% Author of this conversion to LaTeX format: Jan Herca, 2017
\documentclass[twoside, 11pt]{book}
\usepackage[T1]{fontenc} % indica al procesador cómo imprimir los caracteres
\usepackage{fontspec} % permite definir fuentes a partir de las instaladas en el SO
\usepackage{geometry}
\usepackage{graphicx}
\usepackage{float}
\usepackage{tocloft}
\usepackage{titleps}
\usepackage{emptypage}
\usepackage[spanish]{babel}
\usepackage{multicol}
% Text styles
\geometry{paperwidth=16cm, paperheight=24cm, top=2.5cm, bottom=1.7cm, inner=2.5cm, outer=1.2cm}

\makeatletter
\def\@makechapterhead#1{%
	\vspace*{50\p@}%
	{\parindent \z@ \raggedright \normalfont
		\interlinepenalty\@M
		\huge \bfseries #1\par\nobreak
		\vskip 40\p@
}}
\def\@makeschapterhead#1{%
	\vspace*{50\p@}%
	{\parindent \z@ \raggedright
		\normalfont
		\interlinepenalty\@M
		\huge \bfseries  #1\par\nobreak
		\vskip 40\p@
}}
\makeatother

\renewcommand{\cftchapleader}{\cftdotfill{\cftdotsep}}
%\renewcommand{\thechapter}{}
\renewcommand{\cftchapfont}{\large}
\cftsetpnumwidth{3em}
\renewcommand{\cftchappagefont}{\large}

\newcommand{\myfr}[3]{\textit{#1} #2 {\tiny #3}}

\title{La Quinta Revelación \newline Primer Volumen \newline Dios, \newline el Universo Central \newline y los Superuniversos}
\date{}
\begin{document}

\begin{titlepage}
	\centering
	{\Huge\bfseries El Libro de Urantia\par}
	{\huge\bfseries La Quinta Revelación\par}
	\vspace{1cm}
	{\huge\bfseries Primer Volumen\par}
	\vspace{1cm}
	{\huge\bfseries Dios,\par}
	{\huge\bfseries el Universo Central\par}
	{\huge\bfseries y los Superuniversos\par}
	\vfill
	{\scshape\Large URANTIA FOUNDATION\par}
	{\scshape\Large CHICAGO ILLINOIS\par}
	{\Large 2009 Traducción al español Europea\par}
\end{titlepage}

	

\par {\textcopyright} 2019 Jan Herca, de la edición
\par {\textcopyright} 2009 Urantia Foundation, de la traducción
\par {\textcopyright} 1993 Urantia Foundation, de otros materiales
\bigbreak
\par Jan Herca
\par Correo electrónico: janherca@gmail.com
\bigbreak
\par Urantia Foundation
\par 533 West Diversey Parkway
\par Chicago, IL 60614 EE.UU.A
\par Oficina: 1+(773) 525-3319
\par Fax: 1 +(773) 525-7739
\par Website: http://www.urantia.org
\par Correo electrónico: urantia@urantia.org
\bigbreak
\par Todos los derechos reservados, incluyendo el de traducción en los Estados Unidos de América, Canadá y en los demás países de la Unión Internacional de copyright. Todos los derechos reservados en los paises firmantes de la Union Panamericana de la Union internacional de copyright.
\par No todo el libro ni parte de él pueden ser copiados, reproducidos o traducidos en forma alguna, ya sea por medio electrónico, mecánico u otra forma, como fotocopia, grabación o archivo computerizado sin autorización por escrito del editor.
\par URANTIA,'' ``URANTIAN,'' ``EL LIBRO DE URANTIA'' y son marcas registradas de Urantia Foundation y su uso está sujeto a licencia.
\bigbreak
\par La Quinta Revelación es una reedición de El Libro de Urantia (Edición Europea). Está dividido en siete volúmenes para hacerlo más manejable y dispone de contenido adicional en forma de ayudas a la lectura integradas en el texto. El Libro de Urantia (Edición Europea) es una traducción de The Urantia Book realizada por la Fundación Urantia en 2009. 
\newpage

\begin{center}
	{\huge\bfseries Las partes del libro\par}
	\vspace{1cm}
	{\scshape\large PRIMER VOLUMEN\par}
	{\scshape\Large DIOS, EL UNIVERSO CENTRAL Y LOS SUPERUNIVERSOS\par}
	\vspace{1cm}
	
	{\scshape\large SEGUNDO VOLUMEN \par}
	{\scshape\Large EL UNIVERSO LOCAL\par}
	\vspace{1cm}
	
	{\scshape\large TERCER VOLUMEN \par}
	{\scshape\Large LA HISTORIA DE NUESTRO PLANETA, URANTIA\par}
	\vspace{1cm}
	
	{\scshape\large CUARTO VOLUMEN \par}
	{\scshape\Large LA EVOLUCIÓN DE LA CIVILIZACIÓN HUMANA\par}
	\vspace{1cm}
	
	{\scshape\large QUINTO VOLUMEN \par}
	{\scshape\Large LA RELIGIÓN, LA SOBREVIVENCIA A LA MUERTE Y LA DEIDAD EXPERIENCIAL\par}
	\vspace{1cm}
	
	{\scshape\large SEXTO VOLUMEN \par}
	{\scshape\Large LA VIDA Y LAS ENSEÑANZAS DE JESÚS - I\par}
	\vspace{1cm}
	
	{\scshape\large SÉPTIMO VOLUMEN \par}
	{\scshape\Large LA VIDA Y LAS ENSEÑANZAS DE JESÚS - II\par}
\end{center}
\newpage
\begin{center}
{\small \textit {Intencionadamente en blanco}\par}
\end{center}
\newpage

\pagestyle{empty}


\tableofcontents

\newpagestyle{main}{
	%\setheadrule{.4pt}% Header rule
	%\setfootrule{.4pt}% Footer rule
	\sethead[\small \thepage]% odd-left
	[]% odd-center
	[\begin{minipage}{0.9\textwidth}\begin{flushright}\scriptsize \MakeUppercase{\chaptertitle}\end{flushright}\end{minipage}]% odd-right
	{\begin{minipage}{0.9\textwidth}\scriptsize \MakeUppercase{\chaptertitle}\end{minipage}}% even-left
	{}% even-center
	{\small \thepage}% even-right
	\setfoot[]% odd-left
	[]% odd-center
	[]% odd-right
	{}% even-left
	{}% even-center
	{}% even-right
}
\pagestyle{main}
\renewcommand{\makeheadrule}{\rule[-.6\baselineskip]{\linewidth}{.4pt}}

\chapter{Prólogo}
\par
%\textsuperscript{(1.1)}
\textsuperscript{0:0.1} EN LA MENTE de los mortales de Urantia ---éste es el nombre de vuestro mundo--- existe una gran confusión en cuanto al significado de palabras tales como Dios, divinidad y deidad. Los seres humanos se sienten aún más confundidos e inseguros con respecto a las relaciones entre las personalidades divinas designadas con estos numerosos apelativos. Debido a esta pobreza conceptual acompañada de tanta confusión de ideas, se me ha encargado formular esta exposición preliminar para explicar los significados que deberán atribuirse a ciertos símbolos verbales que se van a utilizar más adelante en estos documentos, que el cuerpo de reveladores de la verdad, de Orvonton, ha sido autorizado a traducir al idioma inglés de Urantia.

\par
%\textsuperscript{(1.2)}
\textsuperscript{0:0.2} En nuestro esfuerzo por aumentar la conciencia cósmica y elevar la percepción espiritual, nos resulta extremadamente difícil presentar unos conceptos más amplios y una verdad avanzada cuando estamos limitados por la utilización del lenguaje restringido de un planeta. Pero las instrucciones que hemos recibido nos recomiendan que realicemos todos los esfuerzos posibles para transmitir nuestros significados utilizando los símbolos verbales de la lengua inglesa. Se nos ha ordenado que sólo introduzcamos términos nuevos cuando el concepto a describir no encuentre en inglés ninguna terminología que se pueda emplear para expresar ese nuevo concepto, ya sea parcialmente o incluso distorsionando más o menos su significado.

\par
%\textsuperscript{(1.3)}
\textsuperscript{0:0.3} Con la esperanza de facilitar la comprensión y de impedir la confusión de cualquier mortal que pueda leer detenidamente estos documentos, estimamos oportuno presentar en esta exposición inicial un resumen de los significados que deberán atribuirse a las numerosas palabras inglesas que se van a emplear para designar a la Deidad y a ciertos conceptos asociados de las cosas, los significados y los valores de la realidad universal.

\par
%\textsuperscript{(1.4)}
\textsuperscript{0:0.4} Pero para poder formular este Prólogo de definiciones y limitaciones de terminología, es necesario indicar de antemano cómo se van a utilizar estas palabras en los documentos posteriores. Por consiguiente, este Prólogo no es una exposición completa en sí mismo; sólo es una guía de definiciones, diseñada para ayudar a aquellas personas que lean los documentos adjuntos, que tratan de la Deidad y del universo de universos, y que han sido formulados por una comisión de Orvonton enviada a Urantia con esta finalidad.

\par
%\textsuperscript{(1.5)}
\textsuperscript{0:0.5} Vuestro mundo, Urantia, es uno de los muchos planetas habitados similares que componen el universo local de \textit{Nebadon}. Este universo, junto con otras creaciones semejantes, forman el superuniverso de \textit{Orvonton}, cuya capital es Uversa, de donde procede nuestra comisión. Orvonton es uno de los siete superuniversos evolutivos del tiempo y del espacio que rodean al universo central de \textit{Havona}, la creación sin principio ni fin de la perfección divina. En el núcleo de este universo central y eterno se encuentra la Isla estacionaria del Paraíso, centro geográfico de la infinidad y morada del Dios eterno.

\par
%\textsuperscript{(1.6)}
\textsuperscript{0:0.6} Llamamos generalmente \textit{gran universo} a los siete superuniversos en evolución en asociación con el universo central y divino; éstas son las creaciones organizadas y habitadas actualmente. Todas forman parte del \textit{universo maestro}, que engloba también a los universos del espacio exterior, deshabitados pero en vías de movilización.

\section*{I. Deidad y divinidad}
\par
%\textsuperscript{(2.1)}
\textsuperscript{0:1.1} El universo de universos manifiesta los fenómenos de las actividades de la deidad en los diversos niveles de las realidades cósmicas, los significados mentales y los valores espirituales, pero todos estos ministerios ---personales u otros--- están divinamente coordinados.

\par
%\textsuperscript{(2.2)}
\textsuperscript{0:1.2} LA DEIDAD puede personalizarse como Dios; es prepersonal y superpersonal de maneras no del todo comprensibles para el hombre. La Deidad se caracteriza por la cualidad de la unidad ---actual o potencial--- en todos los niveles supermateriales de la realidad, y las criaturas comprenden mejor esta cualidad unificadora con el apelativo de divinidad.

\par
%\textsuperscript{(2.3)}
\textsuperscript{0:1.3} La Deidad desempeña sus funciones en los niveles personales, prepersonales y superpersonales. La Deidad Total está actuando en los siete niveles siguientes:

\par
%\textsuperscript{(2.4)}
\textsuperscript{0:1.4} 1. \textit{Estático} ---Deidad contenida en sí misma y existente por sí misma.

\par
%\textsuperscript{(2.5)}
\textsuperscript{0:1.5} 2. \textit{Potencial} ---Deidad con una voluntad y una finalidad propias.

\par
%\textsuperscript{(2.6)}
\textsuperscript{0:1.6} 3. \textit{Asociativo} ---Deidad que se ha personalizado a sí misma y divinamente fraternal.

\par
%\textsuperscript{(2.7)}
\textsuperscript{0:1.7} 4. \textit{Creativo} ---Deidad que se distribuye a sí misma y se revela de manera divina.

\par
%\textsuperscript{(2.8)}
\textsuperscript{0:1.8} 5. \textit{Evolutivo} ---Deidad que se expande a sí misma y está identificada con la criatura.

\par
%\textsuperscript{(2.9)}
\textsuperscript{0:1.9} 6. \textit{Supremo} ---Deidad que experimenta por sí misma y que unifica a la criatura con el Creador. Esta Deidad actúa en el primer nivel de identificación con las criaturas bajo la forma de los supercontroladores espacio-temporales del gran universo, y a veces se le llama Supremacía de la Deidad.

\par
%\textsuperscript{(2.10)}
\textsuperscript{0:1.10} 7. \textit{Último} ---Deidad que se proyecta a sí misma y que trasciende el tiempo y el espacio. Deidad omnipotente, omnisciente y omnipresente. Esta Deidad actúa en el segundo nivel de expresión unificadora de la divinidad bajo la forma de los supercontroladores eficaces y los sostenedores absonitos del universo maestro. Comparada con el ministerio de las Deidades en el gran universo, esta actividad absonita en el universo maestro equivale a un supercontrol y a un supersostén universales, a veces llamados Ultimidad de la Deidad.

\par
%\textsuperscript{(2.11)}
\textsuperscript{0:1.11} \textit{El nivel finito} de la realidad está caracterizado por la vida de las criaturas y las limitaciones del espacio-tiempo. Las realidades finitas pueden no tener un final, pero siempre tienen un principio ---son creadas. El nivel de Deidad de la Supremacía se puede concebir como una actividad relacionada con las existencias finitas.

\par
%\textsuperscript{(2.12)}
\textsuperscript{0:1.12} \textit{El nivel absonito} de la realidad está caracterizado por las cosas y los seres sin principio ni fin, y por la trascendencia del tiempo y del espacio. Los absonitarios no son creados; son existenciados ---simplemente existen. El nivel de Deidad de la Ultimidad implica una actividad relacionada con las realidades absonitas. Cada vez que se trasciende el tiempo y el espacio en cualquier parte del universo maestro, este fenómeno absonito es un acto de la Ultimidad de la Deidad.

\par
%\textsuperscript{(2.13)}
\textsuperscript{0:1.13} \textit{El nivel absoluto} está desprovisto de principio, de fin, de tiempo y de espacio. Por ejemplo, en el Paraíso, el tiempo y el espacio no existen; el estado espacio-temporal del Paraíso es absoluto. Las Deidades del Paraíso alcanzan existencialmente este nivel por medio de la Trinidad, pero este tercer nivel de expresión unificadora de la Deidad no está unificado por completo experiencialmente. Los valores y los significados absolutos del Paraíso se manifiestan en cualquier momento, lugar y manera en que funciona el nivel absoluto de la Deidad.

\par
%\textsuperscript{(3.1)}
\textsuperscript{0:1.14} La Deidad puede ser existencial, como en el caso del Hijo Eterno; experiencial, como en el Ser Supremo; asociativa, como en Dios Séptuple; indivisa, como en la Trinidad del Paraíso.

\par
%\textsuperscript{(3.2)}
\textsuperscript{0:1.15} La Deidad es la fuente de todo lo que es divino. La Deidad es característica e invariablemente divina, pero todo lo que es divino no es necesariamente la Deidad, aunque estará coordinado con ella y tenderá hacia alguna fase de unidad ---espiritual, mental o personal--- con la Deidad.

\par
%\textsuperscript{(3.3)}
\textsuperscript{0:1.16} La DIVINIDAD es la cualidad característica, unificadora y coordinadora de la Deidad.

\par
%\textsuperscript{(3.4)}
\textsuperscript{0:1.17} La divinidad es comprensible para las criaturas como verdad, belleza y bondad; está correlacionada en la personalidad como amor, misericordia y ministerio; y se revela en los niveles impersonales como justicia, poder y soberanía.

\par
%\textsuperscript{(3.5)}
\textsuperscript{0:1.18} La Divinidad puede ser perfecta ---completa---, como en los niveles existenciales y de los creadores, los niveles de la perfección del Paraíso; puede ser imperfecta, como en los niveles experienciales y de las criaturas, los niveles de la evolución espacio-temporal; o puede ser relativa, ni perfecta ni imperfecta, como sucede en ciertos niveles de Havona donde se relacionan lo existencial y lo experiencial.

\par
%\textsuperscript{(3.6)}
\textsuperscript{0:1.19} Cuando intentamos concebir la perfección en todas sus fases y formas de relatividad, nos encontramos con siete tipos imaginables:

\par
%\textsuperscript{(3.7)}
\textsuperscript{0:1.20} 1. Perfección absoluta en todos los aspectos.

\par
%\textsuperscript{(3.8)}
\textsuperscript{0:1.21} 2. Perfección absoluta en algunas fases y perfección relativa en todos los demás aspectos.

\par
%\textsuperscript{(3.9)}
\textsuperscript{0:1.22} 3. Aspectos absolutos, relativos e imperfectos en asociaciones variadas.

\par
%\textsuperscript{(3.10)}
\textsuperscript{0:1.23} 4. Perfección absoluta en algunos sentidos e imperfección en todos los demás.

\par
%\textsuperscript{(3.11)}
\textsuperscript{0:1.24} 5. Perfección absoluta en ninguna dirección y perfección relativa en todas las manifestaciones.

\par
%\textsuperscript{(3.12)}
\textsuperscript{0:1.25} 6. Perfección absoluta en ninguna fase, perfección relativa en algunas e imperfecta en las demás.

\par
%\textsuperscript{(3.13)}
\textsuperscript{0:1.26} 7. Perfección absoluta en ningún atributo e imperfección en todos.

\section*{II. Dios}
\par
%\textsuperscript{(3.14)}
\textsuperscript{0:2.1} Las criaturas mortales evolutivas experimentan un impulso irresistible por simbolizar sus conceptos finitos de Dios. La conciencia del deber moral que tiene el hombre, y su idealismo espiritual, representan un nivel de valores ---una realidad experiencial--- que es difícil de simbolizar.

\par
%\textsuperscript{(3.15)}
\textsuperscript{0:2.2} La conciencia cósmica implica el reconocimiento de una Causa Primera, la sola y única realidad sin causa. Dios, el Padre Universal, actúa en tres niveles de personalidad de la Deidad, que tienen un valor subinfinito y expresan de manera relativa la divinidad:

\par
%\textsuperscript{(3.16)}
\textsuperscript{0:2.3} 1. \textit{Prepersonal} ---como en el ministerio de los fragmentos del Padre, tales como los Ajustadores del Pensamiento.

\par
%\textsuperscript{(3.17)}
\textsuperscript{0:2.4} 2. \textit{Personal} ---como en la experiencia evolutiva de los seres creados y procreados.

\par
%\textsuperscript{(3.18)}
\textsuperscript{0:2.5} 3. \textit{Superpersonal} ---como en las realidades existenciadas de ciertos seres absonitos y otros seres asociados.

\par
%\textsuperscript{(3.19)}
\textsuperscript{0:2.6} DIOS es un símbolo verbal con el que se designan todas las personalizaciones de la Deidad. Este vocablo necesita una definición diferente en cada nivel personal donde actúa la Deidad, y debe ser redefinido posteriormente dentro de cada uno de dichos niveles, porque esta palabra se puede utilizar para designar las diversas personalizaciones coordinadas y subordinadas de la Deidad, como por ejemplo los Hijos Creadores Paradisiacos ---los padres de los universos locales.

\par
%\textsuperscript{(4.1)}
\textsuperscript{0:2.7} La palabra Dios, tal como la utilizamos, puede entenderse:

\par
%\textsuperscript{(4.2)}
\textsuperscript{0:2.8} \textit{Por designación} ---como Dios Padre.

\par
%\textsuperscript{(4.3)}
\textsuperscript{0:2.9} \textit{Por el contexto} ---como cuando se utiliza para hablar de algún nivel o asociación de la deidad. Cuando se tengan dudas sobre la interpretación exacta de la palabra Dios, sería aconsejable aplicarla a la persona del Padre Universal.

\par
%\textsuperscript{(4.4)}
\textsuperscript{0:2.10} La palabra Dios siempre indica \textit{la personalidad.} La palabra Deidad puede referirse o no a las personalidades de la divinidad.

\par
%\textsuperscript{(4.5)}
\textsuperscript{0:2.11} La palabra DIOS se utiliza en estos documentos con los siguientes significados:

\par
%\textsuperscript{(4.6)}
\textsuperscript{0:2.12} 1. \textit{Dios Padre} ---Creador\footnote{\textit{Dios como Creador de todo}: Gn 1:1-27; 2:4-23; Ex 20:11; Neh 9:6; Sal 146:6; Is 42:5; Jer 51:15-16; Mc 13:19; Jn 1:1-3; Hch 4:24; 14:15; Ef 3:9; Col 1:16; 1 P 4:19; Ap 4:11; 10:6. \textit{Dios como Creador de cielo y tierra}: Ex 31:17; 2 Re 19:15; 2 Cr 2:12; Sal 115:15-16; 121:2; 124:8; 134:3; Is 37:16; 45:12-18; Jer 10:11-12; 32:17; Ap 14:7. \textit{Dios como Creador del hombre y la mujer}: Gn 5:1-2. \textit{Dios como Creador del hombre}: Eclo 33:10; Mal 2:10. \textit{Dios como Creador de la Tierra}: Is 40:26,28; Am 4:13. \textit{Dios como Creador de la Sabiduría}: Eclo 1:1-4; Bar 3:32-36. \textit{Dios como creador de mundos}: Heb 1:2.}, Controlador\footnote{\textit{Dios como Controlador}: Job 38:1-39:30; Sal 104:1-32; 148:6-12; Hch 14:15.} y Sostén\footnote{\textit{Dios como Sostén}: Sal 37:17,24; 63:8; 145:14; Is 41:10,13; Heb 1:3.}. El Padre Universal, la Primera Persona de la Deidad.

\par
%\textsuperscript{(4.7)}
\textsuperscript{0:2.13} 2. \textit{Dios Hijo} ---Creador Coordinado, Controlador del Espíritu y Administrador Espiritual. El Hijo Eterno, la Segunda Persona de la Deidad.

\par
%\textsuperscript{(4.8)}
\textsuperscript{0:2.14} 3. \textit{Dios Espíritu} ---Actor Conjunto, Integrador Universal y Donador de la Mente. El Espíritu Infinito, la Tercera Persona de la Deidad.

\par
%\textsuperscript{(4.9)}
\textsuperscript{0:2.15} 4. \textit{Dios Supremo}\footnote{\textit{Dios Supremo}: Sal 136:2-3; Dn 2:47; 10:17; Jos 22:22; 1 P 2:13.} ---el Dios del tiempo y del espacio en proceso de actualización o evolución. La Deidad personal que está llevando a cabo, en asociación, la hazaña experiencial del espacio-tiempo: identificar a la criatura con el Creador. El Ser Supremo está experimentando y consiguiendo personalmente la unidad de la Deidad como Dios evolutivo y experiencial de las criaturas evolutivas del tiempo y del espacio.

\par
%\textsuperscript{(4.10)}
\textsuperscript{0:2.16} 5. \textit{Dios Séptuple} ---personalidad de la Deidad que actúa realmente en cualquier parte del espacio-tiempo. Se trata de las Deidades personales del Paraíso y de sus asociados creativos, que actúan dentro y fuera de las fronteras del universo central, y están personalizando el poder como Ser Supremo en el primer nivel de las criaturas donde se revela, en el tiempo y el espacio, la Deidad unificadora. Este nivel es el gran universo, la esfera donde las personalidades del Paraíso descienden al espacio-tiempo, en asociación recíproca con las criaturas evolutivas que ascienden del espacio-tiempo.

\par
%\textsuperscript{(4.11)}
\textsuperscript{0:2.17} 6. \textit{Dios Último} ---el Dios del supertiempo y del espacio trascendido, que se está existenciando. Es el segundo nivel experiencial donde se manifiesta la Deidad unificadora. Dios Último significa que se han hecho realidad los valores superpersonales-absonitos, los valores del espacio-tiempo trascendido y los valores experienciales existenciados, y que han sido sintetizados y coordinados en los niveles creativos finales de la realidad de la Deidad.

\par
%\textsuperscript{(4.12)}
\textsuperscript{0:2.18} 7. \textit{Dios Absoluto} ---el Dios de los valores superpersonales trascendidos y de los significados de la divinidad trascendidos, que se está volviendo experiencial pero que actualmente es existencial como \textit{Absoluto de la Deidad.} Éste es el tercer nivel de expresión y de expansión de la Deidad unificadora. En este nivel supercreativo, la Deidad experimenta el agotamiento del potencial personalizable, encuentra la culminación de la divinidad y sufre la extenuación de su capacidad para revelarse en los niveles progresivos y sucesivos de cualquier otra personalización. Ahora la Deidad encuentra al \textit{Absoluto Incalificado,} incide en él y experimenta su identidad con él.

\section*{III. La Fuente-Centro Primera}
\par
%\textsuperscript{(4.13)}
\textsuperscript{0:3.1} La realidad total e infinita es existencial en siete fases y bajo la forma de siete Absolutos coordinados:

\par
%\textsuperscript{(5.1)}
\textsuperscript{0:3.2} 1. La Fuente-Centro Primera.

\par
%\textsuperscript{(5.2)}
\textsuperscript{0:3.3} 2. La Fuente-Centro Segunda.

\par
%\textsuperscript{(5.3)}
\textsuperscript{0:3.4} 3. La Fuente-Centro Tercera.

\par
%\textsuperscript{(5.4)}
\textsuperscript{0:3.5} 4. La Isla del Paraíso.

\par
%\textsuperscript{(5.5)}
\textsuperscript{0:3.6} 5. El Absoluto de la Deidad.

\par
%\textsuperscript{(5.6)}
\textsuperscript{0:3.7} 6. El Absoluto Universal.

\par
%\textsuperscript{(5.7)}
\textsuperscript{0:3.8} 7. El Absoluto Incalificado.

\par
%\textsuperscript{(5.8)}
\textsuperscript{0:3.9} Dios, como Fuente y Centro Primera, es primordial ---de manera incondicional--- en relación con la realidad total. La Fuente-Centro Primera es infinita así como eterna\footnote{\textit{Dios es eterno}: Dt 33:27; Ro 1:20; 1 Ti 1:17.}, y por lo tanto sólo está limitada o condicionada por su volición.

\par
%\textsuperscript{(5.9)}
\textsuperscript{0:3.10} Dios ---el Padre Universal--- es la personalidad de la Fuente-Centro Primera, y como tal mantiene relaciones personales de control infinito sobre todas las fuentes y centros coordinados y subordinados. Este control es personal e infinito en \textit{potencia,} aunque nunca lo ejerza realmente debido a la perfección con que actúan las citadas fuentes, centros y personalidades coordinados y subordinados.

\par
%\textsuperscript{(5.10)}
\textsuperscript{0:3.11} Por lo tanto, La Fuente-Centro Primera es primordial en todos los ámbitos: deificado y no deificado, personal o impersonal, actual o potencial, finito o infinito. Ninguna cosa o ser, ninguna relatividad o finalidad puede existir a menos que esté en relación directa o indirecta con la primacía de la Fuente-Centro Primera, y bajo su dependencia.

\par
%\textsuperscript{(5.11)}
\textsuperscript{0:3.12} \textit{La Fuente-Centro Primera} está relacionada con el universo de las maneras siguientes:

\par
%\textsuperscript{(5.12)}
\textsuperscript{0:3.13} 1. Las fuerzas gravitatorias de los universos materiales convergen en el centro de gravedad situado en el bajo Paraíso. Por este motivo, el emplazamiento geográfico de su persona está eternamente fijo en relación absoluta con el centro de energía-fuerza del plano inferior o material del Paraíso. Pero la personalidad absoluta de la Deidad se encuentra en el plano superior o espiritual del Paraíso.

\par
%\textsuperscript{(5.13)}
\textsuperscript{0:3.14} 2. Las fuerzas mentales convergen en el Espíritu Infinito; la mente cósmica diferencial y divergente converge en los Siete Espíritus Maestros; la mente del Supremo, que se está volviendo real, converge como experiencia espacio-temporal en Majeston.

\par
%\textsuperscript{(5.14)}
\textsuperscript{0:3.15} 3. Las fuerzas espirituales del universo convergen en el Hijo Eterno.

\par
%\textsuperscript{(5.15)}
\textsuperscript{0:3.16} 4. La capacidad ilimitada de acción de la deidad reside en el Absoluto de la Deidad.

\par
%\textsuperscript{(5.16)}
\textsuperscript{0:3.17} 5. La capacidad ilimitada de reacción de la infinidad existe en el Absoluto Incalificado.

\par
%\textsuperscript{(5.17)}
\textsuperscript{0:3.18} 6. Los dos Absolutos ---Calificado e Incalificado--- están coordinados y unificados en el Absoluto Universal, y a través de él.

\par
%\textsuperscript{(5.18)}
\textsuperscript{0:3.19} 7. La personalidad potencial de un ser moral evolutivo, o de cualquier otro ser moral, está centrada en la personalidad del Padre Universal.

\par
%\textsuperscript{(5.19)}
\textsuperscript{0:3.20} La REALIDAD, tal como la comprenden los seres finitos, es parcial, relativa e imprecisa. La máxima realidad de la Deidad que pueden comprender plenamente las criaturas finitas evolutivas está contenida en el Ser Supremo. Sin embargo, existen realidades anteriores y eternas, realidades superfinitas, que son ancestrales a esta Deidad Suprema de las criaturas evolutivas del espacio-tiempo. Al intentar describir el origen y la naturaleza de la realidad universal, nos vemos obligados a emplear la técnica del razonamiento espacio-temporal para poder acercarnos al nivel de la mente finita. Por consiguiente, muchos acontecimientos simultáneos de la eternidad tenemos que presentarlos como operaciones secuenciales.

\par
%\textsuperscript{(6.1)}
\textsuperscript{0:3.21} Una criatura del espacio-tiempo percibiría el origen y la diferenciación de la Realidad de la manera siguiente: el eterno e infinito YO SOY, ejerciendo su libre albedrío inherente y eterno, consiguió liberar a la Deidad de las trabas de la infinidad incalificada, y esta separación de la infinidad incalificada produjo la primera \textit{tensión absoluta de la divinidad.} Esta tensión, ocasionada por la diferenciación de la infinidad, la resuelve el Absoluto Universal, que se ocupa de unificar y coordinar la infinidad dinámica de la Deidad Total con la infinidad estática del Absoluto Incalificado.

\par
%\textsuperscript{(6.2)}
\textsuperscript{0:3.22} Con esta operación original, el YO SOY teórico consiguió hacer realidad la personalidad al convertirse en el Padre Eterno del Hijo Original, volviéndose simultáneamente la Fuente Eterna de la Isla del Paraíso. Coexistentes con la diferenciación entre el Hijo y el Padre, y en presencia del Paraíso, aparecieron la persona del Espíritu Infinito y el universo central de Havona. Con la aparición de la Deidad personal coexistente ---el Hijo Eterno y el Espíritu Infinito--- el Padre evitó dispersarse, como personalidad, por todo el potencial de la Deidad Total, lo que de otra manera hubiera sido inevitable. Desde entonces, el Padre sólo llena todo el potencial de la Deidad cuando se encuentra en asociación Trinitaria con sus dos iguales en Deidad, mientras que la Deidad experiencial se está actualizando cada vez más en los niveles de divinidad de la Supremacía, la Ultimidad y la Absolutidad.

\par
%\textsuperscript{(6.3)}
\textsuperscript{0:3.23} \textit{El concepto del YO SOY}\footnote{\textit{YO SOY}: Ex 3:13-14.} es una concesión filosófica que hacemos a la mente finita del hombre, atada al tiempo y encadenada al espacio, a la imposibilidad de que las criaturas comprendan las existencias de la eternidad ---las realidades y relaciones sin principio ni fin. Para las criaturas del espacio-tiempo, todas las cosas deben tener un principio, con la sola excepción de la ÚNICA SIN CAUSA--- la causa primigenia de las causas. Por este motivo conceptuamos este nivel de valor filosófico como el YO SOY, y al mismo tiempo enseñamos a todas las criaturas que el Hijo Eterno y el Espíritu Infinito son coeternos con el YO SOY; en otras palabras, que nunca ha existido un momento en el que el YO SOY no fuera el \textit{Padre} del Hijo, y con él, del Espíritu.

\par
%\textsuperscript{(6.4)}
\textsuperscript{0:3.24} El concepto de \textit{Infinito} lo utilizamos para indicar la plenitud ---la finalidad--- implícita en la primacía de la Fuente-Centro Primera. El YO SOY \textit{teórico} es para la criatura una extensión filosófica de <<la infinidad de la voluntad>>, pero el Infinito es un nivel de valor \textit{actual} que representa la connotación, desde la eternidad, de la verdadera infinidad del libre albedrío absoluto y sin trabas del Padre Universal. Este concepto se denomina a veces el Infinito-Padre.

\par
%\textsuperscript{(6.5)}
\textsuperscript{0:3.25} Una gran parte de la confusión que experimentan todas las clases de seres superiores e inferiores, en sus esfuerzos por descubrir al Infinito-Padre, es inherente a sus limitaciones de comprensión. La primacía absoluta del Padre Universal no es evidente en los niveles subinfinitos; por ello, es probable que únicamente el Hijo Eterno y el Espíritu Infinito conozcan realmente al Padre como infinidad; para todas las demás personalidades, este concepto representa un acto de fe.

\section*{IV. La realidad del universo}
\par
%\textsuperscript{(6.6)}
\textsuperscript{0:4.1} La realidad se actualiza de manera diferencial en diversos niveles del universo; la realidad tiene su origen en, y por medio de, la volición infinita del Padre Universal, y es comprensible en tres fases principales en muchos niveles diferentes de actualización del universo:

\par
%\textsuperscript{(6.7)}
\textsuperscript{0:4.2} 1. \textit{La realidad no deificada} se extiende desde los ámbitos energéticos de lo no personal hasta los dominios de la realidad de los valores no personalizables de la existencia universal, e incluso hasta la presencia del Absoluto Incalificado.

\par
%\textsuperscript{(7.1)}
\textsuperscript{0:4.3} 2. \textit{La realidad deificada} engloba todos los potenciales infinitos de la Deidad que se extienden a través de todos los ámbitos de la personalidad, desde el finito más inferior hasta el infinito más elevado, abarcando así el terreno de todo lo que es personalizable, y aún más ---llegando incluso hasta la presencia del Absoluto de la Deidad.

\par
%\textsuperscript{(7.2)}
\textsuperscript{0:4.4} 3. \textit{La realidad interasociada.} Se supone que la realidad del universo es deificada o no deificada, pero para los seres subdeificados, existe un inmenso campo de realidad interasociada, potencial y en vías de actualización, que resulta difícil de identificar. Una gran parte de esta realidad coordinada está incluida en los ámbitos del Absoluto Universal.

\par
%\textsuperscript{(7.3)}
\textsuperscript{0:4.5} He aquí el concepto primordial de la realidad original: El Padre inicia y mantiene la Realidad. Los \textit{diferenciales} primordiales de la realidad consisten en lo deificado y lo no deificado ---el Absoluto de la Deidad y el Absoluto Incalificado. La \textit{relación} primordial que surge es la tensión entre los dos. Esta tensión de la divinidad, iniciada por el Padre, está perfectamente resuelta por el Absoluto Universal, y se eterniza como tal Absoluto.

\par
%\textsuperscript{(7.4)}
\textsuperscript{0:4.6} Desde el punto de vista del tiempo y del espacio, la realidad también se puede dividir como sigue:

\par
%\textsuperscript{(7.5)}
\textsuperscript{0:4.7} 1. \textit{Actual y Potencial.} Son las realidades que existen en su plenitud de expresión, en contraste con las que contienen una capacidad no revelada para el crecimiento. El Hijo Eterno es una actualidad espiritual absoluta; el hombre mortal es en gran parte una potencialidad espiritual no realizada.

\par
%\textsuperscript{(7.6)}
\textsuperscript{0:4.8} 2. \textit{Absoluta y Subabsoluta.} Las realidades absolutas son las existencias de la eternidad. Las realidades subabsolutas están proyectadas en dos niveles: Absonitas ---las realidades que son relativas con respecto al tiempo y a la eternidad. Finitas ---las realidades que se proyectan en el espacio y que se actualizan en el tiempo.

\par
%\textsuperscript{(7.7)}
\textsuperscript{0:4.9} 3. \textit{Existencial y Experiencial.} La Deidad del Paraíso es existencial, pero el Supremo y el Último que emergen son experienciales.

\par
%\textsuperscript{(7.8)}
\textsuperscript{0:4.10} 4. \textit{Personal e Impersonal.} La expansión de la Deidad, la expresión de la personali-dad y la evolución del universo están condicionadas para siempre por el acto voluntario del Padre, que separó definitivamente los significados y valores mentales, espirituales y personales, actuales y potenciales, centrados en el Hijo Eterno, de aquellas cosas que están centradas en la Isla eterna del Paraíso y son inherentes a ella.

\par
%\textsuperscript{(7.9)}
\textsuperscript{0:4.11} EL PARAÍSO\footnote{\textit{El Paraíso}: Lc 23:43; 2 Co 12:4; Ap 2:7.} es un término que incluye a los Absolutos focales, personales y no personales, de todas las fases de la realidad universal. El Paraíso, adecuadamente calificado, puede connotar todas y cada una de las formas de la realidad, la Deidad, la divinidad, la personalidad y la energía ---espiritual, mental o material. Todas comparten el Paraíso como lugar de origen, de función y de destino en lo que se refiere a los valores, los significados y la existencia de hecho.

\par
%\textsuperscript{(7.10)}
\textsuperscript{0:4.12} \textit{La Isla del Paraíso} ---el Paraíso no calificado de otra manera ---es el Absoluto del control de la gravedad material que ejerce la Fuente-Centro Primera. El Paraíso está inmóvil, y es la única cosa estacionaria en el universo de universos. La Isla del Paraíso tiene un emplazamiento en el universo pero ninguna posición en el espacio. Esta Isla eterna es la fuente real de los universos físicos ---pasados, presentes y futuros. La Isla nuclear de Luz es un derivado de la Deidad, pero no es exactamente una Deidad; las creaciones materiales tampoco son una parte de la Deidad, sino una consecuencia.

\par
%\textsuperscript{(7.11)}
\textsuperscript{0:4.13} El Paraíso no es un creador; es el controlador sin igual de numerosas actividades del universo, siendo mucho más controlador que reactivo. En todos los universos materiales, el Paraíso influye en las reacciones y la conducta de todos los seres relacion-ados con la fuerza, la energía y el poder. Pero el Paraíso en sí mismo es único, exclusivo y está aislado en los universos. El Paraíso no representa a nada y nada representa al Paraíso. No es ni una fuerza ni una presencia, sino simplemente \textit{el Paraíso.}

\section*{V. Realidades de la personalidad}
\par
%\textsuperscript{(8.1)}
\textsuperscript{0:5.1} La personalidad es un nivel de realidad deificada, y se extiende desde el nivel humano e intermedio de mayor activación mental de la adoración y la sabiduría, y asciende a través de los niveles morontiales y espirituales hasta alcanzar el estado definitivo de la personalidad. Ésta es la ascensión evolutiva de la personalidad de los mortales y de otras criaturas similares, pero existen otras muchas clases de personalidades en el universo.

\par
%\textsuperscript{(8.2)}
\textsuperscript{0:5.2} La realidad está sometida a la expansión universal, la personalidad a una diversificación infinita, y las dos son capaces de coordinarse casi ilimitadamente con la Deidad y de estabilizarse de manera eterna. Aunque el campo metamórfico de la realidad no personal está claramente limitado, no conocemos ninguna limitación a la evolución progresiva de las realidades de la personalidad.

\par
%\textsuperscript{(8.3)}
\textsuperscript{0:5.3} En los niveles experienciales conseguidos, todas las clases de personalidades y todos los valores de la personalidad son asociables e incluso cocreativos. Incluso Dios y el hombre pueden coexistir en una personalidad unificada, tal como lo demuestra de manera tan exquisita el estado actual de Cristo Miguel ---Hijo del Hombre e Hijo de Dios.

\par
%\textsuperscript{(8.4)}
\textsuperscript{0:5.4} Todas las clases y fases subinfinitas de personalidad son accesibles mediante la asociación y son potencialmente cocreativas. Lo prepersonal, lo personal y lo super-personal están todos unidos por un potencial mutuo de consecución coordinada, de realización progresiva y de capacidad cocreativa. Pero lo impersonal nunca se transmuta directamente en personal. La personalidad nunca es espontánea; es el regalo del Padre Paradisiaco. La personalidad está superpuesta a la energía y sólo se encuentra asociada con los sistemas de energía vivientes; la identidad puede estar asociada con arquetipos de energía no vivientes.

\par
%\textsuperscript{(8.5)}
\textsuperscript{0:5.5} El Padre Universal es el secreto de la realidad de la personalidad, del otorgamiento de la personalidad y del destino de la personalidad. El Hijo Eterno es la personalidad absoluta, el secreto de la energía espiritual, de los espíritus morontiales y de los espíritus perfeccionados. El Actor Conjunto es la personalidad mental y espiritual, la fuente de la inteligencia, de la razón y de la mente universal. Pero la Isla del Paraíso es no personal y extraespiritual; es la esencia del cuerpo universal, la fuente y el centro de la materia física y el arquetipo maestro absoluto de la realidad material universal.

\par
%\textsuperscript{(8.6)}
\textsuperscript{0:5.6} Estas cualidades de la realidad universal se manifiestan en la experiencia humana de los urantianos en los niveles siguientes:

\par
%\textsuperscript{(8.7)}
\textsuperscript{0:5.7} 1. \textit{El cuerpo.} El organismo físico o material del hombre. El mecanismo electroquímico viviente de naturaleza y origen animal.

\par
%\textsuperscript{(8.8)}
\textsuperscript{0:5.8} 2. \textit{La mente.} El mecanismo del organismo humano que piensa, percibe y siente. La totalidad de la experiencia consciente e inconsciente. La inteligencia asociada con la vida emocional, que se eleva hasta el nivel del espíritu mediante la adoración y la sabiduría.

\par
%\textsuperscript{(8.9)}
\textsuperscript{0:5.9} 3. \textit{El espíritu.} El espíritu divino que reside en la mente del hombre ---el Ajustador del Pensamiento. Este espíritu inmortal es prepersonal ---no es una personalidad, aunque está destinado a volverse una parte de la personalidad de la criatura mortal sobreviviente.

\par
%\textsuperscript{(8.10)}
\textsuperscript{0:5.10} 4. \textit{El alma.} El alma del hombre es una adquisición experiencial. A medida que una criatura mortal elige <<\textit{hacer la voluntad del Padre que está en los cielos}>>\footnote{\textit{Hacer la voluntad del Padre}: Sal 143:10; Eclo 15:11-20; Mt 6:10; 7:21; 12:50; Mc 3:35; Lc 8:21; 11:2; Jn 7:16-17; 9:31; 14:21,24; 15:10,14-16.}, el espíritu interno se convierte en el padre de una \textit{nueva realidad} en la experiencia humana. La mente mortal y material es la madre de esta misma realidad emergente. La sustancia de esta nueva realidad no es material ni espiritual ---es \textit{morontial.} Es el alma emergente e inmortal que está destinada a sobrevivir a la muerte física y a empezar la ascensión al Paraíso.

\par
%\textsuperscript{(9.1)}
\textsuperscript{0:5.11} \textit{La personalidad.} La personalidad del hombre mortal no es ni el cuerpo, ni la mente ni el espíritu, y tampoco es el alma. La personalidad es la única realidad invariable en la experiencia por lo demás siempre cambiante de una criatura, y unifica todos los otros factores asociados de la individualidad. La personalidad es el don incomparable que el Padre Universal confiere a las energías vivientes y asociadas de la materia, la mente y el espíritu, y que sobrevive al sobrevivir el alma morontial.

\par
%\textsuperscript{(9.2)}
\textsuperscript{0:5.12} \textit{Morontia} es un término que designa un inmenso nivel intermedio entre lo material y lo espiritual. Puede designar realidades personales o impersonales, energías vivientes o no vivientes. La urdimbre de la morontia es espiritual, su trama es material.

\section*{VI. Energía y arquetipo}
\par
%\textsuperscript{(9.3)}
\textsuperscript{0:6.1} Llamamos personal a todo lo que responde al circuito de personalidad del Padre. Llamamos espíritu a todo lo que responde al circuito espiritual del Hijo. Llamamos mente, mente como un atributo del Espíritu Infinito ---la mente en todas sus fases--- a todo lo que responde al circuito mental del Actor Conjunto. Llamamos materia ---energía-materia en todos sus estados metamórficos ---a todo lo que responde al circuito de gravedad material centrado en el bajo Paraíso.

\par
%\textsuperscript{(9.4)}
\textsuperscript{0:6.2} ENERGÍA es un término que lo incluye todo, y que lo utilizamos para aplicarlo a los reinos espirituales, mentales y materiales. \textit{Fuerza} lo utilizamos también en términos generales. \textit{Poder} se limita generalmente a designar el nivel electrónico de la materia, es decir, la materia sensible a la gravedad lineal en el gran universo. Poder también se emplea para designar la soberanía. No podemos adoptar vuestras definiciones generalmente aceptadas para la fuerza, la energía y el poder. Vuestro lenguaje es tan escaso que tenemos que asignar múltiples significados a estas palabras.

\par
%\textsuperscript{(9.5)}
\textsuperscript{0:6.3} \textit{Energía física} es un término que indica todas las fases y formas del movimiento, la acción y el potencial que se manifiestan en el mundo de los fenómenos.

\par
%\textsuperscript{(9.6)}
\textsuperscript{0:6.4} Al hablar de las manifestaciones de la energía física, utilizamos en general los términos de fuerza cósmica, energía emergente y poder del universo. A menudo se emplean de la manera siguiente:

\par
%\textsuperscript{(9.7)}
\textsuperscript{0:6.5} 1. \textit{La fuerza cósmica} abarca todas las energías derivadas del Absoluto Incalificado pero que aún no responden a la gravedad del Paraíso.

\par
%\textsuperscript{(9.8)}
\textsuperscript{0:6.6} 2. \textit{La energía emergente} abarca aquellas energías que son sensibles a la gravedad del Paraíso, pero que aún no responden a la gravedad local o lineal. Es el nivel pre-electrónico de la energía-materia.

\par
%\textsuperscript{(9.9)}
\textsuperscript{0:6.7} 3. \textit{El poder del universo} incluye todas las formas de energía que son directamente sensibles a la gravedad lineal, aunque todavía responden a la gravedad del Paraíso. Es el nivel electrónico de la energía-materia y de todas sus evoluciones posteriores.

\par
%\textsuperscript{(9.10)}
\textsuperscript{0:6.8} \textit{La mente} es un fenómeno que implica la presencia y la actividad de un \textit{ministerio viviente} además de diversos sistemas de energía, y esto es cierto a todos los niveles de la inteligencia. En la personalidad, la mente siempre media entre el espíritu y la materia; por consiguiente, el universo está iluminado por tres tipos de luz: la luz material, la perspicacia intelectual y la luminosidad espiritual.

\par
%\textsuperscript{(10.1)}
\textsuperscript{0:6.9} \textit{La luz} ---la luminosidad espiritual\footnote{\textit{Luz espiritual}: Esd 7:55; Is 9:2; Mt 4:16; 5:14-16; Lc 1:79; 2:32; Jn 1:4-9; 3:19-21; 8:12; 9:5; 12:46; 1 Jn 1:5; 2:8.}--- es un símbolo verbal, una figura retórica, que implica la manifestación característica de la personalidad de las diversas clases de seres espirituales. Esta emanación luminosa no está relacionada de ninguna manera con el discernimiento intelectual ni con las manifestaciones de la luz física.

\par
%\textsuperscript{(10.2)}
\textsuperscript{0:6.10} UN ARQUETIPO puede ser proyectado con un aspecto material, espiritual o mental, o como cualquier combinación de estas energías. Puede impregnar las personalidades, las identidades, las entidades o la materia no viviente. Pero un arquetipo es un arquetipo y permanece siendo un arquetipo; sólo las \textit{copias} se multiplican.

\par
%\textsuperscript{(10.3)}
\textsuperscript{0:6.11} El arquetipo puede dar forma a la energía, pero no la controla. La gravedad es la única que controla la energía-materia. Ni el espacio ni el arquetipo responden a la gravedad, pero no existe ninguna relación entre el espacio y el arquetipo; el espacio no es un arquetipo ni un arquetipo potencial. El arquetipo es una configuración de la realidad que ya ha pagado todo su débito a la gravedad; la \textit{realidad} de cualquier arquetipo radica en sus energías, en sus componentes mentales, espirituales o materiales.

\par
%\textsuperscript{(10.4)}
\textsuperscript{0:6.12} En contraposición con el aspecto de lo \textit{total,} el arquetipo revela el aspecto \textit{individual} de la energía y de la personalidad. Las formas de la personalidad o de la identidad son arquetipos resultantes de la energía (física, espiritual o mental), pero no son inherentes a ella. Esa cualidad de la energía o de la personalidad que posibilita la aparición de un arquetipo puede atribuirse a Dios ---a la Deidad---, a la dotación de fuerza del Paraíso, a la coexistencia de la personalidad y del poder.

\par
%\textsuperscript{(10.5)}
\textsuperscript{0:6.13} El arquetipo es un diseño maestro a partir del cual se realizan las copias. El Paraíso Eterno es el absoluto de los arquetipos; el Hijo Eterno es el arquetipo de la personalidad; el Padre Universal es el antecesor-fuente directo de los dos. Pero el Paraíso no confiere arquetipos y el Hijo no puede otorgar la personalidad.

\section*{VII. El Ser Supremo}
\par
%\textsuperscript{(10.6)}
\textsuperscript{0:7.1} El mecanismo de Deidad del universo maestro es doble en lo que se refiere a las relaciones de la eternidad. Dios Padre, Dios Hijo y Dios Espíritu son eternos ---son seres existenciales--- mientras que Dios Supremo, Dios Último y Dios Absoluto son personalidades de la Deidad de las épocas posteriores a Havona, que se están \textit{actualizando} en las esferas del espacio-tiempo y del espacio-tiempo trascendido, esferas en expansión evolutiva en el universo maestro. Estas personalidades de la Deidad, que están actualizándose, son eternas en el futuro desde el momento, y a medida que, adquieren personalidad y poder en los universos crecientes mediante la técnica de la actualización experiencial de los potenciales asociativo-creativos de las Deidades eternas del Paraíso.

\par
%\textsuperscript{(10.7)}
\textsuperscript{0:7.2} Por consiguiente, la presencia de la Deidad es doble:

\par
%\textsuperscript{(10.8)}
\textsuperscript{0:7.3} 1. \textit{Existencial} ---seres con una existencia eterna, pasada, presente y futura.

\par
%\textsuperscript{(10.9)}
\textsuperscript{0:7.4} 2. \textit{Experiencial} ---seres que se están actualizando en el presente post-havoniano, pero cuya existencia no tendrá fin en toda la eternidad futura.

\par
%\textsuperscript{(10.10)}
\textsuperscript{0:7.5} El Padre, el Hijo y el Espíritu son existenciales ---existenciales en actualidad (aunque todos los potenciales sean probablemente experienciales). El Supremo y el Último son totalmente experienciales. El Absoluto de la Deidad es experiencial en actualización, pero existencial en potencialidad. La esencia de la Deidad es eterna, pero sólo las tres personas originales de la Deidad son incondicionalmente eternas. Todas las demás personalidades de la Deidad tienen un origen, pero su destino es eterno.

\par
%\textsuperscript{(10.11)}
\textsuperscript{0:7.6} Habiendo logrado expresar la Deidad existencial de sí mismo en el Hijo y el Espíritu, el Padre está consiguiendo ahora expresarse experiencialmente como Dios Supremo, Dios Último y Dios Absoluto en unos niveles de deidad hasta ahora impersonales y no revelados. Pero estas Deidades experienciales no existen actualmente en su plenitud; se encuentran en proceso de actualización.

\par
%\textsuperscript{(11.1)}
\textsuperscript{0:7.7} \textit{Dios Supremo} en Havona es el reflejo espiritual personal de la Deidad trina del Paraíso. Esta relación asociativa de la Deidad se está expandiendo ahora creativamente hacia fuera en Dios Séptuple, y se está sintetizando, en el gran universo, en el poder experiencial del Todopoderoso Supremo. La Deidad del Paraíso, existencial en tres personas, está evolucionando así experiencialmente en dos fases de Supremacía, mientras que estas fases dobles se están unificando, en lo referente al poder y la personalidad, como un solo Señor, el Ser Supremo.

\par
%\textsuperscript{(11.2)}
\textsuperscript{0:7.8} El Padre Universal consigue liberarse voluntariamente de las cadenas de la infinidad y de las trabas de la eternidad mediante la técnica de la trinitización, la personalización triple de la Deidad. El Ser Supremo está evolucionando ahora mismo como unificación personal subeterna de la manifestación séptuple de la Deidad en los segmentos espacio-temporales del gran universo.

\par
%\textsuperscript{(11.3)}
\textsuperscript{0:7.9} \textit{El Ser Supremo} no es un creador directo, salvo que es el padre de Majeston, pero es el coordinador que sintetiza todas las actividades universales de la criatura y del Creador. El Ser Supremo, que ahora se está actualizando en los universos evolutivos, es la Deidad que correlaciona y sintetiza la divinidad espacio-temporal, es decir, la Deidad trina del Paraíso en asociación experiencial con los Creadores Supremos del tiempo y del espacio. Cuando finalmente se haya actualizado, esta Deidad evolutiva constituirá la fusión eterna de lo finito y de lo infinito ---la unión perpetua e indisoluble del poder experiencial y la personalidad espiritual.

\par
%\textsuperscript{(11.4)}
\textsuperscript{0:7.10} Toda la realidad finita del espacio-tiempo, bajo el impulso directivo del Ser Supremo evolutivo, está dedicada a una movilización siempre ascendente y a una unificación cada vez más perfecta (la síntesis del poder con la personalidad) de todas las fases y valores de la realidad finita, en asociación con fases diversas de la realidad del Paraíso, con el objeto y la finalidad de emprender posteriormente el intento de alcanzar los niveles absonitos donde se consigue el estado de supercriatura.

\section*{VIII. Dios Séptuple}
\par
%\textsuperscript{(11.5)}
\textsuperscript{0:8.1} Para resarcirlas por el estado finito y para compensar las limitaciones conceptuales de las criaturas, el Padre Universal ha establecido un séptuple acercamiento a la Deidad para las criaturas evolutivas:

\par
%\textsuperscript{(11.6)}
\textsuperscript{0:8.2} 1. Los Hijos Creadores Paradisiacos.

\par
%\textsuperscript{(11.7)}
\textsuperscript{0:8.3} 2. Los Ancianos de los Días.

\par
%\textsuperscript{(11.8)}
\textsuperscript{0:8.4} 3. Los Siete Espíritus Maestros.

\par
%\textsuperscript{(11.9)}
\textsuperscript{0:8.5} 4. El Ser Supremo.

\par
%\textsuperscript{(11.10)}
\textsuperscript{0:8.6} 5. Dios Espíritu.

\par
%\textsuperscript{(11.11)}
\textsuperscript{0:8.7} 6. Dios Hijo.

\par
%\textsuperscript{(11.12)}
\textsuperscript{0:8.8} 7. Dios Padre.

\par
%\textsuperscript{(11.13)}
\textsuperscript{0:8.9} Esta personalización séptuple de la Deidad en el tiempo y el espacio, y para los siete superuniversos, permite al hombre mortal alcanzar la presencia de Dios, que es espíritu. Para las criaturas finitas del espacio-tiempo, esta Deidad séptuple, cuyo poder y personalidad estarán integrados algún día en el Ser Supremo, es la Deidad funcional de las criaturas mortales evolutivas que emprenden la carrera de ascensión al Paraíso. Esta carrera de descubrimiento experiencial para comprender a Dios empieza por el reconocimiento de la divinidad del Hijo Creador del universo local, se eleva hasta los Ancianos de los Días del superuniverso, y mediante la persona de uno de los Siete Espíritus Maestros, logra descubrir y reconocer la personalidad divina del Padre Universal en el Paraíso.

\par
%\textsuperscript{(12.1)}
\textsuperscript{0:8.10} El gran universo es el triple dominio de Deidad de la Trinidad de Supremacía, Dios Séptuple y el Ser Supremo. Dios Supremo está en potencia en la Trinidad del Paraíso, de la que procede su personalidad y sus atributos espirituales, pero ahora está actualizándose en los Hijos Creadores, los Ancianos de los Días y los Espíritus Maestros, de quienes obtiene su poder como Todopoderoso para los superuniversos del tiempo y del espacio. Esta manifestación de poder del Dios inmediato de las criaturas evolutivas evoluciona realmente en el espacio-tiempo simultáneamente con ellas. El Todopoderoso Supremo, que evoluciona en el nivel de valor de las actividades no personales, y la persona espiritual de Dios Supremo, son una \textit{sola realidad} ---el Ser Supremo.

\par
%\textsuperscript{(12.2)}
\textsuperscript{0:8.11} En la asociación de Deidades de Dios Séptuple, los Hijos Creadores proporcionan el mecanismo por el cual lo mortal se vuelve inmortal y lo finito alcanza el abrazo de lo infinito. El Ser Supremo proporciona la técnica para la movilización del poder y la personalidad, la síntesis divina, de \textit{todas} estas múltiples operaciones, facilitando así que lo finito alcance lo absonito y, a través de otras posibles actualizaciones futuras, intentar alcanzar al Último. Los Hijos Creadores y sus Ministras Divinas asociadas participan en esta movilización suprema, pero es probable que los Ancianos de los Días y los Siete Espíritus Maestros estén establecidos de manera eterna como administradores permanentes del gran universo.

\par
%\textsuperscript{(12.3)}
\textsuperscript{0:8.12} La actividad de Dios Séptuple data desde que se organizaron los siete superuniversos, y probablemente se ampliará cuando comience la evolución futura de las creaciones del espacio exterior. La organización de estos futuros universos en los niveles espaciales primario, secundario, terciario y cuaternario de evolución progresiva presenciará sin duda la inauguración del acercamiento trascendente y absonito a la Deidad.

\section*{IX. Dios Último}
\par
%\textsuperscript{(12.4)}
\textsuperscript{0:9.1} Al igual que el Ser Supremo evoluciona progresivamente a partir de la dotación de divinidad precedente que existe en el potencial de energía y de personalidad incluido en el gran universo, Dios Último se existencia a partir de los potenciales de divinidad que residen en los dominios del universo maestro donde el espacio-tiempo ha sido trascendido. La actualización de la Deidad Última señala la unificación absonita de la primera Trinidad experiencial, e indica la expansión de la Deidad que se unifica en el segundo nivel de autorrealización creativa. Esto constituye el equivalente, en personalidad y poder, de la actualización universal de las realidades absonitas del Paraíso bajo la forma de la Deidad experiencial, produciéndose todo ello en los niveles en vías de existenciarse de los valores espacio-temporales trascendidos. La finalización de este desarrollo experiencial proporcionará un destino y un servicio últimos a todas las criaturas espacio-temporales que hayan alcanzado los niveles absonitos mediante la comprensión completa del Ser Supremo y gracias al ministerio de Dios Séptuple.

\par
%\textsuperscript{(12.5)}
\textsuperscript{0:9.2} \textit{Dios Último} designa a la Deidad personal que actúa en los niveles de divinidad de lo absonito y en las esferas universales del supertiempo y del espacio trascendido. El Último es una existenciación supersuprema de la Deidad. El Supremo es la unificación de la Trinidad tal como la comprenden los seres finitos; el Último es la unificación de la Trinidad del Paraíso tal como la comprenden los seres absonitos.

\par
%\textsuperscript{(13.1)}
\textsuperscript{0:9.3} Por medio del mecanismo de la Deidad evolutiva, el Padre Universal está efectuando realmente el \textit{acto} formidable y asombroso de focalizar la personalidad y movilizar el poder de los valores de la realidad divina de lo finito, lo absonito e incluso lo absoluto, en sus respectivos niveles de significado universales.

\par
%\textsuperscript{(13.2)}
\textsuperscript{0:9.4} Las tres primeras Deidades del Paraíso ---el Padre Universal, el Hijo Eterno y el Espíritu Infinito--- son eternas desde el pasado, y sus personalidades se complementarán en el eterno futuro mediante la actualización experiencial de las Deidades evolutivas asociadas ---Dios Supremo, Dios Último y probablemente Dios Absoluto.

\par
%\textsuperscript{(13.3)}
\textsuperscript{0:9.5} Dios Supremo y Dios Último, que evolucionan ahora en los universos experienciales, no son existenciales ---no son eternos desde el pasado, sino tan sólo eternos en el futuro; son eternos condicionados por el espacio-tiempo y por lo trascendental. Son Deidades que poseen una dotación suprema, última, y posiblemente supremo-última, pero que han experimentado orígenes históricos en el universo. Nunca tendrán fin, pero su personalidad sí ha tenido un principio. Son en verdad las actualizaciones de los potenciales eternos e infinitos de la Deidad, pero por sí mismos no son incondicionalmente eternos ni infinitos.

\section*{X. Dios Absoluto}
\par
%\textsuperscript{(13.4)}
\textsuperscript{0:10.1} La realidad eterna del \textit{Absoluto de la Deidad} posee muchas características que no se pueden explicar plenamente a la mente finita del espacio-tiempo, pero la actualización de \textit{Dios Absoluto} sería la consecuencia de la unificación de la segunda Trinidad experiencial, la Trinidad Absoluta. Esto supondría la realización experiencial de la divinidad absoluta, la unificación de los significados absolutos en los niveles absolutos. Pero no estamos seguros de que todos los valores absolutos estén incluídos, puesto que no se nos ha informado en ningún momento que el Absoluto Calificado sea el equivalente del Infinito. Los destinos superúltimos están implicados en los significados absolutos y la espiritualidad infinita, y si estas dos realidades están inacabadas, no podemos establecer valores absolutos.

\par
%\textsuperscript{(13.5)}
\textsuperscript{0:10.2} Dios Absoluto es la meta por alcanzar y realizar para todos los seres superabsonitos, pero el potencial de poder y de personalidad del Absoluto de la Deidad trasciende nuestros conceptos, y preferimos no hablar de estas realidades que están tan alejadas de la actualización experiencial.

\section*{XI. Los tres Absolutos}
\par
%\textsuperscript{(13.6)}
\textsuperscript{0:11.1} Cuando el pensamiento combinado del Padre Universal y del Hijo Eterno, actuando a través del Dios de Acción, estableció la creación del universo central y divino, el Padre llevó a cabo la expresión de su pensamiento por medio de la palabra de su Hijo y la acción de su Ejecutivo Conjunto, diferenciando su presencia en Havona de los potenciales de la infinidad. Estos potenciales infinitos no revelados permanecen espacialmente ocultos en el Absoluto Incalificado y divinamente disimulados en el Absoluto de la Deidad, mientras que estos dos últimos actúan como uno solo a través del Absoluto Universal, la unidad-infinidad no revelada del Padre Paradisiaco.

\par
%\textsuperscript{(13.7)}
\textsuperscript{0:11.2} Tanto la potencia de la fuerza cósmica como la potencia de la fuerza espiritual están en proceso de realización y revelación progresiva a medida que el crecimiento experiencial enriquece toda la realidad, y gracias a la correlación de lo experiencial con lo existencial por parte del Absoluto Universal. Debido a la presencia equilibradora del Absoluto Universal, la Fuente-Centro Primera efectúa un aumento del poder experiencial, disfruta de la identificación con sus criaturas evolutivas y logra expandir la Deidad experiencial en los niveles de la Supremacía, la Ultimidad y la Absolutidad.

\par
%\textsuperscript{(14.1)}
\textsuperscript{0:11.3} Cuando no es posible distinguir plenamente entre el Absoluto de la Deidad y el Absoluto Incalificado, a su probable labor conjunta o a su presencia coordinada se les denomina la acción del Absoluto Universal.

\par
%\textsuperscript{(14.2)}
\textsuperscript{0:11.4} 1. \textit{El Absoluto de la Deidad} parece ser el activador omnipotente, mientras que el Absoluto Incalificado parece ser el mecanizador totalmente eficaz del universo de universos, e incluso de universos tras universos, supremamente unificados y coordinados de manera última, ya creados, en proceso de creación, o aún por crearse.

\par
%\textsuperscript{(14.3)}
\textsuperscript{0:11.5} El Absoluto de la Deidad no puede reaccionar de manera subabsoluta ante una situación cualquiera del universo, o al menos no lo hace. En cualquier situación determinada, cada respuesta de este Absoluto parece encaminada al bienestar de todas las cosas y seres de la creación, no sólo en su estado actual de existencia, sino también con vistas a las infinitas posibilidades de toda la eternidad futura.

\par
%\textsuperscript{(14.4)}
\textsuperscript{0:11.6} El Absoluto de la Deidad es ese potencial que fue separado de la realidad total e infinita por la libre elección del Padre Universal, y dentro de él tienen lugar todas las actividades de la divinidad ---existenciales y experienciales. Éste es el Absoluto \textit{Calificado,} en contraste con el Absoluto \textit{Incalificado;} pero en la inclusión de todo el potencial absoluto, el Absoluto Universal está sobreañadido a los dos.

\par
%\textsuperscript{(14.5)}
\textsuperscript{0:11.7} 2. \textit{El Absoluto Incalificado} es no personal, extradivino y no deificado. Este Absoluto carece por tanto de personalidad, de divinidad y de todas las prerrogativas de un creador. Ningún hecho o verdad, ninguna experiencia o revelación, ninguna filosofía o absonitidad serán capaces de comprender la naturaleza y el carácter de este Absoluto sin calificación en el universo.

\par
%\textsuperscript{(14.6)}
\textsuperscript{0:11.8} Debemos indicar claramente que el Absoluto Incalificado es una \textit{realidad positiva} que impregna el gran universo, y que al parecer se extiende con idéntica presencia espacial dentro y fuera de las actividades de fuerza y de las evoluciones premateriales de las vertiginosas extensiones de las regiones espaciales situadas más allá de los siete superuniversos. El Absoluto Incalificado no es el mero negativismo de un concepto filosófico, basado en las suposiciones de los sofismas metafísicos sobre la universalidad, el dominio y la primacía de lo incondicionado y lo incalificado. El Absoluto Incalificado es un supercontrol positivo del universo en la infinidad; este supercontrol es ilimitado sobre la fuerza y el espacio, pero está definitivamente condicionado por la presencia de la vida, la mente, el espíritu y la personalidad; y además está condicionado por las reacciones de la voluntad y los mandatos resueltos de la Trinidad del Paraíso.

\par
%\textsuperscript{(14.7)}
\textsuperscript{0:11.9} Estamos convencidos de que el Absoluto Incalificado no es una influencia indiferenciada que lo impregna todo, comparable a los conceptos panteístas de la metafísica o a la antigua hipótesis científica del éter. El Absoluto Incalificado es ilimitado en fuerza y está condicionado por la Deidad, pero no percibimos plenamente la relación de este Absoluto con las realidades espirituales de los universos.

\par
%\textsuperscript{(14.8)}
\textsuperscript{0:11.10} 3. \textit{El Absoluto Universal.} Llegamos a la conclusión lógica de que este Absoluto era inevitable cuando el Padre Universal, mediante un acto de su libre albedrío absoluto, diferenció las realidades del universo en valores deificados y no deificados ---personalizables y no personalizables. El Absoluto Universal es el fenómeno de la Deidad que indica que está resuelta la tensión que se produjo cuando el acto de libre albedrío diferenció así la realidad universal, y este Absoluto actúa como coordinador asociativo de estas sumas totales de potenciales existenciales.

\par
%\textsuperscript{(15.1)}
\textsuperscript{0:11.11} La presencia y la tensión del Absoluto Universal indican que la diferencia entre la realidad de la deidad y la realidad no deificada está ajustada. Esta diferencia era inherente a la separación entre la dinámica de la divinidad con libre albedrío y la estática de la infinidad incalificada.

\par
%\textsuperscript{(15.2)}
\textsuperscript{0:11.12} Recordad siempre que la infinidad potencial es absoluta e inseparable de la eternidad. La infinidad actual que aparece en el tiempo nunca puede ser más que parcial y por tanto debe ser no absoluta; la infinidad de la personalidad actual tampoco puede ser absoluta, excepto en la Deidad incalificada. La diferencia entre el potencial de infinidad del Absoluto Incalificado y el del Absoluto de la Deidad es lo que eterniza al Absoluto Universal, haciendo de este modo cósmicamente posible tener universos materiales en el espacio, y espiritualmente posible tener personalidades finitas en el tiempo.

\par
%\textsuperscript{(15.3)}
\textsuperscript{0:11.13} Lo finito sólo puede coexistir en el cosmos con lo Infinito a causa de la presencia asociativa del Absoluto Universal, que iguala tan perfectamente las tensiones entre el tiempo y la eternidad, la finitud y la infinidad, el potencial de la realidad y la actualidad de la realidad, el Paraíso y el espacio, el hombre y Dios. Asociativamente, el Absoluto Universal constituye la identificación de la zona de realidad evolutiva en progreso que existe en los universos del espacio-tiempo y del espacio-tiempo trascendido, donde se manifiesta la Deidad subinfinita.

\par
%\textsuperscript{(15.4)}
\textsuperscript{0:11.14} El Absoluto Universal es el potencial de la Deidad estático-dinámica que se puede hacer realidad funcionalmente en los niveles del tiempo y de la eternidad bajo la forma de valores finitos y absolutos, y que contiene la posibilidad de un acercamiento experiencial-existencial. Este aspecto incomprensible de la Deidad puede ser estático, potencial y asociativo, pero experiencialmente no es creativo ni evolutivo en lo que respecta a las personalidades inteligentes que actúan ahora en el universo maestro.

\par
%\textsuperscript{(15.5)}
\textsuperscript{0:11.15} \textit{El Absoluto.} Aunque los dos Absolutos ---calificado e incalificado--- parecen actuar de manera tan divergente cuando son observados por las criaturas mentales, están perfecta y divinamente unificados en, y por, el Absoluto Universal. A fin de cuentas y para comprenderlo de manera final, los tres forman un solo Absoluto. En los niveles subinfinitos están diferenciados a causa de sus funciones, pero en la infinidad son UNO SOLO.

\par
%\textsuperscript{(15.6)}
\textsuperscript{0:11.16} Nunca utilizamos el término <<\textit{Absoluto}>> como una negación de algo o para desmentir alguna cosa. Tampoco consideramos que el Absoluto Universal se determine a sí mismo, que sea una especie de Deidad impersonal y panteísta. En todo lo que concierne a la personalidad en el universo, lo Absoluto está estrictamente limitado por la Trinidad y dominado por la Deidad.

\section*{XII. Las Trinidades}
\par
%\textsuperscript{(15.7)}
\textsuperscript{0:12.1} La Trinidad original y eterna del Paraíso es existencial y era inevitable. Cuando la voluntad sin trabas del Padre diferenció lo personal de lo no personal, esta Trinidad sin principio era inherente a ese hecho, y se hizo real cuando la voluntad personal del Padre coordinó estas realidades dobles por medio de la mente. Las Trinidades posteriores a Havona son experienciales ---son inherentes a la creación de los dos niveles subabsolutos y evolutivos en los que se manifiestan la personalidad y el poder en el universo maestro.

\par
%\textsuperscript{(15.8)}
\textsuperscript{0:12.2} \textit{La Trinidad del Paraíso}\footnote{\textit{La Trinidad del Paraíso}: Mt 28:19; Hch 2:32-33; 2 Co 13:14; 1 Jn 5:7. \textit{La visión de Pablo sobre la Trinidad}: 1 Co 12:4-6.} ---la unión de la Deidad eterna del Padre Universal, el Hijo Eterno y el Espíritu Infinito ---es existencial en actualidad, pero todos sus potenciales son experienciales. Por eso esta Trinidad constituye la única realidad de la Deidad que abarca la infinidad, y por eso se producen los fenómenos universales de la actualización de Dios Supremo, Dios Último y Dios Absoluto.

\par
%\textsuperscript{(15.9)}
\textsuperscript{0:12.3} La primera y segunda Trinidad experienciales, las Trinidades posteriores a Havona, no pueden ser infinitas porque contienen \textit{Deidades derivadas,} unas Deidades que han evolucionado mediante la actualización experiencial de unas realidades creadas o existenciadas por la Trinidad existencial del Paraíso. La infinidad de la divinidad se está enriqueciendo constantemente, si no ampliando, gracias a la finitud y a la absonidad de la experiencia de las criaturas y de los Creadores.

\par
%\textsuperscript{(16.1)}
\textsuperscript{0:12.4} Las Trinidades son las verdades de las relaciones y los hechos de la manifestación coordinada de la Deidad. Las funciones de la Trinidad abarcan las realidades de la Deidad, y las realidades de la Deidad siempre tratan de realizarse y de manifestarse en la personalización. Por consiguiente, Dios Supremo, Dios Último e incluso Dios Absoluto son inevitabilidades divinas. Estas tres Deidades experienciales eran potenciales en la Trinidad existencial, la Trinidad del Paraíso, pero su aparición en el universo como personalidades de poder depende, por una parte, de su propia labor experiencial en los universos de poder y de personalidad, y por otra, de los logros experienciales de los Creadores y Trinidades posteriores a Havona.

\par
%\textsuperscript{(16.2)}
\textsuperscript{0:12.5} Las dos Trinidades experienciales post-havonianas, la Trinidad
Última y la Trinidad Absoluta, no están ahora manifestadas por completo; se encuentran en proceso de realización en el universo. Estas asociaciones de la Deidad se pueden describir como sigue:

\par
%\textsuperscript{(16.3)}
\textsuperscript{0:12.6} 1. \textit{La Trinidad Última,} ahora en evolución, constará finalmente del Ser Supremo, las Personalidades Creadoras Supremas y los Arquitectos absonitos del Universo Maestro, esos incomparables planificadores de universos que no son ni creadores ni criaturas. Dios Último adquirirá final e inevitablemente poder y personalidad como consecuencia, en la Deidad, de la unificación de esta Trinidad
Última experiencial en el escenario en expansión del universo maestro casi ilimitado.

\par
%\textsuperscript{(16.4)}
\textsuperscript{0:12.7} 2. \textit{La Trinidad Absoluta} ---la segunda Trinidad experiencial ---ahora en proceso de actualización, constará de Dios Supremo, Dios Último y el Consumador no revelado del Destino del Universo. Esta Trinidad ejerce sus funciones tanto en los niveles personales como en los superpersonales, llegando hasta las fronteras de lo no personal, y su unificación en universalidad haría experiencial a la Deidad Absoluta.

\par
%\textsuperscript{(16.5)}
\textsuperscript{0:12.8} La Trinidad Última se está unificando experiencialmente hasta su finalización, pero dudamos sinceramente que una unificación tan completa sea posible en el caso de la Trinidad Absoluta. Sin embargo, nuestro concepto de la Trinidad eterna del Paraíso es un recordatorio permanente de que la trinitización de la Deidad puede lograr lo que de otra manera es inalcanzable; de ahí que consideremos como un postulado la aparición algún día del \textit{Supremo-Último,} y la posible trinitización-objetivación de Dios Absoluto.

\par
%\textsuperscript{(16.6)}
\textsuperscript{0:12.9} Los filósofos del universo consideran como postulado una \textit{Trinidad deTrinidades,} una Trinidad Infinita existencial-experiencial, pero no son capaces de imaginar su personalización, que tal vez equivaldría a la persona del Padre Universal en el nivel conceptual del YO SOY. Pero independientemente de todo esto, la Trinidad original del Paraíso es potencialmente infinita, puesto que el Padre Universal es realmente infinito.

\par[Agradecimiento]{\textit{Agradecimiento}}
\par
%\textsuperscript{(16.8)}
\textsuperscript{0:12.11} Los documentos siguientes describen el carácter del Padre Universal y la naturaleza de sus asociados del Paraíso, junto con un intento por describir el perfecto universo central y los siete superuniversos que lo rodean. Para formularlos tenemos que guiarnos por las órdenes de los gobernantes del superuniverso que nos aconsejan que, en todos nuestros esfuerzos por revelar la verdad y coordinar el conocimiento fundamental, tenemos que dar preferencia a los conceptos humanos más elevados que existen relacionados con los temas que se van a presentar. Sólo podemos recurrir a la revelación pura cuando el concepto a presentar no haya sido expresado anteriormente de manera adecuada por la mente humana.

\par
%\textsuperscript{(17.1)}
\textsuperscript{0:12.12} Las revelaciones planetarias sucesivas de la verdad divina contienen invariablemente los conceptos más elevados que existen sobre los valores espirituales, como una parte de la coordinación nueva y mejor del conocimiento planetario. En consecuencia, para poder presentar a Dios y a sus asociados del universo, hemos seleccionado como base de estos documentos más de mil conceptos humanos que representan el conocimiento planetario más elevado y avanzado sobre los valores espirituales y los significados universales. Cuando estos conceptos humanos, recopilados entre los mortales del pasado y del presente que conocen a Dios, sean inadecuados para describir la verdad tal como se nos ha ordenado que la revelemos, los completaremos sin vacilar recurriendo para ello a nuestro propio conocimiento superior sobre la realidad y la divinidad de las Deidades del Paraíso y del universo trascendente donde residen.

\par
%\textsuperscript{(17.2)}
\textsuperscript{0:12.13} Conocemos plenamente las dificultades de nuestra misión; reconocemos la imposibilidad de traducir completamente el lenguaje de los conceptos de la divinidad y de la eternidad a los símbolos ling\"uísticos de los conceptos finitos de la mente mortal. Pero sabemos que un fragmento de Dios vive en la mente humana y que el Espíritu de la Verdad reside con el alma humana; y sabemos también que estas fuerzas espirituales conspiran para permitir que el hombre material capte la realidad de los valores espirituales y comprenda la filosofía de los significados universales. Pero sabemos incluso con mayor seguridad que estos espíritus de la Presencia Divina son capaces de ayudar al hombre para que se apropie espiritualmente de toda verdad que contribuya a realzar la realidad siempre en progreso de la experiencia religiosa personal ---la conciencia de Dios.

\par
%\textsuperscript{(17.3)}
\textsuperscript{0:12.14} [Redactado por un Consejero Divino de Orvonton, Jefe del Cuerpo de las Personalidades Superuniversales designadas para describir, en Urantia, la verdad sobre las Deidades del Paraíso y el universo de universos.]


\chapter{Documento 1. El Padre Universal}
\setcounter{chapter}{1}
\par
%\textsuperscript{(21.1)}
\textsuperscript{1:0.1} EL Padre Universal es el Dios de toda la creación, la Fuente-Centro Primera de todas las cosas y de todos los seres. Pensad primero en Dios como en un creador, luego como en un controlador y finalmente como en un sostén infinito. La verdad sobre el Padre Universal había empezado a despuntar sobre la humanidad cuando el profeta dijo: <<\textit{Tú, Dios, eres único; no hay ninguno aparte de ti. Has creado el cielo y el cielo de los cielos, con todas sus huestes; tú los proteges y los controlas. Los universos han sido hechos por los Hijos de Dios. El Creador se cubre de luz como si fuera un vestido y extiende los cielos como una cortina}>>\footnote{\textit{Tú eres el único Dios}: Neh 9:6. \textit{No hay Dios aparte de ti}: 2 Re 19:19; 1 Cr 17:20; Neh 9:6; Sal 86:10; Eclo 36:5; Is 37:16; 44:6,8; 45:5-6,21; Dt 4:35,39; 6:4; Mc 12:29,32; Jn 17:3; Ro 3:30; 1 Co 8:4-6; Gl 3:20; Ef 4:6; 1 Ti 2:5; Stg 2:19; 1 Sam 2:2; 2 Sam 7:22. \textit{Dios creó los cielos y la Tierra}: Gn 1:1; 2:4; Ex 20:11; 31:17; 2 Re 19:15; 2 Cr 2:12; Neh 9:6; Sal 115:15-16; 121:2; 124:8; 134:3; 146:6; Is 37:16; 40:28; 42:5; 45:12,18; Jer 10:11-12; 32:17; 51:15-16; Hch 4:24; 14:15; Col 1:16; Ap 4:11; 10:6; 14:7. \textit{Creador de todas sus huestes}: Sal 33:6. \textit{Universos hechos por los Hijos de Dios}: Sal 33:6; Jn 1:1-3; Heb 1:2. \textit{El Creador se cubre de luz}: Sal 104:2.}. El concepto del Padre Universal ---un solo Dios en lugar de muchos dioses--- es el único que ha permitido al hombre mortal comprender al Padre como creador divino y controlador infinito.

\par
%\textsuperscript{(21.2)}
\textsuperscript{1:0.2} Todas las miríadas de sistemas planetarios fueron hechos para ser finalmente habitados por numerosos tipos diferentes de criaturas inteligentes, de seres que pudieran conocer a Dios, recibir el afecto divino y amarle a cambio. El universo de universos es la obra de Dios y el lugar donde residen sus diversas criaturas. <<\textit{Dios creó los cielos y formó la Tierra; estableció el universo y no creó este mundo en vano; lo formó para que fuera habitado}>>\footnote{\textit{Creó la Tierra para ser habitada}: Sal 115:16; Is 45:18.}.

\par
%\textsuperscript{(21.3)}
\textsuperscript{1:0.3} Todos los mundos iluminados reconocen y adoran al Padre Universal, el autor eterno y el sostén infinito de toda la creación. Las criaturas volitivas de un universo tras otro han emprendido el larguísimo viaje hacia el Paraíso, la lucha fascinante de la aventura eterna para alcanzar a Dios Padre. La meta trascendente de los hijos del tiempo es encontrar al Dios eterno, comprender la naturaleza divina, reconocer al Padre Universal. Las criaturas que conocen a Dios sólo tienen una ambición suprema, un solo deseo ardiente, y es volverse, tal como ellas son en sus esferas, semejantes a como él es en su perfección paradisiaca de personalidad y en su esfera universal de justa supremacía. Del Padre Universal que habita la eternidad\footnote{\textit{Dios habita la eternidad}: Esd 8:20; Is 57:15.} ha salido el mandato supremo: <<\textit{Sed perfectos como yo soy perfecto}>>\footnote{\textit{Sed perfectos}: Gn 17:1; 1 Re 8:61; Lv 19:2; Dt 18:13; Mt 5:48; 2 Co 13:11; Stg 1:4; 1 P 1:16.}. Con amor y misericordia, los mensajeros del Paraíso han llevado esta exhortación divina a través de los tiempos y de los universos, incluso hasta las criaturas de origen animal tan humildes como las razas humanas de Urantia.

\par
%\textsuperscript{(22.1)}
\textsuperscript{1:0.4} Este magnífico mandato universal de esforzarse por alcanzar la perfección de la divinidad es el primer deber, y debería ser la más alta ambición, de todas las criaturas que luchan en la creación del Dios de perfección. Esta posibilidad de alcanzar la perfección divina es el destino cierto y final de todo el eterno progreso espiritual del hombre.

\par
%\textsuperscript{(22.2)}
\textsuperscript{1:0.5} Los mortales de Urantia difícilmente pueden esperar ser perfectos en el sentido infinito, pero a los seres humanos les es enteramente posible, poniéndose en camino como lo hacen en este planeta, alcanzar la meta celestial y divina que el Dios infinito ha fijado para el hombre mortal; y cuando alcancen este destino serán tan completos en su esfera de perfección divina, en todo aquello que se refiere a la autorrealización y a la consecución mental, como Dios mismo lo es en su esfera de infinidad y de eternidad. Una perfección así puede no ser universal en el sentido material, ni ilimitada en comprensión intelectual, ni final en experiencia espiritual, pero es final y completa en todos los aspectos finitos relacionados con la divinidad de la voluntad, la perfección de la motivación de la personalidad, y la conciencia de Dios.

\par
%\textsuperscript{(22.3)}
\textsuperscript{1:0.6} Éste es el verdadero significado del mandato divino: <<\textit{Sed perfectos como yo soy perfecto}>>, que impulsa constantemente al hombre mortal hacia adelante y lo atrae hacia el interior en esa larga y fascinante lucha por alcanzar unos niveles de valores espirituales y unos verdaderos significados universales cada vez más elevados. Esta búsqueda sublime del Dios de los universos es la aventura suprema de los habitantes de todos los mundos del tiempo y del espacio.

\section*{1. El nombre del Padre}
\par
%\textsuperscript{(22.4)}
\textsuperscript{1:1.1} De todos los nombres con que se conoce a Dios Padre en todos los universos, aquellos que se encuentran con más frecuencia son los que lo designan como la Fuente Primera y el Centro del Universo. Al Padre Primero se le conoce por diversos nombres en diferentes universos y en diferentes sectores del mismo universo. Los nombres que las criaturas le asignan al Creador dependen mucho del concepto que las criaturas tengan del Creador. La Fuente Primera y el Centro del Universo no se ha revelado nunca por su nombre, sino sólo por su naturaleza. Si creemos que somos los hijos de este Creador, es muy natural que acabemos por llamarle Padre\footnote{\textit{Llamar a Dios ``Padre''}: Sal 89:26; Eclo 51:10; Mt 6:9; Lc 11:2; Ro 1:7}. Pero éste es un nombre de nuestra propia elección, y tiene su origen en el reconocimiento de nuestra relación personal con la Fuente-Centro Primera.

\par
%\textsuperscript{(22.5)}
\textsuperscript{1:1.2} El Padre Universal no impone nunca ninguna forma de reconocimiento arbitrario, de adoración formal, ni de servicio servil a las criaturas volitivas inteligentes de los universos. Los habitantes evolutivos de los mundos del tiempo y del espacio deben reconocerlo, amarlo y adorarlo voluntariamente ---en su propio corazón--- por sí mismos. El Creador se niega a coaccionar el libre albedrío espiritual de sus criaturas materiales o forzarlas a que se sometan. La dedicación afectuosa de la voluntad humana a hacer la voluntad del Padre es el regalo más selecto que el hombre puede hacerle a Dios; de hecho, una consagración así de la voluntad de la criatura constituye el único obsequio posible de verdadero valor que el hombre puede hacerle al Padre Paradisiaco. En Dios, el hombre vive, se mueve y tiene su existencia\footnote{\textit{En Dios el hombre vive, se mueve}: Hch 17:28a.}; no hay nada que el hombre pueda darle a Dios, salvo esta elección de someterse a la voluntad del Padre, y estas decisiones, efectuadas por las criaturas volitivas inteligentes de los universos, constituyen la realidad de esa verdadera adoración que tanto satisface la naturaleza del Padre Creador, dominada por el amor.

\par
%\textsuperscript{(22.6)}
\textsuperscript{1:1.3} Una vez que os hayáis vuelto verdaderamente conscientes de Dios, después de que hayáis descubierto realmente al majestuoso Creador y hayáis empezado a experimentar la conciencia de la presencia interior del controlador divino, entonces, según vuestra iluminación y de acuerdo con la manera y el método que los Hijos divinos utilizan para revelar a Dios, encontraréis un nombre para el Padre Universal, que expresará de manera adecuada vuestro concepto de la Gran Fuente-Centro Primera. Así es como en diferentes mundos y en diversos universos, al Creador se le conoce por numerosas denominaciones, que en el espíritu de las relaciones todas significan lo mismo pero que, en las palabras y los símbolos, cada nombre representa el grado, la profundidad, de su entronización en el corazón de sus criaturas de un reino determinado.

\par
%\textsuperscript{(23.1)}
\textsuperscript{1:1.4} Cerca del centro del universo de universos, al Padre Universal se le conoce generalmente por unos nombres que se pueden considerar como que significan la Fuente Primera. Cuando nos alejamos hacia los universos del espacio, los términos que se emplean para designar al Padre Universal significan con más frecuencia el Centro Universal. Aún más lejos en la creación estrellada, como en el mundo sede de vuestro universo local, se le conoce como la Primera Fuente Creativa y el Centro Divino. En una constelación cercana, a Dios se le llama el Padre de los Universos. En otra, el Sostén Infinito, y hacia el este, el Controlador Divino. También ha sido llamado el Padre de las Luces\footnote{\textit{Padre de las Luces}: Stg 1:17a.}, el Don de la Vida\footnote{\textit{Don de la Vida}: Hch 17:25; Ro 6:23.} y el Único Todopoderoso\footnote{\textit{El poder de Dios}: Ex 9:16; 15:6; 1 Cr 29:11-12; Neh 1:10; Job 36:22; 37:23; Sal 59:16; 106:8; 111:6; 147:5; Jer 10:12; 27:5; 32:17; 51:15; Nm 14:17; Nah 1:3; Dt 9:29; Mt 28:18; 2 Sam 22:33}.

\par
%\textsuperscript{(23.2)}
\textsuperscript{1:1.5} En aquellos mundos donde un Hijo Paradisiaco ha vivido una vida de donación, a Dios\footnote{\textit{Llamar a Dios ``Dios''}: Gn 46:3; Ex 3:6.} se le conoce generalmente por algún nombre que indica una relación personal, un tierno afecto y una devoción paternal. En la sede de vuestra constelación se refieren a Dios como el Padre Universal, y en diferentes planetas de vuestro sistema local de mundos habitados se le conoce de manera diversa como el Padre de los Padres, el Padre Paradisiaco, el Padre Havoniano y el Padre Espiritual. Aquellos que conocen a Dios gracias a las revelaciones de las donaciones de los Hijos Paradisiacos, ceden finalmente al atractivo sentimental de la conmovedora relación que supone la asociación entre el Creador y la criatura, y se refieren a Dios como <<\textit{nuestro Padre}>>\footnote{\textit{Llamar a Dios ``nuestro Padre''}: Sal 89:26; Eclo 51:10; Mt 6:9; Lc 11:2; Ro 1:7}.

\par
%\textsuperscript{(23.3)}
\textsuperscript{1:1.6} En un planeta de criaturas sexuadas, en un mundo donde los impulsos de la emoción parental son inherentes al corazón de sus seres inteligentes, el término Padre se vuelve un nombre muy expresivo y apropiado para el Dios eterno. En vuestro planeta Urantia, es mejor conocido, más universalmente reconocido, con el nombre de \textit{Dios.} El nombre que se le dé tiene poca importancia; lo importante es que lo conozcáis y aspiréis a pareceros a él. Vuestros profetas de antaño lo llamaron con razón <<\textit{el Dios eterno}>>\footnote{\textit{El Dios eterno}: Gn 21:33; Sal 90:2; Is 40:28.}, y se refirieron a él como aquel que <<\textit{vive en la eternidad}>>\footnote{\textit{Vive en la eternidad}: Esd 8:20; Is 57:15.}.

\section*{2. La realidad de Dios}
\par
%\textsuperscript{(23.4)}
\textsuperscript{1:2.1} Dios es la realidad primordial en el mundo del espíritu; Dios es la fuente de la verdad en las esferas de la mente; Dios cubre con su sombra todas las partes de los reinos materiales. Para todas las inteligencias creadas, Dios es una personalidad, y para el universo de universos, es la Fuente-Centro Primera de la realidad eterna. Dios no se parece ni a un hombre\footnote{\textit{Dios no es un hombre}: Nm 23:19; 1 Sam 15:29.} ni a una máquina. El Padre Primero es un espíritu universal, la verdad eterna, la realidad infinita y una personalidad paternal.

\par
%\textsuperscript{(23.5)}
\textsuperscript{1:2.2} El Dios eterno es infinitamente más que la realidad idealizada o el universo personalizado. Dios no es simplemente el deseo supremo del hombre, la búsqueda humana objetivada. Dios tampoco es un simple concepto, el potencial de poder de la rectitud. El Padre Universal no es un sinónimo de la naturaleza, ni tampoco la ley natural personificada. Dios es una realidad trascendente, y no simplemente el concepto humano tradicional de los valores supremos. Dios no es una focalización psicológica de los significados espirituales, ni tampoco <<\textit{la obra más noble del hombre}>>. Dios puede ser todos o cualquiera de estos conceptos en la mente de los hombres, pero es aún más. Es una persona salvadora y un Padre amoroso para todos los que disfrutan de la paz espiritual en la Tierra, y que anhelan experimentar la supervivencia de la personalidad en el momento de la muerte.

\par
%\textsuperscript{(24.1)}
\textsuperscript{1:2.3} La realidad de la existencia de Dios está demostrada en la experiencia humana mediante la divina presencia interior, el Monitor espiritual enviado desde el Paraíso para vivir en la mente mortal del hombre y ayudarle allí a desarrollar el alma inmortal que sobrevive eternamente. Tres fenómenos experienciales revelan la presencia de este Ajustador divino en la mente humana:

\par
%\textsuperscript{(24.2)}
\textsuperscript{1:2.4} 1. La capacidad intelectual para conocer a Dios ---la conciencia de Dios\footnote{\textit{La conciencia de Dios}: Sal 100:3; Tit 1:16.}.

\par
%\textsuperscript{(24.3)}
\textsuperscript{1:2.5} 2. El impulso espiritual de encontrar a Dios ---la búsqueda de Dios\footnote{\textit{La búsqueda de Dios}: 2 Cr 19:3; 30:18-19; Job 23:3; Sal 14:2; 53:2; 69:32; Eclo 39:1-6.}.

\par
%\textsuperscript{(24.4)}
\textsuperscript{1:2.6} 3. El anhelo de la personalidad por parecerse a Dios ---el deseo sincero de hacer la voluntad del Padre\footnote{\textit{El deseo de hacer la voluntad del Padre}: Sal 143:10; Eclo 15:11-20; Mt 6:10; 7:21; 12:50; 26:39,42,44; Mc 3:35; 14:36,39; Lc 8:21; 11:2; 22:42; Jn 4:34; 5:30; 6:38-40; 7:16-17; 9:31; 14:21-24; 15:10,14,16; 17:4}.

\par
%\textsuperscript{(24.5)}
\textsuperscript{1:2.7} La existencia de Dios nunca se podrá demostrar mediante los experimentos científicos ni las deducciones lógicas de la razón pura. Dios sólo se puede comprender en las esferas de la experiencia humana; sin embargo, el verdadero concepto de la realidad de Dios es razonable para la lógica, plausible para la filosofía, esencial para la religión e indispensable para cualquier esperanza de supervivencia de la personalidad.

\par
%\textsuperscript{(24.6)}
\textsuperscript{1:2.8} Aquellos que conocen a Dios han experimentado el hecho de su presencia; estos mortales que conocen a Dios poseen, en su experiencia personal, la única prueba positiva de la existencia del Dios viviente que un ser humano pueda ofrecer a otro. La existencia de Dios sobrepasa por completo toda posibilidad de demostración, excepto en lo que se refiere al contacto entre la conciencia de Dios que posee la mente humana y la presencia de Dios representada por el Ajustador del Pensamiento que reside en el intelecto mortal, y que es otorgado al hombre en calidad de regalo gratuito del Padre Universal.

\par
%\textsuperscript{(24.7)}
\textsuperscript{1:2.9} En teoría, podéis pensar en Dios como Creador, y es el Creador personal del Paraíso y del universo central de perfección, pero los universos del tiempo y del espacio son todos creados y organizados por el cuerpo paradisiaco de los Hijos Creadores. El Padre Universal no es el creador personal del universo local de Nebadon; el universo en el que vivís es la creación de su Hijo Miguel. Aunque el Padre no crea personalmente los universos evolutivos, los controla en muchas de sus relaciones universales y en algunas de sus manifestaciones de energía física, mental y espiritual. Dios Padre es el creador personal del universo Paradisiaco y, en asociación con el Hijo Eterno, el creador de todos los demás Creadores personales de universos.

\par
%\textsuperscript{(24.8)}
\textsuperscript{1:2.10} Como controlador físico en el universo de universos material, la Fuente-Centro Primera ejerce su actividad en los arquetipos de la Isla eterna del Paraíso, y a través de este centro de gravedad absoluto, el Dios eterno ejerce un supercontrol cósmico sobre el nivel físico tanto en el universo central como en todo el universo de universos. Como mente, Dios actúa por medio de la Deidad del Espíritu Infinito; como espíritu, Dios se manifiesta en la persona del Hijo Eterno y en las personas de los hijos divinos del Hijo Eterno. Estas relaciones mutuas de la Fuente-Centro Primera con las Personas y los Absolutos coordinados del Paraíso no impiden en lo más mínimo la acción personal \textit{directa} del Padre Universal en toda la creación y en todos los niveles de ésta. Por medio de la presencia de su espíritu fragmentado, el Padre Creador mantiene un contacto inmediato con sus hijos criaturas y con sus universos creados.

\section*{3. Dios es un espíritu universal}
\par
%\textsuperscript{(25.1)}
\textsuperscript{1:3.1} <<\textit{Dios es espíritu}>>\footnote{\textit{Dios es espíritu}: Jn 4:24}. Es una presencia espiritual universal. El Padre Universal es una realidad espiritual infinita; es <<\textit{el único verdadero Dios soberano, eterno, inmortal e invisible}>>\footnote{\textit{Único Dios verdadero}: 1 Ti 1:17.}. Aunque seáis <<\textit{la progenitura de Dios}>>\footnote{\textit{Progenie de Dios}: 1 Cr 22:10; Sal 2:7; 89:26; Is 56:5; Mt 5:9,16,45; Lc 20:36; Jn 1:12-13; 11:52; Hch 17:28-29; Ro 8:14-17,19,21; 9:26; 2 Co 6:18; Gl 3:26; 4:5-7; Ef 1:5; Flp 2:15; Heb 12:5-8; 1 Jn 3:1-2,10; 5:2; Ap 21:7; 2 Sam 7:14.}, no deberíais pensar que el Padre se parece a vosotros en la forma y el físico porque se os haya dicho que habéis sido creados <<\textit{a su imagen}>>\footnote{\textit{A imagen de Dios}: Gn 1:26-27; 9:6.} ---habitados por los Monitores de Misterio enviados desde la residencia central de su presencia eterna. Los seres espirituales son reales, a pesar de que sean invisibles para los ojos humanos; aunque no sean de carne y hueso.

\par
%\textsuperscript{(25.2)}
\textsuperscript{1:3.2} El antiguo vidente dijo: <<\textit{!`He aquí!, camina a mi lado, y no lo veo; continúa también su camino, pero no lo percibo}>>\footnote{\textit{Pero no lo percibo}: Job 9:11.}. Podemos observar constantemente las obras de Dios, podemos ser muy conscientes de las pruebas materiales de su comportamiento majestuoso, pero raras veces podemos contemplar la manifestación visible de su divinidad, y ni siquiera percibir la presencia de su espíritu delegado que reside en los hombres.

\par
%\textsuperscript{(25.3)}
\textsuperscript{1:3.3} El Padre Universal no es invisible porque se esconda de las criaturas humildes con obstáculos materiales y dones espirituales limitados. La situación es más bien la siguiente: <<\textit{No podéis ver mi rostro, porque ningún mortal puede verme y vivir}>>\footnote{\textit{No puedes ver mi rostro}: Ex 33:20.}. Ningún hombre material podría contemplar al espíritu de Dios y conservar su existencia mortal. A los grupos inferiores de seres espirituales o a cualquier clase de personalidades materiales les es imposible acercarse a la gloria y al brillo espiritual de la presencia de la personalidad divina. La luminosidad espiritual de la presencia personal del Padre es una <<\textit{luz a la que ningún hombre mortal puede acercarse; que ninguna criatura material ha visto o puede ver}>>\footnote{\textit{Luz a la que nadie puede acercarse}: 1 Ti 6:16.}. Pero no es necesario ver a Dios con los ojos de la carne, para percibirlo con la visión de la fe de la mente espiritualizada.

\par
%\textsuperscript{(25.4)}
\textsuperscript{1:3.4} El Padre Universal comparte plenamente su naturaleza espiritual con su yo coexistente, el Hijo Eterno del Paraíso. De la misma manera, tanto el Padre como el Hijo comparten plenamente y sin reservas el espíritu universal y eterno con su personalidad conjunta y coordinada, el Espíritu Infinito. El espíritu de Dios es, en sí mismo y por sí mismo, absoluto; en el Hijo es incalificado, en el Espíritu es universal, y en todos ellos y por todos ellos es infinito.

\par
%\textsuperscript{(25.5)}
\textsuperscript{1:3.5} Dios es un espíritu universal; Dios es la persona universal. La realidad personal suprema de la creación finita es espíritu; la realidad última del cosmos personal es espíritu absonito. Sólo los niveles de la infinidad son absolutos, y sólo en esos niveles existe una unidad final entre la materia, la mente y el espíritu.

\par
%\textsuperscript{(25.6)}
\textsuperscript{1:3.6} En los universos, Dios Padre es, en potencia, el supercontrolador de la materia, la mente y el espíritu. Dios sólo trata directamente con las personalidades de su inmensa creación de criaturas volitivas por medio de su extenso circuito de personalidad, pero (fuera del Paraíso) sólo se puede contactar con él en las presencias de sus entidades fragmentadas, la voluntad de Dios fuera en los universos. Este espíritu paradisiaco, que reside en la mente de los mortales del tiempo y fomenta allí la evolución del alma inmortal de las criaturas supervivientes, tiene la naturaleza y la divinidad del Padre Universal. Pero la mente de estas criaturas evolutivas tiene su origen en los universos locales, y debe conseguir la perfección divina llevando a cabo aquellas transformaciones experienciales de alcance espiritual que se producen inevitablemente cuando la criatura elige hacer la voluntad del Padre que está en los cielos.

\par
%\textsuperscript{(26.1)}
\textsuperscript{1:3.7} En la experiencia interior del hombre, la mente está unida a la materia. Estas mentes vinculadas a la materia no pueden sobrevivir a la muerte física. La técnica de la supervivencia está incluida en aquellos ajustes de la voluntad humana y en aquellas transformaciones en la mente mortal mediante los cuales ese intelecto consciente de Dios se deja enseñar gradualmente por el espíritu y se deja conducir finalmente por él. Esta evolución de la mente humana desde la asociación con la materia hasta la unión con el espíritu tiene como resultado la transmutación de las fases potencialmente espirituales de la mente mortal en las realidades morontiales del alma inmortal. La mente mortal subordinada a la materia está destinada a volverse cada vez más material y, en consecuencia, a sufrir la extinción final de la personalidad; la mente sometida al espíritu está destinada a volverse cada vez más espiritual y a alcanzar finalmente la unidad con el espíritu divino que sobrevive y la guía, consiguiendo de esta manera la supervivencia y la existencia eterna de la personalidad.

\par
%\textsuperscript{(26.2)}
\textsuperscript{1:3.8} Procedo del Eterno, y he regresado muchas veces a la presencia del Padre Universal. Conozco la realidad y la personalidad de la Fuente-Centro Primera, el Padre Eterno y Universal. Sé que aunque el gran Dios es absoluto, eterno e infinito, es también bueno, divino y misericordioso. Conozco la verdad de las grandes declaraciones: <<\textit{Dios es espíritu}>>\footnote{\textit{Dios es espíritu}: Jn 4:24.} y <<\textit{Dios es amor}>>\footnote{\textit{Dios es amor}: 1 Jn 4:8,16.}, y estos dos atributos son revelados al universo de la manera más completa en el Hijo Eterno.

\section*{4. El misterio de Dios}
\par
%\textsuperscript{(26.3)}
\textsuperscript{1:4.1} La infinidad de la perfección de Dios es tal, que hace eternamente de él un misterio. Y el más grande de todos los misterios insondables de Dios es el fenómeno de la residencia divina en la mente de los mortales. La manera en que el Padre Universal reside en las criaturas del tiempo es el más profundo de todos los misterios del universo; la presencia divina en la mente del hombre es el misterio de los misterios.

\par
%\textsuperscript{(26.4)}
\textsuperscript{1:4.2} Los cuerpos físicos de los mortales son <<\textit{los templos de Dios}>>\footnote{\textit{El cuerpo es el templo de Dios}: Lc 17:21; Ro 8:9-11; 1 Co 3:16-17; 6:19; 2 Co 6:16; 2 Ti 1:14; 1 Jn 4:12-15; Ap 21:3}. Aunque los Hijos Creadores Soberanos se acercan a las criaturas de sus mundos habitados y <<\textit{atraen a todos los hombres hacia ellos}>>\footnote{\textit{Atrae a todos los hombres (gravedad espiritual)}: Jer 31:3; Jn 6:44; 12:32.}; aunque <<\textit{permanecen en la puerta}>> de la conciencia <<\textit{y llaman}>>\footnote{\textit{Está a la puerta y llama}: Ap 3:20.} y les encanta entrar en todos aquellos que <<\textit{abren la puerta de su corazón}>>; aunque existe de hecho esta íntima comunión personal entre los Hijos Creadores y sus criaturas mortales, sin embargo, los hombres mortales poseen algo de Dios mismo que reside realmente dentro de ellos; sus cuerpos son su templo.

\par
%\textsuperscript{(26.5)}
\textsuperscript{1:4.3} Cuando hayáis terminado aquí abajo, cuando hayáis finalizado vuestro recorrido en vuestra forma temporal en la Tierra, cuando concluya vuestro viaje de prueba en la carne, cuando el polvo que compone el tabernáculo mortal <<\textit{regrese a la tierra de donde salió}>>\footnote{\textit{El cuerpo regresa a la tierra}: Gn 2:7; 3:19; Ec 3:20-21; Eclo 33:10.}; entonces, así se ha revelado, <<\textit{el Espíritu}>> que vive en vosotros <<\textit{regresará a Dios que lo concedió}>>\footnote{\textit{El Espíritu regresa a Dios}: Ec 3:21; 12:7.}. Dentro de cada ser moral de este planeta reside un fragmento de Dios, una parte de la divinidad. Todavía no es vuestro por derecho de posesión, pero está intencionalmente destinado a ser una sola cosa con vosotros si sobrevivís a la existencia mortal.

\par
%\textsuperscript{(26.6)}
\textsuperscript{1:4.4} Nos enfrentamos constantemente a este misterio de Dios; estamos perplejos ante el despliegue creciente del panorama sin fin de la verdad de su bondad infinita, su misericordia interminable, su sabiduría incomparable y su carácter extraordinario.

\par
%\textsuperscript{(26.7)}
\textsuperscript{1:4.5} El misterio divino consiste en la diferencia inherente que existe entre lo finito y lo infinito, lo temporal y lo eterno, la criatura espacio-temporal y el Creador Universal, lo material y lo espiritual, la imperfección del hombre y la perfección de la Deidad del Paraíso. El Dios del amor universal se manifiesta infaliblemente a cada una de sus criaturas hasta la plenitud de la capacidad de esa criatura para captar espiritualmente las cualidades de la verdad, la belleza y la bondad divinas.

\par
%\textsuperscript{(27.1)}
\textsuperscript{1:4.6} A todo ser espiritual y a toda criatura mortal, en cada esfera y en cada mundo del universo de universos, el Padre Universal revela todo aquello de su yo misericordioso y divino que puede ser discernido o comprendido por esos seres espirituales y esas criaturas mortales. Dios no hace acepción de personas, ya sean espirituales o materiales\footnote{\textit{Dios no hace acepción de personas}: 2 Cr 19:7; Job 34:19; Eclo 35:12; Hch 10:34; Ro 2:11; Gl 2:6; 3:28; Ef 6:9; Col 3:11.}. La presencia divina que puede disfrutar cualquier hijo del universo en un momento dado sólo está limitada por la capacidad de esa criatura para recibir y discernir las realidades espirituales del mundo supermaterial.

\par
%\textsuperscript{(27.2)}
\textsuperscript{1:4.7} Como realidad en la experiencia espiritual humana, Dios no es un misterio. Pero cuando las realidades del mundo del espíritu se intentan poner de manifiesto a las mentes físicas de tipo material, el misterio aparece: unos misterios tan sutiles y tan profundos, que sólo la captación por la fe del mortal que conoce a Dios puede conseguir el milagro filosófico del reconocimiento del Infinito por medio de lo finito, el discernimiento del Dios eterno por parte de los mortales evolutivos de los mundos materiales del tiempo y del espacio.

\section*{5. La personalidad del Padre Universal}
\par
%\textsuperscript{(27.3)}
\textsuperscript{1:5.1} No permitáis que la magnitud de Dios, su infinidad, oscurezca o eclipse su personalidad. <<\textit{Aquel que diseñó el oído, ¿no oirá? Aquel que formó el ojo, ¿no verá?}>>\footnote{\textit{El que hizo el oído ¿no verá?}: Sal 94:9.} El Padre Universal es la cúspide de la personalidad divina; él es el origen y el destino de la personalidad en toda la creación. Dios es a la vez infinito y personal; es una personalidad infinita. El Padre es verdaderamente una personalidad, a pesar de que la infinidad de su persona lo sitúa para siempre más allá de la plena comprensión de los seres materiales y finitos.

\par
%\textsuperscript{(27.4)}
\textsuperscript{1:5.2} Dios es mucho más que una personalidad, tal como la mente humana entiende la personalidad; es incluso mucho más que cualquier concepto posible de una superpersonalidad. Pero es totalmente inútil discutir estos conceptos incomprensibles de la personalidad divina con las mentes de las criaturas materiales, cuyo máximo concepto de la realidad del ser consiste en la idea y en el ideal de la personalidad. El concepto más elevado posible que posee la criatura material sobre el Creador Universal está contenido en los ideales espirituales de la idea elevada de la personalidad divina. Por eso, aunque podáis saber que Dios debe ser mucho más que el concepto humano de la personalidad, sabéis igualmente muy bien que el Padre Universal no puede ser menos, de ninguna manera, que una personalidad eterna, infinita, verdadera, buena y bella.

\par
%\textsuperscript{(27.5)}
\textsuperscript{1:5.3} Dios no se oculta a ninguna de sus criaturas. Sólo es inaccesible para tantas órdenes de seres porque <<\textit{reside en una luz a la que ninguna criatura material puede acercarse}>>\footnote{\textit{Luz a la que puede acercarse}: 1 Ti 6:16.}. La inmensidad y la grandiosidad de la personalidad divina se encuentran más allá del alcance de la mente imperfecta de los mortales evolutivos. Él <<\textit{mide las aguas con el hueco de su mano, mide un universo con la palma de su mano. Él es el que está sentado sobre la órbita de la Tierra, el que extiende los cielos como una cortina y los despliega como un universo para ser habitado}>>\footnote{\textit{Mide las aguas...} Is 40:12a. \textit{Sentado en la órbita...} Is 40:22.}. <<\textit{Levantad vuestros ojos hacia arriba y contemplad quién ha creado todas estas cosas, quién pone de manifiesto el número de sus mundos y los llama a todos por sus nombres}>>\footnote{\textit{Levantad vuestros ojos}: Is 40:26. \textit{Dios llama a las estrellas por su nombre}: Sal 147:4.}; así pues es cierto que <<\textit{las cosas invisibles de Dios son parcialmente comprendidas por las cosas que están hechas}>>\footnote{\textit{Cosas invisibles parcialmente comprendidas}: Ro 1:20.}. Hoy, tal como sois, debéis discernir al Hacedor invisible a través de su creación múltiple y diversa, así como por medio de la revelación y el ministerio de sus Hijos y de sus numerosos subordinados.

\par
%\textsuperscript{(28.1)}
\textsuperscript{1:5.4} Aunque los mortales materiales no pueden ver la persona de Dios, deberían regocijarse en la seguridad de que es una persona; aceptar por la fe la verdad que indica que el Padre Universal ha amado tanto al mundo que ha tomado precauciones para el progreso espiritual eterno de sus humildes habitantes; que <<\textit{se deleita en sus hijos}>>\footnote{\textit{Se deleita en sus hijos}: Pr 8:31; Jer 9:24; Dt 10:15. \textit{El amor de Dios por el mundo}: Jn 3:16; Ro 5:8; 2 Co 5:18-21; 1 Jn 4:9-10.}. Dios no carece de ninguno de esos atributos superhumanos y divinos que constituyen la personalidad perfecta, eterna, amorosa e infinita del Creador.

\par
%\textsuperscript{(28.2)}
\textsuperscript{1:5.5} En las creaciones locales (a excepción del personal de los superuniversos) Dios no tiene ninguna manifestación personal o residencial aparte de la de los Hijos Creadores Paradisiacos, que son los padres de los mundos habitados y los soberanos de los universos locales. Si la fe de la criatura fuera perfecta, sabría con seguridad que habiendo visto a un Hijo Creador ha visto al Padre Universal\footnote{\textit{Ver al Padre mediante el Hijo}: Jn 12:45; 14:7-11.}; al buscar al Padre, no pediría ni esperaría ver otra cosa que al Hijo\footnote{\textit{Buscar al Padre a través del Hijo}: Mt 11:27; Lc 10:22.}. El hombre mortal no puede simplemente ver a Dios\footnote{\textit{El hombre no puede ver a Dios}: Ex 33:20; Jn 1:18.} hasta que no lleve a cabo una transformación espiritual completa y alcance realmente el Paraíso.

\par
%\textsuperscript{(28.3)}
\textsuperscript{1:5.6} La naturaleza de los Hijos Creadores Paradisiacos no abarca todos los potenciales incalificados de la absolutidad universal de la naturaleza infinita de la Gran Fuente-Centro Primera, pero el Padre Universal está \textit{divinamente} presente de todas las maneras en los Hijos Creadores. El Padre y sus Hijos son una sola cosa\footnote{\textit{El Padre y el Hijo son uno}: Jn 1:1; 5:17-18; 10:30,38; 12:44-45; 14:7-11,20; 17:11,21-22.}. Estos Hijos Paradisiacos de la orden de los Migueles son unas personalidades perfectas, e incluso el modelo para todas las personalidades del universo local, desde la Radiante Estrella Matutina hasta las criaturas humanas más humildes de la evolución animal progresiva.

\par
%\textsuperscript{(28.4)}
\textsuperscript{1:5.7} Sin Dios, y exceptuando su persona magnífica y central, no habría ninguna personalidad en todo el inmenso universo de universos. \textit{Dios es personalidad.}

\par
%\textsuperscript{(28.5)}
\textsuperscript{1:5.8} A pesar de que Dios es un poder eterno, una presencia majestuosa, un ideal trascendente y un espíritu glorioso, aunque es todo esto e infinitamente más, sin embargo es verdadera y eternamente una personalidad perfecta de Creador, una persona que puede <<\textit{conocer y ser conocida}>>\footnote{\textit{Conocer y ser conocido}: Jn 8:19; 10:14; 1 Co 13:12.}, que puede <<\textit{amar y ser amada}>>\footnote{\textit{Amar y ser amado}: Jn 14:21; 1 Jn 4:19.}, alguien que puede manifestarnos amistad; y a vosotros se os puede conocer, como a otros humanos les ha sucedido, como amigos de Dios\footnote{\textit{La amistad de Dios}: 2 Cr 20:7; Jn 15:14-15; Stg 2:23.}. Él es un espíritu real y una realidad espiritual.

\par
%\textsuperscript{(28.6)}
\textsuperscript{1:5.9} Cuando vemos al Padre Universal revelado en todo su universo; cuando lo discernimos habitando en las miríadas de sus criaturas; cuando lo contemplamos en las personas de sus Hijos Soberanos; cuando seguimos sintiendo su presencia divina aquí y allá, cerca y lejos, no dudemos ni pongamos en tela de juicio la primacía de su personalidad. A pesar de todas estas extensas distribuciones, continúa siendo una verdadera persona y mantiene perpetuamente una conexión personal con la multitud incontable de sus criaturas diseminadas por todo el universo de universos.

\par
%\textsuperscript{(28.7)}
\textsuperscript{1:5.10} La idea de la personalidad del Padre Universal es un concepto más amplio y verdadero de Dios, que ha llegado principalmente a la humanidad a través de la revelación. La razón, la sabiduría y la experiencia religiosa infieren e implican la personalidad de Dios, pero no la validan por completo. Incluso el Ajustador del Pensamiento interior es prepersonal. La verdad y la madurez de cualquier religión es directamente proporcional a su concepto de la personalidad infinita de Dios y a su captación de la unidad absoluta de la Deidad. La idea de una Deidad personal se convierte entonces en la medida de la madurez religiosa, después de que la religión ha formulado previamente el concepto de la unidad de Dios.

\par
%\textsuperscript{(29.1)}
\textsuperscript{1:5.11} La religión primitiva tenía muchos dioses personales, y estaban forjados a imagen del hombre. La revelación afirma la validez del concepto de la personalidad de Dios, que no es más que una posibilidad en el postulado científico de una Causa Primera, y sólo está provisionalmente insinuado en la idea filosófica de la Unidad Universal. Una persona sólo puede empezar a comprender la unidad de Dios mediante el enfoque de la personalidad. Negar la personalidad de la Fuente-Centro Primera sólo deja una elección entre los dos dilemas filosóficos: el materialismo o el panteísmo.

\par
%\textsuperscript{(29.2)}
\textsuperscript{1:5.12} Al reflexionar sobre la Deidad, el concepto de la personalidad ha de ser despojado de la idea de corporeidad. Tanto en el hombre como en Dios, un cuerpo material no es indispensable para la personalidad. El error de la corporeidad aparece en los dos extremos de la filosofía humana. En el materialismo, el hombre deja de existir como personalidad puesto que pierde su cuerpo al morir; en el panteísmo, puesto que Dios no tiene cuerpo, por consiguiente no es una persona. El tipo superhumano de personalidad progresiva ejerce su actividad en una unión de mente y de espíritu.

\par
%\textsuperscript{(29.3)}
\textsuperscript{1:5.13} La personalidad no es simplemente un atributo de Dios; representa más bien la totalidad de la naturaleza infinita coordinada y de la voluntad divina unificada que se manifiesta en una expresión perfecta eterna y universal. En el sentido supremo, la personalidad es la revelación de Dios al universo de universos.

\par
%\textsuperscript{(29.4)}
\textsuperscript{1:5.14} Puesto que Dios es eterno, universal, absoluto e infinito, no crece en conocimiento ni aumenta en sabiduría. Dios no adquiere experiencia tal como el hombre finito podría suponerlo o comprenderlo, pero en el ámbito de su propia personalidad eterna, disfruta en verdad de esas expansiones continuas de la realización de sí mismo que son en cierto modo comparables y análogas a la adquisición de una experiencia nueva por parte de las criaturas finitas de los mundos evolutivos.

\par
%\textsuperscript{(29.5)}
\textsuperscript{1:5.15} La perfección absoluta del Dios infinito le conduciría a sufrir las terribles limitaciones de la finalidad incalificada de la perfección, si no fuera un hecho que el Padre Universal participa directamente en las luchas de la personalidad de todas las almas imperfectas del extenso universo, que buscan ascender, con la ayuda divina, a los mundos espiritualmente perfectos de arriba. Esta experiencia progresiva de cada ser espiritual y de cada criatura mortal, en todo el universo de universos, es una parte de la conciencia de Deidad en constante expansión que tiene el Padre respecto al círculo divino sin fin de la realización incesante de sí mismo.

\par
%\textsuperscript{(29.6)}
\textsuperscript{1:5.16} Es literalmente cierto que: <<\textit{en todas vuestras aflicciones, él está afligido}>>\footnote{\textit{Dios comparte nuestras aflicciones}: Is 63:9.}. <<\textit{En todos vuestros triunfos, él triunfa en vosotros y con vosotros}>>\footnote{\textit{Dios triunfa con nosotros}: 2 Co 2:14.}. Su espíritu divino prepersonal es una parte real de vosotros. La Isla del Paraíso reacciona a todas las metamorfosis físicas del universo de universos; el Hijo Eterno incluye todos los impulsos espirituales de toda la creación; el Actor Conjunto abarca todas las expresiones mentales del cosmos en expansión. El Padre Universal es consciente, en la plenitud de la conciencia divina, de toda la experiencia individual de las luchas progresivas de las mentes en expansión y de los espíritus ascendentes de cada entidad, ser y personalidad de toda la creación evolutiva del tiempo y del espacio. Y todo esto es literalmente cierto, porque <<\textit{en Él todos vivimos, nos movemos y tenemos nuestra existencia}>>\footnote{\textit{En Él vivimos y nos movemos}: Hch 17:28.}.

\section*{6. La personalidad en el universo}
\par
%\textsuperscript{(29.7)}
\textsuperscript{1:6.1} La personalidad humana es la sombra-imagen espacio-temporal proyectada por la personalidad divina del Creador. Y ninguna realidad se puede comprender nunca de manera adecuada mediante el examen de su sombra. Las sombras deben interpretarse en función de la verdadera sustancia.

\par
%\textsuperscript{(30.1)}
\textsuperscript{1:6.2} Para la ciencia, Dios es una causa; para la filosofía, una idea; para la religión, una persona e incluso el Padre amoroso y celestial. Para los científicos, Dios es una fuerza primordial; para los filósofos, una hipótesis de unidad; para las personas religiosas, una experiencia espiritual viviente. El concepto inadecuado del hombre sobre la personalidad del Padre Universal sólo puede mejorar mediante el progreso espiritual del hombre en el universo, y sólo se volverá verdaderamente adecuado cuando los peregrinos del tiempo y del espacio alcancen finalmente el abrazo divino del Dios viviente en el Paraíso.

\par
%\textsuperscript{(30.2)}
\textsuperscript{1:6.3} No olvidéis nunca que los puntos de vista de la personalidad, concebidos por Dios y por el hombre, se encuentran en las antípodas los unos de los otros. El hombre considera y comprende la personalidad mirando desde lo finito hacia lo infinito; Dios mira desde lo infinito hacia lo finito. El hombre posee el tipo de personalidad más baja, y Dios, la más elevada, siendo incluso suprema, última y absoluta. Por eso los mejores conceptos sobre la personalidad divina han tenido que esperar pacientemente la aparición de mejores ideas sobre la personalidad humana, en especial la elevada revelación tanto de la personalidad humana como de la divina en la vida de donación de Miguel, el Hijo Creador, en Urantia.

\par
%\textsuperscript{(30.3)}
\textsuperscript{1:6.4} El espíritu divino prepersonal que reside en la mente mortal aporta, con su sola presencia, la prueba válida de su existencia real, pero el concepto de la personalidad divina sólo se puede captar mediante la perspicacia espiritual de la auténtica experiencia religiosa personal. Cualquier persona, humana o divina, puede ser conocida y comprendida, independientemente por completo de las reacciones externas o de la presencia material de esa persona.

\par
%\textsuperscript{(30.4)}
\textsuperscript{1:6.5} Para una amistad entre dos personas, cierto grado de afinidad moral y de armonía espiritual es esencial; una personalidad amorosa difícilmente se puede revelar a una persona desprovista de amor. Incluso para acercarse al conocimiento de una personalidad divina, el hombre debe consagrar enteramente a ese esfuerzo todos los dones de su personalidad; una devoción parcial y poco entusiasta será ineficaz.

\par
%\textsuperscript{(30.5)}
\textsuperscript{1:6.6} Cuanto mejor se comprende el hombre a sí mismo y más aprecia los valores de la personalidad de sus semejantes, más anhelará conocer a la Personalidad Original, y con más ardor luchará ese ser humano que conoce a Dios por parecerse a la Personalidad Original. Podéis discutir sobre las opiniones acerca de Dios, pero la experiencia con él y en él existe por encima y más allá de toda controversia humana y de la simple lógica intelectual. El hombre que conoce a Dios no describe sus experiencias espirituales para convencer a los incrédulos, sino para la edificación y la satisfacción mutua de los creyentes.

\par
%\textsuperscript{(30.6)}
\textsuperscript{1:6.7} Asumir que el universo puede ser conocido, que es inteligible, es asumir que el universo está hecho por la mente y dirigido por la personalidad. La mente del hombre sólo puede percibir los fenómenos mentales de otras mentes, ya sean humanas o superhumanas. Si la personalidad del hombre puede experimentar el universo, hay una mente divina y una personalidad real ocultas en alguna parte de ese universo.

\par
%\textsuperscript{(30.7)}
\textsuperscript{1:6.8} Dios es espíritu\footnote{\textit{Dios es espíritu}: Jn 4:24.} ---una personalidad espiritual; el hombre es también un espíritu ---una personalidad espiritual potencial. Jesús de Nazaret alcanzó la plena realización de este potencial de la personalidad espiritual en la experiencia humana; por eso su vida, en la que llevó a cabo la voluntad del Padre, se ha convertido para el hombre en la revelación más real e ideal de la personalidad de Dios. Aunque la personalidad del Padre Universal sólo se puede captar en una experiencia religiosa efectiva, la vida terrestre de Jesús nos inspira mediante la demostración perfecta de esta comprensión y de esta revelación de la personalidad de Dios en una experiencia verdaderamente humana.

\section*{7. El valor espiritual del concepto de la personalidad}
\par
%\textsuperscript{(31.1)}
\textsuperscript{1:7.1} Cuando Jesús hablaba del <<\textit{Dios vivo}>>\footnote{\textit{El Dios vivo}: Mt 16:16-17; Jn 6:57,69.}, se refería a una Deidad personal ---al Padre que está en los cielos. El concepto de la personalidad de la Deidad facilita la comunión; favorece la adoración inteligente; fomenta la confianza reconfortante. Entre cosas no personales puede haber interacción, pero no comunión. No se puede disfrutar de una relación de comunión entre padre e hijo, como entre Dios y el hombre, a menos que los dos sean personas. Sólo las personalidades pueden comunicarse entre sí, aunque la presencia de una entidad impersonal como el Ajustador del Pensamiento puede facilitar enormemente esta comunión personal.

\par
%\textsuperscript{(31.2)}
\textsuperscript{1:7.2} El hombre no lleva a cabo su unión con Dios como una gota de agua podría encontrar su unidad con el océano. El hombre consigue la unión divina mediante una comunión espiritual progresiva y recíproca, mediante unas relaciones de personalidad con el Dios personal, alcanzando cada vez más la naturaleza divina mediante una conformidad sincera e inteligente a la voluntad divina. Una relación tan sublime sólo puede existir entre personalidades.

\par
%\textsuperscript{(31.3)}
\textsuperscript{1:7.3} El concepto de la verdad quizás podría concebirse separado de la personalidad, el concepto de la belleza puede existir sin la personalidad, pero el concepto de la bondad divina sólo es comprensible en relación con la personalidad. Sólo una \textit{persona} puede amar y ser amada. Incluso la belleza y la verdad estarían separadas de la esperanza de la supervivencia si no fueran atributos de un Dios personal, de un Padre amoroso.

\par
%\textsuperscript{(31.4)}
\textsuperscript{1:7.4} No podemos comprender plenamente cómo Dios puede ser primordial, invariable, todopoderoso y perfecto, y al mismo tiempo estar rodeado de un universo en constante cambio y aparentemente limitado por las leyes, un universo evolutivo con imperfecciones relativas. Pero podemos \textit{conocer} esta verdad en nuestra propia experiencia personal, puesto que todos conservamos la identidad de nuestra personalidad y la unidad de nuestra voluntad a pesar de los cambios constantes tanto en nosotros mismos como en nuestro entorno.

\par
%\textsuperscript{(31.5)}
\textsuperscript{1:7.5} Las matemáticas, la lógica o la filosofía no pueden captar la realidad última del universo, sólo puede hacerlo la experiencia personal que se conforma progresivamente a la voluntad divina de un Dios personal. Ni la ciencia, ni la filosofía ni la teología pueden validar la personalidad de Dios. Sólo la experiencia personal de los hijos por la fe del Padre celestial puede llevar a cabo la verdadera comprensión espiritual de la personalidad de Dios.

\par
%\textsuperscript{(31.6)}
\textsuperscript{1:7.6} Los conceptos más elevados sobre la personalidad en el universo implican: identidad, conciencia de sí mismo, voluntad propia y la posibilidad de revelarse. Y estas características implican además una hermandad con otras personalidades semejantes, tal como existe en las asociaciones de personalidad de las Deidades del Paraíso. La unidad absoluta de estas asociaciones es tan perfecta que la divinidad es conocida por su indivisibilidad, por su unidad. <<\textit{El Señor Dios es uno solo}>>\footnote{\textit{Dios es uno solo}: 2 Re 19:19; 1 Cr 17:20; Neh 9:6; Sal 86:10; Eclo 36:5; Is 37:16; 44:6,8; 45:5-6.21; Dt 4:35,39; 6:4; Mc 12:29,32; Jn 17:3; Ro 3:30; 1 Co 8:4-6; Gl 3:20; Ef 4:6; 1 Ti 2:5; Stg 2:19; 1 Sam 2:2; 2 Sam 7:22.}. La indivisibilidad de la personalidad no interfiere con el hecho de que Dios otorgue su espíritu para que viva en el corazón de los hombres mortales. La indivisibilidad de la personalidad de un padre humano no impide la reproducción de hijos e hijas mortales.

\par
%\textsuperscript{(31.7)}
\textsuperscript{1:7.7} Este concepto de la indivisibilidad, en asociación con el concepto de la unidad, implica la trascendencia tanto del tiempo como del espacio por parte de la Ultimidad de la Deidad; por lo tanto, ni el tiempo ni el espacio pueden ser absolutos o infinitos. La Fuente-Centro Primera es esa infinidad que trasciende de una manera incalificada toda mente, toda materia y todo espíritu.

\par
%\textsuperscript{(31.8)}
\textsuperscript{1:7.8} El hecho de la Trinidad del Paraíso no viola de ninguna manera la verdad de la unidad divina. Las tres personalidades de la Deidad del Paraíso son como una sola en todas sus reacciones a la realidad universal y en todas sus relaciones con las criaturas. La existencia de estas tres personas eternas tampoco viola la verdad de la indivisibilidad de la Deidad. Soy plenamente consciente de que no tengo a mi disposición ningún idioma adecuado para explicar claramente a la mente mortal cómo estos problemas del universo se nos presentan a nosotros. Pero no debéis desanimaros; todas estas cosas no están totalmente claras ni siquiera para las altas personalidades que pertenecen a mi grupo de seres paradisiacos. Tened siempre presente que estas profundas verdades relacionadas con la Deidad se clarificarán cada vez más a medida que vuestra mente se espiritualice progresivamente durante las épocas sucesivas de la larga ascensión de los mortales hacia el Paraíso.

\par
%\textsuperscript{(32.1)}
\textsuperscript{1:7.9} [Presentado por un Consejero Divino, miembro de un grupo de personalidades celestiales designadas por los Ancianos de los Días de Uversa, sede del séptimo superuniverso, para supervisar aquellas partes de la revelación que sigue a continuación y que están relacionadas con los asuntos que sobrepasan las fronteras del universo local de Nebadon. Estoy encargado de patrocinar aquellos documentos que describen la naturaleza y los atributos de Dios, porque represento la fuente de información más elevada que se encuentra disponible para tal fin en cualquier mundo habitado. He servido como Consejero Divino en cada uno de los siete superuniversos y he residido durante mucho tiempo en el centro paradisiaco de todas las cosas. He disfrutado muchas veces del placer supremo de permanecer en la presencia personal inmediata del Padre Universal. Describo la realidad y la verdad de la naturaleza y de los atributos del Padre con una autoridad indiscutible; sé de lo que hablo.]


\chapter{Documento 2. La naturaleza de Dios}
\par
%\textsuperscript{(33.1)}
\textsuperscript{2:0.1} PUESTO que el concepto más elevado posible que el hombre tiene de Dios está contenido dentro de la idea y del ideal humanos de una personalidad primordial e infinita, es lícito, y puede resultar útil, estudiar ciertas características de la naturaleza divina que constituyen el carácter de la Deidad. La naturaleza de Dios se puede comprender mejor mediante la revelación del Padre que Miguel de Nebadon desarrolló en sus múltiples enseñanzas y en su magnífica vida humana en la carne. El hombre también puede comprender mejor la naturaleza divina si se considera a sí mismo como un hijo de Dios y aprecia al Creador Paradisiaco como un verdadero Padre espiritual.

\par
%\textsuperscript{(33.2)}
\textsuperscript{2:0.2} La naturaleza de Dios puede ser estudiada en una revelación de ideas supremas, el carácter divino puede ser contemplado como una descripción de ideales celestiales, pero de todas las revelaciones de la naturaleza divina, la más instructiva y la más espiritualmente edificante ha de buscarse en la comprensión de la vida religiosa de Jesús de Nazaret, tanto antes como después de haber alcanzado la plena conciencia de su divinidad. Si la vida encarnada de Miguel la tomamos como trasfondo de la revelación de Dios al hombre, podemos intentar poner en símbolos verbales humanos ciertas ideas e ideales sobre la naturaleza divina que quizás puedan contribuir a iluminar y a unificar mejor el concepto humano de la naturaleza y del carácter de la personalidad del Padre Universal.

\par
%\textsuperscript{(33.3)}
\textsuperscript{2:0.3} En todos nuestros esfuerzos por ampliar y espiritualizar el concepto humano de Dios, nos vemos enormemente obstaculizados por la capacidad limitada de la mente mortal. También encontramos serias dificultades en la ejecución de nuestra tarea debido a las limitaciones del lenguaje y a la pobreza del material que podemos utilizar, a efectos de aclarar o de comparar, en nuestros esfuerzos por describir los valores divinos y presentar los significados espirituales a la mente mortal y finita del hombre. Todos nuestros esfuerzos por ampliar el concepto humano de Dios serían casi inútiles si no fuera por el hecho de que la mente mortal está habitada por el Ajustador otorgado del Padre Universal e impregnada por el Espíritu de la Verdad del Hijo Creador. Contando pues con la presencia de estos espíritus divinos en el corazón del hombre para que me ayuden a ampliar el concepto de Dios, emprendo alegremente la ejecución del mandato que he recibido de intentar describir más ampliamente la naturaleza de Dios a la mente del hombre.

\section*{1. La infinidad de Dios}
\par
%\textsuperscript{(33.4)}
\textsuperscript{2:1.1} <<\textit{En lo tocante al Infinito, no podemos descubrirlo. Los pasos divinos no se conocen}>>\footnote{\textit{En lo tocante al Infinito}: Job 37:23. \textit{Los pasos divinos}: Sal 77:19.}. <<\textit{Su comprensión es infinita y su grandeza es insondable}>>\footnote{\textit{Su comprensión es infinita}: Job 12:13; Sal 147:5. \textit{Grandeza insondable}: Sal 145:3. \textit{Dios es grande}: Job 36:26.}. La luz cegadora de la presencia del Padre es tal, que para sus criaturas humildes parece <<\textit{residir en espesas tinieblas}>>\footnote{\textit{Residir en espesas tinieblas}: Ex 20:21; 1 Re 8:12; 2 Cr 6:1; Dt 4:11; 5:22-23.}. No solamente sus pensamientos y sus planes son insondables, sino que <<\textit{hace una multitud de cosas grandes y maravillosas}>>\footnote{\textit{Maravillas innumerables}: Job 5:9; 9:10.}. <<\textit{Dios es grande; no lo comprendemos, ni se puede averiguar el número de sus años}>>. <<\textit{¿Vivirá Dios en verdad en la Tierra? Mirad, el cielo (el universo) y el cielo de los cielos (el universo de universos) no pueden contenerlo}>>\footnote{\textit{El cielo de los cielos}: 1 Re 8:27; 2 Cr 2:6; 6:18; Neh 9:6; Sal 148:4; Dt 10:14.}. <<\textit{!`Cuán insondables son sus juicios e indescubribles sus caminos!}>>\footnote{\textit{Juicios insondables}: Ro 11:33.}

\par
%\textsuperscript{(34.1)}
\textsuperscript{2:1.2} <<\textit{No hay más que un solo Dios, el Padre infinito, que es también un Creador fiel}>>\footnote{\textit{Creador fiel y divino}: Gn 1:1-27; 2:4-23; 5:1-2; Ex 20:11; 31:17; 2 Re 19:15; 2 Cr 2:12; Neh 9:6; Sal 115:15; 121:2; 124:8; 134:3; 146:6; Eclo 1:1-4; 33:10; Is 37:16; 40:26,28; 42:5; 45:12,18; Jer 10:11-12; 32:17; 51:15; Bar 3:32-36; Am 4:13; Mc 13:19; Jn 1:1-3; Hch 4:24; 14:15; Ef 3:9; Col 1:16; Heb 1:2; Ap 4:11; 10:6; 14:7. \textit{Un Dios, el Padre}: Mal 2:10; 1 Co 8:6; Ef 4:6. \textit{Creador Fiel}: 1 P 4:19}. <<\textit{El Creador divino es también el Determinador Universal, la fuente y el destino de las almas. Él es el Alma Suprema, la Mente Primordial, y el Espíritu Ilimitado de toda la creación}>>\footnote{\textit{Determinador Universal}: Job 34:13; Pr 16:33. \textit{Fuente y destino}: Is 41:4; 44:6; Ap 1:8,11,17; 21:6; 22:13. \textit{Mente primordial}: Is 40:28; 1 Co 2:16; Flp 2:5. \textit{Espíritu Ilimitado}: Sal 104:30.}. <<\textit{El gran Controlador no comete errores. Resplandece de majestad y de gloria}>>\footnote{\textit{Sin errores}: 2 Sam 22:31. \textit{Resplandece de majestad y de gloria}: 1 Cr 29:11; Sal 45:3; Is 2:19-21. \textit{Gloria de Dios}: Is 35:2; 42:8}. <<\textit{El Dios Creador está totalmente desprovisto de temor y de enemistad. Es inmortal, eterno, existente por sí mismo, divino y generoso}>>\footnote{\textit{Desprovisto de temor}: Job 41:33. \textit{Inmortal, eterno}: Ro 1:20; 1 Ti 1:16-17. \textit{Existente por sí mismo}: Ap 1:8. \textit{Divino}: 2 P 1:3-4. \textit{Generoso}: Sal 65:11; 68:10; Jer 31:12,14.}. <<\textit{!`Cuán puro y hermoso, cuán profundo e insondable es el Antepasado celestial de todas las cosas!}>> <<\textit{El Infinito es muy excelente, ya que se da a sí mismo a los hombres. Es el principio y el fin, el Padre de toda intención buena y perfecta}>>\footnote{\textit{Se da a sí mismo}: Sal 84:11; 1 Co 2:12; Ef 1:3. \textit{Es el principio y el fin}: Is 41:4; 44:6; 48:12; Ap 1:8,11,17; 2:8; 21:6; 22:13. \textit{Padre de las buenas intenciones}: Stg 1:17.}. <<\textit{Con Dios todas las cosas son posibles; el Creador eterno es la causa de las causas}>>\footnote{\textit{Todas las cosas son posibles}: Gn 18:14; Jer 32:17; Mt 19:26; Mc 10:27; 14:36; Lc 1:37; 18:27. \textit{Causa de causas}: Gn 1:1ff.}.

\par
%\textsuperscript{(34.2)}
\textsuperscript{2:1.3} A pesar de la infinidad de las manifestaciones prodigiosas de la personalidad eterna y universal del Padre, él es incondicionalmente consciente de su infinidad y de su eternidad; asimismo, conoce plenamente su perfección y su poder. Aparte de sus divinos coordinados, es el único ser en el universo que experimenta una evaluación perfecta, adecuada y completa de sí mismo.

\par
%\textsuperscript{(34.3)}
\textsuperscript{2:1.4} El Padre satisface de manera constante e infalible las necesidades de la demanda diferencial que se tiene de él a medida que ésta cambia de vez en cuando en las diversas secciones de su universo maestro. El gran Dios se conoce y se comprende; es infinitamente consciente de todos sus atributos primordiales de perfección. Dios no es un accidente cósmico ni un experimentador de universos. Los Soberanos de los Universos pueden emprender aventuras; los Padres de las Constelaciones pueden hacer experimentos; los jefes de los sistemas pueden entrenarse; pero el Padre Universal ve el fin desde el principio\footnote{\textit{Ve el fin desde el principio}: Is 46:9-10.}; su plan divino y su propósito eterno abarcan y comprenden realmente todos los experimentos y todas las aventuras de todos sus subordinados, en todos los mundos, sistemas y constelaciones de todos los universos de sus inmensos dominios.

\par
%\textsuperscript{(34.4)}
\textsuperscript{2:1.5} Ninguna cosa es nueva para Dios, y ningún acontecimiento cósmico se produce nunca por sorpresa; él habita el círculo de la eternidad\footnote{\textit{Habita la eternidad}: Esd 8:20; Is 57:15.}. Sus días no tienen principio ni fin\footnote{\textit{Él es el principio y el fin}: Is 41:4; 44:6; 48:12; Ap 1:8,11,17; 2:8; 21:6; 22:13.}. Para Dios no existe el pasado, el presente o el futuro; todo el tiempo está presente en cualquier momento dado. Él es el gran y único YO SOY\footnote{\textit{YO SOY}: Ex 3:14.}.

\par
%\textsuperscript{(34.5)}
\textsuperscript{2:1.6} El Padre Universal es infinito en todos sus atributos de una manera absoluta y sin restricción; y este hecho, en sí mismo y por sí mismo, lo aísla automáticamente de toda comunicación personal directa con los seres materiales finitos y otras inteligencias inferiores creadas.

\par
%\textsuperscript{(34.6)}
\textsuperscript{2:1.7} Para ponerse en contacto y en comunicación con sus múltiples criaturas, todo esto necesita las siguientes medidas que han sido ordenadas: En primer lugar, la personalidad de los Hijos Paradisiacos de Dios que, aunque son perfectos en divinidad, también comparten a menudo la misma naturaleza de carne y hueso de las razas planetarias, volviéndose uno de vosotros y uno con vosotros; de esta manera, Dios se vuelve por así decirlo hombre, como sucedió en la donación de Miguel, que fue llamado indistintamente Hijo de Dios e Hijo del Hombre. En segundo lugar se encuentran las personalidades del Espíritu Infinito, las diversas órdenes de huestes seráficas y otras inteligencias celestiales, que se acercan a los seres materiales de origen humilde y los ayudan y los sirven de tantas maneras. Y en tercer lugar están los Monitores de Misterio impersonales, los Ajustadores del Pensamiento, el don efectivo del gran Dios mismo, enviados para residir en unos seres tales como los humanos de Urantia, enviados sin previo aviso ni explicación. Desde las alturas de la gloria descienden en una profusión interminable para honrar y residir en las mentes humildes de aquellos mortales que poseen la capacidad o el potencial de tener conciencia de Dios.

\par
%\textsuperscript{(35.1)}
\textsuperscript{2:1.8} De esta forma y de muchas otras, de unas maneras desconocidas para vosotros y que sobrepasan por completo la comprensión finita, el Padre Paradisiaco reduce voluntaria y amorosamente su infinidad, y la modifica, la diluye y la atenúa de otras maneras a fin de poder acercarse a la mente finita de sus hijos creados. Y así, mediante una serie de distribuciones cada vez menos absolutas de su personalidad, el Padre infinito consigue disfrutar de un estrecho contacto con las diversas inteligencias de los numerosos reinos de su extenso universo.

\par
%\textsuperscript{(35.2)}
\textsuperscript{2:1.9} Todo esto lo ha hecho, lo hace ahora y continuará haciéndolo eternamente, sin disminuir en lo más mínimo el hecho y la realidad de su infinidad, su eternidad y su primacía. Estas cosas son absolutamente ciertas a pesar de la dificultad para comprenderlas, del misterio en el que están envueltas, o de la imposibilidad de que unas criaturas como las que viven en Urantia puedan entenderlas plenamente.

\par
%\textsuperscript{(35.3)}
\textsuperscript{2:1.10} Puesto que el Padre Primero es infinito en sus planes y eterno en sus propósitos, a cualquier ser finito le es inherentemente imposible captar o comprender nunca en su plenitud estos planes y estos propósitos divinos. El hombre mortal sólo puede vislumbrar los propósitos del Padre de vez en cuando, aquí y allá, a medida que se revelan en relación con el desarrollo del plan de ascensión de las criaturas en sus niveles sucesivos de progresión en el universo. Aunque el hombre no puede abarcar el significado de la infinidad, el Padre infinito comprende plenamente y engloba amorosamente, con toda seguridad, toda la finitud de todos sus hijos en todos los universos.

\par
%\textsuperscript{(35.4)}
\textsuperscript{2:1.11} El Padre comparte la divinidad y la eternidad con un gran número de seres superiores del Paraíso, pero nos preguntamos si la infinidad y la primacía universal consiguiente las comparte plenamente con otros que no sean sus asociados coordinados de la Trinidad del Paraíso. La infinidad de la personalidad debe englobar forzosamente toda finitud de la personalidad; de ahí la verdad ---una verdad literal--- de la enseñanza que afirma que <<\textit{en Él vivimos, nos movemos y tenemos nuestra existencia}>>\footnote{\textit{En Él vivimos y nos movemos}: Hch 17:28.}. El fragmento de pura Deidad del Padre Universal que reside en el hombre mortal \textit{es} una parte de la infinidad de la Gran Fuente-Centro Primera, el Padre de los Padres.

\section*{2. La perfección eterna del Padre}
\par
%\textsuperscript{(35.5)}
\textsuperscript{2:2.1} Incluso vuestros antiguos profetas comprendieron la eterna naturaleza circular, sin principio ni fin, del Padre Universal. Dios está literal y eternamente presente en su universo de universos. Habita el momento presente con toda su majestad absoluta y su grandeza eterna. <<\textit{El Padre tiene la vida en sí mismo, y esta vida es la vida eterna}>>\footnote{\textit{Vida eterna}: Dn 12:2; Mt 19:16,29; 25:46; Mc 10:17,30; Lc 10:25; 18:18,30; Jn 3:15-16,36; 4:14,36; 5:24,39; 6:27,40,47; 6:54,68; 8:51-52; 10:28; 11:25-26; 12:25,50; 17:2-3; Hch 13:46-48; Ro 2:7; 5:21; 6:22-23; Gl 6:8; 1 Ti 1:16; 6:12,19; Tit 1:2; 3:7; 1 Jn 1:2; 2:25; 3:15; 5:13,20; Jud 1:21; Ap 22:5. \textit{Tiene vida en sí mismo}: Jn 5:26. \textit{Su vida es eterna}: 1 Jn 5:11.}. A lo largo de las épocas eternas, el Padre ha sido el que <<\textit{da la vida a todos}>>\footnote{\textit{Da a todos la vida}: Hch 17:25.}. Existe una perfección infinita en la integridad divina. <<\textit{Yo soy el Señor; yo no cambio}>>\footnote{\textit{Yo soy el Señor, yo no cambio}: Mal 3:6.}. Nuestro conocimiento del universo de universos no solamente revela que él es el Padre de las luces\footnote{\textit{Padre de las luces}: Stg 1:17.}, sino también que en su dirección de los asuntos interplanetarios <<\textit{no hay variabilidad ni sombra de cambio}>>\footnote{\textit{Sin variabilidad}: Is 25:1; Mal 3:6; Stg 1:17.}. Él <<\textit{proclama el fin desde el principio}>>\footnote{\textit{Proclama el fin desde el principio}: Is 46:10.}. Dice: <<\textit{Mi parecer perdurará; haré todo lo que me complace}>>\footnote{\textit{Mi parecer perdurará}: Is 46:10.} <<\textit{de acuerdo con el propósito eterno que me propuse en mi Hijo}>>\footnote{\textit{Propósito eterno}: Ef 3:11.}. Los planes y los propósitos de la Fuente-Centro Primera son pues como ella misma: eternos, perfectos y siempre invariables.

\par
%\textsuperscript{(35.6)}
\textsuperscript{2:2.2} Existe una perfección final y una plenitud completa en los mandatos del Padre. <<\textit{Todo lo que Dios hace será para siempre; no se puede añadir nada ni quitar nada}>>\footnote{\textit{Actos eternos y completos}: Ec 3:14.}. El Padre Universal no se arrepiente de sus propósitos originales de sabiduría y de perfección\footnote{\textit{Se ha dicho que Dios se arrepiente, pero no lo hace}: Gn 6:6-7; Ex 32:14; 1 Cr 21:15; Sal 106:45; Jer 18:8,10; 26:19; 42:10; Am 7:3,6; Jon 3:10; Jue 2:18; Heb 7:21; 1 Sam 15:35; 2 Sam 24:16. \textit{No se arrepiente (por elección)}: Sal 110:4; Jer 4:28; Ez 24:14; Zac 8:14. \textit{No se arrepiente (por naturaleza)}: Nm 23:19; 1 Sam 15:29.}. Sus planes son firmes, su parecer es inmutable, mientras que sus actos son divinos e infalibles\footnote{\textit{Dios es inmutable e infalible}: Sal 33:11; Jer 32:18-19; Heb 6:17.}. <<\textit{Mil años a sus ojos son como el día de ayer cuando ha pasado, y como una vigilia nocturna}>>\footnote{\textit{Dios es atemporal}: Sal 90:4.}. La perfección de la divinidad y la magnitud de la eternidad están para siempre más allá de la plena comprensión de la mente circunscrita del hombre mortal.

\par
%\textsuperscript{(36.1)}
\textsuperscript{2:2.3} Las reacciones de un Dios invariable, en la ejecución de su propósito eterno, pueden parecer que varían con arreglo a la actitud cambiante y a las mentes variables de las inteligencias que ha creado; es decir, que dichas reacciones pueden variar de manera aparente y superficial; pero por debajo de la superficie y debajo de todas las manifestaciones exteriores, continúa estando presente el propósito invariable, el plan perpetuo, del Dios eterno.

\par
%\textsuperscript{(36.2)}
\textsuperscript{2:2.4} Fuera, en los universos, la perfección ha de ser necesariamente un término relativo, pero en el universo central y especialmente en el Paraíso, la perfección es pura; en ciertas fases es incluso absoluta. Las manifestaciones de la Trinidad alteran la demostración de la perfección divina, pero no la atenúan.

\par
%\textsuperscript{(36.3)}
\textsuperscript{2:2.5} La perfección primordial de Dios no consiste en una rectitud ficticia, sino más bien en la perfección inherente de la bondad de su naturaleza divina. Él es final, completo y perfecto. A la belleza y a la perfección de su carácter recto no les falta nada. Todo el proyecto de las existencias vivientes en los mundos del espacio está centrado en el propósito divino de elevar a todas las criaturas volitivas hasta el alto destino de la experiencia de compartir la perfección paradisiaca del Padre. Dios no es ni egocéntrico ni autosuficiente; no deja nunca de darse a todas las criaturas conscientes de sí mismas en el inmenso universo de universos.

\par
%\textsuperscript{(36.4)}
\textsuperscript{2:2.6} Dios es eterna e infinitamente perfecto, no puede conocer personalmente la imperfección como experiencia propia, pero sí comparte la conciencia de toda la experiencia con la imperfección que tienen todas las criaturas que luchan en los universos evolutivos de todos los Hijos Creadores Paradisiacos. El toque personal y liberador del Dios de la perfección cubre con su sombra el corazón, y pone en su circuito la naturaleza, de todas aquellas criaturas mortales que se han elevado hasta el nivel universal del discernimiento moral. De esta manera, así como a través de los contactos de la presencia divina, el Padre Universal participa realmente en la experiencia \textit{con} la inmadurez y la imperfección en la carrera evolutiva de todos los seres morales del universo entero.

\par
%\textsuperscript{(36.5)}
\textsuperscript{2:2.7} Las limitaciones humanas, el mal potencial, no forman parte de la naturaleza divina, pero la experiencia humana \textit{con} el mal y todas las relaciones del hombre con él forman parte con toda seguridad de la autorrealización en constante expansión que Dios efectúa en los hijos del tiempo ---unas criaturas con responsabilidad moral que han sido creadas o desarrolladas por cada Hijo Creador que sale del Paraíso.

\section*{3. La justicia y la rectitud}
\par
%\textsuperscript{(36.6)}
\textsuperscript{2:3.1} Dios es recto; por consiguiente es justo. <<\textit{El Señor es recto en todos sus caminos}>>\footnote{\textit{Recto en todos sus caminos}: Sal 145:17.}. <<\textit{`No he hecho sin razón todo lo que he hecho', dice el Señor}>>\footnote{\textit{No actúa sin una causa}: Ez 14:23.}. <<\textit{Los juicios del Señor son totalmente verdaderos y rectos}>>\footnote{\textit{Juicios rectos y verdaderos}: Sal 19:9.}. La justicia del Padre Universal no puede ser influida por los actos ni las obras de sus criaturas, <<\textit{porque no hay iniquidad en el Señor nuestro Dios, ni acepción de personas, ni aceptación de regalos}>>\footnote{\textit{No hay iniquidad en Él}: 2 Cr 19:7. \textit{No hace acepción de personas}: 2 Cr 19:7; Job 34:19; Eclo 35:12; Hch 10:34; Ro 2:11; Gl 2:6; 3:28; Ef 6:9; Col 3:11.}.

\par
%\textsuperscript{(36.7)}
\textsuperscript{2:3.2} !`Cuán inútil es hacer peticiones pueriles a un Dios semejante para que modifique sus decretos inmutables a fin de que podamos evitar las justas consecuencias del funcionamiento de sus sabias leyes naturales y de sus justos mandatos espirituales! <<\textit{No os engañéis; uno no puede burlarse de Dios, porque aquello que un hombre siembra, eso también recogerá}>>\footnote{\textit{Se recoge lo que siembra}: Job 4:8; Gl 6:7.}. En verdad, incluso al recoger en justicia la cosecha de las maldades, esta justicia divina siempre está templada de misericordia. La sabiduría infinita es el árbitro eterno que determina las proporciones de justicia y de misericordia que se repartirán en cualquier circunstancia dada. El castigo más grande (que es en realidad una consecuencia inevitable) por la maldad y la rebelión deliberada contra el gobierno de Dios es la pérdida de la existencia como súbdito individual de ese gobierno. El resultado final del pecado deliberado es la aniquilación. A fin de cuentas, esos individuos identificados con el pecado se han destruido a sí mismos al volverse completamente irreales por haber abrazado la iniquidad. Sin embargo, la desaparición real de esas criaturas siempre se retrasa hasta que los mandatos ordenados de la justicia, vigentes en ese universo, se han cumplido plenamente.

\par
%\textsuperscript{(37.1)}
\textsuperscript{2:3.3} El cese de la existencia se decreta habitualmente en el momento del juicio dispensacional, o juicio de época, del planeta o de los planetas. En un mundo como Urantia tiene lugar al final de una dispensación planetaria. El cese de la existencia se puede decretar en esos momentos mediante la acción coordinada de todos los tribunales con jurisdicción que se extienden desde el consejo planetario, pasando por las cortes del Hijo Creador, hasta los tribunales de juicio de los Ancianos de los Días. El mandato de disolución parte de las cortes superiores del superuniverso después de una confirmación ininterrumpida de la acusación que se originó en la esfera de residencia del malhechor; luego, cuando la sentencia de extinción ha sido confirmada en las alturas, la ejecución se lleva a cabo mediante la acción directa de aquellos jueces que residen en la sede del superuniverso y que actúan desde allí.

\par
%\textsuperscript{(37.2)}
\textsuperscript{2:3.4} Cuando esta sentencia se confirma definitivamente, el ser identificado con el pecado se vuelve instantáneamente como si no hubiera existido\footnote{\textit{Como si no hubiera sido}: Abd 1:16.}. Este destino no conlleva ninguna resurrección; es perpetuo y eterno. Los factores energéticos vivientes de la identidad se disipan, mediante las transformaciones del tiempo y las metamorfosis del espacio, en los potenciales cósmicos de donde habían surgido anteriormente. En cuanto a la personalidad del ser inicuo, se queda privada de un vehículo vital continuo porque la criatura no ha logrado hacer aquellas elecciones ni ha tomado aquellas decisiones finales que le habrían asegurado la vida eterna. Cuando la mente asociada ha abrazado continuamente el pecado hasta el punto de culminar en una identificación completa del yo con la iniquidad, entonces, después del cese de la vida y de la disolución cósmica, esa personalidad aislada es absorbida en la superalma de la creación, volviéndose una parte de la experiencia evolutiva del Ser Supremo. Nunca más volverá a aparecer como una personalidad; su identidad se vuelve como si nunca hubiera existido. En el caso de una personalidad habitada por un Ajustador, los valores espirituales experienciales sobreviven en la realidad del Ajustador que sigue existiendo.

\par
%\textsuperscript{(37.3)}
\textsuperscript{2:3.5} En cualquier controversia universal entre los niveles manifestados de la realidad, la personalidad del nivel superior terminará por triunfar sobre la personalidad del nivel inferior. Este resultado inevitable de las controversias en el universo es inherente al hecho de que la calidad divina es igual al grado de realidad o de manifestación de cualquier criatura volitiva. El mal no diluido, el error completo, el pecado deliberado y la iniquidad rematada son inherente y automáticamente autodestructivos. Tales actitudes de irrealidad cósmica sólo pueden sobrevivir en el universo debido a una tolerancia misericordiosa transitoria, en espera de la acción de los mecanismos de los tribunales universales de juicio en rectitud, los cuales determinan la justicia y descubren lo que es equitativo.

\par
%\textsuperscript{(37.4)}
\textsuperscript{2:3.6} El deber de los Hijos Creadores en los universos locales consiste en crear y en espiritualizar. Estos Hijos se dedican a ejecutar eficazmente el plan paradisiaco de la ascensión progresiva de los mortales, a rehabilitar a los rebeldes y a los pensadores equivocados, pero cuando todos estos esfuerzos amorosos son rechazados de manera definitiva y para siempre, las fuerzas que actúan bajo la jurisdicción de los Ancianos de los Días ejecutan el decreto final de disolución.

\section*{4. La misericordia divina}
\par
%\textsuperscript{(38.1)}
\textsuperscript{2:4.1} La misericordia es simplemente la justicia, templada por esa sabiduría que procede del conocimiento perfecto y del pleno reconocimiento de la debilidad natural y de los obstáculos ambientales de las criaturas finitas. <<\textit{Nuestro Dios está lleno de compasión, es benévolo, paciente y abundante en misericordia}>>\footnote{\textit{Dios de la compasión y la misericordia}: Sal 145:8; 86:15.}. Por eso <<\textit{quienquiera que recurra al Señor será salvado}>>\footnote{\textit{Quien recurra a Él será salvado}: Sal 50:15; Jl 2:32; Zac 13:9; Hch 2:21; Ro 10:13.}, <<\textit{porque él perdonará en abundancia}>>\footnote{\textit{Perdonará en abundancia}: Is 55:7.}. <<\textit{La misericordia del Señor va de eternidad en eternidad}>>; sí, <<\textit{su misericordia perdura para siempre}>>\footnote{\textit{Su misericordia dura eternamente}: 1 Cr 16:34,41; 2 Cr 5:13; 7:3,6; Sal 100:5; 103:17; 107:1; 118:1-4; 136:1-26; Is 54:8.}. <<\textit{Yo soy el Señor que lleva a cabo la bondad, el juicio y la rectitud en la Tierra, porque me deleito en estas cosas}>>\footnote{\textit{El que lleva a cabo la bondad}: Jer 9:24.}. <<\textit{No aflijo voluntariamente ni apeno a los hijos de los hombres}>>\footnote{\textit{No aflijo a propósito}: Lm 3:33.}, porque yo soy <<\textit{el Padre de las misericordias y el Dios de todo consuelo}>>\footnote{\textit{Padre de las misericordias}: 2 Co 1:3.}.

\par
%\textsuperscript{(38.2)}
\textsuperscript{2:4.2} Dios es inherentemente bondadoso, compasivo por naturaleza y perpetuamente misericordioso. Nunca es necesario ejercer ninguna influencia sobre el Padre para suscitar su bondad. La necesidad de las criaturas es enteramente suficiente para asegurar todo el caudal de la tierna misericordia del Padre y de su gracia salvadora. Puesto que Dios lo sabe todo acerca de sus hijos, le resulta fácil perdonar. Cuanto mejor comprende el hombre a su prójimo, más fácil le resulta perdonarlo, e incluso amarlo.

\par
%\textsuperscript{(38.3)}
\textsuperscript{2:4.3} Sólo el discernimiento de una sabiduría infinita permite a un Dios recto administrar la justicia y la misericordia al mismo tiempo y en cualquier situación dada del universo. El Padre celestial nunca se siente desgarrado por actitudes conflictivas hacia sus hijos del universo; Dios nunca es víctima de antagonismos en su actitud. La omnisciencia de Dios dirige infaliblemente su libre albedrío en la elección de esa conducta universal que satisface de manera perfecta, simultánea y por igual las exigencias de todos sus atributos divinos y las cualidades infinitas de su naturaleza eterna.

\par
%\textsuperscript{(38.4)}
\textsuperscript{2:4.4} La misericordia es el fruto natural e inevitable de la bondad y del amor. La naturaleza bondadosa de un Padre amoroso no podría negar de ninguna manera el sabio ministerio de la misericordia a cada miembro de cada grupo de sus hijos del universo. La justicia eterna y la misericordia divina unidas constituyen lo que en la experiencia humana se llamaría \textit{equidad.}

\par
%\textsuperscript{(38.5)}
\textsuperscript{2:4.5} La misericordia divina representa una técnica de equidad para ajustar los niveles de perfección y de imperfección del universo. La misericordia es la justicia de la Supremacía adaptada a las situaciones de lo finito en evolución, la rectitud de la eternidad modificada para satisfacer los intereses superiores y el bienestar universal de los hijos del tiempo. La misericordia no es una violación de la justicia, sino más bien una interpretación comprensiva de las exigencias de la justicia suprema, tal como ésta es aplicada con equidad a los seres espirituales subordinados y a las criaturas materiales de los universos evolutivos. La misericordia es la justicia de la Trinidad del Paraíso, aplicada con sabiduría y amor a las múltiples inteligencias de las creaciones del tiempo y del espacio, tal como esta justicia es formulada por la sabiduría divina y determinada por la mente omnisciente y el libre albedrío soberano del Padre Universal y de todos sus Creadores asociados.

\section*{5. El amor de Dios}
\par
%\textsuperscript{(38.6)}
\textsuperscript{2:5.1} <<\textit{Dios es amor}>>\footnote{\textit{Dios es amor}: 1 Jn 4:8,16.}; por eso su única actitud personal hacia los asuntos del universo es siempre una reacción de afecto divino. El Padre nos ama lo suficiente como para concedernos su vida. <<\textit{Hace salir su Sol sobre los malos y los buenos, y envía su lluvia sobre los justos y los injustos}>>\footnote{\textit{Hace salir el sol sobre buenos y malos}: Mt 5:45.}.

\par
%\textsuperscript{(39.1)}
\textsuperscript{2:5.2} Es falso pensar que los sacrificios de sus Hijos o la intercesión de sus criaturas subordinadas convenzan a Dios para que ame a sus hijos, <<\textit{porque el Padre mismo os ama}>>\footnote{\textit{El Padre te ama}: Jn 16:27.}. En respuesta a este afecto paternal, Dios envía a los maravillosos Ajustadores para que residan en la mente de los hombres. El amor de Dios es universal; <<\textit{cualquiera que lo desee puede venir}>>\footnote{\textit{Quienquiera puede venir}: Sal 50:15; Jl 2:32; Zac 13:9; Mt 7:24; 10:32-33; 12:50; 16:24-25; Mc 3:35; 8:34-35; Lc 6:47; 9:23-24; 12:8; Jn 3:15-16; 4:13-14; 11:25-26; 12:46; Hch 2:21; 10:42-43; 13:26; Ro 9:33; 10:13; 1 Jn 2:23; 4:15; 5:1; Ap 22:17b.}. Él querría <<\textit{que todos los hombres se salvaran por medio del conocimiento de la verdad}>>\footnote{\textit{¿Se salvarán todos los hombres?}: 1 Ti 2:4.}. <<\textit{No desea que ninguno perezca}>>\footnote{\textit{No desea que nadie perezca}: 2 P 3:9.}.

\par
%\textsuperscript{(39.2)}
\textsuperscript{2:5.3} Los Creadores son los primeros que intentan salvar al hombre de los resultados desastrosos de sus insensatas transgresiones de las leyes divinas. El amor de Dios es por naturaleza un afecto paternal; por eso a veces <<\textit{nos castiga por nuestro propio bien, para que podamos compartir su santidad}>>\footnote{\textit{Nos castiga por nuestro bien}: Heb 12:10.}. Incluso durante vuestras pruebas más duras, recordad que <<\textit{en todas nuestras aflicciones, está afligido con nosotros}>>\footnote{\textit{En nuestras aflicciones se aflige}: Is 53:5; 63:9.}.

\par
%\textsuperscript{(39.3)}
\textsuperscript{2:5.4} Dios es divinamente bondadoso con los pecadores. Cuando los rebeldes vuelven a la rectitud, son recibidos con misericordia, <<\textit{porque nuestro Dios perdonará en abundancia}>>\footnote{\textit{Perdona en abundancia}: Is 55:7.}. <<\textit{Yo soy aquel que borra vuestras transgresiones por mi propia complacencia, y no me acordaré de vuestros pecados}>>\footnote{\textit{Borra nuestros pecados}: Is 43:25.}. <<\textit{Mirad la clase de amor que el Padre nos ha otorgado para que nos llamen hijos de Dios}>>\footnote{\textit{Somos hijos de Dios}: 1 Cr 22:10; Sal 2:7; Is 56:5; Mt 5:9,16,45; Lc 20:36; Jn 1:12-13; 11:52; Hch 17:28-29; Ro 8:14-17,19,21; 9:26; 2 Co 6:18; Gl 3:26; 4:5-7; Ef 1:5; Flp 2:15; Heb 12:5-8; 1 Jn 3:1-2,10; 5:2; Ap 21:7; 2 Sam 7:14.}.

\par
%\textsuperscript{(39.4)}
\textsuperscript{2:5.5} Después de todo, la prueba más grande de la bondad de Dios y la razón suprema para amarlo es el don interior del Padre ---el Ajustador que espera tan pacientemente la hora en que él y vosotros seréis eternamente una sola cosa. Aunque no podáis encontrar a Dios por medio de la investigación, si os sometéis a las directrices del espíritu interior, seréis guiados infaliblemente paso a paso, vida tras vida, de un universo a otro, y era tras era, hasta que os encontréis finalmente en la presencia de la personalidad paradisiaca del Padre Universal.

\par
%\textsuperscript{(39.5)}
\textsuperscript{2:5.6} Cuán irrazonable es que no adoréis a Dios porque las limitaciones de la naturaleza humana y los obstáculos de vuestra creación material os impiden verlo. Entre vosotros y Dios hay una enorme distancia (de espacio físico) que hay que atravesar\footnote{\textit{Un gran espacio entre Dios y nosotros}: Lc 16:26.}. Existe igualmente un gran abismo de diferencia espiritual que hay que colmar; pero a pesar de todo lo que os separa física y espiritualmente de la presencia personal de Dios en el Paraíso, deteneos a reflexionar sobre el hecho solemne de que Dios vive dentro de vosotros; a su propia manera ya ha tendido un puente sobre el abismo. Ha enviado de sí mismo su espíritu para que viva en vosotros\footnote{\textit{Su espíritu vive en nosotros}: Job 32:8,18; Is 63:10-11; Ez 37:14; Mt 10:20; Lc 17:21; Jn 17:21-23; Ro 8:9-11; 1 Co 3:16-17; 6:19; 2 Co 6:16; Gl 2:20; 1 Jn 3:24; 4:12-15; Ap 21:3.} y trabaje con vosotros mientras continuáis vuestra carrera eterna en el universo.

\par
%\textsuperscript{(39.6)}
\textsuperscript{2:5.7} Encuentro fácil y agradable adorar a alguien que es tan grande, y que al mismo tiempo se dedica tan afectuosamente al ministerio de elevar a sus humildes criaturas. Amo de manera natural a alguien que es tan poderoso como para crear y controlar su creación, y que sin embargo es tan perfecto en su bondad y tan fiel en la benevolencia que nos cubre constantemente con su sombra\footnote{\textit{Amar su benevolencia}: Sal 17:7; Jer 9:24; Os 2:19.}. Creo que amaría a Dios de igual forma si no fuera tan grande ni tan poderoso, con tal que siga siendo tan bueno y misericordioso\footnote{\textit{Dios es bondadoso}: Ex 34:6; Ro 2:4. \textit{Dios es misericordioso}: Ex 20:6; 1 Cr 16:34,41; 2 Cr 5:13; 7:3,6; 30:9; Esd 3:11; Sal 25:6; 36:5; 86:15; 100:5; 103:8,17; 107:1; 116:5; 117:2; 118:1,4; 136:1-26; 145:8; Is 54:8; 55:7; Jer 3:12; Nm 14:18-19; Miq 7:18; Dt 4:31; 5:10; Heb 8:12.}. Todos amamos más al Padre por su naturaleza que en reconocimiento de sus atributos asombrosos.

\par
%\textsuperscript{(39.7)}
\textsuperscript{2:5.8} Cuando observo a los Hijos Creadores y a sus administradores subordinados luchando tan valientemente contra las múltiples dificultades del tiempo inherentes a la evolución de los universos del espacio, descubro que tengo un afecto grande y profundo por esos gobernantes menores de los universos. Después de todo, creo que todos nosotros, incluídos los mortales de los mundos, amamos al Padre Universal y a todos los demás seres divinos o humanos porque percibimos que esas personalidades nos aman verdaderamente. La experiencia de amar es en gran medida una respuesta directa a la experiencia de ser amado. Sabiendo que Dios me ama, debería continuar amándolo de manera suprema, aunque estuviera despojado de todos sus atributos de supremacía, ultimidad y absolutidad.

\par
%\textsuperscript{(40.1)}
\textsuperscript{2:5.9} El amor del Padre nos sigue ahora y a lo largo del círculo sin fin de las eras eternas. Cuando meditéis sobre la naturaleza amorosa de Dios, sólo hay una reacción razonable y natural de la personalidad: amaréis a vuestro Hacedor cada vez más; tendréis por Dios un afecto análogo al que un niño siente por su padre terrestre; porque al igual que un padre, un padre real, un verdadero padre, ama a sus hijos, el Padre Universal ama a sus hijos e hijas creados y busca constantemente su bienestar.

\par
%\textsuperscript{(40.2)}
\textsuperscript{2:5.10} Pero el amor de Dios es un afecto parental inteligente y previsor. El amor divino actúa en asociación unificada con la sabiduría divina y con todas las otras características infinitas de la naturaleza perfecta del Padre Universal. Dios es amor\footnote{\textit{Dios es amor}: 1 Jn 4:8,16.}, pero el amor no es Dios. La mayor manifestación del amor divino por los seres mortales se puede observar en la concesión de los Ajustadores del Pensamiento, pero vuestra mayor revelación del amor del Padre se puede contemplar en la vida de donación de su Hijo Miguel cuando vivió en la Tierra la vida espiritual ideal. El Ajustador interior es el que individualiza el amor de Dios para cada alma humana.

\par
%\textsuperscript{(40.3)}
\textsuperscript{2:5.11} A veces casi me apena verme obligado a describir el afecto divino del Padre celestial por sus hijos del universo utilizando el símbolo verbal humano \textit{amor.} Aunque este término conlleva el concepto más elevado que tiene el hombre sobre las relaciones humanas de respeto y de devoción, designa con tanta frecuencia tantas cosas de las relaciones humanas, que es completamente innoble y totalmente inadecuado que sean conocidas con una palabra que se utiliza también para indicar el afecto incomparable del Dios viviente por sus criaturas del universo. Es lamentable que no pueda utilizar un término exclusivo y celestial que pudiera transmitir a la mente del hombre la verdadera naturaleza y el significado exquisitamente hermoso del afecto divino del Padre Paradisiaco.

\par
%\textsuperscript{(40.4)}
\textsuperscript{2:5.12} Cuando el hombre pierde de vista el amor de un Dios personal, el reino de Dios se vuelve simplemente el reino del bien. A pesar de la unidad infinita de la naturaleza divina, el amor es la característica dominante de todas las relaciones personales de Dios con sus criaturas.

\section*{6. La bondad de Dios}
\par
%\textsuperscript{(40.5)}
\textsuperscript{2:6.1} La belleza divina la podemos ver en el universo físico, la verdad eterna podemos discernirla en el mundo intelectual, pero la bondad de Dios se encuentra solamente en el mundo espiritual de la experiencia religiosa personal. La religión es, en su verdadera esencia, una fe mezclada de confianza en la bondad de Dios. En la filosofía, Dios podría ser grande y absoluto, e incluso de algún modo inteligente y personal; pero en la religión Dios ha de ser también moral; debe ser bueno. El hombre podría temer a un gran Dios, pero sólo ama y tiene confianza en un Dios bueno. Esta bondad de Dios forma parte de la personalidad de Dios, y su plena revelación sólo aparece en la experiencia religiosa personal de los hijos creyentes de Dios.

\par
%\textsuperscript{(40.6)}
\textsuperscript{2:6.2} La religión implica que el mundo superior de naturaleza espiritual tiene conocimiento de las necesidades fundamentales del mundo humano, y responde a ellas. La religión evolutiva puede volverse ética, pero sólo la religión revelada se vuelve verdadera y espiritualmente moral. El antiguo concepto de que Dios es una Deidad dominada por una moralidad regia fue elevado por Jesús hasta el nivel afectuosamente conmovedor de la moralidad familiar íntima de la relación entre padres e hijos, no existiendo ninguna más tierna ni más hermosa en la experiencia de los mortales.

\par
%\textsuperscript{(41.1)}
\textsuperscript{2:6.3} La <<\textit{abundancia de la bondad de Dios conduce al hombre equivocado al arrepentimiento}>>\footnote{\textit{Bondad de Dios}: Ex 34:6; Ro 2:4.}. <<\textit{Todo don bueno y todo don perfecto proceden del Padre de las luces}>>\footnote{\textit{Todo don bueno y perfecto}: Stg 1:17.}. <<\textit{Dios es bueno; es el refugio eterno del alma de los hombres}>>\footnote{\textit{Dios es bueno}: Sal 34:8; 73:1; Jer 33:11; Lm 3:25; Nah 1:7. \textit{Es nuestro refugio}: Dt 33:27.}. <<\textit{El Señor Dios es misericordioso y benevolente. Es paciente y abunda en bondad y en verdad}>>\footnote{\textit{Misericordioso y benevolente}: Ex 34:6.}. <<\textit{!`Probad y ved que el Señor es bueno! Bendito sea el hombre que confía en él}>>\footnote{\textit{Probad y ved}: Sal 34:8. \textit{Bendito el que confía en Él}: Sal 2:12; 40:4; 84:12; Jer 17:7.}. <<\textit{El Señor es bondadoso y está lleno de compasión. Es el Dios de la salvación}>>\footnote{\textit{Dios es bondadoso}: Sal 111:4; 145:8. \textit{Dios de salvación}: Ex 15:2; 1 Cr 16:35; Job 13:16; Sal 18:2,46; 24:5; 25:5; 27:9; 51:14; 65:5; 68:19-20; 79:9; 85:4; 88:1; Is 12:2; 17:10; Miq 7:7; Hab 3:18; Flp 1:19; 2 Sam 22:3,47.}. <<\textit{Cura los corazones destrozados y venda las heridas del alma. Es el benefactor todopoderoso del hombre}>>\footnote{\textit{Cura los corazones}: Sal 147:3. \textit{Es todopoderoso}: 1 Cr 29:11-12; Sof 3:17.}.

\par
%\textsuperscript{(41.2)}
\textsuperscript{2:6.4} Aunque el concepto de Dios como rey-juez fomentó un nivel moral elevado y creó un pueblo respetuoso de la ley como grupo, dejaba al creyente individual en una triste posición de inseguridad respecto a su condición en el tiempo y en la eternidad. Los profetas hebreos más tardíos proclamaron que Dios era un Padre para Israel; Jesús reveló a Dios como el Padre de cada ser humano. Todo el concepto humano de Dios está iluminado de manera trascendente por la vida de Jesús. El desinterés es inherente al amor parental. Dios no ama \textit{igual} que un padre, sino \textit{como} un padre. Él es el Padre Paradisiaco de cada personalidad del universo.

\par
%\textsuperscript{(41.3)}
\textsuperscript{2:6.5} La rectitud implica que Dios es la fuente de la ley moral del universo. La verdad muestra a Dios como revelador, como instructor. Pero el amor da afecto y lo desea ardientemente, busca una comunión comprensiva tal como la que existe entre padres e hijos. La rectitud puede ser el pensamiento divino, pero el amor es la actitud de un padre. La suposición errónea de que la rectitud de Dios era incompatible con el amor desinteresado del Padre celestial presuponía una falta de unidad en la naturaleza de la Deidad, y condujo directamente a la elaboración de la doctrina de la expiación, que es un ataque filosófico tanto a la unidad como al libre albedrío de Dios.

\par
%\textsuperscript{(41.4)}
\textsuperscript{2:6.6} El afectuoso Padre celestial, cuyo espíritu reside en sus hijos de la Tierra, no es una personalidad dividida ---una de justicia y otra de misericordia--- ni tampoco se necesita un mediador para conseguir el favor o el perdón del Padre. La rectitud divina no está dominada por una estricta justicia retributiva; Dios como padre trasciende a Dios como juez.

\par
%\textsuperscript{(41.5)}
\textsuperscript{2:6.7} Dios nunca es vengativo, ni está colérico ni enojado. Es verdad que la sabiduría refrena a menudo su amor, a la vez que la justicia condiciona su misericordia cuando ésta es rechazada. Su amor por la rectitud no puede evitar manifestarse como un odio equivalente por el pecado. El Padre no es una personalidad contradictoria; la unidad divina es perfecta. Existe una unidad absoluta en la Trinidad del Paraíso, a pesar de la identidad eterna de los correlacionados de Dios.

\par
%\textsuperscript{(41.6)}
\textsuperscript{2:6.8} Dios ama al pecador y \textit{detesta} el pecado: esta afirmación es filosóficamente cierta, pero Dios es una personalidad trascendente, y las personas sólo pueden amar y odiar a otras personas. El pecado no es una persona. Dios ama al pecador porque es una realidad personal
(potencialmente eterna), mientras que Dios no adopta ninguna actitud personal hacia el pecado, porque el pecado no es una realidad espiritual; no es personal; por eso sólo la justicia de Dios tiene en cuenta su existencia. El amor de Dios salva al pecador; la ley de Dios destruye el pecado. Esta actitud de la naturaleza divina cambiaría en apariencia si el pecador terminara por identificarse totalmente con el pecado, al igual que esta misma mente mortal puede identificarse plenamente también con el Ajustador espiritual interior. La naturaleza de un mortal identificado así con el pecado se volvería entonces completamente antiespiritual (y, por tanto, personalmente irreal) y experimentaría la extinción final de su ser. La irrealidad, e incluso el estado incompleto de la naturaleza de las criaturas, no pueden existir para siempre en un universo que progresa en realidad y que crece en espiritualidad.

\par
%\textsuperscript{(42.1)}
\textsuperscript{2:6.9} De cara al mundo de la personalidad, se descubre que Dios es una persona amorosa; de cara al mundo espiritual, es un amor personal; en la experiencia religiosa es las dos cosas. El amor identifica la voluntad volitiva de Dios. La bondad de Dios descansa en el fondo del libre albedrío divino ---la tendencia universal a amar, a mostrar misericordia, a manifestar paciencia y a ofrecer el perdón.

\section*{7. La verdad y la belleza divinas}
\par
%\textsuperscript{(42.2)}
\textsuperscript{2:7.1} Todo conocimiento finito y toda comprensión por parte de las criaturas son \textit{relativos.} La información y los datos, aunque procedan de fuentes elevadas, sólo son relativamente completos, localmente exactos y personalmente verdaderos.

\par
%\textsuperscript{(42.3)}
\textsuperscript{2:7.2} Los hechos físicos son bastante uniformes, pero la verdad es un factor viviente y flexible en la filosofía del universo. Las comunicaciones de las personalidades evolutivas sólo son parcialmente sabias y relativamente verídicas. Sólo pueden estar seguras dentro de lo que alcanza su experiencia personal. Aquello que puede parecer enteramente cierto en un lugar, sólo puede ser relativamente cierto en otro segmento de la creación.

\par
%\textsuperscript{(42.4)}
\textsuperscript{2:7.3} La verdad divina, la verdad final, es uniforme y universal, pero la historia de las cosas espirituales, tal como la cuentan numerosas personalidades procedentes de esferas diversas, puede variar a veces en los detalles debido a esta relatividad en la totalidad del conocimiento y en la plenitud de la experiencia personal, así como en la longitud y la extensión de esa experiencia. Las leyes y los decretos, los pensamientos y las actitudes de la Gran Fuente-Centro Primera son eterna, infinita y universalmente verdaderos, pero al mismo tiempo su aplicación y su adaptación a cada universo, sistema, mundo e inteligencia creada concuerdan con los planes y la técnica de los Hijos Creadores tal como éstos actúan en sus respectivos universos, y también están en armonía con los planes y los procedimientos locales del Espíritu Infinito y de todas las demás personalidades celestiales asociadas.

\par
%\textsuperscript{(42.5)}
\textsuperscript{2:7.4} La falsa ciencia del materialismo condenaría al hombre mortal a convertirse en un proscrito en el universo. Un conocimiento así de parcial es potencialmente malo; es un conocimiento compuesto a la vez de bien y de mal. La verdad es hermosa porque es a la vez completa y simétrica. Cuando el hombre busca la verdad, persigue aquello que es divinamente real.

\par
%\textsuperscript{(42.6)}
\textsuperscript{2:7.5} Los filósofos cometen su error más grave cuando se extravían en el sofisma de la abstracción, en la práctica de enfocar la atención sobre un aspecto de la realidad, y luego declarar que ese aspecto aislado constituye la verdad total. El filósofo sabio buscará siempre el propósito creativo que se encuentra detrás de, y es anterior a, todos los fenómenos del universo. El pensamiento del creador precede invariablemente a la acción creativa.

\par
%\textsuperscript{(42.7)}
\textsuperscript{2:7.6} La conciencia intelectual puede descubrir la belleza de la verdad, su calidad espiritual, no sólo por la coherencia filosófica de sus conceptos, sino con más certeza y seguridad por la respuesta infalible del Espíritu de la Verdad siempre presente. La felicidad es el resultado del reconocimiento de la verdad porque ésta puede \textit{exteriorizarse;} puede vivirse. La decepción y la tristeza acompañan al error porque, como éste no es una realidad, no se puede llevar a cabo en la experiencia. La verdad divina se conoce mejor por su \textit{sabor espiritual}\footnote{\textit{Sabor espiritual}: Sal 34:8; 119:103.}.

\par
%\textsuperscript{(42.8)}
\textsuperscript{2:7.7} La búsqueda eterna es con vistas a la unificación, a la coherencia divina. El extenso universo físico encuentra su coherencia en la Isla del Paraíso; el universo intelectual halla su coherencia en el Dios de la mente, el Actor Conjunto; el universo espiritual es coherente en la personalidad del Hijo Eterno. Pero los mortales aislados del tiempo y del espacio encuentran su coherencia en Dios Padre a través de la relación directa entre el Ajustador del Pensamiento interior y el Padre Universal. El Ajustador del hombre es un fragmento de Dios y busca perpetuamente la unificación divina; es coherente con la Deidad Paradisiaca de la Fuente-Centro Primera, y en ella.

\par
%\textsuperscript{(43.1)}
\textsuperscript{2:7.8} Discernir la belleza suprema es descubrir e integrar la realidad: Discernir la bondad divina en la verdad eterna, esa es la belleza última. Incluso el encanto del arte humano consiste en la armonía de su unidad.

\par
%\textsuperscript{(43.2)}
\textsuperscript{2:7.9} El gran error de la religión hebrea consistió en que no logró asociar la bondad de Dios con las verdades objetivas de la ciencia y la belleza atractiva del arte. A medida que la civilización progresaba, y puesto que la religión insistía en seguir el mismo camino insensato de acentuar con exceso la bondad de Dios, excluyendo relativamente la verdad y descuidando la belleza, ciertos tipos de hombres desarrollaron una tendencia creciente a desviarse del concepto abstracto y disociado de la bondad aislada. La moralidad aislada y exagerada de la religión moderna, que no logra retener la devoción y la lealtad de muchos hombres del siglo veinte, se rehabilitaría si, además de sus mandatos morales, concediera una consideración equivalente a las verdades de la ciencia, la filosofía y la experiencia espiritual, a las bellezas de la creación física, al encanto del arte intelectual y a la grandeza de la consecución de un carácter auténtico.

\par
%\textsuperscript{(43.3)}
\textsuperscript{2:7.10} El desafío religioso de la época actual es para aquellos hombres y mujeres previsores, con visión de futuro y con perspicacia espiritual, que se atrevan a construir una nueva y atrayente filosofía de la vida a partir de los conceptos modernos ampliados y exquisitamente integrados de la verdad cósmica, la belleza universal y la bondad divina. Una visión así nueva y justa de la moralidad atraerá todo lo que hay de bueno en la mente del hombre y desafiará lo que hay de mejor en el alma humana. La verdad, la belleza y la bondad son realidades divinas, y a medida que el hombre asciende la escala de la vida espiritual, estas cualidades supremas del Eterno se coordinan y se unifican cada vez más en Dios, que es amor.

\par
%\textsuperscript{(43.4)}
\textsuperscript{2:7.11} Toda verdad ---material, filosófica o espiritual--- es a la vez bella y buena. Toda belleza real ---el arte material o la simetría espiritual--- es a la vez verdadera y buena. Toda bondad auténtica ---ya se trate de la moralidad personal, la equidad social o el ministerio divino--- es igualmente verdadera y bella. La salud, la cordura y la felicidad son integraciones de la verdad, la belleza y la bondad tal como se encuentran combinadas en la experiencia humana. Estos niveles de vida eficaz llegan a conseguirse mediante la unificación de los sistemas energéticos, los sistemas de las ideas y los sistemas del espíritu.

\par
%\textsuperscript{(43.5)}
\textsuperscript{2:7.12} La verdad es coherente, la belleza es atractiva y la bondad es estabilizadora. Cuando estos valores de lo que es real se coordinan en la experiencia de la personalidad, el resultado es un elevado tipo de amor condicionado por la sabiduría y capacitado por la lealtad. La verdadera finalidad de toda la educación en el universo consiste en coordinar de la mejor manera a los hijos aislados de los mundos con las realidades más amplias de su experiencia en expansión. La realidad es finita en el nivel humano, y es infinita y eterna en los niveles superiores y divinos.

\par
%\textsuperscript{(43.6)}
\textsuperscript{2:7.13} [Presentado por un Consejero Divino, que actúa por autoridad de los Ancianos de los Días de Uversa]


\chapter{Documento 3. Los atributos de Dios}
\par
%\textsuperscript{(44.1)}
\textsuperscript{3:0.1} Dios está presente en todas partes; el Padre Universal gobierna el círculo de la eternidad. Pero en los universos locales gobierna por medio de las personas de sus Hijos Creadores Paradisiacos, al igual que concede la vida a través de estos Hijos. <<\textit{Dios nos ha dado la vida eterna, y esta vida se encuentra en sus Hijos}>>\footnote{\textit{Vida eterna}: Dn 12:2; Mt 19:16,29; 25:46; Mc 10:17,30; Lc 10:25; 18:18,30; Jn 3:15-16,36; 4:14,36; 5:24,39; 6:27,40,47; 6:54,68; 8:51-52; 10:28; 11:25-26; 12:25,50; 17:2-3; Hch 13:46-48; Ro 2:7; 5:21; 6:22-23; Gl 6:8; 1 Ti 1:16; 6:12,19; Tit 1:2; 3:7; 1 Jn 1:2; 2:25; 3:15; 5:13,20; Jud 1:21; Ap 22:5. \textit{Dios da la vida a través de sus Hijos}: 1 Jn 5:11-12.}. Estos Hijos Creadores de Dios son la expresión personal de él mismo en los sectores del tiempo y para los hijos de los planetas que giran en los universos evolutivos del espacio.

\par
%\textsuperscript{(44.2)}
\textsuperscript{3:0.2} Las órdenes inferiores de inteligencias creadas pueden percibir claramente a los Hijos de Dios altamente personalizados, y éstos compensan así la invisibilidad del Padre, que es infinito, y por lo tanto menos perceptible. Los Hijos Creadores Paradisiacos del Padre Universal son una revelación de un ser que, por otra parte, es invisible, y es invisible a causa de la absolutidad y de la infinidad inherentes al círculo de la eternidad y a las personalidades de las Deidades del Paraíso.

\par
%\textsuperscript{(44.3)}
\textsuperscript{3:0.3} La facultad de crear no es exactamente un atributo de Dios; es más bien el conjunto de su naturaleza activa. Y esta función universal creadora se manifiesta eternamente a medida que es condicionada y controlada por todos los atributos coordinados de la realidad divina e infinita de la Fuente-Centro Primera. Ponemos sinceramente en duda que una característica cualquiera de la naturaleza divina pueda ser considerada como anterior a las demás, pero si éste fuera el caso, entonces la naturaleza creadora de la Deidad tendría prioridad sobre todas sus demás naturalezas, actividades y atributos. Y la facultad creadora de la Deidad culmina en la verdad universal de la Paternidad de Dios.

\section*{1. La omnipresencia de Dios}
\par
%\textsuperscript{(44.4)}
\textsuperscript{3:1.1} La capacidad del Padre Universal para estar presente al mismo tiempo en todas partes constituye su omnipresencia. Sólo Dios puede estar en dos lugares, o en una multitud de lugares, a la vez. Dios está simultáneamente presente <<\textit{arriba en los cielos y abajo en la Tierra}>>\footnote{\textit{Arriba en el Cielo y abajo en la Tierra}: Dt 3:24; 4:39; Jos 2:11.}; tal como el salmista exclamó: <<\textit{¿Adónde iré lejos de tu espíritu? o ¿adónde huiré de tu presencia?}>>\footnote{\textit{¿Adónde huir de Dios?}: Sal 139:7.}.

\par
%\textsuperscript{(44.5)}
\textsuperscript{3:1.2} <<\textit{Soy un Dios al alcance de la mano, y también muy lejano}>>\footnote{\textit{Dios a mano y lejos}: Jer 23:23-24.}, dice el Señor. <<\textit{¿Acaso no lleno los cielos y la Tierra?}>>. El Padre Universal está constantemente presente en todas las partes y en todos los corazones de su extensa creación. Él es <<\textit{la plenitud de aquel que lo llena todo en todo}>>\footnote{Es todo en todo: Hch 17:28; Ro 11:36; 1 Co 8:6; 12:6; 15:28; Ef 1:23; 4:6; Col 1:17; 3:11; Heb 2:10-11. \textit{Lo llena todo en todo}: Ef 1:23; 4:10.}, y <<\textit{que lo efectúa todo en todo}>>\footnote{\textit{Hace todo en todo}: 1 Co 12:6.}, y además el concepto de su personalidad es tal, que <<\textit{el cielo (el universo) y el cielo de los cielos (el universo de universos) no pueden contenerlo}>>\footnote{\textit{El cielo de los cielos}: 1 Re 8:27; 2 Cr 2:6; 6:18; Neh 9:6; Sal 148:4; Dt 10:14.}. Es literalmente cierto que Dios lo es todo y se encuentra en todo, pero ni siquiera esto es \textit{la totalidad} de Dios. Sólo la infinidad puede revelar finalmente al Infinito; la causa nunca puede ser plenamente comprendida por un análisis de los efectos; el Dios vivo es inconmensurablemente más grande que la suma total de la creación que ha surgido a la existencia como resultado de los actos creativos de su libre albedrío sin trabas. Dios está revelado en todo el cosmos, pero el cosmos nunca podrá contener o abarcar la totalidad de la infinidad de Dios.

\par
%\textsuperscript{(45.1)}
\textsuperscript{3:1.3} La presencia del Padre patrulla sin cesar el universo maestro. <<\textit{Aparece por el principio de los cielos y da la vuelta hasta el final de éstos; y no hay nada que pueda ocultarse a su luz}>>\footnote{\textit{Viene del cielo}: Sal 19:6.}.

\par
%\textsuperscript{(45.2)}
\textsuperscript{3:1.4} La criatura no solamente existe en Dios, sino que Dios vive también en la criatura. <<\textit{Sabemos que vivimos en él porque él vive en nosotros; nos ha dado su espíritu. Este don del Padre Paradisiaco es el compañero inseparable del hombre}>>\footnote{\textit{Vive en nosotros}: Job 32:8,18; Is 63:10-11; Ez 36:27; 37:14; Mt 10:20; Lc 17:21; Jn 17:21-23; Ro 8:9-11; 1 Co 3:16-17; 6:19; 2 Co 6:16; Gl 2:20; 2 Ti 1:14; 1 Jn 3:24; 4:12-15; Ap 21:3. \textit{El don del Padre}: 2 Co 9:15. \textit{Dona a un compañero}: Ro 6:23.}. <<\textit{Es el Dios siempre presente que lo impregna todo}>>\footnote{\textit{Dios siempre presente}: Sal 46:1; Mt 28:20; Lc 17:21.}. <<\textit{El espíritu del Padre eterno está escondido en la mente de cada hijo mortal}>>. <<\textit{El hombre sale en busca de un amigo, cuando ese mismo amigo vive dentro de su propio corazón}>>. <<\textit{El verdadero Dios no está lejos, forma parte de nosotros, su espíritu habla desde nuestro interior}>>\footnote{\textit{No está lejos}: Jer 23:23. \textit{Forma parte de nosotros}: 1 Jn 4:4,13,16. \textit{Su espíritu habla desde el interior}: Ez 36:27; Mt 10:20; Jn 3:34; Hch 2:4.}. <<\textit{El Padre vive en el hijo. Dios siempre está con nosotros. Él es el espíritu guía del destino eterno}>>\footnote{\textit{El Padre vive en el hijo}: Jn 14:10-11,20. \textit{Dios siempre está con nosotros}: Mt 28:20. \textit{El espíritu guía}: Jn 14:16-18,26. \textit{Nunca nos abandona}: 1 Re 6:13; Dt 4:31; 31:6,8; 1 Sam 12:22.}.

\par
%\textsuperscript{(45.3)}
\textsuperscript{3:1.5} Se ha dicho con razón de la raza humana: <<\textit{Sois de Dios}>>\footnote{\textit{Sois de Dios}: 1 Jn 4:4,6; 5:19.} porque <<\textit{aquel que vive en el amor vive en Dios y Dios en él}>>\footnote{\textit{El que vive en el amor, vive en Dios}: 1 Jn 4:16.}. Sin embargo, cuando hacéis el mal atormentáis al don interior de Dios, pues el Ajustador del Pensamiento ha de sufrir las consecuencias de los malos pensamientos junto con la mente humana donde está encarcelado.

\par
%\textsuperscript{(45.4)}
\textsuperscript{3:1.6} La omnipresencia de Dios forma parte en realidad de su naturaleza infinita; el espacio no constituye una barrera para la Deidad. Dios sólo está presente de manera perceptible, en su perfección y sin limitaciones, en el Paraíso y en el universo central. Así pues, su presencia no se puede observar en las creaciones que rodean a Havona, porque Dios ha limitado su presencia directa y efectiva en reconocimiento de la soberanía y de las prerrogativas divinas de los creadores y gobernantes coordinados de los universos del tiempo y del espacio. Por eso el concepto de la presencia divina debe tener en cuenta una amplia gama de formas y de canales de manifestación que abarcan los circuitos presenciales del Hijo Eterno, del Espíritu Infinito y de la Isla del Paraíso. Tampoco es siempre posible distinguir entre la presencia del Padre Universal y los actos de sus agentes y coordinados eternos, ya que éstos cumplen a la perfección todas las exigencias infinitas de su propósito invariable. Pero no sucede lo mismo con el circuito de la personalidad y los Ajustadores; en estas materias, Dios actúa de manera única, directa y exclusiva.

\par
%\textsuperscript{(45.5)}
\textsuperscript{3:1.7} El Controlador Universal está potencialmente presente en los circuitos de gravedad de la Isla del Paraíso, en todas las partes del universo, en todo momento y con la misma intensidad, de conformidad con la masa, en respuesta a las demandas físicas de su presencia, y a causa de la naturaleza inherente a toda la creación que hace que todas las cosas se adhieran a él y consistan en él\footnote{\textit{Todas las cosas consisten en Él}: Col 1:15-17.}. La Fuente-Centro Primera está asimismo potencialmente presente en el Absoluto Incalificado, el depósito de los universos increados del eterno futuro. Dios impregna así potencialmente los universos físicos del pasado, del presente y del futuro. La creación llamada material es coherente porque él es su fundamento primordial. Este potencial no espiritual de la Deidad se manifiesta aquí y allá, en todo el nivel de las existencias físicas, mediante la intrusión inexplicable de alguno de sus agentes exclusivos en el campo de acción del universo.

\par
%\textsuperscript{(45.6)}
\textsuperscript{3:1.8} La presencia mental de Dios está correlacionada con la mente absoluta del Actor Conjunto, el Espíritu Infinito. Pero en las creaciones finitas, esta presencia se percibe mejor en el funcionamiento omnipresente de la mente cósmica de los Espíritus Maestros del Paraíso. Al igual que la Fuente-Centro Primera está potencialmente presente en los circuitos mentales del Actor Conjunto, también está potencialmente presente en las tensiones del Absoluto Universal. Pero la mente de tipo humano es un don de las Hijas del Actor Conjunto, las Ministras Divinas de los universos en evolución.

\par
%\textsuperscript{(46.1)}
\textsuperscript{3:1.9} El espíritu omnipresente del Padre Universal está coordinado con la actividad de la presencia espiritual universal del Hijo Eterno y con el potencial divino perpetuo del Absoluto de la Deidad. Pero ni la actividad espiritual del Hijo Eterno y de sus Hijos Paradisiacos, ni las donaciones mentales del Espíritu Infinito parecen excluir la acción directa de los Ajustadores del Pensamiento, los fragmentos interiores de Dios, en el corazón de sus hijos creados.

\par
%\textsuperscript{(46.2)}
\textsuperscript{3:1.10} En lo que se refiere a la presencia de Dios en un planeta, un sistema, una constelación o un universo, el grado de dicha presencia en cualquier unidad creada mide el grado de la presencia evolutiva del Ser Supremo. Este grado está determinado por el reconocimiento masivo de Dios y la lealtad hacia él por parte de la inmensa organización universal, que se extiende hacia abajo hasta los sistemas y los planetas mismos. Por esta razón, y a veces con la esperanza de conservar y de salvaguardar estas fases de la preciosa presencia de Dios, cuando algunos planetas (o incluso algunos sistemas) se han hundido profundamente en las tinieblas espirituales, han sido puestos en cierto modo en cuarentena, o han sido parcialmente aislados sin poder relacionarse con las unidades más grandes de la creación. Todo esto, tal como sucede con Urantia, es una reacción espiritualmente defensiva de la mayoría de los mundos para protegerse, en la medida de lo posible, de sufrir las consecuencias del aislamiento ocasionado por los actos alienantes de una minoría testaruda, perversa y rebelde.

\par
%\textsuperscript{(46.3)}
\textsuperscript{3:1.11} Aunque el Padre incluye paternalmente en su circuito a todos sus hijos ---a todas las personalidades--- su influencia sobre ellos es limitada porque tienen un origen alejado de la Segunda y Tercera Personas de la Deidad; esta influencia aumenta a medida que logran su destino y se acercan a esos niveles. El \textit{hecho} de la presencia de Dios en la mente de las criaturas está determinado por la circunstancia de que estén o no habitadas por los fragmentos del Padre, tales como los Monitores de Misterio, pero la presencia \textit{eficaz} de Dios está determinada por el grado de cooperación que estos Ajustadores interiores reciben de las mentes donde residen.

\par
%\textsuperscript{(46.4)}
\textsuperscript{3:1.12} Las fluctuaciones de la presencia del Padre no se deben a la variabilidad de Dios\footnote{\textit{Dios no es un hombre}: Nm 23:19; 1 Sam 15:29.}. El Padre no se retira a un lugar aislado porque ha sido menospreciado; su afecto no se enajena porque la criatura haya actuado mal. En lugar de eso, como sus hijos han sido dotados del poder de elegir (en lo que se refiere a Él), son ellos los que, al ejercer esta elección, determinan directamente el grado y las limitaciones de la influencia divina del Padre en sus propios corazones y en sus propias almas. El Padre se ha dado gratuitamente a nosotros sin límites ni favoritismos. Él no hace acepción de personas\footnote{\textit{No hace acepción de personas}: 2 Cr 19:7; Job 34:19; Eclo 35:12; Hch 10:34; Ro 2:11; Gl 2:6; 3:28; Ef 6:9; Col 3:11.}, de planetas, de sistemas ni de universos. En los sectores del tiempo, sólo confiere honores diferenciales a las personalidades paradisiacas de Dios Séptuple, los creadores correlacionados de los universos finitos.

\section*{2. El poder infinito de Dios}
\par
%\textsuperscript{(46.5)}
\textsuperscript{3:2.1} Todos los universos saben que <<\textit{el Señor Dios omnipotente reina}>>\footnote{\textit{Omnipotente reina}: Ap 19:6.}. Los asuntos de este mundo y de los otros mundos están divinamente supervisados. <<\textit{Él actúa según su voluntad en los ejércitos del cielo y entre los habitantes de la Tierra}>>\footnote{\textit{Actúa de acuerdo a su voluntad}: Dn 4:35.}. Es eternamente cierto que <<\textit{no existe más poder que el de Dios}>>\footnote{\textit{No hay más poder que Dios}: Jn 19:11; Ro 13:1.}.

\par
%\textsuperscript{(46.6)}
\textsuperscript{3:2.2} Dentro de los límites de lo que es conforme con la naturaleza divina, es literalmente cierto que <<\textit{con Dios todas las cosas son posibles}>>\footnote{\textit{Con Él todo es posible}: Gn 18:14; Jer 32:27; Mt 19:26; Mc 10:27; 14:36; Lc 1:37; 18:27.}. Los procesos evolutivos interminables de los pueblos, los planetas y los universos están perfectamente controlados por los creadores y administradores universales, y se desarrollan según el propósito eterno del Padre Universal, avanzando en orden y armonía de acuerdo con el plan infinitamente sabio de Dios. Sólo hay un legislador\footnote{\textit{Sólo un legislador}: Stg 4:12.}. Él sostiene los mundos en el espacio y hace girar los universos alrededor del círculo sin fin del circuito eterno.

\par
%\textsuperscript{(47.1)}
\textsuperscript{3:2.3} De todos los atributos divinos, su omnipotencia es la mejor comprendida, especialmente tal como predomina en los universos materiales. Visto como un fenómeno no espiritual, Dios es energía. Esta afirmación de un hecho físico está basada en la verdad incomprensible de que la Fuente-Centro Primera es la causa primordial de los fenómenos físicos universales de todo el espacio. Toda la energía física y las demás manifestaciones materiales se derivan de esta actividad divina. La luz, es decir, la luz sin calor, es otra de las manifestaciones no espirituales de las Deidades. Y existe además otra forma de energía no espiritual que es prácticamente desconocida en Urantia; hasta ahora no ha sido reconocida.

\par
%\textsuperscript{(47.2)}
\textsuperscript{3:2.4} Dios controla todo el poder; ha trazado <<\textit{un camino para el rayo}>>\footnote{\textit{Un camino para el rayo}: Job 28:26; 38:25.}; ha ordenado los circuitos de todas las energías. Ha decretado el momento y la manera de manifestarse de todas las formas de energía-materia. Y todas estas cosas se mantienen para siempre bajo su perpetuo dominio ---bajo el control gravitatorio centrado en el bajo Paraíso. La luz y la energía del Dios eterno giran así constantemente alrededor de su circuito majestuoso, la procesión ordenada y sin fin de las multitudes de estrellas que componen el universo de universos. Toda la creación gira eternamente alrededor del centro paradisiaco y personal de todas las cosas y de todos los seres.

\par
%\textsuperscript{(47.3)}
\textsuperscript{3:2.5} La omnipotencia del Padre está relacionada con el predominio omnipresente del nivel absoluto, donde las tres energías, la material, la mental y la espiritual, no pueden distinguirse cuando se encuentran tan cerca de él ---la Fuente de todas las cosas\footnote{\textit{Fuente de todas las cosas}: Is 34:1; Jer 10:16; 51:19; Jn 1:3.}. Como la mente de la criatura no es la monota ni el espíritu del Paraíso, no responde directamente al Padre Universal. Dios \textit{se ajusta} a la mente imperfecta ---a los mortales de Urantia a través de los Ajustadores del Pensamiento.

\par
%\textsuperscript{(47.4)}
\textsuperscript{3:2.6} El Padre Universal no es una fuerza transitoria, un poder cambiante o una energía fluctuante. El poder y la sabiduría del Padre son totalmente adecuados para hacer frente a todas las exigencias del universo. Cuando surgen situaciones críticas en la experiencia humana, él las ha previsto todas, y por eso no reacciona de manera indiferente a los asuntos del universo, sino más bien de acuerdo con los dictados de la sabiduría eterna y en consonancia con los mandatos de su juicio infinito. A pesar de las apariencias, el poder de Dios no funciona como una fuerza ciega en el universo.

\par
%\textsuperscript{(47.5)}
\textsuperscript{3:2.7} A veces surgen situaciones en las que parece que se han tomado decisiones de emergencia, que se han suspendido leyes naturales, que se han reconocido inadaptaciones, y que se está haciendo un esfuerzo por rectificar la situación; pero éste no es el caso. Estos conceptos de Dios tienen su origen en el campo limitado de vuestro punto de vista, en la finitud de vuestra comprensión, y en la esfera circunscrita de vuestra visión de conjunto; este concepto erróneo de Dios se debe a la profunda ignorancia que tenéis acerca de la existencia de las leyes superiores del reino, la magnitud del carácter del Padre, la infinidad de sus atributos, y el hecho de su libre albedrío.

\par
%\textsuperscript{(47.6)}
\textsuperscript{3:2.8} Las criaturas planetarias habitadas por un espíritu de Dios, diseminadas aquí y allá por todos los universos del espacio, están tan cerca de ser infinitas en número y en clases, sus intelectos son tan diversos, sus mentes son tan limitadas y a veces tan toscas, su visión es tan reducida y tan localizada, que es casi imposible formular leyes generales que expresen adecuadamente los atributos infinitos del Padre, y que al mismo tiempo sean hasta cierto punto comprensibles para esas inteligencias creadas. Por esta razón, para vosotros las criaturas, muchos actos del Creador todopoderoso parecen arbitrarios, indiferentes y no raras veces despiadados y crueles. Pero os aseguro de nuevo que esto no es verdad. Todos los actos de Dios son decididos, inteligentes, sabios, generosos y tienen eternamente en cuenta el mayor bien, no siempre de un ser individual, una raza concreta, un planeta particular o incluso un universo determinado, sino que persiguen el bienestar y el mayor bien de todos los interesados, desde los más humildes hasta los más elevados. En las épocas del tiempo, a veces puede parecer que el bienestar de la parte difiere del bienestar del todo; en el círculo de la eternidad, estas diferencias aparentes no existen.

\par
%\textsuperscript{(48.1)}
\textsuperscript{3:2.9} Todos formamos parte de la familia de Dios\footnote{\textit{La familia de Dios}: Ro 8:16-17.}, y por eso a veces tenemos que participar en la disciplina de familia. Muchos actos de Dios que nos perturban y nos confunden tanto son el resultado de las decisiones y los fallos finales de la omnisciencia, la cual faculta al Actor Conjunto para llevar a cabo las elecciones de la voluntad infalible de la mente infinita, para hacer respetar las decisiones de la personalidad perfecta cuya vista de conjunto, visión y cuidados abarcan el bienestar eterno más elevado de toda su enorme y extensa creación.

\par
%\textsuperscript{(48.2)}
\textsuperscript{3:2.10} Así es como vuestro punto de vista aislado, particular, finito, tosco y extremadamente materialista, y las limitaciones inherentes a la naturaleza de vuestro ser, constituyen tal obstáculo que sois incapaces de ver, comprender o conocer la sabiduría y la bondad de muchos actos divinos que os parecen cargados de una crueldad tan aplastante, y que parecen estar caracterizados por una indiferencia tan total hacia la comodidad y el bienestar, hacia la felicidad planetaria y la prosperidad personal de vuestros semejantes. A causa de las limitaciones de la visión humana, debido a vuestro entendimiento circunscrito y a vuestra comprensión finita, interpretáis mal los móviles de Dios y desvirtuáis sus propósitos. Pero en los mundos evolutivos suceden muchas cosas que no son la obra personal del Padre Universal.

\par
%\textsuperscript{(48.3)}
\textsuperscript{3:2.11} La omnipotencia divina está perfectamente coordinada con los demás atributos de la personalidad de Dios. Generalmente, el poder de Dios sólo está limitado, en sus manifestaciones espirituales universales, por tres condiciones o situaciones:

\par
%\textsuperscript{(48.4)}
\textsuperscript{3:2.12} 1. Por la naturaleza de Dios, especialmente por su amor infinito, por la verdad, la belleza y la bondad.

\par
%\textsuperscript{(48.5)}
\textsuperscript{3:2.13} 2. Por la voluntad de Dios, por su ministerio de misericordia y sus relaciones paternales con las personalidades del universo.

\par
%\textsuperscript{(48.6)}
\textsuperscript{3:2.14} 3. Por la ley de Dios, por la rectitud y la justicia de la Trinidad eterna del Paraíso.

\par
%\textsuperscript{(48.7)}
\textsuperscript{3:2.15} Dios tiene un poder ilimitado, una naturaleza divina, una voluntad final, unos atributos infinitos, una sabiduría eterna y es una realidad absoluta. Todas estas características del Padre Universal están unificadas en la Deidad y se expresan de manera universal en la Trinidad del Paraíso y en los Hijos divinos de esta Trinidad. Por lo demás, fuera del Paraíso y del universo central de Havona, todo lo referente a Dios está limitado por la presencia evolutiva del Supremo, condicionado por la presencia en vías de existenciación del Último, y coordinado por los tres Absolutos existenciales ---el Absoluto de la Deidad, el Absoluto Universal y el Absoluto Incalificado. La presencia de Dios está limitada así porque esa es la voluntad de Dios.

\section*{3. El conocimiento universal de Dios}
\par
%\textsuperscript{(48.8)}
\textsuperscript{3:3.1} <<\textit{Dios conoce todas las cosas}>>\footnote{\textit{Conoce todas las cosas}: 1 Jn 3:20.}. La mente divina es consciente de los pensamientos de toda la creación y está familiarizada con ellos. Su conocimiento de los acontecimientos es universal y perfecto. Las entidades divinas que salen de él son una parte de él; aquel que <<\textit{equilibra las nubes}>>\footnote{\textit{El que equilibra las nubes}: Job 37:16.} es también <<\textit{perfecto en conocimiento}>>\footnote{\textit{Perfecto en conocimiento}: Job 36:4; 37:16.}. <<\textit{Los ojos del Señor están en todas partes}>>\footnote{\textit{Los ojos del Señor en todas partes}: Pr 15:3; 1 P 3:12.}. Vuestro gran maestro dijo de los gorriones insignificantes: <<\textit{Ni uno de ellos caerá al suelo sin que lo sepa mi Padre}>>\footnote{\textit{Los gorriones no caen sin que Dios lo sepa}: Mt 10:29; Lc 12:6.}, y también: <<\textit{Los cabellos mismos de vuestras cabezas están contados}>>\footnote{\textit{Vuestros cabellos están contados}: Mt 10:30; Lc 12:7.}. <<\textit{Él sabe el número de las estrellas, y las llama a todas por sus nombres}>>\footnote{\textit{El número y nombre de las estrellas}: Sal 147:4; Is 40:26.}.

\par
%\textsuperscript{(49.1)}
\textsuperscript{3:3.2} El Padre Universal es la única personalidad en todo el universo que sabe realmente el número de las estrellas y de los planetas del espacio. Todos los mundos de cada universo están constantemente en la conciencia de Dios. Él dice también: <<\textit{He visto ciertamente la aflicción de mi pueblo, he oído su llanto y conozco sus penas}>>\footnote{\textit{Conozco sus penas}: Ex 3:7.}. Porque <<\textit{el Señor mira desde los cielos; observa a todos los hijos de los hombres; desde el lugar donde reside contempla a todos los habitantes de la Tierra}>>\footnote{\textit{Mira desde los cielos}: Sal 33:13-14.}. Todo hijo de criatura puede decir en verdad: <<\textit{Él conoce el camino que tomo, y cuando me haya puesto a prueba, resaltaré como el oro}>>\footnote{\textit{Conoce el camino que tomo}: Job 23:10.}. <<\textit{Dios conoce nuestros avances y nuestros retrocesos, comprende nuestros pensamientos desde lejos y conoce todos nuestros caminos}>>\footnote{\textit{Comprende nuestros pensamientos}: Sal 139:2-3.}. <<\textit{Todas las cosas están desnudas y abiertas a los ojos de aquel con quien tratamos}>>\footnote{\textit{Todas las cosas están desnudas para Él}: Heb 4:13.}. Y para todo ser humano debería ser un verdadero consuelo comprender que <<\textit{él conoce vuestra estructura; se acuerda de que sois polvo}>>\footnote{\textit{Conoce vuestra estructura y que somos polvo}: Sal 103:14.}. Hablando del Dios vivo, Jesús dijo: <<\textit{Vuestro Padre sabe lo que necesitáis incluso antes de que se lo pidáis}>>\footnote{\textit{Sabe lo que necesitáis antes de que lo pidáis}: Is 65:24; Mt 6:8.}.

\par
%\textsuperscript{(49.2)}
\textsuperscript{3:3.3} Dios posee un poder ilimitado para conocer todas las cosas; su conciencia es universal. Su circuito personal abarca a todas las personalidades, y su conocimiento de las criaturas, incluidas las humildes, lo completa indirectamente mediante la serie descendente de los Hijos divinos, y directamente a través de los Ajustadores del Pensamiento interiores. Además, el Espíritu Infinito está constantemente presente en todas partes.

\par
%\textsuperscript{(49.3)}
\textsuperscript{3:3.4} No estamos totalmente seguros de si Dios elige o no conocer de antemano los casos de pecado. Pero aunque Dios conociera de antemano los actos del libre albedrío de sus hijos, esta presciencia no abrogaría en absoluto la libertad de sus criaturas. Una cosa es segura: a Dios nunca le coge nada por sorpresa.

\par
%\textsuperscript{(49.4)}
\textsuperscript{3:3.5} La omnipotencia no implica el poder de hacer lo irrealizable, un acto no divino. La omnisciencia tampoco implica conocer lo incognoscible. Pero no es fácil hacer comprender estas afirmaciones a la mente finita. La criatura difícilmente puede comprender el alcance y las limitaciones de la voluntad del Creador.

\section*{4. Dios no tiene límites}
\par
%\textsuperscript{(49.5)}
\textsuperscript{3:4.1} Las donaciones sucesivas de Dios a los universos, a medida que éstos surgen a la existencia, no disminuye de ningún modo el potencial de poder ni la reserva de sabiduría que continúan residiendo y descansando en la personalidad central de la Deidad. El potencial de fuerza, de sabiduría y de amor que posee el Padre nunca ha disminuido en nada, ni tampoco se ha despojado de ningún atributo de su gloriosa personalidad, como resultado de haberse dado sin límites a los Hijos Paradisiacos, a sus creaciones subordinadas, y a las múltiples criaturas de éstas.

\par
%\textsuperscript{(49.6)}
\textsuperscript{3:4.2} La creación de cada nuevo universo necesita un nuevo ajuste de la gravedad; pero aunque la creación continuara creciendo indefinidamente, eternamente, incluso hasta la infinidad, de tal manera que la creación material existiera finalmente sin límites, aún así se descubriría que el poder de control y de coordinación que reside en la Isla del Paraíso estaría a la altura y sería adecuado para dominar, controlar y coordinar ese universo infinito. Después de esta donación de fuerza y de poder ilimitados sobre un universo sin límites, el Infinito continuaría todavía sobrecargado con el mismo grado de fuerza y de energía; el Absoluto Incalificado estaría todavía sin disminuir; Dios poseería todavía el mismo potencial infinito, exactamente como si su fuerza, su energía y su poder nunca hubieran sido derramados para dotar a unos universos tras otros.

\par
%\textsuperscript{(50.1)}
\textsuperscript{3:4.3} Lo mismo sucede con la sabiduría: El hecho de que la mente sea tan abundantemente distribuida a los seres pensantes de los mundos no empobrece de ningún modo la fuente central de la sabiduría divina. A medida que se multiplican los universos y que el número de seres de los mundos va creciendo hasta los límites de la comprensión, aunque la mente continúe siendo otorgada sin fin a estos seres de rango superior e inferior, la personalidad central de Dios seguirá abarcando todavía la misma mente eterna, infinita y omnisapiente\footnote{\textit{Mente infinita}: Sal 147:5.}.

\par
%\textsuperscript{(50.2)}
\textsuperscript{3:4.4} El hecho de que envíe mensajeros espirituales procedentes de sí mismo para que residan en los hombres y las mujeres de vuestro mundo y de otros mundos, no disminuye de ningún modo su capacidad para actuar como una personalidad espiritual divina y todopoderosa; no existe absolutamente ningún límite en cuanto a la cantidad o al número de estos Monitores espirituales que Dios puede y desea enviar. Este don de sí mismo a sus criaturas crea para estos mortales divinamente dotados una posibilidad futura ilimitada y casi inconcebible de existencias progresivas y sucesivas. Esta pródiga distribución de sí mismo bajo la forma de estas entidades espirituales ministrantes no disminuye de ninguna manera la sabiduría y la perfección de la verdad y del conocimiento que descansan en la persona del Padre omnisciente, omnipotente y omnisapiente.

\par
%\textsuperscript{(50.3)}
\textsuperscript{3:4.5} Para los mortales del tiempo hay un futuro, pero Dios vive en la eternidad\footnote{\textit{Dios vive en la eternidad}: Esd 8:20; Is 57:15.}. Aunque vengo de las proximidades del lugar mismo donde reside la Deidad, no puedo atreverme a hablar con una comprensión perfecta sobre la infinidad de muchos atributos divinos. Sólo la infinidad de mente puede comprender plenamente la infinidad de existencia y la eternidad de acción.

\par
%\textsuperscript{(50.4)}
\textsuperscript{3:4.6} El hombre mortal no puede conocer de ninguna manera la infinitud del Padre celestial. La mente finita no puede examinar a fondo un hecho absoluto o una verdad absoluta de este tipo. Pero este mismo ser humano finito puede \textit{sentir} realmente ---puede experimentar literalmente--- el impacto completo y no disminuido del AMOR de un Padre así de infinito. Este amor se puede experimentar realmente, pero aunque la calidad de esta experiencia es ilimitada, su cantidad está estrictamente limitada por la capacidad humana para la receptividad espiritual y por la capacidad asociada para amar al Padre en recíproca correspondencia.

\par
%\textsuperscript{(50.5)}
\textsuperscript{3:4.7} La apreciación finita de las cualidades infinitas trasciende de lejos las capacidades lógicamente limitadas de las criaturas debido al hecho de que el hombre mortal ha sido creado a imagen de Dios\footnote{\textit{El hombre a imagen de Dios}: Gn 1:26-27; 9:6.} ---un fragmento de la infinidad vive dentro de él. Por eso el acercamiento más íntimo y más afectuoso del hombre a Dios ha de realizarlo por amor y a través del amor, porque Dios es amor\footnote{\textit{Dios es amor}: 1 Jn 4:8,16.}. La totalidad de esta relación única es una experiencia real en la sociología cósmica, la relación entre el Creador y la criatura ---el afecto entre Padre e hijo.

\section*{5. El dominio supremo del Padre}
\par
%\textsuperscript{(50.6)}
\textsuperscript{3:5.1} En su contacto con las creaciones posteriores a Havona, el Padre Universal no ejerce su poder infinito y su autoridad final por transmisión directa, sino más bien a través de sus Hijos y de las personalidades subordinadas a ellos. Dios hace todo esto por su propio libre albedrío. Si se presentara el caso, si la mente divina lo eligiera así, cualquiera de estos poderes delegados podría ser ejercido directamente; pero por regla general un acto así sólo tiene lugar cuando la personalidad delegada no ha logrado satisfacer la confianza divina. En esos momentos, en presencia de tal negligencia y dentro de los límites de la reserva del poder y del potencial divinos, el Padre actúa de manera independiente y de acuerdo con los mandatos de su propia elección; y esta elección siempre muestra una perfección infalible y una sabiduría infinita.

\par
%\textsuperscript{(51.1)}
\textsuperscript{3:5.2} El Padre gobierna por medio de sus Hijos; a través de toda la organización universal existe una cadena ininterrumpida de gobernantes que termina en los Príncipes Planetarios, los cuales dirigen los destinos de las esferas evolutivas de los inmensos dominios del Padre. Las exclamaciones siguientes no son simples expresiones poéticas: <<\textit{La Tierra pertenece al Señor en toda su plenitud}>>\footnote{\textit{La Tierra pertenece a Dios en toda su plenitud}: Sal 24:1; 1 Co 10:26,28.}. <<\textit{Destrona a los reyes y establece a los reyes}>>\footnote{\textit{Destrona y corona reyes}: Dn 2:21.}. <<\textit{Los Altísimos gobiernan en los reinos de los hombres}>>\footnote{\textit{Gobierna los reinos de los hombres}: Dn 4:17,25,32; 5:21.}.

\par
%\textsuperscript{(51.2)}
\textsuperscript{3:5.3} En las cuestiones del corazón de los hombres puede ser que el Padre Universal no siempre consiga sus fines, pero en lo que se refiere a la dirección y al destino de un planeta, es el plan divino el que prevalece; el propósito eterno de sabiduría y de amor es el que triunfa.

\par
%\textsuperscript{(51.3)}
\textsuperscript{3:5.4} Jesús dijo: <<\textit{Mi Padre, que me los ha dado, es más grande que todos; y nadie puede arrancarlos de la mano de mi Padre}>>\footnote{\textit{Mi Padre me los ha dado}: Jn 10:29.}. Cuando vislumbráis las múltiples obras de Dios y contempláis la asombrosa inmensidad de su creación casi ilimitada, vuestro concepto de su primacía puede titubear, pero no deberíais dejar de aceptar a Dios como entronizado perpetuamente y con seguridad en el centro paradisiaco de todas las cosas, y como Padre benefactor de todos los seres inteligentes. No hay más que <<\textit{un solo Dios y Padre de todos, que está por encima de todo y en todos}>>, y <<\textit{existe antes que todas las cosas, y todas las cosas consisten en él}>>\footnote{\textit{Existe antes que todas las cosas}: Col 1:17. \textit{Es antes que todo y en todo}: Ef 4:6.}.

\par
%\textsuperscript{(51.4)}
\textsuperscript{3:5.5} Las incertidumbres de la vida y las vicisitudes de la existencia no contradicen de ninguna manera el concepto de la soberanía universal de Dios. La vida de cualquier criatura evolutiva está asaltada por ciertas \textit{inevitabilidades.} Examinad las siguientes:

\par
%\textsuperscript{(51.5)}
\textsuperscript{3:5.6} 1. \textit{La valentía} ---la fuerza de carácter--- ¿es deseable? Entonces el hombre debe educarse en un entorno donde sea necesario luchar contra las dificultades y reaccionar ante las decepciones.

\par
%\textsuperscript{(51.6)}
\textsuperscript{3:5.7} 2. \textit{El altruismo} ---el servicio a los semejantes--- ¿es deseable? Entonces la experiencia de la vida debe proporcionar situaciones donde se encuentren desigualdades sociales.

\par
%\textsuperscript{(51.7)}
\textsuperscript{3:5.8} 3. \textit{La esperanza} ---la grandeza de la confianza--- ¿es deseable? Entonces la existencia humana debe enfrentarse continuamente con inseguridades e incertidumbres recurrentes.

\par
%\textsuperscript{(51.8)}
\textsuperscript{3:5.9} 4. \textit{La fe} ---la afirmación suprema del pensamiento humano--- ¿es deseable? Entonces la mente del hombre ha de encontrarse en esa situación incómoda en la que siempre sabe menos de lo que puede creer.

\par
%\textsuperscript{(51.9)}
\textsuperscript{3:5.10} 5. \textit{El amor a la verdad} ---y la buena disposición a seguirla dondequiera que conduzca--- ¿es deseable? Entonces el hombre debe crecer en un mundo donde el error esté presente y la falsedad sea siempre posible.

\par
%\textsuperscript{(51.10)}
\textsuperscript{3:5.11} 6. \textit{El idealismo} ---el concepto que se acerca a lo divino--- ¿es deseable? Entonces el hombre debe luchar en un entorno de bondad y de belleza relativas, en un ambiente que estimule la aspiración incontenible hacia cosas mejores.

\par
%\textsuperscript{(51.11)}
\textsuperscript{3:5.12} 7. \textit{La lealtad} ---la devoción al deber más elevado--- ¿es deseable? Entonces el hombre debe caminar entre las posibilidades de traición y de deserción. El valor de la devoción al deber consiste en el peligro implícito de incumplirlo.

\par
%\textsuperscript{(51.12)}
\textsuperscript{3:5.13} 8. \textit{El desinterés} ---el espíritu del olvido de sí mismo--- ¿es deseable? Entonces el hombre mortal debe vivir cara a cara con las reivindicaciones incesantes de un ego ineludible que pide reconocimiento y honores. El hombre no podría elegir dinámicamente la vida divina si no hubiera ninguna vida egoísta a la que renunciar. El hombre nunca podría aferrarse a la rectitud para salvarse si no existiera ningún mal potencial para exaltar y diferenciar el bien por contraste.

\par
%\textsuperscript{(51.13)}
\textsuperscript{3:5.14} 9. \textit{El placer} ---la satisfacción de la felicidad--- ¿es deseable? Entonces el hombre debe vivir en un mundo donde la alternativa del dolor y la probabilidad del sufrimiento son posibilidades experienciales siempre presentes.

\par
%\textsuperscript{(52.1)}
\textsuperscript{3:5.15} En todo el universo, cada unidad está considerada como una parte del todo\footnote{\textit{Unidad de mucha partes}: 1 Co 12:12-27.}. La supervivencia de la parte depende de su cooperación con el plan y la intención del todo, del deseo sincero y del consentimiento perfecto de hacer la voluntad divina del Padre. Si existiera un mundo evolutivo sin errores (sin la posibilidad de juicios imprudentes), sería un mundo sin inteligencia \textit{libre.} En el universo de Havona hay mil millones de mundos perfectos con sus habitantes perfectos, pero es necesario que el hombre en evolución sea falible si ha de ser libre. Una inteligencia libre e inexperimentada no puede ser de ninguna manera uniformemente sabia al principio. La posibilidad de un juicio erróneo (el mal) sólo se vuelve pecado cuando la voluntad humana acepta conscientemente y abraza deliberadamente un juicio inmoral premeditado.

\par
%\textsuperscript{(52.2)}
\textsuperscript{3:5.16} La plena apreciación de la verdad, la belleza y la bondad es inherente a la perfección del universo divino. Los habitantes de los mundos de Havona no necesitan el potencial de los niveles de valor relativo para estimular sus elecciones; estos seres perfectos son capaces de identificar y de elegir el bien en ausencia de toda situación moral que sirva de contraste y obligue a pensar. Pero todos estos seres perfectos poseen esa naturaleza moral y ese estado espiritual en virtud del hecho de su existencia. Sólo han conseguido avanzar experiencialmente en el interior de su estado inherente. El hombre mortal consigue incluso su estado de candidato a la ascensión mediante su propia fe y esperanza. Todas las cosas divinas que la mente humana capta y que el alma humana consigue son consecuciones experienciales; son \textit{realidades} de la experiencia personal y son, por lo tanto, posesiones únicas, en contraste con la bondad y la rectitud inherentes a las personalidades infalibles de Havona.

\par
%\textsuperscript{(52.3)}
\textsuperscript{3:5.17} Las criaturas de Havona son valientes por naturaleza, pero no son valerosas en el sentido humano. Son amables y consideradas de forma innata, pero difícilmente altruistas a la manera humana. Esperan un futuro agradable, pero no tienen esperanzas a la manera exquisita de los mortales confiados de las esferas evolutivas inciertas. Tienen fe en la estabilidad del universo, pero desconocen totalmente esa fe salvadora por la cual el hombre mortal se eleva desde el estado de animal hasta las puertas del Paraíso. Aman la verdad, pero no saben nada de sus cualidades que salvan el alma. Son idealistas, pero han nacido así; ignoran por completo el éxtasis de llegar a serlo mediante elecciones estimulantes. Son leales, pero nunca han experimentado la emoción que produce la devoción sincera e inteligente al deber frente a la tentación de no cumplirlo. Son desinteresadas, pero nunca han conseguido estos niveles experienciales mediante la magnífica victoria sobre un yo beligerante. Disfrutan del placer, pero no comprenden el dulzor de escapar por medio del placer al potencial del dolor.

\section*{6. La primacía del Padre}
\par
%\textsuperscript{(52.4)}
\textsuperscript{3:6.1} Con un desinterés divino, con una generosidad consumada, el Padre Universal renuncia a su autoridad y delega su poder, pero continúa siendo primordial; su mano descansa sobre la poderosa palanca de las circunstancias de los reinos universales; se ha reservado todas las decisiones finales y ejerce infaliblemente el cetro todopoderoso del veto de su propósito eterno con una autoridad indiscutible sobre el bienestar y el destino de la extensa creación que gira en las órbitas perpetuas.

\par
%\textsuperscript{(52.5)}
\textsuperscript{3:6.2} La soberanía de Dios es ilimitada; es el hecho fundamental de toda la creación. El universo no era inevitable. El universo no es un accidente, ni existe por sí mismo. El universo es un trabajo de creación y por eso está totalmente sujeto a la voluntad del Creador. La voluntad de Dios es la verdad divina, el amor viviente; por esa razón, las creaciones que se perfeccionan en los universos evolutivos están caracterizadas por la bondad ---acercamiento a la divinidad--- y por el mal potencial ---alejamiento de la divinidad.

\par
%\textsuperscript{(53.1)}
\textsuperscript{3:6.3} Tarde o temprano, todas las filosofías religiosas llegan al concepto de un gobierno universal unificado, de un solo Dios. Las causas universales no pueden ser inferiores a los efectos universales. La fuente de las corrientes de la vida universal y de la mente cósmica tiene que estar por encima de los niveles de su manifestación. La mente humana no puede ser explicada de manera coherente en términos de los tipos inferiores de existencia. La mente del hombre sólo se puede comprender realmente cuando se reconoce la realidad de unos tipos superiores de pensamiento y de voluntad intencional. El hombre como ser moral no tiene explicación, a menos que se reconozca la realidad del Padre Universal.

\par
%\textsuperscript{(53.2)}
\textsuperscript{3:6.4} Los filósofos mecanicistas pretenden rechazar la idea de una voluntad universal y soberana, y veneran profundamente la actividad de esa misma voluntad soberana que ha elaborado las leyes del universo. !`Qué homenaje involuntario rinde el mecanicista al Creador de las leyes, cuando concibe que tales leyes actúan y se explican por sí solas!

\par
%\textsuperscript{(53.3)}
\textsuperscript{3:6.5} Es un gran disparate humanizar a Dios, salvo en el concepto del Ajustador del Pensamiento interior, pero incluso esto no es tan insensato como \textit{mecanizar} por completo la idea de la Gran Fuente-Centro Primera.

\par
%\textsuperscript{(53.4)}
\textsuperscript{3:6.6} ¿Sufre el Padre Paradisiaco? No lo sé. Los Hijos Creadores pueden sufrir con toda seguridad y a veces sufren, como les sucede a los mortales. El Hijo Eterno y el Espíritu Infinito sufren en un sentido modificado. Creo que el Padre Universal sufre, pero no puedo comprender \textit{cómo;} quizás sea a través del circuito de la personalidad, o por medio de la individualidad de los Ajustadores del Pensamiento y de las otras donaciones de su naturaleza eterna. Él ha dicho de las razas mortales: <<\textit{En todas vuestras aflicciones estoy afligido}>>\footnote{\textit{En vuestras aflicciones me aflijo}: Is 63:9.}. Él experimenta indiscutiblemente una comprensión paternal y compasiva; puede ser que sufra realmente, pero no comprendo la naturaleza de ese sufrimiento.

\par
%\textsuperscript{(53.5)}
\textsuperscript{3:6.7} El Gobernante eterno e infinito del universo de universos es poder, forma, energía, proceso, arquetipo, principio, presencia y realidad idealizada. Pero es mucho más: es personal; ejerce una voluntad soberana, experimenta la conciencia de su divinidad, ejecuta los mandatos de una mente creadora, persigue la satisfacción de realizar un proyecto eterno, y manifiesta el amor y el afecto de un Padre por sus hijos del universo. Todas estas características más personales del Padre se comprenden mejor observándolas tal como fueron reveladas en la vida de donación de Miguel, vuestro Hijo Creador, cuando estuvo encarnado en Urantia.

\par
%\textsuperscript{(53.6)}
\textsuperscript{3:6.8} Dios Padre ama a los hombres; Dios Hijo sirve a los hombres; Dios Espíritu inspira a los hijos del universo hacia la aventura siempre ascendente de encontrar a Dios Padre por los caminos ordenados por Dios Hijos a través del ministerio de la gracia de Dios Espíritu.

\par
%\textsuperscript{(53.7)}
\textsuperscript{3:6.9} [Siendo el Consejero Divino designado para presentar la revelación del Padre Universal, he continuado con esta exposición de los atributos de la Deidad.]


\chapter{Documento 4. Las relaciones de Dios con el universo}
\par
%\textsuperscript{(54.1)}
\textsuperscript{4:0.1} EL PADRE Universal tiene un propósito eterno relacionado con los fenómenos materiales, intelectuales y espirituales del universo de universos, y lo lleva a cabo constantemente. Dios creó los universos por su propia voluntad libre y soberana, y los creó de acuerdo con su propósito omnisapiente y eterno. Es dudoso que nadie, salvo las Deidades del Paraíso y sus asociados más elevados, sepa realmente mucho sobre el propósito eterno de Dios. Incluso los ciudadanos elevados del Paraíso tienen opiniones muy diversas acerca de la naturaleza del propósito eterno de las Deidades.

\par
%\textsuperscript{(54.2)}
\textsuperscript{4:0.2} Es fácil deducir que al crear el perfecto universo central de Havona, el propósito era satisfacer puramente la naturaleza divina. Havona puede servir como creación modelo para todos los demás universos, y como escuela final para los peregrinos del tiempo en su camino hacia el Paraíso; sin embargo, esta creación celestial debe existir principalmente para el placer y la satisfacción de los Creadores perfectos e infinitos.

\par
%\textsuperscript{(54.3)}
\textsuperscript{4:0.3} El plan asombroso para perfeccionar a los mortales evolutivos y, después de que han alcanzado el Paraíso y el Cuerpo de la Finalidad, para proporcionarles una formación adicional con vistas a un trabajo futuro no revelado, parece ser actualmente uno de los intereses principales de los siete superuniversos y de sus numerosas subdivisiones; pero este programa de ascensión para espiritualizar y educar a los mortales del tiempo y del espacio no es de ninguna manera la ocupación exclusiva de las inteligencias del universo. Existen en verdad otras muchas tareas fascinantes que ocupan el tiempo y reclutan las energías de las huestes celestiales.

\section*{1. La actitud del Padre hacia el universo}
\par
%\textsuperscript{(54.4)}
\textsuperscript{4:1.1} Durante siglos, los habitantes de Urantia no han comprendido la providencia de Dios. En vuestro mundo existe una providencia de elaboración divina, pero no se trata del ministerio infantil, arbitrario y material que muchos mortales han concebido. La providencia de Dios consiste en las actividades entrelazadas de los seres celestiales y de los espíritus divinos que, de acuerdo con la ley cósmica, trabajan sin cesar por el honor de Dios y por el progreso espiritual de sus hijos del universo.

\par
%\textsuperscript{(54.5)}
\textsuperscript{4:1.2} ¿No podéis elevar vuestro concepto sobre las relaciones de Dios con el hombre hasta el punto de reconocer que la consigna del universo es el \textit{progreso?} La raza humana ha luchado durante largas épocas para alcanzar su estado actual. A lo largo de todos esos milenios, la Providencia ha estado realizando el plan de la evolución progresiva. Estas dos ideas no son opuestas en la práctica, sino únicamente en los conceptos erróneos del hombre. La providencia divina no se opone nunca al verdadero progreso humano, ya sea temporal o espiritual. La providencia está siempre de acuerdo con la naturaleza perfecta e invariable del Legislador supremo.

\par
%\textsuperscript{(55.1)}
\textsuperscript{4:1.3} <<\textit{Dios es fiel}>>\footnote{\textit{Dios es fiel}: Dt 7:9; 1 Co 1:9; 10:13.} y <<\textit{todos sus mandamientos son justos}>>\footnote{\textit{Las leyes de Dios son justas}: Sal 119:172.}. <<\textit{Su fidelidad está establecida en los mismos cielos}>>\footnote{\textit{Fidelidad establecida en los cielos}: Sal 36:5.}. <<\textit{Oh Señor, tu palabra está establecida para siempre en los cielos. Tu fidelidad es para todas las generaciones; has establecido la Tierra, y ésta permanece}>>\footnote{\textit{Palabra establecida en los cielos}: Sal 119:89-90.}. <<\textit{Él es un Creador fiel}>>\footnote{\textit{Creador fiel}: 1 P 4:19.}.

\par
%\textsuperscript{(55.2)}
\textsuperscript{4:1.4} Las fuerzas y las personalidades que el Padre puede utilizar para hacer respetar su propósito y sostener a sus criaturas no tienen límites. <<\textit{El Dios eterno es nuestro refugio, y por debajo están sus brazos eternos}>>\footnote{\textit{El Dios eterno es nuestro refugio}: Dt 33:27.}. <<\textit{Aquel que habita en el lugar secreto del Altísimo permanecerá bajo la sombra del Todopoderoso}>>\footnote{\textit{Habita en lugar secreto}: Sal 91:1.}. <<\textit{Mirad, aquel que nos cuida no dormitará ni se dormirá}>>\footnote{\textit{No duerme ni dormita}: Sal 121:4.}. <<\textit{Sabemos que todas las cosas trabajan unidas por el bien de aquellos que aman a Dios}>>\footnote{\textit{Todas las cosas trabajan unidas}: Ro 8:28.}, <<\textit{porque los ojos del Señor están sobre los justos, y sus oídos están abiertos a sus oraciones}>>\footnote{\textit{Los ojos del Sñor sobre los justos}: Sal 34:15; 1 P 3:12.}.

\par
%\textsuperscript{(55.3)}
\textsuperscript{4:1.5} Dios sostiene <<\textit{todas las cosas con la palabra de su poder}>>\footnote{\textit{Sostiene todas las cosas}: Heb 1:3.}. Y cuando nacen nuevos mundos, <<\textit{envía a sus Hijos y esos mundos son creados}>>\footnote{\textit{Envía a sus Hijos creadores}: Sal 104:30.}. Dios no solamente crea, sino que <<\textit{los protege a todos}>>\footnote{\textit{Los protege a todos}: Neh 9:6.}. Dios sostiene constantemente todas las cosas materiales y a todos los seres espirituales. Los universos son eternamente estables. Existe una estabilidad en medio de una inestabilidad aparente\footnote{\textit{Estabilidad entre la inestabilidad}: Job 26:7.}. Existe un orden y una seguridad subyacentes en medio de las agitaciones energéticas y de los cataclismos físicos de los reinos cuajados de estrellas.

\par
%\textsuperscript{(55.4)}
\textsuperscript{4:1.6} El Padre Universal no se ha retirado de la dirección de los universos; no es una Deidad inactiva. Si Dios se retirara como sostén actual de toda la creación, se produciría inmediatamente un derrumbamiento universal. Exceptuando a Dios, no existiría nada que pudiera calificarse de \textit{realidad.} En este mismo momento, así como durante las épocas lejanas del pasado y en el eterno futuro, Dios continúa sosteniendo\footnote{\textit{Dios continúa sosteniendo}: Job 26:7.}. El alcance divino se extiende por todo el círculo de la eternidad. Al universo no se le da cuerda como a un reloj para que ande durante cierto tiempo y luego deje de funcionar; todas las cosas se renuevan constantemente\footnote{\textit{Constantemente renovado}: 104:30.}. El Padre derrama sin cesar energía, luz y vida. El trabajo de Dios es tangible así como espiritual. <<\textit{Extiende el norte sobre el espacio vacío y cuelga la Tierra en la nada}>>.

\par
%\textsuperscript{(55.5)}
\textsuperscript{4:1.7} Un ser de mi orden es capaz de descubrir una armonía última y de detectar una coordinación trascendental y profunda en los asuntos rutinarios de la administración universal. Muchas cosas que parecen inconexas y fortuitas para la mente mortal, aparecen ordenadas y constructivas para mi comprensión. Pero suceden muchas cosas en los universos que no comprendo plenamente. He estudiado durante mucho tiempo y estoy más o menos familiarizado con las fuerzas, las energías, las mentes, las morontias, los espíritus y las personalidades reconocidas de los universos locales y de los superuniversos. Tengo una comprensión general de cómo funcionan estos agentes y personalidades, y conozco íntimamente los trabajos de las inteligencias espirituales acreditadas del gran universo. A pesar de mi conocimiento de los fenómenos de los universos, me enfrento constantemente con reacciones cósmicas que no puedo comprender plenamente. Encuentro continuamente confabulaciones aparentemente fortuitas de interasociaciones de fuerzas, energías, intelectos y espíritus que no puedo explicar de manera satisfactoria.

\par
%\textsuperscript{(55.6)}
\textsuperscript{4:1.8} Soy enteramente competente para descubrir y analizar el funcionamiento de todos los fenómenos que se derivan directamente de la actividad del Padre Universal, del Hijo Eterno, del Espíritu Infinito y, en gran medida, de la Isla del Paraíso. Mi perplejidad aparece cuando me encuentro con lo que parece ser la actuación de sus misteriosos coordinados, los tres Absolutos de potencialidad. Estos Absolutos parecen reemplazar la materia, trascender la mente y sobrevenir al espíritu. Me siento constantemente confundido y a menudo perplejo debido a mi incapacidad para comprender estas complejas operaciones, que atribuyo a la presencia y a la actividad del Absoluto Incalificado, del Absoluto de la Deidad y del Absoluto Universal.

\par
%\textsuperscript{(56.1)}
\textsuperscript{4:1.9} Estos Absolutos deben ser las presencias no plenamente reveladas fuera en el universo que, en lo referente a los fenómenos de la potencia espacial y a la función de otros superúltimos, hacen que a los físicos, a los filósofos e incluso a las personas religiosas les resulte imposible predecir con certeza de qué manera los orígenes primordiales de la fuerza, del concepto o del espíritu reaccionarán a unas demandas efectuadas en una situación de realidad compleja, que implican ajustes supremos y valores últimos.

\par
%\textsuperscript{(56.2)}
\textsuperscript{4:1.10} Existe también una unidad orgánica en los universos del tiempo y del espacio que parece servir de base a toda la estructura de los acontecimientos cósmicos. Esta presencia viviente del Ser Supremo en evolución, esta Inmanencia del Incompleto Proyectado, se manifiesta inexplicablemente de vez en cuando mediante lo que parece ser una coordinación asombrosamente fortuita de acontecimientos universales aparentemente no relacionados entre sí. Debe tratarse de la función de la Providencia ---el ámbito del Ser Supremo y del Actor Conjunto.

\par
%\textsuperscript{(56.3)}
\textsuperscript{4:1.11} Me inclino a creer que este extenso control, generalmente imposible de reconocer, que coordina e interasocia todas las fases y formas de la actividad universal, es el que hace que esta mezcla variada y en apariencia desesperadamente confusa de fenómenos físicos, mentales, morales y espirituales, trabaje tan infaliblemente para la gloria de Dios y para el bien de los hombres y de los ángeles.

\par
%\textsuperscript{(56.4)}
\textsuperscript{4:1.12} Pero en un sentido más amplio, los <<\textit{accidentes}>> aparentes del cosmos forman parte sin duda del drama finito de la aventura espacio-temporal del Infinito en su eterna manipulación de los Absolutos.

\section*{2. Dios y la naturaleza}
\par
%\textsuperscript{(56.5)}
\textsuperscript{4:2.1} La naturaleza es, en un sentido limitado, la constitución física de Dios. El comportamiento, o la acción de Dios, se encuentra atenuado y provisionalmente modificado por los planes experimentales y las configuraciones evolutivas de un universo local, una constelación, un sistema o un planeta. Dios actúa de acuerdo con una ley bien definida, invariable e inmutable, en todo el extenso universo maestro; pero modifica las pautas de su acción para poder contribuir al comportamiento coordinado y equilibrado de cada universo, constelación, sistema, planeta y personalidad, de conformidad con los objetivos, las intenciones y los planes locales de los proyectos finitos de desarrollo evolutivo.

\par
%\textsuperscript{(56.6)}
\textsuperscript{4:2.2} Por eso la naturaleza, tal como la comprende el hombre mortal, presenta la base subyacente y el trasfondo fundamental de una Deidad invariable y de sus leyes inmutables, las cuales son modificadas, fluctúan y experimentan trastornos debido al funcionamiento de los planes, los objetivos, las configuraciones y las condiciones locales que las fuerzas y las personalidades del universo local, de la constelación, del sistema y del planeta han introducido y están llevando a cabo. Por ejemplo: las leyes de Dios que han sido ordenadas para Nebadon son modificadas por los planes establecidos por el Hijo Creador y el Espíritu Creativo de este universo local; y además de todo esto, el funcionamiento de estas leyes ha sufrido la influencia adicional de los errores, las negligencias y las insurrecciones de ciertos seres residentes en vuestro planeta y que pertenecen a vuestro propio sistema planetario de Satania.

\par
%\textsuperscript{(56.7)}
\textsuperscript{4:2.3} La naturaleza es la resultante espacio-temporal de dos factores cósmicos: en primer lugar, la inmutabilidad, la perfección y la rectitud de la Deidad del Paraíso, y en segundo lugar, los planes experimentales, los desatinos de ejecución, los errores insurreccionales, el desarrollo incompleto y la sabiduría imperfecta de las criaturas extraparadisiacas, desde las más elevadas hasta las más humildes. La naturaleza contiene por tanto un hilo de perfección uniforme, invariable, majestuoso y maravilloso que proviene del círculo de la eternidad; pero en cada universo, en cada planeta y en cada vida individual, esta naturaleza se encuentra modificada, atenuada y quizás desfigurada debido a los actos, los errores y las deslealtades de las criaturas de los sistemas y de los universos evolutivos; por eso la naturaleza ha de estar siempre de humor cambiante, además de ser caprichosa, aunque en el fondo sea estable, y varíe de acuerdo con los procedimientos operativos de un universo local.

\par
%\textsuperscript{(57.1)}
\textsuperscript{4:2.4} La naturaleza es la perfección del Paraíso, dividida por el estado incompleto, el mal y el pecado de los universos inacabados. Este cociente expresa así a la vez lo perfecto y lo parcial, lo eterno y lo temporal. La evolución contínua modifica la naturaleza mediante el aumento del contenido de la perfección paradisiaca y la disminución del contenido del mal, del error y de la falta de armonía de la realidad relativa.

\par
%\textsuperscript{(57.2)}
\textsuperscript{4:2.5} Dios no está personalmente presente ni en la naturaleza ni en ninguna de las fuerzas de la naturaleza, porque el fenómeno de la naturaleza es la superposición de las imperfecciones de la evolución progresiva y, a veces, de las consecuencias de una rebelión insurreccional, sobre los fundamentos paradisiacos de la ley universal de Dios. Tal como aparece en un mundo como Urantia, la naturaleza no puede ser nunca la expresión adecuada, la verdadera representación, el fiel retrato, de un Dios omnisapiente e infinito.

\par
%\textsuperscript{(57.3)}
\textsuperscript{4:2.6} En vuestro mundo, la naturaleza representa las leyes de la perfección, atenuadas por los planes evolutivos del universo local. !`Qué parodia adorar la naturaleza porque esté impregnada de Dios en un sentido limitado y restringido; porque sea una fase del poder universal y, por lo tanto, del poder divino! La naturaleza es también una manifestación de los procesos inacabados, incompletos e imperfectos del desarrollo, del crecimiento y del progreso de un experimento universal en la evolución cósmica.

\par
%\textsuperscript{(57.4)}
\textsuperscript{4:2.7} Los defectos aparentes del mundo natural no indican ningún defecto correspondiente de ese tipo en el carácter de Dios. Las imperfecciones que se observan son más bien las simples detenciones inevitables que se producen durante la exposición de la bobina siempre en movimiento de la película infinita. Estas mismas interrupciones-defectos de la continuidad de la perfección son las que hacen posible que la mente finita del hombre material capte un vislumbre fugaz de la realidad divina en el tiempo y el espacio. Las manifestaciones materiales de la divinidad sólo parecen defectuosas para la mente evolutiva del hombre porque el hombre mortal insiste en mirar los fenómenos de la naturaleza con los ojos físicos, con la visión humana sin la ayuda de la mota morontial o de la revelación, que son sus sustitutos compensatorios en los mundos del tiempo.

\par
%\textsuperscript{(57.5)}
\textsuperscript{4:2.8} Y la naturaleza está desfigurada, su hermoso rostro está marcado, sus rasgos están marchitos por la rebelión, la mala conducta y los pensamientos erróneos de las miríadas de criaturas que forman parte de la naturaleza, pero que han contribuido a desfigurarla en el tiempo. No, la naturaleza no es Dios. La naturaleza no es un objeto de adoración.

\section*{3. El carácter invariable de Dios}
\par
%\textsuperscript{(57.6)}
\textsuperscript{4:3.1} El hombre ha creído durante demasiado tiempo que Dios se parecía a él\footnote{\textit{Dios no es como el hombre}: Nm 23:19; 1 Sam 15:29.}. Dios no tiene, no ha tenido nunca, y nunca tendrá celos del hombre o de cualquier otro ser del universo de universos. Sabiendo que el Hijo Creador tenía la intención de hacer del hombre la obra maestra de la creación planetaria, el soberano de toda la Tierra, cuando ve que su ser se encuentra dominado por sus propias pasiones más bajas, el espectáculo de verlo doblegado ante los ídolos de madera, de piedra, de oro y de su ambición egoísta ---estas sórdidas escenas incitan a Dios y a sus Hijos a estar celosos \textit{por} el hombre, pero nunca del hombre\footnote{\textit{Dios celoso `por' el hombre}: Ez 39:25; Jl 2:18; Zac 1:14; 8:2. \textit{Dios celoso `del' hombre}: Ex 20:5; 34:14; Nah 1:2; Dt 4:24; 5:9; 6:15; Jos 24:19.}.

\par
%\textsuperscript{(57.7)}
\textsuperscript{4:3.2} El Dios eterno es incapaz de cólera y de ira en el sentido de estas emociones humanas y tal como el hombre comprende estas reacciones\footnote{\textit{La visión de Dios como colérico o airado}: Ex 4:14; 1 Re 14:9; 1 Cr 13:10; Neh 4:4; Job 9:13; Sal 6:1; Is 1:4; Jer 3:12; Lm 1:12; Nm 11:1; Ez 5:13; Os 11:9; Jl 2:13; Jon 3:9; Miq 7:18; Nah 1:3; Dt 4:25; Sof 2:2; Jos 7:1; Jue 2:12; 2 Sam 6:7.}. Estos sentimientos son mezquinos y despreciables; apenas son dignos de ser llamados humanos, y mucho menos divinos; estas actitudes son totalmente ajenas a la naturaleza perfecta y al carácter misericordioso del Padre Universal.

\par
%\textsuperscript{(58.1)}
\textsuperscript{4:3.3} Una parte, una gran parte de las dificultades que tienen los mortales de Urantia para comprender a Dios se debe a las consecuencias trascendentales de la rebelión de Lucifer y de la traición de Caligastia. En los mundos no aislados por el pecado, las razas evolutivas son capaces de hacerse unas ideas mucho mejores sobre el Padre Universal; sufren menos confusión, deformación y perversión en sus conceptos.

\par
%\textsuperscript{(58.2)}
\textsuperscript{4:3.4} Dios no se arrepiente de nada de lo que ha hecho antes, de lo que hace ahora, o de lo que hará en el futuro\footnote{\textit{La visión de Dios arrepintiéndose}: Gn 6:6; Ex 32:14; 1 Cr 21:15; Sal 106:45; Jer 18:8,10; 26:19; 42:10; Am 7:3,6; Jon 3:10; Jue 2:18; 1 Sam 15:35; 2 Sam 24:16. \textit{Dios no se arrepiente (por elección)}: Sal 110:4; Jer 4:28; Ez 24:14; Zac 8:14; Heb 7:21. \textit{Dios no se arrepiente (por naturaleza)}: Nm 23:19; 1 Sam 15:29.}. Es omnisapiente así como omnipotente. La sabiduría del hombre surge de las pruebas y de los errores de la experiencia humana; la sabiduría de Dios consiste en la perfección incalificada de su perspicacia universal infinita, y este preconocimiento divino dirige eficazmente su libre albedrío creativo.

\par
%\textsuperscript{(58.3)}
\textsuperscript{4:3.5} El Padre Universal nunca hace nada que produzca tristeza o pesar posteriormente, pero las criaturas volitivas que han sido planeadas y creadas por sus Personalidades Creadoras en los universos exteriores efectúan elecciones desacertadas y, a veces, producen emociones de divina tristeza en la personalidad de sus padres Creadores. Pero aunque el Padre no comete errores, ni tiene penas, ni experimenta tristezas, es un ser con un afecto de padre, y su corazón se aflige indudablemente cuando sus hijos no logran alcanzar los niveles espirituales que son capaces de conseguir con la ayuda que les ha sido proporcionada tan abundantemente mediante los planes de consecución espiritual y las políticas universales para la ascensión de los mortales.

\par
%\textsuperscript{(58.4)}
\textsuperscript{4:3.6} La bondad infinita del Padre se encuentra más allá de la comprensión de la mente finita del tiempo; de ahí que deba proporcionarse siempre un contraste con el mal relativo
(no con el pecado) para mostrar efectivamente todas las fases de la bondad relativa. La perspicacia imperfecta de los mortales sólo puede discernir la perfección de la bondad divina porque ésta se halla en una asociación de contraste con la imperfección relativa en las relaciones del tiempo y la materia en los movimientos del espacio.

\par
%\textsuperscript{(58.5)}
\textsuperscript{4:3.7} El carácter de Dios es infinitamente superhumano; por eso esta naturaleza de la divinidad ha de ser personalizada, como en los Hijos divinos, antes incluso de que pueda ser captada mediante la fe por la mente finita del hombre.

\section*{4. La comprensión de Dios}
\par
%\textsuperscript{(58.6)}
\textsuperscript{4:4.1} Dios es el único ser estacionario, autosuficiente e invariable en todo el universo de universos, y no tiene exterior, ni más allá, ni pasado ni futuro. Dios es energía intencional (espíritu creador) y voluntad absoluta, y estos atributos existen por sí mismos y son universales.

\par
%\textsuperscript{(58.7)}
\textsuperscript{4:4.2} Puesto que Dios existe por sí mismo, es absolutamente independiente. La identidad misma de Dios es contraria al cambio. <<\textit{Yo, el Señor, no cambio}>>\footnote{\textit{Dios no cambia}: Mal 3:6; Stg 1:17.}. Dios es inmutable; pero hasta que no alcancéis el estado paradisiaco, ni siquiera podréis empezar a comprender cómo Dios puede pasar de la simplicidad a la complejidad, de la identidad a la variación, de la quietud al movimiento, de la infinidad a la finitud, de lo divino a lo humano, y de la unidad a la dualidad y a la triunidad. Dios puede modificar así las manifestaciones de su absolutidad porque la inmutabilidad divina no implica la inmovilidad; Dios tiene voluntad ---él \textit{es} voluntad.

\par
%\textsuperscript{(58.8)}
\textsuperscript{4:4.3} Dios es el ser que se determina absolutamente a sí mismo; no existen límites a sus reacciones en el universo, salvo aquellos que se impone a sí mismo, y los actos de su libre albedrío sólo están condicionados por aquellas cualidades divinas y aquellos atributos perfectos que caracterizan de manera inherente su naturaleza eterna. Por eso la relación de Dios con el universo es la de un ser de bondad final más la de un libre albedrío de infinidad creativa.

\par
%\textsuperscript{(58.9)}
\textsuperscript{4:4.4} El Absoluto-Padre es el creador del universo central y perfecto, y el Padre de todos los demás Creadores. Dios comparte con el hombre y con otros seres la personalidad, la bondad y otras muchas características, pero la infinidad de voluntad es sólo suya. Dios sólo está limitado en sus actos creadores por los sentimientos de su naturaleza eterna y por los dictados de su sabiduría infinita. Dios sólo elige personalmente aquello que es infinitamente perfecto, de ahí la perfección celestial del universo central; y aunque los Hijos Creadores comparten plenamente su divinidad, e incluso algunas fases de su absolutidad, no están totalmente limitados por esa sabiduría final que dirige la voluntad infinita del Padre. En consecuencia, el libre albedrío creativo se vuelve incluso más activo, totalmente divino y casi último, si no absoluto, en la orden de filiación de los Migueles. El Padre es infinito y eterno, pero negar la posibilidad de que pueda limitarse voluntariamente a sí mismo equivale a negar el concepto mismo de su absolutidad volitiva.

\par
%\textsuperscript{(59.1)}
\textsuperscript{4:4.5} La absolutidad de Dios impregna cada uno de los siete niveles de la realidad universal. La totalidad de esta naturaleza absoluta está sujeta a la relación entre el Creador y su familia universal de criaturas. La precisión puede caracterizar a la justicia trinitaria en el universo de universos, pero en todas sus extensas relaciones familiares con las criaturas del tiempo, el Dios de los universos está gobernado por el \textit{sentimiento divino.} En primer y en último lugar ---eternamente--- el Dios infinito es un \textit{Padre.} De todos los títulos posibles con los que podría ser conocido de manera apropiada, se me ha encargado describir al Dios de toda la creación como el Padre Universal.

\par
%\textsuperscript{(59.2)}
\textsuperscript{4:4.6} En Dios Padre, las acciones de su libre albedrío no están dirigidas por el poder ni guiadas por el solo intelecto; la personalidad divina se puede definir como que consiste en un espíritu y se manifiesta a los universos como amor. Por eso, en todas sus relaciones personales con las personalidades de las criaturas de los universos, la Fuente-Centro Primera es siempre y consecuentemente un Padre amoroso. Dios es un Padre en el sentido más elevado del término. Está eternamente motivado por el idealismo perfecto del amor divino, y esta tierna naturaleza encuentra su expresión más poderosa y su mayor satisfacción en el hecho de amar y ser amado.

\par
%\textsuperscript{(59.3)}
\textsuperscript{4:4.7} En la ciencia, Dios es la Causa Primera; en la religión, el Padre universal y amoroso; en la filosofía, el único ser que existe por sí mismo, no dependiendo de ningún otro ser para existir, pero que confiere benéficamente la realidad de la existencia a todas las cosas y a todos los demás seres. Pero se necesita la revelación para mostrar que la Causa Primera de la ciencia y la Unidad existente por sí misma de la filosofía son el Dios de la religión, lleno de misericordia y de bondad, y empeñado en llevar a cabo la supervivencia eterna de sus hijos terrestres.

\par
%\textsuperscript{(59.4)}
\textsuperscript{4:4.8} Anhelamos el concepto del Infinito, pero adoramos la idea-experiencia de Dios, nuestra capacidad para captar en cualquier momento y lugar los factores de personalidad y de divinidad de nuestro concepto más elevado de la Deidad.

\par
%\textsuperscript{(59.5)}
\textsuperscript{4:4.9} La conciencia de llevar una vida humana victoriosa en la Tierra nace de esa fe de la criatura que, cuando se enfrenta con el terrible espectáculo de las limitaciones humanas, se atreve a desafiar cada episodio recurrente de la existencia, declarando infaliblemente: Aunque yo no pueda hacer esto, alguien vive en mí que puede hacerlo y lo hará, una parte del Absoluto-Padre del universo de universos. Ésta es <<\textit{la victoria que triunfa sobre el mundo, vuestra fe misma}>>\footnote{\textit{Victoria por la fe}: 1 Jn 5:4.}.

\section*{5. Ideas erróneas sobre Dios}
\par
%\textsuperscript{(59.6)}
\textsuperscript{4:5.1} La tradición religiosa es la historia imperfectamente conservada de las experiencias de los hombres que conocían a Dios en las épocas pasadas, pero estos relatos son poco fiables como guías para llevar una vida religiosa, o como fuentes de información verídica sobre el Padre Universal. Estas creencias antiguas han sido invariablemente alteradas por el hecho de que el hombre primitivo era un creador de mitos.

\par
%\textsuperscript{(60.1)}
\textsuperscript{4:5.2} Una de las mayores fuentes de confusión en Urantia acerca de la naturaleza de Dios proviene de que vuestros libros sagrados no han logrado distinguir claramente entre las personalidades de la Trinidad del Paraíso ni entre la Deidad del Paraíso y los creadores y administradores de los universos locales. Durante las dispensaciones pasadas en las que existía una comprensión parcial, vuestros sacerdotes y profetas no lograron diferenciar claramente entre los Príncipes Planetarios, los Soberanos de los Sistemas, los Padres de las Constelaciones, los Hijos Creadores, los Gobernantes de los Superuniversos, el Ser Supremo y el Padre Universal. Muchos mensajes de personalidades subordinadas, tales como los Portadores de Vida y diversas órdenes de ángeles, han sido presentados en vuestros escritos como procedentes de Dios mismo. El pensamiento religioso urantiano confunde todavía las personalidades asociadas de la Deidad con el propio Padre Universal, de manera que todos están incluídos bajo una misma denominación.

\par
%\textsuperscript{(60.2)}
\textsuperscript{4:5.3} Los habitantes de Urantia continúan sufriendo la influencia de los conceptos primitivos sobre Dios. Los dioses que se comportan de manera violenta en la tormenta\footnote{\textit{La idea de la violencia de Dios en las tormentas}: Is 28:2; 29:6.}, que hacen temblar la tierra en su cólera\footnote{\textit{Idea de que hace temblar la tierra con su cólera}: Nah 1:2-6.} y fulminan a los hombres en su ira\footnote{\textit{Idea de que fulmina hombres con su ira}: 1 Cr 13:10; Hch 5:1-10.}; que infligen el juicio de su descontento\footnote{\textit{Idea de las calamidades como maldición}: Ez 5:16-17.} en las épocas de escasez y de inundaciones\footnote{\textit{Inundaciones como castigo}: Gn 6:6 ff.} ---éstos son los dioses de la religión primitiva; no son los Dioses que viven y gobiernan en los universos. Estos conceptos son una reliquia de los tiempos en que los hombres suponían que el universo estaba dirigido y dominado por los caprichos de estos dioses imaginarios. Pero el hombre mortal empieza a darse cuenta de que vive en un universo de ley y de orden relativos en lo que se refiere a la política y a la conducta administrativas de los Creadores Supremos y de los Controladores Supremos.

\par
%\textsuperscript{(60.3)}
\textsuperscript{4:5.4} La idea bárbara de apaciguar a un Dios enojado, de hacerse propicio a un Señor ofendido, de obtener los favores de la Deidad mediante sacrificios y penitencias e incluso por medio del derramamiento de sangre, representa una religión totalmente pueril y primitiva, una filosofía indigna de una época iluminada por la ciencia y la verdad. Estas creencias son completamente repulsivas para los seres celestiales y los gobernantes divinos que sirven y reinan en los universos. Es una afrenta a Dios creer, sostener o enseñar que hace falta derramar sangre inocente para ganar su favor o desviar una cólera divina ficticia.

\par
%\textsuperscript{(60.4)}
\textsuperscript{4:5.5} Los hebreos creían que <<\textit{sin derramamiento de sangre no podía haber remisión de los pecados}>>\footnote{\textit{Idea de remisión por la sangre}: Heb 9:22.}. No se habían liberado de la antigua idea pagana de que sólo la vista de la sangre podía apaciguar a los Dioses, aunque Moisés había realizado un progreso notable cuando prohibió los sacrificios humanos y los sustituyó por los sacrificios ceremoniales de animales, apropiados para la mentalidad primitiva de sus seguidores que eran beduinos infantiles.

\par
%\textsuperscript{(60.5)}
\textsuperscript{4:5.6} La donación de un Hijo Paradisiaco en vuestro mundo fue inherente a la situación de cierre de una era planetaria; fue inevitable y no era obligatoria para conseguir el favor de Dios. También dio la casualidad de que esta donación fue el acto final personal de un Hijo Creador en su larga aventura por lograr la soberanía experiencial de su universo. La enseñanza de que el corazón paternal de Dios, en toda su frialdad y dureza austeras, era tan insensible a las desgracias y tristezas de sus criaturas que su tierna misericordia no podía manifestarse hasta que viera a su Hijo irreprochable sangrar y morir en la cruz del Calvario, !`qué parodia del carácter infinito de Dios!

\par
%\textsuperscript{(60.6)}
\textsuperscript{4:5.7} Pero los habitantes de Urantia han de encontrar la manera de liberarse de estos antiguos errores y de estas supersticiones paganas respecto a la naturaleza del Padre Universal. La revelación de la verdad sobre Dios está empezando a aparecer, y la raza humana está destinada a conocer al Padre Universal en toda esa belleza de carácter y ese encanto de atributos que fueron tan magníficamente presentados por el Hijo Creador que residió en Urantia como Hijo del Hombre e Hijo de Dios.

\par
%\textsuperscript{(61.1)}
\textsuperscript{4:5.8} [Presentado por un Consejero Divino de Uversa.]


\chapter{Documento 5. Las relaciones de Dios con los individuos}
\par
%\textsuperscript{(62.1)}
\textsuperscript{5:0.1} SI LA mente finita del hombre es incapaz de comprender cómo un Dios tan grande y tan majestuoso como el Padre Universal puede descender de su residencia eterna de perfección infinita para fraternizar con las criaturas humanas individuales, entonces ese intelecto finito debe basar su seguridad de comunión divina en la verdad del hecho de que un fragmento real del Dios viviente reside en el intelecto de cada mortal de Urantia provisto de una mente normal y de una conciencia moral. Los Ajustadores del Pensamiento interiores son una parte de la Deidad eterna del Padre Paradisiaco. El hombre no tiene necesidad de ir más allá de su propia experiencia interior, donde el alma contempla la presencia de esta realidad espiritual, para encontrar a Dios y tratar de comulgar con él.

\par
%\textsuperscript{(62.2)}
\textsuperscript{5:0.2} Dios ha distribuido la infinidad de su naturaleza eterna en todas las realidades existenciales de sus seis coordinados absolutos, pero en cualquier momento puede establecer un contacto directo y personal con cualquier parte, o fase, o tipo de creación por mediación de sus fragmentos prepersonales. Y el Dios eterno también se ha reservado la prerrogativa de conceder la personalidad a los Creadores divinos y a las criaturas vivientes del universo de universos, mientras que además se ha reservado la prerrogativa de mantener un contacto directo y paternal con todos estos seres personales a través del circuito de la personalidad.

\section*{1. El camino de acceso a Dios}
\par
%\textsuperscript{(62.3)}
\textsuperscript{5:1.1} La incapacidad de las criaturas finitas para acercarse al Padre infinito no es inherente a la actitud distante del Padre, sino a la finitud y a las limitaciones materiales de los seres creados. La magnitud de la diferencia espiritual entre la más alta personalidad que existe en el universo y los grupos inferiores de inteligencias creadas es inconcebible. Si a los tipos de inteligencias inferiores les fuera posible ser transportados instantáneamente ante la presencia del Padre mismo, no sabrían que se encuentran allí. Se hallarían allí tan inconscientes de la presencia del Padre Universal como donde se encuentran ahora. El hombre mortal tiene por delante un larguísimo camino antes de que pueda solicitar, de manera coherente y dentro de lo posible, un salvoconducto que le permita llegar ante la presencia paradisiaca del Padre Universal. El hombre ha de ser trasladado espiritualmente muchas veces antes de que pueda alcanzar un plano que le proporcione la visión espiritual adecuada para ver siquiera a uno solo de los Siete Espíritus Maestros.

\par
%\textsuperscript{(62.4)}
\textsuperscript{5:1.2} Nuestro Padre no se oculta; no se encuentra en un retiro arbitrario. Ha movilizado los recursos de su sabiduría divina en un esfuerzo sin fin por revelarse a los hijos de sus dominios universales. La majestad de su amor lleva unidas una grandeza infinita y una generosidad inexpresable que le inducen a anhelar asociarse con cada ser creado que pueda comprenderlo, amarlo o acercarse a él; por consiguiente, vuestras limitaciones inherentes, inseparables de vuestra personalidad finita y de vuestra existencia material, son las que determinan el momento, el lugar y las circunstancias en que podréis alcanzar la meta del viaje de la ascensión humana, y encontraros en la presencia del Padre en el centro de todas las cosas.

\par
%\textsuperscript{(63.1)}
\textsuperscript{5:1.3} Aunque para acercaros a la presencia del Padre en el Paraíso debéis esperar a haber alcanzado los niveles finitos más elevados de la progresión espiritual, deberíais regocijaros en el reconocimiento de la posibilidad siempre presente de poder comulgar inmediatamente con el espíritu otorgado por el Padre, tan íntimamente asociado con vuestra alma interior y con vuestro yo en vías de espiritualización.

\par
%\textsuperscript{(63.2)}
\textsuperscript{5:1.4} Los mortales de los mundos del tiempo y del espacio pueden diferir enormemente en sus capacidades innatas y en sus dones intelectuales, pueden disfrutar de entornos excepcionalmente favorables para el avance social y el progreso moral, o pueden sufrir la carencia de casi toda ayuda humana para cultivarse y avanzar supuestamente en las artes de la civilización; pero las posibilidades para el progreso espiritual en la carrera de la ascensión son iguales para todos; los niveles crecientes de perspicacia espiritual y de significados cósmicos se alcanzan con absoluta independencia de todos los diferenciales sociomorales de los entornos materiales diversificados de los mundos evolutivos.

\par
%\textsuperscript{(63.3)}
\textsuperscript{5:1.5} Por mucho que difieran los mortales de Urantia en sus oportunidades y en sus dones intelectuales, sociales, económicos e incluso morales, no olvidéis que su dotación espiritual es uniforme y única. Todos disfrutan de la misma presencia divina del don procedente del Padre, y todos gozan del mismo privilegio de poder buscar una íntima comunión personal con este espíritu interior de origen divino, mientras que todos pueden elegir igualmente aceptar las directrices espirituales uniformes de estos Monitores de Misterio.

\par
%\textsuperscript{(63.4)}
\textsuperscript{5:1.6} Si un hombre mortal está motivado de manera sincera y espiritual, consagrado sin reservas a hacer la voluntad del Padre, entonces, puesto que está dotado espiritualmente de forma tan cierta y tan eficaz de un Ajustador divino interior, no puede dejar de materializarse en la experiencia de ese individuo la conciencia sublime de conocer a Dios y la seguridad celestial de sobrevivir para encontrar a Dios mediante la experiencia progresiva de volverse cada vez más semejante a él.

\par
%\textsuperscript{(63.5)}
\textsuperscript{5:1.7} El hombre está habitado espiritualmente por un Ajustador del Pensamiento que sobrevive. Si esa mente humana está sincera y espiritualmente motivada, si ese alma humana desea conocer a Dios y volverse semejante a él, si quiere hacer honradamente la voluntad del Padre, no existe ninguna influencia negativa de privaciones mortales ni ningún auténtico poder de interferencia posible que pueda impedir a ese alma divinamente motivada ascender con toda seguridad hasta las puertas del Paraíso.

\par
%\textsuperscript{(63.6)}
\textsuperscript{5:1.8} El Padre desea que todas sus criaturas estén en comunión personal con él. Tiene un lugar en el Paraíso para recibir a todos aquellos cuyo estado de supervivencia y cuya naturaleza espiritual hagan posible esta consecución. Por lo tanto, inscribid en vuestra filosofía, ahora y para siempre, que: para cada uno de vosotros y para todos nosotros, Dios es accesible, el Padre es alcanzable, el camino está abierto; las fuerzas del amor divino y los medios de la administración divina están todos entrelazados en un esfuerzo por facilitar el progreso de todas las inteligencias dignas de todos los universos hasta la presencia del Padre Universal en el Paraíso.

\par
%\textsuperscript{(63.7)}
\textsuperscript{5:1.9} El hecho de que se necesite un tiempo considerable para alcanzar a Dios no hace menos real la presencia y la personalidad del Infinito. Vuestra ascensión es una parte del circuito de los siete superuniversos, y aunque dais la vuelta a su alrededor un número incontable de veces, podéis esperar, en espíritu y en estado, que avanzaréis siempre hacia el interior. Podéis contar con que seréis trasladados de esfera en esfera, desde los circuitos exteriores siempre acercándoos al centro interior, y algún día, no lo dudéis, os encontraréis ante la presencia divina y central, y la veréis, hablando en lenguaje figurado, cara a cara. Es una cuestión de alcanzar los niveles espirituales reales y tangibles; y estos niveles espirituales son accesibles para cualquier ser que haya sido habitado por un Monitor de Misterio, y que haya fusionado posteriormente de manera eterna con ese Ajustador del Pensamiento.

\par
%\textsuperscript{(64.1)}
\textsuperscript{5:1.10} El Padre no se encuentra en un escondite espiritual, pero muchas de sus criaturas se han escondido en las brumas de sus propias decisiones obstinadas, y por el momento se han separado de la comunión con su espíritu y con el espíritu de su Hijo porque han elegido sus propios caminos perversos y porque han dado rienda suelta a la presunción de sus mentes intolerantes y de sus naturalezas no espirituales.

\par
%\textsuperscript{(64.2)}
\textsuperscript{5:1.11} El hombre mortal puede acercarse a Dios y alejarse repetidas veces de la voluntad divina durante tanto tiempo como conserve su poder de elección. El destino final del hombre no se decide hasta que ha perdido el poder de elegir la voluntad del Padre. El Padre no cierra nunca su corazón a las necesidades y a las súplicas de sus hijos. Es su progenitura la que cierra su corazón para siempre al poder de atracción del Padre cuando pierde final y definitivamente el deseo de hacer su voluntad divina ---la de conocerle y ser semejante a él. El destino eterno del hombre está igualmente asegurado cuando su fusión con el Ajustador proclama al universo que este ascendente ha hecho la elección final e irrevocable de vivir la voluntad del Padre.

\par
%\textsuperscript{(64.3)}
\textsuperscript{5:1.12} El gran Dios se pone en contacto directo con el hombre mortal y le concede una parte de su yo infinito, eterno e incomprensible para que viva y resida dentro de él. Dios se ha embarcado en la aventura eterna con el hombre. Si os sometéis a las directrices de las fuerzas espirituales que están en vosotros y alrededor de vosotros, no podréis dejar de alcanzar el alto destino que un Dios amoroso ha establecido como meta universal para sus criaturas ascendentes de los mundos evolutivos del espacio.

\section*{2. La presencia de Dios}
\par
%\textsuperscript{(64.4)}
\textsuperscript{5:2.1} La presencia física del Infinito es la realidad del universo material. La presencia mental de la Deidad ha de estar determinada por la profundidad de la experiencia intelectual individual y por el nivel evolutivo de la personalidad. La presencia espiritual de la Divinidad debe ser forzosamente diferencial en el universo. Está determinada por la capacidad espiritual de receptividad y por el grado en que la voluntad de la criatura está consagrada a hacer la voluntad divina.

\par
%\textsuperscript{(64.5)}
\textsuperscript{5:2.2} Dios vive en cada uno de sus hijos nacidos del espíritu. Los Hijos Paradisiacos siempre tienen acceso a la presencia de Dios, <<\textit{a la derecha del Padre}>>\footnote{\textit{A la derecha del Padre}: Sal 110:1; Mt 22:43-44; Mc 12:36; 16:19; Lc 20:42; Hch 7:55-56; Ro 8:34; Col 3:1; Heb 1:3; 8:1; 10:12; 12:2; 1 P 3:22.}, y todas las personalidades de sus criaturas tienen acceso al <<\textit{seno del Padre}>>\footnote{\textit{Seno del Padre}: Jn 1:18.}. Esto se refiere al circuito de la personalidad, cuando, dónde y comoquiera que se contacte con él, o suponga por lo demás un contacto y una comunión personal y consciente con el Padre Universal, ya sea en su residencia central o en cualquier otro lugar designado, como por ejemplo una de las siete esferas sagradas del Paraíso.

\par
%\textsuperscript{(64.6)}
\textsuperscript{5:2.3} Sin embargo, la presencia divina no se puede descubrir en ninguna parte de la naturaleza, ni siquiera en la vida de los mortales que conocen a Dios, de una manera tan plena y tan segura como en vuestro intento de comunión con el Monitor de Misterio interior, el Ajustador del Pensamiento del Paraíso. !Qué error soñar con un Dios lejano en los cielos, cuando el espíritu del Padre Universal vive dentro de vuestra propia mente!\footnote{\textit{El espíritu viviendo dentro}: Job 32:8,18; Is 63:10-11; Ez 37:14; Mt 10:20; Lc 17:21; Jn 17:21-23; Ro 8:9-11; 1 Co 3:16-17; 6:19; 2 Co 6:16; Gl 2:20; 1 Jn 3:24; 4:12-15; Ap 21:3.}

\par
%\textsuperscript{(64.7)}
\textsuperscript{5:2.4} Debido a este fragmento de Dios que reside en vosotros, y a medida que os armonicéis progresivamente con las directrices espirituales del Ajustador, podéis esperar discernir más plenamente la presencia y el poder transformador de aquellas otras influencias espirituales que os rodean e inciden en vosotros, pero que no funcionan como una parte integrante de vosotros. El hecho de que no seáis intelectualmente conscientes de un contacto estrecho e íntimo con el Ajustador interior no refuta en lo más mínimo una experiencia tan elevada. La prueba de la fraternidad con el Ajustador divino reside enteramente en la naturaleza y la extensión de los frutos del espíritu que produce la experiencia de la vida del creyente individual. <<\textit{Por sus frutos los conoceréis}>>\footnote{\textit{Por sus frutos los conoceréis}: Mt 7:16-20; Lc 6:43-44; Gl 5:22-23; Ef 5:9.}.

\par
%\textsuperscript{(65.1)}
\textsuperscript{5:2.5} A la mente material escasamente espiritualizada del hombre mortal le resulta extremadamente difícil experimentar una conciencia notable de las actividades espirituales de unas entidades divinas tales como los Ajustadores Paradisiacos. A medida que el alma creada conjuntamente por la mente y el Ajustador se vuelve cada vez más real, también se desarrolla una nueva fase de la conciencia del alma que es capaz de experimentar la presencia de los Monitores de Misterio, y de reconocer sus directrices espirituales y sus otras actividades supermateriales.

\par
%\textsuperscript{(65.2)}
\textsuperscript{5:2.6} Toda la experiencia de la comunión con el Ajustador implica poseer un estado moral, una motivación mental y una experiencia espiritual. La conciencia personal de un logro semejante permanece limitada principalmente, aunque no exclusivamente, al ámbito de la conciencia del alma, pero las pruebas aparecen pronto y son abundantes, manifestándose mediante los frutos del espíritu en la vida de todos aquellos que se ponen en contacto con este espíritu interior.

\section*{3. La verdadera adoración}
\par
%\textsuperscript{(65.3)}
\textsuperscript{5:3.1} Desde el punto de vista universal, las Deidades del Paraíso son como una sola, pero en sus relaciones espirituales con los seres como los que viven en Urantia son también tres personas distintas y separadas. Existe una diferencia entre las Divinidades en aquellas cuestiones relacionadas con las súplicas personales, la comunión y otras relaciones íntimas. En el sentido más elevado, adoramos al Padre Universal y sólo a él. Es verdad que podemos adorar y adoramos al Padre tal como se manifiesta en sus Hijos Creadores, pero es el Padre, directa o indirectamente, el que es venerado y adorado.

\par
%\textsuperscript{(65.4)}
\textsuperscript{5:3.2} Las súplicas de todo tipo pertenecen al ámbito del Hijo Eterno y de la organización espiritual del Hijo. Las oraciones, todas las comunicaciones formales, todo, salvo la adoración y la veneración del Padre Universal, son cuestiones que conciernen al universo local; normalmente no sobrepasan el ámbito jurisdiccional de un Hijo Creador. Pero la adoración es incluida sin duda en un circuito y enviada a la persona del Creador por medio del circuito de la personalidad del Padre. Creemos además que este registro del homenaje de una criatura habitada por un Ajustador es facilitado por la presencia del espíritu del Padre. Existe una enorme cantidad de pruebas que justifican esta creencia, y sé que todos los tipos de fragmentos del Padre poseen la facultad de registrar aceptablemente en la presencia del Padre Universal la adoración auténtica de sus súbditos. Los Ajustadores también utilizan indudablemente unos canales prepersonales directos de comunicación con Dios, y son igualmente capaces de utilizar los circuitos de la gravedad espiritual del Hijo Eterno.

\par
%\textsuperscript{(65.5)}
\textsuperscript{5:3.3} La adoración tiene su razón de ser en sí misma; la oración incorpora un elemento de interés personal o para sí mismo; ésta es la gran diferencia entre la adoración y la oración. La verdadera adoración no contiene en absoluto ninguna petición para sí mismo ni ningún otro elemento de interés personal; adoramos simplemente a Dios por lo que comprendemos que él es. La adoración no pide nada ni espera nada para el adorador. No adoramos al Padre porque podamos obtener algo de esa veneración; le rendimos esa devoción y nos dedicamos a esa adoración como reacción espontánea y natural al reconocimiento de la personalidad incomparable del Padre y a causa de su naturaleza encantadora y de sus atributos adorables.

\par
%\textsuperscript{(65.6)}
\textsuperscript{5:3.4} En el momento en que un elemento de interés personal se introduce en la adoración, la devoción pasa de la adoración a la oración, y sería más conveniente dirigirla a la persona del Hijo Eterno o del Hijo Creador. Pero en la experiencia religiosa práctica no existe ninguna razón por la que la oración no pueda dirigirse a Dios Padre como parte de una verdadera adoración.

\par
%\textsuperscript{(66.1)}
\textsuperscript{5:3.5} Cuando os ocupáis de los asuntos prácticos de vuestra vida diaria, estáis en manos de las personalidades espirituales que tienen su origen en la Fuente-Centro Tercera; cooperáis con los agentes del Actor Conjunto. Así es como adoráis a Dios, oráis al Hijo y comulgáis con él, y resolvéis los detalles de vuestra estancia terrestre en conexión con las inteligencias del Espíritu Infinito que trabajan en vuestro mundo y en todo vuestro universo.

\par
%\textsuperscript{(66.2)}
\textsuperscript{5:3.6} Los Hijos Creadores o Hijos Soberanos que presiden los destinos de los universos locales ocupan el lugar tanto del Padre Universal como del Hijo Eterno del Paraíso. Estos Hijos de los Universos reciben en nombre del Padre la adoración del culto, y prestan oído a las súplicas de sus súbditos que oran en todas las partes de sus creaciones respectivas. A efectos prácticos, un Hijo Miguel es Dios para los hijos de su universo local. Es la personificación del Padre Universal y del Hijo Eterno en el universo local. El Espíritu Infinito mantiene un contacto personal con los hijos de esos reinos a través de los Espíritus del Universo, las asociadas administrativas y creativas de los Hijos Creadores Paradisiacos.

\par
%\textsuperscript{(66.3)}
\textsuperscript{5:3.7} La adoración sincera implica la movilización de todos los poderes de la personalidad humana bajo la dominación del alma evolutiva, y sujetos a la dirección divina del Ajustador del Pensamiento asociado. La mente, con sus limitaciones materiales, nunca puede volverse extremadamente consciente del significado real de la verdadera adoración. La comprensión humana de la realidad de la experiencia de la adoración está determinada principalmente por el estado de desarrollo de su alma inmortal en evolución. El crecimiento espiritual del alma tiene lugar de manera totalmente independiente de la conciencia intelectual de sí mismo.

\par
%\textsuperscript{(66.4)}
\textsuperscript{5:3.8} La experiencia de la adoración consiste en el intento sublime del Ajustador prometido por comunicar al Padre divino los anhelos inexpresables y las aspiraciones indecibles del alma humana ---creación conjunta de la mente mortal que busca a Dios y del Ajustador inmortal que revela a Dios. Por consiguiente, la adoración es el acto mediante el cual la mente material consiente que su yo en vías de espiritualizarse intente comunicarse con Dios, bajo la dirección del espíritu asociado, como hijo por la fe del Padre Universal. La mente mortal consiente en adorar; el alma inmortal anhela e inicia la adoración; la presencia divina del Ajustador dirige esta adoración en nombre de la mente mortal y del alma inmortal evolutiva. A fin de cuentas, la verdadera adoración se convierte en una experiencia que se lleva a cabo en cuatro niveles cósmicos: el intelectual, el morontial, el espiritual y el personal ---la conciencia de la mente, del alma y del espíritu, y su unificación en la personalidad.

\section*{4. Dios en la religión}
\par
%\textsuperscript{(66.5)}
\textsuperscript{5:4.1} La moralidad de las religiones evolutivas \textit{empuja} a los hombres hacia adelante en la búsqueda de Dios mediante la fuerza motriz del miedo. Las religiones de la revelación \textit{atraen} a los hombres hacia la búsqueda de un Dios de amor porque anhelan volverse semejantes a él. Pero la religión no es simplemente un sentimiento pasivo de <<\textit{dependencia absoluta}>> y de <<\textit{certeza de la supervivencia}>>; es una experiencia viviente y dinámica consistente en alcanzar la divinidad, basada en el servicio a la humanidad.

\par
%\textsuperscript{(66.6)}
\textsuperscript{5:4.2} El gran servicio inmediato de la verdadera religión es el establecimiento de una unidad duradera en la experiencia humana, una paz constante y una seguridad profunda\footnote{\textit{Paz duradera}: Sal 119:165; Is 26:3; Nm 6:26; Lc 1:79; 2:14; Jn 14:27; 16:33; Ro 14:17; 1 Co 14:33; Flp 4:7. \textit{Seguridad profunda}: Is 32:17; Ef 3:12; Col 2:2; 1 Ts 1:5; 2 Ti 1:12; Heb 6:11; 10:22; 1 Jn 3:19.}. Entre los hombres primitivos, incluso el politeísmo es una unificación relativa del concepto evolutivo de la Deidad; el politeísmo es el monoteísmo en formación. Tarde o temprano, Dios está destinado a ser comprendido como la realidad de los valores, la sustancia de los significados y la vida de la verdad.

\par
%\textsuperscript{(67.1)}
\textsuperscript{5:4.3} Dios no es solamente el que determina el destino; él \textit{es} el destino eterno del hombre. Todas las actividades humanas no religiosas intentan doblegar el universo al servicio deformante del yo; el individuo verdaderamente religioso intenta identificar su yo con el universo, y luego dedicar las actividades de ese yo unificado al servicio de la familia universal de sus semejantes, humanos y superhumanos.

\par
%\textsuperscript{(67.2)}
\textsuperscript{5:4.4} Los dominios de la filosofía y del arte se interponen entre las actividades religiosas y no religiosas del yo humano. A través del arte y la filosofía, el hombre con mentalidad materialista se siente persuadido a contemplar las realidades espirituales y los valores universales que tienen significados eternos.

\par
%\textsuperscript{(67.3)}
\textsuperscript{5:4.5} Todas las religiones enseñan la adoración de la Deidad y alguna doctrina de salvación humana. La religión budista promete salvar del sufrimiento, una paz sin fin; la religión judía promete salvar de las dificultades, una prosperidad basada en la rectitud; la religión griega prometía salvar de la falta de armonía, de la fealdad, gracias al reconocimiento de la belleza; el cristianismo promete salvar del pecado, la santidad; el mahometismo ofrece liberaros de las rigurosas reglas morales del judaísmo y del cristianismo. La religión de Jesús \textit{salva} del yo, libera de los males del aislamiento de la criatura en el tiempo y en la eternidad.

\par
%\textsuperscript{(67.4)}
\textsuperscript{5:4.6} Los hebreos basaban su religión en la bondad; los griegos, en la belleza; las dos religiones buscaban la verdad. Jesús reveló un Dios de amor, y el amor engloba totalmente a la verdad, la belleza y la bondad.

\par
%\textsuperscript{(67.5)}
\textsuperscript{5:4.7} Los zoroástricos tenían una religión de moralidad; los hindúes, una religión de metafísica; los confucionistas, una religión de ética. Jesús vivió una religión de \textit{servicio.} Todas estas religiones son valiosas en la medida en que se aproximan válidamente a la religión de Jesús. La religión está destinada a convertirse en la realidad de la unificación espiritual de todo lo que es bueno, hermoso y verdadero en la experiencia humana.

\par
%\textsuperscript{(67.6)}
\textsuperscript{5:4.8} La religión griega tenía un lema: <<\textit{Conócete a ti mismo}>>; los hebreos centraban su enseñanza en <<\textit{Conoced a vuestro Dios}>>\footnote{\textit{Conoce a Dios}: 1 Cr 28:9; Jer 9:24; 31:34; Os 6:6.}; los cristianos predican un evangelio dirigido al <<\textit{conocimiento del Señor Jesucristo}>>\footnote{\textit{Conoce a Jesucristo}: Flp 3:8; 2 P 1:8; 3:18.}; Jesús proclamó la buena nueva de <<\textit{conoce a Dios y conócete a ti mismo como hijo de Dios}>>\footnote{\textit{Conoce a Dios y a ti como hijo}: Jn 14:7,17,20; Flp 3:9-10; Heb 8:11; 1 Jn 2:3-5. \textit{Somos hijos de Dios}: 1 Cr 22:10; Sal 2:7; Is 56:5; Mt 5:9,16,45; Lc 20:36; Jn 1:12-13; 11:52; Hch 17:28-29; Ro 8:14-17,19,21; 9:26; 2 Co 6:18; Gl 3:26; 4:5-7; Ef 1:5; Flp 2:15; Heb 12:5-8; 1 Jn 3:1-2,10; 5:2; Ap 21:7; 2 Sam 7:14.}. Estos conceptos diferentes sobre la meta de la religión determinan la actitud del individuo en las diversas situaciones de la vida, y presagian la profundidad de su adoración y la naturaleza de sus hábitos personales de oración. El estado espiritual de cualquier religión se puede determinar por la naturaleza de sus oraciones.

\par
%\textsuperscript{(67.7)}
\textsuperscript{5:4.9} El concepto de un Dios semihumano y celoso es una transición inevitable entre el politeísmo y el sublime monoteísmo. Un elevado antropomorfismo es el nivel más alto que puede alcanzar una religión puramente evolutiva. El cristianismo ha elevado el concepto del antropomorfismo desde el ideal de lo humano hasta el concepto trascendente y divino de la persona del Cristo glorificado. Éste es el antropomorfismo más elevado que el hombre pueda concebir jamás.

\par
%\textsuperscript{(67.8)}
\textsuperscript{5:4.10} El concepto cristiano de Dios es un intento por combinar tres enseñanzas diferentes:

\par
%\textsuperscript{(67.9)}
\textsuperscript{5:4.11} 1. \textit{El concepto hebreo} ---Dios como defensor de los valores morales, un Dios justo.

\par
%\textsuperscript{(67.10)}
\textsuperscript{5:4.12} 2. \textit{El concepto griego} ---Dios como unificador, un Dios de sabiduría.

\par
%\textsuperscript{(68.1)}
\textsuperscript{5:4.13} 3. \textit{El concepto de Jesús} ---Dios como amigo viviente, un Padre amoroso, la presencia divina.

\par
%\textsuperscript{(68.2)}
\textsuperscript{5:4.14} Por lo tanto, ha de ser evidente que la teología compuesta cristiana encuentra grandes dificultades para conseguir la coherencia. Estas dificultades se agravan aún más por el hecho de que las doctrinas del cristianismo primitivo estaban basadas generalmente en la experiencia religiosa personal de tres personas diferentes: Filón de Alejandría, Jesús de Nazaret y Pablo de Tarso.

\par
%\textsuperscript{(68.3)}
\textsuperscript{5:4.15} Cuando estudiéis la vida religiosa de Jesús, consideradlo de manera positiva. No penséis tanto en que estaba libre de pecado, sino en su rectitud, en su servicio amoroso. Jesús elevó el amor pasivo, revelado en el concepto hebreo del Padre celestial, hasta el afecto \textit{activo} superior, amoroso por sus criaturas, de un Dios que es el Padre de todos los individuos, incluso de los malhechores.

\section*{5. La conciencia de Dios}
\par
%\textsuperscript{(68.4)}
\textsuperscript{5:5.1} El hecho de ser consciente de sí mismo da origen a la moralidad; ésta es superanimal pero totalmente evolutiva. La evolución humana abarca en su desarrollo todos los dones que preceden a la concesión de los Ajustadores y al derramamiento del Espíritu de la Verdad. Pero alcanzar los niveles de la moralidad no libera al hombre de las luchas reales de su vida como mortal. El entorno físico del hombre implica la lucha por la existencia; el medio ambiente social necesita ajustes éticos; las situaciones morales requieren que se hagan elecciones en las esferas más elevadas de la razón; la experiencia espiritual (una vez que se tiene conciencia de Dios) exige que el hombre lo encuentre y se esfuerce sinceramente por parecerse a él.

\par
%\textsuperscript{(68.5)}
\textsuperscript{5:5.2} La religión no está basada en los hechos de la ciencia, ni en las obligaciones de la sociedad, ni en las suposiciones de la filosofía, ni en los deberes implícitos de la moralidad. La religión es un campo independiente de reacción humana a las situaciones de la vida, y aparece infaliblemente en todas las fases del desarrollo humano posteriores a la moral. La religión puede impregnar los cuatro niveles de la comprensión de los valores y del disfrute de la fraternidad universal: el nivel físico o material de la preservación de sí mismo; el nivel social o emocional de la fraternidad; el nivel moral de la razón o del deber; y el nivel espiritual de la conciencia de la fraternidad universal mediante la adoración divina.

\par
%\textsuperscript{(68.6)}
\textsuperscript{5:5.3} El científico que busca los hechos concibe a Dios como la Causa Primera, un Dios de fuerza. El artista emotivo ve a Dios como el ideal de la belleza, un Dios de estética. El filósofo razonador se siente a veces inclinado a proponer un Dios de unidad universal, e incluso una Deidad panteísta. La persona religiosa que tiene fe cree en un Dios que patrocina la supervivencia, el Padre que está en los cielos, el Dios de amor.

\par
%\textsuperscript{(68.7)}
\textsuperscript{5:5.4} La conducta moral precede siempre a la religión evolutiva e incluso es una parte de la religión revelada, pero nunca es la totalidad de la experiencia religiosa. El servicio social es el resultado de una manera moral de pensar y religiosa de vivir. La moralidad no conduce biológicamente a los niveles espirituales más elevados de la experiencia religiosa. La adoración de la belleza abstracta no es la veneración de Dios; la exaltación de la naturaleza o la veneración de la unidad tampoco son la adoración de Dios.

\par
%\textsuperscript{(68.8)}
\textsuperscript{5:5.5} La religión evolutiva es la madre de la ciencia, del arte y de la filosofía que han elevado al hombre hasta el nivel en que es receptivo a la religión revelada, incluyendo la concesión de los Ajustadores y la venida del Espíritu de la Verdad. El cuadro evolutivo de la existencia humana comienza y termina con la religión, aunque con calidades muy diferentes de religión, una evolutiva y biológica, la otra revelada y periódica. Así pues, aunque la religión es normal y natural para el hombre, es también opcional. El hombre no tiene por qué ser religioso en contra de su voluntad.

\par
%\textsuperscript{(69.1)}
\textsuperscript{5:5.6} Como la experiencia religiosa es esencialmente espiritual, nunca puede ser plenamente comprendida por la mente material; de ahí la función de la teología, que es la psicología de la religión. La doctrina fundamental de la comprensión humana de Dios crea una paradoja en el entendimiento finito. A la lógica humana y a la razón finita les resulta casi imposible armonizar el concepto de la inmanencia divina, un Dios interior que forma parte de cada individuo, con la idea de la trascendencia de Dios, la dominación divina del universo de universos. Estos dos conceptos esenciales de la Deidad deben ser unificados mediante la captación por la fe del concepto de la trascendencia de un Dios personal y la comprensión de la presencia interior de un fragmento de ese Dios, con el objeto de justificar la adoración inteligente y validar la esperanza de la supervivencia de la personalidad. Las dificultades y las paradojas de la religión son inherentes al hecho de que las realidades de la religión sobrepasan por completo la capacidad de comprensión intelectual de los mortales.

\par
%\textsuperscript{(69.2)}
\textsuperscript{5:5.7} El hombre mortal obtiene tres grandes satisfacciones de su experiencia religiosa, incluso durante los días de su estancia temporal en la Tierra:

\par
%\textsuperscript{(69.3)}
\textsuperscript{5:5.8} 1. \textit{Intelectualmente,} adquiere la satisfacción de una conciencia humana más unificada.

\par
%\textsuperscript{(69.4)}
\textsuperscript{5:5.9} 2. \textit{Filosóficamente,} disfruta de la justificación de sus ideales de los valores morales.

\par
%\textsuperscript{(69.5)}
\textsuperscript{5:5.10} 3. \textit{Espiritualmente,} crece en la experiencia del compañerismo divino, en las satisfacciones espirituales de la verdadera adoración.

\par
%\textsuperscript{(69.6)}
\textsuperscript{5:5.11} La conciencia de Dios, tal como la experimentan los mortales evolutivos de los mundos, debe consistir en tres factores variables, en tres niveles diferenciales de comprensión de la realidad. En primer lugar está la conciencia mental ---la comprensión de la \textit{idea} de Dios. Luego le sigue la conciencia del alma ---la comprensión del \textit{ideal} de Dios. Finalmente despunta la conciencia del espíritu ---la comprensión de la \textit{realidad espiritual} de Dios. Mediante la unificación de estos factores de la comprensión divina, por muy incompleta que ésta sea, la personalidad mortal despliega constantemente, sobre todos los niveles conscientes, una comprensión de la \textit{personalidad} de Dios. En aquellos mortales que han alcanzado el Cuerpo de la Finalidad, todo esto conducirá en su momento a la comprensión de la \textit{supremacía} de Dios, y puede traducirse posteriormente en la comprensión de la \textit{ultimidad} de Dios, una fase de la superconciencia absonita del Padre Paradisiaco.

\par
%\textsuperscript{(69.7)}
\textsuperscript{5:5.12} La experiencia de la conciencia de Dios sigue siendo la misma de generación en generación, pero a medida que avanza el conocimiento humano en cada época, el concepto filosófico y las definiciones teológicas de Dios \textit{deben} cambiar. El conocimiento sobre Dios, la conciencia religiosa, es una realidad universal, pero por muy válida (real) que sea la experiencia religiosa, debe estar dispuesta a someterse a la crítica inteligente y a una interpretación filosófica razonable; no debe tratar de ser una cosa separada de la totalidad de la experiencia humana.

\par
%\textsuperscript{(69.8)}
\textsuperscript{5:5.13} La supervivencia eterna de la personalidad depende enteramente de la elección de la mente mortal, cuyas decisiones determinan el potencial de supervivencia del alma inmortal. Cuando la mente cree en Dios y el alma conoce a Dios, cuando con el Ajustador que estimula todos \textit{desean} a Dios, entonces la supervivencia está asegurada. Las limitaciones del intelecto, las restricciones de la educación, la privación de cultura, el empobrecimiento de la posición social e incluso unos criterios morales humanos inferiores ocasionados por la falta desafortunada de ventajas educativas, culturales y sociales, no pueden invalidar la presencia del espíritu divino en esos individuos desafortunados y humanamente perjudicados, pero creyentes. La presencia interior del Monitor de Misterio constituye el comienzo, y asegura la posibilidad, del potencial de crecimiento y de supervivencia del alma inmortal.

\par
%\textsuperscript{(70.1)}
\textsuperscript{5:5.14} La capacidad de los padres mortales para procrear no está basada en su nivel educativo, cultural, social o económico. La unión de los factores parentales en condiciones naturales es completamente suficiente para dar comienzo a una descendencia. Una mente humana que discierne el bien y el mal y que posee la capacidad de adorar a Dios, en unión con un Ajustador divino, es todo lo que necesita ese mortal para dar comienzo y fomentar el nacimiento de su alma inmortal con sus cualidades de supervivencia, si ese individuo dotado de espíritu busca a Dios y desea sinceramente volverse como él, elige honradamente hacer la voluntad del Padre que está en los cielos.

\section*{6. El Dios de la personalidad}
\par
%\textsuperscript{(70.2)}
\textsuperscript{5:6.1} El Padre Universal es el Dios de las personalidades. El campo de la personalidad en el universo, desde las criaturas mortales y materiales más humildes con estatus de personalidad hasta las personas más elevadas con dignidad de creadores y con estatus divino, tiene su centro y su circunferencia en el Padre Universal. Dios Padre es el que concede y conserva cada personalidad. Y el Padre Paradisiaco es igualmente el destino de todas aquellas personalidades finitas que eligen sinceramente hacer la voluntad divina, de aquellos que aman a Dios y anhelan parecerse a él.

\par
%\textsuperscript{(70.3)}
\textsuperscript{5:6.2} La personalidad es uno de los misterios no resueltos de los universos. Podemos formarnos unos conceptos adecuados de los factores que entran en la composición de los diversos tipos y niveles de personalidades, pero no comprendemos plenamente la naturaleza real de la personalidad misma. Percibimos claramente los numerosos factores que, una vez reunidos, constituyen el vehículo de la personalidad humana, pero no comprendemos plenamente la naturaleza y el significado de esa personalidad finita.

\par
%\textsuperscript{(70.4)}
\textsuperscript{5:6.3} La personalidad es potencial en todas las criaturas que poseen una dotación mental comprendida entre el mínimo de conciencia de sí mismo hasta el máximo de conciencia de Dios. Pero la dotación mental por sí sola no es la personalidad, ni tampoco lo es el espíritu ni la energía física. La personalidad es esa cualidad y ese valor, dentro de la realidad cósmica, que es concedida exclusivamente por Dios Padre a aquellos sistemas vivientes donde las energías de la materia, la mente y el espíritu están asociadas y coordinadas. La personalidad tampoco es una consecución progresiva. La personalidad puede ser material o espiritual, pero la personalidad está o no está. Aquello que es distinto a lo personal nunca alcanza el nivel de lo personal, salvo mediante un acto directo del Padre Paradisiaco.

\par
%\textsuperscript{(70.5)}
\textsuperscript{5:6.4} La concesión de la personalidad es una ocupación exclusiva del Padre Universal, es la personalización de los sistemas energéticos vivientes, a los cuales dota de los atributos de una conciencia creativa relativa y del control de la misma por medio del libre albedrío. No hay ninguna personalidad que no provenga de Dios Padre, y no existe ninguna personalidad si no es gracias a Dios Padre. Los atributos fundamentales de la individualidad humana, así como el Ajustador, núcleo absoluto de la personalidad humana, son dones del Padre Universal actuando en su terreno exclusivamente personal de ministerio cósmico.

\par
%\textsuperscript{(70.6)}
\textsuperscript{5:6.5} Los Ajustadores, cuyo estado es prepersonal, residen en numerosos tipos de criaturas mortales, asegurando así a estos mismos seres la posibilidad de sobrevivir a la muerte física para personalizarse como criaturas morontiales, con el potencial de alcanzar el estado espiritual último. Porque, cuando la mente de una criatura dotada de personalidad está habitada por un fragmento del espíritu del Dios eterno, el don prepersonal del Padre personal, entonces esa personalidad finita posee el potencial de lo divino y de lo eterno, y aspira a un destino semejante al del Último, tendiendo incluso hacia la comprensión del Absoluto.

\par
%\textsuperscript{(71.1)}
\textsuperscript{5:6.6} La capacidad para recibir la personalidad divina es inherente al Ajustador prepersonal; la capacidad para recibir la personalidad humana existe en potencia en la dotación mental cósmica del ser humano. Pero la personalidad experiencial del hombre mortal no es observable como realidad activa y funcional hasta después de que el vehículo vital material de la criatura mortal ha sido tocado por la divinidad liberadora del Padre Universal, siendo lanzada así a los mares de la experiencia como una personalidad consciente de sí misma, capaz (relativamente) de determinarse y de crearse a sí misma. El yo material es verdaderamente \textit{personal sin ninguna restricción.}

\par
%\textsuperscript{(71.2)}
\textsuperscript{5:6.7} El yo material posee una personalidad y una identidad, una identidad temporal; el Ajustador espiritual prepersonal posee también una identidad, una identidad eterna. Esta personalidad material y esta prepersonalidad espiritual son capaces de unir sus atributos creadores como para traer a la existencia la identidad sobreviviente del alma inmortal.

\par
%\textsuperscript{(71.3)}
\textsuperscript{5:6.8} Una vez que ha asegurado así el crecimiento del alma inmortal y que ha liberado al yo interior del hombre de las cadenas de la dependencia absoluta a la causalidad precedente, el Padre se retira. Así pues, una vez que el hombre ha sido liberado así de las cadenas de la reacción a la causalidad, al menos en lo relacionado con el destino eterno, y que se ha facilitado el crecimiento del yo inmortal, el alma, queda en manos del hombre mismo el querer o el impedir la creación de ese yo sobreviviente y eterno que será suyo si así lo elige. Ningún otro ser, ninguna fuerza, ningún creador o agente en todo el extenso universo de universos puede interferir en ninguna medida en la soberanía absoluta del libre albedrío humano, tal como éste funciona dentro del campo de la elección, en lo referente al destino eterno de la personalidad del mortal que escoge. En lo que concierne a la supervivencia eterna, Dios ha decretado que la voluntad material y humana es soberana, y este decreto es absoluto.

\par
%\textsuperscript{(71.4)}
\textsuperscript{5:6.9} La concesión de la personalidad a las criaturas les confiere una liberación relativa respecto a la reacción servil a la causalidad precedente, y la personalidad de todos estos seres morales, evolutivos u otros, está centrada en la personalidad del Padre Universal. Siempre es atraída hacia su presencia en el Paraíso por ese parentesco de existencia que constituye el inmenso círculo familiar universal y el circuito fraternal del Dios eterno\footnote{\textit{Gravedad espiritual}: Jer 31:3; Jn 6:44; 12:32.}. Existe un parentesco de espontaneidad divina en toda personalidad.

\par
%\textsuperscript{(71.5)}
\textsuperscript{5:6.10} El circuito de personalidad del universo de universos está centrado en la persona del Padre Universal, y el Padre Paradisiaco es personalmente consciente de todas las personalidades de todos los niveles de existencia consciente, y se mantiene en contacto personal con ellas. Esta conciencia sobre las personalidades de toda la creación existe independientemente de la misión de los Ajustadores del Pensamiento.

\par
%\textsuperscript{(71.6)}
\textsuperscript{5:6.11} Al igual que toda la gravedad está incluida en el circuito de la Isla del Paraíso, toda mente en el circuito del Actor Conjunto y todo espíritu en el Hijo Eterno, del mismo modo toda personalidad está incluida en el circuito de la presencia personal del Padre Universal, y este circuito transmite infaliblemente la adoración de todas las personalidades a la Personalidad Original y Eterna.

\par
%\textsuperscript{(71.7)}
\textsuperscript{5:6.12} En cuanto a aquellas personalidades que no están habitadas por un Ajustador, el Padre Universal también les ha concedido el atributo de la libertad de elección, y estas personas están incluidas igualmente en el gran circuito del amor divino, el circuito de personalidad del Padre Universal. Dios asegura la elección soberana a todas las verdaderas personalidades. Ninguna criatura personal puede ser forzada a emprender la aventura eterna; la puerta de la eternidad sólo se abre en respuesta a la libre elección de los hijos con libre albedrío del Dios del libre albedrío.

\par
%\textsuperscript{(72.1)}
\textsuperscript{5:6.13} Esto representa mis esfuerzos por exponer las relaciones del Dios viviente con los hijos del tiempo. Y cuando todo ha sido dicho y hecho, no puedo hacer nada más útil que reiterar que Dios es vuestro Padre en el universo, y que todos sois sus hijos planetarios.

\par
%\textsuperscript{(72.2)}
\textsuperscript{5:6.14} [Este documento es el quinto y último de la serie que describe al Padre Universal, presentada por un Consejero Divino de Uversa.]


\chapter{Documento 6. El Hijo Eterno}
\par
%\textsuperscript{(73.1)}
\textsuperscript{6:0.1} EL Hijo Eterno es la expresión perfecta y final del <<\textit{primer}>> concepto personal y absoluto del Padre Universal. Por consiguiente, en cualquier momento y de cualquier manera que el Padre se exprese de forma personal y absoluta, lo hace a través de su Hijo Eterno, que siempre ha sido, es ahora, y será siempre el Verbo viviente y divino. Este Hijo Eterno reside en el centro de todas las cosas en asociación con el Padre Eterno y Universal cuya presencia personal envuelve directamente.

\par
%\textsuperscript{(73.2)}
\textsuperscript{6:0.2} Hablamos del <<\textit{primer}>> pensamiento de Dios y aludimos a un imposible origen del Hijo Eterno en el tiempo con el objeto de lograr acceder a los canales de pensamiento del intelecto humano. Estas deformaciones de lenguaje representan nuestros mejores esfuerzos por llegar a un compromiso que permita ponernos en contacto con la mente de las criaturas mortales atadas al tiempo. En sentido secuencial, el Padre Universal no ha podido tener nunca un primer pensamiento, ni el Hijo Eterno un principio. Pero me han ordenado describir las realidades de la eternidad a la mente de los mortales limitada por el tiempo con estos símbolos de pensamiento, y designar las relaciones de la eternidad mediante estos conceptos temporales de secuencia.

\par
%\textsuperscript{(73.3)}
\textsuperscript{6:0.3} El Hijo Eterno es la personalización espiritual del concepto universal e infinito del Padre Paradisiaco sobre la realidad divina, el espíritu incalificado y la personalidad absoluta. Por eso el Hijo constituye la revelación divina de la identidad como creador del Padre Universal. La personalidad perfecta del Hijo revela que el Padre es realmente la fuente eterna y universal de todos los significados y valores de aquello que es espiritual, volitivo, intencional y personal.

\par
%\textsuperscript{(73.4)}
\textsuperscript{6:0.4} En un esfuerzo por permitir que la mente finita del tiempo se forme un concepto secuencial de las relaciones entre los seres eternos e infinitos de la Trinidad del Paraíso, utilizamos licencias de concepción tales como la de referirnos al <<\textit{primer concepto personal, universal e infinito del Padre}>>. Me resulta imposible transmitirle a la mente humana una idea adecuada de las relaciones eternas entre las Deidades; por eso empleo unos términos que le den a la mente finita alguna idea de las relaciones de estos seres eternos en las eras posteriores del tiempo. Creemos que el Hijo surgió del Padre; nos enseñan que los dos son incondicionalmente eternos. Por lo tanto es evidente que ninguna criatura temporal podrá nunca comprender plenamente este misterio de un Hijo que desciende del Padre, y que sin embargo es coordinadamente eterno con el Padre mismo.

\section*{1. La identidad del Hijo Eterno}
\par
%\textsuperscript{(73.5)}
\textsuperscript{6:1.1} El Hijo Eterno es el Hijo original y unigénito de Dios\footnote{\textit{Hijo unigénito}: Jn 1:14,18; 3:16,18; 1 Jn 4:9.}. Es Dios Hijo, la Segunda Persona de la Deidad y el creador asociado de todas las cosas. Así como el Padre es la Gran Fuente-Centro Primera, el Hijo Eterno es la Gran Fuente-Centro Segunda.

\par
%\textsuperscript{(74.1)}
\textsuperscript{6:1.2} El Hijo Eterno es el centro espiritual y el administrador divino del gobierno espiritual del universo de universos. El Padre Universal es en primer lugar un creador y luego un controlador; el Hijo Eterno es en primer lugar un cocreador y luego un \textit{administrador espiritual.} <<\textit{Dios es espíritu}>>\footnote{\textit{Dios es espíritu}: Jn 4:24.}, y el Hijo es una revelación personal de ese espíritu. La Fuente-Centro Primera es el Absoluto Volitivo; la Fuente-Centro Segunda es el Absoluto de la Personalidad.

\par
%\textsuperscript{(74.2)}
\textsuperscript{6:1.3} El Padre Universal no actúa nunca personalmente como creador, excepto en conjunción con el Hijo o con la acción coordinada del Hijo\footnote{\textit{Dios ha hecho todo}: Gn 1:1; 2:4; 5:1-2; Ex 20:11; 31:17; 2 Re 19:15; 2 Cr 2:12; Neh 9:6; Sal 115:15; 121:2; 124:8; 134:3; 146:6; Eclo 1:1; 33:10; Is 37:16; 40:26-28; 42:5; 45:12,18; Jer 10:11-12; 32:17; 51:15; Bar 3:32; Am 4:13; Mal 2:10; Mc 13:19; Hch 4:24; 14:15; Ef 3:9; Col 1:16; Heb 1:2; 1 P 4:19; Ap 4:11; 10:6; 14:7.}. Si el autor del Nuevo Testamento se hubiera referido al Hijo Eterno, habría dicho la verdad cuando escribió: <<\textit{En el principio era el Verbo, y el Verbo estaba con Dios, y el Verbo era Dios\footnote{\textit{El Verbo y Dios}: Jn 1:1; 5:18; 10:30,38; 14:9-11,20; 17:11,21-22}. Todas las cosas fueron hechas por él, y sin él no se habría hecho nada de lo que se ha hecho}>>\footnote{\textit{Dios ha hecho todo}: Jn 1:1-3.}.

\par
%\textsuperscript{(74.3)}
\textsuperscript{6:1.4} Cuando un Hijo del Hijo Eterno apareció en Urantia, aquellos que fraternizaron con este ser divino en su forma humana se refirieron a él como <<\textit{Aquel que existía desde el principio, a quien hemos oído, a quien hemos visto con nuestros ojos, a quien hemos contemplado, y que nuestras manos han tocado, el Verbo mismo de la vida}>>\footnote{\textit{Aquel que existía desde el principio}: 1 Jn 1:1.}. Y este Hijo donador provenía del Padre tan ciertamente como el Hijo Original, tal como lo sugirió en una de sus oraciones terrestres: <<\textit{Y ahora, Padre mío, glorifícame con tu propio ser, con la gloria que tenía contigo antes de que existiera este mundo}>>\footnote{\textit{Glorifícame con tu propio ser}: Jn 17:5.}.

\par
%\textsuperscript{(74.4)}
\textsuperscript{6:1.5} Al Hijo Eterno se le conoce por distintos nombres en los diversos universos. En el universo central se le conoce como la Fuente Coordinada, el Cocreador, y el Absoluto Asociado. En Uversa, sede de vuestro superuniverso, designamos al Hijo como el Centro Espiritual Coordinado y como el Administrador Espiritual Eterno. En Salvington, sede de vuestro universo local, este Hijo es conocido como la Eterna Fuente-Centro Segunda. Los Melquisedeks se refieren a él como el Hijo de los Hijos. En vuestro mundo, pero no en vuestro sistema de esferas habitadas, este Hijo Original ha sido confundido con un Hijo Creador coordinado, con Miguel de Nebadon, que se donó a las razas mortales de Urantia.

\par
%\textsuperscript{(74.5)}
\textsuperscript{6:1.6} Aunque a todos los Hijos Paradisiacos se les puede llamar apropiadamente Hijos de Dios, tenemos la costumbre de reservar el nombre de <<\textit{Hijo Eterno}>> a este Hijo Original, la Fuente-Centro Segunda, cocreador con el Padre Universal del universo central de poder y de perfección, y cocreador de todos los otros Hijos divinos que descienden de las Deidades infinitas.

\section*{2. La naturaleza del Hijo Eterno}
\par
%\textsuperscript{(74.6)}
\textsuperscript{6:2.1} El Hijo Eterno es tan invariable y tan infinitamente digno de confianza como el Padre Universal. Es también tan espiritual como el Padre, un espíritu tan verdaderamente ilimitado como él. Para vosotros que sois de origen humilde, el Hijo parecería ser más personal puesto que se encuentra, en accesibilidad, un paso más cerca de vosotros que el Padre Universal.

\par
%\textsuperscript{(74.7)}
\textsuperscript{6:2.2} El Hijo Eterno es el Verbo eterno de Dios. Es enteramente semejante al Padre; de hecho, el Hijo Eterno \textit{es} Dios Padre manifestado personalmente al universo de universos. Y así se ha podido, se puede, y se podrá decir siempre del Hijo Eterno y de todos los Hijos Creadores coordinados: <<\textit{El que ha visto al Hijo, ha visto al Padre}>>\footnote{\textit{El me ve ha visto al Padre}: Jn 14:9.}.

\par
%\textsuperscript{(74.8)}
\textsuperscript{6:2.3} La naturaleza del Hijo es enteramente semejante a la del Padre espiritual. Cuando adoramos al Padre Universal, adoramos realmente al mismo tiempo a Dios Hijo y a Dios Espíritu. La naturaleza de Dios Hijo es tan divinamente real y eterna como la de Dios Padre.

\par
%\textsuperscript{(75.1)}
\textsuperscript{6:2.4} El Hijo no sólo posee toda la rectitud infinita y trascendente del Padre, sino que el Hijo refleja también toda la santidad de carácter del Padre. El Hijo comparte la perfección del Padre y comparte de manera conjunta la responsabilidad de ayudar a todas las criaturas imperfectas en sus esfuerzos espirituales por alcanzar la perfección divina.

\par
%\textsuperscript{(75.2)}
\textsuperscript{6:2.5} El Hijo Eterno posee todo el carácter divino y todos los atributos espirituales del Padre. El Hijo \textit{es} la plenitud de la absolutidad de Dios en lo referente a la personalidad y al espíritu, y el Hijo revela estas cualidades en su dirección personal del gobierno espiritual del universo de universos.

\par
%\textsuperscript{(75.3)}
\textsuperscript{6:2.6} Dios es en verdad un espíritu universal; Dios es espíritu; y esta naturaleza espiritual del Padre está focalizada y personalizada en la Deidad del Hijo Eterno\footnote{\textit{Dios es espíritu}: Jn 4:24.}. En el Hijo, todas las características espirituales están en apariencia enormemente realzadas por diferenciación con la universalidad de la Fuente-Centro Primera. Y al igual que el Padre comparte su naturaleza espiritual con el Hijo, juntos comparten el espíritu divino plenamente y sin reservas con el Actor Conjunto, el Espíritu Infinito.

\par
%\textsuperscript{(75.4)}
\textsuperscript{6:2.7} En el amor a la verdad y en la creación de la belleza, el Padre y el Hijo son iguales, salvo que el Hijo \textit{parece} dedicarse más a la realización de la belleza exclusivamente espiritual de los valores universales.

\par
%\textsuperscript{(75.5)}
\textsuperscript{6:2.8} En la bondad divina, no discierno ninguna diferencia entre el Padre y el Hijo. El Padre ama a sus hijos del universo como un padre; el Hijo Eterno contempla a todas las criaturas como padre y como hermano a la vez.

\section*{3. El ministerio de amor del Padre}
\par
%\textsuperscript{(75.6)}
\textsuperscript{6:3.1} El Hijo comparte la justicia y la rectitud de la Trinidad, pero la personalización infinita del amor y de la misericordia del Padre eclipsan estas características de la divinidad; el Hijo es la revelación del amor divino a los universos. Al igual que Dios es amor\footnote{\textit{Dios es amor}: 1 Jn 4:8,16.}, el Hijo es misericordia. El Hijo no puede amar más que el Padre, pero puede mostrar misericordia a las criaturas de una manera adicional, porque no sólo es un creador primordial como el Padre, sino que es también el Hijo Eterno de ese mismo Padre, participando así en la experiencia de filiación de todos los otros hijos del Padre Universal.

\par
%\textsuperscript{(75.7)}
\textsuperscript{6:3.2} El Hijo Eterno es el gran ministro de la misericordia para toda la creación\footnote{\textit{Ministro de la misericordia}: Ex 20:6; 1 Cr 16:34; 2 Cr 5:13; 7:3,6; 30:9; Esd 3:11; Sal 25:6; 36:5; 86:5,13,15; 100:5; 103:8,11,17; 107:1; 116:5; 117:2; 118:1,4; 136:1-26; 145:8; Is 54:8; 55:7; Jer 3:12; Nm 14:18-19; Miq 7:18; Dt 4:31; 5:10; Heb 8:12.}. La misericordia es la esencia del carácter espiritual del Hijo. Cuando los mandatos del Hijo Eterno salen por los circuitos espirituales de la Fuente-Centro Segunda, están afinados a los tonos de la misericordia.

\par
%\textsuperscript{(75.8)}
\textsuperscript{6:3.3} Para comprender el amor del Hijo Eterno debéis percibir primero su fuente divina, el Padre, que \textit{es} amor\footnote{\textit{Dios es amor}: 1 Jn 4:8,16.}, y luego contemplar el despliegue de este afecto infinito en el extenso ministerio del Espíritu Infinito y de su multitud casi ilimitada de personalidades ministrantes.

\par
%\textsuperscript{(75.9)}
\textsuperscript{6:3.4} El ministerio del Hijo Eterno está consagrado a la revelación del Dios de amor al universo de universos. Este Hijo divino no se dedica a la tarea innoble de tratar de persuadir a su Padre benevolente para que ame a sus humildes criaturas y manifieste misericordia a los malhechores temporales. !`Qué error imaginar al Hijo Eterno suplicándole al Padre Universal para que muestre misericordia a sus humildes criaturas de los mundos materiales del espacio! Estos conceptos de Dios son vulgares y grotescos. Deberíais daros cuenta más bien de que todos los servicios misericordiosos de los Hijos de Dios son una revelación directa del corazón del Padre, lleno de amor universal y de compasión infinita. El amor del Padre es la fuente real y eterna de la misericordia del Hijo.

\par
%\textsuperscript{(75.10)}
\textsuperscript{6:3.5} Dios es amor\footnote{\textit{Dios es amor}: 1 Jn 4:8,16.}, el Hijo es misericordia\footnote{\textit{El Hijo es misericordia}: 1 Cr 16:34.}. La misericordia es el amor aplicado, el amor del Padre en acción en la persona de su Hijo Eterno. El amor de este Hijo universal es igualmente universal. Tal como el amor se comprende en un planeta sexuado, el amor de Dios es más comparable con el amor de un padre, mientras que el amor del Hijo Eterno se parece más al afecto de una madre. Estos ejemplos son realmente burdos, pero los empleo con la esperanza de transmitir a la mente humana la idea de que existe una diferencia, no en el contenido divino, sino en la calidad y en la técnica de expresión, entre el amor del Padre y el amor del Hijo.

\section*{4. Los atributos del Hijo Eterno}
\par
%\textsuperscript{(76.1)}
\textsuperscript{6:4.1} El Hijo Eterno motiva el nivel espiritual de la realidad cósmica; el poder espiritual del Hijo es absoluto en relación con todas las realidades del universo. Ejerce un control perfecto sobre la interasociación de toda la energía espiritual indiferenciada y sobre toda la realidad espiritual manifestada gracias a su dominio absoluto de la gravedad espiritual. Todo espíritu puro no fragmentado y todos los seres y valores espirituales son sensibles al poder de atracción infinito del Hijo original del Paraíso. Y si el eterno futuro tuviera que presenciar la aparición de un universo ilimitado, la gravedad espiritual y el poder espiritual del Hijo Original resultarían enteramente adecuados para controlar espiritualmente y administrar eficazmente esa creación sin límites.

\par
%\textsuperscript{(76.2)}
\textsuperscript{6:4.2} El Hijo sólo es omnipotente en el ámbito espiritual. En la eterna economía de la administración del universo nunca se encuentra una repetición de funciones derrochadora e innecesaria; las Deidades no son dadas a duplicar inútilmente su ministerio universal.

\par
%\textsuperscript{(76.3)}
\textsuperscript{6:4.3} La omnipresencia del Hijo Original constituye la unidad espiritual del universo de universos. La cohesión espiritual de toda la creación descansa en la presencia ubicua y activa del espíritu divino del Hijo Eterno. Cuando concebimos la presencia espiritual del Padre, nos resulta difícil diferenciarla en nuestro pensamiento de la presencia espiritual del Hijo Eterno. El espíritu del Padre reside eternamente en el espíritu del Hijo.

\par
%\textsuperscript{(76.4)}
\textsuperscript{6:4.4} El Padre debe estar espiritualmente omnipresente, pero esta omnipresencia parece ser inseparable de las actividades espirituales ubicuas del Hijo Eterno. Creemos sin embargo que en todas las situaciones en que la presencia del Padre y del Hijo tiene una naturaleza espiritual doble, el espíritu del Hijo está coordinado con el espíritu del Padre.

\par
%\textsuperscript{(76.5)}
\textsuperscript{6:4.5} En su contacto con las personalidades, el Padre actúa por medio del circuito de la personalidad. En su contacto personal y detectable con la creación espiritual, el Padre aparece en los fragmentos de la totalidad de su Deidad, y estos fragmentos del Padre tienen una función solitaria, única y exclusiva cada vez que aparecen en cualquier lugar de los universos. En todas estas situaciones, el espíritu del Hijo está coordinado con la función espiritual de la presencia fragmentada del Padre Universal.

\par
%\textsuperscript{(76.6)}
\textsuperscript{6:4.6} Espiritualmente, el Hijo Eterno es omnipresente. El espíritu del Hijo Eterno está con toda seguridad con vosotros y alrededor de vosotros, pero no dentro de vosotros ni formando parte de vosotros como el Monitor de Misterio. El fragmento interior del Padre ajusta la mente humana a las actitudes progresivamente divinas, con lo cual esta mente ascendente se vuelve cada vez más sensible al poder de atracción espiritual del todopoderoso circuito de gravedad espiritual de la Fuente-Centro Segunda.

\par
%\textsuperscript{(76.7)}
\textsuperscript{6:4.7} El Hijo Original es universal y espiritualmente consciente de sí mismo. En sabiduría, el Hijo es plenamente igual al Padre. En los dominios del conocimiento, de la omnisciencia, no podemos distinguir entre las Fuentes Primera y Segunda; al igual que el Padre, el Hijo lo sabe todo; ningún acontecimiento del universo le coge nunca por sorpresa; comprende el fin desde el principio.

\par
%\textsuperscript{(77.1)}
\textsuperscript{6:4.8} El Padre y el Hijo conocen realmente el número y el paradero de todos los espíritus y de todos los seres espiritualizados del universo de universos. El Hijo no solamente conoce todas las cosas en virtud de su propio espíritu omnipresente, sino que el Hijo, al igual que el Padre y el Actor Conjunto, conoce plenamente el extenso servicio de información por reflectividad del Ser Supremo, y este servicio de información es consciente en todo momento de todas las cosas que suceden en todos los mundos de los siete superuniversos. Y la omnisciencia del Hijo Paradisiaco está asegurada además por otros medios.

\par
%\textsuperscript{(77.2)}
\textsuperscript{6:4.9} El Hijo Eterno, como personalidad espiritual amorosa, misericordiosa y ministrante, es exacta e infinitamente igual al Padre Universal, mientras que en todos sus contactos personales misericordiosos y afectuosos con los seres ascendentes de las esferas inferiores, el Hijo Eterno es tan bondadoso y considerado, tan paciente y tolerante como sus Hijos Paradisiacos de los universos locales, esos Hijos que se donan con tanta frecuencia a los mundos evolutivos del tiempo.

\par
%\textsuperscript{(77.3)}
\textsuperscript{6:4.10} Es innecesario extenderse más sobre los atributos del Hijo Eterno. Con las excepciones indicadas, es suficiente con estudiar los atributos espirituales de Dios Padre para comprender y evaluar correctamente los atributos de Dios Hijo.

\section*{5. Las limitaciones del Hijo Eterno}
\par
%\textsuperscript{(77.4)}
\textsuperscript{6:5.1} El Hijo Eterno no actúa personalmente en los dominios físicos, ni tampoco ejerce su actividad en los niveles del ministerio mental hacia los seres creados, salvo a través del Actor Conjunto. Pero, por otra parte, estas restricciones no limitan de ninguna manera al Hijo Eterno en el pleno y libre ejercicio de todos sus atributos divinos de omnisciencia, omnipresencia y omnipotencia \textit{espirituales.}

\par
%\textsuperscript{(77.5)}
\textsuperscript{6:5.2} El Hijo Eterno no impregna personalmente los potenciales espirituales inherentes a la infinidad del Absoluto de la Deidad, pero a medida que estos potenciales se van manifestando, entran dentro de la todopoderosa atracción del circuito de gravedad espiritual del Hijo.

\par
%\textsuperscript{(77.6)}
\textsuperscript{6:5.3} La personalidad es el don exclusivo del Padre Universal. El Hijo Eterno deriva su personalidad del Padre, pero no confiere la personalidad sin el Padre. El Hijo da origen a una inmensa multitud de espíritus, pero estas derivaciones no son personalidades. Cuando el Hijo crea una personalidad, lo hace en conjunción con el Padre o con el Creador Conjunto, que puede actuar por el Padre en estas relaciones. El Hijo Eterno es así un cocreador de personalidades, pero no confiere la personalidad a ningún ser; a solas y por sí mismo nunca crea seres personales. Sin embargo, esta limitación en su actividad no priva al Hijo de la capacidad de crear cualquier tipo de realidad distinta a la personal.

\par
%\textsuperscript{(77.7)}
\textsuperscript{6:5.4} El Hijo Eterno está limitado en la transmisión de las prerrogativas de creador. El Padre, al eternizar al Hijo Original, le otorgó el poder y el privilegio de unirse posteriormente a él en el acto divino de engendrar otros Hijos que poseyeran atributos creadores, y esto lo han hecho y lo siguen haciendo. Pero una vez que estos Hijos coordinados han sido engendrados, parece ser que las prerrogativas creadoras no pueden transmitirse más allá. El Hijo Eterno sólo transmite los poderes creativos a la personalización primera o directa. Por consiguiente, cuando el Padre y el Hijo se unen para personalizar a un Hijo Creador, consiguen su propósito; pero el Hijo Creador traído así a la existencia nunca puede transmitir o delegar las prerrogativas creadoras a las diversas órdenes de Hijos que pueda crear posteriormente, a pesar de que en los Hijos superiores del universo local aparece un reflejo muy limitado de los atributos creativos de un Hijo Creador.

\par
%\textsuperscript{(78.1)}
\textsuperscript{6:5.5} El Hijo Eterno, como ser infinito y exclusivamente personal, no puede fragmentar su naturaleza, no puede distribuir ni otorgar porciones individualizadas de su yo a otras personas o entidades como lo hacen el Padre Universal y el Espíritu Infinito. Pero el Hijo puede donarse y se dona como espíritu ilimitado para bañar toda la creación y atraer incesantemente hacia él a todas las personalidades de espíritu y a todas las realidades espirituales.

\par
%\textsuperscript{(78.2)}
\textsuperscript{6:5.6} Recordad siempre que el Hijo Eterno es el retrato personal del Padre espiritual para toda la creación. El Hijo es personal y nada más que personal en el sentido de la Deidad; esta personalidad divina y absoluta no puede disgregarse ni fragmentarse. Dios Padre y Dios Espíritu son verdaderamente personales, pero son también todo lo demás además de ser estas personalidades de la Deidad.

\par
%\textsuperscript{(78.3)}
\textsuperscript{6:5.7} Aunque el Hijo Eterno no puede participar personalmente en la concesión de los Ajustadores del Pensamiento, en el eterno pasado se sentó en consejo con el Padre Universal, y aprobó el plan y prometió una cooperación sin fin cuando el Padre, al proyectar la concesión de los Ajustadores del Pensamiento, le propuso al Hijo: <<\textit{Hagamos al hombre mortal a nuestra propia imagen}>>\footnote{\textit{El hombre hecho a imagen de Dios}: Gn 1:26.}. Y al igual que el fragmento espiritual del Padre habita en vosotros, la presencia espiritual del Hijo os envuelve, y los dos trabajan constantemente como uno solo para vuestro progreso espiritual.

\section*{6. La mente del espíritu}
\par
%\textsuperscript{(78.4)}
\textsuperscript{6:6.1} El Hijo Eterno es espíritu y posee una mente, pero no una mente o un espíritu que la mente humana pueda comprender. El hombre mortal percibe la mente en los niveles finito, cósmico, material y personal. El hombre observa también los fenómenos mentales en los organismos vivientes que funcionan en el nivel subpersonal (animal), pero le resulta difícil captar la naturaleza de la mente cuando ésta se encuentra asociada a los seres supermateriales y forma parte de unas personalidades exclusivamente espirituales. Sin embargo, la mente ha de ser definida de manera diferente cuando se refiere al nivel espiritual de existencia, y cuando se emplea para indicar las funciones espirituales de la inteligencia. El tipo de mente que está unida directamente al espíritu no es comparable ni con la mente que coordina el espíritu y la materia, ni con la mente que sólo está unida a la materia.

\par
%\textsuperscript{(78.5)}
\textsuperscript{6:6.2} El espíritu es siempre consciente, está dotado de mente y posee diversas fases de identidad. Sin una mente de algún tipo, no existiría ninguna conciencia espiritual en la fraternidad de los seres espirituales. El equivalente de la mente, la capacidad para conocer y ser conocido\footnote{\textit{Conocer y ser conocido}: 1 Co 13:12.}, es natural en la Deidad. La Deidad puede ser personal, prepersonal, superpersonal o impersonal, pero la Deidad nunca está desprovista de mente, es decir, nunca carece de la capacidad de comunicarse, al menos con entidades, seres o personalidades similares.

\par
%\textsuperscript{(78.6)}
\textsuperscript{6:6.3} La mente del Hijo Eterno es semejante a la del Padre, pero diferente a cualquier otra mente en el universo, y junto con la mente del Padre, es la antepasada de las extensas mentes diversas del Actor Conjunto. La mente del Padre y del Hijo, ese intelecto que es ancestral a la mente absoluta de la Fuente-Centro Tercera, quizás se encuentra mejor ilustrada en la premente de un Ajustador del Pensamiento, porque, aunque estos fragmentos del Padre están totalmente fuera de los circuitos mentales del Actor Conjunto, poseen alguna forma de premente; conocen y son conocidos\footnote{\textit{Conocer y ser conocido}: 1 Co 13:12.}; disfrutan del equivalente del pensamiento humano.

\par
%\textsuperscript{(78.7)}
\textsuperscript{6:6.4} El Hijo Eterno es totalmente espiritual; el hombre es casi enteramente material; por eso muchas cosas relacionadas con la personalidad espiritual del Hijo Eterno, con sus siete esferas espirituales que rodean al Paraíso, y con la naturaleza de las creaciones impersonales del Hijo Paradisiaco, tendrán que esperar a que alcancéis el estado espiritual después de culminar vuestra ascensión morontial del universo local de Nebadon. Luego, cuando paséis por el superuniverso y continuéis hasta Havona, muchos de estos misterios ocultos del espíritu se clarificarán a medida que empecéis a estar dotados de la <<\textit{mente del espíritu}>>\footnote{\textit{Mente del espíritu}: Ro 8:27; 11:34; 1 Co 2:16; Ef 4:23; Flp 2:5.} ---la perspicacia espiritual.

\section*{7. La personalidad del Hijo Eterno}
\par
%\textsuperscript{(79.1)}
\textsuperscript{6:7.1} El Hijo Eterno es esa personalidad infinita que sufre las trabas de la personalidad incalificada, de las que el Padre Universal se escapó mediante la técnica de la trinitización, y en virtud de la cual ha continuado donándose desde entonces con una prodigalidad sin fin a su universo, en constante expansión, de Creadores y de criaturas. El Hijo es la \textit{personalidad absoluta;} Dios es la \textit{personalidad paternal} ---la fuente de la personalidad, el donador de la personalidad, la causa de la personalidad. Cada ser personal obtiene su personalidad del Padre Universal, tal como el Hijo Original obtiene eternamente su personalidad del Padre Paradisiaco.

\par
%\textsuperscript{(79.2)}
\textsuperscript{6:7.2} La personalidad del Hijo Paradisiaco es absoluta y puramente espiritual, y esta personalidad absoluta es también el arquetipo divino y eterno, en primer lugar, de la concesión de la personalidad por parte del Padre al Actor Conjunto y, posteriormente, de la concesión de la personalidad a las miríadas de sus criaturas en todo un extenso universo.

\par
%\textsuperscript{(79.3)}
\textsuperscript{6:7.3} El Hijo Eterno es verdaderamente un ministro misericordioso, un espíritu divino, un poder espiritual y una personalidad real. El Hijo es la naturaleza espiritual y personal de Dios manifestada a los universos ---la suma y la sustancia de la Fuente-Centro Primera, despojadas de todo lo que es no personal, extradivino, no espiritual y puro potencial. Pero es imposible transmitir a la mente humana una descripción gráfica de la belleza y la grandiosidad de la personalidad celestial del Hijo Eterno. Todo lo que tiende a oscurecer al Padre Universal ejerce una influencia casi equivalente para impedir reconocer conceptualmente al Hijo Eterno. Tendréis que esperar a alcanzar el Paraíso, y entonces comprenderéis por qué he sido incapaz de describir el carácter de esta personalidad absoluta a la comprensión de la mente finita.

\section*{8. La comprensión del Hijo Eterno}
\par
%\textsuperscript{(79.4)}
\textsuperscript{6:8.1} En lo que se refiere a la identidad, la naturaleza y otros atributos de la personalidad, el Hijo Eterno es el pleno equivalente, el complemento perfecto y la contrapartida eterna del Padre Universal. En el mismo sentido que Dios es el Padre Universal, el Hijo es la Madre Universal. Y todos nosotros, elevados y humildes, constituimos su familia universal.

\par
%\textsuperscript{(79.5)}
\textsuperscript{6:8.2} Para apreciar el carácter del Hijo, deberíais estudiar la revelación del carácter divino del Padre; los dos son eterna e inseparablemente uno solo. Como personalidades divinas son prácticamente indistinguibles por las órdenes inferiores de inteligencia. A aquellos que tienen su origen en los actos creadores de las Deidades mismas no les resulta tan difícil reconocerlos por separado. Los seres nacidos en el universo central y en el Paraíso disciernen al Padre y al Hijo no solamente como una unidad personal de control universal, sino también como dos personalidades distintas que ejercen su actividad en ámbitos concretos de la administración del universo.

\par
%\textsuperscript{(79.6)}
\textsuperscript{6:8.3} Como personas, podéis concebir al Padre Universal y al Hijo Eterno como individuos distintos, pues en verdad lo son; pero en la administración de los universos, están tan entrelazados e interrelacionados que no siempre es posible distinguir entre ellos\footnote{\textit{El Padre y el Hijo son uno}: Jn 1:1; 5:17-18; 10:30,38; 12:45; 14:7-11,20; 17:11,21-22.}. Cuando encontramos, en los asuntos de los universos, al Padre y al Hijo en interasociaciones desconcertantes, no siempre es útil intentar separar sus actividades; recordad simplemente que Dios es el pensamiento iniciador y que el Hijo es el verbo expresivo\footnote{\textit{El hijo es la "palabra"}: Jn 1:1.}. En cada universo local, esta inseparabilidad está personalizada en la divinidad del Hijo Creador, que representa tanto al Padre como al Hijo para las criaturas de diez millones de mundos habitados.

\par
%\textsuperscript{(80.1)}
\textsuperscript{6:8.4} El Hijo Eterno es infinito, pero es accesible a través de las personas de sus Hijos Paradisiacos y por medio del paciente ministerio del Espíritu Infinito. Sin el servicio donador de los Hijos Paradisiacos y sin el ministerio amoroso de las criaturas del Espíritu Infinito, los seres de origen material difícilmente podrían esperar alcanzar al Hijo Eterno. Y es igualmente cierto que, con la ayuda y la guía de estos agentes celestiales, los mortales conscientes de Dios alcanzarán indudablemente el Paraíso y algún día se encontrarán en la presencia personal de este majestuoso Hijo de Hijos\footnote{\textit{En presencia de Dios}: Ap 20:12.}.

\par
%\textsuperscript{(80.2)}
\textsuperscript{6:8.5} Aunque el Hijo Eterno es el arquetipo que deberán alcanzar las personalidades mortales, encontraréis más fácil captar la realidad del Padre y del Espíritu, porque el Padre es el verdadero donador de vuestra personalidad humana, y el Espíritu Infinito es la fuente absoluta de vuestra mente mortal. Pero a medida que os elevéis en el sendero paradisiaco del progreso espiritual, la personalidad del Hijo Eterno se volverá cada vez más real para vosotros, y la realidad de su mente infinitamente espiritual se hará más discernible para vuestra mente en vías de espiritualización progresiva.

\par
%\textsuperscript{(80.3)}
\textsuperscript{6:8.6} El concepto del Hijo Eterno nunca podrá brillar intensamente en vuestra mente material ni en vuestra mente morontial posterior; hasta que no seáis un espíritu y comencéis vuestra ascensión espiritual, la comprensión de la personalidad del Hijo Eterno no empezará a igualar la intensidad de vuestro concepto sobre la personalidad del Hijo Creador originario del Paraíso, el cual, en persona y como persona, se encarnó y vivió en otro tiempo en Urantia como un hombre entre los hombres.

\par
%\textsuperscript{(80.4)}
\textsuperscript{6:8.7} Durante toda vuestra experiencia en el universo local, el Hijo Creador, cuya personalidad es comprensible para el hombre, deberá compensar vuestra incapacidad para captar todo el significado del Hijo Eterno del Paraíso, que es más exclusivamente espiritual, pero sin embargo personal. Cuando progreséis a través de Orvonton y de Havona, a medida que dejéis atrás la imagen intensa y los profundos recuerdos del Hijo Creador de vuestro universo local, la desaparición de esta experiencia material y morontial será compensada con unos conceptos siempre más amplios y una comprensión más intensa del Hijo Eterno del Paraíso, cuya realidad y cercanía aumentarán constantemente a medida que progreséis hacia el Paraíso.

\par
%\textsuperscript{(80.5)}
\textsuperscript{6:8.8} El Hijo Eterno es una personalidad grandiosa y gloriosa. Aunque captar la realidad de la personalidad de este ser infinito sobrepasa la capacidad de la mente mortal y material, no lo dudéis, es una persona. Sé de lo que hablo. He permanecido en la presencia divina de este Hijo Eterno en ocasiones casi innumerables, y luego he viajado hasta el universo para llevar a cabo sus bondadosos mandatos.

\par
%\textsuperscript{(80.6)}
\textsuperscript{6:8.9} [Redactado por un Consejero Divino designado para formular esta exposición que describe al Hijo Eterno del Paraíso.]


\chapter{Documento 7. Las relaciones del Hijo Eterno con el universo}
\par
%\textsuperscript{(81.1)}
\textsuperscript{7:0.1} EL HIJO Original se ocupa constantemente de ejecutar los aspectos espirituales del propósito eterno del Padre, a medida que éste se desarrolla progresivamente en los fenómenos de los universos evolutivos con sus múltiples grupos de seres vivientes. Nosotros no comprendemos plenamente este plan eterno, pero el Hijo Paradisiaco lo comprende sin duda alguna.

\par
%\textsuperscript{(81.2)}
\textsuperscript{7:0.2} El Hijo es semejante al Padre en el sentido de que trata de dar todo lo que le es posible de sí mismo a sus Hijos coordinados y a los Hijos subordinados a ellos. Y el Hijo comparte la naturaleza autodistributiva del Padre en la donación ilimitada de sí mismo al Espíritu Infinito, su ejecutivo conjunto.

\par
%\textsuperscript{(81.3)}
\textsuperscript{7:0.3} Como sostén de las realidades espirituales, la Fuente-Centro Segunda es el eterno contrapeso de la Isla del Paraíso, que sostiene tan magníficamente todas las cosas materiales. La Fuente-Centro Primera se revela así eternamente en la belleza material de los arquetipos exquisitos de la Isla central, y en los valores espirituales de la personalidad celestial del Hijo Eterno.

\par
%\textsuperscript{(81.4)}
\textsuperscript{7:0.4} El Hijo Eterno es el sostén efectivo de la inmensa creación de realidades de espíritu y de seres espirituales. El mundo del espíritu es el hábito, la conducta personal del Hijo, y las realidades impersonales de naturaleza espiritual son siempre sensibles a la voluntad y al propósito de la personalidad perfecta del Hijo Absoluto.

\par
%\textsuperscript{(81.5)}
\textsuperscript{7:0.5} Sin embargo, el Hijo no es personalmente responsable de la conducta de todas las personalidades espirituales. La voluntad de las criaturas personales es relativamente libre y, por lo tanto, determina las acciones de esos seres volitivos. Por consiguiente, el mundo espiritual del libre albedrío no siempre representa verdaderamente el carácter del Hijo Eterno, al igual que la naturaleza en Urantia no revela verdaderamente la perfección y la inmutabilidad del Paraíso y de la Deidad. Pero cualesquiera que sean las características de los actos libres de un hombre o de un ángel, el dominio eterno del Hijo sobre el control gravitatorio universal de todas las realidades espirituales continúa siendo absoluto.

\section*{1. El circuito de la gravedad espiritual}
\par
%\textsuperscript{(81.6)}
\textsuperscript{7:1.1} Todo lo que ha sido enseñado acerca de la inmanencia de Dios, su omnipresencia, omnipotencia y omnisciencia, es igualmente cierto del Hijo en el ámbito espiritual. La gravedad espiritual pura y universal de toda la creación, ese circuito exclusivamente espiritual, conduce directamente de vuelta a la persona de la Fuente-Centro Segunda en el Paraíso. Él preside el control y el funcionamiento de esa atracción espiritual siempre presente e infalible sobre todos los verdaderos valores espirituales. El Hijo Eterno ejerce así una soberanía espiritual absoluta. Mantiene literalmente, por así decirlo, en el hueco de su mano\footnote{\textit{Mantiene en el hueco de su mano}: Is 40:12.}, todas las realidades espirituales y todos los valores espiritualizados. El control de la gravedad espiritual\footnote{\textit{Gravedad espiritual}: Jer 31:3; Jn 6:44; 12:32.} universal \textit{es} la soberanía espiritual universal.

\par
%\textsuperscript{(82.1)}
\textsuperscript{7:1.2} Este control gravitatorio de las cosas espirituales funciona independientemente del tiempo y del espacio; por eso la energía espiritual no disminuye cuando es transmitida. La gravedad espiritual nunca sufre los retrasos del tiempo ni tampoco experimenta las disminuciones causadas por el espacio. No decrece con arreglo al cuadrado de la distancia en que es transmitida; la masa de la creación material no retrasa los circuitos del poder espiritual puro. Esta trascendencia del tiempo y del espacio por parte de las energías espirituales puras es inherente a la absolutidad del Hijo; no se debe a la interposición de las fuerzas antigravitatorias de la Fuente-Centro Tercera.

\par
%\textsuperscript{(82.2)}
\textsuperscript{7:1.3} Las realidades del espíritu reaccionan al poder de atracción del centro de la gravedad espiritual con arreglo a su valor cualitativo, a su grado efectivo de naturaleza espiritual. La sustancia espiritual (calidad) es tan sensible a la gravedad espiritual como la energía organizada de la materia física (cantidad) lo es a la gravedad física. Los valores espirituales y las fuerzas espirituales son \textit{reales.} Desde el punto de vista de la personalidad, el espíritu es el alma de la creación; la materia es el oscuro cuerpo físico.

\par
%\textsuperscript{(82.3)}
\textsuperscript{7:1.4} Las reacciones y fluctuaciones de la gravedad espiritual siempre son fieles al contenido en valores espirituales, al estado espiritual cualitativo de un individuo o de un mundo. Este poder de atracción responde instantáneamente a los valores inter e intraespirituales de cualquier situación universal o condición planetaria. Cada vez que una realidad espiritual se manifiesta en los universos, ese cambio necesita el reajuste inmediato e instantáneo de la gravedad espiritual. Ese nuevo espíritu forma parte realmente de la Fuente-Centro Segunda; y con la misma certeza que el hombre mortal se vuelve un ser espiritualizado, alcanzará al Hijo espiritual, el centro y la fuente de la gravedad espiritual.

\par
%\textsuperscript{(82.4)}
\textsuperscript{7:1.5} El poder de atracción espiritual del Hijo es inherente, en menor grado, a muchas órdenes paradisiacas de filiación. Pues existen de hecho, dentro del circuito absoluto de la gravedad espiritual, aquellos sistemas locales de atracción espiritual que funcionan en las unidades más pequeñas de la creación. Estas focalizaciones subabsolutas de la gravedad espiritual forman parte de la divinidad de las personalidades Creadoras del tiempo y del espacio, y están correlacionadas con el supercontrol experiencial emergente del Ser Supremo.

\par
%\textsuperscript{(82.5)}
\textsuperscript{7:1.6} La atracción de la gravedad espiritual, y la respuesta a la misma, funcionan como un todo no solamente en el universo, sino también entre los individuos y los grupos de individuos. Existe una cohesión espiritual entre las personalidades espirituales y espiritualizadas de cualquier mundo, raza, nación o grupo de creyentes. Existe una atracción directa de naturaleza espiritual entre las personas con mentalidad espiritual que tienen gustos y anhelos semejantes. El término \textit{almas gemelas} no es enteramente una figura retórica.

\par
%\textsuperscript{(82.6)}
\textsuperscript{7:1.7} Al igual que la gravedad material del Paraíso, la gravedad espiritual del Hijo Eterno es también absoluta. El pecado y la rebelión pueden dificultar el funcionamiento de los circuitos de un universo local, pero nada puede interrumpir la gravedad espiritual del Hijo Eterno. La rebelión de Lucifer ocasionó muchos cambios en vuestro sistema de mundos habitados y en Urantia, pero no observamos que la cuarentena espiritual resultante de vuestro planeta haya afectado en lo más mínimo a la presencia y al funcionamiento del espíritu omnipresente del Hijo Eterno ni del circuito de la gravedad espiritual asociado.

\par
%\textsuperscript{(82.7)}
\textsuperscript{7:1.8} Todas las reacciones del circuito de la gravedad espiritual del gran universo son previsibles. Reconocemos todas las acciones y reacciones del espíritu omnipresente del Hijo Eterno, y comprobamos que son fiables. Siguiendo unas leyes bien conocidas, podemos medir la gravedad espiritual, y lo hacemos, exactamente igual que los hombres intentan calcular los efectos de la gravedad física finita. El espíritu del Hijo responde de manera invariable a todas las cosas, seres y personas espirituales, y esta respuesta siempre está de acuerdo con el grado de manifestación (con el grado cualitativo de realidad) de todos esos valores espirituales.

\par
%\textsuperscript{(83.1)}
\textsuperscript{7:1.9} Pero al lado de este funcionamiento tan fiable y previsible de la presencia espiritual del Hijo Eterno, se encuentran fenómenos cuyas reacciones no son tan previsibles. Estos fenómenos indican probablemente la acción coordinada del Absoluto de la Deidad en los dominios de los potenciales espirituales emergentes. Sabemos que la presencia espiritual del Hijo Eterno es la influencia de una personalidad majestuosa e infinita, pero difícilmente consideramos como personales las reacciones asociadas a las supuestas actividades del Absoluto de la Deidad.

\par
%\textsuperscript{(83.2)}
\textsuperscript{7:1.10} Considerados desde el punto de vista de la personalidad, y por las personas, el Hijo Eterno y el Absoluto de la Deidad parecen estar relacionados de la manera siguiente: el Hijo Eterno domina el ámbito de los valores espirituales manifestados, mientras que el Absoluto de la Deidad parece impregnar el inmenso dominio de los valores espirituales potenciales. Todo valor manifestado de naturaleza espiritual encuentra su sitio en la atracción gravitatoria del Hijo Eterno, pero si es potencial, entonces encuentra aparentemente su lugar en la presencia del Absoluto de la Deidad.

\par
%\textsuperscript{(83.3)}
\textsuperscript{7:1.11} El espíritu parece surgir de los potenciales del Absoluto de la Deidad; el espíritu evolutivo encuentra su correlación en la atracción experiencial e incompleta del Supremo y del Último; el espíritu encuentra en definitiva su destino final en la atracción absoluta de la gravedad espiritual del Hijo Eterno. Éste parece ser el ciclo del espíritu experiencial, pero el espíritu existencial es inherente a la infinidad de la Fuente-Centro Segunda.

\section*{2. La administración del Hijo Eterno}
\par
%\textsuperscript{(83.4)}
\textsuperscript{7:2.1} En el Paraíso, la presencia y la actividad personal del Hijo Original es profunda, es absoluta en el sentido espiritual. Cuando salimos del Paraíso a través de Havona y entramos en los dominios de los siete superuniversos, detectamos cada vez menos la actividad personal del Hijo Eterno. En los universos posteriores a Havona, la presencia del Hijo Eterno está personalizada en los Hijos Paradisiacos, condicionada por las realidades experienciales del Supremo y del Último, y coordinada con el potencial espiritual ilimitado del Absoluto de la Deidad.

\par
%\textsuperscript{(83.5)}
\textsuperscript{7:2.2} En el universo central, la actividad personal del Hijo Original se puede discernir en la exquisita armonía espiritual de la creación eterna. Havona es tan maravillosamente perfecto que el estado espiritual y las condiciones energéticas de este universo modelo se encuentran en un equilibrio perfecto y perpetuo.

\par
%\textsuperscript{(83.6)}
\textsuperscript{7:2.3} En los superuniversos, el Hijo no está personalmente presente ni reside en ellos; en estas creaciones sólo mantiene una representación superpersonal. Estas manifestaciones espirituales del Hijo no son personales; no están incluidas en el circuito de la personalidad del Padre Universal. No conocemos ningún término mejor para designarlas que el nombre de \textit{superpersonalidades;} y son seres finitos; no son ni absonitos ni absolutos.

\par
%\textsuperscript{(83.7)}
\textsuperscript{7:2.4} Como la administración del Hijo Eterno en los superuniversos es exclusivamente espiritual y superpersonal, no es discernible por las persona-lidades de las criaturas. No obstante, el estímulo espiritual omnipresente de la influencia personal del Hijo se encuentra en todas las fases de las actividades de todos los sectores de los dominios de los Ancianos de los Días. Sin embargo, observamos que en los universos locales el Hijo Eterno está personalmente presente en las personas de los Hijos Paradisiacos. Aquí, el Hijo infinito ejerce su actividad de manera espiritual y creadora por medio de las personas del cuerpo majestuoso de los Hijos Creadores coordinados.

\section*{3. Las relaciones del Hijo Eterno con los individuos}
\par
%\textsuperscript{(84.1)}
\textsuperscript{7:3.1} Durante la ascensión del universo local, los mortales del tiempo consideran al Hijo Creador como el representante personal del Hijo Eterno. Pero cuando empiezan a elevarse en el régimen educativo del superuniverso, los peregrinos del tiempo detectan cada vez más la presencia celestial del espíritu inspirador del Hijo Eterno, y son capaces de beneficiarse de ella mediante el consumo de este ministerio de vigorización espiritual. En Havona, los ascendentes se vuelven aún más conscientes del abrazo amoroso del espíritu omnipresente del Hijo Original. El espíritu del Hijo Eterno no reside en la mente o en el alma de los peregrinos del tiempo en ninguna etapa de toda su ascensión como mortales, pero su acción benéfica siempre está cercana y se ocupa siempre del bienestar y de la seguridad espiritual de los hijos del tiempo que progresan.

\par
%\textsuperscript{(84.2)}
\textsuperscript{7:3.2} La atracción de la gravedad espiritual\footnote{\textit{Gravedad espiritual}: Jer 31:3; Jn 6:44; 12:32.} del Hijo Eterno constituye el secreto inherente a la ascensión al Paraíso de las almas humanas sobrevivientes. Todos los valores espirituales auténticos y todos los individuos sinceros espiritualizados son mantenidos en la atracción infalible de la gravedad espiritual del Hijo Eterno. Por ejemplo, la mente mortal inicia su carrera como un mecanismo material, y finalmente es enrolada en el Cuerpo de la Finalidad como una existencia espiritual casi perfeccionada, volviéndose progresivamente menos sujeta a la gravedad material y, en consecuencia, más sensible durante toda esta experiencia al impulso de atracción hacia el interior de la gravedad espiritual. El circuito de la gravedad espiritual tira literalmente del alma del hombre hacia el Paraíso.

\par
%\textsuperscript{(84.3)}
\textsuperscript{7:3.3} El circuito de la gravedad espiritual es el canal fundamental para transmitir las oraciones sinceras del corazón humano creyente, desde el nivel de la conciencia humana hasta la conciencia efectiva de la Deidad. Aquella parte de vuestras peticiones que representa un verdadero valor espiritual será captada por el circuito universal de la gravedad espiritual, y pasará inmediata y simultáneamente a todas las personalidades divinas interesadas. Cada una de ellas se ocupará de lo que pertenece a su incumbencia personal. Por eso en vuestra experiencia religiosa práctica, cuando dirigís vuestras súplicas es indiferente que visualicéis al Hijo Creador de vuestro universo local o al Hijo Eterno en el centro de todas las cosas.

\par
%\textsuperscript{(84.4)}
\textsuperscript{7:3.4} El funcionamiento discriminatorio del circuito de la gravedad espiritual podría compararse quizás con las funciones de los circuitos neuronales del cuerpo humano material: las sensaciones viajan hacia el interior por los nervios; algunas son detenidas por los centros automáticos inferiores espinales, los cuales reaccionan; otras continúan hasta los centros del cerebro inferior, menos automáticos pero entrenados por la costumbre, mientras que los mensajes entrantes más importantes y vitales atraviesan velozmente estos centros subordinados y se registran inmediatamente en los niveles superiores de la conciencia humana.

\par
%\textsuperscript{(84.5)}
\textsuperscript{7:3.5} Pero !`cuánto más perfecta es la técnica magnífica del mundo espiritual! Si algo que se origine en vuestra conciencia contiene un valor espiritual supremo, una vez que lo hayáis expresado, ningún poder en el universo podrá impedir que sea transmitido directamente como un relámpago a la Personalidad Espiritual Absoluta de toda la creación.

\par
%\textsuperscript{(84.6)}
\textsuperscript{7:3.6} Por el contrario, si vuestras súplicas son puramente materiales y totalmente egocéntricas, no existe ningún plan que permita que esas oraciones indignas puedan encontrar un lugar en el circuito espiritual del Hijo Eterno. El contenido de toda petición que no esté <<\textit{dictada por el espíritu}>> no puede encontrar ningún lugar en el circuito espiritual universal; esos ruegos puramente egoístas y materiales caen muertos; no ascienden por los circuitos de los verdaderos valores espirituales\footnote{\textit{Verdaderos valores espirituales}: 1 Co 13:1.}. Esas palabras son como <<\textit{cobres que resuenan y platillos que tintinean}>>.

\par
%\textsuperscript{(85.1)}
\textsuperscript{7:3.7} El pensamiento motivador, el contenido espiritual, es lo que valida la súplica humana. Las palabras carecen de valor.

\section*{4. Los planes de perfección divina}
\par
%\textsuperscript{(85.2)}
\textsuperscript{7:4.1} El Hijo Eterno está unido perpetuamente al Padre para llevar a cabo con éxito el \textit{plan divino de progreso:} el plan universal para la creación, la evolución, la ascensión y la perfección de las criaturas volitivas. Y en fidelidad divina, el Hijo es eternamente igual al Padre.

\par
%\textsuperscript{(85.3)}
\textsuperscript{7:4.2} El Padre y su Hijo actúan como uno solo para formular y llevar a cabo este gigantesco plan de consecución destinado a hacer avanzar a los seres materiales del tiempo hasta la perfección de la eternidad. Este proyecto para elevar espiritualmente a las almas ascendentes del espacio es una creación conjunta del Padre y del Hijo, y, con la cooperación del Espíritu Infinito, se ocupan de ejecutar en asociación su propósito divino.

\par
%\textsuperscript{(85.4)}
\textsuperscript{7:4.3} Este plan divino para alcanzar la perfección abarca tres empresas únicas, aunque maravillosamente correlacionadas, de aventuras universales:

\par
%\textsuperscript{(85.5)}
\textsuperscript{7:4.4} 1. \textit{El plan de consecución progresiva.} Es el plan del Padre Universal para la ascensión por evolución, un programa aceptado sin reservas por el Hijo Eterno cuando estuvo de acuerdo con la propuesta del Padre: <<\textit{Hagamos a las criaturas mortales a nuestra propia imagen}>>\footnote{\textit{El hombre hecho a imagen de Dios}: Gn 1:26.}. Esta disposición para elevar a las criaturas del tiempo implica que el Padre concede los Ajustadores del Pensamiento y dota a las criaturas materiales de las prerrogativas de la personalidad.

\par
%\textsuperscript{(85.6)}
\textsuperscript{7:4.5} 2. \textit{El plan de donación.} El plan universal siguiente es la gran empresa del Hijo Eterno y de sus Hijos coordinados destinada a revelar al Padre. Es la propuesta del Hijo Eterno, y consiste en su donación de los Hijos de Dios a las creaciones evolutivas para personalizar y convertir allí en un hecho, para encarnar y hacer real, el amor del Padre y la misericordia del Hijo a las criaturas de todos los universos. Inherente al plan de donación, y como característica provisional de este ministerio de amor, los Hijos Paradisiacos actúan como rehabilitadores de aquello que la voluntad desviada de las criaturas ha puesto en peligro espiritual. En cualquier momento y lugar en que se produce un retraso en el funcionamiento del plan de consecución, si por azar una rebelión estropea o complica esta empresa, entonces las disposiciones de emergencia del plan de donación entran inmediatamente en acción. Los Hijos Paradisiacos permanecen comprometidos y dispuestos a actuar como recuperadores, a entrar en el terreno mismo de la rebelión y restablecer allí el estado espiritual de las esferas. Un Hijo Creador coordinado efectuó este tipo de servicio heroico en Urantia en conexión con su carrera experiencial de donación para adquirir la soberanía.

\par
%\textsuperscript{(85.7)}
\textsuperscript{7:4.6} 3. \textit{El plan del ministerio de misericordia.} Cuando el plan de consecución y el plan de donación fueron formulados y proclamados, el Espíritu Infinito, solo y de sí mismo, proyectó y puso en marcha la enorme empresa universal del ministerio de misericordia. Este servicio es esencial para el funcionamiento práctico y eficaz tanto de la empresa de consecución como de la empresa de donación, y todas las personalidades espirituales de la Fuente-Centro Tercera comparten el espíritu del ministerio de misericordia que tanto forma parte de la naturaleza de la Tercera Persona de la Deidad. El Espíritu Infinito actúa verdadera y literalmente como ejecutivo conjunto del Padre y del Hijo no sólo en la creación, sino también en la administración.

\par
%\textsuperscript{(86.1)}
\textsuperscript{7:4.7} El Hijo Eterno es el depositario personal, el custodio divino, del plan universal del Padre para la ascensión de las criaturas. Después de haber promulgado el mandato universal <<\textit{Sed perfectos como yo soy perfecto}>>\footnote{\textit{Sed perfectos}: Gn 17:1; Lv 19:2; 1 Re 8:61; Dt 18:13; Mt 5:48; 2 Co 13:11; Stg 1:4; 1 P 1:16.}, el Padre confió la ejecución de esta empresa extraordinaria al Hijo Eterno; y el Hijo Eterno comparte la promoción de esta empresa celestial con su coordinado divino, el Espíritu Infinito. Las Deidades cooperan así eficazmente en el trabajo de creación, control, evolución, revelación y ministerio ---y, si es necesario, en el de restablecimiento y rehabilitación.

\section*{5. El espíritu de donación}
\par
%\textsuperscript{(86.2)}
\textsuperscript{7:5.1} El Hijo Eterno se unió sin reservas al Padre Universal para transmitir este mandato extraordinario a toda la creación: <<\textit{Sed perfectos como vuestro Padre en Havona es perfecto}>>\footnote{\textit{Sed perfectos}: Gn 17:1; 1 Re 8:61; Lv 19:2; Dt 18:13; Mt 5:48; 2 Co 13:11; Stg 1:4; 1 P 1:16.}. Y desde entonces, este mandato-invitación ha motivado todos los planes de supervivencia y todos los proyectos de donación del Hijo Eterno y de su inmensa familia de Hijos coordinados y asociados. Por medio de estas mismas donaciones, los Hijos de Dios se han convertido en <<\textit{el camino, la verdad y la vida}>>\footnote{\textit{El camino, la verdad y a vida}: Jn 14:6.} para todas las criaturas evolutivas.

\par
%\textsuperscript{(86.3)}
\textsuperscript{7:5.2} El Hijo Eterno no puede ponerse en contacto directo con los seres humanos como lo hace el Padre a través del don de los Ajustadores del Pensamiento prepersonales, pero el Hijo Eterno se acerca a las personalidades creadas mediante una serie de gradaciones descendentes de filiación divina hasta que le resulta posible permanecer en presencia del hombre y, a veces, como un hombre mismo.

\par
%\textsuperscript{(86.4)}
\textsuperscript{7:5.3} La naturaleza puramente personal del Hijo Eterno no puede fragmentarse. El Hijo Eterno ejerce su ministerio como una influencia espiritual o como una persona, pero nunca de otra manera. Al Hijo le resulta imposible convertirse en una parte de la experiencia de la criatura a la manera en que el Ajustador del Padre participa en ella, pero el Hijo Eterno compensa esta limitación mediante la técnica de la donación. Para el Hijo Eterno, las experiencias de encarnación de los Hijos Paradisiacos significan lo mismo que la experiencia de las entidades fragmentadas para el Padre Universal.

\par
%\textsuperscript{(86.5)}
\textsuperscript{7:5.4} El Hijo Eterno no llega hasta el hombre mortal bajo la forma de la voluntad divina, del Ajustador del Pensamiento que reside en la mente humana, pero el Hijo Eterno sí llegó hasta el hombre mortal de Urantia cuando la \textit{personalidad} divina de su hijo, Miguel de Nebadon, se encarnó en la naturaleza humana de Jesús de Nazaret. Para compartir la experiencia de las personalidades creadas, los Hijos Paradisiacos de Dios deben adoptar la misma naturaleza que dichas criaturas y encarnar su personalidad divina bajo la forma real de las criaturas mismas. La encarnación, el secreto de Sonarington, es la técnica que utiliza el Hijo para escapar del absolutismo de la personalidad que, de otra manera, lo encadenaría por completo.

\par
%\textsuperscript{(86.6)}
\textsuperscript{7:5.5} Hace muchísimo tiempo, el Hijo Eterno se donó en cada uno de los circuitos de la creación central para iluminar y hacer progresar a todos los habitantes y peregrinos de Havona, incluyendo a los peregrinos ascendentes del tiempo. En ninguna de estas siete donaciones actuó como un ascendente o como un habitante de Havona, sino que vivió como él mismo. Su experiencia fue única; no la hizo \textit{con} un humano ni \textit{como} un humano u otro peregrino, sino que fue de algún modo asociativa en el sentido superpersonal.

\par
%\textsuperscript{(86.7)}
\textsuperscript{7:5.6} Tampoco pasó por el reposo que media entre el circuito interior de Havona y las orillas del Paraíso. A un ser absoluto como él no le es posible interrumpir la conciencia de la personalidad, porque en él están centradas todas las líneas de la gravedad espiritual. Durante los períodos de estas donaciones, el emplazamiento paradisiaco central de la luminosidad espiritual no se oscureció, y tampoco disminuyó el control del Hijo sobre la gravedad espiritual universal.

\par
%\textsuperscript{(87.1)}
\textsuperscript{7:5.7} Las donaciones del Hijo Eterno en Havona se encuentran fuera del alcance de la imaginación humana; fueron trascendentales. En aquel momento y posteriormente aumentó la experiencia de todo Havona, pero no sabemos si añadió algo a la supuesta capacidad experiencial de su naturaleza existencial. Esto caería dentro del misterio de las donaciones de los Hijos Paradisiacos. Creemos sin embargo que todo lo que el Hijo Eterno adquirió en estas misiones de donación lo ha conservado desde entonces, pero no sabemos de qué se trata.

\par
%\textsuperscript{(87.2)}
\textsuperscript{7:5.8} Cualquiera que sea nuestra dificultad para comprender las donaciones de la Segunda Persona de la Deidad, comprendemos muy bien la donación en Havona de un Hijo del Hijo Eterno, que pasó literalmente por los circuitos del universo central y compartió realmente las experiencias que constituyen la preparación de un ascendente para alcanzar la Deidad. Se trata del Miguel original, del Hijo Creador primogénito, que pasó por las experiencias de vida de los peregrinos ascendentes, de circuito en circuito, atravesando personalmente con ellos una etapa de cada círculo en los tiempos de Grandfanda, el primer mortal que llegó a Havona.

\par
%\textsuperscript{(87.3)}
\textsuperscript{7:5.9} Aparte de cualquier otra cosa que revelara este Miguel original, hizo real la donación trascendente del Hijo-Madre Original para las criaturas de Havona. La hizo tan real que cada peregrino del tiempo que se esfuerza en la aventura de atravesar los circuitos de Havona se siente alentado y fortalecido para siempre jamás por el conocimiento seguro de que el Hijo Eterno de Dios renunció siete veces al poder y a la gloria del Paraíso para participar en las experiencias de los peregrinos del espacio-tiempo en los siete circuitos de consecución progresiva de Havona.

\par
%\textsuperscript{(87.4)}
\textsuperscript{7:5.10} El Hijo Eterno es la inspiración ejemplar para todos los Hijos de Dios en sus ministerios de donación en todos los universos del tiempo y del espacio. Los Hijos Creadores coordinados y los Hijos Magistrales asociados, junto con otras órdenes no reveladas de filiación, comparten todos esta maravillosa buena disposición para donarse a las diversas órdenes de vida de las criaturas y bajo la forma de las criaturas mismas. Por esta razón, en espíritu, y a causa de su parentesco de naturaleza así como al hecho de su origen, se vuelve cierto que, por medio de las donaciones de cada Hijo de Dios en los mundos del espacio, en ellas, a través de ellas y gracias a ellas, el Hijo Eterno se ha donado él mismo a las criaturas volitivas inteligentes de los universos.

\par
%\textsuperscript{(87.5)}
\textsuperscript{7:5.11} En espíritu y en naturaleza, si no en todos sus atributos, cada Hijo Paradisiaco es un retrato divinamente perfecto del Hijo Original. Es literalmente cierto que cualquiera que ha visto a un Hijo Paradisiaco ha visto al Hijo Eterno de Dios.

\section*{6. Los Hijos Paradisiacos de Dios}
\par
%\textsuperscript{(87.6)}
\textsuperscript{7:6.1} La carencia de conocimientos acerca de los múltiples Hijos de Dios es una fuente de gran confusión en Urantia. Esta ignorancia persiste a pesar de las declaraciones tales como el relato de un cónclave de estas personalidades divinas: <<\textit{Cuando los Hijos de Dios proclamaban la alegría y todas las Estrellas Matutinas cantaban juntas}>>\footnote{\textit{Los Hijos proclaman la alegría}: Job 38:7.}. Cada milenio del tiempo oficial de un sector, las diversas órdenes de Hijos divinos se reúnen para celebrar sus cónclaves periódicos.

\par
%\textsuperscript{(87.7)}
\textsuperscript{7:6.2} El Hijo Eterno es la fuente personal de los adorables atributos de misericordia y de servicio que caracterizan tan abundantemente a todas las órdenes de Hijos descendentes de Dios cuando ejercen su actividad en toda la creación. El Hijo Eterno transmite infaliblemente toda su naturaleza divina, si no toda la infinidad de sus atributos, a los Hijos Paradisiacos que salen de la Isla eterna para revelar su carácter divino al universo de universos.

\par
%\textsuperscript{(88.1)}
\textsuperscript{7:6.3} El Hijo Eterno y Original es la persona-descendiente del <<\textit{primer}>> pensamiento completo e infinito del Padre Universal. Cada vez que el Padre Universal y el Hijo Eterno proyectan conjuntamente un pensamiento personal nuevo, original, idéntico, único y absoluto, en ese mismo instante esta idea creativa se personaliza de manera perfecta y final en el ser y la personalidad de un \textit{Hijo Creador} nuevo y original. En naturaleza espiritual, sabiduría divina y poder creador coordinado, estos Hijos Creadores son potencialmente iguales a Dios Padre y a Dios Hijo.

\par
%\textsuperscript{(88.2)}
\textsuperscript{7:6.4} Los Hijos Creadores salen del Paraíso hacia los universos del tiempo y, con la cooperación de los agentes controladores y creadores de la Fuente-Centro Tercera, finalizan la organización de los universos locales de evolución progresiva. Estos Hijos no están conectados ni relacionados con los controles centrales y universales de la materia, la mente y el espíritu. De ahí que estén limitados en sus actos creadores por la preexistencia, la prioridad y la primacía de la Fuente-Centro Primera y de sus Absolutos coordinados. Estos Hijos sólo pueden administrar aquello que traen a la existencia. La administración absoluta es inherente a la prioridad de existencia e inseparable de la eternidad de presencia. El Padre permanece primordial en los universos.

\par
%\textsuperscript{(88.3)}
\textsuperscript{7:6.5} Los Hijos Creadores son personalizados por el Padre y el Hijo, y los \textit{HijosMagistrales} son personalizados de manera muy similar por el Hijo y el Espíritu.
Éstos son los Hijos que, en sus experiencias de encarnación como criaturas, se ganan el derecho de servir como jueces de la supervivencia en las creaciones del tiempo y del espacio.

\par
%\textsuperscript{(88.4)}
\textsuperscript{7:6.6} El Padre, el Hijo y el Espíritu se unen también para personalizar a los \textit{HijosInstructores Trinitarios,} que están dotados de múltiples talentos y recorren el gran universo como instructores celestiales de todas las personalidades, humanas y divinas. Y existen otras muchas órdenes de filiación paradisiaca de las que no se ha informado a los mortales de Urantia.

\par
%\textsuperscript{(88.5)}
\textsuperscript{7:6.7} Existe un canal de comunicación directo y exclusivo entre el Hijo Madre Original y estas multitudes de Hijos Paradisiacos dispersos por toda la creación, un canal cuya función es inherente a la calidad del parentesco espiritual que los une mediante lazos de asociación espiritual casi absoluta. Este circuito interfilial es totalmente diferente al circuito universal de la gravedad espiritual, que también está centrado en la persona de la Fuente-Centro Segunda. Todos los Hijos de Dios que tienen su origen en las personas de las Deidades del Paraíso están en comunicación directa y constante con el Hijo Madre Eterno. Y esta comunicación es instantánea; es independiente del tiempo, aunque a veces está condicionada por el espacio.

\par
%\textsuperscript{(88.6)}
\textsuperscript{7:6.8} El Hijo Eterno no solamente tiene en todo momento un conocimiento perfecto del estado, los pensamientos y las múltiples actividades de todas las órdenes de filiación paradisiaca, sino que tiene también un conocimiento perfecto, en todo momento, de todo aquello que posee un valor espiritual en el corazón de todas las criaturas de la creación primaria central de la eternidad, y de las creaciones temporales secundarias de los Hijos Creadores coordinados.

\section*{7. La revelación suprema del Padre}
\par
%\textsuperscript{(88.7)}
\textsuperscript{7:7.1} El Hijo Eterno es una revelación completa, exclusiva, universal y final del espíritu y de la personalidad del Padre Universal. Todo conocimiento y toda información acerca del Padre deben provenir del Hijo Eterno y de sus Hijos Paradisiacos. El Hijo Eterno procede de la eternidad y es uno con el Padre, totalmente y sin restricción espiritual. En personalidad divina, están coordinados; en naturaleza espiritual, son iguales; en divinidad, son idénticos.

\par
%\textsuperscript{(89.1)}
\textsuperscript{7:7.2} El carácter de Dios no podría mejorar intrínsecamente de ninguna manera en la persona del Hijo, pues el Padre divino es infinitamente perfecto, pero este carácter y esta personalidad, al ser despojados de aquello que no es personal ni espiritual, se amplifican para ser revelados a los seres creados. La Fuente-Centro Primera es mucho más que una personalidad, pero todas las cualidades espirituales de la personalidad paternal de la Fuente-Centro Primera están espiritualmente presentes en la personalidad absoluta del Hijo Eterno.

\par
%\textsuperscript{(89.2)}
\textsuperscript{7:7.3} El Hijo primordial y sus Hijos están dedicados a efectuar una revelación universal de la naturaleza espiritual y personal del Padre a toda la creación. En el universo central, los superuniversos, los universos locales o los planetas habitados, es un Hijo Paradisiaco el que revela el Padre Universal a los hombres y a los ángeles. El Hijo Eterno y sus Hijos revelan el camino\footnote{\textit{Los Hijos son el "camino"}: Jn 14:6.} por el que las criaturas pueden acceder al Padre Universal. E incluso nosotros, que tenemos un origen elevado, comprendemos mucho más plenamente al Padre a medida que estudiamos la revelación de su carácter y de su personalidad en el Hijo Eterno y en los Hijos del Hijo Eterno.

\par
%\textsuperscript{(89.3)}
\textsuperscript{7:7.4} El Padre sólo desciende hacia vosotros como personalidad a través de los Hijos divinos del Hijo Eterno. Y vosotros alcanzáis al Padre por este mismo camino viviente; ascendéis hacia el Padre mediante la guía de este grupo de Hijos divinos. Y esto sigue siendo cierto, a pesar de que vuestra personalidad misma sea un don directo del Padre Universal.

\par
%\textsuperscript{(89.4)}
\textsuperscript{7:7.5} En todas estas extensas actividades de la vasta administración espiritual del Hijo Eterno, no olvidéis que el Hijo es una persona tan real y auténtica como el Padre. En verdad, a los seres que en otro tiempo fueron humanos les será más fácil acercarse al Hijo Eterno que al Padre Universal. Al progresar como peregrinos del tiempo a través de los circuitos de Havona, seréis capaces de alcanzar al Hijo mucho antes de que estéis preparados para discernir al Padre.

\par
%\textsuperscript{(89.5)}
\textsuperscript{7:7.6} Deberíais comprender más cosas sobre el carácter y la naturaleza misericordiosa del Hijo Eterno de la misericordia a medida que reflexionéis sobre la revelación de estos atributos divinos, efectuada como servicio amoroso por vuestro propio Hijo Creador, en otro tiempo Hijo del Hombre en la Tierra, y ahora soberano exaltado de vuestro universo local ---el Hijo del Hombre y el Hijo de Dios.

\par
%\textsuperscript{(89.6)}
\textsuperscript{7:7.7} [Redactado por un Consejero Divino designado para formular esta declaración que describe al Hijo Eterno del Paraíso.]


\chapter{Documento 8. El Espíritu Infinito}
\par
%\textsuperscript{(90.1)}
\textsuperscript{8:0.1} ALLÁ por la eternidad, cuando el <<\textit{primer}>> pensamiento infinito y absoluto del Padre Universal encuentra en el Hijo Eterno un verbo tan perfecto y adecuado para su expresión divina, se produce a continuación tanto en el Dios-Pensamiento como en el Dios-Verbo el deseo supremo de tener un agente universal e infinito que los exprese mutuamente y actúe de manera combinada.

\par
%\textsuperscript{(90.2)}
\textsuperscript{8:0.2} En los albores de la eternidad, tanto el Padre como el Hijo se vuelven infinitamente conscientes de su mutua interdependencia, de su unidad eterna y absoluta; por consiguiente, establecen una alianza infinita y perpetua de asociación divina. Este acuerdo sin fin se efectúa para llevar a cabo sus conceptos unidos a lo largo de todo el círculo de la eternidad; y desde este acontecimiento sucedido en la eternidad, el Padre y el Hijo continúan con esta unión divina.

\par
%\textsuperscript{(90.3)}
\textsuperscript{8:0.3} Nos encontramos ahora frente a frente con el origen en la eternidad del Espíritu Infinito, la Tercera Persona de la Deidad. En el mismo instante en que Dios Padre y Dios Hijo conciben conjuntamente una acción idéntica e infinita ---la ejecución de un plan-pensamiento absoluto--- en ese mismo momento el Espíritu Infinito surge en toda su plenitud a la existencia.

\par
%\textsuperscript{(90.4)}
\textsuperscript{8:0.4} Al exponer de esta manera el orden del origen de las Deidades, lo hago así solamente para permitiros pensar en sus relaciones. En realidad, las tres existen desde la eternidad; son existenciales. No tienen ni principio ni fin en el tiempo; están coordinadas y son supremas, últimas, absolutas e infinitas. Existen, han existido siempre y siempre existirán. Son tres personas claramente individualizadas pero eternamente asociadas: Dios Padre, Dios Hijo y Dios Espíritu.

\section*{1. El Dios de acción}
\par
%\textsuperscript{(90.5)}
\textsuperscript{8:1.1} En la eternidad del pasado, con la personalización del Espíritu Infinito el ciclo divino de la personalidad se vuelve perfecto y completo. El Dios de Acción existe, y el inmenso escenario del espacio está preparado para el prodigioso drama de la creación ---para la aventura universal--- para el panorama divino de las eras eternas.

\par
%\textsuperscript{(90.6)}
\textsuperscript{8:1.2} El primer acto del Espíritu Infinito consiste en examinar y reconocer a sus padres divinos, el Padre-Padre y el Hijo-Madre. Él, el Espíritu, los identifica a los dos sin reserva. Es plenamente consciente de sus personalidades distintas y de sus atributos infinitos, así como de su naturaleza combinada y de su acción unida. Luego, voluntariamente, con una buena disposición trascendente y una espontaneidad inspiradora, la Tercera Persona de la Deidad, a pesar de su igualdad con la Primera y Segunda Personas, promete una lealtad eterna a Dios Padre y reconoce su dependencia perpetua de Dios Hijo.

\par
%\textsuperscript{(90.7)}
\textsuperscript{8:1.3} El ciclo de la eternidad queda establecido; es inherente a la naturaleza de esta operación, al reconocimiento mutuo de la independencia de la personalidad de cada una de las Deidades, y a la unión ejecutiva de las tres. La Trinidad del Paraíso ya existe. El escenario del espacio universal está preparado para el múltiple panorama sin fin donde el propósito del Padre Universal se despliega de forma creativa a través de la personalidad del Hijo Eterno y gracias a la actividad ejecutiva del Dios de Acción, el agente que ejecuta las acciones, en la realidad, de la asociación creadora Padre-Hijo.

\par
%\textsuperscript{(91.1)}
\textsuperscript{8:1.4} El Dios de Acción actúa y las bóvedas inertes del espacio se ponen en movimiento. Mil millones de esferas perfectas surgen de inmediato a la existencia. Antes de este momento hipotético de la eternidad, las energías espaciales inherentes al Paraíso ya existen y están potencialmente operativas, pero aún no se han manifestado; la gravedad física tampoco se puede medir si no es mediante la reacción de las realidades materiales a su atracción incesante. No existe ningún universo material en este
(supuesto) momento eternamente lejano, pero en el mismo instante en que se materializan mil millones de mundos, se pone de manifiesto una gravedad suficiente y adecuada para mantenerlos bajo la atracción perpetua del Paraíso.

\par
%\textsuperscript{(91.2)}
\textsuperscript{8:1.5} Ahora centellea por toda la creación de los Dioses la segunda forma de energía, y este espíritu que mana a raudales es atraído instantáneamente por la gravedad espiritual del Hijo Eterno. Y así, el universo dos veces abrazado por la gravedad es tocado por la energía de la infinidad y sumergido en el espíritu de la divinidad. De esta forma, el terreno de la vida está preparado para la conciencia de la mente, puesta de manifiesto en los circuitos de inteligencia asociados del Espíritu Infinito.

\par
%\textsuperscript{(91.3)}
\textsuperscript{8:1.6} Sobre estas semillas de existencia potencial, difundidas por toda la creación central de los Dioses, el Padre actúa, y la personalidad de las criaturas aparece. Luego, la presencia de las Deidades del Paraíso ocupa todo el espacio organizado y empieza a atraer eficazmente a todas las cosas y a todos los seres hacia el Paraíso.

\par
%\textsuperscript{(91.4)}
\textsuperscript{8:1.7} El Espíritu Infinito se eterniza al mismo tiempo que nacen los mundos de Havona, siendo este universo central creado por él, con él y en él, de conformidad con los conceptos combinados y las voluntades unidas del Padre y del Hijo. La Tercera Persona se deifica mediante este acto mismo de creación conjunta, convirtiéndose así para siempre en el Creador Conjunto.

\par
%\textsuperscript{(91.5)}
\textsuperscript{8:1.8} Son los tiempos grandiosos e impresionantes de la expansión creadora del Padre y del Hijo por medio de, y en la acción de, su asociado conjunto y ejecutivo exclusivo, la Fuente-Centro Tercera. No existe ningún archivo de estos tiempos agitados. Sólo disponemos de las escasas revelaciones del Espíritu Infinito para justificar estas poderosas operaciones, y él se limita a confirmar el hecho de que el universo central y todo lo relacionado con éste se eternizó al mismo tiempo que él conseguía la personalidad y la existencia consciente.

\par
%\textsuperscript{(91.6)}
\textsuperscript{8:1.9} En resumen, el Espíritu Infinito declara que, puesto que él es eterno, el universo central también lo es. Éste es el punto de partida tradicional de la historia del universo de universos. No se sabe absolutamente nada, y no existen archivos, respecto a cualquier acontecimiento o actividad anterior a esta prodigiosa erupción de energía creativa y de sabiduría administrativa que cristalizó el inmenso universo que existe y que funciona tan exquisitamente en el centro de todas las cosas. Más allá de este acontecimiento se extienden las operaciones impenetrables de la eternidad y las profundidades de la infinidad ---misterio absoluto.

\par
%\textsuperscript{(91.7)}
\textsuperscript{8:1.10} Describimos de esta forma el origen secuencial de la Fuente-Centro Tercera como una condescendencia interpretativa hacia la mente de las criaturas mortales, atada al tiempo y condicionada por el espacio. La mente del hombre necesita tener un punto de partida para poder imaginarse la historia del universo, y se me ha ordenado que proporcione esta técnica para que pueda acceder al concepto histórico de la eternidad. Para la mente material, la coherencia exige que exista una Causa Primera; por eso consideramos como un postulado que el Padre Universal es la Fuente Primera y el Centro Absoluto de toda la creación, al mismo tiempo que enseñamos a la mente de todas las criaturas que el Hijo y el Espíritu son coeternos con el Padre en todas las fases de la historia universal y en todos los ámbitos de la actividad creadora. Y hacemos esto sin descuidar de ninguna manera la realidad y la eternidad de la Isla del Paraíso y de los Absolutos Incalificado, Universal y de la Deidad.

\par
%\textsuperscript{(92.1)}
\textsuperscript{8:1.11} Ya es suficiente con que la mente material de los hijos del tiempo sea capaz de concebir al Padre en la eternidad. Sabemos que todo niño puede relacionarse mejor con la realidad dominando primero las relaciones de la situación padre-hijo, y ampliando después este concepto hasta abarcar a la familia como un todo. Posteriormente, la mente en desarrollo del niño será capaz de ajustarse al concepto de las relaciones familiares, de las relaciones de la comunidad, la raza y el mundo, y luego a las del universo, del superuniverso e incluso del universo de universos.

\section*{2. La naturaleza del Espíritu Infinito}
\par
%\textsuperscript{(92.2)}
\textsuperscript{8:2.1} El Creador Conjunto existe desde la eternidad y es uno, de manera total y sin restricción, con el Padre Universal y el Hijo Eterno. El Espíritu Infinito refleja a la perfección no solamente la naturaleza del Padre Paradisiaco, sino también la del Hijo Original.

\par
%\textsuperscript{(92.3)}
\textsuperscript{8:2.2} A la Fuente-Centro Tercera se le conoce por numerosos títulos: el Espíritu Universal, el Guía Supremo, el Creador Conjunto, el Ejecutivo Divino, la Mente Infinita, el Espíritu de los Espíritus, el Espíritu Madre Paradisiaco, el Actor Conjunto, el Coordinador Final, el Espíritu Omnipresente, la Inteligencia Absoluta, la Acción Divina; y en Urantia se le confunde a veces con la mente cósmica.

\par
%\textsuperscript{(92.4)}
\textsuperscript{8:2.3} Es totalmente adecuado denominar Espíritu Infinito a la Tercera Persona de la Deidad, porque Dios es espíritu\footnote{\textit{Dios es espíritu}: Jn 4:24.}. Pero las criaturas materiales, que tienden a cometer el error de considerar la materia como la realidad fundamental, y la mente, así como el espíritu, como postulados enraizados en la materia, comprenderían mejor a la Fuente-Centro Tercera si lo llamaran la Realidad Infinita, el Organizador Universal o el Coordinador de la Personalidad.

\par
%\textsuperscript{(92.5)}
\textsuperscript{8:2.4} El Espíritu Infinito, como revelación universal de la divinidad, es insondable y está totalmente fuera de la comprensión humana. Para percibir la absolutidad del Espíritu, sólo necesitáis contemplar la infinidad del Padre Universal y sentiros asombrados por la eternidad del Hijo Original.

\par
%\textsuperscript{(92.6)}
\textsuperscript{8:2.5} Hay misterio en verdad en la persona del Espíritu Infinito, pero no tanto como en el Padre y el Hijo. De todos los aspectos de la naturaleza del Padre, es su infinidad la que el Creador Conjunto revela de manera más notable. Aunque el universo maestro se extienda finalmente hasta la infinidad, la presencia espiritual, el control energético y el potencial mental del Actor Conjunto serán adecuados para satisfacer las exigencias de esa creación ilimitada.

\par
%\textsuperscript{(92.7)}
\textsuperscript{8:2.6} Aunque el Espíritu Infinito comparte en todos los sentidos la perfección, la rectitud y el amor del Padre Universal, siente inclinación hacia los atributos de misericordia del Hijo Eterno, convirtiéndose así en el ministro de la misericordia de las Deidades del Paraíso para el gran universo. Para siempre jamás ---de manera universal y eterna--- el Espíritu es un ministro de misericordia, porque al igual que los Hijos divinos revelan el amor de Dios, el Espíritu divino describe la misericordia de Dios.

\par
%\textsuperscript{(93.1)}
\textsuperscript{8:2.7} No es posible que el Espíritu pueda tener más bondad que el Padre, puesto que toda bondad tiene su origen en el Padre, pero esta bondad la podemos comprender mejor en los actos del Espíritu. La fidelidad del Padre y la constancia del Hijo se hacen muy reales para los seres espirituales y las criaturas materiales de las esferas gracias al ministerio amoroso y al servicio incesante de las personalidades del Espíritu Infinito.

\par
%\textsuperscript{(93.2)}
\textsuperscript{8:2.8} El Creador Conjunto hereda toda la belleza de pensamiento y todo el carácter veraz del Padre. Estas características sublimes de la divinidad están coordinadas en los niveles casi supremos de la mente cósmica, la cual está subordinada a la sabiduría eterna e infinita de la mente incondicionada e ilimitada de la Fuente-Centro Tercera.

\section*{3. Las relaciones del Espíritu con el Padre y el Hijo}
\par
%\textsuperscript{(93.3)}
\textsuperscript{8:3.1} Al igual que el Hijo Eterno es la expresión verbal del <<\textit{primer}>> pensamiento absoluto e infinito del Padre Universal, el Actor Conjunto es la ejecución perfecta del <<\textit{primer}>> concepto, o plan creador completo, para efectuar la acción combinada de la asociación entre las personalidades del Padre y del Hijo, compuesta por la unión absoluta entre el pensamiento y el verbo. La Fuente-Centro Tercera se eterniza al mismo tiempo que la creación central, hecha por decreto, y sólo esta creación central tiene una existencia eterna entre los universos.

\par
%\textsuperscript{(93.4)}
\textsuperscript{8:3.2} Desde la personalización de la Fuente Tercera, la Fuente Primera ya no participa personalmente en la creación del universo. El Padre Universal delega todo aquello que es posible a su Hijo Eterno; de igual manera, el Hijo Eterno deposita toda la autoridad y todo el poder posibles en el Creador Conjunto.

\par
%\textsuperscript{(93.5)}
\textsuperscript{8:3.3} El Hijo Eterno y el Creador Conjunto han planeado y formado, como asociados y por medio de sus personalidades coordinadas, todos los universos que han sido traídos a la existencia después de Havona. En todas las creaciones posteriores, el Espíritu mantiene con el Hijo la misma relación personal que el Hijo mantiene con el Padre en la primera creación central.

\par
%\textsuperscript{(93.6)}
\textsuperscript{8:3.4} Un Hijo Creador del Hijo Eterno y un Espíritu Creativo del Espíritu Infinito os han creado, a vosotros y a vuestro universo; y aunque el Padre sostiene fielmente aquello que han organizado, a este Hijo Universal y a este Espíritu Universal les incumbe fomentar y sostener su obra, así como aportar su ministerio a las criaturas creadas por ellos mismos.

\par
%\textsuperscript{(93.7)}
\textsuperscript{8:3.5} El Espíritu Infinito es el agente eficaz del Padre amoroso y del Hijo misericordioso que ejecuta su proyecto conjunto de atraer hacia ellos a todas las almas que aman la verdad en todos los mundos del tiempo y del espacio. En el mismo instante en que el Hijo Eterno aceptó el plan de su Padre consistente en que las criaturas de los universos alcanzaran la perfección, en el momento en que el proyecto de ascensión se convirtió en un plan del Padre y del Hijo, en ese instante el Espíritu Infinito se convirtió en el administrador conjunto del Padre y del Hijo para llevar a cabo su propósito eterno y unido. Al hacer esto, el Espíritu Infinito prometió al Padre y al Hijo todos los recursos de su presencia divina y de sus personalidades espirituales; lo ha dedicado \textit{todo} al prodigioso plan de elevar a las criaturas volitivas sobrevivientes a las alturas divinas de la perfección paradisiaca.

\par
%\textsuperscript{(93.8)}
\textsuperscript{8:3.6} El Espíritu Infinito es una revelación completa, exclusiva y universal del Padre Universal y de su Hijo Eterno. Todo conocimiento relacionado con la asociación Padre-Hijo ha de adquirirse a través del Espíritu Infinito, el representante conjunto de la unión divina entre el pensamiento y el verbo.

\par
%\textsuperscript{(93.9)}
\textsuperscript{8:3.7} El Hijo Eterno es el único camino de acceso al Padre Universal\footnote{\textit{El Hijo: el único camino}: Mt 11:27; Lc 10:22; Jn 1:18; 6:44-46; 14:6-11,20.}, y el Espíritu Infinito es el único medio de alcanzar al Hijo Eterno. Los seres ascendentes del tiempo sólo pueden descubrir al Hijo por medio del paciente ministerio del Espíritu.

\par
%\textsuperscript{(94.1)}
\textsuperscript{8:3.8} El Espíritu Infinito es la primera de las Deidades del Paraíso que alcanzan los peregrinos ascendentes en el centro de todas las cosas. La Tercera Persona envuelve a la Segunda y a la Primera Personas, y por eso siempre ha de ser reconocida primero por todos los candidatos que desean ser presentados al Hijo y a su Padre.

\par
%\textsuperscript{(94.2)}
\textsuperscript{8:3.9} El Espíritu representa igualmente y sirve de forma similar al Padre y al Hijo de otras muchas maneras.

\section*{4. El espíritu del ministerio divino}
\par
%\textsuperscript{(94.3)}
\textsuperscript{8:4.1} Paralelamente al universo físico donde la gravedad del Paraíso mantiene unidas todas las cosas, existe el universo espiritual donde la palabra del Hijo interpreta el pensamiento de Dios, y cuando este verbo <<\textit{se hace carne}>>\footnote{\textit{El Verbo hecho carne}: Jn 1:14.}, demuestra la misericordia amorosa de la naturaleza combinada de los Creadores asociados. Pero en toda esta creación material y espiritual, y a través de ella, existe un inmenso escenario en el que el Espíritu Infinito y su progenitura espiritual dan a conocer la misericordia, la paciencia y el afecto perpetuo combinados de los padres divinos hacia los hijos inteligentes que han concebido y creado en cooperación. La esencia del carácter divino del Espíritu es servir perpetuamente a la mente. Y toda la descendencia espiritual del Actor Conjunto participa en este deseo de ofrecer su ministerio, en este impulso divino a servir.

\par
%\textsuperscript{(94.4)}
\textsuperscript{8:4.2} Dios es amor\footnote{\textit{Dios es amor}: 1 Jn 4:8,16.}, el Hijo es misericordia, el Espíritu es ministerio ---el ministerio del amor divino y de la misericordia sin fin para toda la creación inteligente. El Espíritu es la personificación del amor del Padre y de la misericordia del Hijo; en él están los dos eternamente unidos para el servicio universal. El Espíritu es el \textit{amor aplicado} para la creación compuesta de criaturas, el amor combinado del Padre y del Hijo.

\par
%\textsuperscript{(94.5)}
\textsuperscript{8:4.3} En Urantia, el Espíritu Infinito es conocido como una influencia omnipresente, una presencia universal, pero en Havona lo conoceréis como una presencia personal de verdadero servicio. Aquí, el ministerio del Espíritu del Paraíso es el modelo ejemplar e inspirador para cada uno de sus Espíritus coordinados y de sus personalidades subordinadas que sirven a los seres creados en los mundos del tiempo y del espacio. En este universo divino, el Espíritu Infinito participó plenamente en las siete apariciones trascendentales del Hijo Eterno; también participó con el Hijo Miguel original en las siete donaciones sobre los circuitos de Havona, convirtiéndose así en el ministro espiritual compasivo y comprensivo para cada peregrino del tiempo que atraviesa estos círculos perfectos de las alturas.

\par
%\textsuperscript{(94.6)}
\textsuperscript{8:4.4} Cuando un Hijo de Dios Creador acepta la responsabilidad de crear un universo local en proyecto, las personalidades del Espíritu Infinito se comprometen a ser los ministros incansables de este Hijo Miguel cuando emprende su misión de aventura creadora. Al Espíritu Infinito lo encontramos especialmente en las personas de las Hijas Creativas, los Espíritus Madres de los universos locales, y lo encontramos dedicado a la tarea de fomentar la ascensión de las criaturas materiales hacia unos niveles cada vez más elevados de consecución espiritual. Todo este trabajo de servicio hacia las criaturas es efectuado en perfecta armonía con los objetivos, y en estrecha asociación con las personalidades, de los Hijos Creadores de estos universos locales.

\par
%\textsuperscript{(94.7)}
\textsuperscript{8:4.5} Al igual que los Hijos de Dios se ocupan de la gigantesca tarea de revelar a un universo la personalidad amorosa del Padre, el Espíritu Infinito se dedica al ministerio interminable de revelar el amor combinado del Padre y del Hijo a las mentes individuales de todos los hijos de cada universo. En estas creaciones locales, el Espíritu no desciende hasta las razas materiales en la similitud de la carne mortal como lo hacen algunos Hijos de Dios, sino que el Espíritu Infinito y sus Espíritus coordinados rebajan su categoría, experimentan alegremente una serie asombrosa de atenuaciones de su divinidad, hasta que aparecen como ángeles para estar a vuestro lado y guiaros por los humildes caminos de la existencia terrestre.

\par
%\textsuperscript{(95.1)}
\textsuperscript{8:4.6} Mediante esta misma serie decreciente, el Espíritu Infinito se acerca realmente mucho, como persona, a cada ser de las esferas de origen animal. Y el Espíritu hace todo esto sin invalidar en lo más mínimo su existencia como Tercera Persona de la Deidad en el centro de todas las cosas\footnote{\textit{Atracción espiritual}: Heb 7:19; Stg 4:8.}.

\par
%\textsuperscript{(95.2)}
\textsuperscript{8:4.7} El Creador Conjunto es verdaderamente y para siempre la gran personalidad ministrante, el ministro universal de la misericordia. Para comprender el ministerio del Espíritu, reflexionad sobre la verdad de que él es el retrato combinado del amor interminable del Padre y de la misericordia eterna del Hijo. Sin embargo, el ministerio del Espíritu no está únicamente limitado a representar al Hijo Eterno y al Padre Universal. El Espíritu Infinito posee también el poder de servir a las criaturas del universo en su propio nombre y derecho; la Tercera Persona tiene una dignidad divina y dispensa también el ministerio universal de la misericordia por su propia cuenta.

\par
%\textsuperscript{(95.3)}
\textsuperscript{8:4.8} A medida que el hombre aprenda más cosas sobre el ministerio amoroso e infatigable de las órdenes inferiores de la familia de criaturas de este Espíritu Infinito, admirará y adorará más la naturaleza trascendente y el carácter incomparable de esta Acción combinada del Padre Universal y del Hijo Eterno. En verdad, este Espíritu es <<\textit{los ojos del Señor que están siempre sobre los justos}>>\footnote{\textit{Los ojos del Señor están sobre los justos}: Sal 34:15; 1 P 3:12.} y <<\textit{los oídos divinos que siempre están abiertos a sus oraciones}>>\footnote{\textit{Los oídos divinos siempre están abiertos}: Sal 34:15; 1 P 3:12.}.

\section*{5. La presencia de Dios}
\par
%\textsuperscript{(95.4)}
\textsuperscript{8:5.1} El atributo sobresaliente del Espíritu Infinito es su omnipresencia. En todo el universo de universos está presente en todas partes este espíritu que lo impregna todo, y que es tan semejante a la presencia de una mente universal y divina. Tanto la Segunda Persona como la Tercera Persona de la Deidad están representadas en todos los mundos por sus espíritus siempre presentes.

\par
%\textsuperscript{(95.5)}
\textsuperscript{8:5.2} El Padre es \textit{infinito} y, por consiguiente, sólo está limitado por su volición. En la concesión de los Ajustadores y en la incorporación de la personalidad a su circuito, el Padre actúa solo, pero en el contacto de las fuerzas espirituales con los seres inteligentes utiliza los espíritus y las personalidades del Hijo Eterno y del Espíritu Infinito. Está espiritualmente presente a voluntad, y de igual manera, con el Hijo o con el Actor Conjunto; está presente \textit{con} el Hijo y \textit{en} el Espíritu. El Padre está presente con toda seguridad en todas partes, y nosotros discernimos su presencia por y a través de todas estas fuerzas, influencias y presencias diversas pero asociadas.

\par
%\textsuperscript{(95.6)}
\textsuperscript{8:5.3} En vuestras escrituras sagradas, el término \textit{Espíritu de Dios}\footnote{\textit{Espíritu de Dios (Espíritu Santo)}: Gn 1:2; Ex 31:3; 35:31; Job 33:4; Sal 51:10-11; 139:7; Pr 1:23; Is 44:3; 59:21; 61:1; 63:10-11; Lc 4:1; 11:13; Jn 1:33; 3:5; 2 Ti 1:14.} parece haber sido empleado para designar indistintamente tanto al Espíritu Infinito del Paraíso como al Espíritu Creativo de vuestro universo local. El Espíritu Santo es el circuito espiritual de esta Hija Creativa del Espíritu Infinito del Paraíso. El Espíritu Santo es un circuito autóctono de cada universo local y está limitado al ámbito espiritual de esa creación; pero el Espíritu Infinito es omnipresente.

\par
%\textsuperscript{(95.7)}
\textsuperscript{8:5.4} Existen muchas influencias espirituales, y todas funcionan como \textit{una sola.} Incluso el trabajo de los Ajustadores del Pensamiento, aunque es independiente de todas las otras influencias, coincide invariablemente con el ministerio espiritual de las influencias combinadas del Espíritu Infinito y del Espíritu Madre de un universo local. Estas presencias espirituales, tal como funcionan en la vida de los urantianos, no se pueden separar. Actúan en vuestra mente y sobre vuestra alma como un solo espíritu, a pesar de sus orígenes diversos. Y a medida que experimentáis este ministerio espiritual unido, para vosotros se convierte en la influencia del Supremo, <<\textit{que siempre es capaz de evitar que falléis y de presentaros irreprochables ante vuestro Padre en las alturas}>>\footnote{\textit{Capaz de sosteneros y presentaros}: Jud 1:24.}.

\par
%\textsuperscript{(96.1)}
\textsuperscript{8:5.5} Recordad siempre que el Espíritu Infinito es el Actor \textit{Conjunto;} tanto el Padre como el Hijo actúan en él y a través de él; está presente no sólo como él mismo, sino también como Padre y como Hijo, y como Padre-Hijo. En reconocimiento de este hecho y por muchas razones adicionales, a la presencia espiritual del Espíritu Infinito se la califica a menudo de <<\textit{el espíritu de Dios}>>\footnote{\textit{Espíritu de Dios (Espíritu Santo)}: Gn 1:2; Ex 31:3; 35:31; Job 33:4; Sal 51:10-11; 139:7; Pr 1:23; Is 44:3; 59:21; 61:1; 63:10-11; Lc 4:1; 11:13; Jn 1:33; 3:5; 2 Ti 1:14.}.

\par
%\textsuperscript{(96.2)}
\textsuperscript{8:5.6} También sería coherente referirse a la coordinación de todo el ministerio espiritual como el espíritu de Dios\footnote{\textit{Dios es un espíritu}: Jn 4:24.}, porque esta coordinación es realmente la unión de los espíritus de Dios Padre, Dios Hijo, Dios Espíritu y Dios Séptuple ---el espíritu mismo de Dios Supremo.

\section*{6. La personalidad del Espíritu Infinito}
\par
%\textsuperscript{(96.3)}
\textsuperscript{8:6.1} No permitáis que la donación tan difundida y la extensa distribución de la Fuente-Centro Tercera oscurezcan o disminuyan de otra manera el hecho de su personalidad. El Espíritu Infinito es una presencia universal, una acción eterna, un poder cósmico, una influencia sagrada y una mente universal; es todo esto e infinitamente más, pero es también una verdadera personalidad divina.

\par
%\textsuperscript{(96.4)}
\textsuperscript{8:6.2} El Espíritu Infinito es una personalidad completa y perfecta, el coordinado y el igual divino del Padre Universal y del Hijo Eterno. El Creador Conjunto es tan real y visible para las inteligencias superiores de los universos como el Padre y el Hijo; en verdad lo es más, porque es el Espíritu el que todos los ascendentes deben alcanzar antes de poder acercarse al Padre a través del Hijo.

\par
%\textsuperscript{(96.5)}
\textsuperscript{8:6.3} El Espíritu Infinito, la Tercera Persona de la Deidad, posee todos los atributos que vosotros asociáis con la personalidad. El Espíritu está dotado de una mente absoluta: <<\textit{El Espíritu sondea todas las cosas, incluso las cosas profundas de Dios}>>\footnote{\textit{El Espíritu sondea todas las cosas}: 1 Cr 28:9; 1 Co 2:10.}. El Espíritu no sólo está dotado de mente, sino también de voluntad. A propósito de la concesión de sus dones, está escrito: <<\textit{Pero todas estas cosas las hace el solo y mismo Espíritu, repartiendo a cada cual individualmente y como él quiere}>>\footnote{\textit{El mismo Espíritu repartido al hombre}: 1 Co 12:11.}.

\par
%\textsuperscript{(96.6)}
\textsuperscript{8:6.4} <<\textit{El amor del Espíritu}>>\footnote{\textit{El amor del Espíritu}: Ro 15:30.} es real, como lo son también sus tristezas; por ello, <<\textit{no aflijáis al Espíritu de Dios}>>\footnote{\textit{No aflijáis al Espíritu de Dios}: Ef 4:30.}. Cuando observamos al Espíritu Infinito, ya sea como una Deidad del Paraíso o como el Espíritu Creativo de un universo local, descubrimos que el Creador Conjunto no es solamente la Fuente-Centro Tercera sino también una persona divina. Esta personalidad divina reacciona también ante el universo como una persona. El Espíritu os dice: <<\textit{Aquel que tenga oídos, que escuche lo que dice el Espíritu}>>\footnote{\textit{Quien tenga oídos que oiga al Espíritu}: Ap 2:7,11,17,29; 3:6,13,22.}. <<\textit{El Espíritu mismo intercede por vosotros}>>\footnote{\textit{El Espíritu intercede}: Ro 8:26-27.}. El Espíritu ejerce una influencia personal y directa sobre los seres creados, <<\textit{porque todos aquellos que son conducidos por el Espíritu de Dios, son hijos de Dios}>>\footnote{\textit{Hijos de Dios}: 1 Cr 22:10; Sal 2:7; Is 56:5; Mt 5:9,16,45; Lc 20:36; Jn 1:12-13; 11:52; Hch 17:28-29; Ro 9:26; 2 Co 6:18; Gl 3:26; 4:5-7; Ef 1:5; Flp 2:15; Heb 12:5-8; 1 Jn 3:1-2,10; 5:2; Ap 21:7; 2 Sam 7:14.}\footnote{\textit{Los conducidos por el Espíritu son hijos de Dios}: Ro 8:14-17,19,21.}.

\par
%\textsuperscript{(96.7)}
\textsuperscript{8:6.5} Aunque contemplemos el fenómeno del ministerio del Espíritu Infinito en los mundos lejanos del universo de universos, aunque imaginemos a esta misma Deidad coordinadora actuando en, y por medio de, las legiones incalculables de los múltiples seres que tienen su origen en la Fuente-Centro Tercera, aunque reconozcamos la omnipresencia del Espíritu, sin embargo seguimos afirmando que esta misma Fuente-Centro Tercera es una persona, el Creador Conjunto de todas las cosas, de todos los seres y de todos los universos.

\par
%\textsuperscript{(96.8)}
\textsuperscript{8:6.6} En la administración de los universos, el Padre, el Hijo y el Espíritu están perfecta y eternamente interasociados. Aunque cada uno de ellos está consagrado a un ministerio personal hacia toda la creación, los tres están divina y absolutamente entrelazados en un servicio de creación y de control que los convierte para siempre en \textit{uno solo.}

\par
%\textsuperscript{(97.1)}
\textsuperscript{8:6.7} En la persona del Espíritu Infinito, el Padre y el Hijo están siempre mutuamente presentes con una perfección incalificada, porque el Espíritu se parece al Padre y se parece al Hijo, y también se parece al Padre y al Hijo, ya que los dos son eternamente uno solo.

\par
%\textsuperscript{(97.2)}
\textsuperscript{8:6.8} [Presentado en Urantia por un Consejero Divino de Uversa, encargado por los Ancianos de los Días de describir la naturaleza y el trabajo del Espíritu Infinito.]


\chapter{Documento 9. Las relaciones del Espíritu Infinito con el universo}
\par
%\textsuperscript{(98.1)}
\textsuperscript{9:0.1} CUANDO el Padre Universal y el Hijo Eterno se unieron para personalizarse en presencia del Paraíso, se produjo una cosa extraña. Nada, en esta situación de la eternidad, inducía a presagiar que el Actor Conjunto se personalizaría como una espiritualidad ilimitada, coordinada con la mente absoluta y dotada de prerrogativas únicas para manipular la energía. Su nacimiento termina de liberar al Padre de las cadenas de la perfección centralizada y de las trabas del absolutismo de la personalidad. Esta liberación está manifestada en el asombroso poder del Creador Conjunto para crear seres bien adaptados que servirán como espíritus ministrantes incluso a las criaturas materiales de los universos que evolucionarán posteriormente.

\par
%\textsuperscript{(98.2)}
\textsuperscript{9:0.2} El Padre es infinito en amor y en volición, en pensamiento y en propósito espirituales; es el sostén universal. El Hijo es infinito en sabiduría y en verdad, en expresión y en interpretación espirituales; es el revelador universal. El Paraíso es infinito en potencial para dotar de fuerza y en capacidad para dominar la energía; es el estabilizador universal. El Actor Conjunto posee prerrogativas únicas de síntesis, una capacidad infinita para coordinar todas las energías existentes en el universo, todos los espíritus reales del universo y todos los verdaderos intelectos del universo; la Fuente-Centro Tercera es el unificador universal de las múltiples energías y de las diversas creaciones que han aparecido como resultado del plan divino y del propósito eterno del Padre Universal.

\par
%\textsuperscript{(98.3)}
\textsuperscript{9:0.3} El Espíritu Infinito, el Creador Conjunto, es un ministro universal y divino. El Espíritu administra sin cesar la misericordia del Hijo y el amor del Padre, en armonía con la justicia estable, invariable y recta de la Trinidad del Paraíso. Su influencia y sus personalidades siempre están cerca de vosotros; os conocen realmente y os comprenden verdaderamente.

\par
%\textsuperscript{(98.4)}
\textsuperscript{9:0.4} En todos los universos, los agentes del Actor Conjunto manipulan sin cesar las fuerzas y las energías de todo el espacio. Al igual que la Fuente-Centro Primera, el Centro Tercero es sensible tanto a lo espiritual como a lo material. El Actor Conjunto es la revelación de la unidad de Dios, en quien todas las cosas consisten\footnote{\textit{En Dios todas las cosas consisten}: Hch 17:28; Col 1:17,19.} ---cosas, significados y valores; energías, mentes y espíritus.

\par
%\textsuperscript{(98.5)}
\textsuperscript{9:0.5} El Espíritu Infinito impregna todo el espacio; habita el círculo de la eternidad; y el Espíritu, al igual que el Padre y el Hijo, es perfecto e invariable ---absoluto.

\section*{1. Los atributos de la Fuente-Centro Tercera}
\par
%\textsuperscript{(98.6)}
\textsuperscript{9:1.1} A la Fuente-Centro Tercera se le conoce por muchos nombres, y todos ellos designan relaciones y reconocen funciones: como Dios Espíritu, es la personalidad coordinada y el divino igual de Dios Hijo y de Dios Padre. Como Espíritu Infinito, es una influencia espiritual omnipresente. Como Manipulador Universal, es el antepasado de las criaturas que controlan el poder, y el activador de las fuerzas cósmicas del espacio. Como Actor Conjunto, es el representante colectivo y el ejecutivo de la asociación compuesta por el Padre y el Hijo. Como Mente Absoluta, es la fuente de la donación del intelecto en todos los universos. Como Dios de Acción, es el antepasado aparente del movimiento, del cambio y de las relaciones.

\par
%\textsuperscript{(99.1)}
\textsuperscript{9:1.2} Algunos atributos de la Fuente-Centro Tercera proceden del Padre, otros del Hijo, pero se observa que existen otros atributos que no están activa y personalmente presentes ni en el Padre ni en el Hijo ---unos atributos que difícilmente se pueden explicar salvo suponiendo que la asociación Padre-Hijo, que eterniza a la Fuente-Centro Tercera, ejerce sus funciones de manera coherente en consonancia con el hecho eterno de la absolutidad del Paraíso, y en reconocimiento de dicho hecho. El Creador Conjunto personifica la plenitud de los conceptos combinados e infinitos de la Primera y de la Segunda Personas de la Deidad.

\par
%\textsuperscript{(99.2)}
\textsuperscript{9:1.3} Cuando imagináis al Padre como un creador original y al Hijo como un administrador espiritual, deberíais pensar en la Fuente-Centro Tercera como en un coordinador universal, un ministro que coopera de manera ilimitada. El Actor Conjunto es el que pone en correlación toda la realidad manifestada; es la Deidad depositaria del pensamiento del Padre y de la palabra del Hijo, y cuando actúa, es eternamente respetuoso con la absolutidad material de la Isla central. La Trinidad del Paraíso ha decretado la orden universal del \textit{progreso,} y la providencia de Dios es el ámbito del Creador Conjunto y del Ser Supremo en evolución. Ninguna realidad manifestada o en vías de manifestarse puede eludir una relación final con la Fuente-Centro Tercera.

\par
%\textsuperscript{(99.3)}
\textsuperscript{9:1.4} El Padre Universal preside los dominios de la preenergía, del preespíritu y de la personalidad; el Hijo Eterno domina las esferas de las actividades espirituales; la presencia de la Isla del Paraíso unifica el dominio de la energía física y del poder que se materializa; el Actor Conjunto actúa no solamente como un espíritu infinito que representa al Hijo, sino también como manipulador universal de las fuerzas y de las energías del Paraíso, trayendo así a la existencia a la mente universal y absoluta. El Actor Conjunto ejerce su actividad en todo el gran universo como una personalidad verdadera y bien diferenciada, especialmente en las esferas superiores de los valores espirituales, de las relaciones entre la energía y la materia, y de los verdaderos significados mentales. Ejerce sus funciones específicamente en cualquier momento y lugar donde la energía y el espíritu se asocian e interactúan; domina todas las reacciones con la mente, ejerce un gran poder en el mundo espiritual y efectúa una poderosa influencia sobre la energía y la materia. La Fuente Tercera expresa en todo momento la naturaleza de la Fuente-Centro Primera.

\par
%\textsuperscript{(99.4)}
\textsuperscript{9:1.5} La Fuente-Centro Tercera comparte de manera perfecta y sin restricciones la omnipresencia de la Fuente-Centro Primera, y a veces se le llama el Espíritu Omnipresente. El Dios de la mente comparte de una forma especial y muy personal la omnisciencia del Padre Universal y de su Hijo Eterno; el conocimiento del Espíritu es profundo y completo. El Creador Conjunto manifiesta ciertas fases de la omnipotencia del Padre Universal, pero sólo es realmente omnipotente en el ámbito de la mente. La Tercera Persona de la Deidad es el centro intelectual y el administrador universal de los dominios de la mente; en esto es absoluto ---su soberanía es incalificada.

\par
%\textsuperscript{(99.5)}
\textsuperscript{9:1.6} El Actor Conjunto parece estar motivado por la asociación Padre-Hijo, pero todos sus actos parecen reconocer la relación Padre-Paraíso. A veces, y en ciertas funciones, parece compensar el desarrollo incompleto de las Deidades experienciales ---Dios Supremo y Dios Último.

\par
%\textsuperscript{(100.1)}
\textsuperscript{9:1.7} Y en esto reside un misterio infinito: el Infinito reveló simultáneamente su infinidad en el Hijo y bajo la forma del Paraíso, y entonces surge a la existencia un ser igual a Dios en divinidad, que refleja la naturaleza espiritual del Hijo y es capaz de activar el arquetipo del Paraíso, un ser provisionalmente subordinado en soberanía, pero aparentemente el más polifacético, de muchas maneras, en la \textit{acción.} Esta superioridad aparente en la acción se revela en un atributo de la Fuente-Centro Tercera que es superior incluso a la gravedad física ---la manifestación universal de la Isla del Paraíso.

\par
%\textsuperscript{(100.2)}
\textsuperscript{9:1.8} Además de este supercontrol de la energía y de las cosas físicas, el Espíritu Infinito está magníficamente dotado de esos atributos de paciencia, de misericordia y de amor que se revelan tan exquisitamente en su ministerio espiritual. El Espíritu es supremamente competente para dar amor y eclipsar la justicia con la misericordia. Dios Espíritu posee toda la bondad celestial y todo el afecto misericordioso del Hijo Original y Eterno. El universo de vuestro origen se está forjando entre el yunque de la justicia y el martillo del sufrimiento; pero aquellos que manejan el martillo son los hijos de la misericordia, la progenitura espiritual del Espíritu Infinito.

\section*{2. El Espíritu omnipresente}
\par
%\textsuperscript{(100.3)}
\textsuperscript{9:2.1} Dios es espíritu\footnote{\textit{Dios es espíritu}: Jn 4:24.} en un sentido triple: él mismo es espíritu; en su Hijo aparece como un espíritu sin restricción y en el Actor Conjunto como un espíritu aliado a la mente. Además de estas realidades espirituales, creemos discernir unos niveles de fenómenos espirituales experienciales ---los espíritus del Ser Supremo, de la Deidad Última y del Absoluto de la Deidad.

\par
%\textsuperscript{(100.4)}
\textsuperscript{9:2.2} El Espíritu Infinito complementa al Hijo Eterno como el Hijo complementa al Padre Universal. El Hijo Eterno es una personalización espiritualizada del Padre; el Espíritu Infinito es una espiritualización personalizada del Hijo Eterno y del Padre Universal.

\par
%\textsuperscript{(100.5)}
\textsuperscript{9:2.3} Existen muchas líneas ilimitadas de fuerza espiritual y muchas fuentes de poder supermaterial que conectan directamente a la población de Urantia con las Deidades del Paraíso. Existe la conexión directa de los Ajustadores del Pensamiento con el Padre Universal, la influencia general del impulso de la gravedad espiritual del Hijo Eterno, y la presencia espiritual del Creador Conjunto. Existe una diferencia de función entre el espíritu del Hijo y el espíritu del Espíritu. En su ministerio espiritual, la Tercera Persona puede ejercer su actividad como mente más espíritu, o como espíritu solamente.

\par
%\textsuperscript{(100.6)}
\textsuperscript{9:2.4} Además de estas presencias paradisiacas, los urantianos se benefician de las influencias y de las actividades espirituales del universo local y del superuniverso, con su serie casi interminable de personalidades amorosas que conducen siempre a los seres con intenciones sinceras y honrados de corazón hacia arriba y hacia dentro, hacia los ideales de la divinidad y la meta de la perfección suprema.

\par
%\textsuperscript{(100.7)}
\textsuperscript{9:2.5} \textit{Conocemos} la presencia del espíritu universal del Hijo Eterno ---podemos reconocerla de manera inequívoca. Incluso el hombre mortal puede conocer la presencia del Espíritu Infinito, la Tercera Persona de la Deidad, porque las criaturas materiales pueden experimentar realmente la beneficencia de esta influencia divina que actúa bajo la forma del Espíritu Santo del universo local que es otorgado a las razas de la humanidad. Los seres humanos también pueden volverse conscientes en cierta medida del Ajustador, la presencia impersonal del Padre Universal. Todos estos espíritus divinos que trabajan por la elevación y la espiritualización del hombre actúan al unísono y en perfecta cooperación. Se comportan como uno solo en la aplicación espiritual de los planes para que los mortales asciendan y alcancen la perfección.

\section*{3. El Manipulador Universal}
\par
%\textsuperscript{(101.1)}
\textsuperscript{9:3.1} La Isla del Paraíso es la fuente y la sustancia de la gravedad física; y esto debería ser suficiente para informaros de que la gravedad es una de las cosas más \textit{reales} y eternamente fiables en todo el universo de universos físico. La gravedad no se puede modificar ni anular, excepto por parte de las fuerzas y energías patrocinadas conjuntamente por el Padre y el Hijo, las cuales han sido confiadas a la persona de la Fuente-Centro Tercera, con el que están funcionalmente asociadas.

\par
%\textsuperscript{(101.2)}
\textsuperscript{9:3.2} El Espíritu Infinito posee un poder único y asombroso ---la \textit{antigravedad.} Este poder no está presente de manera funcional (observable) ni en el Padre ni en el Hijo. Esta capacidad inherente a la Fuente Tercera de resistir a la atracción de la gravedad material se revela en las reacciones personales del Actor Conjunto ante ciertas fases de las relaciones universales. Y este atributo único es transmisible a algunas personalidades superiores del Espíritu Infinito.

\par
%\textsuperscript{(101.3)}
\textsuperscript{9:3.3} La antigravedad puede anular la gravedad dentro de un marco local; lo hace mediante el ejercicio de una presencia de fuerza equivalente. Sólo funciona con relación a la gravedad material, y no es una acción de la mente. El fenómeno de un giroscopio resistiéndose a la gravedad es un buen ejemplo del \textit{efecto} de la antigravedad, pero no sirve para ilustrar la \textit{causa} de la antigravedad.

\par
%\textsuperscript{(101.4)}
\textsuperscript{9:3.4} El Actor Conjunto muestra además otros poderes que pueden trascender la fuerza y neutralizar la energía. Estos poderes funcionan aminorando la velocidad de la energía hasta el punto de la materialización, y mediante otras técnicas desconocidas por vosotros.

\par
%\textsuperscript{(101.5)}
\textsuperscript{9:3.5} El Creador Conjunto no es la energía, ni la fuente de la energía, ni el destino de la energía; es el \textit{manipulador} de la energía. El Creador Conjunto es acción ---movimiento, cambio, modificación, coordinación, estabilización y equilibrio. Las energías sometidas al control directo o indirecto del Paraíso son sensibles por naturaleza a los actos de la Fuente-Centro Tercera y de sus múltiples agentes.

\par
%\textsuperscript{(101.6)}
\textsuperscript{9:3.6} El universo de universos está penetrado por las criaturas de la Fuente-Centro Tercera que controlan el poder: controladores físicos, directores del poder, centros del poder y otros representantes del Dios de Acción que tienen que ver con la regulación y la estabilización de las energías físicas. Todas estas criaturas únicas en cuanto a sus funciones físicas poseen atributos variables para controlar el poder, tales como la antigravedad, que utilizan en sus esfuerzos por establecer el equilibrio físico de la materia y de las energías del gran universo.

\par
%\textsuperscript{(101.7)}
\textsuperscript{9:3.7} Todas estas actividades materiales del Dios de Acción parecen relacionar su obra con la Isla del Paraíso, y en verdad todos los agentes encargados del poder son respetuosos con la absolutidad de la Isla eterna, e incluso dependen de ésta. Pero el Actor Conjunto no actúa por el Paraíso ni en respuesta al Paraíso. Actúa personalmente por el Padre y el Hijo. El Paraíso no es una persona. Todas las actividades no personales, impersonales y distintas a las no personales de la Fuente-Centro Tercera son actos volitivos del Actor Conjunto mismo; no son reflejos, derivaciones ni repercusiones de nada ni de nadie.

\par
%\textsuperscript{(101.8)}
\textsuperscript{9:3.8} El Paraíso es el arquetipo de la infinidad; el Dios de Acción es el activador de ese arquetipo. El Paraíso es el punto de apoyo material de la infinidad; los agentes de la Fuente-Centro Tercera son las palancas inteligentes que motivan el nivel material e inyectan la espontaneidad en el mecanismo de la creación física.

\section*{4. La mente absoluta}
\par
%\textsuperscript{(102.1)}
\textsuperscript{9:4.1} La Fuente-Centro Tercera posee una naturaleza intelectual que es distinta de sus atributos físicos y espirituales. Es difícil ponerse en contacto con esta naturaleza, pero ésta es asociable ---intelectualmente, aunque no de manera personal. En los niveles donde funciona la mente, se la puede distinguir de los atributos físicos y del carácter espiritual de la Tercera Persona, pero para las personalidades que tratan de discernirla, esta naturaleza no actúa nunca independientemente de las manifestaciones físicas o espirituales.

\par
%\textsuperscript{(102.2)}
\textsuperscript{9:4.2} La mente absoluta es la mente de la Tercera Persona; es inseparable de la personalidad de Dios Espíritu. En los seres que desempeñan su actividad, la mente no está separada de la energía o del espíritu, o de los dos. La mente no es inherente a la energía; la energía es receptiva y sensible a la mente; la mente puede ser superpuesta a la energía, pero la conciencia no es inherente al nivel puramente material. No es preciso que la mente sea añadida al espíritu puro, porque el espíritu es consciente de manera innata y capaz de identificar. El espíritu es siempre inteligente, de alguna forma está dotado de \textit{mente.} Puede tratarse de este o de aquel tipo de mente, puede tratarse de una premente o de una supermente, e incluso de una mente espiritual, pero la facultad en cuestión equivale a pensar y a conocer. La perspicacia del espíritu trasciende, sobreviene y es teóricamente anterior a la conciencia de la mente.

\par
%\textsuperscript{(102.3)}
\textsuperscript{9:4.3} El Creador Conjunto sólo es absoluto en el ámbito de la mente, en el terreno de la inteligencia universal. La mente de la Fuente-Centro Tercera es infinita; trasciende por completo los circuitos mentales activos y funcionales del universo de universos. La dotación mental de los siete superuniversos procede de los Siete Espíritus Maestros, las personalidades primarias del Creador Conjunto. Estos Espíritus Maestros distribuyen la mente por el gran universo bajo la forma de mente cósmica, y vuestro universo local está impregnado de la variante nebadónica del tipo de mente cósmica de Orvonton.

\par
%\textsuperscript{(102.4)}
\textsuperscript{9:4.4} La mente infinita ignora el tiempo, la mente última trasciende el tiempo, la mente cósmica está condicionada por el tiempo. Y lo mismo sucede con el espacio: la Mente Infinita es independiente del espacio, pero a medida que se desciende desde el nivel infinito hasta los niveles de los ayudantes de la mente, el intelecto debe tener cada vez más en cuenta el hecho y las limitaciones del espacio.

\par
%\textsuperscript{(102.5)}
\textsuperscript{9:4.5} La fuerza cósmica reacciona a la mente al igual que la mente cósmica reacciona al espíritu. El espíritu es el propósito divino, y la mente espiritual es el propósito divino en acción. La energía es una cosa, la mente es un significado, el espíritu es un valor. Incluso en el tiempo y el espacio, la mente establece esas relaciones relativas entre la energía y el espíritu que sugieren su parentesco mutuo en la eternidad.

\par
%\textsuperscript{(102.6)}
\textsuperscript{9:4.6} La mente transmuta los valores del espíritu en los significados del intelecto; la volición tiene el poder de hacer que los significados de la mente fructifiquen tanto en los dominios materiales como en los espirituales. La ascensión al Paraíso implica un crecimiento relativo y diferencial en espíritu, mente y energía. La personalidad es la unificadora de estos componentes de la individualidad experiencial.

\section*{5. El ministerio de la mente}
\par
%\textsuperscript{(102.7)}
\textsuperscript{9:5.1} La mente de la Fuente-Centro Tercera es infinita. Si el universo tuviera que crecer hasta la infinidad, su potencial mental continuaría siendo adecuado para dotar a un número ilimitado de criaturas de una mente apropiada y de otros requisitos previos del intelecto.

\par
%\textsuperscript{(102.8)}
\textsuperscript{9:5.2} En el ámbito de la \textit{mente creada,} la Tercera Persona, con sus asociados coordinados y subordinados, gobierna de manera suprema. El campo de la mente de las criaturas tiene su origen exclusivo en la Fuente-Centro Tercera; él es el que concede la mente. Incluso a los fragmentos del Padre les resulta imposible habitar la mente de los hombres hasta que el camino no ha sido debidamente preparado para ellos mediante la acción mental y la actividad espiritual del Espíritu Infinito.

\par
%\textsuperscript{(103.1)}
\textsuperscript{9:5.3} La característica excepcional de la mente es que puede ser conferida a una gran variedad de vida. A través de sus creadores y de sus criaturas asociadas, la Fuente-Centro Tercera aporta su ministerio a todas las mentes en todas las esferas. Aporta su ministerio a los intelectos humanos y subhumanos a través de los ayudantes de los universos locales y, por mediación de los controladores físicos, aporta incluso su ministerio a las entidades más inferiores de los tipos más primitivos de seres vivos incapaces de experimentar. La dirección de la mente es siempre un ministerio de las personalidades dotadas de una mente asociada al espíritu o de una mente asociada a la energía.

\par
%\textsuperscript{(103.2)}
\textsuperscript{9:5.4} Puesto que la Tercera Persona de la Deidad es la fuente de la mente\footnote{\textit{La mente del Espíritu}: Ro 8:27; 11:34; 1 Co 2:16; Ef 4:23; Flp 2:5.}, es perfectamente natural que a las criaturas volitivas evolutivas les resulte más fácil formarse unos conceptos comprensibles sobre el Espíritu Infinito que sobre el Hijo Eterno o el Padre Universal. La realidad del Creador Conjunto se revela imperfectamente en la existencia misma de la mente humana. El Creador Conjunto es el antecesor de la mente cósmica, y la mente del hombre es un circuito individualizado, una porción impersonal, de esa mente cósmica tal como es otorgada en un universo local por una Hija Creativa de la Fuente-Centro Tercera.

\par
%\textsuperscript{(103.3)}
\textsuperscript{9:5.5} Puesto que la Tercera Persona es la fuente de la mente, no os atreváis a suponer que todos los fenómenos mentales son divinos. El intelecto humano está enraizado en el origen material de las razas animales. La inteligencia en el universo no es una verdadera revelación de Dios, que es mente, como la naturaleza física tampoco es una verdadera revelación de la belleza y la armonía del Paraíso. La perfección está en la naturaleza, pero la naturaleza no es perfecta. El Creador Conjunto es la fuente de la mente, pero la mente no es el Creador Conjunto.

\par
%\textsuperscript{(103.4)}
\textsuperscript{9:5.6} En Urantia, la mente es un término medio entre la esencia de la perfección del pensamiento y la mentalidad evolutiva de vuestra naturaleza humana inmadura. El plan concebido para vuestra evolución intelectual es en verdad de una perfección sublime, pero estáis muy lejos de esa meta divina mientras ejercéis vuestra actividad en el tabernáculo de la carne. La mente es realmente de origen divino, y tiene de hecho un destino divino, pero vuestra mente humana no tiene todavía una dignidad divina.

\par
%\textsuperscript{(103.5)}
\textsuperscript{9:5.7} Muy a menudo, demasiado a menudo, desfiguráis vuestra mente con la falta de sinceridad y la marchitáis con la injusticia; la sometéis al miedo animal y la desvirtuáis con ansiedades inútiles. Por lo tanto, aunque la fuente de la mente sea divina, la mente, tal como la conocéis en vuestro mundo ascensional, difícilmente puede convertirse en el objeto de una gran admiración, y mucho menos de adoración o de culto. La contemplación del intelecto humano inmaduro e inactivo sólo debería conducir a reacciones de humildad.

\section*{6. El circuito de la gravedad mental}
\par
%\textsuperscript{(103.6)}
\textsuperscript{9:6.1} La Fuente-Centro Tercera, la inteligencia universal, es personalmente consciente de cada \textit{mente}\footnote{\textit{Dios consciente de cada mente}: Sab 1:6-7.}, de cada intelecto, en toda la creación, y mantiene un contacto personal y perfecto con todas estas criaturas físicas, morontiales y espirituales dotadas de mente en los extensos universos. Todas estas actividades mentales están incluidas en el circuito absoluto de la gravedad mental que se encuentra focalizado en la Fuente-Centro Tercera y que forma parte de la conciencia personal del Espíritu Infinito.

\par
%\textsuperscript{(103.7)}
\textsuperscript{9:6.2} Al igual que el Padre tira de todas las personalidades hacia él, y que el Hijo atrae toda la realidad espiritual, el Actor Conjunto ejerce un poder de atracción sobre todas las mentes\footnote{\textit{Atracción de la mente}: Jer 31:3; Jn 6:44; 12:32.}; domina y controla sin restricción el circuito mental universal. Todos los valores intelectuales auténticos y verdaderos, todos los pensamientos divinos y todas las ideas perfectas son infaliblemente atraídos hacia este circuito absoluto de la mente.

\par
%\textsuperscript{(104.1)}
\textsuperscript{9:6.3} La gravedad mental puede funcionar independientemente de la gravedad material y de la espiritual, pero en cualquier momento y lugar en que estas dos últimas entran en contacto, la gravedad mental funciona siempre. Cuando las tres están asociadas, la gravedad de la personalidad puede abrazar a la criatura material ---física o morontial, finita o absonita. Pero independientemente de esto, el don de la mente, incluso a los seres impersonales, los capacita para pensar y los dota de conciencia a pesar de la ausencia total de personalidad.

\par
%\textsuperscript{(104.2)}
\textsuperscript{9:6.4} Sin embargo, la individualidad con dignidad de personalidad, humana o divina, inmortal o potencialmente inmortal, no tiene su origen ni en el espíritu, ni en la mente ni en la materia; es el don del Padre Universal. La interacción de la gravedad espiritual, mental y material tampoco es un requisito previo para la aparición de la gravedad de la personalidad. El circuito del Padre puede abrazar a un ser mental-material que es insensible a la gravedad espiritual, o puede incluir a un ser mental-espiritual que es insensible a la gravedad material. El funcionamiento de la gravedad de la personalidad es siempre un acto volitivo del Padre Universal.

\par
%\textsuperscript{(104.3)}
\textsuperscript{9:6.5} Aunque la mente está asociada a la energía en los seres puramente materiales, y asociada al espíritu en las personalidades puramente espirituales, innumerables órdenes de personalidades, incluyendo a los humanos, poseen una mente que está asociada tanto a la energía como al espíritu. Los aspectos espirituales de la mente de las criaturas reaccionan infaliblemente a la atracción de la gravedad espiritual del Hijo Eterno; las formas materiales reaccionan al impulso gravitatorio del universo material.

\par
%\textsuperscript{(104.4)}
\textsuperscript{9:6.6} Cuando la mente cósmica no está asociada ni a la energía ni al espíritu, tampoco está sometida a las exigencias gravitatorias de los circuitos materiales o espirituales. La mente pura sólo está sometida a la atracción gravitatoria universal del Actor Conjunto. La mente pura es la pariente más cercana de la mente infinita, y la mente infinita (la coordinada teórica de los absolutos del espíritu y de la energía) es aparentemente una ley en sí misma.

\par
%\textsuperscript{(104.5)}
\textsuperscript{9:6.7} Cuanto mayor es la divergencia entre el espíritu y la energía, mayor es la función observable de la mente; cuanto menor es la diversidad entre la energía y el espíritu, menor es la función observable de la mente. La función máxima de la mente cósmica se encuentra aparentemente en los universos temporales del espacio. La mente parece funcionar aquí en una zona intermedia entre la energía y el espíritu, pero esto no es cierto en lo que se refiere a los niveles superiores de la mente; en el Paraíso, la energía y el espíritu son esencialmente una sola cosa.

\par
%\textsuperscript{(104.6)}
\textsuperscript{9:6.8} El circuito de la gravedad mental es fiable; emana de la Tercera Persona de la Deidad del Paraíso, pero no toda la función observable de la mente es previsible. Paralelamente a este circuito mental, en toda la creación conocida existe una presencia poco comprendida cuya función no es previsible. Creemos que esta imprevisibilidad se puede atribuir en parte a la función del Absoluto Universal. No sabemos en qué consiste esta función; sólo podemos conjeturar qué es lo que la pone en movimiento; y en lo que concierne a su relación con las criaturas, sólo podemos especular.

\par
%\textsuperscript{(104.7)}
\textsuperscript{9:6.9} Algunas fases de la imprevisibilidad de la mente finita pueden deberse al estado incompleto del Ser Supremo, y existe una extensa zona de actividad donde el Actor Conjunto y el Absoluto Universal quizás pueden ser tangentes. Hay muchas cosas que se desconocen acerca de la mente, pero estamos seguros de esto: el Espíritu Infinito es la expresión perfecta de la mente del Creador para todas las criaturas; el Ser Supremo es la expresión evolutiva de las mentes de todas las criaturas para su Creador.

\section*{7. La reflectividad universal}
\par
%\textsuperscript{(105.1)}
\textsuperscript{9:7.1} El Actor Conjunto es capaz de coordinar todos los niveles de la realidad universal de tal manera que hace posible el reconocimiento simultáneo de lo mental, lo material y lo espiritual. Éste es el fenómeno de la \textit{reflectividaduniversal,} ese poder único e inexplicable para ver, oír, sentir y conocer todas las cosas a medida que suceden en todo un superuniverso, y luego focalizar por reflectividad toda esta información y todo este conocimiento en un punto deseado cualquiera. La acción de la reflectividad se manifiesta a la perfección en cada uno de los mundos sede de los siete superuniversos. Funciona también en todos los sectores de los superuniversos y dentro de las fronteras de los universos locales. La reflectividad se focaliza finalmente en el Paraíso.

\par
%\textsuperscript{(105.2)}
\textsuperscript{9:7.2} El fenómeno de la reflectividad, tal como se puede observar en las acciones asombrosas de las personalidades reflectantes estacionadas en los mundos sede de los superuniversos, representa la interasociación más compleja de todas las fases de existencia que se pueden encontrar en toda la creación. Las líneas del espíritu se pueden hacer remontar hasta el Hijo, la energía física hasta el Paraíso, y la mente hasta la Fuente Tercera; pero en el fenómeno extraordinario de la reflectividad universal existe una unificación única y excepcional de las tres, que están asociadas así para permitir que los gobernantes del universo conozcan instantáneamente las circunstancias lejanas en el momento mismo en que se producen.

\par
%\textsuperscript{(105.3)}
\textsuperscript{9:7.3} Comprendemos una gran parte de la técnica de la reflectividad, pero hay muchas fases que nos desconciertan realmente. Sabemos que el Actor Conjunto es el centro universal del circuito mental, que es el antecesor de la mente cósmica, y que la mente cósmica funciona bajo la dominación de la gravedad mental absoluta de la Fuente-Centro Tercera. Sabemos además que los circuitos de la mente cósmica influyen sobre los niveles intelectuales de todas las existencias conocidas; contienen las noticias universales del espacio, que están centradas con toda seguridad en los Siete Espíritus Maestros y convergen en la Fuente-Centro Tercera.

\par
%\textsuperscript{(105.4)}
\textsuperscript{9:7.4} La relación entre la mente cósmica finita y la mente divina absoluta parece estar evolucionando en la mente experiencial del Supremo. Se nos enseña que en los albores del tiempo, el Espíritu Infinito concedió esta mente experiencial al Supremo, y sospechamos que ciertas características del fenómeno de la reflectividad sólo se pueden explicar admitiendo la actividad de la Mente Suprema. Si el Supremo no está implicado en la reflectividad, no sabemos cómo explicar las complicadas actuaciones y las operaciones infalibles de esta conciencia del cosmos.

\par
%\textsuperscript{(105.5)}
\textsuperscript{9:7.5} La reflectividad parece ser la omnisciencia dentro de los límites de lo finito experiencial, y puede representar la aparición de la presencia-conciencia del Ser Supremo. Si esta suposición es cierta, entonces la utilización de la reflectividad en cualquiera de sus fases equivale a un contacto parcial con la conciencia del Supremo.

\section*{8. Las personalidades del Espíritu Infinito}
\par
%\textsuperscript{(105.6)}
\textsuperscript{9:8.1} El Espíritu Infinito posee el pleno poder de transmitir una gran parte de sus poderes y prerrogativas a sus personalidades y agentes coordinados y subordinados.

\par
%\textsuperscript{(105.7)}
\textsuperscript{9:8.2} El primer acto creativo del Espíritu Infinito como Deidad, actuando independientemente de la Trinidad pero asociado de alguna forma no revelada con el Padre y el Hijo, se personalizó en la existencia de los Siete Espíritus Maestros del Paraíso, los distribuidores del Espíritu Infinito para los universos.

\par
%\textsuperscript{(106.1)}
\textsuperscript{9:8.3} No existe ningún representante directo de la Fuente-Centro Tercera en la sede central de un superuniverso. Cada una de estas siete creaciones depende de uno de los Espíritus Maestros del Paraíso, que actúa a través de los siete Espíritus Reflectantes situados en la capital de cada superuniverso.

\par
%\textsuperscript{(106.2)}
\textsuperscript{9:8.4} La actividad creativa siguiente y continua del Espíritu Infinito se revela de vez en cuando en el acto de engendrar a los Espíritus Creativos. Cada vez que el Padre Universal y el Hijo Eterno se vuelven padres de un Hijo Creador, el Espíritu Infinito se convierte en el progenitor del Espíritu Creativo de un universo local, y dicho Espíritu se transforma en el íntimo asociado de ese Hijo Creador en toda la experiencia posterior de ese universo.

\par
%\textsuperscript{(106.3)}
\textsuperscript{9:8.5} Al igual que es necesario distinguir entre el Hijo Eterno y los Hijos Creadores, también es necesario diferenciar entre el Espíritu Infinito y los Espíritus Creativos, los coordinados de los Hijos Creadores en los universos locales. Un Espíritu Creativo representa para un universo local lo mismo que el Espíritu Infinito para la creación total.

\par
%\textsuperscript{(106.4)}
\textsuperscript{9:8.6} La Fuente-Centro Tercera está representada en el gran universo por una inmensa serie de espíritus ministrantes, mensajeros, educadores, jueces, ayudantes y consejeros, así como por los supervisores de ciertos circuitos de naturaleza física, morontial y espiritual. No todos estos seres son personalidades en el estricto sentido de la palabra. La personalidad perteneciente a la variedad de las criaturas finitas está caracterizada por:

\par
%\textsuperscript{(106.5)}
\textsuperscript{9:8.7} 1. La conciencia subjetiva de sí misma.

\par
%\textsuperscript{(106.6)}
\textsuperscript{9:8.8} 2. La reacción objetiva al circuito de personalidad del Padre.

\par
%\textsuperscript{(106.7)}
\textsuperscript{9:8.9} Existen personalidades de creadores y personalidades de criaturas, y además de estos dos tipos fundamentales, existen las \textit{personalidades de la Fuente-CentroTercera,} unos seres que son personales para el Espíritu Infinito, pero que no son incondicionalmente personales para las criaturas. Estas personalidades de la Fuente Tercera no forman parte del circuito de personalidad del Padre. Las personalidades de la Fuente Primera y las personalidades de la Fuente Tercera pueden ponerse mutuamente en contacto; toda personalidad es contactable.

\par
%\textsuperscript{(106.8)}
\textsuperscript{9:8.10} El Padre concede la personalidad por su libre albedrío personal. Sólo podemos conjeturar por qué lo hace, y no sabemos cómo lo hace. Tampoco sabemos por qué la Fuente Tercera confiere la personalidad no procedente del Padre, pero el Espíritu Infinito hace esto en su propio nombre, en conjunción creativa con el Hijo Eterno y de numerosas maneras desconocidas para vosotros. El Espíritu Infinito puede actuar también por el Padre para conceder la personalidad de tipo Fuente Primera.

\par
%\textsuperscript{(106.9)}
\textsuperscript{9:8.11} Existen numerosos tipos de personalidades procedentes de la Fuente Tercera. El Espíritu Infinito concede la personalidad de tipo Fuente Tercera a numerosos grupos que no están incluidos en el circuito de personalidad del Padre, tales como algunos directores del poder. El Espíritu Infinito trata igualmente como personalidades a numerosos grupos de seres, tales como los Espíritus Creativos, que componen una clase por sí mismos en sus relaciones con las criaturas incluidas en el circuito del Padre.

\par
%\textsuperscript{(106.10)}
\textsuperscript{9:8.12} Tanto las personalidades de la Fuente Primera como las de la Fuente Tercera están dotadas de todo aquello que el hombre asocia con el concepto de la personalidad, e incluso de más aún; poseen una mente que abarca la memoria, la razón, el juicio, la imaginación creativa, la asociación de ideas, la decisión, la elección y numerosos poderes intelectuales adicionales totalmente desconocidos por los mortales. Con pocas excepciones, las órdenes que os han sido reveladas poseen una forma y una individualidad bien determinada; son seres reales. La mayoría de ellos son visibles para todas las órdenes de espíritus existentes.

\par
%\textsuperscript{(107.1)}
\textsuperscript{9:8.13} Incluso vosotros seréis capaces de ver a vuestros asociados espirituales de las órdenes inferiores tan pronto como seáis liberados de la visión limitada de vuestros ojos materiales actuales, y hayáis sido dotados de una forma morontial con su mayor sensibilidad a la realidad de las cosas espirituales.

\par
%\textsuperscript{(107.2)}
\textsuperscript{9:8.14} \textit{La familia funcional de la Fuente-Centro Tercera,} tal como está revelada en estas narraciones, se divide en tres grandes grupos:

\par
%\textsuperscript{(107.3)}
\textsuperscript{9:8.15} I. \textit{Los Espíritus Supremos.} Un grupo de origen compuesto que abarca, entre otras, a las órdenes siguientes:

\par
%\textsuperscript{(107.4)}
\textsuperscript{9:8.16} 1. Los Siete Espíritus Maestros del Paraíso.

\par
%\textsuperscript{(107.5)}
\textsuperscript{9:8.17} 2. Los Espíritus Reflectantes de los Superuniversos.

\par
%\textsuperscript{(107.6)}
\textsuperscript{9:8.18} 3. Los Espíritus Creativos de los Universos Locales.

\par
%\textsuperscript{(107.7)}
\textsuperscript{9:8.19} II. \textit{Los Directores del Poder.} Un grupo de criaturas y de agentes de control que ejerce su actividad en todo el espacio organizado.

\par
%\textsuperscript{(107.8)}
\textsuperscript{9:8.20} III. \textit{Las Personalidades del Espíritu Infinito.} Esta designación no implica necesariamente que estos seres sean personalidades de la Fuente Tercera, aunque algunos de ellos son únicos como criaturas volitivas. Habitualmente están agrupados en tres clasificaciones principales:

\par
%\textsuperscript{(107.9)}
\textsuperscript{9:8.21} 1. Las Personalidades Superiores del Espíritu Infinito.

\par
%\textsuperscript{(107.10)}
\textsuperscript{9:8.22} 2. Las Huestes de Mensajeros del Espacio.

\par
%\textsuperscript{(107.11)}
\textsuperscript{9:8.23} 3. Los Espíritus Ministrantes del Tiempo.

\par
%\textsuperscript{(107.12)}
\textsuperscript{9:8.24} Estos grupos sirven en el Paraíso, en el universo central o residencial y en los superuniversos, y engloban a las órdenes que ejercen su actividad en los universos locales, e incluso en las constelaciones, los sistemas y los planetas.

\par
%\textsuperscript{(107.13)}
\textsuperscript{9:8.25} Las personalidades espirituales de la inmensa familia del Espíritu Divino e Infinito están dedicadas para siempre al servicio del ministerio del amor de Dios y de la misericordia del Hijo hacia todas las criaturas inteligentes de los mundos evolutivos del tiempo y del espacio. Estos seres espirituales constituyen la escala viviente por la que el hombre mortal se eleva desde el caos hasta la gloria.

\par
%\textsuperscript{(107.14)}
\textsuperscript{9:8.26} [Revelado en Urantia por un Consejero Divino de Uversa, encargado por los Ancianos de los Días para describir la naturaleza y el trabajo del Espíritu Infinito.]

\chapter{Documento 10. La Trinidad del Paraíso}
\par
%\textsuperscript{(108.1)}
\textsuperscript{10:0.1} LA Trinidad Paradisiaca de las Deidades eternas facilita que el Padre pueda liberarse del absolutismo de la personalidad. La Trinidad asocia perfectamente la expresión ilimitada de la voluntad personal infinita de Dios con la absolutidad de la Deidad. El Hijo Eterno y los diversos Hijos de origen divino, junto con el Actor Conjunto y sus hijos universales, facilitan eficazmente que el Padre pueda liberarse de las limitaciones por lo demás inherentes a la primacía, la perfección, la invariabilidad, la eternidad, la universalidad, la absolutidad y la infinidad.

\par
%\textsuperscript{(108.2)}
\textsuperscript{10:0.2} La Trinidad del Paraíso\footnote{\textit{La Trinidad del Paraíso}: Mt 28:19; Hch 2:32-33; 2 Co 13:14; 1 Jn 5:7.} asegura eficazmente la plena expresión y la revelación perfecta de la naturaleza eterna de la Deidad. Los Hijos Estacionarios de la Trinidad proporcionan igualmente una revelación plena y perfecta de la justicia divina. La Trinidad es la unidad de la Deidad, y esta unidad descansa eternamente sobre los fundamentos absolutos de la unidad divina de las tres personalidades originales, coordinadas y coexistentes: Dios Padre, Dios Hijo y Dios Espíritu\footnote{\textit{Trinidad del Paraíso (primitiva visión de Pablo)}: 1 Co 12:4-6.}.

\par
%\textsuperscript{(108.3)}
\textsuperscript{10:0.3} Partiendo de la situación presente en el círculo de la eternidad, y mirando hacia atrás en el pasado interminable, sólo podemos descubrir una inevitabilidad ineludible en los asuntos del universo, y es la Trinidad del Paraíso. Creo que la Trinidad era inevitable. Cuando examino el pasado, el presente y el futuro del tiempo, considero que ninguna otra cosa en todo el universo de universos era inevitable. El universo maestro actual, visto en retrospectiva o en perspectiva, es impensable sin la Trinidad. Con la Trinidad del Paraíso, podemos admitir maneras alternativas o incluso formas múltiples de hacer todas las cosas, pero sin la Trinidad del Padre, el Hijo y el Espíritu somos incapaces de concebir cómo el Infinito podría lograr una personalización triple y coordinada ante la unidad absoluta de la Deidad. Ningún otro concepto de la creación está a la altura de los niveles de la Trinidad, donde el estado completo de la absolutidad inherente a la unidad de la Deidad está unido a la plenitud de la liberación volitiva inherente a la personalización triple de la Deidad.

\section*{1. La autodistribución de la Fuente-Centro Primera}
\par
%\textsuperscript{(108.4)}
\textsuperscript{10:1.1} Parece ser que el Padre, allá por la eternidad, inauguró una política de profunda distribución de sí mismo. Hay algo inherente a la naturaleza desinteresada, amorosa y adorable del Padre Universal que le induce a reservarse solamente el ejercicio de aquellos poderes y de aquella autoridad que al parecer le resulta imposible delegar o conceder.

\par
%\textsuperscript{(108.5)}
\textsuperscript{10:1.2} El Padre Universal se ha despojado desde el principio de todas las parcelas de sí mismo que podía conferir a cualquier otro Creador o criatura. Ha delegado en sus Hijos divinos y en las inteligencias asociadas a ellos todo el poder y toda la autoridad que se podía delegar. Ha transferido realmente a sus Hijos Soberanos, en sus universos respectivos, todas las prerrogativas de autoridad administrativa que eran transferibles. En los asuntos de un universo local ha hecho a cada Hijo Creador Soberano tan perfecto, competente y con autoridad como el Hijo Eterno lo es en el universo central y original. Junto con la dignidad y la santidad que supone la posesión de la personalidad, ha distribuido, ha dado realmente todo de sí mismo y todos sus atributos, todas las cosas de las que posiblemente podía despojarse, de todas las maneras, en todas las épocas, en todos los lugares, a todas las personas y en todos los universos, salvo en el de su residencia central.

\par
%\textsuperscript{(109.1)}
\textsuperscript{10:1.3} La personalidad divina no es egocéntrica; la distribución de sí misma y el compartir la personalidad caracterizan la individualidad divina con libre albedrío. Las criaturas anhelan asociarse con otras criaturas personales; los Creadores se sienten inducidos a compartir la divinidad con sus hijos del universo; la personalidad del Infinito se revela bajo la forma de Padre Universal, el cual comparte la realidad de su ser y la igualdad de su yo con dos personalidades coordinadas, el Hijo Eterno y el Actor Conjunto.

\par
%\textsuperscript{(109.2)}
\textsuperscript{10:1.4} Para conocer la personalidad del Padre y sus atributos divinos, siempre dependeremos de las revelaciones del Hijo Eterno, porque cuando el acto conjunto de creación se llevó a cabo, cuando la Tercera Persona de la Deidad surgió a la existencia como personalidad y ejecutó los conceptos combinados de sus padres divinos, el Padre dejó de existir como personalidad incalificada. Con la aparición del Actor Conjunto y la materialización del núcleo central de la creación, tuvieron lugar ciertos cambios eternos. Dios se dio como personalidad absoluta a su Hijo Eterno. Así es como el Padre concede la <<\textit{personalidad de la infinidad}>> a su Hijo unigénito, mientras que los dos otorgan la <<\textit{personalidad conjunta}>> de su unión eterna al Espíritu Infinito.

\par
%\textsuperscript{(109.3)}
\textsuperscript{10:1.5} Por estas y otras razones que sobrepasan los conceptos de la mente finita, a las criaturas humanas les resulta extremadamente difícil comprender la infinita personalidad paternal de Dios, excepto tal como está revelada universalmente en el Hijo Eterno y, con el Hijo, es universalmente activa en el Espíritu Infinito.

\par
%\textsuperscript{(109.4)}
\textsuperscript{10:1.6} Puesto que los Hijos Paradisiacos de Dios\footnote{\textit{Hijos Paradisíacos}: Mt 11:27; Lc 10:22; Jn 1:14,18; 6:45-46; 8:26; 12:49-50; 14:7-11,20; 17:6,25-26.} visitan los mundos evolutivos y a veces incluso residen en ellos en la similitud de la carne mortal, y puesto que estas donaciones hacen posible que el hombre mortal pueda conocer realmente algo de la naturaleza y del carácter de la personalidad divina, las criaturas de las esferas planetarias deben recurrir pues a las donaciones de estos Hijos Paradisiacos para obtener una información segura y digna de confianza sobre el Padre, el Hijo y el Espíritu.

\section*{2. La personalización de la Deidad}
\par
%\textsuperscript{(109.5)}
\textsuperscript{10:2.1} El Padre se despoja, mediante la técnica de la trinitización, de esa personalidad espiritual incalificada que es el Hijo, pero al hacerlo, se constituye como Padre de este mismo Hijo, teniendo así la capacidad ilimitada de convertirse en el Padre divino de todos los tipos de criaturas volitivas inteligentes posteriormente creadas, existenciadas o personalizadas de otra manera. Como \textit{personalidadabsoluta e incalificada,} el Padre sólo puede actuar bajo la forma del Hijo y con el Hijo, pero como \textit{Padre personal,} continúa concediendo la personalidad a las multitudes diversas de los diferentes niveles de criaturas volitivas inteligentes, y mantiene para siempre unas relaciones personales de asociación amorosa con esta inmensa familia de hijos universales.

\par
%\textsuperscript{(109.6)}
\textsuperscript{10:2.2} Después de que el Padre hubo donado la plenitud de sí mismo a la personalidad de su Hijo, y cuando este acto de donación de sí mismo fue completo y perfecto, los asociados eternos recurrieron a la naturaleza y al poder infinitos que existen así en la unión Padre-Hijo, y confirieron conjuntamente las cualidades y los atributos que formaron a otro ser parecido a ellos; esta personalidad conjunta, el Espíritu Infinito, completa la personalización existencial de la Deidad.

\par
%\textsuperscript{(110.1)}
\textsuperscript{10:2.3} El Hijo es indispensable para la paternidad de Dios. El Espíritu es indispensable para la fraternidad entre la Segunda y la Tercera Personas. Tres personas forman un grupo social mínimo, pero ésta es la menor de todas las múltiples razones para creer en la inevitabilidad del Actor Conjunto.

\par
%\textsuperscript{(110.2)}
\textsuperscript{10:2.4} La Fuente-Centro Primera es la \textit{personalidad-padre} infinita, la personalidad original ilimitada. El Hijo Eterno es el \textit{absoluto-personalidad} incalificado, ese ser divino que permanece a través de todos los tiempos y de la eternidad como la revelación perfecta de la naturaleza personal de Dios. El Espíritu Infinito es la \textit{personalidad conjunta,} la consecuencia personal única de la unión perpetua entre el Padre y el Hijo.

\par
%\textsuperscript{(110.3)}
\textsuperscript{10:2.5} La personalidad de la Fuente-Centro Primera es la personalidad de la infinidad menos la personalidad absoluta del Hijo Eterno. La personalidad de la Fuente-Centro Tercera es la consecuencia sobreañadida de la unión entre la personalidad liberada del Padre y la personalidad absoluta del Hijo.

\par
%\textsuperscript{(110.4)}
\textsuperscript{10:2.6} El Padre Universal, el Hijo Eterno y el Espíritu Infinito son personas únicas; ninguno de ellos es una copia; cada cual es original; todos están unidos.

\par
%\textsuperscript{(110.5)}
\textsuperscript{10:2.7} Únicamente el Hijo Eterno experimenta la plenitud de las relaciones divinas de la personalidad, la conciencia tanto de su filiación con el Padre como de su paternidad con respecto al Espíritu, y su igualdad divina tanto con el Padre antecesor como con el Espíritu asociado. El Padre conoce la experiencia de tener un Hijo que es igual a él, pero el Padre no conoce antecedentes ancestrales. El Hijo Eterno tiene la experiencia de la filiación, el reconocimiento de un progenitor de su personalidad, y al mismo tiempo el Hijo es consciente de ser el padre conjunto del Espíritu Infinito. El Espíritu Infinito es consciente de la doble ascendencia de su personalidad, pero no es el padre de una personalidad coordinada de la Deidad. El ciclo existencial de la personalización de la Deidad alcanza su culminación con el Espíritu; las personalidades primarias de la Fuente-Centro Tercera son experienciales y su número es de siete.

\par
%\textsuperscript{(110.6)}
\textsuperscript{10:2.8} Tengo mi origen en la Trinidad del Paraíso. Conozco la Trinidad como Deidad unificada; sé también que el Padre, el Hijo y el Espíritu existen y actúan según sus capacidades personales definidas. Sé afirmativamente que no sólo actúan de manera personal y colectiva, sino que también coordinan sus acciones en diversas agrupaciones, de manera que al final ejercen su actividad en siete capacidades diferentes, individuales y plurales. Y puesto que estas siete asociaciones agotan las posibilidades de estas combinaciones de la divinidad, es inevitable que las realidades del universo aparezcan en siete variaciones de valores, de significados y de personalidad.

\section*{3. Las tres personas de la Deidad}
\par
%\textsuperscript{(110.7)}
\textsuperscript{10:3.1} A pesar de que hay una sola Deidad, existen tres personalizaciones verdaderas y divinas de la Deidad. En lo que se refiere a los Ajustadores divinos con los que los hombres han sido dotados, el Padre ha dicho: <<\textit{Hagamos al hombre mortal a nuestra propia imagen}>>\footnote{\textit{El hombre hecho a imagen de Dios}: Gn 1:26.}. Esta referencia a los actos y a las actividades de una Deidad plural aparece repetidas veces en todas las escrituras urantianas, mostrando claramente que se reconoce la existencia y el trabajo de las tres Fuentes y Centros.

\par
%\textsuperscript{(110.8)}
\textsuperscript{10:3.2} Nos enseñan que el Hijo y el Espíritu mantienen con el Padre unas relaciones idénticas de igualdad en la asociación de la Trinidad. En la eternidad y como Deidades lo hacen sin duda alguna, pero en el tiempo y como personalidades revelan ciertamente unas relaciones de naturaleza muy diversa. Mirando desde el Paraíso hacia los universos, estas relaciones parecen muy similares, pero cuando son observadas desde los dominios del espacio, parecen totalmente diferentes.

\par
%\textsuperscript{(111.1)}
\textsuperscript{10:3.3} Los Hijos divinos son en verdad el <<\textit{Verbo de Dios}>>\footnote{\textit{"Verbo" de Dios}: Jn 1:1-4,14.}, pero los hijos del Espíritu son verdaderamente el <<\textit{Acto de Dios}>>\footnote{\textit{Acto de Dios}: Gn 1:1ff.}. Dios habla a través del Hijo y, con el Hijo, actúa a través del Espíritu Infinito, mientras que en todas las actividades del universo, el Hijo y el Espíritu son exquisitamente fraternales y trabajan como dos hermanos iguales con admiración y amor por un Padre común venerado y divinamente respetado.

\par
%\textsuperscript{(111.2)}
\textsuperscript{10:3.4} El Padre, el Hijo y el Espíritu son ciertamente iguales en naturaleza, están coordinados en existencia, pero hay diferencias inequívocas en sus acciones universales, y cuando cada persona de la Deidad actúa sola, está aparentemente limitada en su absolutidad.

\par
%\textsuperscript{(111.3)}
\textsuperscript{10:3.5} Antes de despojarse voluntariamente de la personalidad, de los poderes y de los atributos que constituyen al Hijo y al Espíritu, el Padre Universal parece haber sido (considerado filosóficamente) una Deidad incalificada, absoluta e infinita. Pero esta Fuente-Centro Primera teórica sin un Hijo no podía ser considerada, en ningún sentido de la palabra, el \textit{Padre Universal;} la paternidad no es real sin filiación. Además, para que el Padre haya sido absoluto en un sentido total, debe haber existido solo en algún momento eternamente lejano. Pero nunca ha tenido esa existencia solitaria; tanto el Hijo como el Espíritu son coeternos con el Padre. La Fuente-Centro Primera ha sido siempre, y siempre será, el Padre eterno del Hijo Original y, con el Hijo, el progenitor eterno del Espíritu Infinito.

\par
%\textsuperscript{(111.4)}
\textsuperscript{10:3.6} Observamos que el Padre se ha despojado de todas las manifestaciones directas de su absolutidad, excepto de la paternidad absoluta y de la volición absoluta. No sabemos si la volición es un atributo inalienable del Padre; sólo podemos observar que \textit{no} se ha despojado de su volición. Esta infinidad de voluntad debe haber sido eternamente inherente a la Fuente-Centro Primera.

\par
%\textsuperscript{(111.5)}
\textsuperscript{10:3.7} Al concederle la absolutidad de la personalidad al Hijo Eterno, el Padre Universal se libera de las trabas del absolutismo de la personalidad, pero al hacer esto, toma una medida que le impide para siempre actuar solo como absoluto de la personalidad. Y con la personalización final de la Deidad coexistente ---el Actor Conjunto--- se produce la interdependencia trinitaria crítica de las tres personalidades divinas con relación al funcionamiento total de la Deidad en el sentido absoluto.

\par
%\textsuperscript{(111.6)}
\textsuperscript{10:3.8} Dios es el Absoluto-Padre de todas las personalidades del universo de universos. El Padre es personalmente absoluto en cuanto a su libertad de acción, pero en los universos del tiempo y del espacio ya creados, creándose y todavía por crearse, no se puede discernir que el Padre sea absoluto como Deidad total, salvo en la Trinidad del Paraíso.

\par
%\textsuperscript{(111.7)}
\textsuperscript{10:3.9} Fuera de Havona, la Fuente-Centro Primera ejerce su actividad en los universos fenoménicos de la manera siguiente:

\par
%\textsuperscript{(111.8)}
\textsuperscript{10:3.10} 1. Como creador, a través de los Hijos Creadores, sus nietos.

\par
%\textsuperscript{(111.9)}
\textsuperscript{10:3.11} 2. Como controlador, a través del centro de gravedad del Paraíso.

\par
%\textsuperscript{(111.10)}
\textsuperscript{10:3.12} 3. Como espíritu, a través del Hijo Eterno.

\par
%\textsuperscript{(111.11)}
\textsuperscript{10:3.13} 4. Como mente, a través del Creador Conjunto.

\par
%\textsuperscript{(111.12)}
\textsuperscript{10:3.14} 5. Como Padre, mantiene un contacto parental con todas las criaturas a través de su circuito de personalidad.

\par
%\textsuperscript{(111.13)}
\textsuperscript{10:3.15} 6. Como persona, actúa \textit{directamente} en toda la creación por medio de sus fragmentos exclusivos ---en el hombre mortal, mediante los Ajustadores del Pensamiento.

\par
%\textsuperscript{(111.14)}
\textsuperscript{10:3.16} 7. Como Deidad total, sólo ejerce su actividad en la Trinidad del Paraíso.

\par
%\textsuperscript{(112.1)}
\textsuperscript{10:3.17} Todas estas renuncias y delegaciones de jurisdicción por parte del Padre Universal son totalmente voluntarias y autoimpuestas. El Padre todopoderoso asume intencionalmente estas limitaciones de su autoridad en el universo.

\par
%\textsuperscript{(112.2)}
\textsuperscript{10:3.18} El Hijo Eterno parece actuar como uno solo con el Padre en todos los aspectos espirituales, salvo en la concesión de los fragmentos de Dios y en otras actividades prepersonales. El Hijo tampoco está íntimamente identificado con las actividades intelectuales de las criaturas materiales ni con las actividades energéticas de los universos materiales. Como absoluto, el Hijo ejerce su actividad como una persona y solamente en el ámbito del universo espiritual.

\par
%\textsuperscript{(112.3)}
\textsuperscript{10:3.19} El Espíritu Infinito es asombrosamente universal e increíblemente polifacético en todas sus operaciones. Actúa en las esferas de la mente, la materia y el espíritu. El Actor Conjunto representa la asociación Padre-Hijo, pero también actúa como él mismo. No está directamente relacionado con la gravedad física, la gravedad espiritual o el circuito de la personalidad, pero participa más o menos en todas las demás actividades del universo. Aunque depende aparentemente de tres controles gravitatorios existenciales y absolutos, el Espíritu Infinito parece ejercer tres supercontroles. Este triple don lo emplea de muchas maneras para trascender, y al parecer incluso para neutralizar, las manifestaciones de las fuerzas y de las energías primarias hasta las fronteras superúltimas de la absolutidad. En ciertas situaciones, estos supercontroles trascienden absolutamente incluso las manifestaciones primordiales de la realidad cósmica.

\section*{4. La unión trinitaria de la Deidad}
\par
%\textsuperscript{(112.4)}
\textsuperscript{10:4.1} De todas las asociaciones absolutas, la Trinidad del Paraíso (la primera triunidad) es única como asociación exclusiva de la Deidad personal. Dios sólo actúa como Dios con relación a Dios y a aquellos que pueden conocer a Dios, pero como Deidad absoluta sólo actúa en la Trinidad del Paraíso y con relación a la totalidad del universo.

\par
%\textsuperscript{(112.5)}
\textsuperscript{10:4.2} La Deidad eterna está perfectamente unificada; sin embargo, existen tres personas de la Deidad perfectamente individualizadas. La Trinidad del Paraíso hace posible la expresión simultánea de toda la diversidad de los rasgos de carácter y de los poderes infinitos de la Fuente-Centro Primera y sus eternos coordinados, y de toda la unidad divina de las funciones universales de la Deidad indivisa.

\par
%\textsuperscript{(112.6)}
\textsuperscript{10:4.3} La Trinidad es una asociación de personas infinitas que actúan en una capacidad no personal, pero sin estar en contra de la personalidad. El ejemplo es rudimentario, pero un padre, un hijo y un nieto podrían formar una entidad corporativa que sería no personal, pero que sin embargo estaría sujeta a sus voluntades personales.

\par
%\textsuperscript{(112.7)}
\textsuperscript{10:4.4} La Trinidad del Paraíso es \textit{real.} Existe como la unión del Padre, del Hijo y del Espíritu bajo la forma de Deidad; sin embargo, el Padre, el Hijo o el Espíritu, o dos cualquiera de ellos, pueden ejercer su actividad con relación a esta misma Trinidad del Paraíso. El Padre, el Hijo y el Espíritu pueden colaborar de una manera no trinitaria, pero no como tres Deidades. Como personas pueden colaborar como escojan hacerlo, pero eso no es la Trinidad.

\par
%\textsuperscript{(112.8)}
\textsuperscript{10:4.5} Recordad siempre que aquello que lleva a cabo el Espíritu Infinito es la ocupación del Actor Conjunto. Tanto el Padre como el Hijo ejercen su actividad en él, a través de él y como él. Pero sería inútil tratar de dilucidar el misterio de la Trinidad: tres como uno y en uno, y uno como dos y actuando por dos.

\par
%\textsuperscript{(112.9)}
\textsuperscript{10:4.6} La Trinidad está tan relacionada con los asuntos del universo total que debemos contar con ella cuando intentamos explicar la totalidad de cualquier acontecimiento cósmico o relación de personalidad aislados. La Trinidad ejerce su actividad en todos los niveles del cosmos, y el hombre mortal está limitado al nivel finito; por eso el hombre debe contentarse con un concepto finito de la Trinidad como Trinidad.

\par
%\textsuperscript{(113.1)}
\textsuperscript{10:4.7} Como mortales en la carne, deberíais contemplar la Trinidad según vuestras luces individuales y en armonía con las reacciones de vuestra mente y de vuestra alma. Podéis saber muy pocas cosas sobre la absolutidad de la Trinidad, pero a medida que ascendáis hacia el Paraíso, os asombraréis muchas veces ante las revelaciones sucesivas y los descubrimientos inesperados sobre la supremacía y la ultimidad, si no sobre la absolutidad, de la Trinidad.

\section*{5. Las funciones de la Trinidad}
\par
%\textsuperscript{(113.2)}
\textsuperscript{10:5.1} Las Deidades personales tienen atributos, pero no es muy coherente decir que la Trinidad tiene atributos. Se puede considerar con más propiedad que esta asociación de seres divinos tiene \textit{funciones,} tales como la administración de la justicia, las actitudes de totalidad, la acción coordinada y el supercontrol cósmico. Estas funciones son activamente supremas, últimas y (dentro de los límites de la Deidad) absolutas, en la medida en que conciernen a todas las realidades vivientes con valor de personalidad.

\par
%\textsuperscript{(113.3)}
\textsuperscript{10:5.2} Las funciones de la Trinidad del Paraíso no son simplemente la suma de la aparente dotación de divinidad del Padre, más aquellos atributos especializados que son únicos en la existencia personal del Hijo y del Espíritu. La asociación de las tres Deidades del Paraíso bajo la forma de Trinidad tiene como resultado la evolución, la existenciación y la divinización de unos nuevos significados, valores, poderes y capacidades para la revelación, la acción y la administración universales. Las asociaciones vivientes, las familias humanas, los grupos sociales o la Trinidad del Paraíso no aumentan mediante la simple suma aritmética. El potencial del grupo es siempre muy superior a la simple suma de los atributos de los individuos que lo componen.

\par
%\textsuperscript{(113.4)}
\textsuperscript{10:5.3} La Trinidad mantiene una actitud única, como Trinidad, hacia el universo total del pasado, del presente y del futuro. Y las funciones de la Trinidad se pueden examinar mejor en relación con las actitudes de la Trinidad hacia el universo. Dichas actitudes son simultáneas y pueden ser múltiples con respecto a cualquier situación o acontecimiento aislado:

\par
%\textsuperscript{(113.5)}
\textsuperscript{10:5.4} 1. \textit{Actitud hacia lo Finito.} La limitación máxima que la Trinidad se impone es su actitud hacia lo finito. La Trinidad no es una persona, ni el Ser Supremo es una personalización exclusiva de la Trinidad, pero el Supremo es la máxima aproximación a una focalización de la Trinidad, bajo la forma del poder más la personalidad, que pueden comprender las criaturas finitas. Por eso cuando se habla de la Trinidad en relación con lo finito, a veces se la califica de Trinidad de Supremacía.

\par
%\textsuperscript{(113.6)}
\textsuperscript{10:5.5} 2. \textit{Actitud hacia lo Absonito.} La Trinidad del Paraíso tiene consideración con aquellos niveles de existencia que son más que finitos pero menos que absolutos, y a esta relación se la denomina a veces Trinidad de Ultimidad. Ni el Último ni el Supremo representan totalmente a la Trinidad del Paraíso, pero en un sentido limitado y para sus niveles respectivos, cada uno de ellos parece representar a la Trinidad durante las eras prepersonales en que se desarrolla el poder experiencial.

\par
%\textsuperscript{(113.7)}
\textsuperscript{10:5.6} 3. \textit{La Actitud Absoluta} de la Trinidad del Paraíso está en relación con las existencias absolutas y culmina en la acción de la Deidad total.

\par
%\textsuperscript{(113.8)}
\textsuperscript{10:5.7} La Trinidad Infinita supone la acción coordinada de todas las relaciones triunitarias de la Fuente-Centro Primera ---no deificadas así como deificadas--- y por eso es muy difícil de captar por las personalidades. Al examinar la Trinidad como infinita, no olvidéis las siete triunidades; así se pueden evitar ciertas dificultades de comprensión, y algunas paradojas se pueden resolver parcialmente.

\par
%\textsuperscript{(114.1)}
\textsuperscript{10:5.8} Pero no dispongo de un lenguaje que me permita transmitir a la mente humana limitada la verdad completa y el significado eterno de la Trinidad del Paraíso, ni la naturaleza de la interasociación interminable de los tres seres infinitamente perfectos.

\section*{6. Los Hijos Estacionarios de la Trinidad}
\par
%\textsuperscript{(114.2)}
\textsuperscript{10:6.1} Toda ley tiene su origen en la Fuente-Centro Primera; \textit{él es la ley.} La administración de la ley espiritual es inherente a la Fuente-Centro Segunda. La revelación de la ley, la promulgación y la interpretación de los decretos divinos, es la ocupación de la Fuente-Centro Tercera. La aplicación de la ley, la justicia, es incumbencia de la Trinidad del Paraíso y es llevada a cabo por ciertos Hijos de la Trinidad.

\par
%\textsuperscript{(114.3)}
\textsuperscript{10:6.2} \textit{La justicia} es inherente a la soberanía universal de la Trinidad del Paraíso, pero la bondad, la misericordia y la verdad son el ministerio universal de las personalidades divinas, cuya unión en la Deidad constituye la Trinidad. La justicia no es la actitud del Padre, del Hijo o del Espíritu. La justicia es la actitud trinitaria de estas personalidades de amor, misericordia y ministerio. Ninguna de las Deidades del Paraíso promueve la administración de la justicia. La justicia no es nunca una actitud personal; siempre es una función plural.

\par
%\textsuperscript{(114.4)}
\textsuperscript{10:6.3} \textit{Las pruebas,} la base de la equidad (la justicia en armonía con la misericordia), son proporcionadas por las personalidades de la Fuente-Centro Tercera, el representante conjunto del Padre y del Hijo en todos los universos y para la mente de los seres inteligentes de toda la creación.

\par
%\textsuperscript{(114.5)}
\textsuperscript{10:6.4} \textit{El juicio,} la aplicación final de la justicia de acuerdo con las pruebas presentadas por las personalidades del Espíritu Infinito, es la tarea de los Hijos Estacionarios de la Trinidad, unos seres que comparten la naturaleza trinitaria del Padre, el Hijo y el Espíritu unidos.

\par
%\textsuperscript{(114.6)}
\textsuperscript{10:6.5} Este grupo de Hijos de la Trinidad abarca las personalidades siguientes:

\par
%\textsuperscript{(114.7)}
\textsuperscript{10:6.6} 1. Los Secretos Trinitizados de la Supremacía.

\par
%\textsuperscript{(114.8)}
\textsuperscript{10:6.7} 2. Los Eternos de los Días.

\par
%\textsuperscript{(114.9)}
\textsuperscript{10:6.8} 3. Los Ancianos de los Días.

\par
%\textsuperscript{(114.10)}
\textsuperscript{10:6.9} 4. Los Perfecciones de los Días.

\par
%\textsuperscript{(114.11)}
\textsuperscript{10:6.10} 5. Los Recientes de los Días.

\par
%\textsuperscript{(114.12)}
\textsuperscript{10:6.11} 6. Los Uniones de los Días.

\par
%\textsuperscript{(114.13)}
\textsuperscript{10:6.12} 7. Los Fieles de los Días.

\par
%\textsuperscript{(114.14)}
\textsuperscript{10:6.13} 8. Los Perfeccionadores de la Sabiduría.

\par
%\textsuperscript{(114.15)}
\textsuperscript{10:6.14} 9. Los Consejeros Divinos.

\par
%\textsuperscript{(114.16)}
\textsuperscript{10:6.15} 10. Los Censores Universales.

\par
%\textsuperscript{(114.17)}
\textsuperscript{10:6.16} Somos los hijos de las tres Deidades del Paraíso actuando como Trinidad, pues da la casualidad de que pertenezco a la décima orden de este grupo, los Censores Universales. Estas órdenes no representan la actitud de la Trinidad en un sentido universal; sólo representan esta actitud colectiva de la Deidad en el ámbito del juicio ejecutivo ---la justicia. Fueron concebidos específicamente por la Trinidad para el trabajo preciso al que están asignados, y sólo representan a la Trinidad en aquellas funciones para las que fueron personalizados.

\par
%\textsuperscript{(115.1)}
\textsuperscript{10:6.17} Los Ancianos de los Días y sus asociados de origen trinitario distribuyen el juicio justo de la equidad suprema a los siete superuniversos. En el universo central, estas funciones sólo existen en teoría; allí, la equidad es evidente en su perfección, y la perfección de Havona excluye toda posibilidad de falta de armonía.

\par
%\textsuperscript{(115.2)}
\textsuperscript{10:6.18} La justicia es la idea colectiva de la rectitud; la misericordia es su expresión personal. La misericordia es la actitud del amor; el funcionamiento de la ley está caracterizado por la precisión; el juicio divino es el alma de la equidad, conformándose siempre a la justicia de la Trinidad, satisfaciendo siempre el amor divino de Dios. Cuando la justicia recta de la Trinidad y el amor misericordioso del Padre Universal son percibidos plenamente y comprendidos por completo, coinciden. Pero el hombre no tiene esta plena comprensión de la justicia divina. Así pues, en la Trinidad, tal como el hombre la concibe, las personalidades del Padre, del Hijo y del Espíritu están ajustadas para coordinar el ministerio del amor y de la ley en los universos experienciales del tiempo.

\section*{7. El supercontrol de la Supremacía}
\par
%\textsuperscript{(115.3)}
\textsuperscript{10:7.1} La Primera, la Segunda y la Tercera Personas de la Deidad son iguales entre sí y forman una sola\footnote{\textit{Unidad de la Deidad}: 1 Jn 5:7.}. <<\textit{El Señor nuestro Dios es un solo Dios}>>\footnote{\textit{El Señor nuestro Dios es un solo Dios}: 2 Re 19:19; 1 Cr 17:20; Neh 9:6; Sal 86:10; Eclo 36:5; Is 37:16; 44:6,8; 45:5-6,21; Dt 4:35,39; 6:4; Mc 12:29,32; Jn 17:3; Ro 3:30; 1 Co 8:4-6; Gl 3:20; Ef 4:6; 1 Ti 2:5; Stg 2:19; 1 Sam 2:2; 2 Sam 7:22.}. Existe un propósito perfecto y una unidad de ejecución en la Trinidad divina de las Deidades eternas. El Padre, el Hijo y el Actor Conjunto son verdadera y divinamente uno solo. Se ha escrito en verdad: <<\textit{Yo soy el primero y el último, y fuera de mí no hay ningún Dios}>>\footnote{\textit{El primero, el último, y el único Dios}: Is 41:4; 44:6; 48:12; Ap 1:17; 2:8.} \footnote{\textit{El alpha y el omega}: Ap 1:8,11; 21:6; 22:13.}.

\par
%\textsuperscript{(115.4)}
\textsuperscript{10:7.2} Tal como las cosas aparecen para los mortales en el nivel finito, la Trinidad del Paraíso, al igual que el Ser Supremo, sólo se interesa por lo total ---planeta total, universo total, superuniverso total, gran universo total. Esta actitud de totalidad existe porque la Trinidad es el total de la Deidad, y por otras muchas razones.

\par
%\textsuperscript{(115.5)}
\textsuperscript{10:7.3} El Ser Supremo es algo menos que la Trinidad, y algo distinto a ella, ejerciendo su actividad en los universos finitos; pero dentro de ciertos límites, y durante la presente era en que la personalización y el poder están incompletos, esta Deidad evolutiva parece reflejar la actitud de la Trinidad de Supremacía. El Padre, el Hijo y el Espíritu no actúan personalmente con el Ser Supremo, pero durante la presente era del universo, colaboran con él como Trinidad. Comprendemos que mantienen una relación similar con el Último. A menudo conjeturamos sobre cuál será la relación personal entre las Deidades del Paraíso y Dios Supremo cuando este último haya finalizado su evolución, pero no lo sabemos realmente.

\par
%\textsuperscript{(115.6)}
\textsuperscript{10:7.4} Comprobamos que el supercontrol de la Supremacía no es totalmente previsible. Además, esta imprevisibilidad parece estar caracterizada por cierto estado incompleto de desarrollo, sin duda una marca distintiva del estado incompleto del Supremo y de la reacción finita incompleta a la Trinidad del Paraíso.

\par
%\textsuperscript{(115.7)}
\textsuperscript{10:7.5} La mente humana puede imaginar inmediatamente mil y una cosas ---acontecimientos físicos catastróficos, accidentes espantosos, desastres horribles, enfermedades dolorosas y plagas mundiales--- y preguntarse si estas calamidades están correlacionadas con las maniobras desconocidas de esta actividad probable del Ser Supremo. Francamente, no lo sabemos; no estamos realmente seguros. Pero sí observamos que a medida que pasa el tiempo, todas estas situaciones difíciles y más o menos misteriosas \textit{siempre} se resuelven para el bienestar y el progreso de los universos. Puede ser que la actividad del Supremo y el supercontrol de la Trinidad entremezclen todas las circunstancias de la existencia y todas las vicisitudes inexplicables de la vida en una configuración significativa de alto valor.

\par
%\textsuperscript{(116.1)}
\textsuperscript{10:7.6} Como hijos de Dios, podéis discernir la actitud personal de amor de Dios Padre en todos sus actos. Pero no siempre seréis capaces de comprender cuántos actos universales de la Trinidad del Paraíso redundan en beneficio de los mortales individuales en los mundos evolutivos del espacio. En el progreso de la eternidad, los actos de la Trinidad se revelarán como completamente significativos y considerados, pero no siempre aparecen así a las criaturas del tiempo.

\section*{8. La Trinidad más allá de lo finito}
\par
%\textsuperscript{(116.2)}
\textsuperscript{10:8.1} Muchas verdades y hechos relacionados con la Trinidad del Paraíso sólo se pueden comprender, aunque sea parcialmente, reconociendo una función que trasciende lo finito.

\par
%\textsuperscript{(116.3)}
\textsuperscript{10:8.2} Sería poco aconsejable hablar de las funciones de la Trinidad de Ultimidad, pero podemos revelar que Dios Último es la manifestación de la Trinidad tal como la comprenden los Trascendentales. Nos inclinamos a creer que la unificación del universo maestro es el acto existenciador del Último y refleja probablemente algunas fases, pero no todas, del supercontrol absonito de la Trinidad del Paraíso. El
Último es una manifestación limitada de la Trinidad en relación con lo absonito, pero sólo en el sentido en que el Supremo representa así parcialmente a la Trinidad en relación con lo finito.

\par
%\textsuperscript{(116.4)}
\textsuperscript{10:8.3} El Padre Universal, el Hijo Eterno y el Espíritu Infinito son en cierto sentido las personalidades que constituyen la Deidad total. Su unión en la Trinidad del Paraíso y la función absoluta de la Trinidad equivalen a las funciones de la Deidad total. Esta culminación de la Deidad trasciende tanto lo finito como lo absonito.

\par
%\textsuperscript{(116.5)}
\textsuperscript{10:8.4} Aunque ninguna persona individual de las Deidades del Paraíso llena realmente todo el potencial de la Deidad, colectivamente lo llenan las tres. Tres personas infinitas parecen ser el número mínimo de seres que se necesitan para activar el potencial prepersonal y existencial de la Deidad total ---del Absoluto de la Deidad.

\par
%\textsuperscript{(116.6)}
\textsuperscript{10:8.5} Conocemos al Padre Universal, al Hijo Eterno y al Espíritu Infinito como \textit{personas,} pero no conozco personalmente al Absoluto de la Deidad. Amo y adoro a Dios Padre; respeto y honro al Absoluto de la Deidad.

\par
%\textsuperscript{(116.7)}
\textsuperscript{10:8.6} Una vez residí en un universo donde cierto grupo de seres enseñaba que, en la eternidad, los finalitarios se convertirían finalmente en los hijos del Absoluto de la Deidad. Pero no estoy dispuesto a aceptar esta solución al misterio que envuelve al futuro de los finalitarios.

\par
%\textsuperscript{(116.8)}
\textsuperscript{10:8.7} El Cuerpo de la Finalidad engloba, entre otros, a aquellos mortales del tiempo y del espacio que han alcanzado la perfección en todo lo que se refiere a la voluntad de Dios. Como criaturas, y dentro de los límites de la capacidad de las criaturas, conocen plena y verdaderamente a Dios. Habiendo encontrado así a Dios como Padre de todas las criaturas, estos finalitarios deberán empezar algún día la búsqueda del Padre superfinito. Pero esta búsqueda implica que hay que captar la naturaleza absonita de los atributos y del carácter últimos del Padre Paradisiaco. La eternidad revelará si esta consecución es posible, pero estamos convencidos de que incluso si los finalitarios logran captar este estado último de la divinidad, probablemente serán incapaces de alcanzar los niveles superúltimos de la Deidad absoluta.

\par
%\textsuperscript{(116.9)}
\textsuperscript{10:8.8} Es posible que los finalitarios alcancen parcialmente al Absoluto de la Deidad, pero incluso si lo consiguen, el problema del Absoluto Universal continuará todavía, en la eternidad de las eternidades, intrigando, desorientando, desconcertando y desafiando a los finalitarios que asciendan y progresen, porque percibimos que las relaciones cósmicas insondables del Absoluto Universal tenderán a crecer en la proporción en que los universos materiales y su administración espiritual continúen expandiéndose.

\par
%\textsuperscript{(117.1)}
\textsuperscript{10:8.9} Sólo la infinidad puede revelar al Padre-Infinito.

\par
%\textsuperscript{(117.2)}
\textsuperscript{10:8.10} [Patrocinado por un Censor Universal que actúa por autorización de los Ancianos de los Días que residen en Uversa.]


\chapter{Documento 11. La Isla Eterna del Paraíso}
\par
%\textsuperscript{(118.1)}
\textsuperscript{11:0.1} EL Paraíso es el centro eterno del universo de universos y el lugar donde residen el Padre Universal, el Hijo Eterno, el Espíritu Infinito y sus coordinados y asociados divinos. Esta Isla central es el cuerpo organizado de realidad cósmica más gigantesco de todo el universo maestro. El Paraíso es una esfera material así como una morada espiritual. Toda la creación inteligente del Padre Universal está domiciliada en moradas materiales; por eso el centro de control absoluto debe ser también material, tangible. Y hay que reiterar de nuevo que las cosas de espíritu y los seres espirituales son \textit{reales.}

\par
%\textsuperscript{(118.2)}
\textsuperscript{11:0.2} La belleza material del Paraíso consiste en la magnificencia de su perfección física; la grandiosidad de la Isla de Dios se manifiesta en los logros intelectuales y en el desarrollo mental magníficos de sus habitantes; la gloria de la Isla central se manifiesta en la donación infinita de la personalidad espiritual divina ---la luz de la vida. Pero la intensidad de la belleza espiritual y las maravillas de este conjunto magnífico sobrepasan por completo la comprensión de la mente finita de las criaturas materiales. La gloria y el esplendor espiritual de la morada divina son imposibles de comprender por los mortales. Y el Paraíso existe desde la eternidad; no hay ni archivos ni tradiciones respecto al origen de esta Isla nuclear de Luz y de Vida.

\section*{1. La residencia divina}
\par
%\textsuperscript{(118.3)}
\textsuperscript{11:1.1} El Paraíso sirve para muchos fines en la administración de los reinos universales, pero para los seres creados, existe principalmente como lugar donde vive la Deidad. La presencia personal del Padre Universal reside en el centro mismo de la superficie superior de esta morada casi circular, pero no esférica, de las Deidades. Esta presencia paradisiaca del Padre Universal está rodeada directamente por la presencia personal del Hijo Eterno, mientras que los dos están envueltos por la gloria indecible del Espíritu Infinito.

\par
%\textsuperscript{(118.4)}
\textsuperscript{11:1.2} Dios vive, ha vivido y vivirá perpetuamente en esta misma morada central y eterna\footnote{\textit{Lugar alto y sagrado}: Is 57:15.}. Siempre lo hemos encontrado allí, y allí lo encontraremos siempre. El Padre Universal está cósmicamente focalizado, espiritualmente personalizado y reside geográficamente en este centro del universo de universos.

\par
%\textsuperscript{(118.5)}
\textsuperscript{11:1.3} Todos conocemos el camino directo a seguir para encontrar al Padre Universal. No sois capaces de comprender muchas cosas acerca de la residencia divina debido a que está muy alejada de vosotros y a que el espacio intermedio es inmenso, pero aquellos que pueden comprender el significado de estas distancias enormes, conocen el emplazamiento y la residencia de Dios tan cierta y literalmente como vosotros conocéis el emplazamiento de Nueva York, Londres, Roma o Singapur, ciudades geográficamente situadas con precisión en Urantia. Si fuerais unos navegantes inteligentes, equipados con un barco, unos mapas y una brújula, podríais encontrar fácilmente estas ciudades. De la misma manera, si tuvierais el tiempo y los medios de paso, si estuvierais cualificados espiritualmente y contarais con la orientación necesaria, podríais ser guiados de universo en universo y de circuito en circuito, viajando siempre hacia el interior a través de los reinos estelares, hasta que al fin os encontraríais delante del resplandor central de la gloria espiritual del Padre Universal. Provistos de todo lo necesario para el viaje, es tan posible descubrir la presencia personal de Dios en el centro de todas las cosas como encontrar ciudades lejanas en vuestro propio planeta. El hecho de que no hayáis visitado esos lugares no refuta de ninguna manera su realidad o su existencia efectiva. El hecho de que tan pocas criaturas del universo hayan encontrado a Dios en el Paraíso no refuta de ninguna forma la realidad de su existencia ni la realidad de su persona espiritual en el centro de todas las cosas.

\par
%\textsuperscript{(119.1)}
\textsuperscript{11:1.4} Al Padre siempre se le puede encontrar en este emplazamiento central. Si se trasladara, se produciría un pandemónium universal, porque las líneas universales de la gravedad convergen en él, desde los confines de la creación, en este centro residencial. Que remontemos el circuito de la personalidad a través de los universos o que sigamos a las personalidades ascendentes que viajan hacia el interior hasta el Padre; que sigamos la pista de las líneas de la gravedad material hasta el Paraíso inferior o que sigamos los ciclos crecientes de la fuerza cósmica; que sigamos la pista de las líneas de la gravedad espiritual hasta el Hijo Eterno o que sigamos la procesión hacia el interior de los Hijos Paradisiacos de Dios; que descubramos el rastro de los circuitos mentales o que sigamos a los billones y billones de seres celestiales que proceden del Espíritu Infinito ---cualquiera de estas observaciones o el conjunto de ellas nos conducirá directamente a la presencia del Padre, a su morada central. Aquí, Dios está personal, literal y realmente presente. Y de su ser infinito fluyen las corrientes torrenciales de la vida, la energía y la personalidad hacia todos los universos.

\section*{2. La naturaleza de la Isla Eterna}
\par
%\textsuperscript{(119.2)}
\textsuperscript{11:2.1} Puesto que empezáis a vislumbrar la enormidad del universo material discernible incluso desde vuestro emplazamiento astronómico, desde vuestra posición espacial en los sistemas estelares, debería ser evidente para vosotros que un universo material tan asombroso ha de tener una capital adecuada y digna de él, una sede central proporcionada a la dignidad y a la infinitud del Soberano universal de toda esta inmensa y extensa creación de reinos materiales y de seres vivientes.

\par
%\textsuperscript{(119.3)}
\textsuperscript{11:2.2} El Paraíso difiere, en su forma, de los cuerpos espaciales habitados: no es esférico. Es claramente elipsoide; su diámetro norte-sur es una sexta parte más largo que su diámetro este-oeste. La Isla central es esencialmente plana, y la distancia entre la superficie superior y la superficie inferior es una décima parte del diámetro este-oeste.

\par
%\textsuperscript{(119.4)}
\textsuperscript{11:2.3} Estas diferencias en sus dimensiones, unidas a su estado estacionario y a una mayor presión exterior de la energía-fuerza en el extremo norte de la Isla, permiten establecer direcciones absolutas en el universo maestro.

\par
%\textsuperscript{(119.5)}
\textsuperscript{11:2.4} La Isla central está dividida geográficamente en tres campos de actividad:

\par
%\textsuperscript{(119.6)}
\textsuperscript{11:2.5} 1. El Paraíso Superior.

\par
%\textsuperscript{(119.7)}
\textsuperscript{11:2.6} 2. El Paraíso Periférico.

\par
%\textsuperscript{(119.8)}
\textsuperscript{11:2.7} 3. El Paraíso Inferior.

\par
%\textsuperscript{(119.9)}
\textsuperscript{11:2.8} A la superficie del Paraíso que está ocupada con las actividades de la personalidad la denominamos parte superior, y a la superficie opuesta parte inferior. En la periferia del Paraíso se mantienen actividades que no son ni estrictamente personales ni no personales. La Trinidad parece dominar el plano personal o superior, y el Absoluto Incalificado el plano impersonal o inferior. Al Absoluto Incalificado difícilmente lo concebimos como una persona, pero imaginamos que la presencia espacial funcional de este Absoluto está focalizada en el Paraíso inferior.

\par
%\textsuperscript{(120.1)}
\textsuperscript{11:2.9} La Isla eterna está compuesta de una sola forma de materialización ---de sistemas estacionarios de realidad. Esta sustancia tangible del Paraíso es una organización homogénea de potencia espacial que no se encuentra en ninguna otra parte de todo el extenso universo de universos. Ha recibido muchos nombres en diferentes universos, y los Melquisedeks de Nebadon la han llamado desde hace mucho tiempo \textit{absolutum.} Esta materia fuente del Paraíso no está muerta ni viva; es la expresión original no espiritual de la Fuente-Centro Primera; es el \textit{Paraíso,} y el Paraíso no tiene copias.

\par
%\textsuperscript{(120.2)}
\textsuperscript{11:2.10} A nosotros nos parece que la Fuente-Centro Primera ha concentrado en el Paraíso todo el potencial absoluto de la realidad cósmica como parte de su técnica para liberarse de las limitaciones de la infinidad, como un medio para hacer posible la creación subinfinita e incluso la creación espacio-temporal. Pero de esto no se deduce que el Paraíso esté limitado por el espacio-tiempo, solamente porque el universo de universos revele estas cualidades. El Paraíso existe sin el tiempo y no está ubicado en el espacio.

\par
%\textsuperscript{(120.3)}
\textsuperscript{11:2.11} A grandes rasgos, el espacio se origina aparentemente justo por debajo del Paraíso inferior, y el tiempo justo por encima del Paraíso superior. El tiempo, tal como vosotros lo comprendéis, no es una característica de la existencia en el Paraíso, aunque los habitantes de la Isla Central son plenamente conscientes de la secuencia intemporal de los acontecimientos. El movimiento no es inherente al Paraíso; es volitivo. Pero el concepto de la distancia, e incluso de la distancia absoluta, tiene un gran significado pues puede ser aplicado a emplazamientos relativos en el Paraíso. El Paraíso es no espacial; de ahí que sus áreas sean absolutas y, por consiguiente, utilizables de muchas maneras que sobrepasan los conceptos de la mente humana.

\section*{3. El Paraíso superior}
\par
%\textsuperscript{(120.4)}
\textsuperscript{11:3.1} En el Paraíso superior existen tres grandes esferas de actividad: la \textit{presenciade la Deidad,} la \textit{Esfera Santísima} y el \textit{Área Santa.} La inmensa región que rodea directamente la presencia de las Deidades se encuentra aparte como Esfera Santísima y está reservada para las funciones de la adoración, la trinitización y la consecución espiritual superior. En esta zona no existen estructuras materiales ni creaciones puramente intelectuales; no podrían existir allí. Es inútil intentar por mi parte describirle a la mente humana la naturaleza divina y la hermosa grandiosidad de la Esfera Santísima del Paraíso. Esta zona es totalmente espiritual, y vosotros sois casi enteramente materiales. Para un ser puramente material, una realidad puramente espiritual es aparentemente inexistente.

\par
%\textsuperscript{(120.5)}
\textsuperscript{11:3.2} Aunque no hay materializaciones físicas en el área Santísima, en los sectores de la Tierra Santa existen abundantes recuerdos de vuestros días materiales, y hay aún más en las áreas históricas de reminiscencia del Paraíso periférico.

\par
%\textsuperscript{(120.6)}
\textsuperscript{11:3.3} El Área Santa, la región exterior o residencial, está dividida en siete zonas concéntricas. Al Paraíso se le llama a veces <<\textit{la Casa del Padre}>>\footnote{\textit{La Casa del Padre}: Jn 2:16.}, puesto que es su residencia eterna, y a estas siete zonas se las denomina con frecuencia <<\textit{las mansiones paradisiacas del Padre}>>\footnote{\textit{Muchas mansiones}: Jn 14:2.}. La zona primera o interior está ocupada por los Ciudadanos del Paraíso y por los nativos de Havona que residen circunstancialmente en el Paraíso. La zona siguiente o segunda es el área residencial de los nativos de los siete superuniversos del tiempo y del espacio. Una parte de esta segunda zona está subdividida en siete inmensas divisiones, el hogar paradisiaco de los seres espirituales y de las criaturas ascendentes que proceden de los universos de progresión evolutiva. Cada uno de estos sectores está dedicado exclusivamente al bienestar y al progreso de las personalidades de un solo superuniverso, pero estas instalaciones sobrepasan casi infinitamente las necesidades de los siete superuniversos actuales.

\par
%\textsuperscript{(121.1)}
\textsuperscript{11:3.4} Cada uno de los siete sectores del Paraíso está subdividido en unidades residenciales adecuadas para albergar la sede de mil millones de grupos de trabajo individuales y glorificados. Mil unidades de éstas constituyen una división. Cien mil divisiones son iguales a una congregación. Diez millones de congregaciones constituyen una asamblea. Mil millones de asambleas componen una gran unidad. Y esta serie ascendente continúa con la segunda gran unidad, la tercera y así sucesivamente hasta la séptima gran unidad. Siete grandes unidades forman las unidades maestras, y siete unidades maestras constituyen una unidad superior; y así, por grupos de siete, las series ascendentes se amplían a unidades superiores, supersuperiores, celestiales y supercelestiales, hasta las unidades supremas. Pero incluso esto no llega a ocupar todo el espacio disponible. Este número asombroso de denominaciones residenciales en el Paraíso, un número que sobrepasa vuestros conceptos, ocupa mucho menos del uno por ciento del área asignada de la Tierra Santa. Hay todavía mucho sitio para aquellos que caminan hacia el interior, e incluso para aquellos que no empezarán la ascensión al Paraíso hasta las épocas del eterno futuro.

\section*{4. El Paraíso periférico}
\par
%\textsuperscript{(121.2)}
\textsuperscript{11:4.1} La Isla central termina bruscamente en la periferia, pero su tamaño es tan enorme que este ángulo terminal es relativamente imperceptible desde el interior de un área circunscrita cualquiera. La superficie periférica del Paraíso está ocupada en parte por los campos de aterrizaje y de partida de diversos grupos de personalidades espirituales. Puesto que las zonas de espacio no penetrado casi entran en contacto con la periferia, todos los transportes de personalidades destinados al Paraíso aterrizan en estas regiones. Los supernafines trasportadores o los otros tipos de seres que atraviesan el espacio no pueden acceder ni al Paraíso superior ni al Paraíso inferior.

\par
%\textsuperscript{(121.3)}
\textsuperscript{11:4.2} Los Siete Espíritus Maestros tienen su sede personal de poder y de autoridad en las siete esferas del Espíritu, que giran alrededor del Paraíso en el espacio situado entre los orbes brillantes del Hijo y el circuito interior de los mundos de Havona, pero mantienen unas sedes centrales de fuerza en la periferia del Paraíso. Aquí, las presencias de los Siete Directores Supremos del Poder circulan lentamente e indican la posición de las siete estaciones que transmiten ciertas energías del Paraíso que salen hacia los siete superuniversos.

\par
%\textsuperscript{(121.4)}
\textsuperscript{11:4.3} Aquí, en el Paraíso periférico, se encuentran las enormes áreas de exposiciones históricas y proféticas asignadas a los Hijos Creadores, dedicadas a los universos locales del tiempo y del espacio. Hay exactamente siete billones de estas reservas históricas ya instaladas o en reserva, pero todas estas instalaciones reunidas ocupan solamente alrededor de un cuatro por ciento de la porción del área periférica que les está asignada. Deducimos que estas inmensas reservas pertenecen a las creaciones que algún día estarán situadas más allá de las fronteras de los siete superuniversos conocidos y habitados en la actualidad.

\par
%\textsuperscript{(121.5)}
\textsuperscript{11:4.4} La porción del Paraíso que ha sido designada para el uso de los universos existentes sólo está ocupada entre el uno y el cuatro por ciento, mientras que el área asignada a estas actividades es al menos un millón de veces mayor que la que se necesita realmente para esa finalidad. El Paraíso es lo bastante grande como para acomodar las actividades de una creación casi infinita.

\par
%\textsuperscript{(121.6)}
\textsuperscript{11:4.5} Pero cualquier intento adicional por haceros imaginar las glorias del Paraíso sería inútil. Tenéis que esperar, y ascender mientras esperáis, porque en verdad <<\textit{el ojo no ha visto, el oído no ha percibido, ni la mente del hombre mortal ha concebido las cosas que el Padre Universal ha preparado para aquellos que sobreviven a la vida en la carne en los mundos del tiempo y del espacio}>>\footnote{\textit{El ojo no ha visto ni el oído ha oído}: Is 64:4; 1 Co 2:9.}.

\section*{5. El Paraíso inferior}
\par
%\textsuperscript{(122.1)}
\textsuperscript{11:5.1} En cuanto al Paraíso inferior, sólo sabemos lo que nos han revelado; las personalidades no residen allí. No tiene ninguna relación en absoluto con los asuntos de las inteligencias espirituales, y el Absoluto de la Deidad tampoco ejerce allí su actividad. Se nos informa que todos los circuitos de la energía física y de la fuerza cósmica tienen su origen en el Paraíso inferior, y que éste está formado como sigue:

\par
%\textsuperscript{(122.2)}
\textsuperscript{11:5.2} 1. Directamente debajo del emplazamiento de la Trinidad, en la parte central del Paraíso inferior, se encuentra la Zona desconocida y no revelada de la Infinidad.

\par
%\textsuperscript{(122.3)}
\textsuperscript{11:5.3} 2. Esta Zona está directamente rodeada por un área sin nombre.

\par
%\textsuperscript{(122.4)}
\textsuperscript{11:5.4} 3. Los márgenes exteriores de la superficie inferior están ocupados por una región que está relacionada principalmente con la potencia del espacio y la energía-fuerza. Las actividades de este inmenso centro de fuerza elíptico no se pueden identificar con las funciones conocidas de ninguna triunidad, pero la carga primordial de fuerza del espacio parece estar focalizada en este área. Este centro consta de tres zonas elípticas concéntricas: la más interior es el punto focal de las actividades de la energía-fuerza del Paraíso mismo; la más exterior posiblemente se puede identificar con las funciones del Absoluto Incalificado, pero no estamos seguros en cuanto a las funciones espaciales de la zona intermedia.

\par
%\textsuperscript{(122.5)}
\textsuperscript{11:5.5} \textit{La zona interior} de este centro de fuerza parece actuar como un corazón gigantesco cuyas pulsaciones dirigen las corrientes hacia los límites más exteriores del espacio físico. Dirige y modifica las energías-fuerza, pero no las conduce del todo. La presencia-presión de realidad de esta fuerza primordial es claramente mayor en el extremo norte del centro paradisiaco que en las regiones del sur; es una diferencia que está uniformemente registrada. La fuerza madre del espacio parece entrar a raudales por el sur y salir por el norte gracias al funcionamiento de algún sistema circulatorio desconocido que está relacionado con la difusión de esta forma fundamental de energía-fuerza. De vez en cuando se producen también diferencias notables en las presiones este-oeste. Las fuerzas que emanan de esta zona no responden a la gravedad física observable, pero siempre obedecen a la gravedad del Paraíso.

\par
%\textsuperscript{(122.6)}
\textsuperscript{11:5.6} \textit{La zona intermedia} del centro de fuerza rodea directamente este área. Esta zona intermedia parece ser estática, salvo que se expande y se contrae a lo largo de tres ciclos de actividad. La más pequeña de estas pulsaciones se produce en dirección este-oeste y la siguiente en dirección norte-sur, mientras que la fluctuación más grande tiene lugar en todas direcciones, una expansión y una contracción generalizadas. La función de este área intermedia nunca ha sido realmente identificada, pero debe tener algo que ver con los ajustes recíprocos entre las zonas interior y exterior del centro de fuerza. Muchos creen que la zona intermedia es el mecanismo que controla las zonas de espacio intermedio, o zonas tranquilas, que separan a los niveles espaciales sucesivos del universo maestro, pero no existe ninguna prueba o revelación que lo confirme. Esta deducción se deriva del conocimiento de que este área intermedia está relacionada de alguna manera con el funcionamiento del mecanismo del espacio no penetrado del universo maestro.

\par
%\textsuperscript{(122.7)}
\textsuperscript{11:5.7} \textit{La zona exterior} es la más grande y la más activa de los tres cinturones concéntricos y elípticos del potencial espacial no identificado. Este área es el escenario de unas actividades inimaginables, el punto central de un circuito de emanaciones que se dirigen hacia el espacio en todas direcciones hasta los límites más alejados de los siete superuniversos, y que continúan más allá hasta extenderse sobre los enormes dominios incomprensibles de todo el espacio exterior. Esta presencia espacial es enteramente impersonal, a pesar de que de alguna manera no revelada parece responder indirectamente a la voluntad y a los mandatos de las Deidades infinitas cuando éstas actúan como Trinidad. Se cree que ésta es la focalización central, el centro paradisiaco, de la presencia espacial del Absoluto Incalificado.

\par
%\textsuperscript{(123.1)}
\textsuperscript{11:5.8} Todas las formas de fuerza y todas las fases de la energía parecen estar integradas en circuitos; circulan por todos los universos y regresan por rutas precisas. Pero en lo que se refiere a las emanaciones de la zona activada del Absoluto Incalificado, parece que se produce o una salida o una entrada ---pero nunca las dos a la vez. Esta zona exterior palpita en ciclos seculares de proporciones gigantescas. Durante un poco más de mil millones de años de Urantia, la fuerza espacial de este centro sale hacia el exterior; luego, durante un período de tiempo similar, estará entrando. Y las manifestaciones de la fuerza espacial de este centro son universales; se extienden por todo el espacio penetrable.

\par
%\textsuperscript{(123.2)}
\textsuperscript{11:5.9} Toda fuerza física, toda energía y toda materia son una sola cosa. Toda energía-fuerza procede originalmente del Paraíso inferior y regresará finalmente allí después de completar su circuito espacial. Pero no todas las energías y organizaciones materiales del universo de universos provinieron del Paraíso inferior en sus estados fenoménicos actuales; el espacio es la cuna de diversas formas de materia y de premateria. Aunque la zona exterior del centro de fuerza del Paraíso es la fuente de las energías del espacio, el espacio no se origina allí. El espacio no es ni fuerza, ni energía, ni poder. Las pulsaciones de esta zona tampoco explican la respiración del espacio, pero las fases de entrada y de salida de esta zona están sincronizadas con los ciclos de expansión y de contracción del espacio que duran dos mil millones de años.

\section*{6. La respiración del espacio}
\par
%\textsuperscript{(123.3)}
\textsuperscript{11:6.1} No conocemos el mecanismo concreto de la respiración del espacio; simplemente observamos que todo el espacio se contrae y se expande alternativamente. Esta respiración afecta tanto a la extensión horizontal del espacio penetrado como a las extensiones verticales del espacio no penetrado que existen en los inmensos depósitos de espacio que se hallan por encima y por debajo del Paraíso. Para intentar imaginar la silueta volumétrica de estos depósitos de espacio, podríais pensar en un reloj de arena.

\par
%\textsuperscript{(123.4)}
\textsuperscript{11:6.2} Cuando los universos de la extensión horizontal del espacio penetrado se dilatan, los depósitos de la extensión vertical del espacio no penetrado se contraen, y viceversa. Existe una confluencia de espacio penetrado y no penetrado justo por debajo del Paraíso inferior. Los dos tipos de espacio fluyen allí a través de los canales reguladores que los transmutan, donde se producen cambios que hacen penetrable el espacio no penetrable, y viceversa, durante los ciclos de contracción y de expansión del cosmos.

\par
%\textsuperscript{(123.5)}
\textsuperscript{11:6.3} Espacio <<\textit{no penetrado}>> significa: no penetrado por aquellas fuerzas, energías, poderes y presencias que se sabe que existen en el espacio penetrado. No sabemos si el espacio vertical (depósito) está destinado a funcionar siempre como contrapeso del espacio horizontal (universo); no sabemos si existe una intención creativa con respecto al espacio no penetrado; sabemos realmente muy poco acerca de los depósitos de espacio, simplemente que existen y que parecen contrapesar los ciclos de expansión y de contracción espaciales del universo de universos.

\par
%\textsuperscript{(123.6)}
\textsuperscript{11:6.4} Los ciclos de la respiración del espacio duran en cada fase poco más de mil millones de años de Urantia. Durante una fase los universos se expanden; durante la siguiente se contraen. El espacio penetrado se está acercando ahora al punto medio de su fase de expansión, mientras que el espacio no penetrado se aproxima al punto medio de su fase de contracción, y nos han informado que los límites extremos de las dos extensiones de espacio se encuentran teóricamente en la actualidad casi equidistantes del Paraíso. Los depósitos de espacio no penetrado se extienden ahora verticalmente por encima del Paraíso superior y por debajo del Paraíso inferior a la misma distancia que el espacio penetrado del universo se extiende horizontalmente hacia el exterior del Paraíso periférico hasta el cuarto nivel del espacio exterior, e incluso más allá.

\par
%\textsuperscript{(124.1)}
\textsuperscript{11:6.5} Durante mil millones de años del tiempo de Urantia, los depósitos de espacio se contraen mientras que el universo maestro y las actividades de fuerza de todo el espacio horizontal se expanden. Hace falta pues poco más de dos mil millones de años de Urantia para completar todo el ciclo de expansión-contracción.

\section*{7. Las funciones espaciales del Paraíso}
\par
%\textsuperscript{(124.2)}
\textsuperscript{11:7.1} El espacio no existe en ninguna de las superficies del Paraíso. Si uno <<\textit{mirara}>> directamente hacia arriba desde la superficie superior del Paraíso, no <<\textit{vería}>> nada más que el espacio no penetrado llegando o saliendo, y en este momento llega. El espacio no toca el Paraíso; sólo las \textit{zonas} en reposo del \textit{espacio intermedio} entran en contacto con la Isla central.

\par
%\textsuperscript{(124.3)}
\textsuperscript{11:7.2} El Paraíso es el núcleo realmente inmóvil de las zonas relativamente inactivas que existen entre el espacio penetrado y el espacio no penetrado. Geográficamente, estas zonas parecen ser una extensión relativa del Paraíso, pero es probable que tengan algún movimiento. Sabemos muy poco acerca de ellas, pero observamos que estas zonas de movimiento espacial reducido separan el espacio penetrado del espacio no penetrado. En otro tiempo existieron unas zonas similares entre los niveles del espacio penetrado, pero ahora se encuentran menos inactivas.

\par
%\textsuperscript{(124.4)}
\textsuperscript{11:7.3} La sección transversal vertical del espacio total se parecería un poco a una cruz de Malta, donde los brazos horizontales representarían el espacio penetrado (el universo) y los brazos verticales el espacio no penetrado (el depósito). Las áreas entre los cuatro brazos los separarían en cierto modo, como las zonas de espacio intermedio separan al espacio penetrado del espacio no penetrado. Estas zonas inactivas del espacio intermedio se agrandan cada vez más a medida que se distancian del Paraíso, envolviendo finalmente los bordes de todo el espacio y encerrando por completo tanto los depósitos de espacio como toda la extensión horizontal del espacio penetrado.

\par
%\textsuperscript{(124.5)}
\textsuperscript{11:7.4} El espacio no es ni un estado subabsoluto dentro del Absoluto Incalificado, ni la presencia de éste, ni tampoco es una función del Último. Es un don del Paraíso, y se cree que el espacio del gran universo y el de todas las regiones exteriores está realmente penetrado por la potencia espacial ancestral del Absoluto Incalificado. Este espacio penetrado se extiende horizontalmente desde las proximidades del Paraíso periférico hacia el exterior por todo el cuarto nivel de espacio y más allá de la periferia del universo maestro, pero no sabemos cuánto más allá.

\par
%\textsuperscript{(124.6)}
\textsuperscript{11:7.5} Si os imagináis un plano en forma de V, finito pero inconcebiblemente grande, situado en ángulo recto con respecto a las superficies superior e inferior del Paraíso, con su punta casi tangente al Paraíso periférico, y luego visualizáis este plano rotando elípticamente alrededor del Paraíso, su rotación esbozaría aproximadamente el volumen del espacio penetrado.

\par
%\textsuperscript{(124.7)}
\textsuperscript{11:7.6} El espacio horizontal tiene un límite superior y un límite inferior con relación a cualquier posición dada en los universos. Si alguien pudiera desplazarse lo bastante lejos en ángulo recto con respecto al plano de Orvonton, ya sea hacia arriba o hacia abajo, encontraría finalmente el límite superior o inferior del espacio penetrado. Dentro de las dimensiones conocidas del universo maestro, estos límites se separan cada vez más a medida que se alejan del Paraíso; el espacio se espesa, y se espesa un poco más deprisa que el plano de la creación, es decir, que los universos.

\par
%\textsuperscript{(125.1)}
\textsuperscript{11:7.7} Las zonas relativamente tranquilas que se encuentran entre los niveles de espacio, como la que separa a los siete superuniversos del primer nivel del espacio exterior, son unas enormes regiones elípticas donde las actividades espaciales están en reposo. Estas zonas separan las inmensas galaxias que giran con rapidez en procesión ordenada alrededor del Paraíso. Podéis visualizar el primer nivel del espacio exterior, donde incalculables universos están ahora en proceso de formación, como una enorme procesión de galaxias que giran alrededor del Paraíso, limitadas por arriba y por abajo por las zonas en reposo del espacio intermedio, y limitadas en los márgenes interior y exterior por las zonas de espacio relativamente tranquilas.

\par
%\textsuperscript{(125.2)}
\textsuperscript{11:7.8} Un nivel de espacio funciona pues como una región de movimiento elíptica, rodeada por todas partes por una inmovilidad relativa. Estas relaciones entre el movimiento y la quietud forman un camino espacial curvo de menor resistencia al movimiento, un camino que es seguido universalmente por la fuerza cósmica y la energía emergente a medida que giran eternamente alrededor de la Isla del Paraíso.

\par
%\textsuperscript{(125.3)}
\textsuperscript{11:7.9} Estas zonas alternas del universo maestro, en unión con la circulación alterna de las galaxias en el sentido de las agujas del reloj y en el sentido contrario, es un factor para la estabilización de la gravedad física, destinado a impedir que la presión de la gravedad se acentúe hasta el punto de producirse actividades disruptivas y de dispersión. Esta medida ejerce una influencia antigravitatoria y actúa como un freno sobre unas velocidades que de otra manera serían peligrosas.

\section*{8. La gravedad del Paraíso}
\par
%\textsuperscript{(125.4)}
\textsuperscript{11:8.1} La atracción ineludible de la gravedad sujeta eficazmente todos los mundos de todos los universos de todo el espacio. La gravedad es la atracción todopoderosa de la presencia física del Paraíso. La gravedad es el hilo omnipotente al que están atados los soles resplandecientes, las estrellas brillantes y las esferas que giran, los cuales constituyen el adorno físico universal del Dios eterno, que lo es todo\footnote{\textit{Dios es todas las cosas}: Hch 17:24,28; 1 Co 8:6.}, que lo llena todo\footnote{\textit{Dios llena todas las cosas}: Ef 4:10.}, y en quien todas las cosas consisten\footnote{\textit{En Dios todas las cosas consisten}: Hch 17:22-25,28; Ro 11:36; 1 Co 8:6; Ef 4:10; Col 1:17.}.

\par
%\textsuperscript{(125.5)}
\textsuperscript{11:8.2} El centro y el punto focal de la gravedad material absoluta es la Isla del Paraíso, complementada por los cuerpos de gravedad oscuros que rodean a Havona, y equilibrada por los depósitos de espacio situados por encima y por debajo. Todas las emanaciones conocidas del Paraíso inferior reaccionan invariable e infaliblemente a la atracción de la gravedad central, que actúa sobre los circuitos sin fin de los niveles espaciales elípticos del universo maestro. Toda forma conocida de realidad cósmica tiene la inclinación de los siglos, la tendencia del círculo, el recorrido de la gran elipse.

\par
%\textsuperscript{(125.6)}
\textsuperscript{11:8.3} El espacio es insensible a la gravedad, pero actúa como una fuerza equilibrante sobre la gravedad. Sin el colchón del espacio, la acción explosiva sacudiría a los cuerpos espaciales circundantes. El espacio penetrado ejerce también una influencia antigravitatoria sobre la gravedad física o lineal; el espacio puede neutralizar realmente esta acción de la gravedad, aunque no puede retrasarla. La gravedad absoluta es la gravedad del Paraíso. La gravedad local o lineal es propia del estado eléctrico de la energía o de la materia; actúa dentro del universo central, de los superuniversos y de los universos exteriores, dondequiera que haya tenido lugar una materialización adecuada.

\par
%\textsuperscript{(125.7)}
\textsuperscript{11:8.4} Las numerosas formas de la fuerza cósmica, de la energía física, del poder del universo y de las diversas materializaciones, revelan tres etapas generales de reacción, aunque no perfectamente definidas, a la gravedad del Paraíso:

\par
%\textsuperscript{(126.1)}
\textsuperscript{11:8.5} 1. \textit{Las Etapas de la Pregravedad
(Fuerza).} Éste es el primer paso de la individuación de la potencia espacial hacia las formas preenergéticas de la fuerza cósmica. Este estado es análogo al concepto de la carga de fuerza primordial del espacio, llamada a veces \textit{energía pura o segregata}.

\par
%\textsuperscript{(126.2)}
\textsuperscript{11:8.6} 2. \textit{Las Etapas de la Gravedad (Energía).} La actividad de los organizadores de fuerza del Paraíso produce esta modificación en la carga de fuerza del espacio. Señala la aparición de los sistemas de energía que reaccionan a la atracción de la gravedad del Paraíso. Esta energía emergente es originalmente neutra, pero a consecuencia de metamorfosis ulteriores, manifestará las cualidades llamadas positivas y negativas. A estas etapas las denominamos \textit{ultimata}.

\par
%\textsuperscript{(126.3)}
\textsuperscript{11:8.7} 3. \textit{Las Etapas de la Postgravedad (Poder del Universo).} En esta etapa, la energía-materia revela que reacciona al control de la gravedad lineal. En el universo central, estos sistemas físicos son unas organizaciones triples conocidas como \textit{triata}. Son los sistemas del superpoder que dan nacimiento a las creaciones del tiempo y del espacio. Los sistemas físicos de los superuniversos son movilizados por los Directores del Poder Universal y sus asociados. Estas organizaciones materiales tienen una constitución doble y se conocen como \textit{gravita}. Los cuerpos de gravedad oscuros que rodean a Havona no están hechos ni de triata ni de gravita, y su poder de atracción revela las dos formas de la gravedad física, la lineal y la absoluta.

\par
%\textsuperscript{(126.4)}
\textsuperscript{11:8.8} La potencia del espacio no está sometida a las interacciones de ninguna forma de gravitación. Este don primordial del Paraíso no es un nivel efectivo de realidad, pero es ancestral a todas las realidades relativas funcionales no espirituales ---a todas las manifestaciones de energía-fuerza y a la organización del poder y de la materia. La potencia del espacio es un término difícil de definir. No indica aquello que es ancestral al espacio; su significado debería expresar la idea de las potencias y de los potenciales que existen dentro del espacio. Se puede concebir más o menos como que incluye todas las influencias y potenciales absolutos que emanan del Paraíso y que constituyen la presencia espacial del Absoluto Incalificado.

\par
%\textsuperscript{(126.5)}
\textsuperscript{11:8.9} El Paraíso es la fuente absoluta y el punto focal eterno de toda la energía-materia en el universo de universos\footnote{\textit{Gravedad espiritual}: Jer 31:3; Jn 6:44; 12:32.}. El Absoluto Incalificado es el revelador, el regulador y el depositario de aquello que tiene su fuente y su origen en el Paraíso. La presencia universal del Absoluto Incalificado parece ser equivalente al concepto de que la extensión de la gravedad es potencialmente infinita, de que es una tensión elástica de la presencia del Paraíso. Este concepto nos ayuda a comprender el hecho de que todo es atraído hacia el interior, hacia el Paraíso. El ejemplo es rudimentario, pero sin embargo puede ser útil. También explica por qué la gravedad actúa siempre preferentemente en el plano perpendicular a la masa, un fenómeno que indica que las dimensiones del Paraíso y de las creaciones que lo rodean son diferenciales.

\section*{9. La unicidad del Paraíso}
\par
%\textsuperscript{(126.6)}
\textsuperscript{11:9.1} El Paraíso es único en el sentido de que es la esfera de origen primordial y la meta de destino final de todas las personalidades espirituales. Aunque es cierto que no todos los seres espirituales inferiores de los universos locales son destinados inmediatamente al Paraíso, el Paraíso sigue siendo la meta deseada por todas las personalidades supermateriales.

\par
%\textsuperscript{(126.7)}
\textsuperscript{11:9.2} El Paraíso es el centro geográfico de la infinidad; no es una parte de la creación universal, y ni siquiera forma parte real del eterno universo de Havona. Normalmente nos referimos a la Isla central como si perteneciera al universo divino, pero en realidad no es así. El Paraíso es una existencia eterna y exclusiva.

\par
%\textsuperscript{(127.1)}
\textsuperscript{11:9.3} En la eternidad del pasado, cuando el Padre Universal expresó la personalidad infinita de su yo espiritual en el ser del Hijo Eterno, reveló simultáneamente el potencial de infinidad de su yo no personal bajo la forma del Paraíso. El Paraíso no personal y no espiritual parece haber sido la repercusión inevitable de la voluntad y del acto del Padre que eternizó al Hijo Original. El Padre proyectó así la realidad en dos fases concretas ---la personal y la no personal, la espiritual y la no espiritual. La tensión entre ellas, en presencia de la voluntad de acción del Padre y del Hijo, dio la existencia al Actor Conjunto y al universo central de mundos materiales y de seres espirituales.

\par
%\textsuperscript{(127.2)}
\textsuperscript{11:9.4} Cuando la realidad está diferenciada entre lo personal y lo no personal (entre el Hijo Eterno y el Paraíso), no es muy correcto llamar <<\textit{Deidad}>> a aquello que es no personal, a menos que esté capacitado de alguna manera. A la energía y a las repercusiones materiales de los actos de la Deidad difícilmente se les podría llamar Deidad. La Deidad puede ser la causa de muchas cosas que no son Deidad, y el Paraíso no es una Deidad, ni tampoco es consciente a la manera en que el hombre mortal podría llegar a comprender este término.

\par
%\textsuperscript{(127.3)}
\textsuperscript{11:9.5} El Paraíso no es ancestral a ningún ser o entidad viviente; no es un creador. La personalidad y las relaciones entre la mente y el espíritu son \textit{transmisibles,} pero el arquetipo no lo es. Los arquetipos nunca son reflejos; son copias ---reproducciones. El Paraíso es el absoluto de los arquetipos; Havona es una muestra de estos potenciales hechos manifiestos.

\par
%\textsuperscript{(127.4)}
\textsuperscript{11:9.6} La residencia de Dios es central y eterna, gloriosa e ideal. Su hogar es el hermoso arquetipo para todos los mundos sede del universo; y el universo central donde reside realmente es el arquetipo para los ideales, la organización y el destino último de todos los universos.

\par
%\textsuperscript{(127.5)}
\textsuperscript{11:9.7} El Paraíso es la sede universal de todas las actividades de la personalidad y la fuente-centro de todas las manifestaciones de la energía y de la fuerza espacial. Todo lo que ha existido, existe ahora o está todavía por existir, ha surgido, surge ahora o surgirá después de este lugar central donde residen los Dioses eternos. El Paraíso es el centro de toda la creación, la fuente de todas las energías y el lugar de origen primordial de todas las personalidades.

\par
%\textsuperscript{(127.6)}
\textsuperscript{11:9.8} Después de todo, la cosa más importante para los mortales, en lo que concierne al Paraíso eterno, es el hecho de que esta morada perfecta del Padre Universal es el destino real y lejano de las almas inmortales de los hijos mortales y materiales de Dios, las criaturas ascendentes de los mundos evolutivos del tiempo y del espacio. Cada mortal que conoce a Dios y que ha abrazado la carrera de hacer la voluntad del Padre, ya se ha embarcado en el larguísimo camino hacia el Paraíso a la búsqueda de la divinidad y del logro de la perfección. Y cuando un ser así de origen animal se halla ante los Dioses del Paraíso después de haber ascendido desde las esferas humildes del espacio, como actualmente lo hace un número incontable de sus semejantes, esa hazaña representa la realidad de una transformación espiritual que linda con los límites de la supremacía.

\par
%\textsuperscript{(127.7)}
\textsuperscript{11:9.9} [Presentado por un Perfeccionador de la Sabiduría, encargado por los Ancianos de los Días de Uversa para llevar a cabo esta tarea.]


\chapter{Documento 12. El universo de universos}
\par
%\textsuperscript{(128.1)}
\textsuperscript{12:0.1} LA inmensidad de la extensa creación del Padre Universal sobrepasa por completo el alcance de la imaginación finita; la enormidad del universo maestro hace que se tambaleen incluso los conceptos de los seres de mi orden. Pero se pueden enseñar muchas cosas a la mente mortal sobre el plan y la disposición de los universos; podéis conocer algo de su organización física y de su maravillosa administración; podéis aprender muchas cosas sobre los diversos grupos de seres inteligentes que viven en los siete superuniversos del tiempo y en el universo central de la eternidad.

\par
%\textsuperscript{(128.2)}
\textsuperscript{12:0.2} En principio, es decir, en potencial eterno, concebimos que la creación material es infinita porque el Padre Universal es realmente infinito, pero a medida que estudiamos y observamos la creación material total, sabemos que es limitada en cualquier momento dado del tiempo, aunque para vuestras mentes finitas sea comparativamente ilimitada, prácticamente sin confines.

\par
%\textsuperscript{(128.3)}
\textsuperscript{12:0.3} Por el estudio de las leyes físicas y por la observación de los reinos estelares, estamos convencidos de que el Creador infinito no ha manifestado todavía el carácter definitivo de su expresión cósmica, que una gran parte del potencial cósmico del Infinito sigue estando contenida en él mismo y sin revelarse. El universo maestro puede parecer casi infinito para los seres creados, pero está lejos de encontrarse terminado; la creación material tiene todavía límites físicos, y la revelación experiencial del propósito eterno sigue su curso.

\section*{1. Los niveles espaciales del universo maestro}
\par
%\textsuperscript{(128.4)}
\textsuperscript{12:1.1} El universo de universos no es ni un plano infinito, ni un cubo ilimitado, ni un círculo sin confines; tiene dimensiones con toda seguridad. Las leyes de la organización física y de la administración prueban de manera concluyente que todo el inmenso agregado de energía-fuerza y de poder-materia funciona finalmente como una unidad espacial, como un todo organizado y coordinado. El comportamiento observable de la creación material constituye una evidencia de que el universo físico tiene unos límites definidos. La prueba final de que el universo es circular y está delimitado la proporciona el hecho bien conocido por nosotros de que todas las formas de energía básica giran siempre alrededor de la trayectoria curva de los niveles espaciales del universo maestro, obedeciendo a la atracción incesante y absoluta de la gravedad del Paraíso.

\par
%\textsuperscript{(128.5)}
\textsuperscript{12:1.2} Los niveles espaciales sucesivos del universo maestro forman las divisiones principales del espacio penetrado ---de la creación total organizada y parcialmente habitada, o aún por organizarse y habitarse. Si el universo maestro no fuera una serie de niveles espaciales elípticos con una resistencia reducida al movimiento, alternándose con zonas de quietud relativa, creemos que observaríamos que algunas energías cósmicas saldrían disparadas a escala infinita, disparadas en línea recta hacia un espacio sin explorar; pero nunca observamos que la fuerza, la energía o la materia se comporten de esta manera; dan vueltas constantemente, girando siempre en las trayectorias de los grandes circuitos del espacio.

\par
%\textsuperscript{(129.1)}
\textsuperscript{12:1.3} Partiendo desde el Paraíso hacia el exterior a través de la extensión horizontal del espacio penetrado, el universo maestro existe en seis elipses concéntricas, los niveles espaciales que rodean a la Isla central:

\par
%\textsuperscript{(129.2)}
\textsuperscript{12:1.4} 1. El universo central ---Havona.

\par
%\textsuperscript{(129.3)}
\textsuperscript{12:1.5} 2. Los siete superuniversos.

\par
%\textsuperscript{(129.4)}
\textsuperscript{12:1.6} 3. El primer nivel del espacio exterior.

\par
%\textsuperscript{(129.5)}
\textsuperscript{12:1.7} 4. El segundo nivel del espacio exterior.

\par
%\textsuperscript{(129.6)}
\textsuperscript{12:1.8} 5. El tercer nivel del espacio exterior.

\par
%\textsuperscript{(129.7)}
\textsuperscript{12:1.9} 6. El cuarto nivel del espacio exterior, el más alejado.

\par
%\textsuperscript{(129.8)}
\textsuperscript{12:1.10} \textit{Havona,} el universo central, no es una creación temporal; es una existencia eterna. Este universo sin comienzo ni fin consta de mil millones de esferas de una perfección sublime y está rodeado por los enormes cuerpos gravitatorios oscuros. En el centro de Havona se encuentra la Isla del Paraíso, estacionaria y absolutamente estabilizada, rodeada por sus veintiún satélites. Debido a las enormes masas de los cuerpos gravitatorios oscuros que circulan cerca de los bordes del universo central, el contenido másico de esta creación central es muy superior a la masa total conocida de los siete sectores del gran universo.

\par
%\textsuperscript{(129.9)}
\textsuperscript{12:1.11} \textit{El sistema Paraíso-Havona,} el universo eterno que rodea a la Isla eterna, constituye el núcleo perfecto y eterno del universo maestro; los siete superuniversos y todas las regiones del espacio exterior giran en órbitas establecidas alrededor del gigantesco agregado central compuesto por los satélites del Paraíso y las esferas de Havona.

\par
%\textsuperscript{(129.10)}
\textsuperscript{12:1.12} \textit{Los siete superuniversos} no son unas organizaciones físicas primarias; sus fronteras no dividen en ninguna parte a una familia nebular, ni tampoco atraviesan un universo local, una unidad creativa fundamental. Cada superuniverso es simplemente un enjambre geográfico espacial que contiene aproximadamente una séptima parte de la creación organizada y parcialmente habitada posterior a Havona, y cada uno de ellos es casi equivalente en cuanto al número de universos locales que contiene y al espacio que ocupa. \textit{Nebadon,} vuestro universo local, es una de las creaciones más recientes de \textit{Orvonton,} el séptimo superuniverso.

\par
%\textsuperscript{(129.11)}
\textsuperscript{12:1.13} \textit{El gran universo} es la creación organizada y habitada actual. Está compuesto por los siete superuniversos, con un potencial evolutivo total de unos siete billones de planetas habitados, sin mencionar las esferas eternas de la creación central. Pero este cálculo aproximado no tiene en cuenta las esferas arquitectónicas administrativas, ni tampoco incluye a los grupos exteriores de universos no organizados. El borde actual irregular del gran universo, su periferia desigual y sin acabar, junto con el estado enormemente inestable de todo el terreno astronómico, sugieren a nuestros astrónomos que incluso los siete superuniversos están todavía por terminarse. Cuando partimos desde el interior, desde el centro divino hacia cualquier dirección del exterior, llegamos finalmente a los límites exteriores de la creación organizada y habitada; llegamos a los límites exteriores del gran universo. Y es cerca de este borde exterior, en un rincón remoto de esta creación tan magnífica, donde vuestro universo local tiene su existencia agitada.

\par
%\textsuperscript{(129.12)}
\textsuperscript{12:1.14} \textit{Los niveles del espacio exterior.} A lo lejos en el espacio, a una enorme distancia de los siete superuniversos habitados, se están acumulando unos inmensos circuitos increíblemente formidables de fuerza y de energías en proceso de materialización. Existe una zona espacial de quietud relativa entre los circuitos de energía de los siete superuniversos y este gigantesco cinturón exterior de actividades de fuerza, una zona que varía en anchura pero que alcanza un promedio de casi cuatrocientos mil años-luz. Estas zonas espaciales están libres de polvo estelar ---de niebla cósmica. Aquellos de nosotros que estudian estos fenómenos tienen sus dudas en cuanto al estado exacto de las fuerzas espaciales que existen en esta zona de calma relativa que rodea a los siete superuniversos. Pero cerca de medio millón de años-luz más allá de la periferia del gran universo actual, observamos los comienzos de una zona de actividades energéticas increíbles cuyo volumen e intensidad aumentan durante más de veinticinco millones de años-luz. Estas enormes ruedas de fuerzas energizadoras están situadas en el primer nivel del espacio exterior, un cinturón continuo de actividad cósmica que rodea a toda la creación conocida, organizada y habitada.

\par
%\textsuperscript{(130.1)}
\textsuperscript{12:1.15} Más allá de estas regiones están teniendo lugar unas actividades aún más grandes, pues los físicos de Uversa han detectado indicios iniciales de manifestaciones de fuerza a más de cincuenta millones de años-luz más allá de las zonas más exteriores de los fenómenos del primer nivel del espacio exterior. Estas actividades presagian sin duda la organización de las creaciones materiales del segundo nivel del espacio exterior del universo maestro.

\par
%\textsuperscript{(130.2)}
\textsuperscript{12:1.16} El universo central es la creación de la eternidad; los siete superuniversos son las creaciones del tiempo; los cuatro niveles del espacio exterior están destinados sin duda a desarrollar-existenciar la ultimidad de la creación. Y algunos sostienen que el Infinito nunca podrá alcanzar su plena expresión, salvo en la infinidad; admiten por tanto una creación adicional y no revelada mas allá del cuarto y último nivel del espacio exterior, un posible universo infinito, interminable y en constante expansión. En teoría, no sabemos cómo limitar la infinidad del Creador ni la infinidad potencial de la creación, pero consideramos que el universo maestro, tal como existe y está administrado, tiene limitaciones, está claramente delimitado y confinado en sus márgenes exteriores por el espacio abierto.

\section*{2. Los dominios del Absoluto Incalificado}
\par
%\textsuperscript{(130.3)}
\textsuperscript{12:2.1} Cuando los astrónomos de Urantia miran a través de sus telescopios cada vez más potentes las misteriosas extensiones del espacio exterior, y perciben allí la asombrosa evolución de unos universos físicos casi incontables, deberían comprender que están contemplando el poderoso desarrollo de los planes insondables de los Arquitectos del Universo Maestro. Es verdad que poseemos pruebas que sugieren la presencia de ciertas influencias de personalidades paradisiacas aquí y allá en todas las inmensas manifestaciones de energía que caracterizan actualmente a estas regiones exteriores, pero desde un punto de vista más amplio, se reconoce generalmente que las regiones espaciales que se extienden más allá de los límites exteriores de los siete superuniversos constituyen los dominios del Absoluto Incalificado.

\par
%\textsuperscript{(130.4)}
\textsuperscript{12:2.2} Aunque el ojo humano sólo puede ver a simple vista dos o tres nebulosas más allá de las fronteras del superuniverso de Orvonton, vuestros telescopios revelan literalmente millones y millones de estos universos físicos en proceso de formación. La mayoría de los reinos estelares expuestos a la investigación visual de vuestros telescopios modernos se encuentran en Orvonton, pero con la técnica fotográfica, los telescopios más potentes penetran mucho más allá de las fronteras del gran universo, llegando hasta los dominios del espacio exterior donde innumerables universos están en proceso de organización. Y existen además otros millones de universos que están fuera del alcance de vuestros instrumentos actuales.

\par
%\textsuperscript{(130.5)}
\textsuperscript{12:2.3} En un futuro poco lejano, los nuevos telescopios revelarán a la mirada asombrada de los astrónomos urantianos no menos de 375 millones de nuevas galaxias en las lejanas extensiones del espacio exterior. Al mismo tiempo, estos telescopios más potentes revelarán que muchos universos islas que anteriormente se creía que estaban en el espacio exterior, forman parte en realidad del sistema galáctico de Orvonton. Los siete superuniversos están creciendo todavía; la periferia de cada uno de ellos se expande gradualmente; constantemente se estabilizan y organizan nuevas nebulosas; y algunas nebulosas que los astrónomos urantianos consideran como extragalácticas, se encuentran en realidad en los márgenes de Orvonton y viajan junto con nosotros.

\par
%\textsuperscript{(131.1)}
\textsuperscript{12:2.4} Los astrónomos de Uversa observan que el gran universo está rodeado por los antepasados de una serie de enjambres estelares y planetarios que envuelven por completo a la creación actualmente habitada como anillos concéntricos compuestos de numerosos universos exteriores. Los físicos de Uversa calculan que la energía y la materia de estas regiones exteriores inexploradas igualan muchas veces ya el total de la masa material y de la carga energética que contienen los siete superuniversos. Nos han informado que la metamorfosis de la fuerza cósmica en estos niveles del espacio exterior es una actividad de los organizadores de fuerza del Paraíso. Sabemos también que estas fuerzas son ancestrales a las energías físicas que activan actualmente al gran universo. Sin embargo, los directores del poder de Orvonton no tienen nada que ver con estos reinos tan lejanos, y los movimientos energéticos que se producen allí tampoco están conectados de manera discernible con los circuitos de poder de las creaciones organizadas y habitadas.

\par
%\textsuperscript{(131.2)}
\textsuperscript{12:2.5} Sabemos muy poca cosa sobre el significado de estos fenómenos extraordinarios del espacio exterior. Una creación futura más grande está en proceso de formación. Podemos observar su inmensidad, discernir su extensión y percibir sus dimensiones majestuosas, pero aparte de esto, sobre estos reinos sabemos poco más que lo que conocen los astrónomos de Urantia. Por lo que sabemos, en este anillo exterior de nebulosas, soles y planetas no existen ni seres materiales de la orden de los humanos, ni ángeles u otras criaturas espirituales. Este lejano territorio se encuentra más allá de la jurisdicción y de la administración de los gobiernos de los superuniversos.

\par
%\textsuperscript{(131.3)}
\textsuperscript{12:2.6} En todo Orvonton se cree que se está gestando un nuevo tipo de creación, una clase de universos destinada a convertirse en el escenario de las actividades futuras del Cuerpo de la Finalidad que se está agrupando; y si nuestras suposiciones son correctas, entonces el futuro interminable puede deparar a todos vosotros los mismos espectáculos cautivadores que el pasado sin fin reservó a vuestros mayores y a vuestros predecesores.

\section*{3. La gravedad universal}
\par
%\textsuperscript{(131.4)}
\textsuperscript{12:3.1} Todas las formas de la energía-fuerza ---material, mental o espiritual--- están sometidas de la misma manera a esas atracciones, a esas presencias universales, que llamamos gravedad. La personalidad también es sensible a la gravedad ---al circuito exclusivo del Padre; pero aunque este circuito es exclusivo del Padre, no está excluido de los otros circuitos; el Padre Universal es infinito y actúa en los cuatro circuitos de gravedad absoluta del universo maestro, en \textit{todos} ellos:

\par
%\textsuperscript{(131.5)}
\textsuperscript{12:3.2} 1. La gravedad de personalidad del Padre Universal.

\par
%\textsuperscript{(131.6)}
\textsuperscript{12:3.3} 2. La gravedad espiritual del Hijo Eterno.

\par
%\textsuperscript{(131.7)}
\textsuperscript{12:3.4} 3. La gravedad mental del Actor Conjunto.

\par
%\textsuperscript{(131.8)}
\textsuperscript{12:3.5} 4. La gravedad cósmica de la Isla del Paraíso.

\par
%\textsuperscript{(131.9)}
\textsuperscript{12:3.6} Estos cuatro circuitos no están relacionados con el centro de fuerza del Paraíso inferior; no son circuitos de fuerza, ni de energía, ni de poder. Son circuitos de \textit{presencia} absolutos y, al igual que Dios, son independientes del tiempo y del espacio.

\par
%\textsuperscript{(132.1)}
\textsuperscript{12:3.7} A este respecto, es interesante hacer constar algunas observaciones realizadas en Uversa durante los recientes milenios por el cuerpo de investigadores de la gravedad. Este experto grupo de trabajadores ha llegado a las conclusiones siguientes en relación con los diferentes sistemas de gravedad del universo maestro:

\par
%\textsuperscript{(132.2)}
\textsuperscript{12:3.8} 1. \textit{La gravedad física.} Después de formular una estimación del total de toda la capacidad que tiene el gran universo para la gravedad física, han efectuado laboriosamente una comparación entre este descubrimiento y el total estimado para la presencia de la gravedad absoluta actualmente en vigor. Estos cálculos indican que la acción total de la gravedad en el gran universo es una parte muy pequeña de la atracción de la gravedad estimada del Paraíso, calculada sobre la base de la reacción gravitatoria de las unidades físicas básicas de la materia universal. Estos investigadores llegan a la asombrosa conclusión de que el universo central y los siete superuniversos que lo rodean sólo están utilizando actualmente alrededor de un cinco por ciento del funcionamiento activo de la atracción gravitatoria absoluta del Paraíso. En otras palabras: en el momento actual, cerca del noventa y cinco por ciento de la acción activa de la Isla del Paraíso sobre la gravedad cósmica, calculada según esta teoría de totalidad, está dedicada a controlar unos sistemas materiales situados mas allá de las fronteras de los universos organizados actuales. Todos estos cálculos se refieren a la gravedad absoluta; la gravedad lineal es un fenómeno interactivo que sólo se puede calcular conociendo la gravedad efectiva del Paraíso.

\par
%\textsuperscript{(132.3)}
\textsuperscript{12:3.9} 2. \textit{La gravedad espiritual.} Utilizando la misma técnica de estimación y de cálculo comparativos, estos investigadores han explorado la capacidad de reacción actual de la gravedad espiritual y, con la cooperación de los Mensajeros Solitarios y de otras personalidades espirituales, han llegado a la suma total de la gravedad espiritual activa de la Fuente-Centro Segunda. Y es muy instructivo señalar que encuentran casi el mismo valor para la presencia real y funcional de la gravedad espiritual en el gran universo que lo que dan por sentado con respecto al total actual de la gravedad espiritual activa. Dicho de otra manera: en el momento actual, prácticamente toda la gravedad espiritual del Hijo Eterno, calculada según esta teoría de totalidad, se puede observar funcionando en el gran universo. Si estos resultados son fiables, podemos concluir que los universos que evolucionan ahora en el espacio exterior son en el momento presente enteramente no espirituales. Y si esto es así, explicaría satisfactoriamente por qué los seres dotados de espíritu poseen tan poca o ninguna información sobre estas enormes manifestaciones de energía, aparte de conocer el hecho de su existencia física.

\par
%\textsuperscript{(132.4)}
\textsuperscript{12:3.10} 3. \textit{La gravedad mental.} Utilizando estos mismos principios del cálculo comparativo, estos expertos han atacado el problema de la presencia de la gravedad mental y de la reacción a la misma. La unidad mental de estimación se consiguió calculando el promedio de tres tipos de mentalidad material y tres tipos de mentalidad espiritual, aunque el tipo de mente que se encontró en los directores del poder y en sus asociados resultó ser un factor perturbador en el esfuerzo por llegar a una unidad básica para poder estimar la gravedad mental. Había pocas cosas que impidieran estimar la capacidad actual de la Fuente-Centro Tercera para actuar sobre la gravedad mental de acuerdo con esta teoría de totalidad. Aunque en este caso los resultados no son tan concluyentes como en las estimaciones de la gravedad física y espiritual, considerados comparativamente son muy instructivos e incluso curiosos. Estos investigadores deducen que cerca del ochenta y cinco por ciento de la respuesta de la gravedad mental a la atracción intelectual del Actor Conjunto tiene su origen en el gran universo existente. Esto sugeriría la posibilidad de que hay actividades mentales que están implicadas en las actividades físicas observables que se encuentran ahora en curso en todas las regiones del espacio exterior. Aunque esta estimación está probablemente lejos de ser exacta, concuerda en principio con nuestra creencia de que los organizadores de fuerza inteligentes dirigen ahora la evolución del universo en los niveles espaciales situados más allá de los límites exteriores actuales del gran universo. Cualquiera que sea la naturaleza de esta supuesta inteligencia, no parece sensible a la gravedad espiritual.

\par
%\textsuperscript{(133.1)}
\textsuperscript{12:3.11} Pero todos estos cálculos son, en el mejor de los casos, unas estimaciones basadas en supuestas leyes. Creemos que son bastante fiables. Aunque algunos seres espirituales estuvieran situados en el espacio exterior, su presencia colectiva no influiría notablemente sobre estos cálculos que implican unas mediciones tan enormes.

\par
%\textsuperscript{(133.2)}
\textsuperscript{12:3.12} \textit{La Gravedad de Personalidad} no es calculable. Reconocemos el circuito, pero no podemos medir ninguna realidad cualitativa o cuantitativa que responda a él.

\section*{4. El espacio y el movimiento}
\par
%\textsuperscript{(133.3)}
\textsuperscript{12:4.1} Todas las unidades de la energía cósmica están en rotación primaria, están dedicadas a ejecutar su misión mientras giran alrededor de la órbita universal. Los universos del espacio y los sistemas y los mundos que los componen son todos esferas que giran, que circulan a lo largo de los circuitos sin fin de los niveles espaciales del universo maestro. Nada en absoluto es estacionario en todo el universo maestro, salvo el centro mismo de Havona, la Isla eterna del Paraíso, el centro de la gravedad.

\par
%\textsuperscript{(133.4)}
\textsuperscript{12:4.2} El Absoluto Incalificado está funcionalmente limitado al espacio, pero no estamos tan seguros en cuanto a la relación de este Absoluto con el movimiento. ¿Es el movimiento inherente a él? No lo sabemos. Sabemos que el movimiento no es inherente al espacio; incluso los movimientos \textit{del} espacio no son innatos. Pero no estamos tan seguros en cuanto a la relación del Incalificado con el movimiento. ¿Quién, o qué, es realmente responsable de las gigantescas actividades consistentes en las transmutaciones de la energía-fuerza que se están produciendo ahora más allá de las fronteras de los siete superuniversos actuales? En lo que concierne al origen del movimiento, tenemos las opiniones siguientes:

\par
%\textsuperscript{(133.5)}
\textsuperscript{12:4.3} 1. Creemos que el Actor Conjunto da comienzo al movimiento \textit{en} el espacio.

\par
%\textsuperscript{(133.6)}
\textsuperscript{12:4.4} 2. Si el Actor Conjunto es el que produce los movimientos \textit{del} espacio, no podemos probarlo.

\par
%\textsuperscript{(133.7)}
\textsuperscript{12:4.5} 3. El Absoluto Universal no causa el movimiento inicial, pero sí iguala y controla todas las tensiones originadas por el movimiento.

\par
%\textsuperscript{(133.8)}
\textsuperscript{12:4.6} En el espacio exterior, los organizadores de la fuerza parecen ser los responsables de la producción de las gigantescas ruedas de universos que se encuentran ahora en proceso de evolución estelar, pero su capacidad para actuar así debe haber sido posibilitada por alguna modificación de la presencia espacial del Absoluto Incalificado.

\par
%\textsuperscript{(133.9)}
\textsuperscript{12:4.7} Desde el punto de vista humano, el espacio es la nada ---negativo; sólo existe en relación con algo positivo y no espacial. Sin embargo, el espacio es real. Contiene y condiciona el movimiento. E incluso se mueve. Los movimientos del espacio se pueden clasificar más o menos como sigue:

\par
%\textsuperscript{(133.10)}
\textsuperscript{12:4.8} 1. El movimiento primario ---la respiración del espacio, el movimiento del espacio mismo.

\par
%\textsuperscript{(133.11)}
\textsuperscript{12:4.9} 2. El movimiento secundario ---las rotaciones direccionales alternas de los niveles espaciales sucesivos.

\par
%\textsuperscript{(133.12)}
\textsuperscript{12:4.10} 3. Los movimientos relativos ---relativos en el sentido de que no son evaluados tomando como punto de base al Paraíso. Los movimientos primario y secundario son absolutos, son el movimiento en relación con el Paraíso inmóvil.

\par
%\textsuperscript{(133.13)}
\textsuperscript{12:4.11} 4. El movimiento compensatorio o correlativo destinado a coordinar todos los demás movimientos.

\par
%\textsuperscript{(134.1)}
\textsuperscript{12:4.12} Las relaciones actuales entre vuestro Sol y sus planetas asociados, aunque revelan muchos movimientos relativos y absolutos en el espacio, tienden a dar la impresión a los observadores astronómicos de que estáis comparativamente estacionarios en el espacio y de que los enjambres y corrientes de estrellas circundantes están lanzados en una huida hacia el exterior a velocidades siempre crecientes a medida que vuestros cálculos alcanzan espacios más alejados. Pero éste no es el caso. Olvidáis reconocer que las creaciones físicas de todo el espacio penetrado se encuentran actualmente en una expansión uniforme hacia el exterior. Vuestra propia creación local (Nebadon) participa en este movimiento de expansión universal hacia el exterior. La totalidad de los siete superuniversos, junto con las regiones exteriores del universo maestro, participan en los ciclos de dos mil millones de años de la respiración del espacio.

\par
%\textsuperscript{(134.2)}
\textsuperscript{12:4.13} Cuando los universos se expanden y se contraen, las masas materiales del espacio penetrado se mueven alternativamente a favor o en contra de la atracción de la gravedad del Paraíso. El trabajo que se efectúa al mover la masa energética material de la creación es un trabajo del \textit{espacio}, y no un trabajo de la \textit{energía-poder}.

\par
%\textsuperscript{(134.3)}
\textsuperscript{12:4.14} Aunque vuestras estimaciones espectroscópicas de las velocidades astronómicas son bastante fiables cuando se aplican a los reinos estelares pertenecientes a vuestro superuniverso y a los superuniversos asociados, estos cálculos carecen por completo de fiabilidad cuando se refieren a los dominios del espacio exterior. Las líneas espectrales se desplazan desde lo normal hacia el violeta para una estrella que se acerca; estas líneas se desplazan igualmente hacia el rojo para una estrella que se aleja. Muchas influencias se interponen para dar la impresión de que la velocidad de recesión de los universos exteriores aumenta a razón de más de ciento sesenta kilómetros por segundo por cada millón de años-luz que aumente la distancia. Después de que se perfeccionen unos telescopios más potentes, con este método de cálculo parecerá que estos sistemas tan remotos se alejan de esta parte del universo a la velocidad increíble de cerca de cincuenta mil kilómetros por segundo. Pero esta velocidad aparente de recesión no es real; es el resultado de numerosos factores erróneos entre los que se incluyen los ángulos de observación y otras distorsiones del espacio-tiempo.

\par
%\textsuperscript{(134.4)}
\textsuperscript{12:4.15} Pero la más importante de todas estas distorsiones se produce porque los inmensos universos del espacio exterior, situados en los reinos próximos a los dominios de los siete superuniversos, parecen girar en dirección contraria a la del gran universo. Es decir, esas miríadas de nebulosas, y los soles y las esferas que las acompañan, giran en la actualidad en el sentido de las agujas del reloj alrededor de la creación central. Los siete superuniversos giran alrededor del Paraíso en dirección opuesta a las agujas del reloj. Parece ser que el segundo universo exterior de galaxias, al igual que los siete superuniversos, gira en sentido opuesto a las agujas del reloj alrededor del Paraíso. Y los observadores astronómicos de Uversa creen haber detectado la prueba de movimientos rotatorios, en un tercer cinturón exterior de espacio muy lejano, que están empezando a manifestar la tendencia a orientarse en el sentido de las agujas del reloj.

\par
%\textsuperscript{(134.5)}
\textsuperscript{12:4.16} Es probable que estas direcciones alternas de las sucesivas procesiones espaciales de los universos tengan alguna relación con la técnica de la gravedad empleada por el Absoluto Universal en el interior del universo maestro, una técnica que consiste en coordinar las fuerzas y en igualar las tensiones espaciales. El movimiento, al igual que el espacio, es un complemento o un equilibrador de la gravedad.

\section*{5. El espacio y el tiempo}
\par
%\textsuperscript{(134.6)}
\textsuperscript{12:5.1} Al igual que el espacio, el tiempo es un don del Paraíso, pero no en el mismo sentido, sino sólo indirectamente. El tiempo surge en virtud del movimiento y porque la mente es inherentemente consciente de las secuencias. Desde un punto de vista práctico, el movimiento es esencial para el tiempo, pero no existe ninguna unidad de tiempo universal basada en el movimiento, salvo en la medida en que el día oficial del Paraíso-Havona es reconocido arbitrariamente como tal unidad. La totalidad de la respiración del espacio destruye su valor local como fuente del tiempo.

\par
%\textsuperscript{(135.1)}
\textsuperscript{12:5.2} El espacio no es infinito, aunque tenga su origen en el Paraíso; no es absoluto, pues está penetrado por el Absoluto Incalificado. No conocemos los límites absolutos del espacio, pero sí sabemos que el absoluto del tiempo es la eternidad.

\par
%\textsuperscript{(135.2)}
\textsuperscript{12:5.3} El tiempo y el espacio sólo son inseparables en las creaciones del espacio-tiempo, en los siete superuniversos. El espacio intemporal (el espacio sin tiempo) existe teóricamente, pero el único lugar verdaderamente intemporal es el \textit{área} del Paraíso. El tiempo no espacial (el tiempo sin espacio) existe en la mente del nivel funcional del Paraíso.

\par
%\textsuperscript{(135.3)}
\textsuperscript{12:5.4} Las zonas relativamente inmóviles de espacio intermedio que entran en contacto con el Paraíso y que separan al espacio penetrado del espacio no penetrado son las zonas de transición entre el tiempo y la eternidad, de ahí la necesidad de que los peregrinos que se dirigen hacia el Paraíso se vuelvan inconscientes durante este tránsito cuando ha de culminar en la ciudadanía del Paraíso. Los \textit{visitantes} conscientes del tiempo pueden ir al Paraíso sin dormir de esta manera, pero siguen siendo criaturas del tiempo.

\par
%\textsuperscript{(135.4)}
\textsuperscript{12:5.5} Las relaciones con el tiempo no existen sin un movimiento en el espacio, pero la conciencia del tiempo sí existe. Las secuencias pueden llevar a la conciencia del tiempo incluso en ausencia de movimiento. La mente del hombre está menos atada al tiempo que al espacio debido a la naturaleza inherente de la mente. Incluso durante los tiempos de la vida terrestre en la carne, aunque la mente del hombre esté rígidamente atada al espacio, la imaginación creativa humana está comparativamente libre del tiempo. Pero el tiempo mismo no es genéticamente una cualidad de la mente.

\par
%\textsuperscript{(135.5)}
\textsuperscript{12:5.6} Existen tres niveles diferentes de conocimiento del tiempo:

\par
%\textsuperscript{(135.6)}
\textsuperscript{12:5.7} 1. El tiempo percibido por la mente ---la conciencia de las secuencias, del movimiento y un sentido de la duración.

\par
%\textsuperscript{(135.7)}
\textsuperscript{12:5.8} 2. El tiempo percibido por el espíritu ---la percepción del movimiento hacia Dios y la conciencia del movimiento ascendente hacia niveles de divinidad creciente.

\par
%\textsuperscript{(135.8)}
\textsuperscript{12:5.9} 3. La personalidad \textit{crea} un sentido único del tiempo mediante su percepción de la Realidad, más una conciencia de la presencia y un conocimiento de la duración.

\par
%\textsuperscript{(135.9)}
\textsuperscript{12:5.10} Los animales no espirituales sólo conocen el pasado y viven en el presente. Los hombres habitados por el espíritu tienen poderes de previsión (perspicacia); pueden visualizar el futuro. Sólo las actitudes progresistas y orientadas hacia adelante son personalmente reales. La ética estática y la moralidad tradicional sólo superan ligeramente el nivel animal. El estoicismo tampoco es un tipo elevado de autorrealización. La ética y la moral se vuelven verdaderamente humanas cuando son dinámicas y progresistas, sensibles a la realidad universal.

\par
%\textsuperscript{(135.10)}
\textsuperscript{12:5.11} La personalidad humana no es simplemente una cosa que acompaña a los acontecimientos del tiempo y del espacio; la personalidad humana también puede actuar como causa cósmica de esos acontecimientos.

\section*{6. El supercontrol universal}
\par
%\textsuperscript{(135.11)}
\textsuperscript{12:6.1} El universo no es estático. La estabilidad no es el resultado de la inercia, sino más bien el producto de unas energías equilibradas, unas mentes que cooperan, unas morontias coordinadas, un supercontrol del espíritu y una unificación de la personalidad. La estabilidad siempre es enteramente proporcional a la divinidad.

\par
%\textsuperscript{(135.12)}
\textsuperscript{12:6.2} En el control físico del universo maestro, el Padre Universal ejerce su prioridad y su primacía por medio de la Isla del Paraíso; Dios es absoluto en la administración espiritual del cosmos mediante la persona del Hijo Eterno. En lo que se refiere al terreno de la mente, el Padre y el Hijo actúan de manera coordinada a través del Actor Conjunto.

\par
%\textsuperscript{(136.1)}
\textsuperscript{12:6.3} La Fuente-Centro Tercera ayuda a mantener el equilibrio y la coordinación de las energías y de las organizaciones físicas y espirituales combinadas mediante la absolutidad de su control sobre la mente cósmica y mediante el ejercicio de sus complementos inherentes y universales de gravedad física y espiritual. En cualquier momento y lugar en que se produce una conexión entre lo material y lo espiritual, este fenómeno mental es un acto del Espíritu Infinito. Sólo la mente puede interasociar las fuerzas y las energías físicas del nivel material con los poderes y los seres espirituales del nivel del espíritu.

\par
%\textsuperscript{(136.2)}
\textsuperscript{12:6.4} Cada vez que examinéis los fenómenos universales, aseguraos de que tomáis en consideración la interrelación de las energías físicas, intelectuales y espirituales, y de que tenéis debidamente en cuenta los fenómenos inesperados que aparecen a causa de su unificación por medio de la personalidad, y los fenómenos imprevisibles producidos por las acciones y reacciones de la Deidad experiencial y de los Absolutos.

\par
%\textsuperscript{(136.3)}
\textsuperscript{12:6.5} El universo sólo es muy previsible en el sentido cuantitativo o de medición de la gravedad; incluso las fuerzas físicas fundamentales no responden a la gravedad lineal, ni tampoco lo hacen los significados mentales superiores ni los verdaderos valores espirituales de las realidades últimas del universo. Cualitativamente, el universo no es muy previsible en cuanto a las nuevas asociaciones de fuerzas, ya sean físicas, mentales o espirituales, aunque muchas de estas combinaciones de energías o de fuerzas se vuelven parcialmente previsibles cuando son sometidas a una observación crítica. Cuando la materia, la mente y el espíritu están unificados por la personalidad de una criatura, somos incapaces de predecir plenamente las decisiones de ese ser dotado de libre albedrío.

\par
%\textsuperscript{(136.4)}
\textsuperscript{12:6.6} Todas las fases de la fuerza primordial, del espíritu naciente y de otras ultimidades no personales parecen reaccionar de acuerdo con ciertas leyes relativamente estables pero desconocidas, y están caracterizadas por una amplitud de actuación y una flexibilidad de reacción que son a menudo desconcertantes cuando se las encuentra en los fenómenos de una situación circunscrita y aislada. ¿Cuál es la explicación de que estas realidades universales emergentes revelen esta imprevisible libertad de reacción? Estos sucesos imprevisibles, desconocidos e insondables ---ya se trate del comportamiento de una unidad primordial de fuerza, de la reacción de un nivel mental no identificado, o del fenómeno de un inmenso preuniverso en potencia en los dominios del espacio exterior ---revelan probablemente las actividades del Último y las actuaciones de la presencia de los Absolutos, que son anteriores a la actividad de todos los Creadores universales.

\par
%\textsuperscript{(136.5)}
\textsuperscript{12:6.7} No lo sabemos realmente, pero suponemos que una variedad de talentos tan asombrosa y una coordinación tan profunda significan que los Absolutos están presentes y actúan, y que esta diversidad de reacciones, en presencia de una causalidad aparentemente uniforme, revela la reacción de los Absolutos no sólo a la causalidad inmediata de una situación, sino también a todas las otras causalidades relacionadas, en todas partes del universo maestro.

\par
%\textsuperscript{(136.6)}
\textsuperscript{12:6.8} Los individuos tienen sus guardianes del destino; los planetas, sistemas, constelaciones, universos y superuniversos tienen cada uno de ellos sus gobernantes respectivos que trabajan por el bien de sus dominios. Havona e incluso el gran universo están cuidados por aquellos a quienes se les han confiado estas elevadas responsabilidades. Pero ¿quién fomenta y se ocupa de las necesidades fundamentales del universo maestro como un todo, desde el Paraíso hasta el cuarto y último nivel del espacio exterior? Existencialmente, este cuidado se puede atribuir probablemente a la Trinidad del Paraíso, pero desde un punto de vista experiencial, la aparición de los universos posteriores a Havona depende:

\par
%\textsuperscript{(136.7)}
\textsuperscript{12:6.9} 1. De los Absolutos en cuanto al potencial.

\par
%\textsuperscript{(136.8)}
\textsuperscript{12:6.10} 2. Del Último en cuanto a la dirección.

\par
%\textsuperscript{(137.1)}
\textsuperscript{12:6.11} 3. Del Supremo en cuanto a la coordinación evolutiva.

\par
%\textsuperscript{(137.2)}
\textsuperscript{12:6.12} 4. De los Arquitectos del Universo Maestro en cuanto a la administración anterior a la aparición de los gobernantes específicos.

\par
%\textsuperscript{(137.3)}
\textsuperscript{12:6.13} El Absoluto Incalificado penetra todo el espacio. No tenemos del todo claro el estado exacto del Absoluto de la Deidad y del Absoluto Universal, pero sabemos que este último ejerce su actividad dondequiera que actúan el Absoluto de la Deidad y el Absoluto Incalificado. El Absoluto de la Deidad puede estar universalmente presente, pero difícilmente está presente en el espacio. El Último está presente en el espacio, o lo estará alguna vez, hasta los márgenes exteriores del cuarto nivel de espacio. Dudamos que el
Último esté nunca espacialmente presente más allá de la periferia del universo maestro, pero dentro de estos límites, el Último está integrando progresivamente la organización creativa de los potenciales de los tres Absolutos.

\section*{7. La parte y el todo}
\par
%\textsuperscript{(137.4)}
\textsuperscript{12:7.1} Existe una ley inexorable e impersonal que está en vigor a lo largo de todo el tiempo y del espacio y con respecto a toda realidad de cualquier naturaleza que sea; esta ley equivale al funcionamiento de una providencia cósmica. La misericordia caracteriza la actitud amorosa de Dios por el individuo; la imparcialidad motiva la actitud de Dios hacia la totalidad. La voluntad de Dios no prevalece necesariamente en la parte ---en el corazón de una personalidad determinada--- pero su voluntad gobierna realmente el todo, el universo de universos.

\par
%\textsuperscript{(137.5)}
\textsuperscript{12:7.2} En todas las relaciones de Dios con todos sus seres, es cierto que sus leyes no son inherentemente arbitrarias. Para vosotros, con vuestra visión limitada y vuestro punto de vista finito, los actos de Dios deben parecer a menudo dictatoriales y arbitrarios. Las leyes de Dios son simplemente los hábitos de Dios, su manera de hacer las cosas repetidas veces; y él siempre hace bien todas las cosas. Observáis que Dios hace la misma cosa de la misma manera, repetidas veces, sencillamente porque esa es la mejor manera de hacer esa cosa particular en una circunstancia dada; y la mejor manera es la manera correcta. Por eso la sabiduría infinita ordena siempre que se haga de esa manera precisa y perfecta. Deberíais recordar también que la naturaleza no es el acto exclusivo de la Deidad; otras influencias están presentes en esos fenómenos que el hombre llama naturaleza.

\par
%\textsuperscript{(137.6)}
\textsuperscript{12:7.3} Sufrir cualquier tipo de deterioro o permitir que un acto puramente personal se ejecute alguna vez de manera inferior, es incompatible con la naturaleza divina. Sin embargo debemos indicar claramente que, \textit{si} en la divinidad de cualquier situación, en el extremo de cualquier circunstancia, si en cualquier caso en que la línea de la sabiduría suprema pudiera indicar que se exige una conducta diferente ---si las exigencias de la perfección ordenaran por alguna razón otro método de reacción, uno mejor, el Dios omnisapiente actuaría inmediatamente de esa manera mejor y más adecuada. Esto supondría la expresión de una ley superior, y no la revocación de una ley inferior.

\par
%\textsuperscript{(137.7)}
\textsuperscript{12:7.4} Dios no es un esclavo atado por la costumbre a la repetición crónica de sus propios actos voluntarios. No existe ningún conflicto entre las leyes del Infinito; todas son perfecciones de su naturaleza infalible; todas son los actos incuestionables que expresan unas decisiones sin defecto. La ley es la reacción invariable de una mente infinita, perfecta y divina. Todos los actos de Dios son volitivos, a pesar de esta uniformidad aparente. En Dios <<\textit{no existe ni variabilidad ni sombra de cambio}>>\footnote{\textit{En Dios no hay variabilidad}: Stg 1:17.}. Pero todo esto que se puede decir en verdad del Padre Universal, no se puede decir con igual certeza de todas sus inteligencias subordinadas o de sus criaturas evolutivas.

\par
%\textsuperscript{(137.8)}
\textsuperscript{12:7.5} Puesto que Dios es invariable, podéis contar pues con que hará lo mismo, en todas las circunstancias corrientes, de la misma manera idéntica y habitual. Dios es la seguridad de la estabilidad para todas las cosas y todos los seres creados. Él es Dios, por consiguiente no cambia\footnote{\textit{Dios no cambia}: Mal 3:6.}.

\par
%\textsuperscript{(138.1)}
\textsuperscript{12:7.6} Toda esta conducta constante y toda esta acción uniforme es personal, consciente y altamente volitiva, porque el gran Dios no es un esclavo indefenso de su propia perfección e infinidad. Dios no es una fuerza automática que actúa por sí sola; no es un poder servil atado a la ley. Dios no es ni una ecuación matemática ni una fórmula química. Es una personalidad primordial y con libre albedrío. Es el Padre Universal, un ser sobrecargado de personalidad y la fuente universal de la personalidad de todas las criaturas.

\par
%\textsuperscript{(138.2)}
\textsuperscript{12:7.7} La voluntad de Dios no prevalece de manera uniforme en el corazón de los mortales materiales que buscan a Dios, pero si se amplía el marco del tiempo más allá del momento presente hasta abarcar la totalidad de la primera vida, entonces la voluntad de Dios se hace cada vez más discernible en los frutos del espíritu producidos en la vida de los hijos de Dios guiados por el espíritu. Luego, si la vida humana se amplía aún más hasta incluir la experiencia morontial, se observa que la voluntad divina brilla de manera cada vez más intensa en los actos cada vez más espirituales de las criaturas del tiempo que han empezado a saborear las delicias divinas de experimentar la relación de la personalidad del hombre con la personalidad del Padre Universal.

\par
%\textsuperscript{(138.3)}
\textsuperscript{12:7.8} La Paternidad de Dios y la fraternidad de los hombres presentan la paradoja de la parte y del todo al nivel de la personalidad. Dios ama a \textit{cada} individuo como a un hijo particular de la familia celestial. Sin embargo, Dios ama así a \textit{todos} los individuos; no hace acepción de personas\footnote{\textit{Dios no hace acepción de personas}: 2 Cr 19:7; Job 34:19; Eclo 35:12; Hch 10:34; Ro 2:11; Gl 2:6; 3:28; Ef 6:9; Col 3:11.}, y la universalidad de su amor engendra una relación de totalidad, la fraternidad universal.

\par
%\textsuperscript{(138.4)}
\textsuperscript{12:7.9} El amor del Padre individualiza de manera absoluta a cada personalidad como hijo único del Padre Universal, un hijo sin duplicado en la infinidad, una criatura volitiva irreemplazable en toda la eternidad. El amor del Padre glorifica a cada hijo de Dios, iluminando a cada miembro de la familia celestial, destacando claramente la naturaleza única de cada ser personal, frente a los niveles impersonales situados fuera del círculo fraternal del Padre de todos. El amor de Dios describe de manera impresionante el valor trascendente de cada criatura volitiva, revela inequívocamente el alto valor que el Padre Universal ha atribuido a todos y a cada uno de sus hijos, desde la más alta personalidad creadora con rango paradisiaco hasta la personalidad más humilde con dignidad volitiva entre las tribus salvajes de hombres en los albores de la especie humana en algún mundo evolutivo del tiempo y del espacio.

\par
%\textsuperscript{(138.5)}
\textsuperscript{12:7.10} El mismo amor de Dios por el individuo engendra la familia divina de todos los individuos, la fraternidad universal de los hijos del Padre Paradisiaco dotados de libre albedrío. Y como esta fraternidad es universal, es una relación de totalidad. Cuando la fraternidad es universal, no revela la relación con \textit{cadauno,} sino la relación con \textit{todos.} La fraternidad es una realidad de la totalidad, y revela por tanto las cualidades del conjunto en contraste con las cualidades de la parte.

\par
%\textsuperscript{(138.6)}
\textsuperscript{12:7.11} La fraternidad constituye una relación de hecho entre todas las personalidades en la existencia universal. Ninguna persona puede evitar los beneficios o los perjuicios que pueden surgir como resultado de una relación con otras personas. La parte se beneficia o sufre en proporción con el todo. El buen esfuerzo de cada hombre beneficia a todos los hombres; el error o el mal de cada hombre aumenta las tribulaciones de todos los hombres. Así como se mueve la parte se mueve el todo. Según sea el progreso del todo, así será el progreso de la parte. Las velocidades relativas de la parte y del todo determinan si la parte se retrasa por la inercia del todo, o si es conducida hacia adelante por el impulso de la fraternidad cósmica.

\par
%\textsuperscript{(139.1)}
\textsuperscript{12:7.12} Es un misterio que Dios sea un ser extremadamente personal y consciente de sí mismo con una sede central residencial, y que al mismo tiempo esté personalmente presente en un universo tan inmenso y en contacto personal con un número de seres casi infinito. El hecho de que este fenómeno sea un misterio que sobrepasa la comprensión humana no debería disminuir en lo más mínimo vuestra fe. No permitáis que la magnitud de la infinidad, la inmensidad de la eternidad y la grandiosidad y la gloria del carácter incomparable de Dios os intimiden, os hagan vacilar u os desanimen, pues el Padre no está muy lejos de ninguno de vosotros; vive dentro de vosotros, y en él todos nos movemos literalmente\footnote{\textit{En Dios nos movemos y existimos}: Hch 17:28.}, vivimos realmente y tenemos verdaderamente nuestra existencia\footnote{\textit{Dios vive en nosotros}: Job 32:8,18; Is 63:10-11; Ez 37:14; Mt 10:20; Lc 17:21; Jn 17:21-23; Ro 8:9-11; 1 Co 3:16-17; 6:19; 2 Co 6:16; Gl 2:20; 1 Jn 3:24; 4:12; Ap 21:3.}.

\par
%\textsuperscript{(139.2)}
\textsuperscript{12:7.13} Aunque el Padre Paradisiaco actúa a través de sus creadores divinos y de sus hijos creados, disfruta también del contacto interior más íntimo con vosotros, un contacto tan sublime, tan sumamente personal, que se encuentra incluso más allá de mi comprensión ---se trata de esa misteriosa comunión de un fragmento del Padre con el alma humana y con la mente mortal donde habita realmente. Sabiendo lo que sabéis sobre estos dones de Dios, sabéis por lo tanto que el Padre está en contacto íntimo no sólo con sus asociados divinos, sino también con sus hijos mortales evolutivos del tiempo. El Padre reside realmente en el Paraíso, pero su presencia divina habita también en la mente de los hombres.

\par
%\textsuperscript{(139.3)}
\textsuperscript{12:7.14} Aunque el espíritu de un Hijo haya sido derramado sobre todo el género humano, aunque un Hijo haya vivido en otro tiempo con vosotros en la similitud de la carne mortal, aunque los serafines os guarden y os guíen personalmente, ¿cómo puede esperar nunca cualquiera de estos seres divinos de los Centros Segundo y Tercero acercarse tanto a vosotros o comprenderos tan plenamente como el Padre, que ha dado una parte de sí mismo para que esté en vosotros, para que sea vuestro yo real y divino e incluso vuestro yo eterno?

\section*{8. La materia, la mente y el espíritu}
\par
%\textsuperscript{(139.4)}
\textsuperscript{12:8.1} <<\textit{Dios es espíritu}>>\footnote{\textit{Dios es espíritu}: Jn 4:24.}, pero el Paraíso no lo es. El universo material es siempre el terreno donde tienen lugar todas las actividades espirituales; los seres espirituales y los ascendentes espirituales viven y trabajan en esferas físicas de realidad material.

\par
%\textsuperscript{(139.5)}
\textsuperscript{12:8.2} La concesión de la fuerza cósmica, el ámbito de la gravedad cósmica, es una función de la Isla del Paraíso. Toda la energía-fuerza original procede del Paraíso, y la materia destinada a formar innumerables universos circula actualmente por todo el universo maestro bajo la forma de una presencia supergravitatoria que representa la carga de fuerza del espacio penetrado.

\par
%\textsuperscript{(139.6)}
\textsuperscript{12:8.3} Cualesquiera que sean las transformaciones de la fuerza en los universos exteriores, una vez que la fuerza ha salido del Paraíso continúa su viaje sometida a la atracción interminable, siempre presente e infalible, de la Isla eterna, dando vueltas para siempre de forma obediente e inherente alrededor de las órbitas espaciales eternas de los universos. La energía física es la única realidad que obedece de manera fiel y constante a la ley universal. Únicamente en el terreno de la volición de las criaturas es donde ha habido desviaciones de los caminos divinos y de los planes originales. El poder y la energía son las pruebas universales de la estabilidad, la constancia y la eternidad de la Isla central del Paraíso.

\par
%\textsuperscript{(139.7)}
\textsuperscript{12:8.4} La concesión del espíritu y la espiritualización de las personalidades, el terreno de la gravedad espiritual\footnote{\textit{Gravedad espiritual}: Jer 31:3; Jn 6:44; 12:32.}, es el dominio del Hijo Eterno. Y esta gravedad espiritual del Hijo, que atrae constantemente a todas las realidades espirituales hacia él, es tan real y absoluta como la todopoderosa atracción material de la Isla del Paraíso. Pero el hombre con mentalidad materialista está, de manera natural, más familiarizado con las manifestaciones materiales de naturaleza física que con las operaciones igualmente reales y poderosas de naturaleza espiritual que sólo la perspicacia espiritual del alma es capaz de discernir.

\par
%\textsuperscript{(140.1)}
\textsuperscript{12:8.5} A medida que la mente de cualquier personalidad del universo se vuelve más espiritual ---más semejante a Dios--- es menos sensible a la gravedad material. La realidad, medida por su respuesta a la gravedad física, es la antítesis de la realidad determinada por la calidad de su contenido espiritual. La acción de la gravedad física es un determinador cuantitativo de la energía no espiritual; la acción de la gravedad espiritual es la medida cualitativa de la energía viviente de la divinidad.

\par
%\textsuperscript{(140.2)}
\textsuperscript{12:8.6} Aquello que el Paraíso significa para la creación física, y aquello que el Hijo Eterno significa para el universo espiritual, el Actor Conjunto lo significa para el ámbito de la mente ---para el universo inteligente de los seres y de las personalidades materiales, morontiales y espirituales.

\par
%\textsuperscript{(140.3)}
\textsuperscript{12:8.7} El Actor Conjunto reacciona tanto a las realidades materiales como a las espirituales, y se convierte por tanto, de forma inherente, en el ministro universal para todos los seres inteligentes, unos seres que pueden representar una unión de las fases materiales y espirituales de la creación. El don de la inteligencia, el ministerio aportado a lo material y a lo espiritual en el fenómeno de la mente, es el dominio exclusivo del Actor Conjunto, que se convierte así en el asociado de la mente espiritual, en la esencia de la mente morontial y en la sustancia de la mente material de las criaturas evolutivas del tiempo.

\par
%\textsuperscript{(140.4)}
\textsuperscript{12:8.8} La mente es la técnica por medio de la cual las realidades espirituales se vuelven experienciales para las personalidades de las criaturas. A fin de cuentas, las posibilidades unificadoras de la mente humana misma, la capacidad para coordinar las cosas, las ideas y los valores, es supermaterial.

\par
%\textsuperscript{(140.5)}
\textsuperscript{12:8.9} Aunque a la mente mortal apenas le resulte posible comprender los siete niveles de la realidad cósmica relativa, el intelecto humano debería ser capaz de captar una gran parte del significado de los tres niveles funcionales de la realidad finita:

\par
%\textsuperscript{(140.6)}
\textsuperscript{12:8.10} 1. \textit{La materia.} La energía organizada que está sujeta a la gravedad lineal, excepto cuando es modificada por el movimiento y está condicionada por la mente.

\par
%\textsuperscript{(140.7)}
\textsuperscript{12:8.11} 2. \textit{La mente.} La conciencia organizada que no está totalmente sometida a la gravedad material, y que se vuelve realmente libre cuando es modificada por el espíritu.

\par
%\textsuperscript{(140.8)}
\textsuperscript{12:8.12} 3. \textit{El espíritu.} La realidad personal más elevada. El verdadero espíritu no está sujeto a la gravedad física, pero se vuelve finalmente la influencia motivadora de todos los sistemas energéticos evolutivos que poseen la dignidad de la personalidad.

\par
%\textsuperscript{(140.9)}
\textsuperscript{12:8.13} La meta de la existencia de todas las personalidades es el espíritu; las manifestaciones materiales son relativas, y la mente cósmica sirve de mediadora entre estos opuestos universales. La concesión de la mente y el ministerio del espíritu son obra de las personas asociadas de la Deidad, el Espíritu Infinito y el Hijo Eterno. La realidad total de la Deidad no es la mente sino la mente-espíritu ---el espíritu-mente unificado por la personalidad. Sin embargo, los absolutos tanto del espíritu como de las cosas convergen en la persona del Padre Universal.

\par
%\textsuperscript{(140.10)}
\textsuperscript{12:8.14} En el Paraíso, las tres energías física, mental y espiritual están coordinadas. En el cosmos evolutivo, la energía-materia es la que domina, excepto en la personalidad, donde el espíritu se esfuerza por conseguir la supremacía por mediación de la mente. El espíritu es la realidad fundamental de la experiencia de la personalidad de todas las criaturas, porque Dios es espíritu\footnote{\textit{Dios es espíritu}: Jn 4:24.}. El espíritu es invariable y, por lo tanto, en todas las relaciones entre personalidades, trasciende tanto a la mente como a la materia, que son variables experienciales de consecución progresiva.

\par
%\textsuperscript{(140.11)}
\textsuperscript{12:8.15} En la evolución cósmica, la materia se vuelve una sombra filosófica proyectada por la mente en presencia de la luminosidad espiritual de la iluminación divina, pero esto no invalida la realidad de la energía-materia. La mente, la materia y el espíritu son igualmente reales, pero en lo referente a alcanzar la divinidad no tienen el mismo valor para la personalidad. La conciencia de la divinidad es una experiencia espiritual progresiva.

\par
%\textsuperscript{(141.1)}
\textsuperscript{12:8.16} Cuanto más intenso es el brillo de la personalidad espiritualizada (del Padre en el universo, del fragmento de la personalidad espiritual potencial en la criatura individual) mayor es la sombra proyectada por la mente intermedia sobre su investidura material. En el tiempo, el cuerpo del hombre es tan real como la mente o el espíritu, pero cuando llega la muerte, tanto la mente (la identidad) como el espíritu sobreviven, mientras que el cuerpo no sobrevive. Una realidad cósmica puede no existir en la experiencia de la personalidad. Por eso vuestra figura retórica griega ---la materia es la sombra de la sustancia espiritual más real--- tiene de hecho un significado filosófico.

\section*{9. Las realidades personales}
\par
%\textsuperscript{(141.2)}
\textsuperscript{12:9.1} El espíritu es la realidad personal fundamental en los universos, y la personalidad es fundamental para todas las experiencias progresivas con la realidad espiritual. Cada fase de la experiencia de la personalidad en cada nivel sucesivo de progresión universal rebosa de indicios que conducen al descubrimiento de atractivas realidades personales. El verdadero destino del hombre consiste en crear metas nuevas y espirituales, y luego en responder a los atractivos cósmicos de esas metas celestiales que tienen un valor no material.

\par
%\textsuperscript{(141.3)}
\textsuperscript{12:9.2} El amor es el secreto de las asociaciones beneficiosas entre personalidades. No podéis conocer realmente a una persona como resultado de un solo encuentro. No podéis apreciar la música por medio de deducciones matemáticas, aunque la música sea una forma de ritmo matemático. El número que tiene asignado un abonado telefónico no identifica de ninguna manera a la personalidad de ese abonado, ni indica nada sobre su carácter.

\par
%\textsuperscript{(141.4)}
\textsuperscript{12:9.3} Las matemáticas, la ciencia material, es indispensable para discutir de manera inteligente los aspectos materiales del universo, pero este conocimiento no forma parte necesariamente de una comprensión más elevada de la verdad o de una apreciación personal de las realidades espirituales. No solamente en el terreno de la vida, sino también en el mundo de la energía física, la suma de dos o más cosas es muy a menudo algo \textit{más} que, o algo \textit{diferente} a, las consecuencias previsibles de la adición de esas uniones. Toda la ciencia de las matemáticas, el ámbito total de la filosofía, la física o la química más avanzadas, no podían predecir ni saber que la unión de dos átomos gaseosos de hidrógeno con un átomo gaseoso de oxígeno daría como resultado una sustancia nueva y cualitativamente sobreañadida ---el agua líquida. El conocimiento comprensivo de este solo fenómeno físico-químico debería haber impedido el desarrollo de la filosofía materialista y de la cosmología mecanicista.

\par
%\textsuperscript{(141.5)}
\textsuperscript{12:9.4} El análisis técnico no revela lo que una persona o una cosa pueden hacer. Por ejemplo: el agua se emplea eficazmente para apagar el fuego. Que el agua apaga el fuego es un hecho de la experiencia cotidiana, pero ningún análisis del agua podría haber revelado nunca que posee esta propiedad. El análisis determina que el agua está compuesta de hidrógeno y de oxígeno; un estudio adicional de estos elementos revelaría que el oxígeno es el verdadero soporte de la combustión y que el hidrógeno mismo arde libremente.

\par
%\textsuperscript{(141.6)}
\textsuperscript{12:9.5} Vuestra religión se está volviendo real porque está saliendo de la esclavitud del miedo y de la servidumbre de la superstición. Vuestra filosofía se esfuerza por emanciparse de los dogmas y de la tradición. Vuestra ciencia está enfrascada en la contienda secular entre la verdad y el error, mientras lucha por liberarse de la servidumbre de la abstracción, de la esclavitud de las matemáticas y de la ceguera relativa del materialismo mecanicista.

\par
%\textsuperscript{(142.1)}
\textsuperscript{12:9.6} El hombre mortal posee un núcleo espiritual. La mente es un sistema energético personal que existe alrededor de un núcleo espiritual divino y que funciona en un entorno material. Esta relación viviente entre la mente personal y el espíritu constituye el potencial universal de la personalidad eterna. Los conflictos reales, las decepciones duraderas, los fracasos importantes o la muerte inevitable sólo pueden producirse cuando los conceptos del yo se atreven a reemplazar por completo el poder dominante del núcleo espiritual central, trastornando así el plan cósmico de la identidad de la personalidad.

\par
%\textsuperscript{(142.2)}
\textsuperscript{12:9.7} [Presentado por un Perfeccionador de la Sabiduría, que actúa por autorización de los Ancianos de los Días.]


\chapter{Documento 13. Las esferas sagradas del Paraíso}
\par
%\textsuperscript{(143.1)}
\textsuperscript{13:0.1} ENTRE la Isla central del Paraíso y el circuito planetario más interior de Havona se encuentran situados en el espacio tres circuitos menores de esferas especiales. El circuito más interior está formado por las siete esferas secretas del Padre Universal; el segundo grupo está compuesto por los siete mundos luminosos del Hijo Eterno; en el más exterior se encuentran las siete esferas inmensas del Espíritu Infinito, los mundos sede ejecutivos de los Siete Espíritus Maestros.

\par
%\textsuperscript{(143.2)}
\textsuperscript{13:0.2} Estos tres circuitos de siete mundos del Padre, del Hijo y del Espíritu son unas esferas de una grandiosidad insuperable y de una gloria inimaginable. Incluso su composición física o material es de una índole que no os ha sido revelada. La materia de cada circuito es distinta, y cada mundo de cada circuito es diferente, excepto los siete mundos del Hijo, cuya constitución física es semejante. Las veintiuna esferas son enormes, y cada grupo de siete está eternizado de manera diferente. Por lo que sabemos han existido siempre; son eternas como el Paraíso. No hay ni archivos ni tradiciones sobre su origen.

\par
%\textsuperscript{(143.3)}
\textsuperscript{13:0.3} Las siete esferas secretas del Padre Universal, que circulan alrededor del Paraíso muy cerca de la Isla eterna, reflejan intensamente la luminosidad espiritual del resplandor central de las Deidades eternas, derramando esta luz de la gloria divina por todo el Paraíso e incluso sobre los siete circuitos de Havona.

\par
%\textsuperscript{(143.4)}
\textsuperscript{13:0.4} En los siete mundos sagrados del Hijo Eterno parecen tener su origen las energías impersonales de la luminosidad espiritual. Ningún ser personal puede residir en ninguno de estos siete reinos resplandecientes. Iluminan con una gloria espiritual todo el Paraíso y Havona, y dirigen la luminosidad pura del espíritu hacia los siete superuniversos. Estas esferas brillantes del segundo circuito emiten igualmente su luz (una luz sin calor) hacia el Paraíso y hacia los mil millones de mundos de los siete circuitos del universo central.

\par
%\textsuperscript{(143.5)}
\textsuperscript{13:0.5} Los Siete Espíritus Maestros, que presiden los destinos de los siete superuniversos, ocupan los siete mundos del Espíritu Infinito, y envían la iluminación espiritual de la Tercera Persona de la Deidad hacia estas creaciones del tiempo y del espacio. Y todo Havona, pero no la Isla del Paraíso, está bañada en estas influencias espiritualizantes.

\par
%\textsuperscript{(143.6)}
\textsuperscript{13:0.6} Aunque los mundos del Padre son unas esferas cuyo estado es último para todas las personalidades dotadas por el Padre, ésta no es su ocupación exclusiva. Muchos seres y entidades distintas a las personales residen en estos mundos. Cada mundo del circuito del Padre y del circuito del Espíritu tiene un tipo distinto de ciudadanos permanentes, pero creemos que los mundos del Hijo están habitados por tipos uniformes de seres distintos a los personales. Los fragmentos del Padre forman parte de los nativos de Divinington; las otras órdenes de ciudadanos permanentes no os han sido reveladas.

\par
%\textsuperscript{(143.7)}
\textsuperscript{13:0.7} Los veintiún satélites del Paraíso sirven para muchos fines tanto en el universo central como en los superuniversos, unos fines no revelados en estas narraciones. Sois capaces de comprender tan poca cosa sobre la vida de estas esferas, que no podéis esperar conseguir nada que se parezca a una visión realista de ellas ni en cuanto a su naturaleza ni en cuanto a su función; allí tienen lugar miles de actividades que no os son reveladas. Estas veintiuna esferas abarcan los \textit{potenciales} de la función del universo maestro. Estos documentos sólo proporcionan un vislumbre fugaz de ciertas actividades circunscritas relacionadas con la presente era universal del gran universo ---o más bien, de uno de los siete sectores del gran universo.

\section*{1. Los siete mundos sagrados del Padre}
\par
%\textsuperscript{(144.1)}
\textsuperscript{13:1.1} El circuito del Padre compuesto por las esferas de la vida sagrada contiene los únicos secretos inherentes a la personalidad en el universo de universos. Estos satélites del Paraíso, que forman el circuito más interior de los tres, son los únicos dominios prohibidos en lo que se refiere a la personalidad en el universo central. El Paraíso inferior y los mundos del Hijo están cerrados igualmente a las personalidades, pero ninguno de estos reinos se ocupa de ninguna forma directamente de la personalidad.

\par
%\textsuperscript{(144.2)}
\textsuperscript{13:1.2} Los mundos paradisiacos del Padre están gobernados por la orden más elevada de los Hijos Estacionarios de la Trinidad, los Secretos Trinitizados de Supremacía. De estos mundos puedo decir muy poco; y de sus múltiples actividades puedo decir aún menos. Una información así sólo concierne a aquellos seres que trabajan y que salen de allí. Y aunque estoy un poco familiarizado con seis de estos mundos especiales, nunca he aterrizado en Divinington; ese mundo me está totalmente prohibido.

\par
%\textsuperscript{(144.3)}
\textsuperscript{13:1.3} Una de las razones por las cuales estos mundos son secretos es que cada una de estas esferas sagradas disfruta de una representación, o manifestación, especializada de las Deidades que componen la Trinidad del Paraíso; no se trata de una personalidad, sino de una presencia única de la Divinidad, que sólo pueden apreciar y comprender los grupos especiales de inteligencias que residen en esa esfera particular, o que son admitidos en ella. Los Secretos Trinitizados de Supremacía son los agentes personales de estas presencias especializadas e impersonales de la Divinidad. Y los Secretos de la Supremacía son unos seres extremadamente personales, magníficamente dotados y maravillosamente adaptados a su tarea elevada y exigente.

\par
%\textsuperscript{(144.4)}
\textsuperscript{13:1.4} 1. DIVININGTON. Este mundo es, en un sentido muy especial, el <<\textit{seno del Padre}>>\footnote{\textit{Seno del Padre}: Jn 1:18.}, la esfera de comunión personal del Padre Universal, y en él se encuentra una manifestación especial de su divinidad. Divinington es el punto de reunión paradisiaco de los Ajustadores del Pensamiento, pero es también el hogar de otras muchas entidades, personalidades y otros seres que tienen su origen en el Padre Universal. Muchas personalidades, además del Hijo Eterno, tienen su origen directo en los actos solitarios del Padre Universal. En esta residencia sólo fraternizan y ejercen su actividad los fragmentos del Padre y las personalidades y los otros seres que tienen su origen directo y exclusivo en el Padre Universal.

\par
%\textsuperscript{(144.5)}
\textsuperscript{13:1.5} \textit{Los secretos de Divinington} incluyen el secreto de la donación y de la misión de los Ajustadores del Pensamiento. Su naturaleza, su origen y la técnica de su contacto con las criaturas humildes de los mundos evolutivos son un secreto de esta esfera paradisiaca. Estas operaciones asombrosas no nos conciernen personalmente a los demás, y por eso las Deidades consideran oportuno ocultar a nuestra plena comprensión algunas características de este gran ministerio divino. En la medida en que nos ponemos en contacto con esta fase de la actividad divina, se nos permite conocer plenamente estas operaciones, pero en lo que se refiere a los detalles íntimos de esta gran donación, no estamos informados por completo.

\par
%\textsuperscript{(145.1)}
\textsuperscript{13:1.6} Esta esfera contiene también los secretos de la naturaleza, el propósito y las actividades de todas las otras formas de fragmentos del Padre, de los Mensajeros de Gravedad y de una multitud de otros seres que no os han sido revelados. Es muy probable que si las verdades que se me ocultan sobre Divinington me fueran reveladas, no harían más que confundirme y obstaculizarme en mi trabajo actual, y además, quizás se encuentren más allá de la capacidad conceptual de mi orden de seres.

\par
%\textsuperscript{(145.2)}
\textsuperscript{13:1.7} 2. SONARINGTON. Esta esfera es el <<\textit{seno del Hijo}>>, el mundo receptor personal del Hijo Eterno. Es la sede paradisiaca de los Hijos de Dios descendentes y ascendentes a partir del momento en que son plenamente acreditados y finalmente aprobados. Este mundo es el hogar paradisiaco para todos los Hijos del Hijo Eterno y de sus Hijos coordinados y asociados. Hay numerosas órdenes de filiación divina vinculadas a esta morada celestial que no han sido reveladas a los mortales, puesto que no están relacionadas con los planes del programa ascensional de la progresión espiritual humana a través de los universos y hasta el Paraíso.

\par
%\textsuperscript{(145.3)}
\textsuperscript{13:1.8} \textit{Los secretos de Sonarington} incluyen el secreto de la encarnación de los Hijos divinos. Cuando un Hijo de Dios se convierte en un Hijo del Hombre, cuando nace literalmente de una mujer como sucedió en vuestro mundo hace mil novecientos años, es un misterio universal. Esto está ocurriendo constantemente en todos los universos, y es un secreto de Sonarington relacionado con la filiación divina. Los Ajustadores son un misterio de Dios Padre. La encarnación de los Hijos divinos es un misterio de Dios Hijo; es un secreto encerrado en el séptimo sector de Sonarington, una zona donde nadie penetra salvo aquellos que han pasado personalmente por esta experiencia única. Sólo os han sido comunicadas aquellas fases de la encarnación que tienen que ver con vuestra carrera de ascensión. Existen otras muchas fases del misterio de la encarnación de los tipos no revelados de Hijos Paradisiacos en misiones de servicio universal que no os han sido indicadas. Y Sonarington encierra además otros misterios.

\par
%\textsuperscript{(145.4)}
\textsuperscript{13:1.9} 3. SPIRITINGTON. Este mundo es el <<\textit{seno del Espíritu}>>, el hogar paradisiaco de los seres superiores que representan exclusivamente al Espíritu Infinito. Aquí se reúnen los Siete Espíritus Maestros y algunos de sus descendientes procedentes de todos los universos. En esta morada celestial también se pueden encontrar numerosas órdenes no reveladas de personalidades espirituales, de seres asignados a las múltiples actividades del universo que no están asociadas con los planes destinados a elevar a las criaturas mortales del tiempo hasta los niveles paradisiacos de la eternidad.

\par
%\textsuperscript{(145.5)}
\textsuperscript{13:1.10} \textit{Los secretos de Spiritington} incluyen los misterios impenetrables de la reflectividad. Os hablamos del extenso fenómeno universal de la reflectividad, y más en particular tal como funciona en los mundos sede de los siete superuniversos, pero nunca explicamos plenamente este fenómeno porque no lo entendemos por completo. Comprendemos una gran parte, una grandísima parte, pero muchos detalles fundamentales son todavía un misterio para nosotros. La reflectividad es un secreto de Dios Espíritu. Habéis sido informados de las funciones de la reflectividad en relación con el programa ascensional de la supervivencia humana, y en efecto funciona así, pero la reflectividad es también una característica indispensable del trabajo normal de otras numerosas fases de la actividad universal. Este don del Espíritu Infinito se utiliza también en otros canales distintos a los que sirven para recoger datos y difundir información. Y Spiritington contiene otros secretos.

\par
%\textsuperscript{(145.6)}
\textsuperscript{13:1.11} 4. VICEGERINGTON. Este planeta es el <<\textit{seno del Padre y del Hijo}>> y es la esfera secreta de ciertos seres no revelados que tienen su origen en los actos del Padre y del Hijo. Es también el hogar paradisiaco de muchos seres glorificados cuya ascendencia es compleja, de aquellos cuyo origen es complicado debido a las muy diversas técnicas que funcionan en los siete superuniversos. En este mundo se reúnen muchos grupos de seres cuya identidad no ha sido revelada a los mortales de Urantia.

\par
%\textsuperscript{(146.1)}
\textsuperscript{13:1.12} \textit{Los secretos de Vicegerington} incluyen los secretos de la trinitización, y la trinitización constituye el secreto de la autoridad para poder representar a la Trinidad, para actuar como vicegerentes de los Dioses. La autorización para representar a la Trinidad sólo se concede a aquellos seres revelados y no revelados que son trinitizados, creados, existenciados o eternizados por dos personas cualquiera de la Trinidad del Paraíso, o por las tres. Las personalidades engendradas por los actos trinitizantes de ciertos tipos de criaturas glorificadas no representan nada más que el potencial conceptual movilizado en esa trinitización, aunque esas criaturas pueden elevarse por el camino del abrazo de la Deidad que está abierto a todas las de su clase.

\par
%\textsuperscript{(146.2)}
\textsuperscript{13:1.13} Los seres no trinitizados no comprenden plenamente la técnica de la trinitización empleada por dos o tres Creadores o por ciertas criaturas. Nunca comprenderéis plenamente este fenómeno, a menos que en el lejano futuro de vuestra carrera glorificada intentéis esta aventura y tengáis éxito en ella, porque de otra manera estos secretos de Vicegerington estarán siempre vedados para vosotros. Pero para mí, que soy un ser elevado de origen trinitario, todos los sectores de Vicegerington están abiertos. Comprendo plenamente el secreto de mi origen y de mi destino, y lo protejo igualmente de una manera plena y sagrada.

\par
%\textsuperscript{(146.3)}
\textsuperscript{13:1.14} Existen además otras formas y fases de la trinitización que no han sido indicadas a los pueblos de Urantia, y estas experiencias, en sus aspectos personales, están debidamente protegidas en el sector secreto de Vicegerington.

\par
%\textsuperscript{(146.4)}
\textsuperscript{13:1.15} 5. SOLITARINGTON. Este mundo es el <<\textit{seno del Padre y del Espíritu}>> y es el lugar de reunión de una magnífica multitud de seres no revelados que tienen su origen en los actos conjuntos del Padre Universal y del Espíritu Infinito, unos seres que comparten las características del Padre además de su herencia del Espíritu.

\par
%\textsuperscript{(146.5)}
\textsuperscript{13:1.16} Éste es también el hogar de los Mensajeros Solitarios y de otras personalidades de las órdenes superangélicas. Conocéis a muy pocos de estos seres; existe un gran número de órdenes no reveladas en Urantia. El hecho de que estén domiciliados en el quinto mundo no implica necesariamente que el Padre haya tenido algo que ver con la creación de los Mensajeros Solitarios o de sus asociados superangélicos, pero en esta era del universo, el Padre está relacionado con sus actividades. Durante la presente era del universo, ésta es también la esfera a la que pertenecen los Directores del Poder Universal.

\par
%\textsuperscript{(146.6)}
\textsuperscript{13:1.17} Existen numerosas órdenes adicionales de personalidades espirituales, de seres desconocidos para el hombre mortal, que consideran a Solitarington como su esfera paradisiaca natal. Se debe recordar que todas las divisiones y niveles de las actividades universales están tan plenamente provistos de ministros espirituales como el ámbito que está ocupado en ayudar al hombre mortal a ascender hasta su destino divino en el Paraíso.

\par
%\textsuperscript{(146.7)}
\textsuperscript{13:1.18} \textit{Los secretos de Solitarington.} Además de ciertos secretos de la trinitización, este mundo contiene los secretos de la relación personal entre el Espíritu Infinito y ciertos descendientes superiores de la Fuente-Centro Tercera. En Solitarington se guardan los misterios de la asociación íntima de numerosas órdenes no reveladas con los espíritus del Padre, del Hijo y del Espíritu, con el triple espíritu de la Trinidad, y con los espíritus del Supremo, del Último y del Supremo-Último.

\par
%\textsuperscript{(146.8)}
\textsuperscript{13:1.19} 6. SERAFINGTON. Esta esfera es el <<\textit{seno del Hijo y del Espíritu}>>, y es el mundo de origen de las inmensas multitudes de seres no revelados creados por el Hijo y el Espíritu. Es también la esfera de destino de todas las órdenes ministrantes de las huestes angélicas, incluyendo a los supernafines, los seconafines y los serafines. En el universo central y en los universos de la periferia sirven también muchas órdenes de espíritus magníficos que no son <<\textit{espíritus ministrantes para aquellos que heredarán la salvación}>>\footnote{\textit{Espíritus ministrantes}: Heb 1:14.}. Todos estos trabajadores espirituales, en todos los ámbitos y niveles de las actividades universales, consideran a Serafington como su hogar paradisiaco.

\par
%\textsuperscript{(147.1)}
\textsuperscript{13:1.20} \textit{Los secretos de Serafington} incluyen un triple misterio, y sólo puedo mencionar uno de ellos ---el misterio de los transportes seráficos. La capacidad que poseen diversas órdenes de serafines y de seres espirituales semejantes para envolver dentro de sus formas espirituales a todas las órdenes de personalidades no materiales y transportarlas durante larguísimos viajes interplanetarios, es un secreto encerrado en los sectores sagrados de Serafington. Los serafines transportadores comprenden este misterio, pero no lo comunican al resto de nosotros, o quizás no pueden. Los otros misterios de Serafington están relacionados con las experiencias personales de tipos de servidores espirituales no revelados hasta ahora a los mortales. Nos abstenemos de hablar de los secretos de estos seres estrechamente relacionados porque casi podéis comprender estas órdenes tan cercanas de existencia, y presentar nuestros conocimientos, incluso parciales, de estos fenómenos sería similar a una traición de la confianza.

\par
%\textsuperscript{(147.2)}
\textsuperscript{13:1.21} 7. ASCENDINGTON. Este mundo singular es <<\textit{el seno del Padre, del Hijo y del Espíritu}>>, el lugar de reunión de las criaturas ascendentes del espacio, la esfera receptora de los peregrinos del tiempo que pasan por el universo de Havona en su camino hacia el Paraíso. Ascendington es el verdadero hogar paradisiaco de las almas ascendentes del tiempo y del espacio hasta que alcanzan el estatus del Paraíso. Vosotros los mortales pasaréis la mayor parte de vuestras <<\textit{vacaciones}>> de Havona en Ascendington. Durante vuestra vida en Havona, Ascendington significará para vosotros lo mismo que significaron los directores de la reversión durante la ascensión del universo local y del superuniverso. Aquí os ocuparéis de miles de actividades que sobrepasan el alcance de la imaginación humana. Y al igual que en cada uno de los progresos anteriores de vuestra ascensión hacia Dios, vuestro yo humano emprenderá aquí nuevas relaciones con vuestro yo divino.

\par
%\textsuperscript{(147.3)}
\textsuperscript{13:1.22} \textit{Los secretos de Ascendington} incluyen el misterio de la construcción gradual y segura, en la mente mortal y material, de una contrapartida espiritual y potencialmente inmortal del carácter y de la identidad. Este fenómeno constituye uno de los misterios más desconcertantes de los universos ---la evolución de un alma inmortal en la mente de una criatura mortal y material.

\par
%\textsuperscript{(147.4)}
\textsuperscript{13:1.23} Nunca comprenderéis plenamente esta misteriosa operación hasta que lleguéis a Ascendington. Esta es la razón por la que todo Ascendington estará abierto a vuestras miradas de asombro. Una séptima parte de Ascendington me está prohibida ---ese sector relacionado con este mismo secreto que es (o será) la experiencia y la propiedad exclusivas de vuestro tipo de seres. Esta experiencia pertenece a vuestra orden humana de existencia. Mi orden de personalidad no está relacionada directamente con estas operaciones. Por eso a mí me están prohibidas y a vosotros os serán finalmente reveladas. Pero incluso después de que os sean reveladas, por alguna razón seguirá siendo siempre vuestro secreto. No lo revelaréis ni a nosotros ni a ninguna otra orden de seres. Estamos enterados de la fusión eterna de un Ajustador divino con un alma inmortal de origen humano, pero los finalitarios ascendentes conocen esta misma experiencia como una realidad absoluta.

\section*{2. Las relaciones en los mundos del Padre}
\par
%\textsuperscript{(147.5)}
\textsuperscript{13:2.1} Estos mundos que sirven de hogar para las diversas órdenes de seres espirituales son unas esferas enormes y prodigiosas, y su belleza incomparable y su gloria magnífica son iguales a las del Paraíso. Son mundos de encuentro, esferas para reunirse, que sirven como domicilios cósmicos permanentes. Como finalitarios tendréis vuestro domicilio en el Paraíso, pero Ascendington será vuestra dirección particular en todos los tiempos, incluso cuando empecéis a servir en el espacio exterior. Durante toda la eternidad consideraréis a Ascendington como el hogar de vuestros recuerdos sentimentales y de vuestras memorias del pasado. Cuando os convirtáis en seres espirituales de la séptima fase, es posible que renunciéis a vuestro estado residencial en el Paraíso.

\par
%\textsuperscript{(148.1)}
\textsuperscript{13:2.2} Puesto que los universos exteriores están en proceso de formación, y si han de ser habitados por criaturas temporales con potencial de ascensión, entonces deducimos que estos hijos del futuro también estarán destinados a considerar a Ascendington como el mundo de su hogar paradisiaco.

\par
%\textsuperscript{(148.2)}
\textsuperscript{13:2.3} Cuando lleguéis al Paraíso, Ascendington es la única esfera sagrada que podréis inspeccionar abiertamente y sin reservas. Vicegerington es la única esfera sagrada que está abierta por completo y sin reservas a mi examen. Aunque sus secretos están relacionados con mi origen, en esta era del universo no considero a Vicegerington como mi hogar. Los seres que tienen su origen en la Trinidad y los seres trinitizados no son la misma cosa.

\par
%\textsuperscript{(148.3)}
\textsuperscript{13:2.4} Los seres que tienen su origen en la Trinidad no comparten plenamente los mundos del Padre; tienen su hogar exclusivo en la Isla del Paraíso muy cerca de la Esfera Santísima. A menudo aparecen en Ascendington, <<\textit{el seno del Padre, del Hijo y del Espíritu}>>, donde fraternizan con sus hermanos que han ascendido desde los mundos humildes del espacio.

\par
%\textsuperscript{(148.4)}
\textsuperscript{13:2.5} Podríais suponer que puesto que los Hijos Creadores tienen su origen en el Padre y el Hijo, deberían considerar a Vicegerington como su hogar, pero éste no es el caso en la presente era del universo dominada por las actividades de Dios Séptuple. Y existen muchos problemas similares que os dejarán perplejos, porque encontraréis con seguridad muchas dificultades cuando intentéis comprender estas cosas tan cercanas al Paraíso. Estas cuestiones tampoco las podréis investigar con éxito; sabéis tan poco. Si supierais más cosas sobre los mundos del Padre, encontraríais simplemente más dificultades hasta que lo supiérais \textit{todo} sobre ellos. La pertenencia a cualquiera de estos mundos secretos se adquiere mediante el servicio así como mediante la naturaleza del origen, y las sucesivas eras del universo pueden redistribuir algunas de estas agrupaciones de personalidades, como de hecho lo hacen.

\par
%\textsuperscript{(148.5)}
\textsuperscript{13:2.6} Los mundos del circuito interior son mundos realmente fraternales o de estatus, más que esferas residenciales efectivas. Los mortales alcanzarán algún tipo de estatus en cada uno de los mundos del Padre, salvo en uno. Por ejemplo: cuando vosotros los mortales llegáis a Havona, se os concede el permiso de visitar Ascendington, donde sois muy bien acogidos, pero no se os permite visitar los otros seis mundos sagrados. Después de vuestro paso por el régimen del Paraíso y después de haber sido admitidos en el Cuerpo de la Finalidad, recibís la autorización de ir a Sonarington, puesto que sois hijos de Dios así como ascendentes ---y sois incluso más. Pero siempre habrá una séptima parte de Sonarington, el sector de los secretos de la encarnación de los Hijos divinos, que no estará abierta a vuestra inspección. Estos secretos nunca serán revelados a los hijos ascendentes de Dios.

\par
%\textsuperscript{(148.6)}
\textsuperscript{13:2.7} Al final podréis acceder plenamente a Ascendington y tendréis un acceso relativo a las otras esferas del Padre, salvo a Divinington. Pero después de que seáis finalitarios, aunque se os conceda el permiso de aterrizar en cinco esferas secretas adicionales, no se os autorizará a visitar todos los sectores de esos mundos. Tampoco se os permitirá aterrizar en las orillas de Divinington, el <<\textit{seno del Padre}>>\footnote{\textit{Seno del Padre}: Jn 1:18.}, aunque os halléis repetidas veces con toda seguridad <<\textit{a la diestra del Padre}>>\footnote{\textit{Diestra del Padre}: Sal 110:1; Mt 22:43-44; Mc 12:36; 16:19; Lc 20:42; Hch 7:55-56; Ro 8:34; Col 3:1; Heb 1:3; 8:1; 10:12; 12:2; 1 P 3:22.}. En toda la eternidad, nunca surgirá ninguna necesidad de que estéis presentes en el mundo de los Ajustadores del Pensamiento.

\par
%\textsuperscript{(149.1)}
\textsuperscript{13:2.8} Estos mundos de encuentro de la vida espiritual son territorios prohibidos hasta el punto de que se nos pide que no tratemos de penetrar en aquellas fases de estas esferas que están totalmente fuera del ámbito de nuestra experiencia. Podéis llegar a ser unas criaturas perfectas\footnote{\textit{Ser perfectos}: Gn 17:1; 1 Re 8:61; Lv 19::2; Dt 18:13; Mt 5:48; 2 Co 13:11; Stg 1:4; 1 P 1:16.} al igual que el Padre Universal es una deidad perfecta, pero no podéis conocer todos los secretos experienciales de todas las demás órdenes de personalidades del universo. Cuando el Creador comparte con su criatura un secreto experiencial de la personalidad, el Creador conserva ese secreto en una confidencia eterna.

\par
%\textsuperscript{(149.2)}
\textsuperscript{13:2.9} Todos estos secretos son probablemente conocidos por el cuerpo colectivo de los Secretos Trinitizados de Supremacía. Estos seres sólo son plenamente conocidos por los grupos especiales de sus mundos; son poco comprendidos por otras órdenes. Después de que alcancéis el Paraíso, conoceréis y amaréis ardientemente a los diez Secretos de Supremacía que dirigen Ascendington. A excepción de Divinington, también conseguiréis comprender parcialmente a los Secretos de Supremacía de los otros mundos del Padre, aunque no tan perfectamente como a los de Ascendington.

\par
%\textsuperscript{(149.3)}
\textsuperscript{13:2.10} Los Secretos Trinitizados de Supremacía, como sugiere su nombre, están relacionados con el Supremo; están relacionados igualmente con el Último y con el futuro Supremo-Último. Estos secretos de Supremacía son los secretos del Supremo y también los secretos del Último, e incluso los secretos del Supremo-Último.

\section*{3. Los mundos sagrados del Hijo Eterno}
\par
%\textsuperscript{(149.4)}
\textsuperscript{13:3.1} Las siete esferas luminosas del Hijo Eterno son los mundos de las siete fases de la existencia puramente espiritual. Estos orbes resplandecientes son la fuente de la triple luz del Paraíso y de Havona, y su influencia está ampliamente limitada, pero no del todo, al universo central.

\par
%\textsuperscript{(149.5)}
\textsuperscript{13:3.2} La personalidad no está presente en estos satélites del Paraíso; por eso podemos presentar muy poca cosa a la personalidad mortal y material acerca de estas residencias puramente espirituales. Nos enseñan que estos mundos rebosan de vida distinta a la personal de los seres del Hijo Eterno. Deducimos que estas entidades están siendo agrupadas para ejercer su ministerio en los nuevos universos en proyecto del espacio exterior. Los filósofos del Paraíso sostienen que cada ciclo del Paraíso, que dura unos dos mil millones de años del tiempo de Urantia, presencia la creación de reservas adicionales de estas órdenes en los mundos secretos del Hijo Eterno.

\par
%\textsuperscript{(149.6)}
\textsuperscript{13:3.3} Según mis informaciones, ninguna personalidad ha estado nunca en ninguna de estas esferas del Hijo Eterno. En toda mi larga experiencia dentro y fuera del Paraíso, nunca he sido designado para visitar uno de estos mundos. Ni siquiera las personalidades cocreadas por el Hijo Eterno van a estos mundos. Deducimos que todos los tipos de espíritus impersonales ---cualquiera que sea su origen--- son admitidos en estas moradas espirituales. Como yo soy una persona y tengo una forma espiritual, no hay duda de que un mundo así me parecería vacío y abandonado, aunque se me permitiera hacerle una visita. Las personalidades espirituales superiores no son dadas a satisfacer curiosidades sin objeto, aventuras puramente inútiles. En conjunto, existen en todo momento demasiadas aventuras fascinantes y útiles como para permitirnos el desarrollo de cualquier gran interés por unos proyectos inútiles o irreales.

\section*{4. Los mundos del Espíritu Infinito}
\par
%\textsuperscript{(149.7)}
\textsuperscript{13:4.1} Los siete orbes del Espíritu Infinito circulan entre el circuito interior de Havona y las esferas resplandecientes del Hijo Eterno; se trata de unos mundos habitados por los descendientes del Espíritu Infinito, por los hijos trinitizados de las personalidades creadas glorificadas, y por otros tipos de seres no revelados encargados de administrar eficazmente las numerosas empresas de los diversos campos de actividad del universo.

\par
%\textsuperscript{(150.1)}
\textsuperscript{13:4.2} Los Siete Espíritus Maestros son los representantes supremos y últimos del Espíritu Infinito. Mantienen sus emplazamientos personales, sus centros de poder, en la periferia del Paraíso, pero todas las operaciones relacionadas con su gestión y dirección del gran universo están dirigidas desde estas siete esferas ejecutivas especiales del Espíritu Infinito. Los Siete Espíritus Maestros son en realidad el volante mental-espiritual del universo de universos, un poder centralizado que lo engloba todo, lo abarca todo y lo coordina todo.

\par
%\textsuperscript{(150.2)}
\textsuperscript{13:4.3} Los Espíritus Maestros actúan desde estas siete esferas especiales para igualar y estabilizar los circuitos de la mente cósmica del gran universo. También tienen que ver con la actitud y la presencia espiritual diferencial de las Deidades en todo el gran universo. Las reacciones físicas son uniformes, invariables, y siempre instantáneas y automáticas. Pero la presencia espiritual experiencial está de acuerdo con las condiciones o los estados subyacentes de receptividad espiritual inherentes a cada mente individual de los reinos.

\par
%\textsuperscript{(150.3)}
\textsuperscript{13:4.4} La autoridad, la presencia y la actividad físicas son invariables en todos los universos, grandes o pequeños. En lo que se refiere a la presencia ---o a la reacción--- espiritual, el factor discordante es el diferencial fluctuante con que las criaturas volitivas reconocen y reciben dicha presencia. Aunque a la presencia espiritual de la Deidad absoluta y existencial no le influyen de ninguna manera las actitudes leales o desleales de los seres creados, al mismo tiempo es cierto que a la presencia funcional de la Deidad subabsoluta y experiencial le influyen clara y directamente las decisiones, las elecciones y las actitudes volitivas de estas criaturas finitas ---la lealtad y la devoción de cada ser, planeta, sistema, constelación o universo individual. Pero esta presencia espiritual de la divinidad no es caprichosa ni arbitraria; su variación experiencial es inherente al don del libre albedrío con que están dotadas las criaturas personales.

\par
%\textsuperscript{(150.4)}
\textsuperscript{13:4.5} El factor que determina el diferencial de la presencia espiritual existe en vuestro propio corazón y en vuestra propia mente, y consiste en vuestra propia manera de elegir, en las decisiones de vuestra mente y en la determinación de vuestra propia voluntad. Este diferencial es inherente a las reacciones libres de los seres personales inteligentes, unos seres a quienes el Padre Universal ha ordenado que ejerzan esta libertad de elección. Las Deidades son siempre fieles a los flujos y reflujos de sus espíritus para poder conocer y satisfacer las condiciones y las exigencias de este diferencial en la elección de las criaturas, ya sea otorgando más su presencia en respuesta a un sincero deseo de la misma, o bien retirándose de la escena cuando sus criaturas deciden lo contrario en el ejercicio de la libertad de elección que les ha sido concedida de manera divina. El espíritu de la divinidad se vuelve así humildemente obediente a la elección de las criaturas de los reinos.

\par
%\textsuperscript{(150.5)}
\textsuperscript{13:4.6} Las residencias ejecutivas de los Siete Espíritus Maestros son en realidad las sedes paradisiacas de los siete superuniversos y de sus segmentos correlacionados del espacio exterior. Cada Espíritu Maestro preside un superuniverso, y cada uno de estos siete mundos está exclusivamente asignado a uno de los Espíritus Maestros. No existe literalmente ninguna fase de la administración subparadisiaca de los siete superuniversos que no esté atendida en estos mundos ejecutivos. Estos últimos no son tan exclusivos como las esferas del Padre o las del Hijo, y aunque el estado de residente está limitado a los seres nativos y a aquellos que trabajan allí, estos siete planetas administrativos siempre están abiertos a todos los seres que deseen visitarlos y que puedan disponer de los medios de transporte necesarios.

\par
%\textsuperscript{(151.1)}
\textsuperscript{13:4.7} Para mí, estos mundos ejecutivos son los lugares más interesantes y fascinantes que se encuentran fuera del Paraíso. En ninguna otra parte del vasto universo se pueden observar unas actividades tan variadas, que afectan a tantas órdenes diferentes de seres vivientes, relacionadas con operaciones que se efectúan en tantos niveles diferentes, unas ocupaciones que son a la vez materiales, intelectuales y espirituales. Cuando me conceden un período de descanso de mis tareas, si tengo la suerte de estar en el Paraíso o en Havona, me dirijo habitualmente a uno de estos mundos atareados de los Siete Espíritus Maestros para que mi mente se inspire allí con aquellos espectáculos de iniciativa, devoción, lealtad, sabiduría y eficacia. En ninguna otra parte de los siete niveles de la realidad universal puedo observar una interasociación tan asombrosa de realizaciones de la personalidad. Siempre me siento estimulado por las actividades de aquellos que saben muy bien cómo hacer su trabajo, y que tanto disfrutan haciéndolo.

\par
%\textsuperscript{(151.2)}
\textsuperscript{13:4.8} [Presentado por un Perfeccionador de la Sabiduría, nombrado para esta tarea por los Ancianos de los Días de Uversa.]


\chapter{Documento 14. El universo central y divino}
\par
%\textsuperscript{(152.1)}
\textsuperscript{14:0.1} EL universo perfecto y divino ocupa el centro de toda la creación; es el núcleo eterno alrededor del cual giran las inmensas creaciones del tiempo y del espacio. El Paraíso es la gigantesca Isla nuclear con estabilidad absoluta que reposa inmóvil en el corazón mismo del magnífico universo eterno. Esta familia planetaria central se llama Havona y se encuentra muy alejada del universo local de Nebadon. Sus dimensiones son enormes, su masa es casi increíble, y está compuesta de mil millones de esferas que poseen una belleza inimaginable y una grandiosidad espléndida, pero la verdadera magnitud de esta inmensa creación sobrepasa realmente el alcance de la comprensión de la mente humana.

\par
%\textsuperscript{(152.2)}
\textsuperscript{14:0.2} Éste es el único conjunto de mundos estabilizado, perfecto y establecido. Es un universo totalmente creado y perfecto; no se ha desarrollado por evolución. Es el núcleo eterno de la perfección, alrededor del cual da vueltas la procesión interminable de universos que constituyen el extraordinario experimento evolutivo, la audaz aventura de los Hijos Creadores de Dios, los cuales aspiran a copiar en el tiempo y a reproducir en el espacio el universo modelo, el ideal de la culminación divina, de la finalidad suprema, de la realidad última y de la perfección eterna.

\section*{1. El sistema Paraíso-Havona}
\par
%\textsuperscript{(152.3)}
\textsuperscript{14:1.1} Desde la periferia del Paraíso hasta las fronteras interiores de los siete superuniversos se encuentran las siete condiciones y movimientos espaciales siguientes:

\par
%\textsuperscript{(152.4)}
\textsuperscript{14:1.2} 1. Las zonas en reposo del espacio intermedio que entran en contacto con el Paraíso.

\par
%\textsuperscript{(152.5)}
\textsuperscript{14:1.3} 2. La procesión en el sentido de las agujas del reloj de los tres circuitos del Paraíso y de los siete circuitos de Havona.

\par
%\textsuperscript{(152.6)}
\textsuperscript{14:1.4} 3. La zona semitranquila de espacio que separa a los circuitos de Havona de los cuerpos gravitatorios oscuros del universo central.

\par
%\textsuperscript{(152.7)}
\textsuperscript{14:1.5} 4. El cinturón interior de los cuerpos gravitatorios oscuros, que se mueve en sentido contrario a las agujas del reloj.

\par
%\textsuperscript{(152.8)}
\textsuperscript{14:1.6} 5. La segunda zona de espacio, única en su género, que divide las dos trayectorias espaciales de los cuerpos gravitatorios oscuros.

\par
%\textsuperscript{(152.9)}
\textsuperscript{14:1.7} 6. El cinturón exterior de los cuerpos gravitatorios oscuros, que gira en el sentido de las agujas del reloj alrededor del Paraíso.

\par
%\textsuperscript{(152.10)}
\textsuperscript{14:1.8} 7. Una tercera zona espacial ---una zona semitranquila--- que separa al cinturón exterior de los cuerpos gravitatorios oscuros de los circuitos más interiores de los siete superuniversos\footnote{\textit{Cuerpos gravitarios oscuros}: 1 Re 8:12; 2 Cr 6:1; Job 22:12-14; 38:9-11; Sal 18:11.}.

\par
%\textsuperscript{(152.11)}
\textsuperscript{14:1.9} Los mil millones de mundos de Havona están dispuestos en siete circuitos concéntricos que rodean directamente a los tres circuitos de satélites del Paraíso. Hay más de treinta y cinco millones de mundos en el circuito más interior de Havona, y más de doscientos cuarenta y cinco millones en el más exterior, con cantidades proporcionales intermedias. Cada circuito es diferente, pero todos están perfectamente equilibrados y exquisitamente organizados, y cada uno de ellos está impregnado de una representación especializada del Espíritu Infinito, de uno de los Siete Espíritus de los Circuitos. Además de otras funciones, este Espíritu impersonal coordina la conducta de los asuntos celestiales en todas las partes de cada circuito.

\par
%\textsuperscript{(153.1)}
\textsuperscript{14:1.10} Los circuitos planetarios de Havona no están superpuestos; sus mundos se suceden unos a otros en una procesión lineal ordenada. El universo central gira alrededor de la Isla estacionaria del Paraíso en un inmenso plano compuesto de diez unidades concéntricas estabilizadas ---los tres circuitos de las esferas del Paraíso y los siete circuitos de los mundos de Havona. Desde el punto de vista físico, los circuitos de Havona y del Paraíso forman un solo sistema; los separamos en reconocimiento de su división funcional y administrativa.

\par
%\textsuperscript{(153.2)}
\textsuperscript{14:1.11} El tiempo no se cuenta en el Paraíso; la secuencia de los acontecimientos sucesivos es inherente al concepto que poseen los nativos de la Isla central. Pero el tiempo guarda relación con los circuitos de Havona y con los numerosos seres de origen celestial y terrestre que residen allí. Cada mundo de Havona tiene su propio tiempo local, determinado por su circuito. Todos los mundos de un circuito dado tienen un año de la misma duración, puesto que giran uniformemente alrededor del Paraíso, y la duración de estos años planetarios disminuye desde el circuito más exterior hasta el más interior.

\par
%\textsuperscript{(153.3)}
\textsuperscript{14:1.12} Además del tiempo de los circuitos de Havona, existe el día oficial del Paraíso-Havona y otras denominaciones temporales que están determinadas por los siete satélites paradisiacos del Espíritu Infinito, y emitidas desde allí. El día oficial del Paraíso-Havona está basado en la cantidad de tiempo que necesitan las moradas planetarias del primer circuito, o circuito interior de Havona, para completar una revolución alrededor de la Isla del Paraíso; y aunque su velocidad es enorme debido a que están situadas entre los cuerpos gravitatorios oscuros y el gigantesco Paraíso, estas esferas necesitan casi mil años para completar su circuito. Habéis leído la verdad sin saberlo cuando vuestros ojos se posaron sobre la afirmación: <<\textit{Mil años es como un día con Dios, como una vigilia en la noche}>>. Un día del Paraíso-Havona es exactamente como mil años del calendario bisiesto actual de Urantia, menos siete minutos, tres segundos y un octavo de segundo.

\par
%\textsuperscript{(153.4)}
\textsuperscript{14:1.13} Este día del Paraíso-Havona es la medida oficial de tiempo para los siete superuniversos, aunque cada uno de ellos mantiene sus propios criterios internos de tiempo.

\par
%\textsuperscript{(153.5)}
\textsuperscript{14:1.14} En las afueras de este inmenso universo central, mucho más allá del séptimo cinturón de mundos de Havona, circula una cantidad increíble de enormes cuerpos gravitatorios oscuros. Estas innumerables masas oscuras son totalmente distintas en muchos aspectos a los otros cuerpos espaciales; son muy diferentes incluso en la forma. Estos cuerpos gravitatorios oscuros no reflejan ni absorben la luz; no reaccionan a la luz de la energía física, y rodean y envuelven tan completamente a Havona que la ocultan a la vista de los universos habitados del tiempo y del espacio, incluso de los más cercanos.

\par
%\textsuperscript{(153.6)}
\textsuperscript{14:1.15} El gran cinturón de los cuerpos gravitatorios oscuros está dividido en dos circuitos elípticos iguales por una intrusión de espacio única en su género. El cinturón exterior gira en el sentido de las agujas del reloj, y el cinturón interior en sentido contrario. Estas direcciones alternas del movimiento, unidas a la masa extraordinaria de los cuerpos oscuros, igualan las líneas de la gravedad de Havona de una manera tan eficaz que convierten al universo central en una creación físicamente equilibrada y perfectamente estabilizada.

\par
%\textsuperscript{(153.7)}
\textsuperscript{14:1.16} La procesión interior de los cuerpos gravitatorios oscuros está dispuesta de forma tubular y consiste en tres agrupaciones circulares. Un corte transversal de este circuito mostraría tres círculos concéntricos con una densidad casi igual. El circuito exterior de los cuerpos gravitatorios oscuros está organizado perpendicularmente y es diez mil veces más alto que el circuito interior. El diámetro vertical del circuito exterior es cincuenta mil veces mayor que el diámetro transversal.

\par
%\textsuperscript{(154.1)}
\textsuperscript{14:1.17} El espacio intermedio que existe entre estos dos circuitos de cuerpos gravitatorios es \textit{único,} puesto que no se encuentra nada semejante en ninguna otra parte de todo el extenso universo. Esta zona está caracterizada por enormes movimientos ondulatorios de naturaleza vertical y está impregnada de actividades energéticas extraordinarias de tipo desconocido.

\par
%\textsuperscript{(154.2)}
\textsuperscript{14:1.18} En nuestra opinión, la evolución futura de los niveles del espacio exterior no estará caracterizada por nada que se parezca a los cuerpos gravitatorios oscuros del universo central; consideramos que estas procesiones alternas de los prodigiosos cuerpos equilibradores de la gravedad son únicas en el universo maestro.

\section*{2. La composición de Havona}
\par
%\textsuperscript{(154.3)}
\textsuperscript{14:2.1} Los seres espirituales no viven en un espacio nebuloso; no residen en unos mundos etéreos; están domiciliados en unas esferas concretas de naturaleza material, en unos mundos tan reales como aquellos donde viven los mortales. Los mundos de Havona son reales y tangibles, aunque su sustancia tangible difiere de la organización material de los planetas de los siete superuniversos.

\par
%\textsuperscript{(154.4)}
\textsuperscript{14:2.2} Las realidades físicas de Havona representan un tipo de organización energética radicalmente diferente a cualquier otro que predomine en los universos evolutivos del espacio. Las energías de Havona son triples; las unidades de la energía-materia de los superuniversos contienen una carga energética doble, aunque existe una forma de energía con las fases positiva y negativa. La creación del universo central es triple (procede de la Trinidad); la creación de un universo local es (directamente) doble, efectuada por un Hijo Creador y un Espíritu Creativo.

\par
%\textsuperscript{(154.5)}
\textsuperscript{14:2.3} La materia de Havona está compuesta exactamente de la organización de mil elementos químicos básicos y del funcionamiento equilibrado de las siete formas de energía de Havona. Cada una de estas energías básicas manifiesta siete fases de excitación, de manera que los nativos de Havona responden a cuarenta y nueve estímulos sensoriales diferentes. En otras palabras, visto desde un punto de vista puramente físico, los nativos del universo central poseen cuarenta y nueve formas especializadas de sensaciones. Los sentidos morontiales ascienden a setenta, y los tipos espirituales superiores de respuestas reactivas varían, en las diferentes clases de seres, entre setenta y doscientas diez.

\par
%\textsuperscript{(154.6)}
\textsuperscript{14:2.4} Ninguno de los seres físicos del universo central sería visible para los urantianos. Y ninguno de los estímulos físicos de esos mundos lejanos provocaría tampoco ninguna reacción en vuestros órganos sensoriales rudimentarios. Si un mortal de Urantia pudiera ser transportado hasta Havona, estaría allí sordo, ciego y desprovisto por completo de todas las demás reacciones sensoriales; sólo podría actuar como un ser limitado consciente de sí mismo, privado de todos los estímulos ambientales y de toda reacción a los mismos.

\par
%\textsuperscript{(154.7)}
\textsuperscript{14:2.5} En la creación central se producen numerosos fenómenos físicos y reacciones espirituales que son desconocidos en los mundos tales como Urantia. La organización básica de una creación triple es totalmente distinta a la constitución doble de los universos creados del tiempo y del espacio.

\par
%\textsuperscript{(154.8)}
\textsuperscript{14:2.6} Todas las leyes naturales están coordinadas sobre una base enteramente diferente a la de los sistemas energéticos duales de las creaciones evolutivas. Todo el universo central está organizado con arreglo a un triple sistema de control perfecto y simétrico. En la totalidad del sistema Paraíso-Havona se mantiene un equilibrio perfecto entre todas las realidades cósmicas y todas las fuerzas espirituales. El Paraíso, con su control absoluto sobre la creación material, regula y mantiene perfectamente las energías físicas de este universo central; el Hijo Eterno, como parte de su atracción espiritual que lo abarca todo, sostiene de la manera más perfecta el estado espiritual de todos los que viven en Havona. En el Paraíso nada es experimental, y el sistema Paraíso-Havona es una unidad de perfección creativa.

\par
%\textsuperscript{(155.1)}
\textsuperscript{14:2.7} La gravedad espiritual universal\footnote{\textit{Gravedad espiritual}: Jer 31:3; Jn 6:44; 12:32.} del Hijo Eterno es asombrosamente activa en todo el universo central. Todos los valores de espíritu y todas las personalidades espirituales son atraídos incesantemente hacia el interior, hacia la residencia de los Dioses. Este impulso hacia Dios es intenso e ineludible. La ambición de alcanzar a Dios es más fuerte en el universo central, no porque la gravedad espiritual sea allí más fuerte que en los universos exteriores, sino porque los seres que han llegado hasta Havona están más plenamente espiritualizados y, en consecuencia, son más sensibles a la acción siempre presente de la atracción universal de la gravedad espiritual del Hijo Eterno.

\par
%\textsuperscript{(155.2)}
\textsuperscript{14:2.8} El Espíritu Infinito atrae igualmente todos los valores intelectuales hacia el Paraíso. La gravedad mental del Espíritu Infinito funciona en todo el universo central en unión con la gravedad espiritual del Hijo Eterno, y las dos juntas forman el impulso combinado que sienten las almas ascendentes de encontrar a Dios, alcanzar la Deidad, llegar al Paraíso y conocer al Padre.

\par
%\textsuperscript{(155.3)}
\textsuperscript{14:2.9} Havona es un universo espiritualmente perfecto y físicamente estable. El control y la estabilidad equilibrada del universo central parecen ser perfectos. Todo aquello que es físico o espiritual es perfectamente previsible, pero los fenómenos mentales y la volición de la personalidad no lo son. Deducimos que se puede considerar que es imposible que se produzca el pecado, pero lo deducimos sobre la base de que las criaturas nativas de Havona, dotadas de libre albedrío, nunca han sido culpables de transgredir la voluntad de la Deidad. Durante toda la eternidad, estos seres celestiales han sido firmemente leales a los Eternos de los Días. El pecado tampoco ha aparecido en ninguna criatura que ha entrado como peregrino en Havona. Nunca ha habido un ejemplo de mala conducta por parte de ninguna criatura de ningún grupo de personalidades creadas o admitidas en el universo central de Havona. Los métodos y los medios de selección de los universos del tiempo son tan perfectos y tan divinos que nunca se ha cometido un error en la historia de Havona; nunca se han producido equivocaciones; ningún alma ascendente ha sido nunca prematuramente admitida en el universo central.

\section*{3. Los mundos de Havona}
\par
%\textsuperscript{(155.4)}
\textsuperscript{14:3.1} En cuanto al gobierno del universo central, no existe ninguno. Havona es tan exquisitamente perfecto que no se necesita ningún sistema intelectual de gobierno. No existen tribunales regularmente constituidos, ni tampoco hay asambleas legislativas; Havona sólo necesita una dirección administrativa. Aquí se puede observar la cima de los ideales del verdadero dominio \textit{de sí mismo.}

\par
%\textsuperscript{(155.5)}
\textsuperscript{14:3.2} No hay necesidad de gobierno entre estas inteligencias perfectas y casi perfectas. No tienen ninguna necesidad de reglamentación, porque se trata de unos seres nacidos perfectos, entremezclados con criaturas evolutivas que han pasado hace mucho tiempo el examen de los tribunales supremos de los superuniversos.

\par
%\textsuperscript{(155.6)}
\textsuperscript{14:3.3} La administración de Havona no es automática, pero es maravillosamente perfecta y divinamente eficaz. Es principalmente planetaria y está a cargo del Eterno de los Días residente, pues cada esfera de Havona está dirigida por una de estas personalidades de origen trinitario. Los Eternos de los Días no son creadores, pero son unos administradores perfectos. Enseñan con una habilidad suprema y dirigen a sus hijos planetarios con una sabiduría tan perfecta que linda con la absolutidad.

\par
%\textsuperscript{(156.1)}
\textsuperscript{14:3.4} Los mil millones de esferas del universo central constituyen los mundos educativos de las altas personalidades nativas del Paraíso y de Havona, y sirven además como terreno de prueba final para las criaturas ascendentes de los mundos evolutivos del tiempo. A fin de llevar a cabo el gran plan del Padre Universal para la ascensión de las criaturas, los peregrinos del tiempo son desembarcados en los mundos receptores del circuito exterior, el séptimo, y después de acrecentar su formación y de ampliar su experiencia, avanzan progresivamente hacia el interior, planeta tras planeta y círculo tras círculo, hasta que alcanzan finalmente a las Deidades y consiguen residir en el Paraíso.

\par
%\textsuperscript{(156.2)}
\textsuperscript{14:3.5} En la actualidad, aunque las esferas de los siete circuitos se mantienen en toda su gloria celestial, sólo se utiliza cerca del uno por ciento de toda la capacidad planetaria para la tarea de fomentar el plan universal del Padre para la ascensión de los mortales. Cerca de una décima parte del uno por ciento de la superficie de estos mundos enormes está dedicada a la vida y a las actividades del Cuerpo de la Finalidad, compuesto por unos seres establecidos eternamente en la luz y la vida, que residen a menudo en los mundos de Havona y ejercen allí su ministerio. Estos seres elevados tienen su residencia personal en el Paraíso.

\par
%\textsuperscript{(156.3)}
\textsuperscript{14:3.6} La construcción planetaria de las esferas de Havona es totalmente diferente a la de los mundos y sistemas evolutivos del espacio. No es conveniente utilizar unas esferas tan enormes como mundos habitados en ninguna otra parte de todo el gran universo. Su constitución física triata, unida al efecto equilibrador de los inmensos cuerpos gravitatorios oscuros, hace posible igualar de manera tan perfecta las fuerzas físicas y equilibrar de forma tan exquisita las diversas fuerzas de atracción de esta creación extraordinaria. La antigravedad también se emplea para organizar las funciones materiales y las actividades espirituales de estos mundos enormes.

\par
%\textsuperscript{(156.4)}
\textsuperscript{14:3.7} La arquitectura, la iluminación y el calentamiento, así como el embellecimiento biológico y artístico de las esferas de Havona sobrepasan por completo los mayores esfuerzos que pueda hacer la imaginación humana. No os puedo decir mucho sobre Havona; para comprender su belleza y su grandiosidad tenéis que verla. Pero hay ríos y lagos verdaderos en estos mundos perfectos.

\par
%\textsuperscript{(156.5)}
\textsuperscript{14:3.8} Espiritualmente, estos mundos están equipados de manera ideal; están apropiadamente adaptados a su meta de alojar a las numerosas órdenes de seres diferentes que ejercen su actividad en el universo central. En estos mundos magníficos tienen lugar numerosas actividades que están mucho más allá de la comprensión humana.

\section*{4. Las criaturas del universo central}
\par
%\textsuperscript{(156.6)}
\textsuperscript{14:4.1} En los mundos de Havona hay siete formas fundamentales de cosas y de seres vivientes, y cada una de estas formas fundamentales existe bajo tres fases distintas. Cada una de estas tres fases se divide en setenta divisiones mayores, y cada división mayor está compuesta de mil divisiones menores con otras subdivisiones a su vez, y así sucesivamente. Estos grupos fundamentales de vida podrían clasificarse como sigue:

\par
%\textsuperscript{(156.7)}
\textsuperscript{14:4.2} 1. Materiales.

\par
%\textsuperscript{(156.8)}
\textsuperscript{14:4.3} 2. Morontiales.

\par
%\textsuperscript{(156.9)}
\textsuperscript{14:4.4} 3. Espirituales.

\par
%\textsuperscript{(156.10)}
\textsuperscript{14:4.5} 4. Absonitos.

\par
%\textsuperscript{(156.11)}
\textsuperscript{14:4.6} 5. Últimos.

\par
%\textsuperscript{(156.12)}
\textsuperscript{14:4.7} 6. Coabsolutos.

\par
%\textsuperscript{(156.13)}
\textsuperscript{14:4.8} 7. Absolutos.

\par
%\textsuperscript{(157.1)}
\textsuperscript{14:4.9} El deterioro y la muerte no forman parte del ciclo de la vida en los mundos de Havona. En el universo central, las criaturas vivientes inferiores sufren la transmutación de la materialización. Cambian de forma y de manifestación, pero no se descomponen mediante el proceso del deterioro y de la muerte celular.

\par
%\textsuperscript{(157.2)}
\textsuperscript{14:4.10} Los nativos de Havona descienden todos de la Trinidad del Paraíso. Sus progenitores no han sido las criaturas, y son seres que no se reproducen. No podemos describir la creación de estos ciudadanos del universo central, unos seres que nunca fueron creados. Toda la historia de la creación de Havona es un intento por hacer espacio-temporal un hecho de la eternidad que no tiene ninguna relación con el tiempo ni con el espacio, tal como el hombre mortal los comprende. Pero debemos concederle a la filosofía humana un punto de origen; incluso las personalidades que están muy por encima del nivel humano necesitan el concepto de un <<\textit{comienzo}>>. Sin embargo, el sistema Paraíso-Havona es eterno.

\par
%\textsuperscript{(157.3)}
\textsuperscript{14:4.11} Los nativos de Havona viven en los mil millones de esferas del universo central en el mismo sentido en que otras órdenes de ciudadanos permanentes residen en sus esferas respectivas de nacimiento. Al igual que la orden material de filiación dirige la economía material, intelectual y espiritual de los mil millones de sistemas locales de un superuniverso, en un sentido más amplio los nativos de Havona viven y ejercen su actividad en los mil millones de mundos del universo central. Quizás podríais considerar a estos habitantes de Havona como criaturas materiales en el sentido en que la palabra <<\textit{material}>> se pudiera ampliar para poder describir las realidades físicas del universo divino.

\par
%\textsuperscript{(157.4)}
\textsuperscript{14:4.12} Havona posee una vida autóctona que tiene un significado en sí misma y por sí misma. Los habitantes de Havona ofrecen su ministerio de muchas maneras a los que descienden desde el Paraíso y a los ascendentes de los superuniversos, pero viven también unas vidas que son únicas en el universo central y que tienen un significado relativo con total independencia del Paraíso o de los superuniversos.

\par
%\textsuperscript{(157.5)}
\textsuperscript{14:4.13} Al igual que la adoración de los hijos por la fe de los mundos evolutivos contribuye a satisfacer el amor del Padre Universal, la adoración exaltada de las criaturas de Havona sacia los ideales perfectos de la belleza y de la verdad divinas. Al igual que el hombre mortal se esfuerza por hacer la voluntad de Dios, estos seres del universo central viven para satisfacer los ideales de la Trinidad del Paraíso. En su naturaleza misma, ellos \textit{son} la voluntad de Dios. El hombre se alegra de la bondad de Dios, los habitantes de Havona se regocijan de la belleza divina, mientras que los dos disfrutáis del ministerio de la libertad de la verdad viviente.

\par
%\textsuperscript{(157.6)}
\textsuperscript{14:4.14} Los havonianos tienen a la vez un destino actual optativo y un destino futuro no revelado. Y existe una progresión para las criaturas nativas que es propia del universo central, una progresión que no supone ni la ascensión al Paraíso ni la penetración en los superuniversos. Esta progresión hacia un estado más elevado en Havona se puede indicar como sigue:

\par
%\textsuperscript{(157.7)}
\textsuperscript{14:4.15} 1. Progreso experiencial hacia el exterior, desde el primero hasta el séptimo circuito.

\par
%\textsuperscript{(157.8)}
\textsuperscript{14:4.16} 2. Progreso hacia el interior, desde el séptimo hasta el primer circuito.

\par
%\textsuperscript{(157.9)}
\textsuperscript{14:4.17} 3. Progreso dentro de un circuito ---progresión en los mundos de un circuito dado.

\par
%\textsuperscript{(157.10)}
\textsuperscript{14:4.18} Además de los nativos de Havona, la población del universo central contiene numerosas clases de seres modelo para los diversos grupos del universo ---consejeros, directores y educadores de su misma especie y para los de su misma especie en toda la creación. Todos los seres en todos los universos son creados según algún tipo de criatura modelo que vive en uno de los mil millones de mundos de Havona. Incluso los mortales del tiempo tienen su meta y sus ideales de existencia como criaturas en los circuitos exteriores de estas esferas modelo de las alturas.

\par
%\textsuperscript{(157.11)}
\textsuperscript{14:4.19} Luego están los seres que han alcanzado al Padre Universal, que tienen derecho a ir y venir, y que son destinados aquí y allá en los universos para realizar misiones de servicio especial. Y en cada mundo de Havona se encontrará a los candidatos a la consecución, a aquellos que han alcanzado físicamente el universo central, pero que todavía no han conseguido el desarrollo espiritual que les permitirá solicitar su residencia en el Paraíso.

\par
%\textsuperscript{(158.1)}
\textsuperscript{14:4.20} El Espíritu Infinito está representado en los mundos de Havona por una multitud de personalidades, por unos seres de bondad y de gloria, que administran los detalles de los complejos asuntos intelectuales y espirituales del universo central. En estos mundos de perfección divina, efectúan el trabajo autóctono para la conducción normal de esta enorme creación y, además, llevan adelante las múltiples tareas de enseñar, formar y ayudar a la inmensa cantidad de criaturas ascendentes que se han elevado hasta la gloria desde los mundos tenebrosos del espacio.

\par
%\textsuperscript{(158.2)}
\textsuperscript{14:4.21} Hay numerosos grupos de seres nativos del sistema Paraíso-Havona que no están directamente asociados de ninguna manera con el programa de ascensión que permite a las criaturas alcanzar la perfección; por eso los omitimos de las clasificaciones de personalidades que presentamos a las razas mortales. Sólo presentamos aquí a los grupos principales de seres superhumanos y a aquellas órdenes directamente relacionadas con la experiencia de vuestra supervivencia.

\par
%\textsuperscript{(158.3)}
\textsuperscript{14:4.22} Havona pulula de vida de todas las fases de seres inteligentes, que tratan de avanzar allí desde los circuitos inferiores hasta los circuitos superiores, en sus esfuerzos por alcanzar unos niveles más elevados de comprensión de la divinidad y una apreciación más amplia de los significados supremos, de los valores últimos y de la realidad absoluta.

\section*{5. La vida en Havona}
\par
%\textsuperscript{(158.4)}
\textsuperscript{14:5.1} En Urantia pasáis por una prueba corta e intensa durante la vida inicial de vuestra existencia material. En los mundos de las mansiones y pasando por vuestro sistema, vuestra constelación y vuestro universo local, atravesáis las fases morontiales de la ascensión. En los mundos formativos del superuniverso pasáis por las verdaderas etapas espirituales de la progresión y os preparáis para el tránsito final hacia Havona. En los siete circuitos de Havona, vuestra consecución es intelectual, espiritual y experiencial. Y existe una tarea determinada a realizar en cada uno de los mundos de cada uno de estos circuitos.

\par
%\textsuperscript{(158.5)}
\textsuperscript{14:5.2} La vida en los mundos divinos del universo central es tan rica y tan plena, tan completa y tan repleta, que trasciende totalmente el concepto humano de todo lo que un ser creado podría experimentar. Las actividades sociales y económicas de esta creación eterna son completamente distintas a las ocupaciones de las criaturas materiales que viven en los mundos evolutivos como Urantia. Incluso la técnica del pensamiento en Havona es diferente a los procesos mentales en Urantia.

\par
%\textsuperscript{(158.6)}
\textsuperscript{14:5.3} Las reglamentaciones en el universo central son naturales de forma apropiada e inherente; las normas de conducta no son arbitrarias. En todas las necesidades de Havona se revela la razón de la rectitud y la regla de la justicia. Y estos dos factores combinados equivalen a lo que en Urantia se denominaría \textit{equidad.} Cuando lleguéis a Havona, disfrutaréis haciendo las cosas con naturalidad y de la manera que deben hacerse.

\par
%\textsuperscript{(158.7)}
\textsuperscript{14:5.4} Cuando los seres inteligentes alcanzan por primera vez el universo central, son recibidos y domiciliados en el mundo piloto del séptimo circuito de Havona. A medida que los recién llegados progresan espiritualmente, consiguen comprender la identidad del Espíritu Maestro de su superuniverso, son trasladados al sexto círculo. (Los círculos del progreso de la mente humana han sido denominados según estas disposiciones del universo central). Después de que los ascendentes han conseguido comprender la Supremacía y están preparados así para la aventura de la Deidad, son conducidos al quinto circuito; y después de alcanzar al Espíritu Infinito, son trasladados al cuarto. Una vez que han logrado llegar al Hijo Eterno, son trasladados al tercero; y cuando han reconocido al Padre Universal, van a residir en el segundo circuito de mundos, donde se familiarizan más con las multitudes del Paraíso. La llegada al primer circuito de Havona significa que los candidatos del tiempo han sido aceptados para el servicio en el Paraíso. Según haya sido la duración y la naturaleza de su ascensión como criaturas, se quedarán durante un tiempo indeterminado en el circuito interior de consecución espiritual progresiva. Desde este circuito interior, los peregrinos ascendentes pasan hacia el interior para residir en el Paraíso y para ser admitidos en el Cuerpo de la Finalidad.

\par
%\textsuperscript{(159.1)}
\textsuperscript{14:5.5} Durante vuestra estancia en Havona como peregrinos ascendentes, se os permitirá visitar libremente los mundos del circuito donde estéis destinados. También se os permitirá regresar a los planetas de aquellos circuitos que habréis atravesado previamente. Todo esto es posible para aquellos que residen en los círculos de Havona sin que tengan la necesidad de ser transportados por los supernafines. Los peregrinos del tiempo pueden equiparse ellos mismos para atravesar el espacio <<\textit{conquistado}>>, pero han de depender de las técnicas establecidas para franquear el espacio <<\textit{no conquistado}>>; un peregrino no puede salir de Havona ni avanzar más allá del circuito al que está asignado sin la ayuda de un supernafín transportador.

\par
%\textsuperscript{(159.2)}
\textsuperscript{14:5.6} Existe una originalidad reconfortante en esta inmensa creación central. Aparte de la organización física de la materia y de la constitución fundamental de las órdenes básicas de seres inteligentes y de otras criaturas vivientes, los mundos de Havona no tienen nada en común. Cada uno de estos planetas es una creación original, única y exclusiva; cada planeta es una obra incomparable, magnífica y perfecta. Y esta individualidad tan diversa se extiende a todas las características de los aspectos físicos, intelectuales y espirituales de la existencia planetaria. Cada una de estas mil millones de esferas perfectas ha sido desarrollada y embellecida de acuerdo con los planes del Eterno de los Días residente. Y ésta es precisamente la razón por la que no hay dos que sean iguales.

\par
%\textsuperscript{(159.3)}
\textsuperscript{14:5.7} La tónica de la aventura y el estímulo de la curiosidad no desaparecerán de vuestra carrera hasta que no hayáis atravesado el último circuito de Havona y visitado el último mundo de Havona. Y entonces el estímulo, el impulso hacia adelante de la eternidad, reemplazará a su predecesor, al atractivo de la aventura del tiempo.

\par
%\textsuperscript{(159.4)}
\textsuperscript{14:5.8} La monotonía indica la inmadurez de la imaginación creativa y la inactividad de la coordinación intelectual con la dotación espiritual. Cuando un mortal ascendente empieza a explorar estos mundos celestiales, ya ha alcanzado la madurez emocional, intelectual y social, si no espiritual.

\par
%\textsuperscript{(159.5)}
\textsuperscript{14:5.9} A medida que avancéis de circuito en circuito en Havona, no sólo tendréis que hacer frente a unos cambios inimaginables, sino que vuestro asombro será inexpresable a medida que progreséis de planeta en planeta dentro de cada circuito. Cada uno de estos mil millones de mundos de estudio es una verdadera universidad de sorpresas. Aquellos que atraviesan estos circuitos y recorren estas gigantescas esferas experimentan un asombro continuo, una admiración interminable. La monotonía no forma parte de la carrera en Havona.

\par
%\textsuperscript{(159.6)}
\textsuperscript{14:5.10} El amor de la aventura, la curiosidad y el horror a la monotonía ---esas características inherentes a la naturaleza humana en evolución--- no han sido puestos ahí simplemente para exasperaros y enojaros durante vuestra breve estancia en la Tierra, sino más bien para sugeriros que la muerte sólo es el comienzo de una carrera de aventuras sin fin, de una vida perpetua de anticipaciones, de un eterno viaje de descubrimientos.

\par
%\textsuperscript{(160.1)}
\textsuperscript{14:5.11} La curiosidad ---el espíritu de investigación, el estímulo del descubrimiento, el impulso a la exploración--- forma parte de la dotación innata y divina de las criaturas evolutivas del espacio. Estos impulsos naturales no se os han dado solamente para ser frustrados y reprimidos. Es cierto que estos impulsos ambiciosos han de ser refrenados con frecuencia durante vuestra corta vida en la Tierra, y que a menudo se experimentan decepciones, pero serán plenamente realizados y gloriosamente satisfechos durante las largas eras por venir.

\section*{6. La finalidad del universo central}
\par
%\textsuperscript{(160.2)}
\textsuperscript{14:6.1} La gama de actividades en los siete circuitos de Havona es enorme. En general, se pueden describir como sigue:

\par
%\textsuperscript{(160.3)}
\textsuperscript{14:6.2} 1. Havonianas.

\par
%\textsuperscript{(160.4)}
\textsuperscript{14:6.3} 2. Paradisiacas.

\par
%\textsuperscript{(160.5)}
\textsuperscript{14:6.4} 3. Finito-ascendentes ---evolutivas Supremo-Últimas.

\par
%\textsuperscript{(160.6)}
\textsuperscript{14:6.5} Muchas actividades superfinitas tienen lugar en el Havona de la presente era del universo, incluyendo una incalculable diversidad de fases absonitas y de otros tipos relacionadas con las funciones mentales y espirituales. Es posible que el universo central sirva para muchos fines que no me han sido revelados, ya que funciona de numerosas maneras que sobrepasan la comprensión de la mente creada. Sin embargo, intentaré describir cómo esta creación perfecta atiende las necesidades y contribuye a satisfacer siete órdenes de inteligencias universales.

\par
%\textsuperscript{(160.7)}
\textsuperscript{14:6.6} 1. \textit{El Padre Universal} ---la Fuente-Centro Primera. Dios Padre obtiene una satisfacción parental suprema de la perfección de la creación central. Disfruta de la experiencia de saciar su amor en unos niveles cercanos a la igualdad. El Creador perfecto está divinamente satisfecho con la adoración de las criaturas perfectas.

\par
%\textsuperscript{(160.8)}
\textsuperscript{14:6.7} Havona proporciona al Padre la satisfacción suprema de lo conseguido. La perfección llevada a cabo en Havona compensa el retraso espacio-temporal del impulso eterno a la expansión infinita.

\par
%\textsuperscript{(160.9)}
\textsuperscript{14:6.8} El Padre disfruta con que la belleza divina de Havona se corresponda con la suya. La mente divina se siente satisfecha de proporcionar un modelo perfecto de armonía exquisita a todos los universos en evolución.

\par
%\textsuperscript{(160.10)}
\textsuperscript{14:6.9} Nuestro Padre contempla el universo central con un placer perfecto, porque es una digna revelación de la realidad espiritual para todas las personalidades del universo de universos.

\par
%\textsuperscript{(160.11)}
\textsuperscript{14:6.10} El Dios de los universos considera favorablemente a Havona y al Paraíso como el eterno núcleo de poder para todas las expansiones universales posteriores en el tiempo y el espacio.

\par
%\textsuperscript{(160.12)}
\textsuperscript{14:6.11} El Padre eterno ve con satisfacción interminable la creación de Havona como una meta digna y atractiva para los candidatos ascendentes del tiempo, sus nietos mortales del espacio que alcanzan el hogar eterno de su Creador-Padre. Y Dios disfruta con el universo Paraíso-Havona como hogar eterno de la Deidad y de la familia divina.

\par
%\textsuperscript{(160.13)}
\textsuperscript{14:6.12} 2. \textit{El Hijo Eterno} ---la Fuente-Centro Segunda. La magnífica creación central proporciona al Hijo Eterno la prueba eterna de que la asociación de la familia divina ---el Padre, el Hijo y el Espíritu--- es eficaz. Es la base espiritual y material para tener una confianza absoluta en el Padre Universal.

\par
%\textsuperscript{(160.14)}
\textsuperscript{14:6.13} Havona proporciona al Hijo Eterno una base casi ilimitada para hacer realidad la expansión constante del poder espiritual. El universo central proporcionó al Hijo Eterno el terreno donde pudo demostrar con certidumbre y seguridad el espíritu y la técnica del ministerio de donación para instruir a sus Hijos Paradisiacos asociados.

\par
%\textsuperscript{(161.1)}
\textsuperscript{14:6.14} Havona es la realidad sobre la que se basa el control de la gravedad espiritual del Hijo Eterno sobre el universo de universos. Este universo proporciona al Hijo la satisfacción de su anhelo parental, la reproducción espiritual.

\par
%\textsuperscript{(161.2)}
\textsuperscript{14:6.15} Los mundos de Havona y sus habitantes perfectos son la demostración inicial y eternamente final de que el Hijo es el Verbo del Padre. De esta manera, la conciencia que tiene el Hijo de ser un complemento infinito del Padre está perfectamente satisfecha.

\par
%\textsuperscript{(161.3)}
\textsuperscript{14:6.16} Este universo proporciona la oportunidad de realizar una fraternidad recíproca, en un pie de igualdad, entre el Padre Universal y el Hijo Eterno, y esto constituye la prueba perpetua de que cada uno de ellos es una personalidad infinita.

\par
%\textsuperscript{(161.4)}
\textsuperscript{14:6.17} 3. \textit{El Espíritu Infinito} ---la Fuente-Centro Tercera. El universo de Havona proporciona al Espíritu Infinito la prueba de que él es el Actor Conjunto, el representante infinito del Padre y del Hijo unificados. El Espíritu Infinito obtiene en Havona la satisfacción combinada de ejercer su función como actividad creadora mientras disfruta de la satisfacción de coexistir de manera absoluta con esta consecución divina.

\par
%\textsuperscript{(161.5)}
\textsuperscript{14:6.18} El Espíritu Infinito encontró en Havona un terreno donde pudo demostrar la capacidad y la buena voluntad para servir como ministro potencial de la misericordia. En esta creación perfecta, el Espíritu efectuó su ensayo para la aventura de aportar su ministerio a los universos evolutivos.

\par
%\textsuperscript{(161.6)}
\textsuperscript{14:6.19} Esta creación perfecta proporcionó al Espíritu Infinito la oportunidad de participar en la administración del universo con sus dos padres divinos ---de administrar un universo como descendiente Creador y asociado, preparándose así para la administración conjunta de los universos locales bajo la forma de los Espíritus Creativos asociados a los Hijos Creadores.

\par
%\textsuperscript{(161.7)}
\textsuperscript{14:6.20} Los mundos de Havona son el laboratorio mental de los creadores de la mente cósmica y de los ministros para la mente de todas las criaturas que existen. La mente es diferente en cada mundo de Havona, y sirve de modelo para todos los intelectos espirituales y materiales de las criaturas.

\par
%\textsuperscript{(161.8)}
\textsuperscript{14:6.21} Estos mundos perfectos son las escuelas mentales superiores para todos los seres destinados a la sociedad del Paraíso. Proporcionaron al Espíritu abundantes oportunidades para probar la técnica del ministerio mental sobre unas personalidades a quienes esta prueba no afectó pero que sí dio resultados consultivos.

\par
%\textsuperscript{(161.9)}
\textsuperscript{14:6.22} Havona es una compensación para el Espíritu Infinito por su extenso trabajo desinteresado en los universos del espacio. Havona es el hogar y el retiro perfectos para el Ministro incansable de la Mente del tiempo y del espacio.

\par
%\textsuperscript{(161.10)}
\textsuperscript{14:6.23} 4. \textit{El Ser Supremo} ---la unificación evolutiva de la Deidad experiencial. La creación de Havona es la prueba eterna y perfecta de la realidad espiritual del Ser Supremo. Esta creación perfecta es una revelación de la naturaleza espiritual perfecta y simétrica de Dios Supremo antes de que empezara la síntesis, entre el poder y la personalidad, de los reflejos finitos de las Deidades del Paraíso en los universos experienciales del tiempo y del espacio.

\par
%\textsuperscript{(161.11)}
\textsuperscript{14:6.24} En Havona, los potenciales del poder del Todopoderoso están unificados con la naturaleza espiritual del Supremo. Esta creación central es un ejemplo de la unidad eterna del Supremo en el futuro.

\par
%\textsuperscript{(161.12)}
\textsuperscript{14:6.25} Havona es un modelo perfecto de la universalidad en potencia del Supremo. Este universo es un retrato terminado de la perfección futura del Supremo y sugiere el potencial del
Último.

\par
%\textsuperscript{(162.1)}
\textsuperscript{14:6.26} Havona muestra la finalidad de los valores espirituales que existen bajo la forma de unas criaturas vivientes volitivas con un dominio de sí mismas perfecto y supremo; de la mente que existe como equivalente último del espíritu; de la realidad y de la unidad de la inteligencia con un potencial ilimitado.

\par
%\textsuperscript{(162.2)}
\textsuperscript{14:6.27} 5. \textit{Los Hijos Creadores Coordinados.} Havona es el terreno de entrenamiento educativo donde los Migueles del Paraíso se preparan para sus aventuras posteriores de crear los universos. Esta creación divina y perfecta es un modelo para cada Hijo Creador. Se esfuerzan por hacer que sus propios universos alcancen finalmente estos niveles de perfección del Paraíso-Havona.

\par
%\textsuperscript{(162.3)}
\textsuperscript{14:6.28} Un Hijo Creador utiliza a las criaturas de Havona como posibles modelos de personalidad para sus propios hijos mortales y seres espirituales. Los Migueles y otros Hijos Paradisiacos consideran al Paraíso y a Havona como el destino divino de los hijos del tiempo.

\par
%\textsuperscript{(162.4)}
\textsuperscript{14:6.29} Los Hijos Creadores saben que la creación central es la fuente real de ese supercontrol universal indispensable que estabiliza y unifica sus universos locales. Saben que la presencia personal de la influencia omnipresente del Supremo y del Último se encuentra en Havona.

\par
%\textsuperscript{(162.5)}
\textsuperscript{14:6.30} Havona y el Paraíso son la fuente del poder creador de un Hijo Miguel. Aquí residen los seres que cooperan con él en la creación de un universo. Del Paraíso proceden los Espíritus Madres de los Universos, las cocreadoras de los universos locales.

\par
%\textsuperscript{(162.6)}
\textsuperscript{14:6.31} Los Hijos Paradisiacos consideran a la creación central como el hogar de sus padres divinos ---su hogar. Es el lugar donde disfrutan regresando de vez en cuando.

\par
%\textsuperscript{(162.7)}
\textsuperscript{14:6.32} 6. \textit{Las Hijas Ministrantes Coordinadas.} Los Espíritus Madres de los Unive rsos, las cocreadoras de los universos locales, obtienen su formación prepersonal en los mundos de Havona en estrecha asociación con los Espíritus de los Circuitos. En el universo central, las Hijas Espirituales de los universos locales han sido debidamente entrenadas en los métodos de cooperación con los Hijos del Paraíso, sometidas todo el tiempo a la voluntad del Padre.

\par
%\textsuperscript{(162.8)}
\textsuperscript{14:6.33} En los mundos de Havona, el Espíritu y las Hijas del Espíritu encuentran los modelos mentales para todos sus grupos de inteligencias espirituales y materiales, y este universo central es el destino que tendrán algún día las criaturas que el Espíritu Madre de un Universo apadrina en común con un Hijo Creador asociado.

\par
%\textsuperscript{(162.9)}
\textsuperscript{14:6.34} La Creadora Madre de un Universo se acuerda de que el Paraíso y Havona son el lugar de su origen y el hogar del Espíritu Madre Infinito, la residencia de la presencia de la personalidad de la Mente Infinita.

\par
%\textsuperscript{(162.10)}
\textsuperscript{14:6.35} La concesión de las prerrogativas personales como creadora que la Ministra Divina de un Universo utiliza como complemento de un Hijo Creador en el trabajo de crear a las criaturas vivientes volitivas, también provino de este universo central.

\par
%\textsuperscript{(162.11)}
\textsuperscript{14:6.36} Y por último, puesto que es probable que estas Hijas Espirituales del Espíritu Madre Infinito no regresen nunca a su hogar del Paraíso, obtienen una gran satisfacción del fenómeno universal de la reflectividad asociado al Ser Supremo en Havona y personalizado en Majeston en el Paraíso.

\par
%\textsuperscript{(162.12)}
\textsuperscript{14:6.37} 7. \textit{Los Mortales Evolutivos de la Carrera Ascendente.} Havona es el hogar de la personalidad modelo para todos los tipos de mortales, y el hogar de todas las personalidades superhumanas asociadas a los mortales y que no son nativas de las creaciones del tiempo.

\par
%\textsuperscript{(162.13)}
\textsuperscript{14:6.38} Estos mundos proporcionan el estímulo a todos los impulsos humanos de dirigirse hacia la obtención de los verdaderos valores espirituales en los niveles de realidad más elevados que se puedan concebir. Havona es la meta educativa preparadisiaca de todos los mortales ascendentes. Aquí, los mortales alcanzan a la Deidad preparadisiaca ---al Ser Supremo. Havona se mantiene ante todas las criaturas volitivas como el pórtico que permite entrar en el Paraíso y alcanzar a Dios.

\par
%\textsuperscript{(163.1)}
\textsuperscript{14:6.39} El Paraíso es el hogar, y Havona el taller y el terreno de juego, de los finalitarios. Y todo mortal que conoce a Dios anhela ser un finalitario.

\par
%\textsuperscript{(163.2)}
\textsuperscript{14:6.40} El universo central no es solamente el destino establecido para el hombre, sino que es también el punto de partida de la carrera eterna de los finalitarios cuando emprendan algún día la aventura universal no revelada de explorar por experiencia la infinidad del Padre Universal.

\par
%\textsuperscript{(163.3)}
\textsuperscript{14:6.41} Havona continuará funcionando indiscutiblemente con una importancia absonita incluso en las eras futuras del universo, las cuales quizás presencien cómo los peregrinos del espacio intentarán encontrar a Dios en los niveles superfinitos. Havona tiene capacidad para servir como universo educativo para los seres absonitos. Será probablemente la escuela superior cuando los siete superuniversos funcionen como escuela intermedia para los graduados de las escuelas primarias del espacio exterior. Tendemos a opinar que los potenciales del eterno Havona son realmente ilimitados, que el universo central tiene la capacidad eterna de servir como universo educativo experiencial para todos los tipos de seres creados, pasados, presentes o futuros.

\par
%\textsuperscript{(163.4)}
\textsuperscript{14:6.42} [Presentado por un Perfeccionador de la Sabiduría, encargado para esta tarea por los Ancianos de los Días de Uversa.]


\chapter{Documento 15. Los siete superuniversos}
\par
%\textsuperscript{(164.1)}
\textsuperscript{15:0.1} EN lo que se refiere al Padre Universal ---como Padre--- los universos son prácticamente inexistentes; él se encarga de las personalidades; es el Padre de las personalidades. En lo que se refiere al Hijo Eterno y al Espíritu Infinito ---como asociados creadores--- los universos están localizados y son individuales bajo el gobierno conjunto de los Hijos Creadores y de los Espíritus Creativos. En lo que se refiere a la Trinidad del Paraíso, fuera de Havona sólo existen siete universos habitados, los siete superuniversos que poseen su jurisdicción sobre el círculo del primer nivel de espacio posterior a Havona. Los siete Espíritus Maestros irradian su influencia desde la Isla central, haciendo así de la inmensa creación una rueda gigantesca cuyo eje es la Isla eterna del Paraíso, los siete radios las radiaciones de los Siete Espíritus Maestros, y la llanta las regiones exteriores del gran universo.

\par
%\textsuperscript{(164.2)}
\textsuperscript{15:0.2} Al principio de la materialización de la creación universal se formuló el programa séptuple para organizar y gobernar los superuniversos. La primera creación posterior a Havona fue dividida en siete segmentos formidables, y se diseñaron y se construyeron los mundos sede de estos gobiernos superuniversales. El sistema administrativo actual ha existido desde casi la eternidad, y a los gobernantes de estos siete superuniversos se les llama con razón los Ancianos de los Días.

\par
%\textsuperscript{(164.3)}
\textsuperscript{15:0.3} Poca cosa puedo esperar deciros sobre la enorme masa de conocimientos relacionada con los superuniversos, pero en todos estos reinos se encuentra en vigor una técnica para el control inteligente de las fuerzas tanto físicas como espirituales, y las presencias gravitatorias universales funcionan allí con un poder majestuoso y una armonía perfecta. Es importante que os hagáis primero una idea adecuada de la constitución física y de la organización material de los dominios superuniversales, porque entonces estaréis mejor preparados para captar el significado de la maravillosa organización prevista para su gobierno espiritual y para el progreso intelectual de las criaturas volitivas que residen en las miríadas de planetas habitados diseminados aquí y allá por todos estos siete superuniversos.

\section*{1. El nivel espacial de los superuniversos}
\par
%\textsuperscript{(164.4)}
\textsuperscript{15:1.1} Dentro de la gama limitada de los archivos, las observaciones y los recuerdos de las generaciones de un millón o de mil millones de vuestros cortos años, y a todos los efectos prácticos, Urantia y el universo al que pertenece están experimentando la aventura de una larga inmersión inexplorada en un espacio nuevo; pero según los archivos de Uversa, de acuerdo con las observaciones más antiguas, en armonía con la experiencia y los cálculos más amplios de nuestra orden, y como resultado de las conclusiones basadas en éstos y en otros hallazgos, sabemos que los universos están metidos en una procesión ordenada, bien comprendida y perfectamente controlada, que gira con una grandiosidad majestuosa alrededor de la Gran Fuente-Centro Primera y de su universo residencial.

\par
%\textsuperscript{(165.1)}
\textsuperscript{15:1.2} Hace mucho tiempo que hemos descubierto que los siete superuniversos recorren una gran elipse, un gigantesco círculo alargado. Vuestro sistema solar y los otros mundos del tiempo no se están sumergiendo precipitadamente, sin mapas ni brújula, en un espacio desconocido. El universo local al que pertenece vuestro sistema sigue una trayectoria precisa y bien comprendida, en el sentido contrario a las agujas del reloj, alrededor del inmenso recorrido que rodea al universo central. Esta ruta cósmica está bien trazada, y los observadores de estrellas del superuniverso la conocen tan bien como los astrónomos de Urantia conocen las órbitas de los planetas que forman vuestro sistema solar.

\par
%\textsuperscript{(165.2)}
\textsuperscript{15:1.3} Urantia está situada en un universo local y en un superuniverso no completamente organizados, y vuestro universo local se encuentra en las proximidades inmediatas de numerosas creaciones físicas parcialmente terminadas. Pertenecéis a uno de los universos relativamente recientes. Pero actualmente no os precipitáis al azar en un espacio inexplorado ni dais vueltas a ciegas en unas regiones desconocidas. Estáis siguiendo el camino ordenado y predeterminado del nivel espacial del superuniverso. Estáis pasando ahora por el mismo espacio que vuestro sistema planetario, o sus predecesores, atravesaron en las épocas pasadas; y vuestro sistema o sus sucesores atravesarán de nuevo algún día, en el lejano futuro, el mismo espacio en el cual os precipitáis en la actualidad con tanta rapidez.

\par
%\textsuperscript{(165.3)}
\textsuperscript{15:1.4} En la época actual, y tal como se considera la orientación en Urantia, el superuniverso número uno gira casi derecho hacia el norte, en dirección este, aproximadamente enfrente de la residencia paradisiaca de las Grandes Fuentes y Centros y del universo central de Havona. Esta posición, junto con la correspondiente en el oeste, representa el punto físico en el que las esferas del tiempo se acercan más a la Isla eterna. El superuniverso número dos se encuentra en el norte, preparándose para girar hacia el oeste, mientras que el número tres ocupa actualmente el segmento más septentrional de la gran trayectoria espacial, habiendo sobrepasado ya la curva que lo conduce a su descenso hacia el sur. El número cuatro se encuentra en su camino relativamente recto hacia el sur, y sus regiones avanzadas se acercan ahora frente a los Grandes Centros. El número cinco casi ha dejado su posición frente al Centro de los Centros, y continúa su trayectoria directamente hacia el sur justo antes de girar hacia el este; el número seis ocupa la mayor parte de la curva meridional, segmento que vuestro superuniverso casi ha sobrepasado.

\par
%\textsuperscript{(165.4)}
\textsuperscript{15:1.5} Vuestro universo local de Nebadon pertenece a Orvonton, el séptimo superuniverso, que gira entre los superuniversos uno y seis, y que ha doblado no hace mucho tiempo (tal como nosotros calculamos el tiempo) la curva sudeste del nivel espacial superuniversal. Actualmente, el sistema solar al cual pertenece Urantia ha sobrepasado hace pocos miles de millones de años la curvatura meridional, de manera que ahora estáis avanzando más allá de la curva sudeste y os desplazáis velozmente por la larga ruta relativamente recta hacia el norte. Durante épocas incalculables, Orvonton continuará este recorrido casi directo hacia el norte.

\par
%\textsuperscript{(165.5)}
\textsuperscript{15:1.6} Urantia pertenece a un sistema que se encuentra situado cerca de los límites exteriores de vuestro universo local; y vuestro universo local está atravesando actualmente la periferia de Orvonton. Más allá de vosotros hay otros más, pero estáis muy lejos en el espacio de aquellos sistemas físicos que giran alrededor de la gran órbita a una distancia relativamente cercana de la Gran Fuente-Centro.

\section*{2. La organización de los superuniversos}
\par
%\textsuperscript{(165.6)}
\textsuperscript{15:2.1} El Padre Universal es el único que conoce el emplazamiento y el número real de los mundos habitados del espacio; los llama a todos por su nombre y por su número\footnote{\textit{Número y nombre de las estrellas}: Sal 147:4.}. Sólo puedo daros el número aproximado de planetas habitados o habitables, porque algunos universos locales tienen más mundos adecuados para la vida inteligente que otros. Todos los universos locales en proyecto no han sido organizados. Por eso los cálculos aproximados que ofrezco son únicamente con el objeto de dar una idea de la inmensidad de la creación material.

\par
%\textsuperscript{(166.1)}
\textsuperscript{15:2.2} Hay siete superuniversos en el gran universo, y están constituidos aproxima-damente como sigue:

\par
%\textsuperscript{(166.2)}
\textsuperscript{15:2.3} 1. \textit{El sistema.} La unidad básica del supergobierno está compuesta de unos mil mundos habitados o habitables. Los soles resplandecientes, los mundos fríos, los planetas demasiado cercanos a los soles calientes y otras esferas no adecuadas como moradas para las criaturas no están incluídos en este grupo. A estos mil mundos adaptados para mantener la vida se les llama un sistema, pero en los sistemas más jóvenes, sólo un número relativamente pequeño de estos mundos puede ser habitado. Cada planeta habitado está dirigido por un Príncipe Planetario, y cada sistema local tiene una esfera arquitectónica como sede central, estando gobernada por un Soberano del Sistema.

\par
%\textsuperscript{(166.3)}
\textsuperscript{15:2.4} 2. \textit{La constelación.} Cien sistemas (unos 100.000 planetas habitables) forman una constelación. Cada constelación tiene una esfera sede arquitectónica y está presidida por tres Hijos Vorondadeks, los Altísimos. Cada constelación tiene también como observador a un Fiel de los Días, el embajador de la Trinidad del Paraíso.

\par
%\textsuperscript{(166.4)}
\textsuperscript{15:2.5} 3. \textit{El universo local.} Cien constelaciones (unos 10.000.000 de planetas habitables) constituyen un universo local. Cada universo local tiene un magnífico mundo sede arquitectónico y está gobernado por uno de los Hijos de Dios Creadores coordinados de la orden de los Migueles. Cada universo está bendecido por la presencia de un Unión de los Días, el representante de la Trinidad del Paraíso.

\par
%\textsuperscript{(166.5)}
\textsuperscript{15:2.6} 4. \textit{El sector menor.} Cien universos locales (unos 1.000.000.000 de planetas habitables) constituyen un sector menor del gobierno del superuniverso; posee un maravilloso mundo sede desde donde sus gobernantes, los Recientes de los Días, administran los asuntos del sector menor. En la sede de cada sector menor hay tres Recientes de los Días, que son Personalidades Supremas de la Trinidad.

\par
%\textsuperscript{(166.6)}
\textsuperscript{15:2.7} 5. \textit{El sector mayor.} Cien sectores menores (unos 100.000.000.000 de mundos habitables) forman un sector mayor. Cada sector mayor posee una magnífica sede central y está presidido por tres Perfecciones de los Días, que son Personalidades Supremas de la Trinidad.

\par
%\textsuperscript{(166.7)}
\textsuperscript{15:2.8} 6. \textit{El superuniverso.} Diez sectores mayores (aproximadamente 1.000.000.000.000 de planetas habitables) constituyen un superuniverso. Cada superuniverso está provisto de un mundo sede enorme y glorioso, y está gobernado por tres Ancianos de los Días.

\par
%\textsuperscript{(166.8)}
\textsuperscript{15:2.9} 7. \textit{El gran universo.} Siete superuniversos componen el gran universo actual-mente organizado, que consiste en unos siete billones de mundos habitables, más las esferas arquitectónicas y los mil millones de esferas habitadas de Havona. Los superuniversos están gobernados y administrados indirecta y reflectantemente desde el Paraíso por los Siete Espíritus Maestros. Los mil millones de mundos de Havona están administrados directamente por los Eternos de los Días, y una de estas Personalidades Supremas de la Trinidad preside cada una de estas esferas perfectas.

\par
%\textsuperscript{(167.1)}
\textsuperscript{15:2.10} Exceptuando a las esferas del Paraíso-Havona, el plan de la organización del universo prevé las unidades siguientes:

\par
%\textsuperscript{(167.2)}
\textsuperscript{15:2.11} Superuniversos. . . . . . . . . . 7

\par
%\textsuperscript{(167.3)}
\textsuperscript{15:2.12} Sectores mayores. . . . . . . . 70

\par
%\textsuperscript{(167.4)}
\textsuperscript{15:2.13} Sectores menores. . . . . . . 7.000

\par
%\textsuperscript{(167.5)}
\textsuperscript{15:2.14} Universos locales . . . . . . 700.000

\par
%\textsuperscript{(167.6)}
\textsuperscript{15:2.15} Constelaciones. . . . . . . 70.000.000

\par
%\textsuperscript{(167.7)}
\textsuperscript{15:2.16} Sistemas locales. . . . . 7.000.000.000

\par
%\textsuperscript{(167.8)}
\textsuperscript{15:2.17} Planetas habitables . . 7.000.000.000.000

\par
%\textsuperscript{(167.9)}
\textsuperscript{15:2.18} Cada uno de los siete superuniversos está constituido aproximadamente como sigue:

\par
%\textsuperscript{(167.10)}
\textsuperscript{15:2.19} Un sistema contiene aproximadamente. . . . . . . . . 1.000 mundos

\par
%\textsuperscript{(167.11)}
\textsuperscript{15:2.20} Una constelación (100 sistemas). . . . . . . . . . . . 100.000 mundos

\par
%\textsuperscript{(167.12)}
\textsuperscript{15:2.21} Un universo (100 constelaciones) . . . . . . . . . 10.000.000 de mundos

\par
%\textsuperscript{(167.13)}
\textsuperscript{15:2.22} Un sector menor (100 universos). . . . . . . . . 1.000.000.000 de mundos

\par
%\textsuperscript{(167.14)}
\textsuperscript{15:2.23} Un sector mayor (100 sectores menores) . . .100.000.000.000 de mundos

\par
%\textsuperscript{(167.15)}
\textsuperscript{15:2.24} Un superuniverso (10 sectores mayores) . . 1.000.000.000.000 de mundos

\par
%\textsuperscript{(167.16)}
\textsuperscript{15:2.25} Todos estos cálculos son, a lo sumo, aproximaciones, ya que constantemente están surgiendo nuevos sistemas, mientras que otras organizaciones desaparecen temporalmente de la existencia material.

\section*{3. El superuniverso de Orvonton}
\par
%\textsuperscript{(167.17)}
\textsuperscript{15:3.1} Prácticamente todos los reinos estelares visibles a simple vista desde Urantia pertenecen a la séptima sección del gran universo, al superuniverso de Orvonton. El inmenso sistema estelar de la Vía Láctea representa el núcleo central de Orvonton, que se encuentra mucho más allá de las fronteras de vuestro universo local. Este gran agregado de soles, islas oscuras del espacio, estrellas dobles, grupos globulares, nubes de estrellas, nebulosas espirales y otras, junto con miríadas de planetas individuales, forma una agrupación circular y alargada parecida a un reloj, que ocupa alrededor de una séptima parte de los universos evolutivos habitados.

\par
%\textsuperscript{(167.18)}
\textsuperscript{15:3.2} Desde la posición astronómica de Urantia, cuando miráis la gran Vía Láctea a través del corte transversal de los sistemas cercanos, observáis que las esferas de Orvonton viajan en un inmenso plano alargado cuya anchura es mucho más grande que el espesor, y cuya longitud es mucho mayor que la anchura.

\par
%\textsuperscript{(167.19)}
\textsuperscript{15:3.3} La observación de la llamada Vía Láctea revela que la densidad estelar de Orvonton aumenta comparativamente cuando se mira el cielo en una dirección, mientras que la densidad disminuye a cada lado de dicha dirección; el número de estrellas y de otras esferas decrece al alejarnos del plano principal de nuestro superuniverso material. Cuando el ángulo de observación es propicio y se mira a través del cuerpo principal de esta región que posee la máxima densidad, estáis mirando hacia el universo residencial y el centro de todas las cosas.

\par
%\textsuperscript{(167.20)}
\textsuperscript{15:3.4} De las diez divisiones mayores de Orvonton, los astrónomos urantianos han identificado más o menos ocho. Las otras dos son difíciles de reconocer separadamente porque estáis obligados a contemplar estos fenómenos desde el interior. Si pudierais observar el superuniverso de Orvonton desde una posición muy alejada en el espacio, reconoceríais inmediatamente los diez sectores mayores de la séptima galaxia.

\par
%\textsuperscript{(168.1)}
\textsuperscript{15:3.5} El centro de rotación de vuestro sector menor está situado muy lejos en la enorme y densa nube estelar de Sagitario, alrededor de la cual se desplazan vuestro universo local y sus creaciones asociadas, y a los lados opuestos del inmenso sistema subgaláctico de Sagitario podéis observar dos grandes corrientes de nubes de estrellas que surgen como prodigiosas espirales estelares.

\par
%\textsuperscript{(168.2)}
\textsuperscript{15:3.6} El núcleo del sistema físico al que pertenecen vuestro Sol y sus planetas asociados es el centro de la antigua nebulosa de Andronover. Esta nebulosa en otro tiempo espiral fue ligeramente deformada por los trastornos gravitatorios asociados a los acontecimientos que acompañaron al nacimiento de vuestro sistema solar, y que fueron ocasionados por el estrecho acercamiento de una gran nebulosa vecina. Esta casi colisión transformó a Andronover en un agregado un poco globular, pero no destruyó por completo la procesión en dos direcciones de los soles y de sus grupos físicos asociados. Vuestro sistema solar ocupa ahora una posición bastante central en uno de los brazos de esta espiral deformada, y está situado casi a medio camino entre el centro y el borde exterior de la corriente de estrellas.

\par
%\textsuperscript{(168.3)}
\textsuperscript{15:3.7} El sector de Sagitario y todos los otros sectores y divisiones de Orvonton dan vueltas alrededor de Uversa, y una parte de la confusión de los observadores de estrellas urantianos proviene de las ilusiones y de las distorsiones relativas producidas por los múltiples movimientos rotatorios siguientes\footnote{\textit{Revoluciones del universo}: Ez 1:15-19; 10:9-10.}:

\par
%\textsuperscript{(168.4)}
\textsuperscript{15:3.8} 1. La revolución de Urantia alrededor de su Sol.

\par
%\textsuperscript{(168.5)}
\textsuperscript{15:3.9} 2. El recorrido de vuestro sistema solar alrededor del núcleo de la antigua nebulosa de Andronover.

\par
%\textsuperscript{(168.6)}
\textsuperscript{15:3.10} 3. La rotación de la familia estelar de Andronover y de los grupos asociados alrededor del centro de rotación y de gravedad combinados de la nube de estrellas de Nebadon.

\par
%\textsuperscript{(168.7)}
\textsuperscript{15:3.11} 4. El recorrido de la nube estelar local de Nebadon y de sus creaciones asociadas alrededor del centro de su sector menor, situado en Sagitario.

\par
%\textsuperscript{(168.8)}
\textsuperscript{15:3.12} 5. La rotación de los cien sectores menores, incluyendo a Sagitario, alrededor de su sector mayor.

\par
%\textsuperscript{(168.9)}
\textsuperscript{15:3.13} 6. El torbellino de los diez sectores mayores, las llamadas corrientes de estrellas, alrededor de la sede de Orvonton situada en Uversa.

\par
%\textsuperscript{(168.10)}
\textsuperscript{15:3.14} 7. El movimiento de Orvonton y de los seis superuniversos asociados alrededor del Paraíso y de Havona, la procesión en el sentido contrario a las agujas del reloj del nivel espacial superuniversal.

\par
%\textsuperscript{(168.11)}
\textsuperscript{15:3.15} Estos múltiples movimientos son de diversos tipos: Las trayectorias espaciales de vuestro planeta y de vuestro sistema solar son genéticas, inherentes a su origen. El movimiento absoluto de Orvonton en el sentido opuesto a las agujas del reloj también es genético, inherente a los planes arquitectónicos del universo maestro. Pero los movimientos intermedios son de origen compuesto, procediendo por una parte de la segmentación constitutiva de la energía-materia para formar los superuniversos, y por otra parte son producidos por la acción inteligente e intencional de los organizadores de fuerza del Paraíso.

\par
%\textsuperscript{(168.12)}
\textsuperscript{15:3.16} Los universos locales están más próximos los unos de los otros a medida que se acercan a Havona; los circuitos son más numerosos y se superponen cada vez más, capa tras capa. Pero a mayor distancia del centro eterno hay cada vez menos sistemas, capas, circuitos y universos.

\section*{4. Las nebulosas -antepasadas de los universos}
\par
%\textsuperscript{(169.1)}
\textsuperscript{15:4.1} Aunque la creación y la organización de los universos permanece eternamente bajo el control de los Creadores infinitos y de sus asociados, todo el fenómeno se desarrolla de acuerdo con una técnica ordenada y de conformidad con las leyes gravitatorias de la fuerza, la energía y la materia. Pero hay algo misterioso asociado a la carga de fuerza universal del espacio; comprendemos plenamente la organización de las creaciones materiales desde la etapa ultimatónica en adelante, pero no comprendemos por completo la ascendencia cósmica de los ultimatones. Estamos convencidos de que estas fuerzas ancestrales tienen su origen en el Paraíso, porque giran perpetuamente en el espacio penetrado siguiendo exactamente la silueta gigantesca del Paraíso. Aunque no es sensible a la gravedad del Paraíso, esta carga de fuerza del espacio, antepasada de toda materialización, reacciona siempre a la presencia del Paraíso inferior, pues está aparentemente incorporada en un circuito dentro y fuera del centro del Paraíso inferior.

\par
%\textsuperscript{(169.2)}
\textsuperscript{15:4.2} Los organizadores paradisiacos de la fuerza transmutan la potencia espacial en fuerza primordial, y convierten este potencial prematerial en las manifestaciones energéticas primarias y secundarias de la realidad física. Cuando esta energía alcanza los niveles en que responde a la gravedad, los directores del poder y sus asociados del régimen superuniversal aparecen en escena, y empiezan sus manipulaciones interminables destinadas a establecer los múltiples circuitos de poder y canales de energía de los universos del tiempo y del espacio. Así es como la materia física aparece en el espacio, y el escenario está así preparado para inaugurar la organización del universo.

\par
%\textsuperscript{(169.3)}
\textsuperscript{15:4.3} Esta segmentación de la energía es un fenómeno que nunca ha sido resuelto por los físicos de Nebadon. Su dificultad principal reside en que los organizadores paradisiacos de la fuerza son relativamente inaccesibles, ya que los directores vivientes del poder, aunque son competentes para encargarse de la energía espacial, no tienen la menor idea del origen de las energías que manipulan con tanta habilidad e inteligencia.

\par
%\textsuperscript{(169.4)}
\textsuperscript{15:4.4} Los organizadores paradisiacos de la fuerza son los que originan las nebulosas; son capaces de iniciar alrededor de su presencia espacial los enormes ciclones de fuerza que, una vez que se han desencadenado, nunca se pueden detener ni limitar hasta que estas fuerzas que lo impregnan todo son movilizadas para hacer aparecer al final las unidades ultimatónicas de la material universal. Así es como surgen a la existencia las nebulosas espirales y otras, las ruedas madres de los soles que tienen un origen directo y de sus diversos sistemas. En el espacio exterior se pueden observar diez formas diferentes de nebulosas, las fases de la evolución universal primaria, y estas inmensas ruedas de energía han tenido el mismo origen que las de los siete superuniversos.

\par
%\textsuperscript{(169.5)}
\textsuperscript{15:4.5} El tamaño de las nebulosas, así como el número resultante y la masa total de sus descendientes estelares y planetarios, varían enormemente. Una nebulosa formadora de soles que se encuentra exactamente al norte de las fronteras de Orvonton, pero dentro del nivel espacial superuniversal, ya ha dado origen a unos cuarenta mil soles, y la rueda madre sigue arrojando soles, la mayoría de los cuales tienen un tamaño mucho mayor que el vuestro. Algunas de las nebulosas más grandes del espacio exterior están dando origen a no menos de cien millones de soles.

\par
%\textsuperscript{(169.6)}
\textsuperscript{15:4.6} Las nebulosas no están directamente relacionadas con ninguna de las unidades administrativas tales como los sectores menores o los universos locales, aunque algunos universos locales han sido organizados con los productos de una sola nebulosa. Cada universo local contiene exactamente una cien milésima parte de la carga energética total de un superuniverso, independientemente de su relación con las nebulosas, ya que la energía no está organizada por nebulosas ---está distribuida de manera universal.

\par
%\textsuperscript{(170.1)}
\textsuperscript{15:4.7} Todas las nebulosas espirales no se ocupan de producir soles. Algunas han conservado el control de muchos de sus descendientes estelares separados, y su apariencia espiral resulta del hecho de que sus soles salen del brazo nebular en estrecha formación pero regresan por diversos caminos, lo que facilita observarlos en un punto pero es más difícil verlos cuando se encuentran muy dispersos por sus diferentes caminos de regreso más alejados y fuera del brazo de la nebulosa. No hay muchas nebulosas formadoras de soles que estén activas actualmente en Orvonton, aunque Andrómeda, que está fuera del superuniverso habitado, es muy activa. Esta nebulosa tan distante es visible a simple vista, y cuando la observéis, deteneos a pensar que la luz que contempláis salió de aquellos lejanos soles hace cerca de un millón de años.

\par
%\textsuperscript{(170.2)}
\textsuperscript{15:4.8} La galaxia de la Vía Láctea está compuesta de un gran número de antiguas nebulosas espirales y de otro tipo, y muchas de ellas conservan todavía su configuración original. Pero a consecuencia de las catástrofes internas y de la atracción externa, muchas han sufrido tales deformaciones y adaptaciones que han hecho que estos enormes agregados aparezcan como gigantescas masas luminosas de soles resplandecientes semejantes a la Nube de Magallanes. Los cúmulos de estrellas de tipo globular predominan cerca de los márgenes exteriores de Orvonton.

\par
%\textsuperscript{(170.3)}
\textsuperscript{15:4.9} Las inmensas nubes de estrellas de Orvonton deberían ser consideradas como agregados individuales de materia, comparables a las distintas nebulosas observables en las regiones espaciales exteriores a la galaxia de la Vía Láctea. Sin embargo, muchas de las llamadas nubes de estrellas del espacio sólo están compuestas de materia gaseosa. El potencial energético de estas nubes de gas estelares es increíblemente enorme, y una parte de ellas es absorbida por los soles cercanos y vuelta a enviar al espacio bajo la forma de emanaciones solares.

\section*{5. El origen de los cuerpos espaciales}
\par
%\textsuperscript{(170.4)}
\textsuperscript{15:5.1} La mayor parte de la masa que contienen los soles y los planetas de un superuniverso se origina en las ruedas nebulares; la acción directa de los directores del poder (como en la construcción de las esferas arquitectónicas) organiza una parte muy pequeña de la masa superuniversal, aunque una cantidad constantemente variable de materia se origina en el espacio abierto.

\par
%\textsuperscript{(170.5)}
\textsuperscript{15:5.2} En lo que se refiere a su origen, la mayoría de los soles, planetas y otras esferas se pueden clasificar en uno de los diez grupos siguientes:

\par
%\textsuperscript{(170.6)}
\textsuperscript{15:5.3} 1. \textit{Los anillos de contracción concéntricos.} Todas las nebulosas no son espirales. Muchas nebulosas inmensas sufren una condensación mediante la formación de anillos múltiples, en lugar de dividirse en un sistema estelar doble o de evolucionar como una espiral. Durante largos períodos, este tipo de nebulosa aparece como un enorme sol central rodeado de numerosas nubes gigantescas de formaciones de materia envolventes de apariencia anular.

\par
%\textsuperscript{(170.7)}
\textsuperscript{15:5.4} 2. \textit{Los torbellinos de estrellas} engloban a aquellos soles que son arrojados de las grandes ruedas madres de gases extremadamente calientes. No son arrojados como anillos, sino en procesiones hacia la derecha y la izquierda. Los torbellinos de estrellas también se originan en las nebulosas que no son espirales.

\par
%\textsuperscript{(170.8)}
\textsuperscript{15:5.5} 3. \textit{Los planetas de explosión gravitatoria.} Cuando un sol nace de una nebulosa espiral o bien de una barrada, es expulsado con frecuencia a una distancia considerable. Un sol así es extremadamente gaseoso y, posteriormente, después de haberse enfriado y condensado un poco, quizás gire por casualidad cerca de alguna enorme masa de materia, ya se trate de un sol gigantesco o de una isla oscura del espacio. Un acercamiento así puede no ser suficiente para producir una colisión, pero sin embargo suficiente para permitir que la atracción gravitatoria del cuerpo más grande provoque convulsiones mareomotrices en el más pequeño, iniciándose así una serie de trastornos periódicos que tienen lugar simultáneamente en los lados opuestos del sol dislocado. En su punto culminante, estas erupciones explosivas producen una serie de agregados de materia de tamaños variables que pueden ser proyectados más allá de la zona de recuperación por la gravedad del sol en erupción, estabilizándose así en sus propias órbitas alrededor de uno de los dos cuerpos implicados en este episodio. Más tarde, los grupos más grandes de materia se unen y atraen gradualmente hacia sí a los cuerpos más pequeños. Muchos planetas sólidos de los sistemas menores surgen de esta manera a la existencia. Vuestro propio sistema solar tuvo precisamente este origen.

\par
%\textsuperscript{(171.1)}
\textsuperscript{15:5.6} 4. \textit{Las hijas centrífugas planetarias.} Cuando los soles enormes se encuentran en ciertas etapas de su desarrollo, y si su velocidad de rotación se acelera mucho, empiezan a despedir grandes cantidades de materia que posteriormente se pueden agrupar para formar pequeños mundos que continúan girando alrededor del sol central.

\par
%\textsuperscript{(171.2)}
\textsuperscript{15:5.7} 5. \textit{Las esferas con deficiencias de gravedad.} El tamaño de las estrellas individuales tiene un límite crítico. Cuando un sol alcanza este límite, está condenado a partirse a menos que disminuya su velocidad de rotación; se produce una escisión solar y nace una nueva estrella doble de esta variedad. Posteriormente se pueden formar numerosos planetas pequeños como subproducto de esta ruptura gigantesca.

\par
%\textsuperscript{(171.3)}
\textsuperscript{15:5.8} 6. \textit{Las estrellas contraídas.} En los sistemas más pequeños, el planeta exterior más grande a veces atrae hacia sí a los mundos vecinos, mientras que los planetas más cercanos al sol empiezan su caída final. En vuestro sistema solar, un final así significaría que los cuatro planetas interiores serían reclamados por el Sol, mientras que Júpiter, el planeta mayor, crecería enormemente debido a la captación de los mundos restantes. Esta forma de terminar un sistema solar conduciría al nacimiento de dos soles adyacentes pero desiguales, una manera de formarse las estrellas dobles. Estas catástrofes son poco frecuentes, salvo en la periferia de los agregados estelares de los superuniversos.

\par
%\textsuperscript{(171.4)}
\textsuperscript{15:5.9} 7. \textit{Las esferas acumulativas.} Se pueden acumular lentamente pequeños planetas a partir de la inmensa cantidad de materia que circula en el espacio. Crecen por adición meteórica y debido a colisiones menores. Las condiciones de algunos sectores del espacio favorecen estas formas de nacimiento planetario. Muchos mundos habitados han tenido este origen.

\par
%\textsuperscript{(171.5)}
\textsuperscript{15:5.10} Algunas islas densas y oscuras son el resultado directo de la unión de las energías que se transmutan en el espacio. Otro grupo de estas islas oscuras ha surgido a la existencia debido a la acumulación de enormes cantidades de materia fría, de simples fragmentos y meteoros, que circulan por el espacio. Estos agregados de materia nunca han estado calientes y, a excepción de su densidad, su composición es muy similar a la de Urantia.

\par
%\textsuperscript{(171.6)}
\textsuperscript{15:5.11} 8. \textit{Los soles consumidos.} Algunas islas oscuras del espacio son soles aislados extinguidos que han emitido toda su energía espacial disponible. Estas unidades organizadas de materia se acercan a la condensación total, a una fusión prácticamente completa; estas enormes masas de materia extremadamente condensada necesitan una era tras otra para recargarse en los circuitos del espacio, y prepararse así para nuevos ciclos de funcionamiento en el universo después de una colisión o de algún otro suceso cósmico igualmente revivificante.

\par
%\textsuperscript{(171.7)}
\textsuperscript{15:5.12} 9. \textit{Las esferas producidas por las colisiones.} En aquellas regiones donde los enjambres son densos, las colisiones no son raras. Estos reajustes astronómicos van acompañados de enormes cambios energéticos y de transmutaciones de la materia. Las colisiones que afectan a los soles muertos influyen particularmente en la creación de extensas fluctuaciones de energía. Los desechos de las colisiones constituyen a menudo los núcleos materiales que formarán posteriormente los cuerpos planetarios adaptados para ser habitados por los mortales.

\par
%\textsuperscript{(172.1)}
\textsuperscript{15:5.13} 10. \textit{Los mundos arquitectónicos.} Son los mundos que se construyen de acuerdo con unos planes y unas especificaciones con vistas a una finalidad especial, como es el caso de Salvington, la sede de vuestro universo local, y de Uversa, la sede del gobierno de nuestro superuniverso.

\par
%\textsuperscript{(172.2)}
\textsuperscript{15:5.14} Existen otras numerosas técnicas para producir los soles y separar los planetas, pero los procedimientos anteriormente mencionados indican los métodos por medio de los cuales la inmensa mayoría de los sistemas estelares y de las familias planetarias son traídos a la existencia. Intentar describir todas las diversas técnicas implicadas en las metamorfosis estelares y en la evolución planetaria necesitaría que narráramos casi cien maneras diferentes de formar soles y de dar origen a los planetas. A medida que vuestros astrónomos escruten los cielos, observarán fenómenos que indicarán todas estas formas de evolución estelar, pero raramente detectarán la prueba de la formación de esos pequeños grupos no luminosos de materia que sirven como planetas habitados, las esferas más importantes de las inmensas creaciones materiales.

\section*{6. Las esferas del espacio}
\par
%\textsuperscript{(172.3)}
\textsuperscript{15:6.1} Independientemente de su origen, las diversas esferas del espacio se pueden clasificar en las divisiones mayores siguientes:

\par
%\textsuperscript{(172.4)}
\textsuperscript{15:6.2} 1. Los soles ---las estrellas del espacio.

\par
%\textsuperscript{(172.5)}
\textsuperscript{15:6.3} 2. Las islas oscuras del espacio.

\par
%\textsuperscript{(172.6)}
\textsuperscript{15:6.4} 3. Los cuerpos espaciales menores ---cometas, meteoros y planetesimales.

\par
%\textsuperscript{(172.7)}
\textsuperscript{15:6.5} 4. Los planetas, incluídos los mundos habitados.

\par
%\textsuperscript{(172.8)}
\textsuperscript{15:6.6} 5. Las esferas arquitectónicas ---los mundos hechos a medida.

\par
%\textsuperscript{(172.9)}
\textsuperscript{15:6.7} A excepción de las esferas arquitectónicas, todos los cuerpos espaciales han tenido un origen evolutivo, evolutivo en el sentido de que no han sido traídos a la existencia por orden de la Deidad, evolutivo en el sentido de que los actos creadores de Dios se han desarrollado mediante una técnica espacio-temporal a través del trabajo de muchas inteligencias creadas y existenciadas por la Deidad.

\par
%\textsuperscript{(172.10)}
\textsuperscript{15:6.8} \textit{Los soles.} Son las estrellas del espacio en todas sus diversas fases de existencia. Algunos son sistemas espaciales solitarios en vías de evolución; otros son estrellas dobles, sistemas planetarios en vías de contraerse o de desaparecer. Las estrellas del espacio existen en no menos de mil estados y etapas diferentes. Estáis familiarizados con los soles que emiten luz acompañada de calor; pero hay también soles que brillan sin calor.

\par
%\textsuperscript{(172.11)}
\textsuperscript{15:6.9} Un sol ordinario continuará emitiendo luz y calor durante billones y billones de años, lo cual ilustra la inmensa reserva de energía que contiene cada unidad de materia. La energía real almacenada en estas partículas invisibles de materia física es casi inimaginable. Y esta energía se vuelve casi enteramente disponible bajo la forma de luz cuando es sometida a la enorme presión calorífica y a las actividades energéticas asociadas que prevalecen en el interior de los soles resplandecientes. Otras condiciones aún permiten que estos soles transformen y envíen una gran parte de la energía espacial que les llega por los circuitos espaciales establecidos. Muchas fases de la energía física y todas las formas de materia son atraídas por la dínamo solar y distribuidas posteriormente por ella. Los soles sirven de esta manera como aceleradores locales de la circulación de la energía, actuando como estaciones automáticas de control del poder.

\par
%\textsuperscript{(172.12)}
\textsuperscript{15:6.10} El superuniverso de Orvonton está iluminado y calentado por más de diez billones de soles resplandecientes. Estos soles son las estrellas que se pueden observar desde vuestro sistema astronómico. Más de dos billones están demasiado lejanos y son demasiado pequeños como para ser nunca vistos desde Urantia. Pero en el universo maestro existen tantos soles como vasos de agua en los océanos de vuestro mundo.

\par
%\textsuperscript{(173.1)}
\textsuperscript{15:6.11} \textit{Las islas oscuras del espacio.} Son los soles muertos y los otros grandes agregados de materia desprovistos de luz y de calor. Las islas oscuras tienen a veces una masa enorme y ejercen una poderosa influencia sobre el equilibrio universal y la manipulación de la energía. La densidad de algunas de estas grandes masas es casi increíble. Y esta gran concentración de masa permite que estas islas oscuras funcionen como poderosas ruedas equilibradoras, manteniendo eficazmente a raya a los grandes sistemas vecinos. Mantienen el equilibrio gravitatorio del poder en muchas constelaciones; muchos sistemas físicos que de otra manera se lanzarían rápidamente hacia su destrucción en los soles cercanos, son mantenidos a salvo dentro de la atracción gravitatoria de estas islas oscuras guardianas. Gracias a esta función podemos situarlas con precisión. Hemos medido la atracción gravitatoria de los cuerpos luminosos, y podemos calcular así el tamaño y el emplazamiento exactos de las islas oscuras del espacio, que funcionan con tanta eficacia para mantener firmemente en su trayectoria a un sistema determinado.

\par
%\textsuperscript{(173.2)}
\textsuperscript{15:6.12} \textit{Los cuerpos espaciales menores.} Los meteoros y otras pequeñas partículas de materia que circulan y evolucionan en el espacio constituyen un enorme agregado de energía y de sustancia material.

\par
%\textsuperscript{(173.3)}
\textsuperscript{15:6.13} Muchos cometas son los descendientes salvajes y no estabilizados de las ruedas madres solares, que se van poniendo gradualmente bajo el control del sol central dominante. Los cometas tienen también otros numerosos orígenes. La cola de un cometa se dirige en sentido contrario al cuerpo o al sol que lo atrae debido a la reacción eléctrica de sus gases extremadamente extendidos y a causa de la presión real de la luz y de otras energías que emanan del sol. Este fenómeno constituye una de las pruebas evidentes de la realidad de la luz y de sus energías asociadas; demuestra que la luz tiene peso. La luz es una sustancia real, y no simplemente las ondulaciones de un éter hipotético.

\par
%\textsuperscript{(173.4)}
\textsuperscript{15:6.14} \textit{Los planetas.} Son los mayores agregados de materia que siguen una órbita alrededor de un sol o de algún otro cuerpo espacial; su tamaño varía desde los planetesimales hasta las enormes esferas gaseosas, líquidas o sólidas. Cuando los mundos fríos que se han formado mediante la reunión de la materia espacial circulante se encuentran por casualidad en una relación apropiada con un sol cercano, son los planetas más ideales para albergar a los habitantes inteligentes. Por regla general, los soles muertos no son convenientes para la vida; normalmente están demasiado lejos de un sol vivo y resplandeciente, y además son en conjunto demasiado masivos; la gravedad es enorme en su superficie.

\par
%\textsuperscript{(173.5)}
\textsuperscript{15:6.15} En vuestro superuniverso no hay un planeta frío entre cuarenta que sea habitable por los seres de vuestra orden. Y por supuesto, los soles supercalientes y los mundos alejados muy fríos son inadecuados para albergar una vida superior. En vuestro sistema solar sólo hay tres planetas en la actualidad que convienen para albergar la vida. Por su tamaño, su densidad y su posición, Urantia es ideal en muchos aspectos para el hábitat humano.

\par
%\textsuperscript{(173.6)}
\textsuperscript{15:6.16} Las leyes del comportamiento de la energía física son básicamente universales, pero las influencias locales tienen mucho que ver con las condiciones físicas que prevalecen en los planetas individuales y en los sistemas locales. Los innumerables mundos del espacio están caracterizados por una variedad casi infinita de vida de las criaturas y de otras manifestaciones vivientes. Sin embargo, hay ciertos elementos en común en un grupo de mundos asociados de un sistema dado, aunque existe también un modelo universal de vida inteligente. Hay relaciones físicas entre los sistemas planetarios que pertenecen al mismo circuito físico, y que se siguen de cerca los unos a los otros en su recorrido sin fin alrededor de la órbita de los universos.

\section*{7. Las esferas arquitectónicas}
\par
%\textsuperscript{(174.1)}
\textsuperscript{15:7.1} Aunque cada gobierno superuniversal ejerce su dirección desde cerca del centro de los universos evolutivos de su segmento espacial, ocupa un mundo hecho a medida y poblado de personalidades acreditadas. Estos mundos sede son esferas arquitectónicas, unos cuerpos espaciales construidos específicamente para su finalidad especial. Aunque comparten la luz de los soles cercanos, estas esferas están iluminadas y calentadas de forma independiente. Cada una tiene un sol que emite luz sin calor, como los satélites del Paraíso, y cada una recibe su suministro de calor mediante la circulación de ciertas corrientes de energía cerca de la superficie de la esfera. Estos mundos sede pertenecen a uno de los sistemas más grandes situados cerca del centro astronómico de sus superuniversos respectivos\footnote{\textit{Nuevo cielo}: Is 66:22; 2 P 3:13; Ap 21:1.}.

\par
%\textsuperscript{(174.2)}
\textsuperscript{15:7.2} El tiempo está uniformado en las sedes de los superuniversos. El día oficial del superuniverso de Orvonton es igual a casi treinta días del tiempo de Urantia, y el año de Orvonton equivale a cien días oficiales. Este año de Uversa es oficial en el séptimo superuniverso y corresponde a tres mil días menos veintidós minutos del tiempo de Urantia, unos ocho años más una quinta parte de vuestros años.

\par
%\textsuperscript{(174.3)}
\textsuperscript{15:7.3} Los mundos sede de los siete superuniversos comparten la naturaleza y la grandiosidad del Paraíso, su arquetipo central de perfección. En realidad, todos los mundos sede son paradisiacos. Son en verdad residencias celestiales, y su tamaño material, su belleza morontial y su gloria espiritual van creciendo desde Jerusem hasta la Isla central. Y todos los satélites de estos mundos sede son también esferas arquitectónicas.

\par
%\textsuperscript{(174.4)}
\textsuperscript{15:7.4} Los diversos mundos sede están provistos de todas las fases de la creación material y espiritual. Todos los tipos de seres materiales, morontiales y espirituales se sienten en su hogar en estos mundos de encuentro de los universos. A medida que las criaturas mortales ascienden por el universo, pasando de los mundos materiales a los mundos espirituales, nunca pierden su aprecio por los niveles anteriores de existencia, ni el placer que experimentaron en ellos.

\par
%\textsuperscript{(174.5)}
\textsuperscript{15:7.5} \textit{Jerusem}\footnote{\textit{Jerusem (capital del sistema)}: Ap 21:2 ff.}, la sede de vuestro sistema local de Satania, tiene sus siete mundos de cultura de transición, y cada uno de ellos está rodeado por siete satélites entre los que se encuentran los siete mundos de las mansiones de detención morontial, la primera residencia del hombre después de la muerte. La palabra cielo, tal como se ha utilizado en Urantia, a veces se ha referido a estos siete mundos de las mansiones, denominándose primer cielo al primer mundo de las mansiones, y así sucesivamente hasta el séptimo.

\par
%\textsuperscript{(174.6)}
\textsuperscript{15:7.6} \textit{Edentia,} la sede de vuestra constelación de Norlatiadek, tiene sus setenta satélites de cultura y de preparación para la vida social, y en ellos residen los ascendentes después de finalizar el régimen de Jerusem relacionado con la movilización, la unificación y la comprensión de la personalidad.

\par
%\textsuperscript{(174.7)}
\textsuperscript{15:7.7} \textit{Salvington,} la capital de Nebadon, vuestro universo local, está rodeada de diez grupos universitarios de cuarenta y nueve esferas cada uno. Aquí el hombre es espiritualizado después de haberse hecho sociable en su constelación.

\par
%\textsuperscript{(174.8)}
\textsuperscript{15:7.8} \textit{Umenor la tercera,} la sede de Ensa, vuestro sector menor, está rodeada por las siete esferas dedicadas a los estudios físicos superiores de la vida ascendente.

\par
%\textsuperscript{(174.9)}
\textsuperscript{15:7.9} \textit{Umayor la quinta,} la sede de Splandon, vuestro sector mayor, está rodeada por las setenta esferas de formación intelectual avanzada del superuniverso.

\par
%\textsuperscript{(175.1)}
\textsuperscript{15:7.10} \textit{Uversa,} la sede de Orvonton, vuestro superuniverso, está rodeada directamente por las siete universidades superiores de enseñanza espiritual avanzada para las criaturas volitivas ascendentes. Cada uno de estos siete grupos de esferas maravillosas está compuesto de setenta mundos especializados que contienen miles y miles de instituciones y de organizaciones repletas dedicadas a la educación universal y a la cultura espiritual, donde los peregrinos del tiempo son reeducados y examinados de nuevo con miras a su largo viaje hacia Havona. Los peregrinos del tiempo que llegan son recibidos siempre en estos mundos asociados, pero los graduados que se marchan hacia Havona salen siempre directamente de las orillas de Uversa.

\par
%\textsuperscript{(175.2)}
\textsuperscript{15:7.11} Uversa es la sede espiritual y administrativa para cerca de un billón de mundos habitados o habitables. La gloria, la grandiosidad y la perfección de la capital de Orvonton sobrepasan todas las maravillas de las creaciones del espacio-tiempo.

\par
%\textsuperscript{(175.3)}
\textsuperscript{15:7.12} Si todos los universos locales en proyecto y sus partes componentes estuvieran creados, en los siete superuniversos habría un poco menos de quinientos mil millones de mundos arquitectónicos.

\section*{8. El control y la regulación de la energía}
\par
%\textsuperscript{(175.4)}
\textsuperscript{15:8.1} Las esferas sede de los superuniversos están construidas de tal manera que pueden funcionar como reguladoras eficaces de la energía y del poder para sus diversos sectores, sirviendo como puntos focales para dirigir la energía hacia los universos locales que los componen. Ejercen una poderosa influencia sobre el equilibrio y el control de las energías físicas que circulan a través del espacio organizado.

\par
%\textsuperscript{(175.5)}
\textsuperscript{15:8.2} Los centros de poder y los controladores físicos de los superuniversos, que son entidades inteligentes vivientes y semivivientes constituidas para esta finalidad expresa, realizan otras funciones regulativas. Estos centros y controladores del poder son difíciles de comprender; las órdenes inferiores no son volitivas, no poseen voluntad, no eligen, sus funciones son muy inteligentes pero aparentemente automáticas e inherentes a su organización altamente especializada. Los centros de poder y los controladores físicos de los superuniversos asumen la dirección y el control parcial de los treinta sistemas energéticos con que cuenta el ámbito de la gravita. Los circuitos de la energía física administrados por los centros de poder de Uversa necesitan un poco más de 968 millones de años para completar la circunvalación del superuniverso.

\par
%\textsuperscript{(175.6)}
\textsuperscript{15:8.3} La energía en evolución tiene sustancia; tiene peso, aunque el peso es siempre relativo, dependiendo de la velocidad de rotación, de la masa y de la antigravedad. La masa de la materia tiende a retrasar la velocidad de la energía; y la velocidad siempre presente de la energía representa la velocidad con que ha sido dotada inicialmente, menos el retraso debido a la masa que encuentra a su paso, más la función reguladora de los controladores energéticos vivientes del superuniverso y la influencia física que ejercen los cuerpos cercanos muy calientes o fuertemente cargados.

\par
%\textsuperscript{(175.7)}
\textsuperscript{15:8.4} El plan universal para mantener el equilibrio entre la materia y la energía necesita que las unidades materiales menores se construyan y se destruyan sin cesar. Los Directores del Poder Universal tienen la capacidad de condensar y detener, o de dilatar y liberar, cantidades variables de energía.

\par
%\textsuperscript{(175.8)}
\textsuperscript{15:8.5} Si la influencia retardadora tuviera una duración suficiente, la gravedad terminaría por convertir toda la energía en materia si no fuera por dos factores: en primer lugar, debido a las influencias antigravitatorias de los controladores de la energía, y en segundo lugar, debido a que la materia organizada tiende a desintegrarse bajo ciertas condiciones que se encuentran en las estrellas muy calientes y bajo ciertas condiciones particulares que se dan en el espacio en las proximidades de los cuerpos fríos de materia condensada muy cargados de energía.

\par
%\textsuperscript{(176.1)}
\textsuperscript{15:8.6} Cuando la masa se agrupa en exceso y amenaza con desequilibrar la energía, con agotar los circuitos físicos del poder, los controladores físicos intervienen a menos que la propia tendencia ulterior de la gravedad a materializar excesivamente la energía sea anulada a consecuencia de una colisión entre los gigantes muertos del espacio, disipando por completo en un instante los conjuntos acumulados de gravedad. Durante estas colisiones, las enormes masas de materia se convierten repentinamente en la forma más rara de energía, y la lucha por el equilibrio universal comienza de nuevo. Finalmente, los sistemas físicos más grandes se estabilizan, se asientan físicamente, y se ponen a girar en los circuitos equilibrados y establecidos de los superuniversos. Después de este suceso ya no se producirán más colisiones, ni otras catástrofes devastadoras, en estos sistemas establecidos.

\par
%\textsuperscript{(176.2)}
\textsuperscript{15:8.7} Durante los períodos de mayor cantidad de energía, se producen perturbaciones del poder y fluctuaciones térmicas acompañadas de manifestaciones eléctricas. Durante los períodos de menor cantidad de energía, la materia tiende a reunirse, a condensarse y a descontrolarse cada vez más en los circuitos más delicadamente equilibrados, con los ajustes resultantes debidos a las mareas o a las colisiones, los cuales restablecen rápidamente el equilibrio entre la energía circulante y la materia más literalmente estabilizada. Una de las tareas de los observadores celestiales de estrellas consiste en prever y por otra parte en comprender este comportamiento probable de los soles resplandecientes y de las islas oscuras del espacio.

\par
%\textsuperscript{(176.3)}
\textsuperscript{15:8.8} Somos capaces de reconocer la mayoría de las leyes que gobiernan el equilibrio universal y de predecir una gran parte de aquello que está relacionado con la estabilidad del universo. Nuestras previsiones son fiables en la práctica, pero siempre nos enfrentamos con ciertas fuerzas que no son totalmente sensibles a las leyes que conocemos sobre el control de la energía y el comportamiento de la materia. Todos los fenómenos físicos son cada vez más difíciles de predecir a medida que nos alejamos del Paraíso hacia los universos. Cuando sobrepasamos las fronteras de la administración personal de los Gobernantes del Paraíso, nos enfrentamos con la incapacidad creciente de hacer nuestros cálculos según las normas establecidas y la experiencia adquirida durante las observaciones relacionadas exclusivamente con los fenómenos físicos de los sistemas astronómicos cercanos. Incluso en los reinos de los siete superuniversos, vivimos en medio de unas acciones de fuerza y de unas reacciones energéticas que impregnan todos nuestros dominios y se extienden con un equilibrio unificado por todas las regiones del espacio exterior.

\par
%\textsuperscript{(176.4)}
\textsuperscript{15:8.9} Cuanto más nos alejamos, con más certeza encontramos esos fenómenos variables e imprevisibles que caracterizan tan infaliblemente las actividades y la presencia insondables de los Absolutos y de las Deidades experienciales. Y estos fenómenos deben indicar algún tipo de supercontrol universal de todas las cosas.

\par
%\textsuperscript{(176.5)}
\textsuperscript{15:8.10} En la actualidad, el superuniverso de Orvonton parece descargarse; los universos exteriores parecen estar terminándose con vistas a unas actividades futuras sin precedentes; el universo central de Havona está eternamente estabilizado. La gravedad y la ausencia de calor (el frío) organizan y mantienen unida a la materia; el calor y la antigravedad desorganizan la materia y disipan la energía. Los directores del poder y los organizadores de la fuerza vivientes son el secreto del control especial y de la dirección inteligente de las metamorfosis sin fin que dan como resultado la construcción, la destrucción y la reconstrucción del universo. Las nebulosas pueden dispersarse, los soles consumirse, los sistemas desaparecer y los planetas perecer, pero los universos no se agotan.

\section*{9. Los circuitos de los superuniversos .}
\par
%\textsuperscript{(176.6)}
\textsuperscript{15:9.1} Los circuitos universales del Paraíso impregnan realmente los reinos de los siete superuniversos. Estos circuitos presenciales son los siguientes: la gravedad de personalidad del Padre Universal, la gravedad espiritual del Hijo Eterno, la gravedad mental del Actor Conjunto y la gravedad material de la Isla eterna.

\par
%\textsuperscript{(177.1)}
\textsuperscript{15:9.2} Además de los circuitos universales del Paraíso y además de las actividades y de la presencia de los Absolutos y de las Deidades experienciales, dentro del nivel espacial superuniversal sólo funcionan dos divisiones de circuitos energéticos o separaciones de poder: los circuitos de los superuniversos y los circuitos de los universos locales.

\par
%\textsuperscript{(177.2)}
\textsuperscript{15:9.3} \textit{Los circuitos de los superuniversos:}

\par
%\textsuperscript{(177.3)}
\textsuperscript{15:9.4} 1. El circuito unificador de inteligencia de uno de los Siete Espíritus Maestros del Paraíso. Este circuito de la mente cósmica está limitado a un solo superuniverso.

\par
%\textsuperscript{(177.4)}
\textsuperscript{15:9.5} 2. El circuito del servicio reflectante de los Siete Espíritus Reflectantes de cada superuniverso.

\par
%\textsuperscript{(177.5)}
\textsuperscript{15:9.6} 3. Los circuitos secretos de los Monitores de Misterio, interasociados y dirigidos de alguna manera desde Divinington hacia el Padre Universal en el Paraíso.

\par
%\textsuperscript{(177.6)}
\textsuperscript{15:9.7} 4. El circuito de comunión recíproca entre el Hijo Eterno y sus Hijos Paradisiacos.

\par
%\textsuperscript{(177.7)}
\textsuperscript{15:9.8} 5. La presencia instantánea del Espíritu Infinito.

\par
%\textsuperscript{(177.8)}
\textsuperscript{15:9.9} 6. Las transmisiones del Paraíso, los comunicados espaciales de Havona.

\par
%\textsuperscript{(177.9)}
\textsuperscript{15:9.10} 7. Los circuitos energéticos de los centros de poder y de los controladores físicos.

\par
%\textsuperscript{(177.10)}
\textsuperscript{15:9.11} \textit{Los circuitos de los universos locales:}

\par
%\textsuperscript{(177.11)}
\textsuperscript{15:9.12} 1. El espíritu donador de los Hijos Paradisiacos, el Consolador de los mundos de donación. El Espíritu de la Verdad, el espíritu de Miguel en Urantia.

\par
%\textsuperscript{(177.12)}
\textsuperscript{15:9.13} 2. El circuito de las Ministras Divinas, los Espíritus Madres de los universos locales, el Espíritu Santo de vuestro mundo.

\par
%\textsuperscript{(177.13)}
\textsuperscript{15:9.14} 3. El circuito del ministerio de inteligencia de un universo local, que incluye la presencia de los espíritus ayudantes de la mente que funciona de manera diversa.

\par
%\textsuperscript{(177.14)}
\textsuperscript{15:9.15} Cuando en un universo local se desarrolla tal armonía espiritual que sus circuitos individuales y combinados se vuelven indistinguibles de los del superuniverso, cuando esta identidad de funcionamiento y esta unidad de ministerio predominan realmente, entonces el universo local entra inmediatamente en los circuitos establecidos de la luz y la vida, obteniendo enseguida el derecho a ser admitido en la confederación espiritual de la unión perfeccionada de la supercreación. Los requisitos para ser admitido en los consejos de los Ancianos de los Días, para ser miembro de la confederación superuniversal, son los siguientes:

\par
%\textsuperscript{(177.15)}
\textsuperscript{15:9.16} 1. \textit{Estabilidad física.} Las estrellas y los planetas de un universo local deben estar en equilibrio; los períodos de las metamorfosis estelares inminentes deben haber terminado. El universo debe estar avanzando en una trayectoria clara; su órbita debe estar estabilizada con seguridad y de manera definitiva.

\par
%\textsuperscript{(177.16)}
\textsuperscript{15:9.17} 2. \textit{Lealtad espiritual.} Debe existir un estado de reconocimiento universal y de lealtad hacia el Hijo Soberano de Dios que preside los asuntos de dicho universo local. Debe haber nacido un estado de cooperación armoniosa entre los planetas, los sistemas y las constelaciones individuales de todo el universo local.

\par
%\textsuperscript{(177.17)}
\textsuperscript{15:9.18} A vuestro universo local ni siquiera se le considera que pertenece al orden físico estabilizado del superuniverso, y mucho menos que posee la calidad de miembro en la familia espiritual reconocida del supergobierno. Aunque Nebadon no tiene todavía representantes en Uversa, a nosotros que pertenecemos al gobierno superuniversal nos envían a sus mundos en misiones especiales de vez en cuando, tal como yo he venido a Urantia directamente desde Uversa. Prestamos toda la ayuda posible a vuestros directores y gobernantes para resolver sus difíciles problemas; estamos deseando ver que vuestro universo se cualifique para ser plenamente admitido en las creaciones asociadas de la familia superuniversal.

\section*{10. Los gobernantes de los superuniversos}
\par
%\textsuperscript{(178.1)}
\textsuperscript{15:10.1} Las capitales de los superuniversos son las sedes del gobierno espiritual superior de los dominios del espacio-tiempo. La rama ejecutiva del supergobierno, que tiene su origen en los Consejos de la Trinidad, está dirigida directamente por uno de los Siete Espíritus Maestros con una supervisión suprema, unos seres que ocupan puestos de autoridad paradisiaca y administran los superuniversos a través de los Siete Ejecutivos Supremos estacionados en los siete mundos especiales del Espíritu Infinito, los satélites más exteriores del Paraíso.

\par
%\textsuperscript{(178.2)}
\textsuperscript{15:10.2} Las sedes de los superuniversos son los lugares donde residen los Espíritus Reflectantes y los Ayudantes Reflectantes de Imágenes. Desde esta posición intermedia, estos seres maravillosos dirigen sus extraordinarias operaciones de reflectividad, aportando así su ministerio al universo central que se encuentra por encima de ellos y a los universos locales que están por debajo.

\par
%\textsuperscript{(178.3)}
\textsuperscript{15:10.3} Cada superuniverso está presidido por tres Ancianos de los Días\footnote{\textit{Ancianos de los Días}: Dn 7:9,13,22.}, los jefes ejecutivos conjuntos del supergobierno. En su rama ejecutiva, el personal del gobierno superuniversal está compuesto de siete grupos diferentes:

\par
%\textsuperscript{(178.4)}
\textsuperscript{15:10.4} 1. Los Ancianos de los Días.

\par
%\textsuperscript{(178.5)}
\textsuperscript{15:10.5} 2. Los Perfeccionadores de la Sabiduría.

\par
%\textsuperscript{(178.6)}
\textsuperscript{15:10.6} 3. Los Consejeros Divinos.

\par
%\textsuperscript{(178.7)}
\textsuperscript{15:10.7} 4. Los Censores Universales.

\par
%\textsuperscript{(178.8)}
\textsuperscript{15:10.8} 5. Los Mensajeros Poderosos.

\par
%\textsuperscript{(178.9)}
\textsuperscript{15:10.9} 6. Los Elevados en Autoridad.

\par
%\textsuperscript{(178.10)}
\textsuperscript{15:10.10} 7. Los que no tienen Nombre ni Número.

\par
%\textsuperscript{(178.11)}
\textsuperscript{15:10.11} A los tres Ancianos de los Días los ayuda directamente un cuerpo de mil millones de Perfeccionadores de la Sabiduría, con quienes están asociados tres mil millones de Consejeros Divinos. Mil millones de Censores Universales están destinados a la administración de cada superuniverso. Estos tres grupos son Personalidades Coordinadas de la Trinidad, y tienen su origen directa y divinamente en la Trinidad del Paraíso.

\par
%\textsuperscript{(178.12)}
\textsuperscript{15:10.12} Las otras tres órdenes, los Mensajeros Poderosos, Los Elevados en Autoridad y Los que no tienen Nombre ni Número, son mortales ascendentes glorificados. La primera de estas órdenes se elevó a través del régimen ascendente y pasó por Havona en la época de Grandfanda. Después de alcanzar el Paraíso fueron enrolados en el Cuerpo de la Finalidad, abrazados por la Trinidad del Paraíso, y asignados posteriormente al servicio celestial de los Ancianos de los Días. Como clase, estas tres órdenes son conocidas como los Hijos de la Consecución Trinitizados, han tenido un origen doble pero ahora se encuentran al servicio de la Trinidad. La rama ejecutiva del gobierno superuniversal fue así ampliada para incluir a los hijos glorificados y perfeccionados de los mundos evolutivos.

\par
%\textsuperscript{(178.13)}
\textsuperscript{15:10.13} El consejo coordinado del superuniverso está compuesto de los siete grupos ejecutivos anteriormente mencionados y de los gobernantes de los sectores y otros supervisores regionales siguientes:

\par
%\textsuperscript{(179.1)}
\textsuperscript{15:10.14} 1. Los Perfecciones de los Días ---los gobernantes de los sectores mayores del superuniverso.

\par
%\textsuperscript{(179.2)}
\textsuperscript{15:10.15} 2. Los Recientes de los Días ---los directores de los sectores menores del superuniverso.

\par
%\textsuperscript{(179.3)}
\textsuperscript{15:10.16} 3. Los Uniones de los Días ---los asesores paradisiacos de los gobernantes de los universos locales.

\par
%\textsuperscript{(179.4)}
\textsuperscript{15:10.17} 4. Los Fieles de los Días ---los consejeros paradisiacos de los Altísimos dirigentes de los gobiernos de las constelaciones.

\par
%\textsuperscript{(179.5)}
\textsuperscript{15:10.18} 5. Los Hijos Instructores Trinitarios que pueden encontrarse de servicio en la sede del superuniverso.

\par
%\textsuperscript{(179.6)}
\textsuperscript{15:10.19} 6. Los Eternos de los Días que pueden hallarse presentes en la sede del superuniverso.

\par
%\textsuperscript{(179.7)}
\textsuperscript{15:10.20} 7. Los siete Ayudantes Reflectantes de Imágenes ---los portavoces de los siete Espíritus Reflectantes que, a través de ellos, representan a los Siete Espíritus Maestros del Paraíso.

\par
%\textsuperscript{(179.8)}
\textsuperscript{15:10.21} Los Ayudantes Reflectantes de Imágenes actúan también como representantes de numerosos grupos de seres que ejercen su influencia en los gobiernos superuniversales, pero que por diversas razones no se encuentran en la actualidad plenamente activos en sus aptitudes individuales. En este grupo están incluídos: la manifestación en evolución de la personalidad superuniversal del Ser Supremo, los Supervisores Incalificados del Supremo, los Vicegerentes Calificados del Último, los agentes reflectantes de enlace innominados de Majeston y los representantes espirituales superpersonales del Hijo Eterno.

\par
%\textsuperscript{(179.9)}
\textsuperscript{15:10.22} En los mundos sede de los superuniversos es posible encontrar en casi todo momento a los representantes de todos los grupos de seres creados. Los poderosos seconafines y otros miembros de la inmensa familia del Espíritu Infinito efectúan el trabajo ministrante rutinario de los superuniversos. En las tareas de estos centros maravillosos de administración, control, ministerio y juicio ejecutivo superuniversales, las inteligencias de todas las esferas de la vida universal se mezclan para llevar a cabo un servicio eficaz, una administración sabia, un ministerio amoroso y un juicio justo.

\par
%\textsuperscript{(179.10)}
\textsuperscript{15:10.23} Los superuniversos no mantienen ningún tipo de representación diplomática; están completamente aislados los unos de los otros. Sólo conocen sus asuntos mutuos a través de la cámara paradisiaca de análisis, corrección y distribución de la información, mantenida por los Siete Espíritus Maestros. Sus gobernantes trabajan en los consejos de la sabiduría divina por el bienestar de sus propios superuniversos, sin tener en cuenta lo que pueda estar sucediendo en otras secciones de la creación universal. Este aislamiento continuará hasta el momento en que la soberanía de la personalidad del Ser Supremo experiencial en evolución sea un hecho consumado y consiga la correlación de los superuniversos.

\section*{11. La asamblea deliberante}
\par
%\textsuperscript{(179.11)}
\textsuperscript{15:11.1} En los mundos tales como Uversa es donde los seres que representan la autocracia de la perfección y la democracia de la evolución se encuentran frente a frente. La rama ejecutiva del supergobierno se origina en los reinos de la perfección; la rama legislativa surge del florecimiento de los universos evolutivos.

\par
%\textsuperscript{(179.12)}
\textsuperscript{15:11.2} La asamblea deliberante del superuniverso está limitada al mundo sede. Este consejo legislativo o consultivo está compuesto de siete cámaras, y todos los universos locales admitidos a los consejos superuniversales eligen a un representante nativo para cada una de ellas. Los consejos superiores de dichos universos locales eligen a estos representantes entre los peregrinos ascendentes graduados de Orvonton que se encuentran en Uversa y están acreditados para ser transportados a Havona. El período medio de su servicio es de unos cien años del tiempo oficial superuniversal.

\par
%\textsuperscript{(180.1)}
\textsuperscript{15:11.3} Nunca he conocido un desacuerdo entre los ejecutivos de Orvonton y la asamblea de Uversa. Hasta ahora, en la historia de nuestro superuniverso, el cuerpo deliberante nunca ha aprobado una recomendación que la división ejecutiva del supergobierno haya dudado siquiera en llevar hacia adelante. Siempre ha prevalecido el acuerdo de trabajo y la armonía más perfectos, lo que demuestra el hecho de que los seres evolutivos pueden alcanzar realmente las alturas de una sabiduría perfeccionada que los cualifica para asociarse con las personalidades de origen perfecto y de naturaleza divina. La presencia de las asambleas deliberantes en las sedes de los superuniversos revela la sabiduría, y presagia el triunfo final, de todo el inmenso concepto evolutivo del Padre Universal y de su Hijo Eterno.

\section*{12. Los tribunales supremos}
\par
%\textsuperscript{(180.2)}
\textsuperscript{15:12.1} Cuando hablamos de las ramas ejecutiva y deliberante del gobierno de Uversa, podríais razonar que, por su analogía con ciertas formas de los gobiernos civiles urantianos, debemos tener una tercera rama o rama judicial, y así es; pero ésta no posee un personal independiente. Nuestros tribunales están constituidos como sigue: Según la naturaleza y la gravedad del caso, preside un Anciano de los Días, un Perfeccionador de la Sabiduría o un Consejero Divino. Las pruebas a favor o en contra de un individuo, un planeta, un sistema, una constelación o un universo son presentadas e interpretadas por los Censores. La defensa de los hijos del tiempo y de los planetas evolutivos está a cargo de los Mensajeros Poderosos, los observadores oficiales del gobierno superuniversal en los universos y en los sistemas locales. La actitud del gobierno superior está representada por Los Elevados en Autoridad. Habitualmente, el veredicto es formulado por una comisión de tamaño variable compuesta por igual por Los que no tienen Nombre ni Número y por un grupo de personalidades comprensivas elegidas en la asamblea deliberante.

\par
%\textsuperscript{(180.3)}
\textsuperscript{15:12.2} Las audiencias de los Ancianos de los Días son los tribunales supremos de revisión que dictan las sentencias espirituales para todos los universos que dependen de ellos. Los Hijos Soberanos de los universos locales son supremos en sus propios dominios; sólo están sujetos al supergobierno en la medida en que le someten voluntariamente sus asuntos para recibir el consejo o el juicio de los Ancianos de los Días, excepto en las cuestiones relacionadas con la extinción de las criaturas volitivas. Las órdenes de juicio se originan en los universos locales, pero las sentencias que implican la extinción de las criaturas volitivas siempre se formulan en la sede del superuniverso y son ejecutadas desde allí. Los Hijos de los universos locales pueden decretar la supervivencia del hombre mortal, pero sólo los Ancianos de los Días pueden emitir un juicio ejecutivo sobre las cuestiones de vida y de muerte eternas.

\par
%\textsuperscript{(180.4)}
\textsuperscript{15:12.3} En todos los asuntos que no necesitan un proceso, la presentación de unas pruebas, los Ancianos de los Días o sus asociados pronuncian las sentencias, y estos fallos son siempre unánimes. Aquí estamos tratando con los consejos de la perfección. No existen desacuerdos ni opiniones minoritarias en los decretos de estos tribunales supremos y superlativos.

\par
%\textsuperscript{(180.5)}
\textsuperscript{15:12.4} Con algunas pocas excepciones, los supergobiernos ejercen su jurisdicción sobre todas las cosas y todos los seres de sus dominios respectivos. Los fallos y las decisiones de las autoridades superuniversales no se pueden apelar, puesto que representan las opiniones coincidentes de los Ancianos de los Días y del Espíritu Maestro que preside desde el Paraíso los destinos del superuniverso interesado.

\section*{13. Los gobiernos de los sectores}
\par
%\textsuperscript{(181.1)}
\textsuperscript{15:13.1} Un \textit{sector mayor} consta aproximadamente de una décima parte de un superuniverso y consiste en cien sectores menores, diez mil universos locales y cerca de cien mil millones de mundos habitables. Estos sectores mayores están administrados por tres Perfecciones de los Días, que son Personalidades Supremas de la Trinidad.

\par
%\textsuperscript{(181.2)}
\textsuperscript{15:13.2} Los tribunales de los Perfecciones de los Días están compuestos en gran parte como los de los Ancianos de los Días, salvo que no juzgan espiritualmente a los reinos. El trabajo de los gobiernos de estos sectores mayores está relacionado principalmente con el estado intelectual de una extensa creación. Con vistas a presentar su informe ante los tribunales de los Ancianos de los Días, los sectores mayores retienen, juzgan, distribuyen y clasifican todos los asuntos de importancia superuniversal de naturaleza rutinaria y administrativa que no están relacionados directamente con la administración espiritual de los reinos o con el desarrollo de los planes para la ascensión de los mortales, formulados por los Gobernantes del Paraíso. El personal del gobierno de un sector mayor no es diferente al del superuniverso.

\par
%\textsuperscript{(181.3)}
\textsuperscript{15:13.3} Al igual que los magníficos satélites de Uversa se ocupan de vuestra preparación espiritual final para trasladaros a Havona, los setenta satélites de Umayor la quinta están dedicados a vuestra formación y desarrollo intelectuales de tipo superuniversal. Aquí se reúnen desde todo Orvonton los seres sabios que trabajan incansablemente para preparar a los mortales del tiempo con vistas a su progreso ulterior hacia la carrera de la eternidad. La mayor parte de esta formación de los mortales ascendentes se lleva a cabo en los setenta mundos de estudio.

\par
%\textsuperscript{(181.4)}
\textsuperscript{15:13.4} Los gobiernos de los \textit{sectores menores} están presididos por tres Recientes de los Días. Su administración se ocupa principalmente del control, la unificación y la estabilización físicas, así como de la coordinación rutinaria de la administración de los universos locales que los componen. Cada sector menor abarca no menos de cien universos locales, diez mil constelaciones, un millón de sistemas, o alrededor de mil millones de mundos habitables.

\par
%\textsuperscript{(181.5)}
\textsuperscript{15:13.5} Los mundos sede de los sectores menores son los grandes puntos de reunión de los Controladores Físicos Maestros. Estos mundos sede están rodeados por siete esferas de instrucción que forman las escuelas de admisión al superuniverso, y son los centros donde se enseña el conocimiento físico y administrativo relacionado con el universo de universos.

\par
%\textsuperscript{(181.6)}
\textsuperscript{15:13.6} Los administradores de los gobiernos de los sectores menores están bajo la jurisdicción inmediata de los gobernantes del sector mayor. Los Recientes de los Días reciben todos los informes de las observaciones y coordinan todas las recomendaciones que llegan hasta un superuniverso procedentes de los Uniones de los Días que están estacionados como observadores y consejeros trinitarios en las esferas sede de los universos locales, y procedentes de los Fieles de los Días que están similarmente vinculados a los consejos de los Altísimos en las sedes de las constelaciones. Todos estos informes son transmitidos a los Perfecciones de los Días en los sectores mayores, para ser pasados posteriormente a los tribunales de los Ancianos de los Días. El régimen de la Trinidad se extiende así desde las constelaciones de los universos locales hasta la sede del superuniverso. Las sedes de los sistemas locales no tienen representantes de la Trinidad.

\section*{14. Los objetivos de los siete superuniversos}
\par
%\textsuperscript{(181.7)}
\textsuperscript{15:14.1} La evolución de los siete superuniversos está revelando siete objetivos principales. Cada objetivo principal de la evolución superuniversal sólo encontrará su expresión más plena en uno de los siete superuniversos, y por eso cada superuniverso tiene una función especial y una naturaleza sin igual.

\par
%\textsuperscript{(182.1)}
\textsuperscript{15:14.2} Orvonton, el séptimo superuniverso al que pertenece vuestro universo local, es conocido principalmente por su extraordinaria y generosa donación de ministerio misericordioso hacia los mortales de los reinos. Es célebre por la manera en que prevalece la justicia templada por la misericordia, y donde domina un poder condicionado por la paciencia, mientras que se hacen abundantes sacrificios de tiempo para asegurar la estabilización de la eternidad. Orvonton es una demostración universal del amor y de la misericordia.

\par
%\textsuperscript{(182.2)}
\textsuperscript{15:14.3} Sin embargo, es muy difícil describir nuestro concepto sobre la verdadera naturaleza del objetivo evolutivo que se está desarrollando en Orvonton, pero podríamos sugerirlo diciendo que en esta supercreación sentimos que los seis objetivos singulares de la evolución cósmica, tal como se manifiestan en las seis supercreaciones asociadas, se están interasociando aquí en un significado de totalidad; es por esta razón por lo que a veces hemos conjeturado que, en el lejano futuro, la personalización evolucionada y consumada de Dios Supremo gobernará desde Uversa los siete superuniversos perfeccionados con toda la majestad experiencial del poder soberano todopoderoso que entonces habrá alcanzado.

\par
%\textsuperscript{(182.3)}
\textsuperscript{15:14.4} Orvonton es único en su naturaleza e individual en su destino, y lo mismo sucede con cada uno de los seis superuniversos asociados. Sin embargo, una gran cantidad de cosas que suceden en Orvonton no os son reveladas, y muchas de estas características no reveladas de la vida de Orvonton encontrarán una expresión más completa en algún otro superuniverso. Los siete objetivos de la evolución superuniversal están en vigor en el conjunto de los siete superuniversos, pero cada supercreación sólo expresará de la manera más plena uno de estos objetivos. Para comprender más cosas sobre estos objetivos superuniversales os tendríamos que revelar muchas cosas que no entendéis, e incluso entonces sólo comprenderíais muy pocas de ellas. La totalidad de esta narración sólo presenta una visión fugaz de la inmensa creación a la cual pertenecen vuestro mundo y vuestro sistema local.

\par
%\textsuperscript{(182.4)}
\textsuperscript{15:14.5} Vuestro mundo se llama Urantia y tiene el número 606 en el grupo planetario, o sistema, de Satania. Este sistema tiene actualmente 619 mundos habitados, y más de doscientos planetas adicionales evolucionan favorablemente para convertirse en mundos habitados en algún momento del futuro.

\par
%\textsuperscript{(182.5)}
\textsuperscript{15:14.6} Satania tiene un mundo sede llamado Jerusem y es el sistema número veinticuatro de la constelación de Norlatiadek. Vuestra constelación Norlatiadek está compuesta de cien sistemas locales y tiene un mundo sede llamado Edentia. Norlatiadek tiene el número setenta en el universo de Nebadon. El universo local de Nebadon consta de cien constelaciones y tiene una capital conocida como Salvington. El universo de Nebadon es el número ochenta y cuatro del sector menor de Ensa.

\par
%\textsuperscript{(182.6)}
\textsuperscript{15:14.7} El sector menor de Ensa está compuesto de cien universos locales y tiene una capital llamada Umenor la tercera. Este sector menor es el número tres del sector mayor de Splandon. Splandon está compuesto de cien sectores menores y tiene un mundo sede llamado Umayor la quinta. Es el quinto sector mayor del superuniverso de Orvonton, el séptimo segmento del gran universo. Así es como podéis situar vuestro planeta en el sistema de la organización y de la administración del universo de universos.

\par
%\textsuperscript{(182.7)}
\textsuperscript{15:14.8} El número de vuestro mundo Urantia en el gran universo es el 5.342.482.337.666. Éste es el número con el que está registrado en Uversa y en el Paraíso, vuestro número en el catálogo de los mundos habitados. Conozco el número de registro de las esferas físicas, pero es de una magnitud tan extraordinaria que tiene un significado muy poco práctico para la mente mortal.

\par
%\textsuperscript{(183.1)}
\textsuperscript{15:14.9} Vuestro planeta es miembro de un cosmos inmenso; pertenecéis a una familia casi infinita de mundos, pero vuestra esfera está administrada con tanta precisión y favorecida con tanto amor como si se tratara del único mundo habitado que existe.

\par
%\textsuperscript{(183.2)}
\textsuperscript{15:14.10} [Presentado por un Censor Universal procedente de Uversa.]


\chapter{Documento 16. Los Siete Espíritus Maestros}
\par
%\textsuperscript{(184.1)}
\textsuperscript{16:0.1} LOS siete Espíritus Maestros del Paraíso son las personalidades primarias del Espíritu Infinito. En este séptuple acto creativo de reproducción de sí mismo, el Espíritu Infinito agotó las posibilidades asociativas matemáticamente inherentes a la existencia de hecho de las tres personas de la Deidad. Si hubiera sido posible engendrar un mayor número de Espíritus Maestros, habrían sido creados, pero sólo existen siete posibilidades asociativas, y sólo siete, inherentes a tres Deidades. Esto explica por qué el universo funciona en siete grandes divisiones, y por qué el número siete es básicamente fundamental en su organización y administración.

\par
%\textsuperscript{(184.2)}
\textsuperscript{16:0.2} Los Siete Espíritus Maestros tienen pues su origen en las siete semejanzas siguientes, de las cuales obtienen sus características individuales:

\par
%\textsuperscript{(184.3)}
\textsuperscript{16:0.3} 1. El Padre Universal.

\par
%\textsuperscript{(184.4)}
\textsuperscript{16:0.4} 2. El Hijo Eterno.

\par
%\textsuperscript{(184.5)}
\textsuperscript{16:0.5} 3. El Espíritu Infinito.

\par
%\textsuperscript{(184.6)}
\textsuperscript{16:0.6} 4. El Padre y el Hijo.

\par
%\textsuperscript{(184.7)}
\textsuperscript{16:0.7} 5. El Padre y el Espíritu.

\par
%\textsuperscript{(184.8)}
\textsuperscript{16:0.8} 6. El Hijo y el Espíritu.

\par
%\textsuperscript{(184.9)}
\textsuperscript{16:0.9} 7. El Padre, el Hijo y el Espíritu.

\par
%\textsuperscript{(184.10)}
\textsuperscript{16:0.10} Sabemos muy poca cosa acerca de la actuación del Padre y del Hijo en la creación de los Espíritus Maestros. Aparentemente fueron traídos a la existencia gracias a los actos personales del Espíritu Infinito, pero nos han informado claramente que tanto el Padre como el Hijo participaron en su origen.

\par
%\textsuperscript{(184.11)}
\textsuperscript{16:0.11} Estos Siete Espíritus del Paraíso son como uno solo en lo referente al carácter y a la naturaleza espirituales, pero en todos los demás aspectos de la identidad son muy diferentes, y las diferencias individuales de cada uno de ellos se disciernen inequívocamente en los resultados de sus actividades en los superuniversos. Todos los planes posteriores de los siete segmentos del gran universo ---e incluso de los segmentos correlativos del espacio exterior--- han estado condicionados por la diversidad, distinta a la espiritual, de estos Siete Espíritus Maestros que ejercen una supervisión suprema y última.

\par
%\textsuperscript{(184.12)}
\textsuperscript{16:0.12} Los Espíritus Maestros tienen muchas funciones, pero su terreno particular en el momento actual consiste en la supervisión central de los siete superuniversos. Cada Espíritu Maestro mantiene una enorme sede focal de fuerza que circula lentamente alrededor de la periferia del Paraíso, manteniendo siempre una posición opuesta al superuniverso que supervisa directamente y en el punto focal paradisiaco de control del poder especializado y de la distribución segmentaria de la energía para ese superuniverso. Las líneas radiales que marcan los límites de cualquier superuniverso convergen efectivamente en la sede paradisiaca del Espíritu Maestro que lo supervisa.

\section*{1. La relación con la Deidad trina}
\par
%\textsuperscript{(185.1)}
\textsuperscript{16:1.1} El Creador Conjunto, el Espíritu Infinito, es necesario para completar la personalización trina de la Deidad indivisa. Esta personalización triple de la Deidad posee la posibilidad inherente de expresarse individual y asociativamente de siete maneras; de ahí que el plan posterior consistente en crear unos universos habitados por seres inteligentes y potencialmente espirituales, que expresaran debidamente al Padre, al Hijo y al Espíritu, hizo inevitable la personalización de los Siete Espíritus Maestros. Hemos llegado a hablar de la personalización triple de la Deidad como de la \textit{inevitabilidadabsoluta,} mientras que hemos llegado a considerar la aparición de los Siete Espíritus Maestros como la \textit{inevitabilidad subabsoluta.}

\par
%\textsuperscript{(185.2)}
\textsuperscript{16:1.2} Aunque los Siete Espíritus Maestros no expresan del todo a la Deidad \textit{triple,} son el retrato eterno de la Deidad \textit{séptuple,} de las funciones activas y asociativas de las tres personas eternas de la Deidad. Por medio de estos Siete Espíritus, en ellos y a través de ellos, el Padre Universal, el Hijo Eterno o el Espíritu Infinito, o cualquier asociación de dos de ellos, es capaz de actuar como tal. Cuando el Padre, el Hijo y el Espíritu actúan juntos, pueden ejercer su actividad a través del Espíritu Maestro Número Siete, y así lo hacen, pero no como Trinidad. Los Espíritus Maestros representan individual y colectivamente todas y cada una de las funciones posibles de la Deidad, simples y múltiples, pero no colectivas, no las de la Trinidad. El Espíritu Maestro Número Siete no actúa personalmente con respecto a la Trinidad del Paraíso, y es precisamente por eso por lo que puede actuar \textit{personalmente} por el Ser Supremo.

\par
%\textsuperscript{(185.3)}
\textsuperscript{16:1.3} Pero cuando los Siete Espíritus Maestros dejan sus sedes individuales de poder personal y de autoridad superuniversal, y se reúnen alrededor del Actor Conjunto ante la presencia trina de la Deidad del Paraíso, inmediatamente representan de manera colectiva el poder, la sabiduría y la autoridad funcionales de la Deidad indivisa ---de la Trinidad--- para los universos en evolución y en ellos. Esta unión paradisiaca de la expresión primordial séptuple de la Deidad engloba realmente, abarca literalmente, todos los atributos y actitudes de las tres Deidades eternas en los niveles de la Supremacía y de la Ultimidad. A todos los efectos prácticos, los Siete Espíritus Maestros abarcan de inmediato el ámbito funcional del Supremo-Último para el universo maestro y en él.

\par
%\textsuperscript{(185.4)}
\textsuperscript{16:1.4} Por lo que podemos discernir, estos Siete Espíritus están asociados con las actividades divinas de las tres personas eternas de la Deidad; no detectamos ninguna prueba de que estén asociados directamente con las presencias funcionales de las tres fases eternas del Absoluto. Cuando los Espíritus Maestros están asociados, representan a las Deidades del Paraíso en lo que se puede concebir en líneas generales como el campo de acción finito. Este campo puede englobar muchas cosas que son últimas, pero \textit{no} absolutas.

\section*{2. La relación con el Espíritu Infinito}
\par
%\textsuperscript{(185.5)}
\textsuperscript{16:2.1} Al igual que el Hijo Eterno y Original es revelado a través de las personas de los Hijos divinos cuyo número aumenta constantemente, el Espíritu Infinito y Divino es revelado a través de los canales de los Siete Espíritus Maestros y de sus grupos de espíritus asociados. En el centro de los centros, el Espíritu Infinito es accesible, pero todos los que alcanzan el Paraíso no son capaces de discernir inmediatamente su personalidad y su presencia diferenciada; pero todos los que alcanzan el universo central pueden comunicarse, y de hecho se comunican inmediatamente, con uno de los Siete Espíritus Maestros, con aquel que preside el superuniverso del que procede el peregrino espacial recién llegado.

\par
%\textsuperscript{(186.1)}
\textsuperscript{16:2.2} El Padre Paradisiaco sólo habla al universo de universos a través de su Hijo, mientras que él y el Hijo sólo actúan conjuntamente a través del Espíritu Infinito. Fuera del Paraíso y de Havona, el Espíritu Infinito sólo \textit{habla} a través de las voces de los Siete Espíritus Maestros.

\par
%\textsuperscript{(186.2)}
\textsuperscript{16:2.3} El Espíritu Infinito ejerce la influencia de su \textit{presencia personal} dentro de los confines del sistema Paraíso-Havona; en otras partes, su presencia espiritual personal es ejercida por uno de los Siete Espíritus Maestros y a través de él. Por consiguiente, la presencia espiritual superuniversal de la Fuente-Centro Tercera está condicionada, en cualquier mundo o individuo, por la naturaleza única del Espíritu Maestro que supervisa ese segmento de la creación. A la inversa, las líneas combinadas de la fuerza y de la inteligencia espirituales pasan hacia el interior hasta la Tercera Persona de la Deidad a través de los Siete Espíritus Maestros.

\par
%\textsuperscript{(186.3)}
\textsuperscript{16:2.4} Los Siete Espíritus Maestros están dotados colectivamente de los atributos supremo-últimos de la Fuente-Centro Tercera. Aunque cada uno de ellos comparte individualmente esta dotación, los atributos de la omnipotencia, la omnisciencia y la omnipresencia sólo los revelan de manera colectiva. Ninguno de ellos puede actuar así de forma universal; como individuos y en el ejercicio de estos poderes de supremacía y de ultimidad, cada uno de ellos está limitado personalmente al superuniverso que supervisa directamente.

\par
%\textsuperscript{(186.4)}
\textsuperscript{16:2.5} Todo lo que se os ha dicho acerca de la divinidad y la personalidad del Actor Conjunto se aplica igualmente y por completo a los Siete Espíritus Maestros, que distribuyen tan eficazmente el Espíritu Infinito a los siete segmentos del gran universo de acuerdo con su dotación divina y a la manera de sus naturalezas diferentes e individualmente únicas. Por eso sería apropiado aplicar todos los nombres del Espíritu Infinito, o cualquiera de ellos, al grupo colectivo de los siete. Colectivamente forman una sola cosa con el Creador Conjunto en todos los niveles subabsolutos.

\section*{3. Identidad y diversidad de los Espíritus Maestros}
\par
%\textsuperscript{(186.5)}
\textsuperscript{16:3.1} Los Siete Espíritus Maestros son unos seres indescriptibles, pero son clara y definitivamente personales. Tienen nombres, pero elegimos presentarlos por su número. Como personalizaciones primarias del Espíritu Infinito son semejantes, pero como expresiones primarias de las siete asociaciones posibles de la Deidad trina sus naturalezas son esencialmente distintas, y esta diversidad de naturaleza determina que su comportamiento superuniversal sea diferente. A estos Siete Espíritus Maestros se les puede describir como sigue:

\par
%\textsuperscript{(186.6)}
\textsuperscript{16:3.2} \textit{Espíritu Maestro Número Uno.} Este Espíritu es de una manera especial la representación directa del Padre Paradisiaco. Es una manifestación particular y eficaz del poder, el amor y la sabiduría del Padre Universal. Es el asociado íntimo y el consejero celestial del jefe de los Monitores de Misterio, del ser que preside el Colegio de los Ajustadores Personalizados en Divinington. En todas las asociaciones de los Siete Espíritus Maestros, el Espíritu Maestros Número Uno es siempre el que habla por el Padre Universal.

\par
%\textsuperscript{(186.7)}
\textsuperscript{16:3.3} Este Espíritu preside el primer superuniverso, y aunque manifiesta infaliblemente la naturaleza divina de una personalización primaria del Espíritu Infinito, parece que su carácter se asemeja más especialmente al Padre Universal. Siempre está en conexión personal con los siete Espíritus Reflectantes de la sede del primer superuniverso.

\par
%\textsuperscript{(187.1)}
\textsuperscript{16:3.4} \textit{Espíritu Maestro Número Dos.} Este Espíritu muestra adecuadamente la naturaleza incomparable y el carácter encantador del Hijo Eterno, el primogénito de toda la creación. Siempre está en estrecha asociación con todas las órdenes de Hijos de Dios cada vez que éstos se hallan en el universo residencial como individuos o en alegre cónclave. En todas las asambleas de los Siete Espíritus Maestros, siempre habla por el Hijo Eterno y en nombre de él.

\par
%\textsuperscript{(187.2)}
\textsuperscript{16:3.5} Este Espíritu dirige los destinos del superuniverso número dos y gobierna este inmenso dominio casi como lo haría el Hijo Eterno. Siempre está en conexión con los siete Espíritus Reflectantes situados en la capital del segundo superuniverso.

\par
%\textsuperscript{(187.3)}
\textsuperscript{16:3.6} \textit{Espíritu Maestro Número Tres.} Esta personalidad espiritual se parece especialmente al Espíritu Infinito, y dirige los movimientos y el trabajo de muchas personalidades elevadas del Espíritu Infinito. Preside sus asambleas y está estrechamente asociado con todas las personalidades que tienen su origen exclusivo en la Fuente-Centro Tercera. Cuando los Siete Espíritus Maestros están en consejo, el Espíritu Maestro Número Tres es el que siempre habla por el Espíritu Infinito.

\par
%\textsuperscript{(187.4)}
\textsuperscript{16:3.7} Este Espíritu está a cargo del superuniverso número tres, y administra los asuntos de este segmento casi como lo haría el Espíritu Infinito. Siempre está en conexión con los Espíritus Reflectantes de la sede del tercer superuniverso.

\par
%\textsuperscript{(187.5)}
\textsuperscript{16:3.8} \textit{Espíritu Maestro Número Cuatro.} Como comparte las naturalezas combinadas del Padre y del Hijo, este Espíritu Maestro es la influencia determinante con respecto a las políticas y los procedimientos del Padre-Hijo en los consejos de los Siete Espíritus Maestros. Este Espíritu es el jefe que dirige y aconseja a los seres ascendentes que han alcanzado al Espíritu Infinito y se han vuelto así candidatos para ver al Hijo y al Padre. Patrocina el enorme grupo de personalidades que tienen su origen en el Padre y el Hijo. Cuando es necesario representar al Padre y al Hijo en la asociación de los Siete Espíritus Maestros, el Espíritu Maestro Número Cuatro es siempre el que habla.

\par
%\textsuperscript{(187.6)}
\textsuperscript{16:3.9} Este Espíritu favorece el cuarto segmento del gran universo de acuerdo con la manera particular en que asocia los atributos del Padre Universal y del Hijo Eterno. Siempre está en conexión personal con los Espíritus Reflectantes de la sede del cuarto superuniverso.

\par
%\textsuperscript{(187.7)}
\textsuperscript{16:3.10} \textit{Espíritu Maestro Número Cinco.} Esta personalidad divina que combina de manera tan exquisita el carácter del Padre Universal y del Espíritu Infinito es el consejero del enorme grupo de seres conocidos como directores del poder, centros del poder y controladores físicos. Este Espíritu patrocina también todas las personalidades que tienen su origen en el Padre y el Actor Conjunto. En los consejos de los Siete Espíritus Maestros, cuando la actitud del Padre-Espíritu está en tela de juicio, el Espíritu Maestro Número Cinco es siempre el que habla.

\par
%\textsuperscript{(187.8)}
\textsuperscript{16:3.11} Este Espíritu dirige el bienestar del quinto superuniverso de tal manera que sugiere la acción combinada del Padre Universal y del Espíritu Infinito. Siempre está en conexión con los Espíritus Reflectantes de la sede del quinto superuniverso.

\par
%\textsuperscript{(187.9)}
\textsuperscript{16:3.12} \textit{Espíritu Maestro Número Seis.} Este ser divino parece mostrar el carácter combinado del Hijo Eterno y del Espíritu Infinito. Cada vez que las criaturas creadas conjuntamente por el Hijo y el Espíritu se reúnen en el universo central, este Espíritu Maestro es su consejero; y cada vez que en los consejos de los Siete Espíritus Maestros es necesario hablar conjuntamente por el Hijo Eterno y el Espíritu Infinito, el Espíritu Maestro Número Seis es el que responde.

\par
%\textsuperscript{(188.1)}
\textsuperscript{16:3.13} Este Espíritu dirige los asuntos del sexto superuniverso casi como lo harían el Hijo Eterno y el Espíritu Infinito. Siempre está en conexión con los Espíritus Reflectantes de la sede del sexto superuniverso.

\par
%\textsuperscript{(188.2)}
\textsuperscript{16:3.14} \textit{Espíritu Maestro Número Siete.} El Espíritu que preside el séptimo superuniverso es un retrato extraordinariamente preciso del Padre Universal, el Hijo Eterno y el Espíritu Infinito. El Séptimo Espíritu, el consejero que favorece a todos los seres de origen trino, es también el consejero y el director de todos los peregrinos ascendentes de Havona, de aquellos seres humildes que han alcanzado las cortes de la gloria a través del ministerio combinado del Padre, el Hijo y el Espíritu.

\par
%\textsuperscript{(188.3)}
\textsuperscript{16:3.15} El Séptimo Espíritu Maestro no representa orgánicamente a la Trinidad del Paraíso; pero es un hecho conocido que su naturaleza personal y espiritual \textit{es} el retrato del Actor Conjunto con proporciones equivalentes de las tres personas infinitas cuya unión en la Deidad \textit{es} la Trinidad del Paraíso, y cuya función como tal \textit{es} la fuente de la naturaleza personal y espiritual de Dios Supremo. De ahí que el Séptimo Espíritu Maestro revele una relación personal y orgánica con la persona espiritual del Supremo en evolución. Por eso en los consejos de los Espíritus Maestros en las alturas, cuando es necesario someter a votación la actitud personal combinada del Padre, el Hijo y el Espíritu, o describir la actitud espiritual del Ser Supremo, el Espíritu Maestro Número Siete es el que actúa. Así se convierte de manera inherente en el jefe que preside el consejo paradisiaco de los Siete Espíritus Maestros.

\par
%\textsuperscript{(188.4)}
\textsuperscript{16:3.16} Ninguno de los Siete Espíritus representa orgánicamente a la Trinidad del Paraíso, pero cuando se unen como Deidad séptuple, esta unión en el sentido de la deidad ---no en el sentido personal--- equivale a un nivel funcional asociable con las funciones de la Trinidad. En este sentido, el <<\textit{Espíritu Séptuple}>> es asociable funcionalmente con la Trinidad del Paraíso. También en este sentido, el Espíritu Maestro Número Siete habla a veces para confirmar las actitudes de la Trinidad o, más bien, actúa como portavoz de la actitud de la unión del Espíritu Séptuple en relación con la actitud de la unión de la Deidad Triple, la actitud de la Trinidad del Paraíso.

\par
%\textsuperscript{(188.5)}
\textsuperscript{16:3.17} Las múltiples funciones del Séptimo Espíritu Maestro se extienden así desde ser un retrato combinado de las \textit{naturalezas personales} del Padre, el Hijo y el Espíritu, ser una representación de la \textit{actitud personal} de Dios Supremo, y ser también una revelación de la \textit{actitud como deidad} de la Trinidad del Paraíso. En ciertos aspectos, este Espíritu presidente expresa de forma similar las \textit{actitudes} del Último y del Supremo-Último.

\par
%\textsuperscript{(188.6)}
\textsuperscript{16:3.18} Con sus múltiples aptitudes, el Espíritu Maestro Número Siete es el que patrocina personalmente el progreso de los candidatos a la ascensión procedentes de los mundos del tiempo en sus intentos por conseguir comprender la Deidad indivisa de la Supremacía. Dicha comprensión implica que los candidatos captan la soberanía existencial de la Trinidad de Supremacía, coordinada de tal manera con un concepto de la soberanía experiencial creciente del Ser Supremo como para constituir la comprensión que adquieren las criaturas de la unidad de la Supremacía. La comprehensión por parte de las criaturas de estos tres factores equivale a la comprehensión havoniana de la realidad de la Trinidad, y dota a los peregrinos del tiempo de la capacidad de penetrar finalmente en la Trinidad, de descubrir a las tres personas infinitas de la Deidad.

\par
%\textsuperscript{(188.7)}
\textsuperscript{16:3.19} La incapacidad que tienen los peregrinos en Havona para encontrar plenamente a Dios Supremo es compensada por el Séptimo Espíritu Maestro, cuya naturaleza trina revela de esta manera tan particular la persona espiritual del Supremo. Durante la presente era del universo en que no se puede contactar con la persona del Supremo, el Espíritu Maestro Número Siete actúa en lugar del Dios de las criaturas ascendentes en el tema de las relaciones personales. Es el único ser espiritual superior que todos los seres ascendentes reconocerán con seguridad y comprenderán en cierto modo cuando alcancen los centros de la gloria.

\par
%\textsuperscript{(189.1)}
\textsuperscript{16:3.20} Este Espíritu Maestro está siempre en contacto con los Espíritus Reflectantes de Uversa, la sede del séptimo superuniverso, nuestro propio segmento de la creación. Su manera de administrar Orvonton revela la maravillosa simetría de la mezcla coordinada entre las naturalezas divinas del Padre, el Hijo y el Espíritu.

\section*{4. Atributos y funciones de los Espíritus Maestros}
\par
%\textsuperscript{(189.2)}
\textsuperscript{16:4.1} Los Siete Espíritus Maestros son la plena representación del Espíritu Infinito para los universos evolutivos. Representan a la Fuente-Centro Tercera en las relaciones de la energía, la mente y el espíritu. Aunque actúan como los jefes que coordinan el control administrativo universal del Actor Conjunto, no olvidéis que tienen su origen en los actos creativos de las Deidades del Paraíso. Es literalmente cierto que estos Siete Espíritus son el poder físico, la mente cósmica y la presencia espiritual personalizados de la Deidad trina, <<\textit{los Siete Espíritus de Dios enviados a todo el universo}>>.

\par
%\textsuperscript{(189.3)}
\textsuperscript{16:4.2} Los Espíritus Maestros son únicos en el sentido de que actúan en todos los niveles de realidad del universo, excepto en el absoluto. Son por lo tanto los supervisores eficaces y perfectos de todas las fases de los asuntos administrativos en todos los niveles de las actividades superuniversales. A la mente mortal le resulta difícil comprender muchas cosas sobre los Espíritus Maestros porque el trabajo de éstos es sumamente especializado y sin embargo lo abarca todo, es excepcionalmente material y al mismo tiempo exquisitamente espiritual. Estos creadores polifacéticos de la mente cósmica son los progenitores de los Directores del Poder Universal, y ellos mismos son los directores supremos de la vasta y extensa creación de criaturas espirituales.

\par
%\textsuperscript{(189.4)}
\textsuperscript{16:4.3} Los Siete Espíritus Maestros son los creadores de los Directores del Poder Universal y de sus asociados, unas entidades que son indispensables para organizar, controlar y regular las energías físicas del gran universo. Y estos mismos Espíritus Maestros ayudan de manera muy material a los Hijos Creadores en la tarea de dar forma y organizar los universos locales.

\par
%\textsuperscript{(189.5)}
\textsuperscript{16:4.4} Somos incapaces de encontrar una conexión personal entre el trabajo de los Espíritus Maestros relacionado con la energía cósmica y las actividades del Absoluto Incalificado relacionadas con la fuerza. Todas las manifestaciones energéticas que se encuentran bajo la jurisdicción de los Espíritus Maestros están dirigidas desde la periferia del Paraíso; no parecen estar asociadas de ninguna manera directa con los fenómenos de la fuerza identificados con la superficie inferior del Paraíso.

\par
%\textsuperscript{(189.6)}
\textsuperscript{16:4.5} Cuando nos encontramos con las actividades funcionales de los diversos Supervisores del Poder Morontial, nos hallamos indiscutiblemente cara a cara con ciertas actividades no reveladas de los Espíritus Maestros. Además de estos predecesores de los controladores físicos y de los ministros espirituales, ¿quién podría haber conseguido combinar y asociar de tal manera las energías materiales y espirituales como para dar nacimiento a una fase hasta entonces inexistente de la realidad universal ---la sustancia morontial y la mente morontial?

\par
%\textsuperscript{(189.7)}
\textsuperscript{16:4.6} Una gran parte de la realidad de los mundos espirituales es de tipo morontial, una fase de la realidad universal totalmente desconocida en Urantia. La meta de la existencia de las personalidades es espiritual, pero las creaciones morontiales se interponen siempre para colmar el abismo entre los reinos materiales de origen mortal y las esferas superuniversales con un estado espiritual progresivo. En este ámbito es donde los Espíritus Maestros efectúan su gran contribución al plan de la ascensión del hombre hacia el Paraíso.

\par
%\textsuperscript{(190.1)}
\textsuperscript{16:4.7} Los Siete Espíritus Maestros tienen representantes personales que ejercen su actividad en todo el gran universo; pero puesto que la gran mayoría de estos seres subordinados no se ocupa directamente del programa ascendente de la progresión de los mortales en el camino de la perfección paradisiaca, poco o nada se ha revelado acerca de ellos. Una gran parte, una grandísima parte de la actividad de los Siete Espíritus Maestros permanece oculta para la comprensión humana, porque no está de ninguna manera directamente relacionada con vuestro problema de ascender hasta el Paraíso.

\par
%\textsuperscript{(190.2)}
\textsuperscript{16:4.8} Aunque no podemos ofrecer una prueba definitiva, es muy probable que el Espíritu Maestro de Orvonton ejerza una influencia indudable sobre las esferas de actividad siguientes:

\par
%\textsuperscript{(190.3)}
\textsuperscript{16:4.9} 1. Los procedimientos que utilizan los Portadores de Vida de los universos locales para iniciar la vida.

\par
%\textsuperscript{(190.4)}
\textsuperscript{16:4.10} 2. Las activaciones que efectúan sobre la vida los espíritus ayudantes de la mente otorgados a los mundos por el Espíritu Creativo de un universo local.

\par
%\textsuperscript{(190.5)}
\textsuperscript{16:4.11} 3. Las fluctuaciones que muestran, en sus manifestaciones energéticas, las unidades de materia organizada que responden a la gravedad lineal.

\par
%\textsuperscript{(190.6)}
\textsuperscript{16:4.12} 4. El comportamiento de la energía emergente cuando se libera plenamente de la atracción del Absoluto Incalificado, volviéndose así sensible a la influencia directa de la gravedad lineal y a las manipulaciones de los Directores del Poder Universal y de sus asociados.

\par
%\textsuperscript{(190.7)}
\textsuperscript{16:4.13} 5. La concesión del espíritu ministerial del Espíritu Creativo de un universo local, conocido en Urantia como el Espíritu Santo.

\par
%\textsuperscript{(190.8)}
\textsuperscript{16:4.14} 6. La concesión posterior del espíritu de los Hijos donadores, llamado en Urantia el Consolador o el Espíritu de la Verdad.

\par
%\textsuperscript{(190.9)}
\textsuperscript{16:4.15} 7. El mecanismo de la reflectividad de los universos locales y del super-universo. Muchas características relacionadas con este fenómeno extraordinario apenas se pueden explicar razonablemente, ni comprender racionalmente, si no se admite la actividad de los Espíritus Maestros en asociación con el Actor Conjunto y el Ser Supremo.

\par
%\textsuperscript{(190.10)}
\textsuperscript{16:4.16} A pesar de nuestra incapacidad para comprender adecuadamente los múltiples trabajos de los Siete Espíritus Maestros, estamos convencidos de que hay dos ámbitos en la inmensa gama de las actividades universales donde no tienen absolutamente nada que ver: la concesión y el ministerio de los Ajustadores del Pensamiento y las funciones inescrutables del Absoluto Incalificado.

\section*{5. La relación con las criaturas}
\par
%\textsuperscript{(190.11)}
\textsuperscript{16:5.1} Cada segmento del gran universo, cada universo y cada mundo individuales, disfruta de los beneficios aportados por el consejo y la sabiduría unidos de los Siete Espíritus Maestros, pero recibe el toque y el matiz personales de uno solo de ellos. La naturaleza personal de cada Espíritu Maestro impregna totalmente su superuniverso y lo condiciona de manera única.

\par
%\textsuperscript{(190.12)}
\textsuperscript{16:5.2} Debido a esta influencia personal de los Siete Espíritus Maestros, cada criatura de cada tipo de ser inteligente, fuera del Paraíso y de Havona, debe llevar la marca característica de individualidad que indica la naturaleza ancestral de uno de estos Siete Espíritus del Paraíso. En lo que se refiere a los siete superuniversos, cada criatura nativa, hombre o ángel, llevará para siempre esta marca de identidad natal.

\par
%\textsuperscript{(191.1)}
\textsuperscript{16:5.3} Los Siete Espíritus Maestros no invaden directamente la mente material de las criaturas individuales de los mundos evolutivos del espacio. Los mortales de Urantia no experimentan la presencia personal de la influencia mental-espíritual del Espíritu Maestro de Orvonton. Si este Espíritu Maestro consigue algún tipo de contacto con la mente mortal individual durante las épocas evolutivas primitivas de un mundo habitado, debe producirse a través del ministerio del Espíritu Creativo del universo local, la consorte y asociada del Hijo de Dios Creador que preside los destinos de cada creación local. Pero en su naturaleza y en su carácter, este mismo Espíritu Madre Creativo es exactamente igual al Espíritu Maestro de Orvonton.

\par
%\textsuperscript{(191.2)}
\textsuperscript{16:5.4} La marca física de un Espíritu Maestro es una parte del origen material del hombre. Toda la carrera morontial se vive bajo la influencia continua de este mismo Espíritu Maestro. No es del todo extraño que la carrera espiritual posterior de ese mortal ascendente no erradique nunca por completo la marca característica de este mismo Espíritu supervisor. El sello de un Espíritu Maestro es fundamental para la existencia misma de todas las etapas de la ascensión humana anteriores a Havona.

\par
%\textsuperscript{(191.3)}
\textsuperscript{16:5.5} Los mortales evolutivos manifiestan en la experiencia de su vida unas tendencias distintivas de la personalidad que son características en cada superuniverso y que expresan directamente la naturaleza del Espíritu Maestro dominante; estas tendencias no se borran nunca por completo, ni siquiera después de que estos ascendentes hayan sido sometidos a la larga formación y a la disciplina unificadora que habrán encontrado en los mil millones de esferas educativas de Havona. Incluso la intensa cultura posterior del Paraíso no es suficiente para extirpar las marcas distintivas de origen superuniversal. A lo largo de toda la eternidad, un mortal ascendente mostrará las características indicativas del Espíritu que preside su superuniverso de nacimiento. Incluso en el Cuerpo de la Finalidad, cuando se desea mostrar o llegar a una relación trinitaria \textit{completa} con la creación evolutiva, siempre se reúne a un grupo de siete finalitarios, uno de cada superuniverso.

\section*{6. La mente cósmica}
\par
%\textsuperscript{(191.4)}
\textsuperscript{16:6.1} Los Espíritus Maestros son la fuente séptuple de la mente cósmica, el potencial intelectual del gran universo. Esta mente cósmica es una manifestación subabsoluta de la mente de la Fuente-Centro Tercera, y está relacionada funcionalmente de cierta manera con la mente del Ser Supremo en evolución.

\par
%\textsuperscript{(191.5)}
\textsuperscript{16:6.2} En un mundo como Urantia, la influencia directa de los Siete Espíritus Maestros no la encontramos en los asuntos de las razas humanas. Vivís bajo la influencia directa del Espíritu Creativo de Nebadon. Sin embargo, estos mismos Espíritus Maestros\footnote{\textit{Espíritus ayudantes de la mente}: Ap 5:6.} dominan las reacciones básicas de todas las mentes de las criaturas, porque son la fuente efectiva de los potenciales intelectuales y espirituales que han sido especializados en los universos locales para funcionar en la vida de los individuos que viven en los mundos evolutivos del tiempo y del espacio\footnote{\textit{Mente de Jesús}: Sal 2:5; 1 Co 2:16.}.

\par
%\textsuperscript{(191.6)}
\textsuperscript{16:6.3} El hecho de la mente cósmica explica la afinidad existente entre los diversos tipos de mentes humanas y superhumanas. No solamente los espíritus afines se sienten atraídos los unos hacia los otros, sino que las mentes afines\footnote{\textit{Mentes afine}: Ro 12:15; 15:5-6; 1 Co 1:10; 13:11; Flp 1:27; 2:2; 4:2; 1 P 3:8; 4:1.} son también muy fraternales y tienden a cooperar las unas con las otras. A veces se observa que las mentes humanas funcionan en unas vías que tienen una similitud asombrosa y una concordancia inexplicable.

\par
%\textsuperscript{(191.7)}
\textsuperscript{16:6.4} En todas las asociaciones de personalidad de la mente cósmica existe una cualidad que se podría denominar <<\textit{sensibilidad a la realidad}>>. Esta dotación cósmica universal de las criaturas volitivas es la que las salva de convertirse en víctimas indefensas de las suposiciones implícitas a priori de la ciencia, la filosofía y la religión. Esta sensibilidad de la mente cósmica a la realidad responde a ciertas fases de la realidad exactamente como la energía-materia responde a la gravedad. Sería incluso más correcto decir que estas realidades supermateriales responden así a la mente del cosmos.

\par
%\textsuperscript{(192.1)}
\textsuperscript{16:6.5} La mente cósmica responde infaliblemente (reconoce la respuesta) en tres niveles de la realidad universal. Estas respuestas son evidentes por sí mismas para las mentes que razonan de manera clara y piensan de forma profunda. Estos niveles de realidad son los siguientes:

\par
%\textsuperscript{(192.2)}
\textsuperscript{16:6.6} 1. \textit{La causalidad} ---el ámbito de la realidad relacionado con los sentidos físicos, el campo científico de la uniformidad lógica, la diferenciación entre lo objetivo y lo no objetivo, las conclusiones reflexivas basadas en la reacción cósmica. Es la forma matemática del discernimiento cósmico.

\par
%\textsuperscript{(192.3)}
\textsuperscript{16:6.7} 2. \textit{El deber} ---el ámbito de la realidad relacionado con la moral en el terreno filosófico, el campo de la razón, el reconocimiento del bien y del mal relativos. Es la forma juiciosa del discernimiento cósmico.

\par
%\textsuperscript{(192.4)}
\textsuperscript{16:6.8} 3. \textit{La adoración} ---el ámbito espiritual de la realidad relacionado con la experiencia religiosa, la comprensión personal de la confraternidad divina, el reconocimiento de los valores espirituales, la seguridad de la supervivencia eterna, la ascensión desde el estado de servidores de Dios hasta la alegría y la libertad de los hijos de Dios. Es la perspicacia más elevada de la mente cósmica, la forma reverencial y adoradora del discernimiento cósmico.

\par
%\textsuperscript{(192.5)}
\textsuperscript{16:6.9} Estas perspicacias científicas, morales y espirituales, estas reacciones cósmicas, son innatas en la mente cósmica, la cual dota a todas las criaturas volitivas. La experiencia de la vida no deja nunca de desarrollar estas tres intuiciones cósmicas; forman parte constituyente de la conciencia del pensa-miento reflexivo. Pero hay que indicar con tristeza que muy pocas personas en Urantia se deleitan en cultivar estas cualidades del pensamiento cósmico valiente e independiente.

\par
%\textsuperscript{(192.6)}
\textsuperscript{16:6.10} En las donaciones de la mente a los universos locales, estas tres perspicacias de la mente cósmica constituyen las suposiciones a priori que hacen posible que el hombre actúe como una personalidad racional y consciente de sí misma en los ámbitos de la ciencia, la filosofía y la religión. Dicho de otra manera, el reconocimiento de la \textit{realidad} de estas tres manifestaciones del Infinito se lleva a cabo mediante una técnica cósmica de autorrevelación. La energía-materia es reconocida por la lógica matemática de los sentidos; la razón-mente conoce intuitivamente su deber moral; la fe-espíritu (la adoración) es la religión de la realidad de la experiencia espiritual. Estos tres factores básicos del pensamiento reflexivo pueden unificarse y coordinarse en el desarrollo de la personalidad, o pueden volverse desproporcionados y prácticamente inconexos en sus funciones respectivas. Pero cuando están unificados, producen un carácter fuerte que consiste en la correlación de una ciencia basada en los hechos, de una filosofía moral y de una experiencia religiosa auténtica. Estas tres intuiciones cósmicas son las que le dan una validez objetiva, una realidad, a la experiencia humana con las cosas, los significados y los valores, y en ellos.

\par
%\textsuperscript{(192.7)}
\textsuperscript{16:6.11} La finalidad de la educación es desarrollar y agudizar estos dones innatos de la mente humana; la de la civilización es expresarlos; la de la experiencia de la vida, realizarlos; la de la religión, ennoblecerlos; y la de la personalidad, unificarlos.

\section*{7. La moral, la virtud y la personalidad}
\par
%\textsuperscript{(192.8)}
\textsuperscript{16:7.1} La inteligencia por sí sola no puede explicar la naturaleza moral. La moralidad, la virtud, es innata en la personalidad humana. La intuición moral, la comprensión del deber, es un componente de la dotación mental humana y está asociada con los otros elementos inalienables de la naturaleza humana: la curiosidad científica y la perspicacia espiritual. La mentalidad del hombre trasciende de lejos la de sus primos animales, pero es su naturaleza moral y religiosa la que le distingue especialmente del mundo animal.

\par
%\textsuperscript{(193.1)}
\textsuperscript{16:7.2} La respuesta selectiva de un animal está limitada a su nivel motor de comportamiento. La supuesta perspicacia de los animales superiores se encuentra a un nivel motor y sólo aparece generalmente después de la experiencia de los ensayos y los errores motores. El hombre es capaz de ejercer su perspicacia científica, moral y espiritual antes de explorar o de experimentar cualquier cosa.

\par
%\textsuperscript{(193.2)}
\textsuperscript{16:7.3} Sólo una personalidad puede saber lo que hace antes de hacerlo; sólo las personalidades poseen la perspicacia con antelación a la experiencia. Una personalidad puede mirar antes de saltar y por lo tanto puede aprender tanto mirando como saltando. Un animal no personal sólo aprende generalmente saltando.

\par
%\textsuperscript{(193.3)}
\textsuperscript{16:7.4} Como resultado de la experiencia, un animal es capaz de examinar las diferentes maneras de alcanzar una meta y de elegir un camino de acceso basado en la experiencia acumulada. Pero una personalidad puede examinar también la meta misma y juzgar su validez, su valor. La inteligencia por sí sola puede discernir los mejores medios de conseguir unos fines indistintos, pero un ser moral posee una perspicacia que le permite distinguir entre los fines así como entre los medios. Y un ser moral que elige la virtud es sin embargo inteligente. Sabe lo que hace, por qué lo hace, dónde va y cómo lo conseguirá.

\par
%\textsuperscript{(193.4)}
\textsuperscript{16:7.5} Cuando el hombre no consigue discernir los objetivos de sus esfuerzos como mortal, está actuando en el nivel de existencia animal. No ha conseguido sacar partido de las ventajas superiores de la agudeza material, el discernimiento moral y la perspicacia espiritual que forman parte integrante de su dotación mental cósmica como ser personal.

\par
%\textsuperscript{(193.5)}
\textsuperscript{16:7.6} La virtud es la rectitud ---la conformidad con el cosmos. Nombrar las virtudes no es definirlas, pero vivirlas es conocerlas. La virtud no es el simple conocimiento ni tampoco la sabiduría, sino más bien la realidad de una experiencia progresiva para alcanzar los niveles ascendentes de consecución cósmica. En la vida diaria del hombre mortal, la virtud se hace realidad eligiendo firmemente el bien en lugar del mal, y esta capacidad para elegir es la prueba de que se posee una naturaleza moral.

\par
%\textsuperscript{(193.6)}
\textsuperscript{16:7.7} La elección del hombre entre el bien y el mal no está influida solamente por la agudeza de su naturaleza moral, sino también por otras influencias tales como la ignorancia, la inmadurez y las ilusiones. Cierto sentido de la proporción también está implicado en el ejercicio de la virtud, porque se puede cometer el mal cuando se elige lo menor en lugar de lo mayor, a consecuencia de la deformación o del engaño. El arte de la valoración relativa o de la medida comparativa entra en la práctica de las virtudes del ámbito moral.

\par
%\textsuperscript{(193.7)}
\textsuperscript{16:7.8} La naturaleza moral del hombre se encontraría impotente sin el arte de la medida, sin el discernimiento que está incorporado en su capacidad para examinar a fondo los significados. La elección moral sería igualmente inútil sin esa perspicacia cósmica que proporciona la conciencia de los valores espirituales. Desde el punto de vista de la inteligencia, el hombre se eleva hasta el nivel de un ser moral porque está dotado de personalidad.

\par
%\textsuperscript{(193.8)}
\textsuperscript{16:7.9} Nunca es posible hacer progresar la moralidad por medio de la ley o de la fuerza. Es un asunto personal y de libre albedrío, y ha de propagarse por contagio mediante el contacto entre las personas con fragancia moral y aquellas que son menos sensibles a la moral, pero que tienen también en cierta medida el deseo de hacer la voluntad del Padre.

\par
%\textsuperscript{(193.9)}
\textsuperscript{16:7.10} Los actos morales son las acciones humanas caracterizadas por la inteligencia más elevada, dirigidas por una diferenciación selectiva tanto en la elección de los fines superiores como en la elección de los medios morales para alcanzar dichos fines. Una conducta así es virtuosa. La virtud suprema consiste pues en elegir de todo corazón hacer la voluntad del Padre que está en los cielos.

\section*{8. La personalidad en Urantia}
\par
%\textsuperscript{(194.1)}
\textsuperscript{16:8.1} El Padre Universal confiere la personalidad a las numerosas órdenes de seres que ejercen su actividad en los diversos niveles de la realidad universal. Los seres humanos de Urantia están dotados de una personalidad de tipo finito-mortal que actúa en el nivel de los hijos ascendentes de Dios.

\par
%\textsuperscript{(194.2)}
\textsuperscript{16:8.2} Aunque apenas podemos aventurarnos a definir la personalidad, podemos intentar indicar la manera en que comprendemos los factores conocidos que van a componer el conjunto de energías materiales, mentales y espirituales cuya interasociación constituye el mecanismo en el cual, sobre el cual y con el cual el Padre Universal hace que ejerza su actividad la personalidad conferida por él.

\par
%\textsuperscript{(194.3)}
\textsuperscript{16:8.3} La personalidad es un don único de naturaleza original cuya existencia es independiente de, y anterior a, la concesión del Ajustador del Pensamiento. Sin embargo, la presencia del Ajustador aumenta de hecho la manifestación cualitativa de la personalidad. Cuando los Ajustadores del Pensamiento surgen del Padre, son idénticos en naturaleza, pero la personalidad es variada, original y exclusiva; y la manifestación de la personalidad está condicionada y limitada además por la naturaleza y las cualidades de las energías asociadas de naturaleza material, mental y espiritual que constituyen el vehículo orgánico que sirve para la manifestación de la personalidad.

\par
%\textsuperscript{(194.4)}
\textsuperscript{16:8.4} Las personalidades pueden ser semejantes, pero nunca son iguales. Las personas que pertenecen a una serie, un tipo, una orden o un modelo determinados pueden parecerse las unas a las otras, y de hecho se parecen, pero nunca son idénticas. La personalidad es esa característica que \textit{conocemos} de un individuo, y que nos permitirá identificar a ese ser en algún momento del futuro sin tener en cuenta la naturaleza y la extensión de los cambios que se habrán producido en su forma, su mente o su estado espiritual. La personalidad es esa parte del individuo que nos permite reconocer e identificar con precisión a esa persona como la que hemos conocido anteriormente, por mucho que haya cambiado debido a la modificación del vehículo que expresa y manifiesta su personalidad.

\par
%\textsuperscript{(194.5)}
\textsuperscript{16:8.5} La personalidad de la criatura se distingue por dos fenómenos característicos que se manifiestan por sí mismos en el comportamiento reactivo humano: la conciencia de sí mismo y el libre albedrío relativo asociado.

\par
%\textsuperscript{(194.6)}
\textsuperscript{16:8.6} La conciencia de sí mismo consiste en darse cuenta intelectualmente de la realidad de la personalidad; incluye la aptitud para reconocer la realidad de otras personalidades. Indica la capacidad para llevar a cabo experiencias individualizadas en y con las realidades cósmicas, lo que equivale a alcanzar el estado de identidad en las relaciones entre personalidades en el universo. La conciencia de sí mismo conlleva el reconocimiento de la realidad del ministerio mental y el darse cuenta de la independencia relativa del libre albedrío creativo y determinante.

\par
%\textsuperscript{(194.7)}
\textsuperscript{16:8.7} El libre albedrío relativo que caracteriza a la conciencia de sí mismo de la personalidad humana está implicado en:

\par
%\textsuperscript{(194.8)}
\textsuperscript{16:8.8} 1. La decisión moral, la sabiduría más elevada.

\par
%\textsuperscript{(194.9)}
\textsuperscript{16:8.9} 2. La elección espiritual, el discernimiento de la verdad.

\par
%\textsuperscript{(194.10)}
\textsuperscript{16:8.10} 3. El amor desinteresado, el servicio a la fraternidad.

\par
%\textsuperscript{(194.11)}
\textsuperscript{16:8.11} 4. La cooperación intencional, la lealtad al grupo.

\par
%\textsuperscript{(194.12)}
\textsuperscript{16:8.12} 5. La perspicacia cósmica, la captación de los significados universales.

\par
%\textsuperscript{(194.13)}
\textsuperscript{16:8.13} 6. La dedicación de la personalidad, la consagración incondicional a hacer la voluntad del Padre.

\par
%\textsuperscript{(195.1)}
\textsuperscript{16:8.14} 7. La adoración, la búsqueda sincera de los valores divinos y el amor de todo corazón al divino Dador de los Valores.

\par
%\textsuperscript{(195.2)}
\textsuperscript{16:8.15} Se puede considerar que el tipo de personalidad humana que existe en Urantia ejerce su actividad en un mecanismo físico que consiste en la modifi-cación planetaria del tipo de organismo nebadónico perteneciente a la orden electroquímica de activación vital, y dotado del modelo de reproducción parental de la orden nebadónica de la serie de la mente cósmica de Orvonton. La concesión del don divino de la personalidad a ese mecanismo mortal dotado de una mente le confiere la dignidad de la ciudadanía cósmica y permite que esa criatura mortal reaccione inmediatamente al reconocimiento constitutivo de las tres realidades mentales fundamentales del cosmos:

\par
%\textsuperscript{(195.3)}
\textsuperscript{16:8.16} 1. El reconocimiento matemático o lógico de la uniformidad de la causalidad física.

\par
%\textsuperscript{(195.4)}
\textsuperscript{16:8.17} 2. El reconocimiento razonado de la obligación de tener una conducta moral.

\par
%\textsuperscript{(195.5)}
\textsuperscript{16:8.18} 3. La comprensión por la fe de la adoración con comunión de la Deidad, asociada al servicio amoroso a la humanidad.

\par
%\textsuperscript{(195.6)}
\textsuperscript{16:8.19} El funcionamiento completo de este don de la personalidad es el comienzo de la comprensión del parentesco con la Deidad. Esta individualidad, habitada por un fragmento prepersonal de Dios Padre, es de hecho y en verdad un hijo espiritual de Dios. Esta criatura no sólo revela la capacidad de recibir el don de la presencia divina, sino que muestra también una respuesta reactiva al circuito de la gravedad de personalidad del Padre Paradisiaco de todas las personalidades.

\section*{9. La realidad de la conciencia humana}
\par
%\textsuperscript{(195.7)}
\textsuperscript{16:9.1} La criatura personal dotada de la mente cósmica y habitada por un Ajustador posee la capacidad innata de reconocer y comprender la realidad de la energía, la realidad de la mente y la realidad del espíritu. La criatura volitiva está equipada así para discernir el hecho de Dios, la ley de Dios y el amor de Dios. Aparte de estos tres elementos inalienables de la conciencia humana, toda experiencia humana es realmente subjetiva, excepto esta comprensión intuitiva de lo que es válido vinculada a la \textit{unificación} de estas tres respuestas del reconocimiento cósmico a la realidad universal.

\par
%\textsuperscript{(195.8)}
\textsuperscript{16:9.2} El mortal que discierne a Dios es capaz de sentir el valor unificador de estas tres cualidades cósmicas en la evolución del alma sobreviviente, la empresa suprema del hombre en el tabernáculo físico donde la mente moral colabora con el espíritu divino interior para dualizar el alma inmortal. Desde sus primeros comienzos, el alma es \textit{real;} posee cualidades de supervivencia cósmica.

\par
%\textsuperscript{(195.9)}
\textsuperscript{16:9.3} Si el hombre mortal no logra sobrevivir a la muerte natural, los valores espirituales reales de su experiencia humana sobreviven como una parte de la experiencia continua del Ajustador del Pensamiento. Los valores de la personalidad de ese no sobreviviente subsisten como un factor en la personalidad del Ser Supremo en vías de manifestarse. Estas cualidades sobrevivientes de la personalidad están desprovistas de identidad, pero no de los valores experienciales acumulados durante la vida mortal en la carne. La supervivencia de la identidad depende de la supervivencia del alma inmortal, cuyo estado es morontial y posee un valor cada vez más divino. La identidad de la personalidad sobrevive en y con la supervivencia del alma.

\par
%\textsuperscript{(195.10)}
\textsuperscript{16:9.4} La conciencia humana de sí mismo implica el reconocimiento de la realidad de otros yoes distintos al yo consciente, e implica además que esta conciencia es mutua; que el yo es conocido del mismo modo que conoce\footnote{\textit{Conocerse y conocer}: 1 Co 13:12.}. Esto queda demostrado de una manera puramente humana en la vida social del hombre. Pero no podéis estar tan absolutamente seguros de la realidad de un compañero humano como podéis estarlo de la realidad de la presencia de Dios que vive dentro de vosotros\footnote{\textit{El espíritu de Dios vive en nosotros}: Job 32:8,18; Is 63:10-11; Ez 37:14; Mt 10:20; Lc 17:21; Jn 17:21-23; Ro 8:9-11; 1 Co 3:16-17; 1 Co 6:19; 2 Co 6:16; Gl 2:20; 1 Jn 3:24; 1 Jn 4:12-15; Ap 21:3.}. La conciencia social no es inalienable como la conciencia de Dios; es un desarrollo cultural y depende del conocimiento, de los símbolos y de las contribuciones de las dotaciones constitutivas del hombre ---la ciencia, la moralidad y la religión. Y estos dones cósmicos, adaptados a la sociedad, constituyen la civilización.

\par
%\textsuperscript{(196.1)}
\textsuperscript{16:9.5} Las civilizaciones son inestables porque no son cósmicas; no son innatas en los individuos de las razas. Deben ser alimentadas por las contribuciones combinadas de los factores constitutivos del hombre ---la ciencia, la moralidad y la religión. Las civilizaciones aparecen y desaparecen, pero la ciencia, la moralidad y la religión siempre sobreviven a la destrucción.

\par
%\textsuperscript{(196.2)}
\textsuperscript{16:9.6} Jesús no sólo reveló Dios al hombre, sino que efectuó también una nueva revelación del hombre a sí mismo y a los otros hombres\footnote{\textit{Jesús reveló Dios al hombre}: Mt 5:45-48; 6:1,4,6; 11:25-27; Mc 11:25-26; Lc 6:35-36; 10:22; Jn 1:18; 3:31-34; 4:21-24; 6:45-46; 10:36-38; 14:6-11,20; 15:15; 16:25; 17:8,25-26.}. En la vida de Jesús veis al hombre en su mejor aspecto. El hombre se vuelve así tan hermosamente real porque Jesús poseía tantas cosas de Dios en su vida, y la comprensión (el reconocimiento) de Dios es inalienable y constitutiva en todos los hombres.

\par
%\textsuperscript{(196.3)}
\textsuperscript{16:9.7} Aparte del instinto parental, el desinterés no es totalmente natural; no se ama por naturaleza a las otras personas ni se les sirve socialmente. Para engendrar un orden social desinteresado y altruista se necesita la iluminación de la razón, la moralidad, y el impulso de la religión, el conocimiento de Dios. La conciencia que tiene el hombre de su propia personalidad, la conciencia de sí mismo, depende también directamente de este mismo hecho de la conciencia innata que tiene el hombre de los otros hombres, de esa capacidad innata para reconocer y captar la realidad de las otras personalidades, desde las humanas hasta las divinas.

\par
%\textsuperscript{(196.4)}
\textsuperscript{16:9.8} La conciencia social desinteresada ha de ser, en el fondo, una conciencia religiosa; es decir, si es objetiva; de otra manera es una abstracción filosófica puramente subjetiva y, en consecuencia, desprovista de amor. Sólo un individuo que conoce a Dios puede amar a otra persona del mismo modo que se ama a sí mismo.

\par
%\textsuperscript{(196.5)}
\textsuperscript{16:9.9} La conciencia de sí mismo es en esencia una conciencia comunitaria: Dios y el hombre, Padre e hijo, Creador y criatura. Cuatro comprensiones de la realidad universal se encuentran latentes en la conciencia humana de sí mismo, y son inherentes a ella:

\par
%\textsuperscript{(196.6)}
\textsuperscript{16:9.10} 1. La búsqueda del conocimiento, la lógica de la ciencia.

\par
%\textsuperscript{(196.7)}
\textsuperscript{16:9.11} 2. La búsqueda de los valores morales, el sentido del deber.

\par
%\textsuperscript{(196.8)}
\textsuperscript{16:9.12} 3. La búsqueda de los valores espirituales, la experiencia religiosa.

\par
%\textsuperscript{(196.9)}
\textsuperscript{16:9.13} 4. La búsqueda de los valores de la personalidad, la capacidad para reconocer la realidad de Dios como personalidad, y la comprensión simultánea de nuestra relación fraternal con las personalidades de nuestros semejantes.

\par
%\textsuperscript{(196.10)}
\textsuperscript{16:9.14} Os hacéis conscientes de que el hombre es una criatura hermana vuestra porque ya sois conscientes de que Dios es vuestro Padre Creador. La paternidad es la relación por la que llegamos al reconocimiento de la fraternidad. Y la Paternidad se vuelve, o puede volverse, una realidad universal para todas las criaturas morales porque el Padre mismo ha conferido la personalidad a todos esos seres y los ha colocado bajo el dominio del circuito universal de la personalidad. Adoramos a Dios en primer lugar porque \textit{él es,} luego porque \textit{estáen nosotros,} y finalmente porque \textit{estamos en él.}

\par
%\textsuperscript{(196.11)}
\textsuperscript{16:9.15} ¿Es extraño pues que la mente cósmica se dé cuenta conscientemente de su propia fuente, la mente infinita del Espíritu Infinito, y que al mismo tiempo sea consciente de la realidad física de los extensos universos, de la realidad espiritual del Hijo Eterno, y de la realidad de la personalidad del Padre Universal?

\par
%\textsuperscript{(196.12)}
\textsuperscript{16:9.16} [Patrocinado por un Censor Universal procedente de Uversa.]


\chapter{Documento 17. Los siete grupos de Espíritus Supremos}
\par
%\textsuperscript{(197.1)}
\textsuperscript{17:0.1} LOS siete grupos de Espíritus Supremos son los directores universales que coordinan la administración de los siete segmentos del gran universo. Aunque todos están clasificados dentro de la familia funcional del Espíritu Infinito, los tres grupos siguientes están clasificados generalmente como hijos de la Trinidad del Paraíso:

\par
%\textsuperscript{(197.2)}
\textsuperscript{17:0.2} 1. Los Siete Espíritus Maestros.

\par
%\textsuperscript{(197.3)}
\textsuperscript{17:0.3} 2. Los Siete Ejecutivos Supremos.

\par
%\textsuperscript{(197.4)}
\textsuperscript{17:0.4} 3. Los Espíritus Reflectantes.

\par
%\textsuperscript{(197.5)}
\textsuperscript{17:0.5} Los cuatro grupos restantes son traídos a la existencia mediante los actos creadores del Espíritu Infinito o por medio de sus asociados con poder creativo:

\par
%\textsuperscript{(197.6)}
\textsuperscript{17:0.6} 4. Los Ayudantes Reflectantes de Imágenes.

\par
%\textsuperscript{(197.7)}
\textsuperscript{17:0.7} 5. Los Siete Espíritus de los Circuitos.

\par
%\textsuperscript{(197.8)}
\textsuperscript{17:0.8} 6. Los Espíritus Creativos de los Universos Locales.

\par
%\textsuperscript{(197.9)}
\textsuperscript{17:0.9} 7. Los Espíritus Ayudantes de la Mente.

\par
%\textsuperscript{(197.10)}
\textsuperscript{17:0.10} Estas siete órdenes se conocen en Uversa como los siete grupos de Espíritus Supremos. Su ámbito funcional se extiende desde la presencia personal de los Siete Espíritus Maestros en la periferia de la Isla eterna, pasando por los siete satélites paradisiacos del Espíritu, los circuitos de Havona, los gobiernos de los superuniversos y la administración y la supervisión de los universos locales, llegando incluso hasta el humilde servicio de los ayudantes otorgados a los ámbitos de la mente evolutiva en los mundos del tiempo y del espacio.

\par
%\textsuperscript{(197.11)}
\textsuperscript{17:0.11} Los Siete Espíritus Maestros son los directores que coordinan este extenso ámbito administrativo. En algunos asuntos relacionados con la reglamentación administrativa del poder físico organizado, de la energía mental y del ministerio espiritual impersonal, actúan de manera personal y directa, mientras que en otras materias ejercen su actividad a través de sus múltiples asociados. En todos los asuntos de naturaleza ejecutiva ---resoluciones, reglamentaciones, ajustes y decisiones administrativas--- los Espíritus Maestros actúan a través de las personas de los Siete Ejecutivos Supremos. En el universo central, los Espíritus Maestros pueden desempeñar sus funciones a través de los Siete Espíritus de los Circuitos de Havona; en las sedes de los siete superuniversos, se revelan a través del canal de los Espíritus Reflectantes y actúan a través de las personas de los Ancianos de los Días, con quienes están en comunicación personal a través de los Ayudantes Reflectantes de Imágenes.

\par
%\textsuperscript{(197.12)}
\textsuperscript{17:0.12} Los Siete Espíritus Maestros no se ponen directa y personalmente en contacto con la administración universal que se encuentra por debajo de las cortes de los Ancianos de los Días. Vuestro universo local es administrado como una parte de nuestro superuniverso por el Espíritu Maestro de Orvonton, pero con relación a los seres nativos de Nebadon, su actividad la desempeña directamente y la dirige personalmente el Espíritu Madre Creativo que reside en Salvington, la sede de vuestro universo local.

\section*{1. Los Siete Ejecutivos Supremos}
\par
%\textsuperscript{(198.1)}
\textsuperscript{17:1.1} Las sedes ejecutivas de los Espíritus Maestros ocupan los siete satélites paradisiacos del Espíritu Infinito, que giran alrededor de la Isla central entre las brillantes esferas del Hijo Eterno y el circuito más interior de Havona. Estas esferas ejecutivas se encuentran bajo la dirección de los Ejecutivos Supremos, un grupo de siete seres que fueron trinitizados por el Padre, el Hijo y el Espíritu de acuerdo con las especificaciones de los Siete Espíritus Maestros a fin de producir un tipo de seres que pudieran actuar como representantes universales suyos.

\par
%\textsuperscript{(198.2)}
\textsuperscript{17:1.2} Los Espíritus Maestros se mantienen en contacto con las diversas divisiones de los gobiernos superuniversales a través de estos Ejecutivos Supremos. Estos últimos son los que determinan en gran medida las tendencias constitutivas fundamentales de los siete superuniversos. Son perfectos de manera uniforme y divina, pero también poseen personalidades diversas. No tienen un jefe permanente; cada vez que se reúnen eligen a uno de ellos para que presida ese consejo conjunto. Viajan periódicamente al Paraíso para sentarse en consejo con los Siete Espíritus Maestros.

\par
%\textsuperscript{(198.3)}
\textsuperscript{17:1.3} Los Siete Ejecutivos Supremos actúan como coordinadores administrativos del gran universo; se les podría denominar el consejo de administración que dirige la creación posterior a Havona. No están relacionados con los asuntos internos del Paraíso, y dirigen sus esferas de actividad limitada en Havona a través de los Siete Espíritus de los Circuitos. Por lo demás, la amplitud de su supervisión tiene pocos límites; se ocupan de dirigir las cosas físicas, intelectuales y espirituales; lo ven todo, lo oyen todo, lo sienten todo e incluso saben todo lo que sucede en los siete superuniversos y en Havona.

\par
%\textsuperscript{(198.4)}
\textsuperscript{17:1.4} Estos Ejecutivos Supremos no dan origen a las normas ni modifican los procedimientos universales; se ocupan de ejecutar los planes de la divinidad promulgados por los Siete Espíritus Maestros. Tampoco interfieren en el gobierno de los Ancianos de los Días en los superuniversos, ni en la soberanía de los Hijos Creadores en los universos locales. Son los ejecutivos que coordinan, cuya función consiste en llevar a cabo las políticas combinadas de todos los gobernantes debidamente nombrados en el gran universo.

\par
%\textsuperscript{(198.5)}
\textsuperscript{17:1.5} Cada uno de los ejecutivos y las instalaciones de su esfera están consagrados a la administración eficaz de un solo superuniverso. El Ejecutivo Supremo Número Uno, que ejerce su actividad en la esfera ejecutiva número uno, está enteramente dedicado a los asuntos del superuniverso número uno, y así sucesivamente hasta el Ejecutivo Supremo Número Siete, que trabaja en el séptimo satélite paradisiaco del Espíritu y dedica sus energías a dirigir el séptimo superuniverso. Esta séptima esfera se llama Orvonton, ya que los satélites paradisiacos del Espíritu tienen los mismos nombres que los superuniversos con los que están relacionados; de hecho, a los superuniversos les pusieron los nombres de dichos satélites.

\par
%\textsuperscript{(198.6)}
\textsuperscript{17:1.6} En la esfera ejecutiva del séptimo superuniverso, el personal encargado de mantener en orden los asuntos de Orvonton asciende a una cantidad que sobrepasa la comprensión humana y abarca prácticamente todas las órdenes de inteligencias celestiales. Todos los servicios superuniversales relacionados con el transporte de las personalidades (excepto los Espíritus Inspirados Trinitarios y los Ajustadores del Pensamiento) pasan por uno de estos siete mundos ejecutivos en sus viajes universales hacia el Paraíso y cuando regresan de él, y aquí se mantienen los registros centrales de todas las personalidades creadas por la Fuente-Centro Tercera que ejercen su actividad en los superuniversos. El sistema de archivos materiales, morontiales y espirituales de uno de estos mundos ejecutivos del Espíritu asombra incluso a un ser de mi orden.

\par
%\textsuperscript{(199.1)}
\textsuperscript{17:1.7} La mayor parte de los subordinados inmediatos de los Ejecutivos Supremos está compuesta por los hijos trinitizados de las personalidades del Paraíso-Havona y por los descendientes trinitizados de los mortales glorificados que se han graduado gracias a la formación secular del programa ascendente del tiempo y del espacio. El jefe del Consejo Supremo del Cuerpo Paradisiaco de la Finalidad es el que designa a estos hijos trinitizados para que sirvan con los Ejecutivos Supremos.

\par
%\textsuperscript{(199.2)}
\textsuperscript{17:1.8} Cada Ejecutivo Supremo tiene dos gabinetes consultivos: Los hijos del Espíritu Infinito que se encuentran en la sede de cada superuniverso eligen a sus representantes en sus propias filas para que sirvan durante un milenio en el gabinete consultivo primario de su Ejecutivo Supremo. Para todos los asuntos que afectan a los mortales ascendentes del tiempo, existe un gabinete secundario que está compuesto por los mortales que han alcanzado el Paraíso y por los hijos trinitizados de los mortales glorificados; este cuerpo es elegido por los seres que ascienden y se perfeccionan y que residen transitoriamente en las sedes de los siete superuniversos. Todos los jefes de los demás asuntos son nombrados por los Ejecutivos Supremos.

\par
%\textsuperscript{(199.3)}
\textsuperscript{17:1.9} En estos satélites paradisiacos del Espíritu tienen lugar de vez en cuando grandes cónclaves. Los hijos trinitizados destinados en estos mundos, junto con los ascendentes que han alcanzado el Paraíso, se congregan con las personalidades espirituales de la Fuente-Centro Tercera en las reuniones relacionadas con las luchas y los triunfos de la carrera ascendente. Los Ejecutivos Supremos presiden siempre estas asambleas fraternales.

\par
%\textsuperscript{(199.4)}
\textsuperscript{17:1.10} Una vez cada milenio del Paraíso, los Siete Ejecutivos Supremos dejan sus puestos de autoridad y van al Paraíso, donde celebran su cónclave milenario de saludos y de buenos deseos universales para las multitudes inteligentes de la creación. Este acontecimiento memorable tiene lugar en la presencia inmediata de Majeston, el jefe de todos los grupos de espíritus reflectantes. Así pueden comunicarse simultáneamente con todos sus asociados en el gran universo a través del funcionamiento excepcional de la reflectividad universal.

\section*{2. Majeston -el jefe de la reflectividad}
\par
%\textsuperscript{(199.5)}
\textsuperscript{17:2.1} Los Espíritus Reflectantes tienen su origen divino en la Trinidad. Estos seres excepcionales y un poco misteriosos ascienden a cincuenta. Estas personalidades extraordinarias fueron creadas en grupos de siete, y cada uno de estos episodios creativos se llevó a cabo mediante la unión de la Trinidad del Paraíso con uno de los Siete Espíritus Maestros.

\par
%\textsuperscript{(199.6)}
\textsuperscript{17:2.2} Esta operación trascendental, que sucedió en los albores del tiempo, describe el esfuerzo inicial de las Personalidades Creadoras Supremas, representadas por los Espíritus Maestros, para actuar como cocreadoras con la Trinidad del Paraíso. Esta unión del poder creativo de los Creadores Supremos con los potenciales creativos de la Trinidad es la fuente misma de la realidad del Ser Supremo. Por eso, cuando el ciclo de la creación reflectante terminó su curso, cuando cada uno de los Siete Espíritus Maestros encontró su perfecta sincronía creativa con la Trinidad del Paraíso, cuando el Espíritu Reflectante número cuarenta y nueve fue personalizado, una nueva reacción trascendental se produjo en el Absoluto de la Deidad. Esta reacción concedió al Ser Supremo unas nuevas prerrogativas para su personalidad y culminó en la personalización de Majeston, el jefe de la reflectividad y el centro paradisiaco de todo el trabajo de los cuarenta y nueve Espíritus Reflectantes y de sus asociados en todo el universo de universos.

\par
%\textsuperscript{(200.1)}
\textsuperscript{17:2.3} Majeston es una verdadera persona, el centro personal e infalible de los fenómenos de la reflectividad en los siete superuniversos del tiempo y del espacio. Mantiene su sede paradisiaca permanente cerca del centro de todas las cosas, en el punto de encuentro de los Siete Espíritus Maestros. Se ocupa únicamente de la coordinación y del mantenimiento del servicio de la reflectividad en la extensa creación; no está implicado de otra manera en la administración de los asuntos del universo.

\par
%\textsuperscript{(200.2)}
\textsuperscript{17:2.4} Majeston no está incluido en nuestro catálogo de personalidades paradisiacas porque es la única personalidad divina existente creada por el Ser Supremo en unión funcional con el Absoluto de la Deidad. Es una persona, pero se ocupa exclusivamente, y en apariencia de forma automática, de esta fase única de la economía universal; actualmente no ejerce su actividad en ninguna calidad personal con relación a otras órdenes (no reflectantes) de personalidades del universo.

\par
%\textsuperscript{(200.3)}
\textsuperscript{17:2.5} La creación de Majeston señaló el primer acto creativo supremo del Ser Supremo. Esta voluntad de actuar era volitiva en el Ser Supremo, pero la prodigiosa reacción del Absoluto de la Deidad no se conocía de antemano. Desde la aparición de Havona en la eternidad, el universo no había presenciado una objetivación tan extraordinaria de esta alineación gigantesca y extensa de poder y de esta coordinación de actividades espirituales funcionales. La respuesta de la Deidad a las voluntades creadoras del Ser Supremo y de sus asociados sobrepasó considerablemente las intenciones deliberadas que tenían y excedió enormemente las previsiones que concebían.

\par
%\textsuperscript{(200.4)}
\textsuperscript{17:2.6} En las épocas futuras, el Supremo y el Último podrían alcanzar nuevos niveles de divinidad y elevarse a nuevos ámbitos de funcionamiento de la personalidad; estamos asombrados ante la posibilidad de lo que esas épocas podrán presenciar en el terreno de la deificación de otros seres todavía más inesperados e impensables, que poseerían unos poderes inimaginables para llevar a cabo una coordinación universal creciente. Pareciera ser que no existe ningún límite al potencial de reacción del Absoluto de la Deidad ante esta unificación de las relaciones entre la Deidad experiencial y la Trinidad existencial del Paraíso.

\section*{3. Los Espíritus Reflectivos}
\par
%\textsuperscript{(200.5)}
\textsuperscript{17:3.1} Los cuarenta y nueve Espíritus Reflectantes tienen su origen en la Trinidad, pero cada uno de los siete episodios creativos que acompañaron su aparición produjo un tipo de ser cuya naturaleza se parece a las características del Espíritu Maestro coancestral. Así pues, reflejan de maneras diversas la naturaleza y el carácter de las siete combinaciones asociativas posibles de las características de divinidad del Padre Universal, el Hijo Eterno y el Espíritu Infinito. Por esta razón es necesario tener a siete de estos Espíritus Reflectantes en la sede de cada superuniverso. Hace falta un representante de cada uno de los siete tipos para conseguir reflejar perfectamente todas las fases de todas las manifestaciones posibles de las tres Deidades del Paraíso, ya que estos fenómenos se pueden producir en cualquier parte de los siete superuniversos. Por consiguiente, un miembro de cada tipo fue destinado a servir en cada uno de los superuniversos. Estos grupos de siete Espíritus Reflectantes desiguales mantienen sus sedes en las capitales de los superuniversos en el centro reflectante de cada reino, el cual no coincide con el punto de polaridad espiritual.

\par
%\textsuperscript{(200.6)}
\textsuperscript{17:3.2} Los Espíritus Reflectantes tienen nombres, pero estas denominaciones no se han revelado a los mundos del espacio. Sus nombres tienen relación con la naturaleza y el carácter de estos seres, y forman parte de uno de los siete misterios universales de las esferas secretas del Paraíso.

\par
%\textsuperscript{(201.1)}
\textsuperscript{17:3.3} El atributo de la reflectividad, ese fenómeno de los niveles mentales del Actor Conjunto, el Ser Supremo y los Espíritus Maestros, es transmisible a todos los seres relacionados con el trabajo de este inmenso sistema de información universal. Y aquí reside un gran misterio: Ni los Espíritus Maestros ni las Deidades del Paraíso, por separado o colectivamente, muestran estos poderes de la reflectividad universal coordinada tal como se manifiestan en estas cuarenta y nueve personalidades de enlace de Majeston, y sin embargo aquellos son los creadores de todos estos seres maravillosamente dotados. A veces, la herencia divina revela en la criatura ciertos atributos que no son discernibles en el Creador.

\par
%\textsuperscript{(201.2)}
\textsuperscript{17:3.4} Todo el personal del servicio de la reflectividad, a excepción de Majeston y de los Espíritus Reflectantes, son criaturas del Espíritu Infinito y de sus asociados y subordinados inmediatos. Los Espíritus Reflectantes de cada superuniverso son los creadores de sus Ayudantes Reflectantes de Imágenes, sus voces personales en las cortes de los Ancianos de los Días.

\par
%\textsuperscript{(201.3)}
\textsuperscript{17:3.5} Los Espíritus Reflectantes no son simplemente agentes que transmiten; son también personalidades que retienen. Sus descendientes, los seconafines, son también personalidades que retienen o registran. Todo aquello que posee un verdadero valor espiritual se registra por duplicado, y una copia es conservada en el equipo personal de algún miembro de una de las numerosas órdenes de personalidades secoráficas que pertenecen al extenso personal de los Espíritus Reflectantes.

\par
%\textsuperscript{(201.4)}
\textsuperscript{17:3.6} Los archivos oficiales de los universos son transmitidos hacia las esferas superiores por los archivistas angélicos y a través de ellos, pero los verdaderos anales espirituales son agrupados por reflectividad y conservados en la mente de las personalidades adecuadas y apropiadas que pertenecen a la familia del Espíritu Infinito. Éstos son los archivos \textit{vivientes,} en contraste con los archivos oficiales y \textit{muertos} del universo, y son perfectamente conservados en la mente viviente de las personalidades registradoras del Espíritu Infinito.

\par
%\textsuperscript{(201.5)}
\textsuperscript{17:3.7} La organización de la reflectividad es también el mecanismo que recoge las noticias y difunde los decretos en toda la creación. Está operando continuamente, en contraste con el funcionamiento periódico de los diversos servicios de transmisión.

\par
%\textsuperscript{(201.6)}
\textsuperscript{17:3.8} Todo acontecimiento importante que sucede en la sede de un universo local es reflejado de forma inherente hacia la capital de su superuniverso. Y a la inversa, todo aquello que tiene un significado para los universos locales es reflejado desde la sede del superuniverso hacia las capitales de los universos locales. El servicio de la reflectividad que va desde los universos del tiempo hacia los superuniversos parece que es automático o que funciona por sí solo, pero no es así. Todo este servicio es muy personal e inteligente; su precisión es el resultado de una perfecta cooperación entre personalidades y, por consiguiente, difícilmente se puede atribuir a las acciones o a la presencia impersonales de los Absolutos.

\par
%\textsuperscript{(201.7)}
\textsuperscript{17:3.9} Aunque los Ajustadores del Pensamiento no participan en el funcionamiento del sistema universal de la reflectividad, tenemos todas las razones para creer que todos los fragmentos del Padre conocen plenamente estas operaciones y son capaces de utilizar su contenido.

\par
%\textsuperscript{(201.8)}
\textsuperscript{17:3.10} Durante la presente era del universo, el alcance espacial del servicio de la reflectividad exterior al Paraíso parece estar limitado por la periferia de los siete superuniversos. Por lo demás, el funcionamiento de este servicio parece ser independiente del tiempo y del espacio. Parece ser independiente de todos los circuitos universales subabsolutos conocidos.

\par
%\textsuperscript{(201.9)}
\textsuperscript{17:3.11} En la sede de cada superuniverso, la organización reflectante actúa como una unidad separada; pero en ciertas ocasiones especiales, y bajo la dirección de Majeston, las siete organizaciones pueden actuar al unísono universal, y lo hacen de hecho, como en los casos de un jubileo debido al establecimiento de todo un universo local en la luz y la vida, y en las épocas de los saludos milenarios de los Siete Ejecutivos Supremos.

\section*{4. Los Ayudantes Reflectivos de Imágenes}
\par
%\textsuperscript{(202.1)}
\textsuperscript{17:4.1} Los cuarenta y nueve Ayudantes Reflectantes de Imágenes fueron creados por los Espíritus Reflectantes, y hay exactamente siete Ayudantes en la sede de cada superuniverso. El primer acto creativo de los siete Espíritus Reflectantes de Uversa fue dar nacimiento a sus siete Ayudantes de Imágenes, creando cada Espíritu Reflectante su propio Ayudante. En ciertos atributos y características, los Ayudantes de Imágenes son unas reproducciones perfectas de sus Espíritus Madres Reflectantes; son verdaderos duplicados, menos el atributo de la reflectividad. Son verdaderas imágenes y funcionan constantemente como canales de comunicación entre los Espíritus Reflectantes y las autoridades superuniversales. Los Ayudantes de Imágenes no son simples asistentes; son auténticas representaciones de sus Espíritus ancestrales respectivos; son \textit{imágenes,} y son fieles a su nombre.

\par
%\textsuperscript{(202.2)}
\textsuperscript{17:4.2} Los Espíritus Reflectantes mismos son verdaderas personalidades, pero de tal índole que son incomprensibles para los seres materiales. Incluso en la esfera sede de un superuniverso necesitan la asistencia de sus Ayudantes de Imágenes para todas sus relaciones personales con los Ancianos de los Días y sus asociados. En los contactos entre los Ayudantes de Imágenes y los Ancianos de los Días, a veces un solo Ayudante funciona de manera aceptable, mientras que en otras ocasiones se necesitan dos, tres, cuatro o incluso los siete para presentar de forma plena y adecuada la comunicación que se les ha confiado transmitir. Del mismo modo, los mensajes de los Ayudantes de Imágenes son recibidos de manera variada por uno, dos o los tres Ancianos de los Días, según lo requiera el contenido de la comunicación.

\par
%\textsuperscript{(202.3)}
\textsuperscript{17:4.3} Los Ayudantes de Imágenes sirven permanentemente al lado de sus Espíritus ancestrales, y tienen a su disposición una multitud increíble de seconafines asistentes. Los Ayudantes de Imágenes no funcionan directamente en conexión con los mundos educativos de los mortales ascendentes. Están estrechamente asociados con el servicio de información del programa universal para la progresión de los mortales, pero no os pondréis personalmente en contacto con ellos cuando residáis en las escuelas de Uversa porque estos seres aparentemente personales están desprovistos de voluntad; no ejercen el poder de elección. Son verdaderas imágenes, que reflejan enteramente la personalidad y la mente de su Espíritu ancestral particular. Los mortales ascendentes, como clase, no se ponen en contacto íntimo con la reflectividad. Entre vosotros y el funcionamiento efectivo del servicio siempre se interpondrá algún ser de naturaleza reflectante.

\section*{5. Los Siete Espíritus de los Circuitos}
\par
%\textsuperscript{(202.4)}
\textsuperscript{17:5.1} Los Siete Espíritus de los Circuitos de Havona son la representación impersonal conjunta del Espíritu Infinito y de los Siete Espíritus Maestros para los siete circuitos del universo central. Son los servidores de los Espíritus Maestros, de los cuales son sus descendientes colectivos. Los Espíritus Maestros aportan a los siete superuniversos una individualidad administrativa diversificada y bien determinada. A través de estos Espíritus uniformes de los Circuitos de Havona, pueden proporcionar al universo central una supervisión espiritual unificada, uniforme y coordinada.

\par
%\textsuperscript{(202.5)}
\textsuperscript{17:5.2} Cada uno de los Siete Espíritus de los Circuitos está limitado a impregnar un solo circuito de Havona. No están directamente relacionados con los regímenes de los Eternos de los Días, que son los gobernantes de los mundos individuales de Havona. Pero están en conexión con los Siete Ejecutivos Supremos y se sincronizan con la presencia del Ser Supremo en el universo central. Su trabajo está limitado exclusivamente a Havona.

\par
%\textsuperscript{(203.1)}
\textsuperscript{17:5.3} Estos Espíritus de los Circuitos se ponen en contacto con aquellos que residen en Havona a través de sus descendientes personales, los supernafines terciarios. Aunque los Espíritus de los Circuitos coexisten con los Siete Espíritus Maestros, su acto de crear a los supernafines terciarios no alcanzó una importancia enorme hasta la llegada de los primeros peregrinos del tiempo al circuito exterior de Havona en la época de Grandfanda.

\par
%\textsuperscript{(203.2)}
\textsuperscript{17:5.4} A medida que avancéis de circuito en circuito en Havona, conoceréis a los Espíritus de los Circuitos pero no seréis capaces de comulgar personalmente con ellos, aunque podréis reconocer la presencia impersonal de su influencia espiritual, y disfrutar personalmente de ella.

\par
%\textsuperscript{(203.3)}
\textsuperscript{17:5.5} Los Espíritus de los Circuitos se relacionan con los habitantes nativos de Havona de una manera muy semejante a como lo hacen los Ajustadores del Pensamiento con las criaturas mortales que viven en los mundos de los universos evolutivos. Al igual que los Ajustadores del Pensamiento, los Espíritus de los Circuitos son impersonales y se asocian con la mente perfecta de los seres de Havona de una manera muy similar a como los espíritus impersonales del Padre universal residen en la mente finita de los hombres mortales. Pero los Espíritus de los Circuitos no se vuelven nunca una parte permanente de las personalidades de Havona.

\section*{6. Los Espíritus Creativos de los universos locales}
\par
%\textsuperscript{(203.4)}
\textsuperscript{17:6.1} Una gran parte de lo relacionado con la naturaleza y la función de los Espíritus Creativos de los universos locales pertenece en verdad a la historia de su asociación con los Hijos Creadores para organizar y dirigir las creaciones locales; pero las experiencias de estos seres maravillosos, antes de llegar a sus universos locales, poseen muchas características que se pueden narrar como parte de este análisis de los siete grupos de Espíritus Supremos.

\par
%\textsuperscript{(203.5)}
\textsuperscript{17:6.2} Estamos familiarizados con seis fases de la carrera del Espíritu Madre de un universo local, y especulamos mucho sobre la probabilidad de una séptima etapa de actividad. Estas diferentes fases de su existencia son:

\par
%\textsuperscript{(203.6)}
\textsuperscript{17:6.3} 1. \textit{La Diferenciación Inicial en el Paraíso.} Cuando un Hijo Creador es personalizado gracias a la acción conjunta del Padre Universal y del Hijo Eterno, en la persona del Espíritu Infinito se produce simultáneamente lo que se conoce como la <<\textit{reacción complementaria suprema}>>. No entendemos la naturaleza de esta reacción, pero comprendemos que indica una modificación inherente de las posibilidades personalizables que están incluidas dentro del potencial creativo del Creador Conjunto. El nacimiento de un Hijo Creador coordinado señala el nacimiento, dentro de la persona del Espíritu Infinito, del potencial de la futura consorte de ese Hijo Paradisiaco en el universo local. No tenemos conocimiento de esta nueva identificación prepersonal de una entidad, pero sabemos que este hecho queda registrado en los archivos paradisiacos relacionados con la carrera de ese Hijo Creador.

\par
%\textsuperscript{(203.7)}
\textsuperscript{17:6.4} 2. \textit{La Formación Preliminar como Creador.} Durante el largo período de formación preliminar de un Hijo Miguel en la organización y la administración de los universos, su futura consorte experimenta un desarrollo adicional de su entidad y adquiere una conciencia colectiva de su destino. No lo sabemos, pero sospechamos que esta entidad con conciencia colectiva se vuelve consciente del espacio y empieza su formación preliminar necesaria a fin de adquirir habilidad espiritual para su futura tarea de colaborar con el Miguel complementario en la creación y la administración de un universo.

\par
%\textsuperscript{(204.1)}
\textsuperscript{17:6.5} 3. \textit{La Etapa de la Creación Física.} En la época en que el Hijo Eterno le encarga a un Hijo Miguel la tarea de crear, el Espíritu Maestro que dirige el superuniverso al que está destinado ese nuevo Hijo Creador expresa la <<\textit{petición de identificación}>> en presencia del Espíritu Infinito; y, por primera vez, la entidad del futuro Espíritu Creativo aparece como diferenciada de la persona del Espíritu Infinito. Esta entidad se dirige directamente hacia la persona del Espíritu Maestro peticionario, y desaparece de inmediato para nuestro reconocimiento, volviéndose aparentemente una parte de la persona de ese Espíritu Maestro. El Espíritu Creativo recién identificado permanece con ese Espíritu Maestro hasta el momento de la partida del Hijo Creador hacia la aventura del espacio; después de lo cual, el Espíritu Maestro confía el nuevo Espíritu consorte al cuidado del Hijo Creador, indicándole al mismo tiempo al Espíritu consorte el mandato de tener una fidelidad eterna y una lealtad sin fin. Y luego se produce uno de los episodios más profundamente conmovedores que tienen lugar en el Paraíso. El Padre Universal habla para reconocer la unión eterna del Hijo Creador y del Espíritu Creativo, y para confirmar la concesión de ciertos poderes administrativos conjuntos por parte del Espíritu Maestro que ejerce la jurisdicción sobre ese superuniverso.

\par
%\textsuperscript{(204.2)}
\textsuperscript{17:6.6} El Hijo Creador y el Espíritu Creativo, unidos por el Padre, parten luego hacia su aventura de crear un universo. Y trabajan juntos bajo esta forma de asociación durante todo el largo y arduo período de la organización material de su universo.

\par
%\textsuperscript{(204.3)}
\textsuperscript{17:6.7} 4. \textit{La Era de la Creación de la Vida.} Cuando el Hijo Creador declara su intención de crear la vida, empiezan en el Paraíso las <<\textit{ceremonias de la personalización}>>, en las que participan los Siete Espíritus Maestros y que son experimentadas personalmente por el Espíritu Maestro supervisor. Se trata de una contribución de la Deidad del Paraíso a la individualidad del Espíritu consorte del Hijo Creador, y se manifiesta al universo mediante el fenómeno de la <<\textit{erupción primaria}>> que tiene lugar en la persona del Espíritu Infinito. Simultáneamente a este fenómeno que se produce en el Paraíso, el Espíritu consorte del Hijo Creador, hasta ahora impersonal, se convierte a todos los efectos prácticos en una persona auténtica. De ahora en adelante y para siempre jamás, este mismo Espíritu Madre del universo local será considerado como una persona, y mantendrá relaciones personales con toda la multitud de personalidades de la creación viviente que vendrá a continuación.

\par
%\textsuperscript{(204.4)}
\textsuperscript{17:6.8} 5. \textit{Las Épocas Posteriores a la Donación.} Otro cambio importante se produce en la carrera sin fin de un Espíritu Creativo cuando el Hijo Creador regresa a la sede de su universo después de terminar su séptima donación y de haber conseguido la plena soberanía universal. En esta ocasión, ante los administradores reunidos del universo, el Hijo Creador triunfante eleva al Espíritu Madre del Universo a la cosoberanía y reconoce al Espíritu consorte como su igual.

\par
%\textsuperscript{(204.5)}
\textsuperscript{17:6.9} 6. \textit{Las Épocas de Luz y de Vida.} Después de establecerse la era de luz y de vida, la cosoberana de un universo local empieza la sexta fase de la carrera de los Espíritus Creativos. Pero no podemos describir la naturaleza de esta gran experiencia. Estas cosas pertenecen a una etapa futura de la evolución de Nebadon.

\par
%\textsuperscript{(204.6)}
\textsuperscript{17:6.10} 7. \textit{La Carrera No Revelada.} Conocemos estas seis fases de la carrera del Espíritu Madre de un universo local. Es inevitable que nos preguntemos: ¿Existe una séptima carrera? No olvidamos que cuando los finalitarios alcanzan lo que parece ser el destino final de su ascensión como mortales, hay constancia de que empiezan la carrera de los espíritus de la sexta fase. Suponemos que a los finalitarios les espera otra carrera aún no revelada en el trabajo universal. Es natural que supongamos que los Espíritus Madres Universales tengan también delante de ellas una carrera no revelada que representará la séptima fase de su experiencia personal en el servicio universal y la cooperación leal con la orden de los Migueles Creadores.

\section*{7. Los espíritus ayudantes de la mente}
\par
%\textsuperscript{(205.1)}
\textsuperscript{17:7.1} Estos espíritus ayudantes son el séptuple don mental del Espíritu Madre de un universo local a las criaturas vivientes de la creación conjunta de un Hijo Creador y de ese Espíritu Creativo. Este otorgamiento llega a ser posible en la época en que el Espíritu es elevado al estado en que posee las prerrogativas de la personalidad. El relato de la naturaleza y del funcionamiento de los siete espíritus ayudantes de la mente pertenece más propiamente a la historia de vuestro universo local de Nebadon.

\section*{8. Las funciones de los Espíritus Supremos}
\par
%\textsuperscript{(205.2)}
\textsuperscript{17:8.1} Los siete grupos de Espíritus Supremos constituyen el núcleo de la familia funcional de la Fuente-Centro Tercera actuando a la vez como Espíritu Infinito y como Actor Conjunto. El ámbito de los Espíritus Supremos se extiende desde la presencia de la Trinidad en el Paraíso hasta el funcionamiento de la mente de tipo mortal-evolutivo en los planetas del espacio. Estos Espíritus unifican así los niveles administrativos descendentes y coordinan las múltiples funciones del personal de los mismos. Ya se trate de un grupo de Espíritus Reflectantes en conexión con los Ancianos de los Días, de un Espíritu Creativo que actúa de común acuerdo con un Hijo Miguel, o de los Siete Espíritus Maestros situados en circuito alrededor de la Trinidad del Paraíso, la actividad de los Espíritus Supremos se encuentra en todas las partes del universo central, de los superuniversos y de los universos locales. Trabajan del mismo modo con las personalidades Trinitarias de la orden de los <<\textit{Días}>> y con las personalidades Paradisiacas de la orden de los <<\textit{Hijos}>>.

\par
%\textsuperscript{(205.3)}
\textsuperscript{17:8.2} Junto con su Espíritu Madre Infinito, los grupos de Espíritus Supremos son los creadores directos de la inmensa familia de criaturas de la Fuente-Centro Tercera. Todas las órdenes de espíritus ministrantes nacen de esta asociación. Los supernafines primarios tienen su origen en el Espíritu Infinito; los seres secundarios de esta orden son creados por los Espíritus Maestros, y los supernafines terciarios por los Siete Espíritus de los Circuitos. Los Espíritus Reflectantes, colectivamente, son los autores-madres de una orden maravillosa de huestes angélicas, los poderosos seconafines de los servicios superuniversales. Un Espíritu Creativo es la madre de las órdenes angélicas de una creación local; estos ministros seráficos son originales en cada universo local, aunque son creados según los arquetipos del universo central. Todos estos creadores de espíritus ministrantes sólo reciben una asistencia indirecta por parte del alojamiento central del Espíritu Infinito, la madre original y eterna de todos los ministros angélicos.

\par
%\textsuperscript{(205.4)}
\textsuperscript{17:8.3} Los siete grupos de Espíritus Supremos son los coordinadores de la creación habitada. La asociación de sus jefes dirigentes, los Siete Espíritus Maestros, parece coordinar las extensas actividades de Dios Séptuple:

\par
%\textsuperscript{(205.5)}
\textsuperscript{17:8.4} 1. Colectivamente, los Espíritus Maestros casi equivalen al nivel de divinidad de la Trinidad de las Deidades del Paraíso.

\par
%\textsuperscript{(205.6)}
\textsuperscript{17:8.5} 2. Individualmente, agotan las posibilidades asociables primarias de la Deidad trina.

\par
%\textsuperscript{(206.1)}
\textsuperscript{17:8.6} 3. Como representantes diversificados del Actor Conjunto, son los depositarios de la soberanía espiritual, mental y de poder del Ser Supremo que éste no ejerce personalmente todavía.

\par
%\textsuperscript{(206.2)}
\textsuperscript{17:8.7} 4. A través de los Espíritus Reflectantes, sincronizan los gobiernos superuniversales de los Ancianos de los Días con Majeston, el centro paradisiaco de la reflectividad universal.

\par
%\textsuperscript{(206.3)}
\textsuperscript{17:8.8} 5. Mediante su participación en la individualización de las Ministras Divinas de los universos locales, los Espíritus Maestros aportan su contribución al último nivel de Dios Séptuple, la unión de los Hijos Creadores y de los Espíritus Creativos de los universos locales.

\par
%\textsuperscript{(206.4)}
\textsuperscript{17:8.9} La unidad funcional, inherente al Actor Conjunto, se revela a los universos evolutivos en los Siete Espíritus Maestros, sus personalidades primarias. Pero en los superuniversos perfeccionados del futuro, esta unidad será sin duda inseparable de la soberanía experiencial del Supremo.

\par
%\textsuperscript{(206.5)}
\textsuperscript{17:8.10} [Presentado por un Consejero Divino de Uversa.]


\chapter{Documento 18. Las Personalidades Trinitarias Supremas}
\par
%\textsuperscript{(207.1)}
\textsuperscript{18:0.1} TODAS las Personalidades Trinitarias Supremas son creadas para un servicio específico. Han sido concebidas por la Trinidad divina para desempeñar ciertos deberes específicos, y están cualificadas para servir con una técnica perfecta y una dedicación final. Existen siete órdenes de Personalidades Trinitarias Supremas:

\par
%\textsuperscript{(207.2)}
\textsuperscript{18:0.2} 1. Los Secretos Trinitizados de la Supremacía.

\par
%\textsuperscript{(207.3)}
\textsuperscript{18:0.3} 2. Los Eternos de los Días.

\par
%\textsuperscript{(207.4)}
\textsuperscript{18:0.4} 3. Los Ancianos de los Días.

\par
%\textsuperscript{(207.5)}
\textsuperscript{18:0.5} 4. Los Perfecciones de los Días.

\par
%\textsuperscript{(207.6)}
\textsuperscript{18:0.6} 5. Los Recientes de los Días.

\par
%\textsuperscript{(207.7)}
\textsuperscript{18:0.7} 6. Los Uniones de los Días.

\par
%\textsuperscript{(207.8)}
\textsuperscript{18:0.8} 7. Los Fieles de los Días.

\par
%\textsuperscript{(207.9)}
\textsuperscript{18:0.9} El número de estos seres dotados de perfección administrativa es preciso y definitivo. Su creación es un acontecimiento que pertenece al pasado; ya no se personaliza ninguno más.

\par
%\textsuperscript{(207.10)}
\textsuperscript{18:0.10} En todo el gran universo, estas Personalidades Trinitarias Supremas representan la política administrativa de la Trinidad del Paraíso; representan la justicia y \textit{son} el juicio ejecutivo de la Trinidad del Paraíso. Forman una línea interrelacionada de perfección administrativa que se extiende desde las esferas paradisiacas del Padre hasta los mundos sede de los universos locales, y hasta las capitales de las constelaciones que los componen.

\par
%\textsuperscript{(207.11)}
\textsuperscript{18:0.11} Todos los seres de origen trinitario son creados con la perfección del Paraíso en todos sus atributos divinos. Únicamente en el terreno de la experiencia es donde el paso del tiempo ha aumentado sus aptitudes para el servicio cósmico. Con los seres de origen trinitario nunca existe ningún peligro de negligencia ni riesgo de rebelión. Son de esencia divina, y nunca se ha sabido que se hayan apartado del sendero divino y perfecto de la conducta de la personalidad.

\section*{1. Los Secretos Trinitizados de la Supremacía}
\par
%\textsuperscript{(207.12)}
\textsuperscript{18:1.1} Hay siete mundos en el circuito más interior de los satélites del Paraíso, y cada uno de estos mundos exaltados está presidido por un cuerpo de diez Secretos Trinitizados de la Supremacía. No son creadores sino administradores supremos y últimos. La dirección de los asuntos de estas siete esferas fraternales está totalmente encomendada a este cuerpo de setenta directores supremos. Aunque los descendientes de la Trinidad supervisan estas siete esferas sagradas, las más próximas al Paraíso, a este grupo de mundos se le conoce universalmente como el circuito personal del Padre Universal.

\par
%\textsuperscript{(208.1)}
\textsuperscript{18:1.2} Los Secretos Trinitizados de la Supremacía ejercen su actividad en grupos de diez como directores coordinados y conjuntos de sus esferas respectivas, pero también actúan individualmente en campos de responsabilidad particulares. El trabajo de cada uno de estos mundos especiales está dividido en siete departamentos principales, y uno de estos gobernantes coordinados preside cada una de estas divisiones de actividades especializadas. Los tres restantes actúan como representantes personales de la Deidad trina en relación con los otros siete, uno representando al Padre, el otro al Hijo y el tercero al Espíritu.

\par
%\textsuperscript{(208.2)}
\textsuperscript{18:1.3} Aunque existe una clara semejanza de clase que tipifica a los Secretos Trinitizados de la Supremacía, también revelan siete características colectivas distintas. Los diez directores supremos de los asuntos de Divinington reflejan el carácter y la naturaleza personales del Padre Universal; y lo mismo sucede con cada una de estas siete esferas: cada grupo de diez se parece a esa Deidad o asociación de Deidades que caracteriza a su dominio. Los diez directores que gobiernan Ascendington reflejan la naturaleza combinada del Padre, el Hijo y el Espíritu.

\par
%\textsuperscript{(208.3)}
\textsuperscript{18:1.4} Muy poca cosa puedo revelar sobre el trabajo de estas altas personalidades en los siete mundos sagrados del Padre, porque son en verdad los \textit{Secretos} de la Supremacía. No existen secretos arbitrarios relacionados con el acercamiento al Padre Universal, al Hijo Eterno o al Espíritu Infinito. Las Deidades son un libro abierto para todos los que alcanzan la perfección divina, pero nunca se pueden alcanzar plenamente todos los Secretos de la Supremacía. Siempre seremos incapaces de penetrar por completo en los dominios que contienen los secretos, relacionados con la personalidad, de la asociación de la Deidad con la séptuple agrupación de los seres creados.

\par
%\textsuperscript{(208.4)}
\textsuperscript{18:1.5} Puesto que el trabajo de estos directores supremos tiene que ver con el contacto íntimo y personal de las Deidades con estas siete agrupaciones fundamentales de seres universales cuando tienen su domicilio en estos siete mundos especiales o mientras ejercen su actividad en todo el gran universo, es justo que estas relaciones tan personales y estos contactos extraordinarios se mantengan en un secreto sagrado. Los Creadores Paradisiacos respetan la intimidad y la santidad de la personalidad incluso en sus criaturas humildes. Y esto es tan cierto en lo que se refiere a los individuos como en lo que respecta a las diversas órdenes particulares de personalidades.

\par
%\textsuperscript{(208.5)}
\textsuperscript{18:1.6} Estos mundos secretos siguen siendo siempre una prueba de lealtad incluso para los seres que han alcanzado un alto nivel universal. Nos es dado conocer plena y personalmente a los Dioses eternos, conocer abundantemente sus caracteres de divinidad y de perfección, pero no se nos concede penetrar por completo en todas las relaciones personales de los Gobernantes del Paraíso con todos sus seres creados.

\section*{2. Los Eternos de los Días}
\par
%\textsuperscript{(208.6)}
\textsuperscript{18:2.1} Cada uno de los mil millones de mundos de Havona está dirigido por una Personalidad Trinitaria Suprema. A estos gobernantes se les conoce como los Eternos de los Días y su número se eleva exactamente a mil millones, uno por cada una de las esferas de Havona. Descienden de la Trinidad del Paraíso, pero al igual que sucede con los Secretos de la Supremacía, no existen archivos sobre su origen. Estos dos grupos de padres omnisapientes han gobernado desde siempre sus mundos exquisitos del sistema Paraíso-Havona, y ejercen su actividad sin rotación ni ser nombrados de nuevo.

\par
%\textsuperscript{(208.7)}
\textsuperscript{18:2.2} Los Eternos de los Días son visibles para todas las criaturas volitivas que residen en sus dominios. Presiden los cónclaves planetarios regulares. Periódicamente, y por rotación, visitan las esferas sede de los siete superuniversos. Son los parientes cercanos y los divinos iguales de los Ancianos de los Días que presiden los destinos de los siete supergobiernos. Cuando un Eterno de los Días está ausente de su esfera, su mundo es dirigido por un Hijo Instructor Trinitario.

\par
%\textsuperscript{(209.1)}
\textsuperscript{18:2.3} Excepto en lo que se refiere a las órdenes de vida establecidas, tales como los nativos de Havona y otras criaturas vivientes del universo central, los Eternos de los Días residentes han desarrollado sus esferas respectivas totalmente de acuerdo con sus propias ideas e ideales personales. Visitan mutuamente sus planetas, pero no copian ni imitan; siempre son enteramente originales.

\par
%\textsuperscript{(209.2)}
\textsuperscript{18:2.4} La arquitectura, el embellecimiento natural, las estructuras morontiales y las creaciones espirituales son exclusivas y únicas en cada esfera. Cada mundo es un lugar de belleza perpetua y es totalmente diferente a cualquier otro mundo en el universo central. Cada uno de vosotros pasará un tiempo más corto o más largo en cada una de estas esferas únicas y emocionantes durante vuestro camino hacia el interior, a través de Havona, hasta el Paraíso. En vuestro mundo es natural hablar del Paraíso como situado \textit{hacia arriba,} pero sería más correcto referirse a la meta divina de la ascensión como situada \textit{hacia el interior.}

\section*{3. Los Ancianos de los Días}
\par
%\textsuperscript{(209.3)}
\textsuperscript{18:3.1} Cuando los mortales del tiempo se gradúan en los mundos de formación que rodean a la sede de un universo local y son ascendidos a las esferas educativas de su superuniverso, su desarrollo espiritual ha progresado hasta el punto en que son capaces de reconocer y de comunicarse con los altos gobernantes y directores espirituales de estos reinos elevados, incluyendo a los Ancianos de los Días\footnote{\textit{Ancianos de los Días}: Dn 7:9,13,22.}.

\par
%\textsuperscript{(209.4)}
\textsuperscript{18:3.2} Todos los Ancianos de los Días son básicamente idénticos; revelan el carácter combinado y la naturaleza unificada de la Trinidad. Poseen una individualidad y sus personalidades son diversas, pero no se diferencian los unos de los otros como los Siete Espíritus Maestros. Aseguran la dirección uniforme de los siete superuniversos que por otra parte son diferentes, pues cada uno de ellos es una creación distinta, separada y única. La naturaleza y los atributos de los Siete Espíritus Maestros son diferentes, pero todos los Ancianos de los Días, los gobernantes personales de los superuniversos, son los descendientes uniformes y superperfectos de la Trinidad del Paraíso.

\par
%\textsuperscript{(209.5)}
\textsuperscript{18:3.3} Los Siete Espíritus Maestros que están en las alturas determinan la \textit{naturaleza} de sus respectivos superuniversos, pero los Ancianos de los Días dictan la \textit{administración} de estos mismos superuniversos. Sobreponen la uniformidad administrativa a la diversidad creativa y aseguran la armonía del conjunto en medio de las diferencias de las creaciones subyacentes de las siete agrupaciones segmentarias del gran universo.

\par
%\textsuperscript{(209.6)}
\textsuperscript{18:3.4} Todos los Ancianos de los Días fueron trinitizados al mismo tiempo. Representan el principio de los archivos sobre la personalidad en el universo de universos, de ahí su nombre ---los \textit{Ancianos} de los Días. Cuando lleguéis al Paraíso y examinéis los anales escritos sobre el comienzo de las cosas, encontraréis que la primera inscripción que aparece en la sección sobre la personalidad es el relato de la trinitización de estos veintiún Ancianos de los Días.

\par
%\textsuperscript{(209.7)}
\textsuperscript{18:3.5} Estos seres elevados siempre gobiernan en grupos de tres. Existen muchas fases de actividad en las que trabajan de manera individual, y otras en las que pueden actuar dos cualquiera de ellos, pero en las esferas superiores de su administración deben actuar conjuntamente. Nunca dejan personalmente sus mundos de residencia, pero no necesitan hacerlo, ya que estos mundos son los puntos focales superuniversales del extenso sistema de la reflectividad.

\par
%\textsuperscript{(209.8)}
\textsuperscript{18:3.6} Las residencias personales de cada trío de Ancianos de los Días están situadas en el punto de polaridad espiritual de su esfera sede. Estas esferas están divididas en setenta sectores administrativos y tienen setenta capitales divisionarias en las que los Ancianos de los Días residen de vez en cuando.

\par
%\textsuperscript{(210.1)}
\textsuperscript{18:3.7} En lo que se refiere al poder, al alcance de la autoridad y a la extensión de su jurisdicción, los Ancianos de los Días son los más fuertes y los más poderosos de todos los gobernantes directos de las creaciones del espacio-tiempo. En todo el inmenso universo de universos, ellos son los únicos que están investidos con los altos poderes del juicio ejecutivo final en lo que respecta a la extinción eterna de las criaturas volitivas. Y los tres Ancianos de los Días han de participar en los decretos finales del tribunal supremo de un superuniverso.

\par
%\textsuperscript{(210.2)}
\textsuperscript{18:3.8} Aparte de las Deidades y de sus asociados del Paraíso, los Ancianos de los Días son los gobernantes más perfectos, más polifacéticos y más divinamente dotados de todos los que existen en el espacio-tiempo. En apariencia, son los gobernantes supremos de los superuniversos; pero este derecho a gobernar no se lo han ganado por experiencia, y por consiguiente están destinados a ser reemplazados algún día por el Ser Supremo, el soberano experiencial de quien llegarán a ser vicegerentes sin duda alguna.

\par
%\textsuperscript{(210.3)}
\textsuperscript{18:3.9} El Ser Supremo está consiguiendo la soberanía sobre los siete superuniversos por medio del servicio experiencial, exactamente como un Hijo Creador gana por experiencia la soberanía sobre su universo local. Pero durante la era actual en que la evolución del Supremo no ha terminado, los Ancianos de los Días aseguran el supercontrol administrativo coordinado y perfecto de los universos evolutivos del tiempo y del espacio. Todos los decretos y decisiones de los Ancianos de los Días están caracterizados por la sabiduría de la originalidad y la iniciativa de la individualidad.

\section*{4. Los Perfecciones de los Días}
\par
%\textsuperscript{(210.4)}
\textsuperscript{18:4.1} Hay exactamente doscientos diez Perfecciones de los Días y presiden los gobiernos de los diez sectores mayores de cada superuniverso. Fueron trinitizados para el trabajo especial de ayudar a los directores de los superuniversos, y gobiernan como vicegerentes directos y personales de los Ancianos de los Días.

\par
%\textsuperscript{(210.5)}
\textsuperscript{18:4.2} La capital de cada sector mayor tiene asignados tres Perfecciones de los Días, pero a diferencia de los Ancianos de los Días, no es necesario que los tres estén presentes en todo momento. De vez en cuando uno de los miembros de este trío puede ausentarse para conferenciar en persona con los Ancianos de los Días sobre el bienestar de la creación a su cargo.

\par
%\textsuperscript{(210.6)}
\textsuperscript{18:4.3} Estos gobernantes trinos de los sectores mayores son particularmente perfectos en el dominio de los detalles administrativos, de ahí su nombre ---los \textit{Perfecciones} de los Días. Al indicar los nombres de estos seres del mundo espiritual, nos enfrentamos con el problema de traducirlos a vuestra lengua, y muy a menudo es extremadamente difícil ofrecer una traducción satisfactoria. No nos gusta utilizar denominaciones arbitrarias que carecerían de sentido para vosotros; por eso a menudo nos resulta difícil elegir un nombre adecuado, uno que esté claro para vosotros y que al mismo tiempo represente en cierto modo al original.

\par
%\textsuperscript{(210.7)}
\textsuperscript{18:4.4} Los Perfecciones de los Días poseen un grupo moderadamente importante de Consejeros Divinos, de Perfeccionadores de la Sabiduría y de Censores Universales vinculado a sus gobiernos. Disponen de un número aún más importante de Mensajeros Poderosos, de Elevados en Autoridad y de Los que no tienen Nombre ni Número. Pero una gran parte del trabajo rutinario de los asuntos de un sector mayor es efectuado por los Guardianes Celestiales y los Ayudantes de los Hijos Elevados. Estos dos grupos son extraídos de los descendientes trinitizados por las personalidades del Paraíso-Havona o por los finalitarios mortales glorificados. Las Deidades del Paraíso trinitizan de nuevo a algunos miembros de estas dos órdenes de seres trinitizados por las criaturas, y luego los envían como ayudantes a la administración de los gobiernos superuniversales.

\par
%\textsuperscript{(211.1)}
\textsuperscript{18:4.5} La mayor parte de los Guardianes Celestiales y de los Ayudantes de los Hijos Elevados son asignados al servicio de los sectores mayores y menores, pero los Custodios Trinitizados (serafines e intermedios abrazados por la Trinidad) son los oficiales de las audiencias de las tres divisiones, ejerciendo su actividad en los tribunales de los Ancianos de los Días, los Perfecciones de los Días y los Recientes de los Días. Los Embajadores Trinitizados (mortales ascendentes abrazados por la Trinidad cuya naturaleza está fusionada con el Hijo o con el Espíritu) se pueden encontrar en todas las partes de un superuniverso, pero la mayoría presta sus servicios en los sectores menores.

\par
%\textsuperscript{(211.2)}
\textsuperscript{18:4.6} Antes de la época en que el proyecto gubernamental de los siete superuniversos fuera plenamente desvelado, casi todos los administradores de las diversas divisiones de estos gobiernos, exceptuando a los Ancianos de los Días, efectuaron un aprendizaje de duración variable bajo la dirección de los Eternos de los Días en los diversos mundos del universo perfecto de Havona. Los seres trinitizados posteriormente pasaron también una temporada de entrenamiento bajo la dirección de los Eternos de los Días antes de ser destinados al servicio de los Ancianos de los Días, los Perfecciones de los Días y los Recientes de los Días. Todos son administradores maduros, probados y experimentados.

\par
%\textsuperscript{(211.3)}
\textsuperscript{18:4.7} Veréis pronto a los Perfecciones de los Días cuando avancéis hasta la sede de Splandon después de vuestra estancia en los mundos de vuestro sector menor, ya que estos elevados gobernantes están estrechamente asociados con los setenta mundos de los sectores mayores dedicados a la formación superior de las criaturas ascendentes del tiempo. Los Perfecciones de los Días en persona le toman juramento colectivo a los graduados ascendentes de las escuelas de los sectores mayores.

\par
%\textsuperscript{(211.4)}
\textsuperscript{18:4.8} El trabajo de los peregrinos del tiempo en los mundos que rodean a la sede de un sector mayor es principalmente de naturaleza intelectual, en contraste con el carácter más físico y material de la enseñanza en las siete esferas educativas de un sector menor, y con las empresas espirituales en los cuatrocientos noventa mundos universitarios de la sede de un superuniverso.

\par
%\textsuperscript{(211.5)}
\textsuperscript{18:4.9} Aunque sólo estaréis inscritos en el registro del sector mayor de Splandon, que engloba al universo local de vuestro origen, tendréis que pasar por cada una de las diez divisiones mayores de nuestro superuniverso. Veréis a los treinta Perfecciones de los Días de Orvonton antes de llegar a Uversa.

\section*{5. Los Recientes de los Días}
\par
%\textsuperscript{(211.6)}
\textsuperscript{18:5.1} Los Recientes de los Días son los directores supremos más jóvenes de los superuniversos; presiden en grupos de tres los asuntos de los sectores menores. En cuanto a naturaleza están coordinados con los Perfecciones de los Días, pero en lo que se refiere a la autoridad administrativa son sus subordinados. Hay exactamente veintiuna mil de estas personalidades trinitarias personalmente gloriosas y divinamente eficaces. Fueron creadas simultáneamente y pasaron juntas su entrenamiento en Havona bajo la dirección de los Eternos de los Días.

\par
%\textsuperscript{(211.7)}
\textsuperscript{18:5.2} Los Recientes de los Días disponen de un cuerpo de asociados y de ayudantes similar al de los Perfecciones de los Días. Les han asignado además una gran cantidad de seres celestiales de diversas órdenes subordinadas. En la administración de los sectores menores utilizan grandes cantidades de mortales ascendentes residentes, de personal de las diversas colonias de cortesía y de los diversos grupos que tienen su origen en el Espíritu Infinito.

\par
%\textsuperscript{(211.8)}
\textsuperscript{18:5.3} Los gobiernos de los sectores menores se ocupan sobre todo, aunque no exclusivamente, de los grandes problemas físicos de los superuniversos. Las esferas de los sectores menores son las sedes de los Controladores Físicos Maestros. En estos mundos, los mortales ascendentes prosiguen sus estudios y experimentos relacionados con el examen de las actividades de la tercera orden de los Centros Supremos de Poder y de las siete órdenes de Controladores Físicos Maestros.

\par
%\textsuperscript{(212.1)}
\textsuperscript{18:5.4} Puesto que el régimen de un sector menor se ocupa tan extensamente de los problemas físicos, sus tres Recientes de los Días raramente están juntos en la esfera capital. La mayor parte del tiempo uno está fuera entrevistándose con los Perfecciones de los Días del sector mayor supervisor, o se ha ausentado para representar a los Ancianos de los Días en los cónclaves paradisiacos de los seres elevados de origen trinitario. Se alternan con los Perfecciones de los Días para representar a los Ancianos de los Días en los consejos supremos del Paraíso. Mientras tanto, otro Reciente de los Días puede estar fuera en visita de inspección de los mundos sede de los universos locales que pertenecen a su jurisdicción. Pero al menos uno de estos gobernantes permanece siempre de servicio en la sede de un sector menor.

\par
%\textsuperscript{(212.2)}
\textsuperscript{18:5.5} Todos conoceréis algún día a los tres Recientes de los Días encargados de Ensa, vuestro sector menor, puesto que tendréis que pasar por sus manos durante vuestro camino interior hacia los mundos educativos de los sectores mayores. Al ascender hacia Uversa, sólo pasaréis por un grupo de esferas educativas del sector menor.

\section*{6. Los Uniones de los Días}
\par
%\textsuperscript{(212.3)}
\textsuperscript{18:6.1} Las personalidades trinitarias de la orden de los <<\textit{Días}>> no ejercen su capacidad administrativa por debajo del nivel de los gobiernos superuniversales. En los universos locales en evolución sólo actúan como consejeros y asesores. Los Uniones de los Días son un grupo de personalidades de enlace acreditadas por la Trinidad del Paraíso ante los dobles gobernantes de los universos locales. A cada universo local organizado y habitado se le ha asignado uno de estos consejeros paradisiacos, que actúa como representante de la Trinidad y, en algunos aspectos, del Padre Universal, ante la creación local.

\par
%\textsuperscript{(212.4)}
\textsuperscript{18:6.2} Existen setecientos mil seres de este tipo, aunque no todos están en servicio activo. El cuerpo de reserva de los Uniones de los Días ejerce su actividad en el Paraíso como Consejo Supremo de los Ajustes Universales.

\par
%\textsuperscript{(212.5)}
\textsuperscript{18:6.3} Estos observadores trinitarios coordinan de manera especial las actividades administrativas de todas las ramas del gobierno universal, desde las de los universos locales, pasando por los gobiernos de los sectores, hasta las del superuniverso, de ahí su nombre ---los \textit{Uniones} de los Días. Éstos presentan un informe triple a sus superiores: hacen un informe sobre los datos pertinentes de naturaleza física y semi-intelectual a los Recientes de los Días de su sector menor; presentan un informe sobre los acontecimientos intelectuales y casi espirituales a los Perfecciones de los Días de su sector mayor; y hacen un informe sobre los asuntos espirituales y semiparadisiacos a los Ancianos de los Días en la capital de su superuniverso.

\par
%\textsuperscript{(212.6)}
\textsuperscript{18:6.4} Puesto que son seres de origen trinitario, tienen acceso a todos los circuitos del Paraíso para intercomunicarse, y así siempre están en contacto entre ellos y con todas las otras personalidades necesarias, incluidas las que se encuentran en los consejos supremos del Paraíso.

\par
%\textsuperscript{(212.7)}
\textsuperscript{18:6.5} Un Unión de los Días no está conectado orgánicamente con el gobierno del universo local donde está destinado. Aparte de sus deberes como observador, sólo actúa a petición de las autoridades locales. Es miembro de derecho de todos los consejos primarios y de todos los cónclaves importantes de la creación local, pero no participa en el examen técnico de los problemas administrativos.

\par
%\textsuperscript{(213.1)}
\textsuperscript{18:6.6} Cuando un universo local está establecido en la luz y la vida, sus seres glorificados se asocian libremente con el Unión de los Días, que entonces actúa con una capacidad más amplia en ese reino de perfección evolutiva. Pero continúa siendo ante todo un embajador de la Trinidad y un consejero paradisiaco.

\par
%\textsuperscript{(213.2)}
\textsuperscript{18:6.7} Un universo local está gobernado directamente por un Hijo divino con origen doble en la Deidad, pero tiene constantemente a su lado a un hermano del Paraíso, a una personalidad que tiene su origen en la Trinidad. En el caso de que un Hijo Creador se ausente temporalmente de la sede de su universo local, los consejos del Unión de los Días orientan ampliamente a los gobernantes en funciones a la hora de tomar decisiones importantes.

\section*{7. Los Fieles de los Días}
\par
%\textsuperscript{(213.3)}
\textsuperscript{18:7.1} Estas elevadas personalidades de origen trinitario son los asesores paradisiacos de los gobernantes de las cien constelaciones de cada universo local. Hay setenta millones de Fieles de los Días y, al igual que los Uniones de los Días, no todos están de servicio. Su cuerpo de reserva en el Paraíso es la Comisión Consultiva de la Ética y de la Autonomía Interuniversales. Los Fieles de los Días se turnan en su servicio de acuerdo con las decisiones del consejo supremo de su cuerpo de reserva.

\par
%\textsuperscript{(213.4)}
\textsuperscript{18:7.2} Todo lo que un Unión de los Días significa para un Hijo Creador de un universo local, los Fieles de los Días lo significan para los Hijos Vorondadeks que gobiernan las constelaciones de esa creación local. Están supremamente dedicados y son divinamente fieles al bienestar de las constelaciones donde están destinados, de ahí su nombre ---los \textit{Fieles} de los Días. Sólo actúan como consejeros; no participan nunca en las actividades administrativas a menos de haber sido invitados por las autoridades de la constelación. Tampoco se ocupan directamente del ministerio educativo hacia los peregrinos de la ascensión en las esferas arquitectónicas de entrenamiento que rodean a la sede de una constelación. Todas estas empresas están bajo la supervisión de los Hijos Vorondadeks.

\par
%\textsuperscript{(213.5)}
\textsuperscript{18:7.3} Todos los Fieles de los Días que ejercen su actividad en las constelaciones de un universo local están bajo la jurisdicción del Unión de los Días y le informan directamente a él. No poseen un extenso sistema de intercomunicación, limitándose habitualmente a una interasociación dentro de los límites de un universo local. Cualquier Fiel de los Días que se encuentre de servicio en Nebadon puede comunicarse con todos los otros miembros de su orden que estén de servicio en este universo local, y lo hace de hecho.

\par
%\textsuperscript{(213.6)}
\textsuperscript{18:7.4} Al igual que el Unión de los Días en la sede de un universo, los Fieles de los Días mantienen sus residencias personales en las capitales de las constelaciones, separadas de las de los directores administrativos de esos reinos. Sus domicilios son verdaderamente modestos en comparación con los hogares de los gobernantes Vorondadeks de las constelaciones.

\par
%\textsuperscript{(213.7)}
\textsuperscript{18:7.5} Los Fieles de los Días son el último eslabón de la larga cadena consultivo-administrativa que se extiende desde las esferas sagradas del Padre Universal, cerca del centro de todas las cosas, hasta las divisiones primarias de los universos locales. El régimen de origen trinitario termina en las constelaciones; estos asesores del Paraíso no están permanentemente situados en los sistemas componentes ni en los mundos habitados. Estas últimas unidades administrativas están enteramente bajo la jurisdicción de los seres nativos de los universos locales.

\par
%\textsuperscript{(213.8)}
\textsuperscript{18:7.6} [Presentado por un Consejero Divino de Uversa.]


\chapter{Documento 19. Los seres coordinados de origen trinitario}
\par
%\textsuperscript{(214.1)}
\textsuperscript{19:0.1} ESTE grupo paradisiaco, denominado los Seres Coordinados de Origen Trinitario, engloba a los Hijos Instructores Trinitarios, clasificados también entre los Hijos Paradisiacos de Dios, a tres grupos de altos administradores superuniversales, y a la categoría en cierto modo impersonal de los Espíritus Inspirados Trinitarios. En esta clasificación de personalidades trinitarias también se pueden incluir apropiadamente a los nativos de Havona, junto con numerosos grupos de seres que residen en el Paraíso. Los seres de origen trinitario que vamos a considerar en este estudio son los siguientes:

\par
%\textsuperscript{(214.2)}
\textsuperscript{19:0.2} 1. Los Hijos Instructores Trinitarios.

\par
%\textsuperscript{(214.3)}
\textsuperscript{19:0.3} 2. Los Perfeccionadores de la Sabiduría.

\par
%\textsuperscript{(214.4)}
\textsuperscript{19:0.4} 3. Los Consejeros Divinos.

\par
%\textsuperscript{(214.5)}
\textsuperscript{19:0.5} 4. Los Censores Universales.

\par
%\textsuperscript{(214.6)}
\textsuperscript{19:0.6} 5. Los Espíritus Inspirados Trinitarios.

\par
%\textsuperscript{(214.7)}
\textsuperscript{19:0.7} 6. Los Nativos de Havona.

\par
%\textsuperscript{(214.8)}
\textsuperscript{19:0.8} 7. Los Ciudadanos del Paraíso.

\par
%\textsuperscript{(214.9)}
\textsuperscript{19:0.9} A excepción de los Hijos Instructores Trinitarios y quizás de los Espíritus Inspirados Trinitarios, estos grupos tienen un número de seres definitivo; su creación es un acontecimiento consumado que pertenece al pasado.

\section*{1. Los Hijos Instructores Trinitarios}
\par
%\textsuperscript{(214.10)}
\textsuperscript{19:1.1} De todas las ordenes elevadas de personalidades celestiales que os han sido reveladas, sólo los Hijos Instructores Trinitarios actúan en una doble capacidad. Debido a su origen de naturaleza trinitaria, sus funciones están casi enteramente dedicadas a los servicios de la filiación divina. Son los seres de enlace que colman el abismo universal entre las personalidades de origen trinitario y las de origen doble.

\par
%\textsuperscript{(214.11)}
\textsuperscript{19:1.2} Mientras que el número de Hijos Estacionarios de la Trinidad está al completo, la cantidad de Hijos Instructores aumenta constantemente. No sé cual será el número final de Hijos Instructores. Puedo indicar sin embargo que en el último informe periódico enviado a Uversa, los archivos del Paraíso mencionaban que había 21.001.624.821 Hijos de este tipo en servicio.

\par
%\textsuperscript{(214.12)}
\textsuperscript{19:1.3} Estos seres forman el único grupo de Hijos de Dios que os ha sido revelado cuyo origen se encuentra en la Trinidad del Paraíso. Recorren el universo central y los superuniversos, y un cuerpo muy numeroso está asignado a cada universo local. Sirven también en los distintos planetas tal como lo hacen los otros Hijos Paradisiacos de Dios. Puesto que el proyecto del gran universo no está plenamente desarrollado, un gran número de Hijos Instructores se mantienen en reserva en el Paraíso, y se ofrecen como voluntarios para misiones de urgencia y servicios inhabituales en todas las divisiones del gran universo, en los mundos solitarios del espacio, en los universos locales y en los superuniversos, y en los mundos de Havona. También ejercen su actividad en el Paraíso, pero será más provechoso aplazar su estudio detallado hasta que emprendamos el análisis de los Hijos Paradisiacos de Dios.

\par
%\textsuperscript{(215.1)}
\textsuperscript{19:1.4} Sin embargo, se puede señalar a este respecto que los Hijos Instructores son las personalidades coordinadoras supremas de origen trinitario. En un universo de universos tan extenso, siempre existe el gran peligro de sucumbir al error de un punto de vista circunscrito, al mal inherente a una concepción fragmentaria de la realidad y de la divinidad.

\par
%\textsuperscript{(215.2)}
\textsuperscript{19:1.5} Por ejemplo: la mente humana anhelaría normalmente acercarse a la filosofía cósmica descrita en estas revelaciones procediendo de lo simple y de lo finito a lo complejo y a lo infinito, de los orígenes humanos a los destinos divinos. Pero este camino no conduce a la \textit{sabiduría espiritual.} Este procedimiento es el camino más fácil para llegar a cierta forma de \textit{conocimiento genético,} que en el mejor de los casos sólo puede revelar el origen del hombre, pero que revela poco o nada sobre su destino divino.

\par
%\textsuperscript{(215.3)}
\textsuperscript{19:1.6} Incluso en el estudio de la evolución biológica del hombre en Urantia, el enfoque exclusivamente histórico de su situación actual y de sus problemas corrientes presenta graves objeciones. La verdadera perspectiva de cualquier problema sobre la realidad ---humano o divino, terrestre o cósmico--- sólo se puede obtener mediante el estudio y la correlación completos e imparciales de tres fases de la realidad universal: el origen, la historia y el destino. La comprensión adecuada de estas tres realidades experienciales proporciona la base para apreciar sabiamente el estado actual.

\par
%\textsuperscript{(215.4)}
\textsuperscript{19:1.7} Cuando la mente humana sigue la técnica filosófica de partir desde lo inferior para acercarse a lo superior, ya sea en biología o en teología, siempre corre el peligro de cometer cuatro errores de razonamiento:

\par
%\textsuperscript{(215.5)}
\textsuperscript{19:1.8} 1. Puede dejar totalmente de percibir la meta evolutiva final y completa de la realización personal o del destino cósmico.

\par
%\textsuperscript{(215.6)}
\textsuperscript{19:1.9} 2. Puede cometer el error filosófico supremo simplificando con exceso la realidad cósmica evolutiva (experiencial), lo que conduce a deformar los hechos, a desnaturalizar la verdad y a hacerse una idea falsa de los destinos.

\par
%\textsuperscript{(215.7)}
\textsuperscript{19:1.10} 3. El estudio de la causalidad es la lectura atenta de la historia. Pero el conocimiento de \textit{cómo} un ser se vuelve lo que es no proporciona necesariamente una comprensión inteligente del estado actual ni del verdadero carácter de ese ser.

\par
%\textsuperscript{(215.8)}
\textsuperscript{19:1.11} 4. La historia por sí sola no consigue revelar adecuadamente el desarrollo futuro ---el destino. Los orígenes finitos son útiles, pero sólo las causas divinas revelan los efectos finales. Los fines eternos no se manifiestan en los comienzos temporales. El presente sólo se puede interpretar verdaderamente a la luz de su correlación con el pasado y el futuro.

\par
%\textsuperscript{(215.9)}
\textsuperscript{19:1.12} Por eso, a causa de éstas y de otras razones, la técnica que empleamos para acercarnos al hombre y a sus problemas planetarios es la de embarcarnos en el viaje por el tiempo y el espacio partiendo desde la infinita, eterna y divina Fuente y Centro Paradisiaca de toda realidad con personalidad y de toda existencia cósmica.

\section*{2. Los Perfeccionadores de la Sabiduría}
\par
%\textsuperscript{(215.10)}
\textsuperscript{19:2.1} Los Perfeccionadores de la Sabiduría son una creación especializada de la Trinidad del Paraíso, destinada a personificar la sabiduría de la divinidad en los superuniversos. Existen exactamente siete mil millones de seres de este tipo, y mil millones están asignados a cada uno de los siete superuniversos.

\par
%\textsuperscript{(215.11)}
\textsuperscript{19:2.2} Al igual que los Consejeros Divinos y los Censores Universales, que son sus coordinados, los Perfeccionadores de la Sabiduría pasaron por la sabiduría del Paraíso, de Havona y de las esferas paradisiacas del Padre, a excepción de Divinington. Después de estas experiencias, los Perfeccionadores de la Sabiduría fueron destinados de manera permanente al servicio de los Ancianos de los Días. No sirven ni en el Paraíso ni en los mundos de los circuitos del Paraíso-Havona; están totalmente dedicados a la administración de los gobiernos de los superuniversos.

\par
%\textsuperscript{(216.1)}
\textsuperscript{19:2.3} En cualquier momento y lugar en que actúa un Perfeccionador de la Sabiduría, la sabiduría divina funciona de inmediato. Existe una presencia real y una manifestación perfecta en el conocimiento y en la sabiduría representados en las actuaciones de estas personalidades poderosas y majestuosas. No \textit{reflejan} la sabiduría de la Trinidad del Paraíso; \textit{son} esa sabiduría. Son las fuentes de la sabiduría para todos los instructores a la hora de aplicar el conocimiento universal; son las fuentes de la discreción y los manantiales del discernimiento para las instituciones de enseñanza y de perspicacia en todos los universos.

\par
%\textsuperscript{(216.2)}
\textsuperscript{19:2.4} El origen de la sabiduría es doble, pues procede de la perfección de la perspicacia divina inherente a los seres perfectos y de la experiencia personal adquirida por las criaturas evolutivas. Los Perfeccionadores de la Sabiduría \textit{son} la sabiduría divina de la perfección paradisiaca de la perspicacia de la Deidad. Cuando sus asociados administrativos en Uversa, los Mensajeros Poderosos, Los que no tienen Nombre ni Número y Los Elevados en Autoridad actúan juntos, \textit{son} la sabiduría de la experiencia en el universo. Un ser divino puede tener la perfección del conocimiento divino. Un mortal evolutivo puede alcanzar algún día la perfección del conocimiento ascendente, pero ninguno de estos seres agota por sí solo los potenciales de toda la sabiduría posible. Por consiguiente, cada vez que se desea conseguir el máximo de sabiduría administrativa en la conducta del superuniverso, estos perfeccionadores de la sabiduría de la perspicacia divina se asocian siempre con las personalidades ascendentes que se han elevado hasta las altas responsabilidades de la autoridad superuniversal a través de las tribulaciones experienciales de la progresión evolutiva.

\par
%\textsuperscript{(216.3)}
\textsuperscript{19:2.5} Los Perfeccionadores de la Sabiduría necesitarán siempre este complemento de sabiduría experiencial para completar su sagacidad administrativa. Pero se ha presupuesto que los finalitarios del Paraíso quizás podrían conseguir un alto nivel de sabiduría no alcanzado hasta ahora \textit{después} de que inicien algún día la séptima fase de la existencia espiritual. Si esta deducción es correcta, entonces estos seres perfeccionados de la ascensión evolutiva se convertirían sin duda en los administradores universales más eficaces que se haya conocido nunca en toda la creación. Creo que este es el alto destino de los finalitarios.

\par
%\textsuperscript{(216.4)}
\textsuperscript{19:2.6} La variedad de talentos de los Perfeccionadores de la Sabiduría les permite participar en casi todos los servicios celestiales de las criaturas ascendentes. Los Perfeccionadores de la Sabiduría y mi orden de personalidad, los Consejeros Divinos, junto con los Censores Universales, constituyen los tipos de seres más elevados que pueden y se dedican a la tarea de revelar la verdad a los planetas y a los sistemas individuales, ya sea en sus épocas primitivas o cuando están establecidos en la luz y la vida. De vez en cuando todos nos ponemos en contacto con el servicio de los mortales ascendentes, desde un planeta donde se ha iniciado la vida hasta un universo local o el superuniverso, especialmente en este último.

\section*{3. Los Consejeros Divinos}
\par
%\textsuperscript{(216.5)}
\textsuperscript{19:3.1} Estos seres de origen trinitario son el consejo de la Deidad para las esferas de los siete superuniversos. No son el \textit{reflejo} del consejo divino de la Trinidad; \textit{son} ese consejo. Existen veintiún mil millones de Consejeros en servicio, y tres mil millones están destinados en cada superuniverso.

\par
%\textsuperscript{(217.1)}
\textsuperscript{19:3.2} Los Consejeros Divinos son los asociados y los iguales de los Censores Universales y de los Perfeccionadores de la Sabiduría, y con cada una de estas últimas personalidades están asociados entre uno y siete Consejeros. Las tres órdenes participan en el gobierno de los Ancianos de los Días, incluyendo los sectores mayores y menores, en los universos locales y las constelaciones, y en los consejos de los soberanos de los sistemas locales.

\par
%\textsuperscript{(217.2)}
\textsuperscript{19:3.3} Actuamos como individuos, tal como yo lo hago al redactar esta exposición, pero también ejercemos nuestra actividad como trío cuando lo requieren las circunstancias. Cuando actuamos con capacidad ejecutiva siempre estamos asociados de común acuerdo un Perfeccionador de la Sabiduría, un Censor Universal y entre uno y siete Consejeros Divinos.

\par
%\textsuperscript{(217.3)}
\textsuperscript{19:3.4} Un Perfeccionador de la Sabiduría, siete Consejeros Divinos y un Censor Universal constituyen un tribunal de divinidad trinitaria, el cuerpo consultivo itinerante más elevado de los universos del tiempo y del espacio. A estos grupos de nueve se les conoce o bien como tribunales encargados de descubrir los hechos, o bien de revelar la verdad, y cuando juzgan un problema y pronuncian una decisión, es exactamente como si un Anciano de los Días hubiera juzgado el asunto, porque un veredicto así nunca ha sido revocado por los Ancianos de los Días en todos los anales de los superuniversos.

\par
%\textsuperscript{(217.4)}
\textsuperscript{19:3.5} Cuando los tres Ancianos de los Días actúan, la Trinidad del Paraíso actúa. Cuando el tribunal de los nueve llega a una decisión después de haber deliberado de manera conjunta, a todos los efectos prácticos los Ancianos de los Días han hablado. De esta manera es como los Gobernantes del Paraíso se ponen en contacto personal con los mundos, los sistemas y los universos individuales en materia administrativa y en reglamentación gubernamental.

\par
%\textsuperscript{(217.5)}
\textsuperscript{19:3.6} Los Consejeros Divinos son la perfección del consejo divino de la Trinidad del Paraíso. Nosotros representamos, de hecho \textit{somos,} el consejo de la perfección. Cuando contamos con el complemento del consejo experiencial de nuestros asociados, los seres ascendentes evolutivos perfeccionados y abrazados por la Trinidad, nuestras conclusiones combinadas no sólo son completas, sino plenas. Cuando nuestro consejo unificado ha sido asociado, juzgado, confirmado y promulgado por un Censor Universal, es muy probable que se acerque al umbral de la totalidad universal. Estos veredictos representan el máximo acercamiento posible a la actitud absoluta de la Deidad dentro de los límites espacio-temporales de la situación en juego y del problema en cuestión.

\par
%\textsuperscript{(217.6)}
\textsuperscript{19:3.7} Siete Consejeros Divinos, en conexión con un trío evolutivo trinitizado ---un Poderoso Mensajero, un Elevado en Autoridad y uno que no tiene Nombre ni Número--- representan la mayor aproximación superuniversal a la unión del punto de vista humano con la actitud divina en los niveles casi paradisiacos de los significados espirituales y de los valores de la realidad. Esta aproximación tan estrecha entre las actitudes cósmicas unidas de la criatura y del Creador sólo es sobrepasada por los Hijos donadores Paradisiacos, que son Dios y hombre en todas las fases de la experiencia de la personalidad.

\section*{4. Los Censores Universales}
\par
%\textsuperscript{(217.7)}
\textsuperscript{19:4.1} Existen exactamente ocho mil millones de Censores Universales. Estos seres únicos \textit{son} el juicio de la Deidad. No se limitan a reflejar las decisiones de la perfección; \textit{son} el juicio de la Trinidad del Paraíso. Ni siquiera los Ancianos de los Días se sientan a juzgar a menos que lo hagan en asociación con los Censores Universales.

\par
%\textsuperscript{(217.8)}
\textsuperscript{19:4.2} Un Censor es nombrado para cada uno de los mil millones de mundos del universo central, estando vinculado a la administración planetaria del Eterno de los Días residente. Ni los Perfeccionadores de la Sabiduría ni los Consejeros Divinos están vinculados así de manera permanente a las administraciones de Havona; y tampoco comprendemos plenamente por qué los Censores Universales están estacionados en el universo central. Sus actividades actuales apenas justifican su trabajo en Havona, y por eso sospechamos que se encuentran allí anticipándose a las necesidades de una era universal futura en la que la población de Havona podría cambiar parcialmente.

\par
%\textsuperscript{(218.1)}
\textsuperscript{19:4.3} Mil millones de Censores están destinados en cada uno de los siete superuniversos. Trabajan en todas las divisiones de los siete superuniversos tanto a título individual como en asociación con los Perfeccionadores de la Sabiduría y los Consejeros Divinos. Los Censores actúan así en todos los niveles del gran universo, desde los mundos perfectos de Havona hasta los consejos de los Soberanos de los Sistemas, y forman parte orgánica de todos los juicios dispensacionales de los mundos evolutivos.

\par
%\textsuperscript{(218.2)}
\textsuperscript{19:4.4} En cualquier momento y lugar en que un Censor Universal está presente, allí se encuentra el juicio de la Deidad. Y puesto que los Censores siempre pronuncian sus veredictos en unión con los Perfeccionadores de la Sabiduría y los Consejeros Divinos, tales decisiones engloban la sabiduría, el consejo y el juicio unidos de la Trinidad del Paraíso. En este trío jurídico, el Perfeccionador de la Sabiduría sería el <<\textit{yo era}>> y el Consejero Divino el <<\textit{yo seré}>>, pero el Censor Universal siempre es el <<\textit{yo soy}>>.

\par
%\textsuperscript{(218.3)}
\textsuperscript{19:4.5} Los Censores son las personalidades totalizadoras del universo. Cuando mil testigos ---o un millón de testigos--- han dado su testimonio, cuando la voz de la sabiduría ha hablado y el consejo de la divinidad ha sido registrado, cuando se ha añadido el testimonio de la perfección ascendente, entonces el Censor actúa e inmediatamente se revela una totalización infalible y divina de todo lo que ha sucedido; esta revelación representa la conclusión divina, la suma y la sustancia de una decisión final y perfecta. Por eso cuando un Censor ha hablado, nadie más puede hacerlo, porque el Censor ha descrito la verdadera e inequívoca totalidad de todo lo que ha ocurrido. Cuando habla, no hay apelación.

\par
%\textsuperscript{(218.4)}
\textsuperscript{19:4.6} Comprendo perfectamente el funcionamiento de la mente de un Perfec-cionador de la Sabiduría, pero no entiendo ciertamente por completo el funcionamiento de la mente que juzga de un Censor Universal. Me parece que los Censores expresan nuevos significados y dan origen a nuevos valores asociando los hechos, las verdades y los hallazgos que les han sido presentados en el transcurso de una investigación sobre los asuntos universales. Parece probable que los Censores Universales sean capaces de dar interpretaciones originales de la combinación entre la perspicacia perfecta del Creador y la experiencia perfeccionada de la criatura. Esta asociación entre la perfección paradisiaca y la experiencia universal produce indudablemente un nuevo valor en los niveles últimos.

\par
%\textsuperscript{(218.5)}
\textsuperscript{19:4.7} Pero aquí no terminan nuestras dificultades en lo que concierne al funcionamiento de la mente de los Censores Universales. Después de tener debidamente en cuenta todo lo que sabemos o suponemos sobre la actividad de un Censor en una situación universal dada, descubrimos que aún somos incapaces de predecir sus decisiones y de prever sus veredictos. Determinamos con mucha precisión el resultado probable de la asociación entre la actitud del Creador y la experiencia de la criatura, pero estas conclusiones no siempre son unas previsiones exactas de las revelaciones del Censor. Parece probable que los Censores tengan algún tipo de conexión con el Absoluto de la Deidad, pues somos incapaces de explicar de otra manera una gran parte de sus fallos y decisiones.

\par
%\textsuperscript{(218.6)}
\textsuperscript{19:4.8} Los Perfeccionadores de la Sabiduría, los Consejeros Divinos y los Censores Universales, junto con las siete órdenes de Personalidades Trinitarias Supremas, constituyen los diez grupos que a veces han sido denominados los \textit{Hijos Estacionarios de la Trinidad.} Juntos componen el gran cuerpo de administradores, gobernantes, ejecutivos, asesores, consejeros y jueces de la Trinidad. Su número supera ligeramente los treinta y siete mil millones. Dos mil millones setenta están estacionados en el universo central, y un poco más de cinco mil millones en cada superuniverso.

\par
%\textsuperscript{(219.1)}
\textsuperscript{19:4.9} Es muy difícil describir los límites funcionales de los Hijos Estacionarios de la Trinidad. Sería incorrecto afirmar que sus actos se limitan a lo finito, porque hay operaciones registradas en los superuniversos que indican lo contrario. Actúan en cualquier nivel administrativo o judicial del universo en el que las condiciones espacio-temporales puedan necesitarlo y que tenga relación con la evolución pasada, presente y futura del universo maestro.

\section*{5. Los Espíritus Inspirados Trinitarios}
\par
%\textsuperscript{(219.2)}
\textsuperscript{19:5.1} Seré capaz de deciros muy poca cosa acerca de los Espíritus Inspirados Trinitarios, porque son una de las pocas órdenes de seres existentes enteramente secretas, y son secretas sin duda porque les resulta imposible revelarse plenamente incluso a aquellos de nosotros cuyo origen se encuentra tan cerca de la fuente que los ha creado. Surgen a la existencia mediante un acto de la Trinidad del Paraíso y pueden ser utilizados por una o por dos de las Deidades, así como por las tres. No sabemos si el número de estos Espíritus es definitivo o si crece constantemente, pero nos inclinamos a creer que su número no es fijo.

\par
%\textsuperscript{(219.3)}
\textsuperscript{19:5.2} No comprendemos plenamente ni la naturaleza ni la conducta de los Espíritus Inspirados. Quizás pertenecen a la categoría de los espíritus superpersonales. Parecen efectuar sus operaciones en todos los circuitos conocidos y parecen actuar casi con independencia del tiempo y del espacio. Pero sabemos poca cosa de ellos, salvo que deducimos su carácter a partir de la naturaleza de sus actividades, cuyos resultados observamos con certeza aquí y allá en los universos.

\par
%\textsuperscript{(219.4)}
\textsuperscript{19:5.3} Bajo ciertas condiciones, estos Espíritus Inspirados pueden individualizarse lo suficiente como para ser reconocidos por los seres de origen trinitario. Yo los he visto, pero a las órdenes inferiores de seres celestiales nunca les sería posible reconocer a uno de ellos. De vez en cuando surgen también ciertas circunstancias en la conducta de los universos evolutivos en las que cualquier ser de origen trinitario puede emplear directamente a estos Espíritus para apoyar sus tareas. Sabemos pues que existen, y que bajo ciertas condiciones podemos pedir y recibir su ayuda, y a veces reconocer su presencia. Pero no forman parte de la organización manifiesta y claramente revelada encargada de dirigir los universos espacio-temporales antes de que estas creaciones materiales se establezcan en la luz y la vida. No tienen un lugar claramente discernible en la economía o en la administración actuales de los siete superuniversos en evolución. Son un secreto de la Trinidad del Paraíso.

\par
%\textsuperscript{(219.5)}
\textsuperscript{19:5.4} Los Melquisedeks de Nebadon enseñan que, en algún momento del eterno futuro, los Espíritus Inspirados Trinitarios están destinados a reemplazar a los Mensajeros Solitarios, cuyas filas se están reduciendo de manera lenta pero segura debido a que son asignados como asociados a ciertos tipos de hijos trinitizados.

\par
%\textsuperscript{(219.6)}
\textsuperscript{19:5.5} Los Espíritus Inspirados son los Espíritus solitarios del universo de universos. Como Espíritus se parecen mucho a los Mensajeros Solitarios, salvo que estos últimos son personalidades bien diferenciadas. Una gran parte de nuestro conocimiento sobre los Espíritus Inspirados la obtenemos de los Mensajeros Solitarios, los cuales detectan su proximidad debido a su inherente sensibilidad a la presencia de los Espíritus Inspirados, que funciona de forma tan infalible como una aguja imantada apunta hacia un polo magnético. Cuando un Mensajero Solitario se encuentra cerca de un Espíritu Inspirado Trinitario, es consciente de una indicación cualitativa de esa presencia divina y también de un registro cuantitativo muy preciso que le permite conocer realmente la clasificación de la presencia o presencias de estos Espíritus, y el número de ellas.

\par
%\textsuperscript{(220.1)}
\textsuperscript{19:5.6} Puedo contar otro hecho interesante: Cuando un Mensajero Solitario se encuentra en un planeta cuyos habitantes han recibido Ajustadores del Pensamiento, como sucede en Urantia, es consciente de una excitación cualitativa en su sensibilidad detectora de presencias espirituales. En estos casos no se produce una excitación cuantitativa, sino sólo una agitación cualitativa. Cuando se encuentra en un planeta donde no vienen los Ajustadores, su contacto con los nativos no produce este tipo de reacción. Esto sugiere que los Ajustadores del Pensamiento están conectados o relacionados de alguna manera con los Espíritus Inspirados de la Trinidad del Paraíso. Es posible que estén asociados de alguna forma en ciertas fases de su trabajo, pero no lo sabemos realmente. Los dos tienen su origen cerca del centro y la fuente de todas las cosas, pero no pertenecen a la misma orden de seres. Los Ajustadores del Pensamiento surgen del Padre exclusivamente; los Espíritus Inspirados son los descendientes de la Trinidad del Paraíso.

\par
%\textsuperscript{(220.2)}
\textsuperscript{19:5.7} Aparentemente, los Espíritus Inspirados no pertenecen al proyecto evolutivo de los planetas o de los universos individuales, y sin embargo parecen estar en casi todas partes. Mientras estoy ocupado formulando esta exposición, la sensibilidad personal que posee mi Mensajero Solitario asociado ante la presencia de esta orden de Espíritus indica que en este mismo momento se encuentra con nosotros, a menos de ocho metros de distancia, un Espíritu de la orden de los Inspirados, cuya presencia tiene una fuerza del tercer volumen. La presencia de una fuerza del tercer volumen nos sugiere la probabilidad de que tres Espíritus Inspirados estén actuando en conexión.

\par
%\textsuperscript{(220.3)}
\textsuperscript{19:5.8} Más de doce órdenes de seres están asociados conmigo en este momento, y de ellos el Mensajero Solitario es el único que es consciente de la presencia de estas misteriosas entidades de la Trinidad. Además, aunque estamos avisados así de que estos Espíritus divinos están cerca, todos ignoramos por igual cuál es su misión. No sabemos realmente si se trata de simples observadores interesados en nuestras actividades, o si están contribuyendo efectivamente, de alguna manera desconocida para nosotros, al éxito de nuestra empresa.

\par
%\textsuperscript{(220.4)}
\textsuperscript{19:5.9} Sabemos que los Hijos Instructores Trinitarios están dedicados a la iluminación \textit{consciente} de las criaturas del universo. He llegado a la firme conclusión de que, mediante unas técnicas \textit{superconscientes,} los Espíritus Inspirados Trinitarios también actúan como instructores de los reinos. Estoy persuadido de que existe una inmensa cantidad de conocimientos espirituales esenciales, de verdades indispensables para alcanzar un alto nivel espiritual, que no se pueden recibir de manera consciente; la conciencia del yo pondría efectivamente en peligro la certeza de su recepción. Si este concepto es correcto, y todos los seres de mi orden lo comparten, la misión de estos Espíritus Inspirados puede consistir en vencer esta dificultad, en colmar esta laguna en el programa universal de iluminación moral y de progreso espiritual. Pensamos que estos dos tipos de instructores de origen trinitario efectúan alguna clase de conexión en sus actividades, pero en realidad no lo sabemos.

\par
%\textsuperscript{(220.5)}
\textsuperscript{19:5.10} He fraternizado con los mortales que se perfeccionan ---con las almas ascendentes y espiritualizadas de los reinos evolutivos--- en los mundos educativos de los superuniversos y en los circuitos eternos de Havona, pero nunca han sido conscientes de los Espíritus Inspirados que los Mensajeros Solitarios, con sus poderes de detección residentes, indicaban de vez en cuando que se hallaban muy cerca de nosotros. He conversado abiertamente con todas las órdenes de Hijos de Dios, superiores e inferiores, y éstas tampoco tienen conciencia de las exhortaciones de los Espíritus Inspirados Trinitarios. Pueden recordar sus experiencias, y lo hacen de hecho, y mencionan sucesos que son difíciles de explicar si no se tiene en cuenta la acción de estos Espíritus. Pero a excepción de los Mensajeros Solitarios, y a veces de los seres de origen trinitario, ningún miembro de la familia celestial ha sido nunca consciente de la proximidad de los Espíritus Inspirados.

\par
%\textsuperscript{(221.1)}
\textsuperscript{19:5.11} No creo que los Espíritus Inspirados Trinitarios estén jugando al escondite conmigo. Probablemente intentan revelarse a mí con la misma insistencia con que yo trato de comunicarme con ellos; nuestras dificultades y limitaciones deben ser mutuas e inherentes. Estoy convencido de que no existen secretos arbitrarios en el universo; por eso nunca cesaré en mis esfuerzos por resolver el misterio del aislamiento de estos Espíritus que pertenecen a mi orden de seres creados.

\par
%\textsuperscript{(221.2)}
\textsuperscript{19:5.12} Vosotros los mortales, que estáis dando ahora vuestros primeros pasos hacia el viaje eterno, podéis ver muy bien por todo lo dicho anteriormente que tenéis que recorrer un largo camino antes de progresar por medio de la <<\textit{vista}>> y de la seguridad <<\textit{material}>>. Tendréis que utilizar la fe y depender de la revelación durante mucho tiempo si esperáis progresar con rapidez y seguridad.

\section*{6. Los Nativos de Havona}
\par
%\textsuperscript{(221.3)}
\textsuperscript{19:6.1} Los nativos de Havona son la creación directa de la Trinidad del Paraíso, y su número sobrepasa la capacidad de vuestra mente limitada. A los urantianos tampoco les resulta posible concebir los dones inherentes a estas criaturas divinamente perfectas que pertenecen a estas razas de origen trinitario del universo eterno. Nunca podréis imaginaros realmente a estas criaturas gloriosas; tendréis que esperar a llegar a Havona, y entonces podréis saludarlas como camaradas espirituales.

\par
%\textsuperscript{(221.4)}
\textsuperscript{19:6.2} Durante vuestra larga estancia en los mil millones de mundos de cultura havoniana, desarrollaréis una amistad eterna con estos seres magníficos. !`Y cuán profunda es esta amistad que crece entre las criaturas personales más humildes de los mundos del espacio y estos elevados seres personales nacidos en las esferas perfectas del universo central! Durante su larga y afectuosa asociación con los nativos de Havona, los mortales ascendentes hacen muchas cosas para compensar el empobrecimiento espiritual de las etapas iniciales de la progresión humana. Al mismo tiempo, a través de sus contactos con los peregrinos ascendentes, los havonianos adquieren una experiencia que supera en gran medida la desventaja experiencial de haber vivido siempre una vida de perfección divina. El bien que obtienen tanto los mortales ascendentes como los nativos de Havona es grande y mutuo.

\par
%\textsuperscript{(221.5)}
\textsuperscript{19:6.3} Los nativos de Havona, al igual que todas las otras personalidades de origen trinitario, son proyectados en perfección divina, y lo mismo que sucede con otras personalidades de origen trinitario, el paso del tiempo puede aumentar sus reservas de dones experienciales. Pero a diferencia de los Hijos Estacionarios de la Trinidad, el estado de los havonianos puede evolucionar, pueden tener un futuro destino no revelado en la eternidad. Esto queda ilustrado en aquellos havonianos que, a través del servicio, convierten en un hecho su capacidad para fusionar con un fragmento no Ajustador del Padre, lo cual los capacita para volverse miembros del Cuerpo Finalitario de los Mortales. Y existen otros cuerpos finalitarios que están abiertos a estos nativos del universo central.

\par
%\textsuperscript{(221.6)}
\textsuperscript{19:6.4} La evolución del estado de los nativos de Havona ha provocado muchas especulaciones en Uversa. Puesto que se están infiltrando constantemente en los diversos Cuerpos Paradisiacos de la Finalidad, y puesto que ya no se crean nuevos seres, es evidente que el número de nativos que permanecen en Havona disminuye constantemente. Las consecuencias finales de estas operaciones nunca nos han sido reveladas, pero no creemos que Havona se quede nunca totalmente desprovista de sus nativos. Hemos mantenido la teoría de que los havonianos quizás dejen de entrar algún día en los cuerpos finalitarios durante las eras en que se procederá a crear sucesivamente los niveles del espacio exterior. También hemos albergado la idea de que en estas eras universales futuras el universo central podría estar poblado de un grupo mixto de seres residentes, una ciudadanía que sólo estaría compuesta en parte por los nativos originales de Havona. Así pues, no sabemos qué orden o tipo de criaturas podrían estar destinadas a beneficiarse del estado residencial en el Havona del futuro, pero hemos pensado en:

\par
%\textsuperscript{(222.1)}
\textsuperscript{19:6.5} 1. Los univitatias, que son actualmente los ciudadanos permanentes de las constelaciones de los universos locales.

\par
%\textsuperscript{(222.2)}
\textsuperscript{19:6.6} 2. Los tipos futuros de mortales que puedan nacer en las esferas habitadas de los superuniversos cuando florezcan las eras de luz y de vida.

\par
%\textsuperscript{(222.3)}
\textsuperscript{19:6.7} 3. La aristocracia espiritual procedente de los sucesivos universos exteriores.

\par
%\textsuperscript{(222.4)}
\textsuperscript{19:6.8} Sabemos que el Havona de la era universal anterior era un poco diferente al Havona de la época actual. Consideramos que es simplemente razonable suponer que ahora estamos presenciando en el universo central aquellos lentos cambios anticipadores de las eras por venir. Una cosa es segura: el universo no es estático; sólo Dios es invariable.

\section*{7. Los Ciudadanos del Paraíso}
\par
%\textsuperscript{(222.5)}
\textsuperscript{19:7.1} En el Paraíso\footnote{\textit{Paraíso}: Lc 23:43; 2 Co 12:4; Ap 2:7.} residen numerosos grupos de seres magníficos, los Ciudadanos del Paraíso. No están directamente relacionados con el proyecto de perfeccionar a las criaturas volitivas ascendentes, y por eso no son plenamente revelados a los mortales de Urantia. Existen más de tres mil órdenes de estas inteligencias celestiales, y el último grupo fue personalizado al mismo tiempo que la Trinidad emitía el mandato que promulgaba el plan creativo de los siete superuniversos del tiempo y del espacio.

\par
%\textsuperscript{(222.6)}
\textsuperscript{19:7.2} Los Ciudadanos del Paraíso y los nativos de Havona a veces se conocen por el nombre colectivo de \textit{personalidades del Paraíso-Havona.}

\par
%\textsuperscript{(222.7)}
\textsuperscript{19:7.3} Esto completa la historia de los seres que son traídos a la existencia por la Trinidad del Paraíso. Ninguno de ellos se ha descarriado nunca. Y sin embargo, todos están dotados de libre albedrío en el sentido más elevado.

\par
%\textsuperscript{(222.8)}
\textsuperscript{19:7.4} Los seres de origen trinitario poseen unas prerrogativas de transporte que los hacen independientes de las personalidades transportadoras tales como los serafines. Todos poseemos el poder de desplazarnos libre y rápidamente por el universo de universos. A excepción de los Espíritus Inspirados Trinitarios, no podemos alcanzar la velocidad casi increíble de los Mensajeros Solitarios, pero somos capaces de utilizar la totalidad de los medios de transporte en el espacio de tal manera que, partiendo de su mundo sede, podemos llegar a cualquier punto de un superuniverso en menos de un año del tiempo de Urantia. He necesitado 109 días de vuestro tiempo para viajar desde Uversa hasta Urantia.

\par
%\textsuperscript{(222.9)}
\textsuperscript{19:7.5} Tenemos la capacidad de intercomunicarnos instantáneamente a través de estos mismos medios. Toda nuestra orden creada se encuentra en contacto con todos los individuos incluidos en cada una de las divisiones compuestas por los hijos de la Trinidad del Paraíso, a excepción únicamente de los Espíritus Inspirados.

\par
%\textsuperscript{(222.10)}
\textsuperscript{19:7.6} [Presentado por un Consejero Divino de Uversa.]


\chapter{Documento 20. Los Hijos Paradisiacos de Dios}
\par
%\textsuperscript{(223.1)}
\textsuperscript{20:0.1} SEGÚN sus actividades en el superuniverso de Orvonton, los Hijos de Dios están clasificados en tres secciones generales:

\par
%\textsuperscript{(223.2)}
\textsuperscript{20:0.2} 1. Los Hijos de Dios descendentes\footnote{\textit{Los Hijos de Dios descendentes}: Jn 1:1-2.}.

\par
%\textsuperscript{(223.3)}
\textsuperscript{20:0.3} 2. Los Hijos de Dios ascendentes.

\par
%\textsuperscript{(223.4)}
\textsuperscript{20:0.4} 3. Los Hijos de Dios trinitizados.

\par
%\textsuperscript{(223.5)}
\textsuperscript{20:0.5} Las órdenes descendentes de filiación incluyen a las personalidades que han sido creadas de manera directa y divina. Los hijos ascendentes, tales como las criaturas mortales, consiguen este estado participando experiencialmente en la técnica creativa conocida como evolución. Los Hijos Trinitizados son un grupo de origen compuesto que incluye a todos los seres abrazados por la Trinidad del Paraíso, aunque no tengan su origen directo en la Trinidad.

\section*{1. Los Hijos descendentes de Dios}
\par
%\textsuperscript{(223.6)}
\textsuperscript{20:1.1} Todos los Hijos descendentes de Dios tienen un origen elevado y divino\footnote{\textit{Origen divino}: Jn 1:1-2.}. Están dedicados al ministerio descendente de servir en los mundos y sistemas del tiempo y del espacio para facilitar allí el progreso de las criaturas humildes de origen evolutivo ---de los hijos ascendentes de Dios--- en su ascensión hacia el Paraíso. En esta narración describiremos siete de las numerosas órdenes de Hijos descendentes. A los Hijos que surgen de las Deidades en la Isla central de Luz y de Vida se les llama \textit{Hijos Paradisiacos de Dios}\footnote{\textit{Hijos Paradisiacos de Dios}: Jn 1:1-2; Flp 2:5-7.} y abarcan las tres órdenes siguientes:

\par
%\textsuperscript{(223.7)}
\textsuperscript{20:1.2} 1. Los Hijos Creadores ---los Migueles.

\par
%\textsuperscript{(223.8)}
\textsuperscript{20:1.3} 2. Los Hijos Magistrales ---los Avonales.

\par
%\textsuperscript{(223.9)}
\textsuperscript{20:1.4} 3. Los Hijos Instructores Trinitarios ---los Daynales.

\par
%\textsuperscript{(223.10)}
\textsuperscript{20:1.5} A las cuatro órdenes restantes de filiación descendente se les conoce como los \textit{Hijos de Dios de los Universos Locales:}

\par
%\textsuperscript{(223.11)}
\textsuperscript{20:1.6} 4. Los Hijos Melquisedeks.

\par
%\textsuperscript{(223.12)}
\textsuperscript{20:1.7} 5. Los Hijos Vorondadeks.

\par
%\textsuperscript{(223.13)}
\textsuperscript{20:1.8} 6. Los Hijos Lanonandeks.

\par
%\textsuperscript{(223.14)}
\textsuperscript{20:1.9} 7. Los Portadores de Vida.

\par
%\textsuperscript{(223.15)}
\textsuperscript{20:1.10} Los Melquisedeks son los descendientes conjuntos del Hijo Creador, el Espíritu Creativo y el Padre Melquisedek de un universo local. Tanto los Vorondadeks como los Lanonandeks son engendrados por un Hijo Creador y su Espíritu Creativo asociado. A los Vorondadeks se les conoce mejor como los Altísimos, los Padres de las Constelaciones, y a los Lanonandeks como Soberanos de los Sistemas y Príncipes Planetarios. La orden triple de los Portadores de Vida es traída a la existencia por un Hijo Creador y un Espíritu Creativo asociados con uno de los tres Ancianos de los Días del superuniverso a cuya jurisdicción están sometidos. Pero la naturaleza y las actividades de estos Hijos de Dios de los universos locales se describen más adecuadamente en los documentos que tratan de los asuntos de las creaciones locales.

\par
%\textsuperscript{(224.1)}
\textsuperscript{20:1.11} Los Hijos Paradisiacos de Dios tienen un origen triple: los Hijos Creadores o primarios son traídos a la existencia por el Padre Universal y el Hijo Eterno; los Hijos Magistrales o secundarios son los hijos del Hijo Eterno y del Espíritu Infinito; los Hijos Instructores Trinitarios son los descendientes del Padre, el Hijo y el Espíritu. Desde el punto de vista del servicio, de la adoración y de la súplica, los Hijos Paradisiacos son como uno solo; su espíritu es uno solo, y su trabajo es idéntico en calidad y en perfección.

\par
%\textsuperscript{(224.2)}
\textsuperscript{20:1.12} Al igual que las órdenes paradisiacas de los Días han demostrado ser unos administradores divinos, las órdenes de los Hijos Paradisiacos se han revelado como ministros divinos ---creadores, servidores, donadores, jueces, instructores y reveladores de la verdad. Recorren el universo de universos desde las orillas de la Isla eterna hasta los mundos habitados del tiempo y del espacio, efectuando en el universo central y en los superuniversos múltiples servicios no revelados en estas narraciones. Están organizados de manera diversa, dependiendo de la naturaleza y del lugar de su servicio, pero en un universo local, tanto los Hijos Magistrales como los Hijos Instructores sirven bajo la dirección del Hijo Creador que preside ese dominio.

\par
%\textsuperscript{(224.3)}
\textsuperscript{20:1.13} Los Hijos Creadores parecen poseer una dotación espiritual centrada en su persona, que controlan y que pueden otorgar, tal como lo hizo vuestro propio Hijo Creador cuando derramó su espíritu\footnote{\textit{El hijo derramó su espíritu sobre la carne}: Ez 11:19; 18:31; 36:26-27; Jl 2:28-29; Lc 24:49; Jn 7:39; 14:16-18,23,26; 15:4,26; 16:7-8,13-14; 17:21-23; Hch 1:5,8a; 2:1-4,17-18; 2:33; 2 Co 13:5; Gl 2:20; 4:6; Ef 1:13; 4:30; 1 Jn 4:12-15.} sobre todo el género humano de Urantia. Cada Hijo Creador está dotado de este poder de atracción espiritual\footnote{\textit{Gravedad espiritual}: Jer 31:3; Jn 6:44; 12:32.} en su propio reino; es personalmente consciente de todos los actos y de todas las emociones de cada Hijo descendente de Dios que sirve en sus dominios. Hay aquí un reflejo divino, un duplicado en los universos locales, de ese poder de atracción espiritual absoluto del Hijo Eterno que le permite asociarse con todos sus Hijos Paradisiacos, poniéndose y manteniéndose en contacto con ellos en cualquier lugar donde puedan encontrarse en todo el universo de universos.

\par
%\textsuperscript{(224.4)}
\textsuperscript{20:1.14} Los Hijos Creadores Paradisiacos no sirven solamente como Hijos en sus ministerios descendentes de servicio y de donación, sino que cuando han terminado sus carreras de donación, cada uno de ellos actúa como un Padre en el universo que ellos mismos han creado, mientras que los otros Hijos de Dios continúan su servicio de donación y de elevación espiritual destinado a conseguir que los planetas reconozcan voluntariamente, uno tras otro, el gobierno amoroso del Padre Universal, culminando todo ello en la consagración de la criatura a la voluntad del Padre Paradisiaco y en la lealtad planetaria a la soberanía universal de su Hijo Creador.

\par
%\textsuperscript{(224.5)}
\textsuperscript{20:1.15} En un Hijo Creador séptuple, el Creador y la criatura están mezclados para siempre en una asociación comprensiva, compasiva y misericordiosa. Toda la orden de los Migueles, los Hijos Creadores, es tan excepcional que el estudio de su naturaleza y de sus actividades lo reservamos para el siguiente documento de esta serie, mientras que esta narración tratará principalmente de las dos órdenes restantes de filiación paradisiaca: los Hijos Magistrales y los Hijos Instructores Trinitarios.

\section*{2. Los Hijos Magistrales}
\par
%\textsuperscript{(224.6)}
\textsuperscript{20:2.1} Cada vez que el Hijo Eterno manifiesta un concepto original y absoluto de un ser, y este concepto se une con un ideal nuevo y divino de servicio amoroso concebido por el Espíritu Infinito, se da nacimiento a un Hijo de Dios nuevo y original, a un Hijo Paradisiaco Magistral. Estos Hijos componen la orden de los Avonales, en contraste con la orden de los Migueles, los Hijos Creadores. Aunque no son creadores en el sentido personal, en todo su trabajo están estrechamente asociados con los Migueles. Los Avonales son los ministros y los jueces planetarios, los magistrados de los reinos del espacio-tiempo ---de todas las razas, para todos los mundos y en todos los universos.

\par
%\textsuperscript{(225.1)}
\textsuperscript{20:2.2} Tenemos razones para creer que el número total de Hijos Magistrales en el gran universo es de unos mil millones. Es una orden autónoma, que está dirigida por su consejo supremo en el Paraíso, el cual está compuesto de Avonales experimentados que han sido apartados de los servicios de todos los universos. Pero cuando están destinados y en servicio activo en un universo local, sirven bajo la dirección del Hijo Creador de ese dominio.

\par
%\textsuperscript{(225.2)}
\textsuperscript{20:2.3} Los Avonales son los Hijos Paradisiacos que sirven y se donan en los planetas individuales de los universos locales. Y puesto que cada Hijo Avonal tiene una personalidad exclusiva, puesto que no hay dos de ellos que sean iguales, su trabajo es individualmente único en los reinos donde residen, en los cuales se encarnan a menudo en la similitud de la carne mortal y a veces nacen de madres terrestres en los mundos evolutivos.

\par
%\textsuperscript{(225.3)}
\textsuperscript{20:2.4} Además de sus servicios en los niveles administrativos superiores, los Avonales tienen una triple función en los mundos habitados:

\par
%\textsuperscript{(225.4)}
\textsuperscript{20:2.5} 1. \textit{Acciones judiciales.} Estos Hijos actúan al final de las dispensaciones planetarias. Con el tiempo se pueden ejecutar decenas ---o centenares--- de estas misiones en cada mundo individual, y pueden ir innumerables veces al mismo mundo o a otros mundos para poner fin a las dispensaciones, para liberar a los supervivientes dormidos.

\par
%\textsuperscript{(225.5)}
\textsuperscript{20:2.6} 2. \textit{Misiones magistrales.} Antes de la llegada de un Hijo donador se produce generalmente una visita planetaria de este tipo. En una misión así, un Avonal aparece como un adulto del planeta mediante una técnica de encarnación que no implica el nacimiento como mortal. Después de esta primera visita magistral habitual, los Avonales pueden servir repetidas veces en calidad magistral en el mismo planeta tanto antes como después de la aparición del Hijo donador. Durante estas misiones magistrales adicionales, un Avonal puede aparecer o no bajo la forma material y visible, pero en ninguna de ellas nacerá en el mundo como un bebé indefenso.

\par
%\textsuperscript{(225.6)}
\textsuperscript{20:2.7} 3. \textit{Misiones donadoras.} Todos los Hijos Avonales se donan al menos una vez a alguna raza mortal en algún mundo evolutivo. Las visitas judiciales son numerosas, las misiones magistrales pueden ser múltiples, pero en cada planeta sólo aparece un Hijo donador. Los Avonales donadores nacen de una mujer como Miguel de Nebadon se encarnó en Urantia.

\par
%\textsuperscript{(225.7)}
\textsuperscript{20:2.8} La cantidad de veces que los Hijos Avonales pueden servir en misiones magistrales y donadoras no tiene límites, pero por lo general, cuando han atravesado siete veces esta experiencia, se produce una suspensión a favor de aquellos que han efectuado menos este servicio. Estos Hijos con múltiples experiencias donadoras son destinados entonces al consejo personal superior de un Hijo Creador, llegando a participar así en la administración de los asuntos del universo local.

\par
%\textsuperscript{(225.8)}
\textsuperscript{20:2.9} En todo su trabajo para y en los mundos habitados, los Hijos Magistrales reciben la ayuda de dos órdenes de criaturas de los universos locales, los Melquisedeks y los arcángeles, mientras que durante las misiones donadoras también están acompañados por las Brillantes Estrellas Vespertinas, que tienen igualmente su origen en las creaciones locales. En todos sus esfuerzos planetarios, los Hijos Paradisiacos secundarios, los Avonales, reciben el apoyo de todo el poder y de toda la autoridad de un Hijo Paradisiaco primario, el Hijo Creador del universo local donde están sirviendo. A todos los efectos prácticos, su trabajo en las esferas habitadas es tan eficaz y aceptable como lo sería el servicio de un Hijo Creador en esos mundos habitados por los mortales.

\section*{3. Las acciones judiciales}
\par
%\textsuperscript{(226.1)}
\textsuperscript{20:3.1} A los Avonales se les conoce como Hijos Magistrales porque son los altos magistrados de los reinos, los jueces de las dispensaciones sucesivas de los mundos del tiempo. Presiden el despertar de los supervivientes dormidos, juzgan el reino, llevan a su fin una dispensación de justicia que estaba en suspenso, ejecutan los mandatos de una era de misericordia en período de prueba, reasignan las tareas de la nueva dispensación a las criaturas del espacio encargadas del ministerio planetario, y regresan a la sede de su universo local después de finalizar su misión.

\par
%\textsuperscript{(226.2)}
\textsuperscript{20:3.2} Cuando juzgan los destinos de una era, los Avonales decretan la suerte de las razas evolutivas, pero aunque pueden pronunciar sentencias que extinguen la identidad de las criaturas personales, no ejecutan dichas sentencias. Los veredictos de esta naturaleza son ejecutados exclusivamente por las autoridades de un superuniverso.

\par
%\textsuperscript{(226.3)}
\textsuperscript{20:3.3} La llegada de un Avonal Paradisiaco a un mundo evolutivo con el objeto de poner fin a una dispensación y de inaugurar una nueva era de progreso planetario no es necesariamente una misión magistral o una misión donadora. Las misiones magistrales son a veces encarnaciones, y las misiones donadoras lo son siempre, es decir, para estas tareas los Avonales sirven en un planeta con una forma material ---tangible. Sus otras visitas son <<\textit{técnicas}>>, y en dichos casos un Avonal no se encarna para el servicio planetario. Si un Hijo Magistral viene solamente como juez dispensacional, llega al planeta como un ser espiritual, invisible para las criaturas materiales del reino. Estas visitas técnicas se producen repetidas veces en la larga historia de un mundo habitado.

\par
%\textsuperscript{(226.4)}
\textsuperscript{20:3.4} Los Hijos Avonales pueden actuar como jueces planetarios antes de su experiencia magistral o de su experiencia donadora. Sin embargo, en cualquiera de estas misiones, el Hijo encarnado juzgará la era planetaria que termina; un Hijo Creador actúa del mismo modo cuando está encarnado en una misión de donación en la similitud de la carne mortal. Cuando un Hijo Paradisiaco visita un mundo evolutivo y se vuelve semejante a uno de sus habitantes, su presencia pone fin a una dispensación y representa un juicio del reino.

\section*{4. Las misiones magistrales}
\par
%\textsuperscript{(226.5)}
\textsuperscript{20:4.1} Antes de la aparición planetaria de un Hijo donador, un mundo habitado recibe generalmente la visita de un Avonal Paradisiaco en misión magistral. Si se trata de la primera visita magistral, el Avonal se encarna siempre como un ser material. Aparece en el planeta de su misión como un varón hecho y derecho de las razas mortales, como un ser plenamente visible para las criaturas mortales de su época y de su generación, y en contacto físico con ellas. Durante toda su encarnación magistral, el Hijo Avonal mantiene una conexión completa e ininterrumpida con las fuerzas espirituales locales y universales.

\par
%\textsuperscript{(226.6)}
\textsuperscript{20:4.2} Un planeta puede experimentar muchas visitas magistrales tanto antes como después de la aparición de un Hijo donador. Puede ser visitado muchas veces por el mismo Avonal o por otros Avonales, que actúan como jueces dispensacionales, pero estas misiones técnicas de juicio no son ni donadoras ni magistrales, y los Avonales nunca se encarnan en estas ocasiones. Incluso cuando un planeta es bendecido por repetidas misiones magistrales, los Avonales no se someten siempre a la encarnación mortal; y cuando sirven en la similitud de la carne mortal, siempre aparecen como seres adultos del reino; no nacen de mujer.

\par
%\textsuperscript{(227.1)}
\textsuperscript{20:4.3} Cuando están encarnados en sus misiones donadoras o magistrales, los Hijos Paradisiacos están provistos de Ajustadores experimentados, y estos Ajustadores son diferentes para cada encarnación. Los Ajustadores que ocupan la mente de los Hijos de Dios encarnados nunca pueden tener la esperanza de conseguir la personalidad a través de la fusión con los seres humano-divinos donde habitan, pero a menudo son personalizados por orden del Padre Universal. Estos Ajustadores forman el supremo consejo de dirección de Divinington encargado de administrar, identificar y enviar a los Monitores de Misterio a los reinos habitados. También reciben y acreditan a los Ajustadores que regresan al <<\textit{seno del Padre}>>\footnote{\textit{Seno del Padre}: Jn 1:18.} después de la disolución mortal de su tabernáculo terrestre. De esta manera, los fieles Ajustadores de los jueces del mundo se convierten en los jefes exaltados de su misma especie.

\par
%\textsuperscript{(227.2)}
\textsuperscript{20:4.4} Urantia no ha sido nunca la anfitriona de un Hijo Avonal en misión magistral. Si Urantia hubiera seguido el plan general de los mundos habitados, habría sido bendecida con una misión magistral en algún momento entre la época de Adán y la donación de Cristo Miguel. Pero la secuencia regular de los Hijos Paradisiacos en vuestro planeta fue totalmente perturbada por la aparición de vuestro Hijo Creador para llevar a cabo su donación final hace mil novecientos años.

\par
%\textsuperscript{(227.3)}
\textsuperscript{20:4.5} Urantia puede ser visitada todavía por un Avonal encargado de encarnarse en una misión magistral, pero en lo que se refiere a la aparición futura de los Hijos Paradisiacos, ni siquiera <<\textit{los ángeles del cielo conocen el momento o la manera de estas visitas}>>\footnote{\textit{Ni siquiera los ángeles conocen el momento}: Mt 24:36; Mc 13:32.}, porque el mundo donde se ha donado un Miguel se convierte en el pupilo individual y personal de un Hijo Maestro y, como tal, está totalmente sometido a sus propios planes y decisiones. En vuestro mundo el asunto se complica además debido a la promesa que hizo Miguel de regresar. Independientemente de los malentendidos acerca de la estancia urantiana de Miguel de Nebadon, una cosa es indudablemente auténtica ---su promesa de regresar a vuestro mundo\footnote{\textit{Promesa de regresar}: Mt 24:3-42; Mc 13:4-33; Lc 21:7-27; Jn 14:3.}. En vista de esta perspectiva, sólo el tiempo podrá revelar el orden futuro de las visitas de los Hijos Paradisiacos de Dios a Urantia.

\section*{5. La donación de los Hijos Paradisiacos de Dios}
\par
%\textsuperscript{(227.4)}
\textsuperscript{20:5.1} El Hijo Eterno es el Verbo eterno de Dios\footnote{\textit{El Hijo es la ``Palabra''}: Jn 1:1.}. El Hijo Eterno es la expresión perfecta del <<\textit{primer}>> pensamiento absoluto e infinito de su Padre eterno. Cuando un duplicado personal, o extensión divina, de este Hijo Original empieza una misión donadora de encarnación como mortal, se vuelve literalmente cierto que el divino <<\textit{Verbo se hace carne}>>\footnote{\textit{El Verbo se hizo carne}: Jn 1:14a; 1 Jn 1:1.} y que el Verbo habita así entre los seres humildes de origen animal.

\par
%\textsuperscript{(227.5)}
\textsuperscript{20:5.2} En Urantia existe la creencia muy difundida de que la finalidad de la donación de un Hijo es influir de alguna manera sobre la actitud del Padre Universal. Pero vuestra iluminación debería indicaros que esto no es verdad. Las donaciones de los Hijos Avonales y de los Hijos Migueles son una parte necesaria del proceso experiencial diseñado para hacer de estos Hijos unos magistrados y unos gobernantes compasivos y dignos de confianza para los habitantes y los planetas del tiempo y del espacio. La carrera de donación séptuple es la meta suprema de todos los Hijos Creadores Paradisiacos. Y todos los Hijos Magistrales están motivados por este mismo espíritu de servicio que caracteriza de manera tan abundante a los Hijos Creadores primarios y al Hijo Eterno del Paraíso.

\par
%\textsuperscript{(227.6)}
\textsuperscript{20:5.3} Hace falta que alguna orden de Hijos Paradisiacos se done en cada mundo habitado por los mortales con el objeto de hacer posible que los Ajustadores del Pensamiento habiten en la mente de todos los seres humanos normales de esa esfera, ya que los Ajustadores no vienen \textit{a todos} los seres humanos de buena fe hasta que el Espíritu de la Verdad ha sido derramado sobre toda carne; y el envío del Espíritu de la Verdad depende del regreso a su sede universal de un Hijo Paradisiaco que ha realizado con éxito una misión de donación como mortal en un mundo en evolución.

\par
%\textsuperscript{(228.1)}
\textsuperscript{20:5.4} En el transcurso de la larga historia de un planeta habitado tienen lugar numerosos juicios dispensacionales y puede producirse más de una misión magistral, pero un Hijo donador servirá normalmente una sola vez en la esfera. Sólo se requiere que cada mundo habitado tenga a un Hijo donador que venga a vivir la plena vida humana desde el nacimiento hasta la muerte. Tarde o temprano, independientemente de su estado espiritual, cada mundo habitado por los mortales está destinado a convertirse en el anfitrión de un Hijo Magistral en misión donadora, excepto el único planeta de cada universo local donde un Hijo Creador elige efectuar su donación como mortal.

\par
%\textsuperscript{(228.2)}
\textsuperscript{20:5.5} Al comprender más cosas sobre los Hijos donadores, podéis discernir por qué se concede tanto interés a Urantia en la historia de Nebadon. Vuestro pequeño e insignificante planeta es interesante para el universo local simplemente porque es el mundo del hogar terrenal de Jesús de Nazaret. Fue el escenario de la donación final y triunfante de vuestro Hijo Creador, el terreno donde Miguel consiguió la soberanía personal suprema sobre el universo de Nebadon.

\par
%\textsuperscript{(228.3)}
\textsuperscript{20:5.6} En la sede de su universo local, y especialmente después de terminar su propia donación como mortal, un Hijo Creador pasa una gran parte de su tiempo aconsejando e instruyendo al colegio de los Hijos asociados, los Hijos Magistrales y otros. Con amor y devoción, con una tierna misericordia y una afectuosa consideración, estos Hijos Magistrales se donan a los mundos del espacio. Estos servicios planetarios no son de ninguna manera inferiores a las donaciones como mortales de los Migueles\footnote{\textit{Donaciones de los Hijos Creadores}: Jn 1:1-5,14,18; 3:16-17.}. Es verdad que vuestro Hijo Creador eligió como escenario de su aventura final en la experiencia de las criaturas un mundo que había sufrido desgracias inhabituales. Pero ningún planeta podría encontrarse nunca en tales condiciones como para necesitar la donación de un Hijo Creador a fin de llevar a cabo su rehabilitación espiritual. Cualquier Hijo del grupo de donación bastaría igualmente, porque en todo su trabajo en los mundos de un universo local los Hijos Magistrales son tan divinamente eficaces y tan completamente sabios como lo sería su hermano Paradisiaco, el Hijo Creador.

\par
%\textsuperscript{(228.4)}
\textsuperscript{20:5.7} Aunque la posibilidad de un desastre acompa a siempre a estos Hijos Paradisiacos durante sus encarnaciones donadoras, estoy todavía por ver el informe de un fracaso o de un fallo en la misión de donación de un Hijo Magistral o Creador. Los dos tienen un origen demasiado cercano a la perfección absoluta como para fallar. En verdad asumen el riesgo, se vuelven realmente semejantes a las criaturas mortales de carne y hueso y adquieren así la experiencia única de la criatura, pero dentro del campo de mi observación, siempre tienen éxito. Nunca dejan de conseguir la meta de su misión donadora. El relato de sus servicios donadores y planetarios en todo Nebadon constituye el capítulo más noble y fascinante de la historia de vuestro universo local.

\section*{6. Las carreras de donación como mortales}
\par
%\textsuperscript{(228.5)}
\textsuperscript{20:6.1} El método por el cual un Hijo Paradisiaco se prepara para la encarnación humana como Hijo donador, entra en el seno de su madre en el planeta de la donación, es un misterio universal; y cualquier esfuerzo por detectar el funcionamiento de esta técnica de Sonarington está condenado a un fracaso seguro. Que el conocimiento sublime de la vida humana de Jesús de Nazaret se grabe en vuestra alma, pero no malgastéis vuestros pensamientos en especulaciones inútiles sobre cómo se llevó a cabo esta misteriosa encarnación de Miguel de Nebadon. Regocijémonos todos en el conocimiento y la seguridad de que estas proezas son posibles para la naturaleza divina y no perdamos el tiempo en conjeturas inútiles sobre la técnica empleada por la sabiduría divina para llevar a cabo estos fenómenos.

\par
%\textsuperscript{(229.1)}
\textsuperscript{20:6.2} En una misión de donación como mortal, un Hijo Paradisiaco nace siempre de mujer y crece como un niño varón del reino, tal como Jesús lo hizo en Urantia. Todos estos Hijos que efectúan este servicio supremo pasan de la infancia a la juventud y luego a la madurez exactamente igual que un ser humano. Se vuelven semejantes, en todos los aspectos, a los mortales de la raza en la que han nacido. Hacen peticiones al Padre como los hijos de los reinos en los que sirven. Desde el punto de vista material, estos Hijos humano-divinos viven una vida común y corriente, con una sola excepción: no engendran una descendencia en los mundos donde residen; se trata de una restricción universal impuesta a todas las órdenes de Hijos Paradisiacos donadores.

\par
%\textsuperscript{(229.2)}
\textsuperscript{20:6.3} Al igual que Jesús trabajó en vuestro mundo como hijo del carpintero, otros Hijos Paradisiacos trabajan en diversas capacidades en los planetas de su donación. Difícilmente podríais imaginar una profesión que no haya sido ejercida por algún Hijo Paradisiaco en el transcurso de su donación en uno de los planetas evolutivos del tiempo.

\par
%\textsuperscript{(229.3)}
\textsuperscript{20:6.4} Cuando un Hijo donador ha dominado la experiencia de vivir la vida como mortal, cuando ha conseguido sintonizarse perfectamente con su Ajustador interior, inmediatamente después empieza la parte de su misión planetaria destinada a iluminar la mente y a inspirar el alma de sus hermanos en la carne. Como instructores, estos Hijos se dedican exclusivamente a la iluminación espiritual de las razas mortales en los mundos donde residen.

\par
%\textsuperscript{(229.4)}
\textsuperscript{20:6.5} Aunque las carreras de donación como mortales de los Migueles y de los Avonales son comparables en la mayor parte de sus aspectos, no son idénticas en todos ellos: un Hijo Magistral no proclama nunca <<\textit{Aquel que ha visto al Hijo ha visto al Padre}>>\footnote{\textit{Aquel que ha visto al Hijo ha visto al Padre}: Jn 12:45; 14:9.}, como lo hizo vuestro Hijo Creador cuando estuvo encarnado en Urantia. Pero un Avonal donador sí declara <<\textit{Aquel que me ha visto ha visto al Hijo Eterno de Dios}>>. Los Hijos Magistrales no descienden directamente del Padre Universal, ni tampoco se encarnan sometiéndose a la voluntad del Padre; siempre se donan como \textit{Hijos} Paradisiacos sometidos a la voluntad del Hijo Eterno del Paraíso.

\par
%\textsuperscript{(229.5)}
\textsuperscript{20:6.6} Cuando los Hijos donadores, Creadores o Magistrales, atraviesan las puertas de la muerte, reaparecen al tercer día. Pero no deberíais albergar la idea de que siempre sufren el trágico final que encontró el Hijo Creador que residió en vuestro mundo hace mil novecientos años. La experiencia extraordinaria y excepcionalmente cruel por la que pasó Jesús de Nazaret ha hecho que Urantia sea conocida localmente como <<\textit{el mundo de la cruz}>>. No es necesario que a un Hijo de Dios le inflijan un tratamiento tan inhumano, y la gran mayoría de los planetas les ha concedido un recibimiento más considerado, permitiéndoles terminar su carrera humana, poner fin a la era, juzgar a los supervivientes dormidos e inaugurar una nueva dispensación, sin imponerles una muerte violenta. Un Hijo donador debe enfrentarse a la muerte, debe pasar por toda la experiencia efectiva de los mortales del reino, pero el plan divino no contempla el requisito de que esta muerte sea violenta o fuera de lo normal.

\par
%\textsuperscript{(229.6)}
\textsuperscript{20:6.7} Cuando a los Hijos donadores no les quitan la vida de manera violenta, renuncian voluntariamente a su vida y pasan por las puertas de la muerte, no para satisfacer las exigencias de una <<\textit{justicia severa}>>\footnote{\textit{``Justicia severa''}: Lv 26:13-39.} o de una <<\textit{cólera divina}>>\footnote{\textit{``Cólera divina''}: Ex 15:7.}, sino más bien para finalizar la donación, para <<\textit{beber la copa}>>\footnote{\textit{Beber la copa}: Mt 20:22-23; 26:39; Mc 10:38-39; 14:36; Lc 22:42; Jn 18:11.} de la carrera de la encarnación y de la experiencia personal en todo lo que constituye la vida de una criatura tal como ésta se vive en los planetas de la existencia mortal. La donación es una necesidad planetaria y universal, y la muerte física no es nada más que una parte necesaria de una misión donadora.

\par
%\textsuperscript{(230.1)}
\textsuperscript{20:6.8} Cuando su encarnación como mortal ha terminado, el Avonal que ha realizado este servicio se dirige al Paraíso, es aceptado por el Padre Universal, regresa al universo local donde está destinado y recibe el reconocimiento del Hijo Creador. Inmediatamente después, el Avonal donador y el Hijo Creador envían su Espíritu de la Verdad\footnote{\textit{Espíritu de la Verdad}: Ez 11:19; 18:31; 36:26-27; Jl 2:28-29; Lc 24:49; Jn 7:39; 14:16-18,23,26; 15:4,26; 16:7-8,13-14; 17:21-23; Hch 1:5,8a; 2:1-4,16-18; 2:33; 2 Co 13:5; Gl 2:20; 4:6; Ef 1:13; 4:30; 1 Jn 4:12-15.} conjunto para que ejerza su actividad en el corazón de las razas mortales que viven en el mundo de la donación. En las eras de un universo local anteriores a la soberanía, se trata del espíritu conjunto de los dos Hijos, puesto en ejecución por el Espíritu Creativo. Difiere un poco del Espíritu de la Verdad que caracteriza a las eras del universo local posteriores a la séptima donación de un Miguel.

\par
%\textsuperscript{(230.2)}
\textsuperscript{20:6.9} Cuando un Hijo Creador ha terminado su donación final, el Espíritu de la Verdad que había sido enviado previamente a todos los mundos de ese universo local donde se había donado un Avonal, cambia de naturaleza y se vuelve más literalmente el espíritu del soberano Miguel. Este fenómeno se produce simultáneamente con la liberación del Espíritu de la Verdad que es enviado a servir en el planeta de la donación humana del Miguel. Más tarde, cada mundo honrado con una donación Magistral recibirá del Hijo Creador séptuple, en asociación con el Hijo Magistral, el mismo Consolador espiritual que habría recibido si el Soberano del universo local se hubiera encarnado personalmente como Hijo donador.

\section*{7. Los Hijos Instructores Trinitarios}
\par
%\textsuperscript{(230.3)}
\textsuperscript{20:7.1} Estos Hijos Paradisiacos extremadamente personales y espirituales son traídos a la existencia por la Trinidad del Paraíso. Son conocidos en Havona como la orden de los Daynales. En Orvonton están registrados como Hijos Instructores Trinitarios, llamados así a causa de su origen. En Salvington a veces se les denomina Hijos Espirituales Paradisiacos.

\par
%\textsuperscript{(230.4)}
\textsuperscript{20:7.2} El número de Hijos Instructores aumenta constantemente. El último censo universal transmitido indicaba que el número de estos Hijos Trinitarios que ejercen su actividad en el universo central y en los superuniversos ascendía a un poco más de veintiún mil millones, excluyendo a las reservas que están en el Paraíso, las cuales incluyen a más de un tercio de todos los Hijos Instructores Trinitarios que existen.

\par
%\textsuperscript{(230.5)}
\textsuperscript{20:7.3} La orden de filiación de los Daynales no es una parte orgánica de las adminis-traciones de los universos locales o de los superuniversos. Sus miembros no son ni creadores ni rehabilitadores, ni jueces ni gobernantes. No se ocupan tanto de la administración universal como de la iluminación moral y del desarrollo espiritual. Son los educadores universales, y están dedicados al despertar espiritual y a la orientación moral de todos los reinos. Su ministerio está íntimamente interrelacionado con el de las personalidades del Espíritu Infinito y estrechamente asociado con la ascensión de las criaturas al Paraíso.

\par
%\textsuperscript{(230.6)}
\textsuperscript{20:7.4} Estos Hijos de la Trinidad comparten la naturaleza combinada de las tres Deidades del Paraíso, pero en Havona parecen reflejar más la naturaleza del Padre Universal. En los superuniversos parecen describir la naturaleza del Hijo Eterno, mientras que en las creaciones locales parecen manifestar el carácter del Espíritu Infinito. Son la personificación del servicio y la prudencia de la sabiduría en todos los universos.

\par
%\textsuperscript{(230.7)}
\textsuperscript{20:7.5} A diferencia de sus hermanos Migueles y Avonales del Paraíso, los Hijos Instructores Trinitarios no reciben ningún entrenamiento preliminar en el universo central. Son enviados directamente a las sedes de los superuniversos y desde allí se les destina a servir en algún universo local. En su ministerio hacia esos reinos evolutivos utilizan la influencia espiritual combinada de un Hijo Creador y de los Hijos Magistrales asociados, ya que los Daynales no poseen un poder de atracción espiritual en sí mismos y por sí mismos.

\section*{8. El ministerio de los Daynales en los universos locales}
\par
%\textsuperscript{(231.1)}
\textsuperscript{20:8.1} Los Hijos Espirituales Paradisiacos son unos seres incomparables de origen trinitario y las únicas criaturas de la Trinidad que están completamente asociadas a la dirección de los universos de origen doble. Se dedican afectuosamente al ministerio educativo de las criaturas mortales y de las órdenes inferiores de seres espirituales. Empiezan su trabajo en los sistemas locales y, de acuerdo con su experiencia y sus logros, progresan hacia el interior a través del servicio en las constelaciones hasta las tareas más elevadas de la creación local. Después de recibir sus certificados, pueden convertirse en embajadores espirituales y representar a los universos locales donde sirven.

\par
%\textsuperscript{(231.2)}
\textsuperscript{20:8.2} No conozco el número exacto de Hijos Instructores que hay en Nebadon; hay muchos miles de ellos. Muchos jefes de departamento de las escuelas Melquisedeks pertenecen a esta orden, mientras que el personal combinado de la Universidad regularmente constituida de Salvington engloba a más de cien mil personas, incluyendo a estos Hijos. Un gran número de ellos están estacionados en los diversos mundos educativos morontiales, pero no se ocupan enteramente del progreso espiritual e intelectual de las criaturas mortales; también están relacionados con la instrucción de los seres seráficos y de otros nativos de las creaciones locales. Muchos de sus ayudantes proceden de las filas de los seres trinitizados por las criaturas.

\par
%\textsuperscript{(231.3)}
\textsuperscript{20:8.3} Los Hijos Instructores componen el cuerpo docente que efectúa todos los exámenes y dirige todas las pruebas para obtener la calificación y la certificación en todas las fases subordinadas del servicio universal, desde las funciones de los centinelas de los puestos avanzados hasta las de los estudiantes de estrellas. Dirigen un programa secular de formación que se extiende desde los cursos planetarios hasta el Colegio superior de Sabiduría situado en Salvington. A todos los que finalizan estas aventuras en la sabiduría y la verdad, ya se trate de mortales ascendentes o de querubines ambiciosos, se les concede un reconocimiento por sus esfuerzos y sus logros.

\par
%\textsuperscript{(231.4)}
\textsuperscript{20:8.4} En todos los universos, todos los Hijos de Dios están agradecidos a estos Hijos Instructores Trinitarios siempre fieles y universalmente eficaces. Son los educadores exaltados de todas las personalidades espirituales, e incluso los auténticos educadores probados de los Hijos de Dios mismos. Pero difícilmente puedo informaros sobre los detalles interminables de los deberes y funciones de los Hijos Instructores. El inmenso campo de actividad de la filiación Daynal será mejor comprendido en Urantia cuando hayáis progresado más en inteligencia y después de que el aislamiento espiritual de vuestro planeta haya terminado.

\section*{9. El servicio planetario de los Daynales}
\par
%\textsuperscript{(231.5)}
\textsuperscript{20:9.1} Cuando el progreso de los acontecimientos en un mundo evolutivo indica que ha llegado el momento oportuno de iniciar una era espiritual, los Hijos Instructores Trinitarios se ofrecen siempre como voluntarios para este servicio. No estáis familiarizados con esta orden de filiación porque Urantia no ha experimentado nunca una era espiritual, un milenio de iluminación cósmica. Pero los Hijos Instructores están ya visitando vuestro mundo con el objeto de formular los planes relacionados con su proyecto de residir en vuestra esfera. Deberán aparecer en Urantia después de que sus habitantes se hayan liberado relativamente de las trabas del animalismo y de las cadenas del materialismo.

\par
%\textsuperscript{(231.6)}
\textsuperscript{20:9.2} Los Hijos Instructores Trinitarios no tienen nada que ver con la terminación de las dispensaciones planetarias. No juzgan a los muertos ni trasladan a los vivos, pero en cada misión planetaria vienen acompañados de un Hijo Magistral que realiza estos servicios. Los Hijos Instructores se ocupan enteramente del inicio de una era espiritual, del amanecer de la era de las realidades espirituales en un planeta evolutivo. Hacen realidad las contrapartidas espirituales del conocimiento material y de la sabiduría temporal.

\par
%\textsuperscript{(232.1)}
\textsuperscript{20:9.3} Los Hijos Instructores permanecen generalmente en los planetas que visitan durante mil años del tiempo planetario. Un Hijo Instructor preside el reinado milenario planetario y recibe la ayuda de setenta asociados de su orden. Los Daynales no se encarnan ni se materializan de otras maneras para ser visibles a los seres mortales; el contacto con el mundo que visitan se mantiene pues a través de las actividades de las Brillantes Estrellas Vespertinas, unas personalidades del universo local que están asociadas con los Hijos Instructores Trinitarios.

\par
%\textsuperscript{(232.2)}
\textsuperscript{20:9.4} Los Daynales pueden regresar muchas veces a un mundo habitado, y después de su misión final, el planeta entrará en el estado establecido de una esfera de luz y de vida, la meta evolutiva de todos los mundos habitados por mortales en la era actual del universo. El Cuerpo de los Mortales de la Finalidad tiene mucho que ver con las esferas establecidas en la luz y la vida, y sus actividades planetarias están en contacto con las de los Hijos Instructores. En verdad, toda la orden de filiación Daynal está íntimamente enlazada con todas las fases de actividad de los finalitarios en las creaciones evolutivas del tiempo y del espacio.

\par
%\textsuperscript{(232.3)}
\textsuperscript{20:9.5} Durante las etapas iniciales de las ascensión evolutiva, los Hijos Instructores Trinitarios parecen estar tan completamente identificados con el régimen de la progresión mortal que a menudo nos vemos inducidos a especular sobre su posible asociación con los finalitarios en la carrera no revelada de los universos futuros. Observamos que los administradores de los superuniversos son, en parte, personalidades de origen trinitario y, en parte, criaturas evolutivas ascendentes abrazadas por la Trinidad. Creemos firmemente que los Hijos Instructores y los finalitarios están dedicados ahora a adquirir la experiencia de estar asociados en el tiempo, lo cual podría ser un entrenamiento preliminar a fin de prepararlos para una estrecha asociación en algún destino futuro no revelado. En Uversa creemos que cuando los superuniversos se establezcan finalmente en la luz y la vida, estos Hijos Instructores Paradisiacos, que se habrán familiarizado tan profundamente con los problemas de los mundos evolutivos y que habrán estado asociados durante tanto tiempo con la carrera de los mortales evolutivos, pasarán a tener probablemente una asociación eterna con el Cuerpo Paradisiaco de la Finalidad.

\section*{10. El ministerio unido de los Hijos Paradisiacos}
\par
%\textsuperscript{(232.4)}
\textsuperscript{20:10.1} Todos los Hijos Paradisiacos de Dios son de origen y de naturaleza divinos. El trabajo de cada Hijo Paradisiaco en favor de cada mundo es exactamente como si el Hijo que realiza ese servicio fuera el primero y el único Hijo de Dios\footnote{\textit{Hijo unigénito}: Sal 2:7; Jn 1:14,18; 3:16,18; Hch 13:33; Heb 1:5; 5:5; 1 Jn 4:9.}.

\par
%\textsuperscript{(232.5)}
\textsuperscript{20:10.2} Los Hijos Paradisiacos son la presentación divina de las naturalezas en activo de las tres personas de la Deidad a los dominios del tiempo y del espacio. Los Hijos Creadores, Magistrales e Instructores son los dones de las Deidades eternas a los hijos de los hombres y a todas las otras criaturas del universo dotadas del potencial de ascensión. Estos Hijos de Dios son los ministros divinos que se consagran sin cesar a la tarea de ayudar a las criaturas del tiempo a alcanzar la elevada meta espiritual de la eternidad.

\par
%\textsuperscript{(232.6)}
\textsuperscript{20:10.3} En los Hijos Creadores, el amor del Padre Universal se mezcla con la misericordia del Hijo Eterno y se revela\footnote{\textit{Los Hijos Creadores revelan al Padre}: Mt 5:45-48; 6:1,4,6; 11:25-27; Mc 11:25-26; Lc 6:35-36; 10:22; Jn 1:18; 3:31-34; 4:21-24; 6:45-46; 10:36-38; 14:6-11,20; 15:15; 16:25; 17:8,25-26.} a los universos locales en el poder creativo, el ministerio amoroso y la soberanía comprensiva de los Migueles. En los Hijos Magistrales, la misericordia del Hijo Eterno, unida al ministerio del Espíritu Infinito, se revela a los dominios evolutivos en las carreras de estos Avonales que juzgan, sirven y se donan. En los Hijos Instructores Trinitarios, el amor, la misericordia y el ministerio de las tres Deidades del Paraíso están coordinados en los niveles de valor más elevados del espacio-tiempo, y son presentados a los universos como la verdad viviente, la bondad divina y la verdadera belleza espiritual.

\par
%\textsuperscript{(233.1)}
\textsuperscript{20:10.4} En los universos locales, estas órdenes de filiación colaboran para llevar a cabo la revelación de las Deidades del Paraíso a las criaturas del espacio\footnote{\textit{Los Hijos revelan a las Deidades}: Jn 1:18.}. Como Padre de un universo local, un Hijo Creador muestra el carácter infinito del Padre Universal. Como Hijos donadores misericordiosos, los Avonales revelan la naturaleza incomparable del Hijo Eterno que está lleno de compasión infinita. Como verdaderos educadores de las personalidades ascendentes, los Hijos Daynales Trinitarios revelan la personalidad educadora del Espíritu Infinito. Gracias a su cooperación divinamente perfecta, los Migueles, los Avonales y los Daynales contribuyen a revelar y a hacer realidad la personalidad y la soberanía de Dios Supremo en y para los universos del espacio-tiempo. Gracias a la armonía de sus actividades trinas, estos Hijos Paradisiacos de Dios ejercen siempre su actividad en la vanguardia de las personalidades de la Deidad a medida que siguen la expansión interminable de la divinidad de la Gran Fuente-Centro Primera desde la Isla eterna del Paraíso hasta las profundidades desconocidas del espacio.

\par
%\textsuperscript{(233.2)}
\textsuperscript{20:10.5} [Presentado por un Perfeccionador de la Sabiduría de Uversa.]


\chapter{Documento 21. Los Hijos Creadores Paradisiacos}
\par
%\textsuperscript{(234.1)}
\textsuperscript{21:0.1} LOS Hijos Creadores\footnote{\textit{Lod hijos Creadores}: Sal 33:6; 102:25; Is 45:12,18; Jn 1:1-3; Ef 2:10; 3:9; Col 1:16; Heb 1:2,10; Ap 4:11.} son los constructores y gobernantes de los universos locales del tiempo y del espacio. Estos creadores y soberanos universales tienen un origen doble, personificando las características de Dios Padre y de Dios Hijo. Pero cada Hijo Creador es diferente a todos los demás; la naturaleza de cada uno de ellos es única así como su personalidad; cada uno es el <<\textit{Hijo unigénito}>>\footnote{\textit{Hijo unigénito}: Sal 2:7; Jn 1:14,18; 3:16,18; Hch 13:33; Heb 1:5; 5:5; 1 Jn 4:9.} del ideal perfecto de deidad que le dio origen.

\par
%\textsuperscript{(234.2)}
\textsuperscript{21:0.2} En la inmensa tarea de organizar, desarrollar y perfeccionar un universo local, estos Hijos elevados disfrutan siempre de la aprobación sustentadora del Padre Universal. La relación de los Hijos Creadores con su Padre Paradisiaco es conmovedora y suprema. No hay duda de que el afecto profundo de las Deidades-padres por su progenie divina es la fuente de ese amor hermoso y casi divino que incluso los padres mortales tienen por sus hijos.

\par
%\textsuperscript{(234.3)}
\textsuperscript{21:0.3} Estos Hijos Paradisiacos primarios son personalizados como Migueles. Cuando salen del Paraíso para fundar sus universos, son conocidos como Migueles Creadores. Cuando están establecidos en la autoridad suprema se les llama Migueles Maestros. A veces nos referimos al soberano de vuestro universo de Nebadon como Cristo Miguel. Reinan siempre y para siempre según la <<\textit{orden de Miguel}>>, pues así se denomina el primer Hijo de su orden y de su naturaleza.

\par
%\textsuperscript{(234.4)}
\textsuperscript{21:0.4} El Miguel original o primogénito no ha experimentado nunca la encarnación como ser material, pero pasó siete veces por la experiencia de la ascensión espiritual de las criaturas en los siete circuitos de Havona, avanzando desde las esferas exteriores hasta el circuito más interior de la creación central. La orden de los Migueles conoce el gran universo de un extremo al otro; no existe ninguna experiencia esencial por la que haya pasado un hijo cualquiera del tiempo y del espacio en la que los Migueles no hayan participado personalmente; comparten de hecho no solamente la naturaleza divina sino también vuestra naturaleza, es decir todas las naturalezas, desde las más elevadas hasta las más humildes.

\par
%\textsuperscript{(234.5)}
\textsuperscript{21:0.5} El Miguel original es el jefe que preside los Hijos Paradisiacos primarios cuando éstos se reúnen para conferenciar en el centro de todas las cosas. No hace mucho tiempo que recibimos en Uversa la transmisión universal de un cónclave extraordinario de ciento cincuenta mil Hijos Creadores, reunidos en la Isla eterna en presencia de sus progenitores, y ocupados en unas deliberaciones que tenían que ver con el progreso de la unificación y la estabilización del universo de universos. Se trataba de un grupo selecto de Migueles Soberanos, de Hijos que se han donado siete veces.

\section*{1. Origen y naturaleza de los Hijos Creadores}
\par
%\textsuperscript{(234.6)}
\textsuperscript{21:1.1} Cuando la plenitud de una ideación espiritual absoluta en el Hijo Eterno se encuentra con la plenitud de un concepto absoluto de personalidad en el Padre Universal, cuando esta unión creativa se consigue de manera plena y final, cuando tienen lugar esta identidad absoluta de espíritu y esta unidad infinita de concepto de la personalidad, entonces, en ese mismo instante y sin que ninguna de las Deidades infinitas pierda nada de su personalidad o de sus prerrogativas, un nuevo Hijo Creador original en plena posesión de sus capacidades surge como un relámpago a la existencia, el Hijo unigénito del ideal perfecto y de la idea poderosa cuya unión produce esta nueva personalidad creadora dotada de poder y de perfección.

\par
%\textsuperscript{(235.1)}
\textsuperscript{21:1.2} Cada Hijo Creador es el descendiente unigénito\footnote{\textit{Hijos unigénitos de Dios}: Sal 2:7; Jn 1:14,18; 3:16,18; Hch 13:33; Heb 1:5; 5:5; 1 Jn 4:9.}, y el único engendrable, de la unión perfecta entre los conceptos originales de las dos mentes infinitas, eternas y perfectas de los Creadores eternos del universo de universos. Nunca puede existir otro Hijo semejante, porque cada Hijo Creador es la expresión y la personificación incalificadas, completas y finales de la totalidad de cada fase de cada característica de cada posibilidad de cada realidad divina que en toda la eternidad se podrá encontrar nunca en, expresarse a través de, o desarrollarse a partir de, estos potenciales creativos divinos que se unieron para traer a la existencia a este Hijo Miguel. Cada Hijo Creador es el absoluto de los conceptos divinos unidos que constituyen su origen divino.

\par
%\textsuperscript{(235.2)}
\textsuperscript{21:1.3} En principio, la naturaleza divina de estos Hijos Creadores se deriva por igual de los atributos de sus dos padres paradisiacos. Todos comparten la plenitud de la naturaleza divina del Padre Universal y las prerrogativas creadoras del Hijo Eterno, pero a medida que observamos las manifestaciones prácticas de las actividades de los Migueles en los universos, discernimos diferencias aparentes. Algunos Hijos Creadores parecen ser más semejantes a Dios Padre; otros se parecen más a Dios Hijo. Por ejemplo: la tendencia de la administración en el universo de Nebadon sugiere que su Hijo Creador y gobernante posee una naturaleza y un carácter que se parecen más a los del Hijo Madre Eterno. Debemos indicar además que algunos universos están presididos por Migueles Paradisiacos que parecen asemejarse tanto a Dios Padre como a Dios Hijo. Y estas observaciones no implican una crítica en ningún sentido; se trata simplemente de la constatación de un hecho.

\par
%\textsuperscript{(235.3)}
\textsuperscript{21:1.4} No conozco el número exacto de Hijos Creadores que existen, pero tengo buenas razones para creer que hay más de setecientos mil. Ahora bien, sabemos que hay exactamente setecientos mil Uniones de los Días y que ya no se crea ninguno más. También observamos que los planes ordenados para la presente era del universo parecen indicar que un Unión de los Días deberá estar estacionado en cada universo local como consejero embajador de la Trinidad. Observamos además que el número constantemente creciente de Hijos Creadores sobrepasa ya el número fijo de Uniones de los Días. Pero nunca nos han informado sobre el destino de los Migueles que están más allá de los setecientos mil.

\section*{2. Los Creadores de los universos locales}
\par
%\textsuperscript{(235.4)}
\textsuperscript{21:2.1} Los Hijos Paradisiacos de la orden primaria son los diseñadores, creadores\footnote{\textit{Los Hijos Creadores crean}: Sal 33:6; 102:25; Is 45:12,18; Jn 1:1-3; Ef 2:10; 3:9; Col 1:16; Ap 4:11.}, constructores y administradores de sus dominios respectivos, los universos locales del tiempo y del espacio, las unidades creativas básicas de los siete superuniversos evolutivos. A un Hijo Creador se le permite elegir el lugar espacial de su futura actividad cósmica, pero antes de que pueda empezar siquiera la organización física de su universo, debe pasar por un largo período de observación dedicado al estudio de los esfuerzos de sus hermanos mayores en las diversas creaciones situadas en el superuniverso donde tiene el proyecto de actuar. Y antes de todo esto, el Hijo Miguel habrá finalizado su larga experiencia sin igual como observador en el Paraíso y de entrenamiento en Havona.

\par
%\textsuperscript{(235.5)}
\textsuperscript{21:2.2} Cuando un Hijo Creador parte del Paraíso para emprender la aventura de construir un universo, para convertirse en el jefe ---prácticamente en el Dios--- del universo local que él mismo va a organizar, entonces se encuentra por primera vez en contacto íntimo con la Fuente-Centro Tercera y dependiente de ella en muchos aspectos. Aunque el Espíritu Infinito reside con el Padre y el Hijo en el centro de todas las cosas, está destinado a actuar como colaborador real y efectivo de cada Hijo Creador. Por eso cada Hijo Creador está acompañado de una Hija Creativa del Espíritu Infinito, ese ser destinado a convertirse en la Ministra Divina, en el Espíritu Madre del nuevo universo local.

\par
%\textsuperscript{(236.1)}
\textsuperscript{21:2.3} En esta ocasión, la partida de un Hijo Miguel libera para siempre sus prerrogativas creadoras de su vinculación con las Fuentes y Centros Paradisiacos, permaneciendo sometidas únicamente a ciertas limitaciones inherentes a la preexistencia de estas Fuentes y Centros y a otros determinados poderes y presencias anteriores. Entre las limitaciones a las prerrogativas creadoras, por otra parte todopoderosas, del Padre de un universo local, podemos citar las siguientes:

\par
%\textsuperscript{(236.2)}
\textsuperscript{21:2.4} 1. \textit{La energía-materia} está dominada por el Espíritu Infinito. Antes de que se puedan crear nuevas formas de cosas, grandes o pequeñas, antes de que se pueda intentar cualquier nueva transformación de la energía-materia, un Hijo Creador debe asegurarse el consentimiento y la cooperación activa del Espíritu Infinito.

\par
%\textsuperscript{(236.3)}
\textsuperscript{21:2.5} 2. \textit{Los diseños y los tipos de criaturas} están controlados por el Hijo Eterno. Antes de que un Hijo Creador pueda emprender la creación de cualquier nuevo tipo de ser, de cualquier nuevo diseño de criatura, debe asegurarse el consentimiento del Hijo Madre Original y Eterno.

\par
%\textsuperscript{(236.4)}
\textsuperscript{21:2.6} 3. \textit{La personalidad} es concebida y otorgada por el Padre Universal.

\par
%\textsuperscript{(236.5)}
\textsuperscript{21:2.7} Los tipos y arquetipos de \textit{mentes} están determinados por los factores del ser anteriores a la criatura. Después de que estos factores han sido asociados para formar una criatura (personal u otra), la mente es el don de la Fuente-Centro Tercera, la fuente universal del ministerio de la mente para todos los seres que se encuentran por debajo del nivel de los Creadores Paradisiacos.

\par
%\textsuperscript{(236.6)}
\textsuperscript{21:2.8} El control de los diseños y de los tipos de \textit{espíritus} depende del nivel de su manifestación. A fin de cuentas, el diseño espiritual está controlado por la Trinidad o por las dotaciones espirituales pretrinitarias de las personalidades de la Trinidad ---el Padre, el Hijo y el Espíritu\footnote{\textit{Personalidades de la Trinidad}: Mt 28:19; Hch 2:32-33; 2 Co 13:14; 1 Jn 5:7. \textit{Concepción antigua de la Trinidad}: 1 Co 12:4-6.}.

\par
%\textsuperscript{(236.7)}
\textsuperscript{21:2.9} Cuando ese Hijo perfecto y divino ha tomado posesión del escenario espacial que ha elegido para su universo; cuando los problemas iniciales de la materialización del universo y del equilibrio general han sido resueltos; cuando ha formado una unión de trabajo eficaz y cooperativa con su complementaria, la Hija del Espíritu Infinito ---entonces ese Hijo Universal y ese Espíritu Universal inician el enlace destinado a dar origen a las innumerables multitudes de hijos de su universo local. En conexión con este acontecimiento, el Espíritu Creativo, focalización del Espíritu Infinito Paradisiaco, cambia de naturaleza, adquiriendo las cualidades personales del Espíritu Madre de un universo local.

\par
%\textsuperscript{(236.8)}
\textsuperscript{21:2.10} A pesar de que todos los Hijos Creadores son divinamente semejantes a sus padres Paradisiacos, ninguno se parece exactamente a otro; la \textit{naturaleza} así como la personalidad de cada uno de ellos es única, distinta, exclusiva y original. Y puesto que son los arquitectos y los autores de los planes para la vida de sus reinos respectivos, esta misma diversidad asegura que sus dominios serán también diferentes en todas las formas y fases de existencias vivientes, derivadas de los Migueles, que puedan crearse o evolucionar posteriormente allí. En consecuencia, las órdenes de criaturas nativas de los universos locales son muy variadas. No existen dos universos que estén administrados o habitados por seres nativos de origen doble que sean idénticos en todos los aspectos. Dentro de cualquier superuniverso, la mitad de sus atributos inherentes es bastante semejante, pues procede de los Espíritus Creativos uniformes; la otra mitad es diferente, pues proviene de los diversos Hijos Creadores. Pero esta diversidad no caracteriza a aquellas criaturas que tienen su origen exclusivo en el Espíritu Creativo, ni a aquellos seres importados que han nacido en el universo central o en los superuniversos.

\par
%\textsuperscript{(237.1)}
\textsuperscript{21:2.11} Cuando un Hijo Miguel está ausente de su universo, su gobierno es dirigido por el ser nativo primogénito, la Radiante Estrella Matutina\footnote{\textit{Radiante Estrella Matutina}: Ap 22:16.}, el jefe ejecutivo del universo local. El consejo y el asesoramiento del Unión de los Días es inapreciable en esos momentos. Durante estas ausencias, un Hijo Creador puede conferir al Espíritu Madre asociado el supercontrol de su presencia espiritual en los mundos habitados y en los corazones de sus hijos mortales. El Espíritu Madre de un universo local permanece siempre en su sede central, desde donde extiende sus cuidados protectores y su ministerio espiritual hasta las zonas más alejadas de ese dominio evolutivo.

\par
%\textsuperscript{(237.2)}
\textsuperscript{21:2.12} La presencia personal de un Hijo Creador en su universo local no es necesaria para que esa creación material establecida funcione de manera ordenada. Estos Hijos pueden viajar al Paraíso, y aún así sus universos continuarán dando vueltas en el espacio. Pueden dejar a un lado sus posiciones de poder para encarnarse como hijos del tiempo; y aún así sus reinos continuarán girando alrededor de sus centros respectivos. Ninguna organización material es independiente de la atracción de la gravedad absoluta del Paraíso ni del supercontrol cósmico inherente a la presencia espacial del Absoluto Incalificado.

\section*{3. La soberanía de un universo local}
\par
%\textsuperscript{(237.3)}
\textsuperscript{21:3.1} Un Hijo Creador recibe el campo de actividad de un universo con el consentimiento de la Trinidad del Paraíso y con la confirmación del Espíritu Maestro que supervisa el superuniverso interesado. Esta acción constituye un título de propiedad física, un arrendamiento cósmico. Pero la elevación de un Hijo Miguel, desde esta etapa de gobierno inicial y limitada por su propia voluntad hasta la supremacía experiencial de una soberanía ganada por sí mismo, llega como resultado de sus propias experiencias personales durante la tarea de crear un universo y de donarse de forma encarnada. Hasta que consigue una soberanía ganada mediante sus donaciones, gobierna como vicegerente del Padre Universal.

\par
%\textsuperscript{(237.4)}
\textsuperscript{21:3.2} Un Hijo Creador podría imponer su plena soberanía sobre su creación personal en cualquier momento, pero elige sabiamente no hacerlo. Si antes de pasar por sus donaciones como criatura asumiera una soberanía suprema no ganada, las personalidades paradisiacas residentes en su universo local se retirarían. Pero esto no ha sucedido nunca en ninguna de las creaciones del tiempo y del espacio.

\par
%\textsuperscript{(237.5)}
\textsuperscript{21:3.3} El hecho de poseer la facultad de crear implica la plenitud de la soberanía, pero los Migueles eligen \textit{ganarla} por experiencia, conservando así la plena cooperación de todas las personalidades del Paraíso vinculadas a la administración del universo local. No conocemos a ningún Miguel que haya actuado de otra manera; pero todos podrían haberlo hecho, pues son unos Hijos dotados realmente de libre albedrío.

\par
%\textsuperscript{(237.6)}
\textsuperscript{21:3.4} La soberanía de un Hijo Creador en un universo local pasa por seis, o quizás siete, etapas de manifestación experiencial, que aparecen en el orden siguiente:

\par
%\textsuperscript{(237.7)}
\textsuperscript{21:3.5} 1. La soberanía inicial como vicegerente ---la autoridad provisional solitaria que ejerce un Hijo Creador antes de que el Espíritu Creativo asociado adquiera las cualidades de la personalidad.

\par
%\textsuperscript{(237.8)}
\textsuperscript{21:3.6} 2. La soberanía conjunta como vicegerentes ---el gobierno conjunto de la pareja paradisiaca después de que el Espíritu Madre Universal ha conseguido la personalidad.

\par
%\textsuperscript{(238.1)}
\textsuperscript{21:3.7} 3. La soberanía creciente como vicegerente ---la autoridad progresiva de un Hijo Creador durante el período de sus siete donaciones bajo la forma de sus criaturas.

\par
%\textsuperscript{(238.2)}
\textsuperscript{21:3.8} 4. La soberanía suprema ---la autoridad establecida que sigue a la finalización de la séptima donación. La soberanía suprema en Nebadon data de la terminación de la donación de Miguel en Urantia. Ha existido desde hace poco más de mil novecientos años de vuestro tiempo planetario.

\par
%\textsuperscript{(238.3)}
\textsuperscript{21:3.9} 5. La soberanía suprema creciente ---las relaciones avanzadas que se derivan del establecimiento de la mayoría de los dominios de las criaturas en la luz y la vida. Esta etapa pertenece al futuro aún no alcanzado de vuestro universo local.

\par
%\textsuperscript{(238.4)}
\textsuperscript{21:3.10} 6. La soberanía trinitaria ---que es ejercida después de que todo el universo local se ha establecido en la luz y la vida.

\par
%\textsuperscript{(238.5)}
\textsuperscript{21:3.11} 7. La soberanía no revelada ---las relaciones desconocidas de una era futura del universo.

\par
%\textsuperscript{(238.6)}
\textsuperscript{21:3.12} Al aceptar la soberanía inicial como vicegerente de un universo local en proyecto, un Miguel Creador presta a la Trinidad el juramento de no asumir la soberanía suprema hasta que no haya terminado sus siete donaciones como criatura y éstas hayan sido certificadas por los gobernantes del superuniverso. Pero si un Hijo Miguel no pudiera imponer a voluntad esta soberanía no ganada, no tendría ningún sentido prestar el juramento de no hacerlo.

\par
%\textsuperscript{(238.7)}
\textsuperscript{21:3.13} Incluso en las épocas anteriores a sus donaciones, un Hijo Creador gobierna su dominio de manera casi suprema cuando no hay disensiones en ninguna de sus partes. Las limitaciones de su gobierno difícilmente podrían manifestarse si su soberanía no fuera nunca desafiada. La soberanía que ejerce un Hijo Creador antes de sus donaciones en un universo sin rebelión no es más grande que en un universo con rebelión; pero en el primer caso las limitaciones de su soberanía no son evidentes, mientras que en el segundo sí lo son.

\par
%\textsuperscript{(238.8)}
\textsuperscript{21:3.14} Si la autoridad o la administración de un Hijo Creador es alguna vez desafiada, atacada o puesta en peligro, él se ha comprometido eternamente a sostener, proteger, defender y si es necesario recuperar su creación personal. A estos Hijos sólo los pueden perturbar u hostigar las criaturas que ellos mismos han creado o los seres más elevados que ellos mismos han elegido. Se podría deducir que es poco probable que unos <<\textit{seres más elevados}>>, que tienen su origen en unos niveles superiores al del universo local, puedan causar dificultades a un Hijo Creador, y esto es cierto. Pero podrían hacerlo si así lo eligieran. La virtud es volitiva en la personalidad; la rectitud no es automática en las criaturas dotadas de libre albedrío.

\par
%\textsuperscript{(238.9)}
\textsuperscript{21:3.15} Antes de terminar su carrera de donación, un Hijo Creador gobierna con ciertas limitaciones de soberanía que se impone a sí mismo, pero después de finalizar su servicio de donación, gobierna en virtud de su experiencia real vivida bajo la forma y la similitud de sus múltiples criaturas. Cuando un Creador ha residido siete veces entre sus criaturas, cuando su carrera de donación ha terminado, entonces es establecido de manera suprema en la autoridad sobre su universo; se ha convertido en un Hijo Maestro, en un gobernante soberano y supremo.

\par
%\textsuperscript{(238.10)}
\textsuperscript{21:3.16} La técnica para obtener la soberanía suprema sobre un universo local incluye las siete etapas experienciales siguientes:

\par
%\textsuperscript{(238.11)}
\textsuperscript{21:3.17} 1. Descubrir por experiencia siete niveles de existencia de las criaturas mediante la técnica de donarse de forma encarnada en la similitud misma de las criaturas de un nivel determinado.

\par
%\textsuperscript{(238.12)}
\textsuperscript{21:3.18} 2. Consagrarse de manera experiencial a cada fase de la voluntad séptuple de la Deidad del Paraíso, tal como esta voluntad se encuentra personificada en los Siete Espíritus Maestros.

\par
%\textsuperscript{(239.1)}
\textsuperscript{21:3.19} 3. Atravesar cada una de las siete experiencias en los niveles de las criaturas, y ejecutar simultáneamente una de las siete consagraciones a la voluntad de la Deidad del Paraíso.

\par
%\textsuperscript{(239.2)}
\textsuperscript{21:3.20} 4. En cada nivel de las criaturas, describir experiencialmente el apogeo de la vida de las criaturas a la Deidad del Paraíso y a todas las inteligencias del universo.

\par
%\textsuperscript{(239.3)}
\textsuperscript{21:3.21} 5. En cada nivel de las criaturas, revelar experiencialmente una fase de la voluntad séptuple de la Deidad a los seres del nivel de esa donación y a todo el universo.

\par
%\textsuperscript{(239.4)}
\textsuperscript{21:3.22} 6. Unificar experiencialmente la séptuple experiencia de las criaturas con la séptuple experiencia de consagrarse a revelar la naturaleza y la voluntad de la Deidad.

\par
%\textsuperscript{(239.5)}
\textsuperscript{21:3.23} 7. Conseguir una relación nueva y más elevada con el Ser Supremo. La repercusión de la totalidad de esta experiencia como Creador y criatura aumenta la realidad superuniversal de Dios Supremo y la soberanía espacio-temporal del Todopoderoso Supremo, y convierte en un hecho la soberanía suprema de un Miguel Paradisiaco sobre su universo local.

\par
%\textsuperscript{(239.6)}
\textsuperscript{21:3.24} Al resolver la cuestión de la soberanía en un universo local, el Hijo Creador no se limita a demostrar su propia aptitud para gobernar, sino que revela también la naturaleza y describe la actitud séptuple de las Deidades del Paraíso. La comprensión finita y la apreciación de la primacía del Padre por parte de las criaturas están implicadas en la aventura de un Hijo Creador cuando condesciende a asumir la forma y las experiencias de sus criaturas. Estos Hijos primarios del Paraíso son los verdaderos reveladores de la naturaleza amorosa y de la autoridad benefactora del Padre, del mismo Padre que, en asociación con el Hijo y el Espíritu, es el jefe universal de todo poder, de toda personalidad y de todo gobierno en todos los reinos universales.

\section*{4. Las donaciones de los Migueles}
\par
%\textsuperscript{(239.7)}
\textsuperscript{21:4.1} Hay siete grupos de Hijos Creadores donadores y están clasificados así de acuerdo con el número de veces que se han donado a las criaturas de sus reinos. Van desde la experiencia inicial, pasando por las cinco esferas adicionales de donación progresiva, hasta que alcanzan el episodio séptimo y final de la experiencia como Creador y criatura.

\par
%\textsuperscript{(239.8)}
\textsuperscript{21:4.2} Las donaciones de los Avonales siempre se producen en la similitud de la carne mortal, pero las siete donaciones de un Hijo Creador implican su aparición en siete niveles de existencia de las criaturas y están relacionadas con la revelación de las siete expresiones primarias de la voluntad y la naturaleza de la Deidad. Todos los Hijos Creadores sin excepción pasan siete veces por estas siete entregas de sí mismos a sus hijos creados antes de asumir la jurisdicción estable y suprema sobre el universo que ellos mismos han creado.

\par
%\textsuperscript{(239.9)}
\textsuperscript{21:4.3} Aunque estas siete donaciones varían en los diferentes sectores y universos, siempre engloban la aventura de donarse como mortal. En su donación final, un Hijo Creador aparece como miembro de una de las razas mortales superiores de algún mundo habitado, generalmente como miembro del grupo racial que contiene el mayor legado hereditario del linaje adámico importado anteriormente para elevar el estado físico de los pueblos de origen animal. En su carrera séptuple como Hijo donador, un Hijo Paradisiaco nace de mujer una sola vez, tal como figura en vuestro relato sobre el bebé de Belén. Vive y muere una sola vez como miembro de la orden más humilde de criaturas volitivas evolutivas.

\par
%\textsuperscript{(239.10)}
\textsuperscript{21:4.4} Después de cada una de sus donaciones, un Hijo Creador se dirige a <<\textit{la derecha del Padre}>>\footnote{\textit{La derecha del Padre}: Sal 110:1; Mt 22:43-44; Mc 12:36; 16:19; Lc 20:42; Hch 7:55-56; Ro 8:34; Col 3:1; Heb 1:3; 8:1; 10:12; 12:2; 1 P 3:22.} para conseguir allí que el Padre acepte su donación y para recibir instrucciones con miras al episodio siguiente de servicio universal. Después de la séptima y última donación, un Hijo Creador recibe del Padre Universal la autoridad y la jurisdicción supremas sobre su universo.

\par
%\textsuperscript{(240.1)}
\textsuperscript{21:4.5} Es un hecho establecido que el último Hijo divino que apareció en vuestro planeta era un Hijo Creador Paradisiaco que había completado seis fases de su carrera donadora; en consecuencia, cuando abandonó el dominio consciente de su vida encarnada en Urantia, pudo decir en verdad, y así lo hizo: <<\textit{Todo se ha consumado}>>\footnote{\textit{Todo se ha consumado}: Jn 19:30.} ---todo había terminado literalmente. Su muerte en Urantia concluyó su carrera donadora; era el último paso para cumplir con el juramento sagrado de un Hijo Creador Paradisiaco. Cuando han adquirido esta experiencia, estos Hijos son los soberanos supremos de sus universos; ya no gobiernan como vicegerentes del Padre, sino en su propio nombre y por su propio derecho, como <<\textit{Rey de Reyes y Señor de Señores}>>\footnote{\textit{Rey de Reyes}: 1 Ti 6:15; Ap 17:14; 19:16.}. Con algunas de las excepciones indicadas, estos Hijos donadores séptuples son incondicionalmente supremos en los universos donde residen. En lo que concierne a su universo local, <<\textit{todo poder en el cielo y en la Tierra}>>\footnote{\textit{Todo poder en el cielo y la Tierra}: Mt 28:18.} fue sometido a este Hijo Maestro triunfante y entronizado.

\par
%\textsuperscript{(240.2)}
\textsuperscript{21:4.6} Después de finalizar sus carreras donadoras, los Hijos Creadores son considerados como una orden distinta, la de los Hijos Maestros séptuples. En su persona, los Hijos Maestros son idénticos a los Hijos Creadores, pero han sufrido una experiencia donadora tan excepcional que se les considera generalmente como una orden diferente. Cuando un Creador se digna efectuar una donación, un cambio real y permanente está destinado a producirse. En verdad, el Hijo donador sigue siendo a pesar de todo un Creador, pero ha añadido a su naturaleza la experiencia de una criatura, lo cual lo elimina para siempre del nivel divino de un Hijo Creador y lo eleva al plano experiencial de un Hijo Maestro, de un ser que se ha ganado plenamente el derecho de gobernar un universo y de administrar sus mundos. Estos seres personifican todo lo que se puede obtener del linaje divino y engloban todo lo que puede provenir de la experiencia de una criatura perfeccionada. ¿Por qué el hombre tendría que lamentarse de su origen humilde y de su carrera evolutiva inevitable, cuando los Dioses mismos tienen que pasar por una experiencia equivalente antes de ser considerados experiencialmente dignos y competentes para gobernar final y plenamente sus dominios universales?

\section*{5. La relación de los Hijos Maestros con el universo}
\par
%\textsuperscript{(240.3)}
\textsuperscript{21:5.1} El poder de un Miguel Maestro es ilimitado porque proviene de la asociación experiencial con la Trinidad del Paraíso, y es indiscutible porque procede de una experiencia real obtenida bajo la forma de las criaturas mismas que están sometidas a esa autoridad. La naturaleza de la soberanía de un Hijo Creador séptuple es suprema porque:

\par
%\textsuperscript{(240.4)}
\textsuperscript{21:5.2} 1. Abarca el punto de vista séptuple de la Deidad del Paraíso.

\par
%\textsuperscript{(240.5)}
\textsuperscript{21:5.3} 2. Personifica una actitud séptuple de las criaturas del espacio-tiempo.

\par
%\textsuperscript{(240.6)}
\textsuperscript{21:5.4} 3. Sintetiza perfectamente la actitud paradisiaca y el punto de vista de las criaturas.

\par
%\textsuperscript{(240.7)}
\textsuperscript{21:5.5} Esta soberanía experiencial incluye así toda la divinidad de Dios Séptuple que culmina en el Ser Supremo. Y la soberanía personal de un Hijo séptuple es semejante a la soberanía futura del Ser Supremo que algún día llegará a su culminación, la cual abarca, tal como lo hace, el contenido más completo posible del poder y de la autoridad que la Trinidad del Paraíso puede manifestar dentro de los límites espacio-temporales correspondientes.

\par
%\textsuperscript{(240.8)}
\textsuperscript{21:5.6} Cuando un Hijo Miguel consigue la soberanía suprema sobre su universo local, deja atrás el poder y la oportunidad de crear tipos enteramente nuevos de criaturas durante la presente era del universo. Pero el hecho de que un Hijo Maestro pierda su poder para dar origen a unas órdenes de seres enteramente nuevos no interfiere de ninguna manera el trabajo de elaboración de la vida ya establecido y en proceso de desarrollo; este inmenso programa de evolución universal sigue adelante sin interrupción ni reducción. La adquisición de la soberanía suprema por parte de un Hijo Maestro implica la responsabilidad de dedicarse personalmente a fomentar y a administrar aquello que ya ha sido diseñado y creado, y aquello que será engendrado posteriormente por aquellos que han sido así diseñados y creados. Con el tiempo se puede desarrollar una evolución casi infinita de seres diversos, pero desde este momento en adelante, ningún tipo o modelo enteramente nuevos de criaturas inteligentes tendrá directamente su origen en el Hijo Maestro. Éste es el primer paso, el principio, de una administración estabilizada en cualquier universo local.

\par
%\textsuperscript{(241.1)}
\textsuperscript{21:5.7} La elevación de un Hijo donador séptuple a la soberanía indiscutible de su universo significa el principio del fin de una incertidumbre y de una confusión relativa seculares. Después de este acontecimiento, aquello que no pueda ser algún día espiritualizado será finalmente desorganizado; aquello que no pueda ser algún día coordinado con la realidad cósmica será finalmente destruido. Cuando las disposiciones de una misericordia interminable y de una paciencia indecible se han agotado en un esfuerzo por conseguir la lealtad y la devoción de todas las criaturas volitivas de los reinos, la justicia y la rectitud prevalecerán. La justicia terminará por aniquilar aquello que la misericordia no ha podido rehabilitar.

\par
%\textsuperscript{(241.2)}
\textsuperscript{21:5.8} Los Migueles Maestros son supremos en sus propios universos locales una vez que han sido instalados como gobernantes soberanos. Las pocas limitaciones a su gobierno son las inherentes a la preexistencia cósmica de ciertas fuerzas y personalidades. Por lo demás, estos Hijos Maestros son supremos en autoridad, en responsabilidad y en poder administrativo en sus universos respectivos; como Creadores y Dioses, son supremos en casi todas las cosas. En lo que se refiere al funcionamiento de un universo dado, no existe perspicacia alguna más allá de su sabiduría.

\par
%\textsuperscript{(241.3)}
\textsuperscript{21:5.9} Después de su elevación a la soberanía estable en un universo local, un Miguel Paradisiaco tiene el pleno control sobre todos los otros Hijos de Dios que ejercen su actividad en su dominio, y puede gobernar libremente de acuerdo con el concepto que tenga sobre las necesidades de sus reinos. Un Hijo Maestro puede cambiar a voluntad el orden de los juicios espirituales y de los ajustes evolutivos de los planetas habitados. Y estos Hijos elaboran y llevan a cabo los planes elegidos por ellos mismos en todas las cuestiones relacionadas con las necesidades planetarias especiales, en particular con respecto a los mundos donde han vivido como criaturas, y mucho más en lo que concierne a la esfera de su donación final, al planeta de su encarnación en la similitud de la carne mortal.

\par
%\textsuperscript{(241.4)}
\textsuperscript{21:5.10} Los Hijos Maestros parecen estar en perfecta comunicación con los mundos donde se han donado, no solamente con los mundos donde han residido personalmente, sino con todos los mundos en los que se ha donado un Hijo Magistral. Este contacto se mantiene mediante su propia presencia espiritual, el Espíritu de la Verdad\footnote{\textit{El Espíritu de la Verdad}: Ez 11:19; 18:31; 36:26-27; Jl 2:28-29; Lc 24:49; Jn 7:39; 14:16-18,23,26; 15:4,26; 16:7-8,13-14; 17:21-23; Hch 1:5,8a; 2:1-4,16-18; 2:33; 2 Co 13:5; Gl 2:20; 4:6; Ef 1:13; 4:30; 1 Jn 4:12-15.}, que pueden <<\textit{derramar sobre toda carne}>>\footnote{\textit{Derramar el Espíritu de la Verdad}: Hch 2:16-18.}. Estos Hijos Maestros mantienen también una conexión ininterrumpida con el Hijo Madre Eterno en el centro de todas las cosas. Poseen una facultad compasiva que se extiende desde el Padre Universal en las alturas hasta las razas humildes de la vida planetaria en los reinos del tiempo.

\section*{6. El destino de los Migueles Maestros}
\par
%\textsuperscript{(241.5)}
\textsuperscript{21:6.1} Nadie puede atreverse a hablar con una autoridad final sobre la naturaleza o el destino de los Soberanos Maestros séptuples de los universos locales; sin embargo, todos especulamos mucho sobre estas materias. Nos enseñan, y nosotros creemos, que cada Miguel Paradisiaco es el \textit{absoluto} de los dobles conceptos divinos que le dieron origen; personifica por tanto unas fases reales de la infinidad del Padre Universal y del Hijo Eterno. Los Migueles deben ser parciales en relación con la infinidad total, pero son probablemente absolutos en relación con esa parte de infinidad implicada en su origen. Pero al observar su trabajo en la presente era del universo no detectamos ninguna acción que sea más que finita; cualquier supuesta capacidad superfinita debe estar contenida en ellos mismos y hasta ahora no se ha revelado.

\par
%\textsuperscript{(242.1)}
\textsuperscript{21:6.2} La finalización de la carrera de donación bajo la forma de las criaturas y la elevación a la soberanía suprema de un universo deben significar la liberación completa de las capacidades de acción finita de un Miguel, acompañada de la aparición de la capacidad para llevar a cabo un servicio más que finito. Porque observamos a este respecto que estos Hijos Maestros se encuentran entonces limitados para engendrar nuevos tipos de seres creados, una restricción que se ha hecho indudablemente necesaria debido a la liberación de sus potencialidades superfinitas.

\par
%\textsuperscript{(242.2)}
\textsuperscript{21:6.3} Es muy probable que estos poderes creadores no revelados permanezcan contenidos en estos Hijos durante toda la presente era del universo. Pero creemos que en algún momento del lejano futuro, y en los universos del espacio exterior actualmente en vías de movilización, la unión entre un Hijo Maestro séptuple y un Espíritu Creativo de la séptima fase podría llegar a unos niveles absonitos de servicio acompañados de la aparición de nuevas cosas, significados y valores en unos niveles trascendentales que tendrían una importancia universal última.

\par
%\textsuperscript{(242.3)}
\textsuperscript{21:6.4} Al igual que la Deidad del Supremo se está concretando en virtud del servicio experiencial, los Hijos Creadores están consiguiendo la realización personal de los potenciales paradisiacos de divinidad contenidos en sus naturalezas insondables. Cristo Miguel dijo una vez cuando estaba en Urantia: <<\textit{Yo soy el camino, la verdad y la vida}>>\footnote{\textit{El camino, la verdad y la vida}: Jn 14:6.}. Y creemos que, en la eternidad, los Migueles están destinados a ser literalmente <<\textit{el camino, la verdad y la vida}>>, señalando siempre a todas las personalidades del universo el camino que conduce desde la divinidad suprema, pasando por la absonitidad última, hasta la finalidad eterna de la deidad.

\par
%\textsuperscript{(242.4)}
\textsuperscript{21:6.5} [Presentado por un Perfeccionador de la Sabiduría procedente de Uversa.]


\chapter{Documento 22. Los Hijos de Dios Trinitizados}
\par
%\textsuperscript{(243.1)}
\textsuperscript{22:0.1} HAY tres grupos de seres que son llamados Hijos de Dios. Además de las órdenes de filiación descendentes y ascendentes, existe un tercer grupo conocido como los Hijos de Dios Trinitizados. La orden trinitizada de filiación está subdividida en tres divisiones primarias de acuerdo con los orígenes de sus numerosos tipos de personalidades, reveladas o no reveladas. Estas divisiones primarias son:

\par
%\textsuperscript{(243.2)}
\textsuperscript{22:0.2} 1. Los Hijos trinitizados por la Deidad.

\par
%\textsuperscript{(243.3)}
\textsuperscript{22:0.3} 2. Los Hijos abrazados por la Trinidad.

\par
%\textsuperscript{(243.4)}
\textsuperscript{22:0.4} 3. Los Hijos trinitizados por las criaturas.

\par
%\textsuperscript{(243.5)}
\textsuperscript{22:0.5} Sin tener en cuenta su origen, todos los Hijos de Dios Trinitizados tienen en común la experiencia de la trinitización, ya sea como parte de su origen o bien como una experiencia de abrazo por la Trinidad alcanzada posteriormente. Los Hijos trinitizados por la Deidad no os son revelados en estas narraciones; esta presentación se limitará pues a describir los dos grupos restantes, y más especialmente los hijos de Dios abrazados por la Trinidad.

\section*{1. Los Hijos abrazados por la Trinidad}
\par
%\textsuperscript{(243.6)}
\textsuperscript{22:1.1} Todos los hijos abrazados por la Trinidad tienen originalmente un origen único o doble, pero después de ser abrazados por la Trinidad se dedican para siempre al servicio y a las tareas de la Trinidad. Este cuerpo, tal como está revelado y organizado para el servicio superuniversal, abarca siete órdenes de personalidades:

\par
%\textsuperscript{(243.7)}
\textsuperscript{22:1.2} 1. Los Mensajeros Poderosos.

\par
%\textsuperscript{(243.8)}
\textsuperscript{22:1.3} 2. Los Elevados en Autoridad.

\par
%\textsuperscript{(243.9)}
\textsuperscript{22:1.4} 3. Los que no tienen Nombre ni Número.

\par
%\textsuperscript{(243.10)}
\textsuperscript{22:1.5} 4. Los Custodios Trinitizados.

\par
%\textsuperscript{(243.11)}
\textsuperscript{22:1.6} 5. Los Embajadores Trinitizados.

\par
%\textsuperscript{(243.12)}
\textsuperscript{22:1.7} 6. Los Guardianes Celestiales.

\par
%\textsuperscript{(243.13)}
\textsuperscript{22:1.8} 7. Los Ayudantes de los Hijos Elevados.

\par
%\textsuperscript{(243.14)}
\textsuperscript{22:1.9} De acuerdo con su origen, su naturaleza y su función, estos siete grupos de personalidades están clasificados además en tres divisiones principales: los Hijos de la Consecución Trinitizados, los Hijos de la Elección Trinitizados y los Hijos de la Perfección Trinitizados.

\par
%\textsuperscript{(244.1)}
\textsuperscript{22:1.10} \textit{Los Hijos de la Consecución Trinitizados} ---los Mensajeros Poderosos, Los Elevados en Autoridad y Los que no tienen Nombre ni Número ---son todos mortales ascendentes fusionados con su Ajustador que han alcanzado el Paraíso y el Cuerpo de la Finalidad. Pero no son finalitarios; cuando han sido abrazados por la Trinidad, sus nombres son eliminados de la lista nominal de los finalitarios. Los nuevos hijos de esta orden pasan por unos cursos específicos de formación durante períodos relativamente cortos en los planetas sede de los circuitos de Havona bajo la dirección de los Eternos de los Días. Más tarde son destinados al servicio de los Ancianos de los Días en los siete superuniversos.

\par
%\textsuperscript{(244.2)}
\textsuperscript{22:1.11} \textit{Los Hijos de la Elección Trinitizados} incluyen a los Custodios Trinitizados y a los Embajadores Trinitizados. Son reclutados entre ciertos serafines evolutivos y ciertas criaturas intermedias trasladadas que han atravesado Havona y han llegado al Paraíso, así como entre ciertos mortales fusionados con el Espíritu o con el Hijo que han ascendido igualmente hasta la Isla central de Luz y de Vida. Después de ser abrazados por la Trinidad del Paraíso y de un breve entrenamiento en Havona, los Hijos de la Elección Trinitizados son destinados a las cortes de los Ancianos de los Días.

\par
%\textsuperscript{(244.3)}
\textsuperscript{22:1.12} \textit{Los Hijos de la Perfección Trinitizados.} Los Guardianes Celestiales y sus coordinados, los Ayudantes de los Hijos Elevados, componen un grupo único de personalidades dos veces trinitizadas. Son los hijos trinitizados por las criaturas, las cuales son las personalidades del Paraíso-Havona o los mortales ascendentes perfeccionados que se han distinguido durante mucho tiempo en el Cuerpo de la Finalidad. Algunos de estos hijos trinitizados por las criaturas, después de servir con los Ejecutivos Supremos de los Siete Espíritus Maestros y después de servir bajo las órdenes de los Hijos Instructores Trinitarios, son trinitizados (abrazados) de nuevo por la Trinidad del Paraíso y luego asignados a las cortes de los Ancianos de los Días como Guardianes Celestiales y como Ayudantes de los Hijos Elevados. Los Hijos de la Perfección Trinitizados son destinados directamente al servicio de los superuniversos sin más preparación adicional.

\par
%\textsuperscript{(244.4)}
\textsuperscript{22:1.13} Nuestros asociados de origen trinitario ---los Perfeccionadores de la Sabiduría, los Consejeros Divinos y los Censores Universales--- tienen un número fijo, pero los hijos abrazados por la Trinidad aumentan de forma constante. Las siete órdenes de hijos abrazados por la Trinidad son nombradas como miembros de uno de los siete gobiernos superuniversales, y el número de ellas en servicio en cada superuniverso es exactamente el mismo; ninguno se ha perdido nunca. Los seres abrazados por la Trinidad no se han descarriado nunca; pueden dar un traspié temporalmente, pero ni uno solo ha sido juzgado nunca por desacato a los gobiernos de los superuniversos. Los Hijos de la Consecución y los Hijos de la Elección no han fallado nunca en su servicio en Orvonton, pero los Hijos de la Perfección Trinitizados a veces han cometido errores de juicio y han ocasionado así una confusión transitoria.

\par
%\textsuperscript{(244.5)}
\textsuperscript{22:1.14} Bajo la dirección de los Ancianos de los Días, las siete órdenes funcionan en gran medida como grupos autónomos. El ámbito de su servicio es extenso; los Hijos de la Perfección Trinitizados no salen del superuniverso donde están destinados, pero sus asociados trinitizados recorren el gran universo, viajando desde los mundos evolutivos del tiempo y del espacio hasta la Isla eterna del Paraíso. Pueden ejercer su actividad en cualquier superuniverso, pero siempre lo hacen como miembros del supergobierno al que fueron asignados originalmente.

\par
%\textsuperscript{(244.6)}
\textsuperscript{22:1.15} En apariencia, los hijos abrazados por la Trinidad han sido destinados de forma permanente al servicio de los siete superuniversos; esta misión tendrá seguramente la duración de la presente era del universo, pero nunca se nos ha informado de que vaya a ser eterna.

\section*{2. Los Mensajeros Poderosos}
\par
%\textsuperscript{(245.1)}
\textsuperscript{22:2.1} Los Mensajeros Poderosos pertenecen al grupo ascendente de Hijos Trinitizados. Son una clase compuesta por mortales perfeccionados que han sido puestos a prueba durante una rebelión o han demostrado igualmente de otra manera su lealtad personal; todos han pasado por alguna prueba determinada de lealtad universal. En algún momento durante su ascensión al Paraíso se mantuvieron firmes y leales en medio de la deslealtad de sus superiores, y algunos actuaron de manera activa y leal en el lugar de sus jefes desleales.

\par
%\textsuperscript{(245.2)}
\textsuperscript{22:2.2} Con estos antecedentes personales de fidelidad y de devoción, estos mortales ascendentes atraviesan Havona con la oleada de los peregrinos del tiempo, llegan al Paraíso, se gradúan allí, y son enrolados en el Cuerpo de la Finalidad. Después de esto son trinitizados en el abrazo secreto de la Trinidad del Paraíso, y posteriormente se les destina a asociarse con los Ancianos de los Días en la administración de los gobiernos de los siete superuniversos.

\par
%\textsuperscript{(245.3)}
\textsuperscript{22:2.3} Todo mortal ascendente que sufre una experiencia insurreccional y que actúa lealmente en presencia de una rebelión está destinado a convertirse finalmente en un Mensajero Poderoso del servicio superuniversal. Y lo mismo sucede con toda criatura ascendente que impide eficazmente estos disturbios ocasionados por el error, el mal o el pecado; ya que toda acción destinada a impedir una rebelión o a llevar a cabo unos tipos de lealtad más elevados en una crisis universal es considerada como que tiene un valor aún más grande que la simple lealtad en presencia de una rebelión efectiva.

\par
%\textsuperscript{(245.4)}
\textsuperscript{22:2.4} Los Mensajeros Poderosos más antiguos fueron escogidos entre los mortales ascendentes del tiempo y del espacio que llegaron primero al Paraíso; muchos de ellos habían atravesado Havona en la época de Grandfanda. Pero la primera trinitización de Mensajeros Poderosos no se llevó a cabo hasta que el grupo de candidatos no contuvo representantes de cada uno de los siete superuniversos. El último grupo de esta orden que se capacitó en el Paraíso contenía peregrinos ascendentes del universo local de Nebadon.

\par
%\textsuperscript{(245.5)}
\textsuperscript{22:2.5} Los Mensajeros Poderosos son abrazados por la Trinidad del Paraíso en clases de setecientos mil, y cien mil son asignados a cada superuniverso. Hay casi un billón de Mensajeros Poderosos destinados en Uversa, y existen muchas razones para creer que el número de los que sirven en cada uno de los siete superuniversos es exactamente el mismo.

\par
%\textsuperscript{(245.6)}
\textsuperscript{22:2.6} Soy un Mensajero Poderoso, y a los urantianos quizás les interese saber que la compañera y asociada de mi experiencia humana triunfó también en la gran prueba, que aunque estuvimos separados muchas veces y durante largos períodos en el transcurso de la ascensión secular interior hacia Havona, fuimos abrazados en el mismo grupo de setecientos mil, y el tiempo que estuvimos en Vicegerington lo pasamos en estrecha y amorosa asociación. Finalmente entramos en servicio y fuimos destinados juntos a Uversa de Orvonton, y a menudo nos envían en compañía para ejecutar misiones que necesitan el servicio de dos Mensajeros.

\par
%\textsuperscript{(245.7)}
\textsuperscript{22:2.7} Los Mensajeros Poderosos, al igual que todos los hijos abrazados por la Trinidad, son destinados a todas las fases de las actividades superuniversales. Mantienen una conexión constante con sus sedes centrales a través del servicio de la reflectividad superuniversal. Los Mensajeros Poderosos sirven en todos los sectores de un superuniverso, y con frecuencia realizan misiones en los universos locales e incluso en los mundos individuales, tal como lo estoy haciendo en esta ocasión.

\par
%\textsuperscript{(245.8)}
\textsuperscript{22:2.8} Los Mensajeros Poderosos actúan en los tribunales superuniversales como defensores de los individuos y de los planetas cuando éstos comparecen a juicio; también ayudan a los Perfecciones de los Días en la dirección de los asuntos de los sectores mayores. Su tarea principal, como grupo, es la de observadores superuniversales. Están estacionados en los diversos mundos sede y en los planetas individuales importantes como observadores oficiales de los Ancianos de los Días. Cuando están asignados así, sirven también como asesores de las autoridades que dirigen los asuntos de las esferas donde residen. Los Mensajeros participan activamente en todas las fases del programa ascendente de la progresión de los mortales. Con sus asociados de origen mortal, mantienen a los supergobiernos en contacto estrecho y personal con el estado y el progreso de los planes de los Hijos de Dios descendentes.

\par
%\textsuperscript{(246.1)}
\textsuperscript{22:2.9} Los Mensajeros Poderosos son plenamente conscientes de toda su carrera ascendente, y por eso son unos ministros tan útiles y compasivos, unos mensajeros tan comprensivos, para efectuar su servicio en cualquier mundo del espacio y a cualquier criatura del tiempo. En cuanto seáis liberados de la carne os comunicaréis de manera libre y comprensiva con nosotros, puesto que procedemos de todas las razas de todos los mundos evolutivos del espacio, es decir, de aquellas razas mortales que están habitadas por Ajustadores del Pensamiento y que fusionan posteriormente con ellos.

\section*{3. Los Elevados en Autoridad}
\par
%\textsuperscript{(246.2)}
\textsuperscript{22:3.1} Todos Los Elevados en Autoridad, el segundo grupo de Hijos de la Consecución Trinitizados, son seres de origen mortal fusionados con su Ajustador. Son los mortales perfeccionados que han mostrado una capacidad administrativa superior y que han demostrado una genialidad ejecutiva extraordinaria durante toda su larga carrera ascendente. Son la flor y nata de las aptitudes para gobernar procedentes de los mortales supervivientes del espacio.

\par
%\textsuperscript{(246.3)}
\textsuperscript{22:3.2} Setenta mil Elevados en Autoridad son trinitizados en cada unión con la Trinidad. Aunque el universo local de Nebadon es una creación relativamente joven, posee representantes en una clase recientemente trinitizada de esta orden. Más de diez mil millones de estos hábiles administradores están destinados actualmente en Orvonton. Al igual que todas las distintas órdenes de seres celestiales, mantienen su propia sede central en Uversa y, al igual que los otros hijos abrazados por la Trinidad, sus reservas en Uversa actúan como cuerpo dirigente central de esta orden en Orvonton.

\par
%\textsuperscript{(246.4)}
\textsuperscript{22:3.3} Los Elevados en Autoridad son unos administradores que no tienen limitaciones. Son los ejecutivos presentes en todas partes y siempre eficaces de los Ancianos de los Días. Sirven en cualquier esfera, en cualquier mundo habitado y en cualquier fase de actividad de cualquiera de los siete superuniversos.

\par
%\textsuperscript{(246.5)}
\textsuperscript{22:3.4} Dotados de una magnífica sabiduría administrativa y de una destreza ejecutiva excepcional, estos seres brillantes se encargan de presentar la causa de la justicia en nombre de los tribunales superuniversales; fomentan el cumplimiento de la justicia y la rectificación de los errores de adaptación en los universos evolutivos. Por este motivo, si alguna vez sois citados por errores de juicio mientras ascendéis por los mundos y las esferas de vuestra progresión cósmica ordenada, es muy improbable que sufráis una injusticia, puesto que vuestros acusadores serán antiguas criaturas ascendentes que están familiarizadas personalmente con cada etapa de la carrera que habréis atravesado y que estáis atravesando.

\section*{4. Los que no tienen Nombre ni Número}
\par
%\textsuperscript{(246.6)}
\textsuperscript{22:4.1} Los que no tienen Nombre ni Número constituyen el tercer y último grupo de Hijos de la Consecución Trinitizados; son las almas ascendentes que han desarrollado una capacidad para la adoración que sobrepasa la habilidad de todos los hijos e hijas de las razas evolutivas de los mundos del tiempo y del espacio. Han adquirido un concepto espiritual del objetivo eterno del Padre Universal que trasciende comparativamente la comprensión de las criaturas evolutivas que poseen un nombre o un número; por eso los denominamos Los que no tienen Nombre ni Número. Traducido con más precisión, su nombre debería ser <<\textit{Los que están \textit{más allá} de todo Nombre y de todo Número}>>.

\par
%\textsuperscript{(247.1)}
\textsuperscript{22:4.2} La Trinidad del Paraíso abraza a esta orden de hijos en grupos de siete mil. Más de cien millones de estos hijos destinados en Orvonton se encuentran registrados en Uversa.

\par
%\textsuperscript{(247.2)}
\textsuperscript{22:4.3} Puesto que Los que no tienen Nombre ni Número son las mentes espirituales superiores de las razas supervivientes, están especialmente cualificados para juzgar y ofrecer su opinión cuando se desea tener un punto de vista espiritual, y cuando la experiencia de la carrera ascendente es esencial para comprender adecuadamente las cuestiones implicadas en el problema a juzgar. Son los jurados supremos de Orvonton. En algunos mundos, un sistema de jurado mal administrado puede ser más o menos una parodia de la justicia, pero en Uversa y en sus tribunales anexos empleamos como jueces y jurados los tipos más elevados de mentalidades espirituales evolucionadas. El ejercicio de la justicia es la función más elevada de cualquier gobierno, y aquellos a quienes se les ha encomendado pronunciar los veredictos deberían ser escogidos entre los tipos más elevados y nobles de individuos con mayor experiencia y comprensión.

\par
%\textsuperscript{(247.3)}
\textsuperscript{22:4.4} La elección de candidatos para las clases trinitizadas de los Mensajeros Poderosos, Los Elevados en Autoridad y Los que no tienen Nombre ni Número es inherente y automática. Las técnicas selectivas del Paraíso no son arbitrarias en ningún sentido. La experiencia personal y los valores espirituales determinan quiénes pertenecerán a los Hijos de la Consecución Trinitizados. Estos seres tienen una autoridad equivalente y su estado administrativo es uniforme, pero todos poseen una individualidad y unos caracteres distintos; no se trata de seres estandarizados. Todos poseen unas características diferentes, dependiendo de las diferencias de sus carreras ascendentes.

\par
%\textsuperscript{(247.4)}
\textsuperscript{22:4.5} Además de estas aptitudes experienciales, los Hijos de la Consecución Trinitizados han sido trinitizados en el abrazo divino de las Deidades del Paraíso. En consecuencia, ejercen su actividad como asociados coordinados de los Hijos Estacionarios de la Trinidad, porque el abrazo de la Trinidad parece precipitar fuera de la corriente del futuro muchos potenciales no realizados de los seres creados. Pero esto sólo es cierto en lo que se refiere a la presente era del universo.

\par
%\textsuperscript{(247.5)}
\textsuperscript{22:4.6} Este grupo de hijos se ocupa principalmente, pero no del todo, de los servicios relacionados con la carrera ascendente de los mortales del espacio-tiempo. Si el punto de vista de un criatura mortal se pone alguna vez en duda, la cuestión se resuelve apelando a una comisión ascendente compuesta por un Mensajero Poderoso, un Elevado en Autoridad y uno que no tiene Nombre ni Número.

\par
%\textsuperscript{(247.6)}
\textsuperscript{22:4.7} Vosotros, los mortales que leéis este mensaje, podéis ascender hasta el Paraíso, conseguir el abrazo de la Trinidad, ser destinados en las épocas lejanas futuras al servicio de los Ancianos de los Días en uno de los siete superuniversos, y recibir alguna vez la misión de ampliar la revelación de la verdad en algún planeta habitado en evolución, tal como yo lo hago actualmente en Urantia.

\section*{5. Los Custodios Trinitizados}
\par
%\textsuperscript{(247.7)}
\textsuperscript{22:5.1} Los Custodios Trinitizados son los Hijos de la Elección Trinitizados. Vuestras razas y otros mortales con un valor de supervivencia no son los únicos que atraviesan Havona, llegan al Paraíso y a veces se encuentran destinados al servicio superuniversal con los Hijos Estacionarios de la Trinidad, sino que vuestros fieles guardianes seráficos y vuestros asociados intermedios igualmente fieles también pueden ser candidatos al mismo reconocimiento por la Trinidad y al mismo magnífico destino de la personalidad.

\par
%\textsuperscript{(248.1)}
\textsuperscript{22:5.2} Los Custodios Trinitizados son los serafines ascendentes y las criaturas intermedias trasladadas que han pasado por Havona y han llegado al Paraíso y al Cuerpo de la Finalidad. Posteriormente fueron abrazados por la Trinidad del Paraíso y destinados al servicio de los Ancianos de los Días.

\par
%\textsuperscript{(248.2)}
\textsuperscript{22:5.3} A los serafines ascendentes que son candidatos al abrazo de la Trinidad se les concede este reconocimiento porque han cooperado valientemente con algún ascendente mortal que ha alcanzado el Cuerpo de la Finalidad y ha sido posteriormente trinitizado. El guardián seráfico de mi propia carrera como mortal la atravesó entera conmigo, fue más tarde trinitizado, y ahora está vinculado al gobierno de Uversa como Custodio Trinitizado.

\par
%\textsuperscript{(248.3)}
\textsuperscript{22:5.4} Y lo mismo sucede con las criaturas intermedias; muchas de ellas son trasladadas, alcanzan el Paraíso, y junto con los serafines y por las mismas razones, son abrazadas por la Trinidad y enviadas a servir como Custodios en los superuniversos.

\par
%\textsuperscript{(248.4)}
\textsuperscript{22:5.5} La Trinidad del Paraíso abraza a los Custodios Trinitizados en grupos de setenta mil, y una séptima parte de cada grupo es asignada a un superuniverso. Algo más de diez millones de estos Custodios elevados y fiables están actualmente al servicio de Orvonton. Sirven en Uversa y en las esferas sede de los sectores mayores y menores. Para realizar sus tareas reciben la ayuda de un cuerpo de varios miles de millones de seconafines y de otras hábiles personalidades superuniversales.

\par
%\textsuperscript{(248.5)}
\textsuperscript{22:5.6} Los Custodios Trinitizados empiezan su carrera como custodios y continúan como tales en los asuntos de los supergobiernos. En cierto modo son los funcionarios de los gobiernos de sus superuniversos, pero no se ocupan de los individuos como lo hacen los Guardianes Celestiales. Los Custodios Trinitizados administran los asuntos de los grupos y fomentan los proyectos colectivos. Son los custodios de los archivos, los planes y las instituciones; actúan como fideicomisarios de las empresas, los grupos de personalidades, los proyectos ascendentes, los planes morontiales, los proyectos universales y otras innumerables empresas.

\section*{6. Los Embajadores Trinitizados}
\par
%\textsuperscript{(248.6)}
\textsuperscript{22:6.1} Los Embajadores Trinitizados son la segunda orden de Hijos de la Elección Trinitizados y, al igual que sus asociados los Custodios, son reclutados entre dos tipos de criaturas ascendentes. No todos los mortales ascendentes fusionan con el Ajustador o con el Padre; algunos fusionan con el Espíritu y otros fusionan con el Hijo. Algunos de estos mortales fusionados con el Espíritu o con el Hijo llegan a Havona y alcanzan el Paraíso. Los candidatos son escogidos entre estos ascendentes del Paraíso para ser abrazados por la Trinidad y, de vez en cuando, son trinitizados en clases de siete mil. Luego son enviados a servir en los superuniversos como Embajadores Trinitizados de los Ancianos de los Días. Hay casi quinientos millones registrados en Uversa.

\par
%\textsuperscript{(248.7)}
\textsuperscript{22:6.2} Los Embajadores Trinitizados son elegidos para el abrazo de la Trinidad de acuerdo con los informes de sus instructores de Havona. Representan las mentes superiores de sus grupos respectivos y están por tanto mejor cualificados para ayudar a los gobernantes de los superuniversos a comprender y a administrar los intereses de los mundos de donde proceden los mortales fusionados con el Espíritu. Los Embajadores fusionados con el Hijo nos resultan de una gran ayuda en nuestras relaciones con los problemas que afectan a la orden de personalidades fusionadas con el Hijo.

\par
%\textsuperscript{(248.8)}
\textsuperscript{22:6.3} A todos los efectos, los Embajadores Trinitizados son los emisarios de los Ancianos de los Días para todos los mundos o universos situados dentro del superuniverso donde están destinados. Prestan sus servicios especiales e importantes en las sedes de los sectores menores y efectúan las innumerables y diversas tareas de un superuniverso. Forman el cuerpo de urgencia o de reserva de los Hijos Trinitizados de los supergobiernos, y están pues disponibles para una gran variedad de funciones. Se ocupan de miles y miles de empresas relacionadas con los asuntos de los superuniversos, imposibles de describir a la mente humana, puesto que en Urantia no se realiza nada que se parezca de alguna manera a estas actividades.

\section*{7. La técnica de la trinitización}
\par
%\textsuperscript{(249.1)}
\textsuperscript{22:7.1} No puedo exponer plenamente a la mente material la experiencia de la acción creativa suprema que efectúan los seres espirituales perfectos y perfeccionados ---el acto de la trinitización. Las técnicas de la trinitización figuran entre los secretos de Vicegerington y de Solitarington y no se pueden revelar a nadie salvo a aquellos que han pasado por estas experiencias únicas, y sólo pueden ser comprendidas por ellos. Por eso a cualquier ser le resultará imposible describir con éxito a la mente humana la naturaleza y el contenido de esta operación extraordinaria.

\par
%\textsuperscript{(249.2)}
\textsuperscript{22:7.2} Aparte de las Deidades, sólo las personalidades del Paraíso-Havona y algunos miembros de cada cuerpo finalitario se ocupan de la trinitización. Bajo unas condiciones especializadas de perfección paradisiaca, estos seres magníficos pueden embarcarse en la aventura única de la identidad de concepto, y muchas veces logran dar nacimiento a un nuevo ser, a un hijo trinitizado por las criaturas.

\par
%\textsuperscript{(249.3)}
\textsuperscript{22:7.3} Las criaturas glorificadas que emprenden estas aventuras de trinitización sólo pueden participar en una de estas experiencias, mientras que para las Deidades del Paraíso no parece haber ningún límite en cuanto a la ejecución continuada de los episodios de trinitización. La Deidad sólo parece estar limitada en un único aspecto: sólo puede haber un Espíritu Original e Infinito, un solo ejecutivo infinito de la voluntad unida del Padre y del Hijo.

\par
%\textsuperscript{(249.4)}
\textsuperscript{22:7.4} Los finalitarios mortales ascendentes fusionados con el Ajustador que han alcanzado ciertos niveles de cultura paradisiaca y de desarrollo espiritual figuran entre aquellos seres que pueden intentar trinitizar una criatura. Cuando las compañías de finalitarios mortales están estacionadas en el Paraíso, se les concede un período de descanso cada milenio del tiempo de Havona. Estos finalitarios pueden elegir entre siete maneras diferentes de pasar este período libre de obligaciones, y una de ellas consiste en intentar llevar a cabo la trinitización de una criatura en asociación con algún compañero finalitario o con alguna personalidad del Paraíso-Havona.

\par
%\textsuperscript{(249.5)}
\textsuperscript{22:7.5} Si dos finalitarios mortales se presentan ante los Arquitectos del Universo Maestro y demuestran que han elegido de forma independiente un concepto idéntico para ser trinitizado, los Arquitectos tienen la facultad de promulgar según su propio criterio los mandatos que permitirán a estos ascendentes mortales glorificados prolongar su período de descanso y retirarse durante un tiempo al sector de los Ciudadanos del Paraíso reservado a la trinitización. Al final de este retiro concedido, si comunican que han elegido llevar a cabo de manera individual y conjunta el esfuerzo paradisiaco de espiritualizar, idealizar y hacer manifiesto un concepto seleccionado y original que no ha sido trinitizado hasta ese momento, entonces el Espíritu Maestro Número Siete emite las órdenes que autorizarán esta empresa extraordinaria.

\par
%\textsuperscript{(249.6)}
\textsuperscript{22:7.6} Estas aventuras consumen a veces unos períodos de tiempo increíblemente largos; parece transcurrir una era entera antes de que estos antiguos mortales fieles y decididos ---y a veces estas personalidades del Paraíso-Havona--- alcancen finalmente su objetivo, consigan realmente traer a la existencia efectiva el concepto de la verdad universal que han elegido. Estas parejas dedicadas no siempre tienen éxito; muchas veces fracasan, y esto se produce sin que se pueda descubrir ningún error por parte de ellas. Los candidatos a la trinitización que fracasan así son admitidos en un grupo especial de finalitarios designados como seres que han hecho el esfuerzo supremo y que han soportado la decepción suprema. Cuando las Deidades del Paraíso se unen para trinitizar siempre lo consiguen, pero no sucede lo mismo con una pareja homogénea de criaturas, con el intento de unión de dos miembros de la misma orden de seres.

\par
%\textsuperscript{(250.1)}
\textsuperscript{22:7.7} Cuando los Dioses trinitizan a un ser nuevo y original, el potencial de deidad de los padres divinos no cambia; pero cuando las criaturas exaltadas efectúan este episodio creativo, uno de los individuos participantes y contrayentes sufre una modificación excepcional en su personalidad. En cierto sentido, los dos progenitores de un hijo trinitizado por las criaturas se convierten espiritualmente en uno solo. Creemos que este estado de biunificación de ciertas fases espirituales de la personalidad predominará probablemente hasta el momento en que el Ser Supremo haya alcanzado la manifestación plena y completa de su personalidad en el gran universo.

\par
%\textsuperscript{(250.2)}
\textsuperscript{22:7.8} Esta unión espiritual funcional de los dos progenitores se produce simultáneamente con la aparición de un nuevo hijo trinitizado por las criaturas; los dos padres trinitizadores se vuelven uno solo en el nivel funcional último. Ningún ser creado del universo puede explicar plenamente este fenómeno asombroso; es una experiencia casi divina. Cuando el Padre y el Hijo se unieron para eternizar al Espíritu Infinito, después de lograr su propósito se volvieron inmediatamente como uno solo, y desde entonces siempre han sido uno solo. Aunque la unión trinitizadora de dos criaturas es del mismo estilo que la amplitud infinita de la unión perfecta de la Deidad del Padre Universal y del Hijo Eterno, la naturaleza de las repercusiones de una trinitización efectuada por las criaturas no es eterna; terminarán cuando las Deidades experienciales sean un hecho consumado.

\par
%\textsuperscript{(250.3)}
\textsuperscript{22:7.9} Aunque los padres de los hijos trinitizados por las criaturas se vuelven como uno solo en sus tareas universales, siguen siendo considerados como dos personalidades en la composición y en las listas nominales del Cuerpo de la Finalidad y de los Arquitectos del Universo Maestro. Durante la era universal en curso, el destino y la función de todos los padres unidos por la trinitización son inseparables; donde va el uno va el otro, y lo que hace el uno lo hace el otro. Si la biunificación parental afecta a un finalitario mortal (u otro) y a una personalidad del Paraíso-Havona, los seres parentales unidos no trabajan ni con los habitantes del Paraíso o de Havona ni con los finalitarios. Estas uniones mixtas se reúnen en un cuerpo especial compuesto por seres similares. Y en todas las uniones por trinitización, mixtas o de otro tipo, los seres parentales son conscientes el uno del otro, pueden comunicarse entre sí, y pueden desempeñar funciones que ninguno de los dos podría haber ejercido anteriormente.

\par
%\textsuperscript{(250.4)}
\textsuperscript{22:7.10} Los Siete Espíritus Maestros tienen autoridad para aprobar la unión trinitizante entre los finalitarios y las personalidades del Paraíso-Havona, y estos enlaces mixtos siempre tienen éxito. Los magníficos hijos resultantes trinitizados por estas criaturas representan unos conceptos que las criaturas eternas del Paraíso o las criaturas temporales del espacio no pueden comprender; de ahí que se conviertan en los pupilos de los Arquitectos del Universo Maestro. Estos hijos del destino trinitizados personifican unas ideas, unos ideales y una \textit{experiencia} que pertenecen aparentemente a una era futura del universo, y no tienen por ello un valor práctico inmediato ni para las administraciones de los superuniversos ni para la del universo central. Todos estos hijos excepcionales de los hijos del tiempo y de los ciudadanos de la eternidad se mantienen en reserva en Vicegerington, donde se dedican a estudiar los conceptos del tiempo y las realidades de la eternidad en un sector especial de la esfera ocupado por los colegios secretos del cuerpo de los Hijos Creadores.

\par
%\textsuperscript{(251.1)}
\textsuperscript{22:7.11} El Ser Supremo es la unificación de tres fases de la realidad de la Deidad: Dios Supremo, la unificación espiritual de ciertos aspectos finitos de la Trinidad del Paraíso; el Todopoderoso Supremo, la unificación del poder de los Creadores del gran universo; y la Mente Suprema, la contribución individual de la Fuente-Centro Tercera y de sus coordinados a la realidad del Ser Supremo. En sus aventuras de trinitización, las magníficas criaturas del universo central y del Paraíso se aventuran en una triple exploración de la Deidad del Supremo que tiene como resultado el nacimiento de tres órdenes de hijos trinitizados por las criaturas:

\par
%\textsuperscript{(251.2)}
\textsuperscript{22:7.12} 1. \textit{Los hijos trinitizados por los ascendentes.} En sus esfuerzos creativos, los finalitarios intentan trinitizar ciertas realidades conceptuales del Todopoderoso Supremo que han adquirido experiencialmente en su ascensión al Paraíso a través del tiempo y del espacio.

\par
%\textsuperscript{(251.3)}
\textsuperscript{22:7.13} 2. \textit{Los hijos trinitizados por las criaturas del Paraíso-Havona.} Los esfuerzos creativos de los Ciudadanos del Paraíso y de los havonianos tienen como resultado la trinitización de ciertos aspectos espirituales elevados del Ser Supremo que han adquirido experiencialmente en un trasfondo supersupremo que linda con el Último y el Eterno.

\par
%\textsuperscript{(251.4)}
\textsuperscript{22:7.14} 3. \textit{Los hijos del destino trinitizados.} Pero cuando un finalitario y un ciudadano del Paraíso-Havona trinitizan juntos una nueva criatura, este esfuerzo conjunto repercute en ciertas fases de la Mente Supremo-Última. Los hijos resultantes trinitizados por estas criaturas trascienden la creación; representan unas realidades de la Deidad Supremo-Última que no han sido alcanzadas de otra manera por experiencia y que, por lo tanto, son automáticamente de la incumbencia de los Arquitectos del Universo Maestro, los guardianes de aquellas cosas que trascienden los límites de la actividad creativa de la presente era del universo. Los hijos del destino trinitizados personifican ciertos aspectos de la función no revelada del Supremo-Último en el universo maestro. No sabemos mucho acerca de estos hijos conjuntos del tiempo y de la eternidad, pero sabemos mucho más de lo que nos está permitido revelar.

\section*{8. Los hijos trinitizados por las criaturas}
\par
%\textsuperscript{(251.5)}
\textsuperscript{22:8.1} Además de los hijos trinitizados por las criaturas examinados en esta narración, hay numerosas órdenes no reveladas de seres trinitizados por las criaturas ---los diversos descendientes de los múltiples enlaces entre los siete cuerpos finalitarios y las personalidades del Paraíso-Havona. Pero todos estos seres trinitizados por las criaturas, revelados y no revelados, son dotados de la personalidad por el Padre Universal.

\par
%\textsuperscript{(251.6)}
\textsuperscript{22:8.2} Cuando los nuevos hijos trinitizados por los ascendentes y por las personalidades del Paraíso-Havona son jóvenes e inexpertos, se les envía generalmente para que pasen largos períodos de servicio en las siete esferas paradisiacas del Espíritu Infinito, donde sirven bajo la tutela de los Siete Ejecutivos Supremos. Posteriormente pueden ser adoptados por los Hijos Instructores Trinitarios para recibir una formación adicional en los universos locales.

\par
%\textsuperscript{(251.7)}
\textsuperscript{22:8.3} Estos hijos adoptivos, que tienen su origen en las criaturas elevadas y glorificadas, son los aprendices, los ayudantes estudiantiles, de los Hijos Instructores; y en cuanto a su clasificación, a menudo se les cuenta temporalmente junto con estos Hijos. Pueden llevar a cabo, y así lo hacen, muchas nobles misiones abnegadas a favor de los reinos donde han elegido servir.

\par
%\textsuperscript{(251.8)}
\textsuperscript{22:8.4} En los universos locales, los Hijos Instructores pueden proponer a sus pupilos trinitizados por las criaturas para ser abrazados por la Trinidad del Paraíso. Al surgir de este abrazo como Hijos de la Perfección Trinitizados, entran al servicio de los Ancianos de los Días en los siete superuniversos, y éste es el actual destino conocido de este grupo único de seres dos veces trinitizados.

\par
%\textsuperscript{(252.1)}
\textsuperscript{22:8.5} No todos los hijos trinitizados por las criaturas son abrazados por la Trinidad; muchos de ellos se convierten en los asociados y embajadores de los Siete Espíritus Maestros del Paraíso, de los Espíritus Reflectantes de los superuniversos y de los Espíritus Madres de las creaciones locales. Otros pueden aceptar tareas especiales en la Isla eterna. Y otros aún pueden entrar en los servicios especiales de los mundos secretos del Padre y de las esferas paradisiacas del Espíritu. Muchos encuentran finalmente su camino en el cuerpo conjunto de los Hijos Trinitizados en el circuito interior de Havona.

\par
%\textsuperscript{(252.2)}
\textsuperscript{22:8.6} A excepción de los Hijos de la Perfección Trinitizados y de aquellos que se están reuniendo en Vicegerington, el destino supremo de todos los hijos trinitizados por las criaturas parece ser el de ingresar en el Cuerpo de los Finalitarios Trinitizados, uno de los siete Cuerpos Paradisiacos de la Finalidad.

\section*{9. Los Guardianes Celestiales}
\par
%\textsuperscript{(252.3)}
\textsuperscript{22:9.1} Los hijos trinitizados por las criaturas son abrazados por la Trinidad del Paraíso en clases de siete mil. Todos estos descendientes trinitizados de los humanos perfeccionados y de las personalidades del Paraíso-Havona son abrazados igualmente por las Deidades, pero son destinados a los superuniversos de acuerdo con los informes de sus antiguos profesores, los Hijos Instructores Trinitarios. Aquellos cuyo servicio es más aceptable son nombrados Ayudantes de los Hijos Elevados; aquellos cuya actuación es menos distinguida son denominados Guardianes Celestiales.

\par
%\textsuperscript{(252.4)}
\textsuperscript{22:9.2} Cuando estos seres únicos han sido abrazados por la Trinidad, se convierten en unos adjuntos valiosos para los gobiernos de los superuniversos. Están versados en los asuntos de la carrera ascendente, no por haber ascendido personalmente, sino como resultado de su servicio con los Hijos Instructores Trinitarios en los mundos del espacio.

\par
%\textsuperscript{(252.5)}
\textsuperscript{22:9.3} Cerca de mil millones de Guardianes Celestiales han sido nombrados en Orvonton. Están destinados principalmente en las administraciones de los Perfecciones de los Días en las sedes de los sectores mayores, y reciben la ayuda eficaz de un cuerpo de mortales ascendentes fusionados con el Hijo.

\par
%\textsuperscript{(252.6)}
\textsuperscript{22:9.4} Los Guardianes Celestiales son los funcionarios de los tribunales de los Ancianos de los Días, actuando como mensajeros judiciales y como portadores de las citaciones y de las decisiones de los diversos tribunales de los gobiernos superuniversales. Son los agentes de los Ancianos de los Días encargados de los arrestos; salen de Uversa para traer a los seres cuya presencia se necesita ante los jueces de los superuniversos; ejecutan las órdenes de detener a cualquier personalidad en el superuniverso. También acompañan a los mortales de los universos locales fusionados con el Espíritu cuando su presencia se necesita en Uversa por cualquier razón.

\par
%\textsuperscript{(252.7)}
\textsuperscript{22:9.5} Los Guardianes Celestiales y sus asociados, los Ayudantes de los Hijos Elevados, nunca han sido habitados por Ajustadores. Tampoco están fusionados con el Espíritu ni con el Hijo. Sin embargo, el abrazo de la Trinidad del Paraíso compensa el estado no fusionado de los Hijos de la Perfección Trinitizados. El abrazo de la Trinidad sólo puede actuar sobre la idea que está personificada en un hijo trinitizado por las criaturas, dejando al hijo abrazado sin otro tipo de cambio, pero esta limitación sólo se produce cuando es planificada de esta manera.

\par
%\textsuperscript{(252.8)}
\textsuperscript{22:9.6} Estos hijos dos veces trinitizados son unos seres maravillosos, pero no son tan polifacéticos ni tan fiables como sus asociados ascendentes; les falta esa enorme y profunda experiencia personal que el resto de los hijos que pertenecen a este grupo han adquirido elevándose efectivamente hasta la gloria desde los sombríos dominios del espacio. Nosotros, los de la carrera ascendente, los amamos y hacemos todo lo que podemos para compensar sus deficiencias, pero ellos hacen que siempre nos sintamos agradecidos por nuestro origen humilde y por nuestra capacidad para experimentar. Su buena voluntad para reconocer y admitir sus deficiencias en las realidades experimentables de la ascensión del universo es de una belleza trascendente y a veces de un patetismo de lo más conmovedor.

\par
%\textsuperscript{(253.1)}
\textsuperscript{22:9.7} Los Hijos de la Perfección Trinitizados están limitados, en contraste con otros hijos abrazados por la Trinidad, debido a que su capacidad experiencial está inhibida con respecto al espacio-tiempo. Son deficientes en experiencia, a pesar de su larga formación con los Ejecutivos Supremos y los Hijos Instructores, y si éste no fuera el caso, su saturación experiencial les impediría el ser dejados en reserva con vistas a adquirir experiencia en una era futura del universo. En toda la existencia universal simplemente no hay nada que pueda sustituir a la experiencia personal efectiva, y a estos hijos trinitizados por las criaturas se les mantiene en reserva para una función experiencial en alguna época futura del universo.

\par
%\textsuperscript{(253.2)}
\textsuperscript{22:9.8} He visto a menudo, en los mundos de las mansiones, que estos dignos oficiales de los altos tribunales del superuniverso miraban con nostalgia y atracción incluso a los recién llegados de los mundos evolutivos del espacio, de tal manera que uno no podía evitar darse cuenta de que estos poseedores de una trinitización no experiencial envidiaban realmente a sus hermanos, supuestamente menos afortunados, que ascienden el camino universal por medio de etapas de auténticas experiencias y de vivencias reales. A pesar de sus obstáculos y limitaciones, componen un cuerpo de trabajadores maravillosamente útiles y siempre dispuestos a la hora de ejecutar los complejos planes administrativos de los gobiernos de los superuniversos.

\section*{10. Los Ayudantes de los Hijos Elevados}
\par
%\textsuperscript{(253.3)}
\textsuperscript{22:10.1} Los Ayudantes de los Hijos Elevados son el grupo superior de hijos trinitizados y vueltos a trinitizar de los seres ascendentes glorificados del Cuerpo de los Mortales de la Finalidad y de sus eternos asociados, las personalidades del Paraíso-Havona. Están destinados al servicio superuniversal y ejercen su actividad como ayudantes personales de los hijos elevados de los gobiernos de los Ancianos de los Días. Se les podría denominar adecuadamente secretarios particulares. Actúan de vez en cuando como secretarios de las comisiones especiales y de otras asociaciones colectivas de hijos elevados. Sirven a los Perfeccionadores de la Sabiduría, a los Consejeros Divinos, a los Censores Universales, a los Mensajeros Poderosos, a Los Elevados en Autoridad y a Los que no tienen Nombre ni Número.

\par
%\textsuperscript{(253.4)}
\textsuperscript{22:10.2} Si al hablar de los Guardianes Celestiales he parecido llamar la atención sobre las limitaciones y los obstáculos de estos hijos dos veces trinitizados, permitidme que ahora llame la atención, con toda equidad, sobre su gran punto fuerte, el atributo que los hace casi inapreciables para nosotros. Estos seres deben su existencia misma al hecho de que son la personificación de un concepto único y supremo. Son la encarnación personificada de alguna idea divina, de algún ideal universal, que nunca antes había sido concebido, expresado o trinitizado. Y posteriormente han sido abrazados por la Trinidad; así pues, manifiestan y personifican realmente la sabiduría misma de la Trinidad divina en lo que se refiere a la idea-ideal de la existencia de su personalidad. En la medida en que este concepto particular se puede revelar a los universos, estas personalidades encarnan la totalidad de lo que cualquier inteligencia de criatura o de Creador tiene la posibilidad de concebir, expresar o demostrar. \textit{Son esa ideapersonificada.}

\par
%\textsuperscript{(253.5)}
\textsuperscript{22:10.3} ¿No podéis ver que estas concentraciones vivientes de un solo concepto supremo de la realidad universal pueden prestar un servicio incalculable a aquellos que están encargados de administrar los superuniversos?

\par
%\textsuperscript{(254.1)}
\textsuperscript{22:10.4} No hace mucho tiempo recibí la orden de dirigir una comisión de seis personalidades ---una de cada tipo de hijos elevados--- encargada de estudiar tres problemas relacionados con un grupo de nuevos universos en las regiones meridionales de Orvonton. Me hice plenamente consciente del valor de los Ayudantes de los Hijos Elevados cuando le solicité al jefe de esta orden en Uversa que asignara temporalmente unos secretarios de este tipo a mi comisión. La primera de nuestras ideas estaba representada por un Ayudante de los Hijos Elevados de Uversa, que fue destinado de inmediato a nuestro grupo. Nuestro segundo problema estaba incorporado en un Ayudante de los Hijos Elevados destinado en el superuniverso número tres. Recibimos mucha ayuda de esta fuente a través de la cámara de análisis, corrección y distribución de la información del universo central encargada de la coordinación y la diseminación del conocimiento esencial, pero no hay nada comparable a la ayuda que proporciona la presencia real de una personalidad que \textit{es} un concepto trinitizado en supremacía por las criaturas y trinitizado en finalidad por la Deidad. En cuanto a nuestro tercer problema, los archivos del Paraíso revelaron que dicha idea nunca había sido trinitizada por las criaturas.

\par
%\textsuperscript{(254.2)}
\textsuperscript{22:10.5} Los Ayudantes de los Hijos Elevados son unas personalizaciones únicas y originales de unos conceptos asombrosos y de unos ideales formidables. Como tales, son capaces de aportar de vez en cuando una iluminación inexpresable a nuestras deliberaciones. Cuando estoy trabajando en alguna tarea lejana en los universos del espacio, pensad en la ayuda que significa tener la suerte de contar con que está vinculado a mi misión un Ayudante de los Hijos Elevados que es la plenitud del concepto divino en lo que concierne al problema mismo que me han enviado a atacar y resolver; he tenido repetidas veces esta misma experiencia. La única dificultad que posee este plan es que ningún superuniverso puede tener una versión completa de estas ideas trinitizadas; sólo conseguimos una séptima parte de estos seres; así pues, aproximadamente sólo una vez de cada siete podemos disfrutar de la asociación personal de estos seres, incluso cuando los archivos indican que la idea ha sido trinitizada.

\par
%\textsuperscript{(254.3)}
\textsuperscript{22:10.6} Podríamos utilizar con gran ventaja en Uversa un número mucho mayor de estos seres. Debido a su valor para las administraciones de los superuniversos, animamos de todas las maneras posibles a los peregrinos del espacio, y también a los residentes del Paraíso, a que intenten la trinitización después de haberse aportado mutuamente aquellas realidades experienciales que son esenciales para llevar a cabo estas aventuras creativas.

\par
%\textsuperscript{(254.4)}
\textsuperscript{22:10.7} Actualmente tenemos en nuestro superuniverso cerca de un millón y cuarto de Ayudantes de los Hijos Elevados, y sirven en los sectores mayores así como en los menores, al igual que ejercen su actividad en Uversa. Nos acompañan muy a menudo en nuestras misiones a los universos lejanos. Los Ayudantes de los Hijos Elevados no están asignados de manera permanente a ningún Hijo ni a ninguna comisión. Circulan constantemente, sirviendo allí donde la idea o el ideal que ellos \textit{son} pueda favorecer mejor los objetivos eternos de la Trinidad del Paraíso, de la que han llegado a ser sus hijos.

\par
%\textsuperscript{(254.5)}
\textsuperscript{22:10.8} Son conmovedoramente afectuosos, magníficamente leales, exquisitamente inteligentes, supremamente sabios ---con relación a una sola idea--- y trascendentalmente humildes. Aunque pueden proporcionarnos el saber del universo en cuanto a su idea o ideal únicos, es casi patético observar cómo buscan el conocimiento y la información en una multitud de otros temas, aunque provengan de los mortales ascendentes.

\par
%\textsuperscript{(254.6)}
\textsuperscript{22:10.9} Y éste es el relato del origen, la naturaleza y la función de algunos seres llamados Hijos de Dios Trinitizados, y más especialmente de aquellos que han pasado por el abrazo divino de la Trinidad del Paraíso, y que luego han sido destinados al servicio de los superuniversos para ofrecer allí su cooperación sabia y comprensiva a los administradores de los Ancianos de los Días en sus esfuerzos infatigables por facilitar el progreso interior de los mortales ascendentes del tiempo hacia su destino inmediato en Havona y su meta final en el Paraíso.

\par
%\textsuperscript{(255.1)}
\textsuperscript{22:10.10} [Narrado por un Mensajero Poderoso del cuerpo revelador de Orvonton.]


\chapter{Documento 23. Los Mensajeros Solitarios}
\par
%\textsuperscript{(256.1)}
\textsuperscript{23:0.1} LOS Mensajeros Solitarios componen la legión personal y universal del Creador Conjunto; forman la orden primera y más antigua de Personalidades Superiores del Espíritu Infinito. Representan la acción creativa inicial del Espíritu Infinito actuando de forma solitaria con el fin de traer a la existencia a unos espíritus personales solitarios. Ni el Padre ni el Hijo participaron directamente en esta prodigiosa espiritualización.

\par
%\textsuperscript{(256.2)}
\textsuperscript{23:0.2} Estos mensajeros espirituales fueron personalizados en un solo episodio creativo, y su número es fijo. Aunque uno de estos seres extraordinarios está asociado conmigo en esta misión, no sé cuántas personalidades de este tipo existen en el universo de universos. Sólo conozco, de vez en cuando, cuántos están registrados y ejerciendo su actividad en ese momento dentro de la jurisdicción de nuestro superuniverso. Según el último informe de Uversa, observo que entonces había cerca de 7.690 billones de Mensajeros Solitarios trabajando dentro de las fronteras de Orvonton; y sospecho que esta cifra es considerablemente inferior a la séptima parte de su número total.

\section*{1. Naturaleza y origen de los Mensajeros Solitarios}
\par
%\textsuperscript{(256.3)}
\textsuperscript{23:1.1} Inmediatamente después de crear a los Siete Espíritus de los Circuitos de Havona, el Espíritu Infinito trajo a la existencia al inmenso cuerpo de los Mensajeros Solitarios. Ninguna parte de la creación universal es anterior a la existencia de los Mensajeros Solitarios, excepto el Paraíso y los circuitos de Havona; han desempeñado sus funciones en todo el gran universo desde casi la eternidad. Son fundamentales para llevar a cabo la técnica divina del Espíritu Infinito consistente en revelarse a las extensas creaciones del tiempo y del espacio y en ponerse en contacto personal con ellas.

\par
%\textsuperscript{(256.4)}
\textsuperscript{23:1.2} A pesar de que estos mensajeros existen desde los tiempos cercanos a la eternidad, todos son conscientes del comienzo de su individualidad. Son conscientes del tiempo, siendo los primeros seres creados por el Espíritu Infinito en poseer esta conciencia del tiempo. Son las primeras criaturas nacidas del Espíritu Infinito que fueron personalizadas en el tiempo y espiritualizadas en el espacio.

\par
%\textsuperscript{(256.5)}
\textsuperscript{23:1.3} Estos espíritus solitarios aparecieron en los albores del tiempo como seres espirituales totalmente desarrollados y perfectamente dotados. Todos son iguales, y no existen clases ni subdivisiones basadas en las variaciones personales. Sus clasificaciones están enteramente basadas en el tipo de trabajo al que se les destina de vez en cuando.

\par
%\textsuperscript{(256.6)}
\textsuperscript{23:1.4} Los mortales inician su camino como seres casi materiales en los mundos del espacio y ascienden interiormente hacia los Grandes Centros; estos espíritus solitarios inician su camino en el centro de todas las cosas y anhelan ser destinados a las creaciones lejanas, incluídos los mundos individuales de los universos locales más alejados, e incluso mucho más allá.

\par
%\textsuperscript{(256.7)}
\textsuperscript{23:1.5} Aunque se les llama Mensajeros Solitarios, no son espíritus solitarios, pero les gusta realmente trabajar a solas. Son los únicos seres de toda la creación que pueden disfrutar, y disfrutan, de una existencia solitaria, aunque disfrutan igualmente de su asociación con las poquísimas órdenes de inteligencias universales con las que pueden fraternizar.

\par
%\textsuperscript{(257.1)}
\textsuperscript{23:1.6} Los Mensajeros Solitarios no están aislados cuando efectúan su servicio; se encuentran constantemente en contacto con la riqueza intelectual de toda la creación puesto que son capaces de <<\textit{escuchar}>> todas las transmisiones de los reinos donde residen. También pueden intercomunicarse con los miembros de su propio cuerpo inmediato, con los seres que hacen el mismo tipo de trabajo en el mismo superuniverso. Podrían comunicarse con otros miembros de su orden, pero el consejo de los Siete Espíritus Maestros les ha ordenado que no lo hagan, y son un grupo leal; no desobedecen ni faltan a sus compromisos. No hay ningún dato de que un Mensajero Solitario se haya deslizado nunca en las tinieblas.

\par
%\textsuperscript{(257.2)}
\textsuperscript{23:1.7} Los Mensajeros Solitarios, al igual que los Directores del Poder Universal, figuran entre los poquísimos tipos de seres que trabajan en todos los reinos y que están exentos de ser arrestados o detenidos por los tribunales del tiempo y del espacio. No se les podría citar para que comparecieran ante nadie, salvo ante los Siete Espíritus Maestros, pero este consejo del Paraíso no ha sido llamado nunca, en todos los anales del universo maestro, para juzgar el caso de un Mensajero Solitario.

\par
%\textsuperscript{(257.3)}
\textsuperscript{23:1.8} Estos mensajeros que trabajan de forma solitaria son un grupo de seres creados fiables, independientes, polifacéticos, completamente espirituales y ampliamente compasivos, que proceden de la Fuente-Centro Tercera; actúan por autorización del Espíritu Infinito que reside en la Isla central del Paraíso y tal como está personalizado en las esferas sede de los universos locales. Comparten constantemente el circuito directo que emana del Espíritu Infinito, incluso cuando ejercen su actividad en las creaciones locales bajo la influencia inmediata de los Espíritus Madres de los universos locales.

\par
%\textsuperscript{(257.4)}
\textsuperscript{23:1.9} Estos Mensajeros Solitarios deben viajar y trabajar a solas por una razón técnica. Cuando están situados en un lugar fijo y durante cortos períodos de tiempo, pueden colaborar en un grupo, pero cuando se hallan así en compañía, están totalmente apartados del sostén y de la dirección de su circuito del Paraíso; se encuentran enteramente aislados. Cuando están en tránsito o trabajando en los circuitos del espacio y las corrientes del tiempo, si dos miembros o más de esta orden se hallan muy cerca los unos de los otros, los dos o todos ellos pierden su conexión con las fuerzas circulantes superiores. Sufren un <<\textit{cortocircuito}>>, tal como vosotros podríais describirlo en símbolos ilustrativos. Por consiguiente, poseen dentro de ellos de manera inherente un poder de alarma automática, una señal de peligro, que funciona infaliblemente para avisarlos de un riesgo de colisión y que los mantiene indefectiblemente a una distancia suficiente como para no provocar interferencias en su funcionamiento adecuado y eficaz. También poseen unos poderes inherentes y automáticos que detectan e indican la proximidad tanto de los Espíritus Inspirados Trinitarios como de los Ajustadores del Pensamiento divinos.

\par
%\textsuperscript{(257.5)}
\textsuperscript{23:1.10} Estos mensajeros no poseen el poder de extender o de reproducir su personalidad, pero no existe prácticamente ningún trabajo en los universos al que no puedan dedicarse y al que no puedan contribuir con algo esencial y útil. Son especialmente los grandes ahorradores de tiempo para aquellos que se ocupan de la administración de los asuntos universales; nos ayudan a todos, desde los más elevados hasta los más humildes.

\section*{2. Las funciones de los Mensajeros Solitarios}
\par
%\textsuperscript{(257.6)}
\textsuperscript{23:2.1} Los Mensajeros Solitarios no están vinculados de manera permanente a ningún individuo o grupo de personalidades celestiales. Siempre se les indica el servicio que han de realizar, y durante ese servicio trabajan bajo la supervisión directa de aquellos que dirigen los reinos a los que están vinculados. No poseen entre ellos ninguna organización o gobierno de ningún tipo; son Mensajeros \textit{Solitarios.}

\par
%\textsuperscript{(258.1)}
\textsuperscript{23:2.2} El Espíritu Infinito destina a los Mensajeros Solitarios a las siete divisiones de servicio siguientes:

\par
%\textsuperscript{(258.2)}
\textsuperscript{23:2.3} 1. Los Mensajeros de la Trinidad del Paraíso.

\par
%\textsuperscript{(258.3)}
\textsuperscript{23:2.4} 2. Los Mensajeros de los circuitos de Havona.

\par
%\textsuperscript{(258.4)}
\textsuperscript{23:2.5} 3. Los Mensajeros de los superuniversos.

\par
%\textsuperscript{(258.5)}
\textsuperscript{23:2.6} 4. Los Mensajeros de los universos locales.

\par
%\textsuperscript{(258.6)}
\textsuperscript{23:2.7} 5. Los exploradores en misiones no especificadas.

\par
%\textsuperscript{(258.7)}
\textsuperscript{23:2.8} 6. Los embajadores y emisarios en misiones especiales.

\par
%\textsuperscript{(258.8)}
\textsuperscript{23:2.9} 7. Los reveladores de la verdad.

\par
%\textsuperscript{(258.9)}
\textsuperscript{23:2.10} Estos mensajeros espirituales son intercambiables en todos los sentidos entre un tipo de servicio y otro; estos traslados tienen lugar constantemente. No existen distintas órdenes de Mensajeros Solitarios; son semejantes espiritualmente e iguales en todos los sentidos. Aunque generalmente los llamamos por su número, el Espíritu Infinito los conoce por sus nombres personales. El resto de nosotros los conocemos por el nombre o el número que describe su tarea actual.

\par
%\textsuperscript{(258.10)}
\textsuperscript{23:2.11} 1. \textit{Los Mensajeros de la Trinidad del Paraíso.} No tengo permiso para revelar muchas cosas sobre el trabajo del grupo de mensajeros asignados a la Trinidad. Son los servidores secretos y de confianza de las Deidades, y cuando les confían mensajes especiales que conciernen a la política no revelada y a la conducta futura de los Dioses, nunca se ha sabido que divulguen un secreto o que traicionen la confianza depositada en su orden. Referimos todo esto en este contexto no para jactarnos de su perfección, sino más bien para señalar que las Deidades pueden crear \textit{seres perfectos,} y así lo hacen.

\par
%\textsuperscript{(258.11)}
\textsuperscript{23:2.12} La confusión y el desorden existentes en Urantia no significan que los Gobernantes del Paraíso carezcan de interés o de capacidad para dirigir los asuntos de manera diferente. Los Creadores poseen el pleno poder de hacer de Urantia un verdadero paraíso, pero un Edén así no contribuiría a desarrollar aquellos caracteres fuertes, nobles y experimentados que los Dioses están forjando con tanta seguridad en vuestro mundo entre el yunque de la necesidad y el martillo de la angustia. Vuestras ansiedades y tristezas, vuestras dificultades y decepciones forman tanta parte del plan divino en vuestra esfera como lo forman la perfección exquisita y la adaptación infinita de todas las cosas al propósito supremo de los Dioses en los mundos del universo central y perfecto.

\par
%\textsuperscript{(258.12)}
\textsuperscript{23:2.13} 2. \textit{Los Mensajeros de los circuitos de Havona.} Durante toda la carrera ascendente seréis capaces de detectar la presencia de los Mensajeros Solitarios de manera vaga pero creciente, pero hasta que no lleguéis a Havona no los reconoceréis inequívocamente. Los primeros mensajeros que veréis frente a frente serán los de los circuitos de Havona.

\par
%\textsuperscript{(258.13)}
\textsuperscript{23:2.14} Los Mensajeros Solitarios disfrutan de unas relaciones especiales con los nativos de los mundos de Havona. Estos mensajeros, que tienen tantos obstáculos funcionales cuando están asociados los unos con los otros, pueden disfrutar de una comunión muy estrecha y personal con los nativos de Havona, y así lo hacen. Pero es totalmente imposible transmitir a la mente humana las satisfacciones supremas que produce el contacto entre la mente de estos seres divinamente perfectos y el espíritu de estas personalidades casi trascendentes.

\par
%\textsuperscript{(259.1)}
\textsuperscript{23:2.15} 3. \textit{Los Mensajeros de los superuniversos.} Los Ancianos de los Días, esas personalidades de origen trinitario que presiden los destinos de los siete superuniversos, esos tríos con poder divino y sabiduría administrativa, están abundantemente provistos de Mensajeros Solitarios. Los gobernantes trinos de un superuniverso sólo pueden comunicarse directa y personalmente con los gobernantes de otro por medio de esta orden de mensajeros. Los Mensajeros Solitarios son el único tipo disponible de inteligencias espirituales ---aparte quizás de los Espíritus Inspirados Trinitarios--- que pueden ser enviados directamente desde la sede de un superuniverso hasta la sede de otro. Todas las demás personalidades deben pasar por Havona y los mundos ejecutivos de los Espíritus Maestros para realizar estos viajes.

\par
%\textsuperscript{(259.2)}
\textsuperscript{23:2.16} Hay ciertos tipos de información que no se pueden obtener ni por medio de los Mensajeros de Gravedad, ni por reflectividad, ni por transmisión. Y cuando los Ancianos de los Días quieren saber con seguridad estas cosas, deben enviar a un Mensajero Solitario a la fuente del conocimiento. Mucho antes de que la vida estuviera presente en Urantia, el mensajero que ahora está asociado conmigo fue destinado a una misión fuera de Uversa en el universo central ---estuvo ausente de las listas nominales de Orvonton durante cerca de un millón de años, pero regresó a su debido tiempo con la información deseada.

\par
%\textsuperscript{(259.3)}
\textsuperscript{23:2.17} El servicio de los Mensajeros Solitarios en los superuniversos no tiene limitaciones; pueden actuar como ejecutores de los tribunales superiores o hacer acopio de información para el bien del reino. De todas las supercreaciones, es en Orvonton donde más disfrutan sirviendo, porque aquí las necesidades son mayores y las oportunidades de realizar esfuerzos heroicos se multiplican enormemente. Todos disfrutamos de la satisfacción de una actividad más completa en los reinos más necesitados.

\par
%\textsuperscript{(259.4)}
\textsuperscript{23:2.18} 4. \textit{Los Mensajeros de los universos locales.} Las ocupaciones de los Mensajeros Solitarios no tienen límites en los servicios de un universo local. Son los fieles reveladores de los móviles y de las intenciones del Espíritu Madre del universo local, aunque estén bajo la plena jurisdicción del Hijo Maestro reinante. Y esto es así para todos los mensajeros que trabajan en un universo local, ya sea que se encuentren de viaje partiendo directamente de la sede del universo, o que ejerzan temporalmente su actividad en conexión con los Padres de las Constelaciones, los Soberanos de los Sistemas o los Príncipes Planetarios. Antes de que todos los poderes se concentren entre las manos de un Hijo Creador en la época de su elevación como gobernante soberano de su universo, estos mensajeros de los universos locales trabajan bajo la dirección general de los Ancianos de los Días y son directamente responsables ante su representante residente, el Unión de los Días.

\par
%\textsuperscript{(259.5)}
\textsuperscript{23:2.19} 5. \textit{Los exploradores en misiones no especificadas.} Cuando el cuerpo de reserva de los Mensajeros Solitarios tiene un exceso de miembros, uno de los Siete Directores Supremos del Poder emite un llamamiento solicitando voluntarios para explorar; y nunca faltan voluntarios, puesto que les encanta ser enviados como exploradores libres y sin limitaciones para experimentar la emoción de descubrir los núcleos en vías de organización de los nuevos mundos y universos.

\par
%\textsuperscript{(259.6)}
\textsuperscript{23:2.20} Salen a investigar los indicios proporcionados por los observadores espaciales de los reinos. Las Deidades del Paraíso conocen sin duda la existencia de estos sistemas energéticos espaciales no descubiertos, pero nunca divulgan esta información. Si los Mensajeros Solitarios no exploraran y localizaran estos nuevos centros energéticos en vías de organización, estos fenómenos permanecerían desapercibidos durante mucho tiempo incluso para las inteligencias de los reinos adyacentes. Los Mensajeros Solitarios, como clase, son extremadamente sensibles a la gravedad; en consecuencia, a veces pueden detectar la presencia probable de planetas oscuros muy pequeños, los mundos mismos que están mejor adaptados para experimentar con la vida.

\par
%\textsuperscript{(260.1)}
\textsuperscript{23:2.21} Estos mensajeros exploradores en misiones no especificadas patrullan el universo maestro. Están constantemente fuera en expediciones de exploración en las regiones desconocidas de todo el espacio exterior. Una gran parte de la información que poseemos sobre las actividades de los reinos del espacio exterior la debemos a las exploraciones de los Mensajeros Solitarios, puesto que trabajan y estudian a menudo con los astrónomos celestiales.

\par
%\textsuperscript{(260.2)}
\textsuperscript{23:2.22} 6. \textit{Los embajadores y emisarios en misiones especiales.} Los universos locales situados dentro del mismo superuniverso intercambian habitualmente embajadores escogidos entre sus órdenes de filiación nativas. Pero para evitar retrasos, a los Mensajeros Solitarios se les pide con frecuencia que vayan como embajadores de una creación local a otra para representar e interpretar a un reino en el otro. Por ejemplo: cuando se descubre un reino recién habitado, puede encontrarse tan alejado en el espacio que tendrá que pasar mucho tiempo antes de que un embajador enserafinado pueda llegar hasta ese universo distante. Un ser enserafinado no puede sobrepasar de ninguna manera la velocidad de 899.370 kilómetros de Urantia por segundo de vuestro tiempo. Las estrellas masivas, las corrientes contrarias y los desvíos, así como las tangentes de atracción, tienden todas a retrasar esta velocidad, de manera que durante un largo viaje la velocidad alcanzará una media de unos 885.000 kilómetros por segundo.

\par
%\textsuperscript{(260.3)}
\textsuperscript{23:2.23} Cuando se pone de manifiesto que se necesitarán cientos de años para que un embajador nativo llegue a un universo local muy lejano, se pide con frecuencia a un Mensajero Solitario que se dirija inmediatamente allí para actuar como embajador interino. Los Mensajeros Solitarios pueden desplazarse muy rápidamente, no con independencia del tiempo y del espacio como lo hacen los Mensajeros de Gravedad, pero casi igual que ellos. También sirven en otras circunstancias como emisarios en misión especial.

\par
%\textsuperscript{(260.4)}
\textsuperscript{23:2.24} 7. \textit{Los reveladores de la verdad.} Los Mensajeros Solitarios consideran la tarea de revelar la verdad como el deber más elevado de su orden. De vez en cuando ejercen su actividad en esta capacidad, desde los superuniversos hasta los planetas individuales del espacio. Forman parte con frecuencia de las comisiones que se envían para ampliar la revelación de la verdad a los mundos y a los sistemas.

\section*{3. Los servicios de los Mensajeros Solitarios en el tiempo y el espacio}
\par
%\textsuperscript{(260.5)}
\textsuperscript{23:3.1} Los Mensajeros Solitarios son el tipo más elevado de personalidades perfectas y de confianza que se encuentra disponible en todos los reinos para transmitir rápidamente los mensajes importantes y urgentes cuando no es conveniente utilizar el servicio de transmisión o el mecanismo de la reflectividad. Sirven en una variedad sin fin de misiones, ayudando a los seres materiales y espirituales de los reinos, especialmente allí donde el elemento tiempo está implicado. De todas las órdenes destinadas a los servicios de los dominios superuniversales, ellos son los seres personalizados más elevados y más polifacéticos que están más cerca de desafiar el tiempo y el espacio.

\par
%\textsuperscript{(260.6)}
\textsuperscript{23:3.2} El universo está bien provisto de espíritus que utilizan la gravedad a fin de desplazarse; pueden ir a cualquier parte en cualquier momento ---instantáneamente--- pero no son personas. Algunos otros que se desplazan utilizando la gravedad son seres personales, tales como los Mensajeros de Gravedad y los Registradores Trascendentales, pero no están a la disposición de los administradores de los superuniversos o de los universos locales. Los mundos pululan de ángeles, de hombres y de otros seres extremadamente personales, pero están obstaculizados por el tiempo y el espacio: el límite de velocidad para la mayoría de los seres no enserafinados es de 299.790 kilómetros de vuestro mundo por segundo de vuestro tiempo; las criaturas intermedias y algunas otras pueden alcanzar una velocidad doble ---599.580 kilómetros por segundo--- y a menudo lo consiguen, mientras que los serafines y otros pueden atravesar el espacio a una velocidad triple, en torno a los 899.370 kilómetros por segundo. Sin embargo, no existen personalidades mensajeras o de transporte, a excepción de los Mensajeros Solitarios, que circulen entre las velocidades instantáneas de aquellos que utilizan la gravedad para desplazarse y las velocidades relativamente lentas de los serafines.

\par
%\textsuperscript{(261.1)}
\textsuperscript{23:3.3} Por eso a los Mensajeros Solitarios se les utiliza generalmente para los envíos y los servicios en aquellas situaciones en que la personalidad es esencial para el éxito de la misión, y en las que se desea evitar la pérdida de tiempo que ocasionaría el envío de cualquier otro tipo rápidamente disponible de mensajero personal. Son los únicos seres claramente personalizados que pueden sincronizarse con las corrientes universales combinadas del gran universo. Su velocidad para atravesar el espacio es variable y depende de una gran variedad de influencias interferentes, pero los registros demuestran que durante su viaje para llevar a cabo esta misión, mi mensajero asociado se desplazó a razón de 1.354.458.739.000 kilómetros vuestros por segundo de vuestro tiempo.

\par
%\textsuperscript{(261.2)}
\textsuperscript{23:3.4} Me siento totalmente incapaz de explicar al tipo de mente material cómo un espíritu puede ser una persona real y al mismo tiempo atravesar el espacio a esas velocidades asombrosas. Pero estos mismos Mensajeros Solitarios vienen efectivamente a Urantia, y parten de aquí, a estas velocidades incomprensibles; si esto no fuera un hecho, toda la economía de la administración universal estaría en verdad ampliamente privada de su elemento personal.

\par
%\textsuperscript{(261.3)}
\textsuperscript{23:3.5} Los Mensajeros Solitarios son capaces de actuar como líneas de comunicación de urgencia en todas las regiones lejanas del espacio, en aquellos reinos no incluídos en los circuitos establecidos del gran universo. Cuando un mensajero actúa así, puede transmitir un mensaje o enviar un impulso a través del espacio a otro mensajero que se encuentre a unos cien años luz de distancia, tal como los astrónomos de Urantia estiman las distancias estelares.

\par
%\textsuperscript{(261.4)}
\textsuperscript{23:3.6} De las miríadas de seres que cooperan con nosotros en la dirección de los asuntos del superuniverso, ninguno es más importante en utilidad práctica y en ayudarnos a ahorrar tiempo. En los universos del espacio tenemos que contar con los obstáculos del tiempo; de ahí el gran servicio que prestan los Mensajeros Solitarios, los cuales, gracias a sus prerrogativas personales de comunicación, son en cierto modo independientes del espacio, y en virtud de sus enormes velocidades de tránsito, son casi independientes del tiempo.

\par
%\textsuperscript{(261.5)}
\textsuperscript{23:3.7} No encuentro palabras para explicar a los mortales de Urantia cómo los Mensajeros Solitarios pueden no tener una forma y sin embargo poseer una personalidad real y definida. Aunque no tengan esa forma que se asociaría de manera natural con la personalidad, poseen una presencia espiritual que es discernible por todos los tipos superiores de seres espirituales. Los Mensajeros Solitarios son la única clase de seres que parecen poseer casi todas las ventajas de un espíritu sin forma, unidas a todas las prerrogativas de una personalidad totalmente desarrollada. Son auténticas personas, aunque dotadas de casi todos los atributos de una manifestación espiritual impersonal.

\par
%\textsuperscript{(261.6)}
\textsuperscript{23:3.8} En los siete superuniversos, todo aquello que tiende a liberar cada vez más a cualquier criatura de los obstáculos del tiempo y del espacio, disminuye proporcionalmente ---por lo general, pero no siempre--- las prerrogativas de su personalidad. Los Mensajeros Solitarios son una excepción a esta ley general. En sus actividades casi no tienen restricción para utilizar todas las vías ilimitadas de la expresión espiritual, el servicio divino, el ministerio personal y la comunicación cósmica. Si pudierais ver a estos seres extraordinarios a la luz de mi experiencia en la administración universal, comprenderíais lo difícil que sería coordinar los asuntos superuniversales si no fuera por su polifacética cooperación.

\par
%\textsuperscript{(262.1)}
\textsuperscript{23:3.9} Por mucho que el universo pueda agrandarse, es probable que nunca se creen más Mensajeros Solitarios. A medida que crecen los universos, la mayor cantidad de trabajo de la administración deberá ser efectuada cada vez más por otros tipos de ministros espirituales y por aquellos seres que tienen su origen en estas nuevas creaciones, tales como las criaturas de los Hijos Soberanos y de los Espíritus Madres de los universos locales.

\section*{4. El ministerio especial de los Mensajeros Solitarios}
\par
%\textsuperscript{(262.2)}
\textsuperscript{23:4.1} Los Mensajeros Solitarios parecen ser los coordinadores de la personalidad para todos los tipos de seres espirituales. Su ministerio ayuda a que todas las personalidades del extenso mundo espiritual sean semejantes. Contribuyen mucho a desarrollar en todos los seres espirituales una conciencia de identidad de grupo. Cada tipo de ser espiritual recibe el servicio de unos grupos especiales de Mensajeros Solitarios, los cuales fomentan la capacidad de dichos seres para comprender y fraternizar con todos los demás tipos y órdenes, por muy diferentes que sean.

\par
%\textsuperscript{(262.3)}
\textsuperscript{23:4.2} Los Mensajeros Solitarios demuestran una capacidad tan asombrosa para coordinar todos los tipos y órdenes de personalidades finitas ---e incluso para ponerse en contacto con el régimen absonito de los supercontroladores del universo maestro--- que algunos de nosotros suponen que la creación de estos mensajeros, efectuada por el Espíritu Infinito, está relacionada de alguna manera con la donación de la Mente Supremo-Última llevada a cabo por el Actor Conjunto.

\par
%\textsuperscript{(262.4)}
\textsuperscript{23:4.3} Cuando un finalitario y un Ciudadano del Paraíso cooperan para trinitizar a un <<\textit{hijo del tiempo y de la eternidad}>> ---una operación que afecta a los potenciales mentales no revelados del Supremo-Último--- y cuando esta personalidad no clasificada es enviada a Vicegerington, un Mensajero Solitario (supuesta repercusión bajo la forma de personalidad del otorgamiento de esa mente divina) siempre es nombrado como compañero-guardián de ese hijo trinitizado por las criaturas. Este mensajero acompaña al nuevo hijo del destino al mundo donde ha sido asignado y no abandona Vicegerington nunca más. Cuando está unido así a los destinos de un hijo del tiempo y de la eternidad, el Mensajero Solitario es trasladado para siempre a la supervisión exclusiva de los Arquitectos del Universo Maestro. No sabemos cuál será el futuro de esta asociación extraordinaria. Estas asociaciones de personalidades únicas han continuado reuniéndose en Vicegerington durante épocas enteras, pero ni siquiera una sola pareja ha salido nunca de allí.

\par
%\textsuperscript{(262.5)}
\textsuperscript{23:4.4} El número de Mensajeros Solitarios es fijo, pero la trinitización de los hijos del destino parece ser una técnica ilimitada. Puesto que cada hijo trinitizado del destino tiene asignado un Mensajero Solitario, nos parece que en algún momento del lejano futuro se agotará la provisión de mensajeros. ¿Quién se encargará de su trabajo en el gran universo? ¿Su servicio será asumido por algún progreso nuevo entre los Espíritus Inspirados Trinitarios? En alguna época lejana, ¿es que el gran universo va a ser administrado casi totalmente por los seres de origen trinitario, mientras que las criaturas de origen único y doble se marcharán a los reinos del espacio exterior? Si los mensajeros regresan a su antiguo servicio, ¿los acompañarán estos hijos del destino? ¿Cesarán las trinitizaciones entre los finalitarios y los habitantes del Paraíso-Havona cuando la provisión de Mensajeros Solitarios haya sido absorbida como compañeros-guardianes de estos hijos del destino? Todos nuestros eficaces Mensajeros Solitarios, ¿van a ser concentrados en Vicegerington? Estas personalidades espirituales extraordinarias, ¿van a estar eternamente asociadas con estos hijos trinitizados que tienen un destino no revelado? ¿Qué significado debemos darle al hecho de que estas parejas que se están reuniendo en Vicegerington se encuentren bajo la dirección exclusiva de esos poderosos seres rodeados de misterio, los Arquitectos del Universo Maestro? Nos hacemos estas preguntas y otras muchas similares, e interrogamos a otras numerosas órdenes de seres celestiales, pero no conocemos las respuestas.

\par
%\textsuperscript{(263.1)}
\textsuperscript{23:4.5} Esta operación, junto con muchos sucesos similares en la administración universal, indica sin lugar a dudas que el personal del gran universo, e incluso el del Paraíso y Havona, está sufriendo una reorganización precisa y segura en coordinación con, y con referencia a, las inmensas evoluciones energéticas que están teniendo lugar actualmente en todos los reinos del espacio exterior.

\par
%\textsuperscript{(263.2)}
\textsuperscript{23:4.6} Nos inclinamos a creer que el futuro eterno presenciará unos fenómenos de evolución universal que trascenderán de lejos todo lo que ha experimentado el eterno pasado. Y esperamos estas aventuras extraordinarias, al igual que vosotros deberíais hacerlo, con un intenso entusiasmo y una expectación cada vez mayor.

\par
%\textsuperscript{(263.3)}
\textsuperscript{23:4.7} [Presentado por un Consejero Divino procedente de Uversa.]


\chapter{Documento 24. Las personalidades superiores del Espíritu Infinito}
\par
%\textsuperscript{(264.1)}
\textsuperscript{24:0.1} EN Uversa clasificamos a todas las personalidades y entidades del Creador Conjunto en tres grandes divisiones: las Personalidades Superiores del Espíritu Infinito, las Huestes de Mensajeros del Espacio y los Espíritus Ministrantes del Tiempo, esos seres espirituales que se ocupan de enseñar y de aportar su ministerio a las criaturas volitivas del programa ascendente de progresión de los mortales.

\par
%\textsuperscript{(264.2)}
\textsuperscript{24:0.2} Las Personalidades Superiores del Espíritu Infinito que se mencionan en estas narraciones ejercen su actividad en todo el gran universo en siete divisiones:

\par
%\textsuperscript{(264.3)}
\textsuperscript{24:0.3} 1. Los Mensajeros Solitarios.

\par
%\textsuperscript{(264.4)}
\textsuperscript{24:0.4} 2. Los Supervisores de los Circuitos Universales.

\par
%\textsuperscript{(264.5)}
\textsuperscript{24:0.5} 3. Los Directores del Censo.

\par
%\textsuperscript{(264.6)}
\textsuperscript{24:0.6} 4. Los Ayudantes Personales del Espíritu Infinito.

\par
%\textsuperscript{(264.7)}
\textsuperscript{24:0.7} 5. Los Inspectores Asociados.

\par
%\textsuperscript{(264.8)}
\textsuperscript{24:0.8} 6. Los Centinelas Asignados.

\par
%\textsuperscript{(264.9)}
\textsuperscript{24:0.9} 7. Los Guías de los Graduados.

\par
%\textsuperscript{(264.10)}
\textsuperscript{24:0.10} Los Mensajeros Solitarios, los Supervisores de los Circuitos, los Directores del Censo y los Ayudantes Personales tienen la característica de poseer unos dones asombrosos de antigravedad. Los Mensajeros Solitarios no disponen de una sede general conocida; surcan el universo de universos. Los Supervisores de los Circuitos Universales y los Directores del Censo mantienen sus sedes en las capitales de los superuniversos. Los Ayudantes Personales del Espíritu Infinito están estacionados en la Isla central de Luz. Los Inspectores Asociados y los Centinelas Asignados están estacionados respectivamente en las capitales de los universos locales y en las de los sistemas que los componen. Los Guías de los Graduados residen en el universo de Havona y ejercen su actividad en todos sus mil millones de mundos. La mayor parte de estas personalidades superiores tienen puestos en los universos locales pero no están ligadas orgánicamente a la administración de los reinos evolutivos.

\par
%\textsuperscript{(264.11)}
\textsuperscript{24:0.11} De las siete clases que componen este grupo, sólo los Mensajeros Solitarios y quizás los Ayudantes Personales recorren el universo de universos. Partiendo del Paraíso hacia el exterior, a los Mensajeros Solitarios se les encuentra desde los circuitos de Havona hasta las capitales de los superuniversos, y desde allí, en todos los sectores y los universos locales, con sus subdivisiones, e incluso en los mundos habitados. Aunque los Mensajeros Solitarios pertenecen a las Personalidades Superiores del Espíritu Infinito, su origen, su naturaleza y su servicio han sido analizados en el documento anterior.

\section*{1. Los Supervisores de los Circuitos Universales}
\par
%\textsuperscript{(265.1)}
\textsuperscript{24:1.1} Las inmensas corrientes de poder del espacio y los circuitos de la energía espiritual pueden dar la impresión de que funcionan de manera automática; pueden parecer que actúan sin obstáculos ni trabas, pero éste no es el caso. Todos estos formidables sistemas de energía están bajo control; están sometidos a una supervisión inteligente. Los Supervisores de los Circuitos Universales no se ocupan del ámbito de la energía puramente física o material ---terreno que pertenece a los Directores del Poder Universal--- sino de los circuitos de la energía espiritual relativa y de aquellos circuitos modificados que son esenciales para mantener tanto a los seres espirituales muy desarrollados como al tipo morontial, o de transición, de criaturas inteligentes. Los supervisores no dan origen a los circuitos de energía y de superesencia de la divinidad, pero tienen que ver en general con todos los circuitos espirituales superiores del tiempo y de la eternidad y con todos los circuitos espirituales relativos relacionados con la administración de las partes componentes del gran universo. Dirigen y manipulan, fuera de la Isla del Paraíso, todos estos circuitos de energía espiritual.

\par
%\textsuperscript{(265.2)}
\textsuperscript{24:1.2} Los Supervisores de los Circuitos Universales fueron creados exclusivamente por el Espíritu Infinito y actúan únicamente como agentes del Actor Conjunto. Están personalizados para el servicio en las cuatro órdenes siguientes:

\par
%\textsuperscript{(265.3)}
\textsuperscript{24:1.3} 1. Los Supervisores Supremos de los Circuitos.

\par
%\textsuperscript{(265.4)}
\textsuperscript{24:1.4} 2. Los Supervisores Asociados de los Circuitos.

\par
%\textsuperscript{(265.5)}
\textsuperscript{24:1.5} 3. Los Supervisores Secundarios de los Circuitos.

\par
%\textsuperscript{(265.6)}
\textsuperscript{24:1.6} 4. Los Supervisores Terciarios de los Circuitos.

\par
%\textsuperscript{(265.7)}
\textsuperscript{24:1.7} El número de los supervisores supremos de Havona y de los supervisores asociados de los siete superuniversos está al completo; ya no se crean más seres de estas órdenes. El número de supervisores supremos es de siete y están estacionados en los mundos piloto de los siete circuitos de Havona. Los circuitos de los siete superuniversos están a cargo de un grupo maravilloso de siete supervisores asociados, que mantienen sus sedes en las siete esferas paradisiacas del Espíritu Infinito, en los mundos de los Siete Ejecutivos Supremos. Desde allí supervisan y dirigen los circuitos de los superuniversos del espacio.

\par
%\textsuperscript{(265.8)}
\textsuperscript{24:1.8} En estas esferas paradisiacas del Espíritu, los siete supervisores asociados de los circuitos y la primera orden de los Centros Supremos del Poder efectúan una conexión que, bajo la dirección de los Ejecutivos Supremos, conduce a la coordinación subparadisiaca de todos los circuitos materiales y espirituales que salen hacia los siete superuniversos.

\par
%\textsuperscript{(265.9)}
\textsuperscript{24:1.9} En los mundos sede de cada superuniverso se encuentran estacionados los supervisores secundarios encargados de los universos locales del tiempo y del espacio. Los sectores mayores y menores son divisiones administrativas de los supergobiernos, pero no se ocupan del asunto de supervisar la energía espiritual. No sé cuántos supervisores secundarios de los circuitos hay en el gran universo, pero en Uversa se encuentran 84.691 seres de este tipo. Los supervisores secundarios son creados constantemente; de vez en cuando aparecen en grupos de setenta en los mundos de los Ejecutivos Supremos. Los obtenemos a petición nuestra cuando nos disponemos a establecer los distintos circuitos de energía espiritual y de poder de conexión para los nuevos universos que evolucionan bajo nuestra jurisdicción.

\par
%\textsuperscript{(265.10)}
\textsuperscript{24:1.10} Un supervisor terciario de los circuitos ejerce su función en el mundo sede de cada universo local. Esta orden, al igual que los supervisores secundarios, es creada continuamente, siéndolo en grupos de setecientos. Los Ancianos de los Días destinan a sus miembros a los universos locales.

\par
%\textsuperscript{(266.1)}
\textsuperscript{24:1.11} Los supervisores de los circuitos son creados para sus tareas específicas y sirven eternamente en los grupos donde han sido destinados originalmente. No se turnan en su servicio y, en consecuencia, efectúan un estudio secular de los problemas que encuentran en los reinos donde han sido destinados originalmente. Por ejemplo: el supervisor terciario de los circuitos N{\textdegree} 572.842 ha ejercido su actividad en Salvington desde el principio de la concepción de vuestro universo local, y es miembro del estado mayor personal de Miguel de Nebadon.

\par
%\textsuperscript{(266.2)}
\textsuperscript{24:1.12} Tanto si actúan en los universos locales como si lo hacen en los universos superiores, los supervisores de los circuitos dirigen todo lo relacionado con los circuitos adecuados que se deben emplear para transmitir todos los mensajes espirituales y para el tránsito de todas las personalidades. En su trabajo de supervisión de los circuitos, estos seres eficaces utilizan todos los agentes, fuerzas y personalidades del universo de universos. Emplean las <<\textit{elevadas personalidades espirituales no reveladas que controlan los circuitos}>>, y reciben la hábil ayuda de numerosas agrupaciones compuestas de personalidades del Espíritu Infinito. Son ellos los que aislarían a un mundo evolutivo si su Príncipe Planetario se rebelara contra el Padre Universal y su Hijo vicegerente. Son capaces de excluir a cualquier mundo de ciertos circuitos universales del tipo espiritual más elevado, pero no pueden anular las corrientes materiales de los directores del poder.

\par
%\textsuperscript{(266.3)}
\textsuperscript{24:1.13} Los Supervisores de los Circuitos Universales tienen una relación con los circuitos espirituales un tanto similar a la de los Directores del Poder Universal con los circuitos materiales. Las dos órdenes son complementarias, y juntas aseguran la supervisión de todos los circuitos espirituales y materiales que las criaturas pueden controlar y manipular.

\par
%\textsuperscript{(266.4)}
\textsuperscript{24:1.14} Los supervisores de los circuitos ejercen cierta supervisión sobre los circuitos mentales que están asociados con el espíritu, poco más o menos como los directores del poder poseen cierta jurisdicción sobre las fases de la mente que están asociadas con la energía física ---la mente maquinal. En general, las funciones de cada orden se acrecientan mediante su conexión con la otra, pero los circuitos de la mente pura no están sujetos a la supervisión de ninguna de las dos. Las dos órdenes tampoco están coordinadas; en todas sus múltiples tareas, los Supervisores de los Circuitos Universales están sometidos a los Siete Directores Supremos del Poder y a sus subordinados.

\par
%\textsuperscript{(266.5)}
\textsuperscript{24:1.15} Aunque dentro de sus órdenes respectivas los supervisores de los circuitos son enteramente semejantes, todos son individuos diferentes. Son seres verdaderamente personales, pero poseen un tipo de personalidad que es distinta a la otorgada por el Padre, y que no se encuentra en ningún otro tipo de criatura en toda la existencia universal.

\par
%\textsuperscript{(266.6)}
\textsuperscript{24:1.16} Aunque los reconoceréis y los conoceréis durante vuestro viaje hacia el interior, es decir hacia el Paraíso, no tendréis relaciones personales con ellos. Son los supervisores de los circuitos, y se ocupan estricta y eficazmente de sus tareas. Tratan únicamente con aquellas personalidades y entidades que vigilan aquellas actividades que están relacionadas con los circuitos sujetos a su supervisión.

\section*{2. Los Directores del Censo}
\par
%\textsuperscript{(266.7)}
\textsuperscript{24:2.1} A pesar de que la mente cósmica de la Inteligencia Universal conoce la presencia y el paradero de todas las criaturas \textit{pensantes,} en el universo de universos se encuentra operativo un método independiente de llevar la cuenta de todas las criaturas \textit{volitivas.}

\par
%\textsuperscript{(266.8)}
\textsuperscript{24:2.2} Los Directores del Censo son una creación especial y concluida del Espíritu Infinito, y no conocemos el número que existe de ellos. Son creados de tal manera que pueden mantener un sincronismo perfecto con la técnica de la reflectividad de los superuniversos, mientras que al mismo tiempo son personalmente sensibles y reactivos a la \textit{voluntad} inteligente. Mediante una técnica no comprendida del todo, estos directores se vuelven inmediatamente conscientes del nacimiento de la voluntad en cualquier parte del gran universo. Por lo tanto son siempre capaces de indicarnos el número, la naturaleza y el paradero de todas las criaturas volitivas en cualquier parte de la creación central y de los siete superuniversos. Pero no ejercen su actividad en el Paraíso; allí no hay necesidad de ellos. En el Paraíso, el conocimiento es inherente; las Deidades conocen todas las cosas.

\par
%\textsuperscript{(267.1)}
\textsuperscript{24:2.3} En Havona trabajan siete Directores del Censo, y cada uno de ellos está estacionado en el mundo piloto de cada circuito de Havona. A excepción de estos siete y de las reservas de su orden que se encuentran en los mundos paradisiacos del Espíritu, todos los Directores del Censo desempeñan sus funciones bajo la jurisdicción de los Ancianos de los Días.

\par
%\textsuperscript{(267.2)}
\textsuperscript{24:2.4} Un Director del Censo ejerce como presidente en la sede de cada superuniverso, y bajo el mando de este director general hay miles y miles de directores, uno en la capital de cada universo local. Todas las personalidades de esta orden son iguales, excepto las de los mundos piloto de Havona y los siete jefes superuniversales.

\par
%\textsuperscript{(267.3)}
\textsuperscript{24:2.5} En el séptimo superuniverso hay cien mil Directores del Censo. Y este número está compuesto enteramente de aquellos que son destinables a los universos locales; no incluye al estado mayor personal de Usatia, el jefe superuniversal de todos los directores de Orvonton. Usatia, al igual que los otros jefes superuniversales, no está directamente sintonizado con el registro de la voluntad inteligente. Únicamente está sintonizado con sus subordinados estacionados en los universos de Orvonton; actúa pues como una magnífica personalidad totalizadora de los informes que llegan desde las capitales de las creaciones locales.

\par
%\textsuperscript{(267.4)}
\textsuperscript{24:2.6} Los archivistas oficiales de Uversa inscriben de vez en cuando en sus anales el estado del superuniverso tal como lo indican los registros en y sobre la personalidad de Usatia. Estos datos censales pertenecen de manera autóctona a los superuniversos; estos informes no se transmiten ni a Havona ni al Paraíso.

\par
%\textsuperscript{(267.5)}
\textsuperscript{24:2.7} Los Directores del Censo sólo se ocupan de los seres humanos ---así como de otras criaturas volitivas--- para registrar el hecho de que la voluntad funciona. No se ocupan de la historia de vuestra vida ni de vuestras obras; no son en ningún sentido unas personalidades que registran. El Director del Censo de Nebadon, número 81.412 de Orvonton, estacionado actualmente en Salvington, es personalmente consciente y conocedor en este mismo momento de vuestra presencia viviente aquí en Urantia; y proporcionará a los registros la confirmación de vuestra muerte en el momento en que dejéis de actuar como criatura volitiva.

\par
%\textsuperscript{(267.6)}
\textsuperscript{24:2.8} Los Directores del Censo registran la existencia de una nueva criatura volitiva cuando ésta efectúa su primer acto de voluntad; indican la muerte de una criatura volitiva cuando tiene lugar su último acto de voluntad. La aparición parcial de la voluntad que se observa en las reacciones de algunos animales superiores no pertenece al ámbito de los Directores del Censo. Sólo llevan la cuenta de las auténticas criaturas volitivas, y sólo reaccionan al \textit{funcionamiento de la voluntad.} No sabemos con exactitud cómo registran el funcionamiento de la voluntad.

\par
%\textsuperscript{(267.7)}
\textsuperscript{24:2.9} Estos seres han sido siempre, y siempre serán, los Directores del Censo. Serían relativamente ineficaces en cualquier otra división del trabajo universal. Pero en su actividad son infalibles; no fallan nunca ni tampoco falsifican. Y a pesar de sus poderes maravillosos y de sus increíbles prerrogativas, son personas; tienen una presencia y una forma espirituales reconocibles.

\section*{3. Los Ayudantes Personales del Espíritu Infinito}
\par
%\textsuperscript{(268.1)}
\textsuperscript{24:3.1} No tenemos ningún conocimiento auténtico sobre el momento o la manera en que los Ayudantes Personales fueron creados. Su número debe ser enorme, pero no figura en los archivos de Uversa. Partiendo de unas deducciones prudentes basadas en lo que sabemos sobre su trabajo, me atrevo a estimar que su número se eleva a muchos billones. Mantenemos la opinión de que el Espíritu Infinito no tiene límites numéricos en lo que se refiere a la creación de estos Ayudantes Personales.

\par
%\textsuperscript{(268.2)}
\textsuperscript{24:3.2} Los Ayudantes Personales del Espíritu Infinito existen para ayudar exclusivamente a la presencia paradisiaca de la Tercera Persona de la Deidad. Aunque están vinculados directamente al Espíritu Infinito y situados en el Paraíso, van y vienen como relámpagos hasta las partes más alejadas de la creación. Dondequiera que lleguen los circuitos del Creador Conjunto, estos Ayudantes Personales pueden aparecer con el objeto de ejecutar las órdenes del Espíritu Infinito. Atraviesan el espacio casi como lo hacen los Mensajeros Solitarios, pero no son personas en el mismo sentido que los mensajeros.

\par
%\textsuperscript{(268.3)}
\textsuperscript{24:3.3} Todos los Ayudantes Personales son iguales e idénticos; no revelan ninguna diferenciación en su individualidad. Aunque el Actor Conjunto los mira como verdaderas personalidades, para los demás es difícil considerarlos como personas reales; no manifiestan una presencia espiritual a los otros seres espirituales. Los seres de origen paradisiaco son siempre conscientes de la proximidad de estos Ayudantes; pero no reconocemos la presencia de su personalidad. La ausencia de una forma que indique su presencia los hace indudablemente aún más útiles para la Tercera Persona de la Deidad.

\par
%\textsuperscript{(268.4)}
\textsuperscript{24:3.4} De todas las órdenes reveladas de seres espirituales que tienen su origen en el Espíritu Infinito, los Ayudantes Personales son casi los únicos que no encontraréis durante vuestra ascensión hacia el interior, es decir hacia el Paraíso.

\section*{4. Los Inspectores Asociados}
\par
%\textsuperscript{(268.5)}
\textsuperscript{24:4.1} Los Siete Ejecutivos Supremos, que se encuentran en las siete esferas paradisiacas del Espíritu Infinito, actúan colectivamente como un consejo administrativo de superdirectores para los siete superuniversos. Los Inspectores Asociados son la expresión personal de la autoridad de los Ejecutivos Supremos para los universos locales del tiempo y del espacio. Estos altos observadores de los asuntos de las creaciones locales son los descendientes conjuntos del Espíritu Infinito y de los Siete Espíritus Maestros del Paraíso. En una época cercana a la eternidad fueron personalizados setecientos mil, y su cuerpo de reserva reside en el Paraíso.

\par
%\textsuperscript{(268.6)}
\textsuperscript{24:4.2} Los Inspectores Asociados trabajan bajo la supervisión directa de los Siete Ejecutivos Supremos y son sus poderosos representantes personales ante los universos locales del tiempo y del espacio. Hay un inspector estacionado en la esfera sede de cada creación local, y está estrechamente asociado al Unión de los Días que reside allí.

\par
%\textsuperscript{(268.7)}
\textsuperscript{24:4.3} Los Inspectores Asociados sólo reciben los informes y las recomendaciones de sus subordinados, los Centinelas Asignados, estacionados en las capitales de los sistemas locales de mundos habitados, mientras que sólo presentan sus informes a su superior inmediato, el Ejecutivo Supremo del superuniverso interesado.

\section*{5. Los Centinelas Asignados}
\par
%\textsuperscript{(268.8)}
\textsuperscript{24:5.1} Los Centinelas Asignados son las personalidades coordinadoras y los representantes de enlace de los Siete Ejecutivos Supremos. Fueron persona-lizados en el Paraíso por el Espíritu Infinito y fueron creados para los fines específicos a los que fueron destinados. Su número es fijo, y existen exactamente siete mil millones de estos seres.

\par
%\textsuperscript{(269.1)}
\textsuperscript{24:5.2} Al igual que un Inspector Asociado representa a los Siete Ejecutivos Supremos ante un universo local entero, en cada uno de los diez mil sistemas de esa creación local hay un Centinela Asignado que actúa como representante directo del lejano y supremo consejo de supercontrol para los asuntos de los siete superuniversos. Los centinelas que están de servicio en los gobiernos de los sistemas locales de Orvonton actúan bajo la autoridad directa del Ejecutivo Supremo Número Siete, el coordinador del séptimo superuniverso. Pero en su organización administrativa, todos los centinelas nombrados en un universo local están subordinados al Inspector Asociado estacionado en la sede central de ese universo.

\par
%\textsuperscript{(269.2)}
\textsuperscript{24:5.3} Dentro de una creación local, los Centinelas Asignados sirven por turnos, siendo trasladados de sistema en sistema. Habitualmente se les cambia de puesto cada milenio del tiempo del universo local. Figuran entre las personalidades de mayor categoría estacionadas en la capital de un sistema, pero nunca participan en las deliberaciones que afectan a los asuntos del sistema. En los sistemas locales sirven como jefes de oficio de los veinticuatro administradores procedentes de los mundos evolutivos, pero aparte de esto, los mortales ascendentes tienen poco contacto con ellos. Los centinelas se ocupan casi exclusivamente de mantener plenamente informado al Inspector Asociado de su universo sobre todas las cuestiones relacionadas con el bienestar y el estado de los sistemas donde están destinados.

\par
%\textsuperscript{(269.3)}
\textsuperscript{24:5.4} Los Centinelas Asignados y los Inspectores Asociados no informan a los Ejecutivos Supremos a través de la sede de un superuniverso. Son responsables únicamente ante el Ejecutivo Supremo del superuniverso interesado; sus actividades son distintas a las de la administración de los Ancianos de los Días.

\par
%\textsuperscript{(269.4)}
\textsuperscript{24:5.5} Los Ejecutivos Supremos, los Inspectores Asociados y los Centinelas Asignados, junto con los omniafines y una multitud de personalidades no reveladas, constituyen un sistema eficaz, directo y centralizado, pero muy extenso, de coordinación consultiva y administrativa para todo el gran universo de cosas y de seres.

\section*{6. Los Guías de los Graduados}
\par
%\textsuperscript{(269.5)}
\textsuperscript{24:6.1} Los Guías de los Graduados, como grupo, patrocinan y dirigen la importante universidad de enseñanza técnica y de formación espiritual que es tan esencial para que los mortales alcancen la meta de todos los tiempos: Dios, el descanso, y luego una eternidad de servicio perfeccionado. Estos seres extremadamente personales reciben su nombre de la naturaleza y la finalidad de su trabajo. Se dedican exclusivamente a las tareas de guiar a los graduados mortales de los superuniversos del tiempo a través del programa de enseñanza y de formación de Havona, que sirve para preparar a los peregrinos ascendentes para que sean admitidos en el Paraíso y en el Cuerpo de la Finalidad.

\par
%\textsuperscript{(269.6)}
\textsuperscript{24:6.2} No me está prohibido indicaros el trabajo de estos Guías de los Graduados, pero es tan ultraespiritual que desespero de ser capaz de describir adecuadamente a la mente material una idea de sus múltiples actividades. En los mundos de las mansiones, después de que se amplíe vuestro campo visual y de que estéis liberados de las trabas de las comparaciones materiales, podréis empezar a comprender el significado de esas realidades que <<\textit{el ojo no puede ver ni el oído oír, y que no han existido nunca en los conceptos de la mente humana}>>\footnote{\textit{Lo que el ojo no puede ver ni el oído oír}: Is 64:4.}, e incluso aquellas cosas que <<\textit{Dios ha preparado para aquellos que aman estas verdades eternas}>>\footnote{\textit{Lo que Dios ha preparado}: 1 Co 2:9.}. No siempre estaréis tan limitados en el alcance de vuestra visión y de vuestra comprensión espiritual.

\par
%\textsuperscript{(270.1)}
\textsuperscript{24:6.3} Los Guías de los Graduados se ocupan de dirigir a los peregrinos del tiempo a través de los siete circuitos de los mundos de Havona. El guía que os acoja a vuestra llegada al mundo receptor del circuito exterior de Havona permanecerá con vosotros durante toda vuestra carrera en los circuitos celestiales. Aunque os asociaréis con otras innumerables personalidades durante vuestra estancia en los mil millones de mundos, vuestro Guía de los Graduados os seguirá hasta el final de vuestra progresión en Havona y presenciará vuestra entrada en el sueño final del tiempo, el sueño de transición a la eternidad hacia la meta del Paraíso, donde, cuando os despertéis, seréis recibidos por el Compañero Paradisiaco encargado de daros la bienvenida y quizás de permanecer con vosotros hasta que seáis aceptados como miembros del Cuerpo de los Mortales de la Finalidad.

\par
%\textsuperscript{(270.2)}
\textsuperscript{24:6.4} El número de Guías de los Graduados sobrepasa la capacidad de comprensión de la mente humana, y continúan apareciendo. Su origen es un poco misterioso. No han existido desde la eternidad; aparecen misteriosamente a medida que se necesitan. No existe ningún dato sobre un Guía de los Graduados en todos los reinos del universo central hasta aquella fecha lejana en que el primer peregrino mortal de todos los tiempos se abrió paso hasta el cinturón exterior de la creación central. En el momento en que llegó al mundo piloto del circuito exterior, fue recibido con saludos amistosos por Malvorian, el primer Guía de los Graduados, que es actualmente el jefe de su consejo supremo y el director de su inmensa organización educativa.

\par
%\textsuperscript{(270.3)}
\textsuperscript{24:6.5} En los archivos paradisiacos de Havona, en la sección denominada <<\textit{Guías de los Graduados}>>, aparece esta anotación inicial:

\par
%\textsuperscript{(270.4)}
\textsuperscript{24:6.6} <<\textit{Y Malvorian, el primero de esta orden, acogió e instruyó al peregrino que descubrió Havona y le condujo desde los circuitos exteriores de experiencia inicial, paso a paso y circuito tras circuito, hasta que se halló en la presencia misma de la Fuente y Destino de toda personalidad, cruzando posteriormente el umbral de la eternidad hacia el Paraíso}>>.

\par
%\textsuperscript{(270.5)}
\textsuperscript{24:6.7} En aquella época tan lejana yo estaba vinculado al servicio de los Ancianos de los Días en Uversa, y todos nos regocijamos en la seguridad de que, con el tiempo, los peregrinos de nuestro superuniverso llegarían a Havona. Durante eras nos habían enseñado que las criaturas evolutivas del espacio alcanzarían el Paraíso, y la emoción de todos los tiempos recorrió las cortes celestiales cuando el primer peregrino llegó realmente.

\par
%\textsuperscript{(270.6)}
\textsuperscript{24:6.8} El nombre de este peregrino que descubrió Havona es \textit{Grandfanda,} y procedía del planeta 341 del sistema 84 de la constelación 62 del universo local 1.131 situado en el superuniverso número uno. Su llegada fue la señal para establecer el servicio de transmisión del universo de universos. Hasta entonces sólo habían funcionado las transmisiones de los superuniversos y de los universos locales, pero el anuncio de la llegada de Grandfanda a las puertas de Havona señaló la inauguración de los <<\textit{informes espaciales de gloria}>>, llamados así porque la transmisión universal inicial informó de la llegada a Havona del primer ser evolutivo que había logrado entrar en la meta de la existencia ascendente.

\par
%\textsuperscript{(270.7)}
\textsuperscript{24:6.9} Los Guías de los Graduados no dejan nunca los mundos de Havona; están dedicados al servicio de los peregrinos graduados del tiempo y del espacio. Algún día os encontraréis con estos nobles seres cara a cara si no rechazáis el plan seguro y totalmente perfeccionado destinado a llevar a cabo vuestra supervivencia y vuestra ascensión.

\section*{7. El origen de los Guías de los Graduados}
\par
%\textsuperscript{(270.8)}
\textsuperscript{24:7.1} Aunque la evolución no es la regla del universo central, creemos que los Guías de los Graduados son los miembros perfeccionados, o más experimentados, de otra orden de criaturas del universo central, los Servitales de Havona. Los Guías de los Graduados manifiestan una compasión tan amplia y tal capacidad para comprender a las criaturas ascendentes, que estamos convencidos de que han adquirido esta cultura sirviendo efectivamente en los reinos superuniversales como Servitales Havonianos del ministerio universal. Si esta idea no es correcta, ¿cómo podemos explicar entonces la desaparición continua de los servitales más antiguos o más experimentados?

\par
%\textsuperscript{(271.1)}
\textsuperscript{24:7.2} Un servital estará mucho tiempo ausente de Havona efectuando una tarea superuniversal, habiendo participado previamente en muchas de estas misiones; regresará a su hogar, recibirá el privilegio de un <<\textit{contacto personal}>> con el Resplandor Central del Paraíso, será abrazado por las Personas Luminosas, y desaparecerá al reconocimiento de sus compañeros espirituales para no volver a aparecer nunca más entre sus semejantes.

\par
%\textsuperscript{(271.2)}
\textsuperscript{24:7.3} Al regresar del servicio superuniversal, un Servital de Havona puede disfrutar de numerosos abrazos divinos y salir simplemente de ellos como un servital elevado. El hecho de experimentar el abrazo luminoso no significa necesariamente que el servital deba convertirse en un Guía de los Graduados, pero casi una cuarta parte de los que alcanzan el abrazo divino no regresan nunca al servicio de los reinos.

\par
%\textsuperscript{(271.3)}
\textsuperscript{24:7.4} En los archivos superiores aparece una serie de anotaciones como la siguiente:

\par
%\textsuperscript{(271.4)}
\textsuperscript{24:7.5} <<\textit{Y el servital número 842.842.682.846.782 de Havona, llamado Sudna, volvió del servicio superuniversal, fue recibido en el Paraíso, conoció al Padre, entró en el abrazo divino y ya no existe}>>.

\par
%\textsuperscript{(271.5)}
\textsuperscript{24:7.6} Cuando una anotación así aparece en los archivos, la carrera de ese servital ha terminado. Pero exactamente tres momentos después (poco menos de tres días de vuestro tiempo) un Guía de los Graduados recién nacido aparece <<\textit{espontáneamente}>> en el circuito exterior del universo de Havona. Y el número de Guías de los Graduados, teniendo en cuenta una pequeña diferencia debida sin duda a aquellos que están de transición, es exactamente igual al número de servitales desaparecidos.

\par
%\textsuperscript{(271.6)}
\textsuperscript{24:7.7} Existe una razón adicional para suponer que los Guías de los Graduados son los Servitales de Havona evolucionados, y la razón es la tendencia infalible que tienen estos guías y sus servitales asociados a formar unos vínculos tan extraordinarios. La manera en que estas órdenes de seres supuestamente distintas se entienden y se comprenden es totalmente inexplicable. Es reconfortante e inspirador presenciar su mutua devoción.

\par
%\textsuperscript{(271.7)}
\textsuperscript{24:7.8} Los Siete Espíritus Maestros y los Siete Directores Supremos del Poder asociados son respectivamente los depositarios personales del potencial mental y del potencial de poder del Ser Supremo que éste no emplea, hasta ahora, personalmente. Cuando estos asociados paradisiacos colaboran para crear a los Servitales de Havona, estos últimos se encuentran implicados de manera inherente en ciertas fases de la Supremacía. Los Servitales de Havona son pues, en realidad, un reflejo en el perfecto universo central de ciertas potencialidades evolutivas de los dominios espacio-temporales, todo lo cual se revela cuando un servital sufre su transformación y su nueva creación. Creemos que esta transformación tiene lugar en respuesta a la voluntad del Espíritu Infinito, que actúa indudablemente en nombre del Supremo. Los Guías de los Graduados no son creados por el Ser Supremo, pero todos sospechamos que la Deidad experiencial está implicada de alguna manera en estas operaciones que traen a la existencia a estos seres.

\par
%\textsuperscript{(271.8)}
\textsuperscript{24:7.9} El Havona que atraviesan ahora los mortales ascendentes difiere en muchos aspectos del universo central que existía antes de la época de Grandfanda. La llegada de los ascendentes mortales a los circuitos de Havona ha introducido profundas modificaciones en la organización de la creación central y divina, unas modificaciones iniciadas indudablemente por el Ser Supremo ---el Dios de las criaturas evolutivas--- en respuesta a la llegada de su primer hijo experiencial procedente de los siete superuniversos. La aparición de los Guías de los Graduados, junto con la creación de los supernafines terciarios, es un indicio de estas acciones de Dios Supremo.

\par
%\textsuperscript{(272.1)}
\textsuperscript{24:7.10} [Presentado por un Consejero Divino de Uversa.]


\chapter{Documento 25. Las huestes de mensajeros del espacio}
\par
%\textsuperscript{(273.1)}
\textsuperscript{25:0.1} LAS Huestes de Mensajeros del Espacio se encuentran situadas en un punto intermedio en la familia del Espíritu Infinito. Estos seres polifacéticos actúan como eslabones de conexión entre las personalidades superiores y los espíritus ministrantes. Las huestes de mensajeros incluyen a las órdenes siguientes de seres celestiales:

\par
%\textsuperscript{(273.2)}
\textsuperscript{25:0.2} 1. Los Servitales de Havona.

\par
%\textsuperscript{(273.3)}
\textsuperscript{25:0.3} 2. Los Conciliadores Universales.

\par
%\textsuperscript{(273.4)}
\textsuperscript{25:0.4} 3. Los Asesores Técnicos.

\par
%\textsuperscript{(273.5)}
\textsuperscript{25:0.5} 4. Los Custodios de los Archivos en el Paraíso.

\par
%\textsuperscript{(273.6)}
\textsuperscript{25:0.6} 5. Los Registradores Celestiales.

\par
%\textsuperscript{(273.7)}
\textsuperscript{25:0.7} 6. Los Compañeros Morontiales.

\par
%\textsuperscript{(273.8)}
\textsuperscript{25:0.8} 7. Los Compañeros Paradisiacos.

\par
%\textsuperscript{(273.9)}
\textsuperscript{25:0.9} De los siete grupos enumerados, sólo tres ---los servitales, los conciliadores y los Compañeros Morontiales--- han sido creados como tales; los cuatro restantes representan niveles de consecución de las órdenes angélicas. Las huestes de mensajeros sirven de maneras diversas en el universo de universos de acuerdo con su naturaleza inherente y con el estado que han alcanzado, pero siempre están sometidas a la dirección de aquellos que gobiernan los reinos donde están destinadas.

\section*{1. Los Servitales de Havona}
\par
%\textsuperscript{(273.10)}
\textsuperscript{25:1.1} Aunque se les denomina servitales, estas <<\textit{criaturas intermedias}>> del universo central no son servidores en ningún sentido inferior de la palabra. En el mundo espiritual no existe ningún trabajo de baja categoría; todo servicio es sagrado y estimulante; y las órdenes superiores de seres tampoco miran con menosprecio a las órdenes inferiores de existencia.

\par
%\textsuperscript{(273.11)}
\textsuperscript{25:1.2} Los Servitales de Havona son la obra creativa conjunta de los Siete Espíritus Maestros y de sus asociados, los Siete Directores Supremos del Poder. Esta colaboración creativa es la que más se parece a un modelo para la larga lista de reproducciones de tipo doble que se efectúan en los universos evolutivos, y que se extienden desde la creación de una Radiante Estrella Matutina mediante la unión de un Hijo Creador y de un Espíritu Creativo, hasta la procreación sexuada en los mundos como Urantia.

\par
%\textsuperscript{(273.12)}
\textsuperscript{25:1.3} El número de servitales es enorme, y continuamente se están creando más. Aparecen en grupos de mil en el tercer momento que sigue a la reunión de los Espíritus Maestros y de los Directores Supremos del Poder en su zona conjunta situada en el sector más septentrional del Paraíso. Cada cuarto servital es de un tipo más físico que los demás; es decir, que de cada mil, setecientos cincuenta son aparentemente conformes al tipo espiritual, pero doscientos cincuenta son de naturaleza semifísica. Estas \textit{cuartas criaturas} pertenecen en cierto modo a la orden de los seres materiales (materiales en el sentido havoniano), pareciéndose más a los directores del poder físico que a los Espíritus Maestros.

\par
%\textsuperscript{(274.1)}
\textsuperscript{25:1.4} En las relaciones entre personalidades, lo espiritual domina a lo material, aunque esto no parezca así actualmente en Urantia; y en la creación de los Servitales de Havona, la ley que prevalece es la del predominio del espíritu; la proporción establecida produce tres seres espirituales por uno semifísico.

\par
%\textsuperscript{(274.2)}
\textsuperscript{25:1.5} Todos los Servitales recién creados, junto con los nuevos Guías de los Graduados que van apareciendo, pasan por los cursos de formación que los guías más antiguos dirigen continuamente en cada uno de los siete circuitos de Havona. A los servitales se les destina después a las actividades para las que están mejor adaptados, y puesto que son de dos tipos ---espirituales y semifísicos--- la variedad de tareas que estos seres polifacéticos pueden realizar tiene pocos límites. Los grupos superiores o espirituales son destinados selectivamente al servicio del Padre, del Hijo y del Espíritu y al trabajo de los Siete Espíritus Maestros. De vez en cuando son enviados en grandes cantidades a servir en los mundos de estudio que rodean a las esferas sede de los siete superuniversos, los mundos dedicados a la formación final y a la cultura espiritual de las almas ascendentes del tiempo que se están preparando para avanzar hacia los circuitos de Havona. Tanto los servitales espirituales como sus compañeros más físicos son nombrados también como asistentes y asociados de los Guías de los Graduados para ayudar y enseñar a las diversas órdenes de criaturas ascendentes que han alcanzado Havona y que tratan de llegar al Paraíso.

\par
%\textsuperscript{(274.3)}
\textsuperscript{25:1.6} Los Servitales de Havona y los Guías de los Graduados manifiestan una devoción trascendente por su trabajo y un afecto conmovedor los unos por los otros, un afecto que, aunque es espiritual, sólo podríais comprenderlo comparándolo con el fenómeno del amor humano. Cuando los servitales son enviados a sus misiones más allá de los límites del universo central, como sucede tan a menudo, su separación de los guías presenta un patetismo divino; pero parten con alegría y no con tristeza. En los seres espirituales, la alegría satisfactoria de cumplir con un deber elevado es la emoción que eclipsa a todas las demás. La tristeza no puede existir en presencia de la conciencia de un deber divino fielmente ejecutado. Cuando el alma ascendente del hombre se encuentra ante el Juez Supremo, la decisión de importancia eterna no está determinada por los éxitos materiales ni por los logros cuantitativos; el veredicto que resuena en todas las cortes supremas proclama: <<\textit{Bien hecho, buen y \textit{fiel} servidor; has sido fiel en algunas cosas esenciales; serás establecido como gobernante de las realidades universales}>>\footnote{\textit{Bien hecho, buen y fiel servidor}: Mt 25:21,23; Lc 19:17.}.

\par
%\textsuperscript{(274.4)}
\textsuperscript{25:1.7} En el servicio superuniversal, los Servitales de Havona siempre son destinados al dominio que preside el Espíritu Maestro a quien más se parecen por sus prerrogativas espirituales generales y especiales. Sólo sirven en los mundos educativos que rodean a las capitales de los siete superuniversos, y el último informe de Uversa indica que cerca de 138 mil millones de servitales ejercían su ministerio en sus 490 satélites. Se dedican a una variedad sin fin de actividades relacionadas con el trabajo de estos mundos educativos que componen las superuniversidades del superuniverso de Orvonton. Aquí son vuestros compañeros; han descendido desde el nivel de vuestra próxima carrera para estudiaros y para inspiraros la realidad y la certidumbre de que os graduaréis finalmente en los universos del tiempo para pasar a los reinos de la eternidad. Por medio de estos contactos, los servitales adquieren esa experiencia preliminar de aportar su ministerio a las criaturas ascendentes del tiempo que es tan útil en su trabajo posterior en los circuitos de Havona como asociados de los Guías de los Graduados o ---como servitales trasladados--- ejerciendo como Guías de los Graduados ellos mismos.

\section*{2. Los Conciliadores Universales}
\par
%\textsuperscript{(275.1)}
\textsuperscript{25:2.1} Por cada Servital de Havona que se crea, se engendran siete Conciliadores Universales, uno en cada superuniverso. Esta acción creativa requiere una técnica superuniversal precisa de reacción reflectante a unas operaciones que tienen lugar en el Paraíso.

\par
%\textsuperscript{(275.2)}
\textsuperscript{25:2.2} Los siete reflejos de los Siete Espíritus Maestros desempeñan su actividad en los mundos sede de los siete superuniversos. Es difícil intentar describir a la mente material la naturaleza de estos Espíritus Reflectantes. Son auténticas personalidades; sin embargo, cada miembro de un grupo superuniversal sólo refleja perfectamente a uno de los Siete Espíritus Maestros. Y cada vez que los Espíritus Maestros se asocian con los directores del poder con el objeto de crear un grupo de Servitales de Havona, se produce una focalización simultánea en uno de los Espíritus Reflectantes en cada uno de los grupos superuniversales, y un número igual de Conciliadores Universales plenamente desarrollados aparece de inmediato en los mundos sede de las supercreaciones. Si el Espíritu Maestro Número Siete tomara la iniciativa de crear a los servitales, nadie salvo los Espíritus Reflectantes de la séptima orden quedarían fecundados de conciliadores; y mil conciliadores de la séptima orden aparecerían en cada capital superuniversal coincidiendo con la creación de los mil servitales del tipo de Orvonton. Las siete órdenes creadas de conciliadores que sirven en cada superuniverso surgen de estos episodios que reflejan la naturaleza séptuple de los Espíritus Maestros.

\par
%\textsuperscript{(275.3)}
\textsuperscript{25:2.3} Los conciliadores que poseen un estado preparadisiaco no sirven alternativamente entre los superuniversos, estando limitados a los segmentos nativos donde han sido creados. Cada cuerpo superuniversal, que abarca una séptima parte de cada orden creada, pasa pues un tiempo muy largo bajo la influencia de uno de los Espíritus Maestros, con exclusión de los otros, porque aunque los siete están \textit{reflejados} en las capitales superuniversales, sólo uno \textit{domina} en cada supercreación.

\par
%\textsuperscript{(275.4)}
\textsuperscript{25:2.4} Cada una de las siete supercreaciones está impregnada efectivamente de aquel Espíritu Maestro que preside sus destinos. Cada superuniverso se vuelve así como un espejo gigantesco que refleja la naturaleza y el carácter del Espíritu Maestro supervisor, y todo esto se prolonga además en cada universo local subsidiario mediante la presencia y la actividad de los Espíritus Madres Creativos. El efecto de un entorno así sobre el crecimiento evolutivo es tan profundo que, en sus carreras post-superuniversales, los conciliadores manifiestan colectivamente cuarenta y nueve puntos de vista o percepciones experienciales, cada uno de ellos angular ---por lo tanto incompleto--- pero todos se compensan mutuamente y juntos tienden a abarcar el círculo de la Supremacía.

\par
%\textsuperscript{(275.5)}
\textsuperscript{25:2.5} En cada superuniverso, los Conciliadores Universales se encuentran separados de forma extraña e innata en grupos de cuatro, asociaciones en las cuales continúan sirviendo. En cada grupo, tres de ellos son personalidades espirituales, y uno, al igual que las cuartas criaturas de los servitales, es un ser semimaterial. Este cuarteto forma una comisión conciliadora y está compuesto como sigue:

\par
%\textsuperscript{(275.6)}
\textsuperscript{25:2.6} 1. \textit{El Juez-Árbitro.} Aquel designado por unanimidad por los otros tres como el más competente y el mejor cualificado para actuar como jefe judicial del grupo.

\par
%\textsuperscript{(275.7)}
\textsuperscript{25:2.7} 2. \textit{El Abogado Espiritual.} Aquel que es nombrado por el juez-árbitro para presentar las pruebas y salvaguardar los derechos de todas las personalidades implicadas en cualquier asunto destinado a ser juzgado por la comisión conciliadora.

\par
%\textsuperscript{(276.1)}
\textsuperscript{25:2.8} 3. \textit{El Ejecutor Divino.} El conciliador cualificado por su naturaleza inherente para ponerse en contacto con los seres materiales de los reinos y ejecutar las decisiones de la comisión. Como los ejecutores divinos son cuartas criaturas ---unos seres casi materiales--- son casi visibles, pero no del todo, para la visión limitada de las razas mortales.

\par
%\textsuperscript{(276.2)}
\textsuperscript{25:2.9} 4. \textit{El Registrador.} El miembro restante de la comisión se convierte automáticamente en el registrador, en el secretario del tribunal. Él asegura que todos los registros estén preparados adecuadamente para los archivos del superuniverso y para los anales del universo local. Si la comisión está de servicio en un mundo evolutivo, se prepara un tercer informe con la ayuda del ejecutor para los archivos físicos del gobierno sistémico a cuya jurisdicción pertenecen.

\par
%\textsuperscript{(276.3)}
\textsuperscript{25:2.10} Cuando una comisión está reunida funciona como un grupo de tres, puesto que el abogado se encuentra aparte durante el juicio y sólo participa en la expresión del veredicto al final de la audiencia. Por eso a estas comisiones se les llama a veces tríos arbitrales.

\par
%\textsuperscript{(276.4)}
\textsuperscript{25:2.11} Los conciliadores son de un gran valor para hacer que el universo de universos funcione sin problemas. Atraviesan el espacio a la rapidez seráfica de la velocidad triple, y sirven como tribunales ambulantes de los mundos, como comisiones dedicadas a juzgar con rapidez las dificultades menores. Si no fuera por estas comisiones móviles y sumamente equitativas, los tribunales de las esferas estarían desesperadamente abrumados con los malentendidos menores de los reinos.

\par
%\textsuperscript{(276.5)}
\textsuperscript{25:2.12} Estos tríos arbitrales no juzgan los asuntos de importancia eterna; el alma, las perspectivas eternas de una criatura del tiempo, nunca es puesta en peligro a causa de sus actos. Los Conciliadores no se ocupan de las cuestiones que se extienden más allá de la existencia temporal y del bienestar cósmico de las criaturas del tiempo. Pero una vez que una comisión ha aceptado la jurisdicción sobre un problema, sus decisiones son finales y siempre son unánimes; la decisión del juez árbitro es inapelable.

\section*{3. El amplio servicio de los conciliadores}
\par
%\textsuperscript{(276.6)}
\textsuperscript{25:3.1} Los conciliadores mantienen una sede colectiva en la capital de su superuniverso, donde tienen su cuerpo de reserva primario. Sus reservas secundarias están estacionadas en las capitales de los universos locales. Los comisionados más jóvenes y menos experimentados empiezan su servicio en los mundos inferiores, en los mundos como Urantia, y se les promueve para que juzguen problemas más importantes después de haber adquirido una experiencia más madura.

\par
%\textsuperscript{(276.7)}
\textsuperscript{25:3.2} La orden de los conciliadores es totalmente digna de confianza; ninguno de ellos se ha descarriado nunca. Aunque su juicio y su sabiduría no sean infalibles, su fiabilidad es indiscutible y su fidelidad indefectible. Tienen su origen en la sede de un superuniverso y con el tiempo regresan allí, ascendiendo a través de los siguientes niveles de servicio universal:

\par
%\textsuperscript{(276.8)}
\textsuperscript{25:3.3} 1. \textit{Los Conciliadores en los mundos.} Cada vez que las personalidades supervisoras de los mundos individuales se sienten extremadamente confusas o han llegado realmente a un punto muerto en lo que se refiere al procedimiento adecuado a seguir según las circunstancias existentes, y si el asunto no tiene la importancia suficiente como para ser presentado ante los tribunales regularmente constituidos del reino, entonces, después de recibirse la petición de dos personalidades, una por cada parte en litigio, una comisión conciliadora empezará a funcionar enseguida.

\par
%\textsuperscript{(277.1)}
\textsuperscript{25:3.4} Cuando estas dificultades administrativas y jurisdiccionales han sido puestas en manos de los conciliadores para ser estudiadas y juzgadas, la autoridad que éstos poseen es suprema. Pero no pronunciarán ninguna decisión hasta que no se hayan escuchado todos los testimonios, y su autoridad no tiene ningún límite en absoluto para citar a los testigos de cualquier lugar que procedan. Aunque sus decisiones sean inapelables, a veces los asuntos se desarrollan de tal manera que la comisión cierra sus actas en un punto dado, concluye sus opiniones, y transfiere toda la cuestión a los tribunales superiores del reino.

\par
%\textsuperscript{(277.2)}
\textsuperscript{25:3.5} Las decisiones de los comisionados son colocadas en los archivos planetarios y, si es necesario, el ejecutor divino las pone en práctica. Su poder es muy grande y su campo de actividad en un mundo habitado es muy amplio. Los ejecutores divinos manipulan de manera magistral aquello que es en interés de aquello que debería ser. A veces realizan su tarea por el bienestar aparente del reino, y sus actos en los mundos del tiempo y del espacio a veces son difíciles de explicar. Aunque ejecutan sus decretos sin despreciar las leyes naturales ni las costumbres ordenadas del planeta, a menudo llevan a cabo sus extrañas actividades e imponen los mandatos de los conciliadores de acuerdo con las leyes superiores de la administración del sistema.

\par
%\textsuperscript{(277.3)}
\textsuperscript{25:3.6} 2. \textit{Los Conciliadores en las sedes de los sistemas.} Después de servir en los mundos evolutivos, estas comisiones de cuatro miembros ascienden para desempeñar sus funciones en la sede de un sistema. Aquí tienen mucho trabajo que hacer, y demuestran ser los amigos comprensivos de los hombres, de los ángeles y de los otros seres espirituales. Los tríos arbitrales no se interesan tanto por las diferencias personales como por las controversias colectivas y por los malentendidos que surgen entre las diversas órdenes de criaturas; y en la sede de un sistema viven tanto seres espirituales como seres materiales, así como tipos combinados tales como los Hijos Materiales.

\par
%\textsuperscript{(277.4)}
\textsuperscript{25:3.7} En el momento en que los Creadores traen a la existencia a unos individuos evolutivos que tienen el poder de elegir, en ese mismo momento se produce un cambio con respecto al tranquilo funcionamiento de la perfección divina; los malentendidos van a surgir con toda seguridad, y se deben tomar disposiciones para ajustar equitativamente estas honradas diferencias de puntos de vista. Todos deberíamos recordar que los Creadores omnisapientes y todopoderosos podrían haber creado los universos locales tan perfectos como Havona. Ninguna comisión conciliadora necesita ejercer su actividad en el universo central. Pero en toda su sabiduría, los Creadores no eligieron hacer esto. Y aunque han dado nacimiento a unos universos donde abundan las diferencias y pululan las dificultades, también han suministrado los mecanismos y los medios para poner en orden todas estas diferencias y armonizar toda esta confusión aparente.

\par
%\textsuperscript{(277.5)}
\textsuperscript{25:3.8} 3. \textit{Los Conciliadores en las constelaciones.} Después de servir en los sistemas, a los conciliadores los ascienden para que juzguen los problemas de una constelación, dedicándose a las dificultades menores que surgen entre sus cien sistemas de mundos habitados. Muchos problemas que se desarrollan en la sede de una constelación no caen bajo su jurisdicción, pero se mantienen ocupados yendo de sistema en sistema para reunir pruebas y preparar sus declaraciones preliminares. Si la controversia es honrada, si las dificultades proceden de sinceras diferencias de opinión y de una honrada diversidad de puntos de vista, por muy pocas personas que estén implicadas, por muy aparentemente insignificante que sea el malentendido, siempre se puede conseguir que una comisión conciliadora se pronuncie sobre el fondo de la controversia.

\par
%\textsuperscript{(277.6)}
\textsuperscript{25:3.9} 4. \textit{Los Conciliadores en los universos locales.} En este trabajo más amplio de un universo, los comisionados son de una gran ayuda tanto para los Melquisedeks como para los Hijos Magistrales, y para los gobernantes de las constelaciones y la multitud de personalidades que se ocupan de coordinar y administrar las cien constelaciones. Las diferentes órdenes de serafines y otros residentes de las esferas sede de un universo local utilizan también la ayuda y las decisiones de los tríos arbitrales.

\par
%\textsuperscript{(278.1)}
\textsuperscript{25:3.10} Es casi imposible explicar la naturaleza de las diferencias que pueden surgir en los asuntos pormenorizados de un sistema, de una constelación o de un universo. Las dificultades se producen de hecho, pero son muy diferentes a las pruebas y tribulaciones insignificantes de la existencia material tal como ésta se vive en los mundos evolutivos.

\par
%\textsuperscript{(278.2)}
\textsuperscript{25:3.11} 5. \textit{Los Conciliadores en los sectores menores de un superuniverso.} Después de los problemas de los universos locales, los comisionados son ascendidos al estudio de las cuestiones que surgen en los sectores menores de su superuniverso. Cuanto más se elevan hacia el interior desde los planetas individuales, el ejecutor divino tiene menos deberes materiales que hacer; asume gradualmente un nuevo papel de intérprete de la misericordia y de la justicia, y ---como es casi material--- mantiene al mismo tiempo al conjunto de la comisión en contacto comprensivo con los aspectos materiales de sus investigaciones.

\par
%\textsuperscript{(278.3)}
\textsuperscript{25:3.12} 6. \textit{Los Conciliadores en los sectores mayores de un superuniverso.} El carácter del trabajo de los comisionados continúa cambiando a medida que progresan. Cada vez hay menos malentendidos que juzgar y más fenómenos misteriosos que explicar e interpretar. De etapa en etapa van progresando desde árbitros de las diferencias a \textit{explicadores de misterios} ---unos jueces que se transforman en educadores interpretativos. En otro tiempo fueron los árbitros de aquellos que, por ignorancia, dieron lugar a que se originaran dificultades y malentendidos; pero ahora se están convirtiendo en los instructores de aquellos que son lo suficientemente inteligentes y tolerantes como para evitar los conflictos mentales y las guerras de opinión. Cuanto más elevada es la educación de una criatura, más respeto tiene por el conocimiento, la experiencia y las opiniones de los demás.

\par
%\textsuperscript{(278.4)}
\textsuperscript{25:3.13} 7. \textit{Los Conciliadores en el superuniverso.} Aquí los conciliadores consiguen coordinarse ---cuatro árbitros-educadores que se comprenden mutuamente y que ejercen su actividad de manera perfecta. El ejecutor divino es despojado de su poder punitivo y se convierte en la voz física del trío espiritual. Para entonces estos consejeros y educadores se han familiarizado hábilmente con la mayor parte de los problemas y dificultades reales que se encuentran en la dirección de los asuntos del superuniverso. Se convierten así en unos asesores maravillosos y en unos sabios instructores para los peregrinos ascendentes que residen en las esferas educativas que rodean a los mundos sede de los superuniversos.

\par
%\textsuperscript{(278.5)}
\textsuperscript{25:3.14} Todos los conciliadores sirven bajo la supervisión general de los Ancianos de los Días y bajo la dirección directa de los Ayudantes de Imágenes hasta el momento en que son ascendidos a residir en el Paraíso. Durante su estancia en el Paraíso, están bajo las órdenes del Espíritu Maestro que preside el superuniverso de su origen.

\par
%\textsuperscript{(278.6)}
\textsuperscript{25:3.15} Los registros del superuniverso no enumeran a aquellos conciliadores que han pasado más allá de su jurisdicción, y estas comisiones están muy dispersas por todo el gran universo. El último informe de los registros de Uversa indica que el número de comisiones que trabajan en Orvonton se aproxima a los dieciocho billones ---más de setenta billones de individuos. Pero esto sólo representa una fracción muy pequeña de la multitud de conciliadores que han sido creados en Orvonton; su número es de una magnitud mucho más elevada y equivale al número total de Servitales de Havona, teniendo en cuenta las transmutaciones en Guías de los Graduados.

\par
%\textsuperscript{(278.7)}
\textsuperscript{25:3.16} A medida que crece el número de conciliadores superuniversales, son trasladados de vez en cuando al consejo de la perfección del Paraíso, de donde surgen posteriormente como cuerpo coordinador producido por el Espíritu Infinito para el universo de universos, un grupo maravilloso de seres cuyo número y eficacia aumentan constantemente. Han adquirido una comprensión excepcional de la realidad emergente del Ser Supremo a través de su ascensión experiencial y de su entrenamiento en el Paraíso, y surcan el universo de universos en misiones especiales.

\par
%\textsuperscript{(279.1)}
\textsuperscript{25:3.17} Los miembros de una comisión conciliadora no se separan nunca. Los cuatro miembros de un grupo sirven eternamente juntos tal como se asociaron desde el principio. Incluso en su servicio glorificado continúan ejerciendo su actividad como cuartetos con una experiencia cósmica acumulada y una sabiduría experiencial perfeccionada. Están eternamente asociados como personificación de la justicia suprema del tiempo y del espacio.

\section*{4. Los Asesores Técnicos}
\par
%\textsuperscript{(279.2)}
\textsuperscript{25:4.1} Estas mentes jurídicas y técnicas del mundo espiritual no fueron creadas como tales. El Espíritu Infinito eligió como núcleo de este grupo inmenso y polifacético a un millón de las mentes más metódicas entre los primeros supernafines y omniafines. Y desde aquella época tan lejana, a todos los que aspiran a convertirse en Asesores Técnicos siempre se les ha exigido una experiencia efectiva en la aplicación de las leyes de la perfección a los planes de la creación evolutiva.

\par
%\textsuperscript{(279.3)}
\textsuperscript{25:4.2} Los Asesores Técnicos son reclutados en las filas de las siguientes órdenes de personalidades:

\par
%\textsuperscript{(279.4)}
\textsuperscript{25:4.3} 1. Los supernafines.

\par
%\textsuperscript{(279.5)}
\textsuperscript{25:4.4} 2. Los seconafines.

\par
%\textsuperscript{(279.6)}
\textsuperscript{25:4.5} 3. Los terciafines.

\par
%\textsuperscript{(279.7)}
\textsuperscript{25:4.6} 4. Los omniafines.

\par
%\textsuperscript{(279.8)}
\textsuperscript{25:4.7} 5. Los serafines.

\par
%\textsuperscript{(279.9)}
\textsuperscript{25:4.8} 6. Ciertos tipos de mortales ascendentes.

\par
%\textsuperscript{(279.10)}
\textsuperscript{25:4.9} 7. Ciertos tipos de intermedios ascendentes.

\par
%\textsuperscript{(279.11)}
\textsuperscript{25:4.10} En el momento actual, sin contar a los mortales y a los intermedios cuyas asignaciones son todas transitorias, el número de Asesores Técnicos que están registrados en Uversa y trabajan en Orvonton es ligeramente superior a los sesenta y un billones.

\par
%\textsuperscript{(279.12)}
\textsuperscript{25:4.11} Los Asesores Técnicos desempeñan frecuentemente su actividad de manera individual, pero están organizados para el servicio y mantienen unas sedes comunes en grupos de siete en las esferas donde están destinados. En cada grupo, al menos cinco miembros deben tener un estado permanente, mientras que dos pueden estar asociados temporalmente. Los mortales ascendentes y las criaturas intermedias ascendentes sirven en estas comisiones consultivas mientras continúan su ascensión hacia el Paraíso, pero no participan en los programas regulares de formación para los Asesores Técnicos, ni tampoco se convierten nunca en miembros permanentes de la orden.

\par
%\textsuperscript{(279.13)}
\textsuperscript{25:4.12} Los mortales y los intermedios que sirven de manera transitoria con los asesores son elegidos para este trabajo porque son expertos en el concepto de la ley universal y de la justicia suprema. A medida que viajáis hacia vuestra meta en el Paraíso, adquiriendo constantemente conocimientos adicionales y una habilidad creciente, se os concede continuamente la oportunidad de transmitir a otros seres la sabiduría y la experiencia que ya habéis acumulado; durante todo vuestro trayecto hacia Havona representáis el papel de un alumno-maestro. Os abriréis paso a través de los niveles ascendentes de esta inmensa universidad experiencial transmitiendo a aquellos que están justo por debajo de vosotros el conocimiento recién descubierto en vuestra carrera progresiva. En el régimen universal no se considera que habéis adquirido un conocimiento y una verdad hasta que no habéis demostrado vuestra capacidad y vuestra buena voluntad para transmitir a otras personas ese conocimiento y esa verdad.

\par
%\textsuperscript{(280.1)}
\textsuperscript{25:4.13} Después de un largo entrenamiento y de una experiencia efectiva, cualquier espíritu ministrante que se encuentre por encima del estado de los querubines puede recibir un puesto permanente como Asesor Técnico. Todos los candidatos ingresan voluntariamente en esta orden de servicio; pero una vez que han asumido estas responsabilidades no pueden renunciar a ellas. Sólo los Ancianos de los Días pueden trasladar a estos asesores a otras actividades.

\par
%\textsuperscript{(280.2)}
\textsuperscript{25:4.14} La formación de los Asesores Técnicos, que empezó en las universidades Melquisedeks de los universos locales, continúa hasta las cortes de los Ancianos de los Días. Después de esta formación superuniversal siguen adelante hasta las <<\textit{facultades de los siete círculos}>> situadas en los mundos piloto de los circuitos de Havona. Después de los mundos piloto son recibidos en la <<\textit{facultad de la ética de la ley y de la técnica de la Supremacía}>>, la universidad educativa paradisiaca para perfeccionar a los Asesores Técnicos.

\par
%\textsuperscript{(280.3)}
\textsuperscript{25:4.15} Estos asesores son algo más que unos expertos jurídicos; estudian y enseñan la ley \textit{aplicada,} las leyes del universo aplicadas a la vida y al destino de todos los que habitan los inmensos dominios de la extensa creación. A medida que pasa el tiempo se convierten en las bibliotecas jurídicas vivientes del tiempo y del espacio; impiden trastornos sin fin y retrasos innecesarios enseñando a las personalidades del tiempo las formas y los modos de proceder más aceptables para los gobernantes de la eternidad. Son capaces de aconsejar a los trabajadores del espacio de tal manera que les permiten actuar en armonía con las exigencias del Paraíso; son los educadores de todas las criaturas acerca de la técnica de los Creadores.

\par
%\textsuperscript{(280.4)}
\textsuperscript{25:4.16} Esta biblioteca viviente de la ley aplicada no podría ser creada; estos seres deben evolucionar por medio de la experiencia efectiva. Las Deidades infinitas son existenciales, lo cual compensa su falta de experiencia; lo saben todo incluso antes de experimentarlo, pero este conocimiento no experiencial no lo transmiten a sus criaturas subordinadas.

\par
%\textsuperscript{(280.5)}
\textsuperscript{25:4.17} Los Asesores Técnicos se dedican a la tarea de evitar los retrasos, facilitar el progreso y aconsejar cómo alcanzar los objetivos. Siempre hay una manera \textit{mejor} y \textit{más correcta} de hacer las cosas; siempre está la técnica de la perfección, el método divino, y estos asesores saben cómo dirigirnos a todos hacia el descubrimiento de esa manera mejor.

\par
%\textsuperscript{(280.6)}
\textsuperscript{25:4.18} Estos seres extremadamente sabios y prácticos están siempre estrechamente asociados al servicio y al trabajo de los Censores Universales. Los Melquisedeks tienen a su disposición a un cuerpo capacitado. Todos los gobernantes de los sistemas, las constelaciones, los universos y los sectores de los superuniversos están abundantemente provistos de estas mentes técnicas, o de consultas jurídicas, del mundo espiritual. Un grupo especial actúa como consejero jurídico de los Portadores de Vida, asesorando a estos Hijos sobre el grado de desviación que se pueden permitir con respecto al orden establecido para la propagación de la vida, e informándoles además sobre sus prerrogativas y su libertad de acción. Son los asesores de todas las clases de seres en lo que concierne a los usos y las técnicas adecuados en todas las operaciones del mundo espiritual. Pero no se relacionan de forma directa y personal con las criaturas materiales de los reinos.

\par
%\textsuperscript{(280.7)}
\textsuperscript{25:4.19} Además de aconsejar acerca de los usos legales, los Asesores Técnicos se dedican igualmente a la interpretación eficaz de todas las leyes relacionadas con los seres creados ---físicos, mentales y espirituales. Están a la disposición de los Conciliadores Universales y de todos los otros seres que desean saber la verdad de la ley; en otras palabras, saber cómo se puede esperar que reaccione la Supremacía de la Deidad en una situación dada que contenga factores de un orden establecido físico, mental y espiritual. Intentan incluso dilucidar la técnica del Último.

\par
%\textsuperscript{(281.1)}
\textsuperscript{25:4.20} Los Asesores Técnicos son seres escogidos y probados; nunca me he enterado de que uno solo de ellos se haya descarriado. No tenemos ningún dato en Uversa de que hayan sido juzgados alguna vez por desacato a las leyes divinas que ellos interpretan tan eficazmente y exponen de manera tan elocuente. El ámbito de su servicio no tiene ningún límite conocido, y tampoco se le ha impuesto ninguno a su progreso. Continúan como asesores incluso hasta las puertas del Paraíso; todo el universo de la ley y la experiencia está abierto para ellos.

\section*{5. Los Custodios de los Archivos en el Paraíso}
\par
%\textsuperscript{(281.2)}
\textsuperscript{25:5.1} Entre los supernafines terciarios de Havona, algunos de los jefes archivistas más antiguos son elegidos como Custodios de los Archivos, como conservadores de los archivos oficiales de la Isla de Luz, de aquellos archivos que contrastan con los anales vivientes registrados en la mente de los custodios del conocimiento, a veces denominados la <<\textit{biblioteca viviente del Paraíso}>>.

\par
%\textsuperscript{(281.3)}
\textsuperscript{25:5.2} Los ángeles registradores de los planetas habitados son la fuente de todos los expedientes individuales. Otros registradores efectúan sus anotaciones, en todos los universos, tanto en los archivos oficiales como en los archivos vivientes. Desde Urantia hasta el Paraíso se pueden encontrar los dos tipos de archivos: en un universo local hay más archivos escritos y menos vivientes; en el Paraíso hay más vivientes y menos oficiales; en Uversa los dos se encuentran igualmente disponibles.

\par
%\textsuperscript{(281.4)}
\textsuperscript{25:5.3} Todo suceso significativo que se produce en la creación organizada y habitada es un asunto que ha de ser registrado. Aunque los acontecimientos que no tienen más que una importancia local sólo se registran localmente, aquellos que poseen una significación más amplia son tratados en consecuencia. Todo lo que sucede en los planetas, los sistemas y las constelaciones de Nebadon que tenga una importancia universal se registra en Salvington; y estos episodios se transmiten desde estas capitales universales hasta los archivos superiores relacionados con los asuntos de los gobiernos de los sectores y de los superuniversos. El Paraíso posee también un resumen pertinente de los datos de los superuniversos y de Havona; y este relato histórico y acumulativo del universo de universos se encuentra bajo la custodia de estos elevados supernafines terciarios.

\par
%\textsuperscript{(281.5)}
\textsuperscript{25:5.4} Aunque algunos de estos seres han sido enviados a los superuniversos para prestar sus servicios como Jefes de los Archivos y dirigir las actividades de los Registradores Celestiales, ninguno de ellos ha sido trasladado nunca de la lista nominal permanente de su orden.

\section*{6. Los Registradores Celestiales}
\par
%\textsuperscript{(281.6)}
\textsuperscript{25:6.1} Son los registradores que realizan todas las anotaciones por duplicado, efectuando un registro espiritual original y una contrapartida semimaterial ---lo que se podría llamar una copia al carbón. Pueden hacerlo debido a su capacidad particular para manipular simultáneamente tanto la energía espiritual como la material. Los Registradores Celestiales no son creados como tales; son serafines ascendentes de los universos locales. Son recibidos, clasificados y destinados a sus esferas de trabajo por los consejos de los Jefes de los Archivos ubicados en las sedes de los siete superuniversos. Las facultades para formar a los Registradores Celestiales también están situadas allí. Los Perfeccionadores de la Sabiduría y los Consejeros Divinos dirigen la universidad que se encuentra en Uversa.

\par
%\textsuperscript{(281.7)}
\textsuperscript{25:6.2} A medida que los registradores progresan en el servicio universal, continúan llevando a cabo su sistema de registro doble, posibilitando así que sus archivos estén siempre disponibles para todas las clases de seres, desde los de tipo material hasta los elevados espíritus de luz. En vuestra experiencia de transición, a medida que os elevéis desde este mundo material, siempre seréis capaces de consultar los archivos sobre la historia y las tradiciones de la esfera en la que estáis, y familiarizaros por otra parte con ellas.

\par
%\textsuperscript{(282.1)}
\textsuperscript{25:6.3} Los registradores son un cuerpo probado y seguro. Nunca he oído decir que un Registrador Celestial haya desertado, y nunca se ha descubierto una falsificación en sus registros. Están sometidos a una doble inspección; sus registros son examinados a fondo por sus eminentes compañeros de Uversa y por los Mensajeros Poderosos, los cuales certifican la exactitud de las copias casi físicas de los registros espirituales originales.

\par
%\textsuperscript{(282.2)}
\textsuperscript{25:6.4} Los registradores que progresan y que están estacionados en las esferas de registro subordinadas de los universos de Orvonton ascienden a billones y billones, pero el número de aquellos que han alcanzado este estado en Uversa no llega a ocho millones. Estos registradores graduados, o más antiguos, son los custodios y los promotores superuniversales de los archivos garantizados del tiempo y del espacio. Su sede central permanente se encuentra en las moradas circulares que rodean la zona de los archivos en Uversa. Nunca dejan que otros custodien estos archivos; pueden ausentarse a título individual, pero nunca en gran número.

\par
%\textsuperscript{(282.3)}
\textsuperscript{25:6.5} El cuerpo de los Registradores Celestiales es un destino permanente, al igual que el de los supernafines que se han convertido en Custodios de los Archivos. Una vez que los serafines y los supernafines son enrolados en estos servicios, seguirán siendo respectivamente Registradores Celestiales y Custodios de los Archivos hasta el día en que la plena personalización de Dios Supremo dé nacimiento a una administración nueva y modificada.

\par
%\textsuperscript{(282.4)}
\textsuperscript{25:6.6} Estos Registradores Celestiales más antiguos pueden mostrar en Uversa los archivos de todo lo que ha tenido una importancia cósmica en todo Orvonton desde los tiempos muy lejanos de la llegada de los Ancianos de los Días, mientras que los Custodios de los Archivos protegen en la Isla eterna los archivos de este reino que revelan las operaciones paradisiacas que se han producido desde la época de la personificación del Espíritu Infinito.

\section*{7. Los Compañeros Morontiales}
\par
%\textsuperscript{(282.5)}
\textsuperscript{25:7.1} Estos hijos de los Espíritus Madres de los universos locales son los amigos y los asociados de todos los que viven la vida morontial ascendente. No son indispensables para el trabajo real de progresión como criaturas que tienen que hacer los ascendentes, ni tampoco reemplazan en ningún sentido el trabajo de los guardianes seráficos que a menudo acompañan a sus asociados mortales durante su viaje hacia el Paraíso. Los Compañeros Morontiales son simplemente amables anfitriones para aquellos que acaban de empezar la larga ascensión hacia el interior. Son también unos diestros patrocinadores del entretenimiento, y en esta tarea reciben la hábil ayuda de los directores de la reversión.

\par
%\textsuperscript{(282.6)}
\textsuperscript{25:7.2} Aunque tendréis que realizar unas tareas serias y cada vez más difíciles en los mundos educativos morontiales de Nebadon, siempre podréis disponer de temporadas regulares de descanso y de reversión. Durante todo el viaje hacia el Paraíso, siempre habrá tiempo para el descanso y la diversión espiritual; y en la carrera de luz y de vida siempre hay tiempo para la adoración y los nuevos logros.

\par
%\textsuperscript{(282.7)}
\textsuperscript{25:7.3} Estos Compañeros Morontiales son unos asociados tan amistosos que cuando dejéis finalmente la última fase de la experiencia morontial, cuando os preparéis para emprender la aventura espiritual superuniversal, lamentaréis sinceramente que estas criaturas tan sociables no puedan acompañaros, pero prestan sus servicios exclusivamente en los universos locales. En todas las etapas de la carrera ascendente, todas las personalidades contactables serán amistosas y sociables, pero no encontraréis a otro grupo tan dedicado a la amistad y al compañerismo hasta que no conozcáis a los Compañeros Paradisiacos.

\par
%\textsuperscript{(283.1)}
\textsuperscript{25:7.4} El trabajo de los Compañeros Morontiales está descrito de manera más completa en las narraciones que tratan de los asuntos de vuestro universo local.

\section*{8. Los Compañeros Paradisiacos}
\par
%\textsuperscript{(283.2)}
\textsuperscript{25:8.1} Los Compañeros Paradisiacos son un grupo compuesto, o acumulado, que ha sido reclutado en las filas de los serafines, los seconafines, los supernafines y los omniafines. Aunque sirven durante un período de tiempo que consideraríais extraordinariamente largo, no tienen un estado permanente. Cuando este ministerio ha terminado, regresan por regla general (aunque no invariablemente) a aquellas funciones que realizaban cuando fueron llamados para servir en el Paraíso.

\par
%\textsuperscript{(283.3)}
\textsuperscript{25:8.2} Los Espíritus Madres de los universos locales, los Espíritus Reflectantes de los superuniversos y Majeston del Paraíso son los que designan a los miembros de las huestes angélicas para este servicio. Uno de los Siete Espíritus Maestros los convoca en la Isla central y los nombra como Compañeros Paradisiacos. Aparte del estado permanente en el Paraíso, este servicio temporal como compañeros en el Paraíso es el honor más grande que se pueda conferir nunca a los espíritus ministrantes.

\par
%\textsuperscript{(283.4)}
\textsuperscript{25:8.3} Estos ángeles escogidos se dedican a la tarea de servir de acompañantes y son asignados como asociados a todas las clases de seres que puedan estar casualmente solos en el Paraíso, principalmente a los mortales ascendentes, pero también a todos los demás seres que están solos en la Isla central. Los Compañeros Paradisiacos no tienen nada especial que hacer a favor de aquellos con quienes fraternizan; son simplemente compañeros. Casi todos los demás seres que los mortales encontraréis durante vuestra estancia en el Paraíso ---aparte de vuestros camaradas peregrinos--- tendrán algo preciso que hacer con vosotros o por vosotros; pero estos compañeros tienen la única misión de estar con vosotros y de comulgar con vosotros como asociados de vuestra personalidad. Los amables y brillantes Ciudadanos del Paraíso los ayudan a menudo en su ministerio.

\par
%\textsuperscript{(283.5)}
\textsuperscript{25:8.4} Los mortales proceden de unas razas que son muy sociables. Los Creadores saben muy bien que <<\textit{no es bueno que el hombre esté solo}>>\footnote{\textit{No es bueno que el hombre esté solo}: Gn 2:18.} y, en consecuencia, toman sus disposiciones para que esté acompañado, incluso en el Paraíso.

\par
%\textsuperscript{(283.6)}
\textsuperscript{25:8.5} Si vosotros, como mortales ascendentes, llegarais al Paraíso en compañía de la compañera o íntima asociada de vuestra carrera terrestre, o si vuestro guardián seráfico del destino llegara por casualidad con vosotros o bien os estuviera esperando, entonces no se os asignaría ningún compañero permanente. Pero si llegáis solos, un compañero os dará con toda seguridad la bienvenida cuando os despertéis del sueño final del tiempo en la Isla de Luz. Aunque se sepa que llegaréis acompañados de algún asociado ascendente, se designarán a unos compañeros temporales para que os den la bienvenida a las orillas eternas y para acompañaros hasta el lugar preparado para recibiros a vosotros y a vuestros asociados. Podéis estar seguros de que seréis cálidamente recibidos cuando experimentéis la resurrección para la eternidad en las orillas perpetuas del Paraíso.

\par
%\textsuperscript{(283.7)}
\textsuperscript{25:8.6} Durante los días finales de la estancia del ascendente en el último circuito de Havona se designan a los compañeros que lo van a recibir, y éstos examinan cuidadosamente los datos relacionados con su origen mortal y su agitada ascensión a través de los mundos del espacio y de los círculos de Havona. Cuando reciben a los mortales del tiempo, ya están bien versados en las carreras de estos peregrinos que llegan, y demuestran ser enseguida unos compañeros comprensivos y fascinantes.

\par
%\textsuperscript{(283.8)}
\textsuperscript{25:8.7} Durante vuestra estancia prefinalitaria en el Paraíso, si por alguna razón tuvierais que separaros temporalmente del asociado ---mortal o seráfico--- de vuestra carrera ascendente, se os asignaría inmediatamente un Compañero Paradisiaco para aconsejaros y acompañaros. Una vez que ha sido asignado a un mortal ascendente que reside solitariamente en el Paraíso, el compañero permanece con esa persona hasta que ésta se reúne con sus asociados ascendentes o es enrolada debidamente en el Cuerpo de la Finalidad.

\par
%\textsuperscript{(284.1)}
\textsuperscript{25:8.8} Los Compañeros Paradisiacos son asignados según el orden de su lista de espera, salvo que un ascendente nunca es puesto a cargo de un compañero cuya naturaleza difiere de su tipo superuniversal. Si un mortal de Urantia llegara hoy al Paraíso, se le asignaría el primer compañero que está en la lista de espera y que tiene su origen en Orvonton o bien en la naturaleza del Séptimo Espíritu Maestro. Por eso los omniafines no prestan sus servicios a las criaturas ascendentes de los siete superuniversos.

\par
%\textsuperscript{(284.2)}
\textsuperscript{25:8.9} Los Compañeros Paradisiacos realizan muchos servicios adicionales: si un mortal ascendente llegara solo al universo central y fracasara en alguna fase de la aventura de la Deidad mientras atraviesa Havona, sería devuelto en su debido momento a los universos del tiempo, e inmediatamente se realizaría un llamamiento a las reservas de los Compañeros Paradisiacos. Un miembro de esta orden recibiría la misión de seguir al peregrino rechazado, estar con él, confortarlo y alentarlo, y permanecer con él hasta que volviera al universo central para reanudar la ascensión al Paraíso.

\par
%\textsuperscript{(284.3)}
\textsuperscript{25:8.10} Si un peregrino ascendente fuera rechazado en la aventura de la Deidad mientras atraviesa Havona en compañía de un serafín ascendente, el ángel guardián de su carrera como mortal, este ángel escogería acompañar a su asociado mortal. Estos serafines se ofrecen siempre como voluntarios y se les permite acompañar a sus camaradas mortales de tantos años que regresan al servicio del tiempo y del espacio.

\par
%\textsuperscript{(284.4)}
\textsuperscript{25:8.11} Pero no sucede lo mismo con dos ascendentes mortales íntimamente asociados: si uno de ellos alcanza a Dios mientras que el otro fracasa temporalmente, el individuo que ha tenido éxito elige invariablemente regresar con la personalidad decepcionada a las creaciones evolutivas, pero esto no está permitido. En lugar de eso se hace un llamamiento a las reservas de los Compañeros Paradisiacos, y uno de los voluntarios es elegido para que acompañe al peregrino decepcionado. Un Ciudadano voluntario del Paraíso se asocia entonces con el mortal que ha tenido éxito, el cual se queda en la Isla central esperando que regrese a Havona su camarada rechazado, y mientras tanto enseña en ciertas escuelas del Paraíso, exponiendo la intrépida historia de la ascensión evolutiva.

\par
%\textsuperscript{(284.5)}
\textsuperscript{25:8.12} [Patrocinado por un Elevado en Autoridad procedente de Uversa.]


\chapter{Documento 26. Los espíritus ministrantes del universo central}
\par
%\textsuperscript{(285.1)}
\textsuperscript{26:0.1} LOS supernafines son los espíritus ministrantes\footnote{\textit{Espíritus ministrantes}: Heb 1:14.} del Paraíso y del universo central; son la orden más elevada del grupo más humilde de hijos del Espíritu Infinito ---de las huestes angélicas. Estos espíritus ministrantes se pueden encontrar desde la Isla del Paraíso hasta los mundos del espacio y del tiempo. Ninguna parte importante de la creación organizada y habitada está desprovista de sus servicios.

\section*{1. Los espíritus ministrantes}
\par
%\textsuperscript{(285.2)}
\textsuperscript{26:1.1} Los ángeles son los asociados espirituales ministrantes de las criaturas volitivas evolutivas y ascendentes de todo el espacio; son también los colegas y los asociados de trabajo de las multitudes superiores de personalidades divinas de las esferas. Los ángeles de todas las órdenes tienen personalidades distintas y están sumamente individualizados. Todos tienen una amplia capacidad para apreciar el ministerio de los directores de la reversión. Junto con las Huestes de Mensajeros del Espacio, los espíritus ministrantes disfrutan de períodos de descanso y de cambio; poseen una naturaleza muy sociable y tienen una capacidad para asociarse que trasciende de lejos la de los seres humanos.

\par
%\textsuperscript{(285.3)}
\textsuperscript{26:1.2} Los espíritus ministrantes del gran universo están clasificados como sigue:

\par
%\textsuperscript{(285.4)}
\textsuperscript{26:1.3} 1. Los supernafines.

\par
%\textsuperscript{(285.5)}
\textsuperscript{26:1.4} 2. Los seconafines.

\par
%\textsuperscript{(285.6)}
\textsuperscript{26:1.5} 3. Los terciafines.

\par
%\textsuperscript{(285.7)}
\textsuperscript{26:1.6} 4. Los omniafines.

\par
%\textsuperscript{(285.8)}
\textsuperscript{26:1.7} 5. Los serafines.

\par
%\textsuperscript{(285.9)}
\textsuperscript{26:1.8} 6. Los querubines y los sanobines.

\par
%\textsuperscript{(285.10)}
\textsuperscript{26:1.9} 7. Las criaturas intermedias.

\par
%\textsuperscript{(285.11)}
\textsuperscript{26:1.10} Los miembros individuales de las órdenes angélicas no tienen un estado personal completamente fijo en el universo. Los ángeles de ciertas órdenes pueden convertirse en Compañeros Paradisiacos durante un período de tiempo; algunos se vuelven Registradores Celestiales; otros se elevan hasta las filas de los Asesores Técnicos. Algunos querubines pueden aspirar al estado y al destino seráficos, mientras que los serafines evolutivos pueden alcanzar los niveles espirituales de los Hijos ascendentes de Dios.

\par
%\textsuperscript{(285.12)}
\textsuperscript{26:1.11} Las siete órdenes de espíritus ministrantes, tal como os son reveladas, han sido agrupadas para su presentación de acuerdo con las funciones que tienen mayor importancia para las criaturas ascendentes:

\par
%\textsuperscript{(285.13)}
\textsuperscript{26:1.12} 1. \textit{Los espíritus ministrantes del universo central.} Las tres órdenes de \textit{supernafines} sirven en el sistema Paraíso-Havona. Los supernafines primarios o paradisiacos son creados por el Espíritu Infinito. Las órdenes secundaria y terciaria, que prestan sus servicios en Havona, son los descendientes respectivos de los Espíritus Maestros y de los Espíritus de los Circuitos.

\par
%\textsuperscript{(286.1)}
\textsuperscript{26:1.13} 2. \textit{Los espíritus ministrantes de los superuniversos} ---los seconafines, los terciafines y los omniafines. Los \textit{seconafines,} hijos de los Espíritus Reflectantes, prestan sus servicios de manera diversa en los siete superuniversos. Los \textit{terciafines,} que tienen su origen en el Espíritu Infinito, se dedican finalmente al servicio de enlace entre los Hijos Creadores y los Ancianos de los Días. Los \textit{omniafines} son creados de común acuerdo por el Espíritu Infinito y los Siete Ejecutivos Supremos, y son los servidores exclusivos de estos últimos. El análisis de estas tres órdenes constituye el tema de una narración posterior en esta serie.

\par
%\textsuperscript{(286.2)}
\textsuperscript{26:1.14} 3. \textit{Los espíritus ministrantes de los universos locales} incluyen a los \textit{serafines} y a sus ayudantes, los \textit{querubines.} Los ascendentes mortales tienen un contacto inicial con esta progenie de un Espíritu Madre Universal. Las \textit{criaturas intermedias} son nativas de los mundos habitados y no forman parte realmente de las órdenes angélicas propiamente dichas, aunque a menudo son agrupadas funcionalmente con los espíritus ministrantes. Su historia, con un informe sobre los serafines y los querubines, será presentada en los documentos que tratan de los asuntos de vuestro universo local.

\par
%\textsuperscript{(286.3)}
\textsuperscript{26:1.15} Todas las órdenes de las huestes angélicas están dedicadas a los diversos servicios universales, y aportan su ministerio de una manera u otra a las órdenes superiores de seres celestiales; pero son los supernafines, los seconafines y los serafines los que son empleados en gran número para fomentar el programa ascendente de la perfección progresiva para los hijos del tiempo. Ejercen su actividad en el universo central, en los superuniversos y en los universos locales, y forman esa cadena ininterrumpida de ministros espirituales que ha sido proporcionada por el Espíritu Infinito para ayudar y guiar a todos los que tratan de alcanzar al Padre Universal a través del Hijo Eterno.

\par
%\textsuperscript{(286.4)}
\textsuperscript{26:1.16} Los supernafines sólo están limitados en <<\textit{polaridad espiritual}>> respecto a una sola fase de acción, aquella relacionada con el Padre Universal. Pueden trabajar solos, salvo cuando emplean directamente los circuitos exclusivos del Padre. Cuando reciben el poder del ministerio directo del Padre, los supernafines deben asociarse voluntariamente en parejas para poder ejercer su actividad. Los seconafines están limitados del mismo modo, y además deben trabajar en parejas con el objeto de sincronizarse con los circuitos del Hijo Eterno. Los serafines pueden trabajar solos como personalidades distintas y localizadas, pero sólo son capaces de ponerse en circuito cuando están polarizados como parejas de enlace. Cuando estos seres espirituales están asociados en parejas, se dice que uno es complementario del otro. Las relaciones complementarias pueden ser transitorias; no son necesariamente de naturaleza permanente.

\par
%\textsuperscript{(286.5)}
\textsuperscript{26:1.17} Estas brillantes criaturas de luz se sustentan directamente absorbiendo la energía espiritual de los circuitos primarios del universo. Los mortales de Urantia deben obtener la energía de la luz por medio de la encarnación vegetativa, pero las huestes angélicas están metidas en circuitos; tienen <<\textit{un alimento que vosotros no conocéis}>>\footnote{\textit{Alimento que no conocéis}: Jn 4:32.}. También absorben las enseñanzas circulantes de los maravillosos Hijos Instructores Trinitarios; reciben el conocimiento y absorben la sabiduría de una manera que se parece mucho a la técnica que emplean para asimilar las energías vitales.

\section*{2. Los Poderosos Supernafines}
\par
%\textsuperscript{(286.6)}
\textsuperscript{26:2.1} Los supernafines son los ministros cualificados para todos los tipos de seres que residen en el Paraíso y en el universo central. Estos ángeles elevados son creados en tres órdenes principales: primaria, secundaria y terciaria.

\par
%\textsuperscript{(287.1)}
\textsuperscript{26:2.2} \textit{Los supernafines primarios} son la progenitura exclusiva del Creador Conjunto. Dividen su ministerio de una manera casi igual entre ciertos grupos de Ciudadanos del Paraíso y el cuerpo cada vez más numeroso de peregrinos ascendentes. Estos ángeles de la Isla eterna son muy eficaces en la cuestión de fomentar la formación esencial de los dos grupos de habitantes del Paraíso. Aportan una contribución muy útil a la comprensión mutua entre estas dos órdenes únicas de criaturas universales ---pues una es el tipo más elevado de criatura volitiva divina y perfecta, y la otra la evolución perfeccionada del tipo más humilde de criatura volitiva de todo el universo de universos.

\par
%\textsuperscript{(287.2)}
\textsuperscript{26:2.3} El trabajo de los supernafines primarios es tan excepcional y característico que será estudiado por separado en la próxima narración.

\par
%\textsuperscript{(287.3)}
\textsuperscript{26:2.4} \textit{Los supernafines secundarios} dirigen los asuntos de los seres ascendentes en los siete circuitos de Havona. Se interesan igualmente por ayudar a la preparación educativa de numerosas órdenes de Ciudadanos del Paraíso que residen durante largos períodos en los circuitos de mundos de la creación central, pero no podemos examinar esta fase de su servicio.

\par
%\textsuperscript{(287.4)}
\textsuperscript{26:2.5} Estos ángeles elevados son de siete tipos; cada uno de ellos tiene su origen en uno de los Siete Espíritus Maestros y su naturaleza sigue en consecuencia ese modelo. Los Siete Espíritus Maestros crean colectivamente muchos grupos diferentes de seres y de entidades únicos, y la naturaleza de los miembros individuales de cada orden es relativamente uniforme. Pero cuando estos mismos Siete Espíritus crean individualmente, la naturaleza de las órdenes resultantes es siempre séptuple; los hijos de cada Espíritu Maestro comparten la naturaleza de su creador y son por consiguiente distintos a los demás. Éste es el origen de los supernafines secundarios, y los ángeles de los siete tipos creados desempeñan sus funciones en todos los campos de actividad abiertos a la totalidad de su orden, principalmente en los siete circuitos del universo central y divino.

\par
%\textsuperscript{(287.5)}
\textsuperscript{26:2.6} Cada uno de los siete circuitos planetarios de Havona está bajo la supervisión directa de uno de los Siete Espíritus de los Circuitos, y éstos mismos son la creación colectiva ---y por lo tanto uniforme--- de los Siete Espíritus Maestros. Aunque comparten la naturaleza de la Fuente-Centro Tercera, estos siete Espíritus secundarios de Havona no formaban parte del universo arquetípico original. Empezaron a ejercer su actividad después de la creación original (eterna) pero mucho antes de los tiempos de Grandfanda. Aparecieron indudablemente como una reacción creativa de los Espíritus Maestros al propósito emergente del Ser Supremo, y se descubrió que estaban desempeñando sus funciones en el momento de organizarse el gran universo. El Espíritu Infinito y todos sus asociados creativos, como coordinadores universales, parecen estar abundantemente dotados de la capacidad de proporcionar respuestas creativas adecuadas a los desarrollos simultáneos que se producen en las Deidades experienciales y en los universos en evolución.

\par
%\textsuperscript{(287.6)}
\textsuperscript{26:2.7} \textit{Los supernafines terciarios} tienen su origen en estos Siete Espíritus de los Circuitos. El Espíritu Infinito ha facultado a cada uno de ellos para crear en los distintos círculos de Havona un número suficiente de elevados ministros superáficos de la orden terciaria a fin de satisfacer las necesidades del universo central. Aunque los Espíritus de los Circuitos engendraron un número relativamente pequeño de estos ministros angélicos antes de la llegada de los peregrinos del tiempo a Havona, los Siete Espíritus Maestros ni siquiera empezaron a crear a los supernafines secundarios hasta el aterrizaje de Grandfanda. Como los supernafines terciarios son los más antiguos de las dos órdenes, los examinaremos por tanto en primer lugar.

\section*{3. Los Supernafines Terciarios}
\par
%\textsuperscript{(288.1)}
\textsuperscript{26:3.1} Estos servidores de los Siete Espíritus Maestros son los especialistas angélicos de los diversos circuitos de Havona, y su ministerio se extiende tanto a los peregrinos ascendentes del tiempo como a los peregrinos descendentes de la eternidad. Vuestros asociados superáficos de todas las órdenes serán plenamente visibles para vosotros en los mil millones de mundos de estudio de la perfecta creación central. Allí todos seréis, en el sentido más elevado, seres fraternales y comprensivos con un contacto y una simpatía mutuos. También reconoceréis plenamente y fraternizaréis de manera exquisita con los peregrinos descendentes, los Ciudadanos del Paraíso, que atraviesan estos circuitos desde el interior hacia el exterior, entrando en Havona por el mundo piloto del primer circuito y dirigiéndose hacia el exterior hasta el séptimo.

\par
%\textsuperscript{(288.2)}
\textsuperscript{26:3.2} Los peregrinos ascendentes de los siete superuniversos atraviesan Havona en dirección contraria, entrando por el mundo piloto del séptimo circuito y dirigiéndose hacia el interior. No existe ningún límite de tiempo establecido para que las criaturas ascendentes puedan progresar de mundo en mundo y de circuito en circuito, así como tampoco existe ningún período fijo de tiempo señalado arbitrariamente para residir en los mundos morontiales. Pero, mientras que los individuos adecuadamente desarrollados pueden estar exentos de residir en uno o en más mundos educativos del universo local, ningún peregrino puede evitar pasar por los siete circuitos de espiritualización progresiva de Havona.

\par
%\textsuperscript{(288.3)}
\textsuperscript{26:3.3} Este cuerpo de supernafines terciarios, destinado principalmente al servicio de los peregrinos del tiempo, está clasificado como sigue:

\par
%\textsuperscript{(288.4)}
\textsuperscript{26:3.4} 1. \textit{Los Supervisores de la Armonía.} Debe ser evidente que se necesita algún tipo de influencia coordinadora, incluso en el perfecto Havona, para mantener el sistema y asegurar la armonía en todo el trabajo de preparar a los peregrinos del tiempo para sus consecuciones posteriores en el Paraíso. Ésta es la verdadera misión de los supervisores de la armonía ---cuidar de que todo funcione de manera tranquila y expeditiva. Tienen su origen en el primer circuito y sirven en todo Havona, y su presencia en los circuitos significa que nada puede salir mal de ninguna manera. Estos supernafines tienen una gran capacidad para coordinar una diversidad de actividades que afectan a personalidades de diferentes órdenes ---e incluso de múltiples niveles---, lo que les permite ofrecer su ayuda en cualquier momento y lugar en que sea necesaria. Contribuyen enormemente a que los peregrinos del tiempo y los peregrinos de la eternidad se comprendan mutuamente.

\par
%\textsuperscript{(288.5)}
\textsuperscript{26:3.5} 2. \textit{Los Jefes Registradores.} Estos ángeles son creados en el segundo circuito, pero trabajan en todas las partes del universo central. Efectúan sus registros por triplicado, realizando sus anotaciones para los archivos tangibles de Havona, para los archivos espirituales de su orden y para los archivos oficiales del Paraíso. Además, transmiten automáticamente los informes sobre los acontecimientos de importancia para el conocimiento verdadero a las bibliotecas vivientes del Paraíso, a los custodios del conocimiento de la orden primaria de supernafines.

\par
%\textsuperscript{(288.6)}
\textsuperscript{26:3.6} 3. \textit{Los Transmisores.} Los hijos del tercer Espíritu de los Circuitos ejercen su actividad en todo Havona, aunque su estación oficial está situada en el planeta número setenta del círculo más exterior. Estos técnicos maestros reciben y envían las transmisiones de la creación central y son los directores de los informes que se transmiten al espacio sobre todos los fenómenos relacionados con la Deidad que se producen en el Paraíso. Pueden trabajar con todos los circuitos fundamentales del espacio.

\par
%\textsuperscript{(288.7)}
\textsuperscript{26:3.7} 4. \textit{Los Mensajeros} tienen su origen en el circuito número cuatro. Recorren el sistema Paraíso-Havona como portadores de todos los mensajes que necesitan una transmisión personal. Sirven a sus compañeros, a las personalidades celestiales, a los peregrinos del Paraíso e incluso a las almas ascendentes del tiempo.

\par
%\textsuperscript{(289.1)}
\textsuperscript{26:3.8} 5. \textit{Los Coordinadores de la Información.} Estos supernafines terciarios, hijos del quinto Espíritu de los Circuitos, siempre son los promotores sabios y comprensivos de la asociación fraternal entre los peregrinos ascendentes y descendentes. Aportan su ministerio a todos los habitantes de Havona y especialmente a los ascendentes, manteniéndolos informados y al día sobre los asuntos del universo de universos. Gracias a sus contactos personales con los transmisores y los reflectores, estos <<\textit{periódicos vivientes}>> de Havona conocen instantáneamente toda la información que pasa por los inmensos circuitos de noticias del universo central. Consiguen la información mediante el método gráfico de Havona, el cual les permite asimilar automáticamente en una hora del tiempo de Urantia tanta información como vuestra técnica telegráfica más rápida sería capaz de registrar en mil años.

\par
%\textsuperscript{(289.2)}
\textsuperscript{26:3.9} 6. \textit{Las Personalidades de Transporte.} Estos seres, que tienen su origen en el circuito número seis, trabajan generalmente a partir del planeta número cuarenta situado en el circuito más exterior. Son ellos los que se llevan a los candidatos decepcionados que fracasan de manera transitoria en la aventura de la Deidad. Permanecen preparados para servir a todos los seres que deben ir y venir para el servicio de Havona y que no pueden atravesar el espacio por sí solos.

\par
%\textsuperscript{(289.3)}
\textsuperscript{26:3.10} 7. \textit{El Cuerpo de Reserva.} Las fluctuaciones del trabajo con los seres ascendentes, los peregrinos del Paraíso y otras órdenes de seres que residen en Havona hacen necesario mantener estas reservas de supernafines en el mundo piloto del séptimo círculo, en el cual tienen su origen. Son creados sin un propósito especial y están capacitados para encargarse de servir en las fases menos exigentes de cualquiera de las obligaciones de sus asociados superáficos de la orden terciaria.

\section*{4. Los Supernafines Secundarios}
\par
%\textsuperscript{(289.4)}
\textsuperscript{26:4.1} Los supernafines secundarios ejercen su ministerio en los siete circuitos planetarios del universo central. Una parte de ellos está dedicada al servicio de los peregrinos del tiempo, y la mitad de toda la orden tiene la tarea de formar a los peregrinos paradisiacos de la eternidad. Los voluntarios del Cuerpo de la Finalidad de los Mortales también acompañan a estos Ciudadanos del Paraíso en su peregrinación por los circuitos de Havona, un acuerdo que ha prevalecido desde que se completó el primer grupo de finalitarios.

\par
%\textsuperscript{(289.5)}
\textsuperscript{26:4.2} Según su asignación periódica al ministerio de los peregrinos ascendentes, los supernafines secundarios trabajan en los siete grupos siguientes:

\par
%\textsuperscript{(289.6)}
\textsuperscript{26:4.3} 1. Los Ayudantes de los Peregrinos.

\par
%\textsuperscript{(289.7)}
\textsuperscript{26:4.4} 2. Los Guías de la Supremacía.

\par
%\textsuperscript{(289.8)}
\textsuperscript{26:4.5} 3. Los Guías de la Trinidad.

\par
%\textsuperscript{(289.9)}
\textsuperscript{26:4.6} 4. Los Descubridores del Hijo.

\par
%\textsuperscript{(289.10)}
\textsuperscript{26:4.7} 5. Los Guías del Padre.

\par
%\textsuperscript{(289.11)}
\textsuperscript{26:4.8} 6. Los Consejeros y los Asesores.

\par
%\textsuperscript{(289.12)}
\textsuperscript{26:4.9} 7. Los Complementos del Descanso.

\par
%\textsuperscript{(289.13)}
\textsuperscript{26:4.10} Cada uno de estos grupos de trabajo contiene ángeles de los siete tipos creados, y un peregrino del espacio siempre recibe la enseñanza de los supernafines secundarios que tienen su origen en el Espíritu Maestro que preside el superuniverso donde nació ese peregrino. Cuando vosotros, los mortales de Urantia, lleguéis a Havona, seréis guiados sin duda por los supernafines cuya naturaleza creada ---al igual que vuestra propia naturaleza evolutiva--- procede del Espíritu Maestro de Orvonton. Puesto que vuestros tutores descienden del Espíritu Maestro de vuestro propio superuniverso, están especialmente cualificados para comprenderos, confortaros y ayudaros en todos vuestros esfuerzos por alcanzar la perfección paradisiaca.

\par
%\textsuperscript{(290.1)}
\textsuperscript{26:4.11} Los peregrinos del tiempo son transportados más allá de los cuerpos gravitatorios oscuros hasta el circuito planetario exterior de Havona por las personalidades transportadoras de la orden primaria de seconafines que operan desde las sedes de los siete superuniversos. La mayoría de los serafines, pero no todos, que sirven en los planetas y en los universos locales y que han sido acreditados para ascender hacia el Paraíso, se separarán de sus asociados mortales antes del largo vuelo hacia Havona y empezarán de inmediato una larga e intensa formación para ser asignados a una tarea excelsa, esperando conseguir como serafines la perfección de existencia y la supremacía del servicio. Y esto lo hacen, con la esperanza de reunirse con los peregrinos del tiempo, para ser contados entre aquellos que siguen para siempre el camino de esos mortales que han alcanzado al Padre Universal y han recibido una tarea en el servicio no revelado del Cuerpo de la Finalidad.

\par
%\textsuperscript{(290.2)}
\textsuperscript{26:4.12} El peregrino aterriza en el planeta receptor de Havona, en el mundo piloto del séptimo circuito, con una sola dotación de perfección, la perfección de propósito. El Padre Universal ha decretado: <<\textit{Sed perfectos como yo soy perfecto}>>\footnote{\textit{Sed perfectos}: Gn 17:1; 1 Re 8:61; Lv 19:2; Dt 18:13; Mt 5:48; 2 Co 13:11; Stg 1:4; 1 P 1:16.}. Ésta es la asombrosa orden-invitación transmitida a los hijos finitos de los mundos del espacio. La promulgación de este mandato ha puesto en movimiento a toda la creación en un esfuerzo cooperativo de los seres celestiales por ayudar a llevar a cabo el cumplimiento y la realización de este mandato extraordinario de la Gran Fuente-Centro Primera.

\par
%\textsuperscript{(290.3)}
\textsuperscript{26:4.13} Cuando sois finalmente depositados en el mundo receptor de Havona gracias al ministerio de todas las huestes de ayudantes relacionadas con el plan universal de supervivencia, llegáis con un solo tipo de perfección ---la perfección de propósito. Vuestro propósito ha sido completamente demostrado; vuestra fe ha sido probada. Se sabe que estáis a prueba de decepciones. Ni siquiera el fracaso en discernir al Padre Universal puede hacer vacilar la fe ni perturbar seriamente la confianza de un mortal ascendente que ha pasado por la experiencia que todos deben atravesar para alcanzar las esferas perfectas de Havona. Cuando lleguéis a Havona, vuestra sinceridad se habrá vuelto sublime. La perfección de vuestro propósito y la divinidad de vuestro deseo, junto con la firmeza de vuestra fe, han asegurado vuestra entrada en las moradas permanentes de la eternidad; vuestra liberación de las incertidumbres del tiempo es plena y completa; ahora tenéis que enfrentaros con los problemas de Havona y con las inmensidades del Paraíso, para cuyo encuentro os habéis entrenado durante tanto tiempo en las épocas experienciales del tiempo y en las escuelas de los mundos del espacio.

\par
%\textsuperscript{(290.4)}
\textsuperscript{26:4.14} La fe ha conquistado para el peregrino ascendente una perfección de propósito que deja entrar a los hijos del tiempo por las puertas de la eternidad. Ahora los ayudantes de los peregrinos deben empezar el trabajo de desarrollar esa perfección de entendimiento y esa técnica de comprensión que son tan indispensables para la perfección paradisiaca de la personalidad.

\par
%\textsuperscript{(290.5)}
\textsuperscript{26:4.15} \textit{La capacidad de comprender es el pasaporte de los mortales para el Paraíso.} La buena voluntad para creer es la clave para Havona. La aceptación de la filiación, la cooperación con el Ajustador interior, es el precio de la supervivencia evolutiva.

\section*{5. Los Ayudantes de los Peregrinos}
\par
%\textsuperscript{(291.1)}
\textsuperscript{26:5.1} El primero de los siete grupos de supernafines secundarios que encontraréis es el de los ayudantes de los peregrinos, esos seres que poseen una comprensión rápida y una amplia simpatía, y que dan la bienvenida a los ascendentes del espacio, que tanto han viajado, a los mundos estabilizados y a la economía asentada del universo central. Estos elevados ministros empiezan simultáneamente su trabajo para los peregrinos paradisiacos de la eternidad, el primero de los cuales llegó al mundo piloto del circuito interior de Havona al mismo tiempo que Grandfanda aterrizaba en el mundo piloto del circuito exterior. En aquella época tan lejana, los peregrinos del Paraíso y los peregrinos del tiempo se encontraron por primera vez en el mundo receptor del circuito número cuatro.

\par
%\textsuperscript{(291.2)}
\textsuperscript{26:5.2} Estos ayudantes de los peregrinos, que ejercen su actividad en el séptimo círculo de los mundos de Havona, dirigen su trabajo para los mortales ascendentes en tres divisiones principales: primero, la comprensión suprema de la Trinidad del Paraíso; segundo, la comprensión espiritual de la asociación Padre-Hijo; y tercero, el reconocimiento intelectual del Espíritu Infinito. Cada una de estas fases de enseñanza se divide en siete ramas de doce divisiones menores de setenta grupos secundarios; y cada uno de estos setenta agrupamientos secundarios de enseñanza es presentado en mil clasificaciones. En los círculos posteriores se proporciona una enseñanza más detallada, pero los ayudantes de los peregrinos enseñan un resumen de cada requisito del Paraíso.

\par
%\textsuperscript{(291.3)}
\textsuperscript{26:5.3} Éste es pues el curso primario o elemental con el que se enfrentan los peregrinos del espacio cuya fe ha sido probada y que tanto han viajado. Pero mucho antes de llegar a Havona, estos hijos ascendentes del tiempo han aprendido a deleitarse con las incertidumbres, a enriquecerse con las decepciones, a entusiasmarse con los fracasos aparentes, a estimularse en presencia de las dificultades, a mostrar un valor indomable frente a la inmensidad, y a ejercer una fe invencible cuando se enfrentan con el desafío de lo inexplicable. Hace mucho tiempo que el grito de guerra de estos peregrinos se ha vuelto: <<\textit{En unión con Dios, nada ---absolutamente nada--- es imposible}>>\footnote{\textit{Con Dios nada es imposible}: Gn 18:14; Jer 32:27; Mt 19:26; Mc 10:27; 14:36; Lc 1:37; 18:27.}.

\par
%\textsuperscript{(291.4)}
\textsuperscript{26:5.4} A los peregrinos del tiempo se les exige una cosa precisa en cada uno de los círculos de Havona; y aunque cada peregrino continúa bajo la tutela de los supernafines adaptados por su naturaleza a ayudar a este tipo particular de criatura ascendente, el curso que se ha de superar es bastante uniforme para todos los ascendentes que alcanzan el universo central. Este curso de consecución es cuantitativo, cualitativo y experiencial ---intelectual, espiritual y supremo.

\par
%\textsuperscript{(291.5)}
\textsuperscript{26:5.5} El tiempo tiene poca importancia en los círculos de Havona. Participa de una manera limitada en las posibilidades de progreso, pero el éxito es la prueba final y suprema. En el mismo momento en que vuestro asociado superáfico considere que estáis capacitados para pasar hacia el interior al círculo siguiente, seréis llevados ante los doce ayudantes del séptimo Espíritu de los Circuitos. Aquí se os pedirá que paséis las pruebas del círculo determinado por el superuniverso de vuestro origen y por el sistema donde habéis nacido. La conquista divina de este círculo tiene lugar en el mundo piloto, y consiste en el reconocimiento y en la comprensión espirituales del Espíritu Maestro del superuniverso del peregrino ascendente.

\par
%\textsuperscript{(291.6)}
\textsuperscript{26:5.6} Cuando el trabajo del círculo exterior de Havona ha terminado y el curso ofrecido ha sido superado, los ayudantes de los peregrinos llevan a sus sujetos al mundo piloto del círculo siguiente y los confían a los cuidados de los guías de la supremacía. Los ayudantes de los peregrinos siempre se quedan durante una temporada para contribuir a que el traslado sea agradable y beneficioso a la vez.

\section*{6. Los Guías de la Supremacía}
\par
%\textsuperscript{(292.1)}
\textsuperscript{26:6.1} A los ascendentes del espacio los denominan <<\textit{graduados espirituales}>> cuando los trasladan del séptimo al sexto círculo y los colocan bajo la supervisión directa de los guías de la supremacía. A estos guías no hay que confundirlos con los Guías de los Graduados ---que pertenecen a las Personalidades Superiores del Espíritu Infinito--- y que, con sus asociados servitales, ejercen su ministerio en todos los circuitos de Havona con los peregrinos tanto ascendentes como descendentes. Los guías de la supremacía sólo desempeñan su actividad en el sexto círculo del universo central.

\par
%\textsuperscript{(292.2)}
\textsuperscript{26:6.2} En este círculo es donde los ascendentes consiguen una nueva comprensión de la Divinidad Suprema. Durante su larga carrera en los universos evolutivos, los peregrinos del tiempo han experimentado una conciencia creciente de la realidad de un supercontrol todopoderoso de las creaciones espacio-temporales. Aquí, en este circuito de Havona, están a punto de encontrarse con la fuente de la unidad espacio-temporal residente en el universo central ---con la realidad espiritual de Dios Supremo.

\par
%\textsuperscript{(292.3)}
\textsuperscript{26:6.3} No sé muy bien cómo explicar lo que sucede en este círculo. Ninguna presencia personalizada de la Supremacía es perceptible para los ascendentes. En ciertos aspectos, las nuevas relaciones con el Séptimo Espíritu Maestro compensan esta imposibilidad de ponerse en contacto con el Ser Supremo. Pero independientemente de nuestra incapacidad para captar la técnica, cada criatura ascendente parece experimentar un crecimiento transformador, una nueva integración de su conciencia, una nueva espiritualización de su propósito, una nueva sensibilidad a la divinidad, que casi no se pueden explicar de manera satisfactoria sin suponer la actividad no revelada del Ser Supremo. Para aquellos de nosotros que han observado estas operaciones misteriosas, parece como si Dios Supremo otorgara afectuosamente a sus hijos experienciales, y hasta los mismos límites de sus capacidades experienciales, esos aumentos de comprensión intelectual, de perspicacia espiritual y de extensión de la personalidad que tanto necesitarán en todos sus esfuerzos por penetrar en el nivel de divinidad de la Trinidad de Supremacía, para alcanzar a las Deidades eternas y existenciales del Paraíso.

\par
%\textsuperscript{(292.4)}
\textsuperscript{26:6.4} Cuando los guías de la supremacía consideran que sus alumnos están maduros para avanzar, los llevan ante la comisión de los setenta, un grupo mixto que actúa como examinador en el mundo piloto del circuito número seis. Después de satisfacer a esta comisión en cuanto a su comprensión del Ser Supremo y de la Trinidad de Supremacía, los peregrinos reciben la confirmación de que pueden trasladarse al quinto circuito.

\section*{7. Los Guías de la Trinidad}
\par
%\textsuperscript{(292.5)}
\textsuperscript{26:7.1} Los guías de la Trinidad son los ministros incansables del quinto círculo de instrucción havoniana para los peregrinos progresivos del tiempo y del espacio. A los graduados espirituales los denominan aquí <<\textit{candidatos a la aventura de la Deidad}>>, puesto que es en este círculo, y bajo la dirección de los guías de la Trinidad, donde los peregrinos reciben una enseñanza avanzada sobre la Trinidad divina como preparación para intentar conseguir reconocer la personalidad del Espíritu Infinito. Aquí, los peregrinos ascendentes descubren el significado que tiene el verdadero estudio y el auténtico esfuerzo mental cuando empiezan a discernir la naturaleza del esfuerzo espiritual aún más agotador y mucho más arduo que necesitarán hacer para satisfacer las exigencias de la elevada meta que tienen que alcanzar en los mundos de este circuito.

\par
%\textsuperscript{(292.6)}
\textsuperscript{26:7.2} Los guías de la Trinidad son sumamente fieles y eficaces; y cada peregrino recibe la atención indivisa y disfruta del afecto total de un supernafín secundario perteneciente a esta orden. Un peregrino del tiempo no encontraría nunca a la primera persona accesible de la Trinidad del Paraíso si no fuera por la ayuda y la asistencia de estos guías y de la multitud de otros seres espirituales que se ocupan de instruir a los ascendentes sobre la naturaleza y la técnica de la cercana aventura de la Deidad.

\par
%\textsuperscript{(293.1)}
\textsuperscript{26:7.3} Después de terminar el curso de formación en este circuito, los guías de la Trinidad llevan a sus alumnos a su mundo piloto y los presentan ante una de las muchas comisiones trinas que funcionan para examinar y declarar aptos a los candidatos a la aventura de la Deidad. Estas comisiones están compuestas por un compañero finalitario, por uno de los directores del comportamiento perteneciente a la orden de los supernafines primarios, y por un Mensajero Solitario del espacio o un Hijo Trinitizado del Paraíso.

\par
%\textsuperscript{(293.2)}
\textsuperscript{26:7.4} Cuando un alma ascendente sale realmente hacia el Paraíso, sólo va acom-pañada por el trío de transporte: el asociado superáfico del círculo, el Guía de los Graduados y el siempre presente asociado servital de este último. Estas excursiones desde los círculos de Havona hasta el Paraíso son viajes de prueba; los ascendentes no poseen todavía el estado paradisiaco. No consiguen el estado residencial en el Paraíso hasta que no han pasado por el descanso final del tiempo, que tiene lugar después de haber alcanzado al Padre Universal y de haber recibido la acreditación final de los circuitos de Havona. No comparten la <<\textit{esencia de la divinidad}>> y el <<\textit{espíritu de la supremacía}>> hasta después del descanso divino, y entonces empiezan a trabajar realmente en el círculo de la eternidad y en presencia de la Trinidad.

\par
%\textsuperscript{(293.3)}
\textsuperscript{26:7.5} Los compañeros del trío de transporte del ascendente no son necesarios para permitirle que localice la presencia geográfica de la luminosidad espiritual de la Trinidad, sino más bien para proporcionar toda la ayuda posible a un peregrino en su difícil tarea de reconocer, discernir y comprender suficientemente al Espíritu Infinito como para efectuar el reconocimiento de su personalidad. Cualquier peregrino ascendente que se encuentre en el Paraíso puede discernir la presencia geográfica o localizada de la Trinidad; la gran mayoría es capaz de ponerse en contacto con la realidad intelectual de las Deidades, especialmente de la Tercera Persona, pero no todos pueden reconocer o ni siquiera comprender parcialmente la realidad de la presencia espiritual del Padre y del Hijo. Y todavía es más difícil obtener siquiera un mínimo de comprensión espiritual del Padre Universal.

\par
%\textsuperscript{(293.4)}
\textsuperscript{26:7.6} La búsqueda del Espíritu Infinito raras veces no logra consumarse, y cuando sus sujetos han triunfado en esta fase de la aventura de la Deidad, los guías de la Trinidad se preparan para trasladarlos a los cuidados de los descubridores del Hijo en el cuarto círculo de Havona.

\section*{8. Los Descubridores del Hijo}
\par
%\textsuperscript{(293.5)}
\textsuperscript{26:8.1} Al cuarto circuito de Havona se le llama a veces el <<\textit{circuito de los Hijos}>>. Desde los mundos de este circuito, los peregrinos ascendentes van al Paraíso para conseguir un contacto comprensivo con el Hijo Eterno, mientras que en los mundos de este circuito los peregrinos descendentes consiguen una nueva comprensión de la naturaleza y de la misión de los Hijos Creadores del tiempo y del espacio. En este circuito hay siete mundos en los que el cuerpo de reserva de los Migueles Paradisiacos mantienen escuelas especiales de servicio que ofrecen un ministerio mutuo a los peregrinos ascendentes y descendentes; en estos mundos de los Hijos Migueles es donde los peregrinos del tiempo y los peregrinos de la eternidad llegan por primera vez a una verdadera comprensión mutua. Las experiencias de este circuito son en muchos aspectos las más fascinantes de toda la estancia en Havona.

\par
%\textsuperscript{(294.1)}
\textsuperscript{26:8.2} Los descubridores del Hijo son los ministros superáficos de los mortales ascendentes del cuarto circuito. Además del trabajo general de preparar a sus candidatos para que comprendan las relaciones del Hijo Eterno con la Trinidad, estos descubridores del Hijo han de enseñar a sus sujetos de una manera tan completa que éstos tengan un éxito total: primero, comprendiendo espiritualmente al Hijo de forma adecuada; segundo, reconociendo satisfac-toriamente la personalidad del Hijo; y tercero, diferenciando apropiadamente al Hijo de la personalidad del Espíritu Infinito.

\par
%\textsuperscript{(294.2)}
\textsuperscript{26:8.3} Después de alcanzar al Espíritu Infinito ya no se pasan más exámenes. Las pruebas de los círculos interiores consisten en las acciones de los candidatos peregrinos cuando se encuentran envueltos en el abrazo de las Deidades. El progreso está determinado estrictamente por la espiritualidad del individuo, y nadie salvo los Dioses se atreven a juzgar esta posesión. En caso de fracaso nunca se indica una razón, y tampoco se reprende ni se critica nunca a los candidatos mismos ni a sus diversos tutores y guías. En el Paraíso, una decepción nunca se considera como una derrota; un aplazamiento nunca se contempla como una desgracia; los fracasos aparentes del tiempo nunca se confunden con los retrasos significativos de la eternidad.

\par
%\textsuperscript{(294.3)}
\textsuperscript{26:8.4} Hay pocos peregrinos que experimenten la demora de un fracaso aparente en la aventura de la Deidad. Casi todos alcanzan al Espíritu Infinito, aunque alguna que otra vez un peregrino del superuniverso número uno no lo consiga al primer intento. Los peregrinos que alcanzan al Espíritu raras veces no logran encontrar al Hijo; casi todos los que fracasan en la primera aventura proceden de los superuniversos tres y cinco. La gran mayoría de aquellos que no logran alcanzar al Padre en la primera aventura, después de haber encontrado al Espíritu y al Hijo, proceden del superuniverso número seis, aunque algunos que provienen de los números dos y tres tampoco tienen éxito. Todo esto parece indicar claramente que existe alguna buena y suficiente razón para estos fracasos aparentes; en realidad, se trata simplemente de retrasos inevitables.

\par
%\textsuperscript{(294.4)}
\textsuperscript{26:8.5} Los candidatos que han fracasado en la aventura de la Deidad son puestos bajo la jurisdicción de los jefes de la asignación, un grupo de supernafines primarios, y son devueltos al trabajo de los reinos del espacio durante un período no inferior a un milenio. Nunca regresan a su superuniverso natal, sino siempre a la supercreación más favorable para su reeducación como preparación para la segunda aventura de la Deidad. Después de este servicio regresan al círculo exterior de Havona por su propia iniciativa, se les acompaña de inmediato al círculo de su carrera interrumpida, y reanudan enseguida sus preparativos para la aventura de la Deidad. Los supernafines secundarios nunca dejan de guiar con éxito a sus sujetos en la segunda tentativa, y los mismos ministros superáficos, así como otros guías, atienden siempre a estos candidatos durante esta segunda aventura.

\section*{9. Los Guías del Padre}
\par
%\textsuperscript{(294.5)}
\textsuperscript{26:9.1} Cuando el alma del peregrino alcanza el tercer círculo de Havona, llega bajo la tutela de los guías del Padre, los ministros superáficos más antiguos, más cualificados y más experimentados. Los guías del Padre mantienen en los mundos de este circuito sus escuelas de sabiduría y sus facultades técnicas, donde todos los seres que viven en el universo central sirven como educadores. No se descuida nada que pueda ser de utilidad para una criatura del tiempo en esta aventura trascendente de conseguir la eternidad.

\par
%\textsuperscript{(294.6)}
\textsuperscript{26:9.2} Alcanzar al Padre Universal es el pasaporte para la eternidad, a pesar de los circuitos que queden por atravesar. Por eso se produce un acontecimiento de gran importancia en el mundo piloto del círculo número tres cuando el trío de transporte anuncia que la última aventura del tiempo está a punto de comenzar; que otra criatura del espacio trata de entrar en el Paraíso por las puertas de la eternidad.

\par
%\textsuperscript{(295.1)}
\textsuperscript{26:9.3} La prueba del tiempo casi ha terminado; la carrera hacia la eternidad casi ha concluido. Los días de incertidumbre están finalizando; la tentación de la duda se desvanece; el mandato de ser \textit{perfecto} ha sido obedecido. Desde el fondo mismo de la existencia inteligente, la criatura del tiempo y con una personalidad material ha ascendido las esferas evolutivas del espacio, mostrando así la viabilidad del plan de ascensión y demostrando para siempre la justicia y la rectitud del mandato del Padre Universal a sus humildes criaturas de los mundos: <<\textit{Sed perfectos como yo soy perfecto}>>\footnote{\textit{Sed perfectos}: Gn 17:1; 1 Re 8:61; Lv 19:2; Dt 18:13; Mt 5:48; 2 Co 13:11; Stg 1:4; 1 P 1:16.}.

\par
%\textsuperscript{(295.2)}
\textsuperscript{26:9.4} Paso a paso, vida tras vida, mundo tras mundo, la carrera ascendente ha sido superada y la meta de la Deidad ha sido alcanzada. La supervivencia es completa en su perfección, y la perfección está llena de la supremacía de la divinidad. El tiempo se pierde en la eternidad; el espacio queda engullido en una identidad y una armonía adoradora con el Padre Universal. Las transmisiones de Havona emiten los informes espaciales de gloria, la buena nueva de que en verdad las criaturas concienzudas de naturaleza animal y de origen material se han convertido real y eternamente, por medio de la ascensión evolutiva, en los hijos perfeccionados de Dios.

\section*{10. Los consejeros y los asesores}
\par
%\textsuperscript{(295.3)}
\textsuperscript{26:10.1} Los consejeros y los asesores superáficos del segundo círculo son los instructores de los hijos del tiempo en lo relacionado con la carrera de la eternidad. Alcanzar el Paraíso trae consigo unas responsabilidades de un orden nuevo y más elevado, y la estancia en el segundo círculo proporciona abundantes oportunidades para recibir el consejo provechoso de estos supernafines dedicados.

\par
%\textsuperscript{(295.4)}
\textsuperscript{26:10.2} Aquellos que no tienen éxito en su primer esfuerzo por alcanzar la Deidad son trasladados directamente desde el círculo de su fracaso al segundo círculo antes de ser devueltos al servicio de un superuniverso. Los consejeros y los asesores sirven pues también como consejeros y consoladores de estos peregrinos decepcionados. Acaban de enfrentarse con su mayor decepción, que no difiere de ninguna manera ---salvo en su magnitud--- de la larga lista de este tipo de experiencias sobre las que se han elevado, como por una escala, desde el caos hasta la gloria. Son los seres que han apurado la copa experiencial hasta las heces; y he observado que regresan temporalmente al servicio de los superuniversos como ministros amorosos del tipo más elevado para con los hijos del tiempo y las decepciones temporales.

\par
%\textsuperscript{(295.5)}
\textsuperscript{26:10.3} Después de una larga estancia en el circuito número dos, estos sujetos de la decepción son examinados por los consejos de la perfección que se reúnen en el mundo piloto de este círculo y reciben el certificado de haber pasado la prueba de Havona; y esto les concede, en lo que se refiere a su estado no espiritual, la misma posición en los universos del tiempo que si hubieran tenido realmente éxito en la aventura de la Deidad. El espíritu de estos candidatos era totalmente aceptable; su fracaso era inherente a alguna fase de su técnica de acercamiento o a alguna parte de su trasfondo experiencial.

\par
%\textsuperscript{(295.6)}
\textsuperscript{26:10.4} Los consejeros del círculo los llevan luego ante los jefes de la asignación que están en el Paraíso y son devueltos al servicio del tiempo en los mundos del espacio; y se marchan con regocijo y alegría a realizar las tareas de los tiempos y las épocas anteriores. Más adelante regresarán al círculo de su mayor decepción e intentarán de nuevo la aventura de la Deidad.

\par
%\textsuperscript{(296.1)}
\textsuperscript{26:10.5} Para los peregrinos que han tenido éxito en el segundo circuito, el estímulo de la incertidumbre evolutiva ha terminado, pero la aventura de la tarea eterna aún no ha empezado, y aunque la estancia en este círculo es totalmente agradable y muy provechosa, le falta una parte del entusiasmo esperanzador de los círculos anteriores. Son muchos los peregrinos que en esos momentos contemplan retrospectivamente la larguísima lucha con una envidia gozosa, deseando realmente poder regresar de algún modo a los mundos del tiempo y empezarlo todo otra vez, al igual que vosotros los mortales, cuando os acercáis a una edad avanzada, a veces miráis retrospectivamente las luchas de vuestra juventud y de vuestros primeros años de vida, y desearíais verdaderamente poder vivir vuestra vida otra vez.

\par
%\textsuperscript{(296.2)}
\textsuperscript{26:10.6} Pero la travesía del círculo más interior se encuentra ante ellos; poco después terminará el último sueño de transición y empezará la nueva aventura de la carrera eterna. Los consejeros y los asesores del segundo círculo empiezan a preparar a sus sujetos para este gran descanso final, el sueño inevitable que media siempre entre las etapas que marcan una época en la carrera ascendente.

\par
%\textsuperscript{(296.3)}
\textsuperscript{26:10.7} Cuando los peregrinos ascendentes que han alcanzado al Padre Universal concluyen la experiencia del segundo círculo, sus Guías de los Graduados siempre presentes promulgan la orden que les permitirá entrar en el círculo final. Estos guías conducen personalmente a sus sujetos hasta el círculo interior y los confían allí a la custodia de los complementos del descanso, la última orden de supernafines secundarios encargada de ayudar a los peregrinos del tiempo en los circuitos de los mundos de Havona.

\section*{11. Los Complementos del Descanso}
\par
%\textsuperscript{(296.4)}
\textsuperscript{26:11.1} Una gran parte del tiempo que pasan los ascendentes en el último circuito se dedica a continuar el estudio de los problemas inminentes relacionados con la residencia en el Paraíso. Una amplia y diversa multitud de seres, la mayoría de ellos no revelados, residen de manera permanente o transitoria en este anillo interior de los mundos de Havona. La mezcla de estos múltiples tipos proporciona a los complementos superáficos del descanso un ambiente rico en situaciones que utilizan eficazmente para favorecer la educación de los peregrinos ascendentes, especialmente en relación con los problemas de ajuste a los numerosos grupos de seres que pronto encontrarán en el Paraíso.

\par
%\textsuperscript{(296.5)}
\textsuperscript{26:11.2} Entre los seres que viven en este circuito interior se encuentran los hijos trinitizados por las criaturas. Los supernafines primarios y secundarios son los custodios generales del cuerpo conjunto de estos hijos, incluyendo a los descendientes trinitizados de los finalitarios mortales y a la progenie similar de los Ciudadanos del Paraíso. Algunos de estos hijos son abrazados por la Trinidad y enviados a servir en los supergobiernos, a otros les asignan tareas diversas, pero la gran mayoría se está reuniendo en el cuerpo conjunto que reside en los mundos perfectos del circuito interior de Havona. Aquí, bajo la supervisión de los supernafines, están siendo preparados para un trabajo futuro por un cuerpo especial innominado de Ciudadanos elevados del Paraíso que fueron, antes de la época de Grandfanda, los primeros asistentes ejecutivos de los Eternos de los Días. Existen muchas razones para suponer que estos dos grupos excepcionales de seres trinitizados trabajarán juntos en un lejano futuro, y no es la menor de ellas su destino común en las reservas del Cuerpo Paradisiaco de los Finalitarios Trinitizados.

\par
%\textsuperscript{(296.6)}
\textsuperscript{26:11.3} En este circuito más interior, los peregrinos ascendentes y descendentes fraternizan entre sí y con los hijos trinitizados por las criaturas. Al igual que sus padres, estos hijos obtienen grandes beneficios de la interasociación, y la misión especial de los supernafines es la de facilitar y asegurar la confraternidad entre los hijos trinitizados de los finalitarios mortales y los hijos trinitizados de los Ciudadanos del Paraíso. Los complementos superáficos del descanso no se interesan tanto en instruir a estos hijos como en fomentar su asociación comprensiva con los diversos grupos.

\par
%\textsuperscript{(297.1)}
\textsuperscript{26:11.4} Los mortales han recibido el mandato paradisiaco: <<\textit{Sed perfectos como vuestro Padre Paradisiaco es perfecto}>>\footnote{\textit{Sed perfectos}: Gn 17:1; 1 Re 8:61; Lv 19:2; Dt 18:13; Mt 5:48; 2 Co 13:11; Stg 1:4; 1 P 1:16.}. Los supernafines supervisores no dejan nunca de proclamar a estos hijos trinitizados del cuerpo conjunto: <<\textit{Sed comprensivos con vuestros hermanos ascendentes, al igual que los Hijos Creadores Paradisiacos los conocen y los aman}>>\footnote{\textit{Sed comprensivos con el hombre}: 1 Co 14:19.}.

\par
%\textsuperscript{(297.2)}
\textsuperscript{26:11.5} La criatura mortal debe encontrar a Dios. El Hijo Creador no se detiene nunca hasta que encuentra al hombre ---la criatura volitiva más humilde. No hay duda de que los Hijos Creadores y sus hijos mortales se están preparando para algún futuro servicio desconocido en el universo. Los dos atraviesan la gama del universo experiencial, y de esta manera se educan y se entrenan para su misión eterna. En todos los universos se está produciendo esta combinación única de lo humano y de lo divino, la mezcla de la criatura y del Creador. Los mortales irreflexivos se han referido a la manifestación de la misericordia y de la ternura divinas, especialmente hacia los débiles y a favor de los necesitados, como indicativas de un Dios antropomorfo. !`Qué error! Estas manifestaciones de misericordia y de indulgencia por parte de los seres humanos deberían considerarse más bien como una prueba de que el hombre mortal está habitado por el espíritu del Dios viviente; que la criatura está, después de todo, motivada por la divinidad.

\par
%\textsuperscript{(297.3)}
\textsuperscript{26:11.6} Hacia el final de su estancia en el primer círculo, los peregrinos ascendentes encuentran por primera vez a los instigadores del descanso de la orden primaria de los supernafines. Son los ángeles del Paraíso que salen para dar la bienvenida a aquellos que se hallan en el umbral de la eternidad y para completar su preparación con vistas al sueño de transición de la última resurrección. No sois realmente hijos del Paraíso hasta que no habéis atravesado el círculo interior y habéis experimentado la resurrección de la eternidad después del sueño final del tiempo. Los peregrinos perfeccionados empiezan este descanso, se duermen, en el primer círculo de Havona, pero se despiertan en las orillas del Paraíso. De todos aquellos que ascienden a la Isla eterna, sólo los que llegan de esta manera son los hijos de la eternidad; los demás van como visitantes, como invitados, sin tener la condición de residentes.

\par
%\textsuperscript{(297.4)}
\textsuperscript{26:11.7} Y ahora, en la culminación de la carrera de Havona, cuando vosotros los mortales os dormís en el mundo piloto del circuito interior, no emprendéis a solas vuestro descanso como lo hicisteis en los mundos de vuestro origen cuando cerrasteis los ojos en el sueño natural de la muerte física, ni como lo hicisteis cuando entrasteis en el largo trance de transición antes de viajar hacia Havona. Ahora, mientras os preparáis para el descanso de la consecución, vuestro asociado de tantos años del primer círculo, el majestuoso complemento del descanso, se coloca a vuestro lado, se prepara para emprender el descanso junto a vosotros, como garantía de Havona de que vuestra transición ha concluido y de que sólo estáis a la espera de los toques finales de la perfección.

\par
%\textsuperscript{(297.5)}
\textsuperscript{26:11.8} Vuestra primera transición fue en verdad la muerte; la segunda fue un sueño ideal, y ahora la tercera metamorfosis es el verdadero descanso, la relajación de todos los tiempos.

\par
%\textsuperscript{(297.6)}
\textsuperscript{26:11.9} [Presentado por un Perfeccionador de la Sabiduría procedente de Uversa.]


\chapter{Documento 27. El ministerio de los Supernafines Primarios}
\par
%\textsuperscript{(298.1)}
\textsuperscript{27:0.1} LOS supernafines primarios son los servidores celestiales de las Deidades en la Isla eterna del Paraíso. Nunca se ha sabido que se hayan desviado de los caminos de la luz y de la rectitud. Sus listas nominales están al completo; desde la eternidad, ningún miembro de esta magnífica hueste se ha perdido. Estos elevados supernafines son seres perfectos, supremos en perfección, pero no son absonitos ni tampoco absolutos. Como poseen la esencia de la perfección, estos hijos del Espíritu Infinito trabajan de manera intercambiable y a voluntad en todas las fases de sus múltiples funciones. No ejercen ampliamente su actividad fuera del Paraíso, aunque sí participan en las diversas asambleas milenarias y reuniones colectivas del universo central. También salen al exterior como mensajeros especiales de las Deidades, y ascienden en gran número para convertirse en Asesores Técnicos.

\par
%\textsuperscript{(298.2)}
\textsuperscript{27:0.2} A los supernafines primarios también los ponen al mando de las huestes seráficas que ejercen su ministerio en los mundos aislados debido a una rebelión. Cuando un hijo Paradisiaco se dona en dicho mundo, termina su misión, asciende hacia el Padre Universal, es aceptado y regresa como libertador acreditado de ese mundo aislado, los jefes de la asignación siempre designan a un supernafín primario para que asuma el mando de los espíritus ministrantes que están de servicio en la esfera recién recuperada. Los supernafines que efectúan este servicio especial se turnan periódicamente. En Urantia, el actual <<\textit{jefe de los serafines}>> es el segundo de esta orden que está de servicio desde los tiempos de la donación de Cristo Miguel.

\par
%\textsuperscript{(298.3)}
\textsuperscript{27:0.3} Los supernafines primarios han servido desde la eternidad en la Isla de Luz y han salido a los mundos del espacio en misiones de dirección, pero tal como están clasificados actualmente sólo han ejercido su actividad desde la llegada al Paraíso de los peregrinos del tiempo procedentes de Havona. Estos ángeles elevados desempeñan ahora su ministerio principalmente en los siete tipos de servicio siguientes:

\par
%\textsuperscript{(298.4)}
\textsuperscript{27:0.4} 1. Los Conductores de la Adoración.

\par
%\textsuperscript{(298.5)}
\textsuperscript{27:0.5} 2. Los Maestros de Filosofía.

\par
%\textsuperscript{(298.6)}
\textsuperscript{27:0.6} 3. Los Custodios del Conocimiento.

\par
%\textsuperscript{(298.7)}
\textsuperscript{27:0.7} 4. Los Directores de la Conducta.

\par
%\textsuperscript{(298.8)}
\textsuperscript{27:0.8} 5. Los Intérpretes de la Ética.

\par
%\textsuperscript{(298.9)}
\textsuperscript{27:0.9} 6. Los Jefes de la Asignación.

\par
%\textsuperscript{(298.10)}
\textsuperscript{27:0.10} 7. Los Instigadores del Descanso.

\par
%\textsuperscript{(298.11)}
\textsuperscript{27:0.11} Los peregrinos ascendentes no caen bajo la influencia directa de estos supernafines hasta que no consiguen residir realmente en el Paraíso, y luego pasan por una experiencia de formación bajo la dirección de estos ángeles en el orden inverso al que han sido nombrados. Es decir, entráis en vuestra carrera paradisiaca bajo la tutela de los instigadores del descanso y, después de sucesivas temporadas con las órdenes intermedias, termináis este período de formación con los conductores de la adoración. Después de esto estáis preparados para empezar la carrera sin fin de un finalitario.

\section*{1. Los Instigadores del Descanso}
\par
%\textsuperscript{(299.1)}
\textsuperscript{27:1.1} Los instigadores del descanso son los inspectores del Paraíso que salen de la Isla central hacia el circuito interior de Havona para colaborar allí con sus colegas, los complementos del descanso de la orden secundaria de los supernafines. El elemento esencial para disfrutar del Paraíso es el descanso, el descanso divino; y estos instigadores del descanso son los instructores finales que preparan a los peregrinos del tiempo para su primera toma de contacto con la eternidad. Empiezan su trabajo en el círculo final de consecución del universo central y lo continúan cuando el peregrino se despierta del último sueño de transición, del sueño que confiere a una criatura del espacio el grado de entrar en el reino de lo eterno.

\par
%\textsuperscript{(299.2)}
\textsuperscript{27:1.2} El descanso es de naturaleza séptuple: Existe el descanso del sueño y de la diversión en las órdenes inferiores de vida, el del descubrimiento en los seres superiores y el de la adoración en los tipos más elevados de personalidades espirituales. También existe el descanso normal de la absorción de energía, el de la recarga de los seres en energía física o espiritual. Y luego existe el sueño de transición, el sueño inconsciente cuando un ser está enserafinado, cuando está de paso de una esfera a otra. Completamente diferente a todos los anteriores es el sueño profundo de la metamorfosis, el descanso de transición entre una fase del ser y otra, entre una vida y otra, entre un estado de existencia y otro, el sueño que acompaña siempre a la transición desde un \textit{estado} universal concreto, en contraste con la evolución a través de las diversas \textit{fases} de un estado determinado.

\par
%\textsuperscript{(299.3)}
\textsuperscript{27:1.3} Pero el último sueño metamórfico es algo más que los sueños de transición anteriores que marcaron la obtención de los estados sucesivos de la carrera ascendente; gracias a él las criaturas del tiempo y del espacio atraviesan los límites más interiores de lo temporal y de lo espacial para conseguir el estado residencial en las moradas sin tiempo y sin espacio del Paraíso. Los instigadores y los complementos del descanso son tan esenciales para esta metamorfosis trascendente como los serafines y los seres asociados lo son para que la criatura mortal sobreviva a la muerte.

\par
%\textsuperscript{(299.4)}
\textsuperscript{27:1.4} Emprendéis el descanso en el circuito final de Havona y sois resucitados eternamente en el Paraíso. Y cuando os repersonalizáis espiritualmente allí, reconocéis inmediatamente que el instigador del descanso que os da la bienvenida a las orillas eternas es el mismo supernafín primario que provocó vuestro sueño final en el circuito más interior de Havona; y os acordaréis de vuestro último gran esfuerzo de fe cuando os preparasteis para confiar una vez más la custodia de vuestra identidad en las manos del Padre Universal\footnote{\textit{Identidad durante la muerte}: Lc 23:46.}.

\par
%\textsuperscript{(299.5)}
\textsuperscript{27:1.5} El último descanso del tiempo se ha disfrutado; el último sueño de transición se ha experimentado; ahora os despertáis a la vida perpetua en las orillas de la morada eterna. <<\textit{Y ya no habrá más sueño. La presencia de Dios y de su Hijo están ante vosotros y sois eternamente sus servidores; habéis visto su rostro y su nombre es vuestro espíritu. Allí ya no habrá más noche; y no necesitan la luz del Sol porque la Gran Fuente-Centro les da luz; vivirán para siempre jamás. Y Dios enjugará todas las lágrimas de sus ojos; ya no habrá más muerte, ni tristeza ni llanto, y tampoco habrá más dolor, porque las antiguas cosas han desaparecido}>>\footnote{\textit{Ya no habrá más noche}: Ap 22:3-5. \textit{Vivirán para siempre}: Hch 13:46-48. \textit{Enjugará todas las lágrimas}: Ap 7:17; 21:4. \textit{Vida eterna}: Dn 12:2; Mt 19:16,29; 25:46; Mc 10:17,30; Lc 10:25; 18:18,30; Jn 3:15-16,36; 4:14,36; 5:24,39; 6:27,40,47; 6:51,68; 8:51-52; 10:28; 11:25-26; 12:25,50; 17:2-3; Ro 2:7; 5:21; 6:22-23; Gl 6:8; 1 Ti 1:16; 6:12,19; Tit 1:2; 3:7; 1 Jn 1:2; 2:25; 3:15; 5:11,13,20; Jud 1:21; Ap 22:5.}.

\section*{2. Los Jefes de la Asignación}
\par
%\textsuperscript{(300.1)}
\textsuperscript{27:2.1} Se trata del grupo que es designado de vez en cuando por el jefe de los supernafines, <<\textit{el ángel modelo original}>>\footnote{\textit{Modelo original}: Ex 25:40; Ez 43:10; 1 Ti 1:16; Heb 8:5; 9:23.}, para que presida la organización de las tres órdenes de estos ángeles ---primaria, secundaria y terciaria. Como cuerpo, los supernafines son totalmente autónomos y se reglamentan ellos mismos, excepto en lo que se refiere a las funciones de su jefe mutuo, el primer ángel del Paraíso, que siempre dirige a todas estas personalidades espirituales.

\par
%\textsuperscript{(300.2)}
\textsuperscript{27:2.2} Los ángeles de la asignación tienen mucho que ver con los mortales glorificados que residen en el Paraíso antes de ser admitidos en el Cuerpo de la Finalidad. El estudio y la instrucción no son las ocupaciones exclusivas de los que llegan al Paraíso; el servicio también juega su papel esencial en las experiencias educativas prefinalitarias del Paraíso. Y he observado que cuando los mortales ascendentes disfrutan de períodos de ocio, muestran una predilección por fraternizar con el cuerpo de reserva de los jefes superáficos de la asignación.

\par
%\textsuperscript{(300.3)}
\textsuperscript{27:2.3} Cuando vosotros, los ascendentes mortales, llegáis al Paraíso, vuestras relaciones sociales suponen mucho más que un contacto con una gran cantidad de seres elevados y divinos y con una multitud familiar de compañeros mortales glorificados. También tenéis que fraternizar con más de tres mil órdenes diferentes de Ciudadanos del Paraíso, con los diversos grupos de Trascendentales y con otros numerosos tipos de habitantes del Paraíso, tanto permanentes como transitorios, que no han sido revelados en Urantia. Después de un contacto ininterrumpido con estos poderosos intelectos del Paraíso, es muy reposante charlar con los tipos angélicos de mente; a los mortales del tiempo les trae el recuerdo de los serafines, con quienes han tenido un contacto tan prolongado y una asociación tan reconfortante.

\section*{3. Los Intérpretes de la Ética}
\par
%\textsuperscript{(300.4)}
\textsuperscript{27:3.1} Cuanto más os eleváis en la escala de la vida, más atención tenéis que prestar a la ética universal. La conciencia ética es simplemente el reconocimiento, por parte de un individuo, de los derechos inherentes a la existencia de todos los demás individuos. Pero la ética espiritual trasciende de lejos el concepto mortal e incluso morontial de las relaciones personales y colectivas.

\par
%\textsuperscript{(300.5)}
\textsuperscript{27:3.2} La ética ha sido debidamente enseñada y adecuadamente aprendida por los peregrinos del tiempo durante su larga ascensión hacia las glorias del Paraíso. A medida que esta carrera ascendente hacia el interior se ha desarrollado desde los mundos nativos del espacio, los ascendentes han continuado añadiendo un grupo tras otro a su círculo cada vez mayor de asociados universales. A cada nuevo grupo de colegas que se encuentra hay que añadir un nivel más de ética que hay que reconocer y acatar hasta que, en el momento en que los mortales ascendentes alcanzan el Paraíso, necesitan realmente a alguien que les proporcione un consejo útil y amistoso en relación con las interpretaciones éticas. No necesitan que les enseñen la ética, pero a medida que se enfrentan con la tarea extraordinaria de ponerse en contacto con tantas cosas nuevas, sí necesitan que les \textit{interpreten} adecuadamente aquello que han aprendido tan laboriosamente.

\par
%\textsuperscript{(300.6)}
\textsuperscript{27:3.3} Los intérpretes de la ética son de una ayuda inestimable para los que llegan al Paraíso, pues los ayudan a ajustarse a los numerosos grupos de seres majestuosos durante el agitado período que se extiende desde que consiguen el estado residencial hasta que son admitidos oficialmente en el Cuerpo de los Finalitarios Mortales. Los peregrinos ascendentes ya se han encontrado con una gran parte de los numerosos tipos de Ciudadanos del Paraíso en los siete circuitos de Havona. Los mortales glorificados también han disfrutado de un contacto íntimo con los hijos del cuerpo conjunto, trinitizados por las criaturas, en el circuito interior de Havona, donde estos seres reciben una gran parte de su educación. Y en los otros circuitos, los peregrinos ascendentes se han encontrado con numerosos residentes no revelados del sistema Paraíso-Havona que están siguiendo allí una formación colectiva como preparación para las tareas no reveladas del futuro.

\par
%\textsuperscript{(301.1)}
\textsuperscript{27:3.4} Todo este compañerismo celestial es invariablemente mutuo. Como mortales ascendentes no sólo obtenéis beneficios de estos compañeros universales sucesivos y de estas numerosas órdenes de asociados cada vez más divinos, sino que también comunicáis a cada uno de estos seres fraternales alguna cosa de vuestra propia personalidad y de vuestra experiencia que hará que cada uno de ellos sea para siempre diferente y mejor por haber estado asociado con un mortal ascendente de los mundos evolutivos del tiempo y del espacio.

\section*{4. Los Directores de la Conducta}
\par
%\textsuperscript{(301.2)}
\textsuperscript{27:4.1} Una vez que ya han sido plenamente instruidos en la ética de las relaciones paradisiacas ---que no son ni unas formalidades sin sentido ni los dictados de unas castas artificiales, sino más bien unas convenciones inherentes--- a los mortales ascendentes les resulta útil recibir el consejo de los directores superáficos de la conducta, los cuales enseñan a los nuevos miembros de la sociedad del Paraíso los usos de la conducta perfecta de los seres elevados que residen en la Isla central de Luz y de Vida.

\par
%\textsuperscript{(301.3)}
\textsuperscript{27:4.2} La armonía es la tónica del universo central, y en el Paraíso prevalece un orden perceptible. Una conducta adecuada es esencial para progresar por medio del conocimiento, y a través de la filosofía, hasta las alturas espirituales de la adoración espontánea. Existe una técnica divina para acercarse a la Divinidad; y para adquirir esta técnica los peregrinos deben esperar hasta llegar al Paraíso. El espíritu de esta técnica ha sido impartido en los círculos de Havona, pero los toques finales del entrenamiento de los peregrinos del tiempo sólo se pueden aplicar después de que alcanzan realmente la Isla de Luz.

\par
%\textsuperscript{(301.4)}
\textsuperscript{27:4.3} Toda conducta en el Paraíso es enteramente espontánea, natural y libre en todos los sentidos. Pero existe sin embargo una manera adecuada y perfecta de hacer las cosas en la Isla eterna, y los directores de la conducta siempre están al lado de los <<\textit{extraños que están puertas adentro}>>\footnote{\textit{Extraños que están puertas adentro}: Ex 20:10; Dt 5:14; 31:12.} para instruirlos y guiar sus pasos de tal manera que se encuentren perfectamente a gusto, y capacitar al mismo tiempo a los peregrinos para que eviten la confusión y la incertidumbre que por otra parte serían inevitables. Una confusión sin fin sólo se podía evitar mediante estas disposiciones; y la confusión no aparece nunca en el Paraíso.

\par
%\textsuperscript{(301.5)}
\textsuperscript{27:4.4} Estos directores de la conducta sirven realmente como educadores y guías glorificados. Se ocupan principalmente de instruir a los nuevos residentes mortales acerca de una serie casi interminable de situaciones nuevas y de usos desconocidos. A pesar de toda la larga preparación para residir allí y del largo viaje para llegar hasta allí, el Paraíso sigue siendo indeciblemente extraño e inesperadamente nuevo para aquellos que consiguen finalmente el estado de residentes.

\section*{5. Los Custodios del Conocimiento}
\par
%\textsuperscript{(301.6)}
\textsuperscript{27:5.1} Los custodios superáficos del conocimiento son las <<\textit{epístolas vivientes}>>\footnote{\textit{Epístolas vivientes}: 2 Co 3:1-3.} superiores, conocidas y leídas por todos los que viven en el Paraíso. Son los anales divinos de la verdad, los libros vivientes del conocimiento verdadero. Habéis oído hablar de crónicas en el <<\textit{libro de la vida}>>\footnote{\textit{Libro de la vida}: Flp 4:3; Ap 3:5; Ap 13:8; Ap 17:8; Ap 20:12,15; Ap 21:27; Ap 22:19.}. Los custodios del conocimiento son esos libros vivientes, esas crónicas de la perfección impresas en las tablillas eternas de la vida divina y de la seguridad suprema. Son en realidad unas bibliotecas automáticas y vivientes. Los hechos de los universos son inherentes a estos supernafines primarios, y están efectivamente registrados en estos ángeles; y también es imposible de manera inherente que una falsedad consiga alojarse en la mente de estos depositarios perfectos y repletos de la verdad de la eternidad y de la información del tiempo.

\par
%\textsuperscript{(302.1)}
\textsuperscript{27:5.2} Estos custodios dirigen unos cursos informales de instrucción para los residentes de la Isla eterna, pero su función principal es la de servir de consulta y de comprobación. Todo residente del Paraíso puede tener a su lado a voluntad al depositario viviente del hecho o de la verdad particulares que desea conocer. En el extremo norte de la Isla se encuentran disponibles los descubridores vivientes del conocimiento, que designarán al director del grupo que posee la información que se busca, y aparecerán de inmediato los brillantes seres que \textit{son} la cosa misma que deseáis saber. Ya no necesitáis buscar la iluminación en las páginas escritas con grandes letras; ahora comulgáis cara a cara con la inteligencia viviente. El conocimiento supremo lo obtenéis así de los seres vivientes que son sus custodios finales.

\par
%\textsuperscript{(302.2)}
\textsuperscript{27:5.3} Cuando localicéis al supernafín que es exactamente aquello que deseáis verificar, encontraréis a vuestra disposición \textit{todos} los hechos conocidos de todos los universos, porque estos custodios del conocimiento son los resúmenes finales y vivientes de la inmensa cadena de ángeles registradores que se extiende desde los serafines y los seconafines de los universos locales y los superuniversos hasta los jefes archivistas de los supernafines terciarios en Havona. Y esta acumulación viviente de conocimientos es distinta a la de los archivos oficiales del Paraíso, que son el resumen acumulado de la historia universal.

\par
%\textsuperscript{(302.3)}
\textsuperscript{27:5.4} La sabiduría de la verdad tiene su origen en la divinidad del universo central, pero el conocimiento, el conocimiento experiencial, tiene en gran parte sus comienzos en los dominios del tiempo y del espacio ---de ahí la necesidad de mantener las extensas organizaciones superuniversales de los serafines y los supernafines registradores patrocinadas por los Registradores Celestiales.

\par
%\textsuperscript{(302.4)}
\textsuperscript{27:5.5} Estos supernafines primarios que poseen de manera inherente el conocimiento universal son también los responsables de su organización y de su clasificación. Al constituirse a sí mismos como biblioteca de consulta viviente del universo de universos, han clasificado el conocimiento en siete grandes grupos, y cada uno contiene cerca de un millón de subdivisiones. La facilidad con que los residentes del Paraíso pueden consultar esta inmensa reserva de conocimientos se debe únicamente a los esfuerzos voluntarios y sabios de los custodios del conocimiento. Los custodios son también los elevados educadores del universo central, distribuyendo abundantemente sus tesoros vivientes a todos los seres de cualquier circuito de Havona, y son utilizados ampliamente, aunque de forma indirecta, por las cortes de los Ancianos de los Días. Pero esta biblioteca viviente, que está a la disposición del universo central y de los superuniversos, no está al alcance de las creaciones locales. En los universos locales, los beneficios del conocimiento paradisiaco sólo se pueden conseguir por vía indirecta y por reflectividad.

\section*{6. Los Maestros de Filosofía}
\par
%\textsuperscript{(302.5)}
\textsuperscript{27:6.1} Al lado de la satisfacción suprema de la adoración se encuentra el regocijo de la filosofía. Nunca subiréis tan alto ni avanzaréis tan lejos como para que no queden mil misterios que necesitarán el empleo de la filosofía para intentar solucionarlos.

\par
%\textsuperscript{(302.6)}
\textsuperscript{27:6.2} A los filósofos maestros del Paraíso les encanta guiar la mente de sus habitantes, tanto nativos como ascendentes, en la tarea estimulante de intentar resolver los problemas del universo. Estos maestros superáficos de filosofía son los <<\textit{sabios del cielo}>>, los seres de sabiduría que utilizan la verdad del conocimiento y los hechos de la experiencia en sus esfuerzos por dominar lo desconocido. Con ellos, el conocimiento llega hasta la verdad y la experiencia asciende hasta la sabiduría. En el Paraíso, las personalidades ascendentes del espacio experimentan la cúspide del ser: tienen el conocimiento; conocen la verdad; pueden filosofar ---pensar en la verdad; incluso pueden tratar de abarcar los conceptos del Último e intentar comprender las técnicas de los Absolutos.

\par
%\textsuperscript{(303.1)}
\textsuperscript{27:6.3} En el extremo meridional del inmenso dominio del Paraíso, los maestros de filosofía dirigen cursos minuciosos en las setenta divisiones funcionales de la sabiduría. Aquí disertan sobre los planes y los propósitos de la Infinidad y tratan de coordinar las experiencias, y de componer el conocimiento, de todos los que tienen acceso a su sabiduría. Han desarrollado una actitud muy especializada hacia diversos problemas del universo, pero sus conclusiones finales están siempre de acuerdo de manera uniforme.

\par
%\textsuperscript{(303.2)}
\textsuperscript{27:6.4} Estos filósofos del Paraíso enseñan mediante todos los métodos posibles de instrucción, incluyendo la técnica gráfica superior de Havona y ciertos métodos paradisiacos para comunicar la información. Todas estas técnicas superiores para impartir el conocimiento y transmitir las ideas sobrepasan por completo la capacidad de comprensión de la mente humana incluso más desarrollada. Una hora de instrucción en el Paraíso equivaldría a diez mil años de métodos de memorización de Urantia. No podéis comprender estas técnicas de comunicación, y no existe sencillamente nada en la experiencia de los mortales con las que se puedan comparar, nada a lo que se puedan asemejar.

\par
%\textsuperscript{(303.3)}
\textsuperscript{27:6.5} Los maestros de filosofía disfrutan de manera suprema comunicando su interpretación del universo de universos a aquellos seres que han ascendido desde los mundos del espacio. Y aunque la filosofía nunca pueda ser tan firme en sus conclusiones como los hechos del conocimiento y las verdades de la experiencia, sin embargo, cuando hayáis escuchado a estos supernafines primarios disertar sobre los problemas no resueltos de la eternidad y las actuaciones de los Absolutos, experimentaréis una satisfacción cierta y duradera respecto a estas cuestiones no dominadas.

\par
%\textsuperscript{(303.4)}
\textsuperscript{27:6.6} Estas actividades intelectuales del Paraíso no se retransmiten; la filosofía de la perfección sólo está disponible para aquellos que se encuentran personalmente presentes. Las creaciones que rodean al Paraíso sólo conocen estas enseñanzas por medio de aquellos que han pasado por esta experiencia, y que han llevado posteriormente esta sabiduría a los universos del espacio.

\section*{7. Los Conductores de la Adoración}
\par
%\textsuperscript{(303.5)}
\textsuperscript{27:7.1} La adoración es el privilegio más elevado y el deber primero de todas las inteligencias creadas. La adoración es el acto consciente y gozoso de reconocer y de admitir la verdad y el hecho de las relaciones íntimas y personales entre los Creadores y sus criaturas. La calidad de la adoración está determinada por la profundidad de la percepción de la criatura; y a medida que progresa el conocimiento del carácter infinito de los Dioses, el acto de adorar se vuelve cada vez más global hasta que alcanza finalmente la gloria de la delicia experiencial más elevada y del placer más exquisito que conocen los seres creados.

\par
%\textsuperscript{(303.6)}
\textsuperscript{27:7.2} Aunque la Isla del Paraíso contiene ciertos lugares para la adoración, el Paraíso es más bien un inmenso santuario de servicio divino. La adoración es la pasión primera y dominante de todos los que se elevan hasta sus orillas maravillosas ---el arrebato espontáneo de los seres que han aprendido lo suficiente de Dios como para llegar a su presencia. Círculo tras círculo, durante el viaje hacia el interior a través de Havona, la adoración es una pasión creciente hasta que, en el Paraíso, se hace necesario dirigir su expresión y controlarla de otras maneras.

\par
%\textsuperscript{(304.1)}
\textsuperscript{27:7.3} Las explosiones periódicas, espontáneas, colectivas y otros arrebatos especiales de adoración suprema y de alabanza espiritual que se disfrutan en el Paraíso son conducidos bajo el mando de un cuerpo especial de supernafines primarios. Bajo la dirección de estos conductores de la adoración, este homenaje consigue la meta del placer supremo de la criatura y alcanza las alturas en las que la expresión sublime de sí mismo y el disfrute personal son perfectos. Todos los supernafines primarios anhelan ser conductores de la adoración; y todos los seres ascendentes disfrutarían permaneciendo para siempre en la actitud de adoración si los jefes de la asignación no dispersaran periódicamente estas reuniones. Pero a ningún ser ascendente se le pide nunca que emprenda las tareas del servicio eterno hasta que no haya alcanzado la plena satisfacción en la adoración.

\par
%\textsuperscript{(304.2)}
\textsuperscript{27:7.4} Los conductores de la adoración tienen la tarea de enseñar la adoración a las criaturas ascendentes de tal manera que les permita conseguir esta satisfacción de expresarse ellos mismos y al mismo tiempo sean capaces de prestar atención a las actividades esenciales del régimen del Paraíso. Sin el mejoramiento de la técnica de la adoración, el mortal medio que alcanza el Paraíso necesitaría cientos de años para expresar de forma plena y satisfactoria sus emociones de apreciación inteligente y de gratitud ascendente. Los conductores de la adoración abren unas vías de expresión nuevas y hasta ese momento desconocidas para que estos hijos maravillosos de las entrañas del espacio y de las tribulaciones del tiempo puedan conseguir en mucho menos tiempo las plenas satisfacciones de la adoración.

\par
%\textsuperscript{(304.3)}
\textsuperscript{27:7.5} Todas las artes de todos los seres del universo entero que son capaces de intensificar y de exaltar las aptitudes de la expresión de sí mismo y la comunicación de la apreciación se emplean al máximo de su capacidad para adorar a las Deidades del Paraíso. \textit{La adoración es la alegría supremade la existencia en el Paraíso;} es el entretenimiento refrescante del Paraíso. Aquello que el entretenimiento hace por vuestra mente agotada en la Tierra, la adoración lo hará por vuestra alma perfeccionada en el Paraíso. La forma de adorar en el Paraíso se encuentra totalmente más allá de la comprensión de los mortales, pero podéis empezar a apreciar su espíritu incluso aquí abajo en Urantia, porque los espíritus de los Dioses residen ahora mismo en vosotros, se ciernen sobre vosotros y os incitan a la verdadera adoración.

\par
%\textsuperscript{(304.4)}
\textsuperscript{27:7.6} En el Paraíso hay momentos y lugares designados para la adoración, pero no son adecuados para acomodar el desbordamiento cada vez mayor de las emociones espirituales de la inteligencia creciente y del reconocimiento en expansión de la divinidad en los seres brillantes de la ascensión experiencial a la Isla eterna. Desde los tiempos de Grandfanda, los supernafines nunca han sido capaces de acomodar plenamente el espíritu de adoración en el Paraíso. Siempre hay un exceso de deseo de adorar, si se mide por la preparación para ella. Y esto sucede porque las personalidades con una perfección inherente nunca pueden apreciar plenamente las asombrosas reacciones de las emociones espirituales de unos seres que han efectuado su camino hacia arriba de forma lenta y laboriosa hasta la gloria del Paraíso, partiendo de las profundidades de las tinieblas espirituales de los mundos inferiores del tiempo y del espacio. Cuando estos ángeles y los mortales del tiempo alcanzan la presencia de los Poderes del Paraíso, se produce la expresión de las emociones acumuladas durante siglos, un espectáculo asombroso para los ángeles del Paraíso y que provoca la alegría suprema de la satisfacción divina en las Deidades del Paraíso.

\par
%\textsuperscript{(304.5)}
\textsuperscript{27:7.7} A veces todo el Paraíso se sumerge en una marea dominante de expresión espiritual y adoradora. A menudo los conductores de la adoración no pueden controlar estos fenómenos, hasta que aparece la triple fluctuación de luz de la morada de la Deidad, indicando que el corazón divino de los Dioses está plena y completamente satisfecho con la adoración sincera de los residentes del Paraíso, los ciudadanos perfectos de la gloria y las criaturas ascendentes del tiempo. !Qué triunfo técnico! !Qué fructificación del plan y del propósito eternos de los Dioses cuando el amor inteligente del hijo creado llena de satisfacción el amor infinito del Padre Creador!

\par
%\textsuperscript{(305.1)}
\textsuperscript{27:7.8} Después de conseguir la satisfacción suprema de la plenitud de la adoración, estáis cualificados para ser admitidos en el Cuerpo de la Finalidad. La carrera ascendente casi ha terminado, y se prepara la celebración del séptimo jubileo. El primer jubileo señaló el acuerdo del mortal con su Ajustador del Pensamiento cuando se selló la intención de sobrevivir; el segundo fue el despertar en la vida morontial; el tercero fue la fusión con el Ajustador del Pensamiento; el cuarto fue el despertar en Havona; el quinto celebró el descubrimiento del Padre Universal; y el sexto jubileo fue el acontecimiento del despertar en el Paraíso después del sueño de tránsito final del tiempo. El séptimo jubileo señala la entrada en el cuerpo finalitario de los mortales y el comienzo del servicio en la eternidad. Cuando un finalitario alcance la séptima fase de su realización espiritual, este hecho señalará probablemente la celebración del primer jubileo de la eternidad.

\par
%\textsuperscript{(305.2)}
\textsuperscript{27:7.9} Y así termina la historia de los supernafines del Paraíso, la orden más elevada de todos los espíritus ministrantes, esos seres que, como clase universal, os acompañan siempre desde el mundo de vuestro origen hasta que los conductores de la adoración se despiden finalmente de vosotros cuando prestáis a la Trinidad el juramento de la eternidad y sois enrolados en el Cuerpo de los Mortales de la Finalidad.

\par
%\textsuperscript{(305.3)}
\textsuperscript{27:7.10} El servicio interminable para la Trinidad del Paraíso está a punto de empezar; y ahora el finalitario se encuentra frente a frente con el desafío de Dios Último.

\par
%\textsuperscript{(305.4)}
\textsuperscript{27:7.11} [Presentado por un Perfeccionador de la Sabiduría procedente de Uversa.]


\chapter{Documento 28. Los espíritus ministrantes de los superuniversos}
\par
%\textsuperscript{(306.1)}
\textsuperscript{28:0.1} AL IGUAL que los supernafines son las huestes angélicas del universo central y los serafines lo son de los universos locales, los seconafines son los espíritus ministrantes de los superuniversos. Sin embargo, en grado de divinidad y en potencial de supremacía, estos hijos de los Espíritus Reflectantes se parecen mucho más a los supernafines que a los serafines. No sirven solos en las supercreaciones, y las operaciones patrocinadas por sus asociados no revelados son tan numerosas como fascinantes.

\par
%\textsuperscript{(306.2)}
\textsuperscript{28:0.2} Tal como están presentados en estas narraciones, los espíritus ministrantes de los superuniversos abarcan las tres órdenes siguientes.

\par
%\textsuperscript{(306.3)}
\textsuperscript{28:0.3} 1. Los Seconafines.

\par
%\textsuperscript{(306.4)}
\textsuperscript{28:0.4} 2. Los Terciafines.

\par
%\textsuperscript{(306.5)}
\textsuperscript{28:0.5} 3. Los Omniafines.

\par
%\textsuperscript{(306.6)}
\textsuperscript{28:0.6} Puesto que las dos últimas órdenes no están tan directamente relacionadas con el programa ascendente de la progresión de los mortales, las analizaremos brevemente antes de examinar con más amplitud a los seconafines. Técnicamente, ni los terciafines ni los omniafines son espíritus ministrantes \textit{de} los superuniversos, aunque los dos sirven como ministros espirituales \textit{en} estos dominios.

\section*{1. Los Terciafines}
\par
%\textsuperscript{(306.7)}
\textsuperscript{28:1.1} Estos ángeles elevados están registrados en las sedes de los superuniversos y, a pesar de servir en las creaciones locales, residen técnicamente en estas capitales superuniversales puesto que no son nativos de los universos locales. Los terciafines son hijos del Espíritu Infinito y son personalizados en el Paraíso en grupos de mil. Estos seres celestiales con una originalidad divina y una variedad de talentos casi suprema son el regalo del Espíritu Infinito a los Hijos de Dios Creadores.

\par
%\textsuperscript{(306.8)}
\textsuperscript{28:1.2} Cuando un Hijo Miguel se separa del régimen parental del Paraíso y se prepara para salir hacia la aventura universal del espacio, el Espíritu Infinito da nacimiento a un grupo de mil espíritus compañeros de este tipo. Y estos terciafines majestuosos acompañan a ese Hijo Creador cuando emprende la aventura de organizar su universo.

\par
%\textsuperscript{(306.9)}
\textsuperscript{28:1.3} Durante los primeros tiempos de la construcción de un universo, estos mil terciafines constituyen el único estado mayor personal de un Hijo Creador. Adquieren una gran experiencia como ayudantes del Hijo durante estas épocas agitadas de ensamblaje del universo y otras manipulaciones astronómicas. Sirven al lado del Hijo Creador hasta el día de la personalización de la Radiante Estrella Matutina\footnote{\textit{Radiante Estrella Matutina}: Ap 22:16.}, el primogénito de un universo local. Inmediatamente después, los terciafines presentan su dimisión oficial y ésta es aceptada. Y con la aparición de las órdenes iniciales de vida angélica nativa, se retiran del servicio activo en el universo local y se convierten en los ministros de enlace entre el Hijo Creador al que estaban anteriormente vinculados y los Ancianos de los Días del superuniverso interesado.

\section*{2. Los Omniafines}
\par
%\textsuperscript{(307.1)}
\textsuperscript{28:2.1} Los omniafines son creados por el Espíritu Infinito en unión con los Siete Ejecutivos Supremos, y son los servidores y los mensajeros exclusivos de estos mismos Ejecutivos Supremos. Los omniafines están destinados en el gran universo y, en Orvonton, su cuerpo mantiene una sede central en las regiones septentrionales de Uversa, donde residen como colonia especial de cortesía. No están registrados en Uversa ni vinculados a nuestra administración. Tampoco están directamente relacionados con el programa ascendente de progresión de los mortales.

\par
%\textsuperscript{(307.2)}
\textsuperscript{28:2.2} Los omniafines están totalmente ocupados en la supervisión de los superuniversos en interés de una coordinación administrativa desde el punto de vista de los Siete Ejecutivos Supremos. Nuestra colonia de omniafines situada en Uversa sólo recibe instrucciones del Ejecutivo Supremo de Orvonton y sólo le presenta sus informes a él; este último se encuentra situado en la esfera ejecutiva conjunta número siete del anillo exterior de los satélites del Paraíso.

\section*{3. Los Seconafines}
\par
%\textsuperscript{(307.3)}
\textsuperscript{28:3.1} Las huestes secoráficas son engendradas por los siete Espíritus Reflectantes asignados a la sede de cada superuniverso. Existe una técnica precisa de reacción en el Paraíso asociada a la creación de estos ángeles en grupos de siete. En cada grupo de siete siempre hay un seconafín primario, tres secundarios y tres terciarios; siempre se personalizan en esta proporción exacta. Cuando se crean siete seconafines de este tipo, uno de ellos, el primario, es destinado al servicio de los Ancianos de los Días. Los tres ángeles secundarios se asocian con tres grupos de administradores que tienen su origen en el Paraíso y que operan en los supergobiernos: los Consejeros Divinos, los Perfeccionadores de la Sabiduría y los Censores Universales. Los tres ángeles terciarios son vinculados a los asociados ascendentes trinitizados de los gobernantes del superuniverso: los Mensajeros Poderosos, Los Elevados en Autoridad y Los que no tienen Nombre ni Número.

\par
%\textsuperscript{(307.4)}
\textsuperscript{28:3.2} Estos seconafines de los superuniversos son los descendientes de los Espíritus Reflectantes y, por consiguiente, la reflectividad es inherente a su naturaleza. Son reflectantemente sensibles a todas y cada una de las fases de cada criatura que tiene su origen en la Fuente-Centro Tercera y en los Hijos Creadores Paradisiacos, pero no reflejan directamente a los seres y entidades, personales u otros, que tienen su origen exclusivo en la Fuente-Centro Primera. Poseemos muchas evidencias de la realidad de los circuitos universales de inteligencia del Espíritu Infinito, pero aunque no tuviéramos otras pruebas, las acciones reflectantes de los seconafines serían totalmente suficientes para demostrar la realidad de la presencia universal de la mente infinita del Actor Conjunto.

\section*{4. Los Seconafines Primarios}
\par
%\textsuperscript{(307.5)}
\textsuperscript{28:4.1} Los seconafines primarios, asignados a los Ancianos de los Días, son unos espejos vivientes al servicio de estos gobernantes trinos. Pensad en lo que significa para la economía de un superuniverso poder volverse, por así decirlo, hacia un espejo viviente y ver en él y escuchar además las respuestas seguras de otro ser que se encuentra a mil o a cien mil años luz de distancia, y hacer todo esto de manera instantánea e infalible. Los registros son esenciales para dirigir los universos, las transmisiones son prácticas, el trabajo de los Mensajeros Solitarios y de otros mensajeros es muy útil, pero los Ancianos de los Días, desde su posición a medio camino entre los mundos habitados y el Paraíso ---entre el hombre y Dios--- pueden mirar instantáneamente hacia ambos lados, escuchar ambos lados y \textit{conocer} ambos lados.

\par
%\textsuperscript{(308.1)}
\textsuperscript{28:4.2} Esta capacidad ---para escuchar y ver, por así decirlo, todas las cosas--- sólo los Ancianos de los Días la pueden hacer perfectamente realidad en los superuniversos y solamente en sus mundos sede respectivos. E incluso allí encuentran límites: desde Uversa, esta comunicación está limitada a los mundos y universos de Orvonton, y aunque es inoperante entre los superuniversos, esta misma técnica reflectante mantiene a cada uno de ellos en estrecho contacto con el universo central y con el Paraíso. Los siete supergobiernos, aunque están individualmente separados, reflejan perfectamente así la autoridad situada por encima de ellos y comprenden totalmente las necesidades existentes por debajo de ellos, además de estar perfectamente familiarizados con ellas.

\par
%\textsuperscript{(308.2)}
\textsuperscript{28:4.3} Los seconafines primarios tienden a inclinarse, por su naturaleza inherente, hacia siete tipos de servicio, y resulta apropiado que los primeros seres consecutivos de esta orden estén dotados de tal manera que interpreten de forma inherente la mente del Espíritu a los Ancianos de los Días:

\par
%\textsuperscript{(308.3)}
\textsuperscript{28:4.4} 1. \textit{La Voz del Actor Conjunto.} En cada superuniverso, el primer seconafín primario y cada séptimo de esta orden creado posteriormente muestran un alto grado de adaptabilidad para comprender e interpretar la mente del Espíritu Infinito a los Ancianos de los Días y a sus asociados en los supergobiernos. Esto es de un gran valor en las sedes de los superuniversos porque, a diferencia de las creaciones locales con sus Ministras Divinas, la sede de un supergobierno no cuenta con una personalización especializada del Espíritu Infinito. De ahí que estas voces secoráficas sean las que más se acercan a convertirse en las representantes personales de la Fuente-Centro Tercera en esas esferas capitales. Es verdad que los siete Espíritus Reflectantes se encuentran allí, pero estas madres de las huestes secoráficas reflejan de manera menos verdadera y automática al Actor Conjunto que a los Siete Espíritus Maestros.

\par
%\textsuperscript{(308.4)}
\textsuperscript{28:4.5} 2. \textit{La Voz de los Siete Espíritus Maestros.} El segundo seconafín primario y cada séptimo creado después de él tienden a describir las naturalezas y las reacciones colectivas de los Siete Espíritus Maestros. Aunque cada Espíritu Maestro ya está representado en la capital de un superuniverso por uno de los siete Espíritus Reflectantes estacionado allí, esta representación es individual y no colectiva. Colectivamente sólo están presentes de forma reflectante; por eso los Espíritus Maestros acogen con placer los servicios de estos ángeles sumamente personales, los de la segunda serie de seconafines primarios, que son tan adecuados para representarlos ante los Ancianos de los Días.

\par
%\textsuperscript{(308.5)}
\textsuperscript{28:4.6} 3. \textit{La Voz de los Hijos Creadores.} El Espíritu Infinito debe haber tenido algo que ver con la creación o el entrenamiento de los Hijos Paradisiacos de la orden de los Migueles, porque el tercer seconafín primario y cada séptimo consecutivo posterior poseen el don extraordinario de reflejar la mente de estos Hijos Creadores. Si los Ancianos de los Días desearan conocer ---conocer realmente--- la actitud de Miguel de Nebadon acerca de alguna cuestión que se está examinando, no precisan llamarlo por las líneas del espacio; sólo necesitan llamar al Jefe de las Voces de Nebadon, el cual, a petición de los interesados, presentará al seconafín que según los registros está asociado con Miguel; y los Ancianos de los Días percibirán inmediatamente la voz del Hijo Maestro de Nebadon.

\par
%\textsuperscript{(309.1)}
\textsuperscript{28:4.7} Ninguna otra orden de filiación es <<\textit{reflectible}>> de esta manera, y ninguna otra orden de ángeles puede actuar así. No comprendemos plenamente la manera en que esto se realiza, y dudo mucho de que los Hijos Creadores mismos lo comprendan por completo. Pero sabemos con seguridad que funciona, y también sabemos que funciona infaliblemente de manera aceptable, porque en toda la historia de Uversa las voces secoráficas nunca se han equivocado en sus exposiciones.

\par
%\textsuperscript{(309.2)}
\textsuperscript{28:4.8} Estáis empezando a ver aquí una parte de la manera en que la divinidad abarca el espacio del tiempo y domina el tiempo del espacio. Estáis obteniendo aquí uno de vuestros primeros vislumbres fugaces de la técnica del ciclo de la eternidad, divergente por el momento para ayudar a los hijos del tiempo en su tarea de dominar los difíciles obstáculos del espacio. Y estos fenómenos son adicionales a la técnica universal establecida de los Espíritus Reflectantes.

\par
%\textsuperscript{(309.3)}
\textsuperscript{28:4.9} Aunque aparentemente están privados de la presencia personal de los Espíritus Maestros situados por encima, y de los Hijos Creadores situados por debajo, los Ancianos de los Días tienen a su disposición a unos seres vivientes que están sintonizados con unos mecanismos cósmicos provistos de una perfección reflectante y de una precisión última, y por medio de los cuales pueden disfrutar de la presencia reflectante de todos aquellos seres elevados de cuya presencia personal están privados. A través de estos medios, gracias a ellos y a otros que desconocéis, Dios está potencialmente presente en las sedes de los superuniversos.

\par
%\textsuperscript{(309.4)}
\textsuperscript{28:4.10} Los Ancianos de los Días deducen perfectamente la voluntad del Padre comparando la transmisión de la voz del Espíritu procedente de arriba con la transmisión de las voces de los Migueles provenientes de abajo. Así pueden estar infaliblemente seguros a la hora de suponer cuál es la voluntad del Padre respecto a los asuntos administrativos de los universos locales. Pero, para deducir la voluntad de uno de los Dioses a partir del conocimiento de los otros dos, los tres Ancianos de los Días han de actuar juntos; dos no serían capaces de conseguir la respuesta. Por esta razón, y aunque no hubiera ninguna otra, los superuniversos siempre están presididos por tres Ancianos de los Días, y no por uno solo o ni siquiera por dos.

\par
%\textsuperscript{(309.5)}
\textsuperscript{28:4.11} 4. \textit{La Voz de las Huestes Angélicas.} El cuarto seconafín primario y cada séptimo consecutivo resultan ser unos ángeles particularmente sensibles a los sentimientos de todas las órdenes de ángeles, incluyendo a los supernafines que están por encima y a los serafines que están por debajo. Así, la actitud de cualquier ángel dirigente o supervisor se encuentra inmediatamente disponible para ser examinada en cualquier consejo de los Ancianos de los Días. Nunca pasa un día en vuestro mundo sin que el jefe de los serafines de Urantia tenga conciencia del fenómeno de una transferencia reflectante, de que se recurre a él desde Uversa por alguna razón; pero a menos que un Mensajero Solitario lo prevenga, permanece totalmente ignorante de lo que se busca y de cómo se consigue. Estos espíritus ministrantes del tiempo proporcionan constantemente este tipo de testimonio inconsciente y, por tanto, ciertamente imparcial sobre la serie interminable de cuestiones que llaman la atención y requieren el consejo de los Ancianos de los Días y de sus asociados.

\par
%\textsuperscript{(309.6)}
\textsuperscript{28:4.12} 5. \textit{Los Receptores de las Transmisiones.} Existe una clase especial de mensajes a transmitir que sólo los reciben estos seconafines primarios. Aunque ellos no son los transmisores regulares de Uversa, trabajan en unión con los ángeles de las voces reflectantes con el objeto de sincronizar la visión reflectante de los Ancianos de los Días con ciertos mensajes concretos que llegan por los circuitos establecidos de la comunicación universal. Los receptores de las transmisiones son los quintos consecutivos, el quinto seconafín primario en ser creado y cada séptimo creado después de él.

\par
%\textsuperscript{(310.1)}
\textsuperscript{28:4.13} 6. \textit{Las Personalidades de Transporte.} Son los seconafines que transportan a los peregrinos del tiempo desde los mundos sede de los superuniversos hasta el círculo exterior de Havona. Son el cuerpo de transporte de los superuniversos, y funcionan hacia el interior hasta el Paraíso y hacia el exterior hasta los mundos de sus sectores respectivos. Este cuerpo está compuesto por el sexto seconafín primario y por cada séptimo creado posteriormente.

\par
%\textsuperscript{(310.2)}
\textsuperscript{28:4.14} 7. \textit{El Cuerpo de Reserva.} Un grupo muy amplio de seconafines, los séptimos consecutivos primarios, se mantienen de reserva para las funciones no clasificadas y las misiones de urgencia de los reinos. Como no están muy especializados, pueden ejercer su actividad bastante bien en cualquiera de las capacidades de sus diversos asociados, pero este trabajo especializado sólo lo emprenden en caso de urgencia. Sus tareas habituales consisten en la realización de aquellos deberes generalizados de un superuniverso que no pertenecen al campo de acción de los ángeles que tienen una misión específica.

\section*{5. Los Seconafines Secundarios}
\par
%\textsuperscript{(310.3)}
\textsuperscript{28:5.1} Los seconafines de la orden secundaria no son menos reflectantes que sus compañeros primarios. En el caso de los seconafines, la clasificación en primarios, secundarios y terciarios no indica una categoría o una función diferenciales; simplemente denota unas órdenes de procedimiento. Los tres grupos muestran en sus actividades unas cualidades idénticas.

\par
%\textsuperscript{(310.4)}
\textsuperscript{28:5.2} Los siete tipos reflectantes de seconafines secundarios están destinados al servicio de los asociados coordinados de origen trinitario de los Ancianos de los Días de la manera siguiente:

\par
%\textsuperscript{(310.5)}
\textsuperscript{28:5.3} A los Perfeccionadores de la Sabiduría ---las Voces de la Sabiduría, las Almas de la Filosofía y las Uniones de las Almas.

\par
%\textsuperscript{(310.6)}
\textsuperscript{28:5.4} A los Consejeros Divinos ---los Corazones del Consejo, las Alegrías de la Existencia y las Satisfacciones del Servicio.

\par
%\textsuperscript{(310.7)}
\textsuperscript{28:5.5} A los Censores Universales ---los Discernidores de Espíritus.

\par
%\textsuperscript{(310.8)}
\textsuperscript{28:5.6} Al igual que la orden primaria, este grupo es creado en serie; es decir, el primogénito fue una Voz de la Sabiduría, y el séptimo creado después fue similar, y lo mismo sucede con los otros seis tipos de estos ángeles reflectantes.

\par
%\textsuperscript{(310.9)}
\textsuperscript{28:5.7} 1. \textit{La Voz de la Sabiduría.} Algunos de estos seconafines están en conexión perpetua con las bibliotecas vivientes del Paraíso, con los custodios del conocimiento pertenecientes a los supernafines primarios. En su servicio reflectante especializado, las Voces de la Sabiduría son concentraciones y focalizaciones vivientes, actualizadas, completas y totalmente fiables, de la sabiduría coordinada del universo de universos. Con el volumen casi infinito de información que circula por los circuitos maestros de los superuniversos, estos seres magníficos son tan reflectantes y selectivos, tan sensibles, que son capaces de separar y de recibir la esencia de la sabiduría y de transmitir infaliblemente estas joyas de la acción mental a sus superiores, los Perfeccionadores de la Sabiduría. Y ejercen su actividad de tal manera que los Perfeccionadores de la Sabiduría no solamente escuchan las expresiones reales y originales de esta sabiduría, sino que ven también reflectantemente a los seres mismos, de origen humilde o elevado, que la han expresado.

\par
%\textsuperscript{(310.10)}
\textsuperscript{28:5.8} Está escrito: <<\textit{Si un hombre carece de sabiduría, que la pida}>>\footnote{\textit{Si carece de sabiduría, que la pida}: Stg 1:5.}. En Uversa, cuando es necesario llegar a unas decisiones de sabiduría en las situaciones confusas de los asuntos complejos del gobierno del superuniverso, cuando han de aparecer tanto la sabiduría de la perfección como la sabiduría de la viabilidad, entonces los Perfeccionadores de la Sabiduría convocan a un grupo de Voces de la Sabiduría y, con la habilidad consumada de su orden, sintonizan y orientan de tal manera a estos receptores vivientes de la sabiduría que está en las mentes y que circula en el universo de universos, que se produce enseguida, desde estas voces secoráficas, una oleada de sabiduría de la divinidad procedente del universo situado por encima y un torrente de la sabiduría del sentido práctico proveniente de las mentes superiores de los universos situados por debajo.

\par
%\textsuperscript{(311.1)}
\textsuperscript{28:5.9} Si surge una confusión a la hora de armonizar estas dos versiones de la sabiduría, se recurre inmediatamente a los Consejeros Divinos, los cuales deciden enseguida la combinación apropiada de los procedimientos. Si existe alguna duda sobre la autenticidad de alguna cosa procedente de unos reinos donde ha prevalecido la rebelión, se recurre a los Censores, los cuales, con sus Discernidores de Espíritus, son capaces de decidir inmediatamente <<\textit{qué clase de espíritu}>>\footnote{\textit{Qué clase de espíritu}: Lc 9:55.} impulsó al asesor. Así es como la sabiduría de todas las épocas y el intelecto del momento están siempre presentes para los Ancianos de los Días como un libro abierto ante sus miradas benefactoras.

\par
%\textsuperscript{(311.2)}
\textsuperscript{28:5.10} Apenas podéis comprender lo que todo esto significa para aquellos que son los responsables de la dirección de los gobiernos superuniversales. La inmensidad y la amplitud de estas operaciones sobrepasan por completo la concepción finita. Cuando os encontréis, como yo lo he hecho repetidas veces, en las cámaras receptoras especiales del templo de la sabiduría de Uversa y veáis funcionar todo esto de manera efectiva, os sentiréis impulsados a la adoración por la perfección de la complejidad y por la seguridad del funcionamiento de las comunicaciones interplanetarias de los universos. Rendiréis homenaje a la sabiduría y a la bondad divinas de los Dioses, que hacen planes y los ejecutan con esta técnica tan magnífica. Y estas cosas suceden realmente tal como las he descrito.

\par
%\textsuperscript{(311.3)}
\textsuperscript{28:5.11} 2. \textit{El Alma de la Filosofía.} Estos educadores maravillosos también están vinculados a los Perfeccionadores de la Sabiduría y, cuando no están orientados de otra manera, permanecen en sincronismo focal con los maestros de la filosofía del Paraíso. Imaginad que os acercáis a un inmenso espejo viviente, por así decirlo, pero que en lugar de contemplar la imagen de vuestro yo finito y material, percibís un reflejo de la sabiduría de la divinidad y de la filosofía del Paraíso. Si llega a ser deseable <<\textit{encarnar}>> esta filosofía de la perfección, diluirla de tal manera que se vuelva aplicable y asimilable en la práctica por los pueblos humildes de los mundos inferiores, estos espejos vivientes sólo tienen que volver sus rostros hacia abajo para reflejar los criterios y las necesidades de otro mundo o de otro universo.

\par
%\textsuperscript{(311.4)}
\textsuperscript{28:5.12} Por medio de estas mismas técnicas, los Perfeccionadores de la Sabiduría adaptan las decisiones y las recomendaciones a las necesidades reales y al estado efectivo de los pueblos y de los mundos sometidos a estudio, y siempre actúan de común acuerdo con los Consejeros Divinos y los Censores Universales. Pero la plenitud sublime de estas operaciones sobrepasa incluso mi capacidad de comprensión.

\par
%\textsuperscript{(311.5)}
\textsuperscript{28:5.13} 3. \textit{La Unión de las Almas.} Estos reflectores de los ideales y del estado de las relaciones éticas completan el personal trino vinculado a los Perfeccionadores de la Sabiduría. De todos los problemas que surgen en el universo y que requieren el ejercicio de la sabiduría consumada de la experiencia y de la adaptabilidad, ninguno es más importante que aquellos que surgen en las relaciones y en las asociaciones de los seres inteligentes. Ya sea en las asociaciones humanas del comercio y los negocios, de la amistad y el matrimonio, o en los contactos entre las huestes angélicas, continúan apareciendo pequeñas fricciones, malentendidos menores demasiado banales como para atraer siquiera la atención de los conciliadores, pero lo suficientemente irritantes y perturbadores como para estropear el tranquilo funcionamiento del universo si se les permite multiplicarse y continuar. Por consiguiente, los Perfeccionadores de la Sabiduría ponen a la disposición de todo un superuniverso la sabia experiencia de su orden como <<\textit{el óleo de la reconciliación}>>\footnote{\textit{Óleo de la reconciliación}: Ex 30:23-30.}. En todo este trabajo, estos sabios de los superuniversos son hábilmente secundados por sus asociados reflectantes, las Uniones de las Almas, que hacen asequible la información actual relacionada con el estado del universo y describen al mismo tiempo el ideal paradisiaco adecuado para ajustar mejor estos complicados problemas. Cuando no están orientados específicamente hacia otro lugar, estos seconafines permanecen en contacto reflectante con los intérpretes de la ética que se encuentran en el Paraíso.

\par
%\textsuperscript{(312.1)}
\textsuperscript{28:5.14} Éstos son los ángeles que fomentan y promueven el trabajo en equipo en todo Orvonton. Una de las lecciones más importantes que tenéis que aprender durante vuestra carrera mortal es la del \textit{trabajo en equipo.} Las esferas de perfección están tripuladas por aquellos que han dominado este arte de trabajar con otros seres. En el universo hay pocas obligaciones para el servidor solitario. Cuanto más os eleváis, más solos os sentís cuando temporalmente no estáis asociados con vuestros compañeros.

\par
%\textsuperscript{(312.2)}
\textsuperscript{28:5.15} 4. \textit{El Corazón del Consejo.} Éste es el primer grupo de esos genios reflectantes que están colocados bajo la supervisión de los Consejeros Divinos. Los seconafines de este tipo están en posesión de los hechos del espacio, pues son selectivos para este tipo de datos en los circuitos del tiempo. Reflejan de manera especial a los coordinadores superáficos de la información, pero también reflejan de forma selectiva el consejo de todos los seres, ya sean de rango superior o inferior. Cada vez que se recurre a los Consejeros Divinos para recibir asesoramiento o tomar decisiones importantes, éstos solicitan de inmediato un conjunto de Corazones del Consejo, y enseguida se transmite una decisión que incorpora efectivamente la sabiduría y el asesoramiento coordinados de las mentes más competentes de todo el superuniverso, todo lo cual ha sido censurado y revisado a la luz del consejo de las mentes superiores de Havona e incluso del Paraíso.

\par
%\textsuperscript{(312.3)}
\textsuperscript{28:5.16} 5. \textit{La Alegría de la Existencia.} Estos seres están por naturaleza reflectantemente sintonizados con los supervisores superáficos de la armonía situados por encima, y con ciertos serafines situados por debajo, pero es difícil explicar qué hacen exactamente los miembros de este grupo interesante. Sus actividades principales están dirigidas a promover reacciones de alegría entre las diversas órdenes de las huestes angélicas y de las criaturas volitivas inferiores. Los Consejeros Divinos, a los cuales están vinculados, raras veces los utilizan para descubrir específicamente la alegría. De una manera más general, y en colaboración con los directores de la reversión, ejercen su actividad como cámaras de análisis de la alegría, tratando de aumentar las reacciones de placer de los reinos e intentando mejorar al mismo tiempo el gusto por el humor, desarrollar un superhumor entre los mortales y los ángeles. Se esfuerzan por demostrar que hay una alegría inherente en el hecho de tener una existencia con libre albedrío, independientemente de todas las influencias externas; y tienen razón, aunque encuentran grandes dificultades para inculcar esta verdad en la mente de los hombres primitivos. Las personalidades espirituales superiores y los ángeles responden con más rapidez a estos esfuerzos educativos.

\par
%\textsuperscript{(312.4)}
\textsuperscript{28:5.17} 6. \textit{La Satisfacción del Servicio.} Estos ángeles reflejan muy bien la actitud de los directores de la conducta situados en el Paraíso y, actuando en gran medida como lo hacen las Alegrías de la Existencia, se esfuerzan por realzar el valor del servicio y por aumentar las satisfacciones que se derivan del mismo. Han contribuido mucho a iluminar las recompensas aplazadas inherentes al servicio desinteresado, al servicio para la expansión del reino de la verdad.

\par
%\textsuperscript{(312.5)}
\textsuperscript{28:5.18} Los Consejeros Divinos, a quienes esta orden está vinculada, los utilizan para reflejar de un mundo a otro los beneficios que se pueden obtener del servicio espiritual. Y utilizando las obras de los mejores para inspirar y animar a los mediocres, estos seconafines contribuyen enormemente a la calidad del servicio dedicado en los superuniversos. El espíritu competitivo fraternal se utiliza con eficacia, haciendo circular en un mundo la información sobre lo que se hace en los otros mundos, particularmente en los mejores. Así se promueve una rivalidad refrescante y sana, incluso entre las huestes seráficas.

\par
%\textsuperscript{(313.1)}
\textsuperscript{28:5.19} 7. \textit{Los Discernidores de Espíritus.} Existe una conexión especial entre los consejeros y los asesores del segundo círculo de Havona y estos ángeles reflectantes. Son los únicos seconafines vinculados a los Censores Universales, pero son probablemente los más extraordinariamente especializados de todos sus compañeros. Sin tener en cuenta la fuente o el canal de información, por muy escasas que sean las pruebas que se tengan a mano, cuando son sometidas a su examen reflectante, estos discernidores nos informarán enseguida sobre el verdadero motivo, el propósito real y la auténtica naturaleza de su origen. Me maravillo con el magnífico trabajo de estos ángeles, que reflejan de manera tan infalible el verdadero carácter moral y espiritual de cualquier individuo sometido a una exposición focal.

\par
%\textsuperscript{(313.2)}
\textsuperscript{28:5.20} Los Discernidores de Espíritus efectúan estos complicados servicios en virtud de su <<\textit{discernimiento espiritual}>> inherente, si es que puedo utilizar estas palabras en un esfuerzo por transmitir a la mente humana la idea de que estos ángeles reflectantes actúan así de manera intuitiva, inherente e infalible. Cuando los Censores Universales perciben estas presentaciones, se encuentran frente a frente con el alma desnuda del individuo reflejado; la certidumbre y la perfección mismas de este retrato explica en parte por qué los Censores pueden actuar siempre con tanta justicia como jueces equitativos. Los discernidores acompañan siempre a los Censores en todas sus misiones fuera de Uversa, y son exactamente igual de eficaces en los universos que en su sede central de Uversa.

\par
%\textsuperscript{(313.3)}
\textsuperscript{28:5.21} Os aseguro que todas estas operaciones del mundo espiritual son reales, que tienen lugar de acuerdo con las costumbres establecidas y en armonía con las leyes inmutables de los dominios universales. Los seres de cada orden recién creada, inmediatamente después de recibir el soplo de vida, son reflejados instantáneamente en las alturas; un retrato viviente de la naturaleza y del potencial de la criatura se transmite a la sede del superuniverso. Y así, por medio de los discernidores, los Censores conocen plenamente <<\textit{qué clase de espíritu}>>\footnote{\textit{Qué clase de espíritu}: Lc 9:55.} ha nacido exactamente en los mundos del espacio.

\par
%\textsuperscript{(313.4)}
\textsuperscript{28:5.22} Esto mismo sucede con el hombre mortal: el Espíritu Madre de Salvington os conoce plenamente, porque el Espíritu Santo que está en vuestro mundo <<\textit{sondea todas las cosas}>>\footnote{\textit{El Espíritu sondea todas las cosas}: 1 Co 2:10.}, y todo lo que el Espíritu divino sabe sobre vosotros está inmediatamente disponible cada vez que los discernidores secoráficos reflejan con el Espíritu aquello que el Espíritu conoce de vosotros. Debemos mencionar sin embargo que el conocimiento y los planes de los fragmentos del Padre no son reflejables. Los discernidores pueden reflejar, y reflejan, la presencia de los Ajustadores (y los Censores los declaran divinos), pero no pueden descifrar el contenido de la mente de los Monitores de Misterio.

\section*{6. Los Seconafines Terciarios}
\par
%\textsuperscript{(313.5)}
\textsuperscript{28:6.1} De la misma manera que sus compañeros, estos ángeles son creados en serie y en siete tipos reflectantes, pero estos tipos no son destinados individualmente a los distintos servicios de los administradores de los superuniversos. Todos los seconafines terciarios están asignados colectivamente a los Hijos de la Consecución Trinitizados, y estos hijos ascendentes los emplean de manera intercambiable; es decir, los Mensajeros Poderosos pueden utilizar, y utilizan, cualquiera de los tipos terciarios, y esto mismo hacen sus coordinados, Los Elevados en Autoridad y Los que no tienen Nombre ni Número. Estos siete tipos de seconafines terciarios son:

\par
%\textsuperscript{(314.1)}
\textsuperscript{28:6.2} 1. \textit{La Relevancia de los Orígenes.} Los Hijos Trinitizados ascendentes del gobierno de un superuniverso tienen a su cargo la responsabilidad de tratar todos los asuntos derivados del origen de cualquier individuo, raza o mundo; y la importancia del origen es la cuestión primordial en todos nuestros planes para el avance cósmico de las criaturas vivientes del reino. Todas las relaciones y la aplicación de la ética surgen de los hechos fundamentales del origen. El origen es la base de la reacción de los Dioses con respecto a las relaciones. El Actor Conjunto siempre <<\textit{toma nota del hombre, de la manera en que ha nacido}>>.

\par
%\textsuperscript{(314.2)}
\textsuperscript{28:6.3} En el caso de los seres descendentes superiores, el origen es simplemente un hecho que ha de ser comprobado; pero en el caso de los seres ascendentes, incluyendo a las órdenes inferiores de ángeles, la naturaleza y las circunstancias del origen no siempre están tan claras, aunque sean igualmente de una importancia vital en casi cada giro de los asuntos universales ---de ahí el valor de tener a nuestra disposición a una serie de seconafines reflectantes que pueden mostrar instantáneamente todo lo que se necesita en relación con la génesis de cualquier ser que se encuentre o bien en el universo central o en todo el reino de un superuniverso.

\par
%\textsuperscript{(314.3)}
\textsuperscript{28:6.4} La Relevancia de los Orígenes son las genealogías vivientes, que se pueden consultar con rapidez, de la inmensa multitud de seres ---hombres, ángeles y otros--- que habitan los siete superuniversos. Siempre están preparados para proporcionar a sus superiores una estimación actualizada, completa y digna de confianza, de los factores ancestrales y del estado real actual de cualquier individuo en cualquier mundo de sus respectivos superuniversos; y su cómputo de los hechos conocidos siempre está al minuto.

\par
%\textsuperscript{(314.4)}
\textsuperscript{28:6.5} 2. \textit{La Memoria de la Misericordia.} Son los registros vivientes reales, plenos y completos, de la misericordia que se ha concedido a los individuos y a las razas mediante el tierno ministerio de los intermediarios del Espíritu Infinito en su misión de adaptar la justicia de la rectitud al estado de los reinos, tal como éste se revela en las descripciones de la Relevancia de los Orígenes. La Memoria de la Misericordia revela la deuda moral de los hijos de la misericordia ---su pasivo espiritual--- que debe asentarse en la parte contraria de su activo, el cual contiene la provisión de salvación establecida por los Hijos de Dios. Al revelar la misericordia preexistente del Padre, los Hijos de Dios establecen el crédito necesario para asegurar la supervivencia de todos. Y luego, de acuerdo con los descubrimientos de la Relevancia de los Orígenes, se establece un crédito de misericordia para la supervivencia de cada criatura racional, un crédito de proporciones generosas y de una gracia suficiente como para asegurar la supervivencia de toda alma que desee realmente la ciudadanía divina.

\par
%\textsuperscript{(314.5)}
\textsuperscript{28:6.6} La Memoria de la Misericordia es un saldo viviente a prueba, un extracto actualizado de vuestra cuenta con las fuerzas sobrenaturales de los reinos. Son los registros vivientes del ministerio de la misericordia que se leen durante el testimonio en los tribunales de Uversa cuando se juzga el derecho de cada individuo a la vida sin fin, cuando <<\textit{se levantan los tronos y los Ancianos de los Días se sientan. Las transmisiones de Uversa funcionan y salen de delante de ellos; miles y miles de seres les aportan su ministerio, y diez mil veces diez mil permanecen delante de ellos. El juicio está preparado, y los libros se abren}>>\footnote{\textit{El juicio de los Ancianos de los Días}: Dn 7:9-10.}. Y los libros que se abren en una ocasión tan importante son los registros vivientes de los seconafines terciarios de los superuniversos. Los registros oficiales están en los archivos para corroborar el testimonio de las Memorias de la Misericordia si es necesario.

\par
%\textsuperscript{(314.6)}
\textsuperscript{28:6.7} La Memoria de la Misericordia debe mostrar que el crédito de salvación establecido por los Hijos de Dios ha sido plena y fielmente pagado mediante el ministerio afectuoso de las pacientes personalidades de la Fuente-Centro Tercera. Pero cuando se agota la misericordia, cuando la <<\textit{memoria}>> de la misma atestigua su agotamiento, entonces la justicia prevalece y la rectitud decreta. Porque la misericordia no ha de ser impuesta a aquellos que la desprecian; la misericordia no es un regalo para ser pisoteado por los rebeldes persistentes del tiempo. Sin embargo, aunque la misericordia sea así inapreciable y afectuosamente otorgada, vuestro crédito individual sobrepasa siempre con exceso vuestra capacidad para agotar la reserva, si vuestra intención es sincera y sois honrados de corazón.

\par
%\textsuperscript{(315.1)}
\textsuperscript{28:6.8} Los reflectores de la misericordia, con sus asociados terciarios, se ocupan de numerosos ministerios superuniversales, incluyendo la formación de las criaturas ascendentes. Entre otras muchas cosas, la Relevancia de los Orígenes enseña a estos ascendentes la manera de aplicar la ética espiritual y, después de esta formación, las Memorias de la Misericordia les enseñan cómo ser verdaderamente misericordiosos. Aunque las técnicas espirituales del ministerio de la misericordia sobrepasan vuestros conceptos, deberíais comprender ahora mismo que la misericordia es una cualidad del crecimiento. Deberíais daros cuenta de que existe una gran recompensa de satisfacción personal en ser primero justo, a continuación equitativo, luego paciente y luego bondadoso. Y luego, sobre esta base, si lo elegís y lo tenéis en vuestro corazón, podéis dar el siguiente paso y mostrar realmente misericordia; pero no podéis manifestar la misericordia en sí misma y por sí misma. Hay que atravesar estas etapas; de otra manera no puede haber auténtica misericordia. Puede haber patrocinio, condescendencia o caridad ---e incluso compasión--- pero no misericordia. La verdadera misericordia sólo llega como el hermoso punto culminante de estos complementos anteriores de la comprensión colectiva, la apreciación mutua, el compañerismo fraternal, la comunión espiritual y la armonía divina.

\par
%\textsuperscript{(315.2)}
\textsuperscript{28:6.9} 3. \textit{La Importancia del Tiempo.} El tiempo es la única dotación universal común para todas las criaturas volitivas; es el <<\textit{talento}>>\footnote{\textit{El tiempo es un ``talento'' de todos}: Mt 25:14-28; Lc 19:12-21.} que ha sido confiado a todos los seres inteligentes. Todos tenéis tiempo para asegurar vuestra supervivencia; el tiempo sólo se desperdicia fatalmente cuando se pierde en la negligencia, cuando no lográis utilizarlo de tal manera que asegure la supervivencia de vuestra alma. El fracaso en sacarle el mayor partido posible al tiempo de uno mismo no conlleva consecuencias fatales; simplemente retrasa al peregrino del tiempo en su viaje de ascensión. Si se ha logrado la supervivencia, todas las demás pérdidas se pueden recuperar.

\par
%\textsuperscript{(315.3)}
\textsuperscript{28:6.10} En la asignación de las obligaciones, el consejo de las Importancias del Tiempo es inapreciable. El tiempo es un factor vital en todo lo que se encuentra a este lado de Havona y del Paraíso. En el juicio final ante los Ancianos de los Días, el tiempo es un elemento a justificar. Las Importancias del Tiempo deben siempre aportar su testimonio para demostrar que cada acusado ha tenido tiempo suficiente para tomar sus decisiones, para llegar a una elección.

\par
%\textsuperscript{(315.4)}
\textsuperscript{28:6.11} Estos evaluadores del tiempo son también el secreto de la profecía; describen el elemento de tiempo que será necesario para realizar cualquier empresa, y son tan fiables como indicadores como lo son los frandalanks y los cronoldeks que pertenecen a otras órdenes vivientes. Los Dioses prevén, y por lo tanto conocen de antemano; pero las autoridades ascendentes de los universos del tiempo deben consultar a las Importancias del Tiempo para poder pronosticar los acontecimientos del futuro.

\par
%\textsuperscript{(315.5)}
\textsuperscript{28:6.12} A estos seres los encontraréis por primera vez en los mundos de las mansiones; y allí os enseñarán la utilización ventajosa de aquello que llamáis <<\textit{tiempo}>>, tanto en su empleo positivo, el trabajo, como en su utilización negativa, el descanso. Las dos maneras de utilizar el tiempo son importantes.

\par
%\textsuperscript{(315.6)}
\textsuperscript{28:6.13} 4. \textit{La Solemnidad de la Confianza.} La confianza es la prueba crucial de las criaturas volitivas. La honradez es la verdadera medida del dominio de sí mismo, del carácter. Estos seconafines cumplen una doble finalidad en la economía de los superuniversos: describen a todas las criaturas volitivas el sentido de la obligación, el carácter sagrado y la solemnidad de la confianza. Al mismo tiempo reflejan infaliblemente para las autoridades gobernantes la honradez exacta de cualquier candidato a la confianza o a la fiabilidad.

\par
%\textsuperscript{(316.1)}
\textsuperscript{28:6.14} En Urantia intentáis de manera grotesca adivinar el carácter y estimar las capacidades específicas; pero en Uversa hacemos estas cosas realmente a la perfección. Estos seconafines pesan la honradez en las balanzas vivientes que evalúan infaliblemente el carácter, y una vez que os han mirado, sólo tenemos que mirarlos a ellos para conocer las limitaciones de vuestra capacidad para cumplir con las responsabilidades, llevar a cabo los deberes y realizar misiones. Vuestro activo de honradez está expuesto claramente al lado de vuestro pasivo de faltas o de traiciones posibles.

\par
%\textsuperscript{(316.2)}
\textsuperscript{28:6.15} Vuestros superiores tienen el proyecto de haceros avanzar mediante obligaciones crecientes y con la rapidez con que vuestro carácter se desarrolle lo suficiente como para llevar con elegancia estas responsabilidades adicionales, pero sobrecargar al individuo sólo expone al desastre y asegura la decepción. Y el error de colocar prematuramente una responsabilidad sobre un hombre o un ángel se puede evitar utilizando el ministerio de estos estimadores infalibles de la confianza que pueden merecer los individuos del tiempo y del espacio. Estos seconafines acompañan siempre a Los Elevados en Autoridad, y estos ejecutivos nunca efectúan los nombramientos hasta que sus candidatos no han sido pesados en las balanzas secoráficas y declarados <<\textit{no deficientes}>>.

\par
%\textsuperscript{(316.3)}
\textsuperscript{28:6.16} 5. \textit{La Santidad del Servicio.} El privilegio del servicio sigue directamente al descubrimiento de la honradez. Nada puede interponerse entre vosotros y la oportunidad de efectuar un servicio creciente, salvo vuestra falta de honradez, vuestra falta de capacidad para apreciar la solemnidad de la confianza.

\par
%\textsuperscript{(316.4)}
\textsuperscript{28:6.17} El servicio ---el servicio resuelto, no la esclavitud--- produce la satisfacción más elevada y expresa la dignidad más divina. El servicio ---más servicio, servicio creciente, servicio difícil, servicio aventurero, y al final el servicio divino y perfecto--- es la meta del tiempo y el destino del espacio. Pero los ciclos temporales de esparcimiento siempre alternarán con los ciclos de progreso en el servicio. Y después del servicio del tiempo sigue el superservicio de la eternidad. Durante el esparcimiento temporal deberíais imaginar el trabajo de la eternidad, al igual que durante el servicio de la eternidad recordaréis el esparcimiento del tiempo.

\par
%\textsuperscript{(316.5)}
\textsuperscript{28:6.18} La economía universal está basada en el consumo y la producción; durante toda la carrera eterna nunca encontraréis la monotonía de la inacción o el estancamiento de la personalidad. El progreso es posible gracias al movimiento inherente, el avance surge de la capacidad divina para la acción, y la consecución es hija de la aventura imaginativa. Pero en esta capacidad para alcanzar los objetivos se encuentra de manera inherente la responsabilidad de la ética, la necesidad de reconocer que el mundo y el universo están llenos de una multitud de tipos diferentes de seres. Toda esta magnífica creación, \textit{incluido tú mismo,} no ha sido hecha sólo para ti. Este universo no es egocéntrico. Los Dioses han decretado: <<\textit{Es más noble dar que recibir}>>\footnote{\textit{Es más noble dar que recibir}: Hch 20:35.}, y vuestro Hijo Maestro dijo: <<\textit{Aquel que quiera ser el más grande entre vosotros, que sea el servidor de todos}>>\footnote{\textit{El más grande sea el más servicial}: Mt 20:26-27; 23:11-12; Mc 9:35; 10:43-44; Lc 22:26.}.

\par
%\textsuperscript{(316.6)}
\textsuperscript{28:6.19} La naturaleza real de cualquier servicio, ya sea efectuado por un hombre o por un ángel, se revela plenamente en el rostro de estos indicadores secoráficos del servicio, las Santidades del Servicio. El análisis completo de los motivos verdaderos y ocultos queda expuesto con toda claridad. Estos ángeles son en verdad los lectores de la mente, los indagadores del corazón y los reveladores del alma en el universo. Los mortales pueden emplear palabras para ocultar sus pensamientos, pero estos elevados seconafines ponen al descubierto los motivos profundos del corazón humano y de la mente angélica.

\par
%\textsuperscript{(317.1)}
\textsuperscript{28:6.20} 6 y 7. \textit{El Secreto de la Grandeza y el Alma de la Bondad.} Una vez que los peregrinos ascendentes se han dado cuenta de la importancia del tiempo, el camino está preparado para reconocer la solemnidad de la confianza y para apreciar la santidad del servicio. Aunque éstos son los elementos morales de la grandeza, también hay secretos de la grandeza. Cuando se aplican las pruebas espirituales de la grandeza, los elementos morales no se descuidan, pero la verdadera \textit{medida} de la grandeza planetaria es la calidad de la generosidad revelada en el trabajo desinteresado por el bienestar de los propios compañeros terrenales, en particular por los seres dignos que están necesitados y en un apuro. Y la \textit{manifestación} de la grandeza en un mundo como Urantia es la demostración del control de sí mismo. El gran hombre no es aquel que <<\textit{conquista una ciudad}>>\footnote{\textit{Conquista una ciudad}: Pr 16:32.} o <<\textit{derriba una nación}>>, sino más bien <<\textit{aquel que domina su propia lengua}>>\footnote{\textit{Domina su propia lengua}: Pr 21:23; Stg 3:3-8.}.

\par
%\textsuperscript{(317.2)}
\textsuperscript{28:6.21} Grandeza es sinónimo de divinidad. Dios es supremamente grande\footnote{\textit{Dios es grande}: Job 36:26.} y bueno\footnote{\textit{Dios es bueno}: Sal 73:1.}. \textit{La grandeza y la bondad no se pueden simplemente separar.} Están unidas para siempre en Dios. Esta verdad está ilustrada de manera literal e impresionante en la interdependencia reflectante del Secreto de la Grandeza y del Alma de la Bondad, ya que ninguno de los dos puede actuar sin el otro. Para reflejar otras cualidades de la divinidad, los seconafines de los superuniversos pueden actuar solos, y así lo hacen, pero las estimaciones reflectantes de la grandeza y de la bondad parecen ser inseparables. Por lo tanto, en cualquier mundo, en cualquier universo, estos reflectores de la grandeza y de la bondad deben trabajar juntos, mostrando siempre un informe doble y mutuamente dependiente de cada ser sobre el cual se focalizan. La grandeza no se puede estimar sin conocer su contenido de bondad, mientras que la bondad no se puede describir sin mostrar su grandeza inherente y divina.

\par
%\textsuperscript{(317.3)}
\textsuperscript{28:6.22} La estimación de la grandeza varía de una esfera a otra. Ser grande es ser semejante a Dios. Y puesto que la calidad de la grandeza está totalmente determinada por el contenido de bondad, de ello se deduce que, incluso en vuestro estado humano actual, si a través de la gracia podéis volveros buenos, debido a ello os estáis volviendo grandes. Cuanto más contempléis constantemente y más persigáis insistentemente los conceptos de la bondad divina, más ciertamente creceréis en grandeza, en la verdadera magnitud de un auténtico carácter de supervivencia.

\section*{7. El ministerio de los Seconafines}
\par
%\textsuperscript{(317.4)}
\textsuperscript{28:7.1} Los seconafines tienen su origen y su sede central en las capitales de los superuniversos, pero con sus compañeros de enlace recorren desde las orillas del Paraíso hasta los mundos evolutivos del espacio. Sirven como apreciados asistentes de los miembros de las asambleas deliberantes de los supergobiernos y son de una gran ayuda para las colonias de cortesía de Uversa: los estudiantes de estrellas, los turistas milenarios, los observadores celestiales y una multitud de otras personalidades, incluyendo a los seres ascendentes que esperan ser transportados hacia Havona. Los Ancianos de los Días disfrutan nombrando a ciertos seconafines primarios para que ayuden a las criaturas ascendentes domiciliadas en los cuatrocientos noventa mundos de estudio que rodean a Uversa, y muchos miembros de las órdenes secundaria y terciaria sirven también aquí como instructores. Estos satélites de Uversa son las escuelas finales de los universos del tiempo, y ofrecen el curso de preparación para la universidad compuesta por los siete circuitos de Havona.

\par
%\textsuperscript{(317.5)}
\textsuperscript{28:7.2} De las tres órdenes de seconafines, el grupo terciario, vinculado a las autoridades ascendentes, es el que aporta más ampliamente su ministerio a las criaturas ascendentes del tiempo. Los encontraréis de vez en cuando poco después de vuestra partida de Urantia, aunque no utilizaréis abundantemente sus servicios hasta que no alcancéis los mundos de estancia de Orvonton. Disfrutaréis de su compañía cuando los conozcáis plenamente durante vuestra estancia en los mundos escuela de Uversa.

\par
%\textsuperscript{(318.1)}
\textsuperscript{28:7.3} Estos seconafines terciarios son los ahorradores de tiempo, los acortadores del espacio, los detectores de errores, los instructores fieles y los postes indicadores perpetuos ---los signos vivientes de la seguridad divina--- colocados por misericordia en las encrucijadas del tiempo para guiar allí los pasos de los peregrinos ansiosos en los momentos de gran perplejidad y de incertidumbre espiritual. Mucho antes de llegar a las puertas de la perfección empezaréis a tener acceso a los instrumentos de la divinidad y a poneros en contacto con las técnicas de la Deidad. Desde el momento en que lleguéis al mundo inicial de las mansiones hasta que cerréis los ojos en el sueño de Havona como preparación para vuestro transporte hacia el Paraíso, utilizaréis cada vez más la ayuda de urgencia de estos seres maravillosos, que reflejan de manera tan plena y abundante el conocimiento seguro y la sabiduría cierta de aquellos peregrinos fiables y dignos de confianza que os han precedido en el largo viaje hacia los pórticos de la perfección.

\par
%\textsuperscript{(318.2)}
\textsuperscript{28:7.4} Estamos privados del pleno privilegio de utilizar en Urantia a estos ángeles de la orden reflectante. Visitan con frecuencia vuestro mundo, acompañando a las personalidades destinadas aquí, pero aquí no pueden actuar libremente. Esta esfera se encuentra todavía en una cuarentena espiritual parcial, y algunos circuitos esenciales para su servicio no funcionan aquí en la actualidad. Cuando vuestro mundo sea restablecido una vez más en los circuitos reflectantes correspondientes, una gran parte del trabajo de las comunicaciones interplanetarias e interuniversales se simplificará y se acelerará enormemente. Los trabajadores celestiales que están en Urantia encuentran muchas dificultades debido a esta reducción funcional de sus asociados reflectantes. Pero continuamos dirigiendo alegremente nuestros asuntos con los intermediarios disponibles, a pesar de que estemos privados localmente de muchos servicios de estos seres maravillosos, los espejos vivientes del espacio y los proyectores de presencia del tiempo.

\par
%\textsuperscript{(318.3)}
\textsuperscript{28:7.5} [Patrocinado por un Mensajero Poderoso de Uversa.]


\chapter{Documento 29. Los Directores del Poder Universal}
\par
%\textsuperscript{(319.1)}
\textsuperscript{29:0.1} DE TODAS las personalidades del universo implicadas en la reglamentación de los asuntos interplanetarios e interuniversales, los directores del poder y sus asociados son los que han sido menos comprendidos en Urantia. Aunque vuestras razas han conocido desde hace mucho tiempo la existencia de los ángeles y de las órdenes similares de seres celestiales, se ha comunicado muy poca información sobre los controladores y los reguladores del dominio físico\footnote{\textit{Creación física}: Job 9:4-10; 26:7-14; Sal 8:3; 33:6; 104:2-9.}. Incluso ahora sólo se me permite revelar plenamente el último de los tres grupos siguientes de seres vivientes que tienen que ver con el control de la fuerza y la regulación de la energía en el universo maestro:

\par
%\textsuperscript{(319.2)}
\textsuperscript{29:0.2} 1. Los Organizadores de la Fuerza Maestros Existenciados Primarios.

\par
%\textsuperscript{(319.3)}
\textsuperscript{29:0.3} 2. Los Organizadores de la Fuerza Maestros Trascendentales Asociados.

\par
%\textsuperscript{(319.4)}
\textsuperscript{29:0.4} 3. Los Directores del Poder Universal.

\par
%\textsuperscript{(319.5)}
\textsuperscript{29:0.5} Aunque considero imposible describir la individualidad de los diversos grupos de directores, centros y controladores del poder universal, espero poder explicar alguna cosa sobre el ámbito de sus actividades. Forman un grupo único de seres vivientes que tienen que ver con la regulación inteligente de la energía en todo el gran universo. Incluyendo a los directores supremos, abarcan las divisiones principales siguientes:

\par
%\textsuperscript{(319.6)}
\textsuperscript{29:0.6} 1. Los Siete Directores Supremos del Poder.

\par
%\textsuperscript{(319.7)}
\textsuperscript{29:0.7} 2. Los Centros Supremos del Poder.

\par
%\textsuperscript{(319.8)}
\textsuperscript{29:0.8} 3. Los Controladores Físicos Maestros.

\par
%\textsuperscript{(319.9)}
\textsuperscript{29:0.9} 4. Los Supervisores del Poder Morontial.

\par
%\textsuperscript{(319.10)}
\textsuperscript{29:0.10} Los Directores y los Centros Supremos del Poder han existido desde los tiempos cercanos a la eternidad, y por lo que nosotros sabemos ya no se han creado más seres de estas órdenes. Los Siete Directores Supremos fueron personalizados por los Siete Espíritus Maestros, y luego colaboraron con sus padres para crear a más de diez mil millones de asociados. Antes de la época de los directores del poder, los circuitos energéticos del espacio exteriores al universo central estaban bajo la supervisión inteligente de los Organizadores de la Fuerza Maestros del Paraíso.

\par
%\textsuperscript{(319.11)}
\textsuperscript{29:0.11} Como conocéis a las criaturas materiales, al menos tenéis una idea, por contraste, de los seres espirituales; pero a la mente mortal le resulta muy difícil imaginarse a los directores del poder. En el programa de la progresión ascendente hacia los niveles superiores de existencia, no tenéis nada que ver directamente con los directores supremos ni con los centros del poder. En ciertas ocasiones excepcionales tendréis relaciones con los controladores físicos, y cuando lleguéis a los mundos de las mansiones trabajaréis libremente con los supervisores del poder morontial. Estos Supervisores del Poder Morontial ejercen su actividad de forma tan exclusiva en el régimen morontial de las creaciones locales, que consideramos que es mejor narrar sus actividades en la sección que trata del universo local.

\section*{1. Los siete Directores Supremos del Poder}
\par
%\textsuperscript{(320.1)}
\textsuperscript{29:1.1} Los Siete Directores Supremos del Poder son los reguladores de la energía física del gran universo. Su creación por parte de los Siete Espíritus Maestros es el primer caso registrado de la derivación de una progenie semimaterial surgida de una ascendencia verdaderamente espiritual. Cuando los Siete Espíritus Maestros crean individualmente, engendran personalidades altamente espirituales de tipo angélico; cuando crean colectivamente, a veces traen a la existencia a estos tipos elevados de seres semimateriales. Pero incluso estos seres casi físicos serían invisibles para la visión limitada de los mortales de Urantia.

\par
%\textsuperscript{(320.2)}
\textsuperscript{29:1.2} El número de Directores Supremos del Poder es de siete, y su apariencia y sus funciones son idénticas. Nadie puede distinguir al uno del otro, salvo el Espíritu Maestro con el que cada uno de ellos está directamente asociado y al que cada uno de ellos está total y funcionalmente subordinado. Cada uno de los Espíritus Maestros está unido eternamente así con uno de sus descendientes colectivos. El mismo director siempre está asociado con el mismo Espíritu, y su relación de trabajo conduce a una asociación única de las energías físicas y espirituales, de un ser semifísico y de una personalidad espiritual.

\par
%\textsuperscript{(320.3)}
\textsuperscript{29:1.3} Los Siete Directores Supremos del Poder están estacionados en el Paraíso periférico, donde sus presencias que circulan lentamente indican el paradero de las sedes focales de fuerza de los Espíritus Maestros. Estos directores del poder actúan individualmente para regular la energía-poder de los superuniversos, pero colectivamente para administrar la creación central. Operan desde el Paraíso pero se mantienen como centros eficaces del poder en todas las divisiones del gran universo.

\par
%\textsuperscript{(320.4)}
\textsuperscript{29:1.4} Estos seres poderosos son los ascendientes físicos de la inmensa multitud de centros del poder y, a través de ellos, de los controladores físicos dispersos por los siete superuniversos. Estos organismos subordinados del control físico son básicamente uniformes, idénticos, salvo en lo que se refiere al tono diferencial de cada cuerpo superuniversal. Con el objeto de cambiar de servicio superuniversal, les bastaría simplemente con regresar al Paraíso para modificar la tonalidad. La administración de la creación física es fundamentalmente uniforme.

\section*{2. Los Centros Supremos del Poder}
\par
%\textsuperscript{(320.5)}
\textsuperscript{29:2.1} Los Siete Directores Supremos del Poder no pueden reproducirse individualmente, pero colectivamente y en asociación con los Siete Espíritus Maestros pueden reproducir ---crear--- a otros seres semejantes a ellos, y lo hacen de hecho. Éste es el origen de los Centros Supremos del Poder del gran universo, que ejercen su actividad en los siete grupos siguientes:

\par
%\textsuperscript{(320.6)}
\textsuperscript{29:2.2} 1. Los Supervisores Centrales Supremos.

\par
%\textsuperscript{(320.7)}
\textsuperscript{29:2.3} 2. Los Centros de Havona.

\par
%\textsuperscript{(320.8)}
\textsuperscript{29:2.4} 3. Los Centros de los superuniversos.

\par
%\textsuperscript{(320.9)}
\textsuperscript{29:2.5} 4. Los Centros de los universos locales.

\par
%\textsuperscript{(320.10)}
\textsuperscript{29:2.6} 5. Los Centros de las constelaciones.

\par
%\textsuperscript{(320.11)}
\textsuperscript{29:2.7} 6. Los Centros de los sistemas.

\par
%\textsuperscript{(320.12)}
\textsuperscript{29:2.8} 7. Los Centros no clasificados.

\par
%\textsuperscript{(321.1)}
\textsuperscript{29:2.9} Estos centros del poder, junto con los Directores Supremos del Poder, son unos seres con una elevada libertad y acción volitivas. Todos están dotados de una personalidad de la Fuente Tercera y revelan una capacidad volitiva indiscutible de un orden elevado. Estos centros directivos del sistema de poder del universo poseen una exquisita dotación de inteligencia; son el intelecto del sistema de poder del gran universo y el secreto de la técnica del control mental de toda la inmensa red de las extensas funciones de los Controladores Físicos Maestros y de los Supervisores del Poder Morontial.

\par
%\textsuperscript{(321.2)}
\textsuperscript{29:2.10} 1. \textit{Los Supervisores Centrales Supremos.} Estos siete coordinados y asociados de los Directores Supremos del Poder son los reguladores de los circuitos energéticos maestros del gran universo. Cada supervisor central tiene su sede en uno de los mundos especiales de los Siete Ejecutivos Supremos, y trabajan en estrecha asociación con estos coordinadores de los asuntos generales del universo.

\par
%\textsuperscript{(321.3)}
\textsuperscript{29:2.11} Los Directores Supremos del Poder y los Supervisores Centrales Supremos ejercen su actividad tanto de manera individual como conjunta con relación a todos los fenómenos cósmicos que se producen por debajo de los niveles de la <<\textit{energía gravitatoria}>>. Cuando actúan en conexión, estos catorce seres significan para el poder del universo lo que los Siete Ejecutivos Supremos significan para los asuntos generales del universo, y lo que los Siete Espíritus Maestros significan para la mente cósmica.

\par
%\textsuperscript{(321.4)}
\textsuperscript{29:2.12} 2. \textit{Los Centros de Havona.} Antes de la creación de los universos del tiempo y del espacio, los centros del poder no eran necesarios en Havona, pero desde aquellos tiempos tan lejanos, un millón de ellos ha ejercido su actividad en la creación central, teniendo cada centro la supervisión de mil mundos de Havona. Aquí, en el universo divino, el control de la energía es perfecto, una condición que no existe en otras partes. La perfección de la regulación de la energía es la meta final de todos los centros del poder y de todos los controladores físicos del espacio.

\par
%\textsuperscript{(321.5)}
\textsuperscript{29:2.13} 3. \textit{Los Centros de los superuniversos.} Mil centros del poder del tercer tipo ocupan una enorme zona en la esfera capital de cada uno de los siete superuniversos. Tres corrientes de energía primaria, con diez divisiones cada una, entran en estos centros del poder, pero siete circuitos de poder especializados y bien dirigidos, aunque imperfectamente controlados, salen de sus sedes de acción unificada. Ésta es la organización electrónica del poder del universo.

\par
%\textsuperscript{(321.6)}
\textsuperscript{29:2.14} Toda la energía está incluida en el circuito del ciclo del Paraíso, pero los Directores del Poder Universal \textit{dirigen} las energías-fuerzas del Paraíso inferior tal como las encuentran modificadas en las funciones espaciales del universo central y de los superuniversos, convirtiendo y dirigiendo estas energías hacia canales de aplicación útil y constructiva. Existe una diferencia entre la energía de Havona y las energías de los superuniversos. La carga de poder de un superuniverso consiste en tres fases de energía de diez divisiones cada una. Esta triple carga de energía se propaga por todo el espacio del gran universo; es como un inmenso océano de energía en movimiento, que sumerge y baña a la totalidad de cada una de las siete supercreaciones.

\par
%\textsuperscript{(321.7)}
\textsuperscript{29:2.15} La organización electrónica del poder del universo funciona en siete fases y revela una reacción variable a la gravedad local o lineal. Este circuito séptuple procede de los centros superuniversales del poder e impregna cada supercreación. Estas corrientes especializadas del tiempo y del espacio son movimientos de energía precisos y localizados iniciados y dirigidos con fines específicos, de manera muy semejante a como funciona la Corriente del Golfo como fenómeno circunscrito en medio del Océano Atlántico.

\par
%\textsuperscript{(321.8)}
\textsuperscript{29:2.16} 4. \textit{Los Centros de los universos locales.} Cien centros del poder de la cuarta orden se encuentran estacionados en la sede de cada universo local. Ejercen su actividad para reducir y modificar de otras maneras los siete circuitos del poder que emanan de la sede del superuniverso, haciéndolos así aplicables a los servicios de las constelaciones y de los sistemas. Estos centros del poder conceden un interés pasajero a las catástrofes astronómicas locales del espacio; se dedican al envío ordenado de la energía efectiva a las constelaciones y los sistemas subsidiarios. Son de una gran ayuda para los Hijos Creadores durante las épocas finales de la organización de sus universos y de la movilización de la energía. Estos centros son capaces de proporcionar canales intensificados de energía que son útiles para la comunicación interplanetaria entre los puntos habitados importantes. Estos \textit{canales} o \textit{líneas} de energía, a veces llamados también caminos de energía, son unos circuitos directos de energía entre un centro del poder y otro centro del poder, o entre un controlador físico y otro controlador. Es una corriente individualizada de poder y contrasta con los movimientos de la energía no diferenciada en el espacio libre.

\par
%\textsuperscript{(322.1)}
\textsuperscript{29:2.17} 5. \textit{Los Centros de las constelaciones.} Diez centros vivientes del poder de este tipo están estacionados en cada constelación, donde actúan como proyectores de energía hacia los cien sistemas locales tributarios. De estos seres salen las líneas de poder destinadas a la comunicación y al transporte, y para proporcionar energía a aquellas criaturas vivientes que dependen de ciertas formas de energía física para mantenerse con vida. Pero ni los centros del poder ni los controladores físicos subordinados se ocupan de otra manera de la vida como organización funcional.

\par
%\textsuperscript{(322.2)}
\textsuperscript{29:2.18} 6. \textit{Los Centros de los sistemas.} Un Centro Supremo del Poder está asignado permanentemente a cada sistema local. Estos centros de los sistemas envían los circuitos del poder a los mundos habitados del tiempo y del espacio. Coordinan las actividades de los controladores físicos subordinados y actúan además para asegurar la distribución satisfactoria del poder en el sistema local. El relé del circuito entre los planetas depende de la coordinación perfecta de ciertas energías materiales y de la regulación eficaz del poder físico.

\par
%\textsuperscript{(322.3)}
\textsuperscript{29:2.19} 7. \textit{Los Centros no clasificados.} Son los centros que funcionan en situaciones locales especiales, pero no en los planetas habitados. Los mundos individuales están a cargo de los Controladores Físicos Maestros y reciben las líneas del poder incorporadas en circuitos enviadas por el centro del poder de su sistema. Sólo aquellas esferas que poseen unas relaciones energéticas de las más extraordinarias tienen centros del poder de la séptima orden, que actúan como ruedas equilibradoras universales o gobernadores de la energía. Estos centros del poder son, en todas sus fases de actividad, totalmente equivalentes a aquellos que ejercen sus funciones en las unidades superiores de control, pero ni un solo cuerpo espacial entre un millón contiene este tipo de organización viviente del poder.

\section*{3. El ámbito de los Centros del Poder}
\par
%\textsuperscript{(322.4)}
\textsuperscript{29:3.1} Los Centros Supremos del Poder distribuidos en todos los superuniversos, con sus asociados y subordinados, ascienden a más de diez mil millones. Y todos están en sincronismo perfecto y en completa coordinación con sus progenitores del Paraíso, los Siete Directores Supremos del Poder. El control del poder del gran universo se ha confiado así al cuidado y a la dirección de los Siete Espíritus Maestros, los creadores de los Siete Directores Supremos del Poder.

\par
%\textsuperscript{(322.5)}
\textsuperscript{29:3.2} Los Directores Supremos del Poder y todos sus asociados, asistentes y subordinados, están exentos para siempre de ser arrestados o interferidos por todos los tribunales de todo el espacio; tampoco están sujetos a la dirección administrativa del gobierno superuniversal de los Ancianos de los Días, ni a la administración de los universos locales de los Hijos Creadores.

\par
%\textsuperscript{(323.1)}
\textsuperscript{29:3.3} Estos centros y directores del poder son traídos a la existencia por los hijos del Espíritu Infinito. No están relacionados con la administración de los Hijos de Dios, aunque se asocian con los Hijos Creadores durante las épocas finales de la organización material de sus universos. Pero los centros del poder están de alguna forma estrechamente asociados con el supercontrol cósmico del Ser Supremo.

\par
%\textsuperscript{(323.2)}
\textsuperscript{29:3.4} Los centros del poder y los controladores físicos no sufren ningún entrena-miento; todos son creados perfectos y actúan con perfección de manera inherente. Nunca pasan de una función a otra; siempre sirven en su destino original. No existe ninguna evolución en sus filas y esto es así en las siete divisiones de las dos órdenes.

\par
%\textsuperscript{(323.3)}
\textsuperscript{29:3.5} Como no tienen un pasado ascendente que recordar, los centros del poder y los controladores físicos nunca se dedican a la diversión; son totalmente prácticos en todas sus acciones. Siempre están de servicio; en el plan universal no existen disposiciones para interrumpir las líneas físicas de energía. Estos seres no pueden abandonar nunca, ni siquiera durante una fracción de segundo, su supervisión directa de los circuitos energéticos del tiempo y del espacio.

\par
%\textsuperscript{(323.4)}
\textsuperscript{29:3.6} Los directores, centros y controladores del poder no están relacionados con nada en toda la creación, salvo con el poder, con la energía material o semifísica; no lo originan, pero sí lo modifican, lo manipulan y lo orientan. Tampoco tienen nada que ver en absoluto con la gravedad física, excepto para resistir su poder de atracción. Su relación con la gravedad es totalmente negativa.

\par
%\textsuperscript{(323.5)}
\textsuperscript{29:3.7} Los centros del poder utilizan inmensos mecanismos y coordinaciones de tipo material en conexión con los mecanismos vivientes de las diversas concentraciones separadas de energía. Cada centro individual del poder está compuesto exactamente de un millón de unidades de control funcional, y estas unidades modificadoras de la energía no son estacionarias como los órganos vitales del cuerpo físico del hombre; las posibilidades asociativas de estos <<\textit{órganos vitales}>> de la regulación del poder son móviles y verdaderamente calidoscópicas.

\par
%\textsuperscript{(323.6)}
\textsuperscript{29:3.8} Soy totalmente incapaz de explicar la manera en que estos seres vivientes abarcan la manipulación y la regulación de los circuitos maestros de la energía universal. Emprender la tarea de informaros aún más sobre el tamaño y la función de estos gigantescos centros del poder casi perfectamente eficaces sólo aumentaría vuestra confusión y vuestra consternación. Son al mismo tiempo vivientes y <<\textit{personales}>>, pero están más allá de vuestra comprensión.

\par
%\textsuperscript{(323.7)}
\textsuperscript{29:3.9} Fuera de Havona, los Centros Supremos del Poder sólo ejercen su actividad en las esferas especialmente construidas (arquitectónicas) o en los cuerpos espaciales por otra parte adecuadamente constituidos. Los mundos arquitectónicos están construídos de tal manera que los centros vivientes del poder pueden actuar como conmutadores selectivos para orientar, modificar y concentrar las energías del espacio a medida que se derraman sobre estas esferas. No podrían ejercer sus funciones de esta manera en un sol o en un planeta evolutivos ordinarios. Algunos grupos también se ocupan del calentamiento y de otras necesidades materiales de estos mundos sede especiales. Y aunque sobrepasa el alcance del conocimiento urantiano, puedo indicar que estas órdenes de personalidades vivientes del poder tienen mucho que ver con la distribución de la luz que brilla sin calor. No producen este fenómeno, pero se ocupan de diseminarlo y de orientarlo.

\par
%\textsuperscript{(323.8)}
\textsuperscript{29:3.10} Los centros del poder y sus controladores subordinados están asignados al funcionamiento de todas las energías físicas del espacio organizado. Trabajan con las tres corrientes básicas de diez energías cada una. Ésta es la carga de energía del espacio organizado; y el espacio organizado es su campo de acción. Los Directores del Poder Universal no tienen absolutamente nada que ver con las extraordinarias acciones de fuerza que se están produciendo ahora fuera de las fronteras actuales de los siete superuniversos.

\par
%\textsuperscript{(324.1)}
\textsuperscript{29:3.11} Los centros y los controladores del poder sólo ejercen un perfecto control sobre siete de las diez formas de energía contenidas en cada corriente básica universal; aquellas formas que están total o parcialmente libres de su control deben representar los reinos imprevisibles de la manifestación de la energía dominados por el Absoluto Incalificado. Si ejercen una influencia sobre las fuerzas primordiales de este Absoluto, no conocemos estas funciones, aunque existe alguna pequeña prueba que justificaría la opinión de que algunos controladores físicos reaccionan a veces de manera automática a ciertos impulsos del Absoluto Universal.

\par
%\textsuperscript{(324.2)}
\textsuperscript{29:3.12} Estos mecanismos vivientes del poder no están relacionados conscientemente con el supercontrol energético del universo maestro ejercido por el Absoluto Incalificado, pero suponemos que todo su sistema casi perfecto de dirección del poder está subordinado de alguna manera desconocida a esta presencia supergravitatoria. En cualquier situación energética local, los centros y los controladores ejercen una supremacía casi total, pero siempre son conscientes de la presencia superenergética y de la acción no reconocible del Absoluto Incalificado.

\section*{4. Los Controladores Físicos Maestros}
\par
%\textsuperscript{(324.3)}
\textsuperscript{29:4.1} Estos seres son los subordinados móviles de los Centros Supremos del Poder. Los controladores físicos están dotados de unas aptitudes de metamorfosis de la individualidad de tal naturaleza que pueden efectuar una extraordinaria variedad de autotransportes, siendo capaces de atravesar el espacio local a unas velocidades cercanas a las de los Mensajeros Solitarios. Pero al igual que todos los demás seres que atraviesan el espacio, necesitan la ayuda de sus compañeros así como de algunos otros tipos de seres para vencer la acción de la gravedad y la resistencia de la inercia a la hora de partir de una esfera material.

\par
%\textsuperscript{(324.4)}
\textsuperscript{29:4.2} Los Controladores Físicos Maestros sirven en todo el gran universo. Están gobernados directamente desde el Paraíso hasta las sedes de los superuniversos por los Siete Directores Supremos del Poder; desde aquí son dirigidos y distribuidos por el Consejo del Equilibrio, compuesto por los altos comisionados del poder enviados por los Siete Espíritus Maestros y elegidos entre el personal de los Organizadores de la Fuerza Maestros Asociados. Estos altos comisionados están facultados para interpretar las indicaciones y los registros de los maestros frandalanks, esos instrumentos vivientes que indican la presión del poder y la carga de energía de todo un superuniverso.

\par
%\textsuperscript{(324.5)}
\textsuperscript{29:4.3} Aunque la presencia de las Deidades del Paraíso envuelve al gran universo y se extiende alrededor del círculo de la eternidad, la influencia de cualquiera de los Siete Espíritus Maestros está limitada a un solo superuniverso. Existe una clara segregación de la energía y una separación de los circuitos del poder entre cada una de las siete supercreaciones; de ahí que los métodos individualizados de control deban prevalecer, y prevalecen de hecho.

\par
%\textsuperscript{(324.6)}
\textsuperscript{29:4.4} Los Controladores Físicos Maestros son los descendientes directos de los Centros Supremos del Poder, y entre sus grupos se incluyen a los siguientes:

\par
%\textsuperscript{(324.7)}
\textsuperscript{29:4.5} 1. Los Directores Asociados del Poder.

\par
%\textsuperscript{(324.8)}
\textsuperscript{29:4.6} 2. Los Controladores Maquinales.

\par
%\textsuperscript{(324.9)}
\textsuperscript{29:4.7} 3. Los Transformadores de la Energía.

\par
%\textsuperscript{(325.1)}
\textsuperscript{29:4.8} 4. Los Transmisores de la Energía.

\par
%\textsuperscript{(325.2)}
\textsuperscript{29:4.9} 5. Los Asociadores Primarios.

\par
%\textsuperscript{(325.3)}
\textsuperscript{29:4.10} 6. Los Disociadores Secundarios.

\par
%\textsuperscript{(325.4)}
\textsuperscript{29:4.11} 7. Los Frandalanks y los Cronoldeks.

\par
%\textsuperscript{(325.5)}
\textsuperscript{29:4.12} No todos los miembros de estas órdenes son personas en el sentido de poseer el poder individual de elección. Las cuatro últimas órdenes en especial parecen ser totalmente automáticas y maquinales en su respuesta a los impulsos de sus superiores y en su reacción a las condiciones energéticas existentes. Pero aunque esta respuesta parezca totalmente mecánica, no lo es; estos seres pueden parecer autómatas, pero todos revelan la función diferencial de la inteligencia.

\par
%\textsuperscript{(325.6)}
\textsuperscript{29:4.13} La personalidad no acompaña necesariamente a la mente. La mente puede pensar incluso cuando está privada de todo poder de elección, como sucede en numerosos tipos inferiores de animales y en algunos de estos controladores físicos subordinados. Una gran parte de estos reguladores más automáticos del poder físico no son personas en ningún sentido de la palabra. No están dotados de voluntad ni de la independencia de decisión, permaneciendo totalmente subordinados a la perfección maquinal de su diseño para las tareas a las que están asignados. Sin embargo, todos son seres sumamente inteligentes.

\par
%\textsuperscript{(325.7)}
\textsuperscript{29:4.14} Los controladores físicos se ocupan principalmente de ajustar energías fundamentales no descubiertas en Urantia. Estas energías desconocidas son muy esenciales para el sistema interplanetario de transporte y para ciertas técnicas de comunicación. Cuando instalamos unas líneas de energía con el objeto de transmitir los equivalentes del sonido o de ampliar la visión, estas formas no descubiertas de energía son utilizadas por los controladores físicos vivientes y sus asociados. Las criaturas intermedias también utilizan de vez en cuando estas mismas energías en su trabajo rutinario.

\par
%\textsuperscript{(325.8)}
\textsuperscript{29:4.15} 1. \textit{Los Directores Asociados del Poder.} Estos seres maravillosamente eficaces están encargados de designar y de enviar a todas las órdenes de Controladores Físicos Maestros de acuerdo con las necesidades siempre variables del estado energético en constante cambio de los reinos. Las inmensas reservas de controladores físicos se mantienen en los mundos sede de los sectores menores, y desde estos puntos de concentración son enviados periódicamente por los directores asociados del poder a las sedes de los universos, las constelaciones, los sistemas y los planetas individuales. Cuando efectúan esta misión, los controladores físicos están sometidos provisionalmente a las órdenes de los ejecutores divinos de las comisiones conciliadoras, pero por lo demás únicamente son responsables ante sus directores asociados y ante los Centros Supremos del Poder.

\par
%\textsuperscript{(325.9)}
\textsuperscript{29:4.16} Tres millones de directores asociados del poder están asignados a cada uno de los sectores menores de Orvonton, y el contingente superuniversal de estos seres asombrosamente polifacéticos asciende a un total de tres mil millones. Mantienen sus propias reservas en estos mismos mundos de los sectores menores, donde sirven también como instructores para todos los que estudian las ciencias de las técnicas del control y de la transmutación inteligentes de la energía.

\par
%\textsuperscript{(325.10)}
\textsuperscript{29:4.17} Estos directores alternan sus períodos de servicio ejecutivo en los sectores menores con otros períodos equivalentes de servicio como inspectores en los reinos del espacio. Al menos un inspector en funciones está siempre presente en cada sistema local, manteniendo su sede en la esfera capital. Conservan todo el inmenso agregado de energía viviente en un armonioso sincronismo.

\par
%\textsuperscript{(325.11)}
\textsuperscript{29:4.18} 2. \textit{Los Controladores Maquinales.} Son los asistentes móviles y extremada-mente polifacéticos de los directores asociados del poder. Billones y billones de ellos están de servicio en Ensa, vuestro sector menor. A estos seres los llamamos controladores maquinales porque están completamente dominados por sus superiores, totalmente subordinados a la voluntad de los directores asociados del poder. Sin embargo, son por sí mismos muy inteligentes, y aunque su trabajo sea de naturaleza maquinal y práctica, lo ejecutan con habilidad.

\par
%\textsuperscript{(326.1)}
\textsuperscript{29:4.19} De todos los Controladores Físicos Maestros asignados a los mundos habitados, los controladores maquinales son con mucho los más poderosos. Como cada controlador posee el don viviente de la antigravedad de manera muy superior a todos los demás seres, tiene una resistencia a la gravedad que sólo es igualada por las esferas enormes que giran a una velocidad extraordinaria. Diez controladores de este tipo están estacionados actualmente en Urantia, y una de sus actividades planetarias más importantes consiste en facilitar la partida de los transportes seráficos. Para realizar esta función, los diez controladores maquinales actúan al unísono, mientras que un grupo de mil transmisores de energía proporcionan el impulso inicial para la partida seráfica.

\par
%\textsuperscript{(326.2)}
\textsuperscript{29:4.20} Los controladores maquinales son capaces de orientar el flujo de energía y de facilitar su concentración en las corrientes o circuitos especializados. Estos poderosos seres tienen mucho que ver con la separación, la orientación y la intensificación de las energías físicas, y con la igualación de las presiones de los circuitos interplanetarios. Son expertos en la manipulación de veintiuna de las treinta energías físicas del espacio, que constituyen la carga de poder de un superuniverso. También son capaces de llevar a cabo una gran parte de la gestión y del control de seis de las nueve formas más sutiles de energía física. Al colocar a estos controladores en una relación técnica apropiada entre sí y con algunos centros del poder, los directores asociados del poder son capaces de efectuar unos cambios increíbles en el ajuste del poder y en el control de la energía.

\par
%\textsuperscript{(326.3)}
\textsuperscript{29:4.21} Los Controladores Físicos Maestros ejercen a menudo su actividad en grupos de cientos, de miles, e incluso de millones y, variando sus posiciones y sus formaciones, son capaces de controlar la energía de forma colectiva así como individual. A medida que cambian las necesidades, pueden aumentar y acelerar el volumen y el movimiento de la energía, o detener, condensar y retrasar las corrientes energéticas. En cierto modo, influyen sobre las transformaciones de la energía y del poder como los llamados agentes catalíticos aumentan las reacciones químicas. Desempeñan su actividad por capacidad inherente y en cooperación con los Centros Supremos del Poder.

\par
%\textsuperscript{(326.4)}
\textsuperscript{29:4.22} 3. \textit{Los Transformadores de la Energía.} La cantidad que hay de estos seres en un superuniverso es increíble. Sólo en Satania hay casi un millón, y el contingente habitual es de cien transformadores por cada mundo habitado.

\par
%\textsuperscript{(326.5)}
\textsuperscript{29:4.23} Los transformadores de la energía son la creación conjunta de los Siete Directores Supremos del Poder y de los Siete Supervisores Centrales. Figuran entre las órdenes más personales de los controladores físicos, y salvo en los casos en que un director asociado del poder se encuentre presente en un mundo habitado, los transformadores son los que están al mando. Son los inspectores planetarios de todos los transportes seráficos que parten. Todas las clases de vida celestial sólo pueden utilizar las órdenes menos personales de los controladores físicos mediante el enlace con las órdenes más personales de los directores asociados y de los transformadores de la energía.

\par
%\textsuperscript{(326.6)}
\textsuperscript{29:4.24} Estos transformadores son unos conmutadores vivientes poderosos y eficaces, y son capaces de colocarse a favor o en contra de una disposición u orientación dadas del poder. También son hábiles en sus esfuerzos por aislar a los planetas de las poderosas corrientes energéticas que pasan entre los gigantescos vecinos planetarios y estelares. Sus atributos transmutadores de la energía los hacen sumamente útiles en la importante tarea de mantener el equilibrio energético universal, o equilibrio del poder. En ciertos momentos parecen consumir o almacenar energía; en otros momentos parecen exudar o liberar energía. Los transformadores son capaces de aumentar o disminuir el potencial <<\textit{acumulador}>> de las energías vivas y muertas de sus reinos respectivos. Pero sólo se ocupan de las energías físicas y semimateriales, no actúan directamente en el ámbito de la vida, ni tampoco cambian las formas de los seres vivos.

\par
%\textsuperscript{(327.1)}
\textsuperscript{29:4.25} Los transformadores de la energía son en algunos aspectos las más notables y misteriosas de todas las criaturas semimateriales vivientes. Están físicamente diferenciados de alguna manera desconocida y, variando sus relaciones de conexión, son capaces de ejercer una profunda influencia sobre la energía que pasa a través de sus presencias asociadas. El estado de los reinos físicos parece sufrir una transformación bajo sus hábiles manipulaciones. \textit{Pueden cambiar la forma física de las energías del espacio, y lo hacen de hecho.} Con la ayuda de sus compañeros controladores, son capaces de cambiar realmente la forma y el potencial de veintisiete de las treinta energías físicas de la carga de poder de un superuniverso. El hecho de que tres de estas energías estén más allá de su control demuestra que no son los intermediarios del Absoluto Incalificado.

\par
%\textsuperscript{(327.2)}
\textsuperscript{29:4.26} Los cuatro grupos restantes de Controladores Físicos Maestros apenas son personas según cualquier definición aceptable de esta palabra. Las reacciones de estos transmisores, asociadores, disociadores y frandalanks son totalmente automáticas; sin embargo, son inteligentes en todos los sentidos. Estamos enormemente limitados en nuestro conocimiento de estas entidades maravillosas porque no podemos comunicarnos con ellas. Parecen comprender el lenguaje del reino, pero no pueden comunicarse con nosotros. Parecen plenamente capaces de recibir nuestras comunicaciones, pero totalmente impotentes para responder a ellas.

\par
%\textsuperscript{(327.3)}
\textsuperscript{29:4.27} 4. \textit{Los Transmisores de la Energía.} Estos seres ejercen principalmente su actividad, aunque no por completo, en las operaciones intraplanetarias. Son unos maravillosos expedidores de la energía tal como ésta se manifiesta en los mundos individuales.

\par
%\textsuperscript{(327.4)}
\textsuperscript{29:4.28} Cuando la energía ha de ser desviada hacia un nuevo circuito, los transmisores se despliegan en línea a lo largo del recorrido energético deseado y, debido a sus atributos excepcionales de atracción de la energía, pueden inducir realmente un flujo creciente de energía en la dirección deseada. Esto lo hacen de forma tan literal como ciertos circuitos metálicos orientan el flujo de ciertas formas de energía eléctrica; y son superconductores vivientes para más de la mitad de las treinta formas de energía física.

\par
%\textsuperscript{(327.5)}
\textsuperscript{29:4.29} Los transmisores forman hábiles conexiones que son eficaces para rehabilitar las corrientes debilitadas de energía especializada que pasan de un planeta a otro y de una estación a otra en un planeta individual. Pueden detectar corrientes que son demasiado débiles como para ser reconocidas por cualquier otro tipo de ser viviente, y pueden aumentar estas energías de tal manera que el mensaje que las acompaña se vuelve perfectamente inteligible. Sus servicios son inapreciables para los receptores de las transmisiones.

\par
%\textsuperscript{(327.6)}
\textsuperscript{29:4.30} Los transmisores de la energía pueden ejercer su actividad con respecto a todas las formas de percepción comunicable; pueden hacer que una escena lejana resulte <<\textit{visible}>> así como que un sonido distante se vuelva <<\textit{audible}>>. Proporcionan las líneas de comunicación de urgencia en los sistemas locales y en los planetas individuales. Prácticamente todas las criaturas han de utilizar estos servicios con el fin de comunicarse fuera de los circuitos regularmente establecidos.

\par
%\textsuperscript{(327.7)}
\textsuperscript{29:4.31} Estos seres, junto con los transformadores de la energía, son indispensables para mantener la existencia mortal en aquellos mundos que poseen una atmósfera empobrecida, y forman parte integrante de la técnica de vida en los planetas de los no respiradores.

\par
%\textsuperscript{(328.1)}
\textsuperscript{29:4.32} 5. \textit{Los Asociadores Primarios.} Estas entidades interesantes e inapreciables son los conservadores y los custodios magistrales de la energía. Estos organismos vivientes almacenan la energía durante los períodos en que sus manifestaciones son mayores, en cierto modo como una planta almacena la luz solar. Trabajan a una escala gigantesca, convirtiendo las energías del espacio en un estado físico desconocido en Urantia. También son capaces de llevar adelante estas transformaciones hasta el punto de producir algunas de las unidades primitivas de la existencia material. Estos seres actúan simplemente con su presencia. No se agotan ni se reducen de ninguna manera a causa de esta función; actúan como agentes catalíticos vivientes.

\par
%\textsuperscript{(328.2)}
\textsuperscript{29:4.33} Durante los períodos en que las manifestaciones son menores, tienen la facultad de liberar estas energías acumuladas. Pero vuestro conocimiento de la energía y de la materia no es lo suficientemente avanzado como para permitirnos explicar la técnica de esta fase de su trabajo. Siempre actúan de acuerdo con la ley universal, manejando y manipulando los átomos, los electrones y los ultimatones de una manera muy semejante a como vosotros manipuláis los caracteres de imprenta ajustables para hacer que los mismos símbolos alfabéticos cuenten historias sumamente diferentes.

\par
%\textsuperscript{(328.3)}
\textsuperscript{29:4.34} Los asociadores son el primer grupo viviente que aparece en una esfera material en vías de organización, y pueden ejercer su actividad a temperaturas físicas que vosotros consideraríais totalmente incompatibles con la existencia de los seres vivos. Representan un tipo de vida que está simplemente más allá del alcance de la imaginación humana. Junto con sus colaboradores, los disociadores, son las más serviles de todas las criaturas inteligentes.

\par
%\textsuperscript{(328.4)}
\textsuperscript{29:4.35} 6. \textit{Los Disociadores Secundarios.} Comparados con los asociadores primarios, estos seres dotados de unas enormes facultades antigravitatorias son los trabajadores inversos. Nunca existe ningún peligro de que se agoten las formas especiales o modificadas de la energía física en los mundos locales o en los sistemas locales, porque estas organizaciones vivientes están dotadas del poder excepcional de desprender provisiones ilimitadas de energía. Se ocupan principalmente de la evolución de una forma de energía apenas conocida en Urantia, que proviene de una forma de materia aún menos reconocida. Son en verdad los alquimistas del espacio y los autores de prodigios del tiempo. Pero en todas las maravillas que hacen, nunca infringen los mandatos de la Supremacía Cósmica.

\par
%\textsuperscript{(328.5)}
\textsuperscript{29:4.36} 7. \textit{Los Frandalanks.} Estos seres son la creación conjunta de las tres órdenes de seres que controlan la energía: los organizadores primarios y secundarios de la fuerza y los directores del poder. Los frandalanks son los más numerosos de todos los Controladores Físicos Maestros; el número de ellos que ejercen su actividad solamente en Satania sobrepasa vuestro concepto numérico. Están estacionados en todos los mundos habitados y siempre están vinculados a las órdenes superiores de controladores físicos. Trabajan de manera intercambiable en el universo central, en los superuniversos y en los dominios del espacio exterior.

\par
%\textsuperscript{(328.6)}
\textsuperscript{29:4.37} Los frandalanks son creados en treinta divisiones, una para cada forma de fuerza básica universal, y actúan exclusivamente como indicadores vivientes y automáticos de las presencias, las presiones y las velocidades. Estos barómetros vivientes se ocupan exclusivamente de registrar de manera automática e infalible el estado de todas las formas de energía-fuerza. Representan para el universo físico lo mismo que el inmenso mecanismo de la reflectividad representa para el universo mental. Los frandalanks que registran el tiempo además de la presencia cuantitativa y cualitativa de la energía se llaman \textit{cronoldeks.}

\par
%\textsuperscript{(328.7)}
\textsuperscript{29:4.38} Reconozco que los frandalanks son inteligentes, pero no los puedo clasificar de otro modo que como máquinas vivientes. Casi la única manera en que puedo ayudaros a comprender estos mecanismos vivientes es compararlos con vuestros propios aparatos mecánicos que funcionan con una precisión y una exactitud casi semejantes a la inteligencia. Así pues, si deseáis concebir a estos seres, utilizad vuestra imaginación hasta el punto de reconocer que en el gran universo tenemos realmente unos mecanismos (entidades) inteligentes y \textit{vivientes} que pueden realizar tareas más complicadas, las cuales requieren unos cálculos más prodigiosos, con una delicadeza de exactitud aún más grande e incluso con una precisión última.

\section*{5. Los Organizadores de la Fuerza Maestros}
\par
%\textsuperscript{(329.1)}
\textsuperscript{29:5.1} Los organizadores de la fuerza residen en el Paraíso pero ejercen su actividad en todo el universo maestro, y más especialmente en los dominios del espacio no organizado. Estos seres extraordinarios no son ni creadores ni criaturas, y están clasificados en dos grandes divisiones de servicio:

\par
%\textsuperscript{(329.2)}
\textsuperscript{29:5.2} 1. Los Organizadores de la Fuerza Maestros Existenciados Primarios.

\par
%\textsuperscript{(329.3)}
\textsuperscript{29:5.3} 2. Los Organizadores de la Fuerza Maestros Trascendentales Asociados.

\par
%\textsuperscript{(329.4)}
\textsuperscript{29:5.4} Estas dos poderosas órdenes de manipuladores de la fuerza primordial trabajan exclusivamente bajo la supervisión de los Arquitectos del Universo Maestro, y en el momento actual no ejercen ampliamente su actividad dentro de las fronteras del gran universo.

\par
%\textsuperscript{(329.5)}
\textsuperscript{29:5.5} Los Organizadores de la Fuerza Maestros Primarios son los manipuladores de las fuerzas espaciales primordiales o fundamentales del Absoluto Incalificado; son los creadores de las nebulosas. Son los instigadores vivientes de los ciclones energéticos del espacio y los organizadores y orientadores iniciales de estas manifestaciones gigantescas. Estos organizadores de la fuerza transmutan la \textit{fuerza primordial} (la pre-energía no sensible a la gravedad directa del Paraíso) en \textit{energía poderosa} o primaria, la energía que se transmuta desde la atracción exclusiva del Absoluto Incalificado hasta la atracción gravitatoria de la Isla del Paraíso. Les siguen inmediatamente después los organizadores de la fuerza asociados, los cuales continúan el proceso de transmutación de la energía desde la etapa primaria hasta la etapa secundaria o de la \textit{energía-gravedad.}

\par
%\textsuperscript{(329.6)}
\textsuperscript{29:5.6} Cuando los planes para la creación de un universo local se han ejecutado, lo cual es señalado por la llegada de un Hijo Creador, los Organizadores de la Fuerza Maestros Asociados ceden el paso a las órdenes de los directores del poder que actúan en el superuniverso que posee esa jurisdicción astronómica. Pero en ausencia de dichos planes, los organizadores de la fuerza asociados continúan encargándose indefinidamente de estas creaciones materiales, tal como lo hacen actualmente en el espacio exterior.

\par
%\textsuperscript{(329.7)}
\textsuperscript{29:5.7} Los Organizadores de la Fuerza Maestros resisten unas temperaturas y ejercen su actividad en unas condiciones físicas que serían intolerables incluso para los polifacéticos centros del poder y controladores físicos de Orvonton. Los otros únicos tipos de seres revelados capaces de ejercer sus funciones en estos reinos del espacio exterior son los Mensajeros Solitarios y los Espíritus Inspirados Trinitarios.

\par
%\textsuperscript{(329.8)}
\textsuperscript{29:5.8} [Patrocinado por un Censor Universal que actúa por autorización de los Ancianos de los Días de Uversa.]


\chapter{Documento 30. Las personalidades del gran universo}
\par
%\textsuperscript{(330.1)}
\textsuperscript{30:0.1} LAS personalidades y las entidades distintas a las personales que ejercen actualmente su actividad en el Paraíso y en el gran universo constituyen un número casi ilimitado de seres vivientes. Incluso el número de las órdenes y de los tipos principales haría titubear la imaginación humana, sin hablar de los innumerables subtipos y variaciones. Sin embargo, es deseable presentar alguna información sobre las dos clasificaciones fundamentales de los seres vivientes ---una idea de la clasificación del Paraíso y una abreviación del Registro de Personalidades existente en Uversa.

\par
%\textsuperscript{(330.2)}
\textsuperscript{30:0.2} No es posible formular clasificaciones de conjunto y totalmente coherentes de las personalidades del gran universo porque \textit{todos} los grupos no han sido revelados. Se precisarían numerosos documentos adicionales para abarcar la nueva revelación necesaria para clasificar sistemáticamente todos los grupos. Esta expansión conceptual difícilmente sería deseable, porque privaría a los mortales pensantes de los próximos mil años de ese estímulo a la especulación creativa que proporcionan estos conceptos parcialmente revelados. Es mejor que el hombre no reciba una revelación excesiva; eso ahoga la imaginación.

\section*{1. La clasificación paradisiaca de los seres vivientes}
\par
%\textsuperscript{(330.3)}
\textsuperscript{30:1.1} Los seres vivientes están clasificados en el Paraíso de acuerdo con su relación inherente, y con aquella que han alcanzado, con las Deidades del Paraíso. Durante las grandiosas asambleas del universo central y de los superuniversos, las personas presentes son agrupadas con frecuencia de acuerdo con su origen: las de origen trino o que han alcanzado a la Trinidad; las de origen doble; y las de origen único. Es difícil interpretar para la mente mortal la clasificación paradisiaca de los seres vivientes, pero tenemos la autorización de presentar la siguiente:

\par
%\textsuperscript{(330.4)}
\textsuperscript{30:1.2} I. \textit{SERES DE ORIGEN TRINO.} Los seres creados por las tres Deidades del Paraíso, ya sea como tales o como Trinidad, junto con el Cuerpo Trinitizado, designación que se refiere a todos los grupos de seres trinitizados, revelados y no revelados.

\par
%\textsuperscript{(330.5)}
\textsuperscript{30:1.3} \textit{A. Los Espíritus Supremos.}

\par
%\textsuperscript{(330.6)}
\textsuperscript{30:1.4} 1. Los Siete Espíritus Maestros.

\par
%\textsuperscript{(330.7)}
\textsuperscript{30:1.5} 2. Los Siete Ejecutivos Supremos.

\par
%\textsuperscript{(330.8)}
\textsuperscript{30:1.6} 3. Las Siete Órdenes de Espíritus Reflectantes.

\par
%\textsuperscript{(330.9)}
\textsuperscript{30:1.7} \textit{B. Los Hijos Estacionarios de la Trinidad.}

\par
%\textsuperscript{(330.10)}
\textsuperscript{30:1.8} 1. Los Secretos Trinitizados de la Supremacía.

\par
%\textsuperscript{(330.11)}
\textsuperscript{30:1.9} 2. Los Eternos de los Días.

\par
%\textsuperscript{(330.12)}
\textsuperscript{30:1.10} 3. Los Ancianos de los Días.

\par
%\textsuperscript{(330.13)}
\textsuperscript{30:1.11} 4. Los Perfecciones de los Días.

\par
%\textsuperscript{(331.1)}
\textsuperscript{30:1.12} 5. Los Recientes de los Días.

\par
%\textsuperscript{(331.2)}
\textsuperscript{30:1.13} 6. Los Uniones de los Días.

\par
%\textsuperscript{(331.3)}
\textsuperscript{30:1.14} 7. Los Fieles de los Días.

\par
%\textsuperscript{(331.4)}
\textsuperscript{30:1.15} 8. Los Perfeccionadores de la Sabiduría.

\par
%\textsuperscript{(331.5)}
\textsuperscript{30:1.16} 9. Los Consejeros Divinos.

\par
%\textsuperscript{(331.6)}
\textsuperscript{30:1.17} 10. Los Censores Universales.

\par
%\textsuperscript{(331.7)}
\textsuperscript{30:1.18} \textit{C. Seres de Origen Trinitario y Seres Trinitizados.}

\par
%\textsuperscript{(331.8)}
\textsuperscript{30:1.19} 1. Los Hijos Instructores Trinitarios.

\par
%\textsuperscript{(331.9)}
\textsuperscript{30:1.20} 2. Los Espíritus Inspirados Trinitarios.

\par
%\textsuperscript{(331.10)}
\textsuperscript{30:1.21} 3. Los Nativos de Havona.

\par
%\textsuperscript{(331.11)}
\textsuperscript{30:1.22} 4. Los Ciudadanos del Paraíso.

\par
%\textsuperscript{(331.12)}
\textsuperscript{30:1.23} 5. Los Seres No Revelados de Origen Trinitario.

\par
%\textsuperscript{(331.13)}
\textsuperscript{30:1.24} 6. Los Seres No Revelados Trinitizados por la Deidad.

\par
%\textsuperscript{(331.14)}
\textsuperscript{30:1.25} 7. Los Hijos de la Consecución Trinitizados.

\par
%\textsuperscript{(331.15)}
\textsuperscript{30:1.26} 8. Los Hijos de la Elección Trinitizados.

\par
%\textsuperscript{(331.16)}
\textsuperscript{30:1.27} 9. Los Hijos de la Perfección Trinitizados.

\par
%\textsuperscript{(331.17)}
\textsuperscript{30:1.28} 10. Los Hijos Trinitizados por las Criaturas.

\par
%\textsuperscript{(331.18)}
\textsuperscript{30:1.29} II. \textit{SERES DE ORIGEN DOBLE.} Aquellos que tienen su origen en dos cualquiera de las Deidades del Paraíso o han sido creados de otra manera por dos seres cualquiera que descienden directa o indirectamente de las Deidades del Paraíso.

\par
%\textsuperscript{(331.19)}
\textsuperscript{30:1.30} \textit{A. Las Órdenes Descendentes.}

\par
%\textsuperscript{(331.20)}
\textsuperscript{30:1.31} 1. Los Hijos Creadores.

\par
%\textsuperscript{(331.21)}
\textsuperscript{30:1.32} 2. Los Hijos Magistrales.

\par
%\textsuperscript{(331.22)}
\textsuperscript{30:1.33} 3. Las Radiantes Estrellas Matutinas.

\par
%\textsuperscript{(331.23)}
\textsuperscript{30:1.34} 4. Los Padres Melquisedeks.

\par
%\textsuperscript{(331.24)}
\textsuperscript{30:1.35} 5. Los Melquisedeks.

\par
%\textsuperscript{(331.25)}
\textsuperscript{30:1.36} 6. Los Vorondadeks.

\par
%\textsuperscript{(331.26)}
\textsuperscript{30:1.37} 7. Los Lanonandeks.

\par
%\textsuperscript{(331.27)}
\textsuperscript{30:1.38} 8. Las Brillantes Estrellas Vespertinas.

\par
%\textsuperscript{(331.28)}
\textsuperscript{30:1.39} 9. Los Arcángeles.

\par
%\textsuperscript{(331.29)}
\textsuperscript{30:1.40} 10. Los Portadores de Vida.

\par
%\textsuperscript{(331.30)}
\textsuperscript{30:1.41} 11. Los Ayudantes Universales No Revelados.

\par
%\textsuperscript{(331.31)}
\textsuperscript{30:1.42} 12. Los Hijos de Dios No Revelados.

\par
%\textsuperscript{(331.32)}
\textsuperscript{30:1.43} \textit{B. Las Órdenes Estacionarias.}

\par
%\textsuperscript{(331.33)}
\textsuperscript{30:1.44} 1. Los Abandontarios.

\par
%\textsuperscript{(331.34)}
\textsuperscript{30:1.45} 2. Los Susatias.

\par
%\textsuperscript{(331.35)}
\textsuperscript{30:1.46} 3. Los Univitatias.

\par
%\textsuperscript{(331.36)}
\textsuperscript{30:1.47} 4. Los Espirongas.

\par
%\textsuperscript{(331.37)}
\textsuperscript{30:1.48} 5. Los Seres de Origen Doble No Revelados.

\par
%\textsuperscript{(331.38)}
\textsuperscript{30:1.49} \textit{C. Las Órdenes Ascendentes.}

\par
%\textsuperscript{(331.39)}
\textsuperscript{30:1.50} 1. Los Mortales Fusionados con el Ajustador.

\par
%\textsuperscript{(331.40)}
\textsuperscript{30:1.51} 2. Los Mortales Fusionados con el Hijo.

\par
%\textsuperscript{(331.41)}
\textsuperscript{30:1.52} 3. Los Mortales Fusionados con el Espíritu.

\par
%\textsuperscript{(331.42)}
\textsuperscript{30:1.53} 4. Los Intermedios Trasladados.

\par
%\textsuperscript{(331.43)}
\textsuperscript{30:1.54} 5. Los Ascendentes No Revelados.

\par
%\textsuperscript{(332.1)}
\textsuperscript{30:1.55} III. \textit{SERES DE ORIGEN ÚNICO.} Aquellos que tienen su origen en una cualquiera de las Deidades del Paraíso o han sido creados de otra manera por un ser cualquiera que desciende directa o indirectamente de las Deidades del Paraíso.

\par
%\textsuperscript{(332.2)}
\textsuperscript{30:1.56} \textit{A. Los Espíritus Supremos.}

\par
%\textsuperscript{(332.3)}
\textsuperscript{30:1.57} 1. Los Mensajeros de Gravedad.

\par
%\textsuperscript{(332.4)}
\textsuperscript{30:1.58} 2. Los Siete Espíritus de los Circuitos de Havona.

\par
%\textsuperscript{(332.5)}
\textsuperscript{30:1.59} 3. Los Ayudantes Dodécuples de los Circuitos de Havona.

\par
%\textsuperscript{(332.6)}
\textsuperscript{30:1.60} 4. Los Ayudantes Reflectantes de Imágenes.

\par
%\textsuperscript{(332.7)}
\textsuperscript{30:1.61} 5. Los Espíritus Madres de los Universos.

\par
%\textsuperscript{(332.8)}
\textsuperscript{30:1.62} 6. Los Séptuples Espíritus Ayudantes de la Mente.

\par
%\textsuperscript{(332.9)}
\textsuperscript{30:1.63} 7. Los Seres No Revelados con Origen en la Deidad.

\par
%\textsuperscript{(332.10)}
\textsuperscript{30:1.64} \textit{B. Las Órdenes Ascendentes.}

\par
%\textsuperscript{(332.11)}
\textsuperscript{30:1.65} 1. Los Ajustadores Personalizados.

\par
%\textsuperscript{(332.12)}
\textsuperscript{30:1.66} 2. Los Hijos Materiales Ascendentes.

\par
%\textsuperscript{(332.13)}
\textsuperscript{30:1.67} 3. Los Serafines Evolutivos.

\par
%\textsuperscript{(332.14)}
\textsuperscript{30:1.68} 4. Los Querubines Evolutivos.

\par
%\textsuperscript{(332.15)}
\textsuperscript{30:1.69} 5. Los Ascendentes No Revelados.

\par
%\textsuperscript{(332.16)}
\textsuperscript{30:1.70} \textit{C. La Familia del Espíritu Infinito.}

\par
%\textsuperscript{(332.17)}
\textsuperscript{30:1.71} 1. Los Mensajeros Solitarios.

\par
%\textsuperscript{(332.18)}
\textsuperscript{30:1.72} 2. Los Supervisores de los Circuitos del Universo.

\par
%\textsuperscript{(332.19)}
\textsuperscript{30:1.73} 3. Los Directores del Censo.

\par
%\textsuperscript{(332.20)}
\textsuperscript{30:1.74} 4. Ayudantes Personales del Espíritu Infinito.

\par
%\textsuperscript{(332.21)}
\textsuperscript{30:1.75} 5. Los Inspectores Asociados.

\par
%\textsuperscript{(332.22)}
\textsuperscript{30:1.76} 6. Los Centinelas Asignados.

\par
%\textsuperscript{(332.23)}
\textsuperscript{30:1.77} 7. Los Guías de los Graduados.

\par
%\textsuperscript{(332.24)}
\textsuperscript{30:1.78} 8. Los Servitales de Havona.

\par
%\textsuperscript{(332.25)}
\textsuperscript{30:1.79} 9. Los Conciliadores Universales.

\par
%\textsuperscript{(332.26)}
\textsuperscript{30:1.80} 10. Los Compañeros Morontiales.

\par
%\textsuperscript{(332.27)}
\textsuperscript{30:1.81} 11. Los Supernafines.

\par
%\textsuperscript{(332.28)}
\textsuperscript{30:1.82} 12. Los Seconafines.

\par
%\textsuperscript{(332.29)}
\textsuperscript{30:1.83} 13. Los Terciafines.

\par
%\textsuperscript{(332.30)}
\textsuperscript{30:1.84} 14. Los Omniafines.

\par
%\textsuperscript{(332.31)}
\textsuperscript{30:1.85} 15. Los Serafines.

\par
%\textsuperscript{(332.32)}
\textsuperscript{30:1.86} 16. Los Querubines y los Sanobines.

\par
%\textsuperscript{(332.33)}
\textsuperscript{30:1.87} 17. Los Seres No Revelados con Origen en el Espíritu.

\par
%\textsuperscript{(332.34)}
\textsuperscript{30:1.88} 18. Los Siete Directores Supremos del Poder.

\par
%\textsuperscript{(332.35)}
\textsuperscript{30:1.89} 19. Los Centros Supremos del Poder.

\par
%\textsuperscript{(332.36)}
\textsuperscript{30:1.90} 20. Los Controladores Físicos Maestros.

\par
%\textsuperscript{(332.37)}
\textsuperscript{30:1.91} 21. Los Supervisores del Poder Morontial.

\par
%\textsuperscript{(332.38)}
\textsuperscript{30:1.92} IV. SERES TRASCENDENTALES EXISTENCIADOS. En el Paraíso se encuentra una inmensa multitud de seres trascendentales cuyo origen no se revela generalmente a los universos del tiempo y del espacio hasta que éstos no se establecen en la luz y la vida. Estos Trascendentales no son ni creadores ni criaturas; son los hijos existenciados de la divinidad, la ultimidad y la eternidad. Estos <<\textit{existenciados}>> no son ni finitos ni infinitos ---son absonitos; y la absonitidad no es ni la infinidad ni la absolutidad.

\par
%\textsuperscript{(333.1)}
\textsuperscript{30:1.93} Estos no creadores no creados son siempre leales a la Trinidad del Paraíso y obedecen al Último. Existen en cuatro niveles últimos de actividad de la personalidad y ejercen sus funciones en los siete niveles de lo absonito en doce grandes divisiones compuestas de mil grupos principales de trabajo de siete clases cada uno. Estos seres existenciados incluyen a las órdenes siguientes:

\par
%\textsuperscript{(333.2)}
\textsuperscript{30:1.94} 1. Los Arquitectos del Universo Maestro.

\par
%\textsuperscript{(333.3)}
\textsuperscript{30:1.95} 2. Los Registradores Trascendentales.

\par
%\textsuperscript{(333.4)}
\textsuperscript{30:1.96} 3. Otros Trascendentales.

\par
%\textsuperscript{(333.5)}
\textsuperscript{30:1.97} 4. Los Organizadores de la Fuerza Maestros Existenciados Primarios.

\par
%\textsuperscript{(333.6)}
\textsuperscript{30:1.98} 5. Los Organizadores de la Fuerza Maestros Trascendentales Asociados.

\par
%\textsuperscript{(333.7)}
\textsuperscript{30:1.99} Dios, como superpersona, existencia; Dios, como persona, crea; Dios, como prepersona, fragmenta; y este fragmento de sí mismo, el Ajustador, hace evolucionar el alma espiritual en la mente material y mortal de acuerdo con la libre elección de la personalidad que ha sido conferida a esa criatura mortal por el acto parental de Dios como Padre.

\par
%\textsuperscript{(333.8)}
\textsuperscript{30:1.100} V. \textit{ENTIDADES FRAGMENTADAS DE LA DEIDAD.} Esta orden de existencia viviente, que tiene su origen en el Padre Universal, tiene su mejor representación en los Ajustadores del Pensamiento, aunque estas entidades no son de ninguna manera las únicas fragmentaciones de la realidad prepersonal de la Fuente-Centro Primera. Las funciones de los fragmentos distintos a los Ajustadores son múltiples y poco conocidas. La fusión con un Ajustador o con otro fragmento de este tipo convierte a la criatura en un \textit{ser fusionado con elPadre.}

\par
%\textsuperscript{(333.9)}
\textsuperscript{30:1.101} Aunque las fragmentaciones del espíritu premental de la Fuente-Centro Tercera son difícilmente comparables con los fragmentos del Padre, debemos mencionarlas aquí. Estas entidades difieren enormemente de los Ajustadores; no residen como tales en Spiritington, ni atraviesan como tales los circuitos de la gravedad mental; tampoco habitan en las criaturas mortales durante la vida en la carne. No son prepersonales en el mismo sentido que los Ajustadores, pero estos fragmentos de espíritu premental son otorgados a algunos mortales sobrevivientes, y la fusión con ellos los convierte en \textit{mortales fusionados con elEspíritu,} en contraste con los mortales fusionados con el Ajustador.

\par
%\textsuperscript{(333.10)}
\textsuperscript{30:1.102} El espíritu individualizado de un Hijo Creador es aún más difícil de describir; la unión con él convierte a la criatura en un \textit{mortal fusionado con el Hijo.} Y existen además otras fragmentaciones de la Deidad.

\par
%\textsuperscript{(333.11)}
\textsuperscript{30:1.103} VI. \textit{SERES SUPERPERSONALES.} Hay una inmensa multitud de seres distintos a los personales que tienen un origen divino y que efectúan múltiples servicios en el universo de universos. Algunos de estos seres residen en los mundos paradisiacos del Hijo; otros, como los representantes superpersonales del Hijo Eterno, se encuentran en otros lugares. La mayor parte de ellos no se mencionan en estas narraciones, y sería totalmente inútil intentar describirlos a las criaturas \textit{personales.}

\par
%\textsuperscript{(333.12)}
\textsuperscript{30:1.104} VII. \textit{ÓRDENES NO CLASIFICADAS Y NO REVELADAS.} Durante la presente era del universo no sería posible incluir a todos los seres, personales o de otro tipo, dentro de unas clasificaciones que pertenecen a la presente era del universo; todas estas categorías tampoco han sido reveladas en estas narraciones; por eso se han omitido numerosas órdenes en estas listas. Considerad las siguientes:

\par
%\textsuperscript{(333.13)}
\textsuperscript{30:1.105} El Consumador del Destino del Universo.

\par
%\textsuperscript{(333.14)}
\textsuperscript{30:1.106} Los Vicegerentes Calificados del Último.

\par
%\textsuperscript{(334.1)}
\textsuperscript{30:1.107} Los Supervisores Incalificados del Supremo.

\par
%\textsuperscript{(334.2)}
\textsuperscript{30:1.108} Los Agentes Creativos No Revelados de los Ancianos de los Días.

\par
%\textsuperscript{(334.3)}
\textsuperscript{30:1.109} Majeston del Paraíso.

\par
%\textsuperscript{(334.4)}
\textsuperscript{30:1.110} Los Enlaces Reflectores Innominados de Majeston.

\par
%\textsuperscript{(334.5)}
\textsuperscript{30:1.111} Las Órdenes Midsonitas de los Universos Locales.

\par
%\textsuperscript{(334.6)}
\textsuperscript{30:1.112} No es necesario concederle una importancia especial al hecho de que estas órdenes se enumeren de forma conjunta, salvo que ninguna de ellas aparece en la clasificación paradisiaca tal como ésta se revela aquí. Éstas son las pocas no clasificadas; todavía os queda por conocer a las muchas no reveladas.

\par
%\textsuperscript{(334.7)}
\textsuperscript{30:1.113} Existen espíritus: entidades espirituales, presencias espirituales, espíritus personales, espíritus prepersonales, espíritus superpersonales, existencias espirituales, personalidades espirituales ---pero ni el lenguaje mortal ni el intelecto mortal son adecuados. Podemos afirmar sin embargo que no existen personalidades de <<\textit{mente pura}>>; ninguna entidad posee una personalidad a menos que esté dotada de ella por Dios, que es espíritu. Cualquier entidad mental que no esté asociada con la energía espiritual o física no es una personalidad. Pero en el mismo sentido en que hay personalidades espirituales que poseen una mente, existen personalidades mentales que poseen un espíritu. Majeston y sus asociados son unos ejemplos bastante buenos de unos seres dominados por la mente, pero existen mejores ejemplos de este tipo de personalidad desconocidos por vosotros. Hay incluso órdenes enteras no reveladas de estas \textit{personalidadesmentales,} pero siempre están asociadas al espíritu. Algunas otras criaturas no reveladas son lo que se podría llamar \textit{personalidades de energía mental y física.} Este tipo de ser no es sensible a la gravedad espiritual, pero sin embargo es una verdadera personalidad ---está dentro del circuito del Padre.

\par
%\textsuperscript{(334.8)}
\textsuperscript{30:1.114} Estos documentos ni siquiera empiezan a agotar ---no pueden hacerlo--- la historia de las criaturas vivientes, los creadores, los existenciadores y los seres que existen además de otras maneras, que viven, adoran y sirven en los universos pululantes del tiempo y en el universo central de la eternidad. Vosotros los mortales sois personas; por eso podemos describiros a los seres \textit{personalizados,} pero ¿cómo podríamos explicaros nunca qué es un ser \textit{absonitizado?}

\section*{2. El registro de personalidades existente en Uversa}
\par
%\textsuperscript{(334.9)}
\textsuperscript{30:2.1} La familia divina de seres vivientes está registrada en Uversa en siete grandes divisiones:

\par
%\textsuperscript{(334.10)}
\textsuperscript{30:2.2} 1. Las Deidades del Paraíso.

\par
%\textsuperscript{(334.11)}
\textsuperscript{30:2.3} 2. Los Espíritus Supremos.

\par
%\textsuperscript{(334.12)}
\textsuperscript{30:2.4} 3. Los Seres con Origen en la Trinidad.

\par
%\textsuperscript{(334.13)}
\textsuperscript{30:2.5} 4. Los Hijos de Dios.

\par
%\textsuperscript{(334.14)}
\textsuperscript{30:2.6} 5. Las Personalidades del Espíritu Infinito.

\par
%\textsuperscript{(334.15)}
\textsuperscript{30:2.7} 6. Los Directores del Poder Universal.

\par
%\textsuperscript{(334.16)}
\textsuperscript{30:2.8} 7. El Cuerpo de los Ciudadanos Permanentes.

\par
%\textsuperscript{(334.17)}
\textsuperscript{30:2.9} Estos grupos de criaturas volitivas están divididos en numerosas clases y subdivisiones menores. Sin embargo, la presentación de esta clasificación de personalidades del gran universo se interesa principalmente en exponer aquellas órdenes de seres inteligentes que han sido reveladas en estas narraciones, la mayoría de las cuales serán encontradas en la experiencia ascendente de los mortales del tiempo durante su elevación progresiva hacia el Paraíso. Las siguientes enumeraciones no mencionan las extensas órdenes de seres universales que efectúan su trabajo independientemente del programa de la ascensión de los mortales.

\par
%\textsuperscript{(335.1)}
\textsuperscript{30:2.10} I. \textit{LAS DEIDADES DEL PARAÍSO.}

\par
%\textsuperscript{(335.2)}
\textsuperscript{30:2.11} 1. El Padre Universal.

\par
%\textsuperscript{(335.3)}
\textsuperscript{30:2.12} 2. El Hijo Eterno.

\par
%\textsuperscript{(335.4)}
\textsuperscript{30:2.13} 3. El Espíritu Infinito.

\par
%\textsuperscript{(335.5)}
\textsuperscript{30:2.14} II. \textit{LOS ESPÍRITUS SUPREMOS.}

\par
%\textsuperscript{(335.6)}
\textsuperscript{30:2.15} 1. Los Siete Espíritus Maestros.

\par
%\textsuperscript{(335.7)}
\textsuperscript{30:2.16} 2. Los Siete Ejecutivos Supremos.

\par
%\textsuperscript{(335.8)}
\textsuperscript{30:2.17} 3. Los Siete Grupos de Espíritus Reflectantes.

\par
%\textsuperscript{(335.9)}
\textsuperscript{30:2.18} 4. Los Ayudantes Reflectantes de Imágenes.

\par
%\textsuperscript{(335.10)}
\textsuperscript{30:2.19} 5. Los Siete Espíritus de los Circuitos.

\par
%\textsuperscript{(335.11)}
\textsuperscript{30:2.20} 6. Los Espíritus Creativos de los Universos Locales.

\par
%\textsuperscript{(335.12)}
\textsuperscript{30:2.21} 7. Los Espíritus Ayudantes de la Mente.

\par
%\textsuperscript{(335.13)}
\textsuperscript{30:2.22} III. \textit{LOS SERES CON ORIGEN EN LA TRINIDAD.}

\par
%\textsuperscript{(335.14)}
\textsuperscript{30:2.23} 1. Los Secretos Trinitizados de la Supremacía.

\par
%\textsuperscript{(335.15)}
\textsuperscript{30:2.24} 2. Los Eternos de los Días.

\par
%\textsuperscript{(335.16)}
\textsuperscript{30:2.25} 3. Los Ancianos de los Días.

\par
%\textsuperscript{(335.17)}
\textsuperscript{30:2.26} 4. Los Perfecciones de los Días.

\par
%\textsuperscript{(335.18)}
\textsuperscript{30:2.27} 5. Los Recientes de los Días.

\par
%\textsuperscript{(335.19)}
\textsuperscript{30:2.28} 6. Los Uniones de los Días.

\par
%\textsuperscript{(335.20)}
\textsuperscript{30:2.29} 7. Los Fieles de los Días.

\par
%\textsuperscript{(335.21)}
\textsuperscript{30:2.30} 8. Los Hijos Instructores Trinitarios.

\par
%\textsuperscript{(335.22)}
\textsuperscript{30:2.31} 9. Los Perfeccionadores de la Sabiduría.

\par
%\textsuperscript{(335.23)}
\textsuperscript{30:2.32} 10. Los Consejeros Divinos.

\par
%\textsuperscript{(335.24)}
\textsuperscript{30:2.33} 11. Los Censores Universales.

\par
%\textsuperscript{(335.25)}
\textsuperscript{30:2.34} 12. Los Espíritus Inspirados Trinitarios.

\par
%\textsuperscript{(335.26)}
\textsuperscript{30:2.35} 13. Los Nativos de Havona.

\par
%\textsuperscript{(335.27)}
\textsuperscript{30:2.36} 14. Los Ciudadanos del Paraíso.

\par
%\textsuperscript{(335.28)}
\textsuperscript{30:2.37} IV. \textit{LOS HIJOS DE DIOS.}

\par
%\textsuperscript{(335.29)}
\textsuperscript{30:2.38} \textit{A. Hijos Descendentes.}

\par
%\textsuperscript{(335.30)}
\textsuperscript{30:2.39} 1. Los Hijos Creadores ---los Migueles.

\par
%\textsuperscript{(335.31)}
\textsuperscript{30:2.40} 2. Los Hijos Magistrales ---los Avonales.

\par
%\textsuperscript{(335.32)}
\textsuperscript{30:2.41} 3. Los Hijos Instructores Trinitarios ---los Daynales.

\par
%\textsuperscript{(335.33)}
\textsuperscript{30:2.42} 4. Los Hijos Melquisedeks.

\par
%\textsuperscript{(335.34)}
\textsuperscript{30:2.43} 5. Los Hijos Vorondadeks.

\par
%\textsuperscript{(335.35)}
\textsuperscript{30:2.44} 6. Los Hijos Lanonandeks.

\par
%\textsuperscript{(335.36)}
\textsuperscript{30:2.45} 7. Los Hijos Portadores de Vida.

\par
%\textsuperscript{(335.37)}
\textsuperscript{30:2.46} \textit{B. Hijos Ascendentes.}

\par
%\textsuperscript{(335.38)}
\textsuperscript{30:2.47} 1. Los Mortales Fusionados con el Padre.

\par
%\textsuperscript{(335.39)}
\textsuperscript{30:2.48} 2. Los Mortales Fusionados con el Hijo.

\par
%\textsuperscript{(335.40)}
\textsuperscript{30:2.49} 3. Los Mortales Fusionados con el Espíritu.

\par
%\textsuperscript{(335.41)}
\textsuperscript{30:2.50} 4. Los Serafines Evolutivos.

\par
%\textsuperscript{(335.42)}
\textsuperscript{30:2.51} 5. Los Hijos Materiales Ascendentes.

\par
%\textsuperscript{(335.43)}
\textsuperscript{30:2.52} 6. Los Intermedios Trasladados.

\par
%\textsuperscript{(335.44)}
\textsuperscript{30:2.53} 7. Los Ajustadores Personalizados.

\par
%\textsuperscript{(336.1)}
\textsuperscript{30:2.54} \textit{C. Hijos Trinitizados.}

\par
%\textsuperscript{(336.2)}
\textsuperscript{30:2.55} 1. Los Mensajeros Poderosos.

\par
%\textsuperscript{(336.3)}
\textsuperscript{30:2.56} 2. Los Elevados en Autoridad.

\par
%\textsuperscript{(336.4)}
\textsuperscript{30:2.57} 3. Los que no tienen Nombre ni Número.

\par
%\textsuperscript{(336.5)}
\textsuperscript{30:2.58} 4. Los Custodios Trinitizados.

\par
%\textsuperscript{(336.6)}
\textsuperscript{30:2.59} 5. Los Embajadores Trinitizados.

\par
%\textsuperscript{(336.7)}
\textsuperscript{30:2.60} 6. Los Guardianes Celestiales.

\par
%\textsuperscript{(336.8)}
\textsuperscript{30:2.61} 7. Los Ayudantes de los Hijos Elevados.

\par
%\textsuperscript{(336.9)}
\textsuperscript{30:2.62} 8. Los Hijos Trinitizados por los Ascendentes.

\par
%\textsuperscript{(336.10)}
\textsuperscript{30:2.63} 9. Los Hijos Trinitizados del Paraíso-Havona.

\par
%\textsuperscript{(336.11)}
\textsuperscript{30:2.64} 10. Los Hijos del Destino Trinitizados.

\par
%\textsuperscript{(336.12)}
\textsuperscript{30:2.65} V. \textit{PERSONALIDADES DEL ESPÍRITU INFINITO.}

\par
%\textsuperscript{(336.13)}
\textsuperscript{30:2.66} \textit{A. Personalidades Superiores del Espíritu Infinito.}

\par
%\textsuperscript{(336.14)}
\textsuperscript{30:2.67} 1. Mensajeros Solitarios.

\par
%\textsuperscript{(336.15)}
\textsuperscript{30:2.68} 2. Los Supervisores de los Circuitos del Universo.

\par
%\textsuperscript{(336.16)}
\textsuperscript{30:2.69} 3. Los Directores del Censo.

\par
%\textsuperscript{(336.17)}
\textsuperscript{30:2.70} 4. Los Ayudantes Personales del Espíritu Infinito.

\par
%\textsuperscript{(336.18)}
\textsuperscript{30:2.71} 5. Los Inspectores Asociados.

\par
%\textsuperscript{(336.19)}
\textsuperscript{30:2.72} 6. Los Centinelas Asignados.

\par
%\textsuperscript{(336.20)}
\textsuperscript{30:2.73} 7. Los Guías de los Graduados.

\par
%\textsuperscript{(336.21)}
\textsuperscript{30:2.74} \textit{B. Las Huestes de Mensajeros del Espacio.}

\par
%\textsuperscript{(336.22)}
\textsuperscript{30:2.75} 1. Los Servitales de Havona.

\par
%\textsuperscript{(336.23)}
\textsuperscript{30:2.76} 2. Los Conciliadores Universales.

\par
%\textsuperscript{(336.24)}
\textsuperscript{30:2.77} 3. Los Asesores Técnicos.

\par
%\textsuperscript{(336.25)}
\textsuperscript{30:2.78} 4. Los Custodios de los Registros del Paraíso.

\par
%\textsuperscript{(336.26)}
\textsuperscript{30:2.79} 5. Los Registradores Celestiales.

\par
%\textsuperscript{(336.27)}
\textsuperscript{30:2.80} 6. Los Compañeros Morontiales.

\par
%\textsuperscript{(336.28)}
\textsuperscript{30:2.81} 7. Los Compañeros Paradisiacos.

\par
%\textsuperscript{(336.29)}
\textsuperscript{30:2.82} \textit{C. Los Espíritus Ministrantes.}

\par
%\textsuperscript{(336.30)}
\textsuperscript{30:2.83} 1. Los Supernafines.

\par
%\textsuperscript{(336.31)}
\textsuperscript{30:2.84} 2. Los Seconafines.

\par
%\textsuperscript{(336.32)}
\textsuperscript{30:2.85} 3. Los Terciafines.

\par
%\textsuperscript{(336.33)}
\textsuperscript{30:2.86} 4. Los Omniafines.

\par
%\textsuperscript{(336.34)}
\textsuperscript{30:2.87} 5. Los Serafines.

\par
%\textsuperscript{(336.35)}
\textsuperscript{30:2.88} 6. Los Querubines y los Sanobines.

\par
%\textsuperscript{(336.36)}
\textsuperscript{30:2.89} 7. Los Intermedios.

\par
%\textsuperscript{(336.37)}
\textsuperscript{30:2.90} VI. \textit{LOS DIRECTORES DEL PODER UNIVERSAL.}

\par
%\textsuperscript{(336.38)}
\textsuperscript{30:2.91} \textit{A. Los Siete Directores Supremos del Poder.}

\par
%\textsuperscript{(336.39)}
\textsuperscript{30:2.92} \textit{B. Los Centros Supremos del Poder.}

\par
%\textsuperscript{(336.40)}
\textsuperscript{30:2.93} 1. Los Supervisores Supremos de los Centros.

\par
%\textsuperscript{(336.41)}
\textsuperscript{30:2.94} 2. Los Centros de Havona.

\par
%\textsuperscript{(336.42)}
\textsuperscript{30:2.95} 3. Los Centros de los Superuniversos.

\par
%\textsuperscript{(336.43)}
\textsuperscript{30:2.96} 4. Los Centros de los Universos Locales.

\par
%\textsuperscript{(336.44)}
\textsuperscript{30:2.97} 5. Los Centros de las Constelaciones.

\par
%\textsuperscript{(336.45)}
\textsuperscript{30:2.98} 6. Los Centros de los Sistemas.

\par
%\textsuperscript{(336.46)}
\textsuperscript{30:2.99} 7. Los Centros No Clasificados.

\par
%\textsuperscript{(337.1)}
\textsuperscript{30:2.100} \textit{C. Los Controladores Físicos Maestros.}

\par
%\textsuperscript{(337.2)}
\textsuperscript{30:2.101} 1. Los Directores Asociados del Poder.

\par
%\textsuperscript{(337.3)}
\textsuperscript{30:2.102} 2. Los Controladores Maquinales.

\par
%\textsuperscript{(337.4)}
\textsuperscript{30:2.103} 3. Los Transformadores de la Energía.

\par
%\textsuperscript{(337.5)}
\textsuperscript{30:2.104} 4. Los Transmisores de la Energía.

\par
%\textsuperscript{(337.6)}
\textsuperscript{30:2.105} 5. Los Asociadores Primarios.

\par
%\textsuperscript{(337.7)}
\textsuperscript{30:2.106} 6. Los Disociadores Secundarios.

\par
%\textsuperscript{(337.8)}
\textsuperscript{30:2.107} 7. Los Frandalanks y los Cronoldeks.

\par
%\textsuperscript{(337.9)}
\textsuperscript{30:2.108} \textit{D. Los Supervisores del Poder Morontial.}

\par
%\textsuperscript{(337.10)}
\textsuperscript{30:2.109} 1. Los Reguladores de los Circuitos.

\par
%\textsuperscript{(337.11)}
\textsuperscript{30:2.110} 2. Los Coordinadores de los Sistemas.

\par
%\textsuperscript{(337.12)}
\textsuperscript{30:2.111} 3. Los Custodios Planetarios.

\par
%\textsuperscript{(337.13)}
\textsuperscript{30:2.112} 4. Los Controladores Combinados.

\par
%\textsuperscript{(337.14)}
\textsuperscript{30:2.113} 5. Los Estabilizadores de las Conexiones.

\par
%\textsuperscript{(337.15)}
\textsuperscript{30:2.114} 6. Los Clasificadores Selectivos.

\par
%\textsuperscript{(337.16)}
\textsuperscript{30:2.115} 7. Los Registradores Asociados.

\par
%\textsuperscript{(337.17)}
\textsuperscript{30:2.116} VII. \textit{EL CUERPO DE CIUDADANOS PERMANENTES.}

\par
%\textsuperscript{(337.18)}
\textsuperscript{30:2.117} 1. Los Intermedios Planetarios.

\par
%\textsuperscript{(337.19)}
\textsuperscript{30:2.118} 2. Los Hijos Adámicos de los Sistemas.

\par
%\textsuperscript{(337.20)}
\textsuperscript{30:2.119} 3. Los Univitatias de las Constelaciones.

\par
%\textsuperscript{(337.21)}
\textsuperscript{30:2.120} 4. Los Susatias de los Universos Locales.

\par
%\textsuperscript{(337.22)}
\textsuperscript{30:2.121} 5. Los Mortales de los Universos Locales Fusionados con el Espíritu.

\par
%\textsuperscript{(337.23)}
\textsuperscript{30:2.122} 6. Los Abandontarios de los Superuniversos.

\par
%\textsuperscript{(337.24)}
\textsuperscript{30:2.123} 7. Los Mortales de los Superuniversos Fusionados con el Hijo.

\par
%\textsuperscript{(337.25)}
\textsuperscript{30:2.124} 8. Los Nativos de Havona.

\par
%\textsuperscript{(337.26)}
\textsuperscript{30:2.125} 9. Los Nativos de las Esferas Paradisiacas del Espíritu.

\par
%\textsuperscript{(337.27)}
\textsuperscript{30:2.126} 10. Los Nativos de las Esferas Paradisiacas del Padre.

\par
%\textsuperscript{(337.28)}
\textsuperscript{30:2.127} 11. Los Ciudadanos Creados del Paraíso.

\par
%\textsuperscript{(337.29)}
\textsuperscript{30:2.128} 12. Los Ciudadanos Mortales del Paraíso Fusionados con el Ajustador.

\par
%\textsuperscript{(337.30)}
\textsuperscript{30:2.129} Ésta es la clasificación básica de las personalidades de los universos tal como están registradas en el mundo sede de Uversa.

\par
%\textsuperscript{(337.31)}
\textsuperscript{30:2.130} \textit{LOS GRUPOS DE PERSONALIDADES COMPUESTAS.} En Uversa se encuentran los registros de numerosos grupos adicionales de seres inteligentes, de seres que están también estrechamente relacionados con la organización y la administración del gran universo. Entre estas órdenes figuran los tres grupos siguientes de personalidades compuestas:

\par
%\textsuperscript{(337.32)}
\textsuperscript{30:2.131} \textit{A. El Cuerpo Paradisiaco de la Finalidad.}

\par
%\textsuperscript{(337.33)}
\textsuperscript{30:2.132} 1. El Cuerpo de los Finalitarios Mortales.

\par
%\textsuperscript{(337.34)}
\textsuperscript{30:2.133} 2. El Cuerpo de los Finalitarios Paradisiacos.

\par
%\textsuperscript{(337.35)}
\textsuperscript{30:2.134} 3. El Cuerpo de los Finalitarios Trinitizados.

\par
%\textsuperscript{(337.36)}
\textsuperscript{30:2.135} 4. El Cuerpo de los Finalitarios Trinitizados Conjuntos.

\par
%\textsuperscript{(337.37)}
\textsuperscript{30:2.136} 5. El Cuerpo de los Finalitarios Havonianos.

\par
%\textsuperscript{(337.38)}
\textsuperscript{30:2.137} 6. El Cuerpo de los Finalitarios Trascendentales.

\par
%\textsuperscript{(337.39)}
\textsuperscript{30:2.138} 7. El Cuerpo de los Hijos del Destino No Revelados.

\par
%\textsuperscript{(337.40)}
\textsuperscript{30:2.139} El Cuerpo de los Mortales de la Finalidad será tratado en el próximo y último documento de esta serie.

\par
%\textsuperscript{(338.1)}
\textsuperscript{30:2.140} \textit{B. Los Ayudantes Universales.}

\par
%\textsuperscript{(338.2)}
\textsuperscript{30:2.141} 1. Las Radiantes Estrellas Matutinas.

\par
%\textsuperscript{(338.3)}
\textsuperscript{30:2.142} 2. Las Brillantes Estrellas Vespertinas.

\par
%\textsuperscript{(338.4)}
\textsuperscript{30:2.143} 3. Los Arcángeles.

\par
%\textsuperscript{(338.5)}
\textsuperscript{30:2.144} 4. Los Asistentes Altísimos.

\par
%\textsuperscript{(338.6)}
\textsuperscript{30:2.145} 5. Los Altos Comisionados.

\par
%\textsuperscript{(338.7)}
\textsuperscript{30:2.146} 6. Los Supervisores Celestiales.

\par
%\textsuperscript{(338.8)}
\textsuperscript{30:2.147} 7. Los Educadores de los Mundos de las Mansiones.

\par
%\textsuperscript{(338.9)}
\textsuperscript{30:2.148} En todos los mundos sede de los universos locales y de los superuniversos se prevén estos seres que se ocupan de misiones específicas para los Hijos Creadores, los gobernantes de los universos locales. En Uversa acogemos a estos \textit{Ayudantes Universales,} pero no tenemos jurisdicción sobre ellos. Estos emisarios efectúan su trabajo y llevan adelante sus observaciones bajo la autoridad de los Hijos Creadores. Sus actividades se describen más plenamente en la historia de vuestro universo local.

\par
%\textsuperscript{(338.10)}
\textsuperscript{30:2.149} \textit{C. Las Siete Colonias de Cortesía.}

\par
%\textsuperscript{(338.11)}
\textsuperscript{30:2.150} 1. Los Estudiantes de Estrellas.

\par
%\textsuperscript{(338.12)}
\textsuperscript{30:2.151} 2. Los Artesanos Celestiales.

\par
%\textsuperscript{(338.13)}
\textsuperscript{30:2.152} 3. Los Directores de la Reversión.

\par
%\textsuperscript{(338.14)}
\textsuperscript{30:2.153} 4. Los Instructores de las Facultades Anexas.

\par
%\textsuperscript{(338.15)}
\textsuperscript{30:2.154} 5. Los Diversos Cuerpos de Reserva.

\par
%\textsuperscript{(338.16)}
\textsuperscript{30:2.155} 6. Los Visitantes Estudiantiles.

\par
%\textsuperscript{(338.17)}
\textsuperscript{30:2.156} 7. Los Peregrinos Ascendentes.

\par
%\textsuperscript{(338.18)}
\textsuperscript{30:2.157} A estos siete grupos de seres los encontraréis organizados y gobernados así en todos los mundos sede, desde los sistemas locales hasta las capitales de los superuniversos, sobre todo en estas últimas. Las capitales de los siete superuniversos son los lugares de encuentro de casi todas las clases y órdenes de seres inteligentes. A excepción de numerosos grupos del Paraíso-Havona, aquí se pueden observar y estudiar a las criaturas volitivas de todas las fases de existencia.

\section*{3. Las colonias de Cortesía}
\par
%\textsuperscript{(338.19)}
\textsuperscript{30:3.1} Las siete colonias de cortesía residen en las esferas arquitectónicas durante un período de tiempo más o menos prolongado mientras se dedican a fomentar sus misiones y a ejecutar sus tareas especiales. Su trabajo se puede describir como sigue:

\par
%\textsuperscript{(338.20)}
\textsuperscript{30:3.2} 1. \textit{Los Estudiantes de Estrellas,} los astrónomos celestiales, eligen trabajar en esferas como Uversa porque estos mundos especialmente construidos son extraordinariamente favorables para sus observaciones y sus cálculos. Uversa está favorablemente situada para el trabajo de esta colonia, no sólo debido a su emplazamiento central, sino también porque no hay gigantescos soles cercanos vivos o muertos que perturben las corrientes de energía. Estos estudiantes no están conectados orgánicamente de ninguna manera con los asuntos del superuniverso; son simplemente invitados.

\par
%\textsuperscript{(338.21)}
\textsuperscript{30:3.3} La colonia astronómica de Uversa contiene individuos que proceden de numerosos reinos cercanos, del universo central, e incluso de Norlatiadek. Cualquier ser de cualquier mundo de cualquier sistema de cualquier universo puede convertirse en un estudiante de estrellas, puede aspirar a unirse a algún cuerpo de astrónomos celestiales. Los únicos requisitos son: una vida prolongada y un conocimiento suficiente de los mundos del espacio, especialmente de sus leyes físicas de evolución y de control. A los estudiantes de estrellas no se les exige que sirvan eternamente en este cuerpo, pero nadie que ha sido admitido en este grupo puede retirarse antes de un milenio del tiempo de Uversa.

\par
%\textsuperscript{(339.1)}
\textsuperscript{30:3.4} La colonia de observadores de estrellas de Uversa asciende actualmente a más de un millón de seres. Estos astrónomos van y vienen, aunque algunos se quedan durante períodos relativamente largos. Realizan su trabajo con la ayuda de una multitud de instrumentos mecánicos y de aparatos físicos; también reciben mucha ayuda de los Mensajeros Solitarios y de otros exploradores espirituales. En su trabajo de estudiar las estrellas y de examinar el espacio, estos astrónomos celestiales utilizan constantemente a los transformadores y a los transmisores vivientes de la energía, así como a las personalidades reflectantes. Estudian todas las formas y fases de la materia espacial y de las manifestaciones energéticas, y están tan interesados en la función de la fuerza como en los fenómenos estelares; nada en todo el espacio escapa a su examen.

\par
%\textsuperscript{(339.2)}
\textsuperscript{30:3.5} Unas colonias similares de astrónomos se encuentran también en los mundos sede de los sectores del superuniverso así como en las capitales arquitectónicas de los universos locales y en sus subdivisiones administrativas. Salvo en el Paraíso, el conocimiento no es inherente; la comprensión del universo físico depende ampliamente de la observación y de la investigación.

\par
%\textsuperscript{(339.3)}
\textsuperscript{30:3.6} 2. \textit{Los Artesanos Celestiales} sirven en todas las partes de los siete superuniversos. Los mortales ascendentes tienen su contacto inicial con estos grupos durante la carrera morontial en el universo local, en relación con la cual analizaremos más ampliamente a estos artesanos.

\par
%\textsuperscript{(339.4)}
\textsuperscript{30:3.7} 3. \textit{Los Directores de la Reversión} son los promotores de las distracciones y del humor ---del retorno a los recuerdos del pasado. Prestan un gran servicio en el funcionamiento práctico del programa ascendente de la progresión humana, especialmente durante las fases iniciales de la transición morontial y de la experiencia espiritual. Su historia pertenece a la narración de la carrera de los mortales en el universo local.

\par
%\textsuperscript{(339.5)}
\textsuperscript{30:3.8} 4. \textit{Los Instructores de las Facultades Anexas.} El mundo residencial inmediatamente superior de la carrera ascendente siempre mantiene un importante cuerpo de educadores en el mundo que se encuentra justamente por debajo, una especie de escuela preparatoria para los residentes que progresan en esa esfera; se trata de una fase del programa ascendente para hacer avanzar a los peregrinos del tiempo. Estas escuelas, sus métodos de instrucción y de exámenes, son totalmente diferentes a todo lo que intentáis llevar a cabo en Urantia.

\par
%\textsuperscript{(339.6)}
\textsuperscript{30:3.9} Todo el plan ascendente de la progresión de los mortales está caracterizado por la práctica de transmitir a otros seres las nuevas verdades y experiencias tan pronto como se han adquirido. Os abrís camino a través de la larga escuela que conduce a alcanzar el Paraíso sirviendo como maestros a aquellos alumnos que se encuentran inmediatamente detrás de vosotros en la escala de la progresión.

\par
%\textsuperscript{(339.7)}
\textsuperscript{30:3.10} 5. \textit{Los Diversos Cuerpos de Reserva.} Unas inmensas reservas de seres que no están bajo nuestra supervisión inmediata son movilizados en Uversa como colonia de los cuerpos de reserva. En Uversa hay setenta divisiones primarias de esta colonia, y el permitiros pasar una temporada con estas personalidades extraordinarias constituye una educación liberal. En Salvington y en otras capitales universales se mantienen unas reservas generales similares; y son enviadas al servicio activo a petición de los directores de sus grupos respectivos.

\par
%\textsuperscript{(339.8)}
\textsuperscript{30:3.11} 6. \textit{Los Visitantes Estudiantiles.} Un caudal constante de visitantes celestiales procedentes de todo el universo fluye hacia los diversos mundos sede. Como individuos y como clases, estos diversos tipos de seres acuden en tropel hacia nosotros como observadores, alumnos de intercambio y ayudantes estudiantiles. En Uversa hay actualmente más de mil millones de personas en esta colonia de cortesía. Algunos de estos visitantes pueden quedarse un día, otros pueden permanecer un año, todo depende de la naturaleza de su misión. Esta colonia contiene representantes de casi todas las clases de seres del universo, a excepción de las personalidades Creadoras y de los mortales morontiales.

\par
%\textsuperscript{(340.1)}
\textsuperscript{30:3.12} Los mortales morontiales sólo son visitantes estudiantiles dentro de los confines del universo local de su origen. Sólo pueden hacer visitas en calidad superuniversal después de haber alcanzado el estado de espíritus. Una mitad por lo menos de nuestra colonia de visitantes está compuesta de <<\textit{viajeros de paso}>>, de seres que están de camino hacia otros lugares y que se detienen para visitar la capital de Orvonton. Estas personalidades pueden estar realizando una tarea universal o estar disfrutando de un período de ocio ---de exención de funciones. El privilegio del viaje y de la observación intrauniversales forma parte de la carrera de todos los seres ascendentes. El deseo humano de viajar y de observar nuevos pueblos y nuevos mundos será plenamente satisfecho durante la larga y agitada ascensión hacia el Paraíso a través del universo local, el superuniverso y el universo central.

\par
%\textsuperscript{(340.2)}
\textsuperscript{30:3.13} 7. \textit{Los Peregrinos Ascendentes.} Cuando los peregrinos ascendentes son destinados a diversos servicios en combinación con su progresión hacia el Paraíso, se les domicilia como colonia de cortesía en las diversas esferas sede. Estos grupos son ampliamente autónomos mientras ejercen su actividad aquí y allá en todo un superuniverso. Constituyen una colonia en constante cambio que abarca todas las órdenes de mortales evolutivos con sus asociados ascendentes.

\section*{4. Los mortales ascendentes}
\par
%\textsuperscript{(340.3)}
\textsuperscript{30:4.1} Aunque los supervivientes mortales del tiempo y del espacio se denominan \textit{peregrinos ascendentes} cuando están acreditados para la ascensión progresiva hacia el Paraíso, estas criaturas evolutivas ocupan un lugar tan importante en estas narraciones que deseamos presentar aquí una sinopsis de las siete etapas siguientes de la carrera universal ascendente:

\par
%\textsuperscript{(340.4)}
\textsuperscript{30:4.2} 1. Los Mortales Planetarios.

\par
%\textsuperscript{(340.5)}
\textsuperscript{30:4.3} 2. Los Supervivientes Dormidos.

\par
%\textsuperscript{(340.6)}
\textsuperscript{30:4.4} 3. Los Estudiantes de los Mundos de las Mansiones.

\par
%\textsuperscript{(340.7)}
\textsuperscript{30:4.5} 4. Los Progresores Morontiales.

\par
%\textsuperscript{(340.8)}
\textsuperscript{30:4.6} 5. Los Pupilos Superuniversales.

\par
%\textsuperscript{(340.9)}
\textsuperscript{30:4.7} 6. Los Peregrinos en Havona.

\par
%\textsuperscript{(340.10)}
\textsuperscript{30:4.8} 7. Los que llegan al Paraíso.

\par
%\textsuperscript{(340.11)}
\textsuperscript{30:4.9} La siguiente narración presenta la carrera universal de un mortal habitado por un Ajustador. Los mortales fusionados con el Hijo o con el Espíritu comparten ciertas partes de esta carrera, pero hemos elegido contar esta historia tal como está relacionada con los mortales fusionados con el Ajustador, porque éste es el destino que pueden esperar todas las razas humanas de Urantia.

\par
%\textsuperscript{(340.12)}
\textsuperscript{30:4.10} 1. \textit{Los Mortales Planetarios.} Todos los mortales son seres evolutivos de origen animal con un potencial ascendente. En su origen, su naturaleza y su destino, estos diversos grupos y tipos de seres humanos no son enteramente diferentes a los pueblos de Urantia. Las razas humanas de cada mundo reciben el mismo ministerio de los Hijos de Dios y disfrutan de la presencia de los espíritus ministrantes del tiempo. Después de la muerte natural, todos los tipos de ascendentes fraternizan como una sola familia morontial en los mundos de las mansiones.

\par
%\textsuperscript{(341.1)}
\textsuperscript{30:4.11} 2. \textit{Los Supervivientes Dormidos.} Todos los mortales que tienen el estado de supervivencia y que están bajo la custodia de los guardianes personales del destino pasan por las puertas de la muerte natural y se personalizan en los mundos de las mansiones al tercer período. Aquellos seres acreditados que han sido incapaces de alcanzar, por alguna razón, este nivel de dominio de la inteligencia y de dotación de espiritualidad que les daría derecho a tener unos guardianes personales, no pueden ir así directa e inmediatamente a los mundos de las mansiones. Estas almas supervivientes deben permanecer en un sueño inconsciente hasta el día del juicio de una nueva época, de una nueva dispensación, de la llegada de un Hijo de Dios que realizará el llamamiento nominal de la era y juzgará el reino, y ésta es la práctica general que se sigue en todo Nebadon. Se ha dicho de Cristo Miguel que, cuando ascendió a las alturas al final de su trabajo en la Tierra: <<\textit{Conducía a una gran multitud de cautivos}>>\footnote{\textit{Conducía a una gran multitud}: Mt 27:52; Ef 4:8.}. Estos cautivos eran los supervivientes dormidos desde los tiempos de Adán hasta el día de la resurrección del Maestro en Urantia.

\par
%\textsuperscript{(341.2)}
\textsuperscript{30:4.12} El paso del tiempo no tiene ninguna importancia para los mortales dormidos; están totalmente inconscientes y ajenos a la duración de su descanso. En el momento de reensamblarse su personalidad al final de una era, aquellos que han dormido cinco mil años no reaccionan de manera diferente a los que han descansado cinco días. Aparte de este retraso en el tiempo, estos supervivientes pasan por el régimen de la ascensión exactamente igual que aquellos que evitan el sueño más corto o más largo de la muerte.

\par
%\textsuperscript{(341.3)}
\textsuperscript{30:4.13} Estas clases dispensacionales de peregrinos de los mundos se utilizan para las actividades morontiales de grupo en el trabajo de los universos locales. La movilización de estos enormes grupos tiene una gran ventaja; así se les mantiene unidos durante largos períodos de servicio efectivo.

\par
%\textsuperscript{(341.4)}
\textsuperscript{30:4.14} 3. \textit{Los Estudiantes de los Mundos de las Mansiones.} Todos los mortales supervivientes que se vuelven a despertar en los mundos de las mansiones pertenecen a esta clase.

\par
%\textsuperscript{(341.5)}
\textsuperscript{30:4.15} El cuerpo físico de carne mortal no forma parte del reensamblaje del superviviente dormido; el cuerpo físico ha regresado al polvo. El serafín asignado patrocina el nuevo cuerpo, la forma morontial, como nuevo vehículo de vida para el alma inmortal y para ser habitado por el Ajustador que ha regresado. El Ajustador es el custodio de la transcripción espiritual de la mente del superviviente dormido. El serafín asignado es el guardián de la identidad sobreviviente ---del alma inmortal--- hasta el nivel que haya evolucionado. Y cuando los dos, el Ajustador y el serafín, reúnen los elementos de la personalidad confiados a su cargo, el nuevo individuo completa la resurrección de la antigua personalidad, la supervivencia de la identidad evolutiva morontial del alma. Esta reasociación de un alma y de un Ajustador se denomina de manera muy apropiada resurrección, un reensamblaje de los factores de la personalidad; pero incluso esto no explica plenamente la reaparición de la \textit{personalidad} sobreviviente. Aunque es probable que nunca comprenderéis el hecho de esta operación inexplicable, alguna vez conoceréis por experiencia la verdad de esto si no rechazáis el plan de la supervivencia humana.

\par
%\textsuperscript{(341.6)}
\textsuperscript{30:4.16} El plan de detener inicialmente a los mortales en los siete mundos de formación progresiva es casi universal en Orvonton. En cada sistema local de unos mil planetas habitados hay siete mundos de las mansiones, generalmente satélites o subsatélites de la capital del sistema. Son los mundos donde se recibe a la mayoría de los mortales ascendentes.

\par
%\textsuperscript{(341.7)}
\textsuperscript{30:4.17} A veces todos los mundos educativos donde residen los mortales se llaman <<\textit{mansiones}>> del universo, y es a estas esferas a las que Jesús aludió cuando dijo: <<\textit{En la casa de mi Padre hay muchas moradas}>>\footnote{\textit{Muchas mansiones}: Jn 14:2.}. A partir de aquí, dentro de un grupo dado de esferas como los mundos de las mansiones, los ascendentes progresarán individualmente de una esfera a otra y de una fase de vida a otra, pero siempre avanzarán en formación de clase de una etapa de estudio universal a otra.

\par
%\textsuperscript{(342.1)}
\textsuperscript{30:4.18} 4. \textit{Los Progresores Morontiales.} Desde los mundos de las mansiones hacia arriba, a través de las esferas del sistema, la constelación y el universo, los mortales son clasificados como progresores morontiales; atraviesan las esferas de transición de la ascensión mortal. A medida que los mortales ascendentes progresan desde los mundos morontiales inferiores hasta los más superiores, sirven en innumerables tareas en asociación con sus educadores y en compañía de sus hermanos mayores más avanzados.

\par
%\textsuperscript{(342.2)}
\textsuperscript{30:4.19} La progresión morontial está relacionada con el avance continuo del intelecto, del espíritu y de la forma de la personalidad. Los supervivientes siguen siendo seres de naturaleza triple. Durante toda la experiencia morontial son los pupilos del universo local. El régimen del superuniverso no se aplica hasta que no empieza la carrera espiritual.

\par
%\textsuperscript{(342.3)}
\textsuperscript{30:4.20} Los mortales adquieren una verdadera identidad espiritual justo antes de dejar la sede del universo local para trasladarse a los mundos receptores de los sectores menores del superuniverso. El paso de la etapa morontial final al estado espiritual inicial, o más bajo, no es más que una pequeña transición. La mente, la personalidad y el carácter permanecen invariables con este avance; sólo la forma sufre una modificación. Pero la forma espiritual es tan real como el cuerpo morontial, y es igual de perceptible.

\par
%\textsuperscript{(342.4)}
\textsuperscript{30:4.21} Antes de partir de sus universos locales nativos hacia los mundos receptores del superuniverso, los mortales del tiempo reciben la confirmación espiritual del Hijo Creador y del Espíritu Madre del universo local. A partir de este punto, el estado del mortal ascendente queda establecido para siempre. Nunca se ha sabido que los pupilos del superuniverso se hayan descarriado. La categoría angélica de los serafines ascendentes también se eleva en el momento en que salen de los universos locales.

\par
%\textsuperscript{(342.5)}
\textsuperscript{30:4.22} 5. \textit{Los Pupilos del Superuniverso.} Todos los ascendentes que llegan a los mundos educativos de los superuniversos se convierten en los pupilos de los Ancianos de los Días; han atravesado la vida morontial del universo local y ahora son espíritus acreditados. Como jóvenes espíritus, empiezan la ascensión del sistema superuniversal de formación y de cultura que se extiende desde las esferas receptoras de su sector menor, pasando hacia el interior a través de los mundos de estudio de los diez sectores mayores, y continuando hasta las esferas culturales superiores de la sede del superuniverso.

\par
%\textsuperscript{(342.6)}
\textsuperscript{30:4.23} Hay tres órdenes de espíritus estudiantes según residan en el sector menor, en los sectores mayores o en los mundos sede de progresión espiritual del superuniverso. Al igual que los ascendentes morontiales estudiaban y trabajaban en los mundos del universo local, los ascendentes espirituales continúan dominando nuevos mundos mientras practican el transmitir a otros aquello que han bebido en las fuentes experienciales de la sabiduría. Pero ir a la escuela como un ser espiritual en la carrera superuniversal es muy diferente a cualquier cosa que haya penetrado nunca en los reinos imaginativos de la mente material del hombre.

\par
%\textsuperscript{(342.7)}
\textsuperscript{30:4.24} Antes de partir del superuniverso para dirigirse a Havona, estos espíritus ascendentes reciben, en materia de administración superuniversal, el mismo curso minucioso que habían recibido sobre la supervisión del universo local durante su experiencia morontial. Antes de que los mortales espirituales lleguen a Havona, su estudio principal consiste en el dominio de la administración del universo local y del superuniverso, pero ésta no es su ocupación exclusiva. La razón de toda esta experiencia no es en la actualidad plenamente evidente, pero no hay duda de que este entrenamiento es sabio y necesario considerando su posible destino futuro como miembros del Cuerpo de la Finalidad.

\par
%\textsuperscript{(342.8)}
\textsuperscript{30:4.25} El régimen superuniversal no es el mismo para todos los mortales ascendentes. Reciben la misma educación general, pero hay grupos y clases especiales que realizan cursos especiales de instrucción y pasan por cursos específicos de formación.

\par
%\textsuperscript{(343.1)}
\textsuperscript{30:4.26} 6. \textit{Los Peregrinos en Havona.} Cuando el desarrollo espiritual es completo, aunque no sea total, el mortal sobreviviente se prepara para el largo vuelo hacia Havona, el puerto de los espíritus evolutivos. En la Tierra erais criaturas de carne y hueso; en todo el universo local erais seres morontiales; a lo largo del superuniverso erais espíritus en evolución; con vuestra llegada a los mundos receptores de Havona, vuestra educación espiritual empieza en serio y de verdad; vuestra aparición final en el Paraíso será como espíritus perfeccionados.

\par
%\textsuperscript{(343.2)}
\textsuperscript{30:4.27} El viaje desde la sede del superuniverso hasta las esferas receptoras de Havona siempre se hace en solitario. Desde ahora en adelante ya no se recibirá más enseñanza en clases o en grupos. Ya habéis pasado por la formación técnica y administrativa de los mundos evolutivos del tiempo y del espacio. Ahora empieza vuestra \textit{educación personal}, vuestra formación individual espiritual. Desde el principio hasta el fin, a lo largo de todo Havona, la enseñanza es personal y de naturaleza triple: intelectual, espiritual y experiencial.

\par
%\textsuperscript{(343.3)}
\textsuperscript{30:4.28} El primer acto de vuestra carrera en Havona será reconocer y agradecer a vuestro seconafín transportador el viaje largo y seguro. Luego seréis presentados a aquellos seres que patrocinarán vuestras primeras actividades en Havona. A continuación iréis a registrar vuestra llegada y prepararéis vuestro mensaje de acción de gracias y de adoración que será enviado al Hijo Creador de vuestro universo local, el Padre del universo que ha hecho posible vuestra carrera de filiación. Esto concluye las formalidades de la llegada a Havona; después de esto se os concede un largo período de ocio para observar libremente, y esto os proporciona la oportunidad de ir a visitar a vuestros amigos, compañeros y asociados de la larga experiencia de la ascensión. Podéis consultar también las transmisiones para averiguar quiénes son los compañeros peregrinos vuestros que han partido hacia Havona desde el momento en que dejasteis Uversa.

\par
%\textsuperscript{(343.4)}
\textsuperscript{30:4.29} El hecho de vuestra llegada a los mundos receptores de Havona se transmitirá debidamente a la sede de vuestro universo local y se comunicará personalmente a vuestro guardián seráfico, dondequiera que se encuentre ese serafín.

\par
%\textsuperscript{(343.5)}
\textsuperscript{30:4.30} Los mortales ascendentes han sido preparados a fondo en los asuntos de los mundos evolutivos del espacio; ahora empiezan su largo y beneficioso contacto con las esferas creadas de la perfección. !Qué preparación se proporciona para algún trabajo futuro por medio de esta experiencia combinada, única y extraordinaria! Pero no puedo hablaros de Havona; tenéis que ver esos mundos para apreciar su gloria o comprender su grandiosidad.

\par
%\textsuperscript{(343.6)}
\textsuperscript{30:4.31} 7. \textit{Los que llegan al Paraíso.} Cuando llegáis al Paraíso con estado residencial, empezáis el curso progresivo en divinidad y absonidad. Vuestra residencia en el Paraíso significa que habéis encontrado a Dios y que seréis enrolados en el Cuerpo de los Mortales de la Finalidad. De todas las criaturas del gran universo, sólo aquellos que están fusionados con el Padre son enrolados en el Cuerpo de los Mortales de la Finalidad. Sólo estos individuos prestan el juramento finalitario. Otros seres que tienen o que han alcanzado la perfección paradisiaca pueden estar temporalmente vinculados a este cuerpo de la finalidad, pero no están destinados eternamente a la misión desconocida y no revelada de esta multitud creciente de veteranos evolutivos y perfeccionados del tiempo y del espacio.

\par
%\textsuperscript{(343.7)}
\textsuperscript{30:4.32} A los que llegan al Paraíso les conceden un período de libertad, después del cual empiezan sus asociaciones con los siete grupos de supernafines primarios. Cuando han terminado su curso con los conductores de la adoración se les denomina graduados paradisiacos, y luego, como finalitarios, son destinados a servicios de observación y de cooperación hasta los confines de la extensa creación. Hasta ahora no parece haber una ocupación específica o establecida para el Cuerpo de los Finalitarios Mortales, aunque sirven en numerosos empleos en los mundos establecidos en la luz y la vida.

\par
%\textsuperscript{(344.1)}
\textsuperscript{30:4.33} Si no existiera un destino futuro o no revelado para el Cuerpo de los Mortales de la Finalidad, la tarea actual de estos seres ascendentes ya sería totalmente adecuada y gloriosa. Su destino actual justifica plenamente el plan universal de la ascensión evolutiva. Pero las épocas futuras de la evolución de las esferas del espacio exterior ampliarán indudablemente más, e iluminarán divinamente con más plenitud, la sabiduría y la bondad de los Dioses en la ejecución de su plan divino para la supervivencia humana y la ascensión de los mortales.

\par
%\textsuperscript{(344.2)}
\textsuperscript{30:4.34} Esta narración, junto con lo que os ha sido revelado y con lo que podéis adquirir en conexión con la enseñanza relacionada con vuestro propio mundo, presenta un esbozo de la carrera de un mortal ascendente. La historia varía considerablemente en los diferentes superuniversos, pero este relato proporciona un vislumbre del plan medio de la progresión de los mortales tal como se encuentra en vigor en el universo local de Nebadon y en el séptimo segmento del gran universo, el superuniverso de Orvonton.

\par
%\textsuperscript{(344.3)}
\textsuperscript{30:4.35} [Patrocinado por un Mensajero Poderoso procedente de Uversa. ]


\chapter{Documento 31. El Cuerpo de la Finalidad}
\par
%\textsuperscript{(345.1)}
\textsuperscript{31:0.1} EL CUERPO de los Finalitarios Mortales representa el destino actualmente conocido de los mortales ascendentes del tiempo fusionados con su Ajustador. Pero existen otros grupos que también están asignados a este cuerpo. El cuerpo finalitario primario está compuesto como sigue:

\par
%\textsuperscript{(345.2)}
\textsuperscript{31:0.2} 1. Los Nativos de Havona.

\par
%\textsuperscript{(345.3)}
\textsuperscript{31:0.3} 2. Los Mensajeros de Gravedad.

\par
%\textsuperscript{(345.4)}
\textsuperscript{31:0.4} 3. Los Mortales Glorificados.

\par
%\textsuperscript{(345.5)}
\textsuperscript{31:0.5} 4. Los Serafines Adoptados.

\par
%\textsuperscript{(345.6)}
\textsuperscript{31:0.6} 5. Los Hijos Materiales Glorificados.

\par
%\textsuperscript{(345.7)}
\textsuperscript{31:0.7} 6. Las Criaturas Intermedias Glorificadas.

\par
%\textsuperscript{(345.8)}
\textsuperscript{31:0.8} Estos seis grupos de seres glorificados componen este cuerpo único que tiene un destino eterno. Creemos conocer su trabajo futuro, pero no estamos seguros. Aunque el Cuerpo de la Finalidad de los Mortales se está movilizando en el Paraíso, y aunque ahora ejercen tan ampliamente su ministerio en los universos del espacio y administran los mundos establecidos en la luz y la vida, su destino futuro debe ser los universos que se están organizando actualmente en el espacio exterior. Al menos esto es lo que se conjetura en Uversa.

\par
%\textsuperscript{(345.9)}
\textsuperscript{31:0.9} El cuerpo está organizado con arreglo a las asociaciones de trabajo de los mundos del espacio y de acuerdo con la experiencia asociativa adquirida durante toda la larga y agitada carrera ascendente. Todas las criaturas ascendentes admitidas en este cuerpo son recibidas en un pie de igualdad, pero esta elevada igualdad no abroga de ninguna manera la individualidad ni destruye la identidad personal. Al comunicarnos con un finalitario podemos discernir inmediatamente si es un mortal ascendente, un nativo de Havona, un serafín adoptado, una criatura intermedia o un Hijo Material.

\par
%\textsuperscript{(345.10)}
\textsuperscript{31:0.10} Durante la presente época del universo, los finalitarios regresan a servir en los universos del tiempo. Se les destina a trabajar sucesivamente en los diferentes superuniversos, pero nunca en su superuniverso nativo hasta que no han servido en las otras seis supercreaciones. Así pueden adquirir el concepto séptuple del Ser Supremo.

\par
%\textsuperscript{(345.11)}
\textsuperscript{31:0.11} Una o más compañías de finalitarios mortales están constantemente de servicio en Urantia. No existe ningún ámbito de servicio universal al que no sean destinados; ejercen su actividad en todo el universo, con períodos iguales y alternos de deberes asignados y de servicio libre.

\par
%\textsuperscript{(345.12)}
\textsuperscript{31:0.12} No tenemos ninguna idea de la naturaleza de la organización futura de este grupo extraordinario, pero los finalitarios son en la actualidad un cuerpo totalmente autónomo. Eligen a sus propios jefes y directores permanentes, periódicos y de trabajo. Ninguna influencia exterior puede nunca hacer presión sobre su política, y sólo prestan su juramento de lealtad a la Trinidad del Paraíso.

\par
%\textsuperscript{(346.1)}
\textsuperscript{31:0.13} Los finalitarios mantienen sus propias sedes centrales en el Paraíso, en los superuniversos, en los universos locales y en todas las capitales divisionarias. Forman una orden distinta de creación evolutiva. No los dirigimos ni los controlamos directamente y, sin embargo, son absolutamente leales y siempre cooperan con todos nuestros planes. Son en verdad las almas probadas y sinceras del tiempo y del espacio en vías de reunirse ---la sal evolutiva del universo--- y son para siempre impermeables al mal y están protegidos contra el pecado.

\section*{1. Los Nativos de Havona}
\par
%\textsuperscript{(346.2)}
\textsuperscript{31:1.1} Muchos nativos de Havona que sirven como instructores en las escuelas del universo central donde se forma a los peregrinos se vinculan profundamente con los mortales ascendentes y se sienten aún más fascinados por el trabajo y el destino futuros del Cuerpo de los Finalitarios Mortales. En la sede administrativa que tiene este cuerpo en el Paraíso se mantiene un registro para los voluntarios de Havona, presidido por el asociado de Grandfanda. Hoy encontraríais a millones y millones de nativos de Havona en esta lista de espera. Estos seres perfectos, que han sido creados de manera directa y divina, son de una gran ayuda para el Cuerpo de los Mortales de la Finalidad, y realizarán indudablemente servicios más importantes en el lejano futuro. Proporcionan el punto de vista de los seres nacidos en la perfección y en la plenitud divina. Los finalitarios abarcan así las dos fases de la existencia experiencial ---la perfecta y la perfeccionada.

\par
%\textsuperscript{(346.3)}
\textsuperscript{31:1.2} Los nativos de Havona deben conseguir ciertos desarrollos experienciales, en contacto con los seres evolutivos, que crearán en ellos la capacidad para recibir el don de un fragmento del espíritu del Padre Universal. El Cuerpo Finalitario de los Mortales sólo tiene como miembros permanentes a aquellos seres que han fusionado con el espíritu de la Fuente-Centro Primera o que, al igual que los Mensajeros de Gravedad, poseen de manera innata este espíritu de Dios Padre.

\par
%\textsuperscript{(346.4)}
\textsuperscript{31:1.3} El cuerpo recibe a los habitantes del universo central en la proporción de uno por mil ---una compañía de finalitarios. El cuerpo está organizado para el servicio temporal en compañías de mil, y el número de criaturas ascendentes es de 997 por un nativo de Havona y un Mensajero de Gravedad. Los finalitarios están movilizados así en compañías, pero el juramento de la finalidad se toma individualmente. Es un juramento que tiene implicaciones profundas y es de una importancia eterna. El nativo de Havona presta el mismo juramento y se une para siempre al cuerpo.

\par
%\textsuperscript{(346.5)}
\textsuperscript{31:1.4} Los reclutas de Havona siguen a la compañía en la que están destinados; allá donde va el grupo, van ellos. Y deberíais ver su entusiasmo por su nuevo trabajo como finalitarios. La posibilidad de alcanzar el Cuerpo de la Finalidad es una de las magníficas emociones de Havona; la posibilidad de convertirse en un finalitario es una de las aventuras supremas de estas razas perfectas.

\par
%\textsuperscript{(346.6)}
\textsuperscript{31:1.5} A los nativos de Havona también los reciben en la misma proporción en el Cuerpo de los Finalitarios Trinitizados Conjuntos de Vicegerington y en el Cuerpo de los Finalitarios Trascendentales del Paraíso. Los ciudadanos de Havona consideran estos tres destinos, junto con su posible admisión en el Cuerpo de los Finalitarios de Havona, como las metas supremas de sus carreras celestiales.

\section*{2. Los Mensajeros de Gravedad}
\par
%\textsuperscript{(346.7)}
\textsuperscript{31:2.1} En cualquier momento y lugar donde los Mensajeros de Gravedad ejercen su actividad, los finalitarios están al mando. Todos los Mensajeros de Gravedad están bajo la jurisdicción exclusiva de Grandfanda, y sólo están asignados al Cuerpo primario de la Finalidad. En el momento actual son inapreciables para los finalitarios, y se les podrá utilizar para todo en el eterno futuro. Ningún otro grupo de criaturas inteligentes posee un cuerpo así de mensajeros personalizados capaces de trascender el tiempo y el espacio. Los tipos similares de mensajeros-registradores destinados en otros cuerpos finalitarios no están personalizados; están absonitizados.

\par
%\textsuperscript{(347.1)}
\textsuperscript{31:2.2} Los Mensajeros de Gravedad proceden de Divinington y son Ajustadores modificados y personalizados, pero ningún miembro de nuestro grupo de Uversa se comprometería a explicar la naturaleza de uno de estos mensajeros. Sabemos que son unos seres extremadamente personales, divinos, inteligentes y conmovedoramente comprensivos, pero no comprendemos la técnica que utilizan para atravesar instantáneamente el espacio. Parecen ser capaces de utilizar todas las energías, todos los circuitos e incluso la gravedad. Los Finalitarios del cuerpo de los mortales no pueden desafiar el tiempo y el espacio, pero se encuentran asociadas con ellos y sometidas a su mando unas personalidades espirituales casi infinitas que sí pueden hacerlo. Nos permitimos llamar personalidades a los Mensajeros de Gravedad, pero en realidad son seres superespirituales, unas personalidades sin límites ni trabas. Comparados con los Mensajeros Solitarios, son un tipo de personalidad totalmente diferente.

\par
%\textsuperscript{(347.2)}
\textsuperscript{31:2.3} Los Mensajeros de Gravedad pueden estar vinculados a una compañía finalitaria en cantidades ilimitadas, pero sólo un mensajero, el jefe de sus compañeros, es enrolado en el Cuerpo de los Mortales de la Finalidad. Sin embargo, este jefe tiene asignado un estado mayor permanente de 999 compañeros mensajeros y, según lo requieran las circunstancias, puede recurrir a las reservas de la orden para obtener un número ilimitado de ayudantes.

\par
%\textsuperscript{(347.3)}
\textsuperscript{31:2.4} Los Mensajeros de Gravedad y los finalitarios mortales glorificados llegan a tener un afecto profundo y conmovedor los unos por los otros; tienen muchas cosas en común: unos son la personalización directa de un fragmento del Padre Universal; los otros, una personalidad de criatura que existe en el alma inmortal sobreviviente fusionada con un fragmento del mismo Padre Universal, el Ajustador del Pensamiento espiritual.

\section*{3. Los mortales glorificados}
\par
%\textsuperscript{(347.4)}
\textsuperscript{31:3.1} Los mortales ascendentes fusionados con su Ajustador componen la mayor parte del Cuerpo primario de la Finalidad. Junto con los serafines adoptados y glorificados, ascienden generalmente a 990 en cada compañía finalitaria. La proporción de mortales y de ángeles varía en cada grupo, aunque los mortales son mucho más numerosos que los serafines. Los nativos de Havona, los Hijos Materiales glorificados, las criaturas intermedias glorificadas, los Mensajeros de Gravedad y el miembro desconocido que falta sólo constituyen el uno por ciento del cuerpo; cada compañía de mil finalitarios sólo tiene sitio para diez de estas personalidades no mortales y no seráficas.

\par
%\textsuperscript{(347.5)}
\textsuperscript{31:3.2} Nosotros los de Uversa no conocemos el <<\textit{destino finalitario}>> de los mortales ascendentes del tiempo. En el momento actual residen en el Paraíso y sirven temporalmente en el Cuerpo de Luz y de Vida, pero un programa tan extraordinario de formación ascendente y una disciplina universal tan prolongada deben estar destinados a cualificarlos para unas pruebas de confianza aún más grandes y unos servicios de responsabilidad aún más sublimes.

\par
%\textsuperscript{(347.6)}
\textsuperscript{31:3.3} A pesar de que estos mortales ascendentes han alcanzado el Paraíso, han sido enrolados en el Cuerpo de la Finalidad y han sido enviados de vuelta en gran número para participar en la dirección de los universos locales y para ayudar en la administración de los asuntos superuniversales ---en presencia incluso de este destino \textit{aparente}, subsiste el hecho significativo de que sólo están registrados como espíritus de la sexta fase. Falta indudablemente una etapa más en la carrera del Cuerpo de los Mortales de la Finalidad. No conocemos la naturaleza de dicha etapa, pero hemos tenido en cuenta tres hechos sobre los que llamamos aquí la atención:

\par
%\textsuperscript{(348.1)}
\textsuperscript{31:3.4} 1. Sabemos por los archivos que los mortales son espíritus del primer grado durante su estancia en los sectores menores, que ascienden al segundo grado cuando son trasladados a los sectores mayores, y al tercero cuando avanzan hasta los mundos educativos centrales del superuniverso. Los mortales se vuelven espíritus graduados o de cuarto grado después de llegar al sexto círculo de Havona, y se convierten en espíritus de quinto grado cuando encuentran al Padre Universal. Posteriormente consiguen la sexta fase de la existencia espiritual al prestar el juramento que los enrola para siempre en la tarea para la eternidad del Cuerpo de la Finalidad de los Mortales.

\par
%\textsuperscript{(348.2)}
\textsuperscript{31:3.5} Observamos que la clasificación o designación de los espíritus está determinada por el progreso efectivo desde un reino de servicio universal a otro reino de servicio universal, o desde un universo a otro universo; y suponemos que la concesión de la clasificación como espíritus del séptimo grado en el Cuerpo de los Mortales de la Finalidad se producirá al mismo tiempo que sus miembros asciendan a la misión eterna de servir en unas esferas hasta ahora no registradas y no reveladas, y que esto coincidirá con el hecho de alcanzar a Dios Supremo. Pero aparte de estas conjeturas audaces, realmente no sabemos mucho más que vosotros sobre todo esto; nuestro conocimiento sobre la carrera de los mortales no va más allá del destino paradisiaco actual.

\par
%\textsuperscript{(348.3)}
\textsuperscript{31:3.6} 2. Los finalitarios mortales han cumplido plenamente con el mandato de todos los tiempos: <<\textit{Sed perfectos}>>\footnote{\textit{Sed perfectos}: Gn 17:1; 1 Re 8:61; Lv 19:2; Dt 18:13; Mt 5:48; 2 Co 13:11; Stg 1:4; 1 P 1:16.}; han ascendido el sendero universal de la consecución humana; han encontrado a Dios, y han sido debidamente admitidos en el Cuerpo de la Finalidad. Estos seres han alcanzado el límite actual de la progresión espiritual, pero no \textit{la finalidad del estado espiritual último}. Han llegado al límite actual de la perfección de las criaturas, pero no a la \textit{finalidad del servicio de las criaturas}. Han experimentado la plenitud de la adoración de la Deidad, pero no la \textit{finalidad de alcanzar experiencialmente a la Deidad}.

\par
%\textsuperscript{(348.4)}
\textsuperscript{31:3.7} 3. Los mortales glorificados del Cuerpo Paradisiaco de la Finalidad son seres ascendentes que poseen el conocimiento experiencial de cada etapa de la realidad y la filosofía de vida más completa posible de la existencia inteligente, mientras que durante las eras de esta ascensión desde los mundos materiales más humildes hasta las alturas espirituales del Paraíso, estas criaturas sobrevivientes han sido instruidas hasta los límites de su capacidad en todos los detalles de todos los principios divinos relacionados con la administración justa y eficaz, así como misericordiosa y paciente, de toda la creación universal del tiempo y del espacio.

\par
%\textsuperscript{(348.5)}
\textsuperscript{31:3.8} Estimamos que los seres humanos tienen derecho a compartir nuestras opiniones, y que tenéis la libertad de conjeturar con nosotros sobre el misterio del destino último del Cuerpo Paradisiaco de la Finalidad. Nos parece evidente que las tareas actuales de las criaturas evolutivas perfeccionadas comparten la naturaleza de los cursos postgraduados de comprensión universal y de administración superuniversal; y todos nos preguntamos: <<\textit{¿Por qué los Dioses se preocupan tanto por instruir tan minuciosamente a los mortales sobrevivientes en la técnica de dirigir el universo?}>>

\section*{4. Los serafines adoptados}
\par
%\textsuperscript{(348.6)}
\textsuperscript{31:4.1} A muchos fieles guardianes seráficos de los mortales se les permite recorrer la carrera ascendente con sus pupilos humanos, y muchos de estos ángeles guardianes, después de haber fusionado con el Padre, se unen a sus sujetos para prestar el juramento finalitario de la eternidad y aceptar para siempre el destino de sus asociados mortales. Los ángeles que pasan por la experiencia ascendente de los seres mortales pueden compartir el destino de la naturaleza humana; pueden ser igualmente enrolados de manera eterna en este Cuerpo de la Finalidad. Un gran número de serafines adoptados y glorificados forman parte de los diversos cuerpos finalitarios no mortales.

\section*{5. Los Hijos Materiales glorificados}
\par
%\textsuperscript{(349.1)}
\textsuperscript{31:5.1} Existe una disposición en los universos del tiempo y del espacio por la que, cuando los ciudadanos adámicos de los sistemas locales tardan mucho en recibir una misión planetaria, pueden iniciar una petición para ser liberados del estado de ciudadanos permanentes. Si se les concede, se unen a los peregrinos ascendentes en las capitales de los universos, y desde allí se dirigen hacia el Paraíso y el Cuerpo de la Finalidad.

\par
%\textsuperscript{(349.2)}
\textsuperscript{31:5.2} Cuando un mundo evolutivo avanzado alcanza las épocas finales de la era de luz y de vida, los Hijos Materiales, el Adán y la Eva Planetarios, pueden elegir humanizarse, recibir sus Ajustadores, y emprender el recorrido evolutivo de la ascensión universal que conduce al Cuerpo de los Finalitarios Mortales. Algunos de estos Hijos Materiales han fracasado parcialmente o han fallado técnicamente en su misión como aceleradores biológicos, como le sucedió a Adán en Urantia; entonces se ven obligados a tomar el camino natural de los pueblos del reino, recibir sus Ajustadores, pasar por la muerte, progresar por la fe a través del régimen ascendente, y alcanzar posteriormente el Paraíso y el Cuerpo de la Finalidad.

\par
%\textsuperscript{(349.3)}
\textsuperscript{31:5.3} A estos Hijos Materiales no se les encuentra en muchas compañías finalitarias. Su presencia confiere un gran potencial a las posibilidades de servicio elevado de ese grupo, y son invariablemente elegidos como jefes. Si los dos miembros de la pareja edénica están destinados en el mismo grupo, generalmente se les permite trabajar juntos como una sola personalidad. Estas parejas ascendentes tienen mucho más éxito que los mortales ascendentes en la aventura de la trinitización.

\section*{6. Las criaturas intermedias glorificadas}
\par
%\textsuperscript{(349.4)}
\textsuperscript{31:6.1} En muchos planetas las criaturas intermedias son engendradas en gran número, pero raras veces se quedan en su mundo nativo después de que éste se establece en la luz y la vida. En ese momento, o poco después, son liberados de su estado de ciudadanos permanentes y empiezan su ascensión hacia el Paraíso, pasando por los mundos morontiales, el superuniverso y Havona, en compañía de los mortales del tiempo y del espacio.

\par
%\textsuperscript{(349.5)}
\textsuperscript{31:6.2} El origen y la naturaleza de las criaturas intermedias de los diversos universos son muy diferentes, pero todas están destinadas a uno u otro de los cuerpos paradisiacos de la finalidad. Todos los intermedios secundarios fusionan finalmente con su Ajustador y son enrolados en el cuerpo de los mortales. Muchas compañías finalitarias tienen en su grupo a uno de estos seres glorificados.

\section*{7. Los evángeles de Luz}
\par
%\textsuperscript{(349.6)}
\textsuperscript{31:7.1} En la época actual, cada compañía finalitaria contiene 999 personalidades que han prestado juramento, que son miembros permanentes. La plaza vacante está ocupada por el jefe de los Evángeles de Luz\footnote{\textit{Evángel de Luz}: 2 Co 11:14.} destinados en esa compañía y encargados de una misión determinada. Pero estos seres sólo son miembros transitorios del cuerpo.

\par
%\textsuperscript{(349.7)}
\textsuperscript{31:7.2} Toda personalidad celestial destinada al servicio de cualquier cuerpo finalitario se denomina Evángel de Luz. Estos seres no prestan el juramento finalitario, y aunque están sujetos a la organización del cuerpo, no están vinculados a él de manera permanente. Este grupo puede incluir a los Mensajeros Solitarios, los supernafines, los seconafines, los Ciudadanos del Paraíso o sus descendientes trinitizados ---cualquier ser que sea necesario para ejecutar una tarea finalitaria transitoria. No sabemos si estos seres van a estar vinculados a la misión eterna del cuerpo. Al final de su misión, estos Evángeles de Luz recuperan su estado anterior.

\par
%\textsuperscript{(350.1)}
\textsuperscript{31:7.3} En el Cuerpo de los Mortales de la Finalidad, tal como está constituido actualmente, hay exactamente seis clases de miembros permanentes. Los finalitarios, como se podría esperar, especulan mucho sobre la identidad de sus camaradas futuros, pero hay poco acuerdo entre ellos.

\par
%\textsuperscript{(350.2)}
\textsuperscript{31:7.4} Nosotros los de Uversa conjeturamos a menudo sobre la identidad del séptimo grupo de finalitarios. Albergamos muchas ideas, entre ellas la posible asignación de algunos cuerpos de los numerosos grupos trinitizados que se están acumulando en el Paraíso, en Vicegerington y en el circuito interior de Havona. Se conjetura incluso que al Cuerpo de la Finalidad se le permitirá trinitizar a muchos seres que lo ayudan en el trabajo de la administración universal, en el caso de que sean destinados al servicio de los universos que están actualmente en proceso de formación.

\par
%\textsuperscript{(350.3)}
\textsuperscript{31:7.5} Uno de nosotros mantiene la opinión de que esta plaza vacante del cuerpo será ocupada por algún tipo de ser que tendrá su origen en el nuevo universo donde realizarán su servicio futuro; otro tiende a creer que esta plaza la ocupará algún tipo de personalidad del Paraíso aún no creada, existenciada o trinitizada. Pero es muy probable que tengamos que esperar a que los finalitarios entren en su séptima fase de consecución espiritual para saberlo realmente.

\section*{8. Los trascendentales}
\par
%\textsuperscript{(350.4)}
\textsuperscript{31:8.1} Una parte de la experiencia como finalitario en el Paraíso de un mortal perfeccionado consiste en el esfuerzo por conseguir comprender la naturaleza y la función de más de mil grupos de superciudadanos trascendentales del Paraíso, unos seres existenciados con atributos absonitos. En su asociación con estas superpersonalidades, los finalitarios ascendentes reciben una gran ayuda del útil asesoramiento de numerosas órdenes de ministros trascendentales que tienen la tarea de presentar a los finalitarios evolucionados a sus nuevos hermanos del Paraíso. Toda la orden de los Trascendentales vive en el oeste del Paraíso en una inmensa zona que ocupan de manera exclusiva.

\par
%\textsuperscript{(350.5)}
\textsuperscript{31:8.2} Cuando hablamos de los Trascendentales, no sólo estamos restringidos por las limitaciones de la comprensión humana, sino también por los términos del mandato que regula estas revelaciones sobre las personalidades del Paraíso. Estos seres no están relacionados de ninguna manera con la ascensión de los mortales hasta Havona. La inmensa multitud de Trascendentales del Paraíso no tiene absolutamente nada que ver con los asuntos de Havona o de los siete superuniversos, pues se ocupan solamente de la superadministración de los asuntos del universo maestro.

\par
%\textsuperscript{(350.6)}
\textsuperscript{31:8.3} Tú, como eres una criatura, puedes concebir a un Creador, pero difícilmente puedes comprender que existe un enorme agregado diversificado de seres inteligentes que no son ni Creadores ni criaturas. Estos Trascendentales no crean seres, ni ellos mismos fueron nunca creados. Al hablar de su origen, y a fin de evitar la utilización de un nuevo término ---de una denominación arbitraria y sin sentido--- creemos que es mejor decir que los Trascendentales simplemente se \textit{existencian.} Puede ser muy bien que el Absoluto de la Deidad haya estado relacionado con su origen y pueda estar implicado en su destino, pero estos seres únicos no están actualmente dominados por el Absoluto de la Deidad. Están sometidos a Dios Último, y su estancia actual en el Paraíso está supervisada y dirigida en todos los aspectos por la Trinidad.

\par
%\textsuperscript{(351.1)}
\textsuperscript{31:8.4} Aunque todos los mortales que alcanzan el Paraíso fraternizan a menudo con los Trascendentales, tal como lo hacen con los Ciudadanos del Paraíso, sucede que el primer contacto formal de un hombre con un Trascendental se produce durante el acontecimiento memorable en el que, como miembro de un nuevo grupo finalitario, el ascendente mortal se encuentra en el círculo de recepción finalitario donde el jefe de los Trascendentales toma el juramento trinitario de la eternidad; este jefe es el que preside a los Arquitectos del Universo Maestro.

\section*{9. Los Arquitectos del universo maestro}
\par
%\textsuperscript{(351.2)}
\textsuperscript{31:9.1} Los Arquitectos del Universo Maestro son el cuerpo gobernante de los Trascendentales del Paraíso. Este cuerpo gobernante asciende a 28.011 personalidades que poseen unas mentes maestras, unos espíritus magníficos y unas facultades absonitas celestiales. El Arquitecto Maestro más antiguo, dignatario que preside este grupo magnífico, es el jefe que coordina todas las inteligencias del Paraíso por debajo del nivel de la Deidad.

\par
%\textsuperscript{(351.3)}
\textsuperscript{31:9.2} La decimosexta proscripción del mandato que autoriza estas narraciones dice: <<\textit{Si se considera prudente, se puede revelar la existencia de los Arquitectos del Universo Maestro y de sus asociados, pero su origen, su naturaleza y su destino no se pueden revelar plenamente}>>. Podemos informaros sin embargo que estos Arquitectos Maestros existen en siete niveles de lo absonito. Estos siete grupos están clasificados como sigue:

\par
%\textsuperscript{(351.4)}
\textsuperscript{31:9.3} 1. \textit{El Nivel del Paraíso.} Sólo el primer Arquitecto existenciado, el más antiguo, ejerce su actividad en este nivel superior de lo absonito. Esta personalidad última ---ni Creador ni criatura--- se existenció en los albores de la eternidad, y ahora actúa como coordinador exquisito del Paraíso y de sus veintiún mundos de actividades asociadas.

\par
%\textsuperscript{(351.5)}
\textsuperscript{31:9.4} 2. \textit{El Nivel de Havona.} La segunda existenciación de Arquitectos produjo tres planificadores maestros y administradores absonitos, y siempre se han dedicado a la coordinación de los mil millones de esferas perfectas del universo central. La tradición del Paraíso afirma que estos tres Arquitectos, con el asesoramiento del Arquitecto más antiguo existenciado anteriormente, contribuyeron a la planificación de Havona, pero realmente no lo sabemos.

\par
%\textsuperscript{(351.6)}
\textsuperscript{31:9.5} 3. \textit{El Nivel de los Superuniversos.} El tercer nivel absonito abarca a los siete Arquitectos Maestros de los siete superuniversos, que pasan actualmente, como grupo, un período de tiempo casi igual en compañía de los Siete Espíritus Maestros en el Paraíso y con los Siete Ejecutivos Supremos en los siete mundos especiales del Espíritu Infinito. Son los supercoordinadores del gran universo.

\par
%\textsuperscript{(351.7)}
\textsuperscript{31:9.6} 4. \textit{El Primer Nivel de Espacio.} Este grupo asciende a setenta Arquitectos, y conjeturamos que se ocupan de los planes últimos para el primer universo del espacio exterior, que está ahora en vías de movilización más allá de las fronteras de los siete superuniversos actuales.

\par
%\textsuperscript{(351.8)}
\textsuperscript{31:9.7} 5. \textit{El Segundo Nivel de Espacio.} Este quinto cuerpo de Arquitectos asciende a 490 miembros, y conjeturamos de nuevo que deben ocuparse del segundo universo del espacio exterior, donde nuestros físicos ya han detectado unas claras movilizaciones de energía.

\par
%\textsuperscript{(352.1)}
\textsuperscript{31:9.8} 6. \textit{El Tercer Nivel de Espacio.} Este sexto grupo de Arquitectos Maestros asciende a 3.430 miembros, y deducimos igualmente que deben estar ocupados con los planes gigantescos del tercer universo del espacio exterior.

\par
%\textsuperscript{(352.2)}
\textsuperscript{31:9.9} 7. \textit{El Cuarto Nivel de Espacio.} Este cuerpo, el último y el más numeroso, consta de 24.010 Arquitectos Maestros, y si nuestras conjeturas anteriores son válidas, debe estar relacionado con el cuarto y último de los universos del espacio exterior, cada uno de los cuales es más grande que el anterior.

\par
%\textsuperscript{(352.3)}
\textsuperscript{31:9.10} Estos siete grupos de Arquitectos Maestros suman un total de
28.011 planificadores de universos. En el Paraíso existe una tradición según la cual allá por la lejana eternidad se intentó la existenciación del Arquitecto Maestro número 28.012, pero este ser no consiguió absonitizarse, ya que el Absoluto Universal se incautó de su personalidad. Es posible que la serie ascendente de los Arquitectos Maestros alcanzara el límite de la absonidad con el Arquitecto número 28.011, y que la tentativa 28.012 se encontró con el nivel matemático de la presencia del Absoluto. En otras palabras, en el nivel de existenciación 28.012, la cualidad de la absonidad equivalía al nivel del Universal y alcanzó el valor del Absoluto.

\par
%\textsuperscript{(352.4)}
\textsuperscript{31:9.11} Los tres Arquitectos supervisores de Havona actúan, en su organización funcional, como ayudantes asociados del Arquitecto solitario del Paraíso. Los siete Arquitectos de los superuniversos actúan como coordinados de los tres supervisores de Havona. Los setenta planificadores de los universos del primer nivel del espacio exterior sirven actualmente como ayudantes asociados de los siete Arquitectos de los siete superuniversos.

\par
%\textsuperscript{(352.5)}
\textsuperscript{31:9.12} Los Arquitectos del Universo Maestro tienen a su disposición numerosos grupos de ayudantes y colaboradores, incluyendo a dos extensas órdenes de organizadores de la fuerza, los existenciados primarios y los trascendentales asociados. Estos Organizadores de la Fuerza Maestros no deben ser confundidos con los directores del poder, los cuales están relacionados con el gran universo.

\par
%\textsuperscript{(352.6)}
\textsuperscript{31:9.13} Todos los seres engendrados por la unión de los hijos del tiempo con los hijos de la eternidad, tales como los descendientes trinitizados por los finalitarios y los Ciudadanos del Paraíso, se convierten en los pupilos de los Arquitectos Maestros. Pero de todas las demás criaturas o entidades reveladas que ejercen su actividad en los universos actualmente organizados, sólo los Mensajeros Solitarios y los Espíritus Inspirados Trinitarios mantienen una asociación orgánica con los Trascendentales y con los Arquitectos del Universo Maestro.

\par
%\textsuperscript{(352.7)}
\textsuperscript{31:9.14} Los Arquitectos Maestros contribuyen a que se apruebe técnicamente la concesión a los Hijos Creadores de sus emplazamientos espaciales para que organicen sus universos locales. Existe una asociación muy estrecha entre los Arquitectos Maestros y los Hijos Creadores Paradisiacos y, aunque esta relación no se ha revelado, se os ha informado acerca de la asociación de los Arquitectos con los Creadores Supremos del gran universo en la relación de la primera Trinidad experiencial. Estos dos grupos, junto con el Ser Supremo evolutivo y experiencial, forman la Trinidad Última de valores trascendentales y de significados del universo maestro.

\section*{10. La última aventura}
\par
%\textsuperscript{(352.8)}
\textsuperscript{31:10.1} El Arquitecto Maestro más antiguo tiene la supervisión de los siete Cuerpos de la Finalidad, que son los siguientes:

\par
%\textsuperscript{(352.9)}
\textsuperscript{31:10.2} 1. El Cuerpo de los Finalitarios Mortales.

\par
%\textsuperscript{(352.10)}
\textsuperscript{31:10.3} 2. El Cuerpo de los Finalitarios del Paraíso.

\par
%\textsuperscript{(352.11)}
\textsuperscript{31:10.4} 3. El Cuerpo de los Finalitarios Trinitizados.

\par
%\textsuperscript{(353.1)}
\textsuperscript{31:10.5} 4. El Cuerpo de los Finalitarios Trinitizados Conjuntos.

\par
%\textsuperscript{(353.2)}
\textsuperscript{31:10.6} 5. El Cuerpo de los Finalitarios de Havona.

\par
%\textsuperscript{(353.3)}
\textsuperscript{31:10.7} 6. El Cuerpo de los Finalitarios Trascendentales.

\par
%\textsuperscript{(353.4)}
\textsuperscript{31:10.8} 7. El Cuerpo de los Hijos del Destino No Revelados.

\par
%\textsuperscript{(353.5)}
\textsuperscript{31:10.9} Cada uno de estos cuerpos del destino tiene un jefe que lo preside, y los siete constituyen el Consejo Supremo del Destino en el Paraíso; y durante la presente era del universo, Grandfanda es el jefe de este cuerpo supremo que asigna las misiones universales a los hijos del destino último.

\par
%\textsuperscript{(353.6)}
\textsuperscript{31:10.10} La reunión de estos siete cuerpos finalitarios indica la movilización, dentro de la realidad, de unos potenciales, personalidades, mentes, espíritus, absonitos y realidades experienciales, que trascienden probablemente incluso las funciones futuras del Ser Supremo en el universo maestro. Estos siete cuerpos finalitarios indican probablemente la actividad actual de la Trinidad Última, ocupada en reunir las fuerzas de lo finito y de lo absonito como preparación para unos desarrollos inconcebibles en los universos del espacio exterior. Nada que se parezca a esta movilización ha tenido lugar desde los tiempos cercanos a la eternidad, cuando la Trinidad del Paraíso movilizó de manera similar a las personalidades entonces existentes del Paraíso y de Havona, y las nombró como administradoras y gobernantes de los siete superuniversos del tiempo y del espacio entonces en proyecto. Los siete cuerpos finalitarios representan la respuesta de divinidad del gran universo a las necesidades futuras de los potenciales no desarrollados en los universos exteriores donde tendrán lugar futuras actividades eternas.

\par
%\textsuperscript{(353.7)}
\textsuperscript{31:10.11} Nos aventuramos a pronosticar la existencia de unos universos exteriores futuros y aún mayores de mundos habitados, de nuevas esferas pobladas de nuevos tipos de seres exquisitos y únicos, de un universo material de una ultimidad sublime, de una inmensa creación a la que sólo le faltará un detalle importante ---la presencia de una \textit{experiencia finita} real en la vida universal de la existencia ascendente. Ese universo nacerá con una enorme desventaja experiencial: la privación de participar en la evolución del Todopoderoso Supremo. Todos estos universos exteriores disfrutarán del ministerio incomparable y del supercontrol celestial del Ser Supremo, pero el hecho mismo de su presencia activa impedirá la participación de dichos universos en la manifestación de la Deidad Suprema.

\par
%\textsuperscript{(353.8)}
\textsuperscript{31:10.12} Durante la presente era del universo, las personalidades evolutivas del gran universo sufren muchas dificultades debido a la manifestación incompleta de la soberanía de Dios Supremo, pero todos estamos participando en la experiencia única de su evolución. Evolucionamos en él, y él evoluciona en nosotros. En algún momento del eterno futuro, la evolución de la Deidad Suprema será un hecho consumado de la historia universal, y la oportunidad de participar en esta maravillosa experiencia habrá desaparecido de la escena de la acción cósmica.

\par
%\textsuperscript{(353.9)}
\textsuperscript{31:10.13} Pero aquellos de nosotros que hayan adquirido esta experiencia única durante la juventud del universo, la atesorarán a lo largo de toda la eternidad futura. Muchos de nosotros especulan que la misión de las reservas de mortales ascendentes y perfeccionados del Cuerpo de la Finalidad, que se acumulan gradualmente, en asociación con los otros seis cuerpos que se están reclutando de manera similar, quizás podría ser la de administrar estos universos exteriores, en un esfuerzo por compensar sus deficiencias experienciales por no haber participado en la evolución espacio-temporal del Ser Supremo.

\par
%\textsuperscript{(353.10)}
\textsuperscript{31:10.14} Estas deficiencias son inevitables en todos los niveles de la existencia universal. Durante la presente era del universo, nosotros, los de los niveles superiores de las existencias espirituales, descendemos ahora para administrar los universos evolutivos y aportar nuestro ministerio a los mortales ascendentes, esforzándonos así por compensar sus deficiencias en las realidades de la experiencia espiritual superior.

\par
%\textsuperscript{(354.1)}
\textsuperscript{31:10.15} Pero aunque realmente no sabemos nada sobre los planes de los Arquitectos del Universo Maestro respecto a estas creaciones exteriores, sin embargo estamos seguros de tres cosas:

\par
%\textsuperscript{(354.2)}
\textsuperscript{31:10.16} 1. Existe realmente un sistema nuevo e inmenso de universos que se está organizando gradualmente en los dominios del espacio exterior. En efecto, con vuestros telescopios se pueden ver nuevos tipos de creaciones físicas, enormes círculos gigantescos de universos pululantes tras universos, mucho más allá de los límites actuales de las creaciones pobladas y organizadas. En la actualidad, estas creaciones exteriores son totalmente físicas; están aparentemente deshabitadas y parecen estar desprovistas de administración por parte de las criaturas.

\par
%\textsuperscript{(354.3)}
\textsuperscript{31:10.17} 2. Durante épocas y épocas continúa la movilización en el Paraíso, inexplicada y totalmente misteriosa, de los seres perfeccionados y ascendentes del tiempo y del espacio, en asociación con los otros seis cuerpos finalitarios.

\par
%\textsuperscript{(354.4)}
\textsuperscript{31:10.18} 3. En concomitancia con estas operaciones, la Persona Suprema de la Deidad está aumentando su poder como soberano todopoderoso de las supercreaciones.

\par
%\textsuperscript{(354.5)}
\textsuperscript{31:10.19} Cuando vemos este desarrollo trino, que engloba a las criaturas, a los universos y a la Deidad, ¿podemos ser criticados por indicar de antemano que algo nuevo y no revelado se acerca a su culminación en el universo maestro? ¿No es natural que asociemos esta movilización y esta organización seculares de los universos físicos, a una escala hasta ahora desconocida, y la emergencia de la personalidad del Ser Supremo, con este prodigioso plan de elevar a los mortales del tiempo hasta la perfección divina, y con su movilización posterior en el Paraíso en el Cuerpo de la Finalidad ---un nombramiento y un destino envueltos en un misterio universal? En toda Uversa se cree cada vez más que los Cuerpos de la Finalidad en vías de reunirse están destinados a algún servicio futuro en los universos del espacio exterior, donde ya somos capaces de identificar la agrupación de, al menos, setenta mil agregados de materia, cada uno de los cuales es mayor que cualquiera de los superuniversos actuales.

\par
%\textsuperscript{(354.6)}
\textsuperscript{31:10.20} Los mortales evolutivos nacen en los planetas del espacio, pasan por los mundos morontiales, ascienden a los universos espirituales, atraviesan las esferas de Havona, encuentran a Dios, alcanzan el Paraíso y son enrolados en el Cuerpo primario de la Finalidad, para esperar allí la siguiente misión de servicio universal. Hay otros seis cuerpos de la finalidad que se están reuniendo, pero Grandfanda, el primer ascendente mortal, preside como jefe paradisiaco todas las órdenes de finalitarios. Cuando vemos este espectáculo sublime, todos exclamamos: !Qué glorioso destino para los hijos temporales de origen animal, los hijos materiales del espacio!

\par
%\textsuperscript{(354.7)}
\textsuperscript{31:10.21} [Patrocinado conjuntamente por un Consejero Divino y Uno que no tiene Nombre ni Número, autorizados para actuar así por los Ancianos de los Días de Uversa.]

\par
%\textsuperscript{(354.8)}
\textsuperscript{31:10.22} Estos treinta y un documentos que describen la naturaleza de la Deidad, la realidad del Paraíso, la organización y el funcionamiento del universo central y de los superuniversos, las personalidades del gran universo y el elevado destino de los mortales evolutivos, fueron patrocinados, formulados y traducidos al inglés por una elevada comisión compuesta por veinticuatro administradores de Orvonton que actúan de acuerdo con un mandato promulgado por los Ancianos de los Días de Uversa, ordenando que hiciéramos esto en Urantia, planeta 606 de Satania, en Norlatiadek de Nebadon, en el año 1934 d. de J.C.

\newpage
\pagestyle{empty}

\par {\huge Abreviaturas}
\bigbreak
\bigbreak
\begin{multicols}{2}
	\par LU \textit{(El Libro de Urantia)}
	\bigbreak
	\par Libros bíblicos:
	\bigbreak
	\par Abd \textit{(Abdías)}
	\par Am \textit{(Amós)}
	\par Ap \textit{(Apocalipsis)}
	\par Bar \textit{(Baruc)}
	\par Co \textit{(Epístola a los Corintios)}
	\par Cnt \textit{(El Cantar de los Cantares)}
	\par Col \textit{(Epístola a los Colosenses)}
	\par Cr \textit{(Crónicas)}
	\par Dn \textit{(Daniel)}
	\par Dt \textit{(Deuteronomio)}
	\par Ec \textit{(Eclesiastés)}
	\par Eclo \textit{(Ecclesiástico)}
	\par Ef \textit{(Epístola a los Efesios)}
	\par Esd \textit{(Esdras)}
	\par Est \textit{(Ester)}
	\par Ex \textit{(Éxodo)}
	\par Ez \textit{(Ezequiel)} 
	\par Flm \textit{(Epístola a Filemón)}
	\par Flp \textit{(Epístola a los Filipenses)}
	\par Gl \textit{(Epítosla a los Gálatas)}
    \par Gn \textit{(Génesis)}
    \par Hab \textit{(Habacuc)} 
    \par Hag \textit{(Ageo)}
    \par Hch \textit{(Hechos de los Apóstoles)}
    \par Heb \textit{(Epístola a los Hebreos)}
    \par Is \textit{(Isaías)}
    \par Jer \textit{(Jeremías)}
    \par Jl \textit{(Joel)}
    \par Jn \textit{(Juan, evangelio y epístolas)}
    \par Job \textit{(Job)}
    \par Jon \textit{(Jonás)}
    \par Jos \textit{(Josué)}
    \par Jud \textit{(Epístola de Judas)}
    \par Jue \textit{(Jueces)}
    \par Lc \textit{(Lucas)}
    \par Lm \textit{(Lamentaciones)}
	\par Lv \textit{(Levítico)}
	\par Mac \textit{(Macabeos)}
	\par Mal \textit{(Malaquías)}
	\par Mc \textit{(Marcos)}
	\par Miq \textit{(Miqueas)} 
	\par Mt \textit{(Mateo)}
	\par Nah \textit{(Nahúm)}
	\par Neh \textit{(Nehemías)} 
	\par Nm \textit{(Números)}
	\par Os \textit{(Oseas)}
	\par P \textit{(Epístola de Pedro)}
	\par Pr \textit{(Proverbios)}
	\par Re \textit{(Reyes)}
	\par Ro \textit{(Epístola a los Romanos)}
	\par Rt \textit{(Rut)}
	\par Sab \textit{(Sabiduría)}
	\par Sal \textit{(Salmos)}
	\par Sam \textit{(Samuel)}
	\par Sof \textit{(Sofonías)}
	\par Stg \textit{(Epístola a Santiago)}
	\par Ti \textit{(Epístola a Timoteo)}
	\par Tit \textit{(Epítosla a Tito)}
	\par Ts \textit{(Epístola a los Tesalonicenses)}
	\par Zac \textit{(Zacarías)}
	\bigbreak
	\par Libros bíblicos apócrifos:
	\bigbreak 
	\par AsMo \textit{(Asunción de Moisés)}
	\par Bel \textit{(Bel y el Dragón)} 
	\par Hen \textit{(Enoc)} 
	\par Man \textit{(Oración de Manasés)} 
	\par Tb \textit{(Tobit)}
	\bigbreak
	\par Libros de otras religiones: 
	\bigbreak
	\par XXX \textit{(YYYY)}


\end{multicols}

\end{document}
