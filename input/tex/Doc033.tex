\chapter{Documento 33. La administración del universo local}
\par
%\textsuperscript{(366.1)}
\textsuperscript{33:0.1} AUNQUE el Padre Universal gobierna con toda seguridad sobre su inmensa creación, en la administración de un universo local actúa a través de la persona del Hijo Creador. El Padre no actúa personalmente de otra manera en los asuntos administrativos de un universo local. Estas materias las ha confiado al Hijo Creador, al Espíritu Madre del universo local, y a los múltiples hijos de ambos. Los planes, la política y los actos administrativos del universo local son concebidos y ejecutados por este Hijo, el cual, conjuntamente con su Espíritu asociado, delega el poder ejecutivo en Gabriel, y la autoridad jurisdiccional en los Padres de las Constelaciones, los Soberanos de los Sistemas y los Príncipes Planetarios.

\section*{1. Miguel de Nebadon}
\par
%\textsuperscript{(366.2)}
\textsuperscript{33:1.1} Nuestro Hijo Creador es la personificación del concepto original
611.121 de identidad infinita que tuvo origen simultáneamente en el Padre Universal y el Hijo Eterno. El Miguel de Nebadon es el «Hijo unigénito»\footnote{\textit{Hijo unigénito}: Sal 2:7; Jn 1:14,18; Jn 3:16,18; Hch 13:33; Heb 1:5; Heb 5:5; 1 Jn 4:9.} que personaliza este 611.121{\textordmasculine} concepto universal de divinidad y de infinidad. Su sede se encuentra en la triple mansión de luz en Salvington. Y esta morada está dispuesta así porque Miguel ha experimentado el modo de vivir de las tres fases de la existencia de las criaturas inteligentes: la espiritual, la morontial y la material. Debido al nombre asociado a su séptima y última donación en Urantia, a veces se le llama Cristo Miguel.

\par
%\textsuperscript{(366.3)}
\textsuperscript{33:1.2} Nuestro Hijo Creador no es el Hijo Eterno, el asociado existencial paradisiaco del Padre Universal y del Espíritu Infinito. Miguel de Nebadon no es un miembro de la Trinidad del Paraíso. Sin embargo, nuestro Hijo Maestro posee en su reino todos los atributos y poderes divinos que el mismo Hijo Eterno manifestaría si estuviera efectivamente presente en Salvington y ejerciera su actividad en Nebadon. Miguel posee incluso un poder y una autoridad adicionales, porque no sólo personifica al Hijo Eterno, sino que también representa plenamente y expresa efectivamente la presencia de personalidad del Padre Universal para este universo local, y en él. Representa incluso al Padre-Hijo. Estas relaciones hacen de un Hijo Creador el más poderoso, polifacético e influyente de todos los seres divinos capaces de administrar directamente los universos evolutivos y de ponerse en contacto personal con las criaturas inmaduras.

\par
%\textsuperscript{(366.4)}
\textsuperscript{33:1.3} Desde la sede del universo local, nuestro Hijo Creador ejerce el mismo poder de atracción espiritual, la misma gravedad espiritual, que el Hijo Eterno del Paraíso ejercería si estuviera personalmente presente en Salvington, e incluso \textit{más aún}; este Hijo del Universo es también la personificación del Padre Universal para el universo de Nebadon. Los Hijos Creadores son los centros de personalidad para las fuerzas espirituales del Padre-Hijo Paradisiacos. Los Hijos Creadores son las focalizaciones finales del poder y de la personalidad de los poderosos atributos espacio-temporales de Dios Séptuple.

\par
%\textsuperscript{(367.1)}
\textsuperscript{33:1.4} El Hijo Creador personaliza la vicegerencia del Padre Universal, es el coordinado en divinidad del Hijo Eterno, y el asociado creativo del Espíritu Infinito. A todos los efectos prácticos, el Hijo Soberano es Dios\footnote{\textit{El Padre y el Hijo son uno}: Jn 1:1; 5:17-18; 10:30,38; 14:7-11,20; 17:11,21-22.} para nuestro universo y todos sus mundos habitados. Personifica todo lo que los mortales evolutivos pueden comprender con discernimiento de las Deidades del Paraíso\footnote{\textit{Ver al hijo es ver al Padre}: Jn 12:45; 14:7-9.}. Este Hijo y su Espíritu asociado \textit{son} vuestros padres creadores. Para vosotros, Miguel, el Hijo Creador, es la personalidad suprema; para vosotros, el Hijo Eterno es supersupremo ---una personalidad infinita de la Deidad.

\par
%\textsuperscript{(367.2)}
\textsuperscript{33:1.5} En la persona del Hijo Creador tenemos a un gobernante y a un padre divino que es exactamente tan poderoso, eficaz y benefactor como lo serían el Padre Universal y el Hijo Eterno si los dos estuvieran presentes en Salvington y se ocuparan de la administración de los asuntos del universo de Nebadon.

\section*{2. El Soberano de Nebadon}
\par
%\textsuperscript{(367.3)}
\textsuperscript{33:2.1} El observar a los Hijos Creadores revela que algunos se parecen más al Padre, otros al Hijo, mientras que otros son una mezcla de sus dos padres infinitos. Nuestro Hijo Creador manifiesta muy claramente unas características y unos atributos que se parecen más a los del Hijo Eterno.

\par
%\textsuperscript{(367.4)}
\textsuperscript{33:2.2} Miguel eligió organizar este universo local, y ahora reina aquí de manera suprema. Su poder personal está limitado por los circuitos gravitatorios preexistentes centrados en el Paraíso, y por el hecho de que los Ancianos de los Días del gobierno superuniversal se reservan todos los juicios ejecutivos finales relacionados con la extinción de la personalidad. La personalidad es el don exclusivo del Padre, pero los Hijos Creadores, con la aprobación del Hijo Eterno, inician nuevos proyectos de criaturas, y con la cooperación de trabajo de sus Espíritus asociados pueden intentar nuevas transformaciones de la energía-materia.

\par
%\textsuperscript{(367.5)}
\textsuperscript{33:2.3} Miguel es la personificación del Padre-Hijo Paradisiacos para, y en, el universo local de Nebadon; por consiguiente, cuando el Espíritu Madre Creativo, que representa al Espíritu Infinito en el universo local, se subordinó a Cristo Miguel cuando éste regresó de su donación final en Urantia, el Hijo Maestro adquirió con ello la jurisdicción sobre «todos los poderes en el cielo y en la Tierra»\footnote{\textit{Todo poder en el cielo y en la Tierra}: Mt 28:18.}.

\par
%\textsuperscript{(367.6)}
\textsuperscript{33:2.4} Esta subordinación de las Ministras Divinas a los Hijos Creadores de los universos locales convierte a estos Hijos Maestros en los depositarios personales de la divinidad, manifestable de manera finita, del Padre, del Hijo y del Espíritu, mientras que las experiencias donadoras de los Migueles bajo la forma de sus criaturas los cualifican para representar la divinidad experiencial del Ser Supremo. No existen otros seres en los universos que hayan agotado personalmente así los potenciales de la experiencia finita actual, y no hay otros seres en los universos que posean unas aptitudes semejantes para ejercer la soberanía solitaria.

\par
%\textsuperscript{(367.7)}
\textsuperscript{33:2.5} Aunque la sede central de Miguel está situada oficialmente en Salvington, la capital de Nebadon, pasa una gran parte de su tiempo visitando las sedes de las constelaciones y de los sistemas, e incluso los planetas individuales. Viaja periódicamente al Paraíso y con frecuencia a Uversa, donde mantiene sesiones con los Ancianos de los Días. Cuando está fuera de Salvington, Gabriel ocupa su lugar y actúa entonces como regente del universo de Nebadon.

\section*{3. El Hijo y el Espíritu del universo}
\par
%\textsuperscript{(368.1)}
\textsuperscript{33:3.1} Aunque el Espíritu Infinito impregna todos los universos del tiempo y del espacio, actúa desde la sede de cada universo local como una focalización especializada que adquiere todas las cualidades de la personalidad mediante la técnica de la cooperación creativa con el Hijo Creador. En lo que se refiere a un universo local, la autoridad administrativa de un Hijo Creador es suprema; el Espíritu Infinito, bajo la forma de Ministra Divina, es totalmente cooperativo aunque está perfectamente coordinado.

\par
%\textsuperscript{(368.2)}
\textsuperscript{33:3.2} El Espíritu Madre Universal de Salvington, la asociada de Miguel en el control y la administración de Nebadon, pertenece al sexto grupo de los Espíritus Supremos y lleva el número 611.121 de esta orden. Se ofreció como voluntaria para acompañar a Miguel cuando éste fue liberado de sus obligaciones paradisiacas, y desde entonces siempre ha trabajado con él para crear y gobernar su universo.

\par
%\textsuperscript{(368.3)}
\textsuperscript{33:3.3} El Hijo Maestro Creador es el soberano personal de su universo, pero en todos los detalles de la administración, el Espíritu del Universo es codirector con el Hijo. Aunque el Espíritu siempre reconoce al Hijo como soberano y gobernante, el Hijo siempre le concede al Espíritu una posición coordinada y una autoridad igual a la suya en todos los asuntos del reino. En todo su trabajo de amor y de donación de la vida, el Hijo Creador está siempre y para siempre perfectamente apoyado y hábilmente asistido por el Espíritu del Universo omnisapiente y siempre fiel, y por todo su séquito diversificado de personalidades angélicas. Esta Ministra Divina es en realidad la madre de los espíritus y de las personalidades espirituales, la consejera omnisapiente y siempre presente del Hijo Creador, una manifestación fiel y verdadera del Espíritu Infinito del Paraíso.

\par
%\textsuperscript{(368.4)}
\textsuperscript{33:3.4} El Hijo actúa como un padre en su universo local. El Espíritu, tal como lo podrían comprender las criaturas mortales, representa el papel de una madre, ayudando siempre al Hijo y permaneciendo eternamente indispensable para la administración del universo. En presencia de una insurrección, sólo el Hijo y sus Hijos asociados pueden actuar como liberadores. El Espíritu nunca puede oponerse a una rebelión ni defender la autoridad, pero el Espíritu apoya siempre al Hijo en todo lo que éste necesite experimentar en sus esfuerzos por estabilizar el gobierno y mantener la autoridad en los mundos contaminados por el mal o dominados por el pecado. Sólo un Hijo puede rehabilitar la obra que han creado juntos, pero ningún Hijo podría esperar el éxito final sin la cooperación incesante de la Ministra Divina y de su inmenso conjunto de asistentes espirituales, las hijas de Dios, que luchan con tanta valentía y fidelidad por el bienestar de los hombres mortales y por la gloria de sus padres divinos.

\par
%\textsuperscript{(368.5)}
\textsuperscript{33:3.5} Cuando el Hijo Creador finaliza su séptima y última donación como criatura, las incertidumbres del aislamiento periódico terminan para la Ministra Divina, y la asistente universal del Hijo se instala para siempre en la seguridad y en el control. Durante la entronización del Hijo Creador como Hijo Maestro, en el jubileo de los jubileos, es cuando el Espíritu del Universo reconoce por primera vez pública y universalmente, ante las multitudes reunidas, su subordinación al Hijo, prometiéndole fidelidad y obediencia. Este acontecimiento se produjo en Nebadon cuando Miguel regresó a Salvington después de su donación en Urantia. Antes de este importante acontecimiento, el Espíritu del Universo nunca había reconocido su subordinación al Hijo del Universo, y hasta después de esta renuncia voluntaria al poder y a la autoridad por parte del Espíritu no se pudo proclamar en verdad que «todos los poderes en el cielo y en la Tierra han sido puestos en sus manos»\footnote{\textit{Todo poder en el cielo y en la Tierra}: Mt 28:18.}.

\par
%\textsuperscript{(369.1)}
\textsuperscript{33:3.6} Después de esta promesa de subordinación por parte del Espíritu Madre Creativo, Miguel de Nebadon reconoció noblemente su eterna dependencia de su Espíritu compañero, nombró al Espíritu cogobernante de los dominios de su universo, y pidió a todas sus criaturas que prometieran su lealtad al Espíritu como lo habían hecho con el Hijo; entonces se promulgó y se publicó la «Proclamación final de Igualdad». Aunque era el soberano de este universo local, el Hijo proclamó a los mundos el hecho de que el Espíritu era igual a él en todos los dones de la personalidad y en todos los atributos del carácter divino. Y esto se convierte en el modelo trascendente para organizar y dirigir la familia, incluso entre las criaturas humildes de los mundos del espacio. Éste es, de hecho y en verdad, el elevado ideal de la familia y de la institución humana del matrimonio voluntario.

\par
%\textsuperscript{(369.2)}
\textsuperscript{33:3.7} El Hijo y el Espíritu presiden ahora el universo de manera muy similar a como un padre y una madre velan y cuidan a su familia de hijos e hijas. No está totalmente fuera de lugar referirse al Espíritu del Universo como la compañera creativa del Hijo Creador, y considerar a las criaturas de los reinos como sus hijos e hijas ---una familia grande y gloriosa, que exige responsabilidades incalculables y cuidados sin fin.

\par
%\textsuperscript{(369.3)}
\textsuperscript{33:3.8} El Hijo inicia la creación de ciertos hijos del universo, mientras que el Espíritu tiene la única responsabilidad de traer a la existencia a las numerosas órdenes de personalidades espirituales que ayudan y sirven bajo la dirección y la guía de este mismo Espíritu Madre. En la creación de otros tipos de personalidades universales, tanto el Hijo como el Espíritu actúan juntos, y en ningún acto creativo ninguno de ellos hace nada sin el consejo y la aprobación del otro.

\section*{4. Gabriel ---el Jefe Ejecutivo}
\par
%\textsuperscript{(369.4)}
\textsuperscript{33:4.1} La Radiante Estrella Matutina es la personalización del primer concepto de la identidad y del ideal de personalidad concebido por el Hijo Creador y por la manifestación del Espíritu Infinito en el universo local. Retrocediendo a los primeros tiempos del universo local, antes de la unión del Hijo Creador y del Espíritu Madre en los lazos de una asociación creativa, allá por las épocas anteriores al comienzo de la creación de su polifacética familia de hijos e hijas, el primer acto conjunto de la asociación inicial y libre de estas dos personas divinas dio como resultado la creación de la personalidad espiritual más elevada surgida del Hijo y del Espíritu, la Radiante Estrella Matutina.

\par
%\textsuperscript{(369.5)}
\textsuperscript{33:4.2} En cada universo local sólo nace un ser con esta sabiduría y esta majestad. El Padre Universal y el Hijo Eterno pueden crear un número ilimitado de Hijos iguales en divinidad a ellos mismos, y de hecho lo hacen; pero estos Hijos, en unión con las Hijas del Espíritu Infinito, sólo pueden crear en cada universo una Radiante Estrella Matutina, un ser semejante a ellos mismos que comparte abundantemente sus naturalezas combinadas, pero no sus prerrogativas creadoras. Gabriel de Salvington se parece al Hijo del Universo en divinidad de naturaleza, aunque está considerablemente limitado en atributos de Deidad.

\par
%\textsuperscript{(369.6)}
\textsuperscript{33:4.3} Este primogénito de los padres de un nuevo universo es una personalidad única que posee muchas características maravillosas que no están presentes de manera visible en ninguno de sus progenitores, un ser de una variedad de talentos sin precedentes y de una brillantez inimaginable. Esta personalidad celestial engloba la voluntad divina del Hijo combinada con la imaginación creativa del Espíritu. Los pensamientos y los actos de la Radiante Estrella Matutina representarán siempre plenamente tanto al Hijo Creador como al Espíritu Creativo. Este ser también es capaz de comprender ampliamente y de establecer un contacto compasivo tanto con las huestes seráficas espirituales como con las criaturas volitivas materiales evolutivas.

\par
%\textsuperscript{(370.1)}
\textsuperscript{33:4.4} La Radiante Estrella Matutina no es un creador, pero es un maravilloso administrador, es el representante administrativo personal del Hijo Creador. Aparte de la creación y de la concesión de la vida, el Hijo y el Espíritu nunca deliberan sobre importantes procedimientos universales sin la presencia de Gabriel.

\par
%\textsuperscript{(370.2)}
\textsuperscript{33:4.5} Gabriel de Salvington es el jefe ejecutivo del universo de Nebadon y el árbitro de todas las apelaciones ejecutivas relacionadas con su administración. Este ejecutivo del universo fue creado plenamente dotado para su trabajo, pero ha adquirido experiencia con el crecimiento y la evolución de nuestra creación local.

\par
%\textsuperscript{(370.3)}
\textsuperscript{33:4.6} Gabriel es el director en jefe que ejecuta los mandatos superuniversales relacionados con los asuntos no personales del universo local. La mayor parte de las materias relativas a los juicios en masa y a las resurrecciones dispensacionales, juzgadas por los Ancianos de los Días, son también delegadas en Gabriel y en su estado mayor para que las ejecuten. Gabriel es así el jefe ejecutivo combinado de los gobernantes del superuniverso y del universo local. Tiene a su mando a un cuerpo capaz de asistentes administrativos, creados para su trabajo especial, que no han sido revelados a los mortales evolutivos. Además de estos asistentes, Gabriel puede emplear todas las órdenes de seres celestiales que ejercen su actividad en Nebadon, y es también el comandante en jefe de «los ejércitos del cielo»\footnote{\textit{Los ejércitos celestiales}: Ap 19:14.} ---de las huestes celestiales.

\par
%\textsuperscript{(370.4)}
\textsuperscript{33:4.7} Gabriel y su estado mayor no son educadores; son administradores. Nunca se ha sabido que hayan dejado su trabajo habitual, salvo cuando Miguel se encarnaba para llevar a cabo una donación como criatura. Durante estas donaciones, Gabriel siempre tenía en cuenta la voluntad del Hijo encarnado, y con la colaboración del Unión de los Días, se convirtió en el director efectivo de los asuntos del universo durante las últimas donaciones. Gabriel ha estado estrechamente identificado con la historia y el desarrollo de Urantia desde la donación humana de Miguel.

\par
%\textsuperscript{(370.5)}
\textsuperscript{33:4.8} Aparte de encontrar a Gabriel en los mundos de donación y en las épocas de los llamamientos nominales durante las resurrecciones generales y especiales, los mortales raramente lo encontrarán mientras ascienden por el universo local hasta que no sean admitidos en el trabajo administrativo de la creación local. Como administradores de cualquier tipo o de cualquier grado, estaréis bajo la dirección de Gabriel.

\section*{5. Los Embajadores de la Trinidad}
\par
%\textsuperscript{(370.6)}
\textsuperscript{33:5.1} La administración de las personalidades con origen en la Trinidad termina en el gobierno de los superuniversos. Los universos locales están caracterizados por una supervisión doble, el comienzo del concepto padre-madre. El padre del universo es el Hijo Creador; la madre del universo es la Ministra Divina, el Espíritu Creativo del universo local. Sin embargo, cada universo local está bendecido con la presencia de ciertas personalidades del universo central y del Paraíso. A la cabeza de este grupo paradisiaco en Nebadon se encuentra el embajador de la Trinidad del Paraíso ---Emmanuel de Salvington--- el Unión de los Días asignado al universo local de Nebadon. En cierto sentido, este elevado Hijo de la Trinidad es también el representante personal del Padre Universal ante la corte del Hijo Creador; de ahí su nombre Emmanuel.

\par
%\textsuperscript{(370.7)}
\textsuperscript{33:5.2} Emmanuel de Salvington, número 611.121 de la sexta orden de Personalidades Trinitarias Supremas, es un ser de una dignidad sublime y de una condescendencia tan magnífica que rehúsa el culto y la adoración de todas las criaturas vivientes. Se distingue por ser la única personalidad en todo Nebadon que nunca ha reconocido su subordinación a su hermano Miguel. Actúa como asesor del Hijo Soberano, pero sólo ofrece sus consejos si se le solicitan. En ausencia del Hijo Creador puede presidir cualquier alto consejo del universo, pero no participa de otra manera en los asuntos ejecutivos del universo a menos que se le solicite.

\par
%\textsuperscript{(371.1)}
\textsuperscript{33:5.3} Este embajador del Paraíso en Nebadon no está sometido a la jurisdicción del gobierno del universo local. Tampoco ejerce una jurisdicción autorizada sobre los asuntos ejecutivos de un universo local en evolución, salvo en lo que se refiere a la supervisión de sus hermanos coordinados, los Fieles de los Días, que sirven en las sedes de las constelaciones.

\par
%\textsuperscript{(371.2)}
\textsuperscript{33:5.4} Al igual que el Unión de los Días, los Fieles de los Días nunca proponen su asesoramiento ni ofrecen su ayuda a los gobernantes de las constelaciones a menos que se les pida. Estos embajadores del Paraíso ante las constelaciones representan la presencia personal final de los Hijos Estacionarios de la Trinidad que ejercen sus funciones consultivas en los universos locales. Las constelaciones están relacionadas más estrechamente con la administración superuniversal que los sistemas locales, los cuales están administrados exclusivamente por personalidades nativas del universo local.

\section*{6. La administración general}
\par
%\textsuperscript{(371.3)}
\textsuperscript{33:6.1} Gabriel es el jefe ejecutivo y el administrador efectivo de Nebadon. El hecho de que Miguel se ausente de Salvington no interfiere de ninguna manera la conducta ordenada de los asuntos del universo. Durante la ausencia de Miguel, como lo hizo recientemente para reunirse con los Hijos Maestros de Orvonton en el Paraíso, Gabriel es el regente del universo. En esos momentos, Gabriel siempre busca el consejo de Emmanuel de Salvington para todos los problemas importantes.

\par
%\textsuperscript{(371.4)}
\textsuperscript{33:6.2} El Padre Melquisedek es el primer ayudante de Gabriel. Cuando la Radiante Estrella Matutina está ausente de Salvington, sus responsabilidades son asumidas por este Hijo Melquisedek original.

\par
%\textsuperscript{(371.5)}
\textsuperscript{33:6.3} Las diversas subadministraciones del universo tienen asignados ciertos ámbitos de responsabilidad especiales. Aunque el gobierno de un sistema se ocupa en general del bienestar de sus planetas, se preocupa más particularmente por el estado físico de los seres vivientes, por los problemas biológicos. Los gobernantes de la constelación prestan a su vez una atención especial a las condiciones sociales y gubernamentales que prevalecen en los diferentes planetas y sistemas. El gobierno de una constelación se preocupa principalmente de la unificación y la estabilización. Más arriba aún, los gobernantes del universo se ocupan más del estado espiritual de los reinos.

\par
%\textsuperscript{(371.6)}
\textsuperscript{33:6.4} Los embajadores son nombrados por decreto judicial y representan a los universos ante otros universos. Los cónsules representan a las constelaciones entre sí y ante la sede del universo; son nombrados por decreto legislativo y sólo ejercen sus funciones dentro de los confines del universo local. Los observadores son nombrados por un decreto ejecutivo del Soberano de un Sistema para representar a ese sistema ante otros sistemas y ante la capital de la constelación, y ellos también sólo desempeñan sus funciones dentro de los confines del universo local.

\par
%\textsuperscript{(371.7)}
\textsuperscript{33:6.5} Las transmisiones se emiten simultáneamente desde Salvington hacia las sedes de las constelaciones, las sedes de los sistemas y los planetas individuales. Todas las órdenes superiores de seres celestiales son capaces de utilizar este servicio para comunicarse con sus compañeros dispersos por todo el universo. La transmisión universal se extiende a todos los mundos habitados sin tener en cuenta su estado espiritual. La intercomunicación planetaria sólo se niega a aquellos mundos que están en cuarentena espiritual.

\par
%\textsuperscript{(372.1)}
\textsuperscript{33:6.6} Las transmisiones de las constelaciones se emiten periódicamente desde la sede de la constelación por el jefe de los Padres de la Constelación.

\par
%\textsuperscript{(372.2)}
\textsuperscript{33:6.7} Un grupo especial de seres que se encuentran en Salvington son los que cuentan, calculan y rectifican la cronología. El día oficial de Nebadon equivale a dieciocho días y seis horas del tiempo de Urantia, más dos minutos y medio. El año de Nebadon consiste en un segmento del tiempo del recorrido del universo en relación con el circuito de Uversa, y equivale a cien días del tiempo oficial del universo, unos cinco años del tiempo de Urantia.

\par
%\textsuperscript{(372.3)}
\textsuperscript{33:6.8} El tiempo de Nebadon, que se transmite desde Salvington, es el tiempo oficial para todas las constelaciones y todos los sistemas de este universo local. Cada constelación dirige sus asuntos según el tiempo de Nebadon, pero los sistemas mantienen su propia cronología, tal como lo hacen los planetas individuales.

\par
%\textsuperscript{(372.4)}
\textsuperscript{33:6.9} El día de Satania, tal como se calcula en Jerusem, es un poco menos (en 1 hora, 4 minutos y 15 segundos) de tres días del tiempo de Urantia. Estos tiempos se conocen generalmente como el tiempo de Salvington o universal, y el tiempo de Satania o del sistema. El tiempo oficial es el tiempo del universo.

\section*{7. Los tribunales de Nebadon}
\par
%\textsuperscript{(372.5)}
\textsuperscript{33:7.1} Miguel, el Hijo Maestro, sólo se preocupa de manera suprema de tres cosas: la creación, el sostenimiento y el ministerio. No participa personalmente en la tarea judicial del universo. Los Creadores nunca juzgan a sus criaturas; esta función pertenece exclusivamente a las criaturas que poseen una gran formación y una experiencia real como criaturas.

\par
%\textsuperscript{(372.6)}
\textsuperscript{33:7.2} Todo el mecanismo judicial de Nebadon se encuentra bajo la supervisión de Gabriel. Los tribunales supremos, situados en Salvington, se ocupan de los problemas que tienen una importancia general para el universo y de los casos de apelación que proceden de los tribunales de los sistemas. Estas cortes universales tienen setenta ramas y funcionan en siete divisiones de diez secciones cada una. En todos los asuntos a juzgar, la presidencia es ejercida por una doble magistratura compuesta por un juez con antecedentes perfectos y un magistrado con experiencia ascendente.

\par
%\textsuperscript{(372.7)}
\textsuperscript{33:7.3} En lo que se refiere a la jurisdicción, los tribunales del universo local están limitados en las materias siguientes:

\par
%\textsuperscript{(372.8)}
\textsuperscript{33:7.4} 1. La administración del universo local se ocupa de la creación, la evolución, el mantenimiento y el ministerio. Por consiguiente, a los tribunales del universo se les rehúsa el derecho de juzgar los casos en los que está implicada la cuestión de la vida y de la muerte eternas. Esto no se refiere a la muerte natural tal como ésta prevalece en Urantia, pero si la cuestión del derecho a la existencia continuada, a la vida eterna, ha de ser juzgada, tiene que remitirse a los tribunales de Orvonton, y si el fallo es desfavorable para el individuo, todas las sentencias de extinción se ejecutan bajo las órdenes, y a través de los agentes, de los dirigentes del supergobierno.

\par
%\textsuperscript{(372.9)}
\textsuperscript{33:7.5} 2. La negligencia o la deserción de cualquier Hijo de Dios del Universo Local, que ponga en peligro su estado y su autoridad como Hijo, nunca se juzga en los tribunales de un Hijo; una desavenencia de este tipo sería llevada inmediatamente ante los tribunales del superuniverso.

\par
%\textsuperscript{(372.10)}
\textsuperscript{33:7.6} 3. La cuestión de readmitir a cualquier parte constituyente de un universo local ---por ejemplo un sistema local--- en la comunidad del pleno estado espiritual de la creación local, después de haber estado aislada espiritualmente, debe acordarse en la alta asamblea del superuniverso.

\par
%\textsuperscript{(373.1)}
\textsuperscript{33:7.7} En todos los demás casos, los tribunales de Salvington son definitivos y supremos. Sus decisiones y decretos no se pueden apelar ni eludir.

\par
%\textsuperscript{(373.2)}
\textsuperscript{33:7.8} Por muy injustamente que las controversias humanas parezcan juzgarse a veces en Urantia, la justicia y la equidad divina prevalecen en el universo. Vivís en un universo bien ordenado, y podéis contar con que tarde o temprano seréis tratados con justicia, e incluso con misericordia.

\section*{8. Las funciones legislativas y ejecutivas}
\par
%\textsuperscript{(373.3)}
\textsuperscript{33:8.1} En Salvington, la sede de Nebadon, no existen cuerpos verdaderamente legislativos. Los mundos sede de los universos se ocupan ampliamente de los juicios. Las asambleas legislativas del universo local están situadas en las sedes de las cien constelaciones. Los sistemas se ocupan principalmente del trabajo ejecutivo y administrativo de las creaciones locales. Los Soberanos de los Sistemas y sus asociados hacen cumplir los mandatos legislativos de los gobernantes de las constelaciones y ejecutan los decretos judiciales de los tribunales supremos del universo.

\par
%\textsuperscript{(373.4)}
\textsuperscript{33:8.2} Aunque en la sede del universo no se decreta una verdadera legislación, en Salvington funciona una variedad de asambleas consultivas y de investigación compuestas y dirigidas de manera diversa de acuerdo con su alcance y su propósito. Algunas son permanentes y otras se disuelven después de conseguir sus objetivos.

\par
%\textsuperscript{(373.5)}
\textsuperscript{33:8.3} \textit{El consejo supremo} del universo local está compuesto de tres miembros de cada sistema y de siete representantes de cada constelación. Los sistemas aislados no tienen representación en esta asamblea, pero se les permite enviar a sus observadores, los cuales asisten a todas las deliberaciones y las estudian.

\par
%\textsuperscript{(373.6)}
\textsuperscript{33:8.4} \textit{Los cien consejos de sanciones supremas} también están situados en Salvington. Los presidentes de estos consejos componen el gabinete de trabajo directo de Gabriel.

\par
%\textsuperscript{(373.7)}
\textsuperscript{33:8.5} Todas las conclusiones de los altos consejos consultivos del universo se remiten, o bien a los cuerpos judiciales de Salvington, o a las asambleas legislativas de las constelaciones. Estos altos consejos no tienen autoridad ni poder para hacer cumplir sus recomendaciones. Si su informe está basado en las leyes fundamentales del universo, entonces los tribunales de Nebadon emitirán los mandatos de ejecución; pero si sus recomendaciones tienen que ver con las condiciones locales o de urgencia, han de enviarse a las asambleas legislativas de la constelación para su promulgación deliberativa, y luego a las autoridades del sistema para su ejecución. Estos altos consejos son en realidad las superlegislaturas del universo, pero funcionan sin la autoridad de decretar y sin el poder de ejecutar.

\par
%\textsuperscript{(373.8)}
\textsuperscript{33:8.6} Aunque hablamos de la administración del universo en términos de «tribunales» y de «asambleas», debe comprenderse que estas actuaciones espirituales son muy diferentes a las actividades más primitivas y materiales que llevan estos mismos nombres en Urantia.

\par
%\textsuperscript{(373.9)}
\textsuperscript{33:8.7} [Presentado por el Jefe de los Arcángeles de Nebadon.]