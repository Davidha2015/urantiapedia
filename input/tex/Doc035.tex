\chapter{Documento 35. Los Hijos de Dios de los universos locales}
\par
%\textsuperscript{(384.1)}
\textsuperscript{35:0.1} LOS Hijos de Dios presentados anteriormente han tenido un origen paradisiaco. Son los descendientes de los Gobernantes divinos de los dominios universales. Los Hijos Creadores pertenecen a la primera orden de filiación paradisiaca, y en Nebadon sólo hay uno de ellos, Miguel, el padre y soberano del universo. Los Hijos Avonales o Magistrales pertenecen a la segunda orden de filiación del Paraíso, y Nebadon tiene su contingente al completo ---1.062 miembros. Estos <<Cristos menores>> son tan eficaces y tan todopoderosos en sus donaciones planetarias como el Hijo Creador y Maestro lo fue en Urantia. Como la tercera orden tiene su origen en la Trinidad, no está registrada en un universo local, pero calculo que hay en Nebadon entre quince y veinte mil Hijos Instructores Trinitarios, aparte de los 9.642 asistentes trinitizados por las criaturas que sí están registrados. Estos Daynales del Paraíso no son ni magistrados ni administradores; son supereducadores.

\par
%\textsuperscript{(384.2)}
\textsuperscript{35:0.2} Los tipos de Hijos que vamos a estudiar tienen su origen en el universo local; son los descendientes de un Hijo Creador Paradisiaco en asociación variada con el Espíritu Madre Universal complementario. En estas narraciones mencionaremos las siguientes ordenes de filiación del universo local:

\par
%\textsuperscript{(384.3)}
\textsuperscript{35:0.3} 1. Los Hijos Melquisedeks.

\par
%\textsuperscript{(384.4)}
\textsuperscript{35:0.4} 2. Los Hijos Vorondadeks.

\par
%\textsuperscript{(384.5)}
\textsuperscript{35:0.5} 3. Los Hijos Lanonandeks.

\par
%\textsuperscript{(384.6)}
\textsuperscript{35:0.6} 4. Los Hijos Portadores de Vida.

\par
%\textsuperscript{(384.7)}
\textsuperscript{35:0.7} La Deidad trina del Paraíso actúa para crear tres órdenes de filiación: los Migueles, los Avonales y los Daynales. La Deidad doble del universo local, el Hijo y el Espíritu, también actúa para crear tres órdenes elevadas de Hijos: los Melquisedeks, los Vorondadeks y los Lanonandeks; y después de conseguir esta expresión triple, colaboran con el siguiente nivel de Dios Séptuple para engendrar la orden polifacética de los Portadores de Vida. Estos seres están clasificados con los Hijos descendentes de Dios, pero constituyen una forma de vida única y original en el universo. Todo el documento siguiente lo dedicaremos a su estudio.

\section*{1. El Padre Melquisedek}
\par
%\textsuperscript{(384.8)}
\textsuperscript{35:1.1} Después de traer a la existencia a sus ayudantes personales, tales como la Radiante Estrella Matutina y otras personalidades administrativas, de acuerdo con el propósito divino y los planes creativos de un universo dado, una nueva forma de unión creativa se produce entre el Hijo Creador y el Espíritu Creativo, la Hija del Espíritu Infinito en el universo local. La personalidad resultante de esta asociación creativa es el Melquisedek original ---el Padre Melquisedek--- ese ser único que colabora posteriormente con el Hijo Creador y el Espíritu Creativo para traer a la existencia a todo el grupo que lleva este nombre.

\par
%\textsuperscript{(385.1)}
\textsuperscript{35:1.2} El Padre Melquisedek actúa en el universo de Nebadon como el primer asociado ejecutivo de la Radiante Estrella Matutina. Gabriel se ocupa más de la política del universo, y Melquisedek de los procedimientos prácticos. Gabriel preside los tribunales y consejos regularmente constituidos de Nebadon, y Melquisedek las comisiones y los cuerpos consultivos especiales, extraordinarios y de emergencia. Gabriel y el Padre Melquisedek nunca están fuera de Salvington al mismo tiempo, porque en ausencia de Gabriel, el Padre Melquisedek actúa como jefe ejecutivo de Nebadon.

\par
%\textsuperscript{(385.2)}
\textsuperscript{35:1.3} Todos los Melquisedeks de nuestro universo fueron creados en el transcurso de un solo milenio del tiempo oficial por el Hijo Creador y el Espíritu Creativo en unión con el Padre Melquisedek. Como se trata de una orden de filiación en la que uno de sus propios miembros actuó como creador coordinado, los Melquisedeks son en parte de origen autónomo en su constitución, y en consecuencia son candidatos a llevar a cabo un tipo celestial de gobierno autónomo. Eligen periódicamente a su propio jefe administrativo por un período de siete años del tiempo oficial, y funcionan por lo demás como una orden que se regula ella misma, aunque el Melquisedek original ejerce ciertas prerrogativas inherentes a su calidad de coprogenitor. Este Padre Melquisedek designa de vez en cuando a ciertos individuos de su orden para que actúen como Portadores de Vida especiales en los mundos midsonitos, un tipo de planeta habitado hasta ahora no revelado en Urantia.

\par
%\textsuperscript{(385.3)}
\textsuperscript{35:1.4} Los Melquisedeks no ejercen ampliamente su actividad fuera del universo local, salvo cuando son llamados como testigos para los asuntos que están pendientes ante los tribunales del superuniverso, y cuando son designados como embajadores especiales, como a veces les sucede, para representar a un universo ante otro dentro del mismo superuniverso. El Melquisedek original o primogénito de cada universo puede siempre viajar libremente a los universos vecinos o al Paraíso para misiones relacionadas con los intereses y las obligaciones de su orden.

\section*{2. Los Hijos Melquisedeks}
\par
%\textsuperscript{(385.4)}
\textsuperscript{35:2.1} Los Melquisedeks son la primera orden de Hijos divinos que se acercan lo suficiente a la vida de las criaturas inferiores como para poder actuar directamente en el ministerio de elevar a los mortales, de servir a las razas evolutivas sin necesidad de encarnarse. Estos Hijos se hallan por naturaleza en el punto medio de la gran escala descendente de personalidades, encontrándose por su origen casi exactamente a medio camino entre la Divinidad más elevada y las criaturas dotadas de voluntad más humildes. De este modo se convierten en los intermediarios naturales entre los niveles superiores y divinos de existencia viviente y las formas inferiores, e incluso materiales, de vida de los mundos evolutivos. A las órdenes seráficas, a los ángeles, les encanta trabajar con los Melquisedeks; de hecho, todas las formas de vida inteligente encuentran en estos Hijos a unos amigos comprensivos, unos instructores compasivos y unos consejeros sabios.

\par
%\textsuperscript{(385.5)}
\textsuperscript{35:2.2} Los Melquisedeks son una orden que se gobierna de forma autónoma. En este grupo excepcional encontramos el primer intento de autodeterminación por parte de unos seres del universo local, y observamos el tipo más elevado de un verdadero gobierno autónomo. Estos Hijos organizan su propio mecanismo para administrar su grupo y su planeta nativo, así como las seis esferas asociadas y sus mundos tributarios. Y debemos indicar que nunca han abusado de sus prerrogativas; en todo el superuniverso de Orvonton, estos Hijos Melquisedeks no han traicionado nunca, ni una sola vez, la confianza depositada en ellos. Son la esperanza de todos los grupos del universo que aspiran a la autonomía; son el modelo de la autonomía gubernamental y la enseñan en todas las esferas de Nebadon. Todas las órdenes de seres inteligentes, las superiores que están por encima y las subordinadas que están por debajo, elogian sinceramente el gobierno de los Melquisedeks.

\par
%\textsuperscript{(386.1)}
\textsuperscript{35:2.3} La orden de filiación de los Melquisedeks ocupa la posición, y asume la responsabilidad, del hijo mayor en una gran familia. La mayor parte de su trabajo es regular y un poco rutinario, pero una gran parte del mismo es voluntario y totalmente autoimpuesto. La mayoría de las asambleas especiales que se reúnen de vez en cuando en Salvington se convocan a petición de los Melquisedeks. Estos Hijos patrullan su universo nativo por su propia iniciativa. Mantienen una organización autónoma dedicada al servicio de información universal, y presentan informes periódicos al Hijo Creador que son independientes de toda la información que llega a la sede del universo a través de los agentes regulares relacionados con la administración rutinaria del reino. Son observadores imparciales por naturaleza; gozan de la plena confianza de todas las clases de seres inteligentes.

\par
%\textsuperscript{(386.2)}
\textsuperscript{35:2.4} Los Melquisedeks actúan como tribunales de revisión itinerantes y consultivos de los reinos; estos Hijos del universo van a los mundos en pequeños grupos para servir como comisiones consultivas, tomar declaraciones, recibir sugerencias y actuar como consejeros, ayudando así a serenar las dificultades importantes y a resolver las graves diferencias que surgen de vez en cuando en los asuntos de los dominios evolutivos.

\par
%\textsuperscript{(386.3)}
\textsuperscript{35:2.5} Estos Hijos mayores de un universo son los ayudantes principales de la Radiante Estrella Matutina en la tarea de ejecutar los mandatos del Hijo Creador. Cuando un Melquisedek va a un mundo lejano en nombre de Gabriel, puede desempeñar las funciones, con miras a esa misión particular, en nombre de aquel que le envía, y en ese caso aparecerá en el planeta de destino con la plena autoridad de la Radiante Estrella Matutina. Esto es especialmente cierto en aquellas esferas donde un Hijo más elevado aún no ha aparecido en la similitud de las criaturas del reino.

\par
%\textsuperscript{(386.4)}
\textsuperscript{35:2.6} Cuando un Hijo Creador emprende su carrera de donación en un mundo evolutivo, va solo; pero cuando uno de sus hermanos del Paraíso, un Hijo Avonal, emprende una donación, va acompañado por los Melquisedeks que lo apoyan, doce en total, que contribuyen tan eficazmente al éxito de la misión donadora. También apoyan a los Avonales del Paraíso en sus misiones magistrales en los mundos habitados y, durante estas misiones, los Melquisedeks son visibles para los ojos de los mortales si el Hijo Avonal también se manifiesta de esta manera.

\par
%\textsuperscript{(386.5)}
\textsuperscript{35:2.7} No existe ninguna fase de las necesidades espirituales planetarias a la que no aporten su ministerio. Son los educadores que con tanta frecuencia consiguen que mundos enteros de vida avanzada reconozcan de manera plena y definitiva al Hijo Creador y a su Padre Paradisiaco.

\par
%\textsuperscript{(386.6)}
\textsuperscript{35:2.8} La sabiduría de los Melquisedeks es casi perfecta, pero su juicio no es infalible. Cuando están solos y aislados en sus misiones planetarias, a veces se han equivocado en cuestiones menores, es decir, han elegido hacer ciertas cosas que sus supervisores no han aprobado posteriormente. Este error de juicio incapacita temporalmente a un Melquisedek hasta que va a Salvington y, en el transcurso de una audiencia con el Hijo Creador, recibe la enseñanza que lo purificará eficazmente de la falta de armonía que provocó el desacuerdo con sus compañeros; y luego, después del descanso correccional, se reincorporará al servicio al tercer día. Pero estas inadaptaciones menores en la actividad de los Melquisedeks se han producido raras veces en Nebadon.

\par
%\textsuperscript{(387.1)}
\textsuperscript{35:2.9} Estos Hijos no forman una orden que aumente; su número es fijo, aunque varía en cada universo local. El número de Melquisedeks registrados en su planeta sede de Nebadon supera los diez millones.

\section*{3. Los mundos de los Melquisedeks}
\par
%\textsuperscript{(387.2)}
\textsuperscript{35:3.1} Los Melquisedeks ocupan un mundo propio cerca de Salvington, la sede del universo. Esta esfera, llamada Melquisedek, es el mundo piloto del circuito de setenta esferas primarias de Salvington, cada una de las cuales está rodeada por seis esferas tributarias dedicadas a actividades especializadas. A estas esferas maravillosas ---setenta primarias y 420 tributarias--- se les llama a menudo la Universidad Melquisedek. Los mortales ascendentes de todas las constelaciones de Nebadon pasan por la formación de estos 490 mundos para adquirir el estado residencial en Salvington. Pero la educación de los ascendentes no es más que una fase de las múltiples actividades que tienen lugar en el grupo de esferas arquitectónicas de Salvington.

\par
%\textsuperscript{(387.3)}
\textsuperscript{35:3.2} Las 490 esferas del circuito de Salvington están divididas en diez grupos, y cada uno contiene siete esferas primarias y cuarenta y dos tributarias. Cada uno de estos grupos se encuentra bajo la supervisión general de una de las órdenes principales de la vida universal. El primer grupo, que abarca el mundo piloto y las seis esferas primarias siguientes en la procesión planetaria circundante, se encuentra bajo la supervisión de los Melquisedeks. Estos mundos Melquisedeks son los siguientes:

\par
%\textsuperscript{(387.4)}
\textsuperscript{35:3.3} 1. El mundo piloto ---el mundo nativo de los Hijos Melquisedeks.

\par
%\textsuperscript{(387.5)}
\textsuperscript{35:3.4} 2. El mundo de las escuelas de la vida física y de los laboratorios de las energías vivientes.

\par
%\textsuperscript{(387.6)}
\textsuperscript{35:3.5} 3. El mundo de la vida morontial.

\par
%\textsuperscript{(387.7)}
\textsuperscript{35:3.6} 4. La esfera de la vida espiritual inicial.

\par
%\textsuperscript{(387.8)}
\textsuperscript{35:3.7} 5. El mundo de la vida espiritual intermedia.

\par
%\textsuperscript{(387.9)}
\textsuperscript{35:3.8} 6. La esfera de la vida espiritual avanzada.

\par
%\textsuperscript{(387.10)}
\textsuperscript{35:3.9} 7. El dominio de la autorrealización coordinada y suprema.

\par
%\textsuperscript{(387.11)}
\textsuperscript{35:3.10} Los seis mundos tributarios de cada una de estas esferas Melquisedeks están dedicados a las actividades relacionadas con el trabajo de la esfera primaria asociada.

\par
%\textsuperscript{(387.12)}
\textsuperscript{35:3.11} El mundo piloto, la esfera \textit{Melquisedek}, es el punto de encuentro común para todos los seres que se ocupan de educar y de espiritualizar a los mortales ascendentes del tiempo y del espacio. Para un ascendente, este mundo es probablemente el lugar más interesante de todo Nebadon. Todos los mortales evolutivos que han terminado su formación en las constelaciones están destinados a aterrizar en Melquisedek, donde son iniciados en el régimen de las disciplinas y de la progresión espiritual del sistema educativo de Salvington. Nunca olvidaréis las reacciones de vuestro primer día de vida en este mundo único, ni siquiera después de que hayáis alcanzado vuestro destino en el Paraíso.

\par
%\textsuperscript{(387.13)}
\textsuperscript{35:3.12} Los mortales ascendentes residen en el mundo Melquisedek mientras continúan su formación en los seis planetas circundantes de educación especializada. Y este mismo método se aplica durante toda la estancia en los setenta mundos culturales, las esferas primarias del circuito de Salvington.

\par
%\textsuperscript{(387.14)}
\textsuperscript{35:3.13} Muchas actividades diversas ocupan el tiempo de los numerosos seres que residen en los seis mundos tributarios de la esfera Melquisedek, pero en lo que se refiere a los mortales ascendentes, estos satélites se dedican a las fases especiales de estudio siguientes:

\par
%\textsuperscript{(388.1)}
\textsuperscript{35:3.14} 1. La esfera número uno se ocupa de revisar la vida planetaria inicial de los mortales ascendentes. Este trabajo se efectúa en clases compuestas por aquellos que proceden de un mundo dado de origen mortal. Los que provienen de Urantia realizan juntos esta revisión experiencial.

\par
%\textsuperscript{(388.2)}
\textsuperscript{35:3.15} 2. El trabajo especial de la esfera número dos consiste en una revisión similar de las experiencias vividas en los mundos de las mansiones que rodean al primer satélite de la sede del sistema local.

\par
%\textsuperscript{(388.3)}
\textsuperscript{35:3.16} 3. Las revisiones de esta esfera están relacionadas con la estancia en la capital del sistema local y abarcan las actividades del resto de los mundos arquitectónicos del grupo que forma la sede del sistema.

\par
%\textsuperscript{(388.4)}
\textsuperscript{35:3.17} 4. La cuarta esfera se ocupa de revisar las experiencias de los setenta mundos tributarios de la constelación y de sus esferas asociadas.

\par
%\textsuperscript{(388.5)}
\textsuperscript{35:3.18} 5. En la quinta esfera se realiza la revisión de la estancia ascendente en el mundo sede de la constelación.

\par
%\textsuperscript{(388.6)}
\textsuperscript{35:3.19} 6. El tiempo se dedica, en la esfera número seis, a intentar correlacionar estas cinco épocas y lograr así una coordinación de la experiencia, como preparación para entrar en las escuelas primarias Melquisedeks de formación universal.

\par
%\textsuperscript{(388.7)}
\textsuperscript{35:3.20} Las escuelas de administración universal y de sabiduría espiritual están situadas en el mundo nativo de los Melquisedeks, donde también se encuentran las escuelas dedicadas a una sola línea de investigación, tales como la energía, la materia, la organización, la comunicación, los archivos, la ética y la existencia comparada de las criaturas.

\par
%\textsuperscript{(388.8)}
\textsuperscript{35:3.21} En la Facultad Melquisedek de Dotación Espiritual, todas las órdenes de Hijos de Dios, incluidas las del Paraíso, cooperan con los Melquisedeks y los educadores seráficos para formar a las multitudes de evángeles del destino que salen a proclamar la libertad espiritual y la filiación divina incluso hasta los mundos lejanos del universo. Esta facultad particular de la Universidad Melquisedek es una institución exclusiva del universo; los visitantes estudiantiles procedentes de otros reinos no son admitidos aquí.

\par
%\textsuperscript{(388.9)}
\textsuperscript{35:3.22} El curso de formación más elevado en administración universal es impartido por los Melquisedeks en su mundo nativo. Esta Facultad de Ética Superior está presidida por el Padre Melquisedek original. Es a estas facultades donde los diversos universos envían sus estudiantes de intercambio. Aunque el joven universo de Nebadon se encuentra en un bajo nivel en la escala de los universos en cuanto a los logros espirituales y a un desarrollo ético elevado, sin embargo, nuestros problemas administrativos han convertido de tal manera a todo el universo en una inmensa clínica para otras creaciones cercanas, que las facultades Melquisedeks están atestadas de visitantes estudiantiles y de observadores de otros reinos. Además del inmenso grupo de inscritos locales, siempre hay más de cien mil estudiantes extranjeros que asisten a las escuelas Melquisedeks, porque la orden de los Melquisedeks de Nebadon es famosa en todo Splandon.

\section*{4. El trabajo especial de los Melquisedeks}
\par
%\textsuperscript{(388.10)}
\textsuperscript{35:4.1} Una rama sumamente especializada de las actividades de los Melquisedeks está relacionada con la supervisión de la carrera morontial progresiva de los mortales ascendentes. Una gran parte de esta formación está dirigida por los pacientes y sabios ministros seráficos, ayudados por los mortales que han ascendido hasta unos niveles relativamente superiores de consecución universal, pero todo este trabajo educativo se encuentra bajo la supervisión general de los Melquisedeks en asociación con los Hijos Instructores Trinitarios.

\par
%\textsuperscript{(389.1)}
\textsuperscript{35:4.2} Aunque las órdenes de los Melquisedeks se dedican principalmente al extenso sistema educativo y al régimen de formación experiencial del universo local, también actúan en misiones excepcionales y en circunstancias poco habituales. En un universo evolutivo que terminará por contener aproximadamente diez millones de mundos habitados, muchas cosas fuera de lo normal están destinadas a suceder, y en estos casos de emergencia es cuando actúan los Melquisedeks. En Edentia, la sede de vuestra constelación, se les conoce como Hijos de emergencia. Siempre están preparados para servir en todas las situaciones de necesidad ---físicas, intelectuales o espirituales--- ya sea en un planeta, en un sistema, en una constelación o en el universo. En cualquier momento y lugar en que se necesite una ayuda especial, allí encontraréis a uno o más Hijos Melquisedeks.

\par
%\textsuperscript{(389.2)}
\textsuperscript{35:4.3} Cuando algún aspecto del plan del Hijo Creador está amenazado de fracaso, un Melquisedek irá inmediatamente a prestar ayuda. Pero raras veces se les pide que actúen en presencia de una rebelión pecaminosa, como la que se produjo en Satania.

\par
%\textsuperscript{(389.3)}
\textsuperscript{35:4.4} Los Melquisedeks son los primeros en actuar en todas las emergencias de cualquier naturaleza en todos los mundos donde viven las criaturas volitivas. A veces actúan como guardianes temporales de los planetas desobedientes, sirviendo como síndicos de un gobierno planetario rebelde. En una crisis planetaria, estos Hijos Melquisedeks sirven en muchas tareas excepcionales. A este tipo de Hijo le resulta fácil hacerse visible a los seres mortales, y a veces un miembro de esta orden se ha encarnado incluso en la similitud de la carne mortal. Siete veces en Nebadon ha servido un Melquisedek en un mundo evolutivo en la similitud de la carne mortal, y estos Hijos han aparecido en numerosas ocasiones en la similitud de otras órdenes de criaturas del universo. Son en verdad los ministros de urgencia polifacéticos y voluntarios para todas las órdenes de inteligencias del universo y para todos los mundos y sistemas de mundos.

\par
%\textsuperscript{(389.4)}
\textsuperscript{35:4.5} El Melquisedek que vivió en Urantia en los tiempos de Abraham fue conocido localmente como el Príncipe de Salem\footnote{\textit{Melquisedek, Príncipe de Salem}: Gn 14:18; Heb 7:1-3.}, porque presidía una pequeña colonia de buscadores de la verdad que residían en un lugar llamado Salem. Se ofreció como voluntario para encarnarse en la similitud de la carne mortal, y lo hizo con la aprobación de los síndicos Melquisedeks del planeta, los cuales temían que la luz de la vida podría extinguirse durante aquel período de oscuridad espiritual creciente. Fomentó la verdad de su época y la transmitió de manera segura a Abraham y a sus asociados.

\section*{5. Los Hijos Vorondadeks}
\par
%\textsuperscript{(389.5)}
\textsuperscript{35:5.1} Después de la creación de los ayudantes personales y del primer grupo de los polifacéticos Melquisedeks, el Hijo Creador y el Espíritu Creativo del universo local planificaron, y trajeron a la existencia, a la segunda gran orden variada de filiación universal: los Vorondadeks. Se les conoce de manera más general como los Padres de las Constelaciones porque un Hijo de esta orden se encuentra uniformemente a la cabeza del gobierno de cada constelación en todos los universos locales.

\par
%\textsuperscript{(389.6)}
\textsuperscript{35:5.2} El número de Vorondadeks varía en cada universo local, y el número de ellos registrado en Nebadon se eleva exactamente a un millón. Estos Hijos, al igual que sus coordinados los Melquisedeks, no poseen el poder de reproducirse. No existe ningún método conocido por el cual puedan incrementar su número.

\par
%\textsuperscript{(389.7)}
\textsuperscript{35:5.3} Estos Hijos forman, en muchos aspectos, un cuerpo autónomo; como individuos, como grupos, e incluso como totalidad, se autodeterminan en gran medida como lo hacen los Melquisedeks, pero los Vorondadeks no ejercen sus funciones en una variedad tan amplia de actividades. No tienen la misma brillante diversidad de talentos que sus hermanos Melquisedeks, pero como gobernantes y administradores previsores son incluso más fiables y eficaces. Administrativamente tampoco se parecen por completo a sus subordinados, los Lanonandeks Soberanos de los Sistemas, pero superan a todas las órdenes de filiación del universo en la estabilidad de sus propósitos y en la divinidad de sus juicios.

\par
%\textsuperscript{(390.1)}
\textsuperscript{35:5.4} Aunque los fallos y las decisiones de esta orden de Hijos están siempre de acuerdo con el espíritu de filiación divina y en armonía con la política del Hijo Creador, han sido citados a causa de sus errores ante el Hijo Creador, y sus decisiones relativas a detalles técnicos a veces han sido revocadas en la apelación a los tribunales superiores del universo. Pero estos Hijos raras veces caen en el error, y nunca han emprendido una rebelión; en toda la historia de Nebadon nunca se ha oído decir que un Vorondadek haya cometido desacato al gobierno del universo.

\par
%\textsuperscript{(390.2)}
\textsuperscript{35:5.5} El servicio de los Vorondadeks en los universos locales es amplio y variado. Sirven como embajadores ante otros universos y como cónsules representando a las constelaciones dentro de su universo nativo. De todas las órdenes de filiación de un universo local, es a ellos a quienes más a menudo se les confía la plena delegación de los poderes soberanos a ejercer en las situaciones críticas del universo.

\par
%\textsuperscript{(390.3)}
\textsuperscript{35:5.6} En aquellos mundos aislados en las tinieblas espirituales, en aquellas esferas que han sufrido el aislamiento planetario debido a la rebelión y a la negligencia, un observador Vorondadek está generalmente presente hasta el restablecimiento del estado normal. En ciertos casos de emergencia, este observador Altísimo podría ejercer una autoridad absoluta y arbitraria sobre todos los seres celestiales destinados en ese planeta. Los archivos de Salvington mencionan que los Vorondadeks han ejercido a veces esta autoridad como regentes Altísimos de tales planetas. Y esto también ha sucedido incluso en los mundos habitados que no han sido afectados por la rebelión.

\par
%\textsuperscript{(390.4)}
\textsuperscript{35:5.7} A menudo, un cuerpo de doce o más Hijos Vorondadeks forman un alto tribunal de revisión y de apelación con respecto a casos especiales que afectan al estado de un planeta o de un sistema. Pero su trabajo está principalmente relacionado con las funciones legislativas autóctonas de los gobiernos de las constelaciones. Como resultado de todos estos servicios, los Hijos Vorondadeks se han convertido en los historiadores de los universos locales; están familiarizados personalmente con todas las luchas políticas y todas las agitaciones sociales de los mundos habitados.

\section*{6. Los Padres de las Constelaciones}
\par
%\textsuperscript{(390.5)}
\textsuperscript{35:6.1} Al menos tres Vorondadeks están asignados al gobierno de cada una de las cien constelaciones de un universo local. Estos Hijos son elegidos por el Hijo Creador y son nombrados por Gabriel como Altísimos de las constelaciones para servir allí durante un decamilenio ---10.000 años oficiales, unos 50.000 años del tiempo de Urantia. El Altísimo reinante, el Padre de la Constelación, tiene dos asociados, uno más antiguo y otro más reciente. En cada cambio de administración, el asociado más antiguo se convierte en el jefe del gobierno, y el más reciente asume los deberes del más antiguo, mientras que los Vorondadeks sin tarea asignada que residen en los mundos de Salvington proponen a uno de sus miembros como candidato a ser elegido para asumir las responsabilidades del asociado más reciente. Así, de acuerdo con la política actual, cada uno de los gobernantes Altísimos tiene un período de servicio en la sede de una constelación de tres decamilenios, unos
150.000 años de Urantia.

\par
%\textsuperscript{(390.6)}
\textsuperscript{35:6.2} Los cien Padres de las Constelaciones, los jefes que presiden realmente los gobiernos de las constelaciones, componen el gabinete consultivo supremo del Hijo Creador. Este consejo celebra sesiones frecuentes en la sede del universo, y el alcance y la variedad de sus deliberaciones son ilimitados, pero se ocupa principalmente del bienestar de las constelaciones y de la unificación de la administración de todo el universo local.

\par
%\textsuperscript{(391.1)}
\textsuperscript{35:6.3} Cuando el Padre de una Constelación está ocupándose de sus obligaciones en la sede del universo, como sucede con frecuencia, el asociado más antiguo se convierte en el director interino de los asuntos de la constelación. La actividad normal del asociado más antiguo es la supervisión de los asuntos espirituales, mientras que el asociado más reciente se ocupa personalmente del bienestar físico de la constelación. Sin embargo, ninguna política importante se lleva nunca a cabo en una constelación a menos que los tres Altísimos estén de acuerdo sobre todos los detalles de su ejecución.

\par
%\textsuperscript{(391.2)}
\textsuperscript{35:6.4} Todo el mecanismo de la información espiritual y de los canales de comunicación está a la disposición de los Altísimos de las constelaciones. Se encuentran en contacto perfecto con sus superiores en Salvington y con sus subordinados directos, los soberanos de los sistemas locales. Con frecuencia se reúnen en consejo con estos Soberanos de los Sistemas para deliberar sobre el estado de la constelación.

\par
%\textsuperscript{(391.3)}
\textsuperscript{35:6.5} Los Altísimos se rodean de un cuerpo de consejeros, que varía de vez en cuando en cantidad y en personal con arreglo a la presencia de los diversos grupos en las sedes de las constelaciones, y también a medida que varían las necesidades locales. Durante los períodos difíciles pueden solicitar más Hijos de la orden Vorondadek para que los ayuden en el trabajo administrativo, y los reciben rápidamente. Norlatiadek, vuestra propia constelación, está administrada actualmente por doce Hijos Vorondadeks.

\section*{7. Los mundos Vorondadeks}
\par
%\textsuperscript{(391.4)}
\textsuperscript{35:7.1} El segundo grupo de siete mundos que se encuentra en el circuito de las setenta esferas primarias que rodean a Salvington contiene los planetas Vorondadeks. Cada una de estas esferas, con sus seis satélites circundantes, está dedicada a una fase especial de las actividades Vorondadeks. En estos cuarenta y nueve reinos, los mortales ascendentes alcanzan el apogeo de su educación sobre la legislación del universo.

\par
%\textsuperscript{(391.5)}
\textsuperscript{35:7.2} Los mortales ascendentes han observado el funcionamiento de las asambleas legislativas en los mundos sede de las constelaciones, pero aquí, en estos mundos Vorondadeks, participan en la promulgación de la legislación general real del universo local bajo la tutela de los Vorondadeks más antiguos. Estas promulgaciones están destinadas a coordinar las diversas declaraciones de las asambleas legislativas autónomas de las cien constelaciones. La enseñanza que se recibe en las escuelas Vorondadeks es insuperable incluso en Uversa. Esta formación es progresiva y se extiende desde la primera esfera, con trabajos adicionales en sus seis satélites, hasta las seis esferas primarias restantes y sus grupos de satélites asociados.

\par
%\textsuperscript{(391.6)}
\textsuperscript{35:7.3} Los peregrinos ascendentes iniciarán numerosas actividades nuevas en estos mundos de estudio y de trabajo práctico. No se nos prohíbe emprender la revelación de estas ocupaciones nuevas e inimaginables, pero desesperamos de poder describir estas tareas a la mente material de los seres mortales. No tenemos palabras para transmitir los significados de estas actividades celestiales, y no existen tareas humanas análogas que se puedan utilizar como ejemplos de estas nuevas ocupaciones de los mortales ascendentes que continúan sus estudios en estos cuarenta y nueve mundos. Y otras muchas actividades, que no forman parte del régimen ascendente, están centradas en estos mundos Vorondadeks del circuito de Salvington.

\section*{8. Los Hijos Lanonandeks}
\par
%\textsuperscript{(392.1)}
\textsuperscript{35:8.1} Después de la creación de los Vorondadeks, el Hijo Creador y el Espíritu Madre del Universo se unen con el objeto de traer a la existencia a la tercera orden de filiación del universo: los Lanonandeks. Aunque se ocupan de tareas diversas relacionadas con las administraciones de los sistemas, son mejor conocidos como Soberanos de los Sistemas, los gobernantes de los sistemas locales, y como Príncipes Planetarios, los jefes administrativos de los mundos habitados.

\par
%\textsuperscript{(392.2)}
\textsuperscript{35:8.2} Como forman una orden de filiación más tardía e inferior ---en lo que se refiere a los niveles de divinidad--- estos seres necesitaron pasar por ciertos cursos de formación en los mundos Melquisedeks a fin de prepararse para su servicio posterior. Fueron los primeros estudiantes de la Universidad Melquisedek y fueron clasificados y certificados por sus educadores y examinadores Melquisedeks de acuerdo con sus capacidades, su personalidad y sus logros.

\par
%\textsuperscript{(392.3)}
\textsuperscript{35:8.3} El universo de Nebadon empezó su existencia con doce millones exactos de Lanonandeks, y después de pasar por la esfera Melquisedek, en las pruebas finales fueron divididos en tres clases:

\par
%\textsuperscript{(392.4)}
\textsuperscript{35:8.4} 1. \textit{Los Lanonandeks primarios}. De la categoría más elevada había 709.841 miembros. Éstos son los Hijos designados como Soberanos de los Sistemas y asistentes de los consejos supremos de las constelaciones, y como consejeros en el trabajo administrativo superior del universo.

\par
%\textsuperscript{(392.5)}
\textsuperscript{35:8.5} 2. \textit{Los Lanonandeks secundarios}. Cuando esta orden salió de Melquisedek había 10.234.601 miembros. Son destinados como Príncipes Planetarios y a las reservas de esta orden.

\par
%\textsuperscript{(392.6)}
\textsuperscript{35:8.6} 3. \textit{Los Lanonandeks terciarios}. Este grupo contenía 1.055.558 miembros. Estos Hijos actúan como asistentes subordinados, mensajeros, custodios, comisionados, observadores, y llevan a cabo los diversos deberes de un sistema y de los mundos que lo componen.

\par
%\textsuperscript{(392.7)}
\textsuperscript{35:8.7} A estos Hijos no les resulta posible progresar de un grupo a otro como les sucede a los seres evolutivos. Después de estar sometidos a la formación de los Melquisedeks, una vez que han sido probados y clasificados, sirven continuamente en la categoría asignada. Estos Hijos tampoco pueden reproducirse; su número en el universo es fijo.

\par
%\textsuperscript{(392.8)}
\textsuperscript{35:8.8} En números redondos, la orden de los Hijos Lanonandeks está clasificada en Salvington como sigue:

\par
%\textsuperscript{(392.9)}
\textsuperscript{35:8.9} Coordinadores del Universo y Consejeros de las Constelaciones . .
100.000

\par
%\textsuperscript{(392.10)}
\textsuperscript{35:8.10} Soberanos de los Sistemas y Asistentes . . . . . . . . . . . . .
. . 600.000

\par
%\textsuperscript{(392.11)}
\textsuperscript{35:8.11} Príncipes Planetarios y Reservas . . . . . . . . . . . . . . .
. . . .10.000.000

\par
%\textsuperscript{(392.12)}
\textsuperscript{35:8.12} Cuerpo de Mensajeros . . . . . . . . . . . . . . . . . . . . . .
. . . . . 400.000

\par
%\textsuperscript{(392.13)}
\textsuperscript{35:8.13} Custodios y Archivistas. . . . . . . . . . . . . . . . . . . . .
. . . . . . 100.000

\par
%\textsuperscript{(392.14)}
\textsuperscript{35:8.14} Cuerpo de Reserva. . . . . . . . . . . . . . . . . . . . . . . .
. . . . . . 800.000

\par
%\textsuperscript{(392.15)}
\textsuperscript{35:8.15} Puesto que los Lanonandeks son una orden de filiación un poco inferior a las de los Melquisedeks y los Vorondadeks, prestan un servicio aún mayor en las unidades subordinadas del universo, puesto que son capaces de acercarse más a las humildes criaturas de las razas inteligentes. También corren un mayor peligro de descarriarse, de apartarse de la técnica adecuada del gobierno universal. Pero estos Lanonandeks, especialmente los de la orden primaria, son los más capaces y polifacéticos de todos los administradores de los universos locales. En capacidad ejecutiva sólo son superados por Gabriel y sus asociados no revelados.

\section*{9. Los gobernantes Lanonandeks}
\par
%\textsuperscript{(393.1)}
\textsuperscript{35:9.1} Los Lanonandeks son los gobernantes continuos de los planetas y los soberanos rotativos de los sistemas. Uno de estos Hijos gobierna ahora en Jerusem, la sede de vuestro sistema local de mundos habitados.

\par
%\textsuperscript{(393.2)}
\textsuperscript{35:9.2} Los Soberanos de los Sistemas gobiernan en comisiones de dos o tres miembros en la sede de cada sistema de mundos habitados. El Padre de la Constelación nombra a uno de estos Lanonandeks como jefe cada decamilenio. A veces no se efectúa ningún cambio de jefe en el trío, siendo el asunto totalmente optativo para los gobernantes de la constelación. El personal de los gobiernos de los sistemas no cambia repentinamente, a menos que se produzca una tragedia de algún tipo.

\par
%\textsuperscript{(393.3)}
\textsuperscript{35:9.3} Cuando los Soberanos de los Sistemas o los asistentes son retirados, el consejo supremo situado en la sede de la constelación ocupa dichos puestos mediante una selección efectuada entre las reservas de esta orden, un grupo que es más numeroso en Edentia que la media indicada.

\par
%\textsuperscript{(393.4)}
\textsuperscript{35:9.4} Los consejos supremos de los Lanonandeks están estacionados en las diversas sedes de las constelaciones. Este cuerpo está presidido por el Altísimo asociado más antiguo del Padre de la Constelación, mientras que el asociado más reciente supervisa las reservas de la orden secundaria.

\par
%\textsuperscript{(393.5)}
\textsuperscript{35:9.5} Los Soberanos de los Sistemas son fieles a sus nombres; son casi soberanos en los asuntos locales de los mundos habitados. Son casi paternales en su manera de dirigir a los Príncipes Planetarios, los Hijos Materiales y los espíritus ministrantes. El dominio personal del soberano es casi completo. Estos gobernantes no están supervisados por los observadores trinitarios procedentes del universo central. Forman la división ejecutiva del universo local; como custodios de que se cumplan los mandatos legislativos y como ejecutivos encargados de aplicar los veredictos judiciales, representan el único puesto en toda la administración del universo donde la deslealtad personal hacia la voluntad del Hijo Miguel podría afianzarse y tratar de imponerse con más facilidad y rapidez.

\par
%\textsuperscript{(393.6)}
\textsuperscript{35:9.6} Nuestro universo local ha sido desafortunado, ya que más de setecientos Hijos de la orden Lanonandek se han rebelado contra el gobierno del universo, precipitando así la confusión sobre diversos sistemas y numerosos planetas. De toda esta cantidad de fracasos, sólo tres eran Soberanos de Sistemas; prácticamente todos estos Hijos pertenecían a las órdenes segunda y tercera, las de los Príncipes Planetarios y los Lanonandeks terciarios.

\par
%\textsuperscript{(393.7)}
\textsuperscript{35:9.7} El gran número de estos Hijos que han faltado a su integridad no indica ningún defecto en sus creadores. Podían haber sido creados divinamente perfectos, pero fueron creados de tal manera que pudieran comprender mejor, y acercarse más, a las criaturas evolutivas que viven en los mundos del tiempo y del espacio.

\par
%\textsuperscript{(393.8)}
\textsuperscript{35:9.8} De todos los universos locales de Orvonton, a excepción de Henselon, nuestro universo es el que ha perdido el mayor número de esta orden de Hijos. En Uversa existe la opinión general de que hemos tenido tantos problemas administrativos en Nebadon porque nuestros Hijos de la orden Lanonandek fueron creados con un amplio grado de libertad personal para elegir y hacer planes. No hago este comentario como una crítica. El Creador de nuestro universo tiene pleno poder y autoridad para hacer esto. Nuestros elevados gobernantes opinan que, aunque estos Hijos con libertad de elección provocan excesivos conflictos en los primeros tiempos del universo, cuando las cosas se hayan cribado por completo y establecido definitivamente, los beneficios de una lealtad más elevada y de un servicio volitivo más completo por parte de estos Hijos totalmente probados, compensarán con creces la confusión y las tribulaciones de las épocas anteriores.

\par
%\textsuperscript{(394.1)}
\textsuperscript{35:9.9} En el caso de una rebelión en la sede de un sistema, normalmente se instala a un nuevo soberano dentro de un plazo relativamente corto, pero no sucede lo mismo en los planetas individuales. Éstos son las unidades que componen la creación material, y el libre albedrío de las criaturas es un factor a tener en cuenta en el juicio final de todos estos problemas. Se nombran Príncipes Planetarios sucesores para los mundos aislados, para los planetas cuyos príncipes con autoridad pueden haberse descarriado, pero no asumen el gobierno activo de dichos mundos hasta que los resultados de la insurrección no se hayan superado y eliminado parcialmente gracias a las medidas reparadoras adoptadas por los Melquisedeks y otras personalidades ministrantes. La rebelión de un Príncipe Planetario aísla instantáneamente a su planeta; los circuitos espirituales locales se cortan de inmediato. Sólo un Hijo donador puede restablecer las líneas de comunicación interplanetarias de ese mundo espiritualmente aislado.

\par
%\textsuperscript{(394.2)}
\textsuperscript{35:9.10} Existe un plan para salvar a estos Hijos desobedientes e imprudentes, y muchos de ellos han utilizado esta disposición misericordiosa; pero nunca más podrán ejercer su actividad en aquellos puestos donde fallaron. Después de su rehabilitación son asignados a las tareas de custodia y a los departamentos de la administración física.

\section*{10. Los mundos Lanonandeks}
\par
%\textsuperscript{(394.3)}
\textsuperscript{35:10.1} En el circuito de setenta planetas de Salvington, el tercer grupo de siete mundos con sus cuarenta y dos satélites respectivos constituyen el enjambre de esferas administrativas de los Lanonandeks. En estos reinos, los Lanonandeks experimentados que pertenecen al cuerpo de antiguos Soberanos Sistémicos ejercen sus funciones como instructores administrativos de los peregrinos ascendentes y de las huestes seráficas. Los mortales evolutivos observan el trabajo de los administradores del sistema en las capitales de los sistemas, pero aquí participan en la coordinación efectiva de las declaraciones administrativas de los diez mil sistemas locales.

\par
%\textsuperscript{(394.4)}
\textsuperscript{35:10.2} Estas escuelas administrativas del universo local están supervisadas por un cuerpo de Hijos Lanonandeks que han tenido una larga experiencia como Soberanos Sistémicos y como consejeros en las constelaciones. Estos colegios ejecutivos sólo son superados por las escuelas administrativas de Ensa.

\par
%\textsuperscript{(394.5)}
\textsuperscript{35:10.3} Aunque sirven como esferas de formación para los mortales ascendentes, los mundos Lanonandeks son los centros de extensas empresas relacionadas con las actividades administrativas normales y rutinarias del universo. Durante todo el camino hacia el Paraíso, los peregrinos ascendentes continúan sus estudios en las escuelas prácticas de conocimientos aplicados ---una verdadera formación que consiste en hacer realmente las cosas que les enseñan. El sistema educativo universal patrocinado por los Melquisedeks es práctico, progresivo, significativo y experiencial. Abarca la instrucción en las cosas materiales, intelectuales, morontiales y espirituales.

\par
%\textsuperscript{(394.6)}
\textsuperscript{35:10.4} En conexión con estas esferas administrativas de los Lanonandeks, la mayoría de los Hijos salvados de esta orden sirven como custodios y directores de los asuntos planetarios. Estos Príncipes Planetarios rebeldes, y sus asociados en la rebelión, que eligen aceptar la rehabilitación ofrecida, continuarán sirviendo en estas funciones rutinarias al menos hasta que el universo de Nebadon se establezca en la luz y la vida.

\par
%\textsuperscript{(395.1)}
\textsuperscript{35:10.5} Sin embargo, muchos Hijos Lanonandeks de los sistemas más antiguos han establecido maravillosos historiales de servicio, de administración y de logros espirituales. Forman un grupo noble, fiel y leal, a pesar de su tendencia a caer en el error debido a los sofismas de la libertad personal y a las ficciones de la autodeterminación.

\par
%\textsuperscript{(395.2)}
\textsuperscript{35:10.6} [Patrocinado por el Jefe de los Arcángeles, que actúa por autorización de Gabriel de Salvington.]