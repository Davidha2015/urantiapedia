\chapter{Documento 37. Las personalidades del universo local}
\par
%\textsuperscript{(406.1)}
\textsuperscript{37:0.1} A LA cabeza de todas las personalidades de Nebadon se encuentra Miguel, el Hijo Creador y Maestro, el padre y soberano del universo. Su coordinada en divinidad y su complementaria en atributos creativos es el Espíritu Madre del universo local, la Ministra Divina de Salvington. Y estos creadores son, en un sentido muy literal, el Padre-Hijo y el Espíritu-Madre de todas las criaturas nativas de Nebadon.

\par
%\textsuperscript{(406.2)}
\textsuperscript{37:0.2} Los documentos anteriores han tratado de las órdenes creadas de filiación; las narraciones siguientes describirán a los espíritus ministrantes y a las órdenes ascendentes de filiación. Este documento se ocupa principalmente de un grupo intermedio, los Ayudantes del Universo, pero también examinará brevemente algunos de los espíritus más elevados que están estacionados en Nebadon y ciertas órdenes de ciudadanos permanentes del universo local.

\section*{1. Los Ayudantes del Universo}
\par
%\textsuperscript{(406.3)}
\textsuperscript{37:1.1} Muchas de las órdenes singulares agrupadas generalmente en esta categoría no han sido reveladas, pero los Ayudantes del Universo, tal como se presentan en estos documentos, incluyen a las siete órdenes siguientes:

\par
%\textsuperscript{(406.4)}
\textsuperscript{37:1.2} 1. Las Radiantes Estrellas Matutinas\footnote{\textit{Radiante Estrella Matutina}: Ap 22:16.}.

\par
%\textsuperscript{(406.5)}
\textsuperscript{37:1.3} 2. Las Brillantes Estrellas Vespertinas.

\par
%\textsuperscript{(406.6)}
\textsuperscript{37:1.4} 3. Los Arcángeles.

\par
%\textsuperscript{(406.7)}
\textsuperscript{37:1.5} 4. Los Asistentes Altísimos.

\par
%\textsuperscript{(406.8)}
\textsuperscript{37:1.6} 5. Los Altos Comisionados.

\par
%\textsuperscript{(406.9)}
\textsuperscript{37:1.7} 6. Los Supervisores Celestiales.

\par
%\textsuperscript{(406.10)}
\textsuperscript{37:1.8} 7. Los Educadores de los Mundos de las Mansiones.

\par
%\textsuperscript{(406.11)}
\textsuperscript{37:1.9} De la primera orden de Ayudantes del Universo, las Radiantes Estrellas Matutinas, sólo hay un representante en cada universo local, y es el primogénito de todas las criaturas nativas de ese universo local. A la Radiante Estrella Matutina de nuestro universo se le conoce con el nombre de Gabriel de Salvington. Es el jefe ejecutivo de todo Nebadon, y actúa como representante personal del Hijo Soberano y como portavoz de su consorte creativa.

\par
%\textsuperscript{(406.12)}
\textsuperscript{37:1.10} Durante los primeros tiempos de Nebadon, Gabriel trabajó totalmente solo con Miguel y el Espíritu Creativo. A medida que el universo creció y que los problemas administrativos se multiplicaron, se le proporcionó un estado mayor personal de asistentes no revelados, y este grupo aumentó con el tiempo mediante la creación del cuerpo de Estrellas Vespertinas de Nebadon.

\section*{2. Las Brillantes Estrellas Vespertinas}
\par
%\textsuperscript{(407.1)}
\textsuperscript{37:2.1} Los Melquisedeks proyectaron estas brillantes criaturas y luego fueron traídas a la existencia por el Hijo Creador y el Espíritu Creativo. Sirven en muchas ocupaciones, pero principalmente como agentes de enlace de Gabriel, el jefe ejecutivo del universo local. Uno o más de estos seres actúa como representante suyo en la capital de cada constelación y de cada sistema de Nebadon.

\par
%\textsuperscript{(407.2)}
\textsuperscript{37:2.2} Como jefe ejecutivo de Nebadon, Gabriel es el presidente de oficio de la mayoría de los cónclaves de Salvington, o asiste como observador a ellos, y sucede a menudo que mil de estos cónclaves celebran sus sesiones simultáneamente. Las Brillantes Estrellas Vespertinas representan a Gabriel en esas ocasiones; él no puede estar en dos lugares a la vez, y estos superángeles compensan esta limitación. Prestan un servicio análogo para el cuerpo de los Hijos Instructores Trinitarios.

\par
%\textsuperscript{(407.3)}
\textsuperscript{37:2.3} Aunque Gabriel está personalmente ocupado con sus deberes administrativos, se mantiene en contacto con todas las otras fases de la vida y de los asuntos del universo a través de las Brillantes Estrellas Vespertinas. Éstas siempre le acompañan en sus giras planetarias y van con frecuencia en misiones especiales a los planetas individuales como representantes personales suyos. Durante estas misiones, a veces se les ha conocido como «el ángel del Señor»\footnote{\textit{El ángel del Señor}: Gn 16:7-11; Ex 3:2; Mt 1:20,24; 2:13,19.}. Van a menudo a Uversa para representar a la Radiante Estrella Matutina ante los tribunales y las asambleas de los Ancianos de los Días, pero raras veces viajan más allá de los confines de Orvonton.

\par
%\textsuperscript{(407.4)}
\textsuperscript{37:2.4} Las Brillantes Estrellas Vespertinas forman una orden doble de carácter único, pues algunos de sus miembros lo son por dignidad creada y otros lo han conseguido mediante el servicio. En Nebadon, el cuerpo de estos superángeles asciende actualmente a 13.641 miembros. Hay 4.832 de dignidad creada, mientras que 8.809 son espíritus ascendentes que han alcanzado esta meta de servicio elevado. Muchas de estas Estrellas Vespertinas ascendentes empezaron su carrera universal como serafines; otras han ascendido desde los niveles no revelados de la vida de las criaturas. Como meta a alcanzar, este elevado cuerpo nunca está cerrado para los candidatos a la ascensión mientras un universo no se establezca en la luz y la vida.

\par
%\textsuperscript{(407.5)}
\textsuperscript{37:2.5} Los dos tipos de Brillantes Estrellas Vespertinas son fácilmente visibles para las personalidades morontiales y para ciertos tipos de seres materiales supermortales. Los seres creados de esta interesante y polifacética orden poseen una fuerza espiritual que se puede manifestar con independencia de su presencia personal.

\par
%\textsuperscript{(407.6)}
\textsuperscript{37:2.6} El jefe de estos superángeles es Gavalia, el primogénito de esta orden en Nebadon. Desde que Cristo Miguel regresó de su donación triunfal en Urantia, Gavalia ha estado asignado al ministerio de los mortales ascendentes, y durante los últimos mil novecientos años urantianos su asociado Galantia ha mantenido su sede en Jerusem, donde pasa casi la mitad de su tiempo. Galantia es el primer superángel ascendente que ha alcanzado esta elevada posición.

\par
%\textsuperscript{(407.7)}
\textsuperscript{37:2.7} Para las Brillantes Estrellas Vespertinas no existe ninguna agrupación u organización en compañías más que la de su asociación habitual en parejas a fin de realizar numerosas funciones. No se les destina a muchas misiones relacionadas con la carrera ascendente de los mortales, pero cuando se las encargan, nunca actúan solos. Siempre trabajan en parejas ---uno es un ser creado y el otro una Estrella Vespertina ascendente.

\par
%\textsuperscript{(407.8)}
\textsuperscript{37:2.8} Uno de los deberes elevados de las Estrellas Vespertinas consiste en acompañar a los Hijos donadores Avonales en sus misiones planetarias, tal como Gabriel acompañó a Miguel durante su donación en Urantia. Los dos superángeles acompañantes son las personalidades de mayor categoría de estas misiones, y sirven como comandantes conjuntos de los arcángeles y de todos los otros seres asignados a estas empresas. El comandante más antiguo de estos superángeles es el que, a la edad y en el momento oportunos, le dice al Hijo Avonal donador: «Ocúpate de los asuntos de tu hermano».

\par
%\textsuperscript{(408.1)}
\textsuperscript{37:2.9} Unas parejas similares de estos superángeles son destinadas al cuerpo planetario de los Hijos Instructores Trinitarios que trabajan para establecer la era espiritual naciente, o posterior a la donación, en un mundo habitado. En estas misiones, las Estrellas Vespertinas sirven de enlace entre los mortales del reino y el cuerpo invisible de los Hijos Instructores.

\par
%\textsuperscript{(408.2)}
\textsuperscript{37:2.10} \textit{Los Mundos de las Estrellas Vespertinas}. El sexto grupo de siete mundos de Salvington y sus cuarenta y dos satélites tributarios están destinados a la administración de las Brillantes Estrellas Vespertinas. Las órdenes creadas de estos superángeles presiden los siete mundos primarios, mientras que los satélites tributarios están administrados por las Estrellas Vespertinas ascendentes.

\par
%\textsuperscript{(408.3)}
\textsuperscript{37:2.11} Los satélites de los tres primeros mundos están consagrados a las escuelas de los Hijos Instructores y de las Estrellas Vespertinas, dedicadas a las personalidades espirituales del universo local. Los tres grupos siguientes contienen escuelas conjuntas similares consagradas a la formación de los mortales ascendentes. Los satélites del séptimo mundo están reservados para las deliberaciones trinas de los Hijos Instructores, las Estrellas Vespertinas y los finalitarios. Durante los últimos tiempos, estos superángeles han estado estrechamente identificados con el trabajo del Cuerpo de la Finalidad en el universo local, y han estado asociados durante mucho tiempo con los Hijos Instructores. Existe una conexión de un poder y de una importancia extraordinarios entre las Estrellas Vespertinas y los Mensajeros de Gravedad vinculados a los grupos de trabajo finalitarios. El séptimo mundo primario mismo está reservado a aquellos asuntos no revelados que serán propios de las relaciones futuras que existirán entre los Hijos Instructores, los finalitarios y las Estrellas Vespertinas, después de que la manifestación superuniversal de la personalidad de Dios Supremo haya emergido por completo.

\section*{3. Los Arcángeles}
\par
%\textsuperscript{(408.4)}
\textsuperscript{37:3.1} Los arcángeles son la progenitura del Hijo Creador y del Espíritu Madre del Universo. Son el tipo más elevado de seres espirituales superiores engendrados en grandes cantidades en un universo local, y en el momento del último registro había cerca de ochocientos mil en Nebadon.

\par
%\textsuperscript{(408.5)}
\textsuperscript{37:3.2} Los Arcángeles son uno de los pocos grupos de personalidades del universo local que no están normalmente bajo la jurisdicción de Gabriel. No están relacionados de ninguna manera con la administración rutinaria del universo, estando dedicados a la tarea de la supervivencia de las criaturas y a fomentar la carrera ascendente de los mortales del tiempo y del espacio. Aunque habitualmente no están sujetos a la dirección de la Radiante Estrella Matutina, los arcángeles actúan a veces por autoridad suya. También colaboran con otros Ayudantes del Universo tales como las Estrellas Vespertinas, como queda ilustrado en ciertas actividades descritas en la narración sobre el transplante de la vida en vuestro mundo.

\par
%\textsuperscript{(408.6)}
\textsuperscript{37:3.3} El cuerpo de los arcángeles de Nebadon está dirigido por el primogénito de esta orden y, en tiempos más recientes, una sede divisionaria de arcángeles se ha mantenido en Urantia. Este hecho inhabitual es el que atrae rápidamente la atención de los visitantes estudiantiles procedentes del exterior de Nebadon. Entre las primeras cosas que observan en las operaciones intrauniversales se encuentra el descubrimiento de que muchas actividades ascendentes de las Brillantes Estrellas Vespertinas están dirigidas desde la capital de un sistema local, el de Satania. Al profundizar en su examen descubren que ciertas actividades arcangélicas están dirigidas desde un pequeño mundo habitado, aparentemente insignificante, llamado Urantia. Luego sigue la revelación de que Miguel se donó en Urantia, y estos visitantes se interesan de inmediato vivamente por vosotros y por vuestra humilde esfera.

\par
%\textsuperscript{(409.1)}
\textsuperscript{37:3.4} ¿Captáis la importancia del hecho de que vuestro humilde y confuso planeta se ha convertido en una sede divisionaria de la administración del universo y de la dirección de ciertas actividades arcangélicas relacionadas con el programa de la ascensión al Paraíso? Esto presagia indudablemente la futura concentración de otras actividades ascendentes en el mundo donde Miguel se donó, y confiere una importancia enorme y solemne a la promesa personal del Maestro: «Regresaré»\footnote{\textit{Regresaré de nuevo}: Mt 24:27,37,42; Mt 25:13; Mc 13:32-33; Jn 14:3,28.}.

\par
%\textsuperscript{(409.2)}
\textsuperscript{37:3.5} Los arcángeles están asignados en general al servicio y al ministerio de la orden de filiación Avonal, pero no lo hacen hasta después de haber pasado por una extensa formación preliminar en todas las fases del trabajo de los diversos espíritus ministrantes. Un cuerpo de cien arcángeles acompaña a cada Hijo Paradisiaco que se dona en un mundo habitado, y le están temporalmente asignados mientras dura esa donación. Si el Hijo Magistral se convirtiera en el gobernante temporal del planeta, estos arcángeles actuarían como jefes directores de toda la vida celestial de esa esfera.

\par
%\textsuperscript{(409.3)}
\textsuperscript{37:3.6} Dos arcángeles más antiguos siempre son asignados como ayudantes personales a un Avonal Paradisiaco en todas sus misiones planetarias, ya se trate de acciones judiciales, de misiones magistrales o de encarnaciones donadoras. Cuando este Hijo Paradisiaco ha terminado el juicio de un reino y se realiza el llamamiento de los muertos de acuerdo con los registros (la llamada resurrección), es literalmente cierto que los guardianes seráficos de las personalidades dormidas responden a «la voz del arcángel»\footnote{\textit{La voz del arcángel}: 1 Ts 4:16.}. Un arcángel acompañante es el que promulga el llamamiento nominal al final de una dispensación. Es el arcángel de la resurrección, llamado a veces el «arcángel de Miguel»\footnote{\textit{Arcángel de Miguel}: Mt 27:52-53; Jud 1:9.}.

\par
%\textsuperscript{(409.4)}
\textsuperscript{37:3.7} \textit{Los Mundos de los Arcángeles}. El séptimo grupo de mundos que rodea a Salvington, con sus satélites asociados, está asignado a los arcángeles. La esfera número uno y sus seis satélites tributarios están ocupados por los conservadores de los registros de la personalidad. Este inmenso cuerpo de registradores se ocupa de mantener en orden la historia de cada mortal del tiempo desde el momento de su nacimiento, pasando por su carrera universal, hasta que esa persona o bien deja Salvington para incorporarse al régimen superuniversal, o es «tachada de la existencia registrada»\footnote{\textit{Tachada de los registros}: Sal 69:28.} por mandato de los Ancianos de los Días.

\par
%\textsuperscript{(409.5)}
\textsuperscript{37:3.8} En estos mundos es donde los informes sobre la personalidad y las garantías de la identidad son clasificados, archivados y conservados durante ese período que media entre la muerte física y el momento de la repersonalización, la resurrección después de la muerte.

\section*{4. Los Asistentes Altísimos}
\par
%\textsuperscript{(409.6)}
\textsuperscript{37:4.1} Los Asistentes Altísimos son un grupo de seres voluntarios que tienen su origen fuera del universo local, y que son nombrados temporalmente como representantes u observadores del universo central y de los superuniversos ante las creaciones locales. Su número varía constantemente, pero siempre se eleva a muchos millones.

\par
%\textsuperscript{(409.7)}
\textsuperscript{37:4.2} De vez en cuando nos beneficiamos así del ministerio y de la ayuda de unos seres de origen paradisiaco tales como los Perfeccionadores de la Sabiduría, los Consejeros Divinos, los Censores Universales, los Espíritus Inspirados Trinitarios, los Hijos Trinitizados, los Mensajeros Solitarios, los supernafines, los seconafines, los terciafines y otros ministros misericordiosos que residen con nosotros con el objeto de ayudar a nuestras personalidades nativas en su esfuerzo por conducir a todo Nebadon hacia una armonía más plena con las ideas de Orvonton y los ideales del Paraíso.

\par
%\textsuperscript{(410.1)}
\textsuperscript{37:4.3} Cualquiera de estos seres puede estar sirviendo voluntariamente en Nebadon y sin embargo estar técnicamente fuera de nuestra jurisdicción, pero cuando actúan por haber sido nombrados para ello, estas personalidades de los superuniversos y del universo central no están totalmente exentas de las reglamentaciones del universo local donde residen, aunque continúan ejerciendo como representantes de los universos superiores y trabajando de acuerdo con las instrucciones que constituyen su misión en nuestro reino. Su sede general está situada en el sector del Unión de los Días en Salvington y trabajan en Nebadon sometidos a la supervisión suprema de este embajador de la Trinidad del Paraíso. Cuando sirven en grupos independientes, estas personalidades de los reinos superiores se gobiernan habitualmente de forma autónoma, pero cuando sirven a petición de los interesados, a menudo se colocan voluntariamente bajo la jurisdicción total de los directores que supervisan los reinos donde actúan por encargo.

\par
%\textsuperscript{(410.2)}
\textsuperscript{37:4.4} Los Asistentes Altísimos sirven en los universos locales y en las constelaciones, pero no están directamente vinculados a los gobiernos de los sistemas o de los planetas. Sin embargo pueden ejercer su actividad en cualquier parte del universo local y ser asignados a cualquier fase de la actividad de Nebadon ---administrativa, ejecutiva, educativa u otras.

\par
%\textsuperscript{(410.3)}
\textsuperscript{37:4.5} La mayor parte de este cuerpo se ha reclutado para ayudar a las personalidades paradisiacas de Nebadon ---el Unión de los Días, el Hijo Creador, los Fieles de los Días, los Hijos Magistrales y los Hijos Instructores Trinitarios. En el tratamiento de los asuntos de una creación local, de vez en cuando es sabio ocultar temporalmente ciertos detalles al conocimiento de casi todas las personalidades nativas de ese universo local. Ciertos planes avanzados y ciertas decisiones complejas son también mejor captados y más plenamente comprendidos por el cuerpo más maduro y previsor de los Asistentes Altísimos, y es en estas situaciones y en muchas otras en las que son tan extremadamente útiles para los gobernantes y los administradores del universo.

\section*{5. Los Altos Comisionados}
\par
%\textsuperscript{(410.4)}
\textsuperscript{37:5.1} Los Altos Comisionados son mortales ascendentes que han fusionado con el Espíritu; no están fusionados con el Ajustador. Comprendéis bastante bien la carrera de la ascensión universal de un candidato mortal a la fusión con el Ajustador, pues ése es el alto destino en perspectiva para todos los mortales de Urantia desde la donación de Cristo Miguel. Pero éste no es el destino exclusivo de todos los mortales de las épocas anteriores a la donación en los mundos como el vuestro, y existe otro tipo de mundo cuyos habitantes nunca están permanentemente habitados por Ajustadores del Pensamiento. Esos mortales nunca se unen de manera permanente con un Monitor de Misterio donado desde el Paraíso; sin embargo, los Ajustadores sí habitan en ellos transitoriamente, sirviendo como guías y modelos mientras dura la vida en la carne. Durante esa estancia temporal, favorecen la evolución de un alma inmortal exactamente igual que lo hacen en aquellos seres con quienes esperan fusionar, pero cuando la carrera mortal ha terminado, se despiden eternamente de las criaturas con quienes han estado temporalmente asociados.

\par
%\textsuperscript{(410.5)}
\textsuperscript{37:5.2} Las almas sobrevivientes de este tipo alcanzan la inmortalidad mediante la fusión eterna con un fragmento individualizado del espíritu del Espíritu Madre del universo local. No forman un grupo numeroso, al menos en Nebadon. En los mundos de las mansiones encontraréis a estos mortales fusionados con el Espíritu y fraternizaréis con ellos mientras ascienden con vosotros el camino del Paraíso hasta llegar a Salvington, donde se detienen. Algunos de ellos pueden ascender posteriormente hasta niveles universales superiores, pero la mayoría permanecerá para siempre al servicio del universo local; como clase, no están destinados a alcanzar el Paraíso.

\par
%\textsuperscript{(411.1)}
\textsuperscript{37:5.3} Como no están fusionados con un Ajustador, nunca llegarán a ser finalitarios, pero se integrarán finalmente en el Cuerpo de la Perfección del universo local. Habrán obedecido en espíritu al mandato del Padre: «Sed perfectos»\footnote{\textit{Sed perfectos}: Gn 17:1; 1 Re 8:61; Lv 19:2; Dt 18:13; Mt 5:48; 2 Co 13:11; Stg 1:4; 1 P 1:16.}.

\par
%\textsuperscript{(411.2)}
\textsuperscript{37:5.4} Después de alcanzar el Cuerpo de la Perfección de Nebadon, los ascendentes fusionados con el Espíritu pueden aceptar misiones como Ayudantes del Universo, siendo ésta una de las vías que tienen abiertas para continuar creciendo experiencialmente. Así se vuelven candidatos a ser nombrados para el elevado servicio de interpretar los puntos de vista de las criaturas evolutivas de los mundos materiales ante las autoridades celestiales del universo local.

\par
%\textsuperscript{(411.3)}
\textsuperscript{37:5.5} Los Altos Comisionados empiezan su servicio en los planetas como comisionados de las razas. En esta función interpretan los puntos de vista, y describen las necesidades, de las diversas razas humanas. Están dedicados de manera suprema al bienestar de las razas mortales, de las cuales son portavoces, tratando siempre de conseguir para ellas misericordia, justicia y un trato equitativo en todas sus relaciones con los otros pueblos. Los comisionados de las razas actúan en una serie interminable de crisis planetarias, y sirven como expresión articulada de grupos enteros de mortales que luchan.

\par
%\textsuperscript{(411.4)}
\textsuperscript{37:5.6} Después de una larga experiencia solucionando problemas en los mundos habitados, estos comisionados de las razas son ascendidos a niveles de funcionamiento superiores, alcanzando finalmente el estado de Altos Comisionados del universo local, y en él. El último registro indicaba que había poco más de mil millones y medio de estos Altos Comisionados en Nebadon. Estos seres no son finalitarios, pero son seres ascendentes con una larga experiencia y prestan un gran servicio a su universo nativo.

\par
%\textsuperscript{(411.5)}
\textsuperscript{37:5.7} A estos comisionados los encontramos invariablemente en todos los tribunales de justicia, desde los más humildes hasta los más elevados. No es que participen en los procesos de la justicia, sino que actúan como amigos de los tribunales, asesorando a los magistrados que presiden respecto a los antecedentes, el entorno y la naturaleza inherente de los implicados en el juicio.

\par
%\textsuperscript{(411.6)}
\textsuperscript{37:5.8} Los Altos Comisionados están vinculados a las diversas huestes de mensajeros del espacio, y siempre lo están a los espíritus ministrantes del tiempo. Se les encuentra en los programas de las diversas asambleas universales, y estos mismos comisionados con sabiduría humana siempre forman parte de las misiones de los Hijos de Dios en los mundos del espacio.

\par
%\textsuperscript{(411.7)}
\textsuperscript{37:5.9} Cada vez que la equidad y la justicia exigen que se comprenda cómo una política o un procedimiento previstos podría afectar a las razas evolutivas del tiempo, estos comisionados están disponibles para presentar sus recomendaciones; siempre están presentes para hablar en nombre de aquellos que no pueden estar presentes para expresarse por sí mismos.

\par
%\textsuperscript{(411.8)}
\textsuperscript{37:5.10} \textit{Los Mundos de los Mortales fusionados con el Espíritu}. El octavo grupo compuesto por siete mundos primarios y sus satélites tributarios, en el circuito de Salvington, es propiedad exclusiva de los mortales de Nebadon fusionados con el Espíritu. Los mortales ascendentes fusionados con el Ajustador no están relacionados con estos mundos, salvo para disfrutar de muchas estancias agradables y beneficiosas como huéspedes invitados de los residentes fusionados con el Espíritu.

\par
%\textsuperscript{(411.9)}
\textsuperscript{37:5.11} Estos mundos son la residencia permanente de los supervivientes fusionados con el Espíritu, salvo para aquellos pocos que alcanzan Uversa y el Paraíso. Esta limitación deliberada a la ascensión de los mortales resulta beneficiosa para los universos locales, pues asegura la retención de una población permanente evolucionada cuya experiencia creciente continuará aumentando la estabilización y la diversificación futuras de la administración del universo local. Puede ser que estos seres no alcancen el Paraíso, pero consiguen una sabiduría experiencial en el dominio de los problemas de Nebadon que sobrepasa por completo la que pueden alcanzar los ascendentes transitorios. Y estas almas sobrevivientes continúan como combinaciones únicas de lo humano y de lo divino, siendo cada vez más capaces de unir los puntos de vista de estos dos niveles ampliamente separados, y de presentar este doble punto de vista con una sabiduría cada vez mayor.

\section*{6. Los Supervisores Celestiales}
\par
%\textsuperscript{(412.1)}
\textsuperscript{37:6.1} El sistema educativo de Nebadon está administrado conjuntamente por los Hijos Instructores Trinitarios y el cuerpo de enseñantes Melquisedeks, pero los Supervisores Celestiales llevan a cabo una gran parte del trabajo destinado a mantenerlo y a fortalecerlo. Estos seres forman un cuerpo reclutado que abarca todos los tipos de individuos relacionados con el programa de educar y de instruir a los mortales ascendentes. Hay más de tres millones de ellos en Nebadon, y todos son voluntarios que se han cualificado por experiencia para servir como asesores educativos en todo el reino. Desde su sede en los mundos Melquisedeks de Salvington, estos supervisores recorren el universo local como inspectores de la técnica académica de Nebadon destinada a formar la mente y a educar el espíritu de las criaturas ascendentes.

\par
%\textsuperscript{(412.2)}
\textsuperscript{37:6.2} Esta formación de la mente y esta educación del espíritu se llevan a cabo desde los mundos de origen humano, pasando por los mundos de las mansiones del sistema y las otras esferas de progreso asociadas a Jerusem, hasta los setenta reinos de vida social vinculados a Edentia y las cuatrocientas noventa esferas de progreso espiritual que rodean a Salvington. En la misma sede del universo se encuentran las numerosas escuelas de los Melquisedeks, las facultades de los Hijos del Universo, las universidades seráficas y las escuelas de los Hijos Instructores y del Unión de los Días. Se toman todas las disposiciones posibles a fin de capacitar a las diversas personalidades del universo para que realicen un servicio más elevado y una actividad mejor. Todo el universo es una inmensa escuela.

\par
%\textsuperscript{(412.3)}
\textsuperscript{37:6.3} Los métodos que se emplean en muchas escuelas superiores sobrepasan el concepto humano sobre el arte de enseñar la verdad, pero he aquí la piedra angular de todo el sistema educativo: la adquisición del carácter mediante una experiencia iluminada. Los educadores aportan la iluminación; el lugar que se ocupa en el universo y el estatus del ascendente proporcionan la oportunidad de experimentar; la sabia utilización de estos dos factores acrecienta el carácter.

\par
%\textsuperscript{(412.4)}
\textsuperscript{37:6.4} El sistema educativo de Nebadon asegura fundamentalmente vuestra asignación a una tarea, y luego os proporciona la oportunidad de enseñaros el método ideal y divino de realizar mejor esa tarea. Se os encarga una tarea determinada a realizar, y al mismo tiempo se os proporcionan los educadores cualificados para enseñaros el mejor método de ejecutar vuestro trabajo. El plan de educación divino asegura la íntima asociación entre el trabajo y la enseñanza. Os enseñamos la mejor manera de ejecutar las cosas que os mandamos hacer.

\par
%\textsuperscript{(412.5)}
\textsuperscript{37:6.5} La finalidad de toda esta formación y de toda esta experiencia es la de prepararos para que seáis admitidos en las esferas educativas superiores y más espirituales del superuniverso. El progreso dentro de un reino determinado es individual, pero la transición de una fase a otra se efectúa generalmente por clases.

\par
%\textsuperscript{(412.6)}
\textsuperscript{37:6.6} La progresión de la eternidad no consiste únicamente en el desarrollo espiritual. La adquisición intelectual también forma parte de la educación universal. La experiencia mental se amplía paralelamente a la expansión del horizonte espiritual. La mente y el espíritu reciben oportunidades semejantes para formarse y avanzar. Pero durante toda esta magnífica preparación mental y espiritual, estáis liberados para siempre de los obstáculos de la carne mortal. Ya no tenéis que arbitrar constantemente las contiendas conflictivas entre vuestras naturalezas espiritual y material divergentes. Por fin estáis cualificados para disfrutar del impulso unificado de una mente glorificada, despojada desde hace mucho tiempo de sus tendencias primitivas animales hacia las cosas materiales.

\par
%\textsuperscript{(413.1)}
\textsuperscript{37:6.7} Antes de dejar el universo de Nebadon, la mayoría de los mortales de Urantia recibirán la oportunidad de servir durante un período más o menos largo como miembros del cuerpo de los Supervisores Celestiales de Nebadon.

\section*{7. Los educadores de los mundos de las mansiones}
\par
%\textsuperscript{(413.2)}
\textsuperscript{37:7.1} Los Educadores de los Mundos de las Mansiones son querubines reclutados y glorificados. Al igual que la mayoría de los otros instructores de Nebadon, son nombrados por los Melquisedeks. Ejercen su actividad en la mayoría de las empresas educativas de la vida morontial, y su numero sobrepasa por completo la comprensión de la mente humana.

\par
%\textsuperscript{(413.3)}
\textsuperscript{37:7.2} Como nivel de consecución de los querubines y de los sanobines, los Educadores de los Mundos de las Mansiones serán objeto de un estudio adicional en el documento siguiente, mientras que como educadores que juegan un papel importante en la vida morontial, hablaremos de ellos más extensamente en el documento que lleva ese nombre.

\section*{8. Las órdenes de espíritus superiores asignadas}
\par
%\textsuperscript{(413.4)}
\textsuperscript{37:8.1} Además de los centros del poder y de los controladores físicos, algunos seres espirituales de origen superior, pertenecientes a la familia del Espíritu Infinito, están asignados permanentemente al universo local. De las órdenes espirituales superiores de la familia del Espíritu Infinito, están asignadas así las que se indican a continuación:

\par
%\textsuperscript{(413.5)}
\textsuperscript{37:8.2} Los \textit{Mensajeros Solitarios}, cuando están vinculados funcionalmente a la administración del universo local, nos prestan un servicio inapreciable en nuestros esfuerzos por vencer los obstáculos del tiempo y del espacio. Cuando no están asignados de esta manera, nosotros los de los universos locales no tenemos ninguna autoridad en absoluto sobre ellos, pero incluso entonces estos seres únicos siempre están dispuestos a ayudarnos a resolver nuestros problemas y a cumplir nuestras misiones.

\par
%\textsuperscript{(413.6)}
\textsuperscript{37:8.3} Andovontia es el nombre del \textit{Supervisor} terciario \textit{de los Circuitos Universales} estacionado en nuestro universo local. Sólo se ocupa de los circuitos espirituales y morontiales, y no de aquellos que están bajo la jurisdicción de los directores del poder. Él es el que aisló a Urantia en la época en que Caligastia traicionó el planeta durante los difíciles momentos de la rebelión de Lucifer. Al enviar sus saludos a los mortales de Urantia, expresa de antemano su placer de que algún día seréis reintegrados en los circuitos universales que él supervisa.

\par
%\textsuperscript{(413.7)}
\textsuperscript{37:8.4} Salsatia, el \textit{Director del Censo} de Nebadon, mantiene su sede en Salvington dentro del sector de Gabriel. Conoce de manera automática el nacimiento y la muerte de la voluntad, y registra constantemente el número exacto de criaturas volitivas que ejercen su actividad en el universo local. Trabaja en estrecha asociación con los registradores de la personalidad domiciliados en los mundos de registro de los arcángeles.

\par
%\textsuperscript{(413.8)}
\textsuperscript{37:8.5} Un \textit{Inspector Asociado} reside en Salvington. Es el representante personal del Ejecutivo Supremo de Orvonton. Sus asociados, los \textit{Centinelas Asignados} a los sistemas locales, también representan al Ejecutivo Supremo de Orvonton.

\par
%\textsuperscript{(414.1)}
\textsuperscript{37:8.6} Los \textit{Conciliadores Universales} son los tribunales itinerantes de los universos del tiempo y del espacio, y ejercen su actividad desde los mundos evolutivos hasta cada una de las secciones del universo local, e incluso más allá. Estos árbitros están registrados en Uversa; el número exacto que trabaja en Nebadon no está anotado, pero estimo que en nuestro universo local hay cerca de cien millones de comisiones conciliadoras.

\par
%\textsuperscript{(414.2)}
\textsuperscript{37:8.7} De los \textit{Asesores Técnicos}, las mentes jurídicas del reino, tenemos nuestro cupo, aproximadamente quinientos millones. Estos seres son las bibliotecas legales experienciales, vivientes y circulantes, de todo el espacio.

\par
%\textsuperscript{(414.3)}
\textsuperscript{37:8.8} De los \textit{Registradores Celestiales}, los serafines ascendentes, tenemos setenta y cinco en Nebadon. Son los registradores supervisores o más antiguos. Los estudiantes avanzados de esta orden que se están formando ascienden a casi cuatro mil millones.

\par
%\textsuperscript{(414.4)}
\textsuperscript{37:8.9} El ministerio de los setenta mil millones de \textit{Compañeros Morontiales} en Nebadon se describe en las narraciones que tratan de los planetas de transición de los peregrinos del tiempo.

\par
%\textsuperscript{(414.5)}
\textsuperscript{37:8.10} Cada universo tiene su propio cuerpo angélico nativo; sin embargo, hay circunstancias en las que es muy útil tener la ayuda de los espíritus superiores que tienen su origen fuera de la creación local. Los supernafines prestan ciertos servicios excepcionales y poco frecuentes; el jefe actual de los serafines de Urantia es un supernafín primario del Paraíso. A los seconafines reflectantes se les encuentra en todos los lugares donde trabaja el personal del superuniverso, y un gran número de terciafines están temporalmente de servicio como Asistentes Altísimos.

\section*{9. Los ciudadanos permanentes del universo local}
\par
%\textsuperscript{(414.6)}
\textsuperscript{37:9.1} Al igual que los superuniversos y el universo central, el universo local tiene sus órdenes de ciudadanos permanentes. Estas órdenes incluyen los tipos creados siguientes:

\par
%\textsuperscript{(414.7)}
\textsuperscript{37:9.2} 1. Los Susatias.

\par
%\textsuperscript{(414.8)}
\textsuperscript{37:9.3} 2. Los Univitatias.

\par
%\textsuperscript{(414.9)}
\textsuperscript{37:9.4} 3. Los Hijos Materiales.

\par
%\textsuperscript{(414.10)}
\textsuperscript{37:9.5} 4. Las Criaturas Intermedias.

\par
%\textsuperscript{(414.11)}
\textsuperscript{37:9.6} Estos nativos de la creación local, junto con los ascendentes fusionados con el Espíritu y los espirongas (que están clasificados de otra manera), constituyen una ciudadanía relativamente permanente. Estas órdenes de seres no son, en general, ni ascendentes ni descendentes. Todas son criaturas experienciales, pero su experiencia creciente continúa estando disponible para el universo en su nivel de origen. Aunque esto no es totalmente cierto en lo que concierne a los Hijos Adámicos y a las criaturas intermedias, es relativamente cierto en lo que se refiere a estas órdenes.

\par
%\textsuperscript{(414.12)}
\textsuperscript{37:9.7} \textit{Los Susatias}. Estos seres maravillosos residen y trabajan como ciudadanos permanentes en Salvington, la sede de este universo local. Son los brillantes descendientes del Hijo Creador y del Espíritu Creativo, y están estrechamente asociados con los ciudadanos ascendentes del universo local, los mortales fusionados con el Espíritu integrados en el Cuerpo de la Perfección de Nebadon.

\par
%\textsuperscript{(414.13)}
\textsuperscript{37:9.8} \textit{Los Univitatias}. Cada uno de los grupos de esferas arquitectónicas que componen las sedes de las cien constelaciones disfruta del ministerio continuo de una orden residencial de seres conocidos con el nombre de univitatias. Estos hijos del Hijo Creador y del Espíritu Creativo constituyen la población permanente de los mundos sede de las constelaciones. Son seres que no se reproducen y que existen en un plano de vida situado casi a medio camino entre el estado semimaterial de los Hijos Materiales domiciliados en las sedes de los sistemas, y el plano más claramente espiritual de los mortales fusionados con el Espíritu y de los susatias de Salvington; pero los univitatias no son seres morontiales. Realizan por los mortales ascendentes, durante la travesía de las esferas de la constelación, lo que los nativos de Havona hacen por los espíritus peregrinos que pasan por la creación central.

\par
%\textsuperscript{(415.1)}
\textsuperscript{37:9.9} \textit{Los Hijos Materiales de Dios}. Cuando un enlace creativo entre el Hijo Creador y la representante universal del Espíritu Infinito, el Espíritu Madre del Universo, ha completado su ciclo, cuando ya no aparecen más descendientes de sus naturalezas combinadas, entonces el Hijo Creador personaliza de manera doble su último concepto del ser, confirmando así definitivamente su propio origen doble original. En sí mismo y de sí mismo crea entonces a los hermosos y magníficos Hijos e Hijas de la orden material de filiación universal. Éste es el origen del Adán y la Eva originales de cada sistema local de Nebadon. Son una orden de filiación que se reproduce, pues son creados masculinos y femeninos. Sus descendientes trabajan como ciudadanos relativamente permanentes de la capital de un sistema, aunque algunos de ellos reciben el nombramiento de Adanes Planetarios.

\par
%\textsuperscript{(415.2)}
\textsuperscript{37:9.10} Durante una misión planetaria, el Hijo y la Hija Materiales reciben el encargo de fundar la raza adámica de ese mundo, una raza destinada a amalgamarse finalmente con los habitantes mortales de esa esfera. Los Adanes Planetarios son Hijos descendentes así como ascendentes, pero habitualmente los clasificamos como ascendentes.

\par
%\textsuperscript{(415.3)}
\textsuperscript{37:9.11} \textit{Las Criaturas Intermedias}. En los primeros tiempos de la mayoría de los mundos habitados, algunos seres superhumanos pero materializados son destinados allí, pero generalmente se retiran cuando llegan los Adanes Planetarios. Las actividades de estos seres y los esfuerzos de los Hijos Materiales por mejorar las razas evolutivas tienen a menudo como resultado la aparición de un número limitado de criaturas que son difíciles de clasificar. Estos seres únicos se encuentran con frecuencia a medio camino entre los Hijos Materiales y las criaturas evolutivas; de ahí su denominación de criaturas intermedias. En un sentido comparativo, estos intermedios son los ciudadanos permanentes de los mundos evolutivos. Desde los primeros tiempos de la llegada de un Príncipe Planetario hasta la época lejana del establecimiento del planeta en la luz y la vida, son el único grupo de seres inteligentes que permanecen continuamente en la esfera. En Urantia, los ministros intermedios son en realidad los verdaderos guardianes del planeta; son prácticamente los ciudadanos de Urantia. Los mortales son en verdad los habitantes físicos y materiales de un mundo evolutivo, pero sois todos tan efímeros; permanecéis en vuestro planeta natal un tiempo tan corto. Nacéis, vivís, morís y pasáis a otros mundos de progresión evolutiva. Incluso los seres superhumanos que sirven en los planetas como ministros celestiales están destinados allí de manera transitoria; pocos de ellos están mucho tiempo vinculados a una esfera determinada. Sin embargo, las criaturas intermedias aseguran la continuidad de la administración planetaria a pesar de los ministerios celestiales siempre cambiantes y de los habitantes mortales que varían constantemente. Durante todos estos cambios y modificaciones incesantes, las criaturas intermedias permanecen en el planeta llevando adelante su trabajo sin interrupción.

\par
%\textsuperscript{(415.4)}
\textsuperscript{37:9.12} De la misma manera, todas las divisiones de la organización administrativa de los universos locales y de los superuniversos tienen sus poblaciones más o menos permanentes, sus habitantes con categoría de ciudadanos. Al igual que Urantia tiene sus intermedios, Jerusem, la capital de vuestro sistema, tiene a los Hijos y las Hijas Materiales; Edentia, la sede de vuestra constelación, tiene a los univitatias, mientras que los ciudadanos de Salvington son de dos tipos, los susatias creados y los mortales evolucionados fusionados con el Espíritu. Los mundos administrativos de los sectores menores y mayores de los superuniversos no tienen ciudadanos permanentes. Pero las esferas sede de Uversa están continuamente favorecidas con un asombroso grupo de seres conocidos con el nombre de \textit{abandontarios}, creados por los agentes no revelados de los Ancianos de los Días y los siete Espíritus Reflectantes residentes en la capital de Orvonton. Estos ciudadanos que residen en Uversa administran actualmente los asuntos rutinarios de su mundo bajo la supervisión directa del cuerpo de los mortales fusionados con el Hijo situado en Uversa. Incluso Havona tiene sus seres nativos, y la Isla central de Luz y de Vida es el hogar de los diversos grupos de Ciudadanos del Paraíso.

\section*{10. Otros grupos del universo local}
\par
%\textsuperscript{(416.1)}
\textsuperscript{37:10.1} Además de las órdenes seráficas y mortales, que serán examinadas en documentos posteriores, hay numerosos seres adicionales relacionados con el mantenimiento y el perfeccionamiento de una organización tan gigantesca como el universo de Nebadon, que ahora mismo ya tiene más de tres millones de mundos habitados, con diez millones en perspectiva. Los diversos tipos de vida de Nebadon son demasiado numerosos para ser catalogados en este documento, pero podemos mencionar dos órdenes excepcionales que ejercen ampliamente su actividad en las 647.591 esferas arquitectónicas del universo local.

\par
%\textsuperscript{(416.2)}
\textsuperscript{37:10.2} Los \textit{Espirongas} son los descendientes espirituales de la Radiante Estrella Matutina y el Padre Melquisedek. Están exentos de que se ponga fin a su personalidad, pero no son seres evolutivos ni ascendentes. Tampoco están implicados funcionalmente en el régimen de la ascensión evolutiva. Son los ayudantes espirituales del universo local, y realizan las tareas espirituales rutinarias de Nebadon.

\par
%\textsuperscript{(416.3)}
\textsuperscript{37:10.3} Los \textit{Espornagias}. Los mundos sede arquitectónicos del universo local son mundos reales ---creaciones físicas. Su conservación física requiere mucho trabajo, y para ello contamos con la ayuda de un grupo de criaturas físicas llamadas espornagias. Se dedican al cuidado y al cultivo de las fases materiales de estos mundos sede, desde Jerusem hasta Salvington. Los espornagias no son ni espíritus ni personas; son una orden animal de existencia, pero si pudierais verlos estaríais de acuerdo en que parecen animales perfectos.

\par
%\textsuperscript{(416.4)}
\textsuperscript{37:10.4} Las diversas \textit{colonias de cortesía} están domiciliadas en Salvington y en otros lugares. En las constelaciones nos beneficiamos especialmente del ministerio de los artesanos celestiales, y sacamos provecho de las actividades de los directores de la reversión que trabajan principalmente en las capitales de los sistemas locales.

\par
%\textsuperscript{(416.5)}
\textsuperscript{37:10.5} Un cuerpo de mortales ascendentes, incluyendo a las criaturas intermedias glorificadas, siempre está destinado al servicio del universo. Después de llegar a Salvington, estos ascendentes son empleados en una variedad casi infinita de actividades relacionadas con la dirección de los asuntos del universo. Desde cada nivel que han alcanzado, estos mortales que progresan retroceden y descienden para echar una mano a sus compañeros que los siguen en la ascensión. Estos mortales que residen temporalmente en Salvington son enviados, cuando son solicitados, a casi todos los cuerpos de personalidades celestiales como ayudantes, estudiantes, observadores y educadores.

\par
%\textsuperscript{(416.6)}
\textsuperscript{37:10.6} Existen además otros tipos de vida inteligente relacionados con la administración de un universo local, pero el plan de esta narración no prevé la revelación adicional de estas órdenes creadas. Aquí se describe lo suficiente sobre la vida y la administración de este universo como para proporcionarle a la mente mortal un vislumbre de la realidad y la grandiosidad de la existencia en la supervivencia. Las experiencias ulteriores de vuestra carrera progresiva os revelarán cada vez más estos seres interesantes y encantadores. Esta narración no puede ser más que un breve esbozo de la naturaleza y del trabajo de las múltiples personalidades que atestan los universos del espacio, administrando estas creaciones como enormes escuelas formativas, unas escuelas donde los peregrinos del tiempo avanzan de vida en vida y de mundo en mundo, hasta que son enviados con amor desde las fronteras de su universo de origen hacia el régimen educativo superior del superuniverso, y desde allí hacia los mundos de formación espiritual de Havona, y finalmente hacia el Paraíso y el elevado destino de los finalitarios ---la asignación eterna a misiones aún no reveladas a los universos del tiempo y del espacio.

\par
%\textsuperscript{(417.1)}
\textsuperscript{37:10.7} [Dictado por una Brillante Estrella Vespertina de Nebadon, Número 1.146 del Cuerpo Creado.]