\chapter{Documento 119. Las donaciones de Cristo Miguel}
\par
%\textsuperscript{(1308.1)}
\textsuperscript{119:0.1} ME LLAMO Gavalia, soy el Jefe de las Estrellas Vespertinas de Nebadon, y estoy destinado en Urantia por Gabriel con la misión de revelar la historia de las siete donaciones de Miguel de Nebadon, el Soberano de este universo. En el transcurso de esta presentación, me atendré estrictamente a las limitaciones impuestas por mi cometido.

\par
%\textsuperscript{(1308.2)}
\textsuperscript{119:0.2} El atributo de la donación es inherente a los Hijos Paradisiacos del Padre Universal. En su deseo de acercarse a las experiencias de la vida de sus criaturas vivientes subordinadas, las diversas órdenes de Hijos Paradisiacos reflejan la naturaleza divina de sus padres del Paraíso. El Hijo Eterno de la Trinidad del Paraíso mostró el camino en esta práctica donándose siete veces en los siete circuitos de Havona en la época de la ascensión de Grandfanda y de los primeros peregrinos del tiempo y del espacio. Y el Hijo Eterno continúa donándose en los universos locales del espacio en las personas de sus representantes, los Hijos Migueles y los Hijos Avonales.

\par
%\textsuperscript{(1308.3)}
\textsuperscript{119:0.3} Cuando el Hijo Eterno concede un Hijo Creador a un universo local en proyecto, ese Hijo Creador asume la plena responsabilidad de acabar, controlar y componer ese nuevo universo, incluyendo el solemne juramento a la Trinidad eterna de no asumir la plena soberanía de la nueva creación hasta que sus siete donaciones bajo la forma de sus criaturas hayan sido terminadas con éxito y certificadas por los Ancianos de los Días del superuniverso interesado. Cada Hijo Miguel que se ofrece como voluntario para salir del Paraíso y emprender la organización y la creación de un universo, asume esta obligación.

\par
%\textsuperscript{(1308.4)}
\textsuperscript{119:0.4} La finalidad de estas encarnaciones bajo la forma de criaturas consiste en capacitar a estos Creadores para que se conviertan en unos soberanos sabios, compasivos, justos y comprensivos. Estos Hijos divinos son justos de manera innata, pero se vuelven comprensivamente misericordiosos como resultado de estas experiencias sucesivas de donación; son misericordiosos por naturaleza, pero estas experiencias los hacen misericordiosos de una forma nueva y adicional. Estas donaciones son las últimas etapas de su educación y de su formación para la tarea sublime de gobernar los universos locales con rectitud divina y un juicio justo.

\par
%\textsuperscript{(1308.5)}
\textsuperscript{119:0.5} Aunque estas donaciones aportan numerosos beneficios secundarios a los diversos mundos, sistemas y constelaciones, así como a las diferentes órdenes de inteligencias universales a quienes afectan y benefician, sin embargo están destinadas principalmente a completar la formación personal y la educación universal de un Hijo Creador mismo. Estas donaciones no son imprescindibles para dirigir un universo local de manera sabia, justa y eficaz, pero son absolutamente necesarias para administrar de forma equitativa, misericordiosa y comprensiva esa creación, que rebosa de diversas formas de vida y de innumerables criaturas inteligentes pero imperfectas.

\par
%\textsuperscript{(1308.6)}
\textsuperscript{119:0.6} Los Hijos Migueles empiezan su trabajo de organización universal con una comprensión justa y completa de las diversas órdenes de seres que han creado. Tienen unas enormes reservas de misericordia para todas estas diferentes criaturas, e incluso compasión por aquellas que se equivocan y se tambalean en el lodo egoísta que ellas mismas han fabricado. Pero los Ancianos de los Días estiman que estos dones de justicia y de rectitud no son suficientes. Estos gobernantes trinos de los superuniversos nunca certificarán que un Hijo Creador es el Soberano de su universo hasta que no haya adquirido realmente el punto de vista de sus propias criaturas mediante una experiencia efectiva en el entorno donde existen y bajo la forma de esas mismas criaturas. De esta manera, estos Hijos se convierten en unos gobernantes inteligentes y comprensivos; llegan a \textit{conocer} a los diversos grupos a los que gobiernan y sobre los que ejercen su autoridad universal. Adquieren por medio de la experiencia viviente una misericordia práctica, un juicio equitativo, y la paciencia nacida de la existencia experiencial de la criaturas.

\par
%\textsuperscript{(1309.1)}
\textsuperscript{119:0.7} El universo local de Nebadon está ahora gobernado por un Hijo Creador que ha terminado su servicio de donación; reina con una supremacía justa y misericordiosa sobre todos los inmensos dominios de su universo en vías de evolución y de perfeccionamiento. Miguel de Nebadon es la 611.121{\textordfeminine} donación del Hijo Eterno sobre los universos del tiempo y del espacio, y empezó la organización de vuestro universo local hace unos cuatrocientos mil millones de años. Miguel se preparó para su primera aventura de donación hacia la época en que Urantia estaba adquiriendo su forma actual, hace mil millones de años. Sus donaciones se han producido cada ciento cincuenta millones de años aproximadamente, y la última tuvo lugar en Urantia hace mil novecientos años. Ahora procederé a exponer la naturaleza y el carácter de estas donaciones de una manera tan plena como me lo permita mi cometido.

\section*{1. La primera donación}
\par
%\textsuperscript{(1309.2)}
\textsuperscript{119:1.1} Se produjo un acontecimiento solemne en Salvington cuando, hace casi mil millones de años, la asamblea de directores y de jefes del universo de Nebadon escuchó a Miguel anunciar que su hermano mayor, Emmanuel, asumiría pronto la autoridad de Nebadon mientras que él (Miguel) se ausentaría para llevar a cabo una misión no explicada. No se hizo ninguna otra declaración acerca de esta operación, salvo que en la transmisión de despedida a los Padres de la Constelación, se decía entre otras instrucciones: «Y durante este período os pongo al cargo y cuidado de Emmanuel, mientras voy a hacer lo que me pide mi Padre Paradisiaco».

\par
%\textsuperscript{(1309.3)}
\textsuperscript{119:1.2} Después de enviar esta transmisión de despedida, Miguel apareció en el campo de partida de Salvington exactamente igual que en muchas ocasiones anteriores cuando se había preparado para partir hacia Uversa o el Paraíso, excepto que esta vez venía solo. Terminó su declaración de partida con estas palabras: «Sólo os dejo durante un corto período de tiempo. Sé que muchos de vosotros querrían venir conmigo, pero allá donde voy no podéis venir. Esto que estoy a punto de hacer no podéis hacerlo. Voy a hacer la voluntad de las Deidades del Paraíso, y cuando haya terminado mi misión y haya adquirido esta experiencia, volveré a ocupar mi lugar entre vosotros». Después de hablar así, Miguel de Nebadon desapareció de la vista de todos aquellos que estaban reunidos y no volvió a aparecer durante veinte años del tiempo oficial. En todo Salvington, sólo la Ministra Divina y Emmanuel sabían lo que estaba sucediendo, y el Unión de los Días sólo compartió su secreto con Gabriel, el jefe ejecutivo del universo, la Radiante Estrella Matutina.

\par
%\textsuperscript{(1309.4)}
\textsuperscript{119:1.3} Todos los habitantes de Salvington y aquellos que residían en los mundos sede de las constelaciones y de los sistemas se reunieron alrededor de sus respectivas estaciones receptoras de la información universal, esperando recibir alguna noticia sobre la misión y el paradero del Hijo Creador. No se recibió ningún mensaje de posible importancia hasta el tercer día después de la partida de Miguel. Ese día se registró en Salvington, procedente de la esfera Melquisedek, la sede de esta orden en Nebadon, una comunicación que describía simplemente esta operación extraordinaria que nunca se había oído anteriormente: «Hoy al mediodía ha aparecido en el campo de recepción de este mundo un extraño Hijo Melquisedek, que no es de los nuestros, pero que se parece enteramente a los de nuestra orden. Venía acompañado de un omniafín solitario que traía las credenciales de Uversa y que presentó unas instrucciones dirigidas a nuestro jefe, procedentes de los Ancianos de los Días y con el acuerdo de Emmanuel de Salvington, ordenando que este nuevo Hijo Melquisedek sea recibido en nuestra orden y destinado al servicio de urgencia de los Melquisedeks de Nebadon. Así hemos ordenado que se haga, y así se ha hecho».

\par
%\textsuperscript{(1310.1)}
\textsuperscript{119:1.4} Esto es casi todo lo que aparece en los archivos de Salvington con respecto a la primera donación de Miguel. No aparece nada más hasta cien años después, según el tiempo de Urantia, cuando se registró el hecho de que Miguel regresó y volvió a asumir, sin anunciarlo, la dirección de los asuntos del universo. Pero se puede encontrar una extraña inscripción en el mundo Melquisedek, un relato del servicio de este Hijo Melquisedek excepcional del cuerpo de urgencia de aquella época. Este informe se conserva en un sencillo templo que ocupa actualmente el primer término del hogar del Padre Melquisedek, y contiene la narración del servicio de este Hijo Melquisedek transitorio en relación con su tarea en veinticuatro misiones de urgencia en el universo. Este informe, que he vuelto a examinar tan recientemente, termina así:

\par
%\textsuperscript{(1310.2)}
\textsuperscript{119:1.5} «A mediodía de hoy, sin previo anuncio y observado solamente por tres miembros de nuestra fraternidad, este Hijo visitante de nuestra orden ha desaparecido de nuestro mundo tal como había llegado, acompañado solamente por un omniafín solitario; este informe se cierra ahora con la certificación de que este visitante ha vivido como un Melquisedek, ha trabajado como un Melquisedek en la similitud de un Melquisedek, y ha cumplido fielmente todas sus misiones como Hijo de urgencia de nuestra orden. Por consentimiento universal se ha convertido en el jefe de los Melquisedeks, habiéndose ganado nuestro amor y nuestra adoración con su sabiduría incomparable, su amor supremo y su magnífica devoción al deber. Él nos ha amado, nos ha comprendido y ha servido con nosotros, y seremos para siempre sus fieles y leales compañeros Melquisedeks, pues este desconocido en nuestro mundo se ha vuelto ahora, y para la eternidad, un ministro universal de naturaleza Melquisedek»\footnote{\textit{De la orden de Melquisedek}: Heb 5:10; 6:20; 7:17.}.

\par
%\textsuperscript{(1310.3)}
\textsuperscript{119:1.6} Esto es todo lo que se me permite deciros sobre la primera donación de Miguel. Comprendemos plenamente, por supuesto, que este extraño Melquisedek que sirvió tan misteriosamente con los Melquisedeks hace mil millones de años, no era otro que Miguel, encarnado durante la misión de su primera donación. Los archivos no afirman específicamente que este Melquisedek excepcional y eficaz fuera Miguel, pero se cree universalmente que lo era. Es probable que la afirmación concreta de este hecho no se pueda encontrar fuera de los archivos de Sonarington, y los registros de este mundo secreto no están abiertos para nosotros. Los misterios de la encarnación y de la donación sólo se conocen plenamente en este mundo sagrado de los Hijos divinos. Todos conocemos los hechos de las donaciones de Miguel, pero no comprendemos cómo se realizan. No sabemos de qué forma el gobernante de un universo, el creador de los Melquisedeks, puede convertirse de manera tan repentina y misteriosa en uno de ellos y, como uno de ellos, vivir en medio de ellos y trabajar como un Hijo Melquisedek durante cien años. Pero esto es lo que ocurrió.

\section*{2. La segunda donación}
\par
%\textsuperscript{(1310.4)}
\textsuperscript{119:2.1} Durante cerca de ciento cincuenta millones de años después de la donación de Miguel como Melquisedek, todo fue bien en el universo de Nebadon hasta que empezaron a surgir dificultades en el sistema 11 de la constelación 37. Este conflicto consistía en un malentendido por parte de un Hijo Lanonandek, un Soberano Sistémico; el conflicto había sido juzgado por los Padres de la Constelación y su fallo había sido aprobado por el Fiel de los Días, el consejero del Paraíso en aquella constelación, pero el Soberano Sistémico que protestaba no estaba plenamente conforme con el veredicto. Después de más de cien años de descontento, condujo a sus asociados a una de las rebeliones más extendidas y desastrosas, en contra de la soberanía del Hijo Creador, que jamás se haya fomentado en el universo de Nebadon, una rebelión que fue juzgada y terminó hace mucho tiempo gracias a la actuación de los Ancianos de los Días de Uversa.

\par
%\textsuperscript{(1311.1)}
\textsuperscript{119:2.2} Lutentia, el Soberano Sistémico rebelde, reinó de manera suprema en el planeta donde tenía su sede durante más de veinte años del tiempo oficial de Nebadon, después de lo cual, los Altísimos, con la aprobación de Uversa, ordenaron su aislamiento y solicitaron a los gobernantes de Salvington que designaran a un nuevo Soberano Sistémico para que asumiera la dirección de este sistema de mundos habitados confuso y desgarrado por los conflictos.

\par
%\textsuperscript{(1311.2)}
\textsuperscript{119:2.3} Al mismo tiempo que se recibía esta petición en Salvington, Miguel anunció la segunda de aquellas extraordinarias proclamaciones de intención de ausentarse de la sede del universo con el fin de «hacer el mandato de mi Padre Paradisiaco», prometiendo «regresar en el momento adecuado», y concentrando toda la autoridad en las manos de Emmanuel, su hermano del Paraíso, el Unión de los Días.

\par
%\textsuperscript{(1311.3)}
\textsuperscript{119:2.4} Luego, empleando la misma técnica que se observó en el momento de su partida para la donación como Melquisedek, Miguel se despidió de nuevo de la esfera de su sede central. Tres días después de esta despedida inexplicada, un nuevo miembro desconocido apareció en el cuerpo de reserva de los Hijos Lanonandeks primarios de Nebadon. Este nuevo Hijo apareció al mediodía, sin anunciarse y acompañado de un terciafín solitario que llevaba las credenciales de los Ancianos de los Días de Uversa, certificadas por Emmanuel de Salvington, ordenando que este nuevo Hijo fuera destinado al sistema 11 de la constelación 37 como sucesor del depuesto Lutentia, y con plena autoridad como Soberano en funciones del Sistema hasta que se nombrara un nuevo soberano.

\par
%\textsuperscript{(1311.4)}
\textsuperscript{119:2.5} Durante más de diecisiete años del tiempo universal, este gobernante provisional extraño y desconocido administró los asuntos y juzgó sabiamente las dificultades de este sistema local confuso y desmoralizado. Ningún Soberano Sistémico fue nunca más ardientemente amado u honrado y respetado de manera más generalizada. Este nuevo gobernante puso en orden con justicia y misericordia el turbulento sistema, mientras servía cuidadosamente a todos sus súbditos, ofreciéndole incluso a su predecesor rebelde el privilegio de compartir el trono de autoridad del sistema con que sólo presentara sus excusas a Emmanuel por sus imprudencias. Pero Lutentia despreció estos ofrecimientos de misericordia, sabiendo muy bien que este nuevo y extraño Soberano del Sistema no era otro que Miguel, el dirigente universal mismo a quien tan recientemente había desafiado. Pero millones de seguidores suyos descaminados y engañados aceptaron el perdón de este nuevo gobernante, conocido en aquella época como el Soberano Salvador del sistema de Palonia.

\par
%\textsuperscript{(1311.5)}
\textsuperscript{119:2.6} Luego llegó el día memorable en que se presentó el Soberano Sistémico recién nombrado, designado por las autoridades universales como sucesor permanente del depuesto Lutentia, y toda Palonia lamentó la partida del gobernante sistémico más noble y más benigno que Nebadon hubiera conocido jamás. Era amado por todo el sistema y adorado por sus compañeros de todos los grupos de Hijos Lanonandeks. Su partida no tuvo lugar sin ceremonias; se organizó una gran celebración cuando dejó la sede central del sistema. Incluso su predecesor equivocado le envió este mensaje: «Eres justo y recto en todas tus acciones. Aunque continúo rechazando el gobierno del Paraíso, me veo obligado a confesar que eres un administrador justo y misericordioso»\footnote{\textit{Miguel es justo y misericordioso}: Sal 116:5; Dt 32:4; Ap 15:3.}.

\par
%\textsuperscript{(1312.1)}
\textsuperscript{119:2.7} Entonces, este gobernante provisional del sistema rebelde se despidió del planeta de su breve estancia administrativa, y al tercer día después de esto, Miguel apareció en Salvington y asumió de nuevo la dirección del universo de Nebadon. Poco después se produjo la tercera proclamación de Uversa sobre la extensión jurisdiccional de la soberanía y de la autoridad de Miguel. La primera proclamación tuvo lugar en el momento de su llegada a Nebadon, la segunda se había emitido poco después de concluir su donación como Melquisedek, y ahora seguía la tercera al terminar la segunda misión, o misión Lanonandek.

\section*{3. La tercera donación}
\par
%\textsuperscript{(1312.2)}
\textsuperscript{119:3.1} El consejo supremo de Salvington acababa de estudiar la petición de los Portadores de Vida del planeta 217 del sistema 87 de la constelación 61 para que se enviara en su ayuda a un Hijo Material. Ahora bien, este planeta estaba situado en un sistema de mundos habitados donde otro Soberano Sistémico se había descarriado, la segunda rebelión de este tipo en todo Nebadon hasta aquel momento.

\par
%\textsuperscript{(1312.3)}
\textsuperscript{119:3.2} La respuesta a la solicitud de los Portadores de Vida de este planeta fue aplazada, a petición de Miguel, hasta que Emmanuel la estudiara y presentara su informe. Se trataba de un procedimiento irregular, y recuerdo muy bien que todos nos esperábamos algo fuera de lo normal, y no tuvimos que permanecer mucho tiempo en la incertidumbre. Miguel procedió a poner la dirección del universo en manos de Emmanuel, mientras que confió el mando de las fuerzas celestiales a Gabriel; una vez que traspasó así sus responsabilidades administrativas, se despidió del Espíritu Madre del Universo y desapareció del campo de partida de Salvington exactamente tal como lo había hecho en las dos ocasiones anteriores.

\par
%\textsuperscript{(1312.4)}
\textsuperscript{119:3.3} Como se podía esperar, un Hijo Material desconocido apareció tres días después, sin haberse anunciado, en el mundo central del sistema 87 de la constelación 61, acompañado de un seconafín solitario, acreditado por los Ancianos de los Días de Uversa y certificado por Emmanuel de Salvington. El Soberano en funciones del Sistema nombró inmediatamente a este nuevo y misterioso Hijo Material como Príncipe Planetario en ejercicio del mundo 217, y los Altísimos de la constelación 61 confirmaron enseguida esta designación.

\par
%\textsuperscript{(1312.5)}
\textsuperscript{119:3.4} Este Hijo Material excepcional empezó así su difícil carrera en un mundo en secesión, en rebelión y en cuarentena, situado en un sistema aislado sin ninguna comunicación directa con el universo exterior, y allí trabajó solo durante una generación entera del tiempo planetario. Este Hijo Material de urgencia consiguió el arrepentimiento y la recuperación del Príncipe Planetario rebelde y de todo su estado mayor, y presenció el restablecimiento del planeta al servicio leal del gobierno del Paraíso tal como éste está establecido en los universos locales. Un Hijo y una Hija Materiales llegaron a su debido tiempo a este mundo rejuvenecido y redimido, y después de haber sido debidamente instalados como gobernantes planetarios visibles, el Príncipe Planetario provisional o de urgencia se despidió oficialmente y desapareció un día al mediodía. Tres días después, Miguel apareció en su lugar acostumbrado en Salvington, y las transmisiones del superuniverso difundieron muy pronto la cuarta proclamación de los Ancianos de los Días, anunciando el nuevo avance de la soberanía de Miguel en Nebadon.

\par
%\textsuperscript{(1312.6)}
\textsuperscript{119:3.5} Lamento no tener permiso para narrar la paciencia, la fortaleza y la habilidad con que este Hijo Material hizo frente a las difíciles situaciones de este confuso planeta. La recuperación de este mundo aislado es uno de los capítulos más hermosos y conmovedores en los anales de la salvación de todo Nebadon. Hacia el final de esta misión, para todo Nebadon se había vuelto evidente por qué su amado gobernante escogía embarcarse en estas repetidas donaciones en la similitud de alguna orden subordinada de seres inteligentes.

\par
%\textsuperscript{(1313.1)}
\textsuperscript{119:3.6} Las donaciones de Miguel primero como Hijo Melquisedek, luego como Hijo Lanonandek y después como Hijo Material, son todas igualmente misteriosas y se encuentran más allá de toda explicación. En cada caso apareció \textit{repentinamente} y como un individuo plenamente desarrollado del grupo de la donación. El misterio de estas encarnaciones no será nunca conocido, salvo por aquellos que tienen acceso al círculo interior de los archivos de la esfera sagrada de Sonarington.

\par
%\textsuperscript{(1313.2)}
\textsuperscript{119:3.7} Desde esta maravillosa donación como Príncipe Planetario de un mundo aislado y en rebelión, ninguno de los Hijos o Hijas Materiales de Nebadon ha caído nunca en la tentación de quejarse de sus tareas o de criticar las dificultades de sus misiones planetarias. Los Hijos Materiales saben para siempre que en el Hijo Creador del universo tienen a un soberano comprensivo y a un amigo compasivo, a alguien que ha «sido probado y comprobado en todos los aspectos»\footnote{\textit{Ha sido probado y comprobado en todos los aspectos}: Lc 4:2; 22:28; Heb 2:16-18; 4:15.}, tal como ellos han de ser también probados y comprobados.

\par
%\textsuperscript{(1313.3)}
\textsuperscript{119:3.8} A cada una de estas misiones le siguió una era de servicio y de lealtad crecientes por parte de todas las inteligencias celestiales de origen universal, mientras que cada era donadora sucesiva estuvo caracterizada por un progreso y una mejora en todos los métodos de la administración universal y en todas las técnicas de gobierno. Desde esta donación, ningún Hijo o Hija Material se ha unido nunca deliberadamente a una rebelión en contra de Miguel; lo aman y lo honran con demasiada devoción como para rechazarlo nunca conscientemente. Los Adanes de los tiempos recientes sólo se han desviado debido a los engaños y sofismas de personalidades rebeldes de tipo más elevado.

\section*{4. La cuarta donación}
\par
%\textsuperscript{(1313.4)}
\textsuperscript{119:4.1} Al final de uno de los periódicos llamamientos nominales milenarios de Uversa, Miguel procedió a poner el gobierno de Nebadon en las manos de Emmanuel y Gabriel y, por supuesto, al recordar lo que había sucedido en tiempos pasados después de una acción como ésta, todos nos preparamos para presenciar la desaparición de Miguel a fin de emprender su cuarta misión de donación; y no tuvimos que esperar mucho tiempo, ya que pronto se dirigió al campo de partida de Salvington y lo perdimos de vista.

\par
%\textsuperscript{(1313.5)}
\textsuperscript{119:4.2} Al tercer día después de esta desaparición donadora, observamos esta noticia significativa, en las transmisiones universales hacia Uversa, procedente de la sede seráfica de Nebadon: «Informamos de la llegada no anunciada de un serafín desconocido, acompañado de un supernafín solitario y de Gabriel de Salvington. Este serafín no registrado satisface los requisitos de la orden de Nebadon y trae las credenciales de los Ancianos de los Días de Uversa, certificadas por Emmanuel de Salvington. Este serafín demuestra pertenecer a la orden suprema de ángeles de un universo local, y ya ha sido destinado al cuerpo de consejeros docentes».

\par
%\textsuperscript{(1313.6)}
\textsuperscript{119:4.3} Miguel estuvo ausente de Salvington para esta donación seráfica durante un período de más de cuarenta años del tiempo oficial del universo. Durante este tiempo estuvo vinculado como consejero seráfico docente, lo que podríais denominar un secretario particular, a veintiséis instructores superiores diferentes que ejercían su actividad en veintidós mundos distintos. Su tarea última o final fue como consejero y asistente destinado en una misión donadora de un Hijo Instructor Trinitario en el mundo 462 del sistema 84 de la constelación 3 del universo de Nebadon.

\par
%\textsuperscript{(1314.1)}
\textsuperscript{119:4.4} Durante los siete años de esta misión, este Hijo Instructor Trinitario nunca estuvo plenamente persuadido de la identidad de su asociado seráfico. Es verdad que durante aquel período todos los serafines fueron considerados con un interés y una minuciosidad particulares. Todos sabíamos muy bien que nuestro amado Soberano estaba fuera en el universo bajo la apariencia de un serafín, pero nunca pudimos estar seguros de su identidad. Nunca fue identificado totalmente hasta el momento de ser destinado a la misión donadora de este Hijo Instructor Trinitario. Pero a lo largo de este período, los serafines supremos siempre fueron tratados con una solicitud especial, por temor a que cualquiera de nosotros pudiera descubrir que había sido, sin saberlo, el anfitrión del Soberano del universo en misión de donación bajo la forma de una criatura. Así pues, en lo que se refiere a los ángeles, se ha vuelto eternamente cierto que su Creador y Gobernante ha sido «probado y comprobado, en todos los aspectos, en la similitud de una personalidad seráfica»\footnote{\textit{Ha sido probado y comprobado en todos los aspectos}: Lc 4:2; 22:28; Heb 2:16-18; 4:15.}.

\par
%\textsuperscript{(1314.2)}
\textsuperscript{119:4.5} A medida que estas donaciones sucesivas compartían de manera creciente la naturaleza de las formas más humildes de la vida universal, Gabriel se convirtió cada vez más en un asociado de estas aventuras de encarnación, actuando como enlace universal entre Miguel, que se estaba donando, y Emmanuel, el gobernante en funciones del universo.

\par
%\textsuperscript{(1314.3)}
\textsuperscript{119:4.6} Miguel ha pasado ahora por la experiencia donadora de tres órdenes de Hijos universales creados por él: los Melquisedeks, los Lanonandeks y los Hijos Materiales. Luego condesciende a personalizarse en la similitud de la vida angélica como un serafín supremo, antes de dirigir su atención hacia las diversas fases de la carrera ascendente de las formas más humildes de criaturas volitivas: los mortales evolutivos del tiempo y del espacio.

\section*{5. La quinta donación}
\par
%\textsuperscript{(1314.4)}
\textsuperscript{119:5.1} Hace poco más de trescientos millones de años, tal como el tiempo se calcula en Urantia, fuimos testigos de otra de aquellas transmisiones de autoridad universal a Emmanuel y observamos que Miguel se preparó para partir. Esta vez fue diferente a las anteriores, en el sentido de que anunció que su destino sería Uversa, la sede central del superuniverso de Orvonton. Nuestro Soberano partió a su debido tiempo, pero las transmisiones del superuniverso no mencionaron nunca la llegada de Miguel a las cortes de los Ancianos de los Días. Poco después de su partida de Salvington, en las transmisiones de Uversa apareció esta declaración significativa: «Hoy ha llegado un peregrino ascendente de origen mortal, sin anunciarse y sin número, procedente del universo de Nebadon, certificado por Emmanuel de Salvington y acompañado por Gabriel de Nebadon. Este ser no identificado presenta el estado de un verdadero espíritu y ha sido recibido en nuestra comunidad».

\par
%\textsuperscript{(1314.5)}
\textsuperscript{119:5.2} Si hoy pudierais visitar Uversa, escucharíais el relato de los tiempos en que Eventod residió allí, pues a este peregrino especial y desconocido del tiempo y del espacio se le conoce en Uversa por este nombre. Este mortal ascendente, o al menos esta magnífica personalidad exactamente semejante a los mortales ascendentes de la fase espiritual, vivió y ejerció su actividad en Uversa durante un período de once años del tiempo oficial de Orvonton. Este ser recibió las misiones y cumplió los deberes de un mortal espiritual de la misma manera que sus compañeros procedentes de los diversos universos locales de Orvonton. «Fue probado y comprobado en todos los aspectos, al igual que sus compañeros»\footnote{\textit{Ha sido probado y comprobado en todos los aspectos}: Lc 4:2; 22:28; Heb 2:16-18; 4:15.}, y en todas las ocasiones se mostró digno de la confianza y de la fe de sus superiores, al mismo tiempo que inspiró infaliblemente el respeto y la admiración leal de sus compañeros espirituales.

\par
%\textsuperscript{(1315.1)}
\textsuperscript{119:5.3} En Salvington seguimos la carrera de este peregrino espiritual con un gran interés, sabiendo muy bien, por la presencia de Gabriel, que este espíritu peregrino modesto y sin número no era otro que el gobernante, en misión de donación, de nuestro universo local. Esta primera aparición de Miguel, encarnado en el papel de una fase de la evolución mortal, fue un acontecimiento que emocionó y cautivó a todo Nebadon. Habíamos oído hablar de estas cosas, pero ahora las contemplábamos. Miguel apareció en Uversa como un mortal espiritual plenamente desarrollado y perfectamente entrenado, y continuó su carrera como tal hasta el momento en que un grupo de mortales ascendentes avanzó hacia Havona; después de lo cual, mantuvo una conversación con los Ancianos de los Días y se despidió inmediatamente de Uversa, en compañía de Gabriel, de manera repentina y sin ceremonias, apareciendo poco después en su lugar acostumbrado en Salvington.

\par
%\textsuperscript{(1315.2)}
\textsuperscript{119:5.4} Hasta que no terminó esta donación, no caímos finalmente en la cuenta de que Miguel iba probablemente a encarnarse en la similitud de sus diversas órdenes de personalidades del universo, desde los Melquisedeks más elevados, bajando en la escala hasta los mortales de carne y hueso de los mundos evolutivos del tiempo y del espacio. Hacia esta época, las escuelas de los Melquisedeks empezaron a enseñar la probabilidad de que Miguel se encarnaría algún día como un mortal en la carne, y se hicieron muchas especulaciones sobre la posible técnica de una donación tan inexplicable. El hecho de que Miguel hubiera representado en persona el papel de un mortal ascendente confirió un nuevo interés adicional a todo el programa del progreso de las criaturas a través del universo local y del superuniverso.

\par
%\textsuperscript{(1315.3)}
\textsuperscript{119:5.5} Sin embargo, la técnica de estas donaciones sucesivas continuaba siendo un misterio. Gabriel mismo confiesa que no comprende el método por el cual este Hijo Paradisiaco y Creador del universo puede, a voluntad, asumir la personalidad y vivir la vida de una de sus propias criaturas subordinadas.

\section*{6. La sexta donación}
\par
%\textsuperscript{(1315.4)}
\textsuperscript{119:6.1} Ahora que todo Salvington estaba familiarizado con los preparativos de una donación inminente, Miguel convocó a los residentes de su planeta sede y, por primera vez, reveló el resto de su plan de encarnación, anunciando que pronto iba a dejar Salvington con el fin de asumir la carrera de un mortal morontial en las cortes de los Altísimos Padres en el planeta sede de la quinta constelación. Y entonces escuchamos por primera vez el anuncio de que su séptima y última donación se llevaría a cabo en la similitud de la carne mortal en algún mundo evolutivo.

\par
%\textsuperscript{(1315.5)}
\textsuperscript{119:6.2} Antes de salir de Salvington para su sexta donación, Miguel dirigió la palabra a los habitantes reunidos de la esfera y partió a la vista de todos, acompañado de un serafín solitario y de la Radiante Estrella Matutina de Nebadon. Aunque la dirección del universo se había confiado de nuevo a Emmanuel, las responsabilidades administrativas habían sido distribuidas más ampliamente.

\par
%\textsuperscript{(1315.6)}
\textsuperscript{119:6.3} Miguel apareció en la sede de la quinta constelación como un mortal morontial de estado ascendente, plenamente desarrollado. Lamento que me esté prohibido revelar los detalles de esta carrera de un mortal morontial sin numerar, pues se trata de una de las épocas más extraordinarias y asombrosas de la experiencia donadora de Miguel, sin exceptuar siquiera su estancia dramática y trágica en Urantia. Pero entre las numerosas restricciones que se me impusieron al aceptar esta misión, se encuentra una que me prohíbe intentar revelar los detalles de esta maravillosa carrera de Miguel como mortal morontial de Endantun.

\par
%\textsuperscript{(1316.1)}
\textsuperscript{119:6.4} Cuando Miguel regresó de esta donación morontial, fue evidente para todos nosotros que nuestro Creador se había vuelto uno de nuestros semejantes, que el Soberano del Universo era también el amigo y el ayudante compasivo incluso de las formas de inteligencias creadas más humildes de sus reinos. La adquisición progresiva del punto de vista de las criaturas, el cual se reflejaba en la administración del universo, ya la habíamos notado antes de esto, pues había ido apareciendo gradualmente, pero se hizo más evidente después de terminar su donación como mortal morontial, y mucho más aún después de regresar de su carrera como hijo del carpintero en Urantia.

\par
%\textsuperscript{(1316.2)}
\textsuperscript{119:6.5} Gabriel nos había informado de antemano sobre el momento en que Miguel sería liberado de su donación morontial, y preparamos en consecuencia una recepción adecuada en Salvington. Se reunieron millones y millones de seres procedentes de los mundos sede de las constelaciones de Nebadon, y la mayoría de los residentes de los mundos adyacentes a Salvington se reunieron para darle la bienvenida a su regreso al gobierno del universo. En respuesta a nuestros numerosos discursos de bienvenida y expresiones de apreciación hacia un Soberano tan sumamente interesado en sus criaturas, Miguel se limitó a contestar: «Simplemente me he ocupado de los asuntos de mi Padre. Sólo hago lo que complace a los Hijos Paradisiacos que aman y desean ardientemente comprender a sus criaturas»\footnote{\textit{Los asuntos de mi Padre}: Lc 2:49.}.

\par
%\textsuperscript{(1316.3)}
\textsuperscript{119:6.6} Pero desde aquel día hasta el momento en que Miguel emprendió su aventura como Hijo del Hombre en Urantia, todo Nebadon continuó hablando de las numerosas proezas de su Gobernante Soberano cuando éste ejercía su actividad en Endantun, donándose a través de la encarnación de un mortal morontial en proceso de ascensión evolutiva, y siendo probado en todos los aspectos como sus compañeros allí reunidos procedentes de los mundos materiales de toda la constelación donde residía.

\section*{7. La séptima y última donación}
\par
%\textsuperscript{(1316.4)}
\textsuperscript{119:7.1} Durante decenas de miles de años, todos esperamos con ansia la séptima y última donación de Miguel. Gabriel nos había informado que esta donación final se llevaría a cabo en la similitud de la carne mortal, pero ignorábamos por completo el momento, el lugar y la manera de esta aventura culminante.

\par
%\textsuperscript{(1316.5)}
\textsuperscript{119:7.2} El anuncio público de que Miguel había escogido Urantia como teatro para su donación final se efectuó poco después de que nos enteráramos de la falta de Adán y Eva. Y así, durante más de treinta y cinco mil años, vuestro mundo ocupó un lugar muy notable en los consejos de todo el universo. Ninguna etapa de la donación en Urantia (aparte del misterio de la encarnación) se mantuvo en secreto. Desde el principio hasta el fin, incluido el regreso triunfante y final de Miguel a Salvington como Soberano supremo del Universo, todo lo que sucedió en vuestro pequeño, pero sumamente honrado mundo, recibió la más completa publicidad universal.

\par
%\textsuperscript{(1316.6)}
\textsuperscript{119:7.3} Nunca supimos, hasta el momento mismo del acontecimiento, que Miguel aparecería en la Tierra como un niño indefenso del reino, aunque creíamos que éste sería el método. Hasta ese momento siempre había aparecido como un individuo plenamente desarrollado del grupo de personalidades escogido para la donación; por eso la transmisión enviada desde Salvington informando que el bebé de Belén había nacido en Urantia fue una noticia emocionante\footnote{\textit{Nacimiento de Jesús}: Lc 2:6-14.}.

\par
%\textsuperscript{(1316.7)}
\textsuperscript{119:7.4} Entonces no solamente nos dimos cuenta de que nuestro Creador y amigo estaba dando el paso más precario de toda su carrera, arriesgando aparentemente su posición y su autoridad en esta donación como niño indefenso, sino que comprendimos también que su experiencia en esta donación final como mortal lo colocaría eternamente en el trono como soberano indiscutible y supremo del universo de Nebadon. Durante un tercio de siglo del tiempo terrestre, todas las miradas de todas las partes de este universo local estuvieron clavadas en Urantia. Todas las inteligencias se dieron cuenta de que la última donación estaba en curso, y como conocíamos desde hacía mucho tiempo la rebelión de Lucifer en Satania y el descontento de Caligastia en Urantia, comprendimos muy bien la intensidad de la lucha que se originaría cuando nuestro gobernante condescendiera a encarnarse en Urantia en la humilde forma y en la similitud de la carne mortal.

\par
%\textsuperscript{(1317.1)}
\textsuperscript{119:7.5} Josué ben José, el bebé judío, fue concebido y nació en el mundo exactamente igual que todos los demás bebés antes y después que él, \textit{salvo} que este bebé en particular era la encarnación de Miguel de Nebadon\footnote{\textit{Encarnación de Miguel}: Mc 1:1; Lc 1:30-33; 2:4-7; Jn 1:14. \textit{Visita de los magos}: Mt 2:1-12.}, un Hijo divino Paradisiaco y el creador de todo este universo local de cosas y de seres. Este misterio de la encarnación de la Deidad en la forma humana de Jesús, por lo demás de origen natural en el mundo, permanecerá para siempre sin resolverse. Nunca conoceréis, ni siquiera en la eternidad, la técnica y el método de la encarnación del Creador en la forma y la similitud de sus criaturas. Es el secreto de Sonarington, y estos misterios son propiedad exclusiva de los Hijos divinos que han pasado por la experiencia de la donación.

\par
%\textsuperscript{(1317.2)}
\textsuperscript{119:7.6} Algunos hombres sabios de la Tierra conocían la llegada inminente de Miguel. Mediante los contactos entre mundos, estos hombres sabios con perspicacia espiritual se enteraron de la próxima donación de Miguel en Urantia. Y los serafines lo anunciaron, a través de las criaturas intermedias, a un grupo de sacerdotes caldeos cuyo dirigente era Ardnón. Estos hombres de Dios visitaron al niño recién nacido. El único acontecimiento sobrenatural relacionado con el nacimiento de Jesús fue este anuncio a Ardnón y a sus compañeros por parte de los serafines que habían estado vinculados anteriormente a Adán y Eva en el primer jardín.

\par
%\textsuperscript{(1317.3)}
\textsuperscript{119:7.7} Los padres humanos de Jesús eran unas personas de tipo medio de su época y generación, y este Hijo encarnado de Dios nació así de una mujer y fue criado de la misma manera que los niños de aquella raza y de aquel tiempo.

\par
%\textsuperscript{(1317.4)}
\textsuperscript{119:7.8} La historia de la estancia de Miguel en Urantia, el relato de la donación humana del Hijo Creador en vuestro mundo, es un asunto que sobrepasa la incumbencia y la finalidad de esta narración.

\section*{8. El estado de Miguel después de sus donaciones}
\par
%\textsuperscript{(1317.5)}
\textsuperscript{119:8.1} Después de la donación final y con éxito de Miguel en Urantia, no solamente fue aceptado por los Ancianos de los Días como gobernante soberano de Nebadon, sino que también fue reconocido por el Padre Universal como director establecido del universo local creado por él mismo. A su regreso a Salvington, este Miguel, Hijo del Hombre e Hijo de Dios, fue proclamado gobernante establecido de Nebadon. La octava proclamación de la soberanía de Miguel se recibió desde Uversa, mientras que desde el Paraíso llegó la declaración conjunta del Padre Universal y del Hijo Eterno constituyendo a esta unión de Dios y del hombre como jefe exclusivo del universo, y ordenando al Unión de los Días destinado en Salvington que indicara su intención de retirarse al Paraíso. Los Fieles de los Días de las sedes de las constelaciones también recibieron la orden de retirarse de los consejos de los Altísimos. Pero Miguel no quiso consentir la renuncia de los Hijos Trinitarios como consejeros y cooperadores. Los reunió en Salvington y les rogó personalmente que permanecieran de servicio para siempre en Nebadon. Éstos indicaron a sus directores en el Paraíso el deseo de obedecer esta petición, y poco después se promulgaron los mandatos que separaban del Paraíso y destinaban para siempre a estos Hijos del universo central a la corte de Miguel de Nebadon\footnote{\textit{Estado actual de Miguel}: Mc 16:19; Ef 1:20-23; Heb 1:1-4; 8:1-2; 1 P 3:22.}.

\par
%\textsuperscript{(1318.1)}
\textsuperscript{119:8.2} Se necesitaron casi mil millones de años del tiempo de Urantia para terminar la carrera donadora de Miguel y llevar a cabo el establecimiento definitivo de su autoridad suprema en el universo que él mismo había creado. Miguel nació como creador, fue educado como administrador, formado como dirigente, pero se le exigió que ganara su soberanía por experiencia. Vuestro pequeño mundo ha sido así conocido en todo Nebadon como el lugar donde Miguel terminó la experiencia que se le exige a todo Hijo Creador Paradisiaco antes de concedérsele la dirección y el control ilimitados sobre el universo creado por él mismo. A medida que ascendáis en el universo local, aprenderéis más cosas sobre los ideales de las personalidades implicadas en las donaciones anteriores de Miguel.

\par
%\textsuperscript{(1318.2)}
\textsuperscript{119:8.3} Al concluir sus donaciones como criatura, Miguel no sólo establecía su propia soberanía, sino que también acrecentaba la soberanía evolutiva de Dios Supremo. En el transcurso de estas donaciones, el Hijo Creador no solamente se dedicó a una exploración descendente de las diversas naturalezas de la personalidad de las criaturas, sino que también consiguió revelar las voluntades variadamente diversificadas de las Deidades del Paraíso, cuya unidad sintética, tal como la revelan los Creadores Supremos, pone de manifiesto la voluntad del Ser Supremo.

\par
%\textsuperscript{(1318.3)}
\textsuperscript{119:8.4} Estos diversos aspectos volitivos de las Deidades están eternamente personalizados en las diferentes naturalezas de los Siete Espíritus Maestros, y cada una de las donaciones de Miguel reveló de manera particular una de estas manifestaciones de la divinidad. En su donación como Melquisedek manifestó la voluntad unida del Padre, el Hijo y el Espíritu; en su donación como Lanonandek, la voluntad del Padre y del Hijo; en la donación adámica reveló la voluntad del Padre y del Espíritu; en la donación seráfica, la voluntad del Hijo y del Espíritu; en la donación como mortal en Uversa describió la voluntad del Actor Conjunto; en la donación como mortal morontial, la voluntad del Hijo Eterno; y en la donación material en Urantia vivió la voluntad del Padre Universal\footnote{\textit{Jesús vivió la voluntad del Padre}: Mt 26:39,42,44; Mc 14:36,39; Lc 22:42; Jn 4:34; 5:30; 6:38-40; 15:10; 17:4.}, incluso como un mortal de carne y hueso.

\par
%\textsuperscript{(1318.4)}
\textsuperscript{119:8.5} La finalización de estas siete donaciones condujo a la liberación de la soberanía suprema de Miguel y también a crear la posibilidad de la soberanía del Supremo en Nebadon. Miguel no reveló a Dios Supremo en ninguna de sus donaciones, pero la suma total de las siete donaciones es una nueva revelación del Ser Supremo en Nebadon.

\par
%\textsuperscript{(1318.5)}
\textsuperscript{119:8.6} En la experiencia de descender desde Dios hasta el hombre, Miguel experimentó al mismo tiempo la ascensión desde la posibilidad de manifestarse parcialmente hasta la supremacía de la acción finita y la liberación final de su potencial para actuar de manera absonita. Miguel, el Hijo Creador, es un creador espacio-temporal, pero Miguel, el Hijo Maestro séptuple, es un miembro de uno de los cuerpos divinos que componen la Trinidad Última.

\par
%\textsuperscript{(1318.6)}
\textsuperscript{119:8.7} Al pasar por la experiencia de revelar las voluntades de los Siete Espíritus Maestros surgidos de la Trinidad, el Hijo Creador ha pasado por la experiencia de revelar la voluntad del Supremo. Al actuar como revelador de la voluntad de la Supremacía, Miguel, junto con todos los demás Hijos Maestros, se ha identificado eternamente con el Supremo. En esta era del universo, Miguel revela al Supremo y participa en el proceso de hacer que se manifieste la soberanía de la Supremacía. Pero en la próxima era del universo, creemos que colaborará con el Ser Supremo en la primera Trinidad experiencial a favor de los universos del espacio exterior y en ellos.

\par
%\textsuperscript{(1319.1)}
\textsuperscript{119:8.8} Urantia es el santuario sentimental de todo Nebadon, la esfera principal entre diez millones de mundos habitados, el hogar humano de Cristo Miguel, soberano de todo Nebadon, ministro Melquisedek para los reinos, salvador de un sistema, liberador adámico, compañero seráfico, asociado de los espíritus ascendentes, progresor morontial, Hijo del Hombre en la similitud de la carne mortal y Príncipe Planetario de Urantia. Vuestras escrituras dicen la verdad cuando afirman que este mismo Jesús ha prometido regresar\footnote{\textit{El regreso de Jesús}: Jn 14:3,28.} algún día al mundo de su donación final, al Mundo de la Cruz.

\par
%\textsuperscript{(1319.2)}
\textsuperscript{119:8.9} [Este documento, que describe las siete donaciones de Cristo Miguel, es el sexagésimo tercero de una serie de presentaciones, patrocinadas por numerosas personalidades, que narran la historia de Urantia hasta la época de la aparición de Miguel en la Tierra en la similitud de la carne mortal. Estos documentos fueron autorizados por una comisión de doce seres de Nebadon que actuaron bajo la dirección de Mantutia Melquisedek. Redactamos estas narraciones y las tradujimos a la lengua inglesa mediante una técnica autorizada por nuestros superiores, en el año 1935 d. de J.C. del tiempo de Urantia.]