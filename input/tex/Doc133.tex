\chapter{Documento 133. El regreso de Roma}
\par 
%\textsuperscript{(1468.1)}
\textsuperscript{133:0.1} AL prepararse para dejar Roma, Jesús no se despidió de ninguno de sus amigos. El escriba de Damasco apareció en Roma sin anunciarse y desapareció de la misma manera. Tuvo que transcurrir un año entero para que los que lo conocían y lo amaban renunciaran a la esperanza de volverlo a ver. Antes del final del segundo año, pequeños grupos de los que lo habían conocido empezaron a juntarse debido a su interés común por sus enseñanzas y a los recuerdos mutuos de los buenos momentos pasados con él. Estos pequeños grupos de estoicos, cínicos y miembros de los cultos de misterio continuaron manteniendo estas reuniones irregulares e informales hasta el mismo momento en que los primeros predicadores de la religión cristiana aparecieron en Roma.

\par 
%\textsuperscript{(1468.2)}
\textsuperscript{133:0.2} Gonod y Ganid habían comprado tantas cosas en Alejandría y Roma que enviaron de antemano todas sus pertenencias a Tarento en una caravana de animales de carga, mientras que los tres viajeros caminaban cómodamente a través de Italia por la gran vía Apia. Durante este viaje se encontraron con toda clase de seres humanos. Muchos nobles ciudadanos romanos y colonos griegos vivían a lo largo de esta ruta, pero los descendientes de un gran número de esclavos inferiores ya empezaban a hacer su aparición.

\par 
%\textsuperscript{(1468.3)}
\textsuperscript{133:0.3} Un día mientras descansaban para almorzar, aproximadamente a medio camino de Tarento, Ganid le hizo a Jesús una pregunta directa para saber lo que pensaba del sistema de castas de la India. Jesús contestó: <<Aunque los seres humanos difieren unos de otros de muchas maneras, todos los mortales están en igualdad de condiciones ante Dios y el mundo espiritual. A los ojos de Dios sólo existen dos grupos de mortales: los que desean hacer su voluntad y los que no lo desean. Cuando el universo contempla un mundo habitado, discierne igualmente dos grandes clases: los que conocen a Dios y los que no lo conocen. Los que no pueden conocer a Dios son contados entre los animales de un mundo determinado. Los seres humanos se pueden dividir propiamente en muchas categorías según requisitos diferentes, pues se les puede considerar desde un punto de vista físico, mental, social, profesional o moral, pero cuando estas diferentes clases de mortales comparecen ante el tribunal de Dios, se presentan en igualdad de condiciones. En verdad Dios no hace acepción de personas. Aunque no se puede evitar reconocer las diferencias de aptitudes y dotaciones humanas en los terrenos intelectual, social y moral, no habría que hacer ninguna distinción de este tipo en la fraternidad espiritual de los hombres cuando se reúnen para adorar en la presencia de Dios>>\footnote{\textit{Dios no hace acepción de personas}: 2 Cr 19:7; Job 34:19; Eclo 35:12; Hch 10:34; Ro 2:11; Gl 2:6; 3:28; Ef 6:9; Col 3:11.}.

\section*{1. La misericordia y la justicia}
\par 
%\textsuperscript{(1468.4)}
\textsuperscript{133:1.1} Una tarde se produjo un incidente muy interesante al borde de la carretera, cuando se acercaban a Tarento. Observaron que un joven tosco y fanfarrón estaba atacando brutalmente a un muchacho más pequeño. Jesús se apresuró a socorrer al joven agredido, y una vez que lo hubo rescatado, mantuvo firmemente al agresor hasta que el muchacho más pequeño pudo huir. En cuanto Jesús soltó al pequeño peleón, Ganid se abalanzó sobre el muchacho y empezó a darle una buena paliza; ante el asombro de Ganid, Jesús intervino inmediatamente. Refrenó a Ganid y permitió que el asustado muchacho saliera huyendo. Tan pronto como recobró el aliento, Ganid exclamó con agitación: <<Maestro, no consigo comprenderte. Si la misericordia requiere que rescates al muchacho más pequeño, ¿no exige la justicia que se castigue al agresor más grande?>>. Jesús le respondió:

\par 
%\textsuperscript{(1469.1)}
\textsuperscript{133:1.2} <<Ganid, es bien cierto que no comprendes. El ministerio de la misericordia es siempre un trabajo individual, pero el castigo de la justicia es una función de los grupos administrativos de la sociedad, del gobierno o del universo. Como individuo estoy obligado a mostrar misericordia; tenía que ir a rescatar al muchacho agredido, y con toda lógica, debía emplear la fuerza suficiente para contener al agresor. Eso es exactamente lo que he hecho. He logrado liberar al muchacho agredido y ahí termina el ministerio de la misericordia. Luego he retenido por la fuerza al agresor el tiempo necesario para permitir que la parte más débil de la disputa pudiera huir, después de lo cual me he retirado del asunto. No me he puesto a juzgar al agresor, examinando sus motivos ---determinando todos los factores que entraban en juego en el ataque a su semejante--- para luego proceder a infligir el castigo que mi mente pudiera dictar como justa retribución por su mala acción. Ganid, la misericordia puede ser pródiga, pero la justicia es precisa. ¿No te das cuenta de que no hay dos personas que se pongan de acuerdo sobre el castigo que daría satisfacción a las exigencias de la justicia? Una querría imponer cuarenta latigazos, otra veinte, mientras que una tercera recomendaría la celda de aislamiento como justo castigo. ¿No puedes ver que en este mundo es mejor que tales responsabilidades recaigan sobre la colectividad, o sean administradas por los representantes escogidos de esa colectividad? En el universo, el acto de juzgar está a cargo de aquellos que conocen plenamente los antecedentes de todas las malas acciones, así como sus motivos. En una sociedad civilizada y en un universo organizado, la administración de la justicia presupone el pronunciamiento de una sentencia justa después de un juicio equitativo, y estas prerrogativas corresponden a los cuerpos judiciales de los mundos y a los administradores omniscientes de los universos superiores de toda la creación>>.

\par 
%\textsuperscript{(1469.2)}
\textsuperscript{133:1.3} Durante varios días conversaron sobre este problema de manifestar misericordia y de administrar justicia. Ganid comprendió, al menos en cierta medida, por qué Jesús se negaba a participar en las peleas personales. Pero Ganid hizo una última pregunta, a la que nunca recibió una respuesta plenamente satisfactoria; esta pregunta fue: <<Pero, Maestro, si una criatura de mal carácter y más fuerte te atacara y amenazara con destruirte, ¿qué harías? ¿No harías ningún esfuerzo por defenderte?>> Jesús no podía responder de una manera completa y satisfactoria a la pregunta del muchacho, puesto que no quería revelarle que él (Jesús) estaba viviendo en la Tierra para dar ejemplo del amor del Padre Paradisiaco a un universo que lo contemplaba. Sin embargo le dijo lo siguiente:

\par 
%\textsuperscript{(1469.3)}
\textsuperscript{133:1.4} <<Ganid, comprendo muy bien que algunos de estos problemas te dejen perplejo, y voy a procurar contestar a tu pregunta. Ante cualquier ataque que se pudiera hacer contra mi persona, primero determinaría si el agresor es o no un hijo de Dios ---mi hermano en la carne. Si yo estimara que esa criatura no posee juicio moral ni razón espiritual, me defendería sin vacilar hasta el límite de mi fuerza de resistencia, sin preocuparme por las consecuencias para el agresor. Pero no me comportaría así con un semejante que tuviera la condición de la filiación, ni siquiera en defensa propia. Es decir, no lo castigaría de antemano y sin juicio por haberme atacado. Mediante todas las estratagemas posibles, trataría de impedir y de disuadirlo de que lanzara su ataque, y de mitigarlo en caso de que no consiguiera abortarlo. Ganid, tengo una confianza absoluta en la protección de mi Padre celestial. Estoy consagrado a hacer la voluntad de mi Padre que está en los cielos. No creo que pueda sucederme ningún daño \textit{real;} no creo que la obra de mi vida pueda ser puesta en peligro realmente por cualquier cosa que mis enemigos pudieran desear hacerme, y es seguro que no tenemos que temer ninguna violencia por parte de nuestros amigos. Estoy absolutamente convencido de que el universo entero es amistoso conmigo ---insisto en creer en esta verdad todopoderosa con una confianza total, a pesar de todas las apariencias en contra>>.

\par 
%\textsuperscript{(1470.1)}
\textsuperscript{133:1.5} Pero Ganid no estaba satisfecho por completo. Conversaron muchas veces sobre estos temas, y Jesús le contó algunas de sus experiencias infantiles; le habló también de Jacobo, el hijo del albañil. Al enterarse de cómo Jacobo se había erigido a sí mismo en defensor de Jesús, Ganid dijo: <<¡Oh, empiezo a comprender! En primer lugar, sería muy raro que un ser humano normal quisiera atacar a una persona tan bondadosa como tú, e incluso si alguien fuera tan irreflexivo como para hacerlo, es casi seguro que algún otro mortal estaría a la mano para acudir en tu ayuda, como tú mismo te apresuras siempre a socorrer a cualquier persona que se encuentra en apuros. Maestro, estoy de acuerdo contigo en mi corazón, pero en mi cabeza continúo pensando que si yo hubiera sido Jacobo, hubiera disfrutado castigando a aquellos brutos que se atrevían a atacarte sólo porque pensaban que no te defenderías. Supongo que viajas con bastante seguridad a través de la vida, puesto que pasas mucho tiempo ayudando a otros y socorriendo a tus semejantes en apuros --- así pues, es muy probable que siempre haya alguien al alcance de la mano para defenderte>>. Y Jesús replicó: <<Esa prueba aún no ha llegado, Ganid, y cuando llegue, deberemos atenernos a la voluntad del Padre>>. Esto fue casi todo lo que el muchacho pudo sacarle a su maestro sobre el difícil tema de la defensa propia y de la no resistencia. En otra ocasión consiguió arrancarle a Jesús la opinión de que la sociedad organizada tenía todo el derecho a emplear la fuerza para hacer que se ejecuten sus justos mandatos.

\section*{2. El embarque en Tarento}
\par 
%\textsuperscript{(1470.2)}
\textsuperscript{133:2.1} Mientras que se demoraban en el embarcadero esperando que el barco descargara, los viajeros observaron a un hombre que estaba maltratando a su mujer. Como era su costumbre, Jesús intervino a favor de la persona agredida. Se acercó por detrás del marido enfurecido, y dándole una suave palmadita en el hombro, le dijo: <<Amigo mío, ¿puedo hablar contigo a solas un momento?>> El hombre irritado se quedó desconcertado por esta intervención, y después de un momento de vacilación embarazosa, balbuceó: <<¿Eh...por qué...sí, ¿qué quieres de mí?>> Jesús lo llevó aparte y le dijo: <<Amigo mío, supongo que ha debido sucederte algo terrible; tengo muchísimo deseo de que me cuentes qué le ha podido suceder a un hombre fuerte como tú para inducirle a atacar a su mujer, la madre de sus hijos, y además aquí a la vista de todo el mundo. Estoy seguro de que tienes la sensación de poseer alguna buena razón para esta agresión. ¿Qué ha hecho tu mujer para merecer un trato semejante por parte de su marido? Al observarte, creo discernir en tu rostro el amor por la justicia, si no el deseo de mostrar misericordia. Me aventuro a decir que, si me encontraras a un lado del camino, atacado por unos ladrones, te abalanzarías sin titubeos para socorrerme. Me atrevo a decir que has realizado muchas de estas acciones valientes en el transcurso de tu vida. Ahora, amigo mío, dime de qué se trata. ¿Ha hecho tu mujer algo malo, o has perdido tontamente la cabeza y la has agredido sin reflexionar?>> El corazón de este hombre se sintió conmovido, no tanto por las palabras de Jesús como por la mirada bondadosa y la simpática sonrisa que éste le ofreció al concluir sus observaciones. El hombre dijo: <<Veo que eres un sacerdote de los cínicos, y te agradezco que me hayas refrenado. Mi mujer no ha hecho nada realmente malo; es una buena mujer, pero me irrita por la manera que tiene de buscar camorra en público, y pierdo mi sangre fría. Lamento mi falta de autocontrol y prometo tratar de vivir de acuerdo con la antigua promesa que le hice a uno de tus hermanos, que me enseñó el mejor camino hace muchos años. Te lo prometo>>.

\par 
%\textsuperscript{(1471.1)}
\textsuperscript{133:2.2} Entonces, al decirle adiós, Jesús añadió: <<Hermano mío, recuerda siempre que el hombre no tiene ninguna autoridad legítima sobre la mujer, a menos que la mujer le haya dado de buena gana y voluntariamente esa autoridad. Tu esposa se ha comprometido a atravesar la vida contigo, a ayudarte en las luchas que comporta y a asumir la mayor parte de la carga consistente en dar a luz y criar a tus hijos; a cambio de este servicio especial, es simplemente equitativo que reciba de ti esa protección especial que el hombre puede dar a la mujer como a la compañera que tiene que llevar dentro de sí, dar a luz y alimentar a los hijos. La consideración y los cuidados afectuosos que un hombre está dispuesto a conceder a su esposa y a sus hijos, indican la medida en que ese hombre ha alcanzado los niveles superiores de la conciencia espiritual y creativa. ¿No sabes que los hombres y las mujeres están asociados con Dios, en el sentido de que cooperan para crear seres que crecen hasta poseer el potencial de almas inmortales? El Padre que está en los cielos trata como a un igual al Espíritu Madre de los hijos del universo. Es parecerse a Dios compartir tu vida y todo lo relacionado con ella en términos de igualdad con la compañera y madre que comparte contigo tan plenamente esa experiencia divina de reproduciros en las vidas de vuestros hijos. Si puedes amar a tus hijos como Dios te ama a ti, amarás y apreciarás a tu esposa como el Padre que está en los cielos honra y exalta al Espíritu Infinito, la madre de todos los hijos espirituales de un vasto universo>>.

\par 
%\textsuperscript{(1471.2)}
\textsuperscript{133:2.3} Al subir a bordo del barco, se volvieron para contemplar la escena de la pareja que, con lágrimas en los ojos, permanecía abrazada en silencio. Habiendo oído la última parte del mensaje de Jesús a aquel hombre, Gonod se pasó todo el día meditando en el tema, y decidió reorganizar su hogar cuando regresara a la India.

\par 
%\textsuperscript{(1471.3)}
\textsuperscript{133:2.4} El viaje hasta Nicópolis fue agradable pero lento, porque el viento no era favorable. Los tres pasaron muchas horas reviviendo sus experiencias en Roma y recordando todo lo que les había sucedido desde que se conocieron por primera vez en Jerusalén. Ganid se iba impregnando con el espíritu del ministerio personal. Empezó a ejercerlo con el despensero del barco, pero al segundo día, cuando se metió en las aguas profundas de la religión, llamó a Josué para que le echara una mano.

\par 
%\textsuperscript{(1471.4)}
\textsuperscript{133:2.5} Pasaron varios días en Nicópolis, la ciudad que Augusto había fundado unos cincuenta años antes como <<ciudad de la victoria>>, en conmemoración de la batalla de Actium, pues en este lugar había acampado con su ejército antes de la batalla. Se alojaron en la casa de un tal Jerami, un prosélito griego de la fe judía, a quien habían conocido a bordo del barco. El apóstol Pablo pasó todo el invierno con el hijo de Jerami en esta misma casa, en el transcurso de su tercer viaje misionero\footnote{\textit{Pablo invernando en Nicópolis}: Tit 3:12.}. Desde Nicópolis navegaron en el mismo barco hasta Corinto, la capital de la provincia romana de Acaya.

\section*{3. En Corinto}
\par 
%\textsuperscript{(1471.5)}
\textsuperscript{133:3.1} Por la época en que llegaron a Corinto, Ganid empezaba a interesarse mucho por la religión judía, así que no es extraño que al pasar un día por delante de la sinagoga y ver a la gente que entraba, le pidiera a Jesús que lo llevara al oficio. Aquel día escucharon a un rabino erudito discurrir sobre el <<Destino de Israel>>, y después del servicio religioso conocieron a un tal Crispo\footnote{\textit{Crispo}: Hch 18:8; 1 Co 1:14.}, el jefe principal de esta sinagoga. Regresaron muchas veces a los oficios de la sinagoga, pero principalmente para encontrarse con Crispo. Ganid le tomó un gran afecto a Crispo, a su mujer y a su familia de cinco hijos. Disfrutó mucho observando cómo un judío dirigía su vida familiar.

\par 
%\textsuperscript{(1472.1)}
\textsuperscript{133:3.2} Mientras que Ganid estudiaba la vida de familia, Jesús enseñaba a Crispo los mejores caminos de la vida religiosa. Jesús tuvo más de veinte reuniones con este judío progresista. Años más tarde, Pablo predicó en esta misma sinagoga, los judíos rechazaron su mensaje y votaron la prohibición de que continuara predicando en la sinagoga; entonces Pablo se dirigió hacia los gentiles, y no es sorprendente que Crispo y toda su familia abrazaran la nueva religión, convirtiéndose en uno de los pilares principales de la iglesia cristiana que Pablo organizó posteriormente en Corinto.

\par 
%\textsuperscript{(1472.2)}
\textsuperscript{133:3.3} Durante los dieciocho meses que Pablo predicó en Corinto, donde Silas y Timoteo\footnote{\textit{Silas y Timoteo}: Hch 18:5.} se reunieron con él más tarde, encontró a otras muchas personas que habían sido instruidas por <<el preceptor judío del hijo de un mercader indio>>.

\par 
%\textsuperscript{(1472.3)}
\textsuperscript{133:3.4} En Corinto se encontraron con gentes de todas las razas, procedentes de tres continentes. Después de Alejandría y Roma, ésta era la ciudad más cosmopolita del imperio mediterráneo. En esta ciudad había muchas cosas atractivas que ver, y Ganid nunca se cansó de visitar la ciudadela que se alzaba casi a seiscientos metros por encima del nivel del mar. También pasó una gran parte de su tiempo libre entre la sinagoga y la casa de Crispo. Al principio le escandalizó, y más tarde le encantó, la condición de la mujer en los hogares judíos; fue una revelación para este joven indio.

\par 
%\textsuperscript{(1472.4)}
\textsuperscript{133:3.5} Jesús y Ganid fueron a menudo los huéspedes de otro hogar judío, el de Justo\footnote{\textit{Justo}: Hch 18:7; Col 4:11.}, un piadoso mercader que vivía al lado de la sinagoga. Posteriormente, cuando el apóstol Pablo residió en esta casa, escuchó muchas veces el relato de estas visitas del muchacho indio y de su preceptor judío, y tanto Pablo como Justo se preguntaban qué habría sido de aquel sabio y brillante educador hebreo.

\par 
%\textsuperscript{(1472.5)}
\textsuperscript{133:3.6} Cuando estaban en Roma, Ganid había observado que Jesús rehusaba acompañarlos a los baños públicos. Después de aquello, el joven trató varias veces de persuadir a Jesús para que se explicara más ampliamente respecto a las relaciones entre los sexos. Aunque contestaba a las preguntas del muchacho, nunca parecía dispuesto a extenderse acerca de estos asuntos. Una noche, mientras paseaban por Corinto cerca del lugar donde la muralla de la ciudadela descendía hasta el mar, fueron abordados por dos mujeres públicas. Ganid estaba impregnado con la idea, por otra parte cierta, de que Jesús era un hombre de altos ideales, que aborrecía todo lo que sonara a impureza o tuviera sabor a mal; en consecuencia, se dirigió con sequedad a estas mujeres, indicándoles groseramente que se alejaran. Al ver esto, Jesús dijo a Ganid: <<Tienes buenas intenciones, pero no deberías atreverte a hablarle así a las hijas de Dios, aunque se trate de sus hijas desviadas. ¿Quiénes somos nosotros para juzgar a estas mujeres? ¿Acaso conoces todas las circunstancias que las han llevado a recurrir a estos métodos para ganarse la vida? Quédate aquí conmigo mientras hablamos de estas cosas>>. Al escuchar estas palabras, las prostitutas se quedaron aún más asombradas que Ganid.

\par 
%\textsuperscript{(1472.6)}
\textsuperscript{133:3.7} Mientras permanecían allí de pie, a la luz de la Luna, Jesús continuó diciendo: <<Dentro de cada mente humana vive un espíritu divino, el don del Padre que está en los cielos. Este buen espíritu se esfuerza continuamente por conducirnos a Dios, por ayudarnos a encontrar a Dios y a conocer a Dios. Pero dentro de los mortales existen también muchas tendencias físicas naturales que el Creador ha puesto allí para servir al bienestar del individuo y de la raza. Ahora bien, los hombres y las mujeres se desconciertan muchas veces al esforzarse por comprenderse a sí mismos y luchar con las múltiples dificultades que encuentran para ganarse la vida en un mundo tan ampliamente dominado por el egoísmo y el pecado. Ganid, percibo que ninguna de estas mujeres es voluntariamente mala. Puedo decir, por la expresión de sus rostros, que han padecido muchas penas; han sufrido mucho a manos de un destino aparentemente cruel; no han elegido intencionalmente este tipo de vida. En un desaliento que rozaba la desesperación, han sucumbido a la presión del momento y han aceptado esta manera desagradable de ganarse la vida como el mejor camino para salir de una situación que les parecía desesperada. Ganid, algunas personas son realmente perversas en su corazón, y escogen deliberadamente hacer cosas despreciables. Pero dime, al observar estos rostros ahora llenos de lágrimas, ¿ves algo malo o perverso?>> Mientras que Jesús esperaba su contestación, la voz de Ganid se ahogó al balbucear su respuesta: <<No, Maestro, no veo nada de eso, y me disculpo por mi grosería hacia ellas ---les ruego que me perdonen>>. Entonces dijo Jesús: <<Y yo te digo, en su nombre, que te han perdonado, como digo en nombre de mi Padre que está en los cielos que él las ha perdonado. Ahora venid todos conmigo a la casa de un amigo, donde recobraremos nuestras fuerzas y haremos planes para la vida nueva y mejor que está ante nosotros>>. Hasta ese momento, las asombradas mujeres no habían pronunciado una sola palabra; se miraron entre sí y siguieron silenciosamente a los hombres que mostraban el camino.

\par 
%\textsuperscript{(1473.1)}
\textsuperscript{133:3.8} Imagináos la sorpresa de la mujer de Justo cuando, a esta hora tardía, Jesús apareció con Ganid y estas dos extrañas, diciendo: <<Perdónanos por llegar a esta hora, pero Ganid y yo deseamos tomar un bocado, y quisiéramos compartirlo con estas nuevas amigas, que también necesitan alimentarse. Además de eso, venimos hacia ti con la idea de que estarás interesada en deliberar con nosotros sobre la mejor manera de ayudar a estas mujeres a emprender una nueva vida. Ellas pueden contarte su historia, pero supongo que han tenido muchas dificultades, y su misma presencia aquí en tu casa demuestra cuán seriamente desean conocer a gente de bien, y con cuánto placer aprovecharán la oportunidad de mostrarle a todo el mundo ---e incluso a los ángeles del cielo--- la clase de mujeres nobles y valientes que pueden llegar a ser>>.

\par 
%\textsuperscript{(1473.2)}
\textsuperscript{133:3.9} Cuando Marta, la esposa de Justo, hubo servido la comida en la mesa, Jesús se despidió de manera inesperada diciendo: <<Como se hace tarde y el padre del joven estará esperándonos, rogamos nos disculpen mientras os dejamos aquí juntas ---a tres mujeres--- las hijas amadas del Altísimo. Rogaré por vuestra orientación espiritual, mientras hacéis planes para una vida nueva y mejor en la Tierra y para la vida eterna en el gran más allá>>.

\par 
%\textsuperscript{(1473.3)}
\textsuperscript{133:3.10} Jesús y Ganid se despidieron así de las mujeres. Hasta ese momento, las dos prostitutas no habían dicho nada, y Ganid se quedó igualmente sin habla. A Marta le sucedió lo mismo durante unos instantes, pero pronto se puso a la altura de las circunstancias, e hizo por aquellas extrañas todo lo que Jesús había esperado. La mayor de las dos mujeres murió poco tiempo después, con brillantes esperanzas de supervivencia eterna; la más joven trabajó en el negocio de Justo, y más tarde se hizo miembro de por vida de la primera iglesia cristiana de Corinto.

\par 
%\textsuperscript{(1473.4)}
\textsuperscript{133:3.11} En la casa de Crispo, Jesús y Ganid se encontraron varias veces con un tal Gayo\footnote{\textit{Gayo}: Hch 19:29; 20:4; Ro 16:23; 1 Co 1:14; 3 Jn 1:1.}, que se convirtió posteriormente en un leal partidario de Pablo. Durante estos dos meses en Corinto, mantuvieron conversaciones íntimas con decenas de personas dignas de interés, y como resultado de estos contactos aparentemente casuales, más de la mitad de estas personas se hicieron miembros de la comunidad cristiana posterior.

\par 
%\textsuperscript{(1473.5)}
\textsuperscript{133:3.12} Cuando Pablo fue por primera vez a Corinto, no tenía la intención de quedarse mucho tiempo, pero no sabía hasta qué punto el preceptor judío había preparado bien el terreno para sus trabajos. Descubrió además que Aquila y Priscila\footnote{\textit{Aquila y Priscila}: Hch 18:2,18,26; Ro 16:3; 1 Co 16:19; 2 Ti 4:19.} ya habían despertado un gran interés por su doctrina. Aquila era uno de los cínicos con los que Jesús había entrado en contacto cuando estuvo en Roma. Esta pareja eran refugiados judíos de Roma, y aceptaron rápidamente las enseñanzas de Pablo, que vivió y trabajó con ellos, porque eran también fabricantes de tiendas. Fue debido a estas circunstancias por lo que Pablo prolongó su estancia en Corinto.

\section*{4. Trabajo personal en Corinto}
\par 
%\textsuperscript{(1474.1)}
\textsuperscript{133:4.1} Jesús y Ganid tuvieron otras muchas experiencias interesantes en Corinto. Tuvieron estrechas conversaciones con un gran número de personas, que se beneficiaron mucho de las instrucciones de Jesús.

\par 
%\textsuperscript{(1474.2)}
\textsuperscript{133:4.2} A un molinero le enseñó a moler los granos de la verdad en el molino de la experiencia viviente, para hacer que las cosas difíciles de la vida divina fueran fácilmente aceptables incluso por aquellos compañeros mortales que son frágiles y débiles. Jesús dijo: <<Da la leche de la verdad a aquellos que están en la infancia de la percepción espiritual. En tu ministerio viviente y amante, sirve el alimento espiritual de una manera atractiva y adaptada a la capacidad de recepción de cada uno de los que te pregunten>>\footnote{\textit{La leche de la verdad espiritual}: 1 Co 3:1-2; 1 P 2:2.}.

\par 
%\textsuperscript{(1474.3)}
\textsuperscript{133:4.3} Al centurión romano le dijo: <<Da al César lo que es del César y a Dios lo que es de Dios. No existe conflicto entre el sincero servicio de Dios y el leal servicio del César, a menos que el César se atreva a reclamar el homenaje que sólo puede ser reivindicado por la Deidad. La lealtad a Dios, si llegas a conocerlo, te hará aún más leal y fiel en tu devoción a un emperador digno>>\footnote{\textit{Da al César lo que es del César y a Dios lo que es de Dios}: Mt 22:21; Mc 12:17; Ro 13:7.}.

\par 
%\textsuperscript{(1474.4)}
\textsuperscript{133:4.4} Al jefe sincero del culto mitríaco le dijo: <<Haces bien en buscar una religión de salvación eterna, pero te equivocas al buscar esa gloriosa verdad entre los misterios elaborados por los hombres y en las filosofías humanas. ¿No sabes que el misterio de la salvación eterna reside dentro de tu propia alma? ¿No sabes que el Dios del cielo ha enviado a su espíritu para que viva dentro de ti, y que todos los mortales que aman la verdad y que sirven a Dios serán conducidos por este espíritu más allá de esta vida, a través de las puertas de la muerte, hasta las alturas eternas de la luz, donde Dios aguarda para recibir a sus hijos? Y no olvides nunca que vosotros, los que conocéis a Dios, sois los hijos de Dios si anheláis realmente pareceros a él>>\footnote{\textit{Dios habita en tu alma}: Job 32:8,18; Is 63:10-11; Ez 37:14; Mt 10:20; Lc 17:21; Jn 17:21-23; Ro 8:9-11; 1 Co 3:16-17; 6:19; 2 Co 6:16; Gl 2:20; 1 Jn 3:24; 4:12-15; Ap 21:3. \textit{Los hijos anhelan parecerse a Dios}: Sal 82:6; Jn 10:34-35.}.

\par 
%\textsuperscript{(1474.5)}
\textsuperscript{133:4.5} Al maestro epicúreo le dijo: <<Haces bien en elegir lo mejor y en apreciar lo bueno, pero ¿eres sabio cuando dejas de discernir las grandes cosas de la vida mortal que están incorporadas en los reinos del espíritu derivados de la conciencia de la presencia de Dios en el corazón humano? En toda experiencia humana, la cosa importante es la conciencia de conocer al Dios cuyo espíritu vive dentro de ti y trata de mostrarte el camino en el largo y casi interminable viaje para alcanzar la presencia personal de nuestro Padre común, el Dios de toda la creación, el Señor de los universos>>\footnote{\textit{El espíritu de Dios vive en nosotros}: Job 32:8,18; Is 63:10-11; Ez 37:14; Mt 10:20; Lc 17:21; Jn 17:21-23; Ro 8:9-11; 1 Co 3:16-17; 6:19; 2 Co 6:16; Gl 2:20; 1 Jn 3:24; 4:12-15; Ap 21:3.}.

\par 
%\textsuperscript{(1474.6)}
\textsuperscript{133:4.6} Al contratista y constructor griego le dijo: <<Amigo mío, al mismo tiempo que construyes los edificios materiales de los hombres, desarrolla un carácter espiritual a semejanza del espíritu divino interior de tu alma. No dejes que tus éxitos como constructor temporal sobrepasen a tus realizaciones como hijo espiritual del reino de los cielos. Mientras construyes las mansiones del tiempo para otros, no descuides asegurarte tu propio derecho a las mansiones de la eternidad. Recuerda siempre que existe una ciudad cuyos fundamentos son la rectitud y la verdad, y cuyo constructor y hacedor es Dios>>\footnote{\textit{Muchas mansiones}: Jn 14:2. \textit{La ciudad de Dios}: Heb 11:10. \textit{Lo temporal contra lo espiritual}: Mt 6:19-20; Lc 12:21; 1 Ti 6:17; 1 Jn 2:15-17.}.

\par 
%\textsuperscript{(1474.7)}
\textsuperscript{133:4.7} Al juez romano le dijo: <<Cuando juzgues a los hombres, recuerda que tú mismo comparecerás también algún día ante el tribunal de los Soberanos de un universo. Juzga con justicia e incluso con misericordia, al igual que algún día desearás ardientemente la consideración misericordiosa de las manos del Arbitro Supremo. Juzga como te gustaría ser juzgado en circunstancias semejantes, y así estarás guiado tanto por el espíritu de la ley como por su letra. De la misma manera que otorgas una justicia dominada por la equidad a la luz de las necesidades de los que son traídos ante ti, igualmente tendrás derecho a esperar una justicia templada por la misericordia, cuando algún día comparezcas ante el Juez de toda la Tierra>>\footnote{\textit{Judga justamente}: Mt 7:2; Lc 6:38; Stg 2:12-13. \textit{Juicio de todos}: Gn 18:24-25. \textit{Justicia, equidad, misericordia}: 1 Re 8:30,34,36,39; Sal 32:1; Is 1:18-20; Mt 6:12,14-15; Mc 11:25-26; Lc 6:36-38; 11:4.}.

\par 
%\textsuperscript{(1475.1)}
\textsuperscript{133:4.8} A la dueña de la posada griega le dijo: <<Ofrece tu hospitalidad como alguien que recibe a los hijos del Altísimo. Eleva la faena ingrata de tu trabajo diario hasta los niveles elevados de un arte refinado, mediante la conciencia creciente de que sirves a Dios en las personas en las que él habita por medio de su espíritu, el cual ha descendido para vivir en el corazón de los hombres, intentando así transformar sus mentes y conducir sus almas al conocimiento del Padre Paradisiaco que ha otorgado todos estos dones del espíritu divino>>\footnote{\textit{Hijos del Altísimo}: Sal 82:6.}.

\par 
%\textsuperscript{(1475.2)}
\textsuperscript{133:4.9} Jesús tuvo numerosos encuentros con un mercader chino. Al despedirse de él, le hizo estas advertencias: <<Adora sólo a Dios, que es tu verdadero antepasado espiritual. Recuerda que el espíritu del Padre vive siempre dentro de ti y orienta constantemente tu alma en dirección al cielo. Si sigues las directrices inconscientes de este espíritu inmortal, estarás seguro de perseverar en el camino elevado que conduce a encontrar a Dios. Cuando logres alcanzar al Padre que está en los cielos, será porque al buscarlo te habrás vuelto cada vez más semejante a él. Así pues, adiós, Chang, pero sólo por un tiempo, porque nos encontraremos de nuevo en los mundos de luz, donde el Padre de las almas espirituales ha preparado numerosos lugares de detención encantadores para los que se dirigen hacia el Paraíso>>\footnote{\textit{Adora sólo a Dios}: Ex 20:3; Dt 5:7; Mt 4:10. \textit{Muchas mansiones}: Jn 14:2.}.

\par 
%\textsuperscript{(1475.3)}
\textsuperscript{133:4.10} Al viajero que venía de Bretaña le dijo: <<Hermano mío, percibo que estás buscando la verdad, y sugiero que el espíritu del Padre de toda verdad tal vez resida dentro de ti. ¿Has probado sinceramente alguna vez hablar con el espíritu de tu propia alma? La cosa es ciertamente difícil y es raro que produzca la conciencia del éxito. Pero cualquier intento honrado de la mente material por comunicarse con su espíritu interior alcanza cierto éxito, aunque la mayoría de estas magníficas experiencias humanas deben permanecer mucho tiempo como registros superconscientes en el alma de esos mortales que conocen a Dios>>.

\par 
%\textsuperscript{(1475.4)}
\textsuperscript{133:4.11} Al muchacho fugitivo Jesús le dijo: <<Recuerda que hay dos seres de quienes no puedes escapar: Dios y tú mismo. Dondequiera que vayas, te llevas a ti mismo y al espíritu del Padre celestial que vive dentro de tu corazón. Hijo mío, no trates más de engañarte; asiéntate en la práctica valiente de enfrentarte a los hechos de la vida; aférrate a la seguridad de la filiación con Dios y a la certeza de la vida eterna, como te lo he indicado. Desde hoy en adelante, propónte ser un verdadero hombre, un hombre decidido a afrontar la vida con valentía e inteligencia>>.

\par 
%\textsuperscript{(1475.5)}
\textsuperscript{133:4.12} Al criminal condenado le dijo en su última hora: <<Hermano mío, has pasado por malos tiempos. Te has extraviado; te has enredado en las mallas del crimen. Basándome en lo que he hablado contigo, sé muy bien que no habías planeado hacer lo que ahora está a punto de costarte la vida temporal. Pero cometiste esa mala acción y tus semejantes te han encontrado culpable; han decidido que debes morir. Ni tú ni yo podemos negarle al Estado el derecho a defenderse como le parezca apropiado. Parece que no hay manera de escapar humanamente al castigo de tu delito. Tus semejantes están obligados a juzgarte por lo que has hecho, pero existe un Juez a quien puedes apelar para ser perdonado, y que te juzgará por tus verdaderos móviles y tus mejores intenciones. No debes temer hacer frente al juicio de Dios, si tu arrepentimiento es auténtico y tu fe sincera. El hecho de que tu error lleve consigo la pena de muerte impuesta por los hombres, no afecta a la oportunidad que tiene tu alma de obtener justicia y de gozar de misericordia ante los tribunales celestiales>>.

\par 
%\textsuperscript{(1476.1)}
\textsuperscript{133:4.13} Jesús disfrutó de muchas conversaciones íntimas con un gran número de almas hambrientas, demasiado numerosas para ser incluidas en esta narración. Los tres viajeros disfrutaron de su estancia en Corinto. A excepción de Atenas, que era más famosa como centro de educación, Corinto era la ciudad más importante de Grecia en esta época romana. Su estancia de dos meses en este centro comercial floreciente proporcionó a los tres la oportunidad de adquirir una experiencia valiosísima. Su estancia en esta ciudad fue una de las escalas más interesantes en el camino de regreso de Roma.

\par 
%\textsuperscript{(1476.2)}
\textsuperscript{133:4.14} Gonod tenía muchos intereses en Corinto, pero finalmente terminó sus negocios y se prepararon para navegar hacia Atenas. Viajaron en un pequeño barco que podía ser transportado por tierra sobre un carril desde uno de los puertos de Corinto hasta el otro, a una distancia de dieciséis kilómetros.

\section*{5. En Atenas --- discurso sobre la ciencia}
\par 
%\textsuperscript{(1476.3)}
\textsuperscript{133:5.1} Llegaron poco después al antiguo centro de la ciencia y del saber griegos. Ganid estaba muy emocionado con la idea de encontrarse en Atenas, de estar en Grecia, en el centro cultural del antiguo imperio de Alejandro, que había extendido sus fronteras hasta su propio país de la India. Había pocos negocios que tratar, de manera que Gonod pasó la mayor parte de su tiempo con Jesús y Ganid, visitando los numerosos lugares de interés y escuchando las atractivas discusiones entre el muchacho y su hábil maestro.

\par 
%\textsuperscript{(1476.4)}
\textsuperscript{133:5.2} Una gran universidad florecía aún en Atenas, y el trío hizo frecuentes visitas a sus salas de enseñanza. Jesús y Ganid habían discutido a fondo las enseñanzas de Platón cuando asistieron a las conferencias en el museo de Alejandría. Todos disfrutaron del arte de Grecia, cuyos ejemplos aún podían encontrarse aquí y allá por toda la ciudad.

\par 
%\textsuperscript{(1476.5)}
\textsuperscript{133:5.3} Tanto el padre como el hijo disfrutaron mucho con la discusión sobre la ciencia que tuvo lugar una noche en la posada entre Jesús y un filósofo griego. Después de que aquel pedante se llevara hablando cerca de tres horas y hubo terminado su discurso, Jesús dijo ---en términos adaptados al pensamiento moderno:

\par 
%\textsuperscript{(1476.6)}
\textsuperscript{133:5.4} Algún día, los científicos podrán medir la energía o las manifestaciones de fuerza de la gravedad, de la luz y de la electricidad, pero estos mismos científicos nunca podrán decir (científicamente) qué \textit{son} estos fenómenos del universo. La ciencia trata de las actividades de la energía física; la religión trata de los valores eternos. La verdadera filosofía procede de la sabiduría, que hace todo lo que puede por correlacionar estas observaciones cuantitativas y cualitativas. Siempre existe el peligro de que el científico que se ocupa de lo puramente físico pueda llegar a sufrir de orgullo matemático y de egoísmo estadístico, sin mencionar la ceguera espiritual.

\par 
%\textsuperscript{(1476.7)}
\textsuperscript{133:5.5} La lógica es válida en el mundo material, y las matemáticas son fiables cuando su aplicación se limita a las cosas físicas; pero ninguna de las dos puede considerarse enteramente digna de confianza o infalible cuando se aplican a los problemas de la vida. La vida contiene fenómenos que no son totalmente materiales. La aritmética dice que si un hombre puede esquilar una oveja en diez minutos, diez hombres pueden hacerlo en un minuto. Es un cálculo exacto, pero no es cierto, porque los diez hombres no podrían hacerlo; se estorbarían tanto los unos a los otros que el trabajo se retrasaría considerablemente.

\par 
%\textsuperscript{(1477.1)}
\textsuperscript{133:5.6} Las matemáticas afirman que si una persona representa cierta unidad de valor intelectual y moral, diez personas representarían diez veces ese valor. Pero al tratar de la personalidad humana, sería más exacto decir que una asociación semejante de personalidades es igual al cuadrado del número de personalidades que figuran en la ecuación, en lugar de su simple suma aritmética. Un grupo social de seres humanos que trabaja en armonía coordinada representa una fuerza mucho más grande que la simple suma de sus componentes.

\par 
%\textsuperscript{(1477.2)}
\textsuperscript{133:5.7} La cantidad puede ser identificada como un \textit{hecho}, convirtiéndose así en una uniformidad científica. La calidad, como está sujeta a la interpretación de la mente, representa una estimación de \textit{valores}, y por lo tanto, debe permanecer como una experiencia del individuo. Cuando la ciencia y la religión sean menos dogmáticas y toleren mejor la crítica, la filosofía empezará entonces a conseguir la \textit{unidad} en la comprensión inteligente del universo.

\par 
%\textsuperscript{(1477.3)}
\textsuperscript{133:5.8} Hay unidad en el universo cósmico, si tan sólo pudierais discernir su funcionamiento en su estado actual. El universo real es amistoso para cada hijo del Dios eterno. El verdadero problema es: ¿Cómo puede conseguir la mente finita del hombre una unidad de pensamiento lógica, verdadera y proporcionada? Este estado mental de conocimiento del universo sólo se puede obtener concibiendo la idea de que los hechos cuantitativos y los valores cualitativos tienen una causación común: el Padre Paradisiaco. Una concepción así de la realidad permite una comprensión más amplia de la unidad intencional de los fenómenos del universo; revela incluso una meta espiritual que la personalidad alcanza de manera progresiva. Éste es un concepto de unidad que puede percibir el trasfondo inmutable de un universo viviente donde las relaciones impersonales cambian sin cesar y donde las relaciones personales evolucionan continuamente.

\par 
%\textsuperscript{(1477.4)}
\textsuperscript{133:5.9} La materia, el espíritu y el estado intermedio entre ambos, son tres niveles interrelacionados e interasociados de la verdadera unidad del universo real. Por muy divergentes que puedan parecer los fenómenos universales de los hechos y de los valores, a fin de cuentas están unificados en el Supremo.

\par 
%\textsuperscript{(1477.5)}
\textsuperscript{133:5.10} La realidad de la existencia material está vinculada a la energía no reconocida así como a la materia visible. Cuando las energías del universo son frenadas hasta el punto de adquirir el grado requerido de movimiento, entonces, en condiciones favorables, estas mismas energías se convierten en masa. Y no olvidéis que la mente, la única que puede percibir la presencia de las realidades aparentes, es también real. La causa fundamental de este universo de energía-masa, de mente y de espíritu, es eterna ---existe y consiste en la naturaleza y en las reacciones del Padre Universal y de sus coordinados absolutos.

\par 
%\textsuperscript{(1477.6)}
\textsuperscript{133:5.11} Todos estaban más que asombrados por las palabras de Jesús, y cuando el griego se despidió de ellos, dijo: <<Por fin mis ojos han visto a un judío que piensa en algo más que en la superioridad racial, y que habla de algo más que de religión>>. Y se retiraron para pasar la noche.

\par 
%\textsuperscript{(1477.7)}
\textsuperscript{133:5.12} La estancia en Atenas fue agradable y provechosa, pero no particularmente fructífera en contactos humanos. Demasiados atenienses de aquellos tiempos, o estaban intelectualmente orgullosos de su reputación del pasado, o eran mentalmente estúpidos e ignorantes, pues descendían de los esclavos inferiores traídos en épocas anteriores, cuando había gloria en Grecia y sabiduría en la mente de sus habitantes. Sin embargo, aún se podían encontrar muchas mentes agudas entre los ciudadanos de Atenas.

\section*{6. En Éfeso --- discurso sobre el alma}
\par 
%\textsuperscript{(1477.8)}
\textsuperscript{133:6.1} Al partir de Atenas, los viajeros fueron por el camino de Tróades hasta Éfeso, la capital de la provincia romana de Asia. Efectuaron muchas visitas al célebre templo de Artemisa de los Efesios, a unos tres kilómetros de la ciudad. Artemisa era la diosa más famosa de toda Asia Menor y una perpetuación de la diosa madre aún más antigua de la Anatolia de épocas anteriores. Se decía que el tosco ídolo que se exhibía en el enorme templo dedicado a su culto había caído del cielo. A Ganid se le había enseñado muy pronto a respetar las imágenes como símbolos de la divinidad; no toda esta educación había sido erradicada, y pensó que lo mejor sería comprar un pequeño relicario de plata en honor de esta diosa de la fertilidad de Asia Menor. Aquella noche hablaron largo y tendido sobre la adoración de los objetos hechos con las manos humanas.

\par 
%\textsuperscript{(1478.1)}
\textsuperscript{133:6.2} Durante el tercer día de su estancia, caminaron río abajo para observar el dragado del puerto en su desembocadura. A mediodía conversaron con un joven fenicio muy desanimado y con nostalgia de su país, pero que sobre todo sentía envidia de un joven a quien habían ascendido por encima de él. Jesús le dirigió palabras de aliento y citó el antiguo proverbio hebreo: <<El talento de un hombre es el que le asegura una posición y le lleva ante los grandes hombres>>\footnote{\textit{El talento es el que asegura la posición}: Pr 18:16.}.

\par 
%\textsuperscript{(1478.2)}
\textsuperscript{133:6.3} De todas las grandes ciudades que visitaron en este viaje por el Mediterráneo, fue aquí donde menos pudieron hacer a favor del trabajo posterior de los misioneros cristianos. El cristianismo se estableció inicialmente en Éfeso gracias, en gran medida, a los esfuerzos de Pablo, que residió aquí más de dos años, fabricando tiendas para ganarse la vida y dando conferencias cada noche sobre religión y filosofía en el salón principal de la escuela de Tirano\footnote{\textit{Comienzos del cristianismo en Éfeso}: Hch 19:1-10.}.

\par 
%\textsuperscript{(1478.3)}
\textsuperscript{133:6.4} Había un pensador progresista que tenía relación con esta escuela local de filosofía, y Jesús tuvo varias reuniones provechosas con él. En el transcurso de estas conversaciones, Jesús utilizó repetidas veces la palabra <<alma>>. Este griego erudito acabó por preguntarle qué entendía él por <<alma>>, y Jesús respondió:

\par 
%\textsuperscript{(1478.4)}
\textsuperscript{133:6.5} <<El alma es la parte del hombre que refleja su yo, discierne la verdad y percibe el espíritu, y que eleva para siempre al ser humano por encima del nivel del mundo animal. La conciencia de sí, en sí misma y por sí misma, no es el alma. La autoconciencia moral es la verdadera autorrealización humana y constituye el fundamento del alma humana. El alma es esa parte del hombre que representa el valor potencial de supervivencia de la experiencia humana. La elección moral y la consecución espiritual, la capacidad para conocer a Dios y el impulso de ser semejante a él, son las características del alma. El alma del hombre no puede existir sin pensamiento moral y sin actividad espiritual. Un alma estancada es un alma moribunda. Pero el alma del hombre es distinta al espíritu divino que reside dentro de la mente. El espíritu divino llega al mismo tiempo que la mente humana efectúa su primera actividad moral, y en esa ocasión es cuando nace el alma.>>

\par 
%\textsuperscript{(1478.5)}
\textsuperscript{133:6.6} <<La salvación o la pérdida de un alma dependen de que la conciencia moral alcance o no el estado de supervivencia mediante una alianza eterna con el espíritu inmortal asociado que le ha sido dado. La salvación es la espiritualización de la autorrealización de la conciencia moral, que adquiere de este modo un valor de supervivencia. Todos los tipos de conflictos del alma consisten en la falta de armonía entre la conciencia de sí moral o espiritual, y la conciencia de sí puramente intelectual.>>

\par 
%\textsuperscript{(1478.6)}
\textsuperscript{133:6.7} <<Cuando el alma humana está madura, ennoblecida y espiritualizada, se acerca al estado celestial en el sentido de que casi llega a ser una entidad intermedia entre lo material y lo espiritual, entre el yo material y el espíritu divino. El alma evolutiva de un ser humano es difícil de describir y aun más difícil de demostrar, porque no puede ser descubierta por el método de la investigación material ni por el de la prueba espiritual. La ciencia material no puede demostrar la existencia de un alma, y la prueba puramente espiritual tampoco. A pesar de que la ciencia material y los criterios espirituales no puedan descubrir la existencia del alma humana, todo mortal moralmente consciente \textit{conoce} la existencia de \textit{su} alma como una experiencia personal \textit{real} y efectiva>>.

\section*{7. La estancia en Chipre --- discurso sobre la mente}
\par 
%\textsuperscript{(1479.1)}
\textsuperscript{133:7.1} Poco después, los viajeros se hicieron a la vela para Chipre, con una escala en Rodas. Disfrutaron de este largo viaje marítimo y llegaron a su isla de destino con el cuerpo descansado y el espíritu renovado.

\par 
%\textsuperscript{(1479.2)}
\textsuperscript{133:7.2} Habían planeado disfrutar de un período de verdadero descanso y esparcimiento durante esta visita a Chipre, pues su gira por el Mediterráneo estaba llegando a su fin. Desembarcaron en Pafos y empezaron enseguida a reunir las provisiones para su estancia de varias semanas en las montañas cercanas. Al tercer día de su llegada, partieron hacia las colinas con sus animales bien cargados.

\par 
%\textsuperscript{(1479.3)}
\textsuperscript{133:7.3} El trío pasó quince días sumamente agradables, y luego, de repente, el joven Ganid cayó gravemente enfermo. Durante dos semanas padeció una fiebre intensa, que a menudo lo llevaba hasta el delirio; tanto Jesús como Gonod se dedicaron de lleno a cuidar al muchacho enfermo. Jesús se ocupó del chico con habilidad y ternura, y el padre se quedó asombrado por la delicadeza y la pericia que Jesús demostró en todos sus cuidados hacia el joven enfermo. Estaban lejos de toda morada humana, y el muchacho se encontraba demasiado enfermo como para ser trasladado; así pues, se prepararon lo mejor que pudieron para cuidarlo hasta que se recuperara allí mismo en las montañas.

\par 
%\textsuperscript{(1479.4)}
\textsuperscript{133:7.4} Durante las tres semanas de la convalecencia de Ganid, Jesús le contó muchas cosas interesantes sobre la naturaleza y sus diversas manifestaciones. Se divirtieron mucho mientras correteaban por las montañas, con el muchacho haciendo preguntas, Jesús respondiéndolas y el padre maravillándose con toda la escena.

\par 
%\textsuperscript{(1479.5)}
\textsuperscript{133:7.5} La última semana de su estancia en las montañas, Jesús y Ganid tuvieron una larga conversación sobre las funciones de la mente humana. Después de varias horas de discusión, el joven hizo la pregunta siguiente: <<Pero, Maestro, ¿qué quieres decir cuando afirmas que el hombre experimenta una forma de conciencia de sí más elevada que la que experimentan los animales más evolucionados?>> Transcrito en un lenguaje moderno, Jesús le contestó:

\par 
%\textsuperscript{(1479.6)}
\textsuperscript{133:7.6} Hijo mío, ya te he hablado mucho de la mente del hombre y del espíritu divino que vive en ella, pero ahora, permíteme recalcar que la conciencia de sí es una \textit{realidad}. Cuando un animal se vuelve consciente de sí mismo, se convierte en un hombre primitivo. Este logro es el resultado de una coordinación de funciones entre la energía impersonal y la mente que concibe el espíritu; este fenómeno es el que justifica la donación de un punto focal absoluto a la personalidad humana: el espíritu del Padre que está en los cielos.

\par 
%\textsuperscript{(1479.7)}
\textsuperscript{133:7.7} Las ideas no son simplemente un registro de sensaciones; las ideas son sensaciones, más las interpretaciones reflexivas del yo personal; y el yo es más que la suma de sus sensaciones. En una individualidad que evoluciona empieza a haber un indicio de acercamiento a la unidad, y esa unidad se deriva de la presencia interior de un fragmento de la unidad absoluta, que activa espiritualmente a esa mente consciente de origen animal.

\par 
%\textsuperscript{(1479.8)}
\textsuperscript{133:7.8} Ningún simple animal puede poseer una conciencia del tiempo. Los animales poseen una coordinación fisiológica de sensaciones y reconocimientos asociados, y la memoria correspondiente; pero ninguno de ellos experimenta un reconocimiento de sensaciones que tenga un significado, ni muestra una asociación intencional de estas experiencias físicas combinadas, tal como se manifiestan en las conclusiones de las interpretaciones humanas inteligentes y reflexivas. Este hecho de la existencia autoconsciente, asociado a la realidad de su experiencia espiritual posterior, convierte al hombre en un hijo potencial del universo y prefigura que alcanzará finalmente a la Unidad Suprema del universo.

\par 
%\textsuperscript{(1480.1)}
\textsuperscript{133:7.9} El yo humano tampoco es simplemente la suma de sus estados sucesivos de conciencia. Sin el funcionamiento eficaz de un factor que ordena y asocia la conciencia, no existiría una unidad suficiente como para justificar la denominación de individualidad. Una mente no unificada de este tipo difícilmente podría alcanzar los niveles de conciencia del estado humano. Si las asociaciones de conciencia no fueran más que un accidente, la mente de todos los hombres manifestaría entonces las asociaciones incontroladas y desatinadas de ciertas fases de la locura mental.

\par 
%\textsuperscript{(1480.2)}
\textsuperscript{133:7.10} Una mente humana basada exclusivamente en la conciencia de las sensaciones físicas, nunca podría alcanzar los niveles espirituales; este tipo de mente material carecería totalmente del sentido de los valores morales y estaría desprovista del sentido director de dominación espiritual, que es tan esencial para conseguir la unidad armoniosa de la personalidad en el tiempo, y que es inseparable de la supervivencia de la personalidad en la eternidad.

\par 
%\textsuperscript{(1480.3)}
\textsuperscript{133:7.11} La mente humana empieza pronto a manifestar unas cualidades que son supermateriales; el intelecto humano verdaderamente reflexivo no está atado del todo por los límites del tiempo. El hecho de que los individuos sean tan diferentes en las acciones de su vida, no solamente indica las variadas dotaciones hereditarias y las diferentes influencias del entorno, sino también el grado de unificación que el yo ha conseguido con el espíritu interior del Padre, la medida en que están identificados el uno con el otro.

\par 
%\textsuperscript{(1480.4)}
\textsuperscript{133:7.12} La mente humana no soporta bien el conflicto de la doble fidelidad. Cuando un alma se esfuerza por servir al bien y al mal a la vez, experimenta una tensión extrema. La mente supremamente feliz y eficazmente unificada es la que está dedicada por entero a hacer la voluntad del Padre que está en los cielos. Los conflictos no resueltos destruyen la unidad y pueden terminar en el desquiciamiento mental. No obstante, el carácter de supervivencia de un alma no se favorece intentando asegurarse la paz mental a cualquier precio, mediante el abandono de las nobles aspiraciones o transigiendo con los ideales espirituales. Esta paz se alcanza más bien afirmando constantemente el triunfo de lo que es verdadero, y esta victoria se consigue venciendo al mal con la poderosa fuerza del bien.

\par 
%\textsuperscript{(1480.5)}
\textsuperscript{133:7.13} Al día siguiente partieron hacia Salamina, donde se embarcaron para Antioquía, en la costa de Siria.

\section*{8. En Antioquía}
\par 
%\textsuperscript{(1480.6)}
\textsuperscript{133:8.1} Antioquía era la capital de la provincia romana de Siria, y el gobernador imperial tenía aquí su residencia. Antioquía tenía medio millón de habitantes; era la tercera ciudad del imperio en importancia y la primera en perversidad y flagrante inmoralidad. Gonod tenía que tratar muchísimos negocios, de manera que Jesús y Ganid estuvieron a solas la mayoría del tiempo. Visitaron todas las cosas de esta ciudad políglota excepto el bosquecillo de Dafne. Gonod y Ganid fueron a visitar este notorio paraje de la indecencia, pero Jesús se negó a acompañarlos. Aquellas escenas no eran tan chocantes para los indios, pero eran repelentes para un hebreo idealista.

\par 
%\textsuperscript{(1480.7)}
\textsuperscript{133:8.2} Jesús se fue poniendo serio y pensativo a medida que se acercaba a Palestina y al final de su viaje. Conversó con poca gente en Antioquía y rara vez se paseó por la ciudad. Después de mucho preguntar por qué su maestro manifestaba tan poco interés por Antioquía, Ganid consiguió finalmente que Jesús dijera: <<Esta ciudad no está lejos de Palestina; quizás regrese aquí algún día>>.

\par 
%\textsuperscript{(1481.1)}
\textsuperscript{133:8.3} Ganid tuvo una experiencia muy interesante en Antioquía. Este joven había demostrado ser un alumno capaz y ya había empezado a llevar a la práctica algunas de las enseñanzas de Jesús. Había cierto indio relacionado con los negocios de su padre en Antioquía, que se había vuelto tan desagradable y enfadado que habían pensado en despedirlo. Cuando Ganid se enteró, se dirigió al centro de negocios de su padre y tuvo una larga conversación con su compatriota. Este hombre tenía el sentimiento de que le habían asignado la tarea equivocada. Ganid le habló del Padre que está en los cielos y le amplió de diversas maneras su visión de la religión. Pero de todo lo que dijo Ganid, lo que más le impactó fue la cita de un proverbio hebreo, cuyas palabras de sabiduría decían: <<Cualquier cosa que tu mano tenga que hacer, hazla con todas tus fuerzas>>\footnote{\textit{Haz todo con todas tus fuerzas}: Ec 9:10.}.

\par 
%\textsuperscript{(1481.2)}
\textsuperscript{133:8.4} Después de preparar su equipaje para la caravana de camellos, descendieron hasta Sidón y desde allí fueron a Damasco; tres días después se prepararon para el largo trayecto a través de las arenas del desierto.

\section*{9. En Mesopotamia}
\par 
%\textsuperscript{(1481.3)}
\textsuperscript{133:9.1} El viaje en caravana a través del desierto no era una experiencia nueva para estos grandes viajeros. Después de ver a su maestro ayudar a cargar sus veinte camellos, y al observar que se ofrecía como voluntario para conducir su propio animal, Ganid exclamó: <<Maestro, ¿hay algo que no sepas hacer?>> Jesús se limitó a sonreír, diciendo: <<Un maestro no deja de tener méritos a los ojos de un alumno aplicado>>. Y partieron así para la antigua ciudad de Ur.

\par 
%\textsuperscript{(1481.4)}
\textsuperscript{133:9.2} Jesús se interesó mucho por la historia antigua de Ur\footnote{\textit{Ur, lugar de nacimiento de Abraham}: Gn 11:27-31.}, lugar donde nació Abraham, y también se quedó fascinado con las ruinas y tradiciones de Susa\footnote{\textit{Historias de Ester y Susa}: Est 1:2ff.}, de tal manera que Gonod y Ganid prolongaron su estancia en estas regiones tres semanas más, con el fin de darle más tiempo a Jesús para que continuara sus investigaciones, y también para encontrar la mejor ocasión de persuadirlo para que regresara con ellos a la India.

\par 
%\textsuperscript{(1481.5)}
\textsuperscript{133:9.3} Fue en Ur donde Ganid tuvo una larga conversación con Jesús respecto a la diferencia entre el conocimiento, la sabiduría y la verdad. Se quedó encantado con el proverbio del sabio hebreo: <<La sabiduría es lo principal; por lo tanto, adquiere sabiduría. Junto a tu búsqueda del conocimiento, adquiere la comprensión. Exalta la sabiduría y ella te hará progresar. Te llevará hasta los honores con tal que la practiques>>\footnote{\textit{La sabiduría es lo principal}: Pr 4:7-8.}.

\par 
%\textsuperscript{(1481.6)}
\textsuperscript{133:9.4} Por fin llegó el día de la separación. Todos fueron valientes, especialmente el joven, pero fue una dura prueba. Tenían lágrimas en los ojos, pero valor en el corazón. Al despedirse de su maestro, Ganid le dijo: <<Adiós, Maestro, pero no para siempre. Cuando vuelva a Damasco, te buscaré. Te quiero, pues creo que el Padre que está en los cielos debe parecerse algo a ti; al menos sé que tú te pareces mucho a lo que me has contado de él. Recordaré tu enseñanza, pero por encima de todo, nunca te olvidaré>>. El padre dijo: <<Adiós a un gran maestro, a alguien que nos ha hecho mejores y que nos ha ayudado a conocer a Dios>>. Y Jesús respondió: <<Que la paz esté con vosotros, y que la bendición del Padre que está en los cielos permanezca siempre con vosotros>>. Y Jesús se quedó en la orilla, contemplando cómo la pequeña barca los llevaba hasta el barco anclado. El Maestro se separó así de sus amigos de la India en Charax, para no volver a verlos nunca más en este mundo; ellos tampoco supieron nunca, en este mundo, que el hombre que más tarde apareció como Jesús de Nazaret era este mismo amigo que acababan de dejar: Josué su instructor.

\par 
%\textsuperscript{(1481.7)}
\textsuperscript{133:9.5} En la India, Ganid creció y se volvió un hombre influyente, un digno sucesor de su eminente padre; divulgó por todas partes muchas de las nobles verdades que había aprendido de Jesús, su amado maestro. Más tarde en la vida, cuando Ganid oyó hablar del extraño educador de Palestina que terminó su carrera en una cruz, aunque reconoció la similitud entre el evangelio de este Hijo del Hombre y las enseñanzas de su preceptor judío, nunca se le ocurrió pensar que los dos eran de hecho la misma persona.

\par 
%\textsuperscript{(1482.1)}
\textsuperscript{133:9.6} Así terminó el capítulo de la vida del Hijo del Hombre que podría titularse: \textit{La misión de Josué el educador}.