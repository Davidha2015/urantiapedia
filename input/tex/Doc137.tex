\chapter{Documento 137. El tiempo de espera en Galilea}
\par 
%\textsuperscript{(1524.1)}
\textsuperscript{137:0.1} EL SÁBADO 23 de febrero del año 26, por la mañana temprano, Jesús descendió de las colinas para reunirse con los compañeros de Juan que acampaban en Pella. Todo este día Jesús se mezcló con la multitud. Atendió a un chico que se había lastimado en una caída y se desplazó hasta el cercano pueblo de Pella para poner al niño a salvo en manos de sus padres.

\section*{1. La elección de los cuatro primeros apóstoles}
\par 
%\textsuperscript{(1524.2)}
\textsuperscript{137:1.1} Durante este sábado, dos de los principales discípulos de Juan pasaron mucho tiempo con Jesús. De todos los seguidores de Juan, uno llamado Andrés es el que estaba más profundamente impresionado por Jesús. Lo acompañó hasta Pella con el muchacho lesionado, y por el camino de vuelta al campamento de Juan le hizo muchas preguntas a Jesús; poco antes de llegar a su destino, los dos se detuvieron para tener una breve conversación, durante la cual Andrés dijo: <<Te he estado observando desde que viniste a Cafarnaúm, y creo que eres el nuevo Instructor; aunque no comprendo toda tu enseñanza, estoy plenamente decidido a seguirte. Quisiera sentarme a tus pies para aprender toda la verdad sobre el nuevo reino>>. Con una cordial resolución, Jesús acogió a Andrés como el primer apóstol\footnote{\textit{Selección de Andrés}: Mt 4:18-20; Mc 1:16-18; Jn 1:40.} de aquel grupo de doce que iba a trabajar con él en la obra de establecer el nuevo reino de Dios en el corazón de los hombres.

\par 
%\textsuperscript{(1524.3)}
\textsuperscript{137:1.2} Andrés había observado en silencio la labor de Juan y creía sinceramente en ella. Tenía un hermano muy capaz y entusiasta, llamado Simón, que era uno de los principales discípulos de Juan. No sería impropio decir que Simón era uno de los apoyos más importantes de Juan.

\par 
%\textsuperscript{(1524.4)}
\textsuperscript{137:1.3} Poco después de que Jesús y Andrés regresaran al campamento, Andrés buscó a su hermano Simón y llevándolo aparte le comunicó que estaba convencido de que Jesús era el gran Instructor, y que se había comprometido a ser su discípulo. Continuó diciendo que Jesús había aceptado su propuesta de servicio, y le sugirió que él (Simón) fuera también a Jesús y se ofreciera para unirse al servicio del nuevo reino. Simón dijo: <<Desde que ese hombre vino a trabajar al taller de Zebedeo, he creído que había sido enviado por Dios, pero ¿qué hacemos con Juan? ¿Vamos a abandonarlo? ¿Es esto lo que debemos hacer?>> Con lo cual, acordaron ir enseguida a consultar a Juan. Juan se entristeció con la idea de perder a dos de sus capaces consejeros y más prometedores discípulos, pero contestó valientemente a sus preguntas diciendo: <<Esto sólo es el principio; mi obra terminará dentro de poco y todos nos convertiremos en sus discípulos>>. Entonces Andrés le hizo señas a Jesús y le anunció aparte que su hermano deseaba entrar al servicio del nuevo reino. Al acoger a Simón como su segundo apóstol, Jesús le dijo: <<Simón, tu entusiasmo es loable, pero peligroso para el trabajo del reino. Te recomiendo que seas más cuidadoso con tus palabras. Desearía cambiar tu nombre por el de Pedro>>\footnote{\textit{Selección de Simón Pedro}: Mt 4:18-20; Mc 1:16-18; Lc 5:1-9; Jn 1:41-42.}.

\par 
%\textsuperscript{(1525.1)}
\textsuperscript{137:1.4} Los padres del chico lastimado, que vivían en Pella, habían rogado a Jesús que pasara la noche con ellos, que se considerara como en su casa, y él había prometido volver. Antes de separarse de Andrés y de su hermano, Jesús les dijo: <<Mañana temprano iremos a Galilea>>.

\par 
%\textsuperscript{(1525.2)}
\textsuperscript{137:1.5} Después de que Jesús hubiera regresado a Pella para pasar la noche, y mientras que Andrés y Simón discutían todavía sobre la naturaleza de su servicio en el establecimiento del reino por venir, Santiago y Juan, los hijos de Zebedeo, llegaron al lugar. Acababan de regresar de su larga e inútil búsqueda de Jesús en las colinas. Cuando oyeron contar a Simón Pedro cómo él y su hermano Andrés se habían convertido en los primeros consejeros aceptados del nuevo reino, y que iban a partir a la mañana siguiente con su nuevo Maestro para Galilea, Santiago y Juan se entristecieron. Conocían a Jesús desde hacía tiempo y lo amaban. Lo habían buscado durante muchos días en las colinas, y ahora regresaban para enterarse de que otros habían sido elegidos antes que ellos. Preguntaron adónde había ido Jesús y se dieron prisa en encontrarlo.

\par 
%\textsuperscript{(1525.3)}
\textsuperscript{137:1.6} Jesús estaba durmiendo cuando llegaron a su habitación, pero lo despertaron diciendo: <<Mientras nosotros, que hemos vivido tanto tiempo contigo, te buscábamos en las colinas, ¿cómo es que prefieres a otros antes que a nosotros, y escoges a Andrés y a Simón como tus primeros asociados en el nuevo reino?>> Jesús les respondió: <<Serenad vuestro corazón y preguntaos, `¿quién os ha ordenado buscar al Hijo del Hombre mientras se dedicaba a los asuntos de su Padre?'>>. Después de contar los detalles de su larga búsqueda en las colinas, Jesús continuó enseñándoles: <<Deberíais aprender a buscar el secreto del nuevo reino en vuestro corazón, y no en las colinas. Aquello que buscabais ya estaba presente en vuestra alma. En verdad sois mis hermanos ---no necesitabais que yo os aceptara--- ya pertenecíais al reino. Tened buen ánimo y preparaos también para acompañarnos mañana a Galilea>>. Juan se atrevió entonces a preguntar: <<Pero, Maestro, ¿Santiago y yo seremos tus asociados en el nuevo reino, como lo son Andrés y Simón?>> Jesús puso una mano en el hombro de cada uno de ellos y dijo: <<Hermanos míos, ya estabais conmigo en el espíritu del reino, incluso antes de que los otros solicitaran ser admitidos. Vosotros, mis hermanos, no tenéis necesidad de presentar una petición para entrar en el reino; habéis estado conmigo en el reino desde el principio. Ante los hombres, otros pueden tener prioridad sobre vosotros, pero en mi corazón ya contaba con vosotros para los consejos del reino, incluso antes de que pensarais en pedírmelo. También podríais haber sido los primeros ante los hombres, si no os hubierais ausentado para dedicaros a la tarea bien intencionada, pero impuesta por vosotros mismos, de buscar a alguien que no estaba perdido. En el reino venidero, no os preocupéis por las cosas que alimentan vuestra ansiedad, sino más bien interesaos en hacer solamente, en todo momento, la voluntad del Padre que está en los cielos>>\footnote{\textit{Selección de Santiago y Juan}: Mt 4:21-22; Mc 1:19-20; Lc 5:10-11.}.

\par 
%\textsuperscript{(1525.4)}
\textsuperscript{137:1.7} Santiago y Juan aceptaron la reprimenda de buena gana; nunca más tuvieron envidia de Andrés y de Simón. Se prepararon para salir a la mañana siguiente para Galilea con los otros dos apóstoles asociados. A partir de este día, la palabra `apóstol' fue empleada para diferenciar a la familia elegida de consejeros de Jesús de la vasta multitud de discípulos creyentes que le siguieron posteriormente.

\par 
%\textsuperscript{(1525.5)}
\textsuperscript{137:1.8} Avanzada la noche, Santiago, Juan, Andrés y Simón mantuvieron una conversación con Juan el Bautista. Con lágrimas en los ojos pero con voz firme, el fornido profeta judeo renunció a dos de sus principales discípulos para que fueran los apóstoles del Príncipe galileo del reino por venir.

\section*{2. La elección de Felipe y de Natanael}
\par 
%\textsuperscript{(1526.1)}
\textsuperscript{137:2.1} El domingo por la mañana 24 de febrero del año 26, Jesús se despidió de Juan el Bautista al borde del río cerca de Pella, para no volverlo a ver nunca más en la carne.

\par 
%\textsuperscript{(1526.2)}
\textsuperscript{137:2.2} Aquel día, mientras Jesús y sus cuatro discípulos-apóstoles partían para Galilea\footnote{\textit{Jesús en Galilea}: Mt 4:23; Mc 1:21; Lc 4:14; Jn 1:43.}, un gran alboroto tuvo lugar en el campamento de los seguidores de Juan. La primera gran división estaba a punto de producirse. El día anterior, Juan había dicho explícitamente a Andrés y a Esdras que Jesús era el Libertador\footnote{\textit{Pronunciamiento de Juan}: Jn 1:35-37.}. Andrés decidió seguir a Jesús, pero Esdras rechazó al apacible carpintero de Nazaret, proclamando a sus asociados: <<El profeta Daniel afirma que el Hijo del Hombre vendrá con las nubes del cielo, lleno de poder y gran gloria. Este carpintero galileo, este constructor de barcas de Cafarnaúm, no puede ser el Libertador. Un don semejante de Dios, ¿puede salir de Nazaret? Ese Jesús es un pariente de Juan, y nuestro maestro se ha dejado engañar por la gran bondad de su corazón. Mantengámonos apartados de ese falso Mesías>>\footnote{\textit{Cita de Daniel}: Dn 7:13.}. Cuando Juan le regañó por estas declaraciones, Esdras se retiró llevándose a muchos discípulos y se dirigió apresuradamente hacia el sur. Este grupo continuó bautizando en nombre de Juan y fundó finalmente una secta con aquellos que creían en Juan pero rehusaban aceptar a Jesús. Un resto de este grupo aún sobrevive en Mesopotamia en la actualidad.

\par 
%\textsuperscript{(1526.3)}
\textsuperscript{137:2.3} Mientras estos disturbios se fraguaban entre los seguidores de Juan, Jesús y sus cuatro discípulos-apóstoles avanzaban a buen paso hacia Galilea. Antes de cruzar el Jordán para ir a Nazaret por el camino de Naín, Jesús miró hacia adelante y vio por la carretera a un tal Felipe de Betsaida que venía hacia ellos con un amigo. Jesús había conocido a Felipe anteriormente, y los cuatro nuevos apóstoles también lo conocían bien. Iba de camino con su amigo Natanael para ver a Juan en Pella a fin de informarse mejor sobre la llegada anunciada del reino de Dios, y se sintió encantado de saludar a Jesús. Felipe había admirado a Jesús desde que vino por primera vez a Cafarnaúm. Pero Natanael, que vivía en Caná de Galilea, no conocía a Jesús. Felipe se adelantó para saludar a sus amigos, mientras Natanael descansaba a la sombra de un árbol al borde del camino.

\par 
%\textsuperscript{(1526.4)}
\textsuperscript{137:2.4} Pedro llevó aparte a Felipe y procedió a explicarle que todos ellos, refiriéndose a él mismo, Andrés, Santiago y Juan, se habían vuelto compañeros de Jesús en el nuevo reino, e incitó vivamente a Felipe a que se ofreciera para este servicio. Felipe se encontró en un aprieto. ¿Qué debía hacer? Aquí, sin el menor preaviso ---al borde del camino cerca del Jordán--- había surgido la cuestión más importante de toda una vida, y tenía que tomar una decisión inmediata. Mientras Felipe conversaba seriamente con Pedro, Andrés y Juan, Jesús describía a Santiago el camino a seguir a través de Galilea hasta Cafarnaúm. Finalmente, Andrés sugirió a Felipe: <<¿Por qué no le preguntas al Maestro?>>.

\par 
%\textsuperscript{(1526.5)}
\textsuperscript{137:2.5} Felipe se dio cuenta repentinamente de que Jesús era realmente un gran hombre, posiblemente el Mesías, y decidió atenerse a lo que Jesús decidiera en este asunto. Fue directamente hacia él y le preguntó: <<Maestro, ¿debo ir hasta Juan o unirme a mis amigos que te siguen?>> Y Jesús respondió: <<Sígueme>>. Felipe se emocionó con la certidumbre de haber encontrado al Libertador\footnote{\textit{Selección de Felipe}: Jn 1:43-44.}.

\par 
%\textsuperscript{(1526.6)}
\textsuperscript{137:2.6} Entonces Felipe le hizo señas al grupo para que permanecieran donde estaban, mientras se apresuraba a revelar su decisión a su amigo Natanael\footnote{\textit{Felipe y Natanael}: Jn 1:45-46.}, que aún continuaba debajo de la morera reflexionando sobre todas las cosas que había oído respecto a Juan el Bautista, el reino por venir y el Mesías esperado. Felipe interrumpió esta meditación, exclamando: <<He encontrado al Libertador, aquel de quien han escrito Moisés y los profetas y a quien Juan ha proclamado>>. Natanael levantó la vista e inquirió: <<¿De dónde viene ese maestro?>> Y Felipe replicó: <<Es Jesús de Nazaret, el hijo de José, el carpintero, que reside desde hace poco en Cafarnaúm>>. Entonces Natanael, un poco sobresaltado, preguntó: <<¿Una cosa tan buena puede salir de Nazaret?>> Pero Felipe, cogiéndolo por el brazo, le dijo: <<Ven a ver>>.

\par 
%\textsuperscript{(1527.1)}
\textsuperscript{137:2.7} Felipe condujo a Natanael hasta Jesús, el cual, mirando bondadosamente de frente a este hombre sincero que dudaba, dijo: <<He aquí a un auténtico israelita, en quien no hay falsedad. Sígueme>>. Y Natanael, volviéndose hacia Felipe, le dijo: <<Tienes razón. Es en verdad un maestro de hombres. Yo también le seguiré, si soy digno>>. Jesús hizo un gesto afirmativo con la cabeza a Natanael, diciéndole de nuevo: <<Sígueme>>\footnote{\textit{La selección de Natanael}: Jn 1:47-51.}.

\par 
%\textsuperscript{(1527.2)}
\textsuperscript{137:2.8} Jesús ya había reunido a la mitad de su futuro cuerpo de asociados íntimos, cinco que lo conocían desde hacía algún tiempo más un extraño, Natanael. Sin más dilación, atravesaron el Jordán, pasaron por el pueblo de Naín y al final de la tarde llegaron a Nazaret.

\par 
%\textsuperscript{(1527.3)}
\textsuperscript{137:2.9} Todos pasaron la noche con José, en la casa de la infancia de Jesús. Los compañeros de Jesús no entendieron muy bien por qué su maestro recién descubierto estaba tan preocupado por destruir completamente todos los vestigios de su escritura que permanecían en la casa, tales como los Diez Mandamientos y otras sentencias y refranes. Pero esta conducta, unida al hecho de que nunca más lo vieron escribir ---excepto en el polvo o en la arena--- hizo una profunda impresión en sus mentes.

\section*{3. La visita a Cafarnaúm}
\par 
%\textsuperscript{(1527.4)}
\textsuperscript{137:3.1} Al día siguiente, Jesús envió a sus apóstoles a Caná, ya que todos ellos estaban invitados a la boda\footnote{\textit{La boda en Caná}: Jn 2:1-2.} de una joven sobresaliente de aquella ciudad, mientras él se preparaba para hacerle una breve visita a su madre en Cafarnaúm, deteniéndose en Magdala para ver a su hermano Judá.

\par 
%\textsuperscript{(1527.5)}
\textsuperscript{137:3.2} Antes de salir de Nazaret, los nuevos asociados de Jesús contaron a José y a otros miembros de la familia de Jesús los acontecimientos maravillosos del entonces pasado reciente, y expresaron francamente su creencia de que Jesús era el libertador tanto tiempo esperado. Estos miembros de la familia de Jesús discutieron sobre todo esto, y José dijo: <<Después de todo, quizás mamá tenía razón ---quizás nuestro extraño hermano sea el futuro rey>>.

\par 
%\textsuperscript{(1527.6)}
\textsuperscript{137:3.3} Judá había estado presente en el bautismo de Jesús y, con su hermano Santiago, se había vuelto un firme creyente en la misión de Jesús en la Tierra. Aunque tanto Santiago como Judá estaban muy perplejos respecto a la naturaleza de la misión de su hermano, su madre había resucitado todas sus antiguas esperanzas de que Jesús sería el Mesías, el hijo de David, y animaba a sus hijos a que tuvieran fe en su hermano como libertador de Israel.

\par 
%\textsuperscript{(1527.7)}
\textsuperscript{137:3.4} Jesús llegó a Cafarnaúm el lunes por la noche, pero no fue a su propia casa, donde vivían Santiago y su madre; fue directamente a la casa de Zebedeo. Todos sus amigos de Cafarnaúm advirtieron un cambio grande y agradable en él. Una vez más parecía relativamente contento y más semejante a como había sido durante sus primeros años en Nazaret. En los años anteriores a su bautismo y a los períodos de aislamiento justo antes y después del mismo, se había vuelto cada vez más serio y reservado. Ahora, a todos ellos les parecía que volvía a ser como antes. Había en él algo de importancia majestuosa y de aspecto sublime, pero estaba nuevamente desenfadado y alegre.

\par 
%\textsuperscript{(1528.1)}
\textsuperscript{137:3.5} María se estremecía de esperanza. Preveía que la promesa de Gabriel iba a cumplirse próximamente. Esperaba que pronto toda Palestina se quedaría sorprendida y pasmada ante la revelación milagrosa de su hijo como rey sobrenatural de los judíos. Pero a las numerosas preguntas que le hicieron su madre, Santiago, Judá y Zebedeo, Jesús se limitó a responder sonriendo: <<Es mejor que me quede aquí durante algún tiempo; debo hacer la voluntad de mi Padre que está en los cielos>>.

\par 
%\textsuperscript{(1528.2)}
\textsuperscript{137:3.6} Al día siguiente, martes, todos fueron a Caná para asistir a la boda de Noemí, que iba a celebrarse al otro día. A pesar de las advertencias reiteradas de Jesús de que no hablaran a nadie de él <<hasta que llegara la hora del Padre>>, ellos insistieron en divulgar discretamente la noticia de que habían encontrado al Libertador. Cada uno de ellos esperaba con confianza que Jesús inauguraría la toma de posesión de su autoridad mesiánica en la próxima boda de Caná, y que lo haría con un gran poder y una grandeza sublime. Recordaban lo que les habían dicho sobre los fenómenos que acompañaron a su bautismo, y creían que su carrera futura en la Tierra estaría marcada de manifestaciones crecientes de maravillas sobrenaturales y de demostraciones milagrosas. En consecuencia, toda la región se preparó para reunirse en Caná para la fiesta nupcial de Noemí y Johab, el hijo de Natán.

\par 
%\textsuperscript{(1528.3)}
\textsuperscript{137:3.7} Hacía años que María no estaba tan alegre. Viajó hasta Caná con el ánimo de una reina madre que va a presenciar la coronación de su hijo. Desde que Jesús tenía trece años, su familia y sus amigos no lo habían visto tan despreocupado y feliz, tan atento y comprensivo con los anhelos y deseos de sus asociados, tan tiernamente compasivo. Así que todos cuchicheaban entre ellos, en pequeños grupos, preguntándose qué iba a suceder. ¿Cuál sería el próximo acto de este extraño personaje? ¿Cómo anunciaría la gloria del reino venidero? Todos estaban emocionados con la idea de que iban a estar presentes para contemplar la revelación de la fuerza y del poder del Dios de Israel.

\section*{4. Las bodas de Caná}
\par 
%\textsuperscript{(1528.4)}
\textsuperscript{137:4.1} Hacia el mediodía del miércoles, cerca de mil convidados habían llegado a Caná, más de cuatro veces el número de invitados a la fiesta nupcial. Los judíos tenían la costumbre de celebrar los casamientos los miércoles, y las invitaciones habían sido enviadas con un mes de antelación. Durante la mañana y el principio de la tarde, aquello se parecía más a una recepción pública para Jesús que a una boda. Todo el mundo quería saludar a este galileo casi famoso, y él era sumamente cordial con todos, jóvenes y adultos, judíos y gentiles. Todos se regocijaron cuando Jesús accedió a encabezar la procesión nupcial preliminar.

\par 
%\textsuperscript{(1528.5)}
\textsuperscript{137:4.2} Jesús era ahora enteramente consciente de su existencia humana, de su preexistencia divina, y del estado de sus naturalezas humana y divina combinadas o fusionadas. Con un equilibrio perfecto podía jugar en todo momento su papel humano o asumir inmediatamente las prerrogativas de la personalidad de su naturaleza divina.

\par 
%\textsuperscript{(1528.6)}
\textsuperscript{137:4.3} A medida que pasaba el día, Jesús se fue haciendo cada vez más consciente de que la gente esperaba que efectuara algún prodigio; comprendió especialmente que su familia y sus seis discípulos-apóstoles esperaban que anunciara su futuro reino de una manera apropiada mediante alguna manifestación sorprendente y sobrenatural.

\par 
%\textsuperscript{(1529.1)}
\textsuperscript{137:4.4} Al principio de la tarde, María llamó a Santiago y juntos se atrevieron a acercarse a Jesús para preguntarle si estaría dispuesto a confiar en ellos hasta el punto de informarles en qué momento y lugar de las ceremonias de la boda había planeado manifestarse como un <<ser sobrenatural>>. En cuanto abordaron esta cuestión con Jesús, vieron que habían suscitado su indignación característica. Él se limitó a decir: <<Si me amáis, entonces disponeos a aguardar conmigo mientras espero la voluntad de mi Padre que está en los cielos>>. Pero la elocuencia de su reproche residía en la expresión de su rostro.

\par 
%\textsuperscript{(1529.2)}
\textsuperscript{137:4.5} El Jesús humano se sintió muy decepcionado por esta acción de su madre, y se quedó muy pensativo ante su propia reacción a la propuesta insinuante de ella de que se permitiera darse el gusto de alguna demostración exterior de su divinidad. Ésta era precisamente una de las cosas que había decidido no hacer cuando estuvo recientemente aislado en las colinas. María estuvo muy deprimida durante varias horas. Le dijo a Santiago: <<No puedo comprenderlo. ¿Qué significa todo esto? ¿Su extraña conducta nunca tendrá fin?>> Santiago y Judá trataron de consolar a su madre, mientras que Jesús se retiraba para estar a solas durante una hora. Pero volvió a la reunión, mostrándose una vez más alegre y desenfadado.

\par 
%\textsuperscript{(1529.3)}
\textsuperscript{137:4.6} El casamiento tuvo lugar en medio de un silencio expectante, pero toda la ceremonia finalizó y el huésped de honor no hizo un solo gesto, no pronunció una sola palabra. Entonces se empezó a cuchichear que el carpintero y constructor de barcas, proclamado por Juan como <<el Libertador>>, descubriría su juego durante las fiestas de la noche, quizás en la cena nupcial. Pero Jesús apartó eficazmente de la mente de sus seis discípulos-apóstoles toda esperanza de una demostración de este tipo, cuando los reunió un poco antes de la cena nupcial y les dijo muy seriamente: <<No creáis que he venido a este lugar para efectuar algún prodigio que satisfaga a los curiosos o que convenza a los que dudan. Estamos aquí más bien para esperar la voluntad de nuestro Padre que está en los cielos>>. Cuando María y los demás lo vieron deliberando con sus asociados, estuvieron plenamente persuadidos en su propia mente de que algo extraordinario estaba a punto de suceder. Y todos se sentaron para disfrutar en buena compañía de la cena nupcial y de la noche de fiesta.

\par 
%\textsuperscript{(1529.4)}
\textsuperscript{137:4.7} El padre del novio había suministrado vino en abundancia para todos los huéspedes invitados a la fiesta nupcial, pero ¿cómo iba a suponer que la boda de su hijo se iba a convertir en un acontecimiento tan íntimamente asociado con la esperada manifestación de Jesús como libertador mesiánico? Estaba encantado de tener el honor de contar entre sus huéspedes al célebre galileo, pero antes de que terminara la cena nupcial, los criados le trajeron la noticia desconcertante de que el vino se estaba acabando. Cuando la cena oficial hubo terminado y los invitados se paseaban por el jardín, la madre del novio le confió a María que la provisión de vino se había agotado. Y María le dijo en confianza: <<No se preocupe ---hablaré con mi hijo. Él nos ayudará>>. Y se atrevió a hablar así, a pesar de la reprimenda recibida pocas horas antes.

\par 
%\textsuperscript{(1529.5)}
\textsuperscript{137:4.8} Durante muchos años, María siempre se había dirigido a Jesús para que la ayudara en cada una de las crisis de su vida familiar en Nazaret, de manera que fue muy natural para ella pensar en él en este momento. Pero esta madre con aspiraciones tenía también otros motivos para acudir a su hijo mayor en esta ocasión. Jesús estaba solo en un rincón del jardín, y su madre se le acercó diciendo: <<Hijo mío, no tienen vino>>. Y Jesús contestó: <<Mi buena mujer, ¿en qué me concierne ese asunto?>> María dijo: <<Pero yo creo que ha llegado tu hora. ¿No puedes ayudarnos?>> Jesús replicó: <<Afirmo de nuevo que no he venido para actuar de esa manera. ¿Por qué me molestas otra vez con esos asuntos?>> Entonces, echándose a llorar, María le suplicó: <<Pero, hijo mío, les he prometido que nos ayudarías. ¿No querrías hacer algo por mí, por favor?>> Entonces dijo Jesús: <<Mujer, ¿quién te ha dicho que hagas ese tipo de promesas? Cuídate de no volverlo a hacer. En todas las cosas debemos servir la voluntad del Padre que está en los cielos>>\footnote{\textit{La petición de María}: Jn 2:3-4.}.

\par 
%\textsuperscript{(1530.1)}
\textsuperscript{137:4.9} María, la madre de Jesús, se sintió abatida; ¡estaba aturdida! Mientras permanecía allí inmóvil delante de él, con el rostro lleno de lágrimas, el corazón humano de Jesús se rindió de compasión por la mujer que lo había llevado en su seno. Se inclinó hacia ella, puso tiernamente la mano sobre su cabeza, y le dijo: <<Vamos, vamos, madre María, no te aflijas por mis palabras aparentemente duras. ¿No te he dicho muchas veces que he venido solamente para hacer la voluntad de mi Padre celestial? Con mucho gusto haría lo que me pides si formara parte de la voluntad del Padre..>>. Y Jesús se detuvo en seco, vacilando. María pareció percibir que algo estaba sucediendo. Dando un salto, arrojó sus brazos alrededor del cuello de Jesús, lo besó, y se precipitó hacia la sala de los criados, diciendo: <<Cualquier cosa que mi hijo os diga, hacedla>>\footnote{\textit{Instrucciones a los sirvientes}: Jn 2:5.}. Pero Jesús no dijo nada. Ahora se daba cuenta de que ya había dicho demasiado ---o más bien que había deseado demasiado con su pensamiento.

\par 
%\textsuperscript{(1530.2)}
\textsuperscript{137:4.10} María saltaba de alegría. No sabía cómo se produciría el vino, pero creía confiadamente de que por fin había persuadido a su hijo primogénito para que afirmara su autoridad, para que se atreviera a presentarse resueltamente, reclamara su posición y mostrara su poder mesiánico. A causa de la presencia y de la asociación de ciertos poderes y personalidades universales, que todos los allí presentes ignoraban por completo, ella no iba a ser defraudada. El vino que María deseaba y que Jesús, el Dios-hombre, anhelaba humanamente por simpatía, estaba en camino.

\par 
%\textsuperscript{(1530.3)}
\textsuperscript{137:4.11} Cerca de allí había seis grandes vasijas de piedra, llenas de agua, con unos ochenta litros cada una. Este agua estaba destinada a utilizarse posteriormente en las ceremonias finales de purificación de la celebración matrimonial. La agitación de los criados alrededor de estas enormes vasijas de piedra, bajo la activa dirección de su madre, atrajo la atención de Jesús. Al acercarse, observó que estaban sacando vino a cántaros llenos\footnote{\textit{Conversión del agua en vino}: Jn 2:6-8.}.

\par 
%\textsuperscript{(1530.4)}
\textsuperscript{137:4.12} Jesús se fue dando cuenta gradualmente de lo que había sucedido. De todas las personas presentes en la fiesta matrimonial de Caná, Jesús era el más sorprendido. Los otros habían esperado que efectuara un prodigio, pero eso era precisamente lo que se había propuesto no hacer. Entonces, el Hijo del Hombre recordó la advertencia que su Ajustador del Pensamiento Personalizado le había hecho en las colinas. Recordó cómo el Ajustador le había avisado que ningún poder o personalidad podía privarlo de su prerrogativa como creador de ser independiente del tiempo. En esta ocasión, los transformadores del poder, los intermedios y todas las demás personalidades que se requerían, estaban reunidos cerca del agua y de los otros elementos necesarios, y en presencia del deseo expresado por el Soberano Creador del Universo, no había manera de evitar la aparición instantánea del \textit{vino}. La producción de este incidente estaba asegurada de manera doble, pues el Ajustador Personalizado había notificado que la ejecución del deseo del Hijo no infringía de ninguna manera la voluntad del Padre.

\par 
%\textsuperscript{(1530.5)}
\textsuperscript{137:4.13} Pero esto no fue un milagro en ningún sentido. Ninguna ley de la naturaleza fue modificada, abolida o ni siquiera trascendida. Lo único que se produjo fue la anulación del \textit{tiempo} en asociación con la reunión celestial de los elementos químicos indispensables para la elaboración del vino. En Caná, en esta ocasión, los agentes del Creador hicieron el vino exactamente tal como lo hacen mediante los procesos naturales ordinarios, \textit{salvo} que lo hicieron con independencia del tiempo y con la intervención de agentes sobrehumanos para reunir en el espacio los ingredientes químicos necesarios.

\par 
%\textsuperscript{(1531.1)}
\textsuperscript{137:4.14} Además, era evidente que la realización de este pretendido milagro no era contraria a la voluntad del Padre Paradisiaco, pues de otra manera no se habría producido, ya que Jesús se había sometido en todas las cosas a la voluntad del Padre.

\par 
%\textsuperscript{(1531.2)}
\textsuperscript{137:4.15} Cuando los criados sacaron este nuevo vino y lo llevaron al padrino de boda, el <<maestro de ceremonias>>, y éste lo hubo probado, llamó al novio, diciéndole: <<Es costumbre servir primero el buen vino, y cuando los convidados han bebido bien, se trae el fruto inferior de la vid; pero tú has guardado el mejor vino para el final de la fiesta>>\footnote{\textit{Comentarios del maestresala}: Jn 2:9-10.}.

\par 
%\textsuperscript{(1531.3)}
\textsuperscript{137:4.16} María y los discípulos de Jesús se regocijaron mucho con el supuesto milagro, pensando que Jesús lo había efectuado intencionalmente, pero Jesús se retiró a un rincón abrigado del jardín y se puso a meditar seriamente durante breves momentos. Finalmente concluyó que, dadas las circunstancias, el incidente estaba más allá de su control personal, y al no ser contrario a la voluntad de su Padre, era inevitable. Cuando regresó entre los invitados, éstos lo miraron con temor; todos creían que era el Mesías. Pero Jesús estaba dolorosamente perplejo; sabía que sólo creían en él a causa del extraño suceso que accidentalmente habían contemplado\footnote{\textit{Creencias de los discípulos}: Jn 2:11.}. Jesús se retiró de nuevo durante un rato a la azotea de la casa para meditar sobre todo aquello.

\par 
%\textsuperscript{(1531.4)}
\textsuperscript{137:4.17} Jesús comprendió entonces plenamente que debía mantenerse continuamente alerta para que su inclinación a la simpatía y a la compasión no fuera responsable de otros incidentes de este tipo. Sin embargo, muchos acontecimientos similares se produjeron antes de que el Hijo del Hombre se despidiera definitivamente de su vida mortal en la carne.

\section*{5. De regreso a Cafarnaúm}
\par 
%\textsuperscript{(1531.5)}
\textsuperscript{137:5.1} Aunque muchos de los invitados se quedaron durante toda la semana de las festividades nupciales, Jesús, con sus discípulos-apóstoles recién elegidos ---Santiago, Juan, Andrés, Pedro, Felipe y Natanael--- partió a la mañana siguiente muy temprano para Cafarnaúm\footnote{\textit{Partida del grupo a Cafarnaúm}: Jn 2:12.}, marchándose sin despedirse de nadie. La familia de Jesús y todos sus amigos de Caná estaban muy apenados por su partida tan repentina, y Judá, su hermano menor, salió en su búsqueda. Jesús y sus apóstoles fueron directamente a la casa de Zebedeo en Betsaida. Durante este viaje, Jesús habló con sus asociados recién elegidos de muchas cosas importantes para el reino venidero, y les advirtió especialmente que no mencionaran la transformación del agua en vino. También les aconsejó que evitaran, en su futuro trabajo, las ciudades de Séforis y Tiberiades.

\par 
%\textsuperscript{(1531.6)}
\textsuperscript{137:5.2} Aquella noche, después de la cena, en el hogar de Zebedeo y Salomé, Jesús celebró una de las conferencias más importantes de toda su carrera terrestre. En esta reunión sólo estuvieron presentes los seis apóstoles; Judá llegó cuando estaban a punto de separarse. Estos seis hombres escogidos habían viajado con Jesús desde Caná hasta Betsaida caminando, por así decirlo, sobre las nubes. Estaban llenos de expectación y emocionados con la idea de haber sido elegidos como asociados inmediatos del Hijo del Hombre. Pero cuando Jesús empezó a decirles claramente quién era él, cuál iba a ser su misión en la Tierra y cómo podría terminar quizás, se quedaron aturdidos. No podían comprender lo que les estaba diciendo. Se quedaron sin habla; el mismo Pedro estaba más anonadado de lo que se puede expresar. Sólo Andrés, el profundo pensador, se atrevió a contestar a las recomendaciones de Jesús. Cuando Jesús percibió que no comprendían su mensaje, cuando vio que sus ideas sobre el Mesías judío estaban tan completamente cristalizadas, los envió a descansar mientras él caminaba y conversaba con su hermano Judá. Antes de despedirse de Jesús, Judá le dijo con mucha emoción: <<Mi hermano-padre, nunca te he comprendido. No sé con certidumbre si eres lo que mi madre nos ha enseñado, y no comprendo plenamente el reino venidero, pero sí sé que eres un poderoso hombre de Dios. He oído la voz en el Jordán y creo en ti, sin importarme quien seas>>. Después de hablar así, Judá se marchó para su propio hogar en Magdala.

\par 
%\textsuperscript{(1532.1)}
\textsuperscript{137:5.3} Aquella noche Jesús no durmió. Envolviéndose en sus mantas, se sentó a la orilla del lago para reflexionar, y reflexionó hasta el alba del día siguiente. Durante las largas horas de esta noche de meditación, Jesús llegó a comprender claramente que nunca conseguiría que sus discípulos lo vieran bajo otra forma que no fuera la del Mesías largo tiempo esperado. Al final reconoció que no había manera de emprender su mensaje del reino excepto como cumplimiento de la predicción de Juan, y como aquel que los judíos estaban esperando. Después de todo, aunque él no era el Mesías de tipo davídico, sí era en verdad el cumplimiento de las declaraciones proféticas de los videntes del pasado con mayores inclinaciones espirituales. Nunca más negó por completo que fuera el Mesías. La tarea de desenredar finalmente esta complicada situación decidió dejarla a la manifestación de la voluntad del Padre.

\par 
%\textsuperscript{(1532.2)}
\textsuperscript{137:5.4} A la mañana siguiente, Jesús se reunió con sus amigos en el desayuno, pero formaban un grupo melancólico. Charló con ellos y al final de la comida los reunió a su alrededor, diciendo: <<Es voluntad de mi Padre que nos quedemos por aquí durante una temporada. Habéis oído decir a Juan que había venido a preparar el camino para el reino; por lo tanto, nos conviene esperar a que Juan termine su predicación. Cuando el precursor del Hijo del Hombre haya terminado su obra, empezaremos a proclamar la buena nueva del reino>>. Ordenó a sus apóstoles que volvieran a sus redes, mientras él se preparaba para ir con Zebedeo al astillero. Les prometió que los vería al día siguiente en la sinagoga, donde iba a hablar, y los citó para reunirse con ellos aquel sábado por la tarde.

\section*{6. Los acontecimientos de un sábado}
\par 
%\textsuperscript{(1532.3)}
\textsuperscript{137:6.1} La primera aparición pública de Jesús, después de su bautismo, tuvo lugar en la sinagoga de Cafarnaúm el sábado 2 de marzo del año 26. La sinagoga estaba atestada de gente. A la historia del bautismo en el Jordán se añadían ahora las recientes noticias de Caná sobre el agua y el vino. Jesús dio asientos de honor a sus seis apóstoles, y junto a ellos estaban sentados sus hermanos carnales Santiago y Judá. Su madre había regresado con Santiago a Cafarnaúm la noche anterior, y también se hallaba presente\footnote{\textit{La madre de Jesús está presente}: Jn 2:12.}, sentada en la sección de la sinagoga destinada a las mujeres. Todo el auditorio tenía los nervios de punta; esperaban contemplar alguna manifestación extraordinaria de poder sobrenatural que fuera un testimonio apropiado de la naturaleza y la autoridad de aquel que iba a hablarles aquel día. Pero estaban destinados a sufrir una decepción.

\par 
%\textsuperscript{(1532.4)}
\textsuperscript{137:6.2} Cuando Jesús se levantó, el jefe de la sinagoga le tendió el rollo de las Escrituras, y leyó en el profeta Isaías: <<Así dice el Señor: `El cielo es mi trono, y la Tierra mi escabel. ¿Dónde está la casa que habéis construido para mí? ¿Y dónde está el lugar de mi morada? Todas estas cosas las han hecho mis manos', dice el Señor. `Pero me fijaré en el hombre que es humilde y de espíritu contrito, y que tiembla con mi palabra'. Oíd la voz del Señor, vosotros que tembláis y tenéis miedo: `Vuestros hermanos os han odiado y desechado a causa de mi nombre'. Pero el Señor sea glorificado. Él aparecerá ante vosotros con alegría y todos los demás serán avergonzados. Una voz de la ciudad, una voz del templo, una voz del Señor dice: `Antes de estar de parto, dio a luz; antes de venirle los dolores, dio a luz un hijo varón'. ¿Quién ha oído una cosa semejante? ¿Producirá la tierra en un solo día? ¿O puede una nación nacer de un golpe? Pero así dice el Señor: `He aquí que extenderé la paz como un río, e incluso la gloria de los gentiles se parecerá a un torrente que fluye. Como alguien a quien su madre consuela, así os consolaré yo. Seréis consolados incluso en Jerusalén. Y cuando veáis estas cosas, se alegrará vuestro corazón'.>>\footnote{\textit{Versículos de Isaías leídos por Jesús}: Is 66:1-2; 66:5-8; 66:12-14.}

\par 
%\textsuperscript{(1533.1)}
\textsuperscript{137:6.3} Cuando terminó esta lectura, Jesús devolvió el rollo a su guardián. Antes de sentarse, dijo simplemente: <<Sed pacientes y veréis la gloria de Dios; así es como será para todos aquellos que aguardan conmigo y aprenden así a hacer la voluntad de mi Padre que está en los cielos>>. Y la gente se fue a sus casas, preguntándose por el significado de todo esto.

\par 
%\textsuperscript{(1533.2)}
\textsuperscript{137:6.4} Aquella tarde, Jesús y sus apóstoles, con Santiago y Judá, se subieron en una barca y se alejaron un poco de la orilla, donde echaron el ancla mientras Jesús les hablaba del reino venidero. Y comprendieron más cosas de las que habían entendido la noche del jueves.

\par 
%\textsuperscript{(1533.3)}
\textsuperscript{137:6.5} Jesús les mandó que se ocuparan de sus deberes regulares hasta que <<llegue la hora del reino>>. Y para animarlos, él mismo dio ejemplo volviendo a trabajar regularmente en el astillero. Al explicarles que deberían pasar tres horas cada noche estudiando y preparándose para su trabajo futuro, Jesús añadió: \guillemotleft Todos nos quedaremos por aquí hasta que el Padre me pida que os llame. Cada uno de vosotros debe regresar ahora a su trabajo de costumbre como si nada hubiera ocurrido. No habléis a nadie de mí y recordad que mi reino no ha de venir con estruendo y fascinación, sino más bien debe venir a través del gran cambio que mi Padre habrá efectuado en vuestro corazón y en el corazón de aquellos que serán llamados para unirse a vosotros en los consejos del reino\footnote{\textit{El reino no llegará con estruendo, sino en los corazones}: Lc 17:20-21.}. Ahora sois mis amigos\footnote{\textit{Sois mis amigos}: Jn 15:14-15.}; confío en vosotros y os amo; pronto os convertiréis en mis asociados personales. Sed pacientes, sed dulces. Obedeced siempre a la voluntad del Padre. Preparaos para la llamada del reino. Aunque experimentaréis una gran alegría al servicio de mi Padre, también debéis prepararos para las dificultades, porque os advierto que muchos sólo entrarán en el reino pasando por grandes tribulaciones\footnote{\textit{El reino de la tribulación y la alegría}: Jn 16:33; Hch 14:22; Ap 7:14.}. Para aquellos que han encontrado el reino, su alegría será completa, y serán llamados los bienaventurados de toda la Tierra. Pero no alimentéis falsas esperanzas; el mundo tropezará con mis palabras. Incluso vosotros, mis amigos, no percibís plenamente lo que estoy revelando a vuestras mentes confusas. No os engañéis; saldremos a trabajar para una generación que busca signos. Exigirán la realización de prodigios como prueba de que soy el enviado de mi Padre, y serán lentos en reconocer, en la revelación del \textit{amor} de mi Padre, las cartas credenciales de mi misión\guillemotright.

\par 
%\textsuperscript{(1533.4)}
\textsuperscript{137:6.6} Aquella noche, cuando volvieron a tierra y antes de separarse, Jesús oró de pie al borde del agua: <<Padre mío, te doy las gracias por estos pequeños que ya creen, a pesar de sus dudas. Por amor a ellos, me he apartado para hacer tu voluntad. Ojalá aprendan ahora a ser uno, como nosotros somos uno>>.

\section*{7. Cuatro meses de formación}
\par 
%\textsuperscript{(1533.5)}
\textsuperscript{137:7.1} Durante cuatro largos meses ---marzo, abril, mayo y junio--- continuó este tiempo de espera; Jesús mantuvo más de cien reuniones largas y serias, aunque alegres y animadas, con estos seis asociados y su propio hermano Santiago. Debido a enfermedades en su familia, Judá rara vez pudo asistir a estas clases. Santiago no perdió la fe en su hermano Jesús, pero durante estos meses de pausa y de inacción, María casi llegó a desesperar de su hijo. Su fe, que se había elevado a tales alturas en Caná, se hundió ahora hasta niveles muy bajos. Lo único que hacía era recurrir a su exclamación tantas veces repetida: <<No consigo comprenderlo. No consigo descifrar qué significa todo esto>>. Pero la mujer de Santiago contribuyó mucho a sostener el ánimo de María.

\par 
%\textsuperscript{(1534.1)}
\textsuperscript{137:7.2} Durante estos cuatro meses, estos siete creyentes, uno de ellos su propio hermano carnal, aprendieron a conocer a Jesús; estuvieron acostumbrándose a la idea de vivir con este Dios-hombre. Aunque lo llamaban Rabino\footnote{\textit{Le llamaban Rabbí}: Jn 1:38,49; Jn 3:2,26; Jn 6:25.}, estaban aprendiendo a no temerle. Jesús poseía esa gracia incomparable en su personalidad que le permitía vivir entre ellos de tal manera que no se sentían desalentados por su divinidad. Encontraban sumamente fácil ser <<amigos de Dios>>\footnote{\textit{Amigos de Dios}: Jn 15:14-15.}, Dios encarnado en la similitud de la carne mortal. Este compás de espera fue una dura prueba para todo el grupo de creyentes. Nada milagroso sucedió, absolutamente nada. Día tras día se ponían a hacer su trabajo ordinario, y noche tras noche se sentaban a los pies de Jesús. Se mantenían unidos gracias a su personalidad sin igual y a las atractivas palabras que les dirigía noche tras noche.

\par 
%\textsuperscript{(1534.2)}
\textsuperscript{137:7.3} Este período de espera y de enseñanza fue especialmente duro para Simón Pedro. Intentó repetidas veces persuadir a Jesús para que emprendiera la predicación del reino en Galilea mientras Juan continuaba predicando en Judea. Pero Jesús siempre respondía a Pedro: <<Ten paciencia, Simón. Haz progresos. No estaremos de ningún modo demasiado preparados cuando el Padre nos llame>>. Y Andrés tranquilizaba a Pedro de vez en cuando con sus consejos más moderados y filosóficos. Andrés estaba enormemente impresionado por la naturalidad humana de Jesús. Nunca se cansaba de contemplar cómo alguien que podía vivir tan cerca de Dios, podía ser tan amistoso y considerado con los hombres.

\par 
%\textsuperscript{(1534.3)}
\textsuperscript{137:7.4} A lo largo de todo este período, Jesús no habló en la sinagoga más que dos veces. Hacia el final de estas numerosas semanas de espera, los comentarios sobre su bautismo y el vino de Caná habían empezado a calmarse. Y Jesús tuvo cuidado de que no se produjeran más milagros aparentes durante este período. Pero aunque vivían de manera tan tranquila en Betsaida, las extrañas acciones de Jesús habían sido comunicadas a Herodes Antipas, quien a su vez envió a unos espías para averiguar lo que estaba pasando. Pero Herodes estaba más preocupado por la predicación de Juan. Decidió no molestar a Jesús, cuya obra proseguía tan sosegadamente en Cafarnaúm.

\par 
%\textsuperscript{(1534.4)}
\textsuperscript{137:7.5} Durante este tiempo de espera, Jesús se esforzó por enseñar a sus asociados la actitud que debían adoptar con respecto a los diversos grupos religiosos y partidos políticos de Palestina. Jesús siempre decía: <<Tratamos de ganarlos a todos, pero no \textit{pertenecemos} a ninguno de ellos>>.

\par 
%\textsuperscript{(1534.5)}
\textsuperscript{137:7.6} A los escribas y rabinos, en conjunto, se les llamaba fariseos. Ellos se denominaban a sí mismos los <<asociados>>. Eran, en muchos aspectos, el grupo progresista entre todos los judíos, pues habían adoptado muchas enseñanzas que no figuraban claramente en las escrituras hebreas, como la creencia en la resurrección de los muertos\footnote{\textit{La resurrección de los muertos}: Dn 12:1-2.}, una doctrina que sólo había sido mencionada por Daniel, un profeta reciente.

\par 
%\textsuperscript{(1534.6)}
\textsuperscript{137:7.7} Los saduceos estaban compuestos por el clero y ciertos judíos ricos. No daban tanta importancia a los detalles de la aplicación de la ley. Los fariseos y los saduceos eran en realidad partidos religiosos en lugar de sectas.

\par 
%\textsuperscript{(1534.7)}
\textsuperscript{137:7.8} Los esenios eran una verdadera secta religiosa, que había nacido durante la revuelta de los Macabeos. En algunos aspectos, sus normas eran más exigentes que las de los fariseos. Habían adoptado muchas creencias y prácticas persas, vivían en hermandad en monasterios, practicaban el celibato y lo poseían todo en común. Se especializaban en las enseñanzas sobre los ángeles.

\par 
%\textsuperscript{(1535.1)}
\textsuperscript{137:7.9} Los celotes eran un grupo de fervientes patriotas judíos. Sostenían que todos los métodos estaban justificados en la lucha para liberarse de la esclavitud del yugo romano.

\par 
%\textsuperscript{(1535.2)}
\textsuperscript{137:7.10} Los herodianos eran un partido puramente político que abogaba por la emancipación del gobierno directo de Roma mediante la restauración de la dinastía de Herodes.

\par 
%\textsuperscript{(1535.3)}
\textsuperscript{137:7.11} En el centro mismo de Palestina vivían los samaritanos, con quienes <<los judíos no tenían relaciones>>\footnote{\textit{No tenían trato con los samaritanos}: Jn 4:9.}, a pesar de que tenían muchos puntos de vista similares con las enseñanzas judías.

\par 
%\textsuperscript{(1535.4)}
\textsuperscript{137:7.12} Todos estos partidos y sectas, incluyendo la pequeña hermandad nazarea, creían que el Mesías llegaría algún día. Todos esperaban a un libertador nacional. Pero Jesús fue muy preciso al aclarar que él y sus discípulos no se aliarían con ninguna de estas escuelas de pensamiento o de práctica. El Hijo del Hombre no debía ser ni un nazareo ni un esenio.

\par 
%\textsuperscript{(1535.5)}
\textsuperscript{137:7.13} Aunque más adelante Jesús ordenó a los apóstoles que salieran, como había hecho Juan, a predicar el evangelio e instruir a los creyentes, hizo hincapié en la proclamación de la <<buena nueva del reino de los cielos>>\footnote{\textit{Predicar buenas noticias}: Mt 3:2; Mc 1:14; Lc 8:1; Jn 3:3,5.}. Inculcó incansablemente a sus asociados que debían <<mostrar amor, compasión y simpatía>>. Desde el principio enseñó a sus seguidores que el reino de los cielos era una experiencia espiritual que tenía que ver con la entronización de Dios en el corazón de los hombres.

\par 
%\textsuperscript{(1535.6)}
\textsuperscript{137:7.14} Mientras que Jesús y los siete se demoraban así antes de lanzarse a su predicación pública activa, pasaban dos noches por semana en la sinagoga estudiando las escrituras hebreas. Años más tarde, después de intensos períodos de trabajo público, los apóstoles recordarían estos cuatro meses como los más preciosos y provechosos de toda su asociación con el Maestro. Jesús enseñó a estos hombres todo lo que podían asimilar. No cometió el error de enseñarles con exceso. No los precipitó en la confusión presentándoles una verdad que sobrepasara demasiado su capacidad de comprensión.

\section*{8. El sermón sobre el reino}
\par 
%\textsuperscript{(1535.7)}
\textsuperscript{137:8.1} El sábado 22 de junio, poco antes de partir para su primera gira de predicación, y unos diez días después del arresto de Juan, Jesús ocupó el púlpito de la sinagoga por segunda vez desde que trajo a sus apóstoles a Cafarnaúm.

\par 
%\textsuperscript{(1535.8)}
\textsuperscript{137:8.2} Unos días antes de predicar este sermón sobre <<el Reino>>\footnote{\textit{Evangelio del reino}: Mt 4:23; 9:35; 24:14; Mc 1:14-15.}, mientras Jesús trabajaba en el astillero, Pedro le trajo la noticia del arresto de Juan\footnote{\textit{Jesús oye que Juan está en prisión}: Mt 4:12,17; Mc 1:14-15.}. Jesús dejó sus herramientas una vez más, se quitó el delantal y le dijo a Pedro: <<La hora del Padre ha llegado. Preparémonos para proclamar el evangelio del reino>>.

\par 
%\textsuperscript{(1535.9)}
\textsuperscript{137:8.3} Este martes 18 de junio del año 26 fue el último día que Jesús trabajó en un banco de carpintería. Pedro se precipitó fuera del taller, y hacia media tarde había reunido a todos sus compañeros; los dejó en un bosquecillo cercano a la costa, y fue en busca de Jesús. Pero no pudo encontrarlo, porque el Maestro había ido a otro bosquecillo para orar. No lo vieron hasta una hora avanzada de aquella noche, cuando regresó a la casa de Zebedeo y pidió de comer. Al día siguiente, envió a su hermano Santiago para que solicitara el privilegio de hablar en la sinagoga el sábado siguiente. El jefe de la sinagoga se alegró mucho de que Jesús estuviera dispuesto de nuevo a dirigir los oficios.

\par 
%\textsuperscript{(1536.1)}
\textsuperscript{137:8.4} Antes de que Jesús predicara este memorable sermón sobre el reino de Dios, el primer esfuerzo con pretensiones de su carrera pública, leyó en las Escrituras los pasajes siguientes: <<Seréis para mí un reino de sacerdotes, un pueblo santo. Yahvé es nuestro juez, Yahvé es nuestro legislador, Yahvé es nuestro rey; él nos salvará. Yahvé es mi rey y mi Dios. Él es un gran rey sobre toda la Tierra. La misericordia está sobre Israel en este reino. Bendita sea la gloria del Señor, porque él es nuestro Rey>>\footnote{\textit{Un pueblo santo}: Ex 19:6. \textit{Yahvé en nuestro juez}: Is 33:22. \textit{Yahvé es nuestro rey y Dios}: Sal 84:3. \textit{Él es un gran rey}: Sal 47:2. \textit{La misericordia}: Sal 138:2. \textit{Bendita sea la gloria del Señor}: Ez 3:12. \textit{Porque él es nuestro Rey}: Sal 89:18.}.

\par 
%\textsuperscript{(1536.2)}
\textsuperscript{137:8.5} Cuando terminó de leer, Jesús dijo:

\par 
%\textsuperscript{(1536.3)}
\textsuperscript{137:8.6} <<He venido para proclamar el establecimiento del reino del Padre. Este reino incluirá a las almas adoradoras de los judíos y de los gentiles, de los ricos y de los pobres, de los hombres libres y de los esclavos, porque mi Padre no hace acepción de personas; su amor y su misericordia son para todos>>\footnote{\textit{Proclamar el establecimiento del reino del Padre}: Mt 3:2; 4:17,23; 5:3,10,19-20; 6:33; 7:21; 8:11; 9:35; 10:7; 11:11-12; 12:28; 13:11,14,31-52; 16:19; 18:1-4,23; 19:14,23-24; 20:1; 21:31,43; 22:2; 23:13; 24:14; 25:1,14; Mc 1:14-15; 4:11,26,30; 9:1,47; 10:14-15,23-25; 12:34; 14:25; 15:43; Lc 4:43; 6:20; 7:28; 8:1,10; 9:2,11,27; 9:60,62; 10:9-11; 11:20; 12:31-32; 13:18,20,28,29; 14:15; 16:16; 17:20-21; 18:16-17,24-25; 19:11; 21:31; 22:16,18; 23:51; Jn 3:3,5; Ro 14:17; 1 Co 4:20; 6:9-10. \textit{No hace acepción de personas}: 2 Cr 19:7; Job 34:19; Eclo 35:12; Hch 10:34; Ro 2:11; Gl 2:6; 3:28; Ef 6:9; Col 3:11. \textit{Su amor y su misericordia son para todos}: Ef 2:4. \textit{Reino universal}: 1 Co 12:33.}.

\par 
%\textsuperscript{(1536.4)}
\textsuperscript{137:8.7} <<El Padre que está en los cielos envía su espíritu para que habite en la mente de los hombres, y cuando yo haya terminado mi obra en la Tierra, el Espíritu de la Verdad será igualmente derramado sobre todo el género humano. El espíritu de mi Padre y el Espíritu de la Verdad os establecerán en el reino venidero de comprensión espiritual y de rectitud divina. Mi reino no es de este mundo. El Hijo del Hombre no conducirá los ejércitos a la batalla para establecer un trono de poder o un reino de gloria terrenal. Cuando llegue mi reino, conoceréis al Hijo del Hombre como el Príncipe de la Paz, como la revelación del Padre eterno. Los hijos de este mundo luchan por establecer y ampliar los reinos de este mundo, pero mis discípulos entrarán en el reino de los cielos por medio de sus decisiones morales y de sus victorias espirituales; y una vez que hayan entrado, encontrarán la alegría, la rectitud y la vida eterna>>\footnote{\textit{El espíritu de mi Padre y el Espíritu de la Verdad}: Jn 17:21-23. \textit{Mi reino no es de este mundo}: Jn 18:36. \textit{Espíritu de la Verdad}: Ez 11:19; 18:31; 36:26-27; Jl 2:28-29; Lc 24:49; Jn 7:39; 14:16-18,23,26; 15:4,26; 16:6-8,13-14; 17:21-23; Hch 1:5,8a; 2:1-4,16-18; 2:33; 2 Co 13:5; Gl 2:20; 4:6; Ef 1:13; 4:30; 1 Jn 4:12-15.}.

\par 
%\textsuperscript{(1536.5)}
\textsuperscript{137:8.8} <<Aquellos que intentan en primer lugar entrar en el reino, y empiezan así a esforzarse por conseguir una nobleza de carácter semejante a la de mi Padre, pronto poseerán todas las demás cosas que necesitan. Pero os lo digo con toda sinceridad: a menos que tratéis de entrar en el reino con la fe y la dependencia confiada de un niño pequeño, no seréis admitidos de ninguna manera>>\footnote{\textit{Buscar primero el reino}: Mt 6:33; Lc 12:31. \textit{La fe confiada de un niño}: Mt 18:2-4; 19:13-14; Mc 9:36-37; 10:13-15; Lc 9:47-48; 18:17.}.

\par 
%\textsuperscript{(1536.6)}
\textsuperscript{137:8.9} <<No os dejéis engañar por aquellos que vienen diciendo: el reino está aquí o el reino está allá, porque el reino de mi Padre no tiene nada que ver con las cosas visibles y materiales. Este reino ya se encuentra ahora entre vosotros, porque allí donde el espíritu de Dios enseña y dirige el alma del hombre, allí está en realidad el reino de los cielos. Y este reino de Dios es rectitud, paz y alegría en el Espíritu Santo>>\footnote{\textit{No os dejéis engañar}: Mt 24:23; Mc 13:21; Lc 17:23; Lc 21:8. \textit{Reino de Dios}: Ro 14:17.}.

\par 
%\textsuperscript{(1536.7)}
\textsuperscript{137:8.10} <<Juan os ha bautizado verdaderamente en señal de arrepentimiento y para la remisión de vuestros pecados, pero cuando entréis en el reino celestial, seréis bautizados con el Espíritu Santo>>\footnote{\textit{Bautismo de Juan en señal de arrepentimiento}: Mt 3:11; Lc 3:16; Hch 1:5; Hch 11:16.}.

\par 
%\textsuperscript{(1536.8)}
\textsuperscript{137:8.11} <<En el reino de mi Padre no habrá ni judíos ni gentiles, sino únicamente aquellos que buscan la perfección a través del servicio, porque declaro que aquel que quiera ser grande en el reino de mi Padre, deberá convertirse primero en el servidor de todos. Si estáis dispuestos a servir a vuestros semejantes, os sentaréis conmigo en mi reino, al igual que yo me sentaré dentro de poco con mi Padre en su reino por haber servido en la similitud de la criatura>>\footnote{\textit{No habrá judíos ni gentiles}: 2 Cr 19:7; Job 34:19; Eclo 35:12; Hch 10:34; Ro 2:9-11; 9:24; 10:12; Gl 2:6; 3:28; Ef 6:9; Col 3:11. \textit{El grande es el servidor de todos}: Mt 20:26-27; 23:11-12; Mc 9:35; 10:43-44; Lc 22:26.}.

\par 
%\textsuperscript{(1536.9)}
\textsuperscript{137:8.12} <<Este nuevo reino es igual a una semilla que crece en la tierra fértil de un campo. No alcanza rápidamente su plena fructificación. Hay un intervalo de tiempo entre el establecimiento del reino en el alma del hombre y el momento en que el reino madura hasta su plena fructificación de rectitud perpetua y de salvación eterna>>\footnote{\textit{El reino como una semilla}: Mt 13:8,23; Mc 4:8,20; Lc 8:8,15.}.

\par 
%\textsuperscript{(1536.10)}
\textsuperscript{137:8.13} <<Este reino que os proclamo no es un reinado de poder y de abundancia. El reino de los cielos no es un asunto de comida y de bebida, sino más bien una vida de rectitud progresiva y de alegría creciente en el servicio cada vez más perfecto de mi Padre que está en los cielos. Porque ¿no ha dicho el Padre refiriéndose a sus hijos del mundo: `es mi voluntad que sean finalmente perfectos, como yo soy perfecto'?>>\footnote{\textit{El reino es una vida de servicio}: Ro 14:17. \textit{Sed perfectos}: Gn 17:1; 1 Re 8:61; Lv 19:2; Dt 18:13; Mt 5:48; 2 Co 13:11; Stg 1:4; 1 P 1:16.}

\par 
%\textsuperscript{(1537.1)}
\textsuperscript{137:8.14} <<He venido a predicar la buena nueva del reino. No he venido a aumentar las cargas pesadas de los que quieran entrar en este reino. Proclamo un camino nuevo y mejor, y aquellos que sean capaces de entrar en el reino venidero disfrutarán del descanso divino. Todo lo que os cueste en cosas del mundo, cualquier precio que paguéis por entrar en el reino de los cielos, lo recibiréis multiplicado en alegría y en progreso espiritual en este mundo, y la vida eterna en la era por venir>>\footnote{\textit{Jesús predicó buenas nuevas}: Lc 8:1. \textit{El reino merece el coste}: Mt 19:29.}.

\par 
%\textsuperscript{(1537.2)}
\textsuperscript{137:8.15} <<La entrada en el reino del Padre no depende de los ejércitos en marcha, de los reinos derrocados de este mundo, ni de la ruptura del yugo de los cautivos. El reino de los cielos está cerca, y todos los que entren en él encontrarán una libertad abundante y una gozosa salvación>>\footnote{\textit{El reino está cerca}: Mt 4:17; 10:7; Mc 1:15; Lc 21:31.}.

\par 
%\textsuperscript{(1537.3)}
\textsuperscript{137:8.16} <<Este reino es un dominio perpetuo. Los que entren en el reino ascenderán hasta mi Padre; alcanzarán ciertamente la diestra de su gloria en el Paraíso. Todos los que entren en el reino de los cielos se convertirán en los hijos de Dios, y en la era venidera ascenderán hasta el Padre. No he venido a llamar a los supuestos justos, sino a los pecadores y a todos los que tienen hambre y sed de la rectitud de la perfección divina>>\footnote{\textit{El reino es un dominio perpetuo}: 2 P 1:11. \textit{Alcanzarán la diestra de su gloria}: Mt 25:33-34; Mc 10:37,40; 16:19. \textit{Jesús vino a llamar a los pecadores}: Mt 9:13; Mc 2:17; Lc 5:32.}.

\par 
%\textsuperscript{(1537.4)}
\textsuperscript{137:8.17} <<Juan ha venido a predicar el arrepentimiento para prepararos para el reino; ahora vengo yo para proclamar que la fe, el regalo de Dios, es el precio para entrar en el reino de los cielos. Con que sólo creáis que mi Padre os ama con un amor infinito, ya estáis en el reino de Dios>>\footnote{\textit{La fe es el precio para entrar}: Hab 2:4; Ro 1:17; Gl 3:11; Ef 2:8; 1 P 1:9.}.

\par 
%\textsuperscript{(1537.5)}
\textsuperscript{137:8.18} Cuando terminó de hablar así, Jesús se sentó. Todos los que le oyeron se quedaron asombrados con sus palabras. Sus discípulos se maravillaron. Pero la gente no estaba preparada para recibir la buena nueva de labios de este Dios-hombre. Aproximadamente un tercio de sus oyentes creyó en el mensaje, aunque no pudieran comprenderlo por completo; otro tercio aproximadamente se preparó en su fuero interno para rechazar este concepto puramente espiritual del reino esperado, mientras que el tercio restante no pudo captar su enseñanza, y muchos de éstos creyeron sinceramente que <<había perdido el juicio>>\footnote{\textit{Muchos creían que Jesús estaba loco}: Mc 3:21.}.