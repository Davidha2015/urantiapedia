\chapter{Documento 141. El comienzo de la obra pública}
\par 
%\textsuperscript{(1587.1)}
\textsuperscript{141:0.1} EL 19 de enero del año 27, primer día de la semana, Jesús y los doce apóstoles se prepararon para marcharse de su cuartel general de Betsaida. Los doce no sabían nada de los planes de su Maestro, excepto que subirían a Jerusalén para asistir a la fiesta de la Pascua en abril, y que se tenía la intención de viajar por el camino del valle del Jordán. No salieron de la casa de Zebedeo hasta cerca del mediodía, porque las familias de los apóstoles y de otros discípulos habían venido para despedirlos y desearles buena suerte en la nueva tarea que estaban a punto de empezar.

\par 
%\textsuperscript{(1587.2)}
\textsuperscript{141:0.2} Poco antes de partir, los apóstoles no vieron al Maestro, y Andrés salió a buscarlo. No tardó en encontrarlo sentado en una barca en la playa, y Jesús estaba llorando. Los doce habían visto a menudo a su Maestro cuando parecía apesadumbrado, y habían contemplado sus breves períodos de graves preocupaciones mentales, pero ninguno de ellos lo había visto nunca llorar. Andrés se quedó un poco sorprendido al ver al Maestro así de afectado en vísperas de su partida hacia Jerusalén, y se atrevió a acercarse a Jesús para preguntarle: <<En este gran día, Maestro, cuando estamos a punto de partir hacia Jerusalén para proclamar el reino del Padre, ¿por qué lloras? ¿Quién de nosotros te ha ofendido?>> Y Jesús, regresando con Andrés para reunirse con los doce, le respondió: <<Ninguno de vosotros me ha causado pena. Estoy triste solamente porque ningún miembro de la familia de mi padre José se ha acordado de venir para desearnos buena suerte>>. En aquel momento, Rut estaba de visita en casa de su hermano José, en Nazaret. Otros miembros de su familia se mantenían alejados por orgullo, decepción, incomprensión y pequeños resentimientos a los que habían cedido porque sus sentimientos habían sido heridos.

\section*{1. La salida de Galilea}
\par 
%\textsuperscript{(1587.3)}
\textsuperscript{141:1.1} Cafarnaúm no estaba lejos de Tiberiades, y la fama de Jesús había empezado a propagarse ampliamente por toda Galilea, e incluso más allá. Jesús sabía que Herodes empezaría pronto a prestar atención a su obra; por eso pensó que sería mejor viajar hacia el sur y entrar en Judea con sus apóstoles. Un grupo de más de cien creyentes deseaba ir con ellos, pero Jesús les habló y les rogó que no acompañaran al grupo apostólico en su descenso por el Jordán. Aunque consintieron en quedarse atrás, muchos de ellos siguieron al Maestro pocos días después.

\par 
%\textsuperscript{(1587.4)}
\textsuperscript{141:1.2} El primer día, Jesús y los apóstoles sólo llegaron hasta Tariquea, donde descansaron durante la noche. Al día siguiente viajaron hasta un punto del Jordán, cerca de Pella, donde Juan había predicado aproximadamente un año antes, y donde Jesús había recibido el bautismo. Se detuvieron allí durante más de dos semanas, enseñando y predicando. Hacia el final de la primera semana, varios cientos de personas se habían reunido en un campamento, cerca del lugar donde residían Jesús y los doce; habían venido de Galilea, Fenicia, Siria, la Decápolis, Perea y Judea.

\par 
%\textsuperscript{(1588.1)}
\textsuperscript{141:1.3} Jesús no efectuó ninguna predicación pública. Andrés dividía la multitud y designaba los predicadores para las asambleas de la mañana y de la tarde. Después de la cena, Jesús conversaba con los doce. No les enseñaba nada nuevo, pero repasaba su enseñanza anterior y contestaba a sus numerosas preguntas. Durante una de aquellas noches, contó a los doce algunas cosas sobre los cuarenta días que había pasado en las colinas, cerca de este lugar.

\par 
%\textsuperscript{(1588.2)}
\textsuperscript{141:1.4} Muchas de las personas que venían de Perea y de Judea habían sido bautizadas por Juan y estaban interesadas en saber más cosas sobre las enseñanzas de Jesús. Los apóstoles hicieron muchos progresos enseñando a los discípulos de Juan, ya que no desacreditaban de ninguna manera la predicación de Juan, y además, en aquella época ni siquiera bautizaban a sus nuevos discípulos. Pero siempre fue un escollo para los seguidores de Juan el ver que Jesús, si era todo lo que Juan había anunciado, no hacía nada por sacarlo de la cárcel. Los discípulos de Juan nunca pudieron comprender por qué Jesús no impidió la muerte cruel de su amado jefe.

\par 
%\textsuperscript{(1588.3)}
\textsuperscript{141:1.5} Noche tras noche, Andrés enseñaba cuidadosamente a sus compañeros apóstoles la tarea delicada y difícil de llevarse bien con los seguidores de Juan el Bautista. Durante este primer año del ministerio público de Jesús, más de las tres cuartas partes de sus discípulos habían seguido previamente a Juan y habían recibido su bautismo. Todo este año 27 lo pasaron haciéndose cargo tranquilamente de la obra de Juan en Perea y Judea.

\section*{2. La ley de Dios y la voluntad del Padre}
\par 
%\textsuperscript{(1588.4)}
\textsuperscript{141:2.1} La noche antes de partir de Pella, Jesús dio a los apóstoles algunas enseñanzas adicionales sobre el nuevo reino. El Maestro dijo: <<Se os ha enseñado a esperar la venida del reino de Dios, y ahora vengo para anunciar que este reino tanto tiempo esperado está cerca, que incluso ya está aquí, en medio de nosotros. En todo reino ha de haber un rey sentado en su trono, decretando las leyes del reino. Por eso habéis desarrollado un concepto del reino de los cielos consistente en el gobierno glorificado del pueblo judío sobre todos los pueblos de la Tierra, con el Mesías sentado en el trono de David, promulgando, desde ese lugar de poder milagroso, las leyes del mundo entero. Pero, hijos míos, no veis con los ojos de la fe, y no oís con el entendimiento del espíritu. Declaro que el reino de los cielos es la comprensión y el reconocimiento del gobierno de Dios en el corazón de los hombres. Es verdad que hay un Rey en este reino, y ese Rey es mi Padre y vuestro Padre. Somos en verdad sus súbditos leales, pero mucho más allá de este hecho se encuentra la verdad transformadora de que somos sus \textit{hijos}. En mi vida, esta verdad ha de volverse manifiesta para todos. Nuestro Padre también está sentado en un trono, pero ninguna mano lo ha hecho. El trono del Infinito es la residencia eterna del Padre en el cielo de los cielos; él llena todas las cosas y proclama sus leyes a unos universos tras otros. Y el Padre reina también en el corazón de sus hijos de la Tierra por medio del espíritu que ha enviado a vivir dentro del alma de los hombres mortales>>\footnote{\textit{El reino de Dios ya está aquí}: Mt 3:2; 4:17,23; 6:33; 9:35; 10:7; 24:14; Mc 1:14-15; Lc 4:43; 10:9,11; 17:21; 21:29-32; Jn 3:3,5. \textit{El trono del rey David}: Lc 1:32. \textit{El espíritu de Dios dentro de nosotros}: Job 32:8,18; Is 63:10-11; Ez 37:14; Mt 10:20; Lc 17:20-21; Jn 17:21-23; Ro 8:9-11; 1 Co 3:16-17; 6:19; 2 Co 6:16; Gl 2:20; 1 Jn 3:24; 4:12-15; Ap 21:3.}.

\par 
%\textsuperscript{(1588.5)}
\textsuperscript{141:2.2} <<Cuando sois los súbditos de este reino, debéis oír en verdad la ley del Soberano Universal; pero cuando, a causa del evangelio del reino que he venido a proclamar, descubrís por la fe que sois hijos, ya no seguís considerándoos como criaturas sujetas a la ley de un rey todopoderoso, sino como los hijos privilegiados de un Padre amoroso y divino. En verdad, en verdad os digo que cuando la voluntad del Padre es vuestra \textit{ley}, difícilmente estáis en el reino. Pero cuando la voluntad del Padre se convierte realmente en vuestra \textit{voluntad}, entonces estáis de verdad en el reino, porque el reino se ha vuelto así una experiencia establecida en vosotros. Cuando la voluntad de Dios es vuestra ley, sois unos nobles súbditos esclavos; pero cuando creéis en este nuevo evangelio de filiación divina, la voluntad de mi Padre se convierte en vuestra voluntad, y sois elevados a la alta posición de los hijos libres de Dios, los hijos liberados del reino>>\footnote{\textit{Evangelio del reino}: Mt 4:23; 9:35; 24:14; Mc 1:14-15. \textit{Hijos de Dios por la fe}: 1 Cr 22:10; Sal 2:7; Is 56:5; Mt 5:9,16,45; Lc 20:36; Jn 1:12-13; 11:52; Hch 17:28-29; Ro 8:14-17,19,21; 9:26; 2 Co 6:18; Gl 3:26; 4:5-7; Ef 1:5; Flp 2:15; Heb 12:5-8; 1 Jn 3:1-2,10; 5:2; Ap 21:7; 2 Sam 7:14. \textit{Cuando la voluntad del Padre se convierta en vuestra voluntad}: Sal 143:10; Eclo 15:11-20; Mt 6:10; 7:21; 12:50; Mc 3:35; Lc 8:21; 11:2; Jn 7:16-17; 9:31; 14:21-24; 15:10,14-16.}.

\par 
%\textsuperscript{(1589.1)}
\textsuperscript{141:2.3} Algunos apóstoles captaron algo de esta enseñanza, pero ninguno de ellos comprendió el significado completo de esta formidable declaración, a excepción quizás de Santiago Zebedeo. Sin embargo, estas palabras se grabaron en su corazón y emergieron para alegrar su ministerio durante los años posteriores de servicio.

\section*{3. La estancia en Amatus}
\par 
%\textsuperscript{(1589.2)}
\textsuperscript{141:3.1} El Maestro y sus apóstoles permanecieron cerca de Amatus casi tres semanas. Los apóstoles continuaron predicando a la multitud dos veces al día, y Jesús predicó todos los sábados por la tarde. Resultó imposible continuar con el recreo de los miércoles; por eso, Andrés decidió que dos apóstoles descansarían cada día durante seis días de la semana, y que todos estarían de servicio durante los oficios del sábado.

\par 
%\textsuperscript{(1589.3)}
\textsuperscript{141:3.2} Pedro, Santiago y Juan hicieron la mayor parte de la predicación pública. Felipe, Natanael, Tomás y Simón hicieron una gran parte del trabajo personal y dirigieron clases para grupos especiales de investigadores; los gemelos continuaron con su supervisión general de vigilancia, mientras que Andrés, Mateo y Judas se organizaron en un comité de administración general de tres miembros, aunque cada uno de ellos también realizó un considerable trabajo religioso.

\par 
%\textsuperscript{(1589.4)}
\textsuperscript{141:3.3} Andrés estaba muy ocupado con la tarea de arreglar los malentendidos y desacuerdos que se repetían continuamente entre los discípulos de Juan y los discípulos más recientes de Jesús. Cada pocos días se producían situaciones graves, pero Andrés, con la ayuda de sus colegas apostólicos, se las ingeniaba para persuadir a las partes en conflicto para que llegaran a algún tipo de acuerdo, aunque fuera temporal. Jesús rehusó participar en ninguna de estas conferencias; tampoco quiso dar ningún consejo sobre la manera de arreglar adecuadamente estas dificultades. Ni una sola vez ofreció sugerencias a los apóstoles sobre cómo resolver estos confusos problemas. Cuando Andrés se presentaba con estas cuestiones, Jesús siempre le decía: <<No es prudente que el anfitrión participe en las querellas familiares de sus huéspedes; un padre sabio nunca toma partido en las desavenencias menores de sus propios hijos>>.

\par 
%\textsuperscript{(1589.5)}
\textsuperscript{141:3.4} El Maestro mostraba una gran sabiduría y manifestaba una equidad perfecta en todas sus relaciones con sus apóstoles y con todos sus discípulos. Jesús era realmente un maestro de hombres; ejercía una gran influencia sobre sus semejantes a causa de la fuerza y el encanto combinados de su personalidad. Su vida ruda, nómada y sin hogar producía una sutil influencia dominante. Había un atractivo intelectual y un poder persuasivo espiritual en su manera de enseñar llena de autoridad, en su lógica lúcida, en la fuerza de su razonamiento, en su perspicacia sagaz, en su viveza mental, en su serenidad incomparable y en su sublime tolerancia. Era sencillo, varonil, honrado e intrépido. Junto a toda esta influencia física e intelectual que manifestaba la presencia del Maestro, también se encontraban todos los encantos espirituales del ser que se habían asociado con su personalidad ---la paciencia, la ternura, la mansedumbre, la dulzura y la humildad.

\par 
%\textsuperscript{(1589.6)}
\textsuperscript{141:3.5} Jesús de Nazaret era en verdad una personalidad fuerte y enérgica; era una potencia intelectual y una fortaleza espiritual. Su personalidad no atraía solamente, entre sus discípulos, a las mujeres propensas a la espiritualidad, sino también al culto e intelectual Nicodemo y al endurecido soldado romano, el capitán que estaba de guardia en la cruz, que después de ver morir al Maestro, dijo: <<En verdad, era un Hijo de Dios>>\footnote{\textit{Evangelio del Hijo divino de Dios}: Mt 8:29; 14:33; 16:15-16; 27:54; Mc 1:1; 3:11; 15:39; Lc 1:35; 4:41; Jn 1:34,49; 3:16-18; 10:36; 20:31; Hch 8:37.}. Y los enérgicos y robustos pescadores galileos le llamaban Maestro\footnote{\textit{Le llamaban ``Maestro''}: Mc 9:38; 13:1; Lc 5:5; Jn 4:31.}.

\par 
%\textsuperscript{(1590.1)}
\textsuperscript{141:3.6} Los retratos de Jesús han sido muy desacertados. Esas pinturas de Cristo han ejercido una influencia perjudicial sobre la juventud; los mercaderes del templo difícilmente hubieran huido delante de Jesús si éste hubiera sido el tipo de hombre que vuestros artistas han representado generalmente. Su masculinidad estaba llena de dignidad; era bueno, pero natural. Jesús no tenía la actitud de un místico apacible, dulce, suave y amable. Su enseñanza era conmovedoramente dinámica. No solamente tenía \textit{buenas intenciones}, sino que iba de un sitio para otro \textit{haciendo} realmente \textit{el bien}\footnote{\textit{Jesús pasó haciendo el bien}: Hch 10:38.}.

\par 
%\textsuperscript{(1590.2)}
\textsuperscript{141:3.7} El Maestro nunca dijo: <<Venid a mí todos los que sois indolentes y todos los soñadores>>. Pero sí dijo muchas veces: <<Venid a mí todos los que os \textit{esforzáis}, y yo os daré descanso ---fuerza espiritual>>\footnote{\textit{Venid y os daré descanso}: Mt 11:28.}. En verdad, el yugo del Maestro es ligero\footnote{\textit{El yugo es ligero}: Mt 11:29-30.}, pero incluso así, nunca lo impone; cada persona debe coger ese yugo por su propia voluntad.

\par 
%\textsuperscript{(1590.3)}
\textsuperscript{141:3.8} Jesús describió la conquista como fruto del sacrificio, el sacrificio del orgullo y del egoísmo. Al mostrar misericordia, pretendía ilustrar la liberación espiritual de todos los rencores, agravios, ira y ansias de poder y de venganza egoístas. Cuando dijo: <<No resistáis al mal>>\footnote{\textit{No resistáis al mal}: Mt 5:39-42; Lc 6:28-31.}, explicó más adelante que no quería decir que excusara el pecado o que aconsejara fraternizar con la iniquidad. Intentaba más bien enseñar a perdonar, a <<no resistirse a los malos tratos contra nuestra personalidad, al perjuicio dañino contra nuestros sentimientos de dignidad personal>>.

\section*{4. La enseñanza sobre el Padre}
\par 
%\textsuperscript{(1590.4)}
\textsuperscript{141:4.1} Durante su estancia en Amatus, Jesús pasó mucho tiempo enseñando a los apóstoles el nuevo concepto de Dios; les inculcó una y otra vez que \textit{Dios es un Padre}, y no un contable grande y supremo que se ocupa principalmente de efectuar asientos perjudiciales contra sus hijos desviados de la Tierra, registrando sus pecados y maldades para luego utilizarlos contra ellos cuando se siente a juzgarlos como justo Juez de toda la creación. Desde hacía mucho tiempo, los judíos habían concebido a Dios como un rey por encima de todo\footnote{\textit{Dios como rey por encima de todo}: Sal 22:27; Is 2:2-5; Dn 2:44; Zac 14:7; Mal 1:14.}, e incluso como Padre de la nación\footnote{\textit{Dios como padre de la nación judía}: Is 63:16; 64:8.}, pero nunca antes un gran número de hombres mortales había mantenido la idea de Dios como Padre amoroso del \textit{individuo}.

\par 
%\textsuperscript{(1590.5)}
\textsuperscript{141:4.2} En respuesta a la pregunta de Tomás: <<¿Quién es este Dios del reino?>>, Jesús replicó: <<Dios es \textit{tu} Padre, y la religión ---mi evangelio--- no es ni más ni menos que reconocer la verdad, creyéndolo, de que tú eres su hijo. Y yo estoy aquí, viviendo en la carne entre vosotros, para clarificar estas dos ideas con mi vida y mis enseñanzas>>footnote{\textit{Dios es nuestro Padre}: 1 Cr 22:10; Sal 2:7; 89:26-27; Jer 3:19; Mt 5:16,45,48; 6:1,9,14; 6:26:32; 7:11; 10:32-33; 18:14; 23:9; Mc 11:25-26; Lc 6:36; 11:2,13; Jn 20:17b; Ro 1:7; 8:14-15; 1 Co 1:3; 2 Co 1:2; 6:18; Gl 1:4; 4:6-7; Ef 1:2; Flp 1:2; Col 1:2; 1 Ts 1:1,3; 2 Ts 1:1-2; 1 Ti 1:2; Flm 1:2; 2 Sam 7:14. \textit{Somos los hijos de Dios}: 1 Cr 22:10; Sal 2:7; Is 56:5; Mt 5:9,16,45; Lc 20:36; Jn 1:12-13; 11:52; Hch 17:28-29; Ro 8:14-17,19,21; 9:26; 2 Co 6:18; Gl 3:26; 4:5-7; Ef 1:5; Flp 2:15; Heb 12:5-8; 1 Jn 3:1-2,10; 5:2; Ap 21:7; 2 Sam 7:14.}.

\par 
%\textsuperscript{(1590.6)}
\textsuperscript{141:4.3} Jesús también intentó liberar la mente de sus apóstoles de la idea de que ofrecer sacrificios de animales era un deber religioso. Pero estos hombres, educados en la religión del sacrificio diario, eran lentos en comprender lo que les quería decir. Sin embargo, el Maestro no se cansó de enseñarles. Cuando no conseguía llegar a la mente de todos los apóstoles mediante un solo ejemplo, volvía a repetir su mensaje empleando otro tipo de parábola con objeto de iluminarlos.

\par 
%\textsuperscript{(1590.7)}
\textsuperscript{141:4.4} Por esta misma época, Jesús empezó a enseñar más plenamente a los doce sobre su misión\footnote{\textit{La misión de los doce}: Mt 10:1,8; Lc 9:2; 10:9.} de <<consolar a los afligidos y de cuidar a los enfermos>>. El Maestro les enseñó muchas cosas sobre el hombre completo ---la unión del cuerpo, la mente y el espíritu para formar el individuo, hombre o mujer. Jesús expuso a sus asociados los tres tipos de aflicción que iban a encontrar, y luego les explicó cómo deberían ayudar a todos los que sufren los dolores de las enfermedades humanas. Les enseñó a reconocer:

\par 
%\textsuperscript{(1591.1)}
\textsuperscript{141:4.5} 1. Las enfermedades de la carne ---las aflicciones generalmente consideradas como enfermedades físicas.

\par 
%\textsuperscript{(1591.2)}
\textsuperscript{141:4.6} 2. Las mentes perturbadas ---las aflicciones no físicas, posteriormente consideradas como dificultades y desórdenes emocionales y mentales.

\par 
%\textsuperscript{(1591.3)}
\textsuperscript{141:4.7} 3. La posesión por los malos espíritus.

\par 
%\textsuperscript{(1591.4)}
\textsuperscript{141:4.8} En diversas ocasiones, Jesús explicó a sus apóstoles la naturaleza de estos malos espíritus, y les dijo algunas cosas sobre su origen; en aquella época también se les llamaba a menudo espíritus impuros. El Maestro conocía bien la diferencia entre la posesión por los malos espíritus y la demencia, pero los apóstoles lo ignoraban. En vista de su conocimiento limitado de la historia primitiva de Urantia, Jesús tampoco podía emprender la tarea de hacerles comprender plenamente esta cuestión. Pero les dijo muchas veces, aludiendo a estos malos espíritus: <<No volverán a molestar a los hombres cuando yo haya ascendido hasta mi Padre que está en los cielos, y después de que haya derramado mi espíritu sobre todo el género humano, en la época en que el reino vendrá con gran poder y gloria espiritual>>\footnote{\textit{Jesús derramando su espíritu}: Ez 11:19; 18:31; 36:26-27; Jl 2:28-29; Lc 24:49; Jn 7:39; 14:16-18,23,26; 15:4,26; 16:7-8,13-14; 17:21-23; Hch 1:5,8a; 2:1-4,16-18; 2:33; 2 Co 13:5; Gl 2:20; 4:6; Ef 1:13; 4:30; 1 Jn 4:12-15. \textit{Llegada del reino en poder y en gloria}: Mt 24:30-31; Mc 13:26; Lc 21:27; Hch 1:7-8.}.

\par 
%\textsuperscript{(1591.5)}
\textsuperscript{141:4.9} Semana tras semana y un mes tras otro, a lo largo de todo este año, los apóstoles prestaron cada vez más atención a la tarea de curar a los enfermos.

\section*{5. La unidad espiritual}
\par 
%\textsuperscript{(1591.6)}
\textsuperscript{141:5.1} Una de las conferencias nocturnas más extraordinarias de Amatus fue la sesión en la que se discutió sobre la unidad espiritual. Santiago Zebedeo había preguntado: <<Maestro, ¿cómo podemos aprender a tener el mismo punto de vista, y a disfrutar así de una mayor armonía entre nosotros?>>\footnote{\textit{Armonía apostólica}: Zac 3:10; Jn 17:21; Ef 4:13-16.} Cuando Jesús escuchó esta pregunta, su espíritu se alteró de tal manera que replicó: <<Santiago, Santiago, ¿cuándo te he enseñado que todos debéis tener el mismo punto de vista? He venido al mundo para proclamar la libertad espiritual, con el fin de que los mortales puedan tener la facultad de vivir una vida individual original y libre ante Dios. No deseo que la armonía social y la paz fraternal se adquieran a costa del sacrificio de la personalidad libre y de la originalidad espiritual. Lo que yo os pido, a mis apóstoles, es la \textit{unidad espiritual} ---y eso lo podéis experimentar en la alegría de vuestra dedicación unida a hacer de todo corazón la voluntad de mi Padre que está en los cielos. No necesitáis tener el mismo punto de vista, sentir de la misma manera o ni siquiera pensar de la misma manera, para \textit{ser iguales} espiritualmente. La unidad espiritual procede de la conciencia de que cada uno de vosotros está habitado, y cada vez más gobernado, por el don espiritual del Padre celestial. Vuestra armonía apostólica debe originarse en el hecho de que la esperanza espiritual de cada uno de vosotros es idéntica en su origen, naturaleza y destino>>\footnote{\textit{Se requiere unidad espiritual}: Ef 4:3-12. \textit{Unidad en la diversidad}: 1 Co 12:4-31. \textit{No sacar las creencias fuera de su esencia}: Mt 12:43-45; Lc 11:24-26.}.

\par 
%\textsuperscript{(1591.7)}
\textsuperscript{141:5.2} <<De esta manera podéis experimentar una unidad perfeccionada de intención espiritual y de comprensión espiritual, que tiene su origen en la conciencia mutua de la identidad de cada uno de vuestros espíritus paradisiacos internos; y podéis disfrutar toda esta profunda unidad espiritual en presencia misma de la extrema diversidad de vuestras actitudes individuales en lo referente a la reflexión intelectual, a los sentimientos propios de vuestro temperamento y a la conducta social. Vuestras personalidades pueden ser agradablemente variadas y notablemente diferentes, pero vuestras naturalezas espirituales y los frutos espirituales de vuestra adoración divina y de vuestro amor fraternal pueden estar tan unificados, que todos los que contemplen vuestra vida reconocerán con toda seguridad esta identidad de espíritu y esta unidad de alma. Reconocerán que habéis estado conmigo y que habéis aprendido así a hacer, de una manera aceptable, la voluntad del Padre que está en los cielos. Podéis conseguir la unidad en el servicio de Dios, aunque cada uno de vosotros cumpla ese servicio siguiendo la técnica de sus propias dotaciones originales de mente, de cuerpo y de alma>>.

\par 
%\textsuperscript{(1592.1)}
\textsuperscript{141:5.3} <<Vuestra unidad espiritual implica dos factores, que siempre se armonizarán en la vida de los creyentes individuales: En primer lugar, poseéis un motivo común para una vida de servicio; todos deseáis por encima de todo hacer la voluntad del Padre que está en los cielos. Y en segundo lugar, todos tenéis una meta común en la existencia; todos os proponéis encontrar al Padre que está en los cielos, mostrando así al universo que os habéis vuelto como él>>.

\par 
%\textsuperscript{(1592.2)}
\textsuperscript{141:5.4} Jesús volvió muchas veces sobre este tema durante la preparación de los doce. Les dijo repetidamente que no deseaba que los que creían en él se volvieran dogmatizados y uniformizados según las interpretaciones religiosas incluso de los hombres de bien. Una y otra vez previno a sus apóstoles contra la elaboración de credos y el establecimiento de tradiciones como medio de guiar y controlar a los creyentes en el evangelio del reino.

\section*{6. La última semana en Amatus}
\par 
%\textsuperscript{(1592.3)}
\textsuperscript{141:6.1} Hacia el final de la última semana en Amatus, Simón Celotes llevó ante Jesús a un tal Tejerma, un persa que hacía negocios en Damasco. Tejerma había oído hablar de Jesús y había venido a Cafarnaúm para verlo. Al enterarse de que Jesús se había ido con sus apóstoles bajando por el Jordán hacia Jerusalén, partió en su búsqueda. Andrés había presentado Tejerma a Simón para que lo instruyera. Simón consideraba al persa como un <<adorador del fuego>>, aunque Tejerma se esmeró en explicarle que el fuego sólo era el símbolo visible del Único Puro y Santo. Después de hablar con Jesús, el persa manifestó su intención de permanecer varios días para oír la enseñanza y escuchar la predicación.

\par 
%\textsuperscript{(1592.4)}
\textsuperscript{141:6.2} Cuando Simón Celotes y Jesús se quedaron solos, Simón le preguntó al Maestro: <<¿Por qué no he podido persuadirlo? ¿Por qué se ha resistido tanto conmigo y te ha escuchado tan rápidamente?>> Jesús respondió: <<Simón, Simón, ¿cuántas veces te he enseñado que dejes de esforzarte por \textit{extraer} algo del corazón de los que buscan la salvación? ¿Cuántas veces te he dicho que trabajes solamente para \textit{introducir} algo dentro de esas almas hambrientas? Conduce a los hombres hasta el reino, y las grandes verdades vivientes del reino pronto expulsarán todo error grave. Cuando hayas dado a conocer al hombre mortal la buena nueva de que Dios es su Padre, podrás persuadirlo más fácilmente de que es en realidad un hijo de Dios. Una vez hecho esto, habrás llevado la luz de la salvación a un ser que está en las tinieblas. Simón, cuando el Hijo del Hombre vino a ti por primera vez, ¿llegó acusando a Moisés y a los profetas para proclamar una manera de vivir nueva y mejor? No. No he venido para eliminar lo que poseéis de vuestros antepasados, sino para mostraros la visión completa de lo que vuestro padres sólo vieron en parte. Así pues Simón, ve a enseñar y a predicar el reino, y cuando tengas a un hombre a salvo y seguro en el reino, entonces será momento, si se acerca a ti con sus preguntas, de impartirle una enseñanza relacionada con el avance progresivo del alma dentro del reino divino>>.

\par 
%\textsuperscript{(1592.5)}
\textsuperscript{141:6.3} Simón se quedó asombrado con estas palabras, pero hizo lo que Jesús le había enseñado, y Tejerma el persa fue contado entre los que entraron en el reino.

\par 
%\textsuperscript{(1592.6)}
\textsuperscript{141:6.4} Aquella noche, Jesús dio un discurso a los apóstoles sobre la nueva vida en el reino. Dijo en parte: <<Cuando entráis en el reino, nacéis de nuevo. No podéis enseñar las cosas profundas del espíritu a los que sólo han nacido de la carne; primero cuidad de que los hombres nazcan de espíritu, antes de intentar instruirlos en los caminos avanzados del espíritu. No empecéis a mostrar a los hombres las bellezas del templo hasta que no hayan entrado primero dentro del templo. Presentad los hombres a Dios, \textit{como} hijos de Dios, antes de discurrir sobre las doctrinas de la paternidad de Dios y de la filiación de los hombres. No rivalicéis con los hombres ---sed siempre pacientes. El reino no es vuestro, sólo sois sus embajadores. Salid simplemente a proclamar: He aquí el reino de los cielos ---Dios es vuestro Padre y vosotros sois sus hijos, y si creéis de todo corazón, esta buena nueva \textit{es} vuestra salvación eterna>>\footnote{\textit{Nacer de nuevo}: Jn 3:1-12. \textit{No rivalicéis con los hombres}: 2 Ti 2:23-26.}.

\par 
%\textsuperscript{(1593.1)}
\textsuperscript{141:6.5} Los apóstoles hicieron grandes progresos durante la estancia en Amatus. Pero se sintieron muy decepcionados de que Jesús no les diera ninguna sugerencia sobre las relaciones con los discípulos de Juan. Incluso en la importante cuestión del bautismo, Jesús se limitó a decir: <<En verdad, Juan ha bautizado con agua, pero cuando entréis en el reino de los cielos, seréis bautizados con el Espíritu>>\footnote{\textit{Bautismo con el espíritu}: Mt 3:11; 28:19; Mc 1:8; Lc 3:16; Jn 1:32-33; Hch 1:5; 2:1-4,38; 10:47; 11:16.}.

\section*{7. En Betania más allá del Jordán}
\par 
%\textsuperscript{(1593.2)}
\textsuperscript{141:7.1} El 26 de febrero, Jesús, sus apóstoles y un grupo numeroso de discípulos viajaron siguiendo el Jordán hasta el vado cerca de Betania en Perea, el lugar donde Juan había proclamado por primera vez el reino venidero. Jesús permaneció allí con sus apóstoles, enseñando y predicando durante cuatro semanas, antes de partir para subir a Jerusalén.

\par 
%\textsuperscript{(1593.3)}
\textsuperscript{141:7.2} Durante la segunda semana de su estancia en Betania más allá del Jordán, Jesús se llevó a Pedro, Santiago y Juan para descansar tres días en las colinas situadas al otro lado del río, al sur de Jericó. El Maestro enseñó a estos tres hombres muchas verdades nuevas y avanzadas sobre el reino de los cielos. Dichas enseñanzas las hemos reorganizado y clasificado de la manera siguiente a efectos de este relato:

\par 
%\textsuperscript{(1593.4)}
\textsuperscript{141:7.3} Jesús procuró dejar muy claro que deseaba que sus discípulos, una vez que hubieran probado las buenas realidades espirituales del reino, vivieran de tal manera en el mundo que cuando los hombres \textit{vieran} sus vidas se volvieran conscientes del reino, y se sintieran así inducidos a preguntar a los creyentes sobre los caminos del reino. Todos estos buscadores sinceros de la verdad se alegran siempre de \textit{escuchar} la buena nueva del don de la fe que asegura la admisión en el reino, con sus realidades espirituales eternas y divinas.

\par 
%\textsuperscript{(1593.5)}
\textsuperscript{141:7.4} El Maestro intentó imprimir en el ánimo de todos los educadores del evangelio del reino que lo único que tenían que hacer era revelar al hombre individual que Dios es su Padre ---llevar a ese hombre individual a hacerse consciente de su filiación; y luego, presentar este mismo hombre a Dios como su hijo por la fe. Estas dos revelaciones esenciales se cumplían en Jesús. Él se convirtió, efectivamente, en <<el camino, la verdad y la vida>>\footnote{\textit{Jesús es el camino, la verdad y la vida}: Jn 14:6.}. La religión de Jesús estaba enteramente basada en la manera de vivir su vida de donación en la Tierra. Cuando Jesús se marchó de este mundo, no dejó detrás de él ni libros, ni leyes, ni otras formas de organización humana que afectaran la vida religiosa del individuo.

\par 
%\textsuperscript{(1593.6)}
\textsuperscript{141:7.5} Jesús indicó francamente que había venido para establecer unas relaciones personales y eternas con los hombres, que siempre tendrían prioridad sobre todas las demás relaciones humanas. Y recalcó que esta hermandad espiritual íntima debía extenderse a todos los hombres de todas las épocas y de todas las condiciones sociales, en todos los pueblos. La única recompensa que ofrecía a sus hijos era: en este mundo, la alegría espiritual y la comunión divina; y en el mundo siguiente, la vida eterna en el desarrollo de las realidades espirituales divinas del Padre Paradisiaco.

\par 
%\textsuperscript{(1593.7)}
\textsuperscript{141:7.6} Jesús hizo mucho hincapié en lo que él llamaba las dos verdades de primera importancia en las enseñanzas del reino, que son las siguientes: conseguir la salvación por medio de la fe, y de la fe solamente, asociada con la enseñanza revolucionaria de conseguir la libertad humana mediante el reconocimiento sincero de la verdad. <<Conoceréis la verdad y la verdad os hará libres>>\footnote{\textit{La verdad os hará libres}: Jn 8:32.}. Jesús era la verdad manifestada en la carne\footnote{\textit{Jesús era la verdad hecha manifiesta}: Jn 1:14.}, y prometió enviar a su Espíritu de la Verdad al corazón de todos sus hijos después de regresar al Padre que está en los cielos.

\par 
%\textsuperscript{(1594.1)}
\textsuperscript{141:7.7} El Maestro enseñaba a estos apóstoles los elementos esenciales de la verdad para toda una era de la Tierra. A menudo escuchaban sus enseñanzas, aunque lo que decía estaba destinado en realidad a inspirar y edificar a otros mundos. Dio ejemplo de un plan de vida nuevo y original. Desde el punto de vista humano era en verdad un judío, pero vivió su vida para todo el planeta como un mortal del mundo.

\par 
%\textsuperscript{(1594.2)}
\textsuperscript{141:7.8} Para estar seguro de que su Padre sería reconocido durante el desarrollo del plan del reino, Jesús explicó que había ignorado adrede a los <<grandes de la Tierra>>. Empezó su trabajo con los pobres\footnote{\textit{Empezar su trabajo con los pobres}: Mt 11:5; Lc 4:18; 7:22; 14:13.}, la clase que precisamente había sido tan desdeñada por la mayoría de las religiones evolutivas de las épocas anteriores. No despreciaba a ninguna persona; su plan era mundial, e incluso universal. Fue tan audaz y enérgico en estas declaraciones, que incluso Pedro, Santiago y Juan estuvieron tentados de creer que quizás había perdido el juicio\footnote{\textit{Jesús visto como fuera de sí}: Mc 3:21.}.

\par 
%\textsuperscript{(1594.3)}
\textsuperscript{141:7.9} Intentó impartir suavemente a estos apóstoles la verdad de que había venido a esta misión donadora, no para dar un ejemplo a algunas criaturas de la Tierra, sino para establecer y demostrar un modelo de vida humana para todos los pueblos de todos los mundos en todo su universo. Este modelo de vida se acercaba a la perfección más alta, incluso a la bondad final del Padre Universal. Pero los apóstoles no podían comprender el significado de sus palabras.

\par 
%\textsuperscript{(1594.4)}
\textsuperscript{141:7.10} Declaró que había venido para ejercer como instructor, un instructor enviado del cielo\footnote{\textit{Instructor venido del cielo}: Mt 11:1; 22:16; Mc 4:1; 6:2,34; 8:31; Jn 3:2; Hch 1:1.} para presentar la verdad espiritual a la mente material. Y esto es exactamente lo que hizo. Era un instructor, no un predicador. Desde el punto de vista humano, Pedro era un predicador mucho más eficaz que Jesús. Si la predicación de Jesús era tan eficaz, se debía más a su personalidad excepcional que a una irresistible atracción oratoria o emocional. Jesús hablaba directamente al alma de los hombres. Instruía al espíritu del hombre, pero a través de la mente. Vivía con los hombres.

\par 
%\textsuperscript{(1594.5)}
\textsuperscript{141:7.11} Fue en esta ocasión cuando Jesús insinuó a Pedro, Santiago y Juan que su trabajo en la Tierra estaba limitado en algunos aspectos por encargo de su <<asociado de arriba>>, refiriéndose a las instrucciones recibidas de su hermano paradisiaco Emmanuel antes de la donación. Les dijo que había venido para hacer la voluntad de su Padre\footnote{\textit{Jesús vivió la voluntad del Padre}: Mt 26:39,42,44; Mc 14:36,39; Lc 22:42; Jn 4:34; 5:30; 6:38-40; 15:10; 17:4.}, y únicamente la voluntad de su Padre. Como estaba motivado así por una sola intención sincera, no se preocupaba ansiosamente por el mal en el mundo.

\par 
%\textsuperscript{(1594.6)}
\textsuperscript{141:7.12} Los apóstoles empezaban a reconocer la amistad sin afectación de Jesús. Aunque era fácil acercarse al Maestro, siempre vivía independientemente de todos los seres humanos, y por encima de ellos. Nunca estuvo dominado ni un solo momento por una influencia puramente humana, o sujeto al frágil juicio humano. No prestaba ninguna atención a la opinión pública y no se dejaba influir por los elogios. Rara vez se interrumpió para corregir malentendidos o para ofenderse por una tergiversación. Nunca le pidió consejo a nadie; nunca solicitó oraciones.

\par 
%\textsuperscript{(1594.7)}
\textsuperscript{141:7.13} Santiago estaba asombrado por la manera en que Jesús parecía ver el fin desde el principio. El Maestro rara vez parecía sorprenderse. Nunca estaba excitado, enojado o desconcertado. Nunca pidió disculpas a nadie. A veces estaba triste, pero nunca desanimado.

\par 
%\textsuperscript{(1594.8)}
\textsuperscript{141:7.14} Juan percibió más claramente que, a pesar de todos sus atributos divinos, después de todo Jesús era humano\footnote{\textit{Jesús era ``humano''}: Jn 1:14; 1 Ti 2:5.}. Jesús vivía como un hombre entre los hombres, y los comprendía, los amaba y sabía cómo dirigirlos. En su vida personal era tan humano, y sin embargo tan irreprochable. Y siempre era desinteresado.

\par 
%\textsuperscript{(1595.1)}
\textsuperscript{141:7.15} Aunque Pedro, Santiago y Juan no pudieron comprender gran cosa de lo que Jesús dijo en esta ocasión, sus palabras bondadosas se grabaron en sus corazones, y después de la crucifixión y la resurrección, surgieron abundantemente para enriquecer y alegrar su ministerio posterior. No es de extrañar que estos apóstoles no comprendieran plenamente las palabras del Maestro, porque estaba delineando ante ellos el plan de una nueva era.

\section*{8. Trabajo en Jericó}
\par 
%\textsuperscript{(1595.2)}
\textsuperscript{141:8.1} Durante las cuatro semanas de estancia en Betania más allá del Jordán, Andrés designó varias veces por semana a unas parejas apostólicas para que subieran uno o dos días a Jericó. Juan tenía muchos creyentes en Jericó, y la mayoría de ellos acogieron con placer las enseñanzas más avanzadas de Jesús y sus apóstoles. Durante estas visitas a Jericó, los apóstoles empezaron a llevar a cabo más expresamente las instrucciones de Jesús de ayudar a los enfermos\footnote{\textit{Los apóstoles ayudando a los enfermos}: Mt 10:8; Lc 10:9.}; visitaron cada casa de la ciudad y trataron de confortar a todas las personas afligidas.

\par 
%\textsuperscript{(1595.3)}
\textsuperscript{141:8.2} Los apóstoles efectuaron alguna labor pública en Jericó, pero sus esfuerzos fueron principalmente de naturaleza más tranquila y personal. Ahora hicieron el descubrimiento de que la buena nueva del reino reconfortaba mucho a los enfermos, que su mensaje llevaba la curación a los afligidos. Fue en Jericó donde los doce pusieron en práctica, por primera vez, el encargo de Jesús de predicar la buena nueva del reino y de atender a los afligidos.

\par 
%\textsuperscript{(1595.4)}
\textsuperscript{141:8.3} Se detuvieron en Jericó, de camino hacia Jerusalén, y fueron alcanzados por una delegación de Mesopotamia que había venido para hablar con Jesús. Los apóstoles habían proyectado pasar un solo día allí, pero cuando llegaron estos buscadores orientales de la verdad, Jesús pasó tres días con ellos. Éstos últimos regresaron a sus diversos hogares, a lo largo del Éufrates, con la felicidad de conocer las nuevas verdades del reino de los cielos.

\section*{9. La partida hacia Jerusalén}
\par 
%\textsuperscript{(1595.5)}
\textsuperscript{141:9.1} El último día de marzo, un lunes, Jesús y los apóstoles emprendieron la subida de las colinas hacia Jerusalén. Lázaro de Betania había bajado dos veces al Jordán para ver a Jesús, y se habían tomado todas las disposiciones necesarias para que el Maestro y sus apóstoles instalaran su cuartel general en la casa de Lázaro y sus hermanas, en Betania, durante todo el tiempo que desearan quedarse en Jerusalén.

\par 
%\textsuperscript{(1595.6)}
\textsuperscript{141:9.2} Los discípulos de Juan permanecieron en Betania más allá del Jordán, enseñando y bautizando a las multitudes, de manera que Jesús sólo iba acompañado de los doce cuando llegó a casa de Lázaro. Jesús y los apóstoles se detuvieron allí durante cinco días, descansando y reponiéndose, antes de continuar hacia Jerusalén para la Pascua. Fue un gran acontecimiento en la vida de Marta y María tener al Maestro y a sus apóstoles en el hogar de su hermano, donde pudieron atender sus necesidades.

\par 
%\textsuperscript{(1595.7)}
\textsuperscript{141:9.3} El domingo 6 de abril por la mañana, Jesús y los apóstoles bajaron a Jerusalén\footnote{\textit{A Jerusalén}: Jn 2:23.}; ésta era la primera vez que el Maestro y los doce se encontraban allí todos juntos.