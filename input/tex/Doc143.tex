\chapter{Documento 143. La travesía de Samaria}
\par 
%\textsuperscript{(1607.1)}
\textsuperscript{143:0.1} A FINALES de junio del año 27, debido a la oposición creciente de los dirigentes religiosos judíos, Jesús y los doce partieron de Jerusalén después de enviar sus tiendas y sus escasos efectos personales para que fueran guardados en la casa de Lázaro, en Betania. Se dirigieron al norte hacia Samaria, y el sábado se detuvieron en Betel. Predicaron allí durante varios días a la gente que venía de Gofna y Efraín. Un grupo de ciudadanos de Arimatea y Tamna vino para invitar a Jesús a que visitara sus pueblos. El Maestro y sus apóstoles pasaron más de dos semanas enseñando a los judíos y samaritanos de esta región, muchos de los cuales venían de lugares tan lejanos como Antípatris para escuchar la buena nueva del reino.

\par 
%\textsuperscript{(1607.2)}
\textsuperscript{143:0.2} Los habitantes del sur de Samaria escucharon con placer a Jesús, y los apóstoles, a excepción de Judas Iscariote, consiguieron vencer muchos de los prejuicios que tenían contra los samaritanos. A Judas le resultaba muy difícil amar a estos samaritanos. La última semana de julio, Jesús y sus compañeros se prepararon para partir hacia las nuevas ciudades griegas de Fasaelis y Arquelais, cerca del Jordán.

\section*{1. La predicación en Arquelais}
\par 
%\textsuperscript{(1607.3)}
\textsuperscript{143:1.1} Durante la primera mitad del mes de agosto, el grupo apostólico estableció su cuartel general en las ciudades griegas de Arquelais y Fasaelis, donde efectuaron su primera experiencia de predicación a una concurrencia compuesta casi exclusivamente de gentiles ---griegos, romanos y sirios--- ya que pocos judíos residían en estas dos ciudades griegas. Al ponerse en contacto con estos ciudadanos romanos, los apóstoles encontraron nuevas dificultades para proclamar el mensaje del reino venidero, y tropezaron con nuevas objeciones a las enseñanzas de Jesús. En una de las muchas conversaciones nocturnas con sus apóstoles, Jesús escuchó atentamente estas objeciones al evangelio del reino mientras los doce repasaban sus experiencias con la gente que se había beneficiado de su trabajo personal.

\par 
%\textsuperscript{(1607.4)}
\textsuperscript{143:1.2} Felipe hizo una pregunta que fue representativa de sus dificultades. Felipe dijo: <<Maestro, estos griegos y romanos menosprecian nuestro mensaje, pues dicen que estas enseñanzas sólo son adecuadas para los débiles y los esclavos. Aseguran que la religión de los paganos es superior a nuestra enseñanza, porque estimula a adquirir un carácter fuerte, robusto y dinámico. Afirman que queremos convertir a todos los hombres en unos especímenes debilitados de no resistentes pasivos, que desaparecerían rápidamente de la faz de la Tierra. A ti te aprecian, Maestro, y admiten francamente que tu enseñanza es celestial e ideal, pero no quieren tomarnos en serio. Afirman que tu religión no es para este mundo, que los hombres no pueden vivir según lo que enseñas. Y ahora, Maestro, ¿qué les vamos a decir a estos gentiles?>>

\par 
%\textsuperscript{(1607.5)}
\textsuperscript{143:1.3} Después de haber escuchado otras objeciones similares al evangelio del reino presentadas por Tomás, Natanael, Simón Celotes y Mateo, Jesús dijo a los doce:

\par 
%\textsuperscript{(1608.1)}
\textsuperscript{143:1.4} <<He venido a este mundo para hacer la voluntad de mi Padre y para revelar su carácter afectuoso a toda la humanidad. Ésta es, hermanos míos, mi misión. Y ésta es la única cosa que haré, independientemente de que mis enseñanzas sean mal comprendidas por los judíos o los gentiles de esta época o de otra generación. Pero no deberíais pasar por alto el hecho de que el amor divino también tiene sus disciplinas severas. El amor de un padre por su hijo obliga muchas veces al padre a refrenar las acciones imprudentes de su atolondrado descendiente. El hijo no siempre comprende los motivos sabios y afectuosos de la disciplina restrictiva del padre. Pero os aseguro que mi Padre Paradisiaco gobierna de hecho un universo de universos con el poder predominante de su amor. El amor es la más grande de todas las realidades espirituales. La verdad es una revelación liberadora, pero el amor es la relación suprema. Cualesquiera que sean los desatinos que vuestros compañeros humanos puedan cometer en la administración del mundo de hoy, el evangelio que os proclamo gobernará este mismo mundo en una era por venir. La meta última del progreso humano consiste en reconocer respetuosamente la paternidad de Dios y en materializar con amor la fraternidad de los hombres>>\footnote{\textit{Jesús vino a hacer la voluntad de Dios}: Mt 26:39,42,44; Mc 14:36,39; Lc 22:42; Jn 4:34; 5:30; 6:38-40; 15:10; 17:4. \textit{Amorosa disciplina de los niños}: Heb 12:5-11. \textit{Dios gobierna mediante el amor}: Mt 5:43-48; 22:37-40; Mc 12:29-33; Lc 10:27; Jn 3:16; 13:34-35; 14:21-23; 15:9-13,17; 16:27; 17:22-23; Ro 5:8; 1 Co 13:1-8; 2 Co 13:11; Tit 3:4; 1 Jn 3:1; 4:7-19.}.

\par 
%\textsuperscript{(1608.2)}
\textsuperscript{143:1.5} <<¿Quién os ha dicho que mi evangelio sólo está destinado a los esclavos y a los débiles? ¿Acaso vosotros, mis apóstoles elegidos, parecéis débiles? ¿Tenía Juan aspecto de endeble? ¿Observáis que yo sea esclavo del miedo? Es verdad que el evangelio se predica a los pobres y a los oprimidos de esta generación. Las religiones de este mundo han olvidado a los pobres, pero mi Padre no hace acepción de personas. Además, los pobres de hoy son los primeros en hacer caso de la llamada al arrepentimiento y a aceptar la filiación. El evangelio del reino debe ser predicado a todos los hombres ---judíos y gentiles, griegos y romanos, ricos y pobres, libres y esclavos--- e igualmente a los jóvenes y a los viejos, a los hombres y a las mujeres>>\footnote{\textit{Id predicad el evangelio a todos}: Mt 24:14; 28:19-20a; Mc 13:10; 16:15; Lc 24:47; Jn 17:18; Hch 1:8b. \textit{El evangelio del reino}: Mt 3:2; 4:17,23; 5:3,10,19-20; 6:33; 7:21; 8:11; 9:35; 10:7; 11:11-12; 12:28; 13:11,24,31-52; 16:19; 18:1-4,23; 19:14,23-24; 20:1; 21:31,43; 22:2; 23:13; 24:14; 25:1,14; Mc 1:14-15; 4:11,26,30; 9:1,47; 10:14-15,23-25; 12:34; 14:25; 15:43; Lc 4:43; 6:20; 7:28; 8:1,10; 9:2,11,27; 9:60,62; 10:9-11; 11:20; 12:31-32; 13:18,20,28,29; 14:15; 16:16; 17:20-21; 18:16-17,24-25; 19:11; 21:31; 22:16,18; 23:51; Jn 3:3,5; Ro 14:17; 1 Co 4:20; 6:9-10. \textit{El evangelio predicado a los pobres}: Is 61:1-2; Mt 11:4-5; Lc 7:22. \textit{¿Acaso Juan era un débil?}: Mt 11:7-9; Lc 7:24-26. \textit{Mi padre no hace acepción de personas}: 2 Cr 19:7; Job 34:19; Eclo 35:12; Hch 10:34; Ro 2:11; Gl 2:6; 3:28; Ef 6:9; Col 3:11.}.

\par 
%\textsuperscript{(1608.3)}
\textsuperscript{143:1.6} <<Aunque mi Padre es un Dios de amor y se deleita practicando la misericordia, no os impregnéis de la idea de que el servicio del reino debe ser de una facilidad monótona. La ascensión al Paraíso es la aventura suprema de todos los tiempos, la dura obtención de la eternidad. El servicio del reino en la Tierra exigirá toda la valiente virilidad que vosotros y vuestros colaboradores podáis reunir. Muchos de vosotros seréis ejecutados por vuestra lealtad al evangelio de este reino. Es fácil morir en el campo de batalla cuando la presencia de vuestros camaradas de combate fortalece vuestra valentía, pero se requiere una forma superior y más profunda de valentía y de devoción humanas para dar la vida con serenidad y en solitario por el amor de una verdad guardada en vuestro corazón mortal>>\footnote{\textit{Dios de amor}: Mt 5:43-48; 22:37-40; Mc 12:29-33; Lc 10:27; Jn 3:16; 13:34-35; 14:21-23; 15:9-13,17; 16:27; 17:22-23; Ro 5:8; 1 Co 13:1-8; 2 Co 13:11; Tit 3:4; 1 Jn 3:1; 4:7-19. \textit{Dios se deleita en la misericordia}: Ex 20:6; 1 Cr 16:34; 2 Cr 5:13; 7:3,6; 30:9; Esd 3:11; Sal 25:6; 36:5; 86:5,13,15; 100:5:; 103:8,11,17; 107:1; 116:5; 117:2; 118:1,4; 136:1-26; 145;8; Is 55:7; Jer 3:12; Nm 14:18-19; Miq 7:18; Dt 5:10; Heb 6:12. \textit{Exigencia de un servicio valiente}: Mt 5:10; Lc 21:12,16-17; Ro 5:3-5; 8:35-39; 2 Co 6:3-10; 1 P 4:12-14.}.

\par 
%\textsuperscript{(1608.4)}
\textsuperscript{143:1.7} <<Hoy, los incrédulos pueden mofarse de vosotros porque predicáis un evangelio de no resistencia y porque vivís una vida sin violencia, pero sois los primeros voluntarios de una larga serie de creyentes sinceros en el evangelio de este reino, que asombrarán a toda la humanidad por su consagración heroica a estas enseñanzas. Ningún ejército del mundo ha desplegado nunca más coraje y bravura que los que mostraréis vosotros y vuestros leales sucesores cuando salgáis para proclamar al mundo entero la buena nueva ---la paternidad de Dios y la fraternidad de los hombres. La valentía de la carne es la forma más baja de bravura. La bravura mental es un tipo más elevado de valentía humana, pero la bravura superior y suprema consiste en la fidelidad inflexible a las convicciones iluminadas de las realidades espirituales profundas. Una valentía así constituye el heroísmo del hombre que conoce a Dios. Y todos vosotros sois hombres que conocéis a Dios; sois, en verdad, los asociados personales del Hijo del Hombre>>.

\par 
%\textsuperscript{(1608.5)}
\textsuperscript{143:1.8} Esto no es todo lo que Jesús dijo en esta ocasión, pero es la introducción de su discurso. Luego continuó hablando largamente para ampliar e ilustrar esta declaración. Éste fue uno de los discursos más apasionados que Jesús pronunció nunca ante los doce. El Maestro rara vez hablaba a sus apóstoles mostrando unos poderosos sentimientos, pero ésta fue una de las pocas ocasiones en las que se expresó con una seriedad manifiesta, acompañada de una marcada emoción.

\par 
%\textsuperscript{(1609.1)}
\textsuperscript{143:1.9} El efecto sobre la predicación pública y el ministerio personal de los apóstoles fue inmediato; a partir de aquel mismo día, su mensaje adquirió un nuevo matiz en el que predominaba la valentía. Los doce continuaron adquiriendo el espíritu positivamente dinámico del nuevo evangelio del reino. Desde aquel día en adelante, ya no se ocuparon tanto de predicar las virtudes negativas y los preceptos pasivos de la enseñanza multifacética de su Maestro.

\section*{2. La lección sobre el dominio de sí mismo}
\par 
%\textsuperscript{(1609.2)}
\textsuperscript{143:2.1} El Maestro era un ejemplo perfeccionado de un hombre dueño de sí mismo\footnote{\textit{El dominio de sí mismo de Jesús}: 1 P 2:23-24.}. Cuando fue injuriado, no injurió; cuando sufrió, no profirió ninguna amenaza contra sus torturadores; cuando fue acusado por sus enemigos, simplemente se encomendó al juicio justo del Padre que está en los cielos.

\par 
%\textsuperscript{(1609.3)}
\textsuperscript{143:2.2} En una de las conferencias nocturnas, Andrés le preguntó a Jesús: <<Maestro, ¿debemos practicar la abnegación como Juan nos ha enseñado, o debemos procurar adquirir el autocontrol que tú enseñas? ¿En qué se diferencia tu enseñanza de la de Juan?>> Jesús respondió: <<En verdad, Juan os ha enseñado el camino de la rectitud de acuerdo con las luces y las leyes de sus antepasados; era la religión del examen de conciencia y de la abnegación. Pero yo vengo con un nuevo mensaje de olvido de sí mismo y de dominio de sí mismo. Os muestro el camino de la vida tal como mi Padre que está en los cielos me lo ha revelado>>.

\par 
%\textsuperscript{(1609.4)}
\textsuperscript{143:2.3} <<En verdad, en verdad os digo que aquel que se gobierna a sí mismo es más grande que el que conquista una ciudad. El dominio de sí mismo es la medida de la naturaleza moral de un hombre, y el indicador de su desarrollo espiritual. En el antiguo orden practicabais el ayuno y la oración. Como criaturas nuevas renacidas del espíritu, se os enseña a creer y a regocijaros. En el reino del Padre, debéis convertiros en criaturas nuevas; las cosas viejas deben desaparecer; observad que os muestro cómo todas las cosas deben renovarse. Por medio de vuestro amor recíproco vais a convencer al mundo de que habéis pasado de la esclavitud a la libertad, de la muerte a la vida eterna>>\footnote{\textit{Aquel que se gobierna a sí mismo es grande}: Pr 16:32. \textit{El antiguo orden era ayuno y oración}: Mt 9:14; Mc 2:18; Lc 5:33. \textit{El orden antiguo y el nuevo}: Mt 9:16-17; Mc 2:21-22; Lc 5:36-38; Ro 7:6; 2 Co 5:17. \textit{Testigos a través del amor}: Jn 13:35. \textit{De la esclavitud a la libertad}: Ro 8:21.}.

\par 
%\textsuperscript{(1609.5)}
\textsuperscript{143:2.4} <<En el antiguo camino, intentáis suprimir, obedecer y conformaros a unas reglas de vida; en el nuevo camino, primero sois \textit{transformados} por el Espíritu de la Verdad y, por ello, fortalecidos en vuestra alma interior mediante la constante renovación espiritual de vuestra mente; así estáis dotados con el poder de ejecutar, con certeza y alegría, la voluntad misericordiosa, aceptable y perfecta de Dios. No lo olvidéis ---vuestra fe personal en las promesas extremadamente grandes y preciosas de Dios es la que os garantiza que os convertiréis en partícipes de la naturaleza divina. Así, mediante vuestra fe y la transformación del espíritu, os convertís en realidad en los templos de Dios, y su espíritu vive efectivamente dentro de vosotros. Así pues, si el espíritu reside dentro de vosotros, ya no sois unos esclavos ligados a la carne, sino unos hijos del espíritu, independientes y liberados. La nueva ley del espíritu os dota de la libertad del dominio de sí mismo, reemplazando la antigua ley del miedo, basada en la autoesclavitud y en el yugo de la abnegación>>\footnote{\textit{Transformación}: Sal 51:10; Ez 18:31; 36:26; Ro 12:2; 2 Co 5:17-19. \textit{Fe en las promesas}: 2 P 1:4. \textit{Seréis los templos de Dios}: Lc 17:21; Ro 8:9-11; 1 Co 3:16-17; 6:19-20; 2 Co 6:16; 2 Ti 1:14; 1 Jn 4:12-15; Ap 21:3. \textit{Hijos liberados del Espíritu}: Ro 8:2,15-16; 2 Co 3:17; Gl 4:6-7.}.

\par 
%\textsuperscript{(1609.6)}
\textsuperscript{143:2.5} <<Muchas veces, cuando habéis hecho el mal, habéis pensado en imputar vuestros actos a la influencia del demonio, cuando en realidad simplemente os habéis descarriado a causa de vuestras propias tendencias naturales. ¿No os ha dicho el profeta Jeremías hace mucho tiempo que el corazón humano es más engañoso que nada, e incluso a veces desesperadamente perverso? ¡Qué fácil es engañaros a vosotros mismos y caer así en unos temores tontos, en deseos de todo tipo, placeres esclavizantes, malicia, envidia e incluso en un odio vengativo!>>\footnote{\textit{El corazón humano es engañoso}: Jer 17:9.}

\par 
%\textsuperscript{(1610.1)}
\textsuperscript{143:2.6} <<La salvación se obtiene por la regeneración del espíritu y no por las acciones presuntuosas de la carne. Estáis justificados por la fe y sois aceptados por la gracia, no por el temor y la abnegación de la carne, aunque los hijos del Padre, que han nacido del espíritu, son siempre y para siempre \textit{dueños} de su yo y de todo lo que se refiere a los deseos de la carne. Cuando sabéis que es la fe la que os salva, tenéis una verdadera paz con Dios. Y todos los que siguen el camino de esta paz celestial están destinados a ser santificados en el servicio eterno de los hijos, en constante progreso, del Dios eterno. En lo sucesivo, ya no es un deber, sino que es más bien vuestro elevado privilegio el purificaros de todos los males de la mente y del cuerpo, mientras buscáis la perfección en el amor de Dios>>\footnote{\textit{Salvados por la fe, no por las obras}: Mc 16:16; Jn 3:36; Ro 3:27-30; Gl 2:16. \textit{Los hijos de la fe son dueños de sí mismos}: Ro 8:5,14; Gl 5:16. \textit{Paz con Dios}: Jn 14:27; 16:33; Ro 5:1-2; 1 Ts 5:23. \textit{Purificarse}: Sal 51:10; 2 Co 7:1; Stg 4:8.}.

\par 
%\textsuperscript{(1610.2)}
\textsuperscript{143:2.7} <<Vuestra filiación está fundada en la fe, y debéis permanecer impasibles ante el miedo. Vuestra alegría nace de la confianza en la palabra divina, y por consiguiente, no pondréis en duda la realidad del amor y de la misericordia del Padre. La bondad misma de Dios es la que conduce a los hombres a un arrepentimiento sincero y auténtico. Vuestro secreto para dominar el yo está ligado a vuestra fe en el espíritu interno, que siempre actúa por amor. Incluso esta fe salvadora no la tenéis por vosotros mismos; es también un regalo de Dios. Si sois los hijos de esta fe viviente, ya no sois los esclavos del yo, sino más bien los dueños triunfantes de vosotros mismos, los hijos liberados de Dios>>\footnote{\textit{Filiación por la fe}: Jn 1:12; Ro 8:14; Gl 3:26. \textit{La bondad de Dios}: Sal 86:5; Mt 19:17; Mc 10:18; Lc 18:19; Ro 2:4. \textit{La fe es un regalo de Dios}: Jn 6:40,65; Ef 2:8; Stg 1:17. \textit{Los hijos liberados de Dios}: 1 Cr 22:10; Sal 2:7; Is 56:5; Mt 5:9,16,45; Lc 20:36; Jn 1:12-13; 11:52; Hch 17:28-29; Ro 8:14-15,19,21; 9:26; 2 Co 6:18; Gl 3:26; 4:5-7; Ef 1:5; Flp 2:15; Heb 12:5-8; 1 Jn 3:1-2,10; 5:2; Ap 21:7; 2 Sam 7:14.}.

\par 
%\textsuperscript{(1610.3)}
\textsuperscript{143:2.8} <<Así pues, hijos míos, si habéis nacido del espíritu, estáis liberados para siempre de la esclavitud consciente de una vida de abnegación y de vigilancia continua de los deseos de la carne, y sois trasladados al alegre reino del espíritu, en el que manifestáis espontáneamente los frutos del espíritu en vuestra vida diaria. Los frutos del espíritu son la esencia del tipo más elevado de autocontrol agradable y ennoblecedor, e incluso lo máximo que un mortal terrestre puede alcanzar ---el verdadero dominio de sí mismo>>\footnote{\textit{Liberados de la abnegación}: Ro 8:1-17. \textit{Los frutos del espíritu}: Gl 5:22-23; Ef 5:9.}.

\section*{3. La diversión y el esparcimiento}
\par 
%\textsuperscript{(1610.4)}
\textsuperscript{143:3.1} Por esta época se desarrolló un estado de gran tensión nerviosa y emocional entre los apóstoles y sus discípulos asociados inmediatos. Aún no se habían acostumbrado a convivir y a trabajar juntos. Cada vez tenían más dificultades para mantener relaciones armoniosas con los discípulos de Juan. El contacto con los gentiles y los samaritanos era una gran prueba para estos judíos. Y además de todo esto, las recientes declaraciones de Jesús habían aumentado la alteración de su estado mental. Andrés estaba casi fuera de sí; ya no sabía qué hacer, y por eso acudió al Maestro con sus problemas y perplejidades. Cuando Jesús terminó de escuchar el relato de las dificultades de su jefe apostólico, dijo: <<Andrés, no puedes disuadir a los hombres de sus confusiones cuando llegan a un grado semejante de complicación, y cuando tantas personas con fuertes sentimientos están implicadas. No puedo hacer lo que me pides ---no deseo participar en esas dificultades sociales personales--- pero me uniré a vosotros para disfrutar de un período de tres días de descanso y esparcimiento. Dirígete a tus hermanos y anúnciales que todos vais a subir conmigo al Monte Sartaba, donde deseo descansar un día o dos>>.

\par 
%\textsuperscript{(1610.5)}
\textsuperscript{143:3.2} <<Ahora deberías dirigirte a cada uno de tus once hermanos y decirles en privado: `El Maestro desea que pasemos a solas con él un período de descanso y esparcimiento. Puesto que todos hemos experimentado recientemente mucha inquietud espiritual y tensión mental, sugiero que durante estas vacaciones no mencionemos para nada nuestras pruebas y dificultades. ¿Puedo contar contigo para que cooperes conmigo en este asunto?' Contacta así con cada uno de tus hermanos de manera privada y personal>>. Y Andrés hizo lo que el Maestro le había ordenado.

\par 
%\textsuperscript{(1611.1)}
\textsuperscript{143:3.3} Éste fue un acontecimiento maravilloso en la experiencia de cada uno de ellos; jamás olvidaron el día que subieron a la montaña. A lo largo de todo el trayecto apenas dijeron una sola palabra de sus dificultades. Al llegar a la cima de la montaña, Jesús los sentó a su alrededor mientras les decía: <<Hermanos míos, todos debéis aprender el valor del descanso y la eficacia del esparcimiento. Debéis comprender que el mejor método para resolver algunos problemas embrollados consiste en alejarse de ellos durante algún tiempo. Luego, cuando volvéis renovados por el descanso o la adoración, sois capaces de atacar vuestras dificultades con una cabeza más despejada y una mano más firme, sin mencionar un corazón más resuelto. Además, muchas veces encontraréis que el tamaño y las proporciones de vuestro problema ha disminuido mientras descansabais vuestra mente y vuestro cuerpo>>.

\par 
%\textsuperscript{(1611.2)}
\textsuperscript{143:3.4} Al día siguiente, Jesús asignó un tema de discusión a cada uno de los doce. Consagraron todo el día a los recuerdos y a hablar de asuntos no relacionados con su trabajo religioso. Se quedaron anonadados durante unos momentos cuando Jesús incluso descuidó dar las gracias ---verbalmente--- al romper el pan para su almuerzo del mediodía. Era la primera vez que lo veían omitir esta formalidad.

\par 
%\textsuperscript{(1611.3)}
\textsuperscript{143:3.5} Cuando subieron a la montaña, la cabeza de Andrés estaba llena de problemas. El corazón de Juan estaba excesivamente perplejo. El alma de Santiago estaba dolorosamente perturbada. Mateo tenía mucha necesidad de fondos debido a la estancia del grupo entre los gentiles. Pedro estaba fatigado y había estado recientemente más temperamental que de costumbre. Judas sufría uno de sus ataques periódicos de susceptibilidad y egoísmo. Simón estaba excepcionalmente trastornado debido a sus esfuerzos por conciliar su patriotismo con el amor de la fraternidad de los hombres. Felipe estaba cada vez más confundido por la manera en que se desarrollaban los acontecimientos. El humor de Natanael había disminuido desde que habían entrado en contacto con las poblaciones gentiles, y Tomás se encontraba en medio de un grave período de depresión. Sólo los gemelos estaban en un estado normal y sin inquietudes. Todos se sentían extremadamente confusos en cuanto a la manera de llevarse pacíficamente con los discípulos de Juan.

\par 
%\textsuperscript{(1611.4)}
\textsuperscript{143:3.6} Al tercer día, cuando empezaron a bajar de la montaña para regresar a su campamento, un gran cambio se había producido en ellos. Habían hecho el importante descubrimiento de que muchas perplejidades humanas no existen en realidad, de que muchas dificultades angustiosas son creadas por un miedo exagerado y producidas por un recelo desmedido. Habían aprendido que la mejor manera de tratar todas las confusiones de este tipo era alejarse de ellas; al irse, habían dejado que estos problemas se resolvieran por sí mismos.

\par 
%\textsuperscript{(1611.5)}
\textsuperscript{143:3.7} El regreso de este descanso marcó el principio de un período de relaciones considerablemente mejores con los seguidores de Juan. Una gran parte de los doce cedió realmente a la hilaridad cuando notaron el cambio del estado mental de cada uno y observaron la ausencia de irritabilidad nerviosa que disfrutaban como consecuencia de sus tres días de vacaciones, alejados de los deberes rutinarios de la vida. Siempre existe el peligro de que la monotonía de los contactos humanos multiplique considerablemente las perplejidades y aumente las dificultades.

\par 
%\textsuperscript{(1611.6)}
\textsuperscript{143:3.8} Pocos gentiles de las dos ciudades griegas de Arquelais y Fasaelis creyeron en el evangelio, pero los doce apóstoles adquirieron una valiosa experiencia con este extenso trabajo, el primero que realizaban con unas poblaciones compuestas exclusivamente de gentiles. Un lunes por la mañana hacia mediados de mes, Jesús le dijo a Andrés: <<Entremos en Samaria>>\footnote{\textit{A través de Samaria}: Jn 4:3-5.}. Y se pusieron en camino inmediatamente hacia la ciudad de Sicar, cerca del pozo de Jacob.

\section*{4. Los judíos y los samaritanos}
\par 
%\textsuperscript{(1612.1)}
\textsuperscript{143:4.1} Durante más de seiscientos años, los judíos de Judea, y más tarde también los de Galilea, habían estado enemistados con los samaritanos. Este sentimiento nocivo entre los judíos y los samaritanos surgió de la manera siguiente: Unos setecientos años a. de J.C., Sargón, rey de Asiria, aplastó una revuelta en Palestina central y se llevó como cautivos a más de veinticinco mil judíos del reino septentrional de Israel, instalando en su lugar a un número casi igual de descendientes de los cutitas, sefarvitas y amatitas. Más tarde, Asurbanipal envió también otras colonias para que vivieran en Samaria.

\par 
%\textsuperscript{(1612.2)}
\textsuperscript{143:4.2} La enemistad religiosa entre los judíos y los samaritanos databa desde el regreso de los judíos de su cautividad en Babilonia, cuando los samaritanos se esforzaron por impedir la reconstrucción de Jerusalén. Más adelante ofendieron a los judíos prestando su ayuda amistosa a los ejércitos de Alejandro. En agradecimiento por su amistad, Alejandro concedió un permiso a los samaritanos para que construyeran un templo en el Monte Gerizim, donde adoraron a Yahvé y a sus dioses tribales, y ofrecieron sacrificios muy semejantes a los de los servicios del templo de Jerusalén. Con este culto continuaron por lo menos hasta la época de los macabeos, cuando Juan Hircano destruyó su templo del Monte Gerizim. Durante sus trabajos a favor de los samaritanos después de la muerte de Jesús, el apóstol Felipe mantuvo numerosas reuniones en el lugar de este antiguo templo samaritano.

\par 
%\textsuperscript{(1612.3)}
\textsuperscript{143:4.3} Los antagonismos entre los judíos y los samaritanos eran históricos y se habían afianzado con el paso del tiempo; desde la época de Alejandro, los dos grupos se habían relacionado cada vez menos. Los doce apóstoles no se oponían a predicar en las ciudades griegas y en otras ciudades gentiles de la Decápolis y Siria, pero fue una dura prueba para su fidelidad al Maestro cuando éste les dijo: <<Entremos en Samaria>>. Sin embargo, durante el año y pico que habían pasado con Jesús, habían desarrollado una forma de lealtad personal que trascendía incluso su fe en las enseñanzas del Maestro y sus prejuicios contra los samaritanos.

\section*{5. La mujer de Sicar}
\par 
%\textsuperscript{(1612.4)}
\textsuperscript{143:5.1} Cuando el Maestro y los doce llegaron al pozo de Jacob\footnote{\textit{El pozo de Jacob}: Jn 4:6.}, Jesús estaba cansado del viaje y se detuvo cerca del pozo, mientras Felipe se llevaba a los apóstoles a Sicar para que le ayudaran a traer la comida y las tiendas, pues tenían la intención de permanecer algún tiempo en aquellos parajes\footnote{\textit{Los apóstoles van a por comida}: Jn 4:8.}. Pedro y los hijos de Zebedeo se hubieran quedado con Jesús, pero éste les rogó que se fueran con sus hermanos, diciendo: <<No temáis por mí, estos samaritanos serán amistosos; sólo nuestros hermanos, los judíos, intentan hacernos daño>>. Eran casi las seis de aquella tarde de verano, cuando Jesús se sentó cerca del pozo para esperar el regreso de los apóstoles.

\par 
%\textsuperscript{(1612.5)}
\textsuperscript{143:5.2} El agua del pozo de Jacob contenía menos minerales que la de los pozos de Sicar, y por eso era más apreciada como agua potable. Jesús tenía sed, pero no disponía de ningún medio para sacar el agua. Por eso, cuando una mujer de Sicar llegó con su cántaro y se dispuso a sacar agua del pozo, Jesús le dijo: <<Dame de beber>>\footnote{\textit{Jesús pide de beber a una mujer}: Jn 4:7.}. Esta mujer de Samaria sabía que Jesús era judío debido a su apariencia y a su vestido, y supuso que era un judío de Galilea a causa de su acento. Se llamaba Nalda y era una hermosa criatura. Se quedó muy sorprendida de que un hombre judío le hablara así al lado del pozo y le pidiera de beber, porque en aquellos tiempos no se consideraba correcto que un hombre que se preciara hablara en público con una mujer, y mucho menos que un judío conversara con una samaritana. Por eso Nalda le preguntó a Jesús: <<¿Cómo es que tú, siendo judío, me pides de beber a mí, a una mujer samaritana?>> Jesús contestó: <<En verdad te he pedido de beber, pero si solamente pudieras comprender, me pedirías un trago de agua viva>>\footnote{\textit{Jesús habla del ``agua de vida''}: Jn 4:10.}. Entonces, Nalda dijo\footnote{\textit{La contestación de Nalda}: Jn 4:9.}: <<Pero Señor, no tienes con qué sacarla, y el pozo es profundo; ¿de dónde tienes pues ese agua viva? ¿Eres más grande que nuestro padre Jacob que nos dio este pozo, del que bebió él mismo y también sus hijos y su ganado?>>\footnote{\textit{¿Cómo puedo sacar ``agua viva''?}: Jn 4:11-12.}

\par 
%\textsuperscript{(1613.1)}
\textsuperscript{143:5.3} Jesús respondió: <<Todo el que bebe de este agua volverá a tener sed, pero cualquiera que beba el agua del espíritu vivo nunca tendrá sed. Este agua viva se volverá en él un manantial refrescante que brotará hasta la vida eterna>>. Nalda dijo entonces: <<Dame de ese agua para no tener más sed, ni tener que venir hasta aquí para sacarla. Además, todo lo que una samaritana pueda recibir de un judío tan digno de elogios será un placer>>\footnote{\textit{Conversación con Nalda}: Jn 4:13-15.}.

\par 
%\textsuperscript{(1613.2)}
\textsuperscript{143:5.4} Nalda no sabía cómo interpretar la buena disposición de Jesús para hablar con ella. Veía en el rostro del Maestro la expresión de un hombre recto y santo, pero tomó su cordialidad por una familiaridad ordinaria, y malinterpretó su simbolismo como una manera de hacerle insinuaciones. Como era una mujer de moral descuidada, estaba dispuesta a volverse abiertamente coqueta cuando Jesús, mirándola directamente a los ojos, le dijo con una voz imperativa: <<Mujer, ve a buscar a tu marido y traelo hasta aquí>>. Esta orden devolvió a Nalda su sentido común. Vio que había juzgado mal la bondad del Maestro; percibió que había interpretado mal el sentido de sus palabras. Estaba asustada; empezó a darse cuenta de que estaba en presencia de una persona excepcional, y buscando a ciegas en su mente una respuesta apropiada, dijo con gran confusión: <<Pero Señor, no puedo llamar a mi marido, porque no tengo marido>>. Entonces dijo Jesús: <<Has dicho la verdad porque, aunque una vez tuviste un marido, el hombre con quien vives ahora no es tu marido. Sería mejor que dejaras de jugar con mis palabras, y buscaras el agua viva que te he ofrecido hoy>>\footnote{\textit{Conversación posterior}: Jn 4:16-18.}.

\par 
%\textsuperscript{(1613.3)}
\textsuperscript{143:5.5} Ahora Nalda había recobrado la seriedad, y su lado bueno se había despertado. No era una mujer inmoral por haberlo elegido así plenamente. Había sido repudiada cruel e injustamente por su marido y, en esta situación desesperada, había consentido en vivir como esposa de cierto griego, pero sin casarse. Nalda se sentía ahora muy avergonzada por haberle hablado a Jesús con tanta ligereza, y se dirigió al Maestro muy arrepentida, diciendo: <<Señor, me arrepiento de la manera en que te he hablado, pues percibo que eres un hombre santo o quizás un profeta>>\footnote{\textit{Un ``hombre santo'' o un profeta}: Jn 4:19.}. Y estaba a punto de solicitar al Maestro una ayuda directa y personal, cuando hizo lo que tantas personas han hecho antes y después de ella ---eludió la cuestión de la salvación personal, orientándose hacia una discusión sobre teología y filosofía. Desvió rápidamente la conversación sobre sus propias necesidades espirituales hacia un debate teológico. Señalando al Monte Gerizim, continuó: <<Nuestros padres adoraban en esta montaña, pero sin embargo, \textit{tú} dirías que el lugar donde los hombres deberían adorar se encuentra en Jerusalén; ¿cuál es pues el lugar apropiado para adorar a Dios?>>\footnote{\textit{¿Cuál es el lugar correcto para adorar a Dios?}: Jn 4:20.}

\par 
%\textsuperscript{(1613.4)}
\textsuperscript{143:5.6} Jesús percibió la tentativa del alma de la mujer por evitar un contacto directo y escrutador con su Hacedor, pero también vio que en su alma estaba presente el deseo de conocer la mejor manera de vivir. Después de todo, en el corazón de Nalda había una verdadera sed de agua viva; la trató pues con paciencia, diciéndole: <<Mujer, déjame decirte que se acerca el día en que no adorarás al Padre ni en esta montaña ni en Jerusalén. Actualmente adoráis aquello que no conocéis, una mezcla de la religión de numerosos dioses paganos y de las filosofías gentiles. Los judíos saben al menos a quien adoran; han eliminado toda confusión, concentrando su adoración en un solo Dios, Yahvé. Deberías creerme cuando digo que se acerca la hora ---e incluso ya está aquí--- en que todos los adoradores sinceros adorarán al Padre en espíritu y en verdad, porque estos son precisamente los adoradores que busca el Padre. Dios es espíritu, y aquellos que lo adoran deben adorarlo en espíritu y en verdad. Tu salvación proviene no de saber cómo deberían adorar los demás o dónde deberían hacerlo, sino de recibir en tu propio corazón este agua viva que te ofrezco en este mismo momento>>\footnote{\textit{En ninguna montaña}: Jn 4:21-22. \textit{Adorar en el espíritu y la verdad}: Jn 4:23-24.}.

\par 
%\textsuperscript{(1614.1)}
\textsuperscript{143:5.7} Pero Nalda haría un esfuerzo más por esquivar la discusión del embarazoso problema de su vida personal en la Tierra y del estado de su alma ante Dios. Una vez más recurrió a cuestiones sobre la religión en general, diciendo: <<Sí, ya sé, Señor, que Juan ha predicado sobre la venida del Convertidor, aquel que será llamado el Libertador, y que cuando venga, nos proclamará todas las cosas..>>. y Jesús, interrumpiendo a Nalda, le dijo con una seguridad sorprendente: <<Yo, que te hablo, soy esa persona>>\footnote{\textit{Jesús es el Mesías}: Jn 4:25-26.}.

\par 
%\textsuperscript{(1614.2)}
\textsuperscript{143:5.8} Ésta era la primera declaración directa, positiva y sin disfraz de su naturaleza y filiación divinas que Jesús hacía en la Tierra; y la hizo a una mujer, a una samaritana, a una mujer de reputación dudosa hasta ese momento a los ojos de los hombres. Pero los ojos divinos veían más a esta mujer como una víctima del pecado de los demás que como una pecadora por su propio deseo, y \textit{ahora} la veían como un alma humana que deseaba la salvación, la deseaba sinceramente y de todo corazón, y con eso bastaba.

\par 
%\textsuperscript{(1614.3)}
\textsuperscript{143:5.9} Cuando Nalda estaba a punto de expresar su anhelo real y personal por las cosas mejores y por una manera más noble de vivir, en el momento en que se disponía a hablar del verdadero deseo de su corazón, los doce apóstoles regresaron de Sicar\footnote{\textit{Los discípulos regresan}: Jn 4:27.}. Al encontrarse con esta escena, la de Jesús hablando tan íntimamente con esta mujer ---esta mujer samaritana, y a solas--- se quedaron más que sorprendidos. Depositaron rápidamente sus provisiones y se apartaron a un lado, sin que nadie se atreviera a censurarlo, mientras Jesús le decía a Nalda: <<Mujer, continúa tu camino; Dios te ha perdonado. De ahora en adelante vivirás una nueva vida. Has recibido el agua viva; una nueva alegría brotará dentro de tu alma, y te convertirás en una hija del Altísimo>>. Al percibir la desaprobación de los apóstoles, la mujer abandonó su cántaro y huyó hacia la ciudad.

\par 
%\textsuperscript{(1614.4)}
\textsuperscript{143:5.10} Al entrar en la ciudad, fue diciendo a todo el que encontró: <<Ve al pozo de Jacob, y date prisa, pues allí verás a un hombre que me ha dicho todo lo que he hecho. ¿Podría ser el Convertidor?>> Antes de ponerse el Sol, un gran gentío se había reunido en el pozo de Jacob para escuchar a Jesús. Y el Maestro les contó más cosas sobre el agua de la vida, el don del espíritu interior\footnote{\textit{Jesús enseña a la multitud en el pozo}: Jn 4:28-30.}.

\par 
%\textsuperscript{(1614.5)}
\textsuperscript{143:5.11} Los apóstoles nunca dejaron de escandalizarse por la buena disposición de Jesús para hablar con las mujeres, con unas mujeres de reputación dudosa, e incluso con mujeres inmorales. A Jesús le resultaba muy difícil enseñar a sus apóstoles que las mujeres, incluso las calificadas de inmorales, tienen un alma que puede escoger a Dios como Padre suyo, y convertirse así en las hijas de Dios y en candidatas a la vida eterna. Incluso diecinueve siglos más tarde, mucha gente muestra la misma aversión a captar las enseñanzas del Maestro. La misma religión cristiana ha sido construida insistentemente alrededor del hecho de la muerte de Cristo, en lugar de hacerlo alrededor de la verdad de su vida. El mundo debería interesarse más por su vida feliz, reveladora de Dios, que por su muerte trágica y triste.

\par 
%\textsuperscript{(1614.6)}
\textsuperscript{143:5.12} Al día siguiente, Nalda contó toda esta historia al apóstol Juan, pero éste nunca la reveló íntegramente a los otros apóstoles, y Jesús no habló detalladamente de esto a los doce.

\par 
%\textsuperscript{(1615.1)}
\textsuperscript{143:5.13} Nalda le contó a Juan que Jesús le había dicho <<todo lo que había hecho>>. Juan quiso muchas veces preguntarle a Jesús sobre esta charla con Nalda, pero nunca lo hizo. Jesús sólo le había dicho a Nalda una cosa sobre sí misma\footnote{\textit{Jesús sólo le dijo una cosa sobre sí misma}: Jn 4:18.}, pero su mirada clavada en sus ojos y la manera de tratarla trajeron en un instante a su mente una revisión panorámica de toda su variada vida, de tal forma que asoció toda esta autorrevelación de su vida pasada con la mirada y las palabras del Maestro. Jesús nunca le dijo que había tenido cinco maridos. Había vivido con cuatro hombres diferentes desde que su marido la había repudiado, y este hecho, junto con todo su pasado, surgió tan vívidamente en su mente cuando se dio cuenta de que Jesús era un hombre de Dios, que posteriormente le repitió a Juan que Jesús le había dicho realmente todo sobre sí misma\footnote{\textit{Jesús me dijo todo lo que había hecho}: Jn 4:29.}.

\section*{6. El renacimiento religioso en Samaria}
\par 
%\textsuperscript{(1615.2)}
\textsuperscript{143:6.1} La tarde que Nalda hizo salir a la muchedumbre de Sicar para ver a Jesús, los doce acababan de regresar con los alimentos y rogaron a Jesús que comiera con ellos en lugar de hablarle a la gente, pues llevaban todo el día sin comer y tenían hambre. Pero Jesús sabía que pronto les envolvería la oscuridad, y por ello persistió en su determinación de hablarle a la gente antes de despedirla. Cuando Andrés intentó persuadirlo para que comiera algo antes de dirigirse a la multitud\footnote{\textit{Le piden a Jesús que coma algo}: Jn 4:31.}, Jesús le dijo: <<Tengo un alimento para comer que vosotros no conocéis>>\footnote{\textit{Jesús tiene un alimento que nadie conoce}: Jn 4:32-38.}. Cuando los apóstoles escucharon esto, se dijeron entre ellos: <<¿Alguien le ha traído algo de comer? ¿Puede ser que la mujer le haya dado alimentos además de bebida?>> Cuando Jesús los escuchó hablando entre ellos, antes de dirigirse a la gente se volvió hacia los doce y les dijo: <<Mi alimento es hacer la voluntad de Aquel que me ha enviado y realizar su obra. Deberíais dejar de decir que falta tanto o tanto tiempo para la cosecha. Contemplad a esta gente que sale de una ciudad samaritana para escucharnos; os digo que los campos ya se han puesto blancos para la cosecha. El que siega recibe su salario y recoge este fruto para la vida eterna; en consecuencia, los sembradores y los segadores se regocijan juntos, porque en esto reside la verdad del refrán: `uno siembra y el otro cosecha'. Ahora os envío a cosechar algo que no habéis trabajado; otros han trabajado, y vosotros estáis a punto de formar parte de su trabajo>>. Dijo esto refiriéndose a la predicación de Juan el Bautista.

\par 
%\textsuperscript{(1615.3)}
\textsuperscript{143:6.2} Jesús y los apóstoles fueron a Sicar y predicaron dos días antes de establecer su campamento en el Monte Gerizim\footnote{\textit{Predicación en el Gerizim}: Jn 4:39-41.}. Muchos habitantes de Sicar creyeron en el evangelio y pidieron ser bautizados, pero los apóstoles de Jesús aún no bautizaban\footnote{\textit{Los apóstoles de Jesús no bautizaban}: Jn 3:22b,26b.}.

\par 
%\textsuperscript{(1615.4)}
\textsuperscript{143:6.3} La primera noche de campamento en el Monte Gerizim, los apóstoles suponían que Jesús les regañaría por su actitud hacia la mujer en el pozo de Jacob, pero él no hizo ninguna referencia a este asunto. En lugar de eso, les dio una charla memorable sobre <<las realidades que son centrales en el reino de Dios>>. En cualquier religión, es muy fácil consentir que los valores se vuelvan desproporcionados y permitir que los hechos ocupen el lugar de la verdad en la teología personal. El hecho de la cruz se volvió el centro mismo del cristianismo posterior, pero ésta no es la verdad central de la religión que se puede deducir de la vida y de las enseñanzas de Jesús de Nazaret.

\par 
%\textsuperscript{(1615.5)}
\textsuperscript{143:6.4} El tema de la enseñanza de Jesús en el Monte Gerizim fue el siguiente: deseaba que todos los hombres vieran a Dios como un Padre-amigo, así como él (Jesús) es un hermano-amigo. Les inculcó una y otra vez que el amor es la relación más grande en el mundo ---en el universo---, al igual que la verdad es la proclamación más grande de la observación de estas relaciones divinas.

\par 
%\textsuperscript{(1616.1)}
\textsuperscript{143:6.5} Jesús se manifestó tan plenamente a los samaritanos porque podía hacerlo sin peligro, y porque sabía que no volvería a visitar el corazón de Samaria para predicar el evangelio del reino.

\par 
%\textsuperscript{(1616.2)}
\textsuperscript{143:6.6} Jesús y los doce acamparon en el Monte Gerizim hasta finales de agosto. Durante el día predicaban la buena nueva del reino ---la paternidad de Dios--- a los samaritanos en las ciudades, y pasaban la noche en el campamento. El trabajo que Jesús y los doce efectuaron en estas ciudades samaritanas dio muchas almas al reino y contribuyó ampliamente a preparar el terreno para la obra maravillosa de Felipe\footnote{\textit{La obra de Felipe}: Hch 8:5-8.} en estas regiones, después de la muerte y resurrección de Jesús, y después de que los apóstoles se dispersaran hasta los confines de la Tierra debido a la persecución encarnizada contra los creyentes en Jerusalén.

\section*{7. Las enseñanzas sobre la oración y la adoración}
\par 
%\textsuperscript{(1616.3)}
\textsuperscript{143:7.1} En las conferencias nocturnas en el Monte Gerizim, Jesús enseñó muchas grandes verdades y recalcó particularmente las siguientes:

\par 
%\textsuperscript{(1616.4)}
\textsuperscript{143:7.2} La verdadera religión es la actuación de un alma individual en sus relaciones conscientes con el Creador; la religión organizada es el intento del hombre por \textit{socializar} la adoración de los practicantes individuales de la religión.

\par 
%\textsuperscript{(1616.5)}
\textsuperscript{143:7.3} La adoración ---la contemplación de lo espiritual--- debe alternar con el servicio, el contacto con la realidad material. El trabajo debería alternar con el esparcimiento; la religión debería estar equilibrada con el humor. La filosofía profunda debería ser aliviada con la poesía rítmica. El esfuerzo por vivir ---la tensión de la personalidad en el tiempo--- debería ser mitigado con el reposo de la adoración. Las sensaciones de inseguridad procedentes del miedo al aislamiento de la personalidad en el universo deberían ser contrarrestadas con la contemplación del Padre, a través de la fe, y con el intento de comprender al Supremo.

\par 
%\textsuperscript{(1616.6)}
\textsuperscript{143:7.4} La oración está destinada a hacer que el hombre piense menos y \textit{comprenda} más; no está destinada a incrementar el conocimiento, sino más bien a ampliar la perspicacia.

\par 
%\textsuperscript{(1616.7)}
\textsuperscript{143:7.5} La adoración tiene el propósito de anticiparse a la vida mejor del futuro, y luego reflejar estas nuevas significaciones espirituales sobre la vida presente. La oración es un sostén espiritual, pero la adoración es divinamente creativa.

\par 
%\textsuperscript{(1616.8)}
\textsuperscript{143:7.6} La adoración es la técnica de buscar en el \textit{Uno} la inspiración para servir a la \textit{multitud}. La adoración es la vara que mide el grado en que el alma se ha desprendido del universo material, y se ha adherido de manera simultánea y segura a las realidades espirituales de toda la creación.

\par 
%\textsuperscript{(1616.9)}
\textsuperscript{143:7.7} La oración es recordarse a sí mismo ---un pensamiento sublime; la adoración es olvidarse de sí mismo--- un superpensamiento. La adoración es una atención sin esfuerzo, el verdadero descanso ideal del alma, una forma de ejercicio espiritual sosegado.

\par 
%\textsuperscript{(1616.10)}
\textsuperscript{143:7.8} La adoración es el acto de un fragmento que se identifica con el Todo, lo finito con lo Infinito, el hijo con el Padre, el tiempo en la operación de ajustarse al ritmo de la eternidad. La adoración es el acto de la comunión personal del hijo con el Padre divino, la aceptación de unas actitudes vivificantes, creativas, fraternales y románticas por parte del alma-espíritu del hombre.

\par 
%\textsuperscript{(1616.11)}
\textsuperscript{143:7.9} Aunque los apóstoles sólo comprendieron una pequeña parte de las enseñanzas del Maestro en el campamento, otros mundos las comprendieron, y otras generaciones de la Tierra las comprenderán.