\chapter{Documento 146. La primera gira de predicación en Galilea}
\par 
%\textsuperscript{(1637.1)}
\textsuperscript{146:0.1} LA PRIMERA gira de predicación pública en Galilea empezó el domingo 18 de enero del año 28 y continuó durante unos dos meses, finalizando con el regreso a Cafarnaúm el 17 de marzo. A lo largo de esta gira, Jesús y los doce apóstoles, con la ayuda de los antiguos apóstoles de Juan, predicaron el evangelio y bautizaron a los creyentes en Rimón, Jotapata, Ramá, Zabulón, Irón, Giscala, Corazín, Madón, Caná, Naín y Endor. En estas ciudades se detuvieron para enseñar, mientras que en otras muchas ciudades más pequeñas proclamaron el evangelio del reino a medida que pasaban por ellas.

\par 
%\textsuperscript{(1637.2)}
\textsuperscript{146:0.2} Ésta fue la primera vez que Jesús permitió a sus asociados predicar sin restricciones. En el transcurso de esta gira, sólo les hizo advertencias en tres ocasiones; les recomendó que permanecieran lejos de Nazaret y que fueran discretos cuando pasaran por Cafarnaúm y Tiberiades. Para los apóstoles fue una causa de gran satisfacción sentir que por fin tenían la libertad de predicar y enseñar sin restricciones, y se lanzaron con una gran seriedad y alegría a la tarea de predicar el evangelio, atender a los enfermos y bautizar a los creyentes.

\section*{1. La predicación en Rimón}
\par 
%\textsuperscript{(1637.3)}
\textsuperscript{146:1.1} La pequeña ciudad de Rimón había estado dedicada en otro tiempo a la adoración de Ramán, un dios babilónico del aire. Las creencias de los rimonitas contenían todavía muchas enseñanzas babilónicas primitivas y enseñanzas posteriores de Zoroastro; por esta razón, Jesús y los veinticuatro consagraron mucho tiempo a la tarea de indicarles claramente la diferencia entre estas antiguas creencias y el nuevo evangelio del reino. Pedro predicó aquí sobre «Aarón y el becerro de oro»\footnote{\textit{Aarón y el becerro de oro}: Ex 32:1-35; Dt 9:16-21.}, uno de los grandes sermones del principio de su carrera.

\par 
%\textsuperscript{(1637.4)}
\textsuperscript{146:1.2} Aunque muchos ciudadanos de Rimón se convirtieron en creyentes de las enseñanzas de Jesús, en años posteriores causaron grandes dificultades a sus hermanos. En el corto espacio de una sola vida, es difícil convertir a unos adoradores de la naturaleza a la plena comunión de la adoración de un ideal espiritual.

\par 
%\textsuperscript{(1637.5)}
\textsuperscript{146:1.3} Muchos de los mejores conceptos babilónicos y persas sobre la luz y las tinieblas, el bien y el mal, el tiempo y la eternidad, fueron incorporados más tarde en las doctrinas del llamado cristianismo; esta inclusión hizo que los pueblos del Cercano Oriente aceptaran más rápidamente las enseñanzas cristianas. De la misma manera, la inclusión de muchas teorías de Platón sobre el espíritu ideal o los arquetipos invisibles de todas las cosas visibles y materiales, tal como Filón las adaptó más tarde a la teología hebrea, hizo que las enseñanzas cristianas de Pablo fueran aceptadas más fácilmente por los griegos occidentales.

\par 
%\textsuperscript{(1637.6)}
\textsuperscript{146:1.4} Fue en Rimón donde Todán escuchó por primera vez el evangelio del reino, y más tarde llevó este mensaje a Mesopotamia y mucho más allá. Fue uno de los primeros que predicó la buena nueva a los habitantes de más allá del Éufrates.

\section*{2. En Jotapata}
\par 
%\textsuperscript{(1638.1)}
\textsuperscript{146:2.1} Aunque la gente común y corriente de Jotapata escuchó con gusto a Jesús y sus apóstoles, y muchas personas aceptaron el evangelio del reino, lo más sobresaliente de esta misión en Jotapata fue el discurso de Jesús a los veinticuatro durante la segunda noche de su estancia en esta pequeña ciudad. Natanael tenía ideas confusas sobre las enseñanzas del Maestro respecto a la oración, la acción de gracias y la adoración. En respuesta a su pregunta, Jesús habló muy extensamente para explicar mejor su enseñanza. Resumido en un lenguaje moderno, este discurso se puede presentar para hacer hincapié en los puntos siguientes:

\par 
%\textsuperscript{(1638.2)}
\textsuperscript{146:2.2} 1. Cuando el corazón del hombre alberga una consideración consciente y persistente por la iniquidad, se va destruyendo gradualmente la conexión que el alma humana ha establecido, mediante la oración, con los circuitos espirituales de comunicación entre el hombre y su Hacedor\footnote{\textit{El corazón inicuo destruye la conexión de la oración}: Jer 17:9.}. Naturalmente, Dios escucha la súplica de su hijo, pero cuando el corazón humano alberga los conceptos de la iniquidad de manera deliberada y permanente, la comunión personal entre el hijo terrenal y su Padre celestial se pierde gradualmente.

\par 
%\textsuperscript{(1638.3)}
\textsuperscript{146:2.3} 2. Una oración que es incompatible con las leyes de Dios conocidas y establecidas, es una abominación para las Deidades del Paraíso. Si el hombre no quiere escuchar a los Dioses que hablan a su creación mediante las leyes del espíritu, de la mente y de la materia, un acto así de desprecio deliberado y consciente por parte de la criatura impide que las personalidades espirituales presten atención a las súplicas personales de esos mortales anárquicos y desobedientes. Jesús citó a sus apóstoles las palabras del profeta Zacarías: «Pero se negaron a escuchar, se volvieron de espaldas y se taparon los oídos para no oír. Sí, endurecieron su corazón como una piedra, para no tener que oír mi ley ni las palabras que yo les enviaba por medio de mi espíritu a través de los profetas; por eso, los resultados de sus malos pensamientos recaen como una gran ira sobre sus cabezas culpables. Y sucedió que gritaron para recibir misericordia, pero ningún oído estaba abierto para escucharlos»\footnote{\textit{Endurecieron su corazón}: Zac 7:11-13.}. Jesús citó a continuación el proverbio del sabio que decía: «Si alguien desvía su oído para no escuchar la ley divina, incluso su oración será una abominación»\footnote{\textit{Desviaron su oído y su oración fue una abominación}: Pr 28:9.}.

\par 
%\textsuperscript{(1638.4)}
\textsuperscript{146:2.4} 3. Al abrir el terminal humano del canal de comunicación entre Dios y el hombre, los mortales ponen inmediatamente a su disposición la corriente constante del ministerio divino para con las criaturas de los mundos. Cuando el hombre escucha hablar al espíritu de Dios dentro de su corazón humano, en esa experiencia se encuentra inherente el hecho de que Dios escucha simultáneamente la oración de ese hombre. El perdón de los pecados\footnote{\textit{El perdón}: Eclo 28:1-5; Mt 6:12,14-15; 18:21-35; Mc 11:25-26; Lc 6:37b; 11:4a; 17:3-4; Ef 4:32; 1 Jn 2:12.} también funciona de esta misma manera infalible. El Padre que está en los cielos os ha perdonado incluso antes de que hayáis pensado en pedírselo, pero dicho perdón no está disponible en vuestra experiencia religiosa personal hasta el momento en que perdonáis a vuestros semejantes. El perdón de Dios no está condicionado, de \textit{hecho}, por vuestro perdón a vuestros semejantes, pero como \textit{experiencia} está sometido exactamente a esa condición. Este hecho de la sincronización entre el perdón divino y el perdón humano estaba reconocido e incluido en la oración que Jesús enseñó a los apóstoles.

\par 
%\textsuperscript{(1638.5)}
\textsuperscript{146:2.5} 4. Existe una ley fundamental de justicia en el universo que la misericordia no tiene poder para burlar. Las glorias desinteresadas del Paraíso no pueden ser recibidas por una criatura totalmente egoísta de los reinos del tiempo y del espacio. Ni siquiera el amor infinito de Dios puede imponer la salvación de la supervivencia eterna a una criatura mortal que no escoge sobrevivir. La misericordia dispone de una gran libertad de donación, pero después de todo, hay mandatos de la justicia que ni siquiera el amor combinado con la misericordia pueden revocar eficazmente. Jesús citó de nuevo las escrituras hebreas: «He llamado y habéis rehusado escuchar; he tendido mi mano, pero nadie ha prestado atención. Habéis despreciado todos mis consejos, y habéis rechazado mi desaprobación; debido a esta actitud rebelde, es inevitable que cuando me invoquéis no recibáis respuesta. Como habéis rechazado el camino de la vida, podéis buscarme con diligencia en vuestros momentos de sufrimiento, pero no me encontraréis»\footnote{\textit{Efecto de rechaza a Dios}: Pr 1:24-28.}.

\par 
%\textsuperscript{(1639.1)}
\textsuperscript{146:2.6} 5. Los que quieran recibir misericordia, deberán mostrar misericordia; no juzguéis, para no ser juzgados\footnote{\textit{No juzguéis, para no ser juzgados}: Mt 7:1-2.}. Con el espíritu con que juzguéis a los demás también seréis juzgados. La misericordia no anula totalmente la justicia universal. Al final será cierto que: «Cualquiera que cierra sus oídos al lamento del pobre, también pedirá ayuda algún día, y nadie lo escuchará»\footnote{\textit{A quien desoiga el lamento del podre}: Pr 21:13.}. La sinceridad de cualquier oración es la garantía de que será escuchada; la sabiduría espiritual y la compatibilidad universal de cualquier petición determinan el momento, la manera y el grado de la respuesta. Un padre sabio no responde \textit{literalmente} a las oraciones tontas de sus hijos ignorantes e inexpertos, aunque dichos hijos puedan obtener mucho placer y una satisfacción real para su alma efectuando ese tipo de peticiones absurdas.

\par 
%\textsuperscript{(1639.2)}
\textsuperscript{146:2.7} 6. Cuando estéis totalmente consagrados a hacer la voluntad del Padre que está en los cielos, todas vuestras súplicas serán contestadas\footnote{\textit{Cuándo son contestadas las súplicas}: Mt 7:7-11; 21:22; Mc 11:24-26; Lc 11:9-13; Jn 14:13-14; 15:7,16; 16:23-24.}, porque vuestras oraciones estarán plenamente de acuerdo con la voluntad del Padre, y la voluntad del Padre se manifiesta constantemente en todo su inmenso universo. Aquello que un verdadero hijo desea y el Padre infinito lo quiere, EXISTE. Una oración así no puede permanecer sin respuesta, y es posible que ningún otro tipo de petición pueda ser contestada plenamente.

\par 
%\textsuperscript{(1639.3)}
\textsuperscript{146:2.8} 7. El grito del justo es el acto de fe del hijo de Dios que abre la puerta del almacén de bondad, de verdad y de misericordia del Padre; estos dones preciados han estado esperando mucho tiempo a que el hijo se acerque y se los apropie personalmente. La oración no cambia la actitud divina hacia el hombre, pero sí cambia la actitud del hombre hacia el Padre invariable. Es el \textit{móvil} de la oración lo que le da el derecho de acceso al oído divino, y no el estado social, económico o religioso exterior de aquel que ora.

\par 
%\textsuperscript{(1639.4)}
\textsuperscript{146:2.9} 8. La oración no se puede emplear para evitar las demoras del tiempo ni para trascender los obstáculos del espacio. La oración no es una técnica diseñada para engrandecer el yo ni para conseguir una ventaja injusta sobre los semejantes. Un alma totalmente egoísta es incapaz de orar en el verdadero sentido de la palabra. Jesús dijo: «Que vuestra delicia suprema esté en el carácter de Dios, y él os concederá con seguridad los sinceros deseos de vuestro corazón»\footnote{\textit{Delicia en Dios}: Sal 37:4.}. «Encomendad vuestro camino al Señor; confiad en él, y él actuará»\footnote{\textit{Confiad en Dios}: Sal 37:5.}. «Porque el Señor escucha el lamento del indigente y atenderá la oración del desamparado»\footnote{\textit{Dios escucha el lamento del necesitado}: Sal 72:12; 102:17.}.

\par 
%\textsuperscript{(1639.5)}
\textsuperscript{146:2.10} 9. «Yo he salido del Padre; por lo tanto, si alguna vez tenéis dudas sobre lo que debéis pedirle al Padre, pedidlo en mi nombre, y yo presentaré vuestra petición de acuerdo con vuestras necesidades y deseos reales y en conformidad con la voluntad de mi Padre». Guardaos contra el grave peligro de volveros egocéntricos en vuestras oraciones. Evitad orar mucho por vosotros mismos; orad más por el progreso espiritual de vuestros hermanos. Evitad las oraciones materialistas; orad en espíritu y por la abundancia de los dones del espíritu\footnote{\textit{Orad por necesidades y deseos reales}: Jn 14:13-14.}.

\par 
%\textsuperscript{(1639.6)}
\textsuperscript{146:2.11} 10. Cuando oréis por los enfermos y los afligidos, no esperéis que vuestras súplicas reemplacen los cuidados afectuosos e inteligentes que necesitan esos afligidos. Orad por el bienestar de vuestras familias, amigos y compañeros, pero orad especialmente por aquellos que os maldicen, y efectuad súplicas afectuosas por aquellos que os persiguen\footnote{\textit{Orad por vuestros enemigos}: Mt 5:44; Lc 6:28.}. «En cuanto al momento en que debéis orar, no os lo indicaré. Sólo el espíritu que reside en vosotros puede incitaros a manifestar las peticiones que expresen vuestra relación interior con el Padre de los espíritus»\footnote{\textit{Padre de los espíritus}: Heb 12:9.}.

\par 
%\textsuperscript{(1640.1)}
\textsuperscript{146:2.12} 11. Mucha gente sólo recurre a la oración cuando tiene dificultades. Una práctica así es irreflexiva y descaminada. Es verdad que hacéis bien en orar cuando estáis agobiados, pero también deberíais acordaros de hablar con vuestro Padre como un hijo, incluso cuando todo va bien para vuestra alma. Que vuestras súplicas reales sean siempre en secreto\footnote{\textit{Orad en secreto}: Mt 6:6.}. No permitáis que los hombres escuchen vuestras oraciones personales. Las oraciones de acción de gracias son apropiadas para los grupos de adoradores, pero la oración del alma es un asunto personal. Sólo existe una forma de oración que es apropiada para todos los hijos de Dios, y es: «Sin embargo, que se haga tu voluntad».

\par 
%\textsuperscript{(1640.2)}
\textsuperscript{146:2.13} 12. Todos los que creen en este evangelio deberían orar sinceramente por la expansión del reino de los cielos. De todas las oraciones de las Escrituras hebreas, Jesús hizo un comentario muy favorable sobre esta súplica del salmista: «Crea en mí un corazón limpio, oh Dios, y renueva un espíritu recto dentro de mí. Purifícame de los pecados secretos y preserva a tu servidor de las transgresiones presuntuosas»\footnote{\textit{Crea un corazón limpio}: Sal 51:10. \textit{Purifícame de mis pecados}: Sal 19:12-13.}. Jesús hizo un extenso comentario sobre la relación entre la oración y el lenguaje descuidado y ofensivo, citando el pasaje: «Oh Señor, pon un vigilante delante de mi boca, y guarda la puerta de mis labios»\footnote{\textit{Pon un vigilante en mi boca}: Sal 141:3.}. Jesús dijo: «La lengua humana es un órgano que muy pocos hombres saben domar; pero el espíritu interior puede transformar este miembro indómito en una suave voz de tolerancia y en un ministro inspirador de misericordia»\footnote{\textit{Lengua indómita}: Stg 3:8.}.

\par 
%\textsuperscript{(1640.3)}
\textsuperscript{146:2.14} 13. Jesús enseñó que la oración para recibir la guía divina en el sendero de la vida terrestre seguía en importancia a la súplica para conocer la voluntad del Padre. Esto significa, en realidad, orar para obtener la sabiduría divina. Jesús no enseñó nunca que pudieran obtenerse conocimientos humanos y habilidades especiales por medio de la oración. Pero sí enseñó que la oración es un factor en la ampliación de nuestra capacidad para recibir la presencia del espíritu divino. Cuando Jesús enseñó a sus asociados que oraran en espíritu y en verdad\footnote{\textit{Orar en espíritu y en verdad}: Jn 4:24.}, explicó que se refería a que oraran con sinceridad y de acuerdo con las luces que poseía cada cual, que oraran de todo corazón y con inteligencia, seriedad y constancia.

\par 
%\textsuperscript{(1640.4)}
\textsuperscript{146:2.15} 14. Jesús previno a sus discípulos contra la idea de que sus oraciones serían más eficaces utilizando repeticiones adornadas\footnote{\textit{Oraciones elocuentes y adornadas}: Mt 6:7-8a.}, una fraseología elocuente, el ayuno, la penitencia o los sacrificios. Pero sí exhortó a sus creyentes a que emplearan la oración como un medio de elevarse a la verdadera adoración a través de la acción de gracias. Jesús deploraba que se encontrara tan poco espíritu de acción de gracias en las oraciones y el culto de sus seguidores. En esta ocasión citó las Escrituras, diciendo: «Es bueno dar gracias al Señor y cantar alabanzas al nombre del Altísimo, reconocer su misericordia cada mañana y su fidelidad cada noche, porque Dios me ha hecho feliz con su obra. Daré gracias por todas las cosas en conformidad con la voluntad de Dios»\footnote{\textit{Es bueno dar gracias al Señor}: Sal 92:1-2. \textit{Dios me ha hecho feliz con su obra}: Sal 92:4. \textit{Daré gracias por todas las cosas}: 1 Ts 5:18.}.

\par 
%\textsuperscript{(1640.5)}
\textsuperscript{146:2.16} 15. Jesús dijo a continuación: «No os preocupéis constantemente por vuestras necesidades ordinarias. No sintáis aprensión por los problemas de vuestra existencia terrestre; en todas estas cosas, mediante la oración y la súplica, con un espíritu sincero de acción de gracias, exponed vuestras necesidades ante vuestro Padre que está en los cielos»\footnote{\textit{No os preocupéis constantemente}: Flp 4:6.}. Luego citó de las Escrituras: «Alabaré el nombre de Dios con un cántico y lo ensalzaré con mi acción de gracias. Esto agradará más al Señor que el sacrificio de un buey o de un becerro con cuernos y pezuñas»\footnote{\textit{Alabaré a Dios con un cántico}: Sal 69:30-31.}.

\par 
%\textsuperscript{(1641.1)}
\textsuperscript{146:2.17} 16. Jesús enseñó a sus seguidores que, después de haber hecho sus oraciones al Padre, deberían permanecer algún tiempo en un estado de receptividad silenciosa para proporcionar al espíritu interior las mejores posibilidades de hablarle al alma atenta. El espíritu del Padre le habla mejor al hombre cuando la mente humana se encuentra en una actitud de verdadera adoración. Adoramos a Dios con la ayuda del espíritu interior del Padre y mediante la iluminación de la mente humana a través del ministerio de la verdad. Jesús enseñó que la adoración hace al adorador cada vez más semejante al ser que adora. La adoración es una experiencia transformadora por medio de la cual el finito se acerca gradualmente a la presencia del Infinito, y finalmente la alcanza.

\par 
%\textsuperscript{(1641.2)}
\textsuperscript{146:2.18} Jesús contó a sus apóstoles otras muchas verdades sobre la comunión del hombre con Dios, pero pocos de ellos pudieron abarcar plenamente su enseñanza.

\section*{3. La parada en Ramá}
\par 
%\textsuperscript{(1641.3)}
\textsuperscript{146:3.1} Jesús tuvo en Ramá el debate memorable con el anciano filósofo griego que enseñaba que la ciencia y la filosofía eran suficientes para satisfacer las necesidades de la experiencia humana. Jesús escuchó con paciencia y simpatía a este educador griego, aceptando la verdad de muchas de las cosas que dijo. Pero cuando terminó de hablar, Jesús le señaló que en su examen de la existencia humana había omitido explicar «de dónde, por qué, y hacia dónde», y añadió: «Allí donde tú terminas, empezamos nosotros. La religión es una revelación al alma humana que trata con unas realidades espirituales que la mente sola nunca podría descubrir ni sondear por completo. Los esfuerzos intelectuales pueden revelar los hechos de la vida, pero el evangelio del reino descubre las \textit{verdades} de la existencia. Tú has hablado de las sombras materiales de la verdad; ¿quieres escucharme ahora mientras te hablo de las realidades eternas y espirituales que proyectan esas sombras temporales transitorias de los hechos materiales de la existencia mortal?» Durante más de una hora, Jesús enseñó a este griego las verdades salvadoras del evangelio del reino. Al anciano filósofo le conmovió el modo de acercarse del Maestro, y como era sinceramente honrado de corazón, creyó rápidamente en este evangelio de salvación.

\par 
%\textsuperscript{(1641.4)}
\textsuperscript{146:3.2} Los apóstoles estaban un poco desconcertados por la manera evidente con que Jesús aprobaba muchas de las proposiciones del griego, pero Jesús les dijo más tarde en privado: «Hijos míos, no os asombréis por mi tolerancia con la filosofía del griego. La certidumbre interior verdadera y auténtica no teme en absoluto el análisis exterior, ni la verdad se resiente por una crítica honesta. No deberíais olvidar nunca que la intolerancia es la máscara que cubre las dudas que se mantienen en secreto sobre la autenticidad de las creencias que uno tiene. A nadie le inquieta en ningún momento la actitud de su vecino, cuando tiene una confianza total en la verdad de lo que cree de todo corazón. El coraje es la confianza completamente honesta en las cosas que uno profesa creer. Los hombres sinceros no temen el examen crítico de sus verdaderas convicciones y de sus nobles ideales».

\par 
%\textsuperscript{(1641.5)}
\textsuperscript{146:3.3} La segunda noche en Ramá, Tomás le hizo a Jesús la pregunta siguiente: «Maestro, un nuevo creyente en tus enseñanzas ¿cómo puede saber realmente, estar realmente seguro, de la verdad de este evangelio del reino?»

\par 
%\textsuperscript{(1641.6)}
\textsuperscript{146:3.4} Jesús le dijo a Tomás: «Tu seguridad de que has entrado en la familia del reino del Padre y de que sobrevivirás eternamente con los hijos del reino es enteramente un asunto de experiencia personal ---de fe en la palabra de la verdad. La seguridad espiritual equivale a tu experiencia religiosa personal con las realidades eternas de la verdad divina; dicho de otra manera, es igual a tu comprensión inteligente de las realidades de la verdad, más tu fe espiritual y menos tus dudas sinceras».

\par 
%\textsuperscript{(1642.1)}
\textsuperscript{146:3.5} «El Hijo está dotado por naturaleza de la vida del Padre. Como habéis sido dotados del espíritu viviente del Padre, sois por tanto hijos de Dios. Sobrevivís a vuestra vida en el mundo material de la carne porque estáis identificados con el espíritu viviente del Padre, el don de la vida eterna. En verdad, muchas personas tenían esta vida antes de que yo viniera del Padre, y muchos más han recibido este espíritu porque han creído en mis palabras; pero os aseguro que, cuando yo regrese al Padre, él enviará su espíritu al corazón de todos los hombres»\footnote{\textit{Hijo dotado de la vida del Padre}: Jn 5:26. \textit{Espíritu viviente del Padre}: Job 32:8,18; Is 63:10-11; Ez 37:14; Mt 10:20; Lc 17:21; Jn 17:21-23; Ro 8:9-11; 1 Co 3:16-17; 6:19; 2 Co 6:16; Gl 2:20; 1 Jn 3:24; 4:12-15; Ap 21:3. \textit{Los creyentes reciben el espíritu}: Jn 5:24.}.

\par 
%\textsuperscript{(1642.2)}
\textsuperscript{146:3.6} «Aunque no podéis observar al espíritu divino trabajando en vuestra mente, existe un método práctico para descubrir hasta qué punto habéis cedido el control de los poderes de vuestra alma a la enseñanza y a la dirección de este espíritu interior del Padre celestial: es el grado de vuestro amor por vuestros semejantes humanos. Este espíritu del Padre participa del amor del Padre, y a medida que domina al hombre, lo conduce infaliblemente en la dirección de la adoración divina y de la consideración afectuosa por los semejantes. Al principio, creéis que sois los hijos de Dios porque mi enseñanza os ha hecho más conscientes de las directrices internas de la presencia de nuestro Padre que reside en vosotros; pero el Espíritu de la Verdad será derramado dentro de poco sobre todo el género humano, y vivirá entre los hombres y los enseñará a todos, como yo ahora vivo entre vosotros y os digo las palabras de la verdad. Este Espíritu de la Verdad, que habla para los dones espirituales de vuestra alma, os ayudará a saber que sois los hijos de Dios. Dará testimonio de manera infalible con la presencia interior del Padre, vuestro espíritu, que entonces residirá en todos los hombres, como ahora reside en algunos, y os dirá que sois en realidad los hijos de Dios»\footnote{\textit{El espíritu de Dios ayuda a conocer la filiación}: Ro 8:16. \textit{El espíritu interior confirma}: Ro 8:14. \textit{El Espíritu de la Verdad}: Ez 11:19; 18:31; 36:26-27; Jl 2:28-29; Lc 24:49; Jn 7:39; 14:16-18,23,26; 15:4,26; 16:7-8,13-14; 17:21-23; Hch 1:5,8a; 2:1-4,16-18; 2:33; 2 Co 13:5; Gl 2:20; 4:6; Ef 1:13; 4:30; 1 Jn 4:12-15.}.

\par 
%\textsuperscript{(1642.3)}
\textsuperscript{146:3.7} «Todo hijo terrestre que sigue las directrices de este espíritu terminará conociendo la voluntad de Dios, y aquel que se abandona a la voluntad de mi Padre vivirá para siempre. El camino que va de la vida terrestre al estado eterno no se os ha indicado claramente; sin embargo hay un camino, siempre lo ha habido, y yo he venido para hacerlo nuevo y viviente. Aquel que entra en el reino ya tiene la vida eterna ---no perecerá nunca. Pero muchas de estas cosas las comprenderéis mejor cuando yo haya regresado al Padre, y seáis capaces de contemplar retrospectivamente vuestras experiencias de ahora»\footnote{\textit{Quien siga al espíritu vivirá para siempre}: Jn 10:27-28; 17:2-3. \textit{La vida eterna}: Dn 12:2; Mt 19:16,29; 25:46; Mc 10:17,30; Lc 10:25; 18:18,30; Jn 3:15-16,36; 4:14,36; 5:24,39; 6:27,40,47; 6:54,68; 8:51-52; 10:28; 11:25-26; 12:25,50; 17:2-3; Hch 13:46-48; Ro 2:7; 5:21; 6:22-23; Gl 6:8; 1 Ti 1:16; 6:12,19; Tit 1:2; 3:7; 1 Jn 1:2; 2:25; 3:15; 5:11,13,20; Jud 1:21; Ap 22:5. \textit{Un camino nuevo y viviente}: Heb 10:20.}.

\par 
%\textsuperscript{(1642.4)}
\textsuperscript{146:3.8} Todos los que escucharon estas palabras bienaventuradas se llenaron de regocijo. Las enseñanzas judías sobre la supervivencia de los justos eran confusas e inciertas, y para los discípulos de Jesús resultaba vivificante e inspirador escuchar estas palabras tan precisas y positivas, asegurando la supervivencia eterna para todos los creyentes sinceros.

\par 
%\textsuperscript{(1642.5)}
\textsuperscript{146:3.9} Los apóstoles continuaron predicando y bautizando a los creyentes, conservando la costumbre de ir de casa en casa para confortar a los deprimidos y atender a los enfermos y afligidos. La organización apostólica se había ampliado, en el sentido de que cada apóstol de Jesús tenía ahora como asociado a un apóstol de Juan; Abner era el asociado de Andrés; y este plan prevaleció hasta que bajaron a Jerusalén para la Pascua siguiente.

\par 
%\textsuperscript{(1642.6)}
\textsuperscript{146:3.10} Durante su estancia en Zabulón, la instrucción especial que Jesús les dio consistió principalmente en nuevas discusiones sobre las obligaciones recíprocas en el reino, y englobó una enseñanza destinada a clarificar las diferencias entre la experiencia religiosa personal y las buenas relaciones en las obligaciones religiosas sociales. Ésta fue una de las pocas veces que el Maestro discurrió sobre los aspectos sociales de la religión. A lo largo de toda su vida en la Tierra, Jesús dio a sus discípulos muy pocas instrucciones sobre la socialización de la religión.

\par 
%\textsuperscript{(1643.1)}
\textsuperscript{146:3.11} La población de Zabulón era de raza mixta, ni judía ni gentil, y pocos de ellos creyeron realmente en Jesús, a pesar de que habían oído hablar de la curación de los enfermos en Cafarnaúm.

\section*{4. El evangelio en Irón}
\par 
%\textsuperscript{(1643.2)}
\textsuperscript{146:4.1} En Irón, como también en muchas de las ciudades más pequeñas de Galilea y Judea, había una sinagoga, y durante los primeros tiempos de su ministerio, Jesús tenía la costumbre de hablar los sábados en estas sinagogas. A veces hablaba durante los oficios de la mañana, y Pedro o uno de los otros apóstoles predicaba por la tarde. Jesús y los apóstoles también enseñaban y predicaban a menudo en las asambleas vespertinas de la sinagoga durante los días de la semana. Aunque los jefes religiosos de Jerusalén eran cada vez más hostiles hacia Jesús, no ejercían ningún control directo sobre las sinagogas exteriores a la ciudad. Sólo en una época más tardía del ministerio público de Jesús, consiguieron crear un sentimiento tan generalizado en contra de él que provocaron casi el cierre total de las sinagogas a su enseñanza. Pero en estos momentos, todas las sinagogas de Galilea y Judea estaban abiertas para él\footnote{\textit{Sinagogas abiertas}: Mc 1:39.}.

\par 
%\textsuperscript{(1643.3)}
\textsuperscript{146:4.2} En Irón se encontraban unas minas muy importantes para aquella época, y como Jesús nunca había compartido la vida de los mineros, durante su estancia en Irón pasó la mayor parte de su tiempo en las minas. Mientras los apóstoles visitaban los hogares y predicaban en los lugares públicos, Jesús trabajaba en las minas con estos obreros subterráneos. La fama de Jesús como sanador se había propagado hasta este pueblo remoto, y muchos enfermos y afligidos buscaron su ayuda; la gente se benefició ampliamente de su ministerio curativo. Pero el Maestro no efectuó, en ninguno de estos casos, un pretendido milagro de curación, exceptuando el del leproso.

\par 
%\textsuperscript{(1643.4)}
\textsuperscript{146:4.3} Al final de la tarde del tercer día en Irón, cuando Jesús regresaba de las minas, pasó por casualidad por una angosta calle lateral en dirección a su alojamiento. Al acercarse a la choza miserable de cierto leproso, el afectado, que había oído hablar de la fama de Jesús como sanador, se atrevió a abordarlo cuando pasaba por su puerta, y se arrodilló delante de él, diciendo: «Señor, si tan sólo quisieras, podrías purificarme. He oído el mensaje de tus instructores y quisiera entrar en el reino si pudiera ser purificado». El leproso se expresó de esta manera porque, entre los judíos, a los leprosos se les prohibía incluso asistir a la sinagoga o practicar otro tipo de culto en público. Este hombre creía realmente que no sería recibido en el reino venidero a menos que pudiera curarse de su lepra. Cuando Jesús lo vio así de afligido y escuchó sus palabras impregnadas de fe, su corazón humano se conmovió y su mente divina se enterneció de compasión. Mientras Jesús lo contemplaba, el hombre se echó de bruces y lo adoró. Entonces, el Maestro alargó su mano, lo tocó y le dijo: «Sí quiero ---queda purificado». Y el hombre se curó de inmediato; la lepra había dejado de afligirlo\footnote{\textit{Curación del leproso}: Mt 8:1-3; Mc 1:40-42; Lc 5:12-13.}.

\par 
%\textsuperscript{(1643.5)}
\textsuperscript{146:4.4} Cuando Jesús hubo levantado al hombre del suelo, le encargó: «Cuida de no hablarle a nadie de tu curación, sino más bien dirígete tranquilamente a tus asuntos, preséntate ante el sacerdote y ofrece los sacrificios ordenados por Moisés en testimonio de tu purificación»\footnote{\textit{«No hables de tu curación»}: Mt 8:4.}. Pero este hombre no hizo lo que Jesús le había indicado. En lugar de eso, empezó a anunciar por toda la localidad que Jesús lo había curado de su lepra\footnote{\textit{El leproso habla}: Mc 1:43-45; Lc 5:14-15.}, y como todo el pueblo lo conocía, la gente pudo ver claramente que había sido librado de su enfermedad. No fue a ver a los sacerdotes como Jesús le había recomendado. Como consecuencia de haber divulgado la noticia de que Jesús lo había curado, el Maestro fue tan asediado por los enfermos que se vio obligado a levantarse temprano al día siguiente y dejar el pueblo. Aunque Jesús no volvió a entrar en la ciudad, permaneció dos días en las afueras cerca de las minas, donde continuó enseñando más cosas a los mineros creyentes sobre el evangelio del reino\footnote{\textit{El evangelio del reino}: Mt 4:23; 9:35; 24:14; Mc 1:14-15.}.

\par 
%\textsuperscript{(1644.1)}
\textsuperscript{146:4.5} Esta purificación del leproso era el primer supuesto milagro que Jesús había realizado intencional y deliberadamente hasta ese momento. Y se trataba de un auténtico caso de lepra.

\par 
%\textsuperscript{(1644.2)}
\textsuperscript{146:4.6} Desde Irón fueron a Giscala, donde pasaron dos días proclamando el evangelio, y luego partieron hacia Corazín, donde estuvieron casi una semana predicando la buena nueva, pero en esta ciudad fueron incapaces de conseguir muchos creyentes para el reino. En ningún lugar donde Jesús había enseñado había encontrado un rechazo tan general de su mensaje. La estancia en Corazín fue muy deprimente para la mayoría de los apóstoles; Andrés y Abner tuvieron muchas dificultades para levantar el ánimo de sus asociados. Así pues, atravesaron tranquilamente Cafarnaúm, y continuaron hasta el pueblo de Madón, donde no tuvieron mucho más éxito. En la mente de la mayoría de los apóstoles prevalecía la idea de que su falta de éxito en estas ciudades que habían visitado tan recientemente se debía a la insistencia de Jesús de que, en sus enseñanzas y predicaciones, se abstuvieran de hablar de él como sanador. ¡Cuánto hubieran deseado que purificara a otro leproso o que manifestara su poder de alguna otra manera para atraer la atención de la gente! Pero el Maestro se mantuvo impasible ante sus ardientes deseos.

\section*{5. De vuelta en Caná}
\par 
%\textsuperscript{(1644.3)}
\textsuperscript{146:5.1} El grupo apostólico se alegró enormemente cuando Jesús anunció: «Mañana iremos a Caná»\footnote{\textit{A Galilea, a Caná}: Jn 4:43.}. Sabían que en Caná los escucharían con simpatía, porque Jesús era bien conocido allí. Iban prosperando en su trabajo de atraer a la gente al reino cuando, al tercer día, cierto ciudadano destacado de Cafarnaúm, llamado Tito, se presentó en Caná; era un creyente a medias y su hijo estaba gravemente enfermo\footnote{\textit{El hijo enfermo del noble}: Jn 4:46.}. Había oído que Jesús estaba en Caná, por lo que se apresuró a ir a verlo. Los creyentes de Cafarnaúm consideraban que Jesús podía curar cualquier enfermedad.

\par 
%\textsuperscript{(1644.4)}
\textsuperscript{146:5.2} Cuando este noble hubo localizado a Jesús en Caná, le suplicó que fuera rápidamente a Cafarnaúm para curar a su hijo afligido. Mientras los apóstoles permanecían cerca con la respiración cortada por la expectación, Jesús, mirando al padre del muchacho enfermo, dijo: «¿Cuánto tiempo seré indulgente con vosotros? El poder de Dios está en medio de vosotros, pero a menos que veáis signos y contempléis prodigios, os negáis a creer». Pero el noble le suplicó a Jesús, diciendo: «Señor mío, yo sí creo, pero ven antes de que mi hijo perezca, porque cuando lo dejé ya estaba a punto de morir». Después de inclinar la cabeza unos momentos, en una meditación silenciosa, Jesús dijo súbitamente: «Vuelve a tu hogar; tu hijo vivirá». Tito creyó en la palabra de Jesús y se apresuró a regresar a Cafarnaúm. Cuando iba de vuelta, sus sirvientes salieron a su encuentro, diciendo: «Regocíjate, pues tu hijo ha mejorado ---vive». Entonces Tito les preguntó a qué hora había empezado a mejorar el muchacho, y cuando los criados contestaron «ayer, hacia la hora séptima, desapareció la fiebre», el padre recordó que era aproximadamente esa hora cuando Jesús había dicho: «Tu hijo vivirá». A partir de entonces Tito creyó de todo corazón, y toda su familia también creyó. Su hijo se convirtió en un poderoso ministro del reino y más tarde sacrificó su vida con los que sufrían en Roma. Toda la familia de Tito, sus amigos, e incluso los apóstoles, consideraron este episodio como un milagro, pero no lo fue. Al menos éste no fue un milagro de curación de una enfermedad física\footnote{\textit{Curación no milagrosa}: Jn 4:54.}. Fue simplemente un caso de preconocimiento respecto al proceso de la ley natural, precisamente el tipo de conocimiento al que Jesús recurrió con frecuencia después de su bautismo\footnote{\textit{Jesús previó la recuperación}: Jn 4:47-53.}.

\par 
%\textsuperscript{(1645.1)}
\textsuperscript{146:5.3} Jesús se vio de nuevo forzado a salir apresuradamente de Caná debido a que el segundo episodio de este tipo que acompañó a su ministerio en esta población había llamado excesivamente la atención. Los vecinos del pueblo se acordaban del agua y del vino, y ahora que suponían que Jesús había curado al hijo del noble a una distancia tan grande, acudían a él no solamente para traerle a los enfermos y a los afligidos, sino también para enviarle mensajeros con el ruego de que curara a los pacientes a distancia. Cuando Jesús vio que toda la región estaba alborotada, dijo: «Vamos a Naín».

\section*{6. Naín y el hijo de la viuda}
\par 
%\textsuperscript{(1645.2)}
\textsuperscript{146:6.1} Esta gente creía en los signos; era una generación que buscaba prodigios. Por esta época, los habitantes de la Galilea central y meridional pensaban en Jesús y en su ministerio personal en términos de milagros. Decenas, centenares de personas honradas que sufrían de desórdenes puramente nerviosos y que estaban afligidas por trastornos emocionales, se presentaban delante de Jesús, y luego volvían a sus casas anunciando a sus amigos que Jesús las había curado. Esta gente ignorante y simple consideraba estos casos de curación mental como curaciones físicas, como curas milagrosas.

\par 
%\textsuperscript{(1645.3)}
\textsuperscript{146:6.2} Cuando Jesús intentó alejarse de Caná para ir a Naín, una gran multitud de creyentes y muchos curiosos se fueron detrás de él. Estaban decididos a contemplar milagros y prodigios, y no iban a quedar decepcionados. Cuando Jesús y sus apóstoles se acercaban a la puerta de la ciudad, se encontraron con una procesión fúnebre que se dirigía al cementerio cercano para llevar al hijo único de una madre viuda de Naín\footnote{\textit{El hijo de la viuda}: Lc 7:11-15.}. Esta mujer era muy respetada, y la mitad del pueblo iba detrás de los que llevaban el féretro de este muchacho supuestamente muerto. Cuando la procesión fúnebre llegó a la altura de Jesús y sus seguidores, la viuda y sus amigos reconocieron al Maestro, y le suplicaron que devolviera el hijo a la vida. Sus expectativas de un milagro se habían despertado hasta tal extremo que creían que Jesús podía curar cualquier enfermedad humana y, ¿por qué este sanador no podría incluso revivir a los muertos? Al ser importunado de esta manera, Jesús se adelantó, levantó la tapa del ataúd y examinó al muchacho. Al descubrir que el joven no estaba realmente muerto, percibió la tragedia que su presencia podía evitar. Así pues, se volvió hacia la madre y le dijo: «No llores. Tu hijo no está muerto; está dormido. Te será devuelto». Luego cogió al joven de la mano y le dijo: «Despiértate y levántate». Y el joven supuestamente muerto se incorporó enseguida y empezó a hablar, y Jesús los envió de vuelta a sus casas.

\par 
%\textsuperscript{(1645.4)}
\textsuperscript{146:6.3} Jesús se esforzó por calmar a la multitud y trató en vano de explicarles que el muchacho no estaba realmente muerto, que él no lo había traído de la tumba, pero fue inútil. La multitud que lo seguía, y todo el pueblo de Naín, habían llegado al máximo grado de frenesí emotivo\footnote{\textit{La gente asombrada}: Lc 7:16.}. Muchos fueron dominados por el miedo, otros por el pánico, mientras que otros aún empezaron a rezar y a lamentarse por sus pecados. No se pudo dispersar a la ruidosa multitud hasta mucho después de la caída de la noche. Naturalmente, a pesar de la afirmación de Jesús de que el muchacho no estaba muerto, todos insistían en que se había producido un milagro, que el muerto había sido resucitado. Aunque Jesús les dijo que el muchacho estaba simplemente en un estado de sueño profundo, explicaron que ésa era su manera de hablar, y llamaron la atención sobre el hecho de que siempre trataba de ocultar sus milagros con mucha modestia.

\par 
%\textsuperscript{(1646.1)}
\textsuperscript{146:6.4} Así pues, la noticia de que Jesús había resucitado de entre los muertos al hijo de la viuda se divulgó por toda Galilea y Judea, y muchos de los que la escucharon se la creyeron. Jesús nunca pudo hacer entender por completo, ni siquiera a todos sus apóstoles, que el hijo de la viuda no estaba realmente muerto cuando le ordenó que se despertara y se levantara. Pero sí los convenció lo suficiente como para evitar que este suceso se incluyera en todos los escritos posteriores, salvo en el de Lucas\footnote{\textit{Un supuesto milagro}: Lc 7:17.}, que relató el episodio tal como se lo habían contado. Una vez más Jesús fue tan asediado como médico, que al día siguiente temprano partió para Endor.

\section*{7. En Endor}
\par 
%\textsuperscript{(1646.2)}
\textsuperscript{146:7.1} En Endor, Jesús eludió durante unos días a las ruidosas multitudes que buscaban la curación física. Durante su estancia en este lugar, el Maestro refirió, para instrucción de los apóstoles, la historia del rey Saúl y la bruja de Endor\footnote{\textit{La bruja de Endor}: 1 Sam 28:7-25.}. Jesús indicó claramente a sus apóstoles que los intermedios desviados y rebeldes que habían personificado con frecuencia a los supuestos espíritus de los muertos, pronto serían puestos bajo control de manera que ya no podrían volver a hacer estas cosas extrañas. Dijo a sus discípulos que, después de que volviera al Padre, y después de que hubieran derramado su espíritu sobre todo el género humano, estos seres semiespirituales ---llamados espíritus impuros--- ya no podrían poseer a los débiles mentales ni a los mortales malintencionados.

\par 
%\textsuperscript{(1646.3)}
\textsuperscript{146:7.2} Jesús explicó además a sus apóstoles que los espíritus de los seres humanos fallecidos no regresan a su mundo de origen para comunicarse con sus semejantes vivos. Al espíritu en progreso del hombre mortal sólo le sería posible volver a la Tierra después de haber transcurrido una época dispensacional, e incluso entonces, sólo sería en casos excepcionales y como parte de la administración espiritual del planeta.

\par 
%\textsuperscript{(1646.4)}
\textsuperscript{146:7.3} Después de haber descansado dos días, Jesús dijo a sus apóstoles: «Regresemos mañana a Cafarnaúm para quedarnos allí y enseñar mientras se calman los alrededores. A estas alturas, en nuestro pueblo ya se habrán recuperado en parte de esta especie de agitación».