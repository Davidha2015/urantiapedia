\chapter{Documento 150. La tercera gira de predicación}
\par 
%\textsuperscript{(1678.1)}
\textsuperscript{150:0.1} EL DOMINGO por la tarde 16 de enero del año 29, Abner llegó a Betsaida con los apóstoles de Juan, y al día siguiente mantuvo una conferencia conjunta con Andrés y los apóstoles de Jesús. Abner y sus asociados establecieron su cuartel general en Hebrón y cogieron la costumbre de venir periódicamente a Betsaida para este tipo de conferencias.

\par 
%\textsuperscript{(1678.2)}
\textsuperscript{150:0.2} Entre las numerosas cuestiones que se consideraron en esta conferencia conjunta se encontraba la práctica de ungir a los enfermos con ciertos tipos de aceite en unión con unas oraciones para la curación\footnote{\textit{Unción de los enfermos con aceite}: Mc 6:13b; Stg 5:14.}. Jesús rehusó de nuevo participar en estas discusiones o expresar su opinión sobre las conclusiones. Los apóstoles de Juan siempre habían utilizado el aceite de ungir en su ministerio hacia los enfermos y los afligidos, y trataron de establecer que esta práctica fuera uniforme para ambos grupos, pero los apóstoles de Jesús se negaron a someterse a esta regla.

\par 
%\textsuperscript{(1678.3)}
\textsuperscript{150:0.3} El martes 18 de enero, los evangelistas que habían pasado la prueba, unos setenta y cinco en total, se reunieron con los veinticuatro en la casa de Zebedeo en Betsaida, antes de ser enviados a la tercera gira de predicación por Galilea. Esta tercera misión se prolongó durante siete semanas.

\par 
%\textsuperscript{(1678.4)}
\textsuperscript{150:0.4} Los evangelistas fueron enviados en grupos de cinco, mientras que Jesús y los doce viajaron juntos la mayor parte del tiempo; los apóstoles salían de dos en dos para bautizar a los creyentes cuando lo requería la ocasión. Durante un período de casi tres semanas, Abner y sus asociados trabajaron también con los grupos de evangelistas, aconsejándolos y bautizando a los creyentes. Visitaron Magdala, Tiberiades, Nazaret y todas las principales ciudades y pueblos del centro y sur de Galilea, todos los lugares visitados anteriormente y muchos más. Éste fue su último mensaje para Galilea, exceptuando las regiones del norte\footnote{\textit{Tercera gira de predicación}: Mt 9:35; Lc 8:1.}.

\section*{1. El cuerpo de mujeres evangelistas}
\par 
%\textsuperscript{(1678.5)}
\textsuperscript{150:1.1} De todos los actos audaces que Jesús efectuó en relación con su carrera terrestre, el más asombroso fue su anuncio repentino la tarde del 16 de enero: «Mañana seleccionaremos a diez mujeres para trabajar en el ministerio del reino»\footnote{\textit{El cuerpo de mujeres}: Lc 8:2-3.}. Al empezar el período de dos semanas durante las cuales los apóstoles y los evangelistas iban a estar ausentes de Betsaida debido a sus vacaciones, Jesús le rogó a David que llamara a sus padres para que regresaran a su hogar, y que enviara a unos mensajeros para convocar en Betsaida a diez mujeres devotas que habían servido en la administración del antiguo campamento y la enfermería de tiendas. Todas estas mujeres habían escuchado la enseñanza impartida a los jóvenes evangelistas, pero nunca se les había ocurrido, ni a ellas ni a sus instructores, que Jesús se atrevería a encargar a unas mujeres la enseñanza del evangelio del reino y la atención a los enfermos. Estas diez mujeres escogidas y autorizadas por Jesús eran: Susana, la hija del antiguo chazán de la sinagoga de Nazaret; Juana, la esposa de Chuza, el administrador de Herodes Antipas; Isabel, la hija de un judío rico de Tiberiades y Séforis; Marta, la hermana mayor de Andrés y Pedro; Raquel, la cuñada de Judá, el hermano carnal del Maestro; Nasanta, la hija de Elman, el médico sirio; Milca, una prima del apóstol Tomás; Rut, la hija mayor de Mateo Leví; Celta, la hija de un centurión romano; y Agaman, una viuda de Damasco. Posteriormente, Jesús añadió dos mujeres más a este grupo: María Magdalena y Rebeca, la hija de José de Arimatea.

\par 
%\textsuperscript{(1679.1)}
\textsuperscript{150:1.2} Jesús autorizó a estas mujeres para que establecieran su propia organización, y ordenó a Judas que les proporcionara fondos para equiparse y comprar animales de carga. Las diez eligieron a Susana como jefa y a Juana como tesorera. A partir de este momento se procuraron sus propios fondos\footnote{\textit{Financiación de las mujeres}: Lc 8:3b.}; nunca más recurrieron a la ayuda de Judas.

\par 
%\textsuperscript{(1679.2)}
\textsuperscript{150:1.3} En una época como ésta, en la que ni siquiera se permitía a las mujeres permanecer en el piso principal de la sinagoga (estaban confinadas a la galería de las mujeres), era más que sorprendente observar que se las reconocía como instructoras autorizadas del nuevo evangelio del reino. El encargo que Jesús confió a estas diez mujeres, al seleccionarlas para la enseñanza y el ministerio del evangelio, fue la proclamación de emancipación que liberaba a todas las mujeres para todos los tiempos; los hombres ya no debían considerar a las mujeres como espiritualmente inferiores a ellos. Fue una auténtica conmoción, incluso para los doce apóstoles. A pesar de que habían escuchado muchas veces decir al Maestro que «en el reino de los cielos no hay ni ricos ni pobres, ni libres ni esclavos, ni hombres ni mujeres, sino que todos son igualmente los hijos e hijas de Dios»\footnote{\textit{Igualdad de todos los seres humanos}: 2 Cr 19:7; Job 34:19; Eclo 35:12; Hch 10:34; Ro 2:11; Gl 2:6; 3:28; Ef 6:9; Col 3:11.}, se quedaron literalmente pasmados cuando Jesús propuso autorizar formalmente a estas diez mujeres como instructoras religiosas, e incluso permitirles que viajaran con ellos. Todo el país se conmovió por esta manera de proceder, y los enemigos de Jesús sacaron un gran provecho de esta decisión; pero por todas partes, las mujeres que creían en la buena nueva respaldaron firmemente a sus hermanas escogidas, y expresaron su más plena aprobación a este reconocimiento tardío del lugar de la mujer en el trabajo religioso. Inmediatamente después de la partida del Maestro, los apóstoles pusieron en práctica esta liberación de las mujeres, otorgándoles el debido reconocimiento, pero las generaciones posteriores volvieron a caer en las antiguas costumbres. Durante los primeros tiempos de la iglesia cristiana, las mujeres instructoras y ministras fueron llamadas \textit{diaconisas}, y se les concedió un reconocimiento general. Pero Pablo, a pesar del hecho de que admitía todo esto en teoría, nunca lo incorporó realmente en su propia actitud y le resultó personalmente difícil ponerlo en práctica\footnote{\textit{Visión de Pablo de las mujeres}: 1 Co 11:1-15; 14:34-35; 1 Ti 2:9-12; 1 P 3:1-6.}.

\section*{2. La parada en Magdala}
\par 
%\textsuperscript{(1679.3)}
\textsuperscript{150:2.1} Cuando el grupo apostólico salió de Betsaida, las mujeres viajaron en la retaguardia. Durante las conferencias, siempre se sentaban en grupo enfrente y a la derecha del orador. Cada vez más mujeres se habían hecho creyentes en el evangelio del reino, y cuando habían deseado mantener una conversación personal con Jesús o con uno de los apóstoles, se habían originado muchas dificultades y un sin fin de situaciones embarazosas. Ahora, todo esto había cambiado. Cuando cualquier mujer creyente deseaba ver al Maestro o entrevistarse con los apóstoles, iba a ver a Susana, y acompañada por una de las doce mujeres evangelistas, se dirigían enseguida a la presencia del Maestro o de uno de sus apóstoles.

\par 
%\textsuperscript{(1680.1)}
\textsuperscript{150:2.2} Fue en Magdala donde las mujeres demostraron por primera vez su utilidad y justificaron la sabiduría de haberlas escogido. Andrés había impuesto a sus asociados unas reglas más bien estrictas en lo referente al trabajo personal con las mujeres, especialmente con aquellas de conducta dudosa. Cuando el grupo llegó a Magdala, estas diez mujeres evangelistas pudieron entrar libremente en los lugares depravados y predicar directamente la buena nueva a todas sus inquilinas. Y cuando visitaban a los enfermos, estas mujeres eran capaces de acercarse íntimamente, en su ministerio, a sus hermanas afligidas. A consecuencia del servicio efectuado en este lugar por estas diez mujeres (más tarde conocidas como las doce mujeres), María Magdalena fue ganada para el reino. A través de una serie de desventuras, y como consecuencia de la actitud de la sociedad respetable hacia las mujeres que cometían estos errores de juicio, esta mujer había ido a parar a uno de los lugares ignominiosos de Magdala. Marta y Raquel fueron las que indicaron claramente a María que las puertas del reino estaban abiertas incluso para las personas como ella. María creyó en la buena nueva y fue bautizada por Pedro al día siguiente.

\par 
%\textsuperscript{(1680.2)}
\textsuperscript{150:2.3} María Magdalena se convirtió en la educadora más eficaz del evangelio, dentro de este grupo de doce mujeres evangelistas. Fue seleccionada para este servicio en Jotapata, junto con Rebeca, unas cuatro semanas después de su conversión. Durante el resto de la vida de Jesús en la Tierra, María, Rebeca y sus compañeras de grupo continuaron trabajando fiel y eficazmente para iluminar y elevar a sus hermanas oprimidas. Y cuando el último y trágico episodio del drama de la vida de Jesús se estaba representando, a pesar de que todos los apóstoles, salvo uno, habían huido, todas estas mujeres estuvieron presentes, y ninguna de ellas lo negó ni lo traicionó\footnote{\textit{La fidelidad de las mujeres}: Mt 27:55-56,61; Mc 15:40-41,47; Lc 8:2-3; 23:49; 24:10-11; Jn 19:25.}.

\section*{3. Un sábado en Tiberiades}
\par 
%\textsuperscript{(1680.3)}
\textsuperscript{150:3.1} Andrés, siguiendo las instrucciones de Jesús, había responsabilizado a las mujeres de los oficios del grupo apostólico para el sábado. Esto significaba, naturalmente, que no se podían celebrar en la nueva sinagoga. Las mujeres eligieron a Juana\footnote{\textit{Juana, mujer de Chuza}: Lc 8:3.} para que se encargara de esta contingencia, y la reunión se celebró en la sala de banquetes del nuevo palacio de Herodes, ya que Herodes se encontraba residiendo en Julias, en Perea. Juana leyó en las Escrituras unos pasajes sobre la obra de la mujer en la vida religiosa de Israel, haciendo referencia a Miriam, Débora, Ester y otras.

\par 
%\textsuperscript{(1680.4)}
\textsuperscript{150:3.2} A una hora avanzada de aquella noche, Jesús dio al grupo reunido una charla memorable sobre «La magia y la superstición». En aquellos tiempos, la aparición de una estrella brillante\footnote{\textit{Supersticiones con las estrellas}: Mt 2:1-2.} y supuestamente nueva era considerada como el signo de que un gran hombre había nacido en la Tierra. Como se había observado recientemente una estrella de este tipo, Andrés le preguntó a Jesús si estas creencias estaban bien fundadas. En su larga respuesta a la pregunta de Andrés, el Maestro emprendió un examen completo de todo el tema de la superstición humana. La exposición que Jesús efectuó en esta ocasión se puede resumir, en lenguaje moderno, de la manera siguiente:

\par 
%\textsuperscript{(1680.5)}
\textsuperscript{150:3.3} 1. El camino que siguen las estrellas en el cielo no tiene absolutamente nada que ver con los acontecimientos de la vida humana en la Tierra. La astronomía es una ocupación adecuada de la ciencia, pero la astrología es una masa de errores supersticiosos que no tienen ningún sitio en el evangelio del reino.

\par 
%\textsuperscript{(1680.6)}
\textsuperscript{150:3.4} 2. El examen de los órganos internos de un animal recién degollado no puede revelar nada sobre el tiempo atmosférico, los acontecimientos futuros o el resultado de los asuntos humanos.

\par 
%\textsuperscript{(1680.7)}
\textsuperscript{150:3.5} 3. Los espíritus de los muertos no regresan para comunicarse con sus familiares o con sus antiguos amigos todavía vivos.

\par 
%\textsuperscript{(1681.1)}
\textsuperscript{150:3.6} 4. Los amuletos y las reliquias son impotentes para curar las enfermedades, evitar los desastres o influir en los malos espíritus; la creencia en todos estos medios materiales para influir sobre el mundo espiritual no es más que una vulgar superstición.

\par 
%\textsuperscript{(1681.2)}
\textsuperscript{150:3.7} 5. Echarlo a suertes quizás sea una manera útil de resolver muchas dificultades menores, pero no es un método destinado a descubrir la voluntad divina. Los resultados que se obtienen así son simplemente el producto de la casualidad material. El único medio de comulgar con el mundo espiritual está incluido en la dotación espiritual de la humanidad, el espíritu interior del Padre, junto con el espíritu derramado por el Hijo y la influencia omnipresente del Espíritu Infinito.

\par 
%\textsuperscript{(1681.3)}
\textsuperscript{150:3.8} 6. La adivinación, la hechicería y la brujería son supersticiones de las mentes ignorantes, como también lo son las ilusiones de la magia. La creencia en los números mágicos, en los pronósticos de buena suerte y en los presagios de mala suerte, es una pura superstición sin ningún fundamento.

\par 
%\textsuperscript{(1681.4)}
\textsuperscript{150:3.9} 7. La interpretación de los sueños es ampliamente un sistema supersticioso e infundado de especulaciones ignorantes y fantásticas. El evangelio del reino no ha de tener nada en común con los sacerdotes adivinos de la religión primitiva.

\par 
%\textsuperscript{(1681.5)}
\textsuperscript{150:3.10} 8. Los espíritus del bien o del mal no pueden residir dentro de los símbolos materiales de arcilla, madera o metal; los ídolos no son nada más que el material con el que están fabricados.

\par 
%\textsuperscript{(1681.6)}
\textsuperscript{150:3.11} 9. Las prácticas de los encantadores, los brujos, los magos y los hechiceros provienen de las supersticiones de los egipcios, los asirios, los babilonios y los antiguos cananeos. Los amuletos y todas las clases de encantamientos son inútiles tanto para conseguir la protección de los buenos espíritus como para desviar a los supuestos espíritus impuros.

\par 
%\textsuperscript{(1681.7)}
\textsuperscript{150:3.12} 10. Jesús desenmascaró y censuró la creencia de sus oyentes en los encantamientos, las ordalías, los hechizos, las maldiciones, los signos, las mandrágoras, las cuerdas anudadas y todas las demás formas de superstición ignorante y esclavizante.

\section*{4. El envío de los apóstoles de dos en dos}
\par 
%\textsuperscript{(1681.8)}
\textsuperscript{150:4.1} A la tarde siguiente, después de reunir a los doce apóstoles\footnote{\textit{Los apóstoles se reúnen}: Mt 10:1; Mc 3:14-15; Lc 9:1-2.}, a los apóstoles de Juan y al grupo recién autorizado de mujeres, Jesús dijo: «Podéis ver por vosotros mismos que la cosecha es abundante, pero que los obreros son pocos\footnote{\textit{La cosecha es abundante, pero los obreros pocos}: Mt 9:37-38; Lc 10:2.}. Así pues, oremos todos al Señor de la cosecha para que envíe aún más obreros a sus campos. Mientras yo me quedo aquí para animar e instruir a los educadores más jóvenes, quisiera enviar a los más antiguos de dos en dos para que pasen rápidamente por toda Galilea predicando el evangelio del reino, mientras que aún se puede hacer de manera cómoda y pacífica». Luego designó a las parejas de apóstoles\footnote{\textit{Envía a los apóstoles de dos en dos}: Mc 6:7; Lc 10:1.} tal como él deseaba que salieran, y fueron las siguientes: Andrés y Pedro, Santiago y Juan Zebedeo, Felipe y Natanael, Tomás y Mateo, Santiago y Judas Alfeo, Simón Celotes y Judas Iscariote\footnote{\textit{Los apóstoles son nominados}: Mt 10:2-4; Mc 3:16-19; Lc 6:14-16; Hch 1:13.}.

\par 
%\textsuperscript{(1681.9)}
\textsuperscript{150:4.2} Jesús fijó la fecha en que se encontraría con los doce en Nazaret, y al separarse dijo: «Durante esta misión, no vayáis a ninguna ciudad de los gentiles ni tampoco a Samaria; id más bien donde están las ovejas perdidas de la casa de Israel\footnote{\textit{Predicad a la casa de Israel}: Mt 10:5-7.}. Predicad el evangelio del reino y proclamad la verdad salvadora de que el hombre es un hijo de Dios. Recordad que el discípulo difícilmente está por encima de su maestro y que un siervo no es más grande que su señor. Es suficiente con que el discípulo sea igual a su maestro y el siervo llegue a ser como su señor\footnote{\textit{Relación de maestro y sirviente}: Mt 10:24-25a; Lc 6:40; Jn 13:16; Jn 15:20.}. Si alguna gente se ha atrevido a calificar al dueño de la casa de asociado de Belcebú\footnote{\textit{Belcebú}: Mt 10:25b.}, ¡con cuánta más razón considerarán de esa manera a la gente de su casa! Pero no tenéis que temer a estos enemigos incrédulos. Os aseguro que no hay nada tan encubierto que no se pueda revelar; no hay nada oculto que no se pueda conocer\footnote{\textit{No hay nada oculto que no será revelado}: Mt 10:26-28; Lc 12:2-5.}. Lo que os he enseñado en privado, predicadlo en público con sabiduría. Lo que os he revelado dentro de la casa, proclamadlo a su debido tiempo desde los tejados. Os lo digo, amigos y discípulos míos, no temáis a los que pueden matar el cuerpo, pero no son capaces de destruir el alma; poned más bien vuestra confianza en Aquel que es capaz de sostener el cuerpo y de salvar el alma».

\par 
%\textsuperscript{(1682.1)}
\textsuperscript{150:4.3} «¿No se venden dos gorriones por un céntimo? Y sin embargo os declaro que ninguno de ellos está olvidado a los ojos de Dios. ¿No sabéis que incluso los cabellos de vuestras cabezas están todos contados?\footnote{\textit{Los gorriones no valen nada; los cabellos de vuestra cabeza están contados}: Mt 10:29-31; Lc 12:6-7.} Así pues, no temáis; vosotros valéis más que una gran cantidad de gorriones. No os avergoncéis de mi enseñanza; salid a proclamar la paz y la buena voluntad, pero no os engañéis ---la paz no siempre acompañará vuestra predicación. He venido para traer la paz a la Tierra, pero cuando los hombres rechazan mi regalo, se producen divisiones y disturbios\footnote{\textit{No hay paz, sino tristeza sobre la Tierra}: Mt 10:34-37a; Lc 12:51-53.}. Cuando toda una familia recibe el evangelio del reino, la paz permanece realmente en esa casa; pero cuando algunos miembros de la familia entran en el reino y otros rechazan el evangelio, una división así sólo puede producir pena y tristeza. Trabajad seriamente para salvar a la familia entera, a fin de que un hombre no tenga por enemigos a los miembros de su propia casa. Pero cuando hayáis hecho todo lo posible por todos los miembros de cada familia, os declaro que cualquiera que ame a su padre o a su madre más que a este evangelio, no es digno del reino»\footnote{\textit{Amad el evangelio por encima de a la familia}: Lc 14:26.}.

\par 
%\textsuperscript{(1682.2)}
\textsuperscript{150:4.4} Después de haber escuchado estas palabras, los doce se prepararon para partir\footnote{\textit{Los apóstoles se marchan a predicar}: Mc 6:12-13; Lc 9:6.}. No volvieron a verse hasta el momento en que se reunieron en Nazaret para encontrarse con Jesús y los otros discípulos, tal como el Maestro lo había dispuesto.

\section*{5. ¿Qué debo hacer para salvarme?}
\par 
%\textsuperscript{(1682.3)}
\textsuperscript{150:5.1} Una tarde en Sunem, después de que los apóstoles de Juan hubieran regresado a Hebrón y los apóstoles de Jesús hubieran sido enviados de dos en dos, el Maestro estaba ocupado en enseñar a un grupo de doce de los evangelistas más jóvenes que trabajaban bajo la dirección de Jacobo, junto con las doce mujeres, cuando Raquel le hizo a Jesús la pregunta siguiente: «Maestro, ¿qué debemos responder cuando las mujeres nos preguntan: Qué debo hacer para salvarme?» Cuando Jesús escuchó esta pregunta, respondió:

\par 
%\textsuperscript{(1682.4)}
\textsuperscript{150:5.2} «Cuando los hombres y las mujeres os pregunten qué deben hacer para salvarse, vosotras contestaréis: Creed en este evangelio del reino\footnote{\textit{Evangelio del reino}: Mt 4:23; 9:35; 24:14; Mc 1:14-15. \textit{Creed para salvaros}: Jn 1:12; Ro 3:21--4:5.}; aceptad el perdón divino. Reconoced, por la fe\footnote{\textit{Tan sólo vivid por la fe}: Hab 2:4.}, al espíritu interno de Dios, cuya aceptación os convierte en hijos de Dios. ¿No habéis leído en las Escrituras el pasaje que dice: `Mi rectitud y mi fuerza residen en el Señor?'\footnote{\textit{Mi rectitud reside en el Señor}: Is 45:24.} Y también allí donde el Padre dice: `Mi justicia se acerca; mi salvación se ha hecho pública y mis brazos envolverán a mi pueblo'\footnote{\textit{Mis brazos envolverán al pueblo}: Is 51:5.}. `Mi alma se regocijará en el amor de mi Dios\footnote{\textit{Mi alma se regocija en el amor de Dios}: Is 61:10.}, porque me ha vestido con las vestiduras de la salvación y me ha cubierto con la túnica de su rectitud'. ¿No habéis leído también, refiriéndose al Padre, que su nombre `será llamado el Señor de nuestra rectitud?'\footnote{\textit{Señor de nuestra rectitud}: Jer 23:6; Jer 33:16.} `Quitaos los harapos sucios de la presunción y vestid a mi hijo con la túnica de la rectitud divina y de la salvación eterna'\footnote{\textit{Vestid a mi hijo con la túnica divina}: Is 61:10; Zac 3:4.}. Es eternamente cierto que `el justo vivirá por su fe'. La entrada en el reino del Padre es totalmente libre, pero el progreso ---el crecimiento en la gracia--- es indispensable para permanecer allí».

\par 
%\textsuperscript{(1682.5)}
\textsuperscript{150:5.3} «La salvación es el don del Padre y es revelada por sus Hijos\footnote{\textit{La salvación, el don del Padre revelado por los Hijos}: Ef 2:8.}. Su aceptación, por la fe, os convierte en partícipes de la naturaleza divina, en hijos o hijas de Dios. Por la fe, estáis justificadas; por la fe, sois salvadas\footnote{\textit{Salvados por la fe}: Mc 16:16; Jn 3:36; Hch 13:39; Ro 5:1; Gl 3:24.}; y por esta misma fe, avanzaréis eternamente en el camino de la perfección progresiva y divina. Abraham fue justificado por la fe y tomó conciencia de la salvación gracias a las enseñanzas de Melquisedek. A lo largo de todos los tiempos, esta misma fe ha salvado a los hijos de los hombres, pero ahora un Hijo ha venido del Padre para hacer más real y aceptable la salvación».

\par 
%\textsuperscript{(1683.1)}
\textsuperscript{150:5.4} Cuando Jesús terminó de hablar, los que habían escuchado estas palabras benévolas sintieron un gran regocijo, y en los días que siguieron, todos continuaron proclamando el evangelio del reino con una nueva fuerza y con una energía y un entusiasmo renovados. Las mujeres se regocijaron aún más al saber que estaban incluidas en estos planes para establecer el reino en la Tierra.

\par 
%\textsuperscript{(1683.2)}
\textsuperscript{150:5.5} Al resumir su declaración final, Jesús dijo: «No podéis comprar la salvación; no podéis ganar la rectitud. La salvación es un don de Dios, y la rectitud es el fruto natural de la vida nacida del espíritu, la vida de filiación en el reino. No vais a salvaros porque viváis una vida de rectitud, sino que viviréis una vida de rectitud porque ya habéis sido salvados, porque habéis reconocido la filiación como un don de Dios, y el servicio en el reino como la delicia suprema de la vida en la Tierra. Cuando los hombres creen en este evangelio, que es una revelación de la bondad de Dios, se sienten inducidos a arrepentirse voluntariamente de todos los pecados conocidos. La realización de la filiación es incompatible con el deseo de pecar. Los creyentes en el reino tienen hambre de rectitud y sed de perfección divina».

\section*{6. Las lecciones vespertinas}
\par 
%\textsuperscript{(1683.3)}
\textsuperscript{150:6.1} En las discusiones de la tarde, Jesús habló de muchos temas. Durante el resto de esta gira ---antes de que todos se reunieran en Nazaret--- trató de «El amor de Dios», «Los sueños y las visiones», «La malicia», «La humildad y la mansedumbre», «El coraje y la lealtad», «La música y la adoración», «El servicio y la obediencia», «El orgullo y la presunción», «La relación entre el perdón y el arrepentimiento», «La paz y la perfección», «La calumnia y la envidia», «El mal, el pecado y la tentación», «Las dudas y la incredulidad», «La sabiduría y la adoración». Como los apóstoles más antiguos estaban ausentes, estos grupos más jóvenes de hombres y mujeres participaron más libremente en estos debates con el Maestro.

\par 
%\textsuperscript{(1683.4)}
\textsuperscript{150:6.2} Después de pasar dos o tres días con un grupo de doce evangelistas, Jesús se desplazaba para reunirse con otro grupo, y los mensajeros de David le informaban del paradero y de los movimientos de todos estos trabajadores. Como ésta era su primera gira, las mujeres permanecieron una buena parte del tiempo con Jesús. Cada uno de estos grupos estaba plenamente informado del desarrollo de la gira gracias al servicio de los mensajeros, y la recepción de noticias de los otros grupos siempre era una fuente de estímulo para estos trabajadores dispersos y separados.

\par 
%\textsuperscript{(1683.5)}
\textsuperscript{150:6.3} Antes de separarse, se había acordado que los doce apóstoles, junto con los evangelistas y el cuerpo de mujeres, se congregarían en Nazaret el viernes 4 de marzo para reunirse con el Maestro\footnote{\textit{Regreso a Nazaret}: Mc 6:1; Lc 4:16a.}. En consecuencia, alrededor de esta fecha, los diversos grupos de apóstoles y de evangelistas empezaron a dirigirse desde todas las partes de la Galilea central y meridional hacia Nazaret. A media tarde, Andrés y Pedro, los últimos en llegar, habían entrado en el campamento preparado por los primeros que llegaron y situado en las altas tierras al norte de la ciudad. Ésta era la primera vez que Jesús visitaba Nazaret desde el comienzo de su ministerio público.

\section*{7. La estancia en Nazaret}
\par 
%\textsuperscript{(1683.6)}
\textsuperscript{150:7.1} Este viernes por la tarde, Jesús se paseó por Nazaret totalmente desapercibido y sin ser reconocido. Pasó por la casa de su infancia y por el taller de carpintería y permaneció media hora en la colina donde tanto disfrutaba cuando era un muchacho. Desde el día en que Juan lo bautizó en el Jordán, el Hijo del Hombre no había sentido conmoverse en su alma tal cantidad de emociones humanas. Mientras bajaba de la montaña, escuchó los sonidos familiares del toque de trompeta que anunciaba la puesta del Sol, tal como los había escuchado tantísimas veces cuando era un niño que crecía en Nazaret. Antes de volver al campamento, pasó por la sinagoga donde había ido a la escuela, y se abandonó mentalmente a numerosas reminiscencias de la época de su infancia. Horas antes, Jesús había enviado a Tomás para que se pusiera de acuerdo con el jefe de la sinagoga a fin de poder predicar en los oficios matutinos del sábado.

\par 
%\textsuperscript{(1684.1)}
\textsuperscript{150:7.2} La gente de Nazaret nunca había sido famosa por su religiosidad ni por su manera recta de vivir. Con el transcurso de los años, este pueblo se había contaminado cada vez más con los bajos criterios morales de la cercana ciudad de Séforis. Durante toda la juventud y los primeros años de la vida adulta de Jesús, las opiniones sobre él habían estado divididas en Nazaret; su decisión de mudarse a Cafarnaúm había producido mucho resentimiento. Los habitantes de Nazaret habían oído hablar mucho de las actividades de su antiguo carpintero, pero estaban ofendidos porque nunca había incluído a su pueblo natal en ninguna de sus anteriores giras de predicación. Habían oído hablar, por supuesto, de la fama de Jesús, pero la mayoría de los ciudadanos estaban enojados porque no había realizado ninguna de sus grandes obras en la ciudad de su juventud. Durante meses, la gente de Nazaret había discutido mucho sobre Jesús, pero sus opiniones eran, en general, desfavorables hacia él.

\par 
%\textsuperscript{(1684.2)}
\textsuperscript{150:7.3} El Maestro se encontró pues, no en un ambiente de bienvenida al hogar, sino en medio de una atmósfera decididamente hostil e hipercrítica. Pero esto no era todo. Sabiendo que iba a pasar este sábado en Nazaret y suponiendo que hablaría en la sinagoga, sus enemigos habían sobornado a un buen número de hombres rudos y groseros para que lo hostigaran y provocaran dificultades de todas las maneras posibles.

\par 
%\textsuperscript{(1684.3)}
\textsuperscript{150:7.4} La mayoría de los antiguos amigos de Jesús, incluído el chazán que lo adoraba y que había sido su profesor en la adolescencia, habían muerto o se habían marchado de Nazaret, y la generación más joven era propensa a sentirse muy recelosa con su fama. Ya no se acordaban de su dedicación, siendo adolescente, a la familia de su padre, y lo criticaban severamente por su negligencia en no visitar a su hermano y a sus hermanas casadas que vivían en Nazaret. La actitud de la familia de Jesús hacia él también había contribuido a acrecentar este sentimiento desfavorable de los ciudadanos. Los judíos más ortodoxos se atrevieron incluso a criticar a Jesús por haber caminado demasiado deprisa cuando iba a la sinagoga aquel sábado por la mañana.

\section*{8. Los oficios del sábado}
\par 
%\textsuperscript{(1684.4)}
\textsuperscript{150:8.1} Aquel sábado hacía un día magnífico, y todo Nazaret, amigos y enemigos, salió para escuchar lo que este antiguo habitante de su ciudad iba a decir en la sinagoga\footnote{\textit{Multitud en la sinagoga}: Mt 13:54a; Mc 6:2a; Lc 4:16b.}. Una gran parte del séquito apostólico tuvo que permanecer fuera de la sinagoga, pues no había sitio para todos los que habían venido a escucharlo. Cuando era joven, Jesús había hablado con frecuencia en este lugar de culto. Aquella mañana, cuando el jefe de la sinagoga le pasó el rollo de los escritos sagrados donde iba a leer la lección de las Escrituras, ninguno de los presentes pareció recordar que éste era el mismo manuscrito que Jesús había regalado a esta sinagoga.

\par 
%\textsuperscript{(1684.5)}
\textsuperscript{150:8.2} Los oficios de este día se celebraron exactamente igual que cuando Jesús asistía siendo niño. Subió al estrado de los oradores con el jefe de la sinagoga, y el oficio empezó recitándose dos oraciones: «Bendito sea el Señor, Rey del mundo, que forma la luz y crea las tinieblas, que hace la paz y crea todas las cosas; que en su misericordia da la luz a la Tierra y a los que viven en ella, y que en su bondad renueva las obras de la creación día tras día y cada día. Bendito sea el Señor nuestro Dios por la gloria de las obras de sus manos y por las luces iluminadoras que ha hecho para su alabanza. Selá. Bendito sea el Señor nuestro Dios que ha creado las luces».

\par 
%\textsuperscript{(1685.1)}
\textsuperscript{150:8.3} Después de una breve pausa, siguieron rezando: «El Señor nuestro Dios nos ha amado con un gran amor, y se ha compadecido de nosotros con una piedad desbordante, nuestro Padre y nuestro Rey, por amor a nuestros padres que confiaron en él. Tú les enseñaste las reglas de la vida; ten misericordia de nosotros y enséñanos. Ilumina nuestros ojos con la ley; haz que nuestros corazones se ajusten a tus mandamientos; une nuestros corazones para que amemos y temamos tu nombre, y no nos avergonzaremos por los siglos de los siglos. Porque tú eres un Dios que prepara la salvación, y nos has escogido entre todas las naciones y lenguas, y en verdad nos has acercado a tu gran nombre ---selá--- para que podamos alabar tu unidad con amor. Bendito sea el Señor que, en su amor, ha elegido a su pueblo Israel».

\par 
%\textsuperscript{(1685.2)}
\textsuperscript{150:8.4} La congregación recitó luego el Semá, el credo de la fe judía. Este ritual consistía en repetir numerosos pasajes de la ley, e indicaba que los creyentes aceptaban el yugo del reino de los cielos, y también el yugo de los mandamientos tal como debían aplicarlos de día y de noche.

\par 
%\textsuperscript{(1685.3)}
\textsuperscript{150:8.5} Luego continuaron con la tercera oración: «Es verdad que tú eres Yahvé, nuestro Dios y el Dios de nuestros padres, nuestro Rey y el Rey de nuestros padres; nuestro Salvador y el Salvador de nuestros padres; nuestro Creador y la roca de nuestra salvación; nuestra ayuda y nuestro libertador. Tu nombre existe desde la eternidad, y no hay más Dios que tú. Los que fueron liberados cantaron un nuevo cántico a tu nombre a la orilla del mar; todos juntos te alabaron y te reconocieron como Rey, diciendo: Yahvé reinará por los siglos de los siglos. Bendito sea el Señor que salva a Israel».

\par 
%\textsuperscript{(1685.4)}
\textsuperscript{150:8.6} El jefe de la sinagoga se situó entonces en su puesto delante del arca, o cofre, que contenía las escrituras sagradas, y empezó a recitar las diecinueve oraciones de elogio, o bendiciones. Pero en esta ocasión era conveniente acortar el oficio a fin de que el invitado de honor dispusiera de más tiempo para su discurso; por consiguiente, sólo se recitaron la primera y la última bendiciones. La primera era: «Bendito sea el Señor nuestro Dios y el Dios de nuestros padres, el Dios de Abraham, el Dios de Isaac y el Dios de Jacob; el grande, el poderoso y el terrible Dios, que muestra misericordia y benevolencia, que crea todas las cosas, que recuerda sus bondadosas promesas a nuestros padres y envía con amor un salvador a los hijos de sus hijos para gloria de su propio nombre. Oh Rey, favorecedor, salvador y protector. Bendito eres tú, oh Yahvé, protector de Abraham».

\par 
%\textsuperscript{(1685.5)}
\textsuperscript{150:8.7} Después siguió la última bendición: «Oh, concede a tu pueblo Israel una gran paz perpetua, pues tú eres el Rey y el Señor de toda paz. Y ves con buenos ojos bendecir con la paz a Israel en todo tiempo y a todas horas. Bendito seas, Yahvé, que bendices con la paz a tu pueblo Israel». La asamblea no miraba al jefe mientras éste recitaba las bendiciones. Después de las bendiciones, ofreció una oración no oficial, adecuada a la circunstancia, y cuando concluyó, toda la congregación se unió para decir amén.

\par 
%\textsuperscript{(1685.6)}
\textsuperscript{150:8.8} Luego, el chazán se dirigió al arca y sacó un rollo que entregó a Jesús para que éste pudiera leer la lección de las Escrituras\footnote{\textit{Jesús tomó el rollo}: Lc 4:17a.}. Era habitual llamar a siete personas para que leyeran por lo menos tres versos de la ley, pero en esta ocasión se renunció a esta práctica para que el visitante pudiera leer la lección que él mismo había escogido. Jesús cogió el rollo, se puso de pie y empezó a leer en el Deuteronomio: «Pues este mandamiento que hoy te doy no es un secreto para ti, ni está lejos. No está en el cielo, para que no digas: ¿quién subirá al cielo por nosotros y nos lo traerá para que podamos oírlo y ponerlo en práctica? Tampoco está al otro lado del mar, para que no digas: ¿quién atravesará el mar por nosotros para que nos traiga el mandamiento a fin de que podamos oírlo y ponerlo en práctica? No, la palabra de vida está muy cerca de ti, incluso en tu presencia y en tu corazón, para que puedas conocerla y obedecerla»\footnote{\textit{Este mandamiento no está oculto}: Dt 30:11-14.}.

\par 
%\textsuperscript{(1686.1)}
\textsuperscript{150:8.9} Cuando terminó de leer en el libro de la ley, pasó a Isaías\footnote{\textit{Jesús lee a los profetas}: Lc 4:17-19.} donde empezó a leer: «El espíritu del Señor está sobre mí, porque me ha ungido para que predique la buena nueva a los pobres. Me ha enviado para que proclame la libertad a los cautivos y la recuperación de la vista a los ciegos, para poner en libertad a los que se sienten heridos y proclamar el año favorable del Señor»\footnote{\textit{El espíritu del Señor está sobre mí}: Is 61:1-2a.}.

\par 
%\textsuperscript{(1686.2)}
\textsuperscript{150:8.10} Jesús cerró el libro y, después de devolverlo al jefe de la sinagoga, se sentó y empezó a hablarle a la gente. Comenzó diciendo: «Hoy, estas Escrituras se han cumplido»\footnote{\textit{Hoy estas escrituras se han cumplido}: Lc 4:21.}. Y luego habló cerca de quince minutos sobre «Los hijos y las hijas de Dios». Su discurso agradó a muchos de los asistentes\footnote{\textit{El agrado de la gente}: Lc 4:20-22a.}, que se maravillaron de su gracia y de su sabiduría\footnote{\textit{La gente se quedó atónita}: Mt 13:54b; Mc 6:2b.}.

\par 
%\textsuperscript{(1686.3)}
\textsuperscript{150:8.11} Después de concluir los oficios formales, existía la costumbre de que el orador permaneciera en la sinagoga para que las personas interesadas pudieran hacerle preguntas. En consecuencia, este sábado por la mañana, Jesús descendió para mezclarse con la multitud que se adelantaba para hacerle preguntas. En este grupo había muchos individuos violentos con intenciones dañinas, mientras que alrededor del gentío circulaban aquellos degenerados que habían sido sobornados para causarle problemas a Jesús. Muchos discípulos y evangelistas que habían permanecido fuera avanzaron ahora para entrar en la sinagoga y se dieron cuenta enseguida de que se estaba fraguando un disturbio. Trataron de llevarse al Maestro, pero éste no quiso ir con ellos.

\section*{9. Nazaret rechaza a Jesús}
\par 
%\textsuperscript{(1686.4)}
\textsuperscript{150:9.1} Jesús se encontró rodeado en la sinagoga por una gran multitud de enemigos y muy pocos de sus propios seguidores\footnote{\textit{La oposición crece}: Mt 13:55-57; Mc 6:3-4; Lc 4:22b-24.}. En respuesta a las preguntas groseras y a las burlas siniestras, comentó medio en broma: «Sí, soy el hijo de José; soy el carpintero, y no me sorprende que me recordéis el proverbio `Médico, cúrate a ti mismo', ni que me desafiéis para que haga en Nazaret lo que habéis oído decir que realicé en Cafarnaúm; pero os pongo por testigos de que las mismas Escrituras afirman que `a un profeta no le faltan honores, salvo en su propio país y entre su propia gente'.»\footnote{\textit{Jesús cita la escritura}: Jn 4:44.}

\par 
%\textsuperscript{(1686.5)}
\textsuperscript{150:9.2} Pero lo empujaron y, señalándolo con un dedo acusador, le dijeron: «Crees que eres mejor que la gente de Nazaret; te fuiste de aquí, pero tu hermano es un obrero común y tus hermanas viven todavía entre nosotros. Conocemos a tu madre, María. ¿Donde se encuentran hoy? Hemos escuchado grandes cosas sobre ti, pero observamos que no haces ningún prodigio a tu regreso». Jesús les contestó: «Amo a la gente que vive en la ciudad donde crecí, y me regocijaría veros entrar a todos en el reino de los cielos, pero no me corresponde determinar la realización de las obras de Dios. Las transformaciones de la gracia se forjan como respuesta a la fe viviente de aquellos que son sus beneficiarios»\footnote{\textit{Sin milagros}: Mt 13:58; Mc 6:5-6.}.

\par 
%\textsuperscript{(1686.6)}
\textsuperscript{150:9.3} Jesús hubiera manejado amablemente a la multitud y hubiera desarmado eficazmente incluso a sus enemigos más violentos, si uno de sus propios apóstoles, Simón Celotes, no hubiera cometido un grave error táctico. Con la ayuda de Nacor, uno de los evangelistas más jóvenes, Simón había reunido entretanto a un grupo de amigos de Jesús que estaban entre el gentío y, con una actitud agresiva, advirtieron a los enemigos del Maestro que se fueran de allí. Hacía tiempo que Jesús había enseñado a los apóstoles que una respuesta dulce desvía el furor\footnote{\textit{Una respuesta suave aplaca el furor}: Pr 15:1.}, pero sus partidarios no estaban acostumbrados a que trataran a su amado instructor, a quien tan gustosamente llamaban Maestro, con tanta descortesía y desdén. Aquello fue demasiado para ellos y se pusieron a expresar su resentimiento apasionado y vehemente, lo cual no hizo más que encender los ánimos alborotadores de esta asamblea impía y grosera. Y así, bajo la dirección de los mercenarios, aquellos rufianes agarraron a Jesús y lo sacaron precipitadamente de la sinagoga hasta la cima de una escarpada colina cercana, donde estaban dispuestos a empujarlo al vacío para que se estrellara abajo. Pero cuando estaban a punto de empujarlo por el borde del acantilado, Jesús se revolvió de pronto sobre sus captores y, haciéndoles frente, se cruzó tranquilamente de brazos. No dijo nada, pero sus amigos se quedaron más que asombrados cuando empezó a caminar hacia adelante, mientras que el populacho se apartaba y lo dejaba pasar sin molestarlo\footnote{\textit{Jesús escapa de los rufianes}: Lc 4:28-30.}.

\par 
%\textsuperscript{(1687.1)}
\textsuperscript{150:9.4} Jesús, seguido de sus discípulos, se dirigió al campamento, donde refirieron todo lo sucedido. Aquella tarde se prepararon para volver al día siguiente temprano a Cafarnaúm\footnote{\textit{Regreso a Cafarnaúm}: Lc 4:31.}, tal como Jesús lo había ordenado. Este final turbulento de la tercera gira de predicación pública tuvo un efecto de moderación sobre todos los seguidores de Jesús. Empezaron a darse cuenta del significado de algunas enseñanzas del Maestro; estaban despertando al hecho de que el reino sólo se establecería mediante muchas tristezas y amargas desilusiones.

\par 
%\textsuperscript{(1687.2)}
\textsuperscript{150:9.5} Aquel domingo por la mañana abandonaron Nazaret, y después de viajar por caminos diferentes, todos se congregaron finalmente en Betsaida el jueves 10 de marzo al mediodía. Se reunieron como un grupo sobrio y serio de predicadores desilusionados del evangelio de la verdad, y no como un conjunto, entusiasta y conquistador, de cruzados triunfantes.