\chapter{Documento 155. La huida por el norte de Galilea}
\par 
%\textsuperscript{(1725.1)}
\textsuperscript{155:0.1} POCO después de desembarcar cerca de Jeresa este domingo memorable, Jesús y los veinticuatro caminaron un poco hacia el norte, donde pasaron la noche en un hermoso parque al sur de Betsaida-Julias. Conocían bien este lugar para acampar porque se habían detenido allí en el pasado. Antes de retirarse para pasar la noche, el Maestro convocó a sus seguidores alrededor de él y discutió con ellos los planes del viaje que tenían proyectado hasta la costa de Fenicia, pasando por Batanea y el norte de Galilea.

\section*{1. ¿Por qué están furiosos los paganos?}
\par 
%\textsuperscript{(1725.2)}
\textsuperscript{155:1.1} Jesús dijo: <<Todos deberíais recordar lo que el salmista indicó sobre estos tiempos, cuando dijo: `¿Por qué están furiosos los paganos y los pueblos conspiran en vano? Los reyes de la Tierra se establecen a sí mismos, y los dirigentes del pueblo se aconsejan entre ellos, en contra del Señor y en contra de su ungido, diciendo: Rompamos las ataduras de la misericordia y desechemos los lazos del amor.'>>

\par 
%\textsuperscript{(1725.3)}
\textsuperscript{155:1.2} <<Hoy veis que esto se cumple delante de vuestros ojos. Pero no veréis cumplirse el resto de la profecía del salmista, porque tenía ideas erróneas sobre el Hijo del Hombre y su misión en la Tierra. Mi reino está basado en el amor, es proclamado con misericordia y se establece mediante el servicio desinteresado. Mi Padre no está sentado en el cielo riéndose burlonamente de los paganos. No está colérico en su gran desagrado. Es verdad la promesa de que el Hijo recibirá por herencia a esos llamados paganos (en realidad, sus hermanos ignorantes y faltos de instrucción). Y recibiré a esos gentiles con los brazos abiertos, con misericordia y afecto. Se mostrará toda esta misericordia a los supuestos paganos, a pesar de la desacertada declaración de la escritura que insinúa que el Hijo triunfante `los quebrantará con una barra de hierro y los hará añicos como una vasija de alfarero.' El salmista os exhortaba a: `servir al Señor con temor' ---Yo os invito a que entréis, por la fe, en los elevados privilegios de la filiación divina; él os ordena que os regocijéis temblando. Yo os pido que os regocijéis con seguridad. Él dice: `Besad al Hijo, no sea que se irrite y perezcáis cuando se encienda su cólera.' Pero vosotros, que habéis vivido conmigo, sabéis muy bien que la ira y la cólera no forman parte del establecimiento del reino de los cielos en el corazón de los hombres. Sin embargo, el salmista vislumbró la verdadera luz cuando dijo al final de esta exhortación: `Benditos sean los que ponen su confianza en este Hijo.'>>

\par 
%\textsuperscript{(1725.4)}
\textsuperscript{155:1.3} Jesús continuó enseñando a los veinticuatro, diciendo: <<Los paganos no están faltos de razón al estar furiosos con nosotros. Como su concepto de la vida es limitado y estrecho, pueden concentrar sus energías con entusiasmo. Tienen una meta cercana y más o menos visible; por eso se esfuerzan con una destreza valiente y eficaz. Vosotros, que habéis confesado vuestra entrada en el reino de los cielos, sois en general demasiado vacilantes e imprecisos en vuestra manera de enseñar. Los paganos se dirigen directamente hacia sus objetivos; vosotros sois culpables de tener demasiados anhelos crónicos. Si deseáis entrar en el reino, ¿por qué no os apoderáis de él mediante un asalto espiritual, como los paganos se apoderan de una ciudad sitiada? Difícilmente sois dignos del reino cuando vuestro servicio consiste tan ampliamente en la actitud de lamentaros del pasado, quejaros del presente y tener una esperanza vana para el futuro. ¿Por qué están furiosos los paganos? Porque no conocen la verdad. ¿Por qué languidecéis en anhelos fútiles? Porque no \textit{obedecéis} a la verdad. Poned fin a vuestras ansias inútiles y salid a hacer valientemente lo que está relacionado con el establecimiento del reino>>.

\par 
%\textsuperscript{(1726.1)}
\textsuperscript{155:1.4} <<En todo lo que hagáis, no os volváis parciales y no os especialicéis con exceso. Los fariseos que buscan nuestra destrucción creen de verdad que están sirviendo a Dios. La tradición los ha limitado tanto, que están cegados por los prejuicios y endurecidos por el miedo. Contemplad a los griegos, que tienen una ciencia sin religión, mientras que los judíos tienen una religión desprovista de ciencia. Cuando los hombres se extravían de esta manera, aceptando una desintegración estrecha y confusa de la verdad, su única esperanza de salvación consiste en coordinarse con la verdad ---en convertirse>>.

\par 
%\textsuperscript{(1726.2)}
\textsuperscript{155:1.5} <<Dejadme expresar enérgicamente esta verdad eterna: Si gracias a vuestra coordinación con la verdad, aprendéis a manifestar en vuestra vida esta hermosa integridad de la rectitud, entonces vuestros semejantes os buscarán para conseguir lo que habéis adquirido así. La cantidad de buscadores de la verdad que se sentirán atraídos hacia vosotros representa la medida de vuestra dotación de la verdad, de vuestra rectitud. La cantidad de mensaje que tenéis que llevar a la gente es, en cierto modo, la medida de vuestro fracaso en vivir la vida plena o recta, la vida coordinada con la verdad>>.

\par 
%\textsuperscript{(1726.3)}
\textsuperscript{155:1.6} El Maestro enseñó otras muchas cosas a sus apóstoles y a los evangelistas antes de que le desearan las buenas noches y se retiraran a descansar.

\section*{2. Los evangelistas en Corazín}
\par 
%\textsuperscript{(1726.4)}
\textsuperscript{155:2.1} El lunes 23 de mayo por la mañana, Jesús ordenó a Pedro que fuera a Corazín con los doce evangelistas, mientras que él partía con los once hacia Cesarea de Filipo, dirigiéndose por la ruta del Jordán hasta la carretera de Damasco a Cafarnaúm; desde allí fueron por el nordeste hasta la unión con la carretera de Cesarea de Filipo, continuando luego hasta esta ciudad, donde se detuvieron y enseñaron durante dos semanas. Llegaron en el transcurso de la tarde del martes 24 de mayo.

\par 
%\textsuperscript{(1726.5)}
\textsuperscript{155:2.2} Pedro y los evangelistas permanecieron dos semanas en Corazín, predicando el evangelio del reino a un grupo de creyentes poco numeroso, pero serio. No pudieron conseguir muchos nuevos conversos. Ninguna otra ciudad de Galilea dio menos almas al reino que Corazín. Siguiendo las instrucciones de Pedro, los doce evangelistas hablaron menos sobre las curaciones ---las cosas físicas--- y se dedicaron a predicar y a enseñar, con un vigor acrecentado, las verdades espirituales del reino celestial. Estas dos semanas en Corazín constituyeron un verdadero bautismo de adversidad para los doce evangelistas, en el sentido de que éste fue el período más difícil e improductivo de su carrera hasta ese momento. Al sentirse privados así de la satisfacción de conseguir almas para el reino, cada uno de ellos hizo un inventario más serio y honrado de su propia alma y del progreso de ésta en los senderos espirituales de la nueva vida.

\par 
%\textsuperscript{(1726.6)}
\textsuperscript{155:2.3} Cuando se hizo evidente que no había más gente que estuviera dispuesta a intentar entrar en el reino, Pedro convocó a sus compañeros, el martes 7 de junio, y partió para reunirse con Jesús y los apóstoles en Cesarea de Filipo. Llegaron el miércoles alrededor del mediodía y pasaron toda la tarde narrando sus experiencias con los incrédulos de Corazín. Durante las discusiones de esta tarde, Jesús se refirió de nuevo a la parábola del sembrador y les enseñó muchas cosas sobre el significado de los fracasos aparentes en las empresas de la vida.

\section*{3. En Cesarea de Filipo}
\par 
%\textsuperscript{(1727.1)}
\textsuperscript{155:3.1} Aunque Jesús no efectuó ninguna labor pública durante esta estancia de dos semanas cerca de Cesarea de Filipo, los apóstoles celebraron por las tardes numerosas reuniones tranquilas en la ciudad; muchos creyentes fueron hasta el campamento para hablar con el Maestro, pero muy pocos de ellos fueron agregados al grupo de creyentes como resultado de esta visita. Jesús conversó diariamente con los apóstoles y éstos discernieron con más claridad que ahora estaba empezando una nueva fase de la tarea de predicar el reino de los cielos. Estaban empezando a comprender que el <<reino de los cielos no es comida y bebida, sino la realización de la alegría espiritual de aceptar la filiación divina>>.

\par 
%\textsuperscript{(1727.2)}
\textsuperscript{155:3.2} La estancia en Cesarea de Filipo fue una verdadera prueba para los once apóstoles; fueron dos semanas difíciles de pasar para todos ellos. Estaban casi deprimidos, y echaban de menos el estímulo periódico de la personalidad entusiasta de Pedro. En aquellos momentos, el hecho de creer en Jesús y de ponerse a seguirlo era realmente una gran aventura y una prueba. Hicieron pocas conversiones durante estas dos semanas, pero aprendieron muchas cosas de sus conferencias diarias con el Maestro, que fueron muy beneficiosas para ellos.

\par 
%\textsuperscript{(1727.3)}
\textsuperscript{155:3.3} Los apóstoles aprendieron que los judíos estaban espiritualmente estancados y moribundos porque habían cristalizado la verdad en un credo; que cuando se formula la verdad como una línea divisoria de exclusivismo presuntuoso, en lugar de servir como un poste indicador para la orientación y el progreso espiritual, dichas enseñanzas pierden su poder creativo y vivificante, y acaban por volverse simplemente conservadoras y fosilizantes.

\par 
%\textsuperscript{(1727.4)}
\textsuperscript{155:3.4} Aprendieron cada vez más de Jesús a considerar a las personalidades humanas en términos de sus posibilidades en el tiempo y en la eternidad. Aprendieron que a muchas almas se les puede inducir mejor a amar al Dios invisible, si primero se les enseña a amar a sus hermanos que pueden ver. En relación con estas lecciones, se atribuyó un nuevo significado a la declaración del Maestro sobre el servicio desinteresado a los semejantes: <<Puesto que lo habéis hecho por el más humilde de mis hermanos, lo habéis hecho por mí>>.

\par 
%\textsuperscript{(1727.5)}
\textsuperscript{155:3.5} Una de las grandes lecciones de esta estancia en Cesarea tuvo que ver con el origen de las tradiciones religiosas, con el grave peligro de permitir que se atribuya un carácter sagrado a las cosas no sagradas, a las ideas corrientes o a los acontecimientos cotidianos. De una de estas conferencias salieron con la enseñanza de que la verdadera religión es la lealtad que un hombre siente en el fondo de su corazón hacia sus convicciones más elevadas y más sinceras.

\par 
%\textsuperscript{(1727.6)}
\textsuperscript{155:3.6} Jesús advirtió a sus creyentes que, si sus anhelos religiosos eran únicamente materiales, el conocimiento creciente de la naturaleza acabaría por quitarles su fe en Dios, debido a la sustitución progresiva del origen supuestamente sobrenatural de las cosas. Pero si su religión era espiritual, el progreso de la ciencia física nunca podría perturbar su fe en las realidades eternas y en los valores divinos.

\par 
%\textsuperscript{(1727.7)}
\textsuperscript{155:3.7} Aprendieron que cuando la religión tiene unos móviles enteramente espirituales, hace que toda la vida valga más la pena, llenándola de objetivos elevados, dignificándola con valores transcendentales, inspirándola con móviles magníficos, y confortando todo el tiempo el alma humana con una esperanza sublime y vigorizante. La verdadera religión está destinada a disminuir las tensiones de la existencia; libera la fe y el coraje para la vida diaria y el servicio desinteresado. La fe fomenta la vitalidad espiritual y la fecundidad de la rectitud.

\par 
%\textsuperscript{(1727.8)}
\textsuperscript{155:3.8} Jesús enseñó repetidas veces a sus apóstoles que ninguna civilización puede sobrevivir mucho tiempo a la pérdida de las mejores cosas que posee su religión. Nunca se cansó de señalar a los doce el gran peligro que supone aceptar los símbolos y las ceremonias religiosos como sustitutos de la experiencia religiosa. Toda su vida terrestre estuvo firmemente consagrada a la misión de derretir las formas congeladas de la religión, para darles las libertades líquidas de una filiación iluminada.

\section*{4. En el camino de Fenicia}
\par 
%\textsuperscript{(1728.1)}
\textsuperscript{155:4.1} El jueves 9 de junio por la mañana, después de que los mensajeros de David trajeran de Betsaida las noticias relacionadas con el progreso del reino, este grupo de veinticinco instructores de la verdad abandonó Cesarea de Filipo para emprender su viaje hacia la costa de Fenicia. Rodearon la región pantanosa, pasando por Luz, hasta el empalme con el camino de Magdala hacia el Monte Líbano, y desde allí hasta el cruce con la carretera que conducía a Sidón, donde llegaron el viernes por la tarde.

\par 
%\textsuperscript{(1728.2)}
\textsuperscript{155:4.2} Mientras se detenían para almorzar a la sombra de una cornisa rocosa inclinada, cerca de Luz, Jesús pronunció uno de los discursos más notables que sus apóstoles hubieran escuchado nunca a lo largo de todos sus años de asociación con él. Apenas se habían sentado para partir el pan, Simón Pedro le preguntó a Jesús: <<Maestro, puesto que el Padre que está en los cielos lo sabe todo, y puesto que su espíritu es nuestro sostén para establecer el reino de los cielos en la Tierra, ¿cómo es que huimos de las amenazas de nuestros enemigos? ¿Por qué nos negamos a enfrentarnos con los enemigos de la verdad?>> Pero antes de que Jesús hubiera empezado a contestar la pregunta de Pedro, Tomás interrumpió para interrogar: <<Maestro, me gustaría saber realmente qué hay exactamente de erróneo en la religión de nuestros enemigos de Jerusalén. ¿Cuál es la diferencia real entre su religión y la nuestra? ¿Cómo puede ser que tengamos tanta diversidad de creencias si todos profesamos servir al mismo Dios?>> Cuando Tomás hubo terminado, Jesús dijo: <<No deseo ignorar la pregunta de Pedro, porque sé muy bien lo fácil que es malinterpretar mis razones para evitar un choque abierto con los jefes de los judíos en este preciso momento; pero sin embargo, será más útil para todos vosotros que elija contestar más bien la pregunta de Tomás. Y eso es lo que voy a hacer cuando hayáis terminado de almorzar>>.

\section*{5. El discurso sobre la verdadera religión}
\par 
%\textsuperscript{(1728.3)}
\textsuperscript{155:5.1} Este discurso memorable sobre la religión, resumido y expuesto de nuevo en un lenguaje moderno, expresó las verdades siguientes:

\par 
%\textsuperscript{(1728.4)}
\textsuperscript{155:5.2} Aunque las religiones del mundo tienen un origen doble ---natural y revelado--- en todo momento se pueden encontrar, en cualquier pueblo, tres formas distintas de devoción religiosa. Estas tres manifestaciones del impulso religioso son:

\par 
%\textsuperscript{(1728.5)}
\textsuperscript{155:5.3} 1. \textit{La religión primitiva}. La propensión seminatural e instintiva a tener miedo de las energías misteriosas y a adorar las fuerzas superiores; es principalmente una religión de la naturaleza física, la religión del miedo.

\par 
%\textsuperscript{(1728.6)}
\textsuperscript{155:5.4} 2. \textit{La religión de la civilización}. Los conceptos y las prácticas religiosos progresivos de las razas que se civilizan ---la religión de la mente--- la teología intelectual basada en la autoridad de la tradición religiosa establecida.

\par 
%\textsuperscript{(1728.7)}
\textsuperscript{155:5.5} 3. \textit{La verdadera religión} ---\textit{la religión de la revelación}. La revelación de los valores sobrenaturales, un atisbo parcial de las realidades eternas, un vislumbre de la bondad y la belleza del carácter infinito del Padre que está en los cielos ---la religión del espíritu tal como está demostrada en la experiencia humana.

\par 
%\textsuperscript{(1729.1)}
\textsuperscript{155:5.6} El Maestro se negó a menospreciar la religión de los sentidos físicos y de los temores supersticiosos del hombre común, aunque deploró el hecho de que sobrevivieran tantos elementos de esta forma primitiva de adoración en las prácticas religiosas de las razas más inteligentes de la humanidad. Jesús indicó claramente que la gran diferencia entre la religión de la mente y la religión del espíritu reside en que, mientras la primera está sostenida por la autoridad eclesiástica, la segunda está enteramente basada en la experiencia humana.

\par 
%\textsuperscript{(1729.2)}
\textsuperscript{155:5.7} Luego, durante su hora de enseñanza, el Maestro continuó aclarando las verdades siguientes:

\par 
%\textsuperscript{(1729.3)}
\textsuperscript{155:5.8} Hasta que las razas se vuelvan sumamente inteligentes y más completamente civilizadas, seguirán existiendo muchas de esas ceremonias infantiles y supersticiosas que son tan características de las prácticas religiosas evolutivas de los pueblos primitivos y atrasados. Hasta que la raza humana no alcance el nivel de un reconocimiento más elevado y más general de las realidades de la experiencia espiritual, un gran número de hombres y mujeres continuarán mostrando su preferencia personal por esas religiones de autoridad que sólo requieren un asentimiento intelectual, en contraste con la religión del espíritu, que implica una participación activa de la mente y del alma en la aventura de la fe consistente en luchar con las realidades rigurosas de la experiencia humana progresiva.

\par 
%\textsuperscript{(1729.4)}
\textsuperscript{155:5.9} La aceptación de las religiones tradicionales de autoridad representa la salida fácil para el impulso que siente el hombre de intentar satisfacer las ansias de su naturaleza espiritual. Las religiones de autoridad, asentadas, cristalizadas y establecidas, proporcionan un refugio disponible donde el alma trastornada y angustiada del hombre puede huir cuando se siente abrumada por el miedo y atormentada por la incertidumbre. Como precio a pagar por las satisfacciones y las seguridades que proporciona, una religión así sólo exige a sus devotos un asentimiento pasivo y puramente intelectual.

\par 
%\textsuperscript{(1729.5)}
\textsuperscript{155:5.10} En la Tierra vivirán durante mucho tiempo esos individuos tímidos, miedosos e indecisos que preferirán obtener de esta manera sus consuelos religiosos, aunque al ligar su suerte con las religiones de autoridad, comprometen la soberanía de su personalidad, degradan la dignidad de la autoestima, y renuncian por completo al derecho de participar en la más emocionante e inspiradora de todas las experiencias humanas posibles: la búsqueda personal de la verdad, el regocijo de afrontar los peligros del descubrimiento intelectual, la determinación de explorar las realidades de la experiencia religiosa personal, la satisfacción suprema de experimentar el triunfo personal de conseguir realmente la victoria de la fe espiritual sobre las dudas intelectuales, una victoria que se gana honradamente durante la aventura suprema de toda la existencia humana ---el hombre a la búsqueda de Dios, por sí mismo y como tal hombre, y que lo encuentra.

\par 
%\textsuperscript{(1729.6)}
\textsuperscript{155:5.11} La religión del espíritu significa esfuerzo, lucha, conflicto, fe, determinación, amor, lealtad y progreso. La religión de la mente ---la teología de la autoridad--- exige pocos o ninguno de estos esfuerzos a sus creyentes formales. La tradición es un refugio seguro y un sendero fácil para las almas temerosas y sin entusiasmo que rehuyen instintivamente las luchas espirituales y las incertidumbres mentales que acompañan a esos viajes, en la fe, de aventuras atrevidas por los altos mares de la verdad inexplorada, en búsqueda de las orillas muy lejanas de las realidades espirituales, tal como éstas pueden ser descubiertas por la mente humana progresiva, y experimentadas por el alma humana en evolución.

\par 
%\textsuperscript{(1729.7)}
\textsuperscript{155:5.12} Jesús continuó diciendo: <<En Jerusalén, los jefes religiosos han formulado un sistema establecido de creencias intelectuales, una religión de autoridad, con las diversas doctrinas de sus instructores tradicionales y de los profetas de antaño. Todo ese tipo de religiones recurre principalmente a la mente. Ahora estamos a punto de entrar en un conflicto implacable con ese tipo de religión, puesto que muy pronto vamos a empezar a proclamar audazmente una nueva religión ---una religión que no es una religión en el sentido que hoy se atribuye a esa palabra, una religión que apela principalmente al espíritu divino de mi Padre que reside en la mente del hombre; una religión que obtendrá su autoridad de los frutos de su aceptación, unos frutos que aparecerán con toda seguridad en la experiencia personal de todos los que se conviertan en creyentes reales y sinceros de las verdades de esta comunión espiritual superior>>.

\par 
%\textsuperscript{(1730.1)}
\textsuperscript{155:5.13} Señalando a cada uno de los veinticuatro y llamándolos por su nombre, Jesús dijo: <<Y ahora, ¿quién de vosotros preferiría coger ese sendero fácil del conformismo a una religión establecida y fosilizada, como la que defienden los fariseos de Jerusalén, en lugar de sufrir las dificultades y persecuciones que acompañarán la misión de proclamar un camino mejor de salvación para los hombres, mientras obtenéis la satisfacción de descubrir, por vosotros mismos, las bellezas de las realidades de una experiencia viviente y personal de las verdades eternas y de las grandezas supremas del reino de los cielos? ¿Sois miedosos, blandos y buscáis la facilidad? ¿Tenéis miedo de confiar vuestro futuro entre las manos del Dios de la verdad, de quien sois hijos? ¿Desconfiáis del Padre, de quien sois hijos? ¿Vais a retroceder al sendero fácil de la certidumbre y de la estabilidad intelectual de la religión de autoridad tradicional, o vais a ceñiros para avanzar conmigo en el futuro incierto y agitado en el que proclamaremos las verdades nuevas de la religión del espíritu, el reino de los cielos en el corazón de los hombres?>>

\par 
%\textsuperscript{(1730.2)}
\textsuperscript{155:5.14} Sus veinticuatro oyentes se pusieron todos de pie con la intención de anunciar su respuesta unánime y leal a este llamamiemto emotivo, uno de los pocos que Jesús les hizo nunca, pero él levantó la mano y los detuvo, diciendo: <<Separaos ahora; que cada uno se quede a solas con el Padre, y encuentre allí la respuesta no emotiva a mi pregunta. Una vez que hayáis descubierto la actitud verdadera y sincera de vuestra alma, expresad esa respuesta de manera franca y audaz a mi Padre y vuestro Padre, cuya vida infinita de amor es el espíritu mismo de la religión que proclamamos>>.

\par 
%\textsuperscript{(1730.3)}
\textsuperscript{155:5.15} Los evangelistas y los apóstoles se separaron cada uno por su lado durante un corto período de tiempo. Tenían el espíritu elevado, la mente inspirada y las emociones poderosamente agitadas por las palabras de Jesús. Sin embargo, cuando Andrés los reunió, el Maestro se limitó a decir: <<Reanudemos nuestro viaje. Vamos a Fenicia para quedarnos una temporada, y todos deberíais orar al Padre para que transforme vuestras emociones mentales y corporales en lealtades mentales superiores y en experiencias espirituales más satisfactorias>>.

\par 
%\textsuperscript{(1730.4)}
\textsuperscript{155:5.16} Los veinticuatro permanecieron silenciosos mientras bajaban por el camino, pero pronto empezaron a charlar entre ellos, y a las tres de la tarde ya no pudieron aguantar más. Se detuvieron, y Pedro se acercó a Jesús, diciendo: <<Maestro, nos has dirigido palabras de vida y de verdad. Quisiéramos escuchar más; te rogamos que continúes hablándonos de estas materias>>.

\section*{6. El segundo discurso sobre la religión}
\par 
%\textsuperscript{(1730.5)}
\textsuperscript{155:6.1} Y así, mientras hacían una pausa a la sombra de una ladera, Jesús continuó enseñándoles acerca de la religión del espíritu, diciendo en esencia:

\par 
%\textsuperscript{(1730.6)}
\textsuperscript{155:6.2} Habéis surgido de entre aquellos semejantes vuestros que han elegido permanecer satisfechos con una religión de la mente, que ansían la seguridad y prefieren el conformismo. Habéis elegido cambiar vuestros sentimientos de certidumbre autoritaria por las seguridades del espíritu de una fe aventurera y progresiva. Os habéis atrevido a protestar contra la esclavitud abrumadora de una religión institucional y a rechazar la autoridad de las tradiciones escritas actualmente consideradas como la palabra de Dios. Nuestro Padre habló en verdad a través de Moisés, Elías, Isaías, Amós y Oseas, pero no ha dejado de suministrar al mundo palabras de verdad cuando estos antiguos profetas terminaron sus proclamaciones. Mi Padre no hace acepción de razas ni de generaciones, en el sentido de que la palabra de la verdad sea otorgada a una época y ocultada a la siguiente. No cometáis la locura de llamar divino a lo que es puramente humano, y no dejéis de discernir las palabras de la verdad, aunque no provengan de los oráculos tradicionales de una supuesta inspiración.

\par 
%\textsuperscript{(1731.1)}
\textsuperscript{155:6.3} Os he llamado para que nazcáis de nuevo, para que nazcáis del espíritu. Os he llamado para que salgáis de las tinieblas de la autoridad y del letargo de la tradición, y entréis en la luz trascendente donde obtendréis la posibilidad de hacer por vosotros mismos el mayor descubrimiento posible que el alma humana puede hacer ---la experiencia celestial de encontrar a Dios por vosotros mismos, en vosotros mismos y para vosotros mismos, y efectuar todo esto como un hecho en vuestra propia experiencia personal. Así podréis pasar de la muerte a la vida, de la autoridad de la tradición a la experiencia de conocer a Dios; así pasaréis de las tinieblas a la luz, de una fe racial heredada a una fe personal conseguida mediante una experiencia real; de este modo progresaréis de una teología de la mente, transmitida por vuestros antepasados, a una verdadera religión del espíritu que será edificada en vuestra alma como una dotación eterna.

\par 
%\textsuperscript{(1731.2)}
\textsuperscript{155:6.4} Vuestra religión dejará de ser una simple creencia intelectual en una autoridad tradicional, para convertirse en la experiencia efectiva de esa fe viviente que es capaz de captar la realidad de Dios y todo lo relacionado con el espíritu divino del Padre. La religión de la mente os ata sin esperanzas al pasado; la religión del espíritu consiste en una revelación progresiva y os llama constantemente para que consigáis unos ideales espirituales y unas realidades eternas más elevados y más santos.

\par 
%\textsuperscript{(1731.3)}
\textsuperscript{155:6.5} Aunque la religión de autoridad puede conferir un sentimiento inmediato de seguridad estable, el precio que pagáis por esa satisfacción pasajera es la pérdida de vuestra independencia espiritual y de vuestra libertad religiosa. Como precio para entrar en el reino de los cielos, mi Padre no os exige que os forcéis a creer en cosas que son espiritualmente repugnantes, impías y falsas. No se os pide que ultrajéis vuestro propio sentido de la misericordia, de la justicia y de la verdad por medio de vuestro sometimiento a un sistema anticuado de formalidades y de ceremonias religiosas. La religión del espíritu os deja eternamente libres para seguir la verdad, dondequiera que os lleven las directrices del espíritu. ¿Y quién puede juzgar ---quizás este espíritu tenga algo que comunicar a esta generación, que las otras generaciones han rehusado escuchar?

\par 
%\textsuperscript{(1731.4)}
\textsuperscript{155:6.6} ¡Verg\"uenza deberían sentir esos falsos educadores religiosos, que quisieran arrastrar a las almas hambrientas al oscuro y lejano pasado, para luego abandonarlas allí! Esas personas desgraciadas están condenadas así a asustarse de todo nuevo descubrimiento, y a sentirse desconcertadas con cada nueva revelación de la verdad. El profeta que dijo: <<Aquel cuya mente descansa en Dios se mantendrá en una paz perfecta>> no era un simple creyente intelectual en una teología autoritaria. Este ser humano, que conocía la verdad, había descubierto a Dios; no se limitaba a hablar de Dios.

\par 
%\textsuperscript{(1731.5)}
\textsuperscript{155:6.7} Os recomiendo que abandonéis la costumbre de citar constantemente a los profetas del pasado y de alabar a los héroes de Israel; aspirad más bien a convertiros en profetas vivientes del Altísimo y en héroes espirituales del reino venidero. En verdad, quizás valga la pena honrar a los jefes del pasado que conocían a Dios, pero cuando lo hagáis, ¿por qué tenéis que sacrificar la experiencia suprema de la existencia humana: encontrar a Dios por vosotros mismos y conocerlo en vuestra propia alma?

\par 
%\textsuperscript{(1732.1)}
\textsuperscript{155:6.8} Cada raza de la humanidad tiene su propia perspectiva mental sobre la existencia humana; por consiguiente, la religión de la mente debe siempre armonizarse con estos diversos puntos de vista raciales. Las religiones de autoridad nunca podrán llegar a unificarse. La unidad humana y la fraternidad de los mortales sólo se pueden conseguir por medio, y a través de, la dotación superior de la religión del espíritu. Las mentes de las razas pueden ser diferentes, pero toda la humanidad está habitada por el mismo espíritu divino y eterno. La esperanza de la fraternidad humana sólo se puede realizar cuando, y a medida que, la religión unificante y ennoblecedora del espíritu ---la religión de la experiencia espiritual personal--- impregne y eclipse a las religiones de autoridad mentales y divergentes.

\par 
%\textsuperscript{(1732.2)}
\textsuperscript{155:6.9} Las religiones de autoridad sólo pueden dividir a los hombres y levantar unas conciencias contra otras; la religión del espíritu unirá progresivamente a los hombres y los inducirá a sentir una simpatía comprensiva los unos por los otros. Las religiones de autoridad exigen a los hombres una creencia uniforme, pero esto es imposible de realizar en el estado actual del mundo. La religión del espíritu sólo exige una unidad de experiencia ---un destino uniforme--- aceptando plenamente la diversidad de creencias. La religión del espíritu sólo pide la uniformidad de perspicacia, no la uniformidad de punto de vista ni de perspectiva. La religión del espíritu no exige la uniformidad de puntos de vista intelectuales, sino solamente la unidad de sentimientos espirituales. Las religiones de autoridad se cristalizan en credos sin vida; la religión del espíritu se desarrolla en la alegría y la libertad crecientes de las acciones ennoblecedoras del servicio amoroso y de la ayuda misericordiosa.

\par 
%\textsuperscript{(1732.3)}
\textsuperscript{155:6.10} Pero tened cuidado, no sea que alguno de vosotros considere con desdén a los hijos de Abraham porque les ha tocado vivir en estos malos tiempos de tradición estéril. Nuestros antepasados se dedicaron de lleno a la búsqueda insistente y apasionada de Dios, y lo descubrieron como ninguna otra raza total de hombres lo ha conocido nunca desde los tiempos de Adán, que sabía muchas de estas cosas, porque él mismo era un Hijo de Dios. Mi Padre no ha dejado de observar la larga e incansable lucha de Israel, desde la época de Moisés, por encontrar y conocer a Dios. Durante largas generaciones, los judíos no han dejado de afanarse, sudar, gemir, trabajar, soportar los sufrimientos y experimentar las tristezas de un pueblo incomprendido y despreciado, y todo ello para poder acercarse un poco más al descubrimiento de la verdad acerca de Dios. Desde Moisés hasta los tiempos de Amós y de Oseas, y a pesar de todos los fracasos y titubeos de Israel, nuestros padres revelaron progresivamente a todo el mundo una imagen cada vez más clara y más verdadera del Dios eterno. Así es como se preparó el camino para la revelación aún más grande del Padre, en la que habéis sido llamados a participar.

\par 
%\textsuperscript{(1732.4)}
\textsuperscript{155:6.11} No olvidéis nunca que sólo hay una aventura más satisfactoria y emocionante que la tentativa de descubrir la voluntad del Dios vivo, y es la experiencia suprema de intentar hacer honradamente esa voluntad divina. Y recordad siempre que la voluntad de Dios se puede hacer en cualquier ocupación terrestre. No hay profesiones santas y profesiones laicas. Todas las cosas son sagradas en la vida de aquellos que están dirigidos por el espíritu, es decir, subordinados a la verdad, ennoblecidos por el amor, dominados por la misericordia y refrenados por la equidad ---por la justicia. El espíritu que mi Padre y yo enviaremos al mundo no es solamente el Espíritu de la Verdad, sino también el espíritu de la belleza idealista.

\par 
%\textsuperscript{(1732.5)}
\textsuperscript{155:6.12} Tenéis que dejar de buscar la palabra de Dios únicamente en las páginas de los viejos escritos con autoridad teológica. Aquellos que han nacido del espíritu de Dios discernirán en lo sucesivo la palabra de Dios, independientemente del lugar de donde parezca originarse. No hay que desestimar la verdad divina porque se haya otorgado a través de un canal aparentemente humano. Muchos de vuestros hermanos aceptan mentalmente la teoría de Dios, pero no consiguen darse cuenta espiritualmente de la presencia de Dios. Ésta es precisamente la razón por la que os he enseñado tantas veces que la mejor manera de comprender el reino de los cielos es adquiriendo la actitud espiritual de un niño sincero. No os recomiendo la inmadurez mental de un niño, sino más bien la \textit{ingenuidad espiritual} de un pequeño que cree con facilidad y que confía plenamente. No es tan importante que conozcáis el hecho de Dios, como que desarrolléis cada vez más la habilidad de \textit{sentir la presencia de Dios}.

\par 
%\textsuperscript{(1733.1)}
\textsuperscript{155:6.13} Una vez que empecéis a descubrir a Dios en vuestra alma, no tardaréis en empezar a descubrirlo en el alma de los otros hombres, y finalmente en todas las criaturas y creaciones de un poderoso universo. Pero ¿qué posibilidades tiene el Padre de aparecer, como el Dios de las lealtades supremas y de los ideales divinos, en el alma de unos hombres que dedican poco o ningún tiempo a la contemplación reflexiva de estas realidades eternas? Aunque la mente no es la sede de la naturaleza espiritual, es en verdad la entrada que conduce a ella.

\par 
%\textsuperscript{(1733.2)}
\textsuperscript{155:6.14} Pero no cometáis el error de intentar probar a otros hombres que habéis encontrado a Dios; no podéis presentar conscientemente una prueba así de válida, aunque existen dos demostraciones positivas y poderosas del hecho de que conocéis a Dios, y son las siguientes:

\par 
%\textsuperscript{(1733.3)}
\textsuperscript{155:6.15} 1. La manifestación de los frutos del espíritu de Dios en vuestra vida diaria habitual.

\par 
%\textsuperscript{(1733.4)}
\textsuperscript{155:6.16} 2. El hecho de que todo el plan de vuestra vida proporciona una prueba positiva de que habéis arriesgado sin reserva todo lo que sois y poseéis en la aventura de la supervivencia después de la muerte, persiguiendo la esperanza de encontrar al Dios de la eternidad, cuya presencia habéis saboreado anticipadamente en el tiempo.

\par 
%\textsuperscript{(1733.5)}
\textsuperscript{155:6.17} Y ahora, no os equivoquéis, mi Padre responderá siempre a la más tenue llama vacilante de fe. Él toma nota de las emociones físicas y supersticiosas del hombre primitivo. Y con esas almas honradas pero temerosas, cuya fe es tan débil que no llega a ser mucho más que un conformismo intelectual a una actitud pasiva de asentimiento a las religiones de autoridad, el Padre siempre está alerta para honrar y fomentar incluso todas estas débiles tentativas por llegar hasta él. Pero se espera que vosotros, que habéis sido sacados de las tinieblas y traídos a la luz, creáis de todo corazón; vuestra fe dominará las actitudes combinadas del cuerpo, la mente y el espíritu.

\par 
%\textsuperscript{(1733.6)}
\textsuperscript{155:6.18} Vosotros sois mis apóstoles, y la religión no se convertirá para vosotros en un refugio teológico al que podréis huir cuando temáis enfrentaros con las duras realidades del progreso espiritual y de la aventura idealista. Vuestra religión se convertirá más bien en el hecho de una experiencia real que atestigua que Dios os ha encontrado, idealizado, ennoblecido y espiritualizado, y que os habéis alistado en la aventura eterna de encontrar al Dios que así os ha encontrado y os ha hecho hijos suyos.

\par 
%\textsuperscript{(1733.7)}
\textsuperscript{155:6.19} Cuando Jesús terminó de hablar, hizo una seña a Andrés, apuntó hacia el oeste en dirección a Fenicia, y dijo: <<Pongámonos en camino>>.