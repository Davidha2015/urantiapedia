\chapter{Documento 156. La estancia en Tiro y Sidón}
\par 
%\textsuperscript{(1734.1)}
\textsuperscript{156:0.1} EL VIERNES 10 de junio por la tarde, Jesús y sus compañeros llegaron a las cercanías de Sidón, donde se detuvieron en la casa de una mujer rica que había sido paciente en el hospital de Betsaida durante la época en que Jesús se encontraba en la cumbre del favor popular. Los evangelistas y los apóstoles se alojaron con unos amigos de ella en las proximidades inmediatas, y descansaron el día del sábado en medio de estos paisajes vivificantes. Pasaron casi dos semanas y media en Sidón y sus cercanías antes de prepararse para visitar las ciudades costeras del norte.

\par 
%\textsuperscript{(1734.2)}
\textsuperscript{156:0.2} Este sábado de junio fue un día muy tranquilo. Los evangelistas y los apóstoles estaban totalmente absortos en sus meditaciones sobre los discursos del Maestro acerca de la religión, que habían escuchado en el camino hacia Sidón. Todos eran capaces de apreciar algo de lo que Jesús les había dicho, pero ninguno de ellos captaba plenamente la importancia de su enseñanza.

\section*{1. La mujer siria}
\par 
%\textsuperscript{(1734.3)}
\textsuperscript{156:1.1} Cerca de la casa de Karuska, donde se alojaba el Maestro, vivía una mujer siria que había oído hablar mucho de Jesús como gran sanador e instructor, y este sábado por la tarde vino a verlo con su hijita. La chica, que tenía unos doce años de edad, estaba afligida con un doloroso trastorno nervioso caracterizado por convulsiones y otras manifestaciones angustiosas.

\par 
%\textsuperscript{(1734.4)}
\textsuperscript{156:1.2} Jesús había encargado a sus asociados que no informaran a nadie de su presencia en la casa de Karuska, explicando que deseaba descansar. Aunque habían obedecido las instrucciones de su Maestro, la criada de Karuska había ido a la casa de esta mujer siria, llamada Norana, para informarle que Jesús estaba alojado en la casa de su ama, y había incitado a esta madre ansiosa a que llevara a su hija afligida para que la curara. Esta madre creía, por supuesto, que su hija estaba poseída por un demonio, por un espíritu impuro.

\par 
%\textsuperscript{(1734.5)}
\textsuperscript{156:1.3} Cuando Norana llegó con su hija, los gemelos Alfeo le explicaron, por medio de un intérprete, que el Maestro estaba descansando y que no se le podía molestar, a lo cual Norana replicó que se quedaría allí con la niña hasta que el Maestro hubiera terminado su descanso. Pedro también intentó razonar con ella y persuadirla para que volviera a su casa. Le explicó que Jesús estaba rendido de cansancio de tanto enseñar y curar, y que había venido a Fenicia para pasar un período de tranquilidad y descanso. Pero fue inútil. Norana no quiso irse. Ante las súplicas de Pedro, ella se limitó a responder: <<No me marcharé hasta que haya visto a tu Maestro. Sé que puede echar al demonio de mi niña, y no me iré hasta que el sanador haya visto a mi hija>>.

\par 
%\textsuperscript{(1734.6)}
\textsuperscript{156:1.4} Entonces, Tomás intentó despedir a la mujer, pero tampoco tuvo éxito. Ella le dijo: <<Tengo fe en que tu Maestro será capaz de echar a este demonio que atormenta a mi hija. He oído hablar de sus obras poderosas en Galilea, y creo en él. ¿Qué os ha sucedido a vosotros, sus discípulos, para que queráis despedir a los que vienen buscando la ayuda de vuestro Maestro?>> Cuando ella hubo dicho esto, Tomás se retiró.

\par 
%\textsuperscript{(1735.1)}
\textsuperscript{156:1.5} Luego se adelantó Simón Celotes para amonestar a Norana. Simón dijo: <<Mujer, eres una gentil que habla griego. No es justo que esperes que el Maestro coja el pan destinado a los hijos de la casa favorecida y se lo eche a los perros>>. Pero Norana rehusó ofenderse por el ataque de Simón. Se limitó a replicar: <<Sí, maestro, comprendo tus palabras. No soy más que un perro a los ojos de los judíos, pero en lo que respecta a tu Maestro, soy un perro creyente. Estoy decidida a que él vea a mi hija, porque estoy persuadida de que, con que sólo la mire, la curará. Y ni siquiera tú, buen hombre, te atreverías a privar a los perros del privilegio de conseguir las migajas que puedan caer de la mesa de los hijos>>.

\par 
%\textsuperscript{(1735.2)}
\textsuperscript{156:1.6} En ese preciso momento, la chiquilla sufrió una violenta convulsión delante de todos ellos, y la madre exclamó: <<Ahora podéis ver que mi hija está poseída por un espíritu maligno. Si nuestra miseria no os impresiona, sí conmoverá a vuestro Maestro, que me han dicho que ama a todos los hombres y que se atreve incluso a curar a los gentiles cuando estos creen. No sois dignos de ser sus discípulos. No me iré hasta que mi hija haya sido curada>>.

\par 
%\textsuperscript{(1735.3)}
\textsuperscript{156:1.7} Jesús, que había escuchado toda esta conversación por una ventana abierta, salió entonces, para gran sorpresa de todos, y dijo: <<Oh mujer, tu fe es grande, tan grande que no puedo rehusar lo que deseas; puedes irte en paz. Tu hija ya ha recuperado la salud>>. Y la chiquilla se sintió bien a partir de aquel momento. Cuando Norana y la niña iban a despedirse, Jesús les rogó que no le contaran a nadie este suceso; aunque sus compañeros sí cumplieron esta petición, la madre y la niña no dejaron de proclamar por toda la región, e incluso en Sidón, el hecho de que la chiquilla había sido curada, de tal manera que Jesús estimó conveniente cambiar de residencia pocos días después.

\par 
%\textsuperscript{(1735.4)}
\textsuperscript{156:1.8} Al día siguiente, mientras Jesús enseñaba a sus apóstoles, comentando la curación de la hija de la mujer siria, dijo: <<Siempre ha sido así desde el principio; ya veis por vosotros mismos que los gentiles son capaces de ejercer una fe salvadora en las enseñanzas del evangelio del reino de los cielos. En verdad, en verdad os digo que los gentiles se apoderarán del reino del Padre si los hijos de Abraham no están dispuestos a mostrar la fe suficiente para entrar en él>>.

\section*{2. La enseñanza en Sidón}
\par 
%\textsuperscript{(1735.5)}
\textsuperscript{156:2.1} Al entrar en Sidón, Jesús y sus asociados pasaron por un puente, el primero que muchos de ellos habían visto nunca. Mientras caminaban por él, entre otras cosas, Jesús dijo: <<Este mundo no es más que un puente; podéis atravesarlo, pero no deberíais pensar en construir una morada encima de él>>.

\par 
%\textsuperscript{(1735.6)}
\textsuperscript{156:2.2} Mientras los veinticuatro empezaron sus trabajos en Sidón, Jesús fue a quedarse en una casa situada exactamente al norte de la ciudad, en el hogar de Justa y de su madre Berenice. Todas las mañanas, Jesús enseñaba a los veinticuatro en la casa de Justa, y durante la tarde y la noche se marchaban a Sidón para enseñar y predicar.

\par 
%\textsuperscript{(1735.7)}
\textsuperscript{156:2.3} Los apóstoles y los evangelistas se sintieron muy animados por la manera en que los gentiles de Sidón recibieron su mensaje; durante su corta estancia, muchos de ellos se añadieron al reino. Este período de unas seis semanas en Fenicia fue un momento muy fructífero en la tarea de ganar almas, pero los escritores judíos que redactaron más tarde los evangelios cogieron la costumbre de pasar por alto alegremente la historia de esta cálida recepción que hicieron los gentiles a las enseñanzas de Jesús, en el preciso momento en que un número tan grande de su propia gente adoptaba una postura hostil contra él.

\par 
%\textsuperscript{(1736.1)}
\textsuperscript{156:2.4} En muchos aspectos, estos creyentes gentiles apreciaron las enseñanzas de Jesús de manera más completa que los judíos. Muchos de estos sirofenicios de habla griega no solamente llegaron a discernir que Jesús se parecía a Dios, sino también que Dios se parecía a Jesús. Estos supuestos paganos consiguieron comprender bien las enseñanzas del Maestro sobre la uniformidad de las leyes de este mundo y de todo el universo. Comprendieron la enseñanza de que Dios no hace acepción de personas, de razas o de naciones; que con el Padre Universal no existen los favoritismos; que el universo siempre obedece totalmente a las leyes y es infaliblemente digno de confianza. Estos gentiles no tenían miedo de Jesús; se atrevían a aceptar su mensaje. A lo largo de todos los siglos posteriores, los hombres no han sido incapaces de comprender a Jesús; han tenido miedo de comprenderlo.

\par 
%\textsuperscript{(1736.2)}
\textsuperscript{156:2.5} Jesús indicó claramente a los veinticuatro que no había huido de Galilea porque careciera de coraje para enfrentarse con sus enemigos. Comprendieron que aún no estaba preparado para un conflicto abierto con la religión establecida, y que no trataba de convertirse en un mártir. Durante una de estas conferencias en la casa de Justa, el Maestro dijo por primera vez a sus discípulos que <<aunque el cielo y la Tierra desaparezcan, mis palabras de verdad no desaparecerán>>.

\par 
%\textsuperscript{(1736.3)}
\textsuperscript{156:2.6} Durante la estancia en Sidón, el tema de las enseñanzas de Jesús fue el progreso espiritual. Dijo a sus discípulos que no podían detenerse; que tenían que progresar en rectitud o retroceder hacia el mal y el pecado. Les recomendó que <<se olvidaran de las cosas del pasado, mientras que avanzaban para abrazar las realidades más grandes del reino>>. Les rogó que no se contentaran con seguir siendo niños en el evangelio, sino que se esforzaran por alcanzar la plena estatura de la filiación divina en la comunión del espíritu y en la hermandad de los creyentes.

\par 
%\textsuperscript{(1736.4)}
\textsuperscript{156:2.7} Jesús dijo: <<Mis discípulos no solamente deben dejar de hacer el mal, sino aprender a hacer el bien; no sólo tenéis que purificaros de todo pecado consciente, sino que tenéis que negaros incluso a albergar sentimientos de culpa. Si confesáis vuestros pecados, están perdonados; por eso tenéis que mantener una conciencia desprovista de faltas>>.

\par 
%\textsuperscript{(1736.5)}
\textsuperscript{156:2.8} Jesús disfrutaba mucho con el agudo sentido del humor que mostraban estos gentiles. El sentido del humor manifestado por Norana, la mujer siria, así como su fe grande y perseverante, fueron las cosas que conmovieron tanto el corazón del Maestro y atrajeron su misericordia. Jesús lamentaba mucho que su gente ---los judíos--- estuvieran tan faltos de humor. Una vez le dijo a Tomás: <<Mi gente se toma demasiado en serio a sí misma; son casi incapaces de apreciar el humor. La religión aburrida de los fariseos nunca podría haberse originado en un pueblo con sentido del humor. También les falta coherencia; filtran los mosquitos y se tragan los camellos>>.

\section*{3. El viaje subiendo por la costa}
\par 
%\textsuperscript{(1736.6)}
\textsuperscript{156:3.1} El martes 28 de junio, el Maestro y sus compañeros salieron de Sidón y subieron por la costa hasta Porfireón y Heldua. Fueron bien recibidos por los gentiles, y muchos de éstos ingresaron en el reino durante esta semana de enseñanza y predicación. Los apóstoles predicaron en Porfireón y los evangelistas enseñaron en Heldua. Mientras los veinticuatro se ocupaban así de su trabajo, Jesús los dejó durante un período de tres o cuatro días para hacer una visita a la ciudad costera de Beirut, donde estuvo charlando con un sirio llamado Malac, que era creyente y había estado en Betsaida el año anterior.

\par 
%\textsuperscript{(1737.1)}
\textsuperscript{156:3.2} El miércoles 6 de julio, todos regresaron a Sidón y permanecieron en la casa de Justa hasta el domingo por la mañana; entonces partieron hacia Tiro, dirigiéndose por la costa hacia el sur, por el camino de Sarepta, y llegaron a Tiro el lunes 11 de julio. Por esta época, los apóstoles y los evangelistas se estaban acostumbrando a trabajar entre estos llamados gentiles, que en realidad descendían principalmente de las antiguas tribus cananeas que tenían un origen semítico aún más antiguo. Todos estos pueblos hablaban la lengua griega. Los apóstoles y los evangelistas se quedaron muy sorprendidos al observar la avidez con que estos gentiles escuchaban el evangelio, y al advertir la prontitud con que muchos de ellos creían.

\section*{4. En Tiro}
\par 
%\textsuperscript{(1737.2)}
\textsuperscript{156:4.1} Desde el 11 hasta el 24 de julio enseñaron en Tiro. Cada uno de los apóstoles se llevó consigo a uno de los evangelistas, y así enseñaron y predicaron de dos en dos en todos los rincones de Tiro y sus alrededores. La población políglota de este activo puerto marítimo los escuchaba con placer, y muchos de ellos fueron bautizados en la hermandad exterior del reino. Jesús estableció su cuartel general en la casa de un judío llamado José, un creyente que vivía a cinco o seis kilómetros al sur de Tiro, no lejos de la tumba de Hiram, que había sido rey de la ciudad-Estado de Tiro en la época de David y Salomón.

\par 
%\textsuperscript{(1737.3)}
\textsuperscript{156:4.2} Durante este período de dos semanas, los apóstoles y los evangelistas entraban diariamente en Tiro por el muelle de Alejandro para dirigir pequeñas reuniones, y la mayoría de ellos regresaba cada noche al campamento de la casa de José, al sur de la ciudad. Los creyentes salían cada día de la ciudad para conversar con Jesús en el lugar donde estaba descansando. El Maestro sólo habló en Tiro una vez, el 20 de julio por la tarde, y enseñó a los creyentes sobre el amor del Padre por toda la humanidad y acerca de la misión del Hijo de revelar el Padre a todas las razas humanas. Estos gentiles mostraban tal interés por el evangelio del reino que, en esta ocasión, abrieron a Jesús las puertas del templo de Melcart, y es interesante indicar que en años posteriores se construyó una iglesia cristiana en el mismo lugar donde estaba situado este antiguo templo.

\par 
%\textsuperscript{(1737.4)}
\textsuperscript{156:4.3} En esta región se fabricaba la púrpura de Tiro, el tinte que había hecho famosas a Tiro y a Sidón en el mundo entero, y que había contribuido tanto a su comercio mundial y a su consiguiente enriquecimiento; muchos dirigentes de esta industria creyeron en el reino. Poco tiempo después empezó a disminuir el abastecimiento de animales marinos que servían para extraer este colorante, y estos fabricantes de tinte se fueron en busca de nuevas regiones donde se encontraban dichos mariscos. Así emigraron hasta los confines de la Tierra, llevando con ellos el mensaje de la paternidad de Dios y de la fraternidad de los hombres ---el evangelio del reino.

\section*{5. La enseñanza de Jesús en Tiro}
\par 
%\textsuperscript{(1737.5)}
\textsuperscript{156:5.1} En el transcurso de su alocución de este miércoles por la tarde, Jesús empezó contando a sus seguidores la historia del lirio blanco que alza su cabeza pura y nevada hacia la luz del Sol, mientras que sus raíces están enterradas en el lodo y el estiércol del suelo tenebroso. <<De la misma manera>>, dijo, <<aunque el hombre mortal tiene las raíces de su origen y de su ser en el suelo animal de la naturaleza humana, mediante la fe puede elevar su naturaleza espiritual hacia la luz solar de la verdad celestial, y producir realmente los nobles frutos del espíritu>>.

\par 
%\textsuperscript{(1738.1)}
\textsuperscript{156:5.2} Fue durante este mismo sermón cuando Jesús utilizó la primera y única parábola relacionada con su propio oficio ---la carpintería. En el transcurso de su recomendación sobre <<construir bien los cimientos para el crecimiento de un carácter noble impregnado de dones espirituales>>, dijo: <<Para producir los frutos del espíritu, tenéis que haber nacido del espíritu. El espíritu es el que debe enseñaros y conduciros si queréis vivir una vida de plenitud espiritual entre vuestros semejantes. Pero no cometáis el error del carpintero necio que derrocha un tiempo precioso cuadrando, midiendo y cepillando una madera de construcción carcomida por los gusanos y podrida en su interior, para después de haber consagrado todo su esfuerzo a esa viga podrida, tiene que rechazarla porque es inadecuada para formar parte de los cimientos del edificio que quería construir, y que deberán resistir los ataques del tiempo y de las tempestades. Que todo hombre se asegure de que los cimientos intelectuales y morales de su carácter tengan tal solidez que sostengan adecuadamente la superestructura de su naturaleza espiritual que aumenta y se ennoblece, la cual transformará así la mente mortal para después, en asociación con esa mente recreada, conseguir desarrollar el alma cuyo destino es inmortal. Vuestra naturaleza espiritual ---el alma creada conjuntamente--- es un producto viviente, pero la mente y la moral del individuo son el terreno del que deben brotar esas manifestaciones superiores del desarrollo humano y del destino divino. El suelo del alma evolutiva es humano y material, pero el destino de esta criatura compuesta de mente y de espíritu es espiritual y divino>>.

\par 
%\textsuperscript{(1738.2)}
\textsuperscript{156:5.3} Este mismo día por la tarde, Natanael le preguntó a Jesús: <<Maestro, ¿por qué le pedimos a Dios que no nos induzca a la tentación, cuando sabemos muy bien, por tu revelación del Padre, que él nunca hace tales cosas?>> Jesús contestó a Natanael:

\par 
%\textsuperscript{(1738.3)}
\textsuperscript{156:5.4} <<No es de extrañar que hagas estas preguntas, puesto que estás empezando a conocer al Padre como yo lo conozco, y no como lo veían tan confusamente los antiguos profetas hebreos. Sabes bien que nuestros antepasados tenían la tendencia de ver a Dios en casi todas las cosas que sucedían. Buscaban la mano de Dios en todas los acontecimientos naturales y en cada episodio insólito de la experiencia humana. Asociaban a Dios tanto con el bien como con el mal. Pensaban que Dios había ablandado el corazón de Moisés y endurecido el del faraón. Cuando el hombre sentía un fuerte impulso de hacer algo, bueno o malo, tenía la costumbre de explicar estas emociones poco frecuentes diciendo: `El Señor me ha hablado para decirme: haz esto o haz aquello, ve aquí o ve allí.' En consecuencia, como los hombres caían tan a menudo y con tanta violencia en la tentación, nuestros antepasados cogieron la costumbre de creer que Dios les inducía a ella para probarlos, castigarlos o fortalecerlos. Pero tú, por supuesto, sabes ahora más cosas. Sabes que, con demasiada frecuencia, los hombres son inducidos a la tentación por el ímpetu de su propio egoísmo y los impulsos de su naturaleza animal. Cuando te sientas tentado de esta manera, te recomiendo que, al mismo tiempo que reconoces honrada y sinceramente la tentación exactamente por lo que es, reorientes de manera inteligente las energías espirituales, mentales y corporales que intentan expresarse hacia unos canales superiores y unas metas más idealistas. De esta manera podrás transformar tus tentaciones en los tipos más elevados de servicio humano edificante, y al mismo tiempo evitarás casi por completo los conflictos destructivos y debilitantes entre la naturaleza animal y la naturaleza espiritual>>.

\par 
%\textsuperscript{(1738.4)}
\textsuperscript{156:5.5} <<Pero déjame prevenirte contra la locura de intentar superar la tentación mediante el esfuerzo de reemplazar un deseo por otro deseo supuestamente superior, utilizando la simple fuerza de la voluntad humana. Si quieres triunfar realmente sobre las tentaciones de la naturaleza más baja e inferior, debes alcanzar esa posición de superioridad espiritual en la que habrás desarrollado, de manera real y sincera, un interés efectivo y un amor por esas formas de conducta superiores y más idealistas que tu mente desea sustituir por los hábitos de comportamiento inferiores y menos idealistas que reconoces como tentaciones. De esta manera podrás liberarte gracias a la transformación espiritual, en lugar de sentirte cada vez más sobrecargado por la represión engañosa de los deseos humanos. Lo antiguo y lo inferior serán olvidados en el amor por lo nuevo y lo superior. La belleza siempre triunfa sobre la fealdad en el corazón de todos los que están iluminados por el amor a la verdad. Existe un enorme poder en la energía expulsiva de un afecto espiritual nuevo y sincero. Te lo repito de nuevo, no te dejes vencer por el mal, sino más bien vence al mal con el bien>>.

\par 
%\textsuperscript{(1739.1)}
\textsuperscript{156:5.6} Los apóstoles y los evangelistas continuaron haciendo preguntas hasta muy entrada la noche, y de las numerosas respuestas de Jesús, desearíamos presentar los pensamientos siguientes, que exponemos en un lenguaje moderno:

\par 
%\textsuperscript{(1739.2)}
\textsuperscript{156:5.7} Una ambición enérgica, un juicio inteligente y una sabiduría madura son los factores esenciales para conseguir el éxito material. Las dotes de mando dependen de la aptitud natural, la discreción, el poder de la voluntad y la determinación. El destino espiritual depende de la fe, el amor y la devoción a la verdad ---el hambre y la sed de rectitud--- el deseo entusiasta de encontrar a Dios y parecerse a él.

\par 
%\textsuperscript{(1739.3)}
\textsuperscript{156:5.8} No os desaniméis por el descubrimiento de que sois humanos. La naturaleza humana puede tender hacia el mal, pero no es pecaminosa de manera inherente. No os sintáis abatidos por vuestra incapacidad para olvidar completamente algunas de vuestras experiencias más lamentables. Los errores que no consigáis olvidar en el tiempo, serán olvidados en la eternidad. Aligerad las cargas de vuestra alma mediante la rápida adquisición de una visión a largo plazo de vuestro destino, de una expansión de vuestra carrera en el universo.

\par 
%\textsuperscript{(1739.4)}
\textsuperscript{156:5.9} No cometáis el error de apreciar el valor del alma según las imperfecciones de la mente o los apetitos del cuerpo. No juzguéis el alma ni evaluéis su destino sobre la base de un solo episodio humano desafortunado. Vuestro destino espiritual sólo está condicionado por vuestros anhelos e intenciones espirituales.

\par 
%\textsuperscript{(1739.5)}
\textsuperscript{156:5.10} La religión es la experiencia exclusivamente espiritual del alma inmortal evolutiva del hombre que conoce a Dios, pero el poder moral y la energía espiritual son unas fuerzas poderosas que se pueden utilizar para tratar situaciones sociales difíciles y para resolver problemas económicos complicados. Estos dones morales y espirituales enriquecen más todos los niveles de la vida humana, y los hacen más significativos.

\par 
%\textsuperscript{(1739.6)}
\textsuperscript{156:5.11} Si aprendéis a amar solamente a aquellos que os aman, estáis destinados a vivir una vida limitada y mediocre. Es cierto que el amor humano puede ser recíproco, pero el amor divino es extrovertido en toda su búsqueda de la satisfacción. Cuanto menos amor hay en la naturaleza de una criatura, más grande es su necesidad de amor, y más intenta el amor divino satisfacer esa necesidad. El amor nunca es egoísta, y no puede ser dirigido hacia uno mismo. El amor divino no puede estar encerrado en sí mismo; necesita darse generosamente.

\par 
%\textsuperscript{(1739.7)}
\textsuperscript{156:5.12} Los creyentes en el reino deben poseer una fe implícita, una creencia con toda el alma, en el triunfo seguro de la rectitud. Los constructores del reino no deben dudar de que el evangelio de la salvación eterna es verdadero. Los creyentes deben aprender cada vez más a apartarse de las precipitaciones de la vida ---a huir de los agobios de la existencia material--- mientras que vivifican su alma, inspiran su mente y renuevan su espíritu por medio de la comunión en la adoración.

\par 
%\textsuperscript{(1739.8)}
\textsuperscript{156:5.13} Los individuos que conocen a Dios no se desaniman por las desgracias ni se dejan abatir por las decepciones. Los creyentes están inmunizados contra la depresión que sigue a los cataclismos puramente materiales; los que llevan una vida espiritual no se inquietan por los episodios del mundo material. Los candidatos a la vida eterna practican una técnica vigorizante y constructiva para hacer frente a todas las vicisitudes y agobios de la vida mortal. Un verdadero creyente, cada día que vive, encuentra \textit{más fácil} hacer lo que es justo.

\par 
%\textsuperscript{(1740.1)}
\textsuperscript{156:5.14} La vida espiritual acrecienta poderosamente la verdadera autoestima. Pero la autoestima no es la admiración de sí mismo. La autoestima siempre está coordinada con el amor y el servicio a los semejantes. No es posible estimarse más a sí mismo de lo que se ama al prójimo; lo uno mide la capacidad para hacer lo otro.

\par 
%\textsuperscript{(1740.2)}
\textsuperscript{156:5.15} A medida que pasan los días, todo verdadero creyente se vuelve más hábil en atraer a sus semejantes hacia el amor de la verdad eterna. ¿Sois hoy más ingeniosos que ayer en revelar la bondad a la humanidad? ¿Sabéis recomendar mejor la rectitud este año que el año pasado? ¿Os estáis volviendo cada vez más artistas en vuestra técnica para conducir a las almas hambrientas hacia el reino espiritual?

\par 
%\textsuperscript{(1740.3)}
\textsuperscript{156:5.16} ¿Son vuestros ideales lo suficientemente elevados como para garantizar vuestra salvación eterna, y vuestras ideas son al mismo tiempo tan prácticas como para convertiros en unos ciudadanos útiles que funcionan en la Tierra en asociación con sus compañeros mortales? En el espíritu, vuestra ciudadanía está en los cielos; en la carne, todavía sois ciudadanos de los reinos de la Tierra. Dad a los césares las cosas que son materiales, y a Dios las que son espirituales.

\par 
%\textsuperscript{(1740.4)}
\textsuperscript{156:5.17} La medida de la capacidad espiritual del alma evolutiva es vuestra fe en la verdad y vuestro amor por los hombres; pero la medida de vuestra fuerza de carácter humano es vuestra aptitud para resistir la influencia de los resentimientos y vuestra capacidad para soportar las cavilaciones en presencia de una pena profunda. La derrota es el verdadero espejo donde podéis contemplar honradamente vuestro yo real.

\par 
%\textsuperscript{(1740.5)}
\textsuperscript{156:5.18} A medida que tenéis más años y os volvéis más experimentados en los asuntos del reino, ¿empleáis más tacto en vuestras relaciones con los mortales inoportunos y más tolerancia en la convivencia con vuestros compañeros testarudos? El tacto es el punto de apoyo de la influencia social, y la tolerancia es el distintivo de un alma grande. Si poseéis estos dones raros y encantadores, a medida que pasan los días os volveréis más alertas y expertos en vuestros esfuerzos meritorios por evitar todos los malentendidos sociales inútiles. Estas almas sabias son capaces de evitar un buen número de dificultades que se abaten con seguridad sobre todos los que sufren una falta de adaptación emocional, los que se niegan a crecer, y los que no aceptan envejecer con elegancia.

\par 
%\textsuperscript{(1740.6)}
\textsuperscript{156:5.19} Evitad la falta de honradez y la injusticia en todos vuestros esfuerzos por predicar la verdad y proclamar el evangelio. No busquéis un reconocimiento no ganado y no anheléis una simpatía inmerecida. Recibid libremente el amor que os llegue tanto de fuentes divinas como humanas, independientemente de que lo merezcáis o no, y amad a cambio generosamente. Pero en todas las demás cosas relacionadas con el honor y la adulación, buscad sólo lo que os pertenezca honradamente.

\par 
%\textsuperscript{(1740.7)}
\textsuperscript{156:5.20} El mortal consciente de Dios está seguro de salvarse; no le teme a la vida; es honrado y consecuente. Sabe cómo soportar valientemente los sufrimientos inevitables; no se queja cuando se enfrenta con las penalidades ineludibles.

\par 
%\textsuperscript{(1740.8)}
\textsuperscript{156:5.21} El verdadero creyente no se cansa de hacer el bien simplemente porque se sienta frustrado. Las dificultades estimulan el ardor de los amantes de la verdad, mientras que los obstáculos sólo sirven para desafiar los esfuerzos de los intrépidos constructores del reino.

\par 
%\textsuperscript{(1740.9)}
\textsuperscript{156:5.22} Y Jesús les enseñó otras muchas cosas antes de que se prepararan para marcharse de Tiro.

\par 
%\textsuperscript{(1740.10)}
\textsuperscript{156:5.23} El día antes de salir de Tiro para regresar a la región del Mar de Galilea, Jesús reunió a sus asociados y ordenó a los doce evangelistas que volvieran por una ruta diferente a la que él y los doce apóstoles iban a utilizar. Después de separarse aquí de Jesús, los evangelistas nunca más volvieron a estar tan íntimamente asociados con él.

\section*{6. El regreso de Fenicia}
\par 
%\textsuperscript{(1741.1)}
\textsuperscript{156:6.1} El domingo 24 de julio hacia el mediodía, Jesús y los doce salieron de la casa de José, al sur de Tiro. Bajaron por la costa hasta Tolemaida, donde se detuvieron un día, y expresaron palabras de aliento al grupo de creyentes que residía allí. Pedro predicó para ellos el 25 de julio por la noche.

\par 
%\textsuperscript{(1741.2)}
\textsuperscript{156:6.2} El martes salieron de Tolemaida y se dirigieron tierra adentro hacia el este, por el camino de Tiberiades, hasta cerca de Jotapata. El miércoles se detuvieron en Jotapata y dieron a los creyentes una enseñanza adicional sobre las cosas del reino. El jueves salieron de Jotapata y se encaminaron hacia el norte por la ruta de Nazaret al Monte Líbano hasta llegar al pueblo de Zabulón, pasando por Ramá. El viernes mantuvieron reuniones en Ramá y se quedaron hasta el sábado. El domingo día 31 llegaron a Zabulón, celebraron una reunión aquella noche y partieron a la mañana siguiente.

\par 
%\textsuperscript{(1741.3)}
\textsuperscript{156:6.3} Cuando salieron de Zabulón, viajaron hasta el cruce con la carretera de Magdala a Sidón, cerca de Giscala, y desde allí se dirigieron a Genesaret por la costa occidental del lago de Galilea, al sur de Cafarnaúm, donde habían acordado reunirse con David Zebedeo, y donde tenían la intención de deliberar sobre el siguiente paso a dar en la tarea de predicar el evangelio del reino.

\par 
%\textsuperscript{(1741.4)}
\textsuperscript{156:6.4} Durante una breve conversación con David, se enteraron de que muchos dirigentes se encontraban reunidos en ese momento en la orilla opuesta del lago, cerca de Jeresa, y en consecuencia, aquella misma noche atravesaron el lago en una barca. Pasaron un día descansando tranquilamente en las colinas, y al día siguiente continuaron hasta el parque cercano donde el Maestro había alimentado anteriormente a los cinco mil. Descansaron allí durante tres días y celebraron diariamente unas conferencias a las que asistieron unos cincuenta hombres y mujeres, el resto del antiguo grupo numeroso de creyentes que residían en Cafarnaúm y sus alrededores.

\par 
%\textsuperscript{(1741.5)}
\textsuperscript{156:6.5} Mientras Jesús estaba ausente de Cafarnaúm y Galilea, durante el período de su estancia en Fenicia, sus enemigos consideraron que todo el movimiento había sido destruido; concluyeron que la prisa de Jesús por alejarse de allí indicaba que estaba tan asustado que probablemente no volvería nunca más a molestarlos. Toda la oposición activa a sus enseñanzas casi se había calmado. Los creyentes empezaban de nuevo a celebrar reuniones públicas, y los supervivientes probados y leales de la gran criba por la que acababan de pasar los creyentes en el evangelio se iban consolidando de manera gradual pero eficaz.

\par 
%\textsuperscript{(1741.6)}
\textsuperscript{156:6.6} Felipe, el hermano de Herodes, se había convertido en un creyente a medias en Jesús y envió un mensaje indicando que el Maestro tenía libertad para vivir y trabajar en sus dominios.

\par 
%\textsuperscript{(1741.7)}
\textsuperscript{156:6.7} La orden de cerrar las sinagogas de todo el mundo judío a las enseñanzas de Jesús y de todos sus seguidores se había vuelto en contra de los escribas y fariseos. En cuanto Jesús se retiró como objeto de controversia, se produjo una reacción en toda la población judía; hubo un resentimiento general contra los fariseos y los dirigentes del sanedrín de Jerusalén. Muchos jefes de las sinagogas empezaron a abrir subrepticiamente sus sinagogas a Abner y sus asociados, declarando que estos instructores eran seguidores de Juan, y no discípulos de Jesús.

\par 
%\textsuperscript{(1741.8)}
\textsuperscript{156:6.8} Incluso Herodes Antipas experimentó un cambio de sentimiento. Al enterarse de que Jesús estaba residiendo al otro lado del lago, en el territorio de su hermano Felipe, le envió el recado de que, aunque había firmado unas órdenes para que lo arrestaran en Galilea, no por ello había autorizado su captura en Perea, indicando de esta manera que Jesús no sería molestado si permanecía fuera de Galilea; y esta misma decisión se la comunicó a los judíos de Jerusalén.

\par 
%\textsuperscript{(1742.1)}
\textsuperscript{156:6.9} Ésta era la situación hacia primeros de agosto del año 29, cuando el Maestro regresó de su misión en Fenicia y empezó a reorganizar sus fuerzas dispersas, puestas a prueba y reducidas, con vistas a este último año memorable de su misión en la Tierra.

\par 
%\textsuperscript{(1742.2)}
\textsuperscript{156:6.10} Los resultados de la batalla están claramente delineados mientras el Maestro y sus compañeros se preparan para empezar la proclamación de una nueva religión, la religión del espíritu del Dios vivo que reside en la mente de los hombres.