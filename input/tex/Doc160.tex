\chapter{Documento 160. Rodán de Alejandría}
\par 
%\textsuperscript{(1772.1)}
\textsuperscript{160:0.1} EL DOMINGO 18 de septiembre por la mañana, Andrés anunció que no se planearía ningún trabajo para la semana siguiente. Todos los apóstoles, excepto Natanael y Tomás, fueron a sus casas para visitar a sus familias o permanecer con sus amigos. Esta semana, Jesús disfrutó de un período de descanso casi completo, pero Natanael y Tomás estuvieron muy ocupados discutiendo con cierto filósofo griego de Alejandría llamado Rodán. Este griego se había hecho recientemente discípulo de Jesús gracias a las enseñanzas de uno de los asociados de Abner, que había dirigido una misión en Alejandría. Rodán estaba ahora seriamente ocupado en la tarea de armonizar su filosofía de la vida con las nuevas enseñanzas religiosas de Jesús, y había venido a Magadán con la esperanza de que el Maestro discutiera estos problemas con él. También deseaba obtener una versión autorizada y de primera mano del evangelio, ya fuera de Jesús o de uno de sus apóstoles. Aunque el Maestro declinó participar en este tipo de conversaciones con Rodán, lo recibió amablemente y ordenó de inmediato que Natanael y Tomás escucharan todo lo que tenía que decir, y que a su vez le hablaran sobre el evangelio.

\section*{1. La filosofía griega de Rodán}
\par 
%\textsuperscript{(1772.2)}
\textsuperscript{160:1.1} El lunes por la mañana temprano, Rodán comenzó una serie de diez discursos para Natanael, Tomás y un grupo de unas dos docenas de creyentes que se encontraban casualmente en Magadán. Estas conversaciones, condensadas, combinadas y reexpuestas en un lenguaje moderno, ofrecen para su estudio los pensamientos siguientes:

\par 
%\textsuperscript{(1772.3)}
\textsuperscript{160:1.2} La vida humana consiste en tres grandes estímulos: los impulsos, los deseos y los alicientes. Un carácter fuerte, una personalidad con autoridad, sólo se puede adquirir convirtiendo el impulso natural de la vida en el arte social de vivir, transformando los deseos inmediatos en esos anhelos elevados que son capaces de logros duraderos, mientras que el aliciente común de la existencia debemos transferirlo desde las ideas personales, convencionales y establecidas, hasta los niveles más elevados de las ideas no exploradas y de los ideales por descubrir.

\par 
%\textsuperscript{(1772.4)}
\textsuperscript{160:1.3} Cuanto más compleja se vuelva la civilización, más difícil será el arte de vivir. Cuanto más rápidamente cambien los usos sociales, más complicada será la tarea de desarrollar el carácter. Para que el progreso pueda continuar, la humanidad tiene que aprender de nuevo el arte de vivir cada diez generaciones. Y si el hombre se vuelve tan ingenioso que aumenta con más rapidez las complejidades de la sociedad, el arte de vivir tendrá que ser dominado de nuevo en menos tiempo, quizás en cada generación. Si la evolución del arte de vivir no logra seguir el mismo ritmo que la técnica de la existencia, la humanidad retrocederá rápidamente al simple impulso de vivir ---a la satisfacción de los deseos inmediatos. De esta manera, la humanidad seguirá siendo inmadura; la sociedad no logrará desarrollarse hasta su plena madurez.

\par 
%\textsuperscript{(1773.1)}
\textsuperscript{160:1.4} La madurez social es equivalente al grado en que el hombre está dispuesto a renunciar a satisfacer sus meros deseos pasajeros e inmediatos, para mantener esos anhelos superiores cuya obtención, por medio del esfuerzo, proporciona las satisfacciones más abundantes del avance progresivo hacia objetivos permanentes. Pero el verdadero distintivo de la madurez social es la buena voluntad de un pueblo para renunciar al derecho de vivir satisfecho y en paz bajo las normas que promueven la facilidad, basadas en el aliciente de las creencias establecidas y de las ideas convencionales, para perseguir el aliciente inquietante, y que necesita energía, de las posibilidades inexploradas de alcanzar los objetivos no descubiertos de las realidades espirituales idealistas.

\par 
%\textsuperscript{(1773.2)}
\textsuperscript{160:1.5} Los animales reaccionan noblemente al impulso de la vida, pero sólo el hombre puede alcanzar el arte de vivir, aunque la mayoría de la humanidad sólo experimenta el impulso animal de vivir. Los animales no conocen más que este impulso ciego e instintivo; el hombre es capaz de trascender este impulso que le incita al funcionamiento natural. El hombre puede decidir vivir en el plano elevado del arte inteligente, e incluso en el plano de la alegría celestial y del éxtasis espiritual. Los animales no se preguntan por el propósito de la vida; por eso nunca se preocupan ni tampoco se suicidan. Entre los hombres, el suicidio demuestra que estos seres han sobrepasado el estado puramente animal de la existencia, y el hecho adicional de que los esfuerzos exploratorios de tales seres humanos no han logrado alcanzar los niveles en que la experiencia mortal se vuelve un arte. Los animales no conocen el significado de la vida; el hombre no sólo posee la capacidad de reconocer los valores y de comprender los significados, sino que también tiene conciencia del significado de los significados ---es consciente de su propia perspicacia.

\par 
%\textsuperscript{(1773.3)}
\textsuperscript{160:1.6} Cuando los hombres se atreven a abandonar una vida de intensos deseos naturales a favor de un arte de vivir arriesgado y de una lógica incierta, deben contar con soportar los riesgos correspondientes de los accidentes emocionales ---conflictos, infelicidad e incertidumbres--- al menos hasta el momento en que alcanzan cierto grado de madurez intelectual y emocional. El desaliento, la preocupación y la indolencia son una prueba evidente de la inmadurez moral. La sociedad humana se enfrenta con dos problemas: alcanzar la madurez por parte del individuo, y alcanzar la madurez por parte de la raza. El ser humano maduro empieza pronto a mirar a todos los demás mortales con sentimientos de ternura y con emociones de tolerancia. Los hombres maduros perciben a sus compañeros inmaduros con el amor y la consideración que los padres tienen por sus hijos.

\par 
%\textsuperscript{(1773.4)}
\textsuperscript{160:1.7} El éxito en la vida no es ni más ni menos que el arte de dominar las técnicas fiables para solucionar los problemas ordinarios. El primer paso para solucionar un problema cualquiera consiste en localizar la dificultad, aislar el problema y reconocer francamente su naturaleza y su gravedad. Cuando los problemas de la vida despiertan nuestros temores profundos, cometemos el gran error de negarnos a reconocerlos. Asimismo, cuando reconocer nuestras dificultades implica reducir nuestra vanidad largamente acariciada, admitir que somos envidiosos, o abandonar unos prejuicios profundamente arraigados, la persona de tipo medio prefiere aferrarse a sus viejas ilusiones de seguridad y a sus falsas sensaciones de estabilidad largo tiempo cultivadas. Sólo una persona valiente está dispuesta a admitir honradamente aquello que descubre una mente sincera y lógica, y a enfrentarse a ello sin temor.

\par 
%\textsuperscript{(1773.5)}
\textsuperscript{160:1.8} Para solucionar de manera sabia y eficaz cualquier problema, se necesita una mente libre de inclinaciones, de pasiones y de cualquier otro prejuicio puramente personal que pueda interferir con el análisis imparcial de los factores reales que juntos constituyen el problema que se ha presentado para ser resuelto. La solución de los problemas de la vida requiere valentía y sinceridad. Sólo las personas honradas y valientes son capaces de continuar valerosamente su camino a través del laberinto confuso y desconcertante de la vida al que pueda llevarles la lógica de una mente intrépida. Esta emancipación de la mente y del alma nunca puede producirse sin el poder impulsor de un entusiasmo inteligente que roza el fervor religioso. Se necesita el atractivo de un gran ideal para impulsar al hombre en pos de un objetivo rodeado de problemas materiales difíciles y de riesgos intelectuales múltiples.

\par 
%\textsuperscript{(1774.1)}
\textsuperscript{160:1.9} Aunque estéis eficazmente preparados para afrontar las situaciones difíciles de la vida, no podéis esperar mucho éxito a menos que estéis provistos de esa sabiduría de la mente y de ese encanto de la personalidad que os permita conseguir el apoyo y la cooperación sincera de vuestros semejantes. Tanto en el trabajo seglar como en el trabajo religioso, no podéis esperar mucho éxito a menos que aprendáis a persuadir a vuestros semejantes, a convencer a los hombres. Simplemente debéis de tener tacto y tolerancia.

\par 
%\textsuperscript{(1774.2)}
\textsuperscript{160:1.10} Pero el mejor de todos los métodos para solucionar los problemas lo he aprendido de Jesús, vuestro Maestro. Me refiero a lo que él practica con tanta perseverancia, y que tan fielmente os ha enseñado: la meditación adoradora en solitario. En esta costumbre que tiene Jesús de apartarse con tanta frecuencia para comulgar con el Padre que está en los cielos, se encuentra la técnica, no sólo para acumular las fuerzas y la sabiduría necesarias para los conflictos ordinarios de la vida, sino también para apropiarse de la energía necesaria para resolver los problemas más elevados de naturaleza moral y espiritual. Pero incluso los métodos correctos para solucionar los problemas no compensan los defectos inherentes a la personalidad, ni reparan la ausencia de hambre y de sed de verdadera rectitud.

\par 
%\textsuperscript{(1774.3)}
\textsuperscript{160:1.11} Me impresiona profundamente la costumbre de Jesús de retirarse a solas para emprender esos períodos de examen solitario de los problemas de la vida; para buscar nuevas reservas de sabiduría y de energía para poder enfrentarse a las múltiples exigencias del servicio social; para vivificar y hacer más profundo el propósito supremo de la vida, sometiendo realmente su personalidad total a la conciencia del contacto con la divinidad; para tratar de conseguir métodos nuevos y mejores para adaptarse a las situaciones siempre cambiantes de la existencia viviente; para efectuar esas reconstrucciones y reajustes vitales de las actitudes personales, que son tan esenciales para comprender mejor todo lo que es válido y real. Y hacer todo esto con miras a la sola gloria de Dios ---decir sinceramente la oración favorita de vuestro Maestro: <<Que se haga, no mi voluntad, sino la tuya>>.

\par 
%\textsuperscript{(1774.4)}
\textsuperscript{160:1.12} Esta práctica de adoración de vuestro Maestro aporta ese descanso que renueva la mente, esa iluminación que inspira el alma, ese valor que permite enfrentarse valientemente con los problemas de uno mismo, esa comprensión de sí mismo que elimina el temor debilitante, y esa conciencia de la unión con la divinidad que equipa al hombre con la seguridad que le permite atreverse a ser como Dios. El descanso de la adoración, o comunión espiritual, tal como la practica el Maestro, alivia la tensión, elimina los conflictos y aumenta poderosamente los recursos totales de la personalidad. Y toda esta filosofía, más el evangelio del reino, constituyen la nueva religión tal como yo la comprendo.

\par 
%\textsuperscript{(1774.5)}
\textsuperscript{160:1.13} Los prejuicios ciegan el alma impidiéndole reconocer la verdad, y los prejuicios sólo se pueden eliminar mediante la devoción sincera del alma a la adoración de una causa que abarque e incluya a todos nuestros semejantes humanos. Los prejuicios están inseparablemente vinculados con el egoísmo. Los prejuicios sólo se pueden suprimir abandonando el egocentrismo y reemplazándolo por la búsqueda de la satisfacción de servir a una causa que sea no sólo más grande que uno mismo, sino incluso más grande que toda la humanidad ---la búsqueda de Dios, la adquisición de la divinidad. La prueba de la madurez de la personalidad consiste en la transformación de los deseos humanos de tal manera que busquen constantemente la comprensión de los valores más elevados y más divinamente reales.

\par 
%\textsuperscript{(1774.6)}
\textsuperscript{160:1.14} En un mundo que cambia continuamente, en medio de un orden social en evolución, es imposible mantener unas metas de destino establecidas y asentadas. Sólo pueden experimentar la estabilidad de la personalidad aquellos que han descubierto y abrazado al Dios viviente como meta eterna de consecución infinita. Para transferir así la meta individual del tiempo a la eternidad, de la Tierra al Paraíso, de lo humano a lo divino, es necesario que el hombre se regenere, se convierta, nazca de nuevo, que se vuelva el hijo re-creado del espíritu divino, que logre su entrada en la fraternidad del reino de los cielos. Todas las filosofías y religiones que estén por debajo de estos ideales son inmaduras. La filosofía que yo enseño, unida al evangelio que vosotros predicáis, representa la nueva religión de la madurez, el ideal de todas las generaciones futuras. Y esto es verdad porque nuestro ideal es definitivo, infalible, eterno, universal, absoluto e infinito.

\par 
%\textsuperscript{(1775.1)}
\textsuperscript{160:1.15} Mi filosofía me ha impulsado a buscar las realidades de la consecución verdadera, la meta de la madurez. Pero mi impulso era impotente, mi búsqueda carecía de fuerza motriz, mi indagación sufría la falta de certidumbre de una orientación. Estas deficiencias han sido ampliamente colmadas con este nuevo evangelio de Jesús, con su aumento del discernimiento, su elevación de los ideales y su estabilidad de objetivos. Sin más dudas ni desconfianzas, ahora puedo emprender de todo corazón la aventura eterna.

\section*{2. El arte de vivir}
\par 
%\textsuperscript{(1775.2)}
\textsuperscript{160:2.1} Los mortales sólo tienen dos maneras de vivir juntos: la manera material o animal y la manera espiritual o humana. Por medio de signos y sonidos, los animales pueden comunicarse entre ellos en una medida limitada. Pero estas formas de comunicación no transmiten ni los significados, ni los valores ni las ideas. La única diferencia entre el hombre y el animal es que el hombre puede comunicarse con sus semejantes por medio de \textit{símbolos} que designan e identifican con precisión los significados, los valores, las ideas e incluso los ideales.

\par 
%\textsuperscript{(1775.3)}
\textsuperscript{160:2.2} Puesto que los animales no pueden comunicarse ideas entre sí, no pueden desarrollar una personalidad. El hombre desarrolla una personalidad porque puede comunicar a sus semejantes tanto las ideas como los ideales.

\par 
%\textsuperscript{(1775.4)}
\textsuperscript{160:2.3} Esta capacidad para comunicar y compartir los significados es lo que constituye la cultura humana y permite al hombre, a través de las asociaciones sociales, construir las civilizaciones. El conocimiento y la sabiduría se vuelven acumulativos debido a la capacidad del hombre para comunicar estas posesiones a las generaciones siguientes, surgiendo de esta manera las actividades culturales de la raza: el arte, la ciencia, la religión y la filosofía.

\par 
%\textsuperscript{(1775.5)}
\textsuperscript{160:2.4} La comunicación simbólica entre los seres humanos predetermina la aparición de los grupos sociales. El grupo social más eficaz de todos es la familia, y más concretamente los \textit{dos padres}. El afecto personal es el lazo espiritual que mantiene unidas estas asociaciones materiales. Una relación tan eficaz también es posible entre dos personas del mismo sexo, como lo ilustran tan abundantemente las devociones de las amistades auténticas.

\par 
%\textsuperscript{(1775.6)}
\textsuperscript{160:2.5} Estas asociaciones basadas en la amistad y en el afecto mutuos son socializadoras y ennoblecedoras porque fomentan y facilitan los siguientes factores esenciales de los niveles superiores del arte de vivir:

\par 
%\textsuperscript{(1775.7)}
\textsuperscript{160:2.6} 1. \textit{Expresarse y comprenderse mutuamente}. Muchos nobles impulsos humanos perecen porque no hay nadie que escuche su expresión. En verdad, no es bueno que el hombre esté solo. Cierto grado de reconocimiento y cierta cantidad de aprecio son esenciales para el desarrollo del carácter humano. Sin el amor auténtico del hogar, ningún niño puede alcanzar el pleno desarrollo de un carácter normal. El carácter es algo más que la mera mente y la mera moralidad. De todas las relaciones sociales pensadas para desarrollar el carácter, la más eficaz e ideal es la amistad afectuosa y comprensiva de un hombre y una mujer en el abrazo mutuo de una vida conyugal inteligente. El matrimonio, con sus múltiples relaciones, es el que está mejor destinado a hacer surgir esos preciosos impulsos y esos motivos elevados que son indispensables para el desarrollo de un carácter fuerte. No dudo en glorificar así la vida familiar, porque vuestro Maestro ha elegido sabiamente la relación de padre a hijo como la piedra angular misma de este nuevo evangelio del reino. Esta comunidad incomparable de relaciones, un hombre y una mujer en el abrazo afectuoso de los ideales superiores del tiempo, es una experiencia tan valiosa y satisfactoria que vale cualquier precio, cualquier sacrificio que sea necesario para poseerla.

\par 
%\textsuperscript{(1776.1)}
\textsuperscript{160:2.7} 2. \textit{La unión de las almas ---la movilización de la sabiduría}. Todo ser humano adquiere, tarde o temprano, cierto concepto de este mundo y cierta visión del siguiente. Ahora bien, es posible, mediante la asociación de las personalidades, unificar estos puntos de vista sobre la existencia temporal y las perspectivas eternas. Así, la mente de uno acrecienta sus valores espirituales adquiriendo una gran parte de la perspicacia del otro. De esta manera, los hombres enriquecen su alma poniendo en común sus posesiones espirituales respectivas. Y también de esta misma manera el hombre consigue evitar esa tendencia siempre presente a caer víctima de su visión distorsionada, de su punto de vista parcial y de su estrechez de juicio. El miedo, la envidia y la vanidad sólo se pueden impedir mediante el contacto íntimo con otras mentes. Llamo vuestra atención sobre el hecho de que el Maestro nunca os envía solos a trabajar para la expansión del reino; siempre os envía de dos en dos. Y puesto que la sabiduría es un superconocimiento, de esto se deduce que, al unir su sabiduría, el grupo social, grande o pequeño, comparte mutuamente todo el conocimiento.

\par 
%\textsuperscript{(1776.2)}
\textsuperscript{160:2.8} 3. \textit{El entusiasmo de vivir}. El aislamiento tiende a agotar la carga de energía del alma. La asociación con nuestros semejantes es esencial para renovar el entusiasmo por la vida, y es indispensable para conservar la valentía para librar esas batallas que siguen a la ascensión a unos niveles superiores de vida humana. La amistad aumenta las alegrías y glorifica los triunfos de la vida. Las asociaciones humanas afectuosas e íntimas tienden a quitarle al sufrimiento su tristeza, y a las dificultades mucha parte de su amargura. La presencia de un amigo realza toda belleza y exalta toda bondad. Por medio de símbolos inteligentes, el hombre es capaz de vivificar y de ampliar las capacidades apreciativas de sus amigos. Este poder y esta posibilidad de estimularse mutuamente la imaginación es una de las glorias supremas de la amistad humana. Existe un gran poder espiritual inherente a la conciencia de estar consagrado de todo corazón a una causa común, de ser mutuamente leales a una Deidad cósmica.

\par 
%\textsuperscript{(1776.3)}
\textsuperscript{160:2.9} 4. \textit{La defensa creciente contra todo mal}. La asociación entre personalidades y el afecto mutuo son un seguro eficaz contra el mal. Las dificultades, las tristezas, las decepciones y las derrotas son más dolorosas y desalentadoras cuando se soportan a solas. La asociación no transforma el mal en rectitud, pero ayuda mucho a disminuir las heridas. Vuestro Maestro ha dicho: <<Bienaventurados los que están de luto>> ---si hay un amigo cerca para consolarlos. Hay una fuerza positiva en el conocimiento de que vivís para el bienestar de los demás, y que los demás viven igualmente para vuestro bienestar y vuestro progreso. El hombre languidece en el aislamiento. Los seres humanos se desaniman infaliblemente cuando ven solamente las transacciones transitorias del tiempo. Cuando el presente está separado del pasado y del futuro, se vuelve de una trivialidad exasperante. Vislumbrar el círculo de la eternidad es lo único que puede inspirar al hombre para hacer lo mejor posible, y que puede desafiar lo mejor que hay en él para que haga lo máximo. Cuando el hombre se encuentra así en sus mejores disposiciones, vive de manera muy generosa para el bien de los demás, para sus semejantes que residen con él en el tiempo y en la eternidad.

\par 
%\textsuperscript{(1777.1)}
\textsuperscript{160:2.10} Repito que esta asociación inspiradora y ennoblecedora encuentra sus posibilidades ideales en las relaciones del matrimonio humano. Es verdad que se pueden conseguir muchas cosas fuera del matrimonio, y que muchísimos matrimonios no logran producir en absoluto estos frutos morales y espirituales. Demasiadas veces contraen matrimonio aquellos que buscan otros valores que son inferiores a estos acompañamientos superiores de la madurez humana. El matrimonio ideal debe estar fundamentado en algo más estable que las fluctuaciones del sentimiento y la inconstancia de la simple atracción sexual; debe estar basado en una devoción personal auténtica y mutua. Así pues, si se pueden construir estas pequeñas unidades dignas de confianza y eficaces de asociaciones humanas, cuando se reúnan en conjunto, el mundo contemplará una gran estructura social glorificada, la civilización de la madurez de los mortales. Una raza así podría empezar a realizar una parte del ideal de vuestro Maestro de <<paz en la Tierra y buena voluntad entre los hombres>>. Aunque una sociedad así no sería perfecta ni estaría completamente libre del mal, al menos se acercaría a la estabilización de la madurez.

\section*{3. Los atractivos de la madurez}
\par 
%\textsuperscript{(1777.2)}
\textsuperscript{160:3.1} El esfuerzo por conseguir la madurez necesita trabajo, y el trabajo requiere energía. ¿De dónde viene el poder para realizar todo esto?. Las cosas físicas se pueden dar por sentadas, pero el Maestro bien ha dicho que <<No sólo de pan vive el hombre>>. Una vez que se posee un cuerpo normal y una salud razonablemente buena, debemos buscar a continuación aquellos atractivos que actuarán como estímulo para hacer surgir las fuerzas espirituales dormidas del hombre. Jesús nos ha enseñado que Dios vive en el hombre; entonces, ¿cómo podemos inducir al hombre a que libere estos poderes de la divinidad y de la infinidad que están ligados en su alma? ¿Cómo induciremos a los hombres a que dejen paso a Dios y Éste pueda brotar para refrescar nuestras propias almas mientras transita hacia el exterior, y luego sirva al propósito de iluminar, elevar y bendecir a otras innumerables almas? ¿De qué manera puedo despertar mejor estos poderes latentes para el bien que yace dormido en vuestra alma? De una cosa estoy seguro: la excitación emocional no es el estímulo espiritual ideal. La excitación no aumenta la energía; más bien agota las fuerzas de la mente y del cuerpo. ¿De dónde viene pues la energía para hacer estas grandes cosas? Observad a vuestro Maestro. En este mismo momento se encuentra allá en las colinas, llenándose de fuerza, mientras nosotros estamos aquí gastando energía. El secreto de todo este problema está envuelto en la comunión espiritual, en la adoración. Desde el punto de vista humano, se trata de combinar la meditación y la relajación. La meditación pone en contacto a la mente con el espíritu; la relajación determina la capacidad para la receptividad espiritual. Este intercambio de la debilidad por la fuerza, del temor por el valor, de la mente del yo por la voluntad de Dios, constituye la adoración. Al menos, el filósofo lo ve de esta manera.

\par 
%\textsuperscript{(1777.3)}
\textsuperscript{160:3.2} Cuando estas experiencias se repiten con frecuencia, se cristalizan en hábitos, en unos hábitos de adoración que dan fuerzas, y estos hábitos se traducen con el tiempo en un carácter espiritual, y este carácter es reconocido finalmente por nuestros semejantes como \textit{una personalidad madura}. Al principio, estas prácticas son difíciles y llevan mucho tiempo, pero cuando se vuelven habituales, proporcionan descanso y ahorro de tiempo a la vez. Cuanto más compleja se vuelva la sociedad, cuanto más se multipliquen los atractivos de la civilización, más urgente será la necesidad, para los individuos que conocen a Dios, de adquirir estas prácticas habituales protectoras destinadas a conservar y aumentar sus energías espirituales.

\par 
%\textsuperscript{(1778.1)}
\textsuperscript{160:3.3} Otro requisito para alcanzar la madurez es la adaptación cooperativa de los grupos sociales a un entorno en constante cambio. El individuo inmaduro despierta el antagonismo de sus semejantes; el hombre maduro se gana la cooperación cordial de sus asociados, multiplicando así muchas veces los frutos de los esfuerzos de su vida.

\par 
%\textsuperscript{(1778.2)}
\textsuperscript{160:3.4} Mi filosofía me dice que hay momentos en que debo luchar, si hace falta, para defender mi concepto de la rectitud, pero no dudo de que el Maestro, con su tipo de personalidad más madura, conseguiría fácil y elegantemente una victoria equivalente mediante su técnica superior y encantadora de tacto y de tolerancia. Demasiado a menudo, cuando luchamos por una cosa justa, resulta que tanto el vencedor como el vencido sufren una derrota. Ayer mismo oí decir al Maestro que <<si un hombre sabio trata de entrar por una puerta cerrada, no destruye la puerta, sino que busca la llave para abrirla>>. Con mucha frecuencia nos ponemos a luchar sólo para convencernos de que no tenemos miedo.

\par 
%\textsuperscript{(1778.3)}
\textsuperscript{160:3.5} Este nuevo evangelio del reino presta un gran servicio al arte de vivir, en el sentido de que proporciona un incentivo nuevo y más rico para una vida más elevada. Presenta una meta de destino nueva y sublime, un propósito supremo para la vida. Estos nuevos conceptos de la meta eterna y divina de la existencia son en sí mismos unos estímulos trascendentes que suscitan la reacción de lo mejor que existe en la naturaleza superior del hombre. En toda cima del pensamiento intelectual se encuentra un descanso para la mente, una fuerza para el alma y una comunión para el espíritu. Desde esta posición de ventaja de la vida superior, el hombre es capaz de trascender las irritaciones materiales de los niveles inferiores de pensamiento ---las preocupaciones, los celos, la envidia, la venganza y el orgullo de la personalidad inmadura. Las almas que ascienden a estas alturas se liberan de una multitud de conflictos a contracorriente de las nimiedades de la vida, volviéndose así libres para alcanzar la conciencia de las corrientes superiores de los conceptos espirituales y de las comunicaciones celestiales. Pero el propósito de la vida debe ser celosamente protegido contra la tentación de buscar los logros fáciles y transitorios; asimismo, debe ser fomentado de tal manera que se vuelva inmune a las amenazas desastrosas del fanatismo.

\section*{4. El equilibrio de la madurez}
\par 
%\textsuperscript{(1778.4)}
\textsuperscript{160:4.1} Mientras tenéis la vista puesta en alcanzar las realidades eternas, debéis también atender las necesidades de la vida temporal. Aunque el espíritu sea nuestra meta, la carne es un hecho. Puede suceder que lo que necesitamos para vivir caiga en nuestras manos por casualidad, pero en general, tenemos que trabajar con inteligencia para conseguirlo. Los dos problemas principales de la vida son: ganarse la vida temporal y conseguir la supervivencia eterna. Incluso el problema de ganarse la vida necesita a la religión para solucionarse de manera ideal. Estos dos problemas son muy personales. De hecho, la verdadera religión no funciona separadamente del individuo.

\par 
%\textsuperscript{(1778.5)}
\textsuperscript{160:4.2} Los factores esenciales de la vida temporal, tal como yo los veo, son:

\par 
%\textsuperscript{(1778.6)}
\textsuperscript{160:4.3} 1. Una buena salud física.

\par 
%\textsuperscript{(1778.7)}
\textsuperscript{160:4.4} 2. Un pensamiento claro y limpio.

\par 
%\textsuperscript{(1778.8)}
\textsuperscript{160:4.5} 3. La capacidad y la habilidad.

\par 
%\textsuperscript{(1778.9)}
\textsuperscript{160:4.6} 4. La riqueza ---los bienes de la vida.

\par 
%\textsuperscript{(1778.10)}
\textsuperscript{160:4.7} 5. La capacidad para resistir la derrota.

\par 
%\textsuperscript{(1778.11)}
\textsuperscript{160:4.8} 6. La cultura ---la educación y la sabiduría.

\par 
%\textsuperscript{(1779.1)}
\textsuperscript{160:4.9} Incluso los problemas materiales de la salud y la eficacia físicas se resuelven mejor cuando se ven desde el punto de vista religioso de las enseñanzas de nuestro Maestro: el cuerpo y la mente del hombre son el lugar donde vive el don de los Dioses, el espíritu de Dios que se convierte en el espíritu del hombre. La mente del hombre se vuelve así la mediadora entre las cosas materiales y las realidades espirituales.

\par 
%\textsuperscript{(1779.2)}
\textsuperscript{160:4.10} Se necesita inteligencia para conseguir la parte que nos corresponde de las cosas deseables de la vida. Es totalmente erróneo suponer que hacer fielmente nuestro trabajo diario nos asegurará la recompensa de la riqueza. Exceptuando la adquisición ocasional y accidental de las riquezas, se descubre que las recompensas materiales de la vida temporal fluyen por ciertos canales bien organizados, y sólo aquellos que tienen acceso a esos canales pueden esperar ser bien recompensados por sus esfuerzos temporales. La pobreza será siempre el destino de todos los hombres que buscan la riqueza en canales aislados e individuales. Por consiguiente, una planificación sabia se convierte en la única cosa esencial para la prosperidad material. El éxito requiere no solamente vuestra devoción al trabajo, sino también que funcionéis como una parte de uno de los canales de la riqueza material. Si sois poco sabios, podéis otorgarle a vuestra generación una vida dedicada sin recompensa material; si os beneficiáis accidentalmente del flujo de la riqueza, podréis nadar en el lujo aunque no hayáis hecho nada útil por vuestros semejantes.

\par 
%\textsuperscript{(1779.3)}
\textsuperscript{160:4.11} La capacidad se hereda, mientras que la habilidad se adquiere. La vida es irreal para aquel que no sabe hacer alguna cosa bien, expertamente. La habilidad es una de las verdaderas fuentes de satisfacción en la vida. La capacidad implica el don de la previsión, de la visión de futuro. No os dejéis engañar por las recompensas tentadoras de los logros deshonestos; estad dispuestos a trabajar por las retribuciones posteriores inherentes a los esfuerzos honrados. El hombre sabio es capaz de distinguir entre los medios y los fines; por otra parte, un exceso de planes para el futuro a veces hace fracasar su propio propósito elevado. En cuanto a la búsqueda de los placeres, deberíais siempre aspirar a producirlos tanto como a consumirlos.

\par 
%\textsuperscript{(1779.4)}
\textsuperscript{160:4.12} Entrenad vuestra memoria para que conserve como un depósito sagrado los episodios fortalecedores y valiosos de la vida, a fin de poderlos recordar a voluntad para vuestro placer y edificación. Construid así para vosotros y dentro de vosotros galerías en reserva de belleza, de bondad y de grandeza artística. Pero los recuerdos más nobles de todos son las memorias atesoradas de los grandes momentos de una magnífica amistad. Todos estos tesoros de la memoria irradian su influencia más preciosa y sublime con el contacto liberador de la adoración espiritual.

\par 
%\textsuperscript{(1779.5)}
\textsuperscript{160:4.13} Pero la vida se convertirá en una carga de la existencia si no aprendéis a fracasar con elegancia. Aceptar las derrotas es un arte que las almas nobles siempre adquieren; debéis saber perder con alegría; debéis ser intrépidos ante las decepciones. No dudéis nunca en admitir un fracaso. No intentéis ocultar el fracaso con sonrisas engañosas y un optimismo radiante. Suena muy bien afirmar que siempre se tiene éxito, pero los resultados finales son espantosos. Esta técnica conduce directamente a la creación de un mundo irreal y a la caída inevitable en la desilusión final.

\par 
%\textsuperscript{(1779.6)}
\textsuperscript{160:4.14} El éxito puede generar la valentía y promover la confianza, pero la sabiduría sólo proviene de las experiencias de adaptación a los resultados de los fracasos personales. Los hombres que prefieren las ilusiones optimistas a la realidad, nunca podrán volverse sabios. Sólo aquellos que se enfrentan con los hechos y los adaptan a sus ideales pueden conseguir la sabiduría. La sabiduría engloba los hechos y los ideales, y por eso salva a sus adeptos de los dos extremos estériles de la filosofía ---el hombre cuyo idealismo excluye los hechos, y el materialista desprovisto de perspectiva espiritual. Las almas tímidas que sólo pueden mantener la lucha por la vida mediante la ayuda continua de las falsas ilusiones del éxito, están condenadas a sufrir fracasos y a experimentar derrotas cuando se despierten finalmente del mundo ilusorio de su propia imaginación.

\par 
%\textsuperscript{(1780.1)}
\textsuperscript{160:4.15} En esta cuestión de enfrentarse con el fracaso y de adaptarse a la derrota es donde la visión de gran alcance de la religión ejerce su influencia suprema. El fracaso es simplemente un episodio educativo ---una experiencia cultural para adquirir sabiduría--- en la experiencia del hombre que busca a Dios y que ha emprendido la aventura eterna de explorar un universo. Para este tipo de hombres, la derrota no es más que una nueva herramienta para alcanzar los niveles superiores de la realidad universal.

\par 
%\textsuperscript{(1780.2)}
\textsuperscript{160:4.16} La carrera de un hombre que busca a Dios puede resultar ser un gran éxito a la luz de la eternidad, aunque toda la empresa de su vida temporal pueda parecer un fracaso abrumador, con tal que cada fracaso de su vida haya producido el cultivo de la sabiduría y el logro espiritual. No cometáis el error de confundir el conocimiento, la cultura y la sabiduría. Están relacionados en la vida, pero representan valores espirituales extremadamente diferentes; la sabiduría domina siempre al conocimiento y glorifica siempre a la cultura.

\section*{5. La religión del Ideal}
\par 
%\textsuperscript{(1780.3)}
\textsuperscript{160:5.1} Me habéis dicho que vuestro Maestro considera que la auténtica religión humana es la experiencia del individuo con las realidades espirituales. Yo he considerado la religión como la experiencia del hombre que reacciona ante algo que le parece digno del homenaje y de la devoción de toda la humanidad. En este sentido, la religión simboliza nuestra devoción suprema a aquello que representa nuestro concepto más elevado de los ideales de la realidad, y el máximo alcance de nuestra mente hacia las posibilidades eternas de la consecución espiritual.

\par 
%\textsuperscript{(1780.4)}
\textsuperscript{160:5.2} Cuando los hombres reaccionan ante la religión en un sentido tribal, nacional o racial, es porque consideran que aquellos que no pertenecen a su grupo no son realmente humanos. Siempre consideramos que el objeto de nuestra lealtad religiosa es digno de ser venerado por todos los hombres. La religión nunca puede ser un asunto de simple creencia intelectual o de razonamiento filosófico; la religión es siempre y para siempre una manera de reaccionar ante las situaciones de la vida; es una especie de conducta. La religión abarca el pensamiento, el sentimiento y el actuar con reverencia hacia una realidad que consideramos digna de la adoración universal.

\par 
%\textsuperscript{(1780.5)}
\textsuperscript{160:5.3} Si algo se ha vuelto una religión en vuestra experiencia, es evidente que ya sois evangelistas activos de esa religión, puesto que consideráis que el concepto supremo de vuestra religión es digno de la adoración de toda la humanidad, de todas las inteligencias del universo. Si no sois unos evangelistas convencidos y misioneros de vuestra religión, os engañáis a vosotros mismos, en el sentido de que aquello que llamáis religión no es más que una creencia tradicional o un simple sistema de filosofía intelectual. Si vuestra religión es una experiencia espiritual, el objeto de vuestra adoración debe ser la realidad y el ideal espiritual universal de todos vuestros conceptos espiritualizados. Todas las religiones que se basan en el miedo, la emoción, la tradición y la filosofía, las califico de religiones intelectuales, mientras que aquellas que se basan en la verdadera experiencia espiritual las calificaría de religiones verdaderas. El objeto de la devoción religiosa puede ser material o espiritual, verdadero o falso, real o irreal, humano o divino. Las religiones pueden ser, por tanto, buenas o malas.

\par 
%\textsuperscript{(1780.6)}
\textsuperscript{160:5.4} La moralidad y la religión no son necesariamente la misma cosa. Si un sistema de moralidad se aferra a un objeto de adoración, puede volverse una religión. Cuando una religión pierde su llamamiento universal a la lealtad y a la devoción suprema, puede convertirse en un sistema de filosofía o en un código de moralidad. Esa cosa, ser, estado, orden de existencia o posibilidad de consecución que constituye el ideal supremo de la lealtad religiosa, y que es el receptor de la devoción religiosa de aquellos que adoran, es Dios. Sin tener en cuenta el nombre que se aplique a este ideal de la realidad espiritual, es Dios.

\par 
%\textsuperscript{(1781.1)}
\textsuperscript{160:5.5} La característica social de una verdadera religión consiste en el hecho de que ésta busca invariablemente convertir al individuo y transformar el mundo. La religión implica la existencia de ideales no descubiertos que trascienden de lejos las normas éticas y morales conocidas, incorporadas en los usos sociales, incluso más elevados, de las instituciones más maduras de la civilización. La religión trata de alcanzar ideales no descubiertos, realidades inexploradas, valores sobrehumanos, una sabiduría divina y un verdadero logro espiritual. La verdadera religión hace todo esto; todas las demás creencias no son dignas de este nombre. No podéis tener una religión espiritual auténtica sin el ideal supremo y celestial de un Dios eterno. Una religión sin este Dios es un invento del hombre, una institución humana de creencias intelectuales sin vida y de ceremonias emocionales sin sentido. Una religión puede pretender tener un gran ideal como objeto de su devoción. Pero estos ideales irreales son inaccesibles; un concepto así es ilusorio. Los únicos ideales susceptibles de ser alcanzados por los hombres son las realidades divinas de los valores infinitos que residen en el hecho espiritual del Dios eterno.

\par 
%\textsuperscript{(1781.2)}
\textsuperscript{160:5.6} La palabra Dios, la \textit{idea} de Dios en contraposición con el \textit{ideal} de Dios, puede volverse una parte de cualquier religión, por muy falsa o pueril que pueda ser esa religión. Y aquellos que conciben esta idea de Dios pueden hacer con ella cualquier cosa que quieran. Las religiones inferiores modelan sus ideas de Dios para satisfacer el estado natural del corazón humano; las religiones superiores exigen que el corazón humano cambie para satisfacer las demandas de los ideales de la verdadera religión.

\par 
%\textsuperscript{(1781.3)}
\textsuperscript{160:5.7} La religión de Jesús trasciende todos nuestros conceptos anteriores sobre la idea de la adoración, en el sentido de que no solamente describe a su Padre como el ideal de la realidad infinita, sino que declara categóricamente que esta fuente divina de los valores y el centro eterno del universo es verdadera y personalmente accesible para toda criatura mortal que elija entrar en el reino de los cielos en la Tierra, reconociendo así que acepta la filiación con Dios y la fraternidad con el hombre. Sugiero que éste es el concepto más elevado de la religión que el mundo haya conocido jamás, y declaro que nunca puede haber uno superior puesto que este evangelio abarca la infinidad de las realidades, la divinidad de los valores y la eternidad de los logros universales. Un concepto así constituye la realización de la experiencia del idealismo de lo supremo y de lo último.

\par 
%\textsuperscript{(1781.4)}
\textsuperscript{160:5.8} No solamente me intrigan los ideales consumados de esta religión de vuestro Maestro, sino que me siento poderosamente impulsado a confesar mi creencia en su declaración de que estos ideales de las realidades espirituales son accesibles; que vosotros y yo podemos emprender esta larga y eterna aventura, con su garantía de que al final llegaremos ciertamente a las puertas del Paraíso. Hermanos míos, soy un creyente, me he embarcado; estoy de camino con vosotros en esta aventura eterna. El Maestro dice que ha venido del Padre y que nos mostrará el camino. Estoy totalmente persuadido de que dice la verdad. Estoy definitivamente convencido de que fuera del Padre Universal y eterno no existen ideales de realidad ni valores de perfección que se puedan alcanzar.

\par 
%\textsuperscript{(1781.5)}
\textsuperscript{160:5.9} Vengo pues a adorar, no simplemente al Dios de las existencias, sino al Dios de la posibilidad de todas las existencias futuras. Por lo tanto, vuestra devoción a un ideal supremo, si este ideal es real, debe ser una devoción a este Dios de los universos pasados, presentes y futuros de cosas y de seres. Y no hay otro Dios, porque no puede haber de ninguna manera ningún otro Dios. Todos los demás dioses son invenciones de la imaginación, ilusiones de la mente mortal, distorsiones de la falsa lógica e ídolos engañosos de aquellos que los crean. Sí, podéis tener una religión sin este Dios, pero no significa nada. Si tratáis de sustituir la realidad de este ideal del Dios viviente por la palabra Dios, sólo os engañaréis a vosotros mismos poniendo una idea en el lugar de un ideal, de una realidad divina. Estas creencias son simplemente religiones de quimeras.

\par 
%\textsuperscript{(1782.1)}
\textsuperscript{160:5.10} En las enseñanzas de Jesús veo la religión en su mejor expresión. Este evangelio nos permite buscar al verdadero Dios y encontrarlo. Pero, ¿estamos dispuestos a pagar el precio de esta entrada en el reino de los cielos?. ¿Estamos dispuestos a nacer de nuevo, a ser rehechos?. ¿Estamos dispuestos a someternos a ese terrible proceso probatorio de la destrucción del yo y de la reconstrucción del alma?. ¿Acaso no ha dicho el Maestro: <<El que quiera salvar su vida ha de perderla. No creáis que he venido para traer la paz, sino más bien una lucha del alma>>?. Es verdad que después de pagar el precio de la dedicación a la voluntad del Padre experimentamos una gran paz, a condición de que continuemos caminando en los senderos espirituales de la vida consagrada.

\par 
%\textsuperscript{(1782.2)}
\textsuperscript{160:5.11} Ahora estamos abandonando de verdad los alicientes del orden de existencia conocido, mientras nos dedicamos sin reservas a buscar los encantos del orden de existencia desconocido e inexplorado de una vida futura de aventuras en los mundos espirituales del idealismo superior de la realidad divina. Y buscamos esos símbolos significativos con los que transmitir a nuestros semejantes estos conceptos de la realidad del idealismo de la religión de Jesús, y no dejaremos de rezar por ese día en que toda la humanidad se emocionará con la visión común de esta verdad suprema. En este momento, nuestro concepto focalizado del Padre, tal como lo tenemos en nuestro corazón, es que Dios es espíritu; tal como lo trasmitimos a nuestros semejantes, Dios es amor.

\par 
%\textsuperscript{(1782.3)}
\textsuperscript{160:5.12} La religión de Jesús exige una experiencia viviente y espiritual. Otras religiones pueden consistir en creencias tradicionales, sentimientos emotivos, conciencias filosóficas, y todo eso junto, pero la enseñanza del Maestro requiere que se alcancen los niveles reales del progreso espiritual verdadero.

\par 
%\textsuperscript{(1782.4)}
\textsuperscript{160:5.13} La conciencia del impulso a ser semejante a Dios no es la verdadera religión. Los sentimientos emotivos de adorar a Dios no son la verdadera religión. La convicción consciente de abandonar el yo y servir a Dios no es la verdadera religión. La sabiduría del razonamiento de que esta religión es la mejor de todas, no es la religión como experiencia personal y espiritual. La verdadera religión tiene relación con el destino y la realidad de lo que se logra, así como con la realidad y el idealismo de aquello que se acepta de todo corazón por la fe. Y todo esto debe hacerse personal para nosotros mediante la revelación del Espíritu de la Verdad.

\par 
%\textsuperscript{(1782.5)}
\textsuperscript{160:5.14} Así terminaron las disertaciones del filósofo griego, uno de los más grandes de su raza, que se había vuelto creyente en el evangelio de Jesús.