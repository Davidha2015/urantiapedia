\chapter{Documento 162. En la fiesta de los tabernáculos}
\par 
%\textsuperscript{(1788.1)}
\textsuperscript{162:0.1} CUANDO Jesús partió hacia Jerusalén con los diez apóstoles, decidió pasar a través de Samaria porque era el camino más corto. En consecuencia, se dirigieron por la costa oriental del lago, y entraron en la frontera de Samaria a través de Escitópolis. Al anochecer, Jesús envió a Felipe y Mateo a un pueblo situado en las pendientes orientales del Monte Gilboa, para asegurar el alojamiento del grupo. Pero sucedió que aquellos aldeanos tenían grandes prejuicios contra los judíos, más grandes aún que la mayoría de los samaritanos, y estos sentimientos se encontraban exaltados en aquel preciso momento ya que muchos judíos se dirigían a la fiesta de los tabernáculos. Esta gente sabía muy pocas cosas sobre Jesús, y se negaron a alojarlo porque él y sus asociados eran judíos\footnote{\textit{Se niega cobijo a los apóstoles}: Lc 9:51-53.}. Cuando Mateo y Felipe manifestaron su indignación e informaron a estos samaritanos de que estaban rechazando hospedar al Santo de Israel, los enfurecidos aldeanos los echaron a palos y pedradas de su pequeña ciudad.

\par 
%\textsuperscript{(1788.2)}
\textsuperscript{162:0.2} Felipe y Mateo regresaron con sus compañeros y les contaron cómo habían sido echados del pueblo; entonces, Santiago y Juan se acercaron a Jesús y le dijeron: «Maestro, te rogamos que nos des permiso para pedir que caiga fuego del cielo y destruya a esos samaritanos insolentes e impenitentes». Cuando Jesús escuchó estas palabras de venganza, se volvió hacia los hijos de Zebedeo y les reprendió con severidad\footnote{\textit{Jesús reprende a los apóstoles}: Lc 9:54-55.}: «No sabéis el tipo de actitud que estáis manifestando. La venganza no tiene cabida en el reino de los cielos. En lugar de discutir, vamos hacia el pueblecito que se encuentra cerca del vado del Jordán». Y así, a causa de sus prejuicios sectarios, estos samaritanos se privaron del honor de ofrecer su hospitalidad al Hijo Creador de un universo.

\par 
%\textsuperscript{(1788.3)}
\textsuperscript{162:0.3} Jesús y los diez se detuvieron para pasar la noche en el pueblo cercano al vado del Jordán\footnote{\textit{Utilizan otra ciudad}: Lc 9:56.}. A primeras horas del día siguiente, atravesaron el río y continuaron su camino hacia Jerusalén por la carretera al este del Jordán, llegando a Betania al final de la tarde del miércoles. Tomás y Natanael, que se habían retrasado a causa de sus conversaciones con Rodán, llegaron el viernes.

\par 
%\textsuperscript{(1788.4)}
\textsuperscript{162:0.4} Jesús y los doce permanecieron en las cercanías de Jerusalén hasta el final del mes siguiente (octubre), aproximadamente cuatro semanas y media. El mismo Jesús sólo entró en la ciudad unas pocas veces, y estas breves visitas tuvieron lugar durante los días de la fiesta de los tabernáculos. Una gran parte del mes de octubre la pasó en Belén con Abner y sus asociados.

\section*{1. Los peligros de la visita a Jerusalén}
\par 
%\textsuperscript{(1788.5)}
\textsuperscript{162:1.1} Mucho antes de que huyeran de Galilea, los seguidores de Jesús le habían suplicado que fuera a Jerusalén para proclamar el evangelio del reino, a fin de que su mensaje tuviera el prestigio de haber sido predicado en el centro de la cultura y de la erudición judías; pero ahora que había venido de hecho a Jerusalén para enseñar\footnote{\textit{Predicación en Jerusalén}: Jn 7:2-4.}, temían por su vida. Sabiendo que el sanedrín había intentado llevar a Jesús a Jerusalén para juzgarlo, y al recordar las recientes declaraciones reiteradas del Maestro de que debía someterse a la muerte, los apóstoles se habían quedado literalmente pasmados ante su repentina decisión de asistir a la fiesta de los tabernáculos. A todas sus súplicas anteriores para que fuera a Jerusalén, Jesús había contestado: «Aún no ha llegado la hora»\footnote{\textit{La hora no ha llegado}: Jn 7:6.}. Ahora, ante sus protestas de temor, se limitaba a responder: «Pero ya ha llegado la hora».

\par 
%\textsuperscript{(1789.1)}
\textsuperscript{162:1.2} Durante la fiesta de los tabernáculos, Jesús entró audazmente en Jerusalén en varias ocasiones y enseñó públicamente en el templo. Hizo esto a pesar de los esfuerzos de sus apóstoles por disuadirlo. Aunque le habían insistido durante mucho tiempo para que proclamara su mensaje en Jerusalén, ahora temían verlo entrar en la ciudad en estos momentos, porque sabían muy bien que los escribas y los fariseos estaban decididos a llevarlo a la muerte.

\par 
%\textsuperscript{(1789.2)}
\textsuperscript{162:1.3} La audaz aparición de Jesús en Jerusalén confundió más que nunca a sus seguidores. Muchos discípulos suyos, e incluso el apóstol Judas Iscariote, se habían atrevido a pensar que Jesús había huido precipitadamente a Fenicia porque tenía miedo de los dirigentes judíos y de Herodes Antipas. No comprendían el significado de los desplazamientos del Maestro\footnote{\textit{Incomprensión de las motivaciones de Jesús}: Jn 7:1.}. Su presencia en Jerusalén en la fiesta de los tabernáculos, incluso en contra de los consejos de sus seguidores, bastó para poner fin definitivamente a todos los rumores sobre su miedo y su cobardía.

\par 
%\textsuperscript{(1789.3)}
\textsuperscript{162:1.4} Durante la fiesta de los tabernáculos, miles de creyentes de todas las partes del imperio romano vieron a Jesús, le oyeron enseñar, y muchos de ellos fueron incluso hasta Betania para conversar con él sobre el progreso del reino en sus regiones nativas.

\par 
%\textsuperscript{(1789.4)}
\textsuperscript{162:1.5} Había muchas razones para que Jesús pudiera predicar públicamente en los patios del templo durante los días de la fiesta; la razón principal era el miedo que se había adueñado de los oficiales del sanedrín a consecuencia de la secreta división de sentimientos que se había producido en sus propias filas. Era un hecho de que muchos miembros del sanedrín creían secretamente en Jesús o bien estaban decididamente en contra de que se le arrestara durante la fiesta, cuando tantísimos visitantes estaban presentes en Jerusalén, muchos de los cuales creían en él o al menos simpatizaban con el movimiento espiritual que patrocinaba.

\par 
%\textsuperscript{(1789.5)}
\textsuperscript{162:1.6} Los esfuerzos de Abner y de sus compañeros a través de Judea también habían contribuido mucho a consolidar un sentimiento favorable hacia el reino, de tal manera que los enemigos de Jesús no se atrevían a manifestar demasiado abiertamente su oposición. Ésta fue una de las razones por las que Jesús pudo visitar públicamente Jerusalén y salir de allí con vida. Uno o dos meses antes, le hubieran dado muerte con toda seguridad.

\par 
%\textsuperscript{(1789.6)}
\textsuperscript{162:1.7} El atrevimiento audaz de Jesús de aparecer públicamente en Jerusalén intimidó a sus enemigos; no estaban preparados para un desafío tan atrevido. Durante este mes, el sanedrín hizo débiles tentativas por arrestar al Maestro en varias ocasiones, pero estos esfuerzos no condujeron a nada. Sus enemigos estaban tan sorprendidos por la inesperada aparición pública de Jesús en Jerusalén, que supusieron que las autoridades romanas le habían prometido su protección. Como sabían que Felipe (el hermano de Herodes Antipas) era casi un discípulo de Jesús, los miembros del sanedrín consideraron que Felipe había obtenido unas promesas para proteger a Jesús de sus enemigos. Antes de que se dieran cuenta de que se habían equivocado al creer que su aparición repentina y audaz en Jerusalén se debía a un acuerdo secreto con los funcionarios romanos, Jesús ya había salido del territorio de su jurisdicción.

\par 
%\textsuperscript{(1789.7)}
\textsuperscript{162:1.8} Sólo los doce apóstoles sabían que Jesús se proponía asistir a la fiesta de los tabernáculos cuando partieron de Magadán\footnote{\textit{Jesús viajó en secreto}: Jn 7:8-10.}. Los otros seguidores del Maestro se quedaron muy asombrados cuando apareció en los patios del templo y empezó a enseñar públicamente, y las autoridades judías se llevaron una sorpresa indescriptible cuando les informaron que estaba enseñando en el templo.

\par 
%\textsuperscript{(1790.1)}
\textsuperscript{162:1.9} Aunque los discípulos de Jesús no esperaban que asistiera a la fiesta, la gran mayoría de los peregrinos que venían de lejos, y que habían oído hablar de él, albergaban la esperanza de poder verlo en Jerusalén\footnote{\textit{Las masas esperaban verle}: Jn 7:11.}. Y no quedaron decepcionados, porque enseñó en diversas ocasiones en el Pórtico de Salomón y en otras partes de los patios del templo. En realidad, estas enseñanzas fueron la proclamación oficial o solemne de la divinidad de Jesús al pueblo judío y al mundo entero.

\par 
%\textsuperscript{(1790.2)}
\textsuperscript{162:1.10} Las opiniones de las multitudes que escuchaban las enseñanzas del Maestro estaban divididas\footnote{\textit{División de las opiniones}: Jn 7:12-13.}. Unos decían que era un buen hombre; otros, que era un profeta; otros, que era realmente el Mesías; otros decían que era un intrigante malicioso, que desviaba a la gente con sus doctrinas extrañas. Sus enemigos dudaban en acusarlo abiertamente por temor a los creyentes que estaban a su favor, mientras que sus amigos temían reconocerlo abiertamente por temor a los dirigentes judíos, sabiendo que el sanedrín estaba decidido a matarlo. Pero incluso sus enemigos se maravillaban de su enseñanza\footnote{\textit{Se maravillaban de sus enseñanzas}: Jn 7:15.}, pues sabían que no había sido instruido en las escuelas de los rabinos.

\par 
%\textsuperscript{(1790.3)}
\textsuperscript{162:1.11} Cada vez que Jesús iba a Jerusalén, sus apóstoles se llenaban de terror. Día tras día, se sentían más atemorizados cuando escuchaban sus declaraciones cada vez más audaces sobre la naturaleza de su misión en la Tierra. No estaban acostumbrados a escuchar a Jesús hacer unas proclamaciones tan rotundas y unas afirmaciones tan sorprendentes, ni siquiera cuando predicaba entre sus amigos.

\section*{2. El primer discurso en el templo}
\par 
%\textsuperscript{(1790.4)}
\textsuperscript{162:2.1} La primera tarde que Jesús enseñó en el templo, un número considerable de personas estaban sentadas y escuchaban sus palabras describiendo la libertad del nuevo evangelio y la alegría de los que creen en la buena nueva, cuando un oyente curioso le interrumpió para preguntar: «Maestro, ¿cómo puede ser que puedas citar las Escrituras y enseñar a la gente con tanta facilidad, cuando me dicen que no has sido instruido en la ciencia de los rabinos?» Jesús contestó: «Ningún hombre me ha enseñado las verdades que os proclamo. Esta enseñanza no es mía, sino de Aquél que me ha enviado\footnote{\textit{Origen del mensaje de Jesús}: Jn 7:14-19.}. Si algún hombre desea realmente hacer la voluntad de mi Padre, sabrá con certeza si mi enseñanza viene de Dios o si hablo por mi mismo. El que habla por sí mismo busca su propia gloria, pero cuando proclamo las palabras del Padre, busco así la gloria de aquél que me ha enviado. Pero antes de intentar entrar en la nueva luz, ¿no deberíais seguir más bien la luz que ya poseéis? Moisés os dio la ley, y sin embargo, ¿cuántos de vosotros intentan honradamente satisfacer sus exigencias? En esta ley, Moisés os ordena: `No matarás'; y a pesar de este mandamiento, algunos de vosotros pretenden matar al Hijo del Hombre»\footnote{\textit{No matarás}: Ex 20:13; Dt 5:17; Mt 5:21; Lc 18:20; Ro 13:9; Stg 2:11.}.

\par 
%\textsuperscript{(1790.5)}
\textsuperscript{162:2.2} Cuando la multitud escuchó estas palabras, empezaron a discutir entre ellos. Algunos decían que estaba loco; otros, que estaba poseído por un demonio. Otros decían que éste era en verdad el profeta de Galilea que los escribas y fariseos intentaban matar desde hacía tiempo. Algunos decían que las autoridades religiosas tenían miedo de molestarlo; otros pensaban que no le habían echado mano porque se habían convertido en creyentes suyos. Después de una discusión prolongada, un miembro de la muchedumbre se adelantó y le preguntó a Jesús: «Por qué los dirigentes intentan matarte?» Y él respondió: «Los dirigentes pretenden matarme porque les indigna mi enseñanza sobre la buena nueva del reino, un evangelio que libera a los hombres de las pesadas tradiciones de una religión formalista de ceremonias, que esos educadores están decididos a mantener a toda costa. Practican la circuncisión, de acuerdo con la ley, el día del sábado, pero quieren matarme porque una vez, en un día de sábado, liberé a un hombre que era esclavo de una aflicción. Me siguen durante el sábado para espiarme, pero quieren matarme porque en otra ocasión escogí curar por completo, un día de sábado, a un hombre que estaba gravemente enfermo. Tratan de matarme porque saben muy bien que si creéis honradamente en mi enseñanza y os atrevéis a aceptarla, su sistema de religión tradicional será derrocado, destruido para siempre. Y así se quedarán privados de autoridad sobre aquello a lo que han consagrado su vida, puesto que se niegan firmemente a aceptar este evangelio nuevo y más glorioso del reino de Dios. Y ahora sí que os lo pido a cada uno de vosotros: No juzguéis por las apariencias exteriores, sino juzgad más bien por el verdadero espíritu de estas enseñanzas; juzgad con rectitud»\footnote{\textit{¿Por qué los judíos querían matar a Jesús?}: Jn 7:20-24.}.

\par 
%\textsuperscript{(1791.1)}
\textsuperscript{162:2.3} Entonces, otro indagador dijo: «Sí, Maestro, buscamos al Mesías, pero sabemos que cuando llegue, su aparición se producirá de manera misteriosa. Sabemos de dónde vienes. Has estado entre tus hermanos desde el principio. El libertador vendrá con poder para restaurar el trono del reino de David. ¿Pretendes realmente ser el Mesías?» Jesús respondió: «Pretendes conocerme y saber de dónde vengo. Desearía que tus afirmaciones fueran verdaderas, porque entonces sí que encontrarías una vida abundante en ese conocimiento. Pero os aseguro que no he venido hasta vosotros por mí mismo; he sido enviado por el Padre, y aquél que me ha enviado es verdadero y fiel. Cuando os negáis a escucharme, os negáis a recibir a Aquél que me envía. Si recibís este evangelio, llegaréis a conocer a Aquél que me ha enviado. Yo conozco al Padre, porque he venido del Padre para proclamarlo y revelarlo a vosotros»\footnote{\textit{Acerca del Mesías}: Jn 7:25-29.}.

\par 
%\textsuperscript{(1791.2)}
\textsuperscript{162:2.4} Los agentes de los escribas querían prenderlo, pero le tenían miedo a la multitud porque muchos creían en él. La obra de Jesús desde su bautismo era bien conocida en toda la sociedad judía, y cuando mucha de esta gente refería estas cosas, se decían entre ellos: «Aunque este instructor sea de Galilea, y aunque no satisfaga todas nuestras expectativas del Mesías, nos preguntamos si cuando llegue el libertador hará realmente algo más maravilloso que lo que ya ha hecho este Jesús de Nazaret»\footnote{\textit{La gente se pregunta si Jesús será el libertador}: Jn 7:30-31.}.

\par 
%\textsuperscript{(1791.3)}
\textsuperscript{162:2.5} Cuando los fariseos y sus agentes escucharon al pueblo hablar de esta manera\footnote{\textit{Los fariseos escuchan al pueblo}: Jn 7:32a.}, consultaron con sus dirigentes y decidieron que había que hacer algo inmediatamente para poner fin a estas apariciones públicas de Jesús en los patios del templo. En general, los dirigentes de los judíos estaban dispuestos a evitar un enfrentamiento con Jesús, pues creían que las autoridades romanas le habían prometido la inmunidad. No podían explicarse de otra manera la audacia que tenía de venir en esta época a Jerusalén; pero los funcionarios del sanedrín no creían totalmente en este rumor. Deducían que los gobernantes romanos no hubieran hecho una cosa así en secreto y sin que lo supieran las más altas autoridades de la nación judía.

\par 
%\textsuperscript{(1791.4)}
\textsuperscript{162:2.6} En consecuencia, Eber, el oficial apropiado del sanedrín, fue enviado con dos asistentes para arrestar a Jesús\footnote{\textit{Envían oficiales para arrestrar a Jesús}: Jn 7:32b.}. Mientras Eber se abría paso hacia Jesús, el Maestro dijo: «No tengas miedo de aproximarte a mí. Acércate mientras escuchas mi enseñanza. Sé que has sido enviado para capturarme, pero deberías comprender que al Hijo del Hombre no le sucederá nada hasta que llegue su hora\footnote{\textit{No ocurrirá nada hasta llegada la hora}: Jn 7:30.}. Tú no estás en contra mía; sólo vienes a ejecutar la orden de tus superiores, e incluso esos dirigentes de los judíos creen de verdad que están sirviendo a Dios cuando buscan en secreto mi destrucción».

\par 
%\textsuperscript{(1792.1)}
\textsuperscript{162:2.7} «No os tengo aversión a ninguno de vosotros. El Padre os ama, y por eso deseo vuestra liberación de la esclavitud a los prejuicios y a las tinieblas de la tradición. Os ofrezco la libertad de la vida y la alegría de la salvación. Proclamo el nuevo camino viviente\footnote{\textit{Jesús: el nuevo camino viviente}: Jn 14:6; Heb 10:20.}, la liberación del mal y la ruptura de la servidumbre del pecado. He venido para que podáis tener la vida, y la tengáis eternamente\footnote{\textit{Vida eterna}: Dn 12:2; Mt 19:16,29; 25:46; Mc 10:17,30; Lc 10:25; 18:18,30; Jn 3:15-16,36; 4:14,36; 5:24,39; 6:27,40,47; 6:54,68; 8:51-52; 10:28; 11:25-26; 12:25,50; 17:2-3; Hch 13:46-48; Ro 2:7; 5:21; 6:22-23; Gl 6:8; 1 Ti 1:16; 6:12,19; Tit 1:2; 3:7; 1 Jn 1:2; 2:25; 3:15; 5:11,13,20; Jud 1:21; Ap 22:5.}. Intentáis desembarazaros de mí y de mis enseñanzas inquietantes. ¡Si pudierais daros cuenta de que sólo estaré poco tiempo con vosotros! Dentro de poco volveré hacia Aquél que me ha enviado a este mundo\footnote{\textit{Volveré a Aquel que me ha enviado}: Jn 7:33; 16:5.}. Entonces, muchos de vosotros me buscaréis con diligencia, pero no descubriréis mi presencia\footnote{\textit{Me buscaréis pero no me encontraréis}: Jn 7:34; 8:21a.}, porque no podéis venir adonde estoy a punto de ir\footnote{\textit{No podéis venir conmigo}: Jn 7:34-36; 8:21b-23.}. Pero todos los que traten sinceramente de encontrarme, alcanzarán alguna vez la vida que conduce a la presencia de mi Padre».

\par 
%\textsuperscript{(1792.2)}
\textsuperscript{162:2.8} Algunos de los que se mofaban se dijeron entre ellos: «¿Adónde irá este hombre para que no podamos encontrarlo? ¿Se irá a vivir con los griegos? ¿Se quitará la vida? ¿Qué quiere decir cuando afirma que pronto nos dejará y que no podemos ir adonde él va?»\footnote{\textit{Frases de burla}: Jn 7:35-36; Jn 8:22.}

\par 
%\textsuperscript{(1792.3)}
\textsuperscript{162:2.9} Eber y sus ayudantes se negaron a detener a Jesús, y regresaron sin él a su lugar de reunión. Por consiguiente, cuando los sacerdotes principales y los fariseos reprendieron a Eber y a sus ayudantes por no haber traído a Jesús\footnote{\textit{Los oficiales que regresan sin Jesús, reprendidos}: Jn 7:45-52.}, Eber se limitó a contestar: «Hemos tenido miedo de arrestarlo en medio de la multitud, porque muchos de ellos creen en él. Además, nunca hemos oído a nadie hablar como ese hombre. Ese instructor tiene algo fuera de lo común. Todos haríais bien en ir a escucharlo». Cuando los jefes principales escucharon estas palabras, se quedaron sorprendidos y le dijeron sarcásticamente a Eber: «¿También tú te has extraviado? ¿Estás a punto de creer en ese impostor? ¿Has oído decir que alguno de nuestros sabios o de nuestros dirigentes haya creído en él? ¿Algún escriba o fariseo ha sido engañado por sus hábiles enseñanzas? ¿Cómo puede ser que te dejes influir por la conducta de esa multitud ignorante que no conoce ni la ley ni los profetas? ¿No sabes que esos iletrados están malditos?» Entonces, Eber contestó: «Es verdad, señores, pero ese hombre dirige palabras de misericordia y de esperanza a la multitud. Anima a los abatidos, y sus palabras han confortado incluso nuestras almas. ¿Qué puede haber de malo en esas enseñanzas, aunque no sea el Mesías de las Escrituras? Y aún así, ¿es que nuestra ley no exige la equidad? ¿Condenamos a un hombre antes de escucharlo?» El jefe del sanedrín se encolerizó con Eber y, volviéndose hacia él, le dijo: «¿Te has vuelto loco? ¿Eres por casualidad también de Galilea? Busca en las Escrituras, y descubrirás que no surge ningún profeta de Galilea, y mucho menos el Mesías»\footnote{\textit{Ningún profeta surge de Galilea}: Jn 7:41.}.

\par 
%\textsuperscript{(1792.4)}
\textsuperscript{162:2.10} El sanedrín se dispersó en la confusión, y Jesús se retiró a Betania para pasar la noche.\footnote{\textit{Todos se retiran a casa}: Jn 7:53.}

\section*{3. La mujer sorprendida en adulterio}
\par 
%\textsuperscript{(1792.5)}
\textsuperscript{162:3.1} Fue durante esta visita a Jerusalén\footnote{\textit{Enseñanza en el Templo}: Jn 8:2.} cuando Jesús intervino en el caso de cierta mujer de mala reputación que los acusadores de ella y los enemigos del Maestro trajeron a su presencia. El relato tergiversado que poseéis de este episodio insinúa que los escribas y fariseos habían llevado a esta mujer ante Jesús, y que Jesús los trató de tal manera que daba a entender que estos jefes religiosos de los judíos podían haber sido ellos mismos culpables de inmoralidad. Jesús sabía muy bien que estos escribas y fariseos estaban espiritualmente ciegos y llenos de prejuicios intelectuales a causa de su lealtad a la tradición, pero que había que contarlos entre los hombres más completamente morales de aquella época y de aquella generación.

\par 
%\textsuperscript{(1793.1)}
\textsuperscript{162:3.2} He aquí lo que sucedió en realidad: A primeras horas de la tercera mañana de la fiesta, cuando Jesús se acercaba al templo, se encontró con un grupo de agentes pagados por el sanedrín que arrastraban con ellos a una mujer. Cuando se acercaron, el portavoz dijo: «Maestro, esta mujer ha sido sorprendida in fraganti cometiendo adulterio\footnote{\textit{La adúltera pillada en el acto}: Jn 8:3-5.}. Pues bien, la ley de Moisés ordena que una mujer así debe ser lapidada. Según tú, ¿qué se debería hacer con ella?»

\par 
%\textsuperscript{(1793.2)}
\textsuperscript{162:3.3} Los enemigos de Jesús planeaban lo siguiente: Si apoyaba la ley de Moisés, la cual exigía que la pecadora que confesaba su falta fuera apedreada, lo enredarían en dificultades con los dirigentes romanos, que habían negado a los judíos el derecho de infligir la pena de muerte sin la aprobación de un tribunal romano. Si prohibía apedrear a la mujer, lo acusarían ante el sanedrín de elevarse por encima de Moisés y de la ley judía. Si permanecía en silencio, lo acusarían de cobardía\footnote{\textit{El dilema}: Jn 8:6a.}. Pero el Maestro manejó la situación de tal manera que toda la trama se hizo pedazos por el propio peso de su mezquindad.

\par 
%\textsuperscript{(1793.3)}
\textsuperscript{162:3.4} Esta mujer, en otra época bien parecida, era la esposa de un ciudadano inferior de Nazaret, un hombre que le había causado dificultades a Jesús durante toda su juventud. Después de haberse casado con esta mujer, la forzó de la manera más vergonzosa a ganarse la vida de los dos comerciando con su cuerpo. Había venido a la fiesta de Jerusalén para que su mujer pudiera prostituir así sus encantos físicos y obtener una ganancia monetaria. Había hecho un pacto con los mercenarios de los dirigentes judíos para traicionar así a su propia esposa en su vicio comercializado. Por eso venían con la mujer y su compañero de culpa, a fin de que Jesús cayera en una trampa al efectuar alguna declaración que pudiera ser utilizada contra él en el caso de que fuera arrestado.

\par 
%\textsuperscript{(1793.4)}
\textsuperscript{162:3.5} Jesús examinó superficialmente a la multitud, y vio al marido que se encontraba detrás de los demás. Sabía el tipo de hombre que era y percibió que era cómplice en esta transacción despreciable. Jesús empezó por caminar alrededor de la multitud para acercarse donde se encontraba este marido degenerado, y escribió unas palabras en la arena que le hicieron marcharse precipitadamente\footnote{\textit{Jesús escribe en la arena}: Jn 8:8-11.}. Luego regresó ante la mujer y escribió de nuevo en el suelo en provecho de sus pretendidos acusadores; cuando leyeron sus palabras, también se fueron uno tras otro. Cuando el Maestro escribió por tercera vez en la arena, el compañero de infortunio de la mujer se alejó a su vez, de manera que cuando el Maestro se incorporó después de escribir, observó que la mujer estaba sola delante de él. Jesús dijo: «Mujer, ¿dónde están tus acusadores? ¿Ya no queda nadie para lapidarte?» La mujer levantó la mirada y respondió: «Nadie, Señor». Jesús dijo entonces: «Conozco tu caso, y yo tampoco te condeno. Puedes irte en paz». Y esta mujer, llamada Hildana, abandonó a su perverso marido y se unió a los discípulos del reino.

\section*{4. La fiesta de los tabernáculos}
\par 
%\textsuperscript{(1793.5)}
\textsuperscript{162:4.1} La presencia de una gente que venía de todos los rincones del mundo conocido, desde España hasta la India, hacía que la fiesta de los tabernáculos\footnote{\textit{La fiesta de los Tabernáculos}: Lv 23:39-43.} fuera una ocasión ideal para que Jesús proclamara públicamente, por primera vez en Jerusalén, la totalidad de su evangelio. Durante esta fiesta, la gente vivía mucho al aire libre, en cabañas de hojas. Era la fiesta de la cosecha y al tener lugar, como así era, en el frescor de los meses de otoño, los judíos del mundo asistían en mayor número que a la fiesta de la Pascua, al final del invierno, o a la de Pentecostés al principio del verano. Los apóstoles veían, por fin, a su Maestro proclamar audazmente su misión en la Tierra delante, por así decirlo, del mundo entero.

\par 
%\textsuperscript{(1794.1)}
\textsuperscript{162:4.2} Ésta era la fiesta de las fiestas, pues todo sacrificio que no se hubiera efectuado en las otras festividades se podía hacer en este momento. Ésta era la ocasión en que se recibían las ofrendas en el templo; era una combinación de los placeres de las vacaciones y de los ritos solemnes del culto religioso. Era un momento de regocijo racial, mezclado con los sacrificios, los cantos levíticos y los toques solemnes de las trompetas plateadas de los sacerdotes\footnote{\textit{Toque de las trompetas plateadas}: Nm 10:1-7.}. Por la noche, el impresionante espectáculo del templo y sus multitudes de peregrinos estaba intensamente iluminado por los grandes candelabros que ardían con esplendor en el patio de las mujeres, así como por el resplandor de docenas de antorchas colocadas en los patios del templo. Toda la ciudad estaba decorada alegremente, excepto el castillo romano de Antonia, que dominaba con un contraste siniestro esta escena festiva y de culto. ¡Cuánto odiaban los judíos este recordatorio siempre presente del yugo romano!

\par 
%\textsuperscript{(1794.2)}
\textsuperscript{162:4.3} Durante la fiesta se sacrificaban setenta bueyes, el símbolo de las setenta naciones del mundo pagano. La ceremonia del derramamiento del agua simbolizaba la efusión del espíritu divino. Esta ceremonia del agua tenía lugar después de la procesión de los sacerdotes y levitas a la salida del Sol. Los fieles bajaban por los peldaños que conducían del patio de Israel al patio de las mujeres, mientras sonaba una sucesión de toques en las trompetas de plata. Luego, los fieles continuaban caminando hacia la hermosa puerta, que se abría hacia el patio de los gentiles. Allí se volvían para ponerse frente al oeste, repetir sus cantos y continuar su camino hacia el agua simbólica.

\par 
%\textsuperscript{(1794.3)}
\textsuperscript{162:4.4} Casi cuatrocientos cincuenta sacerdotes, con un número correspondiente de levitas, oficiaban el último día de la fiesta. Al amanecer se congregaban los peregrinos de todas las partes de la ciudad; cada uno llevaba en la mano derecha un haz de mirto, de sauce y de palmas, y en la mano izquierda una rama de manzana del paraíso ---la cidra o «fruta prohibida». Estos peregrinos se dividían en tres grupos para esta ceremonia matutina. Un grupo permanecía en el templo para asistir a los sacrificios de la mañana. Otro grupo bajaba de Jerusalén hasta cerca de Maza para cortar las ramas de sauce destinadas a adornar el altar de los sacrificios. El tercer grupo formaba una procesión que caminaba detrás del sacerdote encargado del agua, que llevaba la jarra de oro destinada a contener el agua simbólica; este grupo salía del templo por Ofel y llegaba hasta cerca de Siloé, donde se encontraba la puerta de la fuente. Después de haber llenado la jarra de oro en el estanque de Siloé, la procesión regresaba al templo, entraba por la puerta del agua y se dirigía directamente al patio de los sacerdotes, donde el sacerdote que llevaba la jarra de agua se unía al sacerdote que llevaba el vino para la ofrenda de la bebida. Estos dos sacerdotes se encaminaban después a los embudos de plata que conducían a la base del altar, y vertían en ellos el contenido de las jarras. La ejecución de este rito de verter el vino y el agua era la señal que esperaban los peregrinos reunidos para empezar a cantar los salmos 113 al 118 inclusive\footnote{\textit{Salmos 113-118}: Sal 113-118.}, alternando con los levitas. A medida que repetían estos versos, ondeaban sus haces hacia el altar. Luego se realizaban los sacrificios del día, asociados con la repetición del salmo del día; el último día de la fiesta se repetía el salmo ochenta y dos a partir del quinto verso\footnote{\textit{Salmo 82 versículos 5-8}: Sal 82:5-8.}.

\section*{5. El sermón sobre la luz del mundo}
\par 
%\textsuperscript{(1794.4)}
\textsuperscript{162:5.1} Al anochecer del penúltimo día de la fiesta, cuando la escena se encontraba intensamente iluminada por las luces de los candelabros y de las antorchas, Jesús se levantó en medio de la multitud reunida y dijo:

\par 
%\textsuperscript{(1795.1)}
\textsuperscript{162:5.2} «Yo soy la luz del mundo\footnote{\textit{Jesús es la luz del mundo}: Is 9:2; Mt 4:16; Lc 1:78-79; 2:32; Jn 1:4-9; 3:19; 8:12; 9:5; 12:35-36.}. El que me sigue no caminará en las tinieblas, sino que tendrá la luz de la vida. Como os atrevéis a enjuiciarme y asumís el papel de jueces, declaráis que si doy testimonio de mí mismo\footnote{\textit{Testimonio de la identidad de Jesús}: Jn 8:13-19.}, mi testimonio no puede ser verdadero. Pero la criatura nunca puede juzgar al Creador. Aunque dé testimonio de mí mismo, mi testimonio es eternamente verdadero, porque sé de dónde he venido, quién soy y adónde voy. Vosotros, que queréis matar al Hijo del Hombre, no sabéis de dónde he venido, quién soy, ni adónde voy. Sólo juzgáis por las apariencias de la carne; no percibís las realidades del espíritu. Yo no juzgo a nadie, ni siquiera a mi mayor enemigo. Pero si decidiera juzgar, mi juicio sería exacto y recto, porque yo no juzgaría solo, sino en asociación con mi Padre que me ha enviado al mundo, y que es la fuente de todo juicio verdadero. Admitís incluso que se puede aceptar el testimonio de dos personas dignas de confianza ---pues bien, doy testimonio de esas verdades, y mi Padre que está en los cielos también lo da. Cuando ayer os dije esto mismo, me preguntasteis en vuestra ignorancia: `¿Dónde está tu Padre?' En verdad, no me conocéis ni a mí ni a mi Padre, porque si me hubierais conocido, habríais conocido también al Padre».

\par 
%\textsuperscript{(1795.2)}
\textsuperscript{162:5.3} «Ya os he dicho que me marcho\footnote{\textit{Jesús se marcha}: Jn 7:33; 8:21a.}, y que me buscaréis pero que no me encontraréis, porque allí donde voy no podéis venir\footnote{\textit{No podéis venir}: Jn 7:34-36; 8:21b-23.}. Vosotros, que quisierais rechazar esta luz, sois de abajo; yo soy de arriba. Vosotros, que preferís permanecer en las tinieblas\footnote{\textit{Muchos prefieren permanecer en las tinieblas}: Is 9:2; 50:10; Mt 4:16; Jn 8:12.}, sois de este mundo; yo no soy de este mundo, y vivo en la luz eterna del Padre de las luces\footnote{\textit{Padre de las luces}: Stg 1:17.}. Todos habéis tenido numerosas oportunidades para saber quién soy, pero tendréis además otras pruebas que confirmarán la identidad del Hijo del Hombre. Yo soy la luz de la vida, y todo aquél que rechaza deliberadamente y a sabiendas esta luz salvadora, morirá en sus pecados. Tengo muchas cosas que deciros, pero sois incapaces de recibir mis palabras. Sin embargo, aquél que me ha enviado es verdadero y fiel; mi Padre ama incluso a sus hijos descarriados. Y todo lo que mi Padre ha dicho, yo también lo proclamo al mundo».

\par 
%\textsuperscript{(1795.3)}
\textsuperscript{162:5.4} «Cuando el Hijo del Hombre sea elevado\footnote{\textit{El Hijo será elevado}: Jn 3:14; 8:28-29; 12:32,34.}, entonces todos sabréis que yo soy él, y que no he hecho nada por mí mismo, sino tan sólo lo que el Padre me ha enseñado. Os dirijo estas palabras a vosotros y a vuestros hijos. Aquél que me ha enviado también está ahora conmigo; no me ha dejado solo, porque siempre hago lo que resulta agradable a sus ojos».

\par 
%\textsuperscript{(1795.4)}
\textsuperscript{162:5.5} Mientras Jesús enseñaba así a los peregrinos en los patios del templo, muchos creyeron\footnote{\textit{Muchos creyeron}: Jn 8:30.}. Y nadie se atrevió a apresarlo.

\section*{6. El discurso sobre el agua de la vida}
\par 
%\textsuperscript{(1795.5)}
\textsuperscript{162:6.1} El último día, el gran día de la fiesta, después de que la procesión del estanque de Siloé pasara por los patios del templo, e inmediatamente después de que los sacerdotes hubieran vertido el agua y el vino en el altar, Jesús, que se hallaba entre los peregrinos, dijo: «Si alguien tiene sed, que acuda a mí y beba. Traigo a este mundo el agua de la vida\footnote{\textit{Traigo el agua de la vida}: Jn 7:37-39.} que procede del Padre que está en lo alto. El que cree en mí se llenará con el espíritu que este agua representa, porque incluso las Escrituras han dicho: `De él manarán ríos de agua viva'\footnote{\textit{Ríos de agua viva}: Is 12:2-3; 44:3; Zac 14:8.}. Cuando el Hijo del Hombre haya terminado su obra en la Tierra, el Espíritu viviente de la Verdad será derramado sobre todo el género humano. Los que reciban este espíritu no conocerán nunca la sed espiritual».

\par 
%\textsuperscript{(1795.6)}
\textsuperscript{162:6.2} Jesús no interrumpió el servicio para pronunciar estas palabras. Se dirigió a los fieles inmediatamente después del canto del Halel, la lectura correspondiente de los salmos que era acompañada por el ondear de las ramas delante del altar. Precisamente entonces se hacía una pausa mientras se preparaban los sacrificios, y fue en ese momento cuando los peregrinos escucharon la voz fascinante del Maestro proclamar que él era el dador del agua viva para todas las almas sedientas de espíritu.

\par 
%\textsuperscript{(1796.1)}
\textsuperscript{162:6.3} Al final de este oficio matutino, Jesús continuó enseñando a la multitud, diciendo: «¿No habéis leído en las Escrituras: `Mirad, así como las aguas descienden sobre la tierra seca y cubren el suelo árido\footnote{\textit{Las aguas descienden sobre tierra seca}: Is 44:3.}, así os daré el espíritu de santidad para que descienda sobre vuestros hijos y bendiga incluso a los hijos de vuestros hijos?' ¿Por qué tenéis sed del ministerio del espíritu, cuando tratáis de regar vuestra alma con las tradiciones de los hombres, que fluyen de las jarras rotas de los oficios ceremoniales? El espectáculo que veis en este templo es la manera en que vuestros padres intentaron simbolizar la donación del espíritu divino a los hijos de la fe, y habéis hecho bien en perpetuar estos símbolos hasta el día de hoy. Pero ahora, la revelación del Padre de los espíritus\footnote{\textit{Padre de los espíritus}: Heb 12:9.} ha llegado hasta esta generación a través de la donación de su Hijo, y a todo esto le seguirá con seguridad la donación del espíritu del Padre y del Hijo a los hijos de los hombres. Para todo el que tiene fe, esta donación del espíritu se convertirá en el verdadero instructor del camino que conduce a la vida eterna, a las verdaderas aguas de la vida en el reino del cielo en la Tierra, y en el Paraíso del Padre en el más allá».

\par 
%\textsuperscript{(1796.2)}
\textsuperscript{162:6.4} Y Jesús continuó contestando a las preguntas de la multitud y de los fariseos. Algunos pensaban que era un profeta; otros creían que era el Mesías; otros decían que no podía ser el Cristo, ya que venía de Galilea, y que el Mesías debía restablecer el trono de David. Sin embargo, no se atrevieron a arrestarlo.\footnote{\textit{Opiniones de los oyentes}: Jn 7:40-44.}

\section*{7. El discurso sobre la libertad espiritual}
\par 
%\textsuperscript{(1796.3)}
\textsuperscript{162:7.1} La tarde del último día de la fiesta, después de que los apóstoles hubieran fracasado en sus esfuerzos por persuadirlo para que huyera de Jerusalén, Jesús entró de nuevo en el templo para enseñar. Al encontrar un gran grupo de creyentes reunidos en el Pórtico de Salomón\footnote{\textit{Los creyentes reunidos}: Jn 8:31a.}, les habló diciendo:

\par 
%\textsuperscript{(1796.4)}
\textsuperscript{162:7.2} «Si mis palabras permanecen en vosotros y estáis dispuestos a hacer la voluntad de mi Padre, entonces sois realmente mis discípulos. Conoceréis la verdad, y la verdad os hará libres\footnote{\textit{Conocer la verdad os hará libres}: Jn 8:31b-36.}. Sé que vais a contestarme: Somos los hijos de Abraham, y no somos esclavos de nadie; ¿cómo vamos pues a ser liberados? Pero no os hablo de una servidumbre exterior a la autoridad de otro; me refiero a las libertades del alma. En verdad, en verdad os digo que todo aquel que comete pecado es esclavo del pecado. Y sabéis que no es probable que el esclavo resida para siempre en la casa del amo. También sabéis que el hijo permanece en la casa de su padre. Por consiguiente, si el Hijo os libera, y os convierte en hijos, seréis verdaderamente libres».

\par 
%\textsuperscript{(1796.5)}
\textsuperscript{162:7.3} «Sé que sois la semilla de Abraham\footnote{\textit{La semilla de Abraham}: Jn 8:37-40.}, y sin embargo vuestros dirigentes intentan matarme porque no han permitido que mi palabra ejerza su influencia transformadora en sus corazones. Sus almas están selladas por los prejuicios y cegadas por el orgullo de la venganza. Os declaro la verdad que me muestra el Padre eterno, mientras que esos educadores engañados sólo tratan de hacer las cosas que han aprendido de sus padres temporales. Cuando contestáis que Abraham es vuestro padre, entonces os digo que, si fuerais los hijos de Abraham, ejecutaríais las obras de Abraham. Algunos de vosotros creéis en mi enseñanza, pero otros tratáis de destruirme porque os he dicho la verdad que he recibido de Dios. Pero Abraham no trató así la verdad de Dios. Percibo que algunos de vosotros estáis decididos a realizar las obras del maligno. Si Dios fuera vuestro Padre, me conoceríais y amaríais la verdad que os revelo. ¿No queréis ver que vengo del Padre, que he sido enviado por Dios\footnote{\textit{He sido enviado por Dios}: Jn 6:44; Jn 9:4.}, que no estoy haciendo esta obra por mí mismo? ¿Por qué no comprendéis mis palabras? ¿Es porque habéis elegido convertiros en los hijos del mal? Si sois los hijos de las tinieblas\footnote{\textit{Hijos de las tinieblas}: Jn 8:41-45.}, difícilmente podréis caminar en la luz de la verdad que os revelo. Los hijos del maligno sólo siguen los caminos de su padre, que era un farsante y no defendía la verdad, porque no llegó a haber ninguna verdad en él. Pero ahora viene el Hijo del Hombre, que dice y vive la verdad, y muchos de vosotros os negáis a creer».

\par 
%\textsuperscript{(1797.1)}
\textsuperscript{162:7.4} «¿Quién de vosotros me condena por pecador?\footnote{\textit{Quién me condena por pecador}: Jn 8:46-50.} Si proclamo y vivo la verdad que me muestra el Padre, ¿por qué no creéis? El que es de Dios escucha con placer las palabras de Dios; por eso, muchos de vosotros no escucháis mis palabras, porque no sois de Dios. Vuestros instructores se han atrevido incluso a decir que realizo mis obras por el poder del príncipe de los demonios. Uno que está aquí cerca acaba de decir que poseo un demonio, que soy un hijo del diablo. Pero todos aquellos de vosotros que os comportáis honradamente con vuestra propia alma sabéis muy bien que no soy un diablo. Sabéis que honro al Padre, aunque vosotros quisierais deshonrarme. No busco mi propia gloria, sino únicamente la gloria de mi Padre Paradisiaco. Y no os juzgo, porque hay alguien que juzga por mí».

\par 
%\textsuperscript{(1797.2)}
\textsuperscript{162:7.5} «En verdad, en verdad os digo a vosotros que creéis en el evangelio, que si un hombre conserva viva en su corazón esta palabra de verdad, nunca conocerá la muerte\footnote{\textit{Los creyentes nunca morirán}: Dn 12:2; Mt 19:16,29; 25:46; Mc 10:17,30; Lc 10:25; 18:18,30; Jn 3:15-16,36; 4:14,36; 5:24,39; 6:27,40,47; 6:54,68; 8:51-52; 10:28; 11:25-26; 12:25,50; 17:2-3; Hch 13:46-48; Ro 2:7; 5:21; 6:22-23; Gl 6:8; 1 Ti 1:16; 6:12,19; Tit 1:2; 3:7; 1 Jn 1:2; 2:25; 3:15; 5:11,13,20; Jud 1:21; Ap 22:5.}. Ahora, un escriba que está a mi lado dice que esta declaración es la prueba de que poseo un demonio, ya que Abraham está muerto y los profetas también. Y pregunta: `¿Eres mucho más grande que Abraham\footnote{\textit{¿Eres tú más grande que Abraham?}: Jn 8:53-56.} y los profetas como para atreverte a estar aquí y decir que el que conserva tu palabra no conocerá la muerte? ¿Quién pretendes ser para atreverte a decir tales blasfemias?' A todos los que piensan así les digo que, si me glorifico a mí mismo, mi gloria no vale nada. Pero es el Padre el que me glorificará, el mismo Padre que llamáis Dios. Pero no habéis conseguido conocer a este Dios, vuestro Dios y mi Padre, y he venido para uniros, para mostraros cómo llegar a ser de verdad los hijos de Dios. Aunque no conocéis al Padre, yo lo conozco realmente. Incluso Abraham se alegró de ver mi día, lo vio por la fe y se regocijó».

\par 
%\textsuperscript{(1797.3)}
\textsuperscript{162:7.6} Cuando los judíos incrédulos y los agentes del sanedrín que para entonces se habían congregado escucharon estas palabras, provocaron un alboroto, gritando: «No tienes cincuenta años, y sin embargo hablas de haber visto a Abraham; ¡eres un hijo del diablo!» Jesús no pudo continuar su discurso. Sólo dijo al partir: «En verdad, en verdad os lo digo, antes de que Abraham fuera, yo soy»\footnote{\textit{Antes de que existiera Abraham, yo existía}: Jn 8:57-59.}. Muchos incrédulos corrieron en busca de piedras para arrojárselas, y los agentes del sanedrín trataron de arrestarlo, pero el Maestro se alejó rápidamente por los corredores del templo, y se escapó hacia un lugar de reunión secreto, cerca de Betania, donde lo esperaban Marta, María y Lázaro.

\section*{8. La charla con Marta y María}
\par 
%\textsuperscript{(1797.4)}
\textsuperscript{162:8.1} Se había acordado que Jesús se alojaría con Lázaro\footnote{\textit{Jesús permanece con Lázaro}: Lc 10:38-39a.} y sus hermanas en la casa de un amigo, mientras que los apóstoles se dispersarían aquí y allá en pequeños grupos; habían tomado estas precauciones porque las autoridades judías se atrevían de nuevo a ejecutar sus planes de arrestar al Maestro.

\par 
%\textsuperscript{(1797.5)}
\textsuperscript{162:8.2} Durante años, los tres jóvenes habían tenido la costumbre de dejarlo todo para escuchar la enseñanza de Jesús cada vez que éste tenía la oportunidad de visitarlos. Después de perder a sus padres, Marta había asumido la responsabilidad del hogar, y por eso en esta ocasión, mientras Lázaro y María estaban sentados a los pies de Jesús bebiendo sus enseñanzas vivificantes\footnote{\textit{María escucha a Jesús}: Lc 10:39b.}, Marta se dispuso a servir la cena. Es necesario explicar que Marta se distraía innecesariamente con numerosas tareas superfluas, y que se embrollaba con muchas inquietudes insignificantes; pero era su manera de ser.

\par 
%\textsuperscript{(1798.1)}
\textsuperscript{162:8.3} Mientras Marta estaba ocupada con todos estos supuestos deberes, se sentía inquieta porque María no hacía nada por ayudarla. Por eso se acercó a Jesús y le dijo: «Maestro, ¿no te importa que mi hermana me haya dejado hacer sola todo el servicio? ¿No quisieras pedirle que venga a ayudarme?» Jesús respondió: «Marta, Marta, ¿por qué te inquietas siempre por tantas cosas, y te preocupas por tantas bagatelas? Sólo hay una cosa que vale realmente la pena, y puesto que María ha escogido esta parte buena y necesaria, no se la voy a quitar. Pero, ¿cuándo aprenderéis las dos a vivir como os he enseñado: a servir en cooperación y a refrescar vuestras almas al unísono? ¿No podéis aprender que hay un tiempo para cada cosa ---que las cuestiones secundarias de la vida deben dejar paso a las cosas más grandes del reino celestial?»\footnote{\textit{Marta sirve la mesa pero se queja}: Lc 10:40-42.}

\section*{9. En Belén con Abner}
\par 
%\textsuperscript{(1798.2)}
\textsuperscript{162:9.1} Durante toda la semana que siguió a la fiesta de los tabernáculos, decenas de creyentes se reunieron en Betania y fueron instruidos por los doce apóstoles. El sanedrín no hizo ningún esfuerzo por importunar estas reuniones, ya que Jesús no estaba presente; durante todo este período, estuvo trabajando en Belén con Abner y sus compañeros. Al día siguiente del final de la fiesta, Jesús había partido para Betania y no volvió a enseñar en el templo durante esta visita a Jerusalén.

\par 
%\textsuperscript{(1798.3)}
\textsuperscript{162:9.2} En esta época, Abner tenía su cuartel general en Belén, y desde aquel centro se habían enviado muchos discípulos a las ciudades de Judea y del sur de Samaria, e incluso a Alejandría. A los pocos días de su llegada, Jesús y Abner completaron los acuerdos para consolidar la obra de los dos grupos de apóstoles.

\par 
%\textsuperscript{(1798.4)}
\textsuperscript{162:9.3} Durante toda su visita a la fiesta de los tabernáculos, Jesús había dividido su tiempo casi por igual entre Betania y Belén. En Betania, pasó mucho tiempo con sus apóstoles; en Belén, impartió muchas enseñanzas a Abner y a los otros antiguos apóstoles de Juan. Este contacto íntimo fue lo que les llevó finalmente a creer en él. Estos antiguos apóstoles de Juan el Bautista se sintieron influidos por el coraje que Jesús había mostrado enseñando públicamente en Jerusalén, así como por la amable comprensión que experimentaron durante su enseñanza privada en Belén. Estas influencias conquistaron de manera plena y final a cada uno de los compañeros de Abner, y les llevaron a aceptar de todo corazón el reino y todo lo que implicaba un paso así.

\par 
%\textsuperscript{(1798.5)}
\textsuperscript{162:9.4} Antes de marcharse de Belén por última vez, el Maestro tomó medidas para que todos se asociaran con él en el esfuerzo unido que iba a preceder el final de su carrera terrenal en la carne. Acordaron que Abner y sus compañeros se reunirían pronto con Jesús y los doce en el Parque de Magadán.

\par 
%\textsuperscript{(1798.6)}
\textsuperscript{162:9.5} En conformidad con este acuerdo, a principios de noviembre Abner y sus once compañeros unieron su suerte a la de Jesús y los doce, y trabajaron con ellos como una sola organización hasta el día de la crucifixión.

\par 
%\textsuperscript{(1798.7)}
\textsuperscript{162:9.6} A finales de octubre, Jesús y los doce se alejaron de las proximidades inmediatas de Jerusalén. El domingo 30 de octubre, Jesús y sus asociados dejaron la ciudad de Efraín, donde el Maestro había descansado aislado durante unos días; tomaron la carretera al oeste del Jordán y se dirigieron directamente al Parque de Magadán, donde llegaron al final de la tarde del miércoles 2 de noviembre.

\par 
%\textsuperscript{(1799.1)}
\textsuperscript{162:9.7} Los apóstoles se sintieron muy aliviados por tener al Maestro de vuelta en una región amistosa; nunca más le insistieron para que fuera a Jerusalén a proclamar el evangelio del reino.