\chapter{Documento 163. La ordenación de los setenta en Magadán}
\par 
%\textsuperscript{(1800.1)}
\textsuperscript{163:0.1} POCOS días después de que Jesús y los doce regresaran de Jerusalén a Magadán, Abner y un grupo de unos cincuenta discípulos llegaron de Belén. En ese momento también se encontraban reunidos en el campamento de Magadán el cuerpo de los evangelistas, el cuerpo de mujeres y aproximadamente otros ciento cincuenta discípulos sinceros y probados de todas las regiones de Palestina. Después de consagrar unos días a los contactos personales y a la reorganización del campamento, Jesús y los doce emprendieron un curso de formación intensiva para este grupo especial de creyentes; de este conjunto de discípulos bien preparados y experimentados, el Maestro escogió posteriormente a los setenta educadores y los envió a proclamar el evangelio del reino. Esta instrucción regular empezó el viernes 4 de noviembre y continuó hasta el sábado 19 de noviembre.

\par 
%\textsuperscript{(1800.2)}
\textsuperscript{163:0.2} Jesús daba una charla, todas las mañanas, a este conjunto de personas. Pedro enseñaba los métodos de predicación pública. Natanael los instruía en el arte de enseñar; Tomás explicaba la manera de contestar a las preguntas, y Mateo dirigía la organización de sus finanzas colectivas. Los otros apóstoles participaron también en esta formación según su experiencia especial y sus talentos naturales.

\section*{1. La ordenación de los setenta}
\par 
%\textsuperscript{(1800.3)}
\textsuperscript{163:1.1} El sábado 19 de noviembre por la tarde, Jesús ordenó a los setenta en el campamento de Magadán, y Abner fue puesto al frente de estos predicadores e instructores del evangelio. Este cuerpo de setenta estaba compuesto por Abner y diez antiguos apóstoles de Juan, cincuenta y uno de los primeros evangelistas y otros ocho discípulos que se habían distinguido en el servicio del reino.

\par 
%\textsuperscript{(1800.4)}
\textsuperscript{163:1.2} Hacia las dos de la tarde de este sábado, en medio de chubascos, un grupo de creyentes, acrecentado por la llegada de David y de la mayoría de su cuerpo de mensajeros, en total más de cuatrocientas personas, se congregó en la orilla del lago de Galilea para presenciar la ordenación de los setenta.

\par 
%\textsuperscript{(1800.5)}
\textsuperscript{163:1.3} Antes de imponer sus manos sobre la cabeza de los setenta para diferenciarlos como mensajeros del evangelio, Jesús se dirigió a ellos diciendo: <<En verdad, la cosecha es abundante pero los trabajadores son pocos; por eso os exhorto a todos a que recéis para que el Señor de la cosecha envíe a más trabajadores a su cosecha. Estoy a punto de seleccionaros como mensajeros del reino; estoy a punto de enviaros hacia los judíos y los gentiles como corderos entre lobos. Cuando emprendáis vuestro camino de dos en dos, os recomiendo que no llevéis ni bolsa ni ropa adicional, porque esta primera misión será de corta duración. No saludéis a nadie por el camino, ocupaos únicamente de vuestro trabajo. Siempre que vayáis a quedaros en un hogar, empezad por decir: Que la paz sea en esta casa. Si los que viven allí aman la paz, residiréis allí; si no, entonces partiréis. Cuando hayáis escogido un hogar, quedaos en él durante toda vuestra estancia en esa ciudad, comiendo y bebiendo lo que os ofrezcan. Haréis esto porque el obrero merece su sustento. No os trasladéis de casa en casa porque os ofrezcan un alojamiento mejor. Recordad que al salir a proclamar la paz en la Tierra y la buena voluntad entre los hombres, tendréis que luchar contra unos enemigos encarnizados que se engañan a sí mismos; sed pues tan prudentes como las serpientes y tan inofensivos como las palomas>>.

\par 
%\textsuperscript{(1801.1)}
\textsuperscript{163:1.4} <<Dondequiera que vayáis, predicad diciendo: `El reino de los cielos está cerca', y ayudad a todos los que estén enfermos de la mente o del cuerpo. Habéis recibido gratuitamente las buenas cosas del reino; dad gratuitamente. Si la gente de una ciudad os recibe, encontrarán una entrada abundante en el reino del Padre; pero si la gente de una ciudad se niega a recibir este evangelio, aun así proclamaréis vuestro mensaje en el momento de marcharos de esa comunidad incrédula; a los que rechacen vuestra enseñanza, les diréis al partir: `Aunque rechazáis la verdad, sin embargo el reino de Dios se ha acercado a vosotros.' Quienquiera que os escuche, me escucha a mí. Y quienquiera que me escucha, escucha a Aquél que me ha enviado. El que rechace vuestro mensaje evangélico, me rechaza a mí. Y el que me rechaza a mí, rechaza a Aquél que me ha enviado>>.

\par 
%\textsuperscript{(1801.2)}
\textsuperscript{163:1.5} Después de que Jesús les hubiera hablado así, los setenta se arrodillaron en círculo a su alrededor, e impuso sus manos sobre la cabeza de cada uno de ellos, empezando por Abner.

\par 
%\textsuperscript{(1801.3)}
\textsuperscript{163:1.6} A primeras horas de la mañana siguiente, Abner envió a los setenta mensajeros a todas las ciudades de Galilea, Samaria y Judea. Estas treinta y cinco parejas salieron a predicar y a enseñar durante unas seis semanas, y el viernes 30 de diciembre todos regresaron al nuevo campamento cerca de Pella, en Perea.

\section*{2. El joven rico y otros casos}
\par 
%\textsuperscript{(1801.4)}
\textsuperscript{163:2.1} Más de cincuenta discípulos que deseaban la ordenación y el nombramiento como miembros de los setenta fueron rechazados por el comité que Jesús había designado para seleccionar a estos candidatos. Este comité estaba compuesto por Andrés, Abner y el jefe en activo del cuerpo evangélico. En todos los casos en que este comité de tres miembros no se ponía de acuerdo unánimemente, llevaban al candidato ante Jesús. El Maestro no rechazó a ninguna persona particular que deseara ardientemente la ordenación como mensajero del evangelio, pero después de haber hablado con Jesús, más de una docena de candidatos ya no desearon convertirse en mensajeros del evangelio.

\par 
%\textsuperscript{(1801.5)}
\textsuperscript{163:2.2} Un discípulo ferviente vino a ver a Jesús, diciendo: <<Maestro, quisiera ser uno de tus nuevos apóstoles, pero mi padre es muy anciano y está a punto de morir; ¿se me permitiría volver a mi casa para enterrarlo?>> Jesús le dijo a este hombre: <<Hijo mío, los zorros tienen guaridas y los pájaros del cielo tienen nidos, pero el Hijo del Hombre no tiene donde recostar su cabeza. Eres un discípulo fiel, y puedes continuar siéndolo mientras regresas a tu hogar para cuidar a tus seres queridos, pero no sucede así con los mensajeros de mi evangelio. Lo han abandonado todo para seguirme y proclamar el reino. Si quieres ser ordenado instructor, debes dejar que otros entierren a los muertos mientras sales a anunciar la buena nueva>>. Y este hombre se alejó, muy desilusionado.

\par 
%\textsuperscript{(1801.6)}
\textsuperscript{163:2.3} Otro discípulo vino a ver al Maestro y le dijo: <<Quisiera ser ordenado mensajero, pero me gustaría ir a mi casa durante un corto período de tiempo para confortar a mi familia>>. Jesús replicó: <<Si deseas ser ordenado, debes estar dispuesto a abandonarlo todo. Los mensajeros del evangelio no pueden tener su afecto dividido. Ningún hombre que ha puesto la mano en el arado, y se vuelve atrás, es digno de convertirse en un mensajero del reino>>.

\par 
%\textsuperscript{(1801.7)}
\textsuperscript{163:2.4} Andrés trajo entonces ante Jesús a cierto joven rico que era un fervoroso creyente y deseaba recibir la ordenación. Este joven, llamado Matadormo, era miembro del sanedrín de Jerusalén; había escuchado enseñar a Jesús y posteriormente había sido instruido en el evangelio del reino por Pedro y los otros apóstoles. Jesús habló con Matadormo sobre los requisitos de la ordenación y le pidió que demorara su decisión hasta que hubiera reflexionado más plenamente sobre el asunto. A primeras horas de la mañana siguiente, cuando Jesús salía a dar un paseo, este joven se acercó y le dijo: <<Maestro, quisiera conocer por ti las seguridades de la vida eterna. Puesto que he cumplido todos los mandamientos desde mi juventud, me gustaría saber qué más debo hacer para conseguir la vida eterna>>. En respuesta a esta pregunta, Jesús dijo: <<Si guardas todos los mandamientos ---no cometerás adulterio, no matarás, no robarás, no darás falso testimonio, no engañarás, honrarás a tus padres ---haces bien, pero la salvación es la recompensa de la fe, y no simplemente de las obras. ¿Crees en este evangelio del reino?>> Y Matadormo contestó: <<Sí, Maestro, creo todo lo que tú y tus apóstoles me habéis enseñado>>. Jesús dijo: <<Entonces, eres en verdad mi discípulo y un hijo del reino>>.

\par 
%\textsuperscript{(1802.1)}
\textsuperscript{163:2.5} El joven dijo entonces: <<Pero Maestro, no me conformo con ser tu discípulo; quisiera ser uno de tus nuevos mensajeros>>. Cuando Jesús escuchó esto, lo miró con un gran amor y dijo: <<Haré que seas uno de mis mensajeros si estás dispuesto a pagar el precio, si suples la única cosa que te falta>>. Matadormo respondió: <<Maestro, haré lo que sea para que se me permita seguirte>>. Jesús besó en la frente al joven arrodillado, y le dijo: <<Si quieres ser mi mensajero, ve a vender todo lo que posees; cuando hayas dado el producto a los pobres o a tus hermanos, ven y sígueme, y tendrás un tesoro en el reino de los cielos>>.

\par 
%\textsuperscript{(1802.2)}
\textsuperscript{163:2.6} Cuando Matadormo escuchó estas palabras, su semblante cambió. Se levantó y se alejó apenado, pues tenía grandes posesiones. Este joven fariseo rico había sido criado en la creencia de que la riqueza era el signo del favor de Dios. Jesús sabía que Matadormo no estaba liberado del amor de sí mismo y de sus riquezas. El Maestro quería liberarlo del \textit{amor} a la riqueza, no necesariamente de la riqueza. Aunque los discípulos de Jesús no se deshacían de todos sus bienes terrenales, los apóstoles y los setenta sí lo hacían. Matadormo deseaba ser uno de los setenta nuevos mensajeros, y por ese motivo Jesús le pidió que se deshiciera de todas sus posesiones temporales.

\par 
%\textsuperscript{(1802.3)}
\textsuperscript{163:2.7} Casi todo ser humano tiene alguna cosa a la que se aferra como a un mal favorito, y tiene que renunciar a ella como parte del precio de admisión en el reino de los cielos. Si Matadormo se hubiera deshecho de su riqueza, probablemente hubiera sido puesta de nuevo en sus manos para que la administrara como tesorero de los setenta. Porque más adelante, después del establecimiento de la iglesia en Jerusalén, Matadormo sí obedeció el mandato del Maestro, aunque ya era demasiado tarde para disfrutar de la asociación con los setenta, y se convirtió en el tesorero de la iglesia de Jerusalén, cuyo jefe era Santiago, el hermano carnal del Señor.

\par 
%\textsuperscript{(1802.4)}
\textsuperscript{163:2.8} Siempre ha sido así y siempre será así: Los hombres deben tomar sus propias decisiones. Los mortales pueden hacer uso de cierta gama de posibilidades dentro de la libertad de elección. Las fuerzas del mundo espiritual no desean coaccionar al hombre; le permiten seguir el camino que él mismo ha elegido.

\par 
%\textsuperscript{(1802.5)}
\textsuperscript{163:2.9} Jesús preveía que Matadormo, con sus riquezas, no podría ser de ninguna manera ordenado como compañero de unos hombres que lo habían abandonado todo por el evangelio; al mismo tiempo veía que, sin sus riquezas, se convertiría en el máximo dirigente de todos ellos. Pero, al igual que los mismos hermanos de Jesús, Matadormo nunca llegó a ser grande en el reino porque él mismo se privó de esa asociación íntima y personal con el Maestro que podría haber experimentado si hubiera estado dispuesto a hacer en ese momento lo que Jesús le pedía, cosa que hizo en efecto, pero varios años después.

\par 
%\textsuperscript{(1803.1)}
\textsuperscript{163:2.10} Las riquezas no tienen ninguna relación directa con la entrada en el reino de los cielos, pero el \textit{amor a la riqueza sí tiene que ver}. Las lealtades espirituales hacia el reino son incompatibles con la servidumbre a la codicia materialista. El hombre no puede compartir su lealtad suprema a un ideal espiritual con una devoción material.

\par 
%\textsuperscript{(1803.2)}
\textsuperscript{163:2.11} Jesús no enseñó nunca que fuera malo poseer riquezas. Sólo a los doce y a los setenta les pidió que dedicaran todas sus posesiones terrenales a la causa común. Incluso entonces, se encargó de que sus bienes se liquidaran de una manera ventajosa, como en el caso del apóstol Mateo. Jesús aconsejó muchas veces a sus discípulos acaudalados lo que le había enseñado al hombre rico de Roma. El Maestro consideraba que la inversión sabia de las ganancias sobrantes era una forma legítima de asegurarse contra la inevitable adversidad futura. Cuando la tesorería apostólica estaba desbordante, Judas ponía fondos en depósito para utilizarlos posteriormente en el caso de que los apóstoles sufrieran una gran disminución de los ingresos. Judas hacía esto después de consultarlo con Andrés. Jesús no se ocupó nunca personalmente de las finanzas apostólicas, excepto de los desembolsos destinados a las limosnas. Pero había un abuso económico que condenó muchas veces, y fue la explotación injusta de los hombres débiles, ignorantes y menos afortunados por parte de sus semejantes fuertes, agudos y más inteligentes. Jesús declaró que este tratamiento inhumano de hombres, mujeres y niños era incompatible con los ideales de la fraternidad del reino de los cielos.

\section*{3. La discusión sobre la riqueza}
\par 
%\textsuperscript{(1803.3)}
\textsuperscript{163:3.1} Mientras Jesús terminaba de hablar con Matadormo, Pedro y algunos apóstoles se habían reunido a su alrededor, y cuando el joven rico se hubo marchado, Jesús se volvió hacia los apóstoles y dijo: <<¡Ya veis lo difícil que es para los que tienen riquezas entrar plenamente en el reino de Dios! La adoración espiritual no se puede compartir con las devociones materiales; ningún hombre puede servir a dos señores. Tenéis un dicho que dice que `es más fácil que un camello pase por el ojo de una aguja, a que los paganos hereden la vida eterna.' Y yo declaro que es igual de fácil que ese camello pase por el ojo de la aguja, a que estos ricos satisfechos de sí mismos entren en el reino de los cielos>>.

\par 
%\textsuperscript{(1803.4)}
\textsuperscript{163:3.2} Cuando Pedro y los apóstoles escucharon estas palabras, se quedaron extremadamente sorprendidos, de tal manera que Pedro dijo: <<¿Entonces, Señor, quién puede salvarse? ¿Todos los que tienen riquezas se quedarán fuera del reino?>> Jesús respondió: <<No, Pedro, pero todos los que ponen su confianza en las riquezas, difícilmente entrarán en la vida espiritual que conduce al progreso eterno. Pero aún así, muchas cosas que son imposibles para el hombre, no están fuera del alcance del Padre que está en los cielos; deberíamos reconocer más bien que con Dios todas las cosas son posibles>>.

\par 
%\textsuperscript{(1803.5)}
\textsuperscript{163:3.3} Mientras cada uno se iba por su lado, a Jesús le entristeció que Matadormo no se quedara con ellos, porque lo amaba profundamente. Cuando bajaron al lago, se sentaron al lado del agua y Pedro, hablando en nombre de los doce (que estaban todos presentes en aquel momento), dijo: <<Estamos confundidos por tus palabras al joven rico. ¿Tenemos que exigir a los que quieran seguirte que renuncien a todas sus riquezas terrenales?>> Jesús dijo: <<No, Pedro, sólo a los que quieran convertirse en apóstoles, y deseen vivir conmigo como vosotros lo hacéis, y como una sola familia. Pero el Padre exige que el afecto de sus hijos sea puro e indiviso. Cualquier cosa o persona que se interponga entre vosotros y el amor a las verdades del reino, debe ser abandonada. Si la riqueza que uno posee no invade los recintos del alma, no tiene ninguna consecuencia sobre la vida espiritual de los que desean entrar en el reino>>.

\par 
%\textsuperscript{(1804.1)}
\textsuperscript{163:3.4} Pedro dijo entonces: <<Pero, Maestro, nosotros lo hemos abandonado todo para seguirte; ¿qué poseeremos entonces?>> Jesús se dirigió a la totalidad de los doce, diciendo: <<En verdad, en verdad os digo que no hay nadie que haya abandonado su riqueza, su hogar, a su esposa, a sus hermanos, a sus padres o a sus hijos, por amor por mí y por el reino de los cielos, que no reciba mucho más en este mundo ---quizás con algunas persecuciones--- y la vida eterna en el mundo venidero. Muchos que son los primeros serán los últimos, mientras que los últimos serán a menudo los primeros. El Padre trata a sus criaturas según sus necesidades y de acuerdo con sus justas leyes de consideración misericordiosa y amorosa por el bienestar de un universo>>.

\par 
%\textsuperscript{(1804.2)}
\textsuperscript{163:3.5} <<El reino de los cielos se parece a un propietario que empleaba a muchos hombres, y que salió por la mañana temprano a contratar a unos obreros para que trabajaran en su viña. Después de acordar con los trabajadores que les pagaría un denario por día, los envió a su viña. Luego salió a eso de las nueve, y al ver a otros parados en la plaza del mercado, les dijo: `Id también a trabajar en mi viña, y os pagaré lo que sea justo.' Y fueron inmediatamente a trabajar. El propietario salió de nuevo a eso de las doce y hacia las tres, e hizo lo mismo. Fue a la plaza del mercado alrededor de las cinco de la tarde, encontró a otros obreros parados, y les preguntó: `¿Por qué estáis aquí todo el día sin hacer nada?' Los hombres contestaron: `Porque nadie nos ha contratado.' El propietario dijo entonces: `Id vosotros también a trabajar en mi viña, y os pagaré lo que sea justo.'>>

\par 
%\textsuperscript{(1804.3)}
\textsuperscript{163:3.6} <<Cuando llegó la noche, el propietario de la viña dijo a su administrador: `Llama a los obreros y págales su salario, empezando por los últimos contratados y terminando por los primeros.' Cuando llegaron los que habían sido contratados a eso de las cinco, cada uno recibió un denario, y todos los demás trabajadores recibieron el mismo salario. Cuando los hombres que habían sido contratados al principio del día vieron lo que habían cobrado los últimos en llegar, esperaron recibir más de la cantidad acordada. Pero al igual que los demás, cada hombre sólo recibió un denario. Cuando todos hubieron recibido su paga, se quejaron al propietario, diciendo: `Los últimos hombres que contrataste sólo han trabajado una hora, y sin embargo les has pagado lo mismo que a nosotros, que hemos aguantado todo el día bajo el Sol abrasador.'>>

\par 
%\textsuperscript{(1804.4)}
\textsuperscript{163:3.7} <<El propietario contestó entonces: `Amigos míos, no soy injusto con vosotros. ¿No aceptasteis trabajar por un denario al día? Tomad ahora lo que es vuestro y seguid vuestro camino, porque es mi deseo dar a los últimos que llegaron lo mismo que os he dado a vosotros. ¿No me es lícito hacer lo que desee con lo que es mío? ¿O acaso os molesta mi generosidad, porque deseo ser bondadoso y mostrar misericordia?'>>

\section*{4. La despedida de los setenta}
\par 
%\textsuperscript{(1804.5)}
\textsuperscript{163:4.1} El día que los setenta salieron para efectuar su primera misión fue un momento emocionante en el campamento de Magadán. Aquella mañana temprano, en su última conversación con los setenta, Jesús hizo hincapié en los puntos siguientes:

\par 
%\textsuperscript{(1804.6)}
\textsuperscript{163:4.2} 1. El evangelio del reino debe ser proclamado en el mundo entero, tanto a los gentiles como a los judíos.

\par 
%\textsuperscript{(1804.7)}
\textsuperscript{163:4.3} 2. Cuando cuidéis a los enfermos, absteneos de enseñarles a esperar milagros.

\par 
%\textsuperscript{(1805.1)}
\textsuperscript{163:4.4} 3. Proclamad una fraternidad espiritual de los hijos de Dios, y no un reino exterior de poder mundano y de gloria material.

\par 
%\textsuperscript{(1805.2)}
\textsuperscript{163:4.5} 4. Evitad perder el tiempo mediante un exceso de visitas sociales y otras trivialidades, que podrían disminuir vuestra consagración entusiasta a la predicación del evangelio.

\par 
%\textsuperscript{(1805.3)}
\textsuperscript{163:4.6} 5. Si la primera casa que hayáis elegido como cuartel general resulta ser un hogar digno, permaneced allí durante toda vuestra estancia en esa ciudad.

\par 
%\textsuperscript{(1805.4)}
\textsuperscript{163:4.7} 6. Indicad claramente a todos los creyentes fieles que ha llegado la hora de romper abiertamente con los jefes religiosos de los judíos de Jerusalén.

\par 
%\textsuperscript{(1805.5)}
\textsuperscript{163:4.8} 7. Enseñad que todo el deber del hombre se encuentra resumido en este mandamiento único: Ama al Señor tu Dios con toda tu mente y con toda tu alma, y a tu prójimo como a ti mismo. (Debían enseñar que esto representaba todo el deber del hombre, en lugar de las 613 reglas de vida expuestas por los fariseos.)

\par 
%\textsuperscript{(1805.6)}
\textsuperscript{163:4.9} Después de que Jesús hubiera hablado así a los setenta en presencia de todos los apóstoles y discípulos, Simón Pedro se los llevó aparte y les predicó su sermón de ordenación; fue una ampliación de las instrucciones que les había dado el Maestro en el momento de imponerles las manos y de seleccionarlos como mensajeros del reino. Pedro exhortó a los setenta a que fomentaran, en su experiencia, las virtudes siguientes:

\par 
%\textsuperscript{(1805.7)}
\textsuperscript{163:4.10} 1. \textit{La devoción consagrada}. Orar siempre para que más obreros sean enviados a la cosecha del evangelio. Explicó que, cuando uno ora así, dirá más probablemente: <<Aquí estoy; envíame>>. Les exhortó a que no olvidaran su culto diario.

\par 
%\textsuperscript{(1805.8)}
\textsuperscript{163:4.11} 2. \textit{El coraje verdadero}. Les advirtió que se encontrarían con hostilidades y que estuvieran seguros de que sufrirían persecuciones. Pedro les dijo que su misión no era una empresa para cobardes, y aconsejó a los que tuvieran miedo que se retiraran antes de partir. Pero ninguno desistió.

\par 
%\textsuperscript{(1805.9)}
\textsuperscript{163:4.12} 3. \textit{La fe y la confianza}. Para esta corta misión, debían salir sin recursos ningunos; debían confiar en el Padre para la comida, el alojamiento y todas las demás necesidades.

\par 
%\textsuperscript{(1805.10)}
\textsuperscript{163:4.13} 4. \textit{El ardor y la iniciativa}. Debían estar dominados por un ardor y un entusiasmo inteligente; debían ocuparse estrictamente de los asuntos de su Maestro. El saludo oriental era una ceremonia bastante larga y elaborada; por eso Jesús les había indicado que <<no saludaran a nadie por el camino>>. Se trataba de una expresión corriente para exhortar a alguien a ocuparse de sus asuntos sin perder el tiempo. No tenía nada que ver con la cuestión del saludo amistoso.

\par 
%\textsuperscript{(1805.11)}
\textsuperscript{163:4.14} 5. \textit{La amabilidad y la cortesía}. El Maestro les había ordenado que evitaran perder el tiempo de manera innecesaria en ceremonias sociales, pero les recomendó la cortesía hacia todos aquellos con quienes se pusieran en contacto. Debían mostrar una gran amabilidad con los que los hospedaran en su hogar. Fueron estrictamente advertidos contra el hecho de dejar un hogar modesto para hospedarse en uno más cómodo o más influyente.

\par 
%\textsuperscript{(1805.12)}
\textsuperscript{163:4.15} 6. \textit{La asistencia a los enfermos}. Pedro encargó a los setenta que buscaran a los que estaban enfermos de la mente y del cuerpo, y que hicieran todo lo que pudieran por aliviar o curar sus enfermedades.

\par 
%\textsuperscript{(1805.13)}
\textsuperscript{163:4.16} Una vez que hubieron recibido sus órdenes y sus instrucciones, partieron de dos en dos para realizar su misión en Galilea, Samaria y Judea.

\par 
%\textsuperscript{(1806.1)}
\textsuperscript{163:4.17} Aunque los judíos tenían una estima particular por el número setenta, considerando a veces que las naciones paganas sumaban un total de setenta, y aunque estos setenta mensajeros debían llevar el evangelio a todos los pueblos, sin embargo, por lo que podemos discernir, el hecho de que este grupo comportara exactamente setenta miembros era una simple coincidencia. Es indudable de que Jesús hubiera aceptado a media docena más, pero no estaban dispuestos a pagar el precio de separarse de sus riquezas y de sus familias.

\section*{5. El traslado del campamento a Pella}
\par 
%\textsuperscript{(1806.2)}
\textsuperscript{163:5.1} Jesús y los doce se prepararon ahora para establecer su último cuartel general en Perea, cerca de Pella, donde el Maestro había sido bautizado en el Jordán. Los últimos diez días de noviembre los pasaron deliberando en Magadán, y el martes 6 de diciembre, el grupo entero compuesto por casi trescientas personas partió al amanecer con todos sus efectos para alojarse aquella noche cerca de Pella, al lado del río. Se trataba del mismo lugar, cerca del manantial, que Juan el Bautista había ocupado con su campamento varios años antes.

\par 
%\textsuperscript{(1806.3)}
\textsuperscript{163:5.2} Después de levantarse el campamento de Magadán, David Zebedeo regresó a Betsaida y empezó inmediatamente a reducir el servicio de mensajeros. El reino estaba entrando en una nueva fase. Los peregrinos llegaban diariamente de todas las partes de Palestina e incluso de las regiones remotas del imperio romano. A veces venían creyentes de Mesopotamia y de los territorios situados al este del Tigris. En consecuencia, el domingo 18 de diciembre, con la ayuda de su cuerpo de mensajeros, David cargó en los animales de carga el equipo del campamento, que estaba entonces almacenado en la casa de su padre; era el material con el que había organizado anteriormente el campamento de Betsaida, al lado del lago. Se despidió de Betsaida por un tiempo y descendió por la orilla del lago, y a lo largo del Jordán, hasta un punto situado aproximadamente a un kilómetro al norte del campamento apostólico; en menos de una semana estaba preparado para ofrecer su hospitalidad a cerca de mil quinientos peregrinos visitantes. El campamento apostólico podía alojar a unas quinientas personas. Como era la estación de las lluvias en Palestina, se necesitaban estos alojamientos para cuidar al creciente número de interesados, en su mayoría serios, que venían hasta Perea para ver a Jesús y escuchar su enseñanza.

\par 
%\textsuperscript{(1806.4)}
\textsuperscript{163:5.3} David hizo todo esto por su propia iniciativa, aunque había consultado con Felipe y Mateo en Magadán. La mayor parte de su antiguo cuerpo de mensajeros los empleó como asistentes para dirigir este campamento; ahora utilizaba menos de veinte hombres en el servicio regular de mensajeros. A finales de diciembre y antes de que regresaran los setenta, cerca de ochocientos visitantes estaban congregados alrededor del Maestro, y encontraron alojamiento en el campamento de David.

\section*{6. El regreso de los setenta}
\par 
%\textsuperscript{(1806.5)}
\textsuperscript{163:6.1} El viernes 30 de diciembre, mientras Jesús estaba ausente en las colinas cercanas con Pedro, Santiago y Juan, los setenta mensajeros fueron llegando de dos en dos al cuartel general de Pella, acompañados por numerosos creyentes. Hacia las cinco de la tarde, cuando Jesús regresó al campamento, los setenta estaban reunidos en el lugar dedicado a la enseñanza. La cena se retrasó más de una hora, mientras estos entusiastas del evangelio del reino contaban sus experiencias. Los mensajeros de David habían traído a los apóstoles muchas de estas noticias durante las semanas anteriores, pero fue realmente inspirador escuchar a estos instructores del evangelio, ordenados recientemente, contar en persona cómo los judíos y los gentiles ávidos habían recibido su mensaje. Por fin Jesús podía ver a unos hombres que salían a difundir la buena nueva fuera de su presencia personal. El Maestro sabía ahora que podía dejar este mundo sin obstaculizar seriamente el progreso del reino.

\par 
%\textsuperscript{(1807.1)}
\textsuperscript{163:6.2} Cuando los setenta contaron que <<hasta los demonios se sometían>> a ellos, se referían a las curas maravillosas que habían realizado en los casos de víctimas con trastornos nerviosos. Sin embargo, estos ministros habían aliviado algunos casos de verdadera posesión por los espíritus, y refiriéndose a ellos, Jesús dijo: <<No es de extrañar que esos espíritus menores desobedientes se sometan a vosotros, puesto que he visto a Satanás caer del cielo como un rayo. Pero no os regocijéis tanto por eso, porque os declaro que, en cuanto regrese al lado de mi Padre, enviaremos nuestros espíritus al interior de la mente misma de los hombres para que esos pocos espíritus perdidos ya no puedan penetrar en la mente de los mortales desafortunados. Me regocijo con vosotros de que tengáis influencia sobre los hombres, pero no os sintáis ensalzados por esta experiencia, sino regocijaos más bien porque vuestros nombres están inscritos en los archivos del cielo, y porque vais a avanzar así en una carrera sin fin de conquista espiritual>>.

\par 
%\textsuperscript{(1807.2)}
\textsuperscript{163:6.3} Fue en ese instante, poco antes de compartir la cena, cuando Jesús experimentó uno de esos raros momentos de éxtasis emocional que sus seguidores tuvieran ocasión de presenciar. Dijo: <<Te doy las gracias, Padre mío, Señor del cielo y de la Tierra, porque el espíritu ha revelado estas glorias espirituales a estos hijos del reino, mientras que este evangelio maravilloso era ocultado a los sabios y a los presuntuosos. Sí, Padre mío, debe haber sido agradable a tus ojos hacer esto, y me regocijo al saber que la buena nueva se difundirá por el mundo entero después de que yo haya vuelto a ti y al trabajo que me has encomendado. Estoy extremadamente emocionado cuando me doy cuenta de que estás a punto de poner en mis manos toda la autoridad, que sólo tú sabes realmente quién soy, y que sólo yo te conozco realmente, así como aquellos a quienes te he revelado. Cuando haya finalizado esta revelación a mis hermanos en la carne, la continuaré con tus criaturas del cielo>>.

\par 
%\textsuperscript{(1807.3)}
\textsuperscript{163:6.4} Después de haberle hablado así al Padre, Jesús se volvió para decirle a sus apóstoles y ministros: <<Benditos sean los ojos que ven y los oídos que oyen estas cosas. Dejadme deciros que muchos profetas y muchos grandes hombres de las épocas pasadas desearon contemplar lo que veis ahora, pero no les fue concedido. Y muchas generaciones venideras de hijos de la luz, cuando oigan estas cosas, os envidiarán porque vosotros las habéis visto y oído>>.

\par 
%\textsuperscript{(1807.4)}
\textsuperscript{163:6.5} Luego se dirigió a todos los discípulos, y dijo: <<Habéis oído cuántas ciudades y pueblos han recibido la buena nueva del reino, y cómo han sido recibidos mis ministros e instructores tanto por los judíos como por los gentiles. Benditas son en verdad esas comunidades que han elegido creer en el evangelio del reino. Pero, ¡ay de los habitantes que rechazan la luz en Corazín, Betsaida-Julias y Cafarnaúm, esas ciudades que no han recibido bien a estos mensajeros! Declaro que si las obras poderosas que se han hecho en esos lugares hubieran sido hechas en Tiro y en Sidón, los habitantes de esas ciudades llamadas paganas se habrían arrepentido hace mucho tiempo dándose golpes de pecho. En el día del juicio, el destino de Tiro y de Sidón será por cierto más llevadero>>.

\par 
%\textsuperscript{(1807.5)}
\textsuperscript{163:6.6} Como el día siguiente era sábado, Jesús se reunió aparte con los setenta y les dijo: <<En verdad, me he regocijado con vosotros cuando habéis regresado con las buenas noticias de que el evangelio del reino ha sido acogido por tanta gente diseminada por toda Galilea, Samaria y Judea. Pero, ¿por qué os sentíais tan sorprendentemente exaltados? ¿No esperabais que la comunicación de vuestro mensaje se manifestaría con poder? ¿Salisteis con tan poca fe en este evangelio como para regresar sorprendidos de su eficacia? Y ahora, aunque no quisiera apagar vuestro entusiasmo, deseo advertiros severamente contra las sutilezas del orgullo, del orgullo espiritual. Si pudierais comprender la caída de Lucifer, el inicuo, evitaríais solemnemente todas las formas de orgullo espiritual>>.

\par 
%\textsuperscript{(1808.1)}
\textsuperscript{163:6.7} <<Habéis emprendido la gran tarea de enseñar al hombre mortal que es un hijo de Dios. Os he mostrado el camino; salid a realizar vuestro deber y no os canséis de hacer el bien. A vosotros y a todos los que sigan vuestros pasos a lo largo de los siglos, dejad que os diga que siempre estoy cerca, y que mi convocatoria es, y será siempre: Venid a mí, todos los que os afanáis y lleváis una carga pesada, que yo os daré el descanso. Haced vuestro mi yugo y aprended de mí, pues soy sincero y leal, y encontraréis el descanso espiritual para vuestra alma>>.

\par 
%\textsuperscript{(1808.2)}
\textsuperscript{163:6.8} Cuando pusieron a prueba las promesas del Maestro, comprobaron que sus palabras eran ciertas. Y desde aquel día, un número incalculable de personas también han probado y comprobado la certeza de estas mismas promesas.

\section*{7. Los preparativos para la última misión}
\par 
%\textsuperscript{(1808.3)}
\textsuperscript{163:7.1} Los días siguientes estuvieron llenos de actividad en el campamento de Pella; los preparativos para la misión en Perea se estaban ultimando. Jesús y sus asociados estaban a punto de emprender su última misión, la gira de tres meses por toda Perea, que sólo llegó a su fin cuando el Maestro entró en Jerusalén para llevar a cabo sus últimos trabajos en la Tierra. Durante todo este período, el cuartel general de Jesús y los doce apóstoles se mantuvo aquí, en el campamento de Pella.

\par 
%\textsuperscript{(1808.4)}
\textsuperscript{163:7.2} Jesús ya no tenía necesidad de salir para enseñar a la gente. Ahora acudían a él en cantidades que aumentaban cada semana y procedentes de todas partes, no solamente de Palestina, sino de todo el mundo romano y del próximo oriente. Aunque el Maestro participó con los setenta en la gira por Perea, pasó una gran parte de su tiempo en el campamento de Pella, enseñando a la multitud e instruyendo a los doce. Durante todo este período de tres meses, al menos diez apóstoles permanecieron con Jesús.

\par 
%\textsuperscript{(1808.5)}
\textsuperscript{163:7.3} El cuerpo de mujeres también se preparó para salir de dos en dos, acompañando a los setenta, para trabajar en las ciudades más importantes de Perea. Este grupo original de doce mujeres había entrenado recientemente a un cuerpo más numeroso de cincuenta mujeres en la tarea de visitar los hogares y en el arte de cuidar a los enfermos y a los afligidos. Perpetua, la esposa de Simón Pedro, se unió a esta nueva división del cuerpo de mujeres y le confiaron la dirección de este trabajo femenino más amplio, bajo las órdenes de Abner. Después de Pentecostés, permaneció con su ilustre marido y lo acompañó en todas sus giras misioneras; el día que Pedro fue crucificado en Roma, ella sirvió de alimento a las bestias feroces en la arena. Este nuevo cuerpo de mujeres también contaba entre sus miembros a las esposas de Felipe y Mateo, y a la madre de Santiago y Juan.

\par 
%\textsuperscript{(1808.6)}
\textsuperscript{163:7.4} El trabajo del reino se preparaba ahora para entrar en su fase final bajo la dirección personal de Jesús. Esta fase estaba caracterizada por la profundidad espiritual, en contraste con aquella en que las multitudes, propensas a los milagros y buscadoras de prodigios, seguían al Maestro durante los primeros días de su popularidad en Galilea. Sin embargo, aún había cierto número de seguidores suyos que tenían tendencias materialistas, y que no lograban captar la verdad de que el reino de los cielos es la fraternidad espiritual de los hombres, basada en el hecho eterno de la paternidad universal de Dios.