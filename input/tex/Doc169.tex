\chapter{Documento 169. La última enseñanza en Pella}
\par 
%\textsuperscript{(1850.1)}
\textsuperscript{169:0.1} JESÚS y los diez apóstoles llegaron al campamento de Pella el lunes 6 de marzo al caer la tarde. Ésta fue la última semana que Jesús pasó allí, y estuvo muy activo enseñando a la muchedumbre e instruyendo a los apóstoles. Todas las tardes predicaba a las multitudes, y todas las noches respondía a las preguntas de los apóstoles y de algunos de los discípulos más avanzados que residían en el campamento.

\par 
%\textsuperscript{(1850.2)}
\textsuperscript{169:0.2} La noticia de la resurrección de Lázaro había llegado al campamento dos días antes de la llegada del Maestro, y toda la asamblea estaba llena de curiosidad. Desde el episodio de la alimentación de los cinco mil, no había sucedido nada que excitara tanto la imaginación de la gente. Así es como en la cumbre misma de la segunda fase del ministerio público del reino, Jesús planeó enseñar durante esta sola y corta semana en Pella, para luego empezar la gira por el sur de Perea, que conduciría directamente a las experiencias finales y trágicas de la última semana en Jerusalén.

\par 
%\textsuperscript{(1850.3)}
\textsuperscript{169:0.3} Los fariseos y los sacerdotes principales habían empezado a formular sus cargos y a cristalizar sus acusaciones. Se oponían a las enseñanzas del Maestro por los motivos siguientes:

\par 
%\textsuperscript{(1850.4)}
\textsuperscript{169:0.4} 1. Es amigo de los publicanos y de los pecadores; recibe a los impíos e incluso come con ellos\footnote{\textit{Come con publicanos y pecadores}: Mt 9:10-11; 11:19a; Mc 2:15-16; Lc 5:30; 7:34; 15:1-2.}.

\par 
%\textsuperscript{(1850.5)}
\textsuperscript{169:0.5} 2. Es un blasfemo\footnote{\textit{Blasfemo: dice ser igual a Dios}: Jn 1:1; 5:17-18; 10:30,38; 12:44-45; 14:7-11,20; 17:11,21-22.}; habla de Dios como si fuera su Padre y piensa que es igual a Dios.

\par 
%\textsuperscript{(1850.6)}
\textsuperscript{169:0.6} 3. Es un infractor de la ley\footnote{\textit{Recoge grano en sábado}: Mt 12:1-2; Mc 2:23-24; Lc 6:1-2.}. Cura las enfermedades durante el sábado\footnote{\textit{Cura en sábado}: Mt 12:9-14; Mc 3:1-5; Lc 6:6-11; 13:10-16; 14:1-4; Jn 5:5-16; 9:1-7,16.} y se burla de otras muchas maneras de la ley sagrada de Israel\footnote{\textit{Los apóstoles no se lavaban las manos}: Mt 15:1-2; Mc 7:1-6; Lc 11:37-41.}.

\par 
%\textsuperscript{(1850.7)}
\textsuperscript{169:0.7} 4. Está aliado con los demonios\footnote{\textit{Aliado con los demonios}: Mt 12:22-28; Mc 3:22; Lc 11:14-20.}. Realiza prodigios y hace milagros aparentes por el poder de Belcebú, el príncipe de los demonios.

\section*{1. La parábola del hijo perdido}
\par 
%\textsuperscript{(1850.8)}
\textsuperscript{169:1.1} El jueves por la tarde, Jesús habló a la multitud sobre la «Gracia de la salvación». En el transcurso de este sermón, volvió a contar la historia de la oveja perdida y de la moneda perdida, y luego añadió su parábola favorita del hijo pródigo. Jesús dijo:

\par 
%\textsuperscript{(1850.9)}
\textsuperscript{169:1.2} «Desde Samuel hasta Juan, los profetas os han exhortado a buscar a Dios\footnote{\textit{Buscar al Señor}: Is 55:6.} ---a buscar la verdad. Siempre os han dicho: `Buscad al Señor mientras se le puede encontrar.' Toda esta enseñanza debería tomarse en serio. Pero yo he venido a mostraros que, mientras tratáis de encontrar a Dios, Dios también trata de encontraros a vosotros. Os he contado muchas veces la historia del buen pastor que dejó a las noventa y nueve ovejas en el redil para salir a buscar a la que se había perdido, y cuando encontró a la oveja descarriada, cómo se la echó al hombro y la devolvió tiernamente al redil. Y cuando la oveja perdida estuvo de nuevo en el redil, recordaréis que el buen pastor llamó a sus amigos y los invitó a que se regocijaran con él porque había encontrado a la oveja que se había extraviado. Os digo de nuevo que hay más alegría en el cielo por un pecador que se arrepiente que por noventa y nueve justos que no necesitan arrepentimiento. El hecho de que unas almas estén \textit{perdidas} no hace más que acrecentar el interés del Padre celestial. He venido a este mundo para ejecutar el mandato de mi Padre, y se ha dicho con razón del Hijo del Hombre que es amigo de los publicanos y de los pecadores»\footnote{\textit{Parábola de la oveja perdida}: Mt 18:12-14; Lc 15:3-7.}.

\par 
%\textsuperscript{(1851.1)}
\textsuperscript{169:1.3} «Os han enseñado que la aceptación divina se produce después de que os hayáis arrepentido y como consecuencia de todas vuestras obras de sacrificio y de penitencia, pero os aseguro que el Padre os acepta incluso antes de que os hayáis arrepentido, y envía al Hijo y a sus asociados para encontraros y devolveros con regocijo al redil, al reino de la filiación y del progreso espiritual. Todos sois como ovejas extraviadas, y yo he venido para buscar y salvar a los que están perdidos»\footnote{\textit{Jesús ha venido a buscar y salvar}: Mt 18:11; Lc 19:10.}.

\par 
%\textsuperscript{(1851.2)}
\textsuperscript{169:1.4} «También deberíais recordar la historia de la mujer que, después de haber hecho un collar de adorno con diez monedas de plata, perdió una de las monedas; entonces encendió la lámpara, barrió cuidadosamente la casa y continuó buscando hasta que encontró la moneda de plata perdida. En cuanto encontró la moneda que había perdido, convocó a sus amigos y vecinos, diciendo: `Regocijaos conmigo porque he encontrado la moneda que se había perdido.'\footnote{\textit{La moneda perdida}: Lc 15:8-9.} Así pues, os digo de nuevo que siempre hay alegría entre los ángeles del cielo por un pecador que se arrepiente y vuelve al redil del Padre\footnote{\textit{Los ángeles se regocijan con un salvado}: Lc 15:7,10.}. Os cuento esta historia para convenceros de que el Padre y su Hijo salen a \textit{buscar} a aquellos que están perdidos, y en esta búsqueda empleamos todas las influencias que puedan ayudarnos en nuestros esfuerzos diligentes por encontrar a los que se han perdido, a los que necesitan ser salvados. Y así, el Hijo del Hombre sale al desierto para buscar a la oveja extraviada, pero también busca la moneda que se ha perdido en la casa. La oveja se extravía de manera involuntaria; la moneda está cubierta por el polvo del tiempo y oscurecida por la acumulación de las cosas humanas».

\par 
%\textsuperscript{(1851.3)}
\textsuperscript{169:1.5} «Ahora me gustaría contaros la historia del hijo atolondrado de un granjero acaudalado, que dejó \textit{deliberadamente} la casa de su padre y se fue a un país extranjero, donde sufrió muchas tribulaciones. Recordáis que la oveja se descarrió sin intención, pero este joven abandonó su hogar con premeditación. Esto fue lo que ocurrió:»

\par 
%\textsuperscript{(1851.4)}
\textsuperscript{169:1.6} «Había un hombre que tenía dos hijos; el más joven era alegre y despreocupado, y trataba siempre de pasarselo bien y de eludir las responsabilidades, mientras que su hermano mayor era serio, sobrio, trabajador y dispuesto a asumir las responsabilidades. Pero estos dos hermanos no se llevaban bien; discutían y reñían constantemente. El más joven era alegre y vivaz pero holgazán, y no se podía confiar en él; el hijo mayor era formal y trabajador, pero al mismo tiempo egocéntrico, hosco y engreído. El hijo más joven disfrutaba con el juego pero rehuía el trabajo; el mayor se consagraba al trabajo pero jugaba pocas veces. Esta asociación se volvió tan desagradable, que el hijo menor fue a ver a su padre y le dijo: `Padre, entrégame la tercera parte de los bienes que yo heredaría, y permíteme salir al mundo para buscar mi propia fortuna.' El padre sabía lo infeliz que era el joven en el hogar con su hermano mayor, y cuando escuchó esta petición, dividió sus bienes y le entregó al joven su parte»\footnote{\textit{El padre que dividió la herencia entre los hijos}: Lc 15:11-12.}.

\par 
%\textsuperscript{(1851.5)}
\textsuperscript{169:1.7} «El joven reunió todos sus fondos en pocas semanas y salió de viaje hacia un país lejano; como no encontró nada provechoso que hacer que fuera también agradable, pronto derrochó toda su herencia viviendo de manera desenfrenada. Cuando lo hubo gastado todo, una hambruna prolongada surgió en aquel país, y el joven se encontró en la miseria. Y así, cuando empezó a pasar hambre y a sufrir una gran angustia, encontró un empleo con uno de los ciudadanos de aquel país, que lo envió a los campos a dar de comer a los cerdos. El joven se hubiera saciado de buena gana con los desperdicios que comían los cerdos, pero nadie quería darle nada»\footnote{\textit{El hijo malgastó todo en una vida disipada}: Lc 15:13-16.}.

\par 
%\textsuperscript{(1852.1)}
\textsuperscript{169:1.8} «Un día que tenía mucha hambre, se le ocurrió decir: `¡Cuántos criados a sueldo de mi padre tienen pan de sobra mientras yo me muero de hambre, alimentando cerdos aquí en un país extranjero! Me levantaré, iré a ver a mi padre y le diré: Padre, he pecado contra el cielo y contra ti. Ya no soy digno de ser llamado hijo tuyo; permíteme ser solamente como uno de tus criados a sueldo.' Y cuando el joven llegó a esta decisión, se levantó y partió hacia la casa de su padre»\footnote{\textit{El hijo decide regresar}: Lc 15:17-20a.}.

\par 
%\textsuperscript{(1852.2)}
\textsuperscript{169:1.9} «Pero aquel padre había llorado mucho por su hijo; había echado de menos al alegre pero irreflexivo muchacho. Este padre amaba a este hijo y vigilaba constantemente su regreso, de manera que el día que el hijo se acercó a la casa, aunque aún estaba muy lejos, el padre lo vio; impulsado por una compasión amorosa, corrió a su encuentro y, saludándolo afectuosamente, lo abrazó y lo besó. Después de haberse reunido así, el hijo levantó los ojos hacia el rostro lleno de lágrimas de su padre y dijo: `Padre, he pecado contra el cielo y ante tus ojos; ya no soy digno de ser llamado tu hijo' ---pero el joven no tuvo la posibilidad de terminar su confesión, porque el padre lleno de alegría dijo a los criados que para entonces habían llegado corriendo: `Traed enseguida su mejor vestido, aquel que guardé, y ponedselo, y poned en su mano el anillo de hijo y buscad unas sandalias para sus pies.'»\footnote{\textit{La bienvenida del Padre}: Lc 15:20b-22.}

\par 
%\textsuperscript{(1852.3)}
\textsuperscript{169:1.10} «Luego, después de que el feliz padre hubiera llevado hasta la casa al muchacho cansado y con los pies doloridos, dijo a sus criados: `Traed el ternero engordado y matadlo; comamos y divirtámonos, porque este hijo mío estaba muerto y vive de nuevo; estaba perdido y lo he encontrado.' Y todos se reunieron alrededor del padre para regocijarse con él por la restitución de su hijo»\footnote{\textit{La celebración}: Lc 15:23-24.}.

\par 
%\textsuperscript{(1852.4)}
\textsuperscript{169:1.11} «En ese momento, mientras lo estaban celebrando, el hijo mayor regresó de su trabajo cotidiano en el campo y, al acercarse a la casa, escuchó la música y el baile. Cuando llegó a la puerta de atrás, llamó a uno de los criados y le preguntó por el significado de toda esta celebración. El criado dijo entonces: `Tu hermano perdido desde hace mucho tiempo ha regresado al hogar, y tu padre ha matado al ternero engordado para regocijarse porque su hijo ha regresado sano y salvo. Entra para que puedas saludar también a tu hermano y acogerlo a su vuelta a la casa de tu padre.'»\footnote{\textit{El hermano mayor escucha el alboroto}: Lc 15:25-27.}

\par 
%\textsuperscript{(1852.5)}
\textsuperscript{169:1.12} «Pero cuando el hermano mayor escuchó esto, se sintió tan herido y enojado que no quiso entrar en la casa. Cuando su padre se enteró de su resentimiento por la bienvenida que le había dado a su hermano menor, salió para rogarle que entrara. Pero el hijo mayor no quiso ceder a la persuasión de su padre, y le contestó diciendo: `Te he servido aquí durante todos estos años sin transgredir nunca el más pequeño de tus mandamientos, y sin embargo, nunca me has dado ni siquiera un cabrito para poder divertirme con mis amigos. He permanecido aquí para cuidarte todos estos años y nunca has dado una fiesta por mi servicio fiel, pero cuando regresa este hijo tuyo, después de haber malgastado tu fortuna con las prostitutas, te apresuras a matar el ternero engordado y a festejar su regreso.'»\footnote{\textit{Las objecciones del hermano}: Lc 15:28-30.}

\par 
%\textsuperscript{(1852.6)}
\textsuperscript{169:1.13} «Como este padre amaba realmente a sus dos hijos, intentó razonar con el mayor: `Pero hijo mío, has estado conmigo todo este tiempo, y todo lo que poseo es tuyo. Hubieras podido coger un cabrito en cualquier momento que hubieras hecho amigos con quienes compartir tu alegría. Pero ahora es sencillamente apropiado que te unas a mí en la alegría y el regocijo por el regreso de tu hermano. Piensa en ello, hijo mío, tu hermano se había perdido y ha sido encontrado; ¡ha regresado vivo a nosotros!'»\footnote{\textit{El regocijo del Padre}: Lc 15:31-32.}

\par 
%\textsuperscript{(1853.1)}
\textsuperscript{169:1.14} Ésta fue una de las parábolas más conmovedoras y eficaces de todas las que Jesús presentó para convencer a sus oyentes de la buena voluntad del Padre en recibir a todos los que intentan entrar en el reino de los cielos.

\par 
%\textsuperscript{(1853.2)}
\textsuperscript{169:1.15} Jesús era muy aficionado a contar estas tres historias al mismo tiempo. Presentaba la historia de la oveja perdida\footnote{\textit{Parábola de la oveja perdida}: Mt 18:12-13; Lc 15:3-7.} para mostrar que, cuando los hombres se desvían involuntariamente del camino de la vida, el Padre tiene presentes a estos hijos \textit{perdidos}, y sale con sus Hijos, los verdaderos pastores del rebaño, a buscar a las ovejas perdidas. Luego narraba la historia de la moneda perdida\footnote{\textit{Parábola de la moneda perdida}: Lc 15:8-10.} en la casa para ilustrar cuán completa es la \textit{búsqueda} divina de todos los que están confusos, desconcertados, o cegados espiritualmente de otros modos por las preocupaciones materiales y las acumulaciones de la vida. Luego, Jesús se lanzaba a contar esta parábola del hijo perdido\footnote{\textit{Parábola del hijo perdido}: Lc 15:11-32.}, la acogida del pródigo que regresa, para mostrar cuán completo es el \textit{restablecimiento} del hijo perdido en la casa y en el corazón de su padre.

\par 
%\textsuperscript{(1853.3)}
\textsuperscript{169:1.16} Durante sus años de enseñanza, Jesús contó y volvió a contar muchísimas veces esta historia del hijo pródigo. Esta parábola y la historia del buen samaritano eran sus medios preferidos para enseñar el amor del Padre y las buenas relaciones entre los hombres.

\section*{2. La parábola del administrador sagaz}
\par 
%\textsuperscript{(1853.4)}
\textsuperscript{169:2.1} Una tarde, al comentar una de las declaraciones de Jesús, Simón Celotes\footnote{\textit{La pregunta de Simón Celotes}: Lc 16:8-9.} dijo: «Maestro, ¿qué has querido decir hoy cuando has afirmado que muchos hijos del mundo son más sensatos en su generación que los hijos del reino, puesto que tienen la habilidad de hacer amigos con las riquezas conseguidas injustamente?» Jesús respondió:

\par 
%\textsuperscript{(1853.5)}
\textsuperscript{169:2.2} «Antes de entrar en el reino, algunos de vosotros erais muy astutos en el trato con vuestros asociados en los negocios. Si erais injustos y a menudo desleales, sin embargo erais prudentes y previsores, en el sentido de que realizabais vuestros negocios con el ojo puesto únicamente en vuestro beneficio presente y en vuestra seguridad futura. Del mismo modo, ahora deberíais ordenar vuestra vida en el reino de tal manera que os proporcione la alegría en el presente y os asegure también el disfrute futuro de los tesoros acumulados en el cielo. Si erais tan diligentes en la obtención de ganancias para vosotros mismos cuando estabais al servicio del ego, ¿por qué tendríais que mostrar menos diligencia en ganar almas para el reino, puesto que ahora sois los servidores de la fraternidad de los hombres y los administradores de Dios?»\footnote{\textit{Sabios espiritualmente}: Mt 6:19-21; Lc 12:33-34.}

\par 
%\textsuperscript{(1853.6)}
\textsuperscript{169:2.3} «Todos podéis aprender una lección de la historia de cierto hombre rico que tenía un administrador astuto, pero injusto. Este administrador no sólo había presionado a los clientes de su señor en su propio beneficio egoísta, sino que también había malgastado y disipado directamente los fondos de su señor. Cuando todo esto llegó finalmente a oídos del dueño, éste llamó al administrador a su presencia y le preguntó por el significado de aquellos rumores; le exigió que le rindiera cuentas inmediatamente de su administración y que se preparara para entregar los asuntos de su señor a otra persona».\footnote{\textit{Parábola del mayordomo deshonesto}: Lc 16:1-2.}

\par 
%\textsuperscript{(1853.7)}
\textsuperscript{169:2.4} «Pero este administrador infiel empezó a decirse para sí: `¿Qué va a ser de mí, puesto que estoy a punto de perder esta administración? No tengo fuerzas para cavar la tierra, y me da verg\"uenza mendigar. Ya sé lo que voy a hacer para asegurarme de que seré bien recibido, cuando me quiten esta administración, en las casas de todos los que hacen negocios con mi señor.' Luego llamó a todos los deudores de su señor, y le dijo al primero: `¿Cuánto le debes a mi señor?' Éste respondió: `Cien medidas de aceite.' Entonces dijo el administrador: `Coge la tablilla de cera de tu deuda, siéntate deprisa, y cámbiala a cincuenta.' Luego dijo a otro deudor: `¿Cuánto debes tú?' Y éste replicó: `Cien medidas de trigo.' Entonces dijo el administrador: `Coge tu cuenta y escribe ochenta.' E hizo esto mismo con otros numerosos deudores. Este administrador poco honrado trataba así de hacer amigos para cuando le quitaran la administración. Incluso su dueño y señor, cuando se enteró posteriormente de esto, se vio obligado a admitir que su infiel administrador al menos había mostrado sagacidad en la manera en que había intentado asegurarse el porvenir para los días futuros de miseria y de adversidad».\footnote{\textit{Cuidarse de uno mismo}: Lc 16:3-8a.}

\par 
%\textsuperscript{(1854.1)}
\textsuperscript{169:2.5} «Así es como los hijos de este mundo muestran a veces más sabiduría que los hijos de la luz en la preparación de su futuro. A vosotros que pretendéis adquirir un tesoro en el cielo, os digo: Aprended de los que hacen amigos con las riquezas conseguidas injustamente, y conducid vuestra vida de tal manera que entabléis una amistad eterna con las fuerzas de la rectitud para que, cuando fallen todas las cosas terrenales, seáis recibidos con alegría en las moradas eternas».

\par 
%\textsuperscript{(1854.2)}
\textsuperscript{169:2.6} «Afirmo que aquel que es fiel en las cosas pequeñas también será fiel en las grandes, mientras que el que es injusto en las cosas pequeñas también lo será en las grandes. Si no habéis mostrado previsión e integridad en los asuntos de este mundo, ¿cómo podéis esperar ser fieles y prudentes cuando se os confíe la administración de las verdaderas riquezas del reino celestial? Si no sois unos buenos administradores y unos banqueros fieles, si no habéis sido fieles en lo que pertenece a otro, ¿quién será lo bastante insensato como para daros un gran tesoro en propiedad?»

\par 
%\textsuperscript{(1854.3)}
\textsuperscript{169:2.7} «Afirmo de nuevo que nadie puede servir a dos señores; o bien odiará a uno y amará al otro, o bien se quedará con uno mientras que despreciará al otro. No podéis servir a Dios y a las riquezas».

\par 
%\textsuperscript{(1854.4)}
\textsuperscript{169:2.8} Cuando los fariseos que estaban presentes escucharon esto, empezaron a burlarse y a reírse, puesto que eran muy dados a conseguir riquezas. Estos oyentes hostiles trataron de implicar a Jesús en un debate inútil, pero éste se negó a discutir con sus enemigos. Cuando los fariseos se pusieron a reñir entre ellos, sus fuertes voces atrajeron a una gran parte de la multitud que estaba acampada en los alrededores; y cuando empezaron a discutir entre sí, Jesús se retiró a su tienda para pasar la noche.

\section*{3. El hombre rico y el mendigo}
\par 
%\textsuperscript{(1854.5)}
\textsuperscript{169:3.1} Cuando la reunión se volvió demasiado ruidosa, Simón Pedro se levantó y se hizo cargo de la situación, diciendo: «Hombres y hermanos, no es apropiado que discutáis así entre vosotros. El Maestro ha hablado, y haríais bien en sopesar sus palabras. No os ha proclamado ninguna nueva doctrina. ¿No habéis oído también la alegoría de los nazareos sobre el rico y el mendigo? Algunos de nosotros hemos escuchado a Juan el Bautista decir a voz en grito esta parábola de advertencia a todos los que aman las riquezas y codician los bienes fraudulentos. Aunque esta antigua parábola no es conforme al evangelio que predicamos, todos haríais bien en prestar atención a sus lecciones, hasta el momento en que podáis comprender la nueva luz del reino de los cielos. La historia, tal como Juan la contaba, era así:»

\par 
%\textsuperscript{(1854.6)}
\textsuperscript{169:3.2} «Había un hombre rico llamado Dives que, vestido de púrpura y de lino fino, vivía todos los días en el regocijo y el esplendor. Y había un mendigo llamado Lázaro, que estaba tendido en la puerta de aquel rico, cubierto de llagas y deseando alimentarse con las migajas que caían de la mesa del rico. Sí, incluso los perros venían y le lamían las llagas. Y sucedió que el mendigo murió y fue llevado por los ángeles a descansar en el seno de Abraham. El rico murió también enseguida y fue enterrado con una gran pompa y un esplendor real. Cuando el rico partió de este mundo, se despertó en el Hades, y al encontrarse atormentado, levantó los ojos y vio a Abraham a lo lejos y a Lázaro en su seno. Entonces Dives gritó: `Padre Abraham, ten misericordia de mí y envíame a Lázaro para que moje la punta de su dedo en el agua y me refresque la lengua, porque sufro un gran suplicio a causa de mi castigo.' Entonces Abraham replicó: `Hijo mío, recuerda que disfrutaste de las cosas buenas durante tu vida, mientras que Lázaro soportaba las malas. Pero ahora todo ha cambiado, pues Lázaro recibe consuelo mientras que tú estás atormentado. Además, existe un gran abismo entre tú y nosotros, de manera que no podemos ir hasta ti, ni tú puedes venir hasta nosotros.' Entonces Dives le dijo a Abraham: `Te ruego que hagas volver a Lázaro a la casa de mi padre, ya que tengo cinco hermanos, para que pueda dar tal testimonio que impida que mis hermanos vengan a este lugar de tormento.' Pero Abraham dijo: {}`Hijo mío, tienen a Moisés y a los profetas; que los escuchen.' Entonces Dives respondió: `¡No, no, padre Abraham! Pero si alguien que ha muerto se presenta ante ellos, se arrepentirán.' Y entonces dijo Abraham: `Si no escuchan a Moisés y a los profetas, tampoco se convencerán aunque alguien resucite de entre los muertos.'»

\par 
%\textsuperscript{(1855.1)}
\textsuperscript{169:3.3} Después de que Pedro hubiera contado esta antigua parábola de la fraternidad nazarea, y como la multitud se había calmado, Andrés se levantó y disolvió la reunión para pasar la noche. Aunque tanto los apóstoles como los discípulos preguntaron con frecuencia a Jesús sobre la parábola de Dives y Lázaro, nunca consintió en comentarla.

\section*{4. El Padre y su reino}
\par 
%\textsuperscript{(1855.2)}
\textsuperscript{169:4.1} Jesús siempre tuvo dificultades cuando intentó explicar a los apóstoles que, aunque proclamaban el establecimiento del reino de Dios, el Padre que está en los cielos \textit{no era un rey}. En la época en que Jesús vivió en la Tierra y enseñó en la carne, los pueblos de Urantia conocían principalmente a reyes y emperadores en el gobierno de las naciones, y los judíos habían esperado durante mucho tiempo la llegada del reino de Dios. Por estas y otras razones, el Maestro pensó que era mejor llamar reino de los cielos a la fraternidad espiritual de los hombres, y \textit{Padre que está en los cielos} al jefe espiritual de esta fraternidad. Jesús nunca se refirió a su Padre como si fuera un rey. En sus conversaciones íntimas con los apóstoles, siempre se refería a sí mismo como el Hijo del Hombre, como el hermano mayor de ellos. Describía a todos sus seguidores como los «servidores de la humanidad» y como los «mensajeros del evangelio del reino».

\par 
%\textsuperscript{(1855.3)}
\textsuperscript{169:4.2} Jesús nunca dio a sus apóstoles una lección sistemática sobre la personalidad y los atributos del Padre que está en los cielos. Nunca pidió a los hombres que creyeran en su Padre, pues daba por hecho que lo hacían. Jesús nunca se rebajó a ofrecer argumentos que probaran la realidad del Padre. Toda su enseñanza acerca del Padre estaba centrada en la declaración de que él y el Padre son uno solo; que aquel que ha visto al Hijo ha visto al Padre; que el Padre, al igual que el Hijo, conoce todas las cosas; que sólo el Hijo conoce realmente al Padre y aquel a quien el Hijo se lo revela; que aquel que conoce al Hijo conoce también al Padre; y que el Padre lo había enviado al mundo para revelar sus naturalezas combinadas y para dar a conocer su trabajo conjunto. Nunca hizo otras declaraciones sobre su Padre, excepto a la mujer de Samaria en el pozo de Jacob, cuando afirmó: «Dios es espíritu».

\par 
%\textsuperscript{(1856.1)}
\textsuperscript{169:4.3} Aprendéis cosas sobre Dios a través de Jesús observando la divinidad de su vida, no dependiendo de sus enseñanzas. Cada uno puede asimilar, de la vida del Maestro, ese concepto de Dios que representa la medida de vuestra capacidad para percibir las realidades espirituales y divinas, las verdades reales y eternas. El finito nunca puede esperar comprender al Infinito, salvo cuando el Infinito estuvo focalizado en la personalidad espacio-temporal de la experiencia finita de la vida humana de Jesús de Nazaret.

\par 
%\textsuperscript{(1856.2)}
\textsuperscript{169:4.4} Jesús sabía muy bien que a Dios sólo se le puede conocer mediante las realidades de la experiencia; nunca se le puede comprender mediante la simple enseñanza de la mente. Jesús enseñó a sus apóstoles que, aunque nunca podrían comprender plenamente a Dios, podrían \textit{conocerlo} con toda certeza, tal como habían conocido al Hijo del Hombre. Podéis conocer a Dios, no comprendiendo lo que Jesús dijo, sino sabiendo lo que Jesús era. Jesús \textit{era} una revelación de Dios.

\par 
%\textsuperscript{(1856.3)}
\textsuperscript{169:4.5} Excepto cuando citaba las escrituras hebreas, Jesús sólo empleaba dos nombres para referirse a la Deidad: Dios y Padre. Cuando el Maestro se refería a su Padre como Dios, empleaba generalmente la palabra hebrea que significaba el Dios plural (la Trinidad), y no la palabra Yahvé, que representaba el concepto progresivo del Dios tribal de los judíos.

\par 
%\textsuperscript{(1856.4)}
\textsuperscript{169:4.6} Jesús nunca llamó rey al Padre, y lamentaba mucho que la esperanza de los judíos de poseer un reino restaurado y la proclamación de Juan sobre un reino venidero le hubieran obligado a denominar «reino de los cielos» a la fraternidad espiritual que se proponía establecer. Con una sola excepción ---la declaración de que «Dios es espíritu»--- Jesús nunca se refirió a la Deidad de manera distinta a los términos que describían su propia relación personal con la Fuente-Centro Primera del Paraíso.

\par 
%\textsuperscript{(1856.5)}
\textsuperscript{169:4.7} Jesús empleó la palabra Dios para designar la \textit{idea} de la Deidad, y la palabra Padre para designar la \textit{experiencia} de conocer a Dios. Cuando la palabra Padre se emplea para designar a Dios, se debería entender en su significado más amplio posible. La palabra Dios no se puede definir y representa por tanto el concepto infinito del Padre, pero como la palabra Padre se puede definir parcialmente, puede ser empleada para representar el concepto humano del Padre divino, tal como éste está asociado con el hombre en el transcurso de la existencia mortal.

\par 
%\textsuperscript{(1856.6)}
\textsuperscript{169:4.8} Elohim era para los judíos el Dios de los dioses, mientras que Yahvé era el Dios de Israel. Jesús aceptaba el concepto de Elohim y llamaba Dios a este grupo supremo de seres. En el lugar del concepto de Yahvé, la deidad racial, Jesús introdujo la idea de la paternidad de Dios y de la fraternidad mundial de los hombres. Elevó el concepto de Yahvé, el de un Padre racial deificado, hasta la idea de un Padre de todos los hijos de los hombres, un Padre divino del creyente individual. Y además enseñó que este Dios de los universos y este Padre de todos los hombres eran la misma y única Deidad Paradisiaca.

\par 
%\textsuperscript{(1856.7)}
\textsuperscript{169:4.9} Jesús nunca pretendió ser la manifestación de Elohim (Dios) en la carne. Nunca declaró que fuera una revelación de Elohim (Dios) para los mundos. Nunca enseñó que cualquiera que lo hubiera visto había visto a Elohim (Dios). Pero sí se proclamó como la revelación del Padre en la carne, y dijo también que cualquiera que lo hubiera visto había visto al Padre. Como Hijo divino afirmó que sólo representaba al Padre.

\par 
%\textsuperscript{(1857.1)}
\textsuperscript{169:4.10} En verdad, él era incluso el Hijo del Dios Elohim; pero en la similitud de la carne mortal y para los hijos mortales de Dios, escogió limitar la revelación de su vida a la descripción del carácter de su Padre hasta donde esta revelación pudiera ser comprensible por el hombre mortal. En cuanto al carácter de las otras personas de la Trinidad del Paraíso, deberemos contentarnos con la enseñanza de que son totalmente como el Padre, cuya descripción personal ha sido revelada en la vida de su Hijo encarnado, Jesús de Nazaret.

\par 
%\textsuperscript{(1857.2)}
\textsuperscript{169:4.11} Aunque Jesús reveló en su vida terrenal la verdadera naturaleza del Padre celestial, pocas cosas enseñó sobre él. De hecho, sólo enseñó dos cosas: que Dios es en sí mismo espíritu y que, en todas las cuestiones de las relaciones con sus criaturas, es un Padre. Aquella noche, Jesús efectuó la declaración final de su relación con Dios cuando afirmó: «He salido del Padre y he venido al mundo; de nuevo, dejaré el mundo e iré al Padre».

\par 
%\textsuperscript{(1857.3)}
\textsuperscript{169:4.12} ¡Pero prestad atención! Jesús nunca dijo: «Cualquiera que me ha escuchado, ha escuchado a Dios». Pero sí dijo: «Aquel que me ha \textit{visto}, ha visto al Padre». Escuchar la enseñanza de Jesús no equivale a conocer a Dios, pero \textit{ver} a Jesús es una experiencia que es en sí misma una revelación del Padre al alma. El Dios de los universos gobierna la extensa creación, pero es el Padre que está en los cielos el que envía a su espíritu para que resida dentro de vuestra mente.

\par 
%\textsuperscript{(1857.4)}
\textsuperscript{169:4.13} Jesús es, en su semejanza humana, la lente espiritual que hace visible para la criatura material a Aquel que es invisible. Es vuestro hermano mayor que, en la carne, os hace \textit{conocer} a un Ser con atributos infinitos que ni siquiera las huestes celestiales pueden vanagloriarse de comprender plenamente. Pero todo esto debe consistir en la experiencia personal del \textit{creyente individual}. Dios, que es espíritu, sólo se puede conocer como experiencia espiritual. A los hijos finitos de los mundos materiales, el Hijo divino de los reinos espirituales sólo les puede revelar a Dios como \textit{Padre}. Podéis conocer al Eterno como Padre, pero podéis adorarlo como el Dios de los universos, el Creador infinito de todo lo que existe.