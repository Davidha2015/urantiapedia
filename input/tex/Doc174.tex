\chapter{Documento 174. El martes por la mañana en el templo}
\par 
%\textsuperscript{(1897.1)}
\textsuperscript{174:0.1} HACIA las siete de la mañana de este martes, Jesús se reunió, en la casa de Simón, con los apóstoles, el cuerpo de mujeres y unas dos docenas de otros discípulos destacados. En esta reunión se despidió de Lázaro, y le dio las instrucciones que le indujeron a huir rápidamente a Filadelfia en Perea, donde se unió más tarde al movimiento misionero que tenía su sede en aquella ciudad. Jesús también se despidió del anciano Simón, y dio sus consejos de despedida al cuerpo de mujeres, pues nunca más se dirigió a ellas de manera oficial.

\par 
%\textsuperscript{(1897.2)}
\textsuperscript{174:0.2} Esta mañana, saludó a cada uno de lo doce con unas palabras personales. A Andrés le dijo: «No te desanimes por los acontecimientos inminentes. Controla firmemente a tus hermanos y procura que no te vean abatido». A Pedro le dijo: «No pongas tu confianza en el vigor de tu brazo ni en las armas de acero. Asiéntate sobre los fundamentos espirituales de las rocas eternas». A Santiago le dijo: «No vaciles ante las apariencias externas. Permanece firme en tu fe, y pronto conocerás la realidad de aquello en lo que crees». A Juan le dijo: «Sé dulce; ama incluso a tus enemigos; sé tolerante. Y recuerda que te he confiado muchas cosas». A Natanael le dijo: «No juzgues por las apariencias; permanece firme en tu fe cuando todo parezca desvanecerse; sé fiel a tu misión de embajador del reino». A Felipe le dijo: «No te dejes conmover por los acontecimientos inminentes. Permanece impasible, aunque no puedas ver el camino. Sé fiel a tu juramento de consagración». A Mateo le dijo: «No olvides la misericordia que te recibió en el reino. No dejes que nadie te robe tu recompensa eterna. Puesto que has resistido las tendencias de la naturaleza humana, dispónte a ser firme». A Tomás le dijo: «Por muy difícil que sea, ahora tienes que caminar por la fe y no por la vista. No dudes de que yo sea capaz de terminar la obra que he empezado, y de que finalmente veré a todos mis fieles embajadores en el reino del más allá». A los gemelos Alfeo les dijo: «No permitáis que os abrumen las cosas que no podéis comprender. Sed fieles a los afectos de vuestro corazón, y no pongáis vuestra confianza ni en los grandes hombres ni en la actitud cambiante de la gente. Permaneced al lado de vuestros hermanos». A Simón Celotes le dijo: «Simón, quizás te sientas abrumado por la decepción, pero tu espíritu se elevará por encima de todo lo que pueda sucederte. Lo que no has conseguido aprender de mí, mi espíritu te lo enseñará. Busca las verdaderas realidades del espíritu, y deja de sentirte atraído por las sombras irreales y materiales». Y a Judas Iscariote le dijo: «Judas, te he amado y he rogado para que ames a tus hermanos. No te canses de hacer el bien; y deseo prevenirte que te guardes de los senderos resbaladizos de la adulación y de los dardos envenenados del ridículo».

\par 
%\textsuperscript{(1897.3)}
\textsuperscript{174:0.3} Una vez que hubo concluido estos saludos, partió para Jerusalén con Andrés, Pedro, Santiago y Juan, mientras que los demás apóstoles se ocupaban de establecer el campamento de Getsemaní, donde iban a dirigirse aquella noche, y donde instalaron su cuartel general durante el resto de la vida mortal del Maestro. Aproximadamente a medio camino del descenso del Olivete, Jesús se detuvo y conversó durante más de una hora con los cuatro apóstoles.

\section*{1. El perdón divino}
\par 
%\textsuperscript{(1898.1)}
\textsuperscript{174:1.1} Durante varios días, Pedro y Santiago habían estado discutiendo sus diferencias de opinión sobre la enseñanza del Maestro acerca del perdón de los pecados. Los dos habían acordado plantear el asunto a Jesús, y Pedro aprovechó esta ocasión como una oportunidad adecuada para obtener el consejo del Maestro. En consecuencia, Simón Pedro interrumpió la conversación sobre las diferencias entre la alabanza y la adoración, y preguntó: «Maestro, Santiago y yo no estamos de acuerdo sobre tus enseñanzas relacionadas con el perdón de los pecados. Santiago afirma que, según tu enseñanza, el Padre nos perdona incluso antes de que se lo pidamos, y yo sostengo que el arrepentimiento y la confesión deben preceder al perdón. ¿Quién de nosotros tiene razón? ¿Qué dices tú?»

\par 
%\textsuperscript{(1898.2)}
\textsuperscript{174:1.2} Después de un breve silencio, Jesús miró de manera significativa a los cuatro y contestó: «Hermanos míos, os equivocáis en vuestras opiniones porque no comprendéis la naturaleza de las relaciones íntimas y amorosas entre la criatura y el Creador, entre el hombre y Dios. No lográis captar la simpatía comprensiva que un padre sabio alberga por su hijo inmaduro y a veces equivocado. En verdad es dudoso que unos padres inteligentes y afectuosos se vean nunca en la necesidad de perdonar a un hijo normal y corriente. Las relaciones comprensivas, asociadas con las actitudes amorosas, impiden eficazmente todos los distanciamientos que necesitan posteriormente un reajuste mediante el arrepentimiento del hijo y el perdón del padre».

\par 
%\textsuperscript{(1898.3)}
\textsuperscript{174:1.3} «En cada hijo vive una fracción de su padre. El padre disfruta de una prioridad y de una superioridad de comprensión en todas las cuestiones relacionadas con la relación entre padre e hijo. El padre es capaz de percibir la inmadurez del hijo a la luz de la madurez paternal más elevada, de la experiencia más madura que posee el compañero de más edad. En el caso del hijo terrestre y del Padre celestial, el padre divino posee, de una manera infinita y divina, la compasión y la capacidad para comprender con amor. El perdón divino es inevitable; es inherente e inalienable a la comprensión infinita de Dios, a su conocimiento perfecto de todo lo relacionado con el juicio erróneo y la elección equivocada del hijo. La justicia divina es tan eternamente equitativa que engloba infaliblemente una misericordia comprensiva».

\par 
%\textsuperscript{(1898.4)}
\textsuperscript{174:1.4} «Cuando un hombre sensato comprende los impulsos internos de sus semejantes, los ama. Y cuando amáis a vuestro hermano, ya lo habéis perdonado. Esta capacidad para comprender la naturaleza del hombre y para perdonar sus aparentes fechorías, es divina. Si sois unos padres sabios, así es como amaréis y comprenderéis a vuestros hijos, e incluso los perdonaréis cuando los malentendidos pasajeros os hayan separado aparentemente. El hijo es inmaduro y no comprende plenamente la profundidad de la relación entre padre e hijo; por eso experimenta con frecuencia un sentimiento de separación culpable cuando no tiene la plena aprobación de su padre, pero un verdadero padre nunca tiene conciencia de una separación semejante. El pecado es una experiencia de la conciencia de la criatura; no forma parte de la conciencia de Dios».

\par 
%\textsuperscript{(1898.5)}
\textsuperscript{174:1.5} «Vuestra incapacidad o vuestra mala disposición para perdonar a vuestros semejantes es la medida de vuestra inmadurez, de vuestro fracaso en alcanzar el nivel adulto de compasión, de comprensión y de amor. Vuestros rencores y vuestras ideas de venganza son directamente proporcionales a vuestra ignorancia de la naturaleza interior y de los verdaderos anhelos de vuestros hijos y de vuestros semejantes. El amor es la manifestación exterior del impulso de vida interior y divino. Está basado en la comprensión, alimentado por el servicio desinteresado y perfeccionado con la sabiduría».

\section*{2. Las preguntas de los dirigentes judíos}
\par 
%\textsuperscript{(1899.1)}
\textsuperscript{174:2.1} El lunes por la noche se había celebrado un consejo entre el sanedrín y unos cincuenta dirigentes adicionales seleccionados entre los escribas, los fariseos y los saduceos. Esta asamblea llegó al consenso de que sería peligroso arrestar a Jesús en público a causa de su influencia sobre los sentimientos de la gente común. La mayoría opinaba también que había que hacer un esfuerzo decidido para desacreditarlo a los ojos de la multitud, antes de arrestarlo y de llevarlo a juicio. En consecuencia, se designaron diversos grupos de hombres eruditos para que estuvieran disponibles a la mañana siguiente en el templo, a fin de intentar hacerlo caer en una trampa con preguntas difíciles, y tratar de desconcertarlo de otras maneras delante de la gente. Al fin, los fariseos, los saduceos e incluso los herodianos se encontraban todos unidos en este esfuerzo por desacreditar a Jesús a los ojos de las multitudes pascuales.

\par 
%\textsuperscript{(1899.2)}
\textsuperscript{174:2.2} El martes por la mañana, cuando Jesús llegó al patio del templo y empezó a enseñar, sólo había pronunciado algunas palabras cuando un grupo de los estudiantes más jóvenes de las academias, que habían sido preparados de antemano con esta finalidad, se adelantaron y se dirigieron a Jesús a través de su portavoz, diciendo: «Maestro, sabemos que eres un instructor honrado; sabemos que proclamas los caminos de la verdad y que sólo sirves a Dios, porque no temes a ningún hombre y no haces acepción de personas. Sólo somos unos estudiantes, y quisiéramos conocer la verdad sobre una cuestión que nos preocupa. Nuestra dificultad es la siguiente: ¿Es lícito que paguemos tributo al César? ¿Hemos de pagarlo o no?» Percibiendo su hipocresía y su astucia, Jesús les dijo: «¿Por qué venís a tentarme de esta manera? Mostradme el dinero del tributo, y os contestaré». Cuando los estudiantes le entregaron un denario, lo examinó y dijo: «¿De quién es la imagen y la inscripción que lleva esta moneda?» Cuando le contestaron: «Del César», Jesús dijo: «Dad al César las cosas que son del César, y dad a Dios las cosas que son de Dios».

\par 
%\textsuperscript{(1899.3)}
\textsuperscript{174:2.3} Después de haber contestado así, los jóvenes escribas y sus cómplices herodianos se retiraron de su presencia, y la gente, incluídos los saduceos, disfrutaron de su turbación. Incluso los jóvenes que habían intentado hacer caer al Maestro en una trampa, se maravillaron enormemente de la inesperada sagacidad de su respuesta.

\par 
%\textsuperscript{(1899.4)}
\textsuperscript{174:2.4} El día anterior, los dirigentes habían intentado que cometiera un desliz delante de la multitud en cuestiones de autoridad eclesiástica, y como habían fracasado, ahora intentaban implicarlo en una discusión perjudicial sobre la autoridad civil. Tanto Pilatos como Herodes se encontraban en Jerusalén en aquel momento, y los enemigos de Jesús supusieron que si se atrevía a aconsejar que no se pagara el tributo al César, podrían ir inmediatamente a las autoridades romanas y acusarlo de sedición. Por otra parte, si aconsejaba expresamente el pago del tributo, calculaban con razón que dicha declaración heriría profundamente el orgullo nacional de sus oyentes judíos, desviando así la buena voluntad y el afecto de la multitud.

\par 
%\textsuperscript{(1899.5)}
\textsuperscript{174:2.5} Los enemigos de Jesús fueron derrotados en todo esto puesto que una orden bien conocida del sanedrín, emitida para orientar a los judíos dispersos por las naciones gentiles, precisaba que el «derecho de acuñar moneda comportaba el derecho de exigir impuestos». De esta manera, Jesús había evitado la trampa. Si hubiera contestado «no» a su pregunta, hubiera sido el equivalente de incitar a la rebelión; si hubiera contestado «sí», habría conmocionado los sentimientos nacionalistas profundamente arraigados de aquella época. El Maestro no eludió la pregunta; simplemente utilizó la sabiduría de ofrecer una respuesta doble. Jesús nunca era evasivo, pero siempre era sabio en su trato con los que intentaban acosarlo y destruirlo.

\section*{3. Los saduceos y la resurrección}
\par 
%\textsuperscript{(1900.1)}
\textsuperscript{174:3.1} Antes de que Jesús pudiera empezar su enseñanza, otro grupo se adelantó para hacerle preguntas, en esta ocasión un grupo de saduceos eruditos y astutos. Su portavoz se acercó y le dijo: «Maestro, Moisés dijo que si un hombre casado moría sin dejar hijos, su hermano se casaría con la mujer y engendraría una descendencia a su hermano muerto. Pues bien, se ha producido un caso en el que un hombre que tenía seis hermanos murió sin hijos; el hermano siguiente se casó con su mujer, pero también murió pronto sin dejar hijos. El segundo hermano tomó asimismo a la mujer, pero también murió sin dejar descendencia. Y así sucesivamente hasta que los seis hermanos se casaron con ella, y los seis murieron sin dejar hijos. Luego, la mujer murió después de todos ellos. Pues bien, lo que quisiéramos preguntarte es lo siguiente: Cuando llegue la resurrección, ¿de quién será la esposa, puesto que los siete hermanos se casaron con ella?»

\par 
%\textsuperscript{(1900.2)}
\textsuperscript{174:3.2} Jesús sabía, y la gente también, que estos saduceos no eran sinceros al hacer esta pregunta, porque no era probable que un caso así se produjera realmente; además, esta costumbre de que los hermanos de un muerto trataran de engendrarle hijos, era prácticamente letra muerta entre los judíos de esta época. Sin embargo, Jesús condescendió a contestar a su pregunta maliciosa. Dijo: «Todos os equivocáis al hacer este tipo de preguntas, porque no conocéis ni las Escrituras ni el poder viviente de Dios. Sabéis que los hijos de este mundo pueden casarse y ser dados en matrimonio, pero no parecéis comprender que aquellos que son considerados dignos de alcanzar los mundos venideros, mediante la resurrección de los justos, no se casan ni son dados en matrimonio. Los que experimentan la resurrección de entre los muertos se parecen más a los ángeles del cielo, y no mueren nunca. Esos resucitados son eternamente los hijos de Dios; son los hijos de la luz resucitados para el progreso de la vida eterna. Incluso vuestro padre Moisés comprendió esto porque, en conexión con sus experiencias junto a la zarza ardiente, oyó decir al Padre:
`Yo \textit{soy} el Dios de Abraham, el Dios de Isaac y el Dios de Jacob.' Y así, junto con Moisés, declaro que mi Padre no es el Dios de los muertos, sino de los vivos. En él todos vivís, os reproducís y poseéis vuestra existencia mortal».

\par 
%\textsuperscript{(1900.3)}
\textsuperscript{174:3.3} Cuando Jesús hubo terminado de contestar estas preguntas, los saduceos se retiraron, y algunos fariseos se olvidaron tanto de sí mismos que exclamaron: «Es verdad, es verdad, Maestro, has contestado bien a esos saduceos incrédulos». Los saduceos no se atrevieron a hacerle más preguntas, y la gente común se maravilló de la sabiduría de su enseñanza.

\par 
%\textsuperscript{(1900.4)}
\textsuperscript{174:3.4} En su choque con los saduceos, Jesús sólo recurrió a Moisés porque esta secta político-religiosa únicamente reconocía la validez de los llamados cinco libros de Moisés; no aceptaban que las enseñanzas de los profetas sirvieran de base para los dogmas doctrinales. En su respuesta, el Maestro afirmó categóricamente el hecho de la supervivencia de las criaturas mortales mediante la técnica de la resurrección, pero no aprobó en ningún sentido las creencias fariseas en la resurrección del cuerpo humano físico. El punto que Jesús deseaba recalcar era que el Padre había dicho:
`Yo \textit{soy} el Dios de Abraham, de Isaac y de Jacob', y no yo \textit{era} su Dios.

\par 
%\textsuperscript{(1900.5)}
\textsuperscript{174:3.5} Los saduceos habían querido someter a Jesús a la influencia debilitante del \textit{ridículo}, sabiendo muy bien que toda persecución en público crearía sin duda una mayor simpatía hacia él en la mente de la multitud.

\section*{4. El gran mandamiento}
\par 
%\textsuperscript{(1901.1)}
\textsuperscript{174:4.1} Otro grupo de saduceos había recibido instrucciones para hacerle a Jesús unas preguntas enredosas sobre los ángeles, pero cuando observaron la suerte de sus compañeros que habían intentado hacerlo caer en una trampa con preguntas relacionadas con la resurrección, decidieron muy juiciosamente permanecer en silencio; se retiraron sin hacer una sola pregunta. Los fariseos, los escribas, los saduceos y los herodianos aliados habían premeditado el plan de pasarse todo el día haciéndole estas preguntas enredosas, esperando así desacreditar a Jesús delante de la gente, y al mismo tiempo impedirle eficazmente que tuviera tiempo para proclamar sus enseñanzas perturbadoras.

\par 
%\textsuperscript{(1901.2)}
\textsuperscript{174:4.2} Uno de los grupos de fariseos se adelantó entonces para hacerle preguntas embarazosas; el portavoz hizo señas a Jesús, y dijo: «Maestro, soy jurista, y me gustaría preguntarte cuál es, en tu opinión, el mandamiento más grande». Jesús respondió: «No hay más que un solo mandamiento, que es el más grande de todos, y ese mandamiento es: `Escucha, oh Israel, al Señor nuestro Dios; el Señor es uno; y amarás al Señor tu Dios con todo tu corazón y con toda tu alma, con toda tu mente y con todas tus fuerzas.' Éste es el primer gran mandamiento. Y el segundo mandamiento se parece a este primero; en efecto, proviene directamente de él, y dice: `Amarás a tu prójimo como a ti mismo.' No hay otros mandamientos más grandes que estos; en estos dos mandamientos se apoyan toda la ley y los profetas».

\par 
%\textsuperscript{(1901.3)}
\textsuperscript{174:4.3} Cuando el jurista percibió que Jesús no solamente había respondido de acuerdo con el concepto más elevado de la religión judía, sino que también había contestado sabiamente a los ojos de la multitud reunida, pensó que era mejor tener el valor de alabar abiertamente la respuesta del Maestro. En consecuencia, dijo: «En verdad, Maestro, has dicho bien que Dios es uno y que no hay nadie aparte de él; y que el primer gran mandamiento es amarlo con todo el corazón, con toda la inteligencia y con todas nuestras fuerzas, y también amar al prójimo como a uno mismo. Estamos de acuerdo en que este gran mandamiento tiene mucha más importancia que todos los holocaustos y sacrificios». Cuando el jurista contestó de esta manera tan prudente, Jesús bajó la mirada hacia él y dijo: «Amigo mío, percibo que no estás muy lejos del reino de Dios».

\par 
%\textsuperscript{(1901.4)}
\textsuperscript{174:4.4} Jesús dijo la verdad cuando indicó que este jurista «no estaba muy lejos del reino», porque aquella misma noche fue al campamento del Maestro, cerca de Getsemaní, confesó su fe en el evangelio del reino y fue bautizado por Josías, uno de los discípulos de Abner.

\par 
%\textsuperscript{(1901.5)}
\textsuperscript{174:4.5} Otros dos o tres grupos de escribas y fariseos estaban presentes y habían tenido la intención de hacerle preguntas, pero se sentían desarmados por la respuesta de Jesús al jurista, o bien estaban acobardados por la derrota de todos los que habían intentado enredarlo. Después de esto, nadie se atrevió a hacerle más preguntas en público.

\par 
%\textsuperscript{(1901.6)}
\textsuperscript{174:4.6} Como no había más preguntas y se estaba acercando la hora del mediodía, Jesús no reanudó su enseñanza, sino que se contentó simplemente con hacer una pregunta a los fariseos y a sus asociados. Jesús dijo: «Puesto que no hacéis más preguntas, me gustaría haceros una. ¿Qué pensáis del Libertador? Es decir, ¿de quién es hijo?» Después de una breve pausa, uno de los escribas contestó: «El Mesías es el hijo de David». Puesto que Jesús sabía que se había discutido mucho, incluso entre sus propios discípulos, sobre si él era o no el hijo de David, hizo esta otra pregunta: «Si el Libertador es en verdad el hijo de David, ¿cómo puede ser que en el salmo que atribuís a David, él mismo dice, hablando según el espíritu: `El Señor dijo a mi señor: Siéntate a mi derecha hasta que ponga a tus enemigos de banqueta para tus pies?' Si David le llama Señor, entonces ¿cómo puede ser su hijo?» Los dirigentes, los escribas y los principales sacerdotes no contestaron a esta pregunta, pero también se abstuvieron de hacerle más preguntas para intentar enredarlo. Nunca contestaron a la pregunta que Jesús les había hecho, pero después de la muerte del Maestro, intentaron eludir la dificultad cambiando la interpretación de este salmo para que se refiriera a Abraham en lugar del Mesías. Otros trataron de evitar este dilema negando que David fuera el autor de este salmo llamado mesiánico.

\par 
%\textsuperscript{(1902.1)}
\textsuperscript{174:4.7} Un rato antes, los fariseos habían disfrutado con la manera en que el Maestro había acallado a los saduceos; ahora los saduceos se regocijaban con el fracaso de los fariseos; pero esta rivalidad sólo era momentánea; rápidamente se olvidaron de sus diferencias tradicionales, en un esfuerzo común por poner fin a las enseñanzas y a las obras de Jesús. Pero durante todas estas experiencias, la gente común le escuchó con agrado.

\section*{5. Los griegos indagadores}
\par 
%\textsuperscript{(1902.2)}
\textsuperscript{174:5.1} Alrededor del mediodía, mientras Felipe compraba unas provisiones para el nuevo campamento que se estaba estableciendo aquel día cerca de Getsemaní, fue abordado por una delegación de extranjeros, un grupo de creyentes griegos de Alejandría, Atenas y Roma, cuyo portavoz le dijo al apóstol: «Los que te conocen nos han dicho que nos dirijamos a ti; por eso venimos a ti, Señor, con la petición de ver a Jesús, tu Maestro». A Felipe le cogió de sorpresa el encontrarse así, en la plaza del mercado, con estos gentiles griegos eminentes e indagadores. Puesto que Jesús había encargado explícitamente a los doce que no efectuaran ninguna enseñanza pública durante la semana de la Pascua, Felipe estaba un poco confuso sobre la manera correcta de manejar esta situación. También estaba desconcertado porque estos hombres eran gentiles extranjeros. Si hubieran sido judíos, o gentiles conocidos de los alrededores, no hubiera dudado tanto. Lo que hizo fue lo siguiente: Pidió a aquellos griegos que permanecieran allí donde estaban. Mientras se alejaba deprisa, los griegos supusieron que había ido a buscar a Jesús, pero en realidad corrió a la casa de José, donde sabía que Andrés y los otros apóstoles estaban almorzando. Llamó a Andrés para que saliera, le explicó el motivo de su venida, y luego regresó con Andrés al lugar donde esperaban los griegos.

\par 
%\textsuperscript{(1902.3)}
\textsuperscript{174:5.2} Como Felipe casi había terminado de comprar las provisiones, regresó con Andrés y los griegos a la casa de José, donde Jesús los recibió. Se sentaron cerca del Maestro, mientras éste hablaba a sus apóstoles y a un grupo de discípulos principales reunidos en este almuerzo. Jesús dijo:

\par 
%\textsuperscript{(1902.4)}
\textsuperscript{174:5.3} «Mi Padre me ha enviado a este mundo para revelar su bondad a los hijos de los hombres, pero los primeros a quienes me he dirigido se han negado a recibirme. Es verdad que muchos de vosotros habéis creído en mi evangelio por vosotros mismos, pero los hijos de Abraham y sus dirigentes están a punto de rechazarme, y al hacerlo, rechazarán a Aquél que me ha enviado. He proclamado sin reservas el evangelio de la salvación a este pueblo; les he hablado de la filiación acompañada de alegría, de libertad y de una vida más abundante en el espíritu. Mi Padre ha realizado muchas obras maravillosas entre estos hijos de los hombres tiranizados por el miedo. Pero el profeta Isaías se refirió con razón a este pueblo cuando escribió: `Señor, ¿quién ha creído en nuestras enseñanzas? ¿Y a quién ha sido revelado el Señor?' En verdad, los dirigentes de mi pueblo se han cegado deliberadamente para no ver, y han endurecido su corazón por temor a creer y a ser salvados. Todos estos años he tratado de curarlos de su incredulidad, para que puedan recibir la salvación eterna del Padre. Sé que no todos me han fallado; algunos de vosotros habéis creído de verdad en mi mensaje. En esta sala hay ahora veinte hombres que han sido anteriormente miembros del sanedrín, o que han ocupado altos puestos en los consejos de la nación, aunque algunos de ellos evitan todavía confesar abiertamente la verdad, por temor a ser expulsados de la sinagoga. Algunos de vosotros tenéis la tentación de amar más la gloria de los hombres que la gloria de Dios. Pero me veo obligado a mostrar paciencia, puesto que temo incluso por la seguridad y la lealtad de algunos de los que han estado tanto tiempo junto a mí, y que han vivido tan cerca de mi».

\par 
%\textsuperscript{(1903.1)}
\textsuperscript{174:5.4} «Observo que en esta sala de banquetes están reunidos los judíos y los gentiles en un número aproximadamente igual, y os dirigiré la palabra como al primer y último grupo de este tipo que voy a instruir en los asuntos del reino antes de ir hacia mi Padre».

\par 
%\textsuperscript{(1903.2)}
\textsuperscript{174:5.5} Estos griegos habían asistido fielmente a las enseñanzas de Jesús en el templo. El lunes por la noche habían celebrado una conferencia en la casa de Nicodemo, que se había prolongado hasta el amanecer, y treinta de ellos habían elegido entrar en el reino.

\par 
%\textsuperscript{(1903.3)}
\textsuperscript{174:5.6} Mientras Jesús permanecía delante de ellos en aquel momento, percibió el final de una dispensación y el principio de otra. Volviendo su atención hacia los griegos, el Maestro dijo:

\par 
%\textsuperscript{(1903.4)}
\textsuperscript{174:5.7} «El que cree en este evangelio, no solamente cree en mí, sino en Aquel que me ha enviado. Cuando me miráis, no veis solamente al Hijo del Hombre, sino también a Aquel que me ha enviado. Yo soy la luz del mundo, y cualquiera que crea en mi enseñanza ya no permanecerá más tiempo en las tinieblas. Si vosotros, los gentiles, queréis escucharme, recibiréis las palabras de la vida y entraréis inmediatamente en la gozosa libertad de la verdad de la filiación con Dios. Si mis compatriotas, los judíos, escogen rechazarme y rehusar mis enseñanzas, no los juzgaré, porque no he venido para juzgar al mundo, sino para ofrecerle la salvación. Sin embargo, los que me rechazan y rehúsan recibir mi enseñanza, serán llevados a juicio a su debido tiempo por mi Padre y por aquellos que él ha designado para que juzguen a los que rechazan el don de la misericordia y las verdades de la salvación. Recordad todos que no hablo por mí mismo, sino que os he proclamado fielmente lo que el Padre mandó que yo debía revelar a los hijos de los hombres. Y estas palabras que el Padre me ordenó que dijera al mundo son palabras de verdad divina, de misericordia perpetua y de vida eterna».

\par 
%\textsuperscript{(1903.5)}
\textsuperscript{174:5.8} «Pero declaro tanto a los judíos como a los gentiles, que está a punto de llegar la hora en que el Hijo del Hombre será glorificado. Sabéis muy bien que un grano de trigo permanece solitario, a menos que caiga en la tierra y muera; pero si muere en una buena tierra, surge de nuevo a la vida y produce mucho fruto. Aquel que ama egoístamente su vida, corre el peligro de perderla; pero aquel que está dispuesto a dar su vida por mí y por el evangelio, gozará de una existencia más abundante en la Tierra, y de la vida eterna en el cielo. Si queréis seguirme sinceramente, incluso después de que haya regresado al Padre, entonces os convertiréis en mis discípulos y en los sinceros servidores de vuestros semejantes».

\par 
%\textsuperscript{(1903.6)}
\textsuperscript{174:5.9} «Sé que se acerca mi hora, y estoy preocupado. Me doy cuenta de que mi pueblo está decidido a despreciar el reino, pero me alegra recibir a estos gentiles que buscan la verdad, y que hoy están aquí para preguntar por el camino de la luz. Sin embargo, mi corazón sufre por mi pueblo, y mi alma está angustiada por lo que me espera. ¿Qué puedo decir cuando miro hacia adelante y percibo lo que está a punto de sucederme? ¿Acaso diré: Padre, sálvame de esta hora terrible? ¡No! Precisamente con esta finalidad he venido al mundo, e incluso he llegado hasta esta hora. Diré más bien, orando para que os unáis a mí: Padre, glorifica tu nombre; que se haga tu voluntad».

\par 
%\textsuperscript{(1904.1)}
\textsuperscript{174:5.10} Cuando Jesús hubo hablado así, el Ajustador Personalizado que había residido en él antes de su bautismo apareció delante de él, y mientras hacía una pausa de manera perceptible, este espíritu ahora poderoso que representaba al Padre le habló a Jesús de Nazaret, diciendo:«He glorificado mi nombre muchas veces en tus donaciones, y lo glorificaré una vez más».

\par 
%\textsuperscript{(1904.2)}
\textsuperscript{174:5.11} Aunque los judíos y los gentiles allí reunidos no escucharon ninguna voz, no pudieron dejar de percibir que el Maestro se había detenido en su discurso mientras le llegaba un mensaje de alguna fuente sobrehumana. Cada uno le dijo al que tenía a su lado: «Un ángel le ha hablado».

\par 
%\textsuperscript{(1904.3)}
\textsuperscript{174:5.12} Entonces Jesús continuó diciendo: «Todo esto no ha sucedido por mi bien, sino por el vuestro. Sé con certeza que el Padre me recibirá y aceptará mi misión en vuestro favor, pero es necesario que os sintáis estimulados y preparados para la prueba de fuego que se avecina. Dejadme aseguraros que la victoria terminará por coronar nuestros esfuerzos unidos por iluminar al mundo y liberar a la humanidad. El antiguo orden de cosas se está juzgando a sí mismo; he derribado al Príncipe de este mundo, y todos los hombres llegarán a ser libres gracias a la luz del espíritu que yo derramaré sobre toda carne, después de haber ascendido hasta mi Padre que está en los cielos».

\par 
%\textsuperscript{(1904.4)}
\textsuperscript{174:5.13} «Y ahora os afirmo que, si soy elevado en la Tierra y en vuestras vidas, atraeré a todos los hombres hacia mí y hacia la comunidad de mi Padre. Habéis creído que el Libertador residiría para siempre en la Tierra, pero declaro que el Hijo del Hombre será rechazado por los hombres, y que regresará al Padre. Sólo estaré con vosotros un corto período de tiempo; la luz viviente sólo estará poco tiempo en medio de esta generación tenebrosa. Caminad mientras tengáis esta luz, para que las tinieblas y la confusión venideras no os cojan por sorpresa. El que camina en las tinieblas, no sabe adonde va; pero si escogéis caminar en la luz, todos os convertiréis en verdad en los hijos liberados de Dios. Y ahora, venid conmigo todos vosotros mientras regresamos al templo, donde voy a decir mis palabras de adiós a los jefes de los sacerdotes, a los escribas, a los fariseos, a los saduceos, a los herodianos y a los dirigentes ignorantes de Israel».

\par 
%\textsuperscript{(1904.5)}
\textsuperscript{174:5.14} Después de haber hablado así, Jesús condujo al grupo de regreso hacia el templo por las estrechas calles de Jerusalén. Acababan de oír decir al Maestro que éste iba a ser su discurso de adiós en el templo, y le siguieron en silencio, meditando profundamente.