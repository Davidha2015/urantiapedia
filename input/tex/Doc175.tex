\chapter{Documento 175. El último discurso en el templo}
\par 
%\textsuperscript{(1905.1)}
\textsuperscript{175:0.1} POCO después de las dos de la tarde de este martes, Jesús llegó al templo en compañía de once apóstoles, José de Arimatea, los treinta griegos y algunos otros discípulos, y empezó a pronunciar su última alocución en los patios del edificio sagrado. Este discurso estaba destinado a ser su último llamamiento al pueblo judío y la acusación final contra sus vehementes enemigos que trataban de destruirlo: los escribas, los fariseos, los saduceos y los dirigentes principales de Israel. A lo largo de toda la mañana, los diversos grupos habían tenido la oportunidad de hacerle preguntas a Jesús; esta tarde, nadie le preguntó nada.

\par 
%\textsuperscript{(1905.2)}
\textsuperscript{175:0.2} Cuando el Maestro empezó a hablar, el patio del templo estaba tranquilo y en orden. Los cambistas y los mercaderes no se habían atrevido a entrar de nuevo en el templo desde que Jesús y la multitud excitada los habían echado el día anterior. Antes de empezar su discurso, Jesús miró con ternura a este auditorio que pronto iba a escuchar su alocución pública de despedida, su mensaje de misericordia para la humanidad, unido a su última denuncia de los falsos educadores y de los fanáticos dirigentes de los judíos.

\section*{1. El discurso}
\par 
%\textsuperscript{(1905.3)}
\textsuperscript{175:1.1} «He estado con vosotros durante mucho tiempo, recorriendo el país de un lado a otro, y proclamando el amor del Padre por los hijos de los hombres. Muchos han visto la luz y han entrado, por la fe, en el reino de los cielos. En conexión con esta enseñanza y esta predicación, el Padre ha realizado muchas obras maravillosas, llegando incluso a resucitar a los muertos. Muchos enfermos y afligidos han recuperado la salud porque creían; pero toda esta proclamación de la verdad y esta curación de enfermedades no ha abierto los ojos a aquellos que se niegan a ver la luz, a aquellos que están decididos a rechazar este evangelio del reino».

\par 
%\textsuperscript{(1905.4)}
\textsuperscript{175:1.2} «De todas las maneras compatibles con la realización de la voluntad de mi Padre, mis apóstoles y yo hemos hecho todo lo posible por vivir en paz con nuestros hermanos, por cumplir con las exigencias razonables de las leyes de Moisés y de las tradiciones de Israel. Hemos buscado la paz constantemente, pero los dirigentes de Israel no la quieren. Al rechazar la verdad de Dios y la luz del cielo, se alinean al lado del error y de las tinieblas. No puede haber paz entre la luz y las tinieblas, entre la vida y la muerte, entre la verdad y el error».

\par 
%\textsuperscript{(1905.5)}
\textsuperscript{175:1.3} «Muchos de vosotros os habéis atrevido a creer en mis enseñanzas y ya habéis entrado en la alegría y la libertad de la conciencia de la filiación con Dios. Y daréis testimonio de que he ofrecido esta misma filiación con Dios a toda la nación judía, incluso a esos mismos hombres que ahora tratan de destruirme. Incluso ahora, mi Padre recibiría a esos educadores ciegos y a esos dirigentes hipócritas, sólo con que se volvieran hacia él y aceptaran su misericordia. Incluso ahora no es demasiado tarde para que esta gente reciba la palabra del cielo y acoja con agrado al Hijo del Hombre».

\par 
%\textsuperscript{(1906.1)}
\textsuperscript{175:1.4} «Mi Padre ha tratado a este pueblo con misericordia durante mucho tiempo. Generación tras generación, hemos enviado a nuestros profetas para enseñarles y advertirles, y generación tras generación, han matado a estos instructores enviados por el cielo. Y ahora, vuestros altos sacerdotes obstinados y vuestros dirigentes testarudos continúan haciendo exactamente lo mismo. Del mismo modo que Herodes ha provocado la muerte de Juan, vosotros también os preparáis ahora para destruir al Hijo del Hombre».

\par 
%\textsuperscript{(1906.2)}
\textsuperscript{175:1.5} «Mientras exista una posibilidad de que los judíos se vuelvan hacia mi Padre y busquen la salvación, el Dios de Abraham, de Isaac y de Jacob mantendrá extendidas sus manos misericordiosas hacia vosotros; pero una vez que hayáis llenado vuestra copa de impenitencia, y una vez que hayáis rechazado finalmente la misericordia de mi Padre, esta nación será abandonada a sí misma y llegará rápidamente a un final ignominioso. Este pueblo estaba destinado a convertirse en la luz del mundo, a mostrar la gloria espiritual de una raza que conocía a Dios, pero os habéis desviado tanto de la realización de vuestros privilegios divinos, que vuestros dirigentes están a punto de cometer la locura suprema de todos los tiempos, en el sentido de que están a punto de rechazar finalmente el don de Dios a todos los hombres y para todos los tiempos ---la revelación del amor del Padre que está en los cielos por todas sus criaturas de la Tierra».

\par 
%\textsuperscript{(1906.3)}
\textsuperscript{175:1.6} «Una vez que hayáis rechazado esta revelación de Dios al hombre, el reino de los cielos será entregado a otros pueblos, a aquellos que lo reciban con alegría y felicidad. En nombre del Padre que me ha enviado, os advierto solemnemente que estáis a punto de perder vuestra posición en el mundo como portaestandartes de la verdad eterna y custodios de la ley divina. En este momento os ofrezco vuestra última oportunidad de adelantaros y arrepentiros, para anunciar vuestra intención de buscar a Dios con todo vuestro corazón y entrar, como niños pequeños y con una fe sincera, en la seguridad y la salvación del reino de los cielos».

\par 
%\textsuperscript{(1906.4)}
\textsuperscript{175:1.7} «Mi Padre ha trabajado durante mucho tiempo por vuestra salvación, y yo he descendido para vivir entre vosotros y mostraros personalmente el camino. Muchos judíos y samaritanos, e incluso los gentiles, han creído en el evangelio del reino, pero los que deberían ser los primeros en adelantarse para aceptar la luz del cielo, se han negado resueltamente a creer en la revelación de la verdad de Dios ---Dios revelado en el hombre y el hombre elevado a Dios».

\par 
%\textsuperscript{(1906.5)}
\textsuperscript{175:1.8} «Esta tarde, mis apóstoles están aquí delante de vosotros en silencio, pero pronto escucharéis sus voces anunciando la llamada a la salvación y la incitación a unirse al reino celestial como hijos del Dios vivo. Y ahora, tomo por testigos a mis discípulos y a los creyentes en el evangelio del reino, así como a los mensajeros invisibles que están a su lado, de que he ofrecido una vez más, a Israel y a sus dirigentes, la liberación y la salvación. Pero todos observáis que la misericordia del Padre es despreciada y que los mensajeros de la verdad son rechazados. Sin embargo, os advierto que esos escribas y fariseos aún están sentados en el puesto de Moisés; por lo tanto, hasta que los Altísimos que gobiernan en los reinos de los hombres no hayan demolido finalmente esta nación y destruido el lugar donde se encuentran sus dirigentes, os pido que cooperéis con esos ancianos de Israel. No es necesario que os unáis a ellos en sus planes para destruir al Hijo del Hombre, pero en todo lo relacionado con la paz de Israel, debéis someteros a ellos. En todas esas cuestiones, haced todo lo que os ordenen y guardad lo esencial de la ley, pero no imitéis sus malas acciones. Recordad que éste es el pecado de esos gobernantes: Dicen lo que es bueno, pero no lo hacen. Sabéis bien que esos dirigentes echan sobre vuestros hombros unas cargas pesadas, unas cargas penosas de llevar, y que no levantarán ni un solo dedo para ayudaros a llevar esas pesadas cargas. Os han oprimido con ceremonias y esclavizado con tradiciones».

\par 
%\textsuperscript{(1907.1)}
\textsuperscript{175:1.9} «Además, a esos dirigentes egocéntricos les deleita hacer sus buenas obras de manera que puedan ser vistos por los hombres. Agrandan sus filacterias y ensanchan los bordes de sus vestidos oficiales. Anhelan los sitios principales en los banquetes y exigen los asientos de honor en las sinagogas. Codician los saludos elogiosos en las plazas públicas y desean que todos los hombres los llamen rabinos. Y mientras buscan ser honrados así por los hombres, se apoderan en secreto de las casas de las viudas y sacan provecho de los servicios del templo sagrado. Esos hipócritas simulan hacer largas oraciones en público, y dan limosnas para atraer la atención de sus semejantes».

\par 
%\textsuperscript{(1907.2)}
\textsuperscript{175:1.10} «Aunque debéis honrar a vuestros dirigentes y respetar a vuestros educadores, no debéis llamar Padre a ningún hombre en el sentido espiritual, porque uno solo es vuestro Padre, y ese es Dios. No tratéis tampoco de dominar a vuestros hermanos en el reino. Recordad que os he enseñado que aquel que quiera ser el más grande entre vosotros, debe convertirse en el servidor de todos. Si os atrevéis a exaltaros delante de Dios, sin duda seréis humillados; pero aquel que se humilla sinceramente, será exaltado con toda seguridad. En vuestra vida diaria, no busquéis vuestra propia glorificación, sino la gloria de Dios. Someted inteligentemente vuestra propia voluntad a la voluntad del Padre que está en los cielos».

\par 
%\textsuperscript{(1907.3)}
\textsuperscript{175:1.11} «No interpretéis mal mis palabras. No albergo ninguna mala intención hacia esos jefes de los sacerdotes y los dirigentes que en este mismo momento intentan destruirme; no tengo ninguna aversión contra esos escribas y fariseos que rechazan mis enseñanzas. Sé que muchos de vosotros creéis en secreto, y sé que confesaréis abiertamente vuestra lealtad hacia el reino cuando llegue mi hora. Pero, ¿cómo se justificarán vuestros rabinos, que declaran hablar con Dios y luego se atreven a rechazar y destruir a aquel que viene a revelar el Padre a los mundos?»

\par 
%\textsuperscript{(1907.4)}
\textsuperscript{175:1.12} «¡Ay de vosotros, escribas y fariseos, hipócritas! Quisierais cerrar las puertas del reino de los cielos a los hombres sinceros, sólo porque ignoran los caminos de vuestra enseñanza. Os negáis a entrar en el reino, y al mismo tiempo hacéis todo lo que podéis para impedir que entren todos los demás. Permanecéis de espaldas a las puertas de la salvación, y lucháis contra todos los que quieren entrar».

\par 
%\textsuperscript{(1907.5)}
\textsuperscript{175:1.13} «¡Ay de vosotros, escribas y fariseos, tan hipócritas como sois! Porque recorréis en verdad la tierra y el mar para hacer un prosélito, y cuando lo habéis conseguido, no os sentís satisfechos hasta hacerlo dos veces peor de lo que era como hijo de los paganos».

\par 
%\textsuperscript{(1907.6)}
\textsuperscript{175:1.14} «¡Ay de vosotros, sacerdotes principales y dirigentes, que os adueñáis de los bienes de los pobres y exigís impuestos opresivos a los que quieren servir a Dios como creen que Moisés lo ordenó! Vosotros, que os negáis a mostrar misericordia, ¿podéis esperar misericordia en los mundos venideros?»

\par 
%\textsuperscript{(1907.7)}
\textsuperscript{175:1.15} «¡Ay de vosotros, falsos educadores y guías ciegos! ¿Qué se puede esperar de una nación cuando los ciegos conducen a los ciegos? Los dos tropezarán y caerán al abismo de la destrucción».

\par 
%\textsuperscript{(1907.8)}
\textsuperscript{175:1.16} «¡Ay de vosotros que disimuláis cuando prestáis juramento! Sois unos tramposos, porque enseñáis que un hombre puede jurar por el templo y violar su juramento; pero que si cualquiera jura por el oro del templo, debe permanecer atado a su juramento. Todos sois necios y ciegos. Ni siquiera sois consistentes en vuestra deshonestidad, porque, ¿que es más grande, el oro o el templo que supuestamente ha santificado al oro? También enseñáis que si un hombre jura por el altar, no significa nada; pero que si alguien jura por la ofrenda que está en el altar, entonces será tenido por deudor. De nuevo estáis ciegos ante la verdad, porque ¿qué es más grande, la ofrenda o el altar que santifica la ofrenda? ¿Cómo podéis justificar una hipocresía y una deshonestidad semejantes a los ojos del Dios del cielo?»

\par 
%\textsuperscript{(1908.1)}
\textsuperscript{175:1.17} «¡Ay de vosotros, escribas y fariseos, y todos los demás hipócritas, que os aseguráis de pagar el diezmo de la menta, el anís y el comino, y al mismo tiempo descuidáis los asuntos más importantes de la ley ---la fe, la misericordia y el juicio! Dentro de lo razonable, deberíais hacer lo primero sin dejar de hacer lo segundo. Sois realmente unos guías ciegos y unos educadores estúpidos; filtráis los mosquitos y os tragáis los camellos».

\par 
%\textsuperscript{(1908.2)}
\textsuperscript{175:1.18} «¡Ay de vosotros, escribas, fariseos e hipócritas! pues limpiáis escrupulosamente el exterior de la copa y del plato, pero dentro permanece la inmundicia de la extorsión, los excesos y el engaño. Estáis espiritualmente ciegos. ¿No reconocéis que sería mucho mejor limpiar primero el interior de la copa, y luego lo que rebosa limpiaría por sí mismo el exterior? ¡Réprobos perversos! Ejecutáis los actos exteriores de vuestra religión para cumplir literalmente con vuestra interpretación de la ley de Moisés, mientras que vuestras almas están impregnadas de iniquidad y llenas de intenciones asesinas».

\par 
%\textsuperscript{(1908.3)}
\textsuperscript{175:1.19} «¡Ay de todos vosotros que rechazáis la verdad y despreciáis la misericordia! Muchos de vosotros os parecéis a los sepulcros blanqueados, que aparecen hermosos por fuera, pero por dentro están llenos de huesos de muertos y de todo tipo de impurezas. Así es como vosotros, que rechazáis a sabiendas el consejo de Dios, aparecéis exteriormente ante los hombres como santos y rectos, pero por dentro vuestro corazón está lleno de hipocresía y de iniquidad».

\par 
%\textsuperscript{(1908.4)}
\textsuperscript{175:1.20} «¡Ay de vosotros, guías falsos de una nación! Habéis construido allí un monumento a los antiguos profetas martirizados, mientras conspiráis para destruir a Aquel de quien ellos hablaban. Adornáis las tumbas de los justos y presumís de que si hubierais vivido en la época de vuestros padres, no hubierais matado a los profetas; y luego, a pesar de este pensamiento presuntuoso, os preparáis para asesinar a aquel de quien hablaban los profetas: el Hijo del Hombre. En vista de que hacéis estas cosas, testificáis contra vosotros mismos de que sois los hijos perversos de aquellos que mataron a los profetas. ¡Continuad pues, y llenad hasta el borde la copa de vuestra condenación!»

\par 
%\textsuperscript{(1908.5)}
\textsuperscript{175:1.21} «¡Ay de vosotros, hijos del mal! Juan os llamó con razón los hijos de las víboras, y yo os pregunto: ¿cómo podéis escapar al juicio que Juan pronunció contra vosotros?»

\par 
%\textsuperscript{(1908.6)}
\textsuperscript{175:1.22} «Pero incluso ahora os ofrezco, en nombre de mi Padre, la misericordia y el perdón; incluso ahora os tiendo la mano amorosa de la hermandad eterna. Mi Padre os ha enviado a los sabios y a los profetas; habéis perseguido a unos y habéis matado a los otros. Luego apareció Juan, proclamando la llegada del Hijo del Hombre, y lo destruisteis después de que muchos hubieran creído en sus enseñanzas. Y ahora os preparáis para derramar más sangre inocente. ¿No comprendéis que llegará un día terrible de rendición de cuentas, cuando el Juez de toda la Tierra exija a este pueblo que explique por qué ha rechazado, perseguido y destruido a estos mensajeros del cielo? ¿No comprendéis que debéis rendir cuentas por toda esta sangre justa, desde el primer profeta asesinado hasta la época de Zacarías, a quien le quitaron la vida entre el santuario y el altar? Si continuáis por ese camino perverso, quizás esta rendición de cuentas le sea requerida a esta misma generación».

\par 
%\textsuperscript{(1908.7)}
\textsuperscript{175:1.23} «¡Oh Jerusalén e hijos de Abraham, vosotros que habéis lapidado a los profetas y matado a los instructores que os fueron enviados, incluso ahora quisiera reunir a vuestros hijos como la gallina reúne a sus polluelos debajo de sus alas, pero no queréis!»

\par 
%\textsuperscript{(1908.8)}
\textsuperscript{175:1.24} «Y ahora me despido de vosotros. Habéis escuchado mi mensaje y habéis tomado vuestra decisión. Aquellos que han creído en mi evangelio ya están a salvo en el reino de Dios. A vosotros, que habéis escogido rechazar el regalo de Dios, os digo que no me veréis más enseñar en el templo. Mi trabajo a favor de vosotros ha terminado. ¡Mirad, ahora salgo con mis hijos, y os dejo vuestra casa desolada!»

\par 
%\textsuperscript{(1908.9)}
\textsuperscript{175:1.25} A continuación, el Maestro hizo señas a sus seguidores para que salieran del templo.

\section*{2. La condición de los judíos}
\par 
%\textsuperscript{(1909.1)}
\textsuperscript{175:2.1} El hecho de que los dirigentes espirituales y los educadores religiosos de la nación judía rechazaran en otra época las enseñanzas de Jesús y conspiraran para provocar su muerte cruel, no afecta para nada a la situación de cada judío en su posición ante Dios. Este hecho no debería incitar a los que afirman ser seguidores de Cristo a tener prejuicios contra los judíos como compañeros mortales. Los judíos como nación y como grupo sociopolítico pagaron plenamente el precio terrible de rechazar al Príncipe de la Paz. Hace mucho tiempo que dejaron de ser los portadores espirituales de la verdad divina para las razas de la humanidad, pero esto no constituye una razón válida para que los descendientes individuales de aquellos antiguos judíos tengan que sufrir las persecuciones que les han infligido algunos seguidores declarados, intolerantes, indignos y fanáticos de Jesús de Nazaret, el cual era también judío de nacimiento.

\par 
%\textsuperscript{(1909.2)}
\textsuperscript{175:2.2} Este odio y esta persecución irrazonables, tan diferentes al espíritu de Cristo, contra los judíos modernos, ha terminado muchas veces en el sufrimiento y la muerte de algún judío inocente e inofensivo, cuyos mismos antepasados, en los tiempos de Jesús, habían aceptado sinceramente su evangelio y luego murieron sin vacilar por aquella verdad en la que creían de todo corazón. ¡Qué estremecimiento de horror recorre a los seres celestiales espectadores, cuando contemplan a los seguidores declarados de Jesús entregarse a perseguir, acosar e incluso asesinar a los descendientes actuales de Pedro, Felipe y Mateo, y de otros judíos palestinos que dieron sus vidas tan gloriosamente como primeros mártires del evangelio del reino de los cielos!

\par 
%\textsuperscript{(1909.3)}
\textsuperscript{175:2.3} ¡Cuán cruel e irrazonable es obligar a unos niños inocentes a que sufran por los pecados de sus progenitores, por unos delitos que ignoran por completo, de los que no pueden ser responsables de ninguna manera! ¡Y llevar a cabo estas acciones perversas en nombre de aquel que enseñó a sus discípulos a que amaran incluso a sus enemigos! En este relato de la vida de Jesús, ha sido necesario describir la manera en que algunos de sus compatriotas judíos lo rechazaron y conspiraron para provocar su muerte ignominiosa; pero queremos advertir a todos los que lean esta narración que la presentación de este relato histórico no justifica de ninguna manera el odio injusto, ni perdona la actitud mental sin equidad, que tantos cristianos declarados han mantenido durante muchos siglos contra personas judías. Los creyentes en el reino, los que siguen las enseñanzas de Jesús, deben dejar de maltratar al judío individual como alguien culpable del rechazo y de la crucifixión de Jesús. El Padre y su Hijo Creador nunca han dejado de amar a los judíos. Dios no hace acepción de personas, y la salvación es tanto para los judíos como para los gentiles.

\section*{3. La nefasta reunión del sanedrín}
\par 
%\textsuperscript{(1909.4)}
\textsuperscript{175:3.1} La nefasta reunión del sanedrín fue convocada para este martes a las ocho de la noche. En muchas ocasiones anteriores, este tribunal supremo de la nación judía había decretado de manera no oficial la muerte de Jesús. Este augusto cuerpo gobernante había decidido muchas veces poner fin a la obra del Maestro, pero nunca antes había resuelto arrestarlo y provocar su muerte a toda costa. Poco antes de la medianoche de este martes 4 de abril del año 30, el sanedrín, tal como estaba constituido en ese momento, votó oficialmente y por \textit{unanimidad} imponer la sentencia de muerte tanto a Jesús como a Lázaro. Ésta fue la respuesta al último llamamiento del Maestro a los dirigentes de los judíos, un llamamiento que había hecho en el templo tan sólo unas horas antes; esta respuesta representaba su reacción de amargo resentimiento hacia Jesús por su última y vigorosa acusación contra estos mismos sacerdotes principales, y saduceos y fariseos impenitentes. La condena a muerte del Hijo de Dios (incluso antes de su juicio) fue la contestación del sanedrín a la última oferta de misericordia celestial que jamás fuera concedida a la nación judía como tal.

\par 
%\textsuperscript{(1910.1)}
\textsuperscript{175:3.2} A partir de aquel momento, los judíos fueron dejados solos para que terminaran su breve y corto período de vida nacional, en total acuerdo con su posición puramente humana entre las naciones de Urantia. Israel había repudiado al Hijo del Dios que había hecho una alianza con Abraham; y el plan de que los hijos de Abraham fueran los portadores de la luz de la verdad en el mundo se había hecho añicos. La alianza divina se había anulado, y el final de la nación hebrea se aproximaba rápidamente.

\par 
%\textsuperscript{(1910.2)}
\textsuperscript{175:3.3} Los funcionarios del sanedrín recibieron la orden de arrestar a Jesús a la mañana siguiente temprano, pero con las instrucciones de que no debía ser apresado en público. Les dijeron que planearan arrestarlo en secreto, preferiblemente de manera repentina y de noche. Comprendiendo que quizás aquel día (miércoles) no regresaría para enseñar en el templo, indicaron a aquellos oficiales del sanedrín que «lo trajeran ante el alto tribunal judío en cualquier momento antes de la medianoche del jueves».

\section*{4. La situación en Jerusalén}
\par 
%\textsuperscript{(1910.3)}
\textsuperscript{175:4.1} Al final del último discurso de Jesús en el templo, los apóstoles se quedaron una vez más confundidos y consternados. Judas había regresado al templo antes de que el Maestro empezara su terrible acusación contra los dirigentes judíos, de manera que los doce al completo escucharon la segunda mitad del último discurso de Jesús en el templo. Es una pena que Judas Iscariote no pudiera escuchar la primera mitad de este discurso de despedida, que ofrecía la misericordia. No escuchó esta última oferta de misericordia a los dirigentes judíos porque aún estaba conferenciando con un grupo de parientes y amigos saduceos con quienes había almorzado, y con quienes estaba conversando sobre la manera más adecuada de separarse de Jesús y de sus compañeros apóstoles. Mientras escuchaba la acusación final del Maestro contra los jefes y dirigentes judíos, Judas tomó su decisión final y completa de abandonar el movimiento evangélico y de lavarse las manos de toda esta empresa. Sin embargo, salió del templo en compañía de los doce y se dirigió con ellos al Monte de los Olivos, donde escuchó, con sus compañeros apóstoles, el discurso fatídico sobre la destrucción de Jerusalén y el final de la nación judía. Aquella noche del martes, Judas permaneció con ellos en el nuevo campamento cerca de Getsemaní.

\par 
%\textsuperscript{(1910.4)}
\textsuperscript{175:4.2} La multitud se quedó atónita y desconcertada cuando escuchó a Jesús pasar de su llamamiento misericordioso a los dirigentes judíos, a una reprimenda repentina y mordaz que rozaba la acusación sin piedad. Aquella noche, mientras el sanedrín pronunciaba la sentencia de muerte contra Jesús, y el Maestro estaba sentado con sus apóstoles y algunos de sus discípulos en el Monte de los Olivos, prediciendo la muerte de la nación judía, todo Jerusalén se había entregado a la discusión seria y callada de una sola pregunta: «¿Qué van a hacer con Jesús?»

\par 
%\textsuperscript{(1910.5)}
\textsuperscript{175:4.3} En la casa de Nicodemo, más de treinta judíos eminentes que creían en secreto en el reino se reunieron para debatir la conducta a seguir en el caso de que se produjera una ruptura abierta con el sanedrín. Todos los presentes acordaron que reconocerían abiertamente su lealtad al Maestro en cuanto se enteraran de su arresto. Y eso fue exactamente lo que hicieron.

\par 
%\textsuperscript{(1911.1)}
\textsuperscript{175:4.4} Los saduceos, que ahora controlaban y dominaban el sanedrín, deseaban eliminar a Jesús por las razones siguientes:

\par 
%\textsuperscript{(1911.2)}
\textsuperscript{175:4.5} 1. Temían que el creciente favor popular con que la multitud consideraba a Jesús amenazara con poner en peligro la existencia de la nación judía, debido a una posible complicación con las autoridades romanas.

\par 
%\textsuperscript{(1911.3)}
\textsuperscript{175:4.6} 2. El ardor de Jesús por la reforma del templo afectaba directamente a sus ingresos; la depuración del templo perjudicaba sus bolsillos.

\par 
%\textsuperscript{(1911.4)}
\textsuperscript{175:4.7} 3. Se sentían responsables de la preservación del orden social, y temían las consecuencias de una expansión posterior de la nueva y extraña doctrina de Jesús sobre la fraternidad de los hombres.

\par 
%\textsuperscript{(1911.5)}
\textsuperscript{175:4.8} Los fariseos tenían unos motivos diferentes para desear quitarle la vida a Jesús. Le tenían miedo porque:

\par 
%\textsuperscript{(1911.6)}
\textsuperscript{175:4.9} 1. Se había opuesto eficazmente al dominio tradicional que los fariseos ejercían sobre el pueblo. Los fariseos eran ultraconservadores, y les encolerizaba amargamente estos ataques, supuestamente radicales, contra su prestigio establecido como instructores religiosos.

\par 
%\textsuperscript{(1911.7)}
\textsuperscript{175:4.10} 2. Sostenían que Jesús violaba la ley; que había mostrado un desprecio total por el sábado y por otras numerosas exigencias legales y ceremoniales.

\par 
%\textsuperscript{(1911.8)}
\textsuperscript{175:4.11} 3. Lo acusaban de blasfemia porque se refería a Dios como si fuera su Padre.

\par 
%\textsuperscript{(1911.9)}
\textsuperscript{175:4.12} 4. Y ahora estaban profundamente irritados contra él a causa del último discurso que había pronunciado aquel día en el templo, donde en la parte final de su alocución de despedida los acusaba severamente.

\par 
%\textsuperscript{(1911.10)}
\textsuperscript{175:4.13} Una vez que hubo decretado oficialmente la muerte de Jesús y dado las órdenes para su arresto, el sanedrín levantó la sesión este martes cerca de la medianoche, después de acordar una reunión para las diez de la mañana del día siguiente en la casa del sumo sacerdote Caifás, con el fin de formular las acusaciones que permitirían llevar a Jesús a juicio.

\par 
%\textsuperscript{(1911.11)}
\textsuperscript{175:4.14} Un pequeño grupo de saduceos había llegado a proponer que se deshicieran de Jesús mediante el asesinato, pero los fariseos se negaron terminantemente a apoyar este procedimiento.

\par 
%\textsuperscript{(1911.12)}
\textsuperscript{175:4.15} Ésta era la situación en Jerusalén y entre los hombres en este día lleno de acontecimientos, mientras una enorme multitud de seres celestiales se cernía sobre esta importante escena en la Tierra, impacientes por hacer algo para ayudar a su amado Soberano, pero sin poder actuar porque los superiores que los dirigían se lo habían prohibido expresamente.