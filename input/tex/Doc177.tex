\chapter{Documento 177. El miércoles, día de descanso}
\par 
%\textsuperscript{(1920.1)}
\textsuperscript{177:0.1} CUANDO la tarea de enseñar al pueblo no les apremiaba, Jesús y sus apóstoles tenían la costumbre de descansar de sus trabajos todos los miércoles. Este miércoles en particular tomaron el desayuno un poco más tarde que de costumbre, y el campamento estaba impregnado de un silencio de mal ag\"uero; se dijeron muy pocas palabras durante la primera mitad de esta comida matutina. Por fin, Jesús habló: «Deseo que descanséis hoy. Dedicad tiempo para reflexionar sobre todo lo que ha sucedido desde que llegamos a Jerusalén, y para meditar en lo que se avecina, de todo lo cual os he informado claramente. Aseguraos de que la verdad permanece en vuestra vida, y de que crecéis diariamente en la gracia».

\par 
%\textsuperscript{(1920.2)}
\textsuperscript{177:0.2} Después del desayuno, el Maestro informó a Andrés que tenía la intención de ausentarse durante todo el día, y sugirió que se autorizara a los apóstoles para que pasaran el tiempo según sus propios deseos, excepto que no debían, en ninguna circunstancia, atravesar las puertas de Jerusalén.

\par 
%\textsuperscript{(1920.3)}
\textsuperscript{177:0.3} Cuando Jesús se preparó para partir solo hacia las colinas, David Zebedeo se le acercó diciendo: «Maestro, sabes bien que los fariseos y los dirigentes intentan destruirte, y sin embargo te preparas para salir solo hacia las colinas. Eso es una locura; por ello, enviaré a tres hombres contigo, bien preparados para que vigilen que no te suceda nada malo». Jesús miró a los tres galileos robustos y bien armados, y dijo a David: «Tu intención es buena, pero te equivocas en el sentido de que no logras comprender que el Hijo del Hombre no necesita a nadie que lo defienda. Nadie me pondrá la mano encima hasta el momento en que esté preparado para abandonar mi vida de acuerdo con la voluntad de mi Padre. Estos hombres no pueden acompañarme. Deseo ir solo, para poder comulgar con el Padre».

\par 
%\textsuperscript{(1920.4)}
\textsuperscript{177:0.4} Al escuchar estas palabras, David y sus guardianes armados se retiraron; pero mientras Jesús partía solo, Juan Marcos se adelantó con una pequeña cesta que contenía alimentos y agua, y sugirió que si Jesús tenía la intención de estar fuera todo el día, podría tener hambre. El Maestro le sonrió a Juan y bajó la mano para coger la cesta.

\section*{1. Un día a solas con Dios}
\par 
%\textsuperscript{(1920.5)}
\textsuperscript{177:1.1} Cuando Jesús estaba a punto de coger la cesta del almuerzo de las manos de Juan, el joven se aventuró a decir: «Pero, Maestro, quizás dejes la cesta en el suelo mientras te alejas para orar y te vayas sin ella. Además, si te acompaño para llevar el almuerzo, estarás más libre para adorar, y permaneceré callado con toda seguridad. No haré ninguna pregunta, y me quedaré con la cesta cuando te apartes para orar a solas».

\par 
%\textsuperscript{(1920.6)}
\textsuperscript{177:1.2} Mientras daba este discurso, cuya temeridad sorprendió a algunos oyentes que se encontraban cerca, Juan tuvo la audacia de retener la cesta. Allí estaban los dos, Juan y Jesús, agarrados a la cesta. Enseguida el Maestro la soltó, bajó la mirada hacia el muchacho, y le dijo: «Puesto que anhelas acompañarme con todo tu corazón, no te será negado. Nos marcharemos juntos y tendremos una buena conversación. Podrás hacerme todas las preguntas que surjan en tu corazón, y nos confortaremos y nos consolaremos mutuamente. Puedes empezar llevando el almuerzo, y cuando te canses, te ayudaré. Sígueme pues».

\par 
%\textsuperscript{(1921.1)}
\textsuperscript{177:1.3} Aquella noche, Jesús no regresó al campamento hasta después de la puesta del Sol. El Maestro pasó este último día de tranquilidad en la Tierra charlando con este joven hambriento de verdad, y hablando con su Padre Paradisiaco. Este acontecimiento se conoce en las alturas como «el día que un joven pasó con Dios en las colinas». Este suceso ejemplifica para siempre la buena voluntad del Creador para fraternizar con la criatura. Hasta un adolescente, si el deseo de su corazón es realmente supremo, puede atraer la atención y disfrutar de la compañía amorosa del Dios de un universo, experimentar realmente el éxtasis inolvidable de estar a solas con Dios en las colinas, y todo ello durante un día entero. Y ésta fue la extraordinaria experiencia de Juan Marcos durante este miércoles en las colinas de Judea.

\par 
%\textsuperscript{(1921.2)}
\textsuperscript{177:1.4} Jesús charló mucho con Juan, y habló libremente sobre los asuntos de este mundo y del siguiente. Juan le dijo a Jesús que lamentaba mucho no haber tenido la edad suficiente para ser uno de los apóstoles, y expresó su gran reconocimiento porque se le había permitido seguir al grupo apostólico desde su primera predicación en el vado del Jordán cerca de Jericó, a excepción del viaje a Fenicia. Jesús le advirtió al joven que no se desanimara por los acontecimientos inminentes, y le aseguró que viviría para convertirse en un poderoso mensajero del reino.

\par 
%\textsuperscript{(1921.3)}
\textsuperscript{177:1.5} Juan Marcos estaba emocionado por el recuerdo de este día con Jesús en las colinas, pero nunca olvidó la recomendación final del Maestro. Cuando estaban a punto de regresar al campamento de Getsemaní, Jesús le dijo: «Bien, Juan, hemos tenido una buena conversación, un verdadero día de descanso, pero procura no contarle a nadie las cosas que te he dicho». Y Juan Marcos nunca reveló nada de lo que había sucedido este día que pasó con Jesús en las colinas.

\par 
%\textsuperscript{(1921.4)}
\textsuperscript{177:1.6} Durante las pocas horas que le quedaban a Jesús por vivir en la Tierra, Juan Marcos nunca dejó que el Maestro estuviera lejos de su vista durante mucho tiempo. El muchacho siempre estaba oculto cerca de él; sólo durmió cuando Jesús dormía.

\section*{2. La infancia en el hogar}
\par 
%\textsuperscript{(1921.5)}
\textsuperscript{177:2.1} En el transcurso de las conversaciones de este día con Juan Marcos, Jesús pasó bastante tiempo comparando sus experiencias de la infancia y de la adolescencia. Aunque los padres de Juan poseían más bienes terrenales que los padres de Jesús, habían tenido en su niñez muchas experiencias muy similares. Jesús dijo muchas cosas que ayudaron a Juan a comprender mejor a sus padres y a otros miembros de su familia. Cuando el muchacho le preguntó al Maestro cómo podía saber que se convertiría en un «poderoso mensajero del reino», Jesús dijo:

\par 
%\textsuperscript{(1921.6)}
\textsuperscript{177:2.2} «Sé que te mostrarás fiel al evangelio del reino, porque puedo contar con la fe y el amor que tienes ahora, ya que estas cualidades están basadas en una educación tan temprana como la que has recibido en el hogar. Eres el producto de un hogar donde los padres se tienen un afecto mutuo y sincero, por lo que no has sido amado con exceso como para exaltar perjudicialmente tu concepto de tu propia importancia. Tu personalidad tampoco ha sufrido una deformación a consecuencia de unas maniobras sin amor efectuadas por tus padres, enfrentados el uno contra el otro para ganar tu confianza y tu lealtad. Has disfrutado de ese amor parental que asegura una loable confianza en sí mismo y que fomenta unos sentimientos normales de seguridad. Pero también has tenido la suerte de que tus padres poseyeran sabiduría al mismo tiempo que amor; fue la sabiduría la que les condujo a negarte la mayoría de las satisfacciones y de los múltiples lujos que se pueden comprar con la riqueza; te enviaron a la escuela de la sinagoga con tus compañeros de juego de la vecindad, y también te animaron a aprender la manera de vivir en este mundo permitiéndote efectuar una experiencia original. Viniste con tu joven amigo Amós al Jordán, donde nosotros predicábamos y los discípulos de Juan bautizaban. Los dos deseabais acompañarnos. Cuando regresasteis a Jerusalén, tus padres dieron su consentimiento; los padres de Amós se negaron; amaban tanto a su hijo que le negaron la experiencia bendita que tú has tenido, incluida la que hoy estás disfrutando. Amós podría haberse escapado de su casa para unirse a nosotros, pero si lo hubiera hecho, habría herido el amor y sacrificado la fidelidad. Aunque esta conducta hubiera sido sabia, hubiera pagado un precio terrible por la experiencia, la independencia y la libertad. Los padres sabios, como los tuyos, procuran que sus hijos no tengan que herir el amor o ahogar la fidelidad para desarrollar su independencia y disfrutar de una libertad vigorizante cuando han llegado a tu edad».

\par 
%\textsuperscript{(1922.1)}
\textsuperscript{177:2.3} «El amor, Juan, es la realidad suprema del universo cuando es otorgado por unos seres infinitamente sabios, pero presenta un rasgo peligroso y a veces semiegoísta tal como es manifestado en la experiencia de los padres mortales. Cuando te cases y tengas que criar tus propios hijos, asegúrate de que tu amor esté aconsejado por la sabiduría y guiado por la inteligencia».

\par 
%\textsuperscript{(1922.2)}
\textsuperscript{177:2.4} «Tu joven amigo Amós cree en este evangelio del reino tanto como tú, pero no puedo contar plenamente con él; no estoy seguro de lo que va a hacer en los años venideros. Su infancia en el hogar no se desarrolló como para producir una persona enteramente digna de confianza. Amós se parece demasiado a uno de mis apóstoles que no pudo disfrutar de una educación familiar normal, amorosa y sabia. Toda tu vida futura será más feliz y digna de confianza porque pasaste tus primeros ocho años en un hogar normal y bien regulado. Posees un carácter fuerte y bien integrado porque creciste en un hogar donde prevalecía el amor y reinaba la sabiduría. Este tipo de formación durante la infancia produce un tipo de fidelidad que me asegura que continuarás en el camino que has empezado».

\par 
%\textsuperscript{(1922.3)}
\textsuperscript{177:2.5} Durante más de una hora, Jesús y Juan continuaron esta conversación sobre la vida familiar. El Maestro siguió explicándole a Juan que un niño depende totalmente de sus padres y de la vida asociada en el hogar para formarse sus primeros conceptos sobre todas las cosas intelectuales, sociales, morales e incluso espirituales, puesto que la familia representa para el niño pequeño todo lo que puede conocer al principio sobre las relaciones humanas o divinas. El niño debe obtener, de los cuidados de su madre, sus primeras impresiones sobre el universo; depende totalmente de su padre terrenal para sus primeras ideas sobre el Padre celestial. La vida mental y emocional de los primeros años, condicionada por estas relaciones sociales y espirituales del hogar, determina si la vida posterior del niño será feliz o infeliz, fácil o difícil. Toda la vida de un ser humano está enormemente influida por lo que sucede durante los primeros años de la existencia.

\par 
%\textsuperscript{(1922.4)}
\textsuperscript{177:2.6} Creemos sinceramente que el evangelio contenido en las enseñanzas de Jesús, basado como lo está en la relación entre padre e hijo, difícilmente podrá disfrutar de una aceptación mundial hasta el momento en que la vida familiar de los pueblos modernos civilizados contenga más amor y más sabiduría. A pesar de que los padres del siglo veinte poseen un gran conocimiento y una mayor verdad para mejorar el hogar y ennoblecer la vida familiar, sigue siendo un hecho que para educar a los niños y a las niñas, muy pocos hogares modernos son tan buenos como el hogar de Jesús en Galilea y el de Juan Marcos en Judea; sin embargo, la aceptación del evangelio de Jesús tendrá como resultado una mejora inmediata de la vida familiar. La vida de amor de un hogar sabio y la devoción fiel a la verdadera religión ejercen una profunda influencia recíproca. Una vida hogareña así realza la religión, y la auténtica religión siempre glorifica el hogar.

\par 
%\textsuperscript{(1923.1)}
\textsuperscript{177:2.7} Es verdad que muchas influencias censurables atrofiadas y otras características restrictivas de estos antiguos hogares judíos han sido prácticamente eliminadas de muchos hogares modernos mejor organizados. Existe en verdad más independencia espontánea y mucha más libertad personal, pero esta libertad no está refrenada por el amor, motivada por la fidelidad, ni dirigida por la disciplina inteligente de la sabiduría. Mientras enseñemos al niño a rezar «Padre nuestro que estás en los cielos», todos los padres terrenales tendrán la inmensa responsabilidad de vivir y ordenar sus hogares de tal manera que la palabra \textit{padre} quede guardada dignamente en la mente y en el corazón de todos los niños que crecen.

\section*{3. El día en el campamento}
\par 
%\textsuperscript{(1923.2)}
\textsuperscript{177:3.1} Los apóstoles pasaron la mayor parte de este día caminando por el Monte de los Olivos y conversando con los discípulos que acampaban con ellos, pero al principio de la tarde sintieron el vivo deseo de ver regresar a Jesús. A medida que pasaba el día, se inquietaron cada vez más por su seguridad; se sentían inexpresablemente solos sin él. Durante todo el día estuvieron discutiendo sobre si deberían haberle permitido al Maestro partir solo hacia las colinas, acompañado solamente por el muchacho de los recados. Aunque nadie expresaba abiertamente sus pensamientos, no había ninguno de ellos, salvo Judas Iscariote, que no hubiera deseado estar en el lugar de Juan Marcos.

\par 
%\textsuperscript{(1923.3)}
\textsuperscript{177:3.2} Fue hacia mediados de la tarde cuando Natanael dio su discurso sobre el «Deseo supremo» a una media docena de apóstoles y a un número igual de discípulos, concluyendo de la manera siguiente: «En lo que estamos equivocados la mayoría de nosotros es en que somos poco entusiastas. No amamos al Maestro como él nos ama. Si todos hubiéramos querido ir con él tanto como Juan Marcos lo deseaba, seguramente nos hubiera llevado a todos. Nos quedamos mirando mientras el muchacho se acercaba al Maestro y le ofrecía la cesta, pero cuando el Maestro la cogió, el muchacho no la soltó. Por eso el Maestro nos dejó aquí mientras partía hacia las colinas con la cesta, el niño y todo».

\par 
%\textsuperscript{(1923.4)}
\textsuperscript{177:3.3} Hacia las cuatro, unos corredores llegaron hasta David Zebedeo trayéndole un mensaje de su madre en Betsaida y de la madre de Jesús. Varios días antes, David había llegado a la conclusión de que los jefes de los sacerdotes y los dirigentes iban a matar a Jesús. David sabía que estaban decididos a destruir al Maestro, y estaba casi convencido de que Jesús no ejercería su poder divino para salvarse, ni permitiría que sus seguidores emplearan la fuerza para defenderlo. Habiendo llegado a estas conclusiones, no tardó en enviar un mensajero a su madre, instándola a que viniera enseguida a Jerusalén y que trajera a María, la madre de Jesús, y a todos los miembros de su familia.

\par 
%\textsuperscript{(1923.5)}
\textsuperscript{177:3.4} La madre de David hizo lo que su hijo le pedía, y los corredores regresaron ahora hasta David trayendo la noticia de que su madre y toda la familia de Jesús estaban de camino hacia Jerusalén, y que llegarían tarde en cualquier momento del día siguiente, o muy temprano la mañana después. Puesto que David había hecho esto por su propia iniciativa, pensó que sería prudente guardarse esta información para sí mismo. Por lo tanto, no le dijo a nadie que la familia de Jesús estaba de camino hacia Jerusalén.

\par 
%\textsuperscript{(1924.1)}
\textsuperscript{177:3.5} Poco después del mediodía, más de veinte de los griegos que se habían encontrado con Jesús y los doce en la casa de José de Arimatea llegaron al campamento, y Pedro y Juan pasaron varias horas conversando con ellos. Estos griegos, o al menos algunos de ellos, tenían un buen conocimiento del reino, pues habían sido instruidos por Rodán en Alejandría.

\par 
%\textsuperscript{(1924.2)}
\textsuperscript{177:3.6} Aquella noche, después de regresar al campamento, Jesús conversó con los griegos, y habría ordenado a estos veinte hombres tal como había hecho con los setenta si no hubiera sido porque esta acción habría perturbado profundamente a sus apóstoles y a muchos de sus discípulos principales.

\par 
%\textsuperscript{(1924.3)}
\textsuperscript{177:3.7} Mientras todo esto sucedía en el campamento, en Jerusalén los jefes de los sacerdotes y los ancianos estaban sorprendidos de que Jesús no regresara para dirigir la palabra a las multitudes. Es verdad que el día anterior había dicho, al abandonar el templo: «Os dejo vuestra casa desolada». Pero no podían comprender por qué estaba dispuesto a renunciar a la gran ventaja que había conseguido con la actitud amistosa de las multitudes. Aunque temían que produjera un tumulto en el pueblo, las últimas palabras del Maestro a la multitud habían sido una exhortación a que se conformaran, de todas las maneras razonables, a la autoridad de aquellos «que estaban sentados en el puesto de Moisés». Pero aquel día estaban muy ocupados en la ciudad, preparándose simultáneamente para la Pascua y para perfeccionar sus planes de destruir a Jesús.

\par 
%\textsuperscript{(1924.4)}
\textsuperscript{177:3.8} Al campamento no acudió mucha gente, porque su ubicación se había mantenido como un secreto bien guardado por todos los que sabían que Jesús contaba con quedarse allí en lugar de dirigirse todas las noches a Betania.

\section*{4. Judas y los jefes de los sacerdotes}
\par 
%\textsuperscript{(1924.5)}
\textsuperscript{177:4.1} Poco después de que Jesús y Juan Marcos dejaran el campamento, Judas Iscariote desapareció del grupo de sus hermanos y no regresó hasta el final de la tarde. A pesar de la recomendación expresa de su Maestro de que no entraran en Jerusalén, este apóstol confundido y descontento se dirigió apresuradamente a su cita con los enemigos de Jesús, en la casa del sumo sacerdote Caifás. Se trataba de una reunión informal del sanedrín, fijada para poco después de las diez de aquella mañana. Esta reunión se había convocado para discutir la naturaleza de las acusaciones que se iban a presentar contra Jesús, y decidir el procedimiento a seguir para llevarlo ante las autoridades romanas a fin de conseguir la confirmación civil necesaria para la sentencia de muerte que ya habían decretado.

\par 
%\textsuperscript{(1924.6)}
\textsuperscript{177:4.2} El día anterior, Judas había revelado a algunos de sus parientes, y a ciertos amigos saduceos de la familia de su padre, que había llegado a la conclusión de que, aunque Jesús era un soñador y un idealista bien intencionado, no era el libertador esperado de Israel. Judas declaró que le gustaría mucho encontrar una manera airosa de retirarse de todo el movimiento. Sus amigos le aseguraron halagadoramente que su retirada sería saludada por los dirigentes judíos como un gran acontecimiento, y que podría lograr cualquier cosa. Le indujeron a creer que recibiría inmediatamente grandes honores del sanedrín, y que por fin se encontraría en condiciones de borrar el estigma de su «asociación bien intencionada, aunque desafortunada, con unos galileos ignorantes».

\par 
%\textsuperscript{(1924.7)}
\textsuperscript{177:4.3} Judas no podía creer del todo que las grandes obras del Maestro habían sido realizadas por el poder del príncipe de los demonios, pero ahora estaba plenamente convencido de que Jesús no ejercería su poder para engrandecerse; al final se había convencido de que Jesús se dejaría destruir por los dirigentes judíos, y no podía soportar la idea humillante de ser identificado con un movimiento condenado al fracaso. Se negaba a considerar la idea de un fracaso aparente. Comprendía plenamente el carácter firme de su Maestro y la agudeza de su mente majestuosa y misericordiosa, pero sin embargo le causaba placer aceptar, aunque fuera parcialmente, la sugerencia de uno de sus parientes de que Jesús, aunque fuera un fanático bien intencionado, es probable que no estuviera realmente bien de la cabeza; que siempre había parecido ser una persona extraña y mal comprendida.

\par 
%\textsuperscript{(1925.1)}
\textsuperscript{177:4.4} Y ahora más que nunca, Judas empezó a sentir un extraño resentimiento porque Jesús nunca le había asignado una posición más honorífica. Durante todo este tiempo había apreciado el honor de ser el tesorero apostólico, pero ahora empezaba a sentir que no era apreciado, que sus capacidades no eran reconocidas. Repentinamente se sintió dominado por la indignación porque Pedro, Santiago y Juan habían sido honrados con una asociación estrecha con Jesús, y en aquel momento, mientras se dirigía a la casa del sumo sacerdote, estaba más resuelto a desquitarse de Pedro, Santiago y Juan que a preocuparse por la idea de traicionar a Jesús. Pero por encima de todo, en aquel preciso momento, una nueva idea dominante empezó a ocupar el primer lugar en su mente consciente: Se había puesto en marcha para conseguir honores para sí mismo, y si podía asegurárselos al mismo tiempo que se desquitaba de los que habían contribuido a la mayor desilusión de su vida, mucho mejor. Cayó presa de una terrible confabulación de confusión, orgullo, desesperación y resolución. Así pues, debe quedar claro que no era por dinero por lo que Judas se dirigía en aquel momento hacia la casa de Caifás para preparar la traición a Jesús.

\par 
%\textsuperscript{(1925.2)}
\textsuperscript{177:4.5} Mientras Judas se acercaba a la casa de Caifás, tomó la decisión definitiva de abandonar a Jesús y a sus compañeros apóstoles; habiendo decidido dejar así la causa del reino de los cielos, estaba resuelto a asegurarse para sí mismo el máximo de honor y de gloria que había esperado recibir algún día, cuando se identificó por primera vez con Jesús y el nuevo evangelio del reino. Todos los apóstoles habían compartido alguna vez esta ambición con Judas, pero a medida que pasaba el tiempo habían aprendido a admirar la verdad y a amar a Jesús, al menos más que Judas.

\par 
%\textsuperscript{(1925.3)}
\textsuperscript{177:4.6} El traidor fue presentado a Caifás y a los dirigentes judíos por su primo. Éste explicó que Judas había descubierto el error que había cometido al dejarse engañar por la sutil enseñanza de Jesús, y había llegado a un punto en que deseaba renunciar pública y oficialmente a su asociación con el galileo; al mismo tiempo, pedía que se le restableciera en la confianza y la hermandad de sus hermanos judeos. El portavoz de Judas continuó explicando que Judas reconocía que sería mejor, para la paz de Israel, que Jesús fuera arrestado. Como demostración de su pesar por haber participado en este movimiento erróneo, y como prueba de la sinceridad de su presente regreso a las enseñanzas de Moisés, había venido para ofrecerse al sanedrín como alguien que podía colaborar con el capitán que tenía la orden de arrestar a Jesús, para que éste pudiera ser detenido discretamente, evitando así el peligro de excitar a las multitudes, o la necesidad de retrasar su arresto hasta después de la Pascua.

\par 
%\textsuperscript{(1925.4)}
\textsuperscript{177:4.7} Cuando hubo terminado de hablar, el primo presentó a Judas, el cual se acercó al sumo sacerdote, y dijo: «Haré todo lo que mi primo ha prometido, pero ¿qué estáis dispuestos a darme por este servicio?» Judas no pareció percibir la expresión de desdén, e incluso de disgusto, que cruzó por el rostro del insensible y vanidoso Caifás; el corazón de Judas estaba demasiado centrado en su propia glorificación y en el anhelo de satisfacer la exaltación de su ego.

\par 
%\textsuperscript{(1926.1)}
\textsuperscript{177:4.8} Caifás bajó entonces la mirada hacia el traidor mientras decía: «Judas, ve a ver al capitán de la guardia y ponte de acuerdo con ese oficial para traernos a tu Maestro esta noche o mañana por la noche. Y cuando nos lo hayas entregado, recibirás tu recompensa por este servicio». Cuando Judas escuchó esto, se retiró de la presencia de los sacerdotes y dirigentes principales, y fue a consultar con el capitán de los guardias del templo sobre la manera en que debían apresar a Jesús. Judas sabía que Jesús estaba entonces ausente del campamento, y no tenía ni idea de la hora en que volvería aquella noche, por lo que acordaron detener a Jesús a la noche siguiente (jueves), después de que el pueblo de Jerusalén y todos los peregrinos visitantes se hubieran retirado a descansar.

\par 
%\textsuperscript{(1926.2)}
\textsuperscript{177:4.9} Judas regresó al campamento con sus compañeros, embriagado con unas ideas de grandeza y de gloria como no había tenido desde hacía mucho tiempo. Se había enrolado con Jesús esperando convertirse algún día en un gran hombre en el nuevo reino, y al final se había dado cuenta de que no habría ningún nuevo reino tal como él lo había esperado. Pero se regocijaba por ser lo bastante sagaz como para canjear la decepción de no conseguir la gloria en el nuevo reino esperado por la obtención inmediata de honores y recompensas en el viejo orden de cosas; ahora creía que este viejo orden sobreviviría, y estaba seguro de que destruiría a Jesús y a todo lo que él representaba. En el móvil final de su intención consciente, la traición de Judas a Jesús fue el acto cobarde de un desertor egoísta cuya única preocupación era su propia seguridad y su glorificación, cualquiera que fueran los resultados de su conducta para su Maestro y sus antiguos compañeros.

\par 
%\textsuperscript{(1926.3)}
\textsuperscript{177:4.10} Pero siempre había sido así. Hacía mucho tiempo que Judas alimentaba esta conciencia deliberada, persistente, egoísta y vengativa de construir progresivamente en su mente, y de albergar en su corazón, estos deseos odiosos y malvados de venganza y deslealtad. Jesús amaba y confiaba en Judas tal como amaba y confiaba en los otros apóstoles, pero Judas no logró desarrollar a cambio una confianza leal ni experimentar un amor sincero. ¡Cuán peligrosa puede ser la ambición cuando está enteramente unida al egoísmo y motivada de manera suprema por la venganza sombría tanto tiempo reprimida! Qué aplastante es la decepción en la vida de aquellas personas necias que fijan sus miras en los atractivos oscuros y evanescentes del tiempo, y se vuelven ciegas a los logros superiores y más reales de las conquistas perpetuas de los mundos eternos de los valores divinos y de las verdaderas realidades espirituales. Judas ansiaba en su mente los honores mundanos y llegó a amar este deseo con todo su corazón; los otros apóstoles también ansiaban en su mente estos mismos honores mundanos, pero amaban a Jesús con el corazón y hacían todo lo posible por aprender a amar las verdades que él les enseñaba.

\par 
%\textsuperscript{(1926.4)}
\textsuperscript{177:4.11} Judas no se daba cuenta de ello en este momento, pero había criticado subconscientemente a Jesús desde que Juan el Bautista había sido decapitado por Herodes. En lo más profundo de su corazón, a Judas siempre le había indignado el hecho de que Jesús no salvara a Juan. No debéis olvidar que Judas había sido discípulo de Juan antes de convertirse en seguidor de Jesús. Toda esta acumulación de resentimiento humano y de amarga decepción que Judas había conservado en su alma con atuendos de odio, se encontraba ahora bien organizada en su mente subconsciente, lista para brotar y sumergirlo en cuanto se atreviera a separarse de la influencia protectora de sus hermanos, exponiéndose al mismo tiempo a las hábiles insinuaciones y a las burlas sutiles de los enemigos de Jesús. Cada vez que Judas permitía que sus esperanzas se elevaran muy alto, y Jesús decía o hacía algo que las hacía añicos, siempre quedaba en el corazón de Judas una cicatriz de amargo resentimiento; y a medida que estas cicatrices se multiplicaron, aquel corazón herido con tanta frecuencia perdió enseguida todo afecto real por aquel que había infligido esta experiencia desagradable a una personalidad bien intencionada, pero cobarde y egocéntrica. Judas no se daba cuenta de ello, pero era un cobarde. En consecuencia, siempre tenía la tendencia de atribuir a la cobardía de Jesús los móviles que le llevaron con tanta frecuencia a no coger el poder o la gloria cuando estaban en apariencia fácilmente a su alcance. Y todo hombre mortal sabe muy bien que el amor, aunque al principio haya sido sincero, puede convertirse finalmente en un odio real a causa de las decepciones, los celos y un resentimiento constante.

\par 
%\textsuperscript{(1927.1)}
\textsuperscript{177:4.12} Los jefes de los sacerdotes y los ancianos pudieron por fin respirar tranquilamente durante algunas horas. No tendrían que arrestar a Jesús en público, y los servicios de Judas como aliado traidor les aseguraba que Jesús no se escaparía de su jurisdicción como lo había hecho tantas veces en el pasado.

\section*{5. Las últimas horas de reunión social}
\par 
%\textsuperscript{(1927.2)}
\textsuperscript{177:5.1} Puesto que era miércoles, aquella noche en el campamento fueron horas de reunión social. El Maestro intentó animar a sus apóstoles abatidos, pero era casi imposible. Todos empezaban a darse cuenta de que se acercaban unos acontecimientos desconcertantes y abrumadores. No podían estar alegres, ni siquiera cuando el Maestro recordó sus años de asociación afectuosa y llena de acontecimientos. Jesús se interesó cuidadosamente por las familias de todos los apóstoles y, mirando a David Zebedeo, preguntó si alguien tenía noticias recientes de su madre, de su hermana menor o de otros miembros de su familia. David bajó la mirada hacia sus pies; tenía miedo de responder.

\par 
%\textsuperscript{(1927.3)}
\textsuperscript{177:5.2} Ésta fue la ocasión en que Jesús advirtió a sus seguidores que desconfiaran del apoyo de la multitud. Recordó sus experiencias en Galilea, cuando las grandes muchedumbres los habían seguido con entusiasmo una y otra vez, y luego se habían predispuesto contra ellos con el mismo ardor, para volver a sus creencias y maneras de vivir anteriores. Luego dijo: «Así pues, no os dejéis engañar por las grandes muchedumbres que nos escucharon en el templo y que parecían creer en nuestras enseñanzas. Esas multitudes escuchan la verdad y la creen superficialmente con su mente, pero pocos de ellos dejan que la palabra de la verdad se fije en su corazón con raíces vivientes. Cuando se presentan las dificultades reales, no se puede contar con el apoyo de aquellos que sólo conocen el evangelio en su mente, y no lo han experimentado en su corazón. Cuando los dirigentes de los judíos lleguen a un acuerdo para destruir al Hijo del Hombre, y golpeen al unísono, veréis que la multitud huirá aterrada o bien permanecerá allí asombrada en silencio, mientras esos dirigentes enloquecidos y ciegos conducen a la muerte a los instructores de la verdad evangélica. Luego, cuando la adversidad y las persecuciones caigan sobre vosotros, otros que creéis que aman la verdad también se dispersarán, y algunos renunciarán al evangelio y os abandonarán. Algunos que han estado muy cerca de nosotros ya han decidido desertar. Habéis descansado hoy como preparación para los acontecimientos inminentes. Vigilad pues, y orad para que mañana os podáis sentir fortalecidos para los días que se acercan».

\par 
%\textsuperscript{(1927.4)}
\textsuperscript{177:5.3} El ambiente del campamento estaba cargado de una tensión inexplicable. Unos mensajeros silenciosos iban y venían, comunicándose únicamente con David Zebedeo. Antes de que terminara la noche, algunos sabían que Lázaro había huido precipitadamente de Betania. Juan Marcos guardaba un silencio siniestro después de regresar al campamento, a pesar de haber pasado todo el día en compañía del Maestro. Todo esfuerzo por persuadirlo para que hablara sólo indicaba claramente que Jesús le había dicho que no hablara.

\par 
%\textsuperscript{(1928.1)}
\textsuperscript{177:5.4} Incluso el buen humor y la sociabilidad poco común del Maestro asustó a los apóstoles. Todos sentían la clara proximidad del terrible aislamiento que estaba a punto de caer sobre ellos con una prontitud arrolladora y un terror ineludible. Sospechaban vagamente lo que iba a suceder, y ninguno se sentía preparado para enfrentarse a la prueba. El Maestro había estado ausente todo el día, y lo habían echado enormemente de menos.

\par 
%\textsuperscript{(1928.2)}
\textsuperscript{177:5.5} Este miércoles por la noche marcó el punto más bajo del estado espiritual de los apóstoles hasta el momento mismo de la muerte del Maestro. Aunque el día siguiente era un día que les acercaba más al viernes trágico, al menos él todavía estaba con ellos, y pudieron pasar esas horas de inquietud más airosamente.

\par 
%\textsuperscript{(1928.3)}
\textsuperscript{177:5.6} Jesús sabía que ésta sería la última noche que podría dormir tranquilo con la familia que había elegido en la Tierra; un poco antes de la medianoche, los despidió diciendo: «Id a dormir, hermanos míos, y que la paz sea con vosotros hasta que nos levantemos mañana, un día más para hacer la voluntad del Padre y experimentar la alegría de saber que somos sus hijos».