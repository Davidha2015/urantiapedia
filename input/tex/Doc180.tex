\chapter{Documento 180. El discurso de despedida}
\par 
%\textsuperscript{(1944.1)}
\textsuperscript{180:0.1} DESPUÉS de cantar el salmo al final de la última cena, los apóstoles pensaron que Jesús tenía la intención de regresar inmediatamente al campamento, pero les indicó que se sentaran. El Maestro dijo:

\par 
%\textsuperscript{(1944.2)}
\textsuperscript{180:0.2} «Recordáis bien cuando os envié sin bolsa ni alforja, e incluso os aconsejé que no llevarais ninguna ropa de repuesto. Y todos recordaréis que no os faltó de nada. Pero ahora os encontráis en tiempos difíciles. Ya no podéis contar con la buena voluntad de las multitudes. De aquí en adelante, el que tenga una bolsa que la lleve con él. Cuando salgáis al mundo para proclamar este evangelio, encargaos de vuestro sostén como os parezca más conveniente\footnote{\textit{Haced preparativos para el trabajo}: Lc 22:35-36.}. He venido para traer la paz, pero ésta no aparecerá durante un tiempo»\footnote{\textit{Cambio en las instrucciones previas}: Lc 10:4.}.

\par 
%\textsuperscript{(1944.3)}
\textsuperscript{180:0.3} «Ha llegado la hora de que el Hijo del Hombre sea glorificado, y el Padre será glorificado en mí\footnote{\textit{Padre e Hijo glorificados}: Jn 13:31-32; 17:1,4.}. Amigos míos, sólo voy a estar con vosotros un poco más de tiempo. Pronto me buscaréis, pero no me encontraréis\footnote{\textit{Me buscaréis y no me encontraréis}: Jn 13:33.}, porque voy a un lugar donde no podéis venir en este momento\footnote{\textit{No podéis venir ahora}: Jn 7:33-34; 8:21.}. Pero cuando hayáis terminado vuestra obra en la Tierra tal como yo he terminado la mía, entonces vendréis a mí\footnote{\textit{Pero me seguiréis después}: Jn 13:36.} como yo me preparo ahora para ir hacia mi Padre. Voy a dejaros dentro de muy poco tiempo y no me veréis más en la Tierra, pero todos me veréis en la era venidera cuando ascendáis al reino que mi Padre me ha dado».

\section*{1. El nuevo mandamiento}
\par 
%\textsuperscript{(1944.4)}
\textsuperscript{180:1.1} Después de unos momentos de conversación informal, Jesús se levantó y dijo: «Cuando representé para vosotros una parábola que indicaba de qué manera deberíais estar dispuestos a serviros los unos a los otros, dije que deseaba daros un nuevo mandamiento; quisiera hacerlo ahora que estoy a punto de dejaros. Conocéis bien el mandamiento que ordena que os améis los unos a los otros; que améis a vuestro prójimo como a vosotros mismos. Pero incluso esta dedicación sincera por parte de mis hijos no me satisface plenamente. Quisiera que realizarais unos actos de amor aún más grandes en el reino de la fraternidad de los creyentes. Y por eso os doy este nuevo mandamiento: Que os améis los unos a los otros como yo os he amado. De esta manera, si os amáis así los unos a los otros, todos los hombres sabrán que sois mis discípulos»\footnote{\textit{El nuevo mandamiento}: Jn 13:34-35; 15:12,17.}.

\par 
%\textsuperscript{(1944.5)}
\textsuperscript{180:1.2} «Al daros este nuevo mandamiento, no pongo ninguna nueva carga sobre vuestra alma; os traigo más bien una nueva alegría y os doy la posibilidad de experimentar un nuevo placer\footnote{\textit{No os coloco nueva carga, sino un nuevo placer}: Jn 15:11.}, conociendo las delicias de dar el afecto de vuestro corazón a vuestros semejantes. Incluso soportando un dolor externo, estoy a punto de experimentar la alegría suprema\footnote{\textit{La muerte de Jesús como suprema alegría}: Heb 12:2.} de daros mi afecto a vosotros y a vuestros compañeros mortales».

\par 
%\textsuperscript{(1944.6)}
\textsuperscript{180:1.3} «Cuando os invito a que os améis los unos a los otros como yo os he amado\footnote{\textit{Amaos unos a otros como yo os he amado}: Jn 13:34-35; 15:12,17.}, os presento la medida suprema del verdadero afecto, porque nadie puede tener un amor más grande que éste: el de dar la vida por sus amigos\footnote{\textit{No hay amor mayor}: Jn 15:12-15.}. Y vosotros sois mis amigos; seguiréis siendo mis amigos con que sólo estéis dispuestos a hacer lo que os he enseñado. Me habéis llamado Maestro, pero yo no os llamo sirvientes. Si tan sólo os amáis los unos a los otros como yo os amo, seréis mis amigos y siempre os hablaré de lo que el Padre me revela».

\par 
%\textsuperscript{(1945.1)}
\textsuperscript{180:1.4} «No simplemente me habéis elegido vosotros, sino que yo también os he elegido, y os he ordenado\footnote{\textit{Os he elegido y ordenado}: Jn 15:16-17.} para que salgáis al mundo a fin de ofrecer el fruto del servicio amoroso a vuestros semejantes, tal como yo he vivido entre vosotros y os he revelado al Padre. El Padre y yo trabajaremos con vosotros, y vosotros experimentaréis la divina plenitud de la alegría con que sólo obedezcáis mi mandamiento de amaros los unos a los otros como yo os he amado».

\par 
%\textsuperscript{(1945.2)}
\textsuperscript{180:1.5} Si queréis compartir el gozo del Maestro, tenéis que compartir su amor\footnote{\textit{Para compartir el goz, compartir el amor}: Jn 14:23.}. Y compartir su amor significa que habéis compartido su servicio. Esta experiencia de amor no os libera de las dificultades de este mundo; no crea un mundo nuevo, pero hace con toda seguridad que el viejo mundo resulte nuevo.

\par 
%\textsuperscript{(1945.3)}
\textsuperscript{180:1.6} Retened en la memoria: Lo que Jesús pide es la lealtad, no el sacrificio. La conciencia de hacer un sacrificio implica la ausencia de ese afecto sincero que hubiera convertido ese servicio amoroso en una alegría suprema. La idea del \textit{deber} significa que tenéis una mentalidad de sirvientes, y a consecuencia de ello no conseguís la grandísima emoción de hacer vuestro servicio como un amigo y por un amigo. El impulso de la amistad trasciende todas las convicciones del deber, y el servicio que un amigo hace por un amigo nunca se puede llamar sacrificio. El Maestro ha enseñado a los apóstoles que son hijos de Dios\footnote{\textit{Los creyentes son hijos de Dios}: 1 Cr 22:10; Sal 2:7; Is 56:5; Mt 5:9,16,16,45; Lc 20:36; Jn 1:12-13; 11:52; 20:17b; Hch 17:28-29; Ro 8:14-17,19,21; 9:26; 2 Co 6:18; Gl 3:26; 4:5-7; Ef 1:5; Flp 2:15; Heb 12:5-8; 1 Jn 3:1-2,10; 5:2; Ap 21:7; 2 Sam 7:14.}. Los ha llamado hermanos, y ahora, antes de irse, los llama sus amigos.

\section*{2. La vid y los sarmientos}
\par 
%\textsuperscript{(1945.4)}
\textsuperscript{180:2.1} Luego, Jesús se levantó de nuevo y continuó enseñando a sus apóstoles: «Yo soy la verdadera vid, y mi Padre el viñador. Yo soy la vid, y vosotros los sarmientos. El Padre sólo me pide que produzcáis muchos frutos\footnote{\textit{Los discípulos producen frutos}: Jn 15:7-9.}. La vid solamente se poda para aumentar la fecundidad de sus sarmientos. Todo sarmiento estéril que sale de mí, el Padre lo cortará. Todo sarmiento que produzca fruto, el Padre lo limpiará para que pueda producir más frutos. Vosotros ya estáis limpios por la palabra que he pronunciado, pero debéis continuar estando limpios. Tenéis que permanecer en mí, y yo en vosotros; el sarmiento muere si se le separa de la vid. Así como el sarmiento no puede producir frutos a menos que permanezca en la vid, vosotros tampoco podéis producir los frutos del servicio amoroso a menos que permanezcáis en mí. Recordad: Yo soy la verdadera vid, y vosotros los sarmientos vivientes\footnote{\textit{La verdadera vid y los verdaderos sarmientos}: Jn 15:1-5.}. El que vive en mí, y yo en él, producirá muchos frutos del espíritu y experimentará la alegría suprema de dar esta cosecha espiritual. Si mantenéis esta unión espiritual viviente conmigo, produciréis un fruto abundante. Si permanecéis en mí y mis palabras viven en vosotros, podréis comulgar libremente conmigo, y entonces mi espíritu viviente se infiltrará en vosotros de tal manera que podréis pedir todo lo que mi espíritu quiere\footnote{\textit{Si pedís cualquier cosa que desee mi espíritu}: Jn 14:13-14.}, y hacer todo esto con la seguridad de que el Padre nos concederá nuestra petición. El Padre es glorificado en esto: que la vid tenga muchos sarmientos vivientes, y que cada sarmiento produzca muchos frutos. Y cuando el mundo vea estos sarmientos fructíferos ---mis amigos que se aman los unos a los otros como yo los he amado--- todos los hombres sabrán que sois realmente mis discípulos»\footnote{\textit{Se conocerá a mis discípulos por su amor}: Jn 13:35.}.

\par 
%\textsuperscript{(1945.5)}
\textsuperscript{180:2.2} «Así como el Padre me ha amado, yo os he amado\footnote{\textit{Jesús te ama}: Jn 15:9-10.}. Vivid en mi amor como yo vivo en el amor del Padre. Si hacéis lo que os he enseñado, permaneceréis en mi amor al igual que yo he guardado la palabra del Padre y permanezco eternamente en su amor».

\par 
%\textsuperscript{(1946.1)}
\textsuperscript{180:2.3} Los judíos habían enseñado desde hacía mucho tiempo que el Mesías sería «un tallo que surgiría de la vid»\footnote{\textit{Un tallo que surgiría de la vid}: Is 4:2; 11:1; Jer 23:5-6; 33:15; Dn 11:7; Zac 3:8; 6:12.} de los antepasados de David, y en conmemoración de esta antigua enseñanza, un gran emblema de la uva unida a su vid decoraba la entrada del templo de Herodes. Todos los apóstoles recordaron estas cosas mientras el Maestro les hablaba esta noche en la habitación de arriba.

\par 
%\textsuperscript{(1946.2)}
\textsuperscript{180:2.4} Pero más adelante, las conclusiones del Maestro sobre la oración fueron malinterpretadas, lo que produjo una gran pesadumbre. Estas enseñanzas hubieran provocado pocas dificultades si se hubieran recordado las palabras exactas del Maestro y hubieran sido transcritas fielmente con posterioridad. Pero de la manera en que se escribió el relato, los creyentes terminaron por considerar la oración en nombre de Jesús como una especie de magia suprema, creyendo que recibirían del Padre todo lo que pidieran. Durante siglos, las almas sinceras han continuado haciendo naufragar su fe contra este escollo. ¿Cuánto tiempo necesitará el mundo de los creyentes para comprender que la oración no es un proceso para conseguir lo que uno desea, sino más bien un programa para emprender el camino de Dios, una experiencia para aprender a reconocer y a ejecutar la voluntad del Padre? Es enteramente cierto que, cuando vuestra voluntad se ha alineado verdaderamente con la suya, podéis pedir cualquier cosa concebida por esta unión de voluntades, y os será concedida. Esta unión de voluntades se efectúa por medio de Jesús y a través de él, al igual que la vida de la vid circula y atraviesa los sarmientos vivientes\footnote{\textit{Error relativo a la «oración»}: Jn 14:13-14; 15:7,16b.}.

\par 
%\textsuperscript{(1946.3)}
\textsuperscript{180:2.5} Cuando existe esta conexión viviente entre la divinidad y la humanidad, si la humanidad reza sin reflexión y de manera ignorante por sus comodidades egoístas y sus éxitos vanidosos, sólo puede haber una respuesta divina: que los tallos de los sarmientos vivientes produzcan una mayor cantidad de frutos del espíritu. Cuando el sarmiento de la vid está vivo, todas sus peticiones sólo pueden recibir una respuesta: que produzca más uvas. De hecho, el sarmiento sólo existe para producir frutos\footnote{\textit{El sarmiento sólo existe para producir frutos}: Jn 15:1-5.}, y no puede hacer otra cosa que producir uvas. Y así, el verdadero creyente sólo existe con la finalidad de producir los frutos del espíritu\footnote{\textit{El fruto del espíritu: el amor}: Jn 15:8-9,12. \textit{Los frutos del espíritu}: Gl 5:22-23; Ef 5:9.}: amar a los hombres como él mismo ha sido amado por Dios ---que nos amemos los unos a los otros como Jesús nos ha amado\footnote{\textit{Amar como Jesús nos ha amado}: Jn 13:34-35; 15:12.}.

\par 
%\textsuperscript{(1946.4)}
\textsuperscript{180:2.6} Cuando la mano disciplinaria del Padre se coloca sobre la vid, lo hace con amor, para que los sarmientos puedan producir muchos frutos. Un sabio viñador sólo corta las ramas muertas y estériles.

\par 
%\textsuperscript{(1946.5)}
\textsuperscript{180:2.7} Jesús tuvo grandes dificultades para hacer que sus mismos apóstoles reconocieran que la oración es una función de los creyentes nacidos del espíritu, en el reino dominado por el espíritu.

\section*{3. La enemistad del mundo}
\par 
%\textsuperscript{(1946.6)}
\textsuperscript{180:3.1} Apenas habían terminado los once sus comentarios sobre el discurso de la vid y los sarmientos, el Maestro les indicó que deseaba continuar hablándoles, pues sabía que le quedaba poco tiempo, y dijo: «Cuando os haya dejado, no os desaniméis por la enemistad del mundo. No os sintáis abatidos ni siquiera cuando los creyentes pusilánimes se vuelvan contra vosotros y se alíen con los enemigos del reino. Si el mundo os odia\footnote{\textit{Si el mundo os odia}: Jn 15:18-21.}, recordad que me ha odiado antes que a vosotros. Si fuerais de este mundo, entonces el mundo amaría lo que es suyo, pero como no lo sois, el mundo se niega a amaros. Estáis en este mundo, pero no debéis vivir a su manera. Os he elegido y apartado del mundo para que representéis el espíritu de otro mundo en este mismo mundo en el que habéis sido elegidos. Pero recordad siempre las palabras que os he dicho: El servidor no es más grande que su señor. Si se atreven a perseguirme, también os perseguirán a vosotros. Si mis palabras ofenden a los incrédulos, vuestras palabras también ofenderán a los impíos. Y os harán todo esto porque no creen en mí ni en Aquel que me ha enviado; por eso sufriréis muchas cosas a causa de mi evangelio; pero cuando estéis soportando esas tribulaciones, deberíais recordar que yo también he sufrido antes que vosotros a causa de este evangelio del reino celestial».

\par 
%\textsuperscript{(1947.1)}
\textsuperscript{180:3.2} «Muchos de los que os atacarán ignoran la luz del cielo, pero éste no es el caso de algunos que nos persiguen ahora. Si no les hubiéramos enseñado la verdad, podrían hacer muchas cosas extrañas sin incurrir en la condenación, pero ahora, puesto que han conocido la luz y se han atrevido a rechazarla, no tienen excusas para su actitud\footnote{\textit{No hay excusa en la ignorancia}: Jn 15:22-25.}. El que me odia, odia a mi Padre. No puede ser de otra manera; la luz que podría salvaros si la aceptáis, sólo puede condenaros si la rechazáis a sabiendas. ¿Qué les he hecho a esos hombres para que me odien con un odio tan terrible? Nada, salvo ofrecerles la hermandad en la Tierra y la salvación en el cielo. Pero ¿no habéis leído en las Escrituras el dicho: `Y me odiaron sin causa'?»\footnote{\textit{Me odiaron sin causa}: Sal 35:19; 69:4.}

\par 
%\textsuperscript{(1947.2)}
\textsuperscript{180:3.3} «Pero no os dejaré solos en el mundo. Poco tiempo después de mi partida, os enviaré un ayudante espiritual. Tendréis con vosotros a alguien que ocupará mi lugar entre vosotros, a alguien que continuará enseñándoos el camino de la verdad, y que incluso os confortará»\footnote{\textit{Promesa del confortador}: Ez 11:19; 18:31; 36:26-27; Jl 2:28-29; Lc 24:49; Jn 7:39; 14:16-18,23,26; 15:4,26; 16:7-8,13-14; 17:21-23; Hch 1:5,8a; 2:1-4,16-18; 2:33; 2 Co 13:5; Gl 2:20; 4:6; Ef 1:13; 4:30; 1 Jn 4:12-15.}.

\par 
%\textsuperscript{(1947.3)}
\textsuperscript{180:3.4} «Que no se turbe vuestro corazón. Creéis en Dios; continuad creyendo también en mí. Aunque tenga que dejaros, no estaré lejos de vosotros. Ya os he dicho que hay muchas residencias en el universo de mi Padre. Si no fuera verdad, no os habría hablado repetidas veces de ellas. Voy a regresar a esos mundos de luz, a esas estaciones en el cielo del Padre a las que ascenderéis algún día\footnote{\textit{Muchas mansiones}: Jn 14:1-2.}. Desde esos lugares he venido a este mundo, y ahora ha llegado el momento en que debo regresar a la obra de mi Padre en las esferas de arriba».

\par 
%\textsuperscript{(1947.4)}
\textsuperscript{180:3.5} «Si os precedo así en el reino celestial del Padre, os enviaré a buscar con seguridad para que podáis estar conmigo\footnote{\textit{Estaréis conmigo}: Jn 14:3-4; 17:24.} en los lugares que fueron preparados para los hijos mortales de Dios antes de que existiera este mundo. Aunque debo dejaros, estaré presente con vosotros en espíritu, y finalmente estaréis conmigo en persona cuando hayáis ascendido hasta mí en mi universo, tal como yo estoy a punto de ascender hasta mi Padre en su universo más grande. Lo que os he dicho es verdad y eterno, aunque no podáis comprenderlo plenamente. Voy hacia el Padre, y aunque ahora no podáis seguirme, me seguiréis sin duda en las eras venideras».

\par 
%\textsuperscript{(1947.5)}
\textsuperscript{180:3.6} Cuando Jesús se sentó, Tomás se levantó y dijo: «Maestro, no sabemos adónde vas; por consiguiente, no conocemos el camino. Pero te seguiremos esta misma noche si nos muestras el camino»\footnote{\textit{«Muéstranos el camino»}: Jn 14:5.}.

\par 
%\textsuperscript{(1947.6)}
\textsuperscript{180:3.7} Cuando Jesús escuchó a Tomás, contestó: «Tomás, yo soy el camino\footnote{\textit{«Yo soy el camino»}: Jn 14:6.}, la verdad y la vida. Nadie va hacia el Padre si no es a través de mí. Todos los que encuentran al Padre, primero me encuentran a mí. Si me conocéis, conocéis el camino hacia el Padre. Y me conocéis de hecho, porque habéis vivido conmigo y ahora me veis».

\par 
%\textsuperscript{(1947.7)}
\textsuperscript{180:3.8} Pero esta enseñanza era demasiado profunda para muchos de los apóstoles, especialmente para Felipe, quien después de hablar unas palabras con Natanael, se levantó y dijo: «Maestro, muéstranos al Padre\footnote{\textit{«Muéstranos al Padre»}: Mt 11:27; Lc 10:22; Jn 1:18; 14:8.}, y todo lo que nos has dicho se aclarará».

\par 
%\textsuperscript{(1947.8)}
\textsuperscript{180:3.9} Cuando Felipe hubo hablado, Jesús dijo: «Felipe, ¿he estado tanto tiempo contigo y sin embargo ni siquiera me conoces ahora? Declaro de nuevo que aquel que me ha visto, ha visto al Padre\footnote{\textit{Quien me ha visto, ha visto al Padre}: Jn 14:9-11.}. ¿Cómo puedes decir entonces: muéstranos al Padre? ¿Acaso no crees que yo estoy en el Padre y el Padre en mí?\footnote{\textit{El Padre está en el Hijo y el Hijo en el Padre}: Jn 5:36; 10:37-38.} ¿No os he enseñado que las palabras que os digo no son mis palabras, sino las palabras del Padre? Hablo por el Padre y no por mí mismo. Estoy en este mundo para hacer la voluntad del Padre\footnote{\textit{He hecho la vountad del Padre}: Mt 26:39,42,44; Mc 14:36,39; Lc 22:42; Jn 4:34; 5:30; 6:38-40; 15:10; 17:4.}, y eso es lo que he hecho. Mi Padre permanece en mí y trabaja a través de mí. Creedme cuando digo que el Padre está en mí, y que yo estoy en el Padre, o si no, creedme por la vida misma que he vivido ---por mi obra».

\par 
%\textsuperscript{(1948.1)}
\textsuperscript{180:3.10} Mientras el Maestro se apartaba para refrescarse con agua, los once emprendieron una viva discusión sobre estas enseñanzas, y Pedro iba a empezar a pronunciar un largo discurso cuando Jesús regresó y les indicó que se sentaran.

\section*{4. El ayudante prometido}
\par 
%\textsuperscript{(1948.2)}
\textsuperscript{180:4.1} Jesús continuó su enseñanza, diciendo: «Cuando haya regresado al Padre, y él haya aceptado plenamente la obra que he realizado por vosotros en la Tierra, y después de que haya recibido la soberanía final sobre mi propio dominio, le diré a mi Padre: Como he dejado solos a mis hijos en la Tierra, es conforme a mi promesa enviarles a otro instructor. Y cuando el Padre lo apruebe, derramaré el Espíritu de la Verdad sobre todo el género humano. El espíritu de mi Padre ya está en vuestro corazón, y cuando llegue ese día, también me tendréis con vosotros como ahora tenéis al Padre. Este nuevo don es el espíritu de la verdad viviente. Los incrédulos no escucharán al principio las enseñanzas de este espíritu, pero todos los hijos de la luz lo recibirán con placer y de todo corazón. Cuando llegue este espíritu, lo conoceréis como me habéis conocido a mí, recibiréis este don en vuestro corazón, y él permanecerá con vosotros. Podéis percibir que no os voy a dejar sin ayuda ni guía. No voy a dejaros abandonados\footnote{\textit{No voy a dejaros abandonados}: Jn 14:16-18.}. Actualmente sólo puedo estar con vosotros en persona. En los tiempos venideros estaré con vosotros y con todos los demás hombres que deseen mi presencia, dondequiera que estéis, y con cada uno de vosotros al mismo tiempo. ¿No discernís que es mejor que me vaya\footnote{\textit{Es mejor que me vaya}: Jn 16:7.}, que os deje físicamente, para poder estar mejor y más plenamente con vosotros en espíritu?»

\par 
%\textsuperscript{(1948.3)}
\textsuperscript{180:4.2} «Dentro de muy pocas horas el mundo dejará de verme; pero continuaréis conociéndome en vuestro corazón hasta que os envíe este nuevo instructor, el Espíritu de la Verdad. De la misma manera que he vivido con vosotros en persona, viviré entonces dentro de vosotros; seré una sola cosa con vuestra experiencia personal en el reino del espíritu. Cuando esto suceda, sabréis con seguridad que estoy en el Padre, y que, aunque vuestra vida está oculta con el Padre que está en mí, yo también estoy en vosotros\footnote{\textit{El Padre está en el Hijo y el Hijo en nosotros}: Jn 14:19-21a; 17:21.}. He amado al Padre y he guardado su palabra; vosotros me habéis amado, y guardaréis mi palabra. Al igual que mi Padre me ha dado de su espíritu, yo os daré del mío. Este Espíritu de la Verdad que os donaré os guiará y os confortará, y os conducirá finalmente a toda la verdad»\footnote{\textit{El confortador os conducirá a la verdad}: Jn 16:13.}.

\par 
%\textsuperscript{(1948.4)}
\textsuperscript{180:4.3} «Os cuento estas cosas mientras aún estoy con vosotros, a fin de que estéis mejor preparados para soportar las pruebas que ahora se avecinan. Cuando llegue ese nuevo día, estaréis habitados tanto por el Hijo como por el Padre. Y esos dones del cielo trabajarán siempre el uno con el otro, al igual que el Padre y yo hemos trabajado en la Tierra delante de vuestros propios ojos como una sola persona: el Hijo del Hombre. Este amigo espiritual os traerá a la memoria todo lo que os he enseñado»\footnote{\textit{El espíritu recordará todo}: Jn 14:25-26.}.

\par 
%\textsuperscript{(1948.5)}
\textsuperscript{180:4.4} Mientras el Maestro se detenía un momento, Judas Alfeo se atrevió a hacer una de las pocas preguntas que él o su hermano hicieron a Jesús en público. Judas dijo: «Maestro, siempre has vivido entre nosotros como un amigo; ¿cómo te conoceremos cuando ya no te manifiestes a nosotros, salvo a través de este espíritu? Si el mundo no te ve, ¿cómo estaremos seguros de ti? ¿Cómo te mostrarás a nosotros?»\footnote{\textit{¿Cómo te mostrarás a nosotros?}: Jn 14:22.}

\par 
%\textsuperscript{(1949.1)}
\textsuperscript{180:4.5} Jesús los miró a todos, sonrió y dijo: «Hijos míos, me voy, voy de vuelta hacia el Padre. Dentro de poco ya no me veréis\footnote{\textit{Debo irme pronto}: Jn 7:33-34; 8:21; 13:33; 14:2,12,28; 16:10,16-17,28.} como me veis aquí, en carne y hueso. Dentro de muy poco tiempo os enviaré a mi espíritu\footnote{\textit{Os enviaré mi «espíritu»}: Ez 11:19; 18:31; 36:26-27; Jl 2:28-29; Lc 24:49; Jn 7:39; 14:16-18,23,26; 15:4,26; 16:7-8,13-14; 17:21-23; Hch 1:5,8a; 2:1-4,16-18; 2:33; 2 Co 13:5; Gl 2:20; 4:6; Ef 1:13; 4:30; 1 Jn 4:12-15.}, que es como yo, a excepción de este cuerpo material. Este nuevo instructor es el Espíritu de la Verdad que vivirá con cada uno de vosotros, en vuestro corazón, y así todos los hijos de la luz\footnote{\textit{Hijos de la luz}: Lc 16:8; Jn 12:36; Ef 5:8; 1 Ts 5:5.} serán como uno solo y serán atraídos\footnote{\textit{Atracción espiritual}: Jer 31:3; Jn 6:44; 12:32.} los unos hacia los otros. De esta manera concreta mi Padre y yo podremos vivir en el alma de cada uno de vosotros, y también en el corazón de todos los demás hombres que nos aman y hacen real ese amor en sus experiencias, amándose los unos a los otros\footnote{\textit{El nuevo mandamiento}: Jn 13:34-35; 15:12,17. \textit{La regla de oro}: Mt 7:12; Lc 6:31. \textit{La regla de oro (negativa)}: Tb 4:15.} como yo os amo ahora»\footnote{\textit{Vivir en el amor}: Jn 14:23.}.

\par 
%\textsuperscript{(1949.2)}
\textsuperscript{180:4.6} Judas Alfeo no comprendió plenamente lo que había dicho el Maestro, pero captó la promesa de un nuevo instructor, y por la expresión de la cara de Andrés, percibió que su pregunta había sido contestada satisfactoriamente.

\section*{5. El Espíritu de la Verdad}
\par 
%\textsuperscript{(1949.3)}
\textsuperscript{180:5.1} El nuevo ayudante que Jesús prometió enviar al corazón de los creyentes, derramar sobre todo el género humano, es el \textit{Espíritu de la Verdad}. Este don divino no es la letra o la ley de la verdad, ni tampoco está destinado a funcionar como la forma o la expresión de la verdad. El nuevo instructor es la \textit{convicciónde la verdad}, la conciencia y la seguridad de los verdaderos significados en los niveles espirituales reales. Este nuevo instructor es el espíritu de la verdad viviente y creciente, de la verdad que se expande, se desarrolla y se adapta.

\par 
%\textsuperscript{(1949.4)}
\textsuperscript{180:5.2} La verdad divina es una realidad viviente que es percibida por el espíritu. La verdad sólo existe en los niveles espirituales superiores de la comprensión de la divinidad y de la conciencia de la comunión con Dios. Podéis conocer la verdad, y podéis vivir la verdad; podéis experimentar el crecimiento de la verdad en el alma, y gozar de la libertad que su luz aporta a la mente, pero no podéis aprisionar la verdad en unas fórmulas, códigos, credos o modelos intelectuales de conducta humana. Cuando intentáis formular humanamente la verdad divina, ésta muere rápidamente. Incluso en el mejor de los casos, el salvamento póstumo de la verdad aprisionada sólo puede terminar en la realización de una forma particular de sabiduría intelectual glorificada. La verdad estática es una verdad muerta, y sólo la verdad muerta puede ser formulada en una teoría. La verdad viviente es dinámica y sólo puede gozar de una existencia experiencial en la mente humana.

\par 
%\textsuperscript{(1949.5)}
\textsuperscript{180:5.3} La inteligencia nace de una existencia material que está iluminada por la presencia de la mente cósmica. La sabiduría consta de la conciencia del conocimiento, elevada a nuevos niveles de significados, y activada por la presencia de la dotación universal del espíritu ayudante de la sabiduría. La verdad es un valor de la realidad espiritual que sólo lo experimentan los seres dotados de espíritu que ejercen su actividad en los niveles supermateriales de conciencia del universo, y que después de reconocer la verdad, permiten que su espíritu activador viva y reine en sus almas.

\par 
%\textsuperscript{(1949.6)}
\textsuperscript{180:5.4} El verdadero hijo que posee perspicacia universal busca el Espíritu viviente de la Verdad en toda palabra sabia. La persona que conoce a Dios eleva constantemente la sabiduría a los niveles de verdad viviente donde se alcanza la divinidad; el alma que no progresa espiritualmente arrastra todo el tiempo a la verdad viviente hacia los niveles muertos de la sabiduría y hacia los dominios de la simple exaltación del conocimiento.

\par 
%\textsuperscript{(1949.7)}
\textsuperscript{180:5.5} Cuando la regla de oro está despojada de la perspicacia suprahumana del Espíritu de la Verdad, no es nada más que una regla de conducta altamente ética. Cuando la regla de oro se interpreta literalmente, puede convertirse en un instrumento muy ofensivo para vuestros semejantes. Sin un discernimiento espiritual de la regla de oro de la sabiduría, podéis razonar que, puesto que deseáis que todos los hombres os digan con franqueza toda la verdad que tienen en su mente, vosotros deberíais expresarles de manera franca y total todos los pensamientos de vuestra mente. Una interpretación tan poco espiritual de la regla de oro podría ocasionar una infelicidad indecible y unas penas sin fin.

\par 
%\textsuperscript{(1950.1)}
\textsuperscript{180:5.6} Algunas personas disciernen e interpretan la regla de oro como una afirmación puramente intelectual de la fraternidad humana. Otras experimentan esta expresión de las relaciones humanas como una satisfacción emocional de los tiernos sentimientos de la personalidad humana. Otros mortales reconocen esta misma regla de oro como la vara que mide todas las relaciones sociales, el modelo de la conducta social. Y otros aún la consideran como el mandato positivo de un gran instructor moral, que incorporó en esta declaración el concepto más elevado de la obligación moral en lo concerniente a todas las relaciones fraternales. En la vida de esos seres morales, la regla de oro se convierte en el centro sabio y la circunferencia de toda su filosofía.

\par 
%\textsuperscript{(1950.2)}
\textsuperscript{180:5.7} En el reino de la fraternidad creyente de los amantes de la verdad que conocen a Dios, esta regla de oro adquiere cualidades vivientes de realización espiritual en aquellos niveles superiores de interpretación que inducen a los hijos mortales de Dios a considerar que este mandato del Maestro les exige que se relacionen con sus semejantes de tal manera, que éstos reciban el mayor bien posible como resultado de su contacto con los creyentes. Ésta es la esencia de la verdadera religión: que améis a vuestro prójimo como a vosotros mismos\footnote{\textit{Amar al prójimo como a uno mismo}: Lv 19:18,34; Mt 5:43-44; 19:19b; 22:39; Mc 12:31,33; Lc 10:27; Ro 13:9b; Gl 5:14; Stg 2:8.}.

\par 
%\textsuperscript{(1950.3)}
\textsuperscript{180:5.8} Pero la comprensión más elevada y la interpretación más verdadera de la regla de oro consiste en la conciencia del espíritu de la verdad de la realidad perdurable y viviente de esta declaración divina. El verdadero significado cósmico de esta regla de las relaciones universales solamente se revela en su comprensión espiritual, en la interpretación que el espíritu del Hijo hace de la ley de la conducta al espíritu del Padre que reside en el alma del hombre mortal. Cuando esos mortales conducidos por el espíritu se dan cuenta del verdadero significado de esta regla de oro, se llenan a rebosar con la certeza de ser ciudadanos de un universo amistoso, y sus ideales de realidad espiritual sólo se satisfacen cuando aman a sus semejantes como Jesús nos amó a todos. Ésta es la realidad de la comprensión del amor de Dios.

\par 
%\textsuperscript{(1950.4)}
\textsuperscript{180:5.9} Esta misma filosofía de flexibilidad viviente y de adaptabilidad cósmica de la verdad divina a las necesidades y capacidades individuales de cada hijo de Dios, ha de ser percibida antes de que podáis esperar comprender adecuadamente la enseñanza y la práctica del Maestro de la no resistencia al mal. La enseñanza del Maestro es básicamente una declaración espiritual. Incluso las implicaciones materiales de su filosofía no pueden considerarse con utilidad independientemente de sus correlaciones espirituales. El espíritu del mandato del Maestro consiste en no oponer resistencia a todas las reacciones egoístas hacia el universo, y al mismo tiempo alcanzar de manera dinámica y progresiva los niveles rectos de los verdaderos valores espirituales: la belleza divina, la bondad infinita y la verdad eterna ---conocer a Dios y volverse cada vez más como él.

\par 
%\textsuperscript{(1950.5)}
\textsuperscript{180:5.10} El amor, el altruismo, debe sufrir una interpretación readaptativa constante y viviente de las relaciones de acuerdo con las directrices del Espíritu de la Verdad. El amor debe captar así los conceptos ampliados y siempre cambiantes del bien cósmico más elevado para la persona que es amada. Luego, el amor continúa adoptando esta misma actitud hacia todas las demás personas que quizás pudieran ser influidas por las relaciones crecientes y vivientes del amor que un mortal conducido por el espíritu siente por otros ciudadanos del universo. Toda esta adaptación viviente del amor debe efectuarse a la luz del entorno de mal presente y de la meta eterna de la perfección del destino divino.

\par 
%\textsuperscript{(1950.6)}
\textsuperscript{180:5.11} Y así, tenemos que reconocer claramente que ni la regla de oro ni la enseñanza de la no resistencia se pueden entender nunca correctamente como dogmas o preceptos. Sólo se pueden comprender viviéndolas, percatándose de sus significados en la interpretación viviente del Espíritu de la Verdad, que dirige el contacto afectuoso entre los seres humanos.

\par 
%\textsuperscript{(1951.1)}
\textsuperscript{180:5.12} Y todo esto indica claramente la diferencia entre la antigua religión y la nueva. La antigua religión enseñaba la abnegación; la nueva religión sólo enseña el olvido de sí mismo, una autorrealización elevada gracias al servicio social unido a la comprensión del universo. La antigua religión estaba motivada por la conciencia del miedo; el nuevo evangelio del reino está dominado por la convicción de la verdad, el espíritu de la verdad eterna y universal. En la experiencia de la vida de los creyentes en el reino, ninguna cantidad de piedad o de lealtad a un credo puede compensar la ausencia de esa amabilidad espontánea, generosa y sincera que caracteriza a los hijos del Dios viviente nacidos del espíritu. Ni la tradición, ni un sistema ceremonial de culto oficial, pueden compensar la falta de compasión auténtica por nuestros semejantes.

\section*{6. La necesidad de partir}
\par 
%\textsuperscript{(1951.2)}
\textsuperscript{180:6.1} Después de que Pedro, Santiago, Juan y Mateo hubieron hecho numerosas preguntas al Maestro, éste continuó su discurso de despedida, diciendo: «Os cuento todo esto antes de dejaros, a fin de que podáis estar preparados de tal manera para lo que os va a suceder\footnote{\textit{Advertencias premonitorias}: Jn 16:1-4.}, que no cometáis graves errores. Las autoridades no se contentarán con expulsaros simplemente de las sinagogas; os advierto que se acerca la hora en que aquellos que os maten pensarán que están haciendo un servicio a Dios. Os harán todas estas cosas a vosotros y a los que conduzcáis al reino de los cielos, porque no conocen al Padre. Se han negado a conocer al Padre al negarse a recibirme; y se negarán a recibirme cuando os rechacen a vosotros, a condición de que hayáis guardado mi nuevo mandamiento\footnote{\textit{Guardad mi nuevo mandamiento}: Jn 13:34; Jn 15:12.} de que os améis los unos a los otros como yo os he amado. Os cuento de antemano estas cosas para que cuando llegue vuestra hora, como ahora ha llegado la mía, os sintáis fortalecidos por el conocimiento de que yo sabía todo esto, y que mi espíritu estará con vosotros en todo lo que sufriréis por mi causa y a causa del evangelio. Por este motivo os he hablado tan claramente desde el principio. Os he advertido incluso que los enemigos de un hombre pueden ser los miembros de su propia familia\footnote{\textit{Enemigos incluso en la propia familia}: Mt 10:21,36; Mc 13:12; Lc 21:16.}. Aunque este evangelio del reino nunca deja de traer una gran paz al alma del creyente individual, no traerá la paz a la Tierra hasta que los hombres no estén dispuestos a creer de todo corazón en mis enseñanzas, y a establecer la práctica de hacer la voluntad del Padre como meta principal de su vida mortal».

\par 
%\textsuperscript{(1951.3)}
\textsuperscript{180:6.2} «Ahora que os dejo, puesto que ha llegado la hora en que estoy a punto de ir hacia el Padre, me sorprende que ninguno de vosotros me haya preguntado: ¿Por qué nos dejas? Sin embargo, sé que os hacéis estas preguntas en vuestro corazón. Os hablaré claramente, como lo hace un amigo a otro. Es realmente beneficioso para vosotros que me vaya\footnote{\textit{La necesidad de la partida}: Jn 16:5-8.}. Si no me voy, el nuevo instructor no podrá venir a vuestro corazón. Debo ser despojado de este cuerpo mortal, y restablecido en mi puesto en el cielo, antes de poder enviar a este instructor espiritual para que viva en vuestra alma y conduzca a vuestro espíritu a la verdad. Cuando mi espíritu llegue para residir en vosotros, iluminará la diferencia entre el pecado y la rectitud, y os permitirá juzgar sabiamente en vuestro corazón acerca de ambas cosas».

\par 
%\textsuperscript{(1951.4)}
\textsuperscript{180:6.3} «Aún tengo que deciros muchas cosas, pero ahora no podéis soportar más. Sin embargo, cuando llegue el Espíritu de la Verdad\footnote{\textit{El Espíritu de la Verdad}: Ez 11:19; 18:31; 36:26-27; Jl 2:28-29; Lc 24:49; Jn 7:39; 14:16-18,23,26; 15:4,26; 16:7-8,13-14; 17:21-23; Hch 1:5,8a; 2:1-4,16-18; 2:33; 2 Co 13:5; Gl 2:20; 4:6; Ef 1:13; 4:30; 1 Jn 4:12-15.}, os guiará finalmente a toda la verdad\footnote{\textit{Seréis guiados por el Espíritu de la Verdad}: Jn 16:13-14.} a medida que paséis por las muchas moradas\footnote{\textit{Muchas mansiones}: Jn 14:2.} del universo de mi Padre».

\par 
%\textsuperscript{(1951.5)}
\textsuperscript{180:6.4} «Este espíritu no hablará de sí mismo\footnote{\textit{El espíritu no hablará de sí mismo}: Jn 16:13b.}, pero os proclamará lo que el Padre le ha revelado al Hijo, e incluso os mostrará cosas por venir; me glorificará como yo he glorificado a mi Padre. Este espíritu sale de mí y os revelará mi verdad\footnote{\textit{El nuevo instructor os revelará todo}: Jn 16:14-15.}. Todo lo que el Padre posee en este dominio ahora es mío; por eso os he dicho que este nuevo instructor tomará lo que es mío y os lo revelará».

\par 
%\textsuperscript{(1952.1)}
\textsuperscript{180:6.5} «Dentro de muy poco os dejaré por un corto período de tiempo. Después, cuando me veáis de nuevo\footnote{\textit{Apareceré otra vez}: Jn 16:16.}, ya estaré camino del Padre, de manera que, incluso entonces, no me veréis por mucho tiempo».

\par 
%\textsuperscript{(1952.2)}
\textsuperscript{180:6.6} Mientras hacía una corta pausa, los apóstoles empezaron a hablar entre ellos: «¿Qué es lo que nos está diciendo? `Dentro de muy poco os dejaré', y `cuando me veáis de nuevo no será por mucho tiempo, porque estaré camino del Padre'. ¿Qué quiere decir con este `poco tiempo' y `no por mucho tiempo'? No podemos comprender lo que nos dice»\footnote{\textit{Jesús no es comprendido}: Jn 16:17-18.}.

\par 
%\textsuperscript{(1952.3)}
\textsuperscript{180:6.7} Puesto que Jesús sabía que se hacían estas preguntas, dijo: «¿Os preguntáis unos a otros sobre lo que he querido decir cuando he indicado que dentro de muy poco ya no estaré con vosotros y que, cuando me veáis de nuevo, estaré camino del Padre? Os he dicho claramente que el Hijo del Hombre debe morir, pero que resucitará\footnote{\textit{Debo morir pero resucitaré}: Mt 16:21; 17:22-23a; 20:17-19; 27:63; Mc 8:31; 9:31; 10:32-34; Lc 9:22,31,43b-44; 18:31-33; 24:7,46; Jn 14:28a; 20:9.}. ¿No podéis discernir pues el significado de mis palabras? Primero os sentiréis apenados, pero más tarde os regocijaréis\footnote{\textit{Primero os apenaréis y luego os regocijaréis}: Jn 16:19-22.} con muchas personas que comprenderán estas cosas después de que hayan sucedido. En verdad, una mujer está angustiada a la hora del parto, pero una vez que ha dado a luz a su hijo, olvida inmediatamente su angustia ante la alegría de saber que un hombre ha nacido en el mundo. De la misma manera, os vais a entristecer por mi partida, pero os veré pronto de nuevo, y entonces vuestra pena se transformará en alegría, y os llegará una nueva revelación de la salvación de Dios que nadie podrá quitaros nunca. Y todos los mundos serán benditos en esta misma revelación de la vida que derrota a la muerte. Hasta ahora habéis hecho todas vuestras peticiones en nombre de mi Padre. Después de que me veáis de nuevo, podréis pedir también en mi nombre\footnote{\textit{Pedir en nombre de Jesús}: Jn 16:23-24,26.}, y yo os escucharé».

\par 
%\textsuperscript{(1952.4)}
\textsuperscript{180:6.8} «Aquí abajo os he enseñado con proverbios y os he hablado en parábolas. Lo he hecho así porque espiritualmente sólo erais niños; pero se acerca el momento en que os hablaré claramente\footnote{\textit{Os hablaré claramente}: Jn 16:25.} sobre el Padre y su reino. Y lo haré porque el Padre mismo os ama y desea ser revelado más plenamente a vosotros. El hombre mortal no puede ver al Padre\footnote{\textit{El hombre no puede ver a Dios}: Jn 16:27-28.} que es espíritu; por eso he venido al mundo para mostrar el Padre a vuestros ojos de criaturas. Pero cuando os hayáis perfeccionado en el crecimiento espiritual, entonces veréis al Padre mismo».

\par 
%\textsuperscript{(1952.5)}
\textsuperscript{180:6.9} Cuando los once le oyeron hablar así, se dijeron unos a otros: «Mirad, ahora nos habla claramente. Es seguro que el Maestro ha venido de Dios. Pero, ¿por qué dice que debe regresar al Padre?» Jesús vio que incluso entonces no le comprendían\footnote{\textit{Los apóstoles no comprenden a Jesús}: Jn 16:29-31.}. Estos once hombres no podían liberarse de las ideas que habían abrigado durante mucho tiempo sobre el concepto judío del Mesías. Cuanto más plenamente creían en Jesús como Mesías, más embarazosas se volvían estas nociones profundamente arraigadas sobre el glorioso triunfo material del reino en la Tierra.