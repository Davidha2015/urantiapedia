\chapter{Documento 183. La traición y el arresto de Jesús}
\par 
%\textsuperscript{(1971.1)}
\textsuperscript{183:0.1} CUANDO Jesús despertó finalmente a Pedro, Santiago y Juan, les sugirió que se fueran a sus tiendas y trataran de dormir a fin de prepararse para las tareas del día siguiente. Pero para entonces, los tres apóstoles estaban totalmente despiertos; habían descansado con las breves cabezadas que habían dado, y además se habían estimulado y excitado con la llegada al lugar de dos mensajeros agitados que preguntaron por David Zebedeo, y que partieron rápidamente en su búsqueda en cuanto Pedro les indicó dónde se encontraba de vigilancia.

\par 
%\textsuperscript{(1971.2)}
\textsuperscript{183:0.2} Aunque ocho de los apóstoles estaban profundamente dormidos, los griegos que estaban acampados junto a ellos tenían un mayor temor de que se produjeran disturbios, de tal manera que habían apostado un centinela para que diera la alarma en caso de que se presentara algún peligro. Cuando los dos mensajeros entraron precipitadamente en el campamento, el centinela griego procedió a despertar a todos sus compatriotas, los cuales salieron de sus tiendas completamente vestidos y armados. Todo el campamento estaba ahora despierto, salvo los ocho apóstoles; Pedro deseaba llamar a sus compañeros, pero Jesús se lo prohibió terminantemente. El Maestro recomendó dulcemente a todos que regresaran a sus tiendas, pero estaban poco dispuestos a someterse a su sugerencia.

\par 
%\textsuperscript{(1971.3)}
\textsuperscript{183:0.3} Como no logró dispersar a sus seguidores, el Maestro los dejó y descendió hacia el lagar, cerca de la entrada del parque de Getsemaní. Los tres apóstoles, los griegos y los otros miembros del campamento dudaron en seguirlo inmediatamente, pero Juan Marcos corrió entre los olivos y se ocultó en un pequeño cobertizo cerca del lagar. Jesús se había alejado del campamento y de sus amigos para que cuando llegaran sus captores pudieran arrestarlo sin molestar a sus apóstoles. El Maestro temía que sus apóstoles se despertaran y estuvieran presentes en el momento de su arresto, no sea que el espectáculo de la traición de Judas suscitara de tal manera su animosidad que ofrecieran resistencia a los soldados y fueran apresados con él. Temía que si eran arrestados con él, pudieran perecer también con él.

\par 
%\textsuperscript{(1971.4)}
\textsuperscript{183:0.4} Aunque Jesús sabía que el plan para matarlo se había originado en los consejos de los dirigentes de los judíos, también era consciente de que todos estos proyectos nefastos tenían la plena aprobación de Lucifer, Satanás y Caligastia. Sabía muy bien que estos rebeldes de los reinos también verían con placer que todos los apóstoles fueran exterminados con él.

\par 
%\textsuperscript{(1971.5)}
\textsuperscript{183:0.5} Jesús se sentó solo en el lagar, donde esperó la llegada del traidor, y en aquel momento solamente era visto por Juan Marcos y una multitud innumerable de observadores celestiales.

\section*{1. La voluntad del Padre}
\par 
%\textsuperscript{(1971.6)}
\textsuperscript{183:1.1} Existe el gran peligro de malinterpretar el significado de numerosos dichos y de muchos acontecimientos que acompañaron el final de la carrera del Maestro en la carne. El tratamiento cruel que los criados ignorantes y los soldados insensibles infligieron a Jesús, la manera injusta en que fue juzgado y la actitud sin piedad de los dirigentes religiosos declarados, no se deben confundir con el hecho de que al someterse pacientemente a todo este sufrimiento y humillación, Jesús estaba haciendo realmente la voluntad del Padre Paradisiaco. De hecho y en verdad, era voluntad del Padre que su Hijo bebiera por completo la copa de la experiencia humana desde el nacimiento hasta la muerte, pero el Padre que está en los cielos no instigó de ninguna manera la conducta bárbara de aquellos seres humanos, supuestamente civilizados, que torturaron tan brutalmente al Maestro y acumularon tan horriblemente unas indignidades tras otras sobre una persona que no ofrecía resistencia. Estas experiencias inhumanas e impactantes que Jesús tuvo que soportar durante las últimas horas de su vida mortal no formaban parte en ningún sentido de la voluntad divina del Padre, una voluntad que la naturaleza humana del Maestro se había comprometido tan triunfalmente a realizar en el momento de la rendición final del hombre a Dios, tal como lo indicaba la triple oración que formuló en el jardín mientras sus cansados apóstoles dormían el sueño del agotamiento físico.

\par 
%\textsuperscript{(1972.1)}
\textsuperscript{183:1.2} El Padre que está en los cielos deseaba que el Hijo donador terminara su carrera terrenal de manera \textit{natural}, exactamente como todos los mortales deben terminar su vida en la Tierra y en la carne. Los hombres y las mujeres corrientes no pueden esperar que una dispensación especial les facilite sus últimas horas en la Tierra y el episodio posterior de la muerte. En consecuencia, Jesús escogió abandonar su vida en la carne de una manera que fuera conforme con el proceso natural de los acontecimientos, y se negó firmemente a librarse de las garras crueles de una malvada conspiración de acontecimientos inhumanos que lo arrastraron, con horrible certeza, hacia su humillación increíble y su muerte ignominiosa. Cada detalle de toda esta asombrosa manifestación de odio y de esta demostración de crueldad sin precedentes fue obra de unos hombres malvados y de unos mortales perversos. Dios que está en el cielo no lo quiso así, ni tampoco fue dictado por los enemigos acérrimos de Jesús, aunque éstos hicieron muchas cosas para asegurarse de que los mortales malvados e irreflexivos rechazaran así al Hijo donador. Incluso el padre del pecado volvió su rostro ante el horror atroz de la escena de la crucifixión.

\section*{2. Judas en la ciudad}
\par 
%\textsuperscript{(1972.2)}
\textsuperscript{183:2.1} Después de abandonar tan precipitadamente la mesa durante la
Última Cena, Judas fue directamente a la casa de su primo, y luego los dos se dirigieron directamente a ver al capitán de los guardias del templo. Judas le pidió al capitán que reuniera a los guardias y le informó que estaba listo para conducirlos hasta Jesús. Como Judas había aparecido en escena un poco antes de lo esperado, hubo cierta demora hasta que partieron hacia la casa de Marcos, donde Judas esperaba que Jesús se encontraría todavía charlando con los apóstoles. El Maestro y los once salieron de la casa de Elías Marcos unos quince minutos antes de que llegaran el traidor y los guardias. Cuando los captores llegaron a la casa de Marcos, Jesús y los once estaban muy lejos de los muros de la ciudad, camino del campamento en el Olivete.

\par 
%\textsuperscript{(1972.3)}
\textsuperscript{183:2.2} A Judas le inquietó mucho el fracaso que supuso no encontrar a Jesús en el domicilio de Marcos y en compañía de once hombres, de los cuales sólo dos estaban armados para defenderse. Sabía por casualidad que, cuando salieron del campamento por la tarde, sólo Simón Pedro y Simón Celotes se habían ceñido sus espadas; Judas había esperado apresar a Jesús mientras la ciudad estaba tranquila y había pocas posibilidades de resistencia. El traidor temía tener que enfrentarse con más de sesenta discípulos fervientes si esperaba que regresaran a su campamento, y también sabía que Simón Celotes tenía en su poder una buena cantidad de armas. Judas se iba poniendo cada vez más nervioso a medida que pensaba en cómo lo detestarían los once apóstoles leales, y temía que todos intentaran aniquilarlo. No solamente era desleal, sino que en el fondo era un verdadero cobarde.

\par 
%\textsuperscript{(1973.1)}
\textsuperscript{183:2.3} Como no lograron encontrar a Jesús en la sala de arriba, Judas le pidió al capitán de los guardias que regresaran al templo. Mientras tanto, los dirigentes habían empezado a congregarse en la casa del sumo sacerdote, preparándose para recibir a Jesús, puesto que su pacto con el traidor exigía que Jesús fuera arrestado aquel día a medianoche. Judas explicó a sus asociados que no habían encontrado a Jesús en la casa de Marcos, y que sería necesario ir a Getsemaní para detenerlo. Luego el traidor continuó diciendo que más de sesenta seguidores fervientes estaban acampados con él, y que todos ellos estaban bien armados. Los dirigentes de los judíos recordaron a Judas que Jesús siempre había predicado la no resistencia, pero Judas replicó que no podían contar con que todos los seguidores de Jesús obedecieran esta enseñanza. Judas temía realmente por su vida, y por ello se atrevió a pedir una compañía de cuarenta soldados armados. Puesto que las autoridades judías no disponían de una fuerza semejante de hombres armados bajo su jurisdicción, se dirigieron inmediatamente a la fortaleza de Antonia y le pidieron al comandante romano que les diera esta guardia; pero cuando éste se enteró de que tenían la intención de arrestar a Jesús, rehusó rápidamente acceder a su petición y los envió a su oficial superior. De esta manera perdieron más de una hora yendo de una autoridad a otra, hasta que finalmente se vieron obligados a presentarse ante el mismo Pilatos para obtener el permiso de emplear los guardias armados romanos. Ya era tarde cuando llegaron a la casa de Pilatos, y éste se había retirado con su mujer a sus aposentos privados. Dudó en inmiscuirse de alguna manera en esta empresa, y aún más porque su mujer le había pedido que no concediera esta petición. Pero puesto que el presidente oficial del sanedrín judío estaba presente y solicitaba personalmente esta ayuda, el gobernador consideró que era sabio conceder la petición, pensando que más adelante podría enmendar cualquier injusticia que estuvieran dispuestos a cometer.

\par 
%\textsuperscript{(1973.2)}
\textsuperscript{183:2.4} En consecuencia, cuando Judas Iscariote salió del templo hacia las once y media de la noche, iba acompañado de más de sesenta personas ---los guardias del templo, los soldados romanos y los criados curiosos de los sacerdotes y dirigentes principales.

\section*{3. El arresto del Maestro}
\par 
%\textsuperscript{(1973.3)}
\textsuperscript{183:3.1} Mientras esta compañía de soldados y guardias armados, provistos de antorchas y linternas, se acercaba al jardín, Judas se adelantó considerablemente al grupo a fin de estar preparado para identificar rápidamente a Jesús, de manera que los captores pudieran prenderlo fácilmente antes de que sus compañeros acudieran a defenderlo. Había también otra razón por la que Judas escogió adelantarse a los enemigos del Maestro: Pensó que así parecería que había llegado a la escena antes que los soldados, de tal manera que los apóstoles y las otras personas reunidas alrededor de Jesús quizás no lo relacionarían directamente con los guardias armados que le seguían tan de cerca. Judas había pensado incluso en alardear de que se había apresurado para prevenirlos de la llegada de los captores, pero este plan fue desbaratado por el saludo sombrío con que Jesús recibió al traidor. Aunque el Maestro le habló a Judas con amabilidad, lo recibió como a un traidor.

\par 
%\textsuperscript{(1973.4)}
\textsuperscript{183:3.2} Tan pronto como Pedro, Santiago, Juan y unos treinta de sus compañeros de campamento vieron al grupo armado y sus antorchas girar en la cima de la colina, supieron que aquellos soldados venían a arrestar a Jesús, y todos descendieron precipitadamente hacia el lagar, donde el Maestro estaba sentado en una soledad iluminada por la Luna. Mientras la compañía de soldados se acercaba por un lado, los tres apóstoles y sus compañeros se acercaban por el otro. Cuando Judas avanzó a zancadas para acercarse al Maestro, los dos grupos se quedaron inmóviles con el Maestro entre ellos, mientras Judas se preparaba para estampar el beso traidor en la frente de Jesús.

\par 
%\textsuperscript{(1974.1)}
\textsuperscript{183:3.3} El traidor había esperado que, después de conducir a los guardias hasta Getsemaní, podría simplemente indicar a los soldados quién era Jesús, o a lo más llevar a cabo la promesa de saludarlo con un beso, y luego alejarse rápidamente de la escena. Judas tenía mucho miedo de que todos los apóstoles estuvieran presentes y que concentraran su ataque sobre él como castigo por haberse atrevido a traicionar a su amado instructor. Pero cuando el Maestro lo saludó como a un traidor, se sintió tan confundido que no hizo ningún intento por huir.

\par 
%\textsuperscript{(1974.2)}
\textsuperscript{183:3.4} Jesús hizo un último esfuerzo por evitarle a Judas que llevara a cabo el gesto efectivo de traicionarlo. Antes de que el traidor pudiera llegar hasta él, se apartó a un lado y se dirigió al soldado principal de la izquierda, el capitán de los romanos, diciendo: «¿A quién buscáis?» El capitán contestó: «A Jesús de Nazaret». Entonces Jesús se presentó inmediatamente delante del oficial y, con la tranquila majestad del Dios de toda esta creación, dijo: «Soy yo». Muchos miembros de este grupo armado habían escuchado a Jesús enseñar en el templo, otros se habían enterado de sus obras poderosas, y cuando le escucharon anunciar tan audazmente su identidad, los que se encontraban en las primeras filas retrocedieron repentinamente. Se quedaron aturdidos de sorpresa ante la tranquila y majestuosa declaración de su identidad. Judas no tenía pues ninguna necesidad de continuar con su plan de traición. El Maestro se había revelado audazmente a sus enemigos, y éstos podían haberlo arrestado sin la ayuda de Judas. Pero el traidor tenía que hacer algo para justificar su presencia con este grupo armado, y además quería hacer alarde de que estaba realizando su papel en el pacto de traición acordado con los jefes de los judíos, para hacerse digno de la gran recompensa y de los honores que creía que se acumularían sobre él como compensación por su promesa de entregarles a Jesús.

\par 
%\textsuperscript{(1974.3)}
\textsuperscript{183:3.5} Mientras los guardias se recuperaban de su primera vacilación al ver a Jesús y escuchar el sonido de su voz excepcional, y mientras los apóstoles y los discípulos se acercaban cada vez más, Judas avanzó hacia Jesús, le dio un beso en la frente, y dijo: «Salve, Maestro e Instructor». Mientras Judas abrazaba así a su Maestro, Jesús dijo: «Amigo, ¡no te basta con hacer esto! ¿Traicionarás también al Hijo del Hombre con un beso?»

\par 
%\textsuperscript{(1974.4)}
\textsuperscript{183:3.6} Los apóstoles y los discípulos se quedaron literalmente anonadados por lo que estaban viendo. Durante un momento nadie se movió. Luego Jesús se desembarazó del abrazo traidor de Judas, se acercó a los guardias y soldados y preguntó de nuevo: «¿A quién buscáis?» El capitán dijo otra vez: «A Jesús de Nazaret». Y Jesús contestó de nuevo: «Os he dicho que soy yo. Así pues, si me buscáis a mí, dejad que estos otros se vayan. Estoy listo para ir con vosotros».

\par 
%\textsuperscript{(1974.5)}
\textsuperscript{183:3.7} Jesús estaba preparado para regresar a Jerusalén con los guardias, y el capitán de los soldados estaba enteramente dispuesto a permitir que los tres apóstoles y sus compañeros se fueran en paz. Pero antes de que pudieran partir, mientras Jesús estaba allí esperando las órdenes del capitán, un tal Malco, el guardaespaldas sirio del sumo sacerdote, se acercó a Jesús y se preparó para atarle las manos a la espalda, aunque el capitán romano no había ordenado que Jesús fuera atado así. Cuando Pedro y sus compañeros vieron que su Maestro era sometido a esta indignidad, ya no fueron capaces de contenerse más tiempo. Pedro sacó su espada y se abalanzó con los demás para golpear a Malco. Pero antes de que los soldados pudieran acudir en defensa del servidor del sumo sacerdote, Jesús levantó la mano delante de Pedro con gesto de prohibición y le habló severamente, diciendo: «Pedro, guarda tu espada. Los que sacan la espada, perecerán por la espada. ¿No comprendes que es voluntad del Padre que yo beba esta copa? ¿Y no sabes además que incluso ahora podría ordenar a más de doce legiones de ángeles y a sus asociados que me liberaran de las manos de estos pocos hombres?»

\par 
%\textsuperscript{(1975.1)}
\textsuperscript{183:3.8} Aunque Jesús puso así fin eficazmente a esta demostración de resistencia física por parte de sus seguidores, lo sucedido bastó para despertar los temores del capitán de los guardias, el cual, con la ayuda de sus soldados, puso sus pesadas manos sobre Jesús y lo ató rápidamente. Mientras le ataban las manos con fuertes cuerdas, Jesús les dijo: «¿Por qué salís contra mí con espadas y palos como para capturar a un ladrón? He estado diariamente con vosotros en el templo, enseñando públicamente a la gente, y no habéis hecho ningún esfuerzo por apresarme».

\par 
%\textsuperscript{(1975.2)}
\textsuperscript{183:3.9} Cuando Jesús estuvo atado, el capitán, temiendo que los seguidores del Maestro intentaran rescatarlo, dio órdenes para que fueran capturados; pero los soldados no fueron lo suficientemente rápidos porque, como los seguidores de Jesús habían escuchado las órdenes del capitán de que fueran arrestados, huyeron precipitadamente por la hondonada. Durante todo este tiempo, Juan Marcos había permanecido recluido en el cobertizo cercano. Cuando los guardias emprendieron el regreso hacia Jerusalén con Jesús, Juan Marcos intentó salir a escondidas del cobertizo para unirse a los apóstoles y discípulos que habían huido; pero en el preciso momento en que salía, uno de los últimos soldados que regresaba de perseguir a los discípulos que huían pasó por allí y, al ver a este joven con su manto de lino, empezó a perseguirlo y casi llegó a atraparlo. De hecho, el soldado se acercó lo suficiente a Juan como para agarrar su manto, pero el joven se liberó de la ropa y se escapó desnudo mientras el soldado se quedaba con el manto vacío. Juan Marcos se dirigió a toda prisa hacia el sendero de arriba donde se encontraba David Zebedeo. Cuando le contó a David lo que había sucedido, los dos regresaron precipitadamente a las tiendas de los apóstoles dormidos e informaron a los ocho que el Maestro había sido traicionado y detenido.

\par 
%\textsuperscript{(1975.3)}
\textsuperscript{183:3.10} Casi en el mismo momento en que los ocho apóstoles eran despertados, los que habían huido por la hondonada arriba empezaron a regresar, y todos se reunieron cerca del lagar para discutir lo que había que hacer. Mientras tanto, Simón Pedro y Juan Zebedeo, que se habían ocultado entre los olivos, ya habían empezado a seguir al grupo de soldados, guardias y sirvientes, que ahora conducían a Jesús de regreso a Jerusalén como si llevaran a un criminal capaz de cualquier cosa. Juan seguía de cerca al grupo, pero Pedro iba detrás a más distancia. Después de escapar de las garras del soldado, Juan Marcos se procuró un manto que había encontrado en la tienda de Simón Pedro y Juan Zebedeo. Sospechaba que los guardias llevarían a Jesús a la casa de Anás, el sumo sacerdote jubilado; así pues, bordeó los huertos de olivos y llegó antes que el grupo al palacio del sumo sacerdote, donde se escondió cerca de la entrada principal.

\section*{4. La discusión en el lagar}
\par 
%\textsuperscript{(1975.4)}
\textsuperscript{183:4.1} Santiago Zebedeo se encontró separado de Simón Pedro y de su hermano Juan, de manera que se unió a los otros apóstoles y sus compañeros de campamento en el lagar para deliberar sobre lo que debían hacer en vista del arresto del Maestro.

\par 
%\textsuperscript{(1975.5)}
\textsuperscript{183:4.2} Andrés había sido liberado de toda responsabilidad como director del grupo de sus compañeros apóstoles; en consecuencia, en esta crisis, que era la más grave de sus vidas, permanecía en silencio. Después de una breve discusión informal, Simón Celotes se subió en el muro de piedra del lagar y, después de hacer una apasionada defensa a favor de la lealtad al Maestro y a la causa del reino, exhortó a sus compañeros apóstoles y a los otros discípulos a que corrieran detrás de la tropa y rescataran a Jesús. La mayoría del grupo habría estado dispuesta a seguir su conducta agresiva si no hubiera sido por la advertencia de Natanael, el cual se levantó en cuanto Simón terminó de hablar y llamó la atención de todos sobre las enseñanzas tantas veces repetidas de Jesús en relación con la no resistencia. Les recordó además que Jesús les había ordenado aquella misma noche que protegieran sus vidas hasta el momento en que salieran al mundo para proclamar la buena nueva del evangelio del reino celestial. Santiago Zebedeo apoyó esta actitud de Natanael, contando ahora cómo Pedro y otros habían sacado la espada para impedir el arresto del Maestro, y cómo Jesús había pedido a Simón Pedro y a sus compañeros armados que envainaran sus hojas. Mateo y Felipe también dieron sus discursos, pero nada concreto surgió de esta discusión hasta que Tomás llamó la atención de todos sobre el hecho de que Jesús había aconsejado a Lázaro que no se expusiera a la muerte; les indicó que no podían hacer nada por salvar a su Maestro puesto que éste se había negado a permitir que sus amigos lo defendieran, y persistía en abstenerse de utilizar sus poderes divinos para burlar a sus enemigos humanos. Tomás los persuadió para que se dispersaran cada uno por su lado, con el acuerdo de que David Zebedeo permanecería en el campamento para mantener un centro de intercambio de información y un cuartel general de mensajeros para el grupo. A las dos y media de aquella mañana, el campamento se quedaba desierto; sólo David permanecía allí con tres o cuatro mensajeros, después de haber enviado a los demás para que obtuvieran información sobre dónde habían llevado a Jesús y qué iban a hacer con él.

\par 
%\textsuperscript{(1976.1)}
\textsuperscript{183:4.3} Cinco apóstoles ---Natanael, Mateo, Felipe y los gemelos--- fueron a esconderse en Betfagé y Betania. Tomás, Andrés, Santiago y Simón Celotes se escondieron en la ciudad. Simón Pedro y Juan Zebedeo siguieron adelante hasta la casa de Anás.

\par 
%\textsuperscript{(1976.2)}
\textsuperscript{183:4.4} Poco después del amanecer, Simón Pedro, con una imagen abatida de profunda desesperación, regresó vagando al campamento de Getsemaní. David lo envió a cargo de un mensajero para que se reuniera con su hermano Andrés, que estaba en la casa de Nicodemo en Jerusalén.

\par 
%\textsuperscript{(1976.3)}
\textsuperscript{183:4.5} Hasta el final mismo de la crucifixión, Juan Zebedeo permaneció siempre cerca, tal como Jesús se lo había ordenado, y era él quien de hora en hora suministraba a los mensajeros la información que llevaban a David en el campamento del jardín, y que luego se transmitía a los apóstoles escondidos y a la familia de Jesús.

\par 
%\textsuperscript{(1976.4)}
\textsuperscript{183:4.6} Ciertamente, ¡el pastor es golpeado y las ovejas se dispersan! Aunque todos se dan vagamente cuenta de que Jesús les había avisado de esta precisa situación, están muy severamente conmocionados por la repentina desaparición del Maestro como para poder utilizar su mente de manera normal.

\par 
%\textsuperscript{(1976.5)}
\textsuperscript{183:4.7} Poco después del amanecer, y justo después de que Pedro hubiera sido enviado a reunirse con su hermano, Judá, el hermano carnal de Jesús, llegó al campamento casi sin aliento y por delante del resto de la familia de Jesús, para enterarse simplemente de que el Maestro ya había sido arrestado, y descendió apresuradamente la carretera de Jericó para llevar esta información a su madre y a sus hermanos y hermanas. David Zebedeo avisó a la familia de Jesús, por medio de Judá, de que se reunieran en la casa de Marta y María en Betania, y esperaran allí las noticias que sus mensajeros les llevarían con regularidad.

\par 
%\textsuperscript{(1976.6)}
\textsuperscript{183:4.8} Ésta era la situación durante la última mitad de la noche del jueves y las primeras horas de la mañana del viernes en lo que concierne a los apóstoles, los discípulos principales y la familia terrenal de Jesús. Todos estos grupos y personas se mantenían en contacto los unos con los otros gracias al servicio de mensajeros que David Zebedeo continuaba dirigiendo desde su cuartel general en el campamento de Getsemaní.

\section*{5. Camino del palacio del sumo sacerdote}
\par 
%\textsuperscript{(1977.1)}
\textsuperscript{183:5.1} Antes de partir del jardín con Jesús, se originó una discusión entre el capitán judío de los guardias del templo y el capitán romano de la compañía de soldados en cuanto al lugar donde debían llevar a Jesús. El capitán de los guardias del templo dio órdenes para que se le llevara ante Caifás, el sumo sacerdote en ejercicio. El capitán de los soldados romanos ordenó que Jesús fuera llevado al palacio de Anás, el antiguo sumo sacerdote y suegro de Caifás. Y lo hizo así porque los romanos tenían la costumbre de tratar directamente con Anás todas las cuestiones relacionadas con la aplicación de las leyes eclesiásticas judías. Y se obedecieron las órdenes del capitán romano; llevaron a Jesús a la casa de Anás para someterlo a un interrogatorio preliminar.

\par 
%\textsuperscript{(1977.2)}
\textsuperscript{183:5.2} Judas caminaba al lado de los capitanes, escuchando todo lo que se decía, pero sin participar en la discusión, porque ni el capitán judío ni el oficial romano querían siquiera hablar con el traidor ---de tal manera lo despreciaban.

\par 
%\textsuperscript{(1977.3)}
\textsuperscript{183:5.3} Casi en aquel momento, Juan Zebedeo recordó las instrucciones de su Maestro de que permaneciera siempre cerca, y se aproximó apresuradamente a Jesús que caminaba entre los dos capitanes. Al ver que Juan se ponía a su lado, el comandante de los guardias del templo dijo a su asistente: «Coge a este hombre y átalo. Es uno de los seguidores de este tipo». Pero cuando el capitán romano escuchó esto, volvió la cabeza, vio a Juan, y dio órdenes para que el apóstol se pusiera a su lado y que nadie lo molestara. Luego el capitán romano le dijo al capitán judío: «Este hombre no es ni un traidor ni un cobarde. Lo he visto en el jardín y no sacó la espada para oponer resistencia. Tiene el coraje de adelantarse para estar con su Maestro, y nadie le pondrá la mano encima. La ley romana permite que todo preso pueda tener al menos a un amigo que permanezca con él delante del tribunal, y no se impedirá que este hombre esté al lado de su Maestro, el detenido». Cuando Judas escuchó esto, se sintió tan avergonzado y humillado que se fue quedando detrás de la comitiva y llegó solo al palacio de Anás.

\par 
%\textsuperscript{(1977.4)}
\textsuperscript{183:5.4} Esto explica por qué se le permitió a Juan Zebedeo permanecer cerca de Jesús a lo largo de las duras experiencias de aquella noche y del día siguiente. Los judíos temían decirle algo a Juan o molestarlo de alguna manera, porque en cierto modo tenía la condición de un consejero romano designado para actuar como observador de las operaciones del tribunal eclesiástico judío. La posición privilegiada de Juan quedó aún más asegurada cuando, en el momento de entregar a Jesús al capitán de los guardias del templo en la puerta del palacio de Anás, el capitán romano se dirigió a su asistente y le dijo: «Acompaña a este preso y asegúrate de que estos judíos no lo maten sin el consentimiento de Pilatos. Cuida de que no lo asesinen, y asegúrate de que a su amigo, el galileo, le permitan permanecer a su lado para observar todo lo que suceda». Así es como Juan pudo estar cerca de Jesús hasta el momento de su muerte en la cruz, aunque los otros diez apóstoles estuvieron obligados a permanecer ocultos. Juan actuaba bajo la protección romana, y los judíos no se atrevieron a molestarlo hasta después de la muerte del Maestro.

\par 
%\textsuperscript{(1977.5)}
\textsuperscript{183:5.5} Durante todo el trayecto hasta el palacio de Anás, Jesús no abrió la boca. Desde el momento de su arresto hasta su aparición delante de Anás, el Hijo del Hombre no dijo ni una palabra.