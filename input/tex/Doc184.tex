\chapter{Documento 184. Ante el tribunal del sanedrín}
\par 
%\textsuperscript{(1978.1)}
\textsuperscript{184:0.1} UNOS representantes de Anás habían ordenado en secreto al capitán de los soldados romanos que llevara a Jesús al palacio de Anás inmediatamente después de arrestarlo. El antiguo sumo sacerdote deseaba mantener su prestigio como principal autoridad eclesiástica de los judíos. También tenía otro objetivo al retener a Jesús en su casa durante varias horas, y era ganar tiempo para reunir legalmente al tribunal del sanedrín. No era legal convocar el tribunal del sanedrín antes de la hora de la ofrenda del sacrificio matutino en el templo, y este sacrificio se ofrecía hacia las tres de la mañana.

\par 
%\textsuperscript{(1978.2)}
\textsuperscript{184:0.2} Anás sabía que un tribunal de sanedristas estaba esperando en el palacio de su yerno Caifás. Unos treinta miembros del sanedrín se habían reunido a medianoche en la casa del sumo sacerdote a fin de estar preparados para juzgar a Jesús cuando fuera traído ante ellos. Únicamente se habían reunido aquellos miembros que se oponían enérgica y abiertamente a Jesús y a sus enseñanzas, puesto que sólo se necesitaban veintitrés para constituir un tribunal procesal.

\par 
%\textsuperscript{(1978.3)}
\textsuperscript{184:0.3} Jesús pasó unas tres horas en el palacio de Anás en el monte Olivete, no lejos del jardín de Getsemaní, donde lo habían arrestado. Juan Zebedeo estaba libre y a salvo en el palacio de Anás, no solamente debido a la palabra del capitán romano, sino también porque él y su hermano Santiago eran bien conocidos por los criados más viejos, pues habían sido invitados muchas veces al palacio, ya que el antiguo sumo sacerdote era un pariente lejano de su madre Salomé.

\section*{1. El interrogatorio de Anás}
\par 
%\textsuperscript{(1978.4)}
\textsuperscript{184:1.1} Enriquecido por los ingresos del templo, con su yerno como sumo sacerdote en ejercicio, y debido a sus relaciones con las autoridades romanas, Anás era en verdad el individuo más poderoso de la sociedad judía. Era un planificador y un conspirador sofisticado y diplomático. Deseaba dirigir este asunto para deshacerse de Jesús; temía confiar enteramente una empresa tan importante como ésta a su brusco y agresivo yerno. Anás quería asegurarse de que el juicio del Maestro permanecería entre las manos de los saduceos; temía la posible simpatía de algunos fariseos, ya que prácticamente todos los miembros del sanedrín que habían abrazado la causa de Jesús eran fariseos.

\par 
%\textsuperscript{(1978.5)}
\textsuperscript{184:1.2} Anás no había visto a Jesús desde hacía varios años, desde la época en que el Maestro se presentó en su casa y se marchó inmediatamente al observar la frialdad y la reserva con que lo recibió. Anás había pensado aprovecharse de esta antigua relación e intentar de este modo persuadir a Jesús para que abandonara sus pretensiones y se fuera de Palestina. Le repugnaba participar en el asesinato de un hombre bueno y había razonado que quizás Jesús escogería dejar el país en lugar de sufrir la muerte. Pero cuando Anás se encontró delante del fornido y resuelto galileo, supo enseguida que hacer tales proposiciones sería inútil. Jesús era aún más majestuoso y bien equilibrado de lo que Anás recordaba.

\par 
%\textsuperscript{(1979.1)}
\textsuperscript{184:1.3} Cuando Jesús era joven, Anás se había interesado mucho por él, pero ahora sus ingresos estaban amenazados por lo que Jesús había hecho tan recientemente echando del templo a los cambistas y a otros mercaderes. Este acto había suscitado la enemistad del antiguo sumo sacerdote mucho más que las enseñanzas de Jesús.

\par 
%\textsuperscript{(1979.2)}
\textsuperscript{184:1.4} Anás entró en su espaciosa sala de audiencias, se sentó en un gran sillón y ordenó que trajeran a Jesús ante él. Después de observar al Maestro en silencio durante unos momentos, dijo: «Comprenderás que habrá que hacer algo con respecto a tu enseñanza, puesto que estás perturbando la paz y el orden de nuestro país». Mientras Anás miraba de manera indagadora a Jesús, el Maestro lo miró directamente a los ojos, pero no respondió. Anás dijo de nuevo: «¿Cuáles son los nombres de tus discípulos, además de Simón Celotes, el agitador?» Jesús lo miró de nuevo, pero no contestó.

\par 
%\textsuperscript{(1979.3)}
\textsuperscript{184:1.5} Anás estaba considerablemente molesto porque Jesús se negaba a contestar a sus preguntas, de tal manera que le dijo: «¿No te preocupa que yo sea benévolo o no contigo? ¿No tienes consideración por el poder que tengo para determinar las cuestiones de tu próximo juicio?» Cuando Jesús escuchó esto, dijo: «Anás, sabes que no podrías tener ningún poder sobre mí si no fuera permitido por mi Padre. Algunos quisieran matar al Hijo del Hombre porque son ignorantes y no conocen nada mejor; pero tú, amigo, sabes lo que estás haciendo. ¿Cómo puedes, por tanto, rechazar la luz de Dios?»

\par 
%\textsuperscript{(1979.4)}
\textsuperscript{184:1.6} Anás se quedó casi desconcertado por la manera amable en que Jesús le habló. Pero ya había decidido en su interior que Jesús debía irse de Palestina o morir; así pues, reunió su coraje y preguntó: «¿Qué es exactamente lo que intentas enseñar a la gente? ¿Qué pretendes ser?» Jesús contestó: «Sabes muy bien que he hablado abiertamente al mundo. He enseñado en las sinagogas y muchas veces en el templo, donde todos los judíos y muchos gentiles me han escuchado. No he dicho nada en secreto; entonces, ¿por qué me preguntas sobre mi enseñanza? ¿Por qué no llamas a los que me han escuchado y les preguntas a ellos? Mira, todo Jerusalén ha oído lo que he dicho, aunque tú mismo no hayas escuchado estas enseñanzas». Pero antes de que Anás pudiera responder, el administrador principal del palacio, que estaba cerca, abofeteó a Jesús, diciendo: «¿Cómo te atreves a contestarle así al sumo sacerdote?» Anás no reprendió a su administrador, pero Jesús se dirigió a él, diciendo: «Amigo mío, si he hablado mal, testifica contra el mal; pero si he dicho la verdad, entonces ¿por qué me golpeas?»\footnote{\textit{Interrogatorio antes Anás}: Jn 18:19-23.}

\par 
%\textsuperscript{(1979.5)}
\textsuperscript{184:1.7} Anás lamentaba que su administrador hubiera abofeteado a Jesús, pero era demasiado orgulloso como para tener en cuenta el asunto. En su confusión, se fue a otra habitación y dejó solo a Jesús casi una hora con los criados de la casa y los guardias del templo.

\par 
%\textsuperscript{(1979.6)}
\textsuperscript{184:1.8} Cuando regresó, se puso al lado del Maestro y dijo: «¿Pretendes ser el Mesías, el libertador de Israel?» Jesús dijo: «Anás, me conoces desde la época de mi juventud. Sabes que no pretendo ser nada más que lo que mi Padre ha decretado, y que he sido enviado a todos los hombres, tanto gentiles como judíos». Entonces Anás dijo: «Me han dicho que has pretendido ser el Mesías; ¿es verdad?» Jesús miró a Anás pero se limitó a contestar: «Tú lo has dicho».\footnote{\textit{Conversación con Anás}: Mt 26:63b-64; Mc 14:61b-62; Lc 22:67-68,70.}

\par 
%\textsuperscript{(1980.1)}
\textsuperscript{184:1.9} Aproximadamente en ese momento, unos mensajeros del palacio de Caifás llegaron para preguntar a qué hora sería llevado Jesús ante el tribunal del sanedrín, y puesto que faltaba poco para el amanecer, Anás pensó que sería mejor enviar a Jesús, atado y custodiado por los guardias del templo, a Caifás\footnote{\textit{Jesús llevado ante Caifás}: Mt 26:57; Mc 14:53; Lc 22:54; Jn 18:24.}. Él mismo los siguió un poco después.

\section*{2. Pedro en el patio}
\par 
%\textsuperscript{(1980.2)}
\textsuperscript{184:2.1} Cuando el grupo de guardias y soldados se acercaba a la entrada del palacio de Anás, Juan Zebedeo caminaba al lado del capitán de los soldados romanos. Judas se había quedado rezagado a cierta distancia, y Simón Pedro los seguía a lo lejos. Después de que Juan hubiera entrado en el patio del palacio con Jesús y los guardias, Judas se acercó a la cancela pero, al ver a Jesús y a Juan, continuó hacia la casa de Caifás, donde sabía que el verdadero juicio del Maestro se llevaría a cabo más tarde. Poco después de que Judas se hubiera marchado, llegó Simón Pedro, y mientras permanecía delante de la cancela, Juan lo vio en el momento en que estaban a punto de meter a Jesús en el palacio. La portera que estaba encargada de la cancela conocía a Juan, y cuando éste le pidió que dejara entrar a Pedro, ella asintió con placer.

\par 
%\textsuperscript{(1980.3)}
\textsuperscript{184:2.2} Al entrar en el patio, Pedro se dirigió hacia el fuego de carbón e intentó calentarse porque la noche era fría\footnote{\textit{Pedro calentándose}: Mc 14:54b; Lc 22:55; Jn 18:18.}. Se sentía aquí totalmente fuera de lugar entre los enemigos de Jesús, y en verdad no estaba en su sitio. El Maestro no le había ordenado que se mantuviera cerca tal como se lo había recomendado a Juan. Pedro pertenecía al grupo de apóstoles que habían sido expresamente advertidos que no arriesgaran su vida durante estas horas del juicio y de la crucifixión de su Maestro.

\par 
%\textsuperscript{(1980.4)}
\textsuperscript{184:2.3} Pedro había tirado su espada poco antes de llegar a la cancela del palacio, de manera que entró desarmado en el patio de Anás. Su mente era un torbellino de confusión; apenas podía darse cuenta de que Jesús había sido arrestado. No conseguía captar la realidad de la situación ---que se encontraba allí en el patio de Anás, calentándose al lado de los criados del sumo sacerdote. Se preguntaba qué estarían haciendo los demás apóstoles y, al darle vueltas en la cabeza a la forma en que Juan había sido admitido en el palacio, llegó a la conclusión de que los criados lo conocían, puesto que Juan le había pedido a la portera que lo dejara entrar.

\par 
%\textsuperscript{(1980.5)}
\textsuperscript{184:2.4} Poco después de que la portera dejara entrar a Pedro, y mientras éste se calentaba junto al fuego, ella se le acercó y le dijo maliciosamente: «¿No eres tú también uno de los discípulos de ese hombre?» Pedro no debería haberse sorprendido de ser reconocido de esta manera, ya que había sido Juan quien le había pedido a la muchacha que le dejara traspasar las cancelas del palacio; pero estaba en tal estado de tensión nerviosa que el ser identificado como discípulo lo desequilibró, y con un solo pensamiento prioritario en su mente ---la idea de escapar con vida--- respondió de inmediato a la pregunta de la sirvienta, diciendo: «No lo soy»\footnote{\textit{Primera negación}: Mt 26:69-70; Mc 14:66-68a; Lc 22:56-57; Jn 18:17.}.

\par 
%\textsuperscript{(1980.6)}
\textsuperscript{184:2.5} Muy pronto, otro criado se acercó a Pedro y le preguntó: «¿No te he visto en el jardín cuando arrestaron a ese tipo? ¿No eres tú también uno de sus seguidores?» Pedro estaba ahora totalmente alarmado; no veía la manera de escapar sano y salvo de estos acusadores; negó pues con vehemencia toda conexión con Jesús, diciendo: «No conozco a ese hombre, ni soy uno de sus seguidores»\footnote{\textit{Segunda negación}: Lc 22:58; Jn 18:25.}.

\par 
%\textsuperscript{(1980.7)}
\textsuperscript{184:2.6} Casi en ese momento, la portera de la cancela llevó a Pedro a un lado y le dijo: «Estoy segura de que eres un discípulo de ese Jesús, no solamente porque uno de sus seguidores me ha pedido que te dejara entrar en el patio, sino que mi hermana que está aquí te ha visto en el templo con ese hombre. ¿Por qué lo niegas?» Cuando Pedro escuchó la acusación de la sirvienta, negó totalmente conocer a Jesús con muchas maldiciones y juramentos, diciendo de nuevo: «No soy un seguidor de ese hombre; ni siquiera lo conozco; nunca he oído hablar de él anteriormente»\footnote{\textit{Tercera negación}: Mt 26:73-74a; Mc 14:70b-71; Lc 22:59-60a; Jn 18:26-27a.}.

\par 
%\textsuperscript{(1981.1)}
\textsuperscript{184:2.7} Pedro se alejó del fuego durante un momento mientras caminaba por el patio. Le hubiera gustado escaparse, pero temía atraer la atención. Como tenía frío, regresó junto al fuego, y uno de los hombres que estaban cerca de él dijo: «Tú eres sin duda uno de los discípulos de ese hombre. Ese Jesús es galileo, y tu manera de hablar te traiciona, pues hablas también como un galileo». Y Pedro negó de nuevo toda conexión con su Maestro.

\par 
%\textsuperscript{(1981.2)}
\textsuperscript{184:2.8} Pedro estaba tan inquieto que intentó evitar el contacto con sus acusadores alejándose del fuego y permaneciendo solo en el porche. Después de más de una hora de aislamiento, la portera y su hermana lo encontraron por casualidad, y las dos le tomaron el pelo de nuevo acusándolo de ser un seguidor de Jesús. Y otra vez negó la acusación. Justo cuando había negado una vez más toda conexión con Jesús, el gallo cantó\footnote{\textit{El gallo cantó}: Mt 26:74b; Mc 14:68b,72a; Lc 22:60b; Jn 18:27b.}, y Pedro recordó\footnote{\textit{Pedro recuerda}: Mt 26:75a; Mc 14:72b; Lc 22:61b.} las palabras de advertencia que su Maestro\footnote{\textit{Las palabras de advertencia}: Mt 26:34; Mc 14:30; Lc 22:34; Jn 13:38.} le había dicho anteriormente aquella misma noche. Mientras permanecía allí, acongojado y abatido por el sentimiento de culpa, las puertas del palacio se abrieron y salieron los guardias conduciendo a Jesús hacia la casa de Caifás. Al pasar cerca de Pedro, el Maestro vio, a la luz de las antorchas, la cara de desesperación de su apóstol, anteriormente tan seguro de sí mismo y superficialmente valiente; volvió la cabeza y miró a Pedro\footnote{\textit{Jesús mira a Pedro}: Lc 22:61a.}. Pedro nunca olvidó aquella mirada durante toda su vida. Fue una mirada de compasión y de amor a la vez como ningún hombre mortal había visto nunca en el rostro del Maestro.

\par 
%\textsuperscript{(1981.3)}
\textsuperscript{184:2.9} Después de que Jesús y los guardias hubieron franqueado las cancelas del palacio, Pedro los siguió, pero sólo durante una corta distancia. No pudo ir más allá. Se sentó a un lado del camino y lloró amargamente\footnote{\textit{Pedro llora amargamente}: Mt 26:75b; Mc 14:72c; Lc 22:62.}. Después de derramar estas lágrimas de desesperación, volvió sobre sus pasos hacia el campamento con la esperanza de encontrar a su hermano, Andrés. Al llegar al campamento, sólo encontró a David Zebedeo, el cual envió a un mensajero para indicarle el camino hasta el lugar donde se había escondido su hermano en Jerusalén.

\par 
%\textsuperscript{(1981.4)}
\textsuperscript{184:2.10} Toda la experiencia de Pedro tuvo lugar en el patio del palacio de Anás en el monte Olivete. No siguió a Jesús hasta el palacio del sumo sacerdote Caifás. El hecho de que Pedro cayera en la cuenta, con el canto de un gallo, de que había negado repetidas veces a su Maestro, indica que todo esto sucedió fuera de Jerusalén, puesto que la ley prohibía tener aves de corral dentro de los límites de la ciudad.

\par 
%\textsuperscript{(1981.5)}
\textsuperscript{184:2.11} Hasta que el canto del gallo no devolvió a Pedro su sentido común, sólo había pensado, mientras iba y venía por el porche para entrar en calor, en la habilidad con que había eludido las acusaciones de los criados, y en cómo había frustrado sus intenciones de identificarlo con Jesús. De momento, sólo había considerado que aquellos criados no tenían el derecho moral ni legal de interrogarlo así, y se felicitaba realmente por la manera en que creía que había evitado ser identificado y quizás arrestado y encarcelado. A Pedro no se le ocurrió que había negado a su Maestro hasta que el gallo cantó. Únicamente cuando Jesús lo miró se dio cuenta de que no había estado a la altura de sus privilegios como embajador del reino.

\par 
%\textsuperscript{(1981.6)}
\textsuperscript{184:2.12} Después de dar el primer paso en el camino del compromiso y de la menor resistencia, a Pedro no parecía quedarle más salida que continuar con el tipo de conducta que había decidido. Se necesita un carácter grande y noble para cambiar de opinión y retomar el camino recto después de haber empezado mal. Demasiado a menudo, nuestra propia mente tiende a justificar nuestra permanencia en el camino erróneo después de haber entrado en él.

\par 
%\textsuperscript{(1982.1)}
\textsuperscript{184:2.13} Pedro nunca creyó por completo que podría ser perdonado hasta el momento en que se encontró con su Maestro después de la resurrección, y vio que era acogido como antes de las experiencias de esta trágica noche de negaciones.

\section*{3. Ante el tribunal de los sanedristas}
\par 
%\textsuperscript{(1982.2)}
\textsuperscript{184:3.1} Eran aproximadamente las tres y media de la madrugada de este viernes cuando el sumo sacerdote, Caifás, declaró constituido el tribunal sanedrista\footnote{\textit{El Sanedrín se reúne}: Lc 22:66a.} de investigación y pidió que Jesús fuera traído ante ellos para ser juzgado oficialmente. En tres ocasiones anteriores, el sanedrín, por una gran mayoría de votos, había decretado la muerte de Jesús, había decidido que merecía la muerte basándose en las acusaciones irregulares de transgredir la ley, blasfemar y burlarse de las tradiciones de los padres de Israel.

\par 
%\textsuperscript{(1982.3)}
\textsuperscript{184:3.2} Esta reunión del sanedrín no se había convocado de manera regular y no se celebraba en el lugar habitual, la cámara de piedras labradas del templo. Se trataba de un tribunal especial compuesto por unos treinta sanedristas, y se había convocado en el palacio del sumo sacerdote. Juan Zebedeo estuvo presente con Jesús durante todo este pretendido juicio.

\par 
%\textsuperscript{(1982.4)}
\textsuperscript{184:3.3} ¡Cuánto se vanagloriaban estos jefes de los sacerdotes, escribas, saduceos y algunos fariseos, de que este Jesús que había comprometido su posición social y desafiado su autoridad, estuviera ahora seguro entre sus manos! Y estaban decididos a no dejarlo escapar vivo de sus garras vengativas.

\par 
%\textsuperscript{(1982.5)}
\textsuperscript{184:3.4} Normalmente, cuando los judíos juzgaban a un hombre por un delito capital, procedían con una gran cautela y proporcionaban todas las garantías de la equidad en la selección de los testigos y en toda la conducta del juicio. Pero en esta ocasión, Caifás era más un fiscal que un juez imparcial.

\par 
%\textsuperscript{(1982.6)}
\textsuperscript{184:3.5} Jesús apareció ante este tribunal vestido con su ropa habitual y con las manos atadas detrás de la espalda\footnote{\textit{Jesús es llevado dentro}: Lc 22:66b.}. Todo el tribunal se quedó sorprendido y algo confuso por su aspecto majestuoso. Nunca habían contemplado a un preso semejante ni habían presenciado aquella sangre fría en un hombre que podía perder la vida.

\par 
%\textsuperscript{(1982.7)}
\textsuperscript{184:3.6} La ley judía exigía que al menos dos testigos estuvieran de acuerdo\footnote{\textit{Dos testigos de acuerdo}: Mt 26:59-60a; Mc 14:55-56.} en un punto cualquiera antes de que se pudiera hacer una acusación contra un preso. Judas no podía servir de testigo contra Jesús, porque la ley judía prohibía expresamente el testimonio de un traidor. Más de veinte falsos testigos estaban disponibles para testificar contra Jesús, pero sus declaraciones eran tan contradictorias y tan evidentemente inventadas que los mismos sanedristas se sentían bastante avergonzados con el espectáculo. Jesús permanecía allí de pie, mirando con benignidad a aquellos perjuros, y su mismo semblante desconcertaba a los testigos mentirosos. Durante todos estos falsos testimonios, el Maestro no dijo ni una sola palabra; no respondió a ninguna de sus numerosas falsas acusaciones.

\par 
%\textsuperscript{(1982.8)}
\textsuperscript{184:3.7} La primera vez que dos de los testigos\footnote{\textit{El acuerdo de los testigos}: Mt 26:60b-61; Mc 14:57-59.} se acercaron algo a una apariencia de acuerdo fue cuando dos hombres atestiguaron que habían escuchado decir a Jesús, en el transcurso de uno de sus discursos en el templo, que «destruiría este templo hecho por las manos del hombre y en tres días construiría otro templo sin emplear las manos del hombre»\footnote{\textit{Destruid este templo, y lo reconstruiré}: Mt 26:61; 27:40; Mc 14:58; 15:29; Jn 2:19.}. Aquello no era exactamente lo que Jesús había dicho, aparte del hecho de que había señalado su propio cuerpo cuando hizo aquel comentario.

\par 
%\textsuperscript{(1982.9)}
\textsuperscript{184:3.8} Aunque el sumo sacerdote le gritó a Jesús: «¿No contestas a ninguna de estas acusaciones?», Jesús no abrió la boca\footnote{\textit{Jesús no se defiende}: Mt 26:62-63a; Mc 14:60-61a.}. Permaneció allí en silencio mientras todos aquellos falsos testigos prestaban sus declaraciones. El odio, el fanatismo y la exageración sin escrúpulos caracterizaban de tal manera las palabras de aquellos perjuros, que sus testimonios caían por su propio peso. La mejor refutación de aquellas falsas acusaciones era el silencio sosegado y majestuoso del Maestro.

\par 
%\textsuperscript{(1983.1)}
\textsuperscript{184:3.9} Anás llegó poco después de que los falsos testigos empezaran a declarar, y tomó asiento al lado de Caifás. Anás se levantó entonces para argumentar que aquella amenaza de Jesús de destruir el templo era suficiente para justificar tres cargos contra él\footnote{\textit{Los cargos establecidos}: Jn 2:19.}:

\par 
%\textsuperscript{(1983.2)}
\textsuperscript{184:3.10} 1. Que era un peligroso embaucador del pueblo. Que les enseñaba cosas imposibles y que los engañaba de otras maneras.

\par 
%\textsuperscript{(1983.3)}
\textsuperscript{184:3.11} 2. Que era un revolucionario fanático pues abogaba por el empleo de la violencia contra el templo sagrado, porque ¿cómo podría destruirlo de otra manera?

\par 
%\textsuperscript{(1983.4)}
\textsuperscript{184:3.12} 3. Que enseñaba la magia, puesto que prometía construir un nuevo templo, y sin ayudarse con las manos.

\par 
%\textsuperscript{(1983.5)}
\textsuperscript{184:3.13} Todo el sanedrín ya estaba de acuerdo en que Jesús era culpable de unas transgresiones de las leyes judías que merecían la muerte, pero ahora les preocupaba más preparar unas acusaciones relacionadas con su conducta y sus enseñanzas, que justificaran la sentencia de muerte que Pilatos debería pronunciar contra su preso. Sabían que tenían que obtener el consentimiento del gobernador romano antes de poder ejecutar legalmente a Jesús. Anás se sentía inclinado a seguir el método de hacer aparecer a Jesús como un peligroso educador que no podía estar por la calle entre la gente.

\par 
%\textsuperscript{(1983.6)}
\textsuperscript{184:3.14} Pero Caifás ya no podía soportar más la vista del Maestro, que permanecía allí de pie con una serenidad perfecta y en un silencio absoluto. Pensó que conocía al menos una manera de incitar al preso a hablar. En consecuencia, se precipitó hacia Jesús, agitó un dedo acusador delante del rostro del Maestro, y dijo: «En nombre del Dios vivo, te ordeno que nos digas si eres el Libertador, el Hijo de Dios». Jesús contestó a Caifás: «Lo soy. Pronto iré hacia el Padre, y dentro de poco el Hijo del Hombre será revestido de poder y reinará de nuevo sobre las huestes del cielo»\footnote{\textit{Jesús confiesa, yo soy el libertador}: Mt 26:63b-64; Mc 14:61b-62; Lc 22:67,69-70.}.

\par 
%\textsuperscript{(1983.7)}
\textsuperscript{184:3.15} Cuando el sumo sacerdote escuchó a Jesús pronunciar estas palabras, se encolerizó enormemente, y rasgando sus vestiduras exteriores, exclamó: «¿Qué necesidad tenemos de más testigos? Mirad, ahora todos habéis escuchado la blasfemia de este hombre. ¿Qué pensáis ahora que podemos hacer con este blasfemo y transgresor de la ley?»\footnote{\textit{La respuesta del sumo sacerdote}: Mt 26:65-66; Mc 14:63-64; Lc 22:71.} Y todos contestaron al unísono: «Merece la muerte; que sea crucificado».\footnote{\textit{Desorden en el tribunal}: Mt 26:67; Mc 14:65; Lc 22:63.}

\par 
%\textsuperscript{(1983.8)}
\textsuperscript{184:3.16} Jesús no manifestó interés por ninguna de las preguntas que le hicieron cuando estaba delante de Anás o de los sanedristas, exceptuando la única pregunta relacionada con su misión donadora. Cuando se le preguntó si era el Hijo de Dios, contestó afirmativamente de manera instantánea e inequívoca.

\par 
%\textsuperscript{(1983.9)}
\textsuperscript{184:3.17} Anás deseaba que continuara el juicio y que se formularan unas acusaciones bien definidas en cuanto a la relación de Jesús con la ley y las instituciones romanas, para presentarlas posteriormente a Pilatos. Los consejeros estaban impacientes por terminar rápidamente este asunto, no sólo porque era el día de la preparación de la Pascua y no se podía hacer ningún trabajo seglar después del mediodía, sino también porque temían que Pilatos regresara en cualquier momento a Cesarea, la capital romana de Judea, puesto que sólo estaba en Jerusalén para la celebración de la Pascua.

\par 
%\textsuperscript{(1983.10)}
\textsuperscript{184:3.18} Pero Anás no logró conservar el control del tribunal. Después de que Jesús contestara tan inesperadamente a Caifás, el sumo sacerdote se adelantó y lo abofeteó. Anás se quedó verdaderamente impresionado cuando los otros miembros del tribunal escupieron a Jesús a la cara al salir de la sala, y muchos de ellos lo abofetearon burlonamente con la palma de la mano. Y así terminó, a las cuatro y media de la mañana, esta primera sesión del juicio de Jesús por parte de los sanedristas, en desorden y en medio de una confusión inaudita.

\par 
%\textsuperscript{(1984.1)}
\textsuperscript{184:3.19} Treinta falsos jueces llenos de prejuicios y cegados por la tradición, con sus falsos testigos, se atreven a sentarse a juzgar al justo Creador de un universo. Y estos acusadores apasionados están exasperados por el silencio majestuoso y el magnífico comportamiento de este Dios-hombre. Su silencio es terrible de soportar; su palabra es un reto intrépido. Permanece impasible ante sus amenazas e impávido ante sus ataques. El hombre juzga a Dios, pero incluso en ese momento Dios los ama y los salvaría si pudiera.

\section*{4. La hora de la humillación}
\par 
%\textsuperscript{(1984.2)}
\textsuperscript{184:4.1} En la cuestión de pronunciar una sentencia de muerte, la ley judía exigía que el tribunal celebrara dos sesiones. Esta segunda sesión debía tener lugar al día siguiente de la primera, y los miembros del tribunal debían pasar las horas intermedias ayunando y lamentándose. Pero estos hombres no podían esperar al día siguiente para confirmar su decisión de que Jesús debía morir. Sólo esperaron una hora. Mientras tanto, dejaron a Jesús en la sala de audiencia al cuidado de los guardias del templo, que junto con los criados del sumo sacerdote, se divirtieron acumulando todo tipo de indignidades sobre el Hijo del Hombre. Se burlaron de él, le escupieron y lo abofetearon cruelmente. Le golpeaban en la cara con una vara y luego le decían: «Profetiza, Libertador, y dinos quién te ha golpeado». Continuaron así durante una hora entera, ultrajando y maltratando a este hombre de Galilea que no ofrecía resistencia\footnote{\textit{Ultrajes contra Jesús}: Mt 26:67-68; Mc 14:65; Lc 22:63-65.}.

\par 
%\textsuperscript{(1984.3)}
\textsuperscript{184:4.2} Durante esta hora trágica de sufrimientos y de juicios burlescos a manos de los guardias y criados ignorantes e insensibles, Juan Zebedeo estuvo esperando a solas, lleno de terror, en una habitación contigua. Cuando empezaron estos abusos, Jesús le indicó a Juan con un gesto de la cabeza que debía retirarse. El Maestro sabía muy bien que si permitía a su apóstol permanecer en la sala presenciando estas indignidades, se despertaría en Juan tal resentimiento que le hubiera conducido a una explosión de protesta indignada que probablemente le hubiera costado la vida.

\par 
%\textsuperscript{(1984.4)}
\textsuperscript{184:4.3} Durante esta hora espantosa, Jesús no pronunció ni una palabra. Para este alma humana dulce y sensible, unida en una relación de personalidad con el Dios de todo este universo, no hubo un período más amargo en la copa de su humillación que esta hora terrible a merced de estos guardias y criados ignorantes y crueles, que se habían sentido estimulados a maltratarlo debido al ejemplo de los miembros de este pretendido tribunal sanedrista.

\par 
%\textsuperscript{(1984.5)}
\textsuperscript{184:4.4} El corazón humano quizás no puede concebir el escalofrío de indignación que recorrió un enorme universo, mientras las inteligencias celestiales presenciaban este espectáculo de su amado Soberano sometiéndose a la voluntad de sus criaturas ignorantes y desviadas, en la esfera ensombrecida por el pecado de la desafortunada Urantia.

\par 
%\textsuperscript{(1984.6)}
\textsuperscript{184:4.5} ¿Qué es esa característica animal en el hombre que le conduce a querer insultar y atacar físicamente aquello que no puede alcanzar espiritualmente ni conseguir intelectualmente? Aún se esconde en el hombre medio civilizado una malvada brutalidad que intenta desahogarse en aquellos que son superiores en sabiduría y en logros espirituales. Observad la malvada tosquedad y la brutal ferocidad de estos hombres supuestamente civilizados, mientras obtenían cierta forma de placer animal atacando físicamente al Hijo del Hombre que no ofrecía resistencia. Mientras estos insultos, burlas y golpes caían sobre Jesús, él no se defendía, pero no estaba indefenso. Jesús no estaba derrotado, se limitaba a no luchar en el sentido material.

\par 
%\textsuperscript{(1985.1)}
\textsuperscript{184:4.6} Éstos son los momentos de las mayores victorias del Maestro en toda su larga y extraordinaria carrera como autor, sostén y salvador de un enorme y extenso universo. Después de vivir hasta su plenitud una vida revelando Dios al hombre, Jesús está dedicado ahora a revelar el hombre a Dios de una manera nueva y sin precedentes. Jesús está revelando ahora a los mundos la victoria final sobre todos los temores del aislamiento de la personalidad que siente la criatura. El Hijo del Hombre ha conseguido finalmente realizar su identidad como Hijo de Dios. Jesús no duda en afirmar que él y el Padre son uno\footnote{\textit{Jesús y el Padre son uno}: Jn 1:1; 5:17-18; 10:30,38; 12:44-45; 14:7-11,20; 17:11,21-22.}; y basándose en el hecho y la verdad de esta experiencia suprema y celestial, exhorta a todo creyente en el reino a que se vuelva uno con él\footnote{\textit{Volverse uno con Jesús}: Jn 17:11,21-22.}, como él y su Padre son uno. La experiencia viviente en la religión de Jesús se convierte así en la técnica cierta y segura mediante la cual los mortales de la Tierra, espiritualmente aislados y cósmicamente solitarios, consiguen escapar del aislamiento de la personalidad, con todos sus efectos de temores y de sentimientos de impotencia asociados. En las realidades fraternales del reino de los cielos, los hijos de Dios por la fe encuentran su liberación final del aislamiento del yo, tanto de manera personal como planetaria. El creyente que conoce a Dios experimenta cada vez más el éxtasis y la grandeza de la socialización espiritual a escala del universo ---la ciudadanía en el cielo asociada a la realización eterna del destino divino consistente en alcanzar la perfección.

\section*{5. La segunda reunión del tribunal}
\par 
%\textsuperscript{(1985.2)}
\textsuperscript{184:5.1} El tribunal se reunió de nuevo a las cinco y media de la mañana184:05.01 \footnote{\textit{Segunda reunión del tribunal}: Mt 27:1; Mc 15:1a.}, y Jesús fue conducido a la habitación contigua donde estaba esperando Juan. Aquí, el soldado romano y los guardias del templo vigilaron a Jesús, mientras el tribunal empezaba a formular las acusaciones que se iban a presentar a Pilatos. Anás indicó claramente a sus asociados que la acusación de blasfemia no tendría ningún peso ante Pilatos. Judas estaba presente durante esta segunda reunión del tribunal, pero no prestó ninguna declaración.

\par 
%\textsuperscript{(1985.3)}
\textsuperscript{184:5.2} Esta sesión de la corte sólo duró media hora, y cuando levantaron la sesión para presentarse ante Pilatos, habían redactado la acusación contra Jesús estimando que merecía la muerte por tres razones:

\par 
%\textsuperscript{(1985.4)}
\textsuperscript{184:5.3} 1. Que pervertía a la nación judía; que engañaba al pueblo y lo incitaba a la rebelión.

\par 
%\textsuperscript{(1985.5)}
\textsuperscript{184:5.4} 2. Que enseñaba al pueblo a que se negara a pagar el tributo al César.

\par 
%\textsuperscript{(1985.6)}
\textsuperscript{184:5.5} 3. Que como pretendía ser rey y el fundador de un nuevo tipo de reino, incitaba a la traición contra el emperador.

\par 
%\textsuperscript{(1985.7)}
\textsuperscript{184:5.6} Todo este procedimiento era irregular y totalmente contrario a las leyes judías. No había dos testigos que se hubieran puesto de acuerdo en ninguna cuestión, excepto los que habían testificado en relación con la declaración de Jesús de que destruiría el templo y lo levantaría de nuevo en tres días. E incluso en este punto, ningún testigo había hablado en nombre de la defensa, y tampoco se le pidió a Jesús que explicara lo que había querido decir.

\par 
%\textsuperscript{(1985.8)}
\textsuperscript{184:5.7} El único punto sobre el que el tribunal podría haberlo juzgado coherentemente era el de la blasfemia, y hubiera estado basado enteramente en el propio testimonio del acusado. Incluso en lo que concierne a la blasfemia, no consiguieron votar oficialmente la pena de muerte.

\par 
%\textsuperscript{(1985.9)}
\textsuperscript{184:5.8} Y ahora, para presentarse ante Pilatos, se atrevían a formular tres cargos sobre los cuales ningún testigo había sido interrogado, y sobre los que se habían puesto de acuerdo en ausencia del acusado. Cuando todo estuvo hecho, tres de los fariseos se marcharon; querían que Jesús fuera aniquilado, pero no querían formular cargos contra él sin testigos y en su ausencia.

\par 
%\textsuperscript{(1986.1)}
\textsuperscript{184:5.9} Jesús no volvió a aparecer ante el tribunal de los sanedristas. Éstos no querían volver a contemplar su rostro mientras juzgaban su vida inocente. Jesús no se enteró (como hombre) de las acusaciones oficiales hasta que las escuchó de boca de Pilatos.

\par 
%\textsuperscript{(1986.2)}
\textsuperscript{184:5.10} Mientras Jesús estaba en la habitación con Juan y los guardias, y el tribunal celebraba su segunda sesión, algunas mujeres del palacio del sumo sacerdote vinieron con sus amigas para contemplar al extraño preso, y una de ellas le preguntó: «¿Eres el Mesías, el Hijo de Dios?» Y Jesús respondió: «Si te lo digo, no me creerás; y si te lo pregunto, no contestarás»\footnote{\textit{Preguntas de los fisgones}: Lc 22:67-68.}.

\par 
%\textsuperscript{(1986.3)}
\textsuperscript{184:5.11} A las seis de aquella mañana, Jesús fue sacado de la casa de Caifás para aparecer ante Pilatos\footnote{\textit{Jesús llevado a Pilatos}: Mt 27:2; Mc 15:1b; Lc 23:1; Jn 18:28a.}, a fin de que éste confirmara la sentencia de muerte que el tribunal de los sanedristas había decretado de manera tan injusta e irregular.