\chapter{Documento 188. El período en la tumba}
\par 
%\textsuperscript{(2012.1)}
\textsuperscript{188:0.1} EL DÍA y medio que el cuerpo mortal de Jesús estuvo en la tumba de José, el período entre su muerte en la cruz y su resurrección, es un capítulo de la carrera terrenal de Miguel que conocemos poco. Podemos narrar el entierro del Hijo del Hombre y contar en este relato los acontecimientos asociados con su resurrección, pero no podemos proporcionar mucha información auténtica sobre lo que sucedió realmente durante este intervalo de casi treinta y seis horas, desde las tres de la tarde del viernes hasta las tres de la mañana del domingo. Este período de la carrera del Maestro comenzó poco antes de que los soldados romanos lo bajaran de la cruz. Permaneció suspendido en la cruz cerca de una hora después de morir. Hubiera sido bajado antes si no se hubieran retrasado para rematar a los dos bandidos.

\par 
%\textsuperscript{(2012.2)}
\textsuperscript{188:0.2} Los dirigentes de los judíos habían planeado arrojar el cuerpo de Jesús a las fosas comunes abiertas de Gehena, al sur de la ciudad; era costumbre deshacerse así de las víctimas de la crucifixión. Si se hubiera seguido este plan, el cuerpo del Maestro habría estado expuesto a las bestias salvajes.

\par 
%\textsuperscript{(2012.3)}
\textsuperscript{188:0.3} Mientras tanto, José de Arimatea, acompañado de Nicodemo, había ido a ver a Pilatos para pedirle que les entregara el cuerpo de Jesús a fin de enterrarlo adecuadamente. No era raro que los amigos de las personas crucificadas ofrecieran sobornos a las autoridades romanas para obtener el privilegio de disponer de los cuerpos. José se presentó ante Pilatos con una gran suma de dinero, por si era necesario pagar la autorización de trasladar el cuerpo de Jesús a un sepulcro privado. Pero Pilatos no quiso aceptar dinero por esto. Cuando escuchó la petición, firmó rápidamente la orden que autorizaba a José a dirigirse al Gólgota y tomar inmediatamente plena posesión del cuerpo del Maestro. Mientras tanto, la tormenta de arena había amainado considerablemente, y un grupo de judíos que representaba al sanedrín había salido hacia el Gólgota con la intención de asegurarse de que el cuerpo de Jesús acompañaría a los de los bandidos hasta la fosa común pública y abierta.

\section*{1. El entierro de Jesús}
\par 
%\textsuperscript{(2012.4)}
\textsuperscript{188:1.1} Cuando José y Nicodemo llegaron al Gólgota, encontraron que los soldados estaban bajando a Jesús de la cruz y que los representantes del sanedrín estaban allí cerca para asegurarse de que ninguno de los seguidores de Jesús impediría que su cuerpo fuera llevado a la fosa común de los criminales. Cuando José presentó al centurión la orden de Pilatos para que le entregara el cuerpo del Maestro, los judíos causaron un alboroto y pidieron a gritos su posesión. En su frenesí, trataron de apoderarse del cuerpo por la fuerza; al ver esto, el centurión llamó a su lado a cuatro de sus soldados, y con las espadas desenvainadas permanecieron a horcajadas sobre el cuerpo del Maestro que yacía allí en el suelo. El centurión ordenó a los otros soldados que dejaran a los dos ladrones y que rechazaran a esta chusma irritada de judíos enfurecidos. Cuando se restableció el orden, el centurión leyó a los judíos el permiso de Pilatos, se apartó a un lado y le dijo a José: «Este cuerpo es tuyo para que hagas con él lo que creas conveniente. Yo y mis soldados nos quedaremos aquí para asegurarnos de que nadie se entremeta».

\par 
%\textsuperscript{(2013.1)}
\textsuperscript{188:1.2} Una persona crucificada no podía ser enterrada en un cementerio judío; había una ley que prohibía estrictamente esta manera de proceder. José y Nicodemo conocían esta ley, y en el camino hacia el Gólgota habían decidido enterrar a Jesús en el nuevo sepulcro de la familia de José, tallado en la roca maciza y situado a corta distancia al norte del Gólgota, al otro lado de la carretera que conducía a Samaria. Nadie había sido nunca enterrado en este sepulcro, y consideraron apropiado que el Maestro reposara allí. José creía realmente que Jesús resucitaría de entre los muertos, pero Nicodemo tenía muchas dudas. Estos antiguos miembros del sanedrín habían mantenido más o menos en secreto su fe en Jesús, aunque sus colegas sanedristas habían desconfiado de ellos desde hacía tiempo, incluso antes de que se retiraran del consejo. A partir de este momento se convirtieron en los discípulos más abiertos de Jesús en todo Jerusalén.

\par 
%\textsuperscript{(2013.2)}
\textsuperscript{188:1.3} Hacia las cuatro y media, el cortejo fúnebre de Jesús de Nazaret partió del Gólgota hacia el sepulcro de José, situado al otro lado de la carretera. El cuerpo estaba envuelto en una sábana de lino y lo llevaban cuatro hombres, seguidos por las fieles mujeres de Galilea que habían estado vigilando. Los mortales que llevaron hasta la tumba el cuerpo material de Jesús fueron: José, Nicodemo, Juan y el centurión romano.

\par 
%\textsuperscript{(2013.3)}
\textsuperscript{188:1.4} Transportaron el cuerpo hasta el sepulcro, una cámara cuadrada de unos tres metros de lado, donde lo prepararon rápidamente para sepultarlo. En realidad, los judíos no enterraban a sus muertos; los embalsamaban. José y Nicodemo habían traído grandes cantidades de mirra y áloes, y entonces envolvieron el cuerpo con unos vendajes empapados en estas soluciones. Cuando terminaron de embalsamarlo, ataron un paño alrededor de la cara, envolvieron el cuerpo en una sábana de lino y lo depositaron respetuosamente en una plataforma del sepulcro.

\par 
%\textsuperscript{(2013.4)}
\textsuperscript{188:1.5} Después de colocar el cuerpo en la tumba, el centurión hizo señas a sus soldados para que ayudaran a rodar la piedra de cierre delante de la entrada del sepulcro. Los soldados partieron después para Gehena con los cuerpos de los ladrones, mientras los demás regresaban entristecidos a Jerusalén para guardar la fiesta de la Pascua según las leyes de Moisés.

\par 
%\textsuperscript{(2013.5)}
\textsuperscript{188:1.6} El entierro de Jesús se llevó a cabo con una prisa y una precipitación extremas, porque era el día de la preparación y el sábado se acercaba rápidamente. Los hombres se apresuraron en regresar a la ciudad, pero las mujeres se quedaron cerca de la tumba hasta que se hizo de noche.

\par 
%\textsuperscript{(2013.6)}
\textsuperscript{188:1.7} Mientras sucedía todo esto, las mujeres estaban ocultas cerca de allí, de manera que lo vieron todo y observaron el lugar donde había sido enterrado el Maestro. Se habían escondido así porque a las mujeres no les estaba permitido asociarse con los hombres en momentos como éste. Estas mujeres pensaban que Jesús no había sido preparado adecuadamente para ser enterrado, y se pusieron de acuerdo para regresar a la casa de José, descansar el sábado, preparar los aromas y los ung\"uentos, y volver el domingo por la mañana para preparar convenientemente el cuerpo del Maestro para el descanso de la muerte. Las mujeres que permanecieron así cerca de la tumba este viernes por la noche fueron: María Magdalena, María la mujer de Clopas, Marta (otra hermana de la madre de Jesús) y Rebeca de Séforis.

\par 
%\textsuperscript{(2013.7)}
\textsuperscript{188:1.8} Aparte de David Zebedeo y José de Arimatea, muy pocos discípulos de Jesús creían o comprendían realmente que iba a resucitar de la tumba al tercer día.

\section*{2. La protección de la tumba}
\par 
%\textsuperscript{(2014.1)}
\textsuperscript{188:2.1} Aunque los seguidores de Jesús no pensaban en su promesa de resucitar de la tumba al tercer día, sus enemigos no lo olvidaban. Los jefes de los sacerdotes, los fariseos y los saduceos recordaban que habían recibido informes según los cuales había dicho que resucitaría de entre los muertos.

\par 
%\textsuperscript{(2014.2)}
\textsuperscript{188:2.2} Este viernes por la noche, después de la cena pascual, un grupo de dirigentes judíos se reunió hacia la medianoche en la casa de Caifás, donde discutieron de sus temores acerca de las afirmaciones del Maestro de que al tercer día resucitaría de entre los muertos. Esta reunión terminó con el nombramiento de un comité de sanedristas que iría a visitar a Pilatos a primeras horas del día siguiente, llevando la petición oficial del sanedrín de que se apostara una guardia romana delante de la tumba de Jesús para impedir que sus amigos trataran de forzarla. El portavoz de este comité le dijo a Pilatos: «Señor, nos acordamos de que ese farsante, Jesús de Nazaret, dijo mientras aún estaba vivo: `Al cabo de tres días resucitaré.' Por eso hemos venido ante ti para pedirte que des las órdenes oportunas para proteger el sepulcro contra sus seguidores, al menos hasta después del tercer día. Tenemos el gran temor de que sus discípulos vayan y lo roben durante la noche, para luego proclamar ante el pueblo que ha resucitado de entre los muertos. Si consentimos que suceda esto, este error podría ser mucho peor que haberle permitido seguir viviendo».

\par 
%\textsuperscript{(2014.3)}
\textsuperscript{188:2.3} Cuando Pilatos escuchó esta petición de los sanedristas, dijo: «Os daré una guardia de diez soldados. Seguid vuestro camino y asegurad la tumba». Regresaron al templo, cogieron a diez de sus propios guardias, y luego se dirigieron hacia la tumba de José con estos diez guardias judíos y los diez soldados romanos, aunque fuera sábado por la mañana, para ponerlos de vigilancia delante de la tumba. Estos hombres rodaron otra piedra más delante del sepulcro, y colocaron el sello de Pilatos en estas piedras y alrededor de ellas para que no fueran removidas sin que ellos lo supieran. Estos veinte hombres permanecieron de guardia hasta el momento de la resurrección, y los judíos les trajeron de comer y de beber.

\section*{3. Durante el sábado}
\par 
%\textsuperscript{(2014.4)}
\textsuperscript{188:3.1} Durante todo este sábado, los discípulos y los apóstoles permanecieron escondidos, mientras todo Jerusalén hablaba de la muerte de Jesús en la cruz. En este momento había en Jerusalén casi un millón y medio de judíos procedentes de todos los lugares del imperio romano y de Mesopotamia. Era el comienzo de la semana de la Pascua, y todos estos peregrinos estarían en la ciudad para enterarse de la resurrección de Jesús y llevar la noticia a sus tierras natales.

\par 
%\textsuperscript{(2014.5)}
\textsuperscript{188:3.2} A últimas horas del sábado por la noche, Juan Marcos llamó a los once apóstoles para que fueran en secreto a la casa de su padre; poco antes de la medianoche, todos se habían reunido en la misma sala de arriba donde dos noches antes habían compartido la Última Cena con su Maestro.

\par 
%\textsuperscript{(2014.6)}
\textsuperscript{188:3.3} Este sábado por la tarde, poco antes de ponerse el Sol, María la madre de Jesús, acompañada de Rut y de Judá, regresó a Betania para reunirse con su familia. David Zebedeo permaneció en la casa de Nicodemo, donde había hecho los arreglos para que sus mensajeros se reunieran allí el domingo por la mañana temprano. Las mujeres de Galilea, que habían preparado los aromas para embalsamar mejor el cuerpo de Jesús, permanecieron en la casa de José de Arimatea.

\par 
%\textsuperscript{(2014.7)}
\textsuperscript{188:3.4} En realidad, no somos capaces de explicar lo que le sucedió exactamente a Jesús de Nazaret durante este período de un día y medio en el que se suponía que estaba descansando en la nueva tumba de José de Arimatea. Aparentemente, murió en la cruz de la misma muerte natural que hubiera muerto cualquier otro mortal en las mismas circunstancias. Le oímos decir: «Padre, en tus manos encomiendo mi espíritu». No comprendemos plenamente el significado de esta declaración, puesto que su Ajustador del Pensamiento había sido personalizado desde hacía tiempo, y mantenía así una existencia separada del ser mortal de Jesús. Al Ajustador Personalizado del Maestro no le podía afectar de ninguna manera su muerte física en la cruz. Lo que Jesús puso por ahora en las manos del Padre debe haber sido el duplicado espiritual del trabajo inicial del Ajustador, consistente en espiritualizar la mente mortal para poder asegurar la transferencia de la transcripción de la experiencia humana a los mundos de las mansiones. En la experiencia de Jesús debe haber habido alguna realidad espiritual análoga a la naturaleza espiritual, o alma, de los mortales de las esferas que crecen en la fe. Pero esto es simplemente nuestra opinión ---en realidad no sabemos lo que Jesús encomendó a su Padre.

\par 
%\textsuperscript{(2015.1)}
\textsuperscript{188:3.5} Sabemos que la forma física del Maestro descansó en la tumba de José hasta cerca de las tres de la mañana del domingo, pero no tenemos ninguna certidumbre en lo que se refiere al estado de la personalidad de Jesús durante ese período de treinta y seis horas. A veces nos hemos atrevido a explicarnos estas cosas a nosotros mismos más o menos como sigue:

\par 
%\textsuperscript{(2015.2)}
\textsuperscript{188:3.6} 1. La conciencia de Miguel como Creador debe haber estado en libertad y totalmente independiente de su mente mortal asociada en la encarnación física.

\par 
%\textsuperscript{(2015.3)}
\textsuperscript{188:3.7} 2. Sabemos que el antiguo Ajustador del Pensamiento de Jesús estaba presente en la Tierra durante este período y dirigía personalmente las huestes celestiales reunidas.

\par 
%\textsuperscript{(2015.4)}
\textsuperscript{188:3.8} 3. El hombre de Nazaret había adquirido una identidad espiritual que había construido durante su vida en la carne, primero gracias a los esfuerzos directos de su Ajustador del Pensamiento, y después mediante la perfecta adaptación personal que efectuó entre las necesidades físicas y las exigencias espirituales de la existencia humana ideal, una adaptación que llevó a cabo escogiendo sin cesar la voluntad del Padre; esa identidad espiritual es la que debe haber sido confiada al cuidado del Padre Paradisiaco. No sabemos si esta realidad espiritual regresó o no para formar parte de la personalidad resucitada, pero creemos que sí. Pero en el universo están aquellos que sostienen que esta identidad de alma de Jesús descansa ahora en el «seno del Padre», y que posteriormente será liberada para dirigir el Cuerpo de la Finalidad de Nebadon hacia su destino no revelado relacionado con los universos increados de los reinos inorganizados del espacio exterior.

\par 
%\textsuperscript{(2015.5)}
\textsuperscript{188:3.9} 4. Creemos que la conciencia humana o mortal de Jesús durmió durante estas treinta y seis horas. Tenemos razones para creer que el Jesús humano no sabía nada de lo que sucedía en el universo durante este período. Para la conciencia mortal no transcurrió ningún período de tiempo; la resurrección a la vida siguió instantáneamente al sueño de la muerte.

\par 
%\textsuperscript{(2015.6)}
\textsuperscript{188:3.10} Y esto es casi todo lo que podemos indicar sobre el estado de Jesús durante este período en la tumba. Existe una serie de hechos correlativos a los que podemos aludir, aunque no somos del todo competentes para emprender su interpretación.

\par 
%\textsuperscript{(2015.7)}
\textsuperscript{188:3.11} En el inmenso patio de las salas de resurrección del primer mundo de las mansiones de Satania, se puede observar actualmente un magnífico edificio material-morontial conocido con el nombre de «Monumento conmemorativo de Miguel», que lleva ahora el sello de Gabriel. Este monumento fue creado poco después de que Miguel partiera de este mundo, y lleva esta inscripción: «En conmemoración del tránsito humano de Jesús de Nazaret por Urantia».

\par 
%\textsuperscript{(2016.1)}
\textsuperscript{188:3.12} Existen documentos que muestran que, durante este período, el consejo supremo de Salvington, compuesto por cien miembros, celebró una reunión ejecutiva en Urantia bajo la presidencia de Gabriel. También hay archivos que muestran que, durante este período, los Ancianos de los Días de Uversa se comunicaron con Miguel en relación con el estado del universo de Nebadon.

\par 
%\textsuperscript{(2016.2)}
\textsuperscript{188:3.13} Sabemos que al menos un mensaje se cruzó entre Miguel y Emmanuel en Salvington, mientras el cuerpo del Maestro yacía en la tumba.

\par 
%\textsuperscript{(2016.3)}
\textsuperscript{188:3.14} Existen buenas razones para creer que cierta personalidad se sentó en el lugar de Caligastia, en el consejo sistémico de los Príncipes Planetarios que se convocó en Jerusem, mientras el cuerpo de Jesús descansaba en la tumba.

\par 
%\textsuperscript{(2016.4)}
\textsuperscript{188:3.15} Los archivos de Edentia indican que el Padre de la Constelación de Norlatiadek estaba en Urantia, y que recibió instrucciones de Miguel durante este período en que estaba en la tumba.

\par 
%\textsuperscript{(2016.5)}
\textsuperscript{188:3.16} Y existen otras muchas pruebas que sugieren que no toda la personalidad de Jesús estaba dormida e inconsciente durante este período de muerte física aparente.

\section*{4. El significado de la muerte en la cruz}
\par 
%\textsuperscript{(2016.6)}
\textsuperscript{188:4.1} Aunque Jesús no sufrió esta muerte en la cruz para expiar la culpabilidad racial del hombre mortal, ni para proporcionar algún tipo de acercamiento eficaz a un Dios por otra parte ofendido e implacable; aunque el Hijo del Hombre no se ofreció como sacrificio para apaciguar la ira de Dios y abrir a los pecadores el camino para obtener la salvación; a pesar de que estas ideas de expiación y de propiciación son erróneas, sin embargo existen unos significados ligados a esta muerte de Jesús en la cruz que no deberían ser pasados por alto. Es un hecho que a Urantia se le conoce, entre los otros planetas vecinos habitados, como «el Mundo de la Cruz».

\par 
%\textsuperscript{(2016.7)}
\textsuperscript{188:4.2} Jesús deseaba vivir en Urantia una vida mortal plena en la carne. La muerte es, generalmente, una parte de la vida. La muerte es el último acto del drama de los mortales. En vuestros esfuerzos bien intencionados por evitar los errores supersticiosos de la falsa interpretación del significado de la muerte en la cruz, deberíais procurar no cometer el grave error de dejar de percibir el verdadero significado y la auténtica importancia de la muerte del Maestro.

\par 
%\textsuperscript{(2016.8)}
\textsuperscript{188:4.3} El hombre mortal nunca ha sido propiedad de los grandes farsantes. Jesús no murió para redimir al hombre de las garras de los gobernantes apóstatas y de los príncipes caídos de las esferas. El Padre que está en los cielos nunca ha concebido una injusticia tan burda como la de condenar al alma de un mortal por las malas acciones de sus antepasados. La muerte del Maestro en la cruz tampoco fue un sacrificio consistente en un esfuerzo por pagarle a Dios una deuda que la raza humana había contraído con él.

\par 
%\textsuperscript{(2016.9)}
\textsuperscript{188:4.4} Antes de que Jesús viviera en la Tierra, quizás podíais tener la justificación de creer en un Dios semejante, pero ya no es posible desde que el Maestro vivió y murió entre vuestros semejantes mortales. Moisés enseñó la dignidad y la justicia de un Dios Creador, pero Jesús describió el amor y la misericordia de un Padre celestial.

\par 
%\textsuperscript{(2016.10)}
\textsuperscript{188:4.5} La naturaleza animal ---la tendencia a la maldad--- puede ser hereditaria, pero el pecado no se transmite de padres a hijos. El pecado es un acto de rebelión consciente y deliberada contra la voluntad del Padre y las leyes de los Hijos, cometido por una criatura volitiva individual.

\par 
%\textsuperscript{(2017.1)}
\textsuperscript{188:4.6} Jesús vivió y murió para un universo entero, y no solamente para las razas de este único mundo. Aunque los mortales de los reinos disponían de la salvación antes incluso de que Jesús viviera y muriera en Urantia, sin embargo es un hecho que su donación en este mundo iluminó enormemente el camino de la salvación; su muerte contribuyó mucho a hacer evidente para siempre la certeza de la supervivencia de los mortales después de la muerte en la carne.

\par 
%\textsuperscript{(2017.2)}
\textsuperscript{188:4.7} Aunque no es muy adecuado hablar de Jesús como de un sacrificador, un rescatador o un redentor, es enteramente correcto referirse a él como un \textit{salvador}. Hizo que el camino de la salvación (de la supervivencia) fuera para siempre más claro y seguro; el camino de la salvación lo mostró mejor y con más seguridad para todos los mortales de todos los mundos del universo de Nebadon.

\par 
%\textsuperscript{(2017.3)}
\textsuperscript{188:4.8} La idea de Dios como Padre verdadero y amoroso es el único concepto que Jesús enseñó. Una vez que captáis esta idea, debéis, con toda coherencia, abandonar por completo y de manera inmediata todas esas nociones primitivas sobre Dios como monarca ofendido, como soberano severo y todopoderoso, cuyo placer principal consiste en sorprender a sus súbditos obrando mal y en asegurarse de que sean castigados adecuadamente, a menos que otro ser casi igual a él se ofrezca para sufrir por ellos, para morir como un sustituto y en lugar de ellos. Toda la idea de la redención y de la expiación es incompatible con el concepto de Dios tal como fue enseñado y ejemplificado por Jesús de Nazaret. El amor infinito de Dios ocupa el primer lugar en la naturaleza divina.

\par 
%\textsuperscript{(2017.4)}
\textsuperscript{188:4.9} Todo este concepto de la expiación y de la salvación a través del sacrificio está arraigado y apoyado en el egoísmo. Jesús enseñó que el \textit{servicio} al prójimo es el concepto más elevado de la fraternidad de los creyentes en el espíritu. La salvación deben darla por sentada aquellos que creen en la paternidad de Dios. La preocupación principal del creyente no debería ser el deseo egoísta de la salvación personal, sino más bien el impulso desinteresado de amar a los semejantes, y por tanto de servirlos tal como Jesús amó y sirvió a los hombres mortales.

\par 
%\textsuperscript{(2017.5)}
\textsuperscript{188:4.10} Los creyentes auténticos tampoco se inquietan mucho por el castigo futuro de los pecados. El verdadero creyente sólo se preocupa por su separación actual de Dios. Es verdad que los padres sabios pueden castigar a sus hijos, pero hacen todo esto con amor y con un propósito correctivo. No disciplinan llenos de indignación, ni tampoco castigan como represalia.

\par 
%\textsuperscript{(2017.6)}
\textsuperscript{188:4.11} Aunque Dios fuera el monarca severo y legal de un universo en el que la justicia reinara de manera suprema, sin duda no estaría satisfecho con el plan infantil de sustituir a un ofensor culpable por una víctima inocente.

\par 
%\textsuperscript{(2017.7)}
\textsuperscript{188:4.12} Lo importante de la muerte de Jesús, tal como está relacionada con el enriquecimiento de la experiencia humana y la ampliación del camino de la salvación, no es el \textit{hecho} de su muerte, sino más bien la manera magnífica y el espíritu incomparable con que se enfrentó a la muerte.

\par 
%\textsuperscript{(2017.8)}
\textsuperscript{188:4.13} Toda esta idea de la redención de la expiación sitúa a la salvación en un plano de irrealidad; un concepto así es puramente filosófico. La salvación humana es \textit{real;} está basada en dos realidades que las criaturas pueden captar por la fe e incorporarlas de este modo en la experiencia individual humana: el hecho de la paternidad de Dios y su verdad correlacionada, la fraternidad de los hombres. Después de todo, es verdad que se os «perdonarán vuestras deudas, así como vosotros perdonáis a vuestros deudores».

\section*{5. Las lecciones de la cruz}
\par 
%\textsuperscript{(2017.9)}
\textsuperscript{188:5.1} La cruz de Jesús representa la medida total de la devoción suprema del verdadero pastor hacia aquellos miembros de su rebaño que incluso no se la merecen. Todas las relaciones entre Dios y el hombre las sitúa para siempre sobre la base de la familia. Dios es el Padre; el hombre es su hijo. El amor, el amor de un padre por su hijo, se convierte en la verdad central de las relaciones entre el Creador y la criatura en el universo ---y no la justicia de un rey que busca su satisfacción en los sufrimientos y el castigo de sus súbditos malvados.

\par 
%\textsuperscript{(2018.1)}
\textsuperscript{188:5.2} La cruz muestra para siempre que la actitud de Jesús hacia los pecadores no era ni una condena ni una remisión, sino más bien una salvación amorosa y eterna. Jesús es en verdad un salvador, en el sentido de que su vida y su muerte atraen a los hombres hacia la bondad y hacia una justa supervivencia. Jesús ama tanto a los hombres que su amor despierta una respuesta de amor en el corazón humano. El amor es realmente contagioso y eternamente creativo. La muerte de Jesús en la cruz ejemplifica un amor que es lo suficientemente fuerte y divino como para perdonar el pecado y absorber toda maldad. Jesús reveló a este mundo una calidad de rectitud superior a la justicia ---el simple concepto técnico del bien y del mal. El amor divino no se limita a perdonar las ofensas; las absorbe y las destruye realmente. El perdón del amor trasciende totalmente el perdón de la misericordia. La misericordia pone a un lado la culpabilidad del mal; pero el amor destruye para siempre el pecado y todas las debilidades que resultan de él. Jesús trajo a Urantia una nueva manera de vivir. Nos enseñó que no resistiéramos al mal, sino que encontráramos a través de él, de Jesús, una bondad que destruye eficazmente el mal. El perdón de Jesús no es una remisión; es una salvación de la condenación. La salvación no menosprecia las ofensas; \textit{lasenmienda}. El verdadero amor no transige con el odio ni lo perdona, lo destruye. El amor de Jesús nunca se siente satisfecho con el simple perdón. El amor del Maestro implica la rehabilitación, la supervivencia eterna. Es perfectamente correcto hablar de la salvación como de una redención, si con ello os referís a esta rehabilitación eterna.

\par 
%\textsuperscript{(2018.2)}
\textsuperscript{188:5.3} Con el poder de su amor personal por los hombres, Jesús pudo romper la influencia del pecado y del mal. De este modo liberó a los hombres para que escogieran mejores maneras de vivir. Jesús describió una liberación del pasado que prometía en sí misma un triunfo para el futuro. El perdón proporcionaba así la salvación. Cuando el amor divino ha sido aceptado plenamente en el corazón humano, su belleza destruye para siempre el encanto del pecado y el poder del mal.

\par 
%\textsuperscript{(2018.3)}
\textsuperscript{188:5.4} Los sufrimientos de Jesús no se limitaron a la crucifixión. En realidad, Jesús de Nazaret pasó más de veinticinco años en la cruz de una existencia humana real e intensa. El verdadero valor de la cruz consiste en el hecho de que fue la expresión suprema y final de su amor, la revelación culminante de su misericordia.

\par 
%\textsuperscript{(2018.4)}
\textsuperscript{188:5.5} En millones de mundos habitados, decenas de billones de criaturas evolutivas que podían haber tenido la tentación de renunciar a la lucha moral y de abandonar el buen combate de la fe, han mirado una vez más a Jesús en la cruz, y luego han continuado avanzando hacia adelante, inspirados por el espectáculo de un Dios que entrega su vida encarnada por devoción al servicio desinteresado de los hombres.

\par 
%\textsuperscript{(2018.5)}
\textsuperscript{188:5.6} Todo el triunfo de la muerte en la cruz está resumido en el espíritu de la actitud de Jesús hacia sus agresores. Convirtió la cruz en un símbolo eterno del triunfo del amor sobre el odio y de la victoria de la verdad sobre el mal, cuando oró: «Padre, perdónalos, porque no saben lo que hacen». Esta devoción amorosa fue contagiosa en todo un inmenso universo; los discípulos se contagiaron de su Maestro. El primer instructor de su evangelio que fue llamado a entregar su vida en este servicio, dijo, mientras lo lapidaban a muerte: «No los acuses de este pecado».

\par 
%\textsuperscript{(2018.6)}
\textsuperscript{188:5.7} La cruz hace un llamamiento supremo a lo mejor que hay en el hombre, porque nos revela a aquél que estuvo dispuesto a entregar su vida al servicio de sus semejantes. Nadie puede tener un amor más grande que éste: el de estar dispuesto a dar su vida por sus amigos ---y Jesús tenía tal amor, que estaba dispuesto a dar su vida por sus enemigos, un amor más grande que cualquier otro que se hubiera conocido hasta ese momento en la Tierra.

\par 
%\textsuperscript{(2019.1)}
\textsuperscript{188:5.8} En otros mundos, así como en Urantia, este sublime espectáculo de la muerte del Jesús humano en la cruz del Gólgota ha conmovido las emociones de los mortales, mientras que ha despertado la más alta devoción de los ángeles.

\par 
%\textsuperscript{(2019.2)}
\textsuperscript{188:5.9} La cruz es el símbolo superior del servicio sagrado, la consagración de vuestra vida al bienestar y la salvación de vuestros semejantes. La cruz no es el símbolo del sacrificio del Hijo inocente de Dios que se pone en el lugar de los pecadores culpables a fin de apaciguar la ira de un Dios ofendido. Pero sí se alza para siempre, en la Tierra y en todo un inmenso universo, como un símbolo sagrado de los buenos dándose a los malos, salvándolos así mediante esta devoción misma de amor. La cruz sí se alza como la prueba de la forma más elevada de servicio desinteresado, la devoción suprema de la plena donación de una vida recta al servicio de un ministerio incondicional, incluso en la muerte, la muerte en la cruz. La sola visión de este gran símbolo de la vida de donación de Jesús nos inspira realmente a todos a querer hacer lo mismo.

\par 
%\textsuperscript{(2019.3)}
\textsuperscript{188:5.10} Cuando los hombres y las mujeres inteligentes contemplan a Jesús ofreciendo su vida en la cruz, difícilmente se atreverán a quejarse de nuevo ni siquiera de las penalidades más duras de la vida, y mucho menos de las pequeñas incomodidades y sus muchas molestias puramente ficticias. Su vida fue tan gloriosa y su muerte tan triunfal, que todos nos sentimos atraídos a querer compartir las dos. Toda la donación de Miguel posee un verdadero poder de atracción, desde la época de su juventud hasta este espectáculo sobrecogedor de su muerte en la cruz.

\par 
%\textsuperscript{(2019.4)}
\textsuperscript{188:5.11} Aseguraos, pues, de que cuando contempléis la cruz como una revelación de Dios, no la miréis con los ojos del hombre primitivo ni desde el punto de vista de los bárbaros posteriores, pues ambos consideraban a Dios como un Soberano implacable de justicia severa que aplicaba la ley con rigidez. Aseguraos más bien de que veis en la cruz la manifestación final del amor y de la devoción de Jesús a la misión de donación de su vida sobre las razas mortales de su inmenso universo. Ved en la muerte del Hijo del Hombre la culminación de la manifestación del amor divino del Padre por sus hijos de las esferas donde viven los mortales. La cruz representa así la devoción de un afecto complaciente y la donación de la salvación voluntaria a aquellos que están dispuestos a recibir estos dones y esta devoción. En la cruz no hubo nada que el Padre exigiera ---sino únicamente lo que Jesús dio tan gustosamente y que rehusó evitar.

\par 
%\textsuperscript{(2019.5)}
\textsuperscript{188:5.12} Si el hombre no puede apreciar a Jesús de otra manera ni entender el significado de su donación en la Tierra, al menos puede comprender que fue compañero suyo en sus sufrimientos humanos. Nadie puede temer nunca que el Creador no conozca la naturaleza o el grado de sus aflicciones temporales.

\par 
%\textsuperscript{(2019.6)}
\textsuperscript{188:5.13} Sabemos que la muerte en la cruz no sirvió para reconciliar al hombre con Dios, sino para estimular en el hombre la \textit{comprensión} del amor eterno del Padre y de la misericordia sin fin de su Hijo, y para difundir estas verdades universales a un universo entero.