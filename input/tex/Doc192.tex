\chapter{Documento 192. Las apariciones en Galilea}
\par 
%\textsuperscript{(2045.1)}
\textsuperscript{192:0.1} CUANDO los apóstoles salieron de Jerusalén hacia Galilea, los dirigentes judíos se habían tranquilizado considerablemente. Puesto que Jesús sólo se aparecía a su familia de creyentes en el reino, y como los apóstoles estaban escondidos y no hacían ninguna predicación pública, los jefes de los judíos concluyeron que, después de todo, el movimiento del evangelio estaba eficazmente aplastado. Por supuesto, estaban desconcertados por la creciente difusión de los rumores de que Jesús había resucitado de entre los muertos, pero contaban con los guardias sobornados para contrarrestar eficazmente todas estas noticias repitiendo la historia de que un grupo de seguidores de Jesús se había llevado el cuerpo.

\par 
%\textsuperscript{(2045.2)}
\textsuperscript{192:0.2} A partir de este momento y hasta que los apóstoles fueron dispersados por la marea creciente de las persecuciones, Pedro fue reconocido de manera general como jefe del cuerpo apostólico. Jesús nunca le confirió esta autoridad, y sus compañeros apóstoles nunca lo eligieron oficialmente para este puesto de responsabilidad; Pedro lo asumió de manera natural y lo conservó por consentimiento general, y también porque era el principal predicador de todos ellos. Desde ahora en adelante, la predicación pública se convirtió en la tarea fundamental de los apóstoles. Después de regresar de Galilea, Matías, a quien habían elegido para sustituir a Judas, se convirtió en su tesorero.

\par 
%\textsuperscript{(2045.3)}
\textsuperscript{192:0.3} Durante la semana que permanecieron en Jerusalén, María la madre de Jesús pasó mucho tiempo con las mujeres creyentes que estaban alojadas en la casa de José de Arimatea.

\par 
%\textsuperscript{(2045.4)}
\textsuperscript{192:0.4} Cuando los apóstoles partieron para Galilea este lunes por la mañana temprano, Juan Marcos salió tras ellos. Los siguió fuera de la ciudad, y cuando se encontraban mucho más allá de Betania, se presentó audazmente entre ellos, confiando en que no lo enviarían para atrás.

\par 
%\textsuperscript{(2045.5)}
\textsuperscript{192:0.5} Los apóstoles se detuvieron varias veces en el camino de Galilea para contar la historia de su Maestro resucitado, y por eso no llegaron a Betsaida hasta el miércoles por la noche muy tarde. Ya era mediodía del jueves cuando todos se despertaron y se prepararon para tomar el desayuno.

\section*{1. La aparición cerca del lago}
\par 
%\textsuperscript{(2045.6)}
\textsuperscript{192:1.1} El viernes 21 de abril hacia las seis de la mañana, el Maestro morontial efectuó su decimotercera aparición, la primera en Galilea, a los diez apóstoles cuando acercaban su barca a la orilla, cerca del desembarcadero habitual de Betsaida.

\par 
%\textsuperscript{(2045.7)}
\textsuperscript{192:1.2} El jueves, después de que los apóstoles hubieron pasado la tarde y las primeras horas de la noche esperando en la casa de Zebedeo, Simón Pedro sugirió que fueran a pescar. Cuando Pedro propuso esta jornada de pesca, todos los apóstoles decidieron ir. Se afanaron toda la noche con las redes, pero no atraparon ningún pez. No se preocuparon mucho por no haber pescado nada, pues tenían muchas experiencias interesantes sobre las que hablar, todas las cosas que tan recientemente les habían sucedido en Jerusalén. Pero cuando llegó la luz del día, decidieron volver a Betsaida. Al acercarse a la orilla, vieron a alguien en la playa, cerca del desembarcadero, de pie al lado de un fuego. Al principio creyeron que se trataba de Juan Marcos, que había bajado a recibirlos cuando regresaban con la pesca, pero al acercarse más a la orilla vieron que se habían equivocado ---el hombre era demasiado alto para ser Juan. A ninguno se le había ocurrido que la persona que estaba en la playa fuera el Maestro. No comprendían del todo por qué Jesús quería encontrarse con ellos entre los paisajes de sus primeras relaciones y al aire libre en contacto con la naturaleza, lejos del ambiente cerrado de Jerusalén, con sus trágicas asociaciones de miedo, de traición y de muerte. Les había dicho que, si iban a Galilea, se encontraría con ellos allí, y estaba a punto de cumplir esta promesa.

\par 
%\textsuperscript{(2046.1)}
\textsuperscript{192:1.3} Mientras echaban el ancla y se preparaban para subir al bote pequeño con el fin de desembarcar, el hombre que estaba en la playa les gritó: «Muchachos, ¿habéis pescado algo?» Cuando respondieron que no, el hombre dijo de nuevo: «Echad la red a la derecha de la barca y encontraréis los peces». Aunque no sabían que era Jesús el que les había orientado, echaron la red al unísono tal como les había indicado, y se llenó inmediatamente de tal manera que casi no podían sacarla. Pero Juan Zebedeo era de percepción rápida, y cuando vio la red cargada hasta los topes, percibió que era el Maestro el que les había hablado. Cuando este pensamiento le vino a la cabeza, se inclinó hacia Pedro y le dijo en voz baja: «Es el Maestro». Pedro fue siempre un hombre de acción irreflexiva y de devoción impetuosa, de manera que, en cuanto Juan le susurró esto al oído, se levantó rápidamente y se arrojó al agua para poder llegar cuanto antes al lado del Maestro. Sus hermanos llegaron inmediatamente después de él, alcanzando la orilla en la barca pequeña y arrastrando la red de peces detrás de ellos.

\par 
%\textsuperscript{(2046.2)}
\textsuperscript{192:1.4} Mientras tanto, Juan Marcos se había levantado, y al ver que los apóstoles llegaban a la orilla con la red cargada hasta los topes, corrió por la playa abajo para saludarlos. Cuando vio a once hombres en lugar de diez, supuso que el desconocido era Jesús resucitado, y mientras los diez hombres asombrados permanecían allí en silencio, el joven se precipitó hacia el Maestro, se arrodilló a sus pies, y dijo: «Señor mío y Maestro mío». Entonces Jesús habló, no como lo había hecho en Jerusalén cuando los saludó diciendo «Que la paz sea con vosotros», sino que se dirigió a Juan Marcos en un tono familiar: «Bien, Juan, me alegro de verte de nuevo en la despreocupada Galilea, donde podremos tener una buena conversación. Quédate con nosotros, Juan, y desayuna».

\par 
%\textsuperscript{(2046.3)}
\textsuperscript{192:1.5} Mientras Jesús hablaba con el joven, los diez estaban tan asombrados y sorprendidos que se olvidaron de arrastrar la red de peces hasta la playa. Jesús dijo entonces: «Traed vuestros peces y preparad algunos para el desayuno. Ya tenemos el fuego y mucho pan».

\par 
%\textsuperscript{(2046.4)}
\textsuperscript{192:1.6} Mientras Juan Marcos rendía homenaje al Maestro, Pedro se sobresaltó por un momento a la vista de las brasas que resplandecían allí en la playa; la escena le recordó vivamente el fuego de carbón de leña a medianoche en el patio de Anás, donde había negado al Maestro. Pero se repuso, y arrodillándose a los pies del Maestro, exclamó: «¡Señor mío y Maestro mío!»

\par 
%\textsuperscript{(2046.5)}
\textsuperscript{192:1.7} Pedro se unió luego a sus compañeros que arrastraban la red. Cuando llevaron a tierra su captura, contaron los peces, y había 153 grandes. Y de nuevo se cometió el error de llamarle a esto otra pesca milagrosa. No hubo ningún milagro asociado a este episodio. El Maestro simplemente había ejercido su preconocimiento. Sabía que los peces estaban allí, y en consecuencia, indicó a los apóstoles dónde debían echar la red.

\par 
%\textsuperscript{(2047.1)}
\textsuperscript{192:1.8} Jesús les habló diciendo: «Ahora, venid todos a desayunar. Incluso los gemelos deberían sentarse mientras charlo con vosotros; Juan Marcos preparará los peces». Juan Marcos trajo siete peces de buen tamaño que el Maestro puso en el fuego, y cuando estuvieron asados, el muchacho los sirvió a los diez. Entonces, Jesús partió el pan y se lo entregó a Juan que, a su vez, lo sirvió a los hambrientos apóstoles. Cuando todos estuvieron servidos, Jesús le rogó a Juan Marcos que se sentara mientras él mismo servía el pescado y el pan al muchacho. Mientras comían, Jesús charló con ellos, recordando sus numerosas experiencias comunes en Galilea y al lado de este mismo lago.

\par 
%\textsuperscript{(2047.2)}
\textsuperscript{192:1.9} Ésta era la tercera vez que Jesús se manifestaba a los apóstoles como grupo. Cuando Jesús se dirigió a ellos al principio preguntándoles si habían pescado, no sospecharon quien era porque para estos pescadores del Mar de Galilea era una experiencia corriente, cuando llegaban a la orilla, que los mercaderes de pescado de Tariquea los abordaran así, ya que habitualmente estaban dispuestos a comprar la pesca fresca para los establecimientos de desecación.

\par 
%\textsuperscript{(2047.3)}
\textsuperscript{192:1.10} Jesús conversó con los diez apóstoles y Juan Marcos durante más de una hora; luego se paseó de un lado a otro de la playa hablando con ellos de dos en dos ---pero no eran las mismas parejas que al principio había enviado juntas a enseñar. Los once apóstoles habían venido juntos desde Jerusalén, pero a medida que se acercaban a Galilea, Simón Celotes se había desalentado cada vez más, de manera que cuando llegaron a Betsaida, dejó a sus hermanos y regresó a su casa.

\par 
%\textsuperscript{(2047.4)}
\textsuperscript{192:1.11} Antes de despedirse de ellos esta mañana, Jesús les encargó que dos apóstoles se ofrecieran voluntarios para ir a por Simón Celotes y lo trajeran de vuelta aquel mismo día. Y esto es lo que hicieron Pedro y Andrés.

\section*{2. Las conversaciones con los apóstoles de dos en dos}
\par 
%\textsuperscript{(2047.5)}
\textsuperscript{192:2.1} Cuando terminaron de desayunar, y mientras los demás permanecían sentados al lado del fuego, Jesús hizo señas a Pedro y a Juan para que le acompañaran a dar un paseo por la playa. Mientras caminaban, Jesús le dijo a Juan: «Juan, ¿me amas?» Y cuando Juan contestó: «Sí, Maestro, con todo mi corazón», el Maestro dijo: «Entonces, Juan, abandona tu intolerancia y aprende a amar a los hombres como yo te he amado. Dedica tu vida a demostrar que el amor es la cosa más grande del mundo. Es el amor de Dios el que impulsa a los hombres a buscar la salvación. El amor es el padre de toda bondad espiritual, la esencia de lo verdadero y de lo bello».

\par 
%\textsuperscript{(2047.6)}
\textsuperscript{192:2.2} Jesús se volvió entonces hacia Pedro y le preguntó: «Pedro, ¿me amas?» Pedro contestó: «Señor, tú sabes que te amo con toda mi alma». Entonces dijo Jesús: «Si me amas, Pedro, apacienta mis corderos. No descuides ayudar a los débiles, a los pobres y a los jóvenes. Predica el evangelio sin temor ni favor; recuerda siempre que Dios no hace acepción de personas. Sirve a tus semejantes como yo te he servido; perdona a tus compañeros mortales como yo te he perdonado. Que la experiencia te enseñe el valor de la meditación y el poder de la reflexión inteligente».

\par 
%\textsuperscript{(2047.7)}
\textsuperscript{192:2.3} Después de caminar un poco más, el Maestro se volvió hacia Pedro y le preguntó: «Pedro, ¿realmente me amas?» Y entonces dijo Simón: «Sí, Señor, tú sabes que te amo». Y Jesús dijo de nuevo: «Entonces, cuida bien a mis ovejas. Sé un pastor bueno y verdadero para el rebaño. No traiciones su confianza en ti. No te dejes sorprender por el enemigo. Permanece alerta en todo momento ---vigila y ora».

\par 
%\textsuperscript{(2047.8)}
\textsuperscript{192:2.4} Cuando dieron unos cuantos pasos más, Jesús se volvió hacia Pedro y le preguntó por tercera vez: «Pedro, ¿me amas de verdad?» Entonces Pedro, ligeramente herido por la aparente desconfianza del Maestro, dijo con una gran emoción: «Señor, tú lo sabes todo, y sabes por tanto que te amo realmente y de verdad». Entonces dijo Jesús: «Apacienta mis ovejas. No abandones al rebaño. Sé un ejemplo y una inspiración para todos tus compañeros pastores. Ama al rebaño como yo te he amado y conságrate a su bienestar como yo he consagrado mi vida a tu bienestar. Y sígueme hasta el fin».

\par 
%\textsuperscript{(2048.1)}
\textsuperscript{192:2.5} Pedro interpretó esta última declaración al pie de la letra ---que debía continuar detrás de Jesús--- y volviéndose hacia él, señaló con el dedo a Juan y preguntó: «Si continúo detrás de ti, ¿qué hará éste?» Entonces, al percibir que Pedro había entendido mal sus palabras, Jesús dijo: «Pedro, no te preocupes por lo que hagan tus hermanos. Si quiero que Juan se quede después de que te hayas ido, o incluso hasta que yo vuelva, ¿en qué te concierne a ti? Asegúrate solamente de que me sigues».

\par 
%\textsuperscript{(2048.2)}
\textsuperscript{192:2.6} Este comentario se difundió entre los hermanos y fue recibido como una declaración de Jesús de que Juan no moriría antes de que regresara el Maestro para establecer el reino con poder y gloria, como muchos pensaban y esperaban. Esta interpretación de lo que Jesús había dicho contribuyó mucho a que Simón Celotes regresara al servicio y permaneciera trabajando.

\par 
%\textsuperscript{(2048.3)}
\textsuperscript{192:2.7} Cuando regresaron donde estaban los demás, Jesús se fue a pasear y a hablar con Andrés y Santiago. Después de recorrer una corta distancia, Jesús le dijo a Andrés: «Andrés, ¿confías en mí?» Cuando el antiguo jefe de los apóstoles escuchó a Jesús hacerle esta pregunta, se detuvo y contestó: «Sí, Maestro, es evidente que confío en ti, y sabes que es así». Entonces dijo Jesús: «Andrés, si confías en mí, confía más en tus hermanos ---incluso en Pedro. Hace tiempo te confié la dirección de tus hermanos. Ahora debes confiar en los demás mientras os dejo para ir hacia el Padre. Cuando tus hermanos empiecen a dispersarse debido a las crueles persecuciones, sé un consejero sabio y prudente para Santiago, mi hermano carnal, cuando le asignen unas pesadas cargas que no está capacitado para llevar por falta de experiencia. Y luego continúa confiando, porque yo no te fallaré. Cuando hayas terminado en la Tierra, vendrás hacia mí».

\par 
%\textsuperscript{(2048.4)}
\textsuperscript{192:2.8} Luego Jesús se volvió hacia Santiago, preguntando: «Santiago ¿confías en mí?» Y Santiago contestó por supuesto: «Sí, Maestro, confío en ti con todo mi corazón». Entonces dijo Jesús: «Santiago, si confías más en mí, serás menos impaciente con tus hermanos. Si quieres confiar en mí, eso te ayudará a ser bondadoso con la fraternidad de los creyentes. Aprende a estimar las consecuencias de tus palabras y de tus actos. Recuerda que se cosecha aquello que se siembra. Reza por la tranquilidad de espíritu y cultiva la paciencia. Estas gracias, junto con la fe viviente, te sostendrán cuando llegue la hora de beber la copa del sacrificio. Pero no te desanimes nunca; cuando hayas terminado en la Tierra, también vendrás para estar conmigo».

\par 
%\textsuperscript{(2048.5)}
\textsuperscript{192:2.9} Jesús habló a continuación con Tomás y Natanael. A Tomás le dijo: «Tomás, ¿me sirves?» Tomás contestó: «Sí, Señor, te sirvo ahora y siempre». Entonces dijo Jesús: «Si quieres servirme, sirve a mis hermanos en la carne como yo te he servido. Y no te canses de obrar bien, sino que persevera como alguien que ha sido ordenado por Dios para realizar este servicio de amor. Cuando hayas terminado tu servicio conmigo en la Tierra, servirás conmigo en la gloria. Tomás, debes dejar de dudar; debes crecer en la fe y en el conocimiento de la verdad. Cree en Dios como un niño, pero deja de actuar de manera tan infantil. Ten coraje; sé fuerte en la fe y poderoso en el reino de Dios».

\par 
%\textsuperscript{(2049.1)}
\textsuperscript{192:2.10} Entonces el Maestro le dijo a Natanael: «Natanael, ¿me sirves?» Y el apóstol contestó: «Sí, Maestro, y con todo mi afecto». Entonces dijo Jesús: «Si me sirves pues de todo corazón, asegúrate de que te consagras al bienestar de mis hermanos en la Tierra con un afecto incansable. Incorpora la amistad a tu consejo y añade el amor a tu filosofía. Sirve a tus semejantes como yo te he servido. Sé fiel a los hombres como yo he velado por ti. Sé menos crítico; espera menos de algunos hombres y disminuye así la magnitud de tus decepciones. Y cuando el trabajo aquí abajo haya terminado, servirás conmigo en el cielo».

\par 
%\textsuperscript{(2049.2)}
\textsuperscript{192:2.11} Después de esto, el Maestro conversó con Mateo y Felipe. A Felipe le dijo: «Felipe, ¿me obedeces?» Felipe respondió: «Sí, Señor, te obedeceré incluso con mi vida». Entonces dijo Jesús: «Si quieres obedecerme, ve pues a los países de los gentiles y proclama este evangelio. Los profetas te han dicho que es mejor obedecer que sacrificar. Te has convertido, por la fe, en un hijo del reino que conoce a Dios. Sólo hay una ley que obedecer ---y es el mandamiento de salir a proclamar el evangelio del reino. Deja de temer a los hombres; no tengas miedo de predicar la buena nueva de la vida eterna a tus semejantes que languidecen en las tinieblas y ansían la luz de la verdad. Felipe, ya no tendrás que ocuparte del dinero ni de los bienes. Ahora eres libre de predicar la buena nueva exactamente igual que tus hermanos. Iré delante de ti y estaré contigo hasta el fin».

\par 
%\textsuperscript{(2049.3)}
\textsuperscript{192:2.12} Luego, el Maestro se dirigió a Mateo y le preguntó: «Mateo, ¿albergas en tu corazón el deseo de obedecerme?» Mateo respondió: «Sí, Señor, estoy plenamente dedicado a hacer tu voluntad». Entonces dijo el Maestro: «Mateo, si quieres obedecerme, sal a enseñar a todos los pueblos este evangelio del reino. Ya no proporcionarás más a tus hermanos las cosas materiales de la vida; de ahora en adelante también deberás proclamar la buena nueva de la salvación espiritual. A partir de ahora ten el único propósito de obedecer tu encargo de predicar este evangelio del reino del Padre. Al igual que yo he hecho la voluntad del Padre en la Tierra, tú cumplirás la misión divina. Recuerda, tanto los judíos como los gentiles son tus hermanos. No temas a nadie cuando proclames las verdades salvadoras del evangelio del reino de los cielos. Y allí donde voy, dentro de poco vendrás tú».

\par 
%\textsuperscript{(2049.4)}
\textsuperscript{192:2.13} Después se paseó y habló con Santiago y Judas, los gemelos Alfeo; dirigiéndose a los dos a la vez, les preguntó: «Santiago y Judas, ¿creéis en mí?» Y cuando los dos contestaron: «Sí, Maestro, creemos», Jesús dijo: «Pronto voy a dejaros. Veis que ya os he dejado de manera carnal. Sólo permaneceré poco tiempo con esta forma antes de ir hacia mi Padre. Creéis en mí ---sois mis apóstoles, y siempre lo seréis. Continuad creyendo y recordando vuestra asociación conmigo cuando me haya ido, y después de que quizás hayáis regresado al trabajo que hacíais antes de venir a vivir conmigo. No permitáis nunca que un cambio en vuestro trabajo exterior influya en vuestra lealtad. Tened fe en Dios hasta el final de vuestros días en la Tierra. No olvidéis nunca que cuando uno es un hijo de Dios por la fe, todo trabajo honrado en la Tierra es sagrado. Nada de lo que hace un hijo de Dios puede ser corriente. De ahora en adelante, haced pues vuestro trabajo como si fuera para Dios. Y cuando hayáis terminado en este mundo, tengo otros mundos mejores donde trabajaréis igualmente para mí. En todo este trabajo, en este mundo y en los otros, yo trabajaré con vosotros y mi espíritu residirá dentro de vosotros».

\par 
%\textsuperscript{(2049.5)}
\textsuperscript{192:2.14} Eran casi las diez cuando Jesús regresó de su conversación con los gemelos Alfeo. Al dejar a los apóstoles, les dijo: «Adiós, hasta que os vea a todos mañana al mediodía en el monte de vuestra ordenación». Después de hablar así, desapareció de su vista.

\section*{3. En el monte de la ordenación}
\par 
%\textsuperscript{(2050.1)}
\textsuperscript{192:3.1} El sábado 22 de abril al mediodía, los once apóstoles se reunieron tal como habían acordado en la colina cerca de Cafarnaúm, y Jesús apareció entre ellos. Esta reunión tuvo lugar en el mismo monte donde el Maestro los había seleccionado como apóstoles suyos y como embajadores del reino del Padre en la Tierra. Ésta era la decimocuarta manifestación morontial del Maestro.

\par 
%\textsuperscript{(2050.2)}
\textsuperscript{192:3.2} En esta ocasión, los once apóstoles se arrodillaron en círculo alrededor del Maestro; le oyeron repetir sus misiones y le vieron reproducir la escena de la ordenación como cuando fueron seleccionados por primera vez para el trabajo especial del reino. Todo esto fue para ellos como un recordatorio de su consagración anterior al servicio del Padre, excepto la oración del Maestro. Cuando el Maestro ---el Jesús morontial--- oró este día, lo hizo con tal tono de majestad y con tales palabras de autoridad como los apóstoles no lo habían escuchado nunca anteriormente. Su Maestro hablaba ahora con los gobernantes de los universos como alguien en cuyas manos se habían depositado todos los poderes y toda la autoridad sobre su propio universo. Estos once hombres no olvidaron nunca esta experiencia de reconsagración morontial a sus compromisos anteriores como embajadores. El Maestro pasó exactamente una hora con sus embajadores en este monte, y después de despedirse afectuosamente de ellos, desapareció de su vista.

\par 
%\textsuperscript{(2050.3)}
\textsuperscript{192:3.3} Nadie vio a Jesús durante una semana entera. Los apóstoles no tenían realmente ninguna idea de lo que debían hacer, pues no sabían si el Maestro había ido hacia el Padre. En este estado de incertidumbre, permanecieron en Betsaida. No se atrevían a salir a pescar por temor a que viniera a visitarlos y no consiguieran verlo. Durante toda esta semana, Jesús estuvo ocupado con las criaturas morontiales que se encontraban en la Tierra y con los asuntos de la transición morontial que estaba experimentando en este mundo.

\section*{4. La reunión a la orilla del lago}
\par 
%\textsuperscript{(2050.4)}
\textsuperscript{192:4.1} La noticia de las apariciones de Jesús se estaba difundiendo por toda Galilea, y cada día llegaban más creyentes a la casa de Zebedeo para informarse sobre la resurrección del Maestro y averiguar la verdad sobre estas supuestas apariciones. A principios de la semana, Pedro hizo saber que el sábado siguiente a las tres de la tarde se celebraría una reunión pública a la orilla del mar.

\par 
%\textsuperscript{(2050.5)}
\textsuperscript{192:4.2} En consecuencia, el sábado 29 de abril a las tres de la tarde, más de quinientos creyentes de los alrededores de Cafarnaúm se reunieron en Betsaida para escuchar a Pedro predicar su primer sermón público desde la resurrección. El apóstol estaba en su mejor momento, y después de terminar su atractivo discurso, pocos oyentes suyos dudaron de que el Maestro había resucitado de entre los muertos.

\par 
%\textsuperscript{(2050.6)}
\textsuperscript{192:4.3} Pedro terminó su sermón diciendo: «Afirmamos que Jesús de Nazaret no está muerto; declaramos que ha salido de la tumba; proclamamos que lo hemos visto y que hemos hablado con él». En el preciso momento en que terminaba de efectuar esta declaración de fe, el Maestro apareció en forma morontial allí a su lado, plenamente a la vista de toda aquella gente, y les habló en un tono familiar, diciendo: «Que la paz sea con vosotros, y mi paz os dejo». Después de aparecer así y de hablarles de esta manera, desapareció de su vista. Ésta fue la decimoquinta manifestación morontial del Jesús resucitado.

\par 
%\textsuperscript{(2051.1)}
\textsuperscript{192:4.4} Debido a ciertas cosas que el Maestro había dicho a los once durante la conferencia en el monte de la ordenación, los apóstoles tuvieron la impresión de que su Maestro haría pronto una aparición pública delante de un grupo de creyentes galileos, y que después de esto debían regresar a Jerusalén. En consecuencia, al día siguiente, domingo 30 de abril, los once partieron temprano de Betsaida hacia Jerusalén. Enseñaron y predicaron bastante por el camino que descendía junto al Jordán, de manera que no llegaron a la casa de los Marcos, en Jerusalén, hasta el miércoles 3 de mayo ya tarde.

\par 
%\textsuperscript{(2051.2)}
\textsuperscript{192:4.5} Para Juan Marcos fue un triste regreso al hogar. Pocas horas antes de llegar a su casa, su padre, Elías Marcos, había muerto repentinamente de una hemorragia cerebral. La certidumbre de la resurrección de los muertos contribuyó mucho a consolar el dolor de los apóstoles, pero al mismo tiempo se afligieron sinceramente por la pérdida de su buen amigo, que los había apoyado incondicionalmente incluso en los momentos de las mayores dificultades y decepciones. Juan Marcos hizo todo lo que pudo por consolar a su madre, y hablando en nombre de ella, invitó a los apóstoles a que continuaran sintiéndose como en su hogar en la casa de ella. Y los once instalaron su cuartel general en la habitación de arriba hasta después del día de Pentecostés.

\par 
%\textsuperscript{(2051.3)}
\textsuperscript{192:4.6} Los apóstoles habían entrado adrede en Jerusalén después de la caída de la noche para no ser vistos por las autoridades judías. Tampoco aparecieron en público en el momento del funeral de Elías Marcos. Todo el día siguiente permanecieron aislados tranquilamente en esta memorable habitación de la parte superior.

\par 
%\textsuperscript{(2051.4)}
\textsuperscript{192:4.7} El jueves por la noche, los apóstoles tuvieron una maravillosa reunión en esta habitación de arriba, y todos se comprometieron a salir a predicar públicamente el nuevo evangelio del Señor resucitado, excepto Tomás, Simón Celotes y los gemelos Alfeo. Ya se estaban dando los primeros pasos para sustituir el evangelio del reino ---la filiación con Dios y la fraternidad con los hombres--- por la proclamación de la resurrección de Jesús. Natanael se opuso a este cambio en la esencia de su mensaje público, pero no pudo oponerse a la elocuencia de Pedro ni pudo vencer el entusiasmo de los discípulos, especialmente de las mujeres creyentes.

\par 
%\textsuperscript{(2051.5)}
\textsuperscript{192:4.8} Y así, bajo la vigorosa dirección de Pedro, y antes de que el Maestro ascendiera hacia el Padre, sus representantes bien intencionados emprendieron este proceso sutil de sustituir de manera gradual y segura la religión \textit{de} Jesús por una forma nueva y modificada de religión \textit{acerca de} Jesús.