\begin{document}

\title{Битие}


\chapter{1}

\par 1 В началото Бог създаде небето и земята.
\par 2 А земята беше пуста и неустроена; и тъмнина покриваше бездната; и Божият Дух се носеше над водата.
\par 3 И Бог каза: Да бъде светлина. И стана светлина.
\par 4 И Бог видя, че светлината беше добро; и Бог раздели светлината от тъмнината.
\par 5 И Бог нарече светлината Ден, а тъмнината нарече Нощ. И стана вечер, и стана утро, ден първи.
\par 6 И Бог каза: Да бъде простор посред водите, който да раздели вода от вода.
\par 7 И Бог направи простора; и раздели водата, която беше под простора
\par 8 И Бог нарече простора Небе. И стана вечер, и стана утро, ден втори.
\par 9 И Бог каза: Да се събере на едно място водата, която е под небето, та да се яви сушата; и стана така.
\par 10 И Бог нарече сушата Земя, и събраната вода нарече Морета; и Бог видя, че беше добро.
\par 11 И Бог каза: Да произрасти земята крехка трева, трева семеносна и плодно дърво, което да ражда плод, според вида си, чието семе да е в него на земята; и стана така.
\par 12 Земята произрасти крехка трева, трева която дава семе, според вида си, и дърво, което да ражда плод, според вида си, чието семе е в него; и Бог видя че беше добро.
\par 13 И стана вечер, и стана утро ден трети.
\par 14 И Бог каза: Да има светлина на небесния простор, за да разделят деня от нощта; нека служат за знаци и за показване времената, дните, годините;
\par 15 и да бъдат за светила на небесния простор, за да осветляват земята; и стана така.
\par 16 Бог създаде двете големи светила: по-голямото светило, за да владее деня, а по-малкото светило, за да владее нощта; създаде и звездите.
\par 17 И Бог ги постави на небесния простор, за да осветляват земята,
\par 18 да владеят деня и нощта, и да разделят светлината от тъмнината; и Бог видя, че беше добро.
\par 19 И стана вечер, и стана утро, ден четвърти.
\par 20 И Бог каза: Да произведе водата изобилно множества одушевени влечуги, и птици да хвърчат над земята по небесния простор.
\par 21 И Бог създаде големите морски чудовища и всяко одушевено същество, което се движи, които водата произведе изобилно, според видовете им, и всяка крилата птица според вида й; и Бог видя, че беше добро.
\par 22 И благослови Бог, казвайки: Плодете се, размножавайте се, и напълнете водите в моретата; нека се размножават и птиците по земята.
\par 23 И стана вечер, и стана утро, ден пети.
\par 24 И Бог каза: Да произведе земята одушевени животни, според видовете им: добитък, влечуги и земни зверове, според видовете им; и стана така.
\par 25 Бог създаде земните зверове според видовете им, добитъка - според видовете му, и всичко което пълзи по земята, според видовете му; И Бог видя, че беше добро.
\par 26 И Бог каза: Да създадем човека по Нашия образ, по Наше подобие; и нека владее над морските риби, над небесните птици, над добитъка, над цялата земя и над всяко животно, което пълзи по земята.
\par 27 и Бог създаде човека по Своя образ; по Божия образ го създаде; мъж и жена ги създаде.
\par 28 И Бог ги благослови. И рече им Бог: Плодете се и размножавайте, напълнете земята и обладайте я, и владейте над морските риби, над въздушните птици и над всяко живо същество, което се движи по земята.
\par 29 И Бог рече: Вижте, давам ви всяка семеносна трева, която е по лицето на цялата земя и всяко дърво, което има в себе си плод на семеносно дърво; те ще ви бъдат за храна.
\par 30 А на всичките земни зверове, на всичките въздушни птици, и на всичко, което пълзи по земята, в което има живот, давам, всяка зелена трева за храна; и стана така.
\par 31 И Бог видя всичко, което създаде; и, ето, беше твърде добро. И стана вечер, и стана утро, ден шести.

\chapter{2}

\par 1 Така се свършиха небето и земята и цялото тяхно войнство.
\par 2 И на седмия ден, като свърши Бог делата, които беше създал, на седмия ден си почина от всичките дела, които беше създал.
\par 3 И благослови Бог седмия ден и го освети, защото в него си почина от всичките си дела, от всичко , което Бог беше създал и сътворил.
\par 4 Това е произходът на небето и на земята при сътворението им във времето, когато Господ Бог създаде земя и небе.
\par 5 А никакво полско растение още нямаше на земята и никаква полска трева не беше още поникнала; защото Господ Бог не беше дал дъжд на земята и нямаше човек, който да обработва земята;
\par 6 но пара се издигаше от земята та напояваше цялото лице на земята.
\par 7 И Господ Бог създаде човека от пръст из земята, и вдъхна в ноздрите му жизнено дихание; и човекът стана жива душа.
\par 8 И Господ Бог насади градината на изток, в Едем, и постави там човека, когото беше създал.
\par 9 И Господ Бог направи да произраства от земята всяко дърво, що е красиво наглед и добро за храна, както и дървото на живота всред градината и дървото на познаване доброто и злото.
\par 10 И река изтичаше от Едем да напоява градината, от гдето се разклоняваше и стана четири главни реки .
\par 11 Името на едната е Фисон; тя е, която обикаля цялата Евилатска земя, гдето има злато.
\par 12 И златото на оная земя е добро там има още бделий и ониксов камък.
\par 13 Името на втората река е Гион; тя е, която обикаля цялата Хуска земя
\par 14 Името на третата река е Тигър: тя е, която тече на изток от Асирия. А четвъртата река е Ефрат.
\par 15 И Господ Бог взе човека и го засели в Едемската градина, за да я обработва и да я пази.
\par 16 И Господ Бог заповяда на човека, казвайки: От всяко дърво в градината свободно да ядеш;
\par 17 но от дървото на познаване доброто и злото, да не ядеш от него; защото в деня, когато ядеш от него, непременно ще умреш.
\par 18 И Господ Бог каза: Не е добре за човека да бъде сам; ще му създам подходящ помощник.
\par 19 И Господ Бог създаде от земята всички полски зверове и всички въздушни птици; и ги приведе при човека, за да види как ще ги наименува; и с каквото име назовеше човекът всяко одушевено същество, това име му остана.
\par 20 Така човекът даде имена на всеки вид добитък, на въздушните птици и на всички полски зверове. Но помощник, подходящ за човека не се намери.
\par 21 Тогава Господ даде на човека дълбок сън, и той заспа; и взе едно от ребрата му, и изпълни мястото му с плът.
\par 22 И Господ Бог създаде жената от реброто, което взе от човека и я приведе при човека.
\par 23 А човекът каза: Тази вече е кост от костите ми и плът от плътта ми; тя ще се нарече Жена, защото от Мъжа бе взета.
\par 24 Затова ще остави човека баща си и майка си и ще се привърже към жена си и те ще бъдат една плът.
\par 25 А и двамата, човекът и жена му бяха голи и не се срамуваха.

\chapter{3}

\par 1 А змията беше най-хитра от всички полски зверове, които Господ Бог беше създал. И тя рече на жената: Истина ли каза Бог да не ядете от всяко дърво в градината?
\par 2 Жената рече на змията: От плода на градинските дървета можем да ядем;
\par 3 но от плода на дървото, което е всред градината, Бог каза: Да не ядете от него, нито да се допрете до него, за да не умрете.
\par 4 А змията рече на жената: Никак няма да умрете;
\par 5 но знае Бог, че в деня, когато ядете от него, ще ви се отворят очите и ще бъдете, като Бога, да познавате доброто и злото.
\par 6 И като видя жената, че дървото беше добро за храна, и че беше приятно за очите, дърво желателно, за да дава знание, взе от плода му та яде, даде и на мъжа си да яде с нея, та и той яде.
\par 7 Тогава се отвориха очите и на двамата и те познаха, че бяха голи; и съшиха смокинови листа та си направиха препаски.
\par 8 И при вечерния ветрец чуха гласа на Господа Бога, като ходеше из градината; и човекът и жена му се скриха от лицето на Господа Бога между градинските дървета.
\par 9 Но Господ Бог повика човека и му рече: Где си?
\par 10 А той рече: Чух гласа Ти в градината и уплаших се, защото съм гол; и се скрих.
\par 11 А Бог му рече: Кой ти каза, че си гол? Да не би да си ял от дървото, от което ти заповядах да не ядеш?
\par 12 И човекът рече: Жената, която си ми дал за другарка, тя ми даде от дървото, та ядох.
\par 13 И Господ Бог рече на жената: Що е това, което си сторила? А жената рече: Змията ме подмами, та ядох.
\par 14 Тогава рече Господ Бог на змията: Понеже си сторила това, проклета да си измежду всеки вид добитък и измежду всички полски зверове; по корема си ще се влачиш, и пръст ще ядеш през всичките дни на живота си.
\par 15 Ще поставя и вражда между тебе и жената и между твоето потомство и нейното потомство; то ще ти нарани главата, а ти ще му нараниш петата.
\par 16 На жената рече: Ще ти преумножа скръбта в бременността; със скръб ще раждаш чада; и на мъжа ти ще бъде подчинено всяко твое желание и той ще те владее.
\par 17 А на човека рече: Понеже си послушал гласа на жена си и си ял от дървото, за което ти заповядах, като казах: Да не ядеш от него, то проклета да бъде земята поради тебе; със скръб ще се прехранваш от нея през всичките дни на живота си.
\par 18 Тръни и бодли ще ти ражда; и ти ще ядеш полската трева.
\par 19 С пот на лицето си ще ядеш хляб, докато се върнеш в земята, защото от нея си взет; понеже си пръст и в пръстта ще се върнеш.
\par 20 И човекът наименува жена си Ева, защото тя беше майка на всички живи.
\par 21 И Господ Бог направи кожени дрехи на Адама и на жена му и ги облече.
\par 22 И Господ Бог каза: Ето човекът стана като един от Нас, да познава доброто и злото; и сега за да не простре ръката си да вземе от дървото на живота, да яде и да живее вечно, -
\par 23 затова Господ Бог го изпъди от Едемската градина да обработва земята, от която бе взет.
\par 24 Така Той изпъди Адама; и постави на изток от Едемската градина херувимите и пламенния меч, който се въртеше, за да пазят пътя към дървото на живота.

\chapter{4}

\par 1 А Адам позна жена си Ева; и тя зачна и роди Каина; и каза: С помощта на Господа придобих човек.
\par 2 Роди още и брата му Авела. Авел пасеше стадо, а Каин беше земледелец.
\par 3 И след време Каин принесе от земните плодове принос на Господа.
\par 4 Тъй също и Авел принесе от първородните на стадото си и от тлъстината му. И Господ погледна благосклонно на Авела и на приноса му;
\par 5 а на Каина и на приноса му не погледна така. Затова Каин се огорчи твърде много и лицето му се помрачи.
\par 6 И Господ рече на Каина: Защо си се разсърдил? и защо е помрачено лицето ти?
\par 7 Ако правиш добро, не ще ли бъде прието? Но ако не правиш добро, грехът лежи на вратата и към тебе се стреми; но ти трябва да го владееш.
\par 8 А Каин каза това на брата си Авела. И когато бяха на полето, Каин стана против брата си Авела и го уби.
\par 9 И Господ рече на Каина: Где е брат ти Авел? А той рече: Не зная; пазач ли съм аз на брата си?
\par 10 И рече Бог : Какво си сторил? Гласът на братовата ти кръв вика към Мене от земята.
\par 11 И сега проклет си от земята, която отвори устата си да приеме кръвта на брата ти от твоята ръка.
\par 12 Когато работиш земята тя няма вече да ти дава силата си; бежанец и скитник ще бъдеш на земята.
\par 13 А Каин рече на Господа: Наказанието ми е толкова тежко, щото не мога да го понеса.
\par 14 Ето гониш ме днес от лицето на тая земя; ще съм скрит от Твоето лице, и ще бъда бежанец и скитник на земята; и тъй всеки, който ме намери, ще ме убие.
\par 15 А Господ му каза: Затова, който убие Каина, него ще му се отмъсти седмократно. И Господ определи белег за Каина, за да не го убива никой, който го намери.
\par 16 Тогава излезе Каин от Господното присъствие и се засели в земята Нод, на изток от Едем.
\par 17 И Каин позна жена си, която зачна и роди Еноха; и Каин съгради град, и наименува града Енох, по името на сина си.
\par 18 И на Еноха се роди Ирад; а Ирад роди Мехуяила; Мехуяил роди Метусаила; и Метусаил роди Ламеха.
\par 19 И Ламех си взе две жени: Името на едната беше Ада, а името на другата Села.
\par 20 Ада роди Явала; той беше родоначалник на ония, които живеят в шатри и имат добитък.
\par 21 Името пък на брат му беше Ювал; той беше родоначалник на всички, които свирят на арфа и кавал.
\par 22 Тъй Села роди Тувал-Каина, ковач на всякакъв вид медно и желязно сечиво; а сестра на Тувал-Каина беше Наама.
\par 23 И Ламех рече на жените си:- Адо и Село, чуйте гласа ми, Жени Ламехови, слушайте думите ми; Понеже мъж убих, задето ме нарани, Да! юноша задето ме смаза.
\par 24 Ако на Каина се отмъсти седмократно, То на Ламеха ще се отмъсти седемдесет и седмократно.
\par 25 И Адам пак позна жена си и тя роди син, когото наименува Сит, защото тя думаше , Бог ми определи друга рожба, вместо Авела, тъй като Каин го уби.
\par 26 Също и на Сита се роди син, когото наименува Енос. Тогава почнаха човеците да призовават Господното име.

\chapter{5}

\par 1 Ето списъкът§ на Адамовото потомство. В деня, когато Бог сътвори човека, Той го направи по Божие подобие;
\par 2 създаде ги мъж и жена, благослови ги, и наименува ги Човек, в деня когато бяха създадени.
\par 3 Адам живя сто и тридесет години, и роди син по свое подобие по своя образ и наименува го Сит.
\par 4 А откак роди Сита, дните на Адама станаха осемстотин години; и той роди синове и дъщери.
\par 5 А всичките дни на Адама колкото живя станаха деветстотин и тридесет години; и умря.
\par 6 Сит живя сто и пет години и роди Еноса.
\par 7 А откак роди Еноса, Сит живя осемстотин и седем години и роди синове и дъщери.
\par 8 И всичките дни на Сита станаха деветстотин и дванадесет години; и умря.
\par 9 Енос живя деветдесет години и роди Кенана.
\par 10 А откак роди Кенана, Енос живя осемстотин и петнадесет години, и роди синове и дъщери.
\par 11 И всичките дни на Еноса станаха деветстотин и пет години; и умря.
\par 12 Кенан живя седемдесет години и роди Маалалеила.
\par 13 А откак роди Маалалеила, Кенан живя осемстотин и четиридесет години и роди синове и дъщери.
\par 14 И всичките дни на Кенана станаха деветстотин и десет години и умря.
\par 15 Маалалеил живя шестдесет и пет години и роди Яреда.
\par 16 А откак роди Яреда, Маалалеил живя осемстотин и тридесет години и роди синове и дъщери.
\par 17 И всичките дни на Маалалеила станаха осемстотин деветдесет и пет години; и умря.
\par 18 Яред живя сто шестдесет и две години и роди Еноха.
\par 19 А откак роди Еноха, Яред живя осемстотин години и роди синове и дъщери.
\par 20 И всичките дни на Яреда станаха деветстотин шестдесет и две години; и умря.
\par 21 Енох живя шестдесет и пет години и роди Матусала.
\par 22 А откак роди Матусала, Енох ходи по Бога триста години и роди синове и дъщери.
\par 23 И всичките дни на Еноха станаха триста шестдесет и пет години.
\par 24 И Енох ходи по Бога и не се намираше вече , защото Бог го взе.
\par 25 Матусал живя сто осемдесет и седем години и роди Ламеха.
\par 26 А откак роди Ламеха, Матусал живя седемстотин осемдесет и две години и роди синове и дъщери.
\par 27 И всичките дни на Матусала станаха деветстотин шестдесет и девет години; и умря.
\par 28 Ламех живя сто осемдесет и две години и роди син;
\par 29 и наименува го Ной, като думаше: Тоя ще ни утеши в умората ни от работата ни и от труда на ръцете ни, който ни иде от земята, която Господ прокле.
\par 30 А откак роди Ноя, Ламех живя петстотин двадесет и пет години и роди синове и дъщери.
\par 31 И всичките дни на Ламеха станаха седемстотин и седем години; и умря.
\par 32 А Ной беше петстотин години; и Ной роди Сима, Хама и Яфета.

\chapter{6}

\par 1 Като почнаха човеците да се размножават по лицето на земята и им се раждаха дъщери,
\par 2 Божиите синове, като гледаха, че човешките дъщери бяха красиви, вземаха си за жени от всички, които избираха.
\par 3 И тогава рече Господ; Духът който съм му дал не ще владее вечно в човека; в блуждаенето си той е плът; затова дните му ще бъдат сто и двадесет години.
\par 4 В ония дни се намираха исполините на земята; а при това, след като Божиите синове влизаха при човешките дъщери, и те им раждаха синове, тези бяха ония силни и прочути старовременни мъже.
\par 5 И като видя Господ, че се умножава нечестието на човека по земята и че всичко, което мислите на сърцето му въображяваха, беше постоянно само зло,
\par 6 разкая се Господ, че беше направил човека на земята, и огорчи се в сърцето Си.
\par 7 И рече Господ: Ще изтребя от лицето на земята човека, когото създадох, - човеци, зверове, влечуги и въздушни птици, - понеже се разкаях, че ги създадох.
\par 8 А Ной придоби Господното благоволение.
\par 9 Ето Ноевото потомство. Ной беше човек праведен, непорочен между съвременниците си; той ходеше по Бога.
\par 10 И Ной роди три сина: Сима, Хама и Яфета.
\par 11 И земята се разврати пред Бога; земята се изпълни с насилие.
\par 12 И Бог видя земята; и, ето, тя бе развратена защото всяка твар се обхождаше развратно на земята.
\par 13 И рече Бог на Ноя: Краят на всяка твар се предвижда от Мене, защото земята се изпълни с насилие чрез тях; затова, ето, ще ги изтребя заедно със земята.
\par 14 Направи си ковчег от гоферово дърво; стаи да направиш в ковчега; и да го измажеш отвътре и отвън със смола.
\par 15 Направи го така: дължината на ковчега да бъде триста лакти, широчината му петдесет лакти, а височината му тридесет лакти.
\par 16 Прозорец направи на ковчега, като изкараш ковчега без един лакът до върха, а вратата на ковчега постави отстрана; направи го с долен, среден и горен етаж.
\par 17 ето, Аз ще докарам на земята потоп от вода, за да изтребя под небето всяка твар, която има в себе си жизнено дихание; всичко що се намира на земята ще измре.
\par 18 Но с тебе ще поставя завета Си; и ще влезеш в ковчега ти, синовете ти, жена ти и снахите ти с тебе.
\par 19 И от всичко, от всякакъв вид твар, която живее, да вкараш в ковчега по две от всеки вид, за да опазиш живота им със себе си; мъжко и женско да бъдат.
\par 20 От птиците според вида им, от добитъка според вида му, и от всичките земни животни според вида им, по две от всички да влязат при тебе, за да им опазиш живота.
\par 21 А ти си вземи от всякаква храна, която се яде, и събери я при себе си, та да послужи за храна на тебе и на тях.
\par 22 И Ной извърши всичко според както му заповяда Бог; така стори.

\chapter{7}

\par 1 Тогава Господ рече на Ноя: Влез ти и целият ти дом в ковчега; защото в това поколение тебе видях праведен пред Мене.
\par 2 Вземи си по седем от всички чисти животни, мъжки и женските им; а от нечистите животни по две, мъжки и женските им;
\par 3 също и от въздушните птици по седем, мъжки и женски; за да опазиш от тях разплод по лицето на цялата земя.
\par 4 Защото още седем дни и Аз ще направя да вали дъжд по земята четиридесет дни и четиридесет нощи; и ще изтребя от лицето на земята всичко живо, що съм направил.
\par 5 И Ной извърши всичко, според както му заповяда Господ.
\par 6 А Ной беше на шестстотин години когато стана на земята потопът от водата.
\par 7 И поради водите на потопа влязоха в ковчега Ной, синовете му, жена му и снахите му с него.
\par 8 От чистите животни и от нечистите животни, от птиците, и от всичко, което пълзи по земята,
\par 9 влязоха в ковчега две по две при Ноя, мъжки и женски, според както заповяда Бог на Ноя.
\par 10 И след седмия ден водите на потопа заляха земята.
\par 11 В шестстотната година на Ноевия живот, във втория месец, на седемнадесетия ден от месеца, в същия ден всичките извори на голямата бездна се разпукнаха и небесните отвори се разкриха.
\par 12 И дъждът валя на земята четиридесет дни и четиридесет нощи.
\par 13 В тоя същи ден влязоха в ковчега Ной и синовете Ноеви: Сим, Хам и Яфет, и Ноевата жена, и с тях трите му снахи -
\par 14 те и всичките животни според вида си, всичкият добитък според вида си, и всичките влечуги, които пълзят по земята според вида си и всичките птици според вида си, всяко пернато от всякакъв вид.
\par 15 Две по две от всяка твар, която има в себе си жизнено дихание, влязоха в ковчега при Ноя.
\par 16 И като влязоха, мъжки и женски от всяка твар влязоха, според както Бог бе заповядал; и Господ го затвори вътре.
\par 17 Четиридесет дни трая потопът на земята; и водите дойдоха та подеха ковчега и той се издигна над земята.
\par 18 Водите се усилваха и прииждаха много на земята, така ковчегът се носеше по повърхността на водите.
\par 19 Водите се усилваха и твърде много на земята, така щото се покриха всичките високи планини намиращи се под цялото небе.
\par 20 Петнадесет лакти по-горе от тях се издигнаха водите, та планините се покриха.
\par 21 И всяка твар, която се движеше по земята, умря, - птици, добитък, зверове, всички животни, които пълзят по земята и всеки човек.
\par 22 От всичко що беше на сушата, всичко, което имаше в ноздрите си жизнено дихание, измря.
\par 23 Всичко живо що се намираше по лицето на земята се изтреби, - човеци, добитък, животни и небесни птици; изтребиха се от земята; останаха само Ной и ония, които бяха с него в ковчега.
\par 24 А водите се застояха по земята сто и петдесет дни.

\chapter{8}

\par 1 Тогава си спомни Бог за Ноя и за всичко живо и за всичкия добитък, що бяха с него в ковчега; и Бог накара вятър да мине по земята, та водите престанаха.
\par 2 Тоже се затвориха изворите на бездната и небесните отвори, и дъждът от небето спря.
\par 3 Малко по малко водите се оттеглиха от земята и подир сто и петдесет дни водите взеха да намаляват.
\par 4 А на седемнадесетия ден от седмия месец ковчегът заседна върху Араратските планини.
\par 5 Водите намаляваха непрестанно до десетия месец; и на първия ден от десетия месец върховете на планините се показаха.
\par 6 После, след четиридесет дни, Ной отвори прозореца на ковчега, що беше направил;
\par 7 и изпрати гарван, който, като излезе, отиваше насам натам докато пресъхнаха водите на земята.
\par 8 Тогава изпрати гълъб, за да види дали са престанали водите по лицето на земята.
\par 9 Но гълъбът, понеже не намери почивка за нозете си, върна се при него в ковчега, защото водата бе още по лицето на цялата земя. И той простря ръка та го взе, и внесе го при себе си в ковчега.
\par 10 А като почака още седем дни, пак изпрати гълъба от ковчега.
\par 11 И надвечер гълъбът се върна при него, и ето, имаше в устата си пресен маслинен лист; така Ной позна, че водата е спаднала по земята.
\par 12 След това той почака още седем дни, и изпрати гълъба; но той не се върна вече при него.
\par 13 В шестстотин и първата година на Ноевия живот , на първия ден от първия месец, водата пресъхна на земята; и Ной, като дигна покрива на ковчега, погледна, и, ето, повърхността на земята бе изсъхнала.
\par 14 А на двадесет и седмия ден от втория месец земята съвършено изсъхна.
\par 15 Тогава говори Бог на Ноя, казвайки:
\par 16 Излез от ковчега, ти, жена ти, синовете ти и жените им с тебе.
\par 17 Изведи със себе си всичко живо от всяка твар, що е с тебе,- птици, добитък и всичките животни които пълзят по земята, за да се разплодяват по земята, да раждат и да се умножават по земята.
\par 18 Ной излезе, и с него синовете му, жена му и снахите му;
\par 19 излязоха от ковчега и всичките животни, всичките птици, всичко, що се движи по земята според родовете си.
\par 20 И Ной издигна олтар на Господа; и взе от всяко чисто животно и от всяка чиста птица, та ги принесе за всеизгаряне на олтара;
\par 21 и Господ помириса сладко благоухание; и рече Господ в сърцето Си: Не ще проклинам вече земята, поради човека, защото помислите на човешкото сърце са зло още от младините му, нито ще поразя вече друг път всичко живо както сторих.
\par 22 Догде съществува земята, сеитба и жътва, студ и горещина, лято и зима, ден и нощ няма да престанат.

\chapter{9}

\par 1 В това време Бог благослови Ноя и синовете му, като им каза: Плодете се, размножавайте се и напълнете земята.
\par 2 Ще се страхуват и ще треперят от вас всички земни животни и всички въздушни птици; те са всичко, що пълзи по земята и с всички земни риби са предадени в ръцете ви.
\par 3 Всичко живо, що се движи ще ви бъде за храна; давам ви всичко също, както дадох зелената трева.
\par 4 Месо обаче с живота му, тоест , с кръвта му, да не ядете.
\par 5 А вашата кръв, кръвта на живота ви, непременно ще изискам; от всяко животно ще я изискам; и от човека, да! от брата на всеки човек, ще изискам живота на човека.
\par 6 Който пролее човешка кръв, и неговата кръв от човек ще се пролее; защото по Своя образ направи Бог човека.
\par 7 А вие, плодете се и се размножавайте, разплодете се по земята и умножавайте се по нея.
\par 8 После Бог говори на Ноя и на синовете му с него, казвайки:
\par 9 Вижте, Аз поставям завета Си с вас и с потомството ви подир вас;
\par 10 и всичко живо, що е с вас, - птиците, добитъкът и всичките земни животни, които са с вас, да! с всяко земно животно от всичко, което е излязло от ковчега.
\par 11 Поставям завета Си с вас, че няма да се изтреби вече никоя твар от водите на потопа, нито ще настане вече потоп да опустоши земята.
\par 12 Бог рече още: Ето белегът на завета, който Аз поставям до вечни поколения между Мене и вас и всичко живо, което е с вас:
\par 13 поставям дъгата Си в облака, и тя ще бъде белег на завет между Мене и земята.
\par 14 Когато докарам облак на земята, дъгата ще се яви в облака:
\par 15 и ще спомня завета Си, който е между мене и вас и всичко живо от всяка твар; и водата няма вече да стане потоп за изтреблението на всяка твар.
\par 16 Дъгата ще бъде в облака: и ще я гледам, за да си напомням всегдашния завет между Бога и всичко живо от всяка твар, което е на земята.
\par 17 Каза Бог на Ноя: Това е белегът на завета, който установих между мене и всяка твар, която е на земята.
\par 18 А излезлите от ковчега Ноеви синове бяха Сим, Хам и Яфет; а Хам беше баща на Ханаана.
\par 19 Тия трима бяха Ноевите синове; и от тях се насели цялата земя.
\par 20 В това време Ной почна да работи земята и насади лозе.
\par 21 като пи от виното опи се и се разголи в шатрата си.
\par 22 И Хам, Ханаановия баща, видя голотата на баща си и каза на двамата си братя отвън.
\par 23 А Сим и Яфет взеха една дреха и, като я туриха двамата на рамената си, пристъпиха заднишком та покриха голотата на баща си; лицата им гледаха назад, та не видяха бащината си голота.
\par 24 Като изтрезня Ной от виното си и се научи за онова, което му бе направил по-младият ме син, рече: -
\par 25 Проклет да е Ханаан; Слуга на слуги ще бъде на братята си.
\par 26 Рече още: Благословен Господ, Симовия Бог; И Ханаан да му бъде слуга.
\par 27 Бог да разшири Яфета. И да се засели в шатрите на Сима; И Ханаан да им бъде слуга.
\par 28 И след потопа Ной живя триста и петдесет години.
\par 29 И всичките дни на Ноя станаха деветстотин и петдесет години; и умря.

\chapter{10}

\par 1 Ето потомството на Ноевите синове, Сима Хама и Яфета; че и на тях се родиха синове след потопа.
\par 2 Яфетови синове: Гомер, Магог, Мадай, Яван, Тувал, Мосох и Тирас.
\par 3 А Гомерови синове: Асханаз, Рифат и Тогарма.
\par 4 А Яванови синове: Елисей, Тарсис, Китим и Доданим.
\par 5 От тях се разделиха островите на народите, в техните земи, всеки според езика си, според племето си, в народите си.
\par 6 Хамови синове: Хус, Мицраим, Фут и Ханаан.
\par 7 А Хусови синове: Сева, Евила, Савта, Раама и Савтека; а Раамови синове: Шева и Дедан.
\par 8 Хус роди и Нимрода. Той пръв стана силен на земята.
\par 9 Той беше голям ловец пред Господа: затова се казва: Като Нимрода, голям Ловец пред Господа.
\par 10 Първо тоя царува над Вавилон Ерех, Акад и Халне, в Сенаарската земя.
\par 11 От тая земя излезе и отиде в Асирия, та съгради Ниневия, Роовот-Ир, Халах
\par 12 и Ресен между Ниневия и Халах (който е големия град).
\par 13 А Мицраим роди Лудим, Анамим, Леавим, Нафтухим.
\par 14 Патрусим, Каслухим (от които произлязоха филистимците) и Кафторим.
\par 15 А Ханаан роди: първородния си син Сидон, после Хет,
\par 16 евусейците, аморейците, гергесейците,
\par 17 евейците, арукейците, асенейците,
\par 18 арвадците, цемарейците и аматейците; и след това ханаанските племена се пръснаха.
\par 19 Пределът на ханаанците беше от Сидон, като се отива за Герар до Газа, и като се отива за Содом и Гомор, Адма и Цевоим да Лаша.
\par 20 Тия са Хамовите синове в земите си, в народите си, според племената си, според езиците си.
\par 21 Родиха се тъй също чада на Сима, баща на всичките Еверовци и по-стар брат на Яфета.
\par 22 Симови синове: Елам, Асур, Арфаксад, Луд и Арам.
\par 23 А Арамови синове: Уз, Ул, Гетер и Маш.
\par 24 И Арфаксад роди Сала, а Сала роди Евера.
\par 25 И на Евера се родиха двама сина; името на единия беше Фалек защото в неговите дни се разпредели земята; а името на брата му беше Иоктан.
\par 26 Иоктан роди Алмодада, Шалефа, Хацармавета, Яраха.
\par 27 Адорама, Узала, Дикла,
\par 28 Овала, Авимаила, Шева,
\par 29 Офира, Евила и Иовава; всички тия бяха Иоктанови синове.
\par 30 Техните поселения бяха от Меша, като се отива за Сефар, източната планина.
\par 31 Тия са Симовите синове в земите си, според племената си, според езиците си, според народите си.
\par 32 Тия са племената на Ноевите синове според поколенията им, в народите им; и от тях се отделиха народите по земята след потопа.

\chapter{11}

\par 1 А по цялата земя се употребяваше един език и един говор.
\par 2 И като потеглюваха човеците към изток, намериха поле в Сенаарската земя, гдето се и заселиха.
\par 3 И рекоха си един на друг: Елате, да направим тухли и да ги изпечем в огъня. Тухли употребяваха вместо камъни, а смола употребяваха вместо кал.
\par 4 И рекоха: Елате, да си съградим град, даже кула, чийто връх да стига до небето; и да си спечелим име, да не би да се разпръснем по лицето на цялата земя.
\par 5 А Господ слезе да види града и кулата, които градяха човеците.
\par 6 И рече Господ: Ето, едни люде са, и всички говорят един език; и това е което са почнали да правят; и не ще може вече да им се забрани, какво да било нещо, що биха намислили да направят.
\par 7 Елате да слезем, и там да разбъркаме езика им, тъй щото един други да не разбират езика си.
\par 8 Така Господ ги разпръсна от там по лицето на цялата земя; а те престанаха да градят града.
\par 9 За това той се наименува Вавилон, защото там Господ разбърка езика на цялата земя; и от там Господ ги разпръсна по лицето на цялата земя.
\par 10 Ето Симовото потомство: Сим беше на сто години, а роди Арфаксада две години след потопа;
\par 11 а откак роди Арфаксада, Сим живя петстотин години, и роди синове и дъщери.
\par 12 Арфаксад живя тридесет и пет години и роди Сала;
\par 13 а откак роди Сала, Арфаксад живя четиристотин и три години, и роди синове и дъщери.
\par 14 Сала живя тридесет години и роди Евера;
\par 15 а откак роди Евера, Сала живя четиристотин и три години, и роди синове и дъщери.
\par 16 Евер живя тридесет и четири години и роди Фалека;
\par 17 а откак роди Фалека, Евер живя четиристотин и тридесет години и роди синове и дъщери.
\par 18 Фалек живя тридесет години и роди Рагава;
\par 19 а откак роди Рагава, Фалек живя двеста и девет години и роди синове и дъщери.
\par 20 Рагав живя тридесет и две години и роди Серуха;
\par 21 а откак роди Серуха, Рагав живя двеста и седем години и роди синове и дъщери.
\par 22 Серух живя тридесет години и роди Нахора;
\par 23 а откак роди Нахора, Серух живя двеста години и роди синове и дъщери.
\par 24 Нахор живя двадесет и девет години и роди Тара;
\par 25 а откак роди Тара, Нахор живя сто и деветнадесет години и роди синове и дъщери.
\par 26 Тара живя седемдесет години и роди Аврама, Нахора и Арана.
\par 27 Ето потомството и на Тара: Тара роди Аврама, Нахора и Арана; а Аран роди Лота.
\par 28 И Аран умря преди баща си Тара в Ур Халдейски, в родната си земя.
\par 29 И Аврам и Нахор си взеха жени; името на Аврамовата жена бе Сарайя, а името на Нахоровата жена Мелха, Дъщеря на Арана, който освен че беше баща на Мелха, беше баща и на Есха.
\par 30 А Сарайя беше бездетна, нямаше чада.
\par 31 И Тара взе сина си Аврама и внука си Лота, Арановия син, и снаха си Сарайя, жената на сина си Аврама та излязоха от Ур Халдейски, за да отидат в Ханаанската земя; и дойдоха в Харан, гдето се и заселиха.
\par 32 И дните на Тара станаха двеста и пет години; и Тара умря в Харан.

\chapter{12}

\par 1 Тогава Господ каза на Аврама: Излез от отечеството си, измежду рода си и из бащиния си дом, та иди в земята, която ще ти покажа.
\par 2 Ще те направя голям народ; ще те благословя, и ще прославя името ти, и ще бъдеш за благословение.
\par 3 Ще благословя ония, които те благославят, а ще прокълна всеки, който те кълне; и в тебе ще се благославят всички земни племена.
\par 4 И тъй, Аврам тръгна според както му рече Господ, и Лот тръгна с него. А Аврам беше на седемдесет и пет години, когато излезе от Харан.
\par 5 Аврам взе жена си Сарайя, братанеца си Лот, всичкия имот, който бяха спечелили, и хората, които бяха придобили в Харан, та излязоха, за да отидат в Ханаанската земя.
\par 6 И Аврам пропътува земята до местността Сихем, до дъба Море. В това време ханаанците живееха в земята.
\par 7 И Господ се яви на Аврама и рече: На твоето потомство ще дам тая земя. И там издигна олтар на Господа, Който му се яви.
\par 8 От там се премести към хълма, който е на изток от Ветил, дето разпъна шатрата си, Ветил оставаше на запад, а Гай - на изток; и там издигна олтар на Господа и призова Господното име.
\par 9 После, като се дигна, Аврам все напредваше към юг.
\par 10 А настана глад в земята; затова Аврам слезе в Египет да поживее там, понеже гладът беше се усилил в Ханаанската земя.
\par 11 И когато наближи да влезе в Египет, рече на жена си Сарайя: Виж сега, зная, че си жена красива наглед.
\par 12 Египтяните, като те видят, ще рекат: Тя му е жена; и мене ще убият, а тебе ще оставят жива.
\par 13 Кажи, моля, че си ми сестра, за да ми бъда добре покрай тебе и да се опази живота ми, поради твоята дума .
\par 14 И като влезе Аврам в Египет, египтяните видяха, че жената беше твърде красива.
\par 15 Видяха я и Фараоновите големци и я похвалиха на Фараона; затова жената беше заведена в дома на Фараона.
\par 16 И заради нея той стори добро на Аврама, който достигна да има овци, говеда, осли, слуги, слугини, ослици и камили.
\par 17 Но Господ порази Фараона и дома му с тежки язви, поради Аврамовата жена Сарайя.
\par 18 Тогава Фараон повика Аврама и рече: Що е това, което ти ми стори? Защо не ми каза, че ти е жена?
\par 19 Защо ми каза: Сестра ми е, така че аз си я взех за жена; сега, прочее, ето жена ти; вземи я и си иди.
\par 20 И Фараон му определи човеци, които изпратиха него, жена му и всичко що имаше.

\chapter{13}

\par 1 Така Аврам излезе от Египет, той, жена му и всичко що имаше, и Лот с него, та замина към южната страна.
\par 2 Аврам беше много богат с добитък, сребро и злато.
\par 3 И от южната страна той минаваше постепенно дори до Ветил, до мястото, гдето от по-напред беше поставена шатрата му между Ветил и Гай.
\par 4 до мястото, гдето първоначално беше издигнат олтара; и там Аврам призова Господното име.
\par 5 Също и Лот, който придружаваше Аврама, имаше овци, говеда и шатри.
\par 6 И понеже земята не ги побираше да живеят заедно, тъй като имотът им беше много, и не можеха да живеят заедно,
\par 7 появи се спречкване между Аврамовите говедари и Лотовите говедари. (По това време ханаанците и ферезейците населяваха тая земя).
\par 8 Тогава Аврам рече на Лота: Да няма, моля ти се, спречкване между мене и тебе между моите говедари и твоите говедари, защото ние сме братя.
\par 9 Не е ли пред тебе цялата земя. Моля ти се, отдели се от мене; ти ако идеш наляво, то аз ще ида надясно; или ако ти идеш надясно, аз ще ида наляво.
\par 10 Лот, прочее, подигна очи и разгледа цялата равнина на Иордан и видя , че беше добре напоявана догде се стигне в Сигор, (преди да беше разорил Господа Содома и Гомора), като Господната градина, като Египетската земя.
\par 11 За това Лот си избра цялата Иорданска равнина. И Лот тръгна към изток и се разделиха един от друг.
\par 12 Аврам се засели в Ханаанската земя; а Лот се засели между градовете на Иорданската равнина, и преместваше шатрите си докато стигна до Содом.
\par 13 А содомските мъже бяха твърде нечестиви и грешни пред Господа.
\par 14 И Господ каза на Аврама, след като се отдели Лот от него: Подигни сега очите си от мястото гдето си, та погледни към север и юг, изток и запад;
\par 15 защото цялата земя, която виждаш, ще дам на тебе и на потомството ти до века.
\par 16 И ще направя потомството ти многочислено като земния прах; така щото, ако може някой да изброи земния прах, то и твоето потомство ще изброи.
\par 17 Стани, обходи дължината на земята, защото на теб ще я дам.
\par 18 Тогава Аврам премести шатрата си, дойде и се засели при Мемриевите дъбове, които са в Хеврон; и там издигна олтар на Господа.

\chapter{14}

\par 1 В дните не Сенаарския цар Амарфал, Еласарският цар Ариох, Еламският цар Ходологомор и Гоимският цар Тидал,
\par 2 тия царе отвориха война против Содомския цар Вера, Гоморския цар Верса, Адманския цар Сенав, Цевоимския цар Симовор и царя на Вала, (която е Сигор).
\par 3 Всички тия се събраха в Сидимската долина (дето е сега Соленото Море).
\par 4 Дванадесет години бяха се подчинявали на Ходологомора, а в тринадесетата въстанаха.
\par 5 В четиринадесетата година дойдоха Ходологомор и царете, които бяха с него, та поразиха рафаимите в Астарот-карнаим, зузимите в Хам, емимите в Сави-кириатаим
\par 6 и хорейците в поляната им Сиир, до Елфаран, който е при пустинята.
\par 7 А като се върнаха, дойдоха в Ен-мишнат, (който е Кадис), и поразиха цялата страна на амаликитяните, както и амореите, които живееха в Асасон-тамар.
\par 8 Тогава Содомският цар, Гоморският цар, Адманският цар, Цевоимският цар и царят на Вала, (която е Сигор), излязоха и се опълчиха против тях на бой в Сидимската долина:
\par 9 против Еламския цар Ходологомор, Гиомския цар Тидал, Сенаарския цар Амарфал и Еласарския цар Ариох. Имаше четирима царе против петимата.
\par 10 А Сидимската долина беше пълна със смолни кладенци; и Содомският и Гоморският царе, като бягаха, паднаха в тях, а останалите избягаха на бърдото.
\par 11 И победителите задигнаха всичкия имот на Содома и Гомора и всичката им храна и си отидоха.
\par 12 Взеха и Аврамовия братанец Лот, който живееше в Содом, заедно с имота му и си отидоха.
\par 13 А един от избавилите се дойде и извести това на евреина Аврам; той живееше при дъбовете на Амореца Мамврий, брат на Есхола и брат на Анера, които бяха Аврамови съюзници.
\par 14 А като чу Аврам, че брат му бил пленен, изведе своите триста и осемнадесет обучени мъже, родени в неговия дом и гони неприятеля до Дан.
\par 15 И през нощта, той и слугите му, като се разделиха против тях, поразиха ги и ги гониха до Хова, която е от ляво на Дамаск.
\par 16 И възвърна всичкия имот, върна на брата си Лота с имота му, както и жените и людете.
\par 17 И като се върна Аврам от поражението на Ходологомора и на царете, които бяха с него, Содомският цар излезе да го посрещне в Савинската долина, (която е Царевата долина).
\par 18 Така Салимският цар Мелхиседек, който беше свещеник на Всемогъщия Бог, изнесе хляб и вино
\par 19 та го благослови, казвайки: Благословен да бъде Аврам от Всевишния Бог, Създател на небето и на земята;
\par 20 благословен и Всевишният Бог, който предаде неприятелите ти в твоята ръка. И Аврам му даде десетък от всичко.
\par 21 А Содомският цар рече на Аврама: Дай ми човеците, а имота задръж за себе си.
\par 22 Но Аврам каза на Содомския цар: Аз дигнах ръката си пред Господа, Всевишния Бог, Създател на небето и земята,
\par 23 и се заклех , че няма да взема нищо от твоето, ни конец ни ремък за обуща, да не би да речеш: Аз обогатих Аврама.
\par 24 Приемам само онова, което изядоха момците; и дела на мъжете които отидоха с мене: Анер, Есхол и Мамври - те нека вземат дела си.

\chapter{15}

\par 1 След тия събития, дойде Господното слово на Аврама във видение и каза: Не бой се, Авраме; Аз съм твой щит, наградата ти е извънредно голяма.
\par 2 А Аврам рече: Господи Иеова, какво ще ми дадеш, като аз си отивам бездетен и тоя Елиезер от Дамаск ще притежава дома ми?
\par 3 Аврам рече още: Ето Ти не ми даде чадо; и, ето, един роден в дома ми ще ми стане наследник.
\par 4 Но, ето, дойде Господното слово и му каза: Тоя човек няма да ти стане наследник; но оня, който ще излезе от твоите чресла, ще ти бъде наследник.
\par 5 Тогава, като го изведе вън, каза: Погледни сега на небето и изброй звездите, ако можеш ги изброи. И рече му: Толкова ще бъде твоето потомство.
\par 6 И Аврам повярва в Господа; и Той му го вмени за правда.
\par 7 После му каза: Аз съм Господ, който те изведох от Ур Халдейски, за да ти дам да наследиш тая земя.
\par 8 А той рече: Господи Иеова, по какво да позная че ще я наследя?
\par 9 Господ му рече: Вземи ми тригодишна юница, тригодишна коза, тригодишен овен, гургулица и гълъбче.
\par 10 И той Му взе всички тия, разсече ги през средата и постави всяка половина срещу другата, но птиците не разсече.
\par 11 И спуснаха се хищни птици на труповете; но Аврам ги разпъди.
\par 12 А около захождането на слънцето, дълбок сън нападна Аврама; и, ето, ужас, като страшен мрак, го обзе.
\par 13 Тогава Господ каза на Аврама: Знаейки - знай, че твоето потомство ще бъде чуждо в чужда земя и ще им бъдат роби; и те ще ги угнетяват четиристотин години.
\par 14 Но Аз ще съдя народа, комуто ще робуват; и подир това ще излязат с голям имот
\par 15 А ти ще отидеш при бащите си в мир; ще бъдеш погребан в честита старост.
\par 16 А в четвъртия род потомците ти ще се върнат тука; защото беззаконието на аморейците не е още стигнало до върха си.
\par 17 А когато слънцето залезе и настана мрак, ето димяща пещ и огнен пламък, който премина между тия части.
\par 18 И в същия ден Господ направи завет с Аврама, като каза: На твоето потомство давам тая земя, от Египетската река до голямата река, реката Ефрат,
\par 19 земята на кенейците, кенезейците, кадмонейците,
\par 20 хетейците, ферезейците, рафаимите,
\par 21 аморейците, ханаанците, гергесейците и евусейците.

\chapter{16}

\par 1 А Сарайя, жената на Аврама, не му раждаше деца; но, като имаше слугиня египтянка, на име Агар,
\par 2 Сарайя рече на Аврама: Виж сега, Господ не ми дава да раждам; моля ти се влез при слугинята ми; може би ще придобия чадо, чрез нея. И Аврам послуша това, което каза Сарайя.
\par 3 И тъй, след като Аврам беше преживял десет години в Ханаанската земя, Сарайя Аврамовата жена, взе слугинята си Агар, египтянката, и я даде на мъжа си Аврама да му бъде жена.
\par 4 И той влезе при Агар и тя зачна; и като видя че зачна, господарката й стана презряна в очите й.
\par 5 Тогава Сарайя рече на Аврама: Поради тебе ми дойде тая обида. Дадох слугинята си в твоите обятия; а като видя, че зачна, аз станах презряна в очите й. Господ нека съди между мене и тебе.
\par 6 А Аврам рече на Сарайя: Ето, слугинята ти; стори с нея, както ти се вижда угодно. Прочее, Сарайя се отнасяше зле с нея, така щото тя побягна от лицето й.
\par 7 Но ангел Господен я намери при един воден извор в пустинята, при извора на пътя за Сур;
\par 8 и рече: Агар, Сараина слугиньо, от где идеш? и къде отиваш? А тя рече: Бягам от лицето на господарката си Сарайя.
\par 9 А ангелът Господен й рече: Върни се при господарката си, и покори се под властта й.
\par 10 Ангелът Господен тоже и рече: Ще преумножа потомството ти, до толкоз, че да не може да се изброи, поради своето множество.
\par 11 После ангелът Господен й каза: Ето ти си зачнала, и ще родиш син; да го наименуваш Исмаил, защото Господ чу скръбта ти.
\par 12 Той ще бъде между човеците, като див осел; ще дига ръката си против всекиго и всеки ще дига ръката си против него; и той ще живее независим от всичките си братя.
\par 13 Тогава Агар даде на Господа, Който й говореше, това име: Ти си Бог, Който вижда; защото рече: Не видях ли тук аз Онзи, Който ме вижда?
\par 14 Затова, тоя кладенец се наименува Вир-лахай-рой§; ето той е между Кадис и Варад.
\par 15 И Агар роди син на Аврама; и Аврам наименува сина си, роден от Агар, Исмаил.
\par 16 А Аврам беше на осемдесет и шест години когато Агар му роди Исмаила.

\chapter{17}

\par 1 Когато Аврам беше на деветдесет и девет години, Господ се яви на Аврама и му рече: Аз съм Всемогъщият; ходи пред Мене и бъди непорочен.
\par 2 И ще направя завета си между Мене и тебе и ще те умножа твърде много.
\par 3 Тогава Аврам падна на лицето си; и Бог му говореше, казвайки:
\par 4 Ето, Моят завет е с тебе; и ти ще станеш отец на множество народи.
\par 5 Не ще се именуваш вече Аврам, но името ти ще бъде Авраам; защото те направих отец на множество народи.
\par 6 Ще те наплодя твърде много и ще произведа народи от тебе; и царе ще произлязат от тебе.
\par 7 И ще утвърдя завета Си между Мене и тебе и потомците ти след тебе през всичките им поколения за вечен завет, че ще бъда Бог на тебе и на потомството след тебе.
\par 8 На тебе и на потомството ти след тебе ще дам за вечно притежание земята, в която си пришелец, цялата Ханаанска земя; и ще им бъда Бог.
\par 9 Бог още каза на Авраама: Пази ти завета Ми, ти и потомците ти след тебе, през всичките си поколения.
\par 10 Ето моят завет, който трябва да пазите между Мене и вас и потомците ти след тебе: всеки между вас от мъжки пол да се обрязва.
\par 11 Да обрязвате краекожието на плътта си; и това ще бъде знак на завета между Мене и вас.
\par 12 Всяко мъжко дете между вас във всичките ви поколения, като стане на осем дни трябва да се обрязва, както роденото у дома ти, така и онова, което не е от твоето потомство, купена с пари от някой чужденец.
\par 13 Непременно трябва да се обрязва и роденият у дома ти и купеният с парите ти; и Моят завет в плътта ви, ще бъде вечен завет.
\par 14 А необрязаният от мъжки пол, чието краекожие на плътта не е обрязано, тоя човек, да се погуби измежду людете си, защото е нарушил завета Ми.
\par 15 После Бог каза на Авраама: Не наричай вече Сарайя жена си Сарайя; но Сара да бъде името й.
\par 16 Аз ще я благословя, още и син ще ти дам от нея; да! ще я благословя, и тя ще стане майка на народи; царе на племена ще произлязат от нея.
\par 17 Тогава Авраам падна на лицето си и се засмя, и рече в сърцето си: На стогодишен човек ли ще са роди дете? И Сара, която име деветдесет години, ще роди ли?
\par 18 И рече Авраам на Бога; Исмаил да е жив пред Тебе.
\par 19 Но Бог каза: Не, а жена ти Сара ще ти роди син, и ще го наречеш Исаак; и с него ще утвърдя завета Си за вечен завет, който ще бъде и за потомството му след него.
\par 20 И за Исмаила те послушах. Ето, благослових го, и ще го наплодя и преумножа; дванадесет племена ще се родят от него, и ще го направя велик народ.
\par 21 Но завета Си ще утвърдя с Исаака, когото Сара ще ти роди до година по това време.
\par 22 А като прекрати думата Си с Авраама, Бог се възлезе от него.
\par 23 В тоя същия ден Авраам взе сина си Исмаила, всичките родени у дома му и всичките купени с парите му, всеки от мъжки пол между човеците на Авраамовия дом и обряза краекожието на плътта им, според както Бог му каза.
\par 24 Авраам беше на деветдесет и девет години когато се обряза краекожието на плътта му;
\par 25 а син му Исмаил беше на тринадесет години, когато се обряза краекожието на плътта му.
\par 26 В един и същи ден се обрязаха Авраам и син му Исмаил.
\par 27 И всичките мъже от дома му както родените у дома, така и купените от чужденци с пари, се обрязаха заедно с него.

\chapter{18}

\par 1 След това Господ се яви на Авраама при Мамвриевите дъбове, когато той седеше при входа на шатрата си в горещината на деня.
\par 2 Като подигна очи и погледна, ето, трима мъже стояха срещу него; и като ги видя, затече се от входа на шатрата да ги посрещне, поклони се до земята, и рече:
\par 3 Господарю мой, ако съм придобил твоето благоволение, моля Ти се, не отминавай слугата слугата Си.
\par 4 Нека донесат малко вода, та си умийте нозете и си починете под дървото.
\par 5 И аз ще донеса малко хляб, да подкрепите сърцата си, и после ще си заминете; понеже затова дойдохте при слугата си. А те рекоха: Стори, както си казал.
\par 6 Тогава Авраам влезе бързо в шатрата при Сара и рече: Приготви по-скоро три мери чисто брашно, замеси и направи пити.
\par 7 А Авраам се завтече при чердата, взе едно младо, добро теле и го даде на слугата, който побърза да го сготви.
\par 8 После взе масло, мляко и сготвеното теле, та сложи пред тях; и стоеше при тях под дървото, докато ядяха.
\par 9 Тогава му рекоха: Где е жена ти Сара? А той рече: Ето, в шатрата е.
\par 10 И рече Господ : До година по това време Аз непременно ще се върна при тебе и, ето, жена ти Сара ще има син. А Сара слушаше от входа на шатрата, която беше зад него.
\par 11 А Авраам и Сара бяха стари, в напреднала възраст; на Сара беше престанало обикновеното на жените.
\par 12 И тъй Сара се засмя в себе си, като думаше: Като съм остаряла ще има ли за мене удоволствие, като и господаря ми е стар?
\par 13 А Господ рече на Авраама: Защо се засмя Сара и каза: Като съм остаряла, дали наистина ще родя?
\par 14 Има ли нещо невъзможно за Господа? На определеното време ще се върна при тебе, до година по това време и Сара ще има син.
\par 15 Тогава Сара, понеже се уплаши, се отрече, казвайки: Не съм се смяла. А той каза: Не е тъй ; ти се засмя.
\par 16 Като станаха от там, мъжете се обърнаха към Содом; и Авраам отиде с тях, за да ги изпрати.
\par 17 И Господ рече: Да открия ли от Авраама това, което ще сторя,
\par 18 тъй като Авраам непременно ще стане велик и силен народ и чрез него ще се благословят всичките народи на земята?
\par 19 Защото съм го избрал, за да заповяда на чадата си и на дома си след себе си да пазят Господния път, като вършат правда и правосъдие, за да направи Господ да стане с Авраама онова, което е говорил за него.
\par 20 И рече Господ: Понеже викът на Содома и Гомора е силен и, понеже грехът им е твърде тежък,
\par 21 ще сляза сега и ще видя дали са сторили напълно според вика, който стигна до Мене; и ако не, ще узная.
\par 22 Тогава мъжете, като се обърнаха от там, отидоха към Содом. Но Авраам още стоеше пред Господа.
\par 23 И Авраам се приближи и рече: Ще погубиш ли праведния с нечестивия?
\par 24 Може да има петдесет праведника в града; ще погубиш ли местата, не ще ли го пощадиш, заради петдесет праведника, които са в него?
\par 25 Далеч от Тебе да сториш такова нещо, да убиеш праведника с нечестивия, така щото праведния да бъде като нечестивия! Далеч от Тебе това! Съдията на цялата земя няма ли да върши правда?
\par 26 И Господ каза: Ако намеря в Содом петдесет праведника, вътре в града, ще пощадя цялото място, заради тях.
\par 27 А в отговор Авраам рече: Ето сега, аз, който съм прах и пепел, се осмелих да говоря на Господа;
\par 28 може би на петдесет праведника да не достигат пет; ще погубиш ли целия град, поради липсата на пет души? Той каза: Няма да го погубя ако намеря там четиридесет и пет.
\par 29 А Авраам пак Му говори, казвайки: Може да се намерят там четиридесет. Той каза: Заради четиридесет няма да сторя това.
\par 30 А Авраам рече: Да се не разгневи Господ и ще кажа: Може да се намерят там тридесет. Той каза: Няма да сторя това, ако намеря там тридесет.
\par 31 А Авраам рече: Ето сега, осмелих се да говоря на Господа; може да се намерят там двадесет. Той каза: Заради двадесетте няма да погубя града .
\par 32 Тогава Авраам рече: Да се не разгневи Господ и аз ще продумам пак, само тоя път. И той каза: Може да се намерят там десет. И той каза: Заради десет няма да го погубя.
\par 33 А като престана да говори с Авраама Господ си отиде; а Авраам се върна на мястото си.

\chapter{19}

\par 1 Привечер дойдоха двама ангела в Содом; а Лот седеше в Содомската порта. И като ги видя, Лот стана да ги посрещне, поклони се с лице до земята и рече:
\par 2 Ето, господари мои, свърнете, моля, в къщата на слугата си, пренощувайте и си умийте нозете, и утре станете та си идете по пътя. А те рекоха: Не, на улицата ще пренощуваме.
\par 3 Но, като настояваше много, те се отбиха към него и влязоха в къщата му; и той им направи угощение и изпече безквасни хлябове; и ядоха.
\par 4 А преди да си легнат те, градските мъже, Содомските жители, млади и стари, всичките люде от всякъде, обиколиха къщата
\par 5 и викаха на Лота, казвайки: Где са мъжете, които дойдоха у тебе тая нощ? Изведи ни ги да ги познаем.
\par 6 А Лот излезе при тях пред вратата, затвори вратата след себе си, и рече:
\par 7 Моля ви се, братя мои, не правете такова нечестие.
\par 8 Вижте сега, имам две дъщери, които не са познали мъж; тях да ви изведа вън, и сторете с тях, каквото ви се вижда угодно; само на тия мъже не струвайте нищо, понеже затова са влезли по покрива на стрехата ми.
\par 9 Но те рекоха: Махни се нататък. Рекоха още: Той дойде тук самичък и пришелец, а иска още и съдия да стане; ей сега на тебе ще сторим по-голямо зло, отколкото на тях. И насилваха премного човека, Лота, и приближиха се да разбият вратата.
\par 10 Но мъжете простряха ръце, дръпнаха Лота при себе си, в къщи, и затвориха вратата.
\par 11 Тоже и поразиха със слепота човеците, които бяха пред вратата на къщата, и малък и голям, тъй щото се умориха, като търсеха вратата.
\par 12 Тогава мъжете рекоха на Лота: Имаш ли тука друг някой: зет, синове, дъщери и които и да било други, що имаш в града, изведи ги из това място;
\par 13 защото ние ще съсипем мястото, понеже силен стана викът им пред Господа, и Господ ни изпрати да го съсипем.
\par 14 И тъй, Лот излезе и говори на зетьовете си, които щяха да водят дъщерите му и рече: Станете, излезте из това място, защото Господ ще съсипе града. Но не зетьовете му се видя, че той се шегува.
\par 15 Когато се зазори, ангелите настояваха пред Лота, казвайки: Стани, вземи жена си и двете си дъщери, които са тука, за да не погинеш всред наказанието на тоя град.
\par 16 Но тоя се бавеше; затова мъжете хванаха за ръка него, жена му и двете му дъщери, изведоха го и поставиха го вън от града; понеже Господ го пожали.
\par 17 И като го изведоха вън, рече единият на Лота : Бягай за живота си; да не погледнеш назад, нито да се спреш някъде в цялата тая равнина; бягай на планината, за да не погинеш.
\par 18 А Лот им рече: Ах, Господи, не така!
\par 19 Ето, слугата ти придоби Твоето благоволение, и с опазването на живота ми, Ти правиш още по-голяма милостта, която си показал към мене; но аз не мога да побягна на планината да не би да ме постигне злато и да умра.
\par 20 Гледай, моля, твоя град е близо, за да побягна там, и малък е. Нека побягна там, (не е ли малък град ?) и така животът ми ще се опази.
\par 21 Той му каза: Ето слушам те и за това нещо, че няма да разоря града, за който ти говори.
\par 22 Бързай, бягай там защото Аз не мога да сторя нищо, догде ти не стигнеш там. Затова тоя град се наименува Сигор.
\par 23 Слънцето изгряваше на земята, когато Лот влезе в Сигор.
\par 24 Тогава Господ изля върху Содом и Гомор сяра и огън от Господа от небето.
\par 25 Той разори тия градове и цялата равнина, всичките жители на градовете и земните растения.
\par 26 Но жена му, след него, погледна назад, и стана стълб от сол.
\par 27 Сутринта Авраам подрани на мястото, дето беше стоял пред Господа;
\par 28 и погледна към Содом и Гомор и към цялата земя на равнината, и, ето видя, че дим, като дим от пещ, се издигаше от земята.
\par 29 И тъй, когато Бог разоряваше градовете на тая равнина, Бог си спомни за Авраама и изпрати Лота изсред разорението, когато разори градовете, дето живееше Лот.
\par 30 А Лот излезе от Сигор и живееше в планината, и с него двете му дъщери, понеже се боеше да остане в Сигор; затова живееше в една пещера, той и двете му дъщери.
\par 31 Тогава по-старата рече на по-младата: Баща ни е стар, и няма мъж на земята да влезе при нас, според обичая на цялата земя.
\par 32 Ела, да упоим баща си с вино и да преспим с него, за да запазим потомството от баща си.
\par 33 И тъй, оная нощ упоиха баща си с вино; и по-старата влезе та преспа с баща си; а той не усети, нито кога легна тя, нито кога влезе.
\par 34 На другия ден по-старата рече на по-младата: Виж, миналата нощ аз преспах с баща си; да го упоим с вино и тая нощ, та влез ти и преспи с него, за да запазим потомството от баща си.
\par 35 И тъй, оная нощ упоиха баща си с вино, и по-младата влезе та преспа с него; а той не усети нито кога легна тя, нито кога стана.
\par 36 Така и двете лотови дъщери зачнаха от баща си.
\par 37 И по-старата роди син и го наименува Моав; той е до днес отец на моавците.
\par 38 Роди и по-младата син и го наименува Бен-ами; той е и до днес отец на амонците.

\chapter{20}

\par 1 От там Авраам пътуваше към южната страна. Той се настани между Кадис и Сур и живееше като пришелец в Герар.
\par 2 И понеже Авраам казваше за жена си Сара: Сестра ми е, то Герарският цар Авимелех прати та взе Сара.
\par 3 Но Бог дойде при Авимелеха в съня му, през нощта и му рече: Ето, ти умираш поради жената, която си взел; защото тя си има мъж.
\par 4 (А Авимелех не беше се приближил при нея). И рече: Господи, ще погубиш ли един праведен народ?
\par 5 Не каза ли сам той: Сестра ми е? Също и сама тя рече: Той ми е брат. С право сърце и с чисти ръце сторих аз това.
\par 6 Бог му каза в съня: Да, Аз зная, че си сторил това с праведно сърце; още Аз те въздържах да не съгрешиш против Мене, и затова не те оставих да се докоснеш до нея.
\par 7 Сега, прочее, върни жената на човека, защото е пророк и ще се помоли за тебе, и ти ще останеш жив; но, ако не я върнеш, знай, че непременно ще умреш, ти и всички твои.
\par 8 На сутринта, като стана рано Авимелех повика всичките си слуги и извести всички тия неща в ушите им; и хората се уплашиха много.
\par 9 Тогава Авимелех повика Авраам и му рече: Що ни стори ти? Какво ти съгреших, та си навлякъл на мене и на царството ми голям грях? Направил си ми работи, които не трябваше да се правят.
\par 10 Рече още Авимелех на Авраама: Що си видял, та стори това нещо?
\par 11 И рече Авраам: Сторих го понеже рекох: Не ще има страх от Бога в това място, и ще ме убият поради жена ми.
\par 12 А при това, тя наистина ми е сестра, дъщеря на баща ми, но не и дъщеря на майка ми; и ми стана жена.
\par 13 И когато Бог ме направи да странствувам от бащиния си дом, рекох й: Ще ми направиш тая добрина - На всяко място, гдето отидем, казвай за мене: Брат ми е.
\par 14 Тогава Авимелех взе овци и говеда, слуги и слугини, та ги даде на Авраама и върна му жена му Сара.
\par 15 И рече Авимилех: Ето, земята ми е пред тебе, засели се дето ти е угодно.
\par 16 А на Сара каза: Виж, дадох на брат ти хиляда сребърника; ето, това ти е покривало за очите пред всички, които са с тебе и пред всички човеци си оправдана.
\par 17 И тъй, Авраам се помоли на Бога; и Бог изцели Авимелеха, и жена му, и слугите му; и раждаха деца.
\par 18 Защото, поради Авраамовата жена Сара, Господ беше заключил съвсем всички утроби на Авимелеховия дом.

\chapter{21}

\par 1 И Господ посети Сара, според както беше рекъл; и Господ стори на Сара както бе казал.
\par 2 Защото Сара зачна и роди син на Авраама в старините му, в определения му от Бога срок.
\par 3 И Авраам наименува сина , който му се роди, когото Сара му роди, Исаак.
\par 4 И на осмия ден Авраам обряза сина си Исаака, според както Бог му беше заповядал.
\par 5 А Авраам беше на сто години, когато се роди сина му Исаак.
\par 6 И Сара каза: Бог ме направи за смях; всеки, който чуе, ще ми се смее.
\par 7 Каза още: Кой би рекъл на Авраама, че Сара ще кърми чада? - Защото му родих син в старините му.
\par 8 А като порасна детето, отбиха го; и в деня, когато отбиха Исаака, Авраам направи голямо угощение.
\par 9 А Сара видя, че синът на египтянката Агар, когото бе родила на Авраама се присмива;
\par 10 затова рече на Авраама: Изпъди тая слугиня и сина й ; защото синът на тая слугиня няма наследство с моя син Исаак.
\par 11 Обаче тая дума се видя на Авраама твърде тежка, поради сина му Исмаил .
\par 12 Но Бог каза на Авраама: Да не ти се види тежко за момчето и за слугинята ти; относно всичко, което ти рече Сара, послушай думите й, защото по Исаака ще се наименува твоето потомство.
\par 13 Но и от сина на слугинята ще направя да стане народ, понеже е твое чадо.
\par 14 Тогава на сутринта Авраам стана рано, взе хляб и мех с вода и даде ги на Агар, като ги тури на рамото й; даде й още детето и я изпрати.А тя отиде и се заблуди в пустинята Вирсавее.
\par 15 Но изчерпи се водата в меха; и майка му хвърли детето под един храст
\par 16 и отиде та седна на среща, далеч колкото един хвърлей на стрела, защото си рече: Да не гледам, като умира детето. И като седна насреща, издигна глас и заплака.
\par 17 И Бог чу гласа на момчето; и ангел Божий извика към Агар от небето и рече й: Що ти е, Агар? Не бой се, защото Бог чу гласа на момчето от мястото гдето е.
\par 18 Стани, дигни момчето и крепи го с ръката си, защото ще направя от него велик народ.
\par 19 Тогава Бог й отвори очите, и тя видя кладенец с вода; и отиде да напълни меха с вода и даде на момчето да пие.
\par 20 Бог беше с момчето, което порасна, засели се в пустинята и стана стрелец.
\par 21 Засели се във Фаранската пустиня; и майка му взе жена от Египетската земя.
\par 22 По онова време Авимелех, с военачалника си Фихола, говори на Авраама, казвайки: Бог е с тебе във всичко що правиш.
\par 23 Сега, прочее, закълни ми се тук в Бога, че не ще постъпваш неверно с мене, ни със сина ми, нито с внука ми; но, според благостта, която съм показал към тебе, ще показваш и ти към мене и към земята, в която си пребивавал.
\par 24 И рече Авраам: Заклевам се.
\par 25 Подир това Авраам изобличи Авимелеха за водния кладенец, който Авимелеховите слуги бяха отнели на сила.
\par 26 Но рече Авимелех: Не знам кой е сторил това нещо; нито ти си ми явил за това, нито аз съм чул, освен днес.
\par 27 Тогава Авраам взе овци и говеда и ги даде на Авимелеха, та двамата сключиха договор помежду си .
\par 28 А Авраам отдели седем женски агнета от стадото.
\par 29 И Авимелех каза на Авраама: Какви са тия женски агнета, които си отделил?
\par 30 А той рече:Тия седем женски агнета ще вземеш от мене, да ми бъдат за свидетелство, че аз съм изкопал тоя кладенец.
\par 31 Затова той наименува онова място Вирсавее, защото там се заклеха двамата.
\par 32 Така те сключиха договор във Вирсавее: и след това станаха Авимелех и военачалникът Фихол и се върнаха във Филистимската земя.
\par 33 И Авраам посади дъбрава във Вирсавее, и там призова името на Иеова, Вечния Бог.
\par 34 И Авраам престоя във Филистимската земя много дни.

\chapter{22}

\par 1 След тия събития Бог изпита Авраама, като му рече: Аврааме. А той рече: Ето ме.
\par 2 И рече Бог : Вземи сега единствения си син, когото любиш, сина си Исаака, та иди в местността Мория и принесе го там във всеизгаряне на един от хълмовете, за който ще ти кажа.
\par 3 На сутринта, прочее, Авраам подрани та оседла осела си и взе със себе се двама от слугите си и сина си Исаака; и, като нацепи дърва за всеизгарянето, стана та отиде на мястото, за което Бог му беше казал.
\par 4 На третия ден Авраам подигна очи и видя мястото надалеч.
\par 5 Тогава рече Авраам на слугите си: Вие останете тук с осела; а аз и момчето ще отидем до там и, като се поклоним, ще се върнем при вас.
\par 6 И взе Авраам дървата за всеизгарянето и натовари ги на сина си Исаака, а той взе в ръката си огън и нож; и двамата отидоха заедно.
\par 7 Тогава Исаак продума на баща си Авраама, казвайки: Тате! А той рече: Ето ме, синко. И рече Исаак : Ето огъня и дървата, а где е агнето за всеизгарянето?
\par 8 И Авраам каза: Синко, Бог ще си промисли агнето за всеизгаряне. И двамата вървяха заедно.
\par 9 А като стигнаха на мястото, за което Бог му беше казал, Авраам издигна там жертвеник, нареди дървата и, като върза сина си Исаака, тури го на жертвеника върху дърветата.
\par 10 И Авраам простря ръката си та взе ножа да заколи сина си.
\par 11 Тогава ангел Господен му викна от небето и рече: Аврааме, Аврааме! И той рече: Ето ме.
\par 12 И ангелът рече: да не вдигнеш ръката си върху момчето, нито да му сториш нещо; защото сега зная, че ти се боиш от Бога, понеже на пожали за Мене и сина си, единствения си син.
\par 13 Тогава Авраам, като подигна очи, видя, и ето зад него един овен вплетен с рогата си в един храст; и Авраам отиде, взе овена и го принесе всеизгаряне вместо сина си.
\par 14 И Авраам наименува това място Иеова-ире; и според това се казва и до днес: На хълма Господ ще промисли.
\par 15 Тогава втори път ангел Господен викна на Авраама от небето и рече:
\par 16 В Себе Си се заклевам, казва Господ, че понеже си сторил това нещо и не пожали сина си, единствения си син,
\par 17 ще те благословя премного и ще умножа и преумножа потомството ти като небесните звезди и като пясъка на морския бряг; и потомството ти ще завладее портата на неприятелите си;
\par 18 в твоето потомство ще се благословят всичките народи на земята, защото си послушал гласа Ми.
\par 19 И тъй, Авраам се върна при слугите си, и станаха та отидоха заедно във Вирсавее; и Авраам се настани във Вирсавее.
\par 20 А след тия събития, известиха на Авраама, казвайки: Ето, и Мелха роди чада на брата ми Нахор:
\par 21 първородния му Уза, брата ми Вуза, Камуила Арамовия баща.
\par 22 Кеседа, Азава, Фалдеса, Елдафа и Ватуила.
\par 23 А Ватуил роди Ревека. Тия осем сина роди Мелха на Нахора Авраамовия брат.
\par 24 Тоже и наложницата му, на име Ревма, роди Тевека, Гаама, Тахаса и Мааха.

\chapter{23}

\par 1 Сара живя сто двадесет и седем години; тия бяха годините на Сариния живот.
\par 2 И Сара умря в Кириат-арва, (който е Хеврон), в Ханаанската земя; и Авраам дойде да жалее Сара и да я оплаче.
\par 3 И като стана Авраам от покойницата си, говори на хетейците, казвайки:
\par 4 Пришелец и заселник съм аз между вас; дайте ми място за погребване между вас, което да е мое , за да погреба покойницата си пред очите си.
\par 5 А хетейците в отговор на Авраама му казаха:
\par 6 Послушай ни, господарю; между нас ти си княз Божий; погреби покойницата си в най-добрата от гробниците ни; никой от нас не ще ти откаже гробницата си, за да погребеш покойницата си.
\par 7 Тогава стана Авраам, поклони се на людете на земята, на хетейците, и говори с тях, казвайки:
\par 8 Ако ви е угодно да погреба покойницата си пред очите си, послушайте ме и станете посредници за мене пред Сааровия син Ефрон,
\par 9 за да ми даде пещерата си Махпелах, която е на края на нивата му; с пълна цена нека ми я даде всред вас, като собственост за гробница.
\par 10 А Ефрон седеше всред хетейците; и Ефрон, хетеецът, отговори на Авраама пред хетейците, които слушаха, пред всичките, които влизаха в портата на града му, казвайки:
\par 11 Не, господарю, послушай ме; давам ти нивата, давам ти и пещерата, която е в нея; давам ти я пред тия мъже от людете си; погреби покойницата си.
\par 12 Тогава Авраам се поклони пред людете на земята
\par 13 и говори на Ефрона пред людете на земята, които слушаха, казвайки: Но, ако обичаш, моля, послушай ме: ще дам стойността на нивата; вземи я от мене и ще погреба покойницата си там.
\par 14 А в отговор Ефрон каза на Авраама:
\par 15 Чуй ме, господарю: земя за четиристотин сребърни сикли, какво е между мене и тебе? Погреби, прочее, покойницата си.
\par 16 И Авраам прие казаното от Ефрона; и Авраам претегли на Ефрона парите, които определи пред хетейците, които слушаха, четиристотин сребърни сикли, каквито вървяха между търговците.
\par 17 И тъй, Ефроновата нива, която беше в Махпелах, срещу Мамврий,- нивата, пещерата, която бе в нея, и всичките дървета в нивата, в междите около цялата нива,
\par 18 станаха собственост на Авраама пред очите на синовете на Хета, пред всичките, които влизаха в портата на неговия град.
\par 19 След това Авраам погреба жена си Сара в пещерата на нивата Махпелах, срещу Мамврий, (който е Хеврон), в Ханаанската земя.
\par 20 Така нивата и пещерата в нея се утвърдиха на Авраама от хетейците, като собствено гробище.

\chapter{24}

\par 1 А когато Авраам беше стар, напреднал във възраст, и Господ беше благословил Авраама във всичко,
\par 2 рече Авраам на най-стария слуга в дома си, който беше настоятел на всичкия му имот: Тури, моля, ръката си под бедрото ми;
\par 3 и ще те закълна в Господа, Бог на небето и Бог на земята, че няма да вземеш жена за сина ми от дъщерите на ханаанците, между които живея;
\par 4 но да отидеш в отечеството ми, при рода ми, и от там да вземеш жена за сина ми Исаака.
\par 5 А слугата му рече: Може да не иска жената да дойде след мене в тая земя; трябва ли да заведа сина ти в оная земя, отгдето си излязъл?
\par 6 Но Авраам му каза: Пази се да не върнеш сина ми там.
\par 7 Господ, небесният Бог, Който ме изведе из бащиния ми дом и от родната ми земя, и Който ми говори, и Който ми се закле, казвайки: На твоето потомство ще дам тая земя; Той ще изпрати ангела си пред тебе и ще вземеш жена за сина ми оттам.
\par 8 Но, ако жената не иска да дойде след тебе, тогава ти ще бъдеш свободен от това мое заклеване; само да не върнеш сина ми там.
\par 9 Тогава слугата тури ръката си под бедрото на господаря си Авраама и му се закле за това нещо.
\par 10 И тъй, слугата взе десет от камилите на господаря си и тръгна, като носеше в ръцете си от всички богатства на господаря си; стана и отиде в Месопотамия, в Нахоровия град.
\par 11 И надвечер, когато жените излизат да си наливат вода, той накара камилите да коленичат вън от града при водния кладенец.
\par 12 Тогава каза: Господи, Боже на господаря ми Авраама, дай ми, моля, добър успех днес и покажи благост към господаря ми Авраама.
\par 13 Ето, аз стоя при извора на водата, и дъщерите на града излизат да си наливат вода.
\par 14 Нека момата, на която река: Я наведи водоноса си да пия; и тя рече: Пий, и ще напоя и камилите ти; -тя нека е оная, която си отредил за слугата си Исаака; от това ще позная, че си показал милост към господаря ми.
\par 15 Докато той още говореше, ето Ревека излизаше с водоноса си на рамото си; тя бе се родила на Ватуила, син на Мелха, жената на Авраамовия брат Нахор.
\par 16 Момата беше твърде красива на глед, девица, която никой мъж не беше познал; тя, като слезе на извора, напълни водоноса си и се изкачи.
\par 17 А слугата се завтече да я посрещне и рече: Я дай ми да пия малко вода от водоноса ти.
\par 18 А тя рече: Пий, господарю; и бърже сне водоноса си на ръката си, и му даде да пие.
\par 19 И като му даде доволно да пие, рече: И за камилите ти ще налея догде се напоят.
\par 20 И като изля бърже водоноса си в поилото, завтече се на кладенеца да налее още и наля за всичките му камили.
\par 21 А човекът я наблюдаваше внимателно и мълчеше, за да узнае дали Господ бе направил пътуването му успешно или не.
\par 22 Като напоиха камилите, човекът взе една златна обица, тежка половин сикъл, и две гривни за ръцете й, тежки десет сикли злато, и рече:
\par 23 Чия си дъщеря? - я ми кажи. Има ли в къщата на баща ти място за нас да пренощуваме?
\par 24 А тя му рече: Аз съм дъщеря на Ватуила, син на Мелха, когото тя е родила на Нахора.
\par 25 Рече му още: Има у нас и слама, и храна доволно, и място за пренощуване.
\par 26 Тогава човекът се наведе, та се поклони на Господа.
\par 27 И рече: Благословен да бъде Господ, Бог на господаря ми Авраама, Който не лиши господаря ми от милостта си и верността си, като отправи Господ пътя ми в дома на братята на господаря ми.
\par 28 А момата се завтече, та разказа тия неща на майка си.
\par 29 И Ревека имаше брат на име Лаван; и Лаван се завтече вън при човека на извора.
\par 30 Защото, като видя обицата и гривните на ръцете на сестра си и чу думите на сестра си Ревека, която казваше: Така ми говори човекът; той отиде при човека; и, ето, той стоеше при камилите до извора.
\par 31 И рече му: Влез, ти, благословен от Господа; защо стоиш вън? Защото приготвих къщата и място за камилите.
\par 32 И тъй, човекът влезе вкъщи; а Лаван разтовари камилите му и даде слама и храна за камилите, и вода за умиване неговите нозе и нозете на хората, които бяха с него.
\par 33 И сложиха пред него да яде; но той рече: Не искам да ям, догде не кажа думата си. А Лаван рече: Казвай.
\par 34 Тогава той каза: Аз съм слуга на Авраама.
\par 35 Господ е благословил господаря ми премного, та стана велик; дал му е овци и говеда, сребро и злато, слуги и слугини, камили и осли.
\par 36 И Сара, жената на господаря ми, роди син на господаря ми, когато беше вече остаряла; и на него той даде всичко, що има.
\par 37 И господарят ми, като ме закле, рече: Да не вземеш за сина ми жена от дъщерите на ханаанците, в чиято земя живея.
\par 38 Но да отидеш в бащиния ми дом и в рода ми, и от там да вземеш жена за сина ми.
\par 39 И рекох на господаря ми: Може да не иска жената да дойде след мене.
\par 40 Но той ми каза: Господ, пред Когото ходя, ще изпрати ангела си с тебе и ще направи пътуването ти успешно; и ще вземеш жена за сина ми от рода ми и от бащиния ми дом.
\par 41 Само тогава ще бъдеш свободен от заклеването ми, когато отидеш при рода ми; и, ако не ти я дадат, тогава ще бъдеш свободен от заклеването ми.
\par 42 И като пристигнах днес на извора, рекох: Господи, Боже на господаря ми Авраама, моля, ако правиш пътуването ми, по което ходя, успешно,
\par 43 ето, аз стоя при водния извор; момата, която излезе да налее вода, и аз й река: Я ми дай да пия малко вода от водоноса ти,
\par 44 и тя ми рече: И ти пий, и ще налея и за камилите ти,- тя нека е жената, която Господ е отредил за сина на господаря ми.
\par 45 Докато още говорех в сърцето си, ето, Ревека излезе с водоноса на рамото си. И като слезе на извора и наля, рекох й: Я ми дай да пия.
\par 46 И тя бързо сне водоноса от рамото си и рече: Пий и ще напоя и камилите ти. И тъй, аз пих, а тя напои и камилите.
\par 47 Тогава я попитах, казвайки: Чия си дъщеря? А тя рече: Аз съм дъщеря на Ватуила, Нахоровия син, когото му роди Мелха. Тогава турих обицата на лицето й и гривните на ръцете й.
\par 48 И наведох се, та се поклоних на Господа и благослових Господа, Бога на господаря ми Авраама, Който ме доведе по правия път, за да взема за сина на господаря ми братовата му дъщеря.
\par 49 Сега, прочее, ако искате да постъпите любезно и верно спрямо господаря ми, кажете ми; и, ако не, пак ми кажете, за да се обърна надясно или наляво.
\par 50 А Лаван и Ватуил в отговор рекоха: От Господа стана това; ние не можем да ти речем ни зло, ни добро.
\par 51 Ето Ревека пред тебе; вземи я и си иди; нека бъде жена на сина на господаря ти, както Господ е говорил.
\par 52 Като чу думите им, Авраамовият слуга се поклони на Господа до земята.
\par 53 И слугата извади сребърни и златни накити и облекла, та ги даде на Ревека; така също даде скъпи дарове на брат й и на майка й.
\par 54 Подир това, той и хората, които бяха с него, ядоха, пиха и пренощуваха; и като станаха сутринта, слугата рече: Изпратете ме при господаря ми.
\par 55 А брат й и майка й казаха: Нека поседи момата с нас известно време, най-малко десетина дни, после нека иде.
\par 56 Но той им каза: Не ме спирайте, тъй като Господ е направил пътуването ми успешно; изпратете ме да ида при господаря си.
\par 57 А те рекоха: Да повикаме момата и да я попитаме, какво ще каже.
\par 58 И тъй, повикаха Ревека и рекоха й: Отиваш ли с тоя човек? А тя рече: Отивам.
\par 59 И така, изпратиха сестра си Ревека с бавачката й и Авраамовия слуга с хората му.
\par 60 И благословиха Ревека, като й рекоха: Сестро наша, да се родят от тебе хиляди по десет хиляди и потомството ти да завладее портата на ненавистниците си.
\par 61 И Ревека и слугите й станаха, възседнаха камилите и отидоха след човека. Така, слугата взе Ревека и си замина.
\par 62 А Исаак идеше от Вир-лахайрои, защото живееше в южната земя.
\par 63 И като излезе Исаак, за да размишлява на полето привечер, повдигна очи и видя; и, ето камили се приближаваха.
\par 64 Също и Ревека повдигна очи и видя Исаака; и слезна от камилата.
\par 65 Защото беше рекла на слугата: Кой е оня човек, който иде през полето насреща ни? И слугата беше казал: Това е господарят ми. Затова тя взе покривалото си и се покри.
\par 66 И слугата разказа на Исаака всичко, което бе извършил.
\par 67 Тогава Исаак въведе Ревека в шатъра на майка си Сара и я взе, и тя му стана жена, и той я обикна. Така Исаак се утеши след смъртта на майка си.

\chapter{25}

\par 1 А Авраам взе и друга жена, на име Хетура.
\par 2 Тя му роди Земрана, Иоксана, Мадана, Медиама, Есвока и Шуаха.
\par 3 И Иоксан роди Сава и Дедана; а синове на Дедана бяха Асурим, Латусиим и Лаомим.
\par 4 А синовете на Мадиама бяха Гефа, Ефер, Енох, Авида и Елдага; всички тия бяха потомци на Хетура.
\par 5 Но Авраам даде целия си имот на Исаака.
\par 6 А на синовете на наложниците си Авраам даде подаръци и, докато беше още жив, изпрати ги към изток, в източната земя, далеч от сина си Исаака.
\par 7 А числото на годините на живота на Авраама, колкото живя, беше сто седемдесет и пет години.
\par 8 И Авраам издъхна, като умря в честита старост, стар и сит от дни ; и прибра се при людете си.
\par 9 А синовете му Исаак и Исмаил го погребаха в пещерата Махпелах, в нивата на Ефрона, син на Саара, хетееца, която е срещу Мамврий,
\par 10 нивата, която Авраам купи от хетейците; там беше погребан Авраам, също и жена му Сара.
\par 11 А подир смъртта на Авраама, Бог благослови сина му Исаака; а Исаак живееше при Вир-лахай-рои.
\par 12 Ето потомството на Авраамовия син Исмаил, когото египтянката Агар, Сарината слугиня, роди на Авраама;
\par 13 и ето имената на Исмаиловите синове, имената им според родовете им: Исмаиловият първороден- Наваиот, после Кидар, Адвеил, Мавсам,
\par 14 Масма, Дума, Маса,
\par 15 Адад, Тема, Етур, Нафис и Кедиа.
\par 16 Тия са Исмаиловите синове, тия са имената им според колибите им и според оградените им села: дванадесет племеначалници според племената им.
\par 17 И ето годините на Исмаиловия живот, години сто тридесет и седем; и като издъхна, умря и прибра се при людете си.
\par 18 А потомците му се населиха в земите от Евила до Сур, който е срещу Египет, като се отива към Асирия; Исмаил се засели независим от всичките си братя.
\par 19 Ето и потомството на Авраамовия син Исаак; Авраам роди Исаака,
\par 20 а Исаак беше на четиридесет години, когато взе за жена Ревека, дъщеря на сириеца Ватуил от Падан-арам, и сестра на сириеца Лавана.
\par 21 И молеше се Исаак на Господа за жена си, защото беше бездетна; Господ го послуша и жена му Ревека зачна.
\par 22 А децата се блъскаха едно друго вътре в нея; и тя рече: Ако е така, защо да живея? И отиде да се допита до Господа.
\par 23 А Господ й рече: - Два народа са в утробата ти, И две племена ще се разделят от корема ти; Едното племе ще бъде по-силно от другото племе; И по-големият ще слугува на по-малкия.
\par 24 И когато се изпълни времето й да роди, ето, близнета имаше в утробата й.
\par 25 Първият излезе червен, цял космат, като кожена дреха; и наименуваха го Исав.
\par 26 После излезе брат му, държейки с ръката си петата на Исава; затова се нарече Яков. А Исаак беше на шестдесет години, когато тя ги роди.
\par 27 И като порастнаха децата, Исав стана изкусен ловец, полски човек; а Яков беше тих човек и живееше в шатрите.
\par 28 И Исаак обичаше Исава, защото ядеше от лова му; а Ревека обичаше Якова.
\par 29 Един ден Яков си вареше вариво, а Исав дойде от полето изнемощял.
\par 30 И Исав каза на Якова: Я ми дай да ям от червеното, това червено вариво , защото съм изнемощял; (затова той се нарече Едом).
\par 31 И рече Яков: Най-напред продай ми първородството си.
\par 32 А Исав рече: Виж, аз съм на умиране, за какво ми е това първородство?
\par 33 И Яков рече: Най-напред закълни ми се; и той му се закле, и продаде първородството си на Якова.
\par 34 Тогава Яков даде на Исава хляб и вариво от леща; и той яде и пи, и стана та си отиде. Така Исав презря първородството си.

\chapter{26}

\par 1 И настана глад по земята, освен първия глад, който беше в Авраамовите дни, та Исаак отиде в Герар, при филистимския цар Авимелех,
\par 2 защото Господ беше му се явил и рекъл: Не слизай в Египет; живей в земята, за която ще ти кажа;
\par 3 остани в тая земя и Аз ще бъда с тебе и ще те благословя, защото на тебе и на потомството ти ще дам всички тия земи, в утвърждение на клетвата, с която се заклех на баща ти Авраама;
\par 4 и ще умножа потомството ти като небесните звезди, и ще дам на потомството ти всички тия земи; и в твоето потомство ще се благославят всичките народи на земята;
\par 5 понеже Авраам послуша гласа Ми и опази заръчването Ми, заповедите Ми, повеленията Ми и законите Ми.
\par 6 Затова Исаак се настани в Герар.
\par 7 И местните жители го запитаха за жена му; а той рече: Сестра ми е; защото се боеше да каже: Жена ми е, като си думаше : Да не би местните жители да ме убият поради Ревека; понеже тя беше красива на глед.
\par 8 А след като беше преседял там дълго време, филистимският цар Авимелех, като погледна от прозореца, видя, че Исаак играеше с жена си Ревека.
\par 9 Тогава Авимелех повика Исаака и рече: Ето, тя наистина ти е жена; а защо каза ти: Сестра ми е? Исаак му каза: Защото си рекох да не би да бъда убит поради нея.
\par 10 И рече Авимелех: Що е това, което си ни сторил? Лесно можеше някой от людете да лежи с жена ти и ти щеше да ни навлечеш грях.
\par 11 Затова Авимелех заръча на всичките люде, казвайки: Който докачи тоя човек или жена му, непременно ще се умъртви.
\par 12 И Исаак пося в оная земя и събра през същото лято стократно; и Господ го благослови.
\par 13 Човекът се възвеличаваше и продължаваше да става велик, догдето стана твърде велик.
\par 14 Той придоби овци, придоби и говеда, и много слуги; а филистимците му завиждаха;
\par 15 и филистимците затрупаха и напълниха с пръст всичките кладенци, които бащините му слуги бяха изкопали в дните на баща му Авраама.
\par 16 И Авимелех каза на Исаака: Иди си от нас, защото си станал много по-силен от нас.
\par 17 Затова Исаак си отиде от там, разпъна шатрите си в Герарската долина и там живееше.
\par 18 А Исаак изкопа наново водните кладенци, които бяха изкопани в дните на баща му Авраама,(защото филистимците ги бяха затрупали след Авраамовата смърт) и нарече ги по имената, с които баща му беше ги нарекъл.
\par 19 И Исааковите слуги копаха в долината и намериха там кладенец с текуща вода.
\par 20 Но Герарските говедари се караха с Исааковите говедари, казвайки: Наша е водата. Затова Исаак нарече кладенеца Есен, понеже се караха за него.
\par 21 После изкопаха друг кладенец, но и за него се караха; затова го нарече Ситна.
\par 22 Тогава той се премести оттам и изкопа друг кладенец; и за него не се караха. И нарече го Роовот§, като думаше: Защото сега Господ ни даде пространно място и ние ще се наплодим в тая земя.
\par 23 От там отиде във Вирсавее.
\par 24 И Господ му се яви през същата нощ и рече: Аз съм Бог на баща ти Авраама; не бой се, защото Аз съм с тебе, ще те благословя и ще умножа твоето потомство, заради слугата ми Авраама.
\par 25 И той издигна там олтар призова Господното име; разпъна и шатъра си там, и там Исааковите слуги изкопаха кладенец.
\par 26 Тогава Авимелех отиде при него от Герар с приятеля си Оховата и военачалника си Фихола
\par 27 И Исаак им рече: Защо сте дошли при мене, като ме мразите и ме изпъдихте изпомежду си?
\par 28 А те казаха: Видяхме явно, че Господ е с тебе и си рекохме: Нека се положи клетва между нас, между нас и тебе, и нека направим договор с тебе,
\par 29 че няма да ни сториш зло, както и ние не те докачихме, и както само добро ти правихме и те изпратихме с мир. Сега виждаме, че ти си благословеният от Господа.
\par 30 Тогава Исаак им даде угощение и те ядоха и пиха.
\par 31 На сутринта станаха рано и се заклеха един за друг; после Исаак ги изпрати и те си отидоха от него с мир.
\par 32 И в същия ден Исааковите слуги дойдоха и му известиха за кладенеца, който бяха изкопали и му рекоха: Намерихме вода.
\par 33 И нарече го Савее; от това името на града е Вирсавее до днес.
\par 34 А когато Исав беше на четиридесет години, взе за жена Юдита, дъщеря на хетееца Веири, и Васемата, дъщеря на хетееца Елон.
\par 35 Те бяха горчивина за душата на Исаака и Ревека.

\chapter{27}

\par 1 Когато остаря Исаак и очите му отслабнаха, та не можеше да види, повика по-големия си син Исав и му рече: Синко. А той му рече: Ето ме.
\par 2 Тогава той му каза: Виж сега, аз вече остарях; не зная кой ден ще умра;
\par 3 вземи, прочее, още сега оръжията си, тула си и лъка си, излез на полето и улови ми лов,
\par 4 та ми сготви вкусно ястие, каквото аз обичам и донеси ми да ям, за да те благослови душата ми преди да умра.
\par 5 А Ревека чу, когато говореше Исаак на сина си Исава. И Исав отиде на полето да улови лов и го донесе.
\par 6 Тогава Ревека продума на сина си Якова, казвайки: Виж, аз чух баща ти да говори на брата ти с тия думи:
\par 7 Донеси ми лов и сготви ми вкусно ястие, за да ям и да те благословя пред Господа преди да умра.
\par 8 Сега, прочее, синко, каквото ти поръчам, послушай думите ми.
\par 9 Иди още сега в стадото и донеси ми от там две добри ярета, и аз ще сготвя вкусно ястие за баща ти, каквото той обича;
\par 10 и ти ще го принесеш на баща си да яде, за да те благослови преди да умре.
\par 11 Но Яков рече на майка си Ревека: Виж, брат ми Исав е космат, а аз съм гладък.
\par 12 Може би ще ме попипа баща ми и аз ще се явя пред него като измамник, та ще навлека на себе си проклятие, а не благословение.
\par 13 А майка му рече: Нека твоето проклятие падне върху мене, синко, само послушай думите ми и иди, та ми ги донеси.
\par 14 И тъй, той отиде, взе ги и ги донесе на майка си; и майка му сготви вкусно ястие, каквото баща му обичаше.
\par 15 После Ревека взе по-добрите дрехи на по-стария си син, Исава, които се намираха вкъщи при нея и с тях облече по-младия си син, Якова.
\par 16 И зави ръцете му и гладкото на шията му с ярешките кожи.
\par 17 Тогава даде в ръцете на сина си Якова вкусното ястие и хляба, които бе приготвила.
\par 18 И тъй, той отиде при баща си и рече: Тате. А Исаак рече: Ето ме. Кой си ти, синко?
\par 19 И Яков рече на баща си: Аз съм първородният ти Исав; сторих каквото ми поръча; стани, моля ти се, седни и яж от лова ми, за да ме благослови душата ти.
\par 20 Но Исаак рече на сина си: Как стана, синко, че го намери толкова скоро? А той рече: Защото Господ, твоят Бог, ми даде добър успех.
\par 21 Тогава Исаак рече на Якова: Моля, синко, приближи се, за да те попипам дали си същият ми син, Исав, или не.
\par 22 И приближи се Яков при баща си Исаака; и той, като го попипа, рече: Гласът е глас Яковов, а ръцете са ръце Исавови.
\par 23 И не го позна, защото ръцете му бяха космати, като ръцете на брата му Исава; и благослови го.
\par 24 И каза: Ти ли си същият ми син, Исав? А той рече: Аз.
\par 25 Тогава каза: Принеси ми и ще ям от лова на сина си, за да те благослови душата ми. И принесе му, та яде; донесе му и вино, та пи.
\par 26 А баща му Исаак му рече: Приближи се сега и целуни ме, синко.
\par 27 И той се приближи, та го целуна; и Исаак помириса дъха на облеклото му и го благослови, казвайки: - Ето, дъхът на сина ми е като дъх на поле, което е благословил Господ.
\par 28 Бог да ти даде от росата на небето и от И от тлъстината на земята, И изобилие на жито и на вино.
\par 29 Племена да ти слугуват И народи да ти се покланят; Бъди господар на братята си, И да ти се покланят синовете на майка ти. Проклет всеки, който те кълне, И благословен всеки, който те благославя.
\par 30 Щом като свърши Исаак да благославя Якова и Якова току що се беше разделил с баща си Исаака, дойде брат му Исав от лова си.
\par 31 Също и той сготви вкусно ястие, принесе на баща си и му рече: Да стане баща ми и да яде от лова на сина си, за да ме благослови душата ти,
\par 32 А баща му Исаак му рече: Кой си ти? И той каза: Аз съм син ти, първородният ти, Исав.
\par 33 Тогава Исаак се разтрепера твърде много и каза: А кой ще е оня, който улови лов и ми принесе, та ядох от всичко преди да дойдеш ти, и го благослових?- Да! И благословен ще бъде.
\par 34 Когато чу Исав бащините си думи, извика със силен и жалостен вик, и рече на баща си: Благослови ме, ей мене, тате.
\par 35 А той рече: Брат ти дойде с измама и взе твоето благословение.
\par 36 И рече Исав : Право са го нарекли Яков, защото сега втори път той ме е изместил: отне първородството ми и, ето, сега е отнел и благословението ми. И рече: Не си ли задържал за мене благословение?
\par 37 А Исаак в отговор рече на Исава: Ето, поставих го господар над тебе, направих всичките му братя негови слуги и надарих го с жито и вино; какво, прочее, да сторя за тебе, синко?
\par 38 И Исав рече на баща си: Само едно ли благословение имаш, тате? Благослови ме, ей мене, тате. И Исав плака с висок глас.
\par 39 Тогава баща му Исаак в отговор му каза: Ето, твоето желание ще бъде в тлъстите места на земята, Наросявани от небето горе;
\par 40 С ножа си ще живееш, а на брата си ще слугуваш; Но когато въстанеш Ще строшиш ярема му от врата си.
\par 41 И Исав мразеше Якова поради благословението, с което баща му го благослови. И думаше Исав в сърцето си: Скоро ще настанат дните, когато ще оплакваме баща си, тогава ще убия брата си Якова.
\par 42 И казаха на Ревека думите на по-стария й син, Исав; затова тя прати да повикат по-младия й син,Якова, и му рече: Виж, брат ти Исав се утешава относно тебе, че ще те убие.
\par 43 Сега, прочее, синко, послушай думите ми: стани, та бягай при брата му Лавана в Харан,
\par 44 и поживей при него известно време, догде премине яростта на брата ти,
\par 45 догде премине от тебе гнева на брата ти и той забрави това, което си му сторил; тогава ще пратя да те доведат от там. Защо да се лиша и от двама ви в един ден?
\par 46 И Ревека каза на Исаака: Омръзна ми живота поради хетейските дъщери. Ако Яков вземе жена от хетейските дъщери, каквито са тия от дъщерите на тая земя, защо живея?

\chapter{28}

\par 1 Тогава Исаак повика Якова и като го благослови, поръча му, казвайки: Да не вземеш жена от ханаанските дъщери.
\par 2 Стани, иди в Падан-арам, в дома на майчиния ти баща Ватуил, и от там вземи жена, от дъщерите на вуйчо ти Лавана.
\par 3 И Бог Всемогъщи да те благослови и да те наплоди и умножи, така щото да произлязат от тебе редица племена;
\par 4 и даденото на Авраама благословение да го даде на тебе и на потомството ти с тебе, за да наследиш земята, в която си пришелец, която Бог даде на Авраама.
\par 5 Така Исаак изпрати Якова; и той отиде в Падан-арам при Лавана, син на сириеца Ватуила, и брат на Ревека, майка на Якова и Исава.
\par 6 А Исав, като видя, че Исаак благослови Якова и го изпрати в Падан-арам, да си вземе жена от там, и, че, като го благослови, поръча му, казвайки: Да не вземеш жена от Ханаанските дъщери;
\par 7 че Яков послуша баща си и майка си, та отиде в Падан-арам;
\par 8 и като видя Исав, че ханаанските дъщери не се нравеха на баща му Исаака,
\par 9 то Исав отиде при Исмаила и, освен другите си жени, взе за жена и Маелета, дъщеря на Авраамовия син Исмаил, сестра на Новаита.
\par 10 Яков, прочее, излезе от Вирсавее и отиде към Харан.
\par 11 като стигна на едно място, пренощува там, защото слънцето беше залязло; и взе от мястото един камък, та го тури за възглавница, и легна да спи на това място.
\par 12 И сънува, и ето стълба изправена на земята, чийто връх стигаше до небето; и Божиите ангели се качваха и слизаха по нея.
\par 13 И Господ стоеше над нея, и каза: Аз съм Господ Бог на баща ти Авраама и Бог на Исаака; земята, на която лежиш, ще дам на тебе и на потомството ти.
\par 14 Твоето потомство ще бъде многочислено , като земния пясък; ти ще се разшириш към запад и към изток, към север и към юг; и чрез тебе и чрез твоето потомство ще се благословят всички племена на земята.
\par 15 Ето, Аз съм с тебе и ще те пазя, където и да идеш, и ще те върна пак в тая земя; защото няма да те оставя, докле не извърша това, за което ти говорих.
\par 16 А като се събуди Яков от съня си, рече: Наистина Господ е на това място, а аз не съм знаел.
\par 17 И убоя се и рече: Колко е страшно това място! Това не е друго, освен Божий дом, това е врата небесна.
\par 18 На сутринта, като стана Яков рано, взе камъка, който си беше турил за възглавница, изправи го за стълб и изля масло на върха му.
\par 19 И наименува онова място Ветил ; а преди името на града беше Луз.
\par 20 Тогава Яков се обрече и каза: Ако бъде Бог с мене и ме опази в това пътуване, по което отивам, и ми даде хляб да ям и дрехи да се облека,
\par 21 така щото да се завърна с мир в бащиния си дом, тогава Господ ще бъде мой Бог,
\par 22 и тоя камък, който изправих за стълб, ще бъде Божий дом; и от всичко, що ми дадеш, ще дам десетък на Тебе.

\chapter{29}

\par 1 Тогава Яков тръгна и отиде в земята на източните жители.
\par 2 И погледна, и, ето, кладенец на полето, и там три стада овци, които почиваха при него, защото от оня кладенец напояваха стадата; и върху отвора на кладенеца имаше голям камък.
\par 3 И когато се събираха там всичките стада, отваляха камъка от отвора на кладенеца, та напояваха стадата; после пак туряха камъка на мястото му над отвора на кладенеца.
\par 4 И Яков каза на овчарите : Братя, от где сте? А те рекоха: От Харан сме.
\par 5 И рече им: Познавате ли Лавана, Нахоровия син? Отговориха: Познаваме.
\par 6 И рече им: Здрав ли е? А те рекоха: Здрав е; и ето дъщеря му Рахил иде с те.
\par 7 А той каза: Вижте, още е много рано, не е време да се прибира добитъкът; напойте те и идете да ги пасете.
\par 8 А те рекоха: Не можем догде не се съберат всичките стада и не отвалят камъка от отвора на кладенеца; тогава напояваме овците.
\par 9 Докато им говореше още, дойде Рахил с бащините си , защото тя ги пасеше.
\par 10 А като видя Яков Рахил, дъщерята на вуйка си Лавана, и овците на вуйка си Лавана, Яков се приближи, та отвали камъка от отвора на кладенеца; и напои стадото на вуйка си Лавана.
\par 11 И Яков целуна Рахил и заплака с висок глас.
\par 12 И Яков каза на Рахил, че е брат на баща й и, че е син на Ревека; а тя се завтече, та извести на баща си.
\par 13 А Лаван, като чу за своя сестриник Яков, завтече се да го посрещне; и прегърна го, целуна го и го заведе у дома си. Тогава Яков разказа на Лавана всичко.
\par 14 И Лаван му рече: Наистина, ти си моя кост и моя плът. И Яков живя при него един месец.
\par 15 След това Лаван рече на Якова: Нима, като си ми брат, ти ще ми работиш безплатно? Кажи ми, каква да ти бъде заплатата?
\par 16 А Лаван имаше две дъщери: името на по-старата беше Лия, а името на по-младата - Рахил.
\par 17 На Лия очите не бяха здрави; а Рахил имаше хубава снага и хубаво лице.
\par 18 И Яков, понеже обикна Рахил, рече: Ще ти работя седем години за по-малката ти дъщеря, Рахил.
\par 19 И рече Лаван: По-добре да я дам на тебе, отколкото да я дам на друг мъж; живей при мене.
\par 20 И тъй, Яков работи за Рахил седем години; но, поради любовта му към нея, те му се видяха като няколко дни.
\par 21 След това Яков каза на Лавана: Дай жена ми, защото дойде вече време да вляза при нея.
\par 22 И тъй, Лаван събра всичките хора от това място и даде угощение.
\par 23 А вечерта взе дъщеря си Лия, та му я доведе; и той влезе при нея.
\par 24 И Лаван даде слугинята си Зелфа за слугиня на дъщеря си Лия.
\par 25 Но на сутринта, ето, че беше Лия. И Яков рече на Лавана: Що е това, което ми стори ти? Нали за Рахил ти работих? Тогава защо ме излъга?
\par 26 А Лаван каза: В нашето място няма обичай да се дава по-младата преди по-старата.
\par 27 Свърши сватбарската седмица с тая; и ще ти дам и оная за работата,която ще ми вършиш още седем години.
\par 28 И Яков стори така; свърши седмицата с Лия и тогава Лаван му даде дъщеря си Рахил за жена.
\par 29 И Лаван даде слугинята си Вала за слугиня на дъщеря си Рахил.
\par 30 И Яков влезе при Рахил, и обикна Рахил повече от Лия; и работи на Лаван още седем години.
\par 31 А Господ, понеже видя, че Лия не беше обичана, отвори утробата й; а Рахил беше бездетна.
\par 32 И тъй, Лия зачна и роди син, и наименува го Рувим, защото си думаше: Господ погледна на неволята ми; сега мъжът ми ще обикне.
\par 33 И пак зачна и роди син; и рече: Понеже чу Господ, че не съм любима, затова ми даде и тоя син; и наименува го Симеон.
\par 34 Пак зачна и роди син; и рече: Сега вече мъжът ми ще се привърже към мене, защото му родих три сина; затова го наименува Левий§.
\par 35 И пак зачна и роди син; и рече: Тоя път ще възхваля Господа; затова го наименува Юда. И престана да ражда.

\chapter{30}

\par 1 А като видя Рахил, че не раждаше деца на Якова, Рахил завидя на сестра си и рече на Якова: Дай ми чада, иначе аз ще умра.
\par 2 А Яков се разгневи на Рахил и рече: Нима съм аз, а не Бог, Който е лишил утробата ти от плод?
\par 3 А тя рече: Ето слугинята ми Вала; влез при нея, и тя да роди на коленете ми, та да придобия и аз деца чрез нея.
\par 4 И тъй, тя му даде слугинята си Вала за жена; и Яков влезе при нея.
\par 5 И Вала зачна и роди син на Якова.
\par 6 Тогава рече Рахил: Бог отсъди за мене, послуша и гласа ми, та ми даде син; затова го наименува Дан.
\par 7 И Вала, слугинята на Рахил, пак зачна и роди втори син на Якова.
\par 8 Тогава рече Рахил: Силна борба водих със сестра си и надвих; затова го наименува Нефтали짧.
\par 9 Когато видя Лия, че престана да ражда, взе слугинята си Зелфа, та я даде на Якова за жена.
\par 10 И Зелфа, слугинята на Лия, роди син на Якова.
\par 11 И рече Лия: Щастие дойде; затова го наименува Гад.
\par 12 И Зелфа, слугинята на Лия, роди втори син на Якова.
\par 13 Тогава рече Лия: Честита съм; защото честита ще ме нарекат жените; затова го наименува Асир§.
\par 14 И през времето на пшеничената жетва, Рувим, като излезе, намери мандрагорови ябълки на полето и ги донесе на майка си Лия; и Рахил рече на Лия: Я ми дай от мандрагоровите ябълки на сина ти.
\par 15 А тя й рече: Малко ли ти е дето си отнела мъжа ми, та искаш да отнемеш мандрагоровите ябълки на сина ми? Тогава Рахил й рече: Като е тъй, нека лежи с тебе тая нощ за мандрагоровите ябълки на сина ти.
\par 16 И така, когато дойде Яков вечерта от полето, Лия излезе да го посрещне и рече: При мене да влезеш, защото наистина те откупих с мандрагоровите ябълки на сина си. И лежа с нея оная нощ.
\par 17 И Бог послуша Лия; и тя зачна, и роди пети син на Якова.
\par 18 Тогава рече Лия:Бог ми даде награда за гдето дадох слугинята си на мъжа си, затова го наименува Исахар.
\par 19 И Лия пак зачна, и роди шести син на Якова.
\par 20 И рече Лия: Бог ми даде добър дар, сега мъжът ми ще живее с мене, защото му родих шест сина; затова го наименува Завулон.
\par 21 И после роди дъщеря, която наименува Дина.
\par 22 След това Бог си спомни за Рахил; Бог я послуша и отвори утробата й.
\par 23 Тя зачна и роди син, и рече: Бог отне от мене позора.
\par 24 И наименува го Иосиф§§, като думаше: Господ да ми прибави и друг син.
\par 25 А като роди Рахил Иосифа, Яков рече на Лавана: Пусни ме да си отида в моето място и в отечеството си.
\par 26 Дай ми жените и децата ми, за които съм ти работил, за да си отида; защото ти знаеш работата, която ти свърших.
\par 27 А Лаван му рече: Ако съм придобил твоето благоволение, остани, защото разбрах, че Господ ме е благословил заради тебе.
\par 28 Рече още: Кажи ми каква заплата искаш и ще ти я дам.
\par 29 А Яков му каза: Ти знаеш как съм ти работил и как е бил добитъкът ти при мене.
\par 30 Защото това, което имаше ти преди моето идване, беше малко; а сега нарасна и стана много. С моето идване Господ те благослови. Но сега, кога ще промисля и за своя си дом?
\par 31 А той рече: Какво да ти дам? И Яков каза: Не ми давай нищо; ако направиш каквото ти кажа, аз пак ще пазя и стадото ти;
\par 32 ще премина днес през цялото ти стадо и ще отлъча от него всяка капчеста и пъстра, и всяка черникава между овците, и всяка пъстра и капчеста между козите; те ще ми бъдат заплатата.
\par 33 И занапред, когато дойдеш да прегледаш заплатата ми, моята правота ще засвидетелствува за мене - всяка коза, която не е капчеста и пъстра, и всяка овца, която не е черна, тя, ако се намери у мене, ще се счита за крадена.
\par 34 И рече Лаван: Нека да бъде според както си казал.
\par 35 И тъй, в същия ден, Лаван отлъчи нашарените с линии и пъстрите козли, всичките капчести и пъстри кози, всичките, на които имаше бяло и всичките черни между овците, та ги предаде в ръцете на Якововите синове
\par 36 и постави тридневен път между себе си и Якова; а Яков пасеше останалите от Лавановите стада.
\par 37 Тогава Яков взе зелени пръчки от топола, от леска и от явор, и изряза по тях бели ивици, така щото да се вижда бялото по пръчките.
\par 38 Тия пръчки, по които беше изрязал белите ивици, тури пред стадата в коритата, в поилата, дето дохождаха стадата да пият; и, като зачеваха, когато дохождаха да пият,
\par 39 то стадата зачеваха пред пръчките; и стадата раждаха нашарени с линии, капчести и пъстри.
\par 40 И Яков отлъчваше агнетата, и обръщаше лицата на овците към нашарените и към всичките черни от Лавановото стадо; а своите стада тури отделно и не ги тури с Лавановите овци.
\par 41 И когато по-силните овци зачеваха, Яков туряше пръчките в коритата пред очите на стадото, за да зачеват между пръчките.
\par 42 А когато те бяха по-слаби, не ги туряше; така че се падаха по-слабите на Лавана, а по-силните на Якова.
\par 43 Така човекът забогатя твърде много и придоби големи стада, слугини и слуги, камили и осли.

\chapter{31}

\par 1 А Яков чу думите на Лавановите синове, които казваха: Яков отне целия имот на баща ни и от бащиния ни имот придоби цялото това богатство.
\par 2 И ето, Яков видя, че лицето на Лавана не беше към него тъй както преди.
\par 3 А Господ рече на Якова: Върни се в отечеството си и в рода си, и Аз ще бъда с тебе.
\par 4 Тогава Яков прати да повикат Рахил и Лия на полето при стадото му;
\par 5 и рече им: Виждам, че бащиното ви лице не е към мене тъй както преди; но Бог на баща ми е бил с мене.
\par 6 А вие знаете, че с цялата си сила работих на баща ви.
\par 7 Но баща ви ме излъга и десет пъти променя заплатата ми; обаче Бог не го остави да ми напакости.
\par 8 Ако кажеше така: Капчестите ще ти бъдат заплатата; тогава цялото стадо раждаше капчести; а ако кажеше така: Шарените ще ти бъдат заплатата; тогава цялото стадо раждаше шарени.
\par 9 Така Бог отне стадото на баща ви и го даде на мене.
\par 10 И едно време, когато зачеваше стадото, подигнах очи и видях насъне, че, ето, козлите, които се качваха на стадото бяха шарени, капчести и сиви.
\par 11 И ангел Божий ми рече в съня: Якове. И аз отговорих: Ето ме.
\par 12 И той каза: Подигни сега очи и виж, че всичките козли, които се качват на стадото, са шарени, капчести и сиви; защото видях всичко, що ти прави Лаван.
\par 13 Аз Съм Бог на Ветил, гдето ти помаза стълб с масло и гдето Ми се обрече. Стани сега, излез из тая земя и се върни в родината си.
\par 14 Рахил и Лия в отговор му рекоха: Имаме ли ние още дял или наследство в бащиния си дом?
\par 15 Не счете ли ни той като чужденки, защото ни продаде и даже изяде дадените за нас пари?
\par 16 Защото цялото богатство, което Бог отне от баща ни, е наше и на нашите чада. Затова, стори сега, каквото Бог ти е казал.
\par 17 Тогава Яков стана и тури децата си и жените на камилите.
\par 18 И подкара всичкия си добитък и целия си имот, що бе придобил, спечеления от него добитък, който бе събрал в Падан-арам, за да отиде в Ханаанската земя при баща си Исаак.
\par 19 А като беше отишъл Лаван да стриже овците си, Рахил открадна домашните идоли на баща си.
\par 20 И тъй, Яков побягна скришно от сириеца Лавана, без да му извести, че си отива.
\par 21 Побягна с целия си имот, стана та премина Ефрат и отправи се към Галаадската поляна.
\par 22 А на третия ден известиха на Лавана, че Яков побягнал.
\par 23 Тогава Лаван , като взе със себе си братята си, гони го седем дни и го стигна на Галаадската поляна.
\par 24 Но Бог дойде насъне, през нощта, при сириеца Лавана и му каза: Внимавай да не речеш на Якова ни зло, ни добро.
\par 25 И така, Лаван стигна Якова. Яков беше разпънал шатъра си на бърдото, а Лаван с братята си разпъна своята на Галаадската поляна.
\par 26 И Лаван рече на Якова: Що стори ти? Защо побягна скришно и отведе дъщерите ми, като с нож запленени?
\par 27 Защо се скри, за да бягаш; и ме измами, а не ми яви, та да можех да те изпратя с веселие и с песни, с тъпани и с арфи,
\par 28 нито ме остави да целуна синовете и дъщерите си? Ти си сторил това без да мислиш.
\par 29 Ръката ми е доволно силна да ви напакости, но Бог на баща ви ми говори нощес, казвайки: Внимавай да не речеш на Якова ни зло, ни добро.
\par 30 И сега вече си тръгнал, понеже ти е много домъчняло за Бащиния ти дом; обаче , защо си откраднал боговете ми?
\par 31 А Яков в отговор каза на Лавана: Побягнах , понеже се уплаших; защото си рекох: Да не би да ми отнемеш на сила дъщерите си.
\par 32 У когото намериш боговете си, той да не остане жив; пред братята ни прегледай, какво твое има у мене и си го вземи (защото Яков не знаеше, че Рахил ги бе откраднала).
\par 33 И тъй, Лаван влезе в Якововата шатра, в Лиината шатра и в шатрите на двете слугини, но не намери боговете . Тогава, като излезе от Лиината шатра, влезе в Рахилината шатра.
\par 34 А Рахил беше взела домашните идоли, турила ги в седлото на камилата и седеше на тях. А Лаван пипаше из цялата шатра, но не ги намери.
\par 35 И Рахил рече на баща си: Да не ти се зловиди, господарю, дето не мога да стана пред тебе, понеже имам обикновеното на жените. И там той търси, но не намери идолите.
\par 36 Тогава Яков се разсърди и скара се с Лавана. Яков проговори и каза на Лавана: Какво е престъплението ми? Какъв е грехът ми, та си се затекъл подир мене толкоз разпалено?
\par 37 Като претърси всичките ми вещи, какво намери от цялата си покъщнина? Сложи го тук пред моите братя и твоите братя, и нека отсъдят между двама ни.
\par 38 Двадесет години вече съм бил при тебе; овците ти и козите ти не се изяловиха; и овните на стадото ти не изядох.
\par 39 Разкъсано от звяр не ти донесох; аз теглех загубата. От мене ти изискваше откраднотото , било че се открадне деня или нощя.
\par 40 Ето как беше с мене: деня пекът ме изнуряваше, а нощя - мразът, и сънят бягаше от очите ми.
\par 41 Двадесет години вече съм бил в дома ти; четиринадесет години ти работих за двете ти дъщери и шест години за овците ти; и ти десет пъти променя заплатата ми.
\par 42 Ако не беше с мене бащиният ми Бог, Бог на Авраама, Страхът на Исаака, ти без друго би ме изпратил сега без нищо. Бог видя моята неволя и труда на ръцете ми, и те изобличи нощес.
\par 43 А Лаван в отговор рече на Якова: Тия дъщери са мои дъщери и децата са мои деца, и стадата са мои стада, всичко що виждаш е мое; и какво да сторя днес на тия мои дъщери или на децата, които са народили?
\par 44 Но сега ела, аз и ти да направим договор, който да бъде свидетелство между мене и тебе.
\par 45 Тогава Яков взе камък и го изправи за стълб.
\par 46 Още Яков рече на братята си: Натрупайте камъни; и те взеха камъни та направиха грамада; и ядоха там край грамадата.
\par 47 Лаван я нарече Иегар Сахадута, а Яков я нарече Галаад.
\par 48 И рече Лаван: Тая грамада е свидетел днес между мене и тебе. Поради това тя се наименува Галаад
\par 49 и Масфа§, защото думаше: Господ да бди между мене и тебе, когато сме далеч един от друг.
\par 50 Ако се обхождаш зле с дъщерите ми или ако вземеш други жени, освен дъщерите ми, няма никой човек с нас за свидетел ; но виж, Бог е свидетел между мене и тебе.
\par 51 Лаван още каза на Якова: Гледай тая грамада и гледай стълба, който изправих между мене и тебе,
\par 52 тая грамада да бъде свидетел и стълбът да бъде свидетел, че аз няма да премина тая грамада към тебе, нито ти ще преминеш тая грамада и тоя стълб към мене, за зло.
\par 53 Бог Авраамов, Бог Нахоров, бащиният им Бог, нека съди между нас. И Яков се закле в Страха на баща си Исаака.
\par 54 Тогава Яков принесе жертва на поляната и повика братята си да ядат хляб; и ядоха хляб, и пренощуваха на поляната.
\par 55 И на утринта Лаван стана рано, целуна синовете си и дъщерите си, благослови ги; и Лаван тръгна, та се върна в своето място.

\chapter{32}

\par 1 Тогава Яков отиде по пътя си и ангели Божии го срещнаха.
\par 2 А като ги видя, Яков рече: Това е Божие войнство; и наименува мястото Маханаим.
\par 3 И Яков изпрати пред себе си вестители до брата си Исава в Сирийската земя, на местността Едом;
\par 4 и заръча им, казвайки: Така да речете на господаря ми Исава: Слугата ти Яков тъй говори: Бях пришелец при Лавана и бавих се до сега;
\par 5 придобих говеда, осли и , слуги и слугини; и изпратих да известят на господаря ми, за да придобия благоволението ти.
\par 6 А вестителите се върнаха при Якова и казаха: Ходихме при брата ти Исава; а и той иде да те посрещне, и четиристотин мъже с него.
\par 7 А Яков, като се уплаши много и се смути, раздели людете, които бяха с него, и стадата, говедата и камилите, на две дружини, казвайки:
\par 8 Ако налети Исав на едната дружина и я удари, останалата дружина ще се избави.
\par 9 Тогава Яков каза: Боже на баща ми Авраам и Боже на баща ми Исаака, Господи, Който си ми рекъл: Върни се в отечеството си и при рода си, и Аз ще ти сторя добро,
\par 10 не съм достоен за най-малката от всичките милости и от всичката вярност, които си показал на слугата си; защото едвам с тоягата си преминах тоя Иордан, а сега станах и два стана.
\par 11 Избави ме, моля Ти се, от ръката на брата ми, от ръката на Исава; защото се боя от него, да не би като дойде, да порази и мен, и майка с чада.
\par 12 А Ти си казал: Наистина ще ти сторя добро и ще направя потомството ти като морския пясък, който поради множеството си не може да се изброи.
\par 13 Като пренощува там оная нощ, взе от онова, що му дойде под ръка, за подарък на брата си Исава:
\par 14 двеста кози и двадесет козли, двеста овци и двадесет овни,
\par 15 тридесет дойни камили с малките им, четиридесет крави и десет юнци, двадесет ослици и десет жречбета,
\par 16 и предаде всяко стадо отделно, в ръцете на слугите си. И рече на слугите си: Минете пред мене и оставете разстояние между едно стадо и друго.
\par 17 На първия заръча, като каза: Когато те срещне брат ми Исав и те попита, казвайки: Чий си? Къде отиваш? Чии са тия пред тебе?
\par 18 Тогава ще кажеш: Те са на слугата ти Якова; подарък е, който изпраща на господаря ми Исава; и той иде подир нас.
\par 19 Така заръча и на втория, на третия и на всичките, които вървяха подир стадата, като казваше: По тоя начин ще говорите на Исава, когато го срещнете;
\par 20 и ще речете: Ето, слугата ти Яков иде подир нас. Защото си думаше: Ще го умилостивя с подаръка, който върви пред мене и после ще видя лицето му; може би ще ме приеме благосклонно .
\par 21 И тъй, подаръкът мина пред него, но той остана през оная нощ в стана.
\par 22 А като стана през нощта, взе двете си жени, двете си слугини и единадесетте си деца, и премина брода на Яков.
\par 23 Взе ги и ги прекара през потока, прекара и всичко, що имаше.
\par 24 А Яков остана сам. И един човек се бореше с него до зазоряване,
\par 25 който, като видя, че не му надви, допря се до ставата на бедрото му; и ставата на Якововото бедро се измести, като се бореше с него.
\par 26 Тогава човекът рече: Пусни ме да си отида, защото се зазори. А Яков каза: Няма да те пусна да си отидеш, догде не ме благословиш
\par 27 А той му каза: Как ти е името? Отговори: Яков.
\par 28 А той рече: Няма да се именуваш вече Яков, но Израил, защото си бил в борба с Бога и с човеци и си надвил.
\par 29 А Яков го попита, като рече: Кажи ми, моля, твоето име. А той рече: Защо питаш за моето име? И благоволи го там.
\par 30 И Яков наименува мястото Фануил, защото, си казваше : Видях Бога лице с лице и животът ми биде опазен.
\par 31 Слънцето го огря, като заминаваше Фануил; и куцаше с бедрото си.
\par 32 Затова и до днес израилтяните не ядат сухата жила, която е върху ставата на бедрото; защото човекът се допря до ставата на Якововото бедро при сухата жила.

\chapter{33}

\par 1 След това, като подигна Яков очи, видя, че идеше Исав и с него четиристотин мъже; и раздели децата си на Лия и Рахил, и на двете слугини.
\par 2 Слугините и децата им тури напред, Лия и децата й подир тях, а Рахил и Иосифа най-назад.
\par 3 А сам той замина пред тях и поклони се до земята седем пъти, доде да стигне при брата си.
\par 4 И Исав се затече да го посрещне, прегърна го, падна на врата му и го целуна; и те заплакаха.
\par 5 И като подигна очи и видя жените и децата, рече: Какви са тия? И той рече: Те са децата, които Бог подари на слугата ти.
\par 6 Тогава пристъпиха слугините и децата им и се поклониха.
\par 7 Така пристъпиха и Лия и децата й и се поклониха; а после пристъпиха Иосиф и Рахил и се поклониха.
\par 8 Тогава рече Исав : За какво ти е тая цяла дружина, която срещнах? А той каза: За да придобия благоволението на господаря си.
\par 9 А Исав рече: Имам доволно, брате мой; ти задръж своите си.
\par 10 Но Яков отвърна: Не, моля ти се, ако съм придобил твоето благоволение, приеми подаръка ми от ръцете ми, тъй като видях лицето ти, като че видях Божие лице, понеже ти беше благосклонен към мене.
\par 11 Приеми, моля, подаръка ми, който ти е принесен; защото Бог е постъпил благо към мене, та имам всичко. И като настояваше, той го прие.
\par 12 Тогава рече Исав : Да тръгнем и да вървим, и аз ще вървя пред тебе.
\par 13 Но Яков му каза: Господарят ми знае, че децата ми са нежни и, че имам със себе си дойни овци и говеда; и ако ги пресилят само един ден, цялото стадо ще измре.
\par 14 Господарят ми нека замине, моля, пред слугата си; и аз ще карам полека според вървежа на добитъка, който е пред мене и според вървежа на децата, доде стигна при господаря си в Сиир.
\par 15 А Исав рече: Поне да оставя с тебе неколцина от хората, които са с мене. Но той каза: Каква нужда? Стига да придобия благоволението на господаря си.
\par 16 И тъй, в същия ден, Исав се върна по пътя си за Сиир.
\par 17 А Яков пътуваше в Сокхот, дето си построи къща и направи кошари за добитъка си; затова мястото се наименува Сокхот.
\par 18 И като се върна от Падан-арам, Яков дойде в Сихемовия град Салим, който е в Ханаанската земя и разположи се пред града.
\par 19 И от синовете на Емора, Сихемовия баща, купи за сто сребърника нивата, гдето разпъна шатрата си.
\par 20 Там издигна олтар; и го наименува Ел-елое-Израил.

\chapter{34}

\par 1 А Дина, дъщеря на Лия, която тя беше родила на Якова, излезе да види дъщерите на онази земя.
\par 2 А Сихем, син на евееца Емор, местния владетел, като я видя, взе я, лежа с нея и я изнасили.
\par 3 И душата му се привърза за Якововата дъщеря Дина; и обикна момата и говори на момата по сърцето й.
\par 4 И Сихем говори на баща си Емора, казвайки: Вземи ми тая мома за жена.
\par 5 А Яков чу, че той осквернил дъщеря му Дина; но понеже синовете му бяха с добитъка му на полето, Яков си замълча до завръщането им.
\par 6 Тогава Емор, Сихемовият баща, отиде при Якова, за да се разговори с него.
\par 7 А като чуха за станалото , Якововите синове дойдоха от полето; и тия мъже се наскърбиха и много се разгневиха, загдето той сторил безчестие на Израиля, като изнасилил Якововата дъщеря, - нещо, което не трябваше да стане.
\par 8 И така, Емор се разговаряше с тях, казвайки: Душата на сина ми Сихема се е привързала към дъщеря ви; дайте му я, моля, за жена.
\par 9 И сродете се с нас; давайте на нас вашите дъщери и взимайте за вас нашите дъщери.
\par 10 Живейте с нас и земята е пред вас; настанете се и търгувайте в нея, и придобивайте владения в нея.
\par 11 Тоже и Сихем рече на баща й и на братята й: Само да придобия вашето благоволение, каквото ми кажете ще дам.
\par 12 Искайте от мене вино и дарове, колкото желаете, ще дам, според както ми речете; само ми дайте момата за жена.
\par 13 А Якововите синове отговориха на Сихема и на баща му Емора с лукавство, понеже той беше осквернил сестра им Дина; и говориха им, казвайки:
\par 14 Не можем да сторим това, да дадем сестра си на необрязан човек; защото това би било укор за нас.
\par 15 Само при това условие ще се съгласим с вас: ако станете вие като нас, като обрязвате всеки от мъжки пол между вас,
\par 16 тогава ще даваме нашите дъщери вам и ще вземаме вашите дъщери за нас, ще живеем с вас и ще станем един народ.
\par 17 Но, ако не приемете условието ни да се обрежете, тогава ще вземем дъщеря си и ще си отидем.
\par 18 И това, което казаха, бе угодно на Емора и на Еморовия син Сихема.
\par 19 Момъкът не се забави да направи това, защото много обичаше Якововата дъщеря; и в целия му бащин дом той беше най-почтен.
\par 20 Тогава Емор и сина му Сихем дойдоха при портата на града си и говориха на градските мъже, казвайки:
\par 21 Тия човеци са миролюбиви към нас; затова, нека живеят в земята и търгуват в нея; защото вижте, земята е доволно пространна и за тях. Нека вземаме дъщерите им за жени и да им даваме нашите дъщери.
\par 22 Тия човеци се съгласиха да живеят с нас и да бъдем един народ, само с това условие, щото всеки от мъжки пол между нас да се обреже, както те се обрязват.
\par 23 Добитъкът им, имотът им и всичките им животни не ще ли станат наши? Само да се съгласим с тях и те ще живеят с нас.
\par 24 Тогава всичките, които излизаха от портата на града му, послушаха Емора и сина му Сихем; всеки от мъжки пол се обряза, всичките, които излизаха от портата на града му.
\par 25 А на третия ден, когато бяха в болките си, двама от Якововите синове, Симеон и Левий, братя на Дина, взеха всеки ножа си, нападнаха дързостно града и избиха всички от мъжки пол.
\par 26 Убиха с острото на ножа Емора и сина му Сихем; а Дина взеха из дома на Сихема и си излязоха.
\par 27 Тогава Якововите синове се спуснаха върху убитите и ограбиха града, понеже бяха осквернили сестра им.
\par 28 Забраха овците им, говедата им, ослите им, каквото имаше в града и по полето, и всичкото им богатство;
\par 29 и откараха в плен всичките им деца и жени, и разграбиха всичко що имаше в къщите.
\par 30 Но Яков рече на Симеона и Левия: Вие ме смутихте, понеже ме направихте да съм омразен между жителите на тая земя, между ханаанците и ферезейците; и понеже аз имам малко хора, те ще се съберат против мене и ще ме поразят, та ще погина аз и домът ми.
\par 31 А те казаха: Трябваше ли той да постъпи със сестра ни, като с блудница?

\chapter{35}

\par 1 След това Бог каза на Якова: Стани, иди на Ветил и живей там; и там издигни олтар на Бога, Който ти се яви, когато бягаше от лицето на брата си Исава.
\par 2 Тогава Яков каза на домочадието си и на всичките, които бяха с него: Махнете чуждите богове, които са между вас, очистете се и променете дрехите си;
\par 3 и да станем да отидем във Ветил, и там да издигна олтар на Бога, Който ме послуша в деня на бедствието ми и беше с мене в пътя, по който ходех.
\par 4 И тъй, те дадоха на Якова всичките чужди богове, що бяха в ръцете им и обиците, които бяха на ушите им; и Яков ги скри под дъба, който бе при Сихем.
\par 5 След това си тръгнаха; и страх Божий беше върху градовете, които бяха наоколо им, така че не преследваха Якововите синове.
\par 6 И тъй, Яков дойде в Луз, ( който е Ветил ), в Ханаанската земя, той и всичките люде, които бяха с него.
\par 7 И там издигна олтар и наименува мястото Ел-Ветил, защото когато бягаше от лицето на брата си, там му се яви Бог.
\par 8 По това време умря Девора, Ревекината бавачка, и я погребаха под дъба, по-долу от Ветил; затова се наименува Дъба на Плача .
\par 9 И Бог пак се яви на Якова, след завръщането му от Падан-арам, и го благослови.
\par 10 Каза му Бог: Името ти наистина е Яков; но не ще се именуваш вече Яков, но Израил ще ти бъде името. И наименува го Израил.
\par 11 Бог му рече още: Аз Съм Бог Всемогъщий; плоди се и размножавай се. Народ, даже редица народи ще произлязат от тебе, и царе ще излязат от чреслата ти;
\par 12 и земята, която дадох на Авраама и на Исаака, на тебе ще я дам; и на потомството ти след тебе ще дам земята.
\par 13 Тогава Бог се възнесе от него, от мястото, гдето му говори.
\par 14 И Яков издигна стълб на мястото, гдето му говори, каменен стълб и принесе възлияние на него и го поля с масло.
\par 15 И Яков наименува мястото, гдето Бог говори с него, Ветил.
\par 16 След това тръгнаха от Ветил; а като стигнаха близо до Ефрата, Рахил роди и много се мъчеше при раждането си.
\par 17 А като се мъчеше да роди, бабата й рече: Не бой се, защото имаш още един син.
\par 18 А като предаваше душа, (защото умря), Рахил го наименува Венони, а баща му го нарече Вениамин.
\par 19 Така Рахил умря и я погребаха край пътя за Ефрата, (която е Витлеем).
\par 20 И над гроба й Яков издигна стълб; той е и до днес стълб на Рахилиния гроб.
\par 21 Подир това Израил тръгна и разпъна шатрата си оттатък Мигдал-едер.
\par 22 И когато Израил живееше в оная земя, Рувим отиде и лежа с наложницата на баща си Вала; и Израил се научи за това. Якововите синове бяха дванадесет души:
\par 23 синовете на Лия: Рувим, Якововия първороден, Симеон, Левий, Юда Исахар и Завулон;
\par 24 синовете от Рахил: Иосиф и Вениамин;
\par 25 а синовете на Рахилината слугиня Вала: Дан и Нефталим;
\par 26 и синовете на Лиината слугиня Зелфа: Гад и Асир. Тия са Якововите синове, които му се родиха в Падан-арам.
\par 27 После Яков дойде при баща си Исаака в Мамврий, в Кириат-арва, (който е Хеврон), дето Авраам и Исаак бяха престояли.
\par 28 И дните на Исаака станаха сто и осемдесет години.
\par 29 И Исаак като издъхна, умря стар и сит от дни, прибра се при людете си и; и синовете му Исав и Яков го погребаха.

\chapter{36}

\par 1 Ето потомството на Исава, който е Едом.
\par 2 Исав си взе жена от Ханаанските дъщери: Ада, дъщеря на хетееца Елон; и Оливема, Анаевата дъщеря, внука на евееца Севегон;
\par 3 и Исмаиловата дъщеря Васемата, сестра на Навиота.
\par 4 Ада роди на Исава Елифаза; Васемата роди Рагуила.
\par 5 А Оливема роди Еуса, Иеглома и Корея. Тия са синовете на Исава, които му се родиха в Ханаанската земя.
\par 6 А Исав взе жените си, синовете си, дъщерите си и всичките човеци от дома си, добитъка си, всичките си животни и цялото си имущество, което беше придобил в Ханаанската земя, та отиде в една земя далеч от брата си Якова.
\par 7 Защото имуществото им беше толкова много, щото не можеха да живеят заедно; земята гдето престояваха не можеше да ги побере поради добитъка им.
\par 8 И така, Исав се засели на Сиирската поляна. Исав е Едом.
\par 9 Ето потомството на Исава, праотец на едомците в Сиирската поляна;
\par 10 ето имената на Исавовите синове: Елифаз, син от Исавовата жена Ада; Рагуил, син от Исавовата жена Васемата.
\par 11 А синовете на Елифаза бяха: Теман, Омар, Сефо, Готом и Кенез.
\par 12 А Тамна беше наложница на Исавовия син Елифаза и роди на Елифаза Амалика; тия са синовете от Исавовата жена Ада.
\par 13 И ето синовете на Рагуила: Нахат, Зара, Сама и Миза; тия са синовете от Исавовата жена Васемата.
\par 14 И ето синовете от Оливема, Анаевата дъщеря, Севегоновата внука, Исавовата жена: тя роди на Исава Еуса, Еглома и Корея.
\par 15 Ето първенците на Исавовите синове; синовете на Исавовия първороден Елифаз: главатар Теман, главатар Омар, главатар Сефо, главатар Кенез,
\par 16 главатар Корей, главатар Готом, главатар Амалик; тия са главатарите произлезли от Елифаза в Едомската земя; тия са синовете от Ада.
\par 17 И ето синовете на Исавовия син Рагуила; главатар Нахат, главатар Зара, главатар Сама, главатар Миза; тия са главатарите произлезли от Рагуила в Едомската земя; тия са синовете от Исавовата жена Васемата.
\par 18 И ето синовете от Исавовата жена Оливема: главатар Еус, главатар Еглом, главатар Корей; тия са главатарите произлезли от Анаевата дъщеря Оливема, Исавовата жена.
\par 19 Тия са синовете на Исава, който е Едом, и тия са главатарите им.
\par 20 Ето синовете на Корееца Сиир, които живееха в оная земя: Лотан, Совал, Севегон, Ана,
\par 21 Дисон, Асар и Дисан; тия са главатарите произлезли от хорейците, Сиировите чада, в Едомската земя.
\par 22 А синовете на Лотана бяха Хори и Емам; и Лотанова сестра беше Тамна.
\par 23 А ето синовете на Совала: Алван, Манахат, Гевал, Сефо и Онам.
\par 24 И ето синовете на Севегона: Ая и Ана; Ана е тоя, който намери горещите извори в пустинята, като пасеше ослите на баща си Севегона.
\par 25 И ето Анаевите чада: Дисон и Оливема, Анаевата дъщеря.
\par 26 И ето синовете на Дисона: Амадан, Асван, Итран и Харан.
\par 27 Ето синовете на Асара: Валаан, Заван и Акан.
\par 28 Ето синовете на Дисана: Уз и Аран.
\par 29 Ето главатарите произлезли от хорейците: главатар Лотан, главатар Совал, главатар Севегон, главатар Ана.
\par 30 Главатар Дисон, главатар Асар, главатар Дисан; тия са произлезлите от хорейците главатари, според главатарствата им в Сиирската земя.
\par 31 Ето и царете, които царуваха в Едомската земя, преди да се възцари цар над израилтяните.
\par 32 Царува, в Едом, Вела, син на Веора; а името на града му беше Денава.
\par 33 Като умря Вела, възцари се вместо него Иовав, син на Зара, от Восора.
\par 34 Като умря Иовав, възцари се вместо него Хусам, от земята на теманците.
\par 35 Като умря Хусам, възцари се вместо него Адад, син на Вадада, който порази мадиамците на моавското поле; а името на града му беше Авит.
\par 36 Като умря Адад, възцари се вместо него Самла, от Марсека.
\par 37 Като умря Самла, възцари се вместо него Саул, от Роовот, който е при Ефрат.
\par 38 Като умря Саул, възцари се вместо него Вааланан, син на Аховора.
\par 39 Като умря Вааланан, Аховоровият син, възцари се вместо него Адар; а името на града му беше Пау; и името на жена му беше Метавеил, дъщеря на Метреда, Мезаавова внука.
\par 40 Ето имената на произлезлите от Исава първенци, според семействата им, според местата им, според имената им: главатар Тамна, главатар Алва, главатар Етет,
\par 41 главатар Оливема, главатар Ила, главатар Финон,
\par 42 главатар Кенез, главатар Теман, главатар Мивсар,
\par 43 главатар Магедиил, главатар Ирам; тия са Едомските първенци, според седалищата им в земята, която притежаваха. Това е Исав, праотец на едомците.

\chapter{37}

\par 1 А Яков живееше в Ханаанската земя, земята в която баща му беше пришелец.
\par 2 Ето словото за Якововото потомство. Иосиф, когато беше момче на седемнадесет години, пасеше овците заедно с братята си, синовете на Вала и синовете на Зелфа, жените на баща му; и Иосиф съобщаваше на баща им за лошото им поведение.
\par 3 А Израил обичаше Иосифа повече от всичките си чада, защото беше син на старостта му; и му беше направил шарена дрешка.
\par 4 Но братята му, като гледаха, че баща им го обичаше повече от всичките му братя, намразиха го и не можеха да му говорят спокойно.
\par 5 А Иосифа видя сън и го разказа на братята си, и те го намразиха още повече.
\par 6 Той им рече: Чуйте, моля, тоя сън, който видях:
\par 7 Ето ние връзвахме снопи на полето; и моят сноп стана и се изправи, и вашите снопи се наредиха наоколо и се поклониха на моя сноп.
\par 8 А братята му рекоха: Ти цар ли ще станеш над нас? Или господар ще ни станеш? И намразиха го още повече, поради сънищата му и поради думите му.
\par 9 А той видя и друг сън и го разказа на братята си, казвайки: Ето, видях още един сън, че слънцето и луната и единадесет звезди ми се поклониха.
\par 10 Но когато разказа това на баща си и на братята си, смъмра го баща му, като му каза: Какъв е тоя сън, който си видял? Дали наистина аз и майка ти, и братята ти, ще дойдем да ти се поклоним до земята?
\par 11 И завидяха му братята му; а баща му запомни тия думи.
\par 12 А когато братята му бяха отишли да пасат стадото на баща си в Сихем,
\par 13 Израил рече на Иосифа: Не пасат ли братята ти стадото в Сихем? Ела да те изпратя при тях. А той му рече: Ето ме.
\par 14 И каза му: Иди, виж, добре ли са братята ти и стадото, и ми донеси известие. И тъй, изпрати го от Хевронската долина, и той дойде в Сихем.
\par 15 И един човек го намери като се луташе из полето; и човекът го попита, казвайки: Що търсиш?
\par 16 А той рече: Търся братята си; кажи ми, моля, где пасат стадото .
\par 17 И човекът рече: Заминаха оттука, защото ги чух да казват: Нека идем в Дотан. И тъй, Иосиф отиде подир братята си и ги намери в Дотан.
\par 18 А те, като го видяха от далеч, доде още не беше се приближил при тях, сговориха се против него да го убият.
\par 19 Рекоха си един на друг: Ето иде тоя съновидец.
\par 20 Елате сега, да го убием и да го хвърлим в един от тия ровове; па ще кажем: Лют звяр го е изял; и ще видим какво ще излезе от сънищата му.
\par 21 Но Рувим, като чу това, избави го от ръката им и рече: Да не го убиваме.
\par 22 Рече им още Рувим: Не проливайте кръв, хвърлете го в тоя ров, който е в пустинята, но ръка да не дигнете на него; каза така , за да го избави от ръката им и да го върне на баща му.
\par 23 И когато дойде Иосиф при братята си, съблякоха от Иосифа дрешката му, шарената дрешка, която носеше.
\par 24 Тогава го взеха и го хвърлиха в рова; а ровът беше празен, нямаше вода.
\par 25 После, като бяха седнали да ядат хляб, подигнаха очи и видяха, ето, един керван исмаиляни идеше от Галаад, с камилите си натоварени с аромати, балсам и смирна, и отиваха да ги закарат в Египет.
\par 26 Тогава Юда рече на братята си: Каква полза ако убием брата си и скрием кръвта му?
\par 27 Елате да го продадем на исмаиляните; да не туряме ръка на него, защото е наш брат, наша плът. И братята му го послушаха.
\par 28 И като минаваха мадиамските търговци, извлякоха и извадиха Иосифа из рова, и продадоха Иосифа на исмаиляните за двадесет сребърника; а те заведоха Иосифа в Египет.
\par 29 А Рувим се върна при рова, и, ето, Иосиф не беше в рова. И раздра дрехите си.
\par 30 И върна се при братята си та рече: Няма детето; а аз, аз къде да се дяна?
\par 31 Тогава взеха Иосифовата дрешка, заклаха козел и, като натопиха дрешката в кръвта,
\par 32 изпратиха шарената дрешка да я занесат на баща им, като рекоха: Намерихме това; познай сега дали е дрешката на сина ти, или не.
\par 33 И той я позна и рече: Това е дрешката на сина ми; лют звяр го е изял; несъмнено Иосиф е разкъсан.
\par 34 И Яков раздра дрехите си, тури вретище около кръста си и оплаква сина си за дълго време.
\par 35 И всичките му синове, и всичките му дъщери станаха, за да го утешават; но той не искаше да се утеши, защото казваше: С жалеене ще сляза при сина си в гроба. И баща му го оплакваше.
\par 36 А мадиамците продадоха Иосифа в Египет на Петефрия, Фараонов придворен, началник на телохранителите.

\chapter{38}

\par 1 По онова време Юда се отдели от братята си и свърна при един одоламец на име Ира.
\par 2 И Юда като видя там дъщерята на един ханаанец, на име Суя, взе я и влезе при нея.
\par 3 И тя зачна и роди син; и той го наименува Ир.
\par 4 И зачна пак и роди син; и тя го наименува Онан.
\par 5 Пак роди и друг син и го наименува Шела. А Юда беше в Ахдив, когато тя го роди.
\par 6 След време Юда взе жена за първородния си Ир, на име Тамар.
\par 7 А Ир, Юдовият първороден, беше нечестив пред Господа; и Господ го уби.
\par 8 Тогава рече Юда на Онана: Влез при братовата си жена и извърши към нея длъжността на девер и въздигни потомството на брата си.
\par 9 Но Онан знаеше, че потомството нямаше да бъде негово; затова, когато влизаше при братовата си жена, изливаше семето си на земята, за да не въздигне потомство на брата си.
\par 10 А това, което правеше, бе зло пред Господа; затова и него уби.
\par 11 Тогава, Юда, рече на снаха си Тамар: Живей, като вдовица в бащиния си дом, догде отрасне син ми Шела; защото си думаше: Да не би и той да умре като братята си. И така, Тамар отиде и живя в бащиния си дом.
\par 12 След дълго време, Юдовата жена, дъщеря на Суя, умря; и като се утеши Юда, отиде, той и приятелят му, одоламецът Ира, при стригачите на овците си в Тамна.
\par 13 И известиха на Тамар, казвайки: Ето, свекърът ти отива в Тамна, за да стриже овците си.
\par 14 Тогава тя съблече вдовишките си дрехи, покри се с покривалото си, обви се и седна при кръстопътя на Енаим, който е по пътя за Тамна; защото видя, че порасна Шела, а тя не му бе дадена за жена.
\par 15 А Юда, като я видя, помисли, че е блудница: защото беше покрила лицето си.
\par 16 Той, прочее, свърна към нея на пътя и рече: Остави ме, моля, да вляза при тебе; ( защото не позна, че беше снаха му ). И тя рече: Какво ще ми дадеш, за да влезеш при мене?
\par 17 А той рече: Ще ти изпратя яре от стадото. И тя отвърна: Даваш ли ми залог, догде го изпратиш?
\par 18 Рече той: Какъв залог да ти дам? И тя каза: Печата си, ширита си и тоягата си, която е в ръката ти. И той й ги даде. И влезе при нея, и тя зачна от него.
\par 19 После тя стана та си отиде, свали покривалото си и облече вдовишките си дрехи.
\par 20 А Юда изпрати ярето, чрез ръката на приятеля си одоламеца, за да вземе залога от ръката на жената; но той не я намери.
\par 21 Затова попита хората от онова място, казвайки: Где е блудницата, която беше на пътя при Енаим? А те рекоха:Тука не е имало блудница.
\par 22 И той се върна при Юда и рече: Не я намерих; още и хората от онова място рекоха: Тук не е имало блудница.
\par 23 И рече Юда: Нека си държи нещата, да не станем за присмех; ето, аз пратих това яре, но ти не я намери.
\par 24 Около три месеца подир това, известиха на Юда, казвайки: Снаха ти Тамар блудствува; а още, ето, непразна е от блудството. А Юда рече: Изведете я да се изгори.
\par 25 А когато я извеждаха, тя изпрати до свекъра си да му кажат: От човека, чиито са тия неща, съм непразна. Рече още: Познай, моля, чии са тия неща - печатът, ширитът и тоягата.
\par 26 И Юда ги позна и рече: Тя е по-права от мене, тъй като не я дадох на сина си Шела. И не я позна вече.
\par 27 И когато дойде времето й да роди, ето, имаше близнета в утробата й.
\par 28 И като раждаше, едното простря ръка: и бабата взе та върза червен конец на ръката му и каза: Тоя пръв излезе.
\par 29 А като дръпна надире ръката си, ето, брат му излезе; и тя рече: Какъв пролом си направи ти? Затова го наименуваха Фарес.
\par 30 После излезе брат му, който имаше червения конец на ръката си, и него наименуваха Зара.

\chapter{39}

\par 1 А Иосифа заведоха в Египет, и египтянинът Петефрий, Фараонов придворен, началник на телохранителите, го купи от ръката на исмаиляните, които го доведоха там.
\par 2 И Господ беше с Иосифа, и той благоуспяваше и се намираше в дома на господаря си египтянина.
\par 3 И като видя господарят му, че Господ бе с него, и че Господ прави да успява в ръката му всичко, което вършеше,
\par 4 Иосиф придоби благоволение пред очите му и му служеше; и той го постави настоятел на дома си, като предаде в ръката му всичко, което имаше.
\par 5 И откакто го постави настоятел на дома си и на всичко, което имаше, Господ благослови дома на египтянина заради Иосифа; Господното благословение беше върху всичко, що имаше в дома и по нивите.
\par 6 А Петефрий остави всичко що имаше в Иосифовата ръка, и, освен хляба, който ядеше, не знаеше нищо за онова, което притежаваше. А Иосиф беше строен и красив.
\par 7 И след време, жената на господаря му хвърли очи на Иосифа и му рече: Легни с мене.
\par 8 Но той отказа и рече на жената на господаря си: Виж, господарят ми не знае нищо за онова, което е с мене в дома и предаде в моята ръка всичко що има;
\par 9 в тоя дом няма никой по-голям от мене, нито е задържал от мене друго нещо освен тебе, защото си му жена; как, прочее, да сторя аз това голямо зло и да съгреша пред Бога?
\par 10 И при все, че тя говореше на Иосифа всеки ден, той не я послуша да лежи с нея, нито да бъде с нея.
\par 11 А един ден, като влезе Иосиф в къщи, за да върши работата си, а никой от домашните мъже не беше там в къщи,
\par 12 тя го хвана за дрехата и му каза: Легни с мене. Но той остави дрехата си в ръката й, избяга и излезе вън.
\par 13 А като видя, че остави дрехата си в ръката й и избяга вън,
\par 14 тя извика домашните си мъже и им говори, казвайки: Вижте, доведе ни един евреин, за да се поругае с нас; той влезе при мене, за да ме изнасили; но аз извиках с висок глас.
\par 15 А той, като чу, че извиках с висок глас, остави дрехата си при мене и избяга та излезе вън.
\par 16 И тя задържа дрехата му при себе си, докато си дойде господарят му в дома си.
\par 17 И според тия думи му говори, казвайки: Еврейският слуга, когото ти си ни довел, влезе при мене, за да ми се поругае;
\par 18 но, като извиках с висок глас, той остави дрехата си при мене и избяга вън.
\par 19 Като чу господарят му думите, които му рече жена му, казвайки: Така ми стори слугата ти, гневът му пламна.
\par 20 И господарят му взе Иосифа и го хвърли в крепостната тъмница, в мястото гдето бяха запирани царските затворници; и той остана там в крепостната тъмница.
\par 21 Но Господ беше с Иосифа и показваше благост към него и му даде благоволение пред очите на тъмничния началник.
\par 22 Тъй че тъмничният началник предаде на Иосифовата ръка всичките затворници, които бяха в крепостната тъмница; и за всичко що се вършеше там, той беше разпоредник.
\par 23 Тъмничният началник не нагледваше нищо от онова, което бе в ръката на Иосифа ; защото Господ беше с него и Господ правеше да благоуспява всичко, каквото той вършеше.

\chapter{40}

\par 1 След това, виночерпецът и хлебарят на Египетския цар се провиниха пред господаря си, Египетския цар.
\par 2 Та Фараон, като се разгневи на двамата си придворни - началника на виночерпците и началника на хлебарите -
\par 3 тури ги под стража, в дома на началника на телохранителите, в крепостната тъмница, в мястото, гдето Иосиф бе затворен.
\par 4 А началникът на телохранителите постави при тях Иосифа, и той им слугуваше; и те останаха известно време в тъмницата.
\par 5 И виночерпецът и хлебарят на Египетския цар, които бяха затворени в крепостната тъмница, сънуваха и двамата сън, всеки видя съня си в същата нощ, всеки според както щеше да се тълкува съновидението му.
\par 6 И Иосиф, като влезе при тях на утринта и видя, че бяха смутени,
\par 7 попита Фараоновите придворни, които бяха заедно с него в тъмницата, в дома на неговия господар, казвайки: Защото изглеждате тъй скръбни днес?
\par 8 А те му казаха: Видяхме сън, а няма кой да го изтълкува. И Иосиф им рече: Тълкуванията не са ли от Бога? Разкажете ми го, моля?
\par 9 Тогава началникът на виночерпците разказа своя сън на Иосифа, като му рече: В съня ми, ето лоза пред мене;
\par 10 и на лозата имаше три пръчки, и виждаше се, като че напъпваше и цветовете й цъфтяха и гроздовете на лозата узряха.
\par 11 А Фараоновата чаша беше в ръката ми; и взех гроздето, изстисках го във Фараоновата чаша и подадох чашата във Фараоновата ръка.
\par 12 И Иосиф му рече: Ето значението му: трите пръчки са три дни.
\par 13 След три дена Фараон ще издигне главата ти и ще те възстанови на службата ти; и ще поднасяш чашата във Фараоновата ръка, както преди, когато ти му беше виночерпец.
\par 14 Но спомни си за мене, когато те постигне благополучието, смили се, моля, за мене, та продумай на Фараона за мене и избави ме от тоя дом.
\par 15 Понеже наистина бях откраднат от Еврейската земя, а пък тук не съм сторил нищо, за да ме хвърлят в тая яма.
\par 16 Като видя началникът на хлебарите, че той тълкува добре, рече на Иосифа: И аз сънувах; и, ето, три кошници с бял хляб бяха на главата ми;
\par 17 в най-горната кошница имаше от всякакви ястия за Фараона; и птици ги ядяха от кошницата на главата ми.
\par 18 А Иосиф в отговор каза: Ето значението му: трите кошници са трите дни.
\par 19 След три дни Фараон ще ти отнеме главата, като те обеси на дърво; и птиците ще изкълват месата ти от тебе.
\par 20 След три дни, на рождения си ден, Фараон направи угощение на всичките си слуги; и издигна главата на началника на виночерпците и главата на началника на хлебарите между слугите си:
\par 21 Началника на виночерпците възстанови на служба и той подаваше чашата във Фараоновата ръка;
\par 22 а началника на хлебарите обеси, според, както Иосиф бе изтълкувал сънищата им.
\par 23 А началникът на виночерпците не си спомни за Иосифа, но го забрави.

\chapter{41}

\par 1 Като изминаха две години, Фараон сънува, че стоеше при Нил .
\par 2 ето, седем крави, хубави и тлъсти, излизаха из реката и пасяха в тръстиката.
\par 3 ето, след тях излизаха из реката други седем крави, грозни и мършави, и стоеха при първите крави на речния бряг.
\par 4 И грозните, мършави крави изядоха седемте хубави и тлъсти крави. Тогава Фараон се пробуди.
\par 5 А като заспа сънува втори път; и, ето, седем класа пълни и добри израснаха из едно стъбло.
\par 6 ето, след тях израснаха други седем класа тънки и прегорели от източния вятър.
\par 7 И тънките класове погълнаха седемте дебели и пълни класове. А като се събуди Фараон, ето, беше сън.
\par 8 На сутринта духът му беше смутен; той, прочее, изпрати да му повикат всичките влъхви и всичките мъдреци в Египет, и Фараон им разказа сънищата си; но нямаше кой да ги изтълкува на Фараона.
\par 9 Тогава началникът на виночерпците говори на Фараона, казвайки: Днес се сещам, че съм виновен.
\par 10 Фараон беше се разгневил на слугите си и ме хвърли, мене и началника на хлебарите, в тъмница, в дома на началника на телохранителите.
\par 11 И сънувахме, аз и той, в същата нощ; сънувахме, всеки според както щеше да се тълкува съновидението му.
\par 12 А заедно с нас беше там един млад евреин, слуга на началника на телохранителите; и като му разказахме, той ни разтълкува сънищата; на всеки от нас , според съновидението му, даде значението.
\par 13 И според както ни изтълкува, така и стана; мене той възстанови на службата ми, а него обеси.
\par 14 Тогава Фараон направи да повикат Иосифа, и бързо го изведоха из тъмницата; и той се обръсна, преоблече се и дойде при Фараона.
\par 15 И Фараон каза на Иосифа: Сънувах, но няма кой да изтълкува съня; но аз чух да казват за тебе, че като чуеш съновидение, умееш да го тълкуваш.
\par 16 А Иосиф в отговор каза на Фараона: Не аз, Бог ще даде на Фараона отговор с мир.
\par 17 Тогава Фараон каза на Иосифа: В съня си, ето, стоях край брега на Нил.
\par 18 И, ето, седем крави тлъсти и хубави излязоха из реката и пасяха в тръстиката.
\par 19 И, ето, след тях излязоха други седем крави, слаби, много грозни и мършави, каквито по грозота никога не съм видял в цялата Египетска земя.
\par 20 И мършавите, грозни крави изядоха първите седем тлъсти крави;
\par 21 но пак, като ги изядоха не се познаваше, че са ги изяли; но изгледът им беше тъй грозен, както и в началото. Тогава се пробудих.
\par 22 После видях в съня си; и, ето, седем класа пълни и добри израстваха из едно стъбро.
\par 23 И, ето, след тях изратнаха други седем класа, сухи, тънки и прегорели от източния вятър;
\par 24 тънките класове погълнаха седемте добри класове. И казах съновидението на влъхвите; но нямаше кой да ми го изтълкува.
\par 25 Тогава Иосиф каза на Фараона: Фараоновият сън е един; Бог е явил на Фараона това, което скоро ще направи.
\par 26 Седемте добри крави са седем години; и седемте добри класове са седем години, сънят е един.
\par 27 А седемте мършави и грозни крави, които излязоха след тях, са седем години, както и седемте празни класове, прегорели от източния вятър; те ще бъдат седем години на глад.
\par 28 Това е, което казах на Фараона; Бог е явил на Фараона това, което скоро ще направи.
\par 29 Ето идат седем години на голямо плодородие по цялата Египетска земя.
\par 30 А след тях ще дойдат седем години на глад; всичкото плодородие ще се забрави в Египетската земя, защото гладът ще опустоши земята.
\par 31 Няма да се познае плодородието на земята, поради оня глад, който ще последва; защото ще бъде твърде тежък.
\par 32 А това, дето съновидението се повтори на Фараона два пъти, означава , че това е решено от Бога и, че Бог скоро ще го извърши.
\par 33 Прочее, нека потърси Фараон умен и мъдър човек и нека го постави над Египетската земя.
\par 34 Нека стори Фараон това и нека постави наздиратели над земята; и в седемте плодородни години нека събере петата част от произведенията на Египетската земя.
\par 35 Нека съберат всичката храна на тия добри години, които идат; и събраното жито нека бъде под Фараоновата власт, за храна на градовете, и нека го пазят.
\par 36 И храната ще се запази за седемте гладни години, които ще настанат в Египетската земя, за да не се опустоши земята от глада.
\par 37 Това нещо беше угодно на Фараона и на всичките му слуги.
\par 38 И Фараон каза на слугите си: Можем ли да намерим човек, като тоя, в когото има Божия Дух?
\par 39 Тогава Фараон каза на Иосифа: Понеже Бог ти откри всичко това, няма никой толкова умен и мъдър, колкото си ти.
\par 40 Ти ще бъдеш над дома ми, и всичките ми люде ще слушат твоите думи; само с престола аз ще бъда по-горен от тебе.
\par 41 Фараон още каза на Иосифа: Виж, поставям те над цялата Египеска земя.
\par 42 Тогава Фараон извади пръстена си от ръката си и го тури на Иосифовата ръка, облече го в дрехи от висон и окачи му златно огърлие на шията.
\par 43 После нареди да го возят на втората колесница, и викаха пред него: Коленичете! И така го постави над цялата Египетска земя.
\par 44 При това, Фараон каза на Иосифа: Аз съм владетел, но без тебе никой няма да издигне ръка или нога по цялата Египетска земя.
\par 45 И Фараон наименува Иосифа Цафнат-панеах и даде му за жена Асенета, дъщеря на Илиополския жрец Потифер. След това, Иосиф излезе по обиколка из Египетската земя.
\par 46 Иосиф беше на тридесет години, когато стоя пред Египетския цар Фараон; и, като се отдалечи от лицето на Фараона, Иосиф обиколи цялата Египетска земя.
\par 47 И през седемте години на плодородие земята роди преизобилно.
\par 48 И Иосиф събра всичката храна от тия седем години, които бяха настанали в Египетската земя, и тури храната в градовете; във всеки град прибра храната от околните му ниви.
\par 49 Иосиф събра жито твърде много, колкото морския пясък, така щото престана да го мери; защото беше без мяра.
\par 50 А преди да настъпят годините на глада, на Иосифа се родиха два сина, които му роди Асенета, дъщеря на Илиополския жрец Потифер.
\par 51 И Иосиф наименува първородния Манасия, защото си думаше : Бог ме направи да забравя всичките си мъки и целия си бащин дом.
\par 52 А втория наименува Ефрем, защото си думаше : Бог ме направи плодовит в земята на страданието ми.
\par 53 А като се изминаха седемте години на плодородие, които бяха настанали в Египетската земя
\par 54 настъпиха седемте години на глад, според както Иосиф бе казал; и настана глад по всичките земи, а по цялата Египетска земя имаше хляб.
\par 55 Защото, когато огладня цялата Египетска земя, и людете извикаха към Фараона за хляб, Фараона рече на всичките египтяни: Идете при Иосифа, и каквото ви каже, сторете.
\par 56 (А гладът бе по цялото лице на земята). Иосиф, прочее, отвори всичките житници и продаваше на египтяните, защото гладът се усилваше по Египетската земя.
\par 57 И от всичките страни дохождаха в Египет при Иосифа да купят жито, понеже гладът се усилваше по цялата земя.

\chapter{42}

\par 1 А като видя Яков, че в Египет се намира жито, Яков рече на синовете си: Защо се гледате един друг?
\par 2 Рече още: Ето, чух, че в Египет се намира жито; слезте там та ни купете от там, за да живеем и да не измрем.
\par 3 Тогава десетте Иосифови братя слязоха да купят жито от Египет.
\par 4 А Яков не изпрати Вениамина, Иосифовия брат, заедно с братята му; защото думаше: Да не би да му се случи нещастие.
\par 5 И тъй, между ония, които идеха, дойдоха и синовете на Израиля да купят; защото имаше глад и в Ханаанската земя.
\par 6 А понеже Иосиф беше управител на земята и той беше, който продаваше на всичките люде на оная земя, затова братята на Иосифа, като дойдоха, поклониха му се с лицата си до земята.
\par 7 А Иосиф, като видя братята си, позна ги, но се престори, като чужд на тях, говореше им грубо и им рече: От где идете? А те рекоха: От Ханаанската земя, за да купим храна.
\par 8 (А при все, че Иосиф позна братята си, те не го познаха).
\par 9 Тогава Иосиф, като си спомни сънищата, които беше видял за тях, рече им: Вие сте шпиони; дошли сте да съглеждате голотата на тая земя.
\par 10 А те му казаха: Не, господарю, слугите ти дойдоха да си купят храна.
\par 11 Ние всички сме синове на един човек, честни човеци сме, слугите ти не са шпиони.
\par 12 Но той им рече: Не, дошли сте да съглеждате голотата на земята.
\par 13 А те казаха: Ние, твоите слуги, сме дванадесет братя, синове на един човек в Ханаанската земя; и, ето, най-младият е днес при баща ни, а единият го няма.
\par 14 А Иосиф им рече: Това е, което ви казах, когато рекох: Шпиони сте.
\par 15 Ето как ще бъдете опитани: В името на Фараона, няма да излезете от тука, ако не дойде и по-младият ви брат тука.
\par 16 Пратете един от вас да доведе брата ви; а вие ще останете затворени догде се проверят думите ви, дали говорите истина; и ако не, в името на Фараона, наистина вие сте шпиони.
\par 17 И ги постави под стража за три дена.
\par 18 А на третия ден Иосиф им рече: Това сторете и ще живеете, защото аз се боя от Бога:
\par 19 Ако сте честни, нека остане един от вашите братя в къщата, в която сте пазени; вие идете, закарайте жито за гладните си челяди,
\par 20 па ми доведете най-младият си брат; така ще се докаже, че думите ви са истинни, и вие няма да умрете. И сториха така.
\par 21 И рекоха си един на друг: Наистина сме виновни за нашия брат, гдето видяхме мъката на душата му, когато ни се молеше и ние не го послушахме; затова ни постигна туй бедствие.
\par 22 А Рувим им отговори казвайки: Не ви ли говорих тия думи: Не съгрешавайте против детето, но вие не послушахте. Затова, вижте, кръвта му се изисква.
\par 23 А те не знаеха, че Иосиф разбираше, защото говореха с него чрез преводач.
\par 24 И той се оттегли от тях и плака; после, като се върна при тях, говореше им; и взе измежду тях Симеона та го върза пред очите им.
\par 25 Тогава Иосиф заповяда да напълнят съдовете им с жито, да върнат парите на всекиго в чувала му, и да им дадат храна за из пътя; и сториха им така.
\par 26 А те натовариха житото на ослите си и си тръгнаха от там.
\par 27 Но когато един от тях развърза чувала си на мястото за пренощуване, за да даде храна на осела си, видя, че парите му бяха отгоре в чувала.
\par 28 И рече на братята си: Парите ми са повърнати; наистина, вижте ги в чувала ми. Тогава сърцата им се ужасиха, и те се обръщаха с трепет един към друг и казваха: Що е това, което ни стори Бог?
\par 29 И като дойдоха при баща си Якова, в Ханаанската земя, разказаха му всичко, което им се бе случило.
\par 30 Рекоха: Човекът, който е господар на оная земя, ни говори грубо, и ни взе за човеци дошли да съгледат страната.
\par 31 Но ние му казахме: Честни човеци сме, не сме шпиони;
\par 32 дванадесет братя сме, синове на един баща; единият се изгуби, а най-младият е днес при баща ни в Ханаанската земя.
\par 33 И човекът, господарят на земята ни каза: Ето как ще позная дали сте честни: оставете един от вашите братя при мене и вземете жито за гладните си домочадия и си идете,
\par 34 па ми доведете най-младия си брат; тогава ще позная, че не сте шпиони, а сте честни, и ще пусна брата ви, и вие ще търгувате в тая земя.
\par 35 А като изпразваха чувалите си, ето, на всеки възела с парите му беше в чувала му; и те и баща им се уплашиха като видяха възлите с парите си.
\par 36 Тогава баща им Яков каза: Вие ме оставихте без чада; Иосифа няма, Симеона няма, а искате и Вениамина да заведете; върху мене падна всичко това!
\par 37 А Рувим, като говореше на баща си, рече: Убий двамата ми сина, ако не ти го доведа; предай го в моята ръка и аз пак ще ти го доведа.
\par 38 А Яков каза: Син ми няма да слезе с вас, защото брат му умря, и само той остана; ако му се случи нещастие по пътя, по който отивате, тогава ще свалите бялата ми коса със скръб в гроба.

\chapter{43}

\par 1 А гладът ставаше голям по земята.
\par 2 И като изядоха житото, което донесоха от Египет, баща им рече: Идете пак, купете ни малко храна.
\par 3 А Юда, като му говореше, каза: Човекът строго ни заръча, казвайки: Няма да видите лицето ми, ако брат ви не бъде с вас.
\par 4 Ако изпратиш брата ни с нас, ще слезем и ще ти купим храна;
\par 5 но ако не го изпратиш, няма да слезем; защото човекът ни рече: Няма да видите лицето ми, ако брат ви не бъде с вас.
\par 6 А Израил рече: Защо ми сторихте това зло та казахте на човека, че имате и друг брат?
\par 7 А те рекоха: Човекът разпита подробно за нас и за рода ни, като ни каза: Баща ви жив ли е още? имате ли и друг брат? И ние му отговорихме според тия негови думи. От где да знаем ние, че щеше да каже: Доведете брата си.
\par 8 Тогава, Юда рече на баща си, Израиля: Изпрати детето с мене, и да станем да отидем, за да живеем и да не измрем - и ние, ти и чадата ни.
\par 9 Аз отговарям за него; от моята ръка го искай; ако не ти го доведа и него представя пред тебе, тогава нека съм виновен пред тебе за винаги.
\par 10 Понеже, ако не бяхме се бавили, без друго до сега бихме се върнали втори път.
\par 11 Тогава, баща им Израил каза: Ако е тъй, това сторете: вземете в съдовете си от най-хубавите плодове на нашата земя - малко балсама и малко мед, аромати и смирна, фъстъци и бадеми - та ги занесете подарък на човека;
\par 12 вземете двойно повече пари в ръцете си, върнатите в чувалите ви пари, повърнете с ръцете си: може да е станало грешка.
\par 13 Вземете и брата си та станете и идете при човека.
\par 14 А Бог Всемогъщи да ви даде да придобиете благоволението на човека, та да пусне другия ви брат и Вениамина. А аз, ако е речено да остана без чада, нека остана без чада.
\par 15 Тогава човеците, като взеха тоя подарък, взеха двойно повече пари в ръцете си и Вениамина; и станаха та слязоха в Египет и представиха се на Иосифа.
\par 16 А Иосиф, като видя Вениамина с тях, рече на домакина си: Заведи тия човеци у дома ми и заколи каквото трябва и приготви; защото човеците ще обядват с мене на пладне.
\par 17 И човекът стори, каквото поръча Иосиф, и той въведе човеците в Иосифовия дом.
\par 18 А човеците се уплашиха, загдето ги поведоха в Иосифовия дом, и рекоха: Поради парите, върнати в чувалите ни първия път, ни повеждат, за да намери повод против нас, да ни нападне и да зароби нас и ослите ни.
\par 19 Затова те се приближиха при Иосифовия домакин и говориха му при вратата на дома, казвайки:
\par 20 Послушай, господарю, слязохме първия път да си купим храна;
\par 21 а на връщане , когато дойдохме до мястото за пренощуване и отворихме чувалите си, ето, на всеки парите бяха в чувала му, парите ни напълно: затова ги донесохме обратно в ръцете си.
\par 22 Донесохме и други пари в ръцете си, за да купим храна; не знаем, кой тури парите ни в чувалите ни.
\par 23 А той каза: Бъдете спокойни, не бойте се; Бог ваш и Бог на баща ви, ви даде скритото в чувалите ви съкровище; аз получих парите ви. Тогава изведе при тях Симеона.
\par 24 И домакинът въведе човеците в Иосифовия дом и им даде вода та си умиха нозете; даде и храна за ослите им.
\par 25 А те приготвиха подаръка за Иосифа преди да дойде на пладне; защото чуха, че там щели да ядат хляб.
\par 26 И когато дойде Иосиф у дома, поднесоха му в къщи подаръка, който беше в ръцете им; и му се поклониха до земята.
\par 27 И след като ги разпита за здравето им, рече: Здрав ли е баща ви, старецът, за когото ми говорехте? Жив ли е още?
\par 28 А те казаха: Здрав е слугата ти, баща ни, жив е още. И наведоха се та се поклониха.
\par 29 И като подигна очи видя едноутробния си брат Вениамин и каза: Тоя ли е най-младият ви брат, за когото ми говорихте? И рече: Бог да бъде милостив към тебе, чадо мое.
\par 30 И Иосиф бързо се оттегли, защото сърцето му се развълнува за брат му; и като искаше да плаче, влезе в стаята си и плака там
\par 31 После уми лицето си и излезе и задържайки се рече: Сложете хляб.
\par 32 И сложиха отделно на него, отделно за тях и отделно за египтяните, които ядяха с него; защото не биваше египтяните да ядат хляб с евреите ,понеже това е гнусота за египтяните.
\par 33 И насядаха пред него, първородният според първородството му, и най-младият според младостта му; и човеците се чудеха помежду си.
\par 34 И Иосиф им пращаше от ястията си; а Вениаминовият дял беше пет пъти по-голям от дела на когото и да било от тях. И те пиха, пиха изобилно с него.

\chapter{44}

\par 1 После заповяда на домакина си, казвайки: Напълни чувалите на човеците с храна, колкото могат да поберат, тури парите на всекиго в чувала му
\par 2 и тури чашата ми, сребърната чаша, отгоре в чувала на най-младия, с парите за житото му. И той стори според това, което рече Иосиф.
\par 3 На утринта, щом съмна, изпратиха човеците и ослите им.
\par 4 А когато бяха излезли из града и не бяха се отдалечили много, Иосиф рече на домакина си: Стани, тичай след човеците и, като ги стигнеш, кажи им: Защо върнахте зло за добро?
\par 5 Не е ли тая чашата с която пие господарят ми, и с която даже гадае? Зле постъпихте като сторихте това.
\par 6 И човекът ,като ги настигна, каза им тия думи.
\par 7 А те му рекоха: Защо говори господарят ни такива думи? Не дай, Боже, слугите ти да сторят такова нещо.
\par 8 Ето, ние ти върнахме от Ханаанската земя, парите които намерихме отгоре в чувалите си; и как бихме откраднали сребро или злато из дома на господаря ти?
\par 9 Този от слугите ти, у когото се намери, нека умре, тоже и ние нека бъдем роби на господаря си.
\par 10 А той рече: Нека бъде според както казахте: У когото се намери, той ще ми бъде роб, а вие не ще бъдете виновни.
\par 11 Тогава те бързо снеха чувалите си на земята, и всеки отвори чувала си.
\par 12 И той претърси, като почна от най-стария и свърши с най-младия; и чашата се намери във Вениаминовия чувал.
\par 13 Тогава раздраха дрехите си, натовариха всеки осела си, и се върнаха в града.
\par 14 И дойдоха Юда и братята му в дома на Иосифа, гдето той още се намираше, и паднаха пред него на земята.
\par 15 И рече им Иосиф: Какво е това що сторихте? Не знаете ли, че човек, като мене, може да гадае безпогрешно?
\par 16 Тогава Юда каза: Що да речем на господаря си? що да говорим? или как да се оправдаем? Бог откри неправдата на слугите ти; ето, роби сме на господаря си, и ние и оня у когото се намери чашата.
\par 17 Но Иосиф рече: Не дай, Боже, да сторя това: Оня, у когото се намери чашата, той ще ми бъде роб; а вие си идете с мир при баща си.
\par 18 Тогава Юда се приближи до него и рече: Моля ти се, господарю мой, позволи на слугата си да каже една дума на господаря си, като слушаш ти; и да не пламне гневът ти против слугата ти, защото ти си като Фараон.
\par 19 Господарят ми попита слугите си, казвайки: Имате ли баща или брат?
\par 20 И рекохме на господаря ми: Имаме стар баща и малко дете на старостта му, и неговият брат умря, тъй че само той остана от майка си,и баща му го обича.
\par 21 И ти рече на слугите си: Доведете ми го, за да го видя с очите си.
\par 22 И ние казахме на господаря ми: Детето не може да остави баща си, защото, ако остави баща си, той ще умре.
\par 23 А ти рече на слугите си: Ако не слезе с вас най-малкият ви брат, няма вече да видите лицето ми.
\par 24 И като отидохме при слугата ти, баща ни, разказахме му това, което беше казал моят господар.
\par 25 А когато баща ни рече: Идете пак, купете ни малко храна,
\par 26 ние казахме: Не можем да слезем. Ако най-младият ни брат е с нас, тогава ще слезем, защото не можем да видим лицето на човека, ако най-младият ни брат не е с нас.
\par 27 И слугата ти, баща ни, ни каза: Вие знаете, че жена ми ми роди два сина:
\par 28 Единият излезе от мене и си рекох: Навярно звяр го е разкъсал; и до сега не съм го видял;
\par 29 и ако ми отнемете и тоя, и му се случи нещастие, ще свалите бялата ми коса със скръб в гроба.
\par 30 И сега, когато отидем при слугата ти, баща ни, и детето не е с нас, то, понеже животът му е свързан с неговия живот,
\par 31 като види, че няма детето, ще умре; и слугите ти ще свалят бялата коса на слугата ти, баща ни, със скръб в гроба.
\par 32 Защото слугата ти стана поръчител пред баща си за детето, като казах: Ако не ти го доведа, тогава ще бъда за винаги виновен пред баща си.
\par 33 Сега, прочее, моля ти се, вместо детето нека остане слугата ти роб на господаря ми, а детето нека отиде с братята си.
\par 34 Защото, как да отида аз при баща си, ако детето не е с мене? Да не би да видя злото, което ще сполети баща ми.

\chapter{45}

\par 1 Тогава Иосиф не може вече да се стърпи пред всички ония, които стояха пред него, и извика: Изведете всички отпред мене. И не остана никой при Иосифа, когато той се откри на братята си.
\par 2 И заплака с глас, та египтяните чуха; чу още и Фараоновият дом.
\par 3 И Иосиф каза на братята си: Аз съм Иосиф. Баща ми жив ли е още? Но братята му не можаха да му отговорят, защото се смутиха от присъствието му.
\par 4 Тогава Иосиф рече на братята си: Елате близо до мен, моля ви. И те се приближиха. И рече: Аз съм брат ви Иосиф, когото вие продадохте в Египет.
\par 5 Сега, не скърбете, нито се окайвайте, че ме продадохте тук, понеже Бог ме изпрати пред вас, за да опази живота.
\par 6 Защото вече две години гладът върлува в страната; а остават още пет години, в които не ще има ни оране, ни жетва.
\par 7 Бог ме изпрати пред вас, за да съхраня от вас остатъка на земята и да опазя живота ви чрез голямо избавление.
\par 8 И тъй, не ме изпратихте вие тук, но Бог, който ме и направи отец на Фараона, господар на целия му дом и управител на цялата Египетска земя.
\par 9 Бързайте, идете при баща ми и му речете: Така казва синът ти Иосиф: Бог ме постави господар над целия Египет; ела при мене незабавно.
\par 10 Ти ще живееш в Гесенската земя и ще бъдеш близо при мене, ти, чадата ти, и внуците ти, стадата ти, добитъкът ти и всичко що имаш.
\par 11 Там ще те храня, (защото остават още пет години на глад), за да не изпаднеш в немотия, ти, домът ти и всичко що имаш.
\par 12 И гледайте, вашите очи и очите на брата ми Вениамина виждат, че моите уста ви говорят.
\par 13 И разкажете на баща ми всичката ми слава в Египет и всичко що видяхте; и бързо доведете баща ми тука.
\par 14 Тогава падна на врата на брата си Вениамина и плака, плака и Вениамина на неговия врат.
\par 15 И целуна всичките си братя и плака над тях; и след това братята му се разговаряха с него.
\par 16 И известие за това се чу във Фараоновия дом, като казаха: Иосифовите братя са дошли; и това стана угодно на Фараона и на слугите му.
\par 17 И Фараон рече на Иосифа: Кажи на братята си: Така направете: натоварете животните си и тръгнете, идете в Ханаанската земя.
\par 18 И като вземете баща си и челядите си, елате при мене; и аз ще ви дам благата на Египетската земя, и ще се храните с тлъстината на земята.
\par 19 А на тебе заповядвам да им речеш : Така сторете: вземете коли от Египетската земя за децата си и за жените си и доведете баща си и елате.
\par 20 При това, не жалете вещите си, защото благата на цялата Египетска земя ще бъдат ваши.
\par 21 И синовете на Израиля сториха така; и Иосиф им даде коли, според Фараоновата заповед, даде им и храна за из пътя.
\par 22 На всеки от тях даде дрехи за премяна; а на Вениамина даде триста сребърника и дрехи за пет премени.
\par 23 И на баща си изпрати десет осли натоварени с Египетски блага, десет ослици натоварени с жито и хляб, и храна за баща си за из пътя.
\par 24 Така изпрати братята си та си отидоха; и поръча им: Да не се карате по пътя.
\par 25 И тъй, те излязоха от Египет та дойдоха в Ханаанската земя, при баща си Якова.
\par 26 И известиха му, казвайки: Иосиф е още жив и е управител на цялата Египетска земя. А сърцето му примря, защото не ги вярваше.
\par 27 Но те му разказаха всичко, което Иосиф им беше говорил; и като видя колите, които Иосиф бе изпратил да го вземат, духът на баща им Якова се съживи.
\par 28 И рече Израил: Доволно е; син ми Иосиф е още жив, ще ида и ще го видя преди да умра.

\chapter{46}

\par 1 И така, Израил тръгна, с всичко що имаше; и като дойде във Вирсавее, принесе жертви на Бога на баща си Исаака.
\par 2 И Бог говори на Израиля в нощно видение, казвайки: Якове, Якове. А той отговори: Ето ме.
\par 3 И рече: Аз съм Бог, Бог на баща ти; не бой се да слезеш в Египет, защото ще те направя там велик народ.
\par 4 Аз ще сляза с тебе в Египет и Аз непременно ще те върна пак; и Иосиф ще тури ръката на очите ти.
\par 5 Тогава Яков стана от Вирсавее; и синовете на Израиля качиха баща си Якова, децата си и жените си на колите които Фараон бе пратил, за да говорят;
\par 6 събраха и добитъка си и имота, който бяха придобили в Ханаанската земя; и Яков и цялото му семейство с него дойдоха в Египет;
\par 7 той доведе със себе си в Египет синовете си и внуците си, дъщерите си и внуките си, - цялото си семейство.
\par 8 А ето имената на синовете на Израиля, които влязоха в Египет: Яков и синовете му: Рувим, първородният на Якова;
\par 9 а Рувимови синове: Енох, Фалу, Есрон и Хармий;
\par 10 Симеонови синове: Емуил, Ямин, Аод, Яхин, Сохар и Саул син на ханаанка;
\par 11 Левиеви синове: Гирсон, Каат и Мерарий;
\par 12 Юдови синове: Ир, Онан, Шела, Фарес и Зара; но Ир и Онан умряха в Ханаанската земя; и Фаресови синове бяха Есрон и Амул;
\par 13 Исахарови синове: Тола, Фуа, Иов и Симрон.
\par 14 Завулонови синове: Серед, Елон и Ялеил.
\par 15 Тия са синовете, които Лия роди на Якова в Падан-арам, и дъщеря му Дина. Те всички - синовете му и дъщерите му - бяха тридесет и трима души.
\par 16 А Гадови синове: Сифон, Агий, Суний, Есвон, Ирий, Ародий и Арилий;
\par 17 Асирови синове: Емна, Есуа, Есуй и Верия, и сестра им Сера; и Вериеви синове: Хевер и Малхиел.
\par 18 Тия са синовете на Зелфа, която Лаван даде на дъщеря си Лия; и тях тя роди на Якова - шестнадесет души.
\par 19 А синовете на Якововата жена Рахил: Иосиф и Вениамин;
\par 20 И на Иосифа се родиха в Египетската земя Манасия и Ефрем, които му роди Асенета, дъщеря на Илиополския жрец Потифер;
\par 21 Вениаминови синове: Вела, Вехер, Асвил, Гира, Нееман, Ихий, Рос, Мупим, Упим и Арет.
\par 22 Тия са Рахилините синове, които се родиха на Якова, всичките четиринадесет души.
\par 23 А Данови синове: Усим;
\par 24 Нефталимови синове: Ясиил, Гуний, Есер и Силим.
\par 25 Тия са синовете на Вала, която Лаван даде на дъщеря си Рахил, и тях тя роди на Якова, всичките седем души.
\par 26 Всичките човеци, които дойдоха с Якова в Египет, които излязоха из чреслата му, всичките бяха шестдесет и шест души, освен Якововите снахи;
\par 27 и синовете, които се родиха на Иосиф в Египет, бяха двама: всичките от Якововия дом, които дойдоха в Египет, бяха седемдесет души.
\par 28 И Яков изпрати Юда пред себе си при Иосифа, за да покаже пътя пред него в Гесен; и тъй, дойдоха в Гесенската земя.
\par 29 Тогава Иосиф впрегна колесницата си и отиде в Гесен да посрещне баща си Израиля, и като се яви пред него, падна на врата му и плака дълго на врата му.
\par 30 И рече Израил на Иосифа: Нека умра сега, като видях лицето ти, че ти си още жив.
\par 31 А Иосиф рече на братята си и на бащиния си дом: Ще отида да известя на Фараона, и ще му река: Братята ми и домът на баща ми, които бяха в Ханаанската земя, дойдоха при мене;
\par 32 а тия човеци, понеже са овчари, и се занимават със скотовъдство, докараха стадата и добитъка си и всичко що имат.
\par 33 И когато ви повика Фараон и попита: Какво ви е занятието?
\par 34 Кажете: Ние, слугите ти, и бащите ни, от младини до сега сме скотовъдци. Кажете това , за да живеете в Гесенската земя, защото всичките овчари са отвратителни за египтяните.

\chapter{47}

\par 1 И тъй, Иосиф влезе при Фараона и му извести, казвайки: Баща ми и братята ми, стадата им и добитъкът им, и всичко що имат дойдоха от Ханаанската земя, и сега са в Гесенската земя.
\par 2 И взе петима от братята си та ги представи на Фараона.
\par 3 И Фараон рече на братята му: Що е занятието ви? А те казаха на Фараона: Ние, слугите ти, и бащите ни, сме овчари.
\par 4 Рекоха още на Фараона: Дойдохме да поживеем в тая земя, защото няма пасище за стадата на слугите ти, понеже гладът се засили в Ханаанската земя; затова, молим ти се нека живеят слугите ти в Гесенската земя.
\par 5 И Фараон говори на Иосифа, казвайки: Баща ти и братята ти дойдоха при тебе;
\par 6 Египетската земя е пред тебе; настани баща си и братята си в най-добрата местност от земята; нека живеят в Гесенската земя. И ако знаеш, че някои от тях са способни мъже, постави ги надзиратели над моя добитък.
\par 7 След това Иосиф въведе баща си Якова и го представи на Фараона; и Яков благослови Фараона.
\par 8 И Фараон рече на Якова: Колко е числото на годините на живота ти?
\par 9 И Яков каза на Фараона: Числото на годините на пришелствуването ми е сто и тридесет години; малко и зло е било числото на годините на живота ми, и не достигна до числото на годините на живота на бащите ми, във времето на тяхното пришелствуване.
\par 10 И като благослови Фараона, Яков излезе от Фараоновото присъствие.
\par 11 Тогава Иосиф настани баща си и братята си,като им даде имот в Египетската земя, в най-добрата местност от земята в Рамесийската земя, според както заповяда Фараон.
\par 12 И Иосиф хранеше баща си, братята си и целия си бащин дом с хляб, според големината на челядите им.
\par 13 А нямаше хляб по цялата земя, защото гладът дотолкова се засилваше, щото Египетската земя и Ханаанската земя се изнуриха от глада.
\par 14 И Иосиф прибра всичките пари, които се намираха в Египетската и Ханаанската земя, за житото, което купуваха; и Иосиф донесе парите във Фараоновия дом.
\par 15 И като се свършиха всичките пари в Египетската земя и Ханаанската земя, всичките египтяни дойдоха при Иосифа и рекоха: Дай ни хляб; защо да умираме пред тебе, понеже се свършиха парите ни ?
\par 16 А Иосиф рече: Ако парите ви са се свършили, докарайте добитъка си, и ще ви дам хляб срещу добитъка ви.
\par 17 И тъй, докарваха добитъка си на Иосифа; и Иосиф им даваше хляб срещу конете, стадата, добитъка и ослите им; и така, през оная година той ги прехрани с хляб срещу всичкия им добитък.
\par 18 И като се измина оная година, дойдоха при него и другата година и му казаха: Няма да скрием от господаря си, че всичките ни пари се свършиха; и добитъкът стана на господаря ни; не остана друго пред господаря ни освен телата ни и земята ни.
\par 19 Защо да измрем пред очите ти и земята ни да запустее ? Купи нас и земята ни за хляб, и ние ще бъдем слуги на Фараона, и земята ни за него ще ражда . Дай ни и семе, за да останем живи да не умрем, и да не запустее земята.
\par 20 И тъй, Иосиф закупи за Фараона цялата Египетска земя, защото египтяните продаваха всеки нивата си, понеже гладът се засилваше над тях; така земята стана Фараонова.
\par 21 И той пресели населението в градовете от единия край до другия край на Египетските предели.
\par 22 Само земята на жреците не закупи, защото жреците имаха определен от Фараона дял и се прехранваха от дела, който Фараон им даде; затова не продадоха земята си.
\par 23 Тогава Иосиф каза на людете: Ето, днес закупих вас и земята ви за Фараона; на ви семе та засейте земята.
\par 24 Затова във време на беритбата ще давате петата част на Фараона, а четирите части ще бъдат за вас да засеете нивите, и за храна за вас и за домашните ви, и за храна на децата ви.
\par 25 И те рекоха: Ти опази живота ни; нека придобием благоволението на господаря си, и ще бъдем слуги на Фараона.
\par 26 И Иосиф постави това за повеление в Египетската земя, както е и до днес, да се дава петата част на Фараона; само земята на жреците единствено не стана Фараонова.
\par 27 И тъй, Израил се засели в Египетската земя, в Гесенската страна, гдето придобиваха имения, наплодяваха се, и твърде се размножаваха.
\par 28 И Яков живя седемнадесет години в Египетската земя, така че числото на годините на Яковия живот стана сто четиридесет и седем години.
\par 29 А като наближи времето, когато Израил трябваше да умре, повика сина си Иосифа и му каза: Ако съм придобил твоето благоволение, моля, тури ръката си под бедрото ми, и закълни ми се, че ще ми покажеш тая благост и вярност: да не ме погребеш в Египет,
\par 30 но когато почина с бащите си, да ме пренесеш от Египет и да ме погребеш в тяхната гробница. А той рече: Ще направя според както си казал.
\par 31 И рече Яков : Закълни ми се; и той му се закле. Тогава Израил се поклони върху възглавницата на леглото.

\chapter{48}

\par 1 След това, известиха на Иосифа: Ето, баща ти е болен. и той поведе със себе си двата си сина, Манасия и Ефрема, да отидат при него .
\par 2 И известиха на Якова, казвайки: Ето, син ти Иосиф иде при тебе. Тогава Израил събра силите си и седна на леглото.
\par 3 И Яков каза на Иосифа: Бог Всемогъщи ми се яви в Луз, в Ханаанската земя, и благослови ме, като ми каза:
\par 4 Ето, Аз ще те наплодя, ще те размножа, и ще направя да произлязат много народи от теб : и ще дам тая земя на потомството ти след теб за всегдашно притежание.
\par 5 И сега, двата ти сина, които ти се родиха в Египетската земя, преди да дойда аз при тебе в Египет, са мои; Ефрем и Манасия ще бъдат мои също, както Рувим и Симеон.
\par 6 А чадата, които родиш подир тях, ще бъдат твои, а колкото за наследството си, ще се наричат с името на тия свои братя.
\par 7 А когато се връщах от Падан, умря ми Рахил на пътя на Ханаанската земя, като оставаше едно малко разстояние да стигнем до Ефрата; и погребах я там, край пътя за Ефрата, (който е Витлеем).
\par 8 И като съгледа Иосифовите синове, рече Израил: Кои са тия?
\par 9 А Иосиф каза на баща си: Тия са синовете, които Бог ми даде тука. А той рече: Моля, доведи ги при мене, за да ги благословя.
\par 10 А очите на Израиля бяха помрачени от старост, та не можеше да вижда. И тъй, Иосиф ги приближи при него; а той ги целуна и ги прегърна.
\par 11 И рече Израил на Иосифа: Не се надявах да видя лицето ти; но, ето, Бог ми показа и потомството ти.
\par 12 И Иосиф, като ги махна от между колената си, поклони се с лицето до земята.
\par 13 После Иосиф ги взе двамата, Ефрема с дясната си ръка към лявата на Израил, а Манасия с лявата си ръка към дясната на Израил, та ги доведе при него.
\par 14 А Израил простря дясната си ръка и я възложи на главата на Ефрема, който беше по-младият, а лявата ръка на Манасиевата глава, като нарочно кръстоса ръцете си; (защото Манасия беше първородният).
\par 15 И като благослови Иосифа рече: Бог, пред Когото ходеха бащите ми Авраам и Исаак, Бог, Който ме е пасъл през целия ми живот до тоя ден;
\par 16 ангелът, който ме избавя от всяко зло, нека благослови момчетата; и нека се наричат с моето име и с името на бащите ми Авраама и Исаака; и нека нарастат в множество всред земята.
\par 17 Но като видя Иосиф, че баща му възложи дясната си ръка на Ефремовата глава, не одобри; и дигна ръката на баща си, за да я премести от Ефремовата глава на Манасиевата глава.
\par 18 И Иосиф рече на баща си: Не така, тате, защото ето първородният, възложи дясната си ръка на неговата глава.
\par 19 Но баща му отказа, като рече: Зная синко, зная; и той ще стане племе; и той ще бъде велик; но по-младият му брат ще бъде по-голям от него, и потомството му ще стане множество народи.
\par 20 И тъй, в същия ден ги благослови, казвайки: С твоето име Израил ще благославя, като казва: Бог да те направи като Ефрема и като Манасия!- като постави Ефрема пред Манасия.
\par 21 След това Израил рече на Иосифа: Ето, аз умирам; но Бог ще бъде с вас, и ще ви върне пак в отечеството ви.
\par 22 Впрочем, аз на тебе давам един дял повече отколкото на братята ти, който дял взех от ръцете на аморейците със сабята си и с лъка си.

\chapter{49}

\par 1 Тогава Яков повика синовете си и рече: Съберете се, за да ви известя какво ще ви се случи в следващите дни: -
\par 2 Съберете се и слушайте, синове Яковови, И послушайте Израиля, баща си.
\par 3 Рувиме, ти си първородният мой, мощта моя, и първият плод на силата ми, Превъзходен по достойнство, и превъзходен по сила.
\par 4 Изврял си като вода; не ще имаш превъзходството, Защото си се качил на леглото на баща си, И тогава си го осквернил. На леглото ми се е качил!
\par 5 Симеон и Левий са братя; Сечива насилствени са ножовете им.
\par 6 В съвета им да не участвуваш, душе моя; Към събранието им да се не присъединиш, славо моя, Защото в гнева си убиха човеци И в упорството си прерязаха жилите на волове.
\par 7 Проклет гневът им, защото беше свиреп, И яростта им, защото бе жестока! Ще ги разделя в Якова, И ще ги разпръсна в Израиля.
\par 8 Юдо, тебе ще похвалят братята ти; Ръката ти ще бъде на врата на неприятелите ти; Синовете на баща ти ще ти се кланят.
\par 9 Млад лъв е Юда; От плячка, сине мой, си се въздигнал; Легнал и разпрострял се е като лъв И като лъвица; кой ще го възбуди?
\par 10 Не ще липсва скиптър от Юда, Нито управителев жезъл отсред нозете му, Докле дойде Сило; И нему ще се покоряват племената.
\par 11 Като връзва за лозата оселчето си И за отборната лоза жребчето на ослицата си, Ще опере с вино дрехата си, И с кръвта на гроздето облеклото си.
\par 12 Очите му ще червенеят от вино. И зъбите му ще белеят от мляко.
\par 13 Завулон ще обитава край брега на езерото, И ще бъде пристанище на кораби; И ще граничи със Сидон.
\par 14 Исахар е як осел, Който се е проснал между кошарите;
\par 15 И като видя, че мястото беше добро за почивка, И че страната беше приятна, Подложи плещите си за товар, И стана слуга подчинен.
\par 16 Дан ще съди людете си, Като едно от Израилевите племена.
\par 17 Дан ще бъде змия на пътя, Ехидна на пътеката, Която хапе петите на коня, Тъй че ездачът му пада назад.
\par 18 Твоето спасение чаках, Господи.
\par 19 Гада ще разбият разбойници; Но и той ще разбие петите им.
\par 20 Хлябът от Асира ще бъде изряден; И той ще доставя царски сладкиши.
\par 21 Нефталим е елен пуснат, Който говори угодни думи.
\par 22 Иосиф е плодоносна вейка, Плодоносна вейка край извор; Клончетата й се простират по стената.
\par 23 Стрелците го огорчиха, И стреляха по него, и преследваха го;
\par 24 Но лъкът му запази якостта си. И мишците на ръцете му се укрепиха, Чрез ръцете на Силния Яковов, - Отгдето е пастирът, Израилевият камък, -
\par 25 Чрез Бога на отца ти, Който ще ти помага. И чрез Всесилния, Който ще те благославя. С небесни благословения от горе, С благословения на бездната, която лежи отдолу С благословения на съсците и на утробата.
\par 26 Благословенията на отца ти превишаваха, Благословенията на праотците ми, До високите върхове на вечните планини; Те ще бъдат на Иосифовата глава, И на темето на превъзходния между братята си.
\par 27 Вениамин е вълк грабител; Заран ще пояжда лов. А вечер ще дели корист.
\par 28 Всички тия са дванадесетте Израилеви племена; и това е, което изговори баща им като ги благослови; всеки благослови според благословението, което му се падаше .
\par 29 Още им заръча, като им каза: Аз се прибирам при людете си; погребете ме с бащите ми в пещерата, която е в нивата на хетееца Ефрона.
\par 30 В пещерата, която е в нивата Махпелах, която е срещу Мамврий в Ханаанската земя, която пещера Авраам купи заедно с нивата от хетееца Ефрона за собствено гробище.
\par 31 Там погребаха Авраама и жена му Сара; там погребаха Исаака и жена му Ревека; и там погребах аз Лия.
\par 32 Нивата и пещерата, която е в нея, бяха купени от хетейците.
\par 33 А като свърши Яков поръчките към синовете си, притегли нозете си в леглото и издъхна, и се прибра при людете си.

\chapter{50}

\par 1 Тогава Иосиф падна на лицето на баща си, плака върху него и го целува.
\par 2 И Иосиф заповяда на служащите нему лекари да балсамират баща му; и лекарите балсамираха Израиля,
\par 3 като работиха над него напълно четиридесет дена; защото толкова е пълното време за балсамиране; и египтяните го оплакваха седемдесет дена.
\par 4 И като преминаха дните на жалейката за него, Иосиф говори на Фараоновия дом, казвайки: Ако съм придобил вашето благоволение, говорете, моля ви се, в ушите на Фараона и речете:
\par 5 Баща ми ме закле, като каза: Виж, аз умирам; в гроба, който се приготвих в Ханаанската земя, там да ме погребеш. Сега, прочее нека отида, моля, да погреба баща; си и ще се върна.
\par 6 А рече Фараон: Иди, погреби баща си, според както те е заклел.
\par 7 И така, Иосиф отиде да погребе баща си; и с него отидоха всичките служители на Фараона, старейшините на дома му, и всичките старейшини на Египетската земя,
\par 8 също и целият дом на Иосифа, братята му, и бащиният му дом; само челядите си, стадата си и добитъка си оставиха в Гесенската земя.
\par 9 Отидоха с него и колесници и конници, тъй че стана много голямо шествие.
\par 10 И когато пристигнаха до Атадовото гумно, което е оттатък Иордан, там ридаха твърде много и силно; и Иосиф направи за баща си седемдневна жалейка.
\par 11 А ханаанците, тамошните жители, като видяха жалейката при Атадовото гумно, рекоха: египтяните имат голяма жалейка; затова мястото , което е оттатък Иордан, се наименува Авел-мисраим.
\par 12 Тогава синовете на Израиля му сториха, според както им беше заповядал;
\par 13 защото синовете му го пренесоха в Ханаанската земя и го погребаха в пещерата на нивата Махпелах, срещу Мамврий, която пещера Авраам купи заедно с нивата за собствено гробище от хетееца Ефрон.
\par 14 И Иосиф, като погреба баща си, върна се в Египет, той и братята му и всички, които бяха отишли с него да погребат баща му.
\par 15 А като видяха Иосифовите братя, че умря баща им, думаха си: Може Иосиф да ни намрази и да ни възвърне жестоко всичкото зло, що сме му сторили.
\par 16 Затова пратиха на Иосифа да му кажат: Преди да умре баща ти ни заповяда с думите:
\par 17 Така да кажете на Иосифа: Прости, моля ти се, престъплението на братята си и греха им за злото, което ти сториха. И тъй, прости молим ти се, престъплението на слугите на бащиния ти Бог. А Иосиф се разплака, когато му говориха.
\par 18 После отидоха и братята му та паднаха пред него и рекоха: Ето, ние сме ти слуги.
\par 19 А Иосиф им каза: Не бойте се; нима съм аз вместо Бога?
\par 20 Вие наистина намислихте зло против мене; но Бог го намисли за добро, за да действува така, щото да спаси живото на много люде, както и стана днес.
\par 21 Прочее, не бойте се; аз ще храня вас и челядите ви. И утеши ги и говори им любезно.
\par 22 Така Иосиф остана да живее в Египет, той и бащиният му дом. И Иосиф живя сто и десет години;
\par 23 и Иосиф видя Ефремови чада от третия род; също и децата на Манасиевия син Махира се родиха на Иосифовите колене.
\par 24 След това, Иосиф рече на братята си: Аз умирам; а Бог непременно ще ви посети, и ще ви заведе от тая земя в земята, за която се е клел на Авраама, Исака и Якова.
\par 25 И Иосиф закле потомците на Израиля, като рече: Понеже Бог непременно ще ви посети, то вие да изнесете костите ми от тука.
\par 26 И тъй, Иосиф умря, на възраст сто и десет години; и балсамираха го и положиха го в ковчег в Египет.

\end{document}