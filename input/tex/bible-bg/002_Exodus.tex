\begin{document}

\title{Изход}


\chapter{1}

\par 1 Ето имената на синовете на Израиля, които дойдоха в Египет заедно с Якова; всеки дойде с челядта си:
\par 2 Рувим, Симеон, Левий и Юда,
\par 3 Исахар, Завулон, и Вениамин,
\par 4 Дан и Нефталим, Гад и Асир.
\par 5 Всичките човеци, които излязоха из чреслата на Якова бяха седемдесет души; а Иосиф беше вече в Египет.
\par 6 И умря Иосиф, всичките му братя, и цялото онова поколение.
\par 7 А потомците на Израиля се наплодиха и размножиха, увеличиха се и толкова много се засилиха, щото земята се изпълни от тях.
\par 8 Тогава се възцари над Египет нов цар, който не познаваше Иосифа.
\par 9 И той рече на людете си: Вижте, тия люде, израилтяните са по-много и по-силни от нас;
\par 10 елате, да постъпим разумно спрямо тях, за да се не размножават, да не би, в случай на война, да се съединят и те с неприятелите ни, да воюват против нас и да си отидат от земята.
\par 11 Затова поставиха над тях настойници, които да ги измъчват с тежък труд; и те съзидаха на Фараона Питом и Рамесий, градове за житници.
\par 12 Но колкото повече ги измъчваха, толкова повече те се размножаваха и се разширяваха, така че египтяните се отвращаваха от израилтяните.
\par 13 Затова египтяните караха жестоко израилтяните да работят;
\par 14 огорчаваха живота им с тежка работа, да правят кал и кирпичи и да вършат всякакъв вид полска работа, всичките работи, с които ги караха да работят, бяха твърде тежки.
\par 15 При това, египетският цар заръча на еврейските баби, (от които едната се именуваше Шифра, а другата Фуа), като им каза:
\par 16 Когато бабувате на еврейките и видите, че раждат , ако родят син, убивайте го, но ако родят дъщеря, тогава нека живее.
\par 17 Но бабите се бояха от Бога, та не правеха каквото им заръча египетския цар, а оставяха живи мъжките деца.
\par 18 Тогава египетския цар повика бабите и им рече: Защо правите това та оставяте живи мъжките деца?
\par 19 И бабите казаха на Фараона: Понеже еврейките не са като египтянките; защото са пъргави и раждат преди да дойдат бабите при тях.
\par 20 Затова Бог правеше добро на бабите. А людете се размножаваха и засилваха твърде много.
\par 21 И понеже бабите се бояха от Бога, Той им направи домове.
\par 22 Тогава Фараон заръча на всичките си люде, като каза: Всеки син, който се роди на евреите, хвърляйте го в Нил, а всяка дъщеря оставяйте жива.

\chapter{2}

\par 1 В това време отиде някой си човек от Левиевия дом та взе за жена една от Левиевите жени.
\par 2 И жената зачна и роди син; и като видя, че беше красиво дете, кри го за три месеца.
\par 3 Но понеже не можеше вече да се крие, взе му рогозен ковчежец и намаза го със смола и катран, тури детето в него и го положи в тръстиката край брега на реката.
\par 4 А сестра му стоеше от далеч, за да види какво ще му се случи.
\par 5 И Фараоновата дъщеря слезе да се окъпе в реката, а слугините й ходеха по брега на реката; и когато съгледа ковчежеца в тръстиката, тя прати слугинята си за го донесе.
\par 6 Като го отвори, видя детето, и, ето малкото плачеше; и пожали го и каза: От еврейските деца е това.
\par 7 Тогава сестра му рече на Фараоновата дъщеря: Да отида ли да ти повикам доилка от еврейките, за да ти дои детето?
\par 8 И Фараоновата дъщеря й рече: Иди. И тъй, момичето отиде та повика майката на детето.
\par 9 И Фараоновата дъщеря й рече: Вземи това дете и ми го отдой, и аз ще ти заплатя. И жената взе детето и го доеше.
\par 10 А когато порасна детето, донесе го на Фараоновата дъщеря, и то стана неин син. И наименува го Моисей: Понеже, каза тя, извлякох го от водата.
\par 11 А във времето, когато Моисей порасна, излезе при братята си и гледаше теглилата им. И видя някой си египтянин, че биеше един евреин, един от братята му;
\par 12 и като поразгледа насам натам и видя, че няма никой, уби египтянина и го скри в пясъка.
\par 13 Излезе и на другия ден, и ето, двама евреи се препираха; и каза на обидника: Защо биеш другаря си?
\par 14 А той рече: Кой те постави началник и съдия над на? Искаш ли и мене да убиеш, както уби египтянина? Тогава Моисей се уплаши и си каза: Явно е, че това нещо стана известно.
\par 15 А като чу това Фараон, търсеше случай да убие Моисея; но Моисей побягна от Фараоновото лице и се засели в Мадиамската земя. И седна до един кладенец.
\par 16 А Мадиамският жрец имаше седем дъщери, които дойдоха и наляха вода и напълниха коритата, за да напоят бащините си овци.
\par 17 Но овчарите дойдоха и ги изпъдиха; а Моисей стана та им помогна, и напои овците им.
\par 18 И като дойдоха при баща си Рагуила, той им рече: Как дойдохте днес толкова скоро.
\par 19 А те казаха: Един египтянин ни избави от ръцете на овчарите, при това и вода ни наля та напои овците.
\par 20 И той рече на дъщерите си: А где е той? защо оставихте човека? повикайте го да яде хляб.
\par 21 И Моисей склони да живее при човека, който е даде на Моисея дъщеря си Сепфора за жена.
\par 22 И тя роди син; и той го наименува Герсам, защото думаше: Пришелец станах в чужда земя.
\par 23 Подир дълго време египетският цар умря; а израилтяните пъшкаха под робството и извикаха: и викат им от робството стигна до Бога.
\par 24 Бог чу пъшканията им; и Бог спомни завета си с Авраама, с Исаака, и с Якова.
\par 25 И Бог погледна на израилтяните, и Бог разбра положението им.

\chapter{3}

\par 1 А Моисей пасеше овците на тъста си Иотора, мадиамския жрец; и като докара овците в задната страна на пустинята, дойде на Божията планина Хорив.
\par 2 И ангел Господен му се яви в огнен пламък изсред една къпина; и Моисей погледна, и ето къпината гореше в огън, а къпината не изгаряше.
\par 3 И Моисей си рече: Да свърна и да погледам това велико явление, защо къпината не изгаря.
\par 4 А като видя Господ, че свърна да прегледа, Бог го извика изсред къпината и рече:Моисее, Моисее! И Той каза: Ето ме.
\par 5 И рече: Да се не приближиш тука; изуй обущата от нозете си, защото мястото, на което стоиш, е света земя.
\par 6 Рече още: Аз съм Бог на баща ти, Бог Авраамов, Бог Исааков, и Бог Яковов. А Моисей затули лицето си, защото се боеше да погледне към Бога.
\par 7 И рече Господ: Наистина видях страданието на людете Ми, които са в Египет, и чух вика им поради настойниците им; защото познах неволите им.
\par 8 И слязох за да ги избавя от ръката на египтяните, и да ги заведа от оная земя, в земя добра и пространна, в земя гдето текат мляко и мед, в земята на ханаанците, хетейците, аморейците, ферезейците, евейците и евусейците.
\par 9 И сега, ето, вика на израилтяните стигна до Мене; още видях и притеснението, с което ги притесняват египтяните.
\par 10 Ела, прочее, сега, и ще те изпратя при Фараона, за да изведеш людете Ми израилтяните из Египет.
\par 11 А Моисей рече Богу: Кой съм аз та да ида при Фараона, и да изведа израилтяните из Египет?
\par 12 И рече му Бог: Аз непременно ще бъда с теб, и ето ти знака, че аз те изпратих; когато изведеш людете из Египет, ще послужите Богу на тая планина.
\par 13 А Моисей рече Богу: Ето, когато отида при израилтяните и им река: Бог на бащите ви ме изпрати при вас, и те ми кажат: Как Му е името? що да им кажа?
\par 14 И Бог рече на Моисея: Аз съм Оня, Който съм. Рече още: Така ще кажеш на израилтяните, Оня, Който съм, ме изпрати при вас.
\par 15 При това, Бог рече още на Моисея: Така ще кажеш на Израилтяните: Господ Бог на бащите ви, Бог Авраамов, Бог Исааков, и Бог Яковов, ме изпрати при вас. Това е Името Ми до века, и това е Името Ми из род в род.
\par 16 Иди, събери Израилевите старейшини и кажи им: Господ Бог на бащите ви Бог Авраамов, Исааков, и Яковов, ми се яви и рече: Наистина ви посетих и видях онова, което ви правят в Египет;
\par 17 и рече: Отсред страданието ви в Египет ще ви заведа в земята на ханаанците, на хетейците, на аморейците, на ферезейците, на евейците и на евусейците, в земята гдето текат мляко и мед.
\par 18 И те ще послушат думите ти; и ще отидеш ти с Израилевите старейшини, при египетския цар, и ще му речете: Иеова, Бог на евреите, ни срещна; и тъй, остави ни, молим ти се, да отидем на тридневен път в пустинята, за да принесем жертва на Иеова нашия Бог.
\par 19 Но зная аз че, египетският цар не ще ви остави да отидете, нито даже под силна ръка.
\par 20 А като издигна ръката Си, ще поразя Египет с всичките Мои чудеса, които ще направя всред него; подир това той ще ви пусне.
\par 21 И ще дам на тия люде благоволението на египтяните, та, когато тръгнете, няма да отидете празни;
\par 22 но всяка жена ще поиска от съседката си от квартирантката си сребърни вещи, златни вещи и облекла; и ще облечете с тях синовете и дъщерите си, и ще оберете египтяните.

\chapter{4}

\par 1 А Моисей в отговор каза: Но, ето, няма да ме повярват, нито да послушат думите ми, защото ще рекат: Не ти се е явил Господ.
\par 2 Тогава Господ ме каза: Що е това в ръката ти? А той рече: Жезъл.
\par 3 И каза: Хвърли го на земята. И той го хвърли на земята, и стана змия; и Моисей побягна от нея.
\par 4 Но Господ каза на Моисея: Простри ръката си и хвани я за опашката; (и той простря ръката си и я хвана, а тя стана жезъл в ръката му);
\par 5 стори това, за да повярват, че ти се яви Господ, Бог на бащите им, Бог Авраамов, Бог Исааков и Бог Яковов.
\par 6 При това му рече Господ: Тури си сега ръката в пазухата. И той си тури ръката в пазухата; а като я извади, ето ръката му прокажена, бяла като сняг.
\par 7 Тогава рече Господ: Тури си пак ръката в пазухата; (и той пак си тури ръката в пазухата, ето, че бе станала пак като останалата му плът);
\par 8 и продължи Господ: Ако не те повярват, нито послушат гласа на първото знамение, ще повярват, поради гласа на второто знамение.
\par 9 Но ако не повярват и поради двете тия знамения, нито послушат думите ти, тогава да вземеш от водата на реката, и да я излееш на сушата; и водата, която извадиш из реката, ще стане кръв на сушата.
\par 10 А Моисей рече Господу: Моля ти се, Господи, аз не съм красноречив, нито до сега, нито откак си почнал да говориш на слугата си, а мъчно говоря и съм тежкоезичен.
\par 11 Но Господ му рече: Кой е направил човешката уста? или кой прави човек да е ням или глух, да има зрение или да е сляп? не Аз ли Господ?
\par 12 Иди прочее; и Аз ще бъда с устата ти, и ще те науча какво да говориш.
\par 13 Тогава каза Моисей: Моля ти се, Господи, прати чрез онзи, чрез когото искаш да пратиш.
\par 14 И гневът Господен пламна против Моисея, и рече: Нямаш ли брат левиецът Аарон? Зная, че той може да говори добре. А и, ето, той излиза да те посрещне и, когато те види ще се зарадва сърдечно.
\par 15 Ти говори нему и тури думите в устата му; и Аз ще бъда с твоите уста и с неговите уста, и ще ви науча що трябва да правите.
\par 16 Вместо тебе нека говори той на людете; той ще бъде на тебе вместо уста, а ти ще бъдеш на него вместо Бога.
\par 17 И вземи в ръката си тоя жезъл, с който ще вършиш знаменията.
\par 18 Тогава Моисей отиде и се върна при тъста си Иотор и му рече: Моля ти се, нека отида и се завърна при братята си, които са в Египет, та да видя живи ли са още. И Иотор рече на Моисея: Иди с мир.
\par 19 А в Мадиам Господ рече на Моисея: Иди, върни се в Египет, защото измряха всички ония, които искаха живота ти.
\par 20 И тъй Моисей взе жена си и синовете си, качи ги на осел, и тръгна да се върне в Египетската земя; Моисей взе и Божия жезъл в ръката си.
\par 21 И Господ каза на Моисея: Когато се завърнеш в Египет, внимавай да вършиш пред Фараона всичките чудеса, които дадох в ръката ти; но Аз ще закоравя сърцето му, и той няма да пусне людете.
\par 22 А ти кажи на Фараона: Така говори Иеова: Израил ми е син, първородният ми;
\par 23 и казвам ти: Пусни син Ми да Ми послужи; но ако откажеш да го пуснеш, ето, Аз ще заколя твоя син, първородния ти.
\par 24 А по пътя, в гостилницата, Господ посрещна Моисея и искаше да го убие.
\par 25 Тогава Сепфора взе кремък та обряза краекожието на сина си, допря го до нозете му, и рече: Наистина ти ми си кървав младоженец.
\par 26 След това Господ се оттегли от него. Тогава каза тя, поради обрязването: Ти ми си кървав младоженец.
\par 27 А Господ беше казал на Аарона: Иди в пустинята да посрещнеш Моисея. И той, като беше отишъл, посрещна го в Божията планина и целува го.
\par 28 И Моисей извести на Аарона всичките думи, с които Господ го беше изпратил, и всичките знамения, за които му беше заръчал.
\par 29 Тогава Моисей и Аарон отидоха та събраха всичките старейшини на израилтяните;
\par 30 и Аарон каза всичките думи, които Господ беше говорил на Моисея, и извърши знаменията пред людете.
\par 31 И людете повярваха; и когато чуха, че Господ посетил израилтяните, и че погледна на неволята им, наведоха главите си и се поклониха.

\chapter{5}

\par 1 След това дойдоха Моисей и Аарон и казаха на Фараона: Така говори Иеова, Израилевият Бог: Пусни людете Ми, за да Ми пазят празник в пустинята.
\par 2 Но Фараон рече: Кой е Иеова та да послушам гласа Му и да пусна Израиля? Не познавам Иеова, и затова няма да пусна Израиля.
\par 3 А те рекоха: Бог на евреите ни срещна. Молим ти се, нека отидем на тридневен път в пустинята, за да принесем жертва на Иеова нашия Бог, да не би да ни нападне с мор или с нож.
\par 4 Но египетския цар им каза: Защо, Моисее и Аароне, отвличате людете от работите им? Идете на определените си работи.
\par 5 Рече още Фараон: Ето, людете на земята са сега много, а вие ги правите да оставят определените си работи.
\par 6 И в същия ден Фараон заповяда на настойниците и на надзирателите на людете, казвайки:
\par 7 Не давайте вече, както до сега плява на тия люде, за да правят тухли; нека идат сами и си събират плява.
\par 8 Но колкото тухли са правили до сега , същото число изисквайте от тях; с нищо да го не намалите; защото остават без работа и затова викат, казвайки: Нека отидем да принесем жертва на нашия Бог.
\par 9 Нека се възлагат още по-тежки работи на тия човеци, за да се трудят с тях; и да не внимават на лъжливите думи.
\par 10 И тъй, настойниците и надзирателите на людете излязоха и говориха на людете казвайки: Така казва Фараон: Не ви давам плява.
\par 11 Вие сами идете та си събирайте плява, гдето можете да намерите; но нищо няма да се намали от работата ви.
\par 12 Затова, людете се разпръснаха по цялата Египетска земя да събират слама вместо плява.
\par 13 А настойниците настояваха, като казваха: Изкарвайте работата си, определената си ежедневна работа, както когато имаше плява.
\par 14 И надзирателите, поставени над израилтяните от Фараоновите настойници, бяха бити, като им казаха: Защо не изкарахте, и вчера и днес, определеното на вас число тухли, както по-напред?
\par 15 Тогова надзирателите на израилтяните дойдоха и извикаха на Фараона, казвайки: Защо постъпвате така със слугите си?
\par 16 Плява не се дава на слугите ти; а казват ни: Правете тухли; и, ето, слугите ти сме бити; а вината е на твоите люде.
\par 17 Но той каза: Без работа останахте, без работа; затова казвате: Нека отидем да принесем жертва Господу.
\par 18 Идете сега та работете, защото плява няма да ви се даде, но ще давате определеното число тухли.
\par 19 И надзирателите на израилтяните видяха, че положението им е лошо, когато им се рече: Нищо не намалявайте от определеното на вас за всеки ден число тухли.
\par 20 И като излизаха от Фараоновото присъствие, срещнаха Моисея и Аарона, които се намираха на пътя;
\par 21 и рекоха им: Господ да погледне на вас и да съди, защото вие ни направихте омразни на Фараона и на слугите му, и турихте меч в ръката им, за да ни избият.
\par 22 Тогава Моисей се върна при Господа и рече: Господи, защо си зле постъпил спрямо тия люде? защо си ме изпратил?
\par 23 Защото откак дойдох при Фараона да говоря, в Твое име, той е зле постъпвал спрямо тия люде; а Ти никак не си избавил людете Си?

\chapter{6}

\par 1 А Господ каза на Моисея: Сега ще видиш, какво ще сторя на Фараона: защото под силна ръка ще ги пусне и под силна ръка ще ги изпъди из земята.
\par 2 Бог говори още на Моисея казвайки: Аз съм Иеова.
\par 3 Явих се на Авраама, на Исаака и на Якова с името Бог Всемогъщи, но не им бях познат с името Си Иеова.
\par 4 Поставих още и завета Си с тях, да им дам Ханаанската земя, земята в която бяха пришелци при пребиванията си.
\par 5 При това чух пъшкането на израилтяните поради робството, което им налагат египтяните, и си спомних завета.
\par 6 Затова, кажи на израилтяните: Аз съм Иеова; ще ви изведа изпод товарите на египтяните, ще ви избавя от робството ви под тях, и ще ви откупя с издигната мишца и с велики съдби.
\par 7 Ще ви взема за Мои люде, и ще бъда ваш Бог; и ще познаете, че Аз съм Господ, вашият Бог, който ви извеждам изпод египетските товари.
\par 8 И ще ви въведа в земята, за която съм се клел, че ще я дам на Аарона, на Исаака и на Якова; и ще я дам на вас за наследство. Аз съм Иеова.
\par 9 И Моисей говори така на израилтяните; но поради утеснението на душите си и поради жестокото си робуване те не послишаха Моисея.
\par 10 Подир това Господ говори на Моисея, казвайки:
\par 11 Влез при египетския цар Фараона и му кажи да пусне израилтяните из земята си.
\par 12 Но Моисей говори пред Господа, казвайки: Ето, израилтяните на ме послушаха; тогава как ще ме послуша Фараон, като съм с вързани устни?
\par 13 Затова Господ говори на Моисея и на Аарона, та им даде поръчки до израилтяните и до египетския цар Фараон, да изведат израилтяните из Египетската земя.
\par 14 Ето родоначалниците на бащините домове на Аарона и Моисея; синове на Рувима, Израилевия първороден: Енох, Фалу, Есрон и Хармий; тия са Рувимови семейства.
\par 15 Симеонови синове: Емуил, Ямин, Аод, Яхин и Сохар, и Саул, син на ханаанка; тия са Симеонови семейства.
\par 16 Имената на Левиевите синове, според поколенията им са тия: Гирсон, Каат и Мерарий; и годините на Левиевия живот станаха сто тридесет и седем години.
\par 17 Синовете на Гирсон са Ливний и Семей, по семействата им.
\par 18 И синовете на Каата са: Амрам, Исаак, Хеврон и Озиил; и годините на Каатовия живот станаха сто тридесет и три години.
\par 19 И синовете на Мерария са: Маалий и Мусий. Тия са Левиевите семейства според поколенията им.
\par 20 И Амрам взе за жена леля си Иохаведа, която му роди Аарона и Моисея; и годините на Амрамовия живот станаха сто тридесет и седем години.
\par 21 И синовете на Исаара са: Коре, Нефег и Зехрий.
\par 22 И синовете на Озиила са: Мисаил, Елисафан и Ситрий.
\par 23 И Аарон взе за жена Елисавета, Аминадавова дъщеря, Наасонова сестра, която му роди Надава, Авиуда, Елеазара и Итамара.
\par 24 И синовете на Корея са: Асир, Елкана и Авиасаф; тия са Кореевите семейства.
\par 25 И Аароновия син Елеазар взе за жена една от Футииловите дъщери, която му роди Финееса. Тия са родоначалниците на бащините домове на Левитите според семействата им.
\par 26 Тия са същите Аарон и Моисей, на които Господ рече: Изведете израилтяните из Египетската земя, според войнствата им.
\par 27 Тия са, които говориха на египетския цар Фараона, за да изведат израилтяните из Египет; тия са същите Моисей и Аарон.
\par 28 И в същия ден, когато говори Господ на Моисея в Египетската земя,
\par 29 Господ говори на Моисея, казвайки: Аз съм Иеова, кажи на Египетския цар Фараон всичко що ти казвам.
\par 30 А Моисей рече пред Господа: Ето, аз съм с вързани устни; и как ще ме послуша Фараон?

\chapter{7}

\par 1 Тогава Господ рече на Моисея: Ето, поставих те бог на Фараона; и брат ти Аарона ще ти бъде пророк.
\par 2 Ти ще казваш всичко, което ти заповядвам, а брат ти Аарон ще говори на Фараона да пусне израилтяните из земята си.
\par 3 Но Аз ще закоравя Фараоновото сърце, и ще умножа знаменията Си и чудесата Си в Египетската земя.
\par 4 Но понеже Фараон не ще ви послуша, Аз ще положа ръката Си на Египет, и с велики съдби ще изведа войнства, людете Си, израилтяните из Египетската Земя.
\par 5 И египтяните ще познаят, че Аз съм Господ, когато дигна ръката Си против Египет, и изведа израилтяните изсред тях.
\par 6 И тъй, Моисей и Аарон сториха така; според както Господ им заповяда, така сториха.
\par 7 А Моисей бе на осемдесет години, а Аарон на осемдесет и три години, когато говориха на Фараона.
\par 8 И Господ говори на Моисея и на Аарона, казвайки:
\par 9 Когато ви говори Фараон и каже: Покажете чудо в себеподкрепа, тогава кажи на Аарона: Вземи жезъла си и хвърли го пред Фараона, за да стане на змия.
\par 10 Тогава Моисей и Аарон влязоха при Фараона и сториха, според както заповяда Господ; Аарон хвърли жезъла си пред Фараона и пред слугите му, и жезълът стана змия.
\par 11 Но Фараон повика мъдреците и чародеите, та и те, египетските магьосници, сториха същото с баянията си.
\par 12 Защото хвърлиха, всеки от тях, жезъла си, и те станаха на змии; обаче Аароновият жезъл погълна техните жезли.
\par 13 А сърцето на Фараона се закорави, и той не ги послуша, според както Господ беше говорил.
\par 14 След това Господ рече на Моисея: Сърцето на Фараона се закорави толкова, щото той отказа да пусне людете.
\par 15 Иди утре при Фараона; ето той излезе да отиде при водата; а ти застани при брега на реката, за да го срещнеш, и вземи в ръката си жезъла, който се беше превърнал на змия.
\par 16 И кажи му: Господ, Бог, на евреите, ме изпрати при тебе, и казва: Пусни людете Ми, за да Ми послужат в пустинята; но, ето, до сега ти не послуша.
\par 17 Така казва Господ: От това ще познаеш, че Аз съм Господ; ето, с жезъла, който е в ръката ми, ще ударя върху водата, която е в реката, и тя ще се превърне на кръв.
\par 18 Рибите, които са в реката ще измрат, и реката ще се усмърди та египтяните ще се гнусят да пият вода от реката.
\par 19 Тогава Господ рече на Моисея: Кажи на Аарона: Вземи жезъла си и простри ръката си над египетските води, над реките им, над потоците им, над езерата им и над всичките им водни локви, за да станат кръв; и по цялата египетска земя ще има кръв, и в дървените и в каменните съдове.
\par 20 И Моисей и Аарон сториха според както Господ заповяда; и Аарон, като дигна жезъла, удари речната вода пред Фараона и пред слугите му; и всичката речна вода се превърна в кръв.
\par 21 И рибите, които бяха в реката, измряха; и реката се усмърдя, така щото египтяните на можеха да пият вода от реката; и кръвта се намираше по цялата Египетска земя.
\par 22 Но и египетските магьосници с баянията си сториха същото; затова, сърцето на Фараона се закорави, и той не ги послуша, според както беше казал Господ.
\par 23 И Фараон се обърна та си отиде у дома си, без да вземе присърце и това.
\par 24 А всичките египтяни копаха около реката, за да намерят вода за пиене, защото не можеха да пият речната вода.
\par 25 И язвата се продължаваше напълно седем дни, след като Господ удари реката.

\chapter{8}

\par 1 Тогава Господ рече на Моисея: Влез при Фараона и кажи му: Така казва Господ, пусни людете Ми, за да Ми послужат.
\par 2 Но ако откажеш да ги пуснеш, ето, Аз ще поразя всичките ти предели с жаби.
\par 3 Реката ще кипне с жаби, които като излизат ще наскачат в къщата ти, в спалнята ти, по леглото ти, в къщата на слугите ти, върху людете ти, в пещите ти и по нощвите ти.
\par 4 На тебе, на людете ти и на всичките ти слуги ще наскачат жабите.
\par 5 И тъй, Господ рече на Моисея: Кажи на Аарона: Простри ръката си с жезъла си над реките, над потоците и над езерата, та стори да наскачат жаби по Египетската земя.
\par 6 И Аарон простря ръката си над египетските води и възлязоха жабите и покриха Египетската земя.
\par 7 Но и магьосниците сториха същото с баянията си, и направиха да наскачат жаби по Египетската земя.
\par 8 Тогава Фараон повика Моисея и Аарона и рече: Помолете се Господу да махне жабите от мене и от людете ми; и ще пусна людете ви, за да пожертвуват Господу.
\par 9 И Моисей рече на Фараона: Определи ми, кога да се помоля за тебе, за слугите ти и за людете ти, за да се изтребят жабите от тебе и от къщите ти, та да останат само в реката.
\par 10 Фараон каза: Утре. А той рече: Ще бъде според както си казал, за да познаеш, че няма никой подобен на нашия Бог.
\par 11 Жабите ще се махнат от тебе от къщите ти, от слугите ти и от людете ти; само в реката ще останат.
\par 12 Тогава Моисей и Аарон излязоха отпред Фараона; и Моисей викна към Господа относно жабите, които беше пратил върху Фараона.
\par 13 И Господ стори според както викна Моисей; жабите измряха от къщите, от дворовете и от нивите.
\par 14 И събраха ги на купове: и земята се усмърдя.
\par 15 А Фараон като видя, че му дойде облекчение, закорави сърцето си и не ги послуша според както Господ беше говорил.
\par 16 След това Господ каза на Моисея: Речи на Аарона: Простри жезъла си та да удари земната пръст, за да се превърне на въшки в цялата Египетска земя.
\par 17 И сториха така; Аарон простря ръката си с жезъла си и удари земната пръст, и явиха се по човеците и по животните; всичката земна пръст се превърна на въшки из цялата Египетска земя.
\par 18 И магьосниците работеха да вършат същото с баянията си, за да произведат въшки, но не можаха; а въшките бяха по човеците и по животните.
\par 19 Тогава рекоха магьосниците на Фараона: Божий пръст е това. Но сърцето на Фараона се закорави, и той не ги послуша, според както Господ беше говорил.
\par 20 После Господ рече на Моисея: Стани утре рано та застани пред Фараона (ето, той излиза да отиде при водата), и кажи му: Така казва Господ: Пусни людете Ми, за да Ми послужат.
\par 21 Защото, ако не пуснеш людете Ми, ето, ще изпратя рояци мухи върху тебе, върху слугите ти, върху людете ти и върху къщите ти; тъй щото къщите на египтяните и земята, на която живеят, ще се пълнят с рояци мухи.
\par 22 Но в оня ден Аз ще отделя Гесенската земя, в която живеят людете Ми, щото да няма там рояци мухи, за да познаят, че Аз съм Господ всред земята.
\par 23 Аз ще поставя преграда между Своите люде и твоите люде; утре ще стане това знамение.
\par 24 И Господ стори така; навлязоха мъчителни рояци мухи във Фараоновата къща и в къщите на слугите му и в цялата Египетска земя; земята се развали от рояците мухи.
\par 25 Тогава Фараон повика Моисея и Аарона и рече: Идете, принесете жертва на вашия Бог в тая земя.
\par 26 Но Моисей каза: Не е прилично да направим така, защото ние ще жертвуваме на Господа нашия Бог онова, от което египтяните се гнусят; ето, ако жертвуваме пред очите на египтяните онова, от което те се гнусят, няма ли да ни избият с камъни?
\par 27 Ще отидем на тридневен път в пустинята и ще жертвуваме на Господа нашия Бог, според както би ни казал.
\par 28 Тогава рече Фараон: Ще ви пусна да пожертвувате на Господа вашия Бог в пустинята, само да не отидете много далеч. Помолете се Богу за мене.
\par 29 И Моисей рече: Ето, аз излизам отпред тебе, и ще се помоля Господу да се махнат утре рояците мухи от Фараона, от слугите му и от людете му; но нека не следва Фараон да лъже и да не пуска людете ни да пожертвуват Господу.
\par 30 И тъй, Моисей излезе отпред Фараона и помоли се Господу.
\par 31 И Господ стори според както Моисей се молеше; той махна рояците мухи от Фараона, от слугите му и от людете му; не остана ни една.
\par 32 Но Фараон и тоя път закорави сърцето си и не пусна людете.

\chapter{9}

\par 1 Тогава Господ рече на Моисея: Влез у Фараона и кажи му: Така казва Господ, Бог на евреите, пусни людете Ми, за да ми послужат.
\par 2 Защото, ако откажеш да ги пуснеш, и око още ги държиш,
\par 3 ето, Господната ръка ще падне на добитъка ти, който е по полето, на конете, на ослите, на камилите, на говедата и на овците, с твърде тежък мор.
\par 4 И Господ ще постави преграда между Израилевия добитък и египетския добитък; от всичкия добитък на израилтяните нищо няма да умре.
\par 5 И Господ определи срок, като рече: Утре Господ ще стори това на земята.
\par 6 На другия ден Господ Стори това; всичкият египетски добитък измря, а от добитъка на израилтяните нищо не умря.
\par 7 И Фараон прати да видят, и, ето, от добитъка на израилтяните нищо не беше умряло. Но сърцето на Фараона бе упорито, и той не пусна людете.
\par 8 Тогава Господ каза на Моисея и Аарона: Напълнете шепите си с пепел от пещ и нека я пръсне Моисей към небето пред Фараона;
\par 9 и пепелта ще стане прах по цялата Египетска земя, и ще причини на човеците и на животните възпаление с гнойни цирки, по цялата Египетска земя.
\par 10 И като взеха пепел от пещ и застанаха пред Фараона, Моисей я пръсна към небето; и стана възпаление с гнойни цирки на човеците и на животните.
\par 11 И магьосниците не можаха да стоят пред Моисея поради възпалението; защото възпалението беше на магьосниците, както и на всичките египтяни.
\par 12 Но Господ закорави сърцето на Фараона, та не ги послуша, според както Господ беше говорил на Моисея.
\par 13 След това Господ рече на Моисея: Стани утре та застани пред Фараона, и речи му: Така казва Господ, Бог на евреите, Пусни людете Ми, за да Ми послужат.
\par 14 Защото в това време Аз изпращам всичките Си язви върху сърцето ти, върху слугите ти и върху людете ти, за да познаеш, че в целия свят няма подобен на Мене.
\par 15 Понеже сега можех да дигна ръката Си и да поразя тебе и людете ти с мор, и ти би бил изтребен от земята,
\par 16 ако не беше, че нарочно затова те издигам, да покажа в тебе силата Си и да се прочуе Името Ми по целия свят.
\par 17 Още ли се надигаш против людете Ми та не ги пускаш?
\par 18 Ето, утре около тоя час ще наваля много тежък град, небивал в Египет, откак се е основал, дори до днес.
\par 19 Сега, прочее, прати да приберат скоро добитъка ти и всичко що имаш по полето; защото града ще падне на всеки човек и всяко животно, що се намери на полето и не се прибере в къщи; и те ще измрат.
\par 20 Прочее, който от Фараоновите слуги се убоят от това, което Господ каза, прибра бързо в къщи слугите си и добитъка си;
\par 21 а който не даде внимание на казаното от Господа, остави слугите си и добитъка си по полето.
\par 22 Тогава Господ каза на Моисея: Простри ръката си към небето, за да удари град по цялата Египетска земя, по човеците, по животните и по всяка трева на полето из цялата Египетска земя.
\par 23 И Моисей простря жезъла си към небето и Господ прати гръм, и град, и огън са спущаше по земята; Господ наваля град по Египетската земя.
\par 24 Така имаше град, и огън размесен с града, град много тежък, небивал в цялата Египетска земя, откак е заживял там народ.
\par 25 В цялата Египетска земя градът изби всичко що имаше по полето, и човек и животно; градът очука всичката трева по полето и изпочупи всичките дървета по полето.
\par 26 Само в Гесенската земя, гдето бяха израилтяните, не удари град.
\par 27 Тогава Фараон изпрати да повикат Моисея и Аарона и рече им: Тоя път съгреших; Господ е праведен, а аз и людете сме нечестиви.
\par 28 Помолете се Господу: защото стига толкова от тия ужасни гръмове и град; и аз ще ви пусна, и няма вече да останете.
\par 29 А Моисей му каза: Щом изляза от града ще простра ръцете си към Господа; и гръмовете ще престанат, и град не ще има вече, за да познаеш, че светът е Господен.
\par 30 Обаче зная, че ти и слугите ти още не ще се убоите от Господа Бога.
\par 31 (Ленът и ечемикът бидоха изпобити, защото ечемикът беше на класове, и ленът връзваше семе;
\par 32 но пшеницата и бялото жито оцеляха, защото бяха късни).
\par 33 И тъй, Моисей излезе отпред Фараона извън града и простря ръцете си към Господа; и гръмовете и градът престанаха, и дъждът не се изливаше вече по земята.
\par 34 Но като видя Фараон, че престанаха дъждът и градът и гръмовете, той продължаваше да греши, и закорави сърцето си, той и слугите ми.
\par 35 Сърцето на Фараона се зекорави, и той не пусна израилтяните, според както Господ бе говорил чрез Моисея.

\chapter{10}

\par 1 Тогава Господ каза на Моисея: Влез при Фараона; защото Аз закоравих и сърцето на слугите му, за да покажа тия Мои знамения между тях,
\par 2 и за да разкажеш в ушите на сина си и на внука си това що направих на египтяните, и знаменията, които показах между тях, та да познаете, че Аз съм Господ.
\par 3 Тогава Моисей и Аарон влязоха при Фараона и му рекоха: Така говори Господ, Бог на евреите: До кога ще отказваш да се смириш пред Мене? Пусни людете Ми, за да Ми послужат.
\par 4 Защото, ако откажеш да пуснеш людете Ми, ето, утре ще докарам скакалци в пределите ти;
\par 5 те ще покрият лицето на земята, така щото да не може човек да види земята, и ще изпоядат останалото, което оцеля, това, което ви остава от града, и ще поядат всичките дървета, които ви растат по полетата,
\par 6 и ще се напълнят с тях къщите ти, и къщите на всичките ти слуги, и къщите на всичките египтяни - нещо, което не са видели, нито бащите ти, нито дедите ти, откак са съществували на земята, дори до днес. И Моисей се обърна та излезе отпред Фараона.
\par 7 Тогава слугите на Фараона му рекоха: До кога ще ни бъде примка тоя човек? Пусни човеците да послужат на Иеова своя Бог. Още ли не знаеш, че Египет погина?
\par 8 Тогава пак доведоха Моисея и Аарона при Фараона, който им рече: Идете, послужете на Иеова вашия Бог; но кои и кои ще отидат?
\par 9 А Моисей каза: Ще отидем с младите си и със старите си, със синовете и дъщерите си, с овците си и с говедата си ще отидем, защото трябва да пазим празник на Иеова.
\par 10 Тогава Фараон им рече: Така нека е Иеова с вас, както аз ще ви пусна с челядите ви. Внимавайте, защото зло има пред вас.
\par 11 Не така; идете сега вие мъжете та послужете на Иеова, защото това поискахте. И изпъдиха ги от Фараоновото присъствие.
\par 12 Тогава Господ рече на Моисея: Простри ръката си над Египетската земя, за да покрият скакалците Египетската земя и да изпоядат всичката трева на земята, всичко което оцеля от града.
\par 13 И Моисей простря жезъла си над Египетската земя; И Господ направи да духа източен вятър на земята през целия оня ден и цялата нощ, и на заранта източният вятър докара скакалците.
\par 14 И скакалците се пръснаха по цялата Египетска земя, и нападнаха по всичките египетски предели; те бяха много страшни; преди това не е имало такива скакалци, нито ще има такива след тях.
\par 15 Защото покриха лицето на цялата земя, така че земята почерня; и изпоядоха всичката трева на земята и всичките плодове на дърветата, които бяха оцелели от града; и по цялата Египетска земя не остана нищо зелено, било дърво или трева на полето.
\par 16 Тогава Фараон бързо повика Моисея и Аарона и рече: Съгреших на Иеова вашия Бог и на вас.
\par 17 Но сега, прости, моля, греха ми само тоя път, и помолете се на Иеова вашия Бог да дигне от мене само тая смърт.
\par 18 И тъй, Моисей излезе отпред Фараона и се помоли Господу.
\par 19 И Господ промени вятъра, като докара много силен западен вятър, който дигна скакалците и ги хвърли в Червеното море; не остана ни един скакалец по всичките египетски предели.
\par 20 Но Господ закорави сърцето на Фараона, и той не пусна израилтяните.
\par 21 Тогава рече Господ на Моисея: Простри ръката си към небето, за да настане тъмнина по Египетската земя, тъмнина, която може да се пипа.
\par 22 И Моисей простря ръката си към небето; и настана тъмнина по цялата Египетска земя за три дена.
\par 23 Хората не се виждаха един друг, и за три дена никой не се помести от мястото си; но в жилищата на всичките израилтяни беше светло.
\par 24 Тогава Фараон повика Моисея и рече: Идете, послужете на Иеова; само овците ви и говедата ви нека останат; а челядите ви нека отидат с вас.
\par 25 Но Моисей каза: Обаче, ти трябва да допуснеш в ръцете ни и жертви и всеизгаряния, за да пожертвуваме на Господа нашия Бог;
\par 26 тъй че и добитъкът ни ще дойде с нас, защото от добитъка трябва да вземем, за да пожертвуваме на Господа нашия Бог; и догдето не пристигнем там, ние не знаем с какво трябва да послужим Господу.
\par 27 Но Господ закорави сърцето на Фараона, и той не склони да ги пусне.
\par 28 Тогава Фараон рече на Моисея: Махни се от мене; пази се да не видиш вече лицето ми, защото в деня, когато видиш лицето ми, ще умреш.
\par 29 А Моисей каза: Добре си решил; не ще видя лицето ти.

\chapter{11}

\par 1 (А Господ беше казал на Моисея: Още една язва ще нанеса на Фараона и на Египет, подир което ще ви пусне от тука; когато ви пусне, съвсем ще ви изпъди от тука.
\par 2 Кажи, прочее, в ушите на людете, и нека поиска всеки мъж от съседа си, и всяка жена от съседката си, сребърни и златни вещи.
\par 3 И Господ беше дал на людете да продибият благоволението на египтяните. При това, Моисей беше станал твърде велик човек в Египетската земя пред Фараоновите слуги и пред людете).
\par 4 Моисей каза на Фараона: Така казва Господ: Около средата на една нощ Аз ще мина през Египет;
\par 5 и всички първороден в Египетската земя ще умре, от първородните на Фараона, който седи на престола си, до първородния на слугинята, която е зад воденицата, и до всяко първородно от добитъка.
\par 6 И по цялата Египетска земя ще се нададе голям писък, какъвто никога не е имало, нито ще има вече такъв.
\par 7 А против израилтяните, против човек или животно, нито куче няма да поклати езика си, за да познаете, че Господ прави разлика между египтяните и израилтяните.
\par 8 И всички тия твои слуги ще дойдат при мене и ще ми припаднат и рекат: Излез ти с всичките люде, които те следват. И подир това ще изляза. И Моисей излезе от Фараоновото присъствие с голям гняв.
\par 9 (А Господ беше казал на Моисея: Фараон няма да ви послуша, за да се умножат Моите чудеса в Египетската земя.
\par 10 И Моисей и Аарон бяха извършили всички тия чудеса пред Фараона, и той не беше пуснал израилтяните из земята си).

\chapter{12}

\par 1 Тогава Господ говори на Моисея и Аарона в Египетската земя, казвайки:
\par 2 Тоя месец ще ви бъде началният месец; ще ви бъде първият месец на годината.
\par 3 Говорете на цялото Израилево общество, като им кажете да си вземат, на десетия ден от тоя месец, всеки по едно агне, според бащините си домове, по едно агне за всеки дом.
\par 4 Но ако домашните са малцина, за агнето, тогава домакинът и най-ближният до къщата му съсед нека вземат, според числото на човеците в тях; смятайте за агнето според онова, което всеки може да изяде.
\par 5 Агнето или ярето ви нека бъде без недостатък, едногодишно мъжко; от овците или от козите да го вземете.
\par 6 И да го пазите до четиринадесетия ден от същия месец; тогава цялото общество на израилтяните, събрани в домовете си да го заколят привечер.
\par 7 После нека вземат от кръвта и турят на двата стълба и на горния праг на вратата на къщите, гдето ще го ядат.
\par 8 През същата нощ нека ядат месото, печено на огън; с безквасен хляб и с горчиви треви да го ядат.
\par 9 Да не ядат от него сурово нито варено във вода, но изпечено на огън, с главата му, краката му и дреболиите му.
\par 10 И да не оставите нищо от него до утринта; ако остане нещо до утринта, изгорете го в огън.
\par 11 И така да го ядете; препасани през кръста си, с обущата на нозете си и тоягите в ръцете си; и да го ядете набързо, понеже е време на Господното минаване.
\par 12 Защото в оная нощ ще мина през Египетската земя, и ще поразя всяко първородно в Египетската земя, и човек и животно; и ще извърша съдби против всичките египетски богове; Аз съм Иеова.
\par 13 И кръвта на къщите, гдето сте, ще ви служи за белег, така че, като видя кръвта, ще ви отмина, и когато поразя Египетската земя, няма да нападна върху вас, погубителна язва.
\par 14 Оня ден ще ви бъде за спомен, и ще го пазите като празник на Господа във всичките си поколения; вечен закон ще ви бъде, да го празнувате.
\par 15 Седем дни да ядете безквасен хляб; още на първия ден ще дигнете кваса от къщите си; защото, който яде квасно от първия ден до седмия ден, оня човек ще се изтреби от Израиля.
\par 16 На първия ден ще имате свет събор, и на седмия ден свет събор; никаква работа да не се върши в тях, освен около онова, което е нужно за ядене на всеки; само това може да вършите.
\par 17 Да пазите, прочее, празника на безквасните, защото в същия тоя ден изведох войнствата ви из Египетската земя; заради което ще ви бъде вечен закон да пазите тоя ден във всичките си поколения.
\par 18 От вечерта на четиринадесетия ден от първия месец до вечерта на двадесет и първия ден от месеца ще ядете безквасни хлябове.
\par 19 Седем дена да се не намира квас в къщите ви; защото който яде квасно, оня човек ще се изтреби отсред обществото на израилтяните, бил той пришелец или туземец.
\par 20 Нищо квасно да не ядете; във всичките си жилища безквасни хлябове да ядете.
\par 21 Тогава на четиринадесетия ден от месеца, Моисей повика всичките Израилеви старейшини и рече им: Идете та си вземете по едно агне според челядите си и заколете пасхата.
\par 22 После да вземете китка от исоп и да я потопите в кръвта, която ще приемете в леген, и с кръвта, що е в легена, да ударите по горния праг и двата стълба на къщната врата; и никой от вас да не излиза от къщната си врата до утринта.
\par 23 Защото Господ ще мине, за да порази египтяните, и когато види кръвта на горния праг и на двата стълба на вратата, Господ ще отмини вратата, и не ще остави погубителят да влезе в къщите ви, за да ви порази.
\par 24 И ще пазите това като вечен закон за себе си и за синовете си.
\par 25 Когато влезете в земята, която Господ ще ви даде според обещанието си, ще пазите тая служба.
\par 26 И когато чадата ви попитат: Какво искате да кажете с тая служба?
\par 27 Ще отговорите: Това е жертва в спомен на минаването на Господа, който отмина къщите на израилтяните в Египет, когато поразяваше египтяните, а избави нашите къщи. Тогава людете се наведоха и се поклониха.
\par 28 И израилтяните отидоха та сториха, според както Господ заповяда на Моисея и Аарона; така направиха.
\par 29 И по среднощ Господ порази всяко първородно в Египетската земя, от първородния на Фараона, който седеше на престола си, до първородния на пленника, който бе в затвора, както и всяко първородно от добитък.
\par 30 И Фараон стана през нощта, той и всичките му слуги, и всичките египтяни; и нададе се голям писък в Египет, защото нямаше къща без мъртвец.
\par 31 И повика Моисея и Аарона още през нощта та рече: Станете и вие и израилтяните, излезте, изсред людете ми и идете, послужете на Иеова, както рекохте;
\par 32 подкарайте и овците си и стадата си, както рекохте, та идете; па благословете и мене.
\par 33 Тоже египтяните принуждаваха людете, за да ги отпратят по-скоро от земята си, защото си рекоха: Ние всички измираме.
\par 34 И людете дигнаха тестото си преди да вкисне, като носеха на рамена нощвите обвити в дрехите си.
\par 35 А израилтяните бяха постъпили както Моисей беше казал, като бяха поискали от египтяните сребърни и златни вещи и дрехи;
\par 36 и Господ беше дал на людете да придобият благоволението на египтяните, тъй щото те бяха им дали колкото искаха. Така те обраха египтяните.
\par 37 Прочее, израилтяните се дигнаха от Рамесий за Сокхот, на брой около шестстотин хиляди мъже пешаци, освен челядите.
\par 38 Още с тях излезе и голямо разноплеменно множество, както и твърде много добитък - овци и говеда.
\par 39 А от тестото, което носеха из Египет, изпекоха безквасни пити; защото не беше вкиснало, понеже ги изпъдиха из Египет, и те не можаха да се бавят, нито да си приготвят ястие.
\par 40 А времето, което израилтяните прекараха като пришълци в Египет, беше четиристотин и тридесет години.
\par 41 И в края на четиристотин и тридесетте години, дори в същия ден, всички войнства Господни излязоха из Египетската земя.
\par 42 Това е нощ, която е за особено опазване за Господа, загдето ги изведе из Египетската земя; това е оная нощ, която всичките израилтяни, във всичките си поколения, трябва особено да пазят за Господа.
\par 43 И Господ рече на Моисея и Аарона: Ето законът за пасхата: никой чужденец да не яде от нея;
\par 44 обаче всеки роб купен с пари да яде от нея тогава, когато се обреже.
\par 45 Никой пришелец или наемник да не яде от нея.
\par 46 В една къща да се изяде; от месото да не изнасяте вън от къщи и кост от нея да не строшите.
\par 47 Цялото общество израилтяни ще я пазят.
\par 48 И ако би някой чужденец да живее като пришелец между тебе и да иска да пази пасхата Господу, нека се обрежат всичките му мъжки, и тогава нека да пристъпи да я пази; той ще бъде като туземец. Но никой необрязан не бива да яде от нея.
\par 49 Един закон ще има за туземеца и за чужденеца, който е пришелец между вас.
\par 50 И всичките израилтяни сториха според както Господ заповяда на Моисея и Аарона; така направиха.
\par 51 И тъй, в същия оня ден Господ изведе израилтяните из Египетската земя според устроените им войнства.

\chapter{13}

\par 1 Господ говори още на Моисея казвайки:
\par 2 Посвети на Мене всяко първородно, всяко, което отваря утроба между израилтяните - и човек и животно; то е Мое.
\par 3 Тогава рече Моисей на людете: Помнете тоя ден, в който излязохте из Египет, из дома на робството; защото със силна ръка Господ ви избави от там. Никой да не яде квасно.
\par 4 Вие излизате днес в месец Авив.
\par 5 И когато Господ те въведе в земята на ханаанците, хетейците, аморейците, евейците и евусейците, за която се кле на бащите ти, че ще ти я даде, земя гдето текат мляко и мед, тогава ще пазиш тая служба в тоя месец.
\par 6 Седем дни ще ядеш безквасно; и седмият ден ще бъде празник Господу.
\par 7 Безквасно ще се яде през седем дни; и да се не намери у тебе квас, из всичките твои предели.
\par 8 И в оня ден ще обясниш на сина си, като речеш: Това правя поради онова, което Господ ми стори, когато излязох от Египет.
\par 9 Това ще ти бъде за белег на ръката, и за спомен между очите, за да бъде Господният закон в устата ти; защото със силна ръка Господ те изведе из Египет.
\par 10 Прочее, ще пазиш тая наредба всяка година на времето й.
\par 11 А когато Господ те въведе в Ханаанската земя, както се закле на тебе и на бащите ти, и ти я даде,
\par 12 тогава ще отделяш за Господа всичко, което отваря утроба, и всяко твое първородно от животно; мъжките да бъдат на Господа.
\par 13 А всяко първородно от осел ще откупиш с агне или яре; и ако не искаш да го откупиш, тогава ще му пресечеш врата. Ще откупиш и всеки първороден човек между синовете си.
\par 14 После, като те запита синът ти, казвайки: Що е това? ще му казваш: Със силна ръка ни изведе Господ из Египет, из дома на робството;
\par 15 понеже, когато Фараон не скланяше да ни пусне, Господ уби всяко първородно в Египетската земя, първородно на човек и първородно на животно; а всеки първороден от синовете си откупувам.
\par 16 Това ще бъде за белег на ръката ти и за надчелие между очите ти; понеже със силна ръка Господ ни изведе из Египет.
\par 17 А когато Фараон пусна людете, Бог не ги преведе през пътя за Филистимската земя, при все че това беше близкият път; защото Бог рече: Да не би да се разкаят людете, като видят война и да се върнат в Египет.
\par 18 Но Бог поведе людете по околен път през пустинята към Червеното море. А израилтяните излязоха от Египетската земя въоръжени.
\par 19 И Моисей взе със себе си костите на Иосифа; защото той беше строго заклел израилтяните, като каза: Бог непременно ще ви посети; тогава ще занесете със себе си костите ми от тука.
\par 20 И като тръгнаха от Сикхот, разположиха се на стан в Етам, към краищата на пустинята.
\par 21 И Господ вървеше пред тях, денем в облачен стълб, за да ги управя из пътя, а нощем в огнен стълб, за да им свети, та да пътуват денем и нощем.
\par 22 Той не отнемаше отпред людете облачния стълб денем, нито огнания стълб нощем.

\chapter{14}

\par 1 Тогава Господ говори на Моисея, казвайки:
\par 2 Заповядай на израилтяните да завият и да се разположат на стан пред Пиаирот, между Мигодол и морето; срещу Веелсефон ще се разположите на стан близо до морето.
\par 3 Защото Фараон ще рече за израилтяните: Те са се впримчили в земята; пустинята ги е затворила.
\par 4 И Аз ще закоравя сърцето на Фараона, и той ще ги погне отдире; но Аз ще се прославя над Фараона и над цялата му войска, и египтяните ще познаят, че Аз съм Гспод. И израилтяните сториха така.
\par 5 А когато се извести на египетския цар, че побягнаха людете, сърцето на Фараона и на слугите му се обърна против людете, и рекоха: Какво е това що сторихме, гдето пуснахме Израиля да не ни работи вече?
\par 6 Затова, Фараон впрегна колесницата си и събра людете си при себе си;
\par 7 взе шестстотин отборни колесници, дори всичките египетски колесници, с началниците над всичките.
\par 8 И Господ закорави сърцето на египетския цар Фараон, така че той погна отдире израилтяните. (защото израилтяните бяха излезли с издигната ръка),
\par 9 и египтяните се спуснаха подир тях, - всичките коне и колесници на Фараона, конниците му и войската му, - и настигнаха ги разположени на стан близо до морето, пред Пиаирот, срещу Веелсефон.
\par 10 А когато се приближи Фараон, израилтяните подигнаха очи, и, ето, египтяните идеха подир тях; и израилтяните, твърде много уплашени, извикаха към Господа.
\par 11 И рекоха на Моисея: Понеже нямаше гробища в Египет, затова ли ни изведе да измрем в пустинята? Защо постъпи така, та ни изведе от Египет?
\par 12 Не е ли това, каквото ти казвахме в Египет, като рекохме: Остави ни; нека работим на египтяните? Защото по-добре би било за нас да работим на египтяните, отколкото да измрем в пустинята.
\par 13 А Моисей каза на людете: Не бойте се; стойте и гледайте избавлението, което Господ ще извърши за вас днес; защото колкото за египтяните, които видяхте днес, няма да ги видите вече до века.
\par 14 Господ ще воюва за вас, а вие ще останете мирни.
\par 15 Тогава рече Господ на Моисея: Защо викаш към Мене? Кажи на израилтяните да вървят напред.
\par 16 А ти дигни жезъла си над морето та го раздели, и израилтяните ще преминат през морето по сухо.
\par 17 И Аз, ето, ще закоравя сърцето на египтяните, та ще влязат подир тях; и ще се прославя над Фараона, над цялата му войска, над колесниците му и над конниците му.
\par 18 И когато се прославя над Фараона, над колесниците му и над конниците му, египтяните ще познаят, че Аз съм Господ.
\par 19 Тогава ангелът Божий, който вървеше пред израилевото множество, се дигна та дойде отдире им; дигна се облачният стълб отпреде им та застана отдире им,
\par 20 и дойде между Египетското множество и Израилевото; на едните беше тъмен облак, а на другите светеше през нощта, така щото едните не се приближаваха до другите през цялата нощ.
\par 21 Моисей, прочее, простря ръката си над морето; и Господ направи да се оттегля морето цялата оная нощ от силен източен вятър та се пресуши морето и водите се разделиха.
\par 22 Така израилтяните влязоха всред морето по сухо; и водите бяха за тях преграда от дясната и от лявата им страна.
\par 23 А египтяните, всичките коне на Фараона, колесниците му и конниците му, - като ги погнаха влязоха подир тях всред морето.
\par 24 Но в утринната стража Господ погледна от огнения и облачния стълб на египетската войска и смути войската на египтяните;
\par 25 Той извади колелата от колесниците им та те се теглеха мъчно, така щото египтяните рекоха: Да бягаме от Израиля, защото Иеова воюва за тях против египтяните.
\par 26 Тогава Господ рече на Моисея: Простри ръката си над морето, за да се върнат водите върху египтяните, върху колесниците им и върху конниците им.
\par 27 И тъй, Моисей простря ръката си над морето; и около зори морето се върна на мястото си; а като бяха египтяните пред него, Господ изтърси египтяните всред морето.
\par 28 Защото водите се върнаха и покриха колесниците, конниците и цялата Фараонова войска, която беше влязла подир тях в морето; не остана ни един от тях.
\par 29 А израилтяните минаха през морето по сухо; и водите бяха за тях преграда от дясната и от лявата им страна.
\par 30 Така в оня ден Господ избави Израиля от ръката на египтяните; и Израил видя египтяните мъртви по морския бряг.
\par 31 Израил видя онова велико дело, което Господ извърши над египтяните; и людете се убояха от Господа, и повярваха в Господа и слугата Му Моисея.

\chapter{15}

\par 1 Тогава запяха Моисей и израилтяните тая песен Господу, като говориха с тия думи: - Ще пея Господу, защото славно възтържествува; Коня и ездача му хвърли в морето.
\par 2 Господ е сила моя, песен моя. И стана ми Спасител; Той ми е Бог и ще Го прославя, Бащиният ми Бог, и ще Го превъзвиша.
\par 3 Господ е силен Воевател; Името Му е Иеова.
\par 4 Колесниците Фараонови и войската му хвърли в морето; Отборните му полководци потънаха в Червеното море.
\par 5 Дълбочините ги покриха Като камък слязоха в бездните.
\par 6 Десницата Ти, Господи, се прослави в сила Десницата Ти, Господи, смаза неприятеля.
\par 7 С превъзходното Си величие изтребил си противниците Си Пратил си гнева Си, и пояде ги като слама.
\par 8 От духането на ноздрите Ти вадите се струпаха на куп, Вълните застанаха като грамада, Бездните се сгъстиха всред морето.
\par 9 Неприятелят рече: Ще погна, ще стигна, ще разделя користите; Похотта ми ще се насити върху тях; Ще изтръгна ножа си, ръката ми ще ги погуби.
\par 10 Подухнал си с вятъра Си, и морето ги покри; Потънаха като олово в силните води.
\par 11 Кой е подобен на Тебе, Господи, между боговете? Дивен та да Те възпяват, правещ чудеса?
\par 12 Прострял си десницата Си, И земята ги погълна.
\par 13 С милостта Си водил си людете, които си откупил; Упътил си ги със силата Си към светото Си обиталище.
\par 14 Племената ще чуят и ще затреперят; Ужас ще обладае филистимските жители,
\par 15 Тогава ще се уплашат едомските началници; Трепет ще обземе моавските силни; Всичките жители на Ханаан ще се стопят.
\par 16 Страх и трепет ще нападне на тях; Чрез великата Ти мишца ще станат неподвижни като камък. Догде заминат людете ти, Господи, Догде заминат людете, които си придобил.
\par 17 Ще ги въведеш и насадиш в хълма - Твоето наследство, На мястото, Господи, което си приготвил за свое обиталище, В светилището, Господи, което ръцете Ти утвърдиха.
\par 18 Господ ще царува до вечни векове.
\par 19 Защото, като бяха влезли в морето Фараоновите коне с колесниците му и с конниците му, Господ беше повърнал върху тях водите на морето; а израилтяните бяха преминали през сред морето.
\par 20 Тогава пророчицата Мариам, Аароновата сестра, взе в ръката си тъпанче, и всички жени излязоха подир нея с тъпанчета и хороиграния,
\par 21 а Мариам им пееше ответно: Пейте Господу, защото славно възтържествува; Коня и ездача му хвърли в морето.
\par 22 Тогава Моисей дигна израилтяните от Червеното Море и излязоха към пустинята Сур; и като вървяха три дена в пустинята не намериха вода.
\par 23 После дойдоха в Мера, но не можеха да пият от водата на Мера защото беше горчива; (затова се наименува Мера).
\par 24 Тогава людете роптаеха против Моисея, казвайки: Що да пием?
\par 25 А той извика към Господа, и Господ му показа дърво; и като го хвърли във водата, водата се подслади. Там им положи повеление и наредба и там ги опита, като рече:
\par 26 Ако прилежно слушаш гласа на Господа своя Бог, и вършиш онова, което Му е угодно, и слушаш заповедите Му, и пазиш всичките Му повеления, то не ще ти нанеса ни една от болестите, които нанесох върху египтяните; защото Аз съм Господ, Който те изцелявам.
\par 27 После дойдоха в Елим, гдето имаше дванадесет палмови дървета; и там се разположиха на стан при водите.

\chapter{16}

\par 1 Като се дигнаха от Елим, цялото общество израилтяни дойдоха в пустинята Син, която е между Елим и Синай, на петнадесетия ден от втория месец, откак излязоха из Египетската земя.
\par 2 А в пустинята цялото общество израилтяни роптаеха против Моисея и Аарона.
\par 3 Израилтяните им казаха: По-добре да бяхме умрели от Господната ръка в Египетската земя, когато седяхме около котлите с месо и когато ядяхме хляб до ситост; защото ни доведохте в тая пустиня, за да изморите това цяло множество с глад.
\par 4 Тогава рече Господ на Моисея: Ето, ще ви наваля хляб от небето; и ще излизат людете всеки ден да събират колкото им трябва за деня, за да ги опитам, ще ходят ли по закона Ми, или не.
\par 5 А на шестия ден нека сготвят внесеното, което да бъде два пъти, колкото събират всеки ден.
\par 6 И тъй, Моисей и Аарон казаха на всичките израилтяни: Довечера ще познаете, че Господ ви е извел из Египетската земя;
\par 7 а на утринта ще видите славата на Господа, понеже Той чу роптаниято ви против Господа. Защото що сме ние та да роптаете против нас?
\par 8 Моисей още рече: Това ще стане, когато Господ ви даде довечера месо да ядете, и на утринта хляб до ситост; понеже Господ чу роптанията ви против Него. Защото що сме ние? Роптанията ви не са против нас, а против Господа.
\par 9 И рече Моисей на Аарона: Кажи на цялото общество израилтяни: Приближете се пред Господа, защото Той чу роптанията ви.
\par 10 А докато Аарон говореше на цялото общество израилтяни, те обърнаха погледа си към пустинята, и, ето, Господната слава се яви в облака.
\par 11 И Господ говори на Моисея, казвайки:
\par 12 Чух роптанията на израилтяните. Говори им така: Довечера ще ядете месо и на утринта ще се наситите с хляб; и ще познаете, че Аз съм Господ вашият Бог.
\par 13 И така, на вечерта долетяха пъдпъдъци и покриха стана; а на утринта на всякъде около стана беше паднала роса.
\par 14 И като се изпари падналата роса, ето, по лицето на пустинята имаше дребно люспообразно нещо, тънко като слана по земята.
\par 15 Като го видяха израилтяните казаха си един на друг: Що е това? защото не знаеха що беше. А Моисей им каза: Това е хлябът, който Господ ви дава да ядете.
\par 16 Ето що заповядва Господ: Съберете от него, всеки толкова, колкото му трябва да яде, по гомор на глава, според числото на човеците ви; всеки да вземе за ония, които са под шатрата му.
\par 17 Израилтяните сториха така и събраха, кой много, кой малко.
\par 18 И когато измериха събраното с гомора, който беше събрал много нямаше излишък, и който беше събрал малко нямаше недостиг; всеки събираше толкова, колкото му трябваше да яде.
\par 19 Моисей още им рече: Никой да не оставя от него до утринта.
\par 20 При все това, те не послушаха Моисея, и някои оставиха от него до утринта; но червяса и се усмърдя; и Моисей се разгневи на тях.
\par 21 И всяка заран го събираха, кой колкото му трябваше за ядене; а като припечеше слънцето стопяваше се.
\par 22 А на шестия ден, когато събраха двойно количество храна, по два гомора за всекиго, всичките началници на обществото дойдоха и известиха на Моисея.
\par 23 А той им рече: Това е точно каквото каза Господ. Утре е събота, света почивка Господу; опечете колкото искате да опечете, и сварете колкото искате да сварите; и турете на страна каквото остане, да ви стои за утре.
\par 24 И тъй, туриха го настрана до утрото, както заповяда Моисей, и не се усмърдя, нито се намери червей в него.
\par 25 Тогава Моисей им каза: Яжте това днес, защото днес е събота Господу; днес няма да го намерите на полето.
\par 26 Шест дена ще го събирате; но седмия ден е събота, в нея няма да се намира.
\par 27 Обаче някои от людете излязоха да съберат на седмия ден, но не намериха.
\par 28 Тогава рече Господ на Моисея: До кога ще отказвате да пазите заповедите Ми и законите Ми?
\par 29 Вижте, понеже Господ ви даде съботата, затова на шестия ден ви даде хляб за два дена. Седете, всеки на мястото си; на седмия ден никой да не излиза от мястото.
\par 30 И тъй, людете си отпочиваха на седмия ден.
\par 31 А Израилевият дом нарече тая храна Манна; тя беше бяла и приличаше на кориандрово семе; и вкусът й беше като на пита смесена с мед.
\par 32 Тогава рече Моисей: Ето какво е заповядал Господ: Напълнете един гомор с нея, за да се пази за всичките ви поколения, за да могат и те да видят хляба, с който ви храних в пустинята, когато ви изведох из Египетската земя.
\par 33 И Моисей рече на Аарона: Вземи една стомна, и, като туриш в нея пълното на един гомор манна, положи я пред Господа, за да се пази за идните ви поколения.
\par 34 И така, Аарон а положи пред плочите на свидетелството, за да се пази, според както Господ заповяда на Моисея.
\par 35 И израилтяните ядоха манната четиридесет години, докато дойдоха в обитаемата земя, ядоха манна, докато пристигнаха до границите на Ханаанската земя.
\par 36 А гоморът е една десета от ефата.

\chapter{17}

\par 1 След това цялото общество израилтяни тръгнаха от пустинята Син, като бяха пътуванията им според Господната заповед; и разположиха стан в Рафидим, гдето нямаше вода да пият людете.
\par 2 Затова людете се караха с Моисея и рекоха: Дай ни вода да пием. А Моисей им рече: Защо се карате с мене? Защо изпитвате Господа?
\par 3 Но людете ожадняха там за вода; и людете роптаха против Моисея, като думаха: Защо ни изведе из Египет да умориш с жажда и нас, и чадата ни, и добитъка ни?
\par 4 Тогава Моисей извика към Господа казвайки: какво да правя с тия люде? още малко и ще ме убият с камъни.
\par 5 А Господ рече на Моисея: Замини пред людете, като вземеш със себе си някои от Израилевите старейшини; вземи в ръката си и жезъла си, с който удари реката, и върви.
\par 6 Ето, Аз ще застана пред тебе там на канарата в Хорив; а ти удари канарата, и ще потече вода из нея, за да пият людете. И Моисей стори така пред очите на Израилевите старейшини.
\par 7 И нараче мястото Маса, и Мерива, поради карането на израилтяните, и понеже изпитаха Господа, като казаха: Да ли е Господ между нас, или не?
\par 8 По това време дойде Амалик и воюва против Израиля в Рафидим.
\par 9 А Моисей каза на Исуса Навиева: Избери ни мъже, и излез да се биеш с Амалика; и утре аз ще застана на върха на хълма и ще държа Божият жезъл в ръката си.
\par 10 И Исус стори според както му каза Моисей, и би се с Амалика; а Моисей, Аарон и Ор се качиха на върха на хълма.
\par 11 И когато Моисей издигаше ръката си, Израил надвиваше; а когато спущаше ръката си Амалик надвиваше.
\par 12 А като натегнаха ръцете му, взеха камък и подложиха на Моисея, и той седна на него; а Аарон и Ор, единият от едната страна и другият от другата, подпираха ръцете му, така щото ръцете му се подкрепяха до захождането на слънцето.
\par 13 Така Исус порази Амалика и людете му с острото на ножа.
\par 14 Тогава рече Господ на Моисея: Запиши в книгата за спомен, и предай в ушите на Исуса, това, че ще излича съвсем спомена на Амалика под небето.
\par 15 И Моисей издигна там олтар, който нараче Иеова Нисий,
\par 16 като рече: Ръка се подигна против Господния престол; затова, Господ ще ратува против Амалика от поколение в поколение.

\chapter{18}

\par 1 И като чу Мадиамският жрец Иотор, Моисеевият тъст, за всичко, което Бог извършил а Моисея и за людете си Израиля, как Господ, извел Израиля из Египет,
\par 2 то Моисеевият тъст Иотор взе Сепфора, Моисеевата жена, (след като я беше изпратил надире),
\par 3 и двата й сина, (от които на единия името бе Гирсом, защото Моисей беше казал: Пришелец станах в чужда земя;
\par 4 на другия името бе Елиезер§, защото беше казал: Бащиният ми Бог ми стана помощник и ме избави от Фараоновия нож);
\par 5 и Иотор Моисеевият тъст, дойде при Моисея със синовете му и с жена му в пустинята до Божията планина, гдето се беше разположил на стан,
\par 6 и извести на Моисея: Аз, тъстът ти Иотор, ида при тебе с жена ти и двата й сина с нея.
\par 7 Тогава Моисей излезе да посрещне тъста си, поклони се, и го целуна; и като се разпитаха един друг за здравето си влязоха в шатъра.
\par 8 И Моисей разказа на тъста си всичко що бе сторил Господ на Фараона и на египтяните, заради Израиля, и всичките мъчнотии, които ги сполетяха из пътя, и как ги избави Господ.
\par 9 И Иотор се зарадва много за всичкото добро, което Господ бе сторил на Израиля, като го избави от ръката на египтяните.
\par 10 И Иотор каза: Благословен Господ, Който ви избави от ръката на египтяните и от Фараоновата ръка, който избави людете от ръката на египтяните.
\par 11 Сега зная, че Господ е по-велик от всичките богове, даже и в това, с което те се гордееха, той стана по-горен от тях.
\par 12 Тогава Моисеевият тъст Иотор взе всеизгаряне и жертви, за да принесе Богу; и Аарон и всичките Израилеви старейшини дойдоха да ядат хляб с Моисеевия тъст пред Бога.
\par 13 На другия ден Моисей седна да съди людете; и людете стояха около Моисея от заран до вечер.
\par 14 А Моисеевият тъст, като видя всичко, което той вършеше за людете рече: Що е това, което правиш с людете? Защо седиш сам и всичките люде стоят около тебе от заран до вечер?
\par 15 А Моисей рече на тъста си: Защото людете дохождат при мене да се допитват до Бога.
\par 16 Когато имат дело дохождат при мене; и аз съдя между единия и другия, и пояснявам им Божиите повеления и закони.
\par 17 Но Моисеевият тъст каза: Това, което правиш, не е добро.
\par 18 Непременно и ти ще се изнуриш и тия люде, които са с тебе, защото това е много тежко за тебе; не можеш го върши сам.
\par 19 Сега послушай думите ми; ще те посъветвам, и Бог да бъде с тебе. Та предстоявай между людете и Бога, за да представяш делата пред Бога;
\par 20 и поучавай ги в повеленията и законите и показвай им пътя, по който трябва да ходят и делата, които трябва да вършат.
\par 21 Но при това измежду всичките люде избери си способни мъже, които се боят от Бога, обичат истината и мразят несправедливата печалба, и постави над людете такива за хилядници, стотници, петдесетници и десетници;
\par 22 и те нека съдят людете всякога, всяко голямо дело нека донасят пред тебе, а всяко малко дело нека съдят сами; така ще ти олекне, и те ще носят товара заедно с тебе.
\par 23 Ако сториш това, и ако Бог така ти заповяда, тогава ще можеш да утраеш; па и всички тия люде ще стигнат на мястото си с мир.
\par 24 И Моисей послуша думите на тъста си и стори всичко що му рече.
\par 25 Моисей избра способни мъже измежду целия Израил, които постави началници над людете - хилядници, стотници, петдесетници и десетници.
\par 26 Те съдеха людете на всяко време; мъчните дела донасяха на Моисея, а всяко малко дело съдеха сами.
\par 27 След това Моисей изпрати тъста си; и той отиде в своята земя.

\chapter{19}

\par 1 Три месеца подир излизането на израилтяните из Египетската земя, на същия ден дойдоха в Синайската пустиня.
\par 2 Като се дигнаха от Рафидим дойдоха в Синайската пустиня и разположиха стан в пустинята, гдето и Израил разпъна шатрите си срещу планината.
\par 3 И като се възкачи Моисей при Бога, Господ го повика от планинета и рече: Така да кажеш на Якововия дом и да известиш на потомците на Израиля:
\par 4 Вие видяхте що сторих на египтяните, а как носих вас на орлови крила и доведох ви при Себе Си.
\par 5 Сега, прочее, ако наистина ще слушате гласа Ми и ще пазите завета Ми, то повече от всичките племена вие ще бъдете Мое собствено притежание, защото Мой е целият свят;
\par 6 и вие още Ми бъдете царство свещеници и свет народ. Тия са думите, които трябва да кажеш на израилтяните.
\par 7 И тъй, Моисей дойде та повика старейшините на людете, и представи пред тях всички тия думи, които Господ му заповяда.
\par 8 И всичките люде отговориха едногласно, казвайки: Всичко, което Господ е казал, ще сторим. И Моисей отнесе Господу думите на людете.
\par 9 Тогава Господ рече на Моисея: Ето, Аз ида при теб в гъст облак, за да чуят людете, когато говоря с тебе и да те вярват вече за винаги. И като каза Моисей Господу думите на людете,
\par 10 Господ рече още на Моисея: Иди при людете, освети ги днес и утре, и нека изперат дрехите си;
\par 11 и нека бъдат готови за третия ден, защото на третия ден Господ ще излезе на Синайската планина пред очите на всичките люде.
\par 12 И да поставиш прегради наоколо за людете, и да кажеш: Внимавайте да се не качите на планината, нито да се допрете до полите му: който се допре до планината непременно ще се умъртви;
\par 13 обаче ръка да се не допре до него, но той да се убие с камъни или стрели, било то животно или човек, който се допре, да не остане жив. Когато тръбата дълго тръби, тогава нека се приближат до планината.
\par 14 И тъй, Моисей слезе от планината при людете и освети людете, а те изпраха дрехите си.
\par 15 И рече на людете: Бъдете готови за третия ден; не се приближавайте при жена.
\par 16 А сутринта на третия ден имаше гръмове и светкавици, и гъст облак на планината, и много силен тръбен глас; и всичките люде, които бяха в стана, потрепераха.
\par 17 Тогава Моисей изведе людете из стана, за да посрещнат Бога; и застанаха под планината.
\par 18 А Синайската планина беше цяла в дим, защото Господ слезе в огън на нея; и димът и се дигаше, като дим от пещ, и цялата планина се тресеше силно.
\par 19 И когато тръбният глас се усилваше Моисей говори, и Бог му отговори с глас.
\par 20 И Господ слезе на върха на Синайската планина. И Господ повика Моисея до върха на планината; и Моисей се възкачи.
\par 21 Тогава рече Господ на Моисея: Слез, заръчай на людете да се не спуснат към Господа, за да гледат и паднат мнозина от тях.
\par 22 Така и свещениците, които се приближават при Господа, нека се осветят, за да не нападне Господ на тях.
\par 23 А Моисей рече Господу: Людете не могат да се възкачат на Синайската планина, Защото Ти си ни заповядал, казвайки: Постави прегради около планината и освети я.
\par 24 Но Господ ме каза: Иди, слез, после да се качиш, ти и Аарон с тебе; свещениците и людете да се не спускат и да се не качват към Господа, за да не нападне Той на тях.
\par 25 Моисей, прочее, слезе при людете и им каза това.

\chapter{20}

\par 1 Тогава Бог изговори всички тия думи , като каза:
\par 2 Аз съм Иеова твоят Бог, Който те изведох из Египетската земя, из дома на робството.
\par 3 Да нямаш други богове освен мене.
\par 4 Не си прави кумир, или каквото да било подобие на нещо, което е на небето горе, или което е на земята долу, или което е във водата под земята;
\par 5 да не им се кланяш нито да им служиш, защото Аз Господ, твоят Бог, съм Бог ревнив, Който въздавам беззаконието на бащите върху чадата до третото и четвъртото поколение на ония, които Ме мразят,
\par 6 а показвам милости към хиляда поколения на ония, които Ме любят и пазят Моите заповеди.
\par 7 Не изговаряйте напразно Името на Господа твоя Бог; защото Господ няма да счита безгрешен онзи, който изговаря напразно Името Му.
\par 8 Помни съботния ден, за да го освещаваш.
\par 9 Шест дни да работиш и да вършиш всичките си дела;
\par 10 а на седмия ден, който е събота на Господа твоя Бог, да не вършиш никаква работа, ни ти, ни слугата ти, ни слугинята ти, нито добитъкът ти, нито чежденецът, който е отвътре вратите ни;
\par 11 защото в шест дни Господ направи небето и земята, морето и всичко що има в тях, а на седмия ден си почина; затова Господ благослови съботния ден и го освети.
\par 12 Почитай баща си и майка си , за да се продължават дните ти на земята, която ти дава Господ твоя Бог.
\par 13 Не убива�
\par 14 Не прелюбодействувай.
\par 15 Не кради.
\par 16 Не свидетелствувай лъжливо против ближния си.
\par 17 Не пожелавай къщата на ближния си, не пожелавай жената на ближния си, нито вола му, нито осела му, нито какво да е нещо, което е на ближния ти.
\par 18 И всичките люде гледаха гърмежите, светкавиците, гласа на тръбата и димящата планина; и, като видяха, людете се оттеглиха и застанаха надалеч.
\par 19 И рекоха на Моисея: Ти говори на нас, и ние ще слушаме; а Бог да не ни говори, за да не умрем.
\par 20 Но Моисей рече на людете: Не бойте се; Бог дойде да ви опита, и за да има всред вас страх от Него, та да не съгрешавате.
\par 21 Така людете стояха надалеч. А Моисей се приближи пре мрака, гдето беше Бог.
\par 22 Тогава рече Господ на Моисея: Така да кажеш на израилтяните: Вие сами видяхте, че ви говорих от небето.
\par 23 Покрай Мене да не правите и сребърни богове, нито да си правите и златни богове.
\par 24 От пръст Ми издигай олтар, и жертвувай на него всеизгарянията си и примирителните си приноси, овците си и говедата си. На всяко място, гдето ще правя да се помни Името Ми, ще дохождам при тебе и ще те благославям.
\par 25 Но ако ти Ми издигнеш каменен олтар, да го не съзидаш от дялани камъни; защото ако дигнеш на него сечиво, ще го оскверниш.
\par 26 И да се не качваш на олтара Ми по стъпала, за да се не открие голотата ти на него.

\chapter{21}

\par 1 Ето съдбите, които ще представиш пред тях.
\par 2 Ако купиш роб евреин, шест години ще работи, а в седмата ще излезе свободен, без откуп.
\par 3 Ако е дошъл сам, сам да си излезе; ако е имал жена, то и жена му да излезе с него.
\par 4 Ако господарят му му е дал жена, и тя му е родила синове или дъщери, то жената и чадата й ще бъдат на гсподаря й, а той ще излезе сам.
\par 5 Но ако робът изрично каже: Обичам господаря си, жена си и чадата си; не желая да изляза свободен,
\par 6 тогава господарят му ще го заведе пред съдиите, и, като го приведе при вратата, или при стълба на вратата, господарят му ще му промуши ухото с шило; и той ще му бъде роб за винаги.
\par 7 Ако някой продаде дъщеря си за робиня, тя няма да излезе така както излизат робите.
\par 8 Ако не бъде угодна на господаря си, който се е сгодил за нея , то нека я остави да бъде откупена; той не ще има власт да я продаде на чужденци, тъй като й е изневерил.
\par 9 Но ако я е сгодил за сина си, то нека постъпи с нея, както е обичайно с дъщерите.
\par 10 Ако си вземе още една жена, да не лиши оная от храната й, от дрехите й и от съпружеско съжитие с нея.
\par 11 И ако не й направи тия трите, тогава тя нека си излезе даром, без откуп.
\par 12 Който удари човек смъртоносно, непременно да се умъртви.
\par 13 Но ако не го е причаквал, но Бог го е предал в ръката му, тогава Аз ще ти определя място гдето да побегне.
\par 14 Ако, обаче, някой от злоба убие ближния си коварно, то и от олтара Ми ще го извадиш, за да се умъртви.
\par 15 Който удари баща си или майка си непременно да се умъртви.
\par 16 Който открадне човек и го продадеде, или ако откраднатият се намери в ръката му, той непременно да се умъртви.
\par 17 Който хули баща си или майка си непременно да се умъртви.
\par 18 Когато се карат някои, ако единият удари другия с камък или с юмрука си, и той не умре, но пази легло;
\par 19 и ако последният се придигне и излиза макар с тояга, тогава оня, който го е ударил, ще бъде невинен, само ще плати за денгубата му и ще направи да бъде съвършено изцерен.
\par 20 Ако някой удари роба си или робинята си с тояга, та умре под ръката му, непременно да се накаже.
\par 21 Обаче, ако удареният поживее един два дена, тогава да се не наказва, понеже той му е стока.
\par 22 Ако се бият някои и ударят трудна жена, така щото да пометне, а не последва друга повреда, тогава оня, който я е ударил непременно да бъде глобен, според както мъжът й би му наложил, и да плати както определят съдиите.
\par 23 Но ако последва повреда, тогава да отсъдиш живот за живот,
\par 24 око за око, зъб за зъб, ръка за ръка, нога за нога,
\par 25 изгаряне за изгаряне, рана за рана, удар за удар.
\par 26 Ако някой удари роба си или робинята си в окото, и то се развали, поради окото му ще го освободи.
\par 27 И ако избие някой зъб на роба си или някой зъб на робинята си, ще го освободи поради зъба му.
\par 28 Ако вол убоде мъж или жена щото да умре, тогава да се убие с камъни, и да се не яде месото му; а стопанинът на вола ще бъде оправдан.
\par 29 Но ако волът е бил бодлив от по-напред, и това е било известно на стопанина му , но той не го е ограничил, та е убил мъж или жена, то волът да се убие с камъни, още и стопанинът му да се умъртви.
\par 30 Обаче, ако му се определи откуп, то за избавление на живота си нека даде, колкото му се определи.
\par 31 Било че волът е убол мъж или е убол жена, според тая съдба ще му се направят.
\par 32 Но ако волът убоде роб или робиня, стопанинът нека плати на господаря им тридесет сребърни сикли, и нека се убие волът с камъни.
\par 33 Ако отвори някой яма, или изкопае яма без да я покрие, и в нея падне вол или осел,
\par 34 притежателят на ямата ще заплати повредата; ще даде пари на стопанина им, а мършата ще бъде негова.
\par 35 Ако волът на някого убоде вола на другиго, така щото умре, тогава да продадат живия вол и да си разделят стойността му, и мършата тоже да си разделят.
\par 36 Но ако се е знаело от по-напред, че волът е бил бодлив, и стопанинът му не го е ограничил, то непременно ще плати вол за вол, а мършата ще бъде негова.

\chapter{22}

\par 1 Ако някой открадне вол или овца та го заколи или го продаде, то да плати пет вола за вола и четири овци за овцата.
\par 2 (Ако се завари крадецът когато подкопава, и го ударят та умре, няма да се пролее кръв за него.
\par 3 Но ако слънцето е било изгряло над него, тогава ще се пролее кръв за него). Крадецът трябва непременно да плати; но ако няма с какво, то да се продаде той за откраднатото.
\par 4 И ако откраднатото, било вол, осел, или овца, се намери живо в ръката му, ще плати двойно.
\par 5 Ако някой направи да се изяде нива или лозе, като развърже животното си и то се напасе в чужда нива, ще плати от най доброто произведение на нивата си и от най-добрия плод на лозето си.
\par 6 Ако избухне огън и запали тръни, така че изгорят копни или непожънати класове, или ниви, то който е запалил огъня непременно ще плати.
\par 7 Ако някой даде на ближния си пари или някакви вещи да ги пази, и те бъдат откраднати от къщата на човека, то, ако се намери крадецът той ще плати двойно,
\par 8 Но ако не се намери крадецът, тогава стопънинът на къщата ще се заведе пред съдиите, за да се издири дали е турил ръка върху имота на ближния си.
\par 9 За всякакъв вид престъпление, - относно вол, осел, овца, дреха, или какво да било загубено нещо, за което би казал някой, че е негово, - делото между двамата ще дойде пред съдиите: и когото осъдят съдиите, той ще плати двойно на ближния си.
\par 10 Ако някой даде на ближния си осел, или вол, или овца, или какво да било животно да го пази, и то умре, или се нарани, или бъде откарано, без да види някой,
\par 11 то между двамата оня ще се закълне в Господа, че не е турил ръката си върху имота на ближния си, и стопанинът му ще приеме това свидетелство, а другият няма да плаща.
\par 12 Но ако животното бъде откраднато от него, ще плати на стопанина му.
\par 13 Обаче ако бъде разкъсано от звяр, нека го донесе за свидетелство; за разкъсаното няма да плаща.
\par 14 Ако заеме някой от ближния си животно, и то се нарани или умре в отсъствието на стопанина му, непременно ще го заплати.
\par 15 Но ако стопанинът му е с него, няма да плаща. ако е било наето с пари, ще отиде за наема си.
\par 16 Ако някой излъсти несгодена девица и легне с нея, непременно ще даде вено за нея и ще я вземе за жена.
\par 17 Но ако баща й съвсем откаже да му я даде, то ще плати в пари според веното на девиците.
\par 18 Магьосница жена да не оставиш.
\par 19 Всеки скотоложник непременно да се умъртви.
\par 20 Който жертвува на кой да бил бог, освен само на Господа, ще се обрече на изтребление.
\par 21 Чужденец да не онеправдаваш, нито да го угнетяваш; защото и вие бяхте чужденци в Египетската земя.
\par 22 Да не угнетявате вдовица или сираче.
\par 23 Защото ако ги угнетявате някак, и те извикат към Мене, непременно ще послушам вика им,
\par 24 и гневът Ми ще падне, и ще ви избия с нож; и вашите жени ще бъдат вдовици и вашите чада сирачета.
\par 25 Ако дадеш в заем пари на някой свой беден съсед между Моите люде, да не постъпваш с него като заемодател, нито да му налагаш лихва.
\par 26 Ако вземеш в залог дрехата на ближния си, до захождане на слънцето да му я върнеш;
\par 27 защото това е едничката му завивка, това е дрехата за кожата му; с какво ще спи? и като викне към Мене, Аз ще чуя, защото съм милостив.
\par 28 Да не хулиш съдиите, нито да кълнеш началник на людете си.
\par 29 Да не забравиш да принесеш първака на гумното си и на жлеба си. Първородния между синовете си ще дадеш на Мене.
\par 30 Така ще постъпиш и с говедото си и с овцата си; седем дена ще остане малкото с майка си, а на седмия ден ще го дадеш на Мене.
\par 31 Бъдете Ми свети човеци; затова не яжте месо разкъсано от зверове на полето; хвърлете го на кучетата.

\chapter{23}

\par 1 Да не разнасяш лъжлив слух. Да не съдействуваш с неправедния и да не свидетелствуваш в полза на неправдата.
\par 2 Да не следваш множеството да правиш зло; нито да свидетелствуваш в съдебно дело, така щото да се увличаш след множеството, за да изкривиш правосъдието;
\par 3 нито да показваш пристрастие към сиромаха в делото му.
\par 4 Ако срещнеш забъркалия се вол или осел на неприятеля си; непременно да му го закараш.
\par 5 Ако видиш, че оселът на неприятеля ти е паднал под товара си, и не ти се иска да му помогнеш пак непременно да помогнеш заедно с него.
\par 6 Да не изкривяваш правото на сиромаха между вас в делото му.
\par 7 Отдалечавай се от всяка несправедлива работа, и не убивай невинния и праведния; защото Аз няма да оправдая нечестивия.
\par 8 Да не приемаш подаръци, защото подаръците заслепяват видещите и извръщат душите на праведните.
\par 9 И да не угнетяваш чужденеца; защото вие знаете що има на сърцето на чужденеца, понеже и вие сте били чужденци в Египетската земя.
\par 10 Шест години да сееш земята си и да събираш плодовете й;
\par 11 а в седмата да я оставиш да си почива и да не я работиш за да се хранят сиромасите между людете ти от самораслото, и полските животни нека ядат от оставеното от тях. Така да правиш и с лозето си и с маслините си.
\par 12 Шест дена да вършиш работата си; а в седмия ден да си почиваш, за да се отмори волът ти и оселът ти, и да си отдъхне синът на слугинята ти и чужденецът.
\par 13 И внимавайте във всичко що съм ви говорил; и име на други богове да не споменавате, нито да се чува то из устата ви.
\par 14 Три пъти в годината да Ми правиш празник.
\par 15 Да пазиш празника на безквасните хлябове; седем дена да ядеш безквасни хлябове, както съм ти заповядал, на определеното време в месец Авив, защото в него си излязъл из Египет; и никой да не се яви пред Мене с празни ръце;
\par 16 и празника на жетвата, на първите плодове на труда ти, на това, което си посял в нивата; и празника на беритбата при края на годината, когато прибираш плодовете си от нивата.
\par 17 Три пъти в годината всичките твои мъжки да се явят пред Господа Иеова.
\par 18 Да не принасяш кръвта на жертвата Ми с квасен хляб, нито тлъстина от празника Ми да остава през нощта до сутринта.
\par 19 Най-първите плодове от земята си да принасяш в дома на Господа твоя Бог. Да не свариш яре в млякото на майка му.
\par 20 Ето, изпращам ангел пред тебе да те пази по пътя и да те заведе на мястото, което съм ти приготвил.
\par 21 Внимавайте на него и слушайте гласа му; не го предизвиквайте, защото той няма да прости престъпленията ви; понеже Моето Име е в него.
\par 22 Но ако слушаш внимателно гласа му, и вършиш всичко каквото говоря, тогава Аз ще бъда неприятел на твоите неприятели и противник на твоите противници.
\par 23 Защото ангелът Ми ще върви пред тебе и ще те въведе при аморейците, хетейците, ферезейците, ханаанците, евейците и евусейците; и ще ги изтребя.
\par 24 Да се не кланяш на техните богове, нито да им служиш, нито да вършиш според делата им; а да ги събаряш съвсем, и да изпотрошиш стълбовете им.
\par 25 Но да служиш на Господа вашия Бог, и Той ще благославя хляба ти и водата ти; и Аз ще отмахвам всяка болест помежду ви.
\par 26 Не ще има пометкиня или бездетна в земята ти, числото на дните ти ще направя пълно.
\par 27 Ще изпратя пред тебе страх от Мене, и ще обезсиля всичките люде, между които отидеш, и ще направя всичките ти неприятели да обърнат гръб пред тебе.
\par 28 Ще изпратя и стършели пред тебе, които ще изгонят отпред тебе, евейците, ханаанците и хетейците.
\par 29 Няма да ги изпъдя отпред тебе в една година, да не би да запустее земята и се размножат против тебе полските зверове.
\par 30 Малко по малко ще ги изпъждам отпред тебе, догде се размножиш и завладееш земята.
\par 31 И ще поставя пределите ти от Червеното море до Филистимското море, и от пустинята до река Евфрат; защото ще предам местните жители в ръката ви, и ти ще ги изпъдиш отпред себе си.
\par 32 Да не направиш завет с тях нито с боговете им.
\par 33 Да не живеят в земята ти, да не би да те накарат да съгрешиш против Мене; защото ако служиш на боговете им, това непременно ще ти бъде примка.

\chapter{24}

\par 1 Рече още на Моисея: Възкачете се към Господа, ти и Аарон, Надав и Авиуд, и седемдесет от Израилевите старейшини, и поклонете се от далеч;
\par 2 само Моисей ще се приближи при Господа, а те не ще се приближат, нито ще се възкачат людете с него.
\par 3 Тогава Моисей дойде та каза та каза на людете всичките думи на Господа и всичките му съдби; и всичките люде едногласно отговориха казвайки: Всичко, което е казал Господ, ще вършим.
\par 4 И Моисей написа всичките Господни думи; и на утринта, като стана рано, гдето изправи и дванадесет стълба, според дванадесетте Израилеви племена.
\par 5 И изпрати момци от израилтяните, та принесоха всеизгаряния и пожертвуваха Господу телци за примирителни жертви.
\par 6 А Моисей взе половината от кръвта и тури я в паници, а с другата половина половина от кръвта поръси върху олтара.
\par 7 После взе книгата за завета и я прочете, като слушаха людете; и те рекоха: Всичко, каквото е казал Господ, ще вършим, и ще бъдем послушни.
\par 8 Тогава Моисей взе кръвта и поръси с нея върху людете, като казваше: Ето кръвта на завета, който Господ направи с вас според всички тия условия.
\par 9 И тъй Моисей и Аарон, Надав и Авиуд и седемдесет от Израилевите старейшини се възкачиха горе.
\par 10 И видяха Израилевия Бог: под нозете Му имаше като настилка от сапфир, чиято бистрота беше също като небе.
\par 11 Но Той не тури ръка на благородните от израилтяните. И те видяха Бога, и там ядоха и пиха.
\par 12 Тогава рече Господ на Моисея: Ела горе при мене на планината и стой там; и ще ти дам каменните плочи, закона и заповедите, които съм написал, за да ги поучаваш.
\par 13 И тъй Моисей стана със слугата си Исуса, и Моисей се изкачи на Божията планина.
\par 14 А рече на старейшините: Чакайте ни тук догде се завърнем при вас; и, ето, Аарон и Ор са при вас; който има тъжба, нека иде при тях.
\par 15 Тогава Моисей се изкачи на планината; и облакът покриваше планината.
\par 16 И Господната слава застана на Синайската планина, и облакът я покриваше шест дена, а на седмия ден Господ извика към Моисея всред облака.
\par 17 И видът на Господната слава по върха на планината се виждаше на израилтяните като огън пояждащ.
\par 18 И тъй Моисей влезе всред облака и се възкачи на планината. И Моисей стоя на планината четиридесет дена и четиридесет нощи.

\chapter{25}

\par 1 Тогава Господ говори на Моисея казвайки:
\par 2 Кажи на израилтяните да Ми съберат принос; от всеки човек, който на радо сърце би дал, ще приемете приноса за Мене.
\par 3 И ето какъв принос ще приемете от тях; злато, сребро и мед,
\par 4 синьо, мораво, червено, висон, и козина,
\par 5 червено боядисани овнешки кожи и язовски кожи, ситимово дърво,
\par 6 масло за осветление, и аромати за мирото за помазване и за благоуханното кадене,
\par 7 оникси, и камъни за влагане на ефода и на нагръдника.
\par 8 И да Ми направят светилище, за да обитавам между тях.
\par 9 По всичко, което ти показвам - образа на скинията и образа на всичките й принадлежности, - така да я направите.
\par 10 Да направите ковчег от ситимово дърво, дълъг два лакътя и половина, широк лакът и половина, и лакът и половина висок.
\par 11 Да го обковеш с чисто злато; отвън и отвътре да го обковеш; и отгоре му да направиш златен венец наоколо.
\par 12 И да излееш за него четири златни колелца, които да поставиш на четирите му долни ъгъла, две колелца на едната му страна, и две колелца на другата му страна.
\par 13 Да направиш и върлини от ситимово дърво и да го обковеш със злато,
\par 14 па да провреш върлините през колелцата от страните на ковчега, за да се носи ковчегът с тях.
\par 15 Върлините да остават в колелцата на ковчега; да се не изваждат от него.
\par 16 И да вложиш в ковчега плочите на свидетелството, което ще ти дам.
\par 17 Да направиш умилостивилище от чисто злато, два лакътя и половина дълго, и лакът и половина широко.
\par 18 И да направиш два херувима от злато, изковани да ги направиш, на двата края на умилостивилището.
\par 19 Да направиш един херувим на единия край, и един херувим на другия край; херувимите да направите част от самото умилостивилище на двата му края.
\par 20 И херувимите да бъдат с разперени отгоре крила, и да покриват с крилата си умилостивилището; и лицата им да са едно срещу друго; към умилостивилището да бъдат обърнати лицата на херувимите.
\par 21 И да положат умилостивилището върху ковчега; а в ковчега да вложиш плочите на свидетелството, което ще ти дам.
\par 22 Там ще се срещам с тебе; и отгоре на умилостивилището, измежду двата херувима, които са върху ковчега с плочите на свидетелството, ще говоря с тебе за всичко, което ще ви заповядам за израилтяните.
\par 23 Да направиш трапеза от ситимово дърво, два лакътя дълга, един лакът широка, и лакът и половина висока.
\par 24 Да я обковеш с чисто злато и да й направиш златен венец на около.
\par 25 Да й направиш и перваз наоколо, една длан широк, и да направиш златен венец около перваза й.
\par 26 Да й направиш и четири златни колелца, и да поставиш колелцата на четирите й ъгъла, които са при четирите й нозе.
\par 27 Колелцата да бъдат до самия перваз, като влагалища на върлините, за да се носи трапезата.
\par 28 Върлините ще направиш от ситимово дърво, и да ги обковеш със злато, и да се носи трапезата с тях.
\par 29 И да направиш блюдата й, темянниците й, поливалниците й, и тасовете й, за употреба при възлиянията; от чисто злато да ги направиш.
\par 30 И на трапезата постоянно да слагаш хлябове за приношение пред Мен.
\par 31 Да направиш и светилника от чисто злато; изкован да направиш светилника; стъблото му, клоновете му, чашките му, топчиците му, и цветята му да са част от самия него.
\par 32 От страните му да се издават шест клона, три клона на светилника от едната му страна, и три клона на светилника от другата му страна.
\par 33 На единия клон да има три чашки, като бадеми, една топчица и едно цвете, и на другия клон три чашки като бадеми, и една топчица и едно цвете; така да има и на шестте клона, които се издават от светилника.
\par 34 И на стъблото на светилника да има четири чашки като бадеми, с топчиците им и цветята им.
\par 35 И на шестте клона, които се издават от светилника, да има под първите два клона една топчица, част от самия него, и под вторите два клона една топчица, част от самия него, и под третите два клона една топчица, част от самия него.
\par 36 Топчиците им и клоновете им да са част от самия него; светилникът да бъде цял изкован от чисто злато.
\par 37 И да му направиш седем светила; и да палят светилата му, за да светят отпреде му.
\par 38 Щипците му и пепелниците му да бъдат от чисто злато.
\par 39 От един талант чисто злато да се направи той и всички тия прибори.
\par 40 Внимавай да ги направиш по образеца им, който ти бе показан на планината.

\chapter{26}

\par 1 При това да направиш скинията от десет завеси от препреден висон, и от синя, морава и червена материя; на тях да навезеш изкусно изработени херувими.
\par 2 Дължината на всяка завеса да бъде двадесет и осем лакътя, и широчината на всяка завеса четири лакътя, всичките завеси да имат една мярка.
\par 3 Петте завеси да бъдат скачени една с друга; и другите пет завеси да бъдат скачени една с друга.
\par 4 И да направиш сини петелки по края на оная завеса, която е последна от първите скачени завеси; така да направиш и по края на последната завеса от вторите скачени завеси.
\par 5 Да направиш петдесет петелки на едната завеса, и петдесет петелки да направиш по края на завесата; която е във вторите скачени завеси; петелките да са една срещу друга.
\par 6 Да направиш и петдесет златни куки, и с куките да скачиш завесите една за друга; така скинията ще бъде едно цяло.
\par 7 Да направиш завеси от козина за покрив над скинията; да направиш единадесет такива завеси;
\par 8 дължината на всяка завеса да бъди тридесет лакътя, и широчината на всяка завеса четири лакътя; единадесет завеси да имат една мярка.
\par 9 И да окачиш петте завеси отделно; а шестата завеса да прегънеш на две към лицето на скинията.
\par 10 И да направиш петдесет петелки по края на оная завеса, която е последна от първите скачени завеси, и петдесет петелки по края на завесата, която е последна от вторите скачени завеси.
\par 11 И да направиш петдесет медни куки и да вкараш куките в петелките, и така да съединиш покрива та да е едно цяло.
\par 12 И оная част, която остава повече от завесите на покрива, половината на завесата, която остава, нека виси над задната страна на скинията.
\par 13 И един лакът от едната страна и един лакът от другата страна от онава, което остава повече от дължината на завесите на покрива, да виси по страните на скинията отсам и оттам, за да я покрива.
\par 14 Тоже да направиш покрив за скинията от червено боядисани овнешки кожи, и отгоре едно покривало от язовски кожи.
\par 15 Да направиш за скинията дъски от ситимово дърво, които да стоят изправени
\par 16 Десет лакътя да бъде дължината на всяка дъска и лакът и половина широчината на всяка дъска;
\par 17 и два шипа да има във всяка дъска, един срещу друг; и така да направиш на всички дъски на скинията.
\par 18 И дъските на скинията да направиш двадесет дъски за южната страна, към пладне;
\par 19 и под двадесетте дъски да поставиш четиридесет сребърни подложки, две подложки под една дъска за двата й шипа, и две подложки под друга дъска за двата й шипа.
\par 20 Също за втората страна на скинията, която е северната, да направиш двадесет дъски,
\par 21 и четиридесетте им сребърни подложки, две подложки под една дъска, и две подложки под друга дъска.
\par 22 А за задната страна на скинията, западната, да направиш шест дъски,
\par 23 И две дъски да направиш за ъглите на скинията от задната страна;
\par 24 да са скачени отдолу, а отгоре да са свързани посредством едно колелце; така да бъде за двете дъски; те нека бъдат за двата ъгъла.
\par 25 Така да бъдат осем дъски, и сребърните им подложки шестнадесет подложки, две подложки под една дъска и две подложки под друга дъска.
\par 26 И да направиш лостове от ситимово дърво, пет за дъските от едната страна на скинията;
\par 27 пет лоста за дъските от другата страна на скинията, и пет лоста за дъските от задната страна, към запад.
\par 28 И средният лост, който е в средата на дъските, да преминава от край до край.
\par 29 Дъските да обковеш със злато, и колелцата им да направиш от злато за влагалища на лостовете; да обковеш и лостовете със злато.
\par 30 Да издигнеш скинията според образеца й, който ти бе показан на планината.
\par 31 И да направиш завеса от синьо, мораво, червено и препреден висон, и да навезеш на нея изкусно изработени херувими.
\par 32 И да я окачиш със златни куки на четири обковани със злато стълба от ситимово дърво, които да стоят на четири сребърни подложки.
\par 33 Под куките да окачиш завесата; и да внесеш там, отвътре завесата, ковчега с плочите на свидетелството, така щото завесата да ви отделя светото място от пресветото.
\par 34 И до положиш умилостивилището върху ковчега с плочите на свидетелството в пресветото място.
\par 35 А трапезата да положиш отвън завесата към южната страна на скинията; а трапезата да положиш към северната страна.
\par 36 И да направиш за вратата на шатъра закривка, везана работа, от синьо, мораво, червено и препреден висон.
\par 37 И за покривката да направиш пет стълба от ситимово дърво, които да обковеш със злато; куките им да бъдат златни; и да излееш за стълбовете пет медни подложки.

\chapter{27}

\par 1 Да направиш олтара от ситимово дърво, пет лакътя широк; четвъртит да бъде олтарът, и височината му да бъде три лакътя.
\par 2 На четирите му ъгъла да му направиш рогове; роговете да бъдат част от самия него; и да го обковеш с мед.
\par 3 Да му направиш и гърнета за изнасяне на пепелта и лопатите му, тасовете му, вилиците му и въглениците му; медни да направиш всичките му прибори.
\par 4 И да му направиш медна решетка във вид на мрежа, и на четирите ъгъла на мрежата да направиш четири медни колелца.
\par 5 И да я положиш под полицата, която е около олтара отдолу, така щото мрежата да стигне до средата на олтара.
\par 6 Да направиш и върлини за олтара, върлини от ситимово дърво, които да обковеш с мед.
\par 7 Върлините да се проврат през колелцата, и върлините да бъдат от двете страни на олтара, за да се носи с тях.
\par 8 Кух, от дъски, да направиш олтара посред, както ти се показа на планината, така да го направиш.
\par 9 Да направиш двора на скинията; за южната страна, към пладне, да има за двора завеси от препреден висон; дължината им за едната страна да бъде сто лакътя.
\par 10 Стълбовете му да бъдат двадесет, и медните им подложки двадесет; а куките на стълбовете и и връзките им да бъдат сребърни.
\par 11 Също на длъж по северната страна да има завеси дълги сто лакътя, и за тях двадесет стълба и двадесетте им медни подложки; а куките на стълбовете и връзките им да бъдат сребърни.
\par 12 После, за широчината на двора, на западната страна, да има петдесет лакътя завеси; за тях да има десет стълба и десетте им подложки.
\par 13 Широчината на двора на предната страна, към изток, да бъде петдесет лакътя.
\par 14 И завесите за едната страна на входа да бъдат дълги петнадесет лакътя, и в тях три стълба и трите им подложки.
\par 15 Също и за другата страна да има завеси петнадесет лакътя дълги, и за тях три стълба с трите им подложки.
\par 16 А за входа на двора да има покривка, дълга двадесет лакътя, везана работа от синьо, мораво, червено и препреден висон, и за тях четирите им подложки.
\par 17 Всичките стълбове около двора да бъдат опасани със сребро; куките им да бъдат сребърни, а подложките им медни.
\par 18 Дължината на двора да е сто лакътя, широчината навсякъде петдесет лакътя, а височината пет лакътя; завесите му да са от препреден висон, и подложките на стълбовете му медни.
\par 19 Всичките прибори на скинията, за всяка служба в нея, всичките й колове, и всичките колове на двора, да бъдат медни.
\par 20 И ти заповядай на израилтяните да ти донесат дървено масло първоток, чисто, за осветление, за да горят винаги светилата.
\par 21 В шатъра за срещане, извън завесата, която е пред плочите на свидетелството, Аарон и синовете му да ги нареждат да горят от вечер до заран пред Господа. Това да бъде вечен закон за израилтяните във всичките им поколания.

\chapter{28}

\par 1 А ти вземи при себе си измежду израилтяните брата си Аарона и синовете му с него, за да Ми свещенодействуат; Аарона, и Аароновите синове Надава, Авиуда, Елеазара и Итамара.
\par 2 И да направиш свети одежди на брата си Аарона, за слава и великолепие.
\par 3 И кажи на всичките умни мъже, които Аз изпълних с дух на мъдрост, да направят одежди на Аарона за освещение, та да Ми свещенодействува.
\par 4 Ето одеждите, които ще направиш: нагръдник и ефод, мантия и пъстротъкан хитон, митра и пояс; и да направят свети одежди на брата ти Аарона и на синовете му, за да Ми свещенодействуват.
\par 5 И те нека приемат от людете приносите им от злато, синьо, мораво, червено, и препреден висон.
\par 6 Да направят ефодът изкусна изработка от злато, синьо, мораво, червено и препреден висон.
\par 7 На двата му края да има две презрамки, които да се връзват за да се държи заедно.
\par 8 И препаската върху ефода, която ще е над него, да бъде еднаква с него по направата му и част от самия ефод, от злато, синьо, мораво, червено и препреден висон.
\par 9 Да вземеш и два ониксови камъка, и да изрежеш на тях имената на синовете на Израиля,
\par 10 шест от имената на единия камък, и имената на останалите шестима на другия камък, по реда на раждането им.
\par 11 С изкуството на кеменорезец, както се изрязва печат, да изрежеш на двата камъка имената на синовете на Израиля.; и да ги вложиш в златни гнездица.
\par 12 И да туриш тия два камъка на презрамките на ефода, камъни за спомен на израилтяните; и Аарон ще носи имената им за спомен пред Господа на двете си рамена.
\par 13 И да направиш златни гнездица,
\par 14 и двете верижки от чисто злато - изплетени от венцеобразна работа да ги направиш, и да закрепиш плетените верижки в гнездицата им.
\par 15 Да направиш съдебния нагръдник изкусна изработка; според направата на ефода да го направиш; от златно, синьо, мораво, червено и препреден висон да го направиш.
\par 16 Да бъде четвъртит, двоен, една педя дълъг и една педя широк.
\par 17 И да закрепиш на него камъни, четири реда камъни, ред сард, топаз и смарагд да е първият ред,
\par 18 вторият ред: антракс, сапфир и адамант;
\par 19 третият ред: лигирий, агат и аметист;
\par 20 а четвъртият ред: хрисолит, оникс и яспис; да бъдат закрепени в златните си гнездица.
\par 21 Камъните да бъдат дванадесет, според имената на синовете на Израиля, според техните имена, както се изрязва печат; да бъдат за дванадесетте племена, всеки камък според името му.
\par 22 И на нагръдника да направиш изплетени верижки, венцеобразна работа от чисто злато.
\par 23 И да направиш на нагръдника две златни колелца и да туриш двете колелца на двата края на нагръдника.
\par 24 Тогава да провреш двете венцеобразни златни верижки през двете колелца на краищата на нагръдника.
\par 25 А другите два края на двете венцеобразни верижки да свържеш с двете гнездица, и да ги туриш на презрамките на ефода откъм външната му страна.
\par 26 Да направиш още две златни колелца; които да туриш на другите два края на нагръдника, на оная му страна, която е откъм вътрешната страна на ефода.
\par 27 Да направиш и други две златни колелца, които да туриш отдолу на двете презрамки на ефода, откъм вътрешната му страна, там гдето се събират краищата му, над изкусно изработената препаска на ефода.
\par 28 И да връзват нагръдника през колелцата му за колелцата на ефода със син ширит, за да бъде над изкусно изработената препаска на ефода, и за да се не отделя нагръдникът от ефода.
\par 29 Така Аарон, когато влиза в светото място, ще носи, винаги имената на синовете на Израиля върху съдебния нагръдник на сърцето си, за спомен пред Господа.
\par 30 На съдебния нагръдник да положиш и Урима и Тумима, които да бъдат на Аароновото сърце когато влиза пред Господа; и Аарон да носи винаги съда на израилтяните на сърцето си пред Господа.
\par 31 Да направиш мантията на ефода цяла от синьо.
\par 32 В средата на върха й да има отвор, какъвто е отворът на бронята; и ще има тъкана обтока около отвора си, за да се не съдира.
\par 33 И по полите й да направиш нарове от синьо, мораво и червено наоколо по полите й, и златни звънци наоколо памежду им, -
\par 34 златен звънец и нар, златен звънец и нар, наоколо по полите на мантията.
\par 35 И Аарон да я носи, когато служи, та звънтенето й да се чува когато влиза в светото място пред Господа, и когато излиза, за да не умре.
\par 36 Да направиш и плочица от злато, на която да ирежеш, както се изрязва печат, Свет Господу.
\par 37 Да я туриш на син ширит, за да бъде на митрата; в предната страна на митрата да бъде;
\par 38 така да бъде на Аароновото чело, когато носи Аарон нечестието на светите неща, които израилтяните ще посвещават във всичките си свети приноси; та да бъде винаги на челото му, за да бъдат те приемани пред Господа.
\par 39 И да направиш пъстротъкания хитон от висон, да направиш митра от висон и да направиш - везана работа.
\par 40 А на Аароновите синове да направиш хитони, и пояси да им направиш, и гъжви да им направиш, за слава и великолепие.
\par 41 С тия одежди да облечеш брата си Аарона и синовете му с него, и да ги помажеш, за да ги посветиш и да ги осветиш, за да Ми свещенодействуват.
\par 42 Да им направиш и ленени, гащи, които да покриват голотата на тялото им; нека покриват бедрата им от кръста надолу;
\par 43 и нека ги носят Аарон и синовете му, когато влизат в скинията за срещане, или когато пристъпват при олтара за да служат в светото място, да не би да се навлекат грях да да умрат. Това да е за вечен закон за него и за потомството му подир него.

\chapter{29}

\par 1 Ето какво да извършиш над тях, за да ги осветиш да Ми свещенодействуват. Вземи един юнец и два овена без недостатък,
\par 2 и безквасен хляб, безквасни пити месени с дървено масло, и безквасни кори намазани с масло; от чисто пшеничено брашно да ги направиш.
\par 3 Да ги туриш всичките в един кош и да ги принесеш в коша с юнеца и двата овена.
\par 4 Тогава да приведеш Аарона и синовете му при вратата на шатъра за срещане, и да ги умиеш с вода.
\par 5 После да вземеш одеждите и да облечеш Аарона с хитона, с мантията на ефода, с ефода и с нагръдника, и да го опашеш с изкусно изработената препаска на ефода,
\par 6 и да туриш митрата на главата му, и на митрата да туриш светия венец.
\par 7 Тогава да вземеш мирото за помазване и да го излееш на главата, му и тъй да го помажеш.
\par 8 После да приведеш синовете му и да ги облечеш с хитони.
\par 9 И да ги опашеш с пояси, Аарона и синовете му, и да им туриш гъжви. И свещенството ще бъде тяхно по вечен закон. Така да посветиш Аарона и синовете му.
\par 10 Тогава да приведеш юнеца пред шатъра за срещане, а Аарон и синовете му да положат ръцете си на главата на юнеца.
\par 11 И да заколиш юнеца пред Господа, при вратата на шатъра за срещане.
\par 12 После, като вземеш от кръвта на юнеца, с пръста си да туриш от нея на роговете на олтара и тогава да излееш всичката кръв при основата на олтара.
\par 13 И да вземеш всичката тлъстина, която покрива вътрешностите, и булото на дроба, и двата бъбрека с тлъстината около тях, и да ги изгориш на олтара.
\par 14 А месото на юнеца, кожата му и изверженията му да изгориш в огън вън от стана; това е жертва за грях.
\par 15 При това, да вземеш единия овен; и Аарон и синовете му да положат ръцете си на главата на овена.
\par 16 И като заколиш овена, да вземеш кръвта му та с нея да поръсиш навред олтара.
\par 17 Тогава да разсечеш овена на късове, и като измиеш вътрешностите му и нозете му, да ги сложиш върху късовете му и главата му.
\par 18 И да изгориш целия овен на олтара; това е всеизгаряне Господу, благоухание, жертва чрез огън Господу.
\par 19 След това да вземеш и другият овен; и Аарон и синовете му да положат ръцете си на главата на овена.
\par 20 Тогава да заколиш овена, да вземеш от кръвта му, и да туриш от нея на края на дясното ухо на Аарона, и на края на дясното ухо на синовете му, и на палеца на дясната им ръка и на палеца на дясната им нога, и с кръвта да поръсиш навред олтара.
\par 21 И да вземеш от кръвта, която е на олтара, и от мирото за помазване, и с тях да поръсиш Аарона и одеждите му, и синовете му и одеждите на синовете му с него; така ще се осветят той и одеждите му, и синовете му и одеждите на синовете му с него.
\par 22 После да вземеш тлъстината на овена, опашката и тлъстината; която покрива вътрешностите, булото на дроба, двата бъбрека с тлъстината, която е около тях, и дясното бедро, (защото е овен на посвещение),
\par 23 и един хляб, една пита месена с дървено масло, и една кора из коша на безквасните ястия положени пред Господа.
\par 24 Всички тия да туриш на ръката на Аарона и на ръцете на синовете му, и да ги подвижиш за движим принос пред Господа.
\par 25 Тогава да ги вземеш от ръцете им и да ги изгориш на олтара над всеизгарянето, за благоухание пред Господа; това е жертва чрез огън Господу.
\par 26 И да вземеш гърдите на овена на посвещението, което е за Аарона и да ги подвижиш за движим принос пред Господа; и това да бъде твой дял.
\par 27 И да осветиш гърдите на движимия принос и бедрото на възвишаемия принос, който се е подвижил и който се е възвисил от овена на посвещението от оня, който е за Аарона, и от оня, който е за синовете му.
\par 28 И това да бъде право на Аарона и на синовете му от израилтяните по вечен закон; защото е възвишаем принос; и ще бъде, възвишаем принос от израилтяните из примирителните им жертви, техният възвишаем принос Господу.
\par 29 И светите одежди на Аарона ще бъдат за синовете му подир него, за да бъдат помазвани в тях и освещавани в тях.
\par 30 Седем дена да се облича с тях оня от синовете му, който е свещеник, вместо него, когато влиза в шатъра за срещане, за да служи в светилището.
\par 31 Тогава да вземеш овена на посвещението и да свариш месото му на свето място.
\par 32 И Аарон и синовете му да ядат месото на овена и хляба, който е в коша, при вратата на шатъра за срещане;
\par 33 да ядат ония приноси, с които се е извършило умилостивение за тяхното посвещение и освещаване; но чужденец да не яде от тях, защото са свети.
\par 34 И ако остане до утринта нещо от месото на посвещението или от хляба, тогава да изгориш останалото в огън; да се не яде, защото е свето.
\par 35 Така, прочее, да направиш на Аарона и на синовете му според всичко що ти заповядах; седем дена ще ги посвещаваш.
\par 36 И всеки ден да принасяш по един юнец за умилостивение за грях; и да очистяш олтара като правиш умилостивение за него, и да го помажеш за да го осветиш.
\par 37 Седем дена да правиш умилостивение за олтара и да го освещаваш; и олтарът ще бъде пресвет; всичко що се докосва до олтара ще бъде свето.
\par 38 А ето какво да правиш на олтара: Всеки ден по две едногодишни агнета, винаги.
\par 39 Едното агне да принасяш заран, и другото агне да принасяш привечер;
\par 40 и с едното агне една десета от ефа чисто брашно смесено с четвърт ин първоток дървено масло, и четвърт ин вино за възлияние.
\par 41 А другото агне да принасяш привечер, и да правиш нему според стореното на утринния принос и според стореното на възлиянието му, за благоухание, жертва чрез огън Господу.
\par 42 Това да бъде във всичките ви поколения всегдашно всеизгаряне пред Господа, при вратата на шатъра за срещане, гдето ще се срещам с вас, да говоря там с тебе.
\par 43 Там ще се срещам с израилтяните; и това място ще се освещава със славата Ми.
\par 44 Ще осветя шатъра за срещане и олтара; тоже Аарона и синовете му ще осветя, за да Ми свещенодействуват.
\par 45 И ще обитавам между израилтяните и ще им бъда Бог;
\par 46 и те ще познаят, че Аз съм Иеова техният Бог, Който ги изведох из Египетската земя, за да обитавам между тях. Аз съм Иеова техният Бог.

\chapter{30}

\par 1 Да направиш олтар за кадене темян, от ситимово дърво да го направиш;
\par 2 един лакът дълъг и един лакът широк; четвъртит да бъде; и височината ме да бъде два лакътя; роговете му да са част от самия него.
\par 3 Да обковеш с чисто злато върха му, страните му наоколо, и роговете му; да му направиш и златен венец наоколо.
\par 4 А под венеца му да му направиш две златни колелца; близо при двата му ъгъла на двете му страни да ги направиш; и да бъдат влагалища на върлините, за да го носят с тях.
\par 5 Да направиш върлините от ситимово дърво, и да ги обковеш със злато.
\par 6 Тоя олтар да туриш пред завесата, която е пред ковчега с плочите на свидетелството, гдето ще се срещам с тебе.
\par 7 И всяка заран Аарон нека кади над него благовонен темян; когато приготвя светилата нека кади с него.
\par 8 И когато запали Аарон светилата вечер, нека кади с тоя темян; това ще бъде вечно кадене пред Господа във всичките ви поколения.
\par 9 На тоя олтар да не принасяш чужд темян, нито всеизгаряне, нито хлебен принос, нито да изливате на него възлияние.
\par 10 Над роговете му веднъж в годината да направи Аарон умилостивение с кръвта на умилостивителния принос за грях; веднъж в годината да прави над него умилостивение във всичките ви поколения; това е пресвето Господу.
\par 11 И Господ говори на Моисея, казвайки:
\par 12 При преброяването на израилтяните, когато вземеш цялото им число, тогава да дадеш откуп Господу, всеки човек за живота си, когато ги преброяваш, за да не ги нападне язва, когато ги преброяваш.
\par 13 Ето какво да дават: всеки, който се причислява към преброените, половин сикъл, според сикъла на светилището (един сикъл е двадесет гери); половин сикъл за принос Господу.
\par 14 Всеки, който се причислява към преброените то ест, който е от двадесет години нагоре, да даде тоя принос Господу.
\par 15 Богатият да не даде повече, и сиромахът да не даде по-малко, от половин сикъл, когато давате тоя принос Господу, за да направите умилостивение за живота си.
\par 16 А като вземеш парите за умилостивението от израилтяните, да ги употребиш в службата в шатъра за срещане; и това ще бъде за спомен на израилтяните пред Господа, за да бъде умилостивение за живота ви.
\par 17 Господ говори още на Моисея, казвайки:
\par 18 Да направиш и меден умивалник, с медна подложка, за да се мият; и да го поставиш между шатъра за срещане и олтара и да налееш вода в него,
\par 19 та Аарон и синовете му да умиват ръцете си и нозете си в него.
\par 20 Когато влизат в шатъра за срещане нека се мият с водата, за да не умират; или когато пристъпват при олтара да служат, като изгарят жертва чрез огън Господу,
\par 21 тогава да умиват ръцете си и нозете си, за да не умрат. Това ще им бъде вечен закон, за него и за потомците му във всичките им поколения.
\par 22 При това Господ говори на Моисея, казвайки:
\par 23 Вземи си изрядни аромати, от чиста смирна петстотин сикли, от благоуханна канела половината на това, сиреч, двеста и петдесет сикли от благоуханна тръст двеста и петдесет сикли,
\par 24 от касия петстотин, според сикъла на светилището, и от дървеното масло един ин;
\par 25 и да ги направиш миро за свето помазване, мас приготвена според изкуството на мироварец; това да бъде моро за свето помазване.
\par 26 И да помажеш с него шатъра за срещане, ковчега с плочите на свидетелството,
\par 27 трапезата и всичките й прибори, светилника и приборите му, кадилния олтар,
\par 28 олтара за всеизгарянето със всичките му прибори, и умивалника с подложката му;
\par 29 така да ги осветиш, за да бъдат пресвети; всичко що се докосва до тях да биде свето.
\par 30 И да помажеш Аарона и синовете му, и да ги осветиш, за да ми свещенодействуват.
\par 31 И да говориш на израилтяните, казвайки: Това ще бъде за Мене свето миро за помазване на всичките ви поколения.
\par 32 Човешка плът да се не помаже с него; и по неговия състав подобно на него да не правите; то да е свето, и свето да бъде за вас.
\par 33 Който направи подобно нему, или който тури от него на чужденец, ще бъде изтребен из людете си.
\par 34 Рече още Господ на Моисея: Вземи си аромати, - стакти, ониха, галбан, - тия аромати с чист ливан; по равни части да бъдат.
\par 35 И от тях да направиш, темян, смесен според изкуството на мироварец, подправен със сол, чист, свет.
\par 36 И да счукаш от него много дребно, и да туриш от него пред плочите на свидетелството в шатъра за срещане, гдето ще се срещам с тебе; тоя темян да ви бъде пресвет.
\par 37 А според състава на тоя темян, който ще направиш да не правите за себе си; той да ти бъде свет за Господа.
\par 38 Който направи подобен нему, за да го мирише, да бъде изтребен из людете си.

\chapter{31}

\par 1 Пак говори Господ на Моисея, казвайки:
\par 2 Виж, Аз повиках по име Веселеила сина на Урия, Оровия син, от Юдовото племе;
\par 3 и изпълних го с Божия дух в мъдрост, в разум, в знание, и във всякакво изкуство;
\par 4 за да изобретява художествени изделия, да работи злато, сребро и мед,
\par 5 и да изсича камъни за влагане, и да изрязва дърва, за изработването на всякаква работа.
\par 6 И, ето, с него Аз определих Елиава, Ахисамахов син, от Давидовото племе; и на всеки който е с мъдро сърце, Аз турих мъдрост в сърцето му, за да направят всичко що съм ти заповядал:
\par 7 шатъра за срещане, ковчега за плочите на свидетелството, умилостивилището, което е над него, и всичките принадлежности на шатъра,
\par 8 трапезата и приборите й, чисто златния светилник с всичките му прибори, и кадилния олтар,
\par 9 олтара за всеизгарянето с всичките му прибори, и умивалника с подложката му;
\par 10 служебните одежди, светите одежди на свещеника Аарона, и одеждите на синовете му, за да свещенодействуват;
\par 11 мирото за помазване, и темяна за благоуханното кадене за светилището; според всичко, що съм ти заповядал, да го направят.
\par 12 Господ говори още на Моисея, казвайки:
\par 13 Говори тъй също на израилтяните, казвайки: Съботите Ми непременно да пазите; защото това е знак между Мене и вас във всичките поколания, за да знаете че Аз съм Господ, Който ви освещавам.
\par 14 Прочее, да пазите съботата, защото ви е света: който я оскверни непременно да се умъртви; защото всеки, който работи в нея, тоя човек да се изтреби изсред людете си.
\par 15 Шест дена да се работи, а седмият ден е събота за света почивка, света Господу; всеки, който работи в съботния ден, непременно да се умъртви.
\par 16 Прочее, израилтяните да пазят съботата, като я празнуват във всичките си поколения по вечен завет.
\par 17 То е знак между Мене и израилтяните за винаги; защото в шест дена направи Господ небето и земята, а на седмия ден си почина и се успокои.
\par 18 И като свърши говоренето си с Моисея на Синайската планина, Господ му даде двете плочи на свидетелството, каменни плочи, написани с Божия пръст.

\chapter{32}

\par 1 А като видяха людете, че Моисей се забави да слезе от планината, людете се събраха срещу Аарона и му казаха: Стани, направи ни богове, които да ходят пред нас; защото тоя Моисей, чавекът, който ни изведе из Египетската земя, не знаем що му стана.
\par 2 Аарон им каза: Извадете златните обеци, които са на ушите на жените ви, на синовете ви, и на дъщерите ви, и донесете ми ги.
\par 3 И тъй, всичките люде, извадиха златните обеци, които бяха на ушите им, и донесоха ги на Аарона.
\par 4 А той, като ги взе от ръцете им, даде на златото образ с резец, след като направи леяно теле; и те рекоха: Тия са боговете ти, о Израилю, които те изведоха из Египетската земя.
\par 5 И като видя това, Аарон издигна олтар пред него; и Аарон прогласи, казвайки: Утре ще бъде празник Господу.
\par 6 И на следния ден, като станаха рано, пожертвуваха всеизгаряне и принесоха примирителни приноси: после людете седнаха да ядат и да пият, и станаха да играят.
\par 7 Тогава Господ каза на Моисея: Иди, слез, защото се развратиха твоите люде, които си извел из Египетската земя.
\par 8 Скоро се отклониха от пътя, в който им съм заповядал да ходят; направиха си леяно теле, поклониха му се, пожертвуваха му и рекоха: Тия са боговете ти, о Израилю, които те изведоха из Египетската земя,
\par 9 Рече още Господ на Моисея: Видях тия люде, и, ето, коравовратни люде са;
\par 10 сега, прочее, остави Ме, за да пламне гневът Ми против тях и да ги изтребя; а тебе ще направя велик народ.
\par 11 Тогава Моисей се помоли на Иеова своя Бог, казвайки: Господи, защо пламна гневът Ти против людете Ти, които си извел из Египетската земя с голяма сила и мощна ръка?
\par 12 Защо да говорят египтяните, казвайки: За зло ги изведе, за да ги измори в планините и да ги изтреби от лицето на земята? Повърни се от разпаления Си гняв, и разкай се за туй зло, което възнамеряваш против людете Си.
\par 13 Спомни си за слугата си Авраама, Исаака и Израиля, на които си се клел в Себе Си, като си им казал: Ще размножа потомството ви като небесните звезди, и тая цялата земя, за която говорих, ще дам на потомството ви, и те ще я наследят за винаги.
\par 14 Тогава Господ се разкая за злото, което бе казал, че ще направи на людете Си.
\par 15 И така, Моисей се обърна и слезе от планината с двете плочи на свидетелството в ръцете си, плочи написани и от двете страни; от едната страна и от другата бяха написани.
\par 16 Плочите бяха Божие дело; и написаното беше Божие писание начертано на плочите:
\par 17 А като чу Исус Навиев гласа на людете, които викаха, рече на Моисея: Боен глас има в стана.
\par 18 А той каза: Това не е глас на вик за победа, нито глас на вик за поражение; но глас на пеене чувам аз.
\par 19 И като се приближи до стана, видя телето и игрите; и пламна Моисеевия гняв, така че хвърли плочите от ръцете си, и строши ги под планината.
\par 20 Тогава взе телето, което бяха направили, изгори го с огън, и като го стри на ситен прах, разпръсна праха по водата и накара Израилевите чада да я изпият.
\par 21 После рече Моисей на Аарона: Що ти сториха тия люде та си им навлякъл голям грях?
\par 22 И Аарон каза: Да не пламне гневът на господаря ми; ти знаеш, че людете упорствуват към злото.
\par 23 Понеже ми рекоха: Направи ни богове, които да ходят пред нас; защото тоя Моисей, човекът, който ни изведе из Египетската земя, не знаем, що му стана.
\par 24 И аз им рекох: Който има злато, нека си го извади; и така те ми го дадоха. Тогава го хвърлих в огъня; и излезе това теле.
\par 25 А като видя Моисей, че людете бяха съблечени, (защото Аарон ги бе съблякъл за срам между неприятелите им),
\par 26 застана Моисей при входа на стена и рече: Който е от към Господа нека дойде при мене. И събраха се при него всичките Левийци.
\par 27 И рече им: Така говори Господ, Израилевият Бог: Препашете всички меча на бедрото си, минете насам натам от врата на врата през стана, и ибийте всеки брата си, и всеки другаря си и всеки ближния си.
\par 28 И Левийците сториха според Моисеевата дума; и в тоя ден паднаха от людете около три хиляди мъже.
\par 29 Защото Моисей беше казал: Посветете себе си днес Господу, като се дигнете всеки против сина си и против брата си, за да ви се даде днес благоволение.
\par 30 А на следния ден Моисей рече на людете: Вие сте сторили голям грях; но сега ще се възкача към Господа, дано да мога да Го умилостивя за греха ви.
\par 31 Тогава Моисей се върна при Господа и рече: Уви! тия люде сториха голям грях, че си направиха златни богове.
\par 32 Но сега, ако щеш прости греха им, - но ако не, моля Ти се, мене заличи от книгата, която си написал.
\par 33 Но Господ рече на Моисея; Който е съгрешил против Мене, него ще залича от книгата Си.
\par 34 А ти иди сега, води людете на мястото, за което съм ти говорил; ето, ангелът Ми ще ходи пред тебе; обаче в деня, когато го посетя, ще въздам върху тях наказанието на греха им.
\par 35 Така Господ порази людете, за гдето направиха телето, което Аарон изработи.

\chapter{33}

\par 1 Рече още Господ на Моисея: Иди, дигни се от тук, ти и людете, които си извел из Египетската земя, та иди в земята за която се клех на Авраама, на Исаака и на Якова, като казах: На твоето потомство ще я дам.
\par 2 И ще изпратя пред тебе ангел, и ще изгоня ханаанеца, аморееце, хетееца, ферезееца, евееца и евусееца;
\par 3 и той ще ви заведе в земя, гдето текат мляко и мед; понеже Аз няма да вървя помежду ви, (защото сте коравовратни люде), да не би да ви довърша из пътя.
\par 4 И когато чуха людете това лошо известие, скърбиха; и никой не си тури украшенията.
\par 5 Защото Господ беше рекъл на Моисея: Кажи на израилтяните: Вие сте коравовратни люде; една минута ако бих дошъл помежду ви, бих ви довършил; за това, снемете си сега украшенията си, та да видя какво ще сторя с вас.
\par 6 И тъй израилтяните махнаха украшенията си, като се отклониха от планината Хорив.
\par 7 Още Моисей взе шатъра и го постави вън от стана, далеч от стана, и нарече го Шатър за Срещане, та всеки, който търсеше Господа излизаше при шатъра за срещане, който беше вън от стана.
\par 8 И когато Моисей излезе към шатъра, всичките люде станаха та стояха, всеки при входа на шатъра си, и гледаха след Моисея догде влезе в шатъра.
\par 9 И когато влезеше Моисей в шатъра, облачният стълб слизаше и заставаше на входа на шатъра, и Господ говореше на Моисея.
\par 10 И всичките люде гледаха облачният стълб, който стоеше на входа на шатъра, и всичките люде ставаха, всеки на входа на шатъра си, та се кланяха.
\par 11 И Господ говореше на Моисея лице с лице, както човек говори с приятеля си; а, като се връщаше Моисей в стана, слугата му, младежът Исус Навиевият син, не се отдалечаваше от шатъра).
\par 12 Моисей, прочее, рече Господу: Ето, Ти ми казваш: Води тия люде; но не си ми явил, кого ще изпратиш с мене; но пак Ти си ми рекъл: Тебе познавам по име, още си придобил Моето благоволение.
\par 13 Сега, прочее, аз съм придобил Твоето благоволение, покажи ми Моля Ти се, пътя Си, за да Те позная и придобия благоволението Ти; и разсъждай, че тоя народ Твои люде са.
\par 14 И Господ каза: Самият Аз ще вървя с тебе, и Аз ще те успокоя.
\par 15 А Моисей ме рече: Ако Ти не дойдеш с мене, не ни извеждай от тука.
\par 16 Защото как ще се познае сега, че съм придобил Твоето благоволение, аз и Твоите люде? нали чрез Твоето дохождане с нас, така щото да се отделим аз и Твоите люде, от всичките люде, които са по лицето на земята.
\par 17 И Господ рече на Моисея: И това, което си рекъл, ще сторя, защото си придобил Моето благоволение и те познавам по име.
\par 18 Тогава рече Моисей: Покажи ми, моля, славата Си.
\par 19 А Господ му каза: Аз ще сторя да мине пред тебе всичката Моя благост, и ще проглася пред тебе Името Иеова; и ще покажа милост към когото ще покажа, и ще пожаля, когото ще пожаля.
\par 20 Рече още: Не можеш видя лицето Ми; защото човек не може да Ме види и да остане жив.
\par 21 Рече още Господ: Ето място при Мене; ти ще застанеш при канарата;
\par 22 и когато мине славата Ми, ще те поставя в една пукнатина на скалата, и ще те прикрия с ръката Си догде премина;
\par 23 после ще дигна ръката Си, и ще Ме видиш изотдире; но Лицето Ми няма да се види.

\chapter{34}

\par 1 След това Господ каза на Моисея: Издялай се две каменни плочи, като първите; и ще напиша на тия плочи думите, които бяха на първите плочи плочи, които ти строши.
\par 2 Бъди готов за утринта, и на утринта качи се на Синайската планина та застани пред Мене, там на върха на планината.
\par 3 Но никой да не дохожда с тебе, нито да се яви някой по цялата планина: и овците и говедата да не пасат пред тая планина.
\par 4 И така, Моисей издяла две каменни плочи, като първите; и на утринта, като стана рано, изкачи се на Синайската планина, както му заповяда Господ, и взе в ръцете си двете каменни плочи.
\par 5 И Господ слезе в облака, застана там до него, и прогласи Господното Име.
\par 6 Господ замина пред него и прогласи: Господ, Господ, Бог жалостив и милосърд, дълготърпелив, Който изобилва с милост и с вярност,
\par 7 Който пази милост за хиляди поколания, прощава беззаконие, престъпление и грях, но никак не обезвинява виновния, въздава беззаконието на бащите върху чадата и върху внуците им, до третото и до четвъртото поколение.
\par 8 Тогава Моисей бързо се наведе до земята и се поклони;
\par 9 и рече: Господи, ако съм придобил сега Твоето благоволение, нека дойде моля Господ между нас; защото тия са коравовратни люде; и прости беззаконието ни и греха ни, и вземи ни за Свое наследство.
\par 10 И Господ ме каза: Ето, Аз правя завет; пред всичките твои люде ще извърша чудеса такива, каквито не са ставали нито в един народ по целия свят; и всичките люде, между които си ти, ще видят Господното работене; защото това, което Аз ще сторя с вас, е страшно.
\par 11 Пази това, което ти заповядвам днес. Ето Аз изгонвам пред тебе аморееца, ханаанеца, хетееца, ферезееца, евееца и евусееца;
\par 12 но внимавай да не направиш договор с жителите на земята, гдето, да не би да стане примка между вас.
\par 13 Но жертвениците им да събориш, стълбовете им да строшиш и ашерите им да изсечеш;
\par 14 защото ти не бива да се кланяш на друг Бог, понеже Иеова, чието име е Ревнив, е ревнив Бог.
\par 15 Внимавай да не би да направиш договор с жителите на земята, та когато те блудствуват след боговете си, и жертвуват на боговете си, ти, ако те поканят, да ядеш от жертвите им;
\par 16 и да не вземеш от дъщерите им за синовете си, та тия техни дъщери, като блудствуват след боговете си, да направят и твоите синове да блудствуват след боговете им.
\par 17 Да не си направиш леяни богове.
\par 18 Празника на безквасните да пазиш. Седем дена да ядеш безквасен хляб, както съм ти заповядал, на отреденото време в месец Авив; защото в месец Авив ти излезе из Египет.
\par 19 Всичко, което отваря утроба, е Мое, и всяко мъжко първородено между добитъка ти, говедо или овца.
\par 20 А първороденото на осела да откупиш с агне; и ако го не откупиш, тогава да му пресечеш врата. Всичките си първородни синове да откупуваш. И никой да се не яви пред Мене с празни ръце.
\par 21 Шест дена да работиш, а в седмия ден да си почиваш: даже и във време на сеитба и на жетва да си почиваш.
\par 22 И да пазиш празника на седмиците, то ест, на първите плодове на жетвата на житото, и празника на беритбата в края на годината.
\par 23 Три пъти през годината всичките твои от мъжки пол да се явят пред Господа Иеова, Израилевия Бог.
\par 24 Защото ще изгоня пред тебе народи, и ще разширя пределите ти; и никой не ще пожелае твоята земя, когато отиваш да се явиш пред Господа твоя Бог три пъти през годината.
\par 25 Да не принасяш кръвта на жертвата Ми с квасни хлябове; нито да остане нещо от жертвата на пасхалния празник до утринта,
\par 26 Най-първите плодове от земята си да принесеш в дома на Господа твоя Бог. Да не свариш яре в млякото на майка му.
\par 27 Тогава Господ рече на Моисея: Напиши си тия думи; защото според тия думи направих Аз завет с тебе и с Израиля.
\par 28 И Моисей стоя там с Господа четиридесет дена и четиридесет нощи без да яде хляб, или да пие вода. И Господ написа на плочите думите на завета, Десетте заповеди.
\par 29 И като слизаше Моисей от Синайската планина, държащ двете плочи на откровението в ръката си, при слизането си от планината Моисей не знаеше, че кожата на лицето му блестеше понеже бе говорил с Бог.
\par 30 Но Аарон и всичките израилтяни, като видяха Моисея, че, ето, кожата на лицето му блестеше, бояха се да се приближат при него.
\par 31 За това Моисей ги повика; тогава Аарон и всичките началници на обществото се върнаха при него, и Моисей говори на тях.
\par 32 След това се приближиха и всичките израилтяни; и той им заповяда всичко що Господ му беше говорил на Синайската планина.
\par 33 И когато Моисей свърши говоренето си с тях, тури на лицето си покривало.
\par 34 Но когато влизаше пред Господа да говори с него, Моисей вдигаше покривалото догде да излезе; тогава излизаше и говореше на израилтяните, онова, което му беше заповядано.
\par 35 И израилтяните виждаха лицето на Моисея, че кожата на лицето му блестеше; а Моисей пак туряше покривалото на лицето си, догде да влезе да говори с Господа.

\chapter{35}

\par 1 Подир това Моисей събра цялото общество израилтяни и им каза: Ето какво заповяда Господ да правите.
\par 2 Шест дена да се работи; а седмият ден да ви бъде свет, събота за почивка посветена Господу; всеки, който работи в тоя ден, да се умъртви.
\par 3 В съботен ден да не кладате огън в никое от жилищата си.
\par 4 Моисей още говори на цялото общество израилтяни, казвайки: Ето какво заповяда Господ, като каза:
\par 5 Съберете помежду си принос за Господа; всеки, който е сърдечно разположен, нека принесе принос за Господа; злато, сребро и мед,
\par 6 синьо, мораво, червено, висон и козина,
\par 7 червено боядисани овчи кожи и язовски кожи, ситимово дърво,
\par 8 масло за осветление, и аромати за мирото за помазване и за благоуханното кадене,
\par 9 ониксови камъни, и камъни за влагане на ефода и на нагръдника.
\par 10 И всеки между вас, който има мъдро сърце, нека дойде, та да се направи всичко, което заповяда Господ:
\par 11 скинията, покривката й, покривалото й, куките й, дъските й, лостовете й, стълбовете й и подложките й;
\par 12 ковчега и върлините му, умилостивилището, и закривателната завеса;
\par 13 трапезата и върлините й със всичките й прибори, и хлябът за постоянно приношение;
\par 14 тоже и светилника за осветление с приборите му, светилата му, и маслото за осветление;
\par 15 кадилния олтар и върлините му, мирото за помазване, благоуханния темян, входната покривка за входа на скинията;
\par 16 олтара за всеизгарянето с медната му решетка, върлините му, и всичките му прибори, умивалника и подножието му;
\par 17 завесите за двора, стълбовете му и подложките им, и закривката за дворния вход;
\par 18 коловете за скинията и клоновете за двора с въжетата им;
\par 19 служебните одежди за служене в светилището, светите одежди за свещеника Аарона, и одеждите за синовете му, за да свещенодействуват.
\par 20 Тогава цялото общество на израилтяните си отиде от Моисеевото лице.
\par 21 И пак дойдоха, всеки човек, когото сърцето подбуждаше, и всеки, когото духа разполагаше, и донесоха принос Господу за направата на шатъра за срещане и за всяка служба, и за светите одежди.
\par 22 Дойдоха, мъже и жени, които имаха сърдечно разположение, и принесоха гривни, обеци, пръстени, мъниста и всякакви златни неща, - както и всички, които принесоха какъв да бил златен принос Господу.
\par 23 И всеки, у когото се намираше синьо, мораво, червено, висон, козина, червено боядисани овчи кожи и язовски кожи, принесоха ги.
\par 24 Всички, които можаха да направяп принос от сребро и мед, принесоха принос Господу; и всички, у които се намираше ситимово дърво, за каква да било работа на службата, принесоха ги.
\par 25 Тоже и всяка жена, която имаше мъдро сърце, предеше с ръцете си и принасяше напреденото - синьото, моравото, червеното и висона.
\par 26 Всичките жени, чието сърце ги подбуждаше, и които умееха, предяха козината.
\par 27 А началниците принесоха ониксовите камъни и камъните за влагане на ефода и на нагръдника,
\par 28 и ароматите, и маслото за осветление, и за мирото за помазване, и за благоуханния темян.
\par 29 Израилтяните принесоха доброволен принос Господу, всеки мъж и жена, които имаха сърдечно разположение да принесат за каква да било работа, която Господ чрез Моисея заповяда да се направи.
\par 30 Тогава рече Моисей на израилтяните: Вижте, Господ повика по име Веселеила, син на Урия, Оровия син, от Юдовото племе,
\par 31 и го изпълни с Божия дух в мъдрост разум, знание и всякакво изкусно работене,
\par 32 за да изобретява художествени изделия, да работи злато, сребро, мед;
\par 33 да изсича камъни за влагане, и да изрязва дърва, и да работи всяка художествена работа.
\par 34 И Той тури в неговото сърце, и в сърцето на Елиава, Ахисамаховия син, от Давидовото племе, да поучават.
\par 35 Изпълни с мъдрост сърцето им, за да работят всякаква работа на резбар, на изкусен художник, и на везач в синьо, в мораво, в червено и във висон, и на тъкач, с една дума, работа на ония, които вършат каква да било работа, и на ония, които изобретяват художествени изделия.

\chapter{36}

\par 1 Моисей каза още: Веселеил, Елиав и всеки, който умее, в чието сърце Господ е турил мъдрост и разум, за да знае да върши каква да била работа, за службата на светилището, нека работят според всичко, което Господ е заповядал.
\par 2 Тогава Моисей повика Веселеила, Елиава и всеки, който умееше, в чието сърце Господ беше турил мъдрост, всеки, когото сърцето подбуждаше да дойде при работата да я извърши;
\par 3 и те приеха от Моисея всичките приноси, които израилтяните бяха принесли, за да изработят работата за службата на светилището. А като му принасяха всяка заран още доброволни приноси,
\par 4 всичките мъдри мъже, които работеха на цялата работа, за светилището, дойдоха, всеки от работата, която вършеше,
\par 5 та говориха на Моисея, казвайки: Людете донасят много повече отколкото е нужно да служи за работата, която Господ заповяда да се върши.
\par 6 Затова Моисей заповяда та прогласиха в стана, като казаха: Никой мъж или жена да не работи вече за принос за светилището. И тъй, людете се въздържаха та не принасяха вече.
\par 7 Защото материалът, който имаха, беше им доволно, за да извършат всичката работа, и даже повече.
\par 8 И всичките изкусни мъже измежду ония, които работеха, направиха скинията от десет завеси от препреден висон и от синя, морава и червена материя, и на тях навезаха изкусно изработени херувими.
\par 9 Дължината на всяка завеса беше двадесет и осем лакътя, а широчината на всяка завеса четири лакътя; всичките завеси имаха една мярка.
\par 10 И скачи петте завеси една с друга, и другите пет завеси скачи една с друга.
\par 11 И направи сини петелки по края на оная завеса, която беше последна от първите скачени завеси; така направи и по края на последната завеса от вторите скачени завеси.
\par 12 Петдесет петелки направи на едната завеса, и петдесет петелки направи по края на завесата, която беше във вторите скачени завеси; петелките бяха една срещу друга.
\par 13 Направи и петдесет златни куки и скачи завесите една за друга с куките; така скинията стана едно цяло.
\par 14 После направи завеси от козина за покрив над скинията; единадесет такива завеси направи;
\par 15 дължината на всяка завеса бе тридесет лакътя, и широчината на всяка завеса четири лакътя; единадесетте завеси имаха една мярка.
\par 16 И скачи петте завеси отделно и шестте завеси отделно.
\par 17 И направи петдесет петелки по края на оная завеса, която беше последна от първите скачени завеси и петдесет петелки по края на завесата, която беше последна от вторите скачени завеси.
\par 18 Направи и петдесет медни куки, за да съедини покрива в едно цяло.
\par 19 И направи покрив на скинията от червено боядисани овнешки кожи, и отгоре му едно покривало от язовски кожи.
\par 20 И направи дъските на скинията от ситимово дърво, които да стоят изправени.
\par 21 Дължината на всяка дъска беше десет лакътя, и широчината на всяка дъска лакът и половина.
\par 22 Във всяка дъска имаше по два шипа, един срещу друг; така направи за всичките дъски на скинията.
\par 23 И направи дъските за скинията двадесет дъски за южната страна, към пладне;
\par 24 и под двадесетте дъски направи четиридесет сребърни подложки, две подложки под една дъска, за двата й шипа.
\par 25 Също за втората страна на скинията, която е северната, направи двадесет дъски,
\par 26 и четиридесетте им сребърни подложки две подложки под една дъска, и две подложки под друга дъска.
\par 27 А за задната страна на скинията, западната, направи шест дъски.
\par 28 И направи две дъски за ъглите на скинията от задната страна.
\par 29 Те бяха скачени отдолу, а отгоре бяха свързани посредством едно колелце; така направи и за двете дъски на двата ъгъла.
\par 30 Така бяха осем дъски, и сребърните им подложки шестнадесет подложки, по две подложки под всяка дъска.
\par 31 И направи лостове от ситимово дърво, пет за дъските от едната страна на скинията,
\par 32 пет лоста за дъските от другата страна на скинията, и пет лоста за дъските от задната страна на скинията, към запад.
\par 33 И направи средният лост да преминава през средата на дъските от край до край.
\par 34 И обкова дъските със злато, и направи колелцата им от злато за влагалища на лостовете и обкова лостовете със злато.
\par 35 И направи завесата от синьо, мораво, червено и препреден висон, и навеза на нея изкусно изработени херувими.
\par 36 И направи на нея четири стълба от ситимово дърво, които обкова със злато; куките им бяха златни; и изля за тях четири сребърни подложки.
\par 37 Направи и покривка за входа на шатъра, везана работа, от синьо, мораво, червено и препреден висон,
\par 38 и петте му стълба и куките им; и обкова върховете им и връзките им със злато; а петте им подложки бяха медни.

\chapter{37}

\par 1 И направи Веселеил ковчега от ситимово дърво, дълъг два лакътя и половина, широк лакът и половина, и лакът и половина висок.
\par 2 Обкова го отвътре и отвън с чисто злато, и направи ме златен венец наоколо.
\par 3 И изля за него четири златни колелца за четирите му долни ъгъла, две колелца на едната му страна, и две колелца на другата ме страна.
\par 4 Направи и върлини от ситимово дърво и обкова ги със злато.
\par 5 И провря върлините през колелцата от страните на ковчега за да се носи ковчегът.
\par 6 И направи умилостивилище от чисто злато, два лакътя и половина дълго, и лакът и половина широко.
\par 7 И направи два херувима от злато, изковани ги направи, на двата края на умилостивилището,
\par 8 един херувим на единия край, и един херувим на другия край; част от самото умилостивилище направи херувимите на двата му края.
\par 9 И херувимите бяха с разперени отгоре крила, и покриваха с крилата си умилостивилището; и лицата им бяха едно срещу друго; към умилостивилището бяха обърнати лицата на херувимите.
\par 10 И направи трапезата от ситимово дърво, два лакътя дълга, един лакът, широка, и лакът и половина висока.
\par 11 Обкова я с чисто злато, и направи й златен венец наоколо.
\par 12 Направи й наоколо и перваз, една длан широк, и направи златен венец около перваза й.
\par 13 И изля за нея четири златни колелца, и постави колелцата на четирите ъгъла, които бяха при четирите й нозе.
\par 14 До самия перваз бяха колелцата, като влагалища на върлините, за да се носи трапезата.
\par 15 Направи върлините от ситимово дърво, и обкова ги със злато, за да се носи трапезата с тях.
\par 16 И направи от чисто злато приборите, които бяха на трапезата, блюдата й, темянниците й, тасовете й, и поливалниците й, за употреба при възлиянията.
\par 17 И направи светилника от чисто злато; изкован направи светилника; стъблото му, клоновете му, чашките му, и цветята му бяха част от самия него.
\par 18 Шест клона се издигаха от страните му, три клона на светилника, от едната му страна, и три клона на светилника от другата му страна.
\par 19 На единия клон имаше три чашки, като бадеми, една топчица и едно цвете; така и на шестте клона, които се издаваха от светилника.
\par 20 И на стъблото на светилника имаше четири чашки като бадеми, с топчиците им и цветята им.
\par 21 И на шестте клона, които се издаваха от светилника, имаше под първите два клона една топчица, под вторите два клона една топчица и под третите два клона една топчица.
\par 22 Топчиците им и клоновете им бяха част от самия него; светилникът беше цял изкован от чисто злато.
\par 23 И направи седемте му светила, щипците му и пепелниците му от чисто злато.
\par 24 От един талант чисто злато направи него и всичките му прибори.
\par 25 И направи кадилния олтар от ситимово дърво, един лакът дълъг и един лакът широк, четвъртит; и височината му беше два лакътя; а роговете му бяха част от самия него.
\par 26 Обкова с чисто злато върха му, страните му наоколо, и роговете му; и направи му златен венец наоколо.
\par 27 А под венеца му му направи две златни колелца, близо при двата му ъгъла, на двете му страни, за да бъдат влагалища на върлините, за да го носят с тях.
\par 28 Върлините направи от ситимово дърво, и обкова ги със злато.
\par 29 И направи светото миро за помазване, и чистия благоуханен темян, според изкуството на мироварец.

\chapter{38}

\par 1 И направи олтара за всеизгаряне от ситимово дърво, пет лакътя дълъг и пет лакътя широк, четвъртит, и три лакътя висок.
\par 2 И на четирите му ъгъла направи роговете му; роговете му бяха част от самия него; и обкова го с мед.
\par 3 Направи и всичките прибори за олтара, гърнетата, лопатите, тасовете, вилиците и въглениците; всичките му прибори направи медни.
\par 4 И направи за олтара медна решетка във вид на мрежа, под полицата, която е около олтара отдолу, така щото да стигне до средата на олтара.
\par 5 Изля и четири колелца за четирите края на медната решетка, които да бъдат влагалища за върлините.
\par 6 Върлините направи от ситимово дърво, и обкова ги с мед.
\par 7 И провря върлините през колелцата от страните на олтара, за да се носи с тях. Кух, от дъски, направи олтара.
\par 8 Направи умивалника от мед, и подложката му от мед, от огледалата на жените, които се събираха при вратата на шатъра за срещане да прислужват.
\par 9 И направи двора; за южната страна, към пладне, завесите на двора бяха от препреден висон, дълги сто лакътя;
\par 10 стълбовете им бяха двадесет, и медните им подложки двадесет, а куките на стълбовете и връзките им бяха сребърни.
\par 11 И за северната страна завесите бяха сто лакътя дълги, стълбовете им двадесет, и медните им подложки двадесет; а куките на стълбовете и връзките им бяха сребърни.
\par 12 После, за западната страна завесите бяха петдесет лакътя дълги, стълбовете им десет и подложките им десет; а куките на стълбовете и и връзките им бяха сребърни.
\par 13 И за източната страна, която гледа към изток, бяха петдесет лакътя завеси;
\par 14 завесите за едната страна на входа бяха дълги петдесет лакътя, стълбовете три и подложките им три;
\par 15 така и на другата им страна; и от двете страни на дворния вход завесите бяха петнадесет лакътя дълги, стълбовете им три и подложките им три.
\par 16 Всичките завеси около двора бяха от препреден висон.
\par 17 Подложките на стълбовете бяха медни, куките на стълбовете и връзките им сребърни, и върховете им обковани със сребро.
\par 18 Покривката за дворния вход беше везана изработка от синьо, мораво, червено и препреден висон; и дължината й беше двадесет лакътя, а височината на шир пет лакътя, както завесите на двора;
\par 19 и стълбовете им бяха четири и медните им подложки четири, куките им сребърни, върховете им обковани със сребро, и връзките им сребърни.
\par 20 Всичките колчета на скинията около двора бяха медни.
\par 21 Това е сборът на вещите за скинията, сиреч, за скинията за плочите на свидетелството, както, според Моисеевото повеление, се изброиха чрез Итамара, син на свещеника Аарона, за служението на Левитите.
\par 22 Веселеил син на Урия, Оровият син, от Юдовото племе, направи всичко, което Господ заповяда на Моисея;
\par 23 и с него беше Елиав, Ахисамаховия син, от Дановото племе, резбар и изкусен художник, и везач на синьо, на мораво, на червено и на висон.
\par 24 Всичкото злато, което се употреби за изработването на цялата работа на светилището, златото на приноса, беше двадесет и девет таланта и седемстотин и тридесет сикъла, според сикъла на светилището.
\par 25 И среброто от данъка наложен върху преброените от обществото беше сто таланта и хиляда седемстотин осемдесет и пет сикъла, според сикъла на светилището, -
\par 26 данък от един веках на глава, сиреч, половин сикъл, според сикъла на светилището, за всеки, който се е причислил към преброените, то ест, ония които бяха на възраст от двадесет години и нагоре, за шестстотин и три хиляди петстотин и петдесет души.
\par 27 От среброто на стоте таланта изляха се подложките на светилището и подложките на стълбовете за завесата - сто подложки от сто таланта, - един талант за една подложка.
\par 28 И от хилядата седемстотин и пет сикъла направи куките за стълбовете, обкова върховете им и направи връзки.
\par 29 А медта на приноса беше седемдесет таланта и две хиляди и четиристотин сикъла.
\par 30 От нея направи подложките за входа на шатъра за срещане, медния олтар, медната решетка за него, с всичките олтарски прибори,
\par 31 подложките за стълбовете около двора, и подложките за дворовия вход, всичките колчета на скинията, и всичките колчета за двора наоколо.

\chapter{39}

\par 1 И от синьото, моравото, и червеното направиха служебните одежди са служене в светилището, и направиха светите одежди за Аарона, както Господ заповяда на Моисея.
\par 2 Направи ефода от злато, синьо, мораво, червено и препреден висон.
\par 3 И изковаха златото на тънки плочи, които нарязаха на тънки нишки, за да ги работят между синьото, моравото, червеното и висона, изкусна изработка.
\par 4 Направиха му презрамки, които да се връзват, за да се държи заедно на двата края,
\par 5 и препаската на ефода от същата материя и според неговата направа, от злато, синьо, мораво, червено и препреден висон, според както Господ заповяда на Моисея.
\par 6 Изработиха ониксови камъни, закрепени със златни гнездица, и изрязаха на тях, както се изрязват печати, имената на синовете на Израиля.
\par 7 И тури ги на презрамките на ефода, като камъни за спомен на израилтяните, според както Господ заповяда на Моисея.
\par 8 Направи нагръдника, според направата на ефода, изкусна изработка от злато, синьо, мораво, червено и препреден висон.
\par 9 Четвъртит беше; направиха нагръдника двоен, една педя дълъг и една педя широк, и двоен.
\par 10 И закрепиха на него четири реда камъни: ред сард, топаз и смарагд беше първият ред;
\par 11 вторият ред: антракс, сапфир и адамант,
\par 12 третият ред: лигирий, агат, аметист;
\par 13 а четвъртият ред: хрисолит, оникс и яспис; те бяха закрепени в златни гнездица на местата си.
\par 14 И камъните бяха според имената на синовете на Израиля; те бяха дванадесет според техните имена; и на всеки от тях бе изрязано, като на печат, по едно име от дванадесетте племена.
\par 15 И на нагръдника направиха венцеобразни верижки, изплетена работа от чисто злато.
\par 16 Направиха и две златни гнездица и две златни колелца, и туриха двете колелца на двата края на нагръдника.
\par 17 И провряха двете изплетени златни верижки през двете колелца по краищата на нагръдника.
\par 18 А другите два края на двете изплетени верижки ставиха с двете гнездица, и туриха ги на презрамките на ефода на външната ме страна.
\par 19 И направиха още две златни колелца, които туриха на двата края на нагръдника, на страната му, която е от вътрешната страна на ефода.
\par 20 И направиха още други две колелца, които положиха отдолу на двете страни на ефода, на външната му страна, там гдето краищата му се събират, над препаската на ефода.
\par 21 И вързаха нагръдника чрез колелцата му, за колелцата на ефода със син ширит, за да бъде над препаската на ефода, така щото нагръдникът да се не отделя от ефода, според както Господ заповяда на Моисея.
\par 22 Направи мантията на ефода, тъкана изработка цяла от синьо.
\par 23 И в средата на мантията имаше отвор, като отвора на броня, с нашивка около отвора, за да се не дере.
\par 24 По полите на мантията направи нарове от синьо, мораво, червено и препреден висон.
\par 25 И направиха звънци от чисто, злато, и туриха звънците между наровете на полите на мантията, наоколо между наровете,
\par 26 звънец и нар, звънец и нар, наоколо по полите на служебната мантия, според както Господ заповяда на Моисея.
\par 27 Направиха хитоните на Аарона и за синовете му от висон, тъкана изработка;
\par 28 и митрата от висон, великолепните гъжви от висон, ленените гащи от препреден висон;
\par 29 и пояса, везана изработка от препреден висон, синьо, мораво, червено , според както Господ заповяда на Моисея.
\par 30 Направиха плочицата на светия венец от чисто злато, и написаха на нея писмо като изрязване на печат, Свет Господу.
\par 31 И туриха и син ширит, за да я привързват отгоре на митрата, според както Господ заповяда на Моисея.
\par 32 Така се свърши всичката работа на скинията, сиреч, на шатъра за срещане; и израилтяните направиха всичко, според както Господ заповяда на Моисея; така направиха.
\par 33 Тогава донесоха скинията на Моисея, шатъра и всичките му принадлежности, куките му, дъските му, лостовете му, стълбовете му и подложките му,
\par 34 покривката от червено боядисани овнешки кожи, и покривката от язовски кожи, и покривателната завеса,
\par 35 ковчега за плочите на свидетелството, с върлините му и умилостивилището,
\par 36 трапезата, всичките й прибори, и хлябовете за приношение,
\par 37 чисто златния светилник, светилата му - светила, които трябваше да се нагледват, всичките му прибори, и маслото за осветлението,
\par 38 златния олтар, мирото за помазване, и благоуханния темян, покривката за входа на шатъра,
\par 39 медния олтар с медната му решетка, върлините му, и всичките му прибори, умивалника и подложката му,
\par 40 завесите за двора, стълбовете му, подложките му, покривката за дворния вход, въжата му, и колчетата му, и всичките прибори за служене в скинията, сиреч, в шатъра за срещане,
\par 41 служебните одежди за служене в светилището, светите одежди за свещеника Аарона и одеждите за синовете му, за да свещенодействуват.
\par 42 Напълно както Господ заповяда на Моисея, така извършиха израилтяните цялата работа.
\par 43 И Моисей видя цялата работа, и, ето, бяха я извършили, според както заповяда Господ; така бяха я извършили. И Моисей ги благослови.

\chapter{40}

\par 1 Тогава Господ говори на Моисея казвайки:
\par 2 На първия ден от първия месец да издигнеш скинията, шатъра за срещане.
\par 3 И да туриш в него ковчега за плочите не свидетелството и да закриеш ковчега с завесата.
\par 4 Да внесеш трапезата, и да наредиш на нея каквото трябва да се нарежда; да внесеш и светилника, и да запалиш светилата му.
\par 5 Да поставиш златния кадилен олтар пред ковчега за плочите на свидетелството, и да наместиш покривката за входа на скинията.
\par 6 Да туриш олтара за всеизгарянето пред входа на скинията, шатъра за срещане.
\par 7 и да туриш умивалника между шатъра за срещане и олтара, и да налееш вода в него.
\par 8 Да поставиш околния двор, и да окачиш покривката на дворния вход.
\par 9 Да вземеш мирото за помазване, и да помажеш скинията и всичко що е в нея; така да я осветиш и всичките нейни принадлежности; и ще бъде света.
\par 10 И да помажеш олтара за всеизгарянето и всичките му прибори, та да осветиш олтара; така ще бъде олтарът пресвет.
\par 11 Да помажеш и умивалника и подложката му, та да го осветиш.
\par 12 После да приведеш Аарона и синовете му пред входа на шатъра за срещане и да ги умиеш с вода;
\par 13 и да облечеш Аарона с светите одежди, да го помажеш, та да го осветиш, за да Ми свещенодействува;
\par 14 да приведеш и синовете му, да ги облечеш с хитони,
\par 15 и да ги помажеш, както си помазал баща им, за да Ми свещенодействуват. От помазването им свещенството ще бъде на тях вечно, във всичките им поколения.
\par 16 И Моисей направи всичко, според както Господ му заповяда; така направи.
\par 17 В първия месец на втората година, на първия ден от месеца, скинията се издигна.
\par 18 Моисей издигна скинията, като подложи подложките й, постави дъските й, намести лостовете й изправи стълбовете й.
\par 19 И разпростря шатъра върху скинията и тури покривалото на шатъра отгоре му, според както Господ беше заповядал на Моисея.
\par 20 И като взе плочите на свидетелството, положи ги в ковчега, и провря върлините през колелцата на ковчега, и положи умилостивилището върху ковчега.
\par 21 И внесе ковчега в скинията, и окачи покривателната завеса та с нея покри ковчега с плочите на свидетелството, според както Господ беше заповядал на Моисея.
\par 22 Положи и трапезата в шатъра за срещане откъм северната страна на скинията, отвън завесата;
\par 23 и нареди на нея хлябовете пред Господа, според както Господ беше заповядал на Моисея.
\par 24 Тури светилника в шатъра за срещане откъм южната страна на скинията, срещу трапезата;
\par 25 и запали светилата пред Господа, според както Господ беше заповядал на Моисея.
\par 26 И положи златния олтар в шатъра за срещане пред завесата;
\par 27 и накади над него с благовонен темян, според както Господ беше заповядал на Моисея.
\par 28 Окачи покривката за входа на скинията.
\par 29 Положи олтара за всеизгарянето при входа на скинията, сиреч шатъра за срещане, и принесе на него всеизгарянето и хлебния принос, според както Господ беше заповядал на Моисея.
\par 30 Положи и умивалника между шатъра за срещане и олтара и наля в него вода, за да се мият,
\par 31 (и Моисей и Аарон и синовете му миеха от него ръцете си и нозете си;
\par 32 когато влизаха в шатъра за срещане, и когато пристъпваха при олтара, миеха се, според както Господ беше заповядал на Моисея.
\par 33 И постави двора около скинията и олтара, и окачи покривката на дворния вход. Така Моисей свърши делото.
\par 34 Тогава облакът покри шатъра за срещане, и Господната слава изпълни скинията.
\par 35 Моисей не можа да влезе в шатъра за срещане, защото облакът стоеше над него и Господната слава пълнеше скинията.
\par 36 И когато облакът се дигаше от скинията, тогава израилтяните тръгваха на път, през всичките си пътувания;
\par 37 но ако облакът не се дигаше, тогава не тръгваха до деня на вдигането му.
\par 38 Защото Господният облак беше над скинията денем, а огън беше над нея нощем, пред очите на целия Израилев дом, през всичките им пътувания.

\end{document}