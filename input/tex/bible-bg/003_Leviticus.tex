\begin{document}

\title{Левит}


\chapter{1}

\par 1 И Господ повика Моисея, и като му говореше от шатъра за срещане каза:
\par 2 Говори на израилтяните, казвайки им: Когато някой от вас принесе принос Господу, от добитъка нека принесе, от чердата или от стадата.
\par 3 Ако приносът му за всеизгаряне е от чердата, нека принесе мъжко без недостатък; да го принесе над входа на шатъра за срещане, за да бъде прието от Господа.
\par 4 Да положи ръката си на главата на животното за всеизгаряне, и ще бъде прието за него, за да извърши умилостивение за него.
\par 5 После да заколи телето пред Господа, и свещениците, Аароновите синове, да принесат кръвта и да поръсят с кръвта наоколо върху олтара, който е пред входа на шатъра за срещане.
\par 6 И да одере животното за всеизгаряне и да го насече на късове.
\par 7 А синовете на свещеника Аарона да турят огън на олтара и да наредят дърва на огъня.
\par 8 И свещениците, Аароновите синове, да сложат тия късове, главата и тлъстината, на дървата, които са върху огъня на олтара;
\par 9 а вътрешностите му и нозете му да измие с вода, и свещеникът да изгори всичките на олтара, като всеизгаряне, жертва чрез огън, благоуханна Господу.
\par 10 Ако пък приносът му за всеизгаряне е от стадата, от овците или от козите, нека принесе мъжко без недостатък.
\par 11 Да го заколи пред Господа на северната страна на олтара; а свещениците, Аароновите синове, да поръсят олтара наоколо с кръвта му.
\par 12 И да го насече на късове, с главата му и тлъстините му; а свещеникът да ги нареди на дървата които са върху огъня на олтара.
\par 13 А вътрешностите и нозете да измие с вода; и свещеникът да принесе всички тия и да го изгори на олтара; това е всеизгаряне, жертва чрез огън, благоуханна Господу.
\par 14 Но ако приносът му за всеизгаряне Господу е от птиците, тогава да принесе от гургулиците или от гълъбчетата.
\par 15 Свещеникът да го донесе при олтара и, като откъсне главата му, да го изгори на олтара; а кръвта му да изцеди до страната на олтара,
\par 16 Да изтръгне гушата му с изверженията му и да ги хвърли на източната страна на олтара, към мястото за пепелта.
\par 17 И да го разчекне между крилата, но без да го разделя на две; и свещеникът да го изгори на огъня на олтара; това е всеизгаряне, жертва чрез огън, благоуханна Господу.

\chapter{2}

\par 1 Когато принесе някой Господу хлебен принос, нека бъде от чисто брашно; и да го полее с дървено масло и да тури на него ливан.
\par 2 И, като го донесе на свещениците, Аароновите синове, свещеникът да вземе една пълна шепа от чистото му брашно и от маслото му и всичкия му ливан, та да ги изгори на олтара за спомен като жертва чрез огън, благоуханна Господу.
\par 3 А останалото от хлебния принос да бъде на Аарона и на синовете му; това е пресвето измежду Господните чрез огън жертви.
\par 4 Когато принесеш хлебен принос печен в пещ, нека бъде безквасни пити от чисто брашно омесено с дървено масло, или безквасни кори намазани с масло.
\par 5 Ако пък приносът ти е хлебен принос на тава, то нека бъде безквасен, от чисто брашно омесено с дървено масло.
\par 6 Да го пречупиш на уломъци и да го полееш с масло; това е хлебен принос.
\par 7 Но ако приносът ти е хлебен принос в гърне, нека бъде от чисто брашно с дървено масло.
\par 8 Направеният от тях хлебен принос да донесеш Господу; и когато се представи на свещеника, той да го донесе при олтара.
\par 9 И свещеникът, като отдели от хлебния принос, колкото е за спомен, да го изгори на олтара като жертва чрез огън, благоуханна Господу.
\par 10 А останалото от хлебния принос да бъде и на синовете му; това е пресвето измежду Господните чрез огън жертви.
\par 11 Никакъв хлебен принос, който принасяте Господу, да се не прави с квас; защото нито квас, нито мед не бива да изгаряте в жертва Господу.
\par 12 Тях принасяйте Господу, като принос от първите плодове; но да се не изгарят на олтара за благоухание.
\par 13 И от хлебните приноси да подправяш със сол всеки свой принос; да не оставяш да липсва от хлебните приноси солта на завета на твоя Бог; с всичките си приноси да принасяш и сол.
\par 14 И ако принесеш Господу хлебен принос от първите плодове, то за хлебен принос от първите си плодове да принесеш класове пържени на огън, жито очукано от пресни класове.
\par 15 Да го полееш с дървено масло и да му туриш ливан; това е хлебен принос.
\par 16 И свещеникът да изгори от очуканото му жито и от маслото му, колкото е за спомен, заедно с всичкия му ливан; това е жертва чрез огън Господу.

\chapter{3}

\par 1 Ако приносът му е примирителна жертва и го принесе от чердата, то, било че принесе пред Господа мъжко или женско, трябва да бъде без недостатък.
\par 2 Нека положи ръката си на главата на приноса си и нека го заколи при входа на шатъра за срещане; а свещениците, Аароновите синове, да поръсят олтара наоколо с кръвта.
\par 3 И от примирителната жертва нека принесе в жертва чрез огън Господу тлъстината, която покрива вътрешностите, и всичката тлъстина, която е върху вътрешностите,
\par 4 двата бъбрека с тлъстината, която е около тях към кръста, и булото на дроба, (което ще извади до бъбреците);
\par 5 и Аароновите синове да изгорят всичко това на олтара, над всеизгарянето, което е върху дървата на огъня; това е жертва чрез огън, благоуханна Господу.
\par 6 И ако приносът му за примирителна жертва Господу е от стадото, било че принесе мъжко или женско, трябва да е без недостатък.
\par 7 Ако принесе агне, нека го принесе пред Господа.
\par 8 Като положи ръката си на главата на приноса си, нака го заколи пред шатъра за срещане; а Аароновите синове да поръсят олтара наоколо с кръвта му.
\par 9 И от примирителния принос нека принесе в жертва чрез огън Господу тлъстината му, цялата опашка, (която да извади чак от гръбнака), тлъстината, която покрива вътрешностите, и всичката тлъстина, която е върху вътрешностите,
\par 10 двата бъбрека с тлъстината, която е около тях към кръста, и булото на дроба, (което да извади до бъбреците);
\par 11 и свещеникът да ги изгори на олтара; това е храна пожертвувана чрез огън Господу.
\par 12 А ако приносът му е от коза, то да я принесе пред Господа.
\par 13 Като положи ръката си на главата й, нека я заколи пред шатъра за срещане; а Аароновите синове да поръсят олтара наоколо с кръвта й.
\par 14 И от нея нека принесе за жертва чрез огън Господу, тлъстината, която покрива вътрешностите, и всичката тлъстина, която е върху вътрешностите,
\par 15 двата бъбрека с тлъстината, която е около тях към кръста, и булото на дроба, (което да извади до бъбреците);
\par 16 и свещеникът да ги изгори на олтара. Това е храна пожертвувана чрез огън за благоухание; всичката тлъстина принадлежи на Господа.
\par 17 Вечен закон ще бъде във всичките ви поколения, във всичките ви жилища, да не ядете нито тлъстина, нито кръв.

\chapter{4}

\par 1 Господ още говори на Моисея, казвайки:
\par 2 Говори на израилтяните, като кажеш: Ако някой съгреши от незнание, като стори нещо, което Господ е заповядал да се не струва, -
\par 3 ако помазаният свещеник съгреши, така щото да се въведат людете в престъпление, тогава за греха що е сторил нека принесе Господу юнец без недостатък в принос за грях.
\par 4 Нека принесе юнеца при входа на шатъра за срещане пред Господа, нека положи ръката си на главата на юнеца и нека го заколи пред Господа.
\par 5 Тогава помазаният свещеник да вземе от кръвта на юнеца и да я принесе три шатъра за срещане;
\par 6 и свещеникът да натопи пръста си в кръвта, и от кръвта да поръси пред завесата на светилището седем пъти пред Господа.
\par 7 Свещеникът да тури от кръвта и върху роговете на олтара за благоуханното кадене, който е пред Господа в шатъра за срещане; тогава всичката кръв на юнеца да излее в подножието на олтара за всеизгаряне, който е пред входа на шатъра за срещане.
\par 8 И да извади всичката тлъстина на принесения за грях юнец: тлъстината, която покрива вътрешностите, и всичката тлъстина, която е върху вътрешностите,
\par 9 двата бъбрека с тлъстината, която е около тях към кръста, и булото на дроба (което да извади до бъбреците,
\par 10 както се изважда от юнеца, за примирителна жертва); и свещеникът да ги изгори на олтара за всеизгаряне.
\par 11 А кожата на юнеца, всичкото му месо, с главата му и с нозете му, и вътрешностите му и изверженията му,
\par 12 сиреч, целият юнец да изнесе вън от стана на чисто място, гдето се изсипва пепелта, и да го изгори на дърва с огън; гдето се изсипва пепелта, там да се изгори.
\par 13 Ако цялото общество израилтяни съгрешат от незнание, като сторят нещо, което Господ е заповядал да се не струва, та стават виновни, а това нещо се укрие от очите на обществото,
\par 14 когато се узнае грехът, който са сторили, тогава обществото да принесе юнец в принос за грях и да го приведе пред шатъра за срещане;
\par 15 и старейшините на обществото да положат ръцете си на главата на юнеца пред Господа; и да заколят юнеца пред Господа.
\par 16 Тогава помазаният свещеник да внесе от кръвта на юнеца в шатъра за срещане;
\par 17 и свещеникът като натопи пръста си в кръвта, да поръси пред завесата седем пъти пред Господа(
\par 18 Да тури от кръвта и върху роговете на олтара, който е пред Господа, в шатъра за срещане; после всичката кръв да излее в подножието на олтара за всеизгаряне, който е при входа на шатъра за срещане.
\par 19 Всичката му тлъстина да извади и да я изгори на олтара.
\par 20 И с тоя юнец да направи така както направи с оня, който беше принос за грях; така да направи и с тоя юнец; и свещеникът да направи умилостивение за тях, и ще им се прости.
\par 21 И да изнесе юнеца вън от стана, и да го изгори както изгори първия юнец; това е принос за грях за обществото.
\par 22 А когато някой първенец съгреши, като от незнание стори нещо, което Господ неговият Бог е заповядал да се не струва, та стане виновен,
\par 23 ако му се посочи греха що е сторил, то за приноса си да донесе козел без недостатък;
\par 24 и да положи ръката си на главата на козела и да го заколи на мястото, гдето колят всеизгарянето пред Господа; това е принос за грях.
\par 25 И свещеникът да вземе с пръста си от кръвта на приноса за грях и да я тури върху роговете на олтара за всеизгаряне, и тогава да излее кръвта ме в подножието на олтара за всеизгаряне;
\par 26 и всичката му тлъстина да изгори на олтара, както тлъстината на примирителната жертва; така да направи свещеникът умилостивение за него поради греха му, и ще ме се прости.
\par 27 Ако пък някой от простолюдието съгреши от незнание, като стори нещо, което Господ е заповядал да се не струва, та стане виновен,
\par 28 ако му се посочи греха, който е сторил да принесе коза без недостатък;
\par 29 и да положи ръката си на главата на приноса за грях, и да заколи приноса за грях на мястото на всеизгарянето.
\par 30 Тогава свещеникът да вземе с пръста си от кръвта му и да я тури върху роговете на олтара за всеизгаряне, и тогава да излее всичката му кръв в подножието на олтара.
\par 31 И да извади всичката му тлъстина така, както се изважда, тлъстината от примирителната жертва; и свещеникът да я изгори на олтара за благоухание Господу; така да направи свещеникът умилостивение и ще му се прости.
\par 32 Или ако принесе агне в принос за грях, то да принесе женско без недостатък;
\par 33 и да положи ръката си на главата на приноса за грях и да го заколи в принос за грях, на мястото гдето колят всеизгарянето.
\par 34 И свещеникът да вземе с пръста си от кръвта на приноса за грях и да я тури върху роговете на олтара за всеизгаряне, и тогава да излее всичката му кръв в подножието на олтара.
\par 35 И да извади всичката му тлъстина така както се изважда тлъстината от агнето на примирителната жертва; и свещеникът да ги изгори на олтара както си изгарят приносите чрез огън Господу; така да направи свещеникът умилостивение за греха що е сторил, и ще ме се прости.

\chapter{5}

\par 1 Ако някой съгреши в това, че, като е свидетел в някое дело , и чуе че го попитат с думи на клетва дали е видял или знае за работата , той не обажда, това ще носи беззаконието си.
\par 2 Или ако някой се допре до какво да е нечисто нещо, било, че е мърша на нечисто животно, или мърша на нечист, добитък, или мърша на нечиста гадина, и ако не му е известно, че е нечист, пак ще бъде виновен.
\par 3 Или ако се прикосне до човешка нечистота; или каквато и да е причината на нечистотата, чрез която някой се осквернява, и ако това е без знанието му, то щом знае за него, ще бъде виновен.
\par 4 Или ако някой се закълне и обяви несмислено с устните си, че ще направи някакво си зло или добро нещо , то каквото и да обяви човек несмислено с клетва, и бъде без знанието му, когато узнае, ще бъде виновен в едно от тях.
\par 5 А когато стане виновен в едно от тия неща, нека изповяда онова, в което е съгрешил;
\par 6 и нека принесе Господу за престъплението си, за греха, който е сторил, женско агне или яре от стадото в приноса за грях; и свещеникът да направи умилостивение за него поради греха му.
\par 7 Но ако му не стига ръка да принесе овца или коза, то за греха що е сторил нека принесе Господу две гургулици или две гълъбчета, едното в принос за грях, а другото за всеизгаряне.
\par 8 Да ги донесе на свещеника, който да принесе първо оня принос, който е за грях, като пречупи главата от шията му, но без да я откъсне;
\par 9 и от кръвта на приноса за грях да поръси страната на олтара; а останалото от кръвта да изцеди в подножието на олтара; това е принос за грях;
\par 10 и второто да направи всеизгаряне според наредбата. Така да направи свещеникът умилостивение за него поради греха що е сторил, и ще му се прости.
\par 11 Но ако му не стига ръка да донесе две гургулици или две гълъбчета, тогава съгрешилият да донесе в принос за себе си една десета от ефа чисто брашно в принос за грях; да не го полива с масло, нито да тури на него ливан, защото е принос за грях.
\par 12 Да го донесе на свещеника; и свещеникът да вземе от него една пълна шепа за спомен и да го изгори на олтара, както се изгарят приносите чрез огън Господу; това е принос за грях.
\par 13 Така да направи свещеникът умилостивение за него поради греха що е сторил в някое от тия неща, и ще му се прости; а останалото да бъде на свещеника, както хлебния принос.
\par 14 Господ още говори на Моисея, казвайки:
\par 15 Ако някой наруши закона, като от незнание съгреши относно посветените Господу вещи, то за престъплението си да принесе Господу овен от стадото без недостатък достатъчен , според твоята оценка в сребърни сикли според сикъла на светилището, за принос за престъпление;
\par 16 и в каквото е съгрешил относно посветените вещи нека го плати, и нака му притури една пета, която да даде на свещеника; и свещеникът да направи умилостивение за него чрез принесения за престъпление овен, и ще ми се прости.
\par 17 И ако някой съгреши като стори какво да било нещо, което Господ е заповядал да се не струва, макар че не го е познал, пак ще бъде виновен и ще носи беззаконието си.
\par 18 Нека принесе при свещеника овен от стадото без недостатък, достатъчен , според твоята оценка, за принос за престъпление; и свещеникът да направи умилостивение за него поради престъплението, което е извършил от незнание, и ще му се прости.
\par 19 Това е принос за престъпление; защото човекът без съмнение е виновен пред Господа.

\chapter{6}

\par 1 Господ още говори на Моисея, казвайки:
\par 2 Ако някой съгреши и направи престъпление против Господа, като излъже ближния си за нещо поверено нему, или за залог, или чрез грабеж, или като онеправдае ближния си;
\par 3 или намери изгубено нещо и излъже за него, като се закълне лъжливо; в каквото и да било нещо съгреши той между всички ония неща, които прави човек та с тях съгрешава,
\par 4 тогава, ако е съгрешил и е виновен, нека повърне това, което е вземал с грабеж, или това, което е придобил с неправда, или повереното нему нещо, или изгубеното нещо, което е намерил,
\par 5 или какво да е нещо за което се е заклел лъжливо; нека го повърне напълно и му притури една пета; в деня когато се намери виновен, нека го даде на онзи, на когото принадлежи.
\par 6 При свещеника нека принесе Господу приноса си за престъпление, овен от стадото без недостатък, достатъчен , според твоята оценка, в принос за престъпление;
\par 7 и свещеникът да направи умилостивение за него пред Господа, и ще му се прости за какво да било от всичко, което е сторил, чрез което е виновен.
\par 8 Господ още говори на Моисея, казвайки:
\par 9 Заповядай на Аарона и на синовете му, като речеш: Ето законът за всеизгарянето: всеизгарянето да гори на олтара цялата нощ до заранта, и да се направи огънят върху олтара да гори на него непрекъснато.
\par 10 Свещеникът да облече ленената си одежда и да носи ленени гащи на тялото си; и да дигне пепелта от всеизгарянето, което огънят е изгорил на олтарът, и да го тури при олтара.
\par 11 Тогава да съблече одеждите си та да облече други одежди, и да изнесе пепелта вън от стана на чисто място.
\par 12 А да се направи огънят, който е върху олтара, да гори на него непрекъснато; не бива да угасва; всяка заран свещеникът да тури дърва на него да горят, и да нарежда всеизгарянето на него, и да изгаря на него тлъстината на примирителните приноси.
\par 13 Да се направи огънят да гори непрекъснато, на олтара; не бива да угасва.
\par 14 Ето и законът за хлебния принос: Аароновите синове да го принасят пред Господа, пред олтара.
\par 15 И свещеникът , като вземе от него една пълна шепа чисто брашно от хлебния принос и от дървеното му масло, и всичкия ливан, който е на хлебния принос, да ги изгори на олтара като негов спомен, за благоухание Господу.
\par 16 А останалото от него да ядат Аарон и синовете му; безквасно да се яде на свето място; в двора на шатъра за срещане да го ядат.
\par 17 Да се не пече с квас.От Моите приноси дадох им това за дял; пресвето е, както са приносът за грях и приносът за престъпление.
\par 18 Всяко мъжко от Аароновите потомци да яде от него като свое вечно право от Господните чрез огън приноси, във всичките ви поколения; всеки, който се допре до тях, ще бъде свет.
\par 19 Господ още говори на Моисея, казвайки:
\par 20 Ето приносът, който Аарон и синовете му трябва да принасят Господу в деня, когато бъдат помазани: една десета от ефа чисто брашно за всегдашен хлебен принос, половината заран и половината вечер.
\par 21 На тава да се сготви с дървено масло; сготвено да го донесеш; във вид на печени уломъци да принесеш хлебния принос за благоухание Господу.
\par 22 Онзи от синовете му да го принесе, който ще стане помазан свещеник вместо него; вечен закон е да се изгаря цял за Господа.
\par 23 Всеки хлебен принос, принесен от свещеника, да се изгаря цял; да се не яде.
\par 24 Господ още говори на Моисея, казвайки:
\par 25 Говори на Аарона и на синовете му, като речеш, Ето законът за приноса за грях: на мястото, гдето се коли всеизгарянето, да се закаля и приноса за грях пред Господа; пресвето е.
\par 26 Свещеникът, който го принася за грях, да го яде; на свето място да се яде, в двора на шатъра за срещане.
\par 27 Всичко, което се допре до месото му ще бъде свето; и ако някоя дреха се опръска от кръвта му опръсканото да се опере на свето място.
\par 28 А пръстният съд, в който е било варено, да се строшава; но ако е било варено в меден съд, той да се затрива и да се мие с вода.
\par 29 Всеки от мъжки пол между свещеническите семейства да яде от него; то е пресвето.
\par 30 И никой принос за грях, от чиято кръв се внася в шатъра за срещане за да извърши умилостивение в светилището да се не яде; с огън да се изгаря.

\chapter{7}

\par 1 Ето и законът за приноса за престъпление; той е пресвет.
\par 2 На мястото, гдето колят всеизгарянето, да закалят и приноса за престъпление; и с кръвта му да се поръсва олтара изоколо.
\par 3 От него да се принася всичката му тлъстина: опашката, тлъстината, която покрива вътрешностите,
\par 4 двата бъбрека с тлъстината, която е около тях към кръста, и булото на дроба, (която да извади до бъбреците;)
\par 5 и свещеникът да ги изгори на олтара за жертвата чрез огън Господу; това е принос за престъпление.
\par 6 Всеки от мъжки пол между свещеническите семейства да го яде; на свето място да се яде; той е пресвет.
\par 7 Както е приносът за грях, така е и принос за престъпление; един закон да има за тях; който свещеник прави умилостивение чрез него, негов да бъде.
\par 8 А който свещеник принася някому приноса за всеизгаряне, тоя свещеник да взима за себе си кожата на всеизгарянето, което е принесъл.
\par 9 И всеки хлебен принос, който е печен в пещ, и всичко, което е сготвено в гърне или на тава, да бъде на свещеника, който го принася.
\par 10 А всеки хлебен принос, омесен с дървено масло или сух, да бъде на всичките Ааронови синове, по равен дял на всекиго.
\par 11 Ето и законът за примирителната жертва, която ще се принася Господу:
\par 12 ако я принесе някой за благодарение, то с благодарствената жертва да принесе и безквасни пити омесени с дървено масло, и с безквасни кори намазани с масло, и пити от чисто брашно омесени с масло.
\par 13 А, освен питите, с благодарствената си примирителна жертва да принесе и квасен хляб.
\par 14 И от приноса си да принесе по едно от всичките тия неща принос за издигане пред Господа; това да бъде на свещеника, който ръси с кръвта на примирителния принос.
\par 15 И месото на благодарствената му примирителна жертва да се яде същия ден, в който се принася; да не оставя той от него до утринта.
\par 16 Но ако жертвата на приноса е обречена, или е доброволен принос, то да се яде в същия ден, в който принася жертвата си; и каквото остане от него, да се яде на другия ден;
\par 17 но каквото остане от месото на жертвата до третия ден, да се изгаря в огън.
\par 18 И ако се изяде нещо от месото на примирителната жертва на третия ден, то оня, който я принася, не ще бъде приет, нито ще му се счете жертвата; отвратителна ще бъде; и оня човек, който би ял от нея, да носи беззаконието си.
\par 19 Месото, до което би се допряло нечисто нещо, да се не яде; в огън да се изгаря; а от неоскверненото месо да яде всеки, който е чист;
\par 20 но оня човек който, като има нечистота на себе си, яде от месото на Господната примирителна жертва, тоя човек ще се изтреби измежду людете си.
\par 21 Също и оня човек, който би се допрял до нечисто нещо, до човешка нечистота, или до нечисто животно, или до каква да било нечиста гнусота, и яде от месото на Господната примирителна жертва, тоя човек ще се изтреби измежду людете си.
\par 22 Господ още говори на Моисея, казвайки:
\par 23 Говори на израилтяните, като речеш: Да не ядете никаква тлъстина, нито от говедо нито от овца, нито от коза.
\par 24 Тлъстината на естествено умряло и тлъстината на разкъсано от звяр може да се употреби за всяка друга нужда, но никак да не ядете от нея.
\par 25 Защото който яде тлъстина от животно, от което се принася жертва чрез огън Господу, тоя човек, който би ял от нея, ще се изтреби измежду людете си.
\par 26 И в никое от жилищата си да не ядете никаква кръв, било от птица или от животно.
\par 27 Всеки човек, който би ял каква да е кръв, тоя човек ще се изтреби измежду людете си.
\par 28 Господ още говори на Моисея, казвайки:
\par 29 Говори на израилтяните, като речеш: Който принася примирителна жертва Господу, нека донесе приноса си Господу от примирителната си жертва.
\par 30 Със своите ръце да донесе Господните чрез огън приноси; да донесе тлъстината с гърдите, щото гърдите да се движат за движим принос пред Господа.
\par 31 А свещеникът да изгаря тлъстината на олтара; обаче гърдите да бъдат на Аарона и на синовете му.
\par 32 И дясното бедро да давате на свещеника като принос за издигане от примирителните си жертви.
\par 33 Който от Аароновите синове принесе кръвта на примирителния принос и тлъстината, да има дясното бедро за свой дял.
\par 34 Защото Аз взех от израилтяните, от примирителните им жертви, гърдите на движимия принос и бедрото на възвишаемия, и дадох ги на свещеника Аарона и на синовете му за тяхно вечно право от израилтяните.
\par 35 Това е делът от Господните чрез огън приноси за Аарона и синовете му, поради помазването им, в деня когато представи синовете си, за да свещенодействуват Господу,
\par 36 който дял Господ заповяда да им се дава от израилтяните в деня когато ги помаза. Това е тяхно вечно право във всичките им поколения.
\par 37 Това е законът за всеизгарянето, за хлебния принос, за приноса за грях, за приноса за престъпление, за посвещаванията и за примирителната жертва,
\par 38 който Господ заповяда на Моисея на Синайската планина, когато заповяда на израилтяните да принасят приносите си Господу, в Синайската пустиня.

\chapter{8}

\par 1 Господ още говори на Моисея, казвайки:
\par 2 Вземи Аарона и синовете му с него, и одеждите, мирото за помазване, юнеца на приноса за грях, двата овена, и коша с безквасните;
\par 3 и събери цялото общество при входа на шатъра за срещане.
\par 4 И Моисей стори според както му заповяда Господ; и събра се обществото пред входа на шатъра за срещане.
\par 5 Тогава рече Моисей на обществото: Ето какво заповяда Господ да направим:
\par 6 Моисей, прочее, доведе Аарона и синовете му и ги изми с вода.
\par 7 После го облече с хитона и го опаса с пояса, облече го с мантията, тури му ефода, препаса го с препаската на ефода, и го стегна с нея.
\par 8 Тогава му тури нагръдника, и в нагръдника положи Урима и Тумима.
\par 9 Положи и митрата на главата му, и отпреде на митрата закачи златната плочица, сиреч , светия венец, според както Господ бе заповядал на Моисея.
\par 10 След това Моисей взе мирото за помазване, и като помаза скинията и всичко в нея, освети ги.
\par 11 С него поръси и върху олтара седем пъти, и помаза олтара с всичките му прибори, и умивалника с подложката му, за да ги освети.
\par 12 Тогава от мирото за помазание изля на главата на Аарона та го помаза, за да го освети.
\par 13 После Моисей приведе Аароновите синове, облече ги с хитони, опаса ги с пояси, и тури им гъжви, според както Господ бе заповядал на Моисея.
\par 14 След това приведе юнеца на приноса за грях; и Аарон и синовете му положиха ръцете си на главата на принесения за грях юнец.
\par 15 И закла го; и Моисей взе кръвта и с пръстта си тури я на роговете на олтара наоколо та очисти олтара, а останалото от кръвта изля в подножието на олтара; така го освети за да направи умилостивение за него.
\par 16 После взе всичката тлъстина, която е върху вътрешностите, булото на дроб, и двата бъбрека с тлъстината им, и Моисей ги изгори на олтара.
\par 17 А юнеца, кожата му, месото му, и изверженията му изгори на огън вън от стана, според както Господ бе заповядал на Моисея.
\par 18 След това приведе овена за всеизгаряне; и Аарон и синовете му положиха ръцете си на главата на овена.
\par 19 И закла го; и Моисей поръси олтара наоколо с кръвта.
\par 20 И насече овена на късове; и Моисей изгори главата, късовете, и тлъстината.
\par 21 А вътрешностите и нозете изми с вода; и Моисей изгори на олтара целия овен; това беше всеизгаряне за благоухание, жертва чрез огън Господу, според както Господ бе заповядал на Моисея.
\par 22 Тогава приведе другия овен, овенът на посвещението; и Аарон и синовете му положиха ръцете си на главата на овена.
\par 23 И закла го; и Моисей взе от кръвта му и тури я на края на дясното ухо на Аарона, на палеца на дясната му ръка, и на палеца на дясната му нога.
\par 24 Приведе и Аароновите синове; и Моисей тури от кръвта на края на дясното им ухо, на палеца на дясната им ръка, и на палеца на дясната им нога; и с кръвта Моисей поръси олтара наоколо.
\par 25 После взе тлъстината и опашката, всичката тлъстина върху вътрешностите, булото на дроба, двата бъбрека с тлъстината им, и дясното бедро;
\par 26 взе и от коша с безквасните, който беше пред Господа, една безквасна пита, един хляб омесен с дървено масло, и една кора, та ги тури на тлъстината и на дясното бедро;
\par 27 и като тури всичкото това в ръцете на Аарона и в ръцете на синовете му, подвижи ги за движим принос пред Господа.
\par 28 После Моисей ги взе от ръцете им и ги изгори на олтара върху всеизгарянето; това беше жертва на посвещение за благоухание; това беше жертва чрез огън Господу.
\par 29 Моисей взе и гредите та ги подвижи за движим принос пред Господа; това беше Моисеевия дял от овена на посвещаването, според както Господ беше заповядал на Моисея.
\par 30 После Моисей взе от мирото за помазване и от кръвта, която беше върху олтара, та поръси Аарона и одеждите му, и синовете му и одеждите на синовете му, с него; така освети Аарона и одеждите му, и синовете му и одеждите на синовете му с него.
\par 31 Тогава каза Моисей на Аарона и на синовете му: Сварете месото при входа на шатъра за срещане, и там го яжте с хляба, който е в коша на посвещаването, както ми биде заповядано когато Господ ми рече: Аарон и синовете му да ги ядат.
\par 32 А колкото остава от месото и от хляба да изгорите в огъня.
\par 33 И да не излизате от входа на шатъра за срещане за седем дена, преди да са се изпълнили дните на посвещаването ви; защото през седем дена ще става посвещаването ви.
\par 34 Според както Господ заповяда да се върши, така и е било извършено днес, за да стане умилостивение за вас.
\par 35 И през тия седем дена да седите пред входа на шатъра за срещане, денем и нощем, та да пазите Господните заръчвания, за да не умрете; защото така ми биде заповядано.
\par 36 И Аарон и синовете му извършиха всичко що Господ заповяда чрез Моисея.

\chapter{9}

\par 1 На осмия ден Моисей повика Аарона и синовете му и Израилевите старейшини;
\par 2 и рече на Аарона: Вземи си мъжко теле в принос за грях, и овен за всеизгаряне, без недостатък, та ги принеси пред Господа.
\par 3 И да говориш на израилтяните, казвайки: Вземете козел в принос за грях, и за всеизгаряне теле и агне, едногодишни, без недостатък,
\par 4 и за примирителен принос юнец и овен, за да ги пожертвувате пред Господа, и хлебен принос омесен с дървено масло; защото днес ще ви се яви Господ.
\par 5 И тъй, донесоха пред шатъра за срещане онова, което заповяда Моисей; и цялото общество се приближи и застана пред Господа.
\par 6 И Моисей каза: Това е, което Господ заповяда да направите; и Господната слава ще ви се яви.
\par 7 Тогава рече Моисей на Аарона: Пристъпи при олтара та принеси приноса си за грях и всеизгарянето си, и направи умилостивение за себе си и за людете; принеси и приноса за людете и направи умилостивение за тях, според както заповяда Господ.
\par 8 Аарон, прочее, пристъпи при олтара та закла телето на приноса за грях, което беше за него.
\par 9 А синовете на Аарона му донесоха кръвта; и той след като натопи пръста си в кръвта и я тури върху роговете на олтара, изля кръвта в подножието на олтара;
\par 10 а тлъстината, бъбреците, и булото на дроба от приноса за грях изгори на олтара, според както Господ беше заповядал на Моисея.
\par 11 А месото и кожата изгори на огън вън от стана.
\par 12 Закла и всеизгарянето; и синовете на Аарона му представиха кръвта, с която поръси олтарът наоколо.
\par 13 Тогава му донесоха всеизгарянето, къс по къс, с главата; и ги изгори на олтара;
\par 14 и изми вътрешностите и нозете, и изгори ги на олтара, върху всеизгарянето.
\par 15 Тогава принесе приноса за людете, като взе козела на приноса за грях, който беше за людете, закла го, и го принесе за грях, както и първото.
\par 16 И представи всеизгарянето и го принесе според наредбата.
\par 17 Принесе и хлебния принос, напълни ръката си от него, и го изгори на олтара, освен утринното всеизгаряне.
\par 18 Закла още юнеца и овена на примирителната жертва, която беше за людете; и синовете на Аарона му представиха кръвта, (с която поръси олтарът наоколо),
\par 19 и тлъстините от юнеца, а от овена - опашката, тлъстината , която покрива вътрешностите , бъбреците и булото на дроба;
\par 20 и като туриха тлъстините върху гърдите, той изгори тлъстините върху олтара.
\par 21 А гърдите и дясното бедро подвижи Аарон за движим принос пред Господа, според както Моисей беше заповядал.
\par 22 Тогава Аарон подигна ръцете си към людете и ги благослови; и, както беше вече принесъл приноса за грях, всеизгарянето, и примирителните приноси, слезе.
\par 23 И Моисей и Аарон, като влязоха в шатъра за срещане, при излизането си благословиха людете; и Господната слава се яви на всичките люде.
\par 24 И огън излезе от пред Господа и пояде всеизгарянето и тлъстините върху олтара; и като видяха това, всичките люде издадоха силен вик и паднаха на лице.

\chapter{10}

\par 1 А Аароновите синове, Надав и Авиуд, взеха всеки кадилницата си, и като туриха в тях огън и на него туриха темян, принесоха чужд огън пред Господа, - нещо което им беше запретил.
\par 2 За това огън излезе от пред Господа и ги пояде; и умряха пред Господа.
\par 3 Тогава рече Моисей на Аарона: Това е, което говори Господ, като каза: Аз ще се осветя в ония, които се приближават при Мене, и ще се прославя пред всичките люде. А Аарон мълчеше.
\par 4 И Моисей повика Мисаила и Елисафана, синовете на Аароновия стрика Озиил и рече им: Пристъпете, вдигнете братята си отпред светилището и изнесете ги вън от стана.
\par 5 Те, прочее, пристъпиха и ги изнесоха с хитоните им вън от стана, според както рече Моисей.
\par 6 Тогава каза Моисей на Аарона и на синовете му Елеазара и Итамара: Главите си да не откриете, и дрехите си да не раздерете, за да не умрете и да не дойде гняв на цялото общество; но за изгарянето, което подклади Господ, нека плачат братята ви, целият Израилев дом.
\par 7 И да не излезете от входа на шатъра за срещане, за да не умрете; защото мирото за Господното помазване е на вас. И те направиха според както каза Моисей.
\par 8 След това Господ говори на Аарона, казвайки:
\par 9 Когато влизате в шатъра за срещане, да не пиете вино или спиртни питиета, ни ти ни синовете ти с тебе, за да не умрете; това ще бъде вечен закон във всичките ви поколения,
\par 10 както за да разпознавате между свето и мръсно и между нечисто и чисто,
\par 11 така и за да учите израилтяните всичките повеления, които Господ им е говорил чрез Моисея.
\par 12 После Моисей каза на Аарона и на останалите му синове Елеазара и Итамара: Вземете хлебния принос, който е останал от Господните чрез огън жертви, и яжте го безквасен при олтара, защото е пресвет,
\par 13 и трябва да го ядете на свето място; понеже това е твоето право и правото на синовете ти от Господните чрез огън жертви; защото така ми биде заповядано.
\par 14 Също и гърдите на движимия принос и бедрото този за издигане трябва да ядете на чисто място, ти, синовете ти и дъщерите ти с тебе; защото това е дадено като твое право и право на синовете ти от жертвите на примирителните приноси на израилтяните.
\par 15 Бедрото на приноса за издигане и гърдите на движимия да донасят заедно с жертвите чрез огън на тлъстината, за да ги подвижат за движим принос пред Господа; и ще бъде твое вечно право и право на синовете ти с тебе, според както заповяда Господ.
\par 16 И Моисей търсеше грижливо козела на приноса за грях, но ето че беше изгорен; за това той се разгневи на Елеазара и Итамара, Аароновите останали синове, и рече:
\par 17 Защо не ядохте приноса за грях на светото място, тъй като е пресвето, и ви е дадено за да отнемате беззаконието на обществото и да правите умилостивение за тях пред Господа?
\par 18 Ето, кръвта му не се внесе вътре в светилището. Трябваше непременно да го ядете в светилището, както заповядах.
\par 19 Но Аарон каза на Моисея: Ето, те принесоха днес приноса си за грях и всеизгарянето си пред Господа; и, като ми се случиха такива работи, ако бях ял днес приноса за грях, щеше ли да се види угодно на Господа?
\par 20 И като чу това Моисей видя му се задоволително.

\chapter{11}

\par 1 И Господ говори на Моисея и Аарона, като им каза:
\par 2 Говорете на израилтяните, казвайки: Ето животните, които можете да ядете измежду всичките животни, които са по земята.
\par 3 Измежду животните всяко що има раздвоени копита и е с разцепени копита, и преживя, него да ядете.
\par 4 Обаче от ония, които преживят, или от ония, които имат раздвоени копита, да не ядете следните: камилата, защото преживя, но няма раздвоени копита; тя е нечиста за вас;
\par 5 питомния заек, защото преживя, но няма раздвоени копита; той е нечист за вас;
\par 6 дивия заек, защото преживя, но няма раздвоени копита; той е нечист за вас;
\par 7 и свинята, защото има раздвоени копита, и е с разцепени копита, но не преживя; тя е нечиста за вас;
\par 8 От тяхното месо да не ядете, и до мършата им да не се допирате; те са нечисти за вас.
\par 9 Измежду всичките, които са във водите, да ядете следните: всички във водите, които имат перки и люспи, в моретата и в реките, тях да ядете.
\par 10 А измежду всичко, което се движи във водите, и измежду всяко одушевено животно, което е във водите, всички в моретата и в реките, които нямат перки и люспи, те са отвратителни за вас.
\par 11 Непременно да бъдат отвратителни за вас; от месото им да не ядете, и от мършата им да се отвращавате.
\par 12 Всичко във водите, което няма ни перки, ни люспи, да бъде отвратително за вас.
\par 13 Измежду птиците да се отвращавате от следните; да се не ядат, понеже са отвратителни: орелът, грифата, морският орел,
\par 14 пилякът, соколът по видовете му,
\par 15 всяка врана по видовете й,
\par 16 камилоптицата, бухалът, кукувицата, ястребът по видовете му
\par 17 малкият бухал, рибарят, ибисът
\par 18 лебедът, пеликанът, лешоядът,
\par 19 щъркът, цаплята по видовете й, папунякът и прилепът.
\par 20 Всички крилати пълзящи, които ходят на четири нозе, да бъдат отвратителни за вас.
\par 21 Обаче измежду всичките пълзящи крилати, можете да ядете ония, които като ходят на четири нозе, имат пищяли над нозете си за да скачат с тях по земята.
\par 22 Измежду тях можете да ядете следните: скакалеца по видовете му, солама по видовете му, харгола по видовете му, и акридата по видовете й.
\par 23 А всички други крилати пълзящи, които имат четири нозе, да бъдат отвратителни за вас.
\par 24 От тях ще бъдат нечисти: всеки, който се допира до мършата им, ще бъде нечист до вечерта;
\par 25 и всеки, който понесе нещо от мършата им, да изпере дрехите си, и да бъде нечист до вечерта.
\par 26 Всяко животно що има раздвоени копита, но не е с разцепени копита нито преживя, е нечисто за вас; всеки, който се допира до такова, ще бъде нечист.
\par 27 И измежду всичките четвероноги животни, ония, които ходят на лапите си, ще бъдат нечисти за вас; всеки, който се допира до мършата им, ще бъде нечист до вечерта.
\par 28 И който понесе мършата им нека изпере дрехите си, и да бъде нечист до вечерта; те са нечисти за вас.
\par 29 Измежду пълзящите, които пълзят по земята, следните да бъдат нечисти за вас: невестулката, мишката, гущерът по видовете му,
\par 30 ящерът, ящерицата, саврата, самиамитът, и хамелионът.
\par 31 Тия са, които са нечисти за вас измежду всичките пълзящи; всеки, който се допира до мършата им, ще бъде нечист до вечерта.
\par 32 И всяко нещо, върху което те биха паднали мъртви, ще бъде нечисто: било дървен съд, дреха, кожа, вретище, или какъв да е съд, който се употребява в работа, всеки трябва да тури във вода, и ще бъде нечист до вечерта; тогава ще бъде чист.
\par 33 И ако някое от тях падне в някой пръстен съд, то всичко що е вътре в него ще бъде нечисто; а него да строшите
\par 34 Всяка храна в него , която се яде, върху която се сипва вода, когато се сготви , ще бъде нечиста; и всяко питие, което се пие, във всеки такъв съд ще бъде нечисто.
\par 35 Каквото и да било нещо, върху което би паднало нещо от мършата им, ще бъде нечисто; било пещ или огнище, трябва да се строшат; нечисти са и нечисти ще бъдат за вас.
\par 36 Обаче извор или кладенец, гдето има събрана вода, ще си бъде чист; но каквото се допре до мършата на тия животни ще бъде нечисто.
\par 37 Но ако падне нещо от мършата им върху някое семе за сеене, което ще се посее, то си е чисто
\par 38 Обаче, ако са полели семето с вода, и падне нещо от мършата им на него, тогава е нечисто за вас.
\par 39 Ако умре някое от животните, които бива да ядете, който се допре до мършата му, ще бъде нечист до вечерта.
\par 40 И който яде от мършата му трябва да изпере дрехите си, и ще бъде нечист до вечерта; и който понесе мършата му нека изпере дрехите си, и ще бъде нечист до вечерта.
\par 41 Всяка гадина, която пълзи по земята, е отвратителна; да ги не ядете.
\par 42 Всичко, което се влачи по корема си, и всичко, което ходи на четири нозе, или всичко, което има много нозе, сиреч, всички гадини, които пълзят по земята, - тях да не ядете,защото са отвратителни.
\par 43 Да се не омърсите с никакви пълзящи гадини, нито да се оскверните с тях, та да бъдете нечисти чрез тях.
\par 44 Защото Аз съм Иеова вашият Бог; осветете се, прочее, и бъдете свети, понеже Аз съм свет; и да се не оскверните с никаква гадина пълзяща по земята.
\par 45 Защото Аз съм Господ, който ви изведох из Египетската земя за да ви бъда Бог; бъдете, прочее, свети, защото Аз съм свет.
\par 46 Това е законът за животните, за птиците, за всяко одушевено, което се движи във водите, и за всяко одушевено, което пълзи по земята,
\par 47 за да правите разлика между чистото и нечистото, и между одушевеното, което бива да се яде, и одушевеното, което не бива да се яде.

\chapter{12}

\par 1 И Господ говори на Моисея, казвайки:
\par 2 Говори на израилтяните, като речеш: Ако зачне жена и роди мъжко дете, тя ще бъде нечиста седем дни; както в дните, когато е разлъчана поради месечините си, ще бъде нечиста.
\par 3 А на осмия ден да се обреже краекожието на детето .
\par 4 Но тя да остава тридесет и три дена за очистването си от кръвта си; до никоя света вещ да се не допира, и в светилището да не влиза, преди да се свършат дните на очистването й.
\par 5 Но ако роди женско дете, тя ще бъде нечиста две седмици, както когато е разлъчана; и да остава шестдесет и шест дена за очистването си от кръвта си.
\par 6 А когато се свършат дните на очистването й, за син или за дъщеря, нека донесе при свещеника едногодишно агне за всеизгаряне, и гълъбче или гургулица в принос за грях, до входа на шатъра за срещане;
\par 7 и той да го принесе пред Господа и да направи умилостивение за нея, и тя ще бъде чиста от кръвта си. Това е законът за оная, която ражда мъжко или женско дете.
\par 8 Но ако не й стига ръка да донесе агне, нека донесе две гургулици или две гълъбчета, едното за всеизгаряне, а другото в принос за грях; и свещеникът да направи умилостивение за нея, и тя ще бъде чиста.

\chapter{13}

\par 1 Господ още говори на Моисея и на Аарона, казвайки:
\par 2 Когато човек има по кожата на месата си оток, или краста, или лъскаво петно, и се обърне на рана от проказа върху кожата на месата му, тогава да се заведе при свещеника Аарона, или при един от свещениците, негови синове.
\par 3 И свещеникът да прегледа раната върху кожата на месата: ако влакната на раната са побелели, и раната се вижда по-дълбоко от кожата на месата му, това е рана от проказа; и свещеникът, като го прегледа, нека го обяви за нечист.
\par 4 Но ако лъскавото петно върху кожата на месата му е бяло, и не изглежда да е по-дълбоко от кожата, и влакната му не са побелели, тогава свещеникът да затвори за седем деня онзи, който има раната.
\par 5 А на седмия ден свещеникът да го прегледа; и, ето, ако види, че раната е в застой, и раната не се е разпростряла по кожата, тогава свещеникът да го затвори за още седем дена;
\par 6 и на седмия ден свещеникът пак да прегледа; и, ето, ако раната е завехнала, и раната на се е разпростряла по кожата, тогава свещеникът да го обяви за чист; това е краста; болният нека изпере дрехите си, и ще бъде чист.
\par 7 Но ако, подир явяването му пред свещеника за очистването си, крастата се разпростира много по кожата, той трябва да се яви пак при свещеника,
\par 8 и ако види свещеникът, че крастата се е разпростряла по кожата, тогава свещеникът да го обяви за нечист; това е проказа.
\par 9 Когато човек има рана от проказа, нека бъде заведен при свещеника;
\par 10 и свещеникът да прегледа, и, ето, ако има бял оток на кожата и влакната са побелели, и има живо месо в отока,
\par 11 тогава е стара проказа по кожата на месата му; свещеникът трябва да го обяви за нечист; да не го затвори, защото е нечист.
\par 12 Но ако, до колкото се вижда на свещеника, проказата се е разпростряла много по кожата, и проказата е покрила цялата кожа на болния с раната, от главата до нозете му,
\par 13 тогава свещеникът да прегледа, и, ето, ако проказата е покрила всичките му меса, нека обяви за чист, онзи, който има раната; понеже тя цяла е побеляла, той е чист.
\par 14 Но щом се появи на него живо месо, ще бъде нечист;
\par 15 свещеникът, като прегледа живото месо, нека го обяви за нечист; живото месо е нечисто; това е проказа.
\par 16 Или пък ако живото месо пак се промени и побелее, нека дойде при свещеника;
\par 17 и, ето, когато го прегледа свещеникът, ако раната е побеляла, тогава свещеникът да обяви за чист онзи, който има раната; той е чист.
\par 18 Ако на кожата на някого по месата му има цирей, който е оздравял,
\par 19 и ако на мястото на цирея се е появил бял оток, или лъскаво бяло червеникаво петно, такъв да се покаже на свещеника;
\par 20 и свещеникът да прегледа, и ето, ако изглежда да е по-дълбоко от кожата, и влакната му са побелели, тогова свещеникът да го обяви за нечист: това е рана от проказа, която е избухнала в цирея.
\par 21 Но ако, като я прегледа свещеникът, ето, няма бели влакна в нея, и не е по-дълбоко от кожата, и е завехнала, тогава свещеникът да го затвори за седем дена;
\par 22 и ако се разпростира много по кожата, тогава свещеникът да го обяви за нечист; това е язва
\par 23 Но ако лъскавото петно остава на мястото си, и не се е разпростряло, това е белег от цирея; свещеникът нека го обяви за чист.
\par 24 Или, ако на кожата на месата на някого има място изгорено от огън, и живото месо на изгореното стане лъскаво, бяло червеникаво, или бяло петно,
\par 25 тогава свещеникът да го прегледа, и, ето, ако влакната на лъскавото петно са побелели, и то изглежда да е по-дълбоко от кожата, това е проказа, която е избухнала в изгореното; свещеникът да го обяви за нечист; това е язва на проказа.
\par 26 Но ако, като го прегледа свещеникът, ето, няма бели влакна в лъскавото петно, и не е по-дълбоко от кожата, но е завехнало, тогава свещеникът да го затвори за седем дена;
\par 27 и не седмия ден свещеникът да го прегледа; ако то се е разпростряло много по кожата, тогава свещеникът да го обяви за нечист; това е язва на проказа.
\par 28 Но ако лъскавото петно си стои на мястото, и не се е разпростряло по кожата, но е завехнало, това е оток от изгореното; свещеникът да го обяви за чист, понеже е белег от изгореното.
\par 29 Ако мъж или жена има рана на главата или на брадата,
\par 30 то свещеникът да прегледа раната; и, ето, ако тя изглежда да е по-дълбоко от кожата, и в нея има тънки руси влакна, то свещеникът да го обяви за нечист; това е кел, проказа на главата или на брадата.
\par 31 Но ако, като прегледа свещеникът раната от кела, ето, тя не изглежда да е по-дълбока от кожата, и няма черни влакна по нея, тогава свещеникът да затвори за седем дена онзи, който има раната от кела;
\par 32 и на седмия ден свещеникът да прегледа раната; и, ето, ако не се е разпрострял келът, и няма в него руси влакна, и келът не изглежда да е по-дълбоко от кожата,
\par 33 такъв да се обръсне, но без да обръсне келевото, и свещеникът да затвори келевия още седем дена.
\par 34 А на седмия ден свещеникът да прегледа кела; и, ето, ако келът не се е разпрострял по кожата, и изглежда да не е по-дълбок от кожата, тогава свещеникът да го обяви за чист; и такъв да изпере дрехите си, и ще бъде чист.
\par 35 Обаче, ако подир очистването му, келът се разпростре много по кожата,
\par 36 тогава свещеникът да го прегледа; и ако келът се е разпрострял по кожата, свещеникът да не търси руси влакна, човекът е нечист.
\par 37 Но ако му се вижда, че келът е в застой, и черни влакна са поникнали по него, келът е оздравял, човекът е чист; свещеникът да го обяви за чист.
\par 38 Ако мъж или жена има по кожата на месата си бели лъскави петна,
\par 39 то свещеникът да прегледа; и, ето, ако лъскавите петна по кожата на месата им са белезникави, лишай е избухнал по кожата; човекът си е чист.
\par 40 Ако някому е окапала косата от главата, такъв е плешив, но пак е чист.
\par 41 Ако е окапала косата му от предната част на главата му, той е полуплешив, но пак е чист.
\par 42 Но ако по плешивото или по полуплешивото място има бяла, червеникава рана, това е проказа, която е избухнала в плешивото или полуплешивото му място.
\par 43 Свещеникът нека го прегледа; и, ето, ако отокът на раната в плешивото или полуплешивото му място е бял червеникав, подобен на изгледа на проказата по кожата на месата,
\par 44 той е прокажен човек, нечист е; свещеникът непременно да го обяви за нечист; той има раната на главата си.
\par 45 Дрехите на прокажения, който има тая рана, да се раздерат, и главата му да бъде непокрита, а нека покрие устните си и нека вика: Нечист! нечист!
\par 46 През всичкото време до когато бъде раната на него той ще бъде нечист; нечист е той; нека седи отлъчен; вън от стана да бъде жилището му.
\par 47 Ако заразата на проказата бъде в дреха, във вълнена дреха или в ленена дреха,
\par 48 било в основата или в вътъка, ленен или вълнен, или в кожа, или в какво да е кожено нещо,
\par 49 и ако заразата е зеленикава или червеникава в дрехата, или в кожата, било в основата или във вътъка, или в коя да е кожена вещ, това е зараза от проказа и трябва да се покаже на свещеника.
\par 50 Свещеникът нека прегледа заразата и затвори за седем дена онова, което има заразата;
\par 51 а на седмия ден да прегледа заразата; ако заразата се е разпростряла по дрехата, било по основата или по вътъка, или по кожата, в кожата, заразата е люта проказа; това е нечисто.
\par 52 Да изгори дрехата, в която е заразата, било в основата или във вътъка, вълнена дреха или ленена, или каква да е заразена кожена вещ; защото е люта проказа; с огън да се изгори.
\par 53 Но ако, като прегледа свещеникът, ето, заразата не се е разпростряла по дрехата, било по основата, или по вътъка, или по каква да е кожена вещ,
\par 54 то свещеникът да заповяда да изперат онова, което е заразено, и да го затвори още за седем дена.
\par 55 Тогава, след изпирането, нека прегледа свещеникът заразата; ако заразата не се е разпростряла, тая вещ е нечиста; с огън да я изгориш; проказата е люта, била тя от вътрешната страна или от външната.
\par 56 Но ако прегледа свещеникът, и, ето, след изпирането, заразата е завехнала, то да я отдере от дрехата, или от кожата, било от основата или от вътъка.
\par 57 Ако пък заразата още се яви на дрехата, било на основата или вътъка, или на кой да е кожена вещ, проказа е избухнала; с огън да изгориш онова, което е заразено.
\par 58 И ако заразата се заличи от дрехата, било основата или вътъка, или от коя да е кожена вещ, която ще изпереш, тогава нека се изпере още веднъж, и ще бъде чиста.
\par 59 Това е законът за заразата на проказата във вълнена или ленена дреха, било в основата или вътъка, или в коя да е кожена вещ, според който да се обявява за чиста или нечиста.

\chapter{14}

\par 1 Господ говори още на Моисея, казвайки:
\par 2 Ето законът за прокажения в деня, когато се очиства: да се заведе при свещеника;
\par 3 и свещеникът, като излезе вън от стана да прегледа; и ако прокаженият е оздравял от раната на проказата,
\par 4 тогава свещеникът да заповяда да вземат за очищаемия две живи чисти птичета, кедрово дърво, червена вълна и исоп;
\par 5 и свещеникът да заповяда да заколят едното птиче в пръстен съд над текуща вода.
\par 6 После да вземе живото птиче, кедровото дърво, червената вълна и исопа и да натопи тях и живото птиче в кръвта на закланото над текущата вода птиче;
\par 7 и седем пъти да поръси очищаемия от проказата и да го обяви за чист, а да пусне живото птиче на полето.
\par 8 Тогава очищаемия да изпере дрехите си, да обръсне всичката си коса и да се окъпе във вода и ще бъде чист. След това нека дойде в стана, но да живее вън от шатъра си седем дена.
\par 9 И на седмия ден да да обръсне всичката коса на главата си, брадата си и веждите си; всичките си косми да обръсне, и за изпере дрехите си, и да окъпе тялото си във вода, и ще бъде чист.
\par 10 А на осмия ден да вземе две мъжки агнета без недостатък, и едно едногодишно женско агне, и три десети от ефа чисто брашно за хлебен принос омесен с дървено масло, и един лог дървено масло;
\par 11 и свещеникът, който го очиства, да представи очищаемия и тия неща пред Господа на входа на шатъра за срещане.
\par 12 Тогава свещеникът да вземе едното мъжко агне и да го принесе в принос за престъпление, и лога дървено масло, и да ги подвижи за движим принос пред Господа;
\par 13 и да заколи агнето на мястото, гдето колят приноса за грях и всеизгарянето, на свето място; защото приносът за престъпление, както и приносът за грях, принадлежи на свещеника; той е пресвет.
\par 14 После свещеникът да вземе от кръвта на приноса за престъпление, и свещеникът да я тури на края на дясното ухо на очищаемия, на палеца на дясната му ръка, и на палеца на дясната му нога.
\par 15 След това, свещеникът да вземе от лога дървено масло и да го излее в дланта на лявата си ръка;
\par 16 и свещеникът да натопи десния си пръст в маслото, което има в лявата си длан, и с пръста си седем пъти да поръси от маслото пред Господа;
\par 17 и от останалото масло, което има в дланта си, свещеникът да тури края на дясното ухо на очищаемия, на палеца на дясната му ръка, и на палеца на дясната му нога, върху кръвта на приноса за престъпление;
\par 18 а останалото масло, което е в дланта на свещеника, да тури на главата на очищаемия, и свещеникът да направи умилостивение за него пред Господа.
\par 19 Тогава свещеникът да принесе приноса за грях и да направи умилостивение за очищаемия от нечистотата му и после да заколи и всеизгарянето;
\par 20 и свещеникът да принесе всеизгарянето и хлебния принос на олтара; така да направи свещеникът умилостивение за него и ще бъде чист.
\par 21 Но ако очищаемият е сиромах та не му стига ръка, то нека вземе едно мъжко агне в движим принос за престъпление, за да се извърши умилостивение за него, и една десета от ефа чисто брашно, смесено с дървено масло за хлебен принос, и един лог дървено масло,
\par 22 и две гургулици или две гълъбчета, както би намерил; едното да бъде принос за грях, а другото всеизгаряне;
\par 23 и на осмия ден да ги донесе за очистването си при свещеника на входа на шатъра за срещане пред Господа.
\par 24 Тогава свещеникът да вземе агнето на приноса за престъпление и лога дървено масло, и свещеникът да ги подвижи за движим принос пред Господа;
\par 25 и да заколи агнето на приноса за престъпление и свещеникът да вземе от кръвта на приноса за престъпление и да я тури на края на дясното ухо на очищаемия, на палеца на дясната му ръка и на палеца на дясната му нога.
\par 26 След това, свещеникът да излее от дървеното масло в дланта на лявата си ръка;
\par 27 и от маслото, което има в лявата си ръка, свещеникът да поръси с десния си пръст седем пъти пред Господа;
\par 28 и свещеникът да тури от маслото, което има в дланта си, на края на дясното ухо на очищаемия, на палеца на дясната му ръка и на палеца на дясната му нога, на туй място, гдето е кръвта от приноса за престъпление;
\par 29 а останалото от маслото, което е в дланта на свещеника, да тури на главата на очищаемия, за да извърши умилостивение за него пред Господа.
\par 30 После да принесе едната от гургулиците или от гълъбчетата, каквото би намерил;
\par 31 каквото би намерил: едното в принос за грях, а другото за всеизгаряне, заедно с хлебния принос; така да направи свещеникът умилостивение за очищаемия пред Господа.
\par 32 Това е законът за онзи, който има рана от проказа, и не му стига ръка да достави всичко за очистването си.
\par 33 Господ говори още на Моисея и Аарона, казвайки:
\par 34 Когато влезете в Ханаанската земя, която Аз ви давам за владение, и туря зараза от проказа в някоя къща в земята, която притежавате,
\par 35 тогава оня, чиято е къщата, нека дойде и заяви на свещеника, като каже: Струва ми се, че има язва в къщата.
\par 36 И свещеникът, преди да влезе да прегледа язвата, да заповяда да изпразнят къщата, за да не стане нечисто всичко що е в къщата; и след това свещеникът да влезе да прегледа къщата.
\par 37 Като разгледа заразата, ако язвата се явява по стените на къщата със зеленикави или червеникави трапчинки, които изглеждат да са по-дълбоко от повърхността на стената,
\par 38 тогава свещеникът да излезе из къщата до вратата й и да затвори къщата за седем дена.
\par 39 А на седмия ден свещеникът да се върне и да прегледа; и, ето, ако заразата се е разпростряла по стените на къщата,
\par 40 тогава свещеникът да заповяда да извадят камъните, на които е язвата и да ги хвърлят вън от града на нечисто място;
\par 41 и да накара да остържат къщата навсякъде извътре и да изхвърлят остърганата вар вън от града на нечисто място;
\par 42 и да вземат други камъни, които да вложат вместо ония камъни, и да вземат друга вар и да измажат къщата.
\par 43 Но ако пак дойде заразата и се появи в къщата, след като извадят камъните и остържат къщата и я измажат,
\par 44 тогава свещеникът да влезе и да прегледа: ако се е разпростряла язвата в къщата; тя е нечиста.
\par 45 Нека съборят къщата, камъните й, дърветата й и всичката вар на къщата, и да ги изнесат вън от града на нечисто място.
\par 46 При това, който влезе в къщата през цялото време, през което тя е била затворена, да бъде нечист до вечерта.
\par 47 А който спи в къщата, нека изпере дрехите си; и който яде в къщата, нека изпере дрехите си.
\par 48 Но, ако влезе свещеникът та прегледа, и, ето язвата в къщата не се е разпростряла след измазването и, тогава свещеникът да обяви къщата за чиста, защото язвата е изцеляла.
\par 49 А за очистването на къщата нека вземе две птичета, кедрово дърво, червена вълна и исоп;
\par 50 и да заколи едното птиче в пръстен съд над текуща вода;
\par 51 да вземе кедровото дърво, исопа, червената вълна и живото птиче, и да ги натопи в кръвта на закланото птиче и в текущата вода, и да поръси къщата седем пъти.
\par 52 Така да очисти къщата с кръвта на птичето, с текущата вода, с живото птиче, с кедровото дърво, с исопа и с червената вълна .
\par 53 А да пусне живото птиче вън от града на полето. Така да направи умилостивение за къщата, и тя ще бъде чиста.
\par 54 Това е законът за всяка рана от проказа и от кел,
\par 55 и за проказа на дреха и на къща,
\par 56 и за оток, за краста и за лъскави петна,
\par 57 за да учи, кога е нещо нечисто и кога е чисто; това е законът за проказата.

\chapter{15}

\par 1 Господ говори още на Моисея и на Аарона, казвайки:
\par 2 Говорете на израилтяните, като им кажете: Ако човек има течение из тялото си, той е нечист поради течението си.
\par 3 Нечистотата му в течението му е тая: или има течение от тялото му, или се е спряло течението в тялото му; това е неговата нечистота.
\par 4 Всяка постелка, на която лежи оня, който има течението, ще бъде нечиста; и всяко нещо, на което седне, ще бъде нечисто.
\par 5 Също и който се допре до постелката му трябва да изпере дрехите си и да се окъпе с вода и ще бъде нечист до вечерта;
\par 6 и който седне на нещо, върху което е седял оня, който има течението, да изпере дрехите си и да се окъпе с вода, и ще бъде нечист до вечерта.
\par 7 И който се допре до тялото на онзи, който име течението, да изпере дрехите си и да се окъпе във вода, и ще бъде нечист до вечерта.
\par 8 Оня, който има течението, ако плюе върху чистия, този да изпере дрехите си и да се окъпе във вода и да бъде нечист до вечерта.
\par 9 И всяко седло, на което би седнал оня, който има течението, ще бъде нечисто.
\par 10 И който се допре до какво да е нещо, което е било под него, ще бъде нечист до вечерта; и който понесе това нещо нека изпере дрехите си и се окъпе във вода, и ще бъде нечист до вечерта.
\par 11 И ако оня, който име течението, се допре до някого, без да е умил ръцете си с вода, този да изпере дрехите си и да се окъпе във вода и ще е нечист до вечерта.
\par 12 И пръстният съд, до който се допрял, оня, който има течението, да се строши; а всеки дървен съд да се измие с вода.
\par 13 А когато се очисти от течението си оня, който име течение, тогава за очистването си да си изброи седем дена и да изпере дрехите си, и да окъпе тялото си в текуща вода и ще бъде чист.
\par 14 И на осмия ден да си вземе две гургулици или две гълъбчета и да дойде пред Господа на входа на шатъра за срещане, та да ги даде на свещеника;
\par 15 и свещеникът да ги принесе, едното в принос за грях, а другото за всеизгаряне; така да направи свещеникът умилостивение за него пред Господа поради течението му.
\par 16 Оня човек, от когото излезе съвъкупително семе, да окъпе цялото си тяло във вода и да бъде нечист до вечерта.
\par 17 Всяка дреха и всяка кожа, на която се намира съвъкупително семе, да се изпере с вода и ще бъде нечиста до вечерта.
\par 18 А и жената, с която се е съвъкупил мъж, и тя и той да се окъпят във вода и да бъдат нечисти до вечерта.
\par 19 Оная жена, който име течение, и течението от тялото й е кръв, да бъде нечиста седем дена; и всеки, който се допре до нея, ще бъде нечист до вечерта.
\par 20 Всяко нещо, на което е лежала в нечистотата си, ще бъде нечисто; също и всяко нещо, на което е седнала, ще бъде нечисто.
\par 21 И всеки, който се допре до постелката й, да изпере дрехите си и да се окъпе във вода и ще бъде нечист до вечерта.
\par 22 И всеки, който се допре до нещо, на което тя е седнала, да изпере дрехите си и да се окъпе във вода и ще бъде нечист до вечерта.
\par 23 И ако има нещо върху постелката или върху това, на което тя е седнала, и той се допре до това нещо , ще бъде нечист до вечерта.
\par 24 А ако някой легне с нея, и кръвотечението й дойде на него, ще бъде нечист седем дена; и всяка постелка, на която той би легнал, ще бъде нечиста.
\par 25 Ако някоя жена има кръвотечение за много дни, не във времето на нечистотата си, или ако има течение за по-дълго от обикновеното време на нечистотата си, то през всичките дни, когато тече нечистотата й, тя ще бъде както през дните на обикновената си нечистота; тя е нечиста.
\par 26 Всяка постелка, на който лежи през цялото време на течението й, ще бъде както постелката, на която лежи в обикновената си нечистота; и всяко нещо, на което седне, ще бъде нечисто, както с нечистотата на обикновената й нечистота.
\par 27 Всеки, който се допре до тия неща, ще бъде нечист; нека изпере дрехите си и се окъпе във вода, и да бъде нечист до вечерта.
\par 28 Но ако тя се очисти от течението си, тогава да си изброи седем дена, и след тях ще бъде чиста.
\par 29 А на осмия ден да си вземе две гургулици или две гълъбчета и да ги донесе при свещеника при входа на шатъра за срещане;
\par 30 и свещеникът да принесе едното в принос за грях, а другото за всеизгаряне; така да направи свещеникът умилостивение за нея пред Господа поради течението на нечистотата й.
\par 31 Така да отлъчите израилтяните от нечистотите им, за да не умрат в нечистотата си, като оскверняват скинията Ми, която е всред тях.
\par 32 Това е законът за онзи, който име течение, и за онзи, от когото излезе съвъкупително семе и се осквернява чрез него,
\par 33 и за оная, която е болна от кръвотечението си, и за човека който име течение, бил той мъж или жена, и за онзи, който легне с оная, която е нечиста.

\chapter{16}

\par 1 И Господ говори на Моисея, след смъртта на двамата Ааронови синове, когато пристъпиха с чужд огън пред Господа и умряха.
\par 2 Господ каза на Моисея: Кажи на брата си Аарона да не влиза на всяко време в светилището, което е отвътре завесата, пред умилостивилището, което е върху ковчега, за да не умре; защото Аз ще се явявам в облак върху умилостивилището.
\par 3 Ето как да влезе Аарон в светилището: с юнец, в принос за грях и овен за всеизгаряне.
\par 4 Да се облече в осветения ленен хитон, да има ленените гащи на тялото си, да се опаше с ленения поят и да носи ленената митра на главата си; те са светите облекла; и след като окъпе тялото си във вода, нека се облече с тях.
\par 5 И от обществото на израилтяните да вземе два козела в принос за грях и един овен за всеизгаряне.
\par 6 И Аарон да приведе принесения за грях юнец, който е за него, и да направи умилостивение за себе си и за дома си.
\par 7 Също да вземе двата козела и да ги представи пред Господа до входа на шатъра за срещане.
\par 8 Тогава да хвърли Аарон жребий за двата козела, един жребий за Господа, и другия жребий за отпущане;
\par 9 и да приведе Аарон козела, на който е паднал жребият в Господа, и да го принесе в принос за грях;
\par 10 а да представи жив пред Господа козела, на който е паднал жребият за отпущане, за да направи с него умилостивение и да го изпрати в пустинята за отпущане.
\par 11 После Аарон да приведе юнеца на приноса за грях, който е за него, и да направи умилостивение за себе си и за дома си, като заколи принесения за грях юнец, който е за него.
\par 12 След това да вземе кадилница пълна с разгорени въглища от олтара, който е пред Господа, и две пълни шепи с благоуханен темян счукан дребно, и да го внесе отвътре завесата;
\par 13 и да тури темян на огъня пред Господа, за да покрие дима от темяна умилостивилището, което е върху плочите на свидетелството, за да не умре.
\par 14 Също да вземе от кръвта на юнеца и с нея да поръси с пръста си умилостивилището откъм изток; а пред умилостивилището да поръси от кръвта седем пъти с пръста си.
\par 15 Тогава да заколи козела на приноса за грях, който е за людете, и да внесе кръвта му да направи както направи с кръвта на юнеца, като поръси с нея над умилостивилището и пред умилостивилището.
\par 16 Така да направи умилостивение за светилището, поради нечистотиите на израилтяните, и поради престъпленията им и всичките им грехове; така да направи и за шатъра за срещане, който стои помежду им всред нечистотиите им.
\par 17 И никой да се не намира в шатъра за срещане, когато влиза той във светилището да направи умилостивение, догде излезе, след като направи умилостивение за себе си, за дома си и за цялото общество израилтяни.
\par 18 Тогава да излезе към олтара, който е пред Господа, и да направи умилостивение за него, като вземе от кръвта на юнеца и от кръвта на козела и я тури по роговете на олтара наоколо;
\par 19 и седем пъти да поръси върху него от кръвта с пръста си, и така да го очисти от нечистотиите на израилтяните, и за го освети.
\par 20 А като свърши да прави умилостивение за светилището, за шатъра за срещане и за олтара, нека приведе живия козел;
\par 21 и като положи Аарон двете си ръце на главата на живия козел нека изповяда над него всичките беззакония на израилтяните, всичките им престъпления и всичките им грехове, и нека ги възложи на главата на козела; тогава да го изпрати в пустинята чрез определен човек;
\par 22 и като пусне козела в пустинята, козелът ще понесе на сабе си всичките им беззакония в необитаема земя.
\par 23 Тогава да влезе Аарон в шатъра за срещане, да съблече ленените одежди, в които се е облякъл, когато е възлязъл в светилището, и да ги остави там;
\par 24 и, като окъпе тялото си във вода на свето място, да облече дрехите си, да излезе и да принесе своето всеизгаряне и всеизгарянето на людете, и така да направи умилостивение за себе си и за людете.
\par 25 А тлъстината на приноса за грях да изгори на олтара.
\par 26 И оня, който откара отпущаемия козел, нека изпере дрехите си и окъпе тялото си във вода и след това да влезе в стана.
\par 27 А юнецът на приноса за грях, и козелът на приноса за грях, чиято кръв се е внесла, за да се извърши умилостивение във светилището, да изнесат вън от стана и да изгорят в огън кожата им, месото им и изверженията им.
\par 28 А оня, който ги изгори, да изпере дрехите си и да окъпе тялото си с вода и, след това, да влезе в стана.
\par 29 Това да бъде вечен закон; в седмия месец, на десетия ден от месеца, да смирите душите си и да не вършите никаква работа, нито туземец нито пришелец, който се е заселил между вас;
\par 30 защото в тоя ден ще се направи умилостивение за вас, за да се очистите от всичките си грехове, та да сте чисти пред Господа.
\par 31 Това да ви бъде събота за тържествена почивка и да смирите душите си; това е вечен закон.
\par 32 Оня свещеник, който бъде помазан и посветен да свещенодействува, вместо баща си, той да направи умилотивението, като се облече в ленените одежди, светите одежди;
\par 33 и да направи умилостивение за светото светилище, и да направи умилостивение за шатъра за срещане, и да направи умилостивение за свещениците и за всичките люде на обществото.
\par 34 Това ще ви бъде за вечен закон, веднъж в годината да правите умилостивение за израилтяните поради всичките им грехове. И Аарон направи, според както Господ заповяда на Моисея.

\chapter{17}

\par 1 Господ говори още на Моисея, казвайки:
\par 2 Говори на Аарона и на синовете му и на всичките израилтяни, като им кажеш: Ето какво заповяда Господ, като рече:
\par 3 Всеки човек от Израилевия дом, който заколи говедо, агне, или коза в стана, или който заколи вън от стана,
\par 4 без да го е привел до входа на шатъра за срещане, за да го принесе Господу пред Господната скиния, той ще се счете виновен за кръв; кръв е пролял; тоя човек ще се изтреби изсред людете си;
\par 5 с цел израилтяните да довеждат жертвите, които колят на полето и да ги принасят Господу до входа на шатъра за срещане, при свещеника, та да ги колят за примирителни жертви Господу.
\par 6 И свещеникът да поръси с кръвта Господния олтар при входа на шатъра за срещане и да изгори тлъстината за благоухание Господу.
\par 7 Да не принасят вече жертвите си на бесовете, след като те блудствуват; това да им бъде вечен закон във всичките им поколения.
\par 8 Кожи им още: Който човек от Израилевия дом, или от заселените между тях пришелци, принесе всеизгаряне или жертва,
\par 9 без да я приведе до входа на шатъра за срещане, за да я принесе Господу, тоя човек ще се изтреби измежду людете си.
\par 10 Ако някой от Израилевия дом, или от заселените между тях пришелци, яде каква да е кръв, ще насоча лицето Си против оня човек, който яде кръвта, и ще го изтребя изсред людете му.
\par 11 Защото живота на тялото е в кръвта, която Аз ви дадох да правите умилостивение на олтара за душите си; защото кръвта е, която, по силата на живота, който е в нея , прави умилостивение.
\par 12 За това казах на израилтяните: Ни един човек от вас да не яде кръв, и пришелецът, който е заселен между вас, да не яде кръв.
\par 13 И ако някой от израилтяните, или от заселените между тях пришелци, отиде на лов и улови животно или птица, що бива да се яде, нека излее кръвта й, па да я покрие с пръст.
\par 14 Защото, колкото за живота на всяка твар, кръвта й - тя е животът й; за това казах на израилтяните: Да не ядете кръвта на никаква твар, защото животът на всяка твар е кръвта й; всеки, който я яде, ще се изтреби.
\par 15 И всеки човек, който яде мърша или разкъсано от звяр, бил той туземец или пришелец, нека изпере дрехите си и да се окъпе във вода, и ще бъде нечист до вечерта: а след това да бъде чист.
\par 16 И ако не ги изпере и не окъпе тялото си, тогава ще носи беззаконието си.

\chapter{18}

\par 1 Господ говори още на Моисея, казвайки:
\par 2 Говори на израилтяните, като им кажеш: Аз съм Иеова вашият Бог.
\par 3 Да не правите, както правят в Египетската земя, гдето сте живели, и да не правите както правят в Ханаанската земя, в която Аз ви завеждам; и да не ходите по техните повеления.
\par 4 А Моите съдби да правите, и Моите повеления да пазите, да ходите в тях. Аз съм Иеова вашият Бог.
\par 5 Затова, пазете повеленията Ми, и съдбите Ми, чрез които, ако човек прави това, ще живее. Аз съм Иеова.
\par 6 Никой да не приближи при никоя своя кръвна роднина, за да открие голотата й. Аз съм Иеова.
\par 7 Голотата на баща си, тоест, голотата на майка си, да не откриеш; тя ти е майка; да не откриеш голотата й.
\par 8 Голотата на бащината си жена да не откриеш; тя е голотата на баща ти.
\par 9 Голотата на сестра си, бащината ти дъщеря, или майчината ти дъщеря, родена у дома или родена вън, - голотата на такива да не откриеш.
\par 10 Голотата на синовата си дъщеря, или на дъщеря на дъщеря ти - тяхната голота да не откриеш; защото тяхната голота е твоя.
\par 11 Голотата на дъщерята на бащината ти жена, родена от баща ти, (сестра ти е), - нейната голота да не откриеш.
\par 12 Голотата на бащината си сестра да не откриеш; тя е близка роднина на баща ти.
\par 13 Голотата на майчината си сестра да не откриеш, защото тя е близка роднина на майка ти.
\par 14 Голотата на бащиния си брат да не откриеш, тоест , при жена му да се не приближиш; тя ти е стрина.
\par 15 Голотата на снаха си да не откриеш; жена е на сина ти; нейната голота да не откриеш.
\par 16 Голотата на братовата си жена да не откриеш; тя е голотата на брата ти
\par 17 Голотата на жена и на дъщеря й да не откриеш, нито да вземеш дъщерята на сина й, или дъщеря на дъщеря й, за да откриеш голотата й; те са нейни близки роднини; това е нечестие.
\par 18 И да не вземаш жена заедно със сестра й докато е жива другата, за да откриеш голотата й, та да й стане съперница.
\par 19 При жена, когато е разлъчена, поради нечистотата си, да не приближиш да откриеш голотата й.
\par 20 И с жената на ближния си да се не съвъкупиш та да се оскверниш с нея.
\par 21 Никак да не посветиш от семето си на Молоха; нито да оскверниш името на своя Бог. Аз съм Иеова.
\par 22 С мъжко да не легнеш като с женско; това е гнусота.
\par 23 И с никое животно да се не съвъкупиш та да се оскверниш с него, нито жена да застане пред животно, за да се съвъкупи с него; това е мърсота.
\par 24 Не се осквернявай ни с едно от тия неща, защото с всички тия се оскверниха народите, които Аз изпъдих отпред вас.
\par 25 Оскверни се и земята; за това въздавам върху нея беззаконието й, и земята избълва жителите си.
\par 26 Вие, прочее, пазете повеленията Ми и съдбите Ми, и не правете ни една от тия гнусоти, ни туземец, ни зеселен между вас пришелец,
\par 27 (защото всички тия гнусоти вършеха човеците, които са били преди вас на тая земя, та се оскверни земята)
\par 28 за да не избълва и вас земята, когато я осквернявате, както избълва населението, което беше преди вас.
\par 29 Защото, всеки който върши коя да е от тия гнусоти, - човеците, които ги вършат, ще се изтребят изсред людете си.
\par 30 И тъй, пазете заръчванията Ми, и не вършете никой от тия гнусни обичаи, които са се вършили преди вас, та да се не оскверните в тях. Аз съм Иеова вашият Бог.

\chapter{19}

\par 1 Господ говори още на Моисея, казвайки:
\par 2 Кажи на цялото общество израилтяни, като им речеш: Свети бъдете, защото Аз Господ, вашият Бог, съм свет.
\par 3 Да се боите всеки от майка си и от баща си; и съботите Ми да пазите. Аз съм Господ вашият Бог.
\par 4 Да се не върнете към идолите, нито да си направите излеяни богове. Аз съм Господ вашият Бог.
\par 5 Когато принесете примирителна жертва Господу, да я принесете така щото да бъдете приети.
\par 6 В същия ден, когато я принесете, или на следния, трябва да се яде; и ако остане нещо до третия ден, да се изгори в огън.
\par 7 И ако някой яде от нея на третия ден, тя е гнусна; не ще бъде приета;
\par 8 а всеки, който я яде, ще носи беззаконието си, защото е осквернил Господната светиня. Тоя човек ще се изтреби изсред людете си.
\par 9 Когато жънете жетвата на земята си, да не дожънеш краищата на нивата си и да не събереш падналите в жетвата си класове.
\par 10 Да не обираш повторно лозето си, нито да събираш пабирък от лозето; да го оставиш на сиромаха и на чужденеца. Аз съм Господ вашият Бог.
\par 11 Да не крадете нито да мамите и никой да не лъже ближния си.
\par 12 И да се не кълнете на лъжа в Моето Име, нито да оскверняваш Името на Бога си. Аз съм Господ.
\par 13 Да не притесняваш ближния си, нито да го ограбиш; да не престои у тебе заплатата на надничаря ти през нощта до сутринта.
\par 14 Да не хулиш глухия, нито да туриш препънка пред слепия, но да се боиш от Бога си. Аз съм Господ.
\par 15 Да не извършиш неправда в съд; да не покажеш лицеприятие към сиромаха, нито да се стесниш от личността на големеца; но по правда да съдиш ближния си.
\par 16 Да не обикаляш между людете си като одумник, нито да се подигнеш в покушение против кръвта на ближния си. Аз съм Господ.
\par 17 Да не мразиш брата си в сърцето си; да изобличиш смело ближния си, та да се не натовариш с грях поради него.
\par 18 Да не отмъщаваш, нито да храниш злоба против ония, които са от людете ти; но да обичаш ближния си както себе си. Аз съм Господ.
\par 19 Да пазите повеленията Ми. Да не пущаш добитъка си с добитък от друга порода; да не сееш разнородни семена в нивата си; нито да обличаш дреха тъкана от разновидна прежда.
\par 20 Ако някой се съвъкупи с жена, която е робиня, годена за мъж, а не откупена нито освободена, те да се накажат, но не със смърт, защото тя не е била свободна.
\par 21 Той да приведе Господу приноса си за престъпление, при входа на шатъра за срещане, - освен в принос за престъпление.
\par 22 И за греха му, що е сторил, свещеникът да направи умилостивение за него пред Господ с принесения за престъплението овен; и ще му се прости греха, що е сторил.
\par 23 Когато влезете в земята и посадите от всякакъв вид плодно дърво, то да имате плода му като необрязан; три години да ви бъде като необрязан; да се не яде.
\par 24 А в четвъртата година всичкия му плод да бъде свет, за да хвалите с него Господа.
\par 25 И в петата година яжте плода му, за да ви се умножава раждането му. Аз съм Господ вашият Бог.
\par 26 Да не ядете месо с кръвта му , нито да си служите с гадания нито с предвещания.
\par 27 Да не стрижете косата на главата си кръгло; и никой да не разваля краищата на брадата си.
\par 28 Заради мъртвец да не правите порязвания по месата си, нито да начертавате белези по себе си. Аз съм Господ.
\par 29 Да не оскверниш дъщеря си, като й допуснеш да стане блудница, да не би земята да изпадне в блудодеяние, и земята да се напълни с беззаконие.
\par 30 Съботите Ми да пазите, и светилището ми да почитате. Аз съм Господ.
\par 31 Да се не отнесете към запитвачите на зли духове нито към врачовете; не ги издиряйте та да се оскверняват чрез тях. Аз съм Господ вашият Бог.
\par 32 Пред белокосия да ставаш, и старческото лице да почиташ, и от Бога си да се боиш. Аз съм Господ.
\par 33 Когато някой чужденец се засели при тебе в земята ви, да му не правите зло;
\par 34 чужденецът, който се е заселил при вас, да ви бъде като ваш туземец, и да го обичаш като себе си; защото и вие бяхте чужденци в Египетската земя. Аз съм Господ вашият Бог.
\par 35 Да не извършите неправда в съд, в мярка, във везни, или в мерило.
\par 36 Верни везни, верни теглилки, верна ефа и верен ин да имате. Аз съм Господ вашият Бог, Който ви изведох из Египетската земя.
\par 37 Да пазите всичките Ми повеления и всичките Ми съдби, и да ги извършвате. Аз съм Господ.

\chapter{20}

\par 1 Господ говори още на Моисея, казвайки:
\par 2 Да речеш на израилтяните: Който от израилтяните, или от заселените в Израиля чужденци, даде от семето си на Молоха, непременно да се умъртви; людете на земята да го убият с камъни.
\par 3 Също Аз ще насоча лицето Си против оня човек и ще го изтребя изсред людете му, защото е дал от семето си на Молоха, та е омърсил светилището Ми и е осквернил светото Ми Име.
\par 4 И ако по какъв да е начин людете на земята затворят очите си, за да не видят тоя човек, когато дава от семето си на Молоха, и не го погубят,
\par 5 тогава Аз ще насоча лицето Си против тоя човек и против семейството му, и ще изтребя изсред людете им него и всички, които го последват в блудството, , за да блудствуват след Молоха.
\par 6 И човек, който се отнесе към запитвачите на зли духове и към врачовете за да блудствува след тях, против оня човек Аз ще насоча лицето Си и ще го изтребя изсред людете му.
\par 7 И тъй, осветете се и бъдете свети, защото Аз съм Господ вашият Бог;
\par 8 и да пазите повеленията Ми и да ги вършите. Аз съм Господ, Който ви освещавам.
\par 9 Всеки, който прокълне баща си или майка си, непременно да се умъртви; баща си или майка си е проклел; кръвта му да бъде върху него.
\par 10 Ако прелюбодействува някой с чужда жена, тоест , ако прелюбодействува някой с жената на ближния си, непременно да се умъртви и прелюбодеецът и прелюбодейката.
\par 11 Който легне с жената на баща си, бащината голота е открил; непременно и двамата да се умъртвят; кръвта им да бъде върху тях.
\par 12 И ако някой легне със снаха си, непременно и двамата да се умъртвят; мърсота са извършили; кръвта им да бъде върху тях.
\par 13 Ако някой легне с мъжко, като с женско, и двамата са извършили гнусота; непременно да се умъртвят; кръвта им да бъде върху тях.
\par 14 Ако някой вземе жена и майка й, това е беззаконие; с огън да се изгорят и той и те, за да няма беззаконие между вас.
\par 15 Ако някой се съвъкупи с животно, той непременно да се умъртви, и животното да убиете.
\par 16 И ако се приближи жена при какво да е животно, за да се съвъкупи с него, да убиеш и жената и животното; непременно да се умъртвят; кръвта им да бъде върху тях.
\par 17 Ако някой вземе сестра си, дъщеря на баща си, или дъщеря на майка си, и види голотата й, и тя види неговата голота, това е нечестие, да се изтребят пред очите на людете си; голотата на сестра си е открил; ще носи беззаконието си.
\par 18 И който легне с жена, която е в нечистотата си, и открие голотата й, той е открил течението й, и тя е открила течението на кръвта си; за това, да се изтребят и двамата изсред людете си.
\par 19 Голотата на майчината си сестра или на бащината си сестра да не откриеш; защото, който е сторил това , открил е голотата на близките си роднини; те ще носят беззаконието си.
\par 20 И ако някой легне със стрина си, голотата на стрина си е открил; те ще носят греха си; бездетни ще умрат.
\par 21 И ако някой вземе братовата си жена, това е нечистота; братовата си голота е открил; баздетни ще останат.
\par 22 Пазете, прочее, всичките Ми повеления и всичките Ми съдби та ги извършвайте, за да не ви избълва земята, гдето Аз ви завеждам да живеете в нея.
\par 23 И да не ходите по обичаите на населението, което Аз изпъждам отпред вас; защото те извършиха всички тия гнусоти , и затова Аз се погнусих от тях.
\par 24 А на вас съм казал: Вие ще наследите земята им, и Аз ще я дам на вас за притежание, земя гдето текат мляко и мед. Аз съм Господ вашият Бог, който ви отделих от племената.
\par 25 За това, правете разлика между чистите животни и нечистите, и между нечистите птици и чистите; и да не осквернявате себе си с животно, или с птица, или с какво да е що пълзи на земята, които Аз ви отделих като нечисти.
\par 26 И бъдете свети на Мене; защото Аз, Иеова съм свет, и ви отделих от племената за да бъдете Мои.
\par 27 Също мъж или жена, която запитва зли духове, или е врач, непременно да се умъртви; с камъни да ги убият; кръвта им да бъде върху тях.

\chapter{21}

\par 1 Господ рече още на Моисея: Говори на свещениците, Аароновите синове, и кажи им: Никой от вас да се не оскверни между людете си, заради мъртвец,
\par 2 освен за близък сродник, сиреч , за майка си, за баща си, за сина си, за дъщеря си, или за брата си;
\par 3 също и за сестра си девица, която е близка нему, и която не е била омъжена, заради нея може да се оскверни.
\par 4 Като началник на людете си да се не оскверни така щото да стане мръсен.
\par 5 Да не оплешат главата си, нито да бръснат краищата на брадата си, нито да правят порязвания по месата си.
\par 6 Свети да бъдат на Бога си, и да не осквернят Името на Бога си; защото те принасят Господните чрез огън приноси, хляба на Бога си; затова, да бъдат свети.
\par 7 Да не вземат жена, която е блудница или осквернена, нито да вземат жена напусната от мъжа си; защото свещеникът е свет на Бога си.
\par 8 Прочее, да го накараш да бъде свет, защото той принася хляба на твоя Бог; свет да ти бъде; защото Аз Господ, Който ви освещавам, съм свет.
\par 9 И ако дъщерята на някой свещеник се омърси с блудство, тя омърсява баща си; с огън да се изгори.
\par 10 Но оня, който между братята си е велик свещеник, на чиято глава се е изляло мирото за помазване, и който се е посветил, за да облича свещените одежди, да не ткрива главата си, нито да раздира дрехите си,
\par 11 нито да влиза при някой умрял, нито да се осквернява за баща си или за майка си;
\par 12 нито да излиза от светилището или да осквернява светилището на Бога си; защото посвещаването чрез помазване с мирото на Бога му е станало на него. Аз съм Господ.
\par 13 И той да вземе за жена девица;
\par 14 вдовица, или напусната, или осквернена, или блудница да не взема; но девица от людете си да вземе за жена.
\par 15 И тъй, да не осквернява потомството си между людете си; защото Аз съм Господ, Който го освещавам.
\par 16 Господ говори още на Моисея, казвайки:
\par 17 Говори на Аарона, като му речеш: Който от твоето потомство, във всичките им поколения, има недостатък, той да не пристъпва да принася хляб на Бога си.
\par 18 Защото никой, който има недостатък, не бива да пристъпва: човек сляп, или куц, или със сплескан нос, или с нещо излишно,
\par 19 или човек със строшена нога, или строшена ръка,
\par 20 или гърбав, или завързляк, или с повредени очи, или със суха краста, или с лишаи, или килав.
\par 21 Ни един човек от потомството на свещеника Аарона, който има недостатък, да не пристъпва за да принесе Господните чрез огън приноси; понеже има недостатък, ни бива да пристъпи да принесе хляба на Бога си.
\par 22 От пресветите и от светите приноси нека яде хляба на Бога си;
\par 23 но да не влиза до завесата нито да се приближава при олтара, понеже има недостатък, за да не омърсява светилищата Ми; защото Аз съм Господ, Който ги освещавам.
\par 24 И тъй, Моисей каза това на Аарона и на синовете му, и на всичките израилтяни.

\chapter{22}

\par 1 Господ говори още на Моисея, казвайки:
\par 2 Речи на Аарона и на синовете му, кога да се въздържат от светите приноси , които израилтяните Ми посвещават, за да не омърсяват Моето свето Име. Аз съм Господ.
\par 3 Кажи им, че всеки човек от цялото ви потомство, във всичките ви поколения, който има нечистота на себе си, и пристъпи до светите приноси , които израилтяните посвещават Господу, тоя човек ще се изтреби от пред Мене. Аз съм Господ.
\par 4 Който от Аароновото потомство е прокажен, или има течение, да не яде от светите неща, догде се очисти. И който се допре до какво да било нещо, което е нечисто от мърша, или до човек, из когото излиза съвъкупително семе,
\par 5 или който се допре до каквото да е животно, от което може да стане нечист, или до човек, от когото можа да стане нечист, каквато и да е нечистотата му,
\par 6 оня човек, който се допре до кое да е от такива, ще бъде нечист до вечерта, и да не яде от светите неща догдето не окъпе тялото си във вода.
\par 7 Когато залезе слънцето ще бъде нечист, и подир това нека яде от светите неща, защото това му е храната.
\par 8 Мърша или звероразкъсано да не яде, за да се не оскверни от тях. Аз съм Господ.
\par 9 Да пазят, прочее, заръчването Ми, за да ни си навлекат грях и умрат поради това, ако са се омърсили. Аз съм Господ, Който ги освещавам.
\par 10 Ни един чужденец да не яде от светите неща: гост на свещеника ако е, или наемник, пак да не яде от светите неща.
\par 11 Но ако някой свещеник купи някого с пари, той може да яде от тях, както и оня, който се е родил у дома му; те могат да ядат от хляба му.
\par 12 Дъщеря на свещеник, ако е омъжена за чужденец, да не яде от приноса за издигане от светите неща.
\par 13 Но ако дъщеря на свещеник овдовее или бъде напусната и няма дете, и се върне в бащиния си дам, както е била в младостта си, тя може да яде от хляба на баща си. Обаче никоя чужденец да не яде от него.
\par 14 И ако някой от незнание яде нещо свето, тогава да даде на свещеника равното на светото нещо и да му притури петата му част.
\par 15 Свещениците да се не отнасят със светите неща, които израилтяните принасят Господу, като с просто нещо,
\par 16 за да не навлекат на себе си беззаконието на престъпление, когато ядат техните свети неща; защото Аз съм Господ, Който ги освещавам.
\par 17 Господ още говори на Моисея, казвайки.
\par 18 Говори на Аарона, на синовете му и на всичките израилтяни, като им кажеш: Всеки човек от Израилевия дом, или от чужденците в Израиля, който принесе принос срещу какви да са обреци, или като какви да са доброволни приноси, които принасят Господу за всеизгаряне,
\par 19 за да ви бъде приет, трябва да принесе мъжко, без недостатък, от говедата, от овците, или от козите.
\par 20 Нищо с недостатък да не принасяте, защото не ще ви бъде прието.
\par 21 Който, за изпълнение на обрек, или за доброволен принос, принесе от говедата или от овците в примирителна жертва Господу, нека я принесе без недостатък, за да бъде приета; никакъв недостатък да няма в нея.
\par 22 Животно сляпо, или със строшена или изкълчена част, или което има оток, суха краста, или лишаи, такива да не принасяте Господу, нито да правите от тях жертва Господу чрез огън на олтара.
\par 23 Но юнец или овце с нещо излишно или с недостатък в частите можеш да принесеш за доброволен принос: обаче срещу обрек не ще бъде приет.
\par 24 Животно превито, или със смазани или изтръгнати мъди, или скопено да не принасяте Господу, нито да правите така в земята си.
\par 25 Нито от страна на чужденец да принасяте едно от всички тия за храна на вашия Бог, защото разтление има в тях, недостатък има в тях; не ще ви бъдат приети.
\par 26 Господ още говори на Моисея, казвайки:
\par 27 Когато се роди теле, или агне, или яре, тогава нека бъде седем дена с майка си; а от осмия ден нататък ще бъде прието за жертвен чрез огън принос Господу.
\par 28 А крава или овца да не заколите в един ден с малкото й.
\par 29 И когато принесете благодарствена жертва Господу, да я принесете така щото да ви бъде приета.
\par 30 В същия ден да се изяде; да не оставяте нищо от нея до сутринта. Аз съм Господ.
\par 31 И тъй, да пазите заповедите Ми и да ги вършите. Аз съм Господ.
\par 32 И да не осквернявате Моето Име; но Аз ще съм осветен между израилтяните. Аз съм Господ, Който ви освещавам,
\par 33 Който ви изведох из Египетската земя, за да бъда вашият Бог, Аз съм Иеова.

\chapter{23}

\par 1 Господ още говори на Моисея, казвайки:
\par 2 Говори на израилтяните, като им кажеш: Господните празници, в които ще свиквате свети събрания, Моите празници, са следните:
\par 3 Шест дена да се работи; а на седмия ден е събота за тържествена почивка, за свето събрание: в нея да не работите никаква работа; във всичките ви жилища е събота Господу.
\par 4 Ето Господните празници, свети събрания, които ще свиквате във времената им:
\par 5 В първия месец, на четиринадесетия ден от месеца, привечер, е Пасха Господна;
\par 6 и на петнадесетия ден от същия месец е Господният празник на безквасните; седем дена да ядете безквасни хлябове.
\par 7 На първия ден да имате свето събрание, и никаква слугинска работа да не вършите.
\par 8 И седем дена да принасяте по една жертва чрез огън Господу; на седмия ден е свето събрание, и никаква слугинска работа да не вършите.
\par 9 Господ говори още на Моисея, казвайки:
\par 10 Говори на израилтяните, като им кажеш: Когато влезете в земята, която Аз ви давам, и пожънете жетвата й, тогава да донесете на свещеника един сноп от първите плодове на жетвата си;
\par 11 и той да подвижи снопа пред Господа, за да ви бъде приет; на другия ден подир съботата да го подвижи свещеникът.
\par 12 И в деня, когато подвижите снопа, да принесете за всеизгаряне Господу едно едногодишно агне без порок;
\par 13 и хлебния му принос, две десети от ефа чисто брашно омесено с дървено масло, в жертва чрез огън Господу за благоухание; и възлиянието му, един четвърт ин вино.
\par 14 А хляб или пържено жито, или пресни класове да не ядете до тоя ден, до деня, когато принесете приноса на вашия Бог. Това да бъде вечен закон във всичките ви поколения, във всичките ви жилища.
\par 15 От другия ден след съботата, в която принесохте снопа на движимия принос, да си изброите седем цели седмици;
\par 16 до следващия ден подир седмата събота да изброите петдесет дена, и тогава да принесете новохлебен принос Господу.
\par 17 Да донесете от жилищата си за движим принос два хляба, които да бъдат две десети от ефа чисто брашно, изпечени с квас, като първи плодове Господу.
\par 18 И заедно с хляба да принесете седем едногодишни агнета без недостатък, един юнец и два овена; да бъдат всеизгаряне Господу заедно с хлебния им принос и заедно с възлиянията им, в принос чрез огън за благоухание Господу.
\par 19 Да принесете и един козел в принос за грях, и две едногодишни агнета за примирителна жертва.
\par 20 И свещеникът да ги подвижи заедно с хляба на първите плодове и заедно с двете агнета за движим принос пред Господа; та да бъдат свети Господу за свещеника.
\par 21 И на седмия ден да свикате свето събрание и никаква слугинска работа да не вършите: това да бъде вечен закон във всичките ви жилища във всичките ви поколения.
\par 22 И когато жънете нивите на земята си, да не жънеш краищата на нивата си, и да не събираш падналите в жетвата ти класове; за сиромаха и за чужденеца да ги оставиш. Аз съм Господ вашият Бог.
\par 23 Господ още говори на Моисея, казвайки:
\par 24 Говори на израилтяните, като речеш: В седмия месец, на първия ден от месеца, да ви бъде тържествена почивка, спомен с тръбно възклицание, свето събрание.
\par 25 В него да не вършите никаква слугинска работа и да принасяте жертва чрез огън Господу.
\par 26 Господ говори още на Моисея, казвайки:
\par 27 Дасетият ден на тоя седми месец да бъде ден на умилостивение; да имате свето събрание, и да смирите душите си, и да принесете жертва чрез огън Господу.
\par 28 Никаква работа да не вършите в тоя ден, защото е ден на умилостивение, за да се извърши умилостивение за вас пред Господа вашия Бог.
\par 29 Защото всеки човек, който не се смири в тоя ден, ще се изтреби измежду людете си.
\par 30 И всеки човек, който извърши каква да е работа в тоя ден, тоя човек ще изтребя изсред людете му.
\par 31 Никаква работа да не вършите; това да бъде вечен закон във всичките ви поколения във всичките ви жилища.
\par 32 Ще ви бъде събота за тържествена почивка и за да смирите душите си; на деветия ден от месеца, вечерта, от вечер до вечер, да пазите съботата си.
\par 33 Господ говори още на Моисея, казвайки:
\par 34 Говори на израилтяните, като речеш: От петнадесетия ден на тоя месец да пазите за седем дена Господния празник на скинопигията.
\par 35 На първия ден да има свето събрание, и никаква слугинска работа да не вършите.
\par 36 Седем дена да принасяте по една жертва чрез огън Господу, а на осмия ден да имате свето събрание и да принесете жертва чрез огън Господу: това е тържествено събрание, и никаква слугинска работа да не вършите.
\par 37 Тия са Господните празници, в които да свиквате свети събрания, за да принасяте жертва чрез огън Господу, всеизгаряне, хлебен принос, жертва и възлияния,, всяко на определения му ден,
\par 38 освен Господните съботи, и освен всичките обреци, и освен всичките доброволни приноси които давате Господу.
\par 39 А на петнадесетия ден от седмия месец, когато ще сте прибрали произведенията на земята, да празнувате Господния празник седем дена; първият ден да бъде тържествена почивка, и осмият ден тържествена почивка.
\par 40 И на първия ден да си вземете плод от хубави дървета, палмови клони, клони от широколистни дървета и речни върби, и седем дена да се веселите пред Господа вашия Бог.
\par 41 Да празнувате тоя празник за Господа седем дена в годината; това да бъде вечен закон във всичките ви поколения; в седмия месец да го празнувате.
\par 42 В колиби да седите седем дена; всички туземци израилтяни да седят в колиби,
\par 43 за да познаят бъдещите ви поколения, че в колиби направих израилтяните да седят, когато ги изведох из Египетската земя. Аз съм Господ вашият Бог.
\par 44 И тъй, Моисей обяви Господните празници на израилтяните.

\chapter{24}

\par 1 Господ говори още на Моисея, казвайки:
\par 2 Заповядай на израилтяните да ти донесат първоток чисто дървено масло за осветление, за да свети постоянно светило.
\par 3 Аарон да го приготвя отвън завесата, която е пред плочите на свидетелството в шатъра за срещане, за да свети от вечер до заран пред Господа винаги; това да бъде вечен закон във всичките ви поколения.
\par 4 На чистозлатния светилник да приготвя светилата пред Господа винаги.
\par 5 И да вземеш чисто брашно, от което да опечеш дванадесет пити; всяка пита да бъде две десети от ефа .
\par 6 И да ги сложиш на два реда, по шест на всеки ред, върху чистозлатната трапеза пред Господа.
\par 7 И на всеки ред да туриш чист ливан; и това ще бъде върху хляба за спомен в принос чрез огън Господу.
\par 8 Всеки съботен ден свещеникът да слага това винаги пред Господа; това е от страна на израилтяните вечен завет.
\par 9 И те ще бъдат за Аарона и за синовете му, които да ги ядат на свето място; защото са пресвети нему от жертвите принасяни чрез огън Господу; това е вечен закон.
\par 10 И синът на една израилтянка, на когото бащата беше египтянин, излезе между израилтяните; и синът на израилтянката и един израилтянин се караха в стана.
\par 11 И синът на израилтянката похули Господното Име и прокле. И доведоха го при Моисея. (А името на майка му беше Саломита, дъщеря на Даврия, от Дановото племе).
\par 12 И туриха го под стража, догде им се обяви от Господа, какво да му сторят .
\par 13 И Господ говори на Моисея, казвайки:
\par 14 Изведи вън от стана онзи, който прокле; и всички, които са го чули, нека турят ръцете си на главата му, и цялото общество нека го убие с камъни.
\par 15 И говори на израилтяните, като речеш: Който прокълне своя Бог ще носи греха си.
\par 16 Който похули Господното име непременно да се умъртви; цялото общество да го убие с камъни; бил той чужденец или туземец, когато похули Господното име, да се умъртви.
\par 17 Който убие някой човек, непременно да се умъртви.
\par 18 А който убие животно, да го плати, Живот за живот.
\par 19 И ако някой причини повреда на ближния си, нека се направи нему така, както е направил той:
\par 20 строшено за строшено, око за око, зъб за зъб; според повредата, която причини той на човека, така да се направи и нему.
\par 21 Който убие животно, да го плати; а който убие човек, да се умъртви.
\par 22 Един закон да имате, както за чужденеца така и за туземеца; защото Аз съм Господ вашият Бог.
\par 23 И тъй, Моисей каза на израилтяните; и те изведоха вън от стана онзи, който бе проклел, и го убиха с камъни; израилтяните сториха, според както Господ заповяда на Моисея.

\chapter{25}

\par 1 Господ говори още на Моисея на Синайската планина, казвайки:
\par 2 Говори на израилтяните, като им кажеш: Когато влезете в земята, която Аз ви давам, тогава земята да пази една събота за Господа.
\par 3 Шест години да сееш нивите си и шест години да режеш лозето си и да събираш плода му;
\par 4 а седмата година да бъде събота за тържествена почивка на земята, събота на Господа; в нея да не сееш нивата си и да не режеш лозето си.
\par 5 Да не жънеш и саморасла жетва, нито да обираш гроздето от нерязано лозе: година за тържествена почивка да бъде на земята.
\par 6 Произведеното през тая събота на земята ще ви бъде за храна: на тебе, на слугата ти, на слугинята ти, на наемника ти и на чужденеца, който се е заселел при тебе;
\par 7 на добитъка ти, и на животните, които са в земята ти, всичкото нейно произведение ще бъде за храна.
\par 8 Да си изброиш седем седмици от години, седем пъти по седем години, та, като ти мине времето на седем седмици от години,
\par 9 тогава, на десетия ден от седмия месец, да накараш да се затръби с възклицание; в деня на умилостивението да накарате да се затръби из цялата ви земя.
\par 10 И да осветите петдесетата година и да прогласите освобождение из цялата земя на всичките й жители; това ще ви бъде юбилей, когато ще се върнете, всеки на на притежанието си, и ще се върнете, всеки при семейството си.
\par 11 Петдесетата година да ви бъде юбилейна година ; през нея да не сеете, нито да жънете самораслото, нито да берете нерязаното лозе.
\par 12 Защото това е юбилей; той нека ви бъде свет; от полето да се храните с произведеното от него.
\par 13 В тая юбилейна година да се върнете, всеки на притежанието си.
\par 14 И ако продадеш нещо на ближния си, или купиш нещо от ближния си, да се не онеправдавате едни други;
\par 15 но според числото на годините след юбилея да купуваш от ближния си, и според числото на годините на плодосъбирането да ти продава.
\par 16 Според колкото са по-много годините, ще повишиш цената му, и според колкото са по-малко годините, ще понижиш цената му; защото той ти продава според числото на плодосъбиранията.
\par 17 Да се не онеправдавате един други; но да се боиш от своя Бог; защото Аз съм Иеова вашият Бог.
\par 18 И тъй, да държите повеленията Ми, да пазите съдбите Ми и да ги вършите, та да живеете безопасно на земята.
\par 19 Земята ще дава плодовете си; и вие ще ядете до насита и ще живеете безопасно на нея.
\par 20 И ако речете: Какво ще ядем в седмата година, като не посеем, нито съберем плодовете си?
\par 21 Тогава ще заповядам така да се благослови за вас шестата година, щото ще роди плод за три години.
\par 22 А в осмата година ще сеете и ще ядете от старите плодове до деветата година; догдето се съберат нейните плодове ще ядете старите запаси.
\par 23 Земята да се не продава за всегда, понеже земята е Моя; защото вие сте чужденци и пришелци при Мене.
\par 24 За това в цялата земя, която притежавате, позволявайте откупуване на земята.
\par 25 Ако осиромашее брат ти и продаде нещо от притежанието си, нека дойде на-близкият му сродник и да купи онова, което брат му е продал.
\par 26 Но ако човекът няма сродник да го откупи, и, като се улесни, сам намери с какво да го откупи,
\par 27 тогава нека сметне годините от продажбата му, и нека повърне излишъка на онзи, комуто го е продал, и нека се върне на притежанието си.
\par 28 Но ако не може да намери колкото трябва да плати, тогава продаденото да остане в ръката на купувача му до юбилейната година; и в юбилейната година; и в юбилея да се освободи, и той да се върне на притежанието си.
\par 29 Ако някой продаде къщата за живеене в ограден град, тогава може да я откупи в разстояние на една година от продажбата й; за една цяла година от продажбата й; за една цяла година той ще има право да я откупи.
\par 30 Но ако се не откупи до изтичането на една цяла година тогава оная къща, който е в ограден град, да се потвърди за всегда на купувача й във всичките му поколения в юбилей да се не освобождава.
\par 31 Но същите в неоградените села да се считат, както полетата на земята; те може да се откупуват и да се освобождават в юбилей.
\par 32 Обаче в левитските градове, лавитите могат когато да е да откупуват къщите в градовете, които притежават.
\par 33 И ако някой купи къща от левитите, тогава продадената къща вътре в града, който притежава да се освободи в юбилей; защото къщите в градовете на левитите са тяхно притежание между израилтяните.
\par 34 А полето на пасбището на градовете им да се не продава, защото им е вечно притежание.
\par 35 Ако осиромашее брат ти, и видиш, че ръката му трепери, тогава да му помогнеш, като на чужденец или пришелец, за да живее при тебе.
\par 36 Да не му вземеш лихва или печалба, но да се боиш от своя Бог та да живеете брат ти при тебе.
\par 37 Парите си да не му дадеш с лихва, нито храната си да му дадеш за печалба.
\par 38 Аз съм Господ вашият Бог, Който ви изведох из Египетската земя, за да ви дам Ханаанската земя, и да бъда ваш Бог.
\par 39 Ако осиромашее брат ти при тебе и ти се продаде, да го не натовариш с робска работа.
\par 40 Нека бъде той като наемник или пришелец при тебе; нека ти работи до юбилейната година;
\par 41 тогава да си излезе от тебе, той и чадата му с него, и да се върне при семейството си, и нека ти работи до юбилейната година;
\par 42 Защото те са Мои слуги, които изведох из Египетската земя; да се не продават като роби;
\par 43 да не господаруваш над него жестоко, но да се боиш от своя Бог.
\par 44 А колкото за робите и робините, които ще имаш, - от народите, които са около вас, от тях да купуваш роби и робини.
\par 45 Още и от чадата на чужденците, които са пришелци между вас, от тях да купувате, и от техните семейства, които са между вас, които те са родили в земята ви; те да ви бъдат притежание.
\par 46 И да ги оставите в наследство на чадата си; те да ги наследяват подир вас, като притежание; винаги от тях да бъдат вашите роби; но над братята си, израилтяните, да не господарувате един над друг жестоко.
\par 47 Ако чужденец или пришелецът, който живее при тебе, се замогне, а брат ти осиромашее при него, та се продаде на чужденеца или на пришелеца при тебе, или на кого да е от семейството на чужденеца,
\par 48 то, след като се продаде, той може да се откупи; един от братята му може да го откупи;
\par 49 или стрикът му или стриковият му син може да го откупи, или някой близък сродник от семейството му може да го откупи, или ако той се е улеснил, може сам да откупи сабе си.
\par 50 Тогава нека сметне с купувача си от годината, когато му се е продал до юбилейната година, така, че цената, за която се е продал, да бъде съразмерно с с числото на годините; да му се сметне свързано с времето на наемник.
\par 51 Ако остават още много години, съразмерно с тях да възвърне за откупването си от парите, с които е бил купен.
\par 52 Но ако остават малко години до юбилейната година, нека пресметне и съразмерно с годините да възвърне за откупуването си.
\par 53 Като годишен наемник да бъде при него; той да не господарува над него жестоко пред тебе.
\par 54 Но ако така се не откупи, тогава да си излезе в юбилейната година, той и чадата му с него,
\par 55 защото израилтяните са Мои слуги; те са Мои слуги, които изведох из Египетската земя. Аз съм Господ вашият Бог.

\chapter{26}

\par 1 Да не си правите идоли или кумири, нито да си издигате стълбове, нито да поставяте в земята си камък с изображения, за да му се кланяте: защото Аз съм Господ вашият Бог.
\par 2 Съдбите Ми да пазите и светилището Ми да почитате. Аз съм Господ.
\par 3 Ако ходите по повеленията Ми, и пазите заповедите Ми и ги вършите,
\par 4 тогава ще ви дам дъждовете на времето им, и земята ще даде плодовете си, и полските дървета ще дадат плода си.
\par 5 Вършитбата ви ще трае до гроздобер и гроздоберът ще трае до сеитба; и ще ядете хляба си до насита и ще живеете безопасно в земята си.
\par 6 Ще дам мир на земята и ще си лягате, и никой няма да ви плаши; и ще изтребя лошите зверове от земята, и нож няма да замине през земята ви.
\par 7 Ще гоните неприятелите си, и те ще падат пред вас от нож.
\par 8 Петима от вас ще гонят стотина, а стотина от вас ще гонят десет хиляди; и неприятелите ви ща падат пред вас от нож.
\par 9 Аз ще погледна благоприятно към вас, ще ви направя да нарастете, ще ви размножа и ще утвърдя завета Си с вас.
\par 10 Ще ядете стари жита, да! ще извадите старите за да наместите новите.
\par 11 Още ще поставя скинията Си между вас; и душата Ми ще се погнуси от вас.
\par 12 Ще ходя между вас и ще съм вашият Бог, и вие ще бъдете Мои люде.
\par 13 Аз съм Господ вашият Бог, Който ви изведох из земята на египтяните, за да не им робувате; и строших жеглата на ярема ви и ви направих да ходите изправени.
\par 14 Но ако не Ме послушате и не извършите всичките тия заповеди,
\par 15 и ако отхвърлите повеленията Ми, и ако душата ви се погнуси от съдбите Ми, та да не вършите всичките Ми заповеди, и нарушите завета Ми,
\par 16 тогава ето какво Аз ще ви направя: ще изпратя върху вас ужас, охтика и треска, които ще развалят очите ви и ще стопят душата ви ; и ще сеете семето си напразно, защото неприятелите ви ще го ядат.
\par 17 Ще насоча лицето Си против вас, и ще бъдете избити пред неприятелите и; ония, които ви мразят, ще владеят над вас, и ще бягате, когато никой не ви гони.
\par 18 И ако при всичко това не Ме послушате, тогава ще ви накажа седмократно повече от греховете ви.
\par 19 Ще строша гордата ви сила и ще направя небето ви като желязо и земята ви като мед.
\par 20 Силата ви ще се иждивява напразно, защото земята ви няма да дава плодовете си и дърветата на земята ви няма да дават плода си.
\par 21 И ако ходите противно на Мене и не склоните да Ме слушате, ще наложа върху вас седмократно повече язви според греховете ви.
\par 22 Ще изпратя между вас диви зверове, които ще ви лишат от чадата ви, ще изтребят добитъка ви и ще ви направят да намалявате, та пътищата ви ще запустеят.
\par 23 И ако от това не се поправите и не се обърнете към Мене, но ходите противно на Мене,
\par 24 тогава ще ходя и Аз противно на вас и ще ви поразя, да! Аз, седмократно поради греховете ви.
\par 25 И ще докарам на вас нож, който ще извърши отмъщение за завета; и когато се съберете в градовете си, ще изпратя между вас мор; и ще бъдете предадени в ръката на неприятеля.
\par 26 И когато ви строша подпорката от хляба, десет жени ще пекат хляба ви в една пещ и ще ви върнат хляба с теглилка; и ще ядете, но няма да се насищате.
\par 27 Но ако и след това не Ме послушате, а ходите противно на Мене,
\par 28 то и Аз ще ходя противно на вас с ярост и ще ви накажа, да! Аз, седмократно за греховете ви.
\par 29 Ще ядете месата на синовете си и месата на дъщерите си ще ядете.
\par 30 И ще разоря оброчищата ви, ще съборя кумирите ви, и труповете ви ще хвърля върху труповете на презрителните ви идоли; и душата Ми ще се погнуси от вас.
\par 31 Ще обърна градовете ви на пустиня, и ще запустя светилищата ви и не ще да помириша дъха на благоуханните ви приноси .
\par 32 Ще запустя и земята ви, тъй щото да се смеят за това неприятелите ви, които живеят в нея.
\par 33 А вас ще разпръсна между народите и ще изтръгна нож след вас; и земята ви ще бъде пуста и градовете ви пустиня.
\par 34 Тогава земята ще се радва на съботите си през всичкото време докато е пуста, и докато вие сте в земята на неприятелите си; тогава ще се успокои земята и ще се радва на съботите си.
\par 35 През всичкото време, когато е запустяла, ще си почива, защото не си е почивала в дължимите от вас съботи, когато вие живеехте в нея.
\par 36 И на останалите от вас ще вложа страх в сърцето им в земите на неприятелите им; шум от поклатен лист ще ги погне, и ще бягат, като бягащи от нож; и ще падат, когато никой не ги гони.
\par 37 Ще подат един върху друг, като че ли пред нож, когато никой не ги гони; и ще бъдете безсилни да устоите пред неприятелите си.
\par 38 Ще погинете между народите; и земята на неприятелите ви ще ви пояде.
\par 39 Останалите от вас ще се стопят за беззаконията си в земите на неприятелите си; а още за беззаконията на бащите си ще се стопят заедно с тях.
\par 40 Но ако изповядат беззаконието си и беззаконието на бащите си в престъплението, което са извършили против Мене, и още, че са ходили противно на Мене,
\par 41 и че Аз ходих противно на тях и ги отведох в земята на неприятелите им, - ако тогава се смири езическото им сърце, и приемат наказанието на беззаконието си,
\par 42 тогава ще спомня завета Си с Якова; още завета Си с Исаака и завета Си с Авраама ще спомня; ще спомня и земята.
\par 43 И земята като ще бъде напусната от тях, ще се радва на съботите си, както стои пуста без тях; и те ще приемат наказанието за беззаконието си, за гдето отхвърлиха съдбите Ми и за гдето душата им се отврати от повеленията Ми.
\par 44 Но и така, както са в земята на неприятелите си, няма да ги отхърля, нито ще се отвратя от тях до там щото да ги изтребя и да наруша завета Си с тях; защото Аз съм Господ техен Бог;
\par 45 но заради тях ще си спомня завета, който направих с праотците им, които съм извел из Египетската земя пред очите на народите, за да бъда техен Бог Аз съм Иеова.
\par 46 Тия са повеленията, съдбите, и законите, които Господ направи между Себе Си и израилтяните, чрез Моисея, на Синайската планина.

\chapter{27}

\par 1 Господ говори още на Моисея, казвайки:
\par 2 Говори на израилтяните, като им кажеш: Когато някой направи обрек, ти да направиш оценката на лицата за Господа.
\par 3 Ето каква трябва да бъде оценката ти: на мъжко лице от двадесет години до шестдесет години оценката ти да бъде петдесет сребърни сикли, според сиклата на светилището.
\par 4 И ако е женско лице, оценката ти да бъде тридесет сикли.
\par 5 А ако е лице от пет години до двадесет години, оценката ти да бъде за мъжко двадесет сикли, а за женско десет сикли.
\par 6 Ако е лице от един месец до пет години, оценката ти да бъде за мъжко пет сребърни сикли; а за женско оценката ти да бъде три сребърни сикли.
\par 7 И ако е лице от шестнадесет години нагоре, оценката ти да бъде петнадесет сикли , ако е мъжко, и десет сикли, ако е женско.
\par 8 Но-ако човекът е по-сиромах отколкото си го оценил, да се представи пред свещеника, и свещеникът да го оцени изново ; нека го оцени свещеникът според средствата на онзи, който е направил обрека.
\par 9 Ако обрекът е за животно от ония, които се принасят Господу, всичко що дава някой Господу от тях ще бъде свето.
\par 10 Да го не промени, нито да замени добро животно с по-лошо или лошо с добра; и ако някога замени животното с животно, тогава и едното и другото, което го е заменило, ще бъдат свети.
\par 11 Но ако обрекът е за някое нечисто животно, от ония, които не се принасят Господу, тогава да представи животното пред свещеника;
\par 12 и свещеникът да го оцени според както е добро или лошо; по твоята оценка, о свещениче, така ще бъде.
\par 13 Но ако поиска човекът да го откупи, то върху твоята оценка да придаде петата й част.
\par 14 Когато някой посвети къщата си да бъде света Господу, то свещеникът да я оцени, според както е добра или лоша; както я оцени свещеникът, така ще остане.
\par 15 Но ако оня, който я посвети, поиска да си откупи къщата, нека придаде на парите, с които си я оценил, петата им част, и ще бъде негова.
\par 16 Ако някой посвети Господу част от нивата, която съставлява притежанието му, оценката ти да бъде според семето, което може да се засее в нея; един корен ечемично семе да се оцени за петнадесет сребърни сикли.
\par 17 Ако посвети нивата си от юбилейната година, то по твоята оценка ще остане.
\par 18 Но ако посвети нивата си след юбилея, свещеникът да му пресметне парите според годините, които остават до юбилейната година, и според това да се спадне от оценката ти.
\par 19 Но ако тоя, който е посветил нивата, поиска да я откупи, нека придаде на парите, петата им част, и ще стане негова.
\par 20 Обаче, ако не откупи нивата, или ако е продал нивата другиму да се не откупва вече;
\par 21 а когато се освободи нивата в юбилея, ще бъде Света Господу като нива обречена; ще бъде притежание на свещеника.
\par 22 И ако някой посвети Господу нива, която е купил, която, обаче , не е част от нивата съставляваща притежанието му,
\par 23 свещеникът да му пресметне цената й до юбилейната година, според твоята оценка; и в същия ден нека даде оцененото от тебе, като свето Господу.
\par 24 В юбилейната година нивата да се върне на онзи, от когото е била купена, сиреч, на онзи, комуто се пада земята като притежание.
\par 25 И всичките твои оценки да стават според сикъла на светилището; сикълът да е равен на двадесет гери.
\par 26 Но никой да не посвещава първородното между животните, което, като първородно, принадлежи Господу; говедо или овца, Господно е.
\par 27 И ако е от нечистите животни, нека го откупи според твоята оценка, като придаде върху нея петата й част; или, ако се не откупува, нека се продаде според твоята оценка.
\par 28 Но нито да се продаде, нито да се откупи нещо обречено, което би обрекъл някой Господу от онова що има, било човек или животно или нива от притежанието си; всяко нещо обречено е пресвето Господу.
\par 29 Никое обречено нещо, обречено от човек, да не се откупи; то трябва непременно да се умъртви.
\par 30 Всеки десетък от земята, било от посевите на земята, или от плода на дърветата, е Господен; свет е Господу.
\par 31 Ако някой поиска да откупи нещо от десетъка си, нека придаде петата част на цената му .
\par 32 И всеки десетък от черда и от стада, десетък от всичко що минава при преброяване под жезъла, да бъде свет на Господа.
\par 33 Да не издирва посветителят добро ли и животното или лошо, нито да го промени; но ако го промени някога, то и едното и другото, което го е заменило, да бъдат свети; животното да се не откупува.
\par 34 Тия са заповедите, които Господ заповяда на Моисея за израилтяните на Синайската планина.

\end{document}