\begin{document}

\title{Numbers}


\chapter{1}

\par 1 И Господ, говорейки на Моисея в Синайската пустиня, в шатъра за срещане, на първия ден от втория месец на втората година откак излязоха от Египетската земя, каза:
\par 2 Пребройте цялото общество израилтяни по семействата им, по бащините им домове, като броите по име всеки от мъжки пол един по един.
\par 3 Всеки в Израиля от двадесет години нагоре, които могат да излизат на бой, ти и Аарон пребройте ги според установените им войнства.
\par 4 И заедно с вас да има по един човек от всяко племе, от които всеки да бъде глава на бащиния си дом.
\par 5 И ето имената на мъжете, които ще стоят заедно с вас: от Рувима, Елисур, Седиуровият син;
\par 6 От Симеона, Селумиил, Сурисадаевият син;
\par 7 от Юда, Наасон, Аминадавовият син;
\par 8 от Исахара, Натанаил, Суаровият син;
\par 9 от Завулона, Елиав, Хелоновият син;
\par 10 от Иосифовите потомци: от Ефрема, Елисама, Амиудовият син, а от Манасия, Гамалиил, Федасуровият син;
\par 11 от Вениамина, Авидан, Гедеониевият син;
\par 12 от Дана, Ахиезар, Амисадаевият син;
\par 13 от Асира, Фагеил, Охрановият син;
\par 14 от Гада, Елиасаф, Деуиловият син;
\par 15 от Нефталима, Ахирей, Енановият син.
\par 16 Тия бяха избраните от обществото, началници на бащините си племена, глави на Израилевите хиляди.
\par 17 И тъй, Моисей и Аарон, като взеха тия мъже, които са били споменати по име,
\par 18 свикаха цялото общество на първия ден от втория месец; и те се записаха един по един по семействата си, по бащините си домове, според числото на имената на мъжете от двадесет години и нагоре.
\par 19 Както заповяда Господ на Моисея, така той ги изброи в Синайската пустиня.
\par 20 Потомците на Рувима, първородния на Израиля, поколенията им по семействата им, по бащините им домове, по числото на имената, един по един, всички от мъжки пол от двадесет години и нагоре, всички които можеха да излизат на бой,
\par 21 преброени от Рувимовото племе, бяха четиридесет и шест хиляди и петстотин души.
\par 22 От симеонците, поколенията им по семействата им, по бащините им домове, които се преброиха по числото на имената, един по един, всички от мъжки пол от двадесет години и нагоре, всички, които, можеха да излизат на бой,
\par 23 преброени от Симеоновото племе, бяха петдесет и девет хиляди и триста души.
\par 24 От гадците, поколенията им по семействата им, по бащините им домове, по числото на имената от двадесет години и нагоре, всички, които можеха да излизат на бой,
\par 25 преброени от Гадовото племе, бяха четиридесет и пет хиляди шестстотин и петдесет души.
\par 26 От юдейците, поколенията им по семействата им, по бащините им домове, по числото на имената от двадесет години и нагоре, всички, които можеха да излизат на бой,
\par 27 преброени от Юдовото племе, бяха седемдесет и четири хиляди и шестстотин души.
\par 28 От исахарците, поколенията им по семействата им, по бащините им домове, по числото на имената от двадесет години и нагоре, всички, които можеха да излизат на бой,
\par 29 преброени от Исахаровото племе, бяха петдесет и четири хиляди и четиристотин души.
\par 30 От завулонците, поколенията им по семействата им, по бащините им домове, по числото на имената от двадесет години и нагоре, всички, които можеха да излизат на бой,
\par 31 преброени от Завулоновото племе, бяха петдесет и седем хиляди и четиристотин души.
\par 32 От Иосифовите потомци: сиреч , от ефремците, поколенията им по семействата им, по бащините им домове, по числото на имената от двадесет години и нагоре, всички, които можеха да излизат на бой,
\par 33 преброени от Ефремовото племе, бяха четиридесет хиляди и петстотин души;
\par 34 а от манасийците, поколенията им по семействата им, по бащините им домове, по числата на имената от двадесет години и нагоре, всички, които можеха да излизат на бой,
\par 35 преброени от Манасиевото племе, бяха тридесет и две хиляди и двеста души.
\par 36 От вениаминците, поколенията им по семействата им, по бащините им домове, по числото на имената от двадесет години и нагоре, всички, които можеха да излизат на бой,
\par 37 преброени от Вениаминовото племе, бяха тридесет и пет хиляди и четиристотин души.
\par 38 От данците, поколенията им по семействата им, по бещините им домове, по числото на имената от двадесет години и нагоре, всички, които можеха да излизат на бой,
\par 39 преброени от Дановото племе, бяха шестдесет и две хиляди и седемстотин души.
\par 40 От асирците, поколенията им по семействата им, по бащините им домове, по числото на имената от двадесет години и нагоре, всички, които можеха да излизат на бой,
\par 41 преброени от Асировото племе, бяха четиридесет и една хиляда и петстотин души.
\par 42 От Нефталимците, поколенията им по семействата им, по бащините им домове, по числото на имената от двадесет години и нагоре, всички, които можеха да излизат на бой,
\par 43 преброени от Нефталимовото племе, бяха петдесет и три хиляди и четиристотин души.
\par 44 Тия са изброените, които преброиха Моисей, Аарон и дванадесетте мъже Израилеви първенци, всеки един за бащиния си дом.
\par 45 И тъй, всичките преброени от израилтяните по бащините им домове, от двадесет години и нагоре, всички между Израиля, които можеха да излизат на бой,
\par 46 всичките преброени бяха шестстотин и три хиляди петстотин и петдесет души.
\par 47 А левитите не бяха преброени помежду им по бащиното им племе.
\par 48 Защото Господ, говорейки на Моисея, беше рекъл:
\par 49 Само Левиевото племе да не преброиш, нито да вземеш числото им между израилтяните;
\par 50 но да поставиш левитите за настоятели на скинията, за плочите на свидетелството, и на всичките нейни принадлежности, и на всичките нейни вещи; те да носят скинията и всичките нейни принадлежности, и те да вършат служението около нея, и да поставят стана си около скинията.
\par 51 И когато трябва да се дига скинията, левитите да я снемат; и когато трябва да се разпъва скинията, левитите да я поставят; а чужденец, който би се приближил до нея, да се умъртви.
\par 52 И израилтяните да поставят шатрите си, всеки в стана си и всеки при знамето си, според установените си войнства.
\par 53 А левитите да поставят шатрите си около скинията за плочите на свидетелството, за да не падне гняв върху обществото на израилтяните; и левитите да пазят заръчаното за скинията на свидетелството.
\par 54 И израилтяните сториха така; напълно както Господ заповяда на Моисея, така направиха.

\chapter{2}

\par 1 И Господ говори на Моисея и Аарона, казвайки:
\par 2 Израилтяните нека поставят шатрите си, всеки при знамето си, със знаковете на бащиния си дом; срещу шатъра за срещане да поставят шатрите си изоколо.
\par 3 Тия, които поставят шатрите си от предната страна, към изток, да бъдат от знамето на Юдовия стан, според установените си войнства; и началникът на юдейците да бъде Наасон, Аминадавовият син.
\par 4 (А неговото войнство, сиреч, преброените от тях, бяха седемдесет и четири хиляди и шестстотин души).
\par 5 До него да поставят шатрите си Исахаровото племе; и началник на исахарците да бъде Натанаил, Суаровият син;
\par 6 (а неговото множество, сиреч, преброените от тях, бяха петдесет и четири хиляди и четиристотин души);
\par 7 и Завулоновото племе; и началникът на завулонците да бъде Елиав, Хелоновият син;
\par 8 (а неговото множество, сиреч, преброените от тях, бяха петдесет и седем хиляди и четиристотин души.
\par 9 Всичките преброени в Юдовия стан бяха сто и осемдесет и шест хиляди и четиристотин души, според устроените им войнства). Те да се дигат първи.
\par 10 Към юг да бъде знамето на Рувимовия стан, според устроените им войнства; и началник на рувимците да бъде Елисур, Седиуровият син.
\par 11 (А неговото войнство, сиреч, преброените от тях, бяха четиридесет и шест хиляди и петстотин души).
\par 12 До него да поставят шатрите си Симеоновото племе; и началник на симеонците да бъде Селумиил, Сурисадаевият син;
\par 13 (а неговото войнство, сиреч, преброените от тях, бяха петдесет и девет хиляди и триста души)
\par 14 и Гадовото племе; и началник на гадците да бъде Елисаф, Деуиловият син;
\par 15 (а неговото войнство, сиреч, преброените от тях, бяха четиридесет и пет хиляди шестстотин и петдесет души.
\par 16 Всичките преброени в Рувимовия стан бяха сто и петдесет и една хиляда и петдесет души, според устроените им войнства). Те да се вдигат втори.
\par 17 После да се вдига шатъра за срещане, със стана на левитите всред становете; както са поставили шатрите си, така и да се дигат, всеки на реда си при знамето си.
\par 18 Към запад да бъде знамето на Ефремовия стан, според устроените им войнства; и началник на ефремците да бъде Елисама, Амиудовият син.
\par 19 (А неговото войнство, сиреч, преброените от тях, бяха четиридесет хиляди и петстотин души).
\par 20 До него да бъде Манасиевото племе; и началникът на манасийците да бъде Гамалиил, Федасуровият син;
\par 21 (а неговото войнство, сиреч, преброените от тях, бяха тридесет и две хиляди и двеста души);
\par 22 и Вениаминовото племе; и началник на вениаминците да бъде Авидан, Гедеоновият син;
\par 23 (а неговото войнство, сиреч, преброените от тях, бяха тридесет и пет хиляди и четиристотин души.
\par 24 Всичките преброени от Ефремовия стан бяха сто и осем хиляди и сто души, според устроените си войнства). Те да се дигат трети.
\par 25 Към север да бъде знамето на Дановия стан, според устроените им войнства; и началник на данците да бъде Ахиезер, Амисадаевият син.
\par 26 (А неговото войнство, сиреч, преброените от тях, бяха шестстотин и две хиляди и седемстотин души).
\par 27 До него да поставят шатрите си Асировото племе; и началник на асирците да бъде Фагеил, Охрановият син;
\par 28 (а неговото войнство, сиреч, преброените от тях, бяха четиридесет и една хиляда и петстотин души);
\par 29 и Нефталимовото племе; и началник на нефталимците да бъде Ахирей, Енановият син;
\par 30 (а неговото войнство, сиреч, преброените от тях бяха петдесет и три хиляди и четиристотин души.
\par 31 Всичките преброени в Дановия стан бяха сто и петдесет и седем хиляди и шестстотин души). Те да се дигат последни при знамената си.
\par 32 Тия са преброените от израилтяните по бащините им домове; всички, които бяха преброени в становете по устроените им войнства, бяха шестстотин и три хиляди петстотин и петдесет души.
\par 33 А левитите не бяха преброени между израилтяните, според както Господ заповяда на Моисея.
\par 34 И тъй, израилтяните правеха напълно това що Господ заповяда на Моисея; така поставяха шатрите си при знамената си, така се и дигаха, всеки според семействата си, по бащините си домове.

\chapter{3}

\par 1 А ето поколенията на Аарона и Моисея във времето, когато Господ говори на Моисея на Синайската планина.
\par 2 Ето и имената на Аароновите синове: Надав, първородният му, Авиуд, Елеазар и Итамар.
\par 3 Тия са имената на Аароновите синове, помазаните свещеници, които Моисей посвети, за да свещенодействуват.
\par 4 А Надав и Авиуд умряха пред Господа, като принасяха чужд огън пред Господа в Синайската пустиня; и нямаха чада; а Елеазар и Итамар свещенодействуваха в присъствието на баща си Аарона.
\par 5 И Господ говори на Моисея, казвайки:
\par 6 Приведи Левиевото племе и представи ги пред свещеника Аарона, за да му слугуват.
\par 7 Нека пазят заръчаното от него и заръчаното от цялото общество пред шатъра за срещане, за да вършат служенето около скинията.
\par 8 И нека пазят всичките принадлежности на шатъра за срещане, и заръчаното от израилтяните, за да вършат служенето около скинията.
\par 9 И да дадеш левитете на Аарона и на синовете му: те са дадени всецяло нему от страна на израилтяните.
\par 10 А Аарона и синовете му да поставиш да вършат свещеническите си служби; а чужденецът, който би се приближил, да се умъртви.
\par 11 Господ още говори на Моисея, казвайки:
\par 12 Ето, Аз взех левитете измежду израилтяните, вместо всичките първородни от израилтяните, които отварят утроба; левитете ще бъдат Мои.
\par 13 Защото всяко първородно е Мое; в деня, когато поразих всяко първородно в Египетската земя, Аз осветих за Себе Си всяко първородно в Израиля, и човек и животно; Мои ще бъдат. Аз съм Иеова.
\par 14 И Господ говори на Моисея в Синайската пустиня, казвайки:
\par 15 Преброй левийците според бащините им домове, по семействата им; да преброиш всичките мъжки от един месец и нагоре.
\par 16 И тъй, Моисей ги преброи според Господното слово, както му беше заповядано.
\par 17 А синовете на Леви, по имената си, бяха тия: Гирсон, Каат и Мерарий.
\par 18 И ето имената на гирсонците по семействата им: Левий и Семей;
\par 19 и каатците по семействата им: Амрам, Исаар, Хеврон и Озиил;
\par 20 и мерарийците по семействата им: Маалий и Мусий. Тия са семействата на левийците, според бащините им домове.
\par 21 От Гирсона произлезе семейството Левиево и семейството Семеево; тия са семействата на гирсонците.
\par 22 Преброените от тях по числото на всичките мъжки от един месец и нагоре, които се преброиха от тях, бяха седем хиляди и петстотин души.
\par 23 Семействата на гирсонците да поставят шатрите си зад скинията към запад.
\par 24 И началник на бащиния дом на гирсонците да бъде Елиасаф, Лаиловият син.
\par 25 А под грижата на гирсонците в шатъра за срещане да бъдат скинията, шатърът, покривът му, закривката за входа на шатъра за срещане,
\par 26 дворните завеси, закривката на входа на двора, който е около скинията и олтарът и въжетата му за цялата ме служба.
\par 27 От Каата произлезе семейството на Амрамовците, и семейството на Хевроновците, и семейството на Озииловците: тия са семействата на Каатовците.
\par 28 Според числото на всичките мъжки от един месец и нагоре, имаше осем хиляди и шестстотин души, които пазеха поръчаното за светилището.
\par 29 Семейството на Каатовците да поставят шатрите си откъм южната страна на скинията.
\par 30 И началник на бащиния дом от семействата на Каатовците да бъде Елисафан, Озииловият син.
\par 31 А под тяхната грижа да бъдат ковчегът, трапезата, светилникът, олтарите, принадлежностите на светилището, с които служат, закривката и всичко, което принадлежи на службата му.
\par 32 И Елеазар, син на свещеника Аарона, да бъде началник над левитските началници и да надзирава ония, които пазят заръчаното за светилището.
\par 33 От Мерария произлезе семейството на Маалиевците и семейството на Мусиевците; тия са Мерариевите семейства.
\par 34 Които от тях се преброиха, според числото на всичките мъжки от един месец и нагоре, бяха шест хиляди и двеста души.
\par 35 И началник на бащиния дом от семействата на Мерариевците да бъде Суриил, Авихаиловият син. Те да поставят шатрите си откъм северната страна на скинията.
\par 36 И под грижата, назначена на мерарийците, да бъдат дъските на скинията, лостовете й, стълбовете й, подложките й, всичките й прибори, всичко, което принадлежи на службата й,
\par 37 стълбовете на околния двор, подложките им, колчетата им и въжетата им.
\par 38 Тия, които ще поставят шатрите си пред лицето на скинията към изток, пред шатъра за срещане към изгрев слънце, да бъдат Моисей и Аарон и синовете му, които да имат грижа за светилището, сиреч, грижа за израилтяните; и чужденец, който бе се приближил, да се умъртви.
\par 39 Всичките преброени от левитите, които Моисей и Аарон преброиха по семействата им, по Господната заповед, всичките мъжки от един месец и нагоре, бяха двадесет и две хиляди души.
\par 40 И Господ рече на Моисея: Преброй всичките мъжки първородни от израилтяните от един месец и нагоре и вземи числото на имената им.
\par 41 И да вземеш левитите за Мене, (Аз съм Господ) вместо всичките първородни между израилтяните, и добитъка на левитите вместо всичките първородни между добитъка на израилтяните.
\par 42 И тъй Моисей преброи всичките първородни между израилтяните, според както Господ му заповяда;
\par 43 и всичките мъжки първородни като се изброиха по име от един месец и нагоре, според преброяването им, бяха двадесет и две хиляди двеста и седемдесет и три души.
\par 44 Господ говори още на Моисея, казвайки:
\par 45 Вземи левитите, вместо всичките първородни между израилтяните, и добитъка на левитите, вместо техния добитък; и левитите ще бъдат Мои. Аз съм Господ.
\par 46 А за откупуване на двестате седемдесет и три души, с които първородните измежду израилтяните са повече от левитите,
\par 47 да вземеш по пет сикли на глава; според сикъла на светилището да ги вземеш (един сикъл е равен на двадесет гери);
\par 48 и парите на откупа от ония, които са повече, да дадеш на Аарона и на синовете му.
\par 49 И така, Моисей взе парите на откупа от ония, които бяха повече от изкупените през размяна с левитите;
\par 50 от първородните на израилтяните взе парите, хиляда триста и шестдесет и пет сикли според сикъла на светилището.
\par 51 И Моисей даде парите от откупа на Аарона и на синовете му, според Господното слово, както Господ заповяда на Моисея.

\chapter{4}

\par 1 И Господ говори на Моисея и Аарона, казвайки:
\par 2 Измежду левийците преброй каатците по семействата им, по бащините им домове,
\par 3 от тридесет години и нагоре до петдесет години, всички, които влизат в отреда да вършат работа в шатъра за срещане.
\par 4 Ето, службата на каатците в шатъра за срещане ще бъде около пресветите неща;
\par 5 когато се дигна станът, Аарон и синовете му ще пристъпват и ще снемат закривалната завеса, ще закрият с нея ковчега за плочите на свидетелството,
\par 6 и ще турят на него покрива от язовски кожи, а отгоре ще разпрострат плат цял от синьо и ще проврат върлините му.
\par 7 Върху трапезата на присъствените хлябове ще разпрострат син плат, на който ще турят блюдата, темянниците, тасовете и поливалките за поливането; и постоянните хлябове ще бъдат на нея;
\par 8 и върху тях ще разпрострат червен плат, и него ще покрият с покрив от язовски кожи, и ще проврат върлините й.
\par 9 После ще вземат син плат и ще покрият светилника за осветлението, светилата му, щипците му, пепелниците му и маслениците му, с които си служат около него;
\par 10 и ще турят него и всичките му прибори вътре в покрива от язовски кожи и ще го окачат на лост.
\par 11 А върху златния олтар ще разпрострат син плат, и него ще покрият с покрив от язовски кожи, и ще проврат върлините му.
\par 12 Ще вземат и всичките му служебни прибори, с които служат в светото място, и ще ги турят в син плат, ще ги покрият с покрив от язовски кожи и ще ги окачат на лост.
\par 13 Тогава, като очистят пепелта от олтара, ще разпрострат на него морав плат,
\par 14 и ще положат на него всичките му прибори, с които си служат около него, - въглениците, вилиците, лопатите и легените, всичките прибори на олтара, - и ще разпрострат върху него покрив от язовски кожи и ще проврат върлините му.
\par 15 И като свършат Аарон и синовете му с покриването на светите вещи и всичките свети прибори, когато трябва да се дига станът, тогава да пристъпват каатците; за да ги носят: но да се не докосват до светите неща, за да не умрат. Тия неща от шатъра за срещане ще носят каатците.
\par 16 И Елеазар, син на свещеника Аарона, ще има надзор над маслото за осветлението, благоуханния темян, постоянния хлебен принос и мирото за помазване, - надзор над цялата скиния и над всичките неща, които са в нея, - над светилището и принадлежностите му.
\par 17 Господ говори още на Моисея и Аарона, казвайки:
\par 18 Да не изтребите измежду левитите племето на семействата на Каатовците;
\par 19 но, за да останат живи левитите и да не умрат, когато пристъпват при пресветите неща, правете им така: Аарон и синовете му да влизат и да ги поставят всекиго на службата му и на товара му;
\par 20 но те да не влизат да видят светите неща ни за минутка, за да не умрат.
\par 21 Господ говори на Моисея, казвайки:
\par 22 Така също преброй гирсонците по бащините им домове, по семействата им;
\par 23 от тридесет години и нагоре до петдесет години да ги преброиш, всички, които влизат в отреда, за да вършат службата на шатъра за срещане.
\par 24 Ето службата на семействата на гирсонците при слугуването им и при носенето им на товари :
\par 25 ще носят завесите на скинията и шатъра за срещане, покрива му, покрива от язовски кожи, който е отгоре му, и закривката на входа на шатъра за срещане,
\par 26 и дворните завеси, закривката на вратата при входа на двора, който е около скинията и олтара, въжата им, всичките прибори за службата им и каквото е потребно за тия неща; така те ще слугуват.
\par 27 Всичкото служене на Гирсоновците, относно всичкото им носене на товари и всичкото им слугуване, ще бъде според повелението на Аарона и на синовете му; и вие ще им определяте всяко нещо, което те са длъжни да носят.
\par 28 Това е службата на семействата на Гирсоновците в шатъра за срещане; и заръчаното на тях ще бъде под надзора на Итамара, сина на свещеника Аарона.
\par 29 Ще преброиш и мерарийците по семействата им, по бещините им домове;
\par 30 от тридесет години и нагоре до петдесет години ще ги преброиш, всички, които влизат в отреда, за да вършат службата на шатъра за срещане.
\par 31 И ето нещата, които са длъжни да носят през всичкото си слугуване в шатъра за срещане: дъските на скинията, лостовете й, стълбовете й, подложките й
\par 32 и стълбовете на околния двор, подложките им, колчетата им, въжетата им, заедно с всичките им прибори и всичко що им принадлежи; и да определяте по име вещите, които те са длъжни да носят.
\par 33 Това е службата на семействата на мерарийците във всичкото им слугуване в шатъра за срещане, под надзора на Итамара, син на свещеника Аарона.
\par 34 И тъй, Моисей и Аарона и първенците на обществото преброиха Каатовците по семействата им и по бещините им домове,
\par 35 от тридесет години и нагоре до петдесет години, всички, които влизаха в отреда, за да вършат службата на шатъра за срещане;
\par 36 и преброените от тях по семействата им бяха две хиляди седемстотин и петдесет души.
\par 37 Тия се преброени от семействата на Каатовците, всички, които слугуваха в шатъра за срещане, който Моисей и Аарона преброиха, според както Господ заповяда чрез Моисея.
\par 38 А преброените от гирсонците по семействата си и по бащините си домове,
\par 39 от тридесет години и нагоре до петдесет години, всички, които влизаха в отреда, за да вършат службата на шатъра за срещане,
\par 40 преброените от тях по семействата им, по бащините им домове, бяха две хиляди шестотин и тридесет души.
\par 41 Тия са преброените от семействата на гирсонците, всички, които слугуваха в шатъра за срещане, които Моисей и Аарон преброиха по Господното повеление.
\par 42 А преброените от семействата на мерарийците по семействата им, по бащините им домове,
\par 43 от тридесет години и нагоре до петдесет години, всички, които влизаха в отреда да слугуват в шатъра за срещане,
\par 44 преброени от тях по семействата им бяха три хиляди и двеста души.
\par 45 Тия са преброените от семействата на мерарийците, които Моисей и Аарон преброиха, според както Господ заповяда чрез Моисея.
\par 46 Всички, които бяха преброени от левитете, които Моисей и Аарон и Израилевите първенци преброиха, по семействата им и по бащините им домове,
\par 47 от тридесет години и нагоре до петдесет години, всички, които влизаха в шатъра за срещане да слугуват и да носят товари,
\par 48 ония от тях, които бяха преброени, бяха осем хиляди петстотин и осемдесет души.
\par 49 Преброиха се, според както Господ заповяда чрез Моисея, всички според службата си и според товара си. Така се преброиха от него, според както Господ заповяда на Моисея.

\chapter{5}

\par 1 И Господ говори на Моисея, казвайки:
\par 2 Заповядай на израилтяните да извеждат вън от стана всеки, който е прокажен, и всеки, който има течение, и всеки, който е нечист от мъртвец.
\par 3 Както от мъжки, така и от женски пол изваждайте ги; вън из стана ги изваждайте, за да не мърсят становете си, всред които Аз обитавам.
\par 4 И израилтяните сториха така, и изведоха ги вън от стана; както рече Господ на Моисея, така сториха израилтяните.
\par 5 Господ говори още на Моисея, казвайки:
\par 6 Кажи на израилтяните: Когато мъж или жена, като човек, направи какъв да е грях, и стори престъпление против Господа, и тоя човек стане виновен,
\par 7 тогава да изповяда греха, който е сторил, и да повърне онова, за което е виновен и, като притури на него една пета част, да го даде на онзи, пред когото се е провинил.
\par 8 Но ако човекът няма сродник, комуто да повърне онова, за което се е провинил, тогава това, което поради виновността трябва да се повърне на Господа, нека бъде на свещеника, заедно с овена на умилостивение за него.
\par 9 Всеки възвишаем принос от всичките осветени неща на израилтяните, който донасят на свещеника, нека бъде негов.
\par 10 Негови да бъдат й посветените неща от всеки човек; всичко, каквото два някой на свещеника, нека бъде негово.
\par 11 Господ говори още на Моисея, казвайки:
\par 12 Говори на израилтяните, казвайки им: Ако жената на някого прегреши и направи престъпление против него,
\par 13 като лежи някой с нея и излее семе, и това се укрие от очите на мъжа й, и тя се оскверни тайно, без да има свидетел против нея, и без да бъде хваната в делото ,
\par 14 и дойде на него духа на ревност и ревнува жена си, а тя е осквернена, или му дойде дух на ревнивост и ревнува жена си, а тя не е осквернена,
\par 15 тогава тоя човек да доведе жена си при свещеника и да донесе приноса й за нея, една десета от ефа ечемичено брашно; но с дървено масло да го не полее, нито да тури върху него ливан, защото за спомен, който напомня за беззаконие.
\par 16 Тогава свещеникът да я приведе и да я постави пред Господа.
\par 17 После свещеникът да вземе света вода в пръстен съд; и свещеникът да вземе от пръстта, която е по пода на скинията и да я тури във водата.
\par 18 И свещеникът, като постави жената пред Господа, да открие главата на жената и да тури в ръцете й приноса за спомен, приноса за ревнивост; и свещеникът да има в ръката си горчивата вода, която докарва проклетия.
\par 19 И свещеникът да я закълне, като рече на жената: Ако не е легнал някой с тебе и ти не си се отклонила в нечистота, като си под закона на мъжа си, бъди неповредена от тая горчива вода, която докарва проклетия;
\par 20 но ако си прегрешила, като си под закона на мъжа си, и си се осквернила и, ако е лежал с тебе някой освен мъжа ти,
\par 21 (тогава свещеникът да закълне жената с клетва на проклетия, и свещеникът да рече на жената:) Господ да те постави за проклинане и клетва в народа ти, като направи Господ да изсъхне бедрото ти и да се надуе коремът ти;
\par 22 и да влезе тая вода, която докарва проклетия, във вътрешностите ти, и да надуе корема ти и да изсуши бедрото ти; и жената нека рече: Амин, амин.
\par 23 После свещеникът да напише тия клетви на книга и да ги заличи с горчивата вода;
\par 24 и да даде на жената да изпие горчивата вода, която докарва проклетия; и като влезе в нея водата, която докарва проклетия, да стане горчива.
\par 25 Тогава свещеникът да вземе от ръката на жената приноса на ревнивостта, да подвижи тоя принос пред Господа и да го принесе на олтара;
\par 26 и свещеникът да вземе една шепа от приноса за спомен, да го изгори на олтара и след това да даде на жената да изпие водата.
\par 27 А когато я напои с водата, тогас, ако е осквернена и е направила престъпление против мъжа си, водата, която докарва проклетия, като влезе в нея ще стане горчива и ще надуе корема й, и бедрото й ще изсъхне; и тая жена ще бъде за проклетия в народа си.
\par 28 Но ако жената не е осквернена и е чиста, тогава ще остане неповредена и ще зачнува,
\par 29 Това е законът за ревнивостта, когато някоя жена, като е под закона на мъжа си, се оскверни,
\par 30 или когато дойде дух на ревнивост върху някой мъж и ревнува жена си, тогава нека постави жената пред Господа, и свещеникът нека постъпи с нея във всичко според тоя закон.
\par 31 Така мъжът ще бъде чист от беззаконие, а жената ще носи беззаконието си.

\chapter{6}

\par 1 Господ говори още на Моисея, казвайки:
\par 2 Говори на израилтяните, казвайки им: Когато мъж или жена направи изричен обрек на назирейство, за да посвети себе си Господу,
\par 3 тогава да се отказва от вино и от спиртни питиета, да не пие оцет от вино или оцет от спиртни питиета, нито да пие какво да е питие направено от грозде, нито да яде прясно или сухо грозде.
\par 4 През всичкото време на назирейството си да не яде нищо, което се прави от лозе, от зърното до ципата.
\par 5 През всичкото време на назирейския си обрек да не тури бръснач на главата си; догдето се изпълни времето, което е обрекъл Господу, ще бъде свет, и нека остави да растат космите на главата му.
\par 6 През всичкото време на обрека си, че ще бъде назирей Господу, да не се приближи при мъртвец;
\par 7 да се не оскверни, заради баща си или майка си, брата си или сестра си, когато умрат; понеже назирейският обрек на Бога му е на главата му.
\par 8 През всичкото време на назирейството си той е свет Господу.
\par 9 И ако някой умре ненадейно при него, и главата на назирейството му се оскверни, тогава, в деня на очищението си, той да обръсне главата си; на седмия ден да я обръсне.
\par 10 А на осмия ден да донесе две гургулици или две гълъбчета при свещеника при входа на шатъра за срещане;
\par 11 и свещеникът да принесе едното в принос за грях, а другото за всеизгаряне; и да направи умилостивение за него, понеже е съгрешил поради мъртвеца, и в същия ден да освети главата му.
\par 12 След това изново да посвети Господу дните на назирейството си и да донесе едногодишно агне в принос за престъпление; а по-предишните дни няма да се считат, защото се е осквернило назирейството му.
\par 13 И ето законът за назирея, когато се изпълни времето на назирейството му: да се приведе при входа на шатъра за срещане;
\par 14 и той да принесе в принос за себе си Господу едно мъжко едногодишно агне без недостатък в принос за грях, един овен без недостатък за примирителен принос,
\par 15 кош с безквасни пити от чисто брашно, смесени с дървено масло, безквасни кори, намазани с масло и хлебния им принос с възлиянията им.
\par 16 И свещеникът да ги представи пред Господа, и да принесе приноса му за грях и всеизгарянето му;
\par 17 и да принесе овена за примирителна жертва Господу, с коша на безквасните хлябове; свещеникът да принесе още хлебния принос с възлиянието му.
\par 18 И назиреят да обръсне главата на назирейството си при входа на шатъра за срещане и да вземе космите на посветената си глава и да ги тури на огъня, който е под примирителната жертва.
\par 19 И свещеникът да вземе сварената плешка на овена, една безквасна пита от коша и една безквасна кора и да ги тури на ръцете на назирея след като обръсне той главата на назирейството си;
\par 20 и свещеникът да ги подвижи за движим принос пред Господа; това, заедно с гърдите на движимия принос и бедрото на възвишаемия принос, е свето на свещеника; и след това назиреят може да пие вино.
\par 21 Това е закон за назирея, който е направил обрек и за приноса му Господ поради назирейството му, освен онова, което му дава ръка; съгласно с обрека, който е направил, така да постъпва според закона за назирейството си.
\par 22 Господ говори още на Моисея, казвайки:
\par 23 Говори на Аарона и на синовете му, казвайки: Така благославяйте израилтяните, като им говорите:
\par 24 Господ да те благослови и да те пази!
\par 25 Господ да осияе с лицето Си над тебе и да ти покаже милост!
\par 26 Господ да издигне лицето Си над тебе и да ти даде мир!
\par 27 Така да възлагат Името Ми върху израилтяните; и Аз ще ги благославям.

\chapter{7}

\par 1 И когато Моисей свърши поставянето на скинията и помаза и свети я с всичките й принадлежности, и олтара с всичките му прибори, и ги помаза и ги освети,
\par 2 тогава Израилевите първенци, началниците на бащините им домове, които бяха първенци на племената и поставени главни надзиратели при преброяването, донесоха принос;
\par 3 и поставиха приносите си пред Господа, шест покрити коли и дванадесет вола, по една кола от двама първенци, и по един вол от всекиго, и представиха ги пред скинията.
\par 4 Тогава Господ говори на Моисея, казвайки:
\par 5 Приеми тия неща от тях, и нека служат за вършенето работата на шатъра за срещане; и дай ги на левитите, на всекиго според работата му.
\par 6 И тъй, Моисей взе колите и воловете и ги даде на лавитите;
\par 7 двете коли и четирите вола даде на гирсонците, според работата им;
\par 8 и четирите коли и осемте вола даде на на мерарийските синове, според работата им, под надзора на Итамара, син на свещеника Аарона.
\par 9 А на каатците не даде; защото тяхната работа в светилището беше до носят на рамена.
\par 10 И в деня, когато олтарът биде помазан, първенците принесоха за освещаването му; и първенците принесоха приносите си пред олтара.
\par 11 И Господ рече на Моисея: Нека принасят приносите си за освещаването на олтара по един първенец на ден.
\par 12 И тоя, който принесе приноса си на първия ден, беше Наасон, Аминадавовият син, от Юдовото племе;
\par 13 и приносът му беше едно сребърно блюдо тежко сто и тридесет сикли ; един сребърен леген от седемдесет сикли, според сикъла на светилището; и двете пълни с чисто брашно, смесено с дървено масло, за хлебния принос;
\par 14 един златен темянник от десет сикли , пълен с темян;
\par 15 един юнец, един овен, едно едногодишно агне за всеизгаряне;
\par 16 един козел в принос за грях;
\par 17 и за примирителна жертва два вола, пет овена, пет козела и пет едногодишни агнета. Това беше приносът на Наасона, Аминадавовият син.
\par 18 На втория ден принесе Натанаил, Суаровият син, първенецът на Исахаровото племе ;
\par 19 и за приноса си принесе едно сребърно блюдо тежко сто и тридесет сикли ; един сребърен леген от седемдесет сикли, според сикъла на светилището; и двете пълни с чисто брашно, смесено с дървено масло за хлебен принос;
\par 20 един златен темянник от десет сикли , пълен с темян;
\par 21 един юнец, един овен, едно едногодишно агне за всеизгаряне;
\par 22 един козел в принос за грях; и един козел в принос за грях;
\par 23 и за примирителна жертва два вола, пет овена, пет козела и пет едногодишни агнета. Това беше приносът на Натанаила, Суаровият син.
\par 24 На третия ден принесе първенецът на завулонците, Елиав, Хелоновият син.
\par 25 Приносът му беше едно сребърно блюдо тежко сто и тридесет сикли ; един сребърен леген от седемдесет сикли, според сикъла на светилището; и двете пълни с чисто брашно, смесено с дървено масло за хлебен принос;
\par 26 един златен темянник от десет сикли , пълен с темян;
\par 27 един юнец, един овен, едно едногодишно агне за всеизгаряне;
\par 28 един козел в принос за грях;
\par 29 и за примирителна жертва два вола, пет овена, пет козела и пет едногодишни агнета. Това беше приносът на Елиава, Хелоновият син.
\par 30 На четвъртия ден принесе Елисур, Седиуровият син, първенецът на рувимците.
\par 31 Приносът му беше едно сребърно блюдо тежко сто и тридесет сикли ; един сребърен леген от седемдесет сикли, според сикъла на светилището; и двете пълни с чисто брашно, смесено с дървено масло за хлебен принос;
\par 32 един златен темянник от десет сикли , пълен с темян;
\par 33 един, юнец, един овен, едно едногодишно агне за всеизгаряне;
\par 34 един козел в принос за грях;
\par 35 и за примирителна жертва два вола, пет овена, пет козела и пет едногодишни агнета. Това беше приносът на Елисура, Седиуровият син.
\par 36 На петия ден принесе първенецът на симеонците, Селумиил, Сурисадаевият син.
\par 37 Приносът му беше едно сребърно блюдо тежко сто и тридесет сикли ; един сребърен леген от седемдесет сикли, според сикъла на светилището; и двете пълни с чисто брашно, смесено; с дървено масло, за хлебен принос
\par 38 един златен темянник от десет сикли , пълен с темян;
\par 39 един юнец, един овен, едно едногодишно агне за всеизгаряне;
\par 40 един козел в принос за грях;
\par 41 и за примирителна жертва два вола, пет овена, пет козела и пет едногодишни агнета. Това беше приносът на Селумиила, Сурисадаевият син.
\par 42 На шестия ден принесе първенецът на гадците, Елиасаф, Деуиловият син.
\par 43 Приносът му беше едно сребърно блюдо тежко сто и тридесет сикли ; един сребърен леген от седемдесет сикли, според сикъла на светилището; и двете пълни с чисто брашно, смесено с дървено масло, за хлебен принос;
\par 44 един златен темянник от десет сикли , пълен с темян;
\par 45 един юнец, един овен, едно едногодишно агне за всеизгаряне;
\par 46 един козел в принос за грях;
\par 47 и за примирителна жертва два вола, пет овена, пет козела и пет едногодишни агнета. Това беше приносът на Елиасафа, Деуиловият син.
\par 48 На седмия ден принесе първенецът на ефремците, Елисама, Амиудовият син.
\par 49 Приносът му беше едно сребърно блюдо тежко сто и тридесет сикли ; един сребърен леген от седемдесет сикли; и двете пълни с чисто брашно, смесено с дървено масло, за хлебен принос;
\par 50 един златен темянник от десет сикли , пълен с темян;
\par 51 един юнец, един овен, едно едногодишно агне за всеизгаряне;
\par 52 един козел в принос за грях;
\par 53 и за примирителна жертва два вола, пет овена, пет козела и пет едногодишни агнета. Това беше приносът на Елисама, Амиудовият син.
\par 54 На осмия ден принесе първенецът на манасийците, Гамалиил, Федасуровият син.
\par 55 Приносът му беше едно сребърно блюдо тежко сто и тридесет сикли ; един сребърен леген от седемдесет сикли, според сикъла на светилището; и двете пълни с чисто брашно, смесено с дървено масло, за хлебен принос;
\par 56 един златен темянник от десет сикли , пълен с темян;
\par 57 един юнец, един овен, едно едногодишно агне за всеизгаряне;
\par 58 един козел в принос за грях;
\par 59 и за примирителна жертва два вола, пет овена, пет козела и пет едногодишни агнета. Това беше приносът на Гамалиила, Федасуровият син.
\par 60 На деветия ден принесе първенецът на вениаминците, Авидан, Гедеониевият син.
\par 61 Приносът му беше едно сребърно блюдо тежко сто и тридесет сикли ; един сребърен леген от седемдесет сикли, според сикъла на светилището; и двете пълни с чисто брашно, смесено с дървено масло за хлебен принос;
\par 62 един златен темянник от десет сикли , пълен с темян;
\par 63 един юнец, един овен, едно едногодишно агне за всеизгаряне;
\par 64 един козел в принос за грях;
\par 65 и за примирителна жертва два вола, пет овена, пет козела и пет едногодишни агнета. Това беше приносът на Авидана, Гедеониевият син.
\par 66 На десетия ден принесе първенецът на данците, Ахиезер, Амисадаевият син.
\par 67 Приносът му беше едно сребърно блюдо тежко сто и тридесет сикли ; един сребърен леген от седемдесет сикли, според сикъла на светилището; и двете пълни с чисто брашно, смесено с дървено масло, за хлебен принос;
\par 68 един златен темянник от десет сикли , пълен с темян;
\par 69 един юнец, един овен, едно едногодишно агне за всеизгаряне;
\par 70 един козел в принос за грях;
\par 71 и за примирителна жертва две вола, пет овена, пет козела и пет едногодишни агнета. Това беше приносът на Ахиезера, Амисадаевият син.
\par 72 На единадесетия ден принесе първенецът на асирците, Фагеил, Охрановият син.
\par 73 Приносът му беше едно сребърно блюдо тежко сто и тридесет сикли ; един сребърен леген от седемдесет сикли, според сикъла на светилището; и двете пълни с чисто брашно, смесено с дървено масло, за хлебен принос;
\par 74 един златен темянник от десет сикли , пълен с темян;
\par 75 един юнец, един овен, едно едногодишно агне за всеизгаряне;
\par 76 един козел в принос за грях;
\par 77 и за примирителна жертва два вола, пет овена, пет козела и пет едногодишни агнета. Това беше приносът на Фагеил, Охрановият син.
\par 78 На дванадесетия ден принесе първенецът на нефталимците, Ахирей, Енановият син.
\par 79 Приносът му беше едно сребърно блюдо тежко сто и тридесет сикли : един сребърен леген от седемдесет сикли, според сикъла на светилището; и двете пълни с чисто брашно, смесено с дървено масло, за хлебен принос;
\par 80 един златен темянник от десет сикли , пълен с темян;
\par 81 един юнец, един овен, едно едногодишно агне за всеизгаряне;
\par 82 един козел в принос за грях;
\par 83 и за примирителна жертва два вола, пет овена, пет козела и пет едногодишни агнета. Това беше приносът на Ахирея, Енановият син.
\par 84 Тия бяха приносите от Израилевите първенци за освещаването на олтара в деня, когато биде помазан: дванадесет сребърни блюда, дванадесет сребърни легена и дванадесет златни темянника;
\par 85 всяко сребърно блюдо беше от сто и тридесет сикли ; всичкото сребро на съдовете беше две хиляди и четиристотин сикли ; всичкото сребро на съдовете беше две хиляди и четиристотин сикли , според сикъла на светилището;
\par 86 дванадесет златни темянника пълни с темян; (всеки темянник беше от десет сикли , според сикъла на светилището; всичкото злато на темянниците беше сто и двадесет сикли );
\par 87 всичкият добитък за всеизгаряне беше дванадесет юнеца, дванадесет овена, дванадесет едногодишни агнета, заедно с хлебния им принос, и дванадесет козела в принос за грях;
\par 88 и всичкият добитък за примирителна жертва беше двадесет и четири юнеца, шестдесет овена, шестдесет козела и шестдесет едногодишни агнета. Така стана освещаването на олтара, след като биде помазан.
\par 89 И когато влезе Моисей в шатъра за срещане, за да говори с Бога , тогава чу гласа; който му говореше отгоре на умилостивилището, което беше върху ковчега за плочите на свидетелството между двата херувима; и говореше му.

\chapter{8}

\par 1 И Господ говори на Моисея, казвайки:
\par 2 Говори на Аарона, казвайки му: Когато палиш светилата, седемте светила да светят на предната страна на светилника.
\par 3 И Аарон направи така; запали светилата на светилника така щото да светят на предната му страна, според както Господ заповяда на Моисея.
\par 4 И ето каква беше направата на светилника: изкован от злато, от стъблото до цветята си беше изкован, според образеца, който Господ показа на Моисея, така направи той светилника.
\par 5 Господ говори още на Моисея, казвайки:
\par 6 Вземи левитите измежду израилтяните, и ги очисти.
\par 7 И така да им направиш за очистването им; поръси ги с очистителна вода, и нека обръснат цялото си тяло, и изперат дрехите си, и се очистят.
\par 8 После да вземат един юнец заедно с хлебния му принос от чисто брашно смесено с дървено масло; а ти да вземеш друг юнец в принос за грях.
\par 9 И да приведеш левитите пред шатъра за срещане, и да събереш цялото общество израилтяни;
\par 10 и когато приведеш левитите пред Господа, нека израилтяните положат ръцете си на лавитите;
\par 11 и Аарон да принесе левитите пред Господа, като принос от страна на израилтяните, за да вършат те Господната служба.
\par 12 И като положат левитите ръцете си на главите на юнците, ти да принесеш единия в принос за грях, а другия за всеизгаряне Господу, за да направиш умилостивение за левитите.
\par 13 И да поставиш левитите пред Аарона и пред синовете му, и да ги принесеш като принос Господу.
\par 14 Така да отделиш левитите измежду израилтяните; и левитите ще бъдат Мои.
\par 15 А след това левитите да влязат за да слугуват в шатъра за срещане, когато си ги очистил и си ги принесъл като принос.
\par 16 Понеже те Ми са всецяло дадени измежду израилтяните; вместо всичките първородни от израилтяните, всички, които отварят утроба, съм ги взел за Себе Си.
\par 17 Защото всичките първородни измежду израилтяните са Мои, и човек и животно; в деня, когато поразих всичките първородни в Египетската земя, осветих ги за Себе Си.
\par 18 А левитите взех вместо всичките първородни измежду израилтяните.
\par 19 Левитите нарочно дадох на Аарона и на синовете му измежду израилтяните, за да вършат служението на израилтяните в шатъра за срещане, и да правят умилостивение за израилтяните, за да се не появи язва между израилтяните, когато израилтяните се приближават при светилището.
\par 20 Тогава Моисей и Аарон и цялото общество израилтяни постъпиха с левитите напълно, както Господ заповяда на Моисея за левитите; така им сториха израилтяните.
\par 21 И тъй, левитите се очистиха от греховете си, и изпраха дрехите си; и Аарон ги принесе като принос пред Господа, и Аарон направи за тях умилостивение, за да ги очисти.
\par 22 И след това левитите влязоха в шатъра за срещане, за да вършат службата си пред Аарона и пред синовете му; според както Господ заповяда на Моисея за левитите, така им сториха.
\par 23 Господ говори още на Моисея, казвайки:
\par 24 Ето определеното за левитите: от двадесет и пет години и нагоре да възлизат в отреда, за да вършат слугуването в шатъра за срещане;
\par 25 а от петдесет години да престават да вършат слугуване и да не слугуват вече,
\par 26 но да помагат на братята си в шатъра за срещане, да пазят заръчаното; а слугуване да не вършат. Така да постъпваш с левитете, колкото за дадените им заръчвания.

\chapter{9}

\par 1 И в първия месец на втората година, откак излязоха из Египетската земя, Господ говори още на Моисея в Сенайската пустиня, казвайки:
\par 2 Нека направят израилтяните пасхата на определеното й време.
\par 3 На четиринадесетия ден от тоя месец привечер да я направите, на определеното й време; според всичките закони за нея и според всичките наредби за нея да я направите.
\par 4 И тъй, Моисей каза на израилтяните да направят пасхата.
\par 5 Направиха пасхата на четиринадесетия ден от първия месец привечер, в Синайската пустиня; напълно според както Господ заповяда на Моисея, така направиха израилтяните.
\par 6 А имаше някои, които бяха нечисти, поради мъртво човешко тяло, та не можаха да направят пасхата в оня ден; и в същия ден те дойдоха при Моисея и пред Аарона,
\par 7 и тия мъже рекоха: Ние сме нечисти, поради мъртво човешко тяло, защо да ни спират да не принесем между израилтяните Господния принос на времето му?
\par 8 А Моисей им рече: Постойте, за да чуя какво ще заповяда Господ за вас.
\par 9 И Господ говори на Моисея, казвайки:
\par 10 Говори на израилтяните, казвайки: Ако някой човек от вас или от потомците ви бъде нечист, поради мъртво тяло, или е далеч на път, нека и той направи пасхата Господу;
\par 11 нека я направят на четиринадесетия ден на втория месец привечер, и нека я ядат с безквасни хлябове и горчиви треви;
\par 12 да не оставят от нея до утринта, нито да трошат кост от нея; да я направят, според всичките повеления за пасхата.
\par 13 А който е чист, и не е на път, ако пренебрегне да направи пасхата, тоя човек ще бъде изтребен измежду людете си; понеже не е принесъл Господния принос на времето му, тоя човек ще носи греха си.
\par 14 И ако някой чужденец, който е пришелец между вас, желае да направи пасхата Господу, нека я направи, според повеленията за пасхата и според наредбата за нея; един закон ще имате и за чужденеца и за туземеца.
\par 15 И в деня, когато се постави скинията, облакът покри скинията, шатъра за плочите на свидетелството; и от вечер до заран над скинията имаше като огнено явление.
\par 16 Така ставаше всякога: облакът я покриваше, и нощем имаше огнено явление.
\par 17 И когато се дигаше облакът от шатъра, тогава, след това, израилтяните тръгваха; и гдето заставаше облакът, там израилтяните разполагаха стан.
\par 18 По Господно повеление тръгваха израилтяните, и по Господно повеление разполагаха стан; до тогаз, до когато облакът стоеше над скинията, те си оставаха в стана.
\par 19 И когато облакът стоеше над скинията много дни, тогава израилтяните пазеха Господното заръчване и не тръгваха;
\par 20 а понякога облакът стоеше над скинията малко дни; но пак по Господно повеление оставаха разположени в стана, и по Господно повеление тръгваха.
\par 21 И понякога облакът стоеше само от вечер до заран; но пак на заранта, когато се дигаше облакът, тогава и те тръгваха; когато се дигаше облакът, било денем или нощем, тогава и те тръгнаха;
\par 22 Ако облакът продължаваше да стои над скинията два дена, или един месец, или една година, то и израилтяните оставаха в стана си и не тръгваха; а когато той се дигаше, те тръгваха.
\par 23 Според Господно повеление разполагаха стан, и според Господно повеление тръгваха; те пазеха заръчаното от Бога, както заповядваше Господ чрез Моисея.

\chapter{10}

\par 1 И Господ говори на Моисея, казвайки:
\par 2 Направи си две сребърни тръби; изковани да ги направиш; и да ти служат за свикване на обществото и за дигане на становете.
\par 3 Когато засвирят с тях, нека се събере цялото общество с тебе до входа на шатъра за срещане.
\par 4 Ако засвирят само с едната тръба , тогава да се събират при тебе първенците, Израилевите хилядници.
\par 5 А когато засвирите тревога, тогава да се дигат становете, които са разположени към изток;
\par 6 и когато засвирите тревога втори път, тогава да се дигат становете, които са разположени към юг. Да свирят тревога, за да се дигат.
\par 7 А когато има да се събере събранието, да свирите, обаче, без да засвирите тревога.
\par 8 И тръбачите да бъдат свещениците, Аароновите синове; това ще ви бъде вечен закон в поколенията ви.
\par 9 И когато излезете на война в земята си против неприятеля, който би ви притеснил, тогава да засвирите тревога; и ще бъдете спомнени пред Господа вашият Бог, и ще бъдете избавени от неприятелите си.
\par 10 И на увеселителните си дни, и на празниците си, и на новолунията си да свирите с тръбите над всеизгарянията си и над примирителните си жертви; и това ще ви бъде за спомен пред вашия Бог. Аз съм Иеова вашият Бог.
\par 11 Във втората година, на двадесетия ден от втория месец, облакът се издигна от скинията за плочите на свидетелството.
\par 12 И израилтяните се дигнаха от синайската пустиня според реда на пътуванието си; и облакът застана във Фаранската пустиня.
\par 13 Дигнаха се първите пет, според както Господ заповяда чрез Моисея.
\par 14 Първо се дигна знамето на стана на юдейците според устроените им множества; и над множеството му беше Наасон Аминадавовият син.
\par 15 Над множеството на племето на исахарците беше Натанаил Суаровият син.
\par 16 А над множеството на племената на завулонците беше Елиав Хелоновият син.
\par 17 Тогава, като се сне скинията, дигнаха се гирсонците и мерарийците, които носеха скинията.
\par 18 После се дигна знамето на Рувимовия стан, според устроените им множества; и над множеството му беше Елисур Седиуровият син.
\par 19 Над множеството на племето на симеонците беше Селумиил Сурисадаевият син.
\par 20 А над множеството на племето на гадците беше Елиасаф Деуиловият син.
\par 21 Тогава се дигнаха Каатовците, които носеха светилището, до пристигането на които скинията се поставяше.
\par 22 После се дигна знамето на стана на ефремците според устроените им множества; и над множеството му беше Елисама Амиудовият син.
\par 23 Над множеството на племето на манасийците беше Гамалиил Федасуровият син.
\par 24 А над множеството на племето на вениаминците беше Авидан Гедеониевият син.
\par 25 После се дигна знамето на стана на данците, последни от всичките станове, според устроените им множества; и над множеството му беше Ахиезер Амисадаевият син.
\par 26 Над множеството на племето на асирците беше Фагеил Охрановият син.
\par 27 А над множеството на племето на нефталимците беше Ахирей Енановият син.
\par 28 Така ставаше пътуването на израилтяните според устроените им множества, когато се дигаха.
\par 29 В това време Моисея каза на Овава, син на мадиамеца Рагуил, Моисеевият тъста: Ние сме на онова място, на което рече Господ: Ще ви го дам. Ела заедно с нас, и ще ти сторим добро; защото Господ е обещал добро на Израиля.
\par 30 Но той му рече: Няма да дойда, но ще отида в своята си земя и при рода си.
\par 31 А Моисей каза: Не ни оставяй, моля, понеже ти знаеш где трябва да разполагаме стан в пустинята, и ще бъдеш око за нас.
\par 32 И ако дойдеш с нас, то доброто, което Господ ще направи на нас, същото добро ще направим и ние на тебе.
\par 33 И тъй, пропътуваха тридневен път от Господната планина; и ковчегът на Господния завет се движеше пред тях тридневен път, за да им търси място за почивка.
\par 34 И Господният облак беше над тях денем, когато тръгваха от стана.
\par 35 И когато ковчегът се дигаше на път, Моисей казваше: Стани Господи, и да се разпръснат враговете Ти, и да побягнат от пред Тебе ония, които Те мразят.
\par 36 А когато се спираше, той казваше: Върни се Господи, при десетките хиляди на Израилевите хиляди.

\chapter{11}

\par 1 И людете зле роптаеха в ушите на Господа; и Господ чу, и гневът Му пламна; и огън от Господа се запали между тях та пояждаше неколцина в края на стана.
\par 2 Тогава людете извикаха към Моисея; и Моисей се помоли Господу, и огънят престана.
\par 3 И нарече се онова място Тавера, защото огън от Господа гореше между тях.
\par 4 И разноплеменното множество, което беше между тях, показа голямо лакомство; също и израилтяните пак плакаха и рекоха: Кой ще ни даде месо да ядем?
\par 5 Ние помним рибата, която ядохме даром в Египет, краставиците, дините, праза и червения и чесновия лук;
\par 6 а сега душата ни е изсъхнала; нищо няма; няма на какво да гледаме освен тая манна.
\par 7 (А манната приличаше на кориандрово семе и беше на вид като бделий.
\par 8 И людете се пръскаха наоколо та я събираха, мелеха я в мелници, или я чукаха в кутли, и варяха я в гърнета, и правеха пити от нея; а вкусът й беше като вкус на пити пържени в масло.
\par 9 Когато падаше росата в стана нощем, падаше с нея и манна).
\par 10 И Моисей чу как людете плачеха в семействата си, всеки при входа на шатъра си; и Господният гняв пламна силно; стана мъчно и на Моисея.
\par 11 Моисей рече на Господа: Защо си оскърбил слугата Си? и защо не съм придобил Твоето благоволение, та си турил върху мене товара на всички тия люде?
\par 12 Аз ли съм зачнал всички тия люде? или аз съм ги родил, та ми казваш: Носи ги в лоното си, както гледач-баща носи бозайничето, до земята, за която Си се клел на бащите им?
\par 13 От где у мене месо да дам на всички тия люде? защото плачат пред мене и казват: Дай ни месо да ядем.
\par 14 Аз сам не мога да нося всички тия люде, защото са много тежки за мене.
\par 15 Ако постъпиш Ти така с мене, то убий ме още сега, моля, ако съм придобил Твоето благоволение, за да не видя злочестината си.
\par 16 Тогава Господ рече на Моисея: Събери ми седемдесет мъже измежду Израилевите старейшини, които познаваш, като старейшини, които познаваш, като старейшини на людете и техни надзиратели, и доведи ги при шатъра за срещане, за да застанат там с тебе.
\par 17 И Аз като сляза ще говоря там с тебе; и ще взема от духа, който е на тебе, и ще го туря на тях; и те ще носят товара на людете заедно с тебе, за да не го носиш ти сам.
\par 18 И кажи на людете: Очистете си за утре, и ще ядете месо; защото плачехте в ушите на Господа и казвахте: Кой ще ни даде месо да ядем? защото добре ни беше в Египет. Затова Господ ще ви даде месо, и ще ядете.
\par 19 Няма да ядете един ден, ни два дена, ни пет дена, ни десет дена, ни двадесет дена,
\par 20 но цял месец, догде ви излезе из ноздрите и ви омръзне; защото отхвърлихте Господа, Който е между вас, и плакахте пред Него, думайки: Защо излязохме из Египет?
\par 21 И Моисей рече: Людете, всред които съм, са шестстотин хиляди пешаци; и Ти рече: Ще им дам да ядат месо цял месец.
\par 22 Да се изколят ли за тях овците и говедата, за да им бъдат достатъчни? или да им се съберат всичките морски риби, за да им бъдат достатъчни?
\par 23 А Господ каза на Моисея: Скъсила ли се е Господната ръка? Сега ще видиш, ще се обърне ли с тебе думата Ми, или не.
\par 24 И тъй, Моисей излезе та каза на людете Господните думи; и събра седемдесет мъже от старейшините на людете, и постави ги около шатъра.
\par 25 Тогава Господ слезе в облака и говори с него, и като взе от духа, който беше на него, тури го на седемдесетте старейшини; и като застана на тях духът, пророкуваха, но не повториха.
\par 26 Обаче двама от мъжете бяха останали в стана, името на единия от тях беше Елдад, а името на другия Модад, та и на тях застана духът; те бяха от записаните, но не бяха отишли до шатъра; и пророкуваха в стана.
\par 27 И завтече се едно момче та извести на Моисея, казвайки; Елдад и Модад пророкуват в стана.
\par 28 И Исус Навиевият син, слугата на Моисея, един от неговите избрани, проговори и рече: Господарю мой, Моисее, запрети им.
\par 29 А Моисей му рече: Завиждаш ли за мене? Дано всичките Господни люде бъдат пророци, та да тури Господ Духа Си на тях!
\par 30 И Моисей отиде в стана, той и Израилевите старейшини.
\par 31 Тогава излезе вятър от Господа и докара от морето пъдпъдъци, и остави ги да слитат долу над стана, до един ден път от едната страна, и до един ден път от другата страна, около стана; а те летяха до два лакътя над повърхността на земята.
\par 32 Тогава людете станаха та събираха пъдпъдъците целия онзи ден и цялата нощ и целия следен ден; оня, който събра на-малко, събра десет кора; и те си ги простираха около стана.
\par 33 А като беше месото още в зъбите им, и не беше още сдъвкано, Господният гняв пламна против людете, и Господ порази людете с много голяма язва.
\par 34 И нарекоха това място Киврот-атаава, защото там бяха погребани лакомите люде.
\par 35 А от Киврот-атаава людете се дигнаха за Асирот, и останаха в Асирот.

\chapter{12}

\par 1 В това време Мариам и Аарон говориха против Моисея поради етиопянката, която бе взел за жена, (защото беше взел една етиопянка); и рекоха:
\par 2 Само чрез Моисея ли говори Господ? не говори ли и чрез нас? И Господ чу това.
\par 3 А Моисей беше човек много кротък, повече от всичките човеци, които бяха на земята.
\par 4 И веднага Господ рече на Моисея, на Аарона и на Мариам: Излезте вие трима към шатъра за срещане. И тъй, излязоха и тримата.
\par 5 Тогава Господ слезе в облачен стълб, застана пред входа на шатъра и повика Аарона и Мариам, и те двамата излязоха.
\par 6 И рече: Слушайте сега думите Ми. Ако има пророк между вас, Аз Господ ще му стана познат чрез видение, на сън ще му говоря.
\par 7 Но слугата Ми Моисей не е така поставен , той, който е верен в целия Ми дом;
\par 8 с него Аз ще говоря уста с уста, ясно, а не загадъчно; и той ще гледа Господния Образ. Как, прочее, не се убояхте вие да говорите против слугата Ми Моисея?
\par 9 И гневът на Господа пламна против тях, и Той си отиде.
\par 10 И като се оттегли облакът от шатъра, ето, Мариам беше прокажена, бяла като сняг; като погледна Аарон на Мариам, ето, тя беше прокажена.
\par 11 Тогава Аарон рече на Моисея: Моля ти се, господарю мой, не ни възлагай тоя грях, с който сторихме безумие и съгрешихме.
\par 12 Да не бъде тя като мъртво дете , на което половина от тялото е изтляло, когато излиза из утробата на майка си.
\par 13 И Моисей викна към Господа, казвайки: О Боже, моля Ти се, изцели я.
\par 14 А Господ каза на Моисея: Ако би я заплюл баща й в лицето, не щеше ли да бъде посрамена седем дена? Нека бъде затворена вън от стана седем дена, и след това да се прибере.
\par 15 И тъй, Мариам бе затворена седем дена вън от стана, и людете не се дигнаха, докато не се прибра Мариам.
\par 16 Подир това людете се дигнаха от Асирот, и разположиха стан в Фаранската пустиня.

\chapter{13}

\par 1 И Господ говори на Моисея, казвайки:
\par 2 Изпрати мъже, за да съгледат Ханаанската земя, която Аз давам на израилтяните; да изпратите по един мъж от всяко племе на бащите им, и всички да са от първенците между тях.
\par 3 И тъй, според Божието повеление Моисей ги изпрати от Фаранската пустиня; всичките мъже бяха главни между израилтяните.
\par 4 Ето имената им: от Рувимовото племе, Самуй Закхуровият син;
\par 5 от Симеоновото племе, Сафат Хориевият син;
\par 6 от Юдовото племе, Халев Ефониевият син;
\par 7 от Исахаровото племе, Игал Иосифовият син;
\par 8 от Ефремовото племе, Осия Навиевият син;
\par 9 от Вениаминовото племе, Фалтий Рефуевият син;
\par 10 от Завулоновото племе, Гадиил Содиевият син;
\par 11 от Иосифовото племе, от Манасиевото племе, Гадий Сусиевият син;
\par 12 от Дановото племе, Амиил Гамалиевият син;
\par 13 от Асировото племе, Сетур Михаиловият син;
\par 14 от Нефталимовото племе, Наавий Вопсиевият син;
\par 15 от Гадовото племе, Геуил Махиевият син;
\par 16 Тия са имената на мъжете, които Моисей изпрати, за да съгледат земята; и Моисей наименува Осия Навиевия син Исус.
\par 17 Като ги изпрати да съгледат Ханаанската земя, Моисей им рече: Качете се по южната страна и изкачете се на бърдата,
\par 18 та вижте каква е земята, и людете, които живеят в нея, силни ли са или слаби, малко ли са или много;
\par 19 и каква е земята, на която те живеят, добра ли е или лоша; и какви са градовете, в които те живеят, от шатри ли са или са укрепени;
\par 20 и каква е земята, плодовита ли е или постна, има ли по нея дървета или не. И бъдете смели, и донесете от плодовете на земята. А тогава беше времето на първозрялото грозде.
\par 21 И така, те се качиха и съгледаха земята от Цинската пустиня до Роов при прохода на Емат.
\par 22 После се изкачиха на южната страна и дойдоха до Хеврон, гдето живееха Енаковите синове Ахиман, Сесай и Талмай. (А Хеврон беше построен седем години преди Египетския Зоан).
\par 23 И като дойдоха до долината Есхол, от там отрязаха една лозова пръчка с един грозд, който двама носеха на върлина; взеха и нарове и смокини.
\par 24 Онова място се нарече долина Есхол по причина на грозда, който израилтяните отрязаха от там.
\par 25 А на края на четиридесетте дена те се върнаха от съгледването на земята.
\par 26 И отивайки дойдоха при Моисея, при Аарона и при цялото общество израилтяни в Фаранската пустиня, в Кадис; и дадоха отчет на тях и на цялото общество, и показаха им плода на земята.
\par 27 И разказаха му, думайки: Ходихме в земята, в която ни изпрати; и наистина там текат масло и мед; ето плода й.
\par 28 Людете обаче, които живеят в земята, са силни, и градовете укрепени и много големи; там видяхме още и Енаковите синове.
\par 29 Амаличаните живеят в земята към юг; хетейците, евусейците и аморейците живеят по планините; а ханаанците живеят при морето и край бреговете на Иордан.
\par 30 Но Халев успокояваше людете пред Моисея, като казваше: Да вървим напред незабавно и да я завладеем, защото можем да я превземем.
\par 31 А мъжете, които бяха дошли заедно с него, рекоха: Не можем да излезем против ония люде, защото са по-силни от нас.
\par 32 И зле представяха пред израилтяните земята, която бяха съгледали, казвайки: Земята, която обходихме, за да я съгледаме, е земя, която изпояжда жителите си; и всичките люде, които видяхме в нея, са превисоки мъже.
\par 33 Там видяхме исполините, Енаковите синове, от исполинския род; и пред тях нам се виждаше, че сме като скакалци; такива се виждахме и на тях.

\chapter{14}

\par 1 Тогава цялото общество извика с висок глас, и людете плакаха през оная нощ.
\par 2 И всичките израилтяни роптаеха против Моисея и Аарона; и цялото общество им рече: Да бяхме измрели в Египетската земя! или в тая пустиня да бяхме измрели!
\par 3 И защо ни води Господ в тая земя да паднем от нож, и жените ни и децата ни да бъдат разграбени? Не щеше ли да ни е по-добре да се върнем в Египет?
\par 4 И рекоха си един на друг: Да си поставим началник и да се върнем в Египет.
\par 5 Тогава Моисей и Аарон паднаха на лицата си пред цялото множество на обществото израилтяни.
\par 6 И Исус Навиевият син и Халев Ефрониевият син, от ония, които съгледаха земята, раздраха дрехите си,
\par 7 и говориха на цялото общество израилтяни, казвайки: Земята, през която минахме, за да я съгледаме, е много добра земя.
\par 8 Ако бъде благоволението на Господа към нас, тогава Той ще ни въведе в тая земя и ще ни я даде, - земя гдето текат мляко и мед.
\par 9 Само недейте въстава против Господа, нито се бойте от людете на земята, защото те са ястие за нас; защитата им се оттегли от върху тях; а Господ е с нас; не бойте се от тях.
\par 10 Но цялото общество рече да ги убият с камъни. Тогава Господната слава се яви в шатъра за срещане пред всичките израилтяни.
\par 11 И Господ рече на Моисея: До кога ще Ме презират тия люде? и до кога няма да Ме вярват, въпреки всичките знамения, които съм извършил пред тях?
\par 12 Ще ги поразя с мор и ще ги изтребя; а тебе ще направя народ по-голям и по-силен от тях.
\par 13 Но Моисей рече на Господа: Тогава египтяните ще чуят; защото Ти със силата Си си извел тия люде изсред тях;
\par 14 и ще кажат това на жителите на тая земя, които са чули, че Ти, Господи, си между тия люде, - че Ти, Господи, се явяваш лице с лице, и че облакът Ти стои над тях, и че Ти вървиш пред тях денем в облачен стълб, а нощем в огнен стълб.
\par 15 И ако изтребиш тия люде, като един човек, тогава народите, които са чули за Тебе, ще говорят казвайки:
\par 16 Понеже не можа Иеова да въведе тия люде в земята, за която им се кле, за това ги погуби в пустинята.
\par 17 И сега, моля Ти се, нека се възвеличи силата на Господа мой, както си говорил, казвайки;
\par 18 Господ е дълготърпелив и многомилостив, прощава беззаконие и престъпление, и никак не обезвинява виновния , и въздава беззаконието на бащите върху чадата до третия и четвъртия род.
\par 19 Прости, моля Ти се, беззаконието на тия люде, и според както си прощавал на тия люде от Египет до тука.
\par 20 И рече Господ: Прощавам им , според както си казал;
\par 21 но наистина заклевам се в живота Си, че ще се изпълни целия свят с Господната слава.
\par 22 Понеже от всички тия мъже, които са виждали славата Ми и знаменията, които извърших в Египет и в пустинята, и са Ме раздразвали до сега десет пъти, и не послушаха гласа Ми,
\par 23 наистина ни един от тях няма да види земята, за която се клех на бащите им, нито ще я види един от ония, които Ме презряха.
\par 24 Но понеже слугата ми Халев има в себе си друг дух, и той напълно Ме последва, за това него ще въведа в земята, в която влезе, и потомството му ще я наследва.
\par 25 (А амаличаните и ханаанците живеят в долината). Утре се върнете та идете в пустинята към Червеното море.
\par 26 Господ говори още на Моисея и Аарона, казвайки:
\par 27 До кога ще търпя това нечестиво общество, което роптае против Мене? Чух роптанията на израилтяните, с които роптаят против Мене.
\par 28 Кажи им: Заклевам се в живота Си, казва Господ, невярно ще направя на вас така, както вие говорихте в ушите Ми;
\par 29 труповете ви ще паднат в тая пустиня; и от преброените между вас, колкото сте на брой от двадесет години и нагоре, които сте роптали против Мене,
\par 30 ни един не ще влезе в земята, за която се клех да ви заселя в нея, освен Халев Ефониевият син и Исуса Навиевият син.
\par 31 Но децата ви, за които рекохте, че ще бъдат разграбени, тях ще въведа; и те ще познаят земята, която вие презряхте.
\par 32 А вашите трупове ще паднат в тая пустиня.
\par 33 И децата ви ще скитат по пустинята четиридесет години, и ще теглят поради вашите блудствувания, докато се изпоядат труповете ви в пустинята.
\par 34 Според числото на дните, през които съгледахте земята, четиридесет дена, всеки ден за една година, четиридесет години ще теглите поради беззаконията си, и ще познаете какво значи Аз да съм неблагоразположен.
\par 35 Аз Господ говорих; наистина така ще направя на цялото това нечестива общество, което се е събрало против Мене; в тая пустиня ще се довършат, и в нея ще измрат.
\par 36 И ония мъже, които Моисей изпрати да съгледат земята, които, като се върнаха, направиха цялото общество да роптае против него, и зле представиха земята,
\par 37 тия мъже, които зле представиха земята, измряха от язвата пред Господа.
\par 38 А Исус Навиевият син и Халев Ефониевият син останаха живи измежду ония мъже, които ходиха да съгледат земята.
\par 39 Тогава Моисей каза тия думи на всичките израилтяни; и людете плакаха горчиво.
\par 40 И сутринта, като станаха рано, изкачиха се на планинския връх и казаха: Ето ни; и ще вървим напред на мястото, което Господ ни е обещал; защото съгрешихме.
\par 41 А Моисей рече: Защо престъпвате така Господното повеление, тъй като това няма да успее?
\par 42 Не вървете напред, защото Господ не е между вас, да не би да ви поразят неприятелите ви.
\par 43 Защото амаличаните и ханаанците са там пред вас, и вие ще паднете от нож. Понеже отстъпихте и не следвахте Господа, затова Господ няма да бъде с вас.
\par 44 Обаче те дръзнаха да се изкачат на планинския връх; но ковчегът на Господния завет и Моисей не излязоха изсред стана.
\par 45 Тогава амаличаните и ханаанците, които живееха на оная планина, слязоха та ги разбиха, и поразяваха ги дори до Хорма.

\chapter{15}

\par 1 И Господ говори на Моисея, казвайки:
\par 2 Говори на израилтяните, казвайки им: Когато влезете в земята, която Аз ви давам да живеете в нея,
\par 3 и пренесете жертва чрез огън Господу, било всеизгаряне, или жертва за изпълнение на обрек, или доброволен принос, или на празниците си, за да направите благоухание Господу от говедата или от овците,
\par 4 тогава оня, който принася приноса си Господу, нека принесе за хлебен принос една десета от ефа чисто брашно, смесено с четвърт ин дървено масло.
\par 5 И за всяко агне на всеизгарянето или на жертвата да притуриш четвърт от ин вино за възлияние,
\par 6 или за всеки овен да притуриш за хлебен принос две десети от ефа чисто брашно, смесено с една трета от ин дървено масло;
\par 7 и за възлияние да принесеш една трета от ин вино за благоухание Господу.
\par 8 И ако принесеш от говедата за всеизгаряне, или жертва за изпълнение на обрек, или примирителен принос Господу,
\par 9 тогава с жертвата от говедата да принесеш за хлебен принос три десети от ефа чисто брашно, смесено с половин ин дървено масло;
\par 10 и за възлияние да принесеш половин ин вино в жертва чрез огън за благоухание Господу.
\par 11 Така трябва да се прави за всяко говедо, или за всеки овен, или за всяко агне или яре.
\par 12 Според числото, което ще принесете, така да направите за всяко според числото им.
\par 13 Всеки туземец да прави така, когато принесе жертва чрез огън за благоухание Господу.
\par 14 Ако някой чужденец е пришелец между вас, или ако какъвто и да е бил е между вас във всичките ви поколения, и принася жертва чрез огън за благоухание Господу, то както правите вие, така да направи и той.
\par 15 Един закон да има за вас, които сте от обществото, и за чужденеца, който е пришелец между вас , един вечен закон във всичките ви поколения; както сте вие така ще бъде и чужденецът пред Господа.
\par 16 Един закон и една наредба да има за вас и за чужденеца, който е пришелец между вас.
\par 17 Господ говори още на Моисея, казвайки:
\par 18 Говори на израилтяните, казвайки им: Когато влезете в земята, в която Аз ви въвеждам,
\par 19 и ядете от хляба на земята, тогава да принесете Господу възвишаем принос;
\par 20 от първото си тесто да принесете пита за възвишаем принос; да го възвишите, както правите с възвишаемия принос от гумно.
\par 21 От първото си тесто да давате Господу възвишаем принос във всичките си поколения.
\par 22 И ако прегрешите и не изпълните всичките тия заповеди, които Господ каза на Моисея,
\par 23 то ест, всичко, което Господ ви заповяда чрез Моисея, от деня когато Господ даде заповед и нататък във всичките ви поколения,
\par 24 то, ако е сторено от незнание, без да знае обществото - цялото общество нека принесе един юнец за всеизгаряне за благоухание Господу, заедно с хлебния му принос и с възлиянието му, според наредбата, и един козел в принос за грях.
\par 25 И свещеникът да направи умилостивение за цялото общество израилтяни, и ще им се прости; защото е станало от незнание, и те са принесли приноса си в жертва чрез огън Господу, и приноса си за грях пред Господа за несъзнателната си погрешка;
\par 26 и ще се прости на цялото общество израилтяни и на чужденеца, който е пришелец между тях, защото колкото за всичките люде стореното е станало от незнание.
\par 27 Но ако един човек съгреши от незнание, той трябва да принесе едногодишна коза в принос за грях.
\par 28 И свещеникът да направи умилостивение пред Господа за човека, който е съгрешил от незнание; когато съгреши от незнание, да направи умилостивение за него, и ще му се прости.
\par 29 Един закон да има за вас, както за туземеца от израилтяните, така и за чужденеца, който е пришелец между тях, когато съгреши от незнание.
\par 30 Но ако някой съгреши с надменна ръка, бил той туземец или чужденец, той показва презрение към Господа; тоя човек ще бъде изтребен измежду людете си.
\par 31 Понеже е презрял словото на Господа и престъпил заповедта Му, тоя човек непременно ще се изтреби, беззаконието му ще бъде върху него.
\par 32 Когато израилтяните бяха в пустинята, намериха един човек, който събираше дърва в съботен ден.
\par 33 И ония, които го намериха като събираше дърва, доведоха го при Моисея и Аарона и при цялото общество.
\par 34 И туриха го под стража, понеже не беше още изявено що трябваше да сторят с него.
\par 35 И Господ каза на Моисея: Човекът непременно трябва да се умъртви; цялото общество да го убие с камъни вън от стана.
\par 36 И тъй, цялото общество го изведе вън от стана и го уби с камъни, та умря, според както Господ заповяда на Моисея.
\par 37 Тогава Господ говори на Моисея, казвайки:
\par 38 Говори на израилтяните и заповядай им да правят, във всичките си поколения, ресни по краищата на дрехите си, и да турят син ширит по ресните на всичките краища.
\par 39 И това да ви бъде за ресни, та като ги гледате, да помните всичките Господни заповеди и да ги изпълнявате, и да не дирите неща по своите си сърца и по своите си очи, подир, които неща вие блудствувате;
\par 40 та да помните и изпълнявате всичките Ми заповеди, и да бъдете свети на вашия Бог.
\par 41 Аз съм Господ вашият Бог, Който ви изведох из Египетската земя, за да ви бъда Бог. Аз съм Иеова вашият Бог.

\chapter{16}

\par 1 А Корей син на Исаара, син на Каата Левиевият син, и Датан и Авирон синове на Елиава, и Он син на Фалета, Рувимуви потомци, като си взеха човеци,
\par 2 дигнаха се против Моисея, с двеста и петдесет човеци от израилтяните, първенци на обществото, избрани за съветници, именити мъже.
\par 3 Събраха се, прочее, против Моисея и против Аарона и рекоха им: Стига ви толкоз! Цялото общество е свето, всеки един от тях, и Господ е всред тях. А защо възвишавате себе си над Господното общество?
\par 4 А Моисей, като чу това, падна на лицето си,
\par 5 и говори на Корея и на цялата му дружина, казвайки: Утре Господ ще покаже кои са Негови, и кой е свет, и кого ще направи да се приближи при Него. Онзи, който Той избере ще направи да се приближи при Него.
\par 6 Това направете: ти, Корее, и цялата ти дружина, вземете си кадилници,
\par 7 турете в тях огън, и турете в тях темян пред Господа утре; и когато Господ избере, той ще бъде свет. Стига толкоз и вам, левийци!
\par 8 И Моисей рече на Корея; Чуйте сега, вий левийци:
\par 9 малко ли ви е това, гдето Израилевият Бог отдели вас от Израилевото общество, та ви направи да се приближавате при Него, за да вършите службата на Господната скиния и да стоите пред обществото, за да им служите?
\par 10 Той те направи да се приближиш при Него, и заедно с тебе всичките ти братя левийците; а искате ли и свещенството?
\par 11 Така, че ти и цялата ти дружина сте се събрали против Господа; защото кой е Аарон, та да роптаете против него?
\par 12 И Моисей изпрати да повикат Датана и Авирона, Елиавовите синове; а те отговориха: Няма да дойдем.
\par 13 Малко ли е това, гдето си ни извел из земя, в която текат мляко и мед, за да ни измориш в пустинята, та още и владетел ли искаш да направиш себе си над нас?
\par 14 При това, ти не си ни довел в земя, гдето текат мляко и мед, нито си ни дал да наследим ниви и лозя. Ще извърташ ли очите на тия хора? Няма да дойдем.
\par 15 Тогава Моисей се разсърди много, и рече Господу: Не погледвай благосклонно на приноса им; аз не съм взел нито един осел от тях, и никому от тях не съм сторил зло.
\par 16 И Моисей рече на Корея: Утре ти и всичките, които си събрал, да се намерите пред Господа, - ти, и те, и Аарон;
\par 17 и вземете всеки кадилницата си, турете в тях темян, и занесете пред Господа всеки кадилницата си, двеста и петдесет кадилници; също и ти и Аарон, - всеки своята кадилница.
\par 18 Прочее, те взеха всеки кадилницата си, туриха огън в тях, туриха в тях и темян, и застанаха пред входа на шатъра за срещане заедно с Моисея и Аарона.
\par 19 Корей събра против тях и цялото общество пред входа на шатъра за срещане; и Господната слава се яви на цялото общество.
\par 20 Тогава Господ говори на Моисея и на Аарона, казвайки:
\par 21 Отделете се отсред това общество, за да ги изтребя в един миг.
\par 22 А те паднаха на лицата си и рекоха: О Боже, Боже на духовете на всяка твар! ако един човек е съгрешил, ще се разгневиш ли на цялото общество?
\par 23 Тогава Господ говори на Моисея, казвайки:
\par 24 Говори на обществото, казвайки: Отстъпете от жилищата на Корея, Датана и Авирона.
\par 25 И тъй, Моисей стана та отиде при Датана и Авирона; подир него отидоха и Израилевите старейшини.
\par 26 И говори на обществото, казвайки: Отстъпете, моля ви се от шатрите на тия нечестиви човеци, и не се допирайте до нищо тяхно, за да не погинете всред всичките техни грехове.
\par 27 И тъй, те навред отстъпиха от жилищата на Корея, Датана и Авирона; а Датан и Авирон излязоха та застанаха при входовете на шатрите си с жените си и малките си деца.
\par 28 И рече Моисей: От това ще познаете, че Господ ме е изпратил да извърша всички тия дела, и че не ги правя от себе си;
\par 29 ако тия човеци умрат, както умират всичките човеци, или ако им се въздаде, според както се въздава на всичките човеци, то Господ не ме е изпратил;
\par 30 но ако Господ направи ново нещо, - ако отвори земята устата си та погълне тях и всичко, което е тяхно, и те слязат живи в ада, тогава ще познаете, че тия човеци презряха Господа.
\par 31 Като изговори той всички тия думи, земята се разпукна под тях.
\par 32 Земята отвори устата си и погълна тях, домочадията им, всичките Корееви човеци и всичкия им имот.
\par 33 Те и всичко тяхно слязоха живи в ада, земята ги покри, и те погинаха отсред обществото.
\par 34 А целият Израил, който бяха около тях, погинаха, като извикаха, думайки: Да не погълне земята и нас.
\par 35 И огън излезе от Господа и пояде ония двеста и петдесет мъже, които принесоха темян.
\par 36 След това Господ говори на Моисея, казвайки:
\par 37 Кажи на Елеазара, син на свещеника Аарона, да прибере кадилниците отсред изгарянето; а ти разпръсни огъня нататък; защото свети са
\par 38 кадилниците на тия човеци, които съгрешиха против своя си живот, и нека ги направят на плочи за обковаване на олтара; понеже те ги принесоха пред Господа, и за това са свети; и те ще бъдат за знамение на израилтяните.
\par 39 И тъй, свещеникът Елеазар прибра медните кадилници, които изгорелите бяха принесли; и направиха ги на плочи за обковаване на олтара,
\par 40 да напомнюват на израилтяните, че никоя чужд човек, който не е от Аароновото потомство, не бива да пристъпва да принася темян пред Господа, за да не остане като Корея и дружината му. Това стори Елеазар , както Господ му рече чрез Моисея.
\par 41 А на следния ден цялото общество израилтяни възроптаха против Моисея и Аарона, като казваха: Вие избихте Господните люде.
\par 42 Но когато обществото се беше събрало против Моисея и Аарона погледнаха към шатъра за срещане, и ето, облакът го покри, и Господната слава се яви.
\par 43 И Моисей и Аарон дойдоха пред шатъра за срещане.
\par 44 И Господ говори на Моисея, казвайки:
\par 45 Оттеглете се отсред това общество, за да ги погубя в един миг. Но те паднаха на лицата си.
\par 46 Тогава Моисей рече на Аарона: Вземи кадилницата си, тури в нея огън от олтара, и тури на него темян та иди скоро в обществото и направи умилостивение за тях; защото гняв излезе от Господа, язвата почна.
\par 47 Аарон, прочее, взе кадилницата си , както рече Моисей, и завтече се сред обществото; и, ето, язвата беше почнала между людете; и той тури темяна и направи умилостивение за людете.
\par 48 А като застана между мъртвите и живите, язвата престана.
\par 49 Умрелите от язвата бяха четиринадесет хиляди и седемстотин човеци, освен ония, които умряха в Кореевата работа.
\par 50 И Аарон се върна при Моисея до входа на шатъра за срещане, защото язвата престана.

\chapter{17}

\par 1 Тогава Господ говори на Моисея, казвайки:
\par 2 Говори на израилтяните, и вземи от тях дванадесет жезли, по един жезъл за всеки дом, от всичките им първенци, според домовете на бащите им, и напиши името на всекиго на жезъла му.
\par 3 На Левиевия жезъл напиши името на Аарона; понеже ще има по един жезъл за всеки началник на бащините им домове.
\par 4 Положи ги в шатъра за срещане пред плочите на свидетелството, гдето Аз се срещам с вас.
\par 5 И жезълът на човека, когото избера, ще процъфти. Така ще направя да престанат пред Мене роптанията на израилтяните, с които те роптаят против вас.
\par 6 Моисей, прочее, каза на израилтяните; и всичките им първенци му дадоха дванадесет жезъла, всеки първенец по един жезъл, според бащините си домове; и Аароновият жезъл беше между жезлите им.
\par 7 И Моисей положи жезлите пред Господа в шатъра за плочите на свидетелството.
\par 8 И на следния ден Моисей влезе в шатъра за плочите на свидетелството; и ето, Аароновият жезъл за Левиевия дом беше покарал и произрастил пъпки, цъфнал, и завързал зрели бадеми.
\par 9 И Моисей изнесе всичките жезли от пред Господа при всичките израилтяни; и те, като ги прегледаха, взеха всеки жезъла си.
\par 10 Тогава Господ каза на Моисея: Върни Аароновия жезъл пред плочите на свидетелството, за да се пази за знак против бунтовническия род; и така да направиш да престанат роптанията им против Мене, та да не измрат.
\par 11 И Моисей направи така: според както Господ му заповяда така направи.
\par 12 Тогава израилтяните говориха на Моисея, казвайки: Ето, ние загиваме, погубени сме, всички сме погубени;
\par 13 всеки, който се приближава, който се приближи до Господната скиния, умира; ще измрем ли ние всички?

\chapter{18}

\par 1 И Господ каза на Аарона: Ти, синовете ти и домът на баща ти с тебе ще носите виновността за светилището; и ти и синовете ти с тебе ще носите виновността за свещенството си.
\par 2 А приближи при себе си и братята си, Левиевото племе, племето на баща ти, за да са свързани с тебе и да ти слугуват; а ти и синовете ти с тебе да бъдете при шатъра за плочите на свидетелството.
\par 3 И левитите ще пазят заръчаното от тебе, и заръчаното за целия шатър; само до принадлежностите на светилището и до олтара да се не приближават, за да не умрат, те и вие.
\par 4 Нека бъдат свързани с тебе, и да пазят заръчаното за шатъра за срещане във всяка служба на шатъра; и чужд човек да се не приближи при вас.
\par 5 Така да пазите заръчаното за светилището и заръчаното за олтара, щото да не падне вече гняв върху израилтяните.
\par 6 И, ето, Аз взех братята ви левитите отсред израилтяните; те ви са всецяло дар, дадени Господу, да вършат службата на шатъра за срещане.
\par 7 А ти и синовете ти с тебе ограничавайте свещенодействието си във всичко, което се отнася до олтара и което е извътре завесата, и около тях да служите. Вам подарявам службата на свещенството; а чуждият човек, който би се приближил, да се умъртви.
\par 8 Господ рече на Аарона: Ето, Аз дадох на тебе надзора на Моите възвишаеми приноси, сиреч, на всичките неща посвещавани от израилтяните; на тебе и на синовете ти ги дадох като ваше вечно право поради това, че сте били помазани.
\par 9 От пресветите неща, това, което не се туря на огъня, ще бъде твое; всичките им приноси, всичките им хлебни приноси, всичките им приноси за грях, и всичките им приноси за престъпление, които те дават на Мене, ще бъдат пресвети за тебе и за синовете ти.
\par 10 Но пресвето място да ги ядете; всеки от мъжки пол да яде от тях; свети да ти бъдат.
\par 11 И ето що е твое: възвишаемият принос от дара им, с всичките движими приноси на израилтяните; давам ги на тебе, на синовете ти и на дъщерите ти с тебе, като ваше вечно право. Който е чист у дома ти да ги яде.
\par 12 Всичко що е най-добро от дървеното масло, и всичко що е най-добро от виното и от житото, първите им плодове, които те дават Господу, на тебе го давам.
\par 13 Първите плодове от всичките произведения на земята им, които те ще донесат Господу, ще бъдат твои; който е чист у дома ти да ги яде.
\par 14 Всяко обречено нещо в Израиля ще бъде твое.
\par 15 Всяко първородно от всякакъв вид, което принасят Господу, било човек или животно, ще бъде твое; но за първородното от човека непременно да вземеш откуп, и за първородното от нечисти животни да вземаш откуп.
\par 16 Като станат на един месец подлежащите на откупуване да вземаш откуп за тях по твоята оценка, пет сребърни сикли, според сикъла на светилището, който е двадесет гери.
\par 17 А за първородните от говедата, или за първородните от овците, и за първородните от козите да не вземаш откуп; те са свети; с тлъстината им да изгаряш, като жертва чрез огън, за благоухание Господу.
\par 18 А месото им да бъде твое, както са твои движимите гърди и дясното бедро.
\par 19 Всичките възвишаеми приноси от светите неща, които израилтяните принасят Господу, давам на тебе, на синовете ти и на дъщерите ти с тебе, като ваше вечно право. Това е вечен завет със сол пред Господа за тебе и за потомството ти с тебе.
\par 20 Господ рече още на Аарона: Ти да нямаш наследство в тяхната земя, нито да имаш дял между тях; Аз съм твоят дял и твоето наследство между израилтяните.
\par 21 А на левийците, ето, Аз давам в наследство всичките десетъци в Израиля, заради службата, която вършат, службата в шатъра за срещане.
\par 22 Израилтяните да не пристъпят вече при шатъра за срещане, да не би да се натоварят с грях и да измрат.
\par 23 Но левитите да вършат службата в шатъра за срещане, и те да носят виновността си; вечен завет ще бъде във всичките ви поколения да нямате наследство между израилтяните.
\par 24 Защото десетъците, които израилтяните принасят за възвишаем принос Господу, давам в наследство на левитите; затова рекох за тях: Те да нямат наследство между израилтяните.
\par 25 И Господ говори на Моисея, казвайки:
\par 26 Говори на левитите, казвайки им: Когато вземате от израилтяните десетъка, който ви дадох от тях за ваше наследство, тогава да принасяте от него десетък от десетъка за възвишаем принос Господу.
\par 27 И тия ваши възвишаеми приноси ще ви се считат като жито от гумното, и като изобилие на вино от лина.
\par 28 Така и вие да принасяте възвишаем принос Господу от всичките десетъци, които вземате от израилтяните; и от тях да давате на свещеника Аарона възвишаем принос Господу.
\par 29 От всичките си дарове да принасяте всеки възвишаем принос Господу, то ест, осветената част от всичко що е най-добро от тях.
\par 30 За това, да им речеш: Когато принасяте на-добрата част от тях, останалото ще се счита за левитите, като доход от гумното и като доход от лина.
\par 31 Можете да го ядете на всяко място, вие и домочадията ви; защото това ви е заплата за служението ви в шатъра за срещане.
\par 32 Няма да понасяте грях поради това, ако принасяте във възвишаем принос най-добрата част от тях; и да не осквернявате светите неща на израилтяните, за да не умрете.

\chapter{19}

\par 1 И Господ говори на Моисея и Аарона, казвайки:
\par 2 Ето повелението на закона, който Господ заповяда, като каза: Говори на израилтяните да ти доведат червеникава юница без недостатък, която няма повреда и на която не е турян ярем;
\par 3 и да я дадете на свещеника Елеазара, и той да я изведе вън от стана, та да я заколят пред него.
\par 4 Тогава свещеникът Елеазар, като вземе от кръвта й с пръста си, да поръси седем пъти от кръвта й към предната част на шатъра за срещане.
\par 5 И да изгорят юницата пред него: кожата й, месото й и кръвта й с изверженията й да изгорят.
\par 6 После свещеникът да вземе кедрово дърво, исоп и червена прежда , и да ги хвърли всред горящата юница.
\par 7 Тогава свещеникът да изпере дрехите си, да окъпе тялото си във вода, и подир това да влезе в стана; и свещеникът да бъде нечист до вечерта.
\par 8 Така и оня, който я е изгорил, нека изпере дрехите си във вода, и да окъпе тялото си във вода, и да бъде нечист до вечерта.
\par 9 Тогава един чист човек да събере пепелта от юницата и да тури вън от стана на чисто място; и пепелта да се пази за обществото израилтяни, за да се направи с нея вода за очищение от грях.
\par 10 И оня, който събере пепелта от юницата, да изпере дрехите си, и да бъде нечист до вечерта; и това ще бъда вечен закон за израилтяните и за пришелците, които живеят между тях.
\par 11 Който се допре до някое мъртво, човешко тяло, да бъде нечисто седем дена.
\par 12 С тая вода тоя да се очисти на третия ден, и на седмия ден ще бъде чист; но ако не се очисти на третия ден, то и на седмия ден не ще бъде чист.
\par 13 Който се допре до мъртвото тяло на умрял човек, и не се очисти, той осквернява Господната скиния; тоя човек ще се изтреби измежду Израиля; той ще бъде нечист, понеже не е поръсен с очистителната вода; нечистотата му е още на него.
\par 14 Ето и законът, когато някой умре в шатър: всеки, който влиза в шатъра и всички, които се намира в шатъра, да бъдат нечисти седем дена;
\par 15 И всеки непокрит съд , който е без привързана покривка, е нечист.
\par 16 И който се допре на полето до някой убит с нож, или до мъртво тяло, или до човешка кост, или до гроб, да бъде нечист седем дена.
\par 17 А за очистване на нечистия нека вземат в съда от пепелта на юницата изгорена в жертва за грях, и да полеят на нея текуща вода.
\par 18 Тогава чист човек да вземе исоп, и, като го натопи във водата, да поръси шатъра, всичките вещи и човеците, които се намират там и онзи, който се е допрял до кост, или до убит човек, или до умрял, или до гроб.
\par 19 И чистият да поръси нечистия на третия ден и на седмия ден; и на седмия ден да го очисти. Тогава нека изпере дрехите си и нека се окъпе във вода и вечерта ща бъде чист.
\par 20 А оня, който, като е нечист, не се очисти, оня човек ще се изтреби измежду обществото, понеже е осквернил Господното светилище; той не е поръсен с очистителната вода; нечист е.
\par 21 И това да им бъде вечен закон, че тоя, който е поръсил с очистителната вода, да изпере дрехите си; и че който се допре до очистителната вода да бъде нечист до вечерта;
\par 22 и че всичко, до което се допре нечистият да бъде нечисто; и че тоя, който се допре до това нещо , да бъде нечист до вечерта.

\chapter{20}

\par 1 И в първия месец дойдоха израилтяните, цялото общество, в Цииската пустиня; и людете останаха в Кадис. Там умря Мариам, и там бе погребана.
\par 2 А вода нямаше за обществото, тъй че те се събраха против Моисея и против Аарона.
\par 3 Людете се скараха с Моисея, като говореха казвайки: О да бяхме измрели и ние, когато братята ни измряха пред Господа!
\par 4 Защо изведохте Господното общество в тая пустиня да измрем в нея, ние и добитъкът ни?
\par 5 И защо ни изведохте из Египет, за да ни доведете на това лоша място, което не е място за сеене, ни за смокини, ни за лозя, ни за нарове, нито има вода за пиене?
\par 6 Тогава Моисей и Аарон отидоха от присъствието на обществото при входа на шатъра за срещане, гдето и паднаха на лицата си; и Господната слава им се яви.
\par 7 И Господ говори на Моисея, казвайки:
\par 8 Вземи жезъла и свикай обществото, ти и брат ти Аарон, и пред очите из говорете на канарата, и тя ще даде водата си; така ще им извадите вода из канарата, и ще напоиш обществото и добитъка им.
\par 9 И тъй, Моисей взе жезъла, който беше пред Господа, според както Той му заповяда;
\par 10 и, като свика Моисей и Аарон обществото пред канарата, той им каза: Чуйте сега, вий бунтовници! да ви извадим ли вода из тая канара?
\par 11 Тогава Моисей дигна ръката си и жезъла си и удари канарата два пъти; и потече много вода, та обществото и добитъкът им пиха.
\par 12 Но Господ каза на Моисея и Аарона: Понеже не Ме вярвахте за да Ме осветите пред израилтяните, за това вие няма да въведете това общество в земята, която им давам.
\par 13 Това е водата на Мерива, защото израилтяните се препираха с Господа, и Той се освети всред тях.
\par 14 След това Моисей изпрати посланици от Кадис до едомския цар да му кажат : Ти знаеш всичките трудности, които не сполетяха,
\par 15 как бащите ни слязоха в Египет, и живееха дълго време в Египет, и как египтяните се отнасяха зле към нас и бащите ни,
\par 16 и как , когато ние извиквахме към Господа, Той чу гласа ни, и изпрати един ангел та ни изведе из Египет; и ето ни в Кадис, град в края на твоите предели.
\par 17 Нека минем, моля, през земята ти. Няма да минем през нивите или през лозята, нито ще пием вода от кладенците; но ще вървим през целия друм; няма да се отбием ни надясно ни наляво, докато не преминем твоите предели.
\par 18 Но Едом му отговори: Няма да минеш през земята ми, да не би да изляза с нож против тебе.
\par 19 А израилтяните пак му рекоха: Ние ще минем през друма; и ако аз и добитъкът ми пием от водата ти, ще я платим; остави ме само с нозете си да премина и нищо друго.
\par 20 А той пак отговори: Няма да преминеш. И Едом излезе против него с много люде и със силна ръка.
\par 21 Така Едом отказа да пусне Израиля да мине през пределите му; за това Израил се отвърна от него.
\par 22 И така, израилтяните, цялото общество, отпътуваха от Кадис, и дойдоха при планината Ор.
\par 23 Тогава Господ говори на Моисея и Аарона на планината Ор, при границите на Едомската земя, казвайки:
\par 24 Аарон ще се прибере при людете си, защото няма да влезе в земята, която съм дал на израилтяните, понеже не се покорихте на думата Му при водата на Мерива.
\par 25 Вземи Аарона и сина му Елеазара и изведи ги на планината Ор;
\par 26 и съблечи от Аарона одеждите му, и облечи с тях сина му Елеазара; и Аарон ще се прибере при людете си и ще умре там.
\par 27 И Моисей стори, според както Господ заповяда ; те се качиха на планината Ор пред очите на цалото общество.
\par 28 И Моисей съблече от Аарона одеждите му и облече с тях сина му Елеазара; и Аарон умря там на върха на планината; а Моисей и Елеазар слязоха от планината.
\par 29 И като видя цялото общество, че Аарон умря, то целият Израилев дом оплакваха Аарона за тридесет дена.

\chapter{21}

\par 1 А арадският цар, ханаанецът, който живееше към юг, като чу, че Израил иде през пътя Атарим воюва против Израиля, и хвана от тях пленници.
\par 2 И Израил направи обрек Господу, като каза: Ако наистина предадеш тия люде в ръката ми, то съвсем ще разоря градовете им.
\par 3 И Господ послуша гласа на Израиля и му предаде ханаанците; и те ги погубиха и разориха градовете им. И мястото се нарече Хорма.
\par 4 А когато отпътуваха от планината Ор, по пътя към Червеното море, за да обиколят Едомската земя, людете излязоха от търпение в душата си поради пътя.
\par 5 И людете говориха против Бога и против Моисея, казвайки: Защо ни изведохте из Египет да измрем в пустинята? защото няма ни хляб ни вода, и душата ни се отвращава от тоя никакъв хляб.
\par 6 За това Господ изпрати между людете горителни змии, които хапеха людете, та измряха много люде от Израиля.
\par 7 Тогава людете дойдоха при Моисея и казаха: Съгрешихме за гдето говорихме против Господа и против тебе; помоли се Господу да махне змиите от нас. И Моисей се помоли за людете.
\par 8 И Господ рече на Моисея: Направи си една горителна змия, и тури я на висока върлина; и всеки ухапан, като погледне на нея, ще остане жив.
\par 9 И тъй, Моисей направи медна змия и я тури на най-високата върлина; и когато змия ухапеше някого, той, като погледнеше на медната змия, оставаше жив.
\par 10 Тогава израилтяните отпътуваха и разположиха стан в Овот.
\par 11 И като отпътуваха от Овот, разположиха стан в Е-Аварим, в пустинята, която е към Моав, към изгрева на слънцето.
\par 12 От там отпътуваха и разположиха стан в долината Заред.
\par 13 От там отпътуваха и разположиха стан оттатък реката Арнон, която е в пустинята и изтича от пределите на аморейците; защото Арнон е моавска граница между моавците и аморейците.
\par 14 За това е казано в книгата на Господните войни: - Ваев в Суфа И потоците на Арнон,
\par 15 И течението на потоците, Което се простира до селището Ар И допира границата на Моав.
\par 16 А от там дойдоха при Вир. Тоя е кладенецът, за който Господ рече на Моисея: Събери людете, и ще им дам вода.
\par 17 Тогава Израил изпя тая песен: - Бликай, о кладенче; пейте за него;
\par 18 Кладенец изкопаха първенците, Благородните от людете изкопаха, Чрез заповедта на законодателя; с жезлите си. А от пустинята отидоха в Матана,
\par 19 и от Матана в Наалиил, и от Наалиил във Вамот,
\par 20 а от Вамот в долината, която е в моавското поле, при върха на Фасга, която гледа към Есимон.
\par 21 Тогава Израил изпрати посланици при аморейския цар Сион да кажат:
\par 22 Остави ме да замина през земята ти; няма да свръщаме ни по нивите ни по лозята; не ще да пием вода от кладенците; през царевия друм ще вървим, докато преминем твоите предели.
\par 23 А Сион не пусна Израиля да мина през пределите му; но Сион събра всичките си люде, излезе та се опълчи срещу Израиля в пустинята, и дойде в Яса та воюва против Израиля.
\par 24 Но Израил го порази с острото на ножа, и завладя земята му от Арнон до Явок, до амонците; защото границата на амонците беше крепка.
\par 25 Израил завладя всички тия градове; и Израил се засели във всичките градове на аморейците, в Есевон и във всичките му села.
\par 26 Понеже Есевон бе град на аморейския цар Сион, който беше воювал с предишния моавски цар и беше отнел от ръката му всичката му земя до Арнон.
\par 27 За това, ония, които говорят с притчи, казват: - Дойдете в Есевон; Да се съгради и да се закрепи града Сионов;
\par 28 Защото огън излезе от Есевон, Пламък из града Сионов; Пояде Ар моавски, И първенците на високите места на Арнон.
\par 29 Горко ти Моаве! Погина ти, Хамосови люде! Даде синовете си на бяг И дъщерите си на плен При Сиона аморейския цар.
\par 30 Ние го застреляхме Есевон погина до Девон; И запустихме ги до Нофа, Която се простира до Медева.
\par 31 Така Израил се засели в земята на Аморейците.
\par 32 После Моисей изпрати човеци да съгледат Язир; и, като превзеха селата му, изпъдиха аморейците, които бяха в тях.
\par 33 Тогава се върнаха и отидоха по пътя към Васан; а васанският цар Ог излезе против тях, той и всичките му люде, на бой в Едраи.
\par 34 Но Господ рече на Моисея: Не бой се от Него, защото Аз ще предам в ръцете ви него, всичките му люде и земята му; ще му направиш така, както направи на аморейския цар Сион, който живееше в Есевон.
\par 35 И тъй, поразиха него, синовете му и всичките му люде, догдето не му остана ни един оцелял; и завладяха земята му.

\chapter{22}

\par 1 Подир това израилтяните отпътуваха и разположиха стан на моавските полета оттатък Иордан, срещу Ерихон.
\par 2 А Валак Сепфоровият син видя всичко що стори Израил на аморейците.
\par 3 И Моав се уплаши много от людете, защото бяха многочислени; и Моав се обезпокояваше поради израилтяните.
\par 4 И Моав рече на мадиамските старейшини: Сега това множество ще пояде всичко около нас, както говедо пояжда полската трева. И Валак Сепфоровият син, който в това време беше цар на моавците,
\par 5 изпрати посланици до Валаама Веоровия син във Фатур, който е при реката Евфрат , в земята на ония, които бяха людете му, за да го повикат като му кажат: Ето, народ излезе из Египет; ето, покриват лицето на земята, и са разположени срещу мене;
\par 6 Ела сега, прочее, моля ти се, прокълни ми тия люде, защото са по-силни от мене, негли бих могъл да преодолея, та да ги поразим, и да мога да ги изпъдя из земята; понеже зная, че оня, когото ти благословиш, е благословен, а когото прокълнеш е проклет.
\par 7 И тъй, моавските старейшини и мадиамските старейшини отидоха, с възнаграждение в ръце за врачуването; и, като дойдоха при Валаама, казаха му Валаковите думи.
\par 8 А той им рече: Пренощувайте тука, и ще ви дам отговор, според каквото ми каже Господ. И така моавските първенци останаха у Валаама.
\par 9 И Бог дойде при Валаама и рече: Какви са тия човеци у тебе?
\par 10 И Валаам рече на Бога: Валак Сепфоровият син, цар на моавците, ги е пратил до мене да кажат :
\par 11 Ето, людете, които излязоха из Египет, покриват лицето на земята; дойди сега, прокълни ми ги, негли бих могъл да се бия с тях и да ги изпъдя.
\par 12 А Бог рече на Валаама: Да не отидеш с тях, нито да прокълнеш людете, защото са благословени.
\par 13 И тъй, Валаам, като стана сутринта, каза на Валаковите първенци: Идете в земята си, защото Господ отказа да ме пусне да дойда с вас.
\par 14 Тогава моавските първенци станаха та дойдоха при Валака и рекоха: Валаам отказа да дойде с нас.
\par 15 А Валак пак изпрати първенци, по-много и по-почтени от ония.
\par 16 И те, като дойдоха при Валаама, му казаха: Така казва Валак Сепфоровият син: Моля ти се, нищо да те не спре да не дойдеш до мене;
\par 17 защото ще те въздигна до голяма почит, и ще сторя все що би ми рекъл; прочее, дойди, моля, прокълни ми тия люде.
\par 18 А Валаам отговори на Валаковите слуги, казвайки: Ако би ми дал Валак и къщата си пълна със сребро и злато, аз не мога да престъпя думата на Господа моя Бог, да направя по-малко или повече.
\par 19 За това, моля, пренощувайте и вие тука, за да се науча какво още ще ми каже Господ.
\par 20 И Бог дойде при Валаама през нощта, та му рече: Щом са дошли човеците да те повикат, стани иди с тях; но само онова, което ти река, него да направиш.
\par 21 За това, Валаам стана на сутринта, оседла ослицата си, и отиде с моавските първенци.
\par 22 Но Божият гняв пламна за гдето отиде; и ангел Господен застана на пътя пред Валаама , за да му се възпротиви; (а той яздеше на ослицата си, и двамата му слуги бяха с него).
\par 23 И понеже ослицата видя, че ангелът Господен стоеше на пътя с гол нож в ръка, ослицата се отби от пътя и отиваше към полето; а Валаам удари ослицата, за да я оправи в пътя.
\par 24 Тогава ангелът Господен застана на един нисък път между лозята, дето имаше преграда отсам и преграда оттам край пътя .
\par 25 И понеже ослицата видя ангела Господен, облегна се към зида и притисна Валаамовата нога до зида; и той я удари пак.
\par 26 После ангелът Господен отиде още напред и застана на едно тясно място, гдето нямаше къде да си отбие ни надясно ни наляво.
\par 27 И понеже ослицата видя ангела Господен, тя падна под Валаама; а Валаам се разлюти и удари ослицата с тоягата си.
\par 28 Тогава Господ отвори устата на ослицата, и тя рече на Валаама: Що съм ти сторила та ме биеш вече три пъти?
\par 29 А Валаам рече на ослицата: Защо се подигна с мене. Ах, да имах нож в ръката си! сега бих те заклал.
\par 30 И ослицата рече на Валаама: Не съм ли аз твоята ослица, на която си яздил през целия си живот до днес? Имала ли съм навик друг път да ти правя така? А той рече: Не.
\par 31 Тогава Господ отвори очите на Валаама, и той видя ангела Господен стоящ на пътя с гол нож в ръката си; и преклони глава и падна на лицето си.
\par 32 И ангелът Господен му каза: Ти защо би ослицата си вече три пъти? Ето, аз излязох да ти се съпротивя, защото пътят ти не е прав пред мене;
\par 33 и ослицата ме видя и се отби от мене, ето, три пъти; ако да не бе се отбила от мене, до сега да съм те убил, а нея да съм оставил жива.
\par 34 Тогава Валаам рече на ангела Господен: Съгреших, защото не знаех, че ти стоеше на пътя против мене; и сега, ако това не ти е угодно, аз ще се върна.
\par 35 А ангелът Господен рече на Валаама: Иди с човеците; но само словото, което ти кажа, него да говориш. И тъй, Валаам отиде с Валаковите първенци.
\par 36 А като чу Валак, че иде Валаам, излезе да го посрещне до един моавски град разположен там , гдето Арнон е границата, в най-далечната част на границата.
\par 37 Тогава Валак рече на Валаама: Не пратих ли до тебе усърдно да те повикат? Защо не дойде при мене? Не мога ли да те въздигна до почит?
\par 38 А Валаам рече на Валака: Ето, дойдох при тебе; но имам ли сега власт да говоря нещо? Каквато дума тури Бог в устата ми, нея ще говоря.
\par 39 И Валаам отиде с Валака, и дойдоха в Кириатузот.
\par 40 И Валак жертвува говеда и овци, и изпрати от тях и на Валаама и на първенците, които бяха с него,
\par 41 А на сутринта Валак взе Валаама и го заведе на високите Ваалови места от гдето, той видя людете до крайната им част.

\chapter{23}

\par 1 Тогава Валаам каза на Валака: Издигни тук седем жертвеника, и приготви ми тук седем юнеца и седем овена.
\par 2 И Валак стори, както рече Валаам; и Валак и Валаам принесоха по юнец и овен на всеки жертвеник.
\par 3 После Валаам рече на Валака: Застани близо при всеизгарянето си, и аз ще отида; може би да дойде Господ да ме посрещне; а каквото ми яви ще ти кажа. И отиде на гола височина.
\par 4 И Бог срещна Валаама; а Валаам му рече: Приготвих седемте жертвеника, и принесох по юнец и овен на всеки жертвеник.
\par 5 И Господ тури дума в устата на Валаама, и рече: Върни се при Валака и според тоя дума говори.
\par 6 Върни се, прочее, при него; и, ето, той стоеше при всеизгарянето си, той и всичките моавски първенци.
\par 7 Тогава почна беседата си, казвайки: - Валак ме доведе от Арам, Моавският цар от източните планини, и каза ми : Дойди, прокълни ми Якова, И дойди, хвърли презрение върху Израиля.
\par 8 Как да прокълна, когато Бог не проклина? Или как да хвърля презрение върху когото Господ не хвърля?
\par 9 Защото от връх канарите го виждам, И от хълмовете го гледам; Ето люде, които ще се заселят отделно, И няма да се считат между народите.
\par 10 Кой може да преброи пясъка Яковов, Или да изчисли четвъртата част от Израиля? Дано умра както умират праведните, И сетнините ми да бъдат както техните!
\par 11 Тогава Валак рече на Валаама: Що ми направи ти? Взех те, за да прокълнеш неприятелите ми; а, ето, ти напълно си ги благословил!
\par 12 А той в отговор каза: Не трябва ли да внимавам да говоря онова, което Господ туря в устата ми?
\par 13 Тогава Валак му рече: Дойди, моля, с мене на друго място, от гдето ще ги видиш; само крайната им част ще видиш а всички тях няма да видиш; и прокълни ми ги от там.
\par 14 И тъй, доведе го на полето Зофим, на върха на Фасга, и издигна седем жертвеника, и принесе по юнец и овен на всеки жертвеник.
\par 15 Тогава Валаам рече на Валака: Застани тука при всеизгарянето си, а аз ще срещна Господа там.
\par 16 И Господ срещна Валаама, тури дума в устата му, и рече: Върни се при Валака и според тая дума говори.
\par 17 Той, прочее, дойде при него; и, ето, той стоеше при всеизгарянето си, и моавските първенци с него. И Валак му рече: Що говори Господ?
\par 18 Тогава той почна притчата си, казвайки: - Стани, Валаче, та слушай; Дай ми ухо, ти сине Сепфоров!
\par 19 Бог не е човек та да лъже, Нито човешки син та да се разкае; Той каза, и няма ли да извърши? Той говори и няма ли да го тури в действие?
\par 20 Ето, аз получих заповед да благославям; И Той като благослови, аз не мога да го отменя.
\par 21 Не гледа беззаконие в Якова, Нито вижда извратеността в Израиля; Господ Бог негов с него е, И царско възклицание има между тях.
\par 22 Бог ги изведе из Египет: Има сила като див вол.
\par 23 Наистина няма чародейство против Якова. И няма врачуване против Израиля; На времето си ще се говори за Якова И за Израиля: Що е извършил Бог!
\par 24 Ето, людете ще въстанат като лъвица, И ще се дигнат като лъв; Няма да легнат, докато не изядат лова И не изпият кръвта на убитите.
\par 25 Тогава Валак рече на Валаама: Никак да не ги проклинаш и никак да не ги благославяш.
\par 26 А Валаам в отговор каза на Валака: Не говорих ли ти казвайки: Всичко що ми каже Господ, това ще сторя?
\par 27 След това Валак пак рече на Валаама: Дойди, моля, ще те заведа още на друго място, негли бъде угодно Богу да ми ги прокълнеш от там.
\par 28 И Валак заведе Валаама на върха на Фегор, който гледа към Иесимон.
\par 29 Тогава Валаам рече на Валака: Издигни ми тук седем жертвеника, и приготви ми тук седем юнеца и седем овена.
\par 30 И Валак направи както рече Валаам, и принесе по юнец и овен на всеки жертвеник.

\chapter{24}

\par 1 И Валаам, като видя, че беше угодно на Господа да благославя Израиля, не отиде както друг път да търси гадания, но обърна лицето си към пустинята.
\par 2 И като подигна очи, Валаам видя Израиля заселен според племената си; и Божият Дух слезе на него.
\par 3 И почна притчата си, казвайки: - Валаам син Веоров каза: И човекът, който има отворени очи, каза:
\par 4 Каза оня, който чу Божиите думи, Който видя видението от Всесилния, Който падна в изстъпление , но очите му бяха отворени:
\par 5 Колко са красиви твоите шатри, Якове, Твоите скинии, Израилю!
\par 6 Като долини са разпрострени, Като градини по речни брегове, Като алоини дървета, които Господ е насадил, Като кедри покрай водите.
\par 7 Ще се излива вода из ведрата му, И потомството му ще се простира в много води; Царят му ще бъде по-висок от Агага, И царството му ще се възвеличи.
\par 8 Бог го изведе из Египет; Има сила както див вол; Ще пояде неприятелските нему народи, Ще строши костите им, и ще ги удари със стрелите си.
\par 9 Легнал е и лежи като лъв И като лъвица; кой ще го възбуди? Благословен, който те благославя! И проклет, който те проклина!
\par 10 Тогава гневът на Валака пламна против Валаама, и изплеска с ръце: и Валак рече на Валаама: Аз те повиках да прокълнеш неприятелите ми; а, ето, три пъти ти все ги благославяш.
\par 11 Сега, прочее, бягай на мястото си, рекох си да те въздигна до голяма почит; но, ето, Господ те въздържа от почит.
\par 12 А Валаам рече на Валака: Не говорих ли аз и на твоите посланици, които ми прати, като казах:
\par 13 Ако би ми дал Валак и къщата си пълна със сребро и злато, не мога да престъпя Господното повеление и да направя добро или зло от себе си, но онова, което Господ проговори, него ще кажа?
\par 14 И сега, ето, аз си отивам при своите люде; ела да ти кажа какво ще направят тия люде на твоите люде в последните дни.
\par 15 И като почна притчата си, рече: - Валаам син Веоров каза, И човекът с отворени очи каза:
\par 16 Каза оня, който чу думите Божии, Който има знание за Всевишния, Който падна в изтъпление , но очите му бяха отворени:
\par 17 Виждам го, но не сега; Гледам го, но не отблизо; Ще излезе звезда от Якова, И ще се въздигне скиптър от Израиля; Ще порази моавските първенци, И ще погуби всичките Ситови потомци.
\par 18 Едом ще бъде притежаван, Още Сиир ще бъде притежаван от неприятелите си; А Израил ще действува мощно.
\par 19 Един произлязъл от Якова ще завладее И ще погуби останалите от града.
\par 20 А като видя Амалика, продължи притчата си и каза: - Амалик е пръв между народите; Но най-сетне съвсем ще се изтреби.
\par 21 А като видя кенейците продължи притчата си и каза: - Яко е твоето жилище, И положил си гнездото си на канарата,
\par 22 Но кенейците ще се разорят Догде те заплени Асур.
\par 23 Още продължи притчата си и каза: - Уви! кой ще остане жив, Когато Бог извърши това?
\par 24 Но кораби ще дойдат от крайморията на Китим, И ще смирят Асура, и ще смирят Евера; Но и дошлите съвсем ще се изтребят.
\par 25 Тогава, като стана Валаам, тръгна и се върна на мястото си; също и Валак отиде по пътя си.

\chapter{25}

\par 1 А докато Израил оставаше в Ситим, людете почнаха да блудствуват с моавките;
\par 2 защото те канеха людете на жертвите на боговете си, и людете ядяха и се кланяха на боговете им.
\par 3 Израил се привърза за Ваалфегора; и Господният гняв пламна против Израиля.
\par 4 Тогава Господ рече на Моисея: Хвани всичките първенци на людете и обеси ги за Господа пред слънцето, за да се отвърне от Израиля Господният яростен гняв.
\par 5 И Моисей рече на Израилевите съдии: Убийте всеки подчинените си човеци, които се привързаха за Ваалфегора.
\par 6 И, ето, един от израилтяните дойде и доведе на братята си една мадиамка пред очите на Моисея и пред цялото общество израилтяни, когато те плачеха пред входа на шатъра за срещане.
\par 7 А Финеес, син на Елеазара син на свещеника Аарона, като видя стана изсред обществото, взе копие в ръката си,
\par 8 и влезе подир израилтянина в спалнята та прободе и двамата - израилтянина и жената в корема й. Така язвата престана от израилтяните.
\par 9 А умрелите от язвата бяха двадесет и четири хиляди души.
\par 10 Тогава Господ говори на Моисея, казвайки:
\par 11 Финеес, син на Елеазара, син на свещеника Аарона, отвърна яростта Ми от израилтяните; понеже показа ревността всред тях подобна на Моята, така щото Аз не изтребих израилтяните в ревността си.
\par 12 За това кажи му: Ето, Аз му давам Моя завет на мир;
\par 13 ще бъде нему и на потомството му подир него завет на вечно свещенство, защото беше ревностен за своя Бог и направи умилостивение за израилтяните.
\par 14 А името на убитият израилтянин, който беше убит с мадиамката, беше Зимрий син на Салу, първенец на един бащин дом от симеонците.
\par 15 И името на убитата мадиамка беше Хазвия, дъщеря на Сура, началник на людете, от един бащин дом в Мадиам.
\par 16 След това Господ говори на Моисея, казвайки:
\par 17 Измъчвайте мадиамците и поразете ги;
\par 18 защото те ви измъчват с коварствата, с които ви примамиха чрез Фегора и чрез сестра си Хазвия, дъщеря на един мадиамски първенец, която беше убита в деня на язвата наложена поради Фегора.

\chapter{26}

\par 1 А подир язвата Господ говори на Моисея и Елеазара син на свещеника Аарона, казвайки:
\par 2 Пребройте цялото общество израилтяни от двадесет години и нагоре, според бащините им домове, всички в Израиля, които могат да излизат на бой.
\par 3 За това, Моисей и свещеникът Елеазар говориха на людете на моавските полета, при Иордан срещу Ерихон, казвайки:
\par 4 Пребройте людете от двадесет години и нагоре, според както Господ заповяда на Моисея и на израилтяните, които излязоха из Египетската земя.
\par 5 Рувим, първородният на Израиля. Рувимците бяха: от Еноха, семейството на Еноховците; от Фалу, семейството на Фалуевците;
\par 6 от Есрона, семейството на Есроновците; от Хармия, семейството на Хармиевците.
\par 7 Тия са семействата на рувимците; а преброените от тях бяха четиридесет и три хиляди седемстотин и тридесет души.
\par 8 Синовете на Фалу бяха Елиав;
\par 9 и синовете на Елиава: Намуил, Датан и Авирон. Това са ония Датан и Авирон, избраните от обществото, които се подигнаха против Моисея и против Аарона между дружината на Корея, когато се подигнаха против Господа;
\par 10 и земята разтвори устата си та ги погълна заедно с Корея при изтреблението на дружината му , когато огънят пояде двеста и петдесет човеци, и те станаха за знамение;
\par 11 а синовете на Корея не умряха.
\par 12 Симеонците по семействата си бяха: от Намуила, семейството на Намуиловците; от Ямина, семейството на Яминовците; от Яхина, семейството на Яхиновците;
\par 13 от Зара, семейството на Заровците; от Саула, семейството на Сауловците.
\par 14 Тия са семействата на Симеоновците, двадесет и две хиляди и двеста души.
\par 15 Гадците по семействата си бяха: от Сафона, семейството на Сафоновците; от Агия, семейството на Агиевците; от Суния, семейството на Суниевците;
\par 16 от Азения, семейството на Азениевците; от Ирия, семейството на Ириевците;
\par 17 от Арода, семейството на Ародовците; от Арилия, семейството на Арилиевците;
\par 18 Тия са семействата на гадците; и преброените от тях бяха четиридесет хиляди и петстотин души.
\par 19 Юдовите синове бяха Ир и Онан; но Ир и Онан умряха в Ханаанската земя.
\par 20 А юдейците по семействата си бяха; от Села, семейството на Селаевците; от Фареса, семейството на Фаресовците; от Зара, семейството на Заровците.
\par 21 И Фаресовите потомци бяха: от Есрона, семейството на Амуловците.
\par 22 Тия са Юдовите семейства; и преброените от тях бяха седемдесет и шест хиляди и петстотин души.
\par 23 Исахарците по семействата си бяха: от Тола, семейството на Толовците; от Фуа, семейството на Фуавците;
\par 24 от Ясува, семейството на Ясувовците; от Симрона, семейството на Симроновците.
\par 25 Тия са Исахаровите семейства; и преброените от тях бяха шестдесет и четири хиляди и триста души.
\par 26 Завулонците по семействата си бяха: от Середа, семейството на Середовците; от Елона, семейството на Елоновците; от Ялеила, семейството на Ялеиловците.
\par 27 Тия са семействата на завулонците; и преброените от тях бяха шестдесет хиляди и петстотин души.
\par 28 Иосифовите синове по семействата си бяха Манасия и Ефрем.
\par 29 Манасийците бяха: от Махира, семейството на Махировците. И Махир роди Галаада; а от Галаада, семейството на Галаадовците.
\par 30 Ето Галаадовите потомци: от Ахиезера, семейството на Ахиезеровците; от Хелека, семейството на Хелековците;
\par 31 от Асриила, семейството на Асрииловците; от Сихема, семейството на Сихемовците;
\par 32 от Семида, семейството на Семидовците; и от Ефера, семейството на Еферовците.
\par 33 А Салпаад Еферовият син нямаше синове, но дъщери; а имената на Салпаадовите дъщери бяха Маала, Нуа, Егла, Мелха и Терса.
\par 34 Тия са Манасиевите семейства; и преброените от тях бяха петдесет и две хиляди и седемстотин души.
\par 35 А ето ефремците по семействата им: от Сутала, семейството на Суталовците; от Вехера, семейството на Вехеровците; от Тахана, семейството на Тахановците.
\par 36 Ето и Суталовите потомци: от Ерана, семейството на Ерановците.
\par 37 Тия са семействата на ефремците; и преброените от тях бяха тридесет и две хиляди и петстотин души. Тия са Иосифовите потомци по семействата си.
\par 38 Вениаминците по семействата си бяха: от Вела, семейството на Веловците; от Асвила, семейството на Асвиловците; от Ахирама, семейството на Ахирамовците;
\par 39 от Суфама, семейството на Суфамовците; от Уфама, семейството на Уфамовците.
\par 40 А Веловите синове бяха Аред и Нееман; от Ареда , семейството на Аредовците; от Неемана, семейството на Неемановците.
\par 41 Тия са вениаминците по семействата си; и преброените от тях бяха четиридесет и пет хиляди и шестстотин души.
\par 42 Ето данците по семействата им: от Суама, семейството на Суамовците. Тия са Дановите семейства по семействата си.
\par 43 Преброените от всичките семейства на Саумовците бяха шестдесет и четири хиляди и четиристотин души.
\par 44 Асирците по семействата си бяха: от Емна, семейството на Емновците; от Есуи, семейството на Есуиевците; от Верия, семейството на Вериевците.
\par 45 От Вериевите потомци бяха: от Хевера, семейството на Хеверовците; от Малхиила, семейството на Малхииловците.
\par 46 А името на Асировата дъщеря беше Сара.
\par 47 Тия са семействата на асирците; и преброените от тях бяха петдесет и три хиляди и четиристотин души.
\par 48 Нефталимците по семействата си бяха: от Ясиила, семейството на Ясииловците; от Гуния, семейството на Гуниевците;
\par 49 от Есера, семейството на Есеровците; от Силима; семейството на Силимовците.
\par 50 Тия са Нефталимовите семейства по семействата си; и преброените от тях бяха четиридесет и пет хиляди и четиристотин души.
\par 51 Числото на преброените от израилтяните беше шестстотин и една хиляда седемстотин и тридесет души.
\par 52 Тогава Господ говори на Моисея, казвайки:
\par 53 На тях нека се раздели земята за наследство, според числото на имената им.
\par 54 На по-многобройните дай по-голямо наследство, а на по-малобройните дай по-малко наследство: на всяко племе да се даде наследството му, според числото на преброените от него.
\par 55 При все това, обаче, земята ще се раздели с жребие; и ще наследят според имената на бащините си племена.
\par 56 С жребие да се раздели наследството им между мнозината и малцината.
\par 57 А ето числото на преброените от левийците по семействата им: от Гирсона, семейството на Гирсоновците; от Каата, семейството на Каатовците; от Мерария, семейството на Мерариевците.
\par 58 Ето Левиевите семейства: семейството на Левиевците, семейството на Хевроновците, семейството на на Маалиевците, семейството на Мусиевците, семейството на Кореевците; а Каат роди Амрама.
\par 59 Името на Амрамовата жена беше Иохавед, Левиева дъщеря, която се роди на Левия в Египет; и тя роди на Амрама Аарона и Моисея и сестра им Мариам.
\par 60 А на Аарона се родиха Надав, Авиуд, Елеазар и Итамар.
\par 61 Но Надав и Авиуд умряха, когато принесоха чужд огън пред Господа.
\par 62 А преброените от Левийците бяха двадесет и три хиляди, всичките мъжки от един месец и нагоре; те не бяха преброени между израилтяните, понеже на тях не се даде наследство между израилтяните.
\par 63 Тия са преброените чрез Моисея и свещеника Елеазар, които преброиха израилтяните на моавските полета при Иордан, срещу Ерихон.
\par 64 Но между тях не се намираше човек от ония, които бяха преброени от Моисея и свещеника Аарон, когато те преброиха израилтяните в Синайската пустиня;
\par 65 защото за тях Господ бе казал: непременно ще измрат в пустинята. От тях не остана ни един, освен Халев Ефониевият син и Исус Навиевият син.

\chapter{27}

\par 1 Тогава дойдоха дъщерите се Салпаада Еферовия син, а Ефер син на Галаада, син на Махира, син на Манасия, от семействата на Манасия Иосифовия син. А ето имената на дъщерите му: Маала, Нуа, Егла, Мелха и Терса.
\par 2 Те застанаха пред Моисея, пред свещеника Елеазара, и пред първенците и цялото общество, при входа на шатъра за срещане, и рекоха:
\par 3 Баща ни умря в пустинята; а той не беше от дружината на ония, които се събраха против Господа в Кореевата дружина, но умря поради своя си грях; и нямаше синове.
\par 4 Защо да изчезне името на баща ни отсред семейството му по причина, че не е имал син? Дай на нас наследството между братята на бащите ни.
\par 5 И Моисей представи делото им пред Господа.
\par 6 Тогава Господ говори на Моисея, казвайки:
\par 7 Право говорят Салпаадовите дъщери; непременно да им дадеш да притежават наследство между братята на баща си, и да направиш да мине върху тях наследството на баща им.
\par 8 И говори на израилтяните, казвайки: Ако умре някой без да има син, тогава да прехвърлите наследството му върху дъщерите му.
\par 9 И ако няма дъщеря, тогава да дадете наследството му на братята му.
\par 10 И ако няма братя, тогава да дадете наследството му на бащините му братя.
\par 11 Но ако баща му няма братя, тогава да дадете наследството му на най-близкия му сродник от семейството му, и той да го притежава. Това да бъде съдебен закон за израилтяните, според както Господ заповяда на Моисея.
\par 12 След това Господ рече на Моисея: Възкачи се на тая планина Аварим и прегледай земята, която съм дал на израилтяните;
\par 13 и като я прегледаш, ще се прибереш и ти при людете си, както се прибра брат ти Аарон;
\par 14 защото в пустинята Цин, когато обществото се противеше, вие не се покорихте на повелението Ми, да Ме осветите при водата пред тях. (Това е водата на Мерива при Кадис в пустинята Цин).
\par 15 А Моисей говори на Господа, казвайки:
\par 16 Господ, Бог на духовете на всяка твар, нека постави над обществото човек,
\par 17 който да излиза пред тях и който да влиза пред тях, който да ги извежда и който да ги въвежда, за да не бъде Господното общество като овци, които нямат пастир.
\par 18 И Господ каза на Моисея: Вземи при себе си Исуса Навиевия син, човек в когото е Духът, и положи на него ръката си;
\par 19 и като го представиш пред свещеника Елеазара и пред цялото общество, дай му поръчка пред тях.
\par 20 И възложи на него част от твоята почетна власт , за да го слуша цялото общество израилтяни.
\par 21 И той да стои пред свещеника Елеазара, който ще се допитва до Господа за него според съда на Урима; и по неговата дума да влизат, той и всичките израилтяни с него, и цялото общество.
\par 22 И Моисей стори според както му заповяда Господ; взе Исуса и представи го пред свещеника Елеазара и пред цялото общество;
\par 23 и като положи ръцете си на него, даде му поръчка, според както Господ заповяда чрез Моисея.

\chapter{28}

\par 1 Господ говори още на Моисея, казвайки:
\par 2 Заповядай на израилтяните, като им кажеш: Внимавайте да Ми принасяте на определеното им време Моите приноси, хляба Ми за благоуханна чрез огън жертва на Мене.
\par 3 И кажи им: Ето приносът чрез огън, който ще принасяте Господу: две едногодишни агнета на ден, без недостатък, за всегдашно всеизгаряне.
\par 4 Едно агне да принасяш заран, а другото агне да принасяш вечер;
\par 5 а за хлебен принос една десета от ефа чисто брашно, смесено с четвърт ин първоток дървено масло.
\par 6 Това е всегдашно всеизгаряне, определено на Синайската планина, за благоуханна жертва чрез огън Господу.
\par 7 И възлиянието му да бъде четвърт ин за едното агне; в светилището да възливаш силно питие за възлияние Господу.
\par 8 А другото агне да принасяш привечер: както утринния хлебен принос, и както възлиянието му, така да го принасяш за благоуханна жертва чрез огън Господу.
\par 9 А в съботен ден да принасяте две две едногодишни агнета, без недостатък, и две десети от ефа чисто брашно, смесено с дървено масло за хлебен принос с възлиянието му.
\par 10 Това е всеизгарянето за всяка събота, освен всегдашното всеизгаряне с възлиянието му.
\par 11 В новолунията си да принасяте за всеизгаряне Господу два юнеца, един овен, седем едногодишни агнета без недостатък;
\par 12 и за всеки юнец три десети от ефа чисто брашно, смесено с дървено масло за хлебен принос; и за единия овен две десети от ефа чисто брашно, смесено с дървено масло за хлебен принос;
\par 13 и за всяко агне по една десета от ефа чисто брашно, смесено с дървено масло за хлебен принос. Това е всеизгаряне, за благоуханна жертва чрез огън Господу.
\par 14 И възлиянието им да бъде вино, половин ин за юнеца, една трета от ин за овена и четвърт ин за агнето. Това е всеизгарянето за един месец през месеците на годината.
\par 15 И освен всегдашното всеизгаряне да се принася Господу един козел в принос за грях, с възлиянието му.
\par 16 На четиринадесетия ден от първия месец е Господната пасха.
\par 17 А на петнадесетия ден от тоя месец е празник; седем дена да се яде безквасен хляб.
\par 18 На първия ден да има свето събрание, и да не работите никаква слугинска работа;
\par 19 а да принесете жертва чрез огън, за всеизгаряне Господу: два юнеца, един овен и седем едногодишни агнета, които да бъдат без недостатък.
\par 20 А хлебният им принос да бъде от чисто брашно, смесено с дървено масло; три десети от ефа да принесете за юнеца; две десети за овена;
\par 21 и по една десета от ефа да принесеш за всяко от седемте агнета;
\par 22 и един козел в принос за грях, за да се извърши умилостивение за вас.
\par 23 Тия да принесете в прибавка на утринното всеизгаряне, което е всегдашно всеизгаряне.
\par 24 Така да принасяте храната всеки ден през седемте дена за благоуханна жертва чрез огън Господу: това да се принася с възлиянието му, в прибавка на всегдашното всеизгаряне.
\par 25 А на седмия ден да имате свето събрание и да не работите никаква слугинска работа.
\par 26 Също и в деня на първите плодове, когато принасяте нов хлебен принос Господу през празника ви на седмиците, да имате своето събрание, и да не работите никаква слугинска работа.
\par 27 И за благоухание Господу да принесете във всеизгаряне два юнеца, един овен и седем едногодишни агнета.
\par 28 А хлебният им принос да бъде от чисто брашно, смесено с дървено масло, - три десети от ефа за всеки юнец, две десети за единия овен,
\par 29 и по една десета за всяко от седемте агнета;
\par 30 и един козел, за да се извърши умилостивение за вас.
\par 31 Тия да принесете (без недостатък да бъдат) с възлиянието им, в прибавка на всегдашното всеизгаряне с хлебния му принос.

\chapter{29}

\par 1 В седмия месец, на първия ден от месеца, да имате свето събрание и да не работите никаква слугинска работа; той да ви бъде ден на тръбно възклицание.
\par 2 Да принесете във всеизгаряне, за благоухание Господу, един юнец, един овен, седем едногодишни агнета без недостатък;
\par 3 а хлебният им принос да бъде чисто брашно, смесено с дървено масло, - три десети от ефа за юнеца, две десети за овена,
\par 4 и по една десета за всяко от седемте агнета;
\par 5 и един козел в принос за грях, за да се извърши умилостивение за вас;
\par 6 в прибавка на новолунното всеизгаряне и хлебния му принос, и всегдашното всеизгаряне и хлебния му принос, с възлиянията им, според определеното за тях, за благоуханна жертва чрез огън Господу.
\par 7 И на десетия ден от тоя седми месец да имате свето събрание, и да смирите душите си, и да не работите никаква слугинска работа;
\par 8 а да принесете във всеизгаряне Господу, за благоухание, един юнец, един овен, седем едногодишни агнета, които да бъдат без недостатък;
\par 9 а хлебният им принос да бъде чисто брашно, смесено с дървено масло, - три десети от ефа за юнеца, две десети за единия овен,
\par 10 и по една десета за всяко от седемте агнета;
\par 11 един козел в принос за грях, в прибавка на приноса в умилостивение за грях и всегдашното всеизгаряне и хлебния му принос, с възлиянията им.
\par 12 И на петнадесетия ден от седмия месец да имате свето събрание, и да не работите никаква слугинска работа, а да пазите празник Господу седем дена.
\par 13 Да принесете във всеизгаряне, за благоуханна жертва чрез огън Господу, тринадесет юнеца, два овена, четиринадесет едногодишни агнета, които да бъдат без недостатък.
\par 14 А хлебният им принос да бъде чисто брашно, смесено с дървено масло, - три десети от ефа за всеки от тринадесетте юнеца, две десети за всеки от двата овена.
\par 15 и по една десета за всяко от четиринадесетте агнета;
\par 16 и един козел в принос за грях, в прибавка на всегдашното всеизгаряне и хлебния му принос, с възлиянието му.
\par 17 На втория ден да принесете дванадесет юнеца, два овена, четиринадесет едногодишни агнета, без недостатък;
\par 18 хлебния им принос с възлиянията им за юнците, за овните и за агнетата, според числото им, както е определено;
\par 19 и един козел в принос за грях, в прибавка на всегдашното всеизгаряне и хлебния му принос, с възлиянията им.
\par 20 На третия ден, единадесет юнеца, два овена, четиринадесет едногодишни агнета, без недостатък;
\par 21 хлебния им принос с възлиянията им за юнците, за овните, и за агнетата според числото им, както е определено;
\par 22 и един козел в принос за грях, в прибавка на всегдашното всеизгаряне и хлебния му принос, с възлиянието му.
\par 23 На четвъртия ден, десет юнеца, два овена, четиринадесет едногодишни агнета, без недостатък;
\par 24 хлебният им принос с възлиянията им за юнците, за овните и за агнетата, според числото им, както е определено
\par 25 и един козел в принос за грях, в прибавка на всегдашното всеизгаряне и хлебния му принос, с възлиянието му.
\par 26 На петия ден, девет юнеца, два овена, четиринадесет едногодишни агнета без недостатък;
\par 27 хлебния им принос с възлиянията им за юнците, за овните и за агнетата, според числото им, както е определено;
\par 28 и един козел в принос за грях, в прибавка на всегдашното всеизгаряне и хлебния му принос, с възлиянието му.
\par 29 На шестия ден, осем юнеца, два овена, четиринадесет едногодишни агнета, без недостатък;
\par 30 хлебният им принос с възлиянията им за юнците, за овните и за агнетата, според числото им, както е определено;
\par 31 и един козел в принос за грях, в прибавка на всегдашното всеизгаряне и хлебния му принос, с възлиянието му.
\par 32 На седмия ден, седем юнеца, два овена, четиринадесет едногодишни агнета, без недостатък;
\par 33 хлебният им принос с възлиянията им за юнците, за овните и за агнетата, според числото им, както е определено за тях;
\par 34 и един козел в принос за грях, в прибавка на всегдашното всеизгаряне и хлебния му принос с възлиянието му.
\par 35 На осмия ден да имате тържествено събрание и да не работите никаква слугинска работа;
\par 36 и да принесете във всеизгаряне, за благоуханна жертва чрез огън Господ, един юнец, един овен, седем едногодишни агнета, без недостатък;
\par 37 хлебния им принос с възлиянията им за юнеца, за овена и за агнетата, според числото им, както е определено;
\par 38 и един козел в принос за грях, в прибавка на всегдашното всеизгаряне и хлебния му принос, с възлиянието му.
\par 39 Тия да принасяте Господу на празниците си, - в прибавка на обреците си и доброволните си приноси, - за всеизгарянията си, за хлебните си приноси, за възлиянията си, и за примирителните си приноси.
\par 40 И Моисей каза тия неща на израилтяните напълно, както Господ заповяда на Моисея.

\chapter{30}

\par 1 Така също Моисей говори на началниците от племената на израилтяните, казвайки: Ето що заповяда Господ:
\par 2 Когато някой мъж направи обрек Господу, или се закълне с клетва та обвързва душата си със задължение, нека не наруши думата си, но нека извърши, според всичко що е излязло из устата му.
\par 3 Тоже, ако някоя жена направи обрек Господу, и обвърже себе си със задължение, в младостта си, в бащиния си дом,
\par 4 и баща й чуе обрека й и задължението, с което е обвързала душата си, и баща й не й каже нищо, тогава всичките й обреци си остават в сила, и всяко нейно задължение, с което е обвързала душата си, си остава в сила.
\par 5 Но ако й забрани баща й, в деня когато чуе, то никой от обреците й или от задълженията, с които е обвързала душата си, няма да остане в сила; и Господ ще й прости понеже й е забранил баща й.
\par 6 Но ако се омъжи, като има на себе си обрека си, или нещо необмислено изговорени с устните й, с което е обвързала душата си,
\par 7 и мъжът й, като чуе, не й каже нищо, в деня когато чуе, тогава обреците и си остават в сила, и задълженията, с които е обвързала душата си, си остават в сила.
\par 8 Но ако й забрани мъжът й, в деня когато чуе, тогава той ще унищожи обрека, който е взела върху себе си, и онова, което е необмислено изговорила с устните си, с което е обвързала душата си; и Господ ще й прости.
\par 9 Обрек направен от вдовица или напусната жена, всичко, с което би обвързала душата си остава върху нея.
\par 10 Но ако една жена е направила обрек в къщата на мъжа си, или е обвързала душата си с клетвено задължение,
\par 11 и мъжът й е чул и не й е казал нищо, нито й е забранил, тогава всичките й обреци си остават в сила, и всичките задължения, с които е обвързала душата си, си остават в сила.
\par 12 Но ако, в деня, когато е чул, мъжът й ги е съвсем унищожил, тогава онова, което е излязло из устните й относно обреците й относно обвързването на душата й не ще остане в сила; мъжът й ги е унищожил, и Господ ще и прости.
\par 13 Мъжът й може да утвърди, и мъжът й може да унищожи всеки обрек и всяко клетвено задължение за смиряването на душата й;
\par 14 но ако мъжът й ден след ден продължава да мълчи, тогава той потвърждава всичките й обреци и всичките задължения, които са върху нея; той ги е потвърдил защото не й е казал нищо в деня, когато ги е чул.
\par 15 Но ако ги унищожи някак по-после, след като ги е чул, тогава той ще носи нейния грях.
\par 16 Тия са повеленията, които Господ заповяда на Моисея, да се пазят между мъж и жена му, и между баща и дъщеря му в младостта й, догдето е в бащиния си дом; относно обреците .

\chapter{31}

\par 1 След това Господ говори на Моисея, казвайки:
\par 2 Въздай на мадиамците за израилтяните, и след това ще се прибереш при людете си.
\par 3 Моисей, прочее, говори на людете, казвайки: Нека се въоръжат от вас мъже за бой и нека отидат против Мадиама за да извършат въздаяние върху Мадиама за Господа.
\par 4 По хиляда души от всяко племе от всичките Израилеви племена да изпратите на войната.
\par 5 И така, от Израилевите хиляди се преброиха по хиляда души от всяко племе, дванадесет хиляди души, въоръжени, за бой.
\par 6 И Моисей ги изпрати на войната, по хиляда души от всяко племе, тях и Финееса, сина на свещеника Елеазара, със светите вещи и с тръбите за тревога в ръцете му.
\par 7 Те воюваха против Мадиама според както Господ заповяда на Моисея, и убиха всяко мъжко.
\par 8 И между убитите убиха и мадиамските царе: Евия, Рекема, Сура, Ура и Рава, петима мадиамски царе; убиха с нож и Валаама Веоровия син.
\par 9 А израилтяните плениха жените на мадиамците, децата им, всичкия им добитък и всичките им стада; и разграбиха всичкия им имот.
\par 10 А всичките им градове, в местата населени от тях, и всичките им станове изгориха с огън.
\par 11 И взеха всичките користи и всичкото плячка, и човек и животно.
\par 12 И докараха пленниците, плячката и користите на Моисея, на свещеника Елеазар и на обществото израилтяни в стана, на моавските полета при Иордан, срещу Ерихон.
\par 13 Тогава Моисей, свещеникът Елеазар и всичките първенци на обществото излязоха да ги посрещнат вън от стана.
\par 14 Но Моисей се разгневи на военачалниците, на хилядниците и на стотниците, които се връщаха от военния поход;
\par 15 и Моисей им рече: Оставихте ли живи всичките жени?
\par 16 Ето, те, по съвета на Валаама, накараха израилтяните да беззаконствуват против Господа в делото на Фегора, тъй че язвата се яви всред Господното общество.
\par 17 За това, убийте сега всичките мъжки от децата, и убийте всяка жена, която е познала мъж, като е лежала с него.
\par 18 А всичките момичета, които не са познали мъж, които не са лежали с такъв, оставете живи за себе си.
\par 19 И вие останете вън от стана седем дена; вие и пленниците ви, всеки от вас , който е убил човек, и който се е допрял до убит, очистете се на третия ден и на седмия ден;
\par 20 очистете и всичките дрехи, всичките кожени вещи, и всичко, което е направено от козина, и всичките дървени съдове..
\par 21 Също и свещеник Елеазар рече на войниците, които бяха ходили на война: Това е повелението, което Господ заповяда на Моисея като закон.
\par 22 Само че златото, среброто, медта, желязото, оловото и калая,
\par 23 всичко, което може да устои на огън, прекарайте през огън, и ще бъде чисто; обаче, трябва да се очисти и с очистителна вода; и всичко що не може да устои на огън прекарай през вода.
\par 24 Тогава на седмия ден да изперете дрехите си, и ще бъдете чисти; и подир това да влезете в стана.
\par 25 И Господ говори на Моисея, казвайки:
\par 26 Ти и свещеникът Елеазар и началниците на бащините домове от обществото пребройте плячката и пленниците, и човек и животно,
\par 27 и раздели плячката на две, между войниците, между войниците, които ходиха на война и цялото общество.
\par 28 От дела на войниците, които ходиха на бой, отдели като данък за Господа то една душа от петстотин - от човеци, от говеда, от осли и от овци;
\par 29 от тяхната половина да вземеш това и да го дадеш на свещеника Елеазара, като възвишаем принос Господу.
\par 30 А от половината, която се дава на израилтяните, да вземеш по един дял от петдесет - от човеци, от говеда, от осли и от овци - от всеки добитък, - и да ги дадеш на лавитите, които пазят заръчаното за Господната скиния.
\par 31 И тъй Моисей и свещеникът Елеазар сториха, според както Господ заповяда на Моисея.
\par 32 А плячката, то ест , това, което остана в плен, което войниците заплениха, беше: овци, шестстотин седемдесет и пет хиляди;
\par 33 говеда, седемдесет и две хиляди:
\par 34 осли, шестдесет и една хиляда
\par 35 и човеци - жени, които не бяха познали мъж, чрез лягане с такъв - всичко тридесет и две хиляди души.
\par 36 Половината, делът на ония, които бяха ходили на бой, беше на брой: овци, триста и тридесет и седем хиляди и петстотин;
\par 37 а данъкът за Господа от овците беше шестстотин седемдесет и пет;
\par 38 говедата бяха тридесет и шест хиляди, от които данъкът за Господа беше седемдесет и две;
\par 39 ослите бяха тридесет хиляди и петстотин, от които данъкът за Господа беше шестдесет и един;
\par 40 и човеците бяха шестнадесет хиляди, от които данъкът за Господа беше тридесет и двама души,
\par 41 И Моисей даде данъка, като възвишаем принос Господу, на свещеника Елеазара, както Господ заповяда на Моисея.
\par 42 А половината, която се даде на израилтяните, който Моисей отдели от войниците,
\par 43 то ест , половината за обществото, беше: овци, триста и тридесет и седем хиляди и петстотин;
\par 44 говеда, тридесет и шест хиляди;
\par 45 осли, тридесет хиляди и петстотин;
\par 46 и човеци - шестнадесет хиляди.
\par 47 От тия половината за израилтяните Моисей взе по един дял от петдесет от човеци и от животни, и ги даде на левитите, които пазеха заръчаното за Господната скиния, според както Господ заповяда на Моисея.
\par 48 Тогава се приближиха при Моисея началниците, които бяха над хилядите на войската, хилядниците и стотниците,
\par 49 и рекоха на Моисея: Слугите ти преброиха войниците, които са под ръката ни; и не липсва ни един от нас.
\par 50 За това принасяме дар Господу, всеки каквото е намерил, златни неща, верижки, гривни, пръстени, обеци и мъниста, за да се извърши умилостивение за душите ни пред Господа.
\par 51 И Моисей и свещеникът Елеазар взеха златото от тях, всичко в изработени украшения.
\par 52 И всичкото злато от възвишаемия принос, който хилядниците и стотниците принесоха Господу, беше шестнадесет хиляди седемстотин и петдесет сикли.
\par 53 (Защото войниците бяха грабили, всеки за себе си).
\par 54 И Моисей и свещеникът Елеазар, като взеха златото от хилядниците и стотниците, донесоха го в шатъра за срещане, за спомен на израилтяните пред Господа.

\chapter{32}

\par 1 А рувимците и гадците имаха твърде много добитък; и когато видяха Язирската земя и Галаадската земя, че, ето, мястото беше място за добитък,
\par 2 то гадците и рувимците дойдоха та говориха на Моисея, на свещеника Елеазара, и на първенците на обществото, казвайки:
\par 3 Атарот, Девон, Язир, Нимра, Есевон, Елеала, Севама, Нево и Веон,
\par 4 земята, която Господ порази пред Израилевото общество, е земя за добитък; а слугите ти имат добитък.
\par 5 За това, рекоха, ако сме придобили твоето благоволение, нека се даде тая земя на слугите ти имат добитък.
\par 6 А Моисей рече на гадците и на рувимците: Да идат ли братята ви на бой, а вие тук да седите?
\par 7 Защо обезсърчавате сърцата на израилтяните, та да не преминат в земята която Господ им е дал?
\par 8 Така сториха бащите ви, когато ги изпратиха от Кадис-варни, за да видят земята;
\par 9 отидоха до долината Есхол, и, като видяха земята, обезсърчиха сърцата на израилтяните, та да не възлязат в земята, която Господ им бе дал.
\par 10 И в оня ден гневът на Господа пламна, и той се закле казвайки:
\par 11 Ни един от ония мъже, които излязоха из Египет, от двадесет години и нагоре, няма да види земята, за която съм се клел на Авраама, Исаака и Якова, защото не Ме последваха напълно,
\par 12 освен Халева син на Ефония Кенезов и Исус Навиевият син, защото те напълно последваха Господа.
\par 13 Гневът на Господа пламна против Израиля, и той ги направи да се скитат из пустинята за четиридесет години, догде се довърши изцяло онова поколение, което беше сторило зло пред Господа.
\par 14 И, ето, вместо бащите си, издигнахте се вие, прибавка на грешни човеци, и ще разпалите повече пламъка на Господния гняв против Израиля.
\par 15 Защото, ако вие се отвърнете от Него, Той ще остави тях още еднъж в пустинята; и така вие ще станете причина да погинат всички тия люде.
\par 16 Но те пристъпиха при Моисея и рекоха: Ще съградим тука огради за добитъка си и градове за челядите си;
\par 17 а сами ние сме готови да вървим въоръжени пред израилтяните, догде ги заведем до мястото им; и челядите ни ще седят в укрепените градове в безопасност от местните жители.
\par 18 Няма да се върнем в домовете си догде израилтяните не наследят, всеки наследството си.
\par 19 Защото ние няма да наследим с тях отвъд Иордан и по-нататък, понеже нашето наследство ни се падна отсам Иордан на изток.
\par 20 Тогава Моисей им рече: Ако направите това, ако отидете въоръжени на бой пред Господа,
\par 21 ако всички въоръжени преминете Иордан пред Господа, догде изгони Той враговете Си от пред Себе Си,
\par 22 и земята се завладее пред Господа, а подир това се върнете, тогава ще бъдете невинни пред Господа и пред Израиля, и ще имате тая земя за притежание пред Господа.
\par 23 Но ако не направите така, ето, ще съгрешите пред Господа; и да знаете, че грехът ви ще ви намери.
\par 24 Съградете градове за челядите си и гради за овците си, и сторете това, което излезе из устата ви.
\par 25 И гадците и рувимците говориха на Моисея, казвайки: Слугите ти ще сторят, както господарят ни каза.
\par 26 Децата ни, жените ни, стадата ни и всичкият ни добитък ще останат тук в галаадските градове;
\par 27 а слугите ти, всички въоръжени и опълчени, ще отидат пред Господа на бой, както господарят ни каза.
\par 28 Тогава Моисей даде поръчка за тях на свещеника Елеазара, на Исуса Навина, и на началниците на бащините домове от племената на израилтяните.
\par 29 Моисей им рече: Ако гадците и рувимците преминат с вас Иордан, всички въоръжени за бой пред Господа, и се завладее земята пред вас, тогава ще им дадете Галаадската земя за притежание.
\par 30 Но ако не щат да преминат с вас въоръжени, тогава да вземат наследство между вас в Ханаанската земя.
\par 31 И гадците и рувимците в отговор рекоха: Както рече Господ на слугите ти, така, ще сторим.
\par 32 Ние ще заминем въоръжени пред Господа в Ханаанската земя, за да притежаваме наследството си оттатък Иордан.
\par 33 И тъй, Моисей им даде, то ест , на гадците, на рувимците и на половината от племето на Иосифовия син Манасия, царството на аморейския цар Сион и царството на васанския цар Ог, земята заедно с градовете в пределите й, градовете на околната земя.
\par 34 И гадците съградиха Девон, Атарот, Ароир,
\par 35 Атротсофан, Язир, Иогвея,
\par 36 Ветнимра и Ветаран, укрепени градове и огради за овци.
\par 37 А рувимците съградиха Есевон, Елеала, Кириатаим,
\par 38 Нево и Ваалмеон (с променени имена) и Севама; и преименува градовете, които съградиха.
\par 39 И потомците на Манасиевия син Махир отидоха в Галаад, завладяха го и изпъдиха аморейците, които бяха в него.
\par 40 За това Моисей даде Галаад на Махира Манасиевия син, и той се засели в него.
\par 41 А Манасиевият син Яир отиде та завладя градовете му и ги наименува Авот-Яир.
\par 42 И Нова отиде та превзе Кенан и селата му и го наименува Нова по своето име.

\chapter{33}

\par 1 Ето пътуванията на израилтяните, които излязоха из Египетската земя по устроените си войнства под Моисеева и Ааронова ръка.
\par 2 Моисея по Господно повеление написа тръгванията им според пътуванията им; и ето пътуванията им според тръгванията им.
\par 3 В първия месец, на петнадесетия ден от първия месец, отпътуваха от Рамесий; на сутринта на пасхата израилтяните излязоха с издигната ръка пред очите на всичките египтяни,
\par 4 когато египтяните погребваха всичките си първородни, които Господ беше поразил помежду им. (И над боговете им Господ извърши съдби).
\par 5 И израилтяните, като отпътуваха от Рамесий, разположиха стан в Сикхот.
\par 6 Като отпътуваха от Сокхот, разположиха стан в Етам, който е в края на пустинята.
\par 7 Като отпътуваха от Етам върнаха се към Пиаирот, който е срещу Веелсефон, и разположиха стан срещу Мигдол.
\par 8 А когато отпътуваха от Пиаирот, преминаха през морето в пустинята, та пътуваха тридневен път през пустинята Етам и разположиха стан в Мера.
\par 9 А като отпътуваха от Мера, дойдоха в Елим; а в Елим имаше дванадесет водни извори и седемдесет палмови дървета, и там разположиха стан.
\par 10 Като отпътуваха от Елим, разположиха стан при Червеното море.
\par 11 Като отпътуваха от Червеното море, разположиха стан в пустинята Син.
\par 12 Като отпътуваха от пустинята Син, разположиха стан в Дофка,
\par 13 Като отпътуваха от Дофка, разположиха стан в Елус.
\par 14 Като отпътуваха от Елус, разположиха стар в Рафидим, гдето имаше вода да пият людете.
\par 15 Като отпътуваха от Рафидим, разположиха стан в Синайската пустиня.
\par 16 Като отпътуваха от Синайската пустиня, разположиха стан в Киврот-атаава.
\par 17 Като отпътуваха от Киврот-атаава, разположиха стан в Асирот.
\par 18 Като отпътуваха от Асирот, разположиха стан в Ритма,
\par 19 Като отпътуваха от Ритма, разположиха стан в Римот-Фарес.
\par 20 Като отпътуваха от Римот-Фарес, разположиха стан в Ливна.
\par 21 Като отпътуваха от Ливна, разположиха стан в Риса.
\par 22 Като отпътуваха от Риса, разположиха стан в Кеелата.
\par 23 Като отпътуваха от Кеелата, разположиха стан в хълма Сафер.
\par 24 Като отпътуваха от хълма Сафер, разположиха стан в Харада.
\par 25 Като отпътуваха от Харада, разположиха стан в Макилот.
\par 26 Като отпътуваха от Макилот, разположиха стан в Тахат.
\par 27 Като отпътуваха от Тахат, разположиха стан в Тара.
\par 28 Като отпътуваха от Тара, разположиха стан в Митка.
\par 29 Като отпътуваха от Митка, разположиха стан в Асемона.
\par 30 Като отпътуваха от Асемона, разположиха стан в Масирот.
\par 31 Като отпътуваха от Масирот, разположиха стан във Венеякан.
\par 32 Като отпътуваха от Венеякан, разположиха стан в Хоргадгад.
\par 33 Като отпътуваха от Хоргадгад, разположиха стан в Иотвата.
\par 34 Като отпътуваха от Иотвата, разположиха стан в Еврона.
\par 35 Като отпътуваха от Еврона, разположиха стан в Есион-гавер.
\par 36 Като отпътуваха от Есион-гавер, разположиха стан в пустинята Цин, която е Кадис.
\par 37 А като отпътуваха от Кадис, разположиха стан в планината Ор, при края на Едомската земя.
\par 38 И по Господно повеление свещеникът Аарон се изкачи на планината Ор, и умря там, в четиридесетата година от излизането на израилтяните от Египетската земя, в петия месец, на петия ден от месеца.
\par 39 Аарон беше на сто и двадесет и три години, когато умря на планината Ор.
\par 40 И арадският цар, ханаанецът, който живееше на юг от Ханаанската земя, чу за идването на израилтяните.
\par 41 А те, като отпътуваха от планината Ор, разположиха стан в Салмона.
\par 42 Като отпътуваха от Салмона, разположиха стан в Финон.
\par 43 Като отпътуваха от Финон, разположиха стан в Овот.
\par 44 Като отпътуваха от Овот, разположиха стан в Е-аварим, на моавската граница.
\par 45 Като отпътуваха от Иим, разположиха стан в Девон-гад.
\par 46 Като отпътуваха от Девон-гад, разположиха стан в Алмон-дивлатаим.
\par 47 Като отпътуваха от Алмон-дивлатаим, разположиха стан на планината Аварим, срещу Нево.
\par 48 А като отпътуваха от планината Аварим, разположиха стан на моавските полета при Иордан, срещу Ерихон.
\par 49 При Иордан разположиха стан, от Ветиесимот до Авел-ситим, на моавските полета.
\par 50 Тогава Господ говори на Моисея на моавските полета при Иордан, срещу Ерихон, казвайки:
\par 51 Говори на израилтяните, като им кажеш: Когато минете през Иордан в Ханаанската земя,
\par 52 изгонете от пред себе си всичките жители на земята, изтребете всичките им изображения, унищожете всичките им леяни идоли, и съборете всичките им оброчища.
\par 53 И завладейте земята и заселете се в нея; защото на вас съм дал тая земя за наследство.
\par 54 И да разделите земята с жребие между семействата си за наследство на по-многобройните да дадете по-голямо наследство, а на по-малобройните да дадете по-малко наследство. На всекиго наследството да бъде там, гдето му падне жребието. Според бащините си племена да наследите.
\par 55 Но ако не изгоните от пред себе си жителите на земята, тогава ония от тях, които оставите, ще бъдат тръни в очите ви и бодли в ребрата ви и ще ви измъчват в земята, в която живеете.
\par 56 А и при туй, онова, което мислеха да сторя на тях, ще го сторя на вас.

\chapter{34}

\par 1 Господ говори още на Моисея, казвайки:
\par 2 Заповядай на израилтяните като им кажеш: Когато влезете в Ханаанската земя, (земята, която ще ви се падне в наследство, Ханаанската земя според границите й),
\par 3 тогава южните ви земи да бъдат от пустинята Цин покрай Едом: и южната ви граница да бъда от края на Соленото море на изток.
\par 4 Южната ви граница да завива към нагорнището на Акравим и да отида до Цин, и да продължава от южната страна до Кадис-варни, да излиза на Асар-адар и да отива до Асмон.
\par 5 И границата да завива от Асмон до египетския поток и да стигне до морето.
\par 6 А за западна граница да имате Голямото море; това да ви бъде западната граница.
\par 7 Северните ви граници да бъдат тия: от Голямото море да прокарате границата до половината Ор;
\par 8 от планината Ор да прокарате границата до прохода на Емат; и границата да продължава до Седад.
\par 9 И границата да продължава до Зифрон и да излиза на Асаренан. Това да ви бъде границата.
\par 10 А източната си граница да прокарате от Асаренан до Шефам.
\par 11 И границата да слиза от Шефам до Ривла на изток от Аин; и границата да слиза и да досяга брега на езерото Хинерот на изток.
\par 12 И границата да слиза до Иордан и да излиза на Соленото море. Това ще бъде земята ви според окръжаващите я граници.
\par 13 Моисея, прочее, заповяда на израилтяните като каза: Това е земята, която ще наследите чрез жребие, която Господ заповяда да се даде на деветте и половина племена.
\par 14 Защото племето на рувимците, според бащините си домове, и племето на гадците, според бащините си домове, взеха наследството си, както и половината от Манасиевото племе взе.
\par 15 Тия две и половина племена взеха наследството си оттатък Иордан, срещу Ерихон, на изток.
\par 16 И Господ говори на Моисея, казвайки:
\par 17 Ето имената на мъжете, които ще ви разделят земята в наследство: свещеникът Елеазар и Исус Навиевият син.
\par 18 Също и от всяко племе да вземете по един първенец, за да разделят земята в наследство.
\par 19 А ето имената на тия мъже: от Юдовото племе: Халев Ефониевият син;
\par 20 от племето на симеонците: Самуил Амиудовият син;
\par 21 от Вениаминовото племе: Елидад Хислоновият син;
\par 22 от племето на данците: първенец Вукий Иоглиевият син;
\par 23 от Иосифовите потомци, от племето на манасийците: първенец Аниил Ефодовият син;
\par 24 а от племето на ефремците: първенец Камуил Сафтановият син;
\par 25 от племето на завулонците: първенец Елисафан Фарнаховият син;
\par 26 от племето на исахарците: първенец Фалтиил Азановият син;
\par 27 от племето на асирците: първенец Ахиуд Шеломиевият син;
\par 28 и от племето на нефталимците: първенец Федаил Амиудовият син.
\par 29 Тия са, на които Господ заповяда да разделят наследството на израилтяните в Ханаанската земя.

\chapter{35}

\par 1 Господ говори още на Моисея в моавските полета при Иордан срещу Ерихон, казвайки:
\par 2 Заповядай на израилтяните да дадат на левитите от наследственото си притежание градове, в които да се зеселят; дайте на левитите и пасбища около градовете им.
\par 3 Градовете да им служат за живеене, а пасбищата да им служат за говедата им, за имота им и за всичките им животни.
\par 4 Пасбищата около градовете, които ще дадете на левитите, да се простират хиляда лакътя навън от градската стена наоколо.
\par 5 Да измерите извън града, на източната страна, две хиляди лакътя, на южната страна две хиляди лакътя, на западната страна две хиляди лакътя, северната страна две хиляди лакътя, а градът да бъде в средата. Такива да бъдат пасбищата около градовете им.
\par 6 И градовете, които ще дадете на левитите, да бъдат шест града за прибягване, които да определите, за да може да прибягва там убиецът; и на тях да притурите още четиридесет и два града.
\par 7 Всичките градове, които ще дадете на левитите, да бъдат четиридесет и осем града; дайте ги заедно с пасбищата им.
\par 8 И когато дадете градовете от притежанието на израилтяните, от многото градове дайте много, а от малкото - дайте малко; всяко племе да даде на левитите от градовете си според наследството, което е наследило.
\par 9 Господ говори още на Моисея казвайки:
\par 10 Говори на израилтяните, казвайки им: Когато минете през Иордан в Ханаанската земя,
\par 11 тогава да си определите градове, които да ви бъдат градове за прибежище, за да може да прибягва там убиецът, който убие човек по погрешка.
\par 12 Те да ви бъдат градове за избягване от сродника - мъздовъздател, за да се не убие убиецът преди да се представи на съд пред обществото.
\par 13 От градовете, които ще дадете, шест да ви бъдат градове за прибягване.
\par 14 Три града да дадете оттатък Иордан, и три града да дадете в Ханаанската земя, да бъдат градове за прибягване.
\par 15 Тия шест града да бъдат прибежище за израилтяните и за чужденеца, и за онзи, който е пришелец помежду им, за да може да прибягват там всеки, който би убил човек по погрешка.
\par 16 Ако някой удари някого с желязно оръдие, та умре, убиец е; убиецът непременно да се умъртви.
\par 17 Ако го е ударил с камък из ръката си, от който може да умре, та умре, убиец е; убиецът непременно да се умъртви.
\par 18 Или ако го е ударил с дървено оръжие в ръката си, от което може да умре, та умре, убиец е; убиецът непременно да се умъртви.
\par 19 Мъздовъздателят за кръвта сам да умъртви убиеца; когато го срещне, да го умъртви.
\par 20 И ако го тласне от омраза, или из засада хвърли нещо върху него, та умре,
\par 21 или от омраза го удари с ръката си, та умре, тоя, който го е ударил, непременно да се умъртви убиец е; мъздовъздателят за кръвта да умъртви убиеца, когато го срещне.
\par 22 Но ако го тласне внезапно, без да го е намразил, или хвърли нещо върху него без да го е причаквал,
\par 23 или ако, без да види, направи да падне на него някакъв камък, от който може да умре, та умре, без да му е бил неприятел, или да е искал да му стори зло,
\par 24 тогава обществото да отсъди между убиеца и мъздовъздателя за кръвта според тия съдби;
\par 25 и обществото да избави убиеца от ръката на мъздовъздателя за кръвта, и обществото да го върне в града, гдето бе прибягнал за прибежище; и той да живее в него до смъртта на първосвещеника, който е помазан със светото миро.
\par 26 Но ако убиецът излезе кога да е вън от пределите на прибежищния град, в който бе прибягнал,
\par 27 и мъздовъздателят за кръвта го намери вън от пределите на прибежищния град, и мъздовъздателят за кръвта умъртви убиеца, тоя няма да бъде виновен за кръвопролитие;
\par 28 защото убиецът трябваше да стои в прибежищния си град до смъртта на първосвещеника. А след смъртта на първосвещеника нека се върне убиецът в земята, която му е притежание.
\par 29 Това да ви бъде съдебен закон във всичките ви поколения по всичките ви заселища.
\par 30 Който убие някого, тоя убиец да се умъртви при думите на свидетели: обаче не бива само един свидетел да свидетелствува против някого, за да се умъртви.
\par 31 Нито да взимате някакъв откуп за живота на убиеца, който като виновен заслужава смърт; но непременно той да се умъртви.
\par 32 Да не взимате откуп и за онзи, който е прибягнал в прибежищен град, за да се върне да живее на мястото си преди смъртта на свещеника.
\par 33 Така няма да оскверните земята, в която се намирате; защото кръвта, тя е, която осквернява земята; и земята не може да се очисти от кръвта, която се е проляла на нея, освен с кръвта на онзи, който я е пролял.
\par 34 Нито един от вас да не осквернява земята, в която живеете, всред която Аз обитавам; защото Аз Иеова обитавам всред израилтяните.

\chapter{36}

\par 1 Тогава началниците на бащините домове от семействата на потомците на Галаада, син на Махира, Манасиевият син, от семействата на Иосифовите потомци, се приближиха и говориха пред Моисея и пред първенците, началниците на бащините домове на израилтяните казвайки:
\par 2 Господ заповяда на господаря ни да даде земята за наследство на израилтяните с жребие; и господаря ни получи заповед от Господа да даде наследство на брата ни Салпаада на дъщерите му,
\par 3 Но ако те се омъжат за някои от синовете на другите племена на израилтяните, тогава наследството им ще се отнеме от наследството на нашите бащи, и ще се приложи към наследството на към наследството на онова племе, което ще ги вземе за жени ; така то ще се отнеме от наследството, което ни дава жребието.
\par 4 Даже , когато дойде юбилеят на израилтяните, тогава наследството им ще се приложи към наследството на онова племе, което ще ги е взело; и така, наследството им ще се отнеме от наследството на отеческото ни племе.
\par 5 Прочее, по Господното слово, Моисей заповяда на израилтяните, като каза: Племето на Иосифовите потомци говори право.
\par 6 Ето какво заповяда Господ за Салпаадовите дъщери; Той казва: Нека се омъжват, за когото им се види добре, стига само да се омъжат за мъже от отеческото си племе.
\par 7 По тоя начин, никое наследство на израилтяните няма да премине от племе на племе; защото израилтяните трябва да се привързват, всеки за наследството на отеческото си племе.
\par 8 Всяка дъщеря, която има наследство в някое племе на израилтяните, трябва да се омъжи за едного от семейството на отеческото си племе, тъй щото израилтяните да притежават всички отеческото си наследство.
\par 9 Да не преминава наследството от едно племе на друго племе, но племената на израилтяните да се привързват, всяко за своето наследство.
\par 10 И Салпаадовите дъщери сториха според както Господ заповяда на Моисея;
\par 11 защото Маала, Терса, Егла, Мелха и Нуа, Салпаадовите дъщери, се омъжиха за синовете на чичовете си;
\par 12 омъжиха се за мъже от семействата на потомците на Манасия Иосифовия син; и така наследството им остана в племето на бащиното им семейство.
\par 13 Тия са заповедите и съдбите, които Господ заповяда на израилтяните чрез Моисея на моавските полета при Иордан срещу Ерихон.

\end{document}