\begin{document}

\title{Joshua}


\chapter{1}

\par 1 Подир смъртта на Господния раб Моисея, Господ говори на Исуса Навиевия син, Моисеевия слуга, казвайки:
\par 2 Слугата ми Моисей умря; сега, прочее, стани та мини през тоя Иордан, ти и всички тия люде, в земята, която Аз давам на тях, на израилтяните.
\par 3 Всяко място, на което стъпи стъпалото на нозете ви, давам ви го, според както казах на Моисея.
\par 4 От пустинята и тоя Ливан до голямата река, реката Ефрат, цялата земя на хетейците, и до голямото море към захождането на слънцето, ще бъдат пределите ви.
\par 5 Никой не ще може да устои против тебе през всичките дни на живота ти; като бях с Моисея, така ще бъда и с тебе; няма да отстъпя то тебе, нито ще те оставя.
\par 6 Бъди силен и смел; защото ти ще разделиш в наследство на тия люде земята, за която съм се клел на бащите им, че ще им я дам.
\par 7 Само бъди силен и твърде дързостен, та да постъпваш внимателно според целия закон, за който слугата Ми Моисей ти даде заповед; не се отклонявай от него нито надясно нито наляво, за да имаш добър успех където и да идеш.
\par 8 Тая книга на закона да се не отдалечава от устата ти; но да размишляваш върху нея денем и нощем, за да постъпваш внимателно, според всичко каквото е написано в нея, защото тогава ще напредваш в пътя си, и тогава ще имаш добър успех.
\par 9 Ето, заповядвам ти: бъди силен и смел; да не се плашиш и да не се страхуваш; защото Господ твоят Бог е с тебе гдето и да идеш.
\par 10 Тогава Исус заповяда на началниците на людете, като каза:
\par 11 Преминете през сред стана и заповядайте на людете, казвайки: Пригответе си храна за път , защото след три дена ще минете през тоя Иордан, да за влезете да завладеете земята, която Господ вашият Бог ви даде да притежавате.
\par 12 А на рувимците на гадците и на половината от Манасиевото племе Исус говори, казвайки:
\par 13 Помнете думата, която Господният слуга Моисей ви заповяда, като каза: Господ вашият Бог ви успокои и ви даде тая земя.
\par 14 Жените ви, чадата ви и добитъкът ви нека останат в земята, която Моисей ви даде оттатък Иордан, а вие, всичките юнаци, да преминете пред братята си въоръжени и да ми помагате,
\par 15 догде Господ успокои и братята ви както вас, та притежават и те земята, която Господ вашият Бог им дава; тогава да се върнете в земята на наследството си, която Господният слуга Моисей ви даде оттатък Иордан към изгряването на слънцето, и да я притежавате.
\par 16 И те в отговор казаха на Исуса: Всичко що ни заповядаш ще извършим, и където и да ни пращаш ще идем.
\par 17 Както слушахме във всичко Моисея, така ще слушаме и тебе; само Господ твоят Бог да е с тебе; както беше с Моисея.
\par 18 Всеки, който би се възпротивил на твоите заповеди и не би послушал думите ти във всичко що му заповядаш, ще се умъртви само ти бъди силен и смел.

\chapter{2}

\par 1 Тогава Исус Навиевият син изпрати от Ситим двама мъже да съгледат тайно, и каза: Идете, разгледайте земята и Ерихон. И отидоха, и като влязоха в къщата на една блудница на име Раав, престояха там.
\par 2 Но известиха на ерихонския цар, казвайки: Ето, нощес дойдоха тук мъже от израилтяните, за да съгледат земята.
\par 3 И тъй, ерихонският цар прати в Раав да кажат: Изведи мъжете, които дойдоха при тебе, и които влязоха в къщата ти; защото са дошли да съгледат цялата земя.
\par 4 Но жената взе двамата мъже та ги скри; и рече: Наистина мъжете дойдоха при мене, но не знаех от где бяха;
\par 5 и като щяха да се затворят портите, на мръкване, мъжете излязоха. Не зная къде отидоха мъжете; тичайте скоро след тях и ще ги стигнете.
\par 6 Но тя ги беше качила на къщния покрив, и бе ги скрила в ленените гръсти, които беше наредила на къщния покрив.
\par 7 И изпратените мъже ги гониха по пътя, който отива за Иордан, гониха ги до бродовете; и щом тръгнаха ония, които ги подгониха, затвориха портата.
\par 8 А преди те да си легнат, тя се качи при тях на къщния покрив
\par 9 и рече на мъжете: Зная, че Господ ви даде тая земя и страх от вас ни нападна, и всичките жители на тая земя се стопиха пред вас;
\par 10 понеже чухме как Господ пресуши водата на Червеното море пред вас, когато излязохте то Египет, и какво направихте на двамата аморейски царе, които бяха оттатък Иордан, - на Сиона и на Ога, които изтребихте.
\par 11 Като чухме сърцата ни се стопиха, и в никого не остана вече душа поради вас; защото Господ вашият Бог, Той е Бог на небето горе и на земята долу.
\par 12 И сега, моля, закълнете ми се в Господа, че както аз показах милост към вас, ще покажете и вие милост към бащиния ми дом; и дайте ми знак за уверение,
\par 13 че ще запазите живота на баща ми, на майка ми, на братята ми и на сестрите ми, и всичко що имат, и ще избавите живота ни от смърт.
\par 14 Мъжете й рекоха: Вместо вас ние сме готови да умрем, ако само не издадете тая наша работа: и когато Господ ни даде земята, ще ти покажем милост и вярност.
\par 15 Тогава тя ги спусна с въже през прозореца; защото къщата й беше на градската стена, и на стената живееше.
\par 16 И рече им: Идете в гората, за да не ви срещнат гонителите и крийте се там три дена догде се върнат гонителите; и после си идете по пътя.
\par 17 И мъжете й рекоха: Ето как ще бъдем свободни от тая клетва, с която ти ни направи да се закълнем:
\par 18 ето, когато влезем в земята, да вържеш това въже от червена прежда на прозореца, през който ти ни спусна; а баща си и майка си братята си и целия си бащин дом да събереш при себе си в къщи.
\par 19 Всеки, който би излязъл из вратата на къщата ти кръвта му ще бъде на главата му, а ние ще бъдем невинни; на всеки, който остане с тебе в къщи, неговата кръв ще бъде на нашата глава, ако се издигне ръка против него.
\par 20 Обаче, ако издадеш тая наша работа, тогава пак ще бъдем свободни от клетвата, с която ти ни направи да се закълнем.
\par 21 А тя каза: Да бъде според както сте казали. И изпрати ги та си отидоха; и тя върза червеното въже на прозореца.
\par 22 И като тръгнаха отидоха в гората, и там останаха три дена догде се върнаха гонителите; а гонителите ги търсиха по целия път, но не ги намериха.
\par 23 Тогава двамата мъже се върнаха и като слязоха от гората, преминаха и дойдоха при Исуса Навиевия син; и казаха му всичко що се бе случило.
\par 24 И рекоха на Исуса: Наистина Господ предаде в ръцете ни цялата земя; а при това, всичките местни жители се стопиха пред нас.

\chapter{3}

\par 1 Тогава Исус стана рано; и като се дигна от Ситим, той и всичките израилтяни, дойдоха до Иордан, и там пренощуваха преди да преминат.
\par 2 А след три дена началниците преминаха през стана
\par 3 и заповядаха на людете като казаха: Когато видите ковчега за плочите на завета на Господа вашия Бог, и че левитските свещеници го носят, тогава и вие да се дигнете от местата си и да вървите след него.
\par 4 Но между вас и него да има около две хиляди лакътя разстояние по мярка; да се не приближите при него, за да знаете пътя, по който трябва да вървите; защото до сега не сте минавали през тоя път.
\par 5 Исус каза още на людете: Осветете се, защото утре Господ ще извърши всред вас чудни дела.
\par 6 Исус говори и на свещениците, казвайки: Дигнете ковчега за плочите на завета и заминете пред людете. Те, прочее, дигнаха ковчега на завета и вървяха пред людете.
\par 7 И Господ рече на Исуса: Днес почвам да те възвеличавам пред целия Израил, за да познаят, че както бях с Моисея, така ще бъда и с тебе.
\par 8 А ти заповядай на свещениците, които носят ковчега за плочите на завета, като кажеш: Когато стигнете при брега на Иорданската вода, застанете в Иордан.
\par 9 Тогава Исус каза на израилтяните: Приближете се тук и слушайте думите на Господа вашия Бог.
\par 10 И рече Исус: От това ще познаете, че живият Бог е всред вас, и че съвсем ще изгони отпред вас ханаанците, хетейците, евейците, ферезейците, гергесейците, евейците, аморейците и евусейците.
\par 11 Ето, ковчега на завета на Господа на целия свят върви пред вас в Иордан.
\par 12 Сега, прочее, изберете си дванадесет мъже от Израилевите племена, по един мъж от всяко племе.
\par 13 И щом стъпят във водата на Иордан стъпалата на нозете на свещениците, които носят ковчега на Иеова, Господа на целия свят, Иорданската вода, водата, която слиза от горе, ще се раздели и ще застане на куп.
\par 14 И тъй, щом се дигнаха людете от шатрите си, като бяха пред людете свещениците, които носеха ковчега на завета, за да преминат Иордан,
\par 15 и щом дойдоха до Иордан ония, които носеха ковчега, и нозете на свещениците, които носеха ковчега, се намокриха в края на водата, (защото Иордан наводнява всичките си брегове през цялото време на жетвата),
\par 16 водата, която слизаше от горе застана и се издигна на куп много надалеч, пред града Адам, който е край Царетан: а водата , която течеше надолу към полското море, то ест, Соленото море, се свърши и изчезна; и людете преминаха срещу Ерихон.
\par 17 А свещениците, които носеха ковчега на Господния завет, стояха твърди на сухо всред Иордан; и целият Израил преминаваха по сухо догдето целият народ съвсем премина Иордан.

\chapter{4}

\par 1 А когато целият народ съвсем беше преминал Иордан, Господ говори на Исуса, казвайки:
\par 2 Вземете си дванадесет човека от людете, по един човек от всяко племе,
\par 3 и заповядайте им кат кажете: Вземете си дванадесет камъни от тук, от сред Иордан, от мястото гдето нозете на свещениците стоеха твърди, и занесете ги със себе си и турете ги на мястото, гдето ще пренощувате тая нощ.
\par 4 Тогава Исус повика дванадесетте човека, които беше определил от израилтяните, по един човек от всяко племе;
\par 5 и Исус им каза: Заминете пред Ковчега на Господа вашия Бог всред Иордан и дигнете всеки от вас по един камък на рамената си, според числото на племената на израилтяните,
\par 6 за да бъде това белег между вас. И утре, когато чадата ви попитат, казвайки: Защо ви са тия камъни?
\par 7 - Тогава ще из кажете: Понеже водата на Иордан се раздели пред ковчега на Господния завет; когато ковчегът преминаваше Иордан, водата на Иордан се раздели. И тия камъни ще бъдат за спомен до века на израилтяните.
\par 8 Тогава израилтяните сториха така както заповяда Исус; взеха дванадесет камъни отсред Иордан, според както рече Господ на Исуса, според числото на племената на израилтяните; и донесоха ги със себе си на мястото гдето пренощуваха, и там го положиха.
\par 9 Още Исус постави дванадесет камъни сред Иордан, на мястото гдето стояха нозете на свещениците, които носеха ковчега на завета; и те са там до днес.
\par 10 А свещениците, които носеха ковчега, стояха всред Иордан догде се свърши всичко що Господ заповяда на Исуса да говори на людете, според всичко, което Моисей беше заповядал на Исуса. И людете побързаха та преминаха.
\par 11 А когато всичките люде бяха съвсем преминали, мина и Господният ковчег и свещениците пред людете.
\par 12 И рувимците, гадците и половината от Манасиевото племе преминаха въоръжени пред израилтяните, според както Моисея им бе казал;
\par 13 около четиридесет хиляди въоръжени ратници преминаха пред Господа на бой към ерихонските полета.
\par 14 В оня ден Господ възвеличи Исуса пред очите на целия Израил; и те се бояха от него, както се бояха от Моисея, през всичките дни на живота му.
\par 15 Тогава Господ говори на Исуса, казвайки:
\par 16 Заповядай на свещениците, които носят ковчега на свидетелството, да излязат от Иордан.
\par 17 И тъй, Исус заповяда на свещениците, като рече: Излезте от Иордан.
\par 18 А щом свещениците, които носеха ковчега на Господния завет, излязоха изсред Иордан, и стъпалата на свещеническите нозе стъпиха на сухо, водата на Иордан се върна на мястото си и преля всичките му брегове както по-напред.
\par 19 На десетия ден от първия месец людете излязоха от Иордан, и разположиха стан в Галгал, на източната страна от Ерихон.
\par 20 И Исус постави в Галгал ония дванадесет камъни, които взеха от Иордан.
\par 21 Тогава говори на израилтяните, казвайки: Утре, когато чадата ви попитат бащите си, думайки: Какви са тия камъни?
\par 22 ( Тогава да разправите на чадата си, като речете: По сухо премина Израил тоя Иордан;
\par 23 Защото Господ вашият Бог пресуши Иорданската вода пред вас догде преминахте, както Господ вашият Бог стори на Червеното море, което пресуши пред нас догде преминахме,
\par 24 за да знаят всички племена на света, че Господната ръка е мощна, та да се боят винаги от Господа вашия Бог.

\chapter{5}

\par 1 А когато всичките аморейски царе, които бяха оттатък Иордан на запад, и всичките ханаански царе, които бяха при морето, чуха, че Господ пресуши водата на Иордан пред израилтяните догде преминаха, сърцата им се стопиха, и душа не остана в тях поради израилтяните.
\par 2 В това време Господ рече на Исуса: Направи си кремъчни ножове и обрежи пак израилтяните, втори път.
\par 3 И тъй, Исус си направи кремъчни ножове и обряза израилтяните на мястото Хълм на краекожията.
\par 4 А ето причината, по която Исус извърши обрязването: всичките мъжки, които излязоха из Египет, всичките военни мъже, измряха в пустинята по пътя, след като бяха излезли из Египет;
\par 5 и всичките люде, които излязоха, бяха обрязани; а всичките люде, които се родиха в пустинята по пътя, след като бяха излезли из Египет, не бяха обрязани.
\par 6 Защото израилтяните ходиха четиридесет години по пустинята, догде се изтребиха всичките люде, излезли из Египет военни мъже, които не послушаха Господния глас, на които Господ се кле, че не ще ги остави да видят земята, за която Господ беше се клел на бащите им, че ще ни я даде, земя гдето текат мляко и мед.
\par 7 А вместо тях Той издигна синовете им, които Исус обряза; защото бяха необрязани, понеже не бяха ги обрязали по пътя.
\par 8 И като се обрязаха всичките люде, седяха на местата си в стана догде оздравяха.
\par 9 Тогава Господ рече на Исуса: Днес отнех от вас египетския позор. За това онова място се нарича Галгал, както се казва и до днес.
\par 10 А израилтяните разположиха стан в Галгал, и направиха пасхата на четиринадесетия ден от месеца, привечер, на ерихонските полета.
\par 11 И на сутринта на пасхата, в същия ден, ядоха безквасни хлябове от житото на земята и изпържено жито .
\par 12 А на сутринта, като ядоха от житото на земята, манната престана; и израилтяните нямаха вече манна, но през тая година ядяха от рожбите на Ханаанската земя.
\par 13 И когато беше Исус при Ерихон, подигна очи и видя, и, ето, насреща му стоеше човек с измъкнат нож в ръка; и Исус пристъпи при него и му рече: Наш ли си, или от неприятелите ни?
\par 14 А той рече: Не; но за Военачалник на Господното войнство сега дойдох аз. И Исус падна с лицето си на земята и се поклони; и рече му: Що заповяда Господарят ми на слугата си?
\par 15 А военачалникът на Господното войнство рече на Исуса: Изуй обущата си от нозете, защото мястото, на което стоиш, е свето. И Исус стори така.

\chapter{6}

\par 1 (А Ерихон беше заключен и затворен поради израилтяните; никой не излизаше и никой не влизаше).
\par 2 И Господ рече на Исуса: Ето, предадох в ръката ти Ерихон, царя му и силните му и храбри мъже.
\par 3 Ходете, прочее, около града, всички военни мъже, и обиколете града веднъж; така да правите шест дена.
\par 4 И седем свещеника нека носят пред ковчега седем гръмливи тръби; и нека на седмия ден обиколете града седем пъти, и свещениците нека свирят с тръбите.
\par 5 И когато засвирят продължително с гръмливата тръба, като чуете гласа на тръбата всички люде да извикат с гръмлив глас; и градската стена ще падне на мястото си, и людете нека вървят всеки право напред.
\par 6 Тогава Исус Навиевия син повика свещениците и рече им: Дигнете ковчега на завета, и седем свещеника нека държат седем гръмливи тръби пред Господния ковчег.
\par 7 Рече на людете: Минете та обиколете града; а въоръжените нека заминат пред Господния ковчег.
\par 8 И тъй, след като говори Исус на людете, седемте свещеника, които държаха седемте гръмливи тръби пред Господа минаха, и свиреха с тръбите; и ковчегът на Господния завет вървеше подире им.
\par 9 И въоръжените вървяха пред свещениците, които свиреха с тръбите, и задната стража следваше зад ковчега, докато свещениците свиреха с тръбите като вървяха.
\par 10 А Исус заповяда на людете, като каза: Да не викате, нито да се чуе глъсат ви, нито да излезе дума из устата ви, до деня, когато ще ви кажа да извикате; тогава извикайте.
\par 11 И тъй, направи да обиколи Господният ковчег около града еднъж; и дойдоха в стана и пренощуваха в стана.
\par 12 И на сутринта Исус стана рано, и свещениците дигнаха Господния ковчег,
\par 13 И седемте, свещеника, които държаха седемте гръмливи тръби., вървяха пред Господния ковчег, като ходеха и свиреха с тръбите; и пред тях и свиреха с тръбите; и пред тях вървяха въоръжените, а задната стража следваше подир Господния ковчег, докато свещениците свиреха с тръбите като вървяха.
\par 14 И тъй на втория ден обиколиха града еднъж и се върнаха в стана така правеха шест дена.
\par 15 А на седмия ден станаха рано, при пукването на зората, и обиколиха града седем пъти по същия начин; само тоя ден обиколиха града седем пъти.
\par 16 И на седмия път, като свиреха свещениците с тръбите, Исус рече на людете: Извикайте, защото Господ ви предаде града.
\par 17 И градът и всичко що е в него ще бъдат обречени на Господа; само блудницата Раав, да остане жива, тя и всички, които са в къщата с нея, защото скри пратениците, които проводихме.
\par 18 Но вие се пазете във всеки случай от всичко обречено, да не би, като го обречете, да вземете от обреченото, и така да нанесете проклетия върху стана на Израиля и да го смутите.
\par 19 Всичкото сребро и злато, и медните и железните съдове, са посветени Господу; да се внесат в Господното съкровище.
\par 20 И тъй, людете извикаха и свещениците свиреха с тръбите; и като чуха людете тръбния глас, и като издадоха людете гръмлив вик, стената падна на мястото си; и людете влязоха в града, всеки право напред, и превзеха града.
\par 21 И обрекоха на изтребление с острото на ножа всичките в града, мъже и жени, млади и стари, и говеда, овци и осли.
\par 22 Тогава Исус каза на двамата мъже, които бяха съгледали земята: Влезте в къщата на блудницата и изведете от там жената и всичко коквото има, според както й се заклехте.
\par 23 И тъй, като влязоха младежите, шпионите, изведоха Раав, баща й, майка й, братята й и всичките й роднини, и туриха ги вън от Израилевия стан.
\par 24 И града и всичко в него изгориха с огън; само среброто и златото и медните и железните съдове туриха в съкровището на Господния дом.
\par 25 А блудницата Раав и бащиното й семейство, и всичко що имаше тя, Исус остави живи и тя живее всред Израиля до днес, защото скри пратениците, които Исус проводи да съгледат Ерихон.
\par 26 В онова време Исус се закле, като каза: Проклет пред Господа оня човек, който стане да съгради тоя град Ерихон; със смъртта на първородния си син ще тури основите му, и със смъртта на най-младия си син ще постави портите му.
\par 27 Така Господ беше с Исуса; и името му се прочу по цялата земя.

\chapter{7}

\par 1 А израилтяните извършиха престъпление относно обреченото: защото Ахан син на Хармия, син на Завдия, син на Зара, от Юдовото племе, взе от обреченото, така че Господният гняв пламна против израилтяните.
\par 2 Между това Исус прати човеците от Ерихон в Гай, който е близо при Витавен, на източната страна от Ветил; и говори им, казвайки: Изкачете се та съгледайте земята. И тъй, човеците се изкачиха та съгледаха Гай.
\par 3 И като се върнаха при Исуса рекоха му: Да не се изкачат всичките люде, но до две или три хиляди мъже да се изкачат и да поразят Гай; да не вкарват в труд всичките люде да отиват там, защото ония са малцина.
\par 4 И така, изкачиха се там от людете около три хиляди мъже; но побягнаха пред гайските мъже.
\par 5 Гайските мъже поразиха от тях около тридесет и шест мъже, и прогониха ги от портата до Сиварим, и поразиха ги в надолнището; поради което сърцата на людете се стопиха и станаха като вода.
\par 6 Тогава Исус раздра дрехите си и падна на лицето си на земята пред Господния ковчег, гдето лежа до вечерта, той и Израилевите старейшини; и туриха пръст на главите си.
\par 7 И рече Исус: Аз! Господи Иеова, защо преведе тия люде през Иордан, за да ни предадеш в ръцете на аморейците да ни погубят? О да бяхме били доволни да си седим оттатък Иордан!
\par 8 О Господи, що да река като Израил обърна гръб пред неприятеля си?
\par 9 И като чуят ханаанците и всичките други жители на земята, ще ни обиколят и ще на правят да изчезне името ни от земята; и какво ще сториш за великото Си Име?
\par 10 А Господ рече на Исуса: Стани, защо си паднал така на лицето си?
\par 11 Израил е съгрешил, а именно, престъпили са завета Ми, за който им дадох заповед; да! те са взели от обреченото, още са откраднали, още са излъгали, още са турили откраднатото между своите вещи.
\par 12 По тая причина израилтяните не могат да устоят пред неприятелите си, но обръщат гръб пред неприятелите си, защото станаха проклети; и Аз не ще бъда вече с вас, ако не изтребите проклетия човек изсред вас.
\par 13 Стани та освети людете, и речи: Осветете се за утре; защото така говори Господ Израилевият Бог: Нещо обречено има всред тебе, Израилю; не можеш да устоиш пред неприятелите си догде не махнете обреченото изсред вас.
\par 14 И тъй, приближете се утре според племената си; и племето, което хване Господ, да се приближи според семействата си; и семейството, което хване Господа, да се приближи според домовете си; а домът, който хване Господ, да се приближи според мъжете.
\par 15 И който се хване с обреченото ще се изгори с огън, той и всичко що има, понеже е престъпил Господния завет, и понеже е сторил безумие в Израиля.
\par 16 Тогава Исус, като стана рано на сутринта, приведе Израиля според племената им; и хвана се Юдовото племе.
\par 17 И като приведе Юдовите семейства, хвана се семейството на Заровците. И като приведе семейството на Заровците, мъж по мъж, хвана се Завдий.
\par 18 А като приведе неговия дом, мъж по мъж, хвана се Ахан син на Хармия, сина на Завдия, сина на Зара, от Юдовото племе.
\par 19 Тогава Исус рече на Ахана: Чадо мое, въздай сега слава на Господа Израилевия Бог и изповядай Му се; и кажи ми що си сторил, не скривай от мене.
\par 20 И Ахан в отговор на Исуса каза: Наистина аз съгреши на Господа Израилевия Бог, като сторих така и така:
\par 21 Когато видях между користите една хубава вавилонска дреха двеста сикли сребро, и една златна плочка тежка петдесет сикли, пожелах ги и ги взех; и, ето, те са скрити в земята всред шатъра ми, и среброто отдолу.
\par 22 Исус, прочее, прати човеци, които се завтекоха в шатъра; и ето, откраднатото беше скрито в шатъра му, и среброто отдолу.
\par 23 И взеха ги отсред шатъра и донесоха ги при Исуса и при всичките израилтяни; и те ги положиха пред Господа.
\par 24 Тогава Исус и целият Израил с него взеха Ахана Заровия син, със среброто, дрехата и златната плоча, и синовете му, дъщерите му, говедата му, ослите му, овците му, шатъра му и всичко каквото имаше, и заведоха ги в долината Ахор.
\par 25 И рече Исус: Защо си ни смутил? Господ ще смути тебе днес. Тогава целият Израил уби Ахана с камъни, и изгориха го с огън след като го убиха с камъни.
\par 26 После натрупаха на него голяма грамада камъни, която стои и до днес. Така се върна Господ от яростния Си гняв. Затуй онова място се нарича и до днес долина Ахор.

\chapter{8}

\par 1 Подир това Господ рече на Исуса: Не бой се нито се ужасявай; вземи със себе си всичките военни мъже, и стани та се изкачи в Гай; ето, Аз предадох в ръката ти гайския цар, людете му, града му и земята му;
\par 2 и да сториш на Гай и на царя му, както стори на Ерихон и на царя му; само че ще заплените за себе си користите му и добитъка му. Постави против града засада от задната му страна.
\par 3 Тогава Исус стана с всичките военни люде, за да отидат в Гай; и Исус избра тридесет хиляди мъже силни и храбри изпрати ги нощем,
\par 4 и заповяда им, казвайки: Гледайте, поставете засада против града от задната страна на града; да се не отдалечите много от града, и да бъдете всички готови;
\par 5 а аз и всичките люде, които са с мене, ще се приближим до града; и когато излязат против нас, както изпърво, тогава ние ще побегнем от тях.
\par 6 Те ще излязат след нас догде ги отдалечим от града; защото ще си рекат: Те бягат от нас, както по-напред; и ние ще побегнем от тях.
\par 7 Тогава вие станете от засадата и превземете града; защото Господ вашият Бог ще го предаде в ръката ви.
\par 8 И като превземете града да запалите града, да сторите според Господното повеление; ето, заповядах ви.
\par 9 И тъй, Исус ги изпрати; и те отидоха в засада, като седнаха между Ветил и Гай, на западната страна от Гай; а Исус остана през оная нощ всред людете.
\par 10 И Исус, като стана рано на сутринта и прегледа людете, отиде, той и Израилевите старейшини, пред людете за Гай.
\par 11 И всичките военни люде, които бяха с него, отидоха и се приближиха, дойдоха срещу града, и разположиха стан на северната страна от Гай; а между тях и Гай имаше долина.
\par 12 И той взе около пет хиляди мъже и ги постави в засада между Ветил и Гай на западната страна от града.
\par 13 И като наредиха людете - цялото войнство, което беше на север от града, и засадата му на запад от града - Исус отиде през оная нощ всред долината.
\par 14 А гайският цар, като видя това, побърза, той и всичките му люде, мъжете на града, та станаха рано, и в определен час излязоха на полето на бой против Израиля; но той не знаеше, че имаше засада против него зад града.
\par 15 А Исус и целият Израил се престориха на разбити пред тях и бягаха през пътя за пустинята.
\par 16 И всичките люде, които бяха в Гай, бяха свикани, за да го гонят; и гониха Исуса, и отдалечиха се от града.
\par 17 Така в Гай и Ветил не остана човек, който не излезе след Израиля; и те оставиха града отворен и гониха Израиля.
\par 18 Тогава Господ каза на Исуса: Простри към Гай копието, което държиш в ръката си; защото ще предам града в ръката ти. Исус, прочее, простря към града копието, което държеше в ръката си.
\par 19 И щом простря ръката си засадата стана бързо от мястото си, завтекоха се изведнъж и влязоха в града и го превзеха; и побързаха та запалиха града.
\par 20 А когато гайските мъже се огледаха надире, видяха, и, ето че дим се издигаше от града към небето; и нямаха где да бягат ни тук ни там, понеже людете, които бягаха към пустинята, се обърнаха назад против гонителите.
\par 21 А Исус и целият Израил, като видяха, че засадата беше превзела града, и че се издигаше дим от града, обърнаха се назад и поразиха гайските мъже.
\par 22 Другите също излязоха из града против тях; и тъй, те се намериха всред израилтяните, които бяха едни отсам а едни оттам; и тия ги поразиха така щото не оставиха никой от тях да живее или да побегне.
\par 23 А царя на Гай хванаха жив, о го доведоха при Исуса.
\par 24 И като изби Израил всичките жители на Гай в полето и пустинята, гдето ги гониха, и те всички паднаха от острото на ножа догде се изтребиха, тогава целият Израил се върна в Гай и поразиха го с острото на ножа.
\par 25 И всичките паднали в оня ден, мъже и жени, бяха дванадесет хиляди души, всичките гайски люде.
\par 26 Защото Исус не оттегли ръката си, с който бе прострял копието, догде не изтреби като обречени, всичките жители на Гай.
\par 27 Само че Израил плени за себе си добитък и користите на оня град, според думата, която Господ заповяда на Исуса.
\par 28 И така, Исус изгори Гай и го направи всегдашен куп развалини, както е до днес.
\par 29 А царя на Гай обеси на дърво и го остави да виси до вечерта; и като зайде слънцето, Исус заповяда та снеха трупа от дървото, хвърлиха го във входа на градската порта, и натрупаха на него голяма грамада камъни, която стои и до днес.
\par 30 Тогава Исус издигна олтар на Господа Израилевия Бог на хълма Гевал,
\par 31 както Господният слуга Моисей беше заповядал на израилтяните, според написаното в книгата на Моисеевия закон, олтар от цели камъни, по които желязно сечиво не се бе издигнало; и принесоха на него всеизгаряния Господу, и пожертвуваха примирителни приноси.
\par 32 И там написа на камъните един препис на Моисеевия закон, който написа пред израилтяните.
\par 33 И целият Израил, със старейшините им, първенците и съдиите им, чужденци и туземци, застанаха отсам и оттам ковчега, срещу левитските свещеници, които държаха ковчега на Господния завет, - едната им половина към хълма Гаризин, и другата им половина към хълма Гевал, - както Господният слуга Моисей беше заповядал по-напред, за да благославят Израилевите люде.
\par 34 И след това прочее всичките думи на закона, благословията и проклетиите, според всичко каквото бе писано в книгата на закона.
\par 35 От всичко що заповяда Моисей нямаше дума, която Исус не прочете пред всичките събрани израилтяни, с жените, децата и чужденците, които се събираха между тях.

\chapter{9}

\par 1 А когато чуха това всичките царе оттатък Иордан, в хълмистите места и в полетата и във всичките крайбрежия на голямото море до срещу Ливан сиреч , хетейците, аморейците, ханаанците, ферезейците, евейците и евусейците,
\par 2 събраха се единодушно да се бият с Исуса и Израиля.
\par 3 А жителите на Гаваон, като чуха що бе сторил Исус на Ерихон и Гай,
\par 4 постъпиха хитро, като отидоха та се престориха на посланици и взеха стари вретища та ги туриха на ослите си и стари и раздрани и вързани мехове за вино,
\par 5 и стари и кърпени обуща на нозете си, и стари дрехи на сабе си, и всичкият хляб, който си доставиха за храна, беше сух и плесенясал.
\par 6 Та дойдоха при Исуса в стана в Галгал, та рекоха нему и на Израилевите мъже: От далечна страна сме дошли; сега, прочее, направете договор с нас.
\par 7 Но Израилевите мъже рекоха на евейците: Да не би вие да живеете всред нас; и как ще направим договор с вас?
\par 8 А те казаха на Исуса: Твои слуги сме. И Исус им рече: Кои сте? и от где идете?
\par 9 А те му казаха: От много далечна земя дойдоха слугите ви поради името на Иеова твоя Бог; защото чухме славата Му и всичко що извършил в Египет,
\par 10 и всичко що сторил на двамата аморейски царе, които бяха оттатък Иордан, на есевонския цар Сион и на васанския цар Ог, който живееше в Астарот.
\par 11 За това ни говориха старейшините ни и всичките жители на нашата земя, казвайки: Вземете в ръката си храна за из пътя та идете да ги посрещнете, и речете им: Ваши слуги сме; и тъй, сега направете договор с нас.
\par 12 Този наш хляб беше топъл, когато си го доставихме от къщите си за храна в деня, когато излязохме, за да дойдем при вас; а, ето, сега е сух и плесенясал.
\par 13 Тия мехове за вино бяха нови, като ги напълнихме, а ето, съдрани са; и тия наши дрехи и обущата ни овехтяха от много дългия ни път.
\par 14 Тогава израилтяните приеха мъжете по причина на храната им; а до Господа не се допитаха.
\par 15 Исус сключи мир с тях, и свърза договор с тях, че ще се опази живота им: и началниците на обществото им се заклеха.
\par 16 А три дена след като свързаха договор с тях чуха, че им били съседи и че живеели между тях;
\par 17 и като пътуваха израилтяните, пристигнаха в градовете им на третия ден. А градовете им бяга Гаваон, Хефира, Вирот и Кириатиарим.
\par 18 Но израилтяните не ги поразиха, защото началниците на обществото бяха им се заклели в Господа Израилевия Бог. А цялото общество роптаеше против началниците.
\par 19 Всичките началници, обаче, рекоха на цялото общество: Ние им се заклехме в Господа Израилевия Бог; и тъй, сега не можем да се допрем до тях.
\par 20 Ето какво ще сторим с тях: ще запазим живота им, за да не бъде Божият гняв на нас поради клетвата, с който им се заклехме.
\par 21 Началниците им казаха още: нека останат живи; но да бъдат дървосечци и водоносци на цялото общество, - според както началниците бяха им говорили.
\par 22 Тогава Исус ги свика та им говори, казвайки: Защо ни излъгахте като рекохте: Много далеч сме от вас, - когато вие сте живеели между нас?
\par 23 Сега, прочее, проклети сте; и никога няма да липсва от вас слуги, които да бъдат и дървосечци и водоносци за дома на моя Бог.
\par 24 А те в отговор на Исуса казаха: Понеже слугите бяха добре предизвестени за онова, което Иеова твоят Бог заповядал на слугата Си Моисей, да ви даде цялата земя, и да изтреби от пред вас всичките жители на земята, за това много се уплашихме от вас за живота си, и сторихме това нещо.
\par 25 Сега, ето, в ръцете ти сме; каквото ти се вижда добро и право да ни сториш, стори.
\par 26 Той, прочее, стори с тях така, и ги избави от ръката на израилтяните, та не ги убиха.
\par 27 И в същия ден Исус ги направи дървосечци и водоносци на обществото и на олтара на Господа в мястото, което Той би избрал, както са и до днес.

\chapter{10}

\par 1 А като чу ерусалимският цар Адониседек, че Исус превзел Гай и го обрекъл на изтребление, и че както сторил на Ерихон и на царя му така сторил на Гай и на царя му, и че жителите на Гаваон сключили мир с Израиля и останали между тях,
\par 2 уплашиха се много, той и людете му , защото Гаваон беше голям град, като един от царските градове, и защото беше по-голям от Гай, и всичките му мъже бяха силни.
\par 3 Затова, ерусалимският цар Адониседек прати до хевронския цар Оам, до ярмутския цар Пирам, до лахийския цар Яфий и до еглонския цар Девир, да рекат:
\par 4 Дойдете при мене та ми помогнете, и нека поразим Гаваон, защото сключи мир с Исуса и с израилтяните.
\par 5 И така, събраха се тия петима аморейски царе: ерусалимският цар, хевронският цар, ярмутския цар, лахийският цар и еглонският цар, та отидоха, те и всичките им войнства, и разположиха стана си пред Гаваон и воюваха против него.
\par 6 Тогава гаваонците пратиха до Исус в стана у Галгал да рекат: Да не оттеглиш ръка от слугите си; скоро дойди пре нас и избави ни и помогни ни, защото се събраха против нас всичките аморейски царе, които живеят в хълмистите места.
\par 7 И тъй, Исус се качи от Галгал, той и всичките военни люде с него, и всичките силни и храбри мъже.
\par 8 И Господ каза на Исуса: Да се не убоиш от тях, защото ги предадох в ръката ти; никой от тях няма да устои пред тебе.
\par 9 Исус, прочее, дойде върху тях внезапно, като беше се качвал цяла нощ от Галгал.
\par 10 И Господ ги смути пред Израиля; и Исус ги порази с голямо поражение в Гаваон, и гони ги из нагорнището, по което се отива за Веторон, и поразяваше ги до Азика и до Мекида.
\par 11 А като бягаха от Израиля и бяха в надолнище то при Веторон, Господ хвърляше на тях големи камъни от небето до Азика, та измряха; умрелите от камъните на градушката бяха по-много от ония, които израилтяните убиха с нож.
\par 12 Тогава говори Исус Господу, в деня когато Господ предаде аморейците на израилтяните, като рече пред очите на Израиля: Застани слънце,над Гаваон, И ти, луно, над долината Еалон.
\par 13 И слънцето застана и луната и спря, Догдето мъздовъздадоха людете на неприятелите си. Това не е ли записано в Книгата на Праведния? Слънцето застана всред небето, и не побърза да дойде почти цял ден.
\par 14 Такъв ден не е имало ни по-напред ни после, щото така да послуша Господ човешки глас; защото Господ воюваше за Израиля,
\par 15 След това Исус се върна, и целият Израил с него, в стана у Галгал.
\par 16 А ония петима царе побягнаха та се скриха в пещерата при Макида.
\par 17 Известиха, прочее, на Исуса, казвайки: Петимата царе се намериха скрити в пещерата при Макида.
\par 18 И Исус каза: Привалете големи камъни на входа на пещерата и поставете часови при нея да ги пазят;
\par 19 а вие не стойте; гонете неприятелите си и поразете най-последните от тях; не ги оставяйте да влязат в градовете си, защото Господ вашият Бог ги предаде в ръцете ви.
\par 20 И когато Исус и израилтяните ги поразиха с твърде голямо клане, догдето бяха изтребени, и останалите от тях, които оцеляха, влязоха в укрепени градове,
\par 21 тогава всичките люде се върнаха с мир в стана при Исуса у Мекида; никой не поклати език против никого от израилтяните.
\par 22 Тогава рече Исус: Отворете входа на пещерата та изведете пи мене ония петима царе и пещерата.
\par 23 И сториха така, и изведоха при него ония петима царе из пещерата: ерусалимския цар, хевронския цар, ярмутския цар, лахиския цар и еглонския цар.
\par 24 И като изведоха при Исуса ония царе, Исус повика всичките Израилеви мъже и рече на началниците на военните мъже, които бяха ходили с него: Приближете се, турете нозете си на вратовете на тия царе, и те се приближиха и туриха нозете си на вратовете им.
\par 25 Исус още им каза: Не бойте се, нито се страхувайте, бъдете силни и дръзновени, понеже така ще стори Господ на всичките ви неприятели, против които воювате.
\par 26 А подир това Исус ги порази, уби ги и ги обеси на пет дървета; и висяха на дърветата до вечерта.
\par 27 А при захождането на слънцето Исус заповяда, та ги снеха от дърветата, хвърлиха ги във входа на пещерата туриха големи камъни, които стоят там и до днес.
\par 28 В същия ден Исус превзе Макида и порази нея и царя й с острото на ножа; изтреби като обречени, тях и колкото души имаше в нея, не остави никого да избяга; стори на макидския цар както бе сторил на ерихонския цар.
\par 29 След това Исус и целият Израил се него премина от Макида в Ливан и воюваше против Ливна.
\par 30 И Господ предаде и нея и царя й в ръката на Израиля; и порази с острото на ножа си нея и колкото души имаше в нея; не остави никого да избяга и стори на царя й както бе сторил на ерихонския цар.
\par 31 После Исус и целият Израил с него премина от Лина в Лахис, разположи стан срещу него, и воюваше против него.
\par 32 И Господ предаде Лахис в ръката на Израиля; и превзеха го на втория ден, и поразиха с острото на ножа него и колкото души имаше в него, според всичко що сториха на Ливна.
\par 33 Тогава гезерският цар Орам дойде да помогне на Лахис; но Исус поразяваше него и людете му, догдето не му остави остатък.
\par 34 И от Лахис, Исус и целият Израил с него, премина в Еглон, разположиха стан срещу него, и воюваха против него;
\par 35 и в същия ден го превзеха и поразиха го с острото на ножа; в същия ден Исус изтреби, като обречени, колкото души имаше в него, според всичко, що стори в Лахис.
\par 36 После Исус и целият Израил с него отиде от Еглон в Хеврон, и воюваха против него;
\par 37 и като го превзеха, поразиха с острото на ножа него, царя му, всичките му градове и колкото души имаше в него; не остави никого да избяга, според всичко, що стори на Еглон, но изтреби него и колкото души имаше в него.
\par 38 А според това Исус и целият Израил с него се върна в Девир и воюва против него;
\par 39 и превзе него, царя му и всичките му градове; и поразиха ги с острото на ножа, и изтребиха колкото души имаше в него; не остави никого да избяга; стори на Девир и на царя му както стори на Хеврон, и както бе сторил на Ливна и на царя й.
\par 40 Така Исус порази цялата хълмиста, южна и полянска земя, и подгорията, и всичките из царе; не остави никого да избяга, но изтреби всичко, що дишаше, според както Господ Израилевият Бог беше заповядал.
\par 41 И Исус ги порази от Кадис-варни до Газа, и цялата Гесенска земя до Гаваон.
\par 42 Всички тия царе и земята им Исус превзе изведнъж, защото Господ Израилевият Бог воюваше за Израиля.
\par 43 Тогава Исус и целият Израил с него се завърна в стана у Галгал.

\chapter{11}

\par 1 А асирийският цар Явин, като чу това, прати до мадонския цар Иовава, до симвронския цар, до ахсафския цар
\par 2 и до царете, които бяха на север в хълмистата и в полянската земя на юг от Хинерот, и в долината, и в Нафат-дор на запад,
\par 3 и до ханаанците, които бяха на изток и на запад, и до аморейците, хетейците, ферезейците и евусейците, под Ермон в земята Масфа.
\par 4 И те, и всичките им войнства с тях, люде много, по множество както е пясъкът, който е покрай морето, излязоха с твърде много коне и колесници.
\par 5 И като се събраха всички тия царе, дойдоха и заедно разположиха стана си близо при водата Мером, за да се бият с Израиля.
\par 6 Тогава Господ каза на Исуса: Да се не убоиш от тях; защото утре, около тоя мас, Аз ще ги предам всички избити пред Израиля; и ще прережеш жилите на конете им, и ще изгориш с огън колесниците им.
\par 7 Исус, прочее, и всичките военни люде с него, отидоха внезапно против тях при водата Мером, и ги нападнаха.
\par 8 И Господ ги предаде в ръката на Израиля; и те ги поразиха, и гониха ги до големия Сидон и до Мисрефот-маим, и до долината на Масфа на изток, и поразяваха ги догдето не оставиха от тях остатък.
\par 9 Исус из стори, според както Господ му заповяда: преряза жилите на конете им, и изгори с огън колесниците из.
\par 10 В същото време Исус, като се върна, превзе Асор и уби царя им с нож; защото по-напред Асор държеше първенство между всички тия царства.
\par 11 И колкото души имаше в него, убиха с острото на ножа и ги изтребиха като обречени; не остана нищо, което дишаше; и изгори Асор с огън.
\par 12 Исус превзе и всичките градове на ония царе, и всичките им царе хвана и ги уби с острото на ножа; изтреби ги според както бе заповядал Господният слуга Моисей.
\par 13 Но Израил не изгори никой от градовете, които стояха върху хълмовете си, освен един Асор; него Исус изгори.
\par 14 Всичките користи на тия градове и добитъкът израилтяните плениха за себе си, а всичките човеци убиваха с острото на ножа, догдето ги изтребиха; не оставиха нищо, което дишаше.
\par 15 Както Господ беше заповядал на слугата Си Моисея, така заповяда Моисей на Исуса, така и стори Исус; не престъпи нищо от всичко, което Господ бе заповядал на Моисея.
\par 16 Така Исус превзе цялата оная хълмиста и цялата южна земя, и цялата Гесанска земя, равнината и полето, хълмистата и равна земя на Израиля,
\par 17 от нагорнището Халак, което отива към Сир, до Ваалгад в Ливанската долина под планината Ермон; и хвана всичките им царе, порази ги и ги уби.
\par 18 Дълго време воюваше Исус против всички тия царе.
\par 19 Нямаше град, който сключи мир с израилтяните, освен евейците, които живееха в Гаваон; те превзеха всичките с бой.
\par 20 Защото от Господа стана да се закоравят сърцата им та да излязат против Израиля на бой, за да го обрече Исус на изтребление, за да не им се покаже милост, но за да ги изтреби, според както Господ беше заповядал на Моисея.
\par 21 В онова време дойде Исус и изтреби енакимите от планинската земя, от Хеврон, от Девир, от Анав, и от цялата Израилева хълмиста земя; Исус ги обрече на изтребление заедно с градовете им.
\par 22 Не оставиха енакими в земята на израилтяните; някои от тях останаха само в Газа, Гет и Азот.
\par 23 Така Исус превзе цялата земя, според всичко що Господ беше казал на Моисея; Исус я даде на Израиля за наледство според както бяха разделени по племената си. И земята утихна от война.

\chapter{12}

\par 1 А ето царете на земята оттатък Иордан, към изгрева на слънцето, които израилтяните поразиха и чиято земя превзеха, от потока Арнон до Ермонската планина, и цялото поле на изток:
\par 2 аморейският цар Сион, който живееше в Есевон и владееше от Ароир, който е при брега на потока Арнон, и е в средата на долината, и владееше половината на Галаад до потока Явок, който е граница на амонците,
\par 3 и на изток, дори до езерото Хинерот и до полското море, сиреч , Соленото море, на изток, до пътя към Вет-иесимот, и на южната страна под Асдот-фасга:
\par 4 и пределите на васанския цар Ог, който беше от останалите исполини и живееше в Астарот и в Едраи,
\par 5 който владееше в Ермонската планина, в Салха и в целия Васан, до границата на гесурците и на мааханците, и половината от Галаад до границата на есевонския цар Сион.
\par 6 Тях поразиха Господния слуга Моисей и израилтяните; и Господният слуга Моисей даде земята им за притежание на рувимците, на гадците и на половината от Манасиевото племе.
\par 7 И ето царят на земята оттатък Иордан, на запад, който Исус и израилтяните поразиха, от Ваалгад в Ливанката долина до нагорнището Халак, което отива към Сиир; и Исус даде земята им за притежание на Израилевите племена, според както щяха да бъдат разделени:
\par 8 в хълмистата земя, в равнините, в полето, в подгорията, в пустинята и в южната страна. Царете бяха от хетейците, аморейците, ханаанците, ферезейците, евейците и евусейците, а именно :
\par 9 ерихонският цар, един; царят на Гай, град близо до Ветил, един;
\par 10 ерусалимският цар, един; хевронският цар, един;
\par 11 ярмутският цар, един; лахийският цар, един;
\par 12 еглонският цар, един; гезерският цар, един;
\par 13 девирският цар, един; гедерският цар, един;
\par 14 хорманският цар, един; арадският цар, един;
\par 15 ливанският цар, един; одоламският цар, един;
\par 16 макиданският цар, един; ветилският цар, един;
\par 17 тапфуанският цар, един; еферският цар, един;
\par 18 афекският цар, един; ласаронският цар, един;
\par 19 мадонският цар, един; асорският цар, един;
\par 20 симрон-меронският цар, един; ахсафският цар, един;
\par 21 таанахският цар, един; магедонският цар, един;
\par 22 кедеският цар, един; царят на Иокнеам в Кармил, един;
\par 23 царят на Дор в Нафатдор, един; царят на Гоим в Галгал, един;
\par 24 тирзанският цар, един; всичките царе тридесет и един.

\chapter{13}

\par 1 А Исус беше стар напреднал на възраст; и Господ му рече: Ти си стар, напреднал на възраст, а остава още много земя да се превзема.
\par 2 Ето земята, която остава още: всичките околности на филистимците и цялата Гесурска земя ,
\par 3 от Сиор, който е към Египет, до границата на Акарон на север, която се счита на ханаанците, - петте околии на филистимците: на газяните, на азотците, на аскалонците, на гетците и на акаронците; и земята , която е на авейците;
\par 4 на южната страна, цялата земя на ханаанците, и Меара, която е на сидонците, до Афек, до аморейските граници;
\par 5 и земята на гевалците, и целият Ливан към изгрева на слънцето, от Ваалгад под Ермонската планина до прохода на Емат;
\par 6 всичките планински жители от Ливан до Мисрефот-маим, то ест , всичките сидонци; тях Аз ще изтребя отпред израилтяните; а ти раздели тая земя в наследство на Израиля, според както съм ти заповядал.
\par 7 Сега, прочее, раздели тая земя в наследство на деветте племена и на половината от Манасиевото племе.
\par 8 А заедно с другата му половина рувимците и гадците взеха наследството си, което Моисея из даде оттатък Иордан, на изток, както Господният слуга Моисей им даде,
\par 9 от Ароир, който е при брега на потока Арнон, и града, който е в средата на долината, и цялата медеванска равнина до Девон,
\par 10 и всичките градове на аморейския цар Сион, който царуваше в Есевон до границите на амонците,
\par 11 и Галаад, и пределите на гесурците й на мааханците, и цялата Ермонска планина, и целия Васан до Салха,
\par 12 цялото царство на Ога във Васан, който царуваше в Астарот и в Едраи, и който бе оцелял от останалите от исполините; защото тях Моисей порази и ги изгони.
\par 13 Обаче, израилтяните не изгониха гесурците и мааханците, та гесурците и мааханците живееха между Израиля, както живеят и до днес.
\par 14 Само на Левиевото племе той не даде наследство: жертвите, принесени чрез огън на Господа Израилевия Бог са тяхно наследство, според както им е казал.
\par 15 А Моисей даде на племето на рувимците наследство според семействата им.
\par 16 Пределите им бяха от Ароир, който е при брега на потока Арнон и града, който е всред долината, и цялата равнина до Медева,
\par 17 Есевон и всичките му градове в равнината: Девон, Вамотваал, Вет-ваалмеон,
\par 18 Яса, Кедимот, Мафаат,
\par 19 Кириатаим, Сивма, Зарет-Саар, в хълма на Емак,
\par 20 Вет-фегор, Асдот-фасга, Вет-есимот,
\par 21 и всичките градове на равнината, и цялото царство на аморейския цар Сион, който царуваше в Есевон, когато Моисей порази заедно с мадиамските първенци: Евия, Рекема, Сура, Ура, и Рева, Сионови първенци, които живееха в оная земя.
\par 22 И между убитите от тях, израилтяните убиха с нож Валаама Веоровия син, чародееца.
\par 23 А границата на рувимците беше Иордан и пределите му. Това е наследството на рувимците според семействата им, с градовете му и техните села.
\par 24 И на Гадовото племе, Моисей даде наследство на гадците, според семействата им.
\par 25 Пределът им беше Язир и всичките Галаадски градове, и половината от земята на амонците до Ароир срещу Рава,
\par 26 и от Есевон до Рамат-масфа и Ветоним, о от Маханаим, до границата на Девир,
\par 27 и, в долината, Ветарам, Вет-нимра, Сокхот и Сафон, останалото от царството на есевонския цар Сион оттатък Иордан на изток, имайки за граница Иордан до края на езерото Хинерот.
\par 28 Това е наследството на гадците според семействата им, с градовете му и техните села.
\par 29 Моисей даде наследство и на половината от Манасиевото племе; и то стана притежание на половината от племето на манасийците според семействата им.
\par 30 Пределът им беше от Маханаим, целия Васан, цялото царство на васанския цар Ог, и всичките зеселища на Яир, които са у Васан, шестдесет града.
\par 31 И половината от Галаад, и Астарот и Едраи, градове на Оговото царство във Васан, са дадоха на потомците, на Махира Манасиевия син, на половината от Махировите потомци според семействата им.
\par 32 Тия са наследствата, които Моисей раздели в моавските полета оттатък Иордан, срещу Ерихон, на изток.
\par 33 А на Левиевото племе Моисей не даде наследство; Господ Израилевият Бог беше наследството им, според както и беше казал.

\chapter{14}

\par 1 И ето наследствата, които израилтяните взеха в Ханаанската земя, които им разделиха свещеникът Елеазар и Исус Навиевият син, и началниците на бащините домове от племената на израилтяните.
\par 2 С жребие се раздели наследството на тия девет и половина племена, според както Господ беше заповядал на Моисея.
\par 3 Защото Моисей беше дал наследството на двете и половина племена оттатък Иордан; но на левитите на деде наследство между тях.
\par 4 Защото Иосифовите потомци бяха две племена: Манасиевото и Ефремовото; но на левитите не даде дял в тая земя, а само градове за живеене, с пасбищата за добитъка им и за стоката им.
\par 5 Според както бе заповядал Господ на Моисея, така сториха израилтяните, като разделиха земята.
\par 6 В това време, като се приближиха юдейците при Исуса в Галгал, Халев, сина на Ефония Кенезов, му каза: Ти знаеш какво е говорил Господ на Божия човек Моисей за мене и за тебе в Кадис-варни.
\par 7 Аз бях на четиридесет години, когато Господният слуга Моисей ме прати от Кадис-варни, за да разгледам земята; и донесох му известие според както ми беше на сърце.
\par 8 Братята ми, обаче, които отидоха с мене, обезсърчиха людете; но аз последвах напълно Господа моя Бог.
\par 9 За това, в същия ден Моисей се закле, казвайки: Земята, на която са стъпили нозете ти, непременно ще бъде наследство на тебе и на потомците ти за винаги, защото ти напълно последва Господа моя Бог.
\par 10 И сега, ето, Господ ме е опазил жив, както каза, тия четиридесет и пет години, от времето когато Господ говори това на Моисея, когато Израил ходеше по пустинята; и сега, ето, днес аз съм на осемдесет и пет години.
\par 11 Днес аз съм тъй силен, както в деня, когато ме прати Моисей; както беше тогава силата ми, така е и сега силата ми да воювам и да излизам и да влизам.
\par 12 Сега, прочее, дай ми тая поляна, за която Господ говори в оня ден; защото ти чу в оня ден, че имало там енакими и големи укрепени градове; та дано бъде Господ с мене, и аз ще ги изгоня, според както Господ каза.
\par 13 Тогава Исус го благослови; и даде Хеврон за наследство на Халева, Ефониевия син.
\par 14 За това Хеврон стана притежание на Халева, син на Ефония Кенезов, както е и до днес, защото той последва напълно Господа Израилевия Бог.
\par 15 А изпърво името на Хеврон беше Кириат-арва; а Арва беше голям човек между енакимите. И земята утихна от война.

\chapter{15}

\par 1 А на племето на Юдейците, според семействата им, се падна по жребие земята до едомската граница; пустинята Цин на юг беше южният й край.
\par 2 И южната им граница беше от най-далечния край на Соленото море, от залива, който се простира към юг;
\par 3 и простираше се към юг до нагорнището на Акравим, преминаваше в Цин, възлизаше на юг от Кадис-варни, и минаваше край Есрон, възлизаше в Адар, завиваше към Карка,
\par 4 минаваше в Асмон, и излизаше при египетския поток; и границата свършваше при морето. Това ще ви бъде южна граница.
\par 5 А южната граница беше Соленото море до устието на Иордан. И границата на северната част почваше от залива на морето при устието на Иордан;
\par 6 и границата възлизаше до Ветагла и минаваше на север от Ветарава; и границата възлизаше до камъка на Воана Рувимовия потомък;
\par 7 и от долината Ахор границата възлизаше към Девир и завиваше на север към Галгал срещу нагорнището на Адумим, който е на южната страна на потока; после границата преминаваше към водата на Енсемес и свършваше до извора Рогил;
\par 8 и границата възлизаше през долината на Еномовия син, на южната страна на Евус (това е Ерусалим); и границата се възкачваше на върха на хълма, който е на запад срещу еномската долина, който е в северния край на долината на рефаимите;
\par 9 и от върха на хълма границата минаваше до извора на водата Нефтоя и излизаше в градовете на ефронската гора; и границата се отправяше към Ваала (която е Кириат-иарим);
\par 10 и от Ваала границата завиваше западно към сиирската гора, и минаваше край северната страна на гората Ярим (който е Хасалон), и слизаше у Ветсемес, и минаваше през Тамна;
\par 11 после границата излизаше към северната страна на Акарон; и границата се отправяше до Сикрон, и минаваше в гората Ваала, и свършваше при морето.
\par 12 А западната граница беше край голямото море и пределите му. Тия бяха границите околовръст, на юдейците според семействата им.
\par 13 И според заповяданото от Господа на Исуса, той даде на Халева Ефониевия син за дял между юдейците града на Арва, Енаковия баща, който град е Хеврон.
\par 14 И Халева изпъди от там тримата Енакови сина: Сесая, Ахимана и Талмая, Енакови чада.
\par 15 И от там отиде против жителите на Девир (а името на Девир по-напред беше Кириат-сефер).
\par 16 И рече Халев: Който порази Кириат-сефер и го превземе нему ще дам дъщеря си Ахса за жена.
\par 17 И превзе го Готониил син на Кенеза, Халевовия брат; и той му даде дъщеря си Ахса за жена.
\par 18 И като отиваше тя му внуши щото да поиска от баща й нива; и тъй, като слезе от осела, Халев й рече: Що ти е?
\par 19 А тя каза: Дай ми благословение: понеже си ми дал южна страна, дай ми и водни извори. И той и даде горните извори и долните извори.
\par 20 Това е наследството на племето на юдейците според семействата им.
\par 21 А най-крайните градове в племето на юдейците към едомските граници на юг бяха: Кавсеил, Едар, Ягур,
\par 22 Кина, Димона, Адада,
\par 23 Кадес, Асор, Итнан,
\par 24 Зиф, Телем, Ваалот,
\par 25 Асор-адата, Кириот, Есрон (който е Асор),
\par 26 Амам, Сема, Молада,
\par 27 Асаргад, Есемон, Ветфалет,
\par 28 Асар-суал, Вирсавее, Визиотия,
\par 29 Ваала, Иим, Асем,
\par 30 Елтолад, Хесил, Хорма,
\par 31 Сиклаг, Мадмана, Сансана,
\par 32 Леваот, Силеим, Аин и Римон; всичките градове със селата си бяха двадесет и девет.
\par 33 В равнината бяха: Естаол, Сараа, Асна,
\par 34 Заноа, Енганим, Тапфуа, Инам,
\par 35 Ярмут, Одолам, Сохо, Азика,
\par 36 Сагарим, Адитаим, Гедира и Гедиротаим; четиринадесет града със селата им;
\par 37 Сенан, Адаса, Мигдалгад,
\par 38 Далаан Масфа, Иоктеил,
\par 39 Лахис, Васкат, Еглон,
\par 40 Хавон, Лахмас, Хитлис,
\par 41 Гедирот, Вет-дагон, Наама и Макида; шестнадесет града със селата им;
\par 42 Ливна, Етер, Асан,
\par 43 Ефта, Асена, Несив,
\par 44 Кеила, Ахзив и Мариса; девет града със селата им;
\par 45 Акарон и заселищата му със селата му;
\par 46 От Акарон до морето всичките градове , които са близо пре Азот, със селата им;
\par 47 Азот и заселищата му със селата му, Газа и заселищата му със селата му, до египетския поток и голямото море с пределите му.
\par 48 А в хълмистите места: Самир, Ятир, Сохо,
\par 49 Дана, Кириат-сана (който е Девир),
\par 50 Анав, Естемо, Аним,
\par 51 Гесен, Олон и Гило; единадесет града със селата им;
\par 52 Арав, Дума, Есан,
\par 53 Янум, Вет-тапфуа, Афека,
\par 54 Хумата, Кириат-арва (който е Хеврон) и Сиор; девет града и селата им;
\par 55 Маон, Кармил, Зиф, Юта,
\par 56 Езраел, Иокдеам, Заноа,
\par 57 Акаин, Гаваа и Тамна, десет града със селата им;
\par 58 Алул, Ветсур, Гедор,
\par 59 Маарат, Ветанот и Елтекон, шест града със селата им;
\par 60 Кириат-ваал (който е Кириатиарим) и Рава, два града със селата им.
\par 61 В пустинята: Ветарава, Мидин, Сехаха,
\par 62 Нивсан, града на солта, и Енгади, шест града със селата им,
\par 63 А юдейците не можеха да изгонят евусейците, които населяваха Ерусалим; но евусейците живееха в Ерусалим с юдейците, и така живеят до днес.

\chapter{16}

\par 1 На Иосифовите потомци се падна по жребие земята от Иордан при Ерихон при ерихонските води на изток в пустинята, земята , която възлиза от Ерихон през хълмистата страна до Ветил;
\par 2 и границата се простираше от Ветил до Луз, и минаваше през околността на Архиатарот,
\par 3 и слизаше на запад в околността на Яфлети до околността на долния Веторон и до Гезер и свършваше при морето.
\par 4 И Иосифовите потомци, племената на Манасия и Ефрема, взеха наследството си.
\par 5 Пределите на ефремците, според семействата им, ето какъв беше: границата на наследството им към изток беше Атарот-адар до горния Веторон;
\par 6 и на север от Михметат границата се простираше към запад; и границата завиваше към изток до Танат-сило, и от там преминаваше на изток от Янох;
\par 7 и слизаше от Янох до Атарот и до Наарат, и достигаше до Ерихон и свършваше при Иордан.
\par 8 От Тапфуа границата отиваше към запад до потока Кана и свършваше при морето. Това е наследството на племето на ефремците според семействата им.
\par 9 Имаше и градове отделени за ефремците между наследството на манасийците, - всички тия градове със селата им.
\par 10 Но те не изгониха ханаанците, които живееха в Гезер; но ханаанците живееха между ефремците, както живеят и до днес, и станаха поданни роби.

\chapter{17}

\par 1 Хвърлено беше жребие и за племето на Манасия, защото той беше първородният на Иосифа. Колкото за Махира, Манасиевият първороден, баща на Галаада, понеже той беше военен мъж, за това Галаад и Васан станаха негови.
\par 2 Така че хвърленото жребие беше за другите Манасиеви потомци според семействата им; за потомците на Авиезера, за потомците на Хелека, за потомците на Асрииля, за потомците на Сихема, за потомците на Ефера и за потомците на Семида. Тия бяха мъжките чада на Иосифовия син Манасия според семействата им.
\par 3 Обаче, Салпаад, сина на Ефера, сина на Галаада, син на Махира, син на Манасия, нямаше синове, но дъщери; и ето имената на дъщерите му: Маала, Нуа, Егла, Мелха и Терса.
\par 4 Те дойдоха пред свещеника Елеазара, и пред Исуса Навиевия син, о пред първенците та казаха: Господ заповяда на Моисея да ни даде наследство между братята ни, За това, според Господното повеление, той им даде наследство между братята на баща им.
\par 5 И тъй, на Манасия се паднаха десет дяла, освен земята Галаад и Васан, която е оттатък Иордан;
\par 6 защото дъщерите на Манасия получиха наследство между синовете му, а Галаадската земя беше на другите потомци на Манасия.
\par 7 Манасиевата граница беше от Асир до Михметат, който е срещу Сихем; и границата се простираше надясно до жителите на Ентапфуя.
\par 8 Земята на Тапфуя принадлежеше на Манасия; а самият Тапфуя, на Манасиевата граница, принадлежеше на ефремците.
\par 9 И границата слизаше до потока Кана, на юг от потока. Тия градове между Манасиевите градове принадлежаха на Ефрема; и Манасиевата граница беше на север от потока и свършваше при морето.
\par 10 Земята на юг беше на Ефрема, а на север на Манасия, и морето беше границата му; и земите им допираха на север до Асир и на изток до Исахар.
\par 11 И в земята на Исахара и Асира, Манасия притежаваше Ветсан и заселищата му; Ивлеам и заселищата му, жителите на Дор и заселищата му, жителите на Ендор и заселищата му, жителите на Таанах и заселищата му и жителите на Маледон и заселищата му, три околии.
\par 12 Но манасийците не можаха да изгонят жителите на тия градове, а ханаанците настояваха да живеят в оная земя.
\par 13 А когато се закрепиха израилтяните, те наложиха на ханаанците данък, без да ги изгонят съвсем.
\par 14 Тогава Иосифовите потомци говориха на Исуса, казвайки: Защо даде ти да се хвърли само едно жребие за нас, и само един дял да наследим, тогаз когато сме много люде, понеже Господ ни е благословил до сега?
\par 15 А Исус им каза: Ако сте много люде, възкачете се на леса и си изсечете една част от него в земята на ферезейците и на рафаимите, тъй като хълместата част на Ефрема е тясна за вас.
\par 16 Но Иосифовите потомци казаха: Тая хълместа част не е сгодна за нас; при това, всичките ханаанци, които живеят в долинската земя, имат железни колесници, както ония, които са в Ветсан и заселищата му, така и ония, които са в Езраелската долина.
\par 17 Тогава Исус говори на Иосифовия дом - на Ефрема и на Манасия - казвайки: Наистина вие сте много люде и имате голяма сила; не ще имате само едно притежание чрез жребие;
\par 18 но хълместата част ще бъде ваша, макар че е залесена, защото ще я изсечете; ще бъде ваша и до краищата си, понеже ще изпъдите ханаанците, при все, че имат железни колесници и са силни.

\chapter{18}

\par 1 След това, цялото общество израилтяни се събра в Сило, гдето и поставиха шатъра за срещане; защото земята вече беше завладяна от тях.
\par 2 Обаче, между израилтяните останаха седем племена, които още не бяха взели наследството си.
\par 3 Исус, прочее, каза на израилтяните: До кога ще немарите да отидете и завладеете земята, която ви е дал Господ Бог на бащите ви?
\par 4 Изберете си по трима мъже от всяко племе, и аз ще ги пратя; и като станат, нека обходят земята и я опишат, според както трябва, да я наследят, и нека дойдат при мене.
\par 5 Нека я разделят на седем дяла, като си остане Юда в пределите си, на юг, и Иосифовият дом си остане в пределите си на север.
\par 6 И като опишете земята на седем дяла и донесете описанието тук при мене, аз ще хвърля жребие за вас тук пред Господа нашият Бог.
\par 7 Защото левитите нямат дял между вас, понеже Господното свещенство е тяхно наследство; а Гад и Рувим и половината от Манасиевото племе получиха оттатък Иордан, на изток, наследството си, което Господният слуга Моисей им даде.
\par 8 И тъй, мъжете станаха та отидоха; и на ония, които отидоха да опишат земята, Исус даде поръчка, като рече: Идете, обходете земята, за да я опишете, и върнете се при мене; и аз ще хвърля жребие за вас, тук в Сило пред Господа.
\par 9 Мъжете, прочее, отидоха и обходиха земята, и описаха я в книга по градове на седем дяла, и дойдоха при Исуса в стана в Сило.
\par 10 Тогава Исус хвърли жребие за тях в Сило пред Господа; и там Исус раздели земята на израилтяните според дяловете им.
\par 11 Излезе жребието на племето на вениаминците по семействата им; и като излезе жребието, пределът им се падна между юдейците и Иосифовите потомци.
\par 12 На север границата им беше от Иордан; и границата отиваше към северната страна на Ерихон, и възлизаше през хълмистата земя на запад, и свършваше при ветавенската пустиня;
\par 13 и от там границата минаваше към Луз, край южната страна на Луз (който е Ветил); и границата слизаше в Атарот-адар, до хълма, който е на юг от долния Ветерон.
\par 14 От там границата се простираше и завиваше на западната страна и завиваше на западната страна към юг, на юг от хълма, който е срещу Ветерон, и свършваше с Кириат-ваал (който е Кириатиарим), град на юдейците. Това беше западната страна.
\par 15 А южната страна беше от края на Кириатиарим, от гдето границата минаваше към запад и свършваше при извора на водата Нефтоя;
\par 16 и границата слизаше до края на хълма, който е срещу долината на Еномовия син, и който е на север от долината на рафаимите, и слизаше през долинката на Енома до южната страна на Евус, и слизаше при извора Рогил,
\par 17 и като се разпростираше от север минаваше в Енсемес, и излизаше в Галилот, който е срещу нагорнището при Адумим, и слизаше при камъка на Рувимовия син Воана,
\par 18 и минаваше в страната, която е на север срещу Арава, и слизаше в Арава;
\par 19 и границата минаваше към северната страна на Ветагла; и границата свършваше при северния залив на Соленото море, при южния край на Иордан. Това беше южната граница.
\par 20 А границата му на източната страна беше Иордан. Това беше според границите си околовръст, наследството на вениаминците, според семействата им.
\par 21 А градовете за племето на вениаминците по семействата им бяха: Ерихон, Ветагла, Емек-кесис,
\par 22 Ветарава, Семараим, Ветил,
\par 23 Авим, Фара, Офра,
\par 24 Хефар-амона, Афни, и Гава; дванадесет града със селата им;
\par 25 Гаваон, Рама, Вирот,
\par 26 Масфа, Хефира, Моса,
\par 27 Рекем, Ерфаил, Тарала,
\par 28 Сила, Елеф, Евус (който е Ерусалим), Гаваот и Кириат; четиридесет града със селата им. Това е наследството на вениаминците, според семействата им.

\chapter{19}

\par 1 Второто жребие излезе за Симеона, за племето на Симеоновците според семействата им; и наследството им беше всред наследството на юдейците.
\par 2 За своето наследство те имаха; Вирсавее (или Савее), Молада,
\par 3 Асур-суал, Вала, Асем,
\par 4 Елтолад, Ветул, Хорма,
\par 5 Сиклаг, Вет-маркавот, Асарсуса,
\par 6 Ветловаот и Саруен; тринадесет града със селата им;
\par 7 Аин, Римон, Етер и Асан; четири града със селата им;
\par 8 и всичките села, които бяха около тия градове, до Ваалат-вир (който е южният Рамат). Това е наследството за племето на симеонците според семействата им.
\par 9 От дяла на юдейците се даде наследството на симеонците, защото делът на юдейците беше голям за тях; за това, симеонците наследиха дял всред тяхното наследство.
\par 10 Третото жребие излезе за завулонците, според семействата им; и пределът на наследството им беше до Сарид;
\par 11 и границата им отиваше западно до Марала, и досягаше Давасес и досягаше потока, който е срещу Иокнеам;
\par 12 и от Сарид завиваше към изгрева на слънцето до междата на Кислот-тавор, и излизаше на Даврат, и възлизаше на Яфия,
\par 13 и от там минаваше към изток в Гетефер, в Ита-касин, и излизаше в Римон-метоар към Нуя;
\par 14 и от север границата завиваше в Анатот, и свършваше в долината Ефтаил,
\par 15 като включваше Катат, Наадал, Симрон, Идала и Витлеем, дванадесет града със селата им.
\par 16 Това е наследството на завулонците по семействата им, -тия градове със селата им.
\par 17 Четвъртото жребие излезе за Исахара, за исахарците по семействата им.
\par 18 Пределът им беше Езраил, Кесулот, Сунам,
\par 19 Афераим, Сеон, Анахарат,
\par 20 Равит, Кисион, Авес,
\par 21 Ремет, Енганим, Енада и Вет-фасис;
\par 22 и границата досягаше Тавор, Сахасима и Ветсемес; и границата им свършваше при Иордан; шестнадесет града със селата им.
\par 23 Това е наследството за племето на исахарците по семействата им, - градовете със селата им.
\par 24 Петото жребие излезе за племето на асирците семействата им.
\par 25 Пределът им беше Хелкат, Али, Ветен, Ахсаф,
\par 26 Аламелех, Амад и Мисал, и достигаше, до Кармил на запад и до Сихор-ливнат;
\par 27 и границата им завиваше към изгрева на слънцето при Вет-дагон, и досягаше Завулон и долината Ефтаил на север от Ветемек и Наил, и излизаше при Хаул на ляво,
\par 28 и Еврон, Роов, Амон и Кана, до големия Сидон;
\par 29 и границата завиваше към Рама и към укрепения град Тир; и границата завиваше към Оса, и свършваше при морето край страната на Ахзив,
\par 30 и Ама, Афен и Роов; двадесет идва града със селата им.
\par 31 Това е наследство за племето на асирците по семействата им - тия градове със селата им.
\par 32 Шестото жребие излезе за нефталимците, за нефталимците по семействата им.
\par 33 Границата им беше от Елеф, от Алон близо при Саананим, и Адами-некев, и Явнеил, до Лакум и свършваше при Иордан;
\par 34 и границата завиваше към запад до Азнот-тавор, и от там излизаше при Укок, и досягаше Завулон на юг, а досягаше Асир на запад и Юда при Иордан към изгрева на слънцето.
\par 35 А укрепените градове бяха: Сидим, Сер, Амат, Ракат Хинерот,
\par 36 Адама, Рама, Асор,
\par 37 Кедес, Едран, Енасор,
\par 38 Ирон, Мигдалил, Орам, Ветанат и Ветсемес; деветнадесет града със селата им,
\par 39 Това е наследството за племето на нефталимците по семействата им - градовете със селата им.
\par 40 Седмото жребие излезе за племето на данците по семействата им.
\par 41 Пределът, който те наследяваха, беше Саара, Естаол, Ерсемес,
\par 42 Салавим, Еалон, Етла,
\par 43 Елон, Тамната, Акарон,
\par 44 Елтеко, Гиветон, Ваалат,
\par 45 Юд, Вани-варак, Гетримон,
\par 46 Меиаркон и Ракон, с околността, която е срещу Иопа,
\par 47 А пределът на данците не им стигаше; за това данците отидоха и воюваха против Лесем, и като го превзеха и го поразиха с острото на ножа, взеха го за притежание и заселиха се в него; и Лесем наименуваха Дан, по името на праотца си Дан.
\par 48 Това е наследството за племето на данците по семействата им, - тия градове със селата им.
\par 49 А като свършиха подялбата на земята, според пределите й, израилтяните дадоха на Исуса Навиевия син наследството всред себе си;
\par 50 според Господната заповед, дадоха му града, който поиска, сиреч , Тамнат-сарах в хълмистата част на Ефрема; и той съгради града и живееше в него.
\par 51 Тия са наследствата, който свещеникът Елеазар и Исус Навиевия син и началниците на бащините домове на племената на израилтяните разделиха с жребие в Сило пред Господа, при входа на шатъра за срещане. Така свършиха подялбата на земята.

\chapter{20}

\par 1 Тогава Господ говори на Исуса, казвайки:
\par 2 Говори на израилтяните, като им речеш: Определете си прибежищните градове, за които съм ви рекъл чрез Моисея,
\par 3 за да прибягва там оня убиец, който убие човек без намерение, по незнание; и те да ви бъдат прибежище от мъздовъздателя за кръвта.
\par 4 И когато оня, които побегне в един от тия градове, застане във входа на градската порта и каже работата си на всеослушание пред старейшините от оня град, те нека го приемат в града при себе си, и нека му дадат място, и той да живее между тях.
\par 5 И ако мъздовъздателят за кръвта го подгони, та да не предават убиеца в ръката му; защото по незнание е убил ближния си, без да го е мразил от по-напред.
\par 6 И да живее в оня град догде се представи на съд пред обществото, и до смъртта на онзи който е първосвещеник в онова време; тогава убиецът да се върне и да отиде в града си, и в дома си, в града, от който е побягнал.
\par 7 И тъй, отделиха Кадес в Галилея, в хълмистата земя на Нефталима, Сихем в хълмистата земя на Ефрема, Кириат-арва (която е Хеврон) в хълмистата земя на Юда.
\par 8 А оттатък Иордан срещу Ерихон, на изток, определиха Восор в полската пустиня на Рувимовото племе, Рамот в Галаад от Гадовото племе, и Голан във Васан от Манасиевото племе.
\par 9 Тия бяха градовете определени за всичките израилтяни, и за чужденците, които живеят между тях, тъй щото всеки, който би убил някого без намерение, за бяга там, та да не бъде убит от мъздовъздателя за кръвта, догдето се представи пред обществото.

\chapter{21}

\par 1 След това, началниците на левитските бащини домове дойдоха при свещеника Елеазара, при Исуса Навиевия син и при началниците на бащините домове от племето на израилтяните,
\par 2 та им говориха в Сило, в Ханаанската земя, като казваха: Господ заповяда чрез Моисея да ни се дадат градове за живеене и пасбищата им за добитъка ни.
\par 3 И ето градовете с пасбищата им, които израилтяните дадоха на левитите от наследството си според Господната заповед:
\par 4 Излезе жребието за семействата на каатовците; и потомците на свещеника Аарона, които бяха от левитите, получиха чрез жребие тринадесет града от Юдовото племе, от Симеоновото племе и от Вениаминовото племе.
\par 5 А останалите Каатови потомци получиха чрез жребие десет града от семействата на Ефремовото племе, от Дановото племе и от половината на Манасиевото племе.
\par 6 И Гирсоновите потомци получиха чрез жребие тринадесет града от семействата на Исахаровото племе, от Асировото племе, от Нефталимовото племе и от половината на Манасиевото племе у Васан.
\par 7 Марариевите потомци, според семействата си получиха дванадесет града от Рувимовото племе, от Гадовото племе и от Завулоновото племе.
\par 8 Тия градове, прочее, и пасбищата и израилтяните дадоха чрез жребие на левитите, според както Господ заповяда чрез Моисея.
\par 9 От племето на юдеите и от племето на симеонците дадоха тия градове, които по-долу се споменават по име,
\par 10 и които се получиха от Аароновите потомци, бидейки от семействата на каатовците, които са от левийците; защото първото жребие се хвърли за тях:
\par 11 дадоха им града на Арва Енаковия баща (който град е Хеврон) с пасбищата му около него, в Юдовата хълместа земя;
\par 12 (а нивите на града и селата му дадоха на Халева Иефониевия син за негова собственост);
\par 13 Хеврон, прочее, с пасбищата му, прибежищния град за убиец, дадоха на потомците на свещеника Аарона, също и Ливна с пасбищата му,
\par 14 Ятир с пасбищата му, Естемой с пасбищата му,
\par 15 Олом с пасбищата му, Девир с пасбищата му,
\par 16 Аин с пасбищата му, Юда с пасбищата му и Ветсемес с пасбищата му; девет града от тия две племена;
\par 17 а от Вениаминовото племе, Гаваон с пасбищата му, Гава с пасбищата му,
\par 18 Анатон с пасбищата му и Алмот с пасбищата му; четири града.
\par 19 Всичките градове на Аароновите потомци, свещениците, бяха тринадесет града с пасбищата им.
\par 20 И семействата на Каатовите потомци, левитите които оставаха от Каатовите потомци, получиха градовете на своето притежание по жребие от Ефремовото племе.
\par 21 Дадоха им прибежищния за убиец град Сихем с пасбищата му в хълмистата земя на Ефрема, също и Гезер с пасбищата му,
\par 22 Кивзаим с пасбищата му и Веторон с пасбищата му; четири града;
\par 23 от Дановото племе, Елтеко с пасбищата му, Гиветон с пасбищата му.
\par 24 Еалон с пасбищата му и Гетримон с пасбищата му; четири града;
\par 25 а от половината на Манасиевото племе, Таанах с пасбищата му; два града.
\par 26 Всичките градове за семействата на останалите Каатови потомци бяха десет с пасбищата им.
\par 27 А на Гирсоновите потомци, от семействата на левитите, дадоха от другата половина на Манасиевото племе, прибежищния за убиец град Голан у Васан с пасбищата му и Веестера с пасбищата му; два града;
\par 28 от Исахаровото племе, Кисион с пасбищата му, Даврат с пасбищата му,
\par 29 Ярмут с пасбищата му и Енганим с пасбищата му; четиридесет града;
\par 30 от Асировото племе, Мисаал с пасбищата му, Авдон с пасбищата му,
\par 31 Хелкат с пасбищата му и Роов с пасбищата му, Авдон с пасбищата му,
\par 32 а от Нефталимовото племе, прибежищния за убиец град Кедес в Галилея с пасбищата му, Амот-дор с пасбищата му и Картан с пасбищата му; три града.
\par 33 Всичките градове на Гирсоновците според семействата им бяха тридесет града с пасбищата им.
\par 34 И на семействата Мерариевите потомци, останалите от левитите, дадоха от Завулоновото племе Иокнеам с пасбищата му, Карта с пасбищата му,
\par 35 Димна с пасбищата му и Наалон с пасбищата му; четири града;
\par 36 от Рувимовото племе, Восор, с пасбищата му, Яса с пасбищата му,
\par 37 Кедимот с пасбищата му и Мефаят с пасбищата му, Маханаим с пасбищата му;
\par 38 а от Гадовото племе, прибежищния за убиец град Рамот в Галаад се пасбищата му, Маханаим с пасбищата му;
\par 39 Есевон с пасбищата му и Язир с пасбищата му; всичко четири града.
\par 40 Всичките градове, които дадоха чрез жребие на Манасиевите потомци, според семействата им, на останалите от семействата на левитите, бяха дванадесет града.
\par 41 Всичките градове на левитите всред притежанието на израилтяните бяха четиридесет и осем града с пасбищата им.
\par 42 Тия градове бяха всеки с околните си пасбища; така бяха всичките тия градове.
\par 43 Така Господ даде на Израиля цялата земя, за която беше се клел на бащите им, че ще им я даде; и завладяха я и заселиха се в нея.
\par 44 И Господ им даде спокойствие от всяка страна, според всичко що беше се клел на бащите им; и от всичките им неприятели никоя не можа да устои против тях; Господ предаде всичките им неприятели в ръката им.
\par 45 Не пропадна ни едно от всичките добри неща, които Господ беше говорил на Израилевия дом; всички се сбъднаха.

\chapter{22}

\par 1 Тогава Исус повика рувимците, гадците и половината от Манасиевото племе та им каза:
\par 2 Вие изпълнихте всичко що ви заповяда Господният слуга Моисей; послушахте и моя глас във всичко, що съм ви заповядал.
\par 3 През това дълго време до днес не оставихте братята си, но изпълнихте поръчката, която Господ вашият Бог ви заповяда.
\par 4 И сега Господ вашият Бог успокои братята ви, според както им се бе обещал; за това, върнете се сега та идете по шатрите си в земята, която е притежанието ви, която Господният слуга Моисей ви даде оттатък Иордан.
\par 5 Но внимавайте добре да изпълнявате заповедите и закона, който Господният слуга Моисей ви заповяда, да любите Господа вашия Бог, да ходите във всичките Му пътища, да пазите заповедите Му, да се прилепяте за Него, и да Му слугувате с цялото си сърце и с цялата си душа.
\par 6 И тъй, Исус ги благослови и ги разпусна; и те си отидоха по шатрите.
\par 7 (На половината от Манасиевото племе Моисей беше дал наследство във Васан; а на другата му половина Исус даде наследство между братята им оттатък Иордан, на запад), Исус, прочее, когато ги разпускаше по шатрите им, благослови ги
\par 8 като им говори казвайки: Върнете се с много богатство по шатрите си, с твърде много добитък, със сребро и злато, с мед и желязо, и с твърде много дрехи; и разделете с братята си користите взети от неприятелите ви.
\par 9 И тъй, рувимците, гадците и половината от Манасиевото племе се върнаха и заминаха от израилтяните от Сило, което е в Ханаанската земя, за да отидат в Галаадската земя, в земята, която бе притежанието им, която придобиха според Господното слово чрез Моисея.
\par 10 И когато дойдоха в Иорданската околност, която е в Ханаанската земя, рувимците, гадците и половината от Манасиевото племе издигнаха там олтар при Иордан, олтар забележително голям.
\par 11 А израилтяните чуха да се говори: Ето, рувимците, гадците и половината от Манасиевото племе издигнали олтар в преднината на Ханаанската земя, в околността на Иордан, на страната, която принадлежи на израилтяните.
\par 12 И когато израилтяните чуха това, цялото общество израилтяни се събра в Сило, за да отидат да воюват против тях.
\par 13 И израилтяните пратиха до рувимците, до гадците и до половината от Манасиевото племе, в Галаадската земя, Финееса син на свещеника Елеазара,
\par 14 и с него десет първенци, по един първенец на бащиния дом за всяко Израилево племе, всеки от които беше глава на бащиния си дом между израилевите хиляди.
\par 15 И те дойдоха при рувимците, при гадците и при половината от Манасиевото племе в Галаадската земя, та им говориха, казвайки:
\par 16 Така говори цялото Господно общество: Какво е това престъпление, което извършихте против Израилевия Бог, та се отвърнахте днес от да следвате Господа, като сте си издигнали олтар, за да отстъпите днес от Господа?
\par 17 Малко ли беше за нас съгрешението относно Фегора, от което и до днес не сме се очистили, макар че, поради него , стана язва в Господното общество,
\par 18 та вие днес искате да се отвърнете от да следвате Господа? Значи, че понеже отстъпвате днес от Господа, то утре Той ще се разгневи против цялото Израилево общество.
\par 19 Но ако земята, която е притежанието ви, е нечиста, тогава минете в земята, която е Господното притежание, гдето стои Господната скиния, и вземете притежание между нас; само не отстъпвайте от Господа, нито от нас отстъпвайте, като си издигнете олтар освен олтара на Господа нашия Бог.
\par 20 Ахан Зараевият син не извърши ли престъпление относно обречените неща, и гняв падна върху цялото Израилево общество? И този човек не погина сам в бъззаконието си.
\par 21 Тогава рувимците, гадците и половината от Манасиевото племе в отговор рекоха на Израилевите хилядоначалници:
\par 22 Могъщият Бог Иеова, могъщият Бог Иеова, Той знае, и Израил сам ще узнае; ако сме сторили това за отстъпление, или за престъпление против Господа, то не ни избавяй, Господи , днес.
\par 23 Ако сме си издигнали олтар, за да се отвърнем от да следваме Господа, или за да принасяме върху него всеизгаряне или, за да принасяме върху него примирителни жертви, то сам Господ нека издири това,
\par 24 и види дали, напротив , не сторихме това от предпазливост и нарочно, като си думахме: Утре е възможно вашите чада да говорят на нашите чада, казвайки: Каква работа имате вие с Господа Израилевия Бог?
\par 25 Защото Господ е положил Иордан за граница между нас и вас, рувимци и гадци; вие нямате дял в Господа. Така щяха вашите чада да направят нашите чада да престанат да се боят от Господа.
\par 26 Затова рекохме: Нека се заемем да си издигнем олтар, не за всеизгаряне, нито за жертва,
\par 27 но за да бъде свидетелство между нас и вас, и между поколенията ни подир нас, че ние имаме право да вършим пред Господа службата му с всеизгарянията си, с жертвите си и с примирителните си приноси, та да не могат утре вашите чада да кажат на нашите чада: Вие нямате дял в Господа.
\par 28 Затова рекохме, ако се случи утре да говорят така нам или на поколенията ни, тогава ще речем: Ето подобието на Господния олтар, който бащите ни издигнаха, не за всеизгаряне, нито за жертва, но за да бъде свидетелство между нас и вас.
\par 29 Не дай, Боже, да отстъпим ние от Господа и да се отвърнем днес от да следваме Господа като издигнем олтар за всеизгаряне, за принос и за жертва, освен олтара на Господа нашия Бог, който е пред скинията Му.
\par 30 А свещеникът Финеес, първенците на обществото и Израилевите хилядници, които бяха с него, като чуха думите, които говориха рувимците, гадците и манасийците, останаха твърде доволни.
\par 31 И Финеес, син на свещеника Елеазара, каза на рувимците, на гадците и на манасийците: Днес познахме, че Господ е всред нас, защото не извършихме това престъпление против Господа; чрез това избавихме израилтяните от Господната ръка.
\par 32 Тогава Финеес, син на свещеника Елеазара и първенците се върнаха от рувимците и от гадците, из Галаадската земя, в Ханаанската земя при израилтяните, и донесоха им известие.
\par 33 И това нещо стана угодно на израилтяните; и израилтяните благословиха Бога, и не говореха вече за отиване против тях на бой; за да разорят земята, гдето живееха рувимците и гадците.
\par 34 И рувимците и гадците наименуваха олтара така: Той е свидетел помежду ни, че Иеова е Бог.

\chapter{23}

\par 1 След много време, когато Господ беше успокоил Израиля от всичките му околни неприятели, и Исус беше остарял и в напреднала възраст.
\par 2 Исус свика целия Израил, старейшините им, началниците им, та из каза: Аз остарях и съм в напреднала възраст;
\par 3 а вие видяхте всичко, що Господ вашият Бог извърши за вас на всички тия народи; защото Господ вашият Бог, Той е, Който е воювал за вас.
\par 4 Ето, разделих между вас с жребие в наследство на племената ви земята на тия останали народи, и на всичките народи, които погубих, от Иордан до голямото море, към захождането на слънцето.
\par 5 И Господ вашият Бог, Той ще ги изпъди от пред вас и ще ги изгони от пред очите ви; и вие ще завладеете земята им, според както Господ вашият Бог ви се е обещал.
\par 6 Бъдете, прочее, много храбри, да пазите и да вършите всичко, що е написано в книгата на Моисеевия закон, без да се отклонявате от него ни на дясно ни на ляво,
\par 7 за да се не смесвате с тия народи, които останаха помежду ви, нито да споменавате имената на боговете им, нито да се кълнете в тях , нито да им служите, нито да им се кланяте;
\par 8 но към Господа вашия Бог да сте привързани, както сте били до днес.
\par 9 Защото Господ е изгонил от пред вас големи и силни народи; и до днес никой не може да устои пред вас.
\par 10 Един от вас е гонил хиляда, защото Господ вашият Бог, Той е Който воюва за вас, според както ви се е обещал.
\par 11 За това, внимавайте добре да любите Господа вашия Бог.
\par 12 Иначе, ако се върнете някога назад та се привържете към остатъка на тия народи, към останалите между вас, и правите сватовство с тях, и се смесвате с тях и се с вас,
\par 13 да знаете добре, че Господ вашият Бог не ще вече да изгони от пред очите ви тия народи; но те ще бъдат клопка и примка бичове по ребрата ви и тръне в очите ви, догдето изчезнете из тая добра земя, която Господ вашият Бог ви е дал.
\par 14 И, ето, днес аз отивам по пътя на целия свят; и вие знаете с цялото си сърце и цялата си душа, че не пропадна ни едно от тия добри неща, които Господ вашият Бог говори за вас; всичките се сбъднаха за вас; ни едно от тях не пропадна.
\par 15 Но, ще стане, че, както се сбъднаха за вас всичките добри неща, които Господ вашият Бог говори на вас, така Господ ще докара на вас всичките зли неща, догдето ви изтреби из тая добра земя, която Господ вашият Бог ви е дал,
\par 16 когато престъпите завета на Господа вашия Бог, който Той ви заповяда, отидете и служите на други богове та им се кланяте; тогава гневът на Господа ще пламне против вас, и вие скоро ще изчезнете из добрата земя, която Той ви е дал.

\chapter{24}

\par 1 След това Исус събра всичките Израилеви племена в Сихем и свика старейшините на Израиля, началниците им, съдиите им надзирателите им; и те се представиха пред Господа.
\par 2 И Исус каза на всичките люде: Така говори Господ Израилевият Бог: В старо време оттатък реката живееха бащите ви, Тара, Авраамовият баща и Нахаровият баща, и служеха на други богове.
\par 3 И Аз взех баща ви Авраама от оная страна на реката, водих го през цялата Ханаанска земя, и умножих потомството му и му дадох Исаака;
\par 4 а на Исаака дадох Якова и Исава. И на Исава дадох половината Сиир за притежание; а Яков и синовете му слязоха в Египет.
\par 5 И против Моисея и Аарона, и поразиха Египет с язви, които извърших всред него, и после ви изведох.
\par 6 И като извеждах бащите ви из Египет, вие дойдохте на морето; и Египтяните се спуснаха след бащите ви с колесници и коне в Червеното море.
\par 7 Тогава те извикаха към Господа, и Той тури мрак между вас и египтяните, и навлече върху тях морето та ги покри; и очите ви видяха що сторих в Египет; и вие живяхте в пустинята дълго време.
\par 8 После ви доведох в земята на аморейците, които живееха оттатък Иордан, и те воюваха против вас; но Аз ги предадох в ръцете ви, и вие завладяхте земята им, и Аз ги изтребих от пред вас.
\par 9 Тогава моавският цар Валак, Сепфоровият син, стана и воюва против Израиля; и прати да повикат Валаама Веоровия син, за да ви прокълне.
\par 10 Но Аз не склоних да послушам Валаама, и той даже ви благослови; и Аз ви избавих от ръката му.
\par 11 И вие преминахте Иордан та дойдохте в Ерихон; и воюваха против вас ерихонските мъже; амонците, ферезейците, ханаанците, хетейците, гергесейците, евейците и евусейците; и Аз ги предадох в ръцете ви.
\par 12 И изпратих пред вас стършели, които изгониха от пред вас двамата аморейски царе, - не с твоя нож, нито с твоя лък.
\par 13 И дадох ви земя на която не бяхте положили труд, и градове, които не бяхте съградили, и вие живеете в тях; и ядете от лозя и маслини, които не сте садили.
\par 14 Сега, прочее, бойте се от Господа, и служете Му с искреност и истина; и махнете боговете, на които служеха бащите ви оттатък реката и в Египет, и служете Господу.
\par 15 Но ако ви се види тежко да служите Господу, изберете днес кому искате да служите, - на боговете ли, на които служиха бащите ви оттатък реката, или на боговете на аморейците, в чиято земя живеете; но аз и моят дом ще служим Господу.
\par 16 И людете в отговор казаха: Не дай, Боже, да оставим Господа, за да служим на други богове!
\par 17 защото Господ нашият Бог, Той е, Който изведе нас и бащите ни из Египетската земя, от дома на робството, и Който извърши пред нас ония големи знамения, и ни опази през целия път, по който пътувахме, и между всичките племена, през сред които минахме.
\par 18 Тоже Господ изгони от пред нас всичките племена, аморейците, които живееха в тая земя. За това и ние ще служим Господу, защото Той е нашият Бог.
\par 19 Но Исус каза на людете: Не ще можете да служите Господу; защото Той е Бог пресвет; Той е Бог ревнив; не ще да прости престъпленията и греховете ви.
\par 20 Защото, ако оставите Господа и служите на чужди Богове тогава Той ще се обърне и ще ви стори зло и ще ви изтреби, въпреки добрата, което ви е сторил.
\par 21 А людете казаха на Исуса: Не, но Господу ще служим.
\par 22 Тогава Исус каза на людете: Вие сте свидетели против себе си, че си избрахте Господа, за да Му служите; (и те рекоха: Свидетели сме);
\par 23 сега, прочее, махнете чуждите богове, които са всред вас, и преклонете сърцата си към Господа Израилевия Бог.
\par 24 И людете казаха на Исуса: На Господа нашия Бог ще служим, и Неговия глас ще слушаме.
\par 25 И тъй, в същия ден Исус направи завет с людете, и им постави закон и постановление в Сихем.
\par 26 И Исус написа тия думи в книгата на Божия закон; и взе голям камък та го изправи там подир дъба, който е близо при Господното светилище.
\par 27 И Исус каза на всичките люде: Ето тоя камък ще ни бъде за свидетел, защото той чу всичките думи, които Господ ни говори, и той ще ви бъде за свидетел, в случай, че се откажете от вашия Бог.
\par 28 Така Исус разпусна людете, да отидат всеки в наследството си.
\par 29 След това, Господният слуга Исус Навиевият син, умря на възраст сто и десет години.
\par 30 И погребаха го в предела на наследството му в Тамнат-сарах, който е в хълмистата земя на Ефрема, на север от хълма Гаас.
\par 31 И Израил служи на Господа през всичките дни на Исуса, и през всичките дни на старейшините, които преживяха Исуса, и които знаеха всичките дела, които Господ бе извършил за Израиля.
\par 32 И в Сихем погребаха Иосифовите кости, които израилтяните изнесоха из Египет, в местността на нивата, която Яков купи за сто сребърника от синовете на Емора, Сихемовия баща; и тя стана наследство на Иосифовите потомци.
\par 33 И Елеазар Аароновият син умря; и погребаха го на хълма на сина му Финееса, който му бе даден в хълмистата земя на Ефрема.

\end{document}