\begin{document}

\title{Judges}


\chapter{1}

\par 1 Подир смъртта на Исуса, израилтяните се допитаха до Господа, казвайки: Кой пръв ще възлезе за нас против ханаанците да воюва против тях?
\par 2 И Господ каза: Юда ще възлезе, ето, предадох земята в ръката му.
\par 3 Тогава Юда каза на брата си Симеона: Възлез с мене в моя предел, за да воюваме против ханаанците, и аз ще отида с тебе в твоя предел. И Симеон отиде с него.
\par 4 Юда, прочее възлезе; и Господ предаде ханаанците и ферезейците в ръката им; и те поразиха от тях в Везек десет хиляди мъже.
\par 5 И намериха Адонивезека във Везек, воюваха против него и поразиха ханаанците и ферезейците.
\par 6 А Адонивезек побягна; но те го погнаха и хванаха го, и отсякоха палците на ръцете му и на нозете му.
\par 7 И рече Адонивезек: Седемдесет царе с отсечени палци на ръцете и на нозете си са събирали трохи под трапезата ми; както съм аз правил, така ми въздаде Бог. И доведоха го в Ерусалим, гдето и умря.
\par 8 И юдейците воюваха против Ерусалим, и, като го завлякоха, поразиха го с острото на ножа и предадоха града на огън.
\par 9 Подир това, юдейците слязоха, за да воюват против ханаанците, които живееха в хълмистата, и в южната и в полската страни.
\par 10 Юда отиде и против ханаанците, които живееха в Хеврон (а по-напред името на Хеврон беше Кириат-арва), и убиха Сесая, Ахимана и Талмая;
\par 11 и от там отиде против жителите на Девир (а по-напред името на Девир беше Кириат-сефер).
\par 12 И рече Халев: Който порази Кириат-сефер и го превземе, нему ще дам дъщеря си Аса за жена.
\par 13 И превзе го Готониил, син на Кенеза, по-малкия брат на Халева; и той му даде дъщеря си Ахса за жена.
\par 14 И като отиваше тя му внуши щото да поиска от баща й нива; и тъй, като слезе от осела, Халев й рече: Що ти е?
\par 15 А тя му рече: Дай ми благословение; понеже си ми дал южна страна, дай ми и водни извори. И Халев й даде гоните извори и долните извори.
\par 16 И потомците на кенееца, Моисевия тъст, отидоха от Града на палмите, заедно с юдейците, в Юдовата пустиня, която е на юг от Арад; и отидоха та се заселиха между людете.
\par 17 Тогава Юда отиде с братята си Симеона, та поразиха ханаанците, които живееха в Сефат; и обрекоха града на изтребление; и градът се нарече Орма.
\par 18 Юда завладя и Газа с околностите му, Аскалон с околностите му и Акарон с околностите му.
\par 19 Господ беше с Юда; и той изгони жителите на хълмистата страна, обаче, не изгони жителите и на долината, защото имаха железни колесници.
\par 20 И дадоха Хеврон на Халева, според както Моисей беше казал; и той изгони от там тримата Енакови синове.
\par 21 А вениаминците не изгониха евусейците, които населяваха Ерусалим но евусейците живееха в Ерусалим заедно с вениаминците, както живеят и до днес.
\par 22 Също и Иосифовият дом, и те отидоха против Ветил; и Господ беше с тях.
\par 23 И Иосифовият дом прати да съгледат Ветил (а по-напред името на града беше Луз).
\par 24 И съгледателите видяха един човек, който излизаше из града, и рекоха му: Покажи ни, молим, входа на града, и ще ти покажем милост.
\par 25 И той им показа входа на града, и те поразиха града с острото на ножа, а човека оставиха да излезе с цялото си семейство.
\par 26 И човекът отиде в Хетейската земя, гдето и съгради град и нарече го Луз, както е името му и до днес.
\par 27 Манасия не изгони жителите на Ветсан и селата му, нито на Таанах и селата му, нито жителите на Дор и селата му, нито жителите на Ивреам и селата му, нито жителите на Магедон и селата му; но ханаанците настояваха да живеят в оная земя.
\par 28 А Израил, когато стана силен наложи на ханаанците данък, без да ги изгони съвсем.
\par 29 Нито Ефрем изгони ханаанците, които живееха в Гезер; но ханаанците живееха в Гезер помежду им.
\par 30 Завулон тоже не изгони жителите на Китрон, нито жителите на Наалол; но ханаанците живееха между тях и бяха обложени с данък.
\par 31 Асир не изгони жителите на Акхо, нито жителите на Сидон, нито на Ахлав, нито на Ахзив, нито на Хелва, нито на Афек, нито на Роов;
\par 32 но асирците обитаваха между ханаанците, местните жители, защото не ги изгониха.
\par 33 Нефталим не изгони жителите на Ветсемес, нито жителите на Ветенат, но живееше между ханаанците, местните жители; обаче жителите на Ветсемес и на Ветенат му плащаха данък.
\par 34 И аморейците принудиха данците да се оттеглят в хълмистата страна, защото не ги оставиха да слизат в долината;
\par 35 но аморейците настояваха да живеят в гората Ерес, в Еалон, и в Саалвим. Но при все това, ръката на Иосифовия дом преодоля, така щото ония бяха обложени с данък.
\par 36 А пределите на аморейците беше от нагорнището на Акравим, от скалата и нагоре.

\chapter{2}

\par 1 И ангел Господен дойде от Галгал у Бохим та каза: Изведох ви из Египет и доведох ви в земята, за която бях се клел на бащите ви, като рекох: Няма да наруша завета Си с вас до века.
\par 2 Вие да не правите договор с жителите на тая земя, а да съсипете олтарите им; но вие не послушахте гласа Ми. Защо сторихте това?
\par 3 За това и Аз рекох: Няма да ги изгоня от пред вас; но те ще се намират между ребрата ви, и боговете им ще ви бъдат примка.
\par 4 И когато ангелът Господен изговори тия думи на всички израилтяни, людете плакаха с висок глас.
\par 5 За туй нарекоха онова място Бохим; и там пренесоха жертва Господу.
\par 6 И когато Исус разпусна людете, израилтяните отидоха всеки в наследството си, за да притежават земята,
\par 7 И людете служиха на Господа през всичките дни на Исуса и през всичките дни на старейшините, които преживяха Исуса, които видяха всичките велики дела, които Господ беше извършил за Израиля.
\par 8 И Господният слуга, Исус Навиевият син, умря, на възраст сто и десет години.
\par 9 И погребаха го в предела на наследството му в Тамнат-арес, в хълмистата земя на Ефрема, на север от хълма Гаас.
\par 10 Също и цялото това поколение се прибра при бащите си; а след тях настана друго поколение, което не знаеше Господа, нито делото, което беше извършил за Израиля.
\par 11 И израилтяните сториха зло пред Господа, като се поклониха на ваалимите,
\par 12 и оставиха Господа Бога на бащите си, Който ги беше извел из Египетската земя, та последваха други богове, от боговете на племената, които бяха около тях, и, като им се поклониха, разгневиха Господа.
\par 13 Те оставиха Господа та служиха на Ваала и на астартите.
\par 14 И гневът на Господа пламна против Израиля, и Тоя ги предаде в ръката на грабители, които ги ограбиха; и предаде ги в ръката на околните им неприятели, така, щото не можаха вече да устоят пред неприятелите си.
\par 15 Където и да излизаха, Господната ръка беше против тях за зло, както Господ беше говорил, и според както Господ им се беше клел; и те изпаднаха в голямо утеснение.
\par 16 Тогава Господ въздигаше съдии, които ги избавяха от ръката на грабителите им.
\par 17 Но те и съдиите си не слушаха, а блудствуваха след други богове и кланяха им се; скоро се отклониха от пътя, в който ходиха бащите им, които слушаха Господните заповеди: те, обаче , не направиха така.
\par 18 И когато Господ им въздигаше съдии, тогава Господ беше със съдията и ги избавяше от ръката на неприятелите им през всичките дни на съдията; защото Господ се смиляваше поради охканията им от ония, които ги угнетяваха и притесняваха.
\par 19 А когато умреше съдията, те се връщаха и се развращаваха по-зле от бащите си, като следваха други богове, за да им служат и да им се кланят; не преставаха от делата си, нито от да ходят упорито в пътя си.
\par 20 От това гневът на Господа пламна против Израиля, и Той каза: Понеже тоя народ престъпи завета Ми, който съм заповядал на бащите им, и не послушаха гласа ми,
\par 21 то и Аз няма да изгоня вече от пред тях ни един от народите, които Исус остави, когато умря,
\par 22 за да изпитам чрез тях Израиля да ли ще пазят Господния път и ходят в него, както го пазеха бащите им, или не.
\par 23 И тъй, Господ остави тия народи, без да ги изгони скоро, и не ги предаде в ръката на Исуса.

\chapter{3}

\par 1 А ето народите, които Господ остави, за да изпита чрез тях Израиля, всички от тях, които не знаеха всичките ханаански богове:
\par 2 само за да знаят и да се научат на бой поне тия от поколенията на израилтяните, които най-напред не са знаели; именно :
\par 3 петте началници на филистимците, и всичките ханаанци, сидонци и евейци, които живеят в Ливанската планина, от планината Ваалермон до прохода на Емат.
\par 4 Те служеха, за да се изпита Израил чрез тях та да се знае, ще слушат ли заповедите, които Господ заповяда на бащите им чрез Моисея.
\par 5 Така също израилтяните се заселиха между ханаанците, хетейците, аморейците, ферезейците, евейците и евусейците;
\par 6 и вземаха си дъщерите им за жени, и даваха дъщерите си на синовете им, и служеха на боговете им.
\par 7 И израилтяните сториха зло пред Господа, като забравиха Господа своя Бог и служеха на ваалимите и на ашерите.
\par 8 Затова гневът на Господа пламна против Израиля, и Той ги предаде в ръката на месопотамския цар Хусанрисатаима; и израилтяните бяха подчинени на Хусанрисатаима осем години.
\par 9 А когато израилтяните извикаха към Господа, Господ въздигна избавител на израилтяните, който ги спаси, Готониила син на Кенеза, по-младият брат на Халева.
\par 10 Дух Господен дойде върху него, и той съди Израиля. Излезе и на бой, и Господ предаде в ръката му месопотамския цар Хусанрисатаима; и ръката му преодоля против Хусанрисатаима.
\par 11 Тогава земята имаше спокойствие четиридесет години. И Готониил Кенезовият син умря.
\par 12 Пак израилтяните сториха зло пред Господа; и Господ укрепи моавския цар Еглон против Израиля, по причина, че сториха зло пред Господа.
\par 13 И той събра при себе си амонците и амаличаните и отиде та порази Израиля, и завладя Града на палмите.
\par 14 И израилтяните бяха подчинени на моавския цар Еглон осемнадесет години.
\par 15 А когато израилтяните извикаха към Господа, Господ им въздигна избавител, Аода син на вениаминеца Гира, мъж левак. Чрез него израилтяните пратиха подарък на моавския цар Еглон;
\par 16 но Аод си направи меч остър и от двете страни, дълъг един лакът, и опаса го под горната си дреха на дясното си бедро.
\par 17 Той, прочее, принесе подаръка на моавския цар Еглон. А Еглон беше човек твърде тлъст.
\par 18 И като свърши да принесе подаръка, и изпрати човеците, които носеха подаръка,
\par 19 той се върна от пограничните камъни които са при Галгал, и каза: Имам тайна дума за тебе, царю. А той му рече: Мълчи. И излязоха от него всичките, които стояха при него.
\par 20 Тогава Аод дойде при него; и той седеше сам в лятната си горна стая. И рече Аод: Имам дума от Бога за тебе. Тогава той стана от стола си.
\par 21 А Аод простря лявата си ръка и, като измъкна меча от дясното си бедро, заби го в корема му
\par 22 толкоз, щото и дръжката влезе след желязото; и тлъстината до там стисна желязото, щото не можеше да изтръгне из корема му меча, който излезе отзад.
\par 23 Тогава Аод излезе през трема и затвори след себе си вратата на горната стоя, и я заключи.
\par 24 А когато излезе той, дойдоха слугите на Еглона , и, като видяха, че, ето, вратата на горната стая беше заключена, рекоха: Без съмнение по нуждата си е в лятната стая.
\par 25 Но макар че чакаха догде ги хвана срам, ето, той не отваряше вратата на горната стая, затова, взеха ключа та отвориха, и, ето, господарят им лежеше мъртъв на земята.
\par 26 А докато те се бавеха, Аод избяга; и премина пограничните камъни, та се отърва в Сеирота.
\par 27 И когато дойде, засвири с тръба в Ефремовата хълмиста земя, та израилтяните слязоха с него от хълмистата земя, и той бе на чело.
\par 28 И рече им: Вървете подир мене, защото Господ предаде неприятелите ви моавците в ръката ви. И те слязоха подир него, и като превзеха бродовете на Иордан към моавската страна, не оставиха да премине ни един човек.
\par 29 И в онова време поразиха от моавците около десет хиляди мъже, всички едри и всички яки; не се избави ни един човек.
\par 30 Така в онова време Моав се покори под ръката на Израиля. Тогава земята имаше спокойствие осемдесет години.
\par 31 След него настана времето на Самегар Анатовият син, който с един волски остен порази шестстотин мъже от филистимците; също и той избави Израиля.

\chapter{4}

\par 1 И подир смъртта на Аода израилтяните пак сториха зло пред Господа.
\par 2 Затова Господ ги предаде в ръката на ханаанския цар Явин, който царуваше в Асор, на чиито войски началник беше Сисара, който живееше в Аросет езически.
\par 3 И израилтяните извикаха към Господа; защото Явин имаше деветстотин железни колесници и жестоко притесняваше израилтяните двадесет години.
\par 4 В онова време пророчица Девора, Лафидотова жена, съдеше Израиля.
\par 5 И тя живееше под Деворцината палма между Рама и Ветил, в Ефремовата хълмиста земя; и израилтяните възлизаха при нея за съд.
\par 6 И тя прати да повикат Варака Авиноамовия син от Нефталимовия Кадис и му рече: Не заповяда ли Господ Израилевият Бог, като каза : Иди, оттегли се в хълма Тавор и вземи със себе си десет хиляди мъже от нефталимците и от завулонците;
\par 7 и Аз ще прекарам при тебе, при реката Кисон, Сисара, началник на Явиновата войска, с колесниците му и множеството му, и ще го предам в ръката ти?
\par 8 А Варак й каза: Ако дойдеш ти с мене, ще отида; но ако не дойдеш с мене, няма да отида.
\par 9 А тя рече: Непременно ще отида с тебе; но няма да придобиеш чест от похода, на който отиваш, защото в ръката на жена Господ ще предаде Сисара. И тъй, Девора стана та отиде с Варака в Кадис.
\par 10 И Варак свика Завулона и Нефталима в Кадис; и вървяха подир него десет хиляди мъже; и Девора отиде с него.
\par 11 А кенеецът Хевер, от потомците на Моисеевия тъст Овав, беше се отделил от кенейците и беше поставил шатъра си до дъба при Саанаим, който е близо да Кадис.
\par 12 И известиха на Сисара, че Варак Авиноамовият син се изкачил на хълма Тавор.
\par 13 Затова, Сисара свика, от Аросет езически, пре реката Кисон, всичките си колесници, деветстотин железни колесници, и всичките люде, които бяха с него.
\par 14 Тогава Девора каза на Варака: Стани, защото тоя е денят, в който Господ предаде Сисара в ръката ти. Не излезе ли Господ пред тебе? И тъй, Варак слезе от хълма Тавор, и десет хиляди мъже подир него.
\par 15 И Господ разби Сисара с острото на ножа пред Варака, с всичките му колесници и цялата му войска; и Сисара слезе от колесницата си та побягна пеш.
\par 16 И Варак преследва колесницата и войската до Аросет езически; и цялата войска на Сисара падна чрез острото на ножа; не остана ни един.
\par 17 А Сисара побягна пеш в шатъра на Яил, жената на кенееца Хевер; защото имаше мир между асорския цар Явин и дома на кенееца Хевер.
\par 18 И Яил излезе да посрещне Сисара и рече му: Свърни, господарю мой, свърни у мене; не бой се. И когато свърна у нея в шатъра, тя го покри с черга.
\par 19 И той й каза: Дай ми, моля, да пия вода, защото ожаднях. И тя развърза мех с мляко та му даде да пие; и пак го покри.
\par 20 И рече й: Застани при входа на шатъра, и ако дойде някой и те попита, казвайки: Има ли някой тук? Кажи: Няма.
\par 21 Тогава Яил, Хеверовата жена, взе един кол от шатъра, взе и млат в ръката си, та отиде тихо при него и заби кола в слепите му очи, така щото премина в земята; а той, като беше уморен, спеше дълбоко; и умря.
\par 22 И, ето, Варак гонеше Сисара; и Яил излезе да го посрещне и рече му: Дойди; ще ти покажа мъжа, когото търсиш. И когато свърна у нея, ето, Сисара лежеше мъртъв, с кола в слепите си очи.
\par 23 Така в оня ден Бог покори ханаанския цар Явин пред израилтяните.
\par 24 И ръката на израилтяните постоянно преодоляваше над ханаанския цар Явина, докато унищожиха ханаанския цар Явина.

\chapter{5}

\par 1 В оня ден Девора и Варак Авиноамовият син пееха, говорейки: -
\par 2 За гдето взеха водителството в Израиля военачалниците, За гдето доброволно се предадоха людете, Хвалете Господа.
\par 3 Чуйте, царе! дайте ухо, първенци! Ще пея, аз ще пея, Господу; На Господа Бога Израилева ще пея хвала.
\par 4 Господи, когато излезе Ти от Сиир, Когато тръгна от полето Едом Земята се потресе, също и небето покапа, Ей, облаците покапаха вода.
\par 5 Планините се разтопиха от присъствието Господно, Самият Синай от присъствието на Господа Бога Израилева.
\par 6 В дните на Самегара Анатовия син, В дните на Яил, пътищата бяха напуснати, И пътниците вървяха по пътеки настрана.
\par 7 Престанаха управниците в Израиля, престанаха До като се въздигнах аз Девора, Въздигнах се майка в Израиля.
\par 8 Избраха си нови богове; Тогава настана бой в портите; Но видя ли се щит или копие Между четиридесетте хиляди в Израиля?
\par 9 Сърцето ми е към началниците на Израиля, Които между людете предадоха себе си доброволно. Хвалете Господа!
\par 10 Вие, които яздите на бели осли, Вие, които седите на меки постелки, И вие, които ходите по път, възвестете това!
\par 11 Далеч от кипежа на стрелците, На места гдето черпят вода, Там да възхваляват правдините на Господа, Праведните дела на владичеството Му в Израиля. Тогава людете Господни слязоха при портите.
\par 12 Събуди се, събуди се, Деворо! Събуди се, събуди се, изпей песен! Стани, Вараче, И заплени пленниците си, сине Авиноамов!
\par 13 Тогава направи остатък от людете да владеят благородните; Господ ме направи да владея силните.
\par 14 Които са от корена Ефремов, слязоха против Амалика След тебе, Вениамине, между твоите племена; От Махира слязоха началници, И от Завулона ония, които държат жезъл на повелител.
\par 15 И първенците Исахарови бяха с Девора, Исахар още с Варака, Спуснаха се подир него в долината. При потоците Рувимови Велики бяха сърдечните решения.
\par 16 Защо си седнал между оградите Да слушаш блеянията на стадата? При потоците Рувимови Големи бяха сърдечните изпитания.
\par 17 Галаад мируваше оттатък Иордан; И Дан защо стоеше в корабите? Асир седеше в крайморията И мируваше в заливчетата си.
\par 18 Завулон са люде, които изложиха живота си на смърт. Също и Нефталим, по опасните места на полето,
\par 19 Дойдоха царете, воюваха; Тогас воюваха ханаанските царе В Таанах, близо при водите на Магедон; Парична корист не взеха.
\par 20 О небето воюваха; Звездите от пътищата си воюваха против Сисара.
\par 21 Реката Кисон ги завлече, Старата река, реката Кисон, Стъпкала си мощ, душе моя.
\par 22 Тогаз се строшиха конските копита От стремливото тичане, стремливото тичане на силните им.
\par 23 Кълнете Мироза, рече ангелът Господен, Горчива кълнете жителите му, Защото не дойдоха на помощ Господу, На помощ Господу против силните.
\par 24 Благословена нека бъде повече от всички жени Яил, жената на Хевера кенееца; Повече от всички жени живеещи в шатри нека бъде благословена.
\par 25 Вода поиска той; тя му мляко даде, Масло принесе във великолепна чаша.
\par 26 Ръката си простря към кола, И десницата си към работническия млат; И с млат удари Сисара и промуши му главата, Ей, проби и прониза слепите му очи.
\par 27 При нозете й се повали, падна; простря се; При нозете й се повали, падна; Дето се повали, там и падна мъртъв.
\par 28 Сисаровата майка надничаше през прозореца И викаше през решетката: Защо се бави да дойде колесницата му? Защо закъсняха колелата на колесницата му?
\par 29 Мъдрите нейни госпожи й отговаряха, Даже и тя давеше отговор на себе си:
\par 30 Не са ли намерили да делят користи? По мома, по две моми, на всеки мъж; На Сисара користи от пъстрошарени дрехи, Користи от пъстрошарени везани дрехи, От пъстрошарени дрехи везани от двете страни, за шиите на момите взети в корист.
\par 31 Така да погинат всички твои врази, Господи; А ония, които Те любят, да бъдат като слънцето, когато изгрява в силата си. След това, земята имаше спокойствие четиридесет години.

\chapter{6}

\par 1 И Израилтяните сториха зло пред Господа; и Господ ги предаде в ръката на Мадиама за седем години.
\par 2 И ръката на Мадиама преодоля над Израиля; и поради мадиамците, израилтяните си направиха ония ровове, които са по горите, и пещерите и укрепленията.
\par 3 И след като Израил посееше, мадиамците, амаличаните и източните жители дохождаха и нападаха против него;
\par 4 и, като разполагаха стан против тях, унищожаваха рожбите на земята чак до Газа, и не оставяха храна за Израиля, нито овца, нито говедо, нито осел.
\par 5 Защото дохождаха с добитъка си и с шатрите си, и влизаха в страната многобройни като скакалци; безбройни бяха и те и камилите им, и навлизаха в земята, за да я опустошават.
\par 6 Така Израил изпадна в голямо униние поради мадиамците; затова, израилтяните извикаха към Господа.
\par 7 И когато израилтяните извикаха към Господа поради мадиамците,
\par 8 тогава Господ изпрати на израилтяните един пророк, който им каза: Така говори Господ Израилевият Бог: Аз ви изведох от Египет, и ви изведох от дома на робството;
\par 9 И избавих ви от ръката на египтяните и от ръката на всички, които ви насилваха, и като ги изпъдих от пред вас, дадох ви земята им.
\par 10 И рекох ви: Аз съм Господ вашият Бог; да не почитате боговете на аморейците, в чиято земя живеете. Обаче вие не послушахте гласа Ми.
\par 11 И ангелът Господен дойде та седна под дъба, който е в Офра, и принадлежеше на авиезереца Иоас; а син му Гедеон чукаше жито в лина, за да го скрие от мадиамците.
\par 12 И ангелът Господен му се яви и му каза: Господ е с тебе, мъжо силни и храбри.
\par 13 А Гедеон ме рече: О господине, ако Господ е с нас, то защо ни постигна всичко това? и къде са всичките Му чудеса, за които бащите ни ни разказваха, като думаха: Не изведе ли ни Господ от Египет? Но сега Господ ни е оставил и ни е предел в ръката на мадиамците.
\par 14 И Господ погледна към него и му каза: Иди с тая твоя сила, и ще освободиш Израиля от ръката на мадиамците; не те ли изпратих Аз?
\par 15 А той Му рече: О Господи! с какво ще освобадя аз Израиля? Ето, моето семейство е най-долно между Манасия, и аз съм най-малък в бащиния си дом.
\par 16 Но Господ му рече: Непременно Аз ще бъда с тебе; и ти ще поразиш мадиамците като един човек.
\par 17 Сетне той му каза: Моля Ти се, ако съм придобил Твоето благоволение, покажи ми знамение, за да зная Кой си Ти, Който говориш с мене;
\par 18 моля Ти се, не си отивай от тук, докато не дойда при Тебе и изнеса приноса си и го положа пред Тебе. И Той ме рече: Ще чакам догдето се върнеш.
\par 19 Тогава Гедеон слезе та приготви яре и пресни пити от една ефа брашно; месото тури в кошница, а чорбата тури в гърне, и изнесе ги вън при него под дъба, та го представи.
\par 20 И ангелът Божия му каза: Вземи месото и пресните пити та ги сложи на тоя камък, а чорбата излей. И стори така.
\par 21 Тогава ангелът Господен простря края на жезъла, който беше в ръката му, та досегна месото и пресните пити; и излезе огън из камъка та пояде месото и пресните пити; а след това ангелът Господен си отиде отпред очите му.
\par 22 И като видя Гедеон, че това бе ангел Господен, Гедеон каза: Горко ми, Господи Иеова! защото видях ангела Господен лице с лице.
\par 23 А Господ му каза: Мир на тебе, не бой се; няма да умреш.
\par 24 И Гедеон издигна там олтар на Господа и нараче го Иеовашалом; той е до днес в Офра на авиезерците.
\par 25 И в същата нощ Господ му каза: Вземи бащиния си вол, втория вол седемгодишния, та съсипи Вааловия жертвеник, който се намира у баща ти, и съсечи ашерата, която е при него.
\par 26 После на върха на тая скала издигни олтар на Господа твоя Бог, според наредбата, и вземи втория вол та го принеси за всеизгаряне с дървата на ашерата, която ще съсечеш.
\par 27 И тъй, Гедеон взе десетина души от слугите си та стори както Господ му каза; а, понеже се боеше от бащиния си дом и от градските жители, затова не го стори денем, но го стори нощем.
\par 28 А когато градските жители станаха на утринта, ето, Вааловият жертвеник беше съборен, и ашерата, която беше при него, съсечена, и вторият вол цял изгорен на издигнатия олтар.
\par 29 И казаха си един на друг: Кой е извършил тая работа? И като разпитаха и издириха, рекоха: Гедеон Иоасовият син е извършил тая работа.
\par 30 Тогава градските жители казаха на Иоаса: Изведете сина си да се умъртви, понеже той е съборил Вааловия жертвеник, и понеже е съсякъл ашерата, която беше при него.
\par 31 А Иоас рече на всички, които бяха се дигнали против него: Вие ли ще защитите делото на Ваала? или вие ще го спасите? Който защити неговото дело, нека бъде умъртвен докато е още заран. Ако той е бог, нека сам защити делото си, като са съборили жертвеника му
\par 32 За това в същия ден той нарече сина си Ероваал, като казваше: Нека се съди Ваал с него, като е съборил жертвеника му.
\par 33 Тогава всичките мадиамци и амаличани и източните жители се събраха заедно та преминаха и разположиха стан в долината Езраел.
\par 34 А Господният Дух дойде на Гедеона, и той засвири с тръба; и авиезерците се събраха да го последват.
\par 35 После изпрати вестители до целия Манасия, та и той се събра да го последва; прати вестители и до Асира, до Завулона и до Нефталима, а те дойдоха та ги посрещнаха.
\par 36 И Гедеон рече на Бога: Ако искаш да освободиш Израиля с моята ръка, според както си казал.
\par 37 ето, аз ще туря руно вълна на гумното; ако падне роса само на руното, а цялата почва остане суха, тогаз ще позная, че ще освободиш Израиля с моята ръка, според както си казал.
\par 38 Така биде; защото, като стана рано на утринта, той изстиска руното, и роса потече из руното, пълен леген вода.
\par 39 И Гедеон рече на Бога: Да не пламне гневът Ти против мене, и аз ще продумам само тоя път; да опитам моля Ти се, само тоя път с руното: сега нека остане сухо само руното, а по цялата почва нека падне роса.
\par 40 И Бог стори така през оная нощ: само руното остана сухо, а по цялата почва падна роса.

\chapter{7}

\par 1 Тогава Ероваал (който е Гедеон) стана рано, и всичките люде, които бяха с него, та разположиха стана си при извора Арод; а стана на мадиамците беше на север от тях, в долината до хълма Море.
\par 2 И Господ каза на Гедеона: Людете, които са с тебе, са твърде много, за да предам мадиамците в ръката им, да не би да се възгордее Израил против Мене и да каже: Моята ръка ме избави.
\par 3 Сега, прочее, иди, прогласи на всеослушание пред людете тия думи: Който се страхува е трепери, нека се върне и си отиде от галаадската гора. И върнаха се от людете двадесет и две хиляди души, и останаха десет хиляди.
\par 4 Но Господ каза на Гедеона: Людете пак са много; заведи ги при водата, и там ще ти ги пресея; и за когото ти кажа: Тоя да отиде с тебе, той нека отиде с тебе; а за когото ти кажа: Той да не отиде с тебе, той нека не отиде.
\par 5 И той, заведе людете при водата; и Господа каза на Гедеона: Всеки, който полочи с езика си от водата, както лочи куче, него да поставиш отделно, така и всеки, който се наведе на коленете си, за да пие.
\par 6 И числото на ония, които полочиха, като поднесоха ръка до устата си, беше триста мъже; а всичките останали люде се наведоха на коленете си, за да пият вода.
\par 7 Тогава Господ каза на Гедеона: Чрез тия триста мъже, които полочиха, ще ви избавя, и ще предам мадиамците в ръката ти; а всичките останали люде нека си отидат, всеки на мястото си.
\par 8 И тъй людете взеха храна в ръцете си, и тръбите си; и той изпрати всичките израилтяни, всеки в шатъра му, а тристата мъже задържа при себе си. А мадиамският стан беше под него в долината.
\par 9 И в същата нощ Господ му каза: Стани слез в стана, защото го предадох в ръката ти.
\par 10 Но ако те е страх да слезеш, слез със слугата си Фура в стана,
\par 11 и ще чуеш какво говорят; и след това ръцете ти ще се подкрепят, за да слезеш в стана. И тъй, той слезе със слугата си Фура до външните части на въоръжените в стана.
\par 12 А мадиамците и амаличаните и всичките източни жители бяха разпрострени в долината по множество като скакалци; и камилите им по множество бяха безбройни, като пясъка край морето.
\par 13 И като пристигна Гедеон, ето, един човек разказваше сън на другаря си, като говореше: Ето, видях сън, и ето, пита от ечемичен хляб, като се търкаляшие в мадиамския стан, дойде до шатъра и го удари, та падна; и прекатури го така, че шатърът се събори.
\par 14 И другарят му в отговор рече: Това не е друго освен сабята на Гедеона Иоасовия син, израилтянина; Бог предаде в ръката му Мадиама и цялото ни множество.
\par 15 И като чу това Гедеон разказа на съня и тълкуването му поклони се; и като се върна в Израилевия стан, рече: Станете, защото Господ предаде в ръката ви мадиамското множество.
\par 16 И раздели тристата мъже на три дружини, и даде тръби в ръцете на всичките, тоже и празни водоноси, с факли във водоносите.
\par 17 И рече им: Гледайте мене; и каквото правя аз , правете и вие подобно; и, ето, когато стигна при външните части на стана, то каквото направя аз, това да направите и вие.
\par 18 Когато засвиря с тръбата, аз и всичките, които са с мене, тогава да засвирите и вие с тръбите от всяка страна на целия стан, и да извикате: За Господа и за Гедеона!
\par 19 И тъй, Гедеон и стоте мъже, които бяха с него, дойдоха при външната част на стана, около началото на средната стража, тъкмо когато бяха поставили часовите; и засвириха с тръбите и строшиха водоносите, които бяха в ръцете им.
\par 20 Па и трите дружини зесвириха с тръбите и, като строшиха водоносите, държаха факлите в левите си ръце, за да свирят; и викаха: Сабя Господна и Гедеонова!
\par 21 Всеки застана на мястото си около стана; и целият стан се разтича, като викаха и бягаха.
\par 22 Защото, като засвириха с тристата тръби, Господ обърна сабята на всеки човек против ближния му в целия стан; и войската избяга до Ветасета към Зерерат, до околностите на Авелмеола при Тават.
\par 23 Тогава израилтяните от Нефталима, от Асира и от целия Манасия се събраха та преследваха Мадиама.
\par 24 И Гедеон прати вестители по цялата Ефремова хълмиста земя да кажат: Слезте против Мадиама и ловете пред тях водите на Иордан до Вет-вара. И тъй, всичките ефремци се събраха та заловиха водите на Иордан до Вет-вара.
\par 25 И хванаха двамата мадиамски началници, Орива и Зива, Орива убиха при скалата Орив, а Зива убиха при лина Зив. И гониха Мадиама. И донесоха главите на Орива и Зива на Гедеона оттатък Иордан.

\chapter{8}

\par 1 Тогава ефремците казаха на Гедеона : Какво ни стори ти, че не ни повика, когато отиде да воюваш против Мадиама? И караха се силно с него.
\par 2 А той им рече: Що съм извършил аз сега в сравнение с вас? Ефремовият пабирък не е ли по-желателен от Авиезеровия гроздобер?
\par 3 В нашите ръце Бог предаде мадиамските началници, Орива и Зива; и що съм могъл аз да извърша в сравнение с вас? Тогава гневът им се укроти към него, когато каза това.
\par 4 И като дойде Гедеон при Иордан, той премина с тристата мъже, които бяха с него, уморени, но пак преследвайки.
\par 5 И рече на сокхотските жители: Дайте, моля, няколко хляба на людете, които вървят подир мене защото са изнемощели; а аз гоня мадиамските царе Зевей и Салман.
\par 6 А сокхотските първенци му казаха: Ръцете на Зевея и на Салмана в ръката ти ли са вече та да дадем хляб на войската ти?
\par 7 А Гедеон рече: Затова, когато Господ предаде Зевея и Салмана в ръката ми, тогава аз ще разкъсам месата ви с пустинните тръни и с глоговете.
\par 8 От там отиде във Фануил и каза на жителите му същото; а фануилските мъже му отговориха, както бяха отговорили сокхотските мъже.
\par 9 А той говори на фануилските мъже, казвайки: Когато се върна с мир, аз ще съборя тая кула.
\par 10 А Зевей и Салман бяха в Каркор и войските им с тях, около петдесет хиляди души, всичките колкото бяха оцелели от цялата войска на източните жители; защото бяха паднали сто и двадесет хиляди мъже, които теглеха нож.
\par 11 И Гедеон отиде през пътя на ония, които живееха в шатри на изток от Нова и на Иогвея, та порази множеството; защото множеството беше в безгрижност.
\par 12 А Зевей и Салман побягнаха; но той ги гони, и хвана двамата мадиамски царе Зевея и Салмана, и разби цялото множество.
\par 13 Тогава Гедеон Иоасовият син се върна от войната през нагорнището на Херес.
\par 14 И като хвана един момък от сокхотските мъже, разпита го, и той му описа първенците на Сокхот и старейшините му, седемдесет и седем мъже.
\par 15 И Гедеон дойде при сокхотските мъже та рече: Ето Зевея и Салмана, за които ми се подиграхте като казахте: Ръцете на Зевея и на Салмана в ръката ти ли са вече та да дадем хляб на човеците ти, че били изнемощели?
\par 16 И взе градските старейшини, и пустинните тръни и глогове, та наказа с тях сокхотските мъже.
\par 17 Също и събори кулата на Фануила и изби градските мъже.
\par 18 Тогава каза на Зевея и на Салмана: Какви бяха ония човеци, които убихте в Тавор? А те отговориха: Какъвто си ти, такива бяха и те; всеки приличаше на царски син.
\par 19 А той каза: Мои братя бяха, синовете на моята майка; заклевам се в живота на Господа, ако бяхте опазили живота им, аз не бих ви убил.
\par 20 И рече на първородния си, Етер: Стани, убий ги. Но младежът не изтегли меча си, защото се боеше, бидейки още млад.
\par 21 Тогава Зевей и Салмана казаха: Стани ти, та не нападни; защото според човека е и силата му. И тъй Гедеон стана та уби Зевея и Салмана, и взе полумесецообразните украшения , които бяха на вратовете на камилите им.
\par 22 Тогава израилевите мъже казаха на Гедеона: Владей над нас, и ти, и синът ти, и внукът ти, защото ти ни освободи от ръката на Мадиама.
\par 23 А Гедеон каза: Нито аз ще владея над вас, нито синът ми ще владее над вас; Господ ще владее над вас.
\par 24 Гедеон, обаче, им рече: Едно нещо ще поискам от вас, - да ми дадете всеки обеците от користите си. (Защото неприятелите , понеже бяха исмаиляни, носеха златни обеци).
\par 25 И те отговориха: На драго сърце ще ги дадем. И като простряха дреха, всеки хвърляше там обеците от користите си.
\par 26 И теглото на златните обеци, които поиска, беше хиляда и седемстотин златни сикли , освен полумесецообразните украшения , огърлията и моравите дрехи, които бяха върху мадиамските царе, и освен веригите, които бяха около вратовете на камилите им.
\par 27 И от тях Гедеон направи ефод, който положи в града си, в Офра; а там целият Израил блудствува след него; и той стана примка на Гедеона и на дома му.
\par 28 Така Мадиам се покори пред израилтяните, и не дигна вече глава. И земята имаше спокойствие четиридесет години в дните на Гедеона.
\par 29 Тогава Ероваал, Иоасовият син, отиде и седна в дома си.
\par 30 И Гедеон имаше седемдесет сине, родени от самия него, защото имаше много жени.
\par 31 Също и наложницата му, която беше в Сихем, му роди син, когото той нарече Авимелех.
\par 32 И Гедеон, Иоасовият син, умря в честита старост, и биде погребан в гроба на баща си Иоаса, в Офра на авиерците.
\par 33 А когато умря Гедеон, израилтяните пак се отвърнаха, и, като блудствуваха след ваалимите, поставиха си Ваалверита за бог.
\par 34 И израилтяните не си спомниха Господа своя Бог, Който ги бе избавил от ръката на всичките им околни неприятели;
\par 35 нито показаха благост към дома на Ероваала (който е Гедеон) съответно на всичките добрини, които той бе сторил на Израиля.

\chapter{9}

\par 1 След това, Авимелех, Еровааловият син, отиде в Сихем при братята на майка си, та говори на тях и на цялото семейство на бащиния дом на майка си, като рече:
\par 2 Кажете, моля, на всеослушание пред всичките сихемски мъже; Кое е по-добре за вас, да владеят над вас всичките Ероваалови синове, седемдесет мъже, или да владее над вас един мъж? Помнете още, че аз съм ваша кост и ваша плът.
\par 3 Братята, прочее, на майка му изказаха всички тия думи за него на всеослушание пред всичките сихемски мъже; и сърцата им склониха към Авимелеха, защото рекоха: Той ни е брат.
\par 4 И дадоха му седемдесет сребърници от капището на Ваалверита; и с тях Авимелех нае едни никакви и развалени мъже, които вървяха подир него.
\par 5 И, като отиде в бащиния си дом в Офра, изкла върху един камък братята си Еровааловите синове, седемдесет души; само, че най-младият Ероваалов син, Иотам, оцеля, защото се скри.
\par 6 Тогава всичките сихемски мъже и целият дом Милов се събраха, и отивайки, поставиха Авимелеха цар, близо при дъба на гарнизона, който бе в Сихем.
\par 7 А когато се извести това на Иотама, той отиде и застана на върха на хълма Гаризин, и като издигна гласа си, извика и каза им: Послушайте ме, сихемски мъже, за да ви послуша и Бог:
\par 8 еднъж дърветата отишли да помажат цар, който да ги владее, и рекли на маслината: Царувай над нас.
\par 9 Но маслината им отговорила: Да оставя ли аз тлъстината си, чрез която отдавам почит на Бога и на човека, та да ида да се развявам над дърветата?
\par 10 Тогава дърветата рекли са смокинята: Дойди и царувай над нас.
\par 11 Но смокинята им отговорила: Да оставя ли сладостта си и добрия си плод, та да ида да се развявам над дърветата?
\par 12 После дърветата рекли на лозата: Дойди та царувай над нас.
\par 13 Но лозата им отговорила: Да оставя ли виното, което весели Бога и човеците, та да ида да се развявам над дърветата?
\par 14 Сетне всичките дървета рекли на тръна: Дойди, та царувай над нас.
\par 15 А трънът рекъл на дърветата: Ако наистина ме помазвате цар над вас, дойдете, прибегнете под сянката ми; но ако не, нека излезе огън из тръна и изпояде ливанските кедри.
\par 16 Сега, прочее, ако сте постъпили вярно и справедливо, като сте поставили Авимелеха цар, и ако сте се обходили добре с Ероваала и с дома му, и ако сте му сторили, според както извършеното от ръцете му заслужава, -
\par 17 (защо баща ми воюва за вас и тури в опасност живота си та ви избави от ръката на Мадиама;
\par 18 а днес сте въстанали против бащиния ми дом, и сте изклали върху един камък синовете му, седемдесет мъже, и сте поставили Авимелеха, син на слугинята му, цар на сихемските мъже, защото ви е брат), -
\par 19 ако, прочее, днес сте постъпили вярно и справедливо с Ероваала и дома му, то радвайте се на Авимелеха, па нека се радва и той с вас!
\par 20 Но ако не, нека излезе огън из Авимелеха и изпояде сихемските мъже и Миловия дом; и нека излезе огън из сихемските мъже и из Миловия дом и изпояде Авимелеха!
\par 21 Тогава Иотам се спусна и побягна, и като отиде във Вир, живееше там, поради страха от брата си Авимелеха.
\par 22 И Авимелех началствува над Израиля три години.
\par 23 А Бог прати злобен дух между Авимелеха и сихемските мъже, та сихемските мъже измениха на Авимелеха,
\par 24 за да дойде наказание за насилието, извършено над седемдесетте Ероваалови синове, върху Авимелеха брата им, който ги изкла, за да се наложи кръвта им върху него и върху сихемските мъже, които подкрепиха ръцете му, за да изколи братята си.
\par 25 Сихемските мъже, прочее, поставиха засади против него по върховете на хълмовете, та обираха всички, които минаваха край тях по пътя. И това се извести на Авимелеха.
\par 26 В това време дойде Гаал, Еведовият син и братята му, та заминаха в Сихем; и сихемските мъже се довериха на него.
\par 27 И като излязоха на полето, обраха лозята си, изтъпкаха гроздето и се развеселиха, и отидоха в капището на бога си, ядоха и пиха, и проклеха Авимелеха.
\par 28 Тогава Гаал, Еведовият син, каза: Кой е Авимелех, и кой е Сихем, та да му слугуваме? Не е ли той Ероваалов син? и не е ли Зевул настойника му? Слугите на мъжете на Сихемовия баща Емора; а нему защо да слугуваме ние?
\par 29 Дано бяха тия люде под моята ръка! тогава аз бих изгонил Авимелеха. И рече на Авимелеха: Увеличи войската си и излез!
\par 30 И когато чу Зевул, управителят на града, думите на Гаал Еведовия син, гневът му пламна.
\par 31 И прати тайно вестители до Авимелеха да кажат: Ето, Гаал Еведовият син и братята му са дошли в Сихем; и, ето, те подигат града против тебе.
\par 32 Затова, стани през нощта, ти и людете, които са с тебе, та постави засади по полето.
\par 33 И на утринта, щом изгрее слънцето, стани рано и спусни се върху града; и, ето, когато Гаал и людете, които са с него, излязат против тебе, тогава ти му направи, каквото ти дойде отръка.
\par 34 И тъй, през нощта Авимелех стана и всичките люде, които бяха с него, и поставиха четири дружини в засада против Сихем.
\par 35 И Гаал, Еведовият син, излезе та застана във входа на градската порта; и Авимелех и людете, които бяха с него, станаха от засадата.
\par 36 А когато видя людете, Гаал рече на Зевула: Ето, люде слизат от върховете на хълмовете. А Зевул му каза: Сянката на хълмовете ти се вижда като човеци.
\par 37 Пак Гаал проговори, казвайки: Виж, люде слизат посред местността и една дружина иде покрай дъба Маоненим.
\par 38 Тогава Зевул му каза; Где е сега хвалението ти, Гдето рече: Кой е Авимелех та да му слугуваме? Не са ли тия людете, които ти презираше? Излез, прочее, сега, та се бий с тях.
\par 39 Тогава Гаал излезе на чело на сихемските мъже, та се би с Авимелеха.
\par 40 И Авимелех го погна; и той побягна пред него, и мнозина падаха мъртви чак до върха на портата.
\par 41 И Авимелех остана в Арума; а Зевул изпъди Гаала и братята му, за да не живеят в Сихем.
\par 42 А на утринта людете излязоха на полето; и това се извести на Авимелеха.
\par 43 Тогава той взе людете си та ги раздели на три дружини, и постави засади на полето; и когато видя, че, ето, людете излизаха из града, стана против тях та ги порази.
\par 44 И Авимелех и дружината, която беше с него, се спуснаха и застанаха във входа на градската порта, а двете дружини се хвърлиха върху всичките, които бяха по полето та ги порази.
\par 45 И Авимелех, като се биеше против града през целия ден, превзе града и изби людете, които бяха в него, разори града и пося го със сол.
\par 46 А всичките мъже на сихемската кула, като чуха това, влязоха в укреплението на капището на бога си Верита.
\par 47 И извести се на Авимелеха, че всичките мъже от сихемската кула се били събрали.
\par 48 Тогава Авимелех се качи на гората Салмон, той и всичките люде, които бяха с него; и Авимелех взе брадвата в ръка та отсече клон от дърво, дигна го, и като го тури на рамото си, каза на людете, които бяха с него: Каквото видяхте, че направих аз, побързайте да направите и вие като мене.
\par 49 И така, всичките люде отсякоха, всеки клона си, и като отидоха подир Авимелеха натрупаха ги на укреплението, и, като бяха човеците вътре, запалиха укреплението, тъй щото и всичките мъже от сихемската кула умряха, около хиляда мъже и жени.
\par 50 След това Авимелех отида в Тевес та разположиха стан против Тевес и го превзе.
\par 51 Но всред града имаше яка кула, гдето прибягнаха всичките мъже и жени, и всичките градски жители, та се затвориха, и качиха се на покрива на кулата.
\par 52 Но Авимелех, като стигна до кулата, би се против нея, и се приближи до вратата на кулата, за да я изгори с огън.
\par 53 Тогава една жена хвърли един горен воденичен камък на Авимелеховата глава та му строши черепа.
\par 54 И той бърже извика към момъка, оръженосеца си, и му каза: Изтегли меча си та ме убий, за да не рекат за мене: Жена го уби. И момъкът му го прободе, та умря.
\par 55 И Израилевите мъже, като видяха, че Авимелех умря, разотидоха се, всеки на мястото си.
\par 56 Така бог въздаде на Авимелеха злодеянието, което стори на баща си, като уби седемдесетте си братя.
\par 57 Тоже Бог въздаде на главите на сихемските мъже всичките им злодеяния; и дойде на тях проклетията, произнесена от Иотама Еровааловия син.

\chapter{10}

\par 1 А след Авимелеха издигна се да избави Израиля един Исахаров мъж Тола, син на Фуя, син на Додо; и той живееше в Самир у Ефремовата хълмиста земя.
\par 2 И като съди Израиля двадесет и три години, умря, и бе погребан в Самир.
\par 3 А подир него се издигна галаадецът Яир, който съди Израиля двадесет и две години.
\par 4 Той имаше тридесет сина, които яздеха на тридесет осела; и имаха тридесет града, намиращи се в галаадската земя, които и до днес се наричат Яирови паланки.
\par 5 И Яир умря и бе погребан в Камон.
\par 6 И израилтяните пак сториха зло пред Господа, като служеха на ваалимите, на астартите, на сирийските богове, на сидонските богове, на моавските богове, на боговете на амонците и на боговете на филистимците, и оставиха Господа и не служиха Нему.
\par 7 Затова гневът на Господа пламна против Израиля и Той ги предаде в ръцете на филистимците и в ръцете на амонците.
\par 8 О тая година те измъчваха и притесняваха израилтяните осемнадесет години, всичките израилтяни, които бяха оттатък Иордан в земята на аморейците, която е в Галаад.
\par 9 При това, амонците преминаха Иордан, за да воюват против Юда, против Вениамина и против Ефремовия дом; така че Израил се намираше в крайно утеснение.
\par 10 Тогава израилтяните извикаха към Господа, като казваха: Съгрешихме Ти, защото оставихме нашия Бог та служихме на ваалимите.
\par 11 А Господ каза на израилтяните: Не избавих ли ви от египтяните, от аморейците, от амонците и от филистимците?
\par 12 Също и когато сидонците, амаличаните и маонците ви притесняваха, и извикахте към Мене, Аз ви избавих от ръката им.
\par 13 А въпреки това, вие Ме оставихте та служихте на други богове; за туй няма да ви избавям вече.
\par 14 Идете, викайте към боговете, които сте си избрали; те нека ви избавят във време на утеснението ви.
\par 15 А израилтяните казаха на Господа: Съгрешихме; стори ни каквото Ти е угодно; само избави ни днес, молим Ти се.
\par 16 И те отмахнаха отпред себе си чуждите богове та служиха на Господа; и Неговата душа се смили за окаянието на Израиля.
\par 17 Тогава Амонците се събраха та разположиха стан в Галаад. Събраха се израилтяните та разположиха стан в Масфа.
\par 18 И людете и галаадските началници си рекоха едни на други: Който започне да се бие с амонците, той ще стане началник на всичките галаадски жители.

\chapter{11}

\par 1 И галаадецът Ефтай беше силен и храбър мъж; той беше син на една блудница; а Ефтаевият баща беше Галаад.
\par 2 И жената на Галаад му роди синове; а когато пораснаха синовете на жена му, те изпъдиха Ефтая, като му казаха: Ти няма да наследиш в бащиния ни дом, защото си син на чужденка.
\par 3 Затова Ефтай побягна от братята си, та живееше в земята Тов; и при Ефтая се събраха безделници, които и излизаха с него.
\par 4 А след известно време амонците воюваха против Израиля.
\par 5 И когато се биеха амонците против Израиля, галаадските старейшини отидоха да доведат Ефтая от земята Тов,
\par 6 като казаха на Ефтая: Дойди та ни стани началник, за да се бием против амонците.
\par 7 А Ефтай рече на галаадските старейшини: Не ме ли намразихте, и не ме ли изпъдихте им бащиния ми дом? А защо сте дошли при мене сега, когато сте на тясно?
\par 8 И галаадските старейшини казаха на Ефтая: За това сме се върнали сега при тебе, за да дойдеш с нас и да се биеш против амонците; и ти ще станеш за нас началник над всичките галаадски жители.
\par 9 И Ефтай каза на галаадските старейшини: Ако ме върнете у дома, за да воювам против амонците, и Господ ми ги предаде, наистина ли ще стана началник над вас?
\par 10 И галаадските старейшини казаха на Ефтая: Господ нека бъде свидетел помежду ни! сигурно ще направим според както си казал.
\par 11 Тогава Ефтай отиде с галаадските старейшини, и людете го поставиха глава и началник над себе си. И Ефтай изговори всичките си думи пред Господа в Масфа.
\par 12 Тогава Ефтай прати посланици до царя на амонците да кажат: Каква работа имаш с мене, та си дошъл при мене да се биеш против земята ми?
\par 13 И царят на амонците отговори на Ефтаевите посланици: Защото Израил, когато идеше от Египет, отне земята ми от Арнон до Явок и до Иордан; сега, прочее, върни тия земи по мирен начин.
\par 14 Тогава Ефтай пак прати посланици до царя на амонците да му рекат:
\par 15 Така казва Ефтай: Израил не е отнел моавската земя, нито земята на амонците;
\par 16 но когато Израил идеше от Египет и отиваше през пустинята към Червеното море, и беше дощъл в Кадис,
\par 17 тогава Израил прати посланици до едомския цар да рекат: Нека премина, моля през земята ти. А едомският цар не послуша. Прати още и до моавския цар; но и той не склони. Затова Израил остана в Кадис.
\par 18 Тогава отиде през пустинята, та обиколи едомската земя и моавската земя, и, като мина по източната страна на моавската земя, разположи стан оттатък Арнон; но не възлезе вътре в моавските предели, защото Арнон беше моавската граница.
\par 19 След това Израил прати посланици до аморейския цар Сион есевонския цар, та Израил му каза: Нека преминем, молим те, през земята ти до нашето място.
\par 20 Но Сион не се довери на Израиля та да го пусне да мине през пределите му; а напротив Сион събра всичките си люде и, като разположи стан в Яса, воюва против Израиля.
\par 21 И Господ Израилевият Бог предаде Сиона и всичките му люде в ръката на Израиля, и те ги поразиха; и така Израил завладя цялата земя на аморейците, жителите на оная земя.
\par 22 Те завладяха всичките предели на аморейците, от Арнон до Явок, и от пустинята дори до Иордан.
\par 23 И сега, когато Господ Израилевият Бог е изгонил аморейците отпред людете Си Израиля, ти ли ще притежаваш земята им?
\par 24 Не ще ли притежаваш онова, което твоят бог Хамос ти дава да притежаваш? Също и ние ще притежаваме земите на всички, които Иеова нашият Бог е изгонил отпред нас.
\par 25 И сега, ти в какво си по-добър от моавския цар Валак Сеяфоровия син? Борил ли се е той някога с Израиля, или воювал ли е някога против него?
\par 26 Триста години са откак се е заселил Израил в Есевон и в селата му, в Ароир и в селата му, и във всичките градове, които са разположени край Арнон; а защо през това време не сте ги превзели обратно?
\par 27 Прочее, аз не съм ти съгрешил; но ти ме онеправдаваш като воюваш против мене. Господ Съдията нека съди днес между израилтяните и амонците.
\par 28 Но царят на амонците не послуша думите, които Ефтай му изпрати.
\par 29 Тогава дойде Господният Дух на Ефтая; и той мина през Галаад и Манасия, мина и през галаадската Масфа, и от галаадската Масфа мина против амонците.
\par 30 И Ефтай направи обрек Господу, като каза: Ако наистина предадеш амонците в ръката ми,
\par 31 тогава онова, което излезе из вратата на къщата ми да ме посрещне, когато се върна с мир от амонците, ще бъде Господу, и ще го принеса всеизгаряне.
\par 32 И тъй, Ефтай замина към амонците, за да воюва против тях, и Господ ги предаде в ръката му;
\par 33 та ги порази от Ароир до Минит, двадесет града, и до Авел-Керамим, с твърде голямо поражение. Така се покориха амонците пред израилтяните.
\par 34 Тогава, като дойде Ефтай у дома си в Месфа, ето, дъщеря му излизаше да го посрещне с тъпанчета и хора; и тя му беше едничка; освен нея нямаше ни син ни дъщеря.
\par 35 И като я видя той, раздра дрехите си и каза: Горко ми, щерко моя! ти съвсем си ме унижила, и си от ония, които ме смущават; защото аз изговорих думи към Господа, и не мога да отстъпя.
\par 36 А тя му каза: Тате, понеже си изговорил думи към Господа, стори с мене според това, тъй като Господ извърши за тебе възмездие на неприятелите ти амонците.
\par 37 Каза още на баща си: Нека ми се направи това: остави ме два месеца да ида да ходя по хълмовете, аз и другарките ми, и да оплача девството си.
\par 38 И той рече: Иди, И пусна я за два месеца. И тя отиде с другарките си, та оплака девството си по хълмовете.
\par 39 И в края на двата месеца се върна при баща си; и той стори с нея според обрека, който бе направил. И тя не беше познала мъж. От това стана обичай в Израил,
\par 40 да ходят Израилевите дъщери всяка година, четири дена в годината, за да оплакват дъщерята на галаадеца Ефтай.

\chapter{12}

\par 1 Тогава Ефремовите мъже се събраха и, като преминаха към север, рекоха на Ефтая: Ти защо отиде да воюваш против амонците, а нас не повика да дойдем с тебе? Ние ще изгорим къщата ти с огън, и тебе в нея.
\par 2 А Ефтай им рече: Аз и людете ми имахме голям спор с амонците; и аз ви повиках, и вие не ме избавихте от ръката им.
\par 3 И като видях, че не ме избавихте, турих в опасност живота си, като отидох против амонците; и Господ ги предаде в ръката ми. И сега защо сте дошли при денес да се биете против мене?
\par 4 Тогава Ефтай събра всичките галаадски мъже, та се би против Ефрема; и галаадските мъже поразиха ефремците, защото тия рекоха: Бежанци от Ефрема сте вие, галаадци, живеещи между ефремците и манасийците!
\par 5 Галаадците взеха бродовете на Иордан към Ефремовата земя ; и когато някой от ефремските бежанци кажеше: Оставете ме да премина, тогава галаадските мъже му казваха: Ти ефремец ли си?
\par 6 Ако речеше: Не съм, тогава му думаха: Речи Шиболет, а той казваше: Сиболет, защото не можеше да го произнесе право. Тогава го хващаха, та го заколваха при бродовете на Иордан. И в онова време паднаха от Ефрема четиридесет и две хиляди души.
\par 7 И Ефтай съди Израиля шест години. Тогава Ефтай, галаадецът, умря, и бе погребан в един от галаадските градове.
\par 8 А подир него, Ивцан от Витлеем стана съдия в Израиля.
\par 9 Той имаше тридесет сина, и тридесет дъщери, които омъжиха навън; и взе от вън тридесет дъщери за синовете си. И съди Израиля седем години.
\par 10 И Ивцан умря, и бе погребан във Витлеем.
\par 11 След него, завулонецът Елон стана съдия в Израиля, и съди Израиля десет години.
\par 12 И Елон завулонецът умря, и бе погребан в Еалон в Завулоновата земя.
\par 13 А след него, Авдон, син на пиратонеца Илела, стана съдия в Израиля.
\par 14 Той имаше четиридесет сина и тридесет внуци, които яздеха на седемдесет ослета; и съди Израиля осем години.
\par 15 И Авдон, син на пиратонеца Илела, умря, и бе погребан в Пиратон в Ефремовата земя, на амаликската хълмиста страна.

\chapter{13}

\par 1 И израилтяните пак сториха зло пред Господа; и Господ ги предаде в ръката на филистимците за четиридесет години.
\par 2 И имаше един човек от Сарая, от Дановото племе, по име Маное, чиято жена бе бездетна, която не раждаше.
\par 3 И ангел Господен се яви на жената и каза й: Ето, сега си бездетна и не раждаш; но ще зачнеш и ще родиш син.
\par 4 Затова, пази се сега да не пиеш вино или спиртно питие, и да не ядеш нищо нечисто.
\par 5 Защото, ето, ще заченеш и ще родиш син; и бръснач да не мине през главата му, защото още от раждането си детето ще бъде Назирей Богу; той ще почне да избавя Израиля от ръката на филистимците.
\par 6 Тогава жената отиде та яви на мъжа си, казвайки: Един Божий човек дойде при мене, чието лице беше като лицето на ангела Божий, много страшно; и аз не го попитах от къде е, нито той ми яви името си.
\par 7 Но рече ми: Ето, ще заченеш и ще родиш син; затова, да не пиеш вино, нито спиртно питие, и да не ядеш нищо нечисто; защото, от рождението си , до деня на смъртта си, детето ще бъде Назирей Богу.
\par 8 Тогава Маное се помоли Господу, казвайки: Моля ти се, Господи, нека дойде пак при нас Божият човек, когото си пратил и нека ни научи що да сторим с детето, което ще се роди.
\par 9 И Бог послуша Маноевия глас: и ангелът Божи пак дойде при жената, като седеше на нивата; а мъжът й Маное не беше с нея.
\par 10 И жената се завтече та побърза да яви на мъжа си, като му каза: Ето, яви ми се оня човек, който дойде при мен завчера.
\par 11 Тогава Маное стана та отиде подир жена си, и, като дойде при човека, каза му: Ти ли си оня човек, който си говорил на тая жена? И той рече: Аз.
\par 12 И рече Маное: Когато вече се сбъдне това, което ти каза, как трябва да се направлява детето, и какво да прави то?
\par 13 И ангелът Господен рече на Маноя: Нека се пази жената от всичко, за което й говорих;
\par 14 да не яде от нищо, що произлиза от лозе, нито да пие вино или спиртно питие, и да не яде нищо нечисто: всичко що й заповядах нека опази.
\par 15 Тогава Маное каза на ангела Господен: Моля, нека те задържим, за да ти сготвим яре.
\par 16 Но ангелът Господен каза на Маноя: И да ме задържиш няма да ям хляба ти; и ако искаш да направиш всеизгаряне, принеси го Господу. (Защото Маное не позна, че това беше ангелът Господен).
\par 17 Тогава Маное каза на ангела Господен: Как ти е името, за да те почетем, когато се сбъдне това, което ти каза?
\par 18 А ангелът Господен му рече: Защо питаш за името ми, то е тайно.
\par 19 Тогава Маное взе ярето и хлебния принос та ги принесе Господу на камъка; и, като гледаха Маное и жена му, ангелът постъпваше чудно;
\par 20 защото, когато се издигаше пламък от олтара към небето, то и ангелът Господен се издигна в пламъка на олтара: а Маное и жена му, като гледаха, паднаха с лице не земята.
\par 21 Но ангелът Господен беше невидим вече за Маноя и за жена му; и тогава Маное позна, че това беше ангел Господен.
\par 22 И Маное каза на жена си: Непременно ще умрем, защото видяхме Бога.
\par 23 А жена му рече: Ако Господ иска да ни умори, не щеше да приеме всеизгаряне и хлебен принос от ръката ни, нито щеше да ни изяви всичко това, нито би ни съобщил такива неща в това време.
\par 24 И жената роди син и нарече го Самсон; и детето порасна и Господ го благослови.
\par 25 И Господният Дух почна да го подбужда в Маханедан, между Сарая и Естаол.

\chapter{14}

\par 1 И Самсон слезе в Тамнат и видя в Тамнат една жена от филистимските дъщери.
\par 2 И отиде да извести на баща си и на майка си, казвайки: Видях в Тамнат една жена от филистимските дъщери; сега, прочее, вземете ми я за жена.
\par 3 А баща му и майка му му рекоха: Няма ли някоя жена между дъщерите на братята ти, или между всичките ми люде, та отиваш да вземеш жена от необрязаните филистимци? А Самсон рече на баща си: Нея ми вземи; защото тя ми е угодна.
\par 4 Но баща му и майка му не знаеха, че това беше от Господа, и че той търсеше причина против филистимците; защото по онова време филистимците владееха над Израиля.
\par 5 Тогава Самсон слезе с баща си в Тамнат; а като стигнаха до тамнатските лозя, ето, едно лъвче ревеше против него.
\par 6 И Господният Дух дойде със сила на него; и той го разкъса както би разкъсал яре, и то без да има нищо в ръката си; но не каза на баща си и на майка си що бе сторил.
\par 7 И слезе та говори с жената: и тя бе угодна на Самсона.
\par 8 И след няколко дни, като се върна да я вземе, той се отби от пътя , за да види трупа на лъва; и ето рой пчели в лъвовия труп, и мед.
\par 9 И взе от него в ръцете си, па вървеше и ядеше, и като застигна баща си и майка си даде и на тях, та ядоха; но не им каза, че беше взел мед от лъвовия труп.
\par 10 И баща му слезе при жената; и там Самсон направи угощение, защото така правеха момците.
\par 11 И като го видяха филистимците доведоха тридесет другари, за да бъдат с него.
\par 12 И рече им Самсон: Сега ще ви предложа една гатанка. Ако можете да ми я отгатнете през седемте дена на угощението, и да я намерите, тогава аз ще ви дам тридесет ленени ризи и тридесет дрехи за премяна;
\par 13 но ако не можете да ми я отгатнете, тогава вие ще ми дадете тридесет ленени ризи и тридесет дрехи за премяна. И те му казаха: Предложи гатанката си, за да я чуем.
\par 14 И той им рече: - От ядящия излезе ястие, И от силния излезе сладост. А до третия ден те не можаха да отгатнат гатанката.
\par 15 Тогава на седмия ден казаха на Самсоновата жена: Предумай мъжа си да ни каже гатанката, за да не би да изгорим тебе и бащиния ти дом с огън. За това ли ни поканихте, да ни оберете? не е ли така ?
\par 16 И тъй, жената на Самсона заплака пред него, като каза: Ти само ме мразиш, и не ме обичаш; предложил си гатанка на ония, които са от людете, ми, а на мене не си я казал. А той й рече: Ето, нито на баща си и на майка си не съм я казал, та на тебе ли ща я кажа?
\par 17 Но тя плачеше пред него през седемте дена, в които ставаше угощението им; а на седмия ден й я откри, защото му бе много досадила и тя каза гатанката на мъжете от людете си.
\par 18 Тогава на седмия ден, преди да зайде слънцето, градските мъже му рекоха: - Що е по-сладко от мед? И що е по-яко от лъв? А той им рече: Ако не бяхте орали с моята юница, не бихте отгатнали гатанката ми.
\par 19 И Господният Дух дойде със сила на него; и той слезе в Аскалон та изби тридесет мъже от тях, и като взе облеклата им, даде премените на ония, които отгатнаха гатанката. И гневът му пламна, и отиде в бащиния си дом.
\par 20 А Самсоновата жена се омъжи за другаря му, който му беше приятел.

\chapter{15}

\par 1 И след известно време, когато се жънеше пшеницата, Самсон посети жена си с едно яре и рече: Ще вляза при жена си в спалнята. Но баща й не го остави да влезе.
\par 2 И баща й каза: Наистина аз си рекох, че ти съвсем си я намразил; затова я дадох на другаря ти. По-малката й сестра не е ли по-хубава от нея? Вземи нея, моля, вместо другата.
\par 3 А Самсон им каза: Този път ще бъда невинен спрямо филистимците като из сторя и аз зло.
\par 4 И тъй, Самсон отиде та хвана триста лисици, и, като взе главни, обърна опашка към опашка, и тури по една главня в средата между двете опашки.
\par 5 И като запали главните, пусна лисиците по сеитбите на филистимците и изгори копните и непожънатите класове, тоже и маслините.
\par 6 Тогава филистимците рекоха: Кой стори това? И думаха: Самсон зетят на тамнатеца, за гдето този взе жена му та я даде на другаря му. Затова, филистимците дойдоха та изгориха нея и баща й с огън.
\par 7 А Самсон им каза: Щом правите така, аз ще си отмъстя на вас, и само тогава ще престана.
\par 8 И порази ги с голямо клане на длъж и на шир; тогава слезе та седна в разцепа на скалата Итам.
\par 9 Тогава филистимците възлязоха та разположиха стан в Юда и се разпростряха в Лехий.
\par 10 И юдейците им рекоха: Защо сте дошли против нас? И те казаха: Дойдохме да вържем Самсона, за да му направим както направи и той на нас.
\par 11 Затова, три хиляди мъже от Юда слязоха в разцепа на скалата Итам, та рекоха на Самсона: Не знаеш ли, че филистимците ни станаха господари? Що е, прочее, това, което ти си ни сторил? А той им рече: Както ми сториха те, така им сторих и аз.
\par 12 А те му рекоха: Ние сме слезли да те вържем, за да те предадем в ръката на филистимците. И Самсон им каза: Закълнете ми се, че вие няма сами да ме нападнете.
\par 13 А те приказваха с него и рекоха: Не, само ще те вържем и ще те предадем в ръката им; но да те убием, това никак няма да направим. И тъй вързаха го с две нови въжета и го изведоха от скалата.
\par 14 Когато дойде в Лехий, филистимците го посрещнаха с възклицание. А Господният Дух дойде със сила на него; и въжетата, които бяха на мишците му, станаха като лен прегорен с огън, и връзките му като че ли се разтопиха от ръцете му.
\par 15 И като намери оселова челюст още прясна, простря ръката си та я взе и уби с нея хиляда мъже.
\par 16 Тогава рече Самсон: - С оселова челюст купове, купове, С оселова челюст убих хиляда мъже.
\par 17 И като престана да говори, хвърли челюстта от ръката си, от което се нарече онова място Рамат-лехий.
\par 18 А подире той ожадня премного; и извика към Господа, казвайки: Ти даде чрез ръката на слугата Си това голямо избавление; а сега да умра ли от жажда и да падна в ръката на наобрязаните?
\par 19 Но Бог разцепи трапа, който е в Лехий, и вода излезе из него; и като пи, духът му се съвзе, и той се съживи; затова, нарече мястото , което е в Лехий, Енакоре, както се нарича и до днес.
\par 20 И той съди Израиля във времето на филистимското господаруване двадесет години.

\chapter{16}

\par 1 След това, Самсон отиде в Газа, гдето видя една блудница, и влезе при нея.
\par 2 И каза се на газяните: Самсон доде тук. А те го обиколиха и цяла нощ седяха в засада за него при градската порта; и таеха се цяла нощ, като казаха: Преди да съмне утре ще то убием.
\par 3 А Самсон спа до среднощ; и, като стана в полунощ, хвана вратите на градската порта и двата стълба, та ги изкърти заедно с лоста, тури ги на рамената си си, и изнесе ги на върха на хълма, който е срещу Хеврон.
\par 4 След това, той залюби една жена в долината Сорик на име Далила.
\par 5 И филистимските началници, като дойдоха при нея, рекоха й: Предумай го и виж в що се състои голямата му сила, и как можем му надви та да го вържем, за да го оскърбим; а ние ще ти дадем всеки по хиляда и сто сребърника.
\par 6 Далила, прочее, каза на Самсона: Яви ми, моля, в що се състои голямата ти сила, и с какво трябва да те вържат, за да те оскърбят.
\par 7 И рече й Самсон: Ако ме вържат със седем пресни тетиви, още не изсъхнали, тогава ще стана безсилен, и ще бъда като всеки друг човек.
\par 8 Тогава филистимските началници й донесоха седем пресни тетиви, още не изсъхнали; и тя го върза с тях.
\par 9 (А в спалнята й имаше засада). И тя му рече: Филистимците връх тебе, Самсоне! А той скъса тетивите, както би се скъсала връв от кълчища, когато се допре до огъня. И силата му не се узна.
\par 10 После Далила каза на Самсона: Ето, ти си ме подиграл и си ме излъгал; кажи ми сега, моля, с какво трябва да те вържат.
\par 11 А той й рече: Ако ме вържат яко с нови въжета, още не употребявани, тогава ще стана безсилен, и ще бъда като всеки друг човек.
\par 12 Далила, прочее, взе нови въжета та го върза с тях, и рече му: Филистимците връх тебе, Самсоне! (А засадата седеше в спалнята). А той ги скъса от мишците си като нишка.
\par 13 Тогава Далила рече на Самсона: До сега си ме подигравал и си ме лъгал; кажи ми с какво трябва да те вържат. А той й рече: Ако втъчеш седемте плитки на главата ми в тъкането.
\par 14 И тя ги втъка и завря колчето; тогава ме рече: Филистимците връх тебе, Самсоне! А той се събуди от съня си, и изтръгна колчето на стана с тъкането.
\par 15 Тогава тя му рече: Как можеш да казваш: Обичам те, като сърцето ти не е с мене? Ето, тия три пъти ти ме излъга и не ми яви в що се състои голямата ти сила.
\par 16 И понеже му досаждаше всеки ден с думите си, и толкоз настояваше пред него, щото душата му се притесни до смърт,
\par 17 той й откри всичкото си сърце като й рече: Бръснач не е минавал през главата ми, защото аз съм Назирей Богу още от утробата на майка си; ако се обръсна, тогава силата ми ще се оттегли от мене, та ще стана безсилен, и ще бъда като всеки друг човек.
\par 18 А като видя Далила че й откри цялото си сърце, прати да повикат филистимските началници, като каза: Дойдете и тоя път, защото той ми откри цялото си сърце. Тогава филистимските началници дойдоха при нея, та донесоха и парите в ръцете си.
\par 19 И тя го приспа на коленете си, па повика човек та обръсна седемте плитки на главата му; и тя започна да го оскърбява. И силата му се оттегли от него.
\par 20 Тогава тя рече: Филистимците връх тебе, Самсоне! И той се събуди от съня си и си каза: Ще изляза както друг път и ще се отърся. Но той не знаеше, че Господ беше се оттеглил от него.
\par 21 И тъй, филистимците го хванаха и избодоха очите му, и като го отведоха в Газа, вързаха го с медни окови; и той мелеше в тъмницата.
\par 22 Но космите на главата му почнаха пак да растат след като бе обръснат.
\par 23 И филистимските началници се събраха, за да принесат голяма жертва на бога си Дагона и да се развеселят, защото си рекоха: Нашият бог предаде в ръката ни неприятеля ни Самсона.
\par 24 И когато го видяха людете, хвалеха бога си, казвайки: Нашият бог предаде в ръката ни неприятеля ни, разорителя на земята ни, който е убил множество от нас.
\par 25 И когато се развеселиха сърцата им, рекоха: Повикайте Самсона, за да му се подиграваме. И тъй, повикаха Самсона из тъмницата, та стана за подигравка пред тях; след което го поставиха между стълбовете на къщата .
\par 26 Тогава Самсон каза на момченцето, което го държеше за ръката: Остави ме да напипам стълбовете, на които се крепи къщата, за да се подпра на тях.
\par 27 А къщата бе пълна с мъже и жени; там бяха всичките филистимски началници, и на покрива около три хиляди мъже и жени, които гледаха Самсона, като беше станал за подигравка.
\par 28 Тогава Самсон извика към Господа, казвайки: Господи, Иеова, помни ме, моля; и подкрепи ме, моля, само тоя път, Боже, за да отмъстя поне еднъж на филистимците за двете си очи.
\par 29 И Самсон прегърна двата средни стълба, на които се крепеше къщата, и опря се на тях, на единия с дясната си ръка, и на другия с лявата.
\par 30 И рече Самсон: Нека умра с филистимците. И наведе се с всичката си сила; и къщата падна върху началниците и върху всичките люде, които бяха в нея. Така, умрелите, които той уби при смъртта си, бяха повече от ония, които бе убил през живота си.
\par 31 Тогава братята му и целият му бащин дом слязоха и като го взеха, занесоха го та го погребаха в гроба на баща му Маноя, между Сарая и Естаол. И той съди Израиля двадесет години.

\chapter{17}

\par 1 Имаше един човек от Ефремовата хълмиста земя на име Михей.
\par 2 Той каза на майка си: Хилядата и сто сребърника, които ти бяха отнети, за които ти и прокълна, още изговори клетвата като слушах аз - ето, среброто е у мене; аз го взех. А майка му рече: Благословен да е мият син от Господа.
\par 3 И като върна хилядата и сто сребърника на майка си, майка му каза: Действително бях посветила от ръката си среброто Господу за сина ми, за да направи изваян идол и леян кумир; и той, сега ще го върне на тебе.
\par 4 Но той върна среброто на майка си; затова майка му взе двеста сребърника и даде ги на златаря, който направи от тях изваян идол и леян кумир; и те бяха поставени в дома на Михея.
\par 5 И тоя Михей, като имаше капище за богове, направи ефод и домашни идоли, и посвети един от синовете си, който му стана свещеник.
\par 6 В ония дни нямаше цар в Израиля; всеки правеше каквото му се виждаше угодно.
\par 7 И имаше един момък от Витлеем Юдов, град на Юдовите семейства, който беше левитин, и е бил там прищелец.
\par 8 Тоя човек замина от града, от Витлеем Юдов, за да пришелствува, гдето намери място , и като пътуваше, дойде до Михеевата къща в Ефремовата хълмиста земя.
\par 9 И Михей му каза: От где идеш? А той му рече: Аз съм левитин, от Витлеем Юдов, и отивам да пришелствувам, гдето намеря място .
\par 10 И Михей му каза: Седи у мене и стани ми отец и свещеник; и аз ще ти давам по десет сребърника на годината, една премяна дрехи и храната ти. И тъй, левитинът влезе у него.
\par 11 И левитинът беше благодарен да седи у човека, и тоя момък му стана като един от синовете му.
\par 12 И Михей посвети левитина и момъкът му стана свещеник; и остана в Михеевата къща.
\par 13 Тогава каза на Михей: сега зная, че Господ ще ми стори добро, защото имам левитин за свещеник.

\chapter{18}

\par 1 В онова време нямаше цар в Израиля; и в ония дни Дановото племе си търсеше притежание, гдето да се зесели, защото до оня ден не беше им се паднало наследство между Израилевите племена.
\par 2 И Данците изпратиха от рода си петима мъже от цялото си число, храбри мъже, от Сарая и от Естаол, за да съгледат земята и да я изследват, като им казаха: Идете, изследвайте земята. И те дойдоха до Михеевата къща, в Ефремовата хълмиста земя и там пренощуваха,
\par 3 защото , като се приближиха до Михеевата къща, познаха гласа на младия левитин и свърнаха там та му казаха: Кой те доведе тук? и що правиш на това място? и какво ти се пада тук?
\par 4 А той им рече: Така и така ми направи Михей, и пазари ме, та му станах свещеник.
\par 5 И те му рекоха: Молим, допитай се до Бога, за да узнаем дали ще бъде благополучно пътешествието, по което отиваме.
\par 6 А свещеникът им каза: Идете с мир; пътешествието, по което отивате, е пред Господа.
\par 7 Тогава петимата мъже тръгнаха: и като дойдоха в Лаис, видяха, че людете в него живееха безгрижно, както сидонците, спокойно и без страх, защото нямаше в земята властелин, който да ги притеснява в нищо; и те бяха далеч от сидонците, и нямаха сношение с никого.
\par 8 Те, прочее, се върнаха при братята си в Сарая и Естаол; и братята им рекоха: Какво ще кажете вие?
\par 9 А те рекоха: Станете, нека отидем против тях; защото видяхме земята, и, ето, много е добра. Вие още седите ли? Не се обленявайте да идете и влезете, за да завладеете земята.
\par 10 Като от дете ще намерите люде, които живеят безгрижно и на обширна земя, (защото Бог я даде в ръката ни), земя, в която няма оскъдност от нищо, каквото има на света.
\par 11 И тъй, потеглиха от там, от Дановия род, от Сарая и от Естаол, шестотин мъже препасани с войнишки оръжия.
\par 12 И отивайки, те разположиха стан у Кириатиарим в Юда; за това нарекоха онова място Маханедан, както се казва и до днес; ето то се намира зад Кириатиарим.
\par 13 И от там преминаха в Ефремовата хълмиста земя та дойдоха до Михеевата къща.
\par 14 Тогава петимата мъже, които бяха ходили да съгледат местността Лаис, проговориха, казвайки на братята си: Знаете ли, че в тия къщи има един и домашни идоли, изваян идол и леян кумир? Сега, прочее, размислете, какво трябва да направите.
\par 15 И така, те се отбиха там, та отидоха в къщата на младия левитин (ще каже , в Михеевата къща) и ги поздравиха.
\par 16 И шестте стотин мъже, които бяха от данците, застанаха, препасани с войнишките си оръжия, във входа на вратата.
\par 17 Тогава петимата мъже, които бяха отишли да съгледат земята, отидоха, влязоха там, та взеха изваяния идол, ефода, домашните идоли, и леяния кумир; а свещеникът стоеше във входа на вратата с шестте стотин мъже, които бяха препасани с войнишки оръжия.
\par 18 И като влязоха те в Михеевата къща и изнесоха изваяния идол, ефода, домашните идоли и леяния кумир, свещеникът им рече: Що правите вие?
\par 19 А те му рекоха: Мълчи! тури ръката си на устата си, та дойди с нас и бъди ни отец и свещеник. По-добре ли ти е да бъдеш свещеник на дома на един човек, или да бъдеш свещеник на едно племе и на един род в Израиля?
\par 20 На това свещеникът сърдечно се зарадва, и, като взе ефода, домашните идоли, и изваяния идол, отиваше си всред людете.
\par 21 И те се обърнаха та потеглиха, като туриха пред себе си децата и добитъка и по-скъпите си вещи.
\par 22 Когато се бяха отдалечили Михеевата къща, човеците от къщата съседни с Михеевата къща, се събраха и застигнаха данците.
\par 23 И като извикаха на данците, тия обърнаха лицата си та рекоха на Михея: Що ти е та си събрал при себе си такова множество?
\par 24 А той каза: Взехте ми боговете, които си направих, и свещеника, и тръгнахте; и що повече ми става? Как, прочее, ми казвате: Що ти е?
\par 25 А данците му рекоха: Да се не чуе гласът ти между нас, да не би да ни нападнат разгневени мъже, та изгубиш живота си и живота на домашните си.
\par 26 И данците вървяха по пътя си; а Михей, като видя, че те са по-силни от него, върна се та дойде у дома си.
\par 27 Те, прочее, взеха това, що бе направил Михей, и свещеника когото имаше, та дойдоха в Лаис, при людете спокойни и живеещи без страх; и поразиха ги с острото на ножа и изгориха града с огън.
\par 28 И нямаше кой да го избави, защото беше далеч от Сидон, и те нямаха сношения с никого. Градът бе в долината до Вет-реов; и те го съградиха изново и се зеселиха в него.
\par 29 И нарекоха града Дан, по името на баща си Дан, който се е родил на Израиля; а по-напред името на града беше Лаис.
\par 30 Тогава данците си поставиха изваяния идол; а Иоанатан, син на Гирсона, син на Моисея, - той и потомците му бяха свещеници на Дановото племе до времето, когато земята се плени.
\par 31 Така те си поставиха изваяния идол, който Михей бе направил, и който оставаше там през цялото време, когато Божият дом беше в Сило.

\chapter{19}

\par 1 В ония дни, когато нямаше цар в Израиля, имаше един левитин, който живееше на отвъдната страна на Ефремовата хълмиста земя, и който се беше взел наложница от Витлеем Юдов.
\par 2 Но наложницата му блудствува против него, и отиде си от него в бащината си къща във Витлеем Юдов, гдето остана около четири месеца.
\par 3 И мъжът й стана та отиде подир нея, за да й говори любезно и да я върне, като водеше със себе си слугата си и два осела. И тя го въведе в бащината си къща; и когато го видя бащата на младата, посрещна го с радост.
\par 4 И тъстът му, бащата на младата го задържа, та преседя с него три дена; и ядоха и пиха и пренощуваха там.
\par 5 На четвъртия ден, като станаха рано, той се дигна да си иде; но бащата на младата рече на зетя си: Подкрепи сърцето си с малко хляб и после ще си отидеш.
\par 6 И така, седнаха та ядоха и пиха двамата заедно; после бащата на младата рече на мъжа: Склони, моля, да пренощуваш, и нека се развесели сърцето ти.
\par 7 Обаче, човекът се дигна да си иде; но понеже тъстът му настояваше пред него, той пак пренощува там.
\par 8 А на петия ден стана рано да си иде; но бащата на младата рече: Подкрепи, моля, сърцето си. И остана догдето превали денят, като ядоха двамата.
\par 9 Сетне, когато човекът стана да си отиде - той и наложницата му и слугата му, рече тъстът му, бащата на младата: Ето, сега денят преваля към вечер; пренощувайте, моля. Ето, денят е на свършване; пренощувайте тук и нека се развесели сърцето ти; а утре тръгнете рано на път, за да отидеш у дома си.
\par 10 Но човекът не склони да пренощува, а като стана, тръгна и дойде срещу Евус (който е Ерусалим), като водеше със себе си два оседлани осела; и наложницата му беше с него.
\par 11 Когато се приближиха до Евус, денят беше много преминал; и слугата рече на господаря си: Дойди, моля, нека се отбием в тоя град на евусите, за да пренощуваме в него.
\par 12 Но господарят му рече: Няма да се отбием в град на чужденци, гдето не се намират от израилтяните, но ще заминем за Гавая.
\par 13 Рече още на слугата си: Дойди, нека се приближим до едно от тия места, и ще пренощуваме в Гавая или в Рама.
\par 14 И тъй, те заминаха та вървяха; а зайде им слънцето близо при Гавая, която принадлежи на Вениамина.
\par 15 И там се отбиха, за да влязат да пренощуват в Гавая; и когато влезе, седна край градската улица, защото никой не ги прибираше в къщата си, за да пренощуват.
\par 16 И, ето, един старец идеше вечерта от работата си на полето, и тоя човек беше от хълмистата земя на Ефрема и пришелствуваше в Гавая, а местните човеци бяха вениаминци.
\par 17 И като подигна очи и видя пътника на градската улица, старецът му каза: Къде отиваш? и от где идеш?
\par 18 А той му рече: Ние заминаваме от Витлеем Юдов към отвъдната страна на хълмистата земя на Ефрема, отдето съм аз. Ходих до Витлеем Юдов, и сега отивам за Господния дом; а никой не ме прибра в къщата си.
\par 19 А пък ние си имаме плява и храна за ослите си, също и хляб и вино за мене и за слугинята ти и за момчето, което е със слугите ти; нямаме нужда от нищо.
\par 20 И старецът рече: Бъди спокоен; обаче, всичките ти нужди нека понеса аз; само да не пренощуваш на улицата.
\par 21 И така, той го въведе в къщата си, и даде зоб на ослите; а те си умиха нозете и ядоха и пиха.
\par 22 Като веселяха сърцата си, ето, едни градски мъже, развратници, обиколиха къщата, блъскаха на вратата, и говориха на стареца домакин, казвайки: Изведи човека, който влезе в къщата ти, за да го познаем.
\par 23 А човекът, то ест , домакинът, излезе при тях та им каза: Не, братя мои! моля, недейте прави това зло; тъй като тоя човек е мой гост, недейте струва това безумие.
\par 24 Ето дъщеря ми, девица, и неговата наложница; тях ще изведа вън сега; опозорете ги, и сторете им каквото ви е угодно; но на тоя човек да не сторите едно такова безумно дело.
\par 25 Но мъжете не искаха да го послушат; за това, човекът взе наложницата му та им я изведе вън; и те я познаха и обезчестяваха я цялата нощ дори до утринта, а като се зазори пуснаха я.
\par 26 И тъй призори жената дойде и падна при вратата на къщата на човека, гдето беше господарят й, и там лежа до съмване.
\par 27 И на утринта господарят й стана та отвори вратата на къщата и излезе, за да си отиде по пътя, и ето наложницата му паднала при врата на къщата, и ръцете й на прага.
\par 28 И рече й: Стани да си отидем. Но нямаше отговор. Тогава човекът я дигна на осела, и стана та отиде на мястото си.
\par 29 И като дойде в дома си, взе нож, хвана наложницата си, и я разсече член по член на дванадесет части, и прати ги във всичките предели на Израиля.
\par 30 И всички, които видяха това, думаха: Не е ставало, нито се е виждало такова нещо от деня, когато израилтяните излязоха из Египетската земя, до днес; размислете, прочее, затова, съветвайте се, и изказвайте се.

\chapter{20}

\par 1 Тогава всичките израилтяни излязоха, и цялото общество, от Дан до Вирсавее, заедно с Галаадската земя, се събра като един човек пред Господа в Масфа.
\par 2 И людете от всичките краища, всички Израилеви племена, четиристотин хиляди мъже пешаци, които теглеха меч, се представиха в събранието на Божиите люде.
\par 3 И вениаминците чуха, че израилтяните отишли в Масфа. И израилтяните казаха: Разправете ни как стана това зло.
\par 4 И левитинът, мъжът на убитата жена, в отговор рече: Дойдох в Гавая Вениаминова, аз и наложницата ми, за да пренощуваме.
\par 5 А гавайските мъже се подигнаха против мене, и през нощта обиколиха къщата в който бях; мене искаха да убият, а наложницата ми изнасилваха, та умря.
\par 6 Затова взех наложницата си та я разсякох и я изпратих по всичките предели на Израилевото наследство; защото те извършиха разврат и безумие в Израиля.
\par 7 Гледайте, вие израилтяните, всинца вие, съветвайте се тук помежду си, и дайте мнението си.
\par 8 Тогава всичките люде станаха като един човек и казаха: Никоя от нас няма да отиде в шатъра си, нито ще се върне някой от нас в къщата си,
\par 9 но ето сега какво ще направим на Гавая, ще отидем против нея по жребие;
\par 10 и ще вземем по десет мъже на сто от всичките Израилеви племена, и по сто на хиляда, които да донесат храна на людете, та, като стигнат в Гавая Вениаминова, да им сторят според всичкото безумие, което те извършиха в Израиля.
\par 11 И тъй, събраха се против града всичките Израилеви мъже, обединени като един човек.
\par 12 Тогава Израилевите племена пратиха мъже по цялото Вениаминово племе да казват: Какво е това нечестие, което се е извършило помежду ви?
\par 13 Сега, прочее, предайте човеците, ония развратници, които са в Гавая, за да ги избием, и да отмахнем това зло от Израиля. Но Вениамин отказа да послуша гласа на братята си израилтяните;
\par 14 и вениаминците се събраха от градовете си на Гавая, за да излязат на бой против израилтяните.
\par 15 И в оня ден вениаминците, излезли от градовете, като се преброиха, бяха двадесет и шест хиляди мъже, които теглеха меч, освен жителите на Гавая, които се преброиха седемстотин отборни мъже.
\par 16 Между всички тия люде имаше седемстотин отборни мъже леваци, които всички можеха с прашка да хвърлят камъни на косъм и всеки път да улучат.
\par 17 А без Вениамина, Израилевите мъже, които се преброиха, бяха четиристотин хиляди мъже, които теглеха меч; всички тия бяха военни мъже.
\par 18 Тогава израилтяните станаха, възлязоха във Ветил, та се допитаха до Бог, казвайки: Кой от вас да възлезе пръв против вениаминците? А Господ каза: Юда да излезе пръв.
\par 19 И тъй, на утринта израилтяните станаха та разположиха стан против Гавая.
\par 20 И Израилевите мъже излязоха на бой против Вениамина; и опълчиха се Израилевите мъже на бой против тях в Гавая.
\par 21 А вениаминците излязоха из Гавая, та в оня ден повалиха на земята двадесет и две хиляди мъже от Израиля.
\par 22 Но людете, Израилевите мъже, се ободриха и опълчиха се пак на бой, на мястото, гдето бяха се опълчили първия ден.
\par 23 Защото израилтяните бяха възлезли и плакали пред Господа до вечерта, и бяха се допитали до Господа, казвайки: Да възлезем ли пак на бой против потомците на брата ни Вениамина? И Господ беше казал: Възлезте против него.
\par 24 И така, на втория ден израилтяните се приближиха при вениаминците.
\par 25 А на втория ден Вениамин излезе из Гавая против тях та повали на земята още осемнадесет хиляди мъже от израилтяните; всички тия теглеха меч.
\par 26 Тогава всичките израилтяни и всичките люде влязоха та дойдоха във Ветил и плакаха, и седнаха там пред Господа, и постиха в оня ден до вечерта; и принесоха всеизгаряния и примирителни жертви пред Господа.
\par 27 После израилтяните се допитаха до Господа, (защото през ония дни ковчегът на Божия завет беше там,
\par 28 и Финеес, син на Елеазара, Аароновият син, служеше пред него през ония дни), и запитаха: Да възлезем ли пак на бой против потомците на брата ни Вениамина? или да престанем? И Господ каза: Възлезте, защото утре ще ги предам в ръката ви.
\par 29 Тогава Израил постави засада около Гавая.
\par 30 И на третия ден израилтяните излязоха против вениаминците, та се опълчиха против Гавая както в предишните дни.
\par 31 А вениаминците излязоха против людете, отвлякоха се от града, и почнаха както в предишните дни да поразяват людете по пътищата, (от които единият отива към Ветил, а другият към Гавая), и убиха в полето около тридесет мъже от Израиля.
\par 32 Затова вениаминците си казаха: Те падат пред нас както по-напред. А израилтяните рекоха: Да побегнем и да ги отвлечем от града към пътищата.
\par 33 Тогава всичките Израилеви мъже станаха от мястото си, та се опълчиха във Ваал-тамар, и засадата на Израиля изкочи от мястото си, от Гавайската ливада.
\par 34 И дойдоха против Гавая десет хиляди отборни мъже от целия Израил, и битката ставаше ожесточена; но вениаминците не знаеха, че бедата ги застигаше.
\par 35 Защото Господ порази Вениамина пред Израиля; и в оня ден израилтяните погубиха от вениаминците двадесет и пет хиляди и сто мъже; те всички теглеха меч.
\par 36 И вениаминците видяха, че бяха поразени. Защото израилевите мъже отстъпиха пред вениаминците, като разчитаха на засадата, която бяха поставили против Гавая.
\par 37 Тогава засадата побърза та се спусна върху Гавая; и засадата дебнеше напред, и поразиха целия град с острото на ножа.
\par 38 А Израилевите мъже бяха определили знак с ония, които бяха в засадата, да направят да се издигне от града голям стълб дим.
\par 39 И когато израилтяните отстъпиха в битката, Вениамин почна да поразява, и уби от израилтяните около тридесет мъже, защото си рекоха: Наистина те падат пред нас както при първата битка.
\par 40 Но когато облакът почна да се издига от града в димен стълб, вениаминците погледнаха на назад, и, ето, целият град се издигаше в дим към небето.
\par 41 Тогава се повърнаха Израилевите мъже; и Вениаминовите мъже се смутиха, защото видяха, че бедата ги постигна.
\par 42 Затова обърнаха се пред Израилевите мъже към пътя за пустинята; но битката ги притисна; и те погубваха всред градовете ония които излизаха из тях.
\par 43 Заобиколиха вениаминците, гониха ги, и тъпкаха ги лесно до срещу Гавая към изгрева на слънцето.
\par 44 И паднаха от Вениамина осемнадесет хиляди мъже, - всички храбри мъже.
\par 45 Прочее, те се обърнаха та побягнаха към пустинята в канарата Римон; а израилтяните набраха от тях пабирък по пътищата, пет хиляди мъже, после ги гониха до Гидом и там убиха от тях две хиляди мъже.
\par 46 Така всичките, които паднаха в оня ден от Вениамина, бяха двадесет и пет хиляди мъже, които теглят меч, - всички храбри мъже.
\par 47 А шестстотин мъже се обърнаха та побягнаха към пустинята в канарата Римон; и седяха в канарата Римон четири месеца.
\par 48 А Израилевите мъже се обърнаха върху вениаминците, та го поразиха с острото на ножа, както градски човек, така и добитък, и всичко що се намираше; и предадоха на огън всичките градове, които намираха.

\chapter{21}

\par 1 А Израилевите мъже бяха се заклели в Месфа, казвайки: Ни един от нас да не даде дъщеря си на вениаминец за жена.
\par 2 И людете дойдоха във Ветил та седяха там до вечерта пред Бога, и те с висок глас плакаха горко.
\par 3 И рекоха: Защо, Господи Израилеви Бог, стана това в Израиля, та липса днес едно племе от Израиля?
\par 4 И на утрешния ден, людете станаха рано та издигнаха олтар, и принесоха всеизгаряния и примирителни жертви.
\par 5 Тогава израилтяните казаха: Кой измежду всичките Израилеви племена не възлезе на събранието при Господа? Защо бяха направили голяма клетва относно онзи, който не би дошъл при Господа в Масфа, като бяха рекли: Непременно да се умъртви.
\par 6 И израилтяните се разкаяха за брата си Вениамина, казвайки: Днес се отне едно племе от Израиля.
\par 7 Какво да сторим за оцелелите от тях, за да имат жени, тъй като се заклехме в Господа да им не дадем жени от дъщерите си?
\par 8 Затова казаха: Кой измежду Израилевите племена не възлезе в Масфа при Господ? И, ето, от Явис галаадски не беше дошъл никой на събранието в стана.
\par 9 Защото, като се преброиха людете, ето, нямаше там ни един от жителите на Явис галаадски.
\par 10 За това, обществото прати там дванадесет хиляди от най-храбрите мъже и казаха им със заповед: Идете, поразете жителите на Явис галаадски с острото на ножа, с жените и децата.
\par 11 И ето какво да направите: погубете съвсем всеки от мъжки пол, и всяка жена, която е лежала с мъж.
\par 12 А между жителите на Явис галаадски намериха четиристотин млади девици, които не бяха познали мъж, като не бяха лежали с мъж; и доведоха ги в стана у Сило, което е в Ханаанската земя.
\par 13 Тогава цялото общество прати да говорят на вениаминците, които бяха в канарата Римон, и да им прогласят мир.
\par 14 И тъй, вениаминците незабавно се върнаха; и те им дадоха за жени ония, които бяха останали живи от Явис галаадските жени; но пак не из стигнаха.
\par 15 И людете се смилиха за Вениамина, защото Господ беше направил пролом между Израилевите племена.
\par 16 Тогава старейшините на обществото казаха: Що да сторим за оцелелите, за да имат жени, тъй като жените са изтребени от Вениамина?
\par 17 И рекоха: Наследство е потребно за оцелелите от Вениамина, за да не изчезне едно племе от Израиля;
\par 18 а пак ние не можем да им дадем жени от дъщерите си, защото израилтяните се заклеха, казвайки: Проклет, който даде жена на Вениамина.
\par 19 Тогава рекоха: Ето, всяка година става празник Господу в Сило, което е на север от Ветил на изток от пътя, който отива от Ветил в Сихем, и на юг от Левона.
\par 20 И така, заповядаха на вениаминците, казвайки: Идете, крийте се в лозята;
\par 21 и гледайте, и, ето, ако силоенските дъщери излязат да играят хоро, тогава излезте из лозята та си грабнете всеки за сабе си жена от силоенските дъщери, па си идете във Вениаминовата земя.
\par 22 И когато бащите им или братята им, дойдат при вас, за да се оплачат, ние ще им речем; Бъдете благосклонни към тях заради нас, понеже в битката ние не задържахме жена за всекиго; а сега вие не сте им ги дали, та да сте виновни.
\par 23 И вениаминците сториха така, и според числото си взеха жени от играещите хоро; после си тръгнаха и се върнаха в наследството си, и съградиха изново градовете и живееха в тях.
\par 24 Тогава израилтяните тръгнаха от там, всеки за в племето си и в рода си, и излязоха от там всеки за в наследството си.
\par 25 В ония дни нямаше цар в Израиля; всеки правеше каквото му се виждаше угодно.

\end{document}