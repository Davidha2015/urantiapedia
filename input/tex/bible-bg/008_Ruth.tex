\begin{document}

\title{Ruth}


\chapter{1}

\par 1 В Дните когато съдиите съдеха, настана глад на земята. И един човек от Витлеем Юдов отиде да престои в Моавската земя, той, и жена му, и двамата му сина.
\par 2 Името на човека беше Елимелех, а името на жена му Ноемин, а имената на двамата му сина Маалон и Хелеон; те бяха ефратци от Витлеем Юдов. И дойдоха в Моавската земя и там останаха.
\par 3 И Елимелех, мъжът на Ноемин, умря; и тя остана с двамата си сина.
\par 4 И те си взеха жени моавки, на които името на едната беше Орфа, а името на другата Рут; и живяха там около десет години.
\par 5 Тогава умряха Маалон и Хелеон, и двамата, тъй че жената се лиши от двамата си сина и от мъжа си.
\par 6 След това тя стана със снахите си да се върне от Моавската земя, защото беше чула в Моавската земя, че Господ посетил людете Си и им дал хляб.
\par 7 И тъй, тя излезе от мястото, гдето беше, и двете и снахи с нея, и вървяха по пътя да се върнат в Юдовата земя.
\par 8 Сетне Ноемин каза на двете си снахи: Идете, върнете се всяка в дома на майка си. Господ да постъпва с благост към вас, както вие постъпихте към умрелите и към мене.
\par 9 Господ да ви даде да намерите спокойствие, всяка в дома на мъжа си. Тогава ги целуна; а те плакаха с висок глас.
\par 10 И рекоха й: Не, но с тебе ще се върнем при твоите люде.
\par 11 Но Ноемин каза: Върнете се, дъщери мои, защо да дойдете с мене? Имам ли още синове в утробата си, за да ви станат мъже?
\par 12 Върнете се, дъщери мои, идете защото остарях и не съм вече за мъж. Ако бих рекла: Имам надежда; даже ако се омъжех тая нощ, па и родих синове,
\par 13 вие бихте ли ги чакали догде пораснат? Бихте ли се въздържали заради тях да се не омъжите? Не, дъщери мои; върнете се, понеже съм много огорчена заради вас, гдето Господната ръка се е простирала против мене.
\par 14 А като плакаха пак с висок глас, Орфа целуна свекърва си, а Рут се привърза при нея.
\par 15 Тогава рече Ноемин: Ето, етърва ти се върна при людете си и боговете си; върни се и ти подир етърва си.
\par 16 А Рут каза: Не ме умолявай да те оставя и да не дойда подире ти; защото, гдето идеш ти, и аз ще ида, и гдето останеш и аз ще остана; твоите люде ще бъдат мои люде, и твоят Бог мой Бог;
\par 17 гдето умреш ти, и аз ще умра, и там ще се погреба; така да ми направи Господ, да! и повече да притури, ако друго, освен смъртта, ме разлъчи от тебе.
\par 18 И Ноемин, като видя, че тя настояваше да иде с нея, престана да й говори.
\par 19 И тъй, двете вървяха докато дойдоха във Витлеем. И когато стигнаха във Витлеем, целият град се раздвижи поради тях; и жените думаха: Това ли е Ноемин?
\par 20 А тя им каза: Не ме наричайте Ноемин наричайте ме Мара защото Всесилният ме твърде огорчи.
\par 21 Пълна излязох; а Господ ме доведе празна. Защо ме наричате Ноемин, тъй като Господ е заявил против мене, и Всесилният ме е оскърбил?
\par 22 Така се върна Ноемин, и със снаха й Рут, моавката, която дойде от Моавската земя. Те стигнаха във Витлеем в началото на ечемичната жътва.

\chapter{2}

\par 1 А Ноемин имаше един сродник по мъжа си, много имотен човек, от рода на Елимелеха, на име Вооз.
\par 2 И моавката, Рут, каза на Ноемин: Да отида на нивата да събирам класове подир онзи, чието благоволение придобия. И тя й рече: Иди, дъщерьо моя.
\par 3 И тя отиде, и като стигна, събираше класове в нивата подир жътварите; и случи се да попадне на нивата, която беше дал на Вооза, човекът от рода на Елимелеха.
\par 4 И, Ето, Вооз дойде от Витлеем та рече на жътварите: Господ с вас! И те му отговориха: Господ да те благослови!
\par 5 Тогава Вооз рече на слугата си, настойника на жътварите: Чия е тая млада жена?
\par 6 И слугата, настойникът на жътварите, рече в отговор: Тая млада жена е моавката, която се върна с Ноемин от Моавската земя;
\par 7 и тя рече: Да бера, моля, класове, и да събера нещо между снопите подир жътварите. И тъй, тя дойде та стоя от сутринта до сега, само че си почина малко в къщи.
\par 8 Тогава Вооз рече на Рут: Чуеш ли ме, дъщерьо моя? не ходи да събираш класове на друга нива, и да си не отидеш от тук, но стой тук при момичетата ми.
\par 9 Нека очите ти бъдат в нивата, гдето ще жънат, и ходи подир тях; ето, аз заръчах на момчетата да се не досягат до тебе; и когато си жадна, иди при съдовете та пий от онова, което момчетата са налели.
\par 10 И тя падна на лице та се поклони до земята, и рече му: Как придобих аз твоето благоволение, та да ме пригледаш, като съм чужденка?
\par 11 А Вооз в отговор и рече: Каза ми се напълно всичко, що се сторила на свекърва си подир смъртта на мъжа си, и как си оставила баща си и майка си и родината си, та си дошла между люде, които по-преди не си познавала.
\par 12 Господ да ти отплати за делото ти, и пълна награда да ти се даде от Господа Израилевия Бог, под чиито крила си дошла да се подслониш.
\par 13 И тя рече: Да придобия твоето благоволение, господарю мой; понеже ти ме утеши и понеже говори благосклонно на слугинята си, ако и да не съм като някоя от твоите слугини.
\par 14 И когато настана времето за ядене, Вооз й рече: Дойди тука, та яж от хляба, и натопи залъка си в оцета. Тя, прочее, седна до жътварите; а той й подаде пържена пшеница, та яде и се насити и остави нещо.
\par 15 А като стана да бере класове, Вооз заповяда на момчетата си, като каза: И между снопите нека събира класове; не я мъмрете;
\par 16 и даже изваждайте нещо за нея от ръкойките и оставяйте го, и нека го събира без да й забранявате.
\par 17 Така тя береше класове в нивата до вечерта; па очука събраното, и то беше около една ефа ечемик.
\par 18 И взе това та влезе в града, и свекърва й видя колко класове бе събрала; и Рут извади та й даде онова, което беше оставила след като се бе наситила.
\par 19 И свекърва й рече: Где събира днес класове? и где работи? Благословен да бъде оня, който те пригледа. И тя яви на свекърва си в чия нива бе работила, като каза: Името на човека, у когото работих днес, е Вооз.
\par 20 И Ноемин рече на снаха си: Благословен от Господа оня, който не лиши от милостта си ни живите, ни умрелите. Рече и още Ноемин: Тоя човек е от нашия род, близък нам сродник.
\par 21 И Рут моавката каза: При това, той ми рече: Да не се делиш от момчетата ми докато не свършат цялата ми жътва.
\par 22 И Ноемин рече на снаха си Рут: Добре е, дъщерьо моя, да излизаш с неговите момчета, и да те не срещат в друга нива.
\par 23 И така, тя се привърза при момчетата на Вооза, за да бере класове догде се свърши ечемичната жътва и пшеничната жътва; и седеше със свекърва си.

\chapter{3}

\par 1 След това, свекърва й Ноемин й рече: Дъщерьо моя, да не потърся ли спокойствие за тебе та да благоденствуваш?
\par 2 И сега, не е ли от нашия род Вооз, с чиито момчета беше ти? Ето, тая нощ той вее ечемика на гумното.
\par 3 Умий се, прочее, и помажи се, и облечи се с дрехите си и слез на гумното; но да се не покажеш на човека, догде не е свършил яденето и пиенето си.
\par 4 И като си ляга, забележи мястото гдето ляга, и иди та подигни покривката из към нозете му и легни; и той ще ти каже що трябва да направиш.
\par 5 А тя й рече: Всичко що ми каза, ще сторя.
\par 6 И тъй слезе на гумното, та стори всичко, що й заповяда свекърва й.
\par 7 А Вооз, като яде и пи и сърцето му се развесели, отиде да си легне край купа ечемик; а тя дойде тихо та подигна покривката откъм нозете му и легна.
\par 8 И около среднощ човекът се стресна и обърна; и ето жена лежеше при нозете му.
\par 9 И рече: Коя си ти? И тя отговори: Аз съм слугинята ти Рут; простри, прочее, полата си над слугинята си, защото си ми близък сродник.
\par 10 А той каза: Благословена да си от Господа, дъщерьо; тая последна доброта, която се показала, е по-голяма от предишната, гдето ти не отиде след млади, били те сиромаси или богати.
\par 11 И сега, дъщерьо, не се бой; аз ще сторя за тебе всичко що казваш; защото целият град на людете ми знае, че си добродетелна жена.
\par 12 И сега, вярно е, че аз съм близък сродник; има обаче друг сродник по-близък от мене.
\par 13 Остани тая нощ и утре, ако иска той да изпълни към тебе длъжността на сродник, добре, нека я изпълни; но, ако не иска да изпълни към тебе длъжността на сродник, тогава заклевам се в живота на Господа, аз ще изпълня тая длъжност към тебе. Пренощувай тук.
\par 14 И тя лежа при нозете му до сутринта, и стана преди да може човек да разпознае човека, защото той рече: Нека не се знае, че жена е дохождала на гумното.
\par 15 Рече още: Донеси покривалото, което е върху тебе, и дръж го. И като го държеше, той премери шест мери ечемик, та го натовари на нея; и тя си отиде в града.
\par 16 И като дойде при свекърва си, тя й каза: Що ти стана, дъщерьо моя. И тя й разправи всичко, що й стори човекът.
\par 17 Каза още: Даде ми тия шест мери ечемик, защото ми рече: Да не идеш празна при свекърва си.
\par 18 А тя рече: Седни, дъщерьо моя, догде видиш как ще се свърши работата; защото човекът няма да се спре, докато не свърши работата още днес.

\chapter{4}

\par 1 И Вооз възлезе на портата и седна там; и, ето, минаваше близкият сродник, за когото беше говорил Вооз. И рече: Господине, свърни, седни тук. И той свърна и седна.
\par 2 Тогава Вооз събра десет мъже от градските старейшини и рече: Седнете тук. И те седнаха.
\par 3 После рече на сродника: Ноемин, която се върна от Моавската земя, продава нивата, дяла, който принадлежеше на брата ни Елимелеха;
\par 4 а аз рекох да ти известя и да ти кажа: Купи го пред седящите тука, пред старейшините на людете ми. Ако щеш да го откупиш като сродник, откупи го; но, ако не щеш да го откупиш, кажи ми, за да зная; защото освен тебе, няма друг да го откупи, като сродник; и аз съм подир тебе. И той рече: Аз ще го откупя.
\par 5 И рече Вооз: В деня, когато купиш нивата от ръката на Ноемин, трябва да я купиш и от моавката Рут, жена на умрелия, за да възстановиш името на умрелия над наследството му.
\par 6 Тогава сродникът рече: Не мога да изпълня длъжността на сродник, да не би да напакостя на своето си наследство; ти приеми върху себе си моето право да откупя, защото не мога да откупя нивата.
\par 7 А в старо време, за да се утвърди всяко дело по откупване и размяна в Израиля, ето що беше обичаят: човекът изуваше обувката си, та я даваше на ближния си: и така се свидетелствуваше в Израиля.
\par 8 За това, сродникът като каза на Вооза: Купи го ти за себе си, изу си обувката.
\par 9 Тогава Вооз каза на старейшините и на всичките люде: Днес сте свидетели, че купувам от ръката на Ноемин всичко, що беше Елимелехово и всичко що беше Хелеоново и Маалоново.
\par 10 А още и моавката Рут Маалоновата жена, придобих за жена, за да възстановя името на умрелия над наследството му, за да се не изличи името на умрелия измежду братята му и из града на обитаването му; вие сте свидетели днес.
\par 11 И всичките люде, които бяха в портата, и старейшините рекоха: Свидетели сме. Господ да направи жената, която влиза в дома ти, като Рахил и като Лия, двете, които са съградили Израилевия дом; и да станеш силен в Ефрата, и да бъдеш знаменит във Витлеем;
\par 12 и от потомството, което Господ ще ти даде от тая млада жена, нека бъде домът ти като дома на Фареса, когото Тамар роди на Юда.
\par 13 И така, Вооз взе Рут, и тя му стана жена; и като влезе при нея, Господ й даде зачатие, и тя роди син.
\par 14 И жените казаха на Ноемин: Благословен Господ, Който днес не те остави без сродник; нека бъде прочуто Името Му в Израиля.
\par 15 Тоя син ще ти бъде обнова на живота и прехрана в старините ти; защото го роди снаха ти, която те обича, която е за тебе по-желателна от седем сина.
\par 16 И Ноемин взе детето и тури го в пазухата си, и стана му кърмилница.
\par 17 И съседките ме дадоха име казвайки: Син се роди на Ноемин. И рекоха го Овид; той е баща на Есея, Давидовия баща.
\par 18 И ето Фаресовото родословие: Фарес роди Есрона,
\par 19 Есрон роди Арама, Арам роди Аминадава,
\par 20 Аминадав роди Наасона, Наасон роди Салмона,
\par 21 Салмон роди Вооза, Вооз роди Овида,
\par 22 Овид роди Есея, а Есей роди Давида.


\end{document}