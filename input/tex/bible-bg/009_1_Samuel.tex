\begin{document}

\title{1 Samuel}


\chapter{1}

\par 1 Имаше един човек от Раматаим-софим, от Ефремовата хълмиста земя, на име Елкана, син Ероамов, син Елиуев, син Тоуев, син на Суфа ефратец.
\par 2 Той имаше две жени; името на едната бе Анна, а името на другата Фенина. И Фенина имаше деца, а Анна нямаше деца.
\par 3 Тоя човек отиваше от града си всяка година, за да се поклони и да принесе жертва на Господа на Силите в Сило, гдето двамата Илиеви синове, Офний и Финеес бяха свещеници пред Господа.
\par 4 И една година , когато настъпи денят, в който Елкана принесе жертва, той даде дялове на жена си Фенина; на всичките й синове, и на дъщерите й;
\par 5 а на Анна даде двоен дял, защото обичаше Анна. Но Господ беше заключил утробата й.
\par 6 А съперницата й я дразнеше много, за да я направи тъжи, загдето Господ беше затворил утробата й.
\par 7 (Така ставаше всяка година; колкото пъти отиваше в Господния дом, така Фенина я дразнеше; а тя плачеше и не ядеше).
\par 8 Но мъжът й Елкана й каза: Анно, защо плачеш? защо не ядеш? и защо е нажалено сърцето ти? Не съм ли ти аз по-желателен от десет сина?
\par 9 А като ядоха в Сило и като пиха, Анна стана. (А свещеникът Илий седеше на стол близо при стълба на вратата при Господния храм).
\par 10 Тя, прочее, преогорчена в духа си, се молеше Господу, и плачеше твърде много.
\par 11 А направи обрек, казвайки: Господи на Силите, ако наистина погледнеш благосклонно към скръбта на слугинята Си, и ме спомниш, и не забравиш слугинята Си, но дадеш на слугинята Си мъжко дете, то ще го дам Господу за през всичките дни на живота му, и бръснач няма да мине през главата му.
\par 12 А като продължаваше да се моли пред Господа, Илий забелязваше устата й.
\par 13 Защото Анна говореше в сърцето си; само устните й мърдаха, а гласът и не се чуваше; затова, на Илия се стори, че беше пияна.
\par 14 За туй Илий й рече: До кога ще си пияна? Остави се от това твое вино.
\par 15 А Анна в отговор рече: Не, господарю мой, аз съм жена преоскърбена в духа си; нито вино, нито спиртно питие съм пила, но излях душата си пред Господа.
\par 16 Не считай слугинята си за лоша жена; защото от голямото си оплакване от скръбта си съм говорила до сега.
\par 17 Тогава Илий в отговор рече: Иди с мир; и Израилевият Бог нека изпълни прошението, което си отправила към Него.
\par 18 И тя рече: Дано слугинята ти придобие благоволението ти. Тогава жената отиде по пътя си, и яде, и лицето й не беше скръбно .
\par 19 И като станаха рано сутринта, та се поклониха пред Господа, върнаха се та отидоха в дома си в Рама. И Елкана позна жена си Анна, и Господ я спомни.
\par 20 И когато се изпълни времето, откак Анна зачна, роди син; и нарече го Самуил; защото, каза : От Господа го изпросих.
\par 21 И Елкана с целия си дом отиде за да принесе Господу годишната жертва и обрека си.
\par 22 Но Анна не отиде, защото рече на мъжа си: Не ще да отида докато не се отбие детето; тогава ще го занеса, за да се яви пред Господа и да живее там за винаги.
\par 23 И мъжът й Елкана й рече: Стори каквото ти се вижда добро; седи докато го отбиеш; само Господ да утвърди словото Си! И тъй, жената седеше и доеше сина си докато го отби.
\par 24 И когато го отби, заведе го със себе си, заедно с един тригодишен юнец, и с една ефа брашно, и с един мех вино, и донесе го в Господния дом в Сило. А детето беше малко.
\par 25 И като заклаха юнеца, донесоха детето при Илия.
\par 26 И Анна рече: О, господарю мой, заклевам се в живота на душата ти, господарю мой, аз съм жената, която бе застанала тук близо при тебе, та се молеше Господу.
\par 27 За това дете се молех: и Господ ми изпълни прошението, което отправих към Него.
\par 28 Затова и аз го доведох на Господа; през всичките дни на живота си ще бъде посветен на Господа. И той се поклони там на Господа.

\chapter{2}

\par 1 Тогава Анна се помоли, като казваше:- Развесели се сърцето ми в Господа; Въздигна се рога ми чрез Господа; Разшириха се устата ми срещу неприятелите ми, Защото се развеселих в спасението Ти.
\par 2 Няма свет какъвто е Господ; Защото няма друг освен Тебе, Нито канара като нашия Бог.
\par 3 Не продължавайте да говорите горделиво; Да не излезе високомерие из устата ви; Защото Господ е Бог на знания, И от него се претеглят делата.
\par 4 Лъковете на силните се строшиха; И немощните се препасаха със сила.
\par 5 Ситите се пазариха за хляб; А гладните престанаха да гладуват . Ей, и не плодната роди седем, А многодетната изнемощя.
\par 6 Господ умъртвява, и съживява; Сваля в ада, и възвежда.
\par 7 Господ осиромашава човека , и обогатява; Смирява човека , и въздига.
\par 8 Въздига бедния от пръстта, И възвишава сиромаха от бунището, За да ги направи да седнат между князете, И да наследят славен престол; Защото стълбовете на земята са на Господа, Който и постави на тях вселената.
\par 9 Ще пази нозете на светиите Си; А нечестивите ще погинат в тъмнината; Понеже със сила не ще надделее човек.
\par 10 Противниците на Господа ще се сломят; Ще гръмне от небето против тях; Господ ще съди краищата на земята, И ще даде сила на царя Си, И ще въздигне рога на помазаника Си.
\par 11 Тогава Елкана си отида у дома си в Рама. А детето слугуваше Господу пред свещеника Илия.
\par 12 А Илиевите синове бяха лоши човеци, които не познаваха Господа.
\par 13 Тия свещеници постъпваха към людете така: когато някой принасяше жертва, като се вареше месото, слугата на свещеника дохождаше с тризъбна вилица в ръка
\par 14 и забождаше я в тенджерата или в котела, или в котлето, или в гърнето; и каквото издигаше вилицата, свещеникът го вземаше за себе си. Така постъпваха в Сило с всичките израилтяни, които дохождаха там.
\par 15 Даже и преди да изгорят тлъстината, слугата на свещеника дохождаше та казваше на човека, който принасяше жертвата: Дай на свещеника месото за печене, защото няма да приеме от тебе варено месо, но сурово.
\par 16 И ако човекът му речеше: Нека изгорят първо тлъстината, и сетне си вземи колкото желае душата ти, тогава казваше: Не, но сега ще дадеш, и ако не, ще взема на сила.
\par 17 Така грехът на тия младежи беше твърде голям пред Господа; защото човеците се отвращаваха от Господната жертва.
\par 18 А Самуил слугуваше пред Господа, дете препасано с ленен ефод.
\par 19 И майка му правеше за него горна дрешка та му донасяше всяка година, когато дохождаше с мъжа си, за да принесе годишната жертва.
\par 20 И Илий благослови Елкана и жена му, като каза: Господ да ти даде рожба от тая жена вместо заема, който дадохте на Господа. И те отидоха на мястото си.
\par 21 И Господ посети Анна; и тя зачна и роди три сина и две дъщери. А детето Самуил растеше пред Господа.
\par 22 А като беше Илий много стар, чу всичко, що прави синовете му на целия Израил, и как лежали с жените, които слугували при входа на шатъра за срещане.
\par 23 И той им рече: Защо правите такива работи? понеже слушам лоши работи за вас от всички тия люде.
\par 24 Недейте, чада мои; защото не е добър слухът, който чувам; вие правите Господните люде да стават престъпници.
\par 25 Ако сгреши човек на човека ще стане моление Богу за него; но ако съгреши някой Господу, кой ще се моли за него? Но те не послушаха гласа на баща си, защото Господ щеше да ги погуби.
\par 26 А детето Самуил растеше и придобиваше благоволението и на Господа и на човеците.
\par 27 Тогава дойде един Божий човек при Илия та му рече: Така казва Господ: Не съм ли Се открил явно на бащиния ти дом когато те бяха в Египет у Фараоновия дом?
\par 28 И не съм ли избрал него измежду всичките Израилеви племена за Мой свещеник, за да принесе жертвата на олтара Ми, да гори темян и да носи ефод пред Мене? И не съм ли дал на бащиния ти дом всичките приноси чрез огън от израилтяните?
\par 29 Защо, прочее, ритате жертвата Ми и приноса Ми, който съм заповядал да принасят в жилището Ми, и почитат синовете си повече от Мене, за да се гоите с по-доброто от всичките приноси на людете Ми Израиля?
\par 30 Затова Господ Израилевият Бог каза: Аз наистина думах, че твоят дом и домът на баща ти ще да ходят пред Мене до века; но сега Господ каза: Далеч от Мене! защото ония, които славят Мене, тях ще прославя Аз, а ония, които Ме презират, ще бъдат презрени.
\par 31 Ето, идат дните, когато ще пресека мишцата на бащиния ти дом, така щото да няма старец в дома ти.
\par 32 И според всичките блага, които ще се дадат на Израиля, в жилището Ми ще видиш утеснение; и не ще има старец в дома ти до века.
\par 33 И оня от твоите, когото не отсека от олтара Си, ще бъде за изнуряване на очите ти и за огорчаване на душата ти; и всичките внуци на дома ти ще умират в средна възраст.
\par 34 И това, което ще дойде върху двамата ти сина, върху Офния и Финееса, ще ти бъде знамение: в един ден и двамата ще умрат.
\par 35 И Аз ще си въздигна верен свещеник, който ще постъпва според това, което е в сърцето Ми и в душата Ми; и ще съградя непоколебим дом; и той ще ходи пред помазаника Ми до века.
\par 36 А всеки, който остане в твоя дом, ще дохожда да му се кланя за малко пари и за един хляб, и ще казва: Назначи ме, моля, на някоя от свещеническите служби, за да ям едно късче хляб.

\chapter{3}

\par 1 А в ония дни, когато детето Самуил слугуваше Господу пред Илия, слово от Господа беше рядкост, и нямаше явно видение.
\par 2 И в онова време, когато Илий лежеше на мястото си, (а очите бяха почнали да ослабват та не можеше да вижда),
\par 3 и Божият светилник не беше още изгаснал в Господния храм, гдето беше Божият ковчег, и Самуил си беше легнал,
\par 4 Господ повика Самуила; и той рече; Ето ме.
\par 5 И завтече се при Илия та рече: Ето ме, защо ме повика. А той рече: Не съм те повикал; върни се та си легни. И той отиде и си легна.
\par 6 А Господ извика още втори път: Самуиле! И Самуил стана та отиде при Илия и рече: Ето, ме, защо ме повика? А той отговори: Не съм те викал, чадо мое; върни се та си легни.
\par 7 Самуил не познаваше още Господа; и слово от Господа не беше му се откривало.
\par 8 И Господ повика Самуила още трети път. И той стана та отиде при Илия и рече: Ето ме, защото ме повика. Тогава Илия разбра, че Господ е повикал детето.
\par 9 Затова, Илий каза на Самуила: Иди та си легни; и ако те повика, кажи: Говори, Господи, защото слугата Ти слуша. И тъй, Самуил отиде та си легна на мястото си.
\par 10 И Господ дойде та застана и извика както по-напред: Самуиле! Самуиле! Тогава Самуил каза: Говори, защото слугата Ти слуша.
\par 11 Тогава Господ каза на Самуила: Ето, Аз ще извърша в Израиля едно такова дело, щото на всеки, който го чуе, ще му писнат двете уши.
\par 12 В оня ден ще извърша против Илия всичко, що говорих за дома му; ща почна и ще свърша.
\par 13 Защото му известих, че ще съдя дома му до века поради беззаконието което той знае; понеже синовете му навлякоха проклетия на себе си, а той не ги възпря.
\par 14 За това се заклех за Илиевия дом, че беззаконието на Илиевия дом няма да се очисти до века с жертва, нито с принос.
\par 15 И Самуил лежа до утринта; после отвори вратата на Господния дом. Но Самуил се боеше да каже видението на Илия.
\par 16 А Илия повика Самуила, казвайки: Самуиле! чадо мое! А той рече: Ето ме.
\par 17 И каза: Какво слово ти говори Господ ? не крий го, моля, от мене. Така да ти направи Бог, да! и повече да притури, ако скриеш от мене някоя от всичките думи, които ти е говорил.
\par 18 Тогава Самуил му каза всичко, и не скри нищо от него. И рече Илий : Господ е; нека стори каквото Му е угодно.
\par 19 И Самуил растеше; и Господ бе с него, и не оставаше да падне на земята ни една от неговите думи.
\par 20 И целият Израил, от Дан до Вир-савее, позна, че Самуил беше потвърден за Господен пророк,
\par 21 И Господ пак се явяваше в Сило; защото Господ се откриваше на Самуила в Сило чрез словеса от Господа. И Самуиловите думи се разнасяха по целия Израил.

\chapter{4}

\par 1 В това време Израил излезе на бой против филистимците и разположиха стан близо при Евен-езер; а филистимците разположиха стан в Афек.
\par 2 И филистимците се опълчиха против Израиля; и когато започнаха битката, Израил се пръсна пред филистимците, и около четири хиляди мъже бяха убити на полесражението.
\par 3 А като дойдоха людете в стана. Израилевите старейшини казаха: Защо ни порази Господ днес пред филистимците? Нека донесем при себе си ковчега за плочите на Господния завет от Сило, тъй щото, като дойде всред нас, да ни избави от ръката на неприятелите ни.
\par 4 И така, людете пратиха в Сило, та донесоха от там ковчега на завета на Господа на Силите, Който обитава между херувимите; и двамата Илиеви сина, Офний и Финеес, бяха там с ковчега на Божия завет.
\par 5 И като дойде ковчегът на Господният завет в стана, целият Израил възкликна с голям глас, тъй щото земята проехтя.
\par 6 А филистимците като чуха шума на възклицанието, рекоха: Що значи това шумно и голямо възклицание в стана на евреите? и научиха се, че Господният ковчег дошъл в стана.
\par 7 Тогава филистимците се уплашиха, защото казваха: Бог е дошъл в стана. И рекоха: Горко ни! Защото такова нещо не е ставало до сега.
\par 8 Горко ни! Кой ще ни избави от ръката на тия мощни богове? Тия са боговете, които поразиха египтяните с всякакви язви в пустинята.
\par 9 Укрепете се, филистимци, бъдете мъжествени, за да не станете слуги на евреите, както те станаха на вас; бъдете мъжествени, та се бийте с тях.
\par 10 Тогава филистимците се биха; и Израил беше поразен, и всеки побягна в шатъра си; и стана много голямо поражение, защото паднаха тридесет хиляди пешаци от Израиля,
\par 11 и Божият ковчег бе хванат; и двамата Илиеви сина, Офний и Финеес, бяха убити.
\par 12 Тогава един човек от Вениамина се завтече от битката, та отиде в Сило, в същия ден, с раздрани дрехи и с пръст на главата си.
\par 13 И като стигна, ето, Илий седеше на стола си край пътя та пазеше, защото сърцето му трепереше за Божия ковчег. И когато дойде човекът в града, та извести това, целият град извика.
\par 14 А Илий, като чу шума на викането каза: Що значи тоя шум и глъч? И човекът дойде бързо та извести на Илия.
\par 15 Илий беше тогава на деветдесет и осем години; и очите му бяха ослабнали та не можеше да вижда.
\par 16 И човекът каза на Илия: Аз дойдох от сражението; още днес избягах от битката. И рече: Що стана, чадо мое?
\par 17 И вестителят в отговор рече: Израил побягна пред филистимците, при това стана голямо поражение между людете, освен това и двамата ти сина, Офний и Финеес, умряха, и Божият ковчег е хванат.
\par 18 А щом спомена за Божия ковчег, Илий падна от стола възнак край портата, и вратът му се строши, и умря; защото беше стар и тежък човек. И той съди Израиля четиридесет години.
\par 19 А снаха му Финеесовата жена, която беше непразна, готова да роди, щом чу известието, че Божият ковчег бил хванат, и че свекърът й и мъжът й умрели, преви се и роди, защото болките й я хванаха.
\par 20 И когато умираше, жените, които стояха около нея, й рекоха: Не бой се, защото си родила син. Но тя не отговори, нито даде внимание.
\par 21 И нарече детето Ихавод, като казваше: Славата се изгуби от Израиля, (защото Божият ковчег се хванал, и защото свекърът й и мъжът й умрели );
\par 22 каза прочее: Славата се изгуби от Израиля, защото Божият ковчег се хвана.

\chapter{5}

\par 1 А филистимците, като хванаха Божия ковчег, занесоха го от Евен-езер в Азот.
\par 2 И филистимците взеха Божия ковчег та го внесоха в капището на Дагона, и поставиха го до Дагона.
\par 3 И на следния ден, когато азотяните станаха рано, ето Дагон паднал с лицето си на земята пред Господния ковчег. И взеха Дагона та го поставиха на мястото му.
\par 4 И на другия ден като станаха рано сутринта, ето Дагон пак паднал с лицето си на земята пред Господния ковчег, и главата на Дагона и двете длани на ръцете му, остечени върху прага; само трупът на Дагона беше останал.
\par 5 (За това, нито Дагоновите жреци, нито някои от ония, които влизат в Дагоновото капище, не стъпват на прага му в Азот до днес).
\par 6 Но ръката на Господа натегна над азотяните, и Той ги изтреби, и порази с хемороиди тях и Азот и околностите му.
\par 7 И като видяха азотските мъже, че такава е работата, рекоха: Ковчегът на Израилевия Бог няма да стои между нас, защото ръката Му тежи върху нас и върху бога ни Дагона.
\par 8 За това, пратиха та събраха при себе си всичките филистимски началници и рекоха: Що да сторим с ковчега на Израилевия Бог? А те отговориха: ковчегът на Израилевия Бог нека се принесе в Гет. И тъй пренесоха в Гет ковчега на Израилевия Бог.
\par 9 Но като го пренесоха, ръката на Господа беше против града с много голямо поражение; и Той удари градските мъже от малък до голям, та избухнаха по тях хемороиди.
\par 10 Затова, пратиха Божия ковчег в Акарон. А като дойде Божият ковчег в Акарон, акаронците извикаха, казвайки: Донесоха ковчега на Израилевия Бог у нас, за да измори нас и людете ни.
\par 11 И тъй, пратиха да съберат всичките филистимски началници, и рекоха: Изпратете ковчега на Израилевия Бог, и нека се върне на мястото си, за да не измори нас и людете ни; защото имаше смъртно поражение по целия град; Божията ръка тежеше там твърде много.
\par 12 И мъжете, които не умряха, бяха поразени с хемороиди; и викът от града се издигна до небето.

\chapter{6}

\par 1 Господният ковчег стоя във Филистимската земя седем месеца;
\par 2 и тогава филистимците повикаха жреците и чародеите и казаха: Що да сторим с Господния ковчег? Кажете ни как да го изпратим на мястото му?
\par 3 А те рекоха: Ако изпратите ковчега на Израилевия Бог, не го изпращайте празен, но непременно му отдайте принос за престъпление; тогава ще оздравеете, и ще узнаете защо ръката Му не се е отделила от вас.
\par 4 И казаха: Какъв принос за престъпление трябва да му отдадем? А те рекоха: Пет златни хемороиди и пет златни мишки, според числото на филистимските началници; защото същата язва беше върху всички вас и върху началниците ви.
\par 5 Затова, да направите подобия на хемороидите си, и подобия на мишките, които повреждат земята ви; и да отдадете слава на Израилевия Бог, та дано би олекчил ръката Си над вас, над боговете ви, и над земята ви.
\par 6 Защо, прочее, закоравявате сърцата си, както египтяните и Фараон закоравяваха сърцата си? След като извърши чудеса всред тях, те не пуснаха ли людете да си отидат, и те тръгнаха?
\par 7 И тъй сега, вземете си една кола, та я пригответе, вземете и две дойни крави, на които хомот не е турят, и впрегнете кравите в колата, а телците им вземете отподир тях и върнете ги у дома.
\par 8 Тогава вземете Господния ковчег та го турете на колата; и златните неща, които Му отдавате в принос за престъпление, турете в ковчежец от страната му; и изпратете ги да иде.
\par 9 И гледайте: ако тръгне по пътя към своята граница у Ветсемес, тогава Той ни е сторил това голямо зло; но ако не, тогава ние ще знаем, че не е Неговата ръка, която ни е поразила, но това ни е постигнало случайно.
\par 10 И мъжете сториха така: взеха две дойни крави, та ги впрегнаха в колата, а телците им затвориха у дома.
\par 11 И положиха на колата Господния ковчег, и ковчежеца със златните мишки и подобията на хемороидите си.
\par 12 И кравите се отправиха по пътя направо за Ветсемес; все по друма вървяха, и ревяха, като отиваха, без да се обръщат ни на дясно, ни на ляво; а филистимските началници идеха подир тях до границата на Ветсемес.
\par 13 А ветсемесците жънеха пшеницата си в долината; и като подигнаха очи, видяха ковчега, и като го видяха, зарадваха се.
\par 14 И колата влезе в нивата на Исуса ветсемесеца та застана там, гдето имаше голям камък; и нацепиха дървата на колата, та принесоха кравите във всеизгаряне Господу.
\par 15 Тогава левитите снеха Господния ковчег и ковчежеца, който беше с него, в който бяха златните неща, и ги положиха на големия камък; и в същия ден ветсемеските мъже принесоха всеизгаряния и пожертвуваха жертви Господу.
\par 16 А петимата филистимски началници, като видяха това, върнаха се в Акарон в същия ден.
\par 17 А златните хемороиди, които филистимците отдадоха Господу в принос за престъпление, бяха следните: един за Азот, един за Газа, един за Аскалон, един за Гет, един за Акарон;
\par 18 а златните мишки бяха според числото на всичките филистимски градове принадлежащи на петимата началници, както оградените градове, така и неоградените села, дори до големия камък, на който положиха Господния ковчег, - камък, който стои до днес в нивата на Исуса ветсемесеца.
\par 19 Но Господ порази ветсемеските мъже за гдето погледнаха в Господния ковчег, като порази от людете петдесет хиляди е седемдесет мъже; и людете плакаха, защото Господ порази людете с голямо изтребление.
\par 20 И ветсемеските мъже казаха: Кой може да застане пред Господа, пред светия тоя Бог? и при кого да отиде Той от нас?
\par 21 Сетне изпратиха вестители до кириатиаримските жители да кажат: Филистимците донесоха назад Господния ковчег; слезте и изкачете го при себе си.

\chapter{7}

\par 1 Тогава кириатиаримските мъже дойдоха та дигнаха Господния ковчег, и донесоха го в Авинадавовата къща на хълма; и осветиха сина му Елеазара, за да пази Господния ковчег.
\par 2 И то деня, когато ковчегът бе положен в Кириатиарим, мина се много време - двадесет години; и целият Израилев дом въздишаше за Господа.
\par 3 И Самуил говори на целия Израилев дом, казвайки: Ако се обръщате от все сърце към Господа, махнете отсред себе си чуждите богове и астартите, та пригответе сърцата си за Господа и само Нему служете; и Той ще ви избави от ръката на филистимците.
\par 4 Тогава израилтяните махнаха ваалимите и астартите та служеха само на Господа.
\par 5 После Самуил каза: Съберете целия Израил в Масфа, и ще се помоля за вас Господу.
\par 6 И тъй събраха се в Масфа, и наляха вода, която изляха пред Господа, и постиха през оня ден, и рекоха там: Съгрешихме на Господа. И Самуил съдеше израилтяните в Масфа.
\par 7 А като чуха филистимците, че израилтяните се събрали в Масфа, филистимските началници излязоха против Израиля. Като чуха това израилтяните уплашиха се от филистимците.
\par 8 И израилтяните рекоха на Самуила: Не преставай да викаш за нас към Господа нашия Бог, за да ни избави от ръката на филистимците.
\par 9 За това, Самуил взе едно млечниче агне, та го принесе цяло всеизгаряне Господу; и Самуил извика към Господа за Израиля, и Господ го послуша;
\par 10 защото, когато принасяше Самуил всеизгарянето, понеже филистимците се приближиха да се бият с Израиля, в същия ден Господ гръмна със силен гръм върху филистимците и ги смути; и те бяха поразени пред Израиля.
\par 11 Израилевите мъже излязоха от Масфа, та гониха филистимците, и поразиха ги до под Вет-хар.
\par 12 Тогава Самуил взе един камък та го постави между Масфа и Сен, и нарече го Евен-езер, като казваше: До тука ни помогна Господ.
\par 13 Така филистимците бяха покорени и не дойдоха вече в Израилевите предели; и Господната ръка беше против филистимците през всичките дни на Самуила.
\par 14 И градовете, които филистимците бяха превзели от Израиля, бяха повърнати на Израиля, от Акарон до Гет; и Израил освободи техните околности от ръката на филистимците. А между Израиля и аморейците имаше мир.
\par 15 И Самуил съдеше Израиля през всичките дни на живота си.
\par 16 И всяка година той отиваше да обикаля Ветил, Галгал и Масфа, и съдеше Израиля във всички тия места;
\par 17 а после се връщаше в Рама, защото домът му беше там, па и там съдеше Израиля. Там издигна и олтар на Господа.

\chapter{8}

\par 1 Самуил, когато остаря, постави синовете си съдии над Израиля.
\par 2 Името на първородния му беше Иоил, а името на втория му Авия; те бяха съдии във Вир-савее.
\par 3 Но синовете му не ходеха в неговите пътища, но се отклониха та отиваха след сребролюбието, и вземаха подкуп, и извръщаха правосъдието.
\par 4 Тогава всичките Израилеви старейшини се събраха, та дойдоха при Самуил в Рама и му рекоха:
\par 5 Ето, ти остаря, и синовете ти не ходят в твоите пътища; постави ни, прочее, цар, който да ни съди, както е у всичките народи.
\par 6 Обаче на Самуила не бе угодно гдето рекоха: Дай ни цар, който да ни съди. И Самуил се помоли на Господа.
\par 7 А Господ каза на Самуила: Послушай гласа на людете за всичко, що ти говорят, защото не отхвърлиха тебе, но Мене отхвърлиха, за да не царувам над тях.
\par 8 Според всичките дела, които те са вършили от деня, когато съм ги извел от Египет дори до тоя ден, като са Ме оставили и са служили на други богове, така правят и на тебе.
\par 9 Сега, прочее, слушай гласа им, обаче тържествено протестирай пред тях, и покажи им как ще постъпва царят, който ще царува над тях.
\par 10 Самуил, прочее, каза всичките Господни думи на людете, които искаха цар от него.
\par 11 Каза още: Така ще постъпва царят, който ще се възцари над вас; ще взема синовете ви и ще ги определя за колесниците си, и да му бъдат конници, и за да тичат пред колесницата му.
\par 12 И ще си ги назначава хилядници и петдесетници, и ще ги поставя да работят земята му, да жънат жетвата му, и да правят военните му оръжия и приборите за колесниците му.
\par 13 Ще взема и дъщерите ви за мироварици и готвачки и хлебарки.
\par 14 И ще взема по-добрите от нивите ви, лозята ви и маслините ви и ще ги дава на слугите си.
\par 15 Ще взема и десетък от посевите ви и от лозята ви и ще ги дава на скопците си и на слугите си.
\par 16 И ще взема слугите ви, слугините ви, по-добрите момчета, и ослите ви и ще ги употребява в своите работи.
\par 17 Ще взема десетък от стадата ви; и вие ще му бъдете слуги.
\par 18 В оня ден ще викате поради церя си, когото ще сте си избрали; но Господ няма да ви послуша в оня ден.
\par 19 Обаче, людете не искаха да послушат Самуиловия глас, а рекоха: Не, но цар нека има над нас,
\par 20 за да бъдем и ние както всичките народи, и нашият цар да ни служи, и да ни предвожда, и да воюва в боевете ни.
\par 21 И Самуил, като изслуша всичките думи на людете, каза ги в ушите на Господа.
\par 22 А Господ каза на Самуила: Послушай гласа им и постави им цар. Тогава Самуил каза на Израилевите мъже: Идете всеки в града си.

\chapter{9}

\par 1 Имаше човек от Вениамина на име Кис, син на Авиила, Сина на Серора, син на Вехората, син на Афия, човек вениаминец, силен и храбър.
\par 2 А той имаше син не име Саул, отборен и твърде красив; между израилтяните нямаше човек по-красив от него; от рамената си и нагоре беше по-висок от всичките люде.
\par 3 И ослиците на Сауловия баща се изгубиха; затова Кис каза на сина си Саула: Вземи сега със себе си един от слугите и стани та иди да търсиш ослиците.
\par 4 И тъй, той мина през Ефремовата хълмиста земя, мина и през земята Салиса, но не ги намериха; после мина през земята Саалим, но и там ги нямаше; и заминаха през Вениаминовата земя, но не ги намериха.
\par 5 А когато дойдоха в земята Суф, Саул каза на слугата, който беше с него: Ила да се върнем; да не би баща ми да престане да се грижи за ослиците, и да започне да мисли за нас.
\par 6 А той му рече: Ето в тоя град има Божий човек, човек, който е на почит; всичко що казва непременно се сбъдва; да идем, прочее, там, негли би могъл да ни каже нещо за това, поради което пътуваме.
\par 7 И Саул каза на слугата си: Но ако отидем, какво да занесем на човека? защото хлябът в съдовете ни се свърши, и няма подарък да занесем на Божия човек. Що имаме?
\par 8 И слугата пак отговори на Саула, казвайки: Ето в ръката ми се намира четвърт сребърен сикъл, който ще дам на Божия човек, за да ни каже пътя.
\par 9 (В старо време в Израиля, когато някой отиваше да се допита до Бога, думаше така: Елате, да идем при гледача; защото оня, който се нарича пророк, се наричаше по-напред гледач).
\par 10 Тогава Саул каза на слугата си: Добре казваш; ела да идем. И тъй отидоха в града, гдето беше Божият човек.
\par 11 И като се изкачваха по нагорнището към града, намериха моми излезли да налеят вода и казаха им: Тук ли е гледачът?
\par 12 А те в отговор казаха им: Тук е; ето го пред тебе; побързай сега, защото днес дойде в града, понеже людете имат днес жертва на високото място.
\par 13 Щом възлезете в града, ще го намерите преди да се изкачи на високото място да яде; защото людете не ядат до като не дойде той, понеже той благославя жертвата; след това поканените ядат. И тъй, качете се сега, защото около тоя час ще го намерите.
\par 14 Прочее, те се изкачиха към града; и като влизаха в града, ето Самуил излизаше насреща им, за да се изкачи на високото място.
\par 15 А Господ беше открил на Самуила, един ден преди да дойде Саул, като беше казал:
\par 16 Утре около тоя час ще изпратя до тебе човек от Вениаминовата земя; него да помажеш княз над людете Ми Израиля, и той ще избави людете Ми Израиля, и той ще избави людете Ми от ръката на филистимците; защото Аз погледнах благосклонно към людете Си, понеже викът им стигна до Мене.
\par 17 И когато Самуил видя Саула, Господ му рече: Ето човекът, за когото ти говорих! Той ще началствува над людете Ми.
\par 18 В същото време Саул се приближи до Самуила в портата и каза: Покажи ми, моля, где е къщата на гледача.
\par 19 И Самуил в отговор рече на Саула: Аз съм гледачът; изкачвай се пред мене на високото място, за да ядете днес с мене; а утре ще те изпратя, и ще ти явя всичко, що имаш на сърцето си.
\par 20 А колкото за ослиците ти, които се изгубиха преди три дена, нямай грижа за тях, защото се намериха. И към кого е всичкото желание на Израиля? не е ли към тебе и към целия дом на баща ти?
\par 21 А Саул в отговор рече: Не съм ли аз Вениаминец, от най-малкото от Израилевите племена? и не е ли семейството ми най-малко от всичките семейства на Вениаминовото племе? Защо, прочее, ми говориш по тоя начин?
\par 22 А Самуил взе Саула и слугата му та ги въведе в гостната стая, и даде им първото място между поканените, които бяха около тридесет души.
\par 23 Тогава Самуил каза на готвача: Донеси дела, който ти дадох, за който ти рекох: Пази това при себе си.
\par 24 И така, готвачът дигна бедрото и онова, що беше върху него та го сложи пред Саула. И Самуил рече: Ето запазеното; тури го пред себе си та яж, защото за тоя час то е било запазено за тебе след като рекох, че ще поканя людете. И тъй Саул яде със Самуила в оня ден.
\par 25 И като слязоха от високото място в града, Самуил се разговори със Саула на къщния покрив.
\par 26 И станаха рано; и около зазоряване Самуил повика Саула на къщния покрив, казвайки: Стани да те изпратя. И Саул стана, та излязоха вън двамата, той и Самуил.
\par 27 А като слизаха към градския край, Самуил каза на Саула: Заповядай на слугата да замине пред нас, (и той замина); а ти постой малко, за да ти известя Божието слово.

\chapter{10}

\par 1 Тогава Самуил взе съда с миро та го изля на главата му, целуна го и каза: Не помаза ли те Господ за княз над наследството Си?
\par 2 Като си заминеш днес от мене, ще намериш двама човека близо при Рахилиния гроб, във Вениаминовата земя у Селса; и те ще ти рекат: Намериха се ослиците, които ти отиде да търсиш; и, ето, баща ти престана да се грижи за ослиците, та много скърби за вас, като казва: Какво да правя за сина си?
\par 3 И като идеш по-нататък, от там ще дойдеш до дъба на Тавор, и там ще те посрещнат трима човека, които отиват към Бога във Ветил, от които един носи три ярета, а друг носи три хляба, а друг носи мех с вино;
\par 4 и те ще те поздравят и ще ти дадат две хляба, които да приемеш от ръцете им.
\par 5 После ще стигнеш до Божия хълм, гдето е филистимският гарнизон; и като стигнеш там, в града, ще срещнеш дружина пророци, слизащи от високото място, предшествувани от псалтир, тъпанче, свирка и китара, и те пророкуващи.
\par 6 Тогава ще дойде Господният Дух върху тебе, та ще пророкуваш заедно с тях, и ще се промениш в друг човек.
\par 7 А когато тия знамения дойдат на тебе, прави каквото случаят позволява; защото Бог е с тебе.
\par 8 После слез ти пред мене в Галгал; и, ето, аз ще сляза при тебе, за да принеса всеизгаряния и да пожертвувам примирителни жертви; чакай седем дена, докато дойда при тебе и ти кажа що да сториш.
\par 9 И когато обърна гърба си да си замине от Самуила, Бог му даде друго сърце; и всички ония знамения се сбъднаха в същия ден.
\par 10 Като дойдоха там на хълма, ето, дружина пророци ги посрещнаха: и Божият Дух дойде със сила върху него, и той пророкува между тях.
\par 11 И като видяха всички, които го познаваха по-напред, че, ето, пророкуваше между пророците, тогава людете казваха едни на други: Що е станало с Кисовия син? и Саул ли е между пророците?
\par 12 А един от ония, които бяха от там, проговори казвайки: Но кой е техният баща? От това стана поговорка: И Саул ли е между пророците?
\par 13 И като свърши пророкуването си, дойде на високото място.
\par 14 И стриката на Саула рече на него и на слугата му: Къде ходите? И той каза: Да търсим ослите; и когато видяхме, че ги няма, отидохме при Самуила.
\par 15 И Сауловият стрика рече: Я ми кажи що ви рече Самуил.
\par 16 И Саул рече на стрика си: Каза ни положително, че ослиците се намериха. Но не му яви това, което Самуил беше казал за царството.
\par 17 След това Самуил събра людете при Господа в Масфа
\par 18 и рече на израилтяните: Така говори Господ Израилевият Бог: Аз изведох Израиля из Египет, и ви избавих от ръката на египтяните и от ръката на всичките царства, които ви притесняваха.
\par 19 А вие днес отхвърлихте вашия Бог, Който сам ви избави от всичките ви бедствия и скърби, и рекохте Му: Непременно да поставиш цар над нас. Сега, прочее, застанете пред Господа според племената си и според хилядите си.
\par 20 И когато Самуил накара да се приближат всичките Израилеви племена, хвана се с жребие Вениаминовото племе.
\par 21 И като накара да се приближи Вениаминовото племе според семействата си, хвана се семейството на Матри; и хвана го се Кисовият син Саул, но като го потърсиха, той не се намери.
\par 22 Затова, допитаха се пак до Господа: Човекът дошъл ли е вече тук? И Господ отговори: Ето, той се е скрил между вещите.
\par 23 Тогава се завтекоха та го взеха от там; и като застана между людете, беше по-висок от всичките люде от рамената си и нагоре.
\par 24 Тогава Самуил каза на всичките люде: Виждате ли онзи, когото Господ избра, че няма подобен на него между всичките люде? И всичките люде извикаха, казвайки: Да живее царят!
\par 25 После Самуил съобщи на людете как ще се реди царството, и като го написа в книга, положи я пред Господа. Тогава Самуил разпусна всичките люде, всеки у дома му.
\par 26 Също и Самуил отиде у дома си в Гавая, и с него отиде един полк силни мъже, до чиито сърца Бог беше се докоснал.
\par 27 Но някои лоши човеци рекоха: Как ще ни избави той? И презираха го, и не му принасяха дарове; а той се правеше на глух.

\chapter{11}

\par 1 След това амонецът Наас възлезе та разположи стан против Явис-галаад; и всичките явиски мъже казаха на Нааса: Направи договор с нас, и ще ти слугуваме.
\par 2 И амонецът Наас из каза: С това условие ще направя договор с вас: да извъртя десните очи на всички ви; и ще положа това за позор върху целия Израил.
\par 3 И явиските старейшини му казаха: Дай ни седем дена срок, за да пратим вестители по всичките предели на Израиля; и тогава, ако няма кой да ни избави, ще излезем към тебе.
\par 4 И тъй вестителите дойдоха в Сауловия град Гавая та казаха тия думи на всеослушание пред людете; и всичките люде плакаха с висок глас.
\par 5 А, ето, Саул идеше подир воловете от полето: и Саул рече: Какво е на людете та плачат? И съобщиха му думите на явиските мъже.
\par 6 Тогава дойде Божият Дух със сила върху Саула, когато чу ония думи, и гневът му пламна твърде много.
\par 7 И взе два вола, и като ги наряза на части прати ги по всичките предели на Израиля чрез вестители, за да кажат: Който не излезе подир Саула и подир Самуила, така ще се направи на воловете му. И страх от Господа обзе людете; и излязоха като един човек.
\par 8 И като ги преброи у Везек, израилтяните бяха триста хиляди души, а Юдовите мъже тридесет хиляди души.
\par 9 И рекоха на вестителите, които бяха дошли: Така да кажете на явис-галаадските мъже, утре като припече слънцето, ще ви дойде избавление. А когато вестителите дойдоха и известиха на явиските мъже, те се зарадваха.
\par 10 И явиските мъже рекоха на амонците : Утре ще излезем към вас, и вие ни направете всичко, що ви се вижда угодно.
\par 11 И на утринта Саул раздели людете на три полка; и във време на утринната стража те навлязоха всред стана, и поразяваха амонците докато се стопли денят; и оцелелите от тях се разпиляха до толкоз, щото нито двама от тях не останаха заедно.
\par 12 Тогава людете казаха на Самуила: Кой е онзи що рече: Саул ли ще се възцари над нас? Доведете тия мъже, за да ги избием.
\par 13 Но Саул каза: Никой няма да бъде убит тоя ден; защото Господ днес извърши избавление в Израиля.
\par 14 Тогава Самуил каза на людете: Дойдете да идем в Галгал и там да възобновим царството.
\par 15 И така, всичките люде отидоха в Галгал, и поставиха Саула Цар пред Господа там в Галгал; там и пожертвуваха примирителни жертви пред Господа; и Саул и всичките Израилеви мъже се развеселиха там твърде много.

\chapter{12}

\par 1 Тогава Самуил каза на целия Израил: Ето, послушах гласа ви за всичко, що ми казахте, и поставих цар над вас.
\par 2 И сега, ето, царят ви предвожда; а аз съм стар и белокос, и, ето, синовете ми са с вас; и обходата ми от младостта ми до днес е пред вас.
\par 3 Ето ме: свидетелствувайте против мене пред Господа и пред помазаника Му - Кому съм взел вола? или кому съм взел осела? или кого съм онеправдал? кого съм притеснил? или из ръката на кого съм взел подкуп, за да ослепя очите си с него, та да ви го върна?
\par 4 А те рекоха: Не си ни онеправдал, нито си ни притеснил, нито се взел нещо от ръката на някого.
\par 5 И рече им: Свидетел ви е Господ, свидетел е и Неговият помазаник Днес, че не намерихте нищо в ръката ми. И те отговориха: Свидетел е.
\par 6 И рече Самуил на людете: Господ е, Който постави Моисея и Аарона и изведе бащите ви от Египетската земя.
\par 7 Сега, прочее, застанете, за да разсъждавам с вас пред Господа за всичките справедливи дела, които Господ направи вам и на бащите ви.
\par 8 Когато Яков дойде в Египет, и бащите ви извикаха към Господа, тогава Господ прати Моисея и Арона, които изведоха бащите ви из Египет и заселиха ги на това място.
\par 9 Те обаче забравиха Господа своя Бог; затова ги предаде в ръката на Сисара, асорския военачалник, в ръката на филистимците и в ръката на моавския цар, които воюваха против тях.
\par 10 Тогава те извикаха към Господа, казвайки: Съгрешихме, понеже оставихме Господа та служихме на ваалимите и на астартите; но сега избави ни от ръката на неприятелите ни, и ще Ти служим.
\par 11 И Господ изпрати Ероваала, Водана, Ефтая и Самуила та ви избави от ръката на неприятелите ви от всякъде; и вие живеехте в безопасност.
\par 12 Но когато видяхте, че Наас, царят на амонците, дойде против вас, рекохте ми: Не, но цар да царува над нас, - когато Господ вашият Бог ви беше цар.
\par 13 Сега, прочее, ето царят, когото избрахте, когато искахте! и, ето Господ постави цар над вас.
\par 14 Ако се боите от Господа и Му служите, и слушате гласът Му, и не въставате против Господното повеление, и следвате Господа вашия Бог, както вие, така и царят, който царува над вас, добре ;
\par 15 но, ако не слушате Господния глас, а въставате против Господното повеление, тогава Господната ръка ще бъде против, вас, както беше против бащите ви.
\par 16 Сега, прочее, застанете та вижте това велико дело, което Господ ще направи пред очите ви.
\par 17 Не е ли днес жетва на пшеницата? Ще призова Господа; и Той ще прати гръмове и дъжд, за да познаете и видите, че злото, което направихте, като си поискахте цар, е голямо пред Господа.
\par 18 Тогава Самуил призова Господа; и Господ прати гръмове и дъжд през същия ден; и всичките люде се уплашиха твърде много от Господа и от Самуила.
\par 19 И всичките люде казаха на Самуила: Помоли се за слугите си на Господа твоя Бог, за да не измрем; защото върху всичките си грехове притурихме и това зло, да искаме за себе си цар.
\par 20 И Самуил каза на людете: Не бойте се; вие наистина сторихте всичко това зло; но да се не отклоните та да не следвате Господа, а служете Господу от все сърце;
\par 21 и да се не отклоните, защото тогава ще идете след суетностите, които не могат да ползват, или да избавят, понеже са суетни.
\par 22 Защото Господ няма да остави людете Си заради великото Си Име, понеже Господ благоволи да ви направи Свои люде.
\par 23 А колкото за мене, да не даде Бог да съгреша на Господа, като престана да се моля за вас! но ще ви уча добрия и правия път.
\par 24 Само бойте се от Господа, и служете Му искрено от все сърце; защото помислете колко велики дела извърши Той за вас.
\par 25 Но ако следвате да струвате зло, това и вие и царят ви ще погинете.

\chapter{13}

\par 1 До това време Саул беше царувал една година; а като царува две години над Израиля,
\par 2 Саул си избра три хиляди мъже от Израиля, от които две хиляди бяха със Саула в Михмас и в хълма Ветил, а хиляда бяха с Ионатан в Гавая Вениаминова; а останалата част от людете изпрати, всеки в шатъра му.
\par 3 И Ионатан порази филистимския гарнизон, който беше в Гава и филистимците чуха.
\par 4 И така, целият Израил чу да говорят, бе Саул поразил филистимския гарнизон, и че Израил станал омразен на филистимците. И людете се събраха в Галгал да следват Саула.
\par 5 А филистимците се събраха да се бият с Израиля, тридесет хиляди колесници и шест хиляди конници и люде по множество, като пясъка край морския бряг; и възлязоха та разположиха стан в Михмас, на изток от Ветавен.
\par 6 Когато Израилевите мъже видяха, че бяха поставени на тясно, (защото людете бяха на тясно), тогова людете се скриха в пещерите, в гъсталаците, в скалите, в канарите и в рововете.
\par 7 И някои от евреите преминаха Иордан към Гадовата и Галаадската земя. А колкото за Саула, той беше още в Галгал; и всичките люде вървяха подир него с трепет.
\par 8 И чака седем дена, според определеното от Самуила време; а понеже Самуил не беше дошъл в Галгал, и людете се разпръсваха от него,
\par 9 затова рече Саул: Донесете тука при мене всеизгарянето и примирителните приноси. И той принесе всеизгарянето.
\par 10 И щом свърши принасянето на всеизгарянето, ето, Самуил дойде; и Саул излезе да го посрещне и да го поздрави.
\par 11 А Самуил каза: Що стори ти? И Саул рече: Понеже видях, че людете се разпръсваха от мана, и че ти не дойде на определеното време, а филистимците се събраха в Михмас,
\par 12 затова си рекох, сега филистимците ще нападнат върху мене в Галгал, а пък аз не съм отправил молба към Господа; и тъй, дръзнах та принесох всеизгарянето.
\par 13 И Самуил каза на Саула: Безумие си сторил ти, гдето не опази повелението, което Господ твоят Бог ти заповяда; защото Господ щеше сега да утвърди царството ти над Израиля до века.
\par 14 Но сега царството ти няма да трае; Господ си потърси човек според сърцето Си, и него определи да бъде княз над людете Му, понеже ти не опази онова, което Господ ти заповяда.
\par 15 Тогава Самуил стана та отиде от Галгал в Гавая Вениаминова. А Саул преброи людете, които се намираха с него; и те бяха около шестотин мъже.
\par 16 Саул и син му Ионатан и людете, които се намираха с тях, седяха в Гавая Вениаминова; а филистимците бяха разположили стан в Михмас.
\par 17 И от филистимския стан излязоха три чети грабители; една чета се отправи по пътя за Офра към земята Согал;
\par 18 друга чета се отправи по пътя за Веторон; а друга чета се отправи по пътя за околността, който гледа към долината Севоим къде пустинята.
\par 19 А в цялата Израилева земя не се намираше ковач, защото филистимците рекоха: Да не би евреите да си направят мечове или копия,
\par 20 но всичките израилтяни слизаха при филистимците, за да наклепят всеки търнокопа си, палешника си, брадвата си и мотиката си.
\par 21 Обаче, имаха пила за търнокопите, палешниците, тризъбците и брадвите, и за да острят остените.
\par 22 Затова, в деня на боя не се намираше нито нож, ни копие в ръката на някого от людете, които бяха със Саула и Ионатана; в Саула, обаче, и в сина му Ионатана намериха се.
\par 23 А филистимския гарнизон излезе към прохода Михмас.

\chapter{14}

\par 1 А един ден Сауловият син Ионатан каза на момъка оръженосеца си; Дойди да преминем към филистимския гарнизон, който е насреща. Но на баща си не каза това.
\par 2 А Саул седеше при Гавайския край, под наровото дърво, което е в Мигрон; и людете, които бяха с него възлизаха на около шестотин мъже.
\par 3 И Ахия син на Ахитова, брат на Ихавода, син на Финееса, син на Илия, бе Господният свещеник в Сило и носеше ефод. И людете не знаеха, че Ионатан е преминал.
\par 4 А между проходите, през които Ионатан искаше да мине към филистимския гарнизон, имаше остра скала от едната страна, и остра скала от другата страна; името на едната беше Восес, а името на другата Сене.
\par 5 Едната скала се издигаше на север срещу Михмас, а другата на юг от Гавая.
\par 6 Ионатан, прочее, каза на момъка оръженосеца си: Дойди да преминем към гарнизона на тия необрязани, негли Господ подействува за нас; защото нищо не пречи на Господ да спаси чрез мнозина или чрез малцина.
\par 7 И рече му оръженосецът му: Стори все що ти е на сърце; върви напред; ето, аз съм с тебе според сърцето ти.
\par 8 Тогава рече Ионатан: Ето, ще преминем към тия мъже и ще им се явим.
\par 9 Ако ни говорят така: Стойте докато дойдем при вас, тогава ще застанем на мястото си и няма да се възкачим при тях.
\par 10 Но ако говорят така: качете се при нас, тогава ще се възкачим, защото Бог ги предаде в ръката ни. Това ще ни служи за знак.
\par 11 И тъй, и двамата се появиха на филистимския гарнизон; и филистимците рекоха: Ето, евреите излизат из дупките, гдето бяха се скрили.
\par 12 И мъжете на гарнизона проговориха на Ионатана и на оръженосеца му казвайки: Качете се при нас и ще ви покажем нещо. Тогава Ионатан каза на оръженосеца си: Възкачи се след мене, защото Господ ги предаде в Израилевата ръка.
\par 13 И тъй, Ионатан пълзя нагоре с ръцете си и с нозете си, и оръженосецът му след него. И те паднаха пред Ионатана; и оръженосецът му ги убиваше след него.
\par 14 И в това първо поражение, което Ионатан и оръженосецът му нанесоха, паднаха около двадесет мъже, в едно пространство от половин уврат земя.
\par 15 И стана трепет в стана, по нивите, и между всичките люде; гарнизона и грабителите също потрепераха, и земята се тресеше, така щото стана твърде голям трепет.
\par 16 И като наблюдаваха Сауловите стражи в Гавая Вениаминова, ето, множеството се разтопяваше и се разотиваше тук таме.
\par 17 Тогава Саул каза на людете, които бяха с него: Пребройте сега, та вижте, кой от нас е отишъл. И като преброиха, ето, Ионатан и оръженосецът му ги нямаше.
\par 18 И Саул каза на Ахия: Донеси тук Божия ковчег, (защото в това време Божият ковчег беше там с израилтяните).
\par 19 А докато говореше Саул на свещеника, смущението във филистимския стан продължаваше да се увеличава; затова Саул каза на свещеника: Оттегли ръката си.
\par 20 И Саул и всичките люде, които бяха с него, се събраха та дойдоха до сражението; и, ето, мечът на всекиго бе против другаря му, и имаше твърде голямо поражение.
\par 21 И евреите, които по-напред бяха с филистимците, и които бяха дошли с тях в стана от околните местности , те също е обърнаха да помогнат на израилтяните, които бяха със Саула и Ионатана.
\par 22 Тоже и всичките Израилеви мъже, които бяха се крили в Ефремовата хълмиста земя, като чуха, че филистимците бягали, завтекоха се и те в сражението да ги преследват.
\par 23 Така в оня ден Господ избави Израиля; и битката се простря до Ветавен.
\par 24 А Израилевите мъже се измъчиха в оня ден, защото Саул закле людете, казвайки: Проклет онзи, който вкуси храна до вечерта, докато отмъстя на неприятелите си. Затова никой от людете не вкуси храна.
\par 25 А като дойдоха всичките люде в един гъсталак, гдето имаше мед по земята,
\par 26 и като влязоха людете в гъсталака, ето, медът покапваше; но никой не приближи ръка до устата си, защото людете се бояха от клетвата.
\par 27 Ионатан, обаче, не беше чул че баща му заклел людете, затова простря края на тоягата, която беше в ръката му, та я натопи в медената пита, и тури ръката си в устата си; и светна му на очите.
\par 28 А един от людете проговори, казвайки: Баща ти строго закле людете, като каза: Проклет оня, който вкуси храна днес, - макар че людете бяха изнемощели.
\par 29 А Ионатан каза: Баща ми смути света. Я вижте как ми светна на очите, защото вкусих малко от тоя мед;
\par 30 колко повече, ако людете бяха яли днес свободно от користите, които намериха у неприятелите си! защото не щеше ли сега да стане по-голямо клане на филистимците?
\par 31 И през оня ден те поразиха филистимците от Михмас до Еалон; но людете бяха много изнемощели.
\par 32 Затова людете се нахвърлиха върху користите, и като взеха овци, говеда и телци заклаха ги по земята; и людете ядоха с кръвта.
\par 33 Тогава известиха на Саула като му казаха: Ето, людете съгрешават на Господа, защото ядат с кръвта. А той каза: Престъпници станахте; търколете голям камък към мене преди да се свърши денят.
\par 34 И рече Саул: Разотивайте се между людете та им кажете: Докарайте ми тук всеки говедото си и всеки овцата си, та заколете тук и яжте; и не съгрешавайте на Господа като ядете с кръвта. И тъй, оная нощ всичките докараха всеки говедото си със себе си и ги заклаха там.
\par 35 И Саул издигна олтар на Господа; това беше първият олтар, който той издигна на Господа.
\par 36 Тогава Саул каза: Да слезем подир филистимците през нощта, и да ги разграбим преди да се развидели и да не оставим ни един от тях. А те казаха: Направи каквото ти се вижда добро. Тогава рече свещеникът: Да се приближим тук при Бога.
\par 37 И Саул се допита до Бога: Да сляза ли против филистимците? ще ги предадеш ли в ръката на Израиля? Но не му отговори оня ден.
\par 38 Тогава рече Саул: Приближете се тук всички краища на людете та се научете и вижте у кого е било прегрешение днес;
\par 39 защото заклевам се в живота на Господа, Който избавя Израиля, даже и в сина ми Ионатана ако бъде; непременно ще се умъртви. Но не му отговори ни един между всичките люде.
\par 40 И рече на целия Израил: Застанете вие на едната страна, и аз и синът ми Ионатан ще застанем на другата страна. И людете казаха на Саула: Стори каквото ти се вижда добро.
\par 41 Тогава рече Саул на Господа Израилевия Бог: Покажи чрез жребието истината. И хванаха се Ионатан и Саул, а людете бяха отпуснати.
\par 42 И рече Саул: хвърлете жребие между мене и сина ми Ионатана. И хвана се сина му Ионатан.
\par 43 Тогава Саул каза на Ионатана: Кажи ми какво си сторил. И Ионатан му яви, като рече: Наистина вкусих малко мед с края на тоягата, който имах в ръката си; и, ето, трябва да умра!
\par 44 И рече Саул: Така да направи Бог. да! и повече да притури; непременно ще умреш, Ионатане.
\par 45 А людете рекоха на Саула: Ионатан ли ще умре, който извърши това велико избавление в Израиля? Да не бъде Бог! Заклеваме се в живота на Господа, нито един косъм от главата му няма да падне на земята; защото той действува с Бога днес. Така людете избавиха Ионатана, та не умря.
\par 46 Тогава Саул се върна от преследването на филистимците; а филистимците отидоха на мястото си.
\par 47 А Саул, като бе поел царуването над Израиля, воюва против всичките си околни неприятели: против Моава, против амонците, против Едома, против совските царе и против филистимците; и на където и да се обръщаше, все побеждаваше;
\par 48 и като действуваше бързо порази и Амалика, и избави Израиля от ръката на ония, които ги разграбваха.
\par 49 А синовете на Саула бяха: Ионатан, Иисуй и Мелхисуе: и имената на двете му дъщери бяха Мерава, името на първородната, и Михала, името на по-младата.
\par 50 А името на жената на Саула беше Ахиноама, Ахимаасова дъщеря; и името на военачалника му беше Авенир, син на Сауловия стрика Нир.
\par 51 А Кис Сауловият баща и Нир Авенировият баща, бяха Авиилови синове.
\par 52 И през всичките дни на Саула се водеше силна война против филистимците; и когато Саул виждаше някой мъж силен или храбър вземаше го при себе си.

\chapter{15}

\par 1 След това Самуил каза на Саула: Господ ме изпрати да те помажа цар над людете Му, над Израиля; сега, прочее, послушай гласа на Господните думи.
\par 2 Така казва Господ на Силите: Забелязал съм онова, което стори Амалик на Израиля, как му се възпротиви на пътя, когато идеше от Египет.
\par 3 Иди сега та порази Амалика, обречи на изтребление всичко, що има, и не го пожали; но избий мъж и жена, дете и бозайниче, говедо и овца, камила и осел.
\par 4 Прочее, Саул повика людете та ги преброи в Телаам и бяха двеста хиляди пешаци израилтяни и десет хиляди Юдови мъже.
\par 5 И Саул дойде до един амаликов град и постави засада в долината.
\par 6 А Саул каза на кенейците: Идете, оттеглете се, слезте отсред амаличаните, за да не ви изтребя с тях: защото вие постъпвахте с благост към всичките израилтяни, когато идеха от Египет. И така, кенейците се оттеглиха отсред амаличаните.
\par 7 И Саул порази амаличаните от Евила до прохода на Сур, срещу Египет.
\par 8 И хвана жив амаличкия цар Агар, а изтреби всичките люде с острото на ножа.
\par 9 Саул, обаче, и людете пощадиха Агага, и по-добрите от овците и от говедата и то угоените, и агнетата и всичко, което беше добро, и не искаха да ги обрекат на изтребление; но всичко, което беше изхабено и нямаше стойност, него изтребиха.
\par 10 Тогава Господното слово дойде към Самуила и каза:
\par 11 Разкаях се, гдето поставих Саул цар, понеже той се отвърна от да Ме следва, и не извърши повеленията Ми. А това нещо възмути Самуила, и той викаше към Господа цялата нощ.
\par 12 И на утринта, като стана Самуил рано, за да посрещне Саула, известиха на Самуила казвайки: Саул дойде в Кармил, и, ето, като си издигна паметник, обърна се та замина и слезе в Галгал.
\par 13 И когато Самуил дойде при Саула, Саул му каза: Благословен да си от Господа! изпълних Господната заповед.
\par 14 Но Самуил каза: Що значи тогава това блеене на овци в ушите ми?, и тоя рев на говеда, що слушам?
\par 15 И Саул отговори: От амаличаните ги докараха; защото людете пощадиха по-добрите от овците и от говедата, за да пожертвуват на Господа твоя Бог; а другите обрекохме на изтребление.
\par 16 Тогава Самуил каза на Саула: Почакай, и ще ти известя какво ми говори Господ нощес. А той му рече:Казвай.
\par 17 И рече Самуил: Когато ти беше малък пред собствените си очи, не стана ли глава на Израилевите племена? Господ те помаза цар над Израиля,
\par 18 и Господ те изпрати на път и рече: Иди та изтреби грешните амаличаните и воювай против тях догде се довършат.
\par 19 Защо, прочее, не послуша ти Господния глас, но се нахвърли на користите, та стори това зло пред Господа?
\par 20 А Саул каза на Самуила: Да! послушах Господния глас, и отидох в пътя, който Господ ме изпрати, и доведох амаличкия цар Агаг, а амаличаните обрекох на изтребление.
\par 21 Но людете взеха от користите овци и говеда, по-добрите от обречените неща, за да пожертвуват на Господа твоя Бог в Галгал.
\par 22 И рече Самуил: Всеизгарянията и жертвите угодни ли са тъй Господу, както слушането на Господния глас? Ето, послушанието е по-приемливо от жертвата, и покорността - от тлъстината на овни.
\par 23 Защото непокорността е като греха на чародейството, и упорството като нечестието и идолопоклонството. Понеже ти отхвърли словото на Господа, то и Той отхвърли тебе да не си цар.
\par 24 Тогава Саул каза на Самуила: Съгреших; защото престъпих Господното повеление и твоите думи, понеже се убоях от людете и послушах техния глас.
\par 25 Сега, прочее, прости, моля, съгрешението ми, и върни се с мене, за да се поклоня Господу.
\par 26 А Самуил каза на Саула: Няма да се върна с тебе, защото ти отхвърли Господното слово, и Господ отхвърли тебе, да не бъдеш цар над Израиля.
\par 27 И като се обърна Самуил да си отиде, той хвана полата на мантията му, и тя се раздра.
\par 28 И рече му Самуил: Господ откъсна днес Израилевото царство от тебе и го даде на един твой ближен, който е по-добър от тебе.
\par 29 Па и Силният Израилев няма да излъже, нито да се разкае; защото Той не е човек та да се разкайва.
\par 30 А той каза: Съгреших; но сега стори ми чест, моля пред старейшините на людете ми и пред Израиля, и върни се с мене, за да се поклоня на Господа твоя Бог.
\par 31 И тъй, Самуил се върна и отиде след Саула; и Саул се поклони на Господа.
\par 32 Тогава рече Самуил: Доведете ми тук амаличкия цар Агаг. И Агаг дойде при него весело, защото си казваше: Непременно горчивината на смъртта ще е преминала.
\par 33 А Самуил рече: Както мечът ти е обезчадил жени, така и твоята майка ще се обезчади между жените. И Самуил съсече Агага пред Господа в Галгал.
\par 34 Тогава Самуил си отиде в Рама, а Саул отиде у дома си в Гавая Саулова.
\par 35 И Самуил не видя вече Саула до деня на смъртта си; обаче Самуил плачеше за Саула. И Господ се разкая, за гдето беше поставил Саула цар над Израиля.

\chapter{16}

\par 1 Тогава Господ каза на Самуила: До кога ще плачеш за Саула, тъй като Аз съм го отхвърлил да не царува над Израиля? Напълни рога си с миро та иди; Аз те изпращам при витлеемеца Есей; защото Си промислих цар измежду неговите синове.
\par 2 А Самуил каза: Как да ида? ако чуе Саул, ще ме убие. И Господ каза: Вземи със себе си юница и речи: Дойдох да пожертвувам Господу;
\par 3 и покани Есея на жертвата. Тогава Аз ще ти покажа какво да правиш; и ще Ми помажеш, когото ти посоча по име.
\par 4 И Самуил стори каквото Господ каза, и дойде във Витлеем. А градските старейшини го посрещнаха разтреперани, и рекоха: С мир ли идеш?
\par 5 И той рече: С мир. Ида да пожертвувам Господу; осветете се и дойдете с мене на жертвата. И той освети Есея и синовете му и ги покани на жертвата.
\par 6 И като влизаха и видя Елиава, каза си: Несъмнено пред Господа е помазаникът му.
\par 7 Но Господ каза на Самуила. Не гледай на лицето му, нито на високия му ръст, понеже съм го отхвърлил; защото не е както гледа човек, понеже човек гледа на лице, а Господ гледа на сърце.
\par 8 Тогава Есей повика Авинадава та го направи да мине пред Самуила; а Самуил каза: И тогова не е избрал Господ.
\par 9 Тогава Есей направи да мине Сама; а той каза: Нито тогова е избрал Господ.
\par 10 И Есей направи да минат седемте му сина пред Самуила; но Самуил каза на Есея: Господ не е избрал тия.
\par 11 Тогава Самуил рече на Есея: Присъствуват ли тук всичките ти чада? И той рече: Остава още най-младият; и, ето, пасе овцете. И Самуил каза на Есея: Прати да го доведат; защото няма да седнем около трапезата догде не дойде тук.
\par 12 И прати та го доведоха. И той беше рус, с хубави очи, и красив на глед. И Господ каза: Стани, помажи го, защото той е.
\par 13 Тогава Самуил взе рога с мирото та го помаза всред братята му; и Господният Дух дойде със сила на Давида от същия ден и после. Тогава Самуил стана та си отиде в Рама.
\par 14 А Господният Дух беше се оттеглил от Саула, и зъл дух от Господа го смущаваше.
\par 15 И така, слугите на Саула му рекоха: Ето сега, зъл дух от Бога те смущава;
\par 16 затова нека заповяда господарят ни на слугите си, които са пред тебе, да потърсят човек, който знае да свири на арфа; и когато злият дух от Бога е на тебе, той ще свири с ръката си, и ще ти стане добре.
\par 17 И Саул каза на слугите си: Намерете ми човек, който свири добре, и доведете го при мене.
\par 18 Тогава едно от момчетата проговори казвайки: Ето видях един от синовете на витлеемеца Есей, който знае да свири и е силен и храбър военен мъж, в слово разумен, и красив човек: и Господ е с него.
\par 19 Тогава Саул проводи пратеници до Есея да рекат: Прати ми сина си Давида, който е с овците.
\par 20 И тъй Есей взе осел натоварен с хляб, и мех вино, и едно яре, та го прати на Саула със сина си Давида.
\par 21 И Давид дойде пре Саула, та застана пред него; и той го обикна много; и Давид му стана оръженосец.
\par 22 И Саул прати да кажат на Есея: Нека стои Давид пред мене, моля, защото придоби моето благоволение.
\par 23 И когато злият дух от Бога беше на Саула, Давид вземаше арфата и свиреше с ръката си; тогава Саул се освежаваше и ставаше му добре, и злият дух се оттеглюваше от него.

\chapter{17}

\par 1 След това, като свикаха филистимците войските си за война, събраха се в Сокхот Юдин, и разположиха стан в Ефес-дамим, между Сокхот и Азика.
\par 2 А Саул и Израилевите мъже се събраха та разположиха стан в долината Ила, и се опълчиха за бой против филистимците.
\par 3 Филистимците стояха на хълма от едната страна, а Израил стоеше на хълма от другата страна, и долината беше помежду им.
\par 4 И от филистимския стан излезе юнак на име Голиат, от Гет, шест лакътя и една педя висок.
\par 5 Той имаше меден шлем на главата си, и бе облечен с люспеста броня, тежината на бронята беше пет хиляди сикли мед,
\par 6 и с медни ногавки на пищялите си и медно щитче между рамената си.
\par 7 Дръжката на копието му беше като кросно на тъкач; и острието на копието му тежеше шестотин сикли желязо; и щитоносецът му вървеше пред него.
\par 8 И той застана та извика към Израилевите редове, като им каза: Защо излязохте да се опълчите за бой? Не съм ли аз филистимец, и вие Саулови слуги? Изберете си един мъж, па нека слезе при мене.
\par 9 Ако може да се бие с мене и да ме убие, тогава ние ще ви бъдем слуги; но ако му надвия аз и го убия, тогава вие ще ни бъдете слуги и ще ни се подчинявате.
\par 10 Филистимецът рече още: Аз хвърлям презрение днес върху Израилевите редове; дайте ми мъж да се бием двама.
\par 11 А Саул и целият Израил, когато чуха тия думи на филистимеца, смаяха се и се уплашиха твърде много.
\par 12 А Давид беше син на оня ефратянин от Витлеем Юдов, който се именуваше Есей, и имаше осем сина; и в Сауловите дни тоя човек имаше старейшински чин между хората.
\par 13 И тримата по-големи сина на Есея бяха отишли подир Саула във войната; и имената на тримата му сина, които отидоха на войната, бяха - на първородния Елиав, на другия след него, Авинадав и на третия Сама.
\par 14 Давид беше най-младият; а тримата по-големи следваха Саула.
\par 15 А Давид отиваше от Саула, за да пасе овците на баща си във Витлеем и се връщаше.
\par 16 И филистимецът се приближаваше заран и вечер и се представяше четиридесет дена.
\par 17 В това време Есей каза на сина си Давида: Вземи сега за братята си една ефа пържено жито и тия десет хляба та ги занеси бърже на братята си в стана;
\par 18 а тия десет пити сирене занеси на хилядника им; и виж, здрави ли са братята ти, и вземи ми белег от тях.
\par 19 Саул и те и всичките Израилеви мъже са в долината Ила, дето се бият с филистимците.
\par 20 Прочее, на сутринта Давид стана рано, остави овците на пазач, и взе нещата та отиде както Иесей му бе заповядал; и той отиде при оградата от коли когато войската, при излизането си да се опълчи, извикваше гръмогласно за битката.
\par 21 И Израил и филистимците се опълчиха войска срещу войска.
\par 22 А Давид остави товара си под грижата на товаропазача, и като се завтече към войската, дойде та попита братята си за здравето им.
\par 23 И като се разговаряше с тях, ето юнакът, филистимецът от Гет, на име Голиат, излизаше от филистимските редове та говореше според същите думи; и Давид го чу.
\par 24 А всичките Израилеви мъже, като видяха тоя мъж, побягнаха от него и много се уплашиха.
\par 25 И Израилевите мъже думаха: Видяхте ли тоя мъж, който възлиза? Наистина той възлезе да хвърли презрение върху Израиля; обаче, който го убие, него царят ще обогати с голямо богатство, и ще му даде дъщеря си и ще направи бащиния му дом свободен в Израиля.
\par 26 И Давид проговори на стоящите при него мъже като каза: Какво ще се направи на онзи, който порази тоя филистимец и отмахне укора, от Израиля? защото коя е тоя необрязан филистимец та да хвърли презрение върху войските на живия Бог?
\par 27 И людете му отговориха според казаните думи, като рекоха: така ща се направи на мъжа, който би го поразил.
\par 28 А като чу Елиав, най-големият му брат, как говореше на мъжете, гневът на Елиава пламна против Давида, и рече: Защо си слязъл тук? и кому си оставил онези малко овци в пустинята? Аз зная гордостта ти и лукавщината на сърцето ти; ти си слязъл, за да видиш битката.
\par 29 И Давид каза: Що съм сторил сега? няма ли причина?
\par 30 И обърна се от него към другиго и говори по същия начин; и людете пак му отговориха според първите думи.
\par 31 И когато се чуха думите, които говори Давид, известиха ги на Саула; и той го повика при себе си.
\par 32 И Давид каза на Саула: Да не отпада сърцето на никого поради този. Слугата ти ще иде и ще се бие с тоя филистимец.
\par 33 Но Саул каза на Давида: ти не можеш да идеш против тоя филистимец да се биеш с него; защото ти си дете, а той е войнствен мъж още от младостта си.
\par 34 А Давид рече на Саула: Слугата ти пасеше овците на баща си; и когато дойдеше лъв или мечка та грабнеше агне от стадото,
\par 35 аз го подгонвах та го поразявах, и отървавах грабнатото от устата му; и когато се дигнеше върху мене, хващах го за брадата, поразявах го, и го убивах.
\par 36 Слугата ти е убивал и лъв и мечка, и тоя необрязан филистимец ще бъде като едно от тях, понеже хвърли презрение върху войската на живия Бог.
\par 37 Рече още Давид: Господ, Който ме отърва от лапата на лъв и от лапата на мечка, Той ще ме отърве и от ръката на тоя филистимец. И Саул каза на Давида: Иди: и Господ да бъде с тебе.
\par 38 Тогава Саул облече Давида с облеклото си, и тури меден шлем на главата му, и облече го с броня.
\par 39 И Давид препаса неговия меч над облеклото му и се постара да походи, защото не беше навикнал с тях. И Давид каза на Саула: Не мога да ходя с тия оръжия , защото не съм навикнал. И така Давид ги съблече.
\par 40 И взе тоягата си в ръка, и като си избра пет гладки камъка от потока и ги тури в овчарската си торба, той се приближаваше към филистимеца с прашката си в ръка.
\par 41 И филистимецът напредваше и се приближаваше към Давида; и щитоносецът вървеше пред него.
\par 42 И когато филистимецът погледна наоколо и видя Давида, презря го, защото беше дете и рус и красив на лице.
\par 43 И филистимецът каза на Давида: Куче ли съм аз, та идеш против мене с тояга? И филистимецът прокле Давида с боговете си.
\par 44 Филистимецът още каза на Давида: Дойди при мене, и ще дам месата ти на въздушните птици и на земните зверове.
\par 45 А Давид каза на филистимеца: Ти идеш против мене с меч и копие и сулица; а аз ида против тебе в името на Господа на Силите, Бога на Израилевите войски, върху които ти хвърли презрение.
\par 46 Днес Господ ще те предаде в ръката ми; и като те поразя ще ти отнема главата, и днес ще предам труповете на филистимското множество на въздушните птици и на земните зверове; за да познае целият свят, че има Бог в Израиля,
\par 47 и да познаят всички тук събрани, че Господ не избавя с меч и копие; защото боят е на Господа, и Той ще ви предаде в нашата ръка.
\par 48 И като стана филистимецът та идваше и се приближаваше да посрещне Давида, Давид побърза и се завтече към редовете да посрещне филистимеца.
\par 49 И Давид тури ръката си в торбата си та взе от там камък, и като го хвърли с прашката, удари филистимеца в челото му; и той падна по лицето си на земята.
\par 50 Така Давид надви филистимеца с прашка и с камък, и удари филистимеца и го уби. Но нямаше меч в ръката на Давида;
\par 51 затова Давид се завтече та застана над филистимеца, и хвана меча му и го изтръгна из ножницата му и като го уби отсече главата му с него. А филистимците, като видяха, че юнакът им умря, побягнаха.
\par 52 Тогава Израилевите и Юдовите мъже станаха, извикаха и подгониха филистимците до прохода на Гая и до портите на Акарон. И ранените филистимци паднаха из пътя за Саараим до Гет и до Акарон.
\par 53 А израилтяните, като се върнаха от преследването на филистимците, разграбиха стана им.
\par 54 И Давид взе главата на филистимеца та я занесе в Ерусалим, а оръжията му тури в шатъра си.
\par 55 А Саул, когато видя Давида, че излизаше против филистимеца, каза на военачалника Авенира: Авенире, чий син е тоя момък? А Авенир рече: Заклевам се в живота на душата ти, царю не зная.
\par 56 И царят каза: Попитай чий син е тоя момък.
\par 57 И като се връщаше Давид от поражението на филистимеца, Авенир го взе та го доведе пред Саула; и главата на филистимеца беше в ръката му.
\par 58 И Саул му рече: Чий син си, младежо? А Давид отговори: Аз съм син на слугата ти витлеемеца Есей.

\chapter{18}

\par 1 И като престана Давид да говори със Саула, душата на Ионатана се свърза с душата на Давида, и Ионатан го обикна както Собствената си душа.
\par 2 И в същия ден Саул го взе при себе си, и не го остави да се върне вече в бащиния си дом.
\par 3 Тогава Ионатан направи завет с Давида, защото го обичаше като собствената си душа.
\par 4 Още Ионатан съблече мантията, който беше с него, та я даде на Давида, и дрехите си, и собствения си меч, лъка си и пояса си.
\par 5 И Давид излизаше навсякъде, гдето го пращаше Саул, о обхождаше се разумно; и Саул го постави над военните мъже; и това беше угодно пред очите на Сауловите служители.
\par 6 А като идеха по случай завръщането на Давида от поражението на филистимците, жените излизаха от всичките Израилеви градове та пееха на царя Саула, с тъпанчета, с радост и с кимвали.
\par 7 И жените, като играеха, пееха ответно и думаха: Саул порази хилядите си, А Давид десетките хиляди.
\par 8 А Саул се разсърди много, и тия думи му бяха оскърбителни, и каза: На Давида отдадоха десетки хиляди, а на мене дадоха хиляди; и какво му липсва още освен царството?
\par 9 И от същия ден и нататък Саул гледаше на Давида с лошо око.
\par 10 И на следния ден зъл дух от Бога нападна Саула, и той лудееше всред къщата си; и Давид свиреше с ръката си, както всеки ден, а Саул държеше копието си в ръка.
\par 11 И Саул хвърли копието, като каза: Ще закова Давида дори до стената. Но Давид се отклони от присъствието му два пъти.
\par 12 И Саул се страхуваше от Давида, понеже Господ беше с него, а от Саула беше се оттеглил.
\par 13 Затова Саул го отстраняваше от при себе си, и постави го хилядник; и той излизаше и влизаше пред людете.
\par 14 И Давид се обхождаше разумно във всичките си пътища; и Господ беше с него.
\par 15 Затова Саул, като гледаше, че Давид се обхождаше твърде разумно, страхуваше се от него.
\par 16 А целият Израил и Юда обичаха Давида, понеже излизаше и влизаше пред тях.
\par 17 Тогава Саул каза на Давида, Ето по-голямата ми дъщеря Мерава; нея ще ти дам за жена; само служи ми храбро, и воювай в Господните войни. Защото Саул си каза: Нека моята ръка не се дига върху него, но ръката на филистимците нека се дигне върху него.
\par 18 И Давид каза на Саула: Кой съм аз, и какъв е животът ми, и какво е бащиното ми семейство между Израиля, та да стана аз царски зет?
\par 19 Обаче, във времето, когато Сауловата дъщеря Мерава трябваше да се даде на Давида, тя бе дадена на меолатянина Адриил за жена.
\par 20 А Сауловата дъщеря Михала обичаше Давида, и като известиха това на Саула, стана му угодно.
\par 21 И рече Саул: Ще му я дам, за да му бъде примка, и за да се дигне върху него ръката на филистимците. Затова Саул каза на Давида втори път: Днес ще ми станеш зет.
\par 22 И Саул заповяда на слугите си, казвайки : Говорете тайно на Давида и кажете: Ето, благоволението на царя е към тебе, и всичките му служители те обичат; сега, прочее, стани зет на царя.
\par 23 И тъй, Сауловите служители говориха тия думи в ушите на Давида. Но Давид рече: Лесно нещо ли ви се вижда да стане някой царски зет? А аз съм човек сиромах и нищожен.
\par 24 И служителите на Саула му известиха, казвайки: Така и така казва Давид.
\par 25 А Саул каза: Така да говорите на Давида: Царят не ще вино, но сто филистимски краекожия, за да си отмъсти на царевите неприятели. Но Саул замислюваше да направи Давида да падне чрез ръцете на филистимците.
\par 26 А когато служителите му известиха на Давида тия думи, угодно беше на Давида да стане зет на царя; за туй, и преди да изминат определените за това дни.
\par 27 Давид стана та отиде, той и мъжете му, и уби от филистимците двеста мъже; и Давид донесе краекожията им и даде ги напълно на царя, за да стане царски зет. И Саул му даде дъщеря си Михала за жена.
\par 28 И Саул видя и позна, че Господ беше с Давида; а Сауловата дъщеря Михала го обичаше.
\par 29 И тъй, Саул още повече се страхуваше от Давида; и Саул стана Давидов враг за винаги.
\par 30 И филистимските началници пак излизаха на бой ; но колкото пъти излизаха, Давид успяваше повече от всичките Саулови служители, така щото името му беше на голяма почит.

\chapter{19}

\par 1 Между това Саул каза на сина си Ионатана и на всичките си служители да убият Давида.
\par 2 Но Сауловият син Ионатан се радваше много на Давида, казвайки: Баща ми Саул търси случай да те убие; и тъй, пази се, моля, до утре, и стой на тайно място и крий се;
\par 3 а аз ще изляза и ще застана при баща си на нивата, гдето ще бъдеш ти, и ще говоря с баща си за тебе: и ако видя нещо, ще ти известя.
\par 4 И Ионатан говори добре за Давида на баща си Саула, като му каза: Да не съгреши царят против слугата си, против Давида; понеже не ти е съгрешил, и делата му са били много полезни за тебе;
\par 5 защото изложи живота си на опасност та уби филистимеца, и Господ извърши голямо избавление за целия Израил. Ти видя и се зарадва; защо, прочее, искаш да съгрешиш против невинна кръв, като убиеш Давида без причина?
\par 6 И Саул послуша Ионатановите думи; и Саул се закле в живота на Господа: Давид няма да бъде убит.
\par 7 Тогава Ионатан повика Давида и му извести всичко това. И Ионатан доведе Давида при Саула; и той стоеше пред него както по-напред.
\par 8 И пак избухна война; и Давид излезе та се би с филистимците, и порази ги с голямо клане, и те бяха пред него.
\par 9 А злият дух от Господа дойде върху Саула като седеше в къщата си с копието си в ръка; а Давид свиреше с ръката си.
\par 10 И Саул поиска да закове Давида с копието до стената; но той се отклони от Сауловото присъствие, и Саул удари копието в стената; а Давид побягна та се избави през оная нощ.
\par 11 Тогава Саул изпрати човеци в дома на Давида, за да го дебнат и да го убият на утринта; а Михала, жената на Давида, му извести, казвайки: Ако не избавиш живота си тая нощ, утре ще бъдеш убит.
\par 12 И Михала спусна Давида през прозореца; и той отиде, побягна и се избави.
\par 13 Тогава Михала взе един домашен идол та го положи на леглото; тури под главата му възглавница от козина, и покри го с дреха.
\par 14 И когато Саул прати човеците, за да хванат Давида, тя каза: Болен е.
\par 15 Но Саул пак прати човеците, за да видят Давида, и каза: Донесете ми го на леглото, за да го убия.
\par 16 А когато влязоха човеците, ето домашният идол беше на леглото с възглавница от козина под главата му.
\par 17 И Саул каза на Михала: Ти защо ме излъга така, и пусна врага ми та се отърва? И Михала отговори на Саула: Той ми рече: Пусни ме, защо да те убия?
\par 18 Така Давид побягна и се отърва; и дойде при Самуила в Рама та му извести всичко, що му беше сторил Саул. Тогава той и Самуил отидоха та седяха в Навиот.
\par 19 След това, известиха на Саула, казвайки: Ето, Давид е в Навиот у Рама.
\par 20 И Саул прати човеци да хванат Давида; но като видяха дружината на пророците, че пророкуваха, с Божият Дух дойде на Сауловите пратеници, та пророкуваха и те.
\par 21 И като се извести това на Саула, той прати още човеци; но и те пророкуваха. И пак трети път, Саул прати човеци; но и те пророкуваха.
\par 22 Тогава и сам той отиде в Рама; и като стигна до големия кладенец в Сокхо, попита казвайки: Где са Самуил и Давид? И казаха: Ето, та са в Навиот у Рама.
\par 23 И отиде там към Навиот у Рама; и Божият Дух дойде и на него, та като вървеше по пътя пророкуваше по пътя пророкуваше, докато стигна в Навиот у Рама.
\par 24 Съблече и той дрехите си, и пророкуваше и той пред Самуила, и лежеше гол през целия онзи ден и цялата оная нощ. Затова казват: И Саул ли е между пророците.

\chapter{20}

\par 1 Тогава Давид побягна от Навиот у Рама, та дойде и рече на Ионатана: Що съм сторил? Каква е неправдата ми? и какъв е грехът ми пред баща ти, та иска живота ми?
\par 2 А той каза: Не дай, Боже! ти няма да умреш; ето баща ми не върши нищо, ни голямо ни малко, без да ми яви; и защо би скрил баща ми това нещо от мене? Не е така.
\par 3 Обаче Давид се закле и рече: Баща ти знае добре, че аз съм придобил твоето благоволение, затова си казва: да не знае това Ионатан, за да се не наскърби, но заклевам се в живота на Господа и в живота на душата ти, само една крачка има между мене и смъртта.
\par 4 Тогава Ионатан каза на Давида: Каквото поиска душата ти, аз ще сторя за тебе.
\par 5 И Давид каза на Ионатана: Ето, утре е новолуние, когато съм длъжен да седя с царя на ядене; но позволи ми да ида да се крия на полето до вечерта на третия ден.
\par 6 Ако баща ти забележи, че ме няма, тогава кажи: Давид настоятелно поиска позволение от мене да отиде бърже в града си Витлеем, защото там става годишната жертва за цялото му семейство.
\par 7 Ако рече така: Добре, тогава слугата ти ще има мир; но ако се разгневи много, да знаеш, че злото е решено от него.
\par 8 И тъй, покажи милост към слугата си; защото ти си въвел слугата си в завет Господен със себе си. Но ако има неправда в мене, убий ме сам ти; защото да ме водиш при баща си?
\par 9 И рече Ионатан: Да ти не стане това никога; защото, ако бих узнал наистина, че от баща ми е решено да дойде зло върху тебе, нямаше ли да ти го известя?
\par 10 Тогава Давид рече на Ионатана: Кой ще ми извести, ако ти отговори баща ти сърдито?
\par 11 А Ионатан рече на Давида: Дойди, да излезем на полето. И тъй, двамата излязоха на полето.
\par 12 И Ионатан каза на Давида: Господ Израилевият Бог да бъде свидетел ако, като изпитам баща си, утре около тоя час, или на третия ден, и, ето, има нещо добро за Давида, не изпратя тогава до тебе да ти го известя.
\par 13 Така да направи Господ на Ионатана, да! и повече да притури, ако, в случай, че баща ми е решил да ти стори зло, не ти известя това, и не те отпратя, за да отидеш с мир; и Господ да бъде с тебе, както е бил с баща ми!
\par 14 И не само докато съм жив показвай към мене милост Господна,
\par 15 но и като умра, до века не отсичай милостта си от дома ми, даже и тогава, когато Господ ще е изтребил от лицето на земята всички от Давидовите неприятели.
\par 16 И така, Ионатан направи завет с Давидовия дом; и рече : Господ да изиска това чрез Давидовите неприятели!
\par 17 И Ионатан накара Давида да се закълне още веднъж, поради любовта, която имаше към него; защото го обичаше, както обичаше собствената си душа.
\par 18 Тогава Ионатан му каза: Утре е новолуние; и ще видят, че те няма, защото твоето място ще бъде празно.
\par 19 Като престоиш три дена, слез по-скоро и дойди на мястото, дето беше се скрил в деня, когато разисквахме оная работа, и седни при скалата Езил.
\par 20 И аз ще изстрелям три стрели в страната й, като че стрелям в прицел;
\par 21 и, ето, ще пратя момчето и ще му кажа: Иди намери стрелите. Ако река нарочно на момчето: Ето стрелите са отсам тебе, вземи ги: тогава ти дойди, защото има мир за тебе, и няма никаква беда - заклевам се за това в живота на Господа.
\par 22 Но ако река на момчето така: Ето, стрелите са оттатък тебе, тогава иди в пътя си, защото Господ те е отпратил.
\par 23 А относно работата, която аз и ти сме говорили, ето, Господ да бъде свидетел между мене и тебе до века.
\par 24 И тъй, Давид се скри на полето; и когато дойде новолунието, царят седна на трапезата да яде.
\par 25 И като седна царят на мястото си, както винаги, на едно място при стената, Ионатан стана, и Авенир седна при Саула, а Давидовото място беше празно.
\par 26 Саул, обаче, не каза нищо през оня ден, защото си каза: Ще му се е случило нещо, та не е чист; без друго ще е нечист.
\par 27 А на следния ден, вторият на месеца, Давидовото място пак беше празно; затова Саул рече на сина си Ионатана: Защо Есеевият син не дойде да яде ни вчера, ни днес?
\par 28 И Ионатан отговори на Саула: Давид настоятелно поиска позволение от мене да отиде във Витлеем, като каза:
\par 29 Пусни ме, моля, защото семейството ни има жертва в града, и брат ми ми заръча да ида; сега, прочее, ако съм придобил твоето благоволение, позволи ми, моля, да отида и да видя братята си. Затова не дойде на царската трапеза.
\par 30 Тогава гневът на Саула пламна против Ионатана, и той му каза: Развратени и отстъпни сине, не зная ли, че ти си избрал Есеевия син за срам на тебе и за срам на майчината ти голота?
\par 31 Защото, докато Есеевият син живее на земята, ни ти, ни царството ти, ще се утвърди. Затова прати сега та го доведи при мене, защото непременно ще умре.
\par 32 А Ионатан отговори на баща си Саула като му каза: Защо да се убие? що е сторил?
\par 33 А Саул хвърли копието на него, за да го удари, от което Ионатан разбра, че баща му беше решил да убие Давида.
\par 34 И така Ионатан стана от трапезата разярен от гняв, и не яде никаква храна втория ден на месеца; защото беше наскърбен за Давида, понеже баща му го беше опозорил.
\par 35 И на утринта Ионатан излезе на полето, на мястото, което беше определил с Давида, и водеше със себе си едно малко момче.
\par 36 И рече на момчето си: Тичай, намери стрелите, които ще изстрелям. И като тичаше момчето, той изстреля една стрела по-нататък от него.
\par 37 И когато дойде момчето на мястото, гдето беше стрелата, която Ионатан изстреля, Ионатан викна след момчето казвайки: Не е ли стрелата по-нататък от тебе?
\par 38 И Ионатан извика след момчето: Скоро побързай, не стой. И Ионатановото момче, като събра стрелите, дойде при господаря си.
\par 39 Но момчето не знаеше нищо; само Ионатан и Давид знаеха работата.
\par 40 Тогава Ионатан даде оръжията си на момчето, което му слугуваше , и му каза: Иди занеси ги в града.
\par 41 А щом отиде момчето. Давид стана от едно място към юг, и като падна с лицето си на земята, поклони се три пъти; и целуваха се един друг и плакаха и двамата - а Давид твърде много
\par 42 И Ионатан каза на Давида: Иди с мир, както се заклехме ние двамата в Господното име, като казахме: Господ да бъде свидетел между мене и тебе, и между моето потомство и твоето потомство до века! И Давид стана та си отиде; а Ионатан влезе в града.

\chapter{21}

\par 1 Тогава Давид дойде в Ноб при свещеника Ахимелех; и Ахимелех посрещна Давида с трепет, като му каза: Защо си сам, и няма никой с тебе?
\par 2 И Давид рече на свещеника Ахимелех: Царят ми възложи една работа и каза: Никой да не знае за тая работа, по която те изпращам, нито каквото съм ти заповядал; и определих на момците, еди-кое и еди-кое място.
\par 3 А сега, що имаш на ръка? Дай в ръката ми пет хляба, или каквото се намира.
\par 4 И свещеникът отговори на Давида казвайки: Нямам на ръка ни един обикновен хляб, но има свещени хлябове, - ако момците са се въздържали поне то жени.
\par 5 И Давид отговори на свещеника, като му каза: Наистина жените са били далеч от нас около тия три дена откак тръгнах; и съдовете на момците са чисти; и хлябът е някак си общ, още повече понеже друг хляб се освещава днес в съда.
\par 6 И тъй свещеникът му даде осветените хлябове ; защото нямаше там друг хляб освен присъствените хлябове, които бяха дигнали от пред Господа, за да положат топли хлябове в деня, когато се дигнаха другите.
\par 7 (А същия ден имаше там, задържан пред Господа, един от Сауловите слуги на име Доик, едомец, началник на Сауловите овчари)
\par 8 Давид още каза на Ахимелеха: А нямаш ли тук на ръка някое копие или меч? защото не взех в ръката си ни меча си, ни оръжията си, понеже царевата работа беше спешна.
\par 9 И рече свещеникът: Мечът на филистимеца Голиат, когото ти уби в долината Ила, ето, обвит е в кърпа зад ефода; ако искаш да го вземеш, вземи го, защото тук няма друг освен него. И рече Давид; Няма друг като него, дай ми го.
\par 10 Така в оня ден Давид стана, та побягна от Саула, и отиде при гетския цар Анхус.
\par 11 А слугите на Анхуса му казаха: Тоя Давид не е ли цар на земята? Не е ли той, за когото пееха ответно в хороигранията, като казваха: - Саул порази хилядите си, А Давид десетките си хиляди?
\par 12 И Давид като пазеше тия думи в сърцето си си, много се уплаши от гетския цар Анхус.
\par 13 Затова измени поведението си пред тях, като се престори на луд в ръцете им, и драскаше по вратите на портата, и оставяше лигите си да текат по брадата му.
\par 14 Тогава Анхус каза на слугите си: Ето, виждате, че тоя човек е луд; тогава защо го доведохте при мене?
\par 15 Малко ли са моите луди, та сте довели тогова да прави лудории пред мене? Ще влезе ли той в дома ми?

\chapter{22}

\par 1 Тогава Давид излезе от там, та избяга в пещерата Одолам; и братята му и целият му бащин дом, когато чуха, слязоха там при него.
\par 2 И всички, които бяха в утеснение, и всички длъжници, и всички огорчени се събраха при него, и той им стана началник; и така, с него имаше около четиристотин мъже.
\par 3 И от там Давид отиде в Масфа моавска, та рече на моавския цар: Нека дойдат, моля, баща ми и майка ми при вас, докато узная какво ще стори Бог с мене.
\par 4 И доведе ги пред моавския цар, та живяха с него през цялото време, докато Давид беше в крепостта.
\par 5 А пророк Гад каза на Давида: Не оставай в крепостта; излез, та иди в Юдовата земя, Тогава Давид отиде и влезе в дъбравата Арет.
\par 6 А като чу Саул, че Давид и мъжете, които бяха с него, се явили (в което време Саул седеше в Гавая под дъба в Рама, с копието си в ръка, и всичките му слуги стояха около него),
\par 7 тогава Саул каза на слугите си, които стояха около него: Чуйте сега, вениаминци; Есеевият син ще даде ли на всички ви хилядници и стотници,
\par 8 та вие всички да направите заговор против мене, и да няма кой да ми открие, че син ми е направил завет с Есеевия син, и да няма ни един от вас да го боли сърцето за мене, или да ми открие, че син ми е подигнал слугата ми против мене, да постави засада, както днес?
\par 9 Тогава едомецът Доик, който стоеше при Сауловите слуги, проговори, казвайки: Видях Есеевият син, че дойде в Ноб при Ахимелеха Ахитововия син,
\par 10 който се допита до Господа за него и му даде храна, даде му и меча на филистимеца Голиат.
\par 11 Тогава царят прати да повикат свещеника Ахимелеха, Ахитововия син и целият му бащин дом, свещениците, които бяха в Ноб; и те всички дойдоха при царя.
\par 12 И рече Саул: Слушай сега, сине Ахитовов. А той отговори: Ето ме, господарю мой.
\par 13 И рече му Саул: Защо заговорихте против мене, ти и Есеевият син, та си му дал хляб и меч и си се допитал до Бога за него, за да се подигне против мене и да постави засада, както днес?
\par 14 А Ахимелех в отговор рече на царя: А кой между всичките ти слуги е тъй верен както Давида, който е зет на царя, и има свободен вход при тебе и който е почитан в дома ти?
\par 15 Днес ли съм почнал да се допитвам до Бога за него? Далеч от мене! Нека не приписва царят нищо на слугата си, нито на целия ми бащин дом; защото слугата ти не знае нищо за всичко това, ни малко ни много.
\par 16 Но рече царят: Непременно ще умреш, Ахимелехе, ти и целият ти бащин дом.
\par 17 И царят заповяда на стражата, който стоеше около него: Обърнете се та убийте Господните свещеници, понеже са знаели, че той бягал, а не ми известиха. Но царевите слуги отказаха да дигнат ръцете си, за да нападнат Господните свещеници.
\par 18 Тогава царят каза на Доика: Обърни се ти, та нападни свещениците. И едомецът Доик се обърна, нападна свещениците, и уби същия ден осемдесет и пет мъже, които носеха ленен ефод.
\par 19 И Ноб, града на свещениците порази с острото на ножа; мъже и жени, деца и бозайничета, говеда, осли и овци порази с острието на ножа.
\par 20 Но един от синовете на Ахимелеха Ахитововия син, не име Авиатар, си избави, та побягна при Давида.
\par 21 И Авиатар извести на Давида, че Саул изби Господните свещеници.
\par 22 И Давид каза на Авиатара: Оня ден, когато едомецът Доик беше там, знаех, че непременно щеше да яви работата на Саула. Аз станах причина за смъртта на всички човеци от бащиния ти дом.
\par 23 Остани с мене, но бой се; защото, който иска моя живот, онзи е който иска и твоя; но заедно с мене ти ще бъдеш в безопасност.

\chapter{23}

\par 1 След това известиха на Давида казвайки: Ето, филистимците воюват против Кеила и разграбват гумната.
\par 2 Затова Давид се допита до Господа, като каза: Да ида ли да поразя тия филистимци? И Господ каза на Давида: Иди та порази филистимците и избави Кеила.
\par 3 А мъжете на Давида му рекоха: Ето, нас тук в Юдея ни е страх, а колко повече ако отидем в Кеила против филистимските войски!
\par 4 Затова Давид пак се допита до Господа. И Господ пак се допита до Господа. И Господ му отговори казвайки: Стани слез в Кеила, защото ще предам филистимците в ръката ти.
\par 5 Тогава Давид отиде с мъжете си в Кеила, воюва против филистимците, взе добитъка им, и порази ги с голямо клане. Така Давид избави Кеилските жители.
\par 6 А когато Авиатар Ахимелеховият син прибягна при Давида в Кеила, той слезна с ефод в ръката си.
\par 7 И извести се на Саула, че Давид е дощъл в Кеила. И рече Саул: Бог го предаде в ръката ми, защото, щом е влязъл в град, който има порти и лостове, той е затворен.
\par 8 Затова Саул свика всичките люде на война, да слязат в Кеила за да обсадят Давида и мъжете му.
\par 9 Но Давид като се научи, че Саул кроил зло против него, рече на свещеника Авиатара: Донеси тук ефода.
\par 10 И Давид каза: Господи Боже Израилев, слугата Ти чу положително, че Саул възнамерявал да дойде Кеила да съсипе града поради мене.
\par 11 Ще ме предадат ли Кеилските мъже в ръката му? Ще слезе ли Саул според както чу слугата Ти? Господи Боже Израилев, моля Ти се, яви на слугата Си. И рече Господ: Ще слезе.
\par 12 Тогава каза Давид: Кеилските мъже ще предадат ли мене и мъжете ми в Сауловата ръка? И рече Господ: Ще предадат.
\par 13 Тогава Давид и мъжете му, около шестотин души, се дигнаха та излязоха от Кеила, и отидоха гдето можеха. И извести се на Саула, че Давид се избавил от Кеила; затова той се отказа да излезе.
\par 14 А Давид живееше в пустинята в недостъпни места, и седна в хълмистата част на пустинята Зиф. И Саул го търсеше всеки ден; но Бог не го предаде в ръката му.
\par 15 И тъй Давид, като видя, че Саул бе излязъл да иска живота му, остана в пустинята Зиф, вътре в дъбравата.
\par 16 Тогава Ионатан, Сауловият син, стана та отиде при Давида в дъбравата та укрепи ръката му в Бога.
\par 17 И рече му: Не бой се; защото ръката на баща ми Саула няма да те намери; и ти ще царуваш над Израиля, а аз ще бъда втория след тебе; да! и баща ми Саул знае това.
\par 18 И те двамата направиха завет пред Господа; и Давид остана вътре в дъбравата, а Ионатан отиде у дома си.
\par 19 Тогава зифците дойдоха при Саула в Гавая и рекоха: Не се ли крие Давид у нас в недостъпни места вътре в дъбравата, на хълма Ехела, който е на юг от Есимон?
\par 20 Сега, прочее, царю, слез според всичкото желание на душата си да слезеш; а нашата работа ще бъде да го предадем в ръката на царя.
\par 21 И рече Саул: Да сте благословени от Господа, защото ме пожалихте.
\par 22 Идете, прочее, уверете се още по-точно, научете се, и вижте мястото, гдето е свърталището му, и коя го е видял там; защото ми казаха, че бил много хитър.
\par 23 И тъй, вижте и научете се, в кое от всичките скришни места се крие, а когато се върнете при мене с положително известие, ще отида с вас; и, ако е в тая земя, ще го издиря измежду всичките Юдови хиляди.
\par 24 И те станаха та отидоха в Зиф преди Саула. А Давид и мъжете му бяха в пустинята Маон, на полето на юг от Иесимон.
\par 25 И тъй, Саул и мъжете му отидоха да го търсят; защото беше слязъл от скалата и седеше в пустинята Маон; и Саул, като чу, завтече се след Давида в пустинята Маон.
\par 26 И Саул ходеше от едната страна на хълма, а Давид и мъжете му от другата страна на хълма; и Давид побърза да побегне от Саула, защото Саул и мъжете му обикаляха Давида и мъжете му, за да ги хванат.
\par 27 Но дойде вестител при Саула та рече: Побързай да дойдеш, защото филистимците нападнаха земята.
\par 28 Затова, Саул се върна от преследването на Давида и отиде против филистимците, по която причина нарекоха онова място Селаамалекот.

\chapter{24}

\par 1 След това Давид излезе от там и седна в недостъпните места на Ен-гади; и като се върна Саул от преследването на филистимците, известиха му, казвайки: Ето, Давид е в пустинята Ен-Гади.
\par 2 И тогава Саул взе три хиляди мъже избрани, измежду целия Израил, та отиде да търси Давида и мъжете му по скалите на дивите кози.
\par 3 И дойде при кошарите на овците край пътя, гдето имаше пещера; там Саул влезе по нуждата си, а Давид и мъжете му седяха по-навътре в пещерата.
\par 4 И мъжете на Давида му рекоха: Ето денят, за който Господ ти каза: Ето Аз ще предам неприятеля ти в ръката ти, и ще му сториш както ти се вижда за добро. Тогава Давид стана та отряза скришно полата на Сауловата мантия.
\par 5 А по-после Давид се смути в сърцето си, за гдето отряза Сауловата пола.
\par 6 И рече на мъжете си: Да ми не даде Господ да сторя това на господаря си, Господния помазаник, да дигна ръка против него; защото е Господният помазаник.
\par 7 С тия думи Давид спря мъжете си, и не ги остави да се подигнат против Саула. А Саул стана от пещерата и отиде по пътя си.
\par 8 После стана Давид та излезе из пещерата и извика подир Саула казвайки: Господарю мой, царю! И когато Саул погледна отдире си, Давид се наведе с лице до земята, та се поклони.
\par 9 И Давид каза на Саула: Защо слушаш думите на човеци, които казват: Ето, Давид иска злото ти?
\par 10 Ето, днес очите ти виждат как Господ те предаде в ръката ми тоя ден в пещерата: и едни рекоха да те убия; но аз те пожалих, защото рекох: Не ща да дигна ръка против господаря си, защото е Господният помазаник.
\par 11 Виж още, отче мой, виж и полата на мантията ти в ръката ми; и от това, че отрязах полата на мантията ти, но не те убих, познай и виж, че няма ни злоба, ни престъпление в ръката ми, и че не съм съгрешил против тебе, при все че ти гониш живота ми, за да го отнемеш.
\par 12 Господ нека съди между мене и тебе, и Господ нека ми въздаде за тебе: обаче моята ръка не ще се дигне против тебе.
\par 13 Както казва поговорката на древните: От беззаконните произхожда беззаконие, но моята ръка не ще се дигне против тебе.
\par 14 Подир кого е излязъл Израилевият цар? Кого преследваш ти? Подир умряло куче, подир една бълха.
\par 15 Господ, нека, бъде съдия и нека съди между мене и тебе, нека види, нека се застъпи за делото ми, и нека ме избави от ръката ти.
\par 16 И като изговори Давид тия думи на Саула, рече Саул: Това твоят глас ли е, чадо мое Давиде? И Саул плака с висок глас.
\par 17 И рече на Давида: Ти си по-праведен от мене, защото ти ми въздаде добро, а аз ти въздадох зло.
\par 18 И ти показа днес, че си ми сторил добро, защото, когато Господ бе ме предал в ръцете ти, ти не ме уби.
\par 19 Понеже кой, като намери неприятеля си, би го оставил да си отиде по пътя невредим? Господ, прочее, да ти въздаде добро за това, което ти ми направи днес.
\par 20 И сега, ето, познавам че наистина ти ще станеш цар, и че Израилевото царство ще се утвърди в твоята ръка.
\par 21 Сега, прочее, закълни ми се в Господа, че няма да изтребиш потомството ми след мене, и че няма да погребеш името ми от бащиния ми дом.
\par 22 И Давид се закле на Саула. Тогава Саул си отиде у дома си, а Давид и мъжете му се изкачиха на твърдото място.

\chapter{25}

\par 1 В това време Самуил умря; и целият Израил се събра та го оплакаха, и погребаха го в къщата му в Рама. И Давид стана та слезе в пустинята Фаран.
\par 2 И имаше в Маон един човек, чиято работа беше на Кармил; и тоя човек беше много богат, и имаше три хиляди овци и хиляда кози; и стрижеше овците си на Кармил.
\par 3 Името на човека беше Навал, а името на жена му Авигея; и жената беше благоразумна и красива, а мъжът опак и нечестив в делата си; и той беше от Халевовия род.
\par 4 И като чу Давид в пустинята как Навал стрижел овците си,
\par 5 изпрати десет момъка; и Давид каза на момците: Качете се на Кармил та идете при Навала, и поздравете го от мое име като кажете:
\par 6 Здравей! мир и на тебе, мир и на дома ти, мир и на всичко що имаш!
\par 7 И сега чух, че си имал стригачи; ето, не повредихме овчарите ти, които бяха с нас, нито им се изгуби нещо през цялото време, което бяха на Кармил.
\par 8 Попитай момците си, и ще ти кажат. Прочее, нека придобият моите момци твоето благоволение, защото в добър ден дойдохме; дай, моля, на слугите си и на сина си Давида каквото ти дава ръка.
\par 9 И тъй, Давидовите момци дойдоха та говориха на Навала според всички тия думи от името на Давида, - и млъкнаха.
\par 10 Но Навал отговори на Давидовите слуги, казвайки: Кой е Давид? и кой е Есеевият син? Много са станали днес слугите, които бягат всеки от господаря си.
\par 11 И така, да взема ли хляба си, и водата си, и закланото, което заклах за стригачите си, та да ги дам на човеци, които не знам от къде са?
\par 12 И Давидовите момци се върнаха по пътя си, та си отидоха, и като дойдоха известиха на Давида всичките тия думи.
\par 13 Тогава Давид каза на мъжете си: Препашете всеки меча си; и препасаха всеки меча си; и излязоха подир Давида около четиристотин мъже, а двеста останаха при вещите.
\par 14 А един от момците извести на Наваловата жена Авигея, казвайки: Ето, Давид прати човеци от пустинята за да поздравят нашия господар; а той се спусна върху тях.
\par 15 Но тия мъже бяха много добри към нас; ние не бяхме повредени, нито изгубихме нещо, докато дружехме с тях, когато бяхме в полето;
\par 16 те бяха като стена около нас и денем и нощем през всичкото време, докато бяхме с тях и пасяхме овците.
\par 17 Прочее, знай това и размисли, какво ще направиш; защото зло е решено против господаря ни и против целия му дом; понеже той е толкоз злонрав човек, щото никой не може да му продума.
\par 18 Тогава Авигея побърза та взе двеста хляба, два меха вино, пет сготвени овни, пет мери пържено жито, сто грозда сухо грозде и двеста низаници смокини, и ги натовари на осли.
\par 19 И рече на момчетата си: Вървете пред мене; ето аз ида след вас. Но на мъжа си Навала не каза нищо.
\par 20 И като седеше тя на осела и слизаше под сянката на гората, ето, Давид и мъжете му слизаха към нея; и тя ги срещна.
\par 21 А Давид беше казал: Наистина напразно съм пазил всичко що имаше тоз човек в пустинята, и нищо не се изгуби от всичко що имаше; но пак той ми въздаде зло за добро.
\par 22 Така да направи Бог на Давидовите неприятели, да! и повече да притури, ако до утрешната зора оставя едно мъжко от всичко що е негово.
\par 23 И Авигея, като видя Давида побърза та слезе от осела, и падна пред Давида на лицето си, та се поклони до земята.
\par 24 И, като припадна при нозете му, рече: На мене, господарю мой, на мене нека бъде това нечестие; и нека говори, моля, слугинята ти в ушите ти; и послушай думите на слугинята си.
\par 25 Моля, нека господарят ми не обръща никакво внимание на тоя злонрав човек Навала; защото каквото е името му такъв е и той; Навал е името му, и безумие обитава в него; а пък аз, твоята слугиня, не видях момците на господаря си, които си пратил.
\par 26 Сега, прочее, господарю мой, в името на живия Господ и на живота на душата ти, понеже Господ те е въздържал от кръвопролитие и от самоотмъщаване с ръката ти, то вразите ти, и тия, които искат зло на господаря ми, нека бъдат като Навала.
\par 27 И сега тоя подарък, който твоята слугиня донесе на господаря си, нека се даде на момците, които следват господаря ми.
\par 28 Прости, моля, грешката на слугинята си, защото Господ непременно ще направи за господаря ми твърд дом, понеже господарят ми воюва в Господните войни. И зло не се намери в тебе никога.
\par 29 И при все, че се е дигнал човек да те гони и да иска живота ти, пак животът на господаря ми ще бъде вързан във вързопа на живите при Господа твоя Бог; а животът на неприятелите ти Той ще изхвърли като отсред прашка.
\par 30 И когато Господ постъпи към господаря ми според всичките благости, които е говорил за тебе, и те постави управител за тебе, и те постави управител над Израиля,
\par 31 тогава това не ще ти бъде причина за съжаление, нито причина да се спъва сърцето на господаря ми, гдето си пролял невинна кръв, или гдето господарят ми е отмъстил сам за себе си; но когато Господ направи добро на господаря ми, тогава спомни слугинята си.
\par 32 Тогава Давид каза на Авигея: Благословен да бъде Господ Израилевия Бог, Който те изпрати днес да ме посрещнеш;
\par 33 и благословен съвета ти; и благословена ти, която ме въздържа днес от кръвопролитие и от самоотмъщаване с ръката ми.
\par 34 Защото действително, заклевам се в живота на Господа Израилевия Бог, Който ме въздържа да не ти сторя зло, ако не беше побързала да дойдеш да ме посрещнеш, то до утрешната зора нямаше да остани на Навала нито едно мъжко.
\par 35 И така Давид взе от ръката й онова, що му бе донесла; и рече й: Иди в дома си с мир; виж, послушах думите ти и те приех.
\par 36 А Авигея дойде при Навала, и, ето, той имаше в дома си гощавка, като царска гощавка; и Наваловото сърце беше весело в него, понеже той беше крайно пиян, затова, до утрешната зора тя не му извести нищо, ни малко, ни много.
\par 37 Но на утринта, като изтрезня Навал, жена му му разказа тия работи; и сърцето му премря в него, и той стана като камък.
\par 38 И около десет дена подир това Господ порази Навала, та умря.
\par 39 А когато чу Давид, че умрял Навал, рече: Благословен да бъде Господ, Който отсъди съдбата ми за обидата ми от Навала, и въздържа слугата Си да не стори зло; защото Господ обърна злобата на Навала върху главата му. И Давид прати да говорят на Авигея, за да я вземе за жена.
\par 40 И когато Давидовите слуги дойдоха при Авигея в Кармил, говориха й казвайки: Давид ни прати при тебе да те вземе за жена.
\par 41 И тя стана та се поклони с лицето си до земята и рече: Ето, слугинята ти е служителка да мие нозете на господаревите си слуги.
\par 42 Тогава Авигея побърза та стана и се качи на осел, заедно с пет нейни момичета, които отиваха подир нея; и отиде след пратениците на Давида, и му стана жена.
\par 43 Давид взе още и Ахиноам от Езраел; и тия двете му станаха жени.
\par 44 А Саул беше дал дъщеря си Михала, Давидовата жена, на Фалтия Лаисовия син, който беше от Галим.

\chapter{26}

\par 1 Тогава дойдоха зифците при Саула в Гавая и рекоха: Не се ли крие Давид в хълма Ехела срещу Есимон?
\par 2 И тъй Саул стана та слезе в зифската пустиня, като водеше със себе си три хиляди мъже избрани от Израиля, та търси Давида в зифската пустиня.
\par 3 И Саул разположи стана си на върха Ехела, който е срещу Есимос, край пътя. А Давид се намираше в пустинята; и като се научи, че Саул идвал отдире му в пустинята,
\par 4 Давид прати съгледачи та разбра, че Саул наистина бе дошъл.
\par 5 И Давид стана та дойде не мястото, гдето Саул беше разположил стана си, и Давид съгледа мястото, гдето лежеше Саул и Авенир Нировият син, военачалникът му. А Саул лежеше в оградата от коли, и людете бяха разположени около него.
\par 6 Тогава Давид проговори и рече на хетееца Ахимелех и на Ависея Саруиният син, Иоавовия брат: Кой ще слезе с мене при Саула в стана. И рече Ависей: Аз ще сляза с тебе.
\par 7 И тъй, Давид и Ависей дойдоха през нощта при людете; и, ето, Саул лежеше заспал в оградата от коли и копието му беше забито в земята при главата му; а Авенир и людете лежаха около него.
\par 8 Тогава Ависей каза на Давида: Бог предаде днес врага ти в ръката ти; сега, прочее, нека ги поразя с копието до земята с един удар и няма да повторя.
\par 9 А Давид рече на Ависея: Да го не погубиш; защото кой може да дигне ръка против Господния помазаник и да бъде невинен?
\par 10 Рече още Давид: Бъди уверен, както си в живота на Господа, че Господ ще го порази; или денят му ще дойде и ще умре; или ще влезе в сражение и ще загине.
\par 11 Да ми не даде Господ да дигна ръка против Господния помазаник! Но вземи сега, моля, копието което е при главата му, и стомната с водата, па да си отидем.
\par 12 И тъй, Давид взе копието и стомната с водата, които бяха при Сауловата глава, та си отидоха; никой не видя, никой не усети, и никой не се събуди; защото всичките спяха, понеже дълбок сън от Господа бе паднал на тях.
\par 13 Тогава Давид мина насреща та застана на върха на хълма от далеч, като имаше между тях голямо разстояние,
\par 14 и Давид извика към людете и към Авенира Нировия син, като каза: Не отговаряш ли, Авенире? И Авенир в отговор рече: Кой си ти, що викаш към царя?
\par 15 И Давид каза на Авенира: Не си ли ти доблестен мъж? и кой е подобен на тебе между Израиля? Защо, прочее, не пазиш господаря си царя? понеже един от людете влезе да погуби господаря ти царя.
\par 16 Не е добро това, което стори ти; в името на живия Господ вие заслужавате смърт, понеже не опазихте господаря си, Господния помазаник. И сега, вижте, где е копието на царя и стомната с водата, който беше при главата му.
\par 17 А Саул позна Давидовия глас, и рече: Твоят глас ли е, Чадо мое, Давиде? И Давид рече: Моят глас е, господарю мой, царю.
\par 18 Рече още: Защо преследва така господарят ми слугата си? Какво съм сторил? или какво зло има в ръката ми?
\par 19 И сега, моля, нека чуе господарят ми царят думите на слугата си. Ако Господ те е подигнал против мене, нека приеме жертва; но ако това са хора, човеци, те нека бъдат проклети пред Господа, защото са ме пропъдили, така щото днес нямам участие в даденото от Господа наследство, като ми казват: Иди служи на други богове.
\par 20 Сега, прочее, нека падне кръвта ми на земята далеч от Господното присъствие; защото Израилевият цар е излязъл да търси една бълха, като кога гони някой яребица в горите.
\par 21 Тогава рече Саул: Съгреших; върни се, чадо мое, Давиде; защото няма вече да ти сторя зло, понеже животът ми беше днес скъпоценен пред очите ти; ето, безумие сторих и направих голяма погрешка.
\par 22 А Давид в отговор рече: Ето царското копие; нека дойде един от момците да го вземе.
\par 23 А Господ да въздаде на всеки според правдата му и верността му; защото Господ те предаде днес в ръката ми, но аз отказах да дигна ръката си против Господния помазаник.
\par 24 Ето, прочее, както твоят живот бе днес много ценен пред моите очи, така и моят живот нека бъде много ценен пред очите на Господа, и Той да ме избави от всичките скърби.
\par 25 Тогава Саул каза на Давида: Да си благословен, чадо мое, Давиде! ти непременно ще извършиш велики дела , и без друго ще надделееш. И така, Давид отиде в пътя си; а Саул се върна на мястото си.

\chapter{27}

\par 1 Тогава Давид каза в сърцето си: Най-сетне ще загина един ден от Сауловата ръка: няма по-добро за мене отколкото да избягам по-скоро във филистимската земя; тогава Саул ще се отчая и няма вече да ме търси по всичките предели на Израиля; така ще се отърва от ръката му.
\par 2 Затова, Давид стана с шестте стотин мъже, които бяха с него и отиде при гетския цар Анхуса Маоховия син.
\par 3 И Давид остана да живее при Анхуса в Гет, той и мъжете му, всеки със семейството си, и Давид с двете си жени, езраелката Ахиноам, и кармилката Авигея, Наваловата жена.
\par 4 И когато известиха на Саула, че Давид побягнал в Гет, той не го потърси вече.
\par 5 И Давид каза на Анхуса: Ако съм придобил сега твоето благоволение, нека ми се даде място в някой от полските градове, да живея там: защо да живее слугата ти при тебе в царския град?
\par 6 И в същия ден Анхус му даде Сиклаг; затова, Сиклаг принадлежи на Юдовите царе и до днес.
\par 7 А времето, което Давид прекара в земята на филистимците бе една година и четири месеца.
\par 8 И Давид и мъжете му възлязоха та нападнаха гесурийците, гезерейците и амаличаните (защото те бяха жителите на земята от старо време), до Сур и дори до Египетската земя.
\par 9 Давид опустошаваше земята, и не оставаше жив ни мъж, ни жена; и грабеше овци, говеда, осли, камили и дрехи; и като се връщаше, дохождаше при Анхуса.
\par 10 И Анхус казваше на Давида : Где нападнахте днес? А Давид казваше: Южната част на Юда, и южната земя на ерамеилците и южната земя на кенейците.
\par 11 Давид не оставаше жив ни мъж ни жена, които да бъдат доведени в Гет, понеже си думаше: Да не би да долагат против нас, казвайки: Така направи Давид, и такъв е бил обичаят му през цялото време откак е живял във филистимската земя.
\par 12 И Анхус вярваше на Давида и си казваше: Той е направил себе си съвсем омразен на людете си Израиля; затова ще ми бъде слуга за винаги.

\chapter{28}

\par 1 В тия дни филистимците събраха войските си за война, за да воюват против Израиля. И Анхус каза на Давида: Знай положително, че ще отидеш с мене в множеството, ти и мъжете ти.
\par 2 И Давид рече на Анхуса: Ти наистина ще познаеш какво може слугата ти да извърши. А Анхус каза на Давида: Затова ще те направя пазач на главата ми за всегда.
\par 3 А Самуил беше умрял, и целият Израил беше го оплакал и беше го погребал в града му Рама. А Саул беше отстранил от земята запитвачите на зли духове и врачовете.
\par 4 И тъй, филистимците се събраха и дойдоха та разположиха стана си в Сунам; също и Саул събра целият Израил, и разположиха стана си в Гелвуе.
\par 5 А Саул, когато видя стана на филистимците, уплаши се, и сърцето му се разтрепера твърде много.
\par 6 И Саул се допита до Господа; но Господ не му отговори, нито чрез сънища, нито чрез Урима, нито чрез пророци.
\par 7 Тогава Саул каза на слугите си: Потърсете ми някоя запитвачка на зли духове, за да ида при нея и да направя допитване чрез нея. И слугите му му рекоха: Ето, има в Ендор една запитвачка на зли духове.
\par 8 И така, Саул се предреши като облече други дрехи, и отиде, той и двама мъже с него, та дойдоха при жената през нощта; и Саул й рече: Почародействувай ми, моля, чрез запитване зли духове, и възведи ми, когото ти кажа.
\par 9 А жената му каза: Ето, ти знаеш що направи Саул, как изтреби от земята запитвачите на зли духове и врачовете; тогава защо впримчваш живота ми та да ме умъртвят?
\par 10 Но Саул й се закле в Господа, казвайки: Заклевам се в живота на Господа, няма да ти се случи нищо зло за това.
\par 11 Тогава жената рече: Кого да ти възведа? А той каза: Самуила ми възведи.
\par 12 И когато жената видя Самуила, извика със силен глас; и жената говори на Саула и рече: Защо ме измами? Ти си Саул.
\par 13 И рече и царят: Не бой се; но ти какво видя? И жената каза на Саула: Видях един бог който възлизаше из земята.
\par 14 И рече й: Какъв е изгледът му? А тя рече: Един старец възлиза, и е обвит с мантия. И Саул позна, че това беше Самуил, и наведе се с лицето си до земята та се поклони.
\par 15 И Самуил каза на Саула: Защо ме обезпокои, като ме направи да възляза? И Саул отговори: Намирам се в голямо утеснение, защото филистимците воюват против мене, а Бог се е отдалечил от мене, и не ми отговаря вече нито чрез пророци, нито чрез сънища; затова повиках тебе да ми явиш що да правя.
\par 16 Тогава Самуил рече: А защо се допитваш до мене, като Господ се е отдалечил от тебе и ти е станал неприятел?
\par 17 Ето, Господ си извърши, според както бе говорил чрез мене; защото Господ откъсна царството от твоята ръка и го даде на ближния ти Давида.
\par 18 Понеже ти не се покори на гласа на Господа, като не извърши върху Амалика според големия Му гняв, затова Господ ти направи това нещо днес.
\par 19 При това, Господ ще предаде Израиля с тебе в ръката на филистимците; и утре ти и синовете ти ще бъдете при мене; и Господ ще предаде множеството на израилтяните в ръката на филистимците.
\par 20 Тогава Саул изведнъж падна цял прострян на земята, защото се уплаши много от Самуиловите думи; нямаше сила в него, понеже не бе вкусил храна през целия ден и през цялата нощ.
\par 21 А жената дойде при Саула, и като видя, че беше много разтревожен, каза му: Ето, слугинята ти послуша гласа ти; турих живота си в опасност, и покорих се на думите, които ми говори.
\par 22 Сега, прочее, послушай и ти, моля, гласа на слугинята си, и нека сложа малко хляб пред тебе, та яж, за да имаш сила, когато отидеш на път.
\par 23 Но той отказа, като думаше: Не ще ям. Слугите му, обаче, и жената го принудиха, та послуша думите им; и стана от земята та седна на леглото.
\par 24 А жената имаше в къщата угоено теле, и побърза та го закла; и взе брашно та замеси и изпече от него безквасни пити;
\par 25 и го донесе пред Саула и пред слугите му, та ядоха. Тогава те станаха та си отидоха през същата нощ.

\chapter{29}

\par 1 Филистимците, прочее, събраха всичките си войски в Афек; а израилтяните разположиха стана си при извора, който е в Езраел.
\par 2 И филистимските началници заминаваха със стотините си и с хилядите си войници ; а Давид и мъжете му идеха отдире с Анхуса.
\par 3 Тогава рекоха филистимските началници: Що търсят тука тия евреи? И Анхус каза на филистимските военачалници: Това не е ли Давид, слугата на Израилевия цар Саула, който е бил с мене през тия дни или години? и аз не съм намерил в него никаква грешка откакто е преминал откъм мене до днес.
\par 4 Обаче, филистимските военачалници му се разгневиха, и филистимските военачалници му рекоха: Изпрати този човек и нека се върна на мястото, което си му определил и да не слиза с нас в битката, да не би в боя да ни стане противник: защото как би се примирил този с господаря си, ако не с главите на тия мъже?
\par 5 Той не е ли Давид, за когото пееха ответно в хороигранията, като казваха: - Саул порази хилядите си, А Давид десетките си хиляди?
\par 6 Тогава Анхус повика Давида, та му рече: В името на живия Господ, ти си бил честен, и излизането ти и влизането ти с мене във войската е угодно пред очите ми; защото не съм намерил зло в тебе от деня, когато си дошъл при мене до днес; обаче началниците не са добре разположени към тебе.
\par 7 Сега, прочее, върни се та си иди с мир, да не би да причиниш незадоволство у филистимските началници и господаря си царя?
\par 8 А Давид каза на Анхуса: Но що съм сторил? и що си намерил ти в слугата си откак съм пред тебе до днес, та да не мога да воювам против неприятелите на господаря си царя?
\par 9 И Анхус в отговор каза на Давида: Зная, че си угоден пред очите ми, като Божия ангел; все пак, обаче, филистимските военачалници рекоха: Да не отива той с нас в битката.
\par 10 Сега, прочее, стани утре рано със слугите на господаря си, които дойдоха с тебе; и като станете утре рано, щом се развидели, идете си.
\par 11 И тъй, на утринта Давид и мъжете му станаха рано, за да си идат и да се върнат във филистимската земя. А филистимците отидоха в Езраел.

\chapter{30}

\par 1 И когато на третия ден Давид и мъжете му влязоха в Сиклаг, амаличаните бяха нападнали южната страна и Сиклаг, и бяха опустошили Сиклаг и бяха го изгорили с огън.
\par 2 Бяха запленили и жените, които се намираха в него; не бяха убили никого, ни малък, ни голям, но бяха ги откарали и бяха отишли в пътя си.
\par 3 А когато Давид и мъжете му дойдоха в града, ето, беше изгорен с огън, и жените им, синовете им и дъщерите им пленени.
\par 4 Тогава Давид и людете, които бяха с него, плакаха с висок глас, докато не им остана вече сила да плачат.
\par 5 Също и двете Давидови жени бяха пленени, езраелката Ахиноам и Авигея жената на кармилеца Навал.
\par 6 И Давид се наскърби много, защото людете говореха да го убият с камъни, понеже душата на всички люде беше преогорчена, всички за синовете си и дъщерите си; а Давид се укрепи в Господа своя Бог.
\par 7 Тогава Давид каза на свещеника Авиатара, Ахимелеховия син: Донеси ми тук, моля, ефода. И Авиатар донесе ефода при Давида.
\par 8 И Давид се допита до Господа, казвайки: Да преследвам ли тоя полк? ще ги стигна ли? А Той му отговори: Преследвай, защото без друго ще ги стигнеш, и непременно ще отървеш всичко .
\par 9 Тогава Давид отиде с шестте стотин мъже, които бяха с него, и дойдоха до потока Восор, дето се спряха изостаналите надире;
\par 10 понеже останаха надире двеста души, които бяха до там уморени, щото не можеха да преминат потока Восор. А Давид и четиристотин мъже преследваха;
\par 11 и на полето намериха един египтянин, когото доведоха при Давида; и дадоха му хляб та яде, и напиха го с вода,
\par 12 дадоха му и част от низаница смокини и два грозда сухо грозде; и като яде, духът му са върна в него, защото три дена и три нощи не беше ял хляб, нито беше пил вода.
\par 13 И Давид му каза: Чий си? и от где си? И той рече: Аз съм младеж египтянин, слуга на един амаличанин; и господарят ми ме остави понеже се разболях преди три дена.
\par 14 Ние нападнахме южната земя на херетците, и Юдовите земи, и южната земя на Халева, и изгорихме Сиклаг с огън.
\par 15 Тогава Давид му каза: Завеждаш ли ме при тоя полк? А той рече: Закълни ми се в Бога, че няма да ме убиеш, нито ще ме предадеш в ръката на господаря ми, и ще те заведа при тоя полк.
\par 16 И когато го заведе, ето, те бяха разпръснати по цялата местност та ядяха, пиеха и се веселяха поради всичките големи користи, които бяха взели от филистимската земя и от Юдовата земя.
\par 17 И Давид ги поразяваше от зората дори до вечерта срещу другия ден, тъй че ни един от тях не се избави, освен четиристотин момци, които се качиха на камили и побягнаха.
\par 18 Така Давид отърва всичко що бяха взели амаличаните; също и двете си жени отърва Давид.
\par 19 Не им се изгуби нищо, ни малко ни голямо, ни синове, ни дъщери, ни користи, ни каквото да било нещо, което бяха заграбили от тях; Давид върна всичко.
\par 20 Също Давид взе всичките овци и говеда; и като караха тая чарда пред него, казваха: Тия са Давидовите користи.
\par 21 И когато дойде Давид при двестате мъже, които бяха до там уморени, щото не можеха да следват Давида, и които бяха оставили при потока Восор, те излязоха да посрещнат Давида и да посрещнат людете, които бяха с него; и когато се приближи Давид при людете, поздрави ги.
\par 22 А всичките лоши и развратени мъже от ония, които бяха ходили с Давида, проговориха казвайки: Понеже тия не дойдоха с нас, не ще им дадем от користите, които отървахме, освен на всекиго жена му и чадата му; тях нека вземат и да си отидат.
\par 23 Но Давид каза: Не бива да постъпите така, братя мои, с ония неща, които ни даде Господ, Който не опази и предаде в ръката ни полка що бе дошъл против нас.
\par 24 И кой би послушал тия ваши думи? Но какъвто е делът на участвуващия в боя, такъв ще бъде делът и на седящия пре вещите; по равно ще се делят користите помежду им.
\par 25 И така ставаше от оня ден и нататък; той направи това закон и повеление в Израиля, както е и до днес.
\par 26 А когато Давид дойде в Сиклаг, изпрати от тия користи на Юдовите старейшини, на приятелите си, и рече: Ето ви подарък от користите взети от Господните врагове.
\par 27 Изпрати на ония, които живееха във Ветил, на ония в южния Рамот, на ония в Ятир,
\par 28 на ония в Ароир, на ония в Сифмот, на ония в Естемо,
\par 29 на ония в Рахал, на ония в градовете на ерамеилците, на ония в градовете на кенейците,
\par 30 на ония в Хорма, на ония в Хорасан, на ония в Атах,
\par 31 на ония в Хеврон, и на ония във всичките места, гдето Давид и мъжете му се навъртаха.

\chapter{31}

\par 1 А филистимците воюваха против Израиля; и Израилевите мъже побягнаха от филистимците и паднаха убити в хълма Гелвуе.
\par 2 И филистимците настигнаха Саула и синовете му; и филистимците убиха Сауловите синове Ионатана, Авинадава и Мелхусия.
\par 3 И битката се засилваше против Саула, и стрелците го улучиха; и той бе тежко ранен от стрелците.
\par 4 Тогава Саул каза на оръженосеца си: Изтегли меча си, та ме прободи с него , за да не дойдат тия необрязани та ме прободат и се поругаят с мене. Но оръженосецът му не прие, защото много се боеше. Затова Саул взе меча си та падна върху него.
\par 5 И като видя оръженосецът му, че Саул умря, падна и той на меча си и умря с него.
\par 6 Така умря Саул, тримата му сина, оръженосецът му, и всичките му мъже заедно в същия онзи ден.
\par 7 Тогава Израилевите мъже, които бяха оттатък долината и отвъд Иордан, като видяха, че Израилевите мъже бягаха, и че Саул и синовете му измряха, напуснаха градовете, та побягнаха; а филистимците дойдоха и се заселиха в тях.
\par 8 А на следния ден, когато филистимците дойдоха да съблекат убитите, намериха Саула и тримата му сина паднали на хълма Гелвуе.
\par 9 И отсякоха главата му и снеха оръжията му, и изпратиха човеци навред из филистимската земя, за да разнесат известие в капищата на идолите си и между людете.
\par 10 И оръжията му положиха в капището на астартите, а тялото му обесиха на стената на Ветсан.
\par 11 А като чуха жителите на Явис-галаад това, което филистимците направиха на Саула,
\par 12 всичките храбри мъже станаха и, като ходиха цялата нощ, снеха тялото на Саула и телата на синовете му от стената на Ветсан, и като дойдоха в Явис, там ги изгориха.
\par 13 И взеха костите им та ги закопаха под дървото в Явис, и постиха седем дена.


\end{document}