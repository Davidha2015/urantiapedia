\begin{document}

\title{2 Samuel}


\chapter{1}

\par 1 След смъртта на Саула, като се върна Давид от поражението на амаличаните, и Давид беше престоял в Сиклаг два дена,
\par 2 на третия ден, ето, дойде един човек из стана, от Саула, с раздрани дрехи и пръст на главата си; и като влезе при Давида, падна на земята та се поклони.
\par 3 И Давид рече: От где идеш? А той му каза: От Израилевия стан се отървах.
\par 4 И рече Му Давид: Какво стана? Кажи ми, моля. А той отговори: Людете побягнаха от сражението, даже и мнозина от людете паднаха мъртви; умряха още и Саул и син му Ионатан.
\par 5 Тогава Давид рече на момъка, който му съобщаваше това : От къде знаеш, че Саул и син му Ионатан са мъртви?
\par 6 И момъкът, който му съобщаваше това рече: Намерих се случайно в хълма Гелвуе, и, ето, Саул беше се подпрял на копието си, и, ето, колесниците и конниците го застигнаха.
\par 7 И като погледна отдире си, той ме видя, и повика ме. И отговорих: Ето ме.
\par 8 И рече ми: Кой си ти? И отговорих му, амаличанин съм.
\par 9 Пак ми рече: Застани, моля, над мене та ме убий, защото помрачение ме обзе; понеже целият ми живот е още в мене.
\par 10 И тъй, застанах над него и го убих, понеже бях уверен, че не можеше да живее, като беше вече паднал; и взех короната, която беше на главата му, и гривната, която беше на ръката му, та ги донесох тук при господаря си.
\par 11 Тогава Давид хвана дрехите си, та ги раздра: така направиха и всичките мъже, които бяха с него
\par 12 И жалиха и плакаха и постиха до вечерта за Саула и за сина му Ионатана, и за Господните люде и за Израилевия дом, за гдето бяха паднали от меч.
\par 13 А Давид рече на момъка, който му съобщаваше това : От где си ти? И отговори: Аз съм син на един чужденец, амаличанин.
\par 14 И рече му Давид: Ти как се не убоя да дигнеш ръка да убиеш Господния помазаник?
\par 15 И Давид повика един от момците и рече: Пристъпи, нападни го. И той го порази та умря.
\par 16 И Давид му рече: Кръвта ти да бъде на главата ти; защото устата ти свидетелствуваха против тебе, като рече: Аз убих Господния помазаник
\par 17 Тогава Давид оплака Саула и сина му Ионатана с тоя плач;
\par 18 (и заръча да научат юдейците тая Песен на лъка; ето, написана е в Книгата на Праведния): -
\par 19 Славата ти, о Израилю, е застреляна на опасните места на полето ти! Как паднаха Силните.
\par 20 Не възвестявайте това в Гет, Не го прогласявайте по улиците на Аскалон, За да не се зарадват филистимските дъщери, За да не се развеселят дъщерите на необрязаните.
\par 21 Хълми гелвуиски, да няма роса нито дъжд на вас, Нито ниви доставяйки първи плодове за жертви; Защото там бе захвърлен щитът на силните, Щитът на Саула, като да не е бил той помазан с миро.
\par 22 От кръвта на убитите, От лоя на силните, Лъкът Ионатанов не се връщаше назад, И мечът Саулов не се връщаше празен.
\par 23 Саул и Ионатан Бяха любезни и рачителни в живота си, И в смъртта си не се разделиха; По-леки бяха от орлите, По-силни от лъвовете.
\par 24 Израилеви дъщери, плачете за Саула, Който ви обличаше в червено с украшения, Който туряше златни украшения по дрехите ви.
\par 25 Как паднаха силните всред боя! Ионатан, поразен на опасните места на полето ти
\par 26 Прескръбен съм за тебе Ионатане, брате мой! Рачителен ми беше ти; Твоята любов към мене беше чудесна, Превъзхождаше любовта на жените.
\par 27 Как паднаха силните, И погинаха бойните оръжия!

\chapter{2}

\par 1 След това Давид се допита до Господа като каза: Да възляза ли в някой от Юдовите градове? И Господ му каза: Възлез. Пак рече Давид: Где да възляза? А Той му каза: в Хеврон.
\par 2 И тъй, Давид възлезе там с двете си жени, езраелката Ахиноам и Авигея жената на кармилеца Навал.
\par 3 Давид заведе и мъжете, които бяха с него, всеки със семейството му; и заселиха се в хевронските градове.
\par 4 Тогава Юдовите мъже дойдоха и помазаха там Давида за цар над Юдовия дом. И Известиха на Давида, казвайки: явис-галаадските мъже бяха, които погребаха Саула.
\par 5 Затова Давид прати човеци до явис-галаадските мъже да им кажат: Благословени да сте от Господа, за гдето показахте тая благост към господаря си, към Саула, и го погребахте.
\par 6 Сега Господ нека покаже и към вас милост и вярност; така и аз ще ви въздам за тая добрина, понеже сторихте това нещо.
\par 7 Сега, прочее, нека се укрепят ръцете ви, и бъдете мъжествени; защото господарят ви Саул умря, а Юдовият дом си помаза мене за цар.
\par 8 Обаче, Авенир Нировият син, Сауловият военачалник, взе Исвостея, Сауловия син, та го заведе в Маханаим,
\par 9 и направи го цар над Галаад, над асурците, над езраелците, над Ефрема, над Вениамина и над целия Израил;
\par 10 (Исвостей Сауловият син беше на четиридесет години, когато стана цар над Израиля и царува две години); но Юдовият дом последва Давида.
\par 11 А времето, през което Давид царува в Хеврон над Юдовия дом, беше седем години и шест месеца.
\par 12 И тъй, Авенир Нировият син, и слугите на Исвостея Сауловия син, отидоха от Маханаим в Гаваон.
\par 13 Също и Иоав Саруиният син и Давидовите слуги излязоха та се срещнаха близо при гаваонския водоем; и едните седнаха отсам водоема, а другите оттатък водоема.
\par 14 Тогава Авенир каза на Иоава: Нека станат сега момците да поиграят пред нас. И рече Иоав: Нека станат.
\par 15 И тъй, те станаха та преминаха на брой дванадесет души от Вениамина, от страната на Исвостея Сауловия син и дванадесет от Давидовите слуги.
\par 16 И хванаха всеки съперника си за главата, та заби ножа си в ребрата на съперника си, и паднаха заедно; за туй, онова място се нарече Хелкат-асурим, което е в Гаваон.
\par 17 И в оня ден сражението стана много ожесточено; и Авенир и Израилевите мъже бяха победени от Давидовите слуги.
\par 18 И там бяха тримата Саруини синове: Иоав, Ависея и Асаил; а Асаил бе лек в нозете като полска сърна.
\par 19 И Асаил се спусна подир Авенира; и като тичаше не се отби ни надясно, ни наляво от преследването на Авенира.
\par 20 А Авенир погледна надире и рече: Ти ли си, Асаиле? А той отговори: Аз.
\par 21 Тогава Авенир му каза: Отбий се надясно или наляво си та хвани някого от момците и вземи оръжието му. Но Асаил отказа да се отбие от преследването му.
\par 22 И Авенир пак каза на Асаила: Отбий се от да ме преследваш; защо да те поваля до земята? тогава как ще погледна брата ти Иоава в лицето?
\par 23 Но той отказа да се отбие; затова Авенир го порази в корема със задния край на копието си, така че копието излезе отзад; и той падна там и умря на самото място. И всички които дохождаха на мястото, гдето Асаил падна и умря, спираха се.
\par 24 А Иоав и Ависей преследваха Авенира; и слънцето захождаше когато стигнаха до хълма Амма, който е срещу Гия, край пътя за гаваонската пустиня.
\par 25 И вениаминците като се събраха около Авенира, съставиха една дружина и застанаха на върха на един хълм.
\par 26 Тогава Авенир извика към Иоава и рече: Непременно ли ще пояжда мечът? не знаеш ли, че сетнината ще бъде горчива? Тогава, до кога не щеш да заповядаш на людете да се върнат от преследването на братята си?
\par 27 И рече Иоав: заклевам си в живота на Бога, че ако не беше говорил ти да поиграят момците , тогава още в зори людете непременно щяха да се оттеглят всеки от преследването на брата си.
\par 28 И тъй, Иоав засвири с тръбата, та всичките люде се спряха, и не преследваха вече Израиля, нито се биеха вече.
\par 29 Тогава Авенир и мъжете му вървяха през цялата оная нощ им полето, преминаха Иордан та пропътуваха целия Витрон, и дойдоха в Маханаим.
\par 30 А Иоав се върна от преследването на Авенира; и като събра всичките люде, от Давидовите слуги липсваха деветнадесет мъже и Асаил.
\par 31 Но Давидовите слуги бяха поразили от вениаминците и от Авенировите мъже триста и шестдесет мъже, които умряха.
\par 32 И дигнаха Асаила та го погребаха в бащиния му гроб, който бе във Витлеем. А Иоав и мъжете му, като вървяха цяла нощ, стигнаха в Хеврон около зазоряване.

\chapter{3}

\par 1 А войната между Сауловия дом и Давидовия дом трая дълго време; и Давид непрестанно се засилваше, а Сауловият дом непрестанно ослабваше.
\par 2 И народиха се синове на Давида в Хеврон; първородният му беше Амнон, от езраелката Ахиноам;
\par 3 вторият му, Хилеав, от Авигея, жената на кармилеца Навал; третият, Авесалом, син на Мааха, дъщеря на гесурския цар Талмай;
\par 4 четвъртият, Адония, син на Агита; петият Сефатия, син на Авитала;
\par 5 и шестият, Итраам, от Давидовата жена Егла. Тия се родиха на Давида в Хеврон.
\par 6 А докато продължаваше войната между Сауловия дом и Давидовия дом Авенир беше подпорка на Сауловия дом.
\par 7 А Саул имаше наложница на име Ресфа, дъщеря на Аия; и Исвостей каза на Авенира: Ти защо си влязъл при бащината ми наложница?
\par 8 Тогава Авенир много се разяри за думите на Исвостея, и рече: Кучешка глава ли съм аз откъм Юдовата страна? Днес показвам благост към дома на баща ти Саула, към братята му и към приятелите му, като не те предадох в Давидовата ръка, и пак ме обвиняваш днес за тая жена!
\par 9 Така да направи Бог на Авенира, да! и повече да му притури, ако не сторя за Давида тъй, както Господ му се е клел,
\par 10 като направя да премине царството от Сауловия дом, и като поставя Давидовият престол над Израиля, както е и над Юда, от Дан до Вирсавее.
\par 11 И Исвостей не можеше вече да отговори ни дума на Авенира, понеже се боеше от него.
\par 12 Тогава Авенир изпрати човеци до Давида да кажат от негова страна: Чия е земята? Думаше още: Направи договор с мене; и, ето, моята ръка ще бъде с тебе, щото да доведе под твоята власт целия Израил.
\par 13 А той каза: Добре, аз ще направя договор с тебе; но едно нещо искам аз от тебе, а именно, че няма да видиш лицето ми, ако не доведеш предварително Сауловата дъщеря Михала, когато дойдеш да видиш лицето ми.
\par 14 И Давид прати човеци до Сауловия син Исвостея да кажат: Предай жена ми на Михала, която съм взел за жена срещу сто филистимски краекожия
\par 15 Исвостей, прочее, прати та я взе от мъжа й Фалатиила Лаисовия син.
\par 16 И мъжът й дойде с нея, и като вървеше плачеше, и я последва до Ваурим. Тогава Авенир му каза: Иди, върни се; и той се върна.
\par 17 А Авенир влезе във връзка с Израилевите старейшини и рече: В миналото време вие сте искали Давида да царува над вас;
\par 18 сега, прочее, сторете това, защото Господ е говорил за Давида, казвайки: Чрез ръката на слугата Си Давида ще избавя людете Си от ръката на филистимците и от ръката на всичките им неприятели.
\par 19 Авенир говори още и в ушите на вениаминците; също Авенир отиде да говори и в ушите на Давида в Хеврон всичко, що бе угодно на Израиля и на целия Вениаминов дом.
\par 20 И тъй, Авенир дойде при Давида в Хеврон, и с него двадесет мъже. И Давид направи угощение на Авенира и на мъжете, които бяха с него.
\par 21 Тогава Авенир каза на Давида: Ще стана да ида, и ще събера целия Израил при господаря си царя, за да направят завет с тебе, та да царуваш над всички, според желанието на душата си. И Давид изпрати Авенира, и той отиде с мир.
\par 22 И, ето, Давидовите слуги и Иоав идеха от едно нашествие и носеха със себе си много користи; но Авенир не беше с Давида в Хеврон, защото той го беше изпратил, и Авенир беше отишъл с мир.
\par 23 А като дойде Иоав и цялата войска, който беше с него, известиха на Иоава казвайки: Авенир Нировият син дойде при царя; и той го изпрати, та си отиде с мир.
\par 24 Тогава Иоав влезе при царя и рече: Що си сторил? Ето Авенир е дохождал при тебе; защо си го изпратил та си е отишъл?
\par 25 Знаеш какъв е Авенир Нировият син; той е дохождал за да те измами, и да научи излизането ти и влизането ти, и да научи всичко, що правиш.
\par 26 И Иоав, като излезе от при Давида, прати човеци подир Авенира, които го върнаха от кладенеца Сира; Давид, обаче, не знаеше това.
\par 27 И когато се върна Авенир в Хеврон, Иоав го отведе на страна в портата, за да му говори уж тайно; и там го удари в корема за кръвта на брата си Асаила; и той умря.
\par 28 А по-после, като чу Давид, рече: Невинен съм аз и царството ми пред Господа до века за кръвта на Авенира Нировия син;
\par 29 нека остане тя на Иоавовата глава и на целия му бащин дом; и нека не липсва от Иоавовия дом такъв, който има семетечение, или е прокажен, или който се подпира на тояга, или пада от меч, или е лишен от хляб.
\par 30 Така Иоав и брат му Ависей убиха Авенира, защото беше убит брат им Асаила в сражението при Гаваон.
\par 31 Тогава Давид каза на Иоава и на всичките люде, които бяха с него: Раздерете дрехите си, и препашете се с вретище, та плачете пред Авенира. И цар Давид вървеше след носилото.
\par 32 И погребаха Авенира в Хеврон; и царят плака с висок глас над Авенировия гроб; също и всичките люде плакаха.
\par 33 И царят плака над Авенира, и каза: - Трябваше ли Авенир да умре, както умира безумен?
\par 34 Ръцете ти не бяха вързани, Нито нозете ти турнати в окови; Както пада човек пред тези, които вършат неправда, Така падна ти. И всичките люде плакаха над него.
\par 35 Сетне дойдоха всичките люде, за да предумат Давида да яде хляб, докато беше още видело; но Давид се закле, казвайки: Така да ми направи Бог, да! и повече да притури, ако вкуся хляб, или какво да било нещо, преди да зайде слънцето.
\par 36 И всичките люде се научиха за това, и им стана угодно, както беше угодно на всичките люде и все що правеше царят.
\par 37 Защото в оня ден, всичките люде и целият Израил познаха, че не беше от царя да бъде убит Авенир Нировият син.
\par 38 И царят каза на слугите си: Не знаете ли, че велик военачалник падна днес в Израил?
\par 39 Днес аз съм слаб, ако и да съм помазан за цар; а тия мъже, Саруините синове, са много жестоки за мене. Господ да въздаде на злодееца според злодеянието му.

\chapter{4}

\par 1 А когато чу Исвостей Сауловият син, че Авенир умрял всичките израилтяни се смутиха.
\par 2 А тоя Саулов син имаше двама мъже пълководци, от които името на единия беше Ваана, а името на другия Рихав, синове на виротянина Римон от вениаминците; (защото и Вирот се числеше към Вениамина;
\par 3 а виротяните бяха побягнали в Гетаим, дето са били пришелци до днес).
\par 4 А Сауловият син Ионатан имаше син повреден в нозете. Беше на пет годишна възраст, когато дойде известие от Езраел за Саула и Ионатана, и гледачката му беше го дигнала и побягнала; и като бързаше да бяга, той паднал и окуцял. Името му беше Мемфивостей.
\par 5 И Рихав и Ваана, синовете на виротянина Римон, отидоха и в горещината на деня влязоха в къщата на Исвостея, който лежеше на легло по пладне;
\par 6 и влязоха там до сред къщата уж да вземат жито и удариха го в корема; а Рихав и брат му Ваана избягаха.
\par 7 Защото, когато влязоха в къщата, и той лежеше на леглото в спалнята си, удариха го, убиха го и му отсякоха главата. И като взеха главата му, вървяха през полето цялата нощ,
\par 8 и донесоха главата на Исвостея при Давида в Хеврон, и рекоха на царя: Ето главата на Исвостея, син на врага ти Саула, който търсеше живота ти; днес Господ въздаде на Саула и на потомците му за господаря ни царя.
\par 9 А Давид отговори на Рихава и на брата му Ваана, синовете на виротянина Римон, като им рече: Заклевам се в живота на Господа, който избави душата ми от всяко бедствие,
\par 10 когато един ми извести, казвайки: Ето, Саул умря, като мислеше, че носи добро известие, хванах го та го убих в Сиклаг, - което беше наградата, що му дадох за известието му, -
\par 11 а колко повече, когато нечестиви мъже са убили праведен човек в собствената му къща, на леглото му, не ще ли да изискам сега кръвта му от ръката ви и ви изтребя то земята!
\par 12 Тогава Давид заповяда на момците си; и те ги убиха, и като отсякоха ръцете им и нозете им, обесиха ги при водоема в Хеврон. А главата на Исвостея взеха та я заровиха в Авенировия гроб в Хеврон.

\chapter{5}

\par 1 Тогава всичките Израилеви племена дойдоха при Давида в Хеврон и говориха, казвайки: Ето, твоя кост и твоя плът сме ние.
\par 2 И по-напред, още докато Саул царуваше над нас, ти беше, който извеждаше и въвеждаше Израиля; а на тебе рече Господ: Ти ще пасеш людете Ми Израиля, и ти ще бъдеш вожд над Израиля.
\par 3 И така, всичките Израилеви старейшини дойдоха при царя в Хеврон, и цар Давид направи завет с тях пред Господа в Хеврон; и те помазаха Давида цар над Израиля.
\par 4 Давид беше на тридесет години, когато се възцари, и царуваше четиридесет години.
\par 5 В Хеврон царува над Юда седем години и шест месеца; а в Ерусалим царува над целия Израил и Юда тридесет и три години.
\par 6 След това царят отиде с мъжете си в Ерусалим против евусците, жителите на земята; а те говориха на Давида казвайки: Няма да влезеш тук; но и слепите и куците ще те отблъснат; защото си думаха: Давид не ще може да влезе тука.
\par 7 Обаче Давид превзе крепостта Сион; това е Давидовият град.
\par 8 И в оня ден Давид каза: Който удари евусците, нека хвърли във вадата слепите и куците, които са омразни на Давидовата душа; понеже и слепите и куците бяха казали, че той няма да влезе в жилището им.
\par 9 И Давид се засели в крепостта, която и нарече Давидов град. И Давид построи здания около Мило и навътре.
\par 10 Така Давид преуспяваше и ставаше по-велик; и Господ Бог на Силите беше с него.
\par 11 В това време тирският цар Хирам прати посланици при Давида, и кедрови дървета, дърводелци, и зидари, та построиха къща за Давида.
\par 12 И Давид позна, че Господ го бе утвърдил за цар над Израиля, и че беше възвисил царството му заради людете Си Израиля.
\par 13 И Давид си взе още наложници и жени от Ерусалим, след като дойде от Хеврон; и родиха се на Давида още синове и дъщери.
\par 14 Ето имената на ония, които му се родиха в Ерусалим: Самуа, Совав, Натан, Соломон,
\par 15 Евар, Елисуа, Нефег, Яфия,
\par 16 Елисама, Елиада и Елифалет.
\par 17 А когато филистимците чуха, че помазали Давида за цар над Израиля, всичките филистимци дойдоха да търсят Давида; а Давид като чу за това, слезе в крепостта.
\par 18 И тъй, филистимците дойдоха та се разпростряха в долината Рафаил.
\par 19 Тогава Давид се допита до Господа, казвайки: Да възляза ли против филистимците? ще ги предадеш ли в ръката ми? И Господ каза на Давида: Възлез, защото непременно ще предам филистимците в ръката ти.
\par 20 И тъй, Давид дойде във Ваал-ферасим; и Давид ги порази там, като рече: Господ избухна пред мене върху неприятелите ми както избухват водите. За туй, онова място се нарече Ваал-ферасим.
\par 21 Там филистимците оставиха идолите си; а Давид и мъжете му ги дигнаха.
\par 22 И филистимците пак дойдоха та се разпростряха по долината Рафаим.
\par 23 А когато Давид се допита до Господа, Той каза: Не възлизай; обиколи изотзаде им та ги нападни срещу черниците.
\par 24 И когато чуеш шум, като от маршируване по върховете на черниците, тогава да се подвижиш, защото тогава Господ ще излезе пред тебе да порази филистимското множество.
\par 25 И Давид стори както му заповяда Господ, и порази филистимците от Гава до входа на Гезер.

\chapter{6}

\par 1 След това Давид пак събра всичките отборни мъже от Израиля, на брой тридесет хиляди души.
\par 2 И Давид стана от Ваала Юдова та отиде, и всичките люде, които бяха с него, за да пренесе от там, дето се намираше , Божия ковчег, който се нарича с името на Господа на Силите, Който обитава между херувимите.
\par 3 И положиха Божия ковчег на нова кола, и дигнаха го от Авинадавовата къща, която бе на хълма; а Оза и Ахио, Авинадавовите синове, караха новата кола с Божия ковчег,
\par 4 като Ахия вървеше пред ковчега, а Оза край него .
\par 5 А Давид и целият Израилев дом свиреха пред Господа с всякакви видове инструменти от елхово дърво, с арфи, с псалтири, с тъпанчета, с цитри и с кимвали.
\par 6 А когато стигнаха до Нахоновото гумно, Оза простря ръката си към Божия ковчег та го хвана; защото воловете го раздрусаха.
\par 7 И Господният гняв пламна против Оза, и Бог го порази там за грешката му; и той умря там при Божия ковчег.
\par 8 И Давид се наскърби за гдето Господ нанесе поражение на Оза; и нарече онова място Фарез-Оза, както се нарича и до днес.
\par 9 И в оня ден Давид се уплаши от Господа, и рече: Как ще дойде Господният ковчег при мене?
\par 10 Затова Давид отказа да премести Господния ковчег при себе си в Давидовия град; но Давид го отстрани в къщата на гетеца Овид-едом.
\par 11 И Господният ковчег престоя в къщата на гетеца Овид-едома три месеца; и Господ благослови Овид-едома и целия му дом.
\par 12 Известиха, прочее, на цар Давида, казвайки: Господ е благословил дома на Овид-едома и целия му имот заради Божия ковчег. Тогава Давид отиде та пренесе с веселия Божия ковчег от къщата на Овид-едома в Давидовия град.
\par 13 И когато тия, които носеха Господния ковчег, преминаха шест крачки, той пожертвува говеда и угоени телци .
\par 14 И Давид играеше пред Господа с всичката си сила; и Давид беше препасан с ленен ефод.
\par 15 Така Давид и целият Израилев дом пренесоха Господния ковчег с възклицание и с тръбен звук.
\par 16 А като влизаше Господният ковчег в Давидовия град, Михала, Сауловата дъщеря, погледна от прозореца на, и като видя, че цар Давид скачаше и играеше пред Господа, презря го в сърцето си.
\par 17 И донесоха Господния ковчег та го положиха на мястото му, всред шатъра, който Давид принесе всеизгаряния и примирителни приноси пред Господа.
\par 18 И когато Давид свърши принасянето на всеизгарянията и примирителните приноси, благослови людете в името на Господа на Силите.
\par 19 И раздаде на всичките люде, сиреч, на цялото множество израилтяни, мъже и жени, на всеки човек, по един хляб и по една мера вино и по една низаница сухо грозде. Тогава всичките люде си отидоха, всеки в къщата си.
\par 20 А когато Давид се връщаше, за да благослови дома си, Михала, Сауловата дъщеря, излезе да посрещне Давида, и рече: Колко славен беше днес Израилевият цар, който се съблече днес пред очите на слугините на служителите си, както се съблича безсрамно един никакъв човек!
\par 21 А Давид каза на Михала: пред Господа, Който предпочете мене пред баща ти и пред целия негов дом, за да ме постави вожд над Господните люде, над Израиля, да! пред Господа играх.
\par 22 И ще се унижа още повече, и ще се смиря пред собствените си очи; а то слугините, за които ти говори, от тях ще бъда почитан.
\par 23 Затова Михала Сауловата дъщеря остана бездетна до деня на смъртта си.

\chapter{7}

\par 1 И като се настани царят в къщата си, и Господ беше го успокоил от всичките неприятели около него,
\par 2 царят каза на пророк Натана: Виж сега, аз живея в кедрова къща, а Божият ковчег стои под завеси.
\par 3 И Натан каза на царя: Иди, стори всичко, което е в сърцето ти; защото Господ е с тебе.
\par 4 Но през същата нощ Господното слово дойде на Натана и рече:
\par 5 Иди кажи на слугата Ми Давида: Така говори Господ: Ти ли ще Ми построиш дом, в който да обитавам?
\par 6 Защото от деня, когато изведох израилтяните от Египет дори до днес, не съм обитавал в дом, но съм ходил в шатър и в скиния.
\par 7 Където и да съм ходил с всичките израилтяни, говорих ли някога на някое от Израилевите племена, на което заповядах да пасе людете Ми Израиля, да кажа: Защо не Ми построихте кедров дом?
\par 8 Сега, прочее, така да кажеш на слугата Ми Давида: Така казва Господ на Силите: Аз те взех от кошарата, от подир стадото, за да бъдеш вожд на людете Ми, на Израиля;
\par 9 и съм бил с тебе навсякъде, гдето си ходил и изтребил всичките ти неприятели пред тебе, и направих името ти велико, както името на великите, които са на земята.
\par 10 И ще определя място за людете Си Израиля и ще ги насадя, та ще обитават на свое собствено място, и няма да се преместват вече; и тези които, вършат нечестие няма вече да ги притесняват, както по-напред,
\par 11 и както от деня, когато поставих съдии над людете Си Израиля; и ще те успокоя от всичките ти неприятели. При това, Господ ти явява, че Господ ще ти съгради дом.
\par 12 Когато се навършат дните ти и заспиш с бащите си, ще въздигна потомеца ти подир тебе, който ще излезе из чреслата ти, и ще утвърдя царството му.
\par 13 Той ще построи дом за името Ми; и Аз ще утвърдя престола на царството му до века.
\par 14 Аз ще му бъда Отец, и той ще Ми бъде син: ако стори беззаконие, ще го накажа с тояга каквато е за мъже и с биения каквито са за човешкия род;
\par 15 но Моята милост няма да го остави, както я отнех от Саула, когото отмахнах от пред тебе.
\par 16 И домът ти и царството ти ще се утвърдят пред тебе до века.
\par 17 И Натан говори на Давида точно според тия думи и напълно според това видение.
\par 18 Тогава цар Давид влезе та седна пред Господа и рече: Кой съм аз, Господи Иеова, и какъв е моят дом, та си ме довел до това положение ?
\par 19 Но даже и това бе малко пред очите Ти, Господи Иеова; а Ти си говорил още за едно дълго бъдеще за дома на слугата Си, и даваш това като закон на човеците, Господи Иеова!
\par 20 И какво повече може да Ти рече Давид? защото Ти, Господи Иеова, познаваш слугата Си.
\par 21 Заради Своето слово и според Своето сърце Ти си сторил всички тия велики дела, за да ги явиш на слугата Си.
\par 22 Затова Ти си велик, Господи Боже; защото няма подобен на Тебе, нито има бог освен Тебе, според всичко, що сме чули с ушите си.
\par 23 И кой друг народ на света е както Твоите люде, както Израил, при когото Бог дойде да го откупи за Свои люде, да ги направи именити и да извърши за тях велики дела, и страшни дела за земята Ти, пред Твоите люде, които Ти си откупил за Себе Си от Египет, от народите, и от боговете им?
\par 24 Защото Ти си утвърдил людете Си Израиля за Себе Си, за да Ти бъдат люде до века; и Ти, Господи, им стана Бог.
\par 25 И сега, Господи Боже, утвърди до века думата, която си говорил на слугата Си, и за дома му, и стори както си говорил.
\par 26 И нека възвеличат името Ти до века, като казват: Господ на Силите е Бог над Израиля; и нека бъде утвърден пред Тебе домът на слугата Ти Давида,
\par 27 Защото Ти, Господи на Силите, Боже Израилев, откри на слугата Си, като каза: Ще ти съградя дом: затова слугата Ти намери сърцето си разположено да Ти се помоли с тая молитва.
\par 28 И Сега, Господи Иеова, Ти си Бог, и думите Ти са истинни, и Ти си обещал тия блага на слугата Си;
\par 29 сега, прочее, благоволи да благословиш дома на слугата Си, за да пребъдва пред Тебе до века; защото Ти, Господи Иеова, Си говорил тия неща, и под Твоето благоволение нека бъде благословен до века дома на слугата Ти.

\chapter{8}

\par 1 След това Давид порази филистимците и ги покори; и Давид отне Метег-ама от ръката на филистимците.
\par 2 Порази и моавците, и измери ги с въжета, като ги накара да легнат на земята; и отмери две въжета, за да ги погуби, и едно цяло въже, за да ги остави живи. Така моавците станаха Давидови слуги, и плащаха данък.
\par 3 Давид още порази совския цар Ададезер, син на Реова, когато последният отиваше да утвърди властта си на реката Евфрат.
\par 4 Давид му отне хиляда колесници и седемстотин конници и двадесет хиляди пешаци; и Давид пресече жилите на всичките колеснични коне, запази от тях само за сто колесници.
\par 5 И когато дамаските сирийци дойдоха да помогнат на совския цар Ададезер, Давид порази от сирийците двадесет и две хиляди мъже.
\par 6 Тогава Давид постави гарнизони в дамаска Сирия; и сирийците станаха Давидови слуги, и плащаха данък. И Господ запазваше Давида където и да отиваше.
\par 7 И Давид взе златните щитове, които бяха върху слугите на Ададезера, та ги донесе в Ерусалим.
\par 8 И от Ветах и от Веротай, Ададезерови градове, цар Давид взе твърде много мед.
\par 9 А ематският цар Той, когато чу, че Давид поразил всичката сила на Ададезера,
\par 10 тогава Той изпрати сина си Иорама при цар Давида, за да го поздрави и да го благослови, понеже се бил против Ададезера и го поразил; защото Ададезер често воюваше против Тоя. И Иорам донесе със себе си сребърни съдове, златни съдове и медни съдове;
\par 11 па и тях цар Давид посвети на Господа, заедно със среброто и златото, що беше посветил взето от всичките народи, които беше покорил:
\par 12 от Сирия, от Моав, от амонците, от филистимците, от амаличаните, от користите взети от совенския цар Ададезер, Реововия син.
\par 13 И Давид си придоби име, когато се върна от поражението на осемнадесет хиляди сирийци в долината на солта.
\par 14 И постави гарнизони в Едом; в целия Едом постави гарнизони, и всичките едомци се подчиниха на Давида. И Господ запазваше Давида където и да отиваше.
\par 15 Така Давид царува над целия Израил; и Давид съдеше всичките си люде и им раздаваше права.
\par 16 А Иоав, Саруиният син, беше над войската; а Иосафат, Ахилудовият син, летописец;
\par 17 а Садок, Ахитововият син и Ахимелех, Авиатаровият син, свещеници; а Сарая, секретар;
\par 18 А Ванаия, Иодаевият син, беше над херетците и фелетците; а Давидовите синове бяха придворни началници.

\chapter{9}

\par 1 След това Давид каза: Остава ли още някой от Сауловия дом, комуто да покажа благост заради Ионатана?
\par 2 И имаше един слуга от Сауловия дом на име Сива. И повикаха го при Давида; и царят му каза: Ти ли си Сива? И той рече: Слугата ти е.
\par 3 И царят каза: Не остава ли още някой от Сауловия дом, комуто да покажа Божия благост? И Сива рече на царя: Има още един Ионатанов син повреден в нозете.
\par 4 И царят му каза. Где е той? А Сива рече на царя: Ето, той е в къщата на Махира, Амииловия син, в Лодавар.
\par 5 Тогава цар Давид изпрати да го вземат от къщата на Махира Амииловия син, от Лодавар.
\par 6 И когато Мемфивостей, син на Ионатана, Сауловия син, дойде при Давида, падна на лице та се поклони. И рече Давид: Мемфивостее! А той отговори: Ето слугата ти.
\par 7 И Давид му показа: Не бой се; защото непременно ще покажа благост към тебе заради баща ти Ионатана, и ще ти възвърна всичките земи на баща ти Саула; и ти всякога ще ядеш хляб на моята трапеза.
\par 8 А той му се поклони и рече: Кой е слугата ти та да пригледаш такова умряло куче като мене?
\par 9 Тогава царят повика Сауловия слуга Сива та му каза: Всичко, що принадлежи на Саула и на целия му дом, дадох на сина на господаря ти.
\par 10 Ти, прочее, ще му обработваш земята, ти и синовете ти и слугите ти, и ще донасяш доходите , за да има синът на господаря ти хляб да яде; но Мемфивостей, синът на господаря ти, всякога ще се храни на моята трапеза. (А Сива имаше петнадесет сина и двадесет слуги).
\par 11 И Сива рече на царя: Според всичко, що заповяда господарят ми царят на слугата си, така ще направи слугата ти. А Мемфивостей, рече царят , ще яде на моята трапеза, като един от царевите синове.
\par 12 И Мемфивостей имаше малък син на име Миха. А всички, които живееха в къщата на Сива, бяха слуги на Мемфивостея.
\par 13 Така Мемфивостей живееше в Ерусалим; защото винаги ядеше на царската трапеза. И той куцаше и с двата крака.

\chapter{10}

\par 1 След това, царят на амонците умря, и вместо него се възцари син му Анун.
\par 2 Тогава Давид каза: Ща покажа благост към Ануна Наасовия син, както баща му показа благост към мене. И така Давид прати чрез слугите си да го утешат за баща му. А когато Давидовите слуги дойдоха в земята на амонците,
\par 3 първенците на амонците рекоха на господаря си Ануна: Мислиш ли, че от почит към баща ти Давид ти е изпратил утешители? Не е ли пратил Давид слугите си при тебе, за да разузнаят града и да го съгледат та да го съсипе?
\par 4 Затова Анун хвана Давидовите слуги та обръсна половината от брадите им, и отряза дрехите им до половина - до бедрата им, и ги отпрати.
\par 5 А когато известиха това на Давида, той изпрати човеци да ги посрещнат, (понеже мъжете се крайно срамуваха), и да им рекат от царя: Седете в Ерихон догде пораснат брадите ви, и тогава се върнете.
\par 6 А като видяха амонците, че станаха омразни на Давида, амонците пратиха та наеха двадесет хиляди пешаци от вет-реовските сирийци и совенските сирийци, и хиляда мъже от царя на Мааха, и двадесет хиляди души от товските мъже.
\par 7 И когато чу това Давид, прати Иоава и цялото множество силни мъже.
\par 8 И амонците излязоха та се строиха за бой при входа на портата; а сирийците от Сова и Реов, и мъжете от Тов и Мааха бяха отделно на полето.
\par 9 А Иоав, като видя, че бяха се строили за бой против него отпред и отзад, избра измежду всичките отборни Израилеви мъже, та ги опълчи против сирийците;
\par 10 а останалите люде даде в ръката на брата си Ависей, който ги опълчи против амонците.
\par 11 И каза: Ако сирийците надделеят над мене, тогава ти ще ми дойдеш на помощ; а ако амонците надделеят над тебе, тогава аз ще ти дойда на помощ.
\par 12 Дерзай, и нека бъдем мъжествени за людете си и за градовете на нашия Бог; а Господ нека извърши каквото Му се вижда угодно.
\par 13 И тъй, Иоав и людете, които бяха с него, стъпиха в сражение против сирийците; а те побягнаха от него.
\par 14 А когато амонците видяха, че сирийците побягнаха, тогава и те побягнаха от Ависея и влязоха в града. Тогава Иоав се оттегли от амонците та дойде в Ерусалим.
\par 15 А сирийците, като видяха, че бяха поразени от Израиля, пак се събраха всички заедно.
\par 16 И Адарезер прати да известят сирийците, които бяха оттатък реката; и те дойдоха в Елам, с Совака, военачалника на Адарезера, на чело.
\par 17 И когато се извести на Давида, той събра целия Израил и премина Иордан та дойде в Елам. А сирийците се опълчиха против Давида и се биха с него.
\par 18 Но сирийците побягнаха пред Израиля; и Давид изби от сирийците мъжете на седемстотин колесници и четиридесет хиляди конници, и порази военачалника им Совак, и той умря там.
\par 19 И така, всичките царе, подвластни на Адарезера, като видяха, че бяха победени от Израиля, сключиха мир с Израиля и им се подчиниха. И сирийците не смееха вече да помагат на амонците.

\chapter{11}

\par 1 След една година, във времето когато царете отиват на война, Давид прати Иоава и слугите си с него и целия Израил; и те разбиха амонците, и обсадиха Рава. А Давид остана в Ерусалим.
\par 2 И надвечер стана от леглото си и се разхождаше по покрива на царската къща; и от покрива видя една жена, която се къпеше; а жената бе много красива на глед.
\par 3 И Давид прати да разпитат за жената; и рече един : Не е ли това Витсавее, дъщеря на Елиама, жена на хетееца Урия?
\par 4 И Давид прати човеци да я вземат, и когато дойде при него лежа с нея, (защото се бе очистила от нечистотата си); и тя се върна у дома си.
\par 5 А жената зачна; и прати да известят на Давида, казвайки: Непразна съм.
\par 6 Тогава Давид прати до Иоава да кажат : Изпрати ми хетееца Урия. И Иоав изпрати Урия при Давида.
\par 7 И когато дойде Урия при него, Давид го попита как е Иоав, как са людете и как успява войната.
\par 8 После Давид каза на Урия: Слез у дома си та умий нозете си. И Урия излезе из царската къща, а след него отиде дял от царската трапеза .
\par 9 Но Урия спа при вратата на царската къща с всичките слуги на господаря си, и не слезе у дома си.
\par 10 И когато известиха на Давида, казвайки: Урия не слезе у дома си, Давид каза на Урия: Не си ли дошъл от път? защо не слезе у дома си?
\par 11 И Урия каза на Давида: Ковчегът и Израил и Юда прекарват в шатри; и господарят ми Иоав и слугите на господаря ми са разположени в стан на открито поле; а аз ли да ида у дома си, за да ям и да пия и да спя с жена си? заклевам се в твоя живот и в живота на душата ти, не направям аз това нещо.
\par 12 Тогава Давид каза на Урия: Престой тук и днес, па утре ще те изпратя. И тъй, Урия престоя в Ерусалим през оня ден и през другия.
\par 13 И Давид го покани, та яде пред него и пи; и опи го. Но вечерта Урия излезе да спи на леглото си със слугите си, а у дома си не слезе.
\par 14 Затова, на утринта Давид писа писмо на Иоава и го прати чрез Уриева ръка.
\par 15 А в писмото написа, казвайки: Поставете Урия там, гдето сражението е най-люто; сетне се оттеглете от него, за да бъде ударен и да умре.
\par 16 И така, Иоав, като държеше града в обсада, определи Урия на едно място, гдето знаеше, че има храбри мъже.
\par 17 И когато мъжете излязоха от града та се биха с Иоава, паднаха неколцина от людете, ще каже , от Давидовите слуги; умря и хетеецът Урия.
\par 18 Тогава Иоав прати да известят на Давида всичко, що се бе случило във войната.
\par 19 И заповяда на вестителя, казвайки: Когато разкажеш на царя всичко, що се е случило във войната,
\par 20 ако пламне гневът на царя и той ти рече: Защо се приближихте толкоз до града да се биете? не знаехте ли, че щяха да стрелят от стената?
\par 21 Кой порази Авимелеха син на Ерувесета? Една жена не хвърли ли върху него от стената горен воденичен камък, та умря в Тевес? Защо се приближихте толкоз при стената? Тогава ти кажи: Умря и слугата ти хетееца Урия.
\par 22 И така, вестителят замина и, като дойде, извести на Давида всичко, за което Иоав го беше изпратил.
\par 23 Вестителят рече на Давида: Мъжете надделяха над нас та излязоха против нас на полето; а ние ги прогонихме дори до входа на портата;
\par 24 но стрелците стреляха от стената върху слугите ти, и неколцина от слугите на царя, умряха; умря и слугата ти хетеецът Урия.
\par 25 Тогава Давид каза на вестителя: Така да кажеш на Иоава: Да те не смущава това нещо, защото мечът пояжда някога едного и някога другиго; зависи още повече нападението си против града и съсипи го Също и ти го насърчи.
\par 26 И когато чу Уриевата жена, че мъжът й Урия умрял, плака за мъжа си.
\par 27 И като премина жалейката, Давид прати та я взе у дома си, и тя му стана жена, и му роди син. Но делото, което Давид бе сторил, беше зло пред Господа.

\chapter{12}

\par 1 Тогава Господ прати Натана при Давида. И той като дойде при него, каза му: В един град имаше двама човека, единият богат, а другият сиромах.
\par 2 Богатият имаше овци и говеда твърде много;
\par 3 а сиромахът нямаше друго освен едно малко женско агне, което бе купил и което хранеше; а то бе пораснало заедно с него и чадата му, от залъка му ядеше, от чашата му пиеше, и на пазухата му лежеше; и то ме беше като дъщеря.
\par 4 И един пътник дойде у богатия; и нему се посвидя да вземе своите овци и от своите говеда да сготви за пътника, който бе дошъл у него, но взе агнето на сиромаха та го сготви за човека, който бе дошъл у него.
\par 5 Тогава гневът на Давида пламна силно против тоя човек; и той каза на Натана: В името на живия Господ, човекът, който е сторил това, заслужава смърт;
\par 6 и ще плати за агнето четверократно, понеже е сторил това дело, и понеже не се е смирил.
\par 7 Тогава Натан каза на Давида: Ти си тоя човек. Така казва Господ Израилевият Бог: Аз те помазах за цар над Израиля, и те избавих от Сауловата ръка;
\par 8 и дадох ти дом на господаря ти, и жените на господаря ти в пазухата ти, и дадох ти Израилевия и Юдовия дом; и ако това беше малко, приложил бих ти това и това.
\par 9 Защото презря ти словото на Господа, та стори зло пред очите Му? Ти порази с меч хетееца Урия, и си взе за жена неговата жена, а него ти уби с меча на амонците.
\par 10 Сега, прочее, няма никога да се оттегли меч от дома ти, понеже ти Ме презря, та взе жената на хетееца Урия, за да ти бъде жена.
\par 11 Така казва Господ: Ето, отсред твоя дом ще подигна против тебе злини, и ще взема жените ти пред очите ти, та ще ги дам на ближния ти; и той ще лежи с жените ти пред това слънце.
\par 12 Защото ти си извършил това тайно; но Аз ще извърша туй нещо пред целия Израил и пред слънцето.
\par 13 Тогава Давид каза на Натана: Съгреших Господу, А Натан каза на Давида: И Господ отстрани греха ти; няма да умреш.
\par 14 Но понеже чрез това дело ти си дал голяма причина на Господните врагове да хулят, затова детето което ти се е родило, непременно ще умре.
\par 15 И Натан си отиде у дома си. А Господ порази детето, което Уриевата жена роди на Давида, и то се разболя.
\par 16 Давид, прочее, се моли Богу за детето; и Давид пости, па влезе та пренощува легнал на земята.
\par 17 И старейшините на дома му станаха и дойдоха при него, за да го дигнат от земята; но той отказа, нито вкуси хляб с тях.
\par 18 И на седмия ден детето умря. И слугите на Давида се бояха да му явят, че детето бе умряло, защото думаха: Ето, докато детето бе още живо говорехме му, и той не слушаше думите ни; колко, прочее, ще се измъчва, ако му кажем, че детето е умряло!
\par 19 Но Давид, като видя че слугите му шепнеха помежду си, разбра, че бе умряло детето; затова Давид каза на слугите си: Умря ли детето? А те рекоха: Умря.
\par 20 Тогава Давид стана от земята, уми се и се помаза, и като промени дрехите си, влезе в Господния дом та се помоли. После дойде у дома си; и, понеже поиска, сложиха пред него хляб, и той яде.
\par 21 А слугите му му казаха: Що е това, което ти стори? Ти пости и плака за детето, докато беше живо; а като умря детето, ти стана и яде хляб!
\par 22 А той рече: Докато детето беше още живо, постих и плаках, защото си рекох: Кой знае? може Бог да ми покаже милост, и детето да остане живо.
\par 23 Но сега то умря. Защо да постя? Мога ли да го върна надире? Аз ще ида при него, а то няма да се върне при мене.
\par 24 След това, Давид утеши жена си Витсавее, и влезе при нея та лежа с нея; и тя роди син, и нарече го Соломон. И Господ го възлюби,
\par 25 и прати чрез пророка Натана та го нараче Едидия, заради Господа.
\par 26 А Иоав воюва против Рава на амонците и превзе царския град.
\par 27 И Иоав прати вестители до Давида, да кажат: Воювах против Рава, и даже превзех града на водите.
\par 28 Сега, прочее, събери останалите люде та разположи стана си против града и завладей го, да не би аз да завладея града, и той да се нарече с моето име.
\par 29 Затова Давид събра всичките люде та отиде в Рава, би се против нея, и я завладя.
\par 30 И взе от главата на царя им короната му, която тежеше един златен талант, и бе украсена със скъпоценни камъни; и положиха я на главата на Давида. И той изнесе из града твърде много користи.
\par 31 Изведе и людете, които бяха в него, та го тури под триони и под железни дикани и под железни брадви, и преведе ги през пещта за тухли; и така постъпи с всички градове на амонците. Тогава Давид се върна с всичките люде в Ерусалим.

\chapter{13}

\par 1 След това, Амнон Давидовият син, залюби сестрата, която имаше Авесалом, Давидовият син, - една хубавица на име Тамар.
\par 2 А Амнон до толкоз се измъчваше, щото се разболя поради сестра си Тамар; защото беше девица, и на Амнона се виждаше твърде мъчно да стори нещо с нея.
\par 3 Но Амнон имаше един приятел на име Ионадав, син на Давидовия брат Сама: а Ионадав беше много хитър човек.
\par 4 И рече на Амнона : Защо ти, сине на царя, слабееш толкоз от ден на ден? не щеш ли ми яви причината ? И Амнон му каза: Обичам Тамар сестрата на брата ми Авесалома.
\par 5 А Ионадав му рече: Легни на леглото си, та се престори на болен; и когато доде баща ти да те види, кажи му: Нека дойде, моля, сестра ми Тамар, и нека ми даде хляб да ям, и нека сготви ястието пред мене за да видя, и да ям от ръката й.
\par 6 И тъй, Амнон легна та се престори на болен, и като дойде царят да го види, Амнон каза на царя: Нека дойде, моля, сестра ми Тамар, и нека направи пред мене две мекици, за да ям от ръката й.
\par 7 И Давид прати у дома да кажат на Тамар: Иди сега в къщата на брата си Амнон та му сготви ястие.
\par 8 И Тамар отиде в къщата на брата си Амнон, който беше легнал; и като взе та замеси тесто и направи мекици пред него, изпържи мекиците.
\par 9 Сетне взе тавата и ги изсипа пред него; но той отказа да яде. И рече Амнон: Изведете вън всичките човеци, що са при мене. И всичките излязоха от него.
\par 10 Тогава Амнон каза на Тамар: Донеси ястието в спалнята, за да ям от ръката ти. И тъй, Тамар взе мекиците, които бе направила, та ги донесе в спалнята при брата си Амнона.
\par 11 А като ги донесе близо при него за да яде, той я хвана и рече й: Ела, легни с мене, сестро моя.
\par 12 Но тя му отговори: Не, брате мой, не ме насилвай, защото не е прилично да стане такова нещо в Израиля; да не сториш това безумие.
\par 13 И аз де да скрия срама си? па и ти ще бъдеш като един от безумните в Израиля. Сега, прочее, моля, говори на царя; защото той няма да ме откаже на тебе.
\par 14 Но той отказа да послуша гласа й: и понеже бе по-як от нея, насили я и лежа с нея.
\par 15 Тогава Амнон я намрази с много голяма омраза, така щото омразата, с който я намрази бе по-голяма от любовта, с която я бе залюбил. И рече й Амнон: Стани, иди си.
\par 16 А тя му рече: Не така, защото това зло, гдето ме пъдиш е по-голямо от другото, което ми стори. Но той отказа да я послуша.
\par 17 И викна момъка, който му прислужваше, и рече: Изведи сега тази вън от мене, и заключи вратата след нея.
\par 18 А тя бе облечена в шарена дреха, защото царските дъщери, девиците, в такива дрехи се обличаха. И тъй, слугата му я изведе вън та заключи вратата след нея.
\par 19 Тогава Тамар тури пепел на главата си, раздра шарената дреха, която бе на нея, и като положи ръка на главата си отиваше си и викаше като вървеше.
\par 20 А брат й Авесалом й каза: Брат ти Амнон ли се поруга с тебе? Но мълчи сега, сестро моя; брат ти е; не оскърбявай сърцето си за това нещо. И така, Тамар живееше като вдовица в дома на брата си Авесалома.
\par 21 А когато цар Давид чу всички тия работи, много се разгневи.
\par 22 Между това, Авесалом не казваше на Амнона ни зло ни добро; защото Авесалом мразеше Амнона за гдето беше насилил сестра му Тамар.
\par 23 А след цели две години Авесалом имаше стригачи в Ваал-асор, който е близо при Ефрем; и Авесалом покани всичките царски синове.
\par 24 Авесалом дойде и при царя та каза: Ето, слугата ти има сега стригачи; нека дойде, моля царят и слугите му със слугата ти.
\par 25 А царят каза на Авесалома: Не, сине мой, да не идем всички; за да не те отегчаваме. И той го молеше настойчиво; но царят отказа да иде, а го благослови.
\par 26 Тогава рече Авесалом; Ако не, то нека дойде с нас поне брат ми Амнон. И царят му каза: Защо да иде с тебе?
\par 27 Но понеже Авесалом настоя, той позволи да отидат с него Амнон и всичките царски синове.
\par 28 Тогава Авесалом заповяда на слугите си, казвайки: Гледайте когато Амноновото сърце се развесели от виното и аз ще ви кажа: Поразете Амнона, тогава го убийте. Не бойте се; не ви ли заповядвам аз? бъдете дръзновени и храбри.
\par 29 И Авесаломовите слуги сториха на Амнона според както Авесалом заповяда. Тогава всичките царски синове станаха и възседнаха всеки на мъската си та побягнаха.
\par 30 И докато те бяха още по път, слух стигна до Давида, и се казваше: Авесалом избил всичките царски синове, и не останал ни един от тях.
\par 31 Тогава царят стана, раздра дрехите си, и легна на земята; и всичките му слуги, които предстояваха, раздраха дрехите си.
\par 32 А Ионадав, син на Сама, Давидовия брат, проговори, казвайки: Да не мисли господарят ми, че са избили всичките младежи царски синове, защото само Амнон е умрял: понеже с дума от Авесалома това е било решено от деня, когато изнасили сестра му Тамар.
\par 33 Сега, прочее, господарят ми царят да не вложи това в сърцето си и да не помисли, че всичките царски синове са измрели; защото само Амнон е умрял.
\par 34 А Авесалом побягна. В това време момъкът, който пазеше стража, като подигна очи видя, и ето, много люде идеха из пътя зад него край хълма.
\par 35 И Ионадав каза на царя: Ето, царските синове идат; според както каза слугата ти, така е станало.
\par 36 И като изговори това, ето, царските синове дойдоха и плакаха с висок глас; също и царят и всичките плакаха твърде много.
\par 37 Но Авесалом побягна и отиде при гесурския цар Талмая Амиудовия син; а Давид жалееше за сина си всеки ден.
\par 38 Авесалом, прочее, побягна и отиде в Гесур, и там преживя три години.
\par 39 И душата на цар Давида излезе и чезнеше по Авесалома, защото беше се утешил за Амноновата смърт.

\chapter{14}

\par 1 И Иоав Саруиният син позна, че сърцето на царя беше наклонено към Авесалома.
\par 2 Затова Иоав прати в Текое та доведоха от там една умна жена, и рече й: Престори се, моля, че си в жалейка и облечи траурни дрехи, и не се помазвай с масло, но бъди като жена, която жалее дълго време за мъртвец;
\par 3 и иди при царя та му говори според тия думи. И Иоав тури думите в устата й.
\par 4 И когато текойката дойде да говори на царя, падна с лице на земята та се поклони, и рече: Помогни ми царю.
\par 5 И царят й рече: Що имаш? А тя рече: Ах! аз съм вдовица; мъжът ми умря.
\par 6 Слугинята ти имаше двама сина, и те двамата се сбиха на полето; и като нямаше кой да ги раздели, единият удари другият и го уби.
\par 7 И, ето, всичките роднини станаха против слугинята ти и рекоха: Предай оногова, който удари брата си, за да го убием за живота на брата му, когото уби, и така да изтребим и наследника. Така ще угасят въглена, който ми е останал, и не ще оставят на мъжа ми ни име, ни остатък по лицето на света.
\par 8 И царят каза на жената: Иди у дома си; аз ще разпоредя за тебе.
\par 9 А текойката каза на царя: Господарю мой царю, беззаконието нека бъде на мене и на бащиния ми дом; а царят и престолът му нека бъдат невинни.
\par 10 И рече царят: Който проговори против тебе, доведи го при мене, и няма вече да те докачи.
\par 11 И тя рече: Моля, нека помни царят Господа своя Бог, за да не продължава вече отмъстителят за кръвта да изтребва, и да не погуби сина ми. А той рече: заклевам се в живота на Господа, ни един косъм на сина ти няма да падне на земята.
\par 12 Тогава жената рече: Нека каже, моля, слугинята ти една дума на господаря си царя. Той рече: Кажи.
\par 13 И жената рече: Тогава защо си помислил такова нещо против Божиите люде? понеже царят, като каза това, излиза виновен за гдето царят не възвръща своя заточеник.
\par 14 Защото е неизбежно ние да умрем, и сме като вода разляна по земята, която не се събира пак; па и Бог не отнема живот, но изнамерва средства, щото заточеният да не остане отдалечен от Него.
\par 15 Сега по тая причина дойдох да кажа това на господаря си царя, че людете ме уплашиха; и слугинята ти рече: Ще говоря сега на царя; може-би царят да изпълни просбата на слугинята си.
\par 16 Защото царят ще послуша, за да избави слугинята си от ръката на човека, който иска да погуби и мене и сина ми изсред Божието наследство.
\par 17 И слугинята ти рече: Думата на господаря ми нека бъде сега утешителна; защото господарят ми царят е като Божий ангел за да разсъждава за доброто и злото; и Господ твой Бог да бъде с тебе.
\par 18 Тогава царят в отговор каза на жената; Моля, да не скриеш от мене това, за което ще те попитам. И рече жената: Нека проговори, моля, господарят ми царят.
\par 19 И рече царят: С тебе ли е във всичко това ръката на Иоава? И жената в отговор рече: Заклевам се в живота на душата ти господарю мой царю, никоя не може да се отклони ни на дясно ни на ляво от нищо що изрече господарят ми царят; защото слугата ти Иоав, той ми заповяда, и той тури всички тия думи в устата на слугинята ти.
\par 20 Слугата ти Иоав стори това, за да измени лицето на работата; и господарят ми е мъдър с мъдростта на един Божий ангел, за да познава всичко, що има на света.
\par 21 Тогава царят каза на Иоава: Ето сега, сторих това нещо; иди прочее, доведи младежа Авесалом.
\par 22 И Иоав падна с лице на земята та се поклони и благослови царя. И рече Иоав: Днес слугата ти познава, че придобих твоето благоволение, господарю мой царю, тъй като царят изпълни желанието на слугата си.
\par 23 И така Иоав стана и отиде в Гесур, и доведе Авесалома в Ерусалим.
\par 24 А царят рече: Нека се отбие в своята къща, а моето лице да не види. Затова Авесалом се отби в своята си къща и не видя лицето на царя.
\par 25 А в целия Израил нямаше човек толкова много хвален за красотата си колкото Авесалома; от петата на ногата му, дори до върха на главата му нямаше в него недостатък.
\par 26 И когато острижеше главата си, (защото всяка година я стрижеше, понеже косата му натегваше, затова я стрижеше), претегляше косата на главата си, и тя тежеше двеста сикли според царската теглилка.
\par 27 И родиха се на Авесалома трема сина и една дъщеря на име Тамар, която беше красива.
\par 28 И Авесалом седя в Ерусалим цели две години без да види лицето на царя.
\par 29 Тогава Авесалом повика Иоава, за да го изпрати при царя; но той отказа да дойде при него. И повика го втори път; но пак отказа да дойде.
\par 30 Затова, каза на слугите си: Вижте, Иоавовата нива е близо до моята, и там има ечемик; идете запалете го. И тъй Авесаломовите слуги запалиха нивата.
\par 31 Тогава Иоав стана та отиде при Авесалома в къщата му и му каза: Защо са запалили слугите ти нивата ми?
\par 32 А Авесалом отговори на Иоава: Ето, пратих до тебе да кажат: Дойди тук за да те пратя при царя да кажеш: Защо съм дошъл от Гесур? по-добре щеше да е за мене да бях още там; сега, прочее, нека видя лицето на царя; и ако има неправда в мене, нека ме убие.
\par 33 Тогава Иоав дойде при царя та му извести това; и той повика Авесалома, който като дойде при царя, поклони се с лица до земята пред царя; и царят целуна Авесалома.

\chapter{15}

\par 1 След това Авесалом си приготви колесници и коне и петдесет мъже, които да тичат пред него.
\par 2 И Авесалом ставаше рано, та стоеше край пътя до портата; и когато някой имаше дело, за което трябваше да дойде пред царя за съдба, тогава Авесалом го викаше и думаше: От кой си град? А той казваше: Слугата ти е от еди-кое Израилево племе.
\par 3 И Авесалом му казваше: Виж, твоята работа е добра и права; но от страна на царя няма кой да те слуша.
\par 4 Авесалом още казваше: Да бях само поставен съдия на тая страна, за да идва при мене всеки, който име спор или дело, та да го оправдавам!
\par 5 И когато някой се приближаваше да му се поклони, той простираше ръка та го хващаше и го целуваше.
\par 6 Така постъпваше Авесалом с всеки израилтянин, който дохождаше при царя за съд; и по тоя начин Авесалом подмамваше сърцата на Израилевите мъже.
\par 7 И като се свършиха четири години, Авесалом каза на царя: Да отида, моля, в Хеврон, за да изпълня обрека, който направих Господу;
\par 8 защото слугата ти направи обрек, когато живееше в Гесур у Сирия, и казах: Ако наистина ме върне Господ в Ерусалим, тогава ще послужа Господу.
\par 9 И царят му каза: Иди с мир. И така, той стана та отиде в Хеврон.
\par 10 А Авесалом разпрати шпиони по всичките Израилеви племена да казват: Щом чуете тръбния звук, кажете: Авесалом се възцари в Хеврон.
\par 11 А с Авесалома отидоха от Ерусалим двеста мъже, поканени, които отидоха простодушно, без да знаят нищо.
\par 12 После, докато принасяше жертвите, Авесалом покани Давидовия съветник, гилонеца Ахитофел, от града му Гило. И заговорът бе силен, понеже людете постоянно се умножаваха около Авесалома.
\par 13 Тогава дойде вестител при Давида и каза: Сърцата на Израилевите мъже се обърнаха към Авесалома.
\par 14 А Давид каза на всичките си слуги, които бяха с него в Ерусалим: Станете да бягаме, иначе никой от нас не ще може да се опази от Авесалома; побързайте да отидем, да не би да ни застигне скоро, та да докара зло върху нас, и порази града с острото на ножа.
\par 15 И царските слуги рекоха на царя: Ето, слугите ти са готови да вършат все що избере господарят ни царят.
\par 16 И тъй, царят излезе, и целият му дом подир него. А царят остави десетте жени, наложници, да пазят къщата.
\par 17 Царят, прочее, излезе, и всичките люде подир него, и спряха се на едно далечно място.
\par 18 И всичките му слуги вървяха близо до него; а всичките херетци, всичките фелетци и всичките гетци, шестстотин мъже, които бяха го последвали от Гет, вървяха пред царя.
\par 19 Тогава царят каза на гетеца Итай: Защо идеш и ти с нас? върни се и остани с церя, защото си чужденец и преселен от мястото си.
\par 20 Ти вчера дойде; и днес да те направя ли да се скиташ с нас? Ето, аз ще ида където мога; ти се върни, заведи още и братята си; милост и вярност да бъдат с тебе!
\par 21 А Итай в отговор рече на царя: Заклевам се в живота на Господа и в живота на господаря ми царя, гдето и да бъде смърт, било за живот, непременно там ще бъде и слугата ти.
\par 22 Тогава Давид каза на Итая: Иди та премини. И тъй, премина гетецът Итай, и всичките мъже, и всичките деца що бяха с него.
\par 23 А цялата местност плачеше със силен глас като преминаваха всичките люде; а царят премина потока Кедрон, и всичките люде преминаха през пътя за пустинята.
\par 24 А ето дойде и Садок и с него всичките левити, който носеха ковчега за плочите на Божия завет; и сложиха Божия ковчег (при който се качи Авиатар) догде всичките люде излязоха от града.
\par 25 Тогава царят каза на Садока: Върни Божия ковчег в града; ако придобия благоволението на Господа, Той ще ме възвърне за да видя Него и обиталището Му;
\par 26 но ако рече така: Нямам благоволение в тебе, - ето ме, нека ме стори каквото Му се вижда за добро.
\par 27 Царят още каза на свещеника Садока: Ти, гледаче, върни се в града с мир; и с вас нека се върнат и двамата ви сина: твоят син Ахимаас, и Авиатаровият син Ионатан.
\par 28 Вижте, аз ще се бавя при бродовете в пустинята, докато дойде дума от вас да ми извести какво да направя .
\par 29 И тъй, Садок и Авиатар върнаха Божия ковчег в Ерусалим, и там останаха.
\par 30 А Давид се възкачваше по нагорнището на маслинения хълм и плачеше, като се изкачваше, с покрита глава и вървейки бос; и всичките люде що бяха с него покриваха всеки главата си, и плачеха като се изкачваха.
\par 31 И известиха на Давида, казвайки: Ахитофел е между заговорниците с Авесалома. И рече Давид: Господи, моля Ти се, осуети съвета на Ахитофела.
\par 32 И когато стигна Давид на върха на хълма, гдето ставаше богопоклонение, ето архиецът Хусей го посрещна с дрехата си раздрана и с пръст на главата си.
\par 33 И Давид му каза: Ако дойдеш с мене, ще ми бъдеш товар;
\par 34 но ако се върнеш в града и речеш на Авесалома: Царю, ще ти бъда слуга; както бях слуга на баща ти до сега, така ще бъда слуга на тебе, - тогава можеш да осуетиш за мене съвета на Ахитофела.
\par 35 Не са ли там с тебе свещениците Садок и Авиатар? всичко, прочее, каквото би чул от царския дом, съобщи на свещениците Садока и Авиатара;
\par 36 ето, там с тях са двамата ум сина, Ахимаас Садоков и Ионатан Авиатаров; чрез тях ми съобщавайте всичко, каквото чуете.
\par 37 И така, Давидовият приятел Хусей, влезе в града; също и Авесалом влезе в Ерусалим.

\chapter{16}

\par 1 И когато Давид беше преминал малко върха на хълма , ето, Мемфивостеевият слуга Сива го срещна с два оседлани осела, на които имаше двеста хляба, сто грозда сухо грозде, сто низаници летни овощия и мех вино.
\par 2 И царят каза на Сива: Защо носиш това? А Сива рече: Ослите са, за да язди царското семейство, а хлябовете и летните овощия, за да ги ядат момците, а виното, за да пият ония, които изнемощеят в пустинята.
\par 3 Тогава рече царят: А где е синът на господаря ти? И Сива каза на царя: Ето, остава в Ерусалим, защото рече: Днес Израилевият дом ще ми възвърне бащиното ми царство.
\par 4 И царят каза на Сива: Ето, твой е целият Мемфивостеев имот. И рече Сива: Кланям ти се; нека придобия твоето благоволение, господарю мой царю.
\par 5 А когато царят Давид стигна у Ваурим, ето, излезе от там човек от семейството на Сауловия дом на име Семей, Гираев син; и като излезе, вървеше и кълнеше.
\par 6 И хвърляше камъни върху Давида и върху всичките слуги на цар Давида; а всичките люде и всичките силни мъже бяха отдясно му и от ляво му.
\par 7 И Семей като кълнеше говореше така: Излез, излез, мъжо кръвниче и мъжо злосторниче!
\par 8 Господ докара върху тебе всичката кръв на дома на Саула, вместо когото ти се възцари, и Господ предаде царството в ръката на сина ти Авесалома; и ето те в нещастието ти, защото си кръвник.
\par 9 Тогава Ависей Саруиният син каза на царя: Защо да кълне това мъртво куче господаря ми царя? позволи ми, моля, да мина насреща и да му отсека главата.
\par 10 А царят рече: Какво общо има между мене и вас, Саруини синове? Когато кълне, и когато Господ му е казал: Прокълни Давида, който ще му рече: Защо правиш така?
\par 11 Давид още каза на Ависея и на всичките си слуги: Ето, син ми който е излязъл из чреслата ми, търси живота ми, а колко повече сега тоя вениаминец! Остави го нека кълне, защото Господ му е заповядал.
\par 12 Може би Господ ще погледне онеправданието ми, и Господ ще ми въздаде добро вместо неговото проклинане днес.
\par 13 И така, Давид и мъжете му вървяха по пътя; а Семей вървеше край хълма срещу него и кълнеше като вървеше, хвърляше камъни върху него, и хвърляше прах.
\par 14 И царят и всичките люде що бяха с него пристигнаха уморени, та си починаха там.
\par 15 Между това, Авесалом и всичките люде, сиреч , Израилевите мъже дойдоха в Ерусалим; и Ахитофел с него.
\par 16 И когато архиецът Хусей, Давидовият приятел дойде при Авесалома, Хусей каза на Авесалома: Да живее царят! Да живее царят!
\par 17 А Авесалом каза на Хусая: Това ли е благостта ти към приятеля ти? защо не отиде с приятеля си?
\par 18 И Хусей каза на Авесалома: Не, но когато Господ и тия люде и всичките Израилеви мъже избраха, негов ще бъда и с него ще остана.
\par 19 При това кому ще бъде служенето ми? не ли на сина на приятеля ми ? Когато съм служил пред баща ти, така ще бъда и пред тебе.
\par 20 Тогава Авесалом каза на Ахитофела: Дайте си съвета какво да правим.
\par 21 И Ахитофел каза на Авесалома: Влез при бащините си наложници, които е оставил да пазят къщата; и като чуе целият Израил, че си станал омразен на баща си, ще се усилат ръцете на всичките, които са с тебе.
\par 22 Поставиха, прочее, шатъра за Авесалома върху къщния покрив; и Авесалом влезе при бащините си наложници пред целия Израил.
\par 23 А в онова време съветът, който даваше Ахитофела, бе считан като че ли някой бе търсил съвет от слово на Бога; такъв се считаше всеки Ахитофелов съвет и от Давида и от Авесалома.

\chapter{17}

\par 1 Ахитофел каза още на Авесалома: Да избера сега дванадесет хиляди мъже, и да стана да преследвам Давида още тая нощ.
\par 2 Ще налетя върху него като е уморен и ръцете му ослабнали, и ще го уплаша; и всичките люде, що са с него, ще побягнат, и ще поразя само царя.
\par 3 Така ще възвърна всичките люде към тебе, защото убиването на мъжа, когото ти търсиш, значи възвръщането на всичките; всичките люде ще се помирят.
\par 4 И тая дума бе угодна на Авесалома и на всичките Израилеви старейшини.
\par 5 Тогава рече Авесолом: Повикай сега и архиеца Хусай, и нека чуем какво ще каже и той.
\par 6 И когато Хусей влезе при Авесалома, Авесалом му говори, казвайки: Така е говорил Ахитофел. Да постъпим ли според неговия съвет? ако не, говори ти.
\par 7 И Хусей каза на Авесалома: Не е добър съветът, който Ахитофел даде тоя път.
\par 8 Хусей рече още: Ти знаеш баща си и неговите мъже, че са силни мъже, и че са преогорчени в душа, както мечка лишена от малките си в полето; и баща ти, като военен мъж, няма да пренощува с людете;
\par 9 ето, сега скрит в някой трап или в някое друго място, и когато нападне някой от тия свои люде в началото на сражението , всеки, който чуе ще рече: Поражение става между людете, които следват Авесалома.
\par 10 Тогава даже юначният, чието сърце е като лъвско сърце, съвсем ще премре; защото целият Израил знаеше, че баща ти е юнак, и че ония, които са с него, са храбри мъже.
\par 11 Аз съветвам по-добре да се събере при тебе целият Израил, от Дан до Вирсавее, по множество както пясъка край морето, и ти лично да идеш в боя.
\par 12 Така ще налетим върху него в някое място, където се намери, и ще го нападнем, както пада росата на земята, тъй щото от него и от всичките човеци, които са с него, няма да оставим ни един.
\par 13 Или, ако прибегне в някой град, тогава целият Израил ще донесе до оня град въжета, та ще го завлечем до потока, тъй щото да не остане там ни едно камъче.
\par 14 Тогава Авесалом и всичките Израилеви мъже казаха: Съветът на архиеца Хусей е по-добър от съвета на Ахитофела. (Защото Господ беше рекъл да осуети добрия Ахитофелов съвет, за да нанесе Господ зло на Авесалома).
\par 15 Тогава Хусей каза на свещениците Садоха и Авиатара: Така и така съветва Ахитофел Авесалома и Израилевите старейшини; а така и така съветвах аз.
\par 16 Сега, прочее, пратете скоро да известят на Давида, казвайки: Не оставай тая нощ при бродовете в пустинята, но непременно да преминеш Иордан , за да не погинат царят и всичките люде, които са с него.
\par 17 А Иоанатан и Ахимаас стояха при извора Рогил, защото не смееха да влизат явно в града: затова, една слугиня отиде да им извести това, и те отидоха та известиха на цар Давида.
\par 18 А един момък ги видя и обади на Авесалома; но двамата отидоха бърже та влязоха в къщата на един човек във Ваурим, който имаше кладенец в двора си, и спуснаха се в него.
\par 19 И жена му взе една покривка та я простря върху отвора на кладенеца, и насипа върху нея чукано жито, така щото нищо не се позна.
\par 20 И когато Авесаломовите слуги дойдоха при жената в къщата и казаха: Где са Ахимаас и Ионатан? Жената им рече: Преминах водния поток. И те ги потърсиха, но като не ги намериха, върнаха се в Ерусалим.
\par 21 А след заминаването им, ония излязоха от кладенеца и отидоха та известиха на цар Давида. Рекоха на Давида: Станете преминете скоро водата, защото Ахитофел така съветва против вас.
\par 22 Тогава Давид и всичките люде, що бяха с него, станаха та преминаха Иордан; до зори не остана ни един, който не беше преминал Иордан.
\par 23 А Ахитофел, като видя, че съветът му не се възприе, оседла осела си и стана та отиде у дома си, в своя град; и като нареди домашните си работи, обеси се. Така умря; и погребан бе в бащиния си гроб.
\par 24 Тогава Давид дойде в Механаим; а Авесалом премина Иордан, той и всичките Израилеви мъже с него.
\par 25 И Авесалом постави Амаса за военачалник вместо Иоава. А Амаса бе син на един човек на име Итра, израилтянин, който беше влязъл при Авигея Наасовата дъщеря, сестра на Саруия Иоавовата майка).
\par 26 И Израил и Авесалом разположиха стана си в галаадската земя.
\par 27 А когато дойде Давид в Маханаим, Совей Наасовият син, от Рава на амонците, и Махир Амииловият син, от Ло-девар, и галаадецът Верзелай, от Рогелим,
\par 28 донесоха постелки, легени и пръстени съдове, жито, ечемик, брашно, пържено жито, боб, леща и други печени храни,
\par 29 мед, масло, овци и говеждо сирене на Давида и на людете с него за да ядат; защото рекоха: Людете са гладни и изнемощели и жадни в пустинята.

\chapter{18}

\par 1 След това Давид събра людете, които бяха с него, и постави над тях хилядници и стотници.
\par 2 И Давид изпрати людете, една трета под началството на Иоава, една трета под началството на Ависея Саруиния син, Иоавовия брат, и една трета под началството на гетеца Итай. И царят каза на людете: Непременно ще изляза и аз с вас.
\par 3 Людете, обаче, отговориха: Да не излезеш; защото, ако ние се обърнем на бяг, няма да ги е грижа за нас; ако щат умра и половината от нас, пак няма да ги е грижа за нас; а ти си като десет хиляди от нас, затова сега е по-добре ти да си готов да ни помагаш от града.
\par 4 И царят им каза: Каквото ви се вижда добро ще сторя. И тъй, царят застана на едната страна на портата, а всичките люде излязоха по стотини и по хиляди.
\par 5 Тогава царят заповяда на Иоава, на Ависея и на Итая, казвайки: Пощадете ми младежа Авесалома. И всичките люде чуха, когато царят заповядваше на всичките военачалници за Авесалома.
\par 6 И тъй, людете, излизаха на полето против Израиля; и сражението стана в Ефремовия лес.
\par 7 И там Израилевите люде бидоха разбити от Давидовите слуги, и в оня ден там стана голямо клане на двадесет хиляди души;
\par 8 защото сражението в тая местност се разпростря по лицето на цялата страна, и в оня ден лесът погълна повече люде отколкото погълна ножа.
\par 9 И случи се Авесалом да се срещне с Давидовите слуги. А Авесалом яздеше на мъска; и като влезе мъската под гъстите клони на един голям дъб, главата му се хвана в дъба, и той увисна между небето и земята: а мъската мина изпод него.
\par 10 И един човек го видя та извести на Иоава, казвайки: Ето, видях Авесалома увиснал на дъб.
\par 11 А Иоав каза на човека, който му извести: Ето, ти си го видял; а защо не го порази там до земята? и аз бих ти дал десет сребърника и един пояс.
\par 12 А човекът рече на Иоава: И хиляда сребърника, ако бяха претеглени в шепата ми, не бих дигнал ръката си против царския син; защото ние слушахме царят как заповяда на тебе, на Ависея и на Итая, казвайки: Внимавайте всички, никой да се не докосне до младежа Авесалома.
\par 13 Иначе, ако бяха постъпили невярно против живота му, нищо не се укрива от царя, и тогава сам ти би се обърнал против мене.
\par 14 Тогава рече Иоав: Не трябва да губя време така с тебе. И като взе в ръката си три стрели, прониза с тях сърцето на Авесалома, като беше още жив всред дъба.
\par 15 И десет момъка, Иоавови оръженосци, заобиколиха Авесалома, та го удариха и убиха го.
\par 16 Тогава Иоав засвири с тръбата, и людете се върнаха от преследването на Израиля; защото Иоав спря людете.
\par 17 И взеха Авесалома та го хвърлиха в един голям ров, вътре в леса, и натрупаха на него много голям куп камъни. И целият Израил побягна, всеки в шатъра си.
\par 18 А Авесалом, когато беше още жив, бе взел и издигнал за себе си стълба, който е в царската долина; защото си рече: Нямам син, който да опази паметта на името ми; затова нарече стълба по своето име; и той се нарича и до днес Авесаломов паметник.
\par 19 Тогава рече Ахимаас, Садоковият син: Да се завтека сега да занеса на царя известие, че Господ въздаде за него на неприятелите му.
\par 20 А Иоав му рече: Няма да занесеш днес известия; друг ден ще бъдеш известител; а днес няма да занесеш известия, понеже царският син умря.
\par 21 Тогава рече Иоав на Хусина: Иди, извести на царя каквото си видял. И Хусина се поклони на Иоава и се завтече.
\par 22 Тогава Ахимаас, Садоковият син, рече пак на Иоава: Но каквото и да стане, нека тичам и аз, моля след Хусина. А Иоав му рече: Защо искаш да тичаш, синко, като не ще имаш възнаграждение за известията?
\par 23 Но той пак рече : Но каквото и да стане, нека се завтека. Тогава му рече: Тичай. И така Ахимаас се завтече през полския път, и замина Хусина.
\par 24 А Давид седеше между двете порти; и стража се изкачи на покрива на портата към стената, и като подигна очи видя, и, ето, един човек тичаше сам.
\par 25 И стражът извика и извести на царя. А царят рече: Щом е сам, има известия в устата му. И той притичваше и се приближаваше.
\par 26 После стражът видя друг човек, който тичаше; и стражът извика към вратаря, казвайки: Ето още един човек, който тича сам. И рече царят: И той носи известия.
\par 27 И рече стражът: струва ми се, че тичането на първия прилича на тичането на Ахимаас, Садоковия син. И рече царят: Добър човек е той, и иде с добри известия.
\par 28 А Ахимаас извика та рече на царя: Радвай се! И поклони се на царя с лице до земята и рече: Благословен да бъде Господ твоят Бог, Който предаде човеците, които подигнаха ръка против господаря ми царя.
\par 29 И царят рече: Здрав ли е младежът Авесалом? А Ахимаас отговори: Когато Иоав изпрати царския слуга, мене слугата ти, видях едно голямо смущение, но не знаех що беше.
\par 30 И рече царят: Обърни се та застани тук. И той се обърна та застана.
\par 31 И, ето, дойде Хусина. И рече Хусина: Известия, господарю мой царю! защото днес Господ въздаде за тебе на всички, които се подигнаха против тебе.
\par 32 И царят рече на Хусина: Здрав ли е младежът Авесалом? А Хусина отговори: Неприятелите на господаря ми царя, и всички, които се подигат против тебе за зло, дано станат като оня младеж!
\par 33 И царят се смути много, и възкачи се в стаята над портата та плака; и като отиваше говореше така: Сине мой Авесаломе, сине мой, сине мой Авесаломе! да бих умрял аз вместо тебе, Авесаломе, сине мой, сине мой!

\chapter{19}

\par 1 И извести се на Иоава: Ето, царят плаче и жалее Авесалома.
\par 2 И в оня ден победата се обърна в печал между целия народ; защото в оня ден людете чуха да казват: Царят бил печален за сина си.
\par 3 И през оня ден людете влизаха в града скришно, както посрамени люде, които се спотайват, когато бягат изсред сражението.
\par 4 А царят покри лицето си; и царят викаше със силен глас: Сине мой Авесаломе! Авесаломе, сине мой, сине мой!
\par 5 Тогава Иоав влезе при царя в къщата та рече: Ти посрами лицата на всичките си слуги, които опазиха днес живота ти и живота на синовете ти и на дъщерите ти, живота на жените ти и живота на наложниците ти.
\par 6 Понеже обичаш ония, които те мразят, а мразиш ония, които те обичат; защото ти показа днес, че за тебе не са нищо военачалници и слуги; защото днес познах, че ако беше останал Авесалом жив, а ние всички бяхме измрели днес, тогава щеше да ти е угодно.
\par 7 Сега, прочее, стани, излез та говори насърчително на слугите си; защото се заклевам в Господа, че ако не излезеш, няма да остане с тебе тая нощ ни един човек; а това ще бъде по-лошо за тебе от всички злини, които са те сполетели от младостта ти до сега.
\par 8 Тогава царят стана та седна в портата. И известиха на всичките люде, казвайки: Ето, царят седи в портата. И всичките люде дойдоха при царя. А Израил беше побягнал, всеки в шатъра си.
\par 9 Тогава всичките люде на всичките Израилеви племена се препираха, казвайки: Царят ни е избавил от ръката на неприятелите ни, и той ни е освободил от ръката на филистимците, а сега побягна от земята поради Авесалома;
\par 10 сега, прочее, като умря в сражението Авесалом, когото помазахме за цар над нас, защо не говорите нищо за възвръщането на царя?
\par 11 Тогава цар Давид прати до свещениците Садока и Авиатара, да им кажат: Говорете на Юдовите старейшини, като кажете: Защо сте вие последни да възвърнете царя в дома му, тъй като думите на целия Израил (според донесенията до царя) са да ги възвърнат в къщата му?
\par 12 Вие сте мои братя, вие сте моя кост и моя плът; защо, прочее, сте последни да възвърнете царя?
\par 13 А най-вече на Амаса речете: Не си ли ти моя кост и моя плът? Така да ми направи Бог, да! и повече да притури, ако не бъдеш ти винаги военачалник пред мене вместо Иоава.
\par 14 И той склони сърцата на всичките Юдови мъже, като на един човек; тъй че те пратиха до царя да му кажат : Върни се ти и всичките твои слуги.
\par 15 И тъй, царят се върна та дойде до Иордан; а Юда дойде до Галгал, за да иде да посрещне царя, да преведе царя през Иордан.
\par 16 Тогава вениаминецът Семей, Гираевият син, от Ваурим, побърза та слезе с Юдовите мъже да посрещне цар Давида.
\par 17 И с него бяха хиляда мъже от Вениамина, тоже и слугата на Сауловия дом Сива и петнадесетте му сина и двадесет негови слуги с него; и бързо преминаха Иордан да отидат при царя.
\par 18 После премина ладия, за да преведе семейството на царя и да върши каквото би му се видяло за добре. И Семей, Гираевият син, падна пред царя, когато той щеше да премине Иордан, и рече на царя:
\par 19 Нека не ми вменява господарят ми беззаконие, и нека не помни това, в което се провини слугата ти в деня, когато господарят ми царят излизаше из Ерусалим, та да го тури царят в сърцето си;
\par 20 защото аз слугата ти, познавам, че съгреших, затова, ето, дойдох днес пръв от целия Иосифов дом за да сляза и посрещна господаря си царя.
\par 21 Но Ависей Саруиният син, проговаряйки рече: Не трябва ли да бъде убит Семей, за гдето прокле Господния помазаник?
\par 22 А Давид рече: Какво има между мен и вас, Саруини синове, та да ми ставате днес противници? Бива ли да бъде убит днес човек в Израиля? защото не зная ли, че аз съм днес цар над Израиля?
\par 23 И царят рече на Семея: Няма да умреш. И царят му се закле.
\par 24 Също и Сауловият син Мемфивостей слезе да посрещне царя. Той нито нозете си беше умил, нито брадата си пригладил, нито дрехите си изпрал от деня, когато царят беше заминал, до деня когато се върна с мир.
\par 25 А когато дойде от Ерусалим да посрещне царя, царят му рече: Защо не дойде с мене, Мемфивостее?
\par 26 А той отговори: Господарю мой, царю, слугата ми ме измами; защото слугата ти рече: Ще си оседлая един осел за да се кача на него та да ида с царя, понеже слугата ти куца.
\par 27 И той е наклеветил слугата ти на господаря ми царя; но господарят ми царят е като Божий ангел; стори, прочее, коквото ти се вижда угодно.
\par 28 Защото целият ми бащин дом можеше да се определи за смърт пред господаря ми царя; но при все това, ти постави слугата си между ония, които ядяха на трапезата ти. Затова, какво право имам аз вече да се оплаквам още на царя?
\par 29 И царят му рече: Защо продължаваш да говориш за работите си? Аз казвам: Ти и Сива си разделете земите.
\par 30 А Мемфивостей рече на царя: И всичките нека вземе той, тъй като господарят ми царят се е върнал в дома си с мир.
\par 31 Тоже и галаадецът Верзелай слезе от Рогелим та премина Иордан с царя, за да го изпрати оттатък Иордан.
\par 32 А Верзелай беше много стар, осемдесет години на възраст; и беше прехранвал царя, когато седеше в Маханаим, защото беше много богат човек.
\par 33 И царят рече на Верзелая: Премини с мене, и ще те издържам при себе си в Ерусалим.
\par 34 А Верзелай каза на царя: Колко е числото на годините на живота ми та да отида с царя в Ерусалим?
\par 35 Днес съм осемдесет години на възраст. Мога ли да разчитам между добро и лошо? може ли слугата ти да усеща що яде или що пие? мога ли да чуя вече гласа на певците или на певиците? Защо, прочее, да бъде слугата ти още товар на господаря ми царя?
\par 36 Слугата ти е помислил да премине Иордан с царя само до малко разстояние; а защо царят да ми даде затова едно такова възнаграждение?
\par 37 Нека се върне слугата ти, моля, за да умра в града си при гроба на баща си и майка си; но, ето, слугата ти Хамаам, той нека премине с господаря ми царя; и стори с него както ти се вижда угодно.
\par 38 И царят рече: Хамаам ще премине с мене; и аз ще му сторя каквото на тебе се вижда угодно; па и за тебе ще сторя всичко, каквото поискаш от мене.
\par 39 И така всичките люде преминаха Иордан. И когато беше преминал царят, царят целуна Верзелая и го благослови; и той се върна на мястото си.
\par 40 Царят, прочее, продължи пътя си до Галгал, и Хамаам замина с него; и всичките Юдови люде, както и половината от Израилевите люде, преведоха царя.
\par 41 И, ето, всичките Израилеви мъже дойдоха при царя та рекоха на царя: Защо те откраднаха братята ни Юдовите мъже, та преведоха царя и семейството му през Иордан и всичките Давидови мъже с него?
\par 42 А всичките Юдови мъже отговориха на Израилевите мъже: Защото царят е наш сродник; и защо се сърдите за това нещо? дали изядохме нещо от царя? или даде ли ни той някакъв дар?
\par 43 А Израилевите мъже, в отговор на Юдовите мъже, рекоха: Ние имаме десет части в царя, и даже имаме по-голямо право на Давида от вас; защо, прочее, ни презряхте? и не говорихме ли ние първи да възвърнем царя си? Но думите на Юдовите мъже бяха по-остри от думите на Израилевите мъже.

\chapter{20}

\par 1 А случи се да има там един лош човек на име Семей, Вихриевия син, вениаминец; и той засвири с тръбата и рече: Ние нямаме дял в Давида, нито имаме наследство в Иесеевия син! в шатрите си, Израилю, всеки човек!
\par 2 И тъй, всичките Израилеви мъже се оттеглиха от Давида и последваха Савея Вихриевия син; а Юдовите мъже останаха привързани към царя си, от Иордан до Ерусалим.
\par 3 И Давид дойде у дома си в Ерусалим. И царят взе десетте си наложници, които бе оставил да пазят къщата, та ги тури в една къща под стража и хранеше ги, но не влизаше при тях; и те останаха затворени до деня на смъртта си, живеещи като вдовици.
\par 4 Тогава царят рече на Амаса: Събери ми Юдовите мъже в три дена, и тогава ти да се явиш тук.
\par 5 И тъй, Амаса отиде да събере Юда; забави се, обаче повече от назначеното време, което бе му определил.
\par 6 Затова Давид каза на Ависея: Сега, Савей Вихриевият син, ще ни стори по-голяма пакост от Авесалома. Вземи ти слугите на господаря си та го преследвай, да не би да си намери укрепени градове и избегне от очите ни.
\par 7 Излязоха, прочее, подир него Иоавовите мъже, и херетците, фелетците и всички силни мъже излязоха от Ерусалим за да преследват Савея Вихриевия син.
\par 8 Когато стигнаха до голямата скала в Гаваон, Амаса дойде насреща им. А Иоав носеше препасана дрехата, с която беше облечен, а върху нея меч в ножницата му, вързан около кръста му с пояс; и като излезе той към него , мечът падна.
\par 9 И рече Иоав на Амаса: Здрав ли си, брате мой? И Иоав хвана Амаса за брадата с дясната си ръка за да го целуне.
\par 10 А Амаса не се предпази от меча, който беше в другата ръка на Иоава; и така Иоав го удари с него в корема, и изля червата му, на земята без да го удари втори път; и той умря. Тогава Иоав и брат му Ависей продължаваха да преследват Савея Вихриевия син.
\par 11 А един от Иоавовите момци застана при Амаса и казваше: Който благоприятствува на Иоава, и който е за Давида, нека върви подир Иоава.
\par 12 А Амаса лежеше овалян в кръвта си всред пътя. И когато видя тоя човек, че всички люде се спираха, отвлече Амаса от пътя в нивата; и понеже видя, че всеки, който идеше при него, се спираше, хвърли върху него една дреха.
\par 13 И като беше преместен от пътя, всички люде отидоха подир Иоава за да преследват Савея Вихриевия син.
\par 14 А Савей мина през всичките Израилеви племена до Авел и до Вет-мааха; и всичките отборни момци, те също се събраха заедно та го последваха.
\par 15 Тогава дойдоха та обсадиха в Авел на Вет-мааха, и издигнаха могила против града, като я поставиха срещу вала; и всичките люде, които бяха с Иоава, удряха стената със стеноломи за да я съборят.
\par 16 Тогава една благоразумна жена извика от града: Слушайте, слушайте! Моля, кажете на Иоава: Приближи се тук, за да ти поговоря.
\par 17 И когато се приближи до нея, жената рече: Ти ли си Иоав? А той отговори: Аз. Тогава ме рече: Слушай думите на слугинята си. А той отговори: Слушам.
\par 18 И тя продума, казвайки: В старо време имаха обичай да говорят, казвайки: Нека се допитат до Авел, и така да решат работата .
\par 19 Аз съм от мирните и верните на Израиля; ти искаш да съсипеш град и даже столица в Израил. Защо искаш да погълнеш Господното наследство?
\par 20 А Иоав в отговор каза: Далеч от мене, далеч от мене да погълна или да съсипя!
\par 21 Работата не е така; но един мъж от хълмистата земя на Ефрема, на име Савей Вихриевият син, е подигнал ръка против царя, против Давида; предайте само него, и ще си отида от града. И жената рече на Иоава: Ето, главата му ще ти се хвърли през стената.
\par 22 Тогава жената отиде при всичките люде та им говори с мъдростта си. И те отсякоха главата на Савея, Вихриевия син, та я хвърлиха на Иоава. Тогава той засвири с тръбата, и людете се оттеглиха от града, всеки в шатъра си. И Иоав се върна при царя в Ерусалим.
\par 23 И Иоав беше над цялата Израилева войска; а Венаия Иодаевият син, над херетците и над фелетците.
\par 24 Адорам беше над данъка; Иосафат Ахилудовият син, летописец;
\par 25 Сева, секретар; а Садок и Авиатар, свещеници;
\par 26 също и яирецът Ирас беше първенец при Давида.

\chapter{21}

\par 1 В Давидовите дни стана глад три години наред; и когато Давид се допита до Господа за причината , Господ каза: Поради Саула е и поради кръвожадния му дом, гдето изби гаваонците.
\par 2 Тогава царят повика гаваонците та им рече: - (а гаваонците не бяха от израилтяните, но от останалите аморейци; и израилтяните бяха им се заклели да ги оставят да живеят , а Саул от ревност към израилтяните и юдейците беше поискал да ги избие) -
\par 3 Давид, прочее, рече на гаваонците: Що да ви сторя? и с какво да направя умилостивение, за да благословите Господното наследство?
\par 4 А гаваонците му казаха: Не е въпрос за сребро или злато между нас и Саула, или неговия дом, нито е наша работа да убием човек в Израиля. И той рече: Каквото кажете ще ви сторя.
\par 5 И те рекоха на царя: От синовете на човека, който ни е погубвал, и който е изхитрувал против нас с цел да бъдем изтребени, така щото да не оставаме в никой от Израилевите предели,
\par 6 от тях нека ни се дадат седем човека, и ще ги обесим пред Господа в Гавая, града на Саула, Господния избраник. И рече царят: Ще ги предам.
\par 7 Обаче, царят пожали Мемфивостея, син на Ионатана Сауловия син, поради Господната клетва помежду им, между Давида и Ионатана, Сауловия син.
\par 8 Но царят взе Армония и Мемфивостея, двамата сина на Ресфа, дъщерята на Аия, когого беше родила на Саула, и петимата сина на Михала Сауловата дъщеря, които беше родила на Адриила, син на меолатянина Верзелай,
\par 9 та ги предаде в ръцете на гаваонците; и те ги обесиха на бърдото пред Господа. И седмината паднаха заедно, като бяха погубени в първите дни на жътвата, в началото на жътвата на ечемика.
\par 10 Тогава Ресфа, дъщерята на Аия, взе вретище та си го постла на канарата, и от началото на жътвата до като падна на тях дъжд от небето не оставаше въздушните птици да се допрат до тях денем, нито полските зверове нощем.
\par 11 И извести се на Давида онова, което стори Ресфа, дъщерята на Аия, Сауловата наложница.
\par 12 Тогава Давид отиде та взе костите на Саула и костите на сина му Ионатана от мъжете на Явис-галаад, които го бяха грабнали от улицата на Ветасан, гдето ги бяха обесили филистимците в деня, когато филистимците убиха Саула в Гелвуе;
\par 13 и изнесе от там костите на Саула и костите на сина му Ионатана; събраха и костите на обесените.
\par 14 И погребаха костите на Саула и на сина му Ионатана в Сила у Вениаминовата земя, в гроба на баща му Кис; и изпълниха всичко, що заповяда царят. След това Бог се умилостиви към земята.
\par 15 А филистимците пак воюваха против Израиля; и Давид и слугите му с него слязоха та се биха против филистимците; и умори се Давид.
\par 16 А Исви-венов, който беше от синовете на исполина, чието копие тежеше триста медни сикли , и който беше опасан с нов меч , се надяваше да убие Давида.
\par 17 Ависей обаче, Саруиният син, му помогна та порази филистимците и го уби. Тогава Давидовите мъже му се заклеха, като рекоха: Ти няма вече да излезеш с нас на бой, да не би да изгасиш светилото на Израиля.
\par 18 След това, настана пак война с филистимците в Гов, когато хусатецът Сивехай уби Сафа, който беше от синовете на исполина.
\par 19 И пак настана война с филистимците Гов, когато Елханин, син на витлеемеца Яреорегим уби брата на гетеца Голиат, на чието копие дръжката бе като кросно на тъкач.
\par 20 Настана пак война в Гет гдето имаше един високоснажен мъж с по шест пръста на ръцете си и по шест пръста на нозете си, двадесет и четири на брой; също и той бе се родил на исполина.
\par 21 а когато хвърли презрение върху Израиля, Ионатан, син на Давидовия брат Самай, го уби.
\par 22 Тия четирима бяха се родили на исполина в Гет; и паднаха чрез ръката на Давида и чрез ръката на слугите му.

\chapter{22}

\par 1 Тогава Давид изговори Господу думите на тая песен, в деня, когато Господ го беше избавил от ръката на всичките му неприятели и от ръката на Саула;
\par 2 и рече: - Господ е скала моя, крепост моя, и Избавител мой;
\par 3 Бог е канара моя, на Когото се надявам, Щит мой, и рога на избавлението ми; Висока моя кула е, и прибежище ми е, Спасител мой е; Ти ме избавяш от насилие.
\par 4 Ще призова Господа, Който е достохвален; Така ще бъда избавен от неприятелите си.
\par 5 Защото вълните на смъртта ме окръжиха, Порои от беззаконие ме уплашиха;
\par 6 Връзките на ада ме обвиха, Примките на смъртта ме стигнаха
\par 7 В утеснението си призовах Господа, И към Бога мой викнах; И от храма Си Той чу гласа ми, И викането ми стигна в ушите Му.
\par 8 Тогаз са поклати и потресе земята; Основите на небето се разлюляха И поклатиха се, защото се разгневи Той.
\par 9 Дим се издигаше из ноздрите Му, И огън из устата Му поглъщаше; Въглища се разпалиха от Него.
\par 10 Той сведе небето и слезе, И мрак бе под нозете Му.
\par 11 Възседна на херувими и летя, И яви се на ветрени крила.
\par 12 Положи за скиния около Си тъмнината. Събраните води, гъстите въздушни облаци.
\par 13 От святкането пред Него Огнени въглища са разпалиха.
\par 14 Гръмна Господ от небето, Всевишният даде гласа Си;
\par 15 И прати стрели та ги разпръсна, Светкавици та ги смути.
\par 16 Тогава се явиха морските дълбочини, Откриха са основите на света От изобличението на Господа, От духането на духа на ноздрите Му.
\par 17 Прати от височината, взе ме, Извлече ме из големи води;
\par 18 Избави ме от силния ми неприятел, От ония, които ме мразеха, Защото бяха по-силни от мене.
\par 19 Стигнаха ме в деня на бедствието ми; Но Господ ми стана подпорка.
\par 20 И извади ме на широко, Избави ме, защото има благоволение към мене.
\par 21 Въздаде ми Господ според правдата ми; Според чистотата на ръцете ми възнагради ме.
\par 22 Защото съм опазил пътищата Господни, И не съм се отклонил от Бога мой в нечестие.
\par 23 Защото всичките Му съдби са били пред мене; И от повеленията Му не съм се отдалечил.
\par 24 Непорочен бях пред Него, И опазих се от беззаконието си.
\par 25 Затова ми въздаде Господ според правдата ми, Според чистотата ми пред очите Му.
\par 26 Към милостивия, Господи , милостив ще се явиш, Към непорочния, непорочен ще се явиш,
\par 27 Към чистия, чист ще се явиш, А към развратния противен ще се явиш,
\par 28 Оскърбени люде ти ще спасиш; А над горделивите с очите Ти за да ги смириш.
\par 29 Защото Ти, Господи, си светилник мой; И Господ ще озари тъмнината ми.
\par 30 Защото чрез Тебе разбивам полк; Чрез Бога мой прескачам стена.
\par 31 Колкото за Бога, Неговият път е съвършен; Словото на Господа е опитано; Той е щит на всички, които уповават на Него.
\par 32 Защото кой е бог освен Господа? И кой е канара освен нашият Бог?
\par 33 Бог е силната моя крепост, И прави съвършен пътя ми;
\par 34 Прави нозете ми като нозете на елените. И поставя ме на високите ми места;
\par 35 Учи ръцете ми да воюват, Така щото мишците ми запъват меден лък.
\par 36 Ти си ми дал и щита на избавлението Си; И Твоята благост ме е направила велик.
\par 37 Ти си разширил стъпките ми под мене; И нозете ми не се подхлъзнаха.
\par 38 Гоних неприятелите си и ги изтребих, И не се върнах докато не ги довърших.
\par 39 Довърших ги, стрих ги, та не можаха да се подигнат, А паднаха под нозете ми.
\par 40 Защото си ме препасал със сила за бой; Повалил си под мене въставащите против мене.
\par 41 Сторил си на обърнат гръб към мене неприятелите ми, За да изтребя ония, които ме мразят.
\par 42 Погледнаха, но нямаше избавител, - Към Господа, но не ги послуша,
\par 43 Тогава ти стрих като земния прах, Сгазих ги, както калта на пътищата, и стъпках ги;
\par 44 Ти си ме избавил и от съпротивленията на людете ми, Поставил си ме глава на народите; Люде, които не познавах, слугуват ми.
\par 45 Чужденците ми се покориха; Щом чуха за мене, те ме и послушаха.
\par 46 Чужденците ослабнаха, И разтреперани излязоха из местата, гдето са се затворили.
\par 47 Жив е Господ, И благословена да бъде Канарата ми; И да се възвиси Бог, моята спасителна скала,
\par 48 Бог, Който отмъщава за мене, И покорява племена под мене,
\par 49 И Който ме извежда изсред неприятелите ми; Да! възвишаваш ме над въставащите против мене; Избавяш ме от насилника.
\par 50 Затова, ще Те хваля, Господи, между народите, И на името Ти ще пея.
\par 51 Ти си, Който даваш велико избавление на царя Си, И показваш милосърдие към помазаника Си. Към Давида и към потомството му да века.

\chapter{23}

\par 1 А тия са последните Давидови думи: - Давид Есеевият син рече: Мъжът, който бе издигнат на високо, Помазаникът на Бога Яковов И сладкият Израилев псалмопевец, рече:
\par 2 Духът на Господа говори чрез мене, И словото Му дойде на езика ми.
\par 3 Бог Израилев рече, Скалата Израилева ми говори: Оня, който владее над човеци нека бъде праведен, Нека бъде един , който владее със страх от Бога;
\par 4 И ще бъде като утринна виделина, Когато изгрява слънцето В безоблачна зора, Като зеле що никне из земята От сиянието, което блясва след дъжд.
\par 5 Ако домът ми и да не е такъв пред Бога, Пак Той е направил с мене завет вечен. Нареден във всички и твърд, Който е всичкото ми спасение и всичкото ми желание; И не ще ли го направи да процъфти?
\par 6 А всичките беззаконни ще бъдат като тръни, които се изхвърлят, Защото с ръце не се хващат.
\par 7 А който се допре до тях Трябва да е въоръжен с желязо и с дръжка на копие; И ще бъдат изгорени с огън на самото си място.
\par 8 Ето имената на силните мъже, които имаше Давид: техмонецът Иосев-весевет, главен военачалник, който уби в една битка осемстотин души.
\par 9 След него беше Елеазар, син на Додо, син на един ахохиец, един от тримата силни мъже с Давида; когато Израилевите мъже се оттеглиха, след като бяха се заканили на събраните там на бой филистимци,
\par 10 той стана та поразяваше филистимците, докато изнемощя ръката му и се залепи ръката му за ножа; така щото в оня ден Господ извърши голямо избавление, и людете се върнаха само за да съберат користи подир него.
\par 11 И след него беше Сама, син на арареца Агей; когато филистимците се бяха събрали в Лехий, гдето имаше частица земя пълна с леща, и людете побягнаха от филистимците,
\par 12 той застана всред нивата та я защити и порази филистимците; и Господ извърши голямо Избавление.
\par 13 А още трима от тридесетте военачалници слязоха та дойдоха при Давида при Одоломската пещера в жетвено време; и филистимският стан бе разположен в Рафаимската долина.
\par 14 И като беше Давид в това време в канарата, а филистимският гарнизон бе тогава във Витлеем,
\par 15 и Давид пожелавайки рече: Кой би ми дал да пия вода от витлеемския кладенец, който е при портата?
\par 16 тия трима силни мъже пробиха филистимския стан та наляха вода от витлеемския кладенец, който е при портата, и взеха та донесоха на Давида. Но той отказа да я пие, а я възля Господу, като рече:
\par 17 Далеч да бъде от мене, Господи, да сторя аз това! Да пия ли кръвта на мъжете, които ходиха с опасност за живота си? Затова отказа да я пие. Това сториха тия трима силни мъже.
\par 18 А Иоавовият брат Ависея, Серуиният син, беше първият от тримата: той, като махаше копието си против триста души неприятели , уби ги, и си придоби име между тримата.
\par 19 Не беше ли той най-славният от тримата? затова из стана началник; обаче, не стигна до първите трима.
\par 20 И Ванаия, син на Иодая, син на един храбър мъж от Кавсеил, който беше извършил храбри дела, уби двамата лъвовидни моавски мъже; той слезе та уби лъва всред рова в многоснежния ден;
\par 21 при това той уби египтянина, едрия мъж, египтянинът който държеше в ръката си копие; но той слезе при него само с тояга, и като грабна копието от ръката на египтянина, уби го със собственото му копие.
\par 22 Тия неща стори Венаия Иодаевият син, и си придоби име между тия трима силни мъже.
\par 23 По-славен бе от тридесетте, но не достигна до първите трима. И Давид го постави над телохранителите си.
\par 24 Асаил, Иоавовият брат, беше между тридесетте; също бяха и Елханан, син на Додо от Витлеем;
\par 25 Сама ародецът; Елика ародецът:
\par 26 Хелис фалтянинът; Ирас, син на текоеца Екис;
\par 27 Авиезер анатонецът; Мевунай кусатецът;
\par 28 Салмон ахохиецът; Маарай нетофатецът;
\par 29 Хелев, син на нетофатеца Ваана; Итай, син на Ривай от Гавая на вениаминците;
\par 30 Ванаия пиратонецът; Идай от долината на Гаас;
\par 31 Ави-алвон арветецът; Азмавет варумецът;
\par 32 Елиава саалвонецът; Ионатан от Ясиновите синове;
\par 33 Сама арорецът; Ахаим, син на маахатец; Елиам, син на гилонеца Ахитофел;
\par 34 Елифелет, син на Аасве, син на маахатец; Елиам, син на гилонеца Ахитофел;
\par 35 Есрай кармилецът; Фаарай арвиецът;
\par 36 Игал, син на Натана от Сова; Ваний гадецът;
\par 37 Силек амонецът; оръженосец на Иоава Саруиния син; Наарай виротецът;
\par 38 Ирас етерецът; Гарив етерецът;
\par 39 Урия хетеецът; всичко, тридесет и седем души.

\chapter{24}

\par 1 Подир това гневът на Господа пак пламна против Израиля, и Той подбуди Давида против тях, казвайки: Иди, преброй Израиля и Юда.
\par 2 Царят, прочее каза на началника на войската Иоав, който бе с него: Мини сега през всичките Израилеви племена, от Дан до Вирсавее, та преброй людете, за да узная броя на людете.
\par 3 А Иоав каза на царя: Господ твоят Бог дано притури на людете стократно повече отколкото са, и очите на господаря ми царя дано видят това; но защо господарят ми царят намира наслада в това нещо?
\par 4 Обаче царската дума надделя над Иоава и над началниците на войската излязоха от царя за да преброят Израилевите люде.
\par 5 Като преминаха Иордан, разположиха се при Ароир, към Гад и към Язир, отдясно на града, който е всред долината.
\par 6 После дойдоха в Галаад и в земята Тахтим-одси; дойдоха и в Дан-яан и наоколо до Сидон;
\par 7 тогава дойдоха в Тирската крепост и във всичките градове на евейците и на ханаанците; и излязоха в Вирсавее в южна Юда.
\par 8 И тъй, като преминаха през цялата земя, дойдоха в Ерусалим в края на девет месеца и двадесет дни.
\par 9 И Иоав доложи на царя броя на преброените люде; те бяха: от Израиля, осемстотин хиляди силни мъже, които можеха да теглят нож; и от Юдовите мъже, петстотин хиляди.
\par 10 И след това като преброи Давид людете, сърцето му го изобличи. И Давид рече на Господа: Съгреших тежко като извърших това нещо; но сега, моля Ти се, Господи, отмахни беззаконието на слугата Си, защото направих голяма глупост.
\par 11 И когато стана Давид на утринта, Господното слово дойде към пророка Гад, Давидовия гадач, казвайки:
\par 12 Иди, кажи на Давида, Така казва Господ: Три неща ти предлагам; избери си едно от тях, за да го извърша над тебе.
\par 13 Дойде, прочее, Гад при Давида та му извести това ; после му рече: Дали за седем години да има върху тебе глад по земята ти? или три месеца да бягаш от неприятелите си, като те преследват? или три дни да има мор в земята ти? Размисли сега и виж какъв отговор да възвърна на Оногова, Който ме е пратил.
\par 14 И Давид рече на Гада: Намирам се много на тясно; обаче нека паднем в ръката на Господа, защото Неговите милости са много; но в ръката на човека да се изпадна.
\par 15 И тъй, Господ прати мор върху Израиля от оная утрин до определеното време; и измряха от людете, от Дан до Вирсавее, седемдесет хиляди мъже.
\par 16 А когато ангелът простря ръката си към Ерусалим, за да го погуби, Господ се разкая за злото, и рече на ангела, който погубваше людете: Стига вече, оттегли сега ръката си. И ангелът Господен бе близо до гумното на евусеца Орна.
\par 17 И когато видя ангела, който поразяваше людете, Давид проговори Господу, казвайки: Ето, аз съгреших, аз извърших беззаконие; но тия овци що са сторили? Над мене, моля Ти се, нека бъде ръката Ти, и над моя бащин дом.
\par 18 В същия ден Гад дойде при Давида та му рече: Възлез, издигни олтар Господу на гумното на евусеца Орна.
\par 19 И така, Давид възлезе според Гадовата дума, както заповяда Господ.
\par 20 И Орна, като погледна, видя, че царят и слугите му идат към него; и Орна излезе та се поклони на царя с лице до земята.
\par 21 Тогава рече Орна: Защо е дошъл господарят ми царят при слугата си? А Давид каза: Да купя гумното от тебе та да издигна олтар Господу, за да престане язвата между людете.
\par 22 А Орна рече на Давида: Господарят ми царят нека вземе и принесе в жертва каквото му се вижда за добре; ето воловете за всеизгаряне, и диканите и оръдията на воловете за дърва;
\par 23 всички тези, о царю, дава Орна на царя. Орна каза още на царя: Господ твоят Бог да има благоволение към тебе.
\par 24 А царят рече на Орна: Не, но непременно ще го купя от тебе за определена цена; защото не ща да принеса на Господа моя Бог всеизгаряния, за които не съм похарчил. И тъй, Давид купи гумното; а воловете купи за петдесет сребърни сикли.
\par 25 И там Давид издигна олтар Господу, и принесе всеизгаряния и примирителни приноси. И Господ прие молбата за земята, та язвата престана между Израиля.

\end{document}