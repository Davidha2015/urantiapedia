\begin{document}

\title{1 Kings}


\chapter{1}

\par 1 А когато цар Давид бе остарял, в напреднала възраст, при все, че го покриваха с дрехи, не се стоплюваше.
\par 2 Затова, слугите му казаха: Нека потърсят за господаря ни царя млада девица за да престоява пред царя и да му пригажда, и да спи на пазвата му, за да се топли господарят ни царят.
\par 3 И тъй, потърсиха по всичките Израилеви предели красива девица, и намериха сунамката Ависага, и доведоха я при царя.
\par 4 Девицата бе много красива, и пригаждаше на царя и му слугуваше; но царят не я позна.
\par 5 Тогава Адония, Агитиният син, надуто си каза: Аз ще царувам. И приготви си колесници и конници и петдесет мъже, които да тичат пред него.
\par 6 А баща му никога не беше му досаждал с думите: Ти защо правиш това? А той беше и много хубав на глед; и роди се на Давида подир Авесалома.
\par 7 Той се сговори с Иоава Саруиния син и със свещеника Авиатара, и те последваха Адония и му помагаха.
\par 8 Но свещеник Садок, Ванаия Иодаевият син, пророк Натан, Семей и Реий и Давидовите силни мъже не бяха с Адония.
\par 9 И Адония закла овци, говеда и угоени телци при скалата на Зоелет, която е при извора Рогих, и покани всичките си братя царските синове, и всичките Юдови мъже царските слуги.
\par 10 Обаче пророк Натана, Ванаия и силните мъже и брата си Соломона не покани.
\par 11 Тогава Натан говори на Соломоновата майка Витсавее, казвайки: Не си ли чула, че се е възцарил Адония Агитиният син, а господарят ни Давид не знае това?
\par 12 Сега, прочее, ела, моля, да те съветвам, за да избавиш своя живот и живота на сина си Соломона.
\par 13 Иди, влез при цар Давида и кажи му: Господарю мой царю, не закле ли се ти на слугинята си, казвайки: Синът ти Соломон непременно ще царува подир мене, и той ще седи на престола ми? Защо, прочее, се възцари Адония?
\par 14 И ето, докато ти още говориш там с царя, ще дойда и аз подир тебе и ще потвърдя думите ти.
\par 15 И така, Витсавее възлезе при царя, в спалнята, като беше царят много стар, и сунамката Ависага му слугуваше.
\par 16 И Витсавее се наведе та се поклони на царя. И царят рече: Какво желаеш?
\par 17 А тя му каза: Господарю мой, ти се закле в Господа твоя Бог на слугинята си и каза : Твоят син Соломон непременно ще царува подир мене, и той ще седи на престола ми.
\par 18 Но сега, ето, Адония се е възцарил; а ти, господарю мой царю, не знаеш това.
\par 19 И той е заклал говеда, угоени телци и овци в изобилие, и е поканил всичките царски синове, и свещеника Авиатара, и военачалника Иоава; обаче, слугата ти Соломона не е поканил.
\par 20 Но към тебе, господарю мой царю, към тебе са обърнати очите на целия Израил, за да им известиш кой ще седне на престола на господаря ми царя подир него.
\par 21 Иначе, когато господарят ми царят заспа с бащите си, аз и синът ми Соломон ще се считаме за оскърбители.
\par 22 А докато тя още говореше с царя, ето, дойде и пророк Натан.
\par 23 И известявайки на царя, рекоха: Ето пророк Натан. И той, като влезе пред царя, поклони му се с лице до земята.
\par 24 И Натан каза: Господарю мой царю, рекъл ли си ти, Адония ще царува подир мене, и той ще седи на престола ми?
\par 25 Защото той слезе днес та закла говеда, угоени телци , и овци в изобилие, и покани всичките царски синове, и военачалниците, и свещеника Авиатара; и, ето, ядат и пият пред него, и думат: Да живее цар Адония!
\par 26 А мене, мене слугата ти, и свещеника Садока, и Ванаия Иодаевия син, и слугата ти Соломона не покани.
\par 27 От господаря ми царя ли стана това нещо, без да си обявил на слугите си, кой ще седне на престола на господаря ми царя подир него?
\par 28 Тогава цар Давид в отговор рече: Повикайте ми Витсавее. И тя влезе при царя и застана пред него.
\par 29 И царят се закле, казвайки: В името на живия Господ, Който е избавил душата ми от всяко бедствие,
\par 30 непременно, както ти се заклех в Господа Израилевия Бог и рекох, че синът ти Соломон ще царува подир мене, и че той ще седи вместо мене на престола ми, непременно така ще направя днес.
\par 31 Тогава Витсавее се наведе с лице до земята та се поклони на царя, и каза: Да живее господарят ми цар Давид до века!
\par 32 Тогава рече цар Давид: Повикайте ми свещеника Садока, пророка Натана и Ванаия Иодаевия син. И те дойдоха пред царя.
\par 33 И царят им рече: Вземете със себе си слугите на вашия господар, качете сина ми Соломона на моята мъска, и заведете го долу в Гион;
\par 34 и свещеник Садок и порок Натан нека го помажат там за цар над Израиля; и засвирете с тръба и кажете: Да живее цар Соломон!
\par 35 Тогава качете се тук подир него, и нека дойде и седне на престола ми, защото той ще царува вместо мене, и аз го назначих да бъде вожд на Израиля и на Юда.
\par 36 А Ванаия Иодаевият син, в отговор на царя, рече: Амин! така да повели и Господ Бог на господаря ми царя!
\par 37 Както е бил Господ с господаря ми царя, така да бъде и със Соломона, и да направи престола му още по-велик от престола на господаря ми цар Давида.
\par 38 И така, свещеник Садок пророк Натан и Ванаия Иодаевият син, с херетците и фелетците, слязоха та качиха Соломона на цар Давидовата мъска и доведоха го в Гион.
\par 39 И свещеник Садок взе рога с мирото от шатъра, за срещане та помаза Соломона. И засвириха с тръба, и всичките люде рекоха: Да живее цар Соломон!
\par 40 И всичките люде отидоха нагоре подир него; и людете свиреха с кавали и веселяха се с голямо веселие, така че земята се цепеше от виковете им.
\par 41 А Адония и всичките му гости чуха шума , като свършиха яденето си. И когато Иоав чу тръбният звук рече: Защо е тоя шум и градът развълнуван?
\par 42 Докато още говореше, ето, Ионатан, син на свещеника Авиатара дойде; и Адония рече: Влез, защото ти си достоен мъж и носиш дори известия.
\par 43 А Ионатан в отговор рече на Адония: Наистина господарят ни цар Давид направи Соломона цар;
\par 44 и царят прати с него свещеника Садока, пророка Натана и Ванаия Иодаевия син, с херетците и фелетците, та го възкачиха на царевата мъска;
\par 45 И свещеникът Садок и пророк Натан го помазаха за цар в Гион; и от там отидоха нагоре веселящи се, така щото градът екна. Това е шумът, който сте чули.
\par 46 При това, Соломон седна на престола на царството.
\par 47 Освен туй, и царските слуги влязоха да честитят на господаря ни цар Давида, и рекоха: Бог да направи името на Соломона по-светло и от твоето име, и да направи престола му по-велик и от твоя престол. И царят се поклони както беше на леглото си.
\par 48 Царят още говори така: Благословен да бъде Господ Израилевият Бог, Който ми даде днес син да седи на престола ми, докато очите ми го гледат.
\par 49 Тогава всичките гости на Адония се уплашиха, и ставайки отидоха всеки в пътя си.
\par 50 А Адония, понеже се уплаши от Соломона, стана и отиде да се хване за роговете на олтара.
\par 51 И известиха на Соломона, казвайки: Ето, Адония се бои от Цар Соломона, и, ето, хванал се е за роговете на олтара и казва: Нека ми се закълне днес цар Соломон, че няма да убие слугата си с меч.
\par 52 И рече Соломон: Ако се покаже достоен мъж, ни един от космите му няма да падне на земята; но ако се намери зло в него, ще се умъртви.
\par 53 И тъй, цар Соломон прати, та го свалиха от олтара; и той дойде да се поклони на цар Соломона. А Соломон му рече: Иди у дома си.

\chapter{2}

\par 1 И когато наближи на Давида времето да умре, заръча на сина си Соломона, казвайки:
\par 2 Аз отивам в пътя на целия свят; ти, прочее, се крепи и бъди мъжествен.
\par 3 Пази заръчванията на Господа твоя Бог, ходи в пътищата Му и пази повеленията Му, заповедите Му, съдбите Му и заявленията Му, според както е писано в Моисеевия закон, за да успяваш във всичко, каквото правиш и на където и да се обръщаш;
\par 4 За да утвърди Господ думата, с който е говорил за мене, като е казал: Ако внимават чадата ти в пътя си, да ходят пред Мене в истина, от цялото си сърце и от цялата си душа, няма да липсва от тебе мъж върху Израилевия престол.
\par 5 Освен това, ти знаеш какво и стори Иоав Саруиният син, и какво направи на двамата военачалници на Израилевите войски, - на Авенира Нировия син и на Амаса Етеровия син, - който уби, като проля боева кръв в мирно време и обагри с боева кръв пояса, който бе около кръста му, и обущата, които бяха на нозете му.
\par 6 Постъпвай, прочее, според мъдростта си, и не оставяй белите му коси да слязат с мир в гроба.
\par 7 Но покажи благост към синовете на галаадеца Верзелай, и нека бъдат между ония, които ядат на трапезата ти; защото така с храна те дойдоха при мене, когато бягах от брата ти Авесалома.
\par 8 И ето с тебе вениаминецът Семей Гираевият син, от Ваурим, който ме прокле с горчива клетва в деня, когато отидох в Маханаим; слезе обаче да ме посрещне при Иордан, и заклех му се в Господа, като рекох: Няма да те убия с меч.
\par 9 Сега, прочее, да го не считаш за невинен; защото си мъдър човек и ще знаеш, що трябва да му направиш за да сведеш белите му коси с кръв в гроба.
\par 10 И така, Давид заспа с бащите си, и биде погребан в Давидовия град.
\par 11 А времето, през което Давид царува над Израиля, беше четиридесет години: седем години царува в Хеврон, и тридесет и три години царува в Ерусалим.
\par 12 И Соломон седна на престола на баща си Давида; и царството му се закрепи твърдо.
\par 13 Тогава Адония, Агитиният син, дойде при Витсавее, Соломоновата майка. А тя каза: С мир ли идеш? И рече: С мир.
\par 14 После рече: Имам да ти кажа една дума. И тя рече: Казвай.
\par 15 Тогава той каза: Ти знаеш, че на мене принадлежеше царството, и че към мене целият Израил беше обърнал лицето си с ожидане за да царувам; обаче царството се отклони, та се падна на брата ми, защото от Господа му дойде.
\par 16 Сега, прочее, имам една просба към тебе: не ми я отричай. И тя му рече: Казвай.
\par 17 И рече: Кажи, моля, на цар Соломона (защото не ще да ти откаже това ) да ми даде за жена сунамката Ависага.
\par 18 А Витсавее рече: Добре; аз ще говоря за тебе на царя.
\par 19 И така, Витсавее влезе при цар Соломона да му говори за Адония. И царят стана да я посрещне, и поклони й се; сетне, като седна на престола си, каза да положат престол и за царевата майка; и тя седне отдясно му.
\par 20 Тогава тя каза: Една малка просба имам към тебе; на ми я отричай. А царят й рече: Прости, майко моя, защото няма да ти откажа.
\par 21 И тя рече: Нека се даде сунамката Ависага на брата ти Адония за жена.
\par 22 Но цар Соломон в отговор рече на майка си: А защо искаш сунамката Ависага за Адония? искай за него и царството, (защото ми е по-голям брат); да! за него, за свещеника Авиатара и за Иоава Саруиния син.
\par 23 Тогава цар Соломон се закле в Господа, казвайки: Така да ми направи Бог, да! и повече да притури, ако Адония не е изговорил тия думи против живота си.
\par 24 Сега, прочее, заклевам се в живота на Господа, Който ме утвърди и ме тури да седна на престола на баща ми Давида, и Който ми направи дом, според както се е обещал, днес непременно Адония ще бъде умъртвен.
\par 25 И тъй, цар Соломон прати да изпълни заповедта му Ванаия Иодаевият син, който го нападна и той умря.
\par 26 А на свещеника Авиатара царят каза: Иди в Анатот, на нивите си, защото заслужаваш смърт; но няма сега да те умъртвя, понеже ти си носил ковчега на Господа Иеова пред баща ми Давида, и понеже си страдал всичко, което е страдал и баща ми.
\par 27 Така Соломон отблъсна Авиатара да не бъде свещеник Господу, за да се изпълни словото, което Господ бе говорил в Сило за Илиевия дом.
\par 28 И когато за това стигна слух до Иоава, (защото Иоав клонеше след Адония, ако и да не беше клонил след Авесалома), Иоав побягна в Господния шатър и се хвана за роговете на олтара.
\par 29 И извести се на цар Соломона: Иоав побягна в Господния шатър, и ето, той е при олтара. Тогава Соломон прати Ванаия Иодевият син, като каза: Иди, нападни го.
\par 30 И тъй, Ванаия дойде в Господния шатър та му рече: Така казва царят: Излез. А той рече: Не, но тук ще умра. И Ванаия извести на царя, казвайки: Така каза Иоав, о такъв отговор ми даде.
\par 31 А царят му рече: Направи според както е рекъл; нападни го и погреби го, за да изгладиш от мене и от бащиния ми дом невинната кръв, която Иоав е пролял.
\par 32 Господ ще възвърне проляната от него кръв на неговата глава, понеже той нападна двама мъже по-праведни и по-добри от него, та ги уби с меч, без да знае баща ми Давид, - Авенира Нировия син, Израилевия военачалник, и Амаса Етеровия син, Юдовия военачалник.
\par 33 Така ще се възвърне кравта им на Иоавовата глава и на главата на потомството му до века; а на Давида, на потомството му, на дома му и на престола му ще бъде мир от Господа до века.
\par 34 Тогава Ванаия, Иодаевият син, влезе да го нападна и го уби; и погребан бе у дома си в пустинята.
\par 35 И вместо него царят постави свещеника Садока.
\par 36 Тогава царят прати да повикат Семея, и рече му: Построй си къща в Ерусалим и живей там и да не излезеш от там на никъде;
\par 37 защото, знай положително, че в деня, когато излезеш и преминеш потока Кедрон, непременно ще бъдеш убит; кръвта ти ще бъде на твоята глава.
\par 38 И Семей рече на царя: Добра е думата; според както каза господарят ми царят, така ще стори слугата ти. И Семей живя в Ерусалим доволно време.
\par 39 А след три години, двама от Семеевите слуги побягнаха при гетския цар Анхуса, Мааховия син; и известиха на Семея, казвайки: Ето, слугите ти са в Гет.
\par 40 Тогава Семей, като стана та оседла осела си, отиде в Гет при Анхуса за да потърси слугите си; и Семей отиде та доведе слугите си от Гет.
\par 41 И извести се на Соломона, че Семей ходил от Ерусалим в Гет и се върнал.
\par 42 Тогава царят прати да повикат Семея, и му рече: Не те ли заклех в Господа и те предупредих, като рекох: Знай положително, че в деня, когато излезеш и отидеш вън, где и да е, непременно ще умреш; и ти ми рече: Добра е думата, която чух?
\par 43 Защо, прочее, не опази Господната клетва и заповедта, която ти дадох?
\par 44 Царят още каза на Семея: Ти знаеш всичкото зло сторено на баща ми Давида, което се таи в сърцето ти; затова, Господ ще възвърне злото ти на твоята глава;
\par 45 а цар Соломон ще бъде благословен, и Давидовият престол утвърден пред Господа до века.
\par 46 Тогава царят заповяда на Ванаия Иодаевия син; и той излезе та го нападна; и той умря. И царството се утвърди в Соломоновата ръка.

\chapter{3}

\par 1 А Соломон се сроди с египетския цар Фараона, като взе Фараоновата дъщеря; и доведе я да живее в Давидовия град догде да свърши съграждането на своя дом, и на Господния дом и не стената около Ерусалим.
\par 2 Но людете жертвуваха по високите места, понеже до онова време нямаше дом съграден за Господното име.
\par 3 И Соломон възлюби Господа, и ходеше в повеленията на баща си Давида; само че жертвуваше и кадеше по високите места.
\par 4 И царят отиде в Гаваон, за да принесе там жертва, защото това бе главното високо място; хиляда всеизгаряния принесе Соломон на оня олтар.
\par 5 А в Гаваон Господ се яви на Соломона на сън през нощта; и рече Бог: Искай какво да ти дам.
\par 6 А Соломон каза: Ти показа голяма милост към слугата Си баща ми Давида, понеже той ходи пред Тебе във вярност, в правда и в сърдечна правота с Тебе; и Ти си запазил за него тая голяма милост, че си му дал син да седи на престола му, както е днес.
\par 7 И сега, Господи Боже мой, Ти си направил слугата Си цар вместо баща ми Давида; а аз съм малко момче; на зная как да се обхождам.
\par 8 И слугата Ти е всред Твоите люде които Ти си избрал, люде много, които поради множеството си не могат да се изброят, нито да се пресметнат.
\par 9 Дай, прочее, на слугата Си разумно сърце, за да съди людете Ти, за да различава между добро и зло; защото кой може да съди тоя Твой голям народ;
\par 10 И тия думи бяха угодни Господу, понеже Соломон поиска това нещо.
\par 11 И Бог му каза: Понеже ти поиска това нещо, и не поиска за себе си дълъг живот, нито поиска за себе си богатство, нито поиска смъртта на неприятелите си, но поиска за себе си разум за да разбираш правосъдие,
\par 12 ето, сторих според както си казал; ето, дадох ти мъдро и разумно сърце, така щото преди тебе не е имало подобен на тебе, нито подир тебе ще се издигне подобен на тебе.
\par 13 А при това ти дадох каквото не си поискал - и богатство и слава, така щото между царете не ще има подобен на тебе през всичките ти дни.
\par 14 И ако ходиш в Моите пътища, и пазиш повеленията Ми и заповедите Ми, както ходи баща ти Давид, тогава ще продължа дните ти.
\par 15 И събуди се Соломон; и, ето, бе сън. След това, дойде в Ерусалим и като застана пред ковчега на Господния завет, пожертвува всеизгаряния и принесе примирителни приноси; направи и угощение на всичките си слуги.
\par 16 Тогава дойдоха при царя две блудници та застанаха пред него.
\par 17 И едната жена рече: О, господарю мой! аз и тая жена живеем в една къща; и аз родих като живеех с нея в къщата.
\par 18 И на третия ден, откак родих аз, роди и тая жена; и ние бяхме сами заедно, нямаше външен човек с нас в къщата, само ние двете бяхме в къщата.
\par 19 И през нощта умря синът на тая жена, понеже го налегнала.
\par 20 А тя, като станала посред нощ, взела сина ми от при мене, когато слугинята ти спеше, та го турила на своята пазуха, а своя мъртъв син турила на моята пазуха.
\par 21 И в зори, като станах, за да накърмя сина си, ето, той бе мъртъв; но на утринта, като го разгледах, ето, не бе моят син, когото бях родила.
\par 22 А другата жена рече: Не, но живият е моят син, и мъртвият е твоят син. А тая рече: Не, но мъртвият е твоят син, а живият е моят син. Така говориха пред царя.
\par 23 Тогава царят рече: Едната казва: Тоя живият е моят син, а мъртвият е твоят син; а другата казва: Не, но мъртвият е твоят син, а живият е моят син.
\par 24 И царят рече: Донесете ми нож. И донесоха нож пред царя.
\par 25 И царят рече: Разделете на две живото дете, и дайте половината на едната и половината на другата.
\par 26 Тогава оная жена, чието беше живото дете, говори на царя (защото сърцето й я заболя за сина й), казвайки: О господарю мой! дай й живото дете, и недей го убива. А другата рече: Нито мое да е, нито твое; разделете го.
\par 27 Тогава царят в отговор рече: Дайте на тая живото дете, и недейте го убива; тая е майка му.
\par 28 И целият Израил чу за съда който царят отсъди; и бояха се от царя, защото видяха, че Божия мъдрост имаше в него, за да раздава правосъдие.

\chapter{4}

\par 1 Цар Соломон, прочее, царуваше над целия Израил;
\par 2 и ето началниците, които той имаше: Азария, Садоковия син, свещеник;
\par 3 Елиореф и Ахия, синовете на Сиса, секретари; Иосафат, Ахилудовия син, летописец;
\par 4 Ванаия, Иодаевия син, над войската; Садок и Авиатар, свещеници;
\par 5 Азария, Натановия син, над надзирателите на храната ; Завуд, Натановия син, главен чиновник и приятел на царя.
\par 6 Ахисар, домоуправител; а Адонирам, Авдовия син, над набора.
\par 7 А Соломон имаше из целия Израил дванадесет надзиратели; които доставяха храните на царя и за дома му; всеки доставяше за един месец в годината.
\par 8 И ето имената им: Оровия син, надзирател в Ефремовата гора;
\par 9 Декеровият син, в Макас-Саавим, Ветсемес и Елон-Ветанан.
\par 10 Еседовият син, в Арувот; под него бе Сохо и цялата страна Ефер;
\par 11 Авинадавовият син, в целия Нафат-дор; той имаше за жена Соломоновата дъщеря Тафата;
\par 12 Ваана Ахилудовият син, в Таанах и Магедон, и в целия Ветсан, който е при Церетан под Езраил, от Ветсан до Авел-меола до отвъд Иокмеам;
\par 13 Геверовият син, в Рамот-галаад; той имаше градовете на Яира Манасиевият син, които са в Галаад; той имаше и областта Аргов, която е във Васан, - шестдесет големи градове със стени и медни лостове;
\par 14 Ахинадав Иодовият син, в Маханаим;
\par 15 Ахимаас, в Нефталим; и той взе за жена Соломоновата дъщеря Васемата;
\par 16 Ваана, Хусевият син, в Асир и в Алот;
\par 17 Иосафат, Фаруевият син, в Исахар;
\par 18 Семей, Илаевият син, във Вениамин;
\par 19 и Гевер, Уриевият син, в галаадската земя, в земята на аморейския цар Сион и на васанския цар Ог; и в тая земя той бе единственият надзирател.
\par 20 Юда и Израил бяха многобройни, по множество както крайморския пясък; ядяха, пиеха и веселяха се.
\par 21 И Соломон владееше над всичките царства от реката Евфрат до филистимската земя и до египетската граница; и те донасяха подаръци, и бяха подчинени на Соломона през всичките дни на неговия живот.
\par 22 А продоволствието за Соломона за един ден бе тридесет кора чисто брашно и шестдесет кора друго брашно,
\par 23 десет угоени говеда и двадесет охранени говеда, и сто овци, освен елени, сърни, биволи и тлъсти птици.
\par 24 Защото той владееше над цялата земя отсам реката, от Тапса дори до Газа, - над всички царе отсам реката; и той имаше мир навсякъде около себе си.
\par 25 Юда и Израил живееха безопасно, всеки под лозята си и под смоковницата си, от Дан до Вирсавее, през всичките дни на Соломона.
\par 26 И Соломон имаше обори за четиридесет хиляди коне за колесници.
\par 27 И ония надзиратели продоволствуваха, всеки в месеца си, за цар Соломона и за всички, които дохождаха на Соломоновата трапеза; те не допускаха никаква оскъдност.
\par 28 Още ечемик и слама за конете и бързоногите коне донасяха на мястото, гдето бяха, всеки според както му беше определено.
\par 29 И Бог даде на Соломона твърде много мъдрост и разум, и душевен простор, като крайморския пясък.
\par 30 Така Соломоновата мъдрост надмина мъдростта на всичките източни жители и цялата египетска мъдрост;
\par 31 защото беше по-мъдър от всичките човеци, - от езраеца Етан, и от Емана, Халкола и Дарда, синовете на Маола; и името му се прочу между всичките околни народи.
\par 32 Той изрече три хиляди поговорки; а песните му бяха хиляда и пет на брой .
\par 33 Той говори за дърветата, от ливанския кедър дори до исопа, който никне из стената; говори още за животните, за птиците, за пълзящите и за рибите.
\par 34 И от всичките народи, и от всичките царе на света, които бяха чули за мъдростта на Соломона, дохождаха да слушат неговата мъдрост.

\chapter{5}

\par 1 А Тирският цар Хирам, като чу, че помазали Соломона за цар вместо баща му, прати слугите си до него; защото Хирам бе всякога приятел на Давида.
\par 2 И Соломон прати на Хирама да кажат:
\par 3 Ти знаеш, че баща ми Давид не можа да построи дом за името на Господа своя Бог по причина на боевете, които го обикаляха от всякъде, докато Господ не положи неприятелите му под стъпалата на нозете му.
\par 4 Но сега, Господ моят Бог, ми даде спокойствие навред; нямам противник нито злополука.
\par 5 И, ето, аз възнамерявам да построя дом за името на Господа моя Бог според както Господ говори на баща ми Давида, като рече: Стани ти, когото ще поставя вместо тебе на престола ти, той ще построи дом за името Ми.
\par 6 Сега, прочее, заповядай да ми насекат кедри от Ливан; моите слуги ще бъдат с твоите слуги; и ще ти дам заплата за слугите ти напълно според както ще речеш; защото ти знаеш, че между нас няма човек, който знае да сече дървета, толкова изкусно, колкото сидонците.
\par 7 Тогава Хирам, като чу Соломоновите думи, много се зарадва и рече: Благословен да бъда днес Господ, Който даде на Давида мъдър син да царува над тоя велик народ.
\par 8 И Хирам прати до Соломона да кажат: Чух известието , което ти ми прати; аз ще сторя все що искаш за кедровите дървета и за елховите дървета.
\par 9 Моите слуги ще ги снемат от Ливан до морето; и аз ще ги свържа на салове, за да се превозват по море до мястото, което би ми посочил, и там да се развържат и ти да ги прибереш. Също и ти ще сториш каквото искам, да продоволствуваш моя дом.
\par 10 И тъй, Хирам даваше на Соломона кедрови дървета и елхови дървета, колкото той искаше.
\par 11 А Соломон даде на Хирама двадесет хиляди кора пшеница за храна на дома му, и двадесет кора първоток дървено масло; така даваше Соломон на Хирама всяка година.
\par 12 И Господ даде на Соломона мъдрост, според както му беше обещал; и имаше мир между Хирама и Соломона, и те двамата направиха договор помежду си.
\par 13 И цар Соломон събра набор от целия Израил, и събраните мъже бяха тридесет хиляди души.
\par 14 И изпращаше ги в Ливан на смени, по десет хиляди души на месец; един месец бяха в Ливан и два месеца у домовете си; а над набора беше Адонирам.
\par 15 Соломон имаше и седемдесет хиляди бременосци и осемдесет хиляди каменоделци в планините,
\par 16 освен три хиляди и триста Соломонови настойници, които бяха над работите, надзираващи людете, които работеха тая работа.
\par 17 И по царската заповед изкараха големи камъни, камъни с голяма стойност, за да положат основите на дома с дялани камъни.
\par 18 И Соломоновите зидари, и Хирамовите зидари, и гевалците ги издялаха, и приготвиха дърветата и камъните, за да построят дома.

\chapter{6}

\par 1 В четиристотин и осемдесетата година от изхода на израилтяните из Египетската земя, в четвъртата година на царуването си над Израиля, в месец Зив, който е вторият месец, Соломон почна да строи Господния дом.
\par 2 И дължината на дома, който Цар Соломон построи за Господа, беше шестстотин лакътя, широчината му двадесет, а височината му тридесет лакътя.
\par 3 А предхрамието, откъм лицето на храма на дома, беше двадесет лакътя дълго, според широчината на дома, а десет лакътя широко пред дома.
\par 4 И направи за дома неподвижни прозорци с решетки.
\par 5 И пристрои етажи изоколо до стената на дома, изоколо до стените на дома, както на храма, така и на светилището; така направи странични стаи изоколо.
\par 6 На по-долния етаж широчината бе пет лакътя, на средния широчината бе шест лакътя, а на третия широчината бе седем лакътя; защото на външната страна не стената на дома той направи стеснения изоколо, за да не влизат гредите в стените на дома.
\par 7 И като се строеше домът иззида се с камъни приготвени на кариерата; така щото нито чук, нито топор, нито какво да е желязно сечиво не се чу в дома, кат се строеше.
\par 8 Вратата за средните странични стаи беше в дясната страна на дома; и чрез витлообразна стълба се възкачваха в средния етаж , и от средния в третия.
\par 9 Така построи дома и го свърши; и покри дома с кедрови гради и дъски.
\par 10 Пристрои дома и го свърши; и покри дома с кедрови греди и дъски.
\par 11 Тогава Господното слово дойде до Соломона, и рече:
\par 12 Относно тоя дом, който строиш, казвам ти : Ако ходиш в повеленията Ми, изпълняваш съдбите Ми, и пазиш всичките Ми заповеди да ходиш в тях, тогава Аз ще утвърдя с тебе думата, която говорих на баща ти Давида;
\par 13 И ще обитавам всред израилтяните, и няма да оставя людете Си Израиля.
\par 14 Така Соломон построи дома и го свърши.
\par 15 И облече стените на дома извътре с кедрови дъски; облече стените с дърво извътре от пода на дома до покрива, а пода на дома покри с елхови дъски.
\par 16 Още в по-вътрешната част на дома облече едно място от двадесет лакътя с кедрови дъски от пода до върха на стените; облече го извътре за светилището - най-светото място.
\par 17 А домът, то ест, предхрамието, бе четиридесет лакътя дълъг .
\par 18 И по кедровите дървета, извътре дома, бяха изрязани пъпки и цъфнали цветове; всичко бе кедрово; камък не се виждаше.
\par 19 Той приготви светилище в по-вътрешната част на дома, за да положи там ковчега на Господния завет.
\par 20 Отвътре светилището беше двадесет лакътя дълго, двадесет лакътя широко и двадесет лакътя високо; и го обкова с чисто злато, обкова и кедровия олтар.
\par 21 Така Соломон обкова дома извътре с чисто злато; и прокара завесата на златни верижки пред светилището, и покри я със злато.
\par 22 Обковаваше със злато и целия дом, докато свърши целия дом; обкова със злато и целия олтар, който бе при светилището.
\par 23 И в светилището направи два херувима от маслинено дърво, по десет лакътя високи.
\par 24 Едното крило на единия херувим беше пет лакътя дълго , и другото крило на единия херувим пет лакътя; от края на едното му крило, до края на другото му крило, имаше десет лакътя.
\par 25 Така и другият херувим имаше десет лакътя между краищата на крилата си; защото двата херувима имаха една мярка и една направа.
\par 26 Височината на единия херувим беше десет лакътя, така и на другия херувим.
\par 27 И той положи херувимите в средата на вътрешния дом; и крилата на херувимите бяха разперени, така щото крилото на единия достигаше едната стена, и крилото на другия херувим достигаше другата стена; и крилата им се допираха едно с друга в средата на дома.
\par 28 И обкова херувимите със злато.
\par 29 А по всичките стени на дома изоколо изряза образи на херувими и палми и цъфнали цветове, извътре и извън.
\par 30 Също и пода на дома обкова със злато извътре и извън.
\par 31 За входа на светилището направи врата от маслинено дърво; и горният праг със стълбовете на вратата бяха една пета от дължината на стената .
\par 32 И двете крила на вратата бяха от маслинено дърво; и по тях Соломон изряза херувими и палми и цъфнали цветове, и обкова ги със злато, като разстла златото върху херувимите и върху палмите.
\par 33 Така и за храмовата врата направи вратни стълбове от маслинено дърво, ограждайки отверстие широко една четвърт от дължината на стената .
\par 34 И две врати от елхово дърво; двете крила на едната врата се сгъваха и двете крила на другата врата се сгъваха.
\par 35 И изряза на тях херувими и палми и цъфнали цветове; и покри ги със злато натъкмено върху изрязаното,
\par 36 Съгради вътрешния двор с три реда дялани камъни и с един ред кедрови греди.
\par 37 През четвъртата година, в месец Зив, се положиха основите на Господния дим;
\par 38 И през единадесетата година, в месец Вул, който е осмият месец, домът се свърши във всичките си части и съвършено по плана си. Така, за седем години той го построи.

\chapter{7}

\par 1 А Соломон строеше своята къща тринадесет години, докато свърши цялата си къща.
\par 2 Защото построи къщата Ливански лес, на която дължината му сто лакътя, широчината петдесет лакътя, и височината тридесет лакътя, върху четири реда кедрови стълбове, с кедрови греди върху стълбовете.
\par 3 И тя биде покрита с кедър отгоре на четиридесет и петте стаи които се подпираха на стълбовете, по петнадесет в един етаж.
\par 4 И имаше решетки в трите етажа, така че светене беше поставено срещу светене в трите етажа.
\par 5 И всичките врати и вратни стълбове, и решетките, бяха четириъгълни; и светене беше поставено срещу светене в трите етажа.
\par 6 Направи и трем от стълбове, на които дължината беше петнадесет лакътя; и тремът беше пред стълбовете на къщата , така щото стълбовете и стъпалата му бяха към лицето на тия.
\par 7 Направи още престолния трем, гдето щеше да седи, то ест, съдилищния трем; и той беше облечен с кедър от пода до върха.
\par 8 А къщата му, в която жевееше, поставена в един друг двор по-навътре от трема, имаше същата направа. И подобна на тоя трем Соломон направи къща и за Фараоновата дъщеря, която бе взел за жена.
\par 9 Всички тия постройки , от вътрешното и от външното им лице, от основата до върха им, а отвън дори до големия двор, бяха от скъпи камъни, от камъни дялани според мярка, претрити с трион.
\par 10 Тоже и основата му от скъпи камъни, големи камъни, камъни, от десет лакътя и камъни от осем лакътя.
\par 11 И отгоре имаше скъпи камъни, камъни дялани според мярка, и кедрови греди .
\par 12 Също и около големия двор имаше три реда дялани камъни и един ред кедрови греди, подобно на вътрешния двор на Господния дом и подобно на трема на къщата.
\par 13 А цар Соломон бе пратил да доведат Хирама от Тир.
\par 14 Той бе син на една вдовица от Нефталимовото племе, а баща му беше тирянин, който работеше мед; и той бе твърде изкусен, разумен, вещ да изработва всякаква медна работа. И така, той дойде при цар Соломона та му изработи всичката му работа.
\par 15 Защото изля двата медни стълба, всеки стълб осемнадесет лакътя висок; и окръжността на всеки стълб се измерваше с връв от дванадесет лакътя.
\par 16 И направи от леяна мед два капитела, за да ги тури на върховете на стълбовете; височината на единия капител беше пет лакътя и височината на другия капител пет лакътя.
\par 17 Направи още мрежи от плетена изработка и венцеобразни верижки за капителите, които бяха на върховете на стълбовете, седем за единия капител и седем за другия капител.
\par 18 Така направи стълбовете и два реда нарове , изоколо върху всяка мрежа, за да покрие с нарове капителите, които бяха на върховете на стълбовете : и направи същото на другия капител.
\par 19 И капителите, които бяха на върховете на стълбовете в трема, бяха от по четири лакътя, изработени във вид на кремове.
\par 20 И капителите, които бяха на двата стълба, имаха нарове и отгоре, близо до изпъкналата част, която бе при мрежата; и наровете бяха двеста, наредени изоколо върху всеки капител.
\par 21 И Соломон изправи стълбовете за трема на храма; и като изправи десния стълб, нарече го Яхин; а като изправи левия стълб, нарече го Воаз.
\par 22 И на върховете на стълбовете имаше изработени кремове. Така се свърши направата на стълбовете.
\par 23 Направи още и леяното море, с устие десет лакътя широко , кръгло изоколо, а с височина пет лакътя; и връв от тридесет лакътя го измерваше околовръст.
\par 24 Наоколо под устието му имаше цветни пъпки, които го обикаляха, по десет на един лакът; те обикаляха морето изоколо; пъпките бяха на два реда, излеяни в едно цяло с него.
\par 25 И морето стоеше на дванадесет вола, три гледащи към север, три гледащи към запад, три гледащи към юг, и три гледащи към изток; а морето стоеше върху тях; и задните части на всичките бяха навътре.
\par 26 Дебелината му беше една длан; а устието му беше направено като устие на чаша, във вид на кремов цвят; и побираше две хиляди вати вода .
\par 27 Направи още и десет медни подножия, всяко подножие четири лакътя дълго, четири лакътя широко и три лакътя високо.
\par 28 А ето каква беше направата на подножията: имаха странични плочи, и страничните плочи бяха между стълбчета.
\par 29 А върху страничните плочи, които бяха между стълбчетата, имаше лъвове, волове и херувими; и над стълбчетата беше подпорката; а под лъвовете и воловете имаше висящи ресни.
\par 30 Всяко подножие имаше четири медни колела и медни оси, и четирите му крака имаха рамена; рамената, които бяха леяни, се намираха под умивалницата всяко срещу ресните.
\par 31 Отверстието на подножието , извътре капитела и нагоре, беше един лакът, а отверстието му имаше ваяния; а страничните плочи бяха четвъртити, а не кръгли.
\par 32 Под страничните плочи имаше четири колела; и осите на колелата се съединяваха с подножието; а височината на всяко колело бе лакът и половина.
\par 33 Направата на колелата бе като направата на колеснично колело; осите им, наплатите им, спиците им и главините им, - всичките бяха леяни.
\par 34 При четирите ъгъла на всяко подножие имаше четири рамена, и рамената бяха част от самото подножие.
\par 35 На върха на подножието имаше кръгла препаска половин лакът висока; а отверстието и страничните плочи на върха на подножието съставляваха едно цяло с него.
\par 36 И по плочите на отверстието му и по страничните му плочи Хирам издълба херувими, лъвове и палми, съразмерно големината на всяко, с ресни наоколо.
\par 37 Така направи десет подножия, които имаха всички същото леяне, същата мярка и същия образ.
\par 38 Направи още десет медни умивалници; всеки умивалник побираше четиридесет вати вода ; всеки умивалник беше четири лакътя широк ; и върху всяко от десетте подножия се постави един умивалник.
\par 39 И постави подножията, пет от дясната страна на дома, и пет от лявата страна на дома; а морето постави от дясната страна на дома, към изток, къде юг.
\par 40 Хирам направи и умивалниците, лопатите и легените. Така Хирам свърши изработването на всичките работи, които направи на цар Соломона за Господния дом.
\par 41 Двата стълба; изпъкналата част на капителите, които бяха на върховете на двата стълба; двете мрежи, които да покриват двете изпъкналости на капителите, които бяха на върховете на стълбовете;
\par 42 Четирите стотин нарове за двете мрежи - два реда нарове за всяка мрежа, които да покриват двете изпъкналости на кепителите;
\par 43 Десетте подножия, и десетте умивалници върху подножията;
\par 44 едното море и дванадесетте вола под морето;
\par 45 котлите, лопатите и легените - всички тия вещи, които Хирам направи на цар Соломона за Господния дом, бяха от лъскава мед.
\par 46 На Иорданското поле царят ги изля, в глинената земя между Сокхот и Царетан.
\par 47 И Соломон остави всички тия вещи непретеглени , защото бяха твърде много; теглото на медта не можеше да се пресметне.
\par 48 И Соломон направи всичките принадлежности, които бяха за Господния дом, - златния олтар; златната трапеза, на която се полагаха присъствените хлябове;
\par 49 светилниците, пет отдясно и пет отляво пред светилището, от чисто злато; с цветята, светилата и клещите от злато;
\par 50 чашите, щипците, легените, темянниците и кадилниците, от чисто злато и резета от злато, за вратата на най-вътрешния дом, то ест , пресветото място, и за вратата на дома, сиреч , на храма.
\par 51 Така се свърши всичката работа, която цар Соломон направи за Господния дом. И Соломон внесе нещата, посветени от баща му Давида, - среброто, златото и вещите, - и ги положи в съкровищницата на Господния дом.

\chapter{8}

\par 1 Тогава Соломон събра при себе си в Ерусалим Израилевите старейшини, и всичките началници на племената, началниците на бащините домове на израилтяните, за да възнесат ковчега на Господния завет от Давидовия град, който е Сион.
\par 2 И тъй всичките Израилеви мъже се събраха при цар Соломона на празника в месец Етаним, който е седмият месец.
\par 3 А когато дойдоха всичките Израилеви старейшини, свещениците вдигнаха ковчега.
\par 4 Те занесоха Господния ковчег и шатъра за срещане с всичките свети принадлежности, които бяха в шатъра; свещениците и левитите ги занесоха.
\par 5 А цар Соломон и цялото Израилево общество, колкото се бяха събрали при него, бяха с него пред ковчега, и жертвуваха овци и говеда, които по множеството си не можеха да се пресметнат или да се изброят.
\par 6 Така свещениците внесоха ковчега на Господния завет на мястото му, в светилището на дома, в пресветото място, под крилата на херувимите.
\par 7 Защото херувимите бяха с крилата си разперени над мястото на ковчега и херувимите покриваха отгоре ковчега и върлините му.
\par 8 И върлините се издадоха така щото се виждаха краищата на върлините от светото място пред светилището, но извън не се виждаха; и там са до днес.
\par 9 В ковчега нямаше друго освен двете каменни плочи, които Моисей положи там на Хорив, гдето Господ направи завет с израилтяните, когато бяха излезли от Египетската земя.
\par 10 А щом излязоха свещениците из светилището, облакът изпълни Господния дом;
\par 11 така щото поради облака свещениците на можеха да застанат, за да служат, защото Господната слава изпълни Господния дом.
\par 12 Тогава Соломон говори: Господ е казал, че ще обитава в мрак.
\par 13 Аз Ти построих дом за обитаване, място, в което да пребиваваш вечно.
\par 14 После царят обърна лицето си та благоволи цялото Израилево общество, докато цялото Израилево общество стоеше на крака, като каза:
\par 15 Благословен да бъде Господ Израилевият Бог, Който извърши с реката Си онова, което говори с устата Си онова, което говори с устата Си на баща ми Давида, като рече:
\par 16 От деня, когато изведох людете Си Израиля из Египет, не избрах измежду всичките Израилеви племена ни един град, гдето да се построи дом за да се настани името Ми там; но избрах Давида, за да бъде над людете Ми Израиля.
\par 17 И в сърцето на баща ми Давида дойде да построи дом за името на Господа Израилевия Бог;
\par 18 но Господ рече на баща ми Давида: Понеже дойде в сърцето ти да построиш дом за името Ми, добре си сторил, че е дошло това в сърцето ти.
\par 19 Ти, обаче, няма да построиш дома; но синът ти, който ще излезе из чреслата ти, кой ще построи дома за името Ми.
\par 20 Господ, прочее, изпълни словото, което говори; и като се издигнах аз вместо баща си Давида, и седнах на Израилевия престол, според както говори Господ, построих дома за името на Господа Израилевия Бог.
\par 21 И в него приготвих място за ковчега, в който е заветът, който Господ направи с бащите ни, когато ги изведе из Египетската земя.
\par 22 Тогава Соломон застана пред Господния олтар, пред цялото Израилево общество, и като простря ръцете си към небето рече:
\par 23 Господи Боже Израилев, няма гора на небето или долу на земята бог подобен на Тебе, Който пазиш завета и милостта към слугите Си, които ходят пред Тебе с цялото си сърце;
\par 24 Който си изпълнил към слугата Си Давида, баща ми, това, което си му обещал; да! каквото си говорил с устата Си, това си свършил с ръката Си, както се вижда днес.
\par 25 Сега, Господи Боже Израилев, изпълни към слугата Си Давида, баща ми, онова, което си му обещал, като си рекъл: Няма да ти липсва мъж, който да седи пред Мене на Израилевия престол, ако само внимават чадата ти в пътя си, за да ходят пред Мене, както ти си ходил пред Мене.
\par 26 Сега, прочее, Боже Израилев, нека се потвърди, моля, словото, което си говорил на слугата Си Давида баща ми.
\par 27 Но Бог наистина ли ще обитава на земята? Ето, небето и небето на небесата не са достатъчни да Ти поберат; колко по-малко тоя дом, който построих!
\par 28 Но пак погледни благосклонно към молитвата на слугата Си и към молението му, Господи Боже мой, тъй щото да послушаш вика и молитвата, с която слугата Ти се моли днес пред Тебе,
\par 29 за да бъдат очите Ти, нощем и денем, отворени към Твоя дом, към мястото, за което Ти си казвал: Името Ми ще бъде там, за да слушаш молитвата, с която слугата Ти ще се моли на това място.
\par 30 Слушай молението на слугата Си и на людете Си Израиля, когато се молят на това място да! слушай Ти от местообиталището Си, от небето, и като слушаш бивай милостив.
\par 31 Ако съгреши някой на ближния си, и му се наложи клетва за да се закълне, и той дойде та се закълне пред олтара Ти в тоя дом,
\par 32 тогава послушай Ти от небето и подействувай, извърши правосъдие за слугите Си, и осъди беззаконния, така щото да възвърнеш делото му върху главата му, а оправяй праведния като му отдадеш според правдата му.
\par 33 Когато людете Ти Израил бъдат разбити пред неприятеля по причина, че са Ти съгрешили, ако се обърнат към Тебе та изповядат Твоето име, и принесат молитва, като Ти се помолят в тоя дом,
\par 34 тогава Ти послушай от небето, и прости греха на людете Си Израиля, и възвърни ги в земята, която си дал на бащите им.
\par 35 Когато се затвори небето та не вали дъжд по причина, че са Ти съгрешили, ако се помолят на това място и изповядат Твоето име, и се обърнат от греховете си, понеже ги съкрушаваш,
\par 36 тогава Ти послушай от небето и прости греха на слугите Си и на людете Си Израиля, и покажи им добрия път, в който трябва да ходят, и дай дъжд на земята Си, която си дал в наследство на людете Си.
\par 37 Ако настане глад на земята, ако настане мор, ако се появят изсушителен вятър, мана, скакалци, или гъсеници, ако неприятелят им ги обсади в градовете на земята им, - каквато и да е язвата, каквато и да е болестта, -
\par 38 тогава всяка молитва, всяко моление, което би се принесло от кой да бил човек, или от всичките Ти люде Израиля, който ще познае всеки раната на своето сърце, и простре ръцете си към тоя дом,
\par 39 Ти послушай от небето, от местообиталището Си, и прости и подействувай та въздай на всекиго според всичките му постъпки, като познаваш сърцето му, (защото Ти, само Ти, познаваш сърцата на целия човешки род),
\par 40 за да Ти се боят през всичкото време, когато живеят на земята, която си дал на бащите ни.
\par 41 Още за чужденеца, който не е от людете Ти Израиля, но иде от далечна страна, заради Твоето име,
\par 42 (защото ще чуят за великото Ти име, и за Твоята мощна ръка, и за Твоята издигната мишца), - когато дойде та се помоли в тоя дом,
\par 43 Ти послушай от небето, от местообиталището Си, и подействувай според всичко, за което чужденецът Те призове; за да познаят името Ти всичките люде на света, да Ти се боят както людете Ти Израил, и да познаят, че с Твоето име се нарече тоя дом, който построих.
\par 44 Ако людете ти излязат на бой против неприятеля си, където би ги пратил Ти, и се помолят на Господа като се обърнат към града, който Ти си избрал, и към дома, който построих за Твоето име,
\par 45 тогава послушай от небето молитвата им и защити правото им.
\par 46 Ако Ти съгрешат, (защото няма човек, който да не греши), и Ти се разгневиш на тях та ги предадеш на неприятеля, и пленителите им ги заведат пленници в неприятелската земя, далеч или близо,
\par 47 все пак, ако дойдат на себе си в земята, гдето са отведени пленници, та се обърнат и Ти се помолят в земята на пленителите си, и рекат: Съгрешихме, беззаконстувахме, сторихме неправда,
\par 48 и се обърнат към Тебе с цялото си сърце и с цялата си душа в земята на неприятелите, които са ги запленили, и Ти се помолят като се обърнат към земята си, която си дал на бащите им, към града, който си избрал, и към дома, който построих за Твоето име,
\par 49 тогава Ти послушай от небето, от местообиталището Си, молитвата им и молението им, защити правото им,
\par 50 и прости на людете Си, които са Ти съгрешили, всичките им прегрешения, чрез които станаха престъпници против Тебе, и умилостиви към тях пленителите им, за да им покажат милост;
\par 51 защото те са Твои люде и Твое наследство, които си извел из Египет, отсред железарската пещ.
\par 52 Нека прочее, бъдат отворени очите Ти към молението на слугата Ти и към молението на людете Ти Израиля, за да ги слушаш за каквото и да Те призоват;
\par 53 защото Ти, Господи Иеова, си ги отделил от всичките племена на света, за да бъдат Твое наследство, според както говори чрез слугата Си Моисея, когато изведе бащите ни из Египет.
\par 54 И като свърши Соломон да принася цялата тая молитва и това моление към Господа, стана отпред Господния олтар, гдето бе коленичил с ръце прострени към небето.
\par 55 И застана та благослови с висок глас цялото Израилево общество, като каза:
\par 56 Благословен да бъде Господ, Който успокои людете Си Израиля според всичко що е обещал. Не пропадна ни едно от всичките добри обещания, които даде чрез слугата Си Моисея.
\par 57 Дано бъде Господ наш Бог с нас както е бил с бащите ни, да не ни остави, нито да ни отхвърли!
\par 58 да преклони сърцата ни към себе Си, тъй щото да ходим във всичките Му пътища, и да пазим заповедите, повеленията и съдбите, които е заповядал на бащите ни!
\par 59 И тия мои думи, с които се помолих пред Господа, нека бъдат денем и нощем близо при Господа нашия Бог, за да защищава правото на слугата Си и правото на людете Си Израиля според всекидневната нужда;
\par 60 така щото всичките племена на света да познаят, че Иеова Той е Бог; няма друг.
\par 61 Прочее, вашите сърца нека бъдат съвършени спрямо Господа нашия Бог, за да ходите в повеленията Му и да пазите заповедите Му, както правите днес.
\par 62 Тогава царят и целият Израил с него принесоха жертви пред Господа.
\par 63 За примирителни жертви, които принесе Господу, Соломон пожертвува двадесет и две хиляди говеда и сто и двадесет хиляди овци. Така царят и всичките израилтяни посветиха Господния дом.
\par 64 В същия ден царят освети средата на двора, който е към лицето на Господния дом; защото там принесе всеизгарянето и хлебния принос и тлъстината на примирителните приноси, понеже медният олтар, който бе пред Господа, бе твърде малък, за да побере всеизгарянето и хлебния принос и тлъстината на примирителните приноси.
\par 65 По тоя начин Соломон и целият Израил с него, голям събор събран из местностите от прохода на Емат до египетския поток пазеха в онова време празника пред Господа нашия Бог седем дена и седем дена, четиринадесет дена.
\par 66 А на осмия ден разпусна людете: и те благословиха царя, и отидоха си в шатрите си с радостни и весели сърца поради всичкото добро, което Господ бе показал към слугата Си Давида и към людете Си Израиля.

\chapter{9}

\par 1 Когато Соломон беше свършил градежа на Господния дом и на царската къща, и беше извършил всичко, което желаеше Соломон да направи.
\par 2 яви се Господ на Соломона втори път, както бе му се явил в Гаваон.
\par 3 Господ му каза: Чух молитвата ти и молението, с което се моли пред Мене. Тоя дом, който ти построи, Аз го осветих за да настаня там името Си до века: и очите Ми и сърцето Ми ще бъдат там за винаги.
\par 4 А колкото за тебе, ако ходиш пред Мене, както ходи баща ти Давид, с цяло сърце и с правота, тъй щото да постъпваш според всичко, което ти заповядах, като пазиш повеленията Ми и съдбите Ми,
\par 5 тогава ще утвърдя престола на царството ти над Израиля до века, както се обещах на баща ти Давида, като казах: Няма да липсва мъж седящ на Израилевия престол.
\par 6 Но ако се отклоните от Мене, вие или чадата ви, и не опазите заповедите Ми и повеленията, които поставих пред вас, но отидете та послужите на други богове и им се поклоните,
\par 7 тогава ще отсека Израиля от земята, която съм им дал, и ще отхвърля отпред очите Си тоя дом, който осветих за името Си; и Израил ще бъде за поговорка и поругание между всичките племена.
\par 8 А за тоя дом, който стана толкова висок, всеки който минава покрай него, ще се зачуди и ще съска: и ще кажат: Защо направи Господ така на тая земя и на тоя дом?
\par 9 И ще отговарят: Понеже оставиха Господа своя Бог, Който изведе Бащите им из Египетската земя, та си избраха други богове, и им се поклониха и им послужиха, за това Господ нанесе на тях всичкото това зло.
\par 10 А когато се свършиха двадесетте години, в които Соломон построи двата дома, Господния дом и царската къща,
\par 11 и тирският цар Хирам, като беше доставил на Соломона кедрови дървета и елхови дървета и злато, до колкото той желаеше, тогава цар Соломон даде на Хирама двадесет града в галилейската земя.
\par 12 А когато Хирам излезе от Тир, за да види градовете, които му бе дал Соломон, не му се харесаха.
\par 13 И рече: Що за градове са тия, които си ми дал, брате? И нарече ги земя Хавул както се нарича и до днес.
\par 14 И Хирам изпрати на царя сто и двадесет таланта злато.
\par 15 И ето причината на набора, който цар Соломон събра: да съгради Господния дом и своята къща, тоже и Мило, ерусалимската стена, Асор, Магедон и Гезер.
\par 16 (Египетския цар Фараон беше възлязъл и превзел Гезер, и беше го изгорил с огън и избил ханаанците, които живееха в града и беше го дал за зестра на дъщеря си, Соломоновата жена).
\par 17 А Соломон съгради, освен Гезер, и долния Ветерон,
\par 18 Ваалат, Тамар в пустата част на Юдовата земя,
\par 19 всичките градове, в които Соломон имаше житници, градовете за колесниците, градовете за конниците, и все що пожела Соломон да съгради в Ерусалим, в Ливан и в цялата земя на царството си.
\par 20 А относно всичките люде, които останаха от аморейците, хетейците, ферезейците, евейците и евусейците, които не бяха от израилтяните,
\par 21 от техните потомци, останали подир тях в земята, които израилтяните не можаха да изтребят, от тях Соломон събра набор за задължителни работници, каквито са и до днес.
\par 22 Но от израилтяните Соломон не направи никого задължителен работник: а те бяха военни мъже, и неговите служители, неговите първенци, неговите военачалници и началниците на колесниците му и на конниците му.
\par 23 Главните началници, които надзираваха Соломоновите работи, бяха петстотин и петдесет души; те началствуваха над людете, които вършеха работите.
\par 24 А Фараоновата дъщеря се пренесе от Давидовия град в своята къща, който Соломон беше построил за нея, тогава когато той съгради Мило.
\par 25 И три пъти в годината Соломон принасяше всеизгаряния и примирителни приноси върху олтара, който издигна Господ, и кажеше върху оня олтар , който бе пред Господа, след , като свърши дома.
\par 26 Цар Соломон построи и кораби в Есион-гавер, който е при, Елот, на брега на Червеното море, в едомската земя.
\par 27 А в корабите Хирам прати от слугите си, опитни морски корабници, да бъдат със Соломоновите слуги.
\par 28 Те отиваха в Офир, от гдето взеха четиристотин и двадесет таланта злато, и го донесоха на цар Соломона.

\chapter{10}

\par 1 И Савската царица, като чу, че Соломон се прочува като служител на Господното име, дойде да го опита с мъчни за нея въпроси .
\par 2 Дойде в Ерусалим с една твърде голяма свита, с камили натоварени с аромати, и с твърде много злато и скъпоценни камъни; и като дойде при Соломона, говори с него за всичко що имаше на сърцето си.
\par 3 И Соломон отговори на всичките й въпроси; нямаше нищо скрито за царя, което на можа да й обясни.
\par 4 А като видя Савската царица всичката мъдрост на Соломона, и къщата който бе построил,
\par 5 ястията на трапезата му, седенето на слугите му, и прислужването на служителите му, и облеклото им, и виночерпците му, и нагорнището, с което отиваше в Господния дом, не остана дух в нея.
\par 6 И рече на царя: Верен беше слухът, който чух в земята си, за твоето състояние и за мъдростта ти.
\par 7 Аз не вярвах думите, докато не дойдох и не видях с очите си; но, ето, нито половината не ми е била казана; мъдростта ти о благоденствието ти надминават слуха, който бях чула.
\par 8 Честити мажете ти, честити тия твои слуги, които стоят всякога пред тебе та слушат мъдростта ти.
\par 9 Да бъде благословен Господ твоят Бог, Който има благоволение към тебе да те постави на Израилевия престол. Понеже Господ е възлюбил Израиля за винаги, затова те е поставил цар, за да раздаваш правосъдие и да вършиш правда.
\par 10 И тя даде на царя сто и двадесет таланта злато, и твърде много аромати и скъпоценни камъни; не дойде вече такова изобилие от аромати, каквото ония, които Савската царица даде на Соломона.
\par 11 Още и Хирамовите кораби, които донасяха злато от Офир, донасяха от Офир и голямо изобилие алмугови дървета и скъпоценни камъни.
\par 12 А от алмуговите дървета царят направи преградки в Господния дом и в царската къща, тоже и арфи и псалтири за певците; такива алмугови дървета не са дохождали нито са били виждали до днес.
\par 13 И цар Соломон даде на Савската царица всичко що тя желаеше, каквото поиска, освен онова, което цар Соломон й даде доброволно. И тъй тя се върна със слугите си та си отиде в своята си земя.
\par 14 А теглото на златото, което дохождаше на Соломона всяка година, беше шестстотин и шестдесет и шест златни таланта,
\par 15 освен онова, което се внасяше от купувачите, от товарите на търговците, от всичките арабски царе и от управителите на страната.
\par 16 И цар Соломон направи двеста щита от ковано злато; шестстотин сикли злато се иждиви за всеки щит:
\par 17 и триста щитчета от ковано злато; три фунта злато се иждиви за всяко щитче; и царят ги положи в къщата Ливански лес.
\par 18 Царят направи и един великолепен престол от слонова кост, който позлати с най-чисто злато.
\par 19 Престолът имаше шест стъпала, и върхът на престола беше кръгъл отзаде; и имаше облегалки от двете страни на седалището, и два лъва стояха край облегалките.
\par 20 А там, върху шестте стъпала, от двете страни, стояха дванадесет лъва; подобно нещо не се е направило в никое царство.
\par 21 И всичките цар Соломонови съдове за пиене бяха златни, и всичките съдове в къщата Ливански лес от чисто злато; ни един не бе от сребро; среброто се считаше за нищо в Соломоновото време.
\par 22 Защото царят имаше на морето кораби, като тарсийските, заедно с корабите на Хирама; еднъж в три години тия тарсийски кораби дохождаха и донасяха злато и сребро, слонова кост, маймуни и пауни.
\par 23 Така цар Соломон надмина всичките царе на света по богатство и мъдрост.
\par 24 И целият свят търсеше Соломоновото присъствие, за да чуят мъдростта, която Бог бе турил в сърцето му.
\par 25 И всяка година донасяха всеки от тях подаръка си, сребърни вещи, златни вещи, облекла, оръжия и аромати, коне и мъски.
\par 26 Тоже Соломон събра колесници и конници; имаше хиляда и четиристотин колесници, и дванадесет хиляди конници, които настани в градовете за колесниците и при царя в Ерусалим.
\par 27 И царят направи среброто да изобилва в Ерусалим, като камъни, а кедрите направи, като полските черници.
\par 28 И конете, които имаше Соломон, се докарваха из Египет; и царските търговци ги купуваха по стада с определена цена.
\par 29 А всяка колесница излизаше из Египет и възхождаше в Ерусалим за шестстотин сребърни сикли и всеки кон за сто и петдесет; така също за всичките хетейски царе, и за сирийските царе, те се доставяха чрез тия търговци .

\chapter{11}

\par 1 А цар Соломон залюби, освен Фараоновата дъщеря, много чужденки, моавки, амонки, едомки, сидонки, хетейки,
\par 2 от народите за които Господ каза на израилтяните: Не влизайте при тях, нито те да влизат при вас, за да не преклонят сърцата ви към боговете си. Към тях Соломон се страстно прилепи.
\par 3 Имаше седемстотин жени княгини и триста наложници; и жените му отвърнаха сърцето му.
\par 4 Защото, когато остаря Соломон, жените му преклониха сърцето му към други богове; и сърцето му не беше съвършено пред Господа неговия Бог, както сърцето на баща му Давида.
\par 5 Защото Соломон отиде след Астарта, богинята на сидонците, и след Мелхома, мерзостта на амонците.
\par 6 Така Соломон стори зло пред Господа, и не следваше съвършено Господа, както беше сторил баща му Давид.
\par 7 В това време Соломон издигна високо място на Хамоса, мерзостта на Моава, и на Молоха, мерзостта на амонците, в хълма, който е срещу Ерусалим.
\par 8 Така направи и за всичките си чужденки жени, които кадяха и жертвуваха на боговете си.
\par 9 Така направи и за всичките си чужденки жени, които кадяха и жертвуваха на боговете си.
\par 10 А Господ се разгневи на Соломона, понеже сърцето му се отвърна от Господа Израилевия Бог, Който му се яви два пъти.
\par 11 Затова, Господ каза на Соломона: Понеже това е сторено от тебе, и ти не опази завета Ми о повеленията, които ти заповядах, непременно ще откъсна царството от тебе, и ще го дам на слугите ти.
\par 12 Но заради баща ти Давида, в твоите дни няма да направя това; от ръката на сина ти що го откъсна.
\par 13 Обаче, няма да откъсна цялото царство; едно племе ще дам на сина ти, заради слугата Ми Давида и заради Ерусалим, който избрах.
\par 14 След това, Господ подигна противник на Соломона, едомеца Адад, който бе от царското потекло в Едом.
\par 15 Защото, когато Давид бе в Едом, и военачалникът Иоав възлезе да погребе избитите, като беше убил всекиго от мъжки пол в Едом,
\par 16 (понеже Иоав седя там с целия Израил шест месеца, докато изтреби всекиго от мъжки пол в Едом),
\par 17 Адад побягна, и с него няколко едомци от слугите на баща му, за да отидат в Египет; а Адат беше още малко дете.
\par 18 Като станаха от Мадиам, дойдоха във Фаран; и вземайки със себе си мъже от Фаран, дойдоха в Египет при египетския цар Фараон, който му даде къща, определи му храна и му даде земя.
\par 19 И Адад придоби голямо благоволение пред Фараона, така щото той му даде за жена балдъза си, сестра на царицата Тахпенеса,
\par 20 И Тахпенесината сестра му роди сина му Генуват, когото Тахпенеса отби във Фараоновата къща; и Генуват бе в дома на Фараона между неговите синове.
\par 21 И когато Адад чу в Египет, че Давид заспал с бащите си, и че военачалникът Иоав умрял, Адад каза на Фараона: Отпусни ме да си отида в моята страна.
\par 22 А Фараон му каза: Ами от що си лишен при мене та искаш да отидеш в страната си? И отговори: От нищо; но както и да е, отпусни ме.
\par 23 Бог му подигна и друг противник, Резона Елиадевия син, който бе побягнал от господаря си Ададзера, софския цар.
\par 24 И който, като събра около себе си мъже, стана главатар на чета, когато Давид порази совците ; и те отидоха в Дамаск та се зеселиха там, и царуваха в Дамаск.
\par 25 Той бе противник на Израиля през всичките Соломонови дни; и освен пакостите които направи Адад, Резон досаждаше на Израиля като царуваше над Сирия.
\par 26 Също и Еревоам Наватовия син, ефремец от Сарида, Соломонов слуга, чиято майка, една вдовица, се наричаше Серуа, и той дигна ръка против царя.
\par 27 А ето причината, по която той дигна ръка против царя: Соломон беше съградил Мило, и когато поправяше разваленото в стената на града на баща си Давида,
\par 28 тоя Еровоам, човек силен и храбър, бе там ; и Соломон, като видя, че момъкът беше способен за работа, постави го надзирател над всичката работа наложена върху Иосифовия дом.
\par 29 А в това време, като беше излязъл Еровоам из Ерусалим, намери го на пътя пророк Ахия силонецът, облечен в нова дреха; и двамата бяха сами на полето.
\par 30 И Ахия хвана новата дреха, която носеше, и разкъса я на дванадесет части.
\par 31 И рече на Еровоама: Вземи си десет части; защото така казва Господ Израилевият Бог: Ето, ще откъсна царството от Соломоновата ръка, и ще дам на тебе десет племена;
\par 32 (ще остане, обаче, нему едни племе, заради слугата Ми Давида, и заради Ерусалим, града който избрах между всичките Израилеви племена);
\par 33 защото оставиха Мене та послужиха на Астарта, богинята на сидонците, на Хамоса, бога на моавците, и на Мелхома, бога на амонците, и не ходиха в пътищата Ми, да вършат онова, което е право пред Мене, и да пазят повеленията Ми, и съдбите Ми, както правеше баща му Давид.
\par 34 Не ща, обаче, да отнема цялото царство от ръката му, но ще го поставя владетел през всичките дни на живота ме, заради слугата Ми Давида, когото избрах, защото той пазеше заповедите Ми и повеленията Ми.
\par 35 Обаче, ще отнема царството от ръката на сина му, и ще го дам на тебе, сиреч , десет племена;
\par 36 А на сина му ще дам едно племе, тъй щото слугата Ми Давид да има всякога светилник пред мене в Ерусалим, в града, който избрах за себе Си, за да настаня там името Си.
\par 37 А тебе ще взема, и ти ще царуваш над всичко, което желае душата ти, и ще бъдеш цар над Израиля.
\par 38 И ако слушаш всичко, което ти заповядвам, и ходиш в пътищата Ми, и вършиш всичко , което е право пред Мене, като пазиш повеленията Ми и заповедите Ми, както правеше слугата Ми Давид, тогава ще бъда с тебе, и ще ти съградя непоклатим дом, както съградих на Давида, и ще ти дам Израиля.
\par 39 И чрез това ще оскърбя Давидовия род, но не за винаги.
\par 40 Затова, Соломон поиска да убие Еровоама. А Еровоам, като стана, побягна в Египет при египетския цар Сисак, и остана в Египет до Соломоновата смърт.
\par 41 А останалите дела на Соломона, всичко, което той извърши, и мъдростта му, не са ли записани в книгата на Соломоновите дела?
\par 42 А времето на Соломоновото царуване в Ерусалим, над целия Израил, беше четиридесет години.
\par 43 Така Соломон заспа с бащите си, и беше погребан в града на баща си Давида; а вместо него се възцари син му Ровоам.

\chapter{12}

\par 1 И Ровоам отиде в Сихем; защото в Сихем беше се стекъл целият Израил, за да го направи цар.
\par 2 И Еровоам, Наватовият син, който бе още в Египет, гдето бе побягнал от присъствието на цар Соломона, когато чу това, остана в Египет.
\par 3 Но пратиха та го повикаха. Тогава Еровоам и цялото Израилево общество дойдоха та говориха на Ровоама, като рекоха:
\par 4 Баща ти направи непоносим хомота ни; сега, прочее, ти олекчи жестокото ни работене на баща ти и тежкия хомот, който наложи върху нас, и ще ти работим.
\par 5 А той им рече: Идете си, чакайте за три дена; после се върнете при мене. И людете си отидоха.
\par 6 Тогава цар Ровоам се съветва със старейшините, които бяха служители пред баща му Соломона, когато беше още жив, и им каза: Как ме съветвате да отговоря на тия люде?
\par 7 Те му говориха казвайки: Ако станеш днес слуга на тия люде, да им слугуваш, и им отговориш като им продумаш благи думи, тогава те ще ти бъдат слуги за винаги.
\par 8 Но той отхвърли съвета, който старейшините му дадоха, та се съветва с младите си служители, които бяха пораснали заедно с него.
\par 9 Рече им: Как ме съветвате вие да отговорим на тия люде, които ми говориха, казвайки; Олекчи хомота, който баща ти наложи върху нас?
\par 10 И младежите, които бяха пораснали заедно с него, му отговориха, казвайки: Така да кажеш на тия люде, които ти говориха казвайки: Баща ти направи тежък хомота ни, но ти да ни го олекчиш, - така да им речеш: Малкият ми пръст ще бъде по-дебел от бащиния ми кръст.
\par 11 Сега, ако баща ми ви е товарил с тежък хомот, то аз ще направя още по-тежък хомота ви; ако баща ми ви е наказвал с бичове, то аз ще ви наказвам със скорпии.
\par 12 Тогава Еровоам и всичките люде дойдоха при Ровоама на третия ден, според както царят бе говорил, казвайки: Върнете се при мене на третия ден.
\par 13 И царят отговори на людете остро, като остави съвета, който старейшините му дадоха,
\par 14 и говори им по съвета на младежите та каза: Баща ми направи тежък хомота ви, но аз ще приложа на хомота ви; баща ми ви наказа с бичове, но аз ще ви накажа със скорпии.
\par 15 Така царят не послуша людете; защото това нещо стана от Господа, за да изпълни словото, което Господ бе Говорил чрез силонеца Ахия на Еровоама Наватовия син.
\par 16 А като видя целият Израил, че царят не ги послуша, людете в отговор на царя рекоха: Какъв дял имаме ние в Давида? Никакво наследство нямаме в Есеевия син! В шатрите си Израилю! Промишлявай сега, Давиде за дома си.
\par 17 А колкото за израилтяните, които живееха в Юдовите градове, Ровоам царуваше над тях.
\par 18 Тогава цар Ровоам прати при израилтяните Адорама, който бе над набора; но целият Израил го биха с камъни, та умря. Затова цар Ровоам побърза да се качи на колесницата си, за да побегне в Ерусалим.
\par 19 Така Израил въстана против Давидовия дом и остава въстанал до днес.
\par 20 А когато чу целият Израил, че Еровоам се завърнал, пратиха да го повикат пред обществото, и го направиха цар пред целия Израил; никое друго племе освен Юдовото не последва Давидовия дом.
\par 21 Тогава Ровоам, като дойде в Ерусалим, събра целия Юдов дом и Вениаминовото племе, сто и осемдесет хиляди отборни войници, за да се бият против Израилевия дом та дано възвърнат царството пак на Ровоама Соломоновия син.
\par 22 Но Божието слово дойде към Божия човек Самаия, и рече:
\par 23 Говори на Ровоама Соломоновия син, Юдовия цар и на целия Юдов и Вениаминов дом, и на останалите от людете, като речеш:
\par 24 Така казва Господ: Не излизайте нито се бийте против братята си израилтяните, върнете се всеки у дома си; защото от Мене стана това нещо. И те послушаха Господното слово та се върнаха и си отидоха, според Господното слово.
\par 25 Тогава Еровоам съгради Сихем в хълмистата земя на Ефрема и се зесели в него; после излезе от там та съгради Фануил.
\par 26 А Еровоам рече в сърцето си: Сега ще се повърне царството в Давидовия дом;
\par 27 защото , ако тия люде отиват да принасят жертви на Господния дом в Ерусалим, тогава сърцето на тия люде ще се обърне пак към господаря им Юдовия цар Ровоама, а мене ще убият; и ще се върнат при Юдовия цар Ровоама.
\par 28 Затова, царят като се посъветва, направи две златни телета: и рече на людете : Трудно ви е да възлизате в Ерусалим; ето, боговете ти, Израилю, които те изведоха от Египетската земя;
\par 29 И единия идол постави и във Ветил, а другия постави в Дан.
\par 30 А това нещо беше грях; защото людете, за да се кланят пред всеки от тях, отиваха дори до Дан.
\par 31 Царят направи и капища на високите места, и постави жреци от всякакви люде, които не бяха от Левиевите потомци.
\par 32 Тогава Еровоам установи празник в осмия месец, на петнадесетия ден от месеца, подобен на празника, който става в Юда, и принесе жертва на жертвеника. Така направи и във Ветил, като пожертвува на телетата, които бе направил; и настани във Ветил жреците на високите места, които бе направил.
\par 33 И на петнадесетия ден от осмия месец, в месеца, който бе измислил от сърцето си, принесе жертва на жертвеника, който бе издигнал във Ветил; и установи празник за израилтяните, и възкачи се на жертвеника за да покади.

\chapter{13}

\par 1 И, ето, един Божий човек дойде от Юда във Ветил чрез Господното слово; а Еровоам стоеше при жертвеника за да покади.
\par 2 И човекът извика против жертвеника чрез Господното слово, казвайки: Жертвениче, жертвениче, така говори Господ: Ето, син ще се роди на Давидовия дом, на име Иосия, и ще заколи върху тебе жреците от високите места, които кадят върху тебе; и човешки кости ще се изгорят върху тебе.
\par 3 И в същия ден той даде знамение, като рече: Ето знамението, което изговори Господ: Ето, жертвеникът ще се разцепи, и пепелта, която е на него, ще се разсипе.
\par 4 А когато цар Еровоам чу думите, които Божият човек извика против жертвеника във Ветил, простря ръката си от жертвеника и рече: Хванете го. И ръката му, която простря против него, изсъхна, така щото не можа да я потегли надире към себе си.
\par 5 Също и жертвеникът се разцепи и пепелта се разсипа от жертвеника, според знамението, което Божият човек даде чрез Господното слово.
\par 6 Тогава царят проговаряйки, каза на Божия човек: Изпроси, моля, благоволението на Господа твоя Бог, и моли се за мене, за да ми се възстанови ръката. И тъй, Божият човек се помоли Господу; и ръката на царя му се възстанови и стана както беше по-напред.
\par 7 Тогава царят каза на Божия човек: Дойди с мене у дома и обядвай, и ще ти дам подарък.
\par 8 Но Божият човек рече на царя: Ако щеш да ми дадеш половината от дома си, няма да вляза с тебе, нито ще ям хляб, нито ще пия вода на това място;
\par 9 защото така ми бе заръчано чрез Господното слово, което ми каза: Не яж хляб там , нито пий вода, и не се връщай през пътя, по който бе отишъл.
\par 10 Така той си тръгна по друг път, и не се върна през пътя по който бе дошъл във Ветил.
\par 11 А във Ветил живееше един стар пророк; и синовете му дойдоха та му разказаха всичките дела, които Божият човек бе сторил онзи ден във Ветил; разказаха още на баща си и думите, които бе говорил на царя.
\par 12 И баща им каза: През кой път си отиде? А синовете му бяха видели през кой път си отиде Божият човек, който беше дошъл от Юда.
\par 13 Той, прочее, каза на синовете си: Пригответе ми осела. И те му приготвиха осела; а той, като го възседна,
\par 14 отиде подир Божия човек, и го намери седнал под един дъб; и каза му: Ти ли си Божият човек дошъл от Юда? И рече: аз съм.
\par 15 Тогава му каза: Дойди с мене у дома и яж хляб.
\par 16 А той рече: Не мога да се върна с тебе, нито да сляза у тебе, нито ще ям хляб, нито ще пия вода с тебе на това място;
\par 17 защото ми се каза чрез Господното слово: Да не ядеш хляб, нито да пиеш вода там, нито, като се върнеш, да отидеш през пътя по който си дошъл.
\par 18 А той му каза: И аз съм пророк както си ти; и ангел ми говори чрез Господното слово, казвайки: Върни го със себе си у дома си за да яде хляб и да пие вода. Обаче той го лъжеше.
\par 19 И тъй, човекът се върна с него та яде хляб в къщата му и пи вода.
\par 20 А като седяха на трапезата, Господното слово дойде към пророка, който беше го върнал,
\par 21 та извика към Божия човек, който беше дошъл от Юда, и рече: Така говори Господ, понеже ти не послуша Господното слово, и не опази заповедта, която ти даде Господ твоят Бог,
\par 22 но се върна та яде хляб и пи вода на мястото, за което Той ти рече да не ядеш хляб, нито да пиеш вода там, затова тялото ти няма да се положи в гроба на бащите ти.
\par 23 И след като човекът беше ял хляб и пил, старият пророк оседла осела на пророка, когото бе върнал.
\par 24 А като си отиде той, лъв го намери по пътя и го уби; и тялото му бе простряно на пътя, и оселът стоеше при него; също и лъвът стоеше при тялото.
\par 25 И, ето, човеци като минаваха видяха тялото простряно на пътя и лъва стоящ при тялото; и като дойдоха, известиха това в града, гдето жевееше старият пророк.
\par 26 А когато пророкът, който го беше върнал чу, рече: Това е Божият човек, който не послуша Господното слово; затова го предаде Господ на лъва, та го разкъса и го уби, според словото, което Господ му говори.
\par 27 Тогава говори на синовете си, казвайки: Оседлайте ми осела. И те го оседлаха.
\par 28 И той отиде та намери тялото му простряно на пътя, и оселът и лъвът стоящи при тялото; лъвът не бе изял тялото, нито бе разкъсал осела.
\par 29 И пророкът дигна тялото на Божия човек та го качи на осела и занесе го; и старият пророк дойде в града си за да го оплаче и да го погребе.
\par 30 И положи тялото му в своя ; и плакаха над него, казвайки : Уви, брате мой!
\par 31 И като го погреба, говори на синовете си казвайки: Когато умра, погребете и мене в гроба, гдето е погребан Божият човек; турете костите ми при неговите кости;
\par 32 защото непременно ще се изпълни това, което той извика чрез Господното слово против жертвеника във Ветил, и против всичките капища по високите места, които са в самарийските градове.
\par 33 Но след това Еровоам не се върна от лошия си път, но пак правеше от всякакви люде жреци за високите места, които се в самарийските градове.
\par 34 И това нещо бе грях за Еровоамовия дом, поради което да бъде изтребен и погубен от лицето на земята.

\chapter{14}

\par 1 В онова време се разболя Авия, Еровоамовият син.
\par 2 И Еровоам каза на жена си: Стани, моля, предреши се така, щото да не познаят, че си Еровоамовата жена, и иди в Сило; ето там е пророк Ахия, който ми каза, че ще царувам над тия люде.
\par 3 Вземи със себе си десет хляба, сухари и едно гърне мед та иди при него; той ще ти яви какво ще стане на детето.
\par 4 И тъй, Еровоамовата жена стори така; тя стана та отиде в Сило, и дойде в Ахиевата къща. А Ахия не можеше да вижда, защото очите му бяха се помрачили от старостта му.
\par 5 А Господ беше рекъл на Ахия: Ето, Еровоамовата жена иде да се допита до тебе за сина си, защото е болен. Така и така да й кажеш; защото, когато влезе, ще се престори на друга.
\par 6 И тъй, като чу Ахия стъпването на нозете й като влизаше във вратата, рече: Влез, Еровоамова жено; защо се преструваш на друга? Но аз съм пратен при тебе с тежки известия.
\par 7 Иди, кажи на Еровоама: Така казва Господ Израилевият Бог: Понеже Аз те издигнах отсред людете, и те поставих вожд на людете Си Израиля,
\par 8 и, като откъснаха царството от Давидовия дом, дадоха го на тебе, но пак ти не биде както слугата Ми Давида, който опази заповедите Ми и като Ме следва с цялото си сърце, за да върши само онова, което е пред Мене,
\par 9 но ти надмина в зло всички, които са били преди тебе, защото отиде и си направи други богове и леяни идоли, та Ме разгневи и отхвърли зад гърба си,
\par 10 затова, ето, ще докарам зло на Еровоамовия дом, и ще изтребя от Еровоамовия род всеки от мъжки пол, както малолетния, така пълнолетния в Израиля, и ще измета Еровоамовия дом, както някой измита тора, догде не остане нищо.
\par 11 Който от Еровоамовия род умре в града, него кучетата ще изядат; а който умре в полето, въздушните птици ще го изядат; защото Господ каза това.
\par 12 Ти, прочее, остани, иди у дома си; и като влязат нозете ти в града, детето ще умре.
\par 13 Целият Израил ще го оплаче, и ще го погребат; защото само то от Еровоамовия род ще се положи в гроб, понеже в него измежду Еровоамовия дом се намери нещо добро пред Господа Израилевия Бог.
\par 14 А Господ ще си въздигне цар над Израиля, който ще изтреби Еровоамовия дом в оня ден; но що? даже и сега!
\par 15 И Господ ще порази Израиля като тръст, която се клати у водата; и ще изкорени Израиля из тая добра земя, която е дал на бащите им, и ще ги разпръсне оттатък реката Евфрат , понеже направиха ашерите си та разгневиха Господа.
\par 16 И ще предадете Израиля поради греховете, с които Еровоам съгреши, и с които направи Израиля да съгреши.
\par 17 Тогава Еровоамовата жена стана та си отиде, и дойде в Терса; и като стъпи на прага на къщната врата , детето умря.
\par 18 И целият Израил го оплака, и погребаха го, според словото, което Господ говори чрез слугата Си пророка Ахия.
\par 19 А останалите дела на Еровоама, как воюва и как царува, ето, те са написани в Книгата на летописите на Израилевите царе.
\par 20 Времето, през което Еровоам царуваше, беше двадесет и две години; и той заспа с бащите си, и вместо него се възцари син му Надав.
\par 21 А Ровоам, Соломоновият син царуваше над Юда. Ровоам бе на четиридесет и една година, когато стана цар, и царува седемнадесет години в Ерусалим, града, който Господ бе избрал измежду всичките Израилеви племена за да настани името Си там. Името на майка му, амонката, бе Наама.
\par 22 А Юда върши зло пред Господа; и раздразниха Го до ревнуване с извършените си грехове, които бяха повече от всичко, що бяха извършили бащите им.
\par 23 Защото и те си издигнаха високи места и кумири и ашери на всеки висок хълм и под всяко зелено дърво.
\par 24 Още в земята имаше мъжеложци, които постъпваха според всичките мерзости на народите, които Господ беше изгонил отпред израилтяните.
\par 25 И в петата година от Ровоамовото царуване, египетският цар Сисак дойде против Ерусалим,
\par 26 та отнесе съкровищата на Господния дом и съкровищата на царската къща; отнесе всичко; отнесе още всички златни щитове, които Соломон бе направил.
\par 27 А вместо тях цар Ровоам направи медни щитове и ги предаде в ръцете на началниците на телохранителите, които пазеха вратата и царската къща.
\par 28 И когато влизаше царят в Господния дом, телохранителите ги държаха: после пак ги занасяха в залата на телохранителите.
\par 29 А останалите дела на Ровоама, и всичко що извърши, не са ли написани в Книгата на летописите на Юдовите царе?
\par 30 А между Ровоама и Еровоама имаше постоянна война.
\par 31 И Ровоам заспа с бащите си, и погребан биде с бащите си в Давидовия град. Името на майка му, амонката, бе Наама. А вместо него се възцари син му Авия.

\chapter{15}

\par 1 В осемнадесетата година от царуването на Еровоама Наватовия син, Авия се възцари над Юда;
\par 2 и царува три години в Ерусалим. Името на майка му бе Мааха, Авесаломова дъщеря.
\par 3 Той ходи във всичките грехове, които баща му бе сторил преди него; и сърцето му не бе съвършенно пред Господа неговия Бог, както сърцето на баща му Давида.
\par 4 Но заради Давида Господ неговият Бог му даде светилник в Ерусалим, като издигна сина му подир него и утвърди Ерусалим;
\par 5 защото Давид стори онова, което бе право пред Господа, и през всичките дни на живота си, не се отклоняваше от нищо, което му заповяда, освен в делото относно хетееца Урия.
\par 6 И между Ровоама е Еровоама имаше война през всичките дни на Ровоамовия живот.
\par 7 А останалите дела на Авия, и всичко що извърши, не са ли написани в книгата на Юдовите царе? Имаше война и между Авия и Еровоама.
\par 8 И Авия заспа с бащите си, и погребаха го в Давидовия град; и вместо него се възцари син му Аса.
\par 9 Аса се възцари над Юда в двадесетата година на Израилевия цар Еровоама,
\par 10 и царува в Ерусалим четиридесет и една година. Името на майка му бе Мааха, Авесаломова дъщеря.
\par 11 И Аса върши това, което бе право пред Господа, както баща му Давид.
\par 12 Той отмахна мъжеложците от земята, и дигна всичките идоли, които бяха направили бащите му..
\par 13 А още свали и майка си Мааха да не бъде царица, понеже тя бе направила отвратителен идол на Ашера; и Аса съсече нейния идол та го изгори при потока Кедрон.
\par 14 Но високите места не се премахнаха; сърцето обаче на Аса бе съвършено пред Господа през всичките му дни.
\par 15 И Той донесе в Господния дом посветените от баща му вещи, и посветените от самия него вещи, сребро, злато и съдове.
\par 16 И между Аса и Израилевия цар Вааса имаше война през всичките им дни.
\par 17 И Израилевият цар Вааса като възлезе против Юда, съгради Рама, за да не оставя никого да излиза от Юдовия цар Аса, нито да влиза при него.
\par 18 Тогава Аса все всичкото сребро и злато, останало в съкровищата на Господния дом и в съкровищата на царската къща, та ги даде в ръцете на слугата си; и цар Аса ги прати при сирийския цар Венадад, син на Тавримона, син на Есиона, който живееше в Дамаск, да рекат:
\par 19 Нека има договор между мене и тебе, както е имало между баща ми и твоя баща; ето, пратих ти подарък сребро и злато; иди, развали договора си с Израилевия цар Вааса, за да се оттегли от мене.
\par 20 И Венадад послуша цар Аса та прати началниците на силите си против Израилевите градове, и порази, Иион, Дан, Авел-вет-мааха и целия Хинерот с цялата Нефталимова земя.
\par 21 А Вааса, като чу това, престана да гради Рама, и установи се в Терса.
\par 22 Тогава цар Аса възгласи на целия Юда, без никакво изключение та дигнаха камъните на Рама и дърветата й, с които Вааса градеше; и с тях цар Аса съгради Гава Вениаминова и Масфа.
\par 23 А всичките останали дела на Аса, всичкото му юначество, всичко що извърши, и градовете, които съгради, не са ли написани в Книгата на летописите на Юдовите царе? А във времето на старостта си той се разболя от болест в нозете.
\par 24 И Аса заспа с бащите си, и биде погребан с бащите си в града на баща си Давида; а вместо него се възцари син му Иосафат.
\par 25 А във втората година на Юдовия цар Аса, над Израиля се възцари Надав Еровоамовия син, и царува над Израиля две години.
\par 26 Той върши зло пред Господа, като ходи в пътя на баща си, и в греховете му, чрез които направи Израиля да греши.
\par 27 А Вааса, Ахиевият син, от Исахаровия дом, направи заговор против него; и Вааса го уби в Гиветон, който принадлежеше на филистимците; защото Надав и целият Израил обсаждаха Гиветон.
\par 28 А Вааса го уби в третата година на Юдовия цар Аса, и се възцари вместо него.
\par 29 И щом се възцари, изби целия Еровоамов род; не остави на Еровоама нищо живо, което не изтреби, според словото, което Господ говори чрез слугата Си Ахия силонеца,
\par 30 поради греховете, с които Еровоам съгреши, и чрез които направи Израиля да греши, и поради раздразнението, с което предизвика гнева на Господа Израилевия Бог.
\par 31 А останалите дела на Надава, и всичко що извърши, не са ли написани в Книгата на летописите на Израилевите царе?
\par 32 И между Аса и Израилевия цар Вааса имаше война през всичките им дни.
\par 33 В третата година на Юдовия цар Аса се възцари в Терса Вааса Ахиевият син над целия Израил и царува двадесет и четири години.
\par 34 Той върши зло пред Господа, като ходи в пътя на Еровоама, и в греха му, чрез който направи Израил да греши.

\chapter{16}

\par 1 Тогава дойде Господното слово към Ииуя Ананиевия син против Вааса, и рече:
\par 2 Понеже, като те издигнах от пръстта и те поставих вожд на людете Си Израиля, ти ходи в пътя на Еровоама, и направи людете Ми Израиля да съгрешат та да Ме разгневят с греховете си,
\par 3 ето, изтребвам съвършено Вааса и рода му, и ще направя твоя род както рода на Еровоама Наватовия син.
\par 4 Който от Ваасовия род умре в града, него кучетата ще изядат; а който негов умре в полетата, него въздушните птици ще изядат.
\par 5 А останалите дела на Вааса, и онова, което извърши, и юначеството му, не са ли написани в Книгата на летописите на Израилевите царе?
\par 6 И Вааса заспа с бащите си, и биде погребан в Терса; а вместо него се възцари син му Ила.
\par 7 При това, Господното слово дойде чрез пророка Ииуй, Ананиевия син, против Вааса и против рода му, както поради всичките злини, които извърши пред Господа та Го разгневи с делата на ръцете си и стана подобен на рода на Еровоама, така и понеже порази тоя род .
\par 8 В двадесет и шестата година на Юдовия цар Аса, Ила Ваасовият син се възцари над Израиля в Терса, и царува две години.
\par 9 А слугата му Зимрий, началник на половината от военните му колесници, направи заговор против него, когато беше в Терса, та пиеше и се опиваше в къщата на домоуправителя си Арса в Терса.
\par 10 Зимрий влезе, та като го порази уби го, в двадесет и седмата година на Юдовия цар Аса, и възцари се вместо него.
\par 11 А щом се възцари и седна на престола си, веднага порази целия Ваасов род; не му остави никого от мъжки пол, нито сродниците му, нито приятелите му.
\par 12 Така Зимрий изтреби целия Ваасов род, според словото, което Господ говори против Вааса чрез пророка Ииуй,
\par 13 за всичките грехове на Вааса, и за греховете на сина му Ила, с които те съгрешиха, и чрез които направиха Израиля да съгреши, та разгневиха Господа Израилевия Бог със суетите си.
\par 14 А останалите дела на Ила, и всичко, което извърши, не са ли написани в Книгата на летописите на Израилевите царе?
\par 15 В двадесет и седмата година на Юдовия цар Аса, Зимрий царува седем дена в Терса. А людете бяха разположени в стан против Гиветон, който принадлежеше на филистимците.
\par 16 И людете, които бяха в стана, като чуха да казват: Зимрий направил заговор, при това убил и царя, то същия ден, в стана, целият Израил направи военачалника на войската Амрий цар над Израиля.
\par 17 Тогава Амрий и целият Израил с него възлязоха от Гиветон та обсадиха Терса.
\par 18 А Зимрий, като видя, че градът беше превзет, влезе във вътрешната царска къща та изгори царската къща с огън, и себе си в нея, и умря,
\par 19 за греховете, с които съгреши, като върши зло пред Господа, и ходи в пътя на Еровоама и в греха, който стори, като направи Израиля да греши.
\par 20 А останалите дела на Зимрия, и измяната, която извърши не са ли написани в Книгата на летописите на Израилевите царе?
\par 21 Тогава Израилевите люде се разделиха на две части: половината от людете последваха Тивния Гинатовия син, за да направят него цар, а половината последваха Амрия.
\par 22 Но людете, които последваха Амрия, надделяха над людете, които последваха Тивния Гинатовия син; и Тивний умря, а възцари се Амрий.
\par 23 В тридесет и първата година на Юдовия цар Аса, се възцари Амрий над Израиля и царува дванадесет години. И когато беше царувал шест години в Терса,
\par 24 купи самарийския хълм от Семера за два таланта сребро; и съгради град на хълма и нарече града, който съгради, Самария, по името на Семера, притежателя на хълма.
\par 25 Амрий върши зло пред Господа, даже вършеше по-зле от всичките що бяха преди него;
\par 26 защото ходеше във всичките пътища на Еровоама Наватовия син, и в неговия грях, чрез който направи Израиля да греши та да разгневят Господа Израилевия Бог със суетите си.
\par 27 А останалите дела, които Амрий направи, и юначествата, които показа, не са ли написани в Книгата на летописите на Израилевите царе?
\par 28 И Амрий заспа с бащите си, и биде погребан в Самария; а вместо него се възцари син му Ахаав.
\par 29 И в тридесет и осмата година на Юдовия цар Аса, се възцари над Израиля Ахаав Амриевият син; и Ахаав Амриевият син царува над Израиля в Самария двадесет и две години.
\par 30 А Ахаав, Амриевият син, върши зло пред Господа повече от всичките, които бяха преди него.
\par 31 И като, че беше малко това, че ходеше в греховете на Еровоама Наватовия син, при това той взе за жена Езавел, дъщеря на сидонския цар Етваал, и отиде да служи на Ваала и да му се покланя.
\par 32 И издигна жертвеник на Ваала във Вааловото капище, което построи в Самария.
\par 33 Ахаав направи и ашера; тъй че от всичките Израилеви царе, които бяха преди него, Ахаав извърши най-много да разгневи Господа Израилевия Бог.
\par 34 В неговите дни ветилецът Хиил съгради Ерихон; тури основите му със смъртта на първородния си син Авирон, и постави вратите му със смъртта на най-младия си син Селув, според словото, което Господ говори чрез Исуса Навиевия син.

\chapter{17}

\par 1 А тесвиецът Илия, който бе от галаадските жители, рече на Ахава: В името на живия Господ, Израилевия Бог, Комуто служа, заявявам ти че през тия години няма да падне роса или дъжд освен чрез дума от мене.
\par 2 И Господното слово дойде към него и рече:
\par 3 Иди от тука, обърни се към изток, и скрий се при потока Херит, който е срещу Иордан.
\par 4 Ще пиеш от потока; а на враните заповядах да те хранят там.
\par 5 И той отиде та стори според Господното слово; защото отиде и седна при потока Херит, който е срещу Иордан.
\par 6 И враните му донасяха хляб и месо заран, и хляб и месо вечер; а той пиеше от потока.
\par 7 А след известно време потокът пресъхна, понеже на валя дъжд по земята.
\par 8 Тогава Господното слово дойде към него и рече:
\par 9 Стани, иди в Сарепта сидонска и седи там; ето, заповядах на една вдовица там, да те храни.
\par 10 И тъй, той стана та отиде в Сарепта. И като дойде при градската порта, ето там една вдовица, която събираше дърва; и той извика към нея и рече: Донеси ми, моля, малко вода в съд да пия.
\par 11 И като отиваше да донесе, той извика към нея и рече: Донеси ми, моля и залък хляб в ръката си.
\par 12 А тя рече: Заклевам се в живота на Господа твоя Бог, нямам ни една пита, но само една шепа брашно в делвата и малко дървено масло в стомната и, ето, събирам две дръвчета, за да ида и да го приготвя за мене и за сина ми да го изядем и да умрем.
\par 13 А Илия й рече: Не бой се; иди, стори както каза; но омеси от него първо за мене една малка пита та ми донеси, а после приготви за себе си и за сина си;
\par 14 защото така казва Господ Израилевият Бог: Делвата с брашното няма да се изпразни, нито стомната с маслото ще намалее, до деня, когато Господ даде дъжд на земята.
\par 15 И тя отиде та стори според каквото каза Илия; и тя и той и домът й ядоха много дни.
\par 16 Делвата с брашното не се изпразни, нито стомната с маслото намаля, според словото, което Господ говори на Илия.
\par 17 А след това, синът на жената, домакинята, се разболя; и болестта му бе тъй тежка, щото не остана дишане в него.
\par 18 Тогава тя рече на Илия: Какво има между мене и тебе, Божий човече? Дошъл ли си при мене, за да ми припомниш греховете и да умориш сина ми?
\par 19 А той каза: Дай ми сина си. И като го взе от пазухата й та го изнесе на горната стая, гдето живееше, положи го на леглото си.
\par 20 И извика към Господа и рече: Господи Боже мой! нанесъл ли си зло и на вдовицата, при която живея, като си уморил сина й?
\par 21 Тогава той се простря три пъти върху детето, и извика към Господа, казвайки: Господи Боже мой, моля Ти се, нека се върне душата на това дете в него.
\par 22 И Господ послуша Илиевия глас, та се върна душата на детето в него и то оживя.
\par 23 Тогава Илия взе детето та го занесе от горната стая в къщата и даде го на майка му; и Илия рече: Виж, синът ти е жив.
\par 24 И жената каза на Илия: Сега познавам, че си Божий човек, и че Господното слово, което говориш, е истина.

\chapter{18}

\par 1 А след дълго време, в третата година, Господното слово дойде към Илия и рече: Иди, яви се на Ахава; и ще дам дъжд на земята.
\par 2 Илия, прочее, отиде да се яви на Ахава. А гладът бе тежък в Самария.
\par 3 А Ахав беше повикал домоуправителя Авдия. (А Авдия се боеше много от Господа;
\par 4 защото, когато Езавел изтребваше Господните пророци, Авдия бе взел сто пророка та бе ги скрил, петдесет в една пещера и петдесет в друга, и беше ги хранил с хляб и вода).
\par 5 И Ахаав беше казал на Авдия: Обиколи земята и иди към всичките водни извори и към всичките потоци, дано намерим трева за да запазим живота на конете и на мъските, и да не се лишим от животните.
\par 6 И тъй, те бяха разделили земята помежду си, за да я обиколят: Ахаав беше отишъл съм по един път, а Авдия беше отишъл сам по друг път.
\par 7 И като беше Авдия на пътя, ето, Илия го срещна; и той го позна и падна на лице и рече: Ти ли си, господарю мой Илие:
\par 8 А той каза: Аз съм. Иди, кажи на господаря си: Ето Илия.
\par 9 А той каза: В що съм съгрешил, та искаш да предадеш слугата си в ръката на Ахаава, за да ме убие?
\par 10 Заклевам ти се в живота на Господа твоя Бог, че няма народ или царство, гдето да не е пращал господарят ми да те търси; и когато кажеха: Няма го тук , той зеклеваше царството и народа, че не са те намерили.
\par 11 А сега ти казваш: Иди, кажи на господаря си: Ето Илия.
\par 12 А щом се отделя от тебе, Господният Дух ще те отведе, гдето аз не знай; и така, когато отида да известя на Ахаава, че си тук , и той не те намери, ще ме убие. Но аз, твоят слуга, се боя от Господа още от младостта си.
\par 13 Не е ли известно на господаря ми що сторих, когато Езавел убиваше Господните пророци, как скрих сто души от Господните пророци, петдесет в една пещера и петдесет в друга, и храних ги с хляб и вода?
\par 14 А сега ти казваш: Иди, кажи на господаря си: Ето Илия; и той ще ме убие!
\par 15 Но Илия рече: Заклевам ти се в живота на Господа на Силите, Комуто служа, днес ще му се явя.
\par 16 И тъй, Авдия отиде да посрещне Ахаава и му извести. А Ахаав отиде да посрещне Илия.
\par 17 А като видя Илия, Ахаав му рече: Ти ли си, смутителю на Израиля?
\par 18 А той отговори: Не смущавам аз Израиля, но ти и твоят бащин дом; защото вие оставихте Господните заповеди, и ти последва ваалимите.
\par 19 Сега, прочее, прати та събери при мене целия Израил на планината Кармил, и четиристотин и петдесет Ваалови пророци, и четирите стотин пророци на Ашера, които ядат на Езавелината трапеза.
\par 20 И така, Ахаав прати до всичките израилтяни та събра пророците на планината Кармил.
\par 21 Тогава Илия дойде при всичките люде та рече: До кога ще се колебаете между две мнения? Иеова, ако е Бог, следвайте Го; но ако е Ваал, следвайте него. А людете не му отговориха ни дума.
\par 22 Тогава Илия рече на людете: Само аз останах Господен пророк; а Вааловите пророци са четиристотин и петдесет мъже.
\par 23 Те, прочее, нека ни дадат две юнеца; и нека изберат единия юнец за себе си, нека го разсекат и го турят на дървата, но огън да не турят отдолу; и аз ще приготвя другия юнец и ще го туря на дървата, но огън няма да туря отдолу.
\par 24 Тогава вие призовете името на вашия бог, и аз ще призова името на Господа; и оня бог, който отговори с огън, той нека е Бог. И всичките люде в отговор казаха: Добро е каквото си казал.
\par 25 И тъй, Илия каза на Вааловите пророци: Изберете си единия юнец та го пригответе вие първо, защото сте мнозина; и призовете името на бога си, огън обаче не туряйте отдолу.
\par 26 И те взеха юнеца, който им се даде, та го приготвиха, и призоваха името на Ваала от сутринта дори до пладне, като викаха: Послушай ни, Ваале! Но нямаше глас, нито кой да отговори; и те скачаха около жертвеника, който бяха издигнали.
\par 27 А около пладне Илия им се присмиваше като казваше: Викайте със силен глас, защото е бог! той или размишлява, или има някаква работа, или е на път, или - може би - спи и трябва да се събуди.
\par 28 И те викаха със силен глас и режеха се според обичая си с мечове и с ножове, догде бликна кръв от тях.
\par 29 И като мина пладне, те пророкуваха до часа на вечерния принос; но нямаше глас нито кой да отговори, нито кой да внимава.
\par 30 Тогава Илия каза на всичките люде: Приближете се при мене. И всичките люде се приближиха при него. И той поправи Господния олтар, който беше съборен;
\par 31 защото Илия взе дванадесет камъни, според числото на племената на синовете на Якова, към когото дойде Господното слово и рече: Израил ще бъде името ти.
\par 32 И с камъните издигна олтар в Господното име; и около олтара направи окоп, доволно голям да побира две сати семе.
\par 33 И като нареди дървета, насече юнеца на късове та го положи на дърветата, и каза: Напълнете четири бъчви с вода, та излейте на всеизгарянето и на дървата.
\par 34 И рече: Повторете. И повториха.
\par 35 И водата обикаляше около олтара, още и окопът се напълни с вода.
\par 36 А в часа на вечерния принос, пророк Илия се приближи и каза: Господи, Боже Авраамов, Исаков и Израилев, нека стане известно днес, че Ти си Бог в Израиля, и аз Твой слуга, и че според Твоето слово аз сторих всички тия неща.
\par 37 Послушай ме, Господи, послушай ме, за да познаят тия люде, че Ти, Господи, си Бог, и че Ти си възвърнал сърцата им надире.
\par 38 Тогава огън от Господа падна та изгори всеизгарянето, дървата, камъните и пръстта, и облиза водата, която бе в окопа.
\par 39 И всичките люде, когато видяха това, паднаха на лицата си и рекоха: Иеова, Той е Бог; Иеова, Той е Бог.
\par 40 И Илия им каза: Хванете Вааловите пророци; ни един от тях да не избяга. И хванаха ги; и Илия ги заведе при потока Кисон и там ги изкла.
\par 41 Тогава Илия каза на Ахаава: Качи се, яж и пий, защото се чува глас на изобилен дъжд.
\par 42 И тъй, Ахаав възлезе да яде и да пие: а Илия се възкачи на връх Кармил, и като се наведе до земята тури лицето си между коленете си,
\par 43 и рече на слугата си: Възлез сега, погледни към морето. И той възлезе та погледна и рече: Няма нищо. А Илия рече: Иди пак, до седем пъти.
\par 44 И седем пъти той рече: Ето, малък облак колкото човешка длан, се издига от морето. Тогава Илия каза: Иди, кажи на Ахаава: Впрегни колесницата си та слез, за да те не спре дъждът.
\par 45 А между това небето се помрачи от облаци и вятър, и заваля силен дъжд. И Ахаав, возейки се, отида в Езраил.
\par 46 И Господната ръка, бидейки върху Илия, той стегна кръста си, та се завтече пред Ахаава до входа на Езраел.

\chapter{19}

\par 1 И Ахаав съобщи на Езавел всичко, що бе сторил Илия, и как бе избил с меч всичките пророци.
\par 2 Тогава Езавел прати човек до Илия да каже: Така да ми направят боговете, да! и повече да притурят, ако утре, около тоя час, не направя твоя живот както живота на един от тях.
\par 3 А като видя това, Илия стана и отиде за живота си, и като дойде във Вирсавее Юдово, остави слугата си там.
\par 4 А сам той отиде на еднодневен път в пустинята, и дойде та седна под една смрика: и поиска за себе си да умре, казвайки: Доволно, е, сега, Господи, вземи душата ми, защото не съм по-добър от бащите си.
\par 5 Тогава легна и заспа под смриката; после, ето, ангел се допря до него и му рече: Стани, яж.
\par 6 И погледна, и ето при главата му пита печена на жаравата и стомна с вода. И яде и пи, и пак легна.
\par 7 А ангелът Господен дойде втори път та се допря до него, и рече: Стани, яж, защото пътят е много дълъг за тебе.
\par 8 И той стана та яде и пи, и със силата от онова ястие пътува четиридесет да не и четиридесет нощи до Божията планина Хорив.
\par 9 И там слезе в една пещера, гдето се зесели; и ето, Господното слово дойде към него та му рече: Що правиш тук, Илие?
\par 10 А той каза: Аз съм бил много ревнив за Господа Бога на Силите; защото израилтяните оставиха завета Ти, събориха олтарите Ти и избиха с меч пророците Ти; само аз останах, но и моя живот искат да отнемат.
\par 11 И словото му каза: Излез та застани на планината пред Господа. И, ето, Господ минаваше и голям силен вятър цепеше бърдата и сломяваше скалите пред Господа, но Господ не бе във вятъра; а подир вятъра земетръс, но Господ не бе в земетръса;
\par 12 И подир земетръса огън, но Господ не бе в огъня; а подир огъня тих и тънък глас.
\par 13 И Илия, като го чу, покри лицето си с кожуха си, излезе и застана при входа на пещерата. И, ето, глас дойде към него, който рече: Що правиш тук Илие?
\par 14 И той каза: Аз съм бил много ревнив за Господа Бога на Силите; защото израилтяните оставиха завета Ти, събориха олтарите Ти, и избиха с меч пророците Ти; само аз останах но и моя живот искат да отнемат.
\par 15 Но Господ му рече: Иди, върни се по пътя си през пустинята в Дамаск, и когато пристигнеш, помажи Азаила за цар над Сирия;
\par 16 а Ииуя Намесиевия син помажи за цар над Израиля: и Елисея Сафатовия син, от Авелмеола, помажи за пророк вместо тебе.
\par 17 И ще стане, че който се избави от Азаиловия меч него Ииуй ще убие; и който се избави от Ииуевия меч, него Елисей ще убие.
\par 18 Оставил съм Си, обаче, в Израиля седем хиляди души, всички ония, които не са преклонили колена пред Ваала, и всички, чиито уста не са го целували.
\par 19 И тъй, Илия тръгна от там и намери Елисея Сафатовия син, който ореше с дванадесет двойки волове пред себе си; и сам бе с дванадесетата; и Илия мина към него и хвърли кожуха си върху него.
\par 20 А той остави воловете та се завтече подир Илия и рече: Нека целуна, моля, баща си и майка си, и тогава ще те последвам. А Илия му каза: Иди, върни се, защото какво съм ти сторил?
\par 21 И той се върна отподире му, та взе двойката волове и ги закла, а с приборите на воловете опече месото им, и даде на людете, та ядоха. Тогава стана та последва Илия, и му слугуваше.

\chapter{20}

\par 1 Тогава сирийският цар Венадад, събра цялата си войска; (а имаше с него тридесет и двама царе, и коне и колесници); и влезе та обсади Самария и воюваше против нея.
\par 2 И прати човеци на Израилевия цар Ахаава, в града, да му кажат: Така казва Венадад:
\par 3 Среброто и златото ти е мое; тоже и жените ти и най-добрите от чадата ти са мои.
\par 4 И Израилевият цар в отговор рече: Според както казваш, господарю мой царю, твой съм аз и все що имам.
\par 5 А пратениците пак дойдоха та рекоха: Така говори Венадад, казвайки: Наистина пратих до тебе да кажат: Ще дадеш на мене среброто си, златото си, жените си и чадата си;
\par 6 обаче утре, около тоя час, ще пратя слугите си при тебе, които да претърсят къщата ти и къщите на слугите ти; и все що ти се вижда желателно ще го турят в ръцете си и ще го отнемат.
\par 7 Тогава Израилевият цар повика всичките старейшини на страната та рече: Забележете, моля и вижте, че тоя човек търси повод за зло; защото прати до мене за жените ми, за чадата ми, за среброто ми и за златото ми, и не му отказах нищо.
\par 8 А всичките старейшини и всичките люде му рекоха: Да не го послушаш, нито да склониш.
\par 9 Затова, той каза на Венададовите пратеници: Кажете на господаря ми царя: Всичко каквото ти заръча на слугата си изпърво ще го направя, но това нещо не мога да направя. И тъй, пратениците си отидоха та му занесоха отговор.
\par 10 Тогава Венадад прати до него да кажат: Така да ми направят боговете, да! и повече да притурят, ако пръстта на Самария бъде достатъчна по за една шепа на всичките люде, които ме следват.
\par 11 Но Израилевият цар в отговор рече: кажете му: Ония, който опасва оръжия , нека не се хвали както оня, който ги разпасва.
\par 12 А когато Венадад чу тая дума, той и царете, които бяха с него , пиеха в шатрите: и рече на слугите си: Опълчете се. И те се опълчиха против града.
\par 13 И, ето, един пророк дойде против Израилевия цар Ахаав и каза: Така казва Господ, виждаш ли цялото това голямо множество? ето, Аз днес го предавам в ръцете ти; и ще познаеш, че Аз съм Господ.
\par 14 А Ахаав рече: Чрез кого? А той каза: Така казва Господ. Чрез слугите на областните управители. Тогава рече: Кой ще почне сражението? И той отговори: Ти.
\par 15 Тогава Ахаав събра слугите на областните управители, които бяха двеста и тридесет и двама души, и след тях събра всичките люде, всичките израилтяни, които бяха седем хиляди души;
\par 16 и те излязоха около пладне. А Венадад пиеше и се опиваше в шатрите, той и царете, тридесет и двамата съюзници с него царе.
\par 17 Първи излязоха слугите на областните управители; и когато прати Венадад да се научи , известиха му казвайки: Мъже излязоха из Самария.
\par 18 А той рече: Било, че са излезли с мир, хванете ги живи, или ако са излезли за бой, пак ги живи хванете.
\par 19 И тъй, слугите на областните управители и войската, която ги следваше, излязоха из града.
\par 20 И всеки уби човека насреща си; и сирийците побягнаха, и Израил ги преследва; а сирийският цар Венадад се отърва на кон с конниците.
\par 21 Тогава Израилевият цар излезе та порази бягащите на конете и в колесниците, и нанесе на сирийците голямо поражение.
\par 22 След това, пророкът дойде при Израилевия цар и му рече: Иди, укрепи се: размисли, и внимавай какво да направиш; защото, след като се измине една година, сирийският цар ще дойде против тебе.
\par 23 А слугите на сирийския цар му рекоха: Техният бог е планински бог, затова израилтяните надделяха над нас; но ако се бием с тях на полето, непременно ние ще надделеем над тях.
\par 24 Затова, ето какво да направиш: отмахни всеки от царете от мястото му, и вместо тях постави военачалници.
\par 25 После ти си събери войска колкото войската, която ти изгуби, кон вместо кон, колесница вместо колесница; и нека се бием с тях на полето, и непременно ще надделеем над тях. И той послуша гласа им та направи така.
\par 26 Тогава, след като се измина една година, Венадад събра сирийците и възлезе в Афек, за да се бие с Израиля.
\par 27 И израилтяните, като бяха събрани и продоволствувани, отидоха насреща им; и израилтяните разположиха стана си срещу тях, като две малки кози стада, а сирийците изпълниха страната.
\par 28 Тогава Божият човек дойде та говори на Израилевия цар, казвайки: Така говори Господ, понеже сирийците рекоха: Иеова е Бог на планините, а не Бог на долините, затова ще предам в ръката ти цялото това голямо множество; и ще познаете, че Аз съм Иеова.
\par 29 И стояха разположени в стан един срещу други седем дена, а на седмия ден завързаха бой; и израилтяните избиха от сирийците сто хиляди пешаци в един ден.
\par 30 А останалите, като побягнаха до Афек, в града, стената падна върху двадесет и седем хиляди от останалите мъже. Побягна и Венадад та влезе в града, гдето се криеше из клет в клет.
\par 31 Тогава слугите му му казаха: Ето, сега, чули сме, че царете от Израилевия дом били милостиви царе; нека турим, прочее, вретища на кръстовете си и въжета на главите си, и нека излезем при Израилевия цар, дано би ти пощадил живота.
\par 32 И тъй, като се препасаха с вретища около кръстовете си и туриха въжета на главите си, дойдоха при Израилевия цар та рекоха: Слугата ти Венадад казва: Моля ти се, остави ме да живея. А той каза: Жив ли е още? брат ми е.
\par 33 А мъжете взеха това за добър знак, и побързаха да го уловят от него, и казаха: Брат ти Венадад. И той каза: Идете, доведете го. И когато дойде Венадад при него, той го качи на колесницата си.
\par 34 Тогава Венадад му рече: Градовете, които баща ми превзе от твоя баща, ще ти ги върна; и ти си направи улици в Дамаск както си направи баща ми в Самария. А Ахаав отговори : И аз ще те пусна с тоя договор. Така направи договор с него и го пусна.
\par 35 Тогава един човек от пророческите ученици рече на другаря си чрез Господното слово: Удари ме, моля. Но човекът отказа да го удари.
\par 36 Затова, той му рече: Понеже ти не послуша Господния глас, ето, щом си тръгнеш от мене, лъв ще те убие. И щом тръгна от него, намери го лъв и го уби.
\par 37 Сетне той намери друг човек и рече: Удари ме, моля. И човекът го удари и с удара го нарани.
\par 38 Тогава пророкът си отиде и чакаше на пътя за царя, като бе се предрешил с покривало на очите си.
\par 39 И когато минаваше царят, той извика към царя и рече: Слугата ти отиде всред сражението; и ето, един човек, като се отби на страна доведе едного при мене и рече: Пази тоя човек; ако побегне някак, тогава твоят живот ще бъде вместо неговия живот, или ще платиш един талант сребро.
\par 40 И като се занимаваше слугата ти тук-таме, той се изгуби. А Израилевият цар му каза: Ето присъдата ти: Ти сам си я изрекъл.
\par 41 Тогава той побърза та дигна покривалото от очите си; и Израилевият цар го позна, че беше един от пророците.
\par 42 А той му рече: Така казва Господ, понеже ти пусна от ръката си човека, когото Аз бях обрекъл на погубване, затова твоят живот ще бъде вместо неговия живот, и твоите люде вместо неговите люде.
\par 43 И Израилевият цар дойде в Самария и отиде у дома си тъжен и огорчен.

\chapter{21}

\par 1 След тия събития, понеже езраелецът Навутей имаше лозе в Езраил, близо до палата на самарийския цар Ахаава,
\par 2 Ахаав говори на Навутея казвайки: Дай ми лозето си да го имам за бостан, понеже е близо до къщата ми; и вместо него ще ти дам лозе по-добро от него, или, ако ти се види добре, ще ти дам стойността му в пари.
\par 3 А Навутей рече на Ахаава: Да ми не даде Господ да ти дам бащиното си наследство.
\par 4 И Ахаав дойде у дома си тъжен и огорчен поради думата, която езраелецът Навутей му каза, като рече: Не ща да ти дам бащиното си наследство. И като легна на леглото си, отвърна лицето си и не яде хляб.
\par 5 Тогава жена му Езавел дойде при него та му рече: Защо е духът ти тъжен, та не ядеш хляб?
\par 6 А той каза: Понеже говорейки на езраелеца Навутей рекох му: Дай ми лозето си с пари, или, ако обичаш, ще ти дам друго лозе вместо него; а той отговори: Не ща да ти дам лозето си.
\par 7 А жена му Езавел му рече: Царуваш ли ти наистина над Израиля? Стани, яж хляб, и нека е весело на сърцето ти; аз ще ти дам лозето на езраелеца Навутей.
\par 8 И така, тя писа писма от Ахаавово име, и като ги запечата с печата му, прати писмата до старейшините и благородните, които бяха в града му, живеещи с Навутея.
\par 9 В писмата писа, казвайки: Прогласете пост, и поставете Навутея на видно място пред людете;
\par 10 и срещу него поставете двама лоши човеци да засвидетелствуват против него казвайки: Ти похули Бога и царя. Тогава го изведете вън и убийте го с камъни, и нека умре.
\par 11 И мъжете от града му, старейшините и благородниците, които живееха в града му, сториха според заповедта, която Езавел им бе пратила, според написаното в писмата, които им бе пратила;
\par 12 прогласиха пост, и поставиха Навутея на видно място пред людете.
\par 13 И двамата лоши човеци влязоха и седнаха пред него; и лошите човеци свидетелствуваха против него, против Навутея, пред людете, казвайки: Навутей похули Бога и царя. Тогава го изведоха вън от града, та го убиха с камъни; и умря.
\par 14 После пратиха да Езавел да кажат: Навутей е убит с камъни и умря.
\par 15 И като чу Езавел, че Навутей бил убит с камъни и умрял, Езавел рече на Ахаава: Стани, присвои си лозето, което езраелецът Навутей отказа да ти даде с пари; защото Навутей не е жив, но е умрял.
\par 16 И като чу Ахаав, че Навутей е умрял, Ахаав стана та слезе в лозето на езраелеца Навутей за да го присвои.
\par 17 Но Господното слово дойде към тесвиеца Илия и рече:
\par 18 Стани, слез да посрещнеш Израилевия цар Ахаава, който живее в Самария; ето, той е в Навутеевото лозе, гдето слезе да го присвои.
\par 19 И да му говориш казвайки: Така казва Господ: Уби ли ти, а още присвои ли ти? Да му говориш още казвайки: Така казва Господ: На мястото, гдето кучетата лизаха Навутеевата кръв, кучетата ще лижат твоята кръв, да! твоята.
\par 20 А Ахаава каза на Илия: Намери ли ме, враже мой? А той отговори: Намерих те, защото ти си продал себе си да вършиш зло пред Господа.
\par 21 Ето казва Господ : Аз ще докарам зло върху тебе, ще те измета, и ще изтребя от Ахаава всеки от мъжки пол, както малолетния, така и пълнолетния в Израиля;
\par 22 и ще направя дома ти като дома на Еровоама Наватовия син, и като дома на Вааса Ахиевия син, поради раздразнението, с което Ме ти разгневи, като направи Израиля да съгреши.
\par 23 Също и на Езавел говори Господ, казвайки: Кучетата ще изядат Езавел при рова на Езраел.
\par 24 Който от Ахаавовия род умре в града, него кучетата ще изядат; а който умре в полето, него въздушните птици ще изядат.
\par 25 (Никой, наистина, не биде подобен на Ахаава, който продаде себе си да върши зло пред Господа, като го подбуждаше жена му Езавел,
\par 26 и който извърши много мерзости, като следваше идолите, съвсем, както вършеха аморейците, които Господ бе изгонил пред израилтяните).
\par 27 А Ахаав, като чу тия думи, раздра дрехите си, тури вретище на снагата си, пости и лежеше обвит във вретище, и ходеше внимателно.
\par 28 Тогава Господното слово дойде към тесвиеца Илия и рече:
\par 29 Видя ли как се смири Ахаав пред Мене? Понеже той се смири пред Мене, няма да докарам злото в неговите дни; в дните на сина му ще докарам злото върху дома му.

\chapter{22}

\par 1 И минаха се три години без война между Сирия и Израиля.
\par 2 А в третата година, когато Юдовият цар Иосафат слезе при Израилевия цар,
\par 3 рече Израилевият цар на слугите си: Знаете ли, че Рамот-галаад е наш; а ние немарим да си го вземем от ръката на сирийския цар?
\par 4 Рече и на Иосафата: Дохождаш ли с мена на бой в Рамот-галаад? И Иосафат каза на Израилевия цар: Аз съм както си ти, моите люде както твоите люде, моите коне както твоите коне.
\par 5 Иосафат каза още на Израилевия цар: Моля, допитайте се сега до Господното слово.
\par 6 Тогава Израилевият цар събра прорците си , около четиристотин мъже, та из каза: Да ида ли на бой против Рамот-галаад, или да не ида? А те казаха: Възлез и Господ ще го предаде в ръката на царя.
\par 7 Обаче Иосафат каза: Няма ли тук освен тия , някой Господен пророк, за да се допитаме чрез него?
\par 8 И Израилевият цар рече на Иосафата: Има още един човек, Михей, син на Емла, чрез когото можем да се допитаме до Господа; но аз го мразя, защото не пророкува добро за мене, но зло. А Иосафат каза: Нека не говори така царят.
\par 9 Тогава Израилевият цар повика един скопец и рече: Доведи скоро Михея син на Емла.
\par 10 А Израилевият цар и Юдовият цар Иосафат седяха, всеки на престола си, облечени в одеждите си, на открито място при входа на самарийската порта: и всичките пророци пророкуваха пред тях.
\par 11 А Седекия, Ханаановият син, си направи железни рогове, и рече: Така казва Господ: С тия ще буташ сирийците догде ги довършиш.
\par 12 Също и всичките пророци така пророкуваха, казвайки: Иди в Рамот-галаад, и ще имаш добър успех; защото Господ ще го предаде в ръката на церя.
\par 13 А пратеникът, който отиде да повика Михея, му говори казвайки: Ето сега, думите на пророците, като из едни уста са добри за царя; моля, и твоята дума да бъде като думата на един от тях, и ти говори доброто.
\par 14 А Михей рече: В името на живия Господ заявявам, че каквото ми рече Господ, това ще говоря.
\par 15 И тъй, дойде при царя. И царят му каза: Михее, да идем ли на бой в Рамот-галаад, или да не идем? А той му говори: Възлез и ще имаш добър успех; защото Господ ще го предаде в ръката на царя.
\par 16 А царят му каза: Колко пъти ще те заклевам да ми не говориш друга освен истината в Господното име!
\par 17 А той рече: Видях целият Израил пръснат по планините, като овци, които нямат овчар; и Господ рече: Тия нямат господар; нека се върнат всеки у дома си с мир.
\par 18 Тогава Израилевият цар каза на Иосафата: Не рекох ли ти, че не ще прорече добро за мене, но зло?
\par 19 А Михей рече: Чуй, прочее, Господното слово. Видях Господа седящ на престола Си, и цялото небесно множество стоящо около него отдясно и отляво.
\par 20 И Господ рече: Кой ще примами Ахаава, за да отиде и да падне в Рамот-галаад? И един каза едно, а друг каза друго.
\par 21 Сетне излезе един дух та застана пред Господа и рече: Аз ще го примамя.
\par 22 И Господ му рече: Как? А той каза: Ще изляза и ще бъда лъжлив дух в устата на всичките му пророци. И Господ рече: Примамвай го, още и ще сполучиш; излез, стори така.
\par 23 Сега, прочее, ето, Господ е турил лъжлив дух в устата на всички тия твои пророци; обаче Господ е говорил зла за тебе.
\par 24 Тогава Седекия, Ханаановият син, се приближи та плесна Михея по бузата и каза: През кой път мина Господният Дух от мене, за да говори на тебе?
\par 25 А Михей рече: Ето, ще видиш в оня ден, когато ще отиваш из клет в клет за да се криеш.
\par 26 Тогава Израилевият цар каза: Хванете Михея та го върнете при градския управител Амон и при царския син Иоас.
\par 27 И речете: Така казва царят: Турете тогова в тъмницата, и хранете го със затворническа порция хляб и вода догде си дойда с мир.
\par 28 И рече Михей: Ако някога се върнеш с мир, то Господ не е говорил чрез мене. Рече още: Слушайте вие, всички племена.
\par 29 И така, Израилевият цар и Юдовият цар Иосафат отидоха в Рамот-галаад.
\par 30 И Израилевият цар рече на Иосафата: Аз ще се предреша като вляза в сражението, а ти облечи одеждите си. Прочее, Израилевият цар се предреши та влезе в сражението.
\par 31 А сирийският цар бе заповядал на тридесет и двамата свои колесниценачалници, казвайки: Не се бийте нито с малък, нито с голям, но само с Израилевия цар.
\par 32 А колесниценачелниците, като видяха Иосафата, рекоха: Несъмнено тоя ще е Израилевия цар; и отклониха се, да го ударят; но Иосафат извика.
\par 33 И колесниценачалниците, като видяха, че не беше Израилевият цар, престанаха да го преследват и се върнаха.
\par 34 А един човек стреля без да мери, и удари Израилевия цар между ставите на бронята му; за това той рече на колесничаря си: Обърни ръката си та ме изведи из сражението, защото съм тежко ранен.
\par 35 И в оня ден сражението се усили; а царят биде подкрепен в колесницата си срещу сирийците, но привечер умря; и кръвта течеше от раната в дъното на колесницата.
\par 36 И около захождането на слънцето, нададе се в стана вик, който казваше: Всеки да иде в града си и всеки на мястото си!
\par 37 Така царят умря, и донесоха в Самария, и погребаха царя в Самария.
\par 38 И като миеха колесницата в самарийския водоем, гдето се миеха и блудниците, кучетата лижеха кръвта му, според словото, което Господ бе говорил.
\par 39 А останалите дела на Ахаава, и всичко що върши, и къщата която построи от слонова кост, и всичките градове, които съгради, не са ли написани в Книгата на летописите на Израилевите царе?
\par 40 Така Ахаав заспа с бащите си; и вместо него се възцари син му Охозия.
\par 41 А над Юда се възцари Иосафат, син на Аса, в четвъртата година на Израилевия цар Ахаав.
\par 42 Иосафат бе тридесет и пет години на възраст, когато се възцари, и царува двадесет и пет години в Ерусалим; а името на майка му беше Азува, дъщеря на Силея.
\par 43 Той ходи съвършено в пътя на баща си Аса: не се отклони от него а вършеше това, което бе право пред Господа. Високите места, обаче, не се отмахнаха; людете още жертвуваха и кадяха по високите места.
\par 44 И Иосафат сключи мир с Израилевия цар.
\par 45 А останалите дела на Иосафата, юначествата, които показа, и как воюваше, не са ли написани в Книгата на летописите на Юдовите царе.
\par 46 Също той изтреби от земята останалите мъжеложци, които бяха останали от времето на баща му Аса.
\par 47 В това време нямаше цар в Едом, но наместник царуваше.
\par 48 Иосафат построи кораби като тарсийските, които да идат в Офир за злато; обаче не отидоха, защото корабите се разбиха в Есевон-гавер.
\par 49 Тогава Охозия Ахаавовият син каза на Иосафата: Нека отидат моите слуги с твоите слуги в корабите. Но Иосафат отказа.
\par 50 И Иосафат заспа с бащите си, и биде погребан с бащите си в града на баща си Давида; и вместо него се възцари син му Иорам.
\par 51 Охозия, Ахаавовия син, се възцари над Израиля в Самария, в седемдесетата година на Юдовия цар Иосафат, и царува две години над Израиля.
\par 52 Той върши зло пред Господа, като ходи в пътя на баща си, и в пътя на майка си, и в пътя на Еровоама, Наватовия син, който направи Израиля да греши;
\par 53 защото служи на Ваала и му се поклони, и разгневи Господа Израилевия Бог според всичко, което бе вършил баща му.

\end{document}