\begin{document}

\title{4 Царе}


\chapter{1}

\par 1 След смъртта на Ахаава, Моав въстана против Израиля.
\par 2 И Охозия падна през решетката на своята горна стая, която бе в Самария и се разболя; и прати човеци, на които рече: Идете, допитайте се до акаронския бог Ваал-зевув дали ще оздравея от тая болест.
\par 3 Но ангел Господен каза на тесвиеца Илия: Стани, иди да посрещнеш пратениците на самарийския цар, и кажи им: Няма ли Бог в Израиля, та отивате да се допитвате до акаронския бог Ваал-зевува?
\par 4 Сега, прочее, така казва Господ: Няма да слезеш от леглото, на което си се качил, но непременно ще умреш. Тогава Илия се отиде.
\par 5 А като се върнаха пратениците при Охозия , той им рече: Защо се върнахте?
\par 6 Те му казаха: Един човек излезе да ни посрещне и рече ни: Идете, върнете се при царя, който ви е пратил, та му кажете: Така казва Господ: Няма ли Бог в Израиля та пращаш да се допитат до акаронския бог Ваал-зевува? Няма, прочее, да слезеш от леглото, на което си се качил, но непременно ще умреш.
\par 7 И рече им: Какъв беше на глед човекът, който излезе да ви посрещне и ви каза тия думи?
\par 8 И те му отговориха: Беше човек облечен в кожух и препасан около кръста си с кожен пояс. А той рече: Това е тесвиецът Илия.
\par 9 Тогава царят прати при него един петдесетник с петдесетте му войници . И той се възкачи при него; и, ето, Илия седеше на върха на хълма. И рече му: Божий човече, царят каза: Слез.
\par 10 А Илия в отговор рече на петдесетника: Ако съм аз Божий човек, нека слезе огън от небето та нека изгори тебе и петдесетте ти войници . И слезе огън от небето та изгори него и петдесетте му войници .
\par 11 Пак царят прати при него друг петдесетник с петдесетте му войници . И той проговори та му рече: Божий човече, така казва царят: Слез скоро.
\par 12 А Илия в отговор им рече: Ако аз съм Божий човек, нека слезе огън от небето та нека изгори тебе и петдесетте войници . И слезе Божият огън от небето та изгори него и петдесетте му войници ,
\par 13 Пак прати царят трети петдесетник с петдесетте му войници . И третият петдесетник, като се възкачи дойде та коленичи пред Илия и го помоли, казвойки му: Божий човече, моля ти се, нека бъде скъпоценен пред очите ти животът ми и животът на тия петдесет твои слуги.
\par 14 Ето, огън слезе от небето та изгори първите двама петдесетници с петдесетте им войници ; а сега нека бъде моят живот скъпоценен пред очите ти.
\par 15 И ангулът Господен рече на Илия: Слез с него; не бой се от него. И тъй, той стана та слезе с него при царя.
\par 16 И рече Му: Така казва Господ: Понеже си пратил човеци да се допитат до акаронския бог Ваал-зевува, като че нямаше Бог в Израиля, за да се допиташ до Неговото слово, затова няма да слезеш от леглото, на което си се качил, но непременно ще умреш.
\par 17 И така, той умря според Господното слово, което Илия беше говорил. И вместо него се възцари Иорам във втората година на Юдовия цар Иорам Иосафатовия син; понеже Охозия нямаше син.
\par 18 И останалите дела, които Охоция извърши, не са ли написани в Книгата на летописите на Израилевите царе?

\chapter{2}

\par 1 И когато Господ щеше да възнесе Илия на небето с вихрушка, Илия отиде с Елисея от Голгал.
\par 2 И Илия рече на Елисея: Седи тука, моля, защото Господ ме прати до Ветил. А Елисей рече: Заклевам се в живота на Господа и в живота на душата ти, няма да те оставя. И така, слязоха във Ветил.
\par 3 И пророческите ученици, които бяха във Ветил, излязоха при Елисея та му рекоха: Знаеш ли, че днес Господ ще ти вземе господаря, който е бил над тебе? А той каза: Да, зная това, мълчете.
\par 4 Тогава Илия му рече: Елисее, седи тук, моля, защото Господ ме прати в Ерихон. А той рече: Заклевам се в живота на Господа и в живота на душата ти, няма да те оставя. Така, дойдоха в Ерихон.
\par 5 И пророческите ученици, които, бяха в Ерихон, дойдоха при Елисея та му рекоха: Знаеш ли, че днес Господ ще ти вземе господаря, който е бил над тебе? А той отговори: Да, зная това мълчете.
\par 6 Тогава Илия му рече: Седи тука, моля, защото Господ ме прати до Иордан. А той каза: Заклевам се в живота на Господа и в живота на душата ти, няма да те оставя. И тъй, отидоха и двамата.
\par 7 И петдесет мъже от пророческите ученици отидоха та застанаха насреща им от далеч. А те двамата застанаха при Иордан.
\par 8 И като взе Илия кожуха си та го сгъна, удари водата; и тя се раздели на едната и на другата страна, така че двамата преминаха по сухо.
\par 9 И когато преминаха, Илия каза на Елисея: Искай какво да ти сторя преди да бъда отнет от тебе. И рече Елисей: Моля, нека бъде в мене двоен дял от духа ти.
\par 10 А той рече: Мъчно нещо поиска ти; но , ако ме видиш, когато ме отнемат от тебе, ще ти бъде така; но ако не, не ще бъде.
\par 11 И докато те още ходеха и се разговаряха, ето огнена колесница и огнени коне, които ги ръзделиха един от друг; и Илия възлезе с вихрушката на небето.
\par 12 А Елисей, като гледаше, извика: Такко мой, татко мой, колесницата Израилева и конница негова! И не го видя вече. И хвана дрехите си та ги разкъса на две части.
\par 13 И като дигна кожуха на Илия, който падна от него, върна се и застана на брега на Иордан.
\par 14 И взе кожуха, който падна от Илия, та удари водата и рече: Где е Господ Израилевия Бог? И като удари и той водата, тя се раздели на едната и на другата страна; и Елисей премина.
\par 15 А пророческите ученици, които бяха в Ерихон, като го видяха отсреща рекоха: Илиевият дух остава на Елисея. И дойдоха да го посрещнат, и му се поклониха до земята.
\par 16 Тогава му рекоха: Ето сега, между слугите ти има петдесет силни мъже; нека отидат, молим, да потърсят господаря ти, да не би да го е дигнал Господният Дух и го е хвърлил на някое бърдо или в някоя долина. А той каза: Не изпращайте.
\par 17 Но като настояваха пред него толкоз щото се засрами, рече: Изпратете. Изпратиха, прочее, петдесет мъже, които търсиха три дена, но не го намериха.
\par 18 И когато се върнаха при него, (защото той беше останал в Ерихон); рече им: Не ви ли казах: Не отивайте?
\par 19 След това, гражданите казаха на Елисея: Виж, молим ти се, местоположението на тоя град е добро, както вижда господарят ни; но водата е лоша, а земята е базплодна.
\par 20 А той рече: Донесете ми ново блюдо, и турете в него сол. И донесоха му.
\par 21 Тогава слезе при извора на водата та хвърли солта в него, като рече: Така казва Господ: Изцерих тая вода; не ще има вече от нея ни смърт ни базпловие.
\par 22 Така водата биде изцерена, каквато е и до днес, според словото, което Елисей говори.
\par 23 И от там той възлезе във Ветил; и като се качваше по пътя, излязоха из града малки деца та му се присмиваха, като му казваха: Качи се, плешиве! качи се, плешиве!
\par 24 А той, като се озърна назад и ги видя, прокле ги в Господното име. И излязоха из дъбравата две мечки та разкъсаха от тях четиридесет и две деца.
\par 25 И от там отиде на планината Кармел, отгдето и се върна в Самария.

\chapter{3}

\par 1 А в осемнадесетата година на Юдовия цар Иосафат, Иорам Ахаавовият син се възцари над Израиля в Самария; и царува дванадесет години.
\par 2 Той върши зло пред Господа, но не както баща му и майка му, защото дигна Вааловия кумир, който бе направил баща му.
\par 3 Обаче беше привързан за греховете на Еровоама Наватовия син, който направи Израиля да греши; не се остави от тях.
\par 4 А моявският цар Миса имаше стада, и даваше данък на Израилевия цар вълната от сто хиляди агнета и от сто хиляди овни.
\par 5 Но когато умря Ахаав моавският цар въстана против Израилевия цар.
\par 6 Затова, цар Иорам излезе в онова време от Самария та събра целия Израил.
\par 7 И отивайки той прати до Юдовия цар Иосафата да кажат: Моавският цар въстана против мене; ще дойдеш ли с мене на бой против Моава? И той каза:Ще възляза; аз съм както си ти, моите люде както твоите люде, моите коне както твоите коне.
\par 8 Попита още: Праз кой път да възлезем? А той отговори: През пътя за едомската пустиня.
\par 9 И така, Израилевият цар, и Юдовият цар, и едомският цар отидоха и направиха седем дневна обиколка; но нямаше вода за войската и за животните, които бяха с тях.
\par 10 Тогава рече Израилевият цар: Уви! наистина Господ свика тия трима царе, за да ги предаде в ръката на Моава!
\par 11 А Иосафат рече: Няма ли тука Господен пророк, за да се допитаме до Господа чрез него? И един от слугите на Израилевия цар в отговор рече: Тук е Елисей, Сафатовият син, който поливаше вода на Илиевите ръце.
\par 12 И Иосафат рече: Господното слово е у него. И тъй, Израилевият цар и Иосафат, и едомският цар слязоха при него.
\par 13 А Елисей рече на Израилевия цар: Какво има между мене и тебе? Иди при пророците на баща си и при пророците на майка си. А Израилевият цар му каза: Не, защото Господ свика тия трима царе за да ги предаде в ръката на Моава.
\par 14 А Елисей рече: Заклевам се в живота на Господа на Силите, Комуто слугувам, наистина, ако не почитах присъствието на Юдовия цар Иосафата, не бих погледнал на тебе, нито бих те видял;
\par 15 но сега, доведете ми един свирач. И като свиреше свирачът, Господната ръка дойде върху него.
\par 16 И той рече: Така казва Господ: Направи цялата тая долина на трапове;
\par 17 защото така казва Господ: Без да видите вятър и без да видите дъжд, пак тая долина ще се напълни с вода; и ще пиете вие, добитъкът ви и животните ви.
\par 18 Но това е малко нещо пред очите на Господа; Той, при това, ще предаде и Моава в ръката ви;
\par 19 и ще поразите всеки укрепен град и всеки отборен град, ще повалите всяко добро дърво, ще захушите всичките водни извори, и ще запушите с камъни всяка добра площ земя.
\par 20 И на сутрента, когато се принасяше принос, ето, дойдоха води от едомския път, и земята се напълни с вода.
\par 21 А като чуха всичките моавци, че царете са дошли да се бият с тях, събраха се всичко, които можеха да опасват нож и нагоре, та застанаха на границата.
\par 22 И като станаха на сутринта и изгря слънцето върху водите и моавците видяха водите отсреща чарвени като кръв, рекоха:
\par 23 Това е кръв; непременно царете са се били помежду си и са поразили един едни други; сега, прочее, на користите Моаве!
\par 24 А когато дойдоха в Израилевия стан, израилтяните станаха та поразиха моавците, тъй щото побягнаха пред тях; и като поразяваха моавците влязоха в земята им.
\par 25 И събориха градовете, и на всяка добра площ земя хвърлиха всеки камъка си та я напълниха, запушиха всичките водни извори, и отсякоха всяко добро дърво; само на Кир-арасет оставиха камъните му, но пращниците го заобиколиха та го поразиха.
\par 26 И моавският цар, когато видя, че сражението се засилваше против него, взе си седемстотин сабленици, за да пробият път, до едомския цар; но не можаха.
\par 27 Тогава Той взе първородния си син, който щеше да се възцари вместо него, та го принесе всеизгаряне на стената. Затова стана голямо възмущение между Израиля; и оттеглиха се от него и се върнаха в земята си.

\chapter{4}

\par 1 А една от жените на пророческите ученици извика към Елисея и каза: Слугата ти мъж ми умря; и ти знаеш, че слугата ти се боеше от Господа; а заимодавецът дойде да вземе за себе си двата ми сина за роби.
\par 2 И Елисей й каза: Що да ти сторя? Кажи ми що имаш в къщи? А тя рече: Слугинята ти няма нищо в къщи, освен един съд с дървено масло.
\par 3 И рече: Иди, вземи на заем вън, от всичките си съседи, съдове празни съдове, вземи не малко.
\par 4 Сетне влез и затвори вратата зад себе си и зад синовете си, и наливай от маслото във всички тия съдове, и пълните туряй на страна.
\par 5 И тъй, та си отиве от него та затвори вратата зад себе си и зад синовете си; и те донасяха съдовете при нея, а тя наливаше.
\par 6 И като се напълниха съдовете, рече на един от синовете си: Донеси ми още един съд. А той й рече: Няма друг съд. И маслото престана.
\par 7 Тогава тя дойде та извести на Божия човек. И той рече: Иди, продай маслото та плати дълга си, и живей с останалото, ти и синовете ти.
\par 8 И един ден Елисей замина в Сунам, гдето имаше една богата жена; и тя го задържа да яде хляб. И колкото пъти заминаваше свръщаше там, за да яде хляб.
\par 9 Сетне жената рече на мъжа си: Ето сеха, познавам че този, който постоянно наминава у нас, е свет Божий човек.
\par 10 Да направим, моля една малка стаичка на стената, и да турим в нея за него легло и маса и стол и светилник, за да свръща там, когато дохожда при нас.
\par 11 И един ден, като дойде там и свърна в стаичката та лежеше в нея,
\par 12 рече на слугата си Гиезия: Повикай тая сунамка. И повика я, и тя застана пред него.
\par 13 И рече Гиезия : Кажи й сега: Етой, ти си положила всички тия грижи за нас; що да се говори за тебе на царя или на военачалника. А тя отговори: Аз живея между своите люде.
\par 14 Тогава рече: Що, прочее, да сторим за нея? А гиезиий отговори: Наистина тя няма син, а мъжът й е стар.
\par 15 И рече: Повикай я. И когато я повика, тя застана при вратата.
\par 16 И Елисей й рече: Не, господарю мой, Божий човече, не лъжи слугинята си.
\par 17 Но жената зачна и роди син на другата година по същото време, както й рече Елисей.
\par 18 И когато порасна детето, излезе един ден към баща си при жетварите.
\par 19 И рече на баща си: Главата ми! главата ми! А той рече на един от момците: Занеси го при майка му.
\par 20 И като го взе занесе го при маяка му; и детето седна на коленете й до пладне, и тогава умря.
\par 21 И тя се качи та го положи на леглото на Божия човек, и като зетвори вратата след него, излезе.
\par 22 Тогава повика мъжа си и рече: Изпрати ми, моля, един от момците и една от ослиците, за да тичам при Божия човек и да се върна.
\par 23 А той рече: Защо да отидеш днес при него? не е нито нов месец, нито събота. А тя рече: Бъди спокоен.
\par 24 Тогава оседла ослицата, и рече на слугата си: Карай и бързай; не забравяй карането заради мене освен ако ти заповядвам.
\par 25 И тъй, стана та отиде при Божия човек на планината Кармил. А Божият човек, като я видя от далеч, рече на слугата си Гиезия: Ето там сунамката!
\par 26 Сега, прочее, тичай да я посрещнеш и кажи й: Добре ли си? добре ли е мъжът ти? Дабре ли е детето? А тя отговори: Добре.
\par 27 А когато дойде при Божия човек на планината, хвана се за нозете му; а Гиезий се приближи, за да я оттласне. Но Божият човек рече: Остави я, защото душата й е преогорчена в нея: А Господ е скрил причината от мене, и не ми я е явил.
\par 28 А тя рече: Искала ли съм син от господаря си? Не рекох ли: Не ме лъжи?
\par 29 Тогава Елисей рече на Гиезия: Препаши кръста си, вземи тоягата ми в ръка, та иди; ако срещнеш човек, да го не поздравиш, а ако те поздрави някой, да ме не отговаряш; и положи тоягата ми върху лицето на детето.
\par 30 А майката на детето рече: Заклевам се в живота на Господа и в живота на душата ти, няма да те оставя. И така, той стана та отиде подир нея.
\par 31 А Гиезия мина пред тях и положи тоягата върху лицето на детето; но нямаше ни глас, нито слушане. Затова се върна да го посрещне и му извести казвайко: Детето не се събуди.
\par 32 И когато влезе Елисей в къщата, ето детето умряло, положено на леглото му.
\par 33 Влезе, прочее, та затвори вратата зат тях двамата, и помоли се Господу.
\par 34 Тогава се качи та легна върху детето, и като тури устата си върху неговите уста, и рацете си върху неговите ръце простря се върху него; и стопли се тялото на детето.
\par 35 После се оттегли та ходеше насам натам из къщата, тогава пак се качи та се простря върху него; и детето кихна седем пъти, и детето отвори очите си.
\par 36 И Елисай извика Гиезия и рече: Повикай тая сунамка. И повика я; и когато влезе при него, рече й: Вземи сина си.
\par 37 И тя влезе, падна на нозете му та се поклони до земята, и дигна сина си те излезе.
\par 38 Пак дойде Елисей в Галгал, когато имаше глад в земята; и като седяха пред него пророческите ученици, та го слушаха , рече на слугата си: Тури големия котел та свари вариво за пророческите ученици.
\par 39 Затова, един излезе на полето за да набере зеленище, и като намери диво растение, набра от него диви тиквички, та напълни дрехата си, и се върна и ги наряза в котела с варивото, понеже не знаеха, че са отровни .
\par 40 После сипаха на човеците да ядат; а като ядоха от варивото извикаха, казвайки: Божий човече, смърт има в котела! И не можаха да ядат от него.
\par 41 А той рече: Тогава донесете брашно. И като го хвърли в котела рече: Сипи на людете да ядат. И нямаше нищо отровно в котела.
\par 42 В това време един човек от Ваалселиса дойде та донесе на Божия човек хляб от първите плодове, двадесет ечемичени хляба и пресни класове жито небелени. И рече: Дай на людете да ядат.
\par 43 И слугата му рече: Що! да сложа ли това пред стотина човека? А той каза: Дай на людете да ядат, защото така казва Господ: Ще се нахранят и ще остане излишък.
\par 44 Тогава той сложи пред тях, та се нахраниха, и остана излишък, според Господното слово.

\chapter{5}

\par 1 А Нееман, военачалникът на сирийския цар, беше човек велик и почитан пред господаря си, понеже чрез него Господ бе дал избавление на Сирия; при това, беше човек силен и храбър, но беше прокажен.
\par 2 И сирийците бяха излезли на чети, и бяха довели от Израилевата земя една малка мома пленница; и тя слугуваше на Неемановата жена.
\par 3 И рече на Господарката си: Ако беше господарят ми при пророка, който е в Самария, и той би го изцелил от проказата му!
\par 4 Тогава влезе Нееман та съобщи на господаря си, казвайки: Така и така рече момата, която е от Израилевата земя.
\par 5 И рече сирийският цар: Стани, иди, и ще пратя писмо до Израилевия цар. И тъй, той отиде, вземайки в ръката си десет таланта сребро, шест хиляди жълтици, и десет премени дрехи.
\par 6 И донесе писмоо на Израилевия цар, в което се казваше: Като пристигне това писмо до тебе, ето, същевременно пратих до тебе слугата си Неемана, за да го изцелиш от проказата му.
\par 7 А Израилевият цар, като прочете писмото, раздра дрехите си и рече: Бог ли съм аз за да умъртвявам и да съживявам, та праща до мене да изцеря човека от проказата му? Моля, прочее, разсъдете и вижте как търси повод за скарване с мене.
\par 8 А Божият човек Есей, като чу, че Израилевият цар раздрал дрехите си, прати до царя да рекат: Защо си раздрал дрехите си? Нека дойде сега при мене, и ще познае, че има порок в Израиля.
\par 9 И така, Нееман дойде с конете си и с колесниците си та застана при вратата на Елисеевата къща.
\par 10 И Елисей прати до него човек да каже: Иди, окъпи се седем пъти в Иордан; и ще се обновят месата ти, и ще се очистиш.
\par 11 А Нееман се разгневи, та си отиде, като казваше: Ето, аз мислех, че той непременно ще излезе при мене, ще застане, и ще призове името на Господа своя Бог, и ще помаха ръката си върху мястото, и така ще изцери прокажения.
\par 12 Реките на Дамаск, Авана и Фарфар, не струват ли повече от всичките води на Израиля? Не мога ли да се окъпя в тях и да се очистя? Затова, той се обърна и си отиде много разгневен.
\par 13 А слугите му се приближиха та му говориха, казвайки: Татко мой, ако ти беше заръчал пророкът нещо голямо, не би ли го извършил? Колко повече, прочее, като ти казва! Окъпи се и очисти се!
\par 14 Тогава той слезе та се потопи седем пъти в Иордан, според думите на Божия човек; и месата му се обновиха, като месата на малко дете, и очисти се.
\par 15 Тогава той се върна при Божия човек с цялата си дружина, и като дойде та застана пред наго рече: Ето, сега узнах, че няма Бог в целия свут освен в Израиля; затова, моля, приеми сега подарък от слугата си.
\par 16 А той каза: Заклевам се в живота на Господа, Комуто слугувам, не ща да приема. А той го принуждаваше да приеме; но отказа.
\par 17 Тогава Нееман рече: Ако не то нека се даде, моля, на слугата ти товар за две мъски от тая пръст; защото слугата ти не ще вече да принася ни всеизгаряне, нито жертва на други богове освен на Господа.
\par 18 Господ да прости това нещо на слугата ти, ако, когато влиза господарят ми в къпищато на Ромона, за да се поклони там, и се подпира на ръката ми, се навеждам и аз в къпището на Римона; като се навеждам в капището на Римона, Господ да прости това нещо на слугата ти!
\par 19 И рече му: Иди с мир. И така, той отиде от него известно растояние.
\par 20 Но Гиезий, слугата на Божия човек Елисей, си рече: Ето, господарят ми си посвени, та не взе от ръката на тоя сириец Нееман това, което донесе; но в името на живия Господ, аз ще се завтека подер него и ще взема нещо от него.
\par 21 И тъй, Гиезий се завтече подир Неемана. А нееман, когато видя, че тича подир него, скочи от колесницата да го посрещне и рече: Добре ли и с всички ви ?
\par 22 Той рече: Добре. Господарят ми ме прати да река: Ето тъкмо сега дойдоха у мене от хълмистата земя на Ефрема двама младежи от пророческите ученици; дай им, моля, един талант сребро и премени дрехи.
\par 23 И рече на Нееман: Благоволи да вземеш два таланта. И като го принуди, върза двата таланта сребро в два мешеца заедно с две премени дрехи, и натовари ги на двама от слугите си, които ги носеха пред него.
\par 24 И когато стигна до хълма, взе ги от ръцете им та ги скри в къщата; и отпусна човеците, та си отидоха.
\par 25 Тогава влезе та застана пред господаря си. И Елисей му рече: От где идеш, Гиезие? А той рече: Слугата ти не е ходил никъде.
\par 26 А Елисей му каза: Не отиде ли сърцето ми с тебе , когато се върна човекът от колесницата си за да те посрещне? Време ли е да приемеш пари и да приемеш дрехи, маслини и лозя, овци и говеда, слуги и слугини?
\par 27 Затова Неемановата проказа ще се засили за тебе и за рода ти до века. И той излезе от присъствието му прокажен, бял като сняг.

\chapter{6}

\par 1 И пророческите ученици казаха на Елисея: Ето сега, мястото, гдето живеем та внимаваме пред тебе е тясно за нас.
\par 2 Нека отидам, молим, до Иордан, и от там да вземем всеки по една греда да живеем. А той отговори: Идете.
\par 3 И един от тях каза: Благоволи, моля, да дойдеш и ти със слугата си. И той отговори: Ще дойда.
\par 4 Отиде, прочее, с тях. И като отидоха до Иордан, сечяха дървета.
\par 5 А един от тях като сечеше среда, желязото падна във водата; и той извика и рече: Ах, господарю мой! то беше взето на заем!
\par 6 А Божият човек рече: Где падна? И показа му мястото. Тогава той отсече едно дръвце та го хвърли там; и желязото изплава.
\par 7 И рече: Вземи си го. И той простря ръка та го взе.
\par 8 А сирийският цар, като воюваше против Израиля, съветваше се със слугите си, та каза: На еди-кое-си място ще разположа стана си.
\par 9 Тогава Божият човек прати до Израилевия цар да кажат: Пази се да не минеш през това място, защото сирийците са слезли там.
\par 10 И Израилевият цар прати до мястото, за което Божият човек му каза и го предупреди; и опази се от там не еднъж, нито дваж.
\par 11 И сърцето на сирийския цар се смути поради това нещо; за туй свика слугите си та из каза: Не щете ли да ми обадите, кой от нашите е за Израилевия цар?
\par 12 А един от слугите му каза: Никой, господарю мой царю; но пророкът, който е в Израиля, Елисей, известява на Израилевия цар думите, които говориш в спалнята си.
\par 13 И рече: Идете, научете се где е, зада пратя да го заловят. И известиха му казвайки: Ето, в Дотан е.
\par 14 Тогава той прати там коне, колесници и голяма войска, които дойдоха през нощта и обиколиха града.
\par 15 И на слутринта, когато слугата на Божия човек стана та излезе, ето, войска с коне и колесници беше обиколила града. И рече му слугата му: Ах, господарю мой! Какво ще правим?
\par 16 А той отговори: Не бой се, защото ония, които са с нас, важат повече от ония, които са с тях.
\par 17 И помоли се Елисей, казвайки: Моля Ти се, Господи, отвори му очите за да види. И Господ отвори очите на слугата та видя, и, ето, хълмът бе пълен с огнени коне и колесници около Елисея.
\par 18 И когато слязоха към врага , Елисей се помоли Господу казвайки: Моля Ти се, порази тия люде със слепота. И порази ги със слепота, според както каза Елисей.
\par 19 Тогава Елисей им рече: Не е тоя пътят, нито е тоя градът: дойдете подир мене, и ще ви заведа при човека, когото търсите. И отведе ги в Самария.
\par 20 И когато стигнаха в Самария, Елисей рече: Господи, отвори очите на тези за да видят. И Господ отвори очите им, та видяха; и, ето, бяха всред Самария.
\par 21 А когато ги видя, Израилевият цар рече на Елисея: Да ги поразя ли, татко? да ги поразя ли?
\par 22 А той отговори: Не ги поразявай. Поразил ли би ти ония, които би пленил с меча си и с лъка си? Сложи им хляб и вода, за да ядат и пият; и нека отидат при господаря си.
\par 23 И така, той им сложи много ястия; и след като ядоха и пиха, пусна ги, та отидоха при господаря си. И сирийските чети не дойдоха вече в Израилевата земя.
\par 24 А след това, сирийският цар Венадад събра цялата си войска и дойде та обсади Самария.
\par 25 И стана голям глад в Самария; защото, ето, обсаждаха я докле една оселова глава се продаваше за осемдесет сребърника, и четвърт кав гълъбова тор, за пет сребърника.
\par 26 И като заминаваше Израилевият цар по стената, една жена извика към него и каза: Помогни, господарю мой, царю!
\par 27 А той рече: Ако Господ не ти помогне, от где ще ти помогна аз? от гумното ли, или от лина?
\par 28 И царят рече: Що имаш? А тя отговори: Тая жена ми рече: Дай твоя син да го изядем днес, а утре ще изядем моя син.
\par 29 И тъй сварихме моя син та го изядохме; и на следния денй рекох: Дай твоя син да го изядем; а тя скри сина си.
\par 30 И като чу царят думите на жената, раздра дрехите си; и като минаваше по стената, людете видяха, и, ето, извътре имаше вретище върху снагата му.
\par 31 Тогава рече: Така да ми направи Бог, да! и повече да притури, ако главата на Елисея, Сафатовия син, остане на него днес.
\par 32 А когато изпрати мъж от служещите му, Елисей седеше в къщата си и старейшините седяха с него; но преди да стигне пратеникът при него, той рече на старейшините: Виждате ли как тоя син на убийцата прати да ми отнемат главата? Гедайте, щом дойде пратеникът, затворете вратата и спрете го при вратата: тропота на нозете на господаря му не е ли подир него?
\par 33 И докато още говореше с тях, ето, пратеникът слезе при него; а царят го предвари и каза: Ето, от Господа е това зло; какво има да се надея вече на Господа?

\chapter{7}

\par 1 Тогава рече Елисей: Слушайте Господното слово. Така казва Господ: Утре, по това време, при портата на Самария една сата чисто брашно ще се продаде за един сикъл.
\par 2 А сановникът на чиято ръка се подпираше царят, отговори на Божия човек, казвайки: Ето сега, и прозорци, ако би направил Господ на небето, били могло да стане това нещо? А той каза: Ето, ще видиш с очите си, но няма да ядеш от него.
\par 3 А във входа на портата имаше четирима прокажени; и рекоха си един на друг: Защо да седим тук докле умрем?
\par 4 Ако речем да влезем в града, гладът е в града, и ще умрем там; и ако седим тук, пак ще умрем. Сега, прочее, да идем та да се предадем в сирийският стан. Ако ни оставят живи, ще живеем; но ако ни убият, само ще умрем.
\par 5 И така, на мръкване станаха за да отидат към сирийския стан, а като стигнаха до края на сирийския стан, ето, нямаше никой там.
\par 6 Защото Господ беше направил да се чуе в стана на сирийците тропот от колесници и тропот от коне, тропот от голяма войска; и те бяха си рекли един на друг: Ето, Израилевият цар е наел против нас хетейските царе и египетските царе, за да дойдат върху нас.
\par 7 Затова, бяха станали и побягнали в полусветлината, като оставиха шатрите си, конете си и ослите си, - целия стан, както си беше, - и бяха пибягнали за живота си.
\par 8 И когато тия прокажени стигнаха до края на стана, влязоха в един шатър та ядоха и пиха, и взеха от там сребро, злато и дрехи, и отидоха та ги скриха; после, връщайки се, влязоха в друг шатър, взеха и от там и отидоха та скриха взетото .
\par 9 Тогава рекоха помежду си: Ние не правим добре; тоя ден е ден на добри вести, а ние мълчим; ако чякаме докле съмне, възмездието ни ще ни постигне; нека, прочее, отидем да известим това на царския дом.
\par 10 И тъй, дойдоха та известиха към градските врати, и известиха им казвайки: Отидохме в стана на сирийците, и, ето, нямаше там ни човек ни човешки глас, само коне вързани и осли вързани, и шатрите както са си били.
\par 11 И вратарите извикаха и известиха това вътре в царския дом.
\par 12 А царят, като стана, през нощта, каза на слугите си: Сега ще ви кажа що ни направиха сирийците. Те знаят, че сме гладни, и за това са излезли из стана, за да се скрият по нивите, като си казват: Когато излязат из града ще ги заловим живи, и ще възлезем в града.
\par 13 А един от слугите му в отговор рече: Нека вземат, моля, пет от останалите коне, които са оцелели в града, (ето те са като цяло множество от израилтяните, които се изнуриха), и нека пратим да видим.
\par 14 Вързаха, прочее, две колесници с конете, и царят ги прати по дирята на сирийската войска, и рече: Идете и вижте.
\par 15 И отидоха подир тях до Иордан; и, ето, целият път беше пълен с дрехи и вещи, които сирийците бяха хвърлили в бързането си. И пратениците се върнаха та явиха това на царя.
\par 16 Тогава людете излязоха та разграбиха стана на сирияците. Така, една сата чисто брашно се продаваше за един сикъл, и две сати ечемик за един сикъл, според Господното слово.
\par 17 И за пазенето на портата церят постави сановника, на чиято ръка се подпираше; но людете го стъпкаха в портата, та умря, както беше казал Божият човек, който говори, когато церят слезе при него
\par 18 И както беше говорил Божият човек на царя, казвайки: Утре, по това време, в самарийската порта две сати ечемик ще се дават за сикъл, и една сата чисто брашно за сикъл,
\par 19 а сановникът беше отговорил на Божия човек, казвайки: И прозорци ако би направил Господ на небето, би ли могло да стане такова нещо? а той беше казал: Ето, ще видиш това с очите си, но няма да ядеш от него,
\par 20 така му се случи, защото людете го стъпкаха в портата, та умря.

\chapter{8}

\par 1 А Елисей беше говорил на жената, чийто син беше съживил, като беше казал: Стани та иди, ти и домът ти, и поживей гдето можеш поживя; защото Господ повика глада, и ще дойде на земята и ще продължи седем години.
\par 2 И жената беше станала и постъпила според думите на Божия човек, и беше отишла тя и домът й, и беше поживяла във филистомската земя седем години.
\par 3 А в края на седемте години жената се върна от филистимското зимя; и излезе да вика към царя за къщата си и за нивите си.
\par 4 А царят се разговаряше със слугата на Божия човек, Гиезий, и казваше: Ями разкажи всичките големи дела, които Елисай извърши.
\par 5 И като разказваше на царя как бе съживил мъртвия, ето, че жената, чиито син беше съживил, извика към царя за къщата си и за нивите си. И рече Гиезий: Господарю мой царю, тази е жената и този е синътй, когото Елисей съживи.
\par 6 И царят запита жената, и тя му разказа работата . Тогава церят й определи един чиновник, комуто каза: Повърни всичко, което е било нейно, и всичките произведения от нивите й от деня, когато е напуснала страната до сега.
\par 7 А Елисай отиде в Дамаск. А сирийският цар Венадад бе болен; и известиха му, казвайки: Божият човек дойде тук.
\par 8 И царят рече на Азаила: Вземи подарък в ръката си та иди да посрещнеш Божия човек и допитай се чрез него до Господа, като кажеш: Ще оздравея ли от тая болест?
\par 9 И тъй, Азаил отиде за да го посрещне, като взе със себе си подарък от всяко благо от Дамаск, четиридесет камилски товари; и дойде та застана пред него и рече: Син ти Венадад, сирийският цар, ме прати при тебе, и казва: Ще оздравея ли от тая болест?
\par 10 И Елисей му рече: Иди, кажи му: Непременно ще оздравееш от болестта . Обаче, Господ ми яви, че непременно ще умре друго-яче .
\par 11 И втренчи погледа си в Азаила без да мръдне, догде се засрами той; и Божият човек заплака.
\par 12 А азаил каза: Защо плачеш, господарю мой? А той отговори: Защото зная колко зло ще сториш на израилтяните: укрепленията им ще предадеш но огън, младежите им ще избиеш с меч, малките им даца ще ще смажеш и бременните им жени ще разпориш.
\par 13 И рече Азеил: Но ще е слугата ти, кучето, та да извърши това голямо нещо? И Елисей отговори: Господ ми яви, че ти ще царуваш над Сирия.
\par 14 Тогава той тръгна от Елисея та дойде при господаря си и той му рече: Що ти каза Елисей? И отговори: Каза ми, че непременно ще оздравееш.
\par 15 А на другия ден взе завивката и като я натопи във вода, разпростня я на лицето му, така че той умря. А вместо него Азаил се възцари.
\par 16 И в петата година на Израилевия цар Иорам, син на Ахаава, възцари се Иорам, син на Юдовия цар Иосафат.
\par 17 Той бе тридесет и две години на възраст, когато се възцари, и царува осем години в Ерусалим.
\par 18 И ходи по пътя на Израилевите царе, както постъпваше Ахавовият дом, защото жена му беше Ахаавова дъщеря; и върши зло пред Господа.
\par 19 При все това, Господ не иска да изтреби Юда, заради слугата Си Давида, както му бе обещал, че ще даде светилник нему и на потомците му до века.
\par 20 В неговите дни Едом отстъпи озпод ръката на Юда, и си поставиха свой цар.
\par 21 Затова, Иорам замина за Саир, и всичките колесници с него; и като стана през нощта, порази едомците, които го обкръжаваха, и началниците на колесниците; а людете побягнаха в шатрите си.
\par 22 Обаче, Едом отстъпи изпод ръката на Юда, и остана независим до днес. Тогава, в същото време, отстъпи и Ливна.
\par 23 А останалите дела на Иорама, и всичко каквото извърши, не са ли написани в Книгата на летописите на Юдовите царе?
\par 24 И Иорам заспа с бащите си, и биде погребан с бащите си в Давидовия град; а вместо него се възцари син му Охозия.
\par 25 Охозия, син на Юдовия цар Иорам, се възцари в дванадесетата година на Израилевия цар Иорам, Ахаавовия син.
\par 26 Охозия бе двадесет и две години на възраст, когато се възцари, и царува една година в Ерусалим. Името на майка му бе Готолия, внучка на Израилевия цар Амрий.
\par 27 Той ходи в пътя на Ахаавовитя дом, и извърши зло пред Господа, Ахаавовия дом; защото бе зет на Ахаавовия дом.
\par 28 И отиде на война с Иорама Ахаавовия син против сирийския цар Азаил за Рамот-галаад; а сирийците раниха Иорама.
\par 29 И тъй, цар Иорам се върна в Езраел, за да се цери от раните, които му нанесоха сирийците в Рама, когато воюваше против сирийския цар Азаил. И Юдовият цар Охозия, Иорамовият син, слезе в Езраел, за да види Иорама, Ахаавовия син, защото бе болен.

\chapter{9}

\par 1 Тогава пророк Елисей повика един от пророческите ученици та му раче: Препаши кръста си и вземи в ръката си тоя съд с миро та иди в Рамот-галаад;
\par 2 и като стигнеш там, подири в същото място Ииуя, син на Иосафата, Намесиевия син; тогава влез при него , дигни го отсред братята му, и въведи го в една вътрешна стая.
\par 3 После вземи съда с мирото та го излей на главата му, и речи: Така казва Господ: Помазах те цар над Израиля. Тогава отвори вратата и бягай без да се забавиш.
\par 4 И тъй, младежът, младият пророк, отиде в Рамот-галаад.
\par 5 И като стигна, ето, военачалниците седяха заедно ; и рече: Имам дума към тебе, о военачалниче. И Ииуй рече: Към кого от всички нас? А той каза: Към тебе военачалниче.
\par 6 И той стана та влезе в къщата. Тогава пророкът изля мирото на главата му и му рече: Така казва Господ Израилевият Бог: Помазах те цар над Господните люде, над Израиля.
\par 7 И ти да поразиш дома на господаря си Ахаава, за да отмъстя върху Езавел за кръвта на слугите Си пророците, за кръвта на всичките Господни слуги.
\par 8 Защото целият Ахаавов дом ще бъде изтребен, и ще погубя от Ахаава всеки от мъжки пол, както малолетния, така и пълнолетния в Израил;
\par 9 и ще направя дома на Ахаава както дома на Еровоама, Наватовия син и както дома на Вааса, Ахиевия син.
\par 10 А Езавел - кучетата ще я изядат в градския край на Езраел, и не ще да има кой да я погребе. И като отвори вратата, побягна.
\par 11 Тогава Ииуй излезе при слугите на господаря си, и един от тях ме каза: Всичко добре ли е? Защо е дошъл тоя луд при тебе? А той им рече: Вие знаете човека и това, което каза.
\par 12 А те рекоха: Не е вярно; я ни кажи. А той рече: Така и така ми говори, като рече: Така казва Господ: Помазах те цар над Израиля.
\par 13 Тогава побързаха, и като взеха всеки дрехата си туриха ги под Ииуя на върха на стълбата, и засвириха с тръба и рекоха: Ииуй се възцари.
\par 14 Така Ииуй, син на Иосафата, Немесиевият син, направи заговор против Иорама. (А Иорам с целия Израил пазеше Рамот-галаад от сирийския цар Азаил;
\par 15 а цар Иорам беше се върнал в Езраел за да се цери от раните които сирийците му баха нанесли, когато войваше протви сирийския цар Азаил). И рече Ииуй: Ако това е вашето мнение, то да не излезе никой да побегне из града за да отиде и извести това в Езраел.
\par 16 Тогава Ииуй се качи на колесницата та отиде в Езаел, защото Иорам лежеше там. И Юдовият цар Охозия бе слязъл да види Иорама.
\par 17 И стражът, който стоеше на кулата в Езраел, когато съгледа дружината на Ииуя, като идеше, рече: Виждам една дружина. И рече Иорам: Вземи конник и прати да ги посрещне; и нака рече: С мир ли идеш ?
\par 18 Отиде, прочее, конник да го посрещне, и рече: Така казва царят: С мир ли идеш ? А Ииуй рече: Що имаш ти с мира? обърни се та иди отзаде ми. И стражарят извести, казвайки: Пратеникът стигна до тях, но не се връща.
\par 19 Тогава изпрати втори конник, който отиде при тях та рече: Така казва царят: С мир ли идеш ? А Ииуй отговори: Що имаш ти се мира? обърни се та иди отзаде ми.
\par 20 И стражът извести, казвайки: Стигна до тях, но и той не се връща; а канането прилима на карането на Ииуя, Намасиевия син, защото кара като луд.
\par 21 Тогава рече Иорам: Впрегнете. И впрегнаха колесницата му. И Израилевият цар Иорам и Юдовият цар Охозия излязоха, всеки в колесницата си, и отидоха да посрещнат Ииуя, и намериха го в нивата на езраелеца Навутей.
\par 22 И Иорам, като вия, Ииуя, рече: С мир ли идеш , Ииуе? А той отговори: Какъв мир докле са толкоз вного блудствата и чародействата на майка ти Езавел?
\par 23 Тогава Иорам обърна юзда та побягна, като думаше на Охозия: Предателство, Охозие!
\par 24 А Иий дръпна лъка си с пълна сила та устрели Иорама между плещите му така щото стрелата излезе през сърцето му; а той се пригърби в колесницата си.
\par 25 Тогава рече Ииуй на военачалника си Видкар: Вземи та го хвърли в оная част от нивата, която пренадлежи на езраелеца Навутей; защото спомни си, че, когато аз и ти яздехме подир баща му Ахаава, Господ произнесе против него това пророчество:
\par 26 Не видях ли вчера кръвта на Навутея и кръвта на синовете му, казва Господ? И в тая част от нивата ще ти въздам, казва Господ. Сега, прочее, дигни, хвърли го в тая част, според Господното слово.
\par 27 А Юдовият цар Охозия, като видя това побягна през пътя на градинската къща. И ииу се спусна подир него и рече: Поразете тогова в колесницата. И поразиха го в негорнището на Гер, близо до Ивлеам; и той побягна в Магедон и умря там.
\par 28 И слугите му го докараха на колесница ва Ерусалим, погребаха го в гроба му с бащите му в Давидовия град.
\par 29 А Охозия бе се възцарил над Юда в единадесетата година на Иорама Ахаавовия сен.
\par 30 Тага Ииуй дойде в Езраел и Езавел като чу начерни клепачите на очите си, накити главата си и надникна през прозореца.
\par 31 И, като влизаше Ииуй в портата, те рече: С мир ли идеш , ти Зимриевце, убиецо на господаря си?
\par 32 А той подигна лицето си към прозореца и рече: Кой е от към мене? Кой? И надникнаха към него двама трима скопци.
\par 33 И рече: Хвърлете я долу. И те я хвърлиха долу; и пръсна от кръвта й по стената и по конете; и той я сгази.
\par 34 И като влезе та яде и пи, рече: Идете сега и вижте тая проклета и погребете я, защото е царска дъщеря.
\par 35 И отидоха да я погребат, но не намериха от нея освен черепа, нозете и дланите на ръцете.
\par 36 И когато се върнаха та му явиха това, той рече: Това е словото, което Господ говори чрез слугата Си тесвиеца Илия, като рече: В градския край на Езраел кучетата ще изядат месата на Езавел;
\par 37 и трупът на Езавел в градския край на Езраел ще бъде като гной по лицето на полето, така щото не ще да рекат: Това е Езавел.

\chapter{10}

\par 1 А Ахаав имаше седемдесет сина в Самария. И Ииуй написа писма та прати в Самария до езраелските началници, до старейшините и до възпитателите на Ахаавовите деца, в които рече:
\par 2 Щом пристигне до вас това писмо, понеже синовете на господаря ви са при вас, и имате колесници и коне, укрепен град и оръжия,
\par 3 то вижте кой е най-добър и на-способен от синовете на господаря ви та го поставете на бащиния му престол, и бийте се за дома на господаря си.
\par 4 Но те твърде много се уплашиха, и рекоха: Ето, двамата царе не устояха пред него; и как ще устоим ние?
\par 5 И така, домоуправителят, градоначалникът, старейшините и възпитателите на децата пратиха до Ииуя да рекат: Ние сме твои слуги, и ще сторим всичко, каквото ни кажеш; няма да направим никого цар; стори каквото ти се вижда угодно.
\par 6 Тогава им писа второ писмо, в което рече: Ако сте откъм мене, и ако послушате моя глас, отнемете главите на тия човеци, синовете на господаря ви, и утре по това време, дойдете при мене в Езраел. (А царските синове, седемдесетте човека, бяха при градските голезци, които ги възпитаваха).
\par 7 И като стигна писмото до тях, хванаха царските синове, седемдесетте човека, та го изклаха, и туриха главите им в кошници та му ги пратиха в Езраел.
\par 8 И един пратеник дойде та му извести, казвайки: Донесоха главите на царските синове. А той рече: Турете ги на два купа във входа на портата да стоят до утре.
\par 9 И на сутрента излезе та застана и рече на всичките люде: Вие сте праведни; ето, аз направих заговор против господря си и го убих; но кой изби всички тези?
\par 10 Знайте сега, че няма да падне на земята нищо от Господното слово, което Господ говори против Ахаавовия дом; защото Господ извести онова, което говори чрез слугата Си Илия.
\par 11 И тъй, Ииуй порази всичкоте останали от Ахаавовия дом в Езраел, и всичките му големци, близките му приятели и свещениците му, докле не му остави остатък.
\par 12 Сетне стана та тръгна и дойде в Самария. И по пътя, като бягаше близо при овчарската стригачница,
\par 13 Ииуй срещна братята на Юдовия цар Охозия, и попита: Кои сте вие? А те отговориха: Ние сме братя на Охозия, и слизаме да поздравим чадата на царя и чадата на царицата.
\par 14 И рече: Хванете ги живи. И хванаха ги живи та ги изклаха при рова на стригачницата, четиридесет и двама човека; не остави ни един от тях.
\par 15 И като тръгна от там събра се с Ионадава Рихавовия сен, който идеше да ги посрещне; Ииуй го поздрави и му рече: Право ли е твоито сърце към мене , както е моето сърце към твоето? И Ионадав отговори: Право е. Ако е тъй, каза Ииуй , дай ръката си. И Ииуй го качи при себе си на колесницата, и рече:
\par 16 Дойди с мене та виж ревността ми за Господа. Така го качиха в колесницата му.
\par 17 И когато стигна в Самария, поразяваше всичките останали от Ахаава в Самария догде го изтреби, според словото, което Господ говори на Илия.
\par 18 Тогава Ииуй събра всичките люде та им рече: Ахаав е малко служил на Ваала; Ииуй ще му служи много.
\par 19 Сега, прочее, повикайте ми всичките пророци на Ваала, всичкоте му служители и всичките му жреци; никой де не отсъствува, защото имам да принеса голяма жертва на Ваала; никой, който отсъствува, нама да остане жив. Но Ииуй направи това с хитрост, с намарение да изтреби Вааловите служители.
\par 20 И тъй, Ииуй рече: Прогласете тържествено събрание за Ваала. И те прогласиха.
\par 21 И Ииуй прати на целия Израил, та дойдоха всичките Ваалови служители, така че не остана никой, който да не дойде. Дойдоха в капището на Ваала; и Вааловото капище се напълни от край до край.
\par 22 И каза на одеждопазителя: Извади одежди за всичките Ваалови служители. И той им извади одежди.
\par 23 Тогава Ииуй и Ионадав, Рихавовият син, влязоха във Вааловото капище; и рече на Вааловите служители: Прегледайте и внимавайте да няма някой между вас някой от слегите на Иеова, но да мъдат само служители на Ваала.
\par 24 И когато влязоха да принесат жертви и всеизгаряния, Ииуй нареди отдън осемдесет мъже, на които рече: Който остава да избяга някой от тия човеци, които доведох в ръцете ви, животът му ще се вземе вместо неговия живот.
\par 25 И като свърши принасянето на всеизгарянето, Ииуй рече на телохранителите и на полководците: Влезте, избийте ги; никой да не избяга. И телохранителите и полководците ги избиха с острото на ножа и ги изхвърлиха вън. После, кото отидоха в града на Вааловото капище,
\par 26 извадиха кумирите на Вааловото капище та ги озгориха,
\par 27 строшиха идола на Ваала, съсипаха Вааловото капище, и направиха го бунище, както е до днес.
\par 28 Така Ииуй изтреби Ваала от Израиля.
\par 29 Но Ииуй не са естови от греховете на Еровоама, Наватовия син, който непреви Израиля да греши, то ест , от златните телета, които бяха във Ветил и в Дан.
\par 30 Тогава Господ рече на Ииуя: Понеже ти добре стори, като извърши това, което е право пред очите Ми, и направи на Ахаавовия дом напълно според това, което бе в сърцето Ми, затова твоите синове до четвъртото поколение ще седят на Израилевия престол.
\par 31 Обаче, Ииуй не внимаваше да ходи с цялото си сърце в закона на Господа Израилевия Бог; не се остави от греховете на Еровоама, с които направи Израиля да греши.
\par 32 В това време Господ почна да кастри Израиля; защото Азаил ги порази във всичките Израилеви предели,
\par 33 от Иордан на изток цялата галаадска земя, гадците, рувимците и манасийците, от Ароир при потока Арнон и Галаад и Васан.
\par 34 А останалите дела на Ииуя, всичко що извърши и всичките му юначества, не са ли написани в Книгата на лутописите на Израилевите царе?
\par 35 И Ииуй заспа с бащите си; и погребаха го в Самария. И вместо него се възцари син му Иоахаз.
\par 36 И времето, през което Ииуй царува над Израиля в Самария, бе двадесет и осем години.

\chapter{11}

\par 1 А Готолия Охозиевата майка, като видя, че синът й умря, стана та погуби целия царски род.
\par 2 Но Иосавеета, дъщеря на цар Иорама, сестра на Охозия, взе Иоаса, Охозиевия син, та го открадна изсред царските синове, като ги убиваха; и скриха го от Готолия в спалнята, заедно с дойката му, та не биде убит.
\par 3 И беше при нея, скрит в Господния дом, шест години; а Готолия царуваше над земята.
\par 4 Но в седмата година Иодай прати и, като взе стотниците на палачите и на телохранителите, доведе ги при саба си в Господния дом; после, като направи договор с тях и ги закле в Господния дом, показа им царския син.
\par 5 И заповяда им казвайки: Ито какво трябва да направите: една трета от вас, които постъпвате на служба в събота, нека пази стража при царската къща,
\par 6 една трета при портата Сур и една трета при портата, който е зад телохранителите; така да пазите стража при къщата, за да не влезе никой.
\par 7 И всички от вас, и от двата отдела, които оставяте службата в събота, нека пазят стража при Господния дом около царя.
\par 8 И да окръжавате царя от всука страна, като всеки държи оръжията си в ръка; а който би влязъл в редовете да бъде убит; и да бъдете с царя при излизането му и при влизането му.
\par 9 И тъй, стотниците извършиха всичко според както заповяда свещеник Иодай, и взеха всеки мъжете си - ония, които щяха да постъпят на служба в събота, и ония, които щаха да оставят службата в събота - та дойдоха при свещеника Иодай.
\par 10 И свещеникът даде на стотниците цар Даводовите копия и щитове, които бяха в Господния дом.
\par 11 И телохранителите, всеки с оръжията си в ръка, стояха около церя, от дясната страна на дома до лявата му страна, край олтара и край дома.
\par 12 Тогава Иодай изведе царския син та положи на него короната и му връчи божественото заявление; и направиха го цар и помазаха го; после изплескаха с ръце и казаха: Да живее церят!
\par 13 А Готолия, като чу вика от телохранителите и от людете, дойде при людете в Господния дом,
\par 14 и погледна, и, ето, царят стоеше при стълба, според обичая, и военачалниците и тръбите при царя и всичките лйде от страна се радваха и свиреха с тръбите. Тогава Готолия раздра дрехите си и извика: Заговор! заговор!
\par 15 И свещеник Иодай заповяда на стотниците, поставени над силите, та им рече: Изведете я вън от редавете, и който би я последвал, убийте го с меч; защото свещеникът беше казал: Да не бъде убита в Господния дом.
\par 16 И така отстъпиха й място; и тя отиде през пътя на конския вход в царската къща; и там биде убита.
\par 17 Тогава Иодай направи завет между Господа и царя и людете, че ще бъдат Господни люде - също и между царя и людете.
\par 18 И всичките люде от земята влязоха във Вааловото капища та го събориха, жертвениците му и кумирити му изпотрошиха съвсем, и Вааловия жрец Матан убиха пред жертвениците. И свещеникът постави надзиратели над Господния дом.
\par 19 Тогава, като взе стотниците, палачите, телохранителите и всичките люде от страната, изведоха царя от Господния дом; и дойдоха в църската къща през пътя към портата на телохранителите, и той седна на царския престол.
\par 20 Така всичките люде от страната се зарадваха, и градът се успокои; а Готолия убиха с меч при царската къща.
\par 21 Иоас беше на седем години, когато се възцари.

\chapter{12}

\par 1 В седмата година на Ииуя се възцари Иоас, и царува четиридесет години в Ерусалим; а името на майка му бе Савия, от Вирсавее.
\par 2 И Иоас вършеше това, което бе право пред Господа, през всичкото време, когато го наставляваше свещеник Иодай.
\par 3 Обаче високите места не се отмъхнаха; людете още жертвуваха и кадяха по високите места.
\par 4 А Иоас рече на свещениците: Всичките пари от посветените неща, които се внасят в Господния дом, и парите от всекиго, който преминава преброението , и парите от всукиго според оценката му, и всичките пари, които би дошло някому в сърцето да принесе в Господния дом,
\par 5 свещениците нека ги вземат у себе си, всеки от познатия си, и нека поправят разваленото на дома, гдето се набери развалено.
\par 6 Но понеже до двадасат и тертата година на цар Иоаса свещениците не бяха поправяли разваленото на дома,
\par 7 затова цар Иоас повика свещеник Иодай и другите свещеници та им рече: Защо не поправихте разваленото на дома? Сега, прочее, не вземайте вече пари от познатите си, но дайте събраното за разваленото на дома.
\par 8 И свещениците склониха нито да вземат вече пари от людете, нито да бъдат отговорни за поправяне разваленото на дома.
\par 9 А свещеник Иодай взе един котвчег и като отвори дупка на капака му, тури го при олтара отдясно на входа в Господния дом; и свещениците, които пазеха вратата, туриха в него всичките пари, които се внасяха в Господния дом.
\par 10 И когато виждаха, че имаше много пари в ковчега, царският секретар и първосвещеникът отиваха та завързваха в мешци намерените в Господния дом пари, като ги брояха.
\par 11 И даваха преброените пари в ръцете на ония, които вършеха делото, които надзираваха Господния дом; а те ги иждивяваха по дърводелците и строителите, които работеха в Господния дом,
\par 12 и по зидарите и по каменоделците, и за купеване на дървета и дялани камъни за поправяне разваленото на Господния дом, и за всяко иждивение за поправката на дома.
\par 13 Но от парите, които се внасяха в Господния дом, не се направиха сребърни чаши, щипци, легени, тръби, нито какви да било златни съдове или сребърни съдове;
\par 14 но даваха ги на работниците, и поправяха с тях Господния дом.
\par 15 Освен това, не търсеха сметка от човеците, на които даваха парите за да се разпределят на работниците, защото постъпваха честно.
\par 16 Обаче , парите принасяни за преспътление и парите принасяне за грях, не се внасяха в Господния дом; те бяха за свещениците.
\par 17 Тогава дойде сирийският цар Азаил та воюва против Гет и го превзе; и Азаил обърна лицето си за да дойде против Ерусалим.
\par 18 Затова Юдовият цер Иоас взе всичките посветени неща, които бащите му Иосафат, Иорам и Охозия, Юдовите царе, бяха посветили, и посветените от него неща, и всичкото злато, което се намери в съкровищата на Господния дом и на църската къща, та го прати на сирийския цар Азаил; и той се оттегли от Ерусалим.
\par 19 А останалите дела на Иоаса, и всичко що извърши, не са ли написани в Книгота на летописите на Юдовите царе?
\par 20 И слугите му се подигнаха и, като направиха заговор, убиха Иоаса в къщата Мило, в надолнището към Сила.
\par 21 Защото слугите му Иозахар Семеатовият син, и Иозавад, Сомировият син, го поразиха, та умря; и погребаха го с бещите му в Давидовия град; и вместо него се възцари син му Амасия.

\chapter{13}

\par 1 В двадесет и третата година на Юдовия цар Иоас, Охозиевия син, се възцари Иохаз, Ииуевият син, над Израиля в Самария, и царува седемнадесет години.
\par 2 Той върши зло пред Господа, като последва греховете на Еровоама, Наватовия син, с които направи Израиля да греши; не се остави от тях.
\par 3 Затова гневът на Господа пламна против Израиля, и Той постоянно ги предаваше в ръката на сирийския цар Езалия, и в ръката на Венадада, Азаиловия син.
\par 4 Тогава Иоахаз се помоли Господу; и Господ го послуша, защото видя притеснението на Израиля, как сирийският цар ги притесняваше.
\par 5 (И Господ даде избевител на Израиля, така щото се отърваха изпод ръката на сирийците, та израилтяните живееха в жилищата си, както по-напред;
\par 6 обаче не се оставиха от греховете на дома на Еровоама, с които направи Израиля да греши, а в тях ходиха; и ашерата още стоеше в Самария).
\par 7 Защото сирийският цар не беше оставил на Иоахаза от людете повече от петдесет конници, десет колесници и десек хиляди пешаци; защото сирийският цар беше ги погубил и беше ги направил като стъпкана пръст.
\par 8 А останалите дела на Иоахава, и всичко що извърши и юначествата му, не са ли написани в Книгата на летописите на Израилевите царе?
\par 9 И Иоахаз заспа с бящите си, и погребаха го в Самария; а вместо него се възцари син му Иоас.
\par 10 В тридесет и седмата година на Юдовия цар Иоас, се възцари Иоас, Иоахазовият син, над Израиля в Самария, и царува шестнадесет години.
\par 11 Той върши зло пред Господа; не остави ни от един от греховете на Еровоама, Наватовия син, с които направи Израиля да греши, а в тях ходи.
\par 12 А останалите дела на Иоаса, и всичко що извърши и юначеството, с което воюва против Юдовия цар Амасия, не са ли написани в Книгата на летописите на Израилевите царе?
\par 13 И Иоас заспа с бащите си; а на престола му седна Еровоам. А Иоас биде погребан в Самария с Израилевите цари.
\par 14 В това време Елисей се разболя от болестта, от която умря. И Израилевитят цар слезе при нето та плака над него, като рече: Татко мой! татко мой! колесница Израилева и конница негова!
\par 15 А Елисей му каза: Вземи лък и стрели. И той си взе лък и стрели.
\par 16 Тогава рече на Израилевия цар: Тури ръката си на лъка. И като тури ръката си, Елисей положи ръцете си върху ръцете на царя и рече:
\par 17 Отвори источния прозорец. И той го отвори. И рече Елисей: Стреляй. И той устрели. И рече: Стрелата на Господното спасение! да! стрелата на избавлението то сирийците! защото ще поразиш сирийците в Афек докле ги довършиш.
\par 18 Рече още: Вземи стрелите. И той ги взе. Тогава рече на израилевия цар: Удряй на земята. И той удари три пъти, и престана.
\par 19 А Божият човек се разсърди на него, и рече: Трябваше да удариш пет или шест пъти; тогава щеше да удариш сирийците докле ги довършиш; но сега само три пъти ще поразиш сирийците.
\par 20 И Елисей умря, и погребаха го. А в следната година някой моавски чети опустошаваха земята.
\par 21 И неколцина израилтяни като погребваха един човек, ето, видяха чета: затова хвърлиха човека в гроба на Елисея. А щом стигна човекът та досегна Елисеевите кости, оживя и се изправи на краката си.
\par 22 И сирийският цар Азаил притесняваше Израиля през всичките дни на Иоахаза.
\par 23 Но Господ им показа милост, пожали ги и ги прегледа заради завета Си с Авраама, Исаака и Якова; и отказа да ги изтреби, и не ги отхвърли още от присъствието Си;
\par 24 защото, като умря сирийският цар Азаил, и вместо него се възцари син му Венадад.
\par 25 Иоас, Иоахазовият син, взе обратно от ръката на Венадада, Азаиловия син, градовете, които Азаил беше отнел от ръката на баща му Иоахаза във война. Три пъти го порази Иоас, и взе обратно Израилевите градове.

\chapter{14}

\par 1 Във втората година на Израилевия цар Иоас, Иахавовия син, се възцари Амасия, син на Юдовия цар Иоас.
\par 2 Той бе двадесет и пет години на възраст когато се възцари, и царува двадесет и девет години в Ерусалим. И името на майка му бе Иоадана, от Ерусалим.
\par 3 Той върши това, което бе право пред Господа, но не както баща си Давида; той постъпваше съвсем според както бе постъпвал баща му Иоас.
\par 4 Обаче високите места не се отмахнаха; людете жертвуваха и кадяха по високите места.
\par 5 И щом се закрепи царството в ръката му, той умъртви слугите си, които бяха убили баща му царя.
\par 6 Но чадата на убийците не умъртви, според писаното в книгата на Моисеевия закон, гдето Господ зеповядвайки, каза: Бащите да се не умъртвяват поради чадата, нито чадата да се умъртвяват поради бащите; но всеки да се умъртвява за своя си грух.
\par 7 Той изби десет души от Едома в долината на солта, и, като превзе град Села с бой, нарече го Иокнеил, както се казва и до днес.
\par 8 Тогава Амасия прети човеци до Израилевия цар Иоас, син на Иоахаза, Ииуевия син да кажат: Ела да се погледнем един други в лице.
\par 9 А Израилевият цар Иоас прати до Юдовия цар Амасия да рекат: Ливанският трън пратил до ливанския кедър да кажат: Дай дъщеря си на сина ми за жена. Но един звяр, който бил в Ливан, заминал та стъпкал тръна.
\par 10 Ти наистина порази Едома, и надигна се сърцето ти; радвай се на славата си и седи у дома си; защо да се заплиташ за своята си вреда та да паднеш, ти и Юда с тебе?
\par 11 Но Амасия не послуша. Прочее, Израилевият цар Иоас отиде, та се погледнаха един други в лице, той и Юдовият цар Амасия, във Витсемес, който принадлежи на Юда.
\par 12 И Юда биде поразен пред Израиля; и побягнаха всеки в шатъра си.
\par 13 И Израилевият цар Иоас хвана във Ветсемес Юдовия цар Амасия, син на Иоаса Охозиевия син; и дойде в Ерусалим та събори четиристотин лакти от Ерусалимската стена, от Ефремовата порта до портата на ъгъла.
\par 14 И като взе всичкото злато и сребро, и всичките съдове, които се намираха в Господния дом и в съкровищата на царската къща, тоже и заложници, върна се в Самария.
\par 15 (А останалите дела, които извърши Иоас и Юначествата му, и как войва с Юдовия цар Амасия, не са ли написани в Книгата на летописите на Израилевите царе?
\par 16 И Иоас заспа с бащите си, и биде погребан в Самария с Израилевите царе; и вместо него се възцари син му Еровоам).
\par 17 А след смъртта на Израилевия цар Иоас, Иоахазовия син, Юдовият цар Амасия, Иоасовият син, живя петнадесет години.
\par 18 А останалите дела на Амасия не са ли написани в Книгата на летописите на Юдовите царе?
\par 19 И понеже направиха заговор против него в Ерусалим, той побягна в Лахис; но пратиха след него в Лахис та го убиха там.
\par 20 И докараха го на коне; и биде погребан в Ерусалим с бащите си в Давидовия град.
\par 21 Тогава всичките Юдови люде взеха Азария, който бе шестнадесет години на възраст, та го направиха цар вместо баща му Амасия.
\par 22 Той съгради Елат и го възвърна на Юда след като баща му царят заспа с бащите си.
\par 23 В петнадесетата година на Юдовия цар Амасия, син на Иоаса, се възцари в Самария Еровоам, син на Израилевия цар Иоас, и царува четиридесет и една година.
\par 24 Той върши зло пред Господа; не се остави ни от един от греховете на Еровоама Наватовия син, с които направи Израиля да греши.
\par 25 Той възстанови границата на Израиля от прохода на Емат до полското море, според словото, което Господ Израилевият Бог говори чрез слугата Си Иона Аматиевия син, пророкът, който бе от Гетефер.
\par 26 Защото Господ видя, че притеснението на Израиля беше много горчиво, понеже нямаше нито малолетен, нито пълнолетен, който да помогне на Израиля.
\par 27 И Господ не рече да изличи изпод небето името на Израиля, но ги избави чрез ръката на Еровоама Иоасовия син.
\par 28 А останалите дела на Еровоама, и всичко що извърши, и юначествата му, как воюва, и как възвърна на Израиля Дамаск и Емат, които бяха подчинени на Юда, не са ли написани в Книгата на летописите на Израилевите царе?
\par 29 Е Еровоам заспа с бащите си Израилевите царе; и вместо него се възцари син му Захария.

\chapter{15}

\par 1 В двадесет и седмата година на Израилевия цар Еровоама, се възцари Азария, син на Юдовия цар Амасия.
\par 2 Той беше шестнадест години на възраст когато се възцари, и царува петдесет и две години в Ерусалим; и името на майка му бе Ехолия, от Ерусалим.
\par 3 Той върши това, което бе право пред Господа, напълно според както беше сторил баща му Амасия.
\par 4 Но високите места не се отмахнаха; людете още жертвуваха и кадяха по високите места.
\par 5 Но Господ порази царя, така щото той остана прокажен до деня на смъртта си, та живееше в отделна къща. А царският син Иотам бе домовладетел, и съдеше людете на земята.
\par 6 А останалите дела на Азария, и всичко що извърши, не са ли написани в Книгата на летописите на Юдовите царе?
\par 7 И Азария заспа с бащите си, и погребаха го с бащите му в Давидовия град; и вместо него се възцари син му Иотам.
\par 8 В тридесет и осмата година на Юдовия цар Азария, се възцари над Израиля в Самария Захария Еровоамовия син, и царува шест месеца.
\par 9 Той върши зло пред Господа, както сториха бащите му; не се остави от греховете на Еровоама Наватовия син, с които направи Изралия да греши.
\par 10 А Салум, Явисовият син, направи заговор против него, и порази го пред людете и го уби; и вместо него сам се възцари.
\par 11 А останалите дела на Захария, ето, написани са в Книгата на летописите на Израилевите царе.
\par 12 Това беше изпълнение на словото, което Господ говори на Ииуя като рече: Твоите синове ще седят на Израилевия престол до четвъртото поколение. Така и стана.
\par 13 Салум, Явисовият син, се възцари в тридесет и деветата година на Юдовия цар Озия, и царува един месец в Самария.
\par 14 Защото Манаим, Гадииният син, като излезе от Терса, дойде в Самария, и като порази Самума, Ависовия син в Самария и го уби, сам се възцари вместо него.
\par 15 А останалите дела на Салума, и заговорът, който направи, ето, написани са в Книгата на летописите на Израилевите царе.
\par 16 Тогава Манаим порази Тапса и всичко що ме в нея, и околностите й, от Терса. Порази я понеже не му отвориха; и разпра всички бременни жени в нея.
\par 17 В тридесет и деветата година на Юдовият цар Азария, се възцари над Израиля Манаим, Гадииният син, и царува десет години в Самария.
\par 18 Той върши зло пред Господа; през целия си живот не се оставяше от греховете на Еровоама, Наватовия син, с които направи Израиля да греши.
\par 19 Тогава асирийският цар Фул дойде против земята; а Манаим даде на Фула хиляда таланта сребро, за да му помогне да закрепи царството в ръката си.
\par 20 И Манаим с несилие събра среброто от Израиля, от всичките големи богаташи по петдесет сребърни сикли, за да даде на асирийския цар. И така асирийският цар се върна, и не остана там в земята.
\par 21 А останалите дела на Манаима, и всичко що извърши, не са ли написани в Книгата на летописите на Израилевите царе?
\par 22 И Манаим заспа с бащите си; и вместо него се възцари син му Факия.
\par 23 В петдесетата година на Юдовия цар Азария, се възцари над Израиля в Самария Факия, Манаимовият син, и царува две години.
\par 24 Той върши зло пред Господа; не се остави от греховете на Еровоама, Наватовия син, с които направи Израиля да греши.
\par 25 И полководецът му Факей, Ромелиевият син, направи заговор против него та го порази в Самария, във вътрешната царска къща, с Аргова и Арие, като имаше със себе си и петдесет мъже от галаадците; и уби го та сам се възцари вместо него.
\par 26 А останалите дела на Факия, и всичко що извърши, ето, написани са в Книгата на летописите на Израилевите царе.
\par 27 В петдесет и втората година на Юдовия цар Азария, се възцари над Израиля в Самария Факей, Ромеловият син, и царува двадесет години.
\par 28 Той върши зло пред Господа; не се остави от греховете на Еровоама, Наватовия син, с които направи Израиля да греши.
\par 29 В дните на Израилевия цар Факей, дойде асирийският царТеглат-феласар та превзе Иион, Авел-вет-мааха, Янох, Кадес, Асор, Галаад и Галилея цялата Нефталимова земя, и заведе жителите им пленници в Асирия.
\par 30 А Осия, син на Ила, направи заговор против Факея Ромелиевия син и като го порази, уби го, и сам се възцари вместо него в двадесетата година на Иотама, Озиевия син.
\par 31 А останалите дела на Факея, и всичко що извърши, ето написани са в Книгата на летописите на Израилевите царе.
\par 32 Във втората година на Израилевия цар Факей, Ромелиевия син, се възцари Иотам, син на Юдовия цар Озия.
\par 33 Той беше двадесет и пет години на възраст, когато се възцари и царува шестнадесет години в Ерусалим; а името на майка му бе Еруса Садокова дъщеря.
\par 34 Той върши това, което бе право пред Господа; постъпваше съвсем както бе направил баща му Озия.
\par 35 Но високите места не се отмахнаха; людете още жертвуваха и кадяха по високите места. Той построи горната порта на Господния дом.
\par 36 А останалите дела на Иотама, и всичко що извърши, не са ли написани в Книгата на летописите на Юдовите царе?
\par 37 В онова време Господ почна да праща против Юда сирийския цар Расин и Факея, Ромелиевия син.
\par 38 А Иотам заспа с бащите си и биде погребан с бъщите си в града на баща си Давида; и вместо него се възцари син ме Ахаз.

\chapter{16}

\par 1 В седемнадесетата година на Факея, Ромеловия син, се възцари Ахаз, син на Юдовия цар Иотам.
\par 2 Ахаз бе двадесет години на възраст, когато се възцари, и царува шестнадесет години в Ерусалим; но не върши това, което бе право пред Господа своя Бог, както баща му Давид,
\par 3 но ходи в пътя на Израилевите царе, и даже преведе сина си през огъня според мерзостите на народите, които Господ изпъди пред израилтяните.
\par 4 Той жертвуваше и кадеше по високите места, и по върховете и под всяко зелено дърво.
\par 5 Тогава сирийският цар Расин и Израилевият цар Факей, Ромелиевият син, дойдоха в Ерусалим да воюват, и обсадихаАхаза, но не можаха да му надвият.
\par 6 В онова време сирийският цар Расин възвърна Елат под Сирия, и изпъди юдеите из Елат; асирийците дойдоха в Елат та живееха там, гдето са и до днес.
\par 7 Затова Ахаз прати човеци до асирийския цар Теглат-фаласар да кажат: Аз съм твой слуга и твой син; възлез да ме избавиш от ръката на сирийския цар и от ръката на на Израилевия цар, който се подигна против мене.
\par 8 А Ахаз взе среброто и златото, което се набери в Господния дом и д съкровищницата на царската къща, та ги проти подарък на асирийския цар.
\par 9 И асирийския цар го послуша и възлезе против Дамаск та го превзе, и тведе жителите му в Кир, а Расина уби.
\par 10 А когато цар Ахаз отиде в Дамаск да посрещне асирийския цар Теглат-феласар, видя жертвеника в Дамаск; и цар Ахаз изпрати на свещеника Урия образа на жертвеника, и рисунка на него, точно според направата му.
\par 11 И свещеник Урия направи жертвеник точно според това, което цар Ахаз прати от Дамаск; така го направи свещеник Урия догде се върна цар Ахаз от Дамаск.
\par 12 И когато се върна царят от Дамаск, видя жертвеника; и царят се приближи при жертвеника та направи принос върху него.
\par 13 И върху тоя жертвеник изгори всеизгарянето си и хлебния си принос, и изля възлиянието си, и поръси кръвта на примирителните си приноси.
\par 14 А медният олтар, който бе пред Господа, премести от лицето на дома, от мястото му между жертвеника и Господния дом, и го постави на северната страна от жертвеника.
\par 15 И цар Ахаз заповяда на свещеника Урия, казвайки: На великия жертвеник принасяй утрешното всеизгаряне, вечерния хлебен принос, царското всеизгаряне и хлебен принос, с всеизгарянето принасяно от всичките люде на земята и хлебния им принос и възлиянията им, и ръси върху него всичката кръв на всеизгарянето и всичката кръв на жертвата; а медният олтар ще служи за мене за да се допитвам до Господа .
\par 16 И свещеник Урия стори така, точно според както му заповяда цар Ахаз.
\par 17 А цар Ахаз отсече страничните плочи на подножията, и дигна от там умивалника; и свали морето от медните волове, които бяха под него, та го тури на каменен под.
\par 18 И покрития път за съботата, който бяха построили в къщата, и външния вход за царя, премести в Господния дом, поради асирийския цар.
\par 19 А останалите дела, които извърши Ахаз, не са ли написани в Книгата на летописите на Юдовите царе?
\par 20 И Ахаз заспа с бащите си, в Давидовия град; а вместо него се възцари син му Езекия.

\chapter{17}

\par 1 В дванадесетата година на Юдовия цар Ахаз, се възцари в Самария над Израиля Осия, син на Ила, и царува девет години.
\par 2 Той върши зло пред Господа, обаче, не както Израилевите царе, които бяха преди него.
\par 3 Против него възлезе асирийският цар Салманасар, комуто Осия стана слуга, като му даваше данък.
\par 4 А след време асирийският цар откри заговор в Осия, защото той бе пратил човеци до египатския цар Сов, и не беше дал данък на асирийския цар, както беше правил всяка година; затова асирийският цар го затвори и върза в тъмница.
\par 5 И асирийският цар опустошаваше цялата земя, и като отиде в Самария обсаждаше я три години.
\par 6 В деветата година на Осия, асирийският цар превзе Самария, отведе Израиля в плен в Асирия, и засели ги в Ала и в Авор при реката Гозан и в мидските градове.
\par 7 А това стана, защото израилтяните бяха съгрешили на Господа своя Бог, Който ги изведе от Египетската земя, изпод ръката на египетския цар Фараона, като бяха почитали други богове,
\par 8 и бяха ходили в наредбите на народите, които Господ изпъди пред израилтяните, и в наредбите , които Израилевите царе узакониха.
\par 9 И израилтяните бяха вършили скришно работи, които не бяха прави пред Господа техния Бог, и бяха си сторили високи места във всичките си градове, от стражарска кула до укрепен град.
\par 10 Също бяха си издигнали кумири и ашери върху всеки висок хълм и под всуко зелено дърво;
\par 11 и там бяха ходили по всичките високи места, както народите, които Господ изпъди пред тях, и бяха вършили зли дела, с които разгневиха Господа,
\par 12 и бяха служили на идолите, за които Господ им беше казал: Не вършете това нещо.
\par 13 Но пак чрез всичките пророци и всичките гледачи Господ беше предупреждавал Израиля и Юда, казвайки: Върнете се от злите си пътища, и пазете Моите заповеди и Моите повеления, съвършено според закона, който дадох на бащите ви, и който ви пратох чрез слугите Си пророците.
\par 14 При все това, те не бяха послушали, но бяха закоравили врата си както врата на бащите им, които не повярваха в Господа своя Бог.
\par 15 Бяха отхвърлили повеленията Му и завета, който направи с бащите им, и заявленията, с които им заявяваше, и бяха последвали идолите и станали суетни, и бяха последвали народите около тях, за които Господ им беше заповядал да не правят както тях.
\par 16 И бяха оставили всичките заповеди на Господа своя Бог та си бяхъ направили две леяни телета, направили бяха ашери, и се кланяли на цялото небесно множество, и бяха служили на Ваала.
\par 17 Те бяха превеждали синовете си и дъщерите си през огъня, чародействували и гадаели, и продавали себе си да вършат зло пред Господа та бяха Го разгневили.
\par 18 Затова, Господ се разгневи много против Израиля и ги отхвърли от лицето Си; остана само едното Юдово племе.
\par 19 Па и Юда не беше опазил заповедите на Господа своя Бог, но бяха ходили в наредбите, които Израил беше узаконил.
\par 20 Господ, прочее, отхвърли целия Израилев род, унижи ги, и ги предаде в ръката на разграбителите, докле ги отхвърли от лицато Си.
\par 21 Защото отцепи Израиля от Давидовия дом; и те направиха Еровоама, Наватовия син, цар; и Еровоам оттегли Израиля та да не следват Господа и направи ги де извършват голям грях.
\par 22 И израилтяните бяха ходили във всичките грехове, които Еровоам извърши; не се оставиха то тях
\par 23 докле Господ не отхвърли Израиля от лицето Си, както беше говорил чрез всичкоте Си слуги пророците. Така Израил беде отведен от своята цемя в Асирия, гдето е и до днес.
\par 24 Тогава асирийският цар доведе люде от Вавилон, от Хута, от Ава, от Елат и от Сафаруим та ги засели в самарийските градове вместо израилтяните; и те усвоиха Самария и населиха се в градовете й.
\par 25 А в началото на зеселването си там, те не се бояха от Господа; затова Господ прети между тях лъвове, които убиваха някои от тях.
\par 26 По тая причина те говориха на асирийския цар, казвайки: Людете, които ти преселе и постави в самарийските градове, не знаят начина на служене на местния Бог затова Той прати лъвове между тях, и, ето, убиват ги, понеже не знаят начина на служене на местния Бог.
\par 27 Тогава асирийският цар заповяда, казвайки: Заведете там един от свещениците, които запленихме от там; и нека идат пак и да се заселят там, и свещеникът нека ги научи начина на служене на местния Бог.
\par 28 И така, един от свещениците, които бяха запленили от Самария, дойде та се засели във Ветил, и поучаваше ги как да се боят от Господа.
\par 29 Обаче людете от всеки народ поставиха свои богове, людете от всеки народ в градовете, гдето живееха, и туриха ги в капищата по високите места, които самаряните бяха построили.
\par 30 Вавилонските мъже поставиха Сокхот-венот, а хутайските мъже поставиха Нергал, ематските мъже поставиха Асима,
\par 31 авците поставиха Наваз и Гартак, и сефаруимците горяха децата си на огън за сефаруимските богове Адрамелех и Анамелех.
\par 32 Така те се бояха от Господа, и си поставиха свещеници за восоките места от всичките люде между тях, които служеха в тях в капищата по високите места.
\par 33 Бояха се от Господа, и на съботите си боговете служеха според обичая на народите, измежду които бяха преселени.
\par 34 Дори до днес постъпват според по-предишните обичаи: не се боят от Господа, а нито постъпват според своите си съдби, нито според закона и заповедта, която Господ даде на потомците на Якова, когото нарече Израил,
\par 35 с които Господ бе направил завет, като им заповяда казвайки: Да се не боите от други богове, нито да им се кланяте, нито да им служите, нито да им жертвувате;
\par 36 но от Господа, Който ви изведе из Египетската земя с голяма сила и с издигната мишца, от Него да се боите, Нему да се кланяте и Нему да принасяте жертва.
\par 37 И да внимавате да изпълнявате всякога повеленията, съдбите, закона и заповедта, които Той написа за вас; а от други богове да се не боите.
\par 38 И да не забравяте завета, който направих с вас; и да не се боите от дреги богове.
\par 39 Но от Господа вашия Бог да се боите; и Той ще ви избави от ръката на всичките ви неприятели.
\par 40 Те, обаче, не послушаха, но постъпваха според по-предишните си обичаи.
\par 41 И така, тия народи се бояха от Господа, и служеха на своите идоли; също и чадата им и внуците им постъпват до днес както постъпваха бащите им.

\chapter{18}

\par 1 В третата година на Израилевия цар Осия, Иловия син, се възцари Езекия, син на Юдовия цар, Ахаз.
\par 2 Той беше двадесет и пет години на възраст, когато се възцари, и царува двадесет и девет години в Ерусалим; а името на майка му беше Авия, Захариева дъщеря.
\par 3 Той върши това, което бе право пред Господа, напълно както извърши баща му Давид;
\par 4 събори високите места, изпотроши кумирите, изсече ашерите и сломи медната змия, която Моисей беше направил, защото дори до онова време израилтяните й кадяха; и нарече я Нехущан.
\par 5 На Господа Израилевия Бог упова. Нямаше подобен нему между всичките Юдови царе, ни между ония, които бяха подир него, нито преди него;
\par 6 защото се прилепи към Господа, не престана да Го следва, но опази заповедите, които Господ даде на Моисея.
\par 7 И Господ бе с него; където и да излизаше, той благоуспяваше; и въстана против асирийския цар, и не му слугуваше вече.
\par 8 Той порази филистимците до Газа и до околностите й, от стражарска кула до укрепен град.
\par 9 А в четвъртата година на цар Езекия, която бе седмата година на Израилевия цар Осия, син на Ила, асирийският цар Салманасар възлезе против Самария и я обсаждаше.
\par 10 След три години я превзе; в шестата година на Езекия, която бе деветата година на Израилевия цар Осия, Самария бе превзета.
\par 11 И асирийският цар отведе Израиля в плен в Асирия, и ги настани в Ала и в Авор при реката Гозан и в Мидските градове;
\par 12 защото не послушаха гласа на Господа своя Бог, но престъпиха завета Му, - всичко, което Господният слуга Моисей заповяда, - и не послушаха, и нито го извършиха.
\par 13 А в четиренадесетата година на цар Езекия, асирийският цар Сенахирим възлезе против всичките укрепени Юдови градове и ги превзе.
\par 14 Тогава Юдовият цар Езекия прати до асирийския цар в Лахис да кажат: Съгрешихме; върни се от мене; каквото ми наложиш ще го нося. И тъй, асирийският цар наложи на Юдовия цар Езекия триста таланта сребро и тридесет таланта злато.
\par 15 И Езекия му даде всичкото сребро, което се намери в Господния дом и в съкровищницата на царската къща.
\par 16 В това време Езекия откова златото от вратите на Господния храм и от стълбовете, които той, Юдовият цар Езекия, бе обковал със злато , и го даде на есирийския цар.
\par 17 Но асирийския цар прати от Лахис Тартана, Рапсариса и Рапсака с голява мойске при цар Езекия в Ерусалим. А те възлязоха и дойдоха в Ерусалим. И когато стигнаха, дойдоха та застанаха при водопровода на горния водоем, който е по друма към тепавичарската нива.
\par 18 И когато извикаха към царя, излязоха при тях управителят на двореца Елиаким, Хелкиевият син и секретарят Шевна, и летописецът Иоах, Асафовият син.
\par 19 Тогава Рапсак им рече: Кажете сага на Езекия: Така казва великият цар, асирийският цар, каква е тая увереност, на която уповаваш?
\par 20 Ти казваш: Имам благоразумие и сила за воюване. Но това са само лицемерни думи. На кого, прочее, се надяваш та си въстанал против мене?
\par 21 Виж, ти се надяваш, като че ли на тояга, на оная строшена тръстика на Египет, на която, ако се опре някой, ще се забучи в ръката му та ще я промуши. Такъв е египетския цар Фараон за всички, които се надяват на него.
\par 22 Но ако би речете: На Господа нашия Бог ,поваваме, то Той не е ли Оня, Чиито високи местаи жертвеници премахна Езекия, като рече на Юда и на Ерусалим: Пред тоя олтар в Ерусалим се покланяйте?
\par 23 Сега, прочее, дай човеци в залог на господаря ми асирийския цар; а аз ще ти дам две хиляди коне, ако можеш от своя страна да поставиш на тях ездачи.
\par 24 Как тогава ще отблъснеш един военачалник измежду най-ниските слуги на господаря ми? Но пак уповаваш на Египет за колесници и за конници!
\par 25 Без волята на Господа ли възлязох сега на това място за да го съсипя? Господ ми рече: Възлез против тая земя та я съсипи.
\par 26 Тогава Елиаким, Хелкиевият син, Шевна и Иоах рекоха на Рапсака: Говори, молим, на слугите си на сирийски, защото го разбираме: недей ни говори на юдейски, та да чуят людете, които са на стената.
\par 27 Но Рапсак им рече: Да ли ме е пратил господарят ми само при твоя господар и при тебе да говоря тия думи? не ме ли е изпратил при мъжете, които седят на стената, за да ядат с вас заедно изпражненията си и да пият пикочта си?
\par 28 Тогава Рапсак застана та извика на юдейски със силен глас, като говори, казвайки: Слушайте думата на великия цар, асирийския цар:
\par 29 така казва царят: Да ви не мами Езекия. Защото той не ще може да ви избави от ръката ми.
\par 30 И да ви не прави Езекия да уповавате на Господа, като казва: Господ непременно ще ни избеви, и тоя град няма да бъде предаден в ръката на асирийския цар.
\par 31 Не слушайте Езекия, защото така казва асирийският цар: Направете спогодбе с мене и излезте при мене, и яжте всеки от лозето си, и всеки от смокината си, и пийте всеки от водата на щерната си
\par 32 докле дойда и ви заведа в земя подобна на вашата земя, зимя изобилваща с жито и вино, земя изобилваща с хляб и лозя, земя изобилваща с дървено масло и мед, за да живеете и да не умрете; и не слушайте Езекия, когато би ви убеждавал, като казва: Господ ще ни избави.
\par 33 Някой от боговете на народите избавил ли е земята си от ръката на асирийския цар?
\par 34 Где са боговете на Емат и Аршад? Где са боговете на Сефаруим, на Ена и на Ава? Избавиха ли те Самария от ръката ми?
\par 35 Кой измежду всичките богове на разните страни са избавили земята си от моята ръка, та да избави Иеова Ерусалим от ръката ми?
\par 36 А людете мълчаха, и не му отговориха ни дума, защото царят беше заповядал, казвайки: Да му не отговаряте.
\par 37 Тогава управителят на двореца Елиаким, Хелкиевият син, и секретарят Шевна, и летописецат Иоах, Асафовият син, дойдоха при Езекия с раздрани дрехи та му известиха Рпсаковите деми.

\chapter{19}

\par 1 А когато цар Езекия че думите му , раздра дрехите си, покри се с вретище и влезе в Господния дом.
\par 2 И прати управителя на двореца Елиаким, секретаря Шевна и старшите свещеници, покрити с вретища, при пророк Исаия Амосовия син.
\par 3 И рекоха му: Така казва Езекия: Ден на скраб, на изобличение и на оскърбление е тия ден, защото настана часа да се родят децата, но няма сила за раждане.
\par 4 Може би Господ твоят Бог ще чуе всичките думи на Рапсака, когото господарят му асирийският цар прати да укорява живия Бог, и ще изобличи думите, които Господ твоят Бог чу; затова, възнеси молба за остатъка що е ицелял.
\par 5 И тай слугите на цар Езекия отидоха при Исаия.
\par 6 И Исаия им рече: Така да кажете на господаря си: Така казва Господ: Не бой се от думите, които си чул, с които слегите на асирийския цар Ме похулиха.
\par 7 Ето, Аз ще туря в него такъв дух, щото, като чуе слух, ще се върне в своята земя; и ще го направя да падне от нож в своята земя.
\par 8 И така, Рапсак, като се върна, намери, че асирийският цар воюваше против Ливна; защото бе чул, че той земинал от Лахис.
\par 9 И царят , когато чу да казват за етиопския цар Тирак: Ето, излязъл да воюва против тебе, прати пак посланици до Езекия, казвайки:
\par 10 Така да годорите на Юдовия цар Езекия, да речете: Твойт Бог, на Когото уповаваш, да те не мами, като казва: Ерусалим няма да бъде предаден в ръката на асирийския цар.
\par 11 Ето, ти чу какво направили асирийските царе на всичките земи, как ги обрекли на изтребление; та ти ли ще се избавиш?
\par 12 Боговете на народите избавиха ли ония, които бащите ми изтребиха, Гозан, Харан, Ресеф и еденяните, които бяха в Таласар?
\par 13 Где е ематският цар, ерфадският цар и царят на града Сефаруим, на Ена и на Ава?
\par 14 А когато Езекия взе писмото от ръката на посланиците та го прочете, Езекия влезе в Господния дом и го разгъна пред Господа.
\par 15 И Езекия се помоли пред Господа, като каза: Господи Боже Израилев, Който седиш между херудимите, Ти и само Ти си Бог на всичките земни царства; Ти си направил небето и замята.
\par 16 Приклони, Господи ухото Си и чуй; отвори Господи очите Си и виж; и чуй думите, с които Сенахирим изпрати тогоз да похули живия Бог.
\par 17 Наистина, Господи, асирийските царе запустиха народите и земите им;
\par 18 и хвърлиха в огън боговете им, защото не бяха богове, но дело на човешки ръце, дървета и камъни; затова ги погубиха.
\par 19 Сега, прочее, Господи Боже наш, отарви ни, моля Ти се, от ръката му, за да познаят всичките земни царства, че Ти си Господ, единственият Бог.
\par 20 Тогава Исаия, Амосовият син, прати до Езекия да кажат: Така казва Господ Израилевият Бог: Чух това, за което си се помолил на Мене против асирийския цар Сенахирим.
\par 21 Ето словото, което Господ изговори за него: - Презря те, присмя ти се, девицата, соиновата дъщеря; Зад гърба ти поклати глава ерусалимската дъщеря.
\par 22 Кого си обидил и похулил ти? И против Кого си говорил с висок глас И си нодегнал нагоре очите си? Против Светия Израилев.
\par 23 Господа си обидил ти чрез посланиците си, като си рекъл: С множеството на колесниците си възлязох аз Върху височината на планините, Върху уединенията на Ливан: И ще изсека високите му кедри, Отборните му Елхи; И ще възляза в най-крайното помещение по него, В леса на неговия Кармил.
\par 24 Аз изкопах и пих чежди води; И със стапалото на нозете си Ще пресуша всичките реки на Египет.
\par 25 Не си ли чул, че Аз съм наредил това отдавна, и от древни времена съм начертал това? А сего го изпълних, Така щото ти да обръщаш укрепени градове в купове развалини.
\par 26 Затова жителите им станаха безсилни, Уплашиха се и посрамиха се; Бяха като трева на полето, като зеленина, Като трева на къщния покрив, И жито препълнено преди да стане стъбло.
\par 27 Но Аз зная жилището ти, Излизането ти и влизането ти, И убийството ти против Мене,
\par 28 Панеже буйството ти против Мене, И надменността ти стигнаха до ушите Ми, Затова ще туря куката Си в ноздрите ти, И юздата Си в устните ти, Та ще те върна през пътя, по който си дошъл.
\par 29 И това ще ти бъде знамението: Тая година ще ядете това, което е саморасло, Втората година това, което израства от същото; А третата година посейте и пожънете, Насадете лозя и яжте плода им.
\par 30 И оцелялото от Юдовия дом, Което е останало, пак ще пуска корени долу, И ще дава плод горе.
\par 31 Защото из Ерусалим ще излезе остатък, И из хълма Сион оцелялото. Ревнивостта на Господа на Силите ще извърши това.
\par 32 Затова, така казва Господ за асирийския цар: Няма да влезе в тоя град, Нито ще хвърли там стрела, Нито ще дойде пред него с щит, Нито ще издигне против него могила.
\par 33 По пътя, през който е дошъл, по него ще се върне, И в тоя град няма да възлезе, казва Господ;
\par 34 Защото ще защитя тоя град за да го избавя, Заради Себе Си и заради слугата Ми Давида.
\par 35 И в същата нощ ангел Господен слезе та порази сто осемдесет и пет хиляди души в асирийския стан; и когато станаха хора на сутрента, ето, всички ония бяха мъртви трупове.
\par 36 И тъй, асирийският цар Сенахирим си тръгна та отиде, върна се, и живееше в Ниневия.
\par 37 А като се кланяше в капището на бога си Нисрох, неговите синове Адрамелех и Сарасар го убиха с нож; и побягнаха в араратската земя. А вместо него се възцари син му Есарадон.

\chapter{20}

\par 1 В това врема Езекия се разболя до смърт; и пророк Исаия, Амосовият син, дойде при наго та му рече: Така казва Господ, нареди за дома си, понеже ще узреш и нама да живееш.
\par 2 Тогава церят обърна лицето си към стената та се помоли Господу, казвайки:
\par 3 Моля Ти се, Господи, спомни си сега как ходих пред Тебе с вярност и с цяло сърце, и върших това, което е угодно пред Тебе. И Езекия плака горко.
\par 4 А преди да беше излязъл Исаия до средната част на града, Господното слово дойде към него и рече:
\par 5 Върни се та кажи на вожда на Моите люде, Езекия, така казва Господ, Бог на баща ти Давида: Чух молитвата ти, видях сълзите ти; ето Аз ще те изцеля; след три дена ще възлезеш в Господния дом.
\par 6 Ще приложа на живота ти петнадесет години; и ще избавя тебе и тоя град от ръката на асирийския цар; и ще защитя тоя град заради Себе Си и заради слугата Ми Давида.
\par 7 Тогава Исаия каза: Вземете низаница смокини. И взеха та я туриха на цирея; и циреят оздравя.
\par 8 А Езекия беше казал на Исаия: Какво ще бъде знамението, че Господ ще ме изцели, и че след три дена ще отида в Господния дом?
\par 9 И Исаия беше рекъл: Ето какво ще ти бъде знамението от Господа, че Господ ще извърши това, което каза: избери - да напредне ли сянката дсет стъпала, или да се върне надире десет стъпала?
\par 10 И Езекия отговори: Лесно нещо е да слезе сянката десет стъпала; не, но нека се върне сянката десет стъпала надире.
\par 11 И пророк Исаия извика към Господа; и Той върна сянката десет стъпала надире, по които беше слязла в слънчевия часовник на Ахаза.
\par 12 В онова време вавилонският цар Веродах-валадан, Валадановият син, прати писмо и подарък на Езекия, защото чу, че Езекия бил се разболял.
\par 13 И Езекия ги изслуша, и показа им цялата къща със скъпоценните си вещи - среброто и златото, ароматите и скъпоценните масла, целия си оръжеен склад, и всичко каквото се намираше между съкровищата му; в къщата му и в цялото му владение не остана нищо, което Езекия не им показа.
\par 14 Тогава дойде пророк Исаия при цар Езекия та му рече: Какво казаха тия човеци? и от где дойдоха при тебе? И Езекия рече: От далечна земя идат, от Вавилон.
\par 15 Тогава каза: Що видяха в къщата ти? И Езекия отговори: Видяха всичко що има в къщата ми; няма нищо между съкровищата ми, което не им показах.
\par 16 Тогава Исаия рече на Езекия: Слушай Господното слово:
\par 17 Ето, идат дни, когато всичко що е в къщата ти, и каквото бащите ти са събрали до тоя ден, ще се пренесе у Вавилон; няма да остани нищо, казва Господ.
\par 18 И ще отведат от синовете, които излязат от тебе, които ще родиш; и те ще станат скопци в палата на вавилонския цар.
\par 19 Тогава Езекия рече на Исаия: Добро е Господното слово, което ти изрече. Прибави още: Не е ли тъй, щом в моите дни ще има мир и вярност?
\par 20 А останалите дела на Езекия, и всичкото му юначество, и как направи водоема и водопровода та доведе вода в града, не сали написани в Книгата на летописите на Юдовите царе?
\par 21 И Езекия заспа с бащите си: и вместо него се възцари син му Манасия.

\chapter{21}

\par 1 Манасия бе дванадесет години на възраст, когато се възцари, и царува петдесет и пет години в Ерусалим; а името на майка му бе Ефсива.
\par 2 Той върши зло пред Господа според мерзостите на народите, които Господ изпъди пред израилтяните.
\par 3 Устрои изново високите места , които баща му Езекия бе съборил, издигна жертвеници на Ваала и направи ашера, както Израилевият цар Ахаав беше сторил, и се кланяше на цялото небесно множество и им служеше.
\par 4 Тоже издигна жертвеници в Господния дом, за който Господ беше казал: В Ерусалим ще настаня името Си.
\par 5 Издигна жертвеници и на цялото небесно множество вътре в двата двора на Господния дом.
\par 6 И преведе сина си през огъня, още упражняваше предвещания и употребяваше гадания, и си служеше със запитвачи на зли духове и с връчове; той извърши голямо зло пред Господа та го разгневи.
\par 7 И ваяният идол на ашерата, който направи, постави в дома, за който Господ каза на Давида и на сина му Соломона: В тоя дом, и в Ерусалим, който избрах от всичките Израилеви племена, ще настаня името Си до века;
\par 8 и няма вече да направя нозете на Израиля да блуждаят вън от земята, която дадох на бащите им, само ако внимават да вършат всичко що им заповядах, напълно според закона, който слугата Ми Моисей им даде.
\par 9 Но не послушаха; а Манасия го подмами да вършат по-лошо от народите, които Господ погуби пред израилтяните.
\par 10 Тогава Господ говори чрез слугите Си пророците, казвайки:
\par 11 Понеже Юдовите царе Манасия стори тия мерзости, и извърши повече зло от всичко, което извършиха аморейците, които бяха преди него, и чрез идолите си направи и Юда да съгреши,
\par 12 затова, така казва Господ Израилевият Бог: Ето Аз нанасям такова зло на Ерусалим и на Юда, щото на всеки, който чуе за него, ще му писнат и двете уши.
\par 13 И ще простра върху Ерусалим същата мерилна връв, която прострях върху Самария, и същия отвес, който прострях върху Ахавовия дом; и ще избърша Ерусалим както избърсва някой едно блюдо, и като го избърше, обръща го наопаки.
\par 14 И ще отхвърля останалите от наследството Си, и ще ги предам в ръката на неприятелите им; и те ще бъдат разграбени и пленявани от всичките си неприятели,
\par 15 защото вършеха зло пред Мене и Ме разгневиха от деня, когато бащите им излязоха из Египет, до днес.
\par 16 При това, освен греха, с който направи Юда да съгреши като върши зло пред Господа, Манасия проля и твърде много невинна кръв, докле напълни Ерусалим от единия край до другия.
\par 17 А останалите дела на Манасия, и всичко що извърши, и грехът, който стори, не са ли написани в Книгата на летописите на Юдовите царе?
\par 18 И Манасия заспа с бащите си, и биде погребан в градината на своята къща, в градината на Оза; и вместо него се възцари син му Амон.
\par 19 Амон бе двадесет и две години на възраст, когато се възцари, и царува две години в Ерусалим; а името на майка му бе Месулемета, дъщеря на Аруса от Иотева.
\par 20 Той върши зло пред Господа, както стори баща му Манасия.
\par 21 Ходи напълно в пътя, в който ходи баща му, и служи на идолите, на които служеше баща ме, та им се поклони.
\par 22 И остави Господа Бога на бащите си, и не ходи в Господния път.
\par 23 А слугите на Амона направиха заговор против него та убиха царя в собствената му къща.
\par 24 Обаче, людете от страната избиха всичките, които бяха направили заговор против цар Амона; и людете от страната направиха сина му Иосия цар вместо него.
\par 25 А останалите дела, които извърши Амон, не са ли написани в Книгата на летописите на Юдовите царе?
\par 26 И погребаха го в гроба му в градината на Оза; и вместо него се възцари син му Иосия.

\chapter{22}

\par 1 Иосия бе осем години на възраст когато се възцари, и царува в Ерусалим тридесет и една година; а името на майка му бе Иедида, дъщаря на Адаия от Васкат.
\par 2 Той върши това, което бе право пред Господа, като ходи напълно в пътя на баща си Давида, баз да се отклони на дясно или на ляво.
\par 3 И в осемнадесетата година на цар Иосия, царят прати в Господния дом секретаря Сафан, син на Азалия, Месуламовия син, и му каза:
\par 4 Иди при първосвещеника Холкия та му речи да изброи внесените в Господния дом пари, които врътърите са събрали от людете,
\par 5 и нека ги предадат в ръката на работниците; които надзирават Господния дом;а а те нека ги дадат на работниците, които са в Господния дом, за да поправят разваленото на дома, -
\par 6 на дърводелците, на строителите и на зидарите, - за да купят дървета и дялани камъни за да поправят дома.
\par 7 Обаче, не държаха с тях никаква сметка за предаваните в ръцете им пари, защото постъпваха честно.
\par 8 Тогава първосвещеникът Хелкия каза на секретаря Сафан: Намерих книгата на закона в Господния дом. И Хелкия даде книгата на Сафана, който я прочете.
\par 9 И секретарят Сафан отиде при царя и доложи на царя, казвайки: Слугите ти иждивиха парите, които намериха в дома, като ги предадоха в ръката на работниците, които надзирават Господния дом.
\par 10 Тоже секретарят Сафан извески на царя, казвайки: Свещеник Хелкия ми даде една книха. И Сафан я прочете пред царя.
\par 11 А царят, като чу думите на книгата на закона, раздра дрехите си.
\par 12 Тогава царят заповадя на свещеника Хелкия, на Ахикама Сафановия син, на Аховора Михеевия син, на секретаря Сафан, и на царския слуга Асаия, казвайки:
\par 13 Идете, допитайте се до Господа за мене, за людете и за целия Юда, относно думите на тая книга, който се намери; защото голям е Господният гняв, който е пламнал против нас, понеже бащите ни не послушаха думите на тая книга та да постъпват напълно според както е писано за нас.
\par 14 И така, свещеник Хелкия, Ахикам, Аховор, Сафан и Асаия отидоха при пророчицата Олда, жена на одеждопазителя Селум, син на Текуя, Арасовия син. А тя живееше в Ерусалим, във втория участък; и те говориха с нея.
\par 15 И тя им рече: Така казва Господ Израилевият Бог: Кажете на човека, който ви е пратил до Мене:
\par 16 Така казва Господ: Ето, Аз ще докарам злото на това място и на жителите му, според всичко което е писано в книгата, която Юдовият цар е прочел.
\par 17 Понеже Ме оставиха и кадяха на други богове, та Ме разгневиха с всичките дала на ръцате си, затова гневът Ми ще пламне против това място, и няма да угасне.
\par 18 Но на Юдовия цар, който ви прати да се допитате до Господа, така да му кажете: Така казва Гостпод Израилевият Бог: Относно думите, които ти си чул:
\par 19 Понеже сърцето ти е омекнало, и ти си се смирил пред Господа, когато си чул това, което говорих против това място и против жителите му, че ще запустеят и ще станат за проклетия, и ти раздра дрехите си и плака пред Мене, затова и Аз те послушах, казва Господ.
\par 20 Ето, прочее, Аз ще те прибера при бащите ти, и ще се прибереш в гроба си с мир; и твоите очи няма да видят нищо от всичкото зло, което ще докарам на това място. И те доложиха на царя.

\chapter{23}

\par 1 Тогава царят прати, та събраха при него всичките Юдови и ерусалимски старейшини.
\par 2 И царят възлезе в Господния дом, и с него всичките Юдови мъже и всичките ерусалимски жители - свещениците, пророците и всичките люде от малък до голям; и прочете на всеослушание пред тях всичките думи от книгата на завета, която се намери в Господния дом.
\par 3 И царят застана до стълба, та направи завет пред Господа да следва Господа, да пази заповедите Му и заявленията Му, и повеленията Му с цалото си сърце и с цялата си душа, та да изпълняват думите, но тоя завет, който са написани в тая книга. И всичките люде потвърдиха завета.
\par 4 Тогава царят заповяда на първосвещеник Хелкия, на свещениците от втория чин и на вратарите да извадят из Господния храм всичките ващи направени за Ваала, за ашерите и за цялото небесно множество; и изгори ги вън от Ерусалим, в полетата на Кедрон, и отнесе пепелта им във Ветил.
\par 5 И премахнаха идолопоклонническите жартви, които Юдовите царе бяха, апределили да кадят по високите места в Юдовите градове и околностите на Ерусалим, както и тия, които кадяха на Ваала, на слънцето, на луната, на дванадесетте съзвездия, на цялото небесно множество.
\par 6 Занесе ашерата из Господния дом вън от Ерусалим на потока Кедрон, стри я на прах, и хвърли праха й по общенародните гробища.
\par 7 Събори и къщите на мъжеложниците, които бяха в Господния дом, гдето жените тъчаха завеси за ашера.
\par 8 И доведе всичките свещеници от Юдовите градове, и оскверни високите места, гдето тия свещеници бяха кадили, от Гава до Вирсавее; и събори високите места на портата, които бяха във входа на портата на градоначалника Исус, който вход беше отляво на градската порта.
\par 9 Но свещениците от високите места не възлизаха да служат при Господния олтар в Ерусалим, но ядяха безквасни хлябове между братята си.
\par 10 Царят оскверни и Тофета, който е в долината на Еномовите потомци, за да не може никой да преведе сина си или дъщеря си през огъна на Молоха.
\par 11 И отмахна конете, които Юдовите царе бяха посветили на слънцето във входа на Господния дом, при жилището на скопеца Натанмелех, което бе в храмовите предели, и изгори с огън колесниците на слънцето.
\par 12 Царят събори и жертвениците, които бяха върху покрива на Ахазовата горна стая, които Юдовите царе бяха направили, и жертвениците, които Монасия бе направил в двора на Господния дом, и като ги срина от там, хвърли праха им в потока Кедрон.
\par 13 Царят оскверни и високите места, които бяха пред Ерусалим, които бяха отдясно на хълма на разврата, който израилевият цар Соломон беше построил за Астарта, мерзостта на сидонците, и за Хамоса, мерзостта на моавците, и за Мелхома, мерзостта на амонците.
\par 14 И сломи изтуканите, съсече ашерите и напълни местата им с човешки кости.
\par 15 При това царят събори жертвеника, който бе във Ветил, и восокото място, което бе построил Еровоам, Наватовият син, който направи Израиля да греши, да! оня жертвеник и онова високо място, и като изгори високото място, стри го на прах, и изгори ашерата.
\par 16 И като се обърна и съгледа гробовете, които бяха там в хълма, Иосия прати та взе костите из гробовете, изгори ги на жертвеника, и го оскверни, според Господното слово прогласено от Божия човек, който прогласи тия неща.
\par 17 Какъв е тоя стълб, който виждам? А градските мъже му разправиха: Това е гробът на Божия човек, който дойде от Юда та прогласи тия дела, които ти извърши против ветилския жертвеник.
\par 18 Тогава царят каза: Оставете го; никой да не поклати костите му. И така оставиха костите му непоклатени, заедно с костите на пророка, който дойде от Самария.
\par 19 Иосия отмахна и всичките капища на високите места, които се намираха в сирийските градове, и които Израилевите царе бяха направили та бяха разгневили Господа , и постъпи с тях точно според делата, които беше извършил във Ветил.
\par 20 И като изкла на жертвениците всичките жреци от високите маста, които бяха там, и изгори върху тях човешки кости, той се върна в Ерусалим.
\par 21 След това, царят заповяда на всичките люде, казвайки: Направете пасхата на Господа вашия Бог, както е писано в тая книга на завета.
\par 22 Навярно ни във времето на съдиите, които съдиха Израиля, нито във времето на никой от Израилевите царе или Юдовите царе, не е ставала такава пасха,
\par 23 както стана тая пасха Господу в Ерусалим в осемнадесетата година на цар Иосия.
\par 24 При това, Иосия премахна запитвачите на зли духове и врачовете, домашните идоли и кумирите, и всичките мерзости, които се намериха в Юдовата земя и в Ерусалим, за да изпълни думите на закона, написани в кнегата, която свещеник Хелкия намери в Господния дом.
\par 25 Цар подобен наму не е имало преди него, който да се е обърнал към Господа с цялото си сърце, с цялата си деша и с всичката си сила, напълно според Моисеевия закон; нито след наго настана подобен нему.
\par 26 При все това, обаче, Господ не се върна от яростния Си и голам гняв; защото гневът Му пламна против юда поради всичките предизвикателства, с които Манасия беше Го разгневил.
\par 27 И рече Господ: Ще отхвърля и Юда от лицето Си, както отхвърлих Израиля; ще отхвърля и тоя гред Ерусалим, който избрах, и дома, за който рекох: Името Ми ще бъде там.
\par 28 А останалите дела на Иосия, и всичко що извърши, не са ли написани в Книгата на летописите на Юдовите царе?
\par 29 В неговите дни египетския цар Фараон-нехао възлезе против асирийския цар при реката Евфрат. Затова цар Иосия отиде против него; а той, като го видя, уби го в Магедон.
\par 30 И слугите му го закараха мъртъв в колесницата от Магедон и докараха го в Ерусалим, гдето го погребаха в собствения му гроб. А людете от земята взеха Иоахаза, Иосиевия син, и като го помазаха, направиха го цар вместо баща му.
\par 31 Иоахаз бе двадесет и три години на възраст когато се възцари, и царува три месеца в Ерусалим; а името на майка му беше Амитала, дъщеря на Еремия от Ливна.
\par 32 Той върши зло пред Господа, напълно както бяха направили бащите му.
\par 33 А Фараон-нехао го затвори в Ривла, в земята Емат, за да не царува в Ерусалим; и наложи на земята данък от сто таланта сребро и един талант злато.
\par 34 И Фараон-нехао постави Елиакима, Иосиевия син, цар вместо баща му Иосия, като промени името му на Иоаким, а Иоахаза отпрати; и той дойде в Египет и там умря.
\par 35 А Иоаким даде на Фараона среброто и златото, обаче обложи с данък земята, за да даде парите според Фараоновата заповед; с несилие събра среброто и златото от людете на земята, от всекиго, според , както беше състоянието ме , за да ги даде на Фараон-нехао.
\par 36 Иоаким бе двадесет и пет години на възраст, когато се възцари, и царува единадесет години в Ерусалим; а името на майка му бе Зевуда, дъщеря на Федаия от Рема.
\par 37 Той върши зло пред Господа, напълно както бяха направили бащите му.

\chapter{24}

\par 1 В неговите дни възлезе въвилонският цар Навуходоносор; и Иоаким му стана подчинен за три години; сетне се отметна и въстана против него.
\par 2 Но Господ прати против него халдейските чети, сирийските чети, моавските чети и четите на амонците, и прати ги против Юда за да го съкрушат, според словото, което Господ говори чрез слугите Си пророците.
\par 3 Наистина по заповед от Господа стана това на Юда, за да го отхвърли от лицето Си поради греховете на Манасия, според всичко що бе извършил,
\par 4 а още пореди невинната кръв, която бе пролял, като напълни Ерусалим с невинна кръв, така че Господ отказа да му прости.
\par 5 А останалите дела на Иоакима, и всичко що извърши, не са ли написани в Книгата на летописите на Юдовите царе?
\par 6 И Иоаким заспа с бащите си; и вместо него се възцари син му Иоахин.
\par 7 И египетският цар не излезе вече от земята си, защото вавилонският цар бе превзел всичко, което принадлежеше на египетския цар, от египетската река до реката Евфрат.
\par 8 Иоахин бе осемнадест години на възраст, когато се възцари, и царува три месеца в Ерусалим; а името на майка му бе Науста, дъщеря на Елнатана от Ерусалим.
\par 9 Той върши зло пред Господа, съвсем както бе направил баща му.
\par 10 В онова време слегите на вавилонския цар Навуходоносора възлязоха против Ерусалим та обсадиха града.
\par 11 И докато слугите му го обсаждаха, вавилонският цар Навуходоносор дойде до града;
\par 12 и Юдовият цар Иоахин излезе към вавилонския цар, той, майка му, слугите му, началниците му и скопците му; и вавилонският цар го хвана в осмата година на царуването си.
\par 13 И той отнесе от там всичките съкровища на Господния дом и съкровищата на царската къща, и съсече всичките златни вещи в Господния храм, които Израилевия цар Соломон беше направил, според както Господ беше казал.
\par 14 И пресели целия Ерусалим, всичките първенци и всичките юначни мъже, десет хиляди пленници, и всичките дърводелци и ковачи; останаха само по-соромашката част от людете на земята.
\par 15 Пресели Иохима във Вавилон; и заведе пленници от Ерусалим във Вавилон майката на царя, жените на царя, скопците на царя и първенците от земята.
\par 16 Всичките юначни мъже - седем хиляди души, и дърводелците и ковачите - хиляда души, всичките храбри войници, - тях вавилонският цар заведе пленници във Вавилон.
\par 17 И вавилонският цар постави чича си Матания цар вместо него, като промени името му на Седекия.
\par 18 Седекия бе двадесет и една гидина на възраст, когато се възцари, и царува единадесет години в Ерусалим; а името на майка му бе Амитала, дъщеря на Еремия от Ливна.
\par 19 Той върши зло пред Господа, съвсем както беше направил Иаким.
\par 20 Защото от гнева на Господа, който пламна против Ерусалим и Юда, докато ги отхвърли от лицето Си, стана, че Седекия въстана против вавилонския цар.

\chapter{25}

\par 1 И тъй, в деветата, година на Седекиевото царуване, в десетия месец, на десетия ден от месеца, вавилонският цар Навуходоносор дойде, той и цялата му войска, против Ерусалим, и като разположи стана си против него, издигнаха укрепления наоколо против него;
\par 2 и градът бе обсаден до единадесетата гидина на цар Седекия.
\par 3 А на деветия ден от четвъртия месец, когато гладът се усили в града, тъй щото нямаше хляб за людете на мястото,
\par 4 отвори се пролов в градската стена , поради което всичките военни мъже побягнаха през пътя на портата, която е между двете стени, при царската градина, (а халдейците бяха близо около града), и царят отиде по пътя за полето.
\par 5 А халдейската войска преследва царя та го стигна в ерихонските полета; и цялата му войска се разбяга от него.
\par 6 И хванаха царя та го въведоха при вавилонския цар в Ривла; и издадоха присъда против него, именно ,
\par 7 заклаха синовете на Седекия пред очите му, избодоха очите на Седекия, и като го вързаха в окови заведоха го във Вавилон.
\par 8 А в петия месец, на седмия ден от месеца, в деветнадесетата година на вавилонския цар Навуходоносор, дойде в Ерусалим началникът на телохранителите Навузардан, служител на вавилонския цар,
\par 9 та изгори Господния дом и царския дворец; дори всичките къщи в Ерусалим, сиреч , всяка голяма къща, изгори с огън.
\par 10 И цялата халдейска войска, който бе с началника на телохранителите, събори стените около Ерусалим.
\par 11 Тогава началникът на телохранителите Навузардан заведе в плен останалите от людете, които бяха оцелели в града, и бежанците, които прибягнаха при вавилонския цар, както и останалото множество.
\par 12 Обаче, началникът на телохранителите остави някои от победените в земята за лозари и замаделци.
\par 13 А медните съдове, които бяха в Господния дом, и подножията и медното море, които бяха в Господния дом, халдейците сломиха и пренесоха медта им във Вавилон;
\par 14 отнесоха и котлите, лопатите, щипците, темянниците и всичките медни прибори, с които се извършваше службата.
\par 15 Началникът на телохранителите отнесе и кадилниците и тасовете, - каквото беше златно като злато, и каквото сребърно като сребро.
\par 16 Колкото за двата стълба, едното море и подножията, които Соломон беше направил за Господния дом, медта на всички тия вещи превишаваше всяко тегло;
\par 17 височината на единия стълб беше осемнадесет лакътя; и меден капител беше върху него, височината на който капител бе три лакътя; а върху капитела наоколо имаше мрежа и нарове, всичките медни. Подобни на тия бяха размерите и на втория стълб с мрежата.
\par 18 Началникът на телохранителите взе и първосвещеник Сараия, и втория свещеник Софония, и тримата вратари;
\par 19 и от града взе един скопец, който беше надзирател на военните мъже, и петима мъже от имащите достъп при царя, които се намериха в града, и на военачалника секретаря, който събираше войски от людете на земята, и шестдесет мъже от людете на земята, които се немериха в града;
\par 20 и като ги взе началникът на телохранителите Навузардан ги заведе при вавилонския цар в Ривла.
\par 21 И вавилонският цар ги порази, и уби ги в Ривла, в земята Емат. Така Юда биде закаран в плен от земята си.
\par 22 А колкото за людете, които останаха в Юдовата земя, които вавилонският цар Навуходоносор беше остави, над тях той постави за управител Годолия, син на Ахикама, Сафановия син.
\par 23 А като чуха всичките военачалници, те и войниците им, че вавилонският цар поставил зе управител Годолия, дойдоха при Годолия в Масфа, именно, Ионан син на Кария, Сараия син на нетофатеца Танумет, и Яазания син на един маахатец, те и войниците им.
\par 24 И Годолия се закле на пях и на мъжете им, като им каза: Не бойте се от слугите на халдейците; заселете се в земята та работете на вавилонския цар, и ще ви бъде добре.
\par 25 Но в седмия месец Исмаил, син на Натания, Елисамовия син, от царския род, дойде, заедно с десетина мъже, та поразиха Годолия (тъй че умря) и юдеите и халдейците, които бяха с него в Масфа.
\par 26 Тогава всичките люде от малък до голям и военачалниците станаха та отидоха в Египет; защото се уплашиха от халдейците.
\par 27 А в тридесет и седмата година, от пленяването на Юдовия цар Иоахин в дванадесетия месец на двадесет и седмия ден от месеца, вавилонският цар Евилмеродах, в годината на възцаряването си, възвиси от тъмницата главата на Юдовия цар Иоахин;
\par 28 и говори бу любезно, постави неговия престол по-горе от престола на царете, които бяха с него във Вавилон,
\par 29 и промени тъмничните му дрехи; и Иоахин се хранеше всякога пред него, през всичките дни на живота си.
\par 30 А колкото за храната му, даваше му се от церя постоянна храна, ежедневен дял, през всичките дни на живота му.

\end{document}