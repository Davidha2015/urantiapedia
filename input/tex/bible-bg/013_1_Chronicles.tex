\begin{document}

\title{1 Chronicles}


\chapter{1}

\par 1 Адам, Сит, Енос,
\par 2 Каинан, Маалалеил, Яред,
\par 3 Енох, Матусал, Ламех,
\par 4 Ное, Сим, Хам и Яфет.
\par 5 Яфетови синове: Гомер, Магог, Мадай, Яван, Тувал, Мосох, и Тирас;
\par 6 а Гомерови синове: Асханаз, Дифат, и Тогарма;
\par 7 а Яванови синове: Елиса, Тарсис, Китим и Родамим.
\par 8 Хамови синове: Хус, Мицраим, Фут и Ханаан;
\par 9 а Хусови синове: Сева, Евила, Савта, Раама и Савтека; а Раамови синове: Шева и Дедан.
\par 10 И Хус роди Нимрода; той пръв стана силен на земята;
\par 11 а Мицраим роди лудимите, анамимите, леавимите, нафтухимите,
\par 12 патрусимите, каслухимите (от които произлязоха филистимците), и кафторимите;
\par 13 а Ханаан роди първородния си Сидон и Хета,
\par 14 и евусейците, аморейците, гергесейците,
\par 15 евейците, арукейците, асенейците,
\par 16 арвадците, цемарейците и аматейците.
\par 17 Симови синове: Елам, Асур, Арфаксад, Луд и Арам; а Арамови синове : Уз, Ул, Гетер и Мосох§;
\par 18 а Арфаксад роди Сала, а Сала роди Евера.
\par 19 И на Евера се родиха два сина: името на единия бе Фалек, защото в неговите дни се разпредели земята; а името на брата му бе Иоктан.
\par 20 А Иоктан роди Алмодада Шалефа, Хацармавета, Яраха,
\par 21 Адорама, Узала, Дикла,
\par 22 Гевала, Авимаила, Шева,
\par 23 Офира, Евила, и Иовава; всички тия бяха Иоктанови синове,
\par 24 Сим, Арфаксад, Сала,
\par 25 Евер, Фалек, Рагав,
\par 26 Серух, Нахор, Тара,
\par 27 Аврам, който е Авраам,
\par 28 А Авраамови синове: Исаак и Исмаил.
\par 29 Ето техните поколения: първородният на Исмаила, Наваиот; после Кидар, Адвеил, Мавсам,
\par 30 Масма, Дума, Маса, Адад, Тема,
\par 31 Етур, Нафис и Кедма; тия бяха Исмаиловите синове.
\par 32 А Ето синовете на Авраамовата наложница Хетура: тя роди Земрана, Иоксана, Мадана, Мадиама, Есвока и Шуаха; а Иоксанови синове: Шева и Дедан;
\par 33 а Мадиамови синове: Гефа, Ефер, Енох, Авида и Елдага; всички тия бяха потомци на Хетура.
\par 34 И Авраам роди Исаака: а Исааковите синове бяха Исав и Израил,
\par 35 Исавови синове: Елифаз, Рагуил, Еус, Еглом и Корей;
\par 36 а Елифазови синове: Теман, Омар, Сефи, Готом, Кенез, Тамна и Амалик;
\par 37 Рагуилови синове: Нахат, Зара, Сама и Миза.
\par 38 А Сиирови синове: Лотан, Совал, Севегон, Ана, Дисон, Асар и Дисан;
\par 39 а Лотови синове: Хори и Омам§; а сестра на Лота бе Тамна;
\par 40 Совалови синове: Алиан, Манахат, Гевал, Сефи и Онам; а Севегонови синове: Ана и Ана:
\par 41 Анов син, Дисон; а Дисонови синове: Амадан, Асван, Итрам и Харан;
\par 42 Асарови синове: Валаан Заван и Акан; Дисанови синове: Уз и Аран.
\par 43 А ето царете, които царуваха в едомската земя, преди да се възцари цар над израилтяните: Вела Веоровият син; и името на града му бе Денава.
\par 44 А като умря Вела, вместо него се възцари Иовав, Заровият син, от Восора.
\par 45 А като умря Иовав, вместо него се възцари Хусам, от земята на теманците.
\par 46 А като умря Хусам, вместо него се възцари Адад Вададовият син, който порази мадиамците на моавското поле; и името на града му бе Авит.
\par 47 А като умря Адад, възцари се Самла, от Масрека.
\par 48 А като умря Самла, вместо него се възцари Саул, от Роовот при Евфрат§.
\par 49 А като умря Саул, вместо него се възцари Вааланан, Аховоровият син.
\par 50 А като умря Вааланан, вместо него се възцари Адад; и името на града му бе Пау, а името на жена му Метавеил, дъщеря на Метреда, Мезаавова внука.
\par 51 А като умря Адад, едомските първенци бяха: първенец Тамна, първенец Алия, първенец Етет,
\par 52 първенец Оливема, първенец Ила, първенец Финон,
\par 53 първенец Кенез, първенец Теман, първенец Мивсар,
\par 54 първенец Магедиил и първенец Ирам. Тия бяха едомските първенци.

\chapter{2}

\par 1 Ето синовете на Израиля: Рувим, Симеон, Левий, Юда, Исахар, Завулон,
\par 2 Дан, Иосиф, Вениамин, Нефталим, Гад и Асир.
\par 3 Юдови синове: Ир, Онан и Шела: тримата му се родиха от ханаанката, дъщерята на Суя. А първородният на Юда, Ир, бе лош пред Господа, та Той го умъртви.
\par 4 И снаха му Тамар му роди Фареса и Зара, Всичките Юдови синове бяха петима.
\par 5 Фаресови синове: Есрон и Амул.
\par 6 А Зарови синове: Зимрий , Етан, Еман, Халкол и Дара( ; всичко петима.
\par 7 А Хармиев син: Ахар§, смутителят на Израиля, който извърши престъпление относно обреченото.
\par 8 А Етанов син - Азария.
\par 9 А синове, които се родиха на Есрона: Ерамеил, Арам и Халев
\par 10 И Арам роди Аминадава: а Аминадав роди Наасона, първенец на Юдовите потомци.
\par 11 А Наасон роди Салмона: Салмон роди Вооза;
\par 12 Вооз роди Овида: Овид роди Есея;
\par 13 а Есей роди първородния си Елиав, втория Авинадав, третия Сама,
\par 14 четвъртия Натанаил, петия Радай,
\par 15 шестия Осем и седмия Давид.
\par 16 А техни сестри бяха Саруия и Авигея; а Саруините синове бяха трима: Ависей, Иоав и Асаил;
\par 17 и Авигея роди Амаса; а баща на Амаса беше исмаилецът Иетер.
\par 18 И Халев, Есроновият син, роди синове от жена си Азува и от Ериота: синовете му бяха Есер, Совав и Ардон.
\par 19 И като умря Азува, Халев си взе Ефрата, която му роди Ора.
\par 20 А Ор роди Урия; а Урия роди Веселеила.
\par 21 И после Есрон влезе при дъщерята на Махира Галаадовия баща; той я взе като беше на шестдесет години на възраст, и тя му роди Сегува;
\par 22 А Сегув роди Яира, който имаше двадесет и три града в галаадската земя.
\par 23 И Гесур и Арам превзеха от тях Яировите паланки, с Кенат и селата му, шестдесет града, Всички тия бяха синове на Махира, Галаадовия баща.
\par 24 А след като умря Есрон в Халев-ефрата, тогава Есроновата жена Авия му роди Асхора бащата на Текуе.
\par 25 И синовете на първородния на Есрона, Ерамеила, бяха: първородният Арам и Вуна, Орен, Осем и Ахия.
\par 26 Ерамеил взе и друга жена, чието име бе Атара: тя бе майка на Онана.
\par 27 А синовете на първородния на Ерамеила, Арам, бяха: Маас, Ямин и Екер.
\par 28 И синовете на Анама бяха: Самай и Ядай; а синовете на Самая: Надав и Ависур.
\par 29 И името на Ависуровата жена бе Авихаила, която му роди Аавана и Молида.
\par 30 А Надавови синове бяха: Селед Апаим; и Селед умря бездетен.
\par 31 А син на Апаима беше Есия; а син на Есия, Сисан; а син на Сисана, Аалай.
\par 32 А синове на Ядая Самаевия брат бяха: Етер и Ионатан; и Етер умря бездетен;
\par 33 а синовете на Ионатана: Фалет и Зиза. Тия бяха потомците на Ерамеила.
\par 34 А Сисан нямаше синове, но дъщери. А Сисан имаше слуга египтянин на име Яраа;
\par 35 и Сисан даде дъщеря си на слугата си Яраа за жена; и тя му роди Атая;
\par 36 а Атай роди Натана; Натан роди Завада;
\par 37 Завад роди Ефлала; Ефлал роди Овида;
\par 38 Овид роди Ииуя; Ииуй роди Азария;
\par 39 Азария роди Хелиса; Хелис роди Елеаса;
\par 40 Елеас роди Сисамая; Сисамай роди Селума;
\par 41 Селум роди Екамия;а Екамия роди Елисама.
\par 42 А синовете на Хелева Ерамеиловия брат, бяха: първородният му Миса, който бе баща на Зиф; и син на Мариса бе Ави-хеврон.
\par 43 А Хевронови синове: Корей, Тапфуа, Рекем и Сема;
\par 44 А Сема роди Раама, Иоркоамовия баща; а Рекем роди Самая.
\par 45 И Самаев син беше Маон; а Маон беше Ветсуровият баща.
\par 46 А наложницата на Халева, Гефа, роди Харана, Моса и Газера; а Харан роди Газера.
\par 47 А Ядаеви синове: Регем, Иотам, Гисан, Фелет, Гефа и Сагаф.
\par 48 И наложницата на Халева, Мааха, роди Севера и Тирхана;
\par 49 роди още Сагафа, Мадмановия баща, Сева Махвановия баща и бащата на Гавая; и Халевова дъщеря бе Ахса.
\par 50 Ето синовете на Халева, син на първородния на Ефрата, Ор: Совал Кириатиаримовият баща,
\par 51 Салма Витлеемовият баща, и Ареф Вет-гадеровият баща,
\par 52 А на Совала Кириатиаримовия баща синове бяха Арое и половината от манахатците.
\par 53 А Кириатиаримовите семейства бяха: етерците, футците, суматците и мисрайците от тях произлязоха сарайците и естаолците.
\par 54 Салмови синове: Витлеем, нетофатците, Атарот-вит-Иоав, половината от манахатците, сарайците,
\par 55 семействата на писците, които живеят в Явис, тиратците, симеатците и сухатците. Това са кенейците, които произлязоха от Амата, прадед на Рихавовия дом.

\chapter{3}

\par 1 А ето синовете, които се родиха на Давида в Хеврон: първородният, Амнон, от езраелката Ахиноама; вторият Даниил, от кармилката Авигея;
\par 2 третият Авесалом, син на Мааха, дъщеря на гесурския цар Талмай; четвъртият Адония, син на Агита;
\par 3 петият Сефатия, от Авитала; шестият Итраам, от жена му Егла,
\par 4 Шестимата му се родиха в Хеврон, гдето царува седем години и шест месеца; а Ерусалим царува тридесет и три години.
\par 5 А тия му се родиха в Ерусалим: Сима, Совав, Натан и Соломон, четирима, от Витсавее, дъщеря на Амиила;
\par 6 и Евар, Елисама§, Елифалет,
\par 7 Ногах, Нефес, Яфия,
\par 8 Елисама, Елиада и Елифалет, девет души;
\par 9 всички тия бяха Давидови синове освен синовете от наложниците; и Тамар беше тяхна сестра.
\par 10 А Соломонов син бе Ровоам; негов син, Авия; негов син, Аса; негов син, Иосафат;
\par 11 негов син, Иорам; негов син, Охозия; негов син, Иоас;
\par 12 негов син, Амасия; негов син, Азария; негов син, Иотам;
\par 13 негов син, Ахаз; негов син, Езекия; негов син, Иотам;
\par 14 негов син, Амон; негов син Иосия.
\par 15 А синовете на Иосия бяха: първородният Иоанан, вторият Иоаким, третият Седекия, четвъртият Селум.
\par 16 А Иоакимови синове: Ехония син му, Седекия негов син.
\par 17 А Ехониев син бе Асир; негов син Салатиил,
\par 18 и Малхирам, Федаия, Сенасар, Екамия, Осама и Недавия.
\par 19 А Федаиеви синове: Зоровавел и Семей; а Зоровавелови синове: Масулам и Анания, и сестра им Саломита,
\par 20 и Асува, Оел, Варахия, Асавия и Юсав-есед, петима.
\par 21 А Ананиеви синове: Фелатия и Исаия, синовете на Рафаия, синовете на Арнана, синовете на Авдия, синовете на Сехания.
\par 22 А Сеханиев син, Семаия; и Семаиеви синове: Хатус, Игал, Вария, Неария и Сафат, шестима.
\par 23 А Неариеви синове: Елиоинай, Езекия и Азрикам, трима.
\par 24 А Елиоинаеви синове: Одаия, Елиасив, Фалаия, Акув, Иоанан, Далаия и Ананий, седмина.

\chapter{4}

\par 1 Юдови синове: Фареса, Есрон, Хармий, Ор и Совал.
\par 2 И Реаия Соваловият син роди Яата; и Яат роди Ахумая и Лаада. Тия са семействата на сарайците.
\par 3 И ето синовете на Итамовия баща: Езраел, Есма и Едвас; и името на сестра им бе Аселелфония,
\par 4 и Фануил Гедеровият баща и Езер Хусовият баща. Тия са потомците на Орна първородния на Ефрата, Витлеемовия баща.
\par 5 А Асхор, Текуевият баща, имаше две жени, Хала и Наара.
\par 6 Наара му роди Ахузам, Ефера, Темания и Ахастара; тия бяха синове на Наара.
\par 7 А синовете на Хала: Серет, Есуар и Етнан.
\par 8 И Кос роди Анува, Совива и семействата на Ахарила Арумовия син.
\par 9 А Явис беше най-много почитан между братята си; и майка му го нарече Явис, като думаше: Понеже го родих в скръб.
\par 10 И Явис призова Израилевия Бог, казвайки: Дано действително ме благословиш, и дано разшириш пределите ми, и ръката Ти да бъде с мене, и да ме пазиш от зло, та да нямам скръб! И Бог му даде това, което поиска.
\par 11 А Хелуви, Суевият брат, роди Махира; той бе Естоновият баща.
\par 12 А Естон роди Ветрафа, Фасея и Техина основател на град Наас; тия са мъжете на Риха.
\par 13 И Кенезови синове бяха: Готониил и Сараия; а Готониилов син, Атат.
\par 14 А Меонотай роди Офра; а Сараия роди Иоава начинателя на долината на дърводелците, защото бяха дърводелци.
\par 15 А синовете на Халева, Ефониевия син: Иру, Иле и Наам; а Илевият син беше Кенез.
\par 16 И Ялелеилови синове: Зиф, Зифа, Тирия и Асареил.
\par 17 А Езраеви синове: Етер, Меред, Ефер и Ялон: а жената на Мереда роди Мариама, Самая и Есва Естемовия баща;
\par 18 а другата му жена, Юдея, роди Яреда Гедоровия баща, Хевера Соховия баща и Екутиила Заноевия баща. И тия са синовете на Вития Фараоновата дъщеря, която взе Меред;
\par 19 И синовете на Одиевата жена, сестра Нахамова, бяха бащата на гармиеца Кеила и маахатеца Естемо.
\par 20 А Симови синове бяха: Амион, Рина, Венанан и Тилон. И Есиеви синове: Зохет и Вензохет.
\par 21 Синове на Шела, Юдовия син, бяха Ир баща на Лиха, и Лаада, баща на Мариса, и семействата от дома на тия, които работеха висон, от Асвеевия дом,
\par 22 и Иоаким, и мъжете на Хозива, и Иоас, и Сараф, които владееха в Моав и в Ясувилехем. Това според стари записки.
\par 23 Те бяха грънчари и жителите в Нетаим и в Гедира; там живееха с церя за да му работят.
\par 24 Симеонови синове бяха: Намуил, Ямин, Ярив, Зара и Саул;
\par 25 негов син, Селум; негов син, Мавсам; негов син, Масма.
\par 26 А Масмови синове: син му Амуил; негов син, Закхур; негов син, Семей.
\par 27 А Семей роди шестнадесет сина и шест дъщери; братята му обаче нямаха много синове, нито се умножи цялото им семейство като Юдовите потомци.
\par 28 Те се заселиха във Вирсавее, Молада и Асар-суал,
\par 29 във Вела, Есем и Толад(,
\par 30 във Ватуил, Хорма, и Сиклаг,
\par 31 във Вет-мархавот, Асар-сусим, Вет-вирей и в Саараим; тия бяха техни градове до царуването на Давида.
\par 32 И селищата им бяха: Итам, Аим, Римон, Тохен и Асан, пет града,
\par 33 и всичките села, които бяха около тия градове до Ваал; тия бяха местожителствата им родословията им.
\par 34 А Месовав, Ямлих, Иоса, Амасиевият син,
\par 35 Иоил, Ииуй син на Иосивия син на Сараия, син на Асиила,
\par 36 и Елиоинай, Якова, Есохаия, Асаия, Адиил, Есимиил, Ванаия
\par 37 и Зиза, син на Сифия, син на Алона, син на Едаия, син на Симрия, син на Семаия, -
\par 38 тия споменати по имена бяха първенци на семействата им; и бащините им домове нараснаха твърде много.
\par 39 И Симеоновите потомци отидоха дори до прохода на Гедор, на изток от долината, за да търсят паша за стадата си.
\par 40 И намериха тлъста и добра паша; и земята бе широка, спокойна и мирна, защото тия, които по-напред живееха там, бяха Хамови потомци.
\par 41 И тия написани по име дойдоха в дните на Юдовия цар Езекия та разрушиха шатрите им и поразиха моавците, които намериха там, и погубиха ги така щото не остана ни един от тях до днес, и на тяхното място сами се заселиха, защото там имаше паша за стадата им.
\par 42 И някои от тях, именно , петстотин мъже от Симеоновите потомци, отидоха в хълмистата земя Сиир, като имаха за свои водители, Есиевите синове: Фелатия, Неария, Рафаия и Озиила,
\par 43 та поразиха останалите от амаличаните, които бяха оцелели, и се заселиха там, гдето са и до днес.

\chapter{5}

\par 1 И синовете на първородния на Израиля, Рувим, (защото той бе първородният; но, понеже оскверни леглото на баща си, неговото първородство се даде на синовете на Израилевия син Иосифа, а родословието не се смята според първородството;
\par 2 защото Юда превъзмогна над братята си, и от него се определи да произлезе вождът; първородството, обаче, беше на Иосифа,)-
\par 3 синовете на първородния на Израиля, Рувим, бяха Енох, Фалу, Есрон и Хармий.
\par 4 Синове на Иоила: негов син, Семаия; негов син, Гог; негов син, Семей;
\par 5 негов син, Михей; негов син Реаия; негов син, Ваал;
\par 6 негов син, Веера, когото асирийския цар Теглат-Фелнасар заведе в плен; той бе първенец на рувимците.
\par 7 А началник на братята му, според семействата им, когато родословието на поколенията им се изброи, бяха Еиил, Захария
\par 8 и Вела, син на Азаса, син на Сема, син на Иоила; той се засели в Ароир и до Нево и Ваалмеон;
\par 9 и към изток се засели дори до входа на пустинята от река Евфрат, защото добитъкът им беше се умножил в галаадската земя.
\par 10 А в дните на Саула воюваха против агаряните, които паднаха от ръката им; и те се заселиха в шатрите им по цялата източна страна на Галаад.
\par 11 И Гадовите потомци се заселиха срещу тях във васанската земя до Салха,
\par 12 от които Иоил беше началникът, а Сафам вторият, и Янай, и Сафат във Васан.
\par 13 И братята им, от бащините им домове, бяха: Михаил, Месулам, Сева, Иорай, Яхан, Зия и Евер, седмина.
\par 14 Тия бяха синовете на Авихаила, син на Урия, син на Яроя, син на Галаада, син на Михаила, син на Есисая, сина на Ядо, син на Вуза.
\par 15 Ахия, син на Авдиила Гуниевият син, беше началник на бащиния им дом.
\par 16 Те се заселиха в Галаад, във Васан, в селата му, и във всичките околности на Сарон до пределите им.
\par 17 Всички тия се изброиха според родословията си в дните на Юдовия цар Иотам и в дните на Израилевия цар Еровоам.
\par 18 От Рувимовите потомци, гадците и половината от Манасиевото племе, юначните мъже, които носеха щит и нож, стреляха с лъкове и бяха обучени на бой, възлизаха на четиридесет и четири хиляди седемстотин и шестдесет души, които можеха да излизат на война.
\par 19 И те воюваха против агаряните, етуряните, нафисците и нодавците.
\par 20 И те бидоха подпомогнати против тях, тъй че агаряните и всички, които бяха с тях, се предадоха в ръцете им; защото в битката те извикаха към Бога, и Той ги послуша, понеже уповаваха на Него.
\par 21 И плениха добитъка им: петдесет хиляди от камилите им, двеста и петдесет хиляди от овците им, и две хиляди от ослите им, също и сто хиляди души човеци;
\par 22 защото мнозина паднаха убити, понеже изходът на боя беше от Бога. И заселиха се на мястото им, и там живяха до пленението.
\par 23 А потомците на половината Манасиево племе се заселиха в оная земя; те нараснаха от Васан до Ваал-ермон и Санир и планината Ермон.
\par 24 И ето началниците на бащините им домове: Ефер, Есий, Елиил, Азрил, Еремия, Одуия и Адиил, мъже силни и храбри, мъже именити, началници на бащините си домове.
\par 25 Но понеже те престъпваха против Бога на бащите си, като блудствуваха след боговете на народите на оная земя, които Бог беше погубил пред тях,
\par 26 затова, Израилевия Бог подбуди духа на асирийския цар Фул, и духа на асирийския цар Теглат-Фелнасар, та плени рувимците и гадците и половината от Манасиевото племе, и заведе ги в Ала, в Авор, в Ара и до реката Гозан, гдето са и до днес.

\chapter{6}

\par 1 Левиеви синове бяха Гирсон, Каат и Мерарий.
\par 2 А Каатови синове: Амрам, Исаар, Хеврон и Озиил;
\par 3 а Амрамови чада: Аарон, Моисей и Мариам; а Ааронови синове: Надав, Авиуд, Елеазар и Итамар.
\par 4 Елеазар роди Финееса; Финеес роди Ависуя;
\par 5 Ависуй роди Вукия; Вукия роди Озия;
\par 6 Озий роди Зараия; Зараия роди Мераиота;
\par 7 Мераиот роди Амария; Амария роди Ахитова;
\par 8 Ахитов роди Садока; Садок роди Ахимааса;
\par 9 Ахимаас роди Азария; Азария роди Ионатана;
\par 10 Ионатан роди Азария (той е оня, който Соломон построи в Ерусалим);
\par 11 а Азария роди Амария: Амария роди Ахитова;
\par 12 Ахитов роди Садока; Садок роди Селума ;
\par 13 Селум роди Хелкия; Хелкия роди Азария;
\par 14 Азария роди Сараия; а Сараия роди Иоседека.
\par 15 А Иоседек отиде в плен, когато Господ закара в плен Юда и Ерусалим чрез ръката на Навуходоносора.
\par 16 Левиеви синове: Гирсом, Каат и Мерарий.
\par 17 А ето имената на Гирсомовите синове: Ливний и Семей.
\par 18 А Каатови синове бяха: Амрам, Исаар, Хеврон и Озиил.
\par 19 Мерариеви синове бяха: Маалий и Мусий. И ето семействата на лавитите, според бащините им домове:
\par 20 на Гирсона: негов син, Ливний: негов син, Яат; негов син, Зима;
\par 21 негов син, Иоах§ ; негов син, Идо; негов син, Зара; и негов син. Етрай.
\par 22 Синовете на Каата: син му Аминада⧧; негов син, Корей; негов син, Асир;
\par 23 негов син, Елкана; негов син, Авиасаф; негов син, Асир;
\par 24 негов син, Тахат; негов син, Уриил; негов син, Озия; и негов син, Саул.
\par 25 А Елканови синове: Амасай и Ахимот.
\par 26 А относно Елкана: синовете на Елкана: негов син, Суфай; негов син, Нахат§;
\par 27 негов син, Елиав; негов син Ероам; негов син, Елкана.
\par 28 А Самуилови синове: първородният Иоил , а вторият Авия.
\par 29 Мерариеви синове: Маалий; негов син, Ливний; негов син, Семей; негов син, Оза;
\par 30 негов син, Сама; негов син, Агия; негов син, Асаия.
\par 31 И ето ония, които постави Давид над пеенето в службата в Господния дом, когато се намери трайно място за ковчега,
\par 32 които слугуваха с пеене пред скинията, шатъра за срещане, докле Соломон построи Господния дом в Ерусалим, и стоеха на службата си според чина си;
\par 33 ето ония, които стоеха с чадата си: от Каатови потомци - певецът Еман, син на Иоила, син на Самуила,
\par 34 син на Елкана, син на Ероама, син на Елиила, сина на Тоя,
\par 35 син на Суф১, син на Елкана, син на Маата, син на Амасия,
\par 36 син на Елкана, син на Иоила, син на Азария, син, на Софония,
\par 37 син на Тахата, син на Асира, син на Авиасафа, син на Корея,
\par 38 син на Исаара, син на Каата, син на Левия, син на Израиля.
\par 39 И брат му Асаф, който стоеше отдясно нему: Асаф, син на Варахия, син на Сама,
\par 40 син на Михаила, син на Ваасия, син на Малхия,
\par 41 син на Етния, син на Зара син на Адаия,
\par 42 син на Етана, син на Зима, син на Семея,
\par 43 син на Яата, син на Гирсома, син на Левия.
\par 44 И братята им Мерариеви потомци, които бяха от ляво: Етан син на Кисия, син на Авдия, син на Малуха,
\par 45 син на Асавия, син на Амасия, син на Хелкия,
\par 46 син на Амсия, син на Вания, син на Самира,
\par 47 син на Маалия, син на Мусия, син на Мерария, син на Левия;
\par 48 и братята на левитите, които бяха определени за цялата служба в скинията на Божия дом.
\par 49 А Аарон и потомците му принасяха принос върху олтара за всеизгарянето и върху кадилния олтар, като определени за цялата служба на пресветото място, и да правят умилостивение за Израиля, точно както заповяда Божият слуга Моисей.
\par 50 И ето Аароновите потомци: негов син, Елеазар; негов син, Финеес; негов син, Ависуй;
\par 51 негов син, Вукий; негов син, Озий; негов син, Зараия;;
\par 52 негов син, Мараиот; негов син, Амария; негов син, Ахитов;
\par 53 негов син, Садок; и негов син, Ахимаас.
\par 54 И ето жилищата им според заселванията им в пределите им: на Аароновите потомци от семейството на каатците (защото на тях падна първото жребие),
\par 55 на тях дадоха Хеврон в Юдовата земя, с пасбищата около него.
\par 56 Обаче, нивите, на града и селата му дадоха на Халева Ефониевия син.
\par 57 А на Аароновите потомци дадоха тия Юдови градове: Хеврон прибежищния град , Ливна с пасбищата му,
\par 58 Илен с пасбищата му, Девир с пасбищата му,
\par 59 Асан с пасбищата му, и Витсемес с пасбищата му,
\par 60 а от Вениаминовото племе, Гава с пасбищата му, Алемет с пасбищата му, и Анатот с пасбищата му; всичките им градове, според семействата им, бяха тринадесет града.
\par 61 И на Каатовите потомци, останали от семейството на племето, дадоха се с жребие десет града от половината Манасиево племе.
\par 62 А на Гирсоновите потомци, според семействата им, от Исахаровото племе, от Асировото племе, от Нефталимовото племе и от Манасиевото племе във Васан се дадоха тринадесет града.
\par 63 На Мерариевите потомци, според семействата им, дадоха се с жребие от Рувимовото племе, от Гадовото племе и от Завулоновото племе дванадесет града.
\par 64 Израилтяните дадоха на левитите тия градове и пасбищата им.
\par 65 Дадоха с жребие тия градове, споменати по име, от племето на Юдовите потомци, и от племето на Симеоновите потомци, и от племето на Симеоновите потомци, и от племето на Симеоновите потомци, и от племето на Вениаминовите потомци.
\par 66 А някои от семействата на Каатовите потомци получиха градове за притежанията си от Ефремовото племе.
\par 67 Дадоха им прибежищните градове: Сихем с пасбищата му, в Ефремовата хълмиста земя; Гезер с пасбищата му,
\par 68 Иокмеам с пасбищата му, Веторон с пасбищата му,
\par 69 Еалон с пасбищата му, Гетримон с пасбищата му,
\par 70 а, от половината на Манасиевото племе, Анир с пасбищата му, и Вилеам с пасбищата му; тях дадох на семействата на останалите Каатови потомци
\par 71 На Гирсоновите потомци дадоха , от семействата на половината Манасиево племе, Голан във Васан с пасбищата му, и Астарот с пасбищата му;
\par 72 от Исахаровото племе Кедес с пасбищата му, Даврат с пасбищата му,
\par 73 Рамот с пасбищата му и Аним с пасбищата му;
\par 74 от Асировото племе, Масал с пасбищата му, Авдон с пасбищата му.
\par 75 Хукок с пасбищата му, и Роов с пасбищата му;
\par 76 а, от Нефталимовото племе, Кедес в Галилея с пасбищата му, Амон с пасбищата ме, и Кириатаим с пасбищата му.
\par 77 А на останалите Мерариеви потомци дадоха , от Завулоновото племе, Римон с пасбищата му и Тавор с пасбищата му;
\par 78 а оттатък Иордан при Ерихон, на изток от Иордан, дадоха , от Рувимовото племе, Восор в пустинята с пасбищата му, Яса с пасбищата му,
\par 79 Кедимот с пасбищата му, и Мефаат с пасбищата му;
\par 80 а, от Гадовото племе, Рамот в Галаад с пасбищата му, Маханаим с пасбищата му,
\par 81 Есевон с пасбищата му и Язир с пасбищата му.

\chapter{7}

\par 1 А Исахарови синове бяха: Тола, Фуа, Ясув и Симрон, четирима;
\par 2 а Толови синове: Озий, Рафаия, Ериил, Ямай, Евсам и Самуил, началници на бащиния им Толов дом, силни и храбри в поколенията им; в дните на Давида, числото им бе двадесет и две хиляди и шестстотин;
\par 3 а Озиеви синове: Езраия; и Езраиеви синове; Михаил, Авдия, Иоил и Есия, петима, всички началници.
\par 4 И с тях, според поколенията им, според бащините им домове, имаше полкове войска за бой, тридесет и шест хиляди души; защото имаха много жени и синове.
\par 5 А братята им, между всичките Исахарови семейства, силни и храбри, всичките преброени според родословията си, бяха осемдесет и седем хиляди души.
\par 6 Вениаминови синове: Вела, Вехер и Едиил, трима;
\par 7 а Велови синове: Есвон, Озий, Озиил, Еримот и Ирий, петима, началници на бащини домове, силни и храбри, които се преброиха според родословията си; и бяха двадесет и две хиляди и тридесет и четири души;
\par 8 а Вехерови синове: Земира, Иоас, Илиезер, Елиоинай, Амрий, Еримот, Авия, Анатот и Аламет; всички тия бяха Вехерови синове.
\par 9 А броят на родословието им, според поколенията им, бе двадесет хиляди и двеста началници на бащините из домове, силни и храбри.
\par 10 А Едиилов син бе Валаан; а Валаанови синове: Еус, Вениамин, Аод, Ханаана, Зитан, Тарсис и Ахисаар;
\par 11 всички тия Едиилови потомци, началници на бащини домове, силни и храбри, бяха седемнадесет хиляди и двеста души, които можеха да излизат на война.
\par 12 А Суфим и Уфим бяха Ирови синове; а Ахиров§ син бе Усим.
\par 13 Нефталимови синове; Ясиил, Гуний, Есер и Селум внуци на Вала.
\par 14 Манасиеви синове: Асриил, когото му роди жена му; (а наложницата му сирианката роди Махира Галаадовия баща;
\par 15 а Махир взе за жена сестрата на Уфама и Суфама, и името на сестра им бе Мааха); а името на втория бе Салпаад; и Салпаад роди дъщеря.
\par 16 А Махировата жена Мааха роди син и нарече го Фарес; а името на брат му бе Сарес, а името на брата му бе Сарес, е синовете му: Улам и Раким;
\par 17 и Уламов син бе Ведан. Тия бяха синовете на Галаада, син на Махира, Манасиевия син.
\par 18 А сестра му Амолехет роди Исуда, Авиезера и Маала.
\par 19 А Семидови синове бяха Ахиан, Тихем, Ликхий и Аниам.
\par 20 А Ефремови синове: Сутала; негов син, Веред; негов син, Тахат; негов син, Елеада; негов син, Тахат;
\par 21 негов син, Завад; негов син, Сутала; и Езер и Елеад; а гетските мъже, ги убиха защото бяха слезли да отнемат добитъка им.
\par 22 А баща им Ефрем жали много дни; и братята му дойдоха да го утешат.
\par 23 Сетне влезе при жена си, която зачна и роди син; и нарече го Верия, по причина на нещастието, което се бе случило в дома му;
\par 24 (а дъщеря му бе Сеера, която съгради долния и горния Веторон и Узен-сеера);
\par 25 и негов син бе Рефа; негови синове, Ресеф и Тела; негов син, Тахан;
\par 26 негов син, Ладан; негов син, Амиуд; негов син, Елисама;
\par 27 негов син, Нави; и негов син, Исус.
\par 28 А притежанията им и жилищата им бяха Ветил със селата му, и към изток Нааран и към запад Гезер със селата му, и Сихем със селата му, до Газа със селата му;
\par 29 а, в пределите на Манасиевите потомци, Ветсан със селата му, Таанах със селата му, Магедон със селата му, и Дор със селата му. В тях се заселиха потомците на Иосифа Израилевия син.
\par 30 Асирови синове: Емна, Есуа, Есуий и Верия, тяхна сестра Сера;
\par 31 а Вериеви синове бяха: Хевер и Малхиил, който е баща на Вирзавит.
\par 32 А Хевер роди Яфлета, Сомира и Хотама, и тяхна сестра Суя;
\par 33 а Яфлетови синове бяха: Фасах, Вимал и Асуат; тия са Яфлетови синове;
\par 34 а Семирови§ синове бяха: Ахий, Рога, Ехува и Арам;
\par 35 а синовете на брата му Елам: Софа, Емна, Селис и Амал;
\par 36 Софови синове бяха: Суа, Арнефер, Согал, Верий, Емра,
\par 37 Восор, Од, Сама, Силса, Итран и Веера.
\par 38 А Етерови синове бяха: Ефоний, Фасфа и Ара.
\par 39 А Улови синове: Арах, Аниил и Рисия.
\par 40 Всички тия бяха Асирови потомци, началници на бащини домове, отборни, силни и храбри, главни първенци. И броят им, според родословието им, за военна служба, възлизаше на двадесет и шест хиляди мъже.

\chapter{8}

\par 1 А Вениамин роди първородния си Вела, втория Асвил, третия Аара,
\par 2 четвъртия Ноя и петия Рафа,
\par 3 А Велови синове бяха: Адар, Гира, Авиуд,
\par 4 Ависуй, Неемана, Ахоа,
\par 5 Гира, Сефуфан и Урам.
\par 6 И ето Аодовите синове, които бяха началници на бащините домове на ония, които жевееха в Гава, а бидоха заведени в Манахат;
\par 7 с Нееман, Ахия и Гира който ги заведе, и роди Аза и Ахиуда.
\par 8 А Саараим роди синове в моавската земя след като напусна жените си Усима и Ваара:
\par 9 от жена си Одеса роди Иоавава, Савия, Миса, Малхама,
\par 10 Еуса, Сахия и Мирма; тия бяха синовете му, началници на бащини домове.
\par 11 А от Усима беше родил Авитова и Елфаала.
\par 12 А Елфаалови синове бяха: Евер Мисаам, Самер, (който съгради Оно, Лод и селата му),
\par 13 и Верия и Сема§, които бяха началници на бащините домове на живеещите в Еалон, и които изпъдиха гетските жители.
\par 14 А Ахио, Сасак, Еримот,
\par 15 Зевадия, Арад, Адер,
\par 16 Михаил, Есна и Иоах бяха Вериеви синове;
\par 17 и Зевадия, Масулам, Езекий, Хевер,
\par 18 Есмерай, Езлия и Иовав бяха Елфаалови синове;
\par 19 Яким, Зехрий, Завдий,
\par 20 Елиинай, Силатай, Елиил,
\par 21 Адаия, Вераия и Симрат бяха Симееви синове;
\par 22 а Есфан, Евер, Елиил,
\par 23 Авдон, Зехрий, Анан,
\par 24 Анания, Елам, Анатотия,
\par 25 Ефадия и Фануил бяха Сасакови синове:
\par 26 а Самсерай, Сеария, Готолия,
\par 27 Яресия, Илия и Зехрий бяха Ероамови синове, -
\par 28 те бяха началници на бащини домове, началници според семействата им; те се заселиха в Ерусалим.
\par 29 Е в Гаваон се засели Гаваоновият баща Еил , името на чиято жена бе Мааха;
\par 30 а първородният му син бе Авдон, сетне Сур, Кис, Ваал, Надав,
\par 31 Гедор, Ахио, Захер
\par 32 и Макелот, който роди Сама§ и те също се заселиха с брата си в Ерусалим, срещу братята си.
\par 33 А Нир роди Киса; Кис роди Саула; а Саул роди Ионатана, Мелхисуе, Авинадава и Ес-ваала.
\par 34 А Ионатановият син бе Мерив-ваа맧, а Мерив-ваал роди Михея.
\par 35 А Михееви синове бяха Фитон, Мелех, Тарея и Ахаз.
\par 36 А Ахаз роди Иоада, Иоада роди Алемета, Азмавета и Зимрия; а Зимрия роди Моса;
\par 37 Моса роди Винея; негов син бе Рафа; негов син, Елеаса; негов син, Асиил.
\par 38 А Асиил имаше шест сина, чиито имена са тия: Азрикам, Вохеру, Исмаил, Сеария, Авдия и Анан; всички тия бяха Асиилови синове.
\par 39 А синовете на брата му Исек бяха: първородният му Улам, вторият Еус и третият Елифалет.
\par 40 А Уламовите синове бяха силни и храбри мъже, които стреляха с лък, и имаха много синове и внуци, сто и петдесет души. Всички тия бяха от Вениаминовите потомци.

\chapter{9}

\par 1 Така целият Израил се преброи по родословия; и, ето, записани са в Книгата на Израилевите и Юдовите Царе. Те бидоха пленени у Вавилон поради беззаконията си.
\par 2 А първите жители, които се настаниха в притежанията им, в градовете им, бяха израилтяните, свещениците, левитите и нетинимите.
\par 3 И в Ерусалим се заселиха от Юдейците, от вениаминците, от ефремците и от манасийците:
\par 4 Утай, син на Амиуда, син на Амрия, който бе син на Имрия Ваниевия син, от синовете на Фареса Юдовия син;
\par 5 и от Шелаевците: първородният Асаия и синовете му;
\par 6 от Заровите синове: Еуил и братята им, шестстотин и деветдесет души;
\par 7 а от вениаминците: Салу, син на Месулама, син на Одия, Асенуевия син,
\par 8 Евния Ероамовият син, Ила син на Озия, Михриевият син и Месулам, син на Сефатия, син на Рагуила, Евниевия син;
\par 9 и братята им, според семействата им, деветстотин и петдесет и шест души; всички тия мъже бяха началници на бащини домове , според бащините им домове.
\par 10 А от свещениците; Едаия, Иоиарив, Яхин
\par 11 и Азария син на Хелкия, син на Месулама, който бе син на Садока, син на Мераиота, син на Ахитова, началник на Божия дом;
\par 12 и Адаия, син на Ероама, син на Пасхора Мелхиевия син, и Маасай син на Адиила, син на Язира, който бе син на Месулама, син на Месилемита Емировия син,
\par 13 и братята им, началници на бащините им домове, хиляда и седемстотин и шестдесет души, мъже много способни за делото на службата на Божия дом.
\par 14 А от левитите: Семаия, син на Асува, син на Азрикама Асавиевия син от Мерариевите потомци;
\par 15 и Ваквакар, Ерес, Галал и Матания син на Михея, син на Зехрия Асафовия син;
\par 16 и Авдия син на Сомаия§ син на Галала Едутуновия син, и Варахия син на Аса, син на Елкана, който се зесели в селата на нетофатците.
\par 17 А вратарите бяха: Селум, Акув, Талмон, Ахиман и братята им; Селум беше началник;
\par 18 те до сега бяха вратари при царската източна врата за полковете на Левиевите потомци;
\par 19 а Селум син на Коре, който бе син на Авиасафа, Коревия син, и братята му от бащиния му дом, Киреевците, бяха над работата на службата, пазачи на входовете на скинията; и бащите им са били стражи на входа на Господния стан;
\par 20 по-напред началник над тях беше Финеес Елеазеровият син, с когото бе Господ;
\par 21 а Захария, Меселемиевият син, беше пазач при входа на шатъра за срещане;
\par 22 всички ония избрани за пазачи при входовете, бяха двеста и дванадесет души. Те, които Давид и гледачът Самуил бяха поставили на службата им, бяха преброени по родословия в селата си.
\par 23 И така те и потомците им надзираваха портите на Господния дом, на дома на скинията като вардеха по ред.
\par 24 Вратарите бяха на четирите страни, на изток, на запад, на север и на юг.
\par 25 А братята им, които бяха по селата си, трябваше да дохождат в определени времена за да бъдат с тях по седем дни.
\par 26 Защото тия левити, четиримата главни вратари, оставаха в службата си, и надзираваха стаите и съкровищата на Божия дом.
\par 27 Те и нощуваха около Божия дом, защото грижата за него бе възложена на тях, и те трябваше да го отварят всяка заран.
\par 28 И някои от тях бяха над служебните съдове, защото по брой ги внасяха и по брой ги изнасяха.
\par 29 Още някои от тях бяха определени над другите вещи, над всичките свещени съдове, и над чистото брашно, виното, дървеното масло, ливана и ароматите.
\par 30 А някои от свещеническото съсловие приготовляваха благоуханното миро.
\par 31 А Мататия, който бе от левитите, първородният на корееца Серум, надзираваше печените в тава жертви .
\par 32 И други от братята им, от потомците на каатците, бяха над присъствените хлябове за да ги приготовляват всяка събота.
\par 33 И от тях бяха певците, началниците на бащините домове на левитите, които жевееха в стаите свободни от друго служене , защото са упражняваха в работата си денем и нощем,
\par 34 Тия бяха началници на бащините домове на левитите, началници според семействата им; те живееха в Ерусалим.
\par 35 А в Гаваон се засели Гаваоновият баща Еил, името на чиято жена бе Мааха:
\par 36 а първородният му син бе Авдон, после Сур, Кис, Ваал, Нир, Надав,
\par 37 Гедор, Ахио, Захария и Макелот;
\par 38 а Макелот роди Симеама: също и те се заселиха с братята си в Ерусалим, срещу братята си.
\par 39 А Нир роди Киса; Кис роди Саула; а Саула роди Ионатана, Мелсисуя, Авинадава и Есваала.
\par 40 А Ионатановият син бе Мерив-ваал: а Мерив-ваал роди Михея.
\par 41 А Михееви синове бяха: Фитон Мелех, Терея
\par 42 и Ахаз, който роди Яра; а Яра роди Алемета, Азмавета и Зимрия; а Зимрий роди Моса:
\par 43 и Моса роди Винея; а негов син бе Рафаия; негов син, Елеаса; и негов син, Асил.
\par 44 А Асил имаше шест сина, чиито имена са тия: Азрикам, Вохеру, Исмаил, Сеария, Авдия и Анан; те бяха Асилови синове.

\chapter{10}

\par 1 А филистимците воюваха против Израиля; и Израилевите мъже побягнаха от филистимците, и паднаха убити в хълма Гелвуе.
\par 2 И в преследването филистимците настигнаха Саула и синовете му; и филистимците убиха Сауловите синове Ионатана, Авинадава и Мелхусия.
\par 3 И като се засилваше боят против Саула, стрелците го улучиха, и той биде наранен от стрелците.
\par 4 Тогава Саул каза на оръженосеца си: Изтегли меча си та ме прободи с него, да не би да дойдат тия необрязани и се поругаят с мене. Но оръженосецът му не искаше, защото много се боеше. За това Саул взе меча си та падна върху него.
\par 5 И оръженосецът му като видя, че Саул умря, падна и той на меча си та умря.
\par 6 Така умря Саул и тримата му сина; и целият му дом умря едновременно.
\par 7 Тогава всичките Израилеви мъже, които бяха в долината, като видяха, че побягнаха, и че Саул и синовете му умряха, напуснаха градовете си и побягнаха; а филистимците дойдоха та се заселиха в тях.
\par 8 А на следния ден, когато филистимците дойдоха да съблекат убитите, намериха Саула и синовете му паднали на хълма Гелвуе.
\par 9 И като го съблякоха, взеха главата му и оръжията му, та ги разпратиха наоколо из филистимската земя за да разнесат известие на идолите си и на людете.
\par 10 И положиха оръжията му в капището на боговете си, а главата му приковаха в Дагоновото капище.
\par 11 А когато чуха всичките жители на Явис-галаад все що направили филистимците на Саула,
\par 12 всичките храбри мъже станаха та дигнаха тялото на Саула и телата на синовете му, донесоха ги в Явис, и погребаха костите им под дъба на Явис; и постиха седем дена.
\par 13 Така умря Саул за престъплението, което извърши против Господа, против Господното слово, което не опази, а още, понеже бе се съвещал със запитвачка на зли духове, за да се допита до тях ,
\par 14 а до Господа не се допита, затова Той го умъртви, и обърна царството към Давида, Есеевия син.

\chapter{11}

\par 1 Тогава целият Израил се събра при Давида в Хеврон и рекоха: Ето, ние сме твоя кост и твоя плът,
\par 2 И по-напред още, и докато Саул царуваше, ти беше, който извеждаше и въвеждаше Израиля, и на тебе Господ твоят Бог каза: Ти ще пасеш людете Ми Израиля, и ти ще бъдеш вожд над людете Ми Израиля.
\par 3 И така, всичките Израилеви старейшини дойдоха при царя в Хеврон; и Давид направи завет с тях пред Господа в Хеврон; и те помазаха Давида цар над Израиля, според Господното слово чрез Самуила.
\par 4 Тогава Давид и целият Израил отидоха в Ерусалим (който бе Евус), гдето бяха жителите на земята, евусците.
\par 5 А жителите на Евус рекоха на Давида: Няма да влезеш тук. Обаче, Давид превзе крепостта Сион; това е Давидовият град.
\par 6 И рече Давид: Който пръв удари евусците, той ще бъде военачалник и вожд. И Иоав, Саруиният син, се качи пръв; и стана военачалник.
\par 7 Тогава Давид се зесели в крепостта; затова тя се нарече Давидов град.
\par 8 И той съгради града околовръст от Мило и наоколо; а Иоав поправи останалата част от града.
\par 9 И Давид преуспяваше и ставаше по-велик, защото Господ на Силите бе с него.
\par 10 А ето началниците на силните мъже, които имаше Давид, които заедно с целия Израил се подвизаваха с него за царството му, за да го направят цар, според Господното слово относно Израиля.
\par 11 А ето изчислението на силните мъже, които имаше Давид: Ясовеам, Ахмоновия син, главен военачалник; той като махаше копието си против триста души неприятели , уби ги в едно сражение.
\par 12 И след него бе ахохиецът Елеазар Додовият син, който бе един от тримата силни мъже.
\par 13 Той бе с Давида във Фас-дамим, когато филистимците се събраха за бой там, гдето имаше частица земя пълна с ечемик; и когато людете побягнаха пред филистимците,
\par 14 те застанаха всред нивата и я защитаваха и поразиха филистимците; и Господ извърши голямо избавление.
\par 15 После, трима от тридесетте военачалници слязоха до скалата при Давида в одоламската пещера; а филистимският стан бе разположен в рафаимската долина.
\par 16 И като беше Давид тогава в канарата, а филистимският гарнизон бе в това време във Витлеем,
\par 17 и Давид пожелавайки рече: Кой би ми дал да пия вода от витлеемския кладенец, който е при портата;
\par 18 то тия трима пробиха филистимския стан та наляха вода от витлеемския кладенец, който е при портата, и като взеха донесоха на Давида. Но Давид отказа да пие, а я възля Господу, като рече:
\par 19 Да ми не даде моят Бог да сторя това! Да пия ли кравта на тия мъже, които туриха живота си в опасност? защото с опасност за живота си я донесоха. Затова отказа да я пие. Това сториха тия трима силни мъже.
\par 20 И Иоавовият брат Ависей беше главен между другите , трима; защото той като махаше копието си против триста души неприятели , уби ги и си придоби име между тримата.
\par 21 Между тримата той бе по-славен от двамата, и стана им началник; но не стигна до първите трима.
\par 22 Ванаия, Иодаевият син, син на един храбър мъж от Кавсеил, който беше извършил храбри дела, - той уби двамата лъвовидни моавски мъже; тоже той слезе та уби лъва всред рова в многоснежния ден;
\par 23 при това, той уби египтянина, мъж с голям ръст, пет лакътя висок; в ръката на египтянина имаше копие като кросно на тъкач; а Ванаия слезе при него само с тояга, и като грабна копието от ръката на египтянина, уби го със собственото му копие.
\par 24 Тия неща стори Ванаия, Иодаевият син, и си придоби име между тия трима силни мъже.
\par 25 Ето, той стана по-славен от тридесетте, но не стигна до първите трима. И Давид го постави над телохранителите си.
\par 26 А силните мъже между войските бяха: Асаил, Иоавовият брат, Елханан, син на Додо от Витлеем.
\par 27 Самот арорецът, Хелис фелонецът,
\par 28 Ирас, сина на текоеца Екис, Авезер анатонецът,
\par 29 Сивехай хустанецът, Илай ахохиецът,
\par 30 Маарай нетофатецът, Хелед, син на нетофатеца Ваана,
\par 31 Итай, син на Риваия от Гавая, която принадлежеше на Вениаминовите потомци, Ванаия пиратонецът,
\par 32 Урай от долините Гаас, Авиил арватецът,
\par 33 Азмавет варумецът, Елиава саалвонецът,
\par 34 синовете на Асима гизонеца, Ионатан син на арареца Сагий,
\par 35 Ахиам син на арареца Сахар, Елифал Уровият син,
\par 36 Ефер мехиратецът, Ахия фелонецът,
\par 37 Есро кармилецът, Наарай Есвеевият син,
\par 38 Иоил, Натановият брат, Мивар Аргиевият син,
\par 39 Селек амонецът, Нахарай виротецът, оръженосецът на Иоава Саруиния син,
\par 40 Ираз иетерецът, Гарив етерецът,
\par 41 Урия хетеецът, Завад Аалаевият син,
\par 42 Адина син на рувимеца Сиза, началник на рувимците, и тридесет души с него,
\par 43 Анан, син на Мааха, Иосафат митнецът,
\par 44 Озия астеротецът, Сама и Еиил синове на Хотама ароирецът,
\par 45 Един син на Симрия, брат му Иоха тисецът,
\par 46 Елиил маавецът, Еривай и Иосавия Елнаамови синове, Етема моавецът,
\par 47 Елиил, Овид и Ясиил месоваецът.

\chapter{12}

\par 1 И ето ония, които дойдоха при Давида в Сиклаг, докато още се затваряше, за да се крие от Саула, Кисовия син; те бяха от силните мъже, които му помагаха във война.
\par 2 Те бяха въоръжени с лъкове, и можеха да си служат и с дясната и с лявата ръка за хвърляне камъни с прашка и за стреляне с лък, и бяха от Сауловите братя, от Вениамина.
\par 3 Началник беше Ахиезер, после Иоас, синове на гаваатеца Сама; Езиил и Филет, Азмаветови синове; Вераха, анатотецът Ииуй,
\par 4 гаваонецът Исмаия, силен между тридесетте, и над тридесетте; Еремия, Яазаил, Иоанан, гедиротецът Иозавад,
\par 5 Елузай, Еримот, Ваалия, Семария, еруфецът Сафатия,
\par 6 Елкана, Есия, Азареил, Иозер и Ясовеам, Кореовците;
\par 7 и Иоила и Зевадия, синове на Ероама от Гедор.
\par 8 И някои от гадците се отделиха та дойдоха при Давида в канарата в пустинята, мъже силни и храбри, обучени за бой, които знаеха да употребяват щит и копие, чиито лица бяха като лъвови лица, и които бяха бързи като сърните по хълмовете.
\par 9 Те бяха : Езер, първият; Авдия, вторият; Елиав, третият;
\par 10 Мисмана, четвъртият; Еремия петият;
\par 11 Атай, шестият; Елиил, седмият;
\par 12 Иоанан, осмият; Елзавад, деветият;
\par 13 Еремия, десетият; и Махванай, единадесетият.
\par 14 Тия от гадците бяха военачалници; който беше най-долен, между тях , беше над сто войници , а който беше най-горен, беше над хиляда.
\par 15 Тия бяха, които преминаха Иордан в първия месец, когато реката заливаше всичките си брегове и разгониха всичките жители на долините към изток и към запад.
\par 16 Също и от вениаминците и юдейците дойдоха при Давида в канарата.
\par 17 А Давид излезе да ги посрещне та им проговори, казвайки: Ако идете при мене с мир, за да ми помогнете, сърцето ми ще се привърже към вас; но ако сте дошли за да ме предадете на неприятелите ми, като няма насилие в ръцете ми, Бог на бащите ни нека види и нака изобличи това.
\par 18 Тогава дойде дух на Амасия, гладния между началниците, и рече : Твои сме, Давиде, и от към твоя страна, Есеев сине! Мир, мир на тебе! мир и на помощниците ти! защото твоят Бог ти помага. Тогава Давид ги прие, и постави ги началници на дружината си.
\par 19 и от Манасия се присъединиха някои към Давида, когато дойде с филистимците на бой против Саула; обаче не им помогнаха, защото филистимските управители, като се посъветваха, върнаха го, понеже думаха: Той ще мине към господаря си Саула с опасност за главите ни.
\par 20 А като се връщаше в Сиклаг, присъединиха се към него от Манасия: Адна, Иозавад, Едиил, Михаил, Иозавад, Елиу и Салатай, военачалници на Манасиевите хиляди.
\par 21 Те помогнаха на Давида против разбойниците, защото те всички бяха силни и храбри; и станаха военачалници.
\par 22 Защото от ден на ден прииждаха мъже при Давида, за да му помагат, докле стана голяма войска, като войска Божия.
\par 23 И ето броят на въоръжените за бой началници, които дойдоха при Давида в Хеврон, за да възвърнат нему Сауловото царство според Господното слово:
\par 24 от Юдейците шест хиляди и осемнадесет щитоносци и колесници, въоръжени за бой;
\par 25 от симеонците седем хиляди и сто души, силни и храбри за бой;
\par 26 от левийците, четири хиляди и шестстотин души;
\par 27 а Иодай беше началник на Аароновците; и с него бяха три хиляди и седемстотин души,
\par 28 и Садок силен и храбър младеж, и от бащиния му дом двадесет и двама началника;
\par 29 а от вениаминците, Сауловите братя три хиляди души; защото до тогаз по-голямата част от тях бяха поддържали Сауловия дом;
\par 30 от ефремците, двадесет хиляди и осемстотин души, силни и храбри, именити мъже от бащиния си дом;
\par 31 от половината на Манасиевото племе, осемнадесет хиляди души, които се определиха по име да дойдат и направят Давида цар;
\par 32 от исахарците, въже, които разбираха времената, та знаеха как трябваше да постъпва Израил, с двадесетте им началници, и всичките им братя действуващи под тяхна заповед;
\par 33 от Завулона, петдесет хиляди души, които можеха да излизат с войската, да го поставят в строй с всякакви военни оръжия, и да управляват сражението, с непоколебимо сърце;
\par 34 от Нефталима, хиляда началници, и с тях тридесет и седем хиляди щитоносци и копиеносци;
\par 35 от данците, двадесет и осем хиляди и шестстотин мъже, които можеха да се строят за бой;
\par 36 от Асира, четиридесет хиляди души, които можаха да излизат с войската, да я поставят в строй;
\par 37 а оттатък Иордан, от рувимците, от гадците и от половината племе на Манасия, стои двадесет хиляди души с всякакви военни оръжия за бой.
\par 38 Всички тия военни мъже нареждани в строй, дойдоха с цяло сърце в Хеврон за да направят Давида цар над целия Израил; така всички други в Израиля бяха с едно сърце да направят Давида цар.
\par 39 И там бяха с Давида три дни та ядяха и пиеха, защото братята им бяха приготвили за тях.
\par 40 Освен това, и ония, които бяха близо до тях, дори до Исахара, Завулона и Нефталима донесоха им храна с осли, с камили, с мъски и с волове, - храни от брашно, низаници смокини, сухо грозде, вино, дървено масло, говеда и изобилно овце; защото беше веселие в Израиля.

\chapter{13}

\par 1 Тогава Давид се съветва с хилядниците , стотниците и всичките началници;
\par 2 и Давид каза на цялото Израилево общество: Ако ви е угодно и ако е от Господа нашия Бог, нека пратим навсякъде до братята си, останали по цялата Израилева земя, и освен тях, и до свещениците и левитите, които са в градовете им и околностите, за да се съберат при нас;
\par 3 и нека донесем при нас ковчега на нашия Бог, защото не го потърсихме в Сауловите дни.
\par 4 И целият събор каза, че ще направят така, защото това се видя право на всичките люде.
\par 5 Тогава Давид събра целия Израил от египетския поток Сихор дори до прохода на Емат, за да донесат Божия ковчег от Кириатиарим.
\par 6 И Давид отиде с целия Израил във Ваала, то ест , в Юдов Кириатиарим, за да дигне от там Божия ковчег, който се нарича с името на Господа, Който обитава между херувимите.
\par 7 И като изнесоха Божия ковчег от Авинадавовата къща, туриха го на нова кола; и Оза и Ахио караха колата.
\par 8 И Давид и целият Израил играеха пред Господа с всичката си сила, с песни, с арфи, с псалтири, с тъпанчета, с кимвали и с тръби.
\par 9 А когато стигнаха до Хидоновото гумно, Оза простря ръката си та хвана ковчега, защото воловете го раздрусаха.
\par 10 И гневът на Господа пламна против Оза и го порази за гдето простря ръката си на ковчега; и умря там пред Бога.
\par 11 И Давид се наскърби за гдето Господ нанесе поражение на Оза; и нарече онова място Фарез-оза както се казва и до днес.
\par 12 И в оня ден Давид се уплаши от Бога, и каза: Как ще донеса при себе си Божия ковчег?
\par 13 Затова Давид не премести при себе си ковчега в Давидовия град, но го отпрати в къщата на гетеца Овид-едома.
\par 14 И Божият ковчег престоя със семейството на Овид-едома в къщата му три месеца; и Господ благослови дома на Овид-едома и всичко що имаше.

\chapter{14}

\par 1 Подир това тирският цар Хирам прати посланици при Давида и кедрови дървета, зидари и дърводелци, за да му построят къща.
\par 2 И Давид позна, че Господ го бе утвърдил цар над Израиля, защото царството му се възвиси на високо заради людете Му Израиля,
\par 3 И Давид си взе още жени в Ерусалим; и Давид роди още синове и дъщери.
\par 4 И ето имената на чадата, които му се родиха в Ерусалим: Самуа, Совав, Натан, Соломон,
\par 5 Евар, Елисуа, Елфалет,
\par 6 Ногах, Нефег, Авия,
\par 7 Елисама, Веелиада и Елифалет.
\par 8 А когато филистимците чуха, че Давид бил помазан за цар над целия Израил, всичките филистимци възлязоха да търсят Давида; а Давид, като чу за това, излезе против тях.
\par 9 И тъй филистимците дойдоха та нахлуха в долината Рафаим.
\par 10 Тогава Давид се допита до Бога, казвайки: Да възляза ли против филистимците? ще ги предадеш ли в ръката ми? И Господ му отговори: Възлез, защото ще ги предам в ръката ти.
\par 11 И тъй, те отидоха на Ваал-ферасим, и там Давид ги порази, Тогава рече Давид: Бог избухна чрез моята ръка върху неприятелите ми, както избухват водите. За туй, онова място се нарече Ваал-ферасим§.
\par 12 И филистимците , оставиха там боговете си; а Давид заповяда, та ги изгориха с огън.
\par 13 И филистимците пак нахлуха в долината.
\par 14 Затова Давид пак се допита до Бога; и Бог му каза: Не възлизай против тях, но ги заобиколи, та ги нападни отсреща черниците.
\par 15 И когато чуеш шум, като от маршируване по върховете на черниците, тогава да излезеш пред на бой, защото Бог ще излезе пред тебе да порази филистимското множество.
\par 16 И Давид стори според както му заповяда Бог; и поразяваха филистимското множество от Гаваон дори до Гезер.
\par 17 И Давидовото име се прочу по всичките земи; и Господ нанесе страх от него върху всичките народи.

\chapter{15}

\par 1 След това Давид си построи и други къщи в Давидовия град; приготви и място за Божия ковчег, и постави шатър за него.
\par 2 Тогава рече Давид: Не бива да дигат Божия ковчег други освен левитите, защото тях избра Господ да носят Божия ковчег и да Му слугуват винаги.
\par 3 И Давид събра целия Израил в Ерусалим за да принесат Господния ковчег на мястото, което беше ковчег на мястото, което беше приготвил за него.
\par 4 Давид събра Аароновите потомци и левитите;
\par 5 от Каатовите потомци, началника Уриил и братята му, сто и двадесет души;
\par 6 от Марариевите потомци началника Асаия и братята му, двеста и двадесет души;
\par 7 от Гирсоновите потомци, началника Иоил и братята му, сто и тридесет души;
\par 8 от Елисафановите потомци, началника Семаия и братята му, двеста души;
\par 9 от Хевроновите потомци, началника Семаия и братята му, двеста души;
\par 10 от Озииловите потомци, началника Аминадав и братята му, сто и двадесет души.
\par 11 И Давид повика свещениците Садок и Авиатар и левитите Уриил, Асаия, Иоил, Семаия, Емиил и Аминадав, та им рече:
\par 12 Вие сте началници на бащините домове на левитите, осветете се вие и братята ви за да пренесете ковчега на Господа Израилевия Бог на мястото , което съм приготвил за него.
\par 13 Защото, понеже първия път вие не го дигнахте , Господ нашият Бог нанесе поражение върху нас, защото не Го потърсихме според както е заповядано.
\par 14 И той, свещениците и левитите осветиха себе си за да пренесат ковчега на Господа Израилевия Бог.
\par 15 (Левийците бяха, които носеха Божия ковчег с върлините горе на рамената си, както заповяда Моисей, според Господното слово).
\par 16 И Давид каза на левитските началници да поставят братята си певците с музикалните инструменти, псалтири и арфи и кимвали, за да свирят весело със силен глас.
\par 17 И така левитите поставиха Емана, Иоилевия син, а от братята му Асафа Варахииния син; а от братята им, Мерариевите потомци, Етана Кисиевия син;
\par 18 и с тях братята им от втория чин, вратарите Захария, Вена, Яазиил, Семирамот, Ехиил, Уний, Елиав, Ванаия, Маасия, Мататия, Елифалей, Микнеия, Овид-едом и Еиил.
\par 19 Така певците Еман, Асаф и Етан се определиха да дрънкат с медни кимвали;
\par 20 а Захария, Азиил, Семирамот, Ехиил, Уний, Елиав, Маасия и Ванаия с псалтири по Аламот;
\par 21 а Мататия, Елифалей, Минеия, Авид-едом, Еиил и Азазия, с арфи по Семинит, за да ръководят пеенето.
\par 22 А Ханания, главен певец на левитите, ръководеше пеенето, понеже бе изкусен.
\par 23 А Варахия и Елкана бяха вратари за ковчега.
\par 24 А свещениците на Севания, Иосафата, Натанаил, Амасий, Захария, Ванаия и Елиезер свиреха с тръбите пред Божия ковчег; а Овид-едом и Ехия бяха вратари за ковчега.
\par 25 И тъй, Давид и Израилевите старейшини и хилядниците отидоха да дигнат с веселие ковчега на Господния завет от Овид-едомовата къща.
\par 26 И понеже Бог помагаше на левитите, които носеха ковчега на Господния завет, те пожертвуваха седем юнеца и седем овена.
\par 27 И Давид бе облечен с одежда от висон както и всичките левити, които носеха ковчега и певците, и Ханания ръководител на певците; а Давид носеше ленен ефод.
\par 28 Така целият Израил възвеждаше ковчега на Господния завет с възклицание с глас от рог, с тръби и с кимвали, като свиреха с псалтири и с арфи.
\par 29 А когато ковчегът на Господния завет влизаше в Давидовия град, Михала, Сауловата дъщеря, погледна от прозореца, и като видя, че цар Давид скачаше и играеше, презря го в сърцето си.

\chapter{16}

\par 1 Така внесоха Божия ковчег та го положиха всред шатъра, който Давид беше поставил за него; и принесоха всеизгаряния и примирителни приноси пред Бога.
\par 2 И когато Давид свърши принасянето на всеизгарянията и примирителните приноси, благослови людете в Господното име.
\par 3 И даде на всеки човек, мъж и жена, от Израиля, на всекиго по един хляб, по една мръвка месо и по една низаница сухо грозде.
\par 4 И определи известни левити, да служат пред Господния ковчег, да възпоменават, да благодарят и да хвалят Господ Израилевия Бог:
\par 5 първият, Асаф; вторият, Захария; после, Еиил, Семирамот, Ехиил, Мататия, Елиав, Ванаия, Овид-едом и Еиил с псалтири и арфи; Асаф с дрънкане на кимвали;
\par 6 а свещениците Ванаия и Яазиил с тръби, винаги пред ковчега на Божия завет.
\par 7 Тогава, в оня ден, Давид за пръв път нареди славославенето на Господа чрез Асафа и братята му, с тия думи : -
\par 8 Славословете Господа: призовавайте името Му, Възвестявайте между народите делата Му.
\par 9 Пейте Му, и псалмопейте Му; Говорете за всичките Му чудни дела.
\par 10 Хвалете Го с неговото Свето име; Нека се весели сърцето на ония, които търсят Господа.
\par 11 Търсете Господа и Неговата сила. Търсете лицето Му винаги.
\par 12 Помнете пречудните дела, които е извършил. Знаменията Му и съдбите на устата Му.
\par 13 Вие, потомство на неговия слуга Израиля, Чада Яковови, избрани Негови.
\par 14 Той е Господ Бог наш; Съдбите Му са по целия свят.
\par 15 Помнете всякога завета Му, Словото, което заповяда на хиляда поколения,
\par 16 Завета , който направи с Авраама, И клетвата Му към Исаака,
\par 17 Която и утвърди на Якова за повеление, На Израиля за вечен завет,
\par 18 Като рече: На тебе ще дам ханаанската земя За дял на наследството ви,
\par 19 Когато бяха малцина на брой, Малцина и пришелци в нея,
\par 20 И когато прехождаха от народ в народ, И от едно царства в други люде.
\par 21 Не остави никого да им напакости; Ей, заради тях царе изобличи,
\par 22 Казвайки : Да не се допирате до помазаните Ми, И да не сторите зло на пророците Ми.
\par 23 Пейте Господу, жители на целия свят; Благовествувайте от ден в ден спасението Му.
\par 24 Възвестете между народите славата Му, Между всичките племена чудните Му дела.
\par 25 Защото е велик Господ и твърде достоен за хвала, И за страхопочитание повече от всички богове.
\par 26 Защото всичките богове на племената са суетни, А Господ направи небесата,
\par 27 Слава и великолепие са пред Него, Сила и радост на мястото Му.
\par 28 Отдайте Господу вие семейства на племената. Отдайте Господу слава и сила.
\par 29 Отдайте Господу славата дължима на името Му; Донесете принос и влезте пред Него; Поклонете се Господу с великолепие свето.
\par 30 Треперете пред Него, жители на целия свят; Защото вселената е утвърдена, та не може да се поклати.
\par 31 Да се веселят небесата, и да се радва светът: И да казват между народите: Господ царува.
\par 32 Да бучи морето и всичко що има в него: Нека се радват полетата и всичко що е в тях.
\par 33 Тогава ще се радват пред Господа дърветата на дъбравата; Защото иде да съди света,
\par 34 Славословете Господа, защото е благ, Защото милостта Му е до века;
\par 35 И речете: Спаси ни, Боже на спасението ни, Събери ни, и избави ни от народите, За да славословим светото Твое име, И да тържествуваме в Твоята хвала.
\par 36 Благословен да бъда Господ Бог Израилев От века и до века. И всичките люде рекоха: Амин! и възхвалиха Господа.
\par 37 Тогава Давид остави там, пред ковчега на Господния завет, Асафа и братята му за да служат постоянно пред ковчега, според както беше нужно за всеки ден;
\par 38 остави и Овид-едома и братята му, шестдесет и осем души; тоже и Овид-едома Едутуновия син и Оса за вратари;
\par 39 и свещеника Садок и братята му свещениците, пред Господната скиния на високото място, което бе в Гаваон,
\par 40 за да принасят всеизгаряне Господу върху олтара за всеизгарянията винаги заран и вечер, точно според както е писано в закона, който Господ даде на Израиля;
\par 41 и са тях постави Емана, Едутуна, и другите по име определени, които бяха избрани да славословят Господа, защото неговата милост е до века;
\par 42 и при тях, то ест, при Емана и Едутуна, имаше тръби и кимвали за ония, които трябваше да свирят с висок глас, и инструменти за Божиите песни. А Едутуновите синове бяха вратари.
\par 43 И така, всичките люде си отидоха, всеки в къщата си; и Давид се върна да благослови дома си

\chapter{17}

\par 1 А като се настани Давид в къщата си, рече Давид на пророк Натана: Виж, аз живея в кедрова къща, а ковчегът на Господния завет стои под завеси.
\par 2 И Натан каза на Давида: Стори всичко, що е в сърцето ти, защото Бог е с тебе.
\par 3 Но през същата нощ Господното слово дойде към Натана и рече:
\par 4 Иди, кажи на слугата Ми Давида: Така говори Господ: Ти няма да Ми построиш дом, в който да обитавам,
\par 5 защото от деня, когато изведох Израиля от Египет , дори до днес, не съм обитавал в дом, но съм ходил из шатър в шатър и из скиния в скиния .
\par 6 В кое от всичките места, гдето съм ходил с целия Израил, говорих Аз някога на някой от Израилевите съдии, на които заповядах да пасат людете Ми, като казах: Защо Ми не построихте кедров дом?
\par 7 Сега, прочее, така да кажеш на слугата Ми Давид: Така каза Господ на Силите: Аз те взех от кошарата, от подир стадото, за да бъдеш вожд на людете Ми Израиля;
\par 8 и съм бил с тебе на всякъде, гдето си ходил, и изтребих, всичките ти неприятели пред тебе, и направих името ти да е както името на великите, които са на земята.
\par 9 И ще определя мястото за людете Си Израиля и ще ги насадя, то ще обитават на своето собствено място и няма да се преместят вече; и тези които вършат неправда няма вече да ги притесняват, както по-напред,
\par 10 и както от времето, когато поставих съдии над людете Си Израиля; и ще покоря всичките ти неприятели. При това ти явявам, че Господ ще ти съгради дом.
\par 11 Когато се навършат дните ти, та трябва да отидеш при бащите си, ще въздигна потомеца ти подир тебе, който ще бъде един от твоите синова, и ще утвърдя царството му.
\par 12 Той ще Ми построи дом; и Аз ще утвърдя престола му до века.
\par 13 аз ще му бъда отец, и той ще Ми бъде син; и няма да отнема милостта Си от него, както я отнех от оногова, който бе преди тебе;
\par 14 но ще го закрепя в Моя дом и в Моето царство до века; престолът му ще бъде утвърден до века.
\par 15 И Натан говори на Давида точно според тия думи и напълно според това видение.
\par 16 Тогава цар Давид влезе та седна пред Господа и рече: Кой съм аз, Господи Боже, и какъв е моят дом, та си ме довел до това положение ?
\par 17 Но даже и това бе малко пред очите Ти, Боже; а Ти си говорил още за едно дълго бъдеще за дома на слугата Си, и благосклонно си погледнал на мене, като на човек от висока степен, Господи Боже.
\par 18 Какво повече може да Ти рече Давид за честта сторена на слугата Ти? защото Ти познаваш слугата Си.
\par 19 Господи, заради слугата Си и според Своето сърце, Ти си сторил всичко за това велико дело, за да явиш всички тия велики дела.
\par 20 Господи, няма подобен на Тебе, според всичко, що сме чули с ушите си.
\par 21 И кой друг народ на света е както Твоите люде Израил, при когото Бог дойде да го откупи за Свои люде, за да придобиеш име чрез велики и ужасни дела, като изпъди Ти народите пред людете Си, които си откупил от Египет?
\par 22 Защото си направил Людете Си Израиля Свои люде до века; и Ти, Господи, им стана Бог.
\par 23 И сега, Господи, нека се утвърди до века думата, която си говорил за слугата Си и за дома му, и стори както си говорил.
\par 24 Нека се утвърди; така щото да се възвеличи името Ти до века, та да казват: Господ на Силите, Израилевият Бог, е Бог на Израиля; и нека бъде утвърден пред Тебе домът на слугата Ти Давида.
\par 25 Защото Ти, Боже мой, откри на слугата Си, че ще му съградиш дом; затова слугата Ти намери сърцето си разположено да се помоли пред Тебе.
\par 26 И сега, Господи, Ти си Бог, и си обещал тия блага на слугата Си;
\par 27 сега, прочее, благоволи да благословиш дома на слугата Ти, за да пребъдва пред Тебе до века, защото Ти, Господи, си го благословил и ще бъде благословен до века.

\chapter{18}

\par 1 След това Давид порази филистимците и ги покори, и отне Гет и селата му от ръката на филистимците.
\par 2 Порази и моавците; и моавците станаха Давидови слуги и плащаха данък.
\par 3 Давид още порази совския цар Ададезер в Емат, като последният отиваше да утвърди властта си на реката Евфрат.
\par 4 Давид му отне хиляда колесници, седем хиляди конници и двадесет хиляди пешаци и Давид пресече жилите на всичките колеснични коне, само че запази от тях за сто колесници.
\par 5 И когато дамаските сирийци дойдоха за да помогнат на совския цар Ададезер, Давид порази от сирийците двадесет и две хиляди мъже.
\par 6 Тогава Давид постави гарнизон в дамаска Сирия; и сирийците станаха Давидови слуги и плащаха данък. И Господ запазваше Давида където и да отиваше.
\par 7 И Давид взе златните щитове, които бяха върху слугите на Ададезера, та ги донесе в Ерусалим.
\par 8 И от Тиват и от Хун, Ададезериви градове, Давид взе твърде много мед, от която Соломон направи медното море, стълбовете и медните съдове.
\par 9 А ематският цар Той, когато чу, че Давид порази всичката сила на совския цар Ададезер,
\par 10 прати сина си Адорам при цар Давида за да го поздрави и да го благослови понеже се бил против Ададезера и го поразил, защото Ададезер често воюваше против Тоя. И Адорам донесе със себе си всякакви златни, сребърни и медни съдове;
\par 11 па и тях цар Давид посвети на Господа, заедно със среброто и златото, което беше отнел от всичките народи, от Едом, от Моав, от амонците, от филистимците и от Амалика.
\par 12 При това, Ависей, Саруиният син, порази осемнадесет хиляди едомци в долината на солта.
\par 13 И постави гарнизони в Едом, и всичките едомци се подчиниха на Давида. И Господ запазваше Давида където и да отиваше.
\par 14 Така Давид царува над целия Израил, и съдеше всичките си люде и им раздаваше правда.
\par 15 А Иоав, Саруиният син беше над войската; а Иосафат, Ахилудовият син, летописец;
\par 16 а Садок, Ахитововият син и Авимелех, Авиатаровият син, свещеници; а Суса, секретар;
\par 17 а Ванаия, Иодаевият син, беше над херетците и фелетците; а Давидовите синове бяха първенци около царя.

\chapter{19}

\par 1 След това, царят на амонците, Наас, умря, и вместо него се възцари син му Анун .
\par 2 Тогава Давид каза: Ще покажа благост към Ануна Наасовия син, понеже баща му показа благост към мене. И така, Давид прати посланици да го утешат за баща му. А когато Давидовите слуги дойдоха при Ануна в земята на амонците за да го утешат,
\par 3 първенците на амонците рекоха на Ануна: Мислиш ли, че от почит към баща ти, Давид ти е изпратил утешители? Не дойдоха ли слугите му при тебе за да разузнаят да разорят и да съгледат земята?
\par 4 Затова , Анун хвана Давидовите слуги та ги обръсна, отряза дрехите им до половина - до бедрата им, и ги отпрати.
\par 5 И някои отидоха та известиха на Давида за стореното на мъжете; и той изпрати човеци да ги посрещнат, (понеже мъжете се крайно срамуваха), и да им рекат от царя: Седете в Ерихон, догде пораснат брадите ви, и тогава се върнете.
\par 6 А като видяха амонците, че бяха станали омразни на Давида, Анун и амонците пратиха хиляда таланта сребро за да си наемат колесници и конници от Месопотамия, от Сириямаах и от Сова.
\par 7 И наеха си тридесет и две хиляди колесници, също и царя на Мааха с людете му, които дойдоха та разположиха стан срещу Медева. И амонците, като се събраха от градовете си, дойдоха да воюват.
\par 8 И когато чу това Давид, прати Иоава и цялото множество силни мъже.
\par 9 А амонците излязоха та се строиха за бой при градската порта; а надошлите царе бяха отделно на полето.
\par 10 А Иоав, като видя, че бяха се строили за бой против него отпред и отзад, избра между всичките отбрани Израилеви мъже, та ги опълчи против сирийците;
\par 11 а останалите люде даде в ръката на брата си Ависея; и те се опълчиха против амонците.
\par 12 И рече: ако сирийците надделеят над мене, тогава ти ще ми помогнеш; а ако амонците надделеят над тебе, тогава аз ще ти помогна.
\par 13 Дързай, и нека бъдем мъжествени за людете си и за градовете на нашия Бог; а Господ нека извърши каквото Му се вижда угодно.
\par 14 И тъй, Иоав и людете, които бяха с него, стъпиха в сражение против сирийците; и те побягнаха пред него.
\par 15 А когато амонците видяха, че сирийците побягнаха, тогава и те побягнаха пред брата му Ависей и влязоха в града. Тогава Иоав си дойде в Ерусалим.
\par 16 А сирийците, като видяха, че бидоха победени пред Израиля, пратиха посланици да изведат сирийците, които бяха оттатък реката Евфрат , със Совака, военачалника на Ададезера, на чело.
\par 17 И когато се извести на Давида, той събра целият Израил, и като премина Иордан, дойде до тях и се опълчи против тях. И тъй, като се опълчи Давид на бой против сирийците, те се биха с него.
\par 18 Но сирийците побягнаха пред Израиля; и Давид изби от сирийците мъжете на седем хиляди колесници и четиридесет хиляди пешаци; уби и военачалника Совак.
\par 19 И слугите на Ададезера, като видяха, че бяха победени от Израиля, сключиха мир с Давида и му се подчиниха. И сирийците отказаха да помагат вече на амонците.

\chapter{20}

\par 1 След една година, във времето когато царете отиват на война, Иоав изведе всичката сила на войската, и, като опустоши земята на амонците, дойде та обсади Рава; Давид остана в Ерусалим. И Иоав порази Рава и я съсипа.
\par 2 И Давид взе от главата на царя им короната му, и намери, че тежеше един златен талант, и по нея имаше скъпоценни камъни; и положиха я на главата на Давида; и той изнесе из града твърде много користи.
\par 3 Изведе и людете, които бяха в него, та ги пресече с железни дикани и с брадви. И така постъпи Давид с всичките градове на амонците. Тогава Давид се върна с всичките люде в Ерусалим.
\par 4 След това настана война с филистимците в Гезер, когато хусатецът Сивехай уби Сифая, от синовете на исполина; и филистимците бяха победени.
\par 5 И пак настана война с филистимците, когато Елханан, Яировият син, уби Лаамия, брата на гетеца Голиат, на чието копие дръжката бе като кросно на тъкач.
\par 6 Настана пак война в Гет, гдето имаше един високоснажен мъж с по шест пръста на ръцете си и по шест пръста на нозете си, всичко двадесет и четири; също и кой бе се родил на исполина;
\par 7 а когато се закани на Израиля Ионатан, син на Давидовия брат Самай, го уби.
\par 8 Тия бяха се родили на исполина в Гет; и паднаха чрез ръката на слугите му.

\chapter{21}

\par 1 Но Сатана се подигна против Израиля, и подбуди Давида да преброи Израиля.
\par 2 И тъй, Давид каза на Иоава и на първенците на людете: Идете, пребройте Израиля от Вирсавее до Дан, и доложете ми, за да науча броя им.
\par 3 А Иоав рече: Господ дано притури на людете Си стократно повече отколкото са! но господарю мой царю, те всички не са ли слуги на господаря ми? защо господарят ми желае това? защо да стане причина на вина у Израиля?
\par 4 Обаче, царската дума надделя над Иоава. И тъй, Иоав тръгна, и като обходи целия Израил, върна се в Ерусалим.
\par 5 И Иоав доложи на Давида броя на преброените люде. И целият Израил беше един милион и сто хиляди, мъже, които можеха да теглят нож; а Юда, четиристотин седемдесет хиляди мъже, които можеха да теглят нож.
\par 6 А между тях той не брои левитите и вениаминците; защото царската дума беше гнусна на Иеова.
\par 7 Но това нещо се видя зло пред Бога; затова Той, порази Израиля.
\par 8 Тогава Давид каза на Бога: Съгреших тежко като извърших това нещо; но сега, моля Ти се, отмахни беззаконието на слугата Си, защото направих голяма глупост.
\par 9 И Господ говори на Давидовия гледач Гад, казвайки:
\par 10 Иди, говори на Давида, казвайки: Така казва Господ: Три неща ти предлагам; избери си едно от тях за да го извърша над тебе.
\par 11 Дойде, прочее, Гад при Давида та му рече: Така казва Господ:
\par 12 Избери си или тригодишен глад, или три месеца да гинеш пред неприятелите си като те стига ножът на неприятелите ти, или три дни да поразява Господният меч, сиреч, мор по земята, като погубва ангелът Господен по всичките Израилеви предели. Сега, прочее, виж какъв отговор да възвърна на Оногова, Който ме е пратил.
\par 13 И Давид рече на Гада: Намирам се много на тясно; обаче, да падна в ръката на Господ, защото Неговите милости са твърде много; но в ръката на човека да не изпадна..
\par 14 И тъй, Господ прати мор върху Израиля; и паднаха седемдесет хиляди мъже от Израиля.
\par 15 И Бог прати ангела в Ерусалим за да го погуби; но, като щеше да го погубва, Господ погледна и се разкая за злото, и рече на ангела, който погубваше юдеите : Стига вече; оттегли ръката си. И ангелът Господен стоеше при гумното на евусеца Орна.
\par 16 И Давид, като подигна очи, и видя че ангелът Господен стоеше между земята и небето с гол меч в ръката си, прострян над Ерусалим, тогава Давид и старейшините облечени във вретища, паднаха на лицето си.
\par 17 И Давид рече на Бога: Не аз ли заповядах да преброят людете? и наистина аз съм, който съгреших и извърших голямо зло; но тия овце що са сторили? Над мене, моля Ти се, Господи Боже, нека бъде ръката Ти, и над моя бащин дом, а не над Твоите люде, за да ги погуби.
\par 18 Тогава ангелът Господен заповяда на Гада да рече на Давида да отиде и издигне олтар Господу на гумното на Евусеца Орна.
\par 19 И така Давид възлезе според думата, която Гад говори в Господното име.
\par 20 И като се обърна Орна и видя ангела, четиримата му синове, които бяха с него, се скриха. (А Орна вършееше жито).
\par 21 И Като идеше Давид към Орна, Орна погледна и го видя; и излезе из гумното та се поклони на Давида с лице до земята.
\par 22 Тогава Давид каза на Орна: Дай ми мястото, гдето е това гумно, за да издигна на него олтар Господу; дай ми го за пълна цена, за да престане язвата между людете.
\par 23 А Орна рече на Давида: Вземи си го, и нека направи господарят ми царят, каквото му се вижда за добре; ето, давам воловете за всеизгаряне, диканите за дърва и житото за принос; всичкото давам.
\par 24 А цар Давид рече на Орна: Не, но непременно ще го купя с пълна цена; защото не ща да взема за жертва Господу това, което е твое, нито ще принеса всеизгаряне без да платя.
\par 25 И така, Давид даде на Орна за мястото шестстотин сикли злато по теглилка.
\par 26 И там Давид издигна олтар Господу, и принесе всеизгаряния и примирителни приноси, и призова Господа; и Той му отговори от небето с огън паднал върху олтара за всеизгарянето.
\par 27 Тогава Господ заповяда на ангела, та повърна меча си в ножницата му.
\par 28 В онова време, когато видя Давид, че Господ го послуша на гумното на евусеца Орна, започна да принася жертви там.
\par 29 Защото в онова време Господната скиния, която Моисей направи в пустинята, и олтарът за всеизгарянето, бяха на високото място в Гаваон.
\par 30 Но Давид не можа да иде пред скинията да се допита до Бога, понеже се боеше от меча на ангела Господен.

\chapter{22}

\par 1 Тогава рече Давид: Това е домът на Господа Бога, и това е олтарът за всеизгаряне за Израиля.
\par 2 И Давид заповяда да съберат чужденците, които бяха в Израилевата земя; и накара каменосечци да насекат и издялат камъни за построяването на Божия дом.
\par 3 Давид приготви и много желязо за гвоздеи, за вратите на портите и за скобите, и мед толкоз много, щото превишаваше теглото,
\par 4 и кедрови дървета без брой; защото сидонците и тиряните донасяха на Давида много кедрови дървета.
\par 5 Защото Давид си казваше: Понеже син ми Соломон е млад и нежен, и домът, който ще се построи Господу, трябва да бъде извънредно великолепен, прочут и славен в цялата вселена, затова ще направя приготовление за него. И тъй, Давид направи голямо приготовление преди смъртта си.
\par 6 Тогава повика сина си Соломона, та му заръча да построи дом за Господа Израилевия Бог.
\par 7 И Давид каза на Соломона: Сине мой, аз пожелах в сърцето си да построя дом за името на Господа моя Бог;
\par 8 но Господното слово дойде до мене, което каза: Много кръв си пролял, и големи войни си водил; ти няма да построиш дом за името Ми, защото си пролял много кръв на земята пред Мене.
\par 9 Ето, ще ти се роди син, който ще бъде спокоен човек, защото ще го успокоя от всичките неприятели около него; понеже Соломон ще му бъде името, и в неговите дни ще дам мир и тишина на Израиля;
\par 10 той ще построи дом за името Ми; и той ще Ми бъде син, и Аз ще му бъда Отец, и ще утвърдя престола на царството му над Израиля до века.
\par 11 Сега, сине мой, Господ да бъде с тебе; а ти да благоуспееш, та да построиш дома на Господа твоя Бог, както Той е говорил за тебе.
\par 12 Само да ти даде Господ мъдрост и разум, и да те упътва за управлението на Израиля, за да пазиш закона на Господа твоя Бог.
\par 13 Ако внимаваш да изпълняваш повеленията и съдбите, които Господ заповяда на Моисея за Израиля, тогава ще благоуспееш. Бъди силен и храбър, не бой се, и не се страхувай.
\par 14 И, ето, аз, в смирението си, приготвил съм за Господния дом сто хиляди таланта злато, и един милион таланта сребро, а мед и желязо без тегло, защото е много; приготвил съм и дървета и камъни, на които може ти да притуриш.
\par 15 А имам и доста много работници - каменоделци, зидари, дърводелци и работници изкусни във всякакъв вид работа.
\par 16 Златото, среброто, медта и желязото са без брой. Стани та действувай; и Господ да бъде с тебе!
\par 17 Давид заповяда и на всичките Израилеви първенци да помагат на сина му Соломона, като каза :
\par 18 Не е ли с вас Господ вашият Бог? и не даде ли ви спокойствие от всякъде? защото предаде в моята ръка жителите на тая земя; и земята се подчини пред Господа и пред Неговите люде.
\par 19 Утвърдете, прочее, сърцето си и душата си да търсите Господа вашия Бог; и станете та постройте светилището на Господа Бога, за да донесете ковчега на Господния завет и светите Божии вещи в дома, който ще се построи за Господното име.

\chapter{23}

\par 1 Така Давид, като остаря и се насити с дни, направи сина си Соломона цар над Израиля.
\par 2 И събра всичките Израилеви първенци, и свещениците и левитите.
\par 3 А левийците бяха преброени от тридесет годишна възраст нагоре; и като се преброиха мъжете един по един, броят им бе тридесет и осем хиляди души,
\par 4 от които двадесет и четири хиляди да надзирават делото на Господния дом, а шест хиляди да бъдат управители и съдии,
\par 5 четири хиляди вратари, и четири хиляди да хвалят Господа с инструменти, които направих, рече Давид , за да хвалят с тях Господа .
\par 6 И Давид ги раздели на отреди според Левиевите синове, Гирсона, Каата и Мерария
\par 7 От Гирсоновците бяха Ладан и Семей.
\par 8 Ладанови синове бяха: Ехиил - главният, Зетам и Иоил, трима.
\par 9 Семееви синове: Селомит, Азиил и Аран, трима. Тия бяха началници на Ладановите бащини домове .
\par 10 А Семееви синове: Яат, Зиза, Еус и Верия. Тия четирима бяха Семееви синове.
\par 11 А Яат бе главният и Зиза вторият: а Еус и Верия нямаха много синове, затова бяха преброени заедно като един бащин дом.
\par 12 Каатови синове бяха: Амрам, Исаар, Хеврон и Озиил, четирима.
\par 13 Амрамови синове: Аарон и Моисей. А Аарон бе отделен, той и синовете му, за да освещават пресветите неща за всякога, да кадят пред Господа, да Му служат, и за благославят в Неговото име за винаги.
\par 14 А синовете на Божия човек Моисей, се числяха между Левиевото племе.
\par 15 Синовете на Моисея бяха: Гирсам и Елиезер.
\par 16 От Гирсамовите синове, Суваил бе главният.
\par 17 И Елиезеровият син беше Равия - главният. Елиезер нямаше други синове; а Равиевите синове бяха премного.
\par 18 От Исааровите синове, Селомит бе главният.
\par 19 Хевроновите синове: Ерия, първият; Амария, вторият; Язиил, третият; и Екамеам, четвъртият.
\par 20 Азииловите синове: Михей, първият; и Есия, вторият.
\par 21 Мерариеви синове бяха: Маалий и Мусий. Маалиеви синове: Елеазар и Кис.
\par 22 А Елеазар умря без да има синове, а само дъщери; затова братовчедите им, Кисовите синове, ги взеха за жени .
\par 23 Мусиеви синове бяха: Маалий, Едер и Еримот трима.
\par 24 Тия бяха левийците, според бащините си домове, началници на бащините си домове според преброяването си, броени един по един по име, които извъшваха работите по службата на Господния дом, от двадесет годишна възраст и нагоре.
\par 25 Защото Давид рече: Господ Израилевият Бог успокои людете Си, за да живеят в Ерусалим за винаги.
\par 26 При това, не ще има нужда вече левитите да носят скинията и всичките нейни вещи за службата й;
\par 27 защото, според последните думи на Давида, левийците бяха преброени от двадесет годишна възраст нагоре.
\par 28 Понеже тяхната работа бе да помагат на Аароновите потомци в службата на Господния дом, в дворовете, в стаите, в очистването на всичките свети неща, - във всичката работа на службата на Господния дом,
\par 29 също в грижата за присъствените хлябове, за чистото брашно на хлебните приноси, било за безквасните кори, или за нещата, които се пекат в тава, или за препечените, и от всякаква мярка и големина.
\par 30 и всяка заран да стоят за да славят и хвалят Господа, тоже и вечер,
\par 31 и да принасят Господу всичките всеизгаряния в съботите, на новолунията, и на презниците, по число, според определеното за тях, всякога пред Господа,
\par 32 и да пазят заръчаното за шатъра за срещане, и заръчаното за светилището, и заръчаното от братята си Аароновите потомци по службата на Господния дом.

\chapter{24}

\par 1 Ето отредите на Аароновите потомци, Ааронови синове бяха: Надав, Авиуд, Елеазар, и Итамар.
\par 2 А Надав и Авиуд умряха преди баща си и нямаха чада; за това, Елеазар и Итамар свещенодействуваха.
\par 3 И Давид заедно със Садока от Елеазаровите потомци, ги разпредели според наредената за тях работа.
\par 4 А от Елеазаровите потомци се намериха повече началници отколкото от Итамаровите потомци, според бащините им домове, осем началника .
\par 5 Разпределиха и едните и другите с жребий; защото имаше управители на светилището и управители на Божия дом , както от Елеазаровите потомци, така и от Итамаровите потомци.
\par 6 И секретарят Семаия, Натаналовият син, който бе от левитите, ги записа в присъствието на царя, на първенците, на свещеника Садока, на Ахимелеха, Авиатаровия син и на началниците на бащините домове на свещениците и на левитите, като се вземеше един бащин дом от Елеазара и един от Итамара.
\par 7 А първият жребий излезе за Иоиарива, вторият за Едаия,
\par 8 третият за Харима, четвъртият за Сеорима,
\par 9 петият за Мелхия, шестият за Менамина,
\par 10 седмият за Акоса, осмият за Авия,
\par 11 деветият за Иисуя, десетият за Сехания,
\par 12 единадесетият за Елиасива, дванадесетият за Якима,
\par 13 тринадесетият за Уфа, четиринадесетият за Есевава,
\par 14 петнадесетият за Петаия, шестнадесетият за Емира,
\par 15 седемнадесетият за Изира, осемнадесетият за Афисиса,
\par 16 деветнадесетият за Петаия, двадесетият за Езекиила,
\par 17 двадесет и първият за Яхаина, двадесет и вторият за Гамула,
\par 18 двадесет и третият за Делаия, и двадесет и четвъртият за Маазия.
\par 19 Тоя беше редът на служението им, според който да влизат в Господния дом, по наредбата дадена им чрез баща им Аарона, според както му бе заповядал Господ Израилевият Бог.
\par 20 А останалите левийци бяха: от Амрамовите потомци, Суваил; от Суваиловите синове, Ядаия;
\par 21 от Равия, от Равиевите синове, Есия първият;
\par 22 от исаарците, Селомот; от Селомотовите синове, Яат;
\par 23 а Хевронови синове Ерия, първият ; Амария, вторият; Яазиил, третият; Екамеам, четвъртият;
\par 24 от Озииловите синове, Михей; от Михеевите синове, Самир;
\par 25 Михеев брат бе Есия; от Есиевите синове, Захария;
\par 26 Мерариеви синове бяха: Мааяий и Мусий; Яазиевият син, Вено;
\par 27 Мерариеви потомци чрез Яазия: Вено, Соам, Закхур и Иврий.
\par 28 от Маалия бе Елеазар, който нямаше синове;
\par 29 от Киса, Кисовият син Ерамеил;
\par 30 И Мусиеви синове: Маалий, Едер и Еримот. Тия бяха левиевите потомци според бащините им домове.
\par 31 Както братята им, Аароновите потомци, така и те хвърлиха жребия в присъствието на цар Давида, на Садока, на Ахимелеха и на началниците на бащините домове , така и за по-малките си братя.

\chapter{25}

\par 1 При това, Давид и военачалниците определиха за службата някои от Асафовите, Емановите и Едутуновите синове да пророкуват с арфи, с псалтири и с кимвали; а броят на ония, които се занимаваха с тая служба беше:
\par 2 от Асафовите синове: Закхур, Иосиф, Натания и Асарила, Асафови синове, който пророкуваше по наредба на царя;
\par 3 от Едутуна, Едутуновите синове: Годолия, Езрий, Исаия, Семей , Асавия и Мататия, шестима, под наставлението на баща си Едутуна, който пророкуваше с арфа и славословеше Господа;
\par 4 от Емана, Емановите синове: Вукия, Матания, Озиил, Суваил, Еримот, Анания, Ананий, Елиата, Гидалтий, Ромамтиезер, Иосвекаса, Малотий, Отир и Маазиот;
\par 5 всички тия бяха синове на царския гледач в Божиите слова Еман, определен да свири високо с рог. И Бог даде на Емана четиринадесет сина и три дъщери.
\par 6 Всички тия, под наставлението на баща си, бяха певци в Господния дом с кимвали, псалтири и арфи за службата на Божия дом; а Асаф, Едутун и Еман бяха под нареждането на царя.
\par 7 И броят им, заедно с братята им, обучени в Господните пеения, всичките изкусни, беше двеста и осемдесет и осем души.
\par 8 А те хвърлиха жребия за реда на служението си, малък и гялам, учител и ученик, наравно.
\par 9 И първият жребий излезе за Асафа, тоест за сина му Иосифа; вторият за Годолия, - той, братята му и синовете му бяха дванадесет души;
\par 10 третият за Закхура, - той, синовете му и братята му дванадесет души;
\par 11 четвъртият, за Езрий, - той, синовете му и братята му дванадесет души;
\par 12 петият, за Натания, - той, синовете му и братята му дванадесет души;
\par 13 шестият, за Вукия, - той, синовете му и братята му дванадесет души;
\par 14 седмият, за Асарила, - той, синовете му и братята му дванадесет души;
\par 15 осмият, за Исаия, - той, синовете му и братята му дванадесет души;
\par 16 деветият, за Матания, - той, синовете му и братята му дванадесет души;
\par 17 десетият, за Семея, - той, синовете му и братята му дванадесет души;
\par 18 единадесетият, за Азареила, - той, синовете му и братята му дванадесет души;
\par 19 дванадесетият, за Асавия, - той, синовете му и братята му дванадесет души;
\par 20 тринадесетият, за Суваила, - той, синовете му и братята му дванадесет души;
\par 21 четиринадесетият, за Мататия, - той, синовете му и братята му дванадесет души;
\par 22 петнадесетият, за Еримота, - той, синовете му и братята му дванадесет души;
\par 23 шестнадесетият, за Анания, - той, синовете му и братята му дванадесет души;
\par 24 седемнадасетият, за Иосвекаса, - той, синовете му и братята му дванадесет души;
\par 25 осемнадесетият, за Анания, - той, синовете му и братята му дванадесет души;
\par 26 деветнадесетият, за Малотия, - той, синовете му и братята му дванадесет души;
\par 27 двадесетият, за Елиата, - той, синовете му и братята му дванадесет души;
\par 28 двадесет и първият, за Отира, - той, синовете му и братята му дванадесет души;
\par 29 двадесет и вторият, за Гидалтия, - той, синовете му и братята му дванадесет души;
\par 30 двадесет и третият, за Маазиота, - той, синовете му и братята му дванадесет души;
\par 31 и двадесет и четвъртият, за Ромамтиезера, - той, синовете му и братята му дванадесет души.

\chapter{26}

\par 1 За отредите на вратарите: от Кореевите бе Меселемия, Кореевият потомец, от Асафовите синове.
\par 2 А Меселемиеви синове бяха: Захария, първородният; Едиил, вторият; Зевадия, третият; Ятниил, четвъртият;
\par 3 Елам, петият; Иоанан, шестият; и Елиоинай, седмият.
\par 4 И Овид-едомови синове: Семаия, първородният; Иозавад, вторият; Иоах третият; Сахар четвъртият; Натанаил, петият;
\par 5 Амиил, шестият; Исахар, седмият; и Феулатай, осмият; защото Бог го благослови.
\par 6 А на сина му Семаия се родиха синове, които властвуваха над бащиния си дом, защото бяха силни и храбри мъже:
\par 7 Семаевите синове бяха: Офний Рафаил, Овид, Елзавад, чиито братя, Елиу и Семахия, бяха шестдесет и двама души.
\par 8 Всички тия от Овид-едомовите потомци, те, синовете им и братята им от Овид-едома, силни и способни за службата, бяха шестдесет и двама души.
\par 9 А Меселемия имаше осемнадесет храбри синове и братя.
\par 10 А синовете на Оса от Мерариевите потомци бяха: Симрий, първият, (защото, ако и да не беше първороден, пак баща му го направи пръв);
\par 11 Хелкия, вторият; Тевалия, третият: и Захария четвъртият, Всичките синове и братя на Оса бяха тринадесет души.
\par 12 От тия се образуваха отредите на вратарите, ще каже на главните мъже, имащи заръчвания наравно с братята си да слугуват в Господния дом.
\par 13 И за всяка порта хвърлиха жребий, малък и голям еднакво според бещините си домове.
\par 14 И жребият за Селемия падна към изток. Тогава хвърлиха жребий за сина му Захария, мъдър съветник; и на него жребият излезе към север;
\par 15 за Овид-едома към юг, а за синовете му влагалището;
\par 16 за Суфима и Оса към запад, с портата Салехет, при пътя, който върви нагоре, стража слещу стража.
\par 17 Към изток бяха шестима левити, към север четирима на ден, към юг четирима на ден, и към влагалището по двама;
\par 18 в Първото към запад, четирима към пътя, който върви на горе , и двама в Парвар.
\par 19 Тия бяха отредите на вратарите между кореевите потомци и Мерариевите потомци.
\par 20 А от левитите Ахия бе над съкровищата на Божия дом и над съкровищата на посветените неща.
\par 21 Ладанови синове: синовете на гирсонеца Ладан, началници на бащини домове на гирсонеца Ладан, бяха Ехиил;
\par 22 а Ехииловите синове, Зетан и брат му Иоил, бяха над съкровищата на Господния дом.
\par 23 От Амрамовците, Исааровците, Хевроновците и Озииловците,
\par 24 Суваил, син на Гирсама, Моисеевия син, бе надзирател над съкровищата.
\par 25 А братята му от Елиезера, чийто син бе Равия, а негов син Исаия, негов син Иорам, негов син Зехрий, а негов син Селомит; -
\par 26 тоя Селомит и братята му бяха над всичките съкровища на посветените неща, които посветиха цар Давид, първенците на бащините домове , хилядниците, стотниците и военачалниците.
\par 27 От користите, вземани в боевете, бяха посветили за построяването на Господния дом.
\par 28 Също и всичко, каквото бяха посветили гледачът Самуил, Саул Кисовият син, Авенир Нировият син, и Иоав Саруиният син, - всичко посветено от кого да било, - бе под ръката на Селомита и на братята му.
\par 29 От Исааровците: Ханания и синовете му бяха надзиратели и съдии над Израиля за външните дела.
\par 30 От Хевроновците: Асавия и братята му, хиляда и седемстотин храбри мъже, бяха надзиратели над Израиля отсам Иордан, на запад, за всичките Господни дела и за слугуването на царя.
\par 31 Между Хевроновците Ерия беше началник, между Хевроновците, според поколенията им, според бащините им домове . В четиридесетата година на Давидовото царуване те бяха прегледани, и между тях се намериха силни и храбри мъже в Язир Галаад.
\par 32 И братята му, храбри мъже, бяха две хиляди и седемстотин началници на бащини домове , които цар Давид постави над Рувимците, гадците и половината Манасиево племе за всяко божие дело и за царските работи.

\chapter{27}

\par 1 А според преброяването им израилтяните, то ест , началниците на бащините домове , хилядниците, стотниците, и чиновниците им, които слугуваха на царя във всичко, що се касаеше до отредите, които влизаха в служба и излизаха от месец на месец през всичките месеци на годината, бяха двадесет и четири хиляди във всеки отред.
\par 2 Над първия отред, за първия месец, бе Ясовеам, Завдииловият син; и в неговия отред имаше двадесет и четири хиляди души.
\par 3 Той беше от Фаресовите потомци първенец на всички военачалници, за първия месец.
\par 4 И над отреда за втория месец беше ахохецът Додай, с Макелот за първенец на неговия отред; също и в неговия отряд имаше двадесет и четири хиляди души.
\par 5 Третият военачалник за третия месец беше Ванаия, син на свещеник Иодай, главен; и в неговия отред имаше двадесет и четири хиляди души.
\par 6 Това е оня Ванаия, който бе силният между тридесетте, и над тридесетте; и в неговия отред беше син му Амизавад.
\par 7 Четвъртият военачалник , за четвъртия месец, бе Иоавовият брат Асаил, и с него син му Зевадия; и в неговия отряд имаше двадесет и четири хиляди души.
\par 8 Петият военачалник за петия месец бе езраецът Самот; и в неговия отряд имаше двадесет и четири хиляди души.
\par 9 Шестият военачалник за шестия месец, бе Ирас, син на текоеца Екис; и в неговия отряд имаше двадесет и четири хиляди души.
\par 10 Седмият военачалник , за седмия месец бе фелонецът Хелис, от ефремците; и в неговия отряд имаше двадесет и четири хиляди души.
\par 11 Осмият военачалник , за осмия месец, бе хусатецът Сивехай, от Заровците; и в неговия отред имаше двадесет и четири хиляди души.
\par 12 Деветият военачалник , за деветия месец, бе анатотецът Авиезер, от вениаминците; и в неговия отред имаше двадесет и четири хиляди души.
\par 13 Десетият военачалник , за десетия месец, бе нетофатеца Маарай, от Заровците; и в неговия отред имаше двадесет и четири хиляди души.
\par 14 Единадесетият военачалник , за единадесетия месец, бе пиратонецът, Ванаия от ефремците; и в неговия отряд имаше двадесет и четири хиляди души.
\par 15 Дванадесетият военачалник , за дванадесетия месец, бе нетофатецът Хелдай, от Готониила; и неговият отряд имаше двадесет и четири хиляди души.
\par 16 После, над Израилевите племена; първенец на Рувимците бе Елиезер, Зехриевият син; на Симеонците, Сафатия, Мааховият син;
\par 17 на левийците, Асавия, Кемуиловият син; на Аароновците, Садок;
\par 18 на Юда, Елиу, от Давидовите братя; на Исахара, Амрий, Михаиловият син;
\par 19 на Завулона, Исмая Авдиевият син; на Нефталима, Еримот Азрииловият син;
\par 20 на ефремците, Осия Азазиевият син; на половината Манасиево племе, Иоил Федаиевият син;
\par 21 на другата половина на Манасиевото племе в Галгал, Идо Захариевият син; на Вениамина, Ясиил Авенировият син;
\par 22 на Дана, Азареил Ероамовият син. Тия бяха първенци на Израилевите племена.
\par 23 Но между тях Давид не преброи ония, които бяха на двадесет годишна възраст и по-долу; защото Господ беше казал, че ще умножава Израиля както небесните звезди.
\par 24 Иоав, Саруиният син, почна да брои, но не свърши; и поради това преброяване падна гняв на Израиля, и броят не се вписа между работниците записани в летописите на цар Давида.
\par 25 А над царските съкровища бе Азмавет, Адииловият син; над съкровищата на полетата, на градовете, на селата и на крепостите, Ионатан Озиевият син;
\par 26 над полските работници, които обработваха земята, Езрий Хелувовият син;
\par 27 над лозята, раматецът Семей; над плода от лозята влаган във винените влагалища, Завдий Сифамиевият син;
\par 28 над маслините и черниците, които бяха на полето, гедерецът Вааланан; над влагалищата за дървено масло, Иоас;
\par 29 над говедата, които пасяха в Сарон, саронецът Ситрай; над говедата, които бяха в долините, Сафат Адлаиевият син;
\par 30 над камилите, исмаилецът Овил; над ослите, меронотецът Ядаия;
\par 31 а над овцете, агарецът Язиз. Всички тия бяха надзиратели на цар Давидовия имот.
\par 32 А Давидовият стрика Ионатан, човек разумен и книжник, беше съветник; Ехиил Ахмониевият син, беше с царските синове;
\par 33 Ахитофел, царски съветник; архиецът Хусай, царски приятел;
\par 34 а съветниците след Ахитофела, Иодай, Ванаиевият син и Авиатар; а началник на царската войска, Иоав.

\chapter{28}

\par 1 Тогава Давид събра в Ерусалим всичките Израилеви първенци, първенците на племената, началниците на отредите, които слугуваха на царя на ред , и хилядниците, стотниците и надзирателите на царя и на синовете му, както и скопците му, юнаците му и всичките силни и храбри мъже.
\par 2 И цар Давид се изправи на нозете си та рече: Чуйте ме, братя мои и люде мои. Аз имах в сърцето си желание да построя успокоителен дом за ковчега на Господния завет и за подножието на нашия Бог; и бях направил приготовление за построяването.
\par 3 Но Бог ми каза: Ти няма да построиш дом на името Ми, защото си войнствен мъж и си пролял много кръв.
\par 4 А Господ Израилевият Бог избра мене, измежду целия ми бащин дом, за да бъда цар над Израиля до века; защото избра Юда за вожд, а от Юдовия дом избра моя бащин дом, и между синовете на баща ми благоволи да направи мене цар над целия Израил.
\par 5 А измежду всичките ми синове, (защото Господ ми даде много синове), Той избра сина ми Соломона да седи на престола на Господното царство над Израиля;
\par 6 и рече ми: Син ти Соломон, той ще построи дома Ми и дворовете Ми; защото него избрах да бъде Мой син, и Аз ще бъда негов Отец.
\par 7 И ако постоянствува да изпълнява заповедите Ми и съдбите Ми, както прави днес, Аз ще утвърдя царството му до века.
\par 8 Сега, прочее, пред целия Израил - Господното общество, и при слушането на нашия Бог, заръчвам ви : Пазете и изпитвайте всички заповеди на Господа вашия Бог, за да продължавате да владеете тая добра земя и да я оставите след себе си за наследството на потомците си за винаги.
\par 9 И ти, сине мой Соломоне, познавай Бога на баща си, и служи Му с цяло сърце и с драговолна душа; защото Господ изпитва всичките сърца; и знае всичките помисли на ума; ако Го търсиш, Той ще бъде намерен от тебе; но ако Го оставиш, ще те отхвърли за винаги.
\par 10 Внимавай сега; защото Господ избра тебе да построиш дом за светилище; бъди твърд и действувай.
\par 11 Тогава Давид даде на сина си Соломона чертежа на трема на храма , на обиталищата му, на съкровищниците му, на горните му стаи, на вътрешните му стаи и на мястото на умилостивилището,
\par 12 и чертежа на всичко, което прие чрез Духа за дворовете на Господния дом, за всичките околни стаи, за съкровищниците на Божия дом и съкровищниците на посветените неща,
\par 13 и разписанието за отредите на свещениците и левитите и за всяка работа по службата в Господния дом, и чертежа на всичките вещи за службата в Господния дом.
\par 14 Даде му наставления и за златото, по колко на тегло да се употреби за златните вещи , за всичките вещи, за всякакъв вид служба; и за среброто, по колко на тегло за всичките сребърни вещи, за всичките вещи, за всякакъв вид служба;
\par 15 също наставления за теглото на златните светилници и на златните им светила, с теглото на всеки светилник и на светилата му; и теглото на сребърните светилници, с теглото на светилника и на светилата му, според употребата на всеки светилник.
\par 16 Даде му наставления и за златото, по колко на тегло да се употреби за трапезите на присъствените хлябове, колко за всяка трапеза; и за среброто за сребърните трапези;
\par 17 и за чистото злато, за вилиците, за легените и за поливалниците, и за златните паници с теглото на всяка паница; също и за сребърните паници с теглото на всяка паница;
\par 18 и за пречистеното злато, колко на тегло да се употреби за кадилния олтар; и колко злато за начертаната колесница, то ест , херувимите, ковчега на Господния завет.
\par 19 Всичко това каза Давид , Господ ми даде да разбера, като написа с ръката Си всичките подробности на чертежа.
\par 20 Давид още каза на Сина си Соломона: Бъди твърд и насърчен и действувай; не бой се, нито се страхувай; защото Господ Бог, моят Бог, е с теб; няма да те остави, нито да те напусне докле свършиш всичката работа за службата на Господния дом.
\par 21 И, ето, определени са отредите на свещениците и левитите за всяка служба на Божия дом; и с тебе ще бъдат, за всякаква работа, всичките усърдни люде, изкусни за всяка работа; също първенците и всичките люде ще бъдат всецяло с тебе да изпълняват твоите заповеди.

\chapter{29}

\par 1 Тогава цар Давид каза на цялото събрание: Син ми Соломон, когото сам Бог избра, е още млад и нежен, а работата е голяма; защото тоя палат не е за човека, но за Господа Бога.
\par 2 Аз, прочее, приготвих, с всичката си сила, за дома на моя Бог златото за златните вещи, среброто за сребърните вещи, желязото за железните вещи, и дърветата за дървените вещи, тоже ониксови камъни, камъни за влагане, камъни лъскави и разноцветни, и всякакви скъпоценни камъни, и голямо количество мрамори.
\par 3 При това, понеже утвърдиха сърцето към дома на моя Бог, освен всичко що съм приготвил за светия дом, частното си съкровище от злато и сребро,
\par 4 а именно, три хиляди таланта злато от офирското злато и седем хиляди таланта пречистено сребро, с което да облекат стените на обиталищата,
\par 5 златото за златните вещи, и среброто за сребърните вещи, и за всякаква работа, която ще се изработи с ръцете на художниците . Кой, прочее, ще направи днес доброволен принос Господу?
\par 6 Тогава първенците на бащините домове , първенците на Израилевите племена, на хилядниците, стотниците и надзирателите на царските работи пожертвуваха усърдно;
\par 7 и дадоха за работата на Божия дом пет хиляди таланта и десет хиляди драхми злато, десет хиляди таланта сребро, осемнадесет хиляди таланта мед и сто хиляди таланта желязо.
\par 8 И у които се намериха скъпоценни камъни, и тях дадоха за съкровищницата на Господния дом чрез ръката на гирсонеца Ехиил.
\par 9 Тогава людете се зарадваха, защото жертвуваха усърдно, понеже с цялото сърце принасяха доброволно Господу; също и цар Давид се зарадва твърде много.
\par 10 Затова Давид благослови Господа пред цялото събрание; и Давид каза: Благословен си, Господи, от века и до века, Бог на нашия баща Израил.
\par 11 Твое, Господи, е величието, и силата, и великолепието, и сиянието, и славата; защото сичко е Твое що е на небето и на земята; Твое е царството, Господи, и Ти си на високо, като глава над всичко.
\par 12 Богатствата и славата са от Тебе, и Ти владееш над всичко; в Твоята ръка е магъществото и силата и в Твоята ръка е да възвеличаваш и да укрепяваш всички.
\par 13 Сега, прочее, Боже наш, ние Ти благодарим, и хвалим Твоето славно име.
\par 14 Но кой съм аз, и кои са людете ми, та да можем да принесем доброволно принос като тоя? защото всичко е от Тебе, и от Твоето славно име.
\par 15 Защото сме чужденци пред Тебе, и пришелци, както всичките ни бащи; дните ни на земята са като сянка, и трайност няма.
\par 16 Господи Боже наш, целият тоя куп материал , който сме приготвили за да Ти построим дом за Твоето свето име, иде от Твоята ръка, и всичко е Твое.
\par 17 И зная, Боже мой, че Ти изпитваш сърцето и че благоволението Ти е в правдата. С правотата на сърцето си аз принесох доброволно всичко това; и сега с радост видях, че и Твоите люде, които присъствуват тук, принасят на Тебе доброволно.
\par 18 Господи Боже на бащите ни Авраама, Исаака и Израиля, опази това за винаги в сърдечните размишления на людете Си, и оправи сърцето им към Себе Си;
\par 19 и дай на сина ми Соломона съвършено сърце за да пази заповедите Ти, изявленията Ти, и повеленията Ти, и да върши всичко това, и да построи палата, за който направих приготовление.
\par 20 Тогава Давид каза на цялото събрание: Благословете сега Господа вашия Бог. И тъй, цялото събрание благослови Господа Бога на бащите си, и като се наведоха поклониха се Господу и на царя.
\par 21 И на следния ден пожертвуваха жертви Господу, като принесоха във всеизгаряния Господу хиляда юнеца, хиляда овни, и хиляда агнета, с възлиянията им, също и голямо количество жертви за целия Израил;
\par 22 и на същия ден ядоха и пиха пред Господа с голяма радост. И втори път прогласиха Давидовия син Соломона за цар, и помазаха го Господу за управител, а Соломона за свещеник.
\par 23 Тогава Соломон седна на Господния престол като цар, вместо баща си Давида, и благоуспяваше; и целият Израил му се покори.
\par 24 Също и всичките първенци, и силните мъже, още и всичките цар Давидови синове се покориха на цар Соломона.
\par 25 И Господ възвеличи Соломона твърде много пред целия Израил, и даде му такова царско величие, каквото никой цар не бе имал преди него в Израиля.
\par 26 Така Давид, Есеевият син, царува над целия Израил.
\par 27 И времето, през което царува над Израиля, бе четиридесет години7 седем години царува в Хеврон, и тридесет и три в Ерусалим.
\par 28 И умря в честите старост, сит от дни, богатство и слава; и вместо него се възцари син му Соломон.
\par 29 А делата на цар Давида, първите и последните, ето, написани са в Книгата на гледача Самуил, и в Книгата на пророка Натан, и в Книгата на гледача Гад,
\par 30 с всичките събития на царуването му, и с могъществото му, и с времената, които преминаха над него, над Израиля, и над всичките царства на околните страни.

\end{document}