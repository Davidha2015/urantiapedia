\begin{document}

\title{2 Chronicles}


\chapter{1}

\par 1 Соломон, Давидовият син, се закрепи на царството си; и Господ неговият Бог бе с него и твърде много го възвеличи.
\par 2 Тогава Соломон говори на целия Израил, на хилядниците и стотниците, на съдиите и на всичките първенци от целия Израил, началниците на бащините домове ;
\par 3 и така, Соломон и цялото общество с него, отидоха на високото място, което е в Гаваон; защото там беше Божият шатър за срещане, който Господният слуга Моисей беше направил в пустинята.
\par 4 А Давид беше пренесъл Божия ковчег от Кариатиарим на мястото , което му бе приготвил; защото Давид бе поставил за него шатър в Ерусалим.
\par 5 При това, медният олтар, който Везелеил, син на Урия, Оровият син, беше направил, бе там пред Господната скиния; и Соломон и обществото се обърнаха ревностно към него.
\par 6 Там отиде Соломон, при медния олтар пред Господа, който олтар бе пред шатъра за срещане, та принесе на него хиляда всеизгаряния.
\par 7 В същата нощ Бог се яви на Соломона и му рече: Искай какво да ти дам.
\par 8 А Соломон каза на Бога: Ти си показал голяма милост към баща ми Давида, като си ме поставил цар вместо него.
\par 9 Сега, Господи Боже, нека се утвърди обещанието, което Ти даде на баща ми Давида; защото Ти ме направи цар над люде многочислени като праха на земята.
\par 10 Дай ми, прочее, мъдрост и разум за да се обхождам прилично пред тия люде; защото кой може да съди тоя Твой голям народ?
\par 11 И Бог каза на Соломона: Понеже си имал това в сърцето си, и не поиска богатство, имоти и слава, нито живота на ония, които те мразят, нито поиска дълъг живот, но поиска за себе си мъдрост и разум за да съдиш людете Ми, над които те направих цар,
\par 12 дават ти се мъдрост и разум; при това ще ти дам богатство, имоти и слава, каквито не са имали царете, които са били преди тебе, нито ще имат някои след тебе.
\par 13 Тогава Соломон се върна в Ерусалим от високото място, което е в Гаваон от пред шатъра за срещане; и царуваше над Израиля.
\par 14 И Соломон събра колесници и конници, и имаше хиляда и четиристотин колесници, които настани по градовете за колесниците и при царя в Ерусалим.
\par 15 И царят направи среброто и златото да изобилват в Ерусалим като камъни, а кедрите направи по множество като полските черници.
\par 16 И за Соломона докарваха коне из Египет; кервани от царски търговци ги купуваха по стада с определена цена.
\par 17 А извеждаха и докарваха из Египет всяка колесница за шестстотин сребърни сикли , и всеки кон за сто и петдесет; така също за всичките хетейски царе и за сирийските царе конете им се доставяха чрез тия търговци .

\chapter{2}

\par 1 Тогава Соломон, като реши да построи дом на Господното име, и царска къща,
\par 2 Соломон изброи седемдесет, хиляди мъже за бременосци, и осемдесет хиляди за каменоделци в планините, и три хиляди и шестстотин за надзиратели долу.
\par 3 Тогава Соломон прати до тирския цар Хирам да кажат: Както си сторил на баща ми Давида, и си му пратил кедри, за да си построи къща, в която да живее, стори така и на мене .
\par 4 Ето, аз строя дом за името на Господа моя Бог, да го посветя Нему, за да се принася пред Него благоуханен темян и постоянните присъствени хлябове , и утринните и вечерните всеизгаряния, в съботите, на новолунията, и в празниците на Господа нашия Бог; това е всегдашна наредба в Израил.
\par 5 А домът, който строя, е голям, защото нашият Бог е велик повече от всичките богове.
\par 6 Но кой може да Му построи дом, когато небето и небето на небесата на са достатъчни да Го поберат? кой прочее съм аз, та да Му построя дом, освен само за да кадя пред Него?
\par 7 Затова прати ми сега човек изкусен да работи злато и сребро, мед и желязо, мораво, червено и синьо, който знае да прави ваяни работи, и нека работи с изкусните човеци, които са с мене в Юда и в Ерусалим, който баща ми Давид е промислил.
\par 8 Изпрати ми и кедрови дървета, елхови дървета и алмугови дървета от Ливан; защото зная, че твоите слуги умеят да секат дървета в Ливан; и, ето, моите слуги ще бъдат с твоите слуги,
\par 9 за да ми приготвят изобилно дървета; защото домът, който строя, ще бъде голям и чудесен.
\par 10 И ето, ще дам на твоите слуги, дървосечците, двадесет хиляди кора чукано жито, двадесет хиляди кора ечемик, двадесет хиляди вати вино и двадесет хиляди вати дървено масло.
\par 11 Тогава тирският цар Хирам отговори с писмо, което прати на Соломона: Понеже Господ люби людете Си, за това те е поставил цар над тях.
\par 12 Рече още Хирам: Благословен да бъде Господ Израилевият Бог, създател на небето и на земята, Който даде на цар Давида мъдър син, надарен с остроумие и разум, който да построи дом Господу и царска къща.
\par 13 Сега, прочее, пращам ти човек изкусен и разумен, майстора си Хирам,
\par 14 син на една жена от Давидовите дъщери и на баща тирянин, изкусен и да изработва злато и сребро, мед и желязо, камъни и дърва, и мораво, синьо, висон и червено, тоже да прави всякакви ваяни работи и да изработва всякакво измишление каквото му се внуши, за да работи той заедно с твоите изкусни човеци и с изкусните човеци промислени от господаря ми баща ти Давида.
\par 15 Сега, прочее, нека господарят ми изпрати на слугите си житото и ечемика, дървено масло и виното, който е обещал;
\par 16 а ние ще насечим дърветата от Ливан, колкото ти са потребни, и ще ти ги докараме на салове по море до Иопия, а ти ще ги превозиш в Ерусалим.
\par 17 И Соломон преброи всичките чужденци, които бяха в Израилевата земя, след преброяването, с което баща му Давид ги бе преброил; и намериха се сто и петдесет и три хиляди и шестстотин души.
\par 18 И от тях назначи седемдесет хиляди за бременосци, и осемдесет хиляди за каменоделци в планините, и три хиляди и шестстотин за надзиратели върху работенето на людете.

\chapter{3}

\par 1 Тогава Соломон почна да строи Господният дом в Ерусалим на хълма Мория, гдето се яви Господ на баща му Давида, на мястото, което Давид беше приготвил на гумното на евусеца Орна,
\par 2 Той почна да строи на втория ден от втория месец на четвъртата година от възцаряването си.
\par 3 А основата, която Соломон положи, за да построи Божия дом, имаше тия мерки : дължината в лакти, според старата мярка, бе шестдесет лакътя, и широчината двадесет лакътя.
\par 4 А тремът, който бе пред лицето на дома , имаше дължина според широчината на дома, двадесет лакътя, а височината сто и двадесет; и обкова го извътре с чисто злато.
\par 5 И облече великия дом с елхови дървета, които обкова с чисто злато, и извая по него палми и верижки.
\par 6 И украси дома със скъпоценни камъни за красота; а златото бе от Фаруим.
\par 7 Обкова още със злато дома, гредите, вратите, стълбовете и стените му; и извая херувими по стените.
\par 8 И направи пресветото място на на дома, с дължина, според широчината на дома, двадесет лакътя, и с широчина двадесет лакътя; и обкова го с шестстотин таланта чисто злато.
\par 9 А теглото на гвоздеите беше петдесет сикли злато. Обкова и горните стаи със злато.
\par 10 И в пресветото място на дома направи два херувима ваяна изработка, и обкова ги със злато.
\par 11 А крилата на херувимите имаха, заедно , дължина двадесет лакътя; едното крило, на единия херувим , имаше пет лакътя и досягаше стените на дома; и другото крило имаше пет лакътя и допираше крилото на другия херувим.
\par 12 Така и едното крило на другия херувим имаше пет лакътя и досягаше стената на дома; и другото крило имаше пет лакътя и досягаше крилото на другия херувим,
\par 13 Крилата на тия херувими се простираха на двадесет лакътя; и те стояха на нозете си, и лицата им гледаха на навътре.
\par 14 И направи завесата от синьо, мораво, червено и висон, и изработи по нея херувими.
\par 15 Направи пред дома и два стълба тридесет и пет лакътя високи, с капител на върха на всеки от тях пет лакътя висок .
\par 16 Направи и верижки, както в светилището, и тури ги на върховете на стълбовете; и направи сто нара, които окачи на верижките.
\par 17 И изправи стълбовете пред храма, единият отдясно, а другият отляво; и нарече оня, които беше отдясно, Яхин, а оня, който беше отляво, Воаз.

\chapter{4}

\par 1 При това направи меден олтар, с дължина двадесет лакътя, а височина десет лакътя.
\par 2 Направи още леяно море, с устие десет лакътя широко , кръгло изоколо, а с височина пет лакътя; и връв от тридесет лакътя го измерваше околовръст.
\par 3 И под устието му имаше образи на волове, които го обикаляха наоколо, по десет на един лакът; те обикаляха морето изоколо; воловете бяха на два реда, излеяни в едно цяло с него.
\par 4 То стоеше на дванадесет вола; три гледаха към север, три гледаха към запад, три гледаха към юг, и три гледаха към изток, а морето стоеше върху тях; и задните части на всички бяха навътре.
\par 5 Дебелината му беше една длан; а устието беше направено като устие на чаша, във вид на кремов цвят; и побираше, когато бе пълно, три хиляди вати вода ,
\par 6 Направи още десет умивалници, от които тури пет отдясно и пет от ляво на дома , за да мият в тях; там миеха това, което беше за всеизгаряне; морето обаче бе за да се мият в него свещениците.
\par 7 Направи и десет златни светилника, според наставлението за тях, и тури ги в храма, пет отдясно и пет отляво.
\par 8 И направи десет трапези, които тури в храма, пет отдясно и пет отляво. Направи и сто златни легена.
\par 9 При това направи двора за свещениците, и големия двор, и порти за двора, и покри вратите им с зад.
\par 10 И постави морето от дясната страна на дома , към изток, къде юг.
\par 11 Хирам направи и котлите, лопатите и легените Така Хирам свърши изработването на работите, които направи на цар Соломона за Божия дом:
\par 12 двата стълба; изпъкналите части; двата капитела, които бяха на върха на стълбовете; двете мрежи за покриване двете изпъкнали части на капителите, които бяха на върховете на стълбовете;
\par 13 четиристотинте нарове за двете мрежи - два реда нарове за всяка мрежа - за да покриват двете изпъкнали части на капителите, които бяха върху стълбовете.
\par 14 Тоже направи подножията; направи и умивалниците върху подножията;
\par 15 и едното море с дванадесетте вола под него.
\par 16 Също и котлите, лопатите, вилиците и всичките им прибори майсторът му Хирам направи на цар Соломона за Господния дом от лъскава мед.
\par 17 На Иорданското поле царят ги изля в глинената земя между Сокхот и Середата.
\par 18 Така Соломон направи всички вещи, толкоз много, щото теглото на медта не можеше да се пресметне.
\par 19 Също Соломон направи всички вещи, които бяха за Божия дом; и златния олтар; и трапезите, на които се полагаха присъствените хлябове;
\par 20 и светилниците със светилата им, от чисто злато, за да горят пред светилището според наредбата;
\par 21 и цветята, светилата и клещите, от злато, и то чисто злато;
\par 22 и щипците, легените, темянниците и кадилниците, от чисто злато. А колкото за входа на дома, вътрешните му врати за в пресветото място, и вратите на дома, то ест , на храма, бяха от злато.

\chapter{5}

\par 1 Така се свърши всичката работа, която Соломон направи за Господния дом. И Соломон внесе нещата посветени от баща му Давида, - среброто златото и всичките вещи, - и ги положи в съкровищниците на Божия дом.
\par 2 Тогава Соломон събра в Ерусалим Израилевите старейшини, и всичките началници на племената, началниците на бащините домове на израилтяните, за да пренесат ковчега на Господния завет от Давидовия град, който е Сион.
\par 3 И тъй всичките Израилеви мъже се събраха пред царя на празника в седмия месец.
\par 4 А когато дойдоха всичките Израилеви старейшини, левитите дигнаха ковчега.
\par 5 Те пренесоха ковчега и шатъра за срещане с всичките свети принадлежности, които бяха в шатъра; свещениците и левитите ги пренесоха.
\par 6 А цар Соломон и цялото Израилево общество, които се бяха събрали пред него, бяха пред ковчега, и жертвуваха овци и говеда, които по множеството си не можеха да се пресметнат или да се изброят.
\par 7 Така свещениците внесоха ковчега на Господния завет на мястото му, в светилището на дома, в пресветото място, под крилата на херувимите.
\par 8 Защото херувимите бяха с крилата си разперени над мястото на ковчега, и херувимите покриваха отгоре ковчега и върлините му.
\par 9 И върлините се издадоха така, щото се виждаха краищата на върлините издадени от ковчега пред светилището, но извън не се виждаха; и там са до днес.
\par 10 В ковчега нямаше друго освен двете плочи, които Моисей положи там на Хорив, гдето Господ направи завет със израилтяните, когато бяха излезли из Египет.
\par 11 А щом излязоха свещениците из светилището, (защото всичките свещеници, които се намираха там, бяха се осветили без да чакат реда на отредите си;
\par 12 е всичките певци левити - Асаф, Еман, Едутун и синовете им и братята им облечени във висон, и държащи кимвали, псалтири и арфи стояха на изток от олтара, и с тях сто и двадесет свещеника, които свиреха с тръби),
\par 13 и като възгласиха тръбачите и певците заедно, пеещи и славословещи Господа с един глас, и като издигнаха глас с тръби и кимвали и музикални инструменти та хвалеха Господа, защото е благ, защото милостта Му е до века, тогава домът, Господният дом, се изпълни с облак,
\par 14 така щото поради облака свещениците не можаха да застанат за да служат, защото Господната слава изпълни Божия дом.

\chapter{6}

\par 1 Тогава Соломон говори: Господ е казал, че ще обитава в мрак;
\par 2 но аз Ти построих дом за обитаване, място, в което да пребиваваш вечно.
\par 3 После царят обърна лицето си та благослови цялото Израилево общество, докато цялото Израилево общество стоеше на крака, като каза:
\par 4 Благословен да бъде Господ Израилевият Бог, Който извършва с ръцете Си онова, което говори с устата Си на баща ми Давида, като рече:
\par 5 От деня, когато изведох людете Си из Египетската земя, не избрах измежду измежду всичките Израилеви племена ни един град, гдето да се построи дом, за да се настани името Ми там, нито избрах мъж за да бъде вожд на людете Ми Израиля;
\par 6 но избрах Ерусалим за да се настани името Ми там, и избрах Давида за да бъде над людете Ми Израиля;
\par 7 И в сърцето на баща ми Давида дойде да построи дом за името на Господа Израилевия Бог;
\par 8 но Господ рече на баща ми Давида: Понеже дойде в сърцето ти да построиш дом за името Ми, добре си сторил, че е дошло това в сърцето ти.
\par 9 Ти, обаче, няма да построиш дома; но синът ти, който ще излезе из чреслата ти, той ще построи дом за името Ми.
\par 10 Господ прочее, изпълни словото, което говори; и като се издигнах аз вместо баща си Давида, и седнах на Израилевия престол, според както говори Господ, построих дома за името на Господа Израилевия Бог.
\par 11 И там турих ковчега, в който е заветът, който Господ направи с израилтяните.
\par 12 Тогава Соломон застана пред Господния олтар, пред цялото Израилево общество, и простря ръцете си.
\par 13 (Защото Соломон беше направил медна платформа, която беше пет лакътя дълга, пет лакътя широка и пет лакътя висока, която беше положил всред двора, и като застана на нея, падна на коленете си пред цялото Израилево общество). Той, прочее, простря ръцете си към небето и рече:
\par 14 Господи Боже Израилев, няма на небето и на земята Бог подобен на Тебе, Който пазиш завета и милостта към слугите Си, които ходят пред Тебе с цялото си сърце;
\par 15 Който си изпълнил към слугата Си Давида, баща ми, това, което си му обещал; да! това, което си говорил с устата Си, него си и свършил с ръката Си, както се вижда днес.
\par 16 Сега, Господи Боже Израилев, изпълни към слугата Си Давида баща ми, онова което си му обещал, като си рекъл: Няма да ти липсва мъж, който да седи пред Мен на Израилевия престол, ако само внимават чадата ти в пътя си, за да ходят в закона Ми, както ти си ходил пред Мене.
\par 17 Сега, прочее, Господи Боже Израилев, нека се потвърди словото, което си говорил на слугата си Давида.
\par 18 Но Бог наистина ли ще обитава с човека на земята? Ето, небето и небето на небесата не са достатъчни да Те поберат; колко по-малко тоя дом, който построих!
\par 19 Но пак погледни благосклонно към молитвата на слугата Си и към молението му, Господи Боже мой, тъй щото да послушаш вика и молитвата, с която слугата Ти се моли пред Тебе,
\par 20 за да бъдат очите Ти денем и нощем отворени към тоя дом, към мястото, за което Ти си казал, че ще настаниш името Си там, за да слушаш молитвата, с която слугата Ти ще се моли на това място.
\par 21 Слушай моленията на слугата Си и на людете Си Израиля, когато се молят на това място; да! слушай Ти от местообиталището Си от небето, и като слушаш бивай милостив.
\par 22 Ако съгреши някой на ближния си, и му се наложи клетва, за да се закълне, и той дойде та се закълне пред олтара Ти в тоя дом,
\par 23 тогава слушай Ти от небето и подействувай, извърши правосъдие за слугите Си, и въздай на беззаконния, така щото да възвърнеш делото му върху главата му, а оправдай праведния като му отдадеш според правдата му.
\par 24 Ако людете Ти Израил бъдат разбити пред неприятеля по причина, че са Ти съгрешили, и се обърнат та изповядат Твоето име, и принесат молитва като се помолят пред Тебе в тоя дом,
\par 25 тогава Ти послушай от небето, и прости греха на людете Си Израиля, и възвърни ги в земята, която си дал на тях и на бащите им.
\par 26 Когато се затвори небето, та не вали дъжд по причина, че са Ти съгрешили, ако се помолят на това място и изповядат Твоето име, и се обърнат от греховете си понеже ги съкрушаваш,
\par 27 тогава Ти послушай от небето и прости греха на слугите Си и на людете Си Израиля, и покажи им добрия път, в който трябва да ходят, и дай дъжд на земята Си, която си дал в наследство на людете Си.
\par 28 Ако настане глад на земята, ако настане мор, ако се появят изсушителен вятър, мана, скакалци или гъсеници, ако неприятелите им ги обсадят в градовете на земята им, - каква да е язва или каква да е болест ако се появи между тях , -
\par 29 тогава всяка молитва, всяко моление, което би се принесло от кой да бил човек или от всичките Ти люде Израиля, когато всеки познае раната си и болката си и простре ръцете си към Твоя дом,
\par 30 Ти послушай от небето, от местообиталището Си, и прости, и въздай на всекиго според всичките постъпки, като познаваш сърцето му, (защото Ти, само Ти, познаваш сърцата на човешкия род),
\par 31 за да Ти се боят, като ходят в Твоите пътища, през цялото време, когато живеят на земята, която си дал на бащите ни.
\par 32 А чужденеца, още който не е от людете Ти Израиля, но иде от далечна страна, заради Твоето велико име, и заради Твоята мощна ръка, и заради Твоята издигната мишца, - когато дойдат та се помолят в тоя дом,
\par 33 тогава послушай от небето, от местообиталището Си, и подействувай според всичко, за което чужденецът Те призове; за да познаят името Ти всичките племена на света, и Ти се боят както людете Ти Израил, и да познаят, че с Твоето име се нарече тоя дом който построих.
\par 34 Ако людете Ти излязат на бой против неприятелите си, където би ги пратил Ти, и Ти се помолят като се обърнат към тоя град, който Ти си избрал, и към дома, който построих на Твоето име,
\par 35 тогава послушай от небето молитвата им и моленето им, и защити правото им.
\par 36 Ако Ти съгрешат, (защото няма човек, който да не греши), и Ти се разгневиш на тях та го предадеш на неприятеля, и пленителите им ги заведат пленници в земя далечна или близка,
\par 37 и дойдат на себе си в земята, гдето са отведени пленници, та се обърнат и Ти се помолят в земята, гдето са запленени, и рекат: Съгрешихме, беззаконстувахме, сторихме неправда,
\par 38 и се обърнат към Тебе с цялото си сърце и с цялата си душа в земята, гдето са запленени, гдето са ги завели пленници, и се помолят, като се обърнат към земята си, която се дал на бащите им към града, който си избрал, и към дома, който построих за Твоето име,
\par 39 тогава Ти послушай от небето, от местообиталището Си, молитвата им и моленията им, защити правото им, и прости на людете Си, които са Ти съгрешили.
\par 40 Сега, моля Ти се, Боже мой, нека бъдат отворени очите Ти, и внимателни ушите Ти, към молитвата, който се принася на това място.
\par 41 Стани, прочее, сега, Господи Боже, влез в покоя Си, Ти и ковчегът на Твоята сила; свещениците Ти, Господи Боже, да бъдат облечени със спасение, и светиите Ти нака се веселят в блага.
\par 42 Господи Боже, недей отблъсва лицето на помазаника Си: помни милостта, която си показал към слугата Си Давида.

\chapter{7}

\par 1 Като свърши Соломон молитвата си, огън излезе от небето та изгори всеизгарянията и жертвите, и Господната слава изпълни дома.
\par 2 И свещениците не можаха да влязат в Господния дом, защото славата на Господа изпълни дома Му.
\par 3 И всичките израилтяни, като видяха, че огънят слезе, и че Господната слава бе над дома, наведоха се с лице до земята върху постланите камъни та се поклониха, прославящи Господа, защото е благ защото милостта Му е до века.
\par 4 Тогава царят и всичките люде принесоха жертви пред Господа.
\par 5 Цар Соломон принесе в жертви двадесет и две хиляди говеда и сто и двадесет хиляди овци. Така царят и всичките люде посветиха Божият дом.
\par 6 И свещениците стояха според службите си, също и левитите с инструментите за песнопеене Господу, които цар Давид беше направил, за да славословят Господа, защото милостта Му е до века, когато Давид въздаваше хвала чрез тяхното служене; а свещениците тръбяха срещу тях; докато целият Израил стоеше.
\par 7 Соломон освети и средата на двора, който е към лицето на Господния дом; защото там принесе всеизгарянията и тлъстината на примирителните приноси, понеже медният олтар, който Соломон бе направил, не можеше да побере всеизгарянията и хлебния принос и тлъстината.
\par 8 По тоя начин Соломон и целият Израил с него, твърде голям събор събран из местностите от прохода на Емат до Египетския проток, пазеха в онова време празника седем дни.
\par 9 А на осмия ден държаха тържествено събрание; защото пазеха посвещението на олтара седем дни и празника седем дни.
\par 10 И на двадесет и третия ден от седмия месец царят изпрати людете в шатрите им с радостни и весели сърца за благостите, които Господ бе показал към Давида, към Соломона и към людете Си Израиля.
\par 11 Така Соломон свърши Господния дом и царската къща; всичко, каквото дойде в сърцето на Соломона да направи в Господния дом и в своята къща, свърши го благополучно.
\par 12 Тогава Господ се яви на Соломона през нощта и му каза: Чух молитвата ти и избрах това място да Ми бъде дом за жертви.
\par 13 Ако заключа небето да не вали дъжд, или ако заповядам на скакалците да изпоядат земята, или ако изпратя мор между людете Си,
\par 14 и людете Ми, които се наричат с Моето име, смирят себе си та се помолят и потърсят лицето Ми, и се върнат от нечестивите си пътища, тогава ще послушам от небето, ще простя греха им и ще изцеля земята им.
\par 15 Сега очите Ми ще бъдат отворени и ушите Ми внимателни, към молитвата, която се принася на това място.
\par 16 Защото сега избрах и осветих тоя дом за да бъде името Ми в него до века; и очите Ми и сърцето Ми ще бъдат там за винаги.
\par 17 А колкото за тебе, ако ходиш пред Мене както ходи баща ти Давид, и постъпваш напълно според както съм ти заповядал, и пазиш повеленията Ми и съдбите Ми,
\par 18 тогава ще утвърдя престола на царството ти според завета, който направих с баща ти Давида, като рекох: Няма да ти липсва мъж да управлява Израиля.
\par 19 Но ако се отклоните, ако оставите повеленията и заповедите, които поставих пред вас, и отидете та послужите на други богове и им се поклоните,
\par 20 тогава ще ги изкореня от земята Си, която съм им дал, и ще отхвърля отпред очите Си тоя дом, който осветих за името Си, и ще го направя за поговорка и поругание между всичките племена.
\par 21 А за тоя дом, който стана толкова висок, всеки, който минава край него, ще се зачуди и ще рече: Защо направи Господ така на тая земя и на тоя дом?
\par 22 И ще отговорят: Понеже оставиха Господа Бога на бащите си, Който ги изведе из Египетската земя, та се хванаха за други богове и им се поклониха и им послужиха, за това нанесе на тях всичкото това зло.

\chapter{8}

\par 1 А когато се свършиха двадесетте години, в които Соломон построи Господния дом и своята къща,
\par 2 Соломон съгради изново градовете, които Хирам му бе дал, и засели в тях израилтяни.
\par 3 И Соломон отиде в Ематсова и пределите над него.
\par 4 И съгради Тадмор в пустата част на Юдовата земя, и всичките градове за житниците, които съгради в Емат.
\par 5 Съгради още горния Веторон и долния Веторон, градовете укрепени със стени, порти и лостове,
\par 6 и Ваалат и всичките градове, в които Соломон имаше житници, и всичките градове за колесниците, и градовете за конниците, и все що пожела Соломон да съгради в Ерусалим, в Ливан и в цялата земя на царството си.
\par 7 А относно всичките люде, които останаха от хетейците, аморейците, ферезейците, евейците и евусейците, които не бяха от Израиля,
\par 8 от техните потомци, останали подир тях в земята, които израилтяните не бяха изтребили, от тях Соломон събра набор за задължителни работници, каквито са и до днес.
\par 9 Но от израилтяните Соломон не направи никого слуга за работата си; а те бяха военни мъже, и главните му военачалници, и началниците на колесниците му и на конниците му;
\par 10 тоже те бяха главните началници, които имаше цар Соломон (двеста и петдесет души), които началствуваха над людете.
\par 11 Тогава Соломон възведе Фараоновата дъщеря от Давидовия град в къщата, която беше построил за нея; защото рече: Жена ми да не живее в къщата на Израилевия цар Давида, понеже местата, в които е влизал Господният ковчег, са свети.
\par 12 Тогава Соломон почна да принася всеизгаряния Господу на Господния олтар, който бе издигнал пред трема,
\par 13 като принасяше потребното за всеки ден според Моисеевата заповед, в съботите, на новолунията и на празниците, които ставаха три пъти в годината, - на празника на безквасните хлябове , на празника на седмиците и на празника на шатроразпъването.
\par 14 И според наредбата на баща си Давида, постави отредите на свещениците на службата им, и левитите на заръчаното за тях, да пеят и да служат пред свещениците, според както беше потребно за всеки ден, и вратарите според отредите им на всяка порта; защото така бе заповедта от Божия човек Давид.
\par 15 И в нищо не се отклониха от царската заповед относно свещениците и левитите, нито относно съкровищата.
\par 16 А цялата Соломонова работа беше предварително приготвена, от дена, когато се положи основата на Господния дом докле се свърши. Така се свърши Господния дом.
\par 17 Тогава Соломон отиде в Есион-гавер, и в Елот, на морския бряг в едомската земя.
\par 18 И Хирам му прати, под грижата на слугите си, кораби и опитни морски слуги, та отидоха със Соломоновите слуги в Офир; и от там взеха четиристотин и петдесет таланта злато та го донесоха на цар Соломона.

\chapter{9}

\par 1 И савската царица, като слушаше да се прочува Соломон, дойде в Ерусалим да опита Соломона с мъчни за нея въпроси; дойде с една твърде голяма свита, с камили натоварени с аромати, и с много злато и скъпоценни камъни, и като дойде при Соломона, говори с него за всичко що имаше на сърцето си.
\par 2 И Соломон отговори на всичките й въпроси; нямаше нищо скрито за Соломона, което не можа да й обясни.
\par 3 А като видя савската царица мъдростта на Соломона и къщата, която бе построил,
\par 4 ястията на трапезата му, сяденето на слугите му, и прислужването на служителите му и облеклото им, също и виночерпците му и тяхното облекло, и нагорнището, с което отиваше в Господния дом, не остана дух в нея.
\par 5 И рече на царя: Верен беше слухът, който чух в земята си, за твоето състояние и за мъдростта ти.
\par 6 Аз не вярвах думите им докато не дойдох и не видях с очите си; но, ето, нито половината от величието на мъдростта ти не ми е била казана; ти надминаваш слуха, който бях чула.
\par 7 Честити мъжете ти, и честити тия твои слуги, които стоят всякога пред тебе, та слушат мъдростта ти.
\par 8 Да бъде благословен Господ Бог, който има благоволение към тебе да те постави на престола Си цар за Господа твоя Бог. Понеже твоят Бог е възлюбил Израиля за да го утвърди до века, за това те е поставил цар над тях, за да раздаваш правосъдие и да вършиш правда.
\par 9 И тя даде на царя сто и двадесет таланта злато и твърде много аромати и скъпоценни камъни; не е имало никога такива аромати, каквито савската царица даде на цар Соломона.
\par 10 (Още и Хирамовите слуги и Соломоновите слуги, които донасяха злато от Офир, донасяха и алмугови дървета и скъпоценни камъни.
\par 11 А от алмуговите дървета царят направи подпорки за Господния дом и за царската къща, тоже и арфи и псалтири за певците; такива дървета не бяха се виждали по-напред в Юдовата земя).
\par 12 И цар Соломон даде на савската царица всичко, що тя желаеше, каквото поиска, освен равното на онова, което тя беше донесла на царя. И тъй, тя се върна със слугите си та си отиде в своята си земя.
\par 13 А теглото на златото, което дохождаше на Соломона всяка година, беше шестстотин шестдесет и шест златни таланта,
\par 14 освен онова, което се внасяше от купувачите, от търговците, от всичките арабски царе, и от управителите на страната, които донасяха на Соломона злато и сребро.
\par 15 И цар Соломон направи двеста щита от ковано злато; шестстотин сикли злато се иждиви за всеки щит;
\par 16 и триста щитчета от ковано злато; три фунта злато се иждиви за всяко щитче; и царят ги положи в къщата Ливански лес.
\par 17 Царят направи и един великолепен престол от слонова кост, който позлати с чисто злато.
\par 18 Престолът имаше шест стъпала и едно златно подножие закрепени за престола, и облегалки от двете страни на седалището, и два лъва стоящи край облегалките.
\par 19 А там, върху шестте стъпала, от двете страни стояха дванадесет лъва; подобно нещо не се е направило в никое царство.
\par 20 И всичките цар Соломонови съдове за пиене бяха златни, а всичките съдове в къщата Ливански лес от чисто злато; ни един не бе от сребро; среброто се считаше за нищо в Соломоновите дни.
\par 21 Защото царят имаше, заедно с Хирамовите слуги, кораби като ония , които отиваха в Тарсис; еднъж в три години тия тарсийски кораби дохождаха и донасяха злато и сребро, слонова кост, маймуни и пауни.
\par 22 Така цар Соломон надмина всичките царе на света по богатство и мъдрост.
\par 23 И всичките царе на света търсеха Соломоновото присъствие, за да чуят мъдростта, която Бог бе турил в сърцето му.
\par 24 И всяка година донасяха всеки от подаръка си, сребърни вещи, златни вещи, облекла, оръжия и аромати, коне и мъски.
\par 25 Соломон имаше тоже четири хиляди обора за коне и за колесници и дванадесет хиляди конници, които настани в градовете за колесниците и при царя в Ерусалим.
\par 26 И владееше над всичките царе от реката Евфрат до филистимската земя и до границите на Египет.
\par 27 И царят направи среброто да изобилва в Ерусалим като камъни, а кедрите направи като полските черници.
\par 28 И докарваха коне за Соломона от Египет и от всичките страни.
\par 29 А останалите дела на Соломона, първите и последните, не са ли написани в Книгата на пророк Натана, и в пророчеството на силонеца Ахия, и във Виденията на гледача Идо, които изрече против Еровоама, Наватовия син?
\par 30 И Соломон царува в Ерусалим над целия Израил четиридесет години.
\par 31 Така Соломон заспа с бащите си, и биде погребан в града на баща си Давида; и вместо него се възцари син му Ровоам.

\chapter{10}

\par 1 И Ровоам отиде в Сихем; защото в Сихем беше се стекъл целият Израил за да го направи цар.
\par 2 И Еровоам, Наватовият син, който бе в Египет, гдето бе побягнал от присъствието на цар Соломона, когато чу това, Еровоам се върна от Египет;
\par 3 защото пратиха та го повикаха. Тогава Еровоам и целият Израил дойдоха та говориха на Ровоама, като рекоха:
\par 4 Баща ти направи непоносим хомота ни; сега, прочее, ти олекчи жестокото ни работене на баща ти и тежкия хомот, който наложи върху нас и ще ти работим.
\par 5 А той им рече: Върнете се при мене подир три дена. И людете си отидоха.
\par 6 Тогава цар Ровоам се съветва със старейшините, които бяха служили пред баща му Соломона, когато беше още жив, като каза: Как ме съветвате да отговоря на тия люде?
\par 7 Те му отговориха казвайки: Ако се отнесеш благосклонно към тия люде, та им угодиш, и им говориш благи думи, тогава те ще ти бъдат слуги за винаги.
\par 8 Но той отхвърли съвета, който старейшините му дадоха, та се съветва с младите си служители, които бяха пораснали заедно с него.
\par 9 Рече им: Как ме съветвате вие да отговорим на тия люде, които ми говориха казвайки: Олекчи хомота, който баща ти наложи върху нас?
\par 10 И младите, които бяха пораснали заедно с него, му отговориха казвайки: Така да кажеш на людете, които ти говориха, казвайки: Баща ти направи тежък хомота ни, но ти да ни го олекчиш, - така да им речеш: Малкият ми пръст ще бъде по-дебел от бащиния ми кръст.
\par 11 Сега, ако баща ми ви е товарил с тежък хомот, то аз ще направя още по-тежък хомота ви; ако баща ми ви е наказвал с бичове, то аз ще ви наказвам със скорпии.
\par 12 Тогава Еровоам и всичките люде дойдоха при Ровоама на третия ден, според както царят бе говорил, казвайки: Върнете се при мене на третия ден.
\par 13 И царят им отговори остро, като остави цар Ровоам съвета на старейшините,
\par 14 и отговори им по съвета на младежите, та каза: Баща ми направи тежък хомота ви, но аз ще му приложа; баща ми ви наказа с бичове, но аз ще ви накажа със скорпии.
\par 15 Така царят не послуша людете: защото това нещо стана от Бога, за да изпълни словото, което Господ бе говорил чрез силонеца Ахия на Еровоама Наватовия син.
\par 16 А като видя целият Израил, че царят не ги послуша, людете в отговор на царя рекоха: Какъв дял имаме ние в Давида? Никакво наследство нямаме в Есеевия син! Всеки в шатрите си, Израилю! Промишлавай сега, Давиде, за дома си. И така целият Израил си отиде в шатрите.
\par 17 А колкото за израилтяните, които живееха в Юдовите градове, Ровоам царуваше над тях.
\par 18 Тогава цар Ровоам прати при другите израилтяни Адорама, който бе над набора; но израилтяните го биха с камъни, та умря. Затова цар Ровоам побърза да се качи на колесницата си за да побегне в Ерусалим.
\par 19 Така Израил въстана против Давидовия дом, и остана въстанал до днес.

\chapter{11}

\par 1 Тогава Ровоам, като дойде в Ерусалим, събра Юдовия и Вениаминовия дом, сто и осемдесет хиляди отборни войници, за да се бият против Израиля, та дано възвърнат царството пак на Ровоама.
\par 2 Но Господното слово дойде към Божия човек Самаия, и рече:
\par 3 Говори на Ровоама Соломоновия син, Юдовия цар, и на целия Израил в Юда и във Вениамин, като речеш:
\par 4 Така казва Господ: Не възлизайте, нито се бийте против братята си; върнете се всеки у дома си, защото от Мене става това нещо. И те послушаха Господните думи та се върнаха, и не отидоха против Еровоама.
\par 5 А Ровоам, като се установи, в Ерусалим, съгради укрепени градове в Юда;
\par 6 съгради Витлеем, Итам, Текуе,
\par 7 Ветсур, Сохо, Одолам,
\par 8 Гет, Мариса, Зиф,
\par 9 Адораим, Лахис, Азика,
\par 10 Сарая, Еалон и Хеврон, които са укрепени градове в Юда и във Вениамин,
\par 11 Укрепи тия крепости, и тури в тях военачалници и запаси от храна, дървено масло и вино.
\par 12 Още във всеки град тури щитове и копия, и укрепи ги твърде много. Така Юда и Вениамин останаха под него.
\par 13 И всичките свещеници и левити, които бяха в Израил, се събраха при него от всичките си краища.
\par 14 Защото левитите оставиха пасбищата си и притежанията си та дойдоха в Юда и в Ерусалим; понеже Еровоама и синовете му бяха ги изпъдили, за да не свещенодействуват Господу,
\par 15 и Еровоам беше си поставил жреци за високите места, за бесовете и за телетата, които бе направил.
\par 16 И след тях, колкото души от всичките Израилеви племена утвърдиха сърцата си да търсят Господа Израилевия Бог, дойдоха в Ерусалим за да жертвуват на Господа Бога на бащите си.
\par 17 Така за три години те подкрепиха Юдовото царство и поддържаха Ровоама Соломоновия син; защото три години ходиха в пътя на Давида и на Соломона.
\par 18 И Ровоам си взе за жена Меалетя, дъщеря на Давидовия син Еримот, и Авихаила, дъщеря на Есеевия син Елиав,
\par 19 която му роди синове: Еуса, Самария и Заама.
\par 20 А подир нея взе Авия, Атай, Зиза и Селомита.
\par 21 И Ровоам възлюби Мааха Авесаломовата дъщеря повече от всичките си жени и наложници: (защото взе осемнадесет жени и шест наложници, и роди двадесет и осем сина и шестдесет дъщери);
\par 22 и Ровоам постави Маахиния син Авия за княз да началствува над братята си, защото мислеше да го направи цар.
\par 23 И постъпваше разумно, като разпръсна всичките си синове, по неколцина във всеки укрепен град, по всичките Юдови и Вениаминови земи, и даваше им изобилна храна. И за тях потърси много жени.

\chapter{12}

\par 1 Но след като се закрепи царството на Ровоама, и той стана силен, остави Господния закон, а заедно с него и целия Израил.
\par 2 И понеже бяха престъпили пред Господа, затова, в петата година на Ровоамовото царуване, египетския цар Сисак възлезе против Ерусалим
\par 3 с хиляда и двеста колесници и шестдесет хиляди конници; и людете, които дойдоха с него от Египет, ливийци, сукияни и етиопяни, бяха безбройни.
\par 4 И като превзе Юдовите укрепени градове, дойде до Ерусалим.
\par 5 Тогава пророк Семаия дойде при Ровоама и при Юдовите първенци, като се бяха събрали в Ерусалим, поради нахлуването на Сисака, та им рече: Така казва Господ: Вие оставихте Мене; затова и Аз оставих вас в ръцете на Сисака.
\par 6 По това Израилевите първенци и царят се смириха, и казаха: Праведен е Господ.
\par 7 А когато видя Господ, че се смириха, Господното слово дойде към Семаия и каза: Те се смириха; няма да ги изтребя, но ще им дам някакво избавление; и гневът Ми няма да се излее върху Ерусалим чрез Сисака.
\par 8 Обаче те ще му станат слуги, за да познаят, що е да слугуват на Мене и що да слугуват на земните царства.
\par 9 И тъй, египетския цар Сисак възлезе против Ерусалим, та отнесе съкровищата на Господния дом и съкровищата на царската къща; отнесе всичко; отнесе още златните щитове, които Соломон бе направил.
\par 10 (А вместо тях цар Ровоам направи медни щитове и ги предаде в ръцете на началниците на телохранителите, които пазеха вратата на царската къща.
\par 11 И когато влизаше царят в Господния дом, телохранителите дохождаха та ги вземаха; после пак ги занасяха в залата на телохранителите).
\par 12 И като се смири Ровоам , гневът на Господа се отвърна от него, та да не го погуби съвсем; па и в Юда се намираше някакво добро.
\par 13 Така цар Ровоам се закрепи в Ерусалим и царуваше; защото, когато се възцари, Ровоам бе четиридесет и една години на възраст; и царува седемнадесет години в Ерусалим, града който Господ бе избрал измежду всичките Израилеви племена за да настани името Си там. Името на майка му, амонката, бе Наама.
\par 14 А той стори зло, като не оправи сърцето си да търси Господа.
\par 15 А делата на Ровоама, първите и последните, не са ли написани в книгите на пророка Семаия и на гледача Идо за архивите? А между Ровоама и Еровоама имаше постоанни войни.
\par 16 И Ровоам заспа с бащите си и погребан биде в Давидовия град. И вместо него се възцари син му Авия.

\chapter{13}

\par 1 В осемнадесетата година от царуването на Еровоама, Авия се възцари над Юда,
\par 2 и царува три години в Ерусалим. Името на майка му бе Михаия, дъщеря на Уриила от Гавая. И имаше война между Авия и Еровоама.
\par 3 И Авия се опълчи за бой с войска от силни войници, на брой четиристотин хиляди отборни мъже; а Еровоам се опълчи за бой против него са осемстотин хиляди отборни мъже, силни и храбри.
\par 4 Тогава Авия застана на хълма Семараим, който е в Ефремовата хълмиста земя, и каза: Слушайте ме, Еровоаме и целий Израилю.
\par 5 Не трябва ли да знаете, че Господ Израилевият Бог, даде за винаги на Давида да царува над Израиля, той и потомците му, чрез завет със сол?
\par 6 При все това, Еровоам, Наватовият син, слуга на Давидовия син Соломона, стана и се повдигна против господаря си;
\par 7 и при него се събраха нищожни и лоши човеци, та се засилиха против Ровоама, Соломоновия син, когато Ровоам бе млад и с крехко сърца и не можеше да им противостои.
\par 8 И сега вие замислювате да се противите на Господното царство, което е в ръцете на Давидовите потомци, защото сте голямо множество, и между вас има златни телета, които Еровоам ви е направил за богове.
\par 9 Не изпъдихте ли Господните свещеници, Аароновите потомци и левитите, и не направихте ли си жреци, както правят людете на другите земи, така щото всеки, които иде да се освети с теле и седем овена, той може да стане жрец на ония, които не са богове?
\par 10 А колкото за нас, Иеова е нашият Бог, и ние не Го оставяме; и ние имаме за свещеници, които служат Господу, Ааронови потомци и левити на службата им,
\par 11 които всяка заран и всяка вечер горят пред Господа всеизгаряния и благоуханен темян, и нареждат присъствени хлябове върху чисто - златната трапеза, и златния светилник и светилата му, за да гори всяка вечер; защото ние пазим заръчаното от Господа нашия Бог; вие обаче Го оставихте.
\par 12 И, ето, с нас е Бог начело, и свещениците Му с гръмливи тръби, за да свирят тревога против вас. Чада на Израиля, не воювайте против Господа Бога на бащите си, защото няма да успеете.
\par 13 Обаче, Еровоам накара една засада да обиколи и да иде зад тях, така щото те бяха пред Юдовите мъже , а засадата зад тях.
\par 14 И когато Юда назърна надире, ето, боят бе и отпреде им и отзаде; затова извикаха към Господа, и свещениците засвириха с тръбите.
\par 15 Тогава Юдовите мъже нададоха вик; и като извикаха Юдовите мъже, Бог порази Еровоама и целия Израил пред Авия и Юда.
\par 16 Израилтяните бягаха пред Юда; и Бог ги предаде в ръката им,
\par 17 така щото Авия и неговите люде им нанесоха голямо поражение, и от Израиля паднаха убити петстотин хиляди отборни мъже.
\par 18 Така в онова време израилтяните се смириха; а юдейците превъзмогнаха, понеже уповаваха на Господа Бог на бащите си.
\par 19 И Авия преследва Еровоама и му отне градовете: Ветил и селата му, Есана и селата му, Ефрон и селата му.
\par 20 И в дните на Авия, Еровоам не си възвърна вече силата; и Господ го порази, та умря.
\par 21 Но Авия се засили и взе четиринадесет жени, и роди двадесет и два сина и шестнадесет дъщери.
\par 22 А останалите дела на Авия, и постъпките му, и изреченията му са написани в повестите на пророк Идо.

\chapter{14}

\par 1 Така Авия заспа с бащите си и погребаха го в Давидовия град; и вместо него се възцари син му Аса. В неговите дни земята беше спокойна десет години.
\par 2 И Аса върши това, което бе добро пред Господа своя Бог;
\par 3 защото махна жертвениците на чуждите богове и високите места, изпотроши идолите и изсече ашерите;
\par 4 и заповяда на Юда да търсят Господа Бога на бащите си, и да изпълняват закона и заповедите Му.
\par 5 Махна още от всичките Юдови градове високите места и кумирите на слънцето. И царството утихна пред него.
\par 6 И тъй като утихна, земята и нямаше война в ония години, той съгради укрепените градове в Юда; понеже Господ му даде покой.
\par 7 Затова той каза на Юда: Да съградим тия градове, и да направим около тях стени и кули, порти и лостове, докато земята е още свободна пред нас; понеже потърсихме Господа нашия Бог, потърсихме Го, и Той ни е дал покой от всяка страна. И тъй, съградиха и благоуспяха.
\par 8 А Аса имаше войска: от Юда, триста хиляди мъже, които носеха щитове и копия; а от Вениамина, двеста и осемдесет хиляди, които носеха щитчета и запъваха лъкове; всички тия бяха силни и храбри.
\par 9 След това етиопянинът Зара излезе против тях с една войска от един милион мъже и с триста колесници и стигна до Мариса.
\par 10 И Аса излезе против него; и се опълчи за бой в долината Сефата, при Мариса.
\par 11 Тогава Аса извика към Господа своя Бог и рече: Господи, безразлично е за Тебе да помагаш на мощния или на оня, който няма никаква сила; помогни нам, Господи Боже наш, защото на Тебе уповаваме, и в Твоето име идем против това множество, Господи, Ти си нашият Бог; да не надделее човек против Тебе.
\par 12 И Господ порази етиопяните пред Аса и пред Юда; и етиопяните побягнаха.
\par 13 А Аса и людете, които бяха с него, ги преследваха до Герар; и от етиопяните паднаха толкоз много, щото не можаха вече да се съвземат, защото бидоха смазани пред Господа и пред Неговото множество. И Юдовите мъже взеха твърде много користи.
\par 14 И поразиха всичките градове около Герар, защото страх от Господа ги обзе; и ограбиха всичките градове, защото имаше в тях много користи.
\par 15 Нападнаха още и шатрите на добитъка, и като взеха много овци и камили, върнаха се в Ерусалим.

\chapter{15}

\par 1 Тогава Божият дух дойде на Азария, Одидовия син,
\par 2 и излезе да посрещне Аса и му рече: Слушай ме, Асо и целий Юдо и Вениамине: Господ е с вас докато сте вие с Него; и ако Го търсите, ще бъде намерен от вас, но ако Го оставите, Той ще ви остави.
\par 3 Дълго време Израил остана без истинския Бог, без свещеник, който да поучава, и без закон;
\par 4 но когато в бедствието си се обърнаха към Господа Израилевия Бог и Го потърсиха, Той биде намерен от тях.
\par 5 И в ония времена не е имало мир нито за излизащия, нито за влизащия, но големи смутове върху всичките жители на земите.
\par 6 Народ се сломяваше от народ, и град от град; защото Бог ги смущаваше с всякакво бедствие.
\par 7 А вие се усилвайте, и да не ослабват ръцете ви; защото делото ви ще се възнагради.
\par 8 И когато чу Аса тия думи и предсказанието на пророк Одида, ободри се, и отмахна мерзостите от цялата Юдова и Вениаминова земя, и от градовете, които бе отнел от Ефремовата хълмиста земя; и поднови Господния олтар, който бе пред Господния трем.
\par 9 И събра целия Юда и Вениамина, и живеещите между тях пришелци от Ефрема, Манасия и Симеона; защото мнозина от Израиля прибягваха при него като виждаха, че Господ неговият Бог бе с него.
\par 10 Те се събраха в Ерусалим в третия месец, в петнадесетата година от царуването на Аса.
\par 11 В това време принесоха жертви Господу от донесените користи, седемстотин говеда и седем хиляди овци.
\par 12 И стъпиха в завет да търсят Господа Бога на бащите си от цялото си сърце и от цялата си душа,
\par 13 и да се умъртвява всеки, малък или голям, мъж или жена, който не би потърсил Господа Израилевия Бог.
\par 14 И заклеха се Господу със силен глас, с възклицание, с тръби, и с рогове.
\par 15 И целият Юда се развесели поради клетвата; защото се заклеха от цялото си сърце, и потърсиха Бога с цялата си воля, и Господ им даде покой от всякъде.
\par 16 А още и майка си Мааха цар Аса свали да не бъде царица, понеже тя бе направила отвратителен идол на Ашера; и Аса съсече нейния идол, стри го, та го изгори при потока Кедрон.
\par 17 Но високите места не се премахнаха от Израиля; сърцето, обаче, на Аса бе съвършено през всичките му дни.
\par 18 И той донесе в Божия дом посветените от баща му вещи, и посветените от самия него, сребро, злато и съдове.
\par 19 И нямаше вече война до тридесет и петата година от царуването на Аса.

\chapter{16}

\par 1 В тридесет и шестата година от царуването на Аса, Израилевият цар Вааса, като възлезе против Юда, съгради Рама, за да не остави никого да излиза от Юдовия цар Аса, нито да влиза при него.
\par 2 Тогава Аса извади сребро и злато из съкровищницата на Господния дом и на царската къща та прати при сирийския цар Венадад, който жевееше в Дамаск, да рекат:
\par 3 Нека има договар между мене и тебе, както е имало между баща ми и твоя баща; ето пратих ти сребро и злато; иди, развали договора си с Израилевия цар Вааса, за да се оттегли от мене.
\par 4 И Венадад послуша цар Аса та прати началниците на силите си против Израилевите градове; и те поразиха Иион, Дан, Авел-маим и всичките Нефталимови житници-градове.
\par 5 И Вааса, като чу това, престана да гради Рама, и преустанови работата си.
\par 6 Тогава цар Аса взе целия Юда, та дигнаха камъните на Рама и дърветата й, с които Вааса градеше; и с тях съгради Гава и Масфа.
\par 7 А в това време гледачът Ананий отиде при Юдовия цар Аса та му рече: Понеже си уповал на сирийския цар, а не си уповавал на Господа своя Бог, затова войската на сирийския цар избегна и ръката ти.
\par 8 Етоипяните и ливийците не бяха ли огромно множество, с твърде много колесници и конници? Но пак, понеже ти упова на Господа, Той ги предаде в ръката ти.
\par 9 Защото очите на Господа се обръщат насам-натам през целия свят, за да се показва Той мощен в помощ на ония, чиито сърца са съвършенно разположени към Него. В това ти си постъпил безумно, защото от сега нататък ще имаш войни.
\par 10 По това, Аса се разгневи на гледача та го тури в затвор, защото се разсърди против него за туй нещо. И в същото време Аса притесни някои от людете.
\par 11 И, ето, делата на Аса, първите и последните, ето, написани са в Книгата на Юдовите и Израилевите царе.
\par 12 А в тридесет и деветата година от царуването си, Аса се разболя от болест в нозете; обаче, макар болестта му да ставаше твърде тежка, пак в болестта си той не потърси Господа, но лекарите.
\par 13 И Аса заспа с бащите си; и умря в четиридесет и първата година от царуването си.
\par 14 И погребаха го в гроба, който се бе изкопал в Давидовия град, и положиха го на легло пълно с благоухания и с различни аптекарски аромати ; и изгориха за него твърде много аромати .

\chapter{17}

\par 1 А вместо Аса се възцари син му Иосафат, който се засили против Израиля;
\par 2 защото тури военни сили във всичките укрепени градове на Юда, и постави гарнизони в Юдовата земя и в Ефремовите градове, които баща му Аса бе превзел.
\par 3 И Господ бе с Иосафата, понеже той ходи в първите пътища на баща си Давида и не потърси ваалимите,
\par 4 но потърси Бога на бащите си, и в Неговите заповеди ходи, а не по делата на Израиля.
\par 5 Затова Господ, утвърди царството в ръката му; и целият Юда даде подаръци на Иосафата; и той придоби богатство и голяма слава.
\par 6 При това, като се поощри сърцето му в Господните пътища, той отмахна от Юда високите места и ашерите.
\par 7 Тоже в третата година от царуването си прати първенците си Венхаила, Авдия, Захария, Натанаила и Михаия да поучават в Юдовите градове,
\par 8 и с тях левитите Семаия, Натания, Зевадия, Асаила, Семирамота, Ионатана, Адония, Товия и Товадония, левитите, и заедно с тях свещениците Елисама и Иорама.
\par 9 И те поучаваха в Юда, като имаха със себе си книгата на Господния закон; и обхождаха всичките Юдови градове та поучаваха людете.
\par 10 И страх от Господа обзе всичките царства в земите, които бяха около Юда, та не воюваха против Иосафата.
\par 11 Някои и от филистимците донесоха подаръци на Иосафата, и сребро като данък; арабите още му докараха стада: седем хиляди и седемстотин овена, и седем хиляди и седемстотин козли.
\par 12 И Иосафат продължаваше да се възвеличава твърде много; и съгради в Юда крепости и житници-градове.
\par 13 И имаше много работи в Юдовите градове, и военни мъже, силни и храбри, в Ерусалим.
\par 14 А ето броят им, според бащините им домове: от Юда хилядници: Адна, началникът, и с него триста хиляди души силни и храбри;
\par 15 след него, Иоанан, началникът, и с него двеста и осемдесет хиляди души;
\par 16 след него, Амасия Захриевият син, който драговолно принесе себе си Господу, и с него двеста хиляди силни и храбри мъже;
\par 17 а от Вениамина: Елиада, силен и храбър, и с него двеста хиляди души въоръжени с лъкове и щитчета;
\par 18 а след него, Иозавад, и с него сто и осемдесет хиляди души въоръжени за война.
\par 19 Тия бяха мъжете, които слугуваха на царя, освен ония, които царят постави в укрепените градове в целия Юда.

\chapter{18}

\par 1 А Иосафат имаше богатство и голяма слава; и направи сватовщина с Ахава.
\par 2 И подир няколко години слезе при Ахава в Самария. И Ахав закла много овци и говеда за него и за людете, които бяха с него; още ме положи да отиде с него в Рамот-галаад.
\par 3 Защото Израилевият цар Ахав рече на Юдовия цар Иосафата: Дохождаш ли с мене в Рамот-галаад? И той му отговори: Аз съм както си ти, и мойте люде както твоите люде; ще бъдем с тебе във войната.
\par 4 Иосафат каза още на Израилевия цар: Моля, допитай се сега до Господното слово.
\par 5 Тогава Израилевият цар събра пророците си , четиристотин мъже, та им рече: Да идем ли на бой в Рамот-галаад, или да не ида? А те казаха: Иди, и Бог ще го предаде в ръката на царя.
\par 6 Обаче, Иосафат каза: Няма ли тук, освен тия някой Господен пророк, за да се допитаме чрез него?
\par 7 И Израилевият цар рече на Иосафата: Има още един човек, чрез когото можем да се допитаме до Господа; но аз го мразя, защото никога не пророкува добро за мене, но всякога зло; той е Михей, син на Емла. А Иосафат каза: Нека не говори така царят.
\par 8 Тогава Израилевият цар повика един скопец и рече: Доведи скоро Михея син на Емла.
\par 9 А Израилевият цар и Юдовият цар Иосафат седяха, всеки на престола си, облечени в одеждите си; и седяха на открито място при входа на самарийската порта; и всичките пророци пророкуваха пред тях.
\par 10 А Седекия, Ханаановият син, си направи железни рогове, и рече: Така казва Господ: С тия ще буташ сирийците догде ги довършиш.
\par 11 Също и всичките пророци така пророкуваха, казвайки: Иди в Рамат-галаад, и ще имаш добър успех; защото Господ ще го предаде в ръката на царя.
\par 12 А пратеникът, който отиде да повика Михея, му говори казвайки: Ето, думите на пророците като из една уста са добри за царя; моля, прочее, и твоята дума да бъде като думата на един от тях, и ти говори доброто.
\par 13 А Михей рече: Заклевам се в живота на Господа, каквото рече Бог мой, това ще говоря.
\par 14 И той, дойде при царя. И царят му каза: Михее, да идем ли на бой в Рамот-галаад, или да не ида? А той рече: Идете и ще имате успех; защото неприятелите ще бъдат предадени в ръката ви.
\par 15 А царят му каза: Колко пъти ще те заклевам да не ми говориш друго освен истината в Господното име!
\par 16 А той рече: Видях целият Израил пръснат по планините като овци, които нямат овчар; и Господ рече: Тия нямат господар; нека се върнат всеки у дома си с мир.
\par 17 Тогава Израилевият цар каза на Иосафата: Не рекох ли ти, че не ще прорече добро за мене, но зло?
\par 18 А Михей рече: Чуйте, прочее, Господното слово. Видях Господа седящ на престола си, и цялото небесно множество стоящо около него отдясно и отляво.
\par 19 И Господ рече: Кой ще примами Израилевия цар Ахава за да отиде и да падне в Рамот-галаад? И един, проговаряйки, рече едно, а друг рече друго.
\par 20 Сетне излезе един дух та застана пред Господа и рече: Аз ще го примамя. И Господ му рече: Как?
\par 21 А той каза: Ще изляза и ще бъда лъжлив дух в устата на всичките му пророци. И Господ рече: Примамвай го, още и ще сполучиш; излез, стори така.
\par 22 Сега, прочее, ето, Господ е турил лъжлив дух в устата на тия твои пророци, обаче Господ е говорил зло за тебе.
\par 23 Тогава Седекия, Ханаановият син, се приближи та плесна Михея по бузата и каза: През кой път мина Господният Дух от мене за да говори на тебе?
\par 24 А Михей рече: Ето, ще видиш в оня ден, когато ще отиваш из клет в клет за да се криеш.
\par 25 Тогава Израилевият цар каза: Хванете Михея та го върнете при градския управител Амон и при царския син Иоас,
\par 26 и речете: Така казва царят: Турете тогова в тъмницата, и хранете го със затворническа порция хляб и вода догде се завърне с мир.
\par 27 И рече Михей: Ако някога се върнеш с мир, то Господ не е говорил чрез мене. Рече още: Слушайте вие, всички люде.
\par 28 И така, Израилевият цар и Юдовият цар Иосафат възлязоха в Рамот-галаад.
\par 29 И Израилевият цар рече на Иосафата: Аз ще се предреша като вляза в сражението, а ти облечи одеждите си. Прочее, Израилевият цар се предреши, та влязоха в сражението.
\par 30 А сирийският цар бе заповядал на колесниценачалниците си, казвайки: Не се бийте нито с малък, нито с голям, но само с Израилевия цар.
\par 31 А колесниценачалниците, като видяха Иосафата, рекоха: Тоя ще е Израилевият цар; и отклониха се за да го ударят. Но Иосафат извика; И Господ му помогна, защото Бог ги повлия да се отвърнат от него.
\par 32 Понеже колесниценачалниците, като видяха, че не беше Израилевият цар, престанаха да го преследват и се върнаха.
\par 33 А един човек устрели без да мери, и удари Израилевия цар между ставите на бронята му; затова той рече на колесничаря: Обърни юздата та ме извади из войската, защото съм тежко ранен.
\par 34 И в оня ден сражението се усили; но Израилевият цар се удържа в колесницата си срещу сирийците до вечерта, и около захождането на слънцето умря.

\chapter{19}

\par 1 А като се връщаше Юдовият цар Иосафат с мир у дома си в Ерусалим,
\par 2 гледачът Ииуй, Ананиевият син, излезе да го посрещне, и рече на цар Иосафата: На нечестивия ли помагаш, и тия ли, които мразят Господа, обичаш? Затова, гневът от Господа, обичаш? Затова гняв от Господа има върху тебе.
\par 3 Все пак, обаче, намериха се в тебе добри неща, защото ти отмахна ашерите от земята, и утвърди сърцето си да търсиш Бога.
\par 4 И като се засели Иосафат в Ерусалим, сетне излезе пак между людете от Вирсавее до Ефремовата хълмиста земя, та го обърна към Господа Бога на бащите им.
\par 5 И по всичките укрепени градове на Юда постави съдии в земята, град по град.
\par 6 И рече на съдиите: Внимавайте какво правите; защото не съдите за човека, но за Господа, Който е с вас в съдопроизнасянето.
\par 7 Затова нека бъде върху вас страх от Господа; внимавайте в делата си; защото у Господа нашия Бог няма неправда, нито лицеприятие, нито дароприятие.
\par 8 Тогава, като се върнаха в Ерусалим, Иосафат постави и в Ерусалим някои от левитите и свещениците и от началниците на Израилевите бащини домове , за Господния съд и за спорове.
\par 9 И заръча им като им рече: Така да постъпвате със страх от Господа, вярно и със съвършено сърце.
\par 10 И ако дойде при вас от братята ви, които живеят в градовете си, какъвто и да било спор, между кръв и кръв, между закон и заповед, между повеления и узаконения, увещавайте ги да не стават виновни пред Господа, да не би да дойде гняв върху вас и върху братята ви. Така постъпвайте и няма да станете виновни.
\par 11 И, ето, първосвещеникът Амария ще бъде над вас във всяко Господно дело; и началникът на Юдовия дом, Зевадия, Исмаиловия син, ще бъде във всяко царско дело; а левитите ще бъдат надзиратели пред вас. Действувайте мъжествено и Господ ще бъде с добрия.

\chapter{20}

\par 1 След това моавците и амонците, и с тях някои от маонците дойдоха да воюват против Иосафата.
\par 2 Тогава някои дойдоха та известиха на Иосафата, като рекоха: Голямо множество иде против тебе изотвъд Соленото море, от Сирия; и ето ги в Асасон-тамар, (който е Енгади).
\par 3 А Иосафат се уплаши, и предаде се да търси Господа, и прогласи пост по целия Юда.
\par 4 И юдейците се събраха за да искат помощ от Господа; дори от всичките Юдови градове дойдоха да търсят Господа.
\par 5 И Иосафат застана всред събраните юдейци и ерусалимляни в Господния дом, пред новия двор, и рече:
\par 6 Господи Боже на бащите ни, не си ли Ти Бог на небето? и не си ли Ти владетел над всичките царства на народите? И не е ли в Твоята ръка сила и могъщество, така щото никой не може да устои против Тебе?
\par 7 Не беше ли Ти, Боже наш, Който си изпъдил жителите на тая земя пред людете Си Израиля, и дал си я за винаги на потомството на приятеля Си Аврама?
\par 8 И те се заселиха в нея, и построиха Ти светилище в нея за твоето име, и рекоха:
\par 9 Ако ни връхлети зло, и меч, съдба, мор, или глад; и - ние застанем пред тоя дом и пред Тебе, (защото Твоето име е в тоя дом), и извикаме към Тебе в бедствието си, тогава Ти ще ни послушаш и избавиш.
\par 10 И сега, ето амонците и моавците и ония от гората Сиир, които не си оставил Израиля да ги нападне, когато идеха от Египетската земя, но се отклониха от тях и не ги изтребиха, -
\par 11 ето как ни възнаграждават като идат да ни изпъдят от Твоето притежание, което си ни дал да наследим.
\par 12 Боже наш, не щеш ли да го съдиш? Защото в нас няма сила да противостоим на това голямо множество, което иде против, нас, и не знаем що да правим; но към Тебе са очите ни.
\par 13 А целият Юда стоеше пред Господа с челядите си, жените си и чадата си.
\par 14 Тогава всред събранието дойде Господният Дух на левитина Яазиил, Захариевия син, (а Захария бе син на Матания), от Асафовите потомци; и рече:
\par 15 Слушайте, целий Юдо, вие ерусалимски жители, и ти царю Иосафате; така казва Господ на вас: Не бойте се, нито да се уплашите от това голямо множество; защото боят не е ваш, но Божий.
\par 16 Слезте утре против тях; ето, те възлизат, и ще го намерите при края на долината, пред пустинята Еруил.
\par 17 Не ще да е потребно вие да се биете в тоя бой ; поставете се, застанете, и вижте със себе си извършеното от Господа избавление, Юдо и Ерусалиме; на бойте се, нито да се уплашите; утре излезте против тях, защото Господ е с вас.
\par 18 Тогава Иосафат се наведе с лицето до земята; и целият Юда и ерусалимските жители паднаха пред Господа, та се поклониха Господу.
\par 19 И левитите, от потомците на Каатовците и от потомците на Кореевците, станаха да хвалят с много силен глас Господа Израилевия Бог.
\par 20 И тъй на сутринта станаха рано, та излязоха към пустинята Текуе; а когато бяха излезли, Иосафат застана та рече: Слушайте ме, Юдо и вие ерусалимски жители;
\par 21 Тогава, като се съветва с людете, нареди някои от тях да пеят Господу и да хвалят великолепието на Неговата светост като излизат пред войската казвайки: Славословете Господа, защото милостта Му е до века.
\par 22 И когато почнаха да пеят и да хвалят, Господ постави засади против амонците и моавците, и против ония от хълмистата страна Сиир, които бяха дошли против Юда; и те бидоха поразени.
\par 23 Защото амонците и моавците станаха против жителите на хълма Сиир за да го изтребят и заличат; и като довършиха жителите на Сиир, те си помогнаха взаимно да се изтребят.
\par 24 А Юда, като стигна до стражарската кула на пустинята, погледна към множеството и, ето, те бяха мъртви тела паднали на земята, и никой не беше се избавил.
\par 25 И когато Иосафат и людете му дойдоха да ги оберат, намериха между мъртвите им тела твърде много богатство и скъпи вещи и взеха да го отнесат; а три дена обираха, защото користите бяха много.
\par 26 А на четвъртия ден се събраха в Долината на благословението, защото там благословиха Господа; за туй онава място се нарече Долина на благословение, и така са нарича до днес.
\par 27 Тогава всичките Юдови и ерусалимски мъже с Иосафата начело, тръгнаха да се върнат в Ерусалим с веселие, понеже Господ ги беше развеселил с поражението на неприятелите им.
\par 28 И дойдоха в Ерусалим с псалтири и арфи и тръби до Господния дом.
\par 29 И страх от Бога обзе всичките царства на околните страни, когато чуха, че Господ воювал против неприятелите на Израиля.
\par 30 И така царството на Иосафат се успокои; защото неговият Бог му даде спокойствие от всякъде.
\par 31 Така Иосафат царуваше над Юда. Той бе тридесет и пет години на възраст, когато се възцари и царува двадесет и пет години в Ерусалим; а името на майка му беше Азува, дъщеря на Силея.
\par 32 Той ходи в пътя на баща си Аса; не се отклони от него, а вършеше това, което бе право пред Господа.
\par 33 Високите места обаче не се отмахнаха, нито още бяха людете утвърдили сърцата си към Бога на бащите си.
\par 34 А останалите дела на Иосафата, първите и последните, ето, написани са в историята на Ииуя, Ананиевия син, който се споменува в Книгата на Израилевите царе.
\par 35 А след това Юдовият цар Иосафат се сдружи с Израилевия цар Охозия, чиито дела бяха нечестиви.
\par 36 Сдружи се с него за да построят кораби, които да идат в Тарсис; и построиха корабите в Есион-гавер.
\par 37 Тогава Елиезер, Додовият син, от Мариса, пророкува против Иосафата, казвайки: Понеже си се сдружил с Охозия, Господ съсипа твоите дела. И така, корабите се разбиха, тъй че не можеха да идат в Тарсис.

\chapter{21}

\par 1 След това, Иосафат заспа с бащите си и биде погребан с бащите си в Давидовия град; и вместо него се възцари син му Иорам.
\par 2 А Иорам имаше братя, Иосафатови синове, - Азария, Ехиил, Захария, Азария, Михаил и Сафатия; всичките тия бяха синове на Израилевия цар Иосафат.
\par 3 И баща им беше им дал много подаръци, - сребро, злато и скъпоценни неща, заедно с укрепени градове в Юда; но царството беше дал на Иорама, понеже той бе първороден.
\par 4 А Иорам, когато се издигна на бащиното си царство и се закрепи, изби всичките си братя с меч, още и неколцина от Израилевите първенци.
\par 5 Иорам бе тридесет и две години на възраст когато се възцари, и царува осем години в Ерусалим.
\par 6 Той ходи в пътя на Израилевите царе, както постъпваше Ахавовия дом, защото жена му беше Ахавова дъщеря; и върши зло пред Господа.
\par 7 При все това, Господ не иска да изтреби Давидовия дом, заради завета, който бе направил с Давида, и понеже беше обещал, че ща даде светилник нему и на потомците му до века.
\par 8 В неговите дни Едом отстъпи изпод ръката на Юда, и си поставиха свой цар.
\par 9 Затова, Иорам замина с началниците си, и всичките колесници с него; и като стана през нощ порази, които го обкръжаваха, и началниците на колесниците.
\par 10 Обаче, Едом отстъпи из под ръката на Юда, и остава независим до днес. Тогава, в същото време, отстъпи и Ливна изпод ръката му, понеже беше оставил Господа Бога на бащите си.
\par 11 Той построи и високи места по Юдовите планини, и направи ерусалимските жители да блудствуват, и разврати Юда.
\par 12 Тогава дойде до него писмо от пророк Илия, което казваше: Така казва Господ Бог на баща ти Давида: Понеже ти не си ходил в пътищата на баща си Иосафата, нито в пътищата на Юдовия цар Аса,
\par 13 но си ходил в пътя на Израилевите царе, и си направил Юда и ерусалимските жители да блудствуват, както блудствува Ахавовият дом, още си избил и братята си, дома на баща си, които бяха по-добри от тебе,
\par 14 ето, Господ порази с тежък удар людете ти, чадата ти, жените ти и целият ти имот;
\par 15 и ти тежко ще боледуваш от разстройство на червата си, догдето от болестта червата ти почнат да изтичат от ден на ден.
\par 16 Прочее, Господ подигна против Иорама духа на филистимците и на арабите, които са близо да етиопяните;
\par 17 и те възлязоха против Юда, спуснаха се на него, и заграбиха целия имот що се намери в царската къща, още и синовете му и жените му, така щото не му остана син, освен Иоахаз, най младият от синовете му.
\par 18 А подир всичко това Господ го порази с неизцелима болест в червата;
\par 19 и след известно време, на края, след две години, червата му изтекоха поради болестта му; и умря с люти болки. А людете му не гориха аромати за него, както бяха горили за бащите му.
\par 20 Той беше тридесет и две години на възраст, когато се възцари, и царува в Ерусалим осем години; и пресели се неоплакан; и погребаха го в Давидовия град, обаче не в царските гробища.

\chapter{22}

\par 1 Ерусалимските жители направиха цар вместо Иорама , най-младият му син Охозия; защото четите надошли в стана с арабите бяха избили всичките по-стори. Така се възцари Охозия, син на Юдовия цар Иорам.
\par 2 Охозия бе четиридесет и две години на възраст когато се възцари, и царува една година в Ерусалим. Името на майка му бе Готолия, внука на Амрия.
\par 3 Също и той ходи в пътищата на Ахавовия дом; защото майка му беше негова съветница в нечестието му.
\par 4 Той върши зло пред Господа, както Ахавовия дом; защото подир смъртта на баща му, те му станаха съветници, за неговото погубване.
\par 5 По техния съвет като ходеше, той отиде на бой заедно с Израилевия цар Иорам, син на Ахааза, против Сирийския цар Азаил в Рамот-галаад гдето сирийците и раниха Иорама.
\par 6 И той се върна в Езраел за да се цери от раните, които му нанесоха в Рама, когато воюваше против сирийския цар Азаил. Тогава Юдовият цар Охозия Иорамовият син, влезе в Езраел за да види Иорама Ахавовия син, защото бе болен.
\par 7 И загиването на Охозия стана от Бога чрез отиването му при Иорама; защото, когато дойде, излезе с Иорама против Ииуя, Намесиевия син, когото Господ бе помазал да изтреби Ахавовия дом.
\par 8 И Ииуй, когато извършваше съдбата против Ахавовия дом, намери Юдовите първенци и синовете на Охозиевите братя, които служеха на Охозия, и ги изби.
\par 9 Тогава потърси и Охозия; и заловиха го, като бе скрит в Самария, и доведоха го при Ииуя та го убиха; и погребаха го, защото рекоха: Син е на Иосафата, който потърси Господа с цялото си сърце. И Охозиевият дом нямаше вече сила да задържи царството.
\par 10 А Готолия, Охозиевата майка, като видя, че синът и умря, стана та погуби целия царски род от Юдовия дом.
\par 11 Но Иосавеета, царската дъщеря, взе Иоаса, Охозиевият син та го открадна изсред царските синове като ги убиваха, и тури него и доилката му в спалнята. Така Иосавеета, дъщеря на цар Иорама, жена на свещеник Иодай, (защото бе сестра на Охозия), скри го от Готолия, та не го уби.
\par 12 И беше при тях, скрит в Божия дом, шест години; а Готолия царуваше на земята.

\chapter{23}

\par 1 А в седмата година Иодай се усили като взе стотниците Азария, Ероамовия син, Исмаила Иоанановия син, Азария Овидовия син, Маасия Адаиевия син и Елисафата Зехриевия син, та направи завет с тях.
\par 2 И обходиха Юда та събраха левитите от всичките Юдови градове, и началниците на Израилевите бащини домове , и дойдоха в Ерусалим.
\par 3 Тогава цялото събрание направи завет с царя в Божия дом. И Иодай им каза: Ето, царският син ще се възцари, както е говорил Господ за Давидовите потомци.
\par 4 Ето какво трябва да направите: една трета от вас, от свещениците и от левитите, които постъпват на служба в съботата, нека стоят вратари при вратите,
\par 5 една трета в царската къща, и една трета при портата на основата; а всичките люде да бъдат в дворовете на Господния дом.
\par 6 И никой да не влиза в Господния дом, освен свещениците и ония от левитите, които служат; те нека влизат, защото са свети; а всичките люде да пазят заръчаното от Господа,
\par 7 И левитите да окръжават царя, като има всеки оръжията си в ръка; и който би влязъл в дома, да бъде убит; и да бъдете с царя при влизането му и при излизането му.
\par 8 И тъй, левитите и целият Юда извършиха всичко според както заповяда свещеник Иодай, и взеха всеки мъжете си - ония, които щяха да постъпят на служба в събота, и ония, които щяха да оставят службата в съботата; защото свещеник Иодай не разпущаше отредите.
\par 9 И свещеник Иодай даде на стотниците цар Давидовите копия и щитчетата и щитове, които бяха в Божия дом.
\par 10 И постави всичките люде около царя, всеки мъж с оръжията му в ръка, от дясната страна на дома до лявата страна на дома, край олтара и край храма.
\par 11 Тогава изведоха царския син, та положиха на него короната и му връчиха божественото заявление, и направиха го цар. И Иодай и синовете му го помазаха и казаха: Да живее царят!
\par 12 А Годолия, като чу вика от людете, които се стичаха и хвалеха царя, дойде при людете в Господния дом,
\par 13 и погледна, и, ето, царят стоеше при стълба си във входа и военачалниците и тръбите при царя, и всичките люде от страната са радваха и свиреха с тръбите, а певците и ония, които бяха изкусни да псалмопеят, свиреха с музикалните инструменти. Тогава Годолия раздра дрехите си и извика: Заговор! заговор!
\par 14 И свещеник Иодай изведе стотниците, поставени над силите, та им рече: Изведете я вън от редовете, и който би я последвал, да бъде убит с меч; защото свещеникът беше казал: Да не я убиете в Господния дом.
\par 15 И така отстъпиха й място: и когато стигна до входа на царската къща, убиха я там.
\par 16 Тогава Иодай направи завет между себе си и всичките люде и царя, че ще бъдат Господни люде.
\par 17 И всичките люде влязоха във Вааловото капище, та го събориха, жертвениците му и кумирите му изпотрошиха, а Вааловият жрец, Матан, убиха пред жертвениците.
\par 18 И Иодай постави службите на Господния дом в ръцете на левитските свещеници, които Давид беше разпределил из Господния дом, за да принасят Господните всеизгаряния, според както е писано в Моисеевия закон, с веселие и с песни според Давидовата наредба.
\par 19 Постави и вратарите при портите на Господния дом, да не би да влиза някой нечист от какво да е нещо.
\par 20 Тогава, като взе стотниците, високопоставените, началниците на людете, и всичките люде от страната, изведоха царя от Господния дом; и като заминаха през горната порта в царската къща, поставиха царя на царския престол.
\par 21 Така всичките люде от страната се зарадваха, и градът се успокои; а Готолия убиха с меч.

\chapter{24}

\par 1 Иоас беше на седем години, когато се възцари, и царува четиридесет години в Ерусалим; а името на майка му бе Савия, от Вирсавее.
\par 2 И Иоас вършеше това, което бе право пред Господа, през всичките дни на свещеник Иодай.
\par 3 И Иодай му взе две жени; и той роди синове и дъщери.
\par 4 След това, Иоас си науми да обнови Господния дом.
\par 5 И тъй събра свещениците и левитите, та им рече: Излезте по Юдовите градове, та съберете от целия Израил пари за да се поправя дома на вашия Бог от година до година, и гледайте да побързате с работата. Обаче, левитите не побързаха.
\par 6 Тогава царят повика началника на работата Иодай, та му рече: Защо не си изискал от левитите да съберат от Юда и Ерусалим данъка, определен от Господния слуга Моисей да се събира от Израилевото общество, за шатъра на свидетелството?
\par 7 (Защото нечестивата Готолия и синовете й бяха разстроили Божия дом; още и всичките посветени неща в Господния дом бяха посветили на ваалимите).
\par 8 Прочее, по царската заповед направиха един ковчег, който туриха извън, при вратата на Господния дом.
\par 9 И прогласиха в Юда и в Ерусалим да принасят Господу данък наложен върху Израиля от Божия слуга Моисей в пустинята.
\par 10 И всичките първенци и всичките люде се зарадваха и донасяха и туряха в ковчега догде се напълнеше.
\par 11 И когато левитите донасяха ковчега при царските настоятели, и те виждаха, че имаше много пари, царският секретар и настоятелят на първосвещеника дохождаха та изпразваха ковчега, и пак го занасяха и поставяха на мястото му. Така правеха от ден на ден, и събираха много пари.
\par 12 И царят и Иодай ги даваха на ония, които вършеха делото на служенето в Господния дом; и тези наемаха зидари и дърводелци за да обновят Господния дом, още и ковачи и медникари за да поправят Господния дом.
\par 13 И така, работниците вършеха работата, и поправянето успяваше с работенето им, тъй че възстановиха Божия дом в първото му състояние и го закрепиха.
\par 14 И когато свършиха, донесоха останалите пари пред царя и Иодай, и с тях направиха съдове за Господния дом, съдовете за служене и за принасяне жертви , темянници, и други златни и сребърни съдове. Така, непрестанно принасяха всеизгаряния в Господния дом през целия живот на Иодая.
\par 15 Но Иодай остаря и стана сит от дни, и умря; сто и тридесет години бе на възраст когато умря.
\par 16 И погребаха го в Давидовия град между царете, понеже бе извършил добро в Израиля, и пред Бога, и за дома Му.
\par 17 А след смъртта на Иодая, Юдовите началници дойдоха та се поклониха на царя, Тогава царят ги послуша;
\par 18 и те оставиха дома на Господа Бога на бащите си, и служиха на ашерите и на идолите; и гняв дойде върху Юда и Ерусалим поради това тяхно престъпление.
\par 19 При все това, Бог им прати пророци за да ги обърнат към Господа, които заявяваха против тях; но те не послушаха.
\par 20 Тогава Божият Дух дойде на Захария, син на свещеник Иодай, та застана на високо място над людете и рече им: Така казва Бог: Защо престъпвате Господните заповеди? Няма да успеете; понеже вие оставихте Господа, то и Той остави вас.
\par 21 Обаче, те направиха заговор против него и с царска заповед го убиха с камъни в двора на Господния дом.
\par 22 Цар Иоас не си спомни доброто, което му беше показал Иодай, неговият баща, но уби сина му; а той, като умираше рече: Господ да погледни и да издири.
\par 23 И в края на годината, сирийската войска възлезе против Иоаса ; и като дойде в Юда и Ерусалим, изтребиха всичките народни първенци изсред людете, и изпратиха всичките користи, взети от тях, до царя на Дамаск.
\par 24 При все, че сирийската войска, която дойде, беше малолюдна, пак Господ предаде в ръката им едно твърде голямо множество, понеже те бяха оставили Господа Бога на бащите си. Така сирийците извършиха съдба против Иоаса.
\par 25 А като заминаха от него, (и го оставиха страдащ с тежки рани, собствените му слуги направиха заговор против него, поради кръвта на сина на свещеник Иодай, и убиха го на леглото му, та умря; и погребаха го в Давидовия град, но не го погребаха в царските гробища.
\par 26 А ония, които направиха заговор против него, бяха: Завад, син на амонката Симеата и Иозавад, син на моавката Самарита.
\par 27 А колкото за синовете му, и за тежките товари за откуп наложени върху него, и за поправяне на Божия дом, ето, писано е в повестите на Книгата на царете. И вместо него се възцари син му Амасия.

\chapter{25}

\par 1 Амасия бе двадесет и пет години на възраст когато се възцари, и царува двадесет и девет години в Ерусалим. И името на майка му бе Иодана, от Ерусалим.
\par 2 Той върши това, което бе право пред Господа, но не със съвършено сърце.
\par 3 А като му се утвърди царството, той умъртви слугите си, които бяха убили баща му царя.
\par 4 Но чадата им не умъртви, според писаното в закона на Моисеевата книга, гдето Господ заповядвайки, каза: Бащите да не умират поради чадата, нито чадата да умират поради бащите; но всеки да умира за своя грях.
\par 5 Тогава Амасия събра Юда и постави от тях хилядници и стотници, според бащините им домове, по целия Юда и Вениамин; и като ги преброи от двадесет години и нагоре, намери ги триста хиляди отборни мъже, способни да излизат на война, които можеха да държат копие и щит.
\par 6 А от Израиля нае още сто хиляди силни и храбри мъже за сто таланта сребро.
\par 7 А дойде един Божий човек при него и рече: Царю, да не отиде с тебе Израилевата войска; защото Господ не е с Израиля, с никой от ефремците.
\par 8 Но ако си решил да идеш, действувай храбро , укрепи се за бой; иначе , Бог ще те направи да паднеш пред неприятеля, защото Бог има сила да помага и да сваля.
\par 9 А Амасия рече на Божия човек: Но какво да направим за стоте таланта, които дадох на Израилевата войска? А Божият човек отговори: Господ може да ти даде повече от това.
\par 10 Тогава Амасий ги отдели, то ест , войската, която му беше дошла от Ефрема, за да се върнат у дома си; по която причина гневът им пламна силно против Юда, и се върнаха у дома си твърде разярени.
\par 11 А Амасия се одързости, изведе людете си, и отиде в долината на солта, та порази десет хиляди от Сиировите потомци.
\par 12 А други десет хиляди юдейците плениха живи, и като ги закараха на върха на скалата, хвърлиха ги долу от върха на скалата, така че всичките се смазаха.
\par 13 Но мъжете от войската, която Амасия върна надире, за да не идат с него на бой, нападнаха Юдовите градове, от Самария дори до Ветерон, та поразиха от тях три хиляди души, и взеха много користи.
\par 14 А Амасия, като се върна от поразяването на едомците, донесе боговете на Сиировите потомци та ги постави богове за себе си, кланяше се пред тях, и кадеше им тамян.
\par 15 Затова, гневът на Господа пламна против Амасия, и прати при него пророк, който му рече: Защо си прибягнал към боговете на тия люде, които не можеха да избавят своите си люде от ръката ти?
\par 16 А като му говореше той, царят му каза: Съветник ли те поставиха на царя? Престани: защо да бъдеш убит? И пророкът престана, като рече: Зная, че Бог е решил да те изтреби понеже ти стори това, и не послуша моя съвет.
\par 17 Тогава Юдовият цар Амасия, като се съветва, прати до Израилевия цар Иоас, син на Иоахаза, Ииуевия син, да кажат: Дойди да се погледнем един друг в лице.
\par 18 А Израилевият цар Иоас прати до Юдовия цар Амасия да рекат: Ливанският трън пратил до ливанския кедър да кажат: Дай дъщеря си на сина ми за жена. Но един звяр, който бил в ливан, заминал та стъпкал тръна.
\par 19 Ти казваш: Ето, поразих Едома; и сърцето ти те надигна да се хвалиш. Седи сега у дома си; защо да се заплиташ за своята си вреда та да паднеш, ти и Юда с тебе?
\par 20 Но Амасия не послуша; защото от Бога бе това, за да ги предаде в ръката на неприятелите , понеже бяха прибегнали към Едомските богове.
\par 21 Прочее, Израилевият цар Иоас възлезе, та се погледнаха един друг в лице, той и Юдовият цар Амасия, във Ветсемес, който принадлежи на Юда.
\par 22 И Юда биде поразен пред Израиля; и побягнаха всеки в шатъра си.
\par 23 И Израилевият цар Иоас хвана във Ветсемес Юдовия цар Амасия, син на Иоаса, Иоахазовия син и го доведе в Ерусалим, и събори четиристотин лакътя от Ерусалимската стена, от Ефремовата порта до портата на ъгъла.
\par 24 И като взе всичкото злато и сребро, и всичките съдове, които се намираха в Божия дом при Овидедома, и съкровищата на царската къща, тоже и заложници, върна се в Самария.
\par 25 А след смъртта на Израилевия цар Иоас, Иоахазовият син, Юдовият цар Амасия, Иасовият син, живя петнадесет години.
\par 26 А останалите дела на Амасия, първите и последните, ето, написани са в Книгата на Юдовите и Израилевите царе.
\par 27 А след като се отклони Амасия от Господа, направиха заговор против него в Ерусалим, и той побягна в Лахис; но пратиха след него в Лахис та го убиха там.
\par 28 И докараха го на коне та го погребаха с бащите му в Юдовия столичен град .

\chapter{26}

\par 1 А всичките Юдови люде взеха Озия, който бе на шестнадесет години, та го направиха цар вместо баща му Амасия.
\par 2 Той съгради Елот и го възвърна на Юда след като баща му царят заспа с бащите си.
\par 3 Озия беше шестнадесет години на възраст, когато се възцари, и царува петдесет и две години в Ерусалим; и името на майка му бе Ехолия, от Ерусалим.
\par 4 Той върши това, което бе право пред Господа, напълно според както беше сторил баща му Амасия.
\par 5 И търсеше Бог в дните на Захария, който разбираше Божиите видения; и до когато търсеше Господа, Бог му даваше успех.
\par 6 Той излезе та воюва против филистимците, и събори стената на Гет, стената на Явни, и стената на Азот, и съгради градове в азотската околност и между филистимците.
\par 7 И Бог му помогна против филистимците, и против арабите, които живееха в Гур-ваал, и против маонците.
\par 8 И амонците даваха подаръци на Озия; и името му се прочу дори до входа на Египет, защото стана премного силен.
\par 9 Озия съгради и кули в Ерусалим, върху портата на ъгъла върху портата на долината и върху ъгъла на стената , и ги укрепи.
\par 10 Съгради още кули в пустинята, и изкопа много кладенци, защото имаше много добитък и по ниските места и по поляната, имаше и орачи и лозари в планините и на Кармил; защото обичаше земеделието.
\par 11 При това, Озия имаше войска от военни мъже, които излизаха на война по полкове, според числото им, което се преброи от секретаря Еиил и настоятеля Маасия, под ръководството на Анания, един от царските военачалници.
\par 12 Цялото число на началниците на бащините домове , на силните и храбри мъже, беше две хиляди и шестстотин.
\par 13 Под тяхната ръка имаше военна сила от триста и седем хиляди и петстотин души, които се биеха с голямо юначество, за да помагат на царя против неприятелите.
\par 14 Озия приготви за тях, за цялата войска, щитове и копия, шлемове и брони, лъкове и камъни за прашки.
\par 15 И в Ерусалим направи машини, изобретени от изкусни мъже, да бъдат поставени на кулите и на крепостите при ъглите, за хвърляне на стрели и на големи камъни. И името му се прочу на далеч; защото му се помагаше чудно, догдето стана силен.
\par 16 Но, когато стана силен, сърцето му се надигна та се отдаде на поквара; и извърши престъпление против Господа своя Бог, като влезе в Господния храм за да покади върху кадилния олтар.
\par 17 А свещеник Азария влезе подире му, и с него осемдесет Господни свещеници, храбри мъже;
\par 18 и възпротивиха се на цар Озия, и му рекоха: Не принадлежи на тебе, Озие, да кадиш Господу, но на свещениците, Аароновите потомци, които са посветени, за да кадят; излез из светилището, защото си извършил престъпление, което не ще ти бъде за почит от Господа Бога.
\par 19 А Озия, който държеше в ръката си кадилница, за да кади, разяри се; и като се разяри на свещениците, проказата му избухна на челото му пред свещениците в Господния дом, близо при кадилния олтар.
\par 20 И първосвещеник Азария и всичките свещеници погледнаха на него, и, ето, бе прокажен на челото си; и даже сам той побърза да излезе, защото Господ го беше поразил.
\par 21 И цар Озия остана прокажен до деня на смъртта си; и живееше в отделна къща като прокажен, понеже беше отлъчен от Господния дом; а син му Иотам бе над царския дом; а син му Иотам бе над царския дом и съдеше людете на земята.
\par 22 А останалите дела на Озия, първите и последните, написа пророк Исаия, Амосовият син.
\par 23 И Озия заспа с бащите си, и погребаха го с бащите му в оградата на царските гробища, защото рекоха: Прокажен е. И вместо него се възцари син му Иотам.

\chapter{27}

\par 1 Иотам беше двадесет и пет години на възраст когато се възцари, и царува шестнадесет години в Ерусалим; а името на майка му бе Еруса, Садокова дъщеря.
\par 2 Той върши това, което е право пред Господа, съвсем както бе направил баща му Озия; обаче не влезе в Господния храм. Но людете още се покваряваха.
\par 3 Той построи горната порта на Господния дом и съгради много върху стената на Офил.
\par 4 Съгради още и градовете в хълмистата страна на Юда, и в дъбравите съгради крепости и кули.
\par 5 И като воюва против царя на амонците, надделя над тях; и в същата година амонците му дадоха сто таланта сребро, десет хиляди кора жито и десет хиляди кора ечемик. Толкоз му платиха амонците и втората и третата година.
\par 6 Така Иотам стана силен, понеже оправяше пътищата си пред Господа своя Бог.
\par 7 А останалите дела на Иотама, и всичките му войни, и постъпките му, ето, написани са в Книгата на Израилевите и Юдовите царе.
\par 8 Той бе на двадесет и пет години когато се възцари, и царува шестнадесет години в Ерусалим.
\par 9 И Иотам заспа с бащите си; и погребаха го в Давидовия град; и вместо него се възцари син му Ахаз.

\chapter{28}

\par 1 Ахаз бе двадесет години на възраст когато се възцари, и царува шестнадесет години в Ерусалим; но не върши това, което бе право пред Господа, както баща му Давид,
\par 2 но ходи в пътищата на Израилевите царе, направи още и леяни идоли за ваалимите.
\par 3 При това той кади в долината на Еномовия син, и изгори от чадата си в огъня, според мерзостите на народите, които Господ беше изпълнил пред израилтяните.
\par 4 И жертвуваше и кадеше по високите места, по хълмовете и под всяко зелено дърво.
\par 5 Затова Господ неговият Бог го предаде в ръката на сирийския цар; и сирийците го поразиха, и взеха от него голямо множество пленници, които отведоха в Дамаск. Също биде предаден в ръката на Израилевия цар, който му нанесе голямо поражение.
\par 6 Защото Факей, Ромелиевият син, изби от Юда сто и двадесет хиляди души в един ден, всички силни и храбри мъже, понеже бяха оставили Господа Бога на бащите си;
\par 7 и Захарий, един силен мъж от Ефрема, уби царския син Маасия, и надзирателя на двореца Азрикам, и втория подир царя Елкана;
\par 8 тоже израилтяните заплениха от братята си двеста хиляди жени, синове и дъщери, и при това взеха много користи от тях, и отнесоха користите в Самария.
\par 9 А там имаше един Господен пророк на име Одид; и той излезе да посрещне войската, която дохождаше в Самария та им рече: Ето, понеже Господ Бог на бащите ви се разгневи на Юда, предаде ги в ръката ви; и вие ги убихте с ярост, която стигна до небето.
\par 10 И сега възнамерявате да държите юдейците и ерусалимляните като роби и робини под себе си; обаче няма ли у вас, у сами вас, престъпления против Господа вашия Бог?
\par 11 Сега, прочее, послушайте ме, и върнете пленниците, които заробихте от братята си; защото яростен гняв от Господа е върху вас.
\par 12 Тогава някои от първенците от ефремците, Азария Ионановият син, Варахия Месилемотовият син, Езекия Селумовият син и Амаса Адлаевият син, станаха против завърналите се от войната та им рекоха:
\par 13 Не въвеждайте тук пленниците: защото като беззаконствувахме вече против Господа, вие искате да притурите на греховете ни и на престъпленията ни; защото престъплението ни е голямо, и яростен гняв е върху Израиля.
\par 14 Тогава войниците оставиха пленниците и користите пред първенците и пред цялото събрание.
\par 15 И споменатите по име мъже станаха та взеха пленниците и облякоха от користите всичките голи между тях; облякоха ги, обуха ги, дадоха им да ядат и да пият, и помазаха ги, а всичките слаби от тях пренесоха на осли и ги заведоха в Ерихон, града на палмите, при братята им. Тогава се върнаха в Самария.
\par 16 В онова време цар Ахаз прати до асирийските царе да иска помощ,
\par 17 защото едомците бяха пак дошли и поразили Юда и взели пленници.
\par 18 Филистимците тоже бяха нападнали полските градове и южната страна на Юда, и бяха превзели Ветсемес, Еалон, Гедирот, Сокхо и селата му. Тамна и селата му, и Гимзо и селата му, и бяха се заселили в тях.
\par 19 Защото Господ смири Юда преди Израилевия цар Ахаз; понеже той беше развратил Юда и нечествувал много пред Господа.
\par 20 И асирийският цар Теглатфеласар дойде при него: но той го притесни, и не го подкрепи;
\par 21 защото Ахаз, като оголи Господния дом и царската къща и първенците, даде всичкото на асирийския цар; но това не му помогна.
\par 22 И във времето на притеснението си той още повече престъпваше против Господа; такъв бе цар Ахаз.
\par 23 Защото жертвуваше на дамасковите богове, които го бяха поразили, като думаше: Понеже боговете на сирийските царе им помагат, на тях ще принасям жертви, за да помагат и на мене. Но те станаха причина да падне той и целият Израил.
\par 24 И Ахаз събра съдовете на Божия дом, та съсече съдовете на Божия дом, затвори вратите на Господния дом, и си направи жертвеници на всеки ъгъл в Ерусалим.
\par 25 И във всеки Юдов град устрои високи места, за да кади на други богове, и разгневи Господа Бога на бащите си.
\par 26 А останалите негови дела, и всичките му постъпки, първите и последните, ето, написани са в Книгата на Юдовите и Израилевите царе.
\par 27 И Ахаз заспа с бащите си; и погребаха го в града, в Ерусалим, но не го занесоха в гробищата на Израилевите царе. А вместо него се възцари син му Езекия.

\chapter{29}

\par 1 Езекия беше двадесет и пет години на възраст когато се възцари, и царува двадесет и девет години в Ерусалим; а името на майка му беше Авия, Захариева дъщеря.
\par 2 Той върши това, което бе право пред Господа, напълно както извърши баща му Давид.
\par 3 В първия месец от първата година на царуването си той отвори вратата на Господния дом и ги поправи.
\par 4 И като доведе свещениците и левитите, събра ги на източния площад и рече им:
\par 5 Слушайте ме, левити, осветете се сега, осветете и храма на Господа Бога на бащите си, и изнесете нечистотата от светото място.
\par 6 Защото бащите ни са отстъпили, сторили са зло пред Господа нашия Бог, оставили са Го и са отвърнали лицата си от Господното обиталище, и са обърнали гръб;
\par 7 те са затворили вратите на предхрамието и са изгасили светилниците, а не са кадили темян нито пренасяли всеизгаряния в светото място на Израилевия Бог.
\par 8 Затова гняв от Господа е бил върху Юда и Ерусалим, и Той ги предаде да бъдат тласкани, и да бъдат предмет на учудване и на съскане, както виждате с очите си.
\par 9 Защото, ето, бащите ни са паднали от нож, а синовете ни, дъщерите ни и жените ни са поради това в плен.
\par 10 Сега, прочее, имам сърдечно желание да направим завет с Господа Израилевия Бог, за да отвърне от нас яростния Си гняв.
\par 11 Чада мои, не бивайте сега небрежни; защото вас е избрал Господ да стоите пред Него, да му служите, да сте пред Него, да му служите, да сте Му служители и да кадите.
\par 12 Тогава станаха левитите: Маат Амасаевият син и Иоил Азариевият син, от Каатовите потомци; а от Мерариевите потомци, Кис Авдиевият син и Азария Ялелеиловият син; от Герсоновците, Иоах Земовият син и Един Иоаховият син;
\par 13 от Елисафановите потомци, Симрий и Еиил; от Асафовите потомци, Захария и Матания;
\par 14 от Емановите потомци, Ехиил и Семей; а от Едутуновие потомци, Семаия и Озиил;
\par 15 и те, като събраха братята си та се осветиха, влязоха според царската заповед, съгласно Господното слово, да очистят Господния дом.
\par 16 Свещениците влязоха в по-вътрешната част на Господния дом, за да я очистят, и изнесоха в двора на Господния дим всичката нечистота, която намериха в Господния храм; а левитите взеха та я изнесоха вън в потока Кедрон.
\par 17 На първия ден от първия месец почнаха да освещават, а на осмия ден от месеца стигнаха до Господния трем; и за осем дни осветиха Господния дом, и на шестнадесетия ден от първия месец свършиха.
\par 18 Тогава влязоха в двореца при цар Езекия и рекоха: Очистихме изцяло Господния дом, олтара за всеизгарянето с всичките му прибори и трапезата за присъствените хлябове с всичките й прибори;
\par 19 при това, приготвихме и осветихме всичките вещи, които цар Ахаз, в царуването си, оскверни когато отстъпи: и ето ги пред Господния олтар.
\par 20 Тогава цар Езекия стана рано, и като събра градските първенци, възлезе в Господния дом.
\par 21 И докараха седем юнеца, седем овена, седем агнета и седем козела в принос за грях за царството, за светилището и за Юда; и той заповяда на свещениците, Аароновите потомци, да ги принесат на Господния олтар.
\par 22 И тъй, те заклаха юнците, и свещениците, като взеха кръвта, попръскаха я по олтара, заклаха и овните и попръскаха кравта им по олтара; тоже и агнетата заклаха и попръскаха кръвта им по олтара.
\par 23 Сетне преведоха козлите, които бяха в принос за грях, пред царя и събранието; и те положиха ръцете си на тях;
\par 24 тогава свещениците ги заклаха и попръскаха кръвта им по олтара в принос за грях, за да направят умилостивение за целия Израил; защото царят заповяда, щото всеизгарянето и приноса за грях да станат за целия Израил.
\par 25 И той постави левитите в Господния дом с кимвали, с псалтири и с арфи, според заповедта на Давида и на царевия гледач Гад и на пророк Натана; защото заповедта бе от Господа чрез пророците Му.
\par 26 И левитите стояха с Давидовите музикални инструменти, а свещениците с тръбите.
\par 27 Тогава Езекия заповяда да принесат всеизгарянето на олтара. И когато почна всеизгарянето, почна и пеенето Господу с тръбите и с инструментите на Израилевия цар Давида.
\par 28 А цялото събрание се кланяше, и певците пееха, и тръбите свиреха непрекъснато докле се извърши всеизгарянето.
\par 29 И като свършиха принасянето царят и всички, които присъствуваха с него, коленичиха и се поклониха.
\par 30 Тогава цар Езекия и първенците заповядаха на левитите да славословят Господа с думите на Давида и на гледача Асаф. И те пееха с веселие; и наведоха се та се поклониха.
\par 31 След това, Езекия проговаряйки, каза: Сега, като сте се осветили Господу, пристъпете та принесете жертви и благодарителни приноси в Господния дом. И така, обществото принесе жертви и благодарителни приноси; и всеки, който имаше сърдечно съизволение, принесе и всеизгаряне.
\par 32 И числото на всеизгарянията, които обществото принесе, възлезе на седемдесет юнеца, сто овена и двеста агнета; всички тия бяха за всеизгаряния Господу;
\par 33 а посветените животни бяха шестстотин говеда и три хиляди овце.
\par 34 Свещениците, обаче, бяха малцина, и не можеха да одерат всичките всеизгаряния; затова, братята им, левитите им, помогнаха докле се свърши работата и докле се осветиха свещениците; защото левитите показаха повече сърдечна правота да се освещават отколкото свещениците.
\par 35 А пък всеизгарянията бяха много, заедно с тлъстината на примирителните приноси и с всеизгарянията на всяко всеизгаряне. Така се възстанови службата на Господния дом.
\par 36 И Езекия и всичките люде се радваха за гдето Бог беше предразположил людете; понеже това нещо стана ненадейно.

\chapter{30}

\par 1 След това Езекия прати човеци по целия Израил и Юда, писа още и писма на Ефрема и Манасия, да ги поканят да дойдат на Господния дом в Ерусалим, за да направят пасха на Господа Израилевия Бог.
\par 2 Защото царят и първенците му и цялото общество в Ерусалим бяха се съветвали да направят пасхата във втория месец.
\par 3 Понеже не бяха могли да я направят в онова време, защото нямаше доволно осветени свещеници, и людете не бяха се събрали в Ерусалим.
\par 4 И това нещо се видя право на царя и на цялото общество.
\par 5 Затова решиха да прогласят по целия Израил, от Вирсавее дори до Дан, покана да дойдат за да направят пасха на Господа Израилевия Бог в Ерусалим; защото отдавна не бяха я празнували според предписанието.
\par 6 И тъй бързоходците отидоха с писмата от царя и първенците му по целия Израил и Юда, според царската заповед, и казаха: Чада на Израиля, обърнете се към Господа Бога на Авраама, на Исаака и на Израиля, за да се обърне Той към останалите от вас, които се избавихте от ръката на асирийските царе.
\par 7 Не бъдете както бащите ви и както братята ви, които престъпваха против Господа Бога на бащите си, така щото Той ги предаде на опустошение, както виждате.
\par 8 Сега не закоравявайте врата си, както сториха бащите ви, но предайте се Господу и влезте в светилището Му, което е осветил за винаги, и служете на Господа вашия Бог, за да отвърне от вас яростния Си гняв.
\par 9 Защото, ако се обърне, към Господа, то на братята ви и чадата ви ще се покаже милост от тия, които са ги пленили, и те ще се върнат в тая земя; защото щедър и милостив е Господ вашият Бог, и няма да отвърне лицето Си от вас, ако вие се обърнете към Него.
\par 10 Така бързоходците минаха от град на град през Ефремовата и Манасиевата земя дори до Завулон; но те им се присмяха и подиграха се с тях.
\par 11 Някои обаче от Асира, Манасия и Завулона се смириха и дойдоха в Ерусалим.
\par 12 А и над Юда бе Божията ръка, за да им даде едно сърце да постъпят по заповедта на царя и на първенците според Господното слово.
\par 13 И събраха се в Ерусалим много народ, за да пазят празника на безквасните хлябове във втория месец; беше твърде голямо събрание.
\par 14 И станаха та отметнаха жертвениците, които бяха в Ерусалим, отмахнаха и всичките кадилни олтари, та ги хвърлиха в потока Кедрон.
\par 15 Тогава заклаха пасхалните агнета в четиринадесетия ден от втория месец; и свещениците и левитите, засрамени се осветиха, и внесоха всеизгаряния в Господния дом.
\par 16 И застанаха на мястото си, според чина си, съгласно закона на Божия човек Моисей; и свещениците пръскаха кръвта, която я вземаха от ръката на левитите.
\par 17 Защото имаше мнозина в събранието, които не бяха се осветили; за това левитите взеха грижата да заколят пасхалните агнета , за да се осветят Господу.
\par 18 Понеже голяма част от людете, мнозина от Ефрема, Манасия, Исахара и Завулона, не бяха се очистили, но ядоха пасхата не според предписанието; защото Езекия бе се помолил за тях, казвайки: Благий Господ да бъде милостив към всекиго,
\par 19 който утвърждава сърцето си да търси Бога, Господа Бога на бащите си, ако и да не е очистен според очистването изискано за светилището.
\par 20 И Господ послуша Езекия та прости людете.
\par 21 И израилтяните, които се намираха в Ерусалим, пазиха празника на безквасните хлябове седем дни с голямо веселие; и всеки ден левитите и свещениците славословеха Господа, пеещи с музикални инструменти на Господа.
\par 22 И Езекия говори насърчително на всичките левити, които разбираха добре Господната служба . Така през седемте празнични дни те ядяха, като жертвуваха примирителни жертви и славословеха Господа Бога на бащите си.
\par 23 Тогава цялото общество се съветва да празнуват още седем дни; и празнуваха още седем дни с веселие.
\par 24 Защото Юдовият цар Езекия подари на обществото за жертви хиляда юнеца и седем хиляди овце; и първенците подариха на обществото хиляда юнеца и десет хиляди овце; и много свещеници осветиха себе си.
\par 25 И развеселиха се всички събрани от Юда, със свещениците и левитите, и всички събрани надошли от Израиля, и чужденците надошли от Израилевата земя, които живееха в Юда.
\par 26 Така стана голямо веселие в Ерусалим; защото от времето на Израилевия цар Соломона, Давидовия син, не бе станало такова нещо в Ерусалим.
\par 27 След това левитите, свещениците, станаха та благословиха людете; и гласът им биде послушан от Господа , и молитвата им възлезе на небето. Неговото свето обиталище.

\chapter{31}

\par 1 А като се свърши всичко това, целият Израил, които се намираха там, излязоха по Юдовите градове та изпотрошиха кумирите, изсякоха ашерите и събориха високите места и жертвениците из целия Юда и Вениамина, също и в Ефрем и Манасия, докле ги унищожиха всички. Тогава всичките израилтяни се върнаха в градовете си, всеки в своята собственост.
\par 2 Тогава Езекия определи отредите на свещениците и на левитите, според разпределенията им, всекиго според службата му, и свещениците и левитите, за всеизгарянията и примирителните приноси, за да служат и да славословят и да хвалят в портите на Господния стан.
\par 3 Определи и царския дял от имота си за всеизгарянията, а именно , за утринните и вечерните всеизгаряния, и за всеизгарянията в съботите и новолунията и на празниците, според предписанието в Господния закон.
\par 4 Заповяда още на людете, които жевееха в Ерусалим, да дават дела на свещениците и на левитите, за да се затвърдят в Господния закон.
\par 5 И щом се издаде тая заповед, израилтяните донесоха много от първите плодове на житото, виното, дървеното масло, меда и на всичките земни произведения; донесоха още в изобилие и десетъците от всяко нещо.
\par 6 И израилтяните и юдейците, които живееха в Юдовите градове, също е те донесоха десетъците от говеда и овци, и десетъците от светите неща, които бяха посветени на Господа техния Бог, и туриха ги на купове.
\par 7 В третия месец почнаха да правят куповете, и в седмия месец свършиха.
\par 8 И когато дойдоха Езекия и първенците и видяха куповете, благословиха Господа и Неговите люде Израиля.
\par 9 Сетне Езекия запита свещениците и левитите за куповете.
\par 10 И първосвещеник Азария, от Садоковия дом, в отговор му рече: Откак почнаха да донасят приносите в Господния дом, яли сме до насита, и остана много; защото Господ е благословил людете Си; и останалото е тоя голям склад.
\par 11 Тогава Езекия заповяда да приготвят помещения в Господния дом; и като ги приготвиха,
\par 12 в тях внесоха вярно приносите, десетъците и посветените неща; а надзирател над тях бе левитинът Хонения, и подир него брат му Семей.
\par 13 А по заповед на цар Езекия и на настоятеля на Божия дом Азария, Ехиил, Азазия, Нахат, Асаил, Еримот, Иозавад, Елиил, Исмахия, Маат и Ванаия бяха надзиратели под властта на Хонения и брат му Семей.
\par 14 А Коре, син на левитина Емна, вратарят при източната порта , бе над нещата доброволно принесени Богу, за да раздава Господните приноси и пресветите неща.
\par 15 А под него бяха, в градовете на свещениците, Еден, Миниамин, Исус, Семаия, Амария и Сехания, на които бе поръчано да раздават на братята си според отредите им , както на големия, тъй и на малкия,
\par 16 на всекиго, който постъпваше в Господния дом, ежедневния му дял за служенето им в службата, според отредите им, освен ония, които се преброиха според родословието на мъжките от три години на възраст и нагоре;
\par 17 да раздават и на преброените от свещениците и от левитите, според бащините им домове, на възраст от двадесет години и нагоре, според службите им, по отредите им,
\par 18 и на всичките им челяди, на жените им, и на синовете им, и на дъщерите им, в цялото общество, които бяха преброени по родословие; защото те се посветиха вярно на светите неща.
\par 19 А колкото за Аароновите потомци, свещениците, които живееха в полетата на градските пасбища, имаше във всеки град човеци определени по име да раздават дялове на всичките мъжки между свещениците, и на всичките преброени между левитите.
\par 20 Така направи Езекия и в целия Юда; и върши това, което бе добро, и право, и вярно пред Господа своя Бог.
\par 21 Във всяка работа, която почна, относно служенето на Божия дом, и закона, и заповедите, като търсеше своя Бог, вършеше я от все сърце, и успяваше.

\chapter{32}

\par 1 След тия неща и това явление на вярност, асирийският цар Сенахирим дойде та влезе в Юда, и разположи стана си против укрепените градове, като намисли да ги усвои.
\par 2 А Езекия, като видя, че Сенахирим дойде и че намерението му бе да воюва против Ерусалим,
\par 3 съветва се с първенците си и със силните си мъже да запълни водните извори, които бяха вън от града; и те му помогнаха.
\par 4 И като се събраха много люде, запълниха всичките извори и потока, който тече сред земята, защото казваха: Асирийските царе, като дойдат, защо да намерят много вода?
\par 5 И Езекия се ободри и съгради пак цялата стена, която бе съборена, направи кулите й по-високи, съгради и другата стена извън, и поправи Мило в Давидовия град, и направи много копия и щитове.
\par 6 И постави военачалници над людете и като ги събра пре себе си на площада, при градската порта, говори им насърчително, казвайки:
\par 7 Бъдете силни и храбри, не бойте се, нито се уплашвайте от асирийския цар, нито от голямото множество, което е с него; защото с нас има Един по-велик отколкото има с него.
\par 8 С него са плътски мишци; с нас е Господ нашият Бог, да ни помогне и да воюва в боевете ни. И людете се успокоиха от думите на Юдовия цар Езекия.
\par 9 След това, асирийският цар Сенахирим прати слугите си в Ерусалим (а той и всичките му военачалници с него обсаждаха Лахис) до Юдовия цар Езекия и до целия Юда, който бе в Ерусалим, да рекат:
\par 10 Така казва асирийският цар Сенахирим: На що се надявате та чакате в Ерусалим обсадата му?
\par 11 Не мами ли ви Езекия, та ще ви предаде на смърт от глад и от жажда, като казва: Господ нашият Бог ще ни избави от ръката на асирийския цар?
\par 12 Езекия не е ли същият, който махна Неговите високи места и Неговите жертвеници, и заповяда на Юда и на Ерусалим, като каза: Само пред един олтар да се кланяте, и върху него да кадите?
\par 13 Не знаете ли що сторих аз и бащите ми на всичките племена на земите? Можаха ли боговете на народите на тия земи да избавят някак земята си от ръката ми?
\par 14 Кой от всичките богове на ония народи, които бащите ми изтребиха и могъл да избави людете си от ръката ми, та да може вашият Бог да ви избави от ръката ми?
\par 15 Сега, прочее, да ви не мами Езекия, и да ви не убеждава по тоя начин, и не го вярвайте; защото никой бог, на кой да било народ или царство, не е могъл да избави людете си от моята ръка и от ръката на бащите ми, та вашият ли Бог ще може да ви избави от ръката ми?
\par 16 И слугите му говориха още повече против Господа Бога и против слугата Му Езекия.
\par 17 Той писа и писма да хули Господа Израилевия Бог, и да говори против Него, като казваше: Както боговете на народите на тия земи не избавиха своите люде от ръката ми, така и Езекиевият Бог няма да избави своите люде от ръката ми.
\par 18 Тогава извикаха на Юдейски със силен глас към ерусалимските люде, които бяха на стената, за да ги уплашат и да ги смутят, та да превземат града;
\par 19 и говориха за ерусалимския Бог, както за боговете на племената на света, които са дело на човешки ръце.
\par 20 Затова цар Езекия и пророк Исаия, Амосовият син, се помолиха и викаха към небето.
\par 21 И Господ прати ангела, който погуби всичките силни и храбри мъже, и първенците, и военачалниците в стана на асирийския цар. Така той се върна с посрамено лице в земята си. и когато влезе в капището на бога си, тия, които бяха излезли из чреслата му, го убиха там с нож.
\par 22 Така Господ избави Езекия и ерусалимските жители от ръката на асирийския цар Сенахирим, и от ръката на всички други , и ги ръководеше на всяка страна.
\par 23 И мнозина донесоха дарове Господу в Ерусалим, и скъпоценности на Юдовия цар Езекия; така че от тогава нататък той се възвеличаваше пред всичките народи.
\par 24 В това време Езекия се разболя до смърт: и като се помоли Господу, Той му говори и му даде знамение.
\par 25 Но Езекия не отдаде Господу според стореното нему благоволение, защото сърцето му се надигна; за това гняв падна на него и на Юда и Ерусалим.
\par 26 Обаче, Езекия се смири поради надигането на сърцето си, той и ерусалимските жители, тъй че Господният гняв не дойде на тях в Езекиевите дни.
\par 27 И Езекия придоби много голямо богатство и слава; и направи съкровищници за сребро и злато, за скъпоценни камъни, за аромати, за щитове и за всякакви отборни вещи,
\par 28 също и житници за произведението на житото, на виното и на дървеното масло, и обори за всякакви животни и огради за стада.
\par 29 При това, той си направи градове, и придоби множество овце и говеда; защото Бог му даваше твърде много имот.
\par 30 Същият тоя Езекия запълни и горния извор на водата та Гион, и я доведе направо долу на запад от Давидовия град. И Езекия успя във всичките си дела.
\par 31 Но относно посланиците, които вавилонските първенци пратиха до него да разпитват за знамението станало в страната, Бог го остави, за да го изпита и да узнае всичко що беше на сърцето му.
\par 32 А останалите дела на Езекия, и добрините му, ето, написани са във Видението на пророк Исаия, Амосовия син, в Книгата на Юдовите и Израилевите царе.
\par 33 И Езекия заспа с бащите си, и погребаха го в най-видния от гробовете на Давидовите потомци; и целият Юда и ерусалимските жители му отдадоха почест при смъртта му. И вместо него се възцари син му Манасия.

\chapter{33}

\par 1 Манасия бе двадесет години на възраст когато се възцари и царува петдесет и пет години в Ерусалим.
\par 2 Той върши зло пред Господа, според мерзостите на народите, които Господ изпъди пред израилтяните.
\par 3 Устрои изново високите места, които баща му Езекия беше съборил, издигна жертвеници на ваалимите и направи ашери, и се кланяше на цялото небесно множество и им служеше.
\par 4 Тоже издигна жертвеници в Господния дом, за който Господ беше казал: В Ерусалим ще бъде името Ми до века.
\par 5 Издигна жертвеници и на цялото небесно множество вътре в двата двора на Господния дом.
\par 6 И преведе чадата си през огъня в долината на Еномовия син, още упражняваше предвещания и употребяваше гадания, упражняваше омайвания, и си служеше със запитвачи на зли духове и с врачове; той извърши много зла пред Господа та Го разгневи.
\par 7 И ваяния идол, образа, който направи, постави в Божия дом, за който Бог каза на Давида и на сина му Соломона: В тоя дом, и в Ерусалим, който избрах от всичките Израилеви племена, ще настаня името Си до века;
\par 8 нито ще местя вече ногата на Израиля от земята, която определих за бащите ви, само ако внимават да вършат всичко, що им заповядах, целия закон и повеленията и съдбите дадени чрез Моисея.
\par 9 Но Манасия подмами Юда и ерусалимските жители да вършат по-лошо от народите, които Господ изтреби пред израилтяните.
\par 10 Тогава Господ говори на Манасия и на людете му; но те не послушаха.
\par 11 Затова Господ доведе против тях военачалниците на асирийския цар, та хванаха Манасия, и като го туриха в окови и вързаха го с вериги, заведоха го във Вавилон.
\par 12 А когато беше в бедствие, помоли се на Господа своя Бог, и смири се много пред Бога на бащите си.
\par 13 И когато му се помоли, Бог даде внимание на него, и послуша молбата му, та го доведе пак в Ерусалим, на царството му. Тогава Манасия позна, че Господ - Той е Бог.
\par 14 А след това той съгради една външна стена на Давидовия град, на запад от Гион, в долината, дори до входа на рибната порта, и прекара я около Офир, та я издигна много високо; и постави военачалници във всичките укрепени Юдови градове.
\par 15 И отмахна чуждите богове, и идола от Господния дом, и всичките жертвеници, които бе издигнал върху хълма на Господния дом и в Ерусалим, и ги хвърли вън от града.
\par 16 И поправи Господния олтар, и пожертвува на него примирителни и благодарствени жертви, и заповяда на Юда да служи на Господа Израилевия Бог.
\par 17 Но людете още жертвуваха по високите места, обаче, само на Господа своя Бог.
\par 18 А останалите дела на Манасия, и молитвата му към неговия Бог, и думите на гледача, които му говориха в името на Господа Израилевия Бог, ето, написани са между делата на Израилевите царе.
\par 19 И молитвите му, и как Бог го послуша, и всичките му грехове, и престъплението му, местностите, гдето устрои високите места и постави ашерите и леяните идоли, преди да се смири, ето, написани са в историята на гледачите.
\par 20 И Манасия заспа с бащите си; и погребаха го в собствената му къща; а вместо него се възцари, и царува две години в Ерусалим.
\par 21 Амон бе двадесет и две години на възраст когато се възцари, и царува две години в Ерусалим.
\par 22 Той върши зло пред Господа, както стори баща му Манасия; и Амон жертвуваше на всичките леяни идоли, които баща му Манасия бе направил, и им служеше.
\par 23 А не се смири пред Господа, както се смири баща му Манасия; но тоя Амон вършеше все по-много и по-много престъпления.
\par 24 А слугите му направиха заговор против него, та го убиха в собствената му къща.
\par 25 Обаче, людете от страната избиха всичките, които бяха направили заговора против цар Амона; и людете от страната направиха сина му Иосия цар вместо него.

\chapter{34}

\par 1 Иосия бе осем години на възраст когато се възцари, и царува в Ерусалим тридесет и една година.
\par 2 Той върши това, което бе право пред Господа, като ходи в пътищата на баща си Давида, без да се отклони на дясно или на ляво.
\par 3 В осмата година на царуването си, като бе още млад, почна да търси Бога на баща си Давида; а в дванадесетата година почна да чисти Юда и Ерусалим от високите места, от ашерите, от ваяните идоли, от леяните.
\par 4 В неговото присъствие събориха жертвениците на ваалимите; и изсече кумирите на слънцето, които бяха върху тях и сломи ашерите и ваяните и леянити идоли, и като ги стри на прах, разпръсна го върху гробовете на ония, които бяха им жертвували.
\par 5 Изгори и костите на жреците на жертвениците им, и така очисти Юда и Ерусалим.
\par 6 Същото направи и в градовете на Манасия, Ефрема и Симеона, и дори до Нефталима, всред околните им развалини;
\par 7 събори жертвениците, стри на прах ашерите и ваяните идоли, и изсече всичките кумири на слънцето по цялата Израилева земя. Тогава се върна в Ерусалим.
\par 8 А в осемнадесетата година на царуването си, когато беше очистил земята и Божия дом, прати Сафана, Азалиевия син, и градския началник Маасия, и летописеца Иоах, син на Иоахаза, за да поправят дома на Господа неговия Бог.
\par 9 И те отидоха при първосвещеник Хелкия, та предадоха парите внесени в Божия дом, които вратарите левити бяха събрали от Манасия и Ефрема, и от всички останали от Израил, и от целия Юда и Вениамина, и от жителите на Ерусалим.
\par 10 Предадоха ги в ръката на работниците, които надзираваха Господния дом; а те ги дадоха на работниците, които работеха в Господния дом, за да поправят и обновят дома, -
\par 11 дадоха ги на дърводелците и на зидарите, за да купят дялани камъни и дървета за греди, и да поправят зданията, които Юдовите царе бяха съборили.
\par 12 И мъжете вършеха работата честно; а надзирателите над тях бяха левитите Яат и Авдия от Мерариевите потомци, Захария и Месулам от Каатовите потомци, за да настояват, а от другите левити всички изкусни свирачи на музикални инструменти.
\par 13 Имаше още настойници над бременосците и над всички, които работеха в каква да било работа; а някои от левитите бяха писари, надзиратели и вратари.
\par 14 А като изнасяха внесените в Господния дом пари, свещеник Хелкия намери книгата на Господния закон даден чрез Моисея.
\par 15 И Хелкия проговаряйки рече на секретаря Сафан: Намериха книгата на закона в Господния дом. И Хелкия даде книгата на Сафана.
\par 16 А Сафан донесе книгата на царя, занесе и дума на царя, казвайки: Слугите ти вършат всичко, що им е определено;
\par 17 и събраха намерените в Господния дом пари, и ги предадоха в ръката на настоятелите и в ръката на работниците.
\par 18 Тоже секретарят Сафан извести на царя, казвайки: Свещеник Хелкия ми даде една книга. И Сафан я прочете пред царя.
\par 19 А царят, като чу думите на закона, раздра дрехите си.
\par 20 И царят заповяда на Хелкия, на Ахикама Сафановия син, на Авдона, Михеевия син, на секретаря Сафан и на царевия слуга Асаия, казвайки:
\par 21 Идете, допитайте се до Господа за мене и за останалите в Израиля и в Юда, относно думите на намерената книга; защото голям е Господният гняв, който се изля на нас, понеже бащите ни не опазиха Господното слово да постъпват напълно според както е писано в тая книга.
\par 22 И така, Хелкия и ония, които царят беше определил , отидоха при пророчицата Олда, жена на одеждопазителя Селум, син на Текуя, Арасовия син. А тя живееше в Ерусалим, във втория участък; и те й говориха според поръчаното.
\par 23 И тя им рече: Така казва Господ Израилевият Бог: Кажете на човека, който ви е пратил до мене:
\par 24 Така казва Господ: Ето, Аз ще докарам зло на това място и на жителите му, всичките проклетии написани в книгата, която прочетоха пред Юдовия цар.
\par 25 Понеже Ме оставиха и кадяха на други богове, та Ме разгневиха с всичките си дела на ръцете си, затова гневът Ми ще се излее на това място, и няма да угасне.
\par 26 Но на Юдовия цар, който ви прати да се допитате до Господа, така да му кажете: Така казва Господ Израилевият Бог: Относно думите, които ти си чул,
\par 27 понеже сърцето ти е омекнало, и ти си се смирил пред Бога, когато си чул Неговите думи против това място и против жителите му, и ти си се смирил пред Мене, раздрал си дрехите си, и си плакал пред Мене, затова и Аз те послушах, казва Господ.
\par 28 Ето, Аз ще те прибера при бащите ти, и ще се прибереш в гроба си с мир; и твоите очи няма да видят нищо от цялото зло, което ще докарам на това място и на жителите му. И те доложиха на царя.
\par 29 Тогава царят прати та събра всичките Юдови и ерусалимски старейшини.
\par 30 И царят възлезе в Господния дом, заедно с всичките Юдови мъже и ерусалимските жители, - свещениците, левитите и всичките люде от голям до малък; и прочете на всеослушание пред тях всичките думи от книгата на завета, която се намери в Господния дим.
\par 31 И царят застана на мястото си, та направи завет пред Господа да следва Господа, да пази заповедите Му, заявленията Му и повеленията Му с цялото си сърце и с цялата си душа, та да изпълнява думите на завета, които са написани в тая книга.
\par 32 И накара всички, които се намериха в Ерусалим и във Вениамин, да потвърдят завета . И ерусалимските жители постъпиха според завета на Бога, Бога на бащите си.
\par 33 И Иосия отмахна всичките мерзости от всичките места принадлежащи на израилтяните, и накара всички, които се намериха в Израил, да служат, да служат на Господа своя Бог; през всичките му дни те не се отклониха от следване Господа Бога на бащите си.

\chapter{35}

\par 1 При това, Иосия направи пасха Господу в Ерусалим; и на четиринадесетия ден от първия месец заклаха пасхалните агнета .
\par 2 И той постави свещениците в длъжностите им, и ги насърчи да служат на Господния дом.
\par 3 И рече на осветените Господу левити, които поучаваха целия Израил: Положете светия ковчег в дома, който построи Израилевият цар Соломон, Давидовият син; няма да го носите вече на рамена; сега слугувайте на Господа вашия Бог и на людете Му Израил.
\par 4 Пригответе се според бещините си домове, по отредите си, според предписанието на Израилевия цар Давида, и според предписанието на сина му Соломона;
\par 5 и стойте в светилището, според отделенията на бащините домове на братята си миряните, и за всяко нека има по един отдел от левитските бащини домове.
\par 6 И като заколите пасхалните агнета , осветете се, и пригответе за братята си, за да направят и те според Господното слово дадено чрез Моисея.
\par 7 И Иосия подари на людете за жертви овце, агнета и ярета, на брой тридесет хиляди, всичките за пасхалните жертви, за всички, които се намериха там, и три хиляди говеда; тия бяха от царския имот.
\par 8 И първенците му подариха на драго сърце на людете, на свещениците и на левитите. А управителите на Божия дом, Хелкия, Захария и Ехиил, подариха на свещениците за пасхалните жертви две хиляди и шестстотин агнета и ярета и триста говеда.
\par 9 Също и Хонения и братята му Семаия, и Натанаил, и началниците на левитите, Асавия, Еиил и Иозавад, подариха на левитите за пасхалните жертви пет хиляди агнета и ярета и петстотин говеда.
\par 10 Така се приготви службата. Тогава свещениците застанаха на мястото си, и левитите по отредите си, според царската заповед.
\par 11 И заклаха пасхалните агнета , и свещениците поръсиха кръвта, която вземаха от ръката на левитите, които и одраха жертвите .
\par 12 И отделиха частите за всеизгарянето, за да ги дадат според отделенията на бащините домове на людете, за да принесат Господу според предписанията в Моисеевата книга; така сториха и с говедата.
\par 13 И опекоха пасхалните агнета с огън според наредбата; а светите жертви свариха в котли, в гърнета и в тави, и ги разделиха веднага помежду всичките люде.
\par 14 После приготвиха за себе си и за свещениците; защото свещениците, Аароновите потомци, се занимаваха с принасяне всеизгарянията и тлъстините до нощта; затова левитите приготвиха за себе си и за свещениците, Аароновите потомци.
\par 15 И певците, Асафовите потомци, бяха на мястото си, според заповедта на Давида, на Асафа, на Емана и на царския гледач Едутун; и вратарите пазеха на всяка порта; нямаше нужда да оставят работата си, защото братята им, левитите, приготвиха за тях.
\par 16 И тъй, цялата Господна служба биде приготвена в същия ден, за да направят пасхата и да принесат всеизгарянията на Господния олтар, според заповедта на цар Иосия.
\par 17 И израилтяните, които се намериха там, правиха в онова време пасхата и празника на безквасните хлябове седем дни.
\par 18 Такава пасха не беше ставала в Израил от дните на пророк Самуила; нито един от всичките Израилеви царе не беше направил такава пасха, каквато направиха Иосия, и свещениците, и левитите, и целият Юда и Израил, които се намериха там, и ерусалимските жители.
\par 19 Тая пасха стана в осемнадесетата година на Иосиевото царуване.
\par 20 А след всичко това, когато Иосия беше наредил Божия дом, египетския цар Нехао възлезе за да воюва против Кархамис при реката Евфрат; и Иосия излезе против него.
\par 21 А той прати посланици до него да кажат: Какво има между мене и тебе, царю Юдов? Не ида днес против тебе, но против дома, с който имам война, и Бог ми е заповядал да побързам; остави се от Бога, който е с мене, да не би да те изтреби.
\par 22 Обаче Иосия не отвърна лицето си от него, но преличи се да воюва против него, и не послуша думите на Нехао, които бяха из Божиите уста, и дойде да се бие в долината Магедон.
\par 23 И стрелците устрелиха цар Иосия; и царят рече на слугите си: Изведете ме от тук, защото съм тежко ранен.
\par 24 И тъй, слугите му го изведоха из колесницата, та го качиха на втората му колесница, и докараха го в Ерусалим; и умря, и биде погребан в гробищата на бащите си. И целият Юда и Ерусалим жалееха за Иосия.
\par 25 Също и Еремия плака за Иосия; и всичките певци и певици до днес напомнюват Иосия в плачовете си, които и направиха това обичай в Израиля; и ето, те са написани в Плача.
\par 26 А останалите дела на Иосия, и добродетелите му сторени според предписаното в Господния закон,
\par 27 и делата му, първите и последните, ето, писани са в Книгата на Израилевите и Юдовите царе.

\chapter{36}

\par 1 Тогава людете от Юдовата земя взеха Иохаза, Иосиевия син, та го направиха цар в Ерусалим вместо баща му.
\par 2 Иоахаз бе двадесет и три години на възраст когато са възцари, и царува три месеца в Ерусалим.
\par 3 Защото Египетския цар го свали от престола в Ерусалим, и наложи на земята данък от сто таланта сребро и един талант злато.
\par 4 И египетския цар постави брата му Елиакима цар над Юда и Ерусалим, като промени името му на Иоаким. А брат му Иоахаз, Нехао взе та го заведе в Египет.
\par 5 Иоаким бе двадесет и пет години на възраст когато се възцари, и царува единадесет години в Ерусалим; и върши зло пред Господа своя Бог.
\par 6 И вавилонският цар Навуходоносор възлезе против него, та го върза с окови, за да го заведе във Вавилон.
\par 7 Навуходоносор занесе и от вещите на Господния дом във Вавилон, и ги тури в капището си във Вавилон.
\par 8 А останалите дела на Иоакима, и мерзостите, които извърши, и това което се намери в него, ето, писани са в книгите на Израилевите и Юдовите царе. И вместо него се възцари син му Иоахин.
\par 9 Иоахин бе осемнадесет години на възраст когато се възцари, и царува три месеца и десет дена в Ерусалим; и върши зло пред Господа.
\par 10 А в края на годината цар Навуходоносор прати да го доведат във Вавилон, заедно с отбраните вещи на Господния дом; и направи Седекия, брата на баща му , цар над Юда и Ерусалим.
\par 11 Седекия бе двадесет и една години на възраст когато се възцари и царува единадесет години в Ерусалим.
\par 12 Той върши зло пред Господа своя Бог; не се смири пред пророк Еремия, който му говореше из Господните уста.
\par 13 А още се и подигна против цар Навуходоносора, който го бе заклел в Бога в подчиненост ; и закорави врата си, и упорствува в сърцето си да се не обърне към Господа Израилевия Бог.
\par 14 При това, всичките по-главни свещеници и людете преумножиха престъпленията си, според всичките мерзости на народите, и оскверниха дома на Господа, който Той бе осветил в Ерусалим.
\par 15 И Господ Бог на бащите им ги предупреждаваше чрез Своите посланици, като ставаше рано и ги пращаше, защото жалеше людете Си и обиталището Си.
\par 16 Но те се присмиваха на Божиите посланици, презираха словата на Господа , и се подиграваха с пророците Му, догдето гневът Му се издигна против людете Му така, че нямаше изцеление.
\par 17 Затова Той доведе против тях халдейския цар, който изби юношите им с нож, вътре в дома на светилището им, и не пожали ни юноша, ни девица, ни старец, ни белокос; всичките предаде в ръката му.
\par 18 И всичките вещи на Божия дом, големи и малки, и съкровищата на Господния дом, и съкровищата на царя и на първенците му, - всичките занесе във Вавилон.
\par 19 И изгориха Божия дом, и събориха стената на Ерусалим, и всичките му палати изгориха с огън, и всичките му скъпоценни вещи унищожиха.
\par 20 А оцелелите от нож отведе във Вавилон гдето останаха слуги нему и на синовете му до времето на персийското царство;
\par 21 за да се изпълни Господното слово, изговорено чрез устата на на Еремия, догдето земята се наслаждава със съботите си; защото през цялото време, когато лежеше пуста тя пазеше събота; докле да се изпълнят седемдесет години.
\par 22 А в първата година на персийския цар Кир, за да се изпълни Господното слово изговорено чрез устата на Еремия, Господ подбуди духа на цар Кир, та прогласи из цялото си царство, още и писмено обяви , като рече:
\par 23 Така казва персийският цар Кир: Небесният Бог Иеова ми е дал всичките царства на света; и Той ми е заръчал да Му построя дом в Ерусалим, който е в Юда. Прочее, който между вас, от всичките Негови люде, е наклонен , Иеова неговият Бог да бъде с него, нека възлезе.

\end{document}