\begin{document}

\title{Ezra}


\chapter{1}

\par 1 В първата година на персийския цар Кир, за да се изпълни Господното слово, изговорено чрез устата на Еремия, Господ подбуди духа на персийския цар Кир, та прогласи из цялото си царство, още и писмено обяви , като рече:
\par 2 Така казва персийският цар Кир: Небесният Бог Иеова ми е дал всичките царства на света; и Той ми е заръчал да Му построя дом в Ерусалим, който е в Юда.
\par 3 Прочее, който между вас, от всичките Негови люде, е наклонен , неговият Бог да бъде с него, нека възлезе в Ерусалим, който е в Юда, и нека построи на Иеова Израилевият Бог (Той е Бог) дом, който е в Ерусалим.
\par 4 На всеки, който е оцелял, в каквото и да било място, гдето пришелствува, нека му помогнат мъжете на онова място със сребро и със злато, с имот и с добитък, освен това, което доброволно би се принесло за Божия дом, който е в Ерусалим.
\par 5 Тогава станаха началниците на бащините домове на Юда и Вениамина, и свещениците, и левитите, с всичките, на които Бог подбуди духа да възлязат, за да построят Господния дом, който е в Ерусалим.
\par 6 И всичките им съседи им помагаха със сребърни вещи, със злато, с имот, с добитък и със скъпоценни неща, освен всичко това, което доброволно се принасяше.
\par 7 И цар Кир извади вещите на Господния дом, който Навуходоносор бе донесъл от Ерусалим и турил в капището на боговете си, -
\par 8 тях извади персийският цар Кир чрез съкровищника Митридат, и ги брои на управителя на Юда Сасавасар.
\par 9 И ето числото им: тридесет златни легени, хиляда сребърни легени, двадесет и девет ножове,
\par 10 тридесет златни паници от втори вид, и хиляда други съдове.
\par 11 Всичките златни и сребърни съдове бяха пет хиляди и четиристотин, които всички занесе Сасавасар, когато пленниците се върнаха от Вавилон в Ерусалим.

\chapter{2}

\par 1 А ето човеците от Вавилонската област, които се върнаха от плен, от ония, които вавилонският цар Навуходоносор беше преселил като пленници във Вавилон, и които се върнаха в Ерусалим и в Юдея, всеки в града си,
\par 2 които дойдоха със Зоровавела, Исуса, Неемия, Сараия, Реелия, Мардохея, Валасана, Масфара, Вагуя, Реума и Ваана. Числото на мъжете на Израилевите домове беше:
\par 3 Фаросови потомци, две хиляди и сто и осемдесет и девет души;
\par 4 Сефатиеви потомци, триста и седемдесет и двама души;
\par 5 Арахови потомци, седемстотин и седемдесет и пет души;
\par 6 Фаат-моавови потомци, от Исусовите и Иоавовите потомци, две хиляди и осемстотин и дванадесет души;
\par 7 Еламови потомци, хиляда и двеста и петдесет и четири души;
\par 8 Затуеви потомци, деветстотин и четиридесет и пет души;
\par 9 Закхееви потомци, седемстотин и шестдесет души;
\par 10 Вануеви потомци, шестстотин и четиридесет и двама души;
\par 11 Виваиеви потомци, шестстотин и двадесет и трима души;
\par 12 Азгадови потомци, хиляда и двеста и двадесет и двама души;
\par 13 Адоникамови потомци, шестстотин шестдесет и шест души
\par 14 Вагуеви потомци, две хиляди и петдесет и шест души;
\par 15 Адинови потомци, четиристотин петдесет и четири души;
\par 16 Атирови потомци от Езекия, деветстотин и осем души;
\par 17 Висаеви потомци, триста двадесет и трима души;
\par 18 Иораеви потомци, сто и дванадесет души;
\par 19 Асумови потомци, двеста и двадесет и трима души;
\par 20 потомци от Гивар, деветдесет и пет души;
\par 21 потомци от Витлеем, сто и двадесет и трима души;
\par 22 нетофатски мъже, петдесет и шест души;
\par 23 анатотски мъже, сто и двадесет и осем души;
\par 24 потомци от Азмавет, четиридесет и двама души;
\par 25 потомци от Кириатиарим, от Хефира, и от Вирот, седемстотин и четиридесет и трима души;
\par 26 потомци от Рама и Гавая, шестстотин и двадесет и един души;
\par 27 михмаски мъже, сто и двадесет и двама души;
\par 28 ветилски и гайски мъже, двеста и двадесет и трима души;
\par 29 потомци от Нево, петдесет и двама души;
\par 30 Магвисови потомци, сто и петдесет и шест души;
\par 31 потомци на другия Елам, хиляда и двеста и петдесет и четири души;
\par 32 Харимови потомци, триста и двадесет души;
\par 33 потомци от Лод, от Адид и от Оно, седемстотин и двадесет и пет души;
\par 34 потомци от Ерихон, триста и четиридесет и пет души;
\par 35 потомци от Сеная, три хиляди и шестстотин и тридесет души.
\par 36 Свещениците: Едаиеви потомци, от Исусовия дом, деветстотин и седемдесет и трима души;
\par 37 Емирови потомци, хиляда и петдесет и двама души;
\par 38 Пасхорови потомци, хиляда и двеста и четиридесет и седем души;
\par 39 Харимови потомци, хиляда и седемнадесет.
\par 40 Левитите: Исусови и Кадмиилови потомци, от Одавиевите потомци, седемдесет и четири души.
\par 41 Певците: Асафовите потомци, сто и двадесет и осем души.
\par 42 Потомците на вратарите: Селумови потомци, Атерови потомци, Талмонови потомци, Акувови потомци, Атитаеви потомци, Соваеви потомци, всичко сто и тридесет и девет души.
\par 43 Нетинимите: Сихаеви потомци, Асуфаеви потомци, Таваотови потомци,
\par 44 Киросови потомци, Сиаеви потомци, Фадонови потомци,
\par 45 Леванаеви потомци, Агаваеви потомци, Акувови потомци,
\par 46 Агавови потомци, Салмаеви потомци, Ананови потомци,
\par 47 Гедилови потомци, Гаарови потомци, Реаиеви потомци,
\par 48 Расинови потомци, Некодаеви потомци, Газамови потомци,
\par 49 Озаеви потомци, Фасееви потомци, Висаеви потомци,
\par 50 Асанаеви потомци, Меунимови потомци, Нефусимови потомци,
\par 51 Ваквукови потомци, Акуфаеви потомци, Арурови потомци,
\par 52 Васалотови потомци, Меидаеви потомци, Арсаеви потомци,
\par 53 Варкосови потомци, Сисарови потомци, Тамаеви потомци,
\par 54 Несиеви потомци, Атифаеви потомци.
\par 55 Потомците на Соломоновите слуги: Сотаеви потомци, Софаретови потомци, Ферудаеви потомци,
\par 56 Яалаеви потомци, Дарконови потомци, Гедилови потомци,
\par 57 Сефатиеви потомци, Атилови потомци, Фохертови потомци, от Севаим, Амиеви потомци.
\par 58 Всичките нетиними и потомците на Соломоновите слуги бяха триста и деветдесет и двама души.
\par 59 А ето тия, които възлязоха от Тел-мелах, Тел-ариса, Херув, Адан и Емир, но които не можеха да покажат бащините си домове и рода си, че бяха от Израиля;
\par 60 Далаиеви потомци, Товиеви потомци и Некодаеви потомци, шестстотин и петдесет и двама души;
\par 61 и от свещеническите потомци: Авиеви потомци, Акосови потомци, потомците на Верзелая, който взе жена от дъщерите на галаадеца Верзелай и се нарече по тяхно име.
\par 62 Тия търсиха регистрите си между преброените по родословие, но не се намериха; затова, те бидоха извадени от свещенството като скверни.
\par 63 И управителят им заповяда да не ядат от пресветите неща, докле не настане свещеник с Урим и Тумим.
\par 64 Всичките купно събрани бяха четиридесет и две хиляди и триста и шестдесет души,
\par 65 освен слугите им и слугините им, които бяха седем хиляди и триста и тридесет и тридесет и седем души. Те имаха и двеста певци и певици.
\par 66 Конете им бяха седемстотин и тридесет и шест; мъските, двеста и четиридесет и пет;
\par 67 камилите им, четиристотин и тридесет и пет; и ослите им, шест хиляди и седемстотин и двадесет.
\par 68 А някои от началниците на бащините домове , когато дойдоха в Господния дом, който е в Ерусалим, принесоха доброволно за Божия дом, за да го издигнат на мястото му;
\par 69 според силата си внесоха в съкровищницата за делото шестдесет и една хиляди драхми злато, пет хиляди драхми злато, пет хиляди фунта сребро, и сто свещенически одежди.
\par 70 Така свещениците, левитите, някои от людете, певците, вратарите и нетинимите се заселиха в градовете си, - целият Израил в градовете си.

\chapter{3}

\par 1 И като настъпи седмият месец, и израилтяните бяха в градовете, людете се събраха като един човек в Ерусалим.
\par 2 Тогава стана Исус, Иоседековият син, с братята си свещениците, и Зоровавел, Салатииловия син, с братята си, та издигнаха олтара на Израилевия Бог, за да принесат всеизгаряния върху него, според предписаното в закона на Божия човек Моисей.
\par 3 И понеже се бояха от людете на ония места, поставиха олтара на мястото му, и принасяха върху него всеизгаряния Господу, всеизгаряния заран и вечер.
\par 4 И пазиха празника на шатроразпъването, според предписаното, и принасяха ежедневните всеизгаряния на брой както бе наредено, според определеното за всеки ден,
\par 5 и от тогава на сетне и постоянните всеизгаряния, и приносите по новолунията и по всичките осветени Господни празници, както и тия на всекиго, който би принесъл доброволен принос Господу.
\par 6 От първия ден на седмия месец почнаха да принасят всеизгаряния Господу; но основите на Господния храм не бяха още положени.
\par 7 И дадоха пари на каменоделците и на дърводелците, и ядене, пиене и дървено масло на сидонците и на тиряните, за да докарат кедрови дървета от Ливан в морето при Иопия, според както персийският цар Кир беше им позволил.
\par 8 И във втория месец на втората година от завръщането им при Божия дом в Ерусалим, Зоровавел Салатииловия син, Исус Иоседековият син и другите от братята им свещеници и левити, и всички, които бяха дошли от плена в Ерусалим, почнаха да работят ; и поставиха левитите, от двадесетгодишна възраст и нагоре, да надзирават работата на Господния дом.
\par 9 И Исус, синовете му и братята му, Кадмиил и синовете му, и синовете на Юда, станаха като един човек, за да надзирават работниците по Божия дом; също и синовете на Инадада, с техните синове и братя левитите.
\par 10 И когато зидарите положиха основите на Господния дом, поставиха свещениците в одеждите им с тръби, и левитите, Асафовите потомци, с кимвали, за да хвалят Господа, според наредбата на Израилевия цар Давида.
\par 11 И пееха ответно, като хвалеха Господа и Му благодаряха защото е благ, защото е до века милостта Му към Израиля. И всичките люде нададоха голямо възклицание и хвалеха Господа, понеже основите на Господния дом бидоха положени.
\par 12 Обаче мнозина от свещениците, левитите и началниците на бащините домове , старци, които бяха видели първия дом, плачеха със силен глас като се основаваше тоя дом пред очите им; а мнозина възкликнаха гръмогласно от радост;
\par 13 така щото людете не можеха да различават гласа на веселото възклицание от гласа на плача на людете; защото людете възклицаваха със силен глас, и гласът се чуваше на далеч.

\chapter{4}

\par 1 А неприятелите на Юда и на Вениамина, като чуха, че върналите се от плена строели храма на Господа Израилевия Бог,
\par 2 дойдоха при Зоровавела и при началниците на бащините домове та им рекоха7 Да градим с вас; защото и ние търсим вашия Бог както вие Нему и жертвуваме от времето на Асирийския цар Есарадон, който ни възведе тук.
\par 3 Но Зоровавел, Исус и останалите началници на Израилевите бащини домове им рекоха: Не можете вие заедно с нас да построите дом на нашия Бог, но ние сами ще построим на Господа Израилевия Бог, както ни заповяда персийският цар Кир.
\par 4 Тогава людете на земята ослабваха ръцете на Юдовите люде и им пречеха в граденето,
\par 5 и наемаха съветници против тях, за да осуетят намерението им, през всичките дни на персийския цар Кир, дори до царуването на персийския цар Дарий.
\par 6 И когато царуваше Асуир, в началото на царуването му, написаха обвинение против жителите на Юда и Ерусалим.
\par 7 В дните на Артаксеркса писаха Вислам, Митридат, Тавеил и другите и съслужители до персийския цар Артаксеркс; и писмото се написа със сирийски букви и се съчини на сирийски език .
\par 8 Властникът Реум, и секретарят Самса писаха писмо против Ерусалим до цар Артаксеркса както следва:
\par 9 Властникът Реум, секретарят Самса и другите им съслужители, динците, афарсахците, тарфалците, афарсяните, архевците, вавилоняните, сусанците, деавците, еламците
\par 10 и останалите народи, които великият и славният Асенафар доведе и славният Асенафар доведе та засели в градовете на Самария и в другите градове оттатък реката, и прочее, -
\par 11 (ето препис от писмото, което пратиха на цар Артаксеркса, -) Слугите ти, мъжете, които са оттатък реката, и прочее:
\par 12 Да е известно на царя, че юдеите, които възлязоха от тебе при нас, като стигнаха в Ерусалим, градят бунтовния и злия град, и издигат стената като са свързали основите.
\par 13 Да е известно сега на царя, че, ако се съгради тоя град и се издигнат стени, то няма да плащат данък, мито, или пътна повинност, така щото ще повредят дохода на царете.
\par 14 А понеже ние се храним от палата, и е неприлично за нас да гледаме вредата, която ще се нанесе на царя, за това пратихме да известим на царя,
\par 15 за да се издири в книгата на летописите на бащите ти; и ще намериш в книгата на летописите и ще узнаеш, че тоя град е град бунтовен, пакостен на царете и на областите, и че още от старо време са дигали въстания всред него, за която вина тоя град е бил опустошен.
\par 16 Известявам на царя, че, ако тоя град се съгради наново , и се издигнат стените му, не ще имаш никакво притежание отсам реката.
\par 17 Царят отговори на властника Реум, на секретаря Самса и на другите им съслужители, които живееха в Самария и в другите градове отсам реката: Мир и прочее.
\par 18 Писмото, което ни пратихте, прочете се разумливо пред мене,
\par 19 и като издадох указ, издириха и намериха, че тия град още от старо време се е подигал против царете, и че в него са ставали бунтове и въстания.
\par 20 Имало още и силни царе над Ерусалим, които владеели над всичките страни оттатък реката, на които се плащало данък, мито и пътна повинност.
\par 21 Сега, прочее, заповядайте да престанат ония мъже и да се не съгради тоя град, докле не се издаде указ от мене.
\par 22 И внимавайте да не бъдете небрежливи в това, да не би да порасте злото за щета на царете.
\par 23 А когато се прочете преписът от цар Артаксерксовото писмо пред Реума и секретаря Самса и служителите им, побързаха да възлязат на Ерусалим при юдеите та ги спряха на сила.
\par 24 Така престана работата по Божия дом, който е в Ерусалим, и остана спряна до втората година от царуването на персийския цар Дарий.

\chapter{5}

\par 1 А пророците, пророк Агей, и Захария, Идовия син, пророкуваха в името на Израилевия Бог на юдеите, които бяха в Юда и Ерусалим.
\par 2 Тогава станаха Зоровавел, Салатииловият син, и Исус Иоседековият син, и почнаха да строят Божия дом, който е в Ерусалим; и с тях бяха Божиите пророци та им помагаха.
\par 3 А в същото време дойдоха при тях областният управител отсам реката Татанай, и Сетар-вознай, и служителите им, та им рекоха така: Кой ви разреши да строите тоя дом и да издигнете тая стена?
\par 4 После им казаха: Как са имената на мъжете, които строят това здание?
\par 5 Но върху старейшините на юдеите беше окото на техния Бог, така щото те не можаха да ги спрат докле не се отнесе работата до Дария, и тогава да се даде писмен отговор за работата.
\par 6 Ето препис от писмото, което областният управител отсам реката Татанай, и Сетар-вознай и служителите му афарсахците, които са отсам реката, пратиха до цар Дария.
\par 7 Пратиха му писмо, в което бе писано така: - На Цар Дария всяко благополучие!
\par 8 Да бъде известно на царя, че ходихме в юдейската област при дома на великия Бог; и тоя се зида с големи камъни, и поставят се дървета в стените; и това дело напредва прилежно и успява в ръцете им.
\par 9 Тогава попитахме ония старейшини и им казахме така: Кой ви разреши да строите тоя дом и да издигате тия стени?
\par 10 Още и за имената им попитахме, за да ти явим, и да запишем имената на мъжете, които им са на чело.
\par 11 А те в отговор ни рекоха така: Ние сме слуги на Бога, на небето и на земята, и строим дома, който е бил построен вече преди много години, - който дим един велик Израилев цар построи и довърши.
\par 12 Но понеже бащите ни разгневиха Бога на небето, Той ги предаде в ръката на вавилонския цар Навуходоносора, халдееца, който събори тоя дом и пресели людете във Вавилон.
\par 13 Но в първата година на вавилонския цар Кир, цар Кир издаде указ, за да се построи тоя Божий дом.
\par 14 Още и златните и сребърните вещи на Божия дом, които Навуходоносор бе взел от храма що бе в Ерусалим и бе отнесъл в капището във Вавилон, тях цар Кир извади от вавилонското капище, и те бяха предадени на някой си по име Сасавасар, когото бе направил областен управител;
\par 15 и каза му: Вземи тия вещи, иди, занеси ги в храма, който е в Ерусалим, и нека се построи Божият дом на мястото си.
\par 16 Тогава тоя Сасавасар дойде и положи основите на Божия дом, който е в Ерусалим; и от онова време дори до сега той се строи, и още не е довършен.
\par 17 Сега, прочее, ако е угодно на царя, нека се издири в царската съкровищница там у Вавилон, дали наистина е бил издаден указ от цар Кира, за да се построи тоя Божий дом в Ерусалим; и нека изпрати царя да ни извести волята си в случая.

\chapter{6}

\par 1 Тогава цар Дарий издаде указ, та претърсиха в помещенията на архивите, гдето се полагаха съкровищата във Вавилон;
\par 2 и намериха се в Ахмета, в палата, който е в областта на мидяните, един свитък, в който имаше един такъв летопис: -
\par 3 В първата година на цар Кира, цар Кир издаде указ както следва : За Божия дом в Ерусалим: Нека се построи домът, мястото гдето се принасят жертви, и да се положат здраво основите му; височината му да бъде шесдесет лакътя, широчината му шестдесет лакътя,
\par 4 с три реда големи камъни и един ред нови дървета; и разноските да се дадат от царския дом.
\par 5 Също и златните и сребърните вещи на Божия дом, който Навуходоносор взе от храма, който бе в Ерусалим, та ги донесе във Вавилон, нека се дадат назад и да се повърнат на храма в Ерусалим, всеки на мястото си, и да се положат в Божия дом.
\par 6 Сега, прочее, областни управители отвъд реката Татанае, Сетар-вознае, и съслужителите ви афарсахците, които сте отвъд реката, отдалечете се от там.
\par 7 Не спъвайте работата по тоя Божий дом; областният управител на юдеите и старейшините на юдеите нека построят тоя Божия дом на мястото му.
\par 8 При туй, относно онова, що трябва да направите за тия старейшини на юдеите, за построяването на тоя Божий дом, издавам указ да се дадат незабавно от царския имот, от данъка на жителите отвъд реката, разноските на тия човеци, за да не бъдат възпрепятствувани.
\par 9 И каквото би им потрябвало, било телци, овни, или агнета, за всеизгарянията на небесния Бог, тоже жито, сол, вино, или дървено масло, според искането на свещениците, които са в Ерусалим, нека из се дават всеки ден непременно,
\par 10 за да принасят благоуханни жертви на небесния Бог, и да се молят за живота на царя и на синовете му.
\par 11 Издавам още и указ, щото на всеки, който би изменил тая заповед по тая причина, да се изкърти греда от къщата му та да се изправи, и той да се обеси на нея, и къщата му да стане на бунище.
\par 12 И Бог, Който е настанил името Си там, нека изтреби всички царе и племена, които биха прострели ръка, за да изменят тая заповед , та да се събори тоя Божий дом, който е в Ерусалим. Аз Дарий издадох указа; нека се изпълни незабавно.
\par 13 Тогава областният управител отсам реката Татанае, и Сетарвознай, и съслужителите им направиха незабавно така както бе заповядано от цар Дария.
\par 14 И юдейските старейшини градяха успешно, насърчавани чрез пророкуването на пророк Агея и на Захария, Идовия син. Те градиха и свършиха, според заповедта на Израилевия Бог, и според указа на персийския цар Кир, и Дарий, и Артаксеркс.
\par 15 Така тоя дом се свърши на третия ден от месец Адар, в шестата година от царуването на цар Дария.
\par 16 И израилтяните, свещениците, левитите и останалите от завърналите се от плена, пазиха с веселие посвещението на тоя Божий дом.
\par 17 За посвещението на тоя Божий дом принесоха сто юнеца, двеста овена и четиристотин агнета, и в принос за грях за целия Израил, дванадесет козела, според числото на израилевите племена,
\par 18 И поставиха свещениците в чиновете им, и левитите в отредите им, за Божията служба, която се върши в Ерусалим, според предписаното в Моисеевата книга.
\par 19 И завърналите се от плена направиха пасхата на четиринадесетия ден от първия месец;
\par 20 защото свещениците и левитите бяха се очистили единодушно, та те всички бяха чисти; и левитите заклаха пасхата за всички завърнали се от плена, и за братята си свещениците, и за себе си.
\par 21 И завърналите се от плена израилтяни, и всички, които се бяха отлъчили от нечистотата на народите на земята и се бяха присъединили към тях, за да търсят Господа Израилевия Бог, ядоха пасхата ,
\par 22 и пазиха празника на безквасните хлябове седем дни с веселие; защото Господ ги развесели, като бе обърнал към тях сърцето на асирийския цар, за да укрепи ръцете им в делото на дома на Бога, Израилевия Бог.

\chapter{7}

\par 1 След тия събития, в царуването на персийския цар Артаксеркс, дойде Ездра син на Сараия, син на Азария, син на Хелкия,
\par 2 син на Селума, син на Садока син на Ахитова,
\par 3 син на Амария, син на Азария син на Мараиота,
\par 4 син на Зария, син на Озия, син на Вукия,
\par 5 син на Ависуя, син на Финееса, син на Елеазара, син на първосвещеника Аарон.
\par 6 Тоя Ездра възлезе от Вавилон; и бе книжник вещ в Моисеевия закон, който Господ Израилевият Бог бе дал; и царят му даде всичко що поискаше, понеже ръката на Господа неговия Бог беше над него за добро .
\par 7 Също и някои от израилтяните и от свещениците, и певците, вратарите и нетинимите възлязоха в Ерусалим в седмата година на цар Артаксеркса.
\par 8 Ездра стигна в Ерусалим в петия месец от седмата година на царя.
\par 9 Защото на първия ден, от първия месец, той тръгна от Вавилон, и на първия ден от петия месец, стигна в Ерусалим, понеже добрата ръка на неговия Бог беше с него.
\par 10 Защото Ездра бе утвърдил сърцето си да изучава Господния закон и да го изпълнява, и да учи в Израиля повеления и съдби.
\par 11 А ето препис от писмото, което цар Артаксеркс даде на свещеник Ездра, книжника, книжника вещ в думите на Господните заповеди и в повеленията Му към Израиля: -
\par 12 Артаксеркс, цар на царете, към свещеник Ездра, книжника вещ в закона на небесния Бог, съвършен мир , и прочее.
\par 13 Издавам указ щото всички, които са от Израилевите люде, и от свещениците и левитите, в царството ми, които биха пожелали доброволно да идат в Ерусалим, да отидат с тебе.
\par 14 Тъй като си изпратен от царя и от седмината му съветника да изпиташ за Юда и Ерусалим според закона на твоя Бог, който е в ръката ти,
\par 15 и да занесеш среброто и златото, което царят и съветниците му доброволно принасят на Израилевия Бог, Чието обиталище е в Ерусалим,
\par 16 както и всичкото сребро и злато, което ти би събрал от цялата Вавилонска област, заедно с доброволните приноси на людете и на свещениците, които биха принасяли доброволно за дома на своя Бог в Ерусалим,
\par 17 затова, купи веднага с тия пари юнци, овни, агнета, с хлебните им приноси и възлиянията им, и принеси ги върху олтара на дома на вашия Бог, Който е в Ерусалим.
\par 18 И каквото се види угодно на тебе и на братята ти да сторите с останалото сребро и злато, сторете според волята на вашия Бог.
\par 19 И съдовете, които ти са дадени за службата на дома на твоя Бог, предай ги пред ерусалимския Бог.
\par 20 И каквото повече би потрябвало за дома на твоя Бог, което стане нужда да изразходваш, изразходвай го от царската съкровищница.
\par 21 И аз, аз цар Артаксеркс, издавам указ до всичките хазнатари, които са отвъд реката, щото всичко, което ли поискал от вас свещеник Ездра, книжникът вещ в закона на небесния Бог, да се дава веднага,
\par 22 до сто таланта сребро, до сто кара жито, до сто вати вино, до сто вати дървено масло, и неопределено количество сол.
\par 23 Каквото и да е заповядано от небесния Бог, нека се направи за дома на небесния Бог, да не би да дойде гняв върху царството на царя и на синовете му.
\par 24 При това ви известяваме относно свещениците, левитите, певците, вратарите, нетинимите и слугите на тоя Божий дом, че на никого от тях не ще бъде законно да се наложи данък, мито, или пътна повинност.
\par 25 И ти, Ездра, според мъдростта, която имаш от твоя Бог, постави управници и съдии, които да съдят всичките люде намиращи се отвъд реката, и те всички да са такива, които знаят законите на твоя Бог; и поучавайте всеки, който не ги знае.
\par 26 И против всеки, който не пази законите на твоя Бог и царския закон, нека се изпълнява незабавно против него присъдата, било за смърт, за заточение, за конфискуване на имот, или за затвор.
\par 27 Благословен да бъда Господ, Бог на нашите бащи, Който е турил такава мисъл в сърцето на царя, да прослави дома на Господа, Който е в Ерусалим,
\par 28 и е направил да намери милост пред царя и съветниците му, и пред всичките силни първенци на царя! Така, понеже ръката на Господа моя Бог беше над мене за добро , аз се ободрих, и събрах от Израиля някои по-видни човеци, за да възлязат заедно с мене.

\chapter{8}

\par 1 А ето началниците на бащините им домове , ето и родословието, на ония, които възлязоха с мене от Вавилон в царуването на цар Артаксеркса.
\par 2 От Финеесовите потомци, Гирсон; от Итамаровите потомци, Даниил; от Давидовите потомци, Хатус.
\par 3 От потомците на Сехания, из Фаросовите потомци, Захария, и с него се преброиха по родословието на мъжките сто и петдесет души.
\par 4 От Фаат-моавовите потомци, Зараиевият син Елиоинай, и с него двеста души от мъжки пол.
\par 5 От Сеханиевите потомци, Яазииловият син, и с него триста души от мъжки пол.
\par 6 От Адиновите потомци, Ионатановият син Евед, и с него петдесет души от мъжки пол.
\par 7 От Еламовите потомци, Готолиевият син Исаия, и с него седемдесет души от мъжки пол.
\par 8 От Сефатиевите потомци, Михаиловият син Зевадия, и с него осемдесет души от мъжки пол.
\par 9 От Иоавовите потомци, Ехииловият син Авдия, и с него двеста и осемдесет души от мъжки пол.
\par 10 От Селомитовите потомци, Иосифиевият син, и с него сто и шестдесет души от мъжки пол.
\par 11 От Винаевите потомци, Захария Виваевият син, и с него двадесет и осем души от мъжки пол.
\par 12 От Азгадовите потомци, Иоанан Акатановият син, и с него сто и десет души от мъжки пол
\par 13 От послешните, Адоникамовите потомци, следните , чиито имена са: Елифалет, Еиил и Семаия, и с тях седемдесет души от мъжки пол.
\par 14 А от Вагуевите потомци, Утай и Завуд, и с тях седемдесет души от мъжки пол.
\par 15 Тия събрах при реката, която тече към Аава, и там се разположихме в стан три дни; а като прегледах людете и свещениците, не намерих там ни един от потомците на Левия.
\par 16 Тогава пратих за по-видните човеци Елиезера, Ариила, Семаия, Елнатана, Ярива, Елнатана, Натана, Захария и Месулама, и за благоразумните мъже Иоярива и Елнатана,
\par 17 и дадоха им поръчка до началника на мястото Касифия, Идо; и казах им какво да казват на Идо и на братята му нетинимите на мястото Касифия, да ни изпратят служители за дома на нашия Бог.
\par 18 И понеже добрата ръка на нашия Бог беше над нас, те ни доведоха един разумен човек от потомците на Маалия, син на Левия, син на Израиля; също и Саравия със синовете му и братята му, осемнадесет души;
\par 19 и Асавия, и с него Исаия от Марариевите потомци, братята му и синовете им, двадесет души;
\par 20 и от нетинимите, които Давид и първенците бяха определили за прислуга на левитите, двеста и двадесет нетиними. Всички тия бяха споменати по име.
\par 21 Тогава прогласих пост там при реката Аава, за да се смирим пред нашия Бог, и да просим от Него добър път за нас, за чадата ни, и за всичкия ни имот.
\par 22 Защото ме беше срам да поискам от царя войници и конници, за да ни помогнат против неприятели по пътя; понеже бяхме говорили на царя казвайки: Ръката на нашия Бог е за добро над всички, които Го търсят, а силата Му и гневът Му са против всички, които Го оставят.
\par 23 Постихме, прочее, и молихме се на нашия Бог за това, и Той ни послуша.
\par 24 Тогава отделиха дванадесет от главните свещеници - Саравия, Асавия, и с тях десет от братята им -
\par 25 и претеглих им среброто, златото и вещите, които царят, съветниците му и първенците му, и целият Израил, който се намираше там, бяха дали в принос за дома на нашия Бог;
\par 26 претеглих, прочее, в ръката им шестстотин и петдесет таланти сребро, сто таланта сребърни вещи, сто таланта злато,
\par 27 двадесет златни паници, хиляда драхми на тегло, и два съда от добра лъскава мед, скъпоценни като злато.
\par 28 И рекох им: Вие сте свети Господу, и вещите са свети; защото среброто и златото са доброволен принос на Господа Бога на бащите ви;
\par 29 внимавайте, прочее , и пазете ги докле ги претеглите пред по-първите от свещениците и левитите и началниците на Израилевите бащини домове в Ерусалим, в стаите на Господния дом.
\par 30 И тъй, свещениците и левитите приеха среброто, златото и вещите в размер на това тегло, за да ги донесат в Ерусалим в дома на нашия Бог.
\par 31 Тогава на дванадесетия ден от първия месец станахме от реката Аава, за да отидем; и ръката на нашия Бог бе над нас за добро , та ни избави от неприятелска ръка и от причакващи по пътя.
\par 32 И като стигнахме в Ерусалим, седяхме там три дни.
\par 33 А на четвъртия ден, в дома на нашия Бог, среброто, златото и вещите се претеглиха в ръката на Меримота, син на свещеник Урия, с когото бе Елеазар Финеесовият син, и с тях левитите Иозавад, Исусовият син и Ноадия, Венуевият син, -
\par 34 всичко се предаде под брой и по тегло; и цялото тегло се записа в същото време.
\par 35 Завърналите се от плен, които бяха дошли от заточение, принесоха всеизгаряния на Израилевия Бог, дванадесет юнци за целия Израил, деветдесет и шест овена, седемдесет и седем агнета, и дванадесет козела в принос за грях, всички тия във всеизгаряния на Господа.
\par 36 И предадоха царските поръчки на царските сатрапи и на областните управители отсам реката; и те помагаха на людете и на Божия дом.

\chapter{9}

\par 1 Като се свърши това, първенците дойдоха при мене и казаха: Израилевите люде, и свещениците и левитите, не са се отделили от людете на тия земи, колкото за техните мерзости, мерзостите на ханаанците, хетейците, амонците, моавците, египтяните и аморейците;
\par 2 защото са вземали от дъщерите им за себе си и за синовете си, тъй че светият род се е смесил с людете на тия земи; дори, ръката на първенците и на по-видните люде е била първа в това престъпление.
\par 3 И като чух това нещо, раздрах дрехата си и мантията си, скубах космите от главата си и от брадата си, и седях смутен.
\par 4 Тогава се събраха при мене всички, които трепереха от думите на Израилевия Бог поради престъплението на ония, които са били в плана; и седях смутен до вечерната жертва.
\par 5 А във време на вечерната жертва станах от унижението си, и с раздраната си дреха и мантия преклоних колене, и като прострях ръцете си към Господа моя Бог, рекох:
\par 6 Боже мой, срамувам се и червя се да подигна лицето си към Тебе, Боже мой; защото нашите беззакония превишиха главите ни, и престъпленията ни пораснаха до небесата.
\par 7 От дните на бащите си до днес ние сме били много виновни; и поради беззаконията си ние, царете ни и свещениците ни бяха предадени в ръката на другоземните царе, под нож, в плен, на раграбване, и на посрамяване на лицата ни, както е днес,
\par 8 И сега за твърде малко време Господ нашият Бог показва милост към нас, за да се опази между нас остатък, и за да бъдем закрепени в Неговото свето място, за да може нашият Бог да просвещава очите ни и да ни даде малко отдих в робството ни.
\par 9 Защото сме роби; но и в робството ни нашият Бог не ние оставил, а е благоволил да намерим милост пред персийските царе, та да ни съживи, за да издигнем дома на нашия Бог и да поправим развалините му, и да ни даде ограда в Юда и в Ерусалим.
\par 10 Но сега, Боже наш, какво да кажем след това? защото оставихме заповедите,
\par 11 които Си дал чрез слугите Си пророците, като си рекъл: Земята, в който влизате, за да я владеете, е земя осквернена от гнусотиите и от мерзостите на людете на тия земи, които са я напълнили от край до край с нечистотиите си;
\par 12 сега, прочее, не давайте дъщерите си на синовете им, и не вземайте техните дъщери за синовете си, и не съдействувайте никога за мира или благоденствието им; за да се укрепите, и да ядете благата на тая земя, и да я оставите в наследство на потомците си за винаги.
\par 13 А след всичко, което е дошло върху нас поради нашите лоши дела и поради голямото наше престъпление, тъй като Ти, Боже наш, си се въздържал да не ни накажеш според беззаконията ни, а си ни дал такова избавление,
\par 14 трябва ли ние да нарушим пак Твоите заповеди и да се сродяваме с людете предадени на тия мерзости? Ти не били се разгневил на нас докле ни довършиш, та да не остане никакъв остатък и никой оцелял?
\par 15 Господи Боже Израилев, праведен си, защото ние сме остатък оцелели, както сме днес; ето, пред Тебе сме с престъплението си! защото поради него не можем да се изправим пред Тебе.

\chapter{10}

\par 1 А като се молеше Ездра и се изповядваше с плач, паднал пред Божия дом, събра се при него от Израиля едно много голямо множество мъже, жени и деца; защото людете плачеха много горчиво.
\par 2 Тогава Сехания, Ехииловият син, от Еламовите потомци, проговори та рече на Ездра: Ние извършихме престъпление против своя Бог, като взехме чужденки от племената на земята; все пак, обаче, сега има надежда за Израиля относно това нещо.
\par 3 Сега, прочее, нека направим завет с нашия Бог да напуснем всички тия жени и родените от тях, съгласно съвета на господаря ми и на ония, които треперят от заповедта на нашия Бог; и нека се постъпи според закона.
\par 4 Стани, защото това нещо на тебе принадлежи; и ние сме с тебе; бъди храбър и действувай.
\par 5 Тогава Ездра стана та закле началниците на свещениците, на людете и на целия Израил, че ще постъпят според това заявление. И те се заклеха.
\par 6 Тогава Ездра стана от пред Божия дом та отиде в стаята на Иоанана, Елиасивовия син; но когато стигна там, не яде хляб и не пи вода, защото тъжеше за престъплението на ония, които са били в плена.
\par 7 Сетне прогласиха по Юда и Ерусалим между всичките върнали се от плена да се съберат в Ерусалим,
\par 8 и да бъде конфискуван целият имот на всеки, който не би дошъл за три дни според решението на началниците и старейшините, и сам той да бъде отлъчен от обществото на ония, които са били в плен.
\par 9 И така, в три дни се събраха всичките Юдови и Вениаминови мъже в Ерусалим. Беше деветият месец, на двадесетия ден от месеца; и всичките люде, седейки в двора пред Божия дом, трепереха поради това нещо и от поройния дъжд.
\par 10 Тогава свещеник Ездра стана та им рече: Вие сте извършили престъпление, като сте взели чужденки за жени та сте увеличили виновността на Израиля.
\par 11 Сега, прочее, изповядайте се на Господа Бога на бащите си и изпълнете волята Му, и отлъчете се от племената на тая земя и от чужденките жени.
\par 12 Тогава цялото събрание отговаряйки рече със силен глас: Както си казал, така подобава нам да направим.
\par 13 Людете, обаче, са много, и понеже времето е много дъждовито, не можем да стоим вън; при това, работата не е за един ден, нито за два, защото сме мнозина, които сме съгрешили в това нещо.
\par 14 Затова, нека се назначат нашите първенци да надзирават тая работа за цялото общество, и в определени времена нека дохождат пред тях всички, които са взели за жени чужденки по нашите градове, и заедно с тях старейшините на всеки град, и съдиите му, догдето се върне от нас пламенният гняв на нашия Бог поради това нещо.
\par 15 И понеже само Ионатан Асаиловият син и Яазия Текуевият син станаха да се противят с Месулам и левитина Саватай, които ги подкрепиха,
\par 16 затова завърналите се от плена постъпиха така; и отделиха се от свещеник Ездра някои началници на бащини домове , според бащините си домове, всички по име; и те седнаха на първия ден от десетия месец да изследват тая работа.
\par 17 И до първия ден от първия месец свършиха с всичките мъже, които бяха взели за жени чужденки.
\par 18 Между потомците на свещениците се немериха такива, които бяха взели за жени чужденки: от синовете на Исуса Иоседековият син и братята му, Маасия, Елиезер, Ярив и Годолия.
\par 19 И дадоха ръцете си в обещание , че ще напуснат жените си; и като престъпници принесоха за престъплението си овен от стадото.
\par 20 А от Емировите потомци, Ананий и Зевадия,
\par 21 от Харимовите потомци, Маасия, Илия, Семаия, Ехиил и Озия.
\par 22 От Пасхоровите потомци, Елиоинай, Маасия, Исмаил, Натанаил, Иозавад и Еласа.
\par 23 А от певците: Иозавад, Семей, Калаия, (който е Келита), Петаия, Юда и Елиезер.
\par 24 От певците, Елиасив; и от вратарите: Селум, Телем и Урий,
\par 25 А от Израиля, от Фаросовите потомци: Рамия, Езия, Малхия, Миамин, Елеазар, Мелхия и Ванаия,
\par 26 От Еламовите потомци: Матания, Захария, Ехиил, Авдий, Еримот и Илия.
\par 27 От Затуевите потомци: Елиоинай, Елиасив, Матания, Еримот, Завад и Азиза.
\par 28 От Виваевите потомци: Иоанан, Анания, Завай и Атлай.
\par 29 От Ваниевите потомци: Месулам, Малух, Адаия, Ясув, Сеал и Рамот.
\par 30 От Фаат-моавовите потомци; Адна, Хелал, Ванаия, Маасия, Матания, Веселеил, Вануй и Манасия.
\par 31 От Харимовите потомци: Елиезер, Есия, Мелхия, Семаия и Симеон.
\par 32 Вениамин, Малух и Самария.
\par 33 От Асумовите потомци: Матенай, Матата, Завад, Елифалет, Еремай, Манасия и Семей.
\par 34 От Вениевите потомци: Мадай, Амрам и Уил.
\par 35 Ванаия, Ведеия, Хелуй,
\par 36 Вания, Меримот, Елиасив,
\par 37 Матания, Матенай, Яасо,
\par 38 Ваний, Вануй, Семей,
\par 39 Селемия, Натан, Адаия,
\par 40 Махнадевай, Сасай, Сарай,
\par 41 Азареил, Селемия, Самария,
\par 42 Селум, Амария и Иосиф.
\par 43 От Невоевите потомци: Еиил, Мататия, Завад, Зевина, Ядав, Иоил и Ванаия.
\par 44 Всички тия бяха взели чужденки за жени; и някои от жените бяха им родили чада.

\end{document}