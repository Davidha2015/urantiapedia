\begin{document}

\title{Естир}


\chapter{1}

\par 1 А в дните на Асуира, (оня Асуир, който царуваше от Индия дори до Етиопия над сто и двадесет и седем области),
\par 2 в ония дни, когато цар Асуир бе седнал на престола на царството си в столицата Суса,
\par 3 в третата година от царуването си, той направи угощение на всичките си първенци и на слугите си, (като пред него бяха най-силните персийски и мидийски мъже , големците и първенците на областта),
\par 4 Когато за дълго време, сто и осемдесет дни, показваше богатството на славното ти царство и блясъка на превъзходното си величие.
\par 5 И когато се изминаха тия дни, царят направи седемдневно угощение на всичките люде, които се намираха в столицата Суса, от голям до малък, в двора на градината на царския палат,
\par 6 Който бе украсил със завеси от бял, зелен и син плат , прострени на висонени и морави върви, окачени със сребърни колелца на мраморни стълбове, и със златни и сребърни канапета върху настилка от порфир и от бял мрамор, от алабастър и от черен мрамор.
\par 7 И наливаха им в златни чаши, (като бяха чашите различни едни от други); и имаше изобилно царско вино, както подобаваше на царя.
\par 8 А пиенето ставаше според една издадена заповед, че никой не бива да принуждава, защото царят беше заповядал така на всичките си домостроители, да постъпват според волята на всекиго.
\par 9 А и царица Астин направи на жените угощение в царската къща на царя Асуира.
\par 10 А на седмия ден, когато сърцето на царя бе се развеселило от виното, той заповяда на Меумана, на Визата, на Арвона, на Вигта, на Авагта, на Зетара и на Харкаса, седемте скопци, които слугуваха пред цар Асуира,
\par 11 да доведат царица Астин пред царя, носеща царската корона, за да покаже хубостта й на людете и на първенците; защото тя бе красива на глед.
\par 12 Но царица Астин отказа да дойде по царската заповед чрез скопците; за това царят се разяри твърде много, и гневът му пламна в него.
\par 13 Тогава рече царят на мъдреците, които познаваха времената, (защото царят имаше обичай така да се носи спрямо всички, които знаеха закон и съд;
\par 14 А втори след него бяха Карсена, Сетар, Адмата, Тарсис, Мерес, Марсена и Мемукан, седем персийски и мидийски първенци, които имаха достъп при царя и заемаха първо място в царството):
\par 15 Що можем, според закона, да направим на царица Астин за гдето не изпълни заповедта на царя Асуира чрез скопците?
\par 16 И Мемукан, отговори пред царя и първенците: Царица Астин не е обидила само царя, но и всичките първенци и всичките племена, които са по всичките области на цар Асуира.
\par 17 Защото тая постъпка на царицата ще се разчуе между всичките жени, така щото, когато се разнесе слух, че цар Асуир заповядал да се доведе царица Астин пред него, а тя не дошла, това ще направи мъжете им презрени пред очите им.
\par 18 И днес персийските и мидийските съпруги, които ще са чули за постъпката на царицата, ще говорят по същия начин на всичките царски първенци; и от това ще произлезе голямо презрение и гняв.
\par 19 Ако е угодно на царя, нека се издаде от него царска заповед, която да се впише между персийските и мидийските закони, за да се не отменява, според която Астин да не дохожда вече пред цар Асуира; и царят нека даде царското й достойнство на друга по-добра от нея.
\par 20 И когато указът, който царят ще издаде, бъде обнародван по цялото му царство, (защото е голямо), всичките жени ще отдават чест на мъжете си, на голям и на малък.
\par 21 И тая дума се хареса на царя и на първенците; и царят стори според каквото каза Мемукан.
\par 22 Той прати писма по всичките царски области, във всяка област според азбуката й, и на всеки народ според езика му, за да може всеки мъж да бъде господар в дома си, и в него да говори по езика на людете си.

\chapter{2}

\par 1 След тия събития, като се укроти яростта на цар Асуира, той си спомни за Астин, и за онова що бе сторила, и за какво бе решено против нея.
\par 2 Тогава слугите, които слугуваха на царя, рекоха: Нека се потърсят за царя красиви млади девици;
\par 3 и нека назначи царят чиновници по всичките области на царството си, които да съберат в столицата Суса, в женската къща, всичките красиви млади девици, под надзора на царския скопец Игай, пазач на жените; и нека им се дадат нещата, които из са потребни за приглаждане.
\par 4 И девицата, която се хареса на царя, нека стане царица вместо Астин. И това биде угодно на царя; и той стори така.
\par 5 Имаше в столицата Суса един юдеин на име Мардохей, сина на Яира, син на Семея, син на Киса, вениаминец,
\par 6 който беше откаран от Ерусалим с пленниците, които бяха откарани с Юдовия цар Ехония, които вавилонският цар Навуходоносор беше откарал.
\par 7 И той отхранваше Адаса (която е Естир), чичовата си дъщеря, защото тя нямаше ни баща ни майка. Момичето беше прилично и красиво; и когато баща му и майка му умряха Мардохей го взе за своя дъщеря.
\par 8 И той, когато са разчу заповедта е указа на царя, и когато бяха събрани много момичета в столицата Суса, под надзора на Игая, доведоха и Естир в царската къща, под надзора на Игая, пазача на жените.
\par 9 И момичето му се хареса и придоби неговото благоволение; и той побърза да й даде от царската къща нещата потребни за приглаждането й, както и нейния дял, също и седемте момичета, които подобаваше да й се дадат; и премести нея и момичетата й в най-доброто помещение на женската къща.
\par 10 Естир не беше изявила людете си, нито рода си; защото Мардохей беше й заръчал да не го изявява.
\par 11 И Мардохей ходеше всеки ден пред двора на женската къща, за да се научава как е Естир и какво ще стане с нея.
\par 12 А когато дойдеше реда на всяка девица да влезе при цар Асуира, след като беше стояла дванадесет месеца в женската къща , според нареденото за жените, (защото така се употребяваше времето на приглаждането им, шест месеца се мажеха със смирнено масло, и шест месеца с аромати и с други неща за приглаждане на жените),
\par 13 тогава така приготвена , девицата влизаше при царя; всичко каквото поискаше даваше й се, да го занесе със себе си от женската къща в къщата на царя.
\par 14 Вечерта влизаше, а заран се връщаше във втората женска къща под надзора на Саасгаза, царския скопец, който пазеше наложниците; тя не влизаше вече при царя, освен ако царят я поискаше, и бъдеше повикана по име.
\par 15 А когато дойде редът да влезе при царя Естир, дъщеря на Авихаила, чичо на Мардохея, която той беше взел за своя дъщеря, тя не поиска друго освен каквото определи царският скопец Игай, пазача на жените. И Естир придобиваше благоволението на всички, които я гледаха.
\par 16 И тъй, Естир биде заведена при цар Асуира в царския му дом в десетия месец, който е месец Тевет, в седмата година от царуването му.
\par 17 И царят възлюби Естир повече от всичките жени; и тя придоби неговото благоволение и милост повече от всичките девици; и той тури царската корона на главата й, и направи я царица вместо Астин.
\par 18 Тогава царят направи голямо угощение на всичките си първенци на слугите си, угощение в чест на Естир; и определи деня за празник на областите, и даде подаръци, както подобаваше на царя.
\par 19 И когато се събраха девиците втори път, тогава Мардохей седеше в царската порта.
\par 20 Естир не бе изявила рода си, нито людете си, както й беше заръчал Мардохей; защото Естир изпълняваше заповедта на Мардохея както, когато беше в неговия дом.
\par 21 В тия дни, когато Мардохей седеше в царската порта, двама от царските скопци, Вихтан и Терес, от ония, които пазеха входа, се разгневиха, и поискаха да турят ръка върху цар Асуира.
\par 22 А това стана известно на Мардохея, и той го обади на царица Естир; а Естир обади на царя в името на Мардохея.
\par 23 И като се изследва работата и се намери, че беше така, и двамата бидоха обесени на дърво; и събитието се записа в книгата на летописите пред царя.

\chapter{3}

\par 1 След това, цар Асуир повиши Амана, син на агегеца Амидата, въздигна го, и постави стола му над столовете на всичките първенци, които бяха около него.
\par 2 И всичките царски слуги, които бяха в царската порта, се навеждаха и се кланяха на Амана; защото царят бе заповядал така в него. Но Мардохей не се навеждаше, нито му се кланяше.
\par 3 Затова, царските слуги, които бяха в царската порта, рекоха на Мардохея: Ти защо престъпваш царската заповед?
\par 4 А като му говореха всеки ден, а той ги не слушаше, обадиха на Амана, за да видят дали думите на Мардохея ще устоят, тъй като из беше явил, че е юдеин, та не се покорява на заповедта .
\par 5 И когато видя Аман, че Мардохей не се навеждаше, нито му се кланяше, Аман се изпълни с ярост.
\par 6 Но мислеше, че да тури ръка само на Мардохея ще бъде нищожно нещо; затова, понеже му бяха явили от кои люде беше Мардохей, Аман искаше да изтреби Мардохеевите люде, сиреч , всичките юдеи, които бяха в цялото царство на Асуира.
\par 7 В първия месец, който е месец Нисан, в дванадесетата година на цар Асуира, хвърлиха пур (сиреч, жребие) пред Амана последователно за всеки ден от всеки месец, дори до дванадесетия месец , който е месец Адар.
\par 8 Тогава Аман рече на цар Асуира: Има едни люде пръснати и разсеяни между племената по всичките области на твоето царство; и законите им различават от законите на всичките люде, и те не пазят царските закони; затова не е от полза за царя да ги търпи.
\par 9 Ако е угодно на царя, нека се предпише да се изтребят; и аз ще броя десет хиляди таланта сребро в ръката на чиновниците, за да го внесат в царските съкровищници.
\par 10 И царят извади пръстена си от ръката си та го даде на Амана, сина на агегеца Амедата, неприятеля на юдеите.
\par 11 И царят рече на Амана: Дава ти се среброто, тоже и тия люде, да направиш с тях както обичаш.
\par 12 И така, на тринадесетия ден, от първия месец, царските секретари бяха повикани, та се писа точно според това, което заповяда Аман, на царските сатрапи, на управителите на всяка област, и на първенците на всеки народ, във всяка област, според, азбуката им, и на всеки народ, според езика му; в името на цар Асуира се писа, и се подпечата с царския пръстен.
\par 13 И писма се изпратиха с бързоходци по всичките царски области, за да погубят, да избият, и да изтребят всичките юдеи, млади и стари, деца и жени, в един ден, тринадесетия от дванадесетия месец, който е месец Адар, и да разграбят имота им.
\par 14 Препис от писаното, чрез който щеше да се разнесе тая заповед по всяка област, се обнародва между всичките племена, за да бъдат готови за оня ден.
\par 15 Бързоходците излязоха и бързаха според царската заповед; и указът се издаде в столицата Суса. И царят и Аман седнаха да пируват; но градът Суса се смути.

\chapter{4}

\par 1 А Мардохей, като се научи за всичко, що бе станало, раздра дрехите си, облече се във вретище с пепел, и излезе всред града та викаше със силно и горчиво викане.
\par 2 И дойде пред царската порта; защото никой, облечен във вретище, не можеше да влезе вътре в царската порта.
\par 3 И във всяка област, гдето стигна тая заповед и указът на царя, стана между юдеите голямо тъгуване, пост, плач и ридание; и мнозина лежаха с вретище постлано под себе си и пепел.
\par 4 И момичетата и скопците на Естир влязоха та й известиха за това; и царицата се смути много. И прати дрехи, за да облекат Мардохея, и да съблекат вретището от него; но той не прие.
\par 5 Тогава Естир повика Атаха, един от скопците на царя, когото той бе определил да й слугува, и заповяда му да отиде при Мардохея да се научи какво е това, и защо е то.
\par 6 И тъй, Атах излезе при Мардохея в градския площад, който бе пред царската порта.
\par 7 И Мардохей му съобщи всичко, що му бе станало, и количеството на среброто, точно, което Аман бе обещал да внесе в царските съкровищници, за да изтреби юдеите.
\par 8 Даде му и препис от писаното в указа, който бе издаден в Суса за погубването им, за да го покаже на Естир и да й го обясни, и да й заръча да влезе при царя за да му се помоли, и да направи прошение за людете си.
\par 9 Атах, прочее, дойде та съобщи на Естир Мардохеевите думи.
\par 10 Естир говори на Атаха и даде му заповед да съобщи на Мардохея така :
\par 11 Всичките царски слуги и людете от царските области знаят, че веки човек, мъж или жена, който би влязъл невикан при царя във вътрешния двор, един закон има за него, - да се умъртви, освен оня, към когото царят би прострял златния скиптър, за да остане жив; но има тридесет дни откак аз не съм викана да вляза при царя.
\par 12 И известиха на Мардохея думите на Естир.
\par 13 Тогава Мардохей заръча да отговорят на Естир: Не мисли в себе си, че от всичките юдеи само ти ще се избавиш в царския дом.
\par 14 Защото ако съвсем премълчиш в това време, ще дойде от другаде помощ и избавление на юдеите, но ти и бащиният ти дом ще погинете; а кой знае да ли не си дошла ти на царството за такова време каквото е това?
\par 15 Тогава Естир заповяда да отговорят на Мардохея;
\par 16 Иди, събери всичките юдеи, които се намират в Суса, и постете за мене, не яжте и не пийте три дни, нощем и денем; и аз и момичетата ми ще постим подобно; тогава ще вляза при цар, което не е според закона и ако погина, нека погина.
\par 17 И тъй, Мардохей отиде та извърши всичко, що му бе заповядала Естир.

\chapter{5}

\par 1 А на третия ден Естир се облече в царските си дрехи, и застана в вътрешния дом на царската къща, срещу царската къща; а царят седеше на царския си престол в царската къща, срещу входа на къщата.
\par 2 И като видя царят царица Естир, че стоеше в двора, тя придоби благоволението му; и царят простря към Естир златния скиптър, който държеше в ръката си; и Естир, се приближи та се допря до края на скиптъра.
\par 3 Тогава царят й каза: Що искаш, царице Естир? и каква е молбата ти? Даже до половината от царството ще ти се даде.
\par 4 И Естир рече: Ако е угодно на царя, нека дойде царят, с Амана, днес на угощението, което съм приготвила за него.
\par 5 И царят рече: Карайте Амана да побърза, за да се направи каквото е казала Естир. Така царят и Аман дойдоха на угощението, което Естир бе приготвила.
\par 6 И като пиеха вино, царят каза на Естир: Какво е прошението ти? и ще се удовлетвори; и каква е молбата ти? и ще бъде изпълнена даже до половината от царството.
\par 7 Тогава Естир в отговор рече: Прошението и молбата ми е това :
\par 8 Ако съм придобила благоволението на царя, и ако е угодно на царя да удовлетвори прошението ми и да изпълни молбата ми, нака дойде царят с Амана, на угощението, което ще приготвя за тях, и утре ще направя според както царят е казал.
\par 9 В тоя ден Аман излезе радостен и с весело сърце; но когато вида Мардохея в царската порта, че не става, нито шава за него, Аман се изпълни с ярост против Мардохея.
\par 10 Обаче Аман се въздържа и отиде у дома си; тогава прати да повикат приятелите му и жена му Зареса,
\par 11 и Аман им приказа за славата на богатството си, и за многото си деца, и как царят го бе повишил и как го бе въздигнал над първенците и царските слуги.
\par 12 Рече още Аман: Даже и царица Естир не покани другиго с царя на угощението, което направи, а само мене; още и утре съм поканен у нея с царя.
\par 13 Обаче всичко това не ме задоволява докле гледам юдеина Мардохей да седи при царската порта.
\par 14 Тогава жена му Зареса и всичките му приятели му казаха: Нека се приготви една бесилка петдесет лакътя висока, и утре кажи на царя да бъде обесен Мардохей на нея; тогава иди радостен с царя на угощението. И като се хареса това на Амана, заповяда да се приготви бесилката.

\chapter{6}

\par 1 През оная нощ сънят побягна от царя; и той заповяда да донесат записната книга на летописите; и прочитаха се пред царя.
\par 2 И намери се как Мардохей беше обадил за Вихтана и Тереса, двама от царските скопци, от ония, които пазеха входа, че бяха поискали да турят ръка на цар Асуира.
\par 3 И царят рече: Каква почест и отличие е дадено на Мардохея за това? И слугите на царя, които му прислужваха казаха: Не се е направило нищо за него.
\par 4 Тогава царят рече: Кой е на двора? А Аман беше дошъл във вътрешния двор на царската къща, за да каже на царя да обеси Мардохея на бесилката, която бе приготвил за него.
\par 5 И слугите на царя му казаха: Ето, Аман стои на двора. И царят рече: Да влезе.
\par 6 И като влезе Аман царят му рече: Що да се направи на човека, на когото царя благоволява да направи почест? А Аман помисли в сърцето си: Кому другиму би благоволил царят да направи почест освен на мене?
\par 7 Затова Аман каза на царя: За човека, на когото царят благоволи да направи почест,
\par 8 нека донесат царската одежда, с която царят се облича, и царската корона, който се туря на главата му, и нека се туря на главата му, и нека доведат коня, на който царят язди,
\par 9 и тая одежда и коня да се дадат в ръката на един от по-видните царски първенци, за да облекат човека, когото царят благоволява да почете; и когато го развеждат възседнал на коня през градския площад нека прогласяват пред него: Така ще се направи на човека, когото царят благоволява да почете.
\par 10 Тогава царят рече на Амана: Скоро вземи одеждата и коня както си рекъл, и направи така на юдеина Мардохей, който седи при царската порта; да се не изостави нищо от всичко, което си казал.
\par 11 И така, Аман взе одеждата и коня та облече Мардохея, и го преведе яхнал през градския площад, и прогласяваше пред него: Така ще се направи на човека, когото царят благоволява да почете.
\par 12 И Мардохей се върна в царската порта. А Аман отиде бърже у дома си наскърбен, и с покрита глава.
\par 13 И Аман разказа на жена си Зареса и на всичките си приятели всичко, що му се бе случило. Тогава мъдреците му и жена му Зареса му рекоха: Ако Мардохей, пред когото си почнал да изпадаш, е от юдейски род, ти не ще му надвиеш, но без друго ще паднеш пред него.
\par 14 Докато още се разговаряха с него, царските скопци стигнаха и побързаха да заведат Амана на угощението, което Естир бе приготвила.

\chapter{7}

\par 1 Царят, прочее, и Аман дойдоха да пируват с царица Естир.
\par 2 И на втория ден, като пиеха вино, царят пак каза на Естир: Какво е прошението ти, царице Естир? и ще ти се удовлетвори; и каква е молбата ти? и ще се изпълни даже до половината от царството.
\par 3 Тогава царицата Естир в отговор рече: Ако съм придобила твоето благоволение, царю, и ако е угодно на царя, нека ми се подари живота ми при молбата ми, и людете ми при молбата ми;
\par 4 защото сме продадени, аз и людете ми, да бъдем погубени, избити и изтребени, Ако бяхме само продадени като роби и робини, премълчала бих, при все че неприятелят не би могъл да навакса щетата на царя.
\par 5 Тогава цар Асуир проговаряйки рече на царица Естир: Кой е той, и где е оня, който е дръзнал в сърцето си да направи така?
\par 6 И Естир каза: Противникът и неприятелят е тоя нечестив Аман. Тогава Аман се смути пред царя и царицата.
\par 7 И царят разгневен стана от угощението с вино и отиде в градината на палата, а Аман стана, за да изпроси живота си от царица Естир, защото видя, че зло беше решено против него от царя.
\par 8 В това време царят се върна от градината на палата в мястото на винопиенето; а Аман бе паднал на постелката, на която беше Естир. И царят рече: Още и царицата ли иска да изнасили пред мене у дома? Щом излязоха тия думи из устата на царя, покриха Амановото лице.
\par 9 Тогава Арвона, един от служащите пред царя скопци рече: Ето и бесилката, петдесет лакътя висока, която Аман направи за Мардохея, който говори добро за царя, стои в къщата на Амана. И рече царят: Обесете го на нея.
\par 10 И така, обесиха Амана на бесилката, която бе приготвил за Мардохея. Тогава царската ярост утихна.

\chapter{8}

\par 1 В същия ден цар Асуир даде на царица Естир дома на неприятеля на юдеите Аман. И Мардохей дойде пред царя, защото Естир бе явила що й беше той.
\par 2 И царят извади пръстена си, който бе взел от Амана, и го даде на Мардохея. А Естир постави Мардохея над Амановия дом.
\par 3 Тогава Естир говори пак пред царя, паднала в нозете му, и моли му се със сълзи да отстрани злото скроено от агагеца Аман и замисленото, което беше наредил с хитрост против юдеите.
\par 4 И царят простря златния скиптър към Естир; и тъй, Естир се изправи, застана пред царя, и рече:
\par 5 Ако е угодно на царя, и ако съм придобила неговото благоволение, и това се види право на царя, и ако съм му угодна, нека се пише да се отмени писаното в писмата придобити с хитрост от агегеца Аман, син на Амедата, който писа, за да се погубят юдеите по всичките области на царя;
\par 6 защото как бих могла да търпя да видя злото, което би сполетяло людете ми? или как бих могла да търпя да видя изтреблението на рода си?
\par 7 Тогава цар Асуир каза на царица Естир и на юдеите Мардохей: Ето, дадох на Естир дома на Амана; и него обесих на бесилката, защото простря ръката си против юдеите.
\par 8 Пишете, прочее, и вие на юдеите каквото обичате, в името на царя, и подпечатайте го с царския пръстен; защото писмо писано в името на царя и подпечатано с царския пръстен не се отменява.
\par 9 Тогава в третия месец, който е месец Сиван, на двадесет и третия му ден, царските секретари бяха повикани, та се писа точно според това, което заповяда Мардохей, на юдеите, и на сатрапите, и на управителите и първенците на областите, които бяха от Индия до Етиопия, сто и двадесет и седем области, във всяка област според азбуката й, и на всеки народ според езика му, и на юдеите според азбуката им и според езика им.
\par 10 И Мардохей писа от името на цар Асуира, и подпечати го с царския пръстен, и изпрати писмата с бързоходци, които яздеха на бързи коне, употребявани в царската служба, от отборни кобили,
\par 11 чрез които писма царят разреши на юдеите, във всеки град, гдето се намираха, да се съберат да станат за живота си, да погубят, да избият и да изтребят всичката сила на ония люде и на оная област, които биха ги нападнали заедно с децата и жените им, и да разграбят имота им,
\par 12 и това в един ден, по всичките области на цар Асуира, сиреч , в тринадесетия ден от дванадесетия месец, който е месец Адар.
\par 13 Препис от писаното, чрез който щеше да се разнесе тая заповед по всяка област, се обнародва между всичките племена, за да бъдат готови юдеите за оня ден да въздадат на неприятелите си.
\par 14 И бързоходците, които яздеха на бързи коне, употребявани в царската служба, излязоха бърже, тикани от царската заповед. И указът се издаде в столицата Суса.
\par 15 А Мардохей излезе от присъствието на царя в царските дрехи, сини и бели, и с голяма златна корона, и с висонена и морава мантия; и град Суса се радваше и се веселеше.
\par 16 И юдеите имаха светлина и веселие, радост и слава.
\par 17 И във всяка област и във всеки град, гдето стигна царската заповед и указ, юдеите имаха радост и веселие, пируване и добър ден. И мнозина от людете на земята станаха юдеи; защото страх от юдеите ги обзе.

\chapter{9}

\par 1 А в дванадесетия месец, който е месец Адар, на тринадесетия ден от същия, когато наближаваше времето да се изпълни царската заповед и указ, в деня когато неприятелите на юдеите се надяваха да им станат господари, (но напротив се обърна, че юдеите станаха господари, на ония, които ги мразеха),
\par 2 юдеите се събраха в градовете си по всичките области на цар Асуира, за да турят ръка на ония, които биха искали злото им; и никой не можа да им противостои, защото страх от тях обзе всичките племена.
\par 3 И всичките първенци на областите, и сатрапите, областните управители, и царските надзиратели помагаха на юдеите; защото страх от Мардохея ги обзе.
\par 4 Понеже Мардохей беше големец в царския дом, и славата му се разнесе по всичките области; защото човекът Мардохей ставаше все по-велик и по-велик.
\par 5 И юдеите поразиха всичките си неприятели с удар от меч, с убиване, и с погубване, и сториха каквото искаха на ония, които ги мразеха.
\par 6 И в столицата Суса юдеите, избиха и погубиха петстотин мъже.
\par 7 Убиха и Фарсандата, Далфона, Аспата,
\par 8 Пората, Адалия, Аридата,
\par 9 Фармаста, Арисая, Аридая и Ваезета,
\par 10 десетте сина на неприятеля на юдеите Аман, Амедатаевия син; на имот обаче ръка не туриха.
\par 11 На същия ден, като доложиха пред царя числото на избитите в столицата Суса,
\par 12 царят рече на царица Естир: В столицата Суса юдеите са избили и погубили петстотин мъже и десетте Аманови сина; какво ли са направили и в другите царски области! Сега какво е прошението ти? и ще ти се удовлетвори; и каква е още молбата ти? и ще се изпълни.
\par 13 И Естир каза: Ако е угодно на царя, нека се разреши на юдеите, които са в Суса, да направят и утре според указа за днес, и да обесят на бесилка десетте Аманови сина.
\par 14 И царят заповяда да стане така; и издаде с указ в Суса, та обесиха десетте Аманови сина.
\par 15 И тъй, юдеите, които бяха в Суса, се събраха и на четиринадесетия ден от месец Адар, та избиха триста мъже в Суса; на имот, обаче, ръка не туриха.
\par 16 А другите юдеи, които бяха по царските области, се събраха та стояха за живота си, и се успокоиха от неприятелите си, като избиха от ония, които ги мразеха, седемдесет и пет хиляди души; на имот, обаче, ръка не туриха.
\par 17 Това стана на тринадесетия ден от месец Адар; а на четиринадесетия ден от същия си почиваха, и го направиха ден за пируване и увеселение.
\par 18 Но юдеите, които бяха в Суса, се събраха и на тринадесетия и на четиринадесетия му ден; а на петнадесетия от същия си почиваха, и го направиха ден за пируване и увеселение.
\par 19 Ето защо юдеите селяни, които живееха в градове без стени, правят четиринадесетия ден от месец Адар, ден за увеселение и за пируване, и добър ден, и ден за пращане подаръци едни на други.
\par 20 Тогава Мардохей, като описа тия събития, прати писма до всичките юдеи, които бяха по всичките области на цар Асуира, до ближните и до далечните,
\par 21 с които им заръча да пазят, всяка година, и четиринадесетия ден от месец Адар и петнадесетия от същия,
\par 22 като дните, в които юдеите се успокоиха от неприятелите си, и месеца, в който скръбта им се обърна на радост и плачът им в добър ден, та да ги правят дни за пируване и увеселение, и за пращане подаръци едни на други, и милостиня на сиромасите.
\par 23 И юдеите предприеха да правят както бяха почнали и както Мардохей им беше писал,
\par 24 по причина на агагецът Аман, син на Амедата, неприятелят на всичките юдеи, беше изхитрил против юдеите да ги погуби, и беше хвърлил пур (сиреч, жребие) да ги погуби и да ги изтреби;
\par 25 когато обаче, Естир дойде пред царя, той с писма заповяда да се върне върху неговата глава замисленото зло, което беше наредил с хитрост против юдеите, и да се обесят на бесилката той и синовете му.
\par 26 Затова, нарекоха ония дни Пурим, от името пур. За туй, поради всичките думи на това писмо, и поради онова, което бяха опитали в това събитие, и което им се бе случило,
\par 27 юдеите постановиха, и възприеха за себе си и за потомството си, и за всички, които биха се присъединили към тях, непрестанно да пазят тия два дни, според предписаното за тях и във времето им всяка година;
\par 28 и тия дни да са помнят и да се пазят във всеки век и във всеки род, във всяка области във всеки град; и тия дни Пурим да се не изоставят от юдеите, нито да изчезне споменът им между потомците им.
\par 29 Тогава царица Естир, дъщеря на Авихаила, и юдеина Мардохей писаха втори път със силно наблягане за да утвърдят написаното за Пурима.
\par 30 При това, Мардохей прати писма до всичките юдеи в сто и двадесет и седемте области на Асуировото царство, с думи на мир и искреност,
\par 31 за да утвърди тия дни Пурим във времената им, според както юдеинът Мардохей и царица Естир им бяха заповядали, и както бяха постановили за себе си и за потомството си, в спомен на постите и плача си.
\par 32 И тая наредба за Пурима се потвърди със заповедта на Естир; и записа се в книгата.

\chapter{10}

\par 1 И цар Асуир наложи данък на земята и на морските острови.
\par 2 А всичките му силни и могъществени дела, и описанието на величието, на което царят въздигна Мардохея, на са ли написани в Книгата на летописите на персийските и мидийските царе?
\par 3 Защото юдеинът Мардохей стана втори подир цар Асуира, велик между юдеите и благоугоден на многото си братя, като търсеше доброто на людете си и говореше мир за целия си род.

\end{document}