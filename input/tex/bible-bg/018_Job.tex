\begin{document}

\title{Job}


\chapter{1}

\par 1 Имаше в земята Уз един човек на име Иов. Тоя човек бе непорочен и правдив, боеше се от Бога, и отдалечаваше се от злото.
\par 2 И родиха му се седем сина и три дъщери.
\par 3 А добитъкът му беше седем хиляди овци,три хиляди камили, петстотин чифта волове; и петстотин ослици; имаше и голямо множество слуги; така че тоя човек бе най-големият от всичките жители на изток.
\par 4 И синовете му отиваха и правеха угощения в къщата на всекиго от тях на неговия ден; и пращаха да повикат трите си сестри, за да ядат и пият с тях.
\par 5 И когато се изреждаха дните на угощението, Иов пращаше за чадата си та ги освещаваше, като ставаше рано заранта и принасяше всеизгаряния според числото на всичките тях; защото Иов си думаше: Да не би синовете ми да са съгрешили и да са похулили Бога в сърцата си. Така правеше Иов постоянно.
\par 6 А един ден, като дойдоха Божиите синове да се представят пред Господа, между тях дойде и Сатана.
\par 7 И Господ рече на Сатана: От где идеш? А Сатана в отговор на Господа рече: От обикаляне земята и от ходене насам натам по нея.
\par 8 После Господ рече на Сатана: Обърнал ли си внимание на слугата Ми Иов, че няма подобен нему на земята, човек непорочен и правдив, който се бои от Бога и се отдалечава от злото?
\par 9 А Сатана в отговор на Господа рече: Дали без причина се бои Иов от Бога?
\par 10 Не си ли обградил от всякъде него и дома му и всичко що има? Благословил си делата на ръцете му, и имота му се е умножил на земята.
\par 11 Но сега простри ръка и допри се до всичко що има, и той ще Те похули в лице.
\par 12 И Господ рече на Сатана: Ето, в твоята ръка е всичко, що има той; само на него да не туриш ръка. Тогава Сатана излезе от присъствието на Господа.
\par 13 И един ден, когото синовете му и дъщерите му ядяха и пиеха вино в къщата на най-стария си брат,
\par 14 дойде вестител при Иова та рече: Като оряха воловете, и ослите пасяха при тях,
\par 15 савците нападнаха та ги откараха, още избиха слугите с острото на ножа; и само аз се отървах да ти известя.
\par 16 Докато още говореше, той, дойде и друг та рече: Огън от Бога падна от небето та изгори овцете и слугите и ги погълна; и само аз са отървах да ти известя.
\par 17 Тоя като още говореше, дойде и друг та рече: Халдейците образуваха три чети и се спуснаха върху камилите и ги откараха, още избиха слугите с острото на ножа; и само аз се отървах да ти известя.
\par 18 Тоя като говореше още, дойде и друг та рече: Синовете ти и дъщерите ти като ядяха и пиеха вино в къщата на най-стария си брат,
\par 19 ето, дойде силен вятър от пустинята та удари четирите ъгъла на къщата, и тя падна върху чадата ти та умряха; и само аз се отървах да ти известя.
\par 20 Тогава Иов стана, раздра дрехата си, и обръсна главата си, и като падна на земята поклони се.
\par 21 И рече: Гол излязох из утробата на майка си; гол ще и да се върна там. Господ даде, Господ отне; да бъда благословено Господното име.
\par 22 Във всичко това Иов не съгреши, нито се изрази безумно спрямо Бога.

\chapter{2}

\par 1 И пак един ден, като дойдоха Божиите синове, за да се представят пред Господа, между тях дойде и Сатана да се представи пред Господа.
\par 2 И Господ рече на Сатана: От где идеш? А Сатана в отговор на Господа рече: От обикаляне земята и от ходене насам натам по нея.
\par 3 После Господ рече на Сатана: Обърнал ли си внимание на слугата Ми Иов, че няма подобен нему на земята, човек непорочен и правдив, който се бои от Бога и се отдалечава от злото. И още държи правдивостта си, при все че ти Ме подбуди против него да го погубя без причина.
\par 4 А Сатана в отговор на Господа рече: Кожа за кожа, да! все що има човек ще го даде за живота си.
\par 5 Но простри ръката Си сега та се допри до костите му и до месата му, и той ще Те похули в лице.
\par 6 И Господ рече на Сатана: Ето, той е в ръката ти; само живота му опази.
\par 7 Тогава Сатана излезе от присъствието на Господа та порази Иов с лоши цирки от стъпалата на нозете му до темето му.
\par 8 И той си взе черепка, за да се чеше с нея, и седеше в пепел.
\par 9 Тогава жена му рече: Още ли държиш правдивостта си? Похули Бога и умри.
\par 10 А той й каза: Ти говориш както говори една от безумните жени. Що! доброто ли да приемаме от Бога, и да не приемаме и злото? Във всичко това Иов не съгреши с устните си.
\par 11 А тримата приятели на Иова, като чуха за всичкото това зло, което го сполетяло, дойдоха всеки от мястото си, - теманецът Елифаз, шуахецът Валдад, наамецът Софар; защото се бяха съгласили да дойдат заедно да го съжалят и да го утешат.
\par 12 И като подигнаха очи от далеч и го не познаха, плакаха с висок глас; и всеки раздра дрехата си, и сипаха пръст на главите си като я хвърляха към небето.
\par 13 И седяха при него на земята седем дни и нощи; и никой не му проговори дума, защото виждаха, че скръбта му беше много голяма.

\chapter{3}

\par 1 След това Иов отвори устата си та прокле деня си.
\par 2 Иов, проговаряйки, рече: -
\par 3 Да погине денят, в който се родих, И нощта, в която се каза, роди се мъжко.
\par 4 Да бъде тъмнина оня ден; Бог да го не зачита от горе, И да не изгрее на него светлина.
\par 5 Тъмнина и мрачна сянка да го обладаят; Облак да седи на него; Всичко що помрачава деня нека го направи ужасен.
\par 6 Тъмнина да обладае оная нощ; Да се не брои между дните на годината, Да не влезе в числото на месеците.
\par 7 Ето, пуста да остане оная нощ; Радостен глас да не дойде в нея.
\par 8 Да я прокълнат ония, които кълнат дните, Ония, които са изкусни да събудят левиатана.
\par 9 Да изгаснат звездите на вечерта й; Да очаква видело, и да го няма, И да не види първите лъчи на зората;
\par 10 Защото не затвори вратата на майчината ми утроба, И не скри скръбта от очите ми.
\par 11 Защо не умрях при раждането, И не издъхнах щом излязох из утробата?
\par 12 Защо ме приеха коленете? И защо съсците, за да суча?
\par 13 Защото сега щях да лежа и да почивам; щях да спя; Тогава щях да съм в покой.
\par 14 Заедно с царе и съветници на земята, Които си градят пусти стълбове;
\par 15 Или с князе, които имаха злато, Които напълниха къщите си със сребро;
\par 16 Или, като скрито пометниче, не щеше да ме има. Както младенци, които видело не са видели.
\par 17 Там нечестивите престават да смущават, И там уморените се успокояват.
\par 18 Заедно се успокояват и пленниците. Не чуват гласа на насилника,
\par 19 Там са малък и голям; И слугата е свободен от господаря си,
\par 20 Защо се дава видело на злощастния, И живот на огорчения в душата,
\par 21 Които копнеят за смъртта, и нямат я, Ако и да копнеят за нея повече отколкото за скрити съкровища, -
\par 22 Които се много радват и веселят, Когато намерят гроба?
\par 23 Защо се дава видело на човека, чиито път е скрит, И когото Бог е преградил?
\par 24 Защото вместо ядене, дохожда ми въздишка; И стенанията ми се изливат като вода.
\par 25 Защото онова, от което се боях, случи ми се, И онова, от което треперех, дойде върху мене.
\par 26 Не бях на мир, нито на покой, нито в охолност; Но пак смущение ме нападна.

\chapter{4}

\par 1 Тогава теманецът Елифаз проговаряйки рече: -
\par 2 Ако започнем да ти говорим, ще ти дотегне ли? Но кой може се въздържа да не говори?
\par 3 Ето, ти си научил мнозина, И немощни ръце си укрепил.
\par 4 Твоите думи са заякчили колебаещия, И отслабнали колене си укрепил.
\par 5 А сега това дойде на тебе, и ти е дотегнало; Допира те, и смутил си се.
\par 6 В страха ти от Бога не е ли твоето упование, И в правотата на пътищата ти твоята надежда?
\par 7 Спомни си, моля, кой някога е погивал невинен, Или где са били изтребени праведните.
\par 8 До колко съм аз видял, ония, които орат беззаконие, И сеят нечестие, това и жънат.
\par 9 Изтребват се от дишането на Бога, И от духането на ноздрите Му погиват.
\par 10 Ревът на лъва и гласът на свирепия лъв замират, И зъбите на младите лъвове се изкъртват.
\par 11 Лъвът загива от нямане лов, И малките на лъвицата се разпръсват.
\par 12 Тайно достига до мене едно нещо, И ухото ми долови един шепот от него:
\par 13 Всред мислите от нощните видения, Когато дълбок сън напада човеците,
\par 14 Ужас ме обзе, и трепет, И разтърси всичките ми кости;
\par 15 Тогава дух премина пред мене; Космите на тялото ми настръхнаха;
\par 16 Той застана, но не можах да позная образа му; Призрак се яви пред очите ми; В тишина чух тоя глас:
\par 17 Ще бъде ли смъртен човек праведен пред Бога? Ще бъде ли човека чист пред Създателя си?
\par 18 Ето, Той не се доверява на слугите Си, И на ангелите Си намира недостатък, -
\par 19 Колко повече в ония, които живеят в къщи от кал. Чиято основа е в пръстта, И които се смазват като че ли са молци!
\par 20 Между заранта и вечерта се събират, Без да усети някой загубват се за винаги.
\par 21 Величието, което е в тях, не се ли премахва? Умират и то без мъдрост.

\chapter{5}

\par 1 Извикай сега; има ли някой да ти отговори? И към кого от светите духове ще се обърнеш?
\par 2 Наистина гневът убива безумния, И негодуванието умъртвява глупавия.
\par 3 Аз съм виждал безумният като се е вкоренявал; Но веднага съм проклинал обиталището му;
\par 4 Защото чадата му са далеч от безопасност; Съкрушават ги по съдилищата И няма кой да ги отърве;
\par 5 Гладният изяжда жътвата им, Граби я даже изсред тръните; И грабителят поглъща имота им.
\par 6 Защото скръбта не излиза от пръстта, Нито печалта пониква из земята;
\par 7 Но човек се ражда за печал, Както искрите, за да хвъркат високо,
\par 8 Но аз Бог ще потърся, И делото си ще възложа на Бога,
\par 9 Който върши велики и неизлечими дела И безброй чудеса;
\par 10 Който дава дъжд по лицето на земята, И праща води по нивите;
\par 11 Който възвишава смирените, И въздига в безопасност нажалените,
\par 12 Който осуетява кроежите на хитрите, Така щото ръцете им не могат да извършат предприятието си;
\par 13 Който улавя мъдрите в лукавството им, Тъй че намисленото от коварните се прекатурва.
\par 14 Денем посрещат тъмнина, И по пладне пипат както нощем.
\par 15 Но Бог избавя сиромаха от меча, който е устата им, И от ръката на силния;
\par 16 И така сиромахът има надежда, А устата на беззаконието се запушват.
\par 17 Ето, блажен е оня човек, когото Бог изобличава; Затова не презирай наказанието от Всемогъщия;
\par 18 Защото Той наранява, Той и превързва; Поразява, и Неговите ръце изцеляват.
\par 19 В шест беди ще те избави; Дори в седмата няма да те досегне зло.
\par 20 В глад ще те откупи от смърт, И във война от силата на меча.
\par 21 От бича на език ще бъдеш опазен, И не ще се уплашиш от погибел, когато дойде.
\par 22 На погибелта и на глада ще се присмиваш, И не ще се уплашиш от земните зверове;
\par 23 Защото ще имаш спогодба с камъните на полето; И дивите зверове ще бъдат в мир с тебе.
\par 24 И ще познаеш, че шатърът ти е в мир; И когато посетиш кошарата си, няма да намериш да ти липсва нещо.
\par 25 Ще познаеш още, че е многочислено твоето потомство, И рожбите ти като земната трева.
\par 26 В дълбока старост ще дойдеш на гроба си, Както се събира житен сноп на времето си.
\par 27 Ето, това издигахме; така е; Слушай го, и познай го за своето добро.

\chapter{6}

\par 1 А Иов в отговор рече: -
\par 2 Дано само би се претеглила моята печал, И злополуката ми да би се турила срещу нея на везните!
\par 3 Понеже сега би била по-тежка от морския пясък; Затова думите ми са били необмислени.
\par 4 Защото стрелите на Всемогъщия са вътре в мене, Чиято отрова духът ми изпива; Божиите ужаси се опълчват против мене.
\par 5 Реве ли дивият осел, когато има трева? Или мучи ли волът при яслите?
\par 6 Яде ли се блудкавото без сол? Или има ли вкус в белтъка на яйцето?
\par 7 Душата ми се отвращава да ги допре; Те ми станаха като омразно ястие.
\par 8 Дано получех това, което прося, И Бог да ми дадеше онова, за което копнея! -
\par 9 Да благоволеше Бог да ме погуби, Да пуснеше ръката Си та ме посече!
\par 10 Но, това ще ми бъда за утеха, (Да! ще се утвърдя всред скръб, която не ме жали). Че аз не утаих думите на Светия.
\par 11 Каква е силата та да чакам? И каква е сетнината ми та да издържа?
\par 12 Силата ми сила каменна ли е? Или месата ми са медни?
\par 13 Не изчезна ли в мене помощта ми? И не отдалечи ли се от мене избавлението?
\par 14 На оскърбения трябва да се покаже съжаление от приятеля му, Даже ако той е оставил страха от Всемогъщия.
\par 15 Братята ми ме измамиха като поток; Преминаха като течение на потоци,
\par 16 Които се мътят от леда, И в които се топи снегът;
\par 17 Когато се стоплят изчезват; Когато настане топлина изгубват се от мястото си;
\par 18 Керваните, като следват по криволиченията им, Пристигат в пустота и се губят;
\par 19 Теманските кервани прегледваха; Шевските пътници ги очакваха;
\par 20 Излъгаха се в надеждата си; Дойдоха там и се посрамиха;
\par 21 Сега и вие сте така никакви; Видяхте ужас, и се уплашихте.
\par 22 Рекох ли аз: Донесете ми? Или: Дайте ми подарък от имота си?
\par 23 Или: Отървете ме от ръката на неприятеля? Или: Откупете ме от ръката на насилниците?
\par 24 Научете ме, и аз ще млъкна; И покажете ми в що съм съгрешил.
\par 25 Колко са силни справедливите думи! Но вашите доводи що изобличават?
\par 26 Мислите ли да изобличите думи, Когато думите на човек окаян са като вятър?
\par 27 Наистина вие бихте впримчили сирачето, Бихте копали яма на неприятеля си.
\par 28 Сега, прочее, благоволете да ме погледнете, Защото ще стане явно пред вас ако аз лъжа
\par 29 Повърнете се, моля; нека не става неправда; Да! повърнете се пак; касае се до правдивостта ми.
\par 30 Има ли неправда в езика ми? Не може ли небцето ми да познае лошото?

\chapter{7}

\par 1 Земният живот на човека не е ли воюване? И дните му не са ли като дните на наемник?
\par 2 Като на слуга, който желае сянка, И както на наемник, който очаква заплатата си,
\par 3 Така на мене се даде за притежание месеци на разочарование, И нощи на печал ми се определиха.
\par 4 Когато си лягам, казвам: Кога ще стана? Но нощта се протака; И непрестанно се тласкам насам натам до зори.
\par 5 Снагата ми е облечена с червеи и пръстени буци; кожата ми се пука и тлее.
\par 6 Дните ми са по-бързи от совалката на тъкача, И чезнат без надежда.
\par 7 Помни, че животът ми е дъх; И че окото ми няма да са върне да види добро.
\par 8 Окото на оногова, който ме гледа, няма да ме види вече; Твоите очи ще бъдат върху мене, а, ето, не ще ме има.
\par 9 Както облакът се разпръсва и изчезва, Така и слизащият в преизподнята няма да възлезе пак;
\par 10 Няма да се върне вече у дома си. И мястото му няма да го познае вече.
\par 11 Затова аз няма да въздържа устата си; Ще говоря в утеснението на духа си; Ще плача в горестта на душата си.
\par 12 Море ли съм аз, или морско чудовище, Та туряш над мене стража?
\par 13 Когато си казвам: Леглото ми ще ме утеши, Постелката ми ще облекчи оплакването ми,
\par 14 Тогава ме плашиш със сънища, И ме ужасяваш с видения;
\par 15 Така, че душата ми предпочита удушване И смърт, а не тия мои кости.
\par 16 Додея ми се; не ща да живея вечно; Оттегли се от мене, защото дните ми са суета.
\par 17 Що е човек, та да го възвеличаваш, И да си наумяваш за него,
\par 18 Да го посещаваш всяка заран, И да го изпитваш всяка минута?
\par 19 До кога не ще отвърнеш погледа Си от мене, И не ще ме оставиш ни колкото плюнката си да погълна?
\par 20 Ако съм съгрешил, що правя с това на Тебе, о Наблюдателю на човеците? Защо си ме поставил за Своя прицел, Така щото станах тегоба на себе си?
\par 21 И защо не прощаваш престъплението ми, И не отнемеш беззаконието ми? Защото още сега ще спя в пръстта; И сутринта ще ме търсиш, а няма да ме има.

\chapter{8}

\par 1 Тогава шуахецът Валдад в отговор каза:
\par 2 До кога ще говориш така, И думите на устата ти ще бъдат като силен вятър?
\par 3 Бог променя ли съда? Или Всемогъщият променя правдата?
\par 4 Ако Му са съгрешили чадата ти, И Той ги е предал на последствията от беззаконието им;
\par 5 Ако би ти прилежно потърсил Бога, Ако би се помолил на Всемогъщия,
\par 6 Тогава, ако беше ти чист и праведен, Непременно сега Той би се събудил да работи за тебе, И би направил да благоденствува праведното ти жилище;
\par 7 И при все да е било малко началото ти, Пак сетнините ти биха се уголемили много.
\par 8 Понеже, попитай, моля, предишните родове, И внимавай на изпитаното от бащите им;
\par 9 (Защото ние сме вчерашни и не знаем нищо, Тъй като дните на земята са сянката);
\par 10 Не щат ли те да те научат и да ти явят, И да произнесат думи от сърцата си?
\par 11 Никне ли рогоза без тиня? Расте ли тръстиката без вода?
\par 12 Догде е зелена и неокосена Изсъхва преди всяка друга трева.
\par 13 Така са пътищата на всички, които забравят Бога; И надеждата на нечестивия ще загине.
\par 14 Надеждата му ще се пресече; Упованието му е паяжина.
\par 15 Той ще се опре на къщата си, но тя не ще устои; Ще се хване за нея, но не ще утрае.
\par 16 Той зеленее пред слънцето, И клончетата му се простират в градината му;
\par 17 Корените му се сплитат в грамадата камъни; Той гледа на камъните като дом;
\par 18 Но пак, ако го изтръгне някой от мястото му, Тогава мястото ще се отрече от него, казвайки: Не съм те видял.
\par 19 Ето, това е радостта на пътя му! И от пръстта други ще поникнат.
\par 20 Ето, Бог няма да отхвърли непорочния, Нито ще подири ръката на злотворците.
\par 21 Все пак ще напълни устата ти със смях И устните ти с възклицание.
\par 22 Ония, които те мразят, ще се облекат със срам; И шатърът на нечестивите няма вече да съществува.

\chapter{9}

\par 1 А Иов в отговор рече:
\par 2 Наистина зная, че това е така, Но как ще се оправдае човек пред Бога?
\par 3 Ако поиска да се съди с Него, Не може да му отговори за едно от хиляда.
\par 4 Мъдро сърце и мощна сила има Бог; Кой, като е упорствувал против Него, е благоденствувал?
\par 5 Той премества планините и те не усещат Когато ги е превърнал в гнева Си.
\par 6 Той поклаща земята от мястото й, Тъй щото и стълбовете й треперят.
\par 7 Той заповядва на слънцето, и не изгрява; И туря под печат звездите.
\par 8 Той сам простира небесата, И стъпва на морските вълни.
\par 9 Той прави съзвездията - Мечката, Ориона и Плеядите, И скритите пространства на юг.
\par 10 Той прави велики и неизследими дела. И безбройни чудеса.
\par 11 Ето, минава край мене, и не Го виждам; Преминава и не Го съглеждам;
\par 12 Ако грабна плячка, кой ще Му забрани? Кой ще Му рече: Що правиш?
\par 13 Ако Бог не оттегли гнева Си, Горделивите помощници се повалят под Него!
\par 14 Колко по-малко бих могъл аз да Му отговоря И да избера думите си, за да разисквам с Него!
\par 15 Комуто, и праведен ако бях, не можех отговори, Но щях да повярвам, че е послушал гласа ми.
\par 16 Ако извиках, и ми отговореше, Не щях да повярвам, че е послушал гласа ми.
\par 17 Защото ме смазва с вихрушка, И умножава раните ми без причина.
\par 18 Не ме оставя да си отдъхна, Но ме насища с горчивини.
\par 19 Ако е дума за силата на мощните; Ето ме! Би казал Той; И ако за съд, би казал: Кой ще Ми определи време да съдя?
\par 20 Даже ако бях праведен, осъдили ме биха собствените ми уста; Ако бях непорочен, Той би ме показал опърничав.
\par 21 Макар да бях непорочен, не бих зачитал себе си, Презрял бих живота си.
\par 22 Все едно е; затова казвам: Той погубва и непорочния и нечестивия,
\par 23 Ако бичът Му убива внезапно, Той се смее при изпитанията на невинните.
\par 24 Земята е предадена в ръцете на нечестивите; Той покрива лицата на съдиите; Ако не, тогава кой е, който прави това?
\par 25 А моите дни са по-бързи от бързоходец; Бягат без да видят добро;
\par 26 Преминаха като леки кораби, Като орел, който се спуща върху лова.
\par 27 Ако река: Ще забравя оплакването си, Ще оставя желанието си, и ще се утеша.
\par 28 В ужас съм от всичките си скърби Зная, че Ти няма да ме имаш за невинен;
\par 29 Нечестив ще се считам; Защо, прочее, да се трудя напразно?
\par 30 Ако се умия със снежна вода, И очистя със сапун ръцете си,
\par 31 Ти пак ще ме хвърлиш в тинята, Така щото и самите ми дрехи ще се гнусят от мене.
\par 32 Защото Той не е човек, както съм аз, та да Му отговоря И да дойдем заедно на съд.
\par 33 Няма посредник помежду ни, Който да тури ръката си върху двама ни,
\par 34 Нека оттегли от мене тоягата Си, И ужасът Му да не ме уплашва.
\par 35 Тогава ще говоря, и няма да се боя от Него; Защото в себе си не съм така уплашен.

\chapter{10}

\par 1 Душата ми се отегчи от живота ми; За това, ще се предам на оплакването си, Ще говоря в горестта на душата си.
\par 2 Ще река Богу: Недей ме осъжда; Покажи ми защо ми ставаш противен.
\par 3 Добре ли Ти е да оскърбяваш, И да презираш делото на ръцете Си, А да осветляваш съвещаното от нечестивите?
\par 4 Телесни ли очи имаш? Или гледаш както гледа човек?
\par 5 Твоите дни като дните на човека ли са, Или годините Ти като човешки дни,
\par 6 Та претърсваш беззаконието ми И издирваш греха ми,
\par 7 При все че знаеш, че не съм нечестив, И че никой не може да ме избавя от ръката Ти?
\par 8 Твоите ръце ме създадоха и усъвършенствуваха Кръгло в едно; а пак съсипваш ли ме?
\par 9 Помни, моля, че като глина си ме създал; И в пръст ли ще ме възвърнеш?
\par 10 Не си ли ме излял като мляко? Не си ли ме съсирил като сирене?
\par 11 С кожа и мускули си ме облякъл, И с кости и жили си ме оплел;
\par 12 Живот и благоволение си ми подарил, И провидението Ти е запазило духа ми.
\par 13 Но при все туй, това си криел в сърцето Си; Зная, че това е било в ума Ти;
\par 14 Ако съгреша, наблюдаваш ме, И от беззаконието ми няма да ме считаш невинен,
\par 15 Ако съм нечестив, горко ми! И ако съм праведен, пак няма да дигна главата си. Пълен съм с позор; но гледай Ти скръбта ми,
\par 16 Защото расте. Гониш ме като лъв, И повтаряш да се показваш страшен против мене.
\par 17 Повтаряш да издигаш против мене свидетелите Си, И увеличаваш гнева Си върху мене; Едно подир друго войнства ме нападат.
\par 18 Защо прочее ме извади Ти из утробата? Иначе, бих издъхнал без да ме е виждало око;
\par 19 Бих бил като че не съм бил; От утробата бих бил отнесен в гроба.
\par 20 Дните ми не са ли малко? Престани, прочее, И остави ме да си отдъхна малко
\par 21 Преди да отида отдето няма да се върна, В тъмната земя и в смъртната сянка, -
\par 22 Земя, мрачна като самата тъмнина, Земя на мрачна сянка и без никакъв ред, Дето виделото е като тъмнина.

\chapter{11}

\par 1 Тогава нааматецът Софар, в отговор рече: -
\par 2 Не трябва ли да се отговори на многото думи? Бива ли да се оправдае словоохотлив човек?
\par 3 Твоите самохвалства ще запушат ли хорските уста? И когато ти се присмиваш, тебе никой да не засрами ли?
\par 4 Защото ти казваш: Това, което говоря, е право, И аз съм чист пред Твоите очи.
\par 5 Но дано проговореше Бог, И да отвореше устните Си против тебе.
\par 6 И да ти явеше тайните на мъдростта, Че тя е двояка в проницателността си, Знай, прочее, че Бог изисква от тебе по-малко, отколкото заслужава беззаконието ти.
\par 7 Можеш ли да изброиш Божиите дълбочини? Можеш ли да издириш Всемогъщия напълно?
\par 8 Тия тайни са високи до небето; що можеш да сториш? По-дълбоки са от преизподнята; що можеш да узнаеш?
\par 9 Мярката им е по-дълга от земята И по-широка от морето.
\par 10 Ако мине Той та улови и събере съд, То кой може да Му забрани?
\par 11 Защото Той знае суетните човеци, Той вижда и нечестието, без да Му е нужно да внимава в него.
\par 12 Но суетният човек е лишен от разум; Дори, човек се ражда като диво оселче.
\par 13 Ако управиш ти сърцето си, И простреш ръцете си към Него,
\par 14 Ако има беззаконие в ръцете ти, отстрани го, И не оставяй да обитава нечестие в шатрите ти.
\par 15 Тогава само ще издигнеш лицето си без петно, Да! утвърден ще бъдеш, и няма да се боиш;
\par 16 Защото ще забравиш скръбта си; Ще си я спомняш като води, които са оттекли.
\par 17 Твоето пребивание ще бъде по-светло от пладне; И тъмнина ако си, пак ще станеш като зора.
\par 18 Ще бъдеш в увереност, защото има надежда; Да! Ще се озърнеш наоколо, и ще си легнеш безопасно.
\par 19 Ще легнеш, и не ще има кой да те плаши; Дори мнозина ще търсят твоето благоволение.
\par 20 А очите на нечестивите ще изтекат; Прибежище не ще има за тях; И надеждата им ще бъде, че ще издъхват.

\chapter{12}

\par 1 Тогава Иов в отговор рече: -
\par 2 Наистина само вие сте люде, И с вас ще умре мъдростта!
\par 3 Но и аз имам разум както и вие; Не съм по-долен от вас; И такива работи, кой ги не знае?
\par 4 Станах за поругание на ближния си, Човек, който призовавах Бога, и Той му отговаряше, - Праведният, непорочният човек стана за поругание!
\par 5 Тоя, чиито нозе са близо до подхлъзване, Е като презрян светилник в мисълта на благополучния.
\par 6 Шатрите на разбойниците са в благоденствие, И тия, които разгневяват Бога, са в безопасност; Бог докарва изобилие в ръцете им.
\par 7 Но попитай сега животните, и те ще те научат, И въздушните птици, и те ще ти кажат;
\par 8 Или говори на земята, и тя ще те научи, И морските риби ще ти изявят.
\par 9 От всички тия кой не разбира, Че ръката на Господа е сторила това? -
\par 10 В Чиято ръка е душата на всичко живо, И дишането на цялото човечество.
\par 11 Ухото не изпитва ли думите Както небцето вкусва ястието си?
\par 12 Мъдростта е у белокосите, казвате вие, И разумът в дългия живот.
\par 13 А у Бога е мъдростта и силата; Той има разсъждение и разум.
\par 14 Ето, Той събаря, и не съгражда вече; Затваря човека, и не му се отваря.
\par 15 Ето, задържа водите, и пресъхват; Пуща ги пак, и изравят земята.
\par 16 У Него е силата и мъдростта; Негов е измаменият и измамникът.
\par 17 Закарва съветниците ограбени, И прави съдиите глупави.
\par 18 Разпасва пояса на царете, И опасва кръста им с въже.
\par 19 Закарва първенците ограбени, И поваля силните.
\par 20 Отнема думата от ползуващите се с доверие, И взема ума на старейшините.
\par 21 Излива презрение върху князете, И ослабва силата на яките.
\par 22 Открива дълбоки работи из тъмнината, И изважда на видело мрачната сянка.
\par 23 Умножава народите, и погубва ги, Разширява народите, и стеснява ги.
\par 24 Отнема бодростта на началниците на земните жители, И прави ги да се скитат по непроходна пустиня;
\par 25 Пипат в тъмнината без виделина, И прави ги да залитат като пиян.

\chapter{13}

\par 1 Ето, моето око е видяло всичко това, Ухото ми е чуло и го е разбрало.
\par 2 Което знаете вие, това зная и аз; Не съм по-долен от вас.
\par 3 Но аз бих говорил на Всемогъщия, И желая да разисквам с Бога;
\par 4 Защото вие сте измислители на лъжа; Всинца сте безполезни лекари.
\par 5 Дано млъкнехте съвсем! И това щеше да ви бъде за мъдрост.
\par 6 Слушайте сега доводите ми, И дайте внимание в жалбата на устните ми.
\par 7 Заради Бога несправедливо ли ще говорите? Заради Него измама ли ще изкажете?
\par 8 Заради Него пристрастие ли ще покажете? Заради Бога ще се съдите ли?
\par 9 Добро ли е да ви изпита Той? Или ще можете да Го излъжете, както лъжат човека?
\par 10 Той непременно ще ви изобличи, Ако тайно показвате пристрастие.
\par 11 Величието Му не ще ли ви уплаши? И ужасът Му не ще ли да ви нападне?
\par 12 Вашите паметни думи стават пред Него поговорки от пепел; Защитата ви става укрепление от кал.
\par 13 Млъкнете! оставете ме и аз да говоря; И нека дойде върху мене каквото ще.
\par 14 Каквото и да стане, ще взема месата си в зъбите си, И ще туря живота в шепата си.
\par 15 Ако и да ме убие Той, аз ще Го чакам; Но пак ще защитя пътищата си пред Него.
\par 16 Даже това ще ми бъде спасение, Че нечестив човек няма да дойде пред Него.
\par 17 Послушайте внимателно думата ми; И изявлението ми нека бъде в ушите ви.
\par 18 Ето сега, аз съм наредил делото си; Зная, че ще се оправдая.
\par 19 Кой е оня, който ще се съди с мене? Защото, ако млъкна, сега ще издъхна.
\par 20 Само две неща не ми направяй, Тогава не ще се скрия от лицето Ти, -
\par 21 Не отказвай да оттеглиш ръката Си от мене, И нека ме не уплаши ужасът Ти.
\par 22 Тогава Ти повикай, и аз ще Ти отговоря; Или аз да говоря, и ти ми отговори.
\par 23 Колко са беззаконията ми и греховете ми? Яви ми престъплението ми и греха ми.
\par 24 Защо криеш лицето Си, И ме считаш за Свой неприятел?
\par 25 Ще изморяваш ли лист отвяван? И ще гониш ли суха плява?
\par 26 Защо пишеш горести против мене, И ме правиш да наследявам беззаконията на младостта си,
\par 27 И туряш нозете ми в клада, И наблюдаваш всичките ми пътища, Забелязваш дирите на нозете ми? -
\par 28 При все, че аз като гнила вещ тлея, Като дреха от молец изядена.

\chapter{14}

\par 1 Човекът, роден от жена е кратковременен И пълен със смущение.
\par 2 Цъфти като цвят, и се покосява; Бяга като сянка, и не се държи.
\par 3 И върху такъв ли отваряш очите Си, И ме караш на съд с Тебе?
\par 4 Кой може да извади чисто от нечисто? Никой.
\par 5 Тъй като дните му са определени, И числото на месеците му е у Тебе, И Ти си поставил границите му, които не може да премине,
\par 6 Отвърни погледа Си от него, за да си почине, Догде като наемник доизкара деня си.
\par 7 Защото за дървото има надежда, Че, ако се отсече, пак ще поникне, И че издънката му няма да изчезне,
\par 8 Даже ако коренът му остарее в земята, И ако пънът му умре в пръстта;
\par 9 Понеже от дъха на водата ще поникне, И ще покара клончета като новопосадено.
\par 10 Но човек умира и прехожда; Да! Човек издъхва, и де го?
\par 11 Както водите чезнат из морето, И реката престава и пресъхва,
\par 12 Така човек ляга, и не става вече; Докато небесата не преминат, те няма да се събудят, И няма да станат от съня си.
\par 13 О, дано ме скриеше Ти в преизподнята, Да ме покриеше, догде премине гневът Ти, Да ми определеше срок, и тогава да би ме спомнил!
\par 14 Ако умре човек, ще оживее ли? През всичките дни на воюването си ще чакам, Докато дойде промяната ми.
\par 15 Ще повикнеш, и аз ще Ти се отзова; Ще пожелаеш делото на ръцете Си.
\par 16 А сега броиш стъпките ми; Не наблюдаваш ли греховете ми?
\par 17 Престъплението ми е запечатано в мешец, И зашиваш беззаконието ми.
\par 18 Наистина както и планината като пада унищожава се, И скалата се премества от мястото си;
\par 19 Както водите изтриват камъните; И наводненията им завличат пръстта от земята; Така Ти погубваш надеждата на човека.
\par 20 Надделяваш всякога над него, и той прехожда; Изменяваш лицето му, и го отпращаш.
\par 21 Синовете му достигат до почитание, а той не знае; И биват свалени, а той не забелязва това за тях;
\par 22 Знае само, че снагата му е за него в болки, И душата му е за него в жалеене.

\chapter{15}

\par 1 Тогава теманецът Елифад в отговор рече:
\par 2 Мъдър човек с вятърничаво ли знание отговаря, И с източен вятър ли пълни корема си?
\par 3 С празни думи ли се препира И с безполезни речи?
\par 4 Наистина ти унищожаваш страха от Бога, И намаляваш моленето пред Него.
\par 5 Защото беззаконието ти поучава устата ти, И си избрал езика на лукавите.
\par 6 Твоите уста те осъждат, а не аз; Твоите устни свидетелствуват против тебе.
\par 7 Ти ли си първородният човек? Или създаден ли си преди хълмите?
\par 8 Чул ли си ти Божиите тайни намерения? Или си заключил в себе си мъдростта?
\par 9 Що знаеш ти, което ние не знаем? Що разбираш ти, което няма у нас?
\par 10 Има и между нас и белокоси и престарели, По-напреднали на възраст и от баща ти.
\par 11 Божиите утешения и меките Му към тебе думи Малко нещо ли са за тебе?
\par 12 Какво те блазни сърцето ти, И на какво смигат очите ти,
\par 13 Та обръщаш духа си против Бога, И изпущаш такива думи из устата си?
\par 14 Що е човек та да е чист, И роденият от жена та да е праведен?
\par 15 Ето, на светите Си ангели Той не се доверява, И небесата не са чисти в очите Му;
\par 16 Колко повече е гнусен и непотребен човек, Който пие неправда, като вода!
\par 17 Аз ще ти кажа, послушай ме; И това, което съм видял, ще ти изявя,
\par 18 (Което мъдрите не скриха, но възвестиха, Както бяха чули от бащите си;
\par 19 На които биде дадена земята, и само на тях, И чужденец не замина между тях;)
\par 20 Нечестивият се мъчи през всичките си дни; И преброени години са запазени за мъчителя.
\par 21 Ужасни гласове има в ушите му, Че като е в спокойствие ще го нападне изтребителят;
\par 22 Не вярва, че ще се върне от тъмнината; И той е очакван от ножа;
\par 23 Скита се да търси хляб, казвайки: Где е? Знае, че денят на тъмнината е готов до ръката му;
\par 24 Скръб и тъга го плашат, Като цар приготвен за бой му надвиват.
\par 25 Понеже той простря ръката си против Бога, И възгордя се против Всемогъщия,
\par 26 Спусна се на Него с корав врат, С дебелите изпъкналости на щитовете си.
\par 27 Понеже покри лицето си с тлъстината си, И, затлъсти кръста си,
\par 28 Той се засели в разорени градове, В къщи необитаеми, Готови да станат на купове.
\par 29 Няма да се обогати, и имотът му няма да трае, Нито ще се навеждат до земята произведенията им.
\par 30 Няма да се отърве от тъмнината; Пламък ще изсуши младоците му; И от дишането на Божиите уста ще бъде завлечен.
\par 31 Нека не се доверява на суетата, самоизмамен; Защото суета ще бъде заплатата му.
\par 32 Преди времето си ще се изплати, И клонът му няма да раззеленее,
\par 33 Ще изрони неузрялото си грозде като лозата, И ще хвърли цвета си като маслината.
\par 34 Защото дружината на нечестивите ще запустее; И огън ще пояде шатрите на подкупничеството.
\par 35 Зачват зло, и раждат беззаконие, И сърцето им подготвя измама.

\chapter{16}

\par 1 Тогава Иов в отговор рече:
\par 2 Много такива неща съм слушал; Окаяни утешители сте всички.
\par 3 Свършват ли се празните думи? Или що ти дава смелост, та отговаряш?
\par 4 И аз можех да говоря като вас; Ако беше вашата душа на мястото на моята душа, Можех да натрупам думи против вас И да клатя глава против вас.
\par 5 Но аз бих ви подкрепил с устата си, И утехата си от устните ми би олекчила скръбта ви.
\par 6 Ако говоря, болката ми не олеква; И ако мълча, до колко съм по-на покой?
\par 7 Но сега Той ме много умори; Ти запусти цялото ми семейство.
\par 8 Набръчкал си ме в свидетелство против мене; И мършавостта ми се издига срещу мене И заявява в лицето ми.
\par 9 Разкъсва ме в гнева Си, и ме мрази; Скърца със зъбите Си против мене; Неприятелят ми остри очите си върху мене.
\par 10 Зяпат против мене с устата си, Удрят ме по челюстта с хулене, Трупат се всички против мене.
\par 11 Бог ме предаде на неправедния, И ме хвърли в ръцете на нечестивите.
\par 12 Бях в охолност, но Той ме разкъса, Дори хвана ме за врата и строши ме, И постави ме за Своя прицел.
\par 13 Стрелците Му ме обиколиха; Пронизва бъбреците ми, и не щади; Излива жлъчката ми на земята.
\par 14 Съсипва ме с удар върху удар; Спуска се върху мене като исполин.
\par 15 Вретище съших върху кожата си, И окалях рога си в пръстта.
\par 16 Лицето ми подпухна от плач, И мрачна сянка има върху клепачите ми,
\par 17 Ако и да няма неправда в ръцете ми, И да е чиста молитвата ми.
\par 18 О земле, не покривай кръвта ми! И за вика ми да няма място за почивка.
\par 19 Ето и сега, свидетелят ми е на небеса, И свидетелството ми във височините.
\par 20 Моите приятели ми се присмиват; Но окото ми рони сълзи към Бога,
\par 21 Дано Той сам защити човека пред Бога. И човешкия син пред ближния му!
\par 22 Защото като се изминат малко години Аз ще отида на път, отгдето няма да се върна.

\chapter{17}

\par 1 Духът ми чезне, дните ми гаснат, мене вече гробът чака.
\par 2 Сигурно ми се присмиват; И окото ми трябва постоянно да гледа огорченията им!
\par 3 Дай, моля, поръчителство; стани ми поръчител при Себе Си; Кой друг би дал ръка на мене?
\par 4 Защото си скрил сърцето им от разум; Затова няма да ги възвисиш.
\par 5 Който заради плячка предава приятели - Очите на чадата му ще изтекат.
\par 6 Той ме е поставил и поговорка на людете; И укор станах аз пред тях.
\par 7 Помрачиха очите ми от скръб, И всичките ми телесни части станаха като сянка.
\par 8 Правдивите ще се почудят на това, И невинният ще се повдигне против нечестивия.
\par 9 А праведният ще се държи в пътя си, И който има чисти ръце ще увеличава силата си.
\par 10 А вие всички, моля, пак дойдете; Обаче не ще мога намери между вас един разумен.
\par 11 Дните ми преминаха; Намеренията ми и желанията на сърцето ми се пресякоха.
\par 12 Нощта скоро ще замести деня; Виделото е близо до тъмнината,
\par 13 Ако очаквам преизподнята за мое жилище, Ако съм постлал постелката си в тъмнината,
\par 14 Ако съм викнал към тлението, Баща ми си ти, - Към червеите: Майка и сестра ми сте,
\par 15 То где е сега надеждата ми? Да! кой ще види надеждата ми?
\par 16 При вратите на преизподнята ще слезе тя, Когато едновременно ще има покой в пръстта.

\chapter{18}

\par 1 Тогава шуахецът Валдад в отговор рече:
\par 2 До кога още ще ловите думи? Първо разбирайте, и после ще говорим.
\par 3 Защо сме считани като скотове, И станахме никакви пред вас?
\par 4 О ти, който разкъсваш душата си в гнева си, За тебе ли ще бъде напусната земята, И скалите ще се преместят от мястото си?
\par 5 Наистина светлината на нечестивия ще угасне, И пламъкът на огъня му няма да свети.
\par 6 Светлината ще бъде мрак в шатъра му, И светилникът при него ще изгасне,
\par 7 Силното му стъпване ще се стесни. И собствените му намерения ще го повалят.
\par 8 Защото със своите си нозе той се хвърля в мрежа, И ходи върху примки.
\par 9 Клопка ще го улови за петата, Примка ще го хване.
\par 10 Въжето му е скрито в земята, И примката на пътя му.
\par 11 Ужаси ще го плашат отвред, И ще го гонят в петите.
\par 12 Силата му ще чезне от глад, И бедствие ще бъде готово до хълбока му.
\par 13 Първородният на смъртта ще пояде членовете на тялото му. Да! ще пояде членовете му.
\par 14 Той ще бъде изкоренен от шатъра си, в който уповава, И ще бъде закаран при царя на ужасите.
\par 15 В шатъра му ще се засели това, което не е негово; Сяра ще се разпръсне върху жилището му.
\par 16 Отдолу корените му ще изсъхнат, И отгоре клоните му ще се отсекат.
\par 17 Споменът му ще се изличи от земята, И името му не ще го има вече по улиците.
\par 18 Ще бъде изпъден от светлото в тъмното, И ще бъде изгонен от света.
\par 19 Не ще има ни син, ни внук между людете си, Нито остатък в жилищата си.
\par 20 Идните поколения ще се смаят за деня му, Както и предишните се ужасиха.
\par 21 Наистина такива са жилищата на нечестивия, И това е мястото на онзи, който не познава Бога.

\chapter{19}

\par 1 Тогава Иов в отговор рече:
\par 2 До кога ще оскърбявате душата ми, И ще ме съкрушавате с думи?
\par 3 Десет пъти вече стана ме укорявате, Но пак не ви е срам, че ми смайвате главата.
\par 4 Даже ако наистина съм съгрешил, Грешката ми остава с мене.
\par 5 Ако непременно искате да се големеете над мене И да хвърляте против мене укора ми,
\par 6 Знайте сега, че Бог ме повали И ме обиколи с мрежата Си.
\par 7 Ето, викам: Неправда! Но няма кой да ме чуе; Издавам вик за помощ, но няма съд.
\par 8 Той е преградил пътя ми, та не мога да премина, И турил е тъмнина в пътеките ми,
\par 9 Съблякъл ме е от славата ми, И отнел е венеца от главата ми.
\par 10 Съкрушил ме е отвсякъде, и аз отивам; И изкоренил е надеждата ми като дърво.
\par 11 Запалил е тоже против мене гнева Си, И счита ме като един от враговете Си.
\par 12 Полковете Му настъпват заедно, Та заздравяват пътя си против мене, И разполагат се в стан около шатъра ми.
\par 13 Отдалечил е от мене братята ми; И ония, които ме познаваха, станаха съвсем чужди за мене.
\par 14 Оставиха ме ближните ми, И забравиха ме познайниците ми.
\par 15 Ония, които живеят в дома ми, И слугините ми считат ме като чужд; Странен станах в очите им.
\par 16 Викам слугата си, и не отговаря, При все че с устата си му се моля.
\par 17 Дъхът ми е отвратителен на жена ми, И дъхът ми на чадата на чреслата ми.
\par 18 И самите малки деца ме презират; Когато ставам говорят против мене.
\par 19 Всичките ми по-близки приятели се погнусяват от мене; И ония, които възлюбих, обърнаха се против мене.
\par 20 Костите ми залепват за кожата ми и за месата ми; И отървах се само с кожата на зъбите си.
\par 21 Смилете се за мене, смилете се за мене, вие приятели мои! Защото ръката Божия се допря до мене.
\par 22 Защо ме гоните като че ли сте Бог, И не се насищате от плътта ми?
\par 23 О, да можеха да се напишат думите ми! Да се начертаеха на книга!
\par 24 Да се издълбаеха на скала за всегда С желязна писалка и олово!
\par 25 Защото зная, че е жив Изкупителят ми, И че в последно време ще застане на земята;
\par 26 И, като изтлее след кожата ми това тяло, Пак вън от плътта си ще видя Бога:
\par 27 Когото сам аз ще видя, И очите ми ще гледат, и то не като чужденец, За тая гледка дробовете ми се топят дълбоко в мене.
\par 28 Ако кажете, Как ще го гоним, Тъй като причината на това страдание се намира в самия него!
\par 29 Тогава бойте се от меча; Защото гневни са наказанията, нанесени от меча, За да познаете, че има съд.

\chapter{20}

\par 1 Тогава нааматецът Софар в отговор рече:
\par 2 Понеже ме карат мислите ми да отговоря, Затова бързам.
\par 3 Чух укорително изобличение против мене; И духът на разума ме кара да отговоря.
\par 4 Не знаеш ли това от старо време, От когато е поставен човек на земята,
\par 5 Че тържеството на нечестивите е кратковременно, И радостта на безбожния е минутна?
\par 6 Макар величието му да се издигне до небето, И главата му да стигне до облаците.
\par 7 Пак той ще се изрине за винаги, както нечистотиите му; Ония, които са го гледали, ще кажат, Где е той?
\par 8 Като сън ще отлети и няма да се намери, И като нощно видение ще изчезне.
\par 9 Окото, което го е гледало, не ще го гледа вече; И мястото му няма да го види вече.
\par 10 Чадата му ще потърсят благоволението на сиромасите; И ръцете му ще повърнат имота им.
\par 11 Костите му са пълни със съгрешенията на младостта му; И те ще лежат с него в пръстта.
\par 12 Ако и да е сладко злото в устата му, Та го крие под езика си.
\par 13 Ако и да го жали и не го оставя, Но все още го държи вътре в устата си,
\par 14 Пак храната му ще се измени в червата му, На жлъчка аспидна ще се обърне във вътрешностите му.
\par 15 Погълнал е богатство, но ще го повърне; Бог ще го изтръгне из корема му.
\par 16 Отрова аспидна ще суче; Език ехиднин ще го умъртви.
\par 17 Няма вече да гледа потоците, Реките, които текат с мед и масло.
\par 18 Това, за което се трудим, ще го възвърне, И няма да се наслаждава на него; Съразмерно с имота, който е придобил, Той няма да се радва,
\par 19 Защото е угнетил сиромасите и ги е оставил; Заграбил е къща, която не бе построил.
\par 20 Понеже не е знаел насита на лакомството си, Няма да запази нищо от това, което му е най-мило;
\par 21 Понеже не остана нищо, което не изпояде, Затова благоденствието му няма да трае.
\par 22 Когато е в пълно изобилие, ще го сполети оскъдност; Ръката на всеки окаяник ще го нападне.
\par 23 Когато се кани да напълни корема си, Бог ще хвърли върху него яростния Си гняв, И ще го навали върху него, когато още яде.
\par 24 Когато бяга от желязното оръжие, Стрелата на медния лък ще го прониже.
\par 25 Той я изтръгва, и тя излиза из тялото му, Да! лъскавият й връх излиза из жлъчката му; Ужаси го обземат.
\par 26 Всякаква тъмнина е запазена за съкровищата му; Огън нераздухван от човек ще го пояде; На тия, които останат в шатъра му, зле ще им бъде.
\par 27 Небето ще открие беззаконието му, И земята ще се повдигне против него.
\par 28 Богатството на дома му ще изчезне, В деня на Божия гняв ще се разпилее.
\par 29 Това е от Бога делът на нечестивия, И определеното му от Бога наследство.

\chapter{21}

\par 1 Тогава Иов в отговор рече:
\par 2 Слушайте внимателно говоренето ми, И с това ме утешавайте.
\par 3 Потърпете ме, и аз ще говоря; А след като изговоря, присмивайте се.
\par 4 За човека ли се оплаквам аз? А как да се не утесни духът ми?
\par 5 Погледнете на мене, и почудете се, И турете ръка на устата си.
\par 6 Само да си наумя тия въпроси ужасявам се, И трепет обзема снагата ми.
\par 7 Защо живеят нечестивите, Остаряват; даже стават и много силни.
\par 8 Чадата им се утвърждават заедно с тях пред лицето им, И внуците им пред очите им.
\par 9 Домовете им са свободни от страх; И Божията тояга не е върху тях.
\par 10 Говедата им се гонят и не напразно; Юницата им се тели, и не помята.
\par 11 Пущат чадата си като овце; И децата им скачат.
\par 12 Пеят при музиката на тъпанчето и арфата, И веселят се при звука на свирката.
\par 13 Прекарват дните си в благополучие; И в една минута слизат в гроба.
\par 14 Все пак казват Богу: Оттегли се от нас, Защото не искаме да знаем пътищата Ти.
\par 15 Що е Всемогъщият, та да Му служим? И какво се ползваме, като Го призоваваме?
\par 16 Ето, щастието им не е в тяхна ръка; Далеч да бъде от мене мъдруването на нечестивите!
\par 17 Колко често изгасва светилникът на нечестивите, И дохожда бедствието им върху тях! Бог им разпределя болезни в гнева Си.
\par 18 Те са като плява пред вятъра, И като прах от плява, който вихрушката отвява.
\par 19 Думате, Бог пази наказанието на тяхното беззаконие за чадата им. По-добре нека въздаде на сами тях, за да го усещат;
\par 20 Собствените им очи нека видят гибелта им, И сами те нека пият от гнева на Всемогъщия.
\par 21 Защото какво наслаждение от дома си има нечестивият след себе си, Когато се преполови числото на месеците му?
\par 22 Ще научи ли някой Бога на знание, Тъй като Той съди високите?
\par 23 Един умира в пълно благополучие, Като е във всичко охолен и спокоен;
\par 24 Ребрата му са покрити с тлъстина, И костите му са напоени с мозък.
\par 25 А друг умира в душевна горест, Като никога не е ял с весело сърце.
\par 26 Заедно лежат в пръстта, И червеи ги покриват.
\par 27 Ето, зная мислите ви И хитруванията ви за съсипването ми.
\par 28 Защото думате: Где е къщата на княза? И где е шатърът, гдето живееха нечестивите?
\par 29 Не сте ли попитали минаващите през пътя? И не разбирате ли бележитите им примери, -
\par 30 Че нечестивият се пази за ден на погибел, И че в ден на гняв ще бъде закаран?
\par 31 Кой ще изяви пред лицето му неговия път? И кой ще му въздаде за онова, което е сторил?
\par 32 Но и той ще бъде донесен в гроба, И ще пази над гробницата си.
\par 33 Буците на долината ще му бъдат леки; И всеки човек ще отиде подир него, Както безбройни са отишли преди него,
\par 34 Как, прочее, ми давате празни утешения, Тъй като в отговорите ви остава само лъжа?

\chapter{22}

\par 1 Тогава теманецът Елифаз в отговор рече:
\par 2 Може ли човек да бъде полезен Богу? Ако разумен може да бъде полезен на семе си.
\par 3 Ако си ти праведен, Всемогъщият има ли за какво да се радва? Или ползува ли се Той, ако правиш пътищата си непорочни?
\par 4 Поради твоя ли страх от Него Той те изобличава, И влиза в съд с тебе?
\par 5 Нечестието ти не е ли голямо? И беззаконията ти не са ли безкрайни?
\par 6 Защото без причина си взел залог от брата си. И си лишил голите от дрехите им,
\par 7 Не си напоил с вода уморения, И си задържал хляб от гладния.
\par 8 А който беше як, той придобиваше земята; И който беше почитан, той се заселваше в нея.
\par 9 Вдовици си отпратил празни, И мишците на сирачетата си строшил.
\par 10 За това примки те обикалят, И страх внезапен те ужасява,
\par 11 Или тъмнина, та не виждаш, И множество води те покрива.
\par 12 Бог не е ли на небесните висоти? Сега гледай височината на звездите, колко са на високо!
\par 13 А ти казваш: Где ще знае Бог? През мрака ли може да съди?
\par 14 Облаци Го покриват, та не вижда: И ходи по свода небесен,
\par 15 Забележил ли си ти стария път, По който са ходили беззаконниците? -
\par 16 Тия, които преждевременно бидоха грабнати, И чиято основа порой завлече,-
\par 17 Които рекоха Богу: Отдалечи се от нас, И - какво може Всемогъщият да стори за нас?-
\par 18 При все, че Той напълни с блага домовете им. Но далеч да бъде от мене мъдруването на нечестивите!
\par 19 Праведните гледат и се радват; И невинните им се присмиват, като казват:
\par 20 Не бяха ли погубени въстаналите против нас, И огън погълна останалите от тях?
\par 21 Сприятели се сега с Него и бъди в мир; От това ще дойде добро за тебе.
\par 22 Приеми, прочее, закона от устата Му, И съхрани думите Му в сърцето си.
\par 23 Ако се върнеш към Всемогъщия, пак ще бъдеш утвърден; Отдалечи, прочее, беззаконието от шатрите си,
\par 24 Хвърли злото си в пръстта, И офирското злато между камъните на потоците;
\par 25 И Всемогъщият ще ти бъде злато, И изобилие от сребро за тебе.
\par 26 Защото тогава ще се веселиш във Всемогъщия, И ще въздигаш лицето си към Бога.
\par 27 Ще Му се помолиш, и Той ще те послуша; И ще изпълни обреците Си.
\par 28 И каквото решение направиш, ще ти бъде потвърдено; И светлина ще сияе по пътищата ти.
\par 29 Когато те унижат, Тогава ще речеш: Има въздигане! И Той ще спаси онзи, който има смирен поглед.
\par 30 Даже онзи, който не е невинен, ще избави; Да! с чистотата на твоите ръце ще бъде избавен.

\chapter{23}

\par 1 А Иов в отговор рече:
\par 2 И днес оплакването ми е горчиво; Ръката ми е по-тежка от въздишането ми.
\par 3 Ах, да бих знаел где да Го намеря! Отишъл бих до престола Му,
\par 4 Изложил бих делото си пред Него, И напълнил бих устата си с доводи,
\par 5 Узнал бих думите, които Той би ми отговорил, И разбрал бих какво щеше да ми рече.
\par 6 Щеше ли Той да се препира с мене с голямата Си сила? Не! щеше само да внимава в мен.
\par 7 Тогава би станало явно, че един праведник разисква с Него; И така аз бих се освободил за винаги от Съдията си.
\par 8 Обаче, ето, отивам напред, но няма Го, И назад, но не го виждам,
\par 9 Наляво, гдето работи, но не мога да Го видя; Крие се надясно, и Го не виждам.
\par 10 Знае, обаче, пътя ми; когато ме изпита, Ще изляза като злато.
\par 11 Ногата ми се е държала здраво в Неговите стъпки; Опазил съм пътя Му без да се отклоня;
\par 12 От заповедите на устните Му не съм се оттеглил назад; Съхранил съм думите на устата Му повече от нужната си храна.
\par 13 Но Той е на един ум, и кой може да Го отвърне? И каквото желае душата Му, това прави.
\par 14 Защото върши това, което е определено за мене; И много такива неща има у Него.
\par 15 Затова, смущавам се в присъствието Му; Когато размишлявам треперя от Него.
\par 16 Защото сам Бог е разслабил сърцето ми, И Всемогъщият ме е смутил;
\par 17 Тъй като не по причина на тъмнината се отсичат думите ми Нито по причина на мрака, който покрива лицето ми.

\chapter{24}

\par 1 Защо, ако времената не са открити от Всемогъщия, Ония които Го познават, не виждат дните Му за съд?
\par 2 Едни преместят межди, Грабят стада и ги пасат;
\par 3 Откарват осела на сирачетата; Вземат в залог говедото на вдовицата;
\par 4 Изтласкват бедните от пътя; Сиромасите на земята се крият заедно от тях.
\par 5 Ето, като диви осли в пустинята излизат по работата си, Подраняват да търсят храна; Пустинята из доставя храна за чадата им.
\par 6 Жънат фуража в нивата, за да го ядат. И берат лозата на неправедника;
\par 7 Цяла нощ лежат голи без дрехи, И нямат завивка в студа;
\par 8 Измокрюват се от планинските дъждове, И прегръщат скалата, понеже нямат прибежище.
\par 9 Други грабват сирачето от съседите, И вземат залог от сиромаха.
\par 10 Голи, тия ходят крадешком без дреха, И гладни, носят сноповете;
\par 11 Изтискват дървено масло в техните огради, Тъпчат линовете им, а остават жадни.
\par 12 Умиращите охкат из града, И душата на ранените вика; Но пак това безумие Бог не гледа.
\par 13 Дали са от противниците на виделината; Не знаят пътищата й, И не стоят в пътеките й,
\par 14 Убиецът става в зори и убива сиромаха и нуждаещия се, А нощем е като крадец.
\par 15 Така и окото на прелюбодееца очаква да се мръкне, Като казва: Око не ще ме види; И преличава лицето си.
\par 16 В тъмнината пробиват къщи; Те се затварят през деня, Видело не познават.
\par 17 Защото за всички тях зората е като мрачната сянка; Понеже познават ужасите на мрачната сянка.
\par 18 Бърже се отдалечат по лицето на водата; Делът им е проклет на земята; Не се обръщат вече към пътя за лозята.
\par 19 Както сушата и топлината поглъщат водата от снега, Така и преизподнята грешните.
\par 20 Майчината утроба ще ги забрави; Червеят ще има сладко ястие в тях; Няма вече да се спомнят; И неправдата ще се строши като дърво.
\par 21 Поглъщат неплодната, която ражда; И на вдовицата не правят добро,
\par 22 Влачат и мощните със силата си; Те стават, и никой не е безопасен в живота си.
\par 23 Бог им дава безопасност, и те се успокояват с нея, Но очите Му са върху пътищата им.
\par 24 Въздигнаха се за малко, и, ето, че ги няма! Снишават се; и както всички други си отиват, И отсичат се като главите на класовете.
\par 25 И сега, ако не е така, кой ще ме изкара лъжец, И ще обърне в нищо думите ми?

\chapter{25}

\par 1 Тогава шуахецът Валдат в отговор рече: -
\par 2 Господството и страховдъхновението принадлежат Нему; Прави мир във висините Си.
\par 3 Имат ли брой войнствата Му? И върху кого не изгрява Неговата светлина?
\par 4 И тъй, как може човек да е праведен пред Бога? Или как може да е чист роденият от жена?
\par 5 Ето, и самата луна не е светла, И звездите не са чисти, пред Него,
\par 6 Колко по-малко гадината човек, И червеят човешки син!

\chapter{26}

\par 1 А Иов в отговор рече: -
\par 2 Каква помощ си дал ти на немощния! Как си спасил безсилната мишца!
\par 3 Как си съветвал оня, който няма милост! И какъв здрав разум си изсипал!
\par 4 Към кого си отправил думи? И чий дух те е вдъхновявал?
\par 5 Пред него мъртвите треперят Под водите и обитателите им.
\par 6 Преизподнята е гола пред Него, И Авадон няма покрив.
\par 7 Простира севера върху празния простор; Окача земята на нищо.
\par 8 Връзва водите в облаците Си; Но облак не се продира изпод тях.
\par 9 Покрива лицето на престола Си, Като простира облака Си върху него.
\par 10 Обиколил е водите с граница Дори до краищата на светлината и на тъмнината.
\par 11 Небесните стълбове треперят И ужасяват се от смъмрянето Му.
\par 12 Развълнува морето със силата Си; И с разума Си поразява Рахав§.
\par 13 Чрез духа Си украсява небесата; Ръката Му пробожда бягащия змей
\par 14 Ето, тия са само краищата на пътищата Му; И колко малко шепнене ни дават да чуем за Него! А гърма на силата Му, кой може да разбере?

\chapter{27}

\par 1 И Иов продължи беседата си като казваше:
\par 2 В живота на Бога, Който е отнел правото ми, И на Всемогъщия, Който е огорчил душата ми,
\par 3 Заклевам се, че през всичкото време, докато е дишането ми в мене, И Духът Божий в ноздрите ми,
\par 4 Устните ми няма да изговорят неправда, Нито езикът ми ще продума измама.
\par 5 да не даде Бог да ви оправдая! Докато изсъхна няма да отхвърля непорочността си от мене.
\par 6 Правдата си ще държа, и не ще я оставя; Догдето съм жив сърцето ми няма да ме изобличи.
\par 7 Неприятелят ми нека бъде като нечестивия, И който въстава против мене като беззаконния.
\par 8 Защото каква е надеждата на нечестивия, Че ще спечели, когато изтръгне Бог душата му?
\par 9 Ще послуша ли Бог вика му, Когато го сполети беда?
\par 10 Ще се наслаждава ли във Всемогъщия? Ще призовава ли Бога във всяко време?
\par 11 Ще ви науча това, което е в Божията ръка; Каквото има у Всемогъщия не ще да скрия.
\par 12 Ето, вие всички сте видели това; Защо, прочее, ставате съвсем безполезни?
\par 13 Делът на нечестивия от Бога, И наследството, което притеснителите Ще получат от Всемогъщия е това:
\par 14 Ако се умножават чадата му, умножават се за меч; И внуците му не ще се наситят с хляб.
\par 15 Останалите от него, смърт ще ги погребе, И вдовиците му няма да плачат.
\par 16 Ако и да натрупа сребро много като пръст, И приготви дрехи изобилно като кал,
\par 17 Може да приготви, но праведните не ще ги облекат, И невинните ще си разделят среброто.
\par 18 Построява къщата си както молеца. И както пъдарят, който прави колиба.
\par 19 Богат лъга, но няма да повтори. Веднъж отваря очите си и го няма;
\par 20 Трепет го хваща като потоп; Буря го граби нощем;
\par 21 Източният вятър го дига, и той отива; Той го изтръгва с мястото му.
\par 22 Защото Бог ще хвърли върху него беди, и не ще го пожали; Той ще се старае да избяга от ръката Му.
\par 23 Ще изпляскат с ръце против него, И ще му подсвиркват така, че ще бяга от мястото си.

\chapter{28}

\par 1 Наистина има рудница за сребро, И място, гдето злато се плави.
\par 2 Желязото се взема из земята, И медта се лее от камъка.
\par 3 Човекът туря край на тъмнината, И издирва до най-далечните места, Камъните в тъмнината и в мрачната сянка.
\par 4 Далеч то човешко жилище, гдето нозе не стъпват, Той си отваря рудница; Окачени далеч от човеците рудничарите се люлеят.
\par 5 Колкото за земята, от нея произлиза хлябът? И под нея се разравя като че ли с огън.
\par 6 Камъните и са място на сапфир, И златна пръст има в нея.
\par 7 Хищна птица не знае тоя път И окото на сокол не го е видяло.
\par 8 Горделивите зверове не са стъпвали по него; Лъв не е заминавал през него.
\par 9 Човекът простира ръката си върху канарите, Превръща планините из корен.
\par 10 Разсича проломи между скалите; И окото му открива всичко що е скъпоценно
\par 11 И ограничава капането на водите; И скритото изважда на бял свят.
\par 12 Но мъдростта, где ще се намери? И где е мястото на разума?
\par 13 Човекът не познава цената й; И тя не се намира в земята на живите,
\par 14 Бездната казва: Не е у мене. И морето казва:Не е у мене.
\par 15 Не може да се придобие със злато; И сребро не може да се претегли в замяна с нея.
\par 16 Не може да се оцени с офирско злато, Със скъпоценен оникс и сапфир.
\par 17 Злато и кристал не могат се сравни с нея, Нито може да се размени с вещи от на-чисто злато.
\par 18 Не ще се спомене корал или кристал за покупката й. Защото цената на мъдростта е по-висока от скъпоценните камъни.
\par 19 Топаз етиопски не ще се сравни с нея; Не ще се оцени тя с чисто злато.
\par 20 От, где прочее, дохожда мъдростта? И где е мястото на разума?-
\par 21 Понеже е скрита от очите на всичките живи, И утаена от въздушните птици.
\par 22 Гибелта и смъртта казват: С ушите си чухме слух за нея.
\par 23 Бог разбира пътя й, И Той знае мястото й;
\par 24 Понеже Той гледа до земните краища, И вижда под цялото небе,
\par 25 За да претегля тежината на ветровете, И да измерва водите с мярка.
\par 26 Когато направи закон за дъжда, И път за светкавицата на гръма,
\par 27 Тогава Той я видя и изяви; Утвърди я, да! И я изследва;
\par 28 И каза на човека: Ето, Страх от Господа, туй е мъдрост, И отдалечаване от злото, това е разум.

\chapter{29}

\par 1 И Иов още продължи беседата си като казваше:
\par 2 О, да бях както в предишните месеци, Както в дните, когато Бог ме пазеше,
\par 3 Когато светилникът Му светеше на главата ми, И със светлината Му ходех в тъмнината;
\par 4 Както бях в дните на зрелостта си, Когато съветът от Бога бдеше над шатъра ми;
\par 5 Когато Всемогъщият беше още с мене, И децата ми бяха около мене;
\par 6 Когато миех стъпките си с масло, И скалата изливаше за мене реки от дървено масло!
\par 7 Когато през града излизах на портата, И приготвях седалището си на пазара,
\par 8 Младите, като ме гледаха, се криеха, И старците ставаха и стояха прави;
\par 9 Първенците се въздържаха от говорене, И туряха ръка на устата си;
\par 10 Гласът на началниците замлъкваше, И езикът им залепваше за небцето им;
\par 11 Ухо, като ме чуеше, ублажаваше ме, И око, като ме виждаше, свидетелствуваше за мене;
\par 12 Защото освобождавах сиромаха, който викаше, И сирачето, и онзи, който нямаше помощник.
\par 13 Благословението от този, който бе близо до загиване, идеше на мене; И аз веселях сърцето на вдовицата.
\par 14 Обличах правдата, и тя ми беше одежда; Моята правдивост ми беше като мантия и корона.
\par 15 Аз бях очи на слепия, И нозе на хромия.
\par 16 Бях баща на сиромасите; Изследвах делото на непознатия мене.
\par 17 Трошех челюстите на несправедливия, И изтеглях лова из зъбите му.
\par 18 Тогава думах: Ще умра в гнездото си; И дните ми ще се умножат, както пясъка.
\par 19 Коренът ми е прострян към водите; И росата намокрюва цяла нощ клоните ми.
\par 20 Славата ми зеленее още в мене; И лъкът ми се укрепява в ръката ми.
\par 21 Човеците чакаха да ме слушат, И мълчаха, за да чуят съветите ми.
\par 22 Подир моите думи те не притуряха нищо; Словото ми капеше върху тях;
\par 23 За мене очакваха като за дъжд, И устата ми зееха като за пролетен дъжд.
\par 24 Усмихвах се на тях, когато бяха в отчаяние; И те не можаха да потъмнеят светлостта на лицето ми.
\par 25 Избирах пътя към тях, и седях пръв помежду им, И живеех като цар всред войската, Както онзи, който утешава наскърбените.

\chapter{30}

\par 1 Но сега ми се подсмиват по-младите от мене, Чиито бащи не бих приел да туря с кучетата на стадото си;
\par 2 Защото в що можеше да ме ползува силата на ръцете им, Човеци, чиято жизненост бе изчезнала?
\par 3 От немотия и глад те бяха измършавели; Гризяха изсушената земя, отдавна пуста и опустошена;
\par 4 Между храстите късаха слез, И корените на смрика им бяха за храна.
\par 5 Бяха изпъдени измежду човеците, Които викаха подир тебе като подир крадци.
\par 6 Живееха в пукнатините на долините, В дупките на земята и на скалите.
\par 7 Ревяха между храстите. Събираха се между тръните;
\par 8 Безумни и безчестни, Те бидоха изгонени от земята.
\par 9 А сега аз им станах песен, Още им съм и поговорка.
\par 10 Гнусят се от мене, отдалечават се от мене, И не се свенят да плюят в лицето ми.
\par 11 Тъй като Бог е съсипал достолепието ми и ме е смирил. То и те се разюздаха пред мене.
\par 12 Отдясно въстават тия изроди, Тласкат нозете ми, И приготовляват против мене гибелните си намерения,
\par 13 Развалят пътя ми, Увеличават нещастието ми, И то без да имат помощници.
\par 14 Идат като през широк пролом; Под краха нахвърлят се върху мене.
\par 15 Ужаси се обърнаха върху мене; Като вятър гонят достолепието ми; И благополучието ми премина като облак.
\par 16 И сега душата ми се излива в мене; Скръбни дни ме постигнаха.
\par 17 През нощта костите ми се пронизват в мене, И жилите ми не си почиват.
\par 18 Само с голямо усилие се променява дрехата ми; Тя ме стига както яката на хитона ми.
\par 19 Бог ме е хвърлил в калта; И аз съм заприличал на пръст и на прах.
\par 20 Викам към Тебе, но не ми отговаряш; Стоя, и Ти просто ме поглеждаш.
\par 21 Обърнал си се да се показваш жесток към мене; С мощната Си ръка ми враждуваш;
\par 22 Издигаш ме, възкачваш ме на вятъра, И стопяваш ме в бурята.
\par 23 Зная наистина, че ще ме докараш до смърт, И до дома, който е определен за всичките живи.
\par 24 Обаче в падането си човек няма ли да простре ръка, Или да нададе вик в бедствието си?
\par 25 Не плаках ли аз за онзи, който бе отруден? И не се ли оскърби душата ми за сиромаха?
\par 26 Когато очаквах доброто, тогава дойде злото; И когато ожидах виделината, тогава дойде тъмнината.
\par 27 Червата ми възвират, и не си почиват; Скръбни дни ме постигнаха.
\par 28 Ходя почернял, но не от слънцето; Ставам в събранието и викам за помощ.
\par 29 Станах брат на чакалите, И другар на камилоптиците.
\par 30 Кожата ми почерня на мене, И костите ми изгоряха от огън.
\par 31 Затова арфата ми се измени в ридание, И свирката ми в глас на плачещи.

\chapter{31}

\par 1 Направих завет с очите си; И как бих погледнал на девица?
\par 2 Защото какъв дял се определя от Бога отгоре, И какво наследство от Всемогъщия свише?
\par 3 Не е ли разорение за нечестивия, И погибел за тия, които вършат беззаконие?
\par 4 Не вижда ли Той пътищата ми? И не брои ли всичките ми стъпки?
\par 5 Ако съм ходил с лъжата, И ногата ми е бързала на измама, -
\par 6 (Но нека ме претеглят в прави везни, За да познае Бог непорочността ми, -)
\par 7 Ако се е отклонила ногата ми от пътя, И сърцето ми е последвало очите ми, И ако е залепило петно на ръцете ми,
\par 8 То нека сея аз, а друг да яде, И нека се изкоренят произведенията ми
\par 9 Ако се е прелъстило сърцето ми от жена, И съм причаквал при вратата на съседа си,
\par 10 То нека моята жена меле за другиго, И други да се навеждат над нея;
\par 11 Защото това би било гнусно дело, И беззаконие, което да се накаже от съдиите;
\par 12 Понеже това е огън, който изгорява до погубване, И би изкоренил всичките ми плодове.
\par 13 Ако съм презрял правото на слугата си или на слугинята си, Когато имаха спор с мене,
\par 14 То какво бих сторил, когато се подигне Бог? И какво бих Му отговорил, когато посети?
\par 15 Оня, който е образувал мене в утробата, не образува ли и него? И не същия ли ни образува в утробата?
\par 16 Ако съм въздържал сиромасите от това, което желаеха, Или съм направил да помрачеят очите на вдовицата,
\par 17 Или съм изял сам си залъка си, Без да е яло сирачето от него, -
\par 18 (Напротив, от младостта ми то порасте при мене като при баща, И от утробата на майка си съм наставлявал вдовицата;)
\par 19 Ако съм гледал някого да гине от нямане дрехи, Или сиромах, че няма завивка,
\par 20 И не са ме благославяли чреслата му, Като се е стоплял с вълната от овцете ми;
\par 21 Ако съм подигнал ръка против сирачето, Като виждах, че имам помощ в портата;
\par 22 То да падне мишцата ми от рамото, И ръката ми да се пречупи от лакътя;
\par 23 Защото погибел от Бога беше ужас за мене, И пред Неговото величие не можех да сторя нищо.
\par 24 Ако съм турял надеждата си в злато, Или съм рекъл на чистото злато: Ти си мое упование;
\par 25 Ако съм се веселил, защото богатството ми бе голямо, И защото ръката ми бе намерила изобилие;
\par 26 Ако, като съм гледал слънцето, когато изгряваше, Или луната, когато ходеше в светлостта си,
\par 27 Се е увлякло тайно сърцето ми, И устата ми са целували ръката ми;
\par 28 И това би било беззаконие, което да се накаже от съдиите, Защото бих се отрекъл от Всевишния Бог.
\par 29 Ако съм злорадствувал в загиването на мразещия ме, Или ми е ставало драго, когато го е сполетявало зло, -
\par 30 (Даже не съм допуснал на устата си да съгрешат Та да иксам живота му с проклетия); -
\par 31 Ако хората от шатъра ми не са рекли: Кой може да покаже едного, който не е бил наситен от него с месо?
\par 32 (Чужденец не нощуваше вън; Отварях вратата си на пътника);
\par 33 Ако съм покривал престъпленията си както Адама Като съм скривал беззаконието си в пазухата си,
\par 34 Понеже се боях от голямото множество, И презрението на семействата ме ужасяваше, Така че млъквах и не излизах из вратата; -
\par 35 (О, да имаше някой да ме слуша! - Ето виж тука подписа ми; Всемогъщият нека ми отговори!- И да имах акта който противникът ми е написал!
\par 36 Ето, на рамо щях да го нося, За венец щях да го привържа на себе си!
\par 37 Щях да му дам отчет за стъпките си; Като княз щях да се приближа при него-)
\par 38 Ако нивата ми вика против мене, И браздите й плачат заедно;
\par 39 Ако съм изял плода й без да платя, Или съм изгасил живота на стопаните й;
\par 40 Тогава да израстат тръни вместо жито, И вместо ечемик плевели. Свършиха се думите на Иова.

\chapter{32}

\par 1 И така, тия трима човека престанаха да отговарят на Иова, защото беше праведен пред своите си очи.
\par 2 Тогава пламна гневът на вузеца Елиу, син на Варахиила, от Арамовото семейство. Гневът му пламна против Иова, защото оправдаваше себе си наместо Бога(
\par 3 тоже против тримата му приятели пламна гневът му, защото бяха осъдили Иова без да му намерят отговор.
\par 4 А Елиу беше чакал да говори на Иова, защото другите бяха по-стари от него.
\par 5 Но когато Елиу видя, че нямаше отговор в устата на тия трима мъже, гневът му пламна.
\par 6 Тогава вузецът Елиу, син на Варахиила, в отговор рече:- Аз съм млад, а вие много стари; Затова се посвених, и не смеех да ви явя моето мнение.
\par 7 Аз рекох: Дните нека говорят, И многото години нека учат мъдрост.
\par 8 Но има дух в човека; Вдъхновението на Всемогъщия го вразумява.
\par 9 Не че човеците са велики, за това ще са и мъдри, Нито че са стари, за това ще разбират правосъдието.
\par 10 Прочее, казвам: Слушайте мене; Нека явя и аз мнението си.
\par 11 Ето, чаках докато вие говорехте, Слушах разсъжденията ви, Когато търсехте какво да кажете;
\par 12 Внимателно ви слушах, И ето, ни един от вас не убеди Иова, Нито отговори на думите му;
\par 13 За да не речете: Ние намерихме мъдрост. Бог ще го свали, а не човек.
\par 14 Понеже той не е отправил думите си против мене, То и аз няма да му отговоря според вашите речи.
\par 15 Те се смайват, не отговарят вече, Не намират ни дума да кажат.
\par 16 А да чакам ли аз понеже те не говорят, - Понеже стоят и не отговарят вече?
\par 17 Нека отговоря и аз от моя страна, Нека явя и аз мнението си.
\par 18 Защото съм пълен с думи; Духът в мене дълбоко ме притиска.
\par 19 Ето коремът ми е като вино неотворено, Близо е да се разпукне като нови мехове.
\par 20 Ще проговоря, за да ми стане по-леко; Ще отворя устните си и ще отговоря.
\par 21 Далеч от мене да гледам на лице, Или да полаская човека.
\par 22 Защото не зная да лаская; Иначе Създателят ми би ме отмахнал веднага.

\chapter{33}

\par 1 Затова, Иове, чуй сега словото ми, И слушай всичките мои думи.
\par 2 Ето, сега отворих устата си, Езикът ми с устата ми говори.
\par 3 Думите ми ще бъдат според правотата на сърцето ми, И устните ми ще произнесат чист разум.
\par 4 Духът Божии ме е направил, И дишането на Всемогъщия ме оживотворява.
\par 5 Ако можеш, отговори ми; Опълчи са с думите си пред мене та застани.
\par 6 Ето, и аз съм пред Бога както си ти,- И аз съм от кал образуван.
\par 7 Ето, моят ужас няма да те уплашва. Нито ще тежи ръката ми върху тебе.
\par 8 Безсъмнено ти си говорил, като слушах аз, И аз чух гласа на думите ти, като казваше:
\par 9 Чист и без престъпление съм; Невинен съм, и баззаконие няма в мене;
\par 10 Ето, Бог намира причини против мене, Счита ме за Свой неприятел;
\par 11 Туря нозете ми в клада, Наблюдава всичките ми пътища.
\par 12 Ето, в това ти не си прав; Ще ти отговоря, че Бог е по-велик от човека.
\par 13 Защо се препираш с Него, Загдето Той не дава отчет ни за едно от Своите дела?
\par 14 Защото сигурно Бог говори веднъж и дваж, Само че човекът не внимава.
\par 15 В сън, в нощно видение, Когато дълбок сън напада човеците, Когато сънуват на леглата си,
\par 16 Тогава Той отваря ушите на човеците, И запечатва поука в тях,
\par 17 За да отвърне човека от намерението му, И да извади гордостта из човека;
\par 18 Предпазва душата му от гроба, И животът му, за да не падне от меч.
\par 19 Той бива е наказван с болки на леглото си, Да! С непрестанни болки в костите си,
\par 20 Така щото душата му се отвръща от хляб, И сърцето му от вкусното ястие.
\par 21 Месата му се изнуряват тъй, че не се виждат, А невидимите му по-преди кости се подават.
\par 22 Да! Душата му се приближава при гроба. И животът му при погубителите,
\par 23 Тогава, ако има ангел с него, Посредник, пръв между хиляда, За да възвести на човека що е за него право,
\par 24 И ако Бог му бъди милостив И рече: Избави го, за да не слезе в гроба, Аз промислих откуп за него, -
\par 25 Тогава месата му ще се подмладяват повече от месата на дете? Той се връща в дните на младостта си;
\par 26 Ако се помоли Богу, Той е благосклонен към него, И му дава да гледа лицето Му с радост; И възвръща на човека правдата му.
\par 27 Той пее пред човеците, казвайки: Съгреших и изкривих правото, И не ми се въздаде според греха ми;
\par 28 Той избави душата ми, за да не отиде в рова; И животът ми ще види виделината.
\par 29 Ето всичко това върши Бог Дваж и триж с човека,
\par 30 За да отвърне душата му от рова, Но да се просвети с виделината на живота.
\par 31 Внимавай, Иове, послушай ме, Мълчи, и аз ще говоря.
\par 32 Ако имаш какво да кажеш, отговори ми; Говори, защото желая да бъдеш оправдан;
\par 33 Но ако не, то ти слушай мене; Мълчи, и ще те науча мъдрост.

\chapter{34}

\par 1 И Елиу пак проговаряйки рече:
\par 2 Слушайте думите ми, вие мъдри, И внимавайте към мене, вие разумни;
\par 3 Защото ухото изпитва думите Както небцето вкусва ястието.
\par 4 Нека си изберем правото Та да знаем помежду си доброто.
\par 5 Защото Иов е казал: Праведен съм, И пак Бог отне правото ми;
\par 6 Въпреки правото ми считан съм за лъжец; Раната ми е неизцелима при все, че съм без престъпление.
\par 7 Кой човек е като Иова, Който укорява Бога, както пие вода,
\par 8 И дружи с ония, които вършат беззаконие, И ходи с нечестиви човеци?
\par 9 Защото е казал: Нищо не ползува човека Да съизволява с Бога.
\par 10 Слушайте ме, прочее, вие разумни мъже. Далеч да бъде от Бога неправдата, И от Всемогъщия беззаконието!
\par 11 Защото ще въздаде на човека според делото му, И ще направи всеки да намери според пътищата си.
\par 12 Наистина Бог няма да извърши насилие, Нито ще извърне Всемогъщият правосъдието.
\par 13 Кой е възложил на Него грижата за земята? Или кой Го е натоварил с цялата вселена?
\par 14 Ако прилепи Той сърцето Си само към Себе Си, И оттегли към Себе Си Духа Си и душата Си,
\par 15 То ще издъхне заедно всяка плът, И човекът ще се върне пак в пръстта.
\par 16 Сега, ако си разумен, чуй това; Слушай гласа на думите ми.
\par 17 Ще властвува ли оня, който мрази правдата? И ще изкараш ли виновен мощния Праведник,
\par 18 Който казва на цар: Нечестив си, На князе: Беззаконници сте,
\par 19 Който не лицеприятствува пред първенци, Нито почита богатия повече от сиромаха, Понеже всички са дело на Неговите ръце?
\par 20 В една минута умират - да в полунощ; Людете им се смущават и преминават§ И мощните биват премахнати не с ръка.
\par 21 Защото очите на Бога са върху пътищата на човека, И Той гледа всичките му стъпки.
\par 22 Няма тъмнина, нито мрачна сянка, Гдето да се крият ония, които вършат беззаконие.
\par 23 Понеже Той няма нужда втори път да изпитва човека, За да дойде на съд пред Бога.
\par 24 Без дълго изследване сломява силните, И поставя други вместо тях.
\par 25 Прочее, Той познава делата им; И събаря ги нощем, та те се смазват.
\par 26 Удря ги като нечестиви Явно там гдето има зрители,
\par 27 Понеже се отклониха от Него, И не зачитаха ни един от пътищата Му,
\par 28 Така че направиха да стигне до Него викът на сиромасите, Та Той чу вика на угнетените.
\par 29 И когато Той успокоява, кой ще смути? Когато крие лицето Си, кой може да Го види? Безразлично дали е сторено това спрямо народ или спрямо един човек, -
\par 30 За да не царува нечестив човек, Човек, който би впримчвал людете,
\par 31 Защото, ако някой каже на Бога: Понесох наказание без да съм сторил зло;
\par 32 Каквото аз не виждам, Ти ме научи; Ако съм извършил беззаконие, няма да върша вече, -
\par 33 То трябва ли въздаянието му да бъде, според както ти желаеш, да го отхвърляш, Така щото ти да го избереш, казва Бог, а не Аз? Тогава ти кажи каквото знаеш.
\par 34 Разумни мъже ще ми рекат: Да! Всеки мъдър човек, който ме слуша, ще каже:
\par 35 Иов говори без знание, И думите му са лишени от мъдрост.
\par 36 Желанието ми е, Иов да бъде изпитан до край, Понеже отговори, както нечестивите човеци.
\par 37 Защото на греха си притуря бунтовничество, Поруга се между нас, И умножава думите си против Бога.

\chapter{35}

\par 1 Елиу още проговаряйки рече:
\par 2 Мислиш ли че, е право това, което рече ти: Моята правда в туй дело е повече от Божията?
\par 3 Защото ти рече: Какво ще ми бъде преимуществото? Какво ще ме ползува повече отколкото, ако бях съгрешил?
\par 4 Аз ще отговоря на тебе. И на приятелите ти с тебе.
\par 5 Погледни към небесата и виж; И гледай облаците, колко по-високо са от тебе.
\par 6 Ако съгрешаваш, какво правиш против Него? И ако се умножават престъпленията ти, какво Му вършиш?
\par 7 Ако си праведен, какво Му даваш? Или какво получава от ръката ти?
\par 8 Нечестието ти може да повреди само човек като тебе; А правдата ти може да ползува само човешки син.
\par 9 Поради много угнетения викат праведниците, Пищят поради насилието на мощните;
\par 10 Но пак никой не казва: Где е Бог Творецът ми, Който дава песни нощем,
\par 11 Който ни учи нежели земните животни, И прави ни по-мъдри от въздушните птици?
\par 12 Така те викат; но Той не отговаря Да ги избави от гордостта на нечестивите.
\par 13 Наистина, Бог не слуша празнословието, И Всемогъщият не го зачита,
\par 14 Колко по-малко, когато ти казват, че не Го виждаш, Че делото ти е пред Него, и напразно Го чакаш!
\par 15 И сега, понеже не те е посетил в яростта Си, И не е прегледал със строгост надменността ти,
\par 16 Затова Иов отваря уста да говори суетности, Трупа думи лишени от благоразумие.

\chapter{36}

\par 1 И Елиу продължавайки рече:
\par 2 Потърпи ме малко, и ще ти явя, Защото имам още нещо да ти кажа за Бога.
\par 3 Ще черпя знанието си от далеч, И ще дам правда на Създателя си;
\par 4 Защото наистина думите ми не ще бъдат лъжливи; Един, който е усъвършенствуван в знание, стои пред тебе.
\par 5 Ето, макар Бог да е мощен, не презира никого, - Макар да е мощен в силата Си да разсъждава.
\par 6 Той не запазва живота на нечестивия, А на сиромасите отдава правото.
\par 7 Не оттегля очите Си от праведните, Но даже ги туря и с царе да седят на престол за винаги, И те биват възвишени.
\par 8 А във вериги, ако са вързани, И хванати с въжета на наскърбление,
\par 9 Тогава им явява делата им И престъпленията им, че са се възгордели,
\par 10 Отваря и ухото им за поука, И заповядва да се върнат от беззаконието;
\par 11 И ако послужат и служат на Него, Ще прекарат дните си в благополучие И годините си във веселия;
\par 12 Но ако не послушат, ще загинат от меч, И ще умрат без мъдрост.
\par 13 А нечестивите в сърце питаят яд, Не викат към Бога за помощ когато ти връзва;
\par 14 Те умират в младост, И животът им угасва между мръсните.
\par 15 Той избавя наскърбения чрез скръбта му, И чрез бедствие отваря ушите им
\par 16 И така би извел и тебе из утеснение В широко място, гдето няма теснота; И слаганото на трапезата ти било би пълно с тлъстина.
\par 17 Но ти си пълен със съжденията на нечестивия; Затова съдбата и правосъдието те хващат,
\par 18 Внимавай да не би жегата на страданията ти да те подигне против удара; Тогава нито голям откуп би те отървал.
\par 19 Ще важи ли викането ти да те извади от бедствие, Или всичките напрежения на силата ти?
\par 20 Не пожелавай нощта, Когато людете изчезват от мястото си.
\par 21 Внимавай! Не пожелавай беззаконието; Защото ти си предпочел това повече от наскърблението.
\par 22 Ето, Бог е възвишен в силата Си; Кой е господар като Него?
\par 23 Кой Му е предписал пътя Му? Или кой може да Му рече: Извършил си беззаконие?
\par 24 Помни да възвеличаваш Неговото дело, Което човеците възпяват,
\par 25 В което всичките човеци се взират, Което човекът гледа от далеч.
\par 26 Ето, Бог е велик, и ние Го не познаваме; Числото на годините Му е неизследимо.
\par 27 Той привлича водните капки, Които таят в дъжд от парите Му,
\par 28 Които облаците изливат И оросяват върху множество човеци.
\par 29 Може ли, даже, някой да разбере как се разпростират облаците, Или се произвеждат гърмежите на скинията Му?
\par 30 Ето, простира светлината Си около Себе Си, И се покрива с морските дъна.
\par 31 (Понеже чрез тия неща съди народите; Дава храна изобилно),
\par 32 Покрива ръцете Си със светкавицата, И заповядва й где да удари;
\par 33 Шумът й известява за това, И добитъкът - за пламъка, който възлиза.

\chapter{37}

\par 1 Да! Поради това сърцето ми трепери И се измества от мястото си.
\par 2 Слушайте внимателно гърма на гласа Му, И шума, който излиза из устата Му.
\par 3 Праща го под цялото небе И светкавицата Си до краищата на земята;
\par 4 След нея рече глас, Гърми с гласа на величието Си, И не ги възпира щом се чуе гласа Му.
\par 5 Бог гърми чудно с гласа Си, Върши велики дела, които не можем да разбираме;
\par 6 Защото казва на снега: Вали на земята, - Също и на проливния дъжд и на поройните Си дъждове;
\par 7 Запечатва ръката на всеки човек, Така щото всичките човеци, които е направил, да разбират силата Му.
\par 8 Тогава зверовете влизат в скривалищата И остават в рововете си.
\par 9 От помещението си иде бурята, И студът от ветровете що разпръскват облаците.
\par 10 Чрез духане от Бога се дава лед, И широките води замръзват;
\par 11 Тоже гъстия облак Той натоварва с влага, Простира на широко светкавичния Си облак,
\par 12 Които според Неговото наставление се носят наоколо За да правят всичко що им заповядва По лицето на земното кълбо.
\par 13 Било, че за наказание, или за земята Си, Или за милост, ги докарва.
\par 14 Слушай това, Иове, Застани та размисли върху чудесните Божии дела.
\par 15 Разбираш ли как им налага Бог волята Си. И прави светкавицата да свети от облака Му?
\par 16 Разбираш ли как облаците увисват, Чудесните дела на Съвършения в знание? -
\par 17 Ти, чиито дрехи стават топли, Когато земята е в затишие, поради южния вятър,
\par 18 Можеш ли като Него да разпростреш небето, Което, като леяно огледало е здраво?
\par 19 Научи ни що да Му кажем, Защото поради невежество ние не можем да наредим думите си
\par 20 Ще Му се извести ли, че желая да говоря, Като зная че, ако продума човек непременно ще бъде погълнат?
\par 21 И сега човеците не могат да погледнат на светлината, Когато блещи на небето, като е заминал вятърът и го е очистил,
\par 22 Та е дошло златозарно сияние от север; А как ще погледнат на Бога, у Когото е страшна слава!
\par 23 Всемогъщ е, не можем да Го проумеем, превъзходен е в сила; А правосъдието и преизобилната правда Той няма да отврати.
\par 24 Затова Му се боят човеците; Той не зачита никого от високоумните.

\chapter{38}

\par 1 Тогава Господ отговори на Иова из бурята и каза:
\par 2 Кой е тогава този, който помрачава Моя съвет С неразумни думи?
\par 3 Опаши сега кръста си като мъж, И ще те попитам; и ти ми изяснявай,
\par 4 Где беше ти, когато основах земята? Извести, ако си разумен:
\par 5 Кой определи мерките й? (ако знаеш) Или кой тегли връв за мерене по нея?
\par 6 На какво се вдълбочиха основите й? Или кой положи краеъгълния й камък,
\par 7 Когато звездите на зората пееха заедно, И всички Божии синове възклицаваха от радост?
\par 8 Или кой затвори морето с врати, Когато се устреми та излезе из матка,
\par 9 Когато го облякох с облак И го пових с мъгла,
\par 10 И поставих му граница от Мене, Турих лостове и врати,
\par 11 И рекох: До тук ще дохождаш, но не по-нататък, И тук ще се спират гордите ти вълни?
\par 12 Откак започнаха дните ти заповядал ли си ти на утрото И показал на зората мястото й,
\par 13 За да обхване краищата на земята, Така щото да се изтърсят от нея злодейците
\par 14 Та да се преобразува тя, както глина под печат, И всичко да изпъква като че ли в облекло,
\par 15 А от нечестивите да се отнеме виделината им, И издигнатата им мишца да се строши?
\par 16 Проникнал ли си до изворите на морето? Или ходил ли си да изследваш бездната?
\par 17 Откриха ли се на тебе вратите на смъртта? Или видял ли си сенчестите врати на смъртта?
\par 18 Схванал ли си широчината на земята? Кажи, ако си разбрал всичко това.
\par 19 Где е пътят към обиталището на светлината? И на тъмнината где е мястото й,
\par 20 За да й заведеш до границата й, И да познаеш пътеките към дома й?
\par 21 Без съмнение, ти знаеш, защото тогаз си се родил, И голямо е числото на твоите дни!
\par 22 Влизал ли си в съкровищниците за снега, Или виждал ли си съкровищниците за градушката,
\par 23 Които пазя за време на скръб, За ден на бой и на война?
\par 24 Що е пътят за мястото, гдето се разсява светлината, Или се разпръсва по земята източният вятър?
\par 25 Кой е разцепил водопровод за проливните дъждове, Или път за светкавицата на гръма,
\par 26 За да се докара дъжд върху ненаселена земя, Върху пустинята, гдето няма човек,
\par 27 За да насити пустата и запустяла земя. И да направи нежната трева да изникне?
\par 28 Дъждът има ли баща? Или кой е родил капките на росата?
\par 29 От чия матка излиза ледът? И кой е родил небесната слана? -
\par 30 Когато водите се втвърдяват като камък, И повърхността на бездната се смръзва.
\par 31 Ти ли връзваш връзките на Плеадите, Или развързваш въжетата на Ориона?
\par 32 Извеждаш ли Мазарот на времето му? Или управляваш ли Мечката с малките й?
\par 33 Познаваш ли законите на небето? Установяваш ли неговото владичество върху земята?
\par 34 Издигаш ли гласа си до облаците, За да те покрият изобилни води?
\par 35 Изпращаш ли светкавици, та да излизат И да ти казват: Ето ни?
\par 36 Кой е турил мъдрост в облаците? Или кой е дал разум на гъстите облаци?
\par 37 Кой с мъдрост брои облаците? Или кой излива небесните мехове
\par 38 Та да се сгъстява пръстта в куп, И буците да се слепят?
\par 39 Улавят ли лов за лъвицата? Или насищат ли охотата на лъвовите малки,
\par 40 Когато седят в рововете си, И остават в скривалищата за да причакват?
\par 41 Кой приготвя за враната храната й, Когато пилетата й от нямане храна Се скитат и викат към Бога?

\chapter{39}

\par 1 Знаеш ли времето, когато раждат дивите кози по канарите? Забелязваш ли кога раждат кошутите?
\par 2 Преброяваш ли колко месеци изпълняват те? Или знаеш ли срока за раждането им? -
\par 3 Когато се навеждат, раждат малките си, Освобождават се от болките си.
\par 4 Малките им заякват, растат в полето; Излизат и не се връщат вече при тях.
\par 5 Кой е пуснал на свобода дивия осел? Или кой е развързал връзките на тоя плах бежанец,
\par 6 За кой съм направил пустинята за къща И солената земя за негово жилище?
\par 7 Той се присмива на градския шум, Нито внимава на викането на този, който го кара.
\par 8 Планините, които обикаля, са пасбището му; И търси всякаква зеленина.
\par 9 Ще благоволи ли дивият вол да ти работи, Или ще нощува ли в твоите ясли?
\par 10 Можеш ли да впрегнеш дивия вол за оране? Или ще браносва ли той полетата зад тебе?
\par 11 Ще се облегнеш ли на него, защото силата му е голяма? Или ще повериш ли на него работата си?
\par 12 Ще се довериш ли на него да ти прибере житото ти И да го събере в гумното ти?
\par 13 Крилата на камилоптицата пляскат весело; Но крилата и перата й благи ли са?
\par 14 Защото тя оставя яйцата си на земята И ги топли в пръстта,
\par 15 А забравя, че е възможно нога да ги смаже Или полски звяр да ги стъпче.
\par 16 Носи се жестоко с малките си, като че не са нейни; Трудът й е напразно, защото не я е грижа за опасности:
\par 17 Понеже Бог я е лишил от мъдрост, И не я е обдарил с разум.
\par 18 Когато стане да бяга Присмива се на коня и на ездача му.
\par 19 Ти ли си дал сила на коня? Облякъл си врата му с трептяща грива?
\par 20 Ти ли го правиш да скача като скакалец? Гордото му пръхтене е ужасно.
\par 21 Копае с крак в долината, и се радва на силата си; Излиза срещу оръжията.
\par 22 Присмива се на страха и не се бои. Нито се обръща назад от меча,
\par 23 Тула по страната му трещи, И лъскавото копие, и сулицата.
\par 24 С буйство и ярост той гълта земята; И при гласа на тръбата не вярва от радост.
\par 25 Щом свири тръбата, той казва: Хо, хо! И от далеч подушва боя, Гърменето на военачалниците и викането.
\par 26 Чрез твоята ли мъдрост лети на горе ястребът, И простира крилата си към юг?
\par 27 При твоята ли заповед се възвишава орелът, И при гнездото си по височините?
\par 28 Живее по канарите, и там се помещава, По върховете на скалите, и по непроходимите места.
\par 29 От там си съзира плячка, Очите му я съглеждат от далеч.
\par 30 И пилетата му смучат кръв; И дето има трупове, там е той.

\chapter{40}

\par 1 Господ говори на Иова и каза:
\par 2 Тоя, която е изобличил Всемогъщият, ще ли се бори с Него? Тоя, който се препира с Бога, нека отговори на всичко това.
\par 3 Тогава Иов отговори Господу, казвайки:
\par 4 Ето, аз съм нищожен; какво да Ти отговоря? Турям ръката си на устата си.
\par 5 Веднъж съм говорил, и не ще да отговарям вече, Дори дваж, но няма да повторя.
\par 6 Тогава Господ отговори на Иова из бурята, като каза:
\par 7 Опаши сега кръста си като мъж; Аз ще те попитам, и ти Ми изявявай.
\par 8 Дори не ще ли допускаш Моята съдба? Ще осъдиш ли Мене, за да оправдаеш себе си?
\par 9 Или имаш ли мишца като Бога? И можеш ли да гърмиш с глас като Него?
\par 10 Украси се сега с превъсходство и достолепие, И облечи се в чест и величие.
\par 11 Изсипвай преливащия си гняв; И гледай на всеки горделив, и смирявай го;
\par 12 Гледай на всеки горделив, и снишавай го; И стъпквай нечестивите на мястото им;
\par 13 Скрий го купно в пръстта; Вържи лицата им в скрито място.
\par 14 Тогава и Аз ще изповядам пред тебе, Че твоята десница може да те спаси.
\par 15 Ето сега речния кон който съм направил както и тебе; Яде трева като вол.
\par 16 Ето сега, силата му е в чреслата му, И якостта му е в мускулите на корема му.
\par 17 Клати опашката си като кедър; Жилите на бедрата му са сплотени.
\par 18 Костите му са като медни цеви; Ребрата му са като железни лостове.
\par 19 Той е изящното дело Божие; Оня, Който го е направил, го е снабдил с меча Си.
\par 20 Наистина планините промишляват за него храна. Гдето играят всичките полски зверове.
\par 21 Ляга под сенчестите дървета, В съкровището на тръстиката, и в блатата;
\par 22 Сенчестите дървета го покриват със сянката си; Върбите на потоците го обкръжават.
\par 23 Ето, ако би придошла река, той не трепери; Не се смущава, ако би се и Иордан устремил по устата му.
\par 24 Може ли някой да го хване когато е на щрек. Или да прободе носа му с примка?

\chapter{41}

\par 1 Можеш ли да извлечеш крокодила с въдица, Или да притиснеш езика му с въже?
\par 2 Можеш ли тури оглавник на носа му, Или да пробиеш челюстта му с кука?
\par 3 Ще отправи ли той към тебе много моления? Ще ти говори ли със сладки думи?
\par 4 Ще направи ли договор с тебе, Та да го вземеш за вечен слуга?
\par 5 Можеш ли игра с него както с птица? Или ще то вържеш ли за забава на момичетата си?
\par 6 Дружините риболовци ще търгуват ли с него? Ще го разделят ли между търговците?
\par 7 Можеш ли прониза кожата му със сулици, Или главата му с рибарски копия?
\par 8 Тури ръката си на него; Спомни си боя, и не прави вече това.
\par 9 Ето, надеждата да го хване някой е празна; Даже от изгледа му не отпада ли човек?
\par 10 Няма човек толкова дързък щото да смее да го раздразни. Тогава кой може да застане пред Мене?
\par 11 Кой Ми е дал по-напред, та да му отплатя? Все що има под цялото небе е Мое.
\par 12 Няма да мълча за телесните му части, нито за силата Му. Нито за хубавото му устройство.
\par 13 Кой може да смъкне външната му дреха? Кой може да влезе вътре в двойните му челюсти?
\par 14 Кой може да отвори вратите на лицето му? Зъбите му изоколо са ужасни.
\par 15 Той се гордее с наредените си люспи, Съединени заедно като че ли плътно запечатани;
\par 16 Едната се допира до другата Така щото ни въздух не може да влезе между тях;
\par 17 Прилепени са една за друга, Държат се помежду си тъй щото не могат да се отделят.
\par 18 Когато киха блещи светлина, И очите му са като клепачите на зората.
\par 19 Из устата му излизат запалени факли, И огнени искри изкачат.
\par 20 Из ноздрите му излиза дим, Като на възвряло гърне над пламнали тръстики.
\par 21 Дишането му запаля въглища. И пламъкът излиза из устата му.
\par 22 На врата му обитава сила. И всички заплашени скачат пред него.
\par 23 Пластовете на месата му са слепени, Твърди са на него, не могат се поклати.
\par 24 Сърцето му е твърдо като камък, Даже твърдо като долния воденичен камък.
\par 25 Когато става, силните се ужасяват, От страх се смайват.
\par 26 Мечът на тогова, който би го улучил, не може да удържи, - Ни копие, ни сулица, ни остра стрела.
\par 27 Той счита желязото като плява, Медта като гнило дърво.
\par 28 Стрелите не могат го накара да бяга; Камъните на прашката са за него като слама;
\par 29 Сопи се считат като слама; Той се присмива на махането на копието.
\par 30 Като остри камъни има по долните му части; Простира като белези от диканя върху тинята;
\par 31 Прави бездната да ври като котел; Прави морето като варилница за миро.
\par 32 Оставя подир себе си светла диря, Тъй щото някой би помислил, че бездната е побеляла от старост.
\par 33 На земята няма подобен нему, Създаден да няма страх.
\par 34 Той изглежда всяко високо нещо; Цар е над всичките горделиви зверове.

\chapter{42}

\par 1 Тогава Иов отговори Господу, казвайки:
\par 2 Зная, че всичко можеш, И че никое Твое намерение не може да бъде възпрепятствувано.
\par 3 Наистина, кой е този, който помрачава Твоя съвет неразумно. Ето защо аз говорих за онова, което не съм разбирал, За неща пречудни за мене, които не съм познавал.
\par 4 Слушай, моля Ти се, и аз ще говоря; Ще Те попитам, и Ти ми изявявай.
\par 5 Слушал бях за Тебе със слушането на ухото, Но сега окото ми Те вижда;
\par 6 Затова отричам се от думите си, И се кая в пръст и пепел.
\par 7 А когато Господ изговори тия думи на Иова, Господ рече на теманеца Елифаз: Гневът ми пламна против тебе и против двамата ти приятели, защото не сте говорили за Мене това, което е право, както слугата Ми Иов.
\par 8 Затова вземете си сега седем телци и седем овни, та идете при слугата Ми Иова, и пренесете всеизгаряне за себе си; а слугата Ми Иов ще се помоли за вас, (защото него ще приема), за да не постъпи с вас според безумието ви, защото не сте говорили за Мене това, което е право, както слугата Ми Иов.
\par 9 И тъй теманецът Елифаз, и савхиецът Валдад, и нааматецът Софар отидоха, та сториха както им заповяда Господ; и Господ прие Иова.
\par 10 И Господ преобърна плена на Иова, когато той се помоли за приятелите си; и Господ даде на Иова двойно колкото имаше по-напред.
\par 11 Тогава дойдоха при него всичките му братя, всичките му сестри и всичко, които бяха го познавали по-напред, та ядоха хляб с него в къщата му; и, като плакаха за него, утешиха го относно цялото зло, което Господ му беше нанесъл; и всеки му даде по един сребърник, и всеки по една златна обица.
\par 12 Така Господ благослови последните дни на Иова повече от първите му; тъй щото придоби четиринадесет хиляди овце, шест хиляди камили, хиляда чифта волове и хиляда ослици.
\par 13 Още му се родиха седем сина и три дъщери;
\par 14 и първата си дъщеря нарече Емима, втората Касия, а третата Керенапух.
\par 15 И по цялата страна не се намираха жени тъй красиви както Иововите дъщери; и баща им даде на тях наследство както на братята им.
\par 16 Подир това Иов живя сто и четиринадесет години, и видя синовете си и внуците си до четири поколения.
\par 17 И тъй, Иов умря, стар и сит от дни.

\end{document}