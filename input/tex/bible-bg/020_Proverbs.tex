\begin{document}

\title{Притчи}


\chapter{1}

\par 1 Притчи на Давидовия син Соломон, Израилев цар,
\par 2 Записани за да познае някой мъдрост и поука, За да разбере благоразумни думи,
\par 3 За да приеме поука за мъдро постъпване, В правда, съдба и справедливост,
\par 4 За да се даде остроумие на простите, знание и разсъждение на младежа,
\par 5 За да слуша мъдрият и да стане по-мъдър И за да достигне разумният здрави начала,
\par 6 За да се разбират притча и иносказание, Изреченията на мъдрите и гатанките им.
\par 7 Страх от Господа е начало на мъдростта; Но безумният презира мъдростта и поуката.
\par 8 Сине мой, слушай поуката на баща си, И не отхвърляй наставлението на Майка си,
\par 9 Защото те ще бъдат благодатен венец за главата ти, И огърлица около шията ти.
\par 10 Сине мой, ако грешните те прилъгват, Да се не съгласиш.
\par 11 Ако рекат:Ела с нас, Нека поставим засада за кръвопролитие. Нека причакаме без причина невинния,
\par 12 Както ада нека ги погълнем живи, Даже съвършените, като ония, които слизат в рова,
\par 13 Ще намерим всякакъв скъпоценен имот, Ще напълним къщите си с користи,
\par 14 Ще хвърлим жребието си като един от нас, Една кесия ще имаме всички; -
\par 15 Сине мой, не ходи на пътя с тях, Въздържай ногата си от пътеката им,
\par 16 Защото техните нозе тичат към злото, И бързат да проливат кръв.
\par 17 Защото напразно се простира мрежа Пред очите на каква да било птица.
\par 18 И тия поставят засада против своята си кръв, Причакват собствения си живот.
\par 19 Такива са пътищата на всеки сребролюбец: Сребролюбието отнема живота на завладените от него.
\par 20 Превъзходната мъдрост възгласява по улиците, Издига гласа си по площадите,
\par 21 Вика по главните места на пазарите, При входовете на портите, възвестява из града думите си:
\par 22 Глупави, до кога ще обичате глупостта? Присмивачите до кога ще се наслаждавате на присмивките си, И безумните ще мразят знанието?
\par 23 Обърнете се при изобличението ми. Ето, аз ще излея духа си на вас, Ще ви направя да разберете словата ми.
\par 24 Понеже аз виках, а вие отказахте да слушате, Понеже простирах ръката си, а никой не внимаваше,
\par 25 Но отхвърлихте съвета ми, И не приехте изобличението ми, -
\par 26 То аз ще се смея на вашето бедствие, Ще се присмея, когато ви нападне страхът,
\par 27 Когато ви нападне страхът, като опустошителна буря, И бедствието ви се устреми като вихрушка, Когато скръб и мъки ви нападнат,
\par 28 Тогава те ще призоват, но аз няма да отговоря, Ревностно ще ме търсят, но няма да ме намерят.
\par 29 Понеже намразиха знанието, И не разбраха страха от Господа,
\par 30 Не приеха съвета ми, И презряха всичкото ми изобличение,
\par 31 Затова, ще ядат от плодовете на своя си път, И ще се наситят от своите си измислици.
\par 32 Защото глупавите ще бъдат умъртвени от своето си отстъпване, И безумните ще бъдат погубени от своето си безгрижие,
\par 33 Но всеки, който ме слуша, ще живее в безопасност, И ще бъде спокоен без да се бои от зло.

\chapter{2}

\par 1 Сине мой, ако приемеш думите ми, И запазиш заповедите ми при себе си,
\par 2 Така щото да приклониш ухото си към мъдростта. И да предадеш сърцето си към разума,
\par 3 Ако призовеш благоразумието, И издигнеш гласа си към разума,
\par 4 Ако го потърсиш като сребро, И го подириш като скрити съкровища,
\par 5 Тогава ще разбереш страха от Господа, И ще намериш познанието за Бога.
\par 6 Защото Господ дава мъдрост, из устата Му излизат знание и разум.
\par 7 Той запазва истинска мъдрост за праведните, Щит е за ходещите в незлобие,
\par 8 За да защитава пътищата на правосъдието, И да пази пътя на светиите Си.
\par 9 Тогава ще разбереш правда, правосъдие, Правдивост, да! и всеки добър път.
\par 10 Защото мъдрост ще влезе в сърцето ти, Знание ще услажда душата ти,
\par 11 Разсъждение ще те пази, Благоразумие ще те закриля,
\par 12 За да те избави от пътя на злото. От човека, който говори опако, -
\par 13 От ония, които оставят пътищата на правотата, За да ходят по пътищата на тъмнината, -
\par 14 На които прави удоволствие да вършат зло, И се радват на извратеността на злите, -
\par 15 Чиито пътища са криви И пътеките им опаки, -
\par 16 за да те избави от чужда жена, От чужда, която ласкае с думите си,
\par 17 (Която е оставила другаря на младостта си, И е забравила завета на своя Бог,
\par 18 Защото домът й води надолу към смъртта, И пътеките й към мъртвите;
\par 19 Никой от ония, които влизат при нея, не се връща, Нито стига пътищата на живота,) -
\par 20 За да ходиш ти в пътя на добрите, И да пазиш пътеките на праведните.
\par 21 Защото правдивите ще населят земята, И непорочните ще останат в нея,
\par 22 А нечестивите ще се отсекат от земята, И коварните ще се изкоренят от нея.

\chapter{3}

\par 1 Сине мой, не забравяй поуката ми, И сърцето ти нека пази заповедите ми,
\par 2 Защото дългоденствие, години от живот И мир ще ти притурят те.
\par 3 Благост и вярност нека не те оставят; Вържи ги около шията си, Начертай ги на плочата на сърцето си.
\par 4 Така ще намериш благоволение и добро име Пред Бога и човеците.
\par 5 Уповавай на Господа от все сърце И не се облягай на своя разум.
\par 6 Във всичките си пътища признавай Него, И Той ще оправя пътеките ти.
\par 7 Не мисли себе си за мъдър; Бой се от Господа, и отклонявай се от зло;
\par 8 Това ще бъде здраве за тялото ти И влага за косите ти.
\par 9 Почитай Господа от имота си И от първаците на всичкия доход.
\par 10 Така ще се изпълнят житниците ти с изобилие, И линовете ти ще се преливат с ново вино.
\par 11 Сине мой, не презирай наказанието от Господа, И да ти не дотегва, когато Той те изобличава,
\par 12 Защото Господ изобличава оня, когото люби, Както и бащата сина, който му е мил.
\par 13 Блажен оня човек, който е намерил мъдрост, И човек, който е придобил разум,
\par 14 Защото търговията с нея е по-износна от търговията със сребро, И печалбата от нея по-скъпа от чисто злато..
\par 15 Тя е по-скъпа от безценни камъни И нищо, което би пожелал ти, не се сравнява с нея.
\par 16 Дългоденствие е в десницата й, А в левицата й богатство и слава.
\par 17 Пътищата й са пътища приятни, И всичките й пътеки мир.
\par 18 Тя е дело на живот за тия, които я прегръщат И блажени са ония, които я държат.
\par 19 С мъдрост Господ основа земята, С разум утвърди небето.
\par 20 Чрез Неговото знание се разтвориха бездните И от облаците капе роса.
\par 21 Сине мой, тоя неща да се не отдалечават от очите ти; Пази здравомислие и разсъдителност,
\par 22 Така те ще бъдат живот на душата ти И украшение на шията ти.
\par 23 Тогава ще ходиш безопасно по пътя си, И ногата ти не ще се спъне.
\par 24 Когато лягаш не ще се страхуваш; Да! ще лягаш и сънят ти ще бъде сладък.
\par 25 Не ще се боиш от внезапен страх, Нито от бурята, когато нападне нечестивите,
\par 26 Защото Господ ще бъде твое упование, И ще пази ногата то да се не хване.
\par 27 Не въздържай доброто от ония, на които се дължи, Когато ти дава ръка да им го направиш.
\par 28 Не казвай на ближния си: Иди върни се пак, И ще ти дам утре, Когато имаш при себе си това, което му се пада .
\par 29 Не измисляй зло против ближния си, Който с увереност живее при тебе.
\par 30 Не се карай с него без причина, Като не ти е направил зло.
\par 31 Не завиждай на насилник човек, И не избирай ни един от пътищата му,
\par 32 Защото Господ се гнуси от опакия, Но интимно общува с праведните.
\par 33 Проклетия от Господа има в дома на нечестивия; А Той благославя жилището на праведните.
\par 34 Наистина Той се присмива на присмивачите. А на смирените дава благодат.
\par 35 Мъдрите ще наследят слава, А безумните ще отнесат срам.

\chapter{4}

\par 1 Послушайте, чада, бащина поука, И внимавайте да научите разум.
\par 2 Понеже ви давам добро учение, Не оставяйте наставлението ми.
\par 3 Защото и аз бях син на баща си, Гален и безподобен на майка си,
\par 4 И той ме наставляваше и ми казваше: Нека държи сърцето ти думите ми, Пази заповедите ми и ще живееш,
\par 5 Придобий мъдрост, придобий разум; Не забравяй, нито се отклонявай от думите на устата ми.
\par 6 Не я оставяй и тя ще те пази. Обичай я - и ще те варди.
\par 7 Главното е мъдрост; затова придобивай мъдрост, И при всичко, що си придобил, придобивай разум.
\par 8 Въздигай я и тя ще те въздигне, Когато я прегърнеш, ще ти докара слава.
\par 9 Ще положи на главата ти красив венец. Ще ти даде славна корона
\par 10 Слушай, сине мой, и приеми думите ми, И годините на живота ти ще се умножат.
\par 11 Наставлявал съм те в пътя на мъдростта, Водил съм те по прави пътеки.
\par 12 Когато ходиш стъпките ти не ще бъдат стеснени; И когато тичаш, няма да се спънеш.
\par 13 Хвани се здраво за поуката, недей я оставя; Пази я, понеже тя е животът ти.
\par 14 Не влизай в пътеките на нечестивите, И не ходи по пътя на лошите.
\par 15 Отбягвай от него, не минавай край него. Отклони се от него и замини.
\par 16 Защото те не заспиват, око не сторят зло, И сън не ги хваща, ако не спънат някого.
\par 17 Понеже ядат хляб на нечестие, И пият вино на насилство.
\par 18 Но пътя на праведните е като виделото на разсъмване, Което се развиделява, догдето стане съвършен ден.
\par 19 Пътят на нечестивите е като тъмнина; Не знаят от що се спъват.
\par 20 Сине мой, внимавай на думите ми, Приклони ухото си към беседите ми.
\par 21 Да се не отдалечат от очите ти. Пази ги дълбоко в сърцето си;
\par 22 Защото те са живот за тия, които ги намират, И здраве за цялата им снага.
\par 23 Повече от всичко друго що пазиш, пази сърцето си, Защото от него са изворите на живота.
\par 24 Отмахни от себе си опърничави уста, И отдалечи от себе си развратени устни.
\par 25 Очите ти нека гледат право напред, И клепачите ти нека бъдат оправени право пред тебе.
\par 26 Обмисляй внимателно пътеката на нозете си, И всичките ти пътища нека бъдат добре уредени.
\par 27 Не се отбивай ни на дясно ни на ляво; Отклони ногата си от зло.

\chapter{5}

\par 1 Сине мой, внимавай в мъдростта ми. Приклони ухото си към разума ми
\par 2 За да опазиш разсъдливост, И устните ти да пазят знание.
\par 3 Защото от устните на чуждата жена капе мед. И устата й са по-меки от дървено масло;
\par 4 Но сетнините й са горчиви като пелин, Остри като изострен от двете страни меч.
\par 5 Нозете й слизат в смърт, Стъпките й стигат до ада,
\par 6 Тъй че тя никога не намира пътя на живота; Нейните пътеки са непостоянни, и тя не знае на къде водят.
\par 7 Прочее, чада, слушайте мене, И не отстъпвайте от думите на устата ми.
\par 8 Отдалечи пътя си от нея. И не се приближавай до вратата на къщата й,
\par 9 Да не би да дадеш жизнеността си на други. И годините си на немилостивите; -
\par 10 Да не би да се наситят чужди от имота ти, И трудовете ти да отидат в чужд дом;
\par 11 А ти да охкаш в сетнините си, Когато месата ти и тялото ти се изнурят,
\par 12 И да казваш: Как можах да намразя поуката, И сърцето ми да презре изобличението,
\par 13 И аз да не послушам гласа на учителите си, Нито да приклоня ухото си към наставниците си;
\par 14 Малко остана да изпадна във всяко зло Всред събранието и множеството.
\par 15 Пий вода от своята си щерна, И оная, която извира от твоя кладенец
\par 16 Вън ли да се изливат изворите ти, И водни потоци по улиците?
\par 17 Нека бъдат само на тебе, А не на чужди заедно с тебе.
\par 18 Да бъде благословен твоят извор, И весели се с жената на младостта си.
\par 19 Тя да ти бъде като любезна кошута и мила сърна: Нейните гърди да те задоволяват във всяко време; И възхищавай се винаги от нейната любов
\par 20 Понеже, сине мой, защо да се възхищаваш от чужда жена, И да прегръщаш обятията на чужда жена?
\par 21 Защото пътищата на човека се пред очите на Господа, И Той внимателно измерва всичките му пътеки.
\par 22 Нечестивият ще бъде хванат от собствените си беззакония, И с въжетата на своя грях ще бъде държан.
\par 23 Той ще умре от своето отказване от поука; И от голямото си безумие ще се заблуди.

\chapter{6}

\par 1 Сине мой, ако си станал поръчител на ближния си, Или си дал ръка за някой чужд,
\par 2 Ти си се впримчил с думите на устата си, Хванат си с думите на устата си.
\par 3 Затова, сине мой, направи това и остави се, Тъй като си паднал в ръцете на ближния си, - Иди, припадни и моли настоятелно ближния си.
\par 4 Не давай сън на очите си, Нито дрямка на клепачите си,
\par 5 Догдето не се отървеш, като сърна от ръката на ловеца , И като птица от ръката на птицоловец.
\par 6 Иди при мравката, о ленивецо, Размишлявай за постъпките й и стани мъдър, -
\par 7 Който, макар че няма началник, Надзирател, или управител,
\par 8 Приготвя си храната лете, Събира яденето си в жътва.
\par 9 До кога ще спиш лицемерецо? Кога ще станеш от съня си?
\par 10 Още малко спане, малко дрямка, Малко сгъване на ръце за сън,
\par 11 Така ще дойде сиромашия върху тебе, като разбойник, И немотия, като въоръжен мъж.
\par 12 Човек нехранимайко, човек беззаконен, Е оня който ходи с извратени уста,
\par 13 Намигва с очите си, говори с нозете си, Дава знак с пръстите си;
\par 14 Който има извратено сърце, Непрестанно крои зло, сее раздори,
\par 15 Затова погубването му ще дойде внезапно; Изведнъж ще се съкруши, и то непоправимо.
\par 16 Шест неща мрази Господ, Даже седем са мерзост за душата Му:
\par 17 Надменни очи, лъжлив език, Ръце, които проливат невинна кръв,
\par 18 Сърце, което крои лоши замисли, Нозе, които бърже тичат да вършат зло,
\par 19 Неверен свидетел, който говори лъжа, И оня, който сее раздори между братя.
\par 20 Сине, мой, пази заповедта на баща си, И не отстъпвай от наставлението на майка си,
\par 21 Вържи ги за винаги за сърцето си, Увий ги около шията си.
\par 22 Когато ходиш наставлението ще те води, Когато спиш, ще те пази; Когато се събудиш ще се разговаря с тебе,
\par 23 Защото заповедта им е светилник, И наставлението им е светлина, И поучителните им изобличения са път към живот.
\par 24 За да те пазят от лоша жена, От ласкателния език на чужда жена.
\par 25 Да не пожелаеш хубостта й в сърцето си; Да не те улови с клепачите си;
\par 26 Защото поради блудница човек изпада в нужда за парче хляб; А прелюбодейката лови скъпоценната душа.
\par 27 Може ли някой да тури огън в пазухата си, И дрехите му да не изгорят?
\par 28 Може ли някой да ходи по разпалени въглища, И нозете му да се не опекат?
\par 29 Така е с оня, който влиза при жената на ближния си; Който се допре до нея не ще остане ненаказан.
\par 30 Дори крадецът не се пропуска ненаказан, Даже ако краде да насити душата си, когато е гладен;
\par 31 И ако се хване, той трябва да възвърне седмократно, Трябва да даде целия имот на къщата си.
\par 32 Оня, който прелюбодействува с жена е безумен. Който прави това би погубил душата си.
\par 33 Биене и позор ще намери, И срамът му няма да се изличи,
\par 34 Защото ревнуването на мъжа е една ярост; И той няма да пожали в деня на възмездието;
\par 35 Не ще иска да знае за никакъв откуп, Нито ще се умилостиви, ако и да му дадеш много подаръци.

\chapter{7}

\par 1 Сине мой, пази думите ми, И запазвай заповедите ми при себе си.
\par 2 Пази заповедите ми и ще живееш - И поуката ми, като зеницата на очите си.
\par 3 Вържи ги за пръстите си, Начертай ги на плочата на сърцето си,
\par 4 Кажи на мъдростта: Сестра ми си; И наречи разума сродник,
\par 5 За да те пазят от чужда жена, От чужда жена, която ласкае с думите си.
\par 6 Понеже, като погледнах през решетките На прозореца на къщата си
\par 7 Видях между безумните, Съгледах между младежите, Един млад, безумен човек.
\par 8 Който минаваше по улицата близо до ъгъла й, И отиваше по пътя към къщата й.
\par 9 Беше в дрезгавината, когато се свечери, В мрака на нощта и в тъмнината.
\par 10 И посрещна го жена, Облечена като блудница и с хитро сърце;
\par 11 (Бъбрица и упорита, - Нозете й не остават в къщи
\par 12 Кога по улиците кога по площадите, Тя причаква при всеки ъгъл);
\par 13 Като го хвана, целуна го И с безсрамно лице му каза:
\par 14 Като бях задължена да принеса примирителни жертви, Днес изпълних обреците си,
\par 15 Затова излязох да те посрещна С желание да видя лицето ти и намерих те.
\par 16 Постлала съм леглото с красиви покривки, С шарени платове от египетска прежда.
\par 17 Покрила съм леглото си Със смирна, алой и канела.
\par 18 Ела, нека се наситим с любов до зори. Нека се насладим с милувки.
\par 19 Защото мъжът ми не е у дома. Замина на дълъг път;
\par 20 Взе кесия с пари в ръката си, Чак на пълнолуние ще се върне у дома.
\par 21 С многото си предумки тя го прелъга, Привлече го с ласкателството на устните си.
\par 22 Изведнъж той тръгна подире й, Както отива говедо на клане, Или както безумен в окови за наказание,
\par 23 Докато стрела прониза дроба му, - Както птица бърза към примката, без да знае, че това е против живота й.
\par 24 Сега, прочее, чада, послушайте ме. И внимавайте в думите на устата ми.
\par 25 Да се не уклонява сърцето ти в пътищата й, Да се не заблудиш в пътеките й;
\par 26 Защото мнозина е направила да паднат ранени; И силни са всичките убити от нея.
\par 27 Домът й е път към ада, И води надолу в клетките на смъртта.

\chapter{8}

\par 1 Не вика ли мъдростта? И разум не издава ли гласа си?
\par 2 Тя стои по високите места край пътя, На кръстопътя;
\par 3 Възгласява на портите, при входа на града, При входа на вратите:
\par 4 Към вас, човеци, викам, И гласът ми е към човешките чада.
\par 5 Вие, глупави, разберете благоразумие, И вие, безумни, придобивайте разумно сърце
\par 6 Послушайте,, защото ще говоря хубави неща. И ще отворя устните си да произнеса правото.
\par 7 Защото езикът ми ще изговори истина. И нечестието е мерзост за устните ми.
\par 8 Всичките думи на устата ми са справедливи, Няма в тях нищо лъжливо или опако.
\par 9 Те всички са ясни за разумния човек, И прави за тия, които намират знание.
\par 10 Приемете поуката ми, а не сребро, И по-добре знание, нежели избрано злато.
\par 11 Защото мъдростта е по-добра от скъпоценни камъни, И всичко желателно не се сравнява с нея.
\par 12 Аз, мъдростта, обитавам с благоразумието, И издирвам знание на умни мисли.
\par 13 Страх от Господа е да се мрази злото. Аз мразя гордост и високоумие, Лош път и опаки уста.
\par 14 У мене е съветът и здравомислието; Аз съм разум; у мен е силата.
\par 15 Чрез мене царете царуват И началниците узаконяват правда.
\par 16 Чрез мене князете началствуват, Тоже и големците и всичките земни съдии.
\par 17 Аз любя ония, които ме любят, И ония, които ме търсят ревностно, ще ме намерят.
\par 18 Богатството и славата са с мене; Да! трайният имот и правдата.
\par 19 Плодовете ми са по-добри от злато, даже от най-чисто злато, И приходът от мене от избрано сребро.
\par 20 Ходя по пътя на правдата. Всред пътеките на правосъдието,
\par 21 За да направя да наследят имот тия, които ме любят, И за да напълня съкровищницата им.
\par 22 Господ ме създаде като начало на пътя Си, Като първо от древните Си дела.
\par 23 От вечността бях създадена от начало, Преди създаването на земята.
\par 24 Родих се, когато нямаше бездните, Когато нямаше извори изобилващи с вода.
\par 25 Преди да се поставят планините, Преди хълмовете аз бях родена,
\par 26 Докато Господ още беше направил земята, нито полетата, Нито първите буци пръст на света.
\par 27 Когато приготовляваше небето, аз бях там, когато разпростираше свод над лицето на бездната.
\par 28 Когато закрепваше облаците горе, Когато усилваше изворите на бездната,
\par 29 Когато налагаше закона Си на морето, Щото водите да не престъпват повелението Му, Когато нареждаше основите на земята.
\par 30 Тогава аз бях при Него, като майсторски работник, И всеки ден се наслаждавах, Веселях се винаги пред Него.
\par 31 Веселях се на обитаемата Му земя; И наслаждението ми бе с човешките чада.
\par 32 Сега, прочее, послушайте ме, о, чада, Защото блажени са ония, които пазят моите пътища.
\par 33 Послушайте поука, Не я отхвърляйте и станете мъдри.
\par 34 Блажен тоя човек, който ме слуша, Като бди всеки ден при моите порти, И чака при сълбовете на вратата ми,
\par 35 Защото който ме намери намира живот, И придобива благоволение от Господа;
\par 36 А който ме пропуска онеправдава своята си душа; Всички, които мразят мене, обичат смъртта.

\chapter{9}

\par 1 Мъдростта съгради дома си, Издяла седемте си стълба,
\par 2 Закла животните си, смеси виното си И сложи трапезата си,
\par 3 Изпрати слугите си, Вика по високите места на града:
\par 4 Който е прост, нека се отбие тук. И на безумните казва:
\par 5 Елате, яжте от хляба ми, И пийте от виното, което смесих,
\par 6 Оставете глупостта и живейте, И ходете по пътя на разума,
\par 7 Който поправя присмивателя навлича на себе си срам; И който изобличава нечестивия лепва на себе си петно.
\par 8 Не изобличавай присмивателя, да не би да те намрази. Изобличавай мъдрия и той ще те обикне.
\par 9 Давай наставление на мъдрия и той ще стане по-мъдър; Учи праведния и ще стане по-учен.
\par 10 Страх от Господа е начало на мъдростта; И познаването на Светия е разум.
\par 11 Защото чрез мене ще се умножават дните ти. И ще ти се притурят години на живот.
\par 12 Ако станеш мъдър, ще бъдеш мъдър за себе си; И ако се присмееш, ти сам ще понасяш.
\par 13 Безумната жена е бъбрица, Проста е и не знае нищо.
\par 14 Седи при вратата на къщата си, На стол по високите места на града,
\par 15 И кани ония, които минават, Които вървят право в пътя си, като им казва :
\par 16 Който е прост, нека се отбие тук; А колкото за безумния, нему казва:
\par 17 Крадените води са сладки, И хляб, който се яде скришом, е вкусен,
\par 18 Но той не знае, че мъртвите са там, И че гостите й са в дълбочината на ада.

\chapter{10}

\par 1 Притчи Соломонови. Мъдър син радва баща си. А безумен син е тъга за майка си.
\par 2 Съкровища придобити с неправда не ползуват; А правдата избавя от смърт.
\par 3 Господ не ще остави да гладува душата на праведния; Не отхвърля злобното желание на нечестивите.
\par 4 Ленивата ръка докарва сиромашия, А трудолюбивата ръка обогатява.
\par 5 Който събира лятно време, той е разумен син, А който спи в жетва, той е син, който докарва срам.
\par 6 Благословения почиват на главата на праведния; Но устата на нечестивите покриват насилство.
\par 7 Паметта на праведния е благословена, А името на нечестивите ще изгние.
\par 8 Мъдрият по сърце приема заповеди; А безумен бъбрица пада.
\par 9 Който ходи непорочно, ходи безопасно, А който изкривява пътищата си ще се познае.
\par 10 Който намигва с око докарва скръб, А безумен бъбрица пада.
\par 11 Устата на праведния са извор на живот, А устата на нечестивите покриват насилие.
\par 12 Омразата повдига раздори, А любовта покрива всички погрешки.
\par 13 В устните на разумния се намира мъдрост, А тоягата е за гърба на безумния.
\par 14 Мъдрите запазват знанието, А устата на безумния близка погибел.
\par 15 Имотът на богатия е неговият укрепен град, А съсипия за бедните е тяхната сиромашия.
\par 16 Заплатата на праведния е живот. А благоуспяването на нечестивия е за грях.
\par 17 Който внимава на изобличението е по пътя към живот. А който пренебрегва мъмренето, заблуждава се.
\par 18 Който скришно мрази има лъжливи устни; И който възгласява клевета е безумен.
\par 19 В многото говорене грехът е неизбежен; Но който въздържа устните си е разумен.
\par 20 Езикът на праведния е избрано сребро; Сърцето на нечестивите малко струва.
\par 21 Устните на праведния хранят мнозина; А безумните умират от нямане на разум.
\par 22 Благословението Господно обогатява; И трудът на човека не ще притури нищо.
\par 23 Злотворството е като забавление за безумния. Така и мъдростта на разумния човек.
\par 24 Това, от, което се страхува нечестивият, ще го постигне; А желанието на праведните ще се изпълни.
\par 25 Както отминава вихрушката, така и нечестивият изчезва; А праведният има вечна основа.
\par 26 Както е оцетът за зъбите и димът за очите, Така е ленивият за ония, които го пращат.
\par 27 Страхът от Господа придава дни, А годините на нечестивите се съкратяват.
\par 28 Надеждата на праведните е радост, А очакването на нечестивите е напразно.
\par 29 Пътят Господен е крепост за непорочния И съсипване за ония, които вършат беззаконие.
\par 30 Праведният никога няма да се поклати, А нечестивите няма да населят земята,
\par 31 Из устата на праведния блика мъдрост, А лъжливият език ще се отреже.
\par 32 Устните на праведния знаят приятното за слушане ; А устата на нечестивите говорят извратеното.

\chapter{11}

\par 1 Неточни везни са мерзост Господу; А точни грамове са угодни Нему.
\par 2 Дойде ли гордост, дохожда и срам. А мъдростта е със смирените.
\par 3 Незлобието на справедливите ще ги ръководи, А извратеността на коварните ще ги погуби.
\par 4 В ден на гняв богатството не ползува, А правдата избавя от смърт.
\par 5 Правдата на непорочния ще оправя пътя му, А нечестивият ще падне чрез своето нечестие.
\par 6 Правдата на справедливите ще ги избави, А коварните ще се хванат в злобата си.
\par 7 Като умира нечестивият, ожиданото от него загива; Така загива надеждата на насилниците.
\par 8 Праведният се отървава от беда, А вместо него изпада в нея нечестивият.
\par 9 Безбожният погубва ближния си с устата си. Но чрез знанието на праведните ще се избави.
\par 10 Когато благоденствуват праведните, градът се весели; И когато погиват нечестивите става тържество.
\par 11 Чрез благословението върху праведните градът се въздига, А чрез устата на нечестивите се съсипва.
\par 12 Който презира ближния си е скудоумен; А благоразумният човек мълчи.
\par 13 Одумникът обхожда и открива тайните, А верният духом потайва работата.
\par 14 Дето няма мъдро ръководене народът пропада, А в многото съветници има безопасност.
\par 15 Който поръчителствува за чужд човек, зле ще пострада, А който мрази поръчителството е в безопасност.
\par 16 Благодатната жена придобива чест; И насилниците придобиват богатство.
\par 17 Милостивият човек струва добро на себе си, А жестокият смущава своето тяло.
\par 18 Нечестивият придобива измамлива печалба, А който сее правда има сигурна награда.
\par 19 Който е утвърден в правдата, ще стане живот, А който се стреми към злото съдействува за своята смърт.
\par 20 Развратените в сърце са мерзост Господу, А непорочните в пътя си са угодни Нему,
\par 21 Даже ръка с ръка да се съедини пак  нечестивият няма да остане ненаказан., А потомството на праведните ще се избави.
\par 22 Както е златна халка на носа на свиня, Така е красивата, но безразсъдна жена.
\par 23 Желаното от праведните е само добро, А ожиданото от нечестивите е надменност.
\par 24 Един разпръсва щедро , но пак има повече изобилие, А друг се скъпи без мяра, но пак стига до немотия,
\par 25 Благотворната душа ще бъде наситена; И който пои, сам ще бъде напоен.
\par 26 Който задържа жито ще бъде прокълнат от народа, А който продава, благословение ще почива на главата му.
\par 27 Който усърдно търси доброто, търси и благоволение, А който търси злото, то ще дойде и върху него.
\par 28 Който уповава на богатството си, ще падне, А праведните ще цъфтят като зелен лист.
\par 29 Който смущава своя си дом ще наследи вятър; И безумният ще стане слуга на мъдрия по сърце.
\par 30 Плодът на праведния е дърво на живот; И който е мъдър придобива души.
\par 31 Ето, и на праведния се въздава на земята, - Колко повече на нечестивия и на грешния!

\chapter{12}

\par 1 Който обича поправление, обича знание, Но който мрази изобличения е невеж.
\par 2 Добрият човек намира благоволение пред Господа; А зломилсеника Той осъди.
\par 3 Човек няма да се утвърди чрез беззаконие, А коренът на праведните не ще се поклати.
\par 4 Добродетелната жена е венец на мъжа си; А оная, който докарва срам, е като гнилота в костите му.
\par 5 Мислите на праведните са справедливи, А намеренията на нечестивите са коварство.
\par 6 Думите на нечестивите са засада за кръвопролитие; А устата на праведните ще ги избавят.
\par 7 Нечестивите се съсипват и няма ги, А домът на праведните ще стои.
\par 8 Човек бива похвален според разума си, А опакият в сърце ще бъде поругаван.
\par 9 По-щастлив е скромният, който слугува на себе си, От този, който се надига и няма хляб.
\par 10 Праведният се грижи за живота на добитъка си, А благостите на нечестивите са немилостиви.
\par 11 Който обработва земята си ще се насити с хляб, А който следва суетни неща е без разум.
\par 12 Нечестивият търси такава корист, каквато вземат злите, А коренът на праведния дава плод .
\par 13 В престъплението на устните се намира опасна примка, А праведният ще се отърве от затруднение.
\par 14 От плода на устните си човек се насища с добрини; И според делата на ръцете на човека му се въздава.
\par 15 Пътят на безумния е прав в неговите очи, А който е мъдър, той слуша съвети.
\par 16 Безумният показва явно отегчението си, А благоразумният скрива оскърблението.
\par 17 Който диша истина възвестява правдата, А лъжесвидетелят - измамата.
\par 18 Намират се такива, чието несмислено говорене пронизва като нож, А езикът на мъдрите докарва здраве,
\par 19 Устните, които говорят истината, ще се утвърдят за винаги, А лъжливият език ще трае за минута.
\par 20 Измама има в сърцето на ония, които планират зло; А радост имат тия, които съветват за мир.
\par 21 Никаква пакост няма да се случи на праведния, А нечестивите ще се изпълнят с злощастие.
\par 22 Лъжливите устни са мерзост Господу, А ония, които постъпват вярно, са приятни Нему.
\par 23 Благоразумният човек покрива знанието си . А сърцето на безумните наказва глупостта си .
\par 24 Ръката на трудолюбивите ще властвува, А ленивите ще бъдат подчинени.
\par 25 Теготата смирява човешкото сърце, А благата дума го развеселява.
\par 26 Праведният води ближния си, А пътят на нечестивите въвежда самите тях в заблуждение.
\par 27 Ленивият не пече лова си; Но скъпоценностите на човеците са на трудолюбивия.
\par 28 В пътя на правдата има живот, И в пътеката й няма смърт.

\chapter{13}

\par 1 Мъдрият син слуша бащината си поука, А присмивателят не внимава на изобличение.
\par 2 От плодовете на устата си човек ще се храни с добрини, А душата на коварните ще яде насилство.
\par 3 Който пази устата си, опазва душата си, А който отваря широко устните си ще погине.
\par 4 Душата на ленивия желае и няма, А душата на трудолюбивите ще се насити.
\par 5 Праведният мрази лъжата, А нечестивият постъпва подло и срамно.
\par 6 Правдата пази ходещия непорочно, А нечестието съсипва грешния.
\par 7 Един се преструва на богат, а няма нищо; Друг се преструва на сиромах, но има много имот.
\par 8 Богатството на човека служи за откуп на живота му; А сиромахът не внимава на заплашвания.
\par 9 Виделото на праведните е весело, А светилникът на нечестивите ще изгасне.
\par 10 От гордостта произхожда само препиране, А мъдростта е с ония, които приемат съвети.
\par 11 Богатството придобито чрез измама ще намалее, А който събира с ръката си ще го умножи.
\par 12 Отлагано ожидане изнемощява сърцето, А постигнатото желание е дърво на живот.
\par 13 Който презира словото, сам на себе си вреди, А който почита заповедта има отплата.
\par 14 Поуката на мъдрия е извор на живот. За да отбягва човек примките на смъртта.
\par 15 Здравият разум дава благодат, А пътят на коварните е неравен.
\par 16 Всеки благоразумен човек работи със знание, А безумният разсява глупост,
\par 17 Лошият пратеник изпада в зло, А верният посланик дава здраве.
\par 18 Сиромашия и срам ще постигнат този, който отхвърля поука, А който внимава на изобличение ще бъде почитан.
\par 19 Изпълнено желание услажда душата, А на безумните е омразно да се отклоняват от злото.
\par 20 Ходи с мъдрите, и ще станеш мъдър, А другарят на безумните ще пострада зле.
\par 21 Злото преследва грешните, А на праведните ще се въздаде добро.
\par 22 Добрият оставя наследство на внуците си, А богатството на грешния се запазва за праведния,
\par 23 Земеделието на сиромасите доставя много храна, Но някои погиват от липса на разсъдък.
\par 24 Който щади тоягата си, мрази сина си, А който го обича наказва го на време.
\par 25 Праведният яде до насищане на душата си, А коремът на нечестивите не ще се задоволи.

\chapter{14}

\par 1 Всяка мъдра жена съгражда дома си, А безумната го събаря със собствените си ръце.
\par 2 Който ходи в правотата си, бои се от Господа: Но опакият в пътищата си Го презира.
\par 3 В устата на безумния има пръчка за гордостта му , А устните на мъдрите ще ги пазят.
\par 4 Дето няма волове, яслите са чисти, Но в силата на воловете е голямото изобилие.
\par 5 Верният свидетел няма да лъже, А лъжливият свидетел издиша лъжи.
\par 6 Присмивателят търси мъдрост и нея намира, А за разумният учението е лесно.
\par 7 Отмини безумния човек Щом си узнал, че той няма разумни устни.
\par 8 Мъдростта на благоразумния е да обмисля пътя си, А глупостта на безумните е да заблуждават.
\par 9 Приносът за грях се присмива на безумните, А между праведните има Божие благоволение.
\par 10 Сърцето познава своята си горест И чужд не участвува в неговата радост.
\par 11 Къщата на нечестивите ще се събори, Но шатърът на праведните ще благоденствува.
\par 12 Има път, който се вижда прав на човека, Но краят му е пътища към смърт.
\par 13 Даже и всред смеха сърцето си има болката, И краят на веселието е тегота.
\par 14 Развратният по сърце ще се насити от своите пътища, А добрият човек ще се насити от себе си.
\par 15 Простият вярва всяка дума, А благоразумният внимава добре в стъпките си.
\par 16 Мъдрият се бои и се отклонява от злото, А безумният самонадеяно се хвърля напред.
\par 17 Ядовитият човек постъпва несмислено, И зломисленикът е мразен.
\par 18 Безумниите наследяват глупост, А благоразумните се увенчават със знание.
\par 19 Злите се кланят пред добрите, И нечестивите при портите на праведния,
\par 20 Сиромахът е мразен даже от ближния си, А на богатия приятелите са много.
\par 21 Който презира ближния си, съгрешава, А който показва милост към сиромасите е блажен.
\par 22 Не заблуждават ли се ония, които измислят зло? Но милост и верност ще се покажат към тия, които измислят добро
\par 23 От всеки труд има полза, А от бъбренето с устните само оскъдност.
\par 24 Богатството на мъдрите е венец за тях, А глупостта на безумните е всякога глупост.
\par 25 Верният свидетел избавя души, А който издиша лъжи е цял измама.
\par 26 В страха от Господа има силна увереност, И Неговите чада ще имат прибежище.
\par 27 Страхът от Господа е извор на живот, За да се отдалечава човек от примките на смъртта,
\par 28 Когато людете са многочислени, слава е за царя, А когато людете са малочислени, съсипване е за княза.
\par 29 Който не се гневи скоро, показва голямо благоразумие, А който лесно се гневи проявява безумие.
\par 30 Тихо сърце е живот на тялото, А разяреността е гнилост на костите.
\par 31 Който угнетява бедния нанася укор на Създателя му, А който е милостив към сиромаха показва почит Нему.
\par 32 Нечестивият е смазан във време на бедствитето си, А праведният и в смъртта си име упование.
\par 33 В сърцето на разумния мъдростта почива, А между безумните тя се явява.
\par 34 Правдата възвишава народ, А грехът е позор за племената.
\par 35 Благоволението на царя е към разумния слуга, А яростта му против онзи, който докарва срам.

\chapter{15}

\par 1 Мек отговор отклонява ярост, А оскърбителната дума възбужда гняв.
\par 2 Езикът на мъдрите изказва знание, А устата на безумните изригват глупост.
\par 3 Очите Господни са на всяко място И наблюдава злите и добрите.
\par 4 Благият език е дърво на живот, А извратеността в него съкрушава духа.
\par 5 Безумният презира поуката на баща си, Но който внимава в изобличението, е благоразумен.
\par 6 В дома на праведния има голямо изобилие, А в доходите на нечестивия има загриженост.
\par 7 Устните на мъдрите разсяват знание, А сърцето на безумните не прави така.
\par 8 Жертвата на нечестивите е мерзост Господу, А молитвата на праведните е приятна Нему.
\par 9 Пътят на нечестивия е мерзост Господу, Но Той обича този, който следва правдата.
\par 10 Има тежко наказание за ония, които се отбиват от пътя; И който мрази изобличение ще умре.
\par 11 Адът и погибелта са открити пред Господа, - Колко повече сърцата на човешките чада!
\par 12 Присмивателят не обича изобличителя си, Нито ще отива при мъдрите.
\par 13 Весело сърце прави засмяно лице, А от скръбта на сърцето духът се съкрушава.
\par 14 Сърцето на разумния търси знание А устата на безумните се хранят с глупост.
\par 15 За наскърбения всичките дни са зли А оня, който е с весело сърце, има всегдашно пируване.
\par 16 По-добро е малкото със страх от Господа, Нежели много съкровища с безпокойствие.
\par 17 По-добра е гощавката от зеле с любов, Нежели хранено говедо с омраза.
\par 18 Яростният човек подига препирни, А който скоро не се гневи усмирява крамоли.
\par 19 Пътят на ленивия е като трънен плет, А пътят на праведните е като друм.
\par 20 Мъдър син радва баща си, А безумен човек презира майка си.
\par 21 На безумния глупостта е радост, А разумен човек ходи по прав път.
\par 22 Дето няма съвещание намеренията се осуетяват, А в множеството на съветниците те се утвърждават.
\par 23 От отговора на устата си човек изпитва радост, И дума на време казана , колко е добра!
\par 24 За разумния пътят на живота върви нагоре, За да се отклони от ада долу.
\par 25 Господ съсипва дома на горделивите, А утвърдява предела на вдовицата.
\par 26 Лошите замисли са мерзост Господу! А чистите думи Му са угодни.
\par 27 Користолюбивият смущава своя си дом, А който мрази даровете ще живее.
\par 28 Сърцето на праведния обмисля що да отговаря, А устата на нечестивите изригват зло.
\par 29 Господ е далеч от нечестивите, А слуша молитвата на праведните.
\par 30 Светъл поглед весели сърцето, И добри вести угояват костите.
\par 31 Ухо, което слуша животворното изобличение, Ще пребивава между мъдрите.
\par 32 Който отхвърля поуката презира своята си душа, А който слуша изобличението придобива разум.
\par 33 Страхът от Господа е възпитание в мъдрост, И смирението предшествува славата.

\chapter{16}

\par 1 Плановете на сърцето принадлежат на човека, Но отговорът на езика е от Господа.
\par 2 Всичките пътища на човека са чисти в собствените му очи, Но Господ претегля духовете.
\par 3 Възлагай делата си на Господа. И ще се утвърдят твоите намерения.
\par 4 Господ е направил всяко нещо за Себе Си, Дори и нечестивия за деня на злото.
\par 5 Мерзост е Господу всеки, който е с горделиво сърце, Даже ръка с ръка да се съедини, пак той няма да остане ненаказан.
\par 6 С милост и вярност се отплаща за беззаконието, И чрез страх от Господа хората се отклоняват от злото.
\par 7 Когато са угодни на Господа пътищата на човека, Той примирява с него и неприятелите му.
\par 8 По-добре малко с правда, Нежели големи доходи с неправда,
\par 9 Сърцето на човека начертава пътя му, Но Господ оправя стъпките му.
\par 10 Присъдата в устните на царя е боговдъхновена; Устата му няма да погрешат в съда.
\par 11 Вярната теглилка и везни са от Господа, Всичките грамове в торбата са Негово дело.
\par 12 Да се върши беззаконие е мерзост на царете, Защото престолът се утвърждава с правда.
\par 13 Праведните устни са благоприятни на царете, И те обичат онзи, който говори право.
\par 14 Яростта на царя е вестителка на смърт, Но мъдрият човек я укротява.
\par 15 В светенето пред лицето на царя има живот, И неговото благоволение е като облак с пролетен дъжд.
\par 16 Колко по-желателно е придобиването на мъдрост, нежели на злато! И придобиването на разум е за предпочитане, нежели на сребро.
\par 17 Да се отклонява от зло е друм за праведните; Който пази пътя си, опазва душата си.
\par 18 Гордостта предшествува погибелта, И високоумието - падането.
\par 19 По-добре да е някой със смирен дух между кротките, Нежели да дели користи с горделивите.
\par 20 Който внимава на словото ще намери добро. И който уповава на Господа е блажен.
\par 21 Който е с мъдро сърце ще се нарече благоразумен, И сладостта на устните умножава знание.
\par 22 Разумът е извор на живот за притежателя му, А глупостта на безумните е наказанието им.
\par 23 Сърцето на мъдрия вразумява устата му И притуря знание на устните му.
\par 24 Благите думи са медена пита, Сладост на душата и здраве на костите.
\par 25 Има път, който се вижда прав на човека. Но краят му е пътища към смърт,
\par 26 Охотата на работника работи за него, Защото устата му го принуждават.
\par 27 Лошият човек копае зло, И в устните му има сякаш пламнал огън.
\par 28 Опак човек сее раздори, И шепотникът разделя най-близки приятели.
\par 29 Насилникът измамя ближния си, И го води в недобър път;
\par 30 Склопя очите си, за да измисля извратени неща. И прехапва устните си, за да постигне зло.
\par 31 Белите коси са венец на слава, Когато се намират по пътя на правдата.
\par 32 Който скоро не се гневи е по-добър от храбрия, И който владее духа си - от завоевател на град.
\par 33 Жребието се хвърля в скута, Но решението чрез него е от Господа.

\chapter{17}

\par 1 По-добре сух залък и мир с него, Нежели къща пълна с пирования и разпра с тях.
\par 2 Благоразумен слуга ще владее над син, който докарва срам, И ще вземе дял от наследствотото между братята.
\par 3 Горнилото е за среброто и пещта за златото, А Господ изпитва сърцата.
\par 4 Злосторникът слуша беззаконните устни, И лъжецът дава ухо на лошия език.
\par 5 Който се присмива на сиромаха, нанася позор на Създателя му, И който се радва на бедствия, няма да остане ненаказан.
\par 6 Чада на чада са венец на старците, И бащите са слава на чадата им.
\par 7 Хубава реч не подхожда на безумния, - Много по-малко лъжливи устни на началника.
\par 8 Подаръкът е като скъпоценен камък в очите на притежателя му; дето и да бъде обърнат той се показва изящен.
\par 9 Който покрива престъпление търси любов, А който многодумствува за работата разделя най-близки приятели.
\par 10 Изобличението прави повече впечатление на благоразумния, Нежели сто бича на безумния.
\par 11 Злият човек търси само бунтове, Затова жесток пратеник е изпратен против него.
\par 12 По-добре да срещне някого мечка лишена от малките си, Отколкото безумен човек в буйството му.
\par 13 Който въздава зло за добро, Злото не ще се отдалечи от дома му.
\par 14 Започването на разпрата е като , кога някой отваря път на вода, Затова остави препирнята преди да има каране.
\par 15 Който оправдава нечестивия и който осъжда праведния. И двамата са мерзост за Господа.
\par 16 Що ползват парите в ръката на безумния, за да купи мъдрост, Като няма ум?
\par 17 Приятел обича всякога И е роден, като брат за във време на нужда.
\par 18 Човек без разум дава ръка И става поръчител на ближния си.
\par 19 Който обича препирни обича престъпления, И който построи високо вратата си, търси пагуба.
\par 20 Който има опако сърце не намира добро, И който има извратен език изпада в нечестие.
\par 21 Който ражда безумно чадо ще има скръб, И бащата на глупавия няма радост.
\par 22 Веселото сърце е благотворно лекарство, А унилият дух изсушава костите.
\par 23 Нечестивият приема подарък изпод пазуха, За да изкриви пътищата на правосъдието.
\par 24 Мъдростта е пред лицето на разумния, А очите на безумния са към краищата на земята.
\par 25 Безумен син е тъга на баща си И горест на тая която го е родила.
\par 26 Не е добре да се глобява праведния, Нито да се бие благородния, за справедливостта им .
\par 27 Който щади думите си е умен, И търпеливият човек е благоразумен.
\par 28 Даже и безумният, когато мълчи, се счита за мъдър, И когато затваря устата си се счита за разумен.

\chapter{18}

\par 1 Който се отлъчва от другите , търси само своето желание, И се противи на всеки здрав разум.
\par 2 Безумният не се наслаждава от благоразумието, Но само от изявяване сърцето си.
\par 3 С идването на нечестивия идва и презрение, И с подлостта идва и позор.
\par 4 Думите из устата на човека са като дълбоки води, И изворът на мъдростта е като поток.
\par 5 Не е добре да се приема нечестивия, Или да се изкривява съда на праведния.
\par 6 Устните на безумния причиняват препирни, И устата му предизвикват плесници.
\par 7 Устата на безумния са погибел за него, И устните му са примка за душата му.
\par 8 Думите на шепотника са като сладки залъци, И слизат вътре в корема.
\par 9 Немарливият в работата си Е брат на разсипника.
\par 10 Името Господно е яка кула; Праведният прибягва в нея, и е поставен на високо.
\par 11 Имотът на богатия е укрепен град за него, И той е висока стена във въображението му.
\par 12 Преди загиването сърцето на човека се превъзнася, И преди прославянето то се смирява.
\par 13 Да отговаря някой преди да чуе, Е безумие и позор за него.
\par 14 Духът на човека ще го подпира в немощта му; Кой може да подигне унилия дух?
\par 15 Сърцето на благоразумния придобива разум, И ухото на мъртвите търси знание.
\par 16 Подаръкът, който дава човек, отваря място за него, И го привежда пред големците.
\par 17 Който пръв защитава делото си изглежда да е прав, Но съседът му идва и го изпитва.
\par 18 Жребието прекратява разприте, И решава между силите.
\par 19 Брат онеправдан е по-недостъпен от укрепен град, И разногласията им са като лостове на крепост.
\par 20 От плодовете на устата на човека ще се насити коремът му; От произведението на устните си човек ще се насити.
\par 21 Смърт и живот има в силата на езика, И ония, които го обичат, ще ядат плодовете му.
\par 22 Който е намерил съпруга намерил е добро И е получил благоволение от Господа.
\par 23 Сиромахът говори с умолявания, Но богатият отговаря грубо.
\par 24 Човек, който има много приятели намира в това погубването си; Но има приятел, който се държи по-близко и от брат.

\chapter{19}

\par 1 По-добър е сиромахът, който ходи в непорочността си, Нежели оня, който е с извратени устни а при това безумен.
\par 2 Наистина ожидане без разсъдък не е добро, И който бърза с нозете си, обърква пътя си .
\par 3 Безумието на човека изкривява пътя му, И сърцето му негодува против Господа.
\par 4 Богатството притуря много приятели, А сиромахът бива оставен от приятеля си,
\par 5 Лъжливият свидетел няма да остане ненаказан, И който издиша лъжи няма да избегне.
\par 6 Мнозина търсят благоволението на щедрия, И всеки е приятел на онзи, който дава подаръци.
\par 7 Всичките братя на сиромаха го мразят, - Колко повече отбягват от него приятелите му! - Той тича след тях с умолителни думи, но тях ги няма.
\par 8 Който придобива ум обича своята си душа; Който пази благоразумие ще намери добро.
\par 9 Лъжлив свидетел няма да остане ненаказан, И който издиша лъжи ще загине.
\par 10 Изнежеността не прилича на безумен, - Много по-малко на слуга да властвува над началници.
\par 11 Благоразумието на човека възпира гнева му, И слава е за него да се не взира в престъпление.
\par 12 Гневът на царя е като реване на лъв, А благоволението му е като роса на тревата.
\par 13 Безумен син е бедствие за баща си, И препирните на жена са непрестанно капене.
\par 14 Къща и богатство се оставят наследство от бащите, Но благоразумна жена е от Господа.
\par 15 Леноста хвърля в дълбок сън, И бездейна душа ще гладува
\par 16 Който пази заповедта пази душата си, А който немари пътищата си ще загине.
\par 17 Който показва милост към сиромаха заема Господу, И Той ще му въздаде за благодеянието му.
\par 18 Наказвай сина си докато има надежда, И не закоравявай сърцето си да го оставиш да загине.
\par 19 Яростен човек ще понесе наказание, Защото, ако и да го избавиш, трябва пак същото да направиш.
\par 20 Слушай съвет и приемай поука, За да останеш мъдър в сетнините си.
\par 21 Има много помисли в сърцето на човека, Но намерението Господно, то ще устои.
\par 22 Милосърдието на човека е чест нему, И сиромах човек е по-добър от този, който разорява.
\par 23 Страхът от Господа спомага към живот; Който го има ще си ляга наситен и не ще срещне зло.
\par 24 Ленивият затопява ръката си в паницата И не ще нито в устата си да я повърне.
\par 25 Ако биеш присмивателя, простият ще стане внимателен; И ако изобличиш благоразумния, той ще придобие знание.
\par 26 Който опропастява баща си и пропъжда майка си, Той е син, който причинява срам и нанася позор.
\par 27 Престани, сине мой, да слушаш съвети, Които те отклоняват от мъдростта.
\par 28 Лошият свидетел се присмива на правосъдието; И устата на нечестивите поглъщат беззаконие.
\par 29 Присъди се приготвят за присмивателите, И бой за гърба на безумните.

\chapter{20}

\par 1 Виното е присмивател, и спиртното питие крамолник; И който се увлича по тях е неблагоразумен.
\par 2 Царското заплашване е като реване на лъв; Който го дразни съгрешава против своя си живот.
\par 3 Чест е за човека да страни от препирня; А всеки безумен се кара.
\par 4 Ленивият не иска да оре, поради зимата, Затова, когато търси във време на жътва, не ще има нищо.
\par 5 Намерението в сърцето на човека е като дълбока вода; Но разумен човек ще го извади.
\par 6 Повечето човеци разгласяват всеки своята доброта: Но кой може да намери верен човек?
\par 7 Чадата на праведен човек, който ходи в непорочността си, Са блажени след него.
\par 8 Цар, който седи на съдебен престол, Пресява всяко зло с очите си.
\par 9 Кой може да каже: Очистих сърцето си; Чист съм от греховете си?
\par 10 Различни грамове и различни мерки, И двете са мерзост Господу.
\par 11 Даже и детето се явява чрез постъпките си - Дали делата му са чисти и прави.
\par 12 Слушащото ухо и гледащото око, Господ е направил и двете.
\par 13 Не обичай спането, да не би да обеднееш! Отвори очите си, и ще се наситиш с хляб.
\par 14 Лошо е! лошо е! казва купувачът, Но като си отиде, тогава се хвали.
\par 15 Има злато и изобилие драгоценни камъни, Но устните на знанието са скъпоценнно украшение.
\par 16 Вземи дрехата на този, който поръчителствува за чужд. Да! вземи залог от онзи, който поръчителствува за чужди хора.
\par 17 Хлябът спечелен с лъжа е сладък за човека; Но после устата му ще се напълнят с камъчета.
\par 18 Намеренията се утвърждават чрез съвещание, Затова с мъдър съвет обяви война.
\par 19 Одумникът обхожда и открива тайни, Затова не се събирай с онзи, който отваря широко устните си.
\par 20 Светилникът на този, който злослови баща си или майка си, Ще изгасне в най-мрачната тъмнина.
\par 21 На богатството, което бързо се придобива из начало, Сетнината не ще бъде благословена.
\par 22 Да не речеш: Ще въздам на злото; Почакай Господа и Той ще се избави.
\par 23 Различни грамове са мерзост за Господа, И неверните везни на са добри.
\par 24 Стъпките на човека се оправят от Господа; Как, прочее, би познал човек пътя си?
\par 25 Примка е за човека да казва необмислено: Посвещавам това , И след като се е обрекъл тогава да разпитва.
\par 26 Мъдрият цар пресява нечестивите, И докарва върху тях колелото на вършачката .
\par 27 Духът на човека е светило Господно, Което изпитва всичките най-вътрешни части на тялото.
\par 28 Милост и вярност пазят царя, И той поддържа престола си с милост.
\par 29 Славата на младите е силата им, И украшението на старците са белите им коси.
\par 30 Бой, който наранява, И удари, които стигат до най-вътрешните части на тялото, Очистват злото.

\chapter{21}

\par 1 Сърцето на царя е в ръката на Господа, като водни бразди; Той на където иска го обръща.
\par 2 Всичките пътища на човека са прави в неговите очи, Но Господ претегля сърцата.
\par 3 Да върши човек правда и правосъдие Е по-угодно за Господа от жертва
\par 4 Надигнато око и горделиво сърце, Които за нечестивите са светилник, е грях.
\par 5 Мислите на трудолюбивите спомагат само да има изобилие, А на всеки припрян само - оскъдност.
\par 6 Придобиването на съкровища с лъжлив език е преходна пара; Които ги търсят, търсят смърт.
\par 7 Грабителството на нечестивите ще ги обрече, Защото отказват да върнат това, което е право.
\par 8 Пътят на развратния човек е твърде крив, А делото на чистия е право.
\par 9 По-добре да живее някой в ъгъл на покрива, Нежели в широка къща със свадлива жена.
\par 10 Душата на нечестивия желае зло, Ближният му не намира благоволение пред очите му.
\par 11 Когато се накаже присмивателя, простият става по-мъдър, И когато се поучава мъдрия, той придобива знание,
\par 12 Справедливият Бог наблюдава дома на нечестивия, Той съсипва нечестивите до унищожение.
\par 13 Който затуля ушите си за вика на сиромаха, - Ще викне и той, но няма да бъде послушан.
\par 14 Тайният подарък укротява ярост, И подаръкът в пазуха укротява силен гняв.
\par 15 Радост е на праведния да върши правосъдие, А измъчване е за ония, които вършат беззаконие.
\par 16 Човек, който се отбие в пътя на разума, Ще стигне в събранието на мъртвите.
\par 17 Който обича удоволствие осиромашява, Който обича вино и масло не забогатява.
\par 18 Нечестивият ще бъде откуп за праведния, И коварният наместо праведните.
\par 19 По-добре да живее някой в пуста земя, Нежели със свадлива жена и досада.
\par 20 Скъпоценно съкровище и масло се намират в жилището на мъдрия, А безумният човек ги поглъща.
\par 21 Който следва правда и милост, Намира живот, правда и милост,
\par 22 Мъдрият превзема с пристъп града на мощните, И събаря силата, на която те уповават.
\par 23 Който въздържа устата си и езика си Опазва душата си от смущения.
\par 24 Присмивател се нарича оня горделив и надменен човек, Който действува с високоумна гордост.
\par 25 Желанието на ленивия го умъртвява, Защото ръцете му не искат да работят
\par 26 Той се лакоми цял ден, А праведният дава и не му се свиди.
\par 27 Жертвата на нечестивите е мерзост, - Колко повече, когато я принасят за нечестива цел!
\par 28 Лъжливият свидетел ще загине, А човекът, който слуша поука - ще го търсят да говори всякога.
\par 29 Нечестивият човек прави дръзко лицето си, А праведният оправя пътищата си.
\par 30 Няма мъдрост, няма разум, Няма съвещание против Господа.
\par 31 Конят се приготвя за деня на боя, Но избавлението е от Господа.

\chapter{22}

\par 1 За предпочитане е добро име, нежели голямо богатство, И благоволение е по-добро от сребро и злато.
\par 2 Богат и сиромах се срещат; Господ е Създателят на всички тях.
\par 3 Благоразумният предвижда злото и се укрива. А неразумните вървят напред - и страдат.
\par 4 Наградата на смирението и на страха от Господа Е богатство, слава и живот.
\par 5 Тръне и примки има по пътя на опакия, Който пази душата си се отдалечава от тях.
\par 6 Възпитавай детето отрано в подходящия за него път, И не ще се отклони от него, дори когато остарее.
\par 7 Богатият властвува над сиромасите, И който взема на заем е слуга на заемодавеца.
\par 8 Който сее беззаконие ще пожъне бедствие, И жезълът на буйството му ще изчезне.
\par 9 Който има щедро око ще бъде благословен Защото дава от хляба си на сиромаха.
\par 10 Изпъди присмивателя и препирнята ще се махне, И свадата и позорът ще престанат.
\par 11 Който обича чистота в сърцето И има благодатни устни, царят ще му бъде приятел.
\par 12 Очите на Господа пазят онзи, който има знание, И той осуетява думите на коварния.
\par 13 Ленивецът казва: Лъв има вън! Ще бъда убит всред улиците!
\par 14 Устата на чужди жени са дълбока яма, И оня, на когото Господ се гневи, ще падне в нея.
\par 15 Безумието е вързано в сърцето на детето, Но тоягата на наказанието ще го изгони от него.
\par 16 Който угнетява сиромаха, за да умножи богатството си, И който дава на богатия, непременно ще изпадне в немотия.
\par 17 Приклони ухото си та чуй думите на мъдрите, И взимай присърце моето знание,
\par 18 Защото е приятно, ако ги пазиш вътре в себе си, И ако бъдат всякога готови върху устните ти.
\par 19 За да бъде упованието ти на Господа, Аз те научих на тях днес - да! тебе.
\par 20 Не писах ли ти хубави неща От съвет и знание.
\par 21 За да те направя да познаеш верността на думите на истината, Та да отговаряш с думи на истината на ония, които те пращат?
\par 22 Не оголвай сиромаха, защото той е беден, Нито притеснявай в портата угнетения,
\par 23 Защото Господ ще защити делото им, И ще оголи живота на ония, които са ги оголили.
\par 24 Не завързвай приятелство с ядовит човек, И не ходи с гневлив човек.
\par 25 Да не би да научиш пътищата му, И да приготвиш примка за душата си.
\par 26 Не бъди от тия, които дават ръка, От тия, които стават поръчители за дългове,
\par 27 Ако нямаш с какво да платиш, Защо да вземат постелката ти изпод тебе?
\par 28 Не премествай старите межди, Които са положили бащите ти.
\par 29 Видял ли си човек трудолюбив в работата си? Той ще стои пред царе, няма да стои пред неизвестни хора.

\chapter{23}

\par 1 Когато седнеш да ядеш с началник, Прегледай добре какво има пред тебе
\par 2 Иначе ще туриш нож в гърлото си. Ако те обладава охота,
\par 3 Не пожелавай вкусните му ястия, Защото те са примамливи гозби.
\par 4 Не се старай да придобиеш богатство, Остави се от тая си мисъл.
\par 5 Хвърляш ли на него очите си, - то го няма! Защото наистина богатството си прави крила, Както орел ще лети към небето.
\par 6 Не яж хляба на онзи, който има лошо око, Нито пожелавай вкусните му ястия,
\par 7 Защото, каквито са мислите в душата му - такъв е и той. Каза ти: Яж и пий, Но сърцето му не е с тебе.
\par 8 Залъка, който си изял, ще избълваш, И ще изгубиш сладките си думи.
\par 9 Не говори на ушите на безумния, Защото той ще презре разумността на думите ти.
\par 10 Не премествай стари межди, Нито влизай в нивите на сирачетата,
\par 11 Защото Изкупителят им е мощен; Той ще защити делото им против тебе.
\par 12 Предай сърцето си на поука И ушите си към думи на знание.
\par 13 Да не ти се свиди да наказваш детето, Защото, ако и да го биеш с пръчка, то няма да умре.
\par 14 Ти, като го биеш с пръчката, Ще избавиш душата му от ада.
\par 15 Сине мой, ако бъде сърцето ти мъдро, То и на моето сърце ще е драго.
\par 16 Да! сърцето ми ще се радва, Когато устните ти изговарят правото
\par 17 Сърцето ти да не завижда на грешните, Но да пребъдва в страх от Господа цял ден,
\par 18 Защото наистина има бъдеще, И надеждата ти няма да се отсече.
\par 19 Ти, сине мой, слушай и бъди мъдър, И оправяй сърцето си в пътя,
\par 20 Не бъди между винопийци, Между невъздържани месоядци,
\par 21 Защото пияницата и чревоугодникът ще осиромашеят, И дремливостта ще облече човек в дрипи.
\par 22 Слушай баща си, който те е родил, И не презирай майка си, когато остарее.
\par 23 Купувай истината и не я продавай, Тоже и мъдростта, поуката и разума.
\par 24 Бащата на праведния ще се радва много, И който ражда мъдро чадо ще има радост от него.
\par 25 Прочее , нека се веселят твоят баща и твоята майка, И да се възхищава оная, която те е родила.
\par 26 Сине мой, дай сърцето си на мене, И очите ти нека внимават в моите пътища,
\par 27 Защото блудницата е дълбока яма, И чуждата жена е тесен ров.
\par 28 Да! тя причаква като зла плячка, И умножава числото на неверните между човеците.
\par 29 Кому горко? кому скръб? кому каране? Кому оплакване? кому удари без причина? Кому подпухнали очи? -
\par 30 На ония, които се бавят около виното, Които отиват да вкусят подправено вино.
\par 31 Не гледай виното, че е червено, Че показва цвета си в чашата, Че се поглъща гладко,
\par 32 Защото после то хапе като змия, И жили като ехидна.
\par 33 Очите ти ще гледат чужди жени, И сърцето ти ще изригва развратни неща;
\par 34 Даже ще бъдеш като един, който би легнал всред море, Или като един, който би лежал на върха на мачта.
\par 35 Удариха ме ще речеш , и не ме заболя; Биха ме, и не усетих. Кога ще се събудя за да го търся пак?

\chapter{24}

\par 1 Не завиждай на злите хора, Нито пожелавай да си с тях,
\par 2 Защото сърцето им размишлява насилие, И устните им говорят за пакост
\par 3 С мъдрост се гради къща, И с разум се утвърждава,
\par 4 И чрез знание стаите се напълват С всякакви скъпоценни и приятни богатства.
\par 5 Мъдрият човек е силен, И човек със знание се укрепява в сила,
\par 6 Защото с мъдър съвет ще водиш войната си, И чрез множеството съветници бива избавление.
\par 7 Мъдростта е непостижима за безумния, Той не отваря устата си в портата.
\par 8 Който намисля да прави зло, Ще се нарече пакостен човек;
\par 9 Помислянето на такова безумие е грях, И присмивателят е мерзост на човеците.
\par 10 Ако покажеш малодушие в усилно време, Силата ти е малка.
\par 11 Избавяй ония, които се влачат на смърт, И гледай да задържиш ония, които политат към клане.
\par 12 Ако речеш: Ето, ние не знаехме това! То Оня, Който претегля сърцата, не разбира ли? Оня, Който пази душата ти, на знае ли, И не ще ли въздаде на всеки според делата му?
\par 13 Сине мой, яж мед, защото е добър, И медена пита, защото е сладка на вкуса ти.
\par 14 И ще знаеш, че такава е мъдростта за душата ти, Ако си я намерил; и има бъдеще, И надеждата ти няма да се отсече.
\par 15 Не поставяй засада, о нечестиви човече, против жилището на праведния, Не разваляй мястото му за почивка.
\par 16 Защото праведният ако седем пъти пада, пак става, Докато нечестивите се препъват в злото.
\par 17 Не се радвай когато падне неприятелят ти, И да се не весели сърцето ти, когато се подхлъзне той.
\par 18 Да не би да съгледа Господ, и това да Му се види зло, И Той да оттегли гнева Си от него.
\par 19 Не се раздразвай, поради злодейците, Нито завиждай на нечестивите,
\par 20 Защото злите не ще имат бъдеще; Светилникът на нечестивите ще изгасне.
\par 21 Сине мой, бой се от Господа и от царя, И не се сношавай с непостоянните,
\par 22 Защото бедствие ще се издигне против тях внезапно, И кой знае какво наказание ще им се наложи и от двамата?
\par 23 И тия са изречения на мъдрите: - Лицеприятие в съд не е добро.
\par 24 Който казва на нечестивия: Праведен си, Него народи ще кълнат, него племена ще мразят;
\par 25 Но който го изобличават, към тях ще се показва благоволение, И върху тях ще дойде добро благословение.
\par 26 Който дава прав отговор, Той целува в устни.
\par 27 Нареди си работата навън, И приготви си я на нивата, И после съгради къщата си.
\par 28 Не бивай свидетел против ближния си без причина, Нито мами с устните си.
\par 29 Не казвай: Както ми направи той, така ще му направя и аз, Ще въздам на човека според делата му.
\par 30 Минах край нивата на ленивия И край лозето на нехайния човек,
\par 31 И всичко бе обрасло с тръни, Коприва беше покрила повърхността му, И каменната му ограда беше съборена
\par 32 Тогава, като прегледах, размислих в сърцето си, Видях и взех поука.
\par 33 Още малко спане, малко дрямка, Малко сгъване на ръце за сън, -
\par 34 И сиромашията ще дойде върху тебе, като крадец И немотията - като въоръжен мъж,

\chapter{25}

\par 1 И тия са Соломонови притчи, които събраха човеците на Юдовия цар Езекия.
\par 2 Слава за Бога е да скрива всяко нещо, А слава е на царете да издирват работите.
\par 3 Височината на небето и дълбочината на земята И сърцата на царете са неизследими.
\par 4 Отмахни нечистото от среброто, И ще излезе съд за златаря.
\par 5 Отмахни нечестивите от царя, И престолът му ще се утвъди в правда.
\par 6 Не се надигай пред царя, И не стой на мястото на големците,
\par 7 Защото по-добре е да ти кажат: Мини тук по-горе, Нежели да те турят по-долу в присъствието на началника, когото са видели очите ти.
\par 8 Не бързай да излезеш, за да се караш. Да не би най-сетне да не знаеш що да правиш; Когато те засрами противникът ти.
\par 9 Разисквай делото си с противника си сам . Но не откривай чужди тайни,
\par 10 Да не би да те укори оня, който те слуша, И твоето безчестие да остане незаличимо.
\par 11 Дума казана на място е Като златни ябълки в сребърни съдове.
\par 12 Както е обица и украшение от чисто злато за човек , Така е мъдрият изобличител за внимателното ухо.
\par 13 Както е снежната прохлада в жетвено време, Така е верният посланик на тия, които го изпращат, Защото освежава душата на господаря си.
\par 14 Който лъжливо се хвали за подаръци що дава, Прилича на облаци и вятър без дъжд.
\par 15 Чрез въздържаност се склонява управител, И мек език троши кости.
\par 16 Намерил ли си мед? Яж само колкото ти е нужно, Да не би да се преситиш от него и да го повърнеш.
\par 17 Рядко туряй ногата си в къщата на съседа си, Да не би да му досадиш и той да те намрази.
\par 18 Човек, който лъжесвидетелствува против ближния си, Е като чук, нож и остра стрела.
\par 19 Доверие към неверен човек, в усилно време, Е като счупен зъб и изкълчена нога.
\par 20 Както един, който съблича дрехата си в студено време, И както оцет на сода, Така и оня, който пее песни на оскърбено сърце.
\par 21 Ако е гладен ненавистникът ти, дай му хляб да яде, И ако е жаден, напой го с вода,
\par 22 Защото така ще натрупаш жар на главата му И Господ ще те възнагради.
\par 23 Както северният вятър произвежда дъжд, Така и тайно одумващият език - разгневено лице.
\par 24 По-добре е да живее някой в ъгъл на покрива, Нежели в широка къща със свадлива жена.
\par 25 Както е студената вода за жадна душа, Така е добра вест от далечна земя.
\par 26 Праведният, който отстъпва пред нечестивия, Е като мътен извор и развален източник.
\par 27 Не е добре да яде някой много мед. Така също не е славно да търсят хората своята си слава.
\par 28 Който не владее духа си Е като съборен град без стени.

\chapter{26}

\par 1 Както сняг лятно време, И както дъжда в жетва, Така и чест не прилича на безумния.
\par 2 Както врабче в скитането си, както ластовица в летенето си, Така и проклетия не постига без причина.
\par 3 Бич за коня, юзда за осела И тояга за гърба на безумните.
\par 4 Не отговаряй на безумния според безумието му, Да не би да станеш и ти подобен нему.
\par 5 Отговаряй на безумния според безумието му, Да не би да се има мъдър в своите си очи.
\par 6 Който праща известие чрез безумния Отсича своите си нозе и докарва на себе си вреда.
\par 7 Както безполезни висят краката на куция, Така е притча в устата на безумния.
\par 8 Както оня, който хвърли възел със скъпоценни камъни в грамада, Така е тоя, който отдава чест на безумния.
\par 9 Като трън, който боде ръката на пияницата, Така е притча в устата на безумните.
\par 10 Както стрелец, който безогледно наранява всички, Така е оня, който условя безумен, или оня, който условя скитници.
\par 11 Както кучето се връща в бълвоча си, Така безумният повтаря своята глупост.
\par 12 Видял ли си човек който има себе си за мъдър? Повече надежда има за безумния, нежели за него.
\par 13 Ленивият казва: Лъв има на пътя! Лъв има по улиците!
\par 14 Както вратата се завърта на резетата си, Така и ленивият на постелката си.
\par 15 Ленивият потопява ръката си в паницата, А го мързи да я върне в устата си.
\par 16 Ленивият има себе си за по-мъдър От седмина души, които могат да дадат умен отговор.
\par 17 Минувачът, който се дразни с чужда разпра, Е като оня, който хваща куче за ушите.
\par 18 Както лудият, който хвърля главни, стрели и смърт.
\par 19 Така е човекът, който измамя ближния си, И казва: Не сторих ли това на шега?
\par 20 Дето няма дърва огънят изгасва; И дето няма шепотник раздорът престава.
\par 21 Както са въглищата за жарта и дърва за огъня, Така е и крамолникът, за да разпаля препирня.
\par 22 Думите на шепотника са като сладки залъци И влизат вътре в корема.
\par 23 Усърдните устни с нечестиво сърце Са като сребърна глеч намазана на пръстен съд.
\par 24 Ненавистникът лицемерствува с устните си, Но крои коварство в сърцето си;
\par 25 Когато говори сладко не го вярвай, Защото има седем мерзости в сърцето му;
\par 26 Макар омразата му да се покрива с измама, Нечестието му ще се издаде всред събранието.
\par 27 Който копае ров ще падне в него, И който търкаля камък, върху него ще се обърне.
\par 28 Лъжливият език мрази наранените от него, И ласкателните уста докарват съсипня.

\chapter{27}

\par 1 Недей се хвали с утрешния ден, Защото не знаеш какво ще роди денят.
\par 2 Нека те хвали друг, а не твоите уста, - Чужд, а не твоите устни.
\par 3 Камъкът е тежък и пясъкът много тегли; Но досадата на безумния е по-тежка и от двете.
\par 4 Яростта е жестока и гневът е като наводнение, Но кой може да устои пред завистта?
\par 5 Явното изобличение е по-добро От оная любов, която не се проявява.
\par 6 Удари от приятел са искрени, А целувки от неприятел - изобилни.
\par 7 Наситената душа се отвръща и от медена пита, А на гладната душа всичко горчиво е сладко.
\par 8 Както птица, която е напуснала гнездото си, Така е човек, който е напуснал мястото си.
\par 9 Както благоуханните места и каденията веселят сърцето, Така - и сладостта на сърдечния съвет на приятел.
\par 10 Не оставяй своя приятел нито приятеля на баща си. И не влизай в къщата на брата си, в деня на злощастието си. По-добре близък съсед, отколкото далечен брат.
\par 11 Сине мой, бъди мъдър и радвай сърцето ми, За да имам що да отговарям на онзи, който ме укорява.
\par 12 Благоразумният предвижда злото и се укрива, А неразумните вървят напред - и страдат.
\par 13 Вземи дрехата на този, който поръчителствува за чужд; Да! Вземи залог от онзи, който поръчителствува за чужда жена.
\par 14 Който става рано и благославя ближния си с висок глас, Ще се счете, като че го кълне.
\par 15 Непрестанно капене в дъждовен ден И жена крамолница са еднакви;
\par 16 Който би я обуздал, обуздал би вятъра И би хвърлил дървено масло с десницата си.
\par 17 Желязо остри желязо; Така човек остри лицето си срещу приятеля си.
\par 18 Който пази смоковницата ще яде плода й, И който се грижи за господаря си ще бъде почитан.
\par 19 Както водата отразява лице срещу лице, Така сърцето - човек срещу човека.
\par 20 Адът и погибелта не се насищат; Така и човешките очи не се насищат.
\par 21 Горнилото е за пречистване среброто и пещта за златото. А човекът се изпитва чрез онова, с което се хвали.
\par 22 Ако и с черясло сгрухаш безумния в кутел между грухано жито, Пак безумието му няма да се отдели от него.
\par 23 Внимавай да познаваш състоянието на стадата си, И грижи се за добитъка си;
\par 24 Защото богатството не е вечно, И короната не трае из род в род.
\par 25 Сеното се прибира, зеленината се явява, И планинските билки се събират.
\par 26 Агнетата ти служат за облекло, И козлите за купуване на нива.
\par 27 Ще има достатъчно козе мляко за храна На теб, на дома ти и за живеене на слугините ти.

\chapter{28}

\par 1 Нечестивите бягат без да ги гони някой, А праведните се смели като млад лъв.
\par 2 От бунтовете на страната началниците й биват мнозина, Но чрез умни и вещи човеци един неин управител продължава дълго време.
\par 3 Беден човек, който насилва немотните, Е като пороен дъжд, който не оставя храна.
\par 4 Които отстъпват от закона хвалят нечестивите, Но които пазят закона противят се на тях.
\par 5 Злите човеци не разбират правосъдие, Но тия, които търсят Господа разбират всичко.
\par 6 По-добър е сиромахът, който ходи в непорочността си, Нежели оня, който е опак между два пътя, макар и да е богат.
\par 7 Който пази закона е разумен син, А който дружи с чревоугодниците засрамва баща си.
\par 8 Който умножава имота си с лихварство и грабителство Събира го за този, който показва милост към сиромасите.
\par 9 Който отклонява ухото си от слушане закона, На такъв самата му молитва е мерзост.
\par 10 Който заблуждава праведните в лош път, Той сам ще падне в своята яма, А непорочните ще наследят добрини.
\par 11 Богатият човек мисли себе си за мъдър! Но разумният сиромах го изучава.
\par 12 Когато тържествуват първенците има голяма слава, А когато се издигнат нечестивите човек се крие.
\par 13 Който крие престъпленията си няма да успее, А който ги изповяда и оставя ще намери милост.
\par 14 Блажен оня човек, който се бои винаги, А който закоравява сърцето си ще падне в бедствие.
\par 15 Като ревящ лъв и гладна мечка Е нечестив управител над беден народ.
\par 16 О княже, лишен от разум, но велик да насилствуваш, Знай , че който мрази грабителство ще продължи дните си.
\par 17 Човек, който е товарен с кръвта на друг човек, Ще побърза да отиде в ямата; никой да го не спира.
\par 18 Който ходи непорочно, ще се избави, А който ходи опако между два пътя изведнъж ще падне.
\par 19 Който обработва земята си ще се насити с хляб, А който следва суетни неща ще го постигне сиромашия.
\par 20 Верният човек ще има много благословения; Но който бърза да се обогати не ще остане ненаказан.
\par 21 Не е добре да бъде човек лицеприятен, Защото за един залък хляб такъв човек ще извърши престъпление.
\par 22 Който има лошо око, бърза да се обогати. А не знае, че немотия ще го постигне.
\par 23 Който изобличава човека, той после ще намери по-голямо благоволение, Отколкото оня, който ласкае с езика си.
\par 24 Който краде от баща си или от майка си и казва: Не е грях, Той е другар на разрушителя.
\par 25 Човек с надменна душа подига крамоли, А който уповава на Господа ще затлъстее.
\par 26 Който уповава на своето си сърце е безумен, А който ходи разумно, той ще се избави.
\par 27 Който дава на сиромасите няма да изпадне в немотия, А който покрива очите си от тях ще има много клетви.
\par 28 Когато се възвишат нечестивите, хората се крият, Но когато те загиват, праведните се умножават.

\chapter{29}

\par 1 Човек, който често е изобличаван, закоравява врата си. Внезапно ще се съкруши и то без поправление.
\par 2 Когато праведните са на власт, людете се радват; Но когато нечестивият началствува, людете въздишат.
\par 3 Който обича мъдростта, радва баща си, Но който дружи с блудници, разпилява имота му .
\par 4 Чрез правосъдие царят утвърждава земята си . А който придобива подаръци я съсипва.
\par 5 Човек, който ласкае ближния си, Простира мрежа пред стъпките му.
\par 6 В беззаконието на лош човек има примка. А праведният пее и се радва.
\par 7 Праведният внимава в съдбата на бедните; Нечестивият няма даже разум, за да я узнае.
\par 8 Присмивателите запалят града, Но мъдрите усмиряват гнева.
\par 9 Ако мъдър човек има спор с безумен, Тоя се разярява, смее се и няма спокойствие.
\par 10 Кръвопийци мъже мразят непорочния, Но праведните се грижат за живота му.
\par 11 Безумният изригва целия си гняв, А мъдрият го задържа и укротява.
\par 12 Ако слуша управителят лъжливи думи, То всичките му слуги стават нечестиви.
\par 13 Сиромах и притеснител се срещат: Господ просвещава очите на всички тях.
\par 14 Когато цар съди вярно сиромасите, Престолът му ще бъде утвърден за винаги.
\par 15 Тоягата и изобличението дават мъдрост, А пренебрегнатото дете засрамва майка си.
\par 16 Когато нечестивите са на власт, беззаконието се умножава, Но праведните ще видят падането им.
\par 17 Наказвай сина си, и той ще те успокои, Да! ще даде наследство на душата ти.
\par 18 Дето няма пророческо видение людете се разюздават, А който пази закона е блажен.
\par 19 Слугата не се поправя с думи, Защото, при все че разбира, не обръща внимание.
\par 20 Видял ли си човек прибързан в работите си? Има повече надежда за безумния, отколкото за него.
\par 21 Ако глези някой слугата си от детинство, Най-после той ще му стане като син.
\par 22 Гневлив човек възбужда препирни, И сприхав човек беззаконствува много.
\par 23 Гордостта на човека ще го смири, А смиреният ще придобие чест.
\par 24 Който е съдружник на крадец мрази своята си душа; Той слуша заклеването, а не обажда.
\par 25 Страхът от човека туря примка, А който уповава на Господа ще бъде поставен на високо.
\par 26 Мнозина търсят благословението на управителя, Но съдбата на човека е от Господа.
\par 27 Несправедлив човек е мерзост за праведните; И който ходи в прав път е мерзост за нечестивите.

\chapter{30}

\par 1 Думите на Якововия син Агур, Маасовия цар, Който той изговори: - Изморих се, о Боже, изморих се, о Боже, Изнурих се;
\par 2 Защото съм по-скотски от кой да бил човек, И нямам човешки разум;
\par 3 Понеже не научих мъдрост, Нито имам знание за Всесветия.
\par 4 Кой е възлязъл на небето и слязъл? Кой е събрал вятър в шепите си? Кой е вързал водите в дрехата си? Кой е утвърдил всичките земни краища? Как е името му, и как е името на сина му? Кажи , ако го знаеш!
\par 5 Всяко слово божие е опитано: Той е щит на тия, които уповават на Него.
\par 6 Не притуряй на Неговите думи, Да не би да те изобличи и се окажеш лъжец.
\par 7 Две неща прося от Тебе: Не ми ги отказвай преди да умра: -
\par 8 Отдалечи от мене измамата и лъжата: Не ми давай ни сиромашия, ни богатство; Храни ме с хляба, който ми се пада;
\par 9 Да не би да се преситя и се отрека от Тебе и да кажа: Кой е Господ? Или да не би да осиромашея та да открадна, И да употребя скверно името на моя Бог.
\par 10 Не одумвай слуга пред господаря му, Да не би да те закълне и ти да се намериш виновен.
\par 11 Има поколение, което кълне баща си, И не благославя майка си.
\par 12 Има поколение, което е чисто пред своите си очи, Обаче не е измито от нечистотата си.
\par 13 Има поколение - колко високо са очите им И колко са надигнати клепачите им!
\par 14 Има поколение, чиито зъби са мечове, и челюстите му зъби ножове, За да изпояжда от земята сиромасите, И немотните сред човеците.
\par 15 Пиявицата има две дъщери, които викат : Дай, дай! Три неща има, които са ненаситни, - Дори четири, които не казват: Стига! -
\par 16 Адът и неплодната утроба, Земята, който не се насища с вода, И огънят, който не казва: Стига;
\par 17 Окото, което се присмива на баща си, И презира покорността към майка си, Гарвани от долината ще го изкълват, И орлови пилци ще го изядат.
\par 18 Три неща има, които са непостижими за мене, - Дори четири, които не разбирам:
\par 19 Следите на орел по въздуха, Следите на змия върху канара, Следите на кораб всред морето, И следите на мъж при девица.
\par 20 Такъв е пътят на жена прелюбодейка - Яде, бърше си устата и казва: Не съм извършила беззаконие.
\par 21 Поради три неща се тресе земята, Да! поради четири тя не може да търпи:
\par 22 Поради слуга, когато стане цар, И безумен, когато се насити с хляб;
\par 23 Поради омразна жена, когато се омъжи И слугиня, която измества господарката си.
\par 24 Четири неща има на земята, които са малки, Но са извънредно мъдри:
\par 25 Мравките, които не са силни люде, Но лете приготвят храната си;
\par 26 Кролиците, които са слаби люде Но поставят жилищата си на канара;
\par 27 Скакалците, които нямат цар, Но излизат всички по дружини;
\par 28 И гущерът, който можеш да хванеш в ръка, Но пак се намира в царските палати.
\par 29 Три неща има, които вървят величаво, Дори четири, които ходят благородно:
\par 30 Лъвът, който е най-силен от животните, И не се връща надире пред никого;
\par 31 Стегнатият през корема кон ; козелът; И царят, против когото не може да се въстава.
\par 32 Ако си постъпил безумно, като си се надигнал, Или ако си намислил зло, тури ръка на устата си.
\par 33 Защото както , като се бие мляко, изважда се масло, И като се блъска нос, изважда се кръв, Така и, като се подбужда гняв, изкарва се крамола.

\chapter{31}

\par 1 Думите на Маасовия цар Лемуил, Които го поучи майка му: -
\par 2 Що, сине мой? и що, сине на утробата ми? И що, сине на моите обреци?
\par 3 Не давай силата си на жените, Нито пътищата си на тези, които погубват царете.
\par 4 Не е за царете, Лемуиле, не е за царете да пият вино, Нито за князете да кажат : Где е спиртното питие?
\par 5 Да не би, като се напият, да забравят закона И да онеправдаят угнетяваните.
\par 6 Давайте спиртно питие на оня, който загива И вино на огорчения духом.
\par 7 За да пие и да забрави сиромашията си, И да не помни вече окаяността си.
\par 8 Отваряй устата си за безгласния, За делото на всички, които загиват;
\par 9 Отваряй устата си, съди справедливо. И раздавай правосъдие на сиромаха и немотния.
\par 10 Кой може да намери добродетелна жена? Защото тя е много по-ценна от скъпоценни камъни.
\par 11 Сърцето на мъжа й уповава на нея; И не ще му липсва печалба.
\par 12 Тя ще му донася добро, а не зло, През всичките дни на живота си.
\par 13 Търси вълна и лен, И работи с ръцете си това що й е угодно.
\par 14 Тя е като търговските кораби, - Донася храната си от далеч.
\par 15 При това, става докле е още нощ, И дава храна на дома си, И определената работа на слугините си.
\par 16 Разглежда нива, и я купува; От плода на ръцете си сади лозе.
\par 17 Опасва кръста си със сила И уякчава мишците си.
\par 18 Като схваща, че търгуването и е полезно. Светилникът й не угасва през нощта.
\par 19 Туря ръцете си на вретеното, И държи в ръката си хурката.
\par 20 Отваря ръката си на сиромасите, Да! простира ръцете си към немотните.
\par 21 Не се бои от снега за дома си; Защото всичките й домашни са облечени с двойни дрехи.
\par 22 Прави си завивки от дамаска; Облеклото и е висон и морав плат.
\par 23 Мъжът и е познат в портите, Когато седи между местните старейшини.
\par 24 Тя тъче ленено платно и го продава, И доставя пояси на търговците;
\par 25 Сила и достолепие са облеклото й; И тя гледа весело към бъдещето.
\par 26 Отваря устата си с мъдрост, И законът на езика й е благ.
\par 27 Добре внимава в управлението на дома си, И хляб на леност не яде.
\par 28 Чадата й стават и я ублажават; И мъжът й я хвали казвайки :
\par 29 Много дъщери са се държали достойно, Но ти надмина всичките.
\par 30 Прелестта е измамлива и красотата е лъх; Но жена, която се бои от Господа, тя ще бъде похвалена.
\par 31 Дайте й от плода на ръцете й, И делата й нека я хвалят в портите.

\end{document}