\begin{document}

\title{Еклесиаст}


\chapter{1}

\par 1 Думите на проповедника, Давидовият син, цар в Ерусалим: -
\par 2 Суета на суетите, казва проповедникът; Суета на суетите, всичко е суета.
\par 3 Каква полза на човека от всичкия му труд, В който се труди под слънцето?
\par 4 Едно поколение преминава, и друго поколение дохожда; А земята вечно стои.
\par 5 Също и слънцето изгрява, и слънцето захожда, И бърза да отива към мястото, дето трябва да изгрява.
\par 6 Вятърът отива към юг, И се връща към север; Вятърът непрестанно обикаля в отиването си, И пак се връща в своите кръгообращения.
\par 7 Всичките реки се вливат в морето, И пак морето не се напълня; На мястото, дето отиват реките, Там те непрестанно (Еврейски: се връщат да.) отиват.
\par 8 Всичките неща са досадни, - Човек не може да изкаже до колко; Окото не се насища с гледане, нито се напълня ухото със слушане.
\par 9 Каквото е станало, това е, което ще стане; И каквото е било извършено, това е, което ще се извърши; И няма нищо ново под слънцето.
\par 10 Има ли нещо, за което може да се каже: Виж! това е ново? То е вече станало във вековете, които са били преди нас.
\par 11 Не се помнят предишните поколения; Нито ще се помнят послешните, грядущите, поколения Между ония, които ще идат подир.
\par 12 Аз проповедникът бях цар Над Израиля в Ерусалим;
\par 13 И предадох сърцето си да издиря И да изпитам чрез мъдростта Относно всичко, що става под небето. Тежък е тоя труд, който Бог е дал на човешките чада, За да се трудят в него.
\par 14 Видях всичките дела, що се вършат под слънцето; И, ето, всичко е суета и гонене на вятър.
\par 15 Кривото не може да се изправи; И това, което е недоизпълнено, не може да се брои.
\par 16 Аз се съвещавах със сърцето си и рекох: Ето, станах велик, и съм умножавал мъдростта си Повече от всички, които са били преди мене в Ерусалим; Да! сърцето ми е имало голяма опитност в мъдрост и знание.
\par 17 И предадох сърцето си, за да позная мъдростта, И да позная лудостта и безумието. Познах, че и това е гонене на вятър.
\par 18 Защото в многото мъдрост има много досада; И който увеличава знание увеличава и печал.

\chapter{2}

\par 1 Аз рекох на сърцето си: Ела сега, да те опитам с веселба, Затова, се наслаждавай с благо; И, ето, и това беше суета.
\par 2 Рекох за смеха: Лудост е, - И за веселбата: Що ползва тя?
\par 3 Намислих в сърцето си да веселя плътта си с вино, Докато сърцето ми още се управляваше от мъдростта, И да усвоя безумието докле видя какво е добре да вършат човешките чада Под небето през всичките дни на живота си.
\par 4 Направих си големи работи; Съградих си къщи; насадих си лозя;
\par 5 Направих си градини и садове, И насадих в тях всякакви плодни дървета;
\par 6 Направих си водоеми, за да поя от тях Насадения с дървета лес;
\par 7 Придобих слуги и слугини, И имах слуги родени в дома ми; Имах още чреди и стада Повече от всички, които са били преди мене в Ерусалим;
\par 8 Събрах си и сребро, и злато, И особените скъпоценности на царете и на областите; Набавих си певци и певици, И насладите на човешките чада - Наложници твърде много.
\par 9 Така станах велик и уголемих се Повече от всички, които са били преди мене в Ерусалим; Още и мъдростта ми си остана в мене.
\par 10 И от всичко, което пожелаха очите ми, Нищо не им отрекох; Не спрях сърцето си от никаква веселба; Защото сърцето ми се радваше във всичките ми трудове, И това беше делът ми от всичкия ми труд.
\par 11 Тогава разгледах всичките дела, Които бяха извършили ръцете ми, И трудът, в който бях се трудил; И, ето, всичко беше суета и  гонене на вятър, И нямаше полза под слънцето.
\par 12 И обърнах се да разгледам Мъдростта и лудостта и безумието; Защото що може да стори човек, който е дошъл подир царя? Относно това, което е вече сторено?
\par 13 Тогава видях, че мъдростта превъзхожда безумието Както светлината превъзхожда тъмнината.
\par 14 На мъдрия очите са в главата му, А безумният ходи в тъмнина; Обаче аз познах още, че една участ Постига всички тях.
\par 15 Тогава рекох в сърцето си: Каквото постига безумния Това ще постигне и мене; Защо прочее бях аз по-мъдър? За туй рекох в сърцето си, Че и това е суета.
\par 16 Защото како на безумния, така и на мъдрия, Не остава вечно паметта му, Понеже в идните дни всичко ще е вече забравено; И как умира мъдрият? - Както и безумният.
\par 17 За това, намразих живота, Защото тежки ми се видяха делата, които стават под слънцето; Понеже всичко е суета и гонене на вятър.
\par 18 Намразих още и всичкия си труд, В който съм се трудил под слънцето, Защото трябва да го оставя на човека, който ще бъде подир мене;
\par 19 И кой знае мъдър ли ще бъде той или безумен? Но пак той ще властва над всичкия ми труд, в който съм се трудил, И в който показах мъдрост под слънцето. И това е суета.
\par 20 За това, аз наново направих сърцето си да се отчае Поради всичкия труд, в който съм се трудил под слънцето.
\par 21 Защото има човек, който се е  трудил С мъдрост, със знание, и със сполука; Но пак той ще остави всичко  за дял на едного, Който не е участвал в труда му. И това е суета и голямо зло.
\par 22 Защото каква полза на човека от всичкия му труд И от досадата на сърцето му, В който се изморява под слънцето?
\par 23 Понеже всичките му дни са само печал, И трудовете му скръб; И още и нощя сърцето му не си почива. И това е суета.
\par 24 Няма по-добро за човека освен да яде и да пие, И да прави душата си да се наслаждава от доброто на труда му. И аз видях, че и това е от Божията ръка.
\par 25 Защото кой може да яде, И кой може да се наслаждава, повече от мене?
\par 26 Понеже Бог дава на угодния нему човек Мъдрост и знание и радост; А на грешния дава да се труди, и да събира, и да трупа, - За да даде всичко на угодния Богу. И това е суета и гонене на вятър.

\chapter{3}

\par 1 Има време за всяко нещо, И срок за всяка работа под небето:
\par 2 Време за раждане, и време за умиране; Време за насаждане, и време за изкореняване насаденото;
\par 3 Време за убиване, и време за изцеляване; Време за събаряне, и време за градене;
\par 4 Време за плачене, и време за смеене; Време за жалеене, и време за ликуване;
\par 5 Време за разхвърляне камъни, и време за събиране камъни; Време за прегръщане, и време за въздържане от прегръщането;
\par 6 Време за търсене, и време за изгубване; Време за пазене, и време за хвърляне;
\par 7 Време за раздиране, и време за шиене; Време за мълчание, и време за говорене;
\par 8 Време за обичане, и време за мразене; Време за война, и време за мир.
\par 9 Каква полза на оногова, който работи, От онова, в което се труди той?
\par 10 Видях труда, който даде Бог На човешките чада, за да се трудят в него.
\par 11 Той е направил всяко нещо хубаво на времето му; Положил е и вечността в тяхното сърце, Без обаче да може човек да издири Отначало до край делото, което е направил Бог.
\par 12 Познах, че няма друго по-добро за тях Освен да се весели всеки, и да благоденства през живота си;
\par 13 И още всеки човек да яде и да пие, И да се наслаждава от доброто на всичкия си труд. Това е дар от Бога.
\par 14 Познах, че всичко що прави Бог ще бъде вечно; Не е възможно да се притури на него, нито да се отнеме от него; И Бог е направил това, за да Му се боят човеците.
\par 15 Каквото съществува е станало вече; И каквото ще стане е станало вече; И Бог издирва наново онова, което е било оттласнато.
\par 16 Видях още под слънцето Мястото на съда, а там беззаконието, - И мястото на правдата, а там неправдата.
\par 17 Рекох в сърцето си: Бог ще съди праведния и нечестивия; Защото има време у Него за всяко нещо и за всяко дело.
\par 18 Рекох в сърцето си относно човешките чада, Че това е, за да ги опита Бог, И за да видят те, че в себе си са като животни.
\par 19 Защото каквото постига човешките чада Постига и животните; една участ имат; Както умира единия, така умира и другото; Да! един дух имат всичките; И човек не превъзхожда в нищо животното, Защото всичко е суета.
\par 20 Всички отиват в едно място; Всички са от пръстта, и всички се връщат в пръстта.
\par 21 Кой знае духът на човешките чада, че възлиза горе, И духът на животното, че слиза долу на земята?
\par 22 Видях прочее, че за човека няма по-добро, Освен да се радва в делата си; Защото това е делът му; Понеже кой ще го възвърне надире, за да види Онова, което ще бъде подир него?

\chapter{4}

\par 1 Тогава, като изново размишлявах Всичките угнетения, които стават под слънцето, И видях сълзите на угнетяваните, че нямаше за тях утешител, И че силата беше в ръката на ония, които ги угнетяваха, А за тях нямаше утешител,
\par 2 За това аз ублажавах умрелите, които са вече умрели, Повече от живите, които са още живи;
\par 3 А по-щастлив и от двамата считах оня, който не е бил още, Който не е видял лошите дела, които стават под слънцето.
\par 4 Тогава видях всеки труд и всяко сполучливо дело, Че поради него човек бива завиждан от ближния си. И това е суета и гонене на вятър.
\par 5 Безумният сгъва ръцете си И яде своята си плът,
\par 6 И казва: По-добре една пълна шепа със спокойствие, Отколкото две пълни шепи с труд и с гонене на вятър.
\par 7 Тогава изново видях само суета под слънцето.
\par 8 Има такъв, който е самичък, който няма другар, Да! Няма нито син, нито брат; Но пак няма край на многото му труд, Нито се насища окото му с богатство; И той не дума: За кого, прочее, се трудя аз И лишавам душата си от благо? И това е суета и тежък труд.
\par 9 По-добре са двама, отколкото един, Понеже те имат добра награда за труда си;
\par 10 Защото, ако паднат, единият ще вдигне другаря си; Но горко на оня, който е сам, когато падне, И няма друг да го вдигне.
\par 11 И ако легнат двама заедно ще се стоплят; А един как ще се стопли самичък?
\par 12 И ако някой надвие на един, който е самичък, Двама ще му се опрат; И тройното въже не се къса скоро.
\par 13 По-добър е беден и мъдър младеж, Отколкото стар и безумен цар, Който не знае вече да приема съвет;
\par 14 Защото единият излиза из тъмницата (Еврейски: къщата на веригите), за да царува, А другият, и цар да се е родил, става сиромах.
\par 15 Видях всичките живи, които ходят под слънцето, Че бяха с младежа, втория, който стана вместо него;
\par 16 Нямаше край на всичките люде, На всичките, над които е бил той; А идещите подир него не ще се зарадват в него. Наистина и това е  суета и гонене на вятър.

\chapter{5}

\par 1 Пази ногата си, когато отиваш в Божия дом, Защото да се приближиш да слушаш е по-добро, Отколкото да принесеш жертва на безумните, Които не знаят, че струват зло.
\par 2 Не прибързвай с устата си, Нито да бърза сърцето ти да произнася дума пред Бога; Защото Бог е на небесата, а ти на земята, За това, нека бъдат думите ти малко;
\par 3 Защото както съновидението произхожда от многото занимание, Така и гласът на безумния от многото думи.
\par 4 Когато направиш обрек Богу, Не се бави да го изпълниш, Защото той няма благоволение в безумните; Изпълни това, което си обрекъл.
\par 5 По-добре да се не обричаш, Отколкото да се обречеш и  да не изпълниш.
\par 6 Не позволявай на устата си да вкарат в грях плътта ти; И не казвай пред Божия служител (Еврейски: пред ангела), че е било по небрежение; Защо да се разгневи Бог на гласа ти, И да погуби делото на ръцете ти?
\par 7 Защото, макар да изобилват сънища и суети и много думи, Ти се бой от Бога.
\par 8 Ако видиш, че сиромахът се угнетява, И че правосъдието и правдата в държавата се изнасилват, Да се не почудиш на това нещо; Защото над високия надзирава по-висок, И над тях има по-високи.
\par 9 При това, ползата от земята е за всичките, И сам царят служи на нивите.
\par 10 Който обича среброто не ще се насити от сребро, Нито с доходи оня, който обича изобилието. И това е суета.
\par 11 Когато се умножават благата, Умножават се и ония, които ги ядат; И каква полза има на  притежателите им, Освен да ги гледат с очите си?
\par 12 Сънят на работника е сладък, малко ли ял, или много; А пресищането на богатия не го оставя да спи.
\par 13 има тежко зло, което видях под слънцето, именно, Богатство пазено от притежателя му за негова му вреда;
\par 14 И онова богатство се изгубва чрез зъл случай, И не остава нищо в ръката на сина, когото е родил.
\par 15 Както е излязъл из утробата на майка си, Гол ще отиде пак както е дошъл, Без да вземе нещо от труда си, За да го занесе в ръката си.
\par 16 Още и това е тежко зло, Че, по всичко, както е дошъл, така ще и да отиде; И каква полза за него, че се е трудил за вятъра?
\par 17 Още и през всичките си дни яде в тъмнина, И има много досада и болест и негодуване.
\par 18 Ето какво видях аз за добро и прилично: Да яде някой и да пие, И да се наслаждава от благото на всичкия си труд, В който се труди под слънцето, През всичките дни на живота си, които му е дал Бог; Защото това е делът му.
\par 19 И на който човек е дал Бог богатство и имот, И му е дал и власт да яде от тях, И да взема дела си, и да се весели в труда си, - Това е дар от Бога.
\par 20 Защото няма много да помни дните на живота си, Понеже Бог му отговаря с веселието та сърцето му.

\chapter{6}

\par 1 Има зло, което видях под слънцето, И е тежко върху човеците:
\par 2 Човек, на когото Бог дава богатство и имот и почест, Така щото душата му не се лишава от нищо що би пожелал, На когото обаче Бог не дава власт да яде от тях. Но чужденец ги яде. Това е суета и лоша болест.
\par 3 Ако роди човек сто чада. И живее много години, Така щото дните на годините му да станат много, А душата му не се насити с благо, И още той не приема прилично  погребение, - Казвам, че пометничето е по-щастливо от него;
\par 4 Защото това е дошло в нищожество, и отива в тъмнина; И името му се покрива с тъмнина;
\par 5 При туй, то не е видяло слънцето, и не е познало нищо, - По-добре е на това, отколкото на оногоз.
\par 6 Дори два пъти по хиляда години, ако би живял някой, и не види добро, - Не отиват ли те всички в едно място?
\par 7 Всичкият труд на човека е за устата му; Душата обаче не се насища.
\par 8 Защото какво предимство има мъдрият над безумния? Или какво предимство има сиромахът, който умее как да се обхожда пред живите?
\par 9 По-добре е да гледаш нещо с очите си, Отколкото да блуждаеш с желанието си. И това е суета и гонене на вятър.
\par 10 Всяко нещо, което е съществувало, вече си е  получило името; И известно е, че оня, чието име е Човек, (Или: Адам, т.е. Червеникав - направен от червената пръст) Не може да се състезава с по-силния от него.
\par 11 Понеже има много неща, които умножават суетата, То каква полза на човека?
\par 12 Защото кой знае кое е добро за човека В живота, във всичките дни на суетния му живот, Който минава като сянка? Понеже кой ще извести на човека Какво ще бъде подир него под слънцето?

\chapter{7}

\par 1 Добро  име струва повече от скъпоценно миро. И денят на смъртта повече  от деня на раждането.
\par 2 По-добре да отиде някой в дом на жалеене, Отколкото да отиде в дом на пирувание; Защото това е сетнината на всеки човек, И живият може да го вложи в сърцето си.
\par 3 По-полезна е печалта от смеха; Защото от натъжеността на лицето сърцето се развеселява.
\par 4 Сърцето на мъдрите е в дома на жалеене; А сърцето на безумните е в дома на веселие.
\par 5 По-добре е човек да слуша изобличение от мъдрия Нежели да слуша песен от безумните;
\par 6 Защото какъвто е шумът на търнете под котела, Такъв е смехът на безумния. И това е суета.
\par 7 Наистина изнудването прави мъдрия да избезумява; И подарък разтлява сърцето.
\par 8 По-предпочително е свършването на работата, нежели започването й; По-добър е дълготърпеливият нежели високоумният.
\par 9 Не бързай да се досадиш в духа си; Защото досадата почива в гърдите на безумните.
\par 10 Да не речеш: Коя е причината Дето предишните дни бяха по-добри от сегашните? Защото не питаш разумно за това.
\par 11 Мъдростта е равноценна с едно наследство, Даже и по-ценна е на ония, които гледат слънцето;
\par 12 Защото, не само че мъдростта е защита, както и парите са защита, Но предимството на знанието е, че мъдростта запазва живота на ония, които я имат.
\par 13 Разгледай делото Божие; Защото кой може да изправи онова, което Той е направил криво?
\par 14 Във време на благоденствие бъди весел, А във време на злополука бъди разсъдлив; Защото Бог постави едното до другото, За да не може човек да открие нищо, което ще бъде подир него.
\par 15 Всичко това видях в суетните си дни: Има праведен, който загинва в правдата си, И има нечестив, който дългоденства в злотворството си.
\par 16 Не ставай прекалено праведен, и не мисли себе си чрезмерно мъдър; Защо да се погубиш?
\par 17 Не ставай прекалено зъл, и не бивай безумен; Защо да умреш преди времето си?
\par 18 Добре е да се придържаш за едното, И да не оттегляш ръката си от другото; Защото, който се бои от Бога, ще се отърве и от двете. (Еврейски: от всички тези)
\par 19 Мъдростта крепи мъдрия повече От десетина началника, които са в града.
\par 20 Наистина няма праведен човек на земята, Който да струва добро и да не греши.
\par 21 И не обръщай внимание на всичките думи, които се говорят, Да не би да чуеш слугата си да те кълне;
\par 22 Защото сърцето ти познава, че и ти подобно Си проклинал други много пъти.
\par 23 Всичко това опитах чрез мъдростта. Рекох: Ще бъда мъдър; но мъдростта се отдалечи от мене.
\par 24 Онова, което е, е много далеч и твърде дълбоко; Кой може да го намери?
\par 25 Аз изново се предадох от сърцето си Да науча, и да издиря, и да изследвам мъдростта и разума, И да позная, че нечестието е безумие, и че глупостта е лудост;
\par 26 И намирам, че е по-горчива от смърт Оная жена, чието сърце е примки и мрежи, и ръцете й окови; Който е добър пред Бога ще се отърве от нея. А грешникът ще бъде хванат от нея.
\par 27 Виж, това намерих, казва проповедникът. Като изпитвах нещата едно по едно, за да намеря причината;
\par 28 (И душата ми още го изследва, но не съм го намерил:) Един мъж между хиляда намерих; Но ни една жена между толкова (Еврейски: всички тези) жени не намерих.
\par 29 Ето, това само намерих, Че Бог направи човека праведен, Но те изнамериха много измишления.

\chapter{8}

\par 1 Кой е като мъдрия? И кой знае изяснението на нещата? Мъдростта на човека осветлява лицето му, И коравината на лицето му се променя.
\par 2 Аз те съветвам да пазиш царевата заповед, А най-вече заради клетвата пред Бога.
\par 3 Не бързай да излезеш от присъствието му; Не постоянствай в лоша работа; Защото върши всичко, каквото иска,
\par 4 Тъй като думата на царя има власт, И кой ще му рече: Що правиш?
\par 5 Който пази заповедта няма да види нещо зло; И сърцето на мъдрия познава, че има и време и съдба за непокорството.
\par 6 Понеже за всяко нещо има време и съдба; Защото окаянството на човека е голямо върху него,
\par 7 Понеже не знае какво има да стане; Защото кой може да му яви как ще бъде?
\par 8 Няма човек, който да има власт над духа, та да задържи духа, Нито да има власт над деня на смъртта; И в тая война няма уволнение, Нито ще избави нечестието ония, които са предадени на него.
\par 9 Всичко това видях, като занимах сърцето си С всяко дело, което става под слънцето, Че има време, когато човек властва над човека за негова повреда.
\par 10 При това, видях нечестивите погребани, Които бяха дохождали и отивали в святото място; И те бидоха забравени в града дето бяха така сторили. И това е суета.
\par 11 Понеже присъдата против  нечестиво дело не се изпълнява скоро, За това сърцето на човешките чада е всецяло вдадено да струва зло.
\par 12 Ако и грешникът да струва зло сто пъти и да дългоденства, Пак аз това зная, че ще бъде добре На ония, които се боят от Бога, които се боят пред Него;
\par 13 А на нечестивия не ще бъде добре, Нито ще се продължат дните му, които ще бъдат като сянка,  Защото той не се бои пред Бога.
\par 14 Има една суета, която става на земята: Че има праведни, на които се случва според делата на нечестивите, А пък има нечестиви, на които се случва според делата на праведните. Рекох, че и това е суета.
\par 15 За това, аз похвалих веселбата, Защото на човека няма по-добро под слънцето Освен да яде и да пие и да се весели, И това да му остава от труда му През всичките дни на живота му, които Бог му е дал под слънцето.
\par 16 Когато предадох сърцето си да позная мъдростта, И да видя труденето, което ставаше по земята, Как очите на някои не виждат сън ни деня ни нощя,
\par 17 Тогава видях всичкото Божие дело, Че човек не може да издири делото, което става под слънцето; Понеже колкото и да се труди човек да го търси, Пак няма да го намери; Па дори ако и мъдрият да рече да го познае, Не ще може да го намери;

\chapter{9}

\par 1 Защото всичко това вложих в сърцето си, Да издиря всичко това, Че праведните и мъдрите и делата им са в Божията ръка; Няма човек, който да знае Дали любов или омраза го очаква; Всичко е неизвестно пред тях.
\par 2 Всичко постига всичките еднакво; Една е участта на праведния и на нечестивия, На добрия и на нечестивия, на чистия и на нечистия, На оногоз, който жертва, и на оногоз, който не жертва; Както е добрият, така е и  грешният, И оня, който се кълне, както оня, който се бои да се кълне.
\par 3 Това е злото между всичко, което става под слънцето, Че една е участта на всичките, И най-вече, че сърцето на човешките чада е пълно със зло, И лудост е в сърцето им, докато са живи, И че после слизат при мъртвите.
\par 4 Защото за оногоз, който се съобщава с всичките живи, има надежда; Понеже живо куче струва повече от мъртъв лъв.
\par 5 Защото живите поне знаят, че ще умрат; Но мъртвите не знаят нищо, нито вече придобиват, Понеже споменът за тях е забравен;
\par 6 Още и любовта им, и омразата им, и завистта им, вече са изгубени, Нито ще имат вече някога дял в нещо, що става под слънцето.
\par 7 Иди, яж хляба си с радост, и пий виното си с весело сърце, Защото Бог вече има благоволение в делата ти.
\par 8 Дрехите ти нека бъдат винаги бели, И миро да не липсва от главата ти.
\par 9 Радвай се на живота с жената, която си възлюбил, През всичките дни на суетния си живот, Които ти са дадени под слънцето, - През всичките дни на твоята суета; Защото това ти е делът в живота И в труда ти, в който се трудиш под слънцето.
\par 10 Всичко що намери ръката ти да прави според силата ти, направи го; Защото няма ни работа, ни замисъл, ни знание, ни мъдрост в гроба (Или: Шеол), дето отиваш.
\par 11 Обърнах се, и видях под слънцето, Че надтичването не е на леките, нито боят на силните, Нито хлябът на мъдрите, нито богатството на разумните, Нито благоволението на изкусните; Но на всичките се случва според времето и случая.
\par 12 Защото и човек не знае времето си; Както рибите, които се улавят в жестока (Еврейски: зла) мрежа, И както птиците, които се улавят в примка, Така се улавят човешките чада в лошо време, Когато то внезапно ги връхлети.
\par 13 И това видях като задача (Еврейски: мъдрост) под слънцето, (И тя ми се видя голяма:)
\par 14 Имаше малък град, и малцина мъже в него; И дойде против него велик цар та го обсади, и издигна против него големи могили.
\par 15 Но в него се намери сиромах и мъдър човек, И той с мъдростта си избави града; Но никой не си спомни за оногоз сиромах човека.
\par 16 Тогава рекох: Мъдростта струва повече от силата; А при все това, мъдростта на сиромаха се презира, И думите му не се слушат.
\par 17 Думите на мъдрите тихо изговорени се слушат Повече  от вика на оногоз, който властва между безумните.
\par 18 Мъдростта струва повече от военните оръжия; А един грешник разваля много добри неща.

\chapter{10}

\par 1 Умрели мухи правят мирото на мировареца да вони и да кипи; Така и малко безумие покваря оногоз, който е уважаван за мъдрост и чест.
\par 2 Разумът (Еврейски: сърцето)  на мъдрия е в десницата му, А разумът (Еврейски: сърцето) на безумния в левицата му.
\par 3 Докато безумният още ходи в пътя, разумът (Еврейски: сърцето) не му достига. и той се прогласява на всичките, че е безумен.
\par 4 Ако гневът на управителя се повдигне против тебе, не напущай мястото си: Защото отстъпването отвраща големи грешки,
\par 5 Има зло, което видях под слънцето, - Погрешка като че ли произхождаща от владетеля, - и това е, че
\par 6 Безумният се поставя на висок чин, А богатите седят в долни места.
\par 7 Видях слуги на коне, И князе, ходещи като слуги по земята.
\par 8 Който копае яма, ще падне в нея; и който разбива ограда, него змия ще ухапе.
\par 9 Който кърти камъни ще се повреди от тях; И който цепи дърва се излага на опасност от тях;
\par 10 Ако се затъпи желязото, и не се наточи острието му, Тогава трябва да се напряга повече със силата; А мъдростта е полезна за упътване.
\par 11 Ако ухапе змията преди да бъде омаяна, Тогава няма полза от омайвача.(Или: Наистина, ако няма омайване, змията ще ухапе; и клеветникът не е по-добър)
\par 12 Думите из устата на мъдрия са благодатни; А устните на безумния ще погълнат самия него;
\par 13 Защото  първите думи, които изговаря, са безумие, И свършекът на говоренето му е пакостна лудост.
\par 14 Безумният тъй също умножава думи; Но пак човек не знае какво ще бъде; И кой може да му яви какво ще бъде подир него?
\par 15 Трудът на безумните ги уморява, Понеже ни един от тях не знае пътя за града.
\par 16 Горко ти, земьо, когато царят ти е дете, И началниците ти ядат рано!
\par 17 Блазе ти, земьо, когато царят ти е син на благородни, И началниците ти ядат на време, - за подкрепа, а не за опиване!
\par 18 От голяма леност засяда къщният покрив; И от безделието на ръцете прокапва къщата.
\par 19 Угощения се правят за веселба, и виното весели живота; А парите отговарят на всичко.
\par 20 Да не прокълнеш царя нито даже в мисълта си, И да не прокълнеш богатия нито в спалнята си; Защото въздушна птица ще отнесе гласа, И крилатото ще извести това нещо.

\chapter{11}

\par 1 Хвърли хляба си по водата, Защото след много дни ще го намериш.
\par 2 Дай дял на седмина, и дори на осмина; Защото не знаеш какво зло ще бъде на земята.
\par 3 Ако са пълни облаците, изливат дъжд на земята; И ако падне дърво към юг или към север, На мястото дето падне дървото, там ще си остане.
\par 4 Който се взира във вятъра няма да сее; И който гледа на облаците няма  да жъне.
\par 5 Както не знаеш как се движи (Еврейски: какъв е пътят на) духът, Нито как се образуват костите в утробата на непразната, Така не знаеш и делата на Бога, който прави всичко.
\par 6 Сей семето си заран, и вечер не въздържай ръката си; Защото не знаеш кое ще успее, това ли или онова, Или дали ще са и двете еднакво добри.
\par 7 Наистина светлината е сладка, И приятно е на очите да гледат слънцето;
\par 8 Да! ако и да живее човек много години, Нека се весели през всички тях; Но нека си спомня и за дните на тъмнината, защото ще бъдат много. Все що иде е суета.
\par 9 Весели се, младежо, в младостта си, И нека те радва сърцето ти в дните на младостта ти, И ходи по нравите (Еврейски: пътищата) на сърцето си, и според каквото гледат очите ти! Но знай, че за всичко това Бог ще те доведе на съд.
\par 10 За това отмахни от сърцето си досадата, И отдалечи от плътта си всичко що докарваневоля; Защото младостта и юношеството са суета. Глава 12

\chapter{12}

\par 1 И помни Създателя си в дните на младостта си, Преди да дойдат дните на злото, И стигнат годините, когато ще речеш: Нямам наслада от тях, -
\par 2 Преди да се помрачи слънцето и светлината, луната и звездите, И да се върнат облаците подир дъжда;
\par 3 Когато стражите на къщата ще треперят, И силните мъже ще се прегърбят, И ония, които мелят, ще престанат защото намаляха. И на тия, които гледат през прозорците, ще се стъмни;
\par 4 Когато вратите ще се затворят при пътя, Като ослабее гласа на мелницата; И при гласа на птицата ще стане човек, И всичките звукове (Еврейски: дъщери) на песента ще отслабнат;
\par 5 Още, когато ще се боят от всичко, що е високо, И ще треперят в пътя; Когато миндала се разцъфти, и скакалецът натегне, и всяка охота изчезне; Защото човек отива във вечния си дом, И жалеещите обикалят улиците, -
\par 6 Преди да се скъса сребърната верижка и се счупи златната чаша, Или се строши стомната при извора, Или се счупи колелото над кладенеца,
\par 7 И се върне пръстта в земята както е била, И духът се върне при Бога, който го е дал.
\par 8 Суета на суетите, казва проповедникът, Всичко е суета.
\par 9 И колкото по-мъдър ставаше проповедникът, Толкова повече поучаваше людете на знание; А най-вече измисляше и издирваше И нареждаше много притчи.
\par 10 Проповедникът се стараеше да намери угодни думи, И това, което бе с правота написано, думи на истина.
\par 11 Думите на мъдрите са като остни; и като заковани гвоздеи са думите на  събирачите на изреченията, Дадени от единия пастир.
\par 12 А колкото за нещо повече от това, сине мой, приеми увещание, Че правене много книги няма край, И много четене е труд на плътта.
\par 13 Нека чуем свършека на всичкото слово: Бой се от Бога и пази  заповедите му, Понеже това е всичкото на човека; (Или: това е длъжността на)
\par 14 Защото, относно всяко скрито нещо, Бог ще докара на съд всяко дело, Било то добро или зло.

\end{document}