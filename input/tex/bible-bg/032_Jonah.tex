\begin{document}

\title{Йон}


\chapter{1}

\par 1 Господнето слово дойде към Йона Аматиевия син и рече:
\par 2 Стани, иди в големия град Ниневия та викай против него; защото нечестието му възлезе пред мене.
\par 3 Но Йона стана да побягне в Тарсис от Господнето присъствие; и, като слезе в Иопия, намери кораб, който отиваше в Тарсис, плати за превоза си, и влезе в него, за да отиде с тях в Тарсис от Господнето присъствие.
\par 4 Но Господ повдигна силен вятър по морето, и стана голяма буря в морето, тъй че корабът бедстваше да се разбие.
\par 5 Тогава моряците, уплашени, извикаха всеки към своя бог, и хвърлиха в морето стоките, които бяха в кораба, за да олекне от тях; а Йона бе слязъл и легнал във вътрешностите на кораба, и спеше дълбоко.
\par 6 За това, корабоначалникът се приближи при него, та му рече: Защо така, ти който спиш? Стани, призови Бога си, негли Бог си спомни за нас, и не загинем.
\par 7 После си рекоха един на друг: Елате, да си хвърлим жребие, та да познаем по коя причина е това зло на нас. И като хвърлиха жребие, то падна на Йона.
\par 8 Тогава му рекоха: Кажи ни, молим ти се, по коя причина е това зло на нас. Каква ти е работата? И от де идеш? От коя си земя? И от кои си люде?
\par 9 А той им рече: Аз съм Евреин, и се боя от Господа небесния Бог, който направи морето и сушата.
\par 10 Тогава човеците се много уплашиха, и рекоха му: Защо си сторил това? (Защото човеците знаеха, че бягаше от Господнето присъствие, понеже беше им казал.)
\par 11 Тогава му рекоха: Що да ти сторим, за да утихне за нас морето? (Защото морето ставаше все по-бурно).
\par 12 И рече им: Вземете ме та ме хвърлете в морето, и морето ще утихне за вас; защото зная, че поради мене ви постигна тая голяма буря.
\par 13 Но пак, човеците гребяха силно, за да се върнат към сушата; но не можаха, защото морето ставаше все по-бурно против тях.
\par 14 За това, извикаха към Господа, казвайки: Молим ти се, Господи, молим ти се, да не загинем поради живота на тоя човек; и не налагай върху нас невинна кръв; защото ти, Господи, си сторил каквото си искал.
\par 15 И тъй, взеха Йона та го хвърлиха в морето; и яростта на морето престана.
\par 16 Тогава човеците се убояха твърде много от Господа; и принесоха жертва Господу, и направиха обреци.
\par 17 А Господ бе определил една голяма риба да погълне Йона; и Йона остана във вътрешността на рибата три деня и три нощи.

\chapter{2}

\par 1 Тогава Йона се помоли на Господа своя Бог из вътрешността на рибата, като каза: -
\par 2 В скръбта си извиках към Господа, и Той ме послуша; Из вътрешността на Шеол извиках, и Ти чу гласа ми.
\par 3 Защото Ти ме хвърли в дълбочините, в сърцето на морето, И потоци ме обиколиха; Всичките твои вълни и големи води преминаха над мене.
\par 4 И аз рекох: отхвърлен съм отпред очите Ти; Но пак, ще погледна наново към светия Твой храм.
\par 5 Водите ме обкръжиха дори до душа, Бездната ме обгърна, Морският бурен се обви около главата ми.
\par 6 Слязох до дъното на планините; Земните лостове ме затвориха завинаги; Но пак, Ти, Господи Боже мой, си избавил живота ми из рова.
\par 7 Като чезнеше в мене душата ми, спомних си за Господа, И молитвата ми влезе при Тебе в светия Ти храм.
\par 8 Ония, които уповават на лъжливите суети, Оставят милостта, спазвана за тях.
\par 9 Но аз ще принеса жертва с хвалебен глас; Ще отдам това, което съм обрекъл. Спасението е от Господа.
\par 10 И Господ заповяда на рибата; и тя избълва Йона на сушата.

\chapter{3}

\par 1 И Господнето слово дойде втори път към Йона и рече:
\par 2 Стани, иди в големия град Ниневия, и възгласи му проповедта, която ти казвам.
\par 3 И тъй, Йона стана и отиде в Ниневия според Господнето слово. А Ниневия бе твърде голям град, който изискваше три деня, за да се обходи.
\par 4 И Йона, като започна да върви през града един ден път, викаше и казваше: Още четиридесет деня, и Ниневия ще бъде съсипана.
\par 5 И Ниневийските жители повярваха Бога; и прогласиха пост и се облякоха с вретище, от най-големия между тях до най-нищожния;
\par 6 понеже вестта бе стигнала до Ниневийския цар, който, като стана от престола си, съблече одеждата си, покри се с вретище, и седна на пепел.
\par 7 И обяви и прогласи из Ниневия с указ от царя и от големците му, с тия думи - Човеците и животните, говедата и овците, да не вкусят нищо, нито да пасат, нито да пият вода;
\par 8 но човек и животно да се покрият с вретище; и нека викат силно към Бога, да! да се върнат всеки от лошия си път и от неправдата, която е в ръцете им.
\par 9 Кой знае дали Бог не ще се обърне и разкае, и се отвърне от лютия си гняв, та не погинем?
\par 10 И като видя Бог делата им, как те се обърнаха от лошия си път, Бог се разкая за злото, което бе рекъл да им направи, и го не направи.

\chapter{4}

\par 1 А това стана много мъчно на Йона, и възнегодува.
\par 2 И помоли се Господу, казвайки: О, Господи, не беше ли това каквото казах още когато бях в отечеството си? Това бе причината, по която предварих да бягам в Тарсис, дето знаех, че си Бог жалостив и милосерд, дълготърпелив и многомилостив, който се разкайваш за злото.
\par 3 За това, моля ти се, Господи, вземи още сега живота ми; защото ми е по-добре да умра, отколкото да живея.
\par 4 А Господ рече: Добре ли е да негодуваш?
\par 5 Тогава Йона излезе из града, и седна на източната страна от града, дето си направи колиба, под която седеше на сянка докле види какво ще стане с града.
\par 6 И Господ Бог определи да израсте една тиква, която се издигна над Йона, за да бъде сянка над главата му, та да го олечки от скръбта му. И Йона се зарадва твърде много за тиквата.
\par 7 А когато се зазори на утринта Бог определи един червей, който порази тиквата, и тя изсъхна.
\par 8 И щом изгрея слънцето Бог определи горещ източен вятър; и слънцето биеше върху главата на Йона, така щото премираше и поиска за себе си да умре, като казваше - По-добре ми е да умра, отколкото да живея.
\par 9 А Бог рече на Йона: Добре ли е да негодуваш за тиквата? И той каза: Добре е да негодувам, даже до смърт.
\par 10 Тогава рече Господ: Ти пожали тиквата, за която не си се трудил, нито си я направил да расте, която се роди в една нощ и в една нощ загина.
\par 11 А аз не трябваше ли да пожаля оня голям град Ниневия, в който има повече от сто и двадесет хиляди души, които не умеят да различават дясната си ръка от лявата си ръка, освен многото добитък?

\end{document}