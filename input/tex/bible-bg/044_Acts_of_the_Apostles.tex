\begin{document}

\title{Деяния}


\chapter{1}

\par 1 Първата повест, написах, о Теофиле, за всичко що Исус вършеше и учеше, откак почна
\par 2 до деня, когато се възнесе, след като даде чрез Светия Дух заповеди на апостолите, които беше избрал;
\par 3 на които и представи себе си жив след страданието си с много верни доказателства, като им се явяваше през четиридесет дена и им говореше за Божието царство.
\par 4 И като се събираше с тях, заръча им да не напускат Ерусалим, но да чакат обещаното от Отца, за което, каза той, чухте от мене.
\par 5 Защото Йоан е кръщавал с вода; а вие ще бъдете кръстени със Светия Дух не след много дни.
\par 6 И тъй, веднъж, като се събраха, те го питаха, казвайки: Господи, сега ли, ще възвърнеш на Израиля царството?
\par 7 А той рече: Не с за вас да знаете години или времена, които Отец е положил в собствената си власт.
\par 8 Но ще приемете сила, когато дойде върху вас Светият Дух, и ще бъдете свидетели за мене както в Ерусалим, тъй и в цяла Юдея и Самария, и до края на земята;
\par 9 И като изрече това, и те го гледаха, той се възнесе, и облак го прие от погледа им.
\par 10 И като се взираха към небето, когато възлизаше, ето, двама човека в бели дрехи застанаха при тях,
\par 11 които и рекоха: Галилеяни, защо стоите та гледате към небето тоя Исус, който се възнесе от вас на небето, така ще дойде както го видяхте да отива на небето.
\par 12 Тогава те се върнаха в Ерусалим от хълма, наречен Елеонски, който е близо до Ерусалим, на разстояние един съботен ден път.
\par 13 И когато влязоха в града, качиха се в Горната стая, дето живееха Петър и Йоан, Яков и Андрей: Филип и Тома, Вартоломей и Матей, Яков Алфеев и Симон Зилот, и Юда Яковов.
\par 14 Всички тия единодушно бяха в постоянна молитва, [и моление], с някои жени и Мария, майката на Исуса, и с братята му.
\par 15 През тия дни Петър стана посред братята, (а имаше събрано множество около сто и двадесет души), и рече:
\par 16 Братя, трябваше да се изпълни написаното, което Светият Дух предсказа чрез Давидовите уста за Юда, който стана водител на тия, които хванаха Исуса.
\par 17 Защото той се числеше между нас, и получи дял в това служение.
\par 18 Той, прочее, придоби нива от заплатата на своята неправда; и като падна стремглав, пукна се през сред, и всичките му черва изтекоха.
\par 19 И това стана известно на всичките Ерусалимски жители, така щото тая нива се наименува по езика им Акелдама, сиреч, кръвната нива.
\par 20 Защото е писано в книгата на Псалмите:- "Жилището му да запустее, И да няма кой да живее в него; "Друг нека вземе чина му".
\par 21 И тъй, от човеците, които дружеха с нас през всичкото време, когато Господ Исус влизаше и излизаше м е ж д у нас,
\par 22 като почна от времето, когато Йоан кръщаваше и следва до деня, когато се възнесе от нас, един от тях трябва да стане свидетел с нас на възкресението му.
\par 23 И така, поставиха сред двама, Йосифа наречен Варсава, чието презиме бе Юст, и Матия.
\par 24 И помолиха се, казвайки: Ти, Господи, сърцеведче на всички, покажи оногова от тия двама, когото си избрал
\par 25 да вземе мястото в това служение и апостолството, от което отстъпи Юда, за да отиде на своето място.
\par 26 И хвърлиха жребие за тях, и жребието падна на Матия; и той се причисли към единадесетте апостоли.

\chapter{2}

\par 1 И когато настана денят на Петдесятницата, те всички бяха на едно място.
\par 2 И внезапно стана шум от небето като хвученето на силен вятър, и изпълни цялата къща дето седяха.
\par 3 и явиха им се езици като огнени, които се разделяха, и седна по един на всеки от тях.
\par 4 И те всички се изпълниха със Светия Дух, и почнаха да говорят чужди езици, според както Духът им даваше способност да говорят.
\par 5 А тогава престояваха в Ерусалим юдеи, благочестиви човеци, от всеки народ под небето.
\par 6 И като се чу тоя шум, една навалица се събра; и смутиха се, защото всеки един ги слушаше: да говорят на неговия език.
\par 7 И всички, смаяни и зачудени, си думаха: Ето, всички тия, които говорят, не са ли Галилеяни?
\par 8 Тогава както ги слушаше да говорят всеки на собствения наш език, в който сме родени?
\par 9 Партяни, Мидяни и Еламити и жители от Месопотамия, от Юдея и Кападокия Понт и Азия,
\par 10 Фригия и Памфилия, от Египет и ония страни от Ливия, които граничат с Киринея, и посетители от Рим - и юдеи и прозелити,
\par 11 критяни и араби слушаме ги да говорят по нашите езици за великите Божии дела.
\par 12 И те всички се смаяха, и в недоумението си думаха един на друг: Какво значи това?
\par 13 А други им се присмиваха, казвайки: Те са се напили със сладко вино.
\par 14 А Петър, като се изправи с единадесетте, издигна гласа си и им проговори, казвайки: юдеи и всички вие, които живеете в Ерусалим, нека ви стане знайно това, и внимавайте в моите думи.
\par 15 Защото тия не са пияни, както вие мислите, понеже е едвам третия час на деня;
\par 16 но това е казаното чрез пророк Йоила:
\par 17 "И в последните дни, казва Бог, Ще излея от Духа си на всяка твар; И синовете ви и дъщерите ви ще пророкуват, Юношите ви ще виждат видения, И старците ви ще сънуват сънища;
\par 18 Още и на слугите си и на слугините си ще изливам от Духа си в ония дни, и ще пророкуват.
\par 19 И ще покажа чудеса на небето горе, И знамения на земята долу, Кръв, и огън, и пара от дим;
\par 20 Слънцето, ще се превърне в тъмнина, И луната в кръв, Преди да дойде великият И бележит ден Господен.
\par 21 И всеки, който призове името Господне, ще се спаси".
\par 22 Израиляни, послушайте тия думи: Исуса Назарянина, мъж засвидетелствуван между вас от Бога чрез мощни дела, чудеса, и знамения, които Бог извърши чрез него посред вас, както сами вие знаете,
\par 23 него, предаден според определената Божия воля и предузнание, вие разпнахте и убихте чрез ръката на беззаконници;
\par 24 когото Бог възкреси, като развърза болките на смъртта, понеже не беше възможно да бъде държан той от нея.
\par 25 Защото Давид казва за него: "Винаги гледах Господа пред себе си; Понеже той е отдясно ми, за да се не поклатя;
\par 26 Затова се зарадва сърцето ми, и развесели се езикът ми, А още и плътта ми ще престоява в надежда;
\par 27 Защото няма да оставиш душата ми в ада, нито, ще допуснеш твоя светия да види изтление.
\par 28 Изявил си ми пътищата на живота; в присъствието си, ще ме изпълниш с веселба."
\par 29 Братя, мога да ви кажа свободно за Патриарха Давида, че и умря и биде погребан, и гробът му е у нас до тоя ден.
\par 30 И тъй, понеже беше пророк, и знаеше, че Бог с клетва му се обеща, че от плода на неговите чресла [по плът ще въздигне Христа, да го] постави на престола му,
\par 31 той предвиждаше това, говори за възкресението на Христа, че нито той беше оставен в ада, нито плътта му видя изтление.
\par 32 Тогова Исуса Бог възкреси, на което ние всинца сме свидетели.
\par 33 И тъй, като се възвиси до Божията десница, и взе от Отца обещания Свети Дух, той изля това, което виждате и чуете.
\par 34 Защото Давид не се е възнесъл на небесата; но сам той каза: "Рече Господ на моя Господ: Седи отдясно ми,
\par 35 Докле положа враговете ти за твое подножие".,
\par 36 И тъй, нека знае добре целият Израилев дом, че тогова Исуса, когото вие разпнахте, него Бог е направил и Господ и Помазаник.
\par 37 Като чуха това, те, ужилени в сърцата си, рекоха на Петър и на другите апостоли: Какво да сторим братя?
\par 38 А Петър им рече: Покайте се, и всеки от вас нека се кръсти в името Исус Христово за прощение на греховете ви; и ще приемете тоя дар, Светия Дух.
\par 39 Защото на вас е обещанието и на чадата ви, и на всички далечни, колкото Господ, нашият Бог, ще призове при себе си.
\par 40 И с много други думи заявяваше и ги увещаваше, казвайки: Избавете се от това изпорочено поколение.
\par 41 И тъй, тия, които приеха поучението му, се покръстиха; и в същия ден се прибавиха около три хиляди души.
\par 42 И те постоянствуваха в поучението на апостолите, в общението, в преломяването на хляба, и в молитвите.
\par 43 И страх обзе всяка душа; и много чудеса и знамения ставаха чрез апостолите.
\par 44 И всичките вярващи бяха заедно, и имаха всичко общо;
\par 45 и продаваха стоката и имота си, и разподеляха парите на всички, според нуждата на всекиго.
\par 46 И всеки ден прекарваха единодушно в храма, и разчупваха хляб по къщите си, и приемаха храна с радост и простосърдечие,
\par 47 като хвалеха Бога, и печелеха благоволението на всичките люде. А Господ всеки ден прибавяше на църквата ония, които се спасяваха.

\chapter{3}

\par 1 Един ден, когато Петър и Йоан отиваха в храма в десетия час, часа на молитвата,
\par 2 някои носеха един човек куц от рождението си. Него слагаха всеки ден при тъй наречените Красни врата на храма, да проси милостиня от ония, които влизаха в храма.
\par 3 Той, като видя Петра и Йоана, когато щяха да влязат в храма, попроси да му се даде милостиня.
\par 4 А Петър, с Йоана, се взря в него и рече: Погледни ни.
\par 5 И той внимаваше на тях, като очакваше да получи нещо от тях.
\par 6 Но Петър рече: Сребро и злато аз нямам; но каквото имам това ти давам; в името на Исуса Христа Назарянина, [стани и] ходи.
\par 7 И като го хвана за дясната ръка, дигна го и начаса нозете и глезените му добиха сила.
\par 8 И той, като скочи, изправи се и проходи; и влезе с тях в храма та ходеше и скачаше и славеше Бога.
\par 9 И всичките люде го видяха да ходи и да слави Бога,
\par 10 и познаха го, че беше същия, който седеше за милостиня при Красната порта на храма; и изпълниха се с очудване и удивление за това, което бе станало с него.
\par 11 И понеже [изцеленият куц] човек се държеше за Петра и Йоана, то всичките люде смаяни се стекоха при тях в тъй нареченият Соломонов Трем.
\par 12 А Петър като видя това, проговори на людете: Израиляни, защо се чудите за тоя човек? или защо се взирате на нас като че от своя сила или благочестие сме го направили да ходи?
\par 13 Бог Авраамов, Исааков, и Яковов, Бог на бащите ни, прослави служителя си Исуса, когото вие предадохте, и от когото се отрекохте пред Пилата, когато той бе решил да го пусне.
\par 14 Но вие се отрекохте от Светия и праведния, и, като поискахте да ви се пусне един убиец,
\par 15 убихте Началника на живота. Но Бог го възкреси от мъртвите, за което ние сме свидетели.
\par 16 И на основание на вяра в името му, неговото име укрепи тогова, когото гледате и познавате; да! тая вяра, която е чрез него, му даде пред всички вас това съвършено здраве.
\par 17 И сега, братя, аз зная, че вие, както и началниците ви, сторихте това от незнание.
\par 18 Но Бог по тоя начин изпълни това, което беше предизвестил чрез устата на всичките пророци, че неговият Христос ще пострада.
\par 19 Затова, покайте се и обърнете се, за да се заличат греховете ви, та да дойдат освежителни времена от лицето на Господа,
\par 20 и той да ви изпрати определения за вас Христа Исуса,
\par 21 когото трябва да приемат небесата до времето когато ще се възстанови всичко, за което е говорил Бог от века чрез устата на светите си пророци.
\par 22 Защото Моисей е казал, "Господ Бог ще ви въздигне от братята ви пророк, както въздигна мене; него слушайте във всичко каквото би ви рекъл;
\par 23 и всяка душа, която не би послушала тоя пророк, ще бъде изтребена от людете".
\par 24 И всичките пророци от Самуила и насетне, колцината са говорили, и те са известили за тия дни.
\par 25 Вие сте потомци на пророците, и наследници на завета; който Бог направи с бащите ви когато каза на Авраама, "В твоето потомство ще се благословят всички земни племена".
\par 26 Бог, като възкреси Служителя си, първом до вас го изпрати за да ви благослови, като отвръща всеки от вас от нечестието ви.

\chapter{4}

\par 1 И когато те още говореха на людете, свещениците и началникът на храмовата стража и Садукеите надойдоха върху тях,
\par 2 възмутени задето те поучаваха людете и проповядваха, в лицето на Исуса, възкресението на мъртвите.
\par 3 И тъй, туриха ръце на тях и поставиха ги под стража за утрешния ден, защото беше вече привечер.
\par 4 А мнозина от тия, които чуха словото, повярваха; и числото на повярвалите мъже стигна до пет хиляди.
\par 5 И на утрешния ден се събраха в Ерусалим началниците им, Старейшините, и книжниците;
\par 6 и първосвещеникът Анна, и Каиафа, Йоан, Александър и всички които бяха от първосвещеническия род.
\par 7 И като поставиха Петра и Йоана насред, питаха ги: С каква сила, или в кое име, извършихте това?
\par 8 Тогава Петър, изпълнен със Светия Дух, им рече: Началници народни и Старейшини,
\par 9 ако ни изпитвате днес за едно благодеяние сторено на немощен човек, чрез какво биде той изцелен,
\par 10 да знаете всички вие и всичките Израилеви люде, че чрез името на Исуса Христа Назарянина, когото вие разпнахте, когото Бог възкреси от мъртвите, чрез това име тоя човек стои пред вас здрав.
\par 11 Той е камъкът, който вие зидарите презряхте, който стана глава на ъгъла.
\par 12 И чрез никой друг няма спасение; защото няма под небето друго име дадено между човеците, чрез което трябва да се спасим.
\par 13 А те, като гледаха с дързост на Петра и Йоана и бяха вече забележили, че са неграмотни и неучени човеци, чудеха се; и познаха, че са били с Исуса.
\par 14 А като видяха изцеления човек стоящ с тях, нямаха какво да противоречат.
\par 15 Затова, като им заповядаха да излязат вън от синедриона, съвещаваха се помежду си, казвайки:
\par 16 Какво да сторим на тия човеци, защото на всичките Ерусалимски жители е известно, че бележито знамение стана чрез тях и не можем да го опровергаем.
\par 17 Но, за да се не разнася повече между людете, нека ги заплашим, та да не говорят вече никому в това име.
\par 18 прочее, те ги повикаха та им заръчаха да не говорят никак, нито да поучават в Исусовото име.
\par 19 А Петър и Йоан в отговор им рекоха: Право ли е пред Бога да слушаме вас, а не Бог, разсъдете;
\par 20 Защото ние не можем да не говорим това що сме видели и чули.
\par 21 А те, като ги заплашиха изново, пуснаха ги, понеже не знаеха как да ги накажат, поради людете; защото всички славеха Бога за станалото.
\par 22 Защото човекът, над когото се извърши това чудо на изцеление, беше на повече от четиридесет години.
\par 23 И когато ги пуснаха, те дойдоха при своите си та известиха всичко що им рекоха първосвещениците и старейшините.
\par 24 А те като чуха, издигнаха единодушно глас към Бога и рекоха: Владико, ти си Бог, който си направил небето, земята, морето, и всичко що е в тях,
\par 25 ти чрез Светия Дух, говорещ чрез устата на слугата ти, баща на Давида, си рекъл: "Защо се разяряваха народите, и людете намислюваха суети?
\par 26 Опълчаваха се земните царе, И управниците се събираха заедно, против Господа и против неговия Помазаник".
\par 27 Защото наистина и Ирод и Понтийски Пилат, с езичниците н израилевите люде, се събраха в тоя град против твоя свят Служител Исуса, когото си помазал,
\par 28 за да извършат всичко що твоята ръка и твоята воля са определили да стане.
\par 29 И сега, Господи, погледни на техните заплашвания, и дай на своите слуги да говорят твоето слово с пълна дързост.
\par 30 докато ти простираш ръката си за да изцеляваш и да стават знамения и чудеса чрез името на твоя свет служител Исуса.
\par 31 И като се помолиха, потресе се мястото дето бяха събрани, и всички се изпълниха със Светия Дух, и с дързост говореха Божието слово.
\par 32 А множеството на повярвалите имаше едно сърце и душа; и ни един от тях не казваше, че нещо от имота му е негово, но всичко им беше общо.
\par 33 И апостолите с голяма сила свидетелствуваха за възкресението на Господа Исуса; и голяма благодат почиваше над всички тях.
\par 34 Па и никой от тях не беше в лишение, защото всички, които бяха стопани на ниви или, на къщи, продаваха ги, и донасяха цената на продаденото,
\par 35 и слагаха я при нозете на апостолите; и раздаваше се на всекиго според колкото имаше нужда.
\par 36 Така Йосиф, наречен от апостолите Варава (което значи син на увещание), Левит, родом Кипрянин,
\par 37 като имаше земя, продаде я, и донесе парите та ги сложи пред нозете на апостолите.

\chapter{5}

\par 1 А някой си човек на име Анания, с жена си Сапфира, продаде имот,
\par 2 и задържа нещо от цената, със знанието и на жена си и донесе една част и я сложи пред нозете на апостолите.
\par 3 А Петър, рече: Анание, защо изпълни Сатана сърцето ти, да излъжеш Светия Дух и да задържиш от цената на нивата?
\par 4 Додето стоеше непродадена не беше ли твоя? И след като се не бяха ли патите в твоя власт? Защо си намислил това нещо в сърцето си? Не си излъгал човеци, но Бога.
\par 5 И Анания, като слушаше тия думи, падна и издъхна; и голям страх обзе всички, които чуха това.
\par 6 И по-младите мъже станаха, обвиха го, и го изнесоха па го погребаха.
\par 7 И като се минаха около три часа; влезе и жена му без да знае за станалото.
\par 8 И Петър я попита: Кажи ми, за толкова ли продадохте нивата? И тя рече за толкова.
\par 9 А Петър и рече: защо се съгласихте да изкусите Господния Дух? Ето нозете на тия, които погребаха мъжа ти, са на вратата, и ще изнесат и тебе.
\par 10 И тя начаса падна до нозете му и издъхна; а момците като влязоха, намериха я мъртва и изнесоха я та я погребаха до мъжа й.
\par 11 И голям страх обзе цялата църква и всички, които чуха това.
\par 12 И чрез ръцете на апостолите ставаха много знамения и чудеса между людете, [и те всички бяха единодушни в Соломоновия Трем;
\par 13 а от другите никой не смееше да се присъеднни към тях; людете обаче ги величаеха;
\par 14 и още по-голямо множество повярвали в Господа мъже и жени се прибавяха],
\par 15 така щото даже изнасяха болните по улиците и ги слагаха на постелки и на легла, та, като заминаваше Петър, поне сянката му да засегне някого от тях.
\par 16 Събираше се още и множество от градовете около Ерусалим та носеха болни и измъчваните от нечисти духове и всички се изцеляваха.
\par 17 Тогава станаха първосвещеникът и всички, които бяха с него, съставляващи Садукейската Секта, та изпълнени със завист,
\par 18 туриха ръце на апостолите и положиха ги в тъмница.
\par 19 Но ангел от Господа през нощта отвори вратата на тъмницата та ги изведе и рече.
\par 20 Идете, застанете в храма та говорете на людете всичките думи на тоя живот.
\par 21 Те, като чуха това на съмване влязоха в храма и поучаваха. А първосвещеникът дойде и ония, които бяха с него, и като свикаха синедриона и цялото старейшинство на Израиляните, пратиха в тъмницата да доведат апостолите.
\par 22 Но служителите, като отидоха, не ги намериха в тъмницата; и върнаха се та известиха, казвайки:
\par 23 Тъмницата намерихме заключена твърде здраво, и стражарите да стоят при вратата; но като отворихме, не намерихме никого вътре.
\par 24 А Началникът на храмовата стража и първосвещениците като чуха тия думи, бяха в недоумение поради тях, та се чудеха какво ще последва от това.
\par 25 Но дойде някой си та им извести: Ето, човеците, които турихте в тъмницата, стоят в храма и поучават людете.
\par 26 Тогава отиде началникът със служителите и ги доведе, обаче без насилие; защото се бояха от людете, да не би да ги замерват с камъни.
\par 27 И като ги доведоха, поставиха ги пред синедриона; и първосвещеникът ги попита, казвайки:
\par 28 Строго ви запретихме да не поучавате в това име; но ето, напълнили сте Ерусалим с учението си, и възнамерявате да докарате върху нас кръвта на тоя човек.
\par 29 А Петър и апостолите в отговор рекоха: Подобава да се покоряваме на Бога, а не на човеците.
\par 30 Бог на бащите ни възкреси Исуса, когото вие убихте като го повесихте на дърво.
\par 31 Него Бог, възвиси до десницата си за началник и спасител, да даде покаяние на Израиля и прощение на греховете.
\par 32 И ние сме свидетели [нему] за тия неща, както е и Светия дух, когото Бог даде на ония, които му се покоряват.
\par 33 А те, като чуха това, късаха се от яд, и възнамеряваха да ги убият.
\par 34 Но един фарисей на име Гамалиил, законоучител, почитан от всичките люде, се изправи в синедриона и заповяда да извадят вън апостолите за малко време;
\par 35 и рече на събора: Израиляни, внимавайте добре какво ще направите на тия човеци.
\par 36 Защото в предишни дни възстана Тевда и представяше себе си за голям човек, към когото се присъединиха около четиристотин м ъ ж е на брой, който биде убит, и всички, които му се покоряваха, се разпиляха и изчезнаха.
\par 37 След него възстана Галилеянинът Юда през времето на записването, и отвлече след себе си някои от людете; и той загина, и всички, които му се покоряваха, се разпръснаха.
\par 38 И сега ви казвам, Оттеглете се от тия човеци и оставете ги защото ако това намерение или това дело е от човеци, ще се повали;
\par 39 но ако е от Бога, не ще можете го повали. Пазете се да не би да се намерите и богопротивници.
\par 40 И те го послушаха; и, като повикаха апостолите, биха ги, и заръчаха им да не говорят в Исусово име, и ги пуснаха.
\par 41 А те си отидоха от присъствието на синедриона, възрадвани задето се удостоиха да претърпят опозоряване за Исусовото име.
\par 42 И ни един ден не преставаха да поучават и да благовествуват и в храма и по къщите си, че Исус е Христос.

\chapter{6}

\par 1 А през тия дни, когато се умножаваха учениците, възникна ропот от гръцките юдеи против еврейските, задето във всекидневното раздаване на потребностите техните вдовици били пренебрегвани.
\par 2 По това, дванадесетте свикаха всичките ученици и рекоха: Не е добре ние да оставим Божието слово и да прислужваме на трапези.
\par 3 И тъй, братя, изберете измежду вас седем души с удобрен характер, изпълнени с Духа и с мъдрост, които да поставим на тая работа.
\par 4 А ние ще постоянствуваме в молитвата и в служение на словото.
\par 5 И това предложение се хареса на цялото множество; и избраха Стефана, мъж пълен с вяра и със Светия Дух, и Филипа, Прохора, Никанора, Тимона, Пармена и Николая, е д и н прозелит от Антиохия,
\par 6 Тях поставиха пред апостолите; и те, като се помолиха, положиха ръце на тях.
\par 7 И Божието учение растеше, и числото на учениците в Ерусалим се умножаваше твърде много; и голямо множество от свещениците се подчиняваха на вярата.
\par 8 А Стефан, пълен с благодат и сила, вършеше големи чудеса и знамения между людете.
\par 9 Тогава някои от синагогата, наречена Синагога на Либертинците, и от Киринейците и Александринците, и от Киликия и Азия, подигнаха се и се препираха със Стефана.
\par 10 Но не можаха да противостоят на мъдростта и Духа с който той говореше.
\par 11 Тогава подучиха човеци да казват: Чухме го да говори хулни думи против Мойсея и против Бога.
\par 12 И подбудиха людете със старейшините и книжниците, и, като дойдоха върху него, уловиха го и го докараха в синедриона,
\par 13 дето поставиха лъжесвидетели, които казаха: Т о я човек говори думи против това свето място и против закона;
\par 14 защото го чухме да казва, че тоя Исус Назарянин ще разруши това място, и ще измени обредите, които Моисей ни е предал.
\par 15 И всички, които седяха в синедриона, като се вгледаха в него, видяха лицето му като че беше лице на ангел.

\chapter{7}

\par 1 Тогава първосвещеникът рече: Така ли е това?
\par 2 А той каза: Братя и бащи, слушайте: Бог на славата се яви на отца ни Авраама когато беше в Месопотамия, преди да се засели в Харан, и му рече,
\par 3 "Излез из отечеството си и из рода си, та дойди в земята, която ще ти покажа."
\par 4 Тогава той излезе от Халдейската земя и се засели в Харан. И оттам, след смъртта на баща му, Бог го пресели в тая земя, в която вие сега живеете.
\par 5 И не му даде наследство в нея ни колкото една стъпка от нога, а обеща се да я даде за притежание нему и на потомството му след него, докато той още нямаше чадо.
\par 6 И Бог му говори в смисъл, че неговите потомци щяха да бъдат преселени в чужда земя, дето щяха да ги поробят и притесняват четиристотин години.
\par 7 Но аз, рече Бог, ще съдя народа, на който ще робуват: и подир това ще излязат и ще ми служат на това място.
\par 8 И му даде в завет обрязването; и така Авраам роди Исаака, и обряза го в осмия ден; Исаака роди Якова, а Яков дванадесетте патриарси.
\par 9 А патриарсите завидяха на Йосифа та го продадоха в Египет; Бог обаче беше с него,
\par 10 и го избави от всичките му беди, и му даде благоволение и мъдрост преди Египетския цар Фараона, който го постави управител над Египет и над целия си дом.
\par 11 И настана глад по цялата Египетска и Ханаанска земя и голямо бедствие; и бащите ни не намираха прехрана.
\par 12 А Яков, като чу, че имало жито в Египет, изпрати първи път бащите ни;
\par 13 и на втория път Йосиф се опозна на братята си, и Йосифовия род стана известен на Фараона.
\par 14 Йосиф прати да повикат баща му Якова и целия му род, седемдесет и пет души.
\par 15 И тъй Яков слезе в Египет, дето умря, той и бащите ни;
\par 16 и пренесоха ги в Сихем, та ги положиха в гроба в Сихем, който Авраам беше купил, с цена в сребро от синовете на Емора.
\par 17 А като наближаваше времето да се изпълни обещанието, което Бог беше утвърдил на Авраама, людете бяха нарасли и се умножили в Египет,
\par 18 докле се издигна друг цар над Египет, който не познаваше Йосифа.
\par 19 Той с коварно постъпване против нашия род дотолкова притесняваше бащите ни, щото да хвърля децата им за да не остават живи.
\par 20 В това време се роди Моисей, който беше прекрасно дете, и когото храниха три месеца в бащиния му дом.
\par 21 И когато го хвърлиха, фараоновата дъщеря го взе и го отхрани за свой син.
\par 22 И Моисей биде научен на всичката Египетска мъдрост; и бе силен в слово и в дело.
\par 23 А като навършваше четиридесетата си година, дойде му на сърце да посети братята си Изриляните.
\par 24 И бидейки, един от тях онеправдан, защити го, и отмъсти за притеснения като порази Египтянина.
\par 25 мислейки, че братята му ще разберат какво Бог чрез негова ръка им дава избавление; но те не разбраха.
\par 26 На утрешния ден той им се яви когато двама от тях се биеха, и като искаше да ги помири, каза, Човеци, вие сте братя; защо се онеправдавате един, друг?
\par 27 А тоя, който онеправдаваше ближния си, го отблъсна, и рече, Кой те е поставил началник и съдия над нас?
\par 28 И мене ли искаш да убиеш както уби вчера Египтянина?
\par 29 Поради тая дума, Мойсей побягна, и стана пришелец в Мадиамската земя, дето роди двама сина.
\par 30 И като се навършиха четиридесет години, яви му се ангел Господен в пустинята на Синайската планина, всред пламъка на една запалена къпина.
\par 31 А Мойсей, като видя гледката, почуди и се; но когато се приближаваше да прегледа, дойде глас от Господ,
\par 32 "Аз съм Бог на бащите ти, Бог Авраамов, Исааков, и Яковов". И Моисей се разтрепери и не смееше да погледне.
\par 33 И Господ му рече, "изуй обущата от нозете си, защото мястото, на което стоиш, е света земя.
\par 34 Видях, видях злостраданието на людете, които са в Египет, чух стенанието им, слязох за да ги избавя. Дойди, прочее, и ще те изпратя в Египет".
\par 35 Този Моисей, когото бяха отказали да приемат като му рекоха, Кой те постави началник и съдия? него Бог, чрез ръката на ангела, който му се яви в къпината, прати и за началник и за избавител.
\par 36 Той ги изведе, като върши чудеса и знамения в Египет, в Червеното море, и в пустинята през четиридесет години.
\par 37 Това е същия Моисей, който рече на израиляните: "Бог ще ви въздигне от братята ви пророк, както въздигна и мене".
\par 38 Това е оня, който е бил в църквата в пустинята заедно с ангела, който му говореше на Синайската планина, както и с бащите ни, който и прие животворни думи, да ги предаде на нас;
\par 39 когото нашите бащи не искаха да послушат, но го отхвърлиха, и се повърнаха със сърцата си в Египет,
\par 40 казвайки на Аарона, "Направи ни богове, които да ходят пред нас; защото тоя Моисей, който ни изведе из Египетската земя, не знаем що му стана".
\par 41 И през ония дни те си направиха теле, и принесоха жертва на идола, и се веселяха в това, което техните ръце бяха направили.
\par 42 Затова Бог се отвърна от тях, и ги предаде да служат на небесното войнство, както е писано в книгата на пророците: - "Доме Израилев, на мене ли принасяхте заклани животни и жертви Четиридесет години в пустинята?
\par 43 Напротив, носехте скинията на Молоха, И звездата на бога Рефана, Изображенията, които си направихте за да им се кланяте; Затова ще ви преселя оттатък Вавилон".
\par 44 Скинията на свидетелството беше с бащите ни в пустинята, според както заповяда оня, който каза на Мойсея да я направи по образа който бе видял;
\par 45 която нашите бащи по реда си приеха и внесоха с Исуса Навиева във владенията на народите, които Бог изгони пред нашите бащи: и така стоеше до дните на Давида,
\par 46 който придоби Божието благоволение, и поиска да намери обиталище за Якововия Бог.
\par 47 А Соломон му построи дом.
\par 48 Но всевишния не обитава в ръкотворни храмове, както казва пророка: -
\par 49 "Небето ми е престол, А земята е мое подножие; Какъв дом ще построите за мене? Казва Господ, Или какво е мястото за моя покой?"
\par 50 Не направи ли моята ръка всичко това?
\par 51 Коравовратни и с необрязано сърце и уши! вие всякога се противите на Светия Дух; както правеха бащите ви, така правите и вие.
\par 52 Кого от пророците не гониха бащите ви? а още и избиха ония, които предизвестиха за дохождането на тогова Праведника, на когото вие сега станахте предатели и убийци,
\par 53 вие, които приехте закона чрез ангелско служение, и го не опазихте.
\par 54 А като слушаха това, сърцата им се късаха от яд, и те скърцаха със зъби на него.
\par 55 А Стефан, бидейки пълен със Светия Дух, погледна на небето, и видя Божията слава и Исуса стоящ отдясно на Бога;
\par 56 И Рече: Ето, виждам небесата отворени, и Човешкият Син стоящ отдясно на Бога.
\par 57 Но те, като изкрещяха със силен глас, запушиха си ушите и единодушно се спуснаха върху него.
\par 58 И като го изтласкаха вън от града, хвърляха камъни върху него. И свидетелите сложиха дрехите си при нозете на един момък на име Савел.
\par 59 И хвърлиха камъни върху Стефана, който призоваваше Христа, казвайки: Господи Исусе, приеми духа ми.
\par 60 И като коленичи, извика със силен глас: Господи, не им считай тоя грях. И като рече това, заспа.

\chapter{8}

\par 1 А Савел одобряваше убиването му. И на същия ден се подигна голямо гонение против църквата в Ерусалим; и те всички с изключение на апостолите, се разпръснаха по Юдейските и Самарийските окръзи.
\par 2 И някои благочестиви човеци погребаха Стефана и ридаха за него твърде много.
\par 3 А Савел опустошаваше църквата, като влизаше във всяка къща и завличаше мъже и жени та ги предаваше в тъмница.
\par 4 А тия, които бяха се разпръснали, обикаляха и разгласяваха благовестието.
\par 5 Така Филип слезе в град Самария и им проповядваше Христа.
\par 6 И народът единодушно внимаваше на това, което Филип им говореше, като слушаха всичко, И виждаха знаменията, които вършеше.
\par 7 Защото нечистите духове, като викаха със силен глас, излизаха от мнозина, които ги имаха; и мнозина парализирани и куци бидоха изцелени;
\par 8 тъй щото настана голяма радост в оня град.
\par 9 А имаше от по-напред в града един човек на име Симон, който, като представяше себе си за някаква велика личност, правеше магии и очудваше населението на Самария.
\par 10 На него внимаваха всички, от малък до голям, казвайки: Тоя е така наречената Велика Божия Сила.
\par 11 И внимаваха на него, понеже за доста време ги беше учудвал с магиите си.
\par 12 Но когато повярваха на Филипа, който благовестяваше за Божието царство и за Исус Христовото име, кръщаваха се мъже и жени.
\par 13 И самият Симон повярва и, като се кръсти, постоянно придружаваше Филипа, та, като гледаше, че стават знамения и велики дела, удивяваше се.
\par 14 А апостолите, които бяха в Ерусалим, като чуха, че Самария приела Божието учение, пратиха им Петра и Йоана,
\par 15 които, като слязоха, помолиха се за тях за да приемат Светия Дух;
\par 16 Защото той не беше слязъл още ни на един от тях; а само бяха кръстени в Исус Христовото име.
\par 17 Тогава апостолите полагаха ръце на тях, и те приемаха Светия Дух.
\par 18 А Симон, като видя, че с полагането на апостолските ръце, се даваше Светия Дух, предложи им пари, казвайки:
\par 19 Дайте и на мене тая сила щото на когото положа ръце, да приема Светия Дух.
\par 20 А Петър му рече: Парите ти с тебе заедно де погинат, защото си помислил да придобиеш Божий дар с пари.
\par 21 Ти нямаш нито участие, нито дял в тая работа, защото сърцето ти не е право пред Бога.
\par 22 Затуй, покай се от това твое нечестие, и моли се Господу дано ти се прости тая помисъл на сърцето ти;
\par 23 понеже гледам, че си в горчива жлъчка и си вързан в неправда.
\par 24 А Симон в отговор рече: молете се вие на Господа за мене, да ме не постигне нищо от онова що рекохте.
\par 25 Те прочее, след като засвидетелствуваха и разгласяваха Господнето учение, върнаха се в Ерусалим, като по пътя проповядваха благовестието на много Самарийски села.
\par 26 А ангел от Господа говори на Филипа, казвайки: Стани, иди към юг, по пътя, който слиза от Ерусалим през пустинята за Газа.
\par 27 И той стана та отиде. И, ето, човек от Етиопия, скопец, велможа на Етиопската царица Кандакия, който беше поставен над всичкото и съкровище, и беше дошъл в Ерусалим да се поклони,
\par 28 на връщане седеше в колесницата си и четеше пророка Исаия.
\par 29 А Духът рече на Филипа: приближи се и придружи тая колесница.
\par 30 И Филип се завтече та го чу като прочиташе пророка Исаия, и каза: Ами разбираш ли какво четеш?
\par 31 А той рече: Как да разбера, ако ме не опъти някой? И помоли Филипа да се качи и да седне с него.
\par 32 А мястото от писанието, което четеше, беше това: "Като овца биде заведен на клане: И както агне пред стригача си не издава глас, Така не отваря устата си;
\par 33 В унижение отмени се съдбата му, А поколението му, - Кой ще го изкаже? Защото се взе живота му от земята."
\par 34 И скопецът продума та рече на Филипа: Кажи ми, моля ти се, за кого казва това пророкът, - за себе си ли, или за някой друг?
\par 35 А Филип отвори уста, и като почна от това писание, благовести му Исуса.
\par 36 И като вървяха по пътя, дойдоха до една вода; и скопецът каза: Ето вода; какво ми пречи да се кръстя?
\par 37 [и Филип рече: Ако вярваш от все сърце можеш. А той в отговор каза: Вярвам, че Исус Христос е Син Божий].
\par 38 Тогава заповяда да се спре колесницата; и двамата слязоха във водата, и Филип и скопецът; и кръсти го.
\par 39 А когато излязоха из водата, Господния Дух грабна Филипа; и скопецът вече го не видя, защото възрадван продължи пътя си.
\par 40 А Филип се намери в Азот; и, като преминаваше, проповядваше благовестието по всичките градове докле стигна в Кесария.

\chapter{9}

\par 1 А Савел, като още дишаше заплашване и убийство против Господните ученици, отиде при първосвещеника
\par 2 и поиска от него писма но синагогите в Дамаск, че, ако намери някой от тоя Път, мъже или жени, да ги докара вързани в Ерусалим.
\par 3 И на отиване, като наближаваше Дамаск, внезапно блесна около него светлина от небето.
\par 4 И като падна на земята, чу глас, който му каза: Савле, Савле, защо ме гониш?
\par 5 А той рече: Кой си ти, Господи? И той отговори: Аз съм Исус, когото ти гониш.
\par 6 Но стани, влез в града, и ще ти се каже какво трябва да правиш.
\par 7 А мъжете, които го придружаваха, стояха като вцепенени, понеже чуха гласа, а не видяха никого.
\par 8 И Савел стана от земята, и когато отвори очите си, не виждаше нищо; и водеха го за ръка та го въведоха в Дамаск.
\par 9 И прекара три дни без да види, и не яде нито пи.
\par 10 А в Дамаск имаше един ученик на име Анания; и Господ му рече във видение; Анание! А той рече; Ето ме, Господи.
\par 11 И Господ му рече: Стани та иди на улицата, която се нарича Права, и попитай в къщата на Юда за един Тарсянин на име Савел; защото, ето, той се моли;
\par 12 и е видял един човек на име Анания да влиза и да полага ръце на него за да прогледа.
\par 13 Но Анания отговори: Господи, чул съм от мнозина за тоя човек, колко зло е сторил на твоите светии в Ерусалим.
\par 14 И тука имал власт от първосвещениците да върже всички, които призовават твоето име.
\par 15 А Господ му рече: Иди, защото той ми е съд избран да разгласява моето име пред народите и царе и пред Израиляните;
\par 16 защото аз ще му покажа колко много той трябва да пострада за името ми.
\par 17 И тъй, Анания отиде и влезе в къщата; и като положи ръце на него, рече: Брате Савле, Господ ме изпрати, - същият Исус който ти се яви на пътя, по който ти идеше, - за да прогледаш и да се изпълниш с Светия Дух.
\par 18 И начаса паднаха от очите му като люспи, и той прогледа; и стана та се кръсти.
\par 19 А като похапна, доби сила, и преседя няколко дни с учениците в Дамаск.
\par 20 И почна веднага да проповядва по синагогите, че Исус е Божия Син.
\par 21 И всички, които го слушаха, се удивляваха, и казваха: Не е ли тоя, който в Ерусалим съсипвал тия, които призовавали това име, и дошъл тука за да закара такива вързани при първосвещениците?
\par 22 А Савел се засилваше още повече, и преодоляваше юдеите, които живееха в Дамаск, като доказваше, че това е Христос.
\par 23 И когато се минаха доста дни, юдеите се наговориха да го убият;
\par 24 но техният заговор стана известен на Савла. И те даже вардеха портите деня и нощя за да го убият,
\par 25 но учениците му го взеха през нощта и го свалиха през стената, като го спуснаха с кош.
\par 26 И когато дойде в Ерусалим, той се стараеше да дружи с учениците; но всички се боеха от него, понеже не вярваха че е ученик.
\par 27 Но Варнава го взе та то доведе при апостолите, и разказа им как бил видял Господа по пътя, и че му говорил, и как в Дамаск дързостно проповядвал в Исусовото име.
\par 28 И той влизаше и излизаше с тях в Ерусалим, като дързостно проповядваше в Господнето име.
\par 29 И говореше и се препираше с гръцките Юдеи. а те търсеха случай да го убият.
\par 30 Но братята, като разбраха това, заведоха го в Кесария и изпратиха го в Тарс.
\par 31 И тъй по цяла Юдея, Галилея и Самария църквата имаше мир и се назидаваше, и, като ходеше в страх от Господа и в утехата на Светия Дух, се умножаваше.
\par 32 И Петър, като обикаляше всичките вярващи, слезе и светиите, които живееха в Лида.
\par 33 И там намери един човек на име Еней, който бе пазел легло осем години, понеже беше парализиран.
\par 34 И Петър му рече: Енее, Исус Христос те изцелява; стани, направи леглото си. И веднага той стана.
\par 35 И всички, които живееха в Лида и в Саронското поле видяха, и се обърнаха към Господа.
\par 36 А в Иопия имаше една ученица на име Тавита (което значи Сърна). Тая жена вършеше много добри дела и благодеяния.
\par 37 И през тия дни тя се разболя и умря; и като я окъпаха, положиха я в една горна стая.
\par 38 И понеже Лида беше близо до Иопия, учениците, като чуха, че Петър бил там, изпратиха до него двама човека да го помолят: Не се бави да дойдеш при нас.
\par 39 И Петър стана и отиде с тях. И като дойде, заведоха го в горната стая; и всичките вдовици стояха около него та плачеха, и, му показваха многото ризи и дрехи, които правеше Сърна когато бе с тях.
\par 40 А Петър изкара всички навън, коленичи та се помоли, и се обърна към тялото и рече: Тавито, стани. И тя отвори очите си, и като видя Петра, седна.
\par 41 И той и подаде ръка и я изправи; после повика светиите и вдовиците и представи им я жива.
\par 42 И това стана известно по цяла Иопия; и мнозина повярваха в Господа.
\par 43 А Петър преседя дълго време в Иопия у някого си Симон.

\chapter{10}

\par 1 Имаше в Кесария един човек на име Корнилий, стотник от така наречения Италийски Полк.
\par 2 Той беше благочестив и боеше се от Бога с целия си дом, раздаваше много милостини на людете, и непрестанно се молеше на Бога.
\par 3 Около деветия час през деня той видя ясно във видение един ангел от Бога, който слезе при него и му рече: Корнилие!
\par 4 А той, като се взря в него, уплашен рече: Що е Господи? И ангелът му каза: Твоите молитви и твоите милостини възлязоха пред Бога за спомен.
\par 5 И сега изпрати човеци в Иопия да повикат едного Симона, чието презиме е Петър.
\par 6 Той гостува у някой си кожар Симон, чиято къща е край морето, (той ще ти каже що трябва да правиш).
\par 7 И като си отиде ангелът, който му говореше, той повика двама от слугите си и един благочестив войник от тия, които редовно му служеха;
\par 8 и като им разказа всичко, прати ги в Иопия.
\par 9 А на утрешния ден, когато те пътуваха и наближаваха до града, около шестия час Петър се качи на къщния покрив да се помоли.
\par 10 И като огладня, поиска да яде; но докато приготвяха, той дойде в изстъпление,
\par 11 И видя небето отворено и някакъв съд, като голяма плащаница, да слиза, спускан чрез четирите ъгъла към земята.
\par 12 В него имаше всякакви земни четвероноги и гадини и небесни птици.
\par 13 И дойде глас към него: Стани, Петре, заколи и яж.
\par 14 А Петър рече: Никак, Господи; защото никога не съм ял нищо мръсно и нечисто.
\par 15 И пак дойде към него втори път глас; Което Бог е очистил, ти за мръсно го не считай.
\par 16 И това стана три пъти, след което съда се вдигна веднага на небето.
\par 17 А докато Петър беше в недоумение, какво значеше видението, което беше видял, ето, изпратените от Корнилия човеци, като бяха разпитали за Симоновата къща, застанаха пред портата,
\par 18 и повикаха та попитаха: Тука ли гостува Симон, чието презиме е Петър?
\par 19 И докато Петър още размишляваше за видението, духът му рече: Ето, трима човека те търсят.
\par 20 Стани, слез, та иди с тях; и никак не се съмнявай, защото аз съм ги изпратил.
\par 21 И тъй, Петър слезе при човеците и рече: Ето, аз съм оня, когото търсите. По коя причина дойдохте?
\par 22 А те рекоха: Стотникът Корнилий, човек праведен и който се бои от Бога, и с добро име между целия юдейски народ, биде уведомен от Бога чрез един свет ангел да го повика у дома си и да чуе думи от тебе.
\par 23 Тогава той ги покани вътре та ги нагости. И на сутринта той стана та излезе с тях; а някои от братята от Иопия го придружиха.
\par 24 И на другия ден влязоха в Кесария; а Корнилий ги чакаше, като беше свикал роднините си и близките си приятели.
\par 25 И когато влизаше Петър, Корнилий го посрещна, падна пред нозете му, и се поклони.
\par 26 А Петър го дигна, казвайки: Стани; и аз съм човек.
\par 27 И разговаряйки се с него той влезе и намери мнозина събрани.
\par 28 И рече им: вие знаете колко незаконно е за юдеин да има сношение или да дружи с иноплеменник; Бог обаче ми показа, че не бива да наричам никого мръсен или нечист;
\par 29 Затова, щом ме повикахте, дойдох без да възражавам; и тъй, питам аз, по коя причина сте ме повикали?
\par 30 И Корнилий рече: Преди четири дена, в тоя час, прекарвах деветия час в молитва у дома; и ето, пред мене застана човек със светло облекло, който каза,
\par 31 Корнилие, твоята молитва е послушана и твоите милостини се помнят пред Бога.
\par 32 Прати, прочее, в Иопия да повикай Симона, чието презиме е Петър; той гостува у дома на един кожар Симон, край морето, (той като дойде, ще ти говори).
\par 33 Начаса, прочее, пратих до тебе и ти си сторил добре да дойдеш. И тъй, ние всинца присътствуваме тука пред Бога за да чуем все що ти е заповядано от, Господа.
\par 34 А Петър отвори уста и рече: Наистина виждам, че Бог не гледа на лице;
\par 35 но във всеки народ оня, който му се бои и върши правото, угоден му е.
\par 36 Словото, което той прати на Израиляните та им благовестяваше мир чрез Исуса Христа, (който е господар на всички),
\par 37 това слово вие знаете, което, след кръщението, проповядвано от Йоана, се разпространи по цяла Юдея, начиная от Галилея,
\par 38 именно, Исус от Назарет, - как Бог го помаза със Светия Дух и със сила; който обикаляше да прави благодеяния и да изцелява всички угнетявани от дявола; защото Бог беше с него.
\par 39 И ние сме свидетели на всичко що извърши той и в Юдейската земя и в Ерусалим; когото и убиха, като го повесиха на дърво.
\par 40 Него Бог възкреси на третия ден, и даде му да се яви,
\par 41 не на всичките люде, а на нас предизбраните от Бога свидетели, които ядохме и пихме с него след като възкръсна от мъртвите.
\par 42 И заръча ни да проповядваме на людете, и да свидетелствуваме, че тоя е определеният от Бога Съдия на живите и мъртвите.
\par 43 За него свидетелствуват всичките пророци, че всеки, който повярва в него, ще получи чрез неговото име прощение на греховете си.
\par 44 Докато Петър още говореше тия думи, Светият Дух слезе на всички, които слушаха словото.
\par 45 И обрязаните вярващи, дошли с Петра се смаяха задето дарът на Светия Дух, се изля и на езичниците;
\par 46 защото ги чуеха да говорят чужди езици и да величаят Бога. Тогава Петър проговори:
\par 47 Може ли някой да забрани водата, да се не кръстят тия, които приеха Светия Дух, както и ние?
\par 48 И заповяда да бъдат кръстени в името на Исуса Христа. Тогава му се примолиха да преседи няколко дни у тях.

\chapter{11}

\par 1 И апостолите и братята, които бяха в Юдея, чуха, че и езичниците приели Божието слово.
\par 2 И когато Петър възлезе в Ерусалим, ония, които бяха от обрязаните, се препираха с него, казвайки:
\par 3 При необрязани човеци си влизал и си ял с тях.
\par 4 А Петър захвана та им изложи наред станалото, като каза:
\par 5 Аз бях на молитва в град Иопия, и в изстъпление видях видение: един съд като голяма плащаница слизаше, спускан чрез четирите ъгъла от небето, и дойде дори до мене.
\par 6 В него като се взрях и разсъждавах, видях земните четвероноги, зверове, и гадини, и небесни птици.
\par 7 Чух още и глас който ми каза: Стани, Петре, заколи и яж.
\par 8 Но аз рекох, Никак Господи, защото не е влизало в устата ми нищо мръсно или нечисто.
\par 9 Обаче, втори глас от небето продума, Което е Бог очистил, ти за мръсно го не считай.
\par 10 Това стана три пъти, след което всичко се дръпна пак на небето.
\par 11 И, ето, същия час трима човека, изпратени от Кесария до мене, пристигнаха пред къщата, в която бяхме.
\par 12 И Духът ми каза да отида с тях, и никак да не правя разлика между човеците, а с мене дойдоха и шестимата тия братя, и влязохме в къщата на човека.
\par 13 И той ни разказа как видял ангела да стои в къщата му и да казва, Прати човеци в Иопия да повикат Симона, чието презиме е Петър;
\par 14 той ще ти каже думи чрез които ще се спасиш и целия ти дом.
\par 15 И когато почнах да говоря, Светият Дух слезе на тях както и на нас изначало.
\par 16 Тогава си спомних думата на Господа, как каза, Йоан е кръщавал с вода; а вие ще бъдете кръстени със Светия Дух.
\par 17 И тъй, ако Бог даде същия дар и на тях, когато повярваха в Господа Исуса Христа, както и на нас, кой бях аз та да можех да възпрепятствувам на Бога?
\par 18 Като чуха това, те престанаха да възражават, и славеха Бога, казвайки: И на езичниците Бог даде покаяние за живот.
\par 19 Между това, разпръснатите от гонението, което стана по убиването на Стефана, пътуваха дори до Финикия, Кипър, и Антиохия, като на никой друг не известяваха словото, освен на юдеите.
\par 20 Обаче между тях имаше някои Кипряни и Киринейци, които, като пристигнаха в Антиохия, говореха и на гърците, благовестявайки Господа Исуса.
\par 21 Господнята ръка беше с тях; та голямо число човеци повярваха и се обърнаха към Господа.
\par 22 И стигна известие за тях в ушите на църквата в Ерусалим; и те изпратиха Варнава в Антиохия;
\par 23 който, като дойде и видя делото на Божията благодат, зарадва се, и увещаваше всички да пребъдват в Господа с непоколебимо сърце.
\par 24 Понеже той беше добър човек пълен със Светия Дух и с вяра; и значително множество се прибави към Господа.
\par 25 Тогава той отиде в Тарс да търси Савла;
\par 26 и като го намери, доведе го в Антиохия, та, като се събираха с църквата цяла година, научиха значително множество. И първом в Антиохия учениците се нарекоха Християни.
\par 27 И през тия дни слязоха пророци от Ерусалим в Антиохия,
\par 28 един от които, на име Агав, стана и обяви чрез Духа, че щеше да настане голям глад по цялата вселена; какъвто и стана в дните на Клавдия.
\par 29 Затова, учениците наредиха да изпратят всеки според състоянието си, помощ на братята, които живееха в Юдея;
\par 30 което и сториха, и я изпратиха до презвитерите чрез ръката на Варнава и Савла.

\chapter{12}

\par 1 Около това време цар Ирод простря ръцете си да притесни някои от църквата.
\par 2 Уби с меч Иоановия брат Якова;
\par 3 и, като видя, че беше угодно на юдеите, той при това улови и Петра, това беше през дните на безквасните хлябове,
\par 4 И като го хвана, хвърли го в тъмница, и предаде го на четири четворки войници да го вардят, с намерение да го изведе пред людете подир пасхата.
\par 5 И така, те вардеха Петра в тъмницата; а църквата принасяше пред Бога усърдна молитва за него.
\par 6 И през същата нощ когато Ирод щеше да го изведе, Петър спеше между двама войника, окован с две вериги; и стражари пред вратата вардеха тъмницата.
\par 7 И, ето, един ангел от Господа застана до него, и светлина осия килията; и като побутна Петра по ребрата, разбуди го и рече му: Ставай бърже. И веригите паднаха от ръцете му.
\par 8 И ангелът му рече: Опаши се и обуй сандалите си. И той стори така. Тогава му каза: Облечи дрехата си и дойди подир мене.
\par 9 И Петър излезе и вървеше изподире, без да знае, че извършеното от ангела е действителност, но си мислеше, че вижда видение.
\par 10 А като преминаха първата и втората стража, дойдоха до желязната порта, която води в града, и тя им се отвори сама; и като излязоха през нея, изминаха една улица, и ангелът веднага се оттегли от него.
\par 11 И Петър, когато дойде на себе си, рече: Сега наистина знаят, че Господ изпрати ангела си и ме избави от ръката на Ирода, и от всичко, което юдейските люде очакваха.
\par 12 И като поразмисли, дойде при къщата на Мария, майката на Йоана, чието презиме бе Марко, дето бяха събрани мнозина да се молят.
\par 13 И когато похлопа на вратичката в портата, една слугиня на име Рода излезе да послуша кой е.
\par 14 И щом позна Петровия глас, от радост не отвори портата, а се завтече и извести, че Петър стои пред портата.
\par 15 А те й рекоха: Ти си луда, но тя настояваше, че това що им казва е вярно. Тогава те думаха: Тогава е неговият ангел.
\par 16 А Петър продължаваше да хлопа; и като отвориха, видяха го и се смаяха.
\par 17 А той им помаха с ръка да мълчат, и им разказа как го изведе Господ из тъмницата. И като им каза: Явете това на Якова и на братята, излезе та отиде на друго място.
\par 18 А като се съмна, стана не малко смущение между войниците - какво стана Петър.
\par 19 А Ирод, като го потърси и не го намери изпита стражарите и заповяда да ги погубят. И слезе от Юдея, в Кесария, и там живееше.
\par 20 А Ирод беше много разгневен на Тирците и Сидонците; и те дойдоха единодушно при него, и, като спечелиха подръжката на царевия постелник Власта, просеха помирение; защото тяхната област се хранеше от царевата.
\par 21 И в един определен ден Ирод, облечен в царска одежда, седна на престола и държа реч пред тях.
\par 22 А народът извика: Глас Божий, а не човешки!
\par 23 И понеже не въздаде слава на Бога, начаса един ангел от Господа го порази, та биде изяден от червеи и умря.
\par 24 Между това, Божието учение растеше и се умножаваше.
\par 25 А Варнава И Савел, като свършиха службата си, върнаха се от Ерусалим в Антиохия и взеха със себе си Йоана, чието презиме бе Марко.

\chapter{13}

\par 1 А в Антиохийската църква имаше пророци и учители: Варнава, Симеон, наречен Нигер, Киринееца Луиций, Манаин който беше възпитан заедно с четверовластника Ирода, и Савел.
\par 2 И като служеха на Господа и постеха, Светият Дух рече: Отделете ми Варнава и Савла за работата, на която съм ги призвал.
\par 3 Тогава, като постиха и се помолиха, положиха ръце на тях и ги изпратиха.
\par 4 И така те, изпратени от Светия Дух слязоха в Селевкия, и оттам отплуваха за Кипър.
\par 5 И когато бяха в Саламин, проповядваха Божието слово в юдейските синагоги; и имаха Йоана за свой прислужник.
\par 6 И като преминаха целия остров до Пафа, намериха някой си магьосник, лъжепророк, юдеин, на име Вариисус,
\par 7 който беше с управителя Сергия Павла, един разумен човек. Тоя последния повика Варнава и Савла и поиска да чуе Божието слово.
\par 8 Но магьосникът Елима (защото така се превежда името му) им се противеше, и стараеше се да отвърне управителя от вярата.
\par 9 Но Савел, който се наричаше и Павел, изпълнен със Светия Дух, се вгледа в него и рече:
\par 10 О, ти, пълен с всякакви лукавщини и с всякакво коварство, сине дяволски, враже на всичко що е право, няма ли да престанеш да извращаваш правите пътища на Господа?
\par 11 И сега, ето, Господнята ръка е върху тебе; ти ще ослепееш, и няма да виждаш слънцето за известно време. И начаса падна на него помрачаване и тъмнина; и той се луташе, търсейки да го води някой за ръка.
\par 12 Тогава управителят, като видя станалото, повярва, смаян от Господнето учение.
\par 13 А Павел и дружината му, като отплуваха от Пафа, дойдоха в Перга Памфилийска; а Йоан се отдели от тях и се върна в Ерусалим.
\par 14 А те заминаха от Перга и стигнаха в Антиохия Писидийска; и в съботния ден влязоха в синагогата и седнаха.
\par 15 И след прочитането на закона и пророците, началниците на синагогата пратиха до тях да им кажат: Братя, ако имате някоя увещателна дума за людете, кажете.
\par 16 Павел, прочее, стана та помаха с ръка, и рече: Израиляни, и вие, които се боите от Бога, слушайте.
\par 17 Бог на тоя израилски народ избра бащите ни, и възвиси народа когато престояваха в Египетската земя, и с издигната мишца ги изведе из нея.
\par 18 И за около четиридесет години ги води и храни в пустинята.
\par 19 И като изтреби седем народа в Ханаанската земя, раздели им тяхната земя да им бъде наследство за около четиристотин и петдесет години.
\par 20 След това им дава съдии до пророка Самуил.
\par 21 После поискаха цар; и, Бог им даде Саула, Кисовия син, мъж от Вениаминовото племе, за четиридесет години.
\par 22 И него като отмахна, издигна им за цар Давида, за когото свидетелствува, казвайки, "Намерих Давида, Йесеевия син, човек според сърцето ми, който ще изпълни всичката ми воля".
\par 23 От неговото потомство Бог, според обещанието си, въздигна на Израиля спасител, Исуса,
\par 24 след като Йоан, преди неговото идване, беше проповядвал кръщението на покаяние на целия Израилски народ.
\par 25 И като свършваше попрището си, Йоан казваше, Според както мислите, кой съм аз? Не съм оня, когото очаквате; но, ето, подир мене иде един, комуто не съм достоен да развържа обувката на нозете.
\par 26 Братя, потомци от Авраамовия род, и които измежду вас се боят от Бога, нам се изпрати вестта на това спасение.
\par 27 Защото Ерусалимските жители и техните началници, като не го познаха, а без да разберат пророческите думи, които се прочитат всяка събота, изпълниха ги като го осъдиха.
\par 28 Без да намерят в него нещо достойно за смърт, пак изискаха от Пилата да бъде убит.
\par 29 И когато изпълниха всичко, що бе писано за него, снеха го от дървото и положиха го в гроб.
\par 30 Но Бог го възкреси от мъртвите.
\par 31 И той в продължение на много дни се явяваше на тия, които бяха възлезли с него от Галилея в Ерусалим, които сега са свидетели за него пред людете.
\par 32 И ние ви донесохме блага вест за обещанието, дадено на бащите ни,
\par 33 че Бог го изпълни на техните чада, като възкреси Исуса; както е писано и във втория псалом: - "Ти си мой Син, Аз днес те родих".
\par 34 А че го е възкресил от мъртвите, никога вече да се не връща в изтлението, казва така: - "Ще ви дам Верните милости, обещани на Давида".
\par 35 Защото и в друг Псалом казва, "Няма да оставиш Светеца си да види изтление".
\par 36 Обаче, Давид, след като в своето си поколение послужи на Божието намерение, заспа и биде положен при бащите си, и видя изтление.
\par 37 А тоя, когото Бог възкреси не видя изтление.
\par 38 И така, братя, нека ви бъде известно, че чрез него се проповядва на вас прощение на греховете,
\par 39 и че всеки, който вярва, се оправдава чрез него от всичко, от което не сте могли да се оправдаете чрез Мойсеевия закон.
\par 40 Затова внимавайте да не би да ви постигне казаното от пророците: -
\par 41 "Погледнете, презрители, очудете се, и се погубете; Защото аз ще извърша едно дело във вашите дни, Дело, което никак няма да повярвате, ако и да ви го разкаже някой".
\par 42 Когато си излизаха (из юдейската синагога, езичниците) ги молеха да им се проповядват тия думи и следващата събота.
\par 43 И когато се разиде синагогата, мнозина от юдеите и от благочестивите прозелити тръгнаха след Павла И Варнава; които, като се разговаряха с тях увещаваха ги да постоянствуват в Божията благодат.
\par 44 На следващата събота се събра почти целия град да чуят Божието слово.
\par 45 А юдеите като видяха множествата, изпълниха се със завист, опровергаваха това, което говореше Павел, и хулеха.
\par 46 Но Павел и Варнава говориха дързостно и рекоха: Нужно беше да се възвести първом на вас Божието учение; но понеже го отхвърляте и считате себе си недостойни за вечния живот, ето обръщаме се към езичниците.
\par 47 Защото така ни заповяда Господ, казвайки: - "Поставих те за светлина на народите, За да бъдеш за спасение до края на земята".
\par 48 И езичниците, като слушаха това, радваха се, и славеха Божието учение; и повярваха всички, които бяха отредени за вечния живот.
\par 49 И Господнето учение се разпространяваше по цялата тая страна.
\par 50 А юдеите подбудиха набожните високопоставени жени и градските първенци, и, като повдигнаха гонение против Павла и Варнава, изпъдиха ги из пределите си.
\par 51 А те оттърсиха против тях праха от нозете си, и дойдоха в Икония.
\par 52 А учениците се изпълниха с радост и със Святия Дух.

\chapter{14}

\par 1 А в Икония те влязоха заедно в юдейската синагога, и така говориха, че повярваха голямо множество юдеи и гърци,
\par 2 А непокорните на Божието учение юдеи подбудиха и раздразниха духовете на езичниците против братята.
\par 3 Но пак, те преседяха там доста време и дързостно говореха в Господа, който свидетелствуваше за словото на своята благодат като даваше да стават знамения и чудеса чрез техните ръце.
\par 4 И множеството в града се раздвои; едни бяха с юдеите, а други с апостолите.
\par 5 И когато се породи стремеж у езичниците и юдеите с началниците им за да ги опозорят и да ги убият с камъни,
\par 6 те се научиха и избягаха в Ликаонските градове, Листра и Дервия, и в околните им места,
\par 7 и там проповядваха благовестието.
\par 8 А в Листра седеше някой си човек немощен в нозете си, куц от рождението си, който никога не бе ходил.
\par 9 Той слушаше Павла като говореше; а Павел като се взря в него и видя, че има вяра да бъде изцелен,
\par 10 рече със силен глас: Стани прав на нозете си. И той скочи и ходеше.
\par 11 А народът, като видя какво извърши Павел, извика със силен глас, казвайки по Ликаонски: Боговете, оприличени на човеци са слезли при нас.
\par 12 И наричаха Варнава Юпитер, а Павла Меркурий, понеже той б е ш е главният говорител.
\par 13 И жрецът при Юпитеровото капище, което беше пред града, приведе юнаци и донесе венци на портите, и заедно с народа се канеше да принесе жертва.
\par 14 Като чуха това апостолите Варнава и Павел, раздраха дрехите си, скочиха всред народа, та извикаха, казвайки:
\par 15 О, мъже, защо правите това: и ние сме човеци със същото естество като вас, и благовестяваме ви да се обърнете от тия суети към живия Бог, който е направил небето, земята, морето, и всичко що има в тях;
\par 16 който през миналите поколения е оставял всичките народи да ходят по своите пътища,
\par 17 ако и да не е преставал да свидетелствува за себе си, като е правил добрини и давал ви е от небето дъждове родовити времена, и е пълнил сърцата ви с храна и веселба.
\par 18 И като казваха това, те едвам възпряха множествата да им не принасят жертва.
\par 19 Между това, дойдоха юдеи от Антиохия и Икония, които убедиха народа; и те биха Павла с камъни и го извлякоха вън от града, като мислеха че е умрял.
\par 20 А когато учениците още стояха около него, той стана та влезе в града; и на утринта отиде с Варнава в Дервия.
\par 21 И след като проповядваха благовестието в тоя град и придобиха много ученици, върнаха се в Листра, Икония и Аитиохия,
\par 22 и утвърдяваха душите на учениците, като ги увещаваха да постоянствуват във вярата, и ги учеха, че през много скърби трябва да влезем в Божието царство.
\par 23 И след като им ръкоположиха презвитери във всяка църква и се помолиха с пост, препоръчаха ги на Господа, в когото бяха повярвали.
\par 24 И като минаха през Писидия, дойдоха в Памфилия;
\par 25 И проповядваха учението в Перга, и слязоха в Аталия.
\par 26 Оттам отплуваха за Антиохия, отдето бяха препоръчани на Божията благодат за делото, което сега бяха извършили.
\par 27 И като пристигнаха и събраха църквата, те разказаха всичко, което беше извършил Бог чрез тях, и как беше отворил за езичниците врата за да повярват.
\par 28 И там преседяха доста време с учениците.

\chapter{15}

\par 1 А някои слязоха от Юдея и учеха братята, казвайки: Ако се не обрежете според Мойсеевия обред, не можете се спаси.
\par 2 И тъй, като стана не малко препирня и разискване между тях и Павла и Варнава, братята наредиха Павел и Варнава и някои други от тях да възлязат за тоя въпрос в Ерусалим до апостолите и презвитерите.
\par 3 Те, прочее, изпратени от църквата, минаваха и през Финикия и през Самария, та разказваха за обръщането на езичниците, причиняваха голяма радост на всичките братя.
\par 4 А като стигнаха в Ерусалим, бяха приети от църквата и от апостолите и презвитерите, и разказваха все що беше извършил Бог чрез тях.
\par 5 Но, рекоха те, някои от повярвалите между фарисейската секта станаха та казаха, Нужно е да се обрязват езичниците, и да им се заръча да пазят Мойсеевия закон.
\par 6 Тогава апостолите и презвитерите се събраха да разискат тоя въпрос.
\par 7 И след много разпитване Петър стана та им каза: Братя, вие знаете че в първите дни Бог избра между вас мене, щото езичниците чрез моите уста да чуят евангелското учение и да повярват.
\par 8 И сърцеведец Бог им засвидетелствува като даде и на тях Светия Дух, както и на нас;
\par 9 И не направи никаква разлика между нас и тях, като очисти сърцата им чрез вяра.
\par 10 Сега, прочее, защо изпитвате Бога, та да налагате на шията на учениците хомот, който, нито бащите ни, нито ние можехме да носим?
\par 11 Но вярваме, че ние ще се спасим чрез благодатта на Господа Исуса, също както и те.
\par 12 Тогава цялото множество млъкна и слушаше Варнава и Павла да разказват какви знамения и чудеса Бог беше извършил чрез тях между езичниците.
\par 13 И след като свършиха те да говорят, Яков взе думата и каза: Братя, послушайте мене:
\par 14 Симон обясни по кой начин Бог най-напред посети езичниците, за да вземе измежду тях люде за своето име.
\par 15 С това са съгласни и пророческите думи, както е писано:
\par 16 "След това ще се върна. И пак ще въздигна падналата Давидова скиния, И пак ще издигна развалините й, И ще я изправя;
\par 17 3а да потърсят Господа останалите от човеците, И всичките народи, които се наричат с името ми,
\par 18 Казва Господ, който прави да е известно това от века".
\par 19 Затуй, аз съм на мнение да не отегчаваме тия измежду езичниците, които се обръщат към Бога;
\par 20 но да им пишем да се въздържат от оскверненията чрез идоли, чрез блудство, и чрез яденето удавено и кръв.
\par 21 Защото още от старо време по всичките градове е имало такива, които са проповядвали Мойсеевия закон, който се и прочита всяка събота в синагогите.
\par 22 Тогава апостолите и презвитерите с цялата църква намериха за добре да изберат изпомежду си човеци, и да ги пратят в Антиохия с Павла и Варнава, - А именно: Юда, наречен Варсава, и Сила, водители между братята.
\par 23 И писаха им по тях следното: От апостолите и по-старите братя, поздрав до братята, които са от езичниците в Аитиохия, Сирия, и Киликия.
\par 24 Понеже чухме, че някои, които са излезли от нас, ви смутили с думите си, и извратили душите ви, [като ви казват да се обрязвате и да пазите закона,] без да са приели заповед от нас,
\par 25 то ние, като дойдохме до единодушие, намерихме за добре да изберем мъже и да ги пратим до вас заедно с любимите ни Варнава и Павла,
\par 26 човеци, които изложиха живота си на опасност за името на нашия Господ Исус Христос.
\par 27 И така, изпращаме Юда и Сила, да ви съобщят и те устно същите неща.
\par 28 Защото се видя за добре на Светия Дух и на нас да ви не налагаме никоя друга тегота, освен следните необходими неща:
\par 29 да се въздържате от ядене идоложертвено, кръв, и удавено, тоже и от блудство; от които ако се пазите, добре ще ви бъде. Зравейте.
\par 30 И така, изпратените слязоха в Антиохия, и като събраха всичките вярващи дадоха им посланието.
\par 31 И те, като го прочетоха, зарадваха се за успокоението що им даваше.
\par 32 А Юда и Сила, които бяха и сами пророци, увещаваха братята с много думи и ги утвърдиха.
\par 33 И след като преседяха там няколко време, братята ги оставиха с мир да се върнат при ония, които ги бяха изпратили.
\par 34 [Но Сила видя за добре да поседи още там.]
\par 35 А Павел и Варнава останаха в Антиохия, и, заедно с мнозина други поучаваха и проповядваха Господнето учение.
\par 36 А след няколко дни Павел рече на Варнава: Да се върнем сега по всички градове дето сме проповядвали Господнето учение, и да нагледаме братята, как са.
\par 37 И Варнава беше на мнение да вземат със себе си Йоана, наречен Марко;
\par 38 а Павел не намираше за добре да вземат със себе си тогова, който се бе отделил от тях още от Памфилия, и не отиде с тях на делото.
\par 39 И тъй, възникна разпря (помежду им), така че те се отделиха един от друг; и Варнава взе Марка та отплува за Кипър,
\par 40 а Павел си избра Сила, и тръгна, препоръчан от братята на Господнята благодат.
\par 41 И заминуваше през Сирия и Киликия та утвърждаваше църквите.

\chapter{16}

\par 1 После пристигна и в Дервия и Листра; и, ето, там имаше един ученик на име Тимотей, син на една повярвала еврейка, а чийто баща беше грък.
\par 2 Тоя ученик имаше характер удобрен от братята в Листра и Икония.
\par 3 Него Павел пожела да води със себе си, затова взе та го обряза поради юдеите, които бяха по ония места; понеже всички знаяха, че баща му беше, грък.
\par 4 И като ходеха по градовете, предаваха им наредбите определени от апостолите и презвитерите в Ерусалим, за да ги пазят.
\par 5 Така църквите се утвърдяваха във вярата, и от ден на ден се умножаваха числено.
\par 6 И апостолите преминаха Фригийската и Галатийската земя, като им се забрани от Светия Дух да проповядват словото в Азия;
\par 7 и като дойдоха срещу Мизия, опитаха се да отидат във Витиния, но Исусовия Дух не им допусна.
\par 8 И тъй, като изминаха Мизия слязоха в Троада.
\par 9 И яви се на Павла нощя видение: един македонец стоеше и се молеше, казвайки: Дойди в Македония и помогни ни.
\par 10 И като видя видението, веднага потърсихме случай да отидем в Македония, като дойдохме до заключение, че Бог ни призовава да проповядваме благовестието на тях.
\par 11 И тъй, като отплувахме от Троада, отправихме се право към Самотрак, на утрешния ден в Неапол,
\par 12 и оттам във Филипи, който е главният град на оная Македония, и Римска колония. В тоя град преседяхме няколко дни.
\par 13 А в събота излязохме вън от портата край една река, дето предполагахме, че става молитва; и седнахме та говорихме на събраните там жени.
\par 14 И някоя си богобоязлива жена на име Лидия, от град Тиатир, продавачка на морави платове, слушаше; и Господ отвори сърцето й да внимава на това, което Павел говореше.
\par 15 И като се кръсти тя и домът и, помоли ни, казвайки: Ако ме признавате за вярна Господу, влезте в къщата ми и седете. И принуди ни.
\par 16 И един ден, като отивахме на молитвеното място, срещна ни една мома, която имаше предсказвателен дух и чрез прокобяването си докарваше голяма печалба на господарите си.
\par 17 Тя вървеше подир Павла и нас та викаше, казвайки: Тия човеци са слуги на всевишния Бог, които ви проповядват път за спасение.
\par 18 Това тя правеше много дни наред, А понеже твърде дотегна на Павла, той се обърна и рече на духа: Заповядвам ти в името на Исуса Христа да излезеш от нея. И излезе в същия час.
\par 19 А господарите и като видяха, че излезе и надеждата им за печалба, хванаха Павла и Сила та ги завлякоха на пазаря пред началниците.
\par 20 И като ги изведоха при градските съдии рекоха: Тия човеци са Юдеи и много смущават града ни,
\par 21 като прповядват обичаи, които не е позволено на нас, като Римляни, да приемаме или да пазим.
\par 22 На това, народът купно се подигна против тях, и градските съдии им разкъсаха дрехите и заповядаха да ги бият с тояги.
\par 23 И като ги биха много, хвърлиха ги в тъмница, и заръчаха на тъмничния началник да ги варди здраво;
\par 24 който като получи такава заповед, хвърли ги в по-вътрешната тъмница, и стегна добре нозете им в клада.
\par 25 Но по среднощ, когато Павел и Сила се молеха с химни на Бога, а затворниците ги слушаха,
\par 26 внезапно стана голям трус, така че основите на тъмницата се поклатиха и веднага всички врати се отвориха, и оковите на всичките се развързаха.
\par 27 И началникът, като се събуди и видя тъмничните врати отворени, измъкна ножа си и щеше да се убие, мислейки, че затворниците са избягали.
\par 28 Но Павел извика със силен глас, думайки: Недей струва никакво зло на себе си, защото всички сме тука.
\par 29 Тогава началникът поиска светила, скочи вътре, и разтреперан падна пред Павла и Сила;
\par 30 и изведе ги вън и рече: Господа, що трябва да сторя за да се спася?
\par 31 А те казаха: повярвай в Господа Исуса (Христа), и ще се спасиш, ти и домът ти.
\par 32 И говориха Господнето учение на него и на всички, които бяха в дома му.
\par 33 И той ги взе в същия час през нощта та им изми раните; и без забава се кръсти, той той и всичките негови.
\par 34 И като ги заведе в къщата си, сложи им трапеза; и, повярвал в Бога, зарадва се с целия си дом.
\par 35 А когато се разсъмна, градските съдии пратиха палачите да рекат: Пусни ония човеци.
\par 36 И началникът съобщи думите на Павла, казвайки: Градските съдии са пратили да ви пуснем; сега, прочее, излезте и си идете с мир.
\par 37 Но Павел им рече: Биха ни публично без да сме били осъдени, нас, които сме Римляни, и ни хвърлиха в тъмница; и сега тайно ли ни изваждат? То не става; но те нека дойдат и ни изведат.
\par 38 И палачите съобщиха тия думи на градските съдии; а те като чуха, че били Римляни, уплашиха се;
\par 39 и дойдоха та ги помолиха да бъдат снизходителни, и като ги изведоха поканиха ги да си отидат от града.
\par 40 А те, като излязоха от тъмницата, влязоха у Лидини, и, като видяха братята, увещаха ги, и си заминаха.

\chapter{17}

\par 1 И като минаха през Амфипол и Аполония, пристигнаха в Солун, дето имаше юдейска синагога.
\par 2 И по обичая си Павел влезе при тях, и три съботи наред разискваше с тях от писанията,
\par 3 та им поясняваше и доказваше, че Христос трябваше да пострада и да възкръсне от мъртвите, и че тоя Исус, когото, каза той, аз ви проповядвам, е Христос.
\par 4 и някои от тях се убедиха и дружеха с Павла и Сила, така и голямо множество от набожните гърци и не малко от по-първите жени.
\par 5 Но юдеите, подбудени от завист, взеха си неколцина лоши човеци от мързеливците по пазаря, и като събраха тълпа, размиряваха града; и нападнаха на Ясоновата къща та търсеха Павла и Сила за да ги изведат пред народа.
\par 6 Но като ги не намериха, завлякоха Ясона и някои от братята пред градоначалниците и викаха: Тия, които изопачиха света, дойдоха и тука;
\par 7 и Ясон ги е приел; и те всички действуват против Кесаревите укази, казвайки, че имало друг цар, Исус.
\par 8 И народът и градоначалниците, като чуха това, се смутиха.
\par 9 Но като взеха поръчителство от Ясона и от другите пуснаха ги.
\par 10 А братята незабавно изпратиха Павла и Сила през нощ в Берия; и те като стигнаха там, отидоха в юдейската синагога.
\par 11 И Беряните бяха по-благородни от Солунците, защото приеха учението без всякакъв предразсъдък, и всеки ден изследваха писанията да видят дали това е вярно.
\par 12 И така мнозина от тях повярваха, и от високопоставените гъркини и от мъжете не малко.
\par 13 Но Солунските юдеи, като разбраха, че и в Берия се проповядва от Павла Божието учение, дойдоха и там та подбудиха и смутиха народа.
\par 14 Тогава братята изведнъж изпратиха Павла да отиде към морето; а Сила и Тимотей останаха още там.
\par 15 А ония, които придружаваха Павла, заведоха го до Атина; и като получиха от него заповед до Сила и Тимотея да дойдат колкото се може по-скоро при него, заминаха си.
\par 16 А като ги чакаше Павел в Атина, духът му се възмущаваше дълбоко, като гледаше града пълен с идоли.
\par 17 И тъй разискваше в синагогата с юдеите и с набожните, и по пазаря всеки ден с ония, с които се случеше да среща.
\par 18 Тоже и някои от Епикурейските и Стоическите философи се препираха с него; и едни рекоха: Какво иска да каже тоя празнословец? а други: види се, че е проповедник на чужди богове: защото проповядваше Исуса и възкресението.
\par 19 Прочее, взеха та го заведоха на Ареопага, като казваха: можем ли да знаем какво е това ново учение, което ти проповядваш?
\par 20 Защото внасяш нещо странно в ушите ни; бихме обичали, прочее, да узнаем какво ще е то.
\par 21 (А всичките Атиняни и чужденци, които престояваха там, не си прекарваха времето с нищо друго, освен да разказват или да слушат нещо по-ново).
\par 22 И тъй, Павел застана всред Ареопага и каза: Атиняни, по в с и ч к о гледам, че сте много набожни.
\par 23 Защото, като минавах и разглеждах предметите, на които се кланяте, намерих н един жертвеник, на който бе написано: На непознатия Бог. Онова, прочее, на което се кланяте, без да го знаете, това ви проповядвам.
\par 24 Бог, който е направил света и всичко що е в него, като е Господар на небето и на земята, не обитава в ръкотворни храмове,
\par 25 нито му са потребни служения от човешки ръце, като да би имал нужда от нещо, понеже сам той дава на всички и живот и дишане и всичко;
\par 26 и направил е от една кръв всички човешки народи да живеят по цялото лице на земята, като им е определил предназначени времена и пределите на заселищата им;
\par 27 за да търсят Бога, та дано биха го поне напипали и намерили, ако и да той не е далеч от всеки един от нас;
\par 28 защото в него живеем, движим се, и съществуваме; както и някои от вашите поети са рекли, "Защото дори негов род сме".
\par 29 И тъй, като сме Божий род, не бива да мислим, че Божеството е подобно на злато, или на сребро, или на камък; изработен с човешко изкуство и измишление.
\par 30 Бог, прочее, без да държи бележка за времената на невежеството, сега заповядва на всички човеци навсякъде да се покаят,
\par 31 тъй като е назначил ден, когато ще съди вселената справедливо чрез човека, когото е определил; за което и е дал уверение на всички, като го е възкресил от мъртвите.
\par 32 А като чуха за възкресението на мъртвите, едни се подиграваха, а други рекоха: 3а тоя предмет пак ще те слушаме.
\par 33 И така Павел си излезе измежду тях.
\par 34 А някои мъже се прилепиха при него и повярваха, между които беше и Дионисий Ареопагит, още и една жена на име Дамар, и други с тях.

\chapter{18}

\par 1 Подир това Павел тръгна от Атина и дойде в Коринт,
\par 2 дето намери един Юдеин на име Акила, роден в Понт, и неотдавна пристигнал от Италия с жена си Прискила, защото Клавдий беше заповядал да се махнат всички юдеи от Рим; и Павел дойде при тях.
\par 3 И понеже имаше същото занятие, седеше у тях и работеха; защото занятието им беше да правят шатри.
\par 4 И всяка събота той разискваше в синагогата с юдеи и гърци, и се стараеше да ги убеждава.
\par 5 А когато Сила и Тимотей слязоха от Македония, Павел се притесняваше от своя дух да свидетелствува на юдеите, че Исус е Христос.
\par 6 Но понеже те се противяха и хулеха, той отърси дрехите си и рече: Кръвта ви да бъде на главите ви; аз съм чист от нея; отсега ще отивам между езичниците.
\par 7 И като се премести оттам, дойде в дома на някого си на име Тит Юст, който се кланяше на Бога, и чиято къща беше до синагогата.
\par 8 А Крисп, началникът на синагогата, повярва в Господа с целия си дом; и мнозина от Коринтяните, като слушаха, вярваха и се кръщаваха.
\par 9 И Господ каза на Павла нощя във видение: Не не бой се, но говори и не млъквай;
\par 10 защото аз съм с тебе, и никой няма да те нападне та да ти стори зло; защото имам много люде в тоя град.
\par 11 И той преседя там година и шест месеца та ги поучаваше в Божието слово.
\par 12 А когато Галион беше управител в Ахаия, юдеите се подигнаха единодушно против Павла, доведоха го пред съдилището, и казаха:
\par 13 Тоя убеждава човеците да се кланят на Бога несъгласно закона.
\par 14 Но когато Павел щеше да отвори уста, Галион рече на юдеите: Ако беше ако беше въпрос за някоя неправда или грозно злодеяние, о юдеи, разбира се би трябвало да ви търпя;
\par 15 но ако въпросите са за учение, за имена и за вашия закон, гледайте си сами; аз не ща да съм съдия на такива работи.
\par 16 И изпъди ги от съдилището.
\par 17 Тогава те всички хванаха началника на Синагогата Состена, та го биха пред съдилището; но Галион, не искаше и да знае за това.
\par 18 А Павел, след като поседя там още доволно време, прости се с братята си, и отплува за Сирия (и с него Прискила и Акила), като си острига главата в Кенхрея, защото имаше обрек.
\par 19 Като стигнаха Ефес, той ги остави там, а сам влезе в Синагогата и разискваше с юдеите.
\par 20 А когато го замолиха да поседи повечко време, той не склони;
\par 21 но се прости с тях, казвайки; Ако ще Бог, пак ще се върна при вас. И отплува от Ефес.
\par 22 И като слезе в Кесария, възлезе в Ерусалим та поздрави църквата и после слезе в Антиохия.
\par 23 И като поседя там няколко време, излезе и обикаляше наред Галатийската и Фригийската страна та утвърждаваше всичките ученици.
\par 24 И някой си юдеин на име Аполос, роден в Александрия, човек учен и силен в писанията, дойде в Ефес.
\par 25 Той беше наставен в Господния път, и, бидейки по дух усърден, говореше и поучаваше прилежно за Исуса то а познаваше само Йоановото кръщение,
\par 26 Той почна да говори дързостно в синагогата; но Прискила и Акила, като го чуха, прибраха го и му изложиха по-точно Божия път.
\par 27 И когато се канеше да замине за Ахаия, братята го насърчиха, и писаха до учениците да го приемат; и той като дойде помогна много на повярвалите чрез благодатта;
\par 28 защото силно опровергаваше юдеите, и то публично, като доказваше от писанието, че Исус е Христос.

\chapter{19}

\par 1 А когато Аполос беше в Коринт, Павел, след като беше минал през горните страни, дойде в Ефес, дето намери някои ученици.
\par 2 И рече им: Приехте ли Светия Дух като повярвахте? А те му отговориха: Даже не сме чули дали има Свети Дух.
\par 3 И рече: А в що се кръстихте? А те рекоха: В Йоановото кръщение.
\par 4 А Павел рече: Йоан е кръщавал с кръщението на покаяние, като е казвал на людете да вярват в тогова, който щеше да дойде подир него, сиреч в Исуса.
\par 5 И като чуха това, кръстиха се в името на Господа Исуса.
\par 6 И като положи Павел ръце на тях, Светия Дух дойде на тях; и говореха други езици и пророкуваха.
\par 7 И те всички бяха около дванадесет мъже.
\par 8 И той влезе в синагогата, дето говореше дързостно; и в разстояние на три месеца разискваше с людете и ги увещаваше за някои неща отнасящи се до Божието царство.
\par 9 А понеже някои се закоравяваха и не вярваха, но злословеха учението пред народа, той се оттегли от тях и отдели учениците та разискваше всеки ден в училището на Тирана.
\par 10 И това се продължава две години, така щото всички, които живееха в Азия, и юдеи и гърци, чуха Господнето учение.
\par 11 При това, Бог вършеше особено велики дела чрез ръцете на Павла;
\par 12 дотолкоз щото, когато носеха по болните кърпи или престилки от неговото тяло, болестите се отмахваха от тях, и злите духове излизаха.
\par 13 А някои от юдейските скитници заклинатели предприеха да произнасят името на Господа Исуса над тия, които имаха зли духове, казвайки: Заклевам ви в Исуса, когото Павел проповядва.
\par 14 И между тия, които вършеха това, бяха седемте синове на някой си юдеин Скева, главен свещеник.
\par 15 Но веднъж злият дух в отговор им рече: Исуса признавам, и Павла зная; но вие кои сте?
\par 16 И човекът, в когото беше злия дух, скочи върху тях, и, като надви на двамата, превъзмогна над тях, така щото голи и ранени избягаха от оная къща.
\par 17 И това стана известно на всички Ефески жители, и юдеи и гърци; и страх обзе всички тях, и името на Господа Исуса се възвеличаваше.
\par 18 И мнозина от повярвалите дохождаха та се изповядваха и изказваха делата си.
\par 19 Мнозина още и от тия, които правеха магии, донасяха книгите си и ги изгаряха пред всичките; и като пресметнаха цената им, намериха, че бе петдесет хиляди сребърници.
\par 20 Така силно растеше и преодоляваше Господнето учение.
\par 21 И като свърши това, Павел чрез Духа стори намерение да отиде в Ерусалим след като обиколи Македония и Ахаия, казвайки: като постоя там, трябва да видя и Рим.
\par 22 И прати в Македония двама от тия, които му помагаха, Тимотея и Ераста; а той остана за още няколко време в Азия.
\par 23 А по онова време се подигна голямо размирие относно Господния път.
\par 24 Защото един златар на име Димитър, който правеше сребърни храмчета на Даниното капище, и докарваше не малко печалба на занаятчиите,
\par 25 като събра и тях и ония, които работеха подобни неща, рече: О мъже, вие знаете, че от тая работа иде нашето богатство.
\par 26 И вие виждате и чуете, че не само в Ефес, но почти в цяла Азия, тоя Павел е придумал и обърнал големи множества, казвайки, че не са богове тия, които са с ръка направени.
\par 27 И има опасност не само това наше занятия да изпадне в презрение, но и капището на великата богиня Диана да счита за нищо, и даже да се свали от величието си оная, на която цяла Азия и вселената се кланя.
\par 28 Като чуха това, те се изпълниха с гняв та викаха, казвайки: Велика, е Ефеската Диана!
\par 29 И смущението се разпростря по града; и като уловиха македонците Гаия и Аристарха, Павловите спътници, единодушно се спуснаха в театъра.
\par 30 А когато Павел искаше да влезе между народа, учениците не го пуснаха.
\par 31 Така някой от Азийските началници, понеже му бяха приятели, пратиха до него да го помолят да се не показва в театъра.
\par 32 И тъй, едни викаха едно, а други друго; защото навалицата беше разбъркана и знаеше защо се бяха стекли.
\par 33 А някои се бяха стекли от народа изкараха Александра да говори, понеже юдеите посочиха него; и Александър помаха с ръка и щеше да даде обяснение пред народа.
\par 34 Но като го познаха, че е юдеин, всички едногласно викаха за около два часа: Велика е Ефеската Диана!
\par 35 Тогава градският писар, като въдвори тишина между народа, каза: Ефесяни, кой е оня човек, който не знае, че град Ефес е пазач на капището на великата Диана и на падналия от Юпитера идол?
\par 36 И тъй, понеже това е неоспоримо, вие трябва да мирувате и да не правите нищо несмислено.
\par 37 Защото сте довели тук тия човеци, които нито са светотатци, нито хулят нашата богиня.
\par 38 Прочее, ако Димитър и занаятчиите, които са с него, имат спор с някого. съдилищата заседават, има и съдийски чиновници, нека се съдят едни други.
\par 39 Но ако търсите нещо друго, то ще се реши в редовното събрание.
\par 40 Защото има опасност да ни обвинят поради днешното размирие, понеже няма никаква причина за него; и колкото за това, ние не ще можем да оправдаем това стичане.
\par 41 И като рече това, разпусна събранието.

\chapter{20}

\par 1 След утихването на мълвата Павел повика учениците и, като ги увеща, прости се с тях и тръгна да отиде в Македония.
\par 2 И като мина през ония места та увеща учениците с много думи, дойде в Гърция.
\par 3 И като преседя там три месеца, понеже юдеите направиха заговор против него във времето на тръгването му за Сирия, той реши да се върне през Македония,
\par 4 И придружиха го до Азия Берянина Сосипатър Пиров, и от Солунците Аристарх и Секунд; още и Гаий от Дервия и Тимотей, а от Азия Тихик и Трофим.
\par 5 А тия бяха отишли по-напред, та ни чакаха в Троада;
\par 6 и ние отплувахме от Филипи подир дните на безквасните хлябове, и за пет дена дойдохме при тях в Троада, дето преседяхме седем дена.
\par 7 И в първия ден на седмицата, когато бяхме събрани за разчупването на хляба, Павел беседваше с тях понеже щеше да отпътува на сутринта; и продължи словото си до среднощ.
\par 8 И имаше много светила в горната стая, дето бяхме събрани.
\par 9 И едно момче, на име Ефтих, което седеше на прозореца, беше заспало дълбоко, и когато Павел беседваше още по-надълго, бидейки обладано от сън, падна долу от третия етаж; и дигнаха го мъртво.
\par 10 Но Павел слезе и, като падна на него, прегърна го, и рече: Не се безпокойте, защото животът му е в него.
\par 11 След това той се качи горе, разчупи хляба та похапна, и приказва пак надълго до зори, и така тръгна.
\par 12 А момчето доведоха живо, и не малко се утешиха.
\par 13 А ние тръгнахме по-напред за кораба и отплувахме за Асон, дето щяхме да приберем Павла; понеже така беше поръчал, като щеше да отиде пеш.
\par 14 И когато се събра с нас в Асон, прибрахме го и дойдохме в Митилин.
\par 15 И оттам като отплувахме, на утрешния ден дойдохме срещу Хиос, а на другия стигнахме в Самос; и [като преседяхме в Трогилия] на следващия ден дойдохме в Милит.
\par 16 Защото Павел бе решил да отмине Ефес, за да не се бави в Азия, понеже бързаше, ако му беше възможно, да се намери в Ерусалим за деня на Петдесятницата.
\par 17 А от Милит прати в Ефес да повикат църковните презвитери.
\par 18 И като дойдоха при него, рече им: Вие знаете по какъв начин, още от първия ден когато стъпих в Азия, прекарах всичкото време между вас
\par 19 в служене на Господа с пълно смиреномъдрие, със сълзи, и с напасти, които ме сполетяха от заговорите на юдеите;
\par 20 как не се посвених да ви изява всичко що е било полезно, и да ви поучавам и публично и къщите,
\par 21 като проповядвах и на юдеи и на гърци покаяние спрямо Бога и вяра спрямо нашия Господ Исус Христос.
\par 22 И сега, ето, аз заставен духом, отивам в Ерусалим, без да зная какво ще ме сполети там,
\par 23 освен че Светият Дух ми свидетелствува във всеки град, казвайки, че вързвания и скърби ме очакват.
\par 24 Но не се скъпя за живота си, като че ми се свиди за него, в сравнение с това, да изкарам пътя си и служенето, което приех от Господа Исуса, да проповядвам благовестието на Божията благодат.
\par 25 И сега, ето, аз зная, че ни един от вас, между които минах та проповядвах Божието царство, няма вече да види лицето ми.
\par 26 Затова, свидетелствувам ви в тоя ден, че аз съм чист от кръвта на всички;
\par 27 защото не се посвених да ви изява всичката Божия воля.
\par 28 Внимавайте на себе си и на цялото стадо, в което Светият Дух ви е поставил епископи, да пасете църквата на Бога, която той придоби със собствената си кръв.
\par 29 Аз зная, че подир моето заминаване ще навлязат между вас свирепи вълци, които няма да жалят стадото;
\par 30 и от самите вас ще се издигнат човеци, които ще говорят извратено, та ще отвличат учениците след себе си.
\par 31 Затова, бдете, и помнете, че за три години, дене и нощем, не престанах да поучавам със сълзи всеки един от вас.
\par 32 И сега препоръчвам ви на Бога и на словото на неговата благодат, което може да ви назидава и да ви даде наследството между всичките осветени.
\par 33 Никому среброто или златото или облеклото не съм пожелал.
\par 34 Вие сами знаете, че тия мои ръце послужиха за моите нужди и за нуждите на ония, които бяха с мене.
\par 35 Във всичко ви показах, че така трудещи се трябва да помагате на немощните, и да помните думите на Господа Исуса, как той е казал, По-блажено е да дава човек отколкото да приема.
\par 36 Като изговори това, коленичи и се помоли с всички тях.
\par 37 И всички плакаха много; и паднаха на шията на Павла и го целуваха.
\par 38 наскърбени най-много за думата, която каза, че няма вече да видят лицето му. И го изпратиха до кораба.

\chapter{21}

\par 1 Като се разделихме от тях и отплувахме, дойдохме право на Кос, а на утрешния ден на Родос, и оттам на Патара.
\par 2 И като намерихме кораб, който заминаваше за Финикия, качихме се на него и отплувахме.
\par 3 И когато Кипър се показа, оставихме го отляво, плувахме към Сирия, и слязохме в Тир; защото там щеше кораба да се разтовари.
\par 4 И като издирихме учениците, преседяхме там седем дена; и те чрез Духа казваха на Павла да не стъпва в Ерусалим.
\par 5 И когато прекарахме тия дни, излязохме и отивахме си; и те всичките с жените и децата си, ни изпратиха до отвън града; и, коленичили на брега, помолихме се.
\par 6 И като се простихме един с друг, ние се качихме на кораба, а те се върнаха у дома си.
\par 7 И ние, като отплувахме от Тир, стигнахме в Птолемаида, дето поздравихме братята и преседяхме у тях един ден.
\par 8 А на утрешния ден тръгнахме и стигнахме в Кесария; и влязохме в къщата на благовестителя Филипа, който бе един от Седмината дякони и останахме у него.
\par 9 А той имаше четири дъщери девици, които пророкуваха.
\par 10 И след като бяхме преседяли там много дни, един пророк на име Агав слезе от Юдея.
\par 11 И като дойде при нас, взе Павловия пояс та си върза нозете и ръцете, и рече: Ето що казва Светия Дух, Така юдеите в Ерусалим ще вържат човека чийто е тоя пояс, и ще го предадат в ръцете на езичниците.
\par 12 И като чухме това, и ние и тамошните го молихме да не възлиза в Ерусалим.
\par 13 Тогава Павел отговори: Що правите вие, като плачете та ми съкрушавате сърцето? защото аз съм готов не само да бъда вързан, но и да умра в Ерусалим, за името на Господа Исуса.
\par 14 И понеже той беше неумолим, ние млъкнахме и рекохме: да бъде Господнята воля.
\par 15 И след тия дни приготвихме се за път и възлязохме в Ерусалим.
\par 16 С нас дойдоха и някои от учениците в Кесария, и ни водеха при някого си Мнасона, Кипрянин, отдавнашен ученик, у когото щяхме да бъдем гости.
\par 17 И като стигнахме в Ерусалим, братята ни приеха с радост.
\par 18 И на утрешния ден Павел влезе с нас при Якова, дето присъствуваха всичките презвитери.
\par 19 И като ги поздрави, разказа им едно по едно всичко що Бог беше извършил между езичниците чрез неговото служение.
\par 20 А те, като чуха, прославиха Бога. Тогава му рекоха: Ти виждаш, брате, колко десетки хиляди повярвали юдеи има, и те всички ревностно поддържат закона.
\par 21 А за тебе са уведомени, че ти си бил учил всичките юдеи, които са между езичниците, да отстъпят от Мойсеевия закон, като им казваш да не обрязват чадата си, нито да държат старите обреди.
\par 22 И тъй, какво да се направи? [Без друго ще се събере тълпа, защото] те непременно ще чуят, че си дошъл.
\par 23 Затова направи каквото ти кажем. Между нас има четирима мъже, които имат обрек;
\par 24 земи ги, извърши очищението си заедно с тях, и иждиви за тях за да обръснат главите си; и така всички ще знаят, че не е истинна това, което са чули за тебе, но че и ти постъпваш порядъчно и пазиш закона.
\par 25 А колкото за повярвалите езичници, ние писахме решението си да се вардят от ядене идоложертвено, кръв, удавено, тоже и от блудство.
\par 26 Тогаз Павел взе мъжете; и на утрешния ден, като извърши очищението заедно с тях, влезе в храма и обяви кога щяха да свършат дните, определени за очищението, когато щеше да се принесе приноса за всеки един от тях.
\par 27 И когато седемте дена бяха на свършване, юдеите от Азия, като го видяха в храма, възбудиха целия народ, туриха ръце на него, и викаха:
\par 28 О, Израиляни, помагайте! Това е човекът, който на всякъде учи всичките против народа ни, против закона, и против това място; а освен това въведе и гърци в храма, и оскверни това свето място.
\par 29 (Защото преди това бяха видели с него в града Ефесянина Трофим и мислеха, че Павел го е въвел в храма.)
\par 30 И целият град се развълнува и людете се стекоха; и като уловиха Павла, извлякоха го вън от храма; и веднага се затвориха вратите.
\par 31 И когато щяха да го убият, стигна известие до хилядника на полка, че целият Ерусалим е размирен;
\par 32 и той завчас взе войници и стотници та се завтече долу върху тях. А те, като видяха хилядника и войниците, престанаха да бият Павла.
\par 33 Тогава хилядникът се приближи та го хвана и заповяда да го оковат с две вериги, и разпитваше, кой е той и що е сторил.
\par 34 А между навалицата едни викаха едно а други друго; и понеже не можеше да разбере същността на работата поради смущението, заповяда да го закарат в крепостта.
\par 35 А като стигна до стъпалата, войниците го дигнаха и носеха поради насилието на навалицата;
\par 36 защото всичките люде вървяха подире и викаха: Махни го от света!
\par 37 И когато щяха да въведат Павла в крепостта, той каза на хилядника: Позволено ли ми е да ти кажа нещо? А той рече: Знаеш ли гръцки!
\par 38 Не си ли тогава оня египтянин, който преди няколко време размири и изведе в пустинята четирите хиляди мъже убийци?
\par 39 А Павел рече: Аз съм юдеин от Тарс Киликийски, гражданин на тоя знаменит град; и ти се моля да ми позволиш да поговоря на людете.
\par 40 И като му позволи, Павел застана на стъпалата и помаха с ръка на людете; а като се въдвори голяма тишина, почна да им говори на еврейски, казвайки: -

\chapter{22}

\par 1 Братя и бащи, слушайте сега моята защита пред вас.
\par 2 (И като чуха, че им говори на еврейски, те пазеха още по-голяма тишина; и той каза):
\par 3 Аз съм юдеин, роден в Тарс Киликийски, а възпитан в тоя град при Гамалииловите нозе, изучен строго в предадения от бащите ни закон. И бях ревностен за Бога, както сте и всички вие днес,
\par 4 и гонех смърт последователите на тоя път, като връзвах и предавах на затвор и мъже и жени;
\par 5 както свидетелствува за мене и първосвещеникът и цялото старейшинство, от които бях взел и писма до братята евреи в Дамаск, дето отивах да закарам вързани в Ерусалим и ония, които бяха там, за да ги накажат.
\par 6 И когато вървях и приближих Дамаск, къде пладне, внезапно блесна от небето голяма светлина около мене.
\par 7 И паднах на земята и чух глас, който ми каза: Савле, Савле, защо ме гониш?
\par 8 А аз отговорих, Кой си ти, Господи? И рече ми, Аз съм Исус Назарянин, когото ти гониш.
\par 9 А другарите ми видяха светлината, но не чуха гласа на тогова, който ми говореше.
\par 10 И рекох, Какво да сторя Господи? И Господ ми реше, Стани, иди в Дамаск, и там ще ти се каже за всичко що ти е определено да сториш.
\par 11 И понеже от блясъка на оная светлина изгубих зрението си, другарите ми ме поведоха за ръка, и така влязох в Дамаск.
\par 12 И някой си Анания, човек благочестив по закона, удобрен от всички там, живеещи юдеи,
\par 13 дойде при мене, и като застана и се наведе над мене, рече ми, Брате Савле, прогледай. И аз начаса получих зрението си и прогледнах на него.
\par 14 А той рече, Бог на бащите ни те е предназначил да познаеш неговата воля, да видиш праведника, и да чуеш глас от неговите уста;
\par 15 защото ще бъдеш свидетел за него пред всичките човеци за това, което си видял и чул.
\par 16 И сега, защо се бавиш? Стани, кръсти се и се омий от греховете си, и призови неговото име.
\par 17 И като се върнах в Ерусалим, когато се молех в храма, дойдох в изстъпление,
\par 18 и видях го да ми казва: Побързай да излезеш скоро из Ерусалим; защото няма да приемат твоето свидетелство за мене.
\par 19 И аз рекох, Господи, те знаят, че аз затварях и биех по синагогите ония, които вярваха в тебе;
\par 20 и когато се проливаше кръвта на твоя мъченик Стефана, и аз бях там и одобрявах, като вардех дрехите на тия, които го убиваха.
\par 21 Но той ми рече, Иди, защото ще те пратя далеч между езичниците.
\par 22 До тая дума го слушаха; а тогаз извикаха със силен глас, казвайки: Да се махне такъв от земята! защото не е достоен да живее.
\par 23 И понеже те викаха, мятаха дрехите си, и хвърляха прах по въздуха,
\par 24 хилядникът заповяда да го закарат в крепостта, и заръча да го изпитат с биене, за да узнае, по коя причина викат така против него.
\par 25 И когато го бяха разтегнали с ремъци, Павел рече на стотника, който стоеше там: Законно ли е вам да бичувате един римлянин, и то неосъден?
\par 26 Като чу това, стотникът отиде та извести на хилядника, казвайки: Какво правиш? защото тоя човек е римлянин.
\par 27 Тогава хилядникът се приближи и му рече: Кажи ми, римлянин ли си ти? А той каза: римлянин.
\par 28 Хилядникът отговори: С много пари съм добил това гражданство. А Павел рече: Но аз съм се и родил в него.
\par 29 Тогава веднага се оттеглиха от него тия, които щяха да го изпитват. А хилядникът се уплаши като разбра, че е римлянин, понеже го беше вързал.
\par 30 На утринта, като искаше да разбере същинската причина, по която юдеите го обвиняваха, той го развърза, заповяда да се съберат първосвещениците и целият им синедрион, и доведе долу Павла та го постави пред тях.

\chapter{23}

\par 1 И Павел, като се вгледа в синедриона рече: Братя, до тоя ден съм живял пред Бога със съвършено чиста съвест.
\par 2 А първосвещеникът Анания заповяда на стоящите до него да го ударят по устата.
\par 3 Тогава Павел му рече: Бог ще удари тебе, стено варосана; и ти си седнал да ме съдиш по закона, а против закона заповядваш да ме ударят!
\par 4 А стоящите наоколо рекоха: Божия първосвещеник ли хулиш?
\par 5 И Павел рече: Не знаех, братя, че той е първосвещеник, защото е писано: "Да не злословиш началника на рода си."
\par 6 А когато Павел позна, че едната част са садукеи, а другите фарисеи, извика в синедриона: Братя, аз съм фарисей, син на фарисей; съдят ме поради надеждата и учението за възкресението на мъртвите.
\par 7 И когато рече това, възникна разпря между фарисеите и садукеите; И събранието се раздели.
\par 8 Защото садукеите казват, че няма възкресение, нито ангел, нито дух; а фарисеите признават и двете.
\par 9 И така възникна голяма глъчка; и някои книжници от фарисейската страна станаха та се препираха казвайки: Никакво зло не намираме у тоя човек; и какво да направим ако му е говорил дух или ангел?
\par 10 И понеже разпрята стана голяма, хилядникът, боейки се да не би да разкъсат Павла, заповяда на войниците да слязат и да го грабнат изпомежду им, и да го заведат в крепостта.
\par 11 И през следващата нощ Господ застана до него и рече: Дерзай, защото както си свидетелствувал за мене в Ерусалим, така трябва да свидетелствуваш и в Рим.
\par 12 И като се разсъмна, юдеите направиха заговор, влязоха под проклетия, и рекоха, че няма да ядат нито да пият додето не убият Павла.
\par 13 Тия, които направиха тоя заговор, бяха повече от четиридесет души.
\par 14 Те дойдоха при първосвещениците и старейшините и рекоха; влязохме под проклетия, да не вкусим нищо докле не убием Павла.
\par 15 Сега, прочее, вие със синедриона заявете на хилядника да го доведе долу при вас, уж че искате да изучите по-точно неговото дело; а ние, преди да се приближи той, сме готови да го убием.
\par 16 Но Павловият сестрин син, като ги завари, чу заговора и влезе в крепостта та обади на Павла.
\par 17 Тогава Павел повика един от стотниците и му рече: Заведи това момче при хилядника, защото има да му обади нещо.
\par 18 И той, го взе, заведе го при хилядника, и каза: Запреният Павел ме повика и ми се помоли да доведа това момче при тебе, защото имало да ти каже нещо.
\par 19 (А хилядникът го взе за ръка, и като се оттегли настрана, попита го насаме; Какво да ми обадиш?
\par 20 А той рече: юдеите се нагласиха да те замолят да заведеш Павла утре долу в Синедриона, като че ли искаш да разпиташ по-точно за него.
\par 21 Но ти недей ги слуша; защото го причакват повече от четиридесет души от тях, които влязоха под проклетия; задължавайки се да не ядат нито да пият додето го не убият. Те още сега са готови, и чакат само да им се обещаеш.
\par 22 И тъй хилядникът остави момчето да си отиде, като му заръча: никому да не кажеш, че ми си обадил това.
\par 23 Тогава повика двама от стотниците та им рече: Пригответе двеста пехотинци, седемдесет конници, и двеста копиеносци да заминат за Кесария на третия час през нощта.
\par 24 Пригответе и добитък, на който да възкачат Павла, и да го отведат безопасно до управителя Феликса.
\par 25 Той написа и писмо, което имаше следното съдържание:
\par 26 От Клавдия Лисия до негово превъзходителство управителя Феликса, поздрав.
\par 27 Тоя човек биде уловен от юдеите, които щяха да го убият; но аз пристигнах с войниците та го избавих, понеже се научих, че бил римлянин.
\par 28 И като поисках да разбера причината, по която го обвиняваха, заведох го долу в синедриона им;
\par 29 и намерих, че го обвиняват за въпроси от техния закон; нямаше обаче никакво обвинение в нещо достойно за смърт или окови.
\par 30 И понеже ми се подсказа, че щяло да има заговор против човека, веднага го изпратих при тебе, като заръчах и на обвинителите му да се изкажат пред тебе против него. [Остани със здраве].
\par 31 И тъй, войниците, според дадената им заповед, взеха Павла и го заведоха през нощта в Антипатрида.
\par 32 И на утринта оставиха конниците да отидат с него, а те се върнаха в крепостта.
\par 33 А конниците, като влязоха в Кесария и връчиха писмото на управителя, представиха му и Павла.
\par 34 А като го прочете, попита го от коя област е; и като разбра, че е от Киликия, рече:
\par 35 Ще те изслушам когато дойдат и обвинителите ти. И заповяда да го вардят в Иродовата претория.

\chapter{24}

\par 1 След пет дена първосвещеникът Анания слезе с някои старейшини и с един ритор на име Тертил, които подадоха на управителя жалба против Павла.
\par 2 И като го повикаха, Тертил почна да го обвинява, като казваше: Понеже чрез тебе, честити Феликсе, се радваме на голямо спокойствие, и понеже чрез твоята предвидливост се поправят злини в тоя наш народ,
\par 3 то ние с пълна благодарност по всякакъв начин и всякъде посрещаме това.
\par 4 Но за да те не отегчавам повече, моля те да имаш снизхождение и ни изслушаш накратко,
\par 5 понеже намерихме, че тоя човек е заразител и размирник между всичките юдеи по вселената, още и водач на Назарейската ерес;
\par 6 който се опита и храма да оскверни; но ние го уловихме, [и поискахме да го съдим по нашия закон;
\par 7 но хилядникът Лисий дойде и с голямо насилство го изтръгна от ръцете ни, и заповяда на обвинителите му, да дойдат при тебе].
\par 8 А ти, като сам го изпиташ, ще можеш да узнаеш от него всичко това, за което го обвиняваме.
\par 9 И юдеите потвърдиха, казвайки, че това е вярно.
\par 10 А когато управителят кимна на Павла да вземе думата, той отговори: Понеже зная, че от много години ти си бил съдия на тоя народ, аз на радо сърце говоря в своя защита;
\par 11 защото можеш да се научиш, че няма повече от дванадесет дена откак възлязох на поклонение в Ерусалим.
\par 12 И не са ме намирали нито в храма, нито в синагогите, нито в града, да се препирам с някого или да размирявам народа.
\par 13 И те не могат да докажат пред тебе това, за което ме обвиняват сега.
\par 14 Но това ти изповядвам че, според учението което те наричат ерес, така служа на бащиния ни Бог, като вярвам8всичко що е по закона и е писано в пророците,
\par 15 и че се надявам на Бога, че ще има възкресение на праведни и неправедни, което и те сами приемат.
\par 16 Затова и аз се старая да имам всякога непорочна съвест и спрямо Бога и спрямо човеците.
\par 17 А след изтичането на много години, дойдох да донеса милостини, на народа си и приноси.
\par 18 А когато ги принасях те ме намериха в храма очищен, без да има навалица или размирие;
\par 19 но имаше някои юдеи от Азия, който трябваше да се представят пред тебе и да ме обвинят, ако имаха нещо против мене.
\par 20 Или тия сами нека кажат каква неправда са намерили в мене когато застанах пред синедриона,
\par 21 освен ако е само в тоя вик, който издадох като стоях между тях, Поради учението за възкресението на мъртвите ме съдите днес.
\par 22 А Феликс, като познаваше доста добре това учение, ги отложи, казвайки: Когато слезе хилядникът Лисий ще разреша делото ви.
\par 23 И заповяда на стотника да вардят Павла, но да му дават известна свобода, и да не възпират никого от приятелите му да му прислужва.
\par 24 След няколко дни Феликс дойде с жена си Друсилия, която беше юдейка, и прати да повикат Павла, от когото слуша за вярата в Христа Исуса.
\par 25 И когато той говореше за правда, за себеобуздание, и за бъдещия съд, Феликс уплашен отговори: 3а сега си иди; и когато намеря време, ще те повикам.
\par 26 Между това, той се надяваше, че ще получи пари от Павла; затова и по-честичко го викаше та приказваше с него.
\par 27 Но като се навършиха две години, Феликс биде заместен от Порций Фест, а понеже искаше да спечели благоволението на юдеите, Феликс остави Павла в окови.

\chapter{25}

\par 1 А Фест, като зае областта си, подир три дни възлезе от Кесария в Ерусалим.
\par 2 Тогава първосвещениците и юдейските първенци му подадоха жалба против Павла,
\par 3 и молейки му се искаха да склони на това против него, - да изпрати да го доведат в Ерусалим; като крояха да поставят засада и го убият на пътя.
\par 4 Фест обаче отговори, че Павел вече се пази под стража в Кесария, и че сам той скоро щеше да тръгне за там;
\par 5 затова, рече той, влиятелните между вас нека слязат с мене; а ако има нещо криво в човека, нека го обвинят.
\par 6 И като преседя между тях не повече от осем или десет дена, той слезе в Кесария, и на утрешния ден седна на съдийския стол и заповяда да доведат Павла.
\par 7 И като дойде, юдеите, които бяха слезли от Ерусалим, го обиколиха и обвиняваха го с много и тежки обвинения, които не можеха да докажат;
\par 8 но Павел в защитата си казваше: Нито против юдейския закон, нито против храма, нито против Кесаря съм извършил някакво престъпление.
\par 9 Но Фест, понеже искаше да спечели благоволението, на юдеите, в отговор на Павла каза: Щеш ли да възлезеш в Ерусалим, и там да се съдиш за това пред мене?
\par 10 А Павел каза: Аз стоя пред Кесаревото съдилище, дето трябва да бъда съден. На юдеите не съм сторил никаква вреда, както и ти твърде добре знаеш.
\par 11 Прочее, ако съм злодеец, и съм сторил нещо достойно за смърт, не бягам от смъртта; но ако ни едно от нещата, за които ме обвиняват тия не е истинно, никой не може да ме предаде за да им угоди. Отнасям се до Кесаря.
\par 12 Тогава Фест, след като поразиска въпроса със съвета, отговори: Отнесъл си се до Кесаря; при Кесаря ще отидеш.
\par 13 А като изминаха няколко дни, цар Агрипа и Верникия дойдоха в Кесария да поздравят Феста.
\par 14 И като се бавиха там доста време, Фест доложи Павловото дело пред царя, казвайки: Има един човек оставен от Фелиска в окови.
\par 15 3а него когато бях в Ерусалим, първосвещениците и юдейските старейшини ми подадоха жалба и искаха да го съдя.
\par 16 Но им отговорих, че римляните нямат обичай да предават някой човек [на смърт,] преди обвиняемият да е бил поставен лице с лице с обвинителите си, и да му е дал случай да говори в своя защита относно обвинението.
\par 17 И тъй, когато дойдоха тук заедно с мене, на следния ден незабавно седнах на съдийския стол и заповядах да доведат човека.
\par 18 Но когато обвинителите му застанаха, не го обвиниха в никое от лошите дела каквито аз предполагах;
\par 19 но имаха против него някакви разисквания за техните си вярвания, и за някой си Исус, който бил умрял, за когото Павел твърдеше, че е жив.
\par 20 И аз, понеже бях в недоумение как да изпитам за такива неща, попитах да ли би отишъл в Ерусалим, там да се съди за това.
\par 21 Но понеже Павел се отнесе до решението на Августа, за да се опази за него, заповядах да го пазят докле го изпратя при Кесаря.
\par 22 Тогава Априпа рече на Феста: Искаше ми се и мене да чуя тоя човек. И той каза: Утре ще го чуеш.
\par 23 На утрешния ден, когато Агрипа и Верникия дойдоха с голям блясък и влязоха в съдебната стая с хилядниците и по-видните граждани, Фест заповяда та доведоха Павла.
\par 24 Тогава Фест каза: Царю Агрипо, и всички, които присъствувате с нас, ето човека, за когото целият юдейски народ ми представиха жалба, и в Ерусалим и тука, като крещяха, че той не трябва вече да живее.
\par 25 Но аз намерих, че не е сторил нищо достойно за смърт; и понеже той сам се отнесе до Августа, реших да го изпратя.
\par 26 А за него нямам нищо положително да пиша на господаря си; затова го изведох пред вас, и особено пред тебе, царю Агрипо, та, като стане разпитването му, да имам какво да пиша.
\par 27 Защото ми се вижда неразумно, като изпращам човек вързан, да не изложа и обвиненията против него.

\chapter{26}

\par 1 Тогава Агрипа рече на Павла: Позволява ти се да говориш за себе си. И така, Павел простря ръка и почна да говори в своя защита:
\par 2 Честит се считам, царю Агрипо, задето пред тебе ще се защитя днес относно всичко, за което ме обвиняват юдеите,
\par 3 а най-вече защото си вещ във всичките обреди и разисквания между юдеите; затова ти се моля да ме изслушаш с търпение.
\par 4 Какъв, прочее, беше Моя живот още от младини, това всичките юдеи знаят, понеже се прекара отначало между народа ми в Ерусалим.
\par 5 Защото ме познават от начало, (ако искаха да засвидетелствуват), че според най-строгото учение на нашето вероизповедание живях фарисей,
\par 6 И сега стоя пред съда понеже имам надежда на обещанието, което Бог е дал на бащите ни,
\par 7 до изпълнението на, което нашите дванадесет племена се надяват да достигнат, като служат на Бога усърдно нощя и деня. 3а тая надежда, царю [Агрипо], ме обвиняват юдеите!
\par 8 Защото да се мисли между вас нещо не за вярване, че Бог възкресява мъртвите?
\par 9 И аз си мислех, че трябваше да върша много неща против името на Исуса Назарянина;
\par 10 което и върших в Ерусалим, понеже затварях в тъмница мнозина от светиите, като се снабдих с власт от първосвещениците, и за убиването им давах глас против тях.
\par 11 И като ги мъчех много пъти във всичките синагоги стараех се да ги накарам да хулят; и в чрезмерната си ярост против тях гонех ги даже и по чуждите градове.
\par 12 По която работа когато пътувах за Дамаск с власт и поръка о т първосвещениците,
\par 13 по пладне, царю, видях на пътя светлина от небето, която надминаваше слънчевия блясък, и осия мене и тия, които пътуваха с мене,
\par 14 И като паднахме всинца на земята, чух глас, който ми казваше на еврейски, Савле, Савле, защо ме гониш? Мъчно ти е да риташ срещу остен.
\par 15 И аз рекох, Кой си ти, Господи? А Господ рече, Аз съм Исус, когото ти гониш.
\par 16 Но стани и се изправи на нозете си; понеже за туй ти се явих, да те назнача служител и свидетел на това, че си ме видял, и на онова, което ще ти открия,
\par 17 като те избавям от юдейския народ и от езичниците, между които те пращам,
\par 18 да им отвориш очите, та да се обърнат от тъмнината към светлината, и от властта на Сатана към Бога, и да приемат прощение на греховете си и наследство между осветените, чрез вяра в мене.
\par 19 Затова, царю Агрипо, не бях непокорен на небесното видение.
\par 20 но проповядвах първом на юдеите в Дамаск, в Ерусалим, и в цялата юдейска земя, а после и на езичниците, да се покайват и да се обръщат към Бога, като вършат дела съответствени на покаянието си,
\par 21 По тая причина юдеите ме уловиха в храма и се опитаха да ме убият.
\par 22 Но с помощта, която получих от Бога, стоя до тоя ден т а свидетелствувам и пред скромен и пред високопоставен, без да говоря нищо друго освен това, което пророците и Моисей са говорили, че щеше да бъде,
\par 23 сиреч, че Христос трябваше да пострада, и че той, като възкръсне пръв от мъртвите, щеше да проповядва светлина и на юдейския народ и на езичниците.
\par 24 Когато той така се защищаваше, Фест, извика със силен глас: Полудял си, Павле; голямата ти ученост те докарва до лудост.
\par 25 А Павел рече: Не съм полудял, честити Фесте, но от здрав ум изговарям истинни думи,
\par 26 Защото царят, комуто и говоря дързостно, знае за това понеже съм убеден, че нищо от това не е скрито от него, защото то не е станало в някой ъгъл.
\par 27 Царю Агрипо, вярваш ли пророците? Зная, че ги вярваш.
\par 28 А Агрипа рече на Павла: Без малко ме убеждаваш да стана Християнин;
\par 29 А Павел рече: Молил се бях Богу щото било с малко, било с много, не само ти но и всички, които ме слушат днес, да станат такива, какъвто съм аз, освен тия окови.
\par 30 Тогава царят стана, с управителя и Верникия и седящите с тях.
\par 31 И като се оттеглиха настрана говореха помежду си, казвайки: Тоя човек не върши нищо достойно за смърт или окови.
\par 32 А Агрипа рече на Феста: Тоя човек можеше да се пусне ако не беше се отнесъл до Кесаря.

\chapter{27}

\par 1 И когато бе решено да отплуваме за Италия, предадоха Павла и някои други затворници на един стотник на име Юлий, от Августовия полк.
\par 2 И като се качихме на един Адрамитски кораб, който щеше да отплува за местата покрай Азийския бряг, тръгнахме; и с нас беше Аристарх, македонец от Солун.
\par 3 На другия ден стигнахме в Сидон; и Юлий се отнасяше човеколюбиво към Павла, и му позволи да отиде при приятелите си за да му пригодят.
\par 4 и оттам като станахме, плувахме на завет под Кипър, понеже ветровете бяха противни.
\par 5 И като преплувахме Киликийското и Памфилийското море, стигнахме в Ликийския град Мира.
\par 6 Там стотникът намери един Александрийски кораб, който плуваше за Италия н тури ни в него.
\par 7 И след като бяхме плували бавно за много дни, и едвам стигнахме Книд, понеже вятърът не ни позволяваше да влезем там, плувахме на завет под Крит срещу нос Салмон.
\par 8 И като преминахме него с мъка, стигнахме на едно място, което се казва Добри Пристанища, близо при което бе град Ласей.
\par 9 Но като беше се минало много време и плуването беше вече опасно, защото и постът беше вече минал, Павел ги съветваше, казвайки им:
\par 10 Господа, виждам, че плуването ще бъде придружено с повреда и голяма пагуба, не само на товара и на кораба, но и на живота ни.
\par 11 Но стотникът се доверяваше повече на кормчията и на стопанина на кораба отколкото на Павловите думи,
\par 12 И понеже пристанището не беше сгодно за презимуване, повечето изказаха мнение да се дигнат оттам за да стигнат, ако би било възможно, до Феникс, Критско пристанище, което гледа към югозапад и северозапад, и там да презимуват.
\par 13 И когато подухна южен вятър, мислейки, че сполучиха целта си, те дигнаха котва та плуваха близо покрай Крит.
\par 14 Но след малко, спусна се от острова бурен вятър, наречен Евраквилон;
\par 15 и когато корабът бе настигнат от вятъра и поради него не можеше да устои, оставихме се на вълните да ни носят.
\par 16 И като минахме на завет под едно островче наречено Клавдий, сполучихме с мъка да запазим ладията;
\par 17 и когато я издигнаха, употребяваха всякакви средства, и препасваха кораба изотдолу; и боейки се да не бъдат тласнати върху Сиртис, свалиха платната и се носеха така.
\par 18 И понеже бяхме в голяма беда от бурята, на следния ден хванаха да изхвърлят товара.
\par 19 И на третия ден те със своите ръце изхвърлиха вещите на кораба.
\par 20 И понеже за много дни не се виждаше ни слънце ни звезди, и силната буря напираше, то изчезна вече всяка надежда да бъдем спасени.
\par 21 А подир дълго неядене Павел застана между тях и рече: Господа, трябваше да ме слушате да се не дигаме от Крит, та да ни не постигнеше тая повреда и пагуба.
\par 22 Но и сега ви съветвам да сте бодри, защото ни една душа от вас няма да се изгуби, но само кораба;
\par 23 защото ангел от Бога, чийто съм аз и комуто служа, застана до мене тая нощ и рече,
\par 24 Не бой се, Павле, ти трябва да застанеш пред Кесаря; и, ето, Бог ти подари всички, които плуват с тебе.
\par 25 Затова, господа бъдете бодри; защото вярвам в Бога, че ще бъде тъй както ми бе казано.
\par 26 Обаче ние трябва да бъдем изхвърлени на някой остров.
\par 27 А когато настана четиринадесетата нощ, и ние се тласкахме насам-натам по Адриатическото море, около, среднощ корабниците усетиха, че се приближават до някоя суша.
\par 28 И като измериха дълбочината, намериха, че е двадесет разтега; и отивайки малко по-нататък пак измериха, и намериха че е петнадесет разтега.
\par 29 Затова, боейки се да не бъдат изхвърлени на каменисти места, спуснаха четири котви от задницата, и ожидаха да съмне.
\par 30 И понеже корабниците възнамеряваха да избягат от кораба, и бяха свалили ладията в морето под предлог, че щели да спуснат котви откъм предницата,
\par 31 Павел рече на стотника и на войниците: Ако тия не останат в кораба, вие не можете се избави.
\par 32 Тогава войниците отрязаха въжата на ладията и оставиха я да се носи от морето.
\par 33 А на съмване Павел канеше всички да похапнат, казвайки: Днес е четиринадесетият ден как чакате и стоите гладни, без да сте вкусили нещо.
\par 34 Затуй ви моля да похапнете, защото това ще помогне за вашето избавление; понеже никому от вас ни косъм от главата няма да загине.
\par 35 И като рече това, взе хляб, благодари Богу пред всички та разчупи, и почна да яде.
\par 36 От това всички се ободриха, та ядоха и те.
\par 37 И в кораба бяхме всичко двеста седемдесет и шест души.
\par 38 И като се нахраниха, облекчаваха кораба, като изхвърляха житото в морето.
\par 39 И когато се разсъмна, те не познаваха земята; обаче забелязаха един залив с песъчлив бряг, в който се решиха да тикнат кораба, ако бе възможно.
\par 40 И като откачиха котвите, оставиха ги в морето, развързаха още и връзките на кормилата, развиха малкото платно по посоката на вятъра, и се отправиха към, брега.
\par 41 Но изпаднаха на едно място, дето морето биеше от две страни, и там кораба заседна; предницата се заби и не мърдаше, а задницата взе да се разглобява от напора на вълните.
\par 42 И войниците съветваха да се избият запрените, да не би да изплува някой и да избяга.
\par 43 Но стотникът като искаше да избави Павла, възпря ги от това намерение, и заповяда да скочат в морето първо ония, които знаяха да плават, и да излязат на сухо,
\par 44 а останалите да се спасяват кои на дъски, кои пък на нещо от кораба. И така стана та всички излязоха безопасно на сушата.

\chapter{28}

\par 1 И когато се избавихме, познахме, че островът се наричаше Малта.
\par 2 А туземците ни показаха необикновено човеколюбие; защото приеха всички нас, и, понеже валеше дъжд и беше студено накладоха огън.
\par 3 И когато Павел натрупа един куп храсти и го тури на огъня, една ехидна излезе от топлината и се залепи за ръката му.
\par 4 А туземците, като видяха змията, как висеше от ръката му, думаха помежду си: Без съмнение тоя човек ще е убиец, който, ако и да се е избавил от морето, пак правосъдието не го остави да живее.
\par 5 Но той тръсна змията в огъня и не почувствува никакво зло.
\par 6 А те очакваха, че ще отече, или внезапно ще падне мъртъв; но като чакаха много време и гледаха, че не му става никакво зло, промениха мнението си и думаха, че е бог.
\par 7 А около това място се намираха именията на първенеца в острова, чието име беше Поплий, който ни прие и гощава приятелски три дни.
\par 8 И случи се Поплиевият баща да лежи болен от треска и дизентерия; а Павел влезе при него, и като се помоли, положи ръце на него и го изцели.
\par 9 Като стана това, и другите от острова, които имаха болести, дохождаха и се изцеляваха;
\par 10 които и ни показваха много почести, и , когато тръгнахме, туриха в кораба потребното за нуждите ни.
\par 11 И тъй, подир три месеца отплувахме с един александрийски кораб, който беше презимувал в острова, и който имаше за знак Близнаците.
\par 12 И като стигнахме в Сиракуза, преседяхме там три дни.
\par 13 И оттам, като лъкатушехме, стигнахме до Ригия; и след един ден, като повея южен вятър, на втория ден дойдохме в Потиоли,
\par 14 гдето намерихме братя, които ни замолиха да преседим у тях седем дни. Така дойдохме в Рим.
\par 15 отгдето братята, като чули за нас, бяха дошли до Апиевото тържище и до трите кръчми да ни посрещнат; и Павел като ги видя, благодари на Бога и се ободри.
\par 16 А когато влязохме в Рим, стотникът предаде запряните на войводата; а на Павла се позволи да живее отделно с войника, който го вардеше.
\par 17 И подир три дни той свика по-първите от юдеите и, като се събраха каза им: Братя, без да съм сторил аз нещо против народа ни, или против бащините обичаи, пак от Eрусалим ме предадоха вързан в ръцете на римляните;
\par 18 които, като ме изпитаха, щяха да ме пуснат, защото в мене нямаше нищо достойно за смърт.
\par 19 Но понеже юдеите се възпротивиха на това, принудих се да се отнеса до Кесаря, а не че имах да обвиня в нещо народа си.
\par 20 По тая причина, прочее, ви повиках, за да ви видя и да ви поговоря, защото заради това, за което Израил се надява, съм вързан с тая верига.
\par 21 А те му казаха: Нито сме получавали ние писма от Юдея за тебе, нито е дохождал някой от братята да ни извести, или да ни каже нещо лошо за тебе.
\par 22 Но желаем да чуем от тебе какво мислиш, защото ни е известно, че навсякъде говорят против това учение.
\par 23 И като му определиха ден, мнозина от тях дойдоха при него там гдето живееше; и от сутринта до вечерта той им излагаше с доказателства Божието царство и ги уверяваше за Исуса и от Моисеевия закон и от пророците.
\par 24 И едни повярваха това, което говореше, а други не вярваха.
\par 25 И те, понеже бяха несъгласни помежду си, се разотиваха, като им рече Павел една дума: Добре е говорил Святият Дух чрез пророк Исая на бащите ви, когато е рекъл:
\par 26 "Иди, кажи на тия люде: Със слушане ще чуете, но никак няма да схванете; И с очи ще видите, но никак няма да разберете.
\par 27 Защото затлъстя сърцето на тия люде, И ушите им натегнаха, И очите си затвориха, Да не би да гледат с очите си, И да разберат със сърцето си, И да се обърнат та да ги изцеля"
\par 28 Затуй, да знаете, че това Божие спасение се изпрати на езичниците; и те ще слушат.
\par 29 [И като рече това; юдеите си отидоха с голяма препирня помежду си]
\par 30 А Павел преседя цели две години в отделна под наем къща, гдето приемаше всички, които отиваха при него,
\par 31 като проповядваше Божието царство, и с пълно дръзновение поучаваше за Господа Исуса Христа без да му забранява никой.

\end{document}