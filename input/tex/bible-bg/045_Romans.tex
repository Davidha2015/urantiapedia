\begin{document}

\title{Римляни}


\chapter{1}

\par 1 Павел, слуга Исус Христов, призван за апостол, отделен да проповядва благовестието от Бога,
\par 2 (което по-напред Той беше обещал чрез пророците Си в светите писания),
\par 3 за Сина Му нашия Господ Исус Христос, Който по плът се роди от Давидовото потомство,
\par 4 а по Дух на светост биде със сила обявен като Божий Син чрез възкресението от мъртвите;
\par 5 чрез когото получихме благодат и апостолство, та в Неговото име да привеждаме в послушност към вярата човеци от всичките народи;
\par 6 между които сте и вие призвани от Исуса Христа:
\par 7 до всички в Рим, които са възлюбени от Бога, призвани да бъдат светии: Благодат и мир да бъдат с вас от Бога, нашия Отец, и Господа Исуса Христа.
\par 8 Най-напред благодаря на моя Бог чрез Исуса Христа за всички ви, гдето за вашата вяра се говори по целия свят.
\par 9 Понеже Бог, Комуто служа с духа си в благоденствието на Сина Му, ми е свидетел, че непрестанно ви споменавам в молитвите си,
\par 10 молящ се винаги, дано с Божията воля благоуспея най-после сега да дойда при вас.
\par 11 Защото копнея да ви видя, за да ви предам някоя духовна дарба за вашето утвърждаване,
\par 12 то ест, за да се утеша между вас взаимно с вас чрез общата вяра, която е и ваша и моя,
\par 13 И желая, братя, да знаете, че много пъти се канех да дойда при вас, за да имам някой плод и между вас както между другите народи; но досега съм бил възпиран.
\par 14 Имам длъжност към гърци и към варвари, към учени и към неучени;
\par 15 и така, колкото зависи от мене, готов съм да проповядвам благовестието и на вас, които сте в Рим.
\par 16 Защото не се срамувам от благовестието [Христово]; понеже е Божия сила за спасение на всекиго, който вярва, първо на юдеина, а после на езичника.
\par 17 Защото в него се открива правдата, която е от Бога чрез вяра към вяра, както е писано: "Праведният чрез вяра ще живее".
\par 18 Защото Божият гняв се открива от небето против всяко нечестие и неправда на човеците, които препятсвуват на истината чрез неправда.
\par 19 Понеже, това, което е възможно да се знае за Бога, на тях е известно, защото Бог им го изяви.
\par 20 Понеже от създанието на това, което е невидимо у Него, сиреч вечната Му сила и божественост, се вижда ясно, разбираемо чрез творенията; така щото, човеците остават без извинение.
\par 21 Защото, като познаха Бога, не Го просвавиха като Бог, нито Му благодариха; но извратиха се чрез своите мъдрувания, и несмисленото им сърце се помрачи.
\par 22 Като се представяха за мъдри, те глупееха,
\par 23 и славата на нетленния Бог размениха срещу подобие на образ на смъртен човек, на птици; на четириноги и на гадини.
\par 24 Затова, според страстите на сърцата им, Бог ги предаде на начистота, защото да се обезчестят телата им между сами тях, -
\par 25 те които замениха истинския Бог с лъжлив, и предпочетоха да се покланят и да служат на тварта, а не на Твореца, Който е благословен до века. Амин.
\par 26 Затова Бог ги предаде на срамотни страсти, като и жените им измениха естественото употребление на тялото в противоестествено.
\par 27 Така и мъжете, като оставиха естественото употребление на женския пол, разжегоха се в страстта си един към друг, струвайки безобразие мъже с мъже, и приемаха в себе си заслуженото въздеяние на своето нечестие.
\par 28 И понеже отказваха да познаят Бога, Бог ги предаде на развратен ум да вършат това, което не е прилично,
\par 29 изпълнени с всякакъв вид неправда, нечестие, лакомство, омраза; пълни със завист, убийство, крамола, измама и злоба;
\par 30 шепотници, клеветници, богоненавистници, нахални, горделиви, самохвалци, измислители на злини, непокорни на родителите си,
\par 31 безразсъдни, вероломни, без семейна обич, немилостиви;
\par 32 които, при все че знаят Божията справедлива присъда, че тия, които вършат такива работи, заслужават смърт, не само ги вършат, но и одобряват ония, които ги вършат.

\chapter{2}

\par 1 Затова и ти си без извинение, о човече, който и да си, когато съдиш другиго; защото в каквото съдиш другия, себе си осъждаш; понеже ти, който съдиш, вършиш същото,
\par 2 А знам, че Божията съдба против тия, които вършат такива работи, е според истината.
\par 3 И ти, човече, който съдиш ония, които вършат такива работи, мислиш ли, че ще избегнеш съдбата на Бога, като вършиш и ти същото?
\par 4 Или презираш Неговата богата благост, търпеливост и дълготърпение, без да знаеш, че Божията благост е назначена да те води към покаяние?
\par 5 а с упорството си и непокаяното си сърце трупаш на себе си гняв за деня на гнева, когато ще се открие праведната съдба от Бога,
\par 6 Който ще въздаде на всеки според делата му:
\par 7 вечен живот на тия, които с постоянство в добри дела търсят слава, почест и безсмъртие;
\par 8 а пък гняв и негодувание на ония, които са твърдоглави и не се покоряват на истината, а се покоряват на неправдата;
\par 9 скръб и неволя на всяка човешка душа, която прави зло, първо на юдеина, после и на гърка,
\par 10 а слава и почест и мир на всеки, който прави добро, първо на юдеина, после и на гърка.
\par 11 Понеже Бог не гледа на лице.
\par 12 Защото тия, които са съгрешили без да имат закон, без закон ще и погинат; и които са съгрешили под закон, под закон ще бъдат съдени.
\par 13 Защото не законослушателите са праведни пред Бога; но законоизпълнителите ще бъдат оправдани,
\par 14 (понеже, когато езичниците, които нямат закон, по природа вършат това, което се изисква от закона, то, и без да имат закон, те сами са закон за себе си,
\par 15 по това, че те показват действието на закона написано на сърцата им, на което свидетелствува и съвестта им, а помислите им или ги осъждат помежду си, или ги оправдават),
\par 16 в деня, когато Бог чрез Исуса Христа ще съди тайните дела на човеците според моето благовестие.
\par 17 Но ако ти се наричаш юдеин, облягаш се на закон, хвалиш се с Бога,
\par 18 знаеш Неговата воля, и разсъждаваш между различни мнения, понеже се учиш от закона:
\par 19 ако при това си уверен в себе си, че си водител на слепите, светлина на тия, които са в тъмнина,
\par 20 наставник на простите, учител на младенците, понеже имаш в закона олицитворение на знанието и на истината,
\par 21 тогава ти, който учиш другите, учиш ли себе си? Ти, който проповядваш да не крадат, крадеш ли?
\par 22 Ти,който казваш да не прелюбодействуват, прелюободействуваш ли? Ти, който се гнусиш от идолите, светотатствуваш ли?
\par 23 Ти, който се хвалиш в закона, опозоряваш ли Бога като престъпваш закона?
\par 24 Питам това, защото според както е писано, поради вас се хули Божието име между езичниците.
\par 25 Понеже обрязването наистина ползува, ако изпълняваш закона; но ако си престъпник на закона, тогава твоето обрязване става необрязване.
\par 26 И тъй, ако необрязаният пази наредбите на закона, не ще ли неговото необрязване да му се вмени за обрязване?
\par 27 и оня, който остане в природното си състояние необрязан, но пак изпълнява закона, не ще ли осъди тебе, който имаш писан закон и обрязване, но си престъпник на закона?
\par 28 Защото не е юдеин оня, външно такъв, нито е обрязване онова, което е вънкашно в плътта;
\par 29 но юдеин е тоя, който е такъв вътрешно; а обрязване е това, което е на сърцето, в дух, а не в буквата; чията похвала не е от човеците, а от Бога.

\chapter{3}

\par 1 Тогава, какво предимство има юдеинът? или каква полза има от обрязването?
\par 2 Много във всяко отношение, а първо, защото на юдеите се повериха Божествените писания.
\par 3 Понеже, ако някои бяха без вяра, що от това? тяхното неверие ще унищожи ли Божията вярност?
\par 4 Да не бъде! но Бог нека бъде признат за верен, а всеки човек лъжлив, според както е писано: - "За да се оправдаеш в думите Си, И да победиш, когато се съдиш".
\par 5 Но ако нашата неправда изтъква Божията правда, що има да кажем? Несправедлив ли е Бог, когато нанася гняв? (По човешки говоря).
\par 6 Да не бъде! понеже тогава как Бог ще съди света?
\par 7 Обаче, казваш ти, ако с моята невярност Божията вярност стане по-явна, за Неговата слава, то защо и аз, въпреки това, да бъда осъждан като грешник?
\par 8 И защо да не вършим зло, за да дойде добро? (както някои клеветнически твърдят, че ние така говорим). На такива осъждането е справедливо.
\par 9 Тогава що следва? Имаме ли ние някакво предимство над езичниците? Никак; защото вече обвинихме юдеи и гърци, че те всички са под грях.
\par 10 Както е писано: - "Няма праведен ни един;
\par 11 Няма никой разумен, Няма кой да търси Бога.
\par 12 Всички се отклониха, заедно се развратиха; Няма кой да прави добро, няма ни един".
\par 13 "Гроб отворен е гърлото им; С езиците си ласкаят". "Аспидова отрова има под устните им"
\par 14 "Техните уста са пълни с клевета и горест".
\par 15 "Нозете им бързат да проливат кръв;
\par 16 Опустошение и разорение има в пътищата им;
\par 17 И те не знаят пътя на мира",
\par 18 "Пред очите им няма страх от Бога".
\par 19 А знам, че каквото казва законът, казва го за ония, които са под закона; за да се затулят устата на всекиго, и цял свят да се доведе под съдбата на Бога.
\par 20 Защото ни една твар няма да се оправдае пред Него чрез дела изисквани от закона, понеже чрез закона става само познаването на греха,
\par 21 А сега и независимо от закон се яви правдата от Бога, за която свидетелствуват законът и пророците,
\par 22 сиреч правдата от Бога, чрез вяра в Исус Христа, за всички [и на всички], които вярват; защото няма разликат
\par 23 Понеже всички съгрешиха и не заслужават да се прославят от Бога,
\par 24 а с Неговата благост се оправдават даром чрез изкуплението, което е в Христа Исуса,
\par 25 Когото Бог постави за умилостивение чрез кръвта Му посредством вяра. Това стори за да покаже правдата Си в прощаване на греховете извършени по-напред, когато Бог дълготърпеше, -
\par 26 за да покаже, казвам правдата Си в настоящето време, та да се познае, че Той е праведен и че оправдава този, който вярвя в Исуса.
\par 27 И тъй, где остава хвалбата? Изключена е. Чрез какъв закон? чрез закона на делата ли? Не, но чрез закона на вярата.
\par 28 И така, ние заключаваме, че човек се оправдава чрез вяра, без делата на закона.
\par 29 Или Бог е Бог само на юдеите, а не и на езичниците? Да, и на езичниците е.
\par 30 Понеже същият Бог ще оправдае обрязаните от вяра и необрязаните чрез вяра.
\par 31 Тогава, чрез вяра разваляме ли закона? Да не бъде! но утвърждаваме закона.

\chapter{4}

\par 1 И тъй, какво ще кажем, че нашияо отец Авраам, е намерил по плът?
\par 2 Защото ако Авраам се е оправдал от дела, има с какво да се хвали, само не пред Бога.
\par 3 Понеже какво казва писанието: "Авраам повярва в Бога, и това му се вмени за правда".
\par 4 А на този, който върши дела, наградата му се не счита като благодеяние, но като дълг;
\par 5 а на този, който не върши дела, а вярва в Онзи, Който оправдава нечестивия, неговата вяра му се вменява за правда.
\par 6 Както и Давид говори за блаженството на човека, комуто Бог вменява правда независимо от дела:-
\par 7 "Блажени ония, чиито беззакония са простени Чиито грехове са покрити;
\par 8 Блажен е оня човек, комуто Господ няма да вмени грях".
\par 9 Прочее, това блаженство само за обрязаните ли е, или за необрязаните? Понеже казваме: "На Авраама вярата се вмени за правда".
\par 10 то как му се вмени? Когато беше обрязан ли, или необрязан? Не когато беше обрязан, но необрязан.
\par 11 И той обрязването като знак и печат на правдата от вяра, която имаше, когато беше необрязан, за да бъде той отец на всички, които вярват, ако и необрязани, за да се вмени правдата на тях.
\par 12 и отец на ония обрязани, които не само са обрязани, но и ходят в стъпките на оная вяра, която нашият отец Авраам е имал, когато бе необрязан,
\par 13 Понеже обещанието към Авраама или към потомството му, че ще бъде наследник на света, не стана чрез закон, но чрез правдата от вяра.
\par 14 Защото, ако са наследници тия, които се облягат на закона, то вярата, то вярата е празна, и обещанието осуетено;
\par 15 понеже законът докарва, не обещание, а гняв; но гдето няма закон, там няма нито престъпление.
\par 16 Затова наследството е от вяра, за да бъде по благодат, така щото обещанието да е осигурено за цялото потомство, не само за това, което се обляга на закона, но и за онова, което е от вярата на Авраама, който е отец на всички ни,
\par 17 (както е писано: "Направих те отец на много народи"), пред Бога Когото повярва, Който съживява мъртвите, и повиква в действително съществуване онова, което не съществува.
\par 18 Авраам, надявайки се без да има причина за надежда, повярва, за да стане отец на много народи, според реченото: "Толкова ще бъде твоето потомство".
\par 19 Без да ослабне във вяра, той вземаше предвид, че тялото му е вече замъртвяло, като бе на около сто години, вземаше предвид и мъртвостта на Сарината утроба, -
\par 20 обаче, относно Божието обещание не се усъмни чрез неверие, но се закрепи във вяра, и даде Богу слава,
\par 21 уверен, че това, което е обещал Бог, Той е силен да го изпълни.
\par 22 Затова му се вмени за правда.
\par 23 Това пък, че му се вмени за правда, не се написа само за него,
\par 24 но и за нас, на които ще се вменява за правда, като вярваме в Този, Който е възкресил от мъртвите Исус, нашия Господ,
\par 25 Който биде предаден за прегрешенията ни, и биде възкресен за оправданието ни.

\chapter{5}

\par 1 И тъй, оправдани чрез вяра, имаме мир с Бога, чрез нашия Господ Исус Христос;
\par 2 посредством Когото ние чрез вяра придобихме и достъп до тая благодат, в която стоим, и се радваме поради надеждата за Божията слава.
\par 3 И не само това, но нека се хвалим и в скръбта си, като знаем, че скръбта произвежда твърдост,
\par 4 а твърдостта изпитана правда; а изпитаната правда надежда.
\par 5 А надеждата не посрамява, защото Божията любов е изляна в сърцата ни чрез дадения нам Свети Дух.
\par 6 Понеже, когато ние бяхме още немощни, на надлежното време Христос умря за нечестивите.
\par 7 Защото едва ли ще се намери някой да умре даже за праведен човек; (при все че е възможно да дръзне някой да умре за благия);
\par 8 Но Бог препоръчва Своята към нас любов в това, че,когато бяхме още грешници, Христос умря за нас.
\par 9 Много повече, прочее, сега като се оправдахме чрез кръвта Му, ще се избавим от Божия гняв чрез Него.
\par 10 Защото, ако бидохме примирени с Бога чрез смъртта на Сина Му, когато бяхме неприятели, колко повече сега, като сме примирени, ще се избавим чрез Неговия живот!
\par 11 И не само това, но се и хвалим в Бога чрез Нашия Господ Исус Христос, чрез Когото получихме сега това примирение.
\par 12 Затова, както чрез един човек грехът влезе в света, и чрез греха смъртта, и по тоя начин смъртта мина във всичките човеци, понеже всички съгрешиха -
\par 13 (защото и преди закона грехът беше в света, грях, обаче, не се вменява, когато няма закон;
\par 14 при все това от Адама до Моисея смъртта царува и над ония, които не бяха съгрешили според престъплението на Адама, който е образ на бъдещия;
\par 15 но дарбата не е такава каквото бе прегрешението; защото ако поради прегрешението на единия измряха мнозината, то Божията благодат и дарбата чрез благодатта на един човек, Исус Христос, много повече се преумножи за мнозината;
\par 16 нито е дарбата, каквато бе съдбата, чрез съгрешението на един; защото съдбата беше от един грях за осъждане, а дарбата от много прегрешения за оправдание;
\par 17 защото, ако чрез прегрешението на единия смъртта царува чрез тоя един, то много повече тия, които получават изобилието на благодатта и на дарбата, сиреч, правдата, ще царуват в живот чрез единия, Исус Христос), -
\par 18 и тъй, както чрез едно прегрешение дойде осъждането на всичките човеци, така и чрез едно праведно дело дойде на всичките човеци оправданието, което докарва живот.
\par 19 Защото, както чрез непослушанието на единия човек станаха грешни мнозината, така и чрез послушанието на единия мнозината ще станат праведни.
\par 20 А отгоре на това дойде и законът, та се умножи прегрешението; а гдето се умножи грехът преумножи се благодатта;
\par 21 така щото, както грехът бе царувал и докара смъртта, така де царува благодатта чрез правдата и да докара вечен живот чрез Исуса Христа нашия Господ.

\chapter{6}

\par 1 Тогава какво? Да речем ли: Нека останем в греха, за да се умножи благодатта?
\par 2 Да не бъде! Ние, които сме умрели към греха, как ще живеем вече в него?
\par 3 Или не знаете, че ние всички, които се кръстихме да участвуваме в Исуса Христа, кръстихме се да участвуваме в смъртта Му?
\par 4 Затова, чрез кръщението ние се погребахме с Него да участвуваме в смърт, тъй щото, както Христос биде възкресен от мъртвите чрез славата на Отца, така и ние да ходим в нов живот.
\par 5 Защото, ако сме се съединили с Него чрез смърт подобна на Неговата, ще се съединим и чрез възкресение, подобно на Неговото;
\par 6 като знаем това, че нашето старо естество бе разпнато с Него за да се унищожи тялото на греха, та да не робуваме вече не греха.
\par 7 Защото, който е умрял, той е оправдан от греха.
\par 8 Но ако сме умрели с Христа, вярваме, че ще и живеем с Него,
\par 9 знаейки, че Христос, като биде възкресен от мъртвите, не умира вече; смъртта няма вече власт над Него.
\par 10 Защото, смъртта, с която умря, Той умря за греха еднаж за винаги; а животът, който живее, живее го за Бога.
\par 11 Така и вие считайте себе си за мъртви към греха, а живи към Бога в Христа Исуса.
\par 12 И тъй, да не царува грехът във вашето смъртно тяло, та да се покоряват на неговите страсти.
\par 13 Нито представяйте телесните си части като оръдия на неправдата; но представяйте себе си на Бога като оживели от мъртвите, и телесните си части на Бога като оръдия на правдата.
\par 14 Защото грехът няма да ви владее, понеже не сте под закон, а под благодат.
\par 15 Тогава какво? Да грешим ли, защото не сме под закон, а под благодат? Да не бъде!
\par 16 Не знаете ли, че комуто предавате себе си като послушни слуги, слуги сте на оня, комуто се покорявате, било на греха, който докарва смърт, или на послушанието, което докарва правда?
\par 17 Благодарение, обаче, Богу, че като бяхте слуги на греха, вие се покорихте от сърце на оня образец на вероучението, в който бяхте обучени,
\par 18 и, освободени от греха, станахте слуги на правдата.
\par 19 (По човешки говоря поради немощта на вашето естество). Прочее, както предавахте телесните си части като слуги на нечистотата и на беззаконието, което докарва още беззаконие, така сега предайте частите си като слуги на правдата, която докарва светост.
\par 20 Защото, когато бяхте слуги на греха не бяхте обуздавани от правдата.
\par 21 Какъв плод имахте тогава от ония неща? - неща, за които сега се срамувам, защото сетнината им е смърт.
\par 22 Но сега като се освободихте от греха, и станахте слуги на Бога, имате за плод това, че отивате към светост, на която истината е вечен живот.
\par 23 Защото заплатата на греха е смърт; а Божията дар е вечен живот в Христа Исуса, нашия Господ.

\chapter{7}

\par 1 Или не знаете, братя, (защото говоря на човеци, които знаят що е закон), че законът владее над човека само, докогато той е жив?
\par 2 Защото омъжена жена е вързана чрез закона за мъжа, докогато е жив; но когато мъжът умре тя се освобождава от мъжовия закон.
\par 3 И тъй, ако при живота на мъжа си тя се омъжи за друг мъж, става блудница; но ако умре мъжът й, свободна е от тоя закон, и не става блудница, ако се омъжи за друг мъж.
\par 4 И тъй, братя мои, и вие умряхте спрямо закона чрез Христовото тяло, за да се свържите с друг, сиреч, с възкресения от мъртвите, за да принасяме плод на Бога.
\par 5 Защото, когато бяхме плътски, греховните страсти, които се възбуждаха чрез закона, действуваха във вашите телесни части, за да принасяме плод който докарва смърт;
\par 6 но сега, като умряхме към това, което ни държеше, освободихме се от закона; тъй щото ние служим по нов дух, а не по старата буква.
\par 7 Тогава що? Да речем ли, че законът е грях? Да не бъде! Но напротив, не бих познал греха освен чрез закона, защото не бих познал, че пожеланието е грях, ако законът не беше казвал: "Не пожелавай".
\par 8 Но грехът понеже взе повод чрез заповедта, произведе в мене всякакво пожелание; защото без закон грехът е мъртъв.
\par 9 И аз бях жив някога без закон, но когато дойде заповедта, грехът оживя, а пък аз умрях:
\par 10 намерих, че самата заповед, която бе назначена да докара живот, докара ми смърт.
\par 11 Защото грехът, като взе повод чрез заповедта, измами ме и ме умъртви чрез нея.
\par 12 Тъй щото законът е свят, и заповедта свята, праведна и добра
\par 13 Тогава, това ли, което е добро, стана смърт за мене? Да не бъде! Но грехът ми причинява смърт чрез това добро нещо, за да се показва, че е грях, тъй щото чрез заповедта, грехът да стане много грешен.
\par 14 Защото знаем, че законът е духовен; а пък аз съм от плът, продаден под греха.
\par 15 Защото не зная какво правя: понеже не върша това, което искам; но онова което мразя, него върша.
\par 16 Обаче, ако върша, това което не искам, съгласен съм със закона , че е добър.
\par 17 Затова не аз сега върша това, но грехът, който живее в мене.
\par 18 Защото зная, че в мене, сиреч в плътта ми, не живее доброто; понеже желание за доброто имам; но злото, което не желая, него върша.
\par 19 Защото не върша доброто, което желая; но злото, което не желая, него върша.
\par 20 Но ако върша това, което не желая, то вече не го върша аз, а грехът, който живее в мене.
\par 21 И тъй, намирам тоя закон, че при мене, който желая да върша доброто, злото е близу.
\par 22 Защото, колкото за вътрешното ми естество, аз се наслаждавам в Божия закон;
\par 23 но в телесните си части виждам различен закон, който воюва против закона на ума ми, и ме заробва под греховния закон, който е в частите ми.
\par 24 Окаян аз човек! кой ще ме избави от тялота на тая смърт?
\par 25 Благодарение Богу! има избавление чрез Исуса Христа, нашия Господ. И тъй, сам аз с ума слугувам на Божия закон, а с плътта - на греховния закон.

\chapter{8}

\par 1 Сега прочее, няма никакво осъждане на тия, които са в Христа Исуса, [които ходят, не по плът но по Дух].
\par 2 Защото законът на животворящия Дух ме освободи в Христа Исуса от закона на греха и на смъртта.
\par 3 Понеже това, което бе невъзможно за закона, поради туй, че бе отслабнал чрез плътта, Бог го извърши като изпрати Сина Си в плът подобна на греховната плът и в жертва за грях, и осъди греха в плътта,
\par 4 за да се изпълнят изискванията на закона в нас, които ходим, не по плът, но по Дух.
\par 5 Защото тия, които са плътски, копнеят за плътското; а тия, които са духовни, за духовното.
\par 6 Понеже копнежът на плътта значи смърт; а копнежът на Духа значи живот и мир.
\par 7 Защото копнежът на плътта е враждебен на Бога, понеже не са покорява на Божия закон, нито пък може;
\par 8 и тия, които са плътски не могат да угодят на Бога.
\par 9 Вие, обаче, не сте плътски, а духовни, ако живее във вас Божият Дух. Но ако някой няма Христовия Дух, той не е Негов.
\par 10 Обаче, ако Христос е във вас, то при все, че тялото е мъртво поради греха, духът е жив поради правдата.
\par 11 И ако живее във вас Духът на Този, Който е възкресил Исуса от мъртвите, то Същият, Който възкреси Христа Исуса от мъртвите, ще съживи и вашите смъртни тела чрез Духа Си, който обитава във вас.
\par 12 И тъй, братя, ние имаме длъжност, обаче, не към плътта, та да живеем плътски.
\par 13 Защото, ако живеете плътски, ще умрете; но ако чрез Духа умъртвите телесните действия, ще живеете.
\par 14 Понеже които се управляват от Божия Дух, те са Божии синове
\par 15 Защото не сте приели дух на робство, та да бъдете пак на страх, но приели сте дух на осиновение, чрез който и викаме : Авва Отче!
\par 16 Така самият Дух свидетелствува заедно с нашия дух, че сме Божии чада.
\par 17 И ако сме чада то сме и наследници, наследници на Бога, и сънаследници с Христа, та, ако страдаме с Него, да се и прославяме заедно с Него.
\par 18 Понеже смятам, че сегашните временни страдания не заслужават да се сравнят със славата, която има да се открие към нас.
\par 19 Защото създанието с усърдно очакване ожида откриването ни като Божии синове.
\par 20 Понеже създанието беше подчинено на немощ ( Гръцки: Суетност ), не своеволно, но чрез Този, Който го подчини,
\par 21 с надежда, че и самото създание ще се освободи от робството на тлението, и ще премине в славната свобода на Божиите чада.
\par 22 Понеже знаем, че цялото създание съвокупно въздиша и се мъчи до сега.
\par 23 И не то само, но и ние, които имаме Духа в начатък, и сами ние въздишаме в себе си и ожидаме осиновението си, сиреч, изкупването на нашето тяло.
\par 24 Защото в тая надеждание ние се спасихме; а надежда, когато се вижда вече изпълнена, не е вече надежда; защото кой би се надявал за това, което вижда?
\par 25 Но, ако се надяваме за онова, което не виждаме, тогава с търпение го чакаме.
\par 26 Така също и Духът ни помага в нашата немощ: понеже не знаем да се молим както трябва; но самия Дух ходатайствува в нашите неизговорими стенания;
\par 27 а тоя, който изпитва сърцата, знае какъв е умът на Духа, защото той ходатайствува за светиите по Божията воля.
\par 28 Но знаем, че всичко съдействува за добро на тия, които любят Бога, които са призовани според Неговото намерение.
\par 29 Защото, който предузна, тях и предопредели да бъдат съобразни с образа на Сина Му, за да бъде Той първороден между много братя
\par 30 а които предопредели, тях и призова; а които призова, тях и оправда, а които оправда, тях и прослави.
\par 31 И тъй, какво да кажем за това? Ако Бог е откъм нас, кой ще бъде против нас?
\par 32 Оня, Който не пожали Своя Син но Го предаде за всички ни, как не ще ни подари заедно с Него и всичко?
\par 33 Кой ще обвини Божиите избрани? Бог ли, Който ги оправдава?
\par 34 Кой е оня, който ще ги осъжда? Христос Исус ли, Който умря, а при това и биде възкресен от мъртвите, Който е от дясната страна на Бога, и Който ходатайствува за нас?
\par 35 Кой ще ни отлъчи от Христовата любов? скръб ли, или утеснение, гонение или глад, голота, беда, или нож?
\par 36 (защото, както е писано. "Убивани сме заради Тебе цял ден; Считани сме като овци за клане").
\par 37 Не; във всичко това отиваме повече от победители чрез Този, Който ни е възлюбил.
\par 38 Понеже съм уверен, че нито смърт, нито живот, нито ангели, нито власти, нито сегашното, нито бъдещето, нито сили,
\par 39 нито височина, нито дълбочина, нито кое да било друго създание ще може да ни отлъчи от Божията любов, която е в Христа Исуса, нашия Господ.

\chapter{9}

\par 1 Казвам истината в Христа, не лъжа, и съвестта ми свидетелствува с мене в Светия Дух,
\par 2 че имам голяма скръб и непрестанна мъка в сърцето си.
\par 3 Защото бих желал сам аз да съм анатема ( Сиреч: Отлъчен ) от Христа, заради моите братя, моите по плът роднини:
\par 4 които са израилтяни, на които принадлежат осиновението на славата, заветите и даването на закона, богослужението и обещанията:
\par 5 чиито са и отците, и от които се роди по плът Христос, Който е над всички Бог, благословен до века. Амин.
\par 6 Обаче, не е пропаднало Божието слово; защото не всички ония са Израил, които са от Израиля;
\par 7 нито са всички чада, понеже са Авраамово потомство; но "в Исаака" каза Бог, "ще се наименува твоето потомство".
\par 8 Значи, не чадата, родени по плът, са Божии чада; но чадата, родени според обещанието се считат за потомство.
\par 9 Защото това беше нещо обещано, понеже каза: "Ще дойда по това време, и Сара ще има син".
\par 10 И не само това, но и когато Ребека зачена от едного, сиреч от нашия отец Исаака,
\par 11 макар че близнаците не бяха още родени и не бяха още сторили нещо добро или зло, то, за да почива Божието по избор намерение, не на дела, но на онзи, който призовава,
\par 12 рече й се: "По-големият ще слугува на по-малкия";
\par 13 както е писано: "Якова възлюбих , а Исава намразих".
\par 14 И тъй, какво? Да речем ли, че има неправда у Бога? Да не бъде!
\par 15 Защото казва на Моисея: "Ще покажа милост, към когото ще покажа, и ще пожаля, когото ще пожаля".
\par 16 И тъй, не зависи от този, който иска, нито от този, който тича, но от Бога, Който показва милост.
\par 17 Защото писанието казва на Фараона: "Именно за това те издигнах, за да покажа в тебе силата Си, и да се прочуе името Ми по целия свят"
\par 18 И тъй, към когото ще, Той показва милост, и когото ще закоравява.
\par 19 На това ти ще речеш: А защо още обвинява? Кой може да противостои на волята Му?
\par 20 Но, о човече, ти кой си, що отговаряш против Бога? Направеното нещо ще рече ли на онзи, който го е направил: Защо си ме така направил?
\par 21 Или грънчарят няма власт над глината, с част от същата буца да направи съд за почит, а с друга част - съд за непочтенна употреба?
\par 22 А какво ще кажем, ако Бог, при все, че е искал да покаже гнева Си и да изяви силата Си, пак е търпял с голямо дълготърпение съдовете, предмети на гнева Си, приготвени за погибел,
\par 23 и е търпял, за да изяви богатството на Славата Си, над съдовете, предмети на милостта Си, които е приготвил отнапред за слава -
\par 24 над нас, които призова, не само между юдеите, но и измежду езичниците?
\par 25 както и в Осия казва: - "Ще нарека мои люде ония, които не бяха мои люде, И тая възлюбена, която не беше възбюбена";
\par 26 И на същото място, гдето им се казва: "Не сте мои люде, Там ще се нарекат чада на живия Бог".
\par 27 А Исаия вика на Израиля: - "Ако и да е числото на израелтяните като морски пясък, Само остатък от тях ще се спаси;
\par 28 Защото Господ ще изпълни на земята казаното [по правда] от Него", Като го извърши и свърши скоро.
\par 29 И както Исаия е казал в по-предишно място; "Ако Господ на Силите не бе ни оставил потомство, Като Содом бихме останали и на Гомор бихме се уприличили".
\par 30 И тъй, какво да кажем? Това, че езичниците, които не търсеха правда, получиха правда, и то правда, която е чрез вярване;
\par 31 а Израил, който търсеше закон за придобиване правда, не стигна до такъв закон.
\par 32 Защо? затова, че не го търси чрез вярване, а някак си чрез дела. Те се спънаха о камъка, о който хората се спъват;
\par 33 както е писано: - "Ето, полагам в Сион камък, о който да се спъват, и канара, в която да се съблазняват; И който вярва в Него не ще се посрами".

\chapter{10}

\par 1 Братя, моето сърдечно желание и молбата ми към Бога е за спасението на Израиля.
\par 2 Защото свидетелствувам за тях, че те имат ревност за Бога, само че не е според пълното знание.
\par 3 Понеже, ако не знаят правдата, която е от Бога и искат да поставят своята, те не се покориха на правдата от Бога.
\par 4 Понеже Христос изпълнява целта на закона, да се оправдае всеки, който вярва.
\par 5 Защото Моисей пише, че човек, който върши правдата, която е чрез пазенето на закона, ще живее чрез нея.
\par 6 А правдата, която е чрез вяра, говори така: "Да не речеш в сърцето си: Кой ще се възкачи на небето, сиреч, да свали Христа?
\par 7 или: Кой ще слезе в бездната, сиреч да възведе Христа от мъртвите?"
\par 8 Но що казва тя? Казва, че "думата е близу при тебе, в устата ти и в сърцето ти", сиреч думата на вярата която проповядваме.
\par 9 Защото, ако изповядваш с устата си, че Исус е Господ, и повярваш със сърцето си, че Бог Го е възкресил от мъртвите ще се спасиш;
\par 10 Защото със сърце вярва човек и се оправдава, и с устата прави изповед и се спасява.
\par 11 Защото писанието казва: "Никой, който вярва в Него, не ще се посрами".
\par 12 Понеже няма разлика между юдеин и грък защото същият Господ е Господ на всички, богат към всички, които Го призовават.
\par 13 Защото "всеки, който призове Господното име, ще се спаси".
\par 14 Как, прочее, ще призоват Този, в Когото не са повярвали? И как ще повярват в Този за Когото не са чули? А как ще чуят без проповедник?
\par 15 И как ще проповядват, ако не бъдат пратени? Както е писано: - "Колко са прекрасни Нозете на тия, които благовествуват доброто!"
\par 16 Но не всички послушаха благовестието; Защото Исаия казва: "Господи, кой от нас е повярвал на онова, което сме чули"?
\par 17 И тъй, вярването е от слушане, а слушането - от Христовото слово.
\par 18 Но казвам: те не са ли чули? Наистина чули са: - "По цялата земя е излязъл гласът им, И думите им до краищата на вселената".
\par 19 Но пак казвам: Израил не е ли разбрал? Разбрал е, защото първо Моисей казва: - "Аз ще ви раздразня до ревнуване с тия, които не са с народ; С народ несмислен ще ви разгневя";
\par 20 А Исаия се осмелява да каже: - "Намерен бях от ония, които не Ме търсеха; Явен станах на тия, които не питаха за Мене;
\par 21 а за Израиля казва: - "Простирах ръцете Си цял ден Към люде непокорни и опаки"

\chapter{11}

\par 1 И тъй, казвам: Отхвърлил ли е Бог Своите люде? Да не бъде! Защото и аз съм израилтянин, от Авраамовото потомство, от Вениаминовото племе.
\par 2 Не е отхвърлил Бог людете Си, които е предузнал. Или не знаете що казва писанието за Илия? как вика към Бога против Израиля, казвайки:
\par 3 "Господи, избиха пророците Ти, разкопаха олтарите Ти, и аз останах сам; но и моя живот искат да отнемат".
\par 4 Но що му казва божественият отговор? - "Оставил съм Си седем хиляди мъже, които не са преклонили коляно пред Ваала".
\par 5 Така и в сегашно време има остатък, избран по благодат
\par 6 Но, ако е по благодат, не е вече от дела, иначе благодатта не е вече благодат [а ако е от делата, не е вече благодат, иначе делото не е вече дело].
\par 7 Тогава какво? Онова, което Израил търсеше, това не получи, но избраните го получиха, а останалите се закоравиха даже до днес:
\par 8 както е писано: "Бог им даде дух на безчуствие, очи - да не виждат и уши - да не чуват".
\par 9 И Давид казва: - "Трапезата им нека стане за тях примка и уловка, Съблазън и въздаяние;
\par 10 Да се помрачат очите им, та да не виждат, И сгърби гърба им за винаги".
\par 11 Тогава казвам: Спънаха ли се, та да паднат? Да не бъде! Но чрез тяхното отклонение дойде спасението на езичниците,за да ги възбуди към ревност.
\par 12 А, ако тяхното отклонение значи богатство за света и тяхното отпадане - богатство за езичниците, колко повече тяхното пълно възстановяване!
\par 13 Защото на вас, които бяхте езичници, казвам, че понеже съм апостол на езичниците, аз славя моята служба,
\par 14 дано по някакъв начин възбудя към ревност тия, които са моя плът, и да спася накои от тях.
\par 15 Защото, ако тяхното отхвърляне значи примирение на света, какво ще бъде приемането им, ако не оживяване от мъртвите?
\par 16 А ако първото от тестото е свято, то и цялото засяване е свято; и ако коренът е свят, то и клоните са свети.
\par 17 Но, ако някои клони са били отрязани, и ти, бидейки дива маслина, си бил присаден между тях, и си станал съучастник с тях в тлъстия корен на маслината,
\par 18 не се хвали между клоните; но ако се хвалиш, знай, че ти не държиш корена, а коренът тебе.
\par 19 Но ще речеш: Отрязаха се клони, за да се присадя аз.
\par 20 Добре, поради неверие те се отрязаха, а ти поради вяра стоиш. Не високоумствувай, но бой се.
\par 21 Защото, ако Бог не пощади естествените клони, нито тебе ще пощади.
\par 22 Виж, прочее, благостта и строгостта Божии; строгост към падналите, а божествена благост към тебе, ако останеш в тая благост; иначе, и ти ще бъдеш отсечен.
\par 23 Така и те, ако не останат в неверие, ще се присадят; защото Бог може пак да ги присади.
\par 24 Понеже, ако ти си бил отсечен от маслина, по естество дива, и, против естеството, си бил присаден на питомна маслина, то колко повече ония, които са естествени клони, ще се присадят на своята маслина!
\par 25 Защото, братя, за да не се мислите за мъдри, искам да знаете тая тайна, че частично закоравяване сполетя Израиля, само докато влезе пълното число на езичниците.
\par 26 И така целият Израил ще се спаси, както е писано: - "Избавител ще дойде от Сион; Той ще отвърне нечестията от Якова;
\par 27 И ето завета на Мене към тях: Когато отнема греховете им"
\par 28 Колкото за благовестието, те са неприятели, което е за наша полза, а колкото за избора, те са въблюбени заради бащите.
\par 29 Защото даровете и призванието от Бога са неотменими.
\par 30 Защото както вие някога се непокорявахте (Или: Неповярвахте - ват ) на Бога, но сега чрез тяхното непокорство ( Или: Неверие ) сте придобили милост, та чрез показаната към вас милост и те сега да придобият милост,
\par 31 също така и те сега се не покоряват.
\par 32 Защото Бог затвори всички в непокорство, та към всички да покаже милост.
\par 33 О колко дълбоко е богатството на премъдростта и знанието на Бога! Колко са непостижими Неговите съдби, и неизследими пътищата Му!
\par 34 Защото, "Кой е познал ума на Господа, Или, кой Му е бил съветник?"
\par 35 или, "Кой от по-напред Му е дал нещо, Та да му се отплати?"
\par 36 Защото всичко е от Него, чрез Него и за Него, Нему да бъде слава до векове. Амин.

\chapter{12}

\par 1 И тъй, моля ви, братя, поради Божиите милости, да представите телата си в жертва жива, света, благоугодна на Бога, като ваше духовно служение.
\par 2 И недейте се съобразява с тоя век ( Или: свят )., но преобразявайте се чрез обновяване на ума си, за да познаете от опит що е Божията воля, - това, което е добро, благоугодно Нему и съвършено.
\par 3 Защото, чрез дадената ми благодат, казвам на всеки един измежду вас, който е по-виден да не мисли за себе си по-високо, отколкото трябва да мисли, но да разсъждава така, щото да мисли скромно, според делата на вярата, които Бог е на всекиго разпределил.
\par 4 Защото, както имаме много части в едно тяло, а не всичките части имат същата служба,
\par 5 така и ние мнозината сме едно тяло в Христа, а сме части, всеки от нас, един на друг.
\par 6 И като имаме дарби, които се различават според дадената ни благодат, ако е пророчество, нека пророкуваме съразмерно с вярата;
\par 7 ако ли служене, нека прилежаваме в служенето, ако някой поучава, нека прилежава в поучаването:
\par 8 ако увещава, в увещаването: който раздава, да раздава щедро; който управлява, да управлява с усърдие; който показва милост, да я показва доброволно.
\par 9 Любовта да бъде нелицемерна; отвращавайте се от злото, а прилепявайте се към доброто.
\par 10 В братолюбието си обичайте се един друг, като сродници; изпреваряйте да си отдавате един на друг почит.
\par 11 В усърдието бивайте нелениви, пламенни по дух, като служите на Господа.
\par 12 Радвайте се в надеждата, в скръб бивайте твърди, в молитва постоянни.
\par 13 Помагайте на светиите в нуждите им; предавайте се на гостолюбие.
\par 14 Благославяйте ония, които ви гонят, благославяйте, и не кълнете
\par 15 Радвайте се с ония, които се радват; плачете с ония, които плачат.
\par 16 Бъдете единомислени един към друг; не давайте ума си на високи неща, но предавайте се на скромни неща; не считайте себе си за мъдри.
\par 17 Никому не връщайте зло за зло; промишлявайте за това, което е добро пред всичките човеци;
\par 18 ако е възможно, доколкото зависи от вас, живейте в мир с всичките човеци.
\par 19 Не си отмъстявайте, възлюбени, но дайте място на Божия гняв; защото е писано: "На мене принадлежи отмъщението, Аз ще сторя въздаяние, казва Господ".
\par 20 Но, "Ако е гладен неприятелят ти, нахрани го; Ако е жаден, напой го; Защото, това като правиш, ще натрупаш жар на главата му".
\par 21 Не се оставай да те побеждава злото; но ти побеждавай злото чрез доброто.

\chapter{13}

\par 1 Всеки човек да се покорява на властите, които са над него; защото няма власт, която да не е от Бога, и колкото власти има, те са отредени от Бога.
\par 2 Затова, който се противи на властта, противи се на Божията наретба; а които се противят ще навлекат осъждане на себе си.
\par 3 Защото владетелите не причиняват страх на добротвореца, но на злостореца. Искаш ли, прочее, да се не боиш от властта? Върши добро, и ще бъдеш похвален от нея;
\par 4 понеже владетелят е Божий служител за твоя полза. Но ако вършиш зло, да се боиш; защото той не носи напразно сабята, понеже е Божий служител, мъздовъздател за докарване гняв, върху ( Гръцки: за гнева спрямо ) този, който върши зло.
\par 5 Затова нужно е да се покорите не само поради страх от гнева, но и заради съвестта.
\par 6 Понеже за тая причина и данък плащате. Защото владетелите са Божии служители, които постоянно се занимават с тая длъжност.
\par 7 Отдавайте на всички дължимото: комуто се дължи данък, данъка; комуто мито, митото; комуто страх, страха; комуто почит, почитта.
\par 8 Не оставайте никому длъжни в нищо, освен един друг да се обичате, защото, който обича другиго, изпълнява закона.
\par 9 Понеже заповедите: "Не прелюбодействувай"; "Не убивай"; "Не кради"; "Не пожелавай"; и коя да било друга заповед се заключават в тия думи: "Да обичаш ближния си както себе си".
\par 10 Любовта не върши зло на ближния; следователно, любовта изпълнява закона.
\par 11 И това вършете като знаете времето, че часът е вече настъпил да се събудите от сън; защото спасението е по-близу до нас сега, отколкото, когато изпърво повярвахме.
\par 12 Нощта премина а денят наближи: и тъй нека отхвърлим делата на тъмнината, и да се облечем в оръжието на светлината.
\par 13 Както в бял ден нека ходим благопристойно, не по пирове и пиянства, не по блудство и страстолюбие, не по крамоли и зависти.
\par 14 Но облечете се с Господа Исуса Христа, и не промишлявайте за страстите на плътта.

\chapter{14}

\par 1 Слабия във вярата приемайте, но не за да се препирате за съмненията му,
\par 2 Един вярва, че може всичко да яде; а който е слаб във вярата яде само зеленчук.
\par 3 Който яде, да не презира този, който не яде; и който не яде; да не осъжда този, който яде; защото Бог го е приел.
\par 4 Кой си ти, що съдиш чужд слуга? Пред своя си господар той стои или пада. Но ще стои, защото Господ е силен да го направи да стои.
\par 5 Някой уважава един ден повече от друг ден; а друг човек уважава всеки ден еднакво. Всеки да бъде напълно уверен в своя ум.
\par 6 Който пази деня, за Господа го пази; [а който не пази деня, за Господа не го пази]; който яде, за Господа яде, защото благодари на Бога; и който не яде, за Господа не яде, и благодари на Бога.
\par 7 Защото никой от нас не живее за себе си, и никой не умира за себе си.
\par 8 Понеже, ако живеем, за Господа живеем, и ако умираме, за Господа умираме: и тъй живеем ли, умираме ли, Господни сме.
\par 9 Защото Христос затова умря и оживя - да господствува и над мъртвите и над живите.
\par 10 И тъй, ти защо съдиш брата си? а пък ти защо презираш брата си? Понеже ние всички ще застанем пред Божието съдилище.
\par 11 Защото е писано: "Заклевам се в живота Си, казва Господ, че всяко коляно ще се преклони пред Мене, И всеки език ще словослови Бога"
\par 12 И тъй, всеки от нас за себе си ще отговаря пред Бога.
\par 13 Като е тъй, да не съдим вече един друг; но по-добре е да бъде разсъждението ви това - никой да не полага на брата си спънка или съблазън.
\par 14 Зная и уверен съм в Господа Исуса, че нищо не е само по себе си нечисто; с това изключение, че за този, който счита нещо за нечисто, нему е нечисто.
\par 15 Защото, ако брат ти се оскърби поради това, което ядеш, ти вече не ходиш по любов. С яденето си не погубвай онзи, за когото е умрял Христос.
\par 16 Прочее, да се не хули това, което вие считате за добро,
\par 17 Защото Божието царство не е ядене и пиене, но правда, мир и радост в Светия Дух.
\par 18 Понеже, който така служи на Христа, бива угоден на Бога и одобрен от човеците.
\par 19 И тъй, нека търсим това, което служи за мир и за взаимно назидание.
\par 20 Заради ядене недей съсипва Божията работа. Всичко наистина е чисто; но е зло за човека, който с яденето си причинява съблазън.
\par 21 Добре е да не ядеш месо, нито да пиеш вино, нито да сториш нещо, чрез което се спъва брат ти, [или се съблазнява, или изнемощява].
\par 22 Вярата, която имаш за тия неща, имай я за себе си пред Бога. Блажен оня, който не осъжда себе си в това, което одобрява.
\par 23 Но оня, който се съмнява, осъжда се ако яде, защото не яде от убеждение; а всичко, което не става от убеждение е грях.

\chapter{15}

\par 1 Прочее, ние силните сме длъжни да носим немощите на слабите и да не угаждаме на себе си.
\par 2 Всеки от нас да угождава на ближния си, с цел към това, което е добро за назиданието му.
\par 3 Понеже и Христос не угоди на Себе Си, но, както е писано: - "Укорите на ония, които укоряваха тебе, Паднаха върху Мене".
\par 4 Защото всичко, що е било от по-напред писано, писано е било за наша поука, та чрез твърдостта и утехата от писанията да имаме надежда.
\par 5 А Бог на твърдостта и на утехата да ви даде единомислие помежду ви по примера на Христа Исуса.
\par 6 щото единодушно и с едни уста да славите Бога и Отца на нашия Господ Исус Христос.
\par 7 Затова приемайте се един друг, както и Христос ви прие, за Божията слава.
\par 8 Защото казвам, че Христос, заради Божията вярност, стана служител на обрязаните, за да утвърди обещанията дадени на бащите,
\par 9 и за да прославят езичниците Бога за Неговата милост, както е писано:- "Затова ще Те хваля между народите, И на името ти ще пея".
\par 10 И пак казва: - "Развеселете се, народи, с любовта Му".
\par 11 И пак: - "Хвалете Господа, всички народи, И да Го славословят всички люде".
\par 12 И пак Исаия казва: "Ще се яви Иесаевият корен", и, "Който ще се надигне да владее над народите; На Него ще се надяват народите".
\par 13 А Бог на надеждата да ви изпълни с пълна радост и мир във вярването, тъй щото чрез силата на Светия Дух да се преумножава надеждата ви.
\par 14 И сам аз съм уверен за вас, братя мои, че сами вие сте пълни с благост, изпълнени от всяко знание, и че можете да се наставлявате един друг.
\par 15 Но, за да ви напомня, пиша ви донякъде по-дързостно поради дадената ми от Бога благодат,
\par 16 да бъда служител Исус Христов между езичниците, и да свещенослужа в Божието благовестие, за да бъдат езичниците благоприятен принос, осветен от Светия Дух.
\par 17 И тъй, колкото за това, което се отнася до Бога, аз имам за какво да се похваля в Христа Исуса.
\par 18 Защото не бих се осмелил да говоря за нещо освен онова, което Христос е извършил чрез мене за привеждане езичниците в покорност на вярата чрез моето слово и дело,
\par 19 със силата на знамения и чудеса, със силата на Светия Дух [Божий], така щото от Ерусалим и околностите му дори до Илирик напълно съм проповядвал Христовото благовестие.
\par 20 Обаче имах за цел да проповядвам благовестието така, - не там гдето беше вече известно Христовото име, да не би да гради на чужда основа;
\par 21 но, както е писано: - "Ония ще видят, на които не се е възвестило за Него; И ония ще разберат, които не са чули"
\par 22 Това ме е възпирало много пъти, та не съм дохождал при вас.
\par 23 Но сега, като няма вече място за моето работене по тия страни, и понеже от много години съм желал да дойда при вас,
\par 24 на отиването си в Испания ще дойда, защото се надявам да ви видя като минавам, и вие да ме изпратите до там, след като се наситя донякъде чрез общение с вас.
\par 25 А сега отивам в Ерусалим да послужа на светиите.
\par 26 Защото Македония и Ахаия благоволиха да дадат известна помощ за бедните между светиите в Ерусалим.
\par 27 Благоволиха наистина, но и длъжни им са, защото, ако езичниците участвуват с тях и в духовните неща, длъжни са да им послужат и в телесните.
\par 28 Прочее, когато свърша това, като им осигуря тоя плод, ще мина през вас в Испания.
\par 29 И зная, че когато дойда при вас, ще дойда с изобилно благословение от [благовествуването на ] Христа.
\par 30 А моля ви се, братя, заради нашия Господ Исус Хртистос, и заради любовта, която е плод на Духа, да ме придружават усърдна молитва към Бога за мене,
\par 31 та да се избави от противниците на вярата в Юдея, и моята услуга за Ерусалим да бъде благоприятна на светиите;
\par 32 и с Божията воля да дойда радостен при Вас, и да си почина между вас.
\par 33 А Бог на мира да бъде с всички Вас. Амин.

\chapter{16}

\par 1 Препоръчвам ви нашата сестра Фива, която е служителка на църквата в Кенхрея,
\par 2 да я приемете в Господа, както е прилично на светиите, и да и помогнете в каквото би имала нужда от вас; защото и тя е помагала на мнозина, както на самия мене.
\par 3 Поздравете моите съработници в Христа Исуса, Прискила и Акила,
\par 4 които за моя живот си положиха вратовете под нож, на които не само благодаря, но и всички църкви между езичниците; поздравете и домашната им църква.
\par 5 Поздравете любезния ми Епинета, който е първият плод на Азия за Христа.
\par 6 Поздравете Мария, която се е трудила много за вас.
\par 7 Поздравете Андроника и Юния, моите сродници и някога заедно с мене затворници, които между апостолите се считат за бележити и които още преди мене бяха в Христа.
\par 8 Поздравете любезния ми в Господа Амплият.
\par 9 Поздравете нашия съработник в Христа Урвана и любезния ми Стахия.
\par 10 Поздравете одобрения за верен в Христа Апелият. Поздравете ония, които са от Аристовуловото семейство.
\par 11 Поздравете роднината ми Иродиона. Поздравете от Наркисовото семейство тия, които са в Господа.
\par 12 Поздравете Трифена и Трифоса, които работят в Господа. Поздравете любезната Пероида, която е работила много в Господа.
\par 13 Поздравете избрания от Господа Руфа, и неговата майка, която е и моя.
\par 14 Поздравете Асинкрита, Флегоита, Ерма, Патрова, Ермия и братята, които са с тях.
\par 15 Поздравете Филолога и Юлия, Нирея и сестра му, и Олимпана, и всичките светии, които са с тях.
\par 16 Позравете се един друг с света целувка. Поздравяват ви всичките Христови църкви.
\par 17 И моля ви се, братя, да забележите тия, които причиняват раздори и съблазни, противно на учението, което сте научили, и отстранявайте се от тях.
\par 18 Защото такива човеци не служат на нашия Господ [Исус] Христос, а на своите си охоти ( Гръцки: на своя си търбух ), и съблаги и ласкави думи прилъгват сърцата на простодушните.
\par 19 Защото вашата послушност е известна на всички, по която причина аз се радвам за вас. Но желал бих да бъдете мъдри относно доброто, а прости относно злото.
\par 20 А Бог на мира, скоро ще смаже сатана под нозете ви. Благодатта на нашия Господ Исус Христос да бъде с вас.
\par 21 Поздравяват ви съработникът ми Тимотей, и сродниците ми Лукий, Ясон и Сосипатър.
\par 22 Аз Тертий, който написах това послание, ви поздравявам в Господа.
\par 23 Поздравява ви Гаий, гостоприемникът на мене и на цялата църква. Поздравява ви градския ковчежник Ераст и брат Кварт.
\par 24 [Благодатта на нашия Господ Исус Христос да бъде всички вас. Амин].
\par 25 А на Този, Който може да ви утвърди според моето благовестие и проповедта за Исуса Христа, според откриването на тайната, която е била замълчана от вечни времена,
\par 26 а сега се е явила, и чрез пророческите писания по заповедта на вечния Бог е станала позната на всичките народи за тяхно покоряване на вярата, -
\par 27 на единия премъдър Бог да бъде слава чрез Исуса Христа до века. Амин.

\end{document}