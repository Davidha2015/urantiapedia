\begin{document}

\title{1 Коринтяни}


\chapter{1}

\par 1 Павел, с Божията воля призован да бъде апостол Исус Христов, и брат Состен,
\par 2 до Божията църква, която в Коринт, до осветените в Христа Исуса, призовани да бъдат светии заедно с всички, които призовават на всяко място името на Исуса Христа, нашия Господ, Който е и техен и наш:
\par 3 Благодат и мир да бъде на вас от Бога нашия Отец и от Господа Исуса Христа
\par 4 Винаги въздавам благодарения на моя Бог за вас за Божията благодат, която ви е била дадена в Христа Исуса
\par 5 че обогатихте се чрез Него в всичко, в пълна сила да говорите за Него.
\par 6 (по който начин се потвърди свидетелствуването за Христа между вас),
\par 7 така щото вие не оставате назад в никоя дарба, като чакате явлението на нашия Господ Исус Христос,
\par 8 Който и докрай ще ви отвърждава, та да бъдете безупречни в деня на нашия Господ Исус Христос.
\par 9 Верен е Бог, чрез Когото сте били призовани в общението на Сина Му Исуса Христа нашия Господ.
\par 10 Моля ви се, братя, за името на нашия Господ Исус Христос, всички да говорите в съгласие, и да няма раздори между вас, но да бъдете съвършено съединени в един ум и в една мисъл.
\par 11 Защото някои от Хлоините домашни ми явиха за вас, братя мои, че между вас имало разпри.
\par 12 С това искам да кажа, че всеки от вас дума: Аз съм Павлов; а аз Аполосов; а аз Кифов; а пък аз Христов.
\par 13 Нима се е разделил Христос? Павел ли се разпна за вас? Или в Павловото име се кръстихте?
\par 14 Благодаря Богу, че не съм кръстил никого от вас, освен Криспа и Ганя,
\par 15 да не би да каже някой, че сте били кръстени в мое име.
\par 16 Кръстих още и Стефаниновия дом; освен тия, не помня да съм кръстил никой друг.
\par 17 Защото Христос не ме е пратил да кръщавам, но да проповядвам благовестието; не с мъдри думи, да не се лиши Христовия кръст от значението си.
\par 18 Защото словото на кръста е безумие за тия, които погиват; а за нас, които се спасяваме, то е Божия сила.
\par 19 Понеже е писано: "Ще унищожа мъдростт на мъдрите, И разума на разумните ще отхвърля".
\par 20 Где е мъдрият? Где книжникът? Где е разисквачът на тоз век? Не обърна ли Бог в глупост светската мъдрост?
\par 21 Защото, понеже в Божията мъдра наредба светът с мъдростта си не позна Бога, благоволи Бог чрез глупостта на това, което се проповядва, да спаси вярващите.
\par 22 Понеже юдеите искат знамения, а гърците търсят мъдрост;
\par 23 а ние проповядваме разпнатия Христос, за юдеите съблазън, и за езичниците глупост;
\par 24 но за самите призвани, и юдеи и гърци, Христос е Божия Сила и Божия премъдрост.
\par 25 Защото Божието глупаво е по-мъдро от човеците, и Божието немощно е по-силно от човеците.
\par 26 Понеже, братя, вижте какви сте вие призваните, че между вас няма мнозина мъдри според човеците, нито мнозина силни, нито мнозина благородни.
\par 27 Но Бог избра глупавите неща на света, за да посрами мъдрите; също избра Бог немощните неща на света, за да посрами силните;
\par 28 още и долните и презрените неща на света избра Бог, да! и ония, които ги няма, за да унищожи тия които ги има,
\par 29 за да не се похвали нито една твар пред Бога.
\par 30 А от Него сте вие в Христа Исуса, Който стана за нас мъдрост от Бога, и правда, и освещение, и изкупление;
\par 31 тъй щото, както е писано, "който се хвали, с Господа да се хвали".

\chapter{2}

\par 1 И аз, братя, когато дойдох при вас, не дойдох с превъзходно говорене или мъдрост да ви известя Божията тайна;
\par 2 защото бях решил да не зная между вас нещо друго, освен Исуса Христа, и то Христа (Гръцки: Него. ) разпнат.
\par 3 Аз бях немощен между вас, страхувах се и много треперех.
\par 4 И говоренето ми, и проповядването ми не ставаха с убедителните думи на мъдростта, но с доказателство от Дух и от сила;
\par 5 за да бъде вярването ви основано не на човешка мъдрост, а на Божията сила.
\par 6 Обаче, ние поучаваме мъдрост между съвършените, ала не мъдрост от тоя век, нито от властниците на тоя век, които преминават;
\par 7 но поучаваме Божията тайнствена премъдрост, която е била скрита, която е била предопределена от Бога преди вековете да ни докарва слава.
\par 8 Никой от властниците на тоя век не я е познал; защото, ако я бяха познали, не биха разпнали Господа на славата.
\par 9 А, според както е писано: - "Каквото око не е видяло, и ухо не е чуло, И на човешко сърце не е дохождало, Всичко това е приготвил Бог за тия, които Го любят".
\par 10 А на нас Бог откри това чрез Духа; понеже Духът издирва всичко, даже и Божиите дълбочини.
\par 11 Защото кой човек знае що има у човека, освен духът на човека, който е в Него? Така и никой не знае що има у Бога, освен Божият Дух.
\par 12 А ние получихме не духа на света, но Духа, който е от Бога, за да познаем това, което Бог е благоволил да ни подари;
\par 13 което и възвестяваме, не с думи научени от човешка мъдрост, но с думи научени от Духа, като поясняваме духовните неща на духовните човеци.
\par 14 Но естествения човек не подбира това, което е от Божия Дух, защото за него е глупост; и не може да го разбере, понеже, то се изпитва духовно.
\par 15 Но духовният човек изпитва всичко; а него никой не изпитва.
\par 16 Защото, "Кой е познал ума на Господа, За да може да го научи?" А ние имаме ум Христов.

\chapter{3}

\par 1 И аз, братя, не можах да говоря на вас, като на духовни, но като на плътски, като на младенци в Христа.
\par 2 С мляко ви храних, не с твърда храна; защото още не можехте да го приемете, а и сега още не можете.
\par 3 Понеже и досега сте плътски; защото, докато има между вас завист и разпра, не сте ли плътски, и не постъпвате ли по човешки?
\par 4 Защото, кога един казва: Аз съм Павлов, а друг: Аз съм Аполосов, не сте ли като човеци слаби?
\par 5 Какво е, прочее, Аполос, и какво е Павел? Те са служители, чрез които повярвахте, и то както Господ е дал на всеки от тях.
\par 6 Аз насадих, Аполос напои, но Господ прави да расте.
\par 7 И тъй, нито който сади е нещо, нито който пои, а Господ, Който прави да расте.
\par 8 Прочее, тоя, който сади, и тоя, който пои, са равни, обаче всеки според своя труд ще получи своята награда;
\par 9 защото сме съработници на Бога, като вие сте Божия нива, Божие здание.
\par 10 Според дадената ми Божия благодат, като изкусен строител аз положих основа; а друг гради на нея.
\par 11 Защото никой не може да положи друга основа, освен положената, която е Исус Христос.
\par 12 И ако някой гради на основата злато, сребро, скъпоценни камъни, дърво, сено, слама,
\par 13 всекиму работата ще стане явна каква е; защото Господният ден ще я изяви, понеже тя чрез огън се открива; и самият огън ще изпита работата на всекиго каква е.
\par 14 Тоя, комуто работата, която е градил, устои, ще получи награда.
\par 15 А тоя, комуто работата изгори, ще претърпи загуба; а сам той ще се избави, но тъй като през огън.
\par 16 Не знаете ли, че сте храм на Бога, и че Божият Дух живее във вас?
\par 17 Ако някой развали Божия храм, него Бог ще развали; защото Божият храм е свет, който храм сте вие.
\par 18 Никой да не се лъже. Ако някой между вас мисли, че е мъдър според тоя век, нека стане глупав за да бъде мъдър.
\par 19 Защото мъдростта на тоя свят е глупост пред Бога, понеже е писано: "Улавя мъртвите в лукавството им";
\par 20 и пак: "Господ знае, разсъжденията на мъдрите са суетни".
\par 21 Затова никой да се не хвали с човеците. Защото всичко е ваше;
\par 22 било Павел, или Аполос, или Кифа, или светът, или животът, или смъртта, или сегашното, или бъдещето, всичко е ваше;
\par 23 а вие сте Христови, а Христос Божий.

\chapter{4}

\par 1 Така всеки човек да ни счита за Христови служители и настойници на Божиите тайни.
\par 2 При туй, което тук се изисква от настойниците е, всеки да се намери верен.
\par 3 А за мене е твърде малко нещо да бъда съден от вас или от човешки съд; даже аз не съдя сам себе си.
\par 4 Защото, при все че съвестта ми в нищо не ме изобличава, пак с това не съм оправдан; защото Господ е, Който ще ме съди.
\par 5 Затова недейте съди нищо преждевременно, докле не дойде Господ, Който ще извади на видело скритото в тъмнината, и ще изяви намеренията на сърцата; и тогава всеки ще получи подобаващата нему похвала от Бога.
\par 6 И това, братя, преносно приспособих към себе си и към Аполоса заради вас, за да се научите чрез нас да не престъпвате границата на писаното, да се не гордее някой от вас с един против друг.
\par 7 Защото, кой те прави да се отличаваш от другиго? И що имаш, което да не си получил? Но ако си го получил, защо се хвалиш, като че не си го получил?
\par 8 Сити сте вече, обогатихте се вече, царувате и то без нас. И дано царувате, та ние заедно с вас да царуваме;
\par 9 защото струва ми се, че Бог изложи нас, апостолите, най-последни, като човеци осъдени на смърт; защото станахме показ на света, на ангели и на човеци;
\par 10 ние безумни заради Христа, а вие разумни в Христа, ние немощни, а вие силни, вие славни, а ние опозорени.
\par 11 Ние до тоя час и гладуваме и жадуваме, и сме голи, бити сме и се скитаме.
\par 12 трудим се, работейки със своите ръце; като ни хулят, благославяме; като ни гонят, постоянствуваме;
\par 13 като ни злепоставят, умоляваме; станахме до днес като измет на света, измет на всичко.
\par 14 Не пиша това, да ви посрамя, но да ви увещая, като любезни мои чада.
\par 15 Защото, ако имахме десетки хиляди наставници в Христа, пак мнозина бащи нямате; понеже аз ви родих в Христа Исуса чрез благовестието.
\par 16 Затова ви се моля, бъдете подражатели на мене.
\par 17 По тая причина ви пратих Тимотея, който ми е възлюбено и вярно чадо в Господа; той ще ви напомни моите пътища в Христа, такива пътища, каквито получавам навсякъде във всяка църква.
\par 18 Но някои се възгордяха, като че нямаше да дойда при вас.
\par 19 Но, ако ще Господ, аз скоро ще дойда при вас, и ще изпитам, не думите, но силата на тия, които са се възгордели.
\par 20 Защото Божието царство не се състои в думи, а в сила.
\par 21 Що искате? С тояга ли да дойда при вас? Или с любов и кротък дух?

\chapter{5}

\par 1 Дори се чува, че между вас имало блудодеяние, и то такова блудодеяние, каквото нито между езичниците се намира, именно, че един от вас има бащината си жена.
\par 2 И вие сте се възгордели, вместо да сте скърбили, за да се отлъчи измежду вас тоя, който е сторил туй нещо.
\par 3 Защото аз, ако и да не съм телесно при вас, но като съм при вас с духа си, като че ли съм при вас, - осъдих вече, в името на нашия Господ Исус, оногова, който така е сторил това,
\par 4 (като се събра моят дух с вас заедно с властта на нашия Господ Исус),
\par 5 да предадем такъв човек на сатана за погубване на плътта му, за да се спаси духа му в деня на Господа Исуса.
\par 6 Хвалбата ви не е добра. Не знаете ли, че малко квас заквасва цялото тесто?
\par 7 Очистете стария квас, за да бъдете ново тесто, - тъй като сте безквасни; защото и нашата пасха, Христос, биде заклан [за нас].
\par 8 Затова нека празнуваме, не със стар квас, нито с квас от злоба и безчестие, а с безквасни хлябове от искреност и истина.
\par 9 Писах ви в посланието си да се не сношавате с блудници,
\par 10 не че съм искал да кажа за блудниците на тоя свят, или за сребролюбците и грабителите, или за идолопоклонниците, понеже тогава би трябвало да излезете от света. -
\par 11 но в действителност ви писах да се не сношавате с някого, който се нарича брат, ако е блудник, или сребролюбец, или идолопоклонник, или грабител, с такъв нито да ядете заедно.
\par 12 Защото, каква работа имам да съдя вънкашни човеци? Не съдите ли вие вътрешните.
\par 13 докато вънкашните Бог съди? Отлъчете нечестивия човек изпомежду си.

\chapter{6}

\par 1 Когато някой от вас има нещо против другиго, смее ли да се съди пред неправедните, а не пред светиите?
\par 2 Или не знаете, че светиите ще съдят света? Ако, прочее, вие ще съдите света, не сте ли достойни да съдите ни най малките работи?
\par 3 Не знаете ли, че ние ще съдим ангели? а колко повече житейски работи;
\par 4 Тогава, ако имате житейски тъжби, поставяте ли за съдии ония, които от църквата се считат за нищо?
\par 5 Казвам това за да ви направя да се засрамите. Истина ли е, че няма между вас ни един мъдър човек, който би могъл да отсъди между братята си,
\par 6 но брат с брата се съди, и то пред невярващите?
\par 7 Даже, преди всичко, е голям недостатък у вас гдето имате тъжби помежду си. Защо по-добре не оставате онеправдани? защо по-добре не бъдете ограбени?
\par 8 А напротив, вие сами онеправдавате и ограбвате, и то братя.
\par 9 Или не знаете, че неправедните няма да наследят Божието царство? Недейте се лъга. Нито блудниците, нито идолопоклонниците, нито прелюбодейците, нито малакийците, нито мъжеложниците,
\par 10 нито крадците, нито сребролюбците, нито пияниците, нито хулителите, нито грабителите няма да наследят Божието царство.
\par 11 И такива бяха някои от вас; но вие измихте себе си от такива неща, но се осветихте, но се оправдахте в името на Господа Исуса Христа и в Духа на нашия Бог.
\par 12 Всичко ми е позволено, ала не всичко е полезно; всичко ми е позволено, но не ща да съм обладан от нищо.
\par 13 Храната е за стомаха, и стомахът е за храната; но Бог ще унищожи и него и нея. А тялото не е за блудодеяние, но за Господа, и Господ е за тялото,
\par 14 а Бог, Който е възкресил Господа, ще възкреси и нас със силата Си.
\par 15 Не знаете ли, че вашите тела са части на Христа? И тъй, да отнема ли от Христа частите Му и да ги направя части на блудница? Да не бъде?
\par 16 Или не знаете, че който се съвъкупява с блудница е едно тяло с нея? защото "ще бъдат", казва, "двамата една плът".
\par 17 Но, който се съединява с Господа е един дух с Него.
\par 18 Бягайте от блудодеянието. Всеки друг грях, който би сторил човек, е вън от тялото; но който блудствува, съгрешава против своето си тяло.
\par 19 Или не знаете, че вашето тяло е храм на Светия Дух, който е във вас, когото имате от Бога? И вие не сте свои си,
\par 20 защото сте били с цена купени; затова прославяте Бога с телата си, [и с душите си, които са Божии].

\chapter{7}

\par 1 А относно това, що ми писахте: Добре е човек да се не докосва до жена.
\par 2 Но, за да се избягват блудодеянията, нека всеки мъж има своя си жена, и всяка жена да има свой мъж.
\par 3 Мъжът нека има с жената дължимото към нея сношение; подобно и жената с мъжа.
\par 4 Жената не владее своето тяло, а мъжът; така и мъжът не владее своето тяло, а жената.
\par 5 Не лишавайте един друг от съпружеско сношение, освен ако бъде по съгласие за малко време, за да се предавате на молитва, и пак бъдете заедно, да не би сатана да ви изкушава чрез вашата невъзможност да се въздържате.
\par 6 Но, това казвам, като позволение, а не като заповед.
\par 7 Обаче, бих желал всичките човеци да бъдат какъвто съм аз. Но всеки има своя особен дар от Бога, един така, а друг инак.
\par 8 А на неженените и вдовците казвам: Добро е за тях, ако си останат такива, какъвто съм и аз.
\par 9 Но, ако не могат да се въздържат, нека се женят, защото по-добре е да се женят, отколкото да се разжегват.
\par 10 А на жените заръчвам, и то не аз, но Господ: Жена да не оставя мъжа си;
\par 11 (но ако го остави, нека остане неомъжена, или нека се помири с мъжа си;) и мъж да не напуща жена си.
\par 12 А на другите казвам аз, не Господ: Ако някой брат има невярваща жена, и тя е съгласна да живее с него, да не я напуща.
\par 13 И жена, която има невярващ мъж, и той е съгласен да живее с нея, да не напуща мъжа си.
\par 14 Защото невярващият мъж се освещава чрез жената, и невярващата жена се освещава чрез брата, своя мъж; инак чедата ви щяха да бъдат нечисти, а сега са свети.
\par 15 Но, ако невярващият напусне, нека напусне; в такива случаи братът или сестрата не са поробени на брачния закон. Бог, обаче, ни е призвал към мир.
\par 16 Защото отгде знаеш жено, дали не ще спасиш мъжа си? или отгде знаеш мъжо, дали не ще спасиш жена си?
\par 17 Само нека всеки постъпва така, както Господ му е отделил сили, и всеки, както Бог го е призовал; и така заръчвам по всичките църкви.
\par 18 Обрязан ли е бил призован някой във вярата, да не крие обрязването. Необрязан ли е бил някой призован, да се не обрязва.
\par 19 Обрязването е нищо, и необрязването е нищо, но важното е пазенето на Божиите заповеди.
\par 20 Всеки нека си остава в това звание, в което е бил призван във вярата.
\par 21 В положение на роб ли си бил призован? да те не е грижа, (но ако можеш и свободен да станеш, по-добре използувай случая);
\par 22 защото, който е бил призован в Господа като роб, е свободен човек на Господа; така и който е бил призован като свободен човек, е роб на Христа.
\par 23 С цена сте били купени; не ставайте роби на човеци.
\par 24 Братя, всеки в каквото е бил призован във вярата, в него нека си остане с Бога.
\par 25 Относно девиците нямам заповед от Господа; но давам мнение като един, който е получил милост от Господа да бъде верен.
\par 26 И тъй, поради настоящата неволя, ето какво мисля за доброто, че е добре за човека така да остане както си е .
\par 27 Вързан ли си в жена? не търси развързване. Отвързан ли си от жена? не търси жена.
\par 28 Но, ако се и ожениш, не съгрешаваш; и девица, ако се омъжи, не съгрешава; но такива ще имат житейски скърби, а пък аз ви жаля.
\par 29 Това само казвам, братя, че останалото време е кратко; затова и тия, които имат жени, нека бъдат, като че нямат;
\par 30 и които плачат, като че не плачат; които се радват; като че не се радват; които купуват, като че нищо не притежават;
\par 31 и които си служат със света, като че не са преданни на него: защото сегашното състояние(Гръцки: Образът ) на тоя свят преминава.
\par 32 А аз желая вие да бъдете безгрижни. Нежененият се грижи за това, което е Господно, как да угажда на Господа;
\par 33 а жененият се грижи за това, което е световно, как да угажда на жена си.
\par 34 Тъй също има разлика между жена и девица. Неомъжената се грижи за това, което е Господно, за да бъде света и в тяло и в дух; а омъжената се грижи за това, което е световно, как да угажда на мъжа си .
\par 35 И това казвам за вашата собствена полза, не да ви държа с оглавник, но заради благоприличието, и за да служите на Господа без отвличане на ума.
\par 36 Пак, ако някой мисли, че постъпва неприлично към дъщеря си девица, ако й е минала цветущата възраст, и ако трябва така да стане, нека прави каквото ще; не съгрешава, нека се женят.
\par 37 Но който стои твърдо в сърцето си, и не бива принуден, но има власт да изпълни волята си, и е решил в сърцето си да държи дъщеря си девицата неомъжена, ще направи добре.
\par 38 Така щото, който омъжи дъщеря си девицата добре прави; а който я не омъжи, ще направи по-добре
\par 39 Жената е вързана до когато е жив мъжът й; но ако мъжът умре, свободна е да се омъжи за когото ще, само в Господа.
\par 40 Но, по моето мнение, по-щастлива е, ако си остане така; а мисля, че и аз имам Божия Дух.

\chapter{8}

\par 1 А относно идоложертването: Знам, че ние всички уж имаме знание да разрешим въпроса! Но знанието възгордява, а любовта назидава.
\par 2 Ако някой мисли, че знае нещо, той още не е познал както трябва да познава.
\par 3 Но, ако някой люби Бога, той е познат от Него.
\par 4 Прочее, относно яденето от идоложертвеното, знаем, че никакъв бог, изобразен от идол, няма на света, и че няма друг Бог освен един.
\par 5 Защото, ако и да има така наричани богове, било на небето или на земята, (както има много богове, и господари много),
\par 6 но за нас има само един Бог, Отец, от Когото е всичко, и ние за Него, и един Господ, Исус Христос, чрез Когото е всичко, и ние чрез Него.
\par 7 Но това знание го няма у всички; и някои, както и до сега имат съзнание за идолите, ядат месото като жертва на идолите; и съвестта им като слаба, се осквернява..
\par 8 А това що ядем, не ще ни препоръчва на Бога; нито ако не ядем, губим нещо; нито ако ядем, печелим нещо.
\par 9 Но внимавайте да не би по някакъв начин тая ваша свобода да стане спънка на слабите.
\par 10 Защото, ако някой види, че ти, който имаш знание, седиш на трапеза в идолско капище, не ще ли съвестта му да се одързости, ако е слаб, та и той да яде идоложертвено?
\par 11 И поради твоето знание слабият погива, братът, за когото е умрял Христос.
\par 12 А като съгрешавате така против братята, и наранявате слабата им съвест, вие съгрешавате против Христа,
\par 13 Затуй, ако това що ям съблазнява брата ми, аз няма да ям до века, за да не съблазня брата с.

\chapter{9}

\par 1 Не съм ли свободен? Не съм ли апостол? Не видях ли Исуса нашия Господ? Не сте ли вие моето дело в Господа?
\par 2 На други, ако не съм апостол, то поне на вас съм; защото в Господа вие сте печата на моето апостолство.
\par 3 Ето моето оправдание пред тия, които изпитват поведението ми;
\par 4 Нямаме ли право да ядем и да пием за сметка на църквата?
\par 5 Нямаме ли право и ние, както другите апостоли, и братята на Господа и Кифа, да водим жена от сестрите?
\par 6 Или само аз и Варнава нямаме право да не работим за прехраната си?
\par 7 Кой войник служи някога на свои разноски? Кой насажда лозе и не яде плода му? Или кой пасе стадо и не яде от млякото на стадото?
\par 8 По човешки ли говоря това? Или не казва същото и законът?
\par 9 Защото в Моисеевия закон е писано: "Да не обвързваш устата на вола, когато вършее". За воловете ли тук се грижи Бог,
\par 10 или го казва несъмнено заради нас? Да; заради нас е писано това; защото който оре, с надежда трябва да оре; и който вършее, трябва да вършее с надежда, че ще участвува в плода.
\par 11 Ако ние сме посяли у вас духовното, голямо нещо ли е , ако пожънем от вас телесното?
\par 12 Ако други участвуват в това право над вас, не участвуваме ли ние повече? Обаче, ние не използувахме това право, но търпим всичко, за да не причиним някакво препятствие на Христовото благовестие.
\par 13 Не знаете ли, че тия, които свещенодействуват, се хранят от светилището? и че тия, които служат на олтара, вземат дял от олтара?
\par 14 Така и Господ е наредил, щото проповедниците на благовестието да живеят от благовестието.
\par 15 Но аз не съм използувал ни една от тия наретби, нито пиша това, за да се направи за мене така; защото за мене е по-добре да умра, отколкото да осуети някой моята похвала.
\par 16 Защото, ако проповядвам благовестието, няма с какво да се похваля; понеже нужда ми се налага; защото горко ми ако не благовествувам.
\par 17 Понеже, ако върша това доброволно, имам награда, но ако с принуждение, то само изпълнявам повереното ми настойничество:
\par 18 И тъй, каква е моята награда? Тая, че, като проповядвам евангелието, да мога да напрявя благовестието безплатно, така щото да не използувам напълно моето право в благовестието.
\par 19 Защото, при все че съм свободен от всичките човеци, аз заробих себе си на всички, за да придобия мнозината.
\par 20 На юдеите станах като юдеин, за да придобия юдеи; на тия, които са под закон, станах като под закон, (при все че сам аз не съм под закон), за да придобия тия, които са под закон.
\par 21 На тия, които нямат закон, станах като че нямам закон, (при все че не съм без закон спрямо Христа), за да придобия тия, които нямат закон.
\par 22 На слабите станах слаб, за да придобия слабите. На всички станах всичко, та по всякакъв начин да спася неколцина.
\par 23 Всичко това върша заради благовестието, за да участвувам и аз в него.
\par 24 Не знаете ли, че, които тичат на игрището, всички тичат, а само един получава наградата? Така тичайте, щото да я получите.
\par 25 И всеки, който се подвизава, се въздържа от всичко. Те вършат това за да получат тленен венец, а ние нетленен.
\par 26 И тъй, аз така тичам, не към нещо неизвестно; така удрям, не като че бия въздуха;
\par 27 но уморявам тялото си и го поробвам, да не би, като съм проповядвал на другите, сам аз да стана неодобрен.

\chapter{10}

\par 1 Защото, братя, желая да знаете, че, макар да са били бащите ни всички под облака, и всички да са минали през морето,
\par 2 и в облака и в морето всички да са били кръстени от Моисея,
\par 3 и всички да са яли от същата духовна храна,
\par 4 и всички да са пили от същото духовно питие, (защото пиеха от една духовна канара, която ги придружаваше; и тая канара бе Христос),
\par 5 пак в повечето от тях Бог не благоволи; затова ги измори в пустинята.
\par 6 А в тия неща те ни станаха примери, та да не похотствуваме за злото, както и те похотствуваха.
\par 7 Нито бивайте идолопоклонници, както някои от тях според писаното: "Людете седнаха да ядат и да пият, и станаха да играят".
\par 8 Нито да блудствуваме, както блудствуваха някои от тях, и паднаха в един ден двадесет и три хиляди души.
\par 9 Нито да изпитваме Господа, както някои от тях Го изпитваха, и погинаха от змиите.
\par 10 Нито роптайте, както възроптаха някои от тях, и погинаха от изтребителя.
\par 11 А всичко това им се случи за примери, и се записа за поука нам, върху които са стигнали последните времена.
\par 12 Тако щото, който мисли, че стои, нека внимава да не падне.
\par 13 Никакво изпитание не ви е постигнало освен това, което може да носи човек; обаче, Бог е верен, Който няма да ви остави да бъдете изпитани повече, отколкото ви е силата, но заедно с изпитанието ще даде и изходен път, така щото да можете да го издържите.
\par 14 Затова, въблюбени мои, бягайте от идолопоклонството.
\par 15 Говоря като на разумни човеци; сами вие съдете за това, което казвам.
\par 16 Чашата, която биде благословена, и която ние благославяме, не е ли това да имаме обещание в Христовата кръв? Хлябът, който пречупваме, не е ли да имаме общение в Христовото тяло?
\par 17 тъй като ние, ако и да сме мнозина, сме един хляб, едно тяло, понеже всички в единия хляб участвуваме.
\par 18 Гледайте Израиля по плът; тия, които ядат жертвите, нямат ли общение в олтара? Тогава що?
\par 19 Казвам ли аз, че идоложертвеното е нещо, или че идолът е нещо? Не.
\par 20 Но казвам, че онова, което жертвуват езичниците, жертвуват го на бесовете, а не на Бога; но аз не желая вие да имате общение с бесовете.
\par 21 Не можете да пиете Господната чаша и бесовската чаша; не можете да участвувате в Господната трапеза и в бесовската трапеза.
\par 22 Или искаме да подбудим Господа на ревност? Ние по-силни ли сме от Него?
\par 23 Всичко е позволено, но не всичко е полезно; всичко е позволено, но не всичко е назидателно.
\par 24 Никой да не търси своята лична полза, но всеки ползата на другиго.
\par 25 Всичко, що се продава на месарницата, яжте без да изпитвате за него заради съвестта си;
\par 26 защото "Господна е земята и всичко що има в нея".
\par 27 Ако някой от невярващите ви покани на угощение, и вие желаете да отидете, яжте каквото сложат пред вас, без да изпитвате за него заради съвестта си.
\par 28 Но, ако някой ви рече: Това е било принесено в жертва, не яжте, заради тогова, който ви е известил, и заради съвестта, -
\par 29 съвест, казвам, не твоята, но на другия; (понеже, защо да се съди моята свобода от друга съвест?)
\par 30 Ако аз с благодарение Богу участвувам в яденето, защо да ме злословят за онова, за което благодаря?)
\par 31 И тъй, ядете ли, пиете ли, нещо ли вършите, всичко вършете за Божията слава.
\par 32 Не ставайте съблазън ни на юдеи, ни на гърци, нито на Божията църква;
\par 33 както и аз угождавам на всички във всичко, като търся не своята си полза, но ползата на мнозина, за да се спасят.

\chapter{11}

\par 1 Бивайте подражатели на мене, както съм и аз на Христа.
\par 2 А похвалявам ви, че ме помните за всичко, като държите преданията тъй, както ви ги предадох.
\par 3 Но желая да знаете, че глава на всеки мъж е Христос, а глава на жената е мъжът, глава пък на Христа е Бог.
\par 4 Всеки мъж, който се моли или пророкува с покрита глава, засрамва главата си.
\par 5 А всяка жена, която се моли или пророкува гологлава, засрамва главата си, защото това е едно и също, като да е с бръсната глава.
\par 6 Защото, която жена се не покрива, нека остриже и косата си. Но ако е срамотно за жена да си стриже косата, или да си бръсне главата, то нека се покрива.
\par 7 Защото мъжът не трябва да си покрива си покрива главата, понеже е образ и слава на Бога; а жената е слава на мъжа.
\par 8 (Защото мъжът не е от жената, а жената е от мъжа;
\par 9 понеже, мъжът не бе създаден за жената, а жената за мъжа).
\par 10 Затова жената е длъжна да има на главата си белег на власт, заради ангелите.
\par 11 (Обаче, нито жената е без мъжа, нита мъжът без жената, в Господа;
\par 12 защото, както жената е от мъжа, така и мъжът е чрез жената; а всичко е от Бога).
\par 13 Сами в себе си съдете: Прилично ли е, жената да се моли Богу гологлава?
\par 14 Не учи ли и самото естество, че, ако мъж оставя косата си да расте, това е позор за него,
\par 15 но, ако жена оставя косата си да расте, това е слава за нея, защото косата й е дадена за покривало?
\par 16 Но, ако някой мисли да се препира за това, - ние нямаме такъв обичай, нито Божиите църкви.
\par 17 А като ви заръчвам следното, не ви похвалявам, защото се събирате, не за по-добро, но за по-лошо.
\par 18 Защото, първо, слушам, че когато се събирате в църква, ставали разделения помежду ви; (и отчасти вярвам това;
\par 19 защото е нужно да има и разцепление между вас, за да се яви, кои са удобрените помежду ви);
\par 20 прочее, когато така се събирате заедно, не е възможно да ядете Господната вечеря;
\par 21 защото на яденето всеки бърза да вземе своята вечеря преди другиго; и така един остава гладен, а друг се напива.
\par 22 Що! къщи ли нямате, гдето да ядете и пиете? Или презирате Божията църква и посрамяте тия, които нямат нищо? Що да ви кажа? Да ви похваля ли за това? Не ви похвалявам.
\par 23 Защото аз от Господа приех това, което ви и предадох, че Господ Исус през нощта, когато беше предаден, взе хляб,
\par 24 и, като благодари, разчупи и рече: Това е Моето Тяло, което е [разчупено] за вас; туй правете за Мое възпоменание.
\par 25 Така взе и чашата след вечерята и рече: Тая чаша е новият завет в Моята кръв; това правете всеки път, когато пиете, за Мое възпоменание.
\par 26 Защото всеки път, когато ядете тоя хляб и пиете [тая] чаша, възвестявате смъртта на Господа, докле дойде Той.
\par 27 Затова, който яде хляба или пие Господната чаша недостойно, ще бъде виновен за грях против тялото и кръвта на Господа.
\par 28 Но да изпитва човек себе си, и така да от хляба и да пие от чашата;
\par 29 защото, който яде и пие без да разпознава Господното тяло, той яде и пие осъждане на себе си.
\par 30 По тая причина мнозина между вас са слаби и болнави, а доста и са починали.
\par 31 Но, ако разпознавахме сами себе си, не щяхме да бъдем съдени заедно със света.
\par 32 А когато биваме съдени от Господа, с това се наказваме, за да не бъдем осъдени заедно със света.
\par 33 Затова, братя мои, когато се събирате да ядете, чакайте се един друг.
\par 34 Ако някой е гладен, нека яде у дома си, за да не бъде събирането ви за осъждане. А за останалите работи, ще ги наредя, когато дойда.

\chapter{12}

\par 1 При това, братя, желая да разбирате и за духовните дарби.
\par 2 Вие знаете, че когато бяхте езичници, отвличахте се към нямите идоли, както и да ви водеха.
\par 3 Затова ви уведомявам, че никой, като говори с Божия Дух, не казва: Да бъде проклет Исус! и никой не може да нарече Исуса Господ, освен със Светия Дух.
\par 4 Дарбите са различни; но Духът е същият.
\par 5 Службите са различни; но Господа е същият.
\par 6 Различни са и действията; но Бог е същият. Който върши всичко във всичките човеци.
\par 7 А на всеки се дава проявлението на Духа за обща полза.
\par 8 Защото на един се дава чрез Духа да говори с мъдрост, а на друг да говори със знание, чрез същия Дух;
\par 9 на друг вяра чрез същия Дух, а пък на друг изцелителни дарби чрез единия дух;
\par 10 на друг да върши велики дела, а на друг да пророкува; на друг да разпознава духовете; на друг да говори разни езици; а пък на друг да тълкува езици.т
\par 11 А всичко това се върши от един и същи Дух, който разделя на всеки по особено, както му е угодно.
\par 12 Защото, както тялото е едно, а има много части, и всичките части на тялото, ако и да са много, пак са едно тяло, така е и Христос.
\par 13 Защото ние всички, било юдеи или гърци, било роби или свободни, се кръстихме в един Дух да съставляваме едно тяло, и всички от един Дух се напоихме.
\par 14 Защото тялото не се състои от една част, а от много.
\par 15 Ако речеше ногата: Понеже не съм ръка, не съм от тялото, това не я прави да не е от тялото.
\par 16 И ако рече ухото: Понеже не съм око, не съм от тялото, това не го прави да не е от тялото.
\par 17 Ако цялото тяло беше око, где щеше да е слухът? Ако цялото тяло беше слух, где щеше да е обонянието?
\par 18 Но сега Бог е поставил частите, всяка една от тях, в тялото, както му е било угодно.
\par 19 Пак, ако те бяха всички една част, где щеше да е тялото?
\par 20 Но сега те са много части, а едно тяло.
\par 21 И окото не може да рече на ръката: Не ми трябваш;или пък главата на нозете: Не сте ми потребни.
\par 22 Напротив, тия части на тялото, които се виждат да са по-слаби, са необходими;
\par 23 и тия части на тялото, които ни се виждат по-малко честни, тях обличаме с повече почит; и неблагоприличните ни части получават най-голямо благоприличие.
\par 24 А благоприличните ни части нямат нужда от това. Но Бог е сглобил тялото така, че е дал по-голяма почит на оная част, която не я притежава;
\par 25 за да няма раздор в тялото, но частите му да се грижат еднакво една за друга.
\par 26 И ако страда една част, всичките части страдат с нея; или ако се слави една част, всичките части се радват заедно с нея.
\par 27 А вие сте от Христово тяло, и по отделно части от Него.
\par 28 И Бог е поставил някои в църквата да бъдат: първо апостоли, второ пророци, трето учители, други да правят чудеса, някои имат изцелителни дарби, други с дарби на помагания, на управлявания, на говорене разни езици.
\par 29 Всички апостоли ли са? всички пророци ли са? всички учители ли са? всички вършат ли велики дела?
\par 30 Всички имат ли изцелителни дарби? всички говорят ли езици? всички тълкуват ли?
\par 31 Копнейте за по-големи дарби; а при все това аз ви показвам един превъзходен път.

\chapter{13}

\par 1 Ако говоря с човешки и ангелски езици, а любов нямам, аз съм станал мед що звънти, или кимвал що дрънка.
\par 2 И ако имам пророческа дарба, и зная всички тайни и всяко знание, и ако имам пълна вяра, тъй щото планини да премествам, а любов нямам, нищо не съм.
\par 3 И ако раздам всичкия си имот за прехрана на сиромасите, и ако предам тялото си на изгаряне, а любов нямам, никак не ме ползува.
\par 4 Любовта дъблго търпи и е милостива; любовта не завижда; любовта не се превъзнася, не се гордее,
\par 5 не безобразничи, не търси своето, не се раздразнява, не държи сметка за зло,
\par 6 не се радва на неправдата, а се радва заедно с истината,
\par 7 всичко премълчава, на всичко хваща вяра, на всичко се надява, всичко търпи.
\par 8 Любовта никога не отпада; другите дарби, обаче, пророчества ли са, ще се прекратят; езици ли са, ще престанат; знание ли е, ще се прекрати.
\par 9 Защото отчасти знаем и отчасти проракуваме;
\par 10 но когато дойде съвършеното, това, което е частично, ще се прекрати.
\par 11 Когато бях дете, като дете говорех, като дете чувствувах, като дете разсъждавах; откак станах мъж, напуснал съм детинското.
\par 12 Защото сега виждаме нещата неясно, като в огледало, а тогава ще ги видим лице с лице; сега познавам отчасти, а тогава ще познавам напълно, както и съм бил напълно познат.
\par 13 И тъй, остават тия трите: вяра, надежда и любов; но най-голяма от тях е любовта.

\chapter{14}

\par 1 Следвайте любовта; но копнейте и за духовните дарби, а особено за дарбата да пророкувате.
\par 2 Защото, който говори на непознат език, той не говори на човеци, а на Бога, защото никой не му разбира, понеже с духа си говори тайни.
\par 3 А който пророкува, той говори но човеци за назидание, за уважение и за утеха.
\par 4 Който говори на непознат език, назидава себе си; а който пророкува, назидава църквата.
\par 5 Желал бих всички вие да говорите езици, а повече да пророкувате; и който пророкува, е по-горен от този, който говори разни езици, освен ако тълкува, за да се назидава църквата.
\par 6 Кажете сега, братя, ако дойда при вас и говоря непознати езици, какво ще ви ползувам, ако не ви съобщя или някое откровение, или знание, или пророчество, или поука?
\par 7 Даже и бездушните неща, като свирка и гъдулка, когато издават глас, ако не издават отличителни звукове, как ще се познае това, което свирят със свирката или с гъдулката?
\par 8 Защото тръбата ако издадеше неопределен глас, кой би се приготвил за бой?
\par 9 Също така, ако вие не изговаряте с гласа си думи с някакво значение, как ще се знае какво говорите? защото ще говорите на вятъра.
\par 10 Има, може да се каже, толкова вида гласове на света; и ни един от тях не е без значение.
\par 11 Ако, прочее, не разбера значението на гласа, ще бъда другоезичен за този, който говори; И тоя, който говори ще бъде другоезичен за мене.
\par 12 Така и вие, понеже копнеете за духовните дарби, старайте се да се преумножат те у вас за назидание на църквата.
\par 13 Затова, който говори на непознат език, нека се моли за дарбата и да тълкува.
\par 14 Защото, ако се моли на непознат език, духът ми се моли а умът ми не дава плод.
\par 15 Тогава що? Ще се моля с духа си, но ще се моля и с ума си; ще пея с духа си, но ще пея и с ума си.
\par 16 Иначе, ако славословиш с духа си как ще рече: Амин, на твоето блгодарение оня, който е в положението на простите, като знае що говориш?
\par 17 Защото ти наистина благодариш добре, но другият не се назидава.
\par 18 Благодаря Богу, че аз говоря повече езици от всички ви;
\par 19 обаче, в църквата предпочитам да изговоря пет думи с ума си, за да наставя и други, а не десет хиляди думи на непознат език.
\par 20 Братя, не бивайте деца по ум, но, бидейки дечица по злобата, бивайте пълнолетни по ум.
\par 21 В закона е писано: "Чрез другоезични човеци и чрез устните на чужденци ще говоря на тия люде; и нито така ще Ме послушат", казва Господ.
\par 22 Прочее, езиците са белег не за вярващите, а за невярващите; а пророчеството е белег не за невярващите, а за вярващите.
\par 23 И тъй, ако се събере цялата църква, и всички говорят на непознати езици, и влизат хора прости или невярващи, не ще ли кажат, че вие сте полудели?
\par 24 Но ако всички пророкуват, и влезе някой невярващ или прост, той се обвинява от всички, и се осъжда от всички;
\par 25 тайните на сърцето му стават явни; и тъй, той ще падне на лицето си, ще се поклони Богу и ще изповядва, че наистина Бог е между вас.
\par 26 Тогава, братя, що става между вас? Когато се събирате, всеки има да предлага псалом, има поучение, има откровение, има да говори непознат език, има тълкувание. Всичко да става за назидание.
\par 27 Ако някой говори на непознат език, нека говорят по двама, или най-много по трима, и то по ред; а един да тълкува.
\par 28 Но ако няма тълкувател, такъв нека мълчи в църква, и нека говори на себе си и на Бога.
\par 29 От пророците нека говорят само двама или трима, а другите да разсъждават.
\par 30 Ако дойде откровение на някой друг от седящите, първият нека млъква.
\par 31 Защото един след друг всички можете да пророкувате, за да се поучават всички и всички да се насърчават;
\par 32 и духовете на пророците се покоряват на самите пророци.
\par 33 Защото Бог не е Бог на безредие, а на мир, както и поучавам по всичките църкви на светиите.
\par 34 Жените нека мълчат в църквите, защото не им е позволено да говорят; а нека се подчиняват, както казва законът.
\par 35 Ако искат да научат нещо, нека питат мъжете си у дома; защото е срамотно жена да говори в църква.
\par 36 Що? Божието слово от вас ли излезе? Или само до вас ли е достигнало?
\par 37 Ако някой мисли, че е пророк или духовеня, нека признае, че това, което ви пиша е заповед от Господа.
\par 38 Но ако някой не иска да признае, нека не признае.
\par 39 Затова, братя мои, копнейте за дарбата да пророкувате, и не забранявайте да се говорят и езици.
\par 40 Обаче, всичко нека става с приличие и ред.

\chapter{15}

\par 1 Още, братя, напомнювам ви благовестието, което ви проповядвах, което приехте, в което стоите,
\par 2 чрез което се и спасявате, ако го държите според както съм ви го благовестил, - освен ако сте напразно повярвали.
\par 3 Защото първо ви предадох онова, което приех, че Христос умря за греховете ни според писанията;
\par 4 че бе погребан; че биде възкресен на третия ден според писанията;
\par 5 и че се яви на Кифа, после на дванадесетте,
\par 6 че след това се яви на повече от петстотин братя наведнъж, от които повечето и досега са живи, а някои починаха;
\par 7 че после се яви на Якова, тогава на всички апостоли;
\par 8 а най-после от всички яви се на мене, като на някой изверг.
\par 9 Защото аз съм най-нищожният от апостолите, който не съм достоен и апостол да се нарека защото гоних Божията църква,
\par 10 Но с Божията благодат съм каквото съм; и дадената на мене Негова благодат не бе напразно, но трудих се повече от всички тях, - не аз, обаче, но Божията благодато, която беше с мене.
\par 11 И тъй, било че аз се трудих повече, било че те, така проповядвахме и те и аз, и вие така сте повярвали.
\par 12 Ако се проповядва, че Христос е възкресен от мъртвите, как казват някои между вас, че няма възкресение на мъртвите?
\par 13 Ако няма възкресение на мъртвите, то нито Христос е бил възкресен;
\par 14 и ако Христос не е бил възкресен, то празна е нашата проповед, празна е нашата вяра.
\par 15 При това, ние се намираме лъжесвидетели Божии; защото свидетелствувахме за Бога, че е възкресил Христа, Когото не е възкресил, ако е тъй, че мъртвите не възкресяват;
\par 16 защото, ако мъртвите не се възкресяват, то нито Христос е бил възкресен;
\par 17 и ако Христос не е бил възкресен, суетна е вашата вяря, вие сте още в греховете си.
\par 18 Тогава и тия, които са починали в Христа, са погинали.
\par 19 Ако само в тоя живот се надяваме на Христа, то от всичките човеци ние сме най-много за съжаление.
\par 20 Но сега Христос е бил възкесен, първият плод на починалите.
\par 21 Понеже, както чрез човека дойде смъртта, така чрез човека дойде възкресението на мъртвите.
\par 22 Защото, както в Адама всички умират, така и в Христа всички ще оживеят.
\par 23 Но всеки на своя ред: Христос първия плод, после, при пришествието на Христа, тия, които са Негови;
\par 24 Тогава ще бъде краят, когато ще предаде царството на Бога и Отца, след като унищожи всяко началство и всяка власт и сила.
\par 25 Защото Той трябва да царува, докато положи всички врагове под нозете Си.
\par 26 И смъртта, най-последен враг, и тя ще бъде унищожена,
\par 27 защото Бог "е покорил всичко под нозете Му" А когато ще е казъл, че всичко е вече покорено, (с явно изключение на Този, Който Му е покорил всичко),
\par 28 когато казвам, ще Му е било покорено всичко, тогава и Сам Синът ще покори Този, Който Му е покорил всичко, за да бъде Бог все във все.
\par 29 Иначе, какво ще правят тия, които се кръщават заради мъртвите? Ако мъртвите никак не се възкресяват, защо и ще се кръщават заради тях?
\par 30 Защо и ние се излагаме на бедствия всеки час?
\par 31 Братя, с похвалата, с която се гордея за вас в Христа Исуса нашия Господ, аз всеки ден умирам.
\par 32 Ако, по човешки казано, съм се борил с зверовете в Ефес, какво ме ползува? Ако мъртвите не се възкресяват "нека ядем и пием, защото утре ще умрем".
\par 33 Не се мамете. "Лошите другари покваряват добрите нрави".
\par 34 Отрезнявайте към правдата. и не съгрешавайте, защото някои от вас не познават Бога. Това казвам, за да ви направя да се засрамите.
\par 35 Но някой ще рече; Как се възкресяват мъртвите? и с какво тяло ще дойдат?
\par 36 Безумецо, това което ти сееш, не оживявя, ако не умира.
\par 37 И когато го сееш, не посяваш тялото, което ще изникне, а голо зърно, каквото се случи, пшенично или някое друго;
\par 38 но Бог му дава тяло, каквото му е угодно, и на всяко семе собственото му тяло.
\par 39 Всяка плът не е еднаква; но друга е плътта на човеците, а друга на животните, друга пък на птиците и друга на рибите.
\par 40 Има и небесни тела и земни тела, друга е, обаче, славата на небесните, друга на земните.
\par 41 Друг е блясъкът на слънцето, друг на луната и друг блясъкът на звездите; па и звезда от звезда се различава по блясъка.
\par 42 Така е и възкресението на мъртвите. Тялото се сее в тление, възкръсва в нетление;
\par 43 сее се в безчестие, възкръсва в слава; сее се в немощ, възкръсва в сила;
\par 44 сее се одушевено тяло, възкръсва духовно тяло. Ако има одушевено тяло, то има и духовно тяло.
\par 45 Така е и писано: Първият човек Адам "стана жива душа", а последният Адам стана животворещ дух.
\par 46 Обаче, не е първо духовното, а одушевеното, и после духовното.
\par 47 Първият човек е от земята, пръстен; вторият човек е от небето.
\par 48 Какъвто е пръстният, такива са и пръстните; и какъвто е небесният, такива са и небесните.
\par 49 И както сме се облекли в образа на пръстния, ще се облечем и в образа на небесния.
\par 50 А това казвам, братя, че плът и кръв не могат да наследят Божието царство, нито тленното наследява нетлението.
\par 51 Ето, една тайна ви казвам: Не всички ще починем, но всички ще се изменим,
\par 52 в една минута, в миг на око, при последната тръба; защото тя ще затръби, и мъртвите ще възкръснат нетленни, и ние ще се изменим.
\par 53 Защото това тленното трябва да се облече в нетление, и това смъртното да се облече в безсмъртие.
\par 54 А когато това тленното се облече в безсмъртие, тогава ще се сбъдне писаното слово: "Погълната биде смъртта победоносно".
\par 55 О смърте, где ти е победата? О смърте, где ти е жилото?
\par 56 Жилото на смъртта е грехът, и силата на греха е законът;
\par 57 но благодарение Богу, Който ни дава победа чрез нашия Господ Исус Христос.
\par 58 Затова възлюбени мои братя, бъдете твърди, непоколебими, и преизобилвайте всякога в Господното дело, понеже знаете, че в Господа трудът ви не е празден.

\chapter{16}

\par 1 А колкото за събирането на милостинята за светиите, правете и вие както наредих в галатийските църкви.
\par 2 В първия ден на седмицата всеки от вас да отделя според успеха на работите си, и да го има при себе си, за да не стават събирания, когато дойда.
\par 3 И когато дойда, ще изпратя с писма ония, които ще одобрите, да отнесат подаръка ви в Ерусалим;
\par 4 и ако заслужавам да отида и аз, те ще отидат с мене.
\par 5 Защото ще дойда при вас след като мина през Македония; (понеже през Македония минавам);
\par 6 а може би да поостана при вас, или даже и да презимувам, за да ме изпратите вие на където отида.
\par 7 Защото не ми се иска да ви видя сега на заминаване; но надявам се да остана при вас известно време, ако позволи Господ.
\par 8 А ще се бавя в Ефелс до Петдесетницата,
\par 9 защото ми се отвориха големи врата за работа, има и много противници.
\par 10 Ако дойде Тимотей, внимавайте да бъде без страх между вас; защото и той работи Господното дело както и аз;
\par 11 затова, никой да не го презира. Но изпратете го с мир да дойде при мен, защото го очаквам с братята.
\par 12 А колкото за брат Апостола, много му се молих да дойде при вас с братята; но никак не му се искаше да дойде сега; ще дойде, обаче, когато намери случай.
\par 13 Бдете, стойте твърдо във вярата си, бъдете мъжествени, укрепявайте се.
\par 14 Всичко у вас да става с любов.
\par 15 Още ви моля, братя: вие знаете, че Стефаниновото семейство е първият плод на Ахаия, и че те са посветили себе си да служат на светиите;
\par 16 добре, на такива да се подчинявате и вие, и на всеки, който помага в делото и се труди.
\par 17 Радвам се за дохождането на Стефанина, на Фортуната и на Ахаика, защото те запълниха лишението ми от власт;
\par 18 понеже успокоиха моя дух и вашия, затова признавайте такива човеци.
\par 19 Поздравяват ви църквите, които са в Азия. Нарочно ви поздравяват в Господа Акила и Прискила с домашната си църква.
\par 20 Поздравяват ви всички братя. Поздравете се един друг със света целувка.
\par 21 Поздравът пиша аз, Павел, със собствената си ръка.
\par 22 Който не люби Господа да бъде проклет. Господ наш иде.
\par 23 Благодатта на Господа Исуса Христа да бъде с вас.
\par 24 Любовта ми да бъде с всички вас в Христа Исуса. Амин.

\end{document}