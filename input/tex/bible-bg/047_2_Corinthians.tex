\begin{document}

\title{2 Corinthians}


\chapter{1}

\par 1 Павел, с Божията воля апостол Исус Христос, и брат Тимотей, до Божията църква, която е в Коринт, и до всички светии, които са по цяла Ахаия:
\par 2 Благодат и мир да бъде на вас от Бога, нашия Отец и Господа Исуса Христа.
\par 3 Благословен Бог и Отец на нашия Господ Исус Христос, Отец на милостивите и Бог на всяка утеха.
\par 4 Който ни утешава във всяка наша скръб, за да можем и ние да утешаваме тия, които се намират в каквато и да била скръб, с утехата, с която и ние се утешаваме от Бога.
\par 5 Защото, както изобилват в нас Христовите страдания, така и нашата утеха изобилва чрез Христа.
\par 6 Но, ако ни наскърбяват, това е за ваша утеха и спасение, или ако ни утешават, това е за вашата утеха [и спасение], която действува да устоявате в същите страдания, които понасяме и ние.
\par 7 И надеждата ни за вас е твърда; понеже знаем, че, както сте участници в страданията, така сте и в утехата.
\par 8 Защото желаем да знаете, братя, за скръбта, която ни сполетя в Азия, че се отеготихме чрезмерно вън от силата си, така щото отчаяхме се дори за живота си:
\par 9 даже ние сами счетохме, че бяхме приели смъртна присъда в себе си, - за да не уповаваме на себе си, но на Бога, Който възкресява мъртвите.
\par 10 И Той ни избави от толкоз близка смърт, и още избавя, и надяваме се на Него, че пак ще ни избави,
\par 11 като ни съдействувате вие чрез молитва, тъй щото, поради стореното на нас чрез мнозина добро, да благодарят мнозина за нас.
\par 12 Защото нашата похвала е тая, свидетелството на нашата съвест, че ние живяхме на света, а най-много между вас, със светост и искреност пред Бога, не са плътска мъдрост, а са Божия благодат.
\par 13 Защото не ви пишем друго освен това, което четете и даже признавате и което надявам се че и до край ще признавате,
\par 14 (както и отчасти ни признахте), че сме похвала за вас, както и вие за нас, в деня на нашия Господ Исус.
\par 15 С тая увереност възнамерявам да дойда първо при вас, за да имате двояка полза,
\par 16 като през вас мина за Македония; а от Македония да дойда пак при вас, и тогава вие да ме изпратите за Юдея.
\par 17 Добре, когато имах това намерение, лекоумно ли съм постъпил? или намерението ми е било плътско намерение, та да казвам и: Да, да, и: Не, не?
\par 18 Заради Божията вярност, проповядването ми към вас не е било Да и Не?
\par 19 Защото Божият Син, Исус Христос, Който биде проповядван помежду ви от нас, (от мене, Сила и Тимотея), не стана Да и Не но в него стана Да;
\par 20 понеже в Него е Да за всичките Божи обещания, колкото и много да са; затова и чрез Него е Амин, за Божията слава чрез нас.
\par 21 А тоя, който ни утвърждава заедно с вас в Христа, и Който ни е помазал, е Бог,
\par 22 Който ни е запечатил, и е дал в сърдцата ни Духа в залог.
\par 23 Но аз призовавам Бог за свидетел на моята душа, че за да ви пощадя, въздържах се да дойда в Коринт;
\par 24 защото не господаруваме над вярата ви, но сме помощници на радостта ви; понеже, колкото за вярата, вие стоите твърди.

\chapter{2}

\par 1 Обаче това реших, заради себе си, да не дохождам при вас със скръб.
\par 2 Защото, ако аз ви наскърбявам, то кой ще развесели мене, ако не тоя, който е бил наскърбен от мене?
\par 3 И това писах нарочно, да не би кога дойда да бъда наскърбен от ония, които би трябвало да ме зарадват, като имам уверение във всички ви, че моята радост е радост на всички ви.
\par 4 Защото от голяма скръб и сърдечна тъга ви писах с много сълзи, не за да се наскърбите, а за да познаете любовта, която питая особено към вас.
\par 5 Но ако някой ме е наскърбил, не е наскърбил само мене, но всички ви отчасти (да не кажа премного).
\par 6 За такъв един доста е наказанието, което му е било наложено от повечето от вас;
\par 7 така че сега вече ( Гръцки: Напротив.) вие трябва по-добре да му простите и да го утешите, да не би такъв да бъде погълнат от чрезмерна скръб.
\par 8 Затова ви моля да го уверите в любовта си към него.
\par 9 Понеже затова и писах, за да ви позная чрез опит, дали сте послушни във всичко.
\par 10 А комуто вие прощавате нещо, прощавам и аз; защото, ако съм и простил нещо, простих го заради вас пред Христа,
\par 11 да не би сатана да използува случая против нас; защото ние знаем неговите замисли.
\par 12 А когато дойдох в Троада да проповядвам Христовото благовестие, и когато ми се отвори врата в Господното дело,
\par 13 духът ми не се успокои, понеже не намерих брат си Тита, а, като се простих с тях, отпътувах за Македония.
\par 14 Но благодарение Богу, Който винаги ни води в победително шествие в Христа, и на всяко място изявява чрез нас благоуханието на познанието на Него.
\par 15 Защото пред Бог ние сме Христово благоухание за тия, които се спасяват, и за ония, които погиват.
\par 16 На едните сме смъртоносно ухание, което докарва смърт, а на другите животворно ухание, което докарва живот. И за това дело кой е способен?
\par 17 Ние сме, защото не сме като мнозина, които изопачават Божието слово, но говорим искрено в Христа, като от Бога, пред Бога.

\chapter{3}

\par 1 Пак ли започваме да се препоръчваме? Или имаме нужда, както някои, от препоръчителни писма до вас или от вас?
\par 2 Вие сте нашето писмо, написано в сърдцата ни, узнавано и прочитано от всичките човеци;
\par 3 и явявате се, че сте Христово писмо, произлязло чрез нашето служение, написано, не с мастило, но с Духа на живия Бог, не на плочи от камък, но на плочи от плът - на сърдцето.
\par 4 Такава увереност имаме спрямо Бога чрез Христа.
\par 5 Не че сме способни от само себе си да съдим за нещо като от нас си; но нашата способност е от Бога,
\par 6 Който ни направи способни като служители на един нов завет, - не на буквата, но на духа; защото буквата убива, а духът оживотворява.
\par 7 Но, ако служението на онова, което докарва смърт, написано с букви, издълбани на камък, стана с такава слава, щото израилтяните не можеха да гледат Моисея в лице, поради блясъка на лицето му, който впрочем преминаваше,
\par 8 как не ще бъде служението на духа с по-голяма слава?
\par 9 Защото, ако служението на онова, което докарва осъждане, стана със слава, служението на онова, което докарва правда, го надминава много повече в слава.
\par 10 (И наистина, онова, което е било прославено, изгуби славата си в това отношение, поради славата, която превъзхожда).
\par 11 Защото, ако това, което преминаваше, бе със слава, то много по-славно е трайното.
\par 12 И тъй, като имаме такава надежда, говорим с голяма откровеност,
\par 13 и не сме като Моисея, който туряше покривало на лицето си, за да не могат израилтяните да гледат изчезването на това, което преминаваше.
\par 14 Но техните умове бяха заслепени; защото и до днес, когато прочитат Стария завет, същото покривало остава, като не им е открито, че тоя завет преминава в Христа.
\par 15 А и до днес, при прочитането на Моисея, покривало лежи на сърдцето им,
\par 16 но когато Израил се обърне към Господа, покривалото ще се снеме.
\par 17 А Господ е Духът; и гдето е Господният Дух, там е свобода.
\par 18 А ние всички, с открито лице, като в огледало, гледайки Господната слава, се преобразяваме в същия образ, от слава в слава, както от Духа Господен.

\chapter{4}

\par 1 Затуй, като имаме това служение, както и придобихме милост, не се обезсърдчаваме;
\par 2 но се отрекохме от тайни и срамотни дела, и не постъпваме лукаво, нито изопачаваме Божието слово, но, като изявяваме истината, препоръчваме себе си на съвестта на всеки човек пред Бога.
\par 3 Но ако благовестието, което проповядваме, е покрито, то е покрито за тия, които погиват, -
\par 4 за тия, невярващите, чийто ум богът на този свят е заслепил, за да ги не озари светлината от славното благовестие на Христа, Който е образ на Бога.
\par 5 (Защото ние не проповядваме себе си, но Христа Исуса като Господ, а себе си като ваши слуги заради Исуса).
\par 6 Понеже Бог, Който е казал на светлината да изгрее из тъмнината, Той е, Който е огрял в сърдцата ни, за да се просвети света с познаването на Божията слава в лицето Исус Христово.
\par 7 А ние имаме това съкровище в пръстни съдове, за да се види, че превъзходната сила е от Бога, а не от нас.
\par 8 Угнетявани сме отвсякъде, но не сме утеснени; в недоумение сме, но не до отчаяние;
\par 9 гонени сме, но не оставени; повалени сме, но не погубени.
\par 10 Всякога носим на тялото си убиването на [Господа] Исуса, за да се яви на тялото ни и живота на Исуса.
\par 11 Защото ние живите винаги сме предавани на смърт за Исуса, за да се яви и живота на Исуса в нашата смъртна плът.
\par 12 Така щото смъртта действува в нас, а животът във вас.
\par 13 А като имаме същият дух на вяра, според писаното: "Повярвах, за това и говорих", то и ние, понеже вярваме, затова и говорим;
\par 14 понеже знаем, че Тоя, Който е възкресил Господа Исуса, ще възкреси и нас заедно с Исуса, и ще ни представи заедно с вас.
\par 15 Защото всичко това е заради вас, тъй щото благодатта, увеличена чрез мнозината, които са я получили, да умножи благодарението, за Божията слава.
\par 16 Затова ние не се обезсърдчаваме: но ако и да тлее външният наш човек, пак вътрешният всеки ден се подновява.
\par 17 Защото нашата привременна лека скръб произвежда все повече и повече една вечна тежина на слава за нас,
\par 18 които не гледаме на видимите, но на невидимите; защото видимите са временни, а невидимите вечни.

\chapter{5}

\par 1 Защото знаем, че ако се развали земният ни дом, телесната скиния, имаме от Бога здание на небесата, дом неръкотворен, вечен.
\par 2 Понеже в тоя дом и стенем като ожидаме да се облечем с нашето небесно жилище,
\par 3 стига само, облечени с него, да не се намерим голи.
\par 4 Защото ние, които сме в тая телесна скиния, като обременени, стенем; не че желаем да се съблечем, но да се облечем още повече, за да бъде смъртното погълнато от живота.
\par 5 А Бог е, Който ни е образувал нарочно за това, и ни е дал Духа в залог на това.
\par 6 И тъй, понеже винаги се одързостяваме, като знаем, че, докато сме у дома в тялото, ние сме отстранени от Господа,
\par 7 (защото с вярване ходим, а не с виждане).
\par 8 - понеже, казвам, се одързостяваме, то предпочитаме да сме отстранени от тялото и да бъдем у дома при Господа.
\par 9 Затова и ревностно се стараем, било у дома или отстранени, да бъдем угодни на Него.
\par 10 Защото всички трябва да застанем открити пред Христовото съдилище, за да получи всеки според каквото е правил в тялото, било добро или зло.
\par 11 И тъй, като съзнаваме, що е страхът от Господа, убеждаваме човеците; а на Бога сме познати, - надявам се още, че и на вашите съвести сме познати.
\par 12 С това не препоръчваме себе си изново на вас, но ви даваме причини да се хвалите с нас, за да имате що да отговорите на ония, които се хвалят с това, което е на лице, а не с това, което е в сърдцето.
\par 13 Защото, ако сме отишли до крайности, то е било за Бога; или ако сега умерено мъдруваме, то е заради вас.
\par 14 Защото Христовата любов ни принуждава, като разсъждаваме така, че, понеже един е умрял за всичките, то всички са умрели;
\par 15 и че Той умря за всички, за да не живеят вече живите за себе си, но за Този, Който за тях е умрял и възкръснал.
\par 16 Затова, отсега нататък ние не познаваме никого по плът; ако и да сме познали Христа по плът, пак сега вече така Го не познаваме.
\par 17 За туй, ако е някой в Христа, той е ново създание; старото премина; ето, [всичко] стана ново.
\par 18 А всичко е от Бог, Който ни примири със Себе Си чрез [Исуса] Христа, и даде на нас да служим за примирение;
\par 19 сиреч, че Бог в Христа примиряваше света със Себе Си, като не вменяваше на човеците прегрешенията им, и че повери на нас посланието на примирението.
\par 20 И тъй от Христова страна сме посланици, като че Бог чрез нас умолява; молим ви от Христова страна, примирете се с Бога,
\par 21 Който за нас направи грешен ( Гръцки: Грях ) Онзи, Който не е знаел грях, за да станем ние чрез Него праведни ( Гръцки: Правда ) пред Бога.

\chapter{6}

\par 1 И ние, като съдействуваме с Бога, тоже ви умоляваме да не приемате напраздно Божията благодат.
\par 2 (Защото казва: - " В благоприятно време те послушах, И в спасителен ден ти помогнах"; ето, сега е благоприятно време, ето, сега е спасителен ден).
\par 3 Ние в нищо не даваме никаква причина за съблазън, да не би да се злослови нашето служение;
\par 4 но във всичко биваме одобрени, като божии служители, с голяма твърдост, в скърби, в нужди, в утеснения,
\par 5 чрез бичувания, в затваряния, в смутове, в трудове, в неспане, в неядене,
\par 6 с чистота, с благоразумие, с дълготърпение, с благост, със Светия Дух, с нелицемерна любов,
\par 7 с говорене истината, с Божия сила, чрез оръжията на правдата в дясната ръка и в лявата;
\par 8 всред слава и опозорение, всред укори и похвали; считани като измамници, но пак истинни;
\par 9 като непознати, а пък добре познати; като на умиране, а, ето, живеем; като наказани, а не умъртвявани;
\par 10 като наскърбени, а винаги радостни, като сиромаси, но обогатяваме мнозина; като че нищо нямаме, но притежаваме всичко.
\par 11 О коринтяни, устата ни са отворени към вас, сърдцето ни се разшири.
\par 12 Вам не е тясно в нас, но в сами вас е тясно нам.
\par 13 И тъй, във вид на еднакво възмездие, (като на чада говоря), разширете и вие сърдцата си.
\par 14 Не се впрягайте заедно с невярващите; защото какво общо имат правдата и беззаконието или какво общение има светлината с тъмнината?
\par 15 и какво съгласие има Христос с Велиала? или какво съучастие има вярващия с невярващия?
\par 16 и какво споразумение има Божият храм с идолите? Защото ние сме храм на живия Бог, както рече Бог: "Ще се заселя между тях и между тях ще ходя; ще им бъда Бог, и те ще Ми бъдат люде".
\par 17 Затова - "Излезте изсред тях и отделете се", казва Господ, "И не се допирайте до нечисто"; и "Аз ще ви приема,
\par 18 И ще ви бъда Отец, И вие ще Ми бъдете синове и дъщери", казва всемогъщият Господ.

\chapter{7}

\par 1 И тъй, възлюбени, като имаме тия обещания, нека очистим себе си от всяка плътска и духовна нечистота, като се усъвършенствуваме в светост със страх от Бога.
\par 2 Сторете място в сърдцата си за нас; никого не сме онеправдали, никого не сме развратили, никого не сме изкористили.
\par 3 Не казвам това за да ви осъдя; защото по-напред казах, че сте в сърдцата ни, така щото да бъдем заедно и като умрем и като живеем.
\par 4 Голяма е моята увереност към вас, много се хваля с вас, напълно се утешавам; даже във всяка наша скръб радостта ми е преизобилна.
\par 5 Защото откак дойдохме в Македония, плътта ни нямаше никакво спокойствие, но отвсякъде бяхме в утеснение: отвън борби, отвътре страхове.
\par 6 Но Бог, Който утешава смирените, утеши ни с дохождането на Тита;
\par 7 и не просто с дохождането му, но и поради утехата, с която вие го бяхте утешили, като ни извести копнежа ви, плача ви, ревността ви за мене; така щото още повече се зарадвах.
\par 8 Защото, при все че ви наскърбих с посланието си, не се разкайвам; ако и да бях се поразкаял, когато видях, че онова писмо ви наскърби, макар за малко време.
\par 9 Сега се радвам не за наскърбяването ви, но защото наскърбяването ви доведе до покаяние; понеже скърбяхте по Бога, та да се не повредите от нас в нищо.
\par 10 Защото скръбта по Бога докарва спасително покаяние, което не причинява разкаяние; но светската скръб докарва смърт.
\par 11 Защото, ето, това гдето се наскърбихте по Бога, какво усърдие породи във вас, какво себеочистване, какво негодувание, какъв страх, какво ожидане, каква ревност, какво мъздовъздаване! Във всичко вие показахте, че сте чисти в това нещо.
\par 12 И тъй, ако и да ви писах това, не го писах заради обидника, нито заради обидения, но за да ви се яви пред Бога до колко имате усърдие спрямо нас.
\par 13 Затова се утешихме; и в тая наша утеха още повече се зарадвахме поради радостта на Тита, за гдето вие всички сте успокоили духа му.
\par 14 Защото, ако му съм се похвалил малко с вас, не се засрамих; но както всичко, що ви говорихме, беше истинно, така и похвалата ни с вас пред Тита излезе истинска.
\par 15 И той още повече милее за вас, като си спомня послушността на всички ви, как сте го приели със страх и трепет.
\par 16 Радвам се, че във всичко съм насърдчен за вас.

\chapter{8}

\par 1 При това, братя, известявам ви Божията благодат, дадена на църквите в Македония,
\par 2 че, макар и да търпят голямо утеснение, пак великата им радост и дълбоката им беднотия дадоха повод да преизобилва богатството на тяхната щедрост.
\par 3 Защото свидетелствувам, че те дадоха доброволно според силата си, и даже вън от силата си,
\par 4 като ни умоляваха с голямо настояване, относно това даване ( Гръцки: Относно благодатта) , дано да участвуват и те в служенето на светиите.
\par 5 И те не само дадоха, както се надявахме, но първо предадоха себе си на Господа и , по Божията воля на нас;
\par 6 така щото помолихме Тита да довърши между вас и това благодеяние, както го беше и започнал.
\par 7 И тъй, както изобилвате във всяко нещо, - във вяра, в говорене, в знание, в пълно усърдие и в любов към нас, - така да преизобилвате и в това благодеяние.
\par 8 Не казвам това като по заповед, но за да опитам искреността на вашата любов чрез усърдието на другите.
\par 9 Защото знаете благодатта на нашия Господ Исус Христос, че, богат като бе, за вас стана сиромах, за да се обогатите вие чрез Неговата сиромашия.
\par 10 А относно това нещо аз давам тоя съвет, че така да го направите е полезно за вас, които лани бяхте първи не само да го правите, но и да желаете да го направите.
\par 11 А сега го свършете и на дело, така щото както сте били усърдни в желанието, така да бъдете и в доизкарването, според каквото имате.
\par 12 Защото, ако има усърдие, то е прието според колкото има човек, а не според колкото няма.
\par 13 Понеже не искам други да бъдат облекчени, а вие утеснени;
\par 14 но да има равенство, така щото вашето сегашно изобилие да запълни тяхната оскъдност, та и тяхното изобилие да послужи на вашата оскъдност; така щото да има равенство,
\par 15 според както е писано: "Който беше събрал много, нямаше излишък; и който беше събрал малко, не му беше оскъдно".
\par 16 А благодарение на Бога, Който туря в сърдцето на Тита същото усърдие за вас, което имаме и ние;
\par 17 защото наистина прие молбата ни, а при това, като беше сам много усърден, тръгна към вас самоволно.
\par 18 Пратихме с него и брата, чиято похвала в делотона благовестието е известна във всичките църкви;
\par 19 и не само това, но още беше избран от църквите да пътува с нас за делото на това благодеяние, на което служим за славата на Господа, и за да се покаже нашето усърдие,
\par 20 като избягваме това, - някой да ни упрекне относно тоя щедър подарък, който е поверен на нашето служение;
\par 21 понеже се грижим за това, което е честно, не само пред Господа, но и пред човеците.
\par 22 Пратихме с тях и другия ни брат, чието усърдие много пъти и в много неща сме опитали, и който сега е много по усърден поради голямото му към вас доверие.
\par 23 Колкото за Тита, той е мой другар и съработник между вас; а колкото за нашите братя, те са пратеници на църквите, те са слава на Христа.
\par 24 Покажете им, прочее, пред църквите доказателство на вашата любов и на справедливостта на нашата похвала с вас.

\chapter{9}

\par 1 А за даване помощ на светиите излишно е да ви пиша,
\par 2 понеже зная вашето усърдие, за което се хваля с вас пред македонците, че Ахаия още от лани е приготвена; и вашата ревност е подбудила по-голямата част от тях.
\par 3 А пратих братята, да не би да излезе праздна моята похвала с вас в това отношение, та да бъдете, както казвах, приготвени;
\par 4 да не би, ако дойдат с мене македонци, та ви намерят неприготвени, да се посрамим ние, да не кажа вие, в тая наша увереност.
\par 5 И тъй, намерих за нуждно да помоля братята да идат по-напред при вас и да приготвят предварително вашата по-отрано обещана милостиня, за да бъде готова като милостиня, а не като изнудване.
\par 6 А това казвам, че който сее оскъдно, оскъдно ще и да пожъне; а който сее щедро, щедро ще и да пожъне.
\par 7 Всеки да дава според както е решил в сърдцето си, без да се скъпи, и не от принуждение; защото Бог обича онзи, който дава на драго сърдце.
\par 8 А Бог е силен да преумножи на вас всякакво благо, така щото, като имате всякога и във всичко това, което е достатъчно във всяко отношение, да изобилвате във всяко добро дело;
\par 9 както е писано: - "Разпръсна щедро, даде на сиромасите, Правдата му трае до века".
\par 10 А Тоя, Който дава семе на сеяча и хляб за храна, ще даде и ще умножи вашето семе за сеене, и ще прави да изобилват плодовете на вашата правда,
\par 11 та да бъдете във всяко отношение богати във всякаква щедрост, която чрез вашето служение произвежда благодарение на Бога.
\par 12 Защото извършването на това служение не само запълва нуждите на светиите, но и чрез многото благодарения се излива и пред Бога;
\par 13 понеже те славят Бога поради доказателството, което това служение дава за вашата послушност на Христовото благовестие, което изповядвате, и за щедростта на вашето общение към тях и към всички.
\par 14 А и те, с молитви за вас, копнеят за вас поради дадената вам изобилна Божия благодат.
\par 15 Благодарение Богу за Неговия неизказан дар!

\chapter{10}

\par 1 И сам аз, Павел, ви моля поради Христовата кротост и нежност, аз, който съм смирен когато съм между вас, а когато отсътствувам ставам смел към вас, -
\par 2 моля ви се, когато съм при вас да се не принудя да употребя смелост с оная увереност, с която мисля да се одързостя против някои, които разчитат, че ние плътски се обхождаме.
\par 3 Защото, ако и да живеем в плът, по плът не воюваме.
\par 4 Защото оръжията, с които воюваме, не са плътски, но пред Бога са силни за събаряне на крепости.
\par 5 Понеже събаряме помисли и всичко, което се издига високо против познанието на Бога, и пленяваме всеки разум да се покорява на Христа.
\par 6 И готови сме да накажем всяко непослушание, щом стане пълно вашето послушание.
\par 7 Вие гледате на външното. Ако някой е уверен в себе си, че е Христов, то нека размисли още веднаж в себе си, че, както той е Христов, така и ние сме Христови.
\par 8 Защото, ако бих се и нещо повечко похвалил с нашата власт, която Господ даде за назиданието ви, а не за разорението ви, не бих се засрамил.
\par 9 Обаче нека се не покажа, че желая да ви заплашвам с посланията си.
\par 10 Понеже, казват някои, посланията му са строги и силни, но личното му присъствие е слабо и говоренето нищожно.
\par 11 Такъв нека има предвид това, че, каквито сме на думи в посланията си, когато сме далеч от вас, такива сме и на дело, когато сме при вас.
\par 12 Защото не смеем да считаме или сравняваме себе си с някои от ония, които препоръчват сами себе си; но те, като мерят себе си със себе си, и като сравняват себе си със себе си, не постъпват разумно.
\par 13 А ние няма да се похвалим с това, което е вън от мярката ни, но според мярката на областта, която Бог ни е определил, като мярка, която да достигне дори до вас.
\par 14 Защото ние не се простираме чрезмерно, като че ли не сме достигнали до вас; защото ние първи достигнахме до вас с Христовото благовестие.
\par 15 И не се хвалим с това, което е вън от мярката ни, тоест, с чужди трудове, но имаме надежда, че с растенето на вярата ви, ние ще имаме по-голяма област за работа между вас, и то премного,
\par 16 така щото да проповядваме благовестието и от вас по-нататък, а не да се хвалим с готовото в чужда област.
\par 17 А който се хвали, с Господа да се хвали.
\par 18 Защото не е одобрен тоя, който сам себе си препоръчва, но тоя, когото Господ препоръчва.

\chapter{11}

\par 1 Дано бихте потърпели малко моето безумие; да! потърпете ме,
\par 2 защото ревнувам за вас с божествена ревност, понеже ви сгодих с един мъж, да ви представя като чиста девица на Христа.
\par 3 Но боя се да не би, както змията измами Ева с хитростта си, да се разврати умът ви и отпадне от простотата и чистотата, която дължите на Христа.
\par 4 Защото, ако дойде някой и ви проповядва друг Исус, когото ние не сме проповядвали, или ако получите друг дух, когото не сте получили, или друго благовестие, което не сте приели, вие лесно го търпите.
\par 5 Обаче мисля, че аз не съм в нищо по-долен от тия превъзходни апостоли!
\par 6 А пък, ако и да съм в говоренето прост, в знанието не съм; дори ние по всякакъв начин сме ви показали това във всичко.
\par 7 Грях ли съм сторил, като смирявах себе си, за да се издигнете вие, понеже ви проповядвах Божието благовестие даром?
\par 8 Други църкви обрах, като вземах заплата от тях, за да служа на вас;
\par 9 а когато бях при вас и изпаднах в нужда, не отегчих никого, защото братята, които дойдоха от Македония, задоволиха нуждата ми. Така във всичко се пазих, и ще се пазя, да не ви отегча.
\par 10 Заради Христовата истинност, която е и в мене, никой няма да ми отнеме тая похвала в ахайските места.
\par 11 Защо? Защото ви не обичам ли? Знае Бог!
\par 12 А каквото правя, това и ще правя, за да отсека причината на тия, които търсят причина против мене, та относно това, с което те се хвалят, да се намерят също такива, каквито сме и ние.
\par 13 Защото такива човеци са лъжеапостоли, лукави работници, които се преправят на Христови апостоли.
\par 14 И не е чудно; защото сам сатана се преправя на светъл ангел;
\par 15 тъй че, не е голямо нещо, ако и неговите служители се преправят на служители на правдата. Но тяхната сетнина ще бъде според делата им.
\par 16 Пак казвам, никой да не ме счита за безумен; иначе, приемете ме като безумен, та да се похваля и аз малко нещо.
\par 17 (Това, което казвам, не го казвам по Господа, но като в безумие, в тая моя увереност на хваленето.
\par 18 Тъй като мнозина се хвалят по плът, ще се похваля и аз).
\par 19 Защото вие, като сте разумни, с готовност търпите безумните;
\par 20 понеже търпите, ако някой ви заробва, ако ви изпояжда, ако ви обира, ако се превъзнася, ако ви бие по лицето.
\par 21 За свое унижение го казвам, като че ли сме били слаби; но с каквото се осмелява някой да се хвали, (в безумие говоря), осмелявам се и аз.
\par 22 Евреи ли са? И аз съм; израилтяни ли са? И аз съм; Авраамово потомство ли са? И аз съм;
\par 23 служители Христови ли са? (в безумие говоря), аз повече: бил съм в повече трудове, в тъмници още повече, в бичувания чрезмерно, много пъти и на смърт.
\par 24 Пет пъти юдеите ми удариха по четиридесет удара без един;
\par 25 три пъти бях бит с тояги, веднъж ме биха с камъни, три пъти съм претърпял корабокрушение, една нощ и един ден съм бил по морските дълбочини.
\par 26 Много пъти съм бил и в пътешествия; в опасност от реки, в опасност от разбойници, в опасност от съотечественици, в опасност от езичници, в опасност в град, в опасност в пустиня, в опасност по море, в опасност между лъжебратя;
\par 27 в труд и в мъка, много пъти в неспане, в глад и жажда, много пъти в неядене, в студ и в голота;
\par 28 и, освен другите неща, които не споменавам, има и това, което тежи върху мене всеки ден, грижата за всичките църкви.
\par 29 Кой изнемощява, без да изнемощявам и аз? Кой се съблазнява без да се разпалям аз?
\par 30 Ако трябва да се хваля, ще се похваля с това, което се отнася до немощта ми.
\par 31 Бог и Отец на Господа Исуса [Христа], Който е благословен до века, знае, че не лъжа.
\par 32 (В Дамаск областният управител на цар Арета тури стража в град Дамаск за да ме улови;
\par 33 и през прозорец по стената ме спуснаха с кош, та избягах от ръцете му).

\chapter{12}

\par 1 Принуден съм да се хваля, при все че не е за полза; но сега ще дойда до видения и откровения от Господа.
\par 2 Познавам един човек в Христа, който, преди четиридесет години, (в тялото ли, не зная, вън от тялото ли, не зная, Бог знае), бе занесен до третото небе.
\par 3 И такъв човек, познавам, (в тялото ли, без тялото ли, не зная; Бог знае),
\par 4 който бе занесен в рая, и чу неизразими думи, които на човека не е позволено да изговори.
\par 5 С такъв човек ще се похваля; а със себе си няма да се похваля; освен с немощите си.
\par 6 (Защото, даже ако поискам да се похваля за други неща, не ще бъда безумен, понеже ще говоря истината; но въздържам се, да не би някой да помисли за мене повече от каквото вижда, че съм или каквото чува от мене).
\par 7 А за да се не превъзнасям поради премногото откровения, даде ми се трън ( Или: Мъченически кол. ) в плътта, пратеник от сатана да ме мъчи, та да се не превъзнасям.
\par 8 Затова три пъти се молих на Господа да се отмахне от мене;
\par 9 и Той ми рече: Доволно ти е Моята благодат; защото силата Ми в немощ се показва съвършена. И тъй, с преголяма радост по-добре ще се похваля с немощите си, за да почива на мене Христовата сила.
\par 10 Затова намирам удоволствие в немощи, в укори, в лишения, в гонения, в притеснения за Христа; защото, когато съм немощен, тогава съм силен.
\par 11 Станах безумен. Вие ме принудихте, защото вие трябваше да ме препоръчвате, понеже не съм бил по-долен от тия превъзходни апостоли, ако и да не съм нищо.
\par 12 Наистина, признаците на апостолите се показаха между вас с пълно търпение, чрез знамения, чудеса и велики дела.
\par 13 Защото в какво бяхте поставени по-долу от другите църкви, освен в това, гдето сам аз не ви отегчих? Простете ми тая неправда!
\par 14 Ето, готов съм да дойда при вас трети път и няма да ви отегча, защото не искам вашето, но вас; понеже чадата не са длъжни да събират имот за родителите, но родителите за чадата.
\par 15 А пък аз с преголяма радост ще иждивя и цял ще се иждивя за душите ви. Ако аз ви обичам повeче, вие по-малко ли ще ме обичате?
\par 16 Но, нека е тъй, че аз не съм ви отегчил, обаче като хитър съм ви уловил с измама.
\par 17 Изкорестих ли ви аз чрез някои от ония, които изпратих до вас?
\par 18 Помолих Тита да отиде при вас, и с него пратих брата. Тит ли припечели нещо от вас? Не със същия ли дух се обхождаме? Не в същите ли стъпки?
\par 19 Още ли мислите, че ние се възхищаваме пред вас? Не, пред Бога говорим това в Христа, и то всичко, любезни мои, за ваше назидание.
\par 20 Защото се боя да не би, като дойда, да ви намеря не каквито ви желая, и аз да се намеря за вас не какъвто ме желаете, и да не би да има между вас раздор, завист, гняв, партизанства, одумвания, шушукания, големотвувания, безредици;
\par 21 Да не би, когато дойда пак да ме смири моят Бог между вас, и да оплача мнозина, които отнапред са съгрешили, и са се покаяли за нечистотата, блудството и сладострастието, на които са се предавали.

\chapter{13}

\par 1 Ето, трети път ида при вас. "От устата на двама или трима свидетели ще се потвърди всяка работа".
\par 2 Както, когато бях при вас втори път, ви предупредих така и сега, когато не съм при вас, предупреждавам тия, които отнапред са съгрешили, и всички други, че, ако дойда пак, няма да пощадя.
\par 3 тъй като търсите доказателство, че в мене говори Христос, Който спрямо вас не е немощен, но е силен между вас;
\par 4 защото, при все че Той биде разпнат в немощ, но пак живее чрез Божията сила. И ние също сме немощни в Него, но ще сме живи с Него чрез Божията сила спрямо вас.
\par 5 Изпитвайте себе си, дали сте във вярата; опитвайте себе си. Или за себе си не познавате ли че Христос е във вас, освен ако сте порицани;
\par 6 А надявам се да познаете, че ние не сме порицани;
\par 7 и моля Бога да не сторите никакво зло, - не за да се покажем ние одобрени, но за да струвате вие това, което е честно, ако и да бъдем ние порицани.
\par 8 Защото не можем да вършим нищо против истината но за истината можем.
\par 9 Понеже се радваме, когато ние сме немощни, а вие сте силни; и за това се молим - за вашето усъвършенствуване.
\par 10 Затуй, като отсъствувам, пиша това, та когато съм при вас, да се не отнеса строго според властта, която ми е дал Господ за назидание, а не за събаряне.
\par 11 Най-после, братя, здравейте. Усъвършенствувайте се, утешавайте се, бъдете единомислени, живейте в мир; и Бог на любовта и на мира ще бъде с вас.
\par 12 Поздравете се един друг с света целувка.
\par 13 Поздравяват ви всичките светии.
\par 14 Бzлагодатта на Господа Исуса Христа, и любовта на Бога и общението на Светия Дух да бъде с всички вас. (Aмин)

\end{document}