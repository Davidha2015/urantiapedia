\begin{document}

\title{Galatians}


\chapter{1}

\par 1 Павел, апостол, не от човеци, нито чрез човек, но чрез Исуса Христа и Бога Отца, Който Го е възкресил от мъртвите,
\par 2 и всичките братя, които са с мене, към галатийските църкви:
\par 3 Благодат и мир да бъде на вас от Бога Отца и от нашия Господ Исус Христос,
\par 4 Който даде себе си за нашите грехове, за да ни избави от настоящия нечист свят ( Или Век ) според волята на нашия Бог и Отец.
\par 5 Комуто да бъде славата до вечни векове. Амин
\par 6 Чудя се как вие оставяте Онзи, Който ви призова чрез Христовата благодат, и така скоро преминавате към друго благовестие;
\par 7 което не е друго благовестие, но е дело на неколцината, които ви смущават и искат да изопачат Христовото благовестие.
\par 8 Но ако и сами ние, или ангел от небето ви проповядва друго благовестие освен онова, което ви проповядвахме, нека бъде проклет.
\par 9 Както ей сега казвахме, така пак го казвам: Ако някой ви проповядва друго благовестие освен онова, което приехте, нека бъде проклет.
\par 10 Защото на човеци ли искам да угоднича сега, или на Бога? Или искам да угаждам на човеци? Ако бях още угаждал на човеци, не щях да съм Христов слуга.
\par 11 Защото ви известявам, братя, че проповядваното от мене благовестие не е човешко;
\par 12 понеже аз нито от човек съм го приел, нито съм го научил от човек, но чрез откровение от Исуса Христа.
\par 13 Защото сте чули за някогашната моя обхода в юдейската религия, как чрезмерно гонех Божията църква и я разорявах.
\par 14 И напредвах в юдейската религия повече от мнозина мои връстници между съотечествениците ми, като бях по-голям ревнител за преданията от бащите ми.
\par 15 А когато Бог, Който още от утробата на майка ми беше ме отделил и презовал чрез Своята благодат,
\par 16 благоволи да ми открие Сина Си, за да Го проповядвам между езичниците, от същия час не се допитах до плът и кръв,
\par 17 нито възлязох в Ерусалим при ония, които бяха апостоли преди мене, но заминах за Арабия, и пак се върнах в Дамаск.
\par 18 Тогава, след три години, възлязох в Ерусалим за да се запозная с Кифа, и останах при него петнадесет дни;
\par 19 а друг от апостолите не видях, освен Якова, брата Господен.
\par 20 (А за това, което ви пиша, ето, пред Бог ви уверявам, че не лъжа).
\par 21 После дойдох в сирийските и киликийските страни.
\par 22 И лично не бях още познат на Христовите църкви в Юдея;
\par 23 а само слушаха, че оня, който по едно време ги е гонел, сега проповядвал вярата, която някога разорявал.
\par 24 И славеха Бога поради мене.

\chapter{2}

\par 1 Тогава след четиринадесет години, пак възлязох в Ерусалим с Варнава, като взех със себе си и Тита.
\par 2 (А възлязох по откровение). И изложих пред братята благовестието, което проповядвам между езичниците, но частно пред по-именитите от тях, да не би напразно да тичам или да съм тичал.
\par 3 Но даже Тит, който бе с мене, ако и да беше грък, не бе принуден да се обреже;
\par 4 и то поради лъжебратята, които бяха се вмъкнали да съгледват свободата, която имаме в Христа Исуса, за да ни поробят;
\par 5 на които, ни за час не отстъпихме да им се покорим, за да пребъде с вас истината на благовестието.
\par 6 А тия, които се считаха за нещо, (каквито и да са били, на мене е все едно; Бог не гледа на лицето на човека), - тия именити, казвам, не прибавиха нищо повече на моето учение;
\par 7 а напротив, когато видяха, че на мене беше поверено да проповядвам благовестието между необрязаните, както на Петра между обрязаните,
\par 8 (защото, който подействува в Петра за апостолство между обрязаните, подействува и в мене за апостолство между езичниците),
\par 9 и когато познаха дадената на мене благодат, то Яков, Кифа и Иоан, които се считаха за стълбове, дадоха десници на общение на мене и на Варнава, за да идем ние между езичниците, а те между обрязаните.
\par 10 Искаха само да помним сиромасите, - което също нещо и ревностно желаех да върша.
\par 11 А когато Кифа дойде в Антиохия, аз му се възпротивих в очи, защото беше се провинил.
\par 12 Понеже, преди да дойдеха някои от Якова, той ядеше заедно с езичниците; а когато те дойдоха, оттегли се и странеше от тях, защото се боеше от образованите.
\par 13 И заедно с него лицемерствуваха и другите юдеи, така щото и Варнава се увлече от лицемерието им.
\par 14 Но, като видях, че не постъпват право по истината на благовестието, рекох на Кифа пред всичките: Ако ти, който си юдеин, живееш като езичниците, а не като юдеите, как принуждаваш езичниците да живеят като юдеите?
\par 15 Ние, които сме по природа юдеи, а не грешници от езичниците
\par 16 като знаем все пак, че човек не се оправдава чрез дела по закона, а само чрез вяра в Исуса Христа, - и ние повярвахме в Христа Исуса, за да се оправдаем чрез вяра в Христа, а не чрез дела по закона; защото чрез дела по закона няма да се оправдае никоя твар.
\par 17 Но, когато искахме да се оправдаем чрез Христа, ако и ние сме се намерили грешни, то Христос на греха ли е служител? Да не бъде!
\par 18 Защото, ако градя отново онова, което съм развалил, показвам себе си престъпник.
\par 19 Защото аз чрез закона умрях спрямо закона, за да живея за Бога.
\par 20 Съразпнах се с Христа, и сега вече, не аз живея, но Христос живее в мене; а животът, който сега живея в тялото, живея го с вярата, която е в Божия Син, Който ме възлюби и предаде Себе Си за мене.
\par 21 Не осуетявам Божията благодат; обаче, ако правдата се придобива чрез закона, тогава Христос е умрял напразно.

\chapter{3}

\par 1 О, неомислени галатяни, кой ви омая, вас, пред чиито очи Исус Христос е бил ясно очертан като разпнат?
\par 2 Това само желая да науча от вас: Чрез дела, изисквани от зокона ли получихти Духа, или чрез вяра в евангелското послание?
\par 3 Толкоз ли сте несмислени, че, като почнахте в Духа, сега се усъвършенствувате в плът?
\par 4 Напусто ли толкоз страдахте? Ако наистина е напусто!
\par 5 Прочее, Тоя, Който ви дава Духа и върши велики дела между вас, чрез дела, изисквани от закона да върши това, или чрез вяра в посланието,
\par 6 както с Авраама, който повярва в Бога и му се вмени за правда?
\par 7 Тогава познайте, че тия, които упражняват вяра, те са Авраамови чада;
\par 8 и писанието като предвиде, че Бог чрез вяра щеше да оправдае езичниците, изяви предварително благовестието на Авраама, казвайки: "В тебе ще се благословят всичките народи".
\par 9 Така щото тия, които имат вяра се благословят с вярващия Авраам.
\par 10 Защото всички, които се облягат на дела, изисквани от закона, са под клетва, понеже е писано: "Проклет е всеки, който не постоянствува да изпълнява всичко писано в книгата на закона".
\par 11 А че никой не се оправдава пред Бога чрез закона, е явно от това, "че праведния чрез вяра ще живее";
\par 12 а законът не действува чрез вяра, но казва: "Който върши това, което заповядва законът, ще живее чрез него".
\par 13 Христос ни изкупи от проклетията на закона, като стана проклет ( Гръцки: Проклетия. Виж. 2 Кор-5:21 ) за нас; защото е писано: "Проклет всеки, който виси на дърво";
\par 14 така щото благословението, дадено от Авраама, да дойде чрез Христа Исуса на езичниците, за да приемем обещания Дух чрез вяра.
\par 15 Братя, (по човешки говоря) едно завещание, даже ако е само човешко, еднаж потвърди ли се, не се разваля, нито на него се прибавя нещо, от никого.
\par 16 А обещанията се изрекоха на Авраама и на неговия потомък. Не казва: "и на потомците", като на мнозина, но като за един: "и на твоя потомък", Който е Христос.
\par 17 И това казвам, че завет, предварително потвърден, от Бога, не може да бъде развален от закона, станал на четиристотин и тридесет години по-после, така щото да се унищожи обещанието.
\par 18 Защото, ако наследството е чрез закона, не е вече чрез обещание; но Бог го подари на Авраама с обещание.
\par 19 Тогава, защо се даде закона? Прибави се с цел да се изявят престъпленията, докле да дойде Потомъкът, на Когото биде дадено обещанието; и беше прогласен от ангели чрез един ходатай.
\par 20 Но ходатаят не ходатайствува за един; а Бог, Който дава обещание, е един.
\par 21 Тогава, законът противен ли е на Божиите обещания? Да не бъде! Защото, ако беше даден закон, който да може да оживотвори, то наистина правдата щеше да бъде от закона.
\par 22 Но писанието затвори всичко под грях, та обещанието, изпълняемо чрез вяра в Исуса Христа, да се даде на тия, които вярват.
\par 23 А преди да дойде вярата ние бяхме под стражата на закона, затворени до времето на вярата, която искаше да се открие.
\par 24 Така, законът стана за нас детеводител, да ни доведе при Христа, за да се оправдаем чрез вяра.
\par 25 Но след идването на вярата не сме вече под детеводител.
\par 26 Защото всички сте Божии чада чрез вяра в Исуса Христа.
\par 27 Понеже всички вие, които сте се кръстили в Христа, с Христа сте се облекли.
\par 28 Няма вече юдеин, нито грък, няма роб, нито освободен, няма мъжки пол, ни женски; защото вие всички сте едно в Христа Исуса.
\par 29 И ако сте Христови, то сте Авраамово потомство, наследници по обещание.

\chapter{4}

\par 1 Казвам още: До тогаз, докато наследникът е малолетен, той не се различава в нищо от роб, ако и да е господар на всичко,
\par 2 но е под надзиратели и настойници до назначения от бащата срок.
\par 3 Така и ние, когато бяхме малолетни, бяхме поробени под първоначалните учения на света;
\par 4 а когато се изпълни времето, Бог изпрати Сина Си, Който се роди от жена, роди се и под закона,
\par 5 за да изкупи ония, които бяха под закона, та да получим осиновението.
\par 6 И понеже сте синове, Бог изпрати в сърдцата ни Духа на Сина Си, Който вика: Авва, Отче!
\par 7 Затова не си вече роб, но син; и ако си син, то си Божий наследник чрез Христа.
\par 8 Но тогава, когато не познавахте Бога, вие робувахте на ония, които по естество не са богове;
\par 9 а сега, когато познахте Бога, или по-добре, като станахте познати от Бога, как се връщате надире към слабите и сиромашки първоначални учения, на които наново желаете да робувате?
\par 10 Вие пазите дните, месеците, времената и годините.
\par 11 Боя се за вас, да не би напусто да съм се трудил помежду ви.
\par 12 Моля ви се, братя, станете като мене, защото и аз станах като вас. Не сте ми сторили никаква неправда,
\par 13 но сами знаете, че на първия път ви проповядвах благовестието, вследствие на телесна слабост;
\par 14 но пак, това, което ви беше изпитня в моята снага, вие не го презряхте, нито се погнусихте от него, но ме приехте като Божий ангел, като Христа Исуса.
\par 15 Тогава, къде е онова ваше самочеститяване? Понеже свидетелствувам за вас, че, че ако беше възможно, очите си бихте извъртели и бихте ми ги дали.
\par 16 Прочее, аз неприятел ли ви станах, понеже постъпвам вярно с вас?
\par 17 Тия учители ви търсят ревностно не по добър начин; даже те желаят да ви отлъчат от нас, за да търсите тях ревностно.
\par 18 Но добре е да бъдете търсени ревностно за това, което е добро, и то на всяко време, а не само, когато се намирам аз между вас.
\par 19 Дечица мои, за които съм пак в родилни болки докле се изобрази Христос във вас,
\par 20 желал бих да съм сега при вас и да променя гласа си, защото съм в недоумение за вас.
\par 21 Кажете ми вие, които желаете да бъдете под закона, не чувате ли що казва законът?
\par 22 Защото е писано, че Авраам имаше два сина, един от слугинята и един от свободната;
\par 23 но тоя, който бе от слугинята, се роди по плът, а оня, който бе от свободната, по обещание.
\par 24 И това е иносказание, защото тия жени представляват два завета, единият от Синайската планина, който ражда чада за робство, и той е Агар.
\par 25 А тая Агар представлява планината Синай в Арабия и съответствува на днешния Ерусалим, защото тя е в робство с чадата си.
\par 26 А горният Ерусалим е свободен, който е [на всички] майка;
\par 27 защото е писано: "Весели се, неплодна, която не раждаш; Възгласи и извикай, ти, която не си била в родилни болки, Защото повече са чедата на самотната, нежели чадата на омъжената".
\par 28 А ние, братя, както Исаак, сме чада на обещание.
\par 29 Но, както тогава роденият по плът гонеше родения по Дух, така е и сега.
\par 30 Обаче, що казва писанието? "Изпъди слугинята и сина й; защо, синът на слугинята няма да наследи със сина на свободната".
\par 31 За туй, братя, ние не сме чада на слугиня, а на свободната.

\chapter{5}

\par 1 Стойте, прочее, твърдо в свободата, за която Христос ни освободи, и не се заплитайте отново в робско иго.
\par 2 Ето, аз, Павел, ви казвам, че ако се обрязвате Христос никак нямя да ви ползува.
\par 3 Да! повторно заявявам на всеки човек, който се обрязва, че е длъжен да ивпълни целия закон.
\par 4 Вие, които желаете да се оправдавате чрез закона сте се отлъчили от Христа, отпаднали сте от благодатта.
\par 5 Защото ние чрез Духа ожидаме оправданието чрез вяра, за което се надяваме.
\par 6 Понеже в Христа Исуса нито обрязването има някаква сила, нито необрязването, но вяра, която действува чрез любов.
\par 7 Вие вървяхте добре; кой ви попречи да не бъдете послушни на истината?
\par 8 Това убеждение не беше от Онзи, Който ви е призовал.
\par 9 Малко квас заквасва цялото тесто.
\par 10 Аз съм уверен за вас в Господа, че няма да помислите другояче; а който ви смущава, той ще понесе своето наказание, който и да е бил той.
\par 11 И аз братя, защо още да бъда гонен, ако продължавам да проповядвам обрязване? защото тогава съблазънта на кръста би се махнала.
\par 12 Дано се отсечеха ония, които ви разколебават.
\par 13 Защото вие, братя, на свободата бяхте призовани; само не употребявайте свободата си като повод за угаждане на плътта, но с любов служете си един на друг.
\par 14 Защото целият закон се изпълнява в една дума, сиреч в тая "Да обичаш ближния си както себе си".
\par 15 Но ако се хапете и се ядете един друг, пазете се да не би един друг да се изтребите.
\par 16 Прочее казвам: Ходете по Духа и няма да угаждате на плътските страсти.
\par 17 Защото плътта силно желае противното на Духа, а Духът противното на плътта; понеже те се противят едно на друго, за да можете да правите това, което искате.
\par 18 Но ако се водите от Духа, не сте под закон.
\par 19 А делата на плътта са явни; те са: блудство, нечистота, сладострастие,
\par 20 идолопоклонство, чародейство, вражди, разпри, ревнования, ярости, партизанства, раздори, разцепления,
\par 21 зависти, пиянства, пирувания и там подобни; за които ви предупреждавам, че които вършат такива работи, няма да наследят Божието царство.
\par 22 А плодът на Духа е: любов, радост, мир, дълготърпение, благост, милост, милосърдие, вярност,
\par 23 кротост, себеобуздание; против такива неща няма закон.
\par 24 А които са Исус Христови, разпнали са плътта заедно със страстите и похотите й.
\par 25 Ако по Дух живеем, по Дух и да ходим.
\par 26 Да не ставаме щеславни, един друг да се не дразним и да си не завиждаме един на друг.

\chapter{6}

\par 1 Братя, даже ако падне човек в някое прегрешение, вие духовните поправяйте такъв с кротък дух; но всекиму казвам: Пази себе си, да не би ти да бъдеш изкушен.
\par 2 Един другиму тегобите си носете, и така изпълнявайте Христовия закон.
\par 3 Защото, ако някой мисли себе си да е нещо, като не е нищо, той мами себе си.
\par 4 Но всеки нека изпита своята работа, и тогава ще може да се хвали само със себе си, а не с другиго;
\par 5 защото всеки има да носи своя си товар.
\par 6 А тоя, който се поучава в Божието слово, нека прави участник във всичките си блага този, който го учи.
\par 7 Недейте се лъга; Бог не е за подиграване: понеже каквото посее човек, това ще и пожъне.
\par 8 Защото, който сее за плътта си, от плътта си ще пожъне тление, а който сее за Духа, от Духа ще пожъне вечен живот.
\par 9 Да не ни дотегнува да вършим добро; защото ако се не уморяваме, своевременно ще пожънем.
\par 10 И тъй, доколкото имаме случай, нека струваме добро на всички, а най-вече на своите по вяра.
\par 11 Вижте с колко едри букви ви писах със собствената си ръка!
\par 12 Ония, които желаят да покажат добра представа в плътския живот, те ви заставят да се обрязвате; те търсят само да не бъдат гонени за Христовия кръст.
\par 13 Защото и сами тия, които се обрязват, не пазят закона, но желаят да се обрязвате вие, за да могат да се хвалят с вашия плътски живот
\par 14 А далече от мене да се хваля освен с кръста на нашия Господ Исус Христос, чрез който светът за мене е разпнат и аз за света
\par 15 Защото в Христа Исуса нито обрязването е нещо, нито необрязването, а новото създание.
\par 16 И на всички, които живеят по това правило, мир и милост да бъде на тях и на Божия Израил.
\par 17 Отсега нататък никой да ми не досажда, защото аз нося на тялото си белезите на [Господа] Исуса.
\par 18 Братя, благодатта на нашия Господ Исус Христос да бъде с вашия дух. Амин.

\end{document}