\begin{document}

\title{Филипяни}


\chapter{1}

\par 1 Павел и Тимотей, слуги Исус Христови, до всичките в Исуса Христа светии, които са в Филипи, заедно с епископите и дяконите:
\par 2 Благодат и мир да бъде на Вас от Бога, нашия Отец и от Господа Исуса Христа.
\par 3 Благодаря на моя Бог всеки път, когато си спомням за вас,
\par 4 винаги във всяка моя молитва, като се моля за всички вас с радост,
\par 5 за вашето участие в делото на благовестието, от първия ден дори до сега;
\par 6 като съм уверен именно в това, че оня, който е почнал доброто дело във вас, ще го усъвършенствува до деня на Исуса Христа.
\par 7 И право е да мисля това за всички вас, понеже ви имам на сърце, тъй като вие всички сте съучастници с мене в благодатта, както в оковите ми, тъй и за защитата и в утвърждението на благовестието.
\par 8 Защото Бог ми е свидетел как милея за всички ви с милосърдие Исус Христово.
\par 9 И затова се моля, щото любовта ви да бъде все повече и повече изобилно просветена и всячески проницателна,
\par 10 за да изпитате нещата, които се различават, та да бъдете искрени и незлобни до деня на Христа,
\par 11 изпълнени с плодовете на правдата, които са чрез Исуса Христа, за слава и хвала на Бога.
\par 12 А желая да знаете, братя, че това, което ме сполетя, спомогна повече за преуспяване на благовестието,
\par 13 до толкоз, щото стана известно на цялата претория и на всички други, че съм в окови за Христа;
\par 14 и повечето от братята на Господа, одързостени от успеха в оковите ми, станаха по-смели да говорят Божието слово без страх.
\par 15 Някои наистина проповядват Христа дори от завист и от препирлив дух, а някоъ и от добра воля.
\par 16 Едните правят това от любов, като знаят, че съм изпратен да защищавам благовестието;
\par 17 а другите възвестяват Христа от партизанство, не искрено като мислят да ми турят тяга в оковите ми.
\par 18 Тогава що? Само туй, че по всякакъв начин, било присторено или истинно, Христос се проповядва; и затова аз се радвам, и ще се радвам.
\par 19 Защото зная, че това ще излезе за моето спасение чрез вашата молитва и даването на мене Духа Исус Христов,
\par 20 според усърдното ми очакване и надежда, че няма в нищо да се посрамя, но, че, както всякога, така и сега ще възвелича Христа в тялото си с пълно дръзновение, било чрез живот, или чрез смърт.
\par 21 Защото за мене да живея е Христос, а да умра, придобивка.
\par 22 Но ако живея в тялото, това значи плод от делото ми; и така що да избера не зная,
\par 23 но съм на тясно между двете, понеже имам желание да отида и да бъда с Христа, защото, това би било много по-добре;
\par 24 но да остана в тялото е по-нужно за вас.
\par 25 И като имам тая увереност, зная, че ще остана и ще пребъда с всички вас за вашето преуспяване и радост във вярата;
\par 26 тъй щото, чрез моето завръщане между вас, да можете поради мене много да се хвалите в Исуса Христа.
\par 27 Само се обхождайте достойно на Христовото благовестие, тъй щото, било че дойда и ви видя, или че не съм при вас, да чуя за вас, че стоите твърдо в един дух и се подвизавате единодушно във вярата на благовестието,
\par 28 и че в нищо не се плашите от противниците; което е и доказателство за тяхната погибел, а на вас за спасение, и то от Бога;
\par 29 защото, относно Христа, вам е дадено не само да вярвате в Него, но и да страдате за Него;
\par 30 като имате същата борба, каквато сте видели, че аз имам, и сега чувате, че съм имал.

\chapter{2}

\par 1 И тъй, ако има някоя утеха в Христа, или някоя разтуха от любов, или някое милосърдие и състрадание,
\par 2 направете радостта ми пълна, като мислите все едно, като имате еднаква любов и бъдете единодушни и единомислени.
\par 3 Не правете нищо от партизанство или от щеславие, но със смиреномъдрие нека всеки счита другия по-горен от себе си.
\par 4 Не гледайте всеки само за своето, но всеки и за чуждото.
\par 5 Имайте в себе си същия дух, който беше и в Христа Исуса;
\par 6 Който, както беше в Божия образ, пак не счете, че трябва твърдо да държи равенството с Бога,
\par 7 но се отказва всичко, като взе на Себе Си образ на слуга и стана подобен на човеците;
\par 8 и, като се намери в човешки образ, смири Себе Си и стана послушен до смърт, даже смърт на кръст.
\par 9 Затова и Бог Го превъзвиши и Му подари името, което е над всяко друго име;
\par 10 така щото в Исусовото име да се поклони всяко коляно от небесните и земните и подземните същества,
\par 11 и всеки език да изповяда, че Исус Христос е Господ, за слава на Бога Отца.
\par 12 Затова, възлюбени мои, както сте били винаги послушни, не само както при мое присъствие, но сега много повече при моето отсъствие, изработвайте спасението си със страх и трепет;
\par 13 Защото Бог е, Който според благоволението Си действува във вас и да желаете това и да го изработвате.
\par 14 Вършете всичко без роптане и без препиране,
\par 15 за да бъдете безукорни и незлобиви, непорочни Божии чада всред опако и извратено поколение, между които блестите като светила на света,
\par 16 като явявате словото на живота; за да имам с какво да се хваля в деня на Христа, че не съм тичал напразно, нито съм се трудил напразно.
\par 17 Но макар че се принасям аз като възлияние върху жертвата и служението на вашата вяра, радвам се и с всички вас се радвам.
\par 18 Подобно се радвайте и вие и с мене заедно се радвайте.
\par 19 А надявам се на Господа Исуса да ви изпратя скоро Тимотея, та и аз да се утеша, като узная вашето състояние.
\par 20 Защото нямам никой друг на еднакъв дух с мене, който да се погрижи искрено за вас.
\par 21 Понеже всички търсят своето си, а не онова, което е Исус Христово.
\par 22 А вие знаете неговата изпитана вярност, че той е служител с мене в делото на благовестието, както чадо слугува на баща си.
\par 23 Него, прочее, се надявам да изпратя, щом разбера, как ще стане с мене;
\par 24 а уверен съм в Господа, че и сам скоро ще дойда.
\par 25 Счетох, обаче за нужно да ви изпратя брата Епафродита, моя съработник и сподвижник, изпратен от вас да ми послужи в нуждите;
\par 26 понеже милееше за всички ви, и тъжеше, защото бяхте чули че бил болен.
\par 27 И наистина той боледува близу до смърт; но Бог му показа милост, и не само на него, но и на мене, за да нямам скръб върху скръб.
\par 28 Затова и по-скоро го изпратих, та да го видите, да се зарадвате пак, и аз да бъда по-малко скръбен;
\par 29 Прочее, приемете го в Господа с пълна радост; и имайте такива на почит,
\par 30 понеже заради Христовото дело той дойде близу до смърт, като изложи живота си на опасност, за да допълни липсата на вашите услуги към мене.

\chapter{3}

\par 1 Впрочем, братя мои, радвайте се в Господа. За мене не е досадно да ви пиша все същото, а за вас е безопасно.
\par 2 Пазете се от злите работници, пазете се от поборниците на обрязването ( Гръцки: От връзването: презрителен израз);
\par 3 защото ние сме обрязаните, които с Божия Дух се кланяме, и се хвалим с Хрита Исуса, и не уповаваме на плътта.
\par 4 При все че аз мога и на плътта да уповавам. Ако някой друг мисли, че може да уповава на плътта, то аз повече,
\par 5 бидейки обрязан в осмия ден, от Израиловия род, от Вениаминовото племе, евреин от евреин, досежно закона фарисей,
\par 6 по ревност гонител на църквата, по правдата, която е от закона, непорочен.
\par 7 Но това, което беше за мене придобивка, като загуба го счетох за Христа.
\par 8 А още всичко считам като загуба заради това превъзходно нещо - познаването на моя Господ Христос Исус, за Когото изгубих всичко и считам всичко за измет, само Христа за придобия,
\par 9 и да се намеря в Него, без да имам за своя правда оная, която е от закона, но оная, която е чрез вяра в Христа, то ест, правдата, която е от Бога въз основа на вяра,
\par 10 за да позная Него, силата на Неговото възкресение, и обещанието в Неговите страдания, ставайки съобразуван със смъртта Му,
\par 11 дано всякак достигна възкресението на мъртвите
\par 12 Не че съм сполучил вече, или че съм станал вече съвършен; но гоня изподир, дано уловя, понеже и аз бидох уловен от Христа Исуса.
\par 13 Братя, аз не считам, че съм уловил, но едно правя, - като забравям задното и се простирам към предното,
\par 14 пускам се към прицелната точка за наградата на горното от Бога признание в Хриса Исуса
\par 15 И тъй, ние, които сме зрели, нека мислим така; и ако мислите вие нещо другояче, Бог ще ви открие в него.
\par 16 Само нека [имаме за правило да] живее според това, в което сме достигнали [същото да мъдруваме].
\par 17 Братя, бъдете всички подражатели на мене и внимавайте на тия, които се обхождат така, както имаме пример в нас.
\par 18 Защото мнозина, за които много пъти съм ви казвал, а сега и с плач ви казвам се обхождат като врагове на Христовия кръст;
\par 19 чиято сетнина е погибел, чиито бог е коремът, и чието хвалене е в това, което е срамотно, които дават ума си на земните неща.
\par 20 Защото нашето гражданство е на небесата, отгдето и очакваме Спасител, Господа Исуса Христа,
\par 21 Който ще преобрази нашето унищожено тяло, за да стане съобразно с Неговото славно тяло, по упражнение на силата Си да покори и всичко на Седе Си.

\chapter{4}

\par 1 Затова, възлюбени и многожелани мои братя, моя радост и мой венец, стойте така твърдо в Господа, възлюбени мои.
\par 2 Моля Еводия, моля и Синтихия, да бъдат единомислени в Господа.
\par 3 Да! и тебе умолявам, искрени ми сътруднико, помагай на тия жени, защото се трудеха за делото на благовестието заедно с мене, и с Климента, и от другите ми съработници, чиито имена са в книгата на живота.
\par 4 Радвайте се всякога в Господа; пак ще кажа: Радвайте се.
\par 5 Вашата кротост да бъде позната на всичките човеци; Господ е близу.
\par 6 Не се безпокойте за нищо; но във всяко нещо, с молитва и молба изказвайте прошенията си на Бога с благодарение;
\par 7 и Божият мир, който никой ум не може да схване, ще пази сърцата ви и мислите ви в Христа Исуса.
\par 8 Най-после, братя, всичко, що е истина, що е честно, що е праведно, що е любезно, що е благодатно, - ако има нещо добродетелно, и ако има нещо похвално, - това зачитайте.
\par 9 Това, което сте и научили, и приели, и чули, и видели в мене, него вършете; и Бога на мира ще бъде с вас.
\par 10 Аз много се радвам в Господа, че сега най-после направихте да процъвти наново вашата грижа за мене; за което наистина сте се грижили, ала не сте имали благовремие.
\par 11 Не казвам това поради оскъдност; защото се научих да съм доволен в каквото състояние и да се намеря.
\par 12 Зная и в оскъдност да живея, зная и в изобилие да живея; във всяко нещо и във всички обстоятелства съм научил тайната и да съм сит, и да съм гладен, и да съм в изобилие, и да съм в оскъдност.
\par 13 За всичко имам сила чрез Онзи, Който ме подкрепява.
\par 14 Но сторихте добро, като заехте участие в скръбтта ми.
\par 15 А и вие, филипяни, знаете, че когато излязох от Македония и почнах делото на благовестието, ни една църква, освен едни вие, не влезе във връзка с мене за даване и вземане;
\par 16 защото и в Солун един два пъти ми пращаха за нуждата ми.
\par 17 Не че искам подаръка, но искам плода, който се умножава за ваша сметка.
\par 18 Но получих всичко, и имам изобилно; наситих се като получих от Епафродита изпратеното от вас, благоуханна миризма, жертва приятна, благоугодна на Бога.
\par 19 А моят Бог ще снабди всяка ваша нужда според Своето богатство в слава на Христа Исуса.
\par 20 А на нашия Бог и Отец да бъде слава во вечни векове. Амин.
\par 21 Поздравете всеки светия в Христа Исуса. Поздравяват ви братята, които са с мене.
\par 22 Поздравяват ви всичките светии, а особено тия, които са от Кесаровия дом.
\par 23 Благодатта на Господа Исуса Христа да бъде с духа ви. [Амин].

\end{document}