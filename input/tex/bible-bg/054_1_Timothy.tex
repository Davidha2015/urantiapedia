\begin{document}

\title{1 Timothy}


\chapter{1}

\par 1 Павел апостол Исус Христов, по заповед на Бога, нашия Спасител, и на Христа Исуса, нашата надежда,
\par 2 до Тимотея, истинското ми чадо във вярата; Благодат, милост, мир от Бога Отца и от Христа Исуса, нашия Господ.
\par 3 Както те молих, когато отивах в Македония да останеш в Ефес, за да заръчаш на някои да не предават друго учение,
\par 4 нито да внимават на басни и безконечни родословия, които спомагат повече за препирни, а не за Божията спасителна наредба чрез вяра, така правя и сега.
\par 5 А целта на това поръчване е чистосърдечна любов от добра съвест и нелицемерна вяра;
\par 6 от които неща някои, като не случиха целта, отклониха се в празнословие,
\par 7 и искат да бъдат законоучители, без да разбират нито що говорят, нито що твърдят.
\par 8 А ние знаем, че законът е добър, ако го употребява някой законно,
\par 9 като знае това, че законът се налага за праведния, а за беззаконните и непокорните, за нечестивите и грешните, за неблагоговейните и скверните, за убийците на бащи и убийците на майки, за човекоубийците,
\par 10 за блудниците, мъжеложниците, търгуващите с роби, лъжците, кълнещите се на лъжа, и за всичко друго, що е противно на здравото учение,
\par 11 според славното благовестие на блаженния Бог, което ми биде поверено.
\par 12 Благодаря на Христа Исуса нашия Господ, Който ми даде сила, че ме счете за верен и ме настани на службата,
\par 13 при все, че бях изпърво хулител, гонител и пакостник. Но придобих милост, понеже, като невеж, сторих това в неверие;
\par 14 и благодатта на нашия Господ се преумножи към мене с вяра и любов в Христа Исуса.
\par 15 Вярно е това слово и заслужава пълно приемане, че Христос Исус дойде на света да спаси грешните, от които главният съм аз.
\par 16 Но придобих милост по тая причина, за да покаже Исус Христос в мене, главния грешник, цялото Си дълготърпение, за пример на ония, които щяха да повярват в Него за вечен живот.
\par 17 А на вечния Цар, на безсмъртния, невидимия, единствения Бог, да бъде чест и слава до вечни векове. Амин.
\par 18 Това заръчване ти предавам, чадо Тимотее, според пророчествата, които първо те посочиха, за да воюваш съобразно с тях доброто воинствуване,
\par 19 имайки вяра и чиста съвест, която някои, като отхвърлиха отпаднаха от вярата;
\par 20 от които са Именей и Александър, които предадох на сатаната, за да се научат да не богохулстват.

\chapter{2}

\par 1 И тъй, увещавам, преди всичко, да отправяте молби, молитви, прошения, благодарения на всички човеци,
\par 2 за царе и за всички, които са високопоставени, за да поминем тих и спокоен живот в пълно благочестие и сериозност.
\par 3 Това е добро и благоприятно пред Бога, нашия Спасител,
\par 4 Който иска да се спасят всички човеци и да достигнат до познание на истината.
\par 5 Защото има само един Бог и един ходатай между Бога и човеците, човекът Христос Исус,
\par 6 Който, като своевременно свидетелство за това, даде Себе Си откуп за всички;
\par 7 за което аз бях поставен проповедник и апостол, (истина казвам не лъжа), учител на езичници във вярата и истината.
\par 8 И тъй, искам мъжете да се молят на всяко място, като издигат ръце свети, а не гневни и препирливи,
\par 9 Така и жените да украсят със скромна премяна, с срамежливост и целомъдрие, не с плетена коса и злато или бисери или скъпи дрехи,
\par 10 а с добри дела, както прилича на жени, които са се посветили на благочестието.
\par 11 Жената да се учи мълчаливо с пълно подчинение.
\par 12 А на жената не позволявам да поучава, нито да владее над мъжа, но нека бъде мълчалива.
\par 13 Защото първо Адам бе създаден, а после Ева.
\par 14 И Адам не се излъга; но жената се излъга, та падна в престъпление.
\par 15 Но пак тя ще се спаси чрез чадородието, ако пребъдат във вяра, в любов и в святост с целомъдрие.

\chapter{3}

\par 1 Вярно е това слово: Ако се ревне някому епископство, добро дело желае.
\par 2 Прочее, епископът трябва да бъде непорочен, мъж на една жена, самообладан, разбран, порядъчен, честолюбив, способен да поучава,
\par 3 не навикнал на пияни разправии, не побойник, а кротък, не крамолник, не сребролюбец;
\par 4 който управлява добре своя си дом и държи чадата си в послушание с пълна сериозност;
\par 5 (защото човек ако не знае да управлява своя си дом, как ще се грижи за Божията църква?)
\par 6 да не е нов във вярата, за да се не възгордее и падне под същото осъждане с дявола.
\par 7 При това, той трябва да се ползува с добри отзиви и от външните, за да не падне в укор и в примката на дявола.
\par 8 Така и дяконите трябва да бъдат сериозни, не двоезични, да не обичат много вино, да не бъдат лакоми за гнусна печалба,
\par 9 да държат с чиста съвест тайната на вярата.
\par 10 Също и те първо да се изпитват и после да стават дякони, ако са непорочни.
\par 11 Тъй и жените им ( Или: дякониците) трябва да бъдат сериозни, не клеветници, самообладани, вярни във всичко.
\par 12 Дяконите да бъдат мъже всеки на една жена, да управляват добре чадата си и домовете си.
\par 13 Защото, тия които са служили добре като дякони, придобиват за себе си добро положение и голямо дръзновение във вярата на Христа Исуса.
\par 14 Надявам се скоро да дойда при тебе; но това ти пиша,
\par 15 в случай, че закъснея, за да знаеш, как трябва да се обхождат хората в Божия дом, който е църква на живия Бог, стълб и подпорка на истината.
\par 16 И без противоречие, велика е тайната на благочестието: - Тоя, "Който биде явен в плът, Доказан чрез Духа, Виден от ангели, Проповядван между народите, Повярван в света, Възнесен в слава",

\chapter{4}

\par 1 А Духът изрично казва, че в послешните времена някои ще отстъпят от вярата, и ще слушат измамителни духове и бясовски учения,
\par 2 чрез лицемерието на човеци, които лъжат, чиято съвест е пригоряла,
\par 3 които запрещават жененето и заповядват въздържание от ястия, които Бог създаде, за да се употребяват с благодарение от ония, които вярват и разбират истината.
\par 4 Защото всяко нещо, създадено от Бога, е добро, и нищо не е за отхвърляне, ако се приема с благодарение;
\par 5 Понеже се освещава чрез Божието слово и молитва.
\par 6 В това като съветваш братята, ще бъдеш добър служител Исус Христов, хранен с думите на вярата и доброто учение, което си следвал до сега.
\par 7 А отхвърляй скверните и бабешките басни и обучавай себе си в благочестие.
\par 8 Защото телесното обучение е за малко полезно; а благочестието е за всичко полезно, понеже има обещанието и за сегашния и за бъдещия живот.
\par 9 Това слово е вярно и заслужава приемане;
\par 10 понеже за това се трудим и подвизаваме, защото се надяваме на живия Бог, Който е Спасител на всички човеци, а най-вече на вярващите.
\par 11 Това заръчвай и учи.
\par 12 Никой да не презира твоята младост; но бъди пример на вярващите в слово, в поведение, в любов, във вяра, в чистота.
\par 13 Докато дойда, внимавай на прочитането, на увещанието на проучването.
\par 14 Не пренебрегвай, дарбата която имаш, която ти се даде, съгласно с пророчеството, чрез ръкополагането от презвитерите.
\par 15 В това прилежавай, на това се предавай, за да стане явен на всички твоят напредък.
\par 16 Внимавай на себе си и на поучението си, постоянствувай в това; защото, като правиш това, ще спасиш и себе си и слушателите си.

\chapter{5}

\par 1 Стар човек не изобличавай, а увещавай го като баща, по-младите като братя,
\par 2 старите жени като майки, по-младите като сестри - със съвършенна чистота.
\par 3 Почитай вдовиците, които наистина са вдовици.
\par 4 Но ако някоя вдовица има чада и внуци, те нека се учат да показват благочестие към домашните си и да отдават дължимото на родителите си; защото това е угодно пред Бога.
\par 5 А която е истинска вдовица и е останала сама, тя се надява на Бога и постоянствува в молби и в молитви нощем и денем;
\par 6 но оная, която живее сладострастно, жива е умряла.
\par 7 Заръчвай и това, за да бъдат и непорочни.
\par 8 Но ако някой не промишлява за своите, а най-вече за домашните си, той се е отрекъл от вярата, и от безверник е по-лош.
\par 9 Да се записва само такава вдовица, която не е по-долу от шестдесет години, която е била на един мъж жена,
\par 10 Известна по добрите си дела: ако е отхранила чеда, ако е приемала странки, ако е омивала нозете на светии, ако е помагала на страдащи, ако се е предавала на всякакво добро дело.
\par 11 А по-младите вдовици не приемай, защото, когато страстите им ги отвърнат от Христа, искат да се омъжват,
\par 12 та падат под осъждане, защото са се пометнали от първото си убеждение.
\par 13 А при това, те навикват да стоят празни, да ходят от къща на къща, и не само да бъдат празни, но и бъбриви, като се месят в чужди работи и говорят това, което не трябва да се говори.
\par 14 По тая причина, искам по-младите вдовици да се омъжват, да раждат деца, да управляват дом, да не дават никаква причина на противника да хули;
\par 15 защото някои вече са се отклонили и отишли подир сатана.
\par 16 Ако някой вярващ, мъж или жена, има сродници вдовици, нека ги пригледват, та църквата да се не обременява, за да може да пригледва истинските вдовици.
\par 17 Презвитер който управлява добре, нека се удостояват с двойна почит, особно ония, които се трудят в словото и поучението,
\par 18 защото писанието казва: "Да не обърнеш устата на воля, когато вършее", и: "Работникът заслужава заплатата си".
\par 19 Против презвитер не приемай обвинение, освен ако е нанесено от двама или трима свидетели.
\par 20 Ония, които съгрешават, изобличавай пред всички, та другите да имат страх.
\par 21 Пред Бога, пред Христа Исуса и пред избраните ангели ти заръчвам да пазиш тия заповеди без предразсъдък и нищо да не вършиш с пристрастие.
\par 22 Не прибързвай да ръкополагаш никого, нито участвай в чужди грехове. Пази себе си чист.
\par 23 Не пий вече само вода, но употребявай малко вино за стомаха си и за честите си боледувания.
\par 24 Греховете на някои човеци са явни и предварят ги на съда; а на някои идат отпосле.
\par 25 Така и добрите дела на някои са явни; а ония, които не са, не могат да се укрият за винаги.

\chapter{6}

\par 1 Които са слуги под игото на робството, нека считат господарите си достойни за всяка почит, та да не се хули Божието име и учението.
\par 2 И ония, които имат вярващи господари, да не ги презират, за гдето са братя; но толкова повече нека им работят, защото ония, които се ползуват от усърдието им, са вярващи и възлюбени. Това поучавай и увещавай;
\par 3 и ако някой предава друго учение, и не се съобразява с думите на нашия Господ Исус Христос и учението, което е съгласно с благочетието,
\par 4 той се е възгордял и не знае нищо, а има болничава охота за разисквания и препирни за нищожности, от които произлизат завист, разпри, хули, лукави подозрения,
\par 5 крамоли между човеци с развратен ум и лишени от истина, които мислят, че благочестието е средство за печалба.
\par 6 А благочестието с задоволство е голяма печалба;
\par 7 защото нищо не сме внесли в света, нито можем да изнесем нищо;
\par 8 а, като имаме прехрана и облекло, те ще ни бъдат доволно.
\par 9 А които ламтят за обогатяване, падат в изкушение, в примка и в много глупави и вредни страсти, които потопяват човеците в разорение и погибел.
\par 10 Защото сребролюбието е корен на всякакви злини, към което се стремиха някои от вярата, и пронизваха себе си с много скърби,
\par 11 Но ти, човече Божий, от тия къща; и следвай правдата, благочестието, вярата, любовта, търпението, кротостта.
\par 12 Подвизавай се в доброто воинствуване на вярата; хвани се за вечния живот, на който си бил призван, като си направил добрата изповед мнозина свидетели.
\par 13 Заръчвам ти пред Бога, Който оживява всичко и пред Христа Исуса, Който пред Понтийския Пилат засвидетелствува с добрата изповед,
\par 14 да пазиш тая заповед чисто и безукорно до явлението на нашия Господ Исус Христос,
\par 15 което своевременно ще бъде открито от блажения и единствен Властител, ; Цар на царствуващите и господ на господствуващите,
\par 16 Който сам притежава безсмъртие; обитавайки в непристъпна светлина; Когото никой човек не е видял, нито може да види; Комуто да бъде чест и вечна сила. Амин.
\par 17 На ония, които имат богатство на тоя свят, заръчай да не високоумствуват, нито да се надяват на непостоянното богатство, а на Бога, Който ни дава всичко изобилно да се наслаждаваме;
\par 18 да суруват добро, да богатеят с добри дела, да бъдат щедри, съчувствителни,
\par 19 да събират за себе си имот, който ще бъде добра основа за бъдеще, за да се хванат за истинският живот.
\par 20 О, Тимотее, пази това, което ти е поверено, като се отклоняваш от скверните празнословия и противоречия на криво нареченото знание,
\par 21 на което, като се предадоха някои, отстраниха се от вярата. Благодат да бъде с вас. [Амин].

\end{document}