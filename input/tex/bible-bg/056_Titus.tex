\begin{document}

\title{Тит}


\chapter{1}

\par 1 Павел, слуга Божий и апостол Исус Христов за подпомагане вярата на Божиите избрани и познаване на истината, която е благочестието,
\par 2 в надежда за вечен живот, който преди вечни времена е обещал Бог, Който не лъже,
\par 3 а в своите времена яви словото Си чрез посланието, поверено на мене по заповед на Бога, нашия Спасител,
\par 4 до Тита, истинното ми чадо по общата ни вяра: Благодат и мир да бъде с тебе от Бога Отца и Христа Исуса, нашия спасител.
\par 5 Оставих те в Крит по тая причина, да туриш ред в недовършеното и да поставиш презвитери във всеки град, както аз ти поръчах;
\par 6 ако е някой непорочен, на една жена мъж, и има вярващи чада, които не са обвинени в разпуснат живот или в непокорство.
\par 7 Защото епископът трябва да бъде непорочен, като Божий настойник, не своеволен, нито гневлив, нито навикнал на пияни разправии, нито побойник нито да е лаком за гнусна печалба;
\par 8 а гостолюбив, да обича доброто, разбрану праведен, благочестив, самообладан;
\par 9 който да държи вярното слово според както е било научено, за да може и да увещава със здравото учение, и да опровергава ония, които противоречат.
\par 10 Защото има мнозина непокорни човеци, празнословци и измамници, а особено от обрязаните,
\par 11 чиито уста трябва да се затулят, човеци, които извръщават цели домове, като учат за гнусна печалба това, което не трябва да учат.
\par 12 Един от тях някой си техен пророк е казъл: "Критяните са винаги лъжци, Зли зверове, лениви търбуси".
\par 13 Това свидетелство е вярно. По тая причина изобличавай ги строго, за да бъдат здрави във вярата.
\par 14 и да не дадът внимание на юдейски басни и на заповеди от човеци, които се отвръщат от истината.
\par 15 За чистите всичко е чисто; а за осквернените и невярващите нищо няма чисто, но умът им съвестта им са осквернени.
\par 16 Твърдят, че познават Бога, но с делата си се отричат от Него, като са мръсни и непокорни, неспособни за каквото и да било добро дело.

\chapter{2}

\par 1 Но ти говори това, което приляга на здравото учение именно:
\par 2 Старците да бъдат самообладани, сериозни, разбрани, здрави във вярата, в любовта, в търпението;
\par 3 също и старите жени да имат благоговейно поведение, да не са клеветници, нито предадени много на винопийство, да поучават това, което е добро;
\par 4 за да учат младите жени да обичат мъжете си и децата си,
\par 5 да са разбрани, целомъдри, да работят у домовете си, да са благи, подчинени на мъжете си, за да не се хули Божието учение.
\par 6 Така и потомците увещавай да бъдат разбрани.
\par 7 Във всичко показвай себе си пример за добри дела; в поучението показвай искреност, сериозност,
\par 8 здраво неукорно говорене, за да се засрами противникът, като няма какво лошо да каже за нас.
\par 9 Увещавай слугите да се покоряват на господарите си, да им угаждат във всичко, да им не противоречат,
\par 10 да не присвояват чуждо, а да показват винаги съвършенна вярност; за да украсят във всичко учението на Бога, нашия Спасител.
\par 11 Защото се яви Божията благодат, спасителна за всичките човеци,
\par 12 и ни учи да се отречем от нечестието и от световните страсти и да живеем разбрано, праведно и благочестиво в настоящия свят ( Или: век. ),
\par 13 ожидайки блаженната надежда, славното явление на нашия велик Бог и Спасител Исус Христос,
\par 14 който даде Себе Си за нас, за да ни изкупи от всяко беззаконие и очисти за Себе Си люде за Свое притежание, ревностни за добри дела.
\par 15 Така говори, увещавай с пълна власт. Никой да не те презира.

\chapter{3}

\par 1 Напомняй им да се покоряват на началствата и властите, да ги слушат и да бъдат готови за всяко добро дело,
\par 2 да не злословят никого, да не бъдат крамолници, да бъдат нежни и да показват съвършена кротост към всички човеци.
\par 3 Защото и ние някога бяхме несмислени, непокорни, измамвани и поробени на разни страсти и удоволствия, и като живеехме в злоба и завист, бяхме омразни, и се мразехме един друг.
\par 4 Но когато се яви благостта на Бога, нашия Спасител, и Неговата любов към човеците,
\par 5 Той ни спаси не чрез праведни дела, които ние сме сторили, но по Своята милост чрез окъпването, сиреч, новорождението и обновяването на Светия Дух,
\par 6 когото изля изобилно върху нас чрез Исуса Христа, нашия спасител,
\par 7 та, оправдани чрез Неговата благодат, да станем, според надеждата, наследници на вечния живот.
\par 8 Вярно е това слово. И желая да настояваш върху това, с цел ония които са повярвали в Бога, да се упражняват старателно в добри дела. Това е добро и полезно за човеците.
\par 9 А отбягвай глупавите разисквания, родословия, препирни и карания върху закона, защото те са безполезни и суетни.
\par 10 След като съветваш един два пъти човек, който е раздорник, остави го,
\par 11 като знаеш, че такъв се е извратил и съгрешава, та от само себе си е осъден.
\par 12 Когато изпратя до тебе Артема или Тихика, постарай се да дойдеш при мене в Никопол, защото съм решил там да презимувам.
\par 13 Погрижи се да изпратиш на път законника Зина и Аполоса, тъй щото да не им липсва нищо.
\par 14 Нека се учат и нашите се упражняват старателно в такива добри дела, за да не бъдат безплодни, в посрещане на необходимите нужди.
\par 15 Поздравяват те всички които са с мене. Поздрави ония, които ни любят във вярата. Благодат да бъде с всички ви. [Амин].

\end{document}