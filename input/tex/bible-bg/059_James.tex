\begin{document}

\title{Яков}


\chapter{1}

\par 1 Яков слуга на Бога и на Господа Исуса Христа, до дванадесет пръснати племена, поздрав.
\par 2 Считайте го за голяма радост, братя мои, когато падате в разни изпитни,
\par 3 като знаете, че изпитанието на вашата вяра произвежда твърдост.
\par 4 А твърдостта нека извърши делото си съвършено, за да бъдете съвършени и цели, без никакъв недостатък.
\par 5 Но ако някому от вас не достига мъдрост, нека иска от Бога, Който дава на всички щедро без да укорява, и ще му се даде.
\par 6 Но да проси с вяра без да се съмнява ни най-малко; защото, който се съмнява прилича на морски вълни, които се тласкат и блъскат от ветровете.
\par 7 Такъв човек да не мисли, че ще получи нещо от Господа,
\par 8 понеже е колеблив и непостоянен във всичките си пътища.
\par 9 Братът, който е в по-долно състояние, нека се хвали, когато се въздига,
\par 10 а богатият - когато се смирява, понеже ще прецъфти като цвета на тревата.
\par 11 Защото слънцето изгрява с изсушителния вятър, тревата изсъхва, цветът й окапва, и красотата на изгледа й изчезва: така и богатият ще повехне в пътищата си.
\par 12 Блажен онзи човек, който издържа изпитня; защото като бъде одобрен, ще приеме за венец живота, който Господ е обещал на ония, които Го любят.
\par 13 Никой, който се изкушава, да не казва: Бог ме изкушава, защото Бог се не изкушава от зло, и Той никого не изкушава.
\par 14 А който се изкушава, се завлича и подлъгва от собствената си страст;
\par 15 и тогава страстта зачева и ражда грях, а грехът, като се развие напълно ражда смърт.
\par 16 Не се заблуждавайте, любезни мои братя;
\par 17 Всяко дадено добро и всеки съвършен дар е отгоре, и слиза от Отца на светлините, у Когото няма изменение, или сянка от промяна.
\par 18 От собствената Си воля ни е родил чрез словото на истината, за да бъдем един вид пръв плод на Неговите създания.
\par 19 Вие знаете това, любезни мои братя. Обаче нека човек бъде бърз да слуша, бавен да говори и бавен да се гневи;
\par 20 защото човешкия гняв не върши Божията правда.
\par 21 Затова, като отхвърлите всяка нечистота и преливаща се злоба, приемайте с кротост всаденото слово, което може да спаси душите ви,
\par 22 Бивайте и изпълнители на словото, а не само слушатели, да лъжете себе си.
\par 23 Защото ако някой бъде слушател на словото, а не изпълнител, той прилича на човек, който гледа естественото си лице в огледалото;
\par 24 понеже се огледва, отива си, и завчас забравя какъв бе.
\par 25 Но който вникне в съвършения закон на свободата и постоянствува, той, като не е забравлив слушател, но деятелен изпълнител, ще бъде блажен в дейността си.
\par 26 Ако някой счита себе си за благочестив, а не обуздава езика си, но мами сърцето си, неговото благочестие е суетно.
\par 27 Чисто и непорочно благочестие пред Бога и Отца, ето що е: да пригледва човек сирачетата и вдовиците в неволята им, и да пази себе си неопетнен от света.

\chapter{2}

\par 1 Братя мои, да не държите вярата на прославения наш Господ Исус Христос с лицеприятие.
\par 2 Защото, ако влезе в синагогата ви човек с златен пръстен и с хубави дрехи, а влезе и сиромах с оплескани дрехи,
\par 3 и погледете с почит към оня, който е с хубави дрехи, та речете: Ти седни тука на добро място ; а на сиромаха речете: Ти стой там, или: Седни до подножието ми,
\par 4 не правите ли различия помежду си, и не ставате ли пристрастни съдии?
\par 5 Слушайте любезни ми братя: Не избра ли Бог ония, които от сиромаси в светски неща, богати с вяра и наследници на царството, което е обещал на тия, които Го любят?
\par 6 А вие опозорихте сиромаха. Нали богатите ви угнетяват и сами ви влачат по съдилища?
\par 7 Нали те хулят почтеното име, с което се именувате?
\par 8 Обаче, ако изпълнявате царският закон, според писанието: Да обичаш ближният си като себе си, добре правите.
\par 9 Но ако гледате на лице, грях правите, и от закона се осъждате като престъпници.
\par 10 Защото, който опази целия закон, а сгреши в едно нещо, бива виновен във всичко.
\par 11 Понеже Оня, Който е рекъл: Не прелюбодействувай, рекъл е и: Не убивай, тъй че, ако не прелюбодействуваш, а пък убиваш, станал си престъпник на закона.
\par 12 Така говорете и така постъпвайте като човеци, които ще бъдат осъдени по закона на свободата.
\par 13 Защото съдът е немилостив към този, който е показал милост. Милостта тържествува над съда.
\par 14 Каква полза братя мои, че има вяра, а няма дела? Може ли такава вяра да го спаси?
\par 15 Ако някой брат или някоя сестра са голи и останали без ежедневна храна,
\par 16 и някой от вас им рече: Идете си с мир, дано бъдете стоплени и нахранени, а не им дадете потребното за тялото, каква полза?
\par 17 Така и вярата, ако няма дела, сама по себе си е мъртва.
\par 18 Но ще рече някой: Ти имаш вяра, а пък аз имам дела; ако можеш, покажи ми вярата си без дела, и аз ще ти покажа вярата си от моите дела.
\par 19 Ти вярваш, че има само един Бог, добре правиш; и бесовете вярват и треперят.
\par 20 Обаче искаш ли да познаеш, о суетни човече, че вяра без дела е безплодна?
\par 21 Авраам, нашият отец, не оправда ли се чрез дела като принесе сина си Исаака на жертвеника?
\par 22 Ти виждаш, че вярата действуваше заедно с делата му, и че от делата се усъвършенствува вярата;
\par 23 и изпълни се писанието, което казва: Аврам повярва в Бога; и това му се вмени за правда и се нарече Божий приятел.
\par 24 Виждате, че чрез дела се оправдава човек, а не само чрез вяра.
\par 25 Така също блудницата Раав не оправда ли се чрез дела, когато прие пратеници и ги изпрати бърже през друг път?
\par 26 Защото, както тялото отделено от духа е мъртво, така и вярата отделена от дела е мъртва.

\chapter{3}

\par 1 Братя мои, не ставайте мнозина учители, като знаете, че ще приемем по-тежко осъждане.
\par 2 Защото всички ние в много неща грешим; а който не греши в говорене, той е съвършен мъж, способен да обуздае и цялото тяло.
\par 3 Ето, ние туряме юздите в устата на конете, за да ни се покоряват, и обръщаме цялото им тяло.
\par 4 Ето, и корабите, ако да са толкова големи, и се тласкат от силни ветрове, пак с твърде малко кормило се обръщат на където желае кормчията.
\par 5 Така и езикът е малка част от тялото, но много се хвали. Ето, съвсем малко огън, колко много вещество запалва!
\par 6 И езикът, тоя цял свят от нечестие, е огън. Между нашите телесни части езикът е, който заразява цялото тяло и запалва колелото на живота, ни, а сам той се запалва от пъкъла.
\par 7 Защото всякакъв вид зверове, птици, гадини, и морски животни се укротяват и укротени са били от човечеството;
\par 8 но езикът никой човек не може да укроти; буйно зло е, пълен е със смъртоносна отрова.
\par 9 С него благославяме Господа и Отца, и с него кълнем човеците създадени по Божие подобие!
\par 10 От същите уста излизат благословения и проклятия! Братя мои, не трябва това така да бъде.
\par 11 Изворът пуща ли от същото отверстие сладка и горчива вода?
\par 12 Възможно ли е, братя мои смоковницата да роди маслини, или лозата смокини? Така също не може солената вода да дава сладка.
\par 13 Кой от вас е мъдър и разумен? Нека показва своите дела чрез добрият си живот, с кротостта на мъдростта.
\par 14 Но ако в сърцето си имате горчива завист и крамолничество, не се хвалете и не лъжете против истината.
\par 15 Това не е мъдрост, която слиза отгоре, но е земна, животинска, бесовска;
\par 16 защото, гдето има завист и крамолничество, там има бъркотия и всякакво лошо нещо.
\par 17 Но мъдростта, която е отгоре, е преди всичко чиста, после миролюбива, кротко умолима, пълна с милост и добри плодове, примирителна, нелицемерна.
\par 18 А плодът на правдата се сее с мир от миротворците.

\chapter{4}

\par 1 Отгде произлизат боеве, и отгде крамоли, между вас? Не от там ли, от вашите сласти, които воюват в телесните ви части?
\par 2 Пожелавате, но нямате; ревнувате и завиждате, но не можете да получите; карате се и се биете; но нямате, защо не просите
\par 3 Просите и не получавате, защото зле просите, за да иждивявате в сластите си.
\par 4 Прелюлбодейци! не знаете ли, че приятелството със света е вражда против Бога? И тъй който иска да бъде приятел на света, става враг на Бога.
\par 5 Или мислите,че без нужда казва писанието, че Бог и до завист ревнува за духа, който е турил да живее в нас?
\par 6 Но Той дава една голяма благодат; затова казва: "Бог на горделивите се противи, а на смирените дава благодат".
\par 7 И тъй, покорявай се на Бога, но противете се на дявола, и той ще бяга от вас.
\par 8 Приближавайте се при Бога, и ще се приближава и Той до вас. Измивайте си ръцете, вие грешни, очиствайте сърцата си, вие колебливи.
\par 9 Тъжете, ридайте и плачете; смехът ви нека се обърне на плач, и радостта ви в тъга.
\par 10 Смирявайте се, пред Господа, и Той ще ви възвишава,.
\par 11 Не се одумвайте един друг, братя; който одумва брата или съди брата си, одумва закона и съди закона; а ако съдиш закона, не си изпълнител на закона, но съдия.
\par 12 Само един е законодател и съдия, Който може да спаси и да погуби; а ти кой си та съдиш ближния си?
\par 13 Слушайте сега вие, които казвате: Днес или утре ще отидем в еди-кой-си град, ще преседим там една година, и ще търгуваме и ще спечелим,
\par 14 когато вие не знаете какво ще бъде утре. Що е животът ви? Защото вие сте пара, която се явява, и после изчезва.
\par 15 Вместо това, вие тепърва да казвате: Ако ще Господ, ние ще живеем и ще направим това или онова.
\par 16 Но сега славно ви е да се хвалите. Всяка такава хвалба е зло.
\par 17 Прочее, ако някой знае да прави добро и не го прави, грях е нему.

\chapter{5}

\par 1 Дойдете сега, вие богатите, плачете и ридайте поради бедствията, които идат върху вас.
\par 2 Богатството ви изгни, и дрехите са изядени от молци
\par 3 Златото ви и среброто ви ръждясаха, и ръждата им ще свидетелствува против вас, и ще пояде месата ви като огън. Вие сте събирали съкровища в последните дни.
\par 4 Ето, заплатата за работниците, които са жънали нивите ви, от която ги лишихте, вика; и виковете на жетварите влязоха в ушите на Господа на Силите.
\par 5 Вие живяхте на земята разкошно и разпуснато, угоихте сърцата си като в ден на клане.
\par 6 Осъдихте, убихте Праведния; и Той не ви се противи.
\par 7 И тъй, братя, останете твърди, до Господното пришествие. Ето земеделецът очаква скъпоценния плод от земята и търпи за него докле получи и ранния и късния дъжд.
\par 8 Останете и вие твърди, и укрепете сърцата си, защото Господното пришествие наближи.
\par 9 Не роптайте един против друг, братя, за да не бъдете осъдени, ето, Съдията стои пред вратите.
\par 10 Братя, вземайте за пример на злострадание и на твърдост пророците, които говореха за Господното име.
\par 11 Ето, ублажаваме ония, които са останали твърди. Чули сте за търпението на Йоана, и видели сте сетнината въздадена нему от Господа, че Господ е много жалостив и милостив.
\par 12 А преди всичко, братя мои, не се кълнете нито в небето, нито в земята, нито в друга някоя клетва, но нека бъде вашето говорене: Да, да, и: Не, не, за да не паднете под осъждане.
\par 13 Зле ли страда някой от вас? нека се моли. Весел ли е някой? нека пее хваления.
\par 14 Болен ли е някой от вас? нека повика църковните презвитери, и нека се помолят над него и го помажат с масло в Господното име.
\par 15 И молитвата, която е с вяра, ще избави страдалеца. Господ ще го привдигне, и, ако е извършил грехове, ще му се простят.
\par 16 И тъй, изповядайте един на друг греховете си, и молете се един за друг, за да оздравеете. Голяма сила има усърдната молитва на праведния.
\par 17 Илия беше човек със същото естество като нас; и помоли се усърдно да не вали дъжд, и не вали дъжд на земята три години и шест месеца;
\par 18 и пак се помоли и небето даде дъжд, и земята произведе плода си.
\par 19 Братя мои, ако някой от вас заблуди от истината, и един го обърне,
\par 20 нека знае, че който е обърнал грешния от заблудения му път ще спаси душа от смърт и ще покрие много грехове.

\end{document}