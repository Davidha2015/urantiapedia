\begin{document}

\title{2 Peter}


\chapter{1}

\par 1 Симон Петър, слуга и апостол Исус Христов, на вас, които чрез правдата на нашия Бог и Спасител Исус Христос сте получили еднаква с нас скъпоценна вяра:
\par 2 Благодат и мир да ви се умножи чрез познаването на Бога и на Исуса, нашия Господ.
\par 3 Понеже Неговата божествена сила ни е подарила всичко що е потребно за живота и за благочестието, чрез познаването на Този, Който ни е призовал чрез Своята слава и сила:
\par 4 чрез които се подариха скъпоценните нам и твърде големи обещания, за да станете чрез тях участници на божественото естество, като сте избягали от произлязлото от страстите разтление в света;
\par 5 то по самата тая причина положете всяко старание и прибавете на вярата си добродетел, на добродетелта си благоразумие,
\par 6 на благоразумието си себеобуздание, на себеобузданието си твърдост, на твърдостта си благочестие,
\par 7 на благочестието си братолюбие, и на братолюбието си любов.
\par 8 Защото ако тия добродетели се намират у вас и изобилват, те ви правят да не сте безделни нито безплодни в познаването на нашия Господ Исус Христос.
\par 9 Но оня, у когото те не се намират, е сляп, късоглед, и е забравил, че е бил очистен от старите си грехове.
\par 10 Затова, братя, постарайте се още повече да затвърдявате вашето призвание и избиране; защото като вършите тия добродетели, никога няма да изпаднете.
\par 11 Понеже така ще ви се даде голям достъп във вечното царство на нашия Господ и Спасител Исус Христос.
\par 12 Затова всякога ще бъда готов да ви напомням за тия работи, ако и да ги знаете и да сте утвърдени в истината, която сега държите.
\par 13 И мисля, че е право, докато съм в тая телесна хижа, да ви подтиквам чрез напомняне;
\par 14 понеже зная, че скоро ще напусна хижата си, както ми извести нашият Господ Исус Христос.
\par 15 Даже ще се постарая щото вие и след смъртта ми, да можете всякога да помните тия работи.
\par 16 Защото, когато ви обявихме силата и пришествието на нашия Господ Исус Христос, ние не следвахме хитроизмислени басни, а бяхме очевидци на Неговото величие.
\par 17 Защото Той прие от Бога Отца почест и слава, когато от великолепната слава дойде от Него такъв глас: Този е Моят възлюбен Син, в Когото е Моето благоволение.
\par 18 Тоя глас чухме сами ние, че дойде от небето, когато бяхме с Него на светата планина.
\par 19 И така, пророческото слово повече се потвърждава за нас; и вие добре правите, че внимавате на него, като на светило, което свети в тъмно място, догде се зазори, и зорницата изгрее в сърцата ви.
\par 20 И това да знаете преди всичко, че никое пророчество в писанието не е частно на пророка обяснение на Божията воля:
\par 21 защото никога не е идвало пророчество от човешка воля, но [светите] човеци са говорили от Бога; движими от Святия Дух.

\chapter{2}

\par 1 Но имало е лъжливи пророци между людете, както и между вас ще има лъжливи учители, които ще въведат тайно гибелни ереси, като се отричат даже от Господаря, Който ги е купил, та ще навлекат на себе си бърза погибел.
\par 2 И мнозина ще последват техните похотливи дела, поради които човеци пътят на истината ще се похули.
\par 3 От лакомство те ще ви мамят с присторени думи; но тяхната присъда, отдавна приготвена, не се забавя, и тяхното погубление не дреме.
\par 4 Защото, ако Бог не пощади и ангели, когато съгрешиха, но ги хвърли в мрака на най-дълбоките ровове, и ги предаде да бъдат вардени за съд;
\par 5 и ако не пощади стария свят, но опази с още седем души Ноя, проповедника на правдата, когато нанесе потоп върху нечестивия свят;
\par 6 тъй също, ако Той осъди на разорение содомските и гоморските градове и ги обърна на пепел, и ги постави за пример на ония, които щяха да вършат нечестие,
\par 7 но избави праведния Лот, комуто бе досадил развратния живот на нечестивите;
\par 8 (защото тоя праведен човек, като живееше между тях измъчваше от ден на ден праведната си душа, като гледаше и слушаше беззаконните им дела);
\par 9 то знае Господ как да избави благочестивите от напаст, а неправедните да държи под наказание за съдния ден,
\par 10 а особено тия, които с нечисто пожелание следват плътските страсти и презират властта. Смели и упорити, те не се страхуват да хулят славните същества;
\par 11 докато ангелите, ако и да са по-големи в мощ и сила, не представят против тях хулителна присъда пред Господа.
\par 12 Тия, обаче, като животни без разум, естествено родени, за да бъдат ловени и изтребвани, хулят за неща, които не знаят, и ще погинат в своя разврат,
\par 13 наближаващи да получат заплатата на неправдата - люде, които считат за удоволствие да разкошествуват денем. Те са петна и позор; наслаждават се на примамките си, когато са на угощение при вас;
\par 14 очите им са пълни с блудство и с непрестанен грях; подмамват неутвърдени души; сърцето им е научено на лакомство; те са предадени на проклетия;
\par 15 оставиха правия път и се заблудиха, като последваха пътя на Валаама Веоров, който обикна заплатата на неправдата,
\par 16 но биде изобличен за своето беззаконие, когато ням осел проговори с човешки глас и възпря лудостта на пророка.
\par 17 Те са безводни кладенци, мъгли тласкани от буря, за които е запазена мрачна тъмнина [до века].
\par 18 Защото, като говорят с надуто празнословие, те с разтленността подмамват в плътските страсти ония, които едвам избягват от живеещите в заблуда.
\par 19 Обещават им свобода, а те сами са роби на разврата; защото от каквото е победен някой, на това и роб става.
\par 20 Понеже, ако, след като са избягали от светските мърсотии чрез познаването на Господа и Спасителя Исус Христа, те пак са се сплели в тях и остават победени, то последното им състояние е станало по-лошо от първото.
\par 21 Понеже по-добре би било за тях да не бяха познали пътя на правдата, отколкото след като са го познали, да се отвърнат от предадената на тях света заповед.
\par 22 С тях се е случило това, което казва истинската пословица: Псето се върна на бълвоча си, и Окъпаната свиня се върна да се валя в тинята.

\chapter{3}

\par 1 Ето възлюбени, пиша ви това второ послание; и в двете събуждам чрез напомняне вашия чист разум,
\par 2 за да помните думите, изговорени по-напред от светите пророци и заповедта на Господа Спасителя, дадена чрез изпратените вам апостоли.
\par 3 Преди всичко знайте това, че в последните дни ще дойдат подиграватели, които с подигравките си ще ходят по своите страсти и ще казват:
\par 4 Где е обещаното Му пришествие? защото, откак са се поминали бащите ни всичко си стои така, както от началото на създанието.
\par 5 Защото те своеволно не признават това, че чрез Божието слово от начало е имало небе и земя сплотена от водата, и всред водата,
\par 6 но пак посредством тях тогавашния свят, потопен от водата загина.
\par 7 Така със същото слово, и днешните небе и земя са натрупани за огън, пазени до деня на страшния съд и погибелта на нечестивите човеци
\par 8 Още и това нещо да не забравяте, възлюбени, че за Господ един ден е като хиляда години, и хиляда години като един ден.
\par 9 Господ не забравя това, което е обещал, според както някои смятат бавенето, но заради вас търпи за дълго време; понеже не иска да погинат някои, но всички да дойдат на покаяние.
\par 10 А Господният ден ще дойде като крадец, когато небето ще премине с бучение, а стихиите нажежени ще се стопят, и земята и каквото се е вършило по нея ще изчезнат.
\par 11 Прочее, понеже всичкото това ще се стопи, какви трябва да сте вие в своето живеене и в благочестие,
\par 12 като очаквате и ожидате дохождането на Божия ден, поради който небето възпламенено ще се стопи, и стихиите нажежени ще се разложат!
\par 13 А според обещанието Му очакваме ново небе и нова земя, в която да живее правда.
\par 14 Затова, възлюблени, като очаквате тия неща, старайте се да се намерите чисти и непорочни пред Него, с мир в сърцата си.
\par 15 И считайте дълготърпението на нашия Господ като средство за спасение; както любезният ни брат Павел ви е писал, според дадената му мъдрост.
\par 16 както пише и във всичките си послания, когато говори в тях за тия работи; в които послания има някои неща мъчни за разбиране, които неучените и неутвърдените изопачават, както правят и с другите писания, за своята погибел.
\par 17 И тъй, вие, възлюбени, като сте предизвестени за това, пазете се да не би да се завлечете от заблуждението на беззаконните и да отпаднете от утвърждението си.
\par 18 Но растете в благодатта и познаването на нашия Господ и Спасител Исус Христос. Нему да бъде слава и сега и до вечния ден. Амин.

\end{document}