\begin{document}

\title{3 Йоаново}


\chapter{1}

\par 1 От Презвитера до любезния Гай,  когото любя:
\par 2 Възлюбений, молитствувам да благоуспееш и да си здрав във всичко, както благоуспява душата ти.
\par 3 Защото много се зарадвах, когато дойдоха някои братя и засвидетелствуваха за твоята вярност, според както ти ходиш в истината.
\par 4 По-голяма радост няма за мене от това, да слушам, че моите чада ходят в истината.
\par 5 Възлюбений, ти вършиш вярна работа в каквото правиш за братята, и то за чужденци;
\par 6 които свидетелствуваха пред църквата за твоята любов. Добре ще сториш да ги изпратиш както подобава пред Бога;
\par 7 защото за Христовото име излязоха, без да вземат нищо от езичниците.
\par 8 Ние, прочее, сме длъжни да посрещаме такива радостно, за да бъдем съработници с истината.
\par 9 Писах няколко думи до църквата; но Диотреф, който обича да първенствува между тях, не ни приема.
\par 10 Затова, ако дойда, ще му напомня за делата, които върши, като бръщолеви против нас лоши думи. И като не се задоволява с това, той не просто че сам не приема братята, но възпира и тия, които искат да ги приемат, и ги пъди от църквата.
\par 11 Възлюбений, не подражавай злото, но доброто. Който върши добро, от Бога е; който върши зло, не е видял Бога.
\par 12 За Димитрия се свидетелствува добро от всички, и от самата истина; а още и ние свидетелствуваме и ти знаеш, че нашето свидетелство е истинско.
\par 13 Имах много да ти пиша, но не ми се ще да ти пиша с мастило и перо;
\par 14 а надявам се скоро да те видя, и ще разговорим уста с уста. Мир на тебе. Поздравяват те приятелите. Поздрави приятелите по име.


\end{document}