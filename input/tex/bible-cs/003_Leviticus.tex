\begin{document}

\title{Leviticus}

\chapter{1}

\par 1 Povolal pak Hospodin Mojžíše, a mluvil k nemu z stánku úmluvy, rka:
\par 2 Mluv k synum Izraelským a rci jim: Když by kdo z vás obetoval obet Hospodinu, z hovad, totiž z volu aneb z drobného dobytka obetovati budete obet svou.
\par 3 Jestliže zápalná obet jeho byla by z skotu, samce bez vady obetovati bude. U dverí stánku úmluvy obetovati jej bude dobrovolne, pred oblícejem Hospodinovým.
\par 4 A vloží ruku svou na hlavu obeti zápalné, i bude príjemná jemu k ocištení jeho.
\par 5 Tedy zabije volka toho pred tvárí Hospodinovou, a obetovati budou kneží, synové Aronovi, krev, a pokropí tou krví oltáre svrchu vukol, kterýž jest u dverí stánku úmluvy.
\par 6 I stáhne kuži s obeti zápalné, a rozseká ji na díly své.
\par 7 A dají synové Arona kneze ohen na oltár, a narovnají dríví na tom ohni.
\par 8 Potom zporádají kneží, synové Aronovi, díly ty, hlavu i tuk, na dríví vložené na ohen, kterýž jest na oltári.
\par 9 A droby jeho i nohy jeho vymyjete vodou. I páliti bude knez všecko to na oltári; zápal jest v obet ohnivou, vune spokojující Hospodina.
\par 10 Jestliže pak z drobného dobytka bude obet jeho, z ovcí aneb z koz k obeti zápalné, samce bez poškvrny obetovati bude.
\par 11 A zabije ho pri strane oltáre pulnocní pred tvárí Hospodinovou, a pokropí kneží, synové Aronovi, krví oltáre vukol.
\par 12 I rozseká ho na díly jeho s hlavou jeho i tukem jeho, a zporádá je knez na dríví vložené na ohen, kterýž jest na oltári.
\par 13 A streva jeho i nohy jeho vymyje vodou. Tedy obetovati bude knez všecko to, a páliti bude to na oltári; zápal jest v obet ohnivou, vune spokojující Hospodina.
\par 14 Jestliže pak z ptactva obet zápalnou obetovati bude Hospodinu, tedy at obetuje z hrdlicek aneb z holoubat obet svou.
\par 15 I vloží je knez na oltár a nehtem natrhne hlavy jeho, a zapálí na oltári, vytlace krev jeho na stranu oltáre.
\par 16 Odejme také vole jeho s necistotami jeho, a povrže je blízko oltáre k strane východní na místo, kdež jest popel.
\par 17 A natrhne ho za krídla jeho, kterýchž však neodtrhne. I páliti bude je knez na oltári na dríví, kteréž jest na ohni; zápalt jest v obet ohnivou, vune spokojující Hospodina.

\chapter{2}

\par 1 Když by pak který clovek obetoval dar obeti suché Hospodinu, mouka belná bude obet jeho. I poleje ji olejem a vloží na ni kadidlo.
\par 2 Prinese ji pak k synum Aronovým knežím, a vezme odtud plnou hrst svou té mouky belné a toho oleje se vším kadidlem jejím; i páliti to bude knez na památku její na oltári v obet ohnivou, vune spokojující Hospodina.
\par 3 Ostatek pak té obeti suché bude Aronovi i synum jeho, svaté svatých, z ohnivých obetí Hospodinových.
\par 4 Když bys pak obetoval dar obeti suché, pecené v peci, at jsou z mouky belné kolácové nekvašení, zadelaní olejem, aneb oplatkové presní olejem pomazaní.
\par 5 Jestliže pak obet suchou na pánvici smaženou obetovati budeš, bude z mouky belné olejem zadelané a nenakvašené.
\par 6 Rozlámeš ji na kusy a naleješ na ni oleje, obet suchá jest.
\par 7 Pakli obet suchou v kotlíku pripravenou obetovati budeš, z mouky belné s olejem bude.
\par 8 I prineseš obet suchou, kteráž z tech vecí bude Hospodinu, a dáš ji knezi, kterýžto donese ji k oltári.
\par 9 A vezma knez z obeti té pametné její, páliti je bude na oltári, v obet ohnivou, vune spokojující Hospodina.
\par 10 Což pak zustane z obeti té suché, bude Aronovi i synum jeho, svaté svatých, z ohnivých obetí Hospodinových.
\par 11 Žádná obet suchá, kterouž obetovati budete Hospodinu, nebude kvašena; nebo nižádného kvasu, ani medu nebudete obetovati v obet ohnivou Hospodinu.
\par 12 V obeti prvotin toliko obetovati budete je Hospodinu, ale na tento oltár neobetujte jich u vuni spokojující.
\par 13 Také všeliký dar suché obeti své solí osolíš, a neodejmeš soli smlouvy Boha svého od suché obeti své. Pri každé obeti své sul obetovati budeš.
\par 14 Jestliže bys pak obetoval obet suchou z prvotin Hospodinu, klasy nové ohnem upražíš, a což vymneš z tech klasu nových, to obetovati budeš, suchou obet prvotin svých.
\par 15 A poleješ ji svrchu olejem, kadidlo také vložíš na ni; obet suchá jest.
\par 16 I páliti bude knez pametné její z obilí zetreného, a z oleje toho, se vším tím kadidlem jejím; v obet ohnivou bude Hospodinu.

\chapter{3}

\par 1 Jestliže pak obet pokojná bude obet jeho, když by z skotu obetoval, bud vola neb krávu, bez vady bude obetovati je pred tvárí Hospodinovou.
\par 2 Tedy položí ruku svou na hlavu obeti své, i zabije ji u dverí stánku úmluvy; a kropiti budou kneží synové Aronovi krví na oltár vukol.
\par 3 Potom obetovati bude z obeti pokojné ohnivou obet Hospodinu, tuk streva kryjící, i všeliký tuk, kterýž jest na nich,
\par 4 Též obe ledvinky s tukem, kterýž jest na nich i na slabinách, tolikéž i branici, kteráž jest na jatrách; s ledvinkami odejme ji.
\par 5 I páliti to budou synové Aronovi na oltári spolu s obetí zápalnou, kteráž bude na dríví vloženém na ohen; obet zajisté ohnivá jest, vune spokojující Hospodina.
\par 6 Jestliže pak z drobného dobytka bude obet jeho v obet pokojnou Hospodinu, samce aneb samici, bez poškvrny obetovati bude ji.
\par 7 Jestliže by obetoval beránka, obet svou, tedy obetovati jej bude pred tvárí Hospodinovou.
\par 8 A vloží ruku svou na hlavu obeti své, a zabije ji pred stánkem úmluvy, a kropiti budou synové Aronovi krví její na oltár vukol.
\par 9 Potom obetovati bude z obeti pokojné obet ohnivou Hospodinu, tuk její i ocas celý, kterýž od hrbetu odejme, též i tuk streva kryjící se vším tím tukem, kterýž jest na nich,
\par 10 Též obe dve ledvinky s tukem, kterýž jest na nich i na slabinách, ano i branici, kteráž jest na jatrách; s ledvinkami odejme ji.
\par 11 I páliti to bude knez na oltári; pokrm jest obeti ohnivé Hospodinovy.
\par 12 Jestliže pak koza byla by obet jeho, tedy bude ji obetovati pred oblícejem Hospodinovým.
\par 13 A vloží ruku svou na hlavu její, a zabije ji pred stánkem úmluvy; i kropiti budou synové Aronovi krví její na oltári vukol.
\par 14 Potom obetovati bude z ní obet svou v obet ohnivou Hospodinu, tuk streva kryjící, i všecken tuk, kterýž jest na nich,
\par 15 Též obe dve ledvinky s tukem, kterýž jest na nich i na slabinách, ano i branici, kteráž jest na jatrách; s ledvinkami odejme ji.
\par 16 I páliti to bude knez na oltári, pokrm obeti ohnivé u vuni príjemnou. Všecken tuk Hospodinu bude.
\par 17 Právem vecným po rodech vašich, ve všech príbytcích vašich žádného tuku a žádné krve nebudete jísti.

\chapter{4}

\par 1 Mluvil také Hospodin k Mojžíšovi, rka:
\par 2 Mluv synum Izraelským a rci: Když by kdo zhrešil z poblouzení proti nekterému ze všech prikázaní Hospodinových, cine, cehož býti nemá, a prestoupil by jedno z nich;
\par 3 Jestliže by knez pomazaný zhrešil hríchem jiných lidí: tedy obetovati bude za hrích svuj, kterýmž zhrešil, volka mladého bez poškvrny Hospodinu v obet za hrích.
\par 4 I privede volka toho ke dverím stánku úmluvy pred oblícej Hospodinuv, a položí ruku svou na hlavu volka toho, a zabije ho pred oblícejem Hospodinovým.
\par 5 Tedy vezme knez pomazaný krve z toho volka, a vnese ji do stánku úmluvy.
\par 6 Potom omocí knez prst svuj v té krvi, a kropiti jí bude sedmkráte pred Hospodinem, pred oponou svatyne.
\par 7 Pomaže také knez tou krví rohu oltáre, na nemž se kadí vonnými vecmi pred Hospodinem, kterýž jest v stánku úmluvy, a ostatek krve volka toho vyleje k spodku oltáre zápalu, kterýž jest u dverí stánku úmluvy.
\par 8 Všecken pak tuk volka toho za hrích vyjme z neho, totiž tuk prikrývající streva a všecken tuk, kterýž jest na nich,
\par 9 Též obe dve ledvinky s tukem, kterýž jest na nich i na slabinách, tolikéž i branici, kteráž jest na jatrách, s ledvinkami odejme ji,
\par 10 Tak jako se odjímá od volka obeti pokojné, a páliti to bude knez na oltári zápalu.
\par 11 Kuži pak volka toho, i všecko maso jeho s hlavou i s nohami jeho, streva jeho i s lejny jeho,
\par 12 A tak celého volka vynese ven za stany na místo cisté, tam kdež se popel vysýpá, a spálí jej na dríví ohnem; na míste, kdež se popel vysýpá, spálen bude.
\par 13 Jestliže by pak všecko množství Izraelské pobloudilo, a byla by ta vec skrytá pred ocima shromáždení toho, a ucinili by proti nekterému ze všech prikázaní Hospodinových, cehož by býti nemelo, tak že by vinni byli,
\par 14 A byl by poznán hrích, kterýmž zhrešili: tedy obetovati bude shromáždení volka mladého v obet za hrích, a privedou ho pred stánek úmluvy.
\par 15 I položí starší shromáždení toho ruce své na hlavu volka pred Hospodinem, a zabije volka pred Hospodinem.
\par 16 I vnese knez pomazaný z krve volka toho do stánku úmluvy,
\par 17 A omoce knez prst svuj v té krvi, kropiti jí bude sedmkráte pred Hospodinem, pred oponou.
\par 18 Pomaže také knez tou krví rohu oltáre, kterýž jest pred Hospodinem v stánku úmluvy; potom všecku krev pozustávající vyleje k spodku oltáre zápalu, kterýž jest u dverí stánku úmluvy.
\par 19 Všecken také tuk jeho vyjme z neho, a páliti bude na oltári.
\par 20 S tím pak volkem tak uciní, jako ucinil s volkem za hrích obetovaným, rovne tak uciní s ním. A tak ocistí je knez, i bude jim odpušteno.
\par 21 Volka pak vynese ven z táboru a spálí jej, jako spálil volka prvního; nebo obet za hrích shromáždení jest.
\par 22 Jestliže pak kníže zhreší, a uciní proti nekterému ze všech prikázaní Hospodina Boha svého, cehož býti nemelo, a to z poblouzení, tak že vinen bude,
\par 23 A byl by znám hrích jeho, jímž zhrešil: tedy privede obet svou z koz, samce bez poškvrny.
\par 24 I položí ruku svou na hlavu toho kozla a zabije ho na míste, kdež se bijí obeti zápalné pred Hospodinem; obet za hrích jest.
\par 25 A vezma knez z krve obeti za hrích na prst svuj, pomaže rohu oltáre zápalu, ostatek pak krve jeho vyleje k spodku oltáre zápalu.
\par 26 Všecken také tuk jeho páliti bude na oltári, jako i tuk obetí pokojných. A tak ocistí jej knez od hríchu jeho, a odpušten bude jemu.
\par 27 Jestliže pak zhreší clovek z lidu obecného z poblouzení, ucine proti nekterému prikázaní Hospodinovu, cehož by nemelo býti, tak že vinen bude,
\par 28 A byl by znám hrích jeho, kterýmž zhrešil: tedy privede obet svou z koz, samici bez poškvrny, za hrích svuj, jímž zhrešil.
\par 29 I položí ruku svou na hlavu obeti té za hrích a zabije tu obet za hrích na míste obetí zápalných.
\par 30 A vezma knez z krve její na prst svuj, pomaže rohu oltáre zápalu, a ostatek krve její vyleje u spodku oltáre.
\par 31 Všecken také tuk její odejme, jako se odjímá tuk od obetí pokojných, a páliti jej bude knez na oltári u vuni príjemnou Hospodinu. A tak ocistí jej knez, a bude mu odpušteno.
\par 32 Pakli by z ovcí prinesl obet svou za hrích, samici bez poškvrny prinese.
\par 33 A vloží ruku svou na hlavu té obeti za hrích a zabije ji v obet za hrích na míste, kdež se bijí obeti zápalné.
\par 34 Potom vezma knez krve z obeti té za hrích na prst svuj, pomaže rohu oltáre zápalu, ostatek pak krve její vyleje k spodku oltáre.
\par 35 Všecken také tuk její odejme, jakž se odjímá tuk beránka z obetí pokojných, a páliti jej bude knez na oltári v obet ohnivou Hospodinu. A tak ocistí jej knez od hríchu jeho, kterýmž zhrešil, a odpušten bude jemu.

\chapter{5}

\par 1 Zhrešil-li by clovek, tak že slyše hlas zakletí a jsa svedkem toho, což videl neb slyšel, a neoznámil by, poneset pokutu za nepravost svou.
\par 2 Aneb jestliže by se dotkl clovek nekteré veci necisté, budto tela zveri necisté, budto tela hovada necistého, aneb tela žížaly necisté, a bylo by to skryto pred ním, tedy necistý bude a vinen jest.
\par 3 Aneb jestliže by se dotkl necistoty cloveka, jaká by koli byla necistota jeho, kterouž se poškvrnuje, a bylo by to skryto pred ním, a potom poznal by to, vinen jest.
\par 4 Aneb jestliže by se kdo zaprisáhl, vynášeje to rty svými, že uciní neco zlého aneb dobrého, a to o jakékoli veci, o níž clovek s prísahou obycej má mluviti, a bylo by to skryto pred ním, a potom by poznal, že vinen jest jednou vecí z tech,
\par 5 Když tedy vinen bude jednou vecí z tech: vyzná hrích svuj,
\par 6 A privede obet za vinu svou Hospodinu, za hrích svuj, kterýmž zhrešil, samici z dobytku drobného, ovci aneb kozu za hrích, a ocistít jej knez od hríchu jeho.
\par 7 A pakli by s to býti nemohl, aby dobytce obetoval, tedy prinese obet za vinu svou, kterouž zhrešil, dve hrdlicky aneb dvé holoubátek Hospodinu, jedno v obet za hrích a druhé v obet zápalnou.
\par 8 I prinese je k knezi, a on obetovati bude nejprvé to, kteréž má býti v obet za hrích, a nehtem natrhne hlavy jeho naproti tylu jeho, však nerozdelí jí.
\par 9 I pokropí krví z obeti za hrích strany oltáre, a což zustane krve, vytlací ji k spodku oltáre; nebo obet za hrích jest.
\par 10 Z druhého pak uciní obet zápalnou vedlé obyceje. A tak ocistí jej knez od hríchu jeho, kterýmž zhrešil, a bude mu odpušten.
\par 11 A pakli nemuže s to býti, aby prinesl dve hrdlicky aneb dvé holoubátek, tedy prinese obet svou ten, kterýž zhrešil, desátý díl míry efi mouky belné v obet za hrích. Nenalejet na ni oleje, aniž položí na ni kadidla, nebo obet za hrích jest.
\par 12 Kterouž když prinese k knezi, tedy knez vezma z ní plnou hrst svou, pametné její, páliti to bude na oltári mimo obet ohnivou Hospodinu, obet za hrích jest.
\par 13 I ocistí jej knez od hríchu jeho, kterýmž zhrešil v kterékoli veci z tech, a budet mu odpušten; ostatek pak bude knezi jako pri obeti suché.
\par 14 Mluvil opet Hospodin k Mojžíšovi, rka:
\par 15 Kdyby clovek prestoupil prestoupením, a zhrešil by z poblouzení, ujímaje vecí posvecených Hospodinu: tedy prinese obet za vinu svou Hospodinu, skopce bez poškvrny z drobného dobytka, podlé ceny tvé, nejníž za dva loty stríbra, vedlé lotu svatyne, za vinu.
\par 16 A tak, což zhrešil, ujímaje posvecených vecí, nahradí, a pátý díl nad to pridá, dada to knezi; knez pak ocistí jej skrze skopce obeti za vinu, a bude jemu odpuštena.
\par 17 Jestliže by pak clovek zhrešil, a ucinil by proti nekterému ze všech prikázaní Hospodinových, cehož by nemelo býti, neznaje toho, a byl by vinen, rovne též ponese nepravost svou.
\par 18 A privede skopce bez poškvrny z drobného dobytka vedlé ceny tvé v obet za vinu k knezi. I ocistí jej knez od poblouzení jeho, kterýmž pobloudil a o nemž nevedel, a bude mu odpušteno.
\par 19 Obet za provinení jest, nebo zavinil Hospodinu.

\chapter{6}

\par 1 Mluvil opet Hospodin k Mojžíšovi, rka:
\par 2 Kdyby clovek zhrešil a prestoupením prestoupil proti Hospodinu, bud že by oklamal bližního svého v veci sobe sverené, aneb v spolku nejakém, aneb mocí by vzal neco, aneb lstive podvedl bližního svého,
\par 3 Bud že nalezna ztracenou vec, za prelby mu ji, bud že by prisáhl falešne, kteroukoli vecí z tech, kterouž prihází se cloveku uciniti a zhrešiti jí,
\par 4 Když by tedy zhrešil a vinen byl: navrátí zase tu vec, kterouž mocí sobe vzal, aneb tu, kteréž lstive s útiskem dosáhl, aneb tu, kteráž mu sverena byla, aneb ztracenou vec, kterouž nalezl,
\par 5 Aneb o kterékoli veci falešne prisáhl: tedy navrátí to z cela, a nad to pátý díl toho pridá tomu, cíž bylo; to navrátí v den obeti za hrích svuj.
\par 6 Obet pak za hrích svuj privede Hospodinu z drobného dobytku, skopce bez poškvrny, vedlé ceny tvé, v obet za vinu knezi.
\par 7 I ocistí jej knez pred Hospodinem, a odpušteno mu bude, jedna každá z tech vecí, kterouž by ucinil a jí vinen byl.
\par 8 I mluvil Hospodin k Mojžíšovi, rka:
\par 9 Prikaž Aronovi i synum jeho a rci: Tento bude rád pri obeti zápalné: (slove pak obet zápalná od pálení na oltári celou noc až do jitra, nebo ohen na oltári vždycky horeti bude),
\par 10 Oblece se knez v roucho své lnené, a košilku lnenou vezme na telo své, a vyhrabe popel, když ohen spálí obet zápalnou na oltári, a vysype jej u oltáre.
\par 11 Potom svlece šaty své a oblece se v roucho jiné, a vynese popel ven z stanu na místo cisté.
\par 12 Ohen pak, kterýž jest na oltári, bude horeti na nem, nebude uhašován. A bude zapalovati jím knez dríví každého jitra, a zporádá na nem obet zápalnou, a páliti bude na nem tuk pokojných obetí.
\par 13 Ohen ustavicne horeti bude na oltári, a nebudet uhašen.
\par 14 Tento pak bude rád pri obeti suché, kterouž obetovati budou synové Aronovi Hospodinu u oltáre:
\par 15 Vezme hrst belné mouky z obeti té a z oleje jejího, se vším tím kadidlem, kteréž bude na obeti suché, a páliti to bude na oltári u vuni líbeznou, pametné její Hospodinu.
\par 16 Což pak zustane z ní, to jísti budou Aron i synové jeho; presné jísti se bude na míste svatém, v síni stánku úmluvy jísti to budou.
\par 17 Nebude vareno s kvasem, nebo jsem jim to dal za díl z obetí mých ohnivých; svaté svatých to jest, jako i obet za hrích a jako obet za vinu.
\par 18 Každý mužského pohlaví z synu Aronových jísti bude to právem vecným po rodech vašich, z ohnivých obetí Hospodinových. Což by se koli dotklo toho, svaté bude.
\par 19 I mluvil Hospodin k Mojžíšovi, rka:
\par 20 Tato jest obet Aronova a synu jeho, kterouž obetovati budou Hospodinu v den pomazání jeho: Desátý díl míry efi mouky belné za obet suchou ustavicnou, polovici toho ráno a polovici u vecer.
\par 21 Na pánvici s olejem strojena bude; smaženou prineseš ji, a pecené kusy obeti suché obetovati budeš u vuni spokojující Hospodina.
\par 22 A knez, kterýž z synu jeho po nem pomazán bude, at ji obetuje právem vecným. Hospodinu všecko to páleno bude.
\par 23 A všeliká suchá obet knežská celá spálena bude; nebudet jedena.
\par 24 Mluvil pak Hospodin k Mojžíšovi, rka:
\par 25 Mluv Aronovi a synum jeho a rci: Tento bude rád obeti za hrích: Na míste, kdež se zabijí obet zápalná, zabita bude obet za hrích pred Hospodinem; svatá svatých jest.
\par 26 Knez obetující tu obet za hrích budet jísti ji; na míste svatém jedena bude v síni stánku úmluvy.
\par 27 Což by se koli dotklo masa jejího, svaté bude; a jestliže by krví její šaty skropeny byly, to, což skropeno jest, obmyješ na míste svatém.
\par 28 A nádoba hlinená, v níž by vareno bylo, rozražena bude; pakli by v nádobe medené vareno bylo, vytrena a vymyta bude vodou.
\par 29 Všeliký pohlaví mužského mezi knežími jísti to bude; svaté svatých jest.
\par 30 Ale žádná obet za hrích, (z jejížto krve neco vneseno bylo by do stánku úmluvy k ocištení v svatyni), nebude jedena; ohnem spálena bude.

\chapter{7}

\par 1 Tento pak bude rád obeti za provinení; svatá svatých jest.
\par 2 Na kterémž míste zabijí se obet zápalná, na témž zabijí i obet za vinu, a pokropí krví její oltáre svrchu vukol.
\par 3 Všecken pak tuk její obetovati bude z ní, ocas i tuk streva prikrývající.
\par 4 Též obe dve ledvinky s tukem, kterýž jest na nich i na slabinách; a branici, kteráž jest na jatrách, s ledvinkami odejme.
\par 5 I bude páliti to knez na oltári v obet ohnivou Hospodinu; obet za provinení jest.
\par 6 Všeliký pohlaví mužského mezi knežími jísti bude ji, na míste svatém jedena bude; svatá svatých jest.
\par 7 Jakož obet za hrích, tak obet za vinu, jednostejné právo míti budou; knezi, kterýž by ho ocištoval, prináležeti bude.
\par 8 Knezi pak, kterýž by necí obet zápalnou obetoval, kuže té obeti zápalné, kterouž obetoval, prináležeti bude.
\par 9 Nadto všeliká obet suchá, kteráž v peci pecena bude, a všecko, což na pánvici aneb v kotlíku strojeno bude, knezi, kterýž to obetuje, prináležeti bude.
\par 10 Tolikéž všeliká obet suchá olejem zadelaná aneb upražená, všechnem synum Aronovým prináležeti bude, a to jednomu jako druhému.
\par 11 Tento pak bude rád obeti pokojné, kterouž by obetoval Hospodinu:
\par 12 Jestliže by ji obetoval v obeti chvály, tedy obetovati bude v obet chvály koláce nekvašené, olejem zadelané a oplatky nekvašené, olejem pomazané a mouku belnou smaženou, s temi koláci olejem zadelanými.
\par 13 Mimo ty koláce také chléb kvašený obetovati bude obet svou, v obet chvály pokojných obetí svých.
\par 14 A budet obetovati z neho jeden pecník, ze vší té obeti Hospodinu obet ku pozdvižení, a ten prináležeti bude tomu knezi, kterýž kropil krví té obeti pokojné.
\par 15 Maso pak obet, z té obeti chvály, jenž jest obet pokojná, v den obetování jejího jedeno bude, aniž co zustane z neho do jitra.
\par 16 Jestliže by pak z slibu aneb z dobré vule obetoval obet svou, tolikéž v den obetování jejího jedena bude; a jestliže by co zustalo z toho, tedy na druhý den jísti se bude.
\par 17 Jestliže by pak co masa z té obeti zustalo do tretího dne, ohnem spáleno bude.
\par 18 Pakli by kdo predce jedl maso obeti pokojné dne tretího, nebudet príjemný ten, kterýž ji obetoval, aniž prijata bude, ale ohavnost bude, a kdož by koli jedl je, ponese nepravost svou.
\par 19 Též maso, kteréž by se dotklo neceho necistého, nebude jedeno, ale ohnem spáleno bude; maso pak jiné, kdož by koli cistý byl, bude moci jísti.
\par 20 Nebo clovek, kterýž by jedl maso z obeti pokojné, kteráž jest Hospodinu obetována, a byl by poškvrnený: tedy vyhlazen bude clovek ten z lidu svého.
\par 21 A kdož by se dotkl neceho necistého, budto necistoty cloveka, bud hovada necistého aneb všeliké ohavnosti necisté, a jedl by maso z obeti pokojné, kteráž jest Hospodinu posvecena: tedy vyhlazen bude clovek ten z lidu svého.
\par 22 Mluvil také Hospodin k Mojžíšovi, rka:
\par 23 Mluv k synum Izraelským a rci jim: Žádného tuku z vola, aneb z ovce, aneb z kozy nebudete jísti.
\par 24 Ackoli tuk mrtvého a tuk udáveného hovada muže užíván býti k všeliké potrebe, ale jísti ho nikoli nebudete.
\par 25 Nebo kdož by koli jedl tuk z hovada, kteréž obetovati bude clovek v obet ohnivou Hospodinu, vyhlazen bude clovek ten, kterýž jedl, z lidu svého.
\par 26 Tolikéž krve žádné jísti nebudete ve všech príbytcích svých, bud z ptactva, bud z hovada.
\par 27 Všeliký clovek, kterýž by jedl jakou krev, vyhlazen bude z lidu svého.
\par 28 Mluvil opet Hospodin k Mojžíšovi, rka:
\par 29 Mluv k synum Izraelským a rci: Kdož by obetoval obet svou pokojnou Hospodinu, on sám prinese obet svou Hospodinu z obetí pokojných svých.
\par 30 Ruce jeho obetovati budou obet ohnivou Hospodinu. Tuk s hrudím prinese, a hrudí aby bylo v obet sem i tam obracení pred Hospodinem.
\par 31 Páliti pak bude knez tuk na oltári, ale hrudí to zustane Aronovi i synum jeho.
\par 32 A plece pravé dáte knezi ku pozdvižení z obetí pokojných vašich.
\par 33 Kdožkoli z synu Aronových obetovati bude krev obetí pokojných a tuk, tomu se dostane plece pravé na díl jeho.
\par 34 Nebo hrudí sem i tam obracení a plece vzhuru pozdvižení vzal jsem od synu Izraelských z obetí pokojných jejich, a dal jsem je Aronovi knezi i synum jeho právem vecným od synu Izraelských.
\par 35 Tot jest díl pomazání Aronova, a pomazání synu jeho z ohnivých obetí Hospodinových, ode dne toho, v kterémž jim pristoupiti rozkázal k vykonávání knežství Hospodinu,
\par 36 Kterýž prikázal Hospodin, aby jim ode dne, v kterémž jich pomazal, dáván byl od synu Izraelských právem vecným po rodech jejich.
\par 37 Tent jest rád obeti zápalné, obeti suché, obeti za hrích, obeti za vinu, a posvecování i obetí pokojných,
\par 38 Kteréž prikázal Hospodin Mojžíšovi na hore Sinai toho dne, když prikázal synum Izraelským, aby obetovali obeti své Hospodinu na poušti Sinai.

\chapter{8}

\par 1 I mluvil Hospodin k Mojžíšovi, rka:
\par 2 Vezmi Arona a syny jeho s ním, i roucha jejich a olej pomazání, též volka k obeti za hrích, a dva skopce, a koš s presnými chleby,
\par 3 A shromažd všecko množství ke dverím stánku úmluvy.
\par 4 I ucinil Mojžíš, jakž prikázal jemu Hospodin, a shromáždilo se všecko množství ke dverím stánku úmluvy.
\par 5 Tedy rekl Mojžíš tomu množství: Totot jest to slovo, kteréž prikázal vykonati Hospodin.
\par 6 A rozkázav pristoupiti Mojžíš Aronovi a synum jeho, umyl je vodou.
\par 7 A oblékl jej v sukni a prepásal ho pasem, a odev ho pláštem, dal náramenník svrchu na nej, a pripásal jej pasem náramenníka a otáhl ho jím.
\par 8 A vložil na nej náprsník, do nehožto dal urim a thumim.
\par 9 Potom vstavil cepici na hlavu jeho; a dal na cepici jeho po predu plech zlatý, korunu svatou, jakož prikázal Hospodin Mojžíšovi.
\par 10 Vzal také Mojžíš olej pomazání a pomazal príbytku i všech vecí, kteréž byly v nem, a posvetil jich.
\par 11 A pokropil jím oltáre sedmkrát, a pomazal oltáre i všeho nádobí jeho, též umyvadla i s podstavkem jeho, aby to všecko posveceno bylo.
\par 12 Vlil také oleje pomazání na hlavu Aronovu, a pomazal ho ku posvecení jeho.
\par 13 Rozkázal také Mojžíš pristoupiti synum Aronovým, a zoblácel je v sukne, a opásal je pasem, a vstavil na ne klobouky, jakož byl prikázal Hospodin jemu.
\par 14 A privedl volka k obeti za hrích, i položil Aron a synové jeho ruce své na hlavu volka obeti za hrích.
\par 15 I zabil jej a vzal krev jeho, a pomazal rohu oltáre vukol prstem svým, a tak ocistil oltár. Ostatek pak krve vylil k spodku oltáre a posvetil ho k ocištování na nem.
\par 16 Vzal také všecken tuk, kterýž byl na strevách, a branici s jater a obe ledvinky i tuk jejich, a pálil to Mojžíš na oltári.
\par 17 Volka pak toho i kuži jeho, i maso jeho, i lejna jeho spálil ohnem vne za stany, jakož byl prikázal Hospodin Mojžíšovi.
\par 18 Potom privedl skopce obeti zápalné, a položil Aron i synové jeho ruce své na hlavu toho skopce.
\par 19 I zabil jej, a pokropil Mojžíš krví oltáre po vrchu vukol.
\par 20 Skopce také rozsekal na kusy jeho, a pálil Mojžíš hlavu, kusy i tuk.
\par 21 Streva pak a nohy vymyl vodou, a tak spálil Mojžíš všeho skopce na oltári. I byl zápal u vuni líbeznou, obet ohnivá Hospodinu, jakož prikázal Hospodin Mojžíšovi.
\par 22 Rozkázal také privésti skopce druhého, skopce posvecení, a položil Aron i synové jeho ruce své na hlavu skopce.
\par 23 I zabil jej, a vzav Mojžíš krve jeho, pomazal jí konce pravého ucha Aronova a palce ruky jeho pravé, i palce nohy jeho pravé.
\par 24 Tolikéž synum Aronovým rozkázav pristoupiti, pomazal kraje ucha jejich pravého a palce ruky jejich pravé, též palce nohy jejich pravé, a vykropil krev na oltár svrchu vukol.
\par 25 Potom vzal tuk a ocas i všecken tuk prikrývající droby a branici s jater, též obe dve ledvinky i tuk jejich i plece pravé.
\par 26 Také z koše presných chlebu, kteríž byli pred Hospodinem, vzal jeden kolác presný a jeden pecník chleba s olejem a jeden oplatek, a položil to s tukem a s plecem pravým.
\par 27 A dal to všecko v ruce Aronovy a v ruce synu jeho, rozkázav obraceti sem i tam v obet obracení pred Hospodinem.
\par 28 Potom vzav z rukou jejich, pálil to na oltári v zápal. Posvecení toto jest u vuni rozkošnou, obet ohnivá Hospodinu.
\par 29 Vzal také Mojžíš hrudí a obracel je sem i tam v obet obracení pred Hospodinem;a z skopce posvecení dostal se Mojžíšovi díl, jakož mu byl prikázal Hospodin.
\par 30 Vzal také Mojžíš oleje pomazání a krve, kteráž byla na oltári, a pokropil Arona i roucha jeho, též synu Aronových a roucha jejich s ním. A tak posvetil Arona i roucha jeho, též synu jeho i roucha jejich s ním.
\par 31 I rekl Mojžíš Aronovi a synum jeho: Varte to maso u dverí stánku úmluvy, a jezte je tu, i chléb, kterýž jest v koši posvecení, jakož jsem prikázal, rka: Aron a synové jeho jísti budou je.
\par 32 Což by pak zustalo masa i chleba toho, ohnem to spálíte.
\par 33 A ze dverí stánku úmluvy za sedm dní nevycházejte až do dne, v kterémž by se vyplnili dnové svecení vašeho; nebo za sedm dní posvecovány budou ruce vaše.
\par 34 Jakož se stalo dnešní den, tak prikázal Hospodin ciniti k ocištení vašemu.
\par 35 Protož u dverí stánku úmluvy zustanete ve dne i v noci za sedm dní, a ostríhati budete narízení Hospodinova, abyste nezemreli; nebo tak mi jest prikázáno.
\par 36 Ucinil tedy Aron i synové jeho všecky veci, kteréž prikázal Hospodin skrze Mojžíše.

\chapter{9}

\par 1 Stalo se pak dne osmého, povolal Mojžíš Arona a synu jeho i starších Izraelských.
\par 2 I rekl Aronovi: Vezmi sobe tele k obeti za hrích, a skopce k obeti zápalné, obé bez poškvrny, a obetuj pred Hospodinem.
\par 3 K synum pak Izraelským mluviti budeš, rka: Vezmete kozla k obeti za hrích, a tele a beránka, rocní, bez vady, k obeti zápalné,
\par 4 Vola také a skopce k obeti pokojné, abyste obetovali pred Hospodinem, a obet suchou zadelanou olejem; nebo dnes se vám ukáže Hospodin.
\par 5 Tedy vzali ty veci, kteréž prikázal Mojžíš pred stánkem úmluvy, a pristoupivši všecko shromáždení, stáli pred Hospodinem.
\par 6 I rekl Mojžíš: Toto jest ta vec, kterouž prikázal Hospodin. Vykonejtež ji, a ukáže se vám sláva Hospodinova.
\par 7 Aronovi pak rekl Mojžíš: Pristup k oltári a obetuj obet za hrích svuj, a obet zápalnou svou k vykonání ocištení za sebe i za lid tento; obetuj také obet lidu všeho, a ucin ocištení za ne, jakož prikázal Hospodin.
\par 8 Pristoupiv tedy Aron k oltári, zabil tele své k obeti za hrích.
\par 9 I dali mu synové Aronovi krev. Kterýžto omociv prst svuj ve krvi, pomazal rohu oltáre, ostatek pak krve vylil k spodku oltáre.
\par 10 Ale tuk a ledvinky, i branici s jater té obeti za hrích pálil na oltári, jakož byl prikázal Hospodin Mojžíšovi.
\par 11 Maso pak s kuží spálil vne za stany.
\par 12 Zabil také obet zápalnou. I podali mu synové Aronovi krve, kterouž pokropil po vrchu oltáre vukol.
\par 13 Podali jemu také i obeti zápalné s kusy jejími i hlavy její, a pálil ji na oltári.
\par 14 A vymyv streva i nohy její, pálil je s obetí zápalnou na oltári.
\par 15 Obetoval také obet všeho lidu. A vzav kozla obeti za hrích, kterýž byl všeho lidu, zabil jej a obetoval ho za hrích jako i prvního.
\par 16 Obetoval též obet zápalnou a ucinil ji vedlé obyceje.
\par 17 Tolikéž i obet suchou obetoval, a vzav plnou hrst z ní, pálil to na oltári, mimo obet zápalnou jitrní.
\par 18 Zabil ješte i vola a skopce k obeti pokojné, kteráž byla za lid. I podali mu synové Aronovi krve, kterouž pokropil oltáre po vrchu vukol.
\par 19 Dali jemu také tuk z vola a z skopce ocas, a tuk prikrývající streva i ledvinky a branici s jater.
\par 20 A položili vcecken tuk na hrudí; i pálil ten tuk na oltári.
\par 21 Hrudí pak a plece pravé obracel sem i tam Aron v obet obracení pred Hospodinem, jakož byl prikázal Mojžíšovi.
\par 22 Potom Aron pozdvihna rukou svých k lidu, dal jim požehnání, a sstoupil od obetování obeti za hrích a obeti zápalné i obeti pokojné.
\par 23 Tedy všel Mojžíš s Aronem do stánku úmluvy; a když vycházeli z neho, požehnání dávali lidu. I ukázala se sláva Hospodinova všemu lidu.
\par 24 Nebo sstoupil ohen od tvári Hospodina, a spálil na oltári obet zápalnou i všecken tuk. Což když uzrel veškeren lid, zkrikli a padli na tvári své.

\chapter{10}

\par 1 Synové pak Aronovi Nádab a Abiu, vzavše jeden každý kadidlnici svou, dali do nich ohen a položili na nej kadidlo, a obetovali pred Hospodinem ohen cizí, cehož jim byl neprikázal.
\par 2 Protož sstoupiv ohen od Hospodina, spálil je; a zemreli tu pred Hospodinem.
\par 3 I rekl Mojžíš Aronovi: Tot jest, což mluvil Hospodin, rka: V tech, kteríž pristupují ke mne, posvecen budu, a pred oblícejem všeho lidu oslaven budu. I mlcel Aron.
\par 4 Tedy povolal Mojžíš Mizaele a Elizafana, synu Uziele, strýce Aronova, a rekl jim: Podte a vyneste bratrí své od svatyne ven za stany.
\par 5 I prišli a vynesli je v sukních jejich ven za stany, jakž byl rozkázal Mojžíš.
\par 6 Mluvil pak Mojžíš Aronovi, Eleazarovi a Itamarovi, synum jeho: Hlav svých neodkrývejte a roucha svého neroztrhujte, abyste nezemreli, a aby se Buh na všecko množství nerozhneval; ale bratrí vaši, všecka rodina Izraelská, budou plakati nad tím spálením, kteréž uvedl Hospodin.
\par 7 Vy pak ze dverí stánku úmluvy nevycházejte, abyste nezemreli; nebo olej pomazání Hospodinova jest na vás. I ucinili vedlé rozkázaní Mojžíšova.
\par 8 Mluvil také Hospodin Aronovi, rka:
\par 9 Vína a nápoje opojného nebudeš píti,ty ani synové tvoji s tebou, kdyžkoli budete míti vcházeti do stánku úmluvy, abyste nezemreli, (ustanovení vecné to bude po rodech vašich);
\par 10 Také abyste rozeznati mohli mezi svatým a neposveceným, a mezi cistým a necistým;
\par 11 Též abyste ucili syny Izraelské všechnem ustanovením, kteráž mluvil Hospodin k nim skrze Mojžíše.
\par 12 Mluvil pak Mojžíš Aronovi, Eleazarovi a Itamarovi, synum jeho, kteríž živi zustali: Vezmete obet suchou, kteráž zustala z ohnivých obetí Hospodinových, a jezte ji s presnicemi u oltáre; nebo svatá svatých jest.
\par 13 Protož jísti budete ji na míste svatém, nebo to jest právo tvé a právo synu tvých z ohnivých obetí Hospodinových; tak zajisté jest mi prikázáno.
\par 14 Hrudí pak sem i tam obracení, a plece vznášení jísti budete na míste cistém, ty i synové tvoji, i dcery tvé s tebou; nebo to právem tobe a synum tvým dáno jest z pokojných obetí synu Izraelských.
\par 15 Plece vzhuru vznášení a hrudí sem i tam obracení s obetmi ohnivými tuku, kteréž prinesou, aby sem i tam obracíno bylo, tak jako obet obracení pred Hospodinem, bude tobe i synum tvým s tebou právem vecným, jakož prikázal Hospodin.
\par 16 Mojžíš pak hledal pilne kozla k obeti za hrích, a hle, již spálen byl. Tedy rozhneval se na Eleazara a Itamara, syny Aronovy pozustalé, a rekl:
\par 17 Procež jste nejedli obeti za hrích na míste svatém? Nebo svatá svatých byla, ponevadž ji dal vám, abyste nesli nepravost všeho množství k ocištení jejich pred Hospodinem.
\par 18 A hle, ani krev její není vnesena do vnitrku svatyne. Jísti jste meli ji v svatyni, jakož jsem byl prikázal.
\par 19 Tedy odpovedel Aron Mojžíšovi: Aj, dnes obetovali obet svou za hrích a obet svou zápalnou pred Hospodinem, ale prihodila se mne taková vec, že, kdybych byl jedl obet za hrích dnes, zdali by se to líbilo Hospodinu?
\par 20 Což když uslyšel Mojžíš, prestal na tom.

\chapter{11}

\par 1 I mluvil Hospodin Mojžíšovi a Aronovi, rka jim:
\par 2 Mluvte k synum Izraelským, rkouce: Tito jsou živocichové, kteréž jísti budete ze všech hovad, kteráž jsou na zemi:
\par 3 Všeliké hovado, kteréž má kopyta rozdelená, tak aby rozdvojená byla, a prežívá, jísti budete.
\par 4 A však z tech, kteráž prežívají, a z tech, kteráž kopyta rozdelená mají, nebudete jísti, jako velblouda; nebo ac prežívá, ale kopyta rozdeleného nemá, necistý vám bude.
\par 5 Ani králíka, kterýž ac prežívá, ale kopyta rozdeleného nemá, necistý vám bude.
\par 6 Ani zajíce; nebo ac prežívá, ale kopyta rozdeleného nemá, necistý vám bude.
\par 7 Tolikéž ani svine; nebo ac má rozdelené kopyto, tak že se rozdvojuje, ale neprežívá, necistá bude vám.
\par 8 Masa jejich nebudete jísti, ani tela jejich mrtvého se dotýkati, necistá budou vám.
\par 9 Ze všech pak živocichu, kteríž u vodách jsou, tyto jísti budete: Všecko, což má plejtvy a šupiny u vodách morských i v rekách, jísti budete.
\par 10 Všecko pak, což nemá plejtví a šupin v mori i v rekách, bud jakýkoli hmyz vodný, aneb jakákoli duše živá u vodách, ohavností bude vám.
\par 11 A tak v ohavnosti vám budou, abyste masa jejich nejedli a tela jejich v ohyzdnosti meli.
\par 12 Což tedy nemá plejtví a šupin u vodách, to v ohavnosti míti budete.
\par 13 Z ptactva pak tyto v ohavnosti míti budete, jichžto nebudete jísti, nebo ohavnost jsou, jako jest orel, noh a orlice morská,
\par 14 Též sup, káne a lunák vedlé pokolení svého,
\par 15 A všeliký krkavec vedlé pokolení svého,
\par 16 Také pstros, sova, vodní káne, a jestráb vedlé pokolení svého,
\par 17 A bukac, krehar a kalous,
\par 18 A porfirián, pelikán a labut,
\par 19 Též cáp a kalandra vedlé pokolení svého, dedek a netopýr.
\par 20 Všeliký zemeplaz krídla mající, kterýž na ctyrech nohách chodí, v ohavnosti míti budete.
\par 21 A však ze všelikého zemeplazu krídla majícího, kterýž na ctyrech nohách chodí, jísti budete ty, kteríž mají stehénka na nohách svých, aby skákali na nich po zemi.
\par 22 Titot pak jsou, kteréž jísti budete: Arbes vedlé pokolení svého, sálem vedlé pokolení svého, chargol vedlé pokolení svého, a chagab vedlé pokolení svého.
\par 23 Jiný pak zemeplaz všeliký krídla mající, kterýž na ctyrech nohách chodí, v ohavnosti míti budete.
\par 24 Nebo temi byste se poškvrnovali. Protož kdož by se koli dotkl tela jich, necistý bude až do vecera.
\par 25 A kdož by koli nesl tela jich, zpéret roucho své a necistý bude až do vecera.
\par 26 Všeliké hovado, kteréž má rozdelené kopyto, ale není rozdvojené, a neprežívá, necisté vám bude. Kdož by koli jeho se dotekl, necistý bude.
\par 27 A cožkoli chodí na tlapách svých ze všech hovad ctvernohých, necisté bude vám. Kdož by koli dotkl se tel jejich, necistý bude až do vecera.
\par 28 A kdož by koli nesl tela jejich, zpére roucha svá a necistý bude až do vecera; nebo necistá jsou vám.
\par 29 Také i toto vám necisté bude mezi zemeplazy, kteríž lezou po zemi: Kolcava a myš a žába, každé vedlé pokolení svého;
\par 30 Též ježek a ješterka, hlemejžd, štír a krtice.
\par 31 Tyto veci necisté vám budou ze všelijakého zemeplazu. Kdož by se koli dotkl jich mrtvých, necistý bude až do vecera.
\par 32 A všeliká vec, na kterouž by padlo neco z tech již mrtvých tel, necistá bude, bud nádoba drevená, neb roucho, neb kuže, aneb pytel. A všeliká nádoba, kteréž se k dílu nejakému užívá, do vody vložena bude a necistá zustane až do vecera, potom cistá bude.
\par 33 Všeliká pak nádoba hlinená, do níž by neco toho upadlo, i což by koli v ní bylo, necisté bude, ona pak rozražena bude.
\par 34 Všeliký také pokrm, jehož se užívá, jestliže by na nej voda vylita byla, necistý bude; a všeliký nápoj ku pití príhodný v každé nádobe necisté bude necistý.
\par 35 A všecko, na cež by neco z tela jejich upadlo, necisté bude. Pec a kotlište zboreno bude, nebo necisté jest; protož za necisté je míti budete.
\par 36 Ale studnice a cisterna, i všeliké shromáždení vod, cistá budou; však což by se dotklo umrliny jejich, necisté bude.
\par 37 Jestliže by pak neco z mrchy jejich upadlo na nekteré semeno, kteréž síti obycej jest, cisté bude.
\par 38 Ale když by polito bylo vodou semeno, a potom padlo by neco z umrliny jejich na ne, necisté bude vám.
\par 39 Jestliže by umrelo hovado z tech, kteráž vám jsou ku pokrmu, kdož by se koli mrchy jeho dotkl, necistý bude až do vecera.
\par 40 Kdo by jedl z tela toho, zpéret roucha svá a necistý bude až do vecera. Také i ten, kterýž by vynesl tu mrchu, zpére roucha svá, a necistý bude až do vecera.
\par 41 Tolikéž všeliký zemeplaz, kterýž se plazí po zemi, v ohavnosti bude vám; nebudete ho jísti.
\par 42 Niceho, což se plazí na prsech, aneb cožkoli ctvermo leze, aneb více má noh, ze všeho zemeplazu, kterýž se plazí po zemi, nebudete jísti nebo jsou ohavnost.
\par 43 Nezohavujtež duší svých žádným zemeplazem, kterýž se plazí, a nepoškvrnujte se jimi, abyste necistí nebyli ucineni skrze ne.
\par 44 Nebo já jsem Hospodin Buh váš; protož posvettež se, a svatí budte, nebo já svatý jsem, a nepošvrnujte duší svých žádným zemeplazem, kterýž se plazí po zemi.
\par 45 Nebo já jsem Hospodin, kterýž jsem vás vyvedl z zeme Egyptské, abych vám byl za Boha; protož svatí budte, nebo já jsem svatý.
\par 46 Tot jest právo strany hovada, ptactva a všeliké duše živé, kteráž se hýbe u vodách, a každé duše živé, kteráž se plazí na zemi,
\par 47 Aby rozdíl cinen byl mezi necistým a cistým, a mezi živocichy, kteríž se mají jísti, a mezi živocichy, kteríž se nemají jísti.

\chapter{12}

\par 1 I mluvil Hospodin k Mojžíšovi, rka:
\par 2 Mluv synum Izraelským a rci: Žena pocnuc, porodí-li pacholíka, necistá bude za sedm dní; podlé poctu dnu, v nichž oddeluje se pro nemoc svou, necistá bude.
\par 3 Potom dne osmého obrezáno bude telo neobrízky jeho.
\par 4 Ona pak ješte za tridceti a tri dni zustávati bude v ocištování se od krve. Nižádné veci svaté se nedotkne a k svatyni nepujde, dokudž se nevyplní dnové ocištení jejího.
\par 5 Pakli devecku porodí, necistá bude za dve nedele vedlé necistoty oddelení svého, a šedesáte šest dní zustávati bude v ocištení od krve.
\par 6 Když pak vyplní se dnové ocištení jejího po synu aneb po dceri, prinese beránka rocního na obet zápalnou, a holoubátko aneb hrdlicku na obet za hrích, ke dverím stánku úmluvy, knezi.
\par 7 Kterýž obetovati ji bude pred Hospodinem, a ocistí ji, a tak ocištena bude od toku krve své. Ten jest zákon té, kteráž porodila pacholíka aneb devecku.
\par 8 Pakli nebude moci býti s beránka, tedy vezme dvé hrdlicátek, aneb dvé holoubátek, jedno v obet zápalnou a druhé v obet za hrích. I ocistí ji knez, a tak cistá bude.

\chapter{13}

\par 1 Mluvil také Hospodin Mojžíšovi a Aronovi, rka:
\par 2 Clovek ješto by na kuži tela jeho byla nejaká oteklina aneb prašivina, aneb poškvrna pobelavá, a bylo by na kuži tela jeho neco podobného k ráne malomocenství: tedy priveden bude k Aronovi knezi, aneb k nekterému z synu jeho kneží.
\par 3 I pohledí knez na ránu, kteráž jest na kuži tela jeho. Jestliže chlupové na té ráne zbelejí, a ta rána bude-li na pohledení hlubší nežli jiná kuže tela jeho, rána malomocenství jest. Když tedy spatrí ji knez, za necistého vyhlásí jej.
\par 4 Pakli by poškvrna pobelavá byla na kuži tela jeho, a rána ta nebyla by hlubší nežli jiná kuže, a chlupové její nezbeleli by: tedy rozkáže zavríti knez majícího takovou ránu za sedm dní.
\par 5 Potom pohledí na nej knez v den sedmý, a jestli rána ta zustává tak pred ocima jeho a nerozmáhá se po kuži: tedy rozkáže ho zavríti knez po druhé za sedm dní.
\par 6 I pohledí na ni knez v den sedmý po druhé, a jestliže ta rána pozcernalá bude, a nerozmohla by se po kuži: tedy za cistého vyhlásí ho knez, nebo prašivina jest. I zpére roucha svá a cistý bude.
\par 7 Paklit by se dále rozmohla ta prašivina po kuži jeho, již po ukázání se knezi k ocištení svému, tedy ukáže se znovu knezi.
\par 8 I pohledí na nej knez, a jestliže se rozmohla ta prašivina po kuži jeho, za necistého vyhlásí jej knez, nebo malomocenství jest.
\par 9 Rána malomocenství když bude na cloveku, priveden bude k knezi.
\par 10 I pohledí na nej knez, a bude-li oteklina bílá na kuži, až by ucinila chlupy bílé, byt pak i zdravé maso bylo na té otekline:
\par 11 Malomocenství zastaralé jest na kuži tela jeho. Protož za necistého vyhlásí jej knez, a nedá ho zavríti, nebo zjevne necistý jest.
\par 12 Jestliže by se pak vylilo malomocenství po kuži, a prikrylo by malomocenství všecku kuži nemocného, od hlavy jeho až do noh jeho, všudy kdež by knez ocima videti mohl:
\par 13 I pohledí na nej knez, a prikrylo-li by malomocenství všecko telo jeho, tedy za cistého vyhlásí nemocného. Nebo všecka ta rána v belost se zmenila, protož cistý jest.
\par 14 Ale kdykoli ukáže se na ní živé maso, necistý bude.
\par 15 I pohledí knez na to živé maso a vyhlásí jej za necistého, nebo to maso živé necisté jest, malomocenství jest.
\par 16 Když by pak odešlo zase maso živé a zmenilo by se v belost, tedy prijde k knezi.
\par 17 A vida knez, že obrátila se rána ta v belost, za cistou vyhlásí ji; cistá jest.
\par 18 Když by pak byl na kuži tela vred, a byl by zhojen,
\par 19 Byla-li by na míste vredu toho oteklina bílá, aneb poškvrna pobelavá a náryšavá, tedy ukázána bude knezi.
\par 20 A vida knez, že na pohledení to místo jest nižší nežli jiná kuže, a chlupové na nem by zbeleli, za necistého vyhlásí jej knez. Rána malomocenství jest, kteráž na vredu vyrostla.
\par 21 Pakli, když pohledí na ni knez, uzrí, že chlupové nezbeleli na nem, a není nižší nežli jiná kuže, ale že jest pozcernalé místo, tedy zavre ho knez za sedm dní.
\par 22 Paklit by šíre se rozmáhala po kuži, za necistého vyhlásí jej knez; rána malomocenství jest.
\par 23 Pakli by ta poškvrna pobelavá zustávala na míste svém, a nerozmáhala by se, znamení toho vredu jest; protož za cistého vyhlásí jej knez.
\par 24 Telo, na jehož kuži byla by spálenina od ohne, a po zhojení té spáleniny zustala by poškvrna pobelavá a náryšavá, anebo bílá toliko:
\par 25 Pohledí na ni knez, a jestliže chlupové na ní zbeleli, a lsknou se, zpusob také její jest hlubší nežli jiné kuže vukol: malomocenství jest, kteréž zrostlo na spálenine; protož za necistého vyhlásí jej knez, nebo rána malomocenství jest.
\par 26 Pakli uzrí knez, an na poškvrne pobelavé není žádného chlupu bílého, a že nižší není nežli jiná kuže, ale že jest pozcernalá, porucí ho zavríti knez za sedm dní.
\par 27 I pohledí na ni knez dne sedmého. Jestliže se více rozmohla po kuži, tedy za necistého vyhlásí jej; nebo rána malomocenství jest.
\par 28 Pakli belost lsknutá na svém míste zustávati bude, a nerozmuže se po kuži, ale bude pozcernalá: zprýštení od spáleniny jest; za cistého vyhlásí jej knez, nebo šrám spáleniny jest.
\par 29 Byla-li by pak rána na hlave aneb na brade muže neb ženy,
\par 30 Tedy pohledí knez na tu ránu, a jestliže zpusob její bude hlubší nežli jiná kuže, a bude na ní vlas prožlutlý a tenký: tedy za necistého vyhlásí jej knez, nebo poškvrna cerná jest; malomocenství na hlave aneb na brade jest.
\par 31 Když pak pohledí knez na ránu poškvrny cerné a uzrí, že zpusob její není hlubší nežli jiná kuže, a že není vlasu cerného na ní: zavríti dá knez majícího ránu poškvrny cerné za sedm dní.
\par 32 I pohledí knez na tu ránu dne sedmého, a aj, nerozmohla se ta poškvrna cerná, a není vlasu žlutého na ní, a zpusob té poškvrny cerné není hlubší nežli kuže:
\par 33 Tedy oholen bude clovek ten, ale poškvrny té cerné nedá holiti. I dá zavríti knez majícího tu poškvrnu za sedm dní po druhé.
\par 34 I pohledí knez na poškvrnu cernou dne sedmého, a jestliže se nerozmohla poškvrna cerná dále po kuži, a místo její není-li hlubší nežli jiná kuže: za cistého vyhlásí ho knez, i zpére roucha svá a cistý bude.
\par 35 Pakli by se dále rozmohla poškvrna ta cerná po kuži, již po ocištení jeho,
\par 36 Tedy pohledí na ni knez, a uzrí-li, že se dále rozmohla ta poškvrna cerná po kuži, nebude více šetriti vlasu žlutého; necistý jest.
\par 37 Pakli poškvrna cerná tak zustává pred ocima jeho, a cerný vlas vzrostl by na ní, tedy zhojena jest ta poškvrna cerná; cistý jest, a za cistého vyhlásí jej knez.
\par 38 Když by na kuži tela muže aneb ženy byly poškvrny, totiž poškvrny pobelavé,
\par 39 I pohledí knez, a jestliže budou na kuži tela jejich poškvrny bílé, pozcernalé, tedy poškvrna bílá jest, kteráž na kuži zrostla; cistý jest.
\par 40 Muž, z jehož by hlavy vlasové slezli, lysý jest a cistý jest.
\par 41 Jestliže pak po jedné strane obleze hlava jeho, nálysý jest a cistý jest.
\par 42 Pakli na té lysine, aneb na tom oblysení byla by rána bílá a ryšavá, tedy malomocenství zrostlo na lysine jeho, aneb na oblysení jeho.
\par 43 I pohledí na nej knez, a uzrí-li oteklinu rány bílou a ryšavou na lysine jeho, aneb na oblysení jeho, jako zpusob malomocenství na kuži tela:
\par 44 Clovek malomocný jest a necistý jest. Bez meškání za necistého vyhlásí jej knez, nebo na hlave jeho jest malomocenství jeho.
\par 45 Malomocný pak, na nemž by ta rána byla, bude míti roucho roztržené a hlavu odkrytou a ústa zastrená, a Necistý, necistý jsem! volati bude.
\par 46 Po všecky dny, v nichž ta rána bude na nem, za necistého jmín bude; nebo necistý jest. Sám bydliti bude, vne za stany bude prebývání jeho.
\par 47 Když by pak na rouchu byla rána malomocenství, bud na rouchu soukenném aneb na rouchu lneném,
\par 48 Bud na osnove aneb na outku ze lnu aneb z vlny, bud na kuži aneb na každé veci kožené,
\par 49 A byla by ta rána zelená neb náryšavá na rouchu aneb na kuži, aneb na osnove, aneb na outku, aneb na kterékoli nádobe kožené: rána malomocenství jest, ukázána bude knezi.
\par 50 I pohledí knez na tu ránu, a dá zavríti tu vec mající ránu za sedm dní.
\par 51 A pohlede na tu ránu dne sedmého, uzrí-li, že se dále rozmohla ta rána na rouchu, neb po osnove anebo po outku, aneb na kuži, k cemuž by jí koli užíváno bylo: rána ta malomocenství rozjídavá jest, vec necistá jest.
\par 52 I spálí to roucho aneb osnovu, aneb outek z vlny neb ze lnu, aneb jakoukoli nádobu koženou, na níž by byla rána ta; nebo malomocenství škodlivé jest, protož ohnem spáleno bude.
\par 53 Pakli pohlede knez, uzrel by, že se nerozmohla rána ta na rouchu, aneb na osnove, aneb na outku, aneb na kterékoli nádobe kožené:
\par 54 Rozkáže knez, aby zeprali tu vec, na níž byla by rána, i káže ji zavríti za sedm dní po druhé.
\par 55 I pohledí knez po zeprání té veci na ránu. Jestliže nezmenila rána barvy své, byt se pak nerozmohla dále, predce necistá jest. Ohnem spálíš ji; nebo rozjídavá vec jest na vrchní neb spodní strane její.
\par 56 Pakli pohlede knez, uzrel by, an pozcernala rána po zeprání svém, odtrhne ji od roucha aneb od kuže, aneb od osnovy, aneb od outku.
\par 57 Jestliže se pak ukáže ješte na rouchu aneb na osnove, aneb na outku, aneb na kterékoli nádobe kožené, malomocenství rozjídající se jest. Ohnem spálíš vec tu, na kteréž by ta rána byla.
\par 58 Roucho pak aneb osnovu, aneb outek, aneb kteroukoli nádobu koženou, když bys zepral, a odešla by od ní ta rána, ješte po druhé zpéreš, a cisté bude.
\par 59 Tent jest zákon o ráne malomocenství na rouchu soukenném aneb lneném, aneb na osnove, aneb na outku, aneb na kterékoli nádobe kožené, kterak má za cistou aneb za necistou uznána býti.

\chapter{14}

\par 1 I mluvil Hospodin k Mojžíšovi, rka:
\par 2 Tento bude rád pri malomocném v den ocištování jeho: K knezi priveden bude.
\par 3 I vyjde knez ven z stanu a pohledí na nej, a uzrí-li, že uzdravena jest rána malomocenství malomocného:
\par 4 Rozkáže knez tomu, kterýž se ocištuje, vzíti dva vrabce živé a cisté, a drevo cedrové, a cervec dvakrát barvený, a yzop.
\par 5 I rozkáže knez zabiti vrabce jednoho, a vycediti krev z neho do nádoby hlinené nad vodou živou.
\par 6 A vezme vrabce živého a drevo cedrové, též cervec dvakrát barvený, a yzop, a omocí to všecko i s vrabcem živým ve krvi vrabce zabitého nad vodou živou.
\par 7 Tedy pokropí toho, kterýž se ocištuje od malomocenství, sedmkrát, a za cistého jej vyhlásí, i pustí vrabce živého na pole.
\par 8 Ten pak, kterýž se ocištuje, zpére roucho své a oholí všecky vlasy své, umyje se vodou, a cistý bude. Potom vejde do táboru, a bydliti bude vne, nevcházeje do stanu svého za sedm dní.
\par 9 Dne pak sedmého sholí všecky vlasy své, hlavu i bradu svou, i obocí své, a tak všecky vlasy své sholí; zpére také roucha svá a telo své umyje vodou, a tak ocistí se.
\par 10 Dne pak osmého vezme dva beránky bez poškvrny, a ovci jednu rocní bez poškvrny, a tri desetiny efi mouky belné k obeti suché, olejem zadelané, a oleje jednu mírku.
\par 11 Knez pak, kterýž ocištuje, postaví toho cloveka ocištujícího se s temi vecmi pred Hospodinem u dverí stánku úmluvy.
\par 12 I vezme knez beránka jednoho, kteréhožto obetovati bude v obet za hrích, a mírku oleje, a obraceti bude tím sem i tam v obet obracení pred Hospodinem.
\par 13 A zabije beránka toho na míste, kdež se zabijí obet za hrích a obet zápalná, totiž na míste svatém; nebo jakož obet za hrích, tak obet za vinu knezi prináleží, svatá svatých jest.
\par 14 I vezme knez krve z obeti za hrích, a pomaže jí kraje ucha pravého cloveka toho, kterýž se ocištuje, a palce ruky jeho pravé, a palce nohy jeho pravé.
\par 15 Vezme také knez z té mírky oleje, a naleje na ruku svou levou.
\par 16 A omoce prst svuj pravý v tom oleji, kterýž má na ruce své levé, pokropí z neho prstem svým sedmkrát pred Hospodinem.
\par 17 Z ostatku pak oleje toho, kterýž má na ruce své, pomaže knez kraje ucha pravého cloveka toho, kterýž se ocištuje, a palce pravé ruky jeho, a palce pravé nohy jeho, na krev obeti za vinu.
\par 18 Což pak zustane oleje, kterýž jest v ruce kneze, pomaže tím hlavy toho, kterýž se ocištuje; a tak ocistí jej knez pred Hospodinem.
\par 19 Uciní také knez obet za hrích, a ocistí ocištujícího se od necistoty jeho. A potom zabije obet zápalnou.
\par 20 I bude obetovati knez tu obet zápalnou, i obet suchou na oltári; a tak ocistí jej, i bude cistý.
\par 21 Jestliže pak bude chudý, tak že by s to býti nemohl, tedy vezme jednoho beránka v obet za provinení, k obracení jí sem i tam pro ocištení své, a desátý díl mouky belné olejem zadelané k obeti suché, a mírku oleje,
\par 22 Též dve hrdlicky aneb dvé holoubátek, kteréž by mohl míti, a bude jedno v obet za hrích, a druhé v obet zápalnou.
\par 23 I prinese je v osmý den ocištování svého knezi, ke dverum stánku úmluvy pred Hospodina.
\par 24 Tedy knez vezma beránka obeti za vinu a mírku oleje, obraceti je bude sem i tam v obet obracení pred Hospodinem.
\par 25 I zabije beránka obeti za provinení, a vezma krve z obeti za provinení, pomaže jí kraje ucha pravého cloveka toho, kterýž se ocištuje, a palce ruky jeho pravé, a palce nohy jeho pravé.
\par 26 Oleje také naleje knez na ruku svou levou.
\par 27 A omoce prst svuj pravý v oleji, kterýž bude na ruce jeho levé, pokropí jím sedmkrát pred Hospodinem.
\par 28 Pomaže také knez olejem, kterýž má na ruce své, kraje ucha pravého toho, kdož se cistí, a palce pravé ruky jeho, a palce pravé nohy jeho, na míste krve z obeti za vinu.
\par 29 Což pak zustává oleje, kterýž jest v ruce kneze, pomaže jím hlavy toho, kterýž se ocištuje; a tak ocistí jej pred Hospodinem.
\par 30 Tolikéž uciní i s hrdlickou jednou aneb s holoubátkem z tech, kteréž zjednati mohl.
\par 31 Z tech, kterýchž dostati mohl, obetovati bude jedno za hrích a druhé v obet zápalnou s obetí suchou; a tak ocistí knez ocištujícího se pred Hospodinem.
\par 32 Ten jest zákon toho, na komž by se ukázala rána malomocenství, a kterýž by nemohl býti s ty veci k ocištení svému prináležité.
\par 33 I mluvil Hospodin Mojžíšovi a Aronovi, rka:
\par 34 Když vejdete do zeme Kananejské, kterouž já vám za dedictví dávám, a dopustil bych ránu malomocenství na nekterý dum zeme, kterouž vládnouti budete,
\par 35 Tedy prijde hospodár domu a oznámí knezi rka: Zdá mi se, jako by byla rána malomocenství na dome.
\par 36 I rozkáže knez vyprázdniti dum, dríve než by všel do neho hledeti na tu ránu, aby nebylo poškvrneno neco z tech vecí, kteréž v dome jsou. Potom pak vejde, aby pohledel na ten dum.
\par 37 Tedy vida ránu tu, uzrí-li, že rána jest na stenách domu, totiž dulkové názelení aneb nácervení, a na pohledení jsou nižší než stena jinde:
\par 38 Vyjde knez z domu toho ke dverím jeho, a dá zavríti dum ten za sedm dní.
\par 39 A v den sedmý navrátí se knez, a uzrí-li, ano se rozmohla rána na stenách domu toho:
\par 40 Rozkáže vyníti kamení, na nemž by rána taková byla, a vyvrci je ven za mesto na místo necisté.
\par 41 Dum pak rozkáže vystrouhati vnitr všudy vukol; a vysypou prach ten, kterýž sstrouhali, vne za mestem na místo necisté.
\par 42 A vezmouce jiné kamení, vyplní jím místo onoho kamení; tolikéž hliny jiné vezmouce, vymaží dum.
\par 43 Paklit by se navrátila rána, a vzrostla by v tom dome po vyvržení kamení a vystrouhání domu, i po vymazání jeho,
\par 44 Tedy vejda knez, uzrí-li, an se rozmohla rána v dome, malomocenství rozjídající se jest v tom dome, necistý jest.
\par 45 I rozborí ten dum a kamení jeho, i dríví jeho, a všecko mazání domu toho, a vynosí ven za mesto na místo necisté.
\par 46 Jestliže by pak kdo všel do domu toho v ten cas, když zavrín byl, necistý bude až do vecera.
\par 47 A jestliže by kdo spal v tom dome, zpére roucha svá; tolikéž jestliže by kdo jedl v tom dome, zpére roucha svá.
\par 48 Jestliže by pak knez vejda tam, uzrel, že se nerozmohla rána v dome po obnovení jeho, tedy za cistý vyhlásí dum ten; nebo uzdravena jest rána jeho.
\par 49 A vezma k ocištení domu toho dva vrabce a drevo cedrové, a cervec dvakrát barvený a yzop,
\par 50 I zabije vrabce jednoho, a vycedí krev do nádoby hlinené nad vodou živou.
\par 51 A vezme drevo cedrové a yzop, a cervec dvakrát barvený, a vrabce živého, omocí to všecko ve krvi vrabce zabitého a u vode živé, a pokropí domu toho sedmkrát.
\par 52 A tak když ocistí dum ten krví vrabce a vodou živou a ptákem živým, drevem cedrovým, yzopem a cervcem dvakrát barveným:
\par 53 Vypustí ven vrabce živého z mesta na pole, a ocistí dum ten, i budet cistý.
\par 54 Ten jest zákon o všeliké ráne malomocenství a poškvrny cerné,
\par 55 A malomocenství roucha i domu,
\par 56 I otekliny, prašiviny a poškvrny pobelavé,
\par 57 K ukázání, kdy jest kdo cistý, aneb kdy jest kdo necistý. Tent jest zákon malomocenství.

\chapter{15}

\par 1 Mluvil pak Hospodin Mojžíšovi a Aronovi, rka:
\par 2 Mluvte synum Izraelským a rcete jim: Když by který muž trpel tok semene z tela svého, necistý bude.
\par 3 Tato pak bude necistota jeho v toku jeho: Jestliže vypenuje telo jeho tok svuj, aneb že by se zastavil tok v tele jeho, necistota jeho jest.
\par 4 Každé luže, na nemž by ležel, kdož má tok semene, necisté bude; a všecko, na cemž by sedel, necisté bude.
\par 5 A kdož by se dotekl luže jeho, zpére roucha svá, a umyje se vodou, a bude necistý až do vecera.
\par 6 A kdo by sedl na to, na cemž sedel ten, kdož má tok semene, zpére roucha svá, a umyje se vodou, i bude necistý až do vecera.
\par 7 Jestliže by se kdo dotekl tela trpícího tok semene, zpére roucha svá, a umyje se vodou, i bude necistý až do vecera.
\par 8 A jestliže by ten, kdož trpí tok semene, plinul na cistého, zpére roucha svá, a umyje se vodou, i bude necistý až do vecera.
\par 9 Každé sedlo, na nemž by sedel ten, kdož má tok semene, necisté bude.
\par 10 A kdož by koli dotekl se neceho, což bylo pod ním, necistý bude až do vecera; a kdož by co z toho nesl, zpére roucha svá, a umyje se vodou, a bude necistý až do vecera.
\par 11 Kohož by se pak dotekl ten, kdož má tok semene, a neumyl rukou svých vodou, zpére roucha svá, a umyje se vodou, i bude necistý až do vecera.
\par 12 Nádoba hlinená, kteréž by se dotekl ten, kdo má tok semene, bude rozbita, a každá nádoba drevená vodou vymyta bude.
\par 13 Když by pak ocišten byl ten, kdož tok semene trpel, od toku svého, secte sedm dní po svém ocištení, a zpére roucha svá, a vodou živou zmyje telo své, i bude cist.
\par 14 Potom dne osmého vezme sobe dve hrdlicky aneb dvé holoubátek, a prijda pred Hospodina ke dverím stánku úmluvy, dá je knezi.
\par 15 Kterýž obetovati je bude, jedno za hrích a druhé v obet zápalnou; a ocistí jej knez pred Hospodinem od toku jeho.
\par 16 Muž, z nehož by vyšlo síme scházení, zmyje vodou všecko telo své, a bude necistý až do vecera.
\par 17 Každé roucho i každá kuže, na níž by bylo síme scházení, zeprána bude vodou, a necistá bude až do vecera.
\par 18 Žena také, s kterouž by obýval muž scházením semene, oba zmyjí se vodou, a necistí budou až do vecera.
\par 19 Žena pak, když by trpela nemoc svou, a tok krve byl by z tela jejího, za sedm dní oddelena bude; každý, kdož by se jí dotekl, necistý bude až do vecera.
\par 20 Na cemž by koli ležela v cas oddelení svého, necistét bude; tolikéž, na cem by koli sedela, necisté bude.
\par 21 Také kdož by se dotekl luže jejího, zpére roucho své, a umyje se vodou, i bude necistý až do vecera.
\par 22 Kdož by se koli dotekl toho, na cemž sedela, zpére roucha svá, a umyje se vodou, i bude necistý až do vecera.
\par 23 Což by koli bylo na luži tom, neb na jaké by koli veci sedela, a dotekl by se toho nekdo, necistý bude až do vecera.
\par 24 Jestliže by kdo spal s ní, a byla by necistota její na nem, necistý bude za sedm dní; i každé luže, na nemž by spal, necisté bude.
\par 25 Žena pak trpela-li by krvotok po mnohé dny, krome casu nemoci své, totiž krvotok by trpela pres cas prirozené nemoci: po všecky dny toku necistoty své, jako i v cas nemoci své prirozené, necistá bude.
\par 26 Každé luže, na nemž by spala po všecky dny toku svého, bude jí jako luže v prirozené nemoci její; a každá vec, na kteréž by sedela, necistá bude, podlé necistoty prirozené nemoci její.
\par 27 Kdož by koli dotekl se tech vecí, necistý bude. Protož zpére roucho své, a umyje se vodou, i bude necistý až do vecera.
\par 28 Když pak ocištena bude od toku svého, secte sobe sedm dní, a potom ocištovati se bude.
\par 29 A v den osmý vezme sobe dve hrdlicky aneb dvé holoubátek, a prinese je knezi ke dverím stánku úmluvy.
\par 30 Z nichž jedno obetovati bude knez v obet za hrích, a druhé v obet zápalnou, a ocistí ji knez pred Hospodinem od toku necistoty její.
\par 31 I budete oddelovati syny Izraelské od necistot jejich, aby nezemreli pro necistoty své, když by poškvrnili príbytku mého, kterýž jest u prostred nich.
\par 32 Tot jest právo trpícího tok semene, i toho, z nehož vychází síme scházení, jímž poškvrnen bývá,
\par 33 Též nemocné ženy v oddelení jejím, i všelikého trpícího tok svuj, bud mužského pohlaví neb ženského, a muže, kterýž by spal s necistou.

\chapter{16}

\par 1 Mluvil pak Hospodin k Mojžíšovi po smrti dvou synu Aronových, kterížto, když predstoupili pred Hospodina, zemreli,
\par 2 A rekl Hospodin Mojžíšovi: Mluv Aronovi bratru svému, at nevchází každého casu do svatyne za oponu pred slitovnici, kteráž jest na truhle, aby neumrel; nebo já v oblace ukáži se nad slitovnicí.
\par 3 S tímto vcházeti bude Aron do svatyne: S volkem mladým, kterýž bude v obet za hrích, a s beranem k obeti zápalné.
\par 4 V sukni lnenou svatou oblece se, a košilku lnenou bude míti na tele hanby své, pasem také lneným opáše se, a cepici lnenou vstaví na hlavu, roucha zajisté svatá jsou tato; i umyje vodou telo své a oblece se v ne.
\par 5 Od shromáždení pak synu Izraelských vezme dva kozly k obeti za hrích, a jednoho berana k zápalné obeti.
\par 6 I bude obetovati Aron volka svého v obet za hrích, a ocistí sebe i dum svuj.
\par 7 Potom vezme ty dva kozly a postaví je pred Hospodinem u dverí stánku úmluvy.
\par 8 I dá Aron na ty dva kozly losy, los jeden Hospodinu, a los druhý Azazel.
\par 9 A obetovati bude Aron kozla toho, na nehož by los padl Hospodinu, obetovati jej bude za hrích.
\par 10 Kozla pak toho, na nehož prišel los Azazel, postaví živého pred Hospodinem, aby skrze neho ucinil ocištení, a pustí ho na poušt k Azazel.
\par 11 I bude obetovati Aron volka svého v obet za hrích, a ocistí sebe i dum svuj, a zabije volka svého v obet za hrích.
\par 12 Vezme také plnou kadidlnici uhlí reravého s oltáre, kterýž jest pred tvárí Hospodinovou, a plné obe hrsti své vonných vecí stlucených, a vnese za oponu.
\par 13 A vloží kadidlo to na ohen pred Hospodinem, a dým kadení toho prikryje slitovnici, kteráž jest nad svedectvím, a neumre.
\par 14 Potom vezma krve volka toho, pokropí prstem svým na slitovnicí k východu; tolikéž pred slitovnicí kropiti bude sedmkrát krví tou prstem svým.
\par 15 Zabije také v obet za hrích kozla toho, kterýž jest lidu, a vnese krev jeho do vnitrku za oponu, a uciní se krví jeho, jakož ucinil se krví volka, totiž pokropí jí na slitovnici a pred slitovnicí.
\par 16 A ocistí svatyni od necistot synu Izraelských a od prestoupení jejich i všech hríchu jejich. Totéž uciní i stánku úmluvy, kterýž jest mezi nimi u prostred necistot jejich.
\par 17 (Žádný pak clovek at není v stánku úmluvy, když on vchází k ocištování do svatyne, dokudž by on zase nevyšel a ocištení za sebe, za dum svuj i za všecko množství Izraelské nevykonal.)
\par 18 Vyjde pak k oltári, kterýž jest pred Hospodinem, a ocistí jej. A vezma krve volka toho, a ze krve kozla, dá na rohy oltáre vukol.
\par 19 A pokropí ho svrchu krví tou prstem svým sedmkrát, a ocistí jej i posvetí ho od necistot synu Izraelských.
\par 20 A když by dokonal ocištení svatyne a stánku úmluvy a oltáre, obetovati bude kozla živého.
\par 21 A vlože Aron obe ruce své na hlavu kozla živého, vyznávati bude nad ním všecky nepravosti synu Izraelských, a všecka prestoupení jejich se všemi hríchy jejich, a vloží je na hlavu kozla, a vyžene ho clovek k tomu zrízený na poušt.
\par 22 (Kozel ten zajisté ponese na sobe všecky nepravosti jejich do zeme pusté.) A pustí kozla toho na poušti.
\par 23 Potom pak prijda Aron do stánku úmluvy, svlece s sebe roucha lnená, v než se byl oblékl, když vjíti mel do svatyne, a nechá jich tu.
\par 24 A umyje telo své vodou na míste svatém, a oblece se zase v roucha svá, a vyjda, obetovati bude obet zápalnou svou a obet zápalnou lidu, a ocistí sebe i lid.
\par 25 A tuk obeti za hrích páliti bude na oltári.
\par 26 Ten pak, kterýž vyvedl kozla na Azazel, zpére roucha svá, a zmyje telo své vodou, a potom vejde do stanu.
\par 27 Volka pak za hrích a kozla za hrích, jejichž krev vnesena byla k vykonání ocištení v svatyni, vynese ven z táboru, a spálí ohnem kuže jejich, i maso jejich, i lejna jejich.
\par 28 Kdož by pak spálil je, zpére roucha svá, a umyje telo své vodou, a potom vejde do táboru.
\par 29 Bude vám i toto za vecné ustanovení: Sedmého mesíce, desátého dne téhož mesíce ponižovati budete duší svých, a žádného díla nebudete delati, ani doma zrozený, ani príchozí, kterýž jest pohostinu mezi vámi.
\par 30 Nebo v ten den ocistí vás, abyste ocišteni byli; ode všech hríchu svých pred Hospodinem ocišteni budete.
\par 31 Sobota odpocinutí bude vám, a ponižovati budete duší svých ustanovením vecným.
\par 32 Ocištovati pak bude knez, kterýž jest pomazaný, a jehož ruce posveceny jsou k vykonávání úradu knežského místo otce svého, a oblece se v roucha lnená, roucha svatá.
\par 33 A ocistí svatyni svatou a stánek úmluvy, ocistí také i oltár, i kneží, i všecken lid shromáždený ocistí.
\par 34 A bude vám to ustanovením vecným k ocištování synu Izraelských ode všech hríchu jejich, každého roku jednou. I ucinil Mojžíš tak, jakž jemu byl prikázal Hospodin.

\chapter{17}

\par 1 I mluvil Hospodin Mojžíšovi, rka:
\par 2 Mluv k Aronovi a synum jeho i ke všechnem synum Izraelským, a rci jim: Tato jest vec, kterouž prikázal Hospodin, rka:
\par 3 Kdož by koli z domu Izraelského zabil vola aneb beránka, neb kozu, bud mezi stany, aneb kdož by zabil vne za stany,
\par 4 A ke dverím stánku úmluvy neprivedl by ho, aby obetoval obet Hospodinu pred príbytkem Hospodinovým: vinen bude krví, nebo krev vylil; protož vyhlazen bude muž ten z prostredku lidu svého.
\par 5 Privedou tedy synové Izraelští obeti své, kteréž by na poli zabíjeti chteli, privedou je, pravím, k Hospodinu ke dverím stánku úmluvy, k knezi, a obetovati budou obeti pokojné Hospodinu.
\par 6 A pokropí knez krví na oltári Hospodinovu u dverí stánku úmluvy, a páliti bude tuk u vuni líbeznou Hospodinu.
\par 7 A nikoli více nebudou obetovati obetí svých dáblum, po nichžto odcházejíce, oni smilní. Zákon tento bude jim vecný i všechnem potomkum jejich.
\par 8 Protož povíš jim: Kdož by koli z domu Izraelského aneb z príchozích, kteríž by pohostinu byli mezi vámi, obetoval zápal aneb jinou obet,
\par 9 A ke dverím stánku úmluvy neprivedl by jí, aby ji obetoval Hospodinu; tedy vyhlazen bude clovek ten z lidu svého.
\par 10 A kdož by koli z domu Izraelského aneb z príchozích, kteríž jsou pohostinu mezi nimi, jedl jakou krev: postavím tvár svou proti cloveku tomu, a vyhladím jej z prostredku lidu jeho.
\par 11 Nebo duše všelikého tela ve krvi jest, já pak oddal jsem vám ji k oltári, k ocištování duší vašich. Nebo sama krev na duši ocištuje.
\par 12 Protož rekl jsem synum Izraelským: Nižádný z vás nebude jísti krve; ani príchozí, kterýž pohostinu jest mezi vámi, nebude krve jísti.
\par 13 A kdož by koli z synu Izraelských aneb z príchozích, kteríž jsou pohostinu mezi vámi, hone, ulovil zvíre aneb ptáka, což se jísti muže, tedy vycedí krev jeho a zasype ji prstí.
\par 14 Nebo duše všelikého tela jest krev jeho, kteráž jest v duši jeho. Protož jsem povedel synum Izraelským: Krve žádného tela jísti nebudete, nebo duše všelikého tela jest krev jeho; kdož by koli jedl ji, vyhlazen bude.
\par 15 Kdož by pak koli jedl telo mrtvé aneb udávené, bud on doma zrozený aneb príchozí: zpére roucha svá, a umyje se vodou, a necistý bude až do vecera, potom pak cistý bude.
\par 16 A pakli nezpére roucha svého, a tela svého neumyje, tedy ponese nepravost svou.

\chapter{18}

\par 1 I mluvil Hospodin k Mojžíšovi, rka:
\par 2 Mluv synum Izraelským a rci jim: Já jsem Hospodin Buh váš.
\par 3 Vedlé skutku zeme Egyptské, v níž jste bydlili, necinte, ani podlé skutku zeme Kananejské, do kteréž já vás uvozuji, ciniti budete, a v ustanoveních jejich nechodte.
\par 4 Soudy mé cinte a ustanovení mých ostríhejte, abyste chodili v nich: Já jsem Hospodin Buh váš.
\par 5 Ostríhejte ustanovení mých a soudu mých. Clovek ten, kterýž by je cinil, živ bude v nich: Já jsem Hospodin.
\par 6 Nižádný clovek k žádné prítelkyni krevní nepristupuj k obnažení hanby její: Já jsem Hospodin.
\par 7 Hanby otce svého a matky své neodkryješ; matka tvá jest, neodkryješ hanby její.
\par 8 Hanby ženy otce svého neodkryješ; nebo hanba otce tvého jest.
\par 9 Hanby sestry své, dcery otce svého aneb dcery matky své, kteráž doma zplozena aneb vne zplozena jest, neodkryješ hanby jejich.
\par 10 Hanby vnucky své, bud po synu neb dceri své, neodkryješ; nebo hanba tvá jsou.
\par 11 Hanby dcery manželky otce svého, kteráž jest zplozena od otce tvého, tvá sestra jest, neodkryješ hanby její.
\par 12 Hanby sestry otce svého neodkryješ; nebo krevní prítelkyne otce tvého jest.
\par 13 Hanby sestry matky své neodkryješ; nebo krevní prítelkyne matky tvé jest.
\par 14 Hanby bratra otce svého neodkryješ; k manželce jeho nevejdeš, stryna tvá jest.
\par 15 Hanby nevesty své neodkryješ; manželka jest syna tvého, neodkryješ hanby její.
\par 16 Hanby manželky bratra svého neodkryješ; nebo hanba bratra tvého jest.
\par 17 Hanby ženy a dcery její neodkryješ. Vnucky její po synu neb po dceri její nepojmeš, abys odkryl hanbu její; nebo krevní jsou, a nešlechetnost jest.
\par 18 Nevezmeš sobe ženy k žene první, abys ssoužil ji, odkrývaje hanbu její za života jejího.
\par 19 Také k žene, když jest v své nemoci necisté, nepristoupíš, odkrývaje hanbu její.
\par 20 S manželkou bližního svého nebudeš obcovati, poškvrnuje se s ní.
\par 21 Nedopustíš, aby kdo z semene tvého proveden byl skrze ohen modly Moloch, abys nepoškvrnil jména Boha svého: Já jsem Hospodin.
\par 22 Nebudeš obcovati s mužským pohlavím, scházeje se s ním jako s ženou; nebo ohavnost jest.
\par 23 A s žádným hovadem nebudeš obcovati, poškvrnuje se s ním; ani žena nepoddá se hovadu, aby s ním obývala; nebo mrzkost jest.
\par 24 A protož nepoškvrnujtež se žádnou touto vecí; nebo temito všemi vecmi poškvrnili se pohané, kteréž já vyvrhu od tvári vaší.
\par 25 Nebo poškvrnila se zeme, a navštívím nepravost její na ní, a vyvrátí zeme obyvatele své.
\par 26 Ale vy ostríhejte ustanovení mých a soudu mých, a necinte nižádných ohavností techto, tak domácí jako príchozí, kterýž jest pohostinu u prostred vás.
\par 27 (Nebo všecky ty ohavnosti cinili lidé zeme té, kteríž byli pred vámi, címž poškvrnena jest zeme.)
\par 28 Aby nevyvrátila vás zeme, proto že byste jí poškvrnili, jako vyvrátila národ, kterýž byl pred vámi.
\par 29 Nebo kdož by koli dopustil se nekteré ze všech ohavností tech: duše zajisté, kteréž by cinily to, vyhlazeny budou z prostredku lidu svého.
\par 30 Protož ostríhejte prikázání mých, abyste necinili niceho z obyceju ohavných,kteríž cineni jsou pred vámi, aniž sebe jimi poškvrnujte: Já jsem Hospodin Buh váš.

\chapter{19}

\par 1 I mluvil Hospodin k Mojžíšovi, rka:
\par 2 Mluv ke všemu množství synu Izraelských a rci jim: Svatí budte, nebo svatý jsem, já Hospodin Buh váš.
\par 3 Jeden každý matky své a otce svého báti se budete. A sobot mých ostríhejte, nebo já jsem Hospodin Buh váš.
\par 4 Neobracejte se k modlám, a bohu litých nedelejte sobe: Já jsem Hospodin Buh váš.
\par 5 A když obetovati budete obeti pokojné Hospodinu, z dobré vule své obetovati budete je.
\par 6 Kterého dne obetovati budete, jísti budete je, i nazejtrí; což by pak zustalo až do tretího dne, ohnem spáleno bude.
\par 7 Pakli predce jísti budete dne tretího, vec ohavná bude, a nebude oblíbena.
\par 8 Kdož by to jedl, pokutu nepravosti své ponese; nebo svatosti Hospodinovy poškvrnil, protož vyhlazena bude duše ta z lidu svého.
\par 9 Když budete žíti obilé zeme vaší, nesežneš naskrze všeho pole svého, a nebudeš sbírati pozustalých klasu žne své.
\par 10 Tolikéž ostatku vinice své nebudeš bráti, a zrní vinice své nebudeš sbírati; chudému a príchozímu zanecháš toho: Já jsem Hospodin Buh váš.
\par 11 Nekradte a nelžete, a nižádný neoklamávej bližního svého.
\par 12 Neprisahejte krive ve jméno mé, aniž kdo poškvrnuj jména Boha svého: Já jsem Hospodin.
\par 13 Neutiskneš mocí bližního svého, aniž obloupíš ho. Nezustane mzda delníka u tebe až do jitra.
\par 14 Hluchému nebudeš zloreciti, a pred slepým nepoložíš úrazu, ale báti se budeš Boha svého, nebo já jsem Hospodin.
\par 15 Neuciníš nepráve v soudu. Neprijmeš osoby chudého, aniž šanovati budeš osoby bohatého; spravedlive souditi budeš bližního svého.
\par 16 Nebudeš choditi jako utrhac v lidu svém, aniž státi budeš na hrdlo bližnímu svému: Já jsem Hospodin.
\par 17 Nebudeš nenávideti bratra svého v srdci svém; svobodne potresceš bližního svého, a nesneseš na nem hríchu.
\par 18 Nebudeš se mstíti, aniž držeti budeš hnevu proti synum lidu svého, ale milovati budeš bližního svého jako sebe samého: Já jsem Hospodin.
\par 19 Ustanovení mých ostríhejte. Hovadu svému nedáš se scházeti s hovadem jiného pokolení. Pole svého neposeješ rozdílným semenem, a rouchem z rozdílných vecí, jako z vlny a ze lnu, setkaným nebudeš se odívati.
\par 20 Jestliže by muž spal s ženou, a obcoval s ní, kteráž devkou jsuc, byla by zasnoubená jinému muži, a nebyla by žádnou mzdou vykoupena, ani propuštena: oba dva zmrskáni budou, a neumrout; nebo nebyla svobodná ucinena.
\par 21 On pak privede obet za vinu svou Hospodinu ke dverím stánku úmluvy, skopce za vinu.
\par 22 I ocistí jej knez skopcem, kterýž se obetuje za vinu pred Hospodinem, od hríchu jeho, kterýmž zhrešil, a bude mu odpušten hrích jeho.
\par 23 Když pak vejdete do zeme, a nasázíte všelikého stromoví ovoce nesoucího, tedy obrežete neobrezání jeho, totiž ovoce jeho. Po tri léta budete míti je za neobrezané, protož nebude jedeno.
\par 24 Ctvrtého pak léta bude všeliké ovoce jeho posvecené k chválení Hospodina,
\par 25 (Pátého teprv léta jísti budete ovoce jeho), aby rozmnožil vám úrody jeho; nebo já jsem Hospodin Buh váš.
\par 26 Nic nebudete jísti se krví. Nebudete hádati, aniž na casích zakládati budete.
\par 27 Nebudeš strihati vlasu hlavy své okrouhle, aniž zohaví kdo brady své.
\par 28 Nad mrtvým pak nebudete rezati tela svého, a žádného znamení vyrytého na sobe neuciníte: Já jsem Hospodin.
\par 29 Nepoškvrnuj dcery své, dopoušteje smilniti jí, aby zeme nesmilnila a nebyla naplnena nešlechetností.
\par 30 Sobot mých ostríhati budete, a svatyne mé báti se budete: Já jsem Hospodin.
\par 31 Neobracejte se k hadacum a veštcum, ani od nich rady berte, poškvrnujíce se s nimi: Já jsem Hospodin Buh váš.
\par 32 Pred clovekem šedivým povstan, a cti osobu starého, a boj se Boha svého, nebo já jsem Hospodin.
\par 33 Jestliže bude pohostinu u tebe príchozí v zemi vaší, necinte jemu krivdy.
\par 34 Jakožto jeden z doma zrozených vašich, tak bude vám príchozí, kterýž jest u vás pohostinu, a milovati ho budeš jako sebe samého; nebo i vy pohostinu jste byli v zemi Egyptské: Já jsem Hospodin Buh váš.
\par 35 Necinte nepráve v soudu, v rozmerování, v váze a v míre.
\par 36 Závaží spravedlivé, kámen spravedlivý, korec spravedlivý, pintu spravedlivou míti budete: Já jsem Hospodin Buh váš, kterýž jsem vás vyvedl z zeme Egyptské.
\par 37 Protož ostríhejte všech ustanovení mých a všech soudu mých, a cinte je, nebo já jsem Hospodin.

\chapter{20}

\par 1 Mluvil pak Hospodin k Mojžíšovi, rka:
\par 2 Synum také Izraelským díš: Kdo by koli z synu Izraelských a z príchozích, kteríž by byli pohostinu v Izraeli, dal z semene svého modle Moloch, smrtí umre. Lid zeme té kamením jej uhází.
\par 3 Nebo já postavím tvár svou proti takovému, a vyhladím ho z prostredku lidu jeho, proto že z semene svého dal modle Moloch, a tak zprznil svatyni mou, a poškvrnil jména svatosti mé.
\par 4 Jestliže by pak lid zeme té všelijak prehlídal to na cloveku tom, kterýž by dal z semene svého modle Moloch, a nezahubil by ho:
\par 5 Tedy já postavím tvár svou proti muži tomu a proti celedi jeho, a vyhladím ho i všecky, kteríž smilníce, odcházeli po nem, aby smilnili, následujíce Molocha, z prostredku lidu jeho.
\par 6 Duše, kteráž by se obrátila k hadacum a veštcum, aby smilnila, postupujíc po nich: postavím tvár svou proti duši té, a vyhladím ji z prostredku lidu jejího.
\par 7 A protož posvette se a budte svatí, nebo já jsem Hospodin Buh váš.
\par 8 A ostríhejte ustanovení mých, a cinte je: Já jsem Hospodin posvetitel váš.
\par 9 Kdož by koli zlorecil otci svému neb matce své, smrtí umre. Otci svému a matce své zlorecil, krev jeho bude na nem.
\par 10 Muž, pak, kterýž by se cizoložství dopustil s ženou necí, že zcizoložil s ženou bližního svého, smrtí umre cizoložník ten i cizoložnice.
\par 11 A kdož by koli obcoval s ženou otce svého, hanbu otce svého odkryl. Smrtí umrou oba dva, krev jejich bude na ne.
\par 12 Kdož by pak obcoval s nevestou svou, smrtí umrou oba dva. Mrzkosti se dopustili, krev jejich bude na ne.
\par 13 A kdož by se scházel s pohlavím mužským jako s ženou, ohavnost ucinili oba dva. Smrtí umrou, krev jejich bude na ne.
\par 14 A kdož by vzal ženu a matku její nešlechetnost jest. Ohnem spálí i jej i je, aby nebylo nešlechetností u prostred vás.
\par 15 Kdož by pak obcoval s hovadem, smrtí umre, a hovado zabijete.
\par 16 Tolikéž žena, kteráž by pristoupila k nekterému hovadu, aby obcovala s ním, zabiješ ji i to hovado. Smrtí umrou, krev jejich bude na ne.
\par 17 Kdož by koli vezma sestru svou, dceru otce svého, aneb dceru matky své, videl by hanbu její, a ona také videla by hanbu jeho, mrzkost jest. Protož vyhlazeni budou pred ocima synu lidu svého; nebo hanbu sestry své odkryl, nepravost svou ponese.
\par 18 A kdož by koli spal s ženou v její nemoci a obnažil by hanbu její, a tok její by odkryl, i ona ukázala by krvotok svuj: vyhlazeni budou oba z prostredku lidu svého.
\par 19 Hanby sestry matky své a sestry otce svého neodkryješ. Nebo kdož by to ucinil, krevní prítelkyni svou by obnažil; protož nepravost svou ponesou.
\par 20 Kdož by pak spal s ženou strýce svého, hanbu strýce svého odkryl. Ponesou hrích svuj, bez detí zemrou.
\par 21 Tolikéž kdož by vzal manželku bratra svého, mrzkost jest. Hanbu bratra svého obnažil, protož bez detí budou.
\par 22 Ostríhejtež tedy všech ustanovení mých a všech soudu mých, a cinte je, aby nevyvrátila vás zeme, do níž já uvozuji vás, abyste v ní bydlili.
\par 23 Aniž chodte v ustanoveních národu toho, kterýž já vyvrhu od tvári vaší; nebo všecky ty veci cinili, a mel jsem je v ošklivosti.
\par 24 Vám jsem pak rekl: Vládnouti budete zemí jejich, a já dám ji vám, abyste ji dedicne obdrželi, zemi tekoucí mlékem a strdí. Já jsem Hospodin Buh váš, kterýž jsem vás oddelil od jiných národu.
\par 25 Protož vy mejte rozdíl mezi hovadem cistým a necistým, a mezi ptákem cistým a necistým, a nepoškvrnujte duší svých hovady a ptactvem, a tím vším, což se plazí po zemi, kteréž jsem já vám oddelil, abyste je meli za necisté.
\par 26 Ale budete mi svatí; nebo svatý jsem já Hospodin, a oddelil jsem vás od jiných národu, abyste byli moji.
\par 27 Muž pak neb žena, kteríž by meli ducha carodejného a veštího, smrtí umrou. Kamením uházejí je, krev jejich bude na ne.

\chapter{21}

\par 1 Rekl také Hospodin Mojžíšovi: Mluv knežím synum Aronovým a rci jim: Pri mrtvém nepoškvrní se žádný z vás v lidu svém.
\par 2 Toliko pri krevním príteli svém, materi neb otci svém, synu neb dceri své a bratru svém,
\par 3 Též pri sestre své, panne sobe nejbližší, kteráž nemela muže, pri té poškvrní se.
\par 4 Nepoškvrní se pri knížeti lidu svého, tak aby necistý byl.
\par 5 Nebudou delati sobe lysiny na hlave své, vytrhávajíce sobe vlasy. A brady své nebudou holiti, ani tela svého rezati.
\par 6 Budou svatí Bohu svému, aniž poškvrní jména Boha svého; nebo obeti ohnivé Hospodinovy, chléb Boha svého obetují, protož svatí budou.
\par 7 Ženy nevestky aneb poškvrnené nevezmou sobe, a ženy zahnané od muže jejího nepojmou; nebo svatý jest jeden každý z nich Bohu svému.
\par 8 Ty také, lide, za svatého budeš jej míti, nebo chléb Boha tvého obetuje. Protož svatý bude tobe, nebo já svatý jsem Hospodin, kterýž posvecuji vás.
\par 9 Dcera pak nekterého kneze, když by se smilství dopustila, otce svého poškvrnila. Ohnem spálena bud.
\par 10 Knez pak nejvyšší mezi bratrími svými, na jehožto hlavu vylit jest olej pomazání, a posvetil rukou svých, aby oblácel se v roucho svaté, hlavy své neodkryje a roucha svého neroztrhne.
\par 11 Aniž k kterému telu mrtvému pristoupí, a aniž pri otci aneb materi své poškvrní se.
\par 12 Nevyjde z svatyne a nepoškvrní svatyne Boha svého; nebo koruna oleje pomazání Boha jeho jest na nem: Já jsem Hospodin.
\par 13 On také ženu v panenstvím jejím za manželku sobe pojme.
\par 14 Vdovy, aneb zahnané, aneb poškvrnené, nevestky, žádné z tech nebude sobe bráti, ale pannu z lidu svého vezme sobe za manželku.
\par 15 A nepoškvrní semene svého v rodu svém, nebo já jsem Hospodin posvetitel jeho.
\par 16 Mluvil opet Hospodin Mojžíšovi, rka:
\par 17 Mluv k Aronovi a rci: Kdožkoli z semene tvého po všech rodech svých bude míti na sobe vadu, necht nepristupuje, aby obetoval chléb Boha svého.
\par 18 Nebo žádný muž, kterýž by mel na sobe vadu, nemá pristupovati: Muž slepý, aneb kulhavý, aneb mající oud nekterý príliš malý, aneb príliš veliký,
\par 19 Aneb muž, kterýž by mel zlámanou nohu, aneb zlámanou ruku,
\par 20 Aneb hrbovatý, aneb krhavý, aneb kterýž má belmo na oku svém, aneb prašivost ustavicnou, neb lišeje, aneb stlacené luno.
\par 21 Nižádný muž, kterýž by mel na sobe nejakou vadu, z semene Arona kneze, nepristoupí, aby obetoval obeti ohnivé Hospodinu; nebo vada na nem jest. Nepristoupít, aby obetoval chléb Boha svého.
\par 22 Chléb však Boha svého z vecí svatosvatých a z vecí svatých jísti bude.
\par 23 Ale za oponu nebude vcházeti a k oltári nebude pristupovati, nebo vada jest na nem, aby nepoškvrnil svatyne mé; nebo já jsem Hospodin posvetitel jejich.
\par 24 Ta slova mluvil Mojžíš k Aronovi a k synum jeho, i ke všechnem synum Izraelským.

\chapter{22}

\par 1 I mluvil Hospodin k Mojžíšovi, rka:
\par 2 Mluv Aronovi a synum jeho, at se zdržují od vecí tech, kteréž jsou posveceny od synu Izraelských, a at nepoškvrnují jména svatého mého v tom, což mi oni posvecují: Já jsem Hospodin.
\par 3 Rci jim: Kdo by koli ze všeho semene vašeho v pronárodech vašich pristoupil ku posveceným vecem, kterýchž by posvetili synové Izraelští Hospodinu, když necistota jeho na nem jest, vyhlazena bude duše ta od tvári mé: Já jsem Hospodin.
\par 4 Kdo by koli z semene Aronova byl malomocný, aneb tok semene trpící, nebude jísti z vecí posvecených, dokavadž by se neocistil. A kdož by se koli dotekl necistoty tela mrtvého, aneb toho, z nehož by vyšlo síme scházení,
\par 5 Aneb kdo by se dotkl kterého zemeplazu, jímž by se poškvrnil, aneb cloveka, pro nehož by byl necistý vedlé všelijaké necistoty jeho:
\par 6 Clovek, kterýž by se cehokoli toho dotekl, necistý bude až do vecera, a nebude jísti z vecí posvecených, lec by umyl telo své vodou.
\par 7 A po západu slunce cistý bude, a potom bude moci jísti z vecí posvecených, nebo pokrm jeho jest.
\par 8 Mrchy a udáveného jísti nebude, aby se tím nepoškvrnil: Já jsem, Hospodin.
\par 9 Protož ostríhati budou prisluhování mých, aby hríchu na se neuvedli, a nezemreli v nem, proto že je zprznili: Ját jsem Hospodin posvetitel jejich.
\par 10 Žádný cizí nebude jísti z vecí posvecených, ani podruh knežský, ani nájemník nebude jísti vecí posvecených.
\par 11 Koupil-li by knez cloveka za své peníze, ten jísti bude z vecí tech, též v dome jeho zplozený; ti budou jísti z pokrmu jeho.
\par 12 Ale dcera knezova, kteráž by se vdala za muže z jiného pokolení, ta z obetí vzhuru pozdvižených, totiž vecí svatých, nebude jísti.
\par 13 Kdyby pak dcera knezova ovdovela, aneb zahnána byla od muže, nemající plodu, a navrátila by se do domu otce svého: tak jako v detinství svém chléb otce svého jísti bude, cizí pak žádný nebude jísti z neho.
\par 14 Jedl-li by pak kdo z nedopatrení veci posvecené, pátý díl nad to pridá knezi, a nahradí jemu tu vec posvecenou,
\par 15 Aby nepoškvrnovali vecí svatých, kteréž by synové Izraelští obetovali Hospodinu,
\par 16 A neuvozovali na ne pokuty za provinení, že jedli veci posvecené jejich; nebo já jsem Hospodin, kterýž jich posvecuji.
\par 17 Dále mluvil Hospodin Mojžíšovi, rka:
\par 18 Mluv k Aronovi a k synum jeho i ke všechnem synum Izraelským, a rci jim: Kdož by koli z domu Izraelského aneb z pohostinných, kteríž jsou v Izraeli, obetovali obet svou vedlé všech slibu svých, vedlé všech daru dobrovolných svých, kteréž by obetovali Hospodinu v obet zápalnou:
\par 19 Z dobré vule své obetovati budete samce bez poškvrny, z skotu, z ovcí a z koz.
\par 20 Což by koli melo na sobe vadu, nebudete toho obetovati; nebo nebude príjemné od vás.
\par 21 Pakli by kdo obetoval obet pokojnou Hospodinu, vykonávaje slib svuj, aneb dobrovolný dar, bud z skotu, aneb z bravu; at jest bez vady, aby bylo príjemné; nebudet žádné poškvrny na nem.
\par 22 Slepého aneb polámaného, osekaného aneb uhrivého, prašivého aneb s lišeji, takového neobetujte Hospodinu a nedávejte jich k ohnivé obeti na oltár Hospodinuv.
\par 23 Vola neb dobytce s nedorostlými neb prerostlými oudy, v dobrovolný zajisté dar obetovati je budeš, ale za slib nebude príjemný.
\par 24 Ztlaceného aneb ztluceného, odtrženého, vyklešteného nebudete obetovati Hospodinu; neuciníte toho v zemi vaší.
\par 25 A z ruky cizozemce nebudete obetovati chleba Bohu svému ze všech tech vecí, nebo porušení jejich jest na nich; vadu mají, nebudou príjemné od vás.
\par 26 Mluvil také Hospodin k Mojžíšovi, rka:
\par 27 Vul neb beran, aneb koza, když se urodí, za sedm dní pri matce své zanecháno bude, osmého pak dne i potom bude prijemné k obeti ohnivé Hospodinu.
\par 28 Krávy pak a dobytcete s mladým jeho nezabijete jednoho dne.
\par 29 A když byste obetovali obet chvály Hospodinu, dobrovolne ji obetovati budete.
\par 30 V tentýž den snedena bude a nepozustavíte z ní niceho až do jitra: Já jsem Hospodin.
\par 31 Protož ostríhejte prikázaní mých a cinte je: Já jsem Hospodin.
\par 32 A nepoškvrnujte jména svatého mého, i budut posvecen u prostred synu Izraelských: Já jsem Hospodin posvetitel váš,
\par 33 Kterýž jsem vás vyvedl z zeme Egyptské, abych vám byl za Boha: Já jsem Hospodin.

\chapter{23}

\par 1 Mluvil opet Hospodin Mojžíšovi, rka:
\par 2 Mluv k synum Izraelským a rci jim: Slavnosti Hospodinovy, kteréž nazývati budete shromáždení svatá, tyto jsou slavnosti mé:
\par 3 Šest dní delati budete, dne pak sedmého sobota odpocinutí jest, shromáždení svaté bude. Žádného díla nedelejte, nebo jest sobota Hospodinova, ve všech príbytcích vašich.
\par 4 Protož tyto jsou slavnosti Hospodinovy, shromáždení svatá, kteréž slaviti budete v casy jich urcité:
\par 5 Mesíce prvního, ctrnáctého dne téhož mesíce u vecer bude Fáze Hospodinovo.
\par 6 A patnáctého dne téhož mesíce svátek presnic bude Hospodinu; za sedm dní presné chleby jísti budete.
\par 7 Dne prvního sbor svatý míti budete; žádného díla robotného nebudete delati.
\par 8 Ale obetovati budete obet ohnivou Hospodinu za sedm dní. Dne také sedmého sbor svatý bude; žádného díla robotného nebudete delati.
\par 9 I mluvil Hospodin k Mojžíšovi, rka:
\par 10 Mluv k synum Izraelským a rci jim: Když vejdete do zeme, kterouž já dávám vám, a žíti budete obilí její, tedy prinesete snopek prvotiny žne vaší k knezi.
\par 11 Kterýž obraceti bude sem i tam snopek ten pred Hospodinem, aby byl príjemnou obetí za vás; nazejtrí po sobote obraceti jej bude knez.
\par 12 Kterého dne obraceti budete snopek ten, téhož zabijete beránka rocního bez poškvrny v obet zápalnou Hospodinu.
\par 13 Též i obet suchou jeho, dve desetiny mouky belné olejem zadelané, v obet ohnivou Hospodinu u vuni príjemnou, a mokrou obet jeho, vína ctvrtý díl míry hin.
\par 14 Chleba pak, ani pražmy, ani zrní vymnutého nebudete jísti, až práve do toho dne, když obetovati budete obet Bohu svému. Ustanovení to vecné bude v pronárodech vašich, ve všech príbytcích vašich.
\par 15 Poctete sobe také od prvního dne po sobote, ode dne, v nemž jste obetovali snopek sem i tam obracení, (plných sedm téhodnu at jest),
\par 16 Až do prvního dne po sedmém téhodni, sectete padesáte dní, a tehdy obetovati budete novou obet suchou Hospodinu.
\par 17 Z príbytku svých prinesete chleby sem i tam obracení, dva bochníky ze dvou desetin mouky belné budou; kvašené je upecete, prvotiny jsou Hospodinu.
\par 18 A s tím chlebem obetovati budete sedm beránku rocních bez vady, a volka mladého jednoho, a skopce dva; obet zápalná budou Hospodinu, s obetmi svými suchými i mokrými, obet ohnivá vune spokojující Hospodina.
\par 19 Zabijete také kozla jednoho za hrích, a dva beránky rocní k obeti pokojné.
\par 20 I bude je knez sem i tam obraceti s chleby prvotin v obet sem i tam obracení pred Hospodinem, i s temi dvema beránky; a budou svaté veci Hospodinu, a dostanou se knezi.
\par 21 I vyhlásíte v ten den slavnost, shromáždení svaté míti budete, žádného díla robotného nebudete delati. Ustanovení to bude vecné ve všech príbytcích vašich, v pronárodech vašich.
\par 22 A když budete žíti obilé krajiny vaší, nesežneš všeho až do konce pole svého, a pozustalých klasu po žni své nebudeš sbírati; chudému a príchozímu zanecháš jich: Já jsem Hospodin Buh váš.
\par 23 Mluvil ješte Hospodin k Mojžíšovi, rka:
\par 24 Mluv synum Izraelským takto: Mesíce sedmého, v první den téhož mesíce, budete míti odpocinutí, památku troubení, shromáždení svaté držíce.
\par 25 Žádného díla robotného nebudete delati, a budete obetovati obet ohnivou Hospodinu.
\par 26 Mluvil také Hospodin k Mojžíšovi, rka:
\par 27 Desátý pak den každého mesíce sedmého den ocištování jest. Shromáždení svaté míti budete, a ponižovati budete životu svých, a obetovati obet ohnivou Hospodinu.
\par 28 Žádného díla nebudete delati v ten den; nebo den ocištování jest, k ocištování vás pred Hospodinem Bohem vaším.
\par 29 A všeliká duše, kteráž by neponižovala se toho dne, vyhlazena bude z lidu svého.
\par 30 Kdož by koli dílo nejaké delal toho dne, zatratím cloveka toho z lidu jeho.
\par 31 Žádného díla nedelejte. Ustanovení to bude vecné v pronárodech vašich, ve všech príbytcích vašich.
\par 32 Sobotu odpocinutí míti budete, když ponižovati budete duší svých, devátého dne téhož mesíce u vecer; od vecera až do druhého vecera držeti budete sobotu svou.
\par 33 Mluvil také Hospodin k Mojžíšovi, rka:
\par 34 Mluv synum Izraelským a rci: Každého patnáctého dne mesíce sedmého slavnost stánku za sedm dní bude Hospodinu.
\par 35 Dne prvního shromáždení svaté bude; žádného díla robotného nebudete delati.
\par 36 Za sedm dní obetovati budete obet ohnivou Hospodinu. Dne osmého shromáždení svaté míti budete, a obetovati budete obet ohnivou Hospodinu; svátek jest, žádného díla robotného nebudete delati.
\par 37 To jsou slavnosti Hospodinovy, kteréž slaviti budete, mívajíce shromáždení svatá, abyste v nich obetovali obet ohnivou Hospodinu, zápal, obet suchou, obet pokojnou, a obeti mokré, jedno každé ve dni svém,
\par 38 Krome sobot Hospodinových, a krome daru vašich, i všech slibu vašich a krome všech dobrovolných obetí vašich, kteréž dávati budete Hospodinu.
\par 39 A však dne patnáctého toho mesíce sedmého, když byste shromáždili úrody zeme, svetiti budete svátek Hospodinuv za sedm dní. Dne prvního odpocinutí bude, tolikéž dne osmého bude odpocinutí.
\par 40 A naberouce sobe dne prvního ovoce z stromu krásných, a ratolestí palmových, a vetvoví z stromu hustých, a vrbí od potoku, veseliti se budete pred Hospodinem Bohem svým za sedm dní.
\par 41 A tak držeti budete ten svátek Hospodinuv za sedm dní každého roku. Ustanovení to bude vecné v pronárodech vašich; každého mesíce sedmého slaviti jej budete.
\par 42 V staních zustanete za sedm dní. Kdožkoli doma zrozený jest v Izraeli, v staních zustávati budete,
\par 43 Aby vedeli potomci vaši, že jsem choval v staních syny Izraelské, když jsem je vyvedl z zeme Egyptské: Já Hospodin Buh váš.
\par 44 I oznámil Mojžíš slavnosti Hospodinovy synum Izraelským.

\chapter{24}

\par 1 Mluvil pak Hospodin k Mojžíšovi, rka:
\par 2 Prikaž synum Izraelským, at prinesou tobe oleje olivového, cistého, vytlaceného, k svícení, aby lampy ustavicne rozsvecovány byly.
\par 3 Pred oponou svedectví v stánku úmluvy zporádá je Aron, aby horely od vecera až do jitra pred Hospodinem vždycky. Tot bude ustanovení vecné v národech vašich.
\par 4 Na svícen cistý rozstavovati bude lampy pred Hospodinem vždycky.
\par 5 A vezma mouky belné, upeceš z ní dvanácte kolácu; jeden každý kolác bude ze dvou desetin efi.
\par 6 A rozkladeš je dvema rady, šest v radu jednom, na stole cistém pred Hospodinem.
\par 7 Dáš také na každý rad kadidla cistého, aby bylo za každý chléb ten kourení pametné v obet ohnivou Hospodinu.
\par 8 Každého dne sobotního klásti budete je radem pred Hospodinem vždycky, berouce je od synu Izraelských smlouvou vecnou.
\par 9 I budou Aronovi a synum jeho, kterížto jísti budou je na míste svatém; nebo nejsvetejší vec jest jim z obetí ohnivých Hospodinových právem vecným.
\par 10 Vyšel pak syn ženy Izraelské, kteréhož mela s mužem Egyptským, mezi syny Izraelskými, a vadili se v staních syn ženy té Izraelské s mužem Izraelským.
\par 11 I zlorecil syn ženy té Izraelské a rouhal se jménu Božímu. Tedy privedli ho k Mojžíšovi. (Jméno pak matky jeho bylo Salumit, dcera Dabri, z pokolení Dan.)
\par 12 A dali jej do vezení, až by jim bylo oznámeno, co s ním Buh káže uciniti.
\par 13 Mluvil pak Hospodin k Mojžíšovi, rka:
\par 14 Vyved toho ruhace ven z stanu, a nechat všickni ti, kteríž slyšeli, vloží ruce na hlavu jeho, a všecken lid at ho ukamenuje.
\par 15 Mluve pak k synum Izraelským, díš jim: Kdož by koli zlorecil Bohu svému, poneset hrích svuj.
\par 16 Kdož by zlorecil jménu Hospodinovu, smrtí umre, a všecko shromáždení bez milosti ukamenuje jej. Tak cizí, jako doma zchovaný, když by zlorecil jménu Hospodinovu, smrtí umre.
\par 17 Zabil-li by kdo kterého cloveka, smrtí umre.
\par 18 Jestliže by pak zabil hovado, navrátí jiné, hovado za hovado.
\par 19 Kdož by pak zohavil bližního svého, vedlé toho, jakž on ucinil, tak se stan jemu:
\par 20 Zlámaní za zlámaní, oko za oko, zub za zub. Jakouž by ohavu ucinil na tele cloveka, taková zase ucinena bude jemu.
\par 21 Kdož by zabil hovado, navrátí jiné, ale kdož by zabil cloveka, umre.
\par 22 Jednostejné právo míti budete. Jakož príchozímu, tak domácímu stane se; nebo já jsem Hospodin Buh váš.
\par 23 Tedy mluvil Mojžíš k synum Izraelským ty veci. I vyvedli toho ruhace ven za stany, a kamením ho zametali. Ucinili, pravím, synové Izraelští vedlé toho, jakož prikázal Hospodin Mojžíšovi.

\chapter{25}

\par 1 Mluvil ješte Hospodin k Mojžíšovi na hore Sinai, rka:
\par 2 Mluv synum Izraelským a rci jim: Když vejdete do zeme, kterouž já dávám vám, odpocívati bude zeme, nebo sobota jest Hospodinova.
\par 3 Šest let osívati budeš rolí svou, a šest let obrezovati budeš vinici svou a sbírati úrody její.
\par 4 Sedmého pak léta sobotu odpocinutí bude míti zeme, sobotu Hospodinovu; nebudeš na poli svém síti a vinice své rezati.
\par 5 Což se samo od sebe zrodí obilí tvého, nebudeš toho žíti, a hroznu vinice zanechané od tebe nebudeš sbírati. Rok odpocinutí bude míti zeme.
\par 6 Ale ovoce zeme toho odpocinutí budete míti ku pokrmu, ty i služebník tvuj, i devka tvá, i nájemník tvuj, i príchozí tvuj, kterýž bydlí u tebe,
\par 7 I hovado tvé, i všeliký živocich, kterýž jest v zemi tvé, všecky úrody její budou míti ku pokrmu.
\par 8 Secteš také sobe sedm téhodnu let, totiž sedmkrát sedm let, tak aby cas sedmi téhodnu let ucinil tobe ctyridceti devet let.
\par 9 Tedy dáš troubiti trubou veselé všudy sedmého mesíce v desátý den; v den ocištování dáš troubiti trubou po vší zemi vaší.
\par 10 I posvetíte léta padesátého, a vyhlásíte svobodu v zemi té všechnem obyvatelum jejím. Léto milostivé toto míti budete, abyste se navrátili jeden každý k statku svému, a jeden každý k celedi své zase prijde.
\par 11 Ten rok milostivý padesátého léta míti budete; nebudete síti, ani žíti toho, což by samo od sebe vzrostlo, ani sbírati vína opuštených vinic léta toho.
\par 12 Nebo milostivé léto jest, protož za svaté je míti budete; ze všelikého pole jísti budete úrody jeho.
\par 13 Toho léta milostivého navrátí se jeden každý k statku svému.
\par 14 Když nejakou vec prodáš bližnímu svému, aneb koupíš neco od bližního svého, nikoli neutiskujte jeden druhého.
\par 15 Vedlé poctu let po létu milostivém koupíš od bližního svého, a vedlé poctu let, v kterýchž úrody bráti máš, prodá tobe.
\par 16 Cím více bude let, tím vetší placení bude, a cím méne let, tím, menší placení bude; nebo pocet úrod prodá tobe.
\par 17 Protož nikoli neoklamávejte jeden druhého, ale boj se každý Boha svého; nebo já jsem Hospodin Buh váš.
\par 18 Ostríhejte ustanovení mých, a soudy mé zachovávejte a cinte je, a bydliti budete v zemi té bezpecne.
\par 19 A prinese vám zeme úrody své; i budete jísti až do sytosti, bydlíce bezpecne v ní.
\par 20 Pakli díte: Co budeme jísti léta sedmého, jestliže nebudeme síti, ani shromaždovati užitku svých?
\par 21 Dám požehnání své vám léta šestého, tak že prinese úrody na tri léta.
\par 22 I budete síti léta osmého, a jísti úrody staré až do léta devátého; dokudž by nezrostly úrody jeho, jísti budete staré.
\par 23 Zeme pak nebude prodávána v manství; nebo má jest zeme, a vy jste príchozí a podruzi u mne.
\par 24 A po vší zemi vládarství svého dopustíte dávati výplatu zeme.
\par 25 Jestliže by ochudl bratr tvuj, tak že by prodal neco z statku svého, tedy prijde príbuzný jeho, nejbližší jeho, a vyplatí prodanou vec od bratra svého.
\par 26 Pakli kdo nemaje výplatce, mohl by tomu sám dosti uciniti, tak že shledal by, což potrebí k vyplacení:
\par 27 Tedy pocte léta prodaje svého, a navrátí, což zustane, tomu, jemuž prodal; tak zase prijde k statku svému.
\par 28 Pakli by nemohl shledati toho, což by navrátiti mel, tedy zustane vec prodaná v rukou toho, kterýž ji koupil, až do léta milostivého. I postoupí mu ji v cas léta milostivého, a on navrátí se zase k dedictví svému.
\par 29 Když by kdo prodal dum k bydlení v meste hrazeném, bude míti právo k vyplacení jeho, dokudž nevyplní se rok prodaje jeho. Za celý rok bude míti právo k vyplacení jeho.
\par 30 Paklit ho nevyplatí dríve, než vyjde ten celý rok, tedy zustane dum ten v meste hrazeném tomu, kterýž jej koupil, dedicne v pronárodech jeho, aniž ho postoupí v léte milostivém.
\par 31 Domové pak ve vsech, kteréž nejsou zdí ohrazené, tak jako pole zeme pocítati se budou. Budou moci býti vyplacováni, a léta milostivého navráceni budou.
\par 32 Ale mesta Levítská, a domové v mestech dedictví jejich, ti vždycky vyplaceni mohou býti od Levítu.
\par 33 Ten pak, kdož vyplacuje, at jest z Levítu. Aneb at vyjde kupec z koupeného domu a mesta dedictví jeho v cas léta milostivého; nebo domové mest Levítských jsou dedictví jejich mezi syny Izraelskými.
\par 34 Pole pak na predmestí mest jejich nebude prodáváno, nebo dedictví jejich vecné jest.
\par 35 Jestliže by schudl bratr tvuj, a ustaly by ruce jeho u tebe, posilníš ho; též príchozí aneb podruh živiti se bude pri tobe.
\par 36 Nevezmeš od neho lichvy aneb úroku, ale báti se budeš Boha svého, aby se mohl bratr tvuj živiti u tebe.
\par 37 Penez svých nedáš jemu na lichvu, aniž pro zisk pujcovati budeš obilí svého.
\par 38 Já jsem Hospodin Buh váš, kterýž jsem vyvedl vás z zeme Egyptské, abych vám dal zemi Kananejskou, a byl vám za Boha.
\par 39 Jestliže by pak schudl bratr tvuj u tebe, tak že by se prodal tobe, nebudeš ho podrobovati v dílo otrocké.
\par 40 Jakožto nájemník a jako podruh bude pri tobe; až do léta milostivého sloužiti bude u tebe.
\par 41 Potom vyjde od tebe s detmi svými, a navrátí se k celedi své, a v dedictví otcu svých navrátí se.
\par 42 Nebo jsou služebníci moji, kteréž jsem vyvedl z zeme Egyptské; nebudou prodáváni tak jako jiní služebníci.
\par 43 Nebudeš panovati nad ním tvrde, ale báti se budeš Boha svého.
\par 44 Služebník pak tvuj aneb devka tvá, kteréž míti budeš, budou z národu tech, kteríž jsou vukol vás; z nich kupovati budete služebníky a devky.
\par 45 I od synu bydlitelu, kteríž jsou u vás pohostinu, od tech kupovati budete, a z celedí tech, kteríž jsou s vámi, kteréž zplodili v zemi vaší, a budou vám v dedictví.
\par 46 A vládnouti budete jimi právem dedicným, i synové vaší po vás, abyste je dedicne obdrželi. Na vecnost služby jejich užívati budete, ale nad bratrími svými, syny Izraelskými, nižádný nad bratrem svým nebude tvrde panovati.
\par 47 Jestliže by pak zbohatl príchozí aneb host, kterýž bydlí s tebou, a bratr tvuj prišel by na chudobu pri nem, tak že by se prodal príchozímu aneb hosti tvému, aneb obyvateli, kterýž jest z celedi cizí,
\par 48 Když by se tedy prodal, muže zase vyplacen býti. Nekdo z bratrí jeho vyplatí ho.
\par 49 Budto strýc jeho, aneb syn strýce jeho vyplatí jej, aneb nekdo z prátel krevních jeho, z rodiny jeho, vyplatí ho; aneb jestli potom sám bude moci s to býti, vyplatí se.
\par 50 I pocte se s tím, kterýž ho koupil, od léta, v kterémž se jemu prodal, až do léta milostivého, aby peníze, za než jest prodán, byly vedlé poctu let; a jakožto s nájemníkem, tak se s ním stane.
\par 51 Jestliže ješte mnoho let zustává k létu milostivému, vedlé nich navrátí výplatu svou z penez, za než koupen jest.
\par 52 Pakli málo zustává let do léta milostivého, tedy pocte se s ním, a vedlé poctu let jeho navrátí výplatu svou.
\par 53 Tak jako s celedínem rocním nakládáno bude s ním; nebude nad ním tvrde panovati pred ocima tvýma.
\par 54 Pakli by se nevyplatil v tech letech, tedy vyjde léta milostivého on, i synové jeho s ním.
\par 55 Nebo synové Izraelští jsou moji služebníci, služebníci moji jsou, kteréž jsem vyvedl z zeme Egyptské: Já Hospodin Buh váš.

\chapter{26}

\par 1 Nedelejte sobe modl, ani obrazu rytého, aneb sloupu nevyzdvihnete sobe, ani kamene malovaného v zemi vaší nestavejte, abyste se jemu klaneli; nebo já jsem Hospodin Buh váš.
\par 2 Sobot mých ostríhejte, a svatyne mé se bojte: Já jsem Hospodin.
\par 3 Jestliže v ustanoveních mých choditi budete, a prikázaní mých ostríhajíce, budete je ciniti:
\par 4 Tedy dám vám dešte vaše casy svými, a zeme vydá úrody své, a stromoví polní vydá ovoce své,
\par 5 Tak že mlácení postihne vinobraní, a vinobraní postihne setí. I budete jísti chléb svuj do sytosti, a prebývati budete bezpecne v zemi své.
\par 6 Nebo dám pokoj v zemi, i budete spáti, a nebude, kdo by vás predesil; vypléním i zver zlou z zeme, a mec nebude procházeti zeme vaší.
\par 7 Nýbrž honiti budete neprátely své, a padnou pred vámi od mece.
\par 8 Pet vašich honiti jich bude sto, a sto vašich honiti bude deset tisícu, i padnou neprátelé vaši pred vámi od mece.
\par 9 Nebo obrátím tvár svou k vám, a dám vám zrust, a rozmnožím vás, a utvrdím smlouvu svou s vámi.
\par 10 A jísti budete úrody nekolikaleté, a když nové prijdou, staré vyprázdníte.
\par 11 Vzdelám príbytek svuj u prostred vás, a duše má nebude vás nenávideti.
\par 12 A procházeti se budu mezi vámi, a budu Bohem vaším, a vy budete lidem mým.
\par 13 Já jsem Hospodin Buh váš, kterýž jsem vyvedl vás z zeme Egyptských, abyste jim nesloužili, a polámal jsem závory jha vašeho, abyste chodili prosti.
\par 14 Pakli nebudete mne poslouchati, a nebudete ciniti všech prikázaní techto,
\par 15 A jestli ustanovení má zavržete, a soudy mé zoškliví-li sobe duše vaše, tak abyste necinili všech prikázaní mých, a zrušili byste smlouvu mou:
\par 16 Já také toto uciním vám: Uvedu na vás strach, souchotiny a zimnici pálcivou, což zkazí oci vaše, a bolestí naplní duši. Semeno své nadarmo síti budete, nebo neprátelé vaši snedí je.
\par 17 Postavím zurivou tvár svou proti vám, tak že poraženi budete od neprátel svých, a panovati budou nad vámi ti, kteríž vás nenávidí. Utíkati budete, ano vás žádný nehoní.
\par 18 Jestliže ani tak poslouchati mne nebudete, tedy ješte sedmkrát více trestati vás budu pro hríchy vaše.
\par 19 A potru vyvýšenost síly vaší, a uciním nebe nad vámi jako železo, a zemi vaši jako med.
\par 20 Nadarmo bude vynaložena síla vaše, nebo zeme vaše nevydá vám úrody své, a stromoví zeme nevydá ovoce svého.
\par 21 Jestliže pak se mnou maní zacházeti budete, a nebudete mne chtíti poslouchati, tedy pridám na vás sedmkrát více ran vedlé hríchu vašich.
\par 22 A pustím na vás zver polní, kteráž uvede na vás sirobu, a vyhubí hovada vaše a umenší vás; i zpustnou cesty vaše.
\par 23 Pakli ani potom nenapravíte se, ale vždy se mnou maní zacházeti budete,
\par 24 I já s vámi maní zacházeti budu, a bíti vás budu sedmkrát více pro hríchy vaše.
\par 25 A uvedu na vás mec, kterýž vrchovate pomstí zrušení smlouvy. I shrnete se do mest svých; tam pošli mor mezi vás, a dáni budete v ruce neprátelum.
\par 26 A když polámi vám hul chleba, tedy deset žen péci bude chléb váš v peci jedné, a zase chléb váš odvažovati vám budou. Budete pak jísti, a nenasytíte se.
\par 27 Pakli i s tím nebudete mne poslouchati, ale predce maní se mnou zacházeti budete:
\par 28 I já také v hneve maní s vámi zacházeti budu, a trestati vás budu i já sedmkrát více pro hríchy vaše.
\par 29 A budete jísti tela synu svých, a tela dcer svých.
\par 30 A zkazím výsosti vaše, a vypléním slunecné obrazy vaše, a skladu tela vaše na špalky ukydaných bohu vašich, a duše má bude vás nenávideti.
\par 31 Mesta vaše dám v zpuštení, a uciním, aby zpustly svatyne vaše, aniž více zachutnám obeti vune príjemné vaší.
\par 32 Já zpustím zemi, tak že ztrnou nad ní neprátelé vaši, kteríž bydliti budou v ní.
\par 33 Vás pak rozptýlím mezi národy, a uciním, aby s dobytým mecem vás honili. I bude zeme vaše vyhubena, a mesta vaše zpustnou.
\par 34 A tehdy zeme užive sobot svých po všecky dny, v nichž pustá bude, vy pak budete v zemi neprátel svých. Tehdáž, pravím, odpocine zeme, a užive sobot svých.
\par 35 Po všecky dny, v nichž pustá bude, odpocívati bude; nebo neodpocívala v soboty vaše, když jste vy bydlili v ní.
\par 36 Kteríž pak ješte pozustanou z vás, uvedu strach na srdce jejich v krajinách neprátel jejich, tak že zažene je chrest listu vetrem chrestícího. I budou utíkati, rovne jako pred mecem, a padnou, an jich žádný honiti nebude.
\par 37 A budou padati jeden pres druhého jakožto od mece, ac jich žádný honiti nebude; aniž kdo z vás bude moci ostáti pred neprátely svými.
\par 38 I zahynete mezi národy, a zžíre vás zeme neprátel vašich.
\par 39 Kteríž pak pozustanou z vás, svadnouti budou pro nepravost svou v zemi neprátel vašich, ano i pro nepravosti otcu vašich s nimi usvadnou.
\par 40 Ale jestliže budou vyznávati nepravost svou, a nepravost otcu svých vedlé prestoupení svého, kterýmž prestoupili proti mne, a maní se mnou zacházeli,
\par 41 I já také že jsem maní s nimi zacházel, a uvedl je do zeme neprátel jejich; jestliže, pravím, tehdáž poníží se srdce jejich neobrezané, a schválí pokutu nepravosti své:
\par 42 Tedy rozpomenu se na smlouvu svou s Jákobem, a na smlouvu svou s Izákem, i na smlouvu svou s Abrahamem rozpomenu se; také i na tu zemi pametliv budu.
\par 43 Mezi tím zeme jsuc jich zbavena, užive sobot svých, kdyžto zpuštena bude pro ne. Tehdáž oni schválí pokutu nepravosti své proto, že všelijak soudy mými pohrdali, a ustanovení má zošklivila sobe duše jejich.
\par 44 A však i tak, když by v zemi neprátel svých byli, nezavrhl bych jich, a nezošklivil jich sobe, abych je mel do konce zkaziti, a zrušiti smlouvu svou s nimi; nebo já jsem Hospodin Buh jejich.
\par 45 Ale rozpomenu se na ne pro smlouvu ucinenou s predky jejich, kteréž jsem vyvedl z zeme Egyptské pred ocima pohanu, abych jim byl za Boha: Já Hospodin.
\par 46 Ta jsou ustanovení a soudové, i zákonové, kteréž vydal Hospodin na hore Sinai skrze Mojžíše, aby byli mezi ním a mezi syny Izraelskými.

\chapter{27}

\par 1 Mluvil také Hospodin k Mojžíšovi, rka:
\par 2 Mluv synum Izraelským a rci jim: Když by kdo slibem oddal duše Hospodinu, vedlé ceny tvé dá výplatu.
\par 3 Tato pak bude cena tvá: Osobe mužského pohlaví, pocna od toho, kterýž jest ve dvadcíti letech, až do šedesátiletého, uložíš výplatu padesáte lotu stríbra vedlé lotu svatyne.
\par 4 Pakli ženského pohlaví bude, uložíš výplatu tridceti lotu.
\par 5 A od petiletých až do dvadcítiletých uložíš výplatu, za osobu mužského pohlaví dvadceti lotu, ženského pak deset lotu.
\par 6 A od dítete jednoho mesíce zstárí až do petiletých, uložíš za pachole pet lotu stríbra, a za devce tri loty stríbra.
\par 7 Od šedesáti pak let a výše, bude-li muž, uložíš výplatu patnácte lotu, a žene deset lotu.
\par 8 Pakli bude tak chudý, že by nemohl uložené výplaty dáti, tedy postaven bude pred knezem, aby mu knez uložil výplatu. Podlé toho, sec bude moci býti ten, kdož slib ucinil, uloží mu výplatu.
\par 9 Jestliže by pak kdo hovado z tech, kteréž se obetují Hospodinu, slíbil, každé, kteréž dá z nich Hospodinu, svaté bude.
\par 10 Nesmení ho, aniž dá jiného za ne, lepšího za horší, aneb horšího za lepší. Jestliže by pak jakýmkoli zpusobem je smenil, hovado za hovado, tedy ono i toto, kteréž za ne dáno, svaté bude.
\par 11 Pakli by které necisté hovado slíbil, z jehož pokolení neobetují obeti Hospodinu, tedy postaví to hovado pred knezem.
\par 12 A bude je šacovati knez, bud ono dobré aneb zlé. Jakž je knez cení, tak bud.
\par 13 Pakli bude chtíti vyplatiti je, pridá pátý díl nad cenu tvou.
\par 14 Když by pak nekdo posvetil domu svého, aby byl svatý Hospodinu, bude jej knez ceniti, budto že by dobrý byl aneb zlý. Jakž jej procení knez, tak zustane.
\par 15 Pakli ten, kterýž posvetil domu svého, chtel by jej vyplatiti, pridá pátý díl penez nad cenu tvou, i bude jeho.
\par 16 Jestliže by kdo díl pole dedictví svého posvetil Hospodinu, tedy budeš je ceniti vedlé toho, jakž se osívá. Chomer jecmene kde se vseje, za padesáte lotu stríbra ceneno bude.
\par 17 Jestliže by hned od léta milostivého posvetil pole svého, tedy vedlé ceny tvé zustane.
\par 18 Pakli by po léte milostivém posvetil pole svého, tedy secte mu knez peníze vedlé poctu let zustávajících ješte do léta milostivého, i odejme to z ceny tvé.
\par 19 Chtel-li by pak vyplatiti pole ten, kterýž ho posvetil, pridá pátý díl penez nad cenu tvou, a zustane jemu.
\par 20 Nevyplatil-li by pak pole toho, a prodáno by bylo nekomu jinému, nemuže víc vyplaceno býti.
\par 21 I bude pole to, když svobodné vyjde léta milostivého, svaté Hospodinovo, jakožto pole posvecené; knezi bude v dedictví jeho.
\par 22 Jestliže pak pole koupené, kteréž by nebylo z pole dedictví jeho, posvetil Hospodinu,
\par 23 Tedy secte mu knez summu ceny jeho až do léta milostivého, i dá cenu jeho v ten den, jako posvecenou Hospodinu.
\par 24 V léte milostivém navrátí se pole k tomu, od kohož bylo koupeno, jehož dedictví jest pole to.
\par 25 Všeliká pak cena tvá bude vedlé lotu svatyne. Lot pak váží dvadceti haléru.
\par 26 Ale prvorozeného, což právem prvorozenství dává se Hospodinu z hovad, nižádný neposvetí, budto z skotu aneb z bravu; Hospodinovo jest prvé.
\par 27 A jestliže by z hovad necistých bylo, vyplatí je vedlé ceny tvé, a pridá pátý díl nad ní. Paklit nebude vyplaceno, tedy necht jest prodáno vedlé ceny tvé.
\par 28 Všeliká pak vec posvecená, kterouž by nekdo slibem posvetil Hospodinu ze všech vecí, kteréž má, bud z lidí aneb z hovad, aneb z pole dedictví svého, nebude prodávána, ani vyplacována; nebo všecko, což takovým slibem posveceno jest, vec nejsvetejší bude Hospodinu.
\par 29 Všeliké hovado tak oddané, kteréž slibem tím se oddává od cloveka, nebude vyplacováno, ale smrtí umre.
\par 30 Všickni také desátkové zeme, bud z semene zeme, aneb z ovoce stromu, Hospodinovi budou; nebo posveceni jsou Hospodinu.
\par 31 Bude-li kdo chtíti vyplatiti desátky své, pátý díl jejich pridá nad ne.
\par 32 A všeliký desátek z volu aneb drobného dobytka, jakž prichází pod hul pastýre, každý ten desátek svatý bude Hospodinu.
\par 33 Nebude vyhledávati, dobré-li by bylo cili zlé, aniž ho smení. Pakli je predce smení, bude to i ono odmenené svaté, a nebude vyplaceno.
\par 34 Ta jsou prikázaní, kteráž prikázal Hospodin Mojžíšovi na hore Sinai, aby je oznámil synum Izraelským.

\end{document}