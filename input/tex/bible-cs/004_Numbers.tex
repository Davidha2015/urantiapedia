\begin{document}

\title{Numeri}

\chapter{1}

\par 1 Mluvil pak Hospodin k Mojžíšovi na poušti Sinai, v stánku úmluvy, prvního dne mesíce druhého, léta druhého po vyjití jejich z zeme Egyptské, rka:
\par 2 Sectete summu všeho množství synu Izraelských po celedech jejich, a po domích otcu jejich, vedlé poctu jmen každého pohlaví mužského po hlavách jejich,
\par 3 Od dvadcítiletých a výše všecky, kteríž by mohli jíti k boji v Izraeli, sectete je po houfích jejich, ty a Aron.
\par 4 A bude s vámi z každého pokolení jeden muž, kterýž by prední byl v dome otcu svých.
\par 5 Tato pak jsou jména mužu, kteríž stanou s vámi: Z pokolení Rubenova Elisur, syn Sedeuruv;
\par 6 Z Simeonova Salamiel, syn Surisaddai;
\par 7 Z Judova Názon, syn Aminadabuv;
\par 8 Z Izacharova Natanael, syn Suar;
\par 9 Z Zabulonova Eliab, syn Helonuv;
\par 10 Z synu Jozefových, z pokolení Efraimova Elisama, syn Amiuduv; z Manassesova Gamaliel, syn Fadasuruv;
\par 11 Z Beniaminova Abidan, syn Gedeonuv;
\par 12 Z pokolení Dan Ahiezer, syn Amisaddai;
\par 13 Z Asser Fegiel, syn Ochranuv;
\par 14 Z pokolení Gád Eliazaf, syn Dueluv;
\par 15 Z Neftalímova Ahira, syn Enanuv.
\par 16 Ti jsou slovoutní z lidu, knížata pokolení otcu svých, ti jako hlavy tisícu Izraelských budou.
\par 17 Vzal tedy Mojžíš a Aron muže ty, kteríž jmenováni byli,
\par 18 A shromáždili všecko množství prvního dne mesíce druhého, kteríž priznávali se k rodum svým po celedech svých, po domích otcu svých, a vedlé poctu jmen, od dvadcítiletých a výše po osobách svých.
\par 19 Jakož byl prikázal Hospodin Mojžíšovi, tak scetl je na poušti Sinai.
\par 20 I bylo synu Rubena prvorozeného Izraelova, rodiny jejich, po celedech jejich, po domích otcu jejich, a podlé poctu jmen, po hlavách jejich, všech pohlaví mužského, od dvadcítiletých a výše, všech, kteríž mohli jíti k boji,
\par 21 A nacteno jich z pokolení Rubenova ctyridceti šest tisícu a pet set.
\par 22 Z synu Simeonových, rodiny jejich, po celedech jejich, po domích otcu jejich, sectených jeho vedlé poctu jmen, po hlavách jejich, všech pohlaví mužského od dvadcítiletých a výše, všech, kteríž mohli jíti k boji,
\par 23 Nacteno jich z pokolení Simeonova padesáte devet tisícu a tri sta.
\par 24 Z synu Gádových, rodiny jejich, po celedech jejich, po domích otcu jejich, podlé poctu jmen, od dvadcítiletých a výše, všech, kteríž mohli bojovati,
\par 25 Nacteno jich z pokolení Gádova ctyridceti pet tisícu, šest set a padesát.
\par 26 Z synu Judových, rodiny jejich, po celedech jejich, po domích otcu jejich, vedlé poctu jmen, od dvadcíti let a výše, všech, kteríž mohli jíti k boji,
\par 27 Nacteno jich z pokolení Judova sedmdesáte ctyri tisíce a šest set.
\par 28 Z synu Izacharových, rodiny jejich, po celedech jejich, po domích otcu jejich, podlé poctu jmen, od dvadcítiletých a výše, všech, kteríž mohli jíti k boji,
\par 29 Nacteno jich z pokolení Izacharova padesáte ctyri tisíce a ctyri sta.
\par 30 Z synu Zabulonových, rodiny jejich, po celedech jejich, po domích otcu jejich, podlé poctu jmen, od dvadcítiletých a výše, všech, kteríž mohli jíti k boji,
\par 31 Nacteno jich z pokolení Zabulonova padesáte sedm tisícu a ctyri sta.
\par 32 Z synu Jozefových, a nejprv, synu Efraimových, rodiny jejich, po celedech jejich, po domích otcu jejich, podlé poctu jmen, od dvadcítiletých a výše, všech, kteríž mohli jíti k boji,
\par 33 Nacteno jich z pokolení Efraimova ctyridceti tisíc a pet set.
\par 34 Potom z synu Manassesových, rodiny jejich, po celedech jejich, po domích otcu jejich, vedlé poctu jmen, od dvadcítiletých a výše, všech, kteríž vycházeli k boji,
\par 35 Nacteno jich z pokolení Manassesova tridceti dva tisíce a dve ste.
\par 36 Z synu Beniaminových, rodiny jejich, po celedech jejich, po domích otcu jejich, vedlé poctu jmen, od dvadcítiletých a výše, všech, kteríž mohli jíti do boje,
\par 37 Nacteno jich z pokolení Beniaminova tridceti pet tisícu a ctyri sta.
\par 38 Z synu Dan, rodiny jejich, po celedech jejich, po domích otcu jejich, vedlé poctu jmen, od dvadcítiletých a výše, všech, kteríž vycházeli k boji,
\par 39 Nacteno jich z pokolení Dan šedesáte dva tisíce a sedm set.
\par 40 Z synu Asser, rodiny jejich, po celedech jejich, po domích otcu jejich, vedlé poctu jmen, od dvadcítiletých a výše, všech, kteríž mohli jíti na vojnu,
\par 41 Nacteno jich z pokolení Asser ctyridceti jeden tisícu a pet set.
\par 42 Z synu Neftalímových, rodiny jejich, po celedech jejich, po domích otcu jejich, vedlé poctu jmen, od dvadcítiletých a výše, všech, kteríž mohli jíti k boji,
\par 43 Nacteno jich z pokolení Neftalímova padesáte tri tisíce a ctyri sta.
\par 44 Ten jest pocet tech, kteréž sectl Mojžíš a Aron a knížata Izraelská, dvanácte mužu, kteríž byli vybráni po jednom z domu otcu svých.
\par 45 I bylo všech sectených synu Izraelských po domích otcu jejich, od dvadcítiletých a výše, všech, kteríž mohli vycházeti k boji v Izraeli,
\par 46 Všech sectených bylo šestkrát sto tisíc, a tri tisíce, pet set a padesáte.
\par 47 Levítové pak vedlé pokolení otcu svých nejsou pocítáni mezi ne.
\par 48 Nebo byl mluvil Hospodin k Mojžíšovi, rka:
\par 49 Pokolení Levítského nebudeš pocítati, a nepricteš jich k sy\par Izraelským,
\par 50 Ale ustanovíš Levíty nad príbytkem svedectví, a nade vším nádobím jeho, a nade všemi vecmi, kteréž prináležejí k nemu. Oni nositi budou príbytek i všecka nádobí jeho, oni prisluhovati budou jemu, a vukol príbytku klásti se budou.
\par 51 Když se pak s místa bude míti hýbati príbytek, složí jej Levítové; a když se bude klásti príbytek, vyzdvihnou jej Levítové. Kdož by koli cizí pristoupil, umre.
\par 52 I budout se klásti synové Izraelští, jeden každý v ležení svém, a jeden každý pod praporcem svým, a po houfích svých.
\par 53 Levítové pak klásti se budou vukol príbytku svedectví, aby neprišlo rozhnevání mé na shromáždení synu Izraelských; i budou Levítové držeti stráž u príbytku svedectví.
\par 54 Ucinili tedy to synové Izraelští; všecko, jakž prikázal Hospodin Mojžíšovi, tak ucinili.

\chapter{2}

\par 1 Mluvil také Hospodin k Mojžíšovi a Aronovi, rka:
\par 2 Synové Izraelští klásti se budou jeden každý pod korouhví svou, pri praporci domu otcu svých; vukol stánku úmluvy opodál klásti se budou.
\par 3 Tito pak rozbijí stany k východní strane: Korouhev vojska Judova po houfích svých, a kníže synu Juda Názon, syn Aminadabuv,
\par 4 A u vojšte jeho lidu secteného sedmdesáte ctyri tisíce a šest set.
\par 5 Podlé neho pak položí se pokolení Izachar, a kníže synu Izachar Natanael, syn Suar,
\par 6 A u vojšte jeho lidu secteného padesáte ctyri tisíce a ctyri sta.
\par 7 Pokolení Zabulon podlé nich, a kníže synu Zabulon Eliab, syn Helonuv,
\par 8 A u vojšte jeho lidu secteného padesáte sedm tisícu a ctyri sta.
\par 9 Summa všech sectených u vojšte Judove sto osmdesáte šest tisícu a ctyri sta, po houfích jejich. Ti napred potáhnou.
\par 10 Korouhev vojska Rubenova klásti se bude ku poledni po houfích svých, a kníže synu Ruben Elisur, syn Sedeuruv,
\par 11 A u vojšte jeho lidu secteného ctyridceti šest tisícu a pet set.
\par 12 Podlé neho pak položí se pokolení Simeonovo, a kníže synu Simeon Salamiel, syn Surisaddai,
\par 13 A u vojšte jeho lidu secteného padesáte devet tisícu a tri sta.
\par 14 Potom pokolení Gád, a kníže synu Gád Eliazaf, syn Rueluv,
\par 15 A u vojšte jeho lidu secteného ctyridceti pet tisícu, šest set a padesáte.
\par 16 Summa všech sectených u vojšte Rubenove sto padesáte a jeden tisícu, ctyri sta a padesáte, po houfích svých. A ti za prvními potáhnou.
\par 17 Potom pujde stánek úmluvy s vojskem Levítu u prostred všeho vojska. Jakýmž porádkem klásti se budou, takovým potáhnou, každý v svém šiku pod korouhví svou.
\par 18 Korouhev vojska Efraimova po houfích svých bude k západu, a kníže synu Efraimových Elisama, syn Amiuduv,
\par 19 A u vojšte jeho lidu secteného ctyridceti tisíc a pet set.
\par 20 Podlé neho pak položí se pokolení Manassesovo, a kníže synu Manassesových Gamaliel, syn Fadasuruv,
\par 21 A u vojšte jeho lidu secteného tridceti dva tisíce a dve ste.
\par 22 Potom položí se pokolení Beniaminovo, a kníže synu Beniamin Abidan, syn Gedeonuv,
\par 23 A u vojšte jeho lidu secteného tridceti pet tisícu a ctyri sta.
\par 24 Summa všech sectených u vojšte Efraimove sto osm tisícu a sto osob, po houfích jejich. A tito za druhými potáhnou.
\par 25 Korouhev vojska Dan bude k strane pulnocní, po houfích svých, a kníže synu Dan Ahiezer, syn Amisaddai,
\par 26 A u vojšte jeho lidu secteného šedesáte dva tisíce a sedm set.
\par 27 Podlé neho položí se pokolení Asser, a kniže Asser Fegiel, syn Ochranuv,
\par 28 A u vojšte jeho lidu secteného ctyridceti jeden tisíc a pet set.
\par 29 Za nimi pokolení Neftalímovo, a kníže synu Neftalímových Ahira, syn Enanuv,
\par 30 A u vojšte jeho lidu secteného padesáte tri tisíce a ctyri sta.
\par 31 Summa všech sectených u vojšte Dan sto padesáte sedm tisícu a šest set. Oni nazad potáhnou pri praporcích svých.
\par 32 Ta jest summa synu Izraelských po domích otcu jejich, všech sectených v celém vojšte po houfích jejich, šestkrát sto tisícu, tri tisíce, pet set a padesáte.
\par 33 Levítové pak nejsou pocítáni mezi syny Izraelské, jakož prikázal Hospodin Mojžíšovi.
\par 34 I ucinili synové Izraelští všecko; jakž prikázal Hospodin Mojžíšovi, tak rozbijeli stany pri korouhvech svých, a tak táhli jeden každý po celedech svých a po domích otcu svých.

\chapter{3}

\par 1 Tito jsou príbehové Aronovi a Mojžíšovi od toho dne, když mluvil Hospodin s Mojžíšem na hore Sinai.
\par 2 A tato jsou jména synu Aronových: Prvorozený Nádab, potom Abiu, Eleazar a Itamar.
\par 3 Ta jsou jména synu Aronových, kneží pomazaných, jejichžto ruce naplneny, aby úrad knežství konali.
\par 4 Umrel pak Nádab a Abiu pred Hospodinem, když obetovali cizí ohen pred Hospodinem na poušti Sinai, a nemeli synu. Protož konal úrad knežský Eleazar a Itamar pred tvárí Arona otce svého.
\par 5 Mluvil pak Hospodin k Mojžíšovi, rka:
\par 6 Rozkaž pristoupiti pokolení Léví, a postav je pred Aronem knezem, aby mu prisluhovali,
\par 7 A drželi stráž jeho, i stráž všeho množství pred stánkem úmluvy k vykonávání služby príbytku,
\par 8 Též aby ostríhali všeho nádobí stánku úmluvy, a drželi stráž synu Izraelských a konali služby príbytku.
\par 9 Dáš tedy Levíty Aronovi i sy\par jeho; vlastne dáni jsou mu oni z synu Izraelských.
\par 10 Arona pak a syny jeho predstavíš, aby ostríhali knežství svého; nebo pristoupil-li by kdo cizí, umre.
\par 11 I mluvil Hospodin Mojžíšovi, rka:
\par 12 Aj, já vzal jsem Levíty z prostredku synu Izraelských na místo všelikého prvorozeného, kteréž otvírá život mezi syny Izraelskými. Protož moji budou Levítové.
\par 13 Nebo mne prináleží všecko prvorozené. Od toho dne, když jsem pobil všecko prvorozené v zemi Egyptské, posvetil jsem sobe všeho prvorozeného v Izraeli od cloveka až do hovada. Mne bude: Já Hospodin.
\par 14 Mluvil také Hospodin k Mojžíšovi na poušti Sinai, rka:
\par 15 Secti syny Léví vedlé domu otcu jejich, po celedech jejich, každého pohlaví mužského; zstárí jednoho mesíce a výše pocítati budeš je.
\par 16 I scetl je Mojžíš podlé reci Hospodinovy, jakž rozkázáno mu bylo.
\par 17 I byli synové Léví ze jména tito: Gerson, Kahat a Merari.
\par 18 Tato pak jsou jména synu Gersonových po celedech jejich: Lebni a Semei.
\par 19 A synové Kahat po celedech svých: Amram a Izar, Hebron a Uziel.
\par 20 Synové pak Merari po celedech svých: Moholi a Musi. Ty jsou celedi Léví vedlé domu otcu svých.
\par 21 Od Gersona celed Lebnitská a celed Semejská. Ty jsou celedi Gersonovy.
\par 22 A nacteno jich v poctu všech pohlaví mužského zstárí jednoho mesíce a výše sedm tisícu a pet set.
\par 23 Celedi Gersonovy za príbytkem klásti se budou k strane západní.
\par 24 Kníže pak domu otce Gersonitských bude Eliazaf, syn Laeluv.
\par 25 A k opatrování sy\par Gersonovým pri stánku úmluvy náležeti bude príbytek i stánek, prikrytí jeho i zastrení dverí stánku úmluvy,
\par 26 A ockovaté koltry k síni, i zavešení dverí síne, kteráž jest pred príbytkem a pri oltári vukol, i provazové jeho ke všeliké potrebe jeho.
\par 27 Od Kahat pak pošla celed Amramitská a celed Izaritská, a celed Hebronitská, a celed Uzielitská. Ty jsou celedi Kahat.
\par 28 V poctu všech pohlaví mužského zstárí jednoho mesíce a výše bylo osm tisícu a šest set, držících stráž pri svatyni.
\par 29 Celedi synu Kahat klásti se budou k strane príbytku polední,
\par 30 A kníže domu otcovského v celedech Kahat Elizafan, syn Uzieluv.
\par 31 V jejich pak opatrování bude truhla, stul, svícen, oltárové a nádobí svatyne, jímž prisluhovati budou, a opona i všecko, což prináleží k ní.
\par 32 Kníže pak nad knížaty Levítskými Eleazar, syn Arona kneze, ustavený nad temi, kteríž drží stráž pri svatyni.
\par 33 Od Merari pak celed Moholitská a celed Musitská. Ty jsou celedi Merari.
\par 34 A nacteno jich v poctu všech mužského pohlaví zstárí jednoho mesíce a výše šest tisícu a dve ste.
\par 35 A kníže domu otcovského v celedech Merari Suriel, syn Abichailuv. A ti klásti se budou k strane príbytku pulnocní.
\par 36 Toto pak poruceno k opatrování sy\par Merari: Dsky príbytku a svlakové jeho, sloupové a podstavkové a všecka nádobí jeho i všecka práce pri nich,
\par 37 Tolikéž sloupové síne vukol i podstavkové jejich, kolíkové i provazové jejich.
\par 38 Položí se pak pred príbytkem po prední strane pred stánkem úmluvy od východu slunce Mojžíš a Aron i synové jeho, držíce stráž pri svatyni, stráž za syny Izraelské. Jiný pristoupil-li by kdo, umre.
\par 39 Všech sectených Levítu, kteréž sectl Mojžíš s Aronem vedlé porucení Hospodinova, po celedech jejich, všech pohlaví mužského jednoho mesíce zstárí a výše dvamecítma tisícu.
\par 40 I rekl Hospodin k Mojžíšovi: Secti všecky prvorozené mužského pohlaví mezi syny Izraelskými jednoho mesíce zstárí a výše, a secti summu jmen jejich.
\par 41 A vezmeš mi Levíty, (ját jsem Hospodin), místo všech prvorozených mezi syny Izraelskými, i hovada Levítu místo všeho prvorozeného mezi hovady synu Izraelských.
\par 42 Když pak scetl Mojžíš, jakož mu prikázal Hospodin, všecky prvorozené mezi syny Izraelskými,
\par 43 Bylo všech prvorozených pohlaví mužského vedlé poctu jmen, jednoho mesíce zstárí a výše sectených jich, dvamecítma tisícu, dve ste, sedmdesáte a tri.
\par 44 A mluvil Hospodin k Mojžíšovi, rka:
\par 45 Vezmi Levíty místo všech prvorozených z synu Izraelských, i hovada Levítu za hovada jejich; i budou moji Levítové: Já jsem Hospodin.
\par 46 K vyplacení pak tech dvou set, sedmdesáti a trí, kteríž zbývají nad pocet Levítu z prvorozených synu Izraelských,
\par 47 Vezmeš pet lotu z každé hlavy, (vedlé lotu svatyne bráti budeš, lot pak dvadceti haléru váží),
\par 48 A dáš ty peníze Aronovi a sy\par jeho, výplatu tech, kteríž zbývají nad pocet jejich.
\par 49 Vzal tedy Mojžíš peníze výplaty od tech, kteríž zbývali, krome tech, kteréž vykoupili sebou Levítové,
\par 50 Od prvorozených synu Izraelských vzal penez tisíc, tri sta, šedesáte pet lotu, vedlé lotu svatyne.
\par 51 I dal ty peníze výplaty Aronovi a sy\par jeho, podlé reci Hospodinovy, jakož byl prikázal jemu Hospodin.

\chapter{4}

\par 1 I mluvil Hospodin k Mojžíšovi a k Aronovi, rka:
\par 2 Secti summu synu Kahat z prostredku synu Léví po celedech jejich a po domích otcu jejich,
\par 3 Od tridcítiletých a výše až do padesátiletých, kteríž by zpusobní jsouce k boji, mohli práci vésti pri stánku úmluvy.
\par 4 Tato pak bude práce synu Kahat pri stánku úmluvy svatyne svatých:
\par 5 Když by se mela vojska hnouti s místa, prijde Aron s syny svými a sejmou oponu zastrení, a prikryjí ní truhlu svedectví.
\par 6 A na to dají prikrytí z koží jezevcích, a pristrou svrchu rouchem z samého postavce modrého, a provlekou sochory její.
\par 7 Na stul pak chlebu predložení prostrou roucho z postavce modrého, a dají na nej misy a kadidlnice a koflíky a prikryvadla k prikrývání; a chléb ustavicne na nem bude.
\par 8 A prostrou na to roucho z cervce dvakrát barveného, a prikryjí to pristrením z koží jezevcích, a provlekou sochory jeho.
\par 9 Vezmou také roucho z postavce modrého, a prikryjí svícen svetla a lampy jeho, i uteradla jeho, i nádoby k oharkum jeho, a všecky nádoby k oleji jeho, jichž pri nem užívají.
\par 10 A obvinou jej se všechnemi nádobami jeho prikrytím z koží jezevcích, a vloží na sochory.
\par 11 Na oltár pak zlatý prostrou roucho z postavce modrého, a prikryjí jej prikrytím z koží jezevcích, a provlekou sochory.
\par 12 Vezmou i všecky nádoby k službe, jimiž prisluhovali v svatyni, a zavinouce do roucha z postavce modrého, prikryjí je prikrytím z koží jezevcích, a vloží na sochory.
\par 13 Vyprázdní i popel z oltáre, a prostrou na nej roucho šarlatové.
\par 14 A vloží svrchu všecko nádobí jeho, jímž prisluhují na nem, nádoby k uhlí, vidlicky, pometla, kotlíky a všecko nádobí oltáre, a pristrouce jej prikrytím z koží jezevcích, uvlekou sochory jeho.
\par 15 A když to vykoná Aron s syny svými, a prikryje svatyni i všecka nádobí její, a již by se mela hýbati vojska, tedy prijdou synové Kahat, aby nesli; ale nedotknou se svatyne, aby nezemreli. Ta jest práce synu Kahat pri stánku úmluvy.
\par 16 Eleazar pak, syn Arona kneze, pecovati bude o olej k svícení a kadení vonnými vecmi a obet ustavicnou i o olej pomazání, sverený sobe maje všecken príbytek a všecky veci, kteréž v nem jsou, svatyni a nádoby její.
\par 17 A mluvil Hospodin k Mojžíšovi a Aronovi, rka:
\par 18 Hledtež, abyste nevyhladili pokolení celedi Kahat z prostredku Levítu.
\par 19 Ale toto uciníte jim, aby zachováni byli a nezemreli, když by pristupovali k svatyni svatých: Aron a synové jeho prijdouce, zrídí je, jednoho každého ku práci a bremenu jeho.
\par 20 A onino nepricházejte hledeti na veci svaté, když zavinovány bývají, aby nezemreli.
\par 21 Mluvil opet Hospodin k Mojžíšovi, rka:
\par 22 Secti také syny Gersonovy po domích otcu jich a po celedech jejich.
\par 23 Od tridcítiletých a výše až do padesátiletých secteš je, kteríž by zpusobní jsouce k boji, mohli konati službu pri stánku úmluvy.
\par 24 Tato pak bude práce celedí Gersonových v službe a nošení:
\par 25 Nositi budou kortýny príbytku a stánek úmluvy s prikrytím jeho, a prikrytí z koží jezevcích, kteréž svrchu na nem jest, a zastrení dverí stánku úmluvy,
\par 26 A ockovaté koltry síne, a zastrení brány síne, kteráž jest pri stanu a pri oltári vukol, a provazy její, a všecky nádoby prisluhování jejich, a cehožkoli užívají pri službe své.
\par 27 Vedlé rozkazu Aronova a synu jeho konati budou všecky služby své synové Gersonovi pri všech pracech svých, a pri všech službách svých, a sveríte jim k ostríhání všecka bremena jejich.
\par 28 Ta jest práce celedí synu Gersonových v stánku úmluvy, a Itamar, syn Arona kneze, stráž nad nimi držeti bude.
\par 29 Syny také Merari po celedech jejich a domích otcu jejich secteš.
\par 30 Od tridcítiletých a výše až do padesátiletých secteš všecky, kteríž by zpusobní jsouce k boji, mohli konati službu pri stánku úmluvy.
\par 31 Tato pak bude povinnost práce jejich, nad všecku službu jejich pri stánku úmluvy: Dsky príbytku a svlaky jeho, i sloupy s podstavky jeho nositi,
\par 32 Sloupy také vukol síne s podstavky jejich, kolíky a provazy jejich, se všechnemi potrebami, i se vším prisluhováním jejich. A ze jména vyctete nádoby sverené jim k ostríhání.
\par 33 Ta bude práce celedí synu Merari pri všech službách jejich pri stánku úmluvy, pod spravou Itamara, syna Arona kneze.
\par 34 I sectl Mojžíš s Aronem a s knížaty lidu syny Kahat po celedech jejich, a po domích otcu jejich,
\par 35 Od tridcítiletých a výše až do padesátiletých, kteríž by zpusobní jsouce k boji, mohli konati službu pri stánku úmluvy.
\par 36 A bylo jich sectených po celedech jejich dva tisíce, sedm set, padesáte.
\par 37 Ti jsou secteni z celedi Kahat, všickni služebníci pri stánku úmluvy, kteréž scetl Mojžíš s Aronem podlé rozkazu Hospodinova skrze Mojžíše.
\par 38 Sectených také synu Gersonových po celedech jejich, a po domích otcu jejich,
\par 39 Od tridcítiletých a výše až do padesátiletých, kteríž by zpusobní jsouce k boji, mohli konati službu pri stánku úmluvy,
\par 40 Sectených jich po celedech jejich, po domích otcu jejich, dva tisíce, šest set, tridceti.
\par 41 Ti jsou secteni z celedí synu Gersonových, všickni prisluhující v stánku úmluvy, kteréž sectli Mojžíš s Aronem k rozkazu Hospodinovu.
\par 42 Sectených pak z celedí synu Merari po celedech jejich, po domích otcu jejich,
\par 43 Od tridcítiletých a výše až do padesátiletých, kteríž by zpusobní jsouce k boji, mohli konati službu pri stánku úmluvy,
\par 44 Nacteno jich po celedech jejich tri tisíce a dve ste.
\par 45 Ti jsou secteni z celedí synu Merari, kteréž scetl Mojžíš s Aronem podlé rozkazu Hospodinova skrze Mojžíše.
\par 46 Všech sectených, kteréž scetl Mojžíš s Aronem a s knížaty Izraelskými z Levítu po celedech jejich a po domích otcu jejich,
\par 47 Od tech, kteríž byli ve tridcíti letech a výše, až do padesátiletých, kterížkoli prináležejí k vykonávání služby prisluhování a práce bremena pri stánku úmluvy,
\par 48 Sectených tech bylo osm tisícu, pet set a osmdesáte.
\par 49 Vedlé rozkazu Hospodinova sectl je Mojžíš, jednoho každého vedlé prisluhování jeho, a vedlé bremene jeho. Secteni pak jsou ti, kteréž rozkázal císti Hospodin Mojžíšovi.

\chapter{5}

\par 1 I mluvil Hospodin k Mojžíšovi, rka:
\par 2 Prikaž sy\par Izraelským, at vyženou z stanu každého malomocného a každého trpícího tok semene, i každého nad mrtvým poškvrneného.
\par 3 I muže i ženu vyženete, ven za stany vyženete je, aby nepoškvrnovali vojska tech, mezi nimiž já prebývám.
\par 4 I ucinili tak synové Izraelští, a vyhnali je ven za stany. Jakož byl mluvil Hospodin k Mojžíšovi, tak ucinili synové Izraelští.
\par 5 Mluvil také Hospodin k Mojžíšovi, rka:
\par 6 Mluv k sy\par Izraelským: Muž aneb žena, když uciní nejaký hrích lidský, dopoušteje se výstupku proti Hospodinu, a byla by vinna duše ta:
\par 7 Tedy vyzná hrích svuj, kterýž ucinil, navrátí pak to, címž vinen byl, v cele, a pátý díl pridá nad to, a dá tomu, proti komuž zavinil.
\par 8 A nemel-li by muž ten prítele, jemuž by nahradil tu škodu, pokuta dána bud Hospodinu a knezi, mimo skopce ocištení, jímž ocišten býti má.
\par 9 Též všeliká obet všech vecí posvecených od synu Izraelských, kterouž prinesou knezi, jemu se dostane.
\par 10 Tak i veci posvecené od kohokoli jemu se dostanou; a dal-li kdo co knezi, také jeho bude.
\par 11 Mluvil ješte Hospodin k Mojžíšovi, rka:
\par 12 Mluv k sy\par Izraelským a rci jim: Kdyby od nekterého muže uchýlila se žena, a dopustila by se výstupku proti nemu,
\par 13 Tak že by obýval nekdo jiný s ní, a bylo by to skryto pred ocima muže jejího, a tajila by se, jsuci poškvrnena, a svedka by nebylo proti ní, a ona nebyla by postižena;
\par 14 Pohnul-li by se duch muže horlivostí velikou, tak že by horlil proti žene své, kteráž by poškvrnena byla; aneb pohnul-li by se duch muže velikou horlivostí, tak že by horlil proti žene své, kteráž by poškvrnena nebyla:
\par 15 Tedy privede muž ženu svou k knezi, a prinese obet její pri ní, desátý díl efi mouky jecné. Nenalejet na ni oleje, aniž dá na ni kadidla; nebo obet veliké horlivosti jest, obet suchá pametná, uvozující v pamet nepravost.
\par 16 I bude ji knez obetovati, a postaví ji pred Hospodinem.
\par 17 A nabere vody svaté do nádoby hlinené, a vezma prachu, kterýž jest na zemi v príbytku, dá jej do té vody.
\par 18 Potom postaví knez ženu tu pred Hospodinem, a odkryje hlavu její, a dá jí do rukou obet suchou pametnou, kteráž jest obet veliké horlivosti; v ruce pak kneze bude voda horká zlorecená.
\par 19 I zaklínati bude ji knez a rekne k ní: Jestliže neobcoval s tebou žádný, a jestliže jsi neuchýlila se k necistote od muže svého, budiž cistá od vody této horké zlorecené.
\par 20 Paklis se uchýlila od muže svého a necistá jsi, a obcoval-li nekdo jiný s tebou krome manžela tvého,
\par 21 (Zaklínati pak bude knez tu ženu, cine klatbu zlorecenství, a rekne jí:) Dejž tebe Hospodin v zlorecení a v prokletí u prostred lidu tvého, dopuste, aby luno tvé hnilo a bricho tvé oteklo.
\par 22 Vejdiž voda zlorecená tato do života tvého, aby oteklo bricho tvé, a luno tvé shnilo. I odpoví žena ta: Amen, amen.
\par 23 Napíše pak všecko zlorecenství toto do knihy, a smyje je tou vodou horkou.
\par 24 I dá žene, aby pila vodu horkou a zlorecenou; a vejdet do ní voda zlorecená, a obrátí se v horkosti.
\par 25 Potom vezme knez z ruky ženy obet veliké horlivosti, a obraceti ji bude sem i tam pred Hospodinem, a bude ji obetovati na oltári.
\par 26 A vezma plnou hrst pametného jejíhoz obeti suché, páliti to bude na oltári; a potom dá vypíti žene tu vodu.
\par 27 A když jí dá píti tu vodu, stane se, jestliže necistá byla, a dopustila se výstupku proti muži svému, že vejde do ní voda zlorecená,a obrátí se v horkost, i odme se bricho její, a vyhnije luno její; i bude žena ta v zlorecení u prostred lidu svého.
\par 28 Pakli není poškvrnena žena ta, ale cistá jest, tedy bez viny bude, a roditi bude deti.
\par 29 Ten jest zákon veliké horlivosti, když by se uchýlila žena od muže svého, a byla by poškvrnena,
\par 30 Aneb když by se pohnul duch veliké horlivosti v manželu, tak že by horlil velmi proti žene své, aby postavil ji pred Hospodinem, a aby vykonal pri ní knez všecko vedlé zákona tohoto.
\par 31 I bude ten muž ocišten od hríchu, žena pak ponese nepravost svou.

\chapter{6}

\par 1 I mluvil Hospodin k Mojžíšovi, rka:
\par 2 Mluv k sy\par Izraelským a rci jim: Muž neb žena, když se oddelí, ciníce slib nazareuv, aby se oddali Hospodinu,
\par 3 Od vína i nápoje opojného zdrží se, octa vinného a octa z nápoje opojného nebude píti, ani co vytlaceného z hroznu, zelených hroznu ani suchých nebude jísti.
\par 4 Po všecky dny nazarejství svého nebude jísti žádné veci pocházející z vinného kmene, od zrnka až do šupiny.
\par 5 Po všecky dny slibu nazarejství svého britva nevejde na hlavu jeho, dokavadž by se nevyplnili dnové, v nichž se oddelil Hospodinu. Svatý bude, a nechá rusti vlasu hlavy své.
\par 6 Po všecky dny, v nichž se oddelí Hospodinu, k telu mrtvému nevejde.
\par 7 Nad otcem svým aneb nad matkou svou, nad bratrem svým aneb nad sestrou svou, kdyby zemreli, nebude se poškvrnovati; nebo posvecení Boha jeho jest na hlave jeho.
\par 8 Po všecky dny nazarejství svého svatý bude Hospodinu.
\par 9 Umrel-li by pak kdo blízko neho náhlou smrtí, a poškvrnil-li by hlavy v nazarejství jeho, oholí hlavu svou v den ocištování svého; dne sedmého oholí ji.
\par 10 A dne osmého prinese dve hrdlicky, aneb dvé holoubátek knezi, ke dverím stánku úmluvy.
\par 11 I bude knez obetovati jedno za hrích,a druhé v zápalnou obet, a ocistí jej od toho, címž zhrešil nad mrtvým, a posvetí hlavy jeho v ten den.
\par 12 A oddelí Hospodinu dny nazarejství svého, a prinese beránka rocního za provinení, a dnové první prijdou v nic; nebo poškvrneno jest nazarejství jeho.
\par 13 Tento pak jest zákon nazareuv: V ten den, když se vyplní cas nazarejství jeho, prijde ke dverím stánku úmluvy.
\par 14 A bude obetovati obet svou Hospodinu, beránka rocního bez poškvrny, jednoho v obet zápalnou, a ovci jednu rocní bez poškvrny v obet za hrích, a skopce jednoho bez poškvrny v obet pokojnou,
\par 15 A koš chlebu presných, koláce z mouky belné olejem zadelané, a pokruty nekvašené, olejem pomazané, s obetmi jejími suchými i mokrými.
\par 16 Kteréžto veci obetovati bude knez pred Hospodinem, a uciní obet za hrích jeho, i zápalnou obet jeho.
\par 17 Skopce také obetovati bude v obet pokojnou Hospodinu, spolu s košem chlebu presných; tolikéž obetovati bude knez i obet suchou i mokrou jeho.
\par 18 Tedy oholí nazareus u dverí stánku úmluvy hlavu nazarejství svého, a vezma vlasy z hlavy nazarejství svého, vloží je na ohen, kterýž jest pod obetí pokojnou.
\par 19 Potom vezme knez plece varené z skopce toho, a jeden kolác presný z koše, a pokrutu nekvašenou jednu, a dá v ruce nazarejského, když by oholeno bylo nazarejství jeho.
\par 20 I bude obraceti knez ty veci sem i tam v obet obracení pred Hospodinem; a ta vec svatá prináležeti bude knezi mimo hrudí obracení, i plece pozdvižení. A již potom nazareus bude moci víno píti.
\par 21 Ten jest zákon nazarea, kterýž slib ucinil, a jeho obet Hospodinu za oddelení jeho, krome toho, což by více uciniti mohl. Vedlé slibu svého, kterýž ucinil, tak uciní podlé zákona oddelení svého.
\par 22 I mluvil Hospodin k Mojžíšovi, rka:
\par 23 Mluv k Aronovi a sy\par jeho a rci: Takto budete požehnání dávati sy\par Izraelským, mluvíce k nim:
\par 24 Požehnejž tobe Hospodin, a ostríhejž tebe.
\par 25 Osvet Hospodin tvár svou nad tebou, a bud milostiv tobe.
\par 26 Obratiž Hospodin tvár svou k tobe,a dejž tobe pokoj.
\par 27 I budou vzývati jméno mé nad syny Izraelskými, a já jim žehnati budu.

\chapter{7}

\par 1 I stalo se toho dne, když dokonal Mojžíš a vyzdvihl príbytek, a pomazal i posvetil ho se všechnemi nádobami jeho, také oltáre a všeho nádobí jeho pomazal a posvetil,
\par 2 Že pristupujíce knížata Izraelská, prední v domích otcu svých, (ti byli knížata pokolení, postavení nad sectenými),
\par 3 Obetovali dar svuj pred Hospodinem, šest vozu prikrytých a dvanácte volu. Dvé knížat dalo jeden vuz, a jeden každý jednoho vola, i obetovali to pred príbytkem.
\par 4 I rekl Hospodin Mojžíšovi, rka:
\par 5 Vezmi ty veci od nich, at jsou ku potrebe pri službe stánku úmluvy, a dej je Levítum, každé celedi podlé prisluhování jejího.
\par 6 Vzav tedy Mojžíš ty vozy i voly, dal je Levítum.
\par 7 Dva vozy a ctyri voly dal sy\par Gersonovým vedlé prisluhování jejich.
\par 8 Ctyri pak vozy a osm volu dal sy\par Merari vedlé prisluhování jejich, kteríž byli pod spravou Itamara, syna Aronova, kneze.
\par 9 Sy\par pak Kahat nic nedal; nebo prisluhování svatyne k nim prináleželo, a na ramenou nositi meli.
\par 10 Obetovali tedy knížata ku posvecování oltáre v ten den, když pomazán byl; obetovali, pravím, dar svuj pred oltárem.
\par 11 Rekl pak Hospodin Mojžíšovi: Jedno kníže v jeden den, druhé kníže v druhý den, porád obetovati budou dar svuj ku posvecení oltáre.
\par 12 Protož obetoval prvního dne dar svuj Názon, syn Aminadabuv, z pokolení Judova.
\par 13 Dar pak jeho byl misa stríbrná jedna, sto tridceti lotu ztíží; též báne jedna stríbrná, sedmdesáti lotu ztíží, jakž jest lot svatyne; obe dve nádoby plné mouky belné, olejem zadelané k obeti suché;
\par 14 Kadidlnice jedna z desíti lotu zlata, plná kadidla;
\par 15 Volek jeden mladý, skopec jeden, a beránek rocní jeden k obeti zápalné;
\par 16 Kozel jeden za hrích;
\par 17 A k obeti pokojné volové dva, skopcu pet, kozlu pet, a beránku rocních pet. Ta byla obet Názona, syna Aminadabova.
\par 18 Druhého dne obetoval Natanael, syn Suar, kníže z pokolení Izachar.
\par 19 Obetoval dar svuj misu stríbrnou jednu, sto tridceti lotu ztíží; báni stríbrnou jednu, sedmdesáti lotu ztíží vedlé lotu svatyne; obe dve nádoby plné mouky belné, olejem zadelané k obeti suché;
\par 20 Kadidlnici jednu z desíti lotu zlata, plnou kadidla;
\par 21 Volka mladého jednoho, skopce jednoho, a beránka rocního jednoho k obeti zápalné;
\par 22 Kozla jednoho za hrích;
\par 23 A k obeti pokojné voly dva, skopcu pet, kozlu pet, a beránku rocních pet. Ten byl dar Natanaele, syna Suar.
\par 24 Dne tretího kníže synu Zabulonových Eliab, syn Helonuv.
\par 25 Dar pak jeho byl misa stríbrná jedna, kteráž sto tridceti lotu vážila; báne stríbrná jedna, sedmdesáti lotu ztíží vedlé lotu svatyne; obe dve nádoby plné mouky belné, olejem zadelané k obeti suché;
\par 26 Kadidlnice jedna z desíti lotu zlata, plná kadidla;
\par 27 Volek jeden mladý, skopec jeden, beránek rocní jeden k obeti zápalné;
\par 28 Kozel jeden za hrích;
\par 29 A k obeti pokojné volové dva, skopcu pet, kozlu pet, a beránku rocních pet. Ta byla obet Eliaba, syna Helonova.
\par 30 Ctvrtého dne kníže synu Rubenových Elisur, syn Sedeuruv.
\par 31 Dar jeho byl misa stríbrná jedna, kteráž sto tridceti lotu vážila; báne stríbrná jedna, sedmdesáti lotu ztíží vedlé lotu svatyne; obe dve nádoby plné mouky belné, olejem zadelané v obet suchou;
\par 32 Kadidlnice jedna z desíti lotu zlata, plná kadidla;
\par 33 Volek jeden mladý, skopec jeden, beránek rocní jeden k obeti zápalné;
\par 34 Kozel jeden za hrích;
\par 35 A k obeti pokojné volové dva, skopcu pet, kozlu pet, beránku rocních pet. Ten byl dar Elisura, syna Sedeurova.
\par 36 Dne pátého kníže synu Simeonových, Salamiel, syn Surisaddai;
\par 37 Obet jeho byla misa stríbrná jedna, kteráž sto tridceti lotu vážila; báne stríbrná jedna, sedmdesáti lotu ztíží vedlé lotu svatyne; obe dve nádoby plné mouky belné, olejem zadelané k obeti suché;
\par 38 Kadidlnice jedna z desíti lotu zlata, plná kadidla;
\par 39 Volek jeden mladý, skopec jeden, beránek rocní jeden k obeti zápalné;
\par 40 Kozel jeden za hrích;
\par 41 A na obet pokojnou volové dva, skopcu pet, kozlu pet, beránku rocních pet. Ta byla obet Salamiele, syna Surisaddai.
\par 42 Dne šestého kníže synu Gád, Eliazaf, syn Dueluv.
\par 43 Dar jeho byl misa stríbrná jedna, kteráž sto tridceti lotu vážila; báne stríbrná jedna, sedmdesáti lotu ztíží vedlé lotu svatyne; obe dve nádoby plné mouky belné, olejem zadelané k obeti suché;
\par 44 Kadidlnice jedna z desíti lotu zlata, plná kadidla;
\par 45 Volek jeden mladý, skopec jeden, beránek rocní jeden k obeti zápalné;
\par 46 Kozel jeden za hrích;
\par 47 A k obeti pokojné volové dva, skopcu pet, kozlu pet, a beránku rocních pet. Ten byl dar Eliazafa, syna Duelova.
\par 48 Dne sedmého kníže synu Efraimových, Elisama, syn Amiuduv.
\par 49 Dar jeho byl misa stríbrná jedna, kteráž sto tridceti lotu vážila; báne stríbrná jedna, sedmdesáti lotu ztíží vedlé lotu svatyne; obe dve nádoby plné mouky belné, olejem zadelané na obet suchou;
\par 50 Kadidlnice jedna z desíti lotu zlata, plná kadidla;
\par 51 Volek jeden mladý, skopec jeden, beránek rocní jeden k obeti zápalné;
\par 52 Kozel jeden za hrích;
\par 53 A na obet pokojnou volové dva, skopcu pet, kozlu pet, beránku rocních pet. Ten byl dar Elisamuv, syna Amiudova.
\par 54 Dne osmého kníže synu Manasse, Gamaliel, syn Fadasuruv.
\par 55 Dar jeho byl misa stríbrná jedna, kteráž sto tridceti lotu vážila; báne stríbrná jedna, sedmdesáti lotu ztíží vedlé lotu svatyne; obe dve nádoby plné mouky belné, olejem zadelané k obeti suché;
\par 56 Kadidlnice jedna z desíti lotu zlata, plná kadidla;
\par 57 Volek jeden mladý, skopec jeden, beránek rocní jeden k obeti zápalné;
\par 58 Kozel jeden za hrích;
\par 59 A k obeti pokojné volové dva, skopcu pet, kozlu pet, beránku rocních pet. Ta byla obet Gamaliele, syna Fadasurova.
\par 60 Dne devátého kníže synu Beniaminových, Abidan, syn Gedeonuv.
\par 61 Obet jeho misa stríbrná jedna, kteráž sto tridceti lotu vážila; báne stríbrná jedna, sedmdesáti lotu ztíží vedlé lotu svatyne; obe dve nádoby plné mouky belné, olejem zadelané k obeti suché;
\par 62 Kadidlnice jedna z desíti lotu zlata, plná kadidla;
\par 63 Volek mladý jeden, skopec jeden, beránek rocní jeden k obeti zápalné;
\par 64 Kozel jeden za hrích;
\par 65 A k obeti pokojné volové dva, skopcu pet, kozlu pet, beránku rocních pet. Ten byl dar Abidanuv, syna Gedeonova.
\par 66 Desátého dne kníže synu Dan, Ahiezer, syn Amisaddai.
\par 67 Obet jeho misa stríbrná jedna, jejížto váha byla sto tridceti lotu; báne stríbrná jedna, sedmdesáti lotu ztíží vedlé lotu svatyne; obe plné mouky belné, olejem zadelané k obeti suché;
\par 68 Kadidlnice jedna z desíti lotu zlata, plná kadidla;
\par 69 Volek jeden mladý, skopec jeden, beránek rocní jeden k obeti zápalné;
\par 70 Kozel jeden za hrích;
\par 71 A k obeti pokojné volové dva, skopcu pet, kozlu pet, a beránku rocních pet. Ten byl dar Ahiezera, syna Amisaddai.
\par 72 Jedenáctého dne kníže synu Asser, Fegiel, syn Ochranuv.
\par 73 Obet jeho misa stríbrná jedna, jejížto váha byla sto tridceti lotu; báne stríbrná jedna, sedmdesáti lotu ztíží vedlé lotu svatyne; obe dve nádoby plné mouky belné, olejem zadelané k obeti suché;
\par 74 Kadidlnice jedna z desíti lotu zlata, plná kadidla;
\par 75 Volek jeden mladý, skopec jeden, beránek rocní jeden k obeti zápalné;
\par 76 Kozel jeden za hrích;
\par 77 A k obeti pokojné volové dva, skopcu pet, kozlu pet, beránku rocních pet. Ten byl dar Fegiele, syna Ochranova.
\par 78 Dvanáctého dne kníže synu Neftalím, Ahira, syn Enanuv.
\par 79 Obet jeho byla misa stríbrná jedna, jejížto váha byla sto tridceti lotu; báne stríbrná jedna, sedmdesáti lotu ztíží vedlé lotu svatyne; obe dve nádoby plné mouky belné, olejem zadelané k obeti suché;
\par 80 Kadidlnice jedna z desíti lotu zlata, plná kadidla;
\par 81 Volek mladý jeden, skopec jeden, beránek rocní jeden k obeti zápalné;
\par 82 Kozel jeden za hrích;
\par 83 A k obeti pokojné volové dva, skopcu pet, kozlu pet, beránku rocních pet. Ta byla obet Ahiry, syna Enanova.
\par 84 Tot jest posvecení oltáre (toho dne, když pomazán jest), od knížat Izraelských: Mis stríbrných dvanácte, bání stríbrných dvanácte, kadidlnic zlatých dvanácte.
\par 85 Sto tridceti lotu vážila jedna misa stríbrná, a sedmdesáte báne jedna; všecko nádobí stríbrné vážilo dva tisíce a ctyri sta lotu na lot svatyne.
\par 86 Kadidlnic zlatých dvanácte, plných kadidla; deset lotu vážila každá kadidlnice na váhu svatyne. Všecky kadidlnice zlaté sto a dvadceti lotu vážily.
\par 87 Všeho dobytka k obeti zápalné dvanácte volku, skopcu dvanácte, beránku rocních dvanácte s obetí jejich suchou, a kozlu dvanácte za hrích.
\par 88 Všeho pak dobytka k obeti pokojné ctyrmecítma volu, skopcu šedesáte, kozlu šedesáte, beránku rocních šedesáte. To bylo posvecení oltáre, když pomazán byl.
\par 89 Potom když vcházel Mojžíš do stánku úmluvy, aby mluvil s Bohem, tedy slýchal hlas mluvícího k sobe z slitovnice, kteráž byla nad truhlou svedectví mezi dvema cherubíny,a mluvíval k nemu.

\chapter{8}

\par 1 I mluvil Hospodin k Mojžíšovi, rka:
\par 2 Mluv k Aronovi a rci jemu: Když rozsvecovati budeš lampy, ven z svícnu sedm lamp svítiti má.
\par 3 I ucinil Aron tak; ven z svícnu svetlo od sebe dávající rozsvítil lampy jeho, jakož prikázal Hospodin Mojžíšovi.
\par 4 Bylo pak dílo svícnu takové: Z taženého zlata byl všecken, až i sloupec jeho i kvetové jeho z taženého zlata byli; podlé podobenství toho, kteréž ukázal Hospodin Mojžíšovi, tak udelal svícen.
\par 5 Mluvil také Hospodin k Mojžíšovi, rka:
\par 6 Vezmi Levíty z prostredku synu Izraelských, a ocist je.
\par 7 Tímto pak zpusobem ocištovati je budeš: Pokropíš jich vodou ocištení; oholí všecko telo své, a zperou roucha svá, a ocišteni budou.
\par 8 Potom vezmou volka mladého a obet suchou z mouky belné, olejem skropené; a druhého volka mladého vezmeš k obeti za hrích.
\par 9 Tedy pristoupiti rozkážeš Levítum pred stánek úmluvy, a shromáždíš všecko množství synu Izraelských.
\par 10 Postavíš Levíty pred Hospodinem, a vloží synové Izraelští ruce své na Levíty.
\par 11 A obetovati bude Aron Levíty v obet pred Hospodinem od synu Izraelských, aby vykonávali službu Hospodinu.
\par 12 Levítové pak vloží ruce své na hlavy tech volku; a obetovati budeš jednoho za hrích, a druhého v obet zápalnou Hospodinu k ocištení Levítu.
\par 13 A postavíš Levíty pred Aronem a pred syny jeho, a obetovati je budeš v obet Hospodinu.
\par 14 I oddelíš Levíty z prostredku synu Izraelských, aby moji byli Levítové.
\par 15 Potom pak prijdou Levítové, aby prisluhovali pri stánku úmluvy, když bys ocistil je a obetoval v obet.
\par 16 Nebo vlastne dáni jsou mi z prostredku synu Izraelských za všecky otvírající život, za prvorozené ze všech synu Izraelských vzal jsem je sobe.
\par 17 Nebo mé jest všecko prvorozené mezi syny Izraelskými, tak z lidí jako z hovad; od toho dne, jakž jsem pobil všecko prvorozené v zemi Egyptské, posvetil jsem jich sobe.
\par 18 Vzal jsem pak Levíty za všecky prvorozené synu Izraelských.
\par 19 A dal jsem Levíty darem Aronovi i sy\par jeho z prostredku synu Izraelských, aby konali službu místo synu Izraelských pri stánku úmluvy, a ocištovali od hríchu syny Izraelské; a tak neprijde na syny Izraelské rána,kteráž by prišla, kdyby pristupovali synové Izraelští k svatyni.
\par 20 I ucinili Mojžíš a Aron i všecko množství synu Izraelských pri Levítích všecko to, což prikázal Hospodin Mojžíšovi o Levítích; tak s nimi ucinili synové Izraelští.
\par 21 A ocistili se Levítové a zeprali roucha svá; a obetoval je Aron v obet pred Hospodinem, a ocistil je Aron, aby byli cisti.
\par 22 Potom teprv pristoupili Levítové k vykonávání služby své pri stánku úmluvy, pred Aronem i pred syny jeho; jakž prikázal Hospodin Mojžíšovi o Levítích, tak s nimi ucinili.
\par 23 I mluvil opet Hospodin k Mojžíšovi, rka:
\par 24 I toto k Levítum prináleží: V petmecítma letech zstárí a výše jeden každý z nich pristoupí, a postaví se k ochotnému práce konání v službe pri stánku úmluvy.
\par 25 V padesáti pak letech prestane od práce služby té, a více prisluhovati nebude.
\par 26 Ale prisluhovati bude bratrím svým pri stánku úmluvy stráž držícím, sám pak služeb konati nebude. Tak uciníš s Levíty pri pracech jejich.

\chapter{9}

\par 1 Mluvil pak Hospodin k Mojžíšovi na poušti Sinai, léta druhého po vyjití z zeme Egyptské, mesíce prvního, rka:
\par 2 Slaviti budou synové Izraelští velikunoc v cas svuj vymerený.
\par 3 Ctrnáctého dne mesíce toho u vecer budete ji slaviti jistým casem svým, vedlé všech ustanovení jejích, a podlé všech rádu jejích slaviti ji budete.
\par 4 I mluvil Mojžíš k sy\par Izraelským, aby slavili Fáze.
\par 5 Tedy slavili Fáze v mesíci prvním, ctrnáctého dne u vecer na poušti Sinai; vedlé všeho toho, což prikázal Hospodin Mojžíšovi, tak ucinili synové Izraelští.
\par 6 I byli nekterí muži, ješto se poškvrnili pri mrtvém, kteríž nemohli slaviti Fáze toho dne. I pristoupili pred Mojžíše a Arona v ten den,
\par 7 A promluvili muži ti k nemu: My jsme se poškvrnili nad mrtvým. Nebude-liž nám zbráneno obetovati obeti Hospodinu v jistý cas spolu s syny Izraelskými?
\par 8 I rekl jim Mojžíš: Pocekejte, až uslyším, co vám uciniti rozkáže Hospodin.
\par 9 Mluvil pak Hospodin k Mojžíšovi, rka:
\par 10 Mluv k sy\par Izraelským a rci: Kdož by koli byl poškvrnený nad mrtvým, aneb byl by na ceste daleké, bud z vás aneb z potomku vašich, budet slaviti Fáze Hospodinu.
\par 11 Mesíce druhého, ctrnáctého dne u vecer slaviti budou je, s chleby nekvašenými, a s rerichami jísti je budou.
\par 12 Nezanechajít ho nic až do jitra, a kosti v nem nezlámí; vedlé všelikého ustanovení Fáze budou je slaviti.
\par 13 Ale clovek ten, kterýž by byl cistý, a nebyl na ceste, a však by zanedbal slaviti Fáze, vyhlazena bude duše ta z lidu svého; nebo obeti Hospodinu neobetoval v jistý cas její; hrích svuj ponese clovek ten.
\par 14 Jestliže by s vámi bydlil príchozí, a slavil by Fáze Hospodinu, vedlé ustanovení Fáze, a vedlé rádu jeho bude je slaviti. Ustanovení jednostejné bude vám, tak príchozímu, jako obyvateli v zemi.
\par 15 Toho pak dne, v kterémž vyzdvižen jest príbytek, prikryl oblak príbytek, a stál nad stánkem svedectví; u vecer pak bývalo nad príbytkem na pohledení jako ohen až do jitra.
\par 16 Tak bývalo ustavicne, oblak prikrýval jej ve dne, záre pak ohnivá v noci.
\par 17 A když se zdvihl oblak od stánku, hned také hýbali se synové Izraelští; a na kterém míste pozustal oblak, tu také kladli se synové Izraelští.
\par 18 K rozkazu Hospodinovu hýbali se synové Izraelští, a k rozkazu Hospodinovu kladli se; po všecky dny, dokudž zustával oblak nad príbytkem, i oni leželi.
\par 19 Když pak trval oblak nad príbytkem po mnohé dny, tedy drželi synové Izraelští stráž Hospodinovu, a netáhli odtud.
\par 20 A když oblak byl nad príbytkem po nemnohé dny, k rozkazu Hospodinovu kladli se, a k rozkazu Hospodinovu hýbali se.
\par 21 Kdyžkoli byl oblak od vecera až do jitra, a v jitre se vznesl, hned i oni šli; bud že trval pres den a noc, (jakž kdy vznášel se oblak, tak oni táhli,)
\par 22 Bud že za dva dni, aneb za mesíc, aneb za rok prodléval oblak nad príbytkem, zustávaje nad ním, synové Izraelští také leželi,a nehnuli se; když pak on vznášel se, též i oni táhli.
\par 23 K rozkazu Hospodinovu kladli se, a k rozkazu Hospodinovu hýbali se, stráž Hospodinovu držíce podlé rozkazu jeho skrze Mojžíše.

\chapter{10}

\par 1 I mluvil Hospodin k Mojžíšovi, rka:
\par 2 Udelej sobe dve trouby stríbrné. Dílem taženým udeláš je, kterýchž užívati budeš k svolání všeho množství, a když by se melo hnouti vojsko.
\par 3 Protož kdyžkoli zatroubí na ne, shromáždí se k tobe všecko množství ke dverím stánku úmluvy.
\par 4 Jestliže v jednu toliko zatroubí, tedy shromáždí se k tobe knížata, prední lidu Izraelského.
\par 5 Pakli by s nejakým pretrubováním troubili, hnou se s místa, kteríž leželi k východní strane.
\par 6 Když by pak troubili s pretrubováním po druhé, tedy hnou se ti, kteríž leželi ku poledni. S pretrubováním troubiti budou k tažení svému.
\par 7 Ale když byste meli svolati všecko množství, proste bez pretrubování troubiti budete.
\par 8 Synové Aronovi kneží trubami temi budou troubiti, a bude vám to ustanovení vecné v pronárodech vašich.
\par 9 Když vyjdete na vojnu v zemi vaší proti nepríteli ssužujícímu vás, s pretrubováním troubiti budete v trouby ty, a budete v pameti pred Hospodinem Bohem svým, a zachováni budete od neprátel svých.
\par 10 V den také veselí vašeho, a pri slavnostech svých, a pri zacátcích mesícu vašich troubiti budete v trouby ty k obetem svým zápalným a pokojným, i budou vám na památku pred Bohem vaším: Já Hospodin Buh váš.
\par 11 I stalo se léta druhého, dvadcátý den mesíce druhého, že se vyzdvihl oblak z príbytku svedectví.
\par 12 I táhli synové Izraelští po svých taženích z poušte Sinai; a zastavil se oblak na poušti Fáran.
\par 13 Takto nejprvé brali se z rozkazu Hospodinova skrze Mojžíše:
\par 14 Šla napred korouhev vojska synu Juda po houfích svých, a nad nimi byl Názon, syn Aminadabuv.
\par 15 Nad vojskem pak pokolení synu Izachar byl Natanael, syn Suar.
\par 16 A nad vojskem pokolení synu Zabulon Eliab, syn Helonuv.
\par 17 Složen jest také i príbytek, a šli synové Gersonovi a synové Merari nesouce príbytek.
\par 18 Potom šla korouhev vojska Rubenova po houfích svých, a nad nimi byl Elisur, syn Sedeuruv.
\par 19 Nad vojskem pak pokolení synu Simeon Salamiel, syn Surisaddai.
\par 20 A nad vojskem pokolení synu Gád Eliazaf, syn Dueluv.
\par 21 Šli také i Kahatští, nesouce svatyni; onino pak vyzdvihovali príbytek, až i tito prišli.
\par 22 Potom šla korouhev vojska synu Efraim po houfích svých, a nad nimi byl Elisama, syn Amiuduv.
\par 23 Nad vojskem pak pokolení synu Manasse Gamaliel, syn Fadasuruv.
\par 24 A nad vojskem pokolení synu Beniamin Abidan, syn Gedeonuv.
\par 25 Šla potom i korouhev vojska synu Dan, obsahující ostatek vojska po houfích svých, a nad vojskem jeho Ahiezer, syn Amisaddai.
\par 26 Nad vojskem pak pokolení synu Asser Fegiel, syn Ochranuv.
\par 27 A nad vojskem pokolení synu Neftalím Ahira, syn Enanuv.
\par 28 Ta jsou tažení synu Izraelských po houfích jejich, a tím porádkem táhli.
\par 29 Rekl pak Mojžíš Chobabovi, synu Raguelovu Madianskému, tchánu svému: My se béreme k místu, o kterémž rekl Hospodin: Dám je vám. Protož pod s námi, a dobre uciníme tobe; nebo Hospodin mnoho dobrého zaslíbil Izraelovi.
\par 30 Kterýž odpovedel jemu: Nepujdu, ale k zemi své a k príbuznosti své se navrátím.
\par 31 I rekl Mojžíš: Neopouštej medle nás; nebo ty jsi svedom na poušti, kde bychom se meli klásti, a budeš nám za vudce.
\par 32 Když pak pujdeš s námi, a prijde to dobré, jímž dobre uciní nám Hospodin, tedy i tobe dobre uciníme.
\par 33 A tak brali se od hory Hospodinovy cestou trí dnu, (a truhla smlouvy Hospodinovy predcházela je,) cestou trí dnu, pro vyhlédání sobe místa k odpocinutí.
\par 34 A oblak Hospodinuv byl nad nimi ve dne, když se hýbali z ležení.
\par 35 Když pak pocínali jíti s truhlou, ríkával Mojžíš: Povstaniž Hospodine, a rozptýleni budte neprátelé tvoji, a at utíkají pred tvárí tvou, kteríž te v nenávisti mají.
\par 36 Když pak stavína byla, ríkával: Navratiž se, Hospodine, k desíti tisícum tisícu Izraelských.

\chapter{11}

\par 1 I stalo se, že lid ztežoval a stýskal sobe, cožse nelíbilo Hospodinu. Protož, slyše to Hospodin, rozhneval se náramne, a roznítil se proti nim ohen Hospodinuv, a spálil zadní díl vojska.
\par 2 Tedy volal lid k Mojžíšovi. I modlil se Mojžíš Hospodinu, a uhasl ohen.
\par 3 I nazval jméno místa toho Tabbera; nebo rozpálil se proti nim ohen Hospodinuv.
\par 4 Lid pak k nim primíšený napadla žádost náramná; a obrátivše se, plakali i synové Izraelští a rekli: Kdo nám to dá, abychom se masa najedli?
\par 5 Rozpomínáme se na ryby, jichž jsme dosti v Egypte darmo jídali, na okurky a melouny, též na por, cibuli a cesnek.
\par 6 A nyní duše naše vyprahlá, nic jiného nemá, krome tu mannu pred ocima svýma.
\par 7 (Manna pak byla jako semeno koliandrové, a barva její jako barva bdelium.
\par 8 I vycházíval lid, a sbírali a mleli žernovy, neb tloukli v moždírích, a smažili na pánvici, aneb koláce podpopelné delali z ní; chut pak její byla jako chut nového oleje.
\par 9 Když pak sstupovala rosa na vojsko v noci, tedy sstupovala také i manna).
\par 10 Tedy uslyšel Mojžíš, an lid pláce po celedech svých, každý u dverí stanu svého. Procež roznítila se prchlivost Hospodinova náramne; Mojžíšovi to také težké bylo.
\par 11 I rekl Mojžíš Hospodinu: Proc jsi tak zle ucinil služebníku svému? Proc jsem nenalezl milosti pred ocima tvýma, že jsi vložil bríme všeho lidu tohoto na mne?
\par 12 Zdaliž jsem já pocal všecken lid tento? Zdali jsem já zplodil jej, že mi díš: Nes jej na rukou svých, jako nosí chuva detátko, do zeme té, kterouž jsi s prísahou zaslíbil otcum jejich?
\par 13 Kde mám nabrati masa, abych dal všemu lidu tomuto? Nebo plací na mne, rkouce: Dej nám masa, at jíme.
\par 14 Nemohut já sám nésti všeho lidu tohoto, nebo jest to nad možnost mou.
\par 15 Pakli mi tak delati chceš, prosím, zabí mne radeji, jestliže jsem nalezl milost pred ocima tvýma, abych více nehledel na trápení své.
\par 16 Tedy rekl Hospodin Mojžíšovi: Shromažd mi sedmdesáte mužu z starších Izraelských, kteréž znáš, že jsou starší v lidu a správcové jeho; i privedeš je ke dverím stánku úmluvy, a státi budou tam s tebou.
\par 17 A já sstoupím a mluviti budu s tebou, a vezmu z ducha, kterýž jest na tobe, a dám jim. I ponesou s tebou bríme lidu, a tak ty ho sám neponeseš.
\par 18 Lidu pak díš: Posvettež se k zítrku,a budete jísti maso; nebo jste plakali v uších Hospodinových, rkouce: Kdo nám dá najísti se masa? Jiste že lépe nám bylo v Egypte. I dá vám Hospodin masa, a budete jísti.
\par 19 Nebudete toliko jeden den jísti, ani dva, ani pet, ani deset, ani dvadcet,
\par 20 Ale za celý mesíc, až vám chrípemi poleze, a zoškliví se, proto že jste pohrdli Hospodinem, kterýž jest u prostred vás, a plakali jste pred ním, ríkajíce: Proc jsme vyšli z Egypta?
\par 21 I rekl Mojžíš: Šestkrát sto tisícu peších jest tohoto lidu, mezi nimiž já jsem, a ty pravíš: Dám jim masa, aby jedli za celý mesíc.
\par 22 Zdali ovcí a volu nabije se jim, aby jim postacilo? Aneb zdali všecky ryby morské shromáždí se jim, aby jim dosti bylo?
\par 23 Tedy rekl Hospodin Mojžíšovi: Zdali ruka Hospodinova ukrácena jest? Již nyní uzríš, prijde-li na to, což jsem mluvil, cili nic.
\par 24 Vyšed tedy Mojžíš, mluvil lidu slova Hospodinova, a shromáždiv sedmdesáte mužu z starších lidu, postavil je vukol stánku.
\par 25 I sstoupil Hospodin v oblaku a mluvil k nemu, a vzav z ducha, kterýž byl na nem, dal sedmdesáti mužum starším. I stalo se, když odpocinul na nich ten duch, že prorokovali, ale potom nikdy více.
\par 26 Byli pak zustali v staních dva muži, jméno jednoho Eldad, a druhého Medad, na nichž také odpocinul duch ten, nebo i oni napsani byli, ackoli nevyšli k stánku. I ti také prorokovali v staních.
\par 27 Tedy pribeh služebník, oznámil to Mojžíšovi, rka: Eldad a Medad prorokují v staních.
\par 28 Jozue pak, syn Nun, služebník Mojžíšuv, jeden z mládencu jeho, dí k tomu: Pane muj, Mojžíši, zabran jim.
\par 29 Jemuž odpovedel Mojžíš: Proc ty horlíš pro mne? Nýbrž ó kdyby všecken lid Hospodinuv proroci byli, a aby dal Hospodin ducha svého na ne!
\par 30 I navrátil se Mojžíš do stanu s staršími Izraelskými.
\par 31 Tedy strhl se vítr od Hospodina, a zachvátiv krepelky od more, spustil je na stany tak široce a dlouze, co by mohl za jeden den cesty ujíti všudy vukol táboru, takmer na dva lokty zvýší nad zemí.
\par 32 Protož vstav lid, celý ten den a celou noc, i celý druhý den shromaždovali sobe ty krepelky; a kdož nejméne nashromáždil, mel jich s deset mer. I rozvešeli je sobe porád okolo stanu.
\par 33 Ješte maso vezelo v zubích jejich, a nebylo práve sžvýkáno, když hnev Hospodinuv vzbudil se na lid. I ranil Hospodin lid ranou velikou náramne.
\par 34 Protož nazváno jest jméno místa toho Kibrot Hattáve; nebo tu pochovali lid, kterýž žádal masa.
\par 35 I bral se lid z Kibrot Hattáve na poušt Hazerot, a pozustali v Hazerot.

\chapter{12}

\par 1 A mluvila Maria i Aron proti Mojžíšovi prícinou manželky Madianky, kterouž sobe vzal; nebo byl pojal manželku Madianku.
\par 2 A rekli: Zdaliž jen toliko skrze Mojžíše mluvil Hospodin? Zdaž také nemluvil skrze nás? I slyšel to Hospodin.
\par 3 (Byl pak Mojžíš clovek nejtišší ze všech lidí, kteríž byli na tvári zeme).
\par 4 A ihned rekl Hospodin Mojžíšovi a Aronovi i Marii: Vyjdete vy tri k stánku úmluvy. I vyšli toliko oni tri.
\par 5 Tedy sstoupil Hospodin v sloupu oblakovém, a stál u dverí stánku. I zavolal Arona a Marie, a vyšli oba dva.
\par 6 Jimž rekl: Slyšte nyní slova má: Prorok když jest mezi vámi, já Hospodin u videní ukáži se jemu, ve snách mluviti budu s ním.
\par 7 Ale není takový služebník muj Mojžíš, kterýž ve všem dome mém verný jest.
\par 8 Ústy k ústum mluvím s ním, ne u videní, ani v zavinutí, a podobnost Hospodinovu spatruje. Procež jste tedy neostýchali se mluviti proti služebníku mému Mojžíšovi?
\par 9 I roznícena jest prchlivost Hospodinova na ne, a odšel.
\par 10 Oblak také zdvihl se s stánku. A aj, Maria byla malomocná, bílá jako sníh. A pohledel Aron na Marii, ana malomocná.
\par 11 Tedy rekl Aron Mojžíšovi: Poslyš mne, pane muj, prosím, nevzkládej na nás té pokuty za hrích ten, jehož jsme se bláznive dopustili, a jímž jsme zhrešili.
\par 12 Prosím, at není tato jako mrtvý plod, kterýž když vychází z života matky své, polovici tela jeho zkaženého bývá.
\par 13 I volal Mojžíš k Hospodinu, rka: Ó Bože silný, prosím, uzdraviž ji.
\par 14 Odpovedel Hospodin Mojžíšovi: Kdyby otec její plinul jí na tvár, zdaž by se nemusila stydeti za sedm dní? Protož at jest vyobcována ven z stanu za sedm dní, a potom zase uvedena bude.
\par 15 Tedy vyloucena jest Maria ven z stanu za sedm dní; a lid nehnul se odtud, až zase uvedena jest Maria.

\chapter{13}

\par 1 Potom pak bral se lid z Hazerot, a položil se na poušti Fáran.
\par 2 I mluvil Hospodin k Mojžíšovi, rka:
\par 3 Pošli sobe muže, kteríž by spatrili zemi Kananejskou, kterouž já dám sy\par Izraelským; jednoho muže z každého pokolení otcu jejich vyšlete, ty, kteríž by byli prednejší mezi nimi.
\par 4 Poslal je tedy Mojžíš z poušte Fáran, jakž byl rozkázal Hospodin; a všickni ti muži byli knížata mezi syny Izraelskými.
\par 5 Tato jsou pak jména jejich: Z pokolení Ruben Sammua, syn Zakuruv;
\par 6 Z pokolení Simeon Safat, syn Huri;
\par 7 Z pokolení Juda Kálef, syn Jefonuv;
\par 8 Z pokolení Izachar Igal, syn Jozefuv;
\par 9 Z pokolení Efraim Ozeáš, syn Nun;
\par 10 Z pokolení Beniamin Falti, syn Rafuv;
\par 11 Z pokolení Zabulon Gaddehel, syn Sodi;
\par 12 Z pokolení Jozef a z pokolení Manasses Gaddi, syn Susi;
\par 13 Z pokolení Dan Amiel, syn Gemaluv;
\par 14 Z pokolení Asser Setur, syn Michaeluv;
\par 15 Z pokolení Neftalím Nahabi, syn Vafsi;
\par 16 Z pokolení Gád Guhel, syn Máchuv;
\par 17 Ta jsou jména mužu, kteréž poslal Mojžíš, aby shlédli zemi. I prezdel Mojžíš Ozeovi, synu Nun, Jozue.
\par 18 Tedy poslal je Mojžíš, aby prohlédli zemi Kananejskou, a rekl jim: Jdete tudyto pri strane polední a vejdete na hory.
\par 19 A shlédnete, jaká jest zeme ta, i lid, kterýž bydlí v ní, silný-li jest ci mdlý? málo-li jich, ci mnoho?
\par 20 A jaká jest též zeme ta, v níž on bydlí, dobrá-li jest, ci zlá? a jaká jsou mesta, v nichž prebývají, v staních-li, ci v hrazených místech?
\par 21 Tolikéž jaká jest zeme, úrodná-li, ci neúrodná, jest-li na ní stromoví, ci není? A budte udatné mysli, a prineste nám z ovoce té zeme. A byl tehdáž cas, v nemžto hroznové zamekali.
\par 22 Odšedše tedy, prohlédli tu zemi od poušte Tsin až do Rohob, kudy se jde do Emat.
\par 23 I šli stranou polední, a prišli až do Hebronu, kdežto byli Achiman a Sesai a Tolmai, synové Enakovi. Hebron pak o sedm let prvé ustaveno jest, nežli Soan, mesto Egyptské.
\par 24 Potom prišli až do údolí Eškol, a tu urezali ratolest s hroznem jedním jahodek plným, a nesli jej na sochore dva; též i jablka zrnatá i fíky té zeme.
\par 25 I nazváno jest to místo Nehel Eškol, od hroznu, kterýž tu urezali synové Izraelští.
\par 26 Navrátili se pak zase, prošedše zemi, po ctyridcíti dnech.
\par 27 A jdouce, prišli k Mojžíšovi a k Aronovi i ke všemu množství synu Izraelských, na poušti Fáran v Kádes, a oznámili jim i všemu množství to, co spravili, a ukázali jim ovoce zeme té.
\par 28 A vypravujíce jim, rekli: Prišli jsme do zeme, do kteréž jsi nás poslal, kteráž v pravde oplývá mlékem a strdí, a toto jest ovoce její.
\par 29 Než že lid jest silný, kterýž bydlí v zemi té, mesta také pevná jsou a veliká velmi, k tomu i syny Enakovy tam jsme videli.
\par 30 Amalech bydlí v kraji poledním, a Hetejský, Jebuzejský a Amorejský bydlejí na horách, Kananejský pak bydlí pri mori a pri brehu Jordánském.
\par 31 I krotil Kálef lid bourící se proti Mojžíšovi, a mluvil: Jdeme predce, a opanujme zemi, nebo zmocníme se jí.
\par 32 Ale muži ti, kteríž chodili s ním, pravili: Nikoli nebudeme moci vstoupiti proti lidu tomu, nebo silnejší jest nežli my.
\par 33 I zhaneli a zošklivili zemi shlédnutou sy\par Izraelským, mluvíce: Zeme, již jsme prošli a spatrili, jest zeme taková, ješto hubí obyvatele své; a všecken lid, kterýž jsme videli u prostred ní, jsou muži postavy vysoké velmi.
\par 34 Také jsme tam videli obry, syny Enakovy, kteríž jsou vetší než jiní obrové, ješto se nám zdálo, že jsme proti nim jako kobylky,a takoví jsme se i jim zdáli.

\chapter{14}

\par 1 Tehdy pozdvihše se všecko množství, kriceli, a plakal lid v tu noc.
\par 2 A reptali proti Mojžíšovi i proti Aronovi všickni synové Izraelští, a reklo k nim všecko množství: Ó bychom byli zemreli v zemi Egyptské, aneb na této poušti, ó bychom byli zemreli!
\par 3 A proc Hospodin vede nás do zeme té, abychom padli od mece, ženy naše i dítky naše aby byly v loupež? Není-liž nám lépe navrátiti se zase do Egypta?
\par 4 I rekli jeden druhému: Ustanovme sobe vudci, a navratme se do Egypta.
\par 5 Tedy padli Mojžíš a Aron na tvári své prede vším množstvím shromáždení synu Izraelských.
\par 6 Jozue pak, syn Nun, a Kálef, syn Jefonuv, z tech, kteríž byli shlédli zemi, roztrhli roucha svá,
\par 7 A mluvili ke všemu množství synu Izraelských, rkouce: Zeme, kterouž jsme prošli a vyšetrili, jest zeme velmi velice dobrá.
\par 8 Bude-li Hospodin laskav na nás, uvedet nás do zeme té, a dá ji nám, a to zemi takovou, kteráž oplývá mlékem a strdí.
\par 9 Toliko nepozdvihujte se proti Hospodinu, ani se bojte lidu zeme té, nebo jako chléb náš jsou. Odešlate od nich ochrana jejich, ale s námi jest Hospodin; nebojtež se jich.
\par 10 Tedy mluvilo všecko množství, aby je kamením uházeli, ale sláva Hospodinova ukázala se nad stánkem úmluvy všechnem sy\par Izraelským.
\par 11 I rekl Hospodin Mojžíšovi: I dokavadž popouzeti mne bude lid ten? A dokud nebudou mi veriti pro tak mnohá znamení, kteráž jsem cinil u prostred nich?
\par 12 Raním jej morem a rozženu jej, tebe pak uciním v národ veliký a silnejší, nežli jest tento.
\par 13 I rekl Mojžíš Hospodinu: Ale uslyšít to Egyptští, z jejichž prostredku vyvedl jsi lid tento v síle své,
\par 14 A reknou s obyvateli zeme té: (nebo slyšeli, že jsi ty, ó Hospodine, byl u prostred lidu tohoto, a že jsi okem v oko spatrován byl, ó Hospodine, a oblak tvuj stál nad nimi, a v sloupe oblakovém predcházel jsi je ve dne, a v sloupe ohnivém v noci),
\par 15 Když tedy zmoríš lid ten, všecky až do jednoho, mluviti budou národové, kteríž slyšeli povest o tobe, ríkajíce:
\par 16 Proto že nemohl Hospodin uvésti lidu toho do zeme, kterouž jim s prísahou zaslíbil, zmordoval je na poušti.
\par 17 Nyní tedy, prosím, nechat jest zvelebena moc Páne, jakož jsi mluvil, rka:
\par 18 Hospodin dlouhocekající a hojný v milosrdenství, odpouštející nepravost a prestoupení, kterýž však z vinného neciní nevinného, ale navštevuje nepravost otcu na synech do tretího i ctvrtého pokolení.
\par 19 Odpust, prosím, nepravost lidu tohoto podlé velikého milosrdenství svého, tak jako jsi odpouštel lidu tomuto, jakž vyšel z Egypta až dosavad.
\par 20 I rekl Hospodin: Odpustil jsem vedlé slova tvého.
\par 21 A však živ jsem já, a sláva má naplnuje všecku zemi,
\par 22 Že všickni ti, kteríž videli slávu mou a znamení má, kteráž jsem cinil v Egypte a na poušti této, a kteríž pokoušeli mne již desetkrát, aniž uposlechli hlasu mého,
\par 23 Neuzrí zeme té, kterouž jsem s prísahou zaslíbil otcum jejich, aniž jí kdo z tech, kteríž mne popouzeli, uhlédá.
\par 24 Ale služebníka svého Kálefa, (nebo v nem byl jiný duch, a cele následoval mne), uvedu jej do zeme, do kteréž chodil, a síme jeho dedicne obdrží ji.
\par 25 Ale ponevadž Amalechitský a Kananejský bydlí v tom údolí, obratte se zase zítra, a berte se na poušt cestou k mori Rudému.
\par 26 Mluvil také Hospodin k Mojžíšovi a Aronovi, rka:
\par 27 Až dokud snášeti budu množství toto zlé, kteréž repce proti mne? Až dokud reptání synu Izraelských, kteríž repcí proti mne, slyšeti budu?
\par 28 Rci jim: Živ jsem já, praví Hospodin, žet vám uciním tak, jakž jste mluvili v uši mé:
\par 29 Na poušti této padnou mrtvá tela vaše, a všickni, kteríž jste secteni, podlé všeho poctu vašeho od majících let dvadceti a výše, kteríž jste reptali proti mne,
\par 30 Nevejdete, pravím, vy do zeme, o kteréž, zdvihna ruku svou, prisáhl jsem, že vás osadím v ní, jediné Kálef, syn Jefonuv, a Jozue, syn Nun.
\par 31 Ale dítky vaše malé, o nichž jste rekli, že v loupež budou, ty uvedu, aby užívali zeme té, kterouž jste vy pohrdli.
\par 32 Tela pak vaše mrtvá padnou na poušti této.
\par 33 A synové vaši budou tuláci na poušti této ctyridceti let, a ponesou pokutu smilství vašeho, až do konce vyhynou tela vaše na poušti.
\par 34 Podlé poctu dnu, v nichž jste procházeli zemi tu, totiž ctyridceti dnu, jeden každý den za rok pocítaje, ponesete nepravosti své ctyridceti let, a poznáte pomstu svého odtržení se ode mne.
\par 35 Já Hospodin mluvil jsem, že to uciním všemu tomuto množství zlému, kteréž se zrotilo proti mne; na poušti této do konce zhynou, a tu zemrou.
\par 36 Muži pak ti, kteréž poslal byl Mojžíš k shlédnutí zeme té, a kteríž, navrátivše se, uvedli v reptání proti nemu všecko množství to, hanením ošklivíce zemi,
\par 37 Ti, rku, muži, kteríž haneli zemi, ranou težkou zemreli pred Hospodinem.
\par 38 Jozue pak, syn Nun, a Kálef, syn Jefonuv, živi zustali z mužu tech, kteríž chodili k vyšetrení zeme.
\par 39 Tedy mluvil Mojžíš slova ta ke všechnem sy\par Izraelským; i plakal lid velmi.
\par 40 A vstavše ráno, vstoupili na vrch hory, a rekli: Aj, my hotovi jsme, abychom šli k místu, o nemž mluvil Hospodin; nebo jsme zhrešili.
\par 41 Jimž rekl Mojžíš: Proc jest to, že vy prestupujete prikázaní Hospodinovo? ješto vám to na dobré nevyjde.
\par 42 Nevstupujte, nebo Hospodin není u prostred vás, abyste nebyli poraženi od neprátel vašich.
\par 43 Amalechitský zajisté a Kananejský jest tu pred vámi, a padnete od mece, proto že jste se odvrátili od následování Hospodina, aniž také Hospodin bude s vámi.
\par 44 Oni pak predce usilovali vstoupiti na vrch hory, ale truhla smlouvy Hospodinovy a Mojžíš nevycházeli z stanu.
\par 45 Tedy sstoupili Amalechitský a Kananejský, kteríž bydlili na tech horách, a porazili je, a potírali je až do Horma.

\chapter{15}

\par 1 Mluvil opet Hospodin k Mojžíšovi, rka:
\par 2 Mluv k sy\par Izraelským a rci jim: Když vejdete do zeme prebývání vašich, kterouž já dám vám,
\par 3 A budete chtíti obetovati obet ohnivou Hospodinu v zápal, aneb obet bud slíbenou, bud dobrovolnou, aneb pri slavnostech vašich, abyste ucinili vuni spokojující Hospodina, budto z skotu, aneb z drobného dobytku:
\par 4 Tedy kdož by koli obetoval dar svuj Hospodinu, obetuj obet suchou, desátý díl efi mouky belné, zadelané olejem, jehož bude ctvrtý díl míry hin.
\par 5 A vína v obet mokrou ctvrtinku hin obetovati budeš pri zápalu, aneb pri obeti vítezné k jednomu každému beránku.
\par 6 Pri beranu pak obetovati budeš obet suchou, mouky belné dve desetiny, zadelané olejem, tretinkou míry hin.
\par 7 Vína také k obeti mokré tretí díl míry hin obetovati budeš u vuni spokojující Hospodina.
\par 8 Jestliže pak obetovati budeš volka v obet zápalnou, aneb v obet k splnení slibu, aneb v obet pokojnou Hospodinu:
\par 9 Tedy budeš obetovati spolu s volkem obet suchou, tri desetiny mouky belné, zadelané olejem, jehož by byla pulka hin.
\par 10 A vína k obeti mokré obetovati budeš pulku hin. Ta jest obet ohnivá vune spokojující Hospodina.
\par 11 Tak udeláte s každým volem i s každým skopcem, i s dobytcetem bud ono z ovec neb z koz.
\par 12 Podlé toho, jakž mnoho jich obetovati budete, tak se zachováte pri jednom každém z nich.
\par 13 Všeliký obyvatel tak bude to vykonávati, aby obetoval obet ohnivou vune spokojující Hospodina.
\par 14 Takž bude-li u vás i príchozí, aneb kdokoli mezi vámi v pronárodech vašich, a bude obetovati obet ohnivou vune spokojující Hospodina, jakž vy se pri tom chováte, tak i on chovati se bude.
\par 15 V shromáždení tomto ustanovení jednostejné bud vám i príchozímu, kterýž by u vás byl, ustanovení vecné v pronárodech vašich; jakož vy, tak príchozí bude pred Hospodinem.
\par 16 Zákon jednostejný a porádek jednostejný bude vám i prichozímu kterýž by u vás byl.
\par 17 Mluvil také Hospodin k Mojžíšovi, rka:
\par 18 Mluv k sy\par Izraelským a rci jim: Když prijdete do zeme, do kteréž já uvedu vás,
\par 19 Tedy, když pocnete jísti chléb zeme, obetovati budete obet vzhuru pozdvižení Hospodinu.
\par 20 Z prvotin testa vašeho kolác obetovati budete v obet vzhuru pozdvižení; tak jako obet z humna, obetovati budete ji.
\par 21 Z prvotin testa vašeho dávati budete Hospodinu obet vzhuru pozdvižení, vy i potomci vaši.
\par 22 A když byste pozbloudili, a neucinili všech prikázaní tech, kteráž mluvil Hospodin k Mojžíšovi,
\par 23 Totiž všech vecí, kteréž prikázal Hospodin vám skrze Mojžíše, od toho dne, v kterémž prikázaní vydal Hospodin, i potom v pronárodech vašich:
\par 24 Tedy jestliže z nedopatrení všeho shromáždení to stalo se, obetovati bude všecko množství volka mladého jednoho v obet zápalnou, k vuni spokojující Hospodina, též obet suchou, a obet mokrou pri nem podlé porádku, a kozla jednoho v obet za hrích.
\par 25 I ocistí knez všecko množství synu Izraelských, a odpušteno jim bude; nebo poblouzení jest. A oni také obetovati budou obet svou, obet ohnivou Hospodinu, a obet za hrích svuj pred Hospodinem za poblouzení své.
\par 26 I bude odpušteno všemu shromáždení synu Izraelských i príchozímu, kterýž jest pohostinu u prostred nich; nebo všeho lidu poblouzení jest.
\par 27 Jestliže by pak clovek jeden zhrešil z poblouzení, bude obetovati kozu rocní v obet za hrích.
\par 28 I ocistí knez duši pobloudilou, kteráž zhrešila poblouzením pred Hospodinem; ocistí ji, a budet jí odpušteno.
\par 29 Domácímu mezi syny Izraelskými i príchozímu, kterýž pohostinu mezi nimi jest, zákon tento jednostejný bude vám, když by kdo zhrešil z poblouzení.
\par 30 Clovek pak, kterýž by z pychu svévolne zhrešil, tak doma zrozený jako príchozí, takový potupil velice Hospodina; protož vyhlazen bude z prostredku lidu svého.
\par 31 Nebo slovem Hospodinovým pohrdl, a prikázaní jeho za nic sobe položil; protož konecne vyhlazen bude clovek ten, a nepravost jeho zustane na nem.
\par 32 Stalo se pak, když synové Izraelští byli na poušti, že nalezli jednoho, an sbírá dríví v den sobotní.
\par 33 A ti, kteríž ho nalezli sbírajícího dríví, privedli jej k Mojžíšovi a k Aronovi i ke všemu množství.
\par 34 I dali jej do vezení; nebo ješte nebylo jim oznámeno, co by s ním melo cineno býti.
\par 35 I rekl Hospodin Mojžíšovi: Smrtí at umre clovek ten; nechat ho bez milosti ukamenuje všecko množství vne za stany.
\par 36 A protož vyvedli jej všecko množství ven za stany, a uházeli ho kamením, až umrel, jakož rozkázal Hospodin.
\par 37 I mluvil Hospodin k Mojžíšovi, rka:
\par 38 Mluv k sy\par Izraelským a rci jim, at sobe delají trepení široké na podolcích odevu svých všickni rodové jejich, a at dávají nad trepením šnurku modrou.
\par 39 A to budete míti za premování, na nežto hledíce, rozpomínati se budete na všecka prikázaní Hospodinova, abyste je cinili, a nepustíte se po žádosti srdce svého, a po ocích svých, jichžto následujíce, smilnili byste,
\par 40 Ale abyste pamatovali a cinili všecka prikázaní má, a byli svatí Bohu svému.
\par 41 Já jsem Hospodin Buh váš, kterýž jsem vyvedl vás z zeme Egyptské, abych vám byl za Boha: Já Hospodin Buh váš.

\chapter{16}

\par 1 Chóre pak syn Izaruv, syna Kahat z pokolení Léví, vytrhl se z jiných, tolikéž Dátan a Abiron, synové Eliabovi, také Hon, syn Feletuv, z synu Rubenových,
\par 2 A povstali proti Mojžíšovi, i jiných mužu z synu Izraelských dve ste a padesáte, knížata shromáždení, kteríž svoláváni byli do rady, muži slovoutní.
\par 3 A sebravše se proti Mojžíšovi a proti Aronovi, rekli jim: Prílište to již na vás; všecko zajisté množství toto, všickni tito svatí jsou, a u prostred nich jest Hospodin. Procež se tedy vyzdvihujete nad shromáždením Hospodinovým?
\par 4 To když uslyšel Mojžíš, padl na tvár svou,
\par 5 A mluvil k Chóre a ke vší rote jeho, rka: Ráno ukáže Hospodin, kdo jsou jeho, a kdo jest svatý, i kdo pred nej predstupovati má; nebo kohožkoli vyvolil, tomu rozkáže pristoupiti k sobe.
\par 6 Toto ucinte: Vezmete sobe kadidlnice, ty Chóre i všecko shromáždení tvé,
\par 7 A nakladte do nich uhlí, a vložte na ne kadidla pred Hospodinem zítra; i stane se, že kohožkoli vyvolí Hospodin, ten bude svatý. Prílište to již na vás, synové Léví.
\par 8 I rekl Mojžíš k Chóre: Slyšte, prosím, synové Léví.
\par 9 Zdaliž málo vám to jest, že vás oddelil Buh Izraelský ode všeho množství Izraelského, a rozkázal vám pristupovati k sobe, abyste vykonávali službu príbytku Hospodinova, a abyste stáli pred shromáždením, a sloužili jim,
\par 10 A vzal te sobe, a všecky bratrí tvé syny Léví s tebou, a že ješte pres to i knežství hledáte?
\par 11 Protož vez, že ty a všickni tvoji jste ti, kteríž se rotíte proti Hospodinu; nebo Aron co jest, že jste reptali proti nemu?
\par 12 Tedy poslal Mojžíš, aby zavolali Dátana a Abirona, synu Eliabových. Kteríž odpovedeli: Nepujdem.
\par 13 Cožt se ješte málo zdá, že jsi vyvedl nás z zeme oplývající mlékem a strdí, abys nás zmoril na poušti, že také chceš i panovati nad námi, a rozkazovati nám?
\par 14 A ješte jsi nás neuvedl do zeme oplývající mlékem a strdí, aniž jsi nám dal v dedictví rolí a vinic. Zdali oci mužum temto vyloupiti chceš? Nepujdeme.
\par 15 I rozhneval se Mojžíš velmi, a rekl k Hospodinu: Nepatriž na obeti jejich. Ani jednoho osla od nich jsem nevzal, aniž jsem komu z nich co zlého ucinil.
\par 16 Potom rekl Mojžíš k Chóre: Ty a všickni tvoji, postavte se zítra pred Hospodinem, ty i oni, též i Aron.
\par 17 A vezmouce jeden každý kadidlnici svou, dáte do nich kadidla, a postavíte se pred Hospodinem, jeden každý s kadidlnicí svou; dve ste a padesáte kadidlnic bude, ty také i Aron, každý s kadidlnicí svou.
\par 18 Tedy vzal jeden každý kadidlnici svou, a nabravše do nich uhlí, vložili na ne i kadidla, a stáli u dverí stánku úmluvy, i Mojžíši Aron.
\par 19 Chóre pak již byl sebral proti nim všecko množství ke dverím stánku úmluvy; i ukázala se sláva Hospodinova všemu množství.
\par 20 I mluvil Hospodin k Mojžíšovi a k Aronovi, rka:
\par 21 Oddelte se z prostredku množství tohoto, at je v okamžení zahladím.
\par 22 Kterížto padše na tvári své, rekli: Bože silný, Bože duchu i všelikého tela, jediný tento clovek zhrešil, a což na všecko shromáždení hnevati se budeš?
\par 23 Tedy mluvil Hospodin k Mojžíšovi, rka:
\par 24 Mluv k množství a rci: Odstupte od príbytku Chóre, Dátana a Abirona.
\par 25 A vstav Mojžíš, šel k Dátanovi a Abironovi; i šli za ním starší Izraelští.
\par 26 A mluvil k množství, rka: Odstupte, prosím, od stanu bezbožných mužu techto, aniž se ceho dotýkejte, což jejich jest, abyste nebyli zachváceni ve všech hríších jejich.
\par 27 I odstoupili se všech stran od príbytku Chóre, Dátana a Abirona; ale Dátan a Abiron vyšedše, stáli u dverí stanu svých, i ženy jejich a synové jejich, i maliccí jejich.
\par 28 Tedy rekl Mojžíš: Po tomto poznáte, že Hospodin poslal mne, abych cinil všecky skutky tyto, a že nic o své ujme neciním:
\par 29 Jestliže tak jako jiní lidé mrou, zemrou i tito, a navštívením obecným všechnem lidem jestliže navštíveni budou, neposlal mne Hospodin.
\par 30 Paklit neco nového uciní Hospodin, a zeme, otevra ústa svá, požre je se vším, což mají, a sstoupí-li za živa do pekla, tedy poznáte, že jsou popouzeli muži ti Hospodina.
\par 31 I stalo se, když prestal mluviti slov tech, že rozstoupila se zeme pod nimi.
\par 32 A otevrevši zeme ústa svá, požrela je i domy jejich i všecky lidi, kteríž byli s Chóre, i všecken statek jejich.
\par 33 A tak sstoupili oni se vším, což meli, za živa do pekla, a prikryla je zeme; i zahynuli z prostredku shromáždení.
\par 34 Všickni pak Izraelští, kteríž byli vukol nich, utíkali, slyšíce krik jejich; nebo rekli: Utecme, aby i nás nesehltila zeme.
\par 35 Vyšel také ohen od Hospodina, a spálil tech dve ste a padesáte mužu, kteríž kadili.
\par 36 Tedy mluvil Hospodin k Mojžíšovi, rka:
\par 37 Rci Eleazarovi, synu Arona kneze, at sbére kadidlnice z toho spálenište, a uhlí z nich pryc rozmece, nebo posveceny jsou,
\par 38 Kadidlnice totiž tech, kteríž proti svým dušem hrešili, a at je rozkuje na plechy k obložení oltáre; nebo kadili jimi pred Hospodinem, protož posveceny jsou, a budou na znamení sy\par Izraelským.
\par 39 I sebral Eleazar knez kadidlnice medené, jimiž kadili ti, kteríž spáleni jsou, a rozkovali je k obložení oltáre,
\par 40 Pro budoucí pamet sy\par Izraelským, aby nepristupoval žádný jiný, kdož by nebyl z rodu Aronova, k kadení pred Hospodinem, aby se mu nestalo jako Chóre a jako rote jeho, jakož byl mluvil jemu Hospodin skrze Mojžíše.
\par 41 Nazejtrí pak reptalo všecko množství synu Izraelských na Mojžíše a na Arona, rkouce: Vy jste prícinou smrti lidu Hospodinova.
\par 42 I stalo se, když se opet sbíral lid proti Mojžíšovi a proti Aronovi, že se ohlédli k stánku úmluvy, a aj, prikryl jej oblak, a ukázala se sláva Hospodinova.
\par 43 Prišel také Mojžíš s Aronem pred stánek úmluvy.
\par 44 I mluvil Hospodin k Mojžíšovi, rka:
\par 45 Vyjdete z prostredku množství tohoto, a zahladím je v okamžení. I padli na tvári své.
\par 46 Tedy rekl Mojžíš Aronovi: Vezma kadidlnici, dej do ní uhlí z oltáre, a vlož kadidla, a bež rychle k množství a ocist je; nebo vyšla prchlivost od tvári Hospodinovy, a již rána se zacala.
\par 47 I vzav Aron kadidlnici, jakž rozkázal Mojžíš, bežel do prostred shromáždení, (a aj, rána již se byla zacala v lidu,) a zakadiv, ocistil lid.
\par 48 A stál mezi mrtvými a živými; i zastavena jest rána.
\par 49 Bylo pak tech, kteríž od té rány zemreli, ctrnácte tisícu a sedm set, krome tech, jenž zemreli prícinou Chóre.
\par 50 I navrátil se Aron k Mojžíšovi, ke dverím stánku úmluvy, když ta rána pretržena byla.

\chapter{17}

\par 1 Tedy mluvil Hospodin k Mojžíšovi, rka:
\par 2 Mluv k sy\par Izraelským, a vezmi od nich po jednom prutu z každého domu otcu, totiž ode všech knížat jejich, vedlé domu otcu jich dvanácte prutu, a jednoho každého jméno napíšeš na prutu jeho.
\par 3 Jméno pak Aronovo napíšeš na prutu Léví; nebo jeden každý prut bude místo jednoho predního z domu otcu jejich.
\par 4 I necháš jich v stánku úmluvy pred svedectvím, kdež pricházím k vám.
\par 5 I stane se, že kohož vyvolím, toho prut zkvetne; a tak spokojím a svedu s sebe reptání synu Izraelských, kterýmž repcí na vás.
\par 6 Ty veci když mluvil Mojžíš k sy\par Izraelským, dali jemu všecka jejich knížata pruty své, jedno každé kníže prut z domu otce svého, totiž prutu dvanácte, prut pak Aronuv byl též mezi pruty jejich.
\par 7 I zanechal Mojžíš prutu pred Hospodinem v stánku svedectví.
\par 8 Nazejtrí pak prišel Mojžíš do stánku svedectví, a aj, vyrostl prut Aronuv z domu Léví, a vypustiv z sebe pupence, zkvetl a vydal mandly zralé.
\par 9 I vynesl Mojžíš všecky ty pruty od tvári Hospodinovy ke všechnem sy\par Izraelským; a uzrevše je, vzali jeden každý prut svuj.
\par 10 Rekl pak Hospodin Mojžíšovi: Dones zase prut Aronuv pred svedectví, aby chován byl na znamení proti buricum zpurným, a tak prítrž uciníš reptání jejich na mne, aby nezemreli.
\par 11 I ucinil tak Mojžíš; jakž byl prikázal jemu Hospodin, tak ucinil.
\par 12 Tedy mluvili synové Izraelští k Mojžíšovi, rkouce: Hle, již mreme, mizíme a všickni my hyneme.
\par 13 Kdožkoli blízko pristoupá k príbytku Hospodinovu, umírá. Všickni-liž smrtí zhyneme?

\chapter{18}

\par 1 A protož rekl Hospodin Aronovi: Ty a synové tvoji i dum otce tvého s tebou ponesete nepravost svatyne; ty také i synové tvoji s tebou ponesete nepravost knežství svého.
\par 2 Bratrí také své, pokolení Léví, celed otce svého pripoj k sobe, at jsou pri tobe a posluhují tobe; ty pak a synové tvoji s tebou pred stánkem svedectví sloužiti budete.
\par 3 A pilne ostríhati budou rozkázaní tvého, bedlivi jsouce pri všem stánku; a však at k nádobám svatyne a k oltári nepristupují, aby nezemreli i oni i vy.
\par 4 A prídržeti se budou tebe, pilne ostríhajíce stánku úmluvy pri všech službách jeho; a cizí žádný nepristupuj k vám.
\par 5 Protož pilne ostríhejte svatyne i služby oltáre, at neprichází více zurivá prchlivost na syny Izraelské,
\par 6 Ponevadž jsem já vybral bratrí vaše Levíty z synu Izraelských vám, za dar dané Hospodinu, aby konali službu pri stánku úmluvy.
\par 7 Ty pak a synové tvoji s tebou ostríhati budete knežství svého pri všech vecech prináležejících k oltári, a kteréž jsou za oponou, a sloužiti budete. Úrad knežství vašeho darem jsem vám dal, protož, pristoupil-li by kdo cizí, umre.
\par 8 Mluvil také Hospodin k Aronovi: Aj, já dal jsem tobe k ostríhání obeti své, kteréž se vzhuru pozdvihují; všecko také, což se posvecuje od synu Izraelských, tobe jsem dal pro pomazání i sy\par tvým ustanovením vecným.
\par 9 Toto bude tvé z vecí posvecených, kteréž nepricházejí na ohen: Všeliká obet jejich, bud suchá obet jejich, neb obet za hrích jejich, aneb obet za provinení jejich, kteréž mi dávati budou, svatosvaté tobe bude to i sy\par tvým.
\par 10 V svatyni budeš je jísti, každý pohlaví mužského bude je jísti; svaté bude tobe.
\par 11 Tvá tedy bude obet daru jejich, kteráž vzhuru pozdvižena bývá; každou také obet synu Izraelských, kteráž sem i tam obracína bývá, tobe jsem dal a sy\par tvým, i dcerám tvým s tebou ustanovením vecným. Každý, kdož jest cistý v dome tvém, bude je jísti.
\par 12 Všecko, což predního jest z oleje, a což nejlepšího jest vína a obilé, prvotiny tech vecí, kteréž dají Hospodinu, dal jsem tobe.
\par 13 Prvotiny všech vecí, kteréž rostly v zemi jejich, ty prinesou Hospodinu, a tvé budou. Každý, kdož jest cistý v dome tvém, jísti je bude.
\par 14 Všecko oddané Bohu v Izraeli tvé bude.
\par 15 Cožkoli otvírá život všelikého tela, kteréž obetováno bývá Hospodinu, tak z lidí jako z hovad, tvé bude; prvorozené však z lidí, vyplaceno bude, prvorozené také z necistých hovad vyplatiti kážeš.
\par 16 Výplata pak jeho, když mu již jeden mesíc bude, vedlé ceny tvé bude peti loty stríbra podlé lotu svatyne; dvadceti penez váží lot.
\par 17 Ale prvorozeného z volu, neb z ovcí, neb z koz, nedáš vyplatiti; nebo posveceny jsou. Krev jejich vykropíš na oltár, a tuk jejich zapálíš, aby byl obetí ohnivou vune spokojující Hospodina.
\par 18 Maso pak jejich tvé bude; jako hrudí z obeti sem i tam obracení, a jako plece pravé tvé bude.
\par 19 Každou obet vzhuru pozdvižení z vecí posvecených, kteréž prinášejí synové Izraelští Hospodinu, dal jsem tobe, a sy\par tvým, i dcerám tvým s tebou ustanovením vecným, smlouvou trvánlivou a vecnou pred Hospodinem, tobe i semeni tvému s tebou.
\par 20 Mluvil také Hospodin k Aronovi: V zemi jejich, dedictví ani dílu svého mezi nimi nebudeš míti; já jsem díl tvuj, a dedictví tvé u prostred synu Izraelských.
\par 21 Sy\par pak Léví, aj, dal jsem všecky desátky v Izraeli za dedictví, za službu jejich, kterouž vykonávají, sloužíce pri stánku úmluvy.
\par 22 A synové Izraelští nechat nepristupují k stánku úmluvy, aby nenesli hríchu a nezemreli.
\par 23 Ale sami Levítové konati budou službu pri stánku úmluvy, a sami ponesou nepravost svou. Ustanovení vecné v pronárodech vašich to bude, aby dedictví nemívali mezi syny Izraelskými.
\par 24 Nebo desátky synu Izraelských, kteréž prinášeti budou Hospodinu v obet vzhuru pozdvižení, dal jsem Levítum za dedictví; protož jsem o nich rekl: U prostred synu Izraelských nebudou míti dedictví.
\par 25 I mluvil Hospodin k Mojžíšovi, rka:
\par 26 Mluv k Levítum a rci jim: Když vezmete desátky od synu Izraelských, kteréž jsem vám dal od nich za dedictví vaše, tedy prinesete z nich obet vzhuru pozdvižení Hospodinu, desátky z desátku.
\par 27 A poctena vám bude ta obet vaše jako obilí z humna, a jako víno od presu.
\par 28 Tak vy také prinesete obet vzhuru pozdvižení Hospodinu ze všech desátku svých, kteréž vezmete od synu Izraelských, a dáte z nich obet Hospodinovu Aronovi knezi.
\par 29 Ze všech daru svých prinesete každou obet vzhuru pozdvižení Hospodinu, díl posvecený všeho, což nejlepšího jest.
\par 30 Díš jim také: Když obetovati budete z toho, což nejlepšího jest, pocteno bude Levítum jako úrody z humna, a jako úrody od presu.
\par 31 Jísti pak budete ty desátky na všelikém míste vy i celed vaše; nebo mzda vaše jest za službu vaši pri stánku úmluvy.
\par 32 A neponesete pro to hríchu, když obetovati budete to, což nejlepšího jest; a tak nepoškvrníte vecí posvecených od synu Izraelských, a nezemrete.

\chapter{19}

\par 1 I mluvil Hospodin k Mojžíšovi a k Aronovi, rka:
\par 2 Toto jest ustanovení zákona, kteréž prikázal Hospodin, rka: Mluv k sy\par Izraelským, at vezmouce, privedou k tobe jalovici cervenou bez vady, na níž by nebylo poškvrny, a na kterouž by ješte jho nebylo kladeno.
\par 3 A dáte ji Eleazarovi knezi, kterýž vyvede ji ven z stanu, a rozkáže ji zabiti pred sebou.
\par 4 A vezma Eleazar knez krve její na prst svuj, kropiti bude jí naproti stánku úmluvy sedmkrát.
\par 5 Potom káže tu jalovici spáliti pred ocima svýma; kuži její, i maso její, i krev její s lejny jejími dá spáliti.
\par 6 A vezma knez dreva cedrového, yzopu a cervce dvakrát barveného, uvrže to vše do ohne, v kterémž ta jalovice horí.
\par 7 I zpére knez roucha svá, a telo své umyje vodou; potom vejde do stanu, a bude necistý až do vecera.
\par 8 Takž také i ten, kterýž ji pálil, zpére roucha svá u vode, a telo své obmyje vodou, a necistý bude až do vecera.
\par 9 Popel pak té jalovice spálené smete muž cistý, a vysype jej vne za stany na míste cistém, aby byl všechnem sy\par Izraelským chován k vode ocištení, kteráž bude k ocištení za hrích.
\par 10 A ten, kdož by smetl popel té jalovice, zpére roucha svá a necistý bude až do vecera. A budou to míti synové Izraelští i príchozí, kterýž jest pohostinu u prostred nich, za ustanovení vecné.
\par 11 Kdo by se dotkl tela kteréhokoli mrtvého cloveka, necistý bude za sedm dní.
\par 12 Takový ocištovati se bude tou vodou dne tretího a dne sedmého, i bude cistý; neocistí-li se pak dne tretího a dne sedmého, nebudet cistý.
\par 13 Kdo by koli, dotkna se tela mrtvého cloveka, kterýž umrel, neocistil se, príbytku Hospodinova poškvrnil. Protož vyhlazena bude duše ta z Izraele, nebo vodou ocištování není pokropen; necistýt bude, a necistota jeho zustává na nem.
\par 14 Tento jest také zákon, kdyby clovek umrel v stanu: Kdo by koli všel do toho stanu, a kdo by koli byl v stanu, necistý bude za sedm dní.
\par 15 Tolikéž všeliká nádoba odkrytá, kteráž by nemela svrchu prikryvadla na sobe, necistá bude.
\par 16 Tak kdož by se koli dotkl na poli bud mecem zabitého, aneb mrtvého, budto kosti cloveka, aneb hrobu, necistý bude za sedm dní.
\par 17 A vezmou pro necistého popela z spálené za hrích, a nalejí na nej vody živé do nádoby.
\par 18 Potom vezme clovek cistý yzopu, a omocí jej v té vode, a pokropí stanu, i všeho nádobí i lidí, kteríž by tu byli, tolikéž toho, kterýž dotkl se kosti, aneb zabitého, aneb mrtvého, aneb hrobu.
\par 19 Pokropí tedy cistý necistého v den tretí a v den sedmý; a když jej ocistí dne sedmého, zpére šaty své, a umyje se vodou, a cistý bude u vecer.
\par 20 Jestliže by pak kdo necistý jsa, neocistil se, vyhlazena bude duše ta z prostredku shromáždení; nebo svatyne Hospodinovy poškvrnil, vodou ocištování nejsa pokropen; necistý jest.
\par 21 I bude jim to za ustanovení vecné. Kdož by pak tou vodou ocištování kropil, zpére roucha svá, a kdož by se dotkl vody ocištování, necistý bude až do vecera.
\par 22 Cehokoli dotkl by se necistý, necisté bude; clovek také, kterýž by se dotkl toho, necistý bude až do vecera.

\chapter{20}

\par 1 I pritáhlo všecko množství synu Izraelských na poušt Tsin, mesíce prvního; i pozustal lid v Kádes, kdež umrela Maria, a tu jest pochována.
\par 2 A když množství to nemelo vody, sešli se proti Mojžíšovi, a proti Aronovi.
\par 3 I domlouval se lid na Mojžíše, a mluvili, rkouce: Ó kdybychom i my byli zemreli, když zemreli bratrí naši pred Hospodinem!
\par 4 Proc jste jen uvedli shromáždení Hospodinovo na poušt tuto, abychom zde pomreli i my i dobytek náš?
\par 5 A proc jste nás vyvedli z Egypta, abyste uvedli nás na toto zlé místo, na nemž se nerodí ani obilí, ani fíku, ani hroznu, ani jablek zrnatých, na kterémž ani vody ku pití není?
\par 6 Tedy odšel Mojžíš s Aronem od tvári shromáždení ke dverím stánku úmluvy, a padli na tvári své; i ukázala se sláva Hospodinova nad nimi.
\par 7 A mluvil Hospodin k Mojžíšovi, rka:
\par 8 Vezmi hul, a shromáždíce všecko množství, ty i Aron bratr tvuj, mluvte k skále této pred ocima jejich, a vydá vodu svou. I vyvedeš jim vodu z skály, a dáš nápoj všemu množství i dobytku jejich.
\par 9 Tedy vzal Mojžíš hul pred tvárí Hospodinovou, jakž rozkázal jemu.
\par 10 I svolali Mojžíš s Aronem všecko množství pred skálu, a rekl jim: Slyštež nyní, ó reptáci: Zdali z skály této vyvedeme vám vodu?
\par 11 I pozdvihl Mojžíš ruky své, a uderil v skálu holí svou po dvakrát; i vyšly vody hojné, a pilo všecko množství i dobytek jejich.
\par 12 Potom rekl Hospodin Mojžíšovi a Aronovi: Že jste mi neverili, abyste posvetili mne pred ocima synu Izraelských, proto neuvedete shromáždení tohoto do zeme, kterouž jsem jim dal.
\par 13 Tot jsou ty vody sváru, o kteréž svarili se synové Izraelští s Hospodinem, a posvecen jest v nich.
\par 14 I poslal Mojžíš posly z Kádes k králi Edom, aby rekli: Totot vzkazuje bratr tvuj Izrael: Ty víš o všech težkostech, kteréž prišly na nás,
\par 15 Že sstoupili otcové naši do Egypta, a bydlili jsme tam za mnoho let. Egyptští pak ssužovali nás i otce naše.
\par 16 A volali jsme k Hospodinu, kterýž uslyšel hlas náš, a poslav andela, vyvedl nás z Egypta, a aj, již jsme v Kádes meste, pri pomezí tvém.
\par 17 Necht, prosím, projdeme skrze zemi tvou. Nepujdeme pres rolí, ani pres vinice, aniž píti budeme vody z cí studnice; cestou královskou pujdeme a neuchýlíme se na pravo ani na levo, dokavadž neprejdeme mezí tvých.
\par 18 Jemužto odpovedel Edom: Nechod skrze mou zemi, abych s mecem nevyšel v cestu tobe.
\par 19 I rekli mu synové Izraelští: Obecnou silnicí pujdeme, a jestliže vody tvé napili bychom se, bud my neb dobytek náš, zaplatíme ji; nic jiného nežádáme, toliko peší abychom prošli.
\par 20 Odpovedel: Neprojdeš. A vytáhl proti nim Edom s množstvím lidu a s silou velikou.
\par 21 Když tedy nedopustil Edom Izraelovi, aby prešel meze jeho, uchýlil se Izrael od neho.
\par 22 A hnuvše se synové Izraelští i všecko množství jejich z Kádes, prišli na horu recenou Hor.
\par 23 I mluvil Hospodin k Mojžíšovi a Aronovi na hore Hor, pri pomezí zeme Edom, rka:
\par 24 Pripojen bude Aron k lidu svému; nebo nevejde do zeme, kterouž jsem dal sy\par Izraelským, proto že jste odporni byli reci mé pri vodách sváru.
\par 25 Pojmi Arona a Eleazara syna jeho, a uvedeš je na horu Hor.
\par 26 A svleceš Arona z roucha jeho, a obleceš v ne Eleazara syna jeho; nebo Aron pripojen bude k lidu svému, a tam umre.
\par 27 I ucinil Mojžíš, jakž rozkázal Hospodin, a vstoupili na horu Hor pred ocima všeho množství.
\par 28 A svlékl Mojžíš Arona z roucha jeho, a oblékl v ne Eleazara syna jeho. I umrel tam Aron na pahrbku hory, Mojžíš pak a Eleazar sstoupili s hory.
\par 29 Vidouce pak všecko množství, že umrel Aron, plakali ho za tridceti dní všecken dum Izraelský.

\chapter{21}

\par 1 To když uslyšel Kananejský král v Arad, kterýž bydlil na poledne, že by táhl Izrael tou cestou, kterouž byli špehéri šli, bojoval s ním, a zajal jich množství.
\par 2 Tedy Izrael ucinil slib Hospodinu, rka: Jestliže dáš lid tento v ruce mé, do gruntu zkazím mesta jejich.
\par 3 I uslyšel Hospodin hlas Izraele a dal mu Kananejské, kterýžto do gruntu zkazil je i mesta jejich, a nazval jméno toho místa Horma.
\par 4 Potom hnuli se s hory Hor cestou k mori Rudému, aby obešli zemi Idumejskou; i ustával lid velice na ceste.
\par 5 A mluvil lid proti Bohu a proti Mojžíšovi: Proc jste vyvedli nás z Egypta, abychom zemreli na poušti? Nebo ani chleba ani vody není, a duše naše chléb tento nicemný sobe již zošklivila.
\par 6 Protož dopustil Hospodin na lid hady ohnivé, kteríž jej štípali, tak že množství lidu zemrelo z Izraele.
\par 7 Tedy prišel lid k Mojžíšovi a rekli: Zhrešili jsme nebo jsme mluvili proti Hospodinu a proti tobe; modl se Hospodinu, at odejme od nás ty hady. I modlil se Mojžíš za lid.
\par 8 I rekl Hospodin Mojžíšovi: Udelej sobe hada podobného tem ohnivým, a vyzdvihni jej na sochu; a každý uštknutý, když pohledí na nej, živ bude.
\par 9 I udelal Mojžíš hada medeného, a vyzdvihl jej na sochu; a stalo se, když uštkl had nekoho, a on vzhlédl na hada medeného, že zustal živ.
\par 10 Tedy hnuli se odtud synové Izraelští, a položili se v Obot.
\par 11 Potom pak hnuvše se z Obot, položili se pri pahrbcích hor Abarim na poušti, kteráž jest naproti zemi Moábské k východu slunce.
\par 12 Odtud brali se, a položili se v údolí Záred.
\par 13 Opet hnuvše se odtud, položili se u brodu potoka Arnon, kterýž jest na poušti, a vychází z koncin Amorejských. Nebo Arnon jest pomezí Moábské mezi Moábskými a Amorejskými.
\par 14 Protož praví se v knize boju Hospodinových, že proti Vahebovi u vichrici bojoval, a proti potokum Arnon.
\par 15 Nebo tok tech potoku, kterýž se nachyluje ku položení Ar, ten jde vedlé pomezí Moábského.
\par 16 A odtud táhli do Beer; a to jest to Beer, o nemž byl rekl Hospodin k Mojžíšovi: Shromažd lid, a dám jim vodu.
\par 17 Tedy zpíval lid Izraelský písnicku tuto: Vystupiž studnice, prozpevujte o ní;
\par 18 Studnice, kterouž kopala knížata, kterouž vykopali prední z lidu s vydavatelem zákona holemi svými. Z poušte pak brali se do Matana,
\par 19 A z Matana do Nahaliel, z Nahaliel do Bamot,
\par 20 A z Bamot do údolí, kteréž jest na poli Moábském, až k vrchu hory, kteráž leží na proti poušti.
\par 21 Tedy poslal lid Izraelský posly k Seonovi králi Amorejskému, rka:
\par 22 Necht jdeme skrze zemi tvou. Neuchýlíme se ani do pole, ani do vinic, ani z studnic vody píti nebudeme, ale cestou královskou pujdeme, dokavadž neprejdeme pomezí tvého.
\par 23 I nedopustil Seon jíti lidu Izraelskému skrze krajinu svou, nýbrž sebrav Seon všecken lid svuj, vytáhl proti lidu Izraelskému na poušt, a pritáh do Jasa, bojoval proti Izraelovi.
\par 24 I porazil jej lid Izraelský mecem, a vzal v dedictví zemi jeho, od Arnon až do Jabok, a až do zeme synu Ammon; nebo pevné bylo pomezí Ammonitských.
\par 25 Tedy vzal Izrael všecka ta mesta, a prebýval ve všech mestech Amorejských, v Ezebon a ve všech mesteckách jeho.
\par 26 Nebo Ezebon bylo mesto Seona krále Amorejského, kterýž když bojoval proti králi Moábskému prvnímu, vzal mu všecku zemi jeho z rukou jeho až do Arnon.
\par 27 Protož ríkávali v prísloví: Podte do Ezebon, aby vystaveno bylo a vzdeláno mesto Seonovo.
\par 28 Nebo ohen vyšel z Ezebon, a plamen z mesta Seon, i spálil Ar Moábských a obyvatele výsosti Arnon.
\par 29 Beda tobe Moáb, zahynuls lide Chámos; dal syny své v utíkání, a dcery své v zajetí králi Amorejskému Seonovi.
\par 30 A království jejich zahynulo od Ezebon až do Dibon, a vyhladili jsme je až do Nofe, kteréž jest až k Medaba.
\par 31 A tak bydlil Izrael v zemi Amorejské.
\par 32 Potom poslal Mojžíš, aby shlédli Jazer, kteréž vzali i s mestecky jeho; a tak vyhnal Amorejské, kteríž tam bydlili.
\par 33 Obrátivše se pak, táhli cestou k Bázan. I vytáhl Og, král Bázan, proti nim, on i všecken lid jeho, aby bojoval v Edrei.
\par 34 Tedy rekl Hospodin Mojžíšovi: Neboj se ho; nebo v ruce tvé dal jsem jej, i všecken lid jeho, i zemi jeho, a uciníš jemu tak, jakož jsi ucinil Seonovi, králi Amorejskému, kterýž bydlil v Ezebon.
\par 35 I porazili jej i syny jeho, i všecken lid jeho, tak že žádný živý po nem nezustal, a uvázali se dedicne v zemi jeho.

\chapter{22}

\par 1 I táhli synové Izraelští a položili se na polích Moábských, nedocházeje k Jordánu, naproti Jerichu.
\par 2 A videl Balák, syn Seforuv, všecko, co ucinil Izrael Amorejskému.
\par 3 I bál se Moáb toho lidu velmi, proto že ho bylo mnoho, a svíral se pro prítomnost synu Izraelských.
\par 4 Protož rekl Moáb k starším Madianským: Tudíž toto množství požere všecko, což jest vukol nás, jako sžírá vul trávu polní. Byl pak Balák, syn Seforuv, toho casu králem Moábským.
\par 5 I poslal posly k Balámovi, synu Beorovu, do mesta Petor, kteréž jest pri rece v zemi vlasti jeho, aby povolal ho, rka: Aj, lid vyšel z Egypta, aj, prikryl svrchek zeme, a usazuje se proti mne.
\par 6 Protož nyní pod medle, zlorec mne k vuli lidu tomuto,nebo silnejší mne jest; snad svítezím nad ním, a porazím jej, aneb vyženu z zeme této. Vím zajisté, že komuž ty žehnáš, požehnaný bude, a komuž ty zlorecíš, bude zlorecený.
\par 7 Tedy šli starší Moábští a starší Madianští, nesouce v rukou svých peníze za zlorecení; i prišli k Balámovi, a vypravovali mu slova Balákova.
\par 8 On pak rekl jim: Pobudte zde pres tuto noc, a dám vám odpoved, jakž mi mluviti bude Hospodin. I zustala knížata Moábská s Balámem.
\par 9 Prišel pak Buh k Balámovi a rekl: Kdo jsou ti muži u tebe?
\par 10 Odpovedel Balám Bohu: Balák, syn Seforuv, král Moábský, poslal ke mne, rka:
\par 11 Aj, lid ten, kterýž vyšel z Egypta, prikryl svrchek zeme; protož nyní pod, prokln mi jej, snad svítezím, bojuje s ním, a vyženu jej.
\par 12 I rekl Buh k Balámovi: Nechod s nimi, aniž zlorec lidu tomu, nebo požehnaný jest.
\par 13 Tedy Balám vstav ráno, rekl knížatum Balákovým: Navratte se do zeme své, nebo nechce mi dopustiti Hospodin, abych šel s vámi.
\par 14 A vstavše knížata Moábská, prišli k Balákovi a rekli: Nechtel Balám jíti s námi.
\par 15 Tedy opet poslal Balák více knížat a znamenitejších, nežli první.
\par 16 Kteríž prišedše k Balámovi, rekli jemu: Toto praví Balák, syn Seforuv: Nezpecuj se, prosím, prijíti ke mne.
\par 17 Nebo velikou ctí te ctíti chci, a což mi koli rozkážeš, uciním; protož prid, prosím, prokln mi lid tento.
\par 18 Odpovídaje pak Balám, rekl služebníkum Balákovým: Byt mi pak dal Balák plný dum svuj stríbra a zlata, nemohl bych prestoupiti slova Hospodina Boha svého, a uciniti proti nemu malé neb veliké veci.
\par 19 Nyní však zustante, prosím, zde i vy také této noci, abych zvedel, co dále mluviti bude Hospodin se mnou.
\par 20 Prišed pak Buh k Balámovi v noci, rekl jemu: Ponevadž proto, aby povolali te, prišli muži tito, vstana, jdi s nimi, a však což prikáži tobe, to cin.
\par 21 Tedy Balám vstav ráno, osedlal oslici svou, a bral se s knížaty Moábskými.
\par 22 Ale rozpálil se hnev Boží, proto že jel s nimi, a postavil se andel Hospodinuv na ceste, aby mu prekazil; on pak sedel na oslici své, maje s sebou dva služebníky své.
\par 23 A když uzrela oslice andela Hospodinova, an stojí na ceste, a mec jeho vytržený v ruce jeho, uhnula se s cesty, a šla polem. I bil ji Balám, aby ji navedl zase na cestu.
\par 24 A andel Hospodinuv stál na stezce u vinice mezi dvema zídkami.
\par 25 Viduci pak oslice andela Hospodinova, pritiskla se ke zdi, pritrela také nohu Balámovi ke zdi; procež opet bil ji.
\par 26 Potom andel Hospodinuv šel dále, a stál v úzkém míste, kdež nebylo žádné cesty k uchýlení se na pravo neb na levo.
\par 27 A viduci oslice andela Hospodinova, padla pod Balámem; procež rozhneval se velmi Balám, a bil oslici kyjem.
\par 28 I otevrel Hospodin ústa oslice, a rekla Balámovi: Cožt jsem ucinila, že již po tretí mne biješ?
\par 29 Rekl Balám k oslici: To, že jsi mne v posmech uvedla. Ó bych mel mec v rukou, jiste bych te již zabil.
\par 30 Odpovedela oslice Balámovi: Zdaliž nejsem oslice tvá? Jezdíval jsi na mne, jak jsi mne dostal, až do dnes; zdaliž jsem kdy obycej mela tak ciniti tobe? Kterýž odpovedel: Nikdy.
\par 31 V tom otevrel Hospodin oci Balámovy, i uzrel andela Hospodinova, an stojí na ceste, maje mec dobytý v ruce své; a nakloniv hlavy, poklonu ucinil na tvár svou.
\par 32 I mluvil k nemu andel Hospodinuv: Proc jsi bil oslici svou po trikrát? Aj, já vyšel jsem, abych se protivil tobe; nebo uchýlil jsi se s cesty prede mnou.
\par 33 Když videla mne oslice, vyhnula mi již po trikrát; a byt mi se byla nevyhnula, již bych byl také tebe zabil, a jí živé nechal.
\par 34 Odpovedel Balám andelu Hospodinovu: Zhrešilt jsem, nebo jsem nevedel, že ty stojíš proti mne na ceste; protož nyní, jestliže se nelíbí tobe, radeji navrátím se domu.
\par 35 Rekl andel Hospodinuv k Balámovi: Jdi s muži temi, avšak slovo, kteréž mluviti budu tobe, to mluviti budeš. Tedy šel Balám s knížaty Balákovými.
\par 36 Uslyšev pak Balák o príchodu Balámovu, vyšel proti nemu do mesta Moábského, kteréž bylo pri rece Arnon, jenž jest pri konci pomezí.
\par 37 I rekl Balák Balámovi: Zdaliž jsem víc než jednou neposílal pro te? Procež jsi tedy neprišel ke mne? Proto-li, že bych te náležite uctiti nemohl?
\par 38 Odpovedel Balám Balákovi: Aj, již jsem prišel k tobe; nyní pak zdaliž všelijak budu co moci mluviti? Slovo, kteréž vložil Buh v ústa má, to mluviti budu.
\par 39 I bral se Balám s Balákem, a prijeli do mesta Husot.
\par 40 Kdežto nabiv Balák volu a ovec, poslal k Balámovi a k knížatum, kteríž s ním byli.
\par 41 Nazejtrí pak ráno, pojav Balák Baláma, uvedl ho na výsosti modly Bál, odkudž shlédl i nejdalší díl lidu Izraelského.

\chapter{23}

\par 1 A rekl Balám Balákovi: Udelej mi tuto sedm oltáru, a priprav mi také sedm volku a sedm skopcu.
\par 2 I udelal Balák, jakž mluvil Balám. Tedy obetoval Balák a Balám volka a na každém oltári.
\par 3 Rekl pak Balám Balákovi: Postuj pri obeti své zápalné, a pujdu, zdali by se potkal se mnou Hospodin, a což by koli mne ukázal, povím tobe. I odšel sám.
\par 4 I potkal se Buh s Balámem, a rekl jemu Balám: Sedm oltáru zporádal jsem, a obetoval jsem volka a skopce na každém oltári.
\par 5 Vložil pak Hospodin slovo v ústa Balámova a rekl: Navrat se k Balákovi a tak mluv.
\par 6 I navrátil se k nemu, a nalezl jej, an stojí pri obeti své zápalné, i všecka knížata Moábská.
\par 7 Tedy vzav pred sebe prísloví své, rekl: Z Aram privedl mne Balák král Moábský, z hor východních, rka: Pod, zlorec mi k vuli Jákoba, a pod, vydej klatbu na Izraele.
\par 8 Proc bych zlorecil, komuž Buh silný nezlorecí? A proc bych klel, kohož Hospodin neproklíná?
\par 9 Když s vrchu skal hledím na nej, a s pahrbku spatruji jej, aj, lid ten sám bydlí, a k jiným národum se neprimešuje.
\par 10 Kdo secte prach Jákobuv? a kdo pocet? Kdo secte ctvrtý díl Izraelského lidu? Ó bych já umrel smrtí spravedlivých, a dokonání mé ó by bylo jako i jeho!
\par 11 I rekl Balák Balámovi: Což mi to deláš? Abys zlorecil neprátelum mým, povolal jsem te, a ty pak ustavicne dobrorecíš jim.
\par 12 Kterýž odpovídaje, rekl: Zdali toho, což Hospodin vložil v ústa má, nemám šetriti, abych tak mluvil?
\par 13 I rekl jemu Balák: Pod, prosím, se mnou na jiné místo, odkudž bys videl jej, (toliko zadní díl jeho videti budeš, a všeho nebudeš videti), a prokln mi jej odtud.
\par 14 I pojav jej, vyvedl ho na rovinu Zofim, na vrch jednoho pahrbku, a udelav sedm oltáru, obetoval volka a skopce na každém oltári.
\par 15 Rekl pak Balákovi: Postuj tuto pri zápalné obeti své, a já pujdu tamto vstríc Hospodinu.
\par 16 I potkal se Hospodin s Balámem, a vloživ slovo v ústa jeho, rekl: Navrat se k Balákovi a mluv tak.
\par 17 Prišel tedy k nemu, a hle, on stál pri zápalné obeti své, a knížata Moábská s ním. Jemužto rekl Balák: Co mluvil Hospodin?
\par 18 A on vzav prísloví své, rekl: Povstan Baláku a slyš, pozoruj mne, synu Seforuv.
\par 19 Buh silný není jako clovek, aby klamal, a jako syn cloveka, aby se menil. Což by rekl, zdaliž neuciní? Což by promluvil, zdali neutvrdí toho?
\par 20 Hle, abych dobrorecil, prijal jsem to na sebe; nebo dobrorecilt jest, a já toho neodvolám.
\par 21 Nepatrít na nepravosti v Jákobovi, aniž hledí na prestoupení v Izraeli; Hospodin Buh jeho jestit s ním, a zvuk krále vítezícího v nem.
\par 22 Buh silný vyvedl je z Egypta, jako silou jednorožcovou byv jim.
\par 23 Nebo není kouzlu proti Jákobovi, ani zaklínání proti Izraelovi; již od toho casu vypravováno bude o Jákobovi a Izraelovi, co ucinil s ním Buh silný.
\par 24 Aj, lid jakožto silný lev povstane, a jakožto lvíce vzchopí se; nepoloží se, dokudž by nejedl loupeže, a dokudž by nevypil krve zbitých.
\par 25 I rekl Balák Balámovi: Aniž mu již zlorec více, ani dobrorec.
\par 26 Jemuž odpovedel Balám, rka: Zdaližt jsem nepravil, že, což by mi koli mluvil Hospodin, to uciním?
\par 27 I rekl Balák Balámovi: Pod, prosím, povedu te na jiné místo, odkudž snad líbiti se bude Bohu, abys mi je proklel.
\par 28 A pojav Balák Baláma, uvedl jej na vrch hory Fegor, kteráž leží naproti poušti.
\par 29 Tedy rekl Balám Balákovi: Udelej mi tuto sedm oltáru, a priprav mi také sedm volku a sedm skopcu.
\par 30 I ucinil Balák, jakž rekl Balám, a obetoval volka a skopce na každém oltári.

\chapter{24}

\par 1 Vida pak Balám, že by se líbilo Hospodinu, aby dobrorecil Izraelovi, již více nešel jako prvé, jednou i po druhé, k cárum svým, ale obrátil tvár svou proti poušti.
\par 2 A pozdvih ocí svých, videl Izraele bydlícího v staních po svých pokoleních, a byl nad ním duch Boží.
\par 3 I vzav podobenství své, dí: Rekl Balám, syn Beoruv, rekl muž mající otevrené oci,
\par 4 Rekl slyšící výmluvnosti Boha silného, kterýž videní všemohoucího vidí, kterýž když padne, otevrené má oci:
\par 5 Jak velmi krásní jsou stánkové tvoji, Jákobe, príbytkové tvoji, Izraeli!
\par 6 Tak jako se potokové rozširují, jako zahrady vedlé reky, jako stromoví aloes, kteréž štípil Hospodin, jako cedrové podlé vod.
\par 7 Potece voda z okovu jeho, a síme jeho u vodách mnohých. A král jeho vyvýšen bude více než Agag, a vyzdvihne se království jeho.
\par 8 Buh silný vyvedl jej z Egypta, jako udatnost jednorožcova jest jemu; sžeret národy protivné sobe, a kosti jich potre a strelami svými prostrílí.
\par 9 Složil se a ležel jako lvíce, a jako lev zurivý; kdo jej zbudí? Kdo by požehnání dával tobe, požehnaný bude, a kdožt by zlorecil, bude zlorecený.
\par 10 Tedy zapáliv se hnevem Balák proti Balámovi, tleskl rukama svýma a rekl Balámovi: Abys zlorecil neprátelum mým, povolal jsem tebe, a aj, ustavicne dobrorecil jsi jim již po trikrát.
\par 11 Protož nyní navrat se radeji k místu svému. Bylt jsem rekl: Velikou ctí te ctíti chci, a hle, Hospodin zbavil te cti.
\par 12 Jemuž odpovedel Balám: Však hned poslum tvým, kteréž jsi poslal ke mne, mluvil jsem, rka:
\par 13 Byt mi dal Balák plný dum svuj stríbra a zlata, nebudu moci prestoupiti reci Hospodinovy, abych cinil dobré neb zlé sám od sebe; což mi Hospodin oznámí, to mluviti budu,
\par 14 (Ale však ješte já, již odcházeje k lidu svému, pockej, poradím tobe,) co uciní lid ten lidu tvému v posledních dnech.
\par 15 Tedy vzav podobenství své, dí: Rekl Balám, syn Beoruv, rekl muž mající oci otevrené,
\par 16 Rekl slyšící výmluvnosti Boha silného, kterýž zná naucení nejvyššího, a videní všemohoucího vidí, kterýž když padne, otevrené má oci:
\par 17 Uzrímt jej, ale ne nyní, pohledím na nej, ale ne z blízka. Vyjdet hvezda z Jákoba, a povstane berla z Izraele, kteráž poláme knížata Moábská, a zkazí všecky syny Set.
\par 18 I bude Edom podmanen, a Seir v vládarstí prijde neprátelum svým; nebo Izrael zmužile sobe pocínati bude.
\par 19 Panovati bude pošlý z Jákoba, a zahladí ostatky z každého mesta.
\par 20 A když uzrel Amalecha, vzav podobenství své, rekl: Jakož pocátek národu Amalech, tak konec jeho do gruntu zahyne.
\par 21 Uzrev pak Cinea, vzav podobenství své, rekl: Pevný jest príbytek tvuj, a na skále založils hnízdo své;
\par 22 Ale však vyhnán bude Cineus, Assur zajatého jej povede.
\par 23 Opet vzav podobenství své, rekl: Ach, kdo bude živ, když toto uciní Buh silný?
\par 24 A bárky z zeme Citim priplynou a ssouží Assyrské, ssouží také Židy, ale i ten lid do konce zahyne.
\par 25 Potom vstav Balám, odšel a navrátil se na místo své; Balák také odšel cestou svou.

\chapter{25}

\par 1 V tom, když pobyl Izrael v Setim, pocal lid smilniti s dcerami Moábskými.
\par 2 Kteréž pozvaly lidu k obetem bohu svých; i jedl lid, a klaneli se bohum jejich.
\par 3 I pripojil se lid Izraelský k modle Belfegor, a popudila se prchlivost Hospodinova proti Izraelovi.
\par 4 I rekl Hospodin Mojžíšovi: Vezmi všecka knížata lidu, a zvešej ty nešlechetníky Hospodinu pred sluncem, aby se odvrátila prchlivost Hospodinova od Izraele.
\par 5 I rekl Mojžíš soudcum Izraelským: Zabí jeden každý z svých všelikého, kterýž se pripojil k modle Belfegor.
\par 6 A aj, jeden z synu Izraelských prišel, a privedl k bratrím svým ženu Madianku, na odivu Mojžíšovi i všemu množství synu Izraelských; oni pak plakali u dverí stánku úmluvy.
\par 7 To když uzrel Fínes, syn Eleazara, syna Aronova, kneze, vyvstal z prostredku množství toho, a vzal kopí do ruky své.
\par 8 A všed za mužem Izraelským do stanu, probodl je oba dva, muže Izraelského, i ženu tu skrze bricho její; i odvrácena jest rána od synu Izraelských.
\par 9 Zhynulo jich pak od té rány dvadceti ctyri tisíce.
\par 10 Tedy mluvil Hospodin k Mojžíšovi, rka:
\par 11 Fínes, syn Eleazara kneze, syna Aronova, odvrátil prchlivost mou od synu Izraelských, kdyžto horlil horlivostí mou u prostred nich, tak abych neshladil synu Izraelských v horlivosti své.
\par 12 Protož díš: Aj, já dám jemu smlouvu svou pokoje.
\par 13 I bude míti on i síme jeho po nem smlouvu knežství vecného, proto že horlil pro Boha svého, a ocistil syny Izraelské.
\par 14 A bylo jméno muže Izraelského zabitého, kterýž byl zabit s Madiankou, Zamri, syn Sáluv, kníže domu otce svého z pokolení Simeonova.
\par 15 Jméno pak ženy Madianky zabité, Kozbi, dcera Sur, kterýž byl prední v národu svém, v dome otce svého mezi Madianskými.
\par 16 I mluvil Hospodin k Mojžíšovi, rka:
\par 17 Neprátelsky zacházejte s temi Madianskými, a zbíte je.
\par 18 Nebo i oni neprátelsky a lstive chovali se k vám, a oklamali vás skrze modlu Fegor, a skrze Kozbu, dceru knížete Madianského, sestru svou, kteráž zabita jest v den pomsty prišlé z príciny Fegor.

\chapter{26}

\par 1 I stalo se po té ráne, že mluvil Hospodin k Mojžíšovi a Eleazarovi, synu Arona kneze, rka:
\par 2 Sectete všecko množství synu Izraelských, od dvadcítiletých a výše po domích otcu jejich, všecky, kteríž by mohli jíti k boji v Izraeli.
\par 3 Tedy mluvil Mojžíš a Eleazar knez k nim na polích Moábských, pri Jordánu proti Jerichu, rka:
\par 4 Sectete lid od dvadceti let majících a výše, jakž rozkázal Hospodin Mojžíšovi a sy\par Izraelským, kteríž byli vyšli z zeme Egyptské.
\par 5 Ruben prvorozený byl Izraeluv. Synové Rubenovi: Enoch, z nehož pošla celed Enochitská; Fallu, z nehož celed Fallutská;
\par 6 Ezron, z nehož celed Ezronitská; Charmi, z nehož celed Charmitská.
\par 7 Ty jsou celedi Rubenovy. A bylo jich sectených ctyridceti tri tisíce, sedm set a tridceti.
\par 8 A syn Falluv Eliab.
\par 9 Synové pak Eliabovi: Nemuel, a Dátan, a Abiron. To jsou ti, Dátan a Abiron, prední z shromáždení, kteríž se vadili s Mojžíšem a s Aronem v spiknutí Chóre, když odporni byli Hospodinu.
\par 10 Procež otevrela zeme ústa svá, a požrela je i Chóre, tehdáž když zemrela ta rota, a ohen spálil tech dve ste a padesáte mužu, kteríž byli za príklad jiným.
\par 11 Synové pak Chóre nezemreli.
\par 12 Synové Simeonovi po celedech svých: Namuel, z nehož celed Namuelitská; Jamin, z nehož celed Jaminská; Jachin, z nehož celed Jachinská;
\par 13 Sohar, z nehož celed Soharská; Saul, z nehož celed Saulitská.
\par 14 Ty jsou celedi Simeonovy, jichž bylo dvamecítma tisícu a dve ste.
\par 15 Synové Gád po celedech svých: Sefon, z nehož celed Sefonitská; Aggi, z nehož celed Aggitská; Suni, z nehož celed Sunitská;
\par 16 Ozni, z nehož celed Oznitská; Heri, z nehož celed Heritská;
\par 17 Arodi, z nehož celed Aroditská; Areli, z nehož celed Arelitská.
\par 18 Ty jsou celedi synu Gád, podlé toho, jakž secteni jsou, ctyridceti tisícu a pet set.
\par 19 Synové Judovi: Her a Onan; ale zemreli Her i Onan v zemi Kanánské.
\par 20 Byli pak synové Judovi po celedech svých: Séla, z nehož celed Sélanitská; Fáres, z nehož celed Fáresská; Zára, z nehož celed Záretská.
\par 21 Byli pak synové Fáresovi: Ezron, z nehož celed Ezronitská; Hamul, z nehož celed Hamulská.
\par 22 Ty jsou celedi Judovy, podlé toho, jakž secteni jsou, sedmdesáte šest tisícu a pet set.
\par 23 Synové Izacharovi po celedech svých: Tola, z nehož celed Tolatská; Fua, z nehož celed Fuatská;
\par 24 Jasub, z nehož celed Jasubská; Simron, z nehož celed Simronská.
\par 25 Ty jsou celedi Izacharovy, podlé toho, jakž secteni jsou, šedesáte ctyri tisíce a tri sta.
\par 26 Synové Zabulonovi po celedech svých: Sared, z nehož celed Saredská; Elon, z nehož celed Elonská; Jahelel, z nehož celed Jahelelská.
\par 27 Ty jsou celedi Zabulonovy, podlé toho jakž secteni jsou, šedesáte tisíc a pet set.
\par 28 Synové Jozefovi po celedech svých: Manasses a Efraim.
\par 29 Synové Manassesovi: Machir, z nehož celed Machirská. A Machir zplodil Galáda, z nehož celed Galádská.
\par 30 Ti jsou synové Galád: Jezer, z nehož celed Jezerská; Helek, z nehož celed Helekitská;
\par 31 Asriel, z nehož celed Asrielská; Sechem, z nehož celed Sechemská;
\par 32 Semida, z nehož celed Semidatská; Hefer, z nehož celed Heferská.
\par 33 A Salfad, syn Heferuv, nemel synu, než toliko dcery, jichž jsou tato jména: Mahla, Noa, Hegla, Melcha a Tersa.
\par 34 Ty jsou celedi Manassesovy, a nacteno jich padesáte dva tisíce a sedm set.
\par 35 Ale synové Efraimovi po celedech svých: Sutala, z nehož celed Sutalitská; Becher, z nehož celed Becherská; Tehen, z nehož celed Tehenská.
\par 36 A ti jsou synové Sutalovi: Heran, z nehož celed Heranská.
\par 37 Ty jsou celedi synu Efraimových, podlé toho, jakž secteni jsou, tridceti dva tisíce a pet set. Ti jsou synové Jozefovi po celedech svých.
\par 38 Synové pak Beniaminovi po celedech svých: Béla, z nehož celed Bélitská; Asbel, z nehož celed Asbelská; Ahiram, z nehož celed Ahiramská;
\par 39 Sufam, z nehož celed Sufamská; Hufam, z nehož celed Hufamská.
\par 40 Byli pak synové Béla: Ared a Náman; z Ared celed Aredská, z Náman celed Námanská.
\par 41 Ti jsou synové Beniaminovi po celedech svých, podlé toho, jakž secteni jsou, ctyridceti pet tisícu a šest set.
\par 42 Tito pak synové Danovi po celedech svých: Suham, z nehož celed Suhamská. Ta jest rodina Danova po celedech svých.
\par 43 Všech celedí Suhamských, jakž secteni jsou, šedesáte ctyri tisíce a ctyri sta.
\par 44 Synové Asser po celedech svých: Jemna, z nehož celed Jemnitská; Jesui, z nehož celed Jesuitská;
\par 45 Beria, z nehož celed Berietská. Synové Beriovi: Heber, z nehož celed Heberská; Melchiel, z nehož celed Melchielská.
\par 46 Jméno pak dcery Asser bylo Serach.
\par 47 Ty jsou celedi synu Asser, tak jakž secteni jsou, padesáte tri tisíce a ctyri sta.
\par 48 Synové Neftalímovi po celedech svých: Jasiel, z nehož celed Jasielská; Guni, z nehož celed Gunitská;
\par 49 Jezer, z nehož celed Jezerská; Sallem, z nehož celed Sallemská.
\par 50 Ta jest rodina Neftalímova po celedech svých, tak jakž secteni jsou, ctyridceti pet tisícu a ctyri sta.
\par 51 Ten jest pocet synu Izraelských, šestkrát sto tisícu a jeden tisíc, sedm set a tridceti.
\par 52 Mluvil pak Hospodin k Mojžíšovi, rka:
\par 53 Temto rozdelena bude zeme k dedictví podlé poctu jmen.
\par 54 Vetšímu poctu vetší dedictví dáš, a menšímu menší; jednomu každému vedlé poctu sectených jeho dáno bude dedictví jeho.
\par 55 A však losem at jest rozdelena zeme; vedlé jmen pokolení otcu svých dedictví vezmou.
\par 56 Losem deleno bude dedictví její, bud jich mnoho neb málo.
\par 57 Tito pak jsou secteni z Levítu po celedech svých: Gerson, z nehož celed Gersonitská; Kahat, z nehož celed Kahatská; Merari, z nehož celed Meraritská.
\par 58 Ty jsou celedi Léví: Celed Lebnitská, celed Hebronitská, celed Moholitská, celed Musitská, celed Choritská. Kahat pak zplodil Amrama.
\par 59 A jméno manželky Amramovy Jochebed, dcera Léví, kteráž se mu narodila v Egypte; ona pak porodila Amramovi Arona a Mojžíše, a Marii sestru jejich.
\par 60 Aronovi pak zrozeni jsou: Nádab a Abiu, Eleazar a Itamar.
\par 61 Ale Nádab a Abiu zemreli, když obetovali ohen cizí pred Hospodinem.
\par 62 I bylo jich nacteno trimecítma tisícu, všech pohlaví mužského zstárí mesíce jednoho a výše; nebo nebyli pocteni mezi syny Izraelskými, proto že jim nebylo dáno dedictví mezi syny Izraelskými.
\par 63 Tito secteni jsou od Mojžíše a Eleazara kneze; oni sectli syny Izraelské na polích Moábských pri Jordánu, naproti Jerichu.
\par 64 Mezi temito pak nebyl žádný z onech sectených od Mojžíše a Arona kneze, když pocítali syny Izraelské na poušti Sinai;
\par 65 (Nebo rekl byl Hospodin o nich: Smrtí zemrou na poušti;) a žádný z nich nepozustal, jediné Kálef, syn Jefonuv, a Jozue, syn Nun.

\chapter{27}

\par 1 Tehdy pristoupily dcery Salfada, syna Heferova, syna Galád, syna Machir, syna Manasse, z pokolení Manasse, syna Jozefova. Tato pak jsou jména dcer jeho: Mahla, Noa, Hegla, Melcha a Tersa.
\par 2 A postavily se pred Mojžíšem a pred Eleazarem knezem, i pred knížaty a vším množstvím, u dverí stánku úmluvy, a rekly:
\par 3 Otec náš umrel na poušti, kterýž však nebyl z spolku tech, kteríž se zrotili proti Hospodinu v rote Chóre; nebo pro hrích svuj umrel, synu žádných nemaje.
\par 4 Což vyhlazeno býti má jméno otce našeho z celedi jeho, proto že nemel syna? Dej nám dedictví mezi bratrími otce našeho.
\par 5 I vznesl tu vec Mojžíš na Hospodina.
\par 6 Kterýž odpovedel Mojžíšovi, rka:
\par 7 Dobre mluví dcery Salfad. Dejž jim bez odporu právo dedictví mezi bratrími otce jejich, a prenes dedictví otce jejich na ne.
\par 8 K sy\par pak Izraelským toto mluviti budeš: Když by kdo umrel, nemaje syna, tedy prenesete dedictví jeho na dceru jeho.
\par 9 Pakli by ani dcery nemel, tedy dáte dedictví jeho bratrím jeho.
\par 10 Pakli by ani bratrí nemel, tedy dáte dedictví jeho bratrím otce jeho.
\par 11 Pakli by ani strýcu nemel, tedy dáte dedictví jeho príteli jeho, kterýž jest nejbližší jemu v rodu jeho, aby dedicne obdržel je. A bude to sy\par Izraelským za ustanovení soudné, jakož prikázal Hospodin Mojžíšovi.
\par 12 Rekl také Hospodin Mojžíšovi: Vstup na horu tuto Abarim, a spatr zemi, kterouž jsem dal sy\par Izraelským.
\par 13 A když spatríš ji, pripojen budeš k lidu svému i ty, jako pripojen jest Aron bratr tvuj,
\par 14 Ponevadž jste odporovali reci mé na poušti Tsin, pri odporování všeho množství, kdež jste mne meli posvetiti pri vodách pred ocima jejich. To jsou ty vody sváru v Kádes, na poušti Tsin.
\par 15 I rekl Mojžíš Hospodinu, rka:
\par 16 Opatríš Hospodin Buh duchu, Buh všelikého tela, shromáždení toto mužem hodným,
\par 17 Kterýž by vycházel pred nimi, a kterýž by vcházel pred nimi, kterýž by vyvodil a zase uvodil je, aby nebylo shromáždení Hospodinovo jako ovce, kteréž nemají pastýre.
\par 18 I rekl Hospodin Mojžíšovi: Vezmi k sobe Jozue, syna Nun, muže, v kterémž jest duch muj, a vlož ruku svou na nej.
\par 19 A postave jej pred Eleazarem knezem a prede vším shromáždením, dáš jemu naucení pred ocima jejich,
\par 20 A udelíš jemu slávy své, aby ho poslouchalo všecko množství synu Izraelských.
\par 21 Kterýž pred Eleazarem knezem postave se, ptáti se ho bude na soud urim pred Hospodinem. K rozkazu jeho vyjdou, a k rozkazu jeho vejdou, on i všickni synové Izraelští s ním, a všecko shromáždení.
\par 22 I ucinil Mojžíš tak, jakž mu byl rozkázal Hospodin, a pojav Jozue, postavil jej pred Eleazarem knezem a prede vším shromáždením.
\par 23 A vloživ ruce své na nej, dal jemu naucení, jakž mluvil Hospodin skrze Mojžíše.

\chapter{28}

\par 1 Mluvil také Hospodin k Mojžíšovi, rka:
\par 2 Prikaž sy\par Izraelským a rci jim: Obetí mých, chleba mého, v obetech mých ohnivých u vuni spokojující mne, nezanedbávejte mi prinášeti v cas k tomu vymerený.
\par 3 Díš tedy jim: Tato jest obet ohnivá, kterouž obetovati budete Hospodinu: Beránky rocní bez poškvrny dva, každého dne v obet zápalnou ustavicne.
\par 4 Beránka jednoho obetovati budeš ráno, a beránka druhého obetovati budeš k vecerou.
\par 5 Desátý také díl míry efi mouky belné, v obet suchou, zadelané olejem nejcistším, ctvrtou cástkou míry hin.
\par 6 Ta jest obet zápalná ustavicná, kteráž obetována jest na hore Sinai u vuni príjemnou, v obet ohnivou Hospodinu.
\par 7 A obet mokrá její, ctvrtý díl míry hin na každého beránka; v svatyni obetuj obet mokrou silného nápoje Hospodinu.
\par 8 Beránka pak druhého obetovati budeš k vecerou, tak jako obet suchou jitrní, a jako obet mokrou její obetovati budeš v obet ohnivou, u vuni spokojující Hospodina.
\par 9 Dne také sobotního dva beránky rocní bez poškvrny, a dve desetiny mouky belné, olejem zadelané v obet suchou s obetí její mokrou.
\par 10 Ta bude obet zápalná sobotní každého dne sobotního, mimo zápalnou obet ustavicnou a mokrou obet její.
\par 11 Také na nov mesícu vašich obetovati budete obet zápalnou Hospodinu, volky mladé dva, skopce jednoho, beránku rocních bez poškvrny sedm;
\par 12 A tri desetiny mouky belné, olejem zadelané v obet suchou pri každém volku, a dve desetiny mouky belné, olejem zadelané v obet suchou pri každém skopci;
\par 13 A jednu desetinu mouky belné olejem zadelané v obet suchou pri každém beránku, k obeti zápalné u vuni príjemnou, v obet ohnivou Hospodinu.
\par 14 Tyto pak obeti mokré jejich z vína: Pul míry hin bude pri každém volku, a tretí cástka hin pri skopci, a ctvrtá cástka hin pri každém beránku. Ta jest obet zápalná k nov mesíci, každého mesíce pres celý rok.
\par 15 Kozla také jednoho v obet za hrích, mimo obet ustavicnou, obetovati budete Hospodinu s mokrou obetí jeho.
\par 16 Mesíce pak prvního, ctrnáctý den téhož mesíce velikanoc bude Hospodinu,
\par 17 A v patnáctý den téhož mesíce slavnost; za sedm dní chleby nekvašené jísti budete.
\par 18 Prvního dne shromáždení svaté bude, žádného díla robotného nebudete delati v nem.
\par 19 Obetovati pak budete obet ohnivou v zápal Hospodinu, volky mladé dva, skopce jednoho, a sedm beránku rocních; bez poškvrny budou.
\par 20 A suchou obet pri nich, totiž mouku belnou, olejem zadelanou, tri desetiny pri každém volku, a dve desetiny pri každém skopci obetovati budete.
\par 21 A jednu desetinu obetovati budeš pri každém beránku z tech sedmi beránku,
\par 22 A kozla v obet za hrích jednoho k ocištení vás.
\par 23 Mimo obet zápalnou jitrní, kteráž jest obet ustavicná, obetovati budete to.
\par 24 Tak obetovati budete každého dne za tech sedm dní pokrm obeti ohnivé, u vuni spokojující Hospodina, mimo zápalnou obet ustavicnou, a obet mokrou její.
\par 25 Sedmého pak dne shromáždení svaté míti budete, žádného díla robotného nebudete delati.
\par 26 V den také prvotin, když obetovati budete novou obet suchou Hospodinu, vyplníc téhodny vaše, shromáždení svaté míti budete; žádného díla robotného nebudete delati.
\par 27 A obetovati budete obet zápalnou, u vuni spokojující Hospodina, volky mladé dva, skopce jednoho, beránku rocních sedm,
\par 28 A obet suchou jejich, mouky belné olejem zadelané tri desetiny pri každém volku, dve desetiny pri každém skopci,
\par 29 Jednu desetinu pri každém beránku z tech sedmi beránku,
\par 30 Kozla jednoho k ocištení vašemu.
\par 31 To mimo obet zápalnou ustavicnou, a obet suchou její obetovati budete; bez poškvrny at jsou s obetmi svými mokrými.

\chapter{29}

\par 1 Mesíce pak sedmého v první den jeho shromáždení svaté míti budete, žádného díla robotného nebudete delati; to jest den troubení vašeho.
\par 2 A obetovati budete zápal u vuni príjemnou Hospodinu, volka mladého jednoho, skopce jednoho, beránku rocních bez poškvrny sedm;
\par 3 A obet suchou pri nich z mouky belné olejem zadelané, tri desetiny na každého volka, a dve desetiny na každého skopce;
\par 4 A desetina jedna na každého beránka z sedmi beránku;
\par 5 A kozla jednoho v obet za hrích k ocištení vás.
\par 6 Mimo zápalnou obet novomesícnou s obetí její suchou, a mimo obet zápalnou ustavicnou s obetí její suchou, a s obetmi jejich mokrými vedlé porádku jejich, u vuni príjemnou, v obet ohnivou Hospodinu.
\par 7 V desátý pak den téhož mesíce sedmého shromáždení svaté míti budete, a ponižovati budete životu svých; žádného díla nebudete delati.
\par 8 A obetovati budete obet zápalnou Hospodinu u vuni príjemnou, volka mladého jednoho, skopce jednoho, beránku rocních sedm, a ti at jsou bez poškvrny;
\par 9 A obet suchou jejich z mouky belné olejem zadelané, tri desetiny na každého volka, a dve desetiny na každého skopce;
\par 10 A desetina jedna na každého beránka z tech sedmi beránku;
\par 11 Kozla jednoho v obet za hrích, mimo obet za hrích k ocištení, a mimo zápal ustavicný s obetí suchou jeho, a s obetmi mokrými jejich.
\par 12 V patnáctý také den mesíce sedmého shromáždení svaté míti budete; žádného díla robotného nebudete delati, ale slaviti budete svátek Hospodinu za sedm dní.
\par 13 A obetovati budete zápal v obet ohnivou u vuni spokojující Hospodina, volku mladých trinácte, skopce dva, beránku rocních ctrnácte, a ti at jsou bez poškvrny;
\par 14 Též obet suchou jejich z mouky belné olejem zadelané, tri desetiny na každého volka z trinácti volku, dve desetiny na každého skopce z tech dvou skopcu,
\par 15 A jednu desetinu na každého beránka z tech ctrnácti beránku;
\par 16 A kozla jednoho v obet za hrích, mimo zápal ustavicný s obetí jeho suchou i mokrou.
\par 17 Potom dne druhého volku mladých dvanácte, skopce dva, beránku rocních bez poškvrny ctrnácte,
\par 18 S obetí suchou pri nich, a s obetmi mokrými jejich pri každém volku, skopci i beránku vedlé poctu jejich, jakž jest obycej;
\par 19 A kozla jednoho v obet za hrích, mimo zápal ustavicný s obetí suchou jeho a s obetmi mokrými jeho.
\par 20 Dne pak tretího volku jedenácte, skopce dva, a beránku rocních bez poškvrny ctrnácte,
\par 21 S obetí suchou a s obetmi mokrými jejich pri každém volku, skopci i beránku, vedlé poctu jejich, jakž jest obycej;
\par 22 A kozla v obet za hrích jednoho, mimo obet zápalnou ustavicnou s obetí její suchou i mokrou.
\par 23 Dne pak ctvrtého volku deset, skopce dva, beránku rocních bez poškvrny ctrnácte,
\par 24 S obetí suchou i s obetmi mokrými jejich pri každém volku, skopci a beránku, vedlé poctu jejich, jakž jest obycej;
\par 25 A kozla jednoho v obet za hrích, mimo obet zápalnou ustavicnou s obetí její suchou i mokrou.
\par 26 Dne také pátého volku devet, skopce dva, beránku rocních bez poškvrny ctrnácte,
\par 27 S obetí suchou i s obetmi mokrými jejich pri každém volku, skopci i beránku, vedlé poctu jejich, jakž jest obycej;
\par 28 A kozla jednoho v obet za hrích, mimo obet zápalnou ustavicnou s obetí její suchou i mokrou.
\par 29 A dne šestého volku osm, skopce dva, beránku rocních bez poškvrny ctrnácte,
\par 30 S obetí suchou a s obetmi mokrými jejich pri každém volku, skopci i beránku, vedlé poctu jejich, jakž jest obycej;
\par 31 A kozla jednoho v obet za hrích, mimo obet zápalnou ustavicnou s obetí její suchou i s obetmi mokrými.
\par 32 Tolikéž dne sedmého volku sedm, skopce dva, beránku rocních bez poškvrny ctrnácte,
\par 33 S obetí suchou i s obetmi mokrými jejich na volky, skopce i beránky, vedlé poctu jejich, jakž jest obycej;
\par 34 A kozla jednoho v obet za hrích, mimo obet zápalnou ustavicnou s obetí její suchou i mokrou.
\par 35 Dne pak osmého slavnost míti budete; žádného díla robotného nebudete delati.
\par 36 A obetovati budete obet zápalnou, v obet ohnivou vune spokojující Hospodina, volka jednoho, skopce jednoho, a beránku rocních sedm bez poškvrny,
\par 37 S obetí suchou i s obetmi mokrými jejich pri volku, skopci i beráncích, vedlé poctu jejich, jakž jest obycej;
\par 38 A kozla jednoho v obet za hrích, mimo obet zápalnou ustavicnou s obetí její suchou i mokrou.
\par 39 Ty veci vykonávati budete Hospodinu pri slavnostech vašich, krome toho, což byste z slibu aneb z dobré vule své obetovali, bud zápalné, aneb suché, aneb mokré, aneb pokojné obeti vaše.

\chapter{30}

\par 1 I oznámil Mojžíš sy\par Izraelským všecky ty veci, kteréž prikázal jemu Hospodin.
\par 2 Mluvil také Mojžíš knížatum pokolení synu Izraelských, rka: Totot jest, což prikázal Hospodin:
\par 3 Jestliže by muž slib aneb prísahu ucinil Hospodinu, závazkem zavazuje duši svou, nezruší slova svého; podlé všeho, což vyšlo z úst jeho, uciní.
\par 4 Když by pak osoba ženského pohlaví ucinila slib Hospodinu, a závazkem zavázala se v dome otce svého v mladosti své,
\par 5 A slyše otec její slib a závazek její, jímž zavázala duši svou, mlcel by k tomu: tedy stálí budou všickni slibové její, i každý závazek, jímž zavázala se, stálý bude.
\par 6 Jestliže by pak to zrušil otec její toho dne, když slyšel všecky ty sliby a závazky její, jimiž zavázala duši svou, nebudout stálí; a Hospodin odpustí jí, nebo otec její to zrušil.
\par 7 Pakli by vdaná byla za muže, a mela by slib na sobe, aneb pronesla by ústy svými neco, címž by se zavázala,
\par 8 A slyše muž její, nic by jí nerekl toho dne, kteréhož slyšel: stálí budou slibové její, i závazkové, jimiž zavázala duši svou, stálí budou.
\par 9 Jestliže by pak muž její toho dne, jakž uslyšel, odeprel tomu, a zrušil slib, kterýž na sobe mela, aneb neco rty svými pronesla, címž by se zavázala, také odpustí jí Hospodin.
\par 10 Všeliký pak slib vdovy aneb ženy zahnané, jímž by se zavázala, stálý bude.
\par 11 Ale jestliže v dome muže svého slíbila, a závazkem zavázala se s prísahou,
\par 12 A slyše to muž její, mlcel k tomu a neodeprel: tedy stálí budou všickni slibové její, a všickni závazkové, jimiž se zavázala, stálí budou.
\par 13 Pakli docela odeprel muž její toho dne, jakž uslyšel, všeliký slib, kterýž vyšel z úst jejích, a závazek, jímž zavázala se, nebude stálý; muž její zrušil to, a Hospodin jí odpustí.
\par 14 Všelikého slibu a každého závazku s prísahou ucineného o trápení života jejího, muž její potvrdí jeho, a muž její zruší jej.
\par 15 Pakli by muž její den po dni mlcel, tedy potvrdí všech slibu jejích, a všech závazku jejích, kteréž na sobe má; potvrdilt jest jich, nebo neodeprel jí v den ten, když to uslyšel.
\par 16 Jestliže by pak slyše, potom teprv zrušiti to chtel, tedy on ponese nepravost její.
\par 17 Ta jsou ustanovení, kteráž prikázal Hospodin Mojžíšovi, mezi mužem a ženou jeho, mezi otcem a dcerou jeho v mladosti její, dokudž jest v dome otce svého.

\chapter{31}

\par 1 Mluvil opet Hospodin k Mojžíšovi, rka:
\par 2 Pomsti prvé synu Izraelských nad Madianskými, a potom pripojen budeš k lidu svému.
\par 3 Mluvil tedy Mojžíš k lidu, rka: Vypravte nekteré z sebe k boji, aby šli proti Madianským a vykonali pomstu Hospodinovu nad nimi.
\par 4 Po tisíci z pokolení, ze všech pokolení Izraelských vyšlete k boji.
\par 5 I vydáno jest z mnohých tisícu Izraelských po tisíci z každého pokolení, totiž dvanácte tisícu zpusobných k boji.
\par 6 I poslal je Mojžíš po tisíci z každého pokolení k boji, a Fínesa syna Eleazara kneze s nimi; a nádoby svaté i trouby k troubení byly v ruce jeho.
\par 7 Tedy bojovali proti Madianským, jakož byl prikázal Hospodin Mojžíšovi, a zbili všecky pohlaví mužského.
\par 8 Pobili také krále Madianské mezi jinými, kteréž porazili, totiž Evi, Rekem, Sur, Hur a Rebe, pet králu Madianských; Baláma také, syna Beorova, zabili mecem.
\par 9 A zajali synové Izraelští ženy Madianské i deti jejich; všecka hovada jejich, i všechny dobytky jejich, a všecka zboží jejich pobrali.
\par 10 Všecka také mesta jejich, v kterýchž svá obydlí meli, i všecky hrady jejich vypálili ohnem.
\par 11 A všecku loupež i všecky koristi pobravše, lidi i hovada,
\par 12 Vedli je k Mojžíšovi a k Eleazarovi knezi, a ke všemu množství synu Izraelských,i zajaté i koristi, i loupeže, k vojsku na roviny Moábské, kteréž jsou pri Jordánu naproti Jerichu.
\par 13 I vyšli Mojžíš a Eleazar knez a všecka knížata shromáždení proti nim ven za stany.
\par 14 Tedy rozhneval se Mojžíš na vudce vojska, hejtmany nad tisíci a setníky, kteríž se navraceli z boje,
\par 15 A rekl jim Mojžíš: A což jste zachovali všecky ženy?
\par 16 Ej, onyt jsou hle sy\par Izraelským, podlé rady Balámovy, daly prícinu k prestoupení proti Hospodinu, pri modlárství Fegor, procež ona rána prišla byla na lid Hospodinuv.
\par 17 Protož nyní zmordujte všecky deti pohlaví mužského, a všecky ženy, kteréž poznaly muže.
\par 18 Všecky pak panny, kteréž nepoznaly muže, zachovejte sobe živé.
\par 19 Vy pak zustante vne za stany za sedm dní; všickni, kterížkoli jste nekoho zabili, aneb kteríž jste se zabitého dotkli, ocištovati se budete dne tretího a dne sedmého, sebe i zajaté své.
\par 20 Všeliké také roucho a všecky veci kožené, i všelijaké dílo z í kozích, i všelikou nádobu drevenou ocistíte.
\par 21 I rekl Eleazar knez vojákum, kteríž byli šli k boji: Toto jest ustanovení zákona, kteréž byl prikázal Hospodin Mojžíšovi.
\par 22 Zlato však, stríbro, med, železo, cín a olovo,
\par 23 A cožkoli trpí ohen, ohnem prepálíte, a precišteno bude, však tak, když vodou ocištování obmyto bude; což pak nemuže ohne strpeti, to skrze vodu protáhnete.
\par 24 Zpérete také roucha svá v den sedmý, a cistí budete; a potom vejdete do stanu.
\par 25 Mluvil i to Hospodin k Mojžíšovi, rka:
\par 26 Secti summu koristí zajatých, tak z lidí jako z hovad, ty a Eleazar knez, a prední z celedi otcu v lidu;
\par 27 A rozdelíš ty koristi na dva díly, jeden mezi vojáky, kteríž byli vytáhli na vojnu, a druhý mezi všecko shromáždení.
\par 28 A vezmeš díl na Hospodina od mužu bojovných, kteríž byli vyšli na vojnu, jednu duši z peti set, budto z lidí neb z hovad, neb z oslu, neb z ovcí.
\par 29 Z jejich polovice to vezmete, a dáte Eleazarovi knezi obet vzhuru pozdvižení Hospodinu.
\par 30 Z polovice pak té, kteráž jest synu Izraelských, vezmeš jedno z padesáti, budto z lidí neb z volu, neb z oslu, neb z ovcí, a tak ze všelijakých hovad, a dáš to Levítum, držícím stráž príbytku Hospodinova.
\par 31 I ucinil Mojžíš a Eleazar knez tak, jakž byl rozkázal Hospodin Mojžíšovi.
\par 32 A bylo té koristi z pozustalé ješte loupeže, kteréž nabral lid válecný, ovec šestkrát sto tisíc, sedmdesáte a pet tisícu;
\par 33 A volu sedmdesáte a dva tisíce;
\par 34 Oslu šedesáte a jeden tisícu;
\par 35 A panen, kteréž mužu nepoznaly, všech dva a tridceti tisícu.
\par 36 Dostala se pak polovice jedna na díl tem, kteríž byli vytáhli na vojnu, dobytka drobného v poctu trikrát sto tisíc, tridceti a sedm tisícu a pet set,
\par 37 A na díl vzatý Hospodinu dobytka drobného šest set, sedmdesáte pet.
\par 38 A z volu šest a tridceti tisícu, z nichž prišlo na díl Hospodinu sedmdesáte a dva.
\par 39 Oslu také tridceti tisíc a pet set, z nichž prišlo na díl Hospodinu šedesáte a jeden.
\par 40 A lidí šestnácte tisícu, z nichž prišlo na díl Hospodinu tridceti a dve duše.
\par 41 Dal tedy Mojžíš díl oddelený Hospodinu Eleazarovi knezi, jakž byl prikázal Hospodin Mojžíšovi.
\par 42 Z druhé pak polovice synu Izraelských, kterouž vzal Mojžíš od tech mužu, jenž bojovali,
\par 43 (A bylo té polovice k shromáždení prináležející z ovec trikrát sto tisíc, tridceti a sedm tisícu a pet set;
\par 44 Volu tridceti šest tisícu;
\par 45 Oslu tridceti tisícu a pet set;
\par 46 A lidí šestnácte tisícu;)
\par 47 Z té tedy polovice synu Izraelských vzal Mojžíš po jednom zajatém z padesáti, tak z lidí jako z hovad, a dal to Levítum, držícím stráž príbytku Hospodinova, jakž byl prikázal Hospodin Mojžíšovi.
\par 48 Tedy pristoupili k Mojžíšovi vývodové vojska, hejtmané nad tisíci a setníci,
\par 49 A rekli jemu: My služebníci tvoji sectli jsme pocet bojovníku, kteréž jsme meli pod spravou naší, a neubyl ani jeden z nás.
\par 50 A protož obetujeme obet Hospodinu, každý z toho, cehož jest dostal, nádobí zlaté, zápony, náramky, prsteny, náušnice a retízky, aby ocišteny byly duše naše pred Hospodinem.
\par 51 Vzal tedy Mojžíš a Eleazar knez od nich to zlato všelikého díla remeslného.
\par 52 Bylo pak všeho zlata oddeleného, kteréž obetováno Hospodinu, šestnácte tisícu, sedm set a padesáte lotu, od hejtmanu nad tisíci a od setníku.
\par 53 (Muži zajisté bojovní, což loupeží vzali, to sobe meli.)
\par 54 A vzavše Mojžíš a Eleazar knez od hejtmanu nad tisíci a setníku to zlato, vnesli je do stánku úmluvy, na památku synu Izraelských pred Hospodinem.

\chapter{32}

\par 1 Meli pak synové Ruben a synové Gád dobytka velmi mnoho, a uzreli zemi Jazer a zemi Galád, ano místo to místo príhodné pro dobytek.
\par 2 Protož pristoupivše synové Gád a synové Ruben, mluvili k Mojžíšovi a k Eleazarovi knezi a knížatum shromáždení, rkouce:
\par 3 Atarot a Dibon, a Jazer a Nemra, Ezebon a Eleale, a Saban a Nébo a Beon,
\par 4 Zeme, kterouž zbil Hospodin pred shromáždením Izraelským, jest zeme príhodná ku pastve dobytku, a my služebníci tvoji máme drahne dobytka.
\par 5 (Protož rekli:) Jestliže jsme nalezli milost pred ocima tvýma, necht jest dána krajina ta služebníkum tvým k vládarství, at nechodíme za Jordán.
\par 6 I odpovedel Mojžíš sy\par Gád a sy\par Ruben: Což bratrí vaši pujdou sami k boji, a vy zde zustanete?
\par 7 I proc roztrhujete mysli synu Izraelských, aby nesmeli jíti do zeme, kterouž jim dal Hospodin?
\par 8 Takt jsou ucinili otcové vaši, když jsem je poslal z Kádesbarne, aby prohlédli zemi tu.
\par 9 Kterížto, když prišli až k údolí Eškol a shlédli zemi, potom vrátivše se, odvrátili mysl synu Izraelských, aby nešli do zeme, kterouž dal jim Hospodin.
\par 10 Címž popuzen jsa k hnevu Hospodin v den ten, prisáhl, rka:
\par 11 Zajisté že lidé ti, kteríž vyšli z Egypta, od dvadcítiletých a výše, neuzrí zeme té, kterouž jsem s prísahou zaslíbil Abrahamovi, Izákovi a Jákobovi, nebo ne cele následovali mne,
\par 12 Krome Kálefa, syna Jefonova Cenezejského, a Jozue, syna Nun, nebo cele následovali Hospodina.
\par 13 I popudila se prchlivost Hospodinova na Izraele, a ucinil, aby byli tuláci na poušti za ctyridceti let, dokudž nezahynul všecken ten vek, kterýž cinil zlé pred ocima Hospodinovýma.
\par 14 A hle, vy nastoupili jste na místo otcu svých, pléme lidí hríšných, abyste vždy pridávali k hnevu prchlivosti Hospodinovy na Izraele.
\par 15 Jestliže se odvrátíte od následování jeho, i ont také opustí jej na poušti této, a tak budete prícina zahynutí všeho lidu tohoto.
\par 16 Pristoupivše pak znovu, rekli jemu: Stáje dobytkum a stádum svým zde vzdeláme, a mesta dítkám svým,
\par 17 Sami pak v odení pohotove budeme, statecne sobe pocínajíce pred syny Izraelskými, dokavadž jich neuvedeme na místo jejich; mezi tím zustanou dítky naše v mestech hrazených, pro bezpecnost pred obyvateli zeme.
\par 18 Nenavrátíme se do domu svých, až prvé vládnouti budou synové Izraelští jeden každý dedictvím svým;
\par 19 Aniž vezmeme jakého dedictví s nimi za Jordánem neb dále, když dosáhneme dedictví svého z této strany Jordánu, k východu slunce.
\par 20 I odpovedel jim Mojžíš: Jestliže uciníte tak, jakž jste mluvili, a jestliže pujdete v odení pred Hospodinem k boji,
\par 21 A šli byste za Jordán vy všickni v odení pred Hospodinem, dokavadž by nevyhnal neprátel svých od tvári své,
\par 22 A nebyla podmanena všecka zeme pred Hospodinem: potom navrátíte se, a budete bez viny pred Hospodinem i pred Izraelem; tak prijde zeme tato vám v dedictví pred Hospodinem.
\par 23 Pakli neuciníte toho, hle, zhrešíte proti Hospodinu, a vezte, že pomsta vaše prijde na vás.
\par 24 Stavejte sobe tedy mesta pro dítky, a stáje pro dobytky své, a což vyšlo z úst vašich, ucinte.
\par 25 I odpovedelo pokolení synu Gád a synu Ruben Mojžíšovi, rka: Služebníci tvoji uciní, jakž pán náš rozkazuje.
\par 26 Dítky naše a ženy naše, dobytek náš a všecka hovada naše, tu zustanou v mestech Galád,
\par 27 Služebníci pak tvoji prejdou jeden každý v odení zpusobný pred Hospodinem k boji, jakož mluví pán muj.
\par 28 I porucil o nich Mojžíš Eleazarovi knezi a Jozue, synu Nun, a predním v celedech pokolení synu Izraelských,
\par 29 A rekl jim: Jestliže prejdou synové Gád a synové Ruben s vámi za Jordán všickni hotovi k boji pred Hospodinem, a byla by již podmanena zeme pred vámi, dáte jim zemi Galád k vládarství.
\par 30 Pakli by nešli v odení s vámi, tedy dedictví míti budou u prostred vás v zemi Kanán.
\par 31 I odpovedeli synové Gád a synové Ruben, rkouce: Jakž mluvil Hospodin služebníkum tvým, tak uciníme:
\par 32 My pujdeme v odení pred Hospodinem do zeme Kanán, a zustane nám v dedictví vládarství naše z této strany Jordánu.
\par 33 Tedy dal jim Mojžíš, sy\par totiž Gád a sy\par Ruben a polovici pokolení Manasses, syna Jozefova, království Seona, krále Amorejského, a království Oga, krále Bázanského, zemi s mesty jejími pri pomezích, i mesta zeme té všudy vukol.
\par 34 A vzdelali synové Gád, Dibon, Atarot a Aroer,
\par 35 A Atrot, Sofan, Jazer a Jegbaa,
\par 36 A Betnemra a Betaran, mesta hrazená, a stáje pro dobytky.
\par 37 Synové pak Ruben vystaveli Ezebon, Eleale a Kariataim,
\par 38 A Nébo a Balmeon, zmenivše jim jména; také Sabma, a dali jiná jména mestum, kteráž vzdelali.
\par 39 Táhli pak synové Machir, syna Manassesova, do Galád, a vzavše tu krajinku, vyhnali Amorejského, kterýž tam bydlil.
\par 40 I dal Mojžíš zemi Galád Machirovi, synu Manassesovu, a bydlil v ní.
\par 41 Jair také syn Manassesuv táhl a vzal vsi jejich, a nazval je vsi Jairovy.
\par 42 Nobe také táhl, a vzal Kanat a mestecka jeho, a nazval je Nobe od jména svého.

\chapter{33}

\par 1 Tato jsou tažení synu Izraelských, kteríž vyšli z zeme Egyptské po houfích svých, pod spravou Mojžíše a Arona.
\par 2 Sepsal pak Mojžíš vycházení jejich podlé toho, jakž táhli k rozkazu Hospodinovu. Tato jsou tedy vycházení jejich podlé toho, jakž táhli.
\par 3 Nejprv z Ramesses jdouce prvního mesíce, v patnáctý den téhož prvního mesíce, nazejtrí po slavnosti Fáze vyšli synové Izraelští v ruce silné pred ocima všech Egyptských,
\par 4 Kdyžto Egyptští pochovávali všecky prvorozené, kteréž zbil Hospodin mezi nimi, a pri bozích jejich vykonal Hospodin soudy své.
\par 5 Hnuvše se tedy synové Izraelští z Ramesses, položili se v Sochot.
\par 6 Potom hnuvše se z Sochot, položili se v Etam, jenž jest pri kraji poušte.
\par 7 A hnuvše se z Etam, navrátili se zase k Fiarot, jenž jest pred Belsefon, a položili se pred Magdalem.
\par 8 A hnuvše se z Fiarot, šli prostredkem more na poušt, a ušedše trí dnu cesty po poušti Etam, položili se v Marah.
\par 9 Jdouce pak z Marah, prišli do Elim, kdežto bylo dvanácte studnic vod, a sedmdesáte palm. I položili se tu.
\par 10 A hnuvše se z Elim, položili se u more Rudého.
\par 11 Potom hnuvše se od more Rudého, položili se na poušti Sin.
\par 12 A když se hnuli z poušte Sin, položili se v Dafka.
\par 13 A hnuvše se z Dafka, položili se v Halus.
\par 14 Hnuvše se pak z Halus, rozbili stany v Rafidim, kdežto lid nemel vody ku pití.
\par 15 A hnuvše se z Rafidim, položili se na poušti Sinai.
\par 16 Hnuvše se pak z poušte Sinai, položili se v Kibrot Hattáve.
\par 17 A když se hnuli z Kibrot Hattáve, položili se v Hazerot.
\par 18 Hnuvše se pak z Hazerot, položili se v Retma.
\par 19 A z Retma hnuvše se, položili se v Remmon Fáres.
\par 20 Potom hnuvše se z Remmon Fáres, položili se v Lebna.
\par 21 A hnuvše se z Lebna, položili se v Ressa.
\par 22 A hnuvše se z Ressa, položili se v Cehelot.
\par 23 Z Cehelot pak hnuvše se, položili se na hore Sefer.
\par 24 A když se hnuli s hory Sefer, položili se v Arad.
\par 25 A hnuvše se z Arad, položili se v Machelot.
\par 26 Potom hnuvše se z Machelot, položili se v Tahat.
\par 27 A hnuvše se z Tahat, položili se v Tár.
\par 28 A když se hnuli z Tár, položili se v Metka.
\par 29 A hnuvše se z Metka, položili se v Esmona.
\par 30 Z Esmona pak hnuvše se, položili se v Moserot.
\par 31 A když se hnuli z Moserot, položili se v Benejakan.
\par 32 A hnuvše se z Benejakan, položili se v Chor Gidgad.
\par 33 A hnuvše se z Chor Gidgad, položili se v Jotbata.
\par 34 Když se pak hnuli z Jotbata, položili se v Habrona.
\par 35 A z Habrona hnuvše se, položili se v Aziongaber.
\par 36 A odtud hnuvše se, položili se na poušti Tsin, jenž jest Kádes.
\par 37 A hnuvše se z Kádes, položili se na hore recené Hor, pri koncinách zeme Edomské.
\par 38 Tu vstoupil Aron knez na horu, jenž slove Hor, k rozkazu Hospodinovu, a umrel tam, léta ctyridcátého po vyjití synu Izraelských z zeme Egyptské, v první den mesíce pátého.
\par 39 A byl Aron ve stu ve dvadcíti a trech letech, když umrel na hore Hor.
\par 40 Uslyšel také Kananejský král v Arad, kterýž bydlil na poledne v zemi Kananejské, že by táhli synové Izraelští.
\par 41 Tedy hnuvše se s hory Hor, položili se v Salmona.
\par 42 A hnuvše se z Salmona, položili se v Funon.
\par 43 Z Funon pak hnuvše se, položili se v Obot.
\par 44 A když se hnuli z Obot, rozbili stany pri pahrbcích hor Abarim, na pomezí Moábském.
\par 45 Potom hnuvše se od tech pahrbku, položili se v Dibongad.
\par 46 Z Dibongad hnuvše se, položili se v Helmondeblataim.
\par 47 A když se hnuli z Helmondeblataim, položili se na horách Abarim proti Nébo.
\par 48 Odšedše pak z hor Abarim, položili se na rovinách Moábských, pri Jordánu proti Jerichu.
\par 49 A rozbili stany pri Jordánu, od Betsimot až do Abelsetim, na rovinách Moábských.
\par 50 Mluvil pak Hospodin k Mojžíšovi na rovinách Moábských, pri Jordánu naproti Jerichu, rka:
\par 51 Mluv k sy\par Izraelským a rci jim: Když prejdete Jordán, a vejdete do zeme Kananejské,
\par 52 Vyžente všecky obyvatele zeme té od tvári vaší, a zkazte všecky rytiny jejich; i všecky obrazy slité jejich zkazte, všecky také výsosti jejich zborte.
\par 53 A když vyženete obyvatele zeme, bydliti budete v ní; nebo vám jsem dal tu zemi, abyste jí dedicne vládli.
\par 54 Kteroužto rozdelíte sobe k dedictví losem, vedlé celedí svých. Kterých jest více, tem vetší dedictví dáte, kterých pak jest méne, tem menší dedictví dáte. Na kterém míste komu los padne, to jemu bude; podlé pokolení otcu svých dedictví dosáhnete.
\par 55 Pakli nevyženete obyvatelu zeme od tvári své, tedy ti, kterýchž zanecháte, budou vám jako trní v ocích vašich, a jako ostnové po bocích vašich, a budou vás ssužovati na zemi, na kteréž vy bydliti budete.
\par 56 A na to prijde, abych to, což jsem jim umínil uciniti, vám ucinil.

\chapter{34}

\par 1 Mluvil také Hospodin k Mojžíšovi, rka:
\par 2 Prikaž sy\par Izraelským a rci jim: Když vejdete do zeme Kanán, (tat jest zeme, kteráž se dostane vám v dedictví, zeme Kananejská s pomezími svými),
\par 3 Strana polední vaše bude poušt Tsin vedlé pomezí Edomských; a bude vaše pomezí polední od brehu more slaného k východu.
\par 4 A zatocí se to pomezí polední k Maleakrabim, a pujde až k Tsin, a potáhne se od poledne pres Kádesbarne; a odtud vyjde ke vsi Addar, a vztáhne se až k Asmon.
\par 5 Od Asmon zatocí se pomezí to vukol až ku potoku Egyptskému, a tu se skonávati bude k západu.
\par 6 Pomezí pak západní budete míti more veliké; to bude vaše pomezí západní.
\par 7 A pomezí pulnocní toto míti budete:Od more velikého vymeríte sobe k hore recené Hor.
\par 8 Od hory Hor vymeríte sobe, až kde se vchází do Emat, a skonávati se bude pomezí to u Sedad.
\par 9 A odtud pujde pomezí to k Zefronu, a konec jeho u vsi Enan; to bude pomezí vaše k strane pulnocní.
\par 10 Vymeríte sobe také pomezí k východu od vsi Enan až do Sefama.
\par 11 A od Sefama schýlí se to pomezí až k Reblata, od východu maje Ain; a schýlí se pomezí to, a prijde k strane more Ceneret k východu.
\par 12 A vztáhne se to pomezí k Jordánu, a bude konec jeho u slaného more. Ta zeme vaše bude po svých pomezích vukol.
\par 13 Tedy oznámil to Mojžíš sy\par Izraelským, rka: Ta jest zeme, kterouž dedicne obdržíte losem, jakož prikázal Hospodin, abych ji dal devateru pokolení a polovici pokolení Manassesova.
\par 14 Nebo vzalo pokolení synu Ruben po domích otcu svých, a pokolení synu Gád po domích otcu svých, a polovice pokolení Manassesova vzali dedictví své.
\par 15 Pul tretího pokolení vzali dedictví své pred Jordánem proti Jerichu, k strane na východ slunce.
\par 16 Mluvil opet Hospodin k Mojžíšovi, rka:
\par 17 Tato jsou jména mužu, kteríž v dedictví rozdelí vám zemi: Eleazar knez, a Jozue, syn Nun.
\par 18 Kníže také jedno z každého pokolení vezmete k rozdelování dedictví zeme.
\par 19 A tato jsou jména mužu: Z pokolení Juda Kálef, syn Jefonuv;
\par 20 Z pokolení synu Simeon Samuel, syn Amiuduv;
\par 21 Z pokolení Beniaminova Helidad, syn Chaselonuv;
\par 22 Z pokolení synu Dan kníže Bukci, syn Jogli;
\par 23 Z synu Jozefových z pokolení synu Manasses kníže Haniel, syn Efoduv;
\par 24 Z pokolení synu Efraim kníže Kamuel, syn Seftanuv;
\par 25 A z pokolení synu Zabulon kníže Elizafan, syn Farnachuv;
\par 26 Z pokolení synu Izachar kníže Faltiel, syn Ozanuv;
\par 27 A z pokolení synu Asser kníže Ahiud, syn Salonuv;
\par 28 A z pokolení synu Neftalím kníže Fedael, syn Amiuduv.
\par 29 Ti jsou, jimž prikázal Hospodin, aby rozdelili zeme k dedictví sy\par Izraelským v zemi Kananejské.

\chapter{35}

\par 1 I mluvil Hospodin k Mojžíšovi na rovinách Moábských, pri Jordánu proti Jerichu, rka:
\par 2 Prikaž sy\par Izraelským, at dadí Levítum z dedictví, kterýmž vládnouti budou, mesta k bydlení, i podmestí mest vukol nich,
\par 3 Aby meli mesta k bydlení, podmestí pak jejich pro dobytky jejich, i pro statky jejich, a pro všecka hovada jejich.
\par 4 Podmestí pak mest, kteráž dáte Levítum, vzdálí budou ode zdi mestské na tisíc loktu zevnitr vukol.
\par 5 Protož vymeríte vne za každým mestem na východ slunce dva tisíce loktu, na poledne též dva tisíce loktu, také na západ dva tisíce loktu, i na pulnoci dva tisíce loktu, tak aby bylo mesto v prostredku. Ta bude míra podmestí mest jejich.
\par 6 Z tech pak mest, kteráž dáte Levítum, oddelíte šest mest k útocišti, aby tam utekl, kdož by nekoho zabil; a k tem pridáte jim ješte ctyridceti dve mesta.
\par 7 I bude všech mest, kteráž dáte Levítum, ctyridceti osm mest i s podmestími jejich.
\par 8 Tech pak mest, kteráž dáte z vládarství synu Izraelských, od tech, kteríž více mají, více vezmete, a od tech, kteríž méne mají, méne vezmete; jedno každé pokolení vedlé velikosti dedictví, jímž vládnouti budou, dá z mest svých Levítum.
\par 9 I mluvil Hospodin k Mojžíšovi, rka:
\par 10 Mluv k sy\par Izraelským a rci jim: Když prejdete Jordán, a vejdete do zeme Kananejské,
\par 11 Vybérete sobe mesta, a ta mesta budete míti k utíkání, aby tam utekl ten, kterýž by nekoho zabil z nedopatrení.
\par 12 A budou vám ta mesta k útocišti pred prítelem, aby neumrel ten, kdož zabil, dokudž by se nepostavil pred shromáždením k soudu.
\par 13 Z tech tedy mest, kteráž dáte, šest mest k útocišti míti budete.
\par 14 Tri mesta dáte pred Jordánem, též tri mesta dáte v zemi Kananejské; i budou mesta útocište.
\par 15 Sy\par Izraelským i príchozímu, i podruhu mezi nimi bude tech šest mest k útocišti, aby tam utekl, kdož by koli ranil nekoho z nedopatrení.
\par 16 Jestliže by pak železem ranil nekoho, tak až by umrel, vražedlník jest; smrtí umre vražedlník takový.
\par 17 Pakli by hode kamenem, jímž by mohl zabiti, uderil nekoho, tak že by umrel, vražedlník jest; smrtí umre vražedlník takový.
\par 18 Pakli by hode drevem, kterýmž by mohl zabiti, uderil nekoho, tak že by umrel, vražedlník jest; smrtí umre vražedlník takový.
\par 19 Prítel zabitého zabije vražedlníka toho; kdyžkoli ho dostane, on sám zabije ho.
\par 20 Aneb jestliže by z nenávisti strcil nekým, aneb shodil by neco na neho z úkladu, tak že by od toho umrel;
\par 21 Aneb jestliže by z neprátelství rukou uderil nekoho, tak že by umrel: smrtí umre bitec ten, vražedlník jest; prítel zabitého zabije vražedlníka toho, jakž ho nejprv dostane.
\par 22 Jestliže by pak náhodou a ne z neprátelství strcil nekým, aneb shodil by na neho nejakou vec bez úkladu;
\par 23 Aneb jaký koli kámen, od nehož by umríti mohl, shodil by na nej z nedopatrení, tak že by umrel, nebyv s ním v neprátelství, ani nehledaje zlého jeho:
\par 24 Tedy souditi bude shromáždení mezi bitcem a mezi prítelem zabitého vedlé soudu techto.
\par 25 A vysvobodí shromáždení vražedlníka toho z rukou prítele zabitého, a káže se jemu navrátiti shromáždení k mestu útocište jeho, do nehož utekl; i bude bydliti v nem, dokudž neumre knez nejvyšší, kterýž pomazán jest olejem svatým.
\par 26 Jestliže by pak ten, kterýž zabil cloveka, vyšel z mezí mesta útocište svého, do nehož utekl,
\par 27 A prítel zabitého našel by jej vne, an prešel meze mesta útocište svého, a zabil by prítel zabitého vražedlníka toho, nebude vinen krví.
\par 28 Nebo v meste útocište svého bydliti má, dokudž by neumrel knez nejvyšší. Když by pak umrel knez nejvyšší, navrátí se vražedlník do zeme vládarství svého.
\par 29 A bude vám toto za ustanovení soudné v pronárodech vašich, ve všech príbytcích vašich.
\par 30 Kdož by koli mel na smrt vydati nekoho, podlé vyznání svedku sáhne na vražedlníka; ale jeden svedek nebude moci svedciti proti nekomu na smrt.
\par 31 Nevezmete pak výplaty za cloveka vražedlníka, kterýž, jsa nešlechetný, jest smrti hoden, než smrtí at umre.
\par 32 Aniž také vezmete výplaty od toho, kterýž utekl do mesta útocište svého, aby se navrátil k bydlení do zeme své, prvé než by umrel knez,
\par 33 Abyste nepoškvrnili zeme, v níž jste. Nebo krev taková poškvrnila by zeme, aniž také zeme ocištena býti muže od krve, kteráž jest vylita na ní, jediné krví toho, kterýž vylil ji.
\par 34 Protož nepoškvrnujte zeme, v kteréž bydlíte, kdežto já prebývám; nebo já Hospodin prebývám u prostred synu Izraelských.

\chapter{36}

\par 1 Pristoupili pak starší z celedi synu Galád, syna Machir, syna Manassesova, z celedi synu Jozefových, a mluvili pred Mojžíšem a pred knížaty predními z synu Izraelských,
\par 2 A rekli: Tobe pánu mému prikázal Hospodin, abys losem dal zemi v dedictví sy\par Izraelským; prikázáno jest také pánu mému od Hospodina, aby dal dedictví Salfada, bratra našeho, dcerám jeho.
\par 3 Kteréž jestliže nekomu z pokolení synu Izraelských, krome z pokolení svého, dány budou za manželky, odejde dedictví jejich od dedictví otcu našich, a pridáno bude k dedictví pokolení toho, do kteréhož by se vdaly, a tak z losu dedictví našeho ubude.
\par 4 A když budou míti synové Izraelští léto milostivé, pripojeno bude dedictví jejich k dedictví pokolení toho, do kteréhož by se vdaly, a tak od dedictví pokolení otcu našich odtrženo bude dedictví jejich.
\par 5 Tedy prikázal Mojžíš sy\par Izraelským podlé reci Hospodinovy, rka: Dobre pokolení synu Jozefových mluví.
\par 6 To jest, což prikázal Hospodin o dcerách Salfadových, rka: Za kohož se jim líbiti bude, necht se vdadí, však v celedi domu otce svého at se vdávají,
\par 7 Aby nebylo prenášíno dedictví synu Izraelských z pokolení na pokolení; nebo synové Izraelští jeden každý prídržeti se bude dedictví pokolení otcu svých.
\par 8 A každá dcera z pokolení synu Izraelských, kteráž by mela dedictví, za nekoho z celedi pokolení otce svého vdá se, aby vládli synové Izraelští jeden každý dedictvím otcu svých,
\par 9 Aby nebylo prenášíno vládarství z jednoho pokolení na druhé pokolení, ale jeden každý z pokolení synu Izraelských dedictví svého prídržeti se bude.
\par 10 Jakož prikázal Hospodin Mojžíšovi, tak ucinily dcery Salfadovy.
\par 11 Nebo Mahla, Tersa a Hegla, Melcha a Noa, dcery Salfad, vdaly se za syny strýcu svých.
\par 12 Do celedi synu Manasse, syna Jozefova, vdaly se, a zustalo dedictví jejich pri pokolení celedi otce jejich.
\par 13 Tato jsou prikázaní a soudové, kteréž prikázal Hospodin skrze Mojžíše sy\par Izraelským, na rovinách Moábských, pri Jordánu proti Jerichu.

\end{document}