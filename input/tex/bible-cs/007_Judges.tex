\begin{document}

\title{Soudců}

\chapter{1}

\par 1 Stalo se pak po smrti Jozue, otázali se synové Izraelští Hospodina, rkouce: Kdo z nás potáhne proti Kananejskému napred, aby bojoval s ním?
\par 2 Jimž rekl Hospodin: Juda necht táhne, aj, dal jsem tu zemi v ruku jeho.
\par 3 (Rekl pak byl Juda Simeonovi bratru svému: Potáhni se mnou k dobývání losu mého, abychom bojovali proti Kananejskému, a já také potáhnu s tebou k dobývání losu tvého. I táhl s ním Simeon.
\par 4 Tedy vytáhl Juda, i dal Hospodin Kananejského a Ferezejského v ruce jejich, a porazili z nich v Bezeku deset tisíc mužu.
\par 5 Nebo nalezše Adonibezeka v Bezeku, bojovali proti nemu, a porazili Kananejského i Ferezejského.
\par 6 Když pak utíkal Adonibezek, honili ho, a chytivše jej, zutínali mu palce u rukou i noh.
\par 7 Tedy rekl Adonibezek: Sedmdesáte králu s utatými palci u rukou i noh svých sbírali drobty pod stolem mým; jakž jsem cinil, tak odplatil mi Buh. I privedli jej do Jeruzaléma, a tam umrel.
\par 8 Nebo byli vybojovali synové Juda Jeruzalém, a vzavše jej, zbili obyvatele jeho ostrostí mece a mesto vypálili.
\par 9 Potom také vytáhli synové Judovi, aby bojovali proti Kananejskému, bydlícímu na horách pri strane polední i na rovinách.
\par 10 Nebo byl vytáhl Juda proti Kananejskému, kterýž bydlil v Hebronu, (jméno pak Hebronu prvé bylo Kariatarbe,) a pobil Sesaie, a Achimana, a Tolmaie.
\par 11 A odtud byl táhl na obyvatele Dabir, (jméno pak Dabir prvé bylo Kariatsefer).
\par 12 Kdežto rekl Kálef: Kdo by dobyl Kariatsefer a vzal by je, dám jemu Axu dceru svou za manželku.
\par 13 Tedy dobyl ho Otoniel, syn Cenezuv, bratra Kálefova mladšího, i dal jemu Axu dceru svou za manželku.
\par 14 Stalo se pak, že když prišla k nemu, ponukla ho, aby prosil otce jejího za pole; i ssedla s osla. A rekl jí Kálef: Cožt jest?
\par 15 A ona odpovedela: Dej mi dar; ponevadžs mi dal zemi suchou, dej mi také studnice vod. I dal jí Kálef studnice v horních i dolních koncinách.
\par 16 Synové také Cinejského, tchána Mojžíšova, odebrali se z mesta palmového s syny Juda na poušt Judovu, jenž jest k strane polední mestu Arad; a odšedše, bydlili s lidem tím.
\par 17 Potom táhl Juda s Simeonem, bratrem svým, a porazili Kananejské prebývající v Sefat, a zkazili je. I nazváno jest jméno mesta toho Horma.
\par 18 Dobyl také Juda Gázy a pomezí jeho, i Aškalonu s pomezím jeho, též Akaronu a pomezí jeho.
\par 19 Nebo byl Hospodin s Judou, a vyhnal obyvatele hor, ale nevyhnal obyvatelu údolí, proto že vozy železné meli.
\par 20 I dali Kálefovi Hebron, jakož byl mluvil Mojžíš, a vyhnal odtud tri syny Enakovy.
\par 21 Jebuzejského pak, bydlícího v Jeruzaléme, nevyhnali synové Beniamin; protož bydlil Jebuzejský v Jeruzaléme s syny Beniamin až do tohoto dne.
\par 22 Vytáhla také i celed Jozefova do Bethel, a Hospodin byl s nimi.
\par 23 Nebo shlédla celed Jozefova Bethel, kteréhožto mesta jméno prvé bylo Luza.
\par 24 Uzrevše pak ti špehéri muže vycházejícího z mesta, rekli jemu: Medle ukaž nám, kudy bychom mohli vjíti do mesta, a ucinímet milost.
\par 25 Kterýžto ukázal jim, kudy by mohli vjíti do mesta; i vyhubili to mesto mecem, muže pak toho se vší celedí jeho propustili.
\par 26 I šel muž ten do zeme Hetejských, kdež vystavel mesto, a nazval jméno jeho Luza; to jest jméno jeho až do tohoto dne.
\par 27 Manasses také nevyhnal obyvatelu Betsan a mestecek jeho, ani Tanach a mestecek jeho, ani obyvatelu Dor a Jibleam, a Mageddo a mestecek jejich; i pocal Kananejský svobodne bydliti v zemi té.
\par 28 Když se pak zsilil Izrael, uvedl Kananejského pod plat, a maje jej vyhnati, nevyhnal.
\par 29 Efraim také nevyhnal Kananejského bydlícího v Gázer, protož bydlil Kananejský u prostred neho v Gázer.
\par 30 Zabulon též nevyhnal obyvatelu Cetron, a obyvatelu Naalol, protož bydlil Kananejský u prostred neho, a dával jemu plat.
\par 31 Asser také nevyhnal obyvatelu Acho a obyvatelu Sidonu, ani Ahalab, ani Achzib, ani Helba, ani Afek, ani Rohob.
\par 32 I bydlil Asser mezi Kananejskými obyvateli zeme té, nebo nevyhnal jich.
\par 33 Též Neftalím nevyhnal obyvatelu Betsemes, ani obyvatelu Betanat, protož bydlil mezi Kananajeskými prebývajícími v zemi té; a však obyvatelé Betsemes a Betanat dávali jim plat.
\par 34 Ssužovali pak Amorejští syny Dan na horách, tak že nedali jim scházeti do údolí.
\par 35 Nebo pocal Amorejský svobodne bydliti na hore Heres, v Aialon a v Salbim, ale když se zsilila ruka celedi Jozefovy, uvedeni jsou pod plat.
\par 36 Pomezí pak Amorejského bylo od zacátku hor Akrabim, od skály jejich i výše.

\chapter{2}

\par 1 Vstoupil pak andel Hospodinuv z Galgala do Bochim a rekl: Vyvedl jsem vás z Egypta a uvedl jsem vás do zeme, kterouž jsem s prísahou zaslíbil otcum vašim, a rekl jsem: Nezruším smlouvy své s vámi na veky.
\par 2 Vy také necinte smlouvy s obyvateli zeme této, oltáre jejich rozkopejte; ale neposlechli jste hlasu mého. Co jste to ucinili?
\par 3 Procež také jsem rekl: Nevyhladím jich pred tvárí vaší, ale budou vám jako trní, a bohové jejich budou vám osídlem.
\par 4 I stalo se, když mluvil andel Hospodinuv slova tato všechnem synum Izraelským, že pozdvihl hlasu svého lid a plakal.
\par 5 I nazvali jméno místa toho Bochim, a obetovali tu Hospodinu.
\par 6 Rozpustil pak byl Jozue lid, a rozešli se synové Izraelští, jeden každý do dedictví svého, aby vládli zemí.
\par 7 I sloužil lid Hospodinu po všecky dny Jozue, a po všecky dny starších, kteríž dlouho živi byli po Jozue, jenž videli všecky skutky veliké Hospodinovy, kteréž ucinil Izraelovi.
\par 8 Ale když umrel Jozue syn Nun, služebník Hospodinuv, jsa ve stu a desíti letech,
\par 9 A pochovali ho v krajine dedictví jeho, v Tamnatheres, na hore Efraim, k strane pulnocní hory Gás;
\par 10 Také když všecken vek ten pripojen jest k otcum svým, a povstal jiný vek po nich, kteríž neznali Hospodina, ani skutku, kteréž ucinil Izraelovi:
\par 11 Tedy cinili synové Izraelští to, což jest zlého pred ocima Hospodinovýma, a sloužili modlám,
\par 12 Opustivše Hospodina Boha otcu svých, kterýž je vyvedl z zeme Egyptské, a odešli za bohy cizími, bohy tech národu, kteríž byli vukol nich, a klaneli se jim; procež popudili Hospodina.
\par 13 Nebo opustivše Hospodina, sloužili Bálovi i Astarotum.
\par 14 I rozpálila se prchlivost Hospodinova na Izraele, a vydal je v ruku loupežníku, kteríž je zloupili; vydal je, pravím, v ruku neprátel jejich vukol, tak že nemohli více ostáti pred neprátely svými.
\par 15 Kamžkoli vycházeli, ruka Hospodinova byla proti nim ke zlému, jakož byl mluvil Hospodin, a jakož byl zaprisáhl jim Hospodin; i ssouženi byli náramne.
\par 16 Vzbuzoval pak Hospodin soudce, kteríž vysvobozovali je z rukou zhoubcu jejich.
\par 17 Ale ani soudcu svých neposlouchali, nebo smilnili, odcházejíce za bohy cizími, a klaneli se jim. Odcházeli rychle s cesty, po kteréž chodili otcové jejich, tak že poslouchati majíce prikázaní Hospodinových, necinili toho.
\par 18 A když vzbuzoval jim Hospodin soudce, býval Hospodin s každým soudcím, a vysvobozoval je z ruky neprátel jejich po všecky dny soudce; (nebo želel Hospodin naríkání jejich, k nemuž je privodili ti, kteríž je ssužovali a utiskali).
\par 19 Po smrti pak soudce navracejíce se zase, pohoršovali cest svých více nežli otcové jejich, odcházejíce za bohy cizími, a sloužíce i klanejíce se jim; nic neulevili z skutku jejich zlých a cesty jejich prevrácené.
\par 20 Protož rozpálila se prchlivost Hospodinova proti Izraelovi, a rekl: Proto že prestoupil národ tento smlouvu mou, kterouž jsem ucinil s otci jejich, a neposlouchali hlasu mého:
\par 21 Já také více nevyhladím žádného od tvári jejich z národu, kterýchž zanechal Jozue, když umrel,
\par 22 Abych skrze ne zkušoval Izraele, budou-li ostríhati cesty prikázaní Hospodinových, chodíce v nich, jakož ostríhali otcové jejich, cili nic.
\par 23 I zanechal Hospodin národu tech, a nevyhnal jich rychle, aniž dal jich v ruku Jozue.

\chapter{3}

\par 1 Tito pak jsou národové, kterýchž zanechal Hospodin, aby skrze ne zkušoval Izraele,totiž všech, kteríž nevedeli o žádných válkách Kananejských,
\par 2 Aby aspon zvedeli vekové synu Izraelských, a poznali, co jest to válka, cehož první nevedeli:
\par 3 Patero knížat Filistinských, a všickni Kananejští a Sidonští a Hevejští bydlící na hore Libánské od hory Balhermon až tam, kudy se vchází do Emat.
\par 4 Ti pozustali, aby zkušován byl skrze ne Izrael, a aby známé bylo, budou-li poslouchati prikázaní Hospodinových, kteráž prikázal otcum jejich skrze Mojžíše.
\par 5 Bydlili tedy synové Izraelští u prostred Kananejských, Hetejských a Amorejských, a Ferezejských a Hevejských a Jebuzejských,
\par 6 A brali sobe dcery jejich za manželky, a dcery své dávali synum jejich, a sloužili bohum jejich.
\par 7 Cinili tedy synové Izraelští to, což jest zlého pred ocima Hospodinovýma, a zapomenuvše se na Hospodina Boha svého, sloužili Bálum a Asserotum.
\par 8 Protož rozpálila se prchlivost Hospodinova na Izraele, a vydal jej v ruku Chusana Risataimského, krále Syrského v Mezopotamii; i sloužili synové Izraelští Chusanovi Risataimskému osm let.
\par 9 Volali pak synové Izraelští k Hospodinu, i vzbudil Hospodin vysvoboditele synum Izraelským, aby je vysvobodil, Otoniele syna Cenezova bratra Kálefova mladšího,
\par 10 Na nemž byl duch Hospodinuv, a soudil lid Izraelský. Když pak vytáhl k boji, dal Hospodin v ruce jeho Chusana Risataimského, krále Syrského, a zmocnila se ruka jeho nad Chusanem Risataimským.
\par 11 A tak v pokoji byla zeme za ctyridceti let; i umrel Otoniel, syn Cenezuv.
\par 12 Takž opet cinili synové Izraelští to, což jest zlého pred ocima Hospodinovýma. I zsilil Hospodin Eglona, krále Moábského, proti Izraelovi, proto že cinili zlé veci pred ocima Hospodinovýma.
\par 13 Nebo shromáždil k sobe syny Ammonovy a Amalechovy, a vytáh, porazil Izraele, a opanovali mesto palmové.
\par 14 I sloužili synové Izraelští Eglonovi králi Moábskému osmnácte let.
\par 15 Potom volali synové Izraelští k Hospodinu. I vzbudil jim Hospodin vysvoboditele Ahoda, syna Gery Beniaminského, muže rukou pravou nevládnoucího. I poslali synové Izraelští po nem dar Eglonovi králi Moábskému.
\par 16 (Pripravil pak sobe Ahod mec na obe strane ostrý, lokte zdélí, a pripásal jej sobe pod šaty svými po pravé strane.)
\par 17 I prinesl dar Eglonovi králi Moábskému. Eglon pak byl clovek velmi tlustý.
\par 18 A když dodal daru, propustil lid, kterýž byl prinesl dar.
\par 19 Sám pak vrátiv se od lomu blízko Galgala, rekl: Tajnou vec mám k tobe, ó králi. I rekl král: Mlc. A vyšli od neho všickni, kteríž stáli pri nem.
\par 20 Tehdy Ahod pristoupil k nemu, (on pak sedel na paláci letním sám). I rekl mu Ahod: Rec Boží mám k tobe. I vstal z stolice své.
\par 21 Tedy sáhna Ahod rukou svou levou, vzal mec u pravého boku svého a vrazil jej do bricha jeho,
\par 22 Tak že i jilce za ostrím vešly tam, a zavrel se tukem mec, (nebo nevytáhl mece z bricha jeho;) i vyšla lejna.
\par 23 Potom vyšel Ahod pres sín, a zavrel dvére paláce po sobe, a zamkl.
\par 24 Když pak on odšel, služebníci jeho prišedše, uzreli, a hle, dvére palácu zamcíny. Tedy rekli: Jest na potrebe v pokoji letním.
\par 25 A když ocekávali dlouho, až se stydeli, a aj, neotvíral dverí paláce, tedy vzali klíc a otevreli, a hle, pán jejich leží na zemi mrtvý.
\par 26 Ahod pak mezi tím, když oni prodlévali, ušel, a prešed lomy, prišel do Seirat.
\par 27 A prišed k svým, troubil v troubu na hore Efraim; i sstoupili s ním synové Izraelští s hory, a on napred šel.
\par 28 Nebo rekl jim: Podte za mnou, dalt jest zajisté Hospodin Moábské, neprátely vaše, v ruku vaši. Tedy táhli za ním, a vzavše Moábským brody Jordánské, nedali žádnému prejíti.
\par 29 I pobili tehdáž Moábských okolo desíti tisíc mužu, každého bohatého a všelikého silného muže, aniž kdo ušel.
\par 30 I snížen jest Moáb v ten den pod mocí Izraele, a pokoj mela zeme za osmdesáte let.
\par 31 Po nem pak byl Samgar, syn Anatuv, a pobil Filistinských šest set mužu ostnem volu, a vysvobodil i on Izraele.

\chapter{4}

\par 1 Po smrti pak Ahoda cinili opet synové Izraelští to, což jest zlého pred ocima Hospodinovýma.
\par 2 I vydal je Hospodin v ruce Jabína krále Kananejského, kterýž kraloval v Azor; (a kníže vojska jeho byl Zizara), sám pak bydlil v Haroset pohanském.
\par 3 Tedy volali synové Izraelští k Hospodinu; nebo devet set vozu železných mel, a on násilne ssužoval Izraele za dvadceti let.
\par 4 Debora pak žena prorokyne, manželka Lapidotova, soudila lid Izraelský toho casu.
\par 5 (A bydlila Debora pod palmou, mezi Ráma a mezi Bethel na hore Efraim), i chodili k ní synové Izraelští k soudu.
\par 6 Kterážto poslavši, povolala Baráka, syna Abinoemova, z Kádes Neftalímova, a rekla jemu: Zdaližt nerozkázal Hospodin Buh Izraelský: Jdi, a shromážde lid na horu Tábor, vezmi s sebou deset tisíc mužu z synu Neftalím a z synu Zabulon.
\par 7 Nebo pritáhnu k tobe ku potoku Císon Zizaru kníže vojska Jabínova, a vozy jeho i množství jeho, a dám jej v ruku tvou.
\par 8 I rekl jí Barák: Pujdeš-li se mnou, pujdu; pakli nepujdeš se mnou, nepujdu.
\par 9 Kteráž odpovedela: Ját zajisté pujdu s tebou, ale nebudet s slávou tvou cesta, kterouž pujdeš, nebo v ruku ženy dá Hospodin Zizaru. Tedy vstavši Debora, šla s Barákem do Kádes.
\par 10 Svolav pak Barák Zabulonské a Neftalímské do Kádes, vyvedl za sebou deset tisíc mužu; a šla s ním i Debora.
\par 11 Heber pak Cinejský oddelil se od Kaina, od synu Chobab, tchána Mojžíšova, a rozbil stany své až k Elon v Sananim, jenž jest v Kádes.
\par 12 Oznámeno pak bylo Zizarovi, že vytáhl Barák syn Abinoemuv na horu Tábor.
\par 13 Protož shromáždil Zizara všecky vozy své, devet set vozu železných, a všecken lid, kterýž mel s sebou z Haroset pohanského, ku potoku Císon.
\par 14 I rekla Debora Barákovi: Vstan, nebo tento jest den, v nemž dal Hospodin Zizaru v ruku tvou. Zdali Hospodin nevyšel pred tebou? I sstoupil Barák s hory Tábor, a deset tisíc mužu za ním.
\par 15 A porazil Hospodin Zizaru a všecky ty vozy, i všecka ta vojska ostrostí mece pred Barákem; a sskociv Zizara s vozu, utíkal pešky.
\par 16 Ale Barák honil ty vozy a vojsko až do Haroset pohanského; i padlo všecko vojsko Zizarovo od ostrosti mece, tak že nezustalo ani jednoho.
\par 17 Zizara pak utíkal pešky k stanu Jáhel, manželky Hebera Cinejského; nebo pokoj byl mezi Jabínem králem Azor a mezi celedí Hebera Cinejského.
\par 18 I vyšedši Jáhel v cestu Zizarovi, rekla jemu: Uchyl se, pane muj, uchyl se ke mne, neboj se. I uchýlil se k ní do stanu, a pristrela jej huní.
\par 19 Kterýž rekl jí: Dej mi, prosím, píti malicko vody, nebo žízním. A otevrevši nádobu mlécnou, dala mu píti a prikryla ho.
\par 20 Rekl také jí: Stuj u dverí stanu, a pri-šel-li by kdo, a ptal se tebe, rka: Jest-li zde kdo? odpovíš: Není.
\par 21 Potom vzala Jáhel manželka Heberova hreb od stanu, vzala též kladivo v ruku svou, a všedši k nemu tiše, vrazila hreb do židovin jeho, až uvázl v zemi; (nebo ustav, tvrde byl usnul), a tak umrel.
\par 22 A aj, Barák honil Zizaru. I vyšla Jáhel vstríc jemu, a rekla mu: Pod, a ukážit muže, kteréhož hledáš. I všel k ní, a aj, Zizara ležel mrtvý na zemi, a hreb v židovinách jeho.
\par 23 A tak ponížil Buh toho dne Jabína krále Kananejského pred syny Izraelskými.
\par 24 I dotírala ruka synu Izraelských vždy více, a silila se proti Jabínovi králi Kananejskému, až i vyhladili téhož Jabína krále Kananejského.

\chapter{5}

\par 1 Zpívala pak písnicku Debora a Barák syn Abinoemuv v ten den, rkouc:
\par 2 Pro pomstu ucinenou v Izraeli, a pro lid, kterýž se k tomu dobrovolne mel, dobrorecte Hospodinu.
\par 3 Slyštež králové, a ušima pozorujte knížata, já, já zpívati budu Hospodinu, žalmy zpívati budu Hospodinu Bohu Izraelskému.
\par 4 Hospodine, když jsi vyšel z Seir, když jsi se bral z pole Edomského, trásla se zeme, nebesa dštila, a oblakové déšt vydali.
\par 5 Hory se rozplynuly od tvári Hospodinovy, i ta hora Sinai trásla se pred tvárí Hospodina Boha Izraelského.
\par 6 Za dnu Samgara syna Anatova, a za dnu Jáhel spustly silnice, kteríž pak šli stezkami, zacházeli cestami krivými.
\par 7 Spustly vsi v Izraeli, spustly, pravím, až jsem povstala já Debora, povstala jsem matka v Izraeli.
\par 8 Kterýžto kdyžkoli sobe zvoloval bohy nové, tedy bývala válka v branách, pavézy pak ani kopí nebylo vidíno mezi ctyridcíti tisíci v Izraeli.
\par 9 Srdce mé nakloneno jest k správcum Izraelským a k tem, kteríž tak ochotní byli mezi jinými. Dobrorectež Hospodinu.
\par 10 Kteríž jezdíte na bílých oslicích, kteríž bydlíte pri Middin, a kteríž chodíte po cestách, vypravujtež,
\par 11 Že vzdálen hluk strelcu na místech, kdež se voda váží; i tam vypravujte hojnou spravedlnost Hospodinovu, hojnou spravedlnost k obyvatelum vsí jeho v Izraeli; tehdážt vstupovati bude k branám lid Hospodinuv.
\par 12 Povstan, povstan, Deboro, povstaniž, povstaniž a vypravuj písen, povstan, Baráku, a zajmi jaté své, synu Abinoemuv.
\par 13 Tehdážte potlacenému dopomoženo k opanování silných reku z lidu; Hospodinte mi ku panování dopomohl nad silnými.
\par 14 Z Efraima koren jejich bojoval proti Amalechitským; za tebou, Efraime, Beniamin s lidem tvým; z Machira táhli vydavatelé zákona, a z Zabulona písari.
\par 15 Knížata také z Izachar s Deborou, ano i všecko pokolení Izacharovo, jako i Barák do údolí poslán jest pešky, ale veliké hrdiny u sebe jsou v podílu Rubenovu.
\par 16 Jak jsi mohl mlce sedeti mezi dvema ohradami, poslouchaje rvání stád? Veliké hrdiny u sebe jsou v podílu Rubenovu.
\par 17 Zdali i Galád pred Jordánem nebydlil? Ale Dan proc zustal pri lodech? Asser sedel na brehu morském, a v lomích svých bydlil.
\par 18 Zabulon, lid udatný, vynaložil duši svou na smrt, též i Neftalím na vysokých místech pole.
\par 19 Králové pritáhše, bojovali, tehdáž bojovali králové Kananejští v Tanach pri vodách Mageddo, a však koristi stríbra nevzali.
\par 20 S nebe bojováno, hvezdy z míst svých bojovaly proti Zizarovi.
\par 21 Potok Císon smetl je, potok Kedumim, potok Císon; všecko to pošlapala jsi, duše má, udatne.
\par 22 Tehdáž otloukla se kopyta konu od dupání velikého pod jezdci silnými.
\par 23 Zlorecte Merozu, praví andel Hospodinuv, zlorecte velice obyvatelum jeho, nebo neprišli na pomoc Hospodinu, ku pomoci Hospodinu proti silným.
\par 24 Požehnaná bud nad jiné ženy Jáhel, manželka Hebera Cinejského, nad ženy v staních bydlící bud požehnaná.
\par 25 On vody žádal, ona mléka dala, v koflíku knížecím podala másla.
\par 26 Levou ruku svou k hrebu vztáhla, a pravou ruku svou k kladivu delníku, i uderila Zizaru, a ztloukla hlavu jeho, probodla a prorazila židoviny jeho.
\par 27 U noh jejích skrcil se, padl, ležel, u noh jejích skrcil se, padl; kdež se skrcil, tu padl zabitý.
\par 28 Vyhlídala z okna skrze mríži, a naríkala matka Zizarova, rkuci: Proc se tak dlouho vuz jeho nevrací? Proc prodlévají vraceti se domu vozové jeho?
\par 29 Moudrejší pak z predních služebnic jejích odpovídaly, i ona sama také sobe odpovídala:
\par 30 Zdali ale dosáhli neceho, a delí koristi? Devecku jednu neb dve na každého muže, loupeže rozdílných barev samému Zizarovi, koristi rozdílných barev krumpovaným dílem, roucho rozdílných barev krumpovaným dílem na hrdlo loupežníku.
\par 31 Tak at zahynou všickni neprátelé tvoji, ó Hospodine, tebe pak milující at jsou jako slunce vzcházející v síle své. I byla v pokoji zeme za ctyridceti let.

\chapter{6}

\par 1 Cinili pak synové Izraelští to, což jest zlého pred ocima Hospodinovýma, i vydal je Hospodin v ruku Madianským za sedm let.
\par 2 A zmocnila se velmi ruka Madianských nad Izraelem, tak že synové Izraelští pred Madianskými zdelali sobe jámy, kteréž jsou na horách, a jeskyne a pevnosti.
\par 3 A bývalo, že jakž co oseli Izraelští, pritáhl Madian a Amalech a národ východní, povstávaje proti nim.
\par 4 Kterížto rozbijeli stany proti nim, a kazili úrody zeme, až kudy se vchází do Gázy, a nenechávali potravy v Izraeli, ani ovce, ani vola, ani osla.
\par 5 Nebo i sami i s stády svými i s stany pritáhli, a jako kobylky ve množství pricházívali, aniž jich neb velbloudu jejich byl pocet; tak pricházejíce do zeme, hubili ji.
\par 6 Tedy znuzen byl velmi Izrael od Madianských, protož volali synové Izraelští k Hospodinu.
\par 7 Když pak volali synové Izraelští k Hospodinu prícinou Madianských,
\par 8 Poslal Hospodin muže proroka k synum Izraelským, a rekl jim: Takto praví Hospodin Buh Izraelský: Já jsem vás vyvedl z Egypta, a vyvedl jsem vás z domu služby,
\par 9 A vytrhl jsem vás z rukou Egyptských a z rukou všech, jenž vás ssužovali, kteréž jsem vyhnal pred tvárí vaší, a dal jsem vám zemi jejich.
\par 10 I rekl jsem vám: Já jsem Hospodin Buh váš, nebojte se bohu Amorejských, v jejichž zemi bydlíte. Ale neposlechli jste hlasu mého.
\par 11 Tedy prišel andel Hospodinuv a posadil se pod dubem, jenž byl v Ofra, kteréž bylo Joasa Abiezeritského. Gedeon pak syn jeho mlátil obilí na humne, aby vezma, utekl s tím pred Madianskými.
\par 12 I ukázal se jemu andel Hospodinuv, a rekl jemu: Hospodin s tebou, muži udatný.
\par 13 Odpovedel jemu Gedeon: Ó Pane muj, jestliže Hospodin s námi jest, procež pak na nás prišlo všecko toto? A kde jsou všickni divové jeho, o kterýchž vypravovali nám otcové naši, rkouce: Zdaliž nás z Egypta nevyvedl Hospodin? A nyní opustil nás Hospodin a vydal v ruku Madianských.
\par 14 A pohledev na nej Hospodin, rekl: Jdi v této síle své, a vysvobodíš Izraele z ruky Madianských. Zdaliž jsem te neposlal?
\par 15 Kterýž odpovedel jemu: Poslyš mne, Pane muj, cím já vysvobodím Izraele? Hle, rod muj chaterný jest v pokolení Manassesovu, a já nejmenší v dome otce svého.
\par 16 Tedy rekl jemu Hospodin: Ponevadž já budu s tebou, protož zbiješ Madianské jako muže jednoho.
\par 17 Jemuž on rekl: Prosím, jestliže jsem nalezl milost pred ocima tvýma, dej mi znamení, že ty mluvíš se mnou.
\par 18 Prosím, neodcházej odsud, až zase prijdu k tobe, a vynesu obet svou a položím pred tebou. Odpovedel on: Ját poseckám, až se navrátíš.
\par 19 Tedy Gedeon všed do domu, pripravil kozelce, a po merici mouky presných chlebu. I vložil maso do koše, a polívku vlil do hrnce, i vynesl to k nemu pod dub a obetoval.
\par 20 I rekl jemu andel Boží: Vezmi to maso a chleby ty nekvašené, a polož na tuto skálu, a polívkou polej. I ucinil tak.
\par 21 Potom zdvihl andel Hospodinuv konec holi, kterouž mel v rukou svých, a dotekl se masa a chlebu presných; i vystoupil ohen z skály té, a spálil to maso i chleby presné. A v tom andel Hospodinuv odšel od ocí jeho.
\par 22 A vida Gedeon, že by andel Hospodinuv byl, rekl: Ach, Panovníce Hospodine, proto-liž jsem videl andela Hospodinova tvárí v tvár, abych umrel?
\par 23 I rekl jemu Hospodin: Mej pokoj, neboj se, neumreš.
\par 24 Protož vzdelal tu Gedeon oltár Hospodinu, a nazval jej: Hospodin pokoje, a až do tohoto dne ješte jest v Ofra Abiezeritského.
\par 25 Stalo se pak té noci, že rekl jemu Hospodin: Vezmi volka vyspelého, kterýž jest otce tvého, totiž volka toho druhého sedmiletého, a zboríš oltár Báluv, kterýž jest otce tvého, háj také, kterýž jest vedlé neho, posekáš.
\par 26 A vzdeláš oltár Hospodinu Bohu svému na vrchu skály této, na rovine její, a vezma volka toho druhého, obetovati budeš obet zápalnou s drívím háje, kterýž posekáš.
\par 27 Tedy vzal Gedeon deset mužu z služebníku svých, a ucinil, jakž mluvil jemu Hospodin; a boje se celedi otce svého a mužu mesta, neucinil toho ve dne, ale v noci.
\par 28 Když pak vstali muži mesta ráno, uzreli zborený oltár Báluv, a háj, kterýž byl vedlé neho, že byl posekaný, a volka toho druhého obetovaného v obet zápalnou na oltári v nove vzdelaném.
\par 29 I mluvili jeden k druhému: Kdo to udelal? A vyhledávajíce, vyptávali se a pravili: Gedeon syn Joasuv to ucinil.
\par 30 Tedy rekli muži mesta k Joasovi: Vyved syna svého, at umre, proto že zboril oltár Báluv, a že posekal háj, kterýž byl vedlé neho.
\par 31 Odpovedel Joas všechnem stojícím pred sebou: A což se vy nesnadniti chcete o Bále? Zdali vy jej vysvobodíte? Kdož by se zasazoval o nej, at jest zabit hned jitra tohoto. Jestližet jest bohem, nechat se sám zasazuje o to, že jest rozboren oltár jeho.
\par 32 I nazval jej toho dne Jerobálem, rka: Necht se zasadí Bál proti nemu, že zboril oltár jeho.
\par 33 Tedy všickni Madianští a Amalechitští a národové východní shromáždili se spolu, a prešedše Jordán, položili se v údolí Jezreel.
\par 34 Duch pak Hospodinuv posilnil Gedeona, kterýžto zatroubiv v troubu, svolal Abiezeritské k sobe.
\par 35 A poslal posly ke všemu pokolení Manassesovu, a svoláno jest k nemu; poslal též posly k Asserovi, k Zabulonovi a k Neftalímovi, i pritáhli jim na pomoc.
\par 36 Tedy mluvil Gedeon k Bohu: Vysvobodíš-li skrze ruku mou Izraele, jakož jsi mluvil,
\par 37 Aj, já položím rouno toto na humne. Jestliže rosa bude toliko na roune, a na vší zemi vukol sucho, tedy vedeti budu, že vysvobodíš skrze ruku mou Izraele, jakož jsi mluvil.
\par 38 I stalo se tak. Nebo vstav nazejtrí, stlacil rouno, a vyždal rosu z neho, i byl plný koflík vody.
\par 39 Rekl také Gedeon Bohu: Nerozpaluj se prchlivost tvá proti mne, že promluvím ješte jednou. Prosím, nechažt zkusím ješte jednou na roune. Necht jest, prosím, samo rouno suché, a na vší zemi rosa.
\par 40 I ucinil Buh té noci tak, a bylo samo rouno suché, a na vší zemi byla rosa.

\chapter{7}

\par 1 Tedy vstav ráno Jerobál, (jenž jest Gedeon), i všecken lid, kterýž byl s ním, položili se pri studnici Charod, vojska pak Madianských byla jim na pulnoci pred vrchem More v údolí.
\par 2 I rekl Hospodin Gedeonovi: Príliš mnoho jest lidu s tebou, protož nedám Madianských v ruce jejich, aby se nechlubil Izrael proti mne, rka: Ruka má spomohla mi.
\par 3 A protož provolej hned, at slyší lid, a rekni: Kdo jest strašlivý a lekavý, navrat se zase, a odejdi ráno pryc k hore Galád. I navrátilo se z lidu dvamecítma tisícu, a deset tisíc zustalo.
\par 4 Rekl opet Hospodin Gedeonovi: Ješte jest mnoho lidu; kaž jim sstoupiti k vodám, a tam jej tobe zkusím. I bude, že o komžkoli reknu: Tento pujde s tebou, ten at s tebou jde, a o komžkoli reknu tobe: Tento nepujde s tebou, ten at nechodí.
\par 5 I kázal sstoupiti lidu k vodám, a rekl Hospodin k Gedeonovi: Každého, kdož chlemtati bude jazykem svým z vody, jako chlemce pes, postavíš obzvláštne, tolikéž každého, kdož se prisehne na kolena svá ku pití.
\par 6 I byl pocet tech, kteríž chlemtali, rukama svýma k ústum svým vodu nosíce, tri sta mužu, ostatek pak všeho lidu sehnuli se na kolena svá ku pití vody.
\par 7 I rekl Hospodin Gedeonovi: Ve trech stech mužu, kteríž chlemtali, vysvobodím vás, a vydám Madianské v ruku tvou, ale ostatek všeho lidu necht se vrátí, jeden každý k místu svému.
\par 8 A protož nabral ten lid v ruce své potravy a trouby své; jiné pak muže Izraelské všecky propustil, jednoho každého do stanu jejich, toliko tri sta tech mužu zanechal pri sobe. Vojska pak Madianských ležela pod ním v údolí.
\par 9 I stalo se, že noci té rekl jemu Hospodin: Vstana, sstup k vojsku, nebo dal jsem je v ruku tvou.
\par 10 Pakli nesmíš sjíti sám, sejdi s Purou služebníkem svým do vojska,
\par 11 A uslyšíš, co mluviti budou; i posilní se z toho ruce tvé, tak že smele pujdeš proti vojsku tomu. Sstoupil tedy on a Pura služebník jeho k kraji odencu, kteríž byli v vojšte.
\par 12 Madian pak a Amalech i všecken lid východní leželi v údolí, jako kobylky u velikém množství, ani velbloudu jejich poctu nebylo, jako písek, kterýž jest na brehu morském v nescíslném množství.
\par 13 A když prišel Gedeon, aj, jeden vypravoval bližnímu svému sen, a rekl: Hle, zdálo mi se, že pecen chleba jecmenného valil se do vojska Madianského, a privaliv se na každý stan, uderil na nej, až padl, a podvrátil jej svrchu, a tak ležel každý stan.
\par 14 Jemužto odpovídaje bližní jeho, rekl: Není to nic jiného, jediné mec Gedeona syna Joasova, muže Izraelského; dalt jest Buh v ruku jeho Madianské i všecka tato vojska.
\par 15 Stalo se pak, že když uslyšel Gedeon vypravování snu i vyložení jeho, poklonil se Bohu, a vrátiv se k vojsku Izraelskému, rekl: Vstante, nebo dal Hospodin v ruku vaši vojska Madianská.
\par 16 Rozdelil tedy tri sta mužu na tri houfy, a dal trouby v ruku každému z nich a báne prázdné, a v prostredku tech bání byly pochodne.
\par 17 I rekl jim: Jakž na mne uzríte, tak uciníte; nebo hle, já vejdu na kraj vojska, a jakž já tehdáž budu delati, tak udeláte.
\par 18 Nebo troubiti budu já v troubu i všickni, jenž se mnou budou, tehdy vy také troubiti budete v trouby vukol všeho vojska, a reknete: Mec Hospodinuv a Gedeonuv.
\par 19 Všel tedy Gedeon a sto mužu, kteríž s ním byli, na kraj vojska pri zacátku bdení prostredního, jen že byli promenili stráž; i troubili v trouby, a roztrískali báne, kteréž meli v rukou svých.
\par 20 Tedy ti tri houfové troubili v trouby, rozstrískavše báne, a drželi v ruce své levé pochodne, v pravé pak ruce své trouby, aby troubili, i kriceli: Mec Hospodinuv a Gedeonuv.
\par 21 A postavili se každý na míste svém vukol ležení; i zdešena jsou všecka vojska, a kricíce, utíkali.
\par 22 Když pak troubilo v trouby tech tri sta mužu, obrátil Hospodin mec jednoho proti druhému, a to po všem ležení. Utíkalo tedy vojsko až k Betseta do Zererat, a až ku pomezí Abelmehula u Tebat.
\par 23 Shromáždivše se pak muži Izraelští z Neftalím a z Asser a ze všeho pokolení Manassesova, honili Madianské.
\par 24 I rozeslal Gedeon posly na všecky hory Efraimské, rka: Vyjdete v cestu Madianským, a zastupte jim vody až k Betabare a Jordánu. Tedy shromáždivše se všickni muži Efraim, zastoupili vody až k Betabare a Jordánu.
\par 25 A chytili dvé knížat Madianských, Goréba a Zéba. I zabili Goréba na skále Goréb a Zéba zabili v lisu Zéb, a honili Madianské, hlavu pak Gorébovu a Zébovu prinesli k Gedeonovi za Jordán.

\chapter{8}

\par 1 Muži pak Efraim rekli jemu: Co jsi nám to ucinil, že jsi nepovolal nás, když jsi táhl k boji proti Madianským? A tuze se na nej domlouvali.
\par 2 Jimž odpovedel: Zdaž jsem co takového provedl, jako vy? Zdaliž není lepší paberování Efraimovo, než vinobraní Abiezerovo?
\par 3 V ruce vaše dal Buh knížata Madianská Goréba a Zéba, a co jsem já mohl takového uciniti, jako vy? I spokojil se duch jejich k nemu, když mluvil ta slova.
\par 4 Když pak prišel Gedeon k Jordánu, prešel jej, a tri sta mužu, kteríž byli s ním, ustalí, honíce neprátely.
\par 5 Protož rekl mužum Sochot: Dejte, prosím, po pecníku chleba lidu, kterýž za mnou jde, nebo ustali, a já honím Zebaha a Salmuna, krále ty Madianské.
\par 6 I rekl jemu jeden z knížat Sochot: Což již moc Zebahova a Salmunova jest v ruce tvé, abychom dali vojsku tvému chleba?
\par 7 Jimž rekl Gedeon: Z té príciny, když dá Hospodin Zebaha a Salmuna v ruku mou, tehdy zmrskám tela vaše trním a bodlácím z poušte té.
\par 8 I táhl odtud do Fanuel, a podobne mluvil k nim, ale obyvatelé Fanuel odpovedeli jemu tolikéž, jako muži Sochot.
\par 9 I rekl obyvatelum Fanuel: Když se navrátím v pokoji, zborím veži tuto.
\par 10 Zebah pak a Salmun byli v Karkara spolu s vojsky svými, takmer patnácte tisícu, všickni, což jich bylo pozustalo ze všeho vojska národu východních, zbitých pak bylo sto a dvadceti tisíc mužu bojovníku.
\par 11 I táhna Gedeon cestou bydlících v staních od východní strany Nobe a Jegbaa, uderil na to vojsko, kteréžto vojsko sobe pocínalo bezpecne.
\par 12 Když pak utíkali Zebah a Salmun, honil je, a jal oba dva ty krále Madianské, Zebaha a Salmuna, a všecko vojsko jejich predesil.
\par 13 I navracel se Gedeon syn Joasuv z bitvy pred východem slunce.
\par 14 I jav mládence z obyvatelu Sochotských, vyptával se ho; kterýž sepsal mu knížata Sochot a starší jeho, sedmdesáte sedm mužu.
\par 15 Prišed pak k obyvatelum Sochot, rekl: Aj, Zebah a Salmun, pro než jste mi utrhali, rkouce: Zdali moc Zebaha a Salmuna jest v ruce tvé, abychom dali ustalým mužum tvým chleba?
\par 16 Protož vzav starší mesta a trní s bodlácím z poušte té, dal na nich príklad jiným mužum Sochot.
\par 17 I veži Fanuel rozboril, a pobil muže mesta.
\par 18 Potom rekl Zebahovi a Salmunovi: Jací to byli muži, kteréž jste pobili na hore Tábor? Odpovedeli oni: Takoví jako ty, jeden každý na pohledení byl jako syn královský.
\par 19 I dí: Bratrí moji, synové matky mé byli. Živt jest Hospodin, byste byli živili je, nepobil bych vás.
\par 20 Tedy rekl k Jeter prvorozenému svému: Vstan, pobí je. Ale mládencek nevytáhl mece svého, proto že se bál, nebo byl ješte mládencek.
\par 21 I rekl Zebah a Salmun: Vstan ty a obor se na nás, nebo jaký jest muž, taková i síla jeho. Vstav tedy Gedeon, zabil Zebaha a Salmuna, a vzal halže, kteréž byly na hrdlech velbloudu jejich.
\par 22 Potom rekli muži Izraelští Gedeonovi: Panuj nad námi i ty i syn tvuj, také syn syna tvého, nebo vysvobodil jsi nás z ruky Madianských.
\par 23 I odpovedel jim Gedeon: Nebudut já panovati nad vámi, aniž panovati bude nad vámi syn muj, Hospodin panovati bude nad vámi.
\par 24 Rekl jim také Gedeon: Žádost tuto vzložím toliko na vás, abyste mi dali jeden každý náušnici z loupeží svých; (nebo náušnice zlaté meli, proto že Izmaelitští byli.)
\par 25 I odpovedeli: Rádi dáme. A prostrevše roucho, metali na ne každý náušnici z loupeží svých.
\par 26 Byla pak váha náušnic zlatých, kteréž vyžádal, tisíc a sedm set lotu zlata, krome halží a zlatých jablecek, i roucha šarlatového, kteréž na sobe meli králové Madianští, a krome halží, kteréž byly na hrdlech velbloudu jejich.
\par 27 I udelal z toho Gedeon efod, a složil jej v meste svém Ofra. I smilnil tam všecken Izrael, chode za ním, a byl Gedeonovi i domu jeho osídlem.
\par 28 Madianští pak sníženi jsou pred syny Izraelskými, aniž potom pozdvihli hlavy své; a tak byla v pokoji zeme za ctyridceti let ve dnech Gedeonových.
\par 29 A odšed Jerobál syn Joasuv, prebýval v dome svém.
\par 30 Mel pak Gedeon sedmdesáte synu, kteríž pošli z bedr jeho; nebo mel žen mnoho.
\par 31 Ženina také jeho, kterouž mel v Sichem, i ta porodila jemu syna, a dala mu jméno Abimelech.
\par 32 A když umrel Gedeon syn Joasuv v starosti dobré, pochován jest v hrobe Joasa otce svého v Ofra Abiezeritského.
\par 33 Stalo se pak po smrti Gedeonove, že se odvrátili synové Izraelští a smilnili, jdouce za modlami, a vzali sobe Bále Berit za boha.
\par 34 A nerozpomenuli se synové Izraelští na Hospodina Boha svého, kterýž je vytrhl z ruky všech neprátel jejich vukol.
\par 35 A neucinili milosrdenství s domem Jerobále Gedeona, jako on všecko dobré cinil Izraelovi.

\chapter{9}

\par 1 Nebo odšed Abimelech syn Jerobáluv do Sichem k bratrím matky své, mluvil k nim i ke vší celedi domu otce matky své, rka:
\par 2 Mluvte, prosím, ke všechnem mužum Sichemským: Co jest lépe vám, to-li, aby panovalo nad vámi sedmdesáte mužu, totiž všickni synové Jerobálovi, cili aby panoval nad vámi muž jeden? Pamatujte pak, že jsem já kost vaše a telo vaše.
\par 3 Tedy mluvili bratrí matky jeho o nem ke všechnem mužum Sichemským všecka slova tato, a naklonilo se srdce jejich k Abimelechovi, nebo rekli: Bratr náš jest.
\par 4 I dali jemu sedmdesáte lotu stríbra z domu Bále Berit, na než sobe najal Abimelech služebníky, povalece a tuláky, aby chodili za ním.
\par 5 A prišed do domu otce svého do Ofra, zmordoval bratrí své, syny Jerobálovy, sedmdesáte mužu, na jednom kameni; toliko zustal Jotam syn Jerobáluv nejmladší, nebo se byl skryl.
\par 6 Tedy shromáždili se všickni muži Sichemští i všecken dum Mello, i šli, a ustanovili sobe Abimelecha za krále na rovinách u sloupu, kterýž jest u Sichem.
\par 7 To když povedeli Jotamovi, odejda, postavil se na vrchu hory Garizim, a pozdvihna hlasu svého, volal, a rekl jim: Poslyšte mne muži Sichemští, a uslyš vás Buh.
\par 8 Sešlo se nekterého casu dríví, aby pomazalo nad sebou krále. I rekli olive: Kraluj nad námi.
\par 9 Jimžto odpovedela oliva: Zdali já, opuste svou tucnost, kterouž cten bývá Buh i lidé, pujdu, abych byla postavena nad stromy?
\par 10 I reklo dríví fíku: Pod ty a kraluj nad námi.
\par 11 Jimž odpovedel fík: Zdali já, opuste sladkost svou a ovoce své výborné, pujdu, abych postaven byl nad stromy?
\par 12 Reklo opet dríví vinnému korenu: Pod ty a kraluj nad námi.
\par 13 Jimž odpovedel vinný koren: Zdali já, opuste své víno, kteréž obveseluje Boha i lidi, pujdu, abych postaven byl nad stromy?
\par 14 Naposledy reklo všecko dríví bodláku: Pod ale ty a kraluj nad námi.
\par 15 I odpovedel bodlák dríví: Jestliže v pravde bérete vy mne sobe za krále, podte, odpocívejte pod stínem mým; pakli nic, vyjdi ohen z bodláku a spal cedry Libánské.
\par 16 Tak i nyní, jestliže jste práve a upríme ucinili, ustavivše Abimelecha za krále, a jestliže jste dobre ucinili Jerobálovi a domu jeho, a jestliže podlé dobrodiní rukou jeho odplatili jste se jemu;
\par 17 (Nebo otec muj bojoval za vás a opovážil se života svého, aby vás vysvobodil z ruky Madianských,
\par 18 Vy pak ted povstali jste proti domu otce mého, a zmordovali jste syny jeho, sedmdesáte mužu na jednom kameni, a ustanovili jste krále Abimelecha, syna devky jeho, nad muži Sichemskými, proto že bratr váš jest);
\par 19 Jestliže, rku, práve a upríme udelali jste Jerobálovi i domu jeho dne tohoto, veselte se z Abimelecha, a on také necht se veselí z vás;
\par 20 Pakli nic, necht vyjde ohen z Abimelecha a sžíre muže Sichemské i dum Mello, necht vyjde také ohen od mužu Sichemských a z domu Mello a sžíre Abimelecha.
\par 21 Tedy utekl Jotam, a utíkaje, odšel do Beera, a zustal tam, boje se Abimelecha bratra svého.
\par 22 I panoval Abimelech nad Izraelem tri léta.
\par 23 Poslal pak Buh ducha zlého mezi Abimelecha a mezi muže Sichemské, a zproneverili se muži Sichemští Abimelechovi,
\par 24 Aby pomštena byla krivda sedmdesáti synu Jerobálových, a aby krev jejich prišla na Abimelecha bratra jejich, kterýž zmordoval je, a na muže Sichemské, kteríž posilnili rukou jeho, aby zmordoval bratrí své.
\par 25 Nebo ucinili muži Sichemští jemu zálohy na vrších hor, a loupili všecky chodíci mimo ne tou cestou; kterážto vec povedína jest Abimelechovi.
\par 26 Syn pak Ebeduv Gál, jda s bratrími svými, prišel do Sichem, i tešili se z neho muži Sichemští.
\par 27 A vyšedše na pole, sbírali víno své a tlacili i veselili se; a všedše do chrámu bohu svých, jedli a pili, a zlorecili Abimelechovi.
\par 28 Rekl pak Gál syn Ebeduv: Kdo jest Abimelech? A jaké jest Sichem, abychom sloužili jemu? Zdaliž není syn Jerobáluv, a Zebul úredník jeho? Služte radeji mužum Emora, otce Sichemova; nebo proc my máme sloužiti tomuto?
\par 29 Ale ó kdyby tento lid byl v ruce mé, abych shladil Abimelecha! I rekl Abimelechovi: Sber sobe vojsko, a vyjdi.
\par 30 Uslyšav pak Zebul, úredník mesta toho, slova Gále syna Ebedova, rozhneval se náramne.
\par 31 I poslal posly k Abimelechovi tajne, rka: Hle, Gál syn Ebeduv i bratrí jeho prišli do Sichem, a hle, bojovati budou s mestem proti tobe.
\par 32 Protož nyní vstana nocne, ty i lid, kterýž jest s tebou, zdelej zálohy v poli.
\par 33 A ráno, když bude slunce vycházeti, vstana, uderíš na mesto, a když on i lid, kterýž jest s ním, vyjde proti tobe, uciníš jemu to, což se naskytne ruce tvé.
\par 34 A protož vstal Abimelech i všecken lid, kterýž s ním byl v noci, a ucinili zálohy u Sichem na ctyrech místech.
\par 35 I vyšel Gál syn Ebeduv, a postavil se v bráne mesta; vyvstal pak Abimelech i lid, kterýž s ním byl, z záloh.
\par 36 A uzrev Gál ten lid, rekl Zebulovi: Hle, lid sstupuje s vrchu hor. Jemuž odpovedel Zebul: Stín hor vidíš, jako nejaké lidi.
\par 37 Tedy opet promluvil Gál, rka:Hle, lid sstupuje s vrchu, nebo houf jeden táhne cestou rovin Monenim.
\par 38 Rekl pak jemu Zebul: Kde jsou nyní ústa tvá, jimižs mluvil: Kdo jest Abimelech, abychom sloužili jemu? Zdaliž toto není lid ten, kterýmž jsi pohrdal? Vytáhniž nyní, a bojuj proti nemu.
\par 39 I vytáhl Gál pred lidmi Sichemskými, a bojoval proti Abimelechovi.
\par 40 Ale Abimelech honil ho utíkajícího pred tvárí svou, a padlo ranených mnoho až k bráne.
\par 41 Zustal pak Abimelech v Aruma, a Zebul vyhnal Gále i bratrí jeho, aby nezustávali v Sichem.
\par 42 Nazejtrí pak vytáhl lid do pole, i oznámeno jest to Abimelechovi.
\par 43 Tedy on pojal lid svuj, a rozdelil jej na tri houfy, zdelav zálohy v polích, a vida, an lid vychází z mesta, vyskocil na ne a zbil je.
\par 44 Nebo Abimelech a houf, kterýž byl s ním, uderili na ne a postavili se u brány mesta, druzí pak dva houfové oborili se na všecky ty, kteríž byli v poli, a zbili je.
\par 45 Abimelech pak dobýval mesta celý ten den, až ho i dobyl, a lid, kterýž v nem byl, pobil, a zboriv mesto, posál je solí.
\par 46 Uslyševše pak všickni muži veže Sichemské, vešli do hradu svého, chrámu boha Berit.
\par 47 A oznámeno jest Abimelechovi, že se tam shromáždili všickni muži veže Sichemské.
\par 48 Tedy vstoupil Abimelech na horu Salmon, on i všecken lid, kterýž byl s ním, a nabrav seker s sebou, nasekal ratolestí s stromu, kterýchž nabral a vložil na rameno své. I rekl lidu, kterýž byl s ním: Což jste videli, že jsem ucinil, spešne ucinte tak jako já.
\par 49 I utal sobe jeden každý ze všeho lidu ratolest, a jdouce za Abimelechem, skladli je u hradu, a zapálili jimi hrad. I zemreli tam všickni muži veže Sichemské témer tisíc mužu a žen.
\par 50 Odšel pak Abimelech do Tébes, a položil se u Tébes, i dobyl ho.
\par 51 Byla pak veže pevná u prostred mesta, a utekli tam všickni muži i ženy, i všickni prední mesta toho, a zavreli po sobe, vstoupivše na vrch veže.
\par 52 Tehdy prišed Abimelech až k veži, dobýval jí, a pristoupil až ke dverím veže, aby je zapálil ohnem.
\par 53 V tom žena nejaká svrhla kus žernovu na hlavu Abimelechovu, a prorazila hlavu jeho.
\par 54 A on rychle zavolav mládence, kterýž nosil zbroj jeho, rekl jemu: Vytrhni mec svuj a zabí mne, aby potom nepravili o mne: Žena zabila ho. I probodl jej služebník jeho, a umrel.
\par 55 Uzrevše pak synové Izraelští, že by umrel Abimelech, odešli jeden každý k místu svému.
\par 56 A tak odmenil Buh zlým Abimelechovi za nešlechetnost, kterouž páchal proti otci svému, zmordovav sedmdesáte bratrí svých.
\par 57 Tolikéž i všecku nešlechetnost mužu Sichemských vrátil Buh na hlavu jejich, a prišlo na ne zlorecení Jotama syna Jerobálova.

\chapter{10}

\par 1 Povstal pak po Abimelechovi k obhajování Izraele Tola, syn Fua, syna Dodova, muž z pokolení Izachar, a ten bydlil v Samir na hore Efraim.
\par 2 I soudil Izraele za trimecítma let, a umrev, pochován jest v Samir.
\par 3 Po nem povstal Jair Galádský, a soudil Izraele za dvamecítma let.
\par 4 A mel tridceti synu, kteríž jezdili na tridcíti mezcích; a meli tridceti mest, kteráž sloula vsi Jairovy až do dnešního dne, a ty jsou v zemi Galád.
\par 5 I umrel Jair, a pochován jest v Kamon.
\par 6 Opet pak synové Izraelští cinili to, což jest zlého pred ocima Hospodinovýma; nebo sloužili Bálum a Astarot, to jest, bohum Syrským, bohum Sidonským a bohum Moábským, tolikéž i bohum synu Ammon, i bohum Filistinským, tak že opustili Hospodina, a nesloužili jemu.
\par 7 Protož roznítila se prchlivost Hospodinova na Izraele, a vydal je v ruku Filistinských a v ruku Ammonitských,
\par 8 Kteríž stírali a potlacovali syny Izraelské toho roku, i potom za osmnácte let, všecky syny Izraelské, kteríž byli pred Jordánem v zemi Amorejského, kteráž jest v Galád.
\par 9 Prešli pak Ammonitští i Jordán, aby bojovali také proti Judovi, a proti Beniaminovi, i proti domu Efraimovu; i byl Izrael náramne ssoužen.
\par 10 Tedy volali synové Izraelští k Hospodinu, rkouce: Zhrešilit jsme tobe, tak že jsme opustili te Boha svého, a sloužili jsme Bálum.
\par 11 Ale Hospodin rekl synum Izraelským: Zdaliž jsem od Egyptských a od Amorejských a od Ammonitských a Filistinských,
\par 12 Tolikéž od Sidonských, a Amalechitských, i od Maonitských, vás ssužujících, když jste volali ke mne, nevysvobodil vás z ruky jejich?
\par 13 A vy opustili jste mne a sloužili jste bohum cizím, protož nevysvobodím vás více.
\par 14 Jdete a volejte k bohum, kteréž jste sobe zvolili; oni necht vás vysvobodí v cas ssoužení vašeho.
\par 15 I rekli synové Izraelští Hospodinu: Zhrešili jsme; ucin s námi, cožt se dobre líbí, a však vysvobod nás, prosíme, v tento cas.
\par 16 Protož vyvrhše bohy cizí z prostredku svého, sloužili Hospodinu, a zželelo se duši jeho nad trápením Izraele.
\par 17 Svolali se pak Ammonitští, a položili se v Galád; shromáždili se také synové Izraelští, a položili se v Masfa.
\par 18 I rekli lid s knížaty Galád jedni druhým: Kdokoli pocne bojovati proti Ammonitským, bude vudce všech obyvatelu Galád.

\chapter{11}

\par 1 Jefte pak Galádský byl muž udatný a byl syn ženy nevestky, z níž zplodil Galád receného Jefte.
\par 2 Ale i manželka Galádova naplodila mu synu, a když dorostli synové manželky té, vyhnali Jefte. Nebo rekli jemu: Nebudeš dediti v dome otce našeho, nebo jsi postranní ženy syn.
\par 3 Protož utekl Jefte od tvári bratrí svých, a bydlil v zemi Tob; a sbehli se k Jefte lidé povaleci, a vycházeli s ním.
\par 4 Stalo se pak po tech dnech, že bojovali Ammonitští proti Izraelským.
\par 5 A když pocali bojovati Ammonitští proti nim, odešli starší Galádští, aby zase privedli Jefte z zeme Tob.
\par 6 I rekli k Jefte: Pod a bud vudce náš, abychom bojovali proti Ammonitským.
\par 7 Odpovedel Jefte starším Galádským: Zdaliž jste vy mne nemeli v nenávisti, a nevyhnali jste mne z domu otce mého? Procež tedy nyní prišli jste ke mne, když ssouženi jste?
\par 8 I rekli starší Galádští k Jefte: Proto jsme se nyní navrátili k tobe, abys šel s námi a bojoval proti Ammonitským, a byl nám všechnem obyvatelum Galádským za vudce.
\par 9 Tedy odpovedel Jefte starším Galádským: Ponevadž mne zase uvodíte, abych bojoval proti Ammonitským, když by mi je dal Hospodin v moc, budu-li vám za vudce?
\par 10 I rekli starší Galádští k Jefte: Hospodin bude svedkem mezi námi, jestliže neuciníme vedlé slova tvého.
\par 11 A tak šel Jefte s staršími Galádskými, a predstavil jej sobe lid za vudci a kníže, a mluvil Jefte všecka slova svá pred Hospodinem v Masfa.
\par 12 Potom poslal Jefte posly k králi Ammonitskému s tímto porucením: Co máš ke mne, že jsi vytáhl na mne, abys bojoval proti zemi mé?
\par 13 I odpovedel král Ammonitský poslum Jefte: Že vzal Izrael zemi mou, když vyšel z Egypta, od Arnon až k Jaboku a až k Jordánu; protož nyní vrat mi ji pokojne.
\par 14 Opet pak poslal Jefte posly k králi Ammonitskému,
\par 15 A rekl jemu: Takto praví Jefte: Nevzalt Izrael zeme Moábské, ani zeme synu Ammonových.
\par 16 Ale když vyšli z Egypta, šel Izrael pres poušt až k mori Rudému, a prišel do Kádes.
\par 17 Odkudž poslal Izrael posly k králi Edomskému, rka: Prosím, necht projdu skrze zemi tvou. A nechtel ho slyšeti král Edomský. Poslal také k králi Moábskému, a nechtel povoliti. A tak zustal Izrael v Kádes.
\par 18 Potom když šel pres poušt, obcházel zemi Edomskou a zemi Moábskou, a prišed od východu slunce zemi Moábské, položil se pred Arnon, a nevešli na pomezí Moábské; nebo Arnon jest meze Moábských.
\par 19 Protož poslal Izrael posly k Seonovi králi Amorejskému, to jest, k králi Ezebon, a rekl jemu Izrael: Prosím, necht projdu skrze zemi tvou až k místu svému.
\par 20 Ale Seon nedoveroval Izraelovi, aby prejíti mel pomezí jeho. Protož shromáždil Seon všecken lid svuj, a položili se v Jasa, a bojoval proti Izraelovi.
\par 21 I dal Hospodin Buh Izraelský Seona i všecken lid jeho v ruku Izraelských, a porazili je. I opanoval dedicne Izrael všecku zemi Amorejského, té zeme obyvatele.
\par 22 A tak opanovali všecko pomezí Amorejského od Arnon až k Jaboku a od poušte až k Jordánu.
\par 23 Když tedy Hospodin Buh Izraelský vyhladil Amorejského od tvári lidu svého Izraelského, proc ty chceš panovati nad ním?
\par 24 Zdaliž tím, což tobe dal Chámos buh tvuj k vládarství, vládnouti nemáš? Takž, kterékoli vyhladil Hospodin Buh náš od tvári naší, jejich dedictvím my vládneme.
\par 25 K tomu pak, zdali jsi ty cím lepší Baláka syna Seforova, krále Moábského? Zdali se kdy vadil s Izraelem? Zdali kdy bojoval proti nim?
\par 26 Ješto již bydlí Izrael v Ezebon a ve vsech jeho, i v Aroer, a ve vsech jeho, i ve všech mestech, kteráž jsou pri pomezí Arnon, za tri sta let. Proc jste jí neodjali v tak dlouhém casu?
\par 27 Protož ne já provinil jsem proti tobe, ale ty mne zle ciníš, bojuje proti mne. Necht soudí Hospodin soudce dnes mezi syny Izraelskými a mezi syny Ammonovými.
\par 28 Král pak Ammonitský neuposlechl slov Jefte, kteráž vzkázal jemu.
\par 29 V tom nadšen byl Jefte duchem Hospodinovým, i táhl skrze Galád a Manasse, prošel i Masfa v Galád, a z Masfy v Galád táhl proti Ammonitským.
\par 30 (Ucinil pak Jefte slib Hospodinu, a rekl: Jestliže jistotne dáš mi Ammonitské v ruku mou:
\par 31 I stane se, že což by koli vyšlo ze dverí domu mého mne vstríc, když se vrátím v pokoji od Ammonitských, bude Hospodinovo, abych to obetoval v obet zápalnou.)
\par 32 Pritáhl tedy Jefte na Ammonitské, aby bojoval proti nim, a dal je Hospodin v ruku jeho.
\par 33 I pobil je od Aroer, až kudy se jde do Mennit, dvadceti mest, a až do Abel vinic porážkou velikou velmi; a sníženi jsou Ammonitští pred syny Izraelskými.
\par 34 Když se pak navracoval Jefte do Masfa k domu svému, aj, dcera jeho vyšla jemu vstríc s bubny a s houfem plésajících; kterouž mel toliko jedinou, aniž mel kterého syna aneb jiné dcery.
\par 35 Stalo se pak, že když uzrel ji, roztrhl roucha svá a rekl: Ach, dcero má, velices mne ponížila; nebo jsi z tech, jenž mne kormoutí, ponevadž jsem tak rekl Hospodinu, aniž budu moci odvolati toho.
\par 36 Jemuž ona odpovedela: Muj otce, jestliže jsi tak rekl Hospodinu, ucin mi podlé toho, jakžs mluvil; kdyžt jen dal Hospodin pomstu nad neprátely tvými, Ammonitskými.
\par 37 Rekla také otci svému: Necht toliko toto obdržím: Odpust mne na dva mesíce, at jdu a vejdu na hory a opláci panenství svého, já i družicky mé.
\par 38 Kterýž rekl: Jdi. I propustil ji na dva mesíce. Odešla tedy ona i družicky její, a plakala panenství svého na horách.
\par 39 A pri dokonání dvou mesícu navrátila se k otci svému, a vykonal pri ní slib svuj, kterýž byl ucinil. Ona pak nepoznala muže. I byl ten obycej v Izraeli,
\par 40 Že každého roku scházívaly se dcery Izraelské, aby plakaly nad dcerou Jefte Galádského za ctyri dni v roce.

\chapter{12}

\par 1 Shromáždili se pak muži Efraim, a prešedše k strane pulnocní, rekli k Jefte: Proc jsi vyšel k boji proti Ammonitským, a nepovolal jsi nás, abychom šli s tebou? Dum tvuj i tebe ohnem spálíme.
\par 2 K nimž rekl Jefte: Nesnáz jsem mel já a lid muj s Ammonitskými velikou; tedy povolal jsem vás, ale nevysvobodili jste mne z ruky jejich.
\par 3 Protož vida, že jste mne nevysvobodili, odvážil jsem se života svého, a táhl jsem proti Ammonitským, i dal je Hospodin v ruku mou. Ale proc jste dnes prišli ke mne? Zdali abyste bojovali proti mne?
\par 4 Tedy shromáždil Jefte všecky muže Galádské, a bojoval s Efraimem. I porazili muži Galád Efraimské, nebo pravili pobehlci Efraimští: Vy Galádští jste u prostred Efraima a u prostred Manassesa.
\par 5 Odjali Galádští též i brody Jordánské Efraimským. A bylo, že když kdo z utíkajících Efraimských rekl: Nechat prejdu, rekli jemu muži Galádští: Jsi-li Efratejský? Jestliže rekl: Nejsem,
\par 6 Tedy rekli jemu: Rci hned Šibolet. I rekl: Sibolet, aniž dobre mohl vyrknouti toho. Tedy pochytíce jej, zamordovali ho u brodu Jordánského. I padlo toho casu z Efraima ctyridceti a dva tisíce.
\par 7 Soudil pak Jefte Izraele šest let; a umrel Jefte Galádský, a pochován jest v jednom z mest Galádských.
\par 8 Potom soudil po nem Izraele Abesam z Betléma.
\par 9 A mel tridceti synu a tridceti dcer, kteréž rozevdal od sebe, a tridceti žen privedl od jinud synum svým. I soudil Izraele sedm let.
\par 10 A umrev Abesam, pochován jest v Betléme.
\par 11 Po nem pak soudil Izraele Elon Zabulonský; ten soudil Izraele deset let.
\par 12 I umrel Elon Zabulonský, a pochován jest v Aialon, v zemi Zabulon.
\par 13 Potom soudil Izraele Abdon syn Helleluv Faratonský.
\par 14 Ten mel ctyridceti synu a tridceti vnuku, kteríž jezdili na sedmdesáti mezcích; i soudil Izraele osm let.
\par 15 Umrev pak Abdon syn Helleluv Faratonský, pohrben jest v Faraton, v zemi Efraim na hore Amalechitské.

\chapter{13}

\par 1 Tedy opet synové Izraelští cinili to, což jest zlého pred ocima Hospodinovýma, i vydal je Hospodin v ruce Filistinských za ctyridceti let.
\par 2 Byl pak muž jeden z Zaraha, z celedi Dan, jménem Manue, jehož manželka byla neplodná a nerodila.
\par 3 I ukázal se andel Hospodinuv žene té a rekl jí: Aj, nyní jsi neplodná, aniž jsi rodila; pocneš pak a porodíš syna.
\par 4 Protož nyní vystríhej se, abys vína nepila aneb nápoje opojného, a abys nejedla nic necistého.
\par 5 Nebo aj, pocneš a porodíš syna, a britva at nevchází na hlavu jeho, nebo bude to díte od života Nazarejský Boží, a tent pocne vysvobozovati Izraele z ruky Filistinských.
\par 6 I prišla žena a povedela to muži svému, rkuci: Muž Boží prišel ke mne, jehož oblícej byl jako oblícej andela Božího, hrozný velmi, a neotázala jsem se jeho, odkud by byl, ani mi svého jména neoznámil.
\par 7 Ale rekl mi: Aj, pocneš a porodíš syna, protož nyní nepí vína aneb nápoje opojného, aniž jez co necistého, nebo to díte od života bude Nazarejský Boží až do dne smrti své.
\par 8 Tedy Manue modlil se Hospodinu a rekl: Vyslyš mne, muj Pane, prosím, necht muž Boží, kteréhož jsi byl poslal, zase prijde k nám, a naucí nás, co máme delati s dítetem, kteréž se má naroditi.
\par 9 I vyslyšel Buh hlas Manue; nebo prišel andel Boží opet k žene té, když sedela na poli. Manue pak muž její nebyl s ní.
\par 10 A protož s chvátáním bežela žena ta, a oznámila muži svému, rkuci jemu: Hle, ukázal se mi muž ten, kterýž byl ke mne prvé prišel.
\par 11 Vstav pak Manue, šel za manželkou svou, a prišed k muži tomu, rekl jemu: Ty-li jsi ten muž, kterýž jsi mluvil k manželce mé? Odpovedel: Jsem.
\par 12 I rekl Manue: Necht se nyní stane slovo tvé, ale jaká péce o to díte a správa pri nem býti má?
\par 13 Odpovedel andel Hospodinuv Manue: Ode všeho toho, o cemž jsem oznámil žene, at se ona vystríhá.
\par 14 Niceho, což pochází z vinného kmene, at neužívá, to jest, vína aneb nápoje opojného at nepije, a nic necistého at nejí; cožkoli jsem jí prikázal, at ostríhá.
\par 15 Tedy rekl Manue andelu Hospodinovu: Medle, necht te pozdržíme, a pripravímet kozlíka.
\par 16 I odpovedel andel Hospodinuv Manue: Bys mne i pozdržel, nebudut jísti pokrmu tvého, ale jestliže pripravíš obet zápalnou, Hospodinu ji obetuj. Nebo nevedel Manue, že byl andel Hospodinuv.
\par 17 Rekl opet Manue andelu Hospodinovu: Jaké jest jméno tvé, abychom, když se naplní rec tvá, poctili tebe?
\par 18 Jemuž odpovedel andel Hospodinuv: Proc se ptáš na jméno mé, kteréž jest divné?
\par 19 Vzav tedy Manue kozlíka a obet suchou, obetoval to na skále Hospodinu, a on divnou vec ucinil, an na to hledí Manue a manželka jeho.
\par 20 Nebo když vstupoval plamen s oltáre k nebi, vznesl se andel Hospodinuv v plameni s oltáre, Manue pak a manželka jeho vidouce to, padli na tvár svou na zemi.
\par 21 A již se více neukázal andel Hospodinuv Manue ani manželce jeho. Tedy porozumel Manue, že byl andel Hospodinuv.
\par 22 I rekl Manue manželce své: Jiste my zemreme, nebo jsme Boha videli.
\par 23 Jemuž odpovedela manželka jeho: Kdyby nás chtel Hospodin usmrtiti, nebyl by prijal z rukou našich obeti zápalné a suché, aniž by nám byl ukázal ceho toho, aniž by na tento cas byl nám ohlásil vecí takových.
\par 24 A tak žena ta porodila syna a nazvala jméno jeho Samson. I rostlo díte, a žehnal jemu Hospodin.
\par 25 I pocal ho Duch Hospodinuv ponoukati v Mahane Dan, mezi Zaraha a Estaol.

\chapter{14}

\par 1 Šel pak Samson do Tamnata, a uzrel tam ženu ze dcer Filistinských.
\par 2 A navrátiv se, oznámil otci svému a materi své, rka: Videl jsem ženu v Tamnata ze dcer Filistinských, protož nyní vezmete mi ji za manželku.
\par 3 I rekl mu otec jeho a matka jeho: Zdali není mezi dcerami bratrí tvých a ve všem lidu mém ženy, že sobe vzíti chceš manželku z Filistinských neobrezaných? Odpovedel Samson otci svému: Tuto vezmete mi, nebt mi se líbí.
\par 4 Otec pak jeho a matka jeho nevedeli, by to od Hospodina bylo, a že príciny hledá od Filistinských; nebo toho casu panovali Filistinští nad Izraelem.
\par 5 Tedy šel Samson a otec jeho i matka jeho do Tamnata. Když pak prišli k vinicím Tamnatským, a aj, lev mladý rvoucí potkal se s ním.
\par 6 I sstoupil na nej Duch Hospodinuv, a roztrhl lva, jako by roztrhl kozelce, ackoli nic nemel v rukou svých. A neoznámil otci ani materi své, co ucinil.
\par 7 Prišed tedy, mluvil s ženou tou, a líbila se Samsonovi.
\par 8 Navracuje se pak po nekolika dnech, aby ji pojal, uchýlil se, aby pohledel na mrtvého lva, a aj, v tele jeho byl roj vcel a med.
\par 9 A vybrav jej na ruce své, šel cestou a jedl; a prišed k otci svému a materi své, dal jim, i jedli. Ale nepovedel jim, že z mrtvého lva vynal ten med.
\par 10 Tedy šel otec jeho k žene té, a ucinil tam Samson hody, nebo tak cinívali mládenci.
\par 11 Když pak jej videli tam, vybrali z sebe tridceti tovaryšu, aby byli pri nem.
\par 12 I rekl jim Samson: Vydám vám pohádku, kterouž jestliže mi práve vysvetlíte za sedm dní techto hodu a uhodnete, dám vám tridceti cechlu a tridcatero roucho promenné.
\par 13 Jestliže mi pak nebudete moci uhodnouti, dáte vy mne tridceti cechlu a tridcatero roucho promenné. Kteríž odpovedeli jemu: Vydej pohádku svou, at ji slyšíme.
\par 14 I rekl jim: Z zžírajícího vyšel pokrm, a z silného vyšla sladkost. I nemohli uhodnouti pohádky té za tri dni.
\par 15 Stalo se pak dne sedmého, (nebo byli rekli žene Samsonove: Namluv muže svého, at nám vyloží tu pohádku, at nespálíme te i domu otce tvého ohnem. Proto-liž, abyste našeho statku dostali, pozvali jste nás? Ci co?
\par 16 I plakala žena Samsonova na nej, rkuci: Jiste nenávidíš mne a nemiluješ mne; pohádku jsi vydal synum lidu mého, a mne jí nechceš povedíti. Kterýž rekl jí: Hle, otci mému a materi neoznámil jsem, a tobe mám povedíti?
\par 17 I plakala na nej do sedmého dne,v nichž meli hody). Dne tedy sedmého povedel jí, nebo trápila jej; kterážto oznámila pohádku synum lidu svého.
\par 18 Muži tedy mesta toho dne sedmého, prvé než slunce zapadlo, rekli jemu: Co sladšího nad med, a co silnejšího nad lva? Kterýž rekl jim: Byste byli neorali mou jalovickou, neuhodli byste pohádky mé.
\par 19 I sstoupil na nej Duch Hospodinuv, a šel do Aškalon, a pobil z nich tridceti mužu. A vzav loupeže jejich, dal šaty promenné tem, jenž uhodli pohádku, a rozhnevav se velmi, odšel do domu otce svého.
\par 20 Žena pak Samsonova dostala se jednomu z tovaryšu jeho, kteréhož on byl k sobe pripojil.

\chapter{15}

\par 1 Stalo se pak po nekolika dnech, v cas žne pšenicné, že chteje navštíviti Samson ženu svou, a prinésti s sebou kozlíka, rekl: Vejdu k žene své do pokoje. A nedopustil mu otec její vjíti.
\par 2 I rekl otec její: Domníval jsem se zajisté, že ji máš v nenávisti, protož dal jsem ji tovaryši tvému. Zdaliž není sestra její mladší peknejší než ona? Nechat jest tedy tvá místo oné.
\par 3 I rekl jim Samson: Nebudut já potom vinen Filistinským, když jim zle uciním.
\par 4 Odšed tedy Samson, nalapal tri sta lišek, a vzav pochodne, obrátil jeden ocas k druhému, a dal vše jednu pochodni mezi dva ocasy do prostredka.
\par 5 Potom zapálil ty pochodne, a pustil do obilí Filistinských, a popálil, jakž sžaté tak nesžaté, i vinice i olivoví.
\par 6 I rekli Filistinští: Kdo je to ucinil? Jimž odpovedíno: Samson zet Tamnejského, proto že vzal ženu jeho a dal ji tovaryši jeho. Tedy prišedše Filistinští, spálili ji ohnem i otce jejího.
\par 7 Tedy rekl jim Samson: Ac jste ucinili tak, však až se lépe vymstím nad vámi, teprv prestanu.
\par 8 I zbil je na hnátích i na bedrách porážkou velikou, a odšed, usadil se na vrchu skály Etam.
\par 9 Procež vytáhli Filistinští, a rozbivše stany proti Judovi, rozložili se až do Lechi.
\par 10 Muži pak Juda rekli: Proc jste vytáhli proti nám? I odpovedeli: Vytáhli jsme, abychom svázali Samsona, a ucinili jemu tak, jako on nám ucinil.
\par 11 Tedy vyšlo tri tisíce mužu Juda k vrchu skály Etam, a rekli Samsonovi: Nevíš-liž, že panují nad námi Filistinští? Procež jsi tedy nám to ucinil? I odpovedel jim: Jakž mi ucinili, tak jsem jim ucinil.
\par 12 Rekli také jemu: Prišli jsme, abychom te svázali a vydali v ruku Filistinským. Tedy odpovedel jim Samson: Prisáhnete mi , že vy se na mne neoboríte.
\par 13 Tedy mluvili jemu, rkouce: Nikoli, jediné tuze svížíce, vydáme te v ruku jejich, ale nezabijeme te. I svázali ho dvema provazy novými, a svedli jej s skály.
\par 14 Kterýž když prišel až do Lechi, Filistinští radostí kriceli proti nemu. Prišel pak na nej Duch Hospodinuv, a ucineni jsou provazové, kteríž byli na rukou jeho, jako niti lnené, když v ohni horí, i spadli svazové s rukou jeho.
\par 15 Tedy našel celist oslicí ješte vlhkou, a vztáh ruku svou, vzal ji a pobil ní tisíc mužu.
\par 16 Protož rekl Samson: Celistí oslicí hromadu jednu, nýbrž dve hromady, celistí oslicí zbil jsem tisíc mužu.
\par 17 A když prestal mluviti, povrhl celist z ruky své, a nazval to místo Ramat Lechi.
\par 18 Žíznil pak velice, i volal k Hospodinu a rekl: Ty jsi ucinil skrze ruce služebníka svého vysvobození toto veliké, nyní pak již žízní umru, aneb upadnu v ruku tech neobrezaných.
\par 19 Tedy otevrel Buh skálu v Lechi, i vyšly z ní vody, a napil se; i okrál, a jako ožil. Protož nazval jméno její studnice vzývajícího, kteráž jest v Lechi až do dnešního dne.
\par 20 Soudil pak Izraele za casu Filistinských dvadceti let.

\chapter{16}

\par 1 Odšel pak Samson do Gázy, a uzrev tam ženu nevestku, všel k ní.
\par 2 I povedíno obyvatelum Gázy: Samson prišel sem. Kteríž obeslavše se, stráhli na nej v bráne mesta pres celou noc, a staveli se tiše té celé noci, rkouce: Až ráno zabijeme jej.
\par 3 Spal pak Samson až do pul noci, a o pul noci vstal, a pochytiv vrata brány mestské s obema verejemi a s závorou, vložil na ramena svá a vnesl je na vrch hory, kteráž byla naproti Hebronu.
\par 4 Potom pak zamiloval ženu v údolí Sorek, jejíž jméno bylo Dalila.
\par 5 I prišli knížata Filistinská k ní a rekli jí: Oklamej ho a zvez, v cem jest síla jeho tak veliká, a jak bychom premohli jej, abychom svížíce, skrotili jej; tobe pak jeden každý z nás dáme tisíc a sto lotu stríbra.
\par 6 Tedy rekla Dalila Samsonovi: Prosím, oznam mi, v cem jest tak veliká síla tvá, a cím bys svázán a zemdlen býti mohl?
\par 7 Odpovedel jí Samson: Kdyby mne svázali sedmi houžvemi surovými, kteréž ješte neuschly, tedy zemdlím, a budu jako jiný clovek.
\par 8 I prinesli jí knížata Filistinská sedm houžví surových, kteréž ješte neuschly, a svázala ho jimi.
\par 9 (V zálohách pak nastrojeni byli nekterí u ní v komore.) I rekla jemu: Filistinští na te, Samsone. A on roztrhl houžve, jako by pretrhl nit koudelnou, pristrce k ohni, a není poznána síla jeho.
\par 10 Tedy rekla Dalila Samsonovi: Aj, oklamals mne a lžive jsi mi mluvil. Prosím, oznam mi nyní, cím bys mohl svázán býti?
\par 11 Kterýž odpovedel jí: Kdyby mne tuze svázali novými provazy, jimiž by ješte nic deláno nebylo, tedy zemdlím a budu jako kdokoli jiný z lidí.
\par 12 I vzala Dalila provazy nové a svázala ho jimi, a rekla jemu: Filistinští na te, Samsone. (Zálohy pak nastrojeny byly v komore.) I roztrhl je na rukou svých jako nitku.
\par 13 Tedy rekla Dalila Samsonovi: Až dosavad jsi mne svodil, a mluvils mi lež. Poveziž mi, cím bys svázán býti mohl. Odpovedel jí: Kdybys privila sedm pramenu z vlasu hlavy mé k vratidlu tkadlcovskému.
\par 14 Což ucinivši, zarazila hrebem, a rekla jemu: Filistinští na te, Samsone. A procítiv ze sna svého, vytrhl hreb, osnovu i s vratidlem.
\par 15 Opet rekla jemu: Kterak ty pravíš: Miluji te, ponevadž srdce tvé není se mnou? Již jsi mne potrikrát oklamal a neoznámils mi, v cem jest tak veliká síla tvá.
\par 16 Když tedy trápila jej slovy svými každého dne, a obtežovala jej, umdlena jest duše jeho, jako by již mel umríti.
\par 17 I otevrel jí cele srdce své a rekl jí: Britva nevešla nikdy na hlavu mou, nebo Nazarejský Boží jsem od života matky své. Kdybych oholen byl, odešla by ode mne síla má, a zemdlel bych a byl jako jiný clovek.
\par 18 Viduci pak Dalila, že by cele otevrel jí srdce své, poslala a zavolala knížat Filistinských temi slovy: Podte ješte jednou, nebo otevrel mi cele srdce své. Tedy prišli knížata Filistinská k ní, nesouce stríbro v rukou svých.
\par 19 I uspala ho na klíne svém a povolala holice, i dala oholiti sedm pramenu vlasu hlavy jeho. I pocala jím strkati, když odešla od neho síla jeho.
\par 20 A rekla: Filistinští na te, Samsone. Procítiv pak ze sna svého, rekl: Vyjdu jako i prvé, a probiji se skrze ne. Nevedel však, že Hospodin odstoupil od neho.
\par 21 Tedy javše ho Filistinští, vyloupili mu oci, a dovedše ho do Gázy, svázali jej dvema retezy železnými. A mlel v dome veznu.
\par 22 Potom pocaly mu vlasy na hlave odrostati po oholení.
\par 23 Knížata pak Filistinská shromáždili se, aby obetovali obet velikou bohu svému Dágonovi a aby se veselili; nebo rekli: Dalt jest buh náš v ruce naše Samsona neprítele našeho.
\par 24 A když uzrel jej lid, chválili boha svého; nebo pravili: Dalt jest buh náš v ruce naše neprítele našeho a zhoubce zeme naší, kterýž mnohé z našich zmordoval.
\par 25 I stalo, když se rozveselilo srdce jejich, že rekli: Zavolejte Samsona, aby kratochvílil pred námi. Tedy povolali Samsona z domu veznu, aby hral pred nimi; i postavili ho mezi sloupy.
\par 26 Nebo rekl Samson pacholeti, kteréž ho za ruku vodilo: Prived mne, at mohu omakati sloupy, na nichž dum stojí, a zpodepríti se na ne.
\par 27 Dum pak plný byl mužu a žen, a byla tam všecka knížata Filistinská, ano i na vrchu okolo trí tisíc mužu a žen, kteríž dívali se, když Samson hral.
\par 28 I volal Samson k Hospodinu, a rekl: Panovníce Hospodine, prosím, rozpomen se na mne, a posilni mne, žádám, toliko aspon jednou, ó Bože, abych se jednou pomstíti mohl za své obe oci nad Filistinskými.
\par 29 Objav tedy Samson oba sloupy prostrední, na nichž dum ten stál, zpolehl na ne, na jeden pravou a na druhý levou rukou svou.
\par 30 Potom rekl Samson: Necht umre život muj s Filistinskými. A nalehl silne, i padl dum na knížata a na všecken lid, kterýž byl v nem; i bylo mrtvých, kteréž pobil on umíraje, více než tech, kteréž pobil, živ jsa.
\par 31 Tedy prišli prátelé jeho, a všecken dum otce jeho, a vzavše jej, odešli, a pochovali jej mezi Zaraha a Estaol v hrobe Manue otce jeho. A on soudil lid Izraelský dvadceti let.

\chapter{17}

\par 1 Byl pak muž nejaký s hory Efraim, jehož jméno bylo Mícha.
\par 2 Kterýž rekl matce své: Ten tisíc a sto stríbrných, kteríž vzati byli tobe, pro než jsi zlorecila a mluvilas prede mnou, hle, stríbro to u mne jest, já jsem je vzal. I rekla matka jeho: Požehnaný jsi, synu muj, od Hospodina.
\par 3 Navrátil tedy ten tisíc a sto stríbrných matce své. I rekla matka jeho: Jižt jsem zajisté posvetila stríbro to Hospodinu z ruky své, a tobe synu svému, aby udelán byl obraz rytý a slitý. Protož nyní dám je tobe.
\par 4 On pak navrátil to stríbro matce své, z nehož vzala matka jeho dve ste stríbrných, a dala zlatníku. I udelal z nich obraz rytý a slitý, kterýž byl v dome Míchove.
\par 5 Mel pak ten Mícha chrám bohu, i udelal efod a terafim, a naplnil ruce jednoho z synu svých, aby mu byl knezem.
\par 6 Toho casu nebylo krále v Izraeli; jeden každý, což se mu za dobré videlo, to cinil.
\par 7 Byl pak mládenec z Betléma Judova, totiž z celedi Judovy, kterýž, jsa Levíta, byl tam pohostinu.
\par 8 Odšel tedy clovek ten z mesta Betléma Judova, aby byl pohostinu, kdež by se mu koli nahodilo. I prišel na horu Efraim, až k domu Míchovu, jda cestou svou.
\par 9 Jemuž rekl Mícha: Odkud jdeš? Odpovedel mu: Já jsem Levíta, z Betléma Judova beru se, abych byl pohostinu, kdež by mi se koli nahodilo.
\par 10 I rekl jemu Mícha: Zustan u mne, a bud mi za otce a za kneze, a budut dávati deset stríbrných na každý rok, a dvoje roucho i stravu tvou. I šel Levíta.
\par 11 Líbilo se pak Levítovi zustati u muže toho, a byl u neho mládenec ten, jako jeden z synu jeho.
\par 12 I posvetil Mícha rukou Levíty, a byl mu mládenec ten za kneze; i bydlil v dome jeho.
\par 13 Rekl pak Mícha: Nynít vím, že mi dobre uciní Hospodin, proto že mám toho Levítu za kneze.

\chapter{18}

\par 1 V tech dnech nebylo krále v Izraeli, a toho casu pokolení Dan hledalo sobe dedicného místa k bydlení, nebo se mu ješte nebylo dostalo dílu u prostred synu Izraelských až do dne toho.
\par 2 Tedy poslali synové Dan z celedi své pet mužu z koncin svých, mužu silných z Zaraha a Estaol, aby spatrili zemi a pilne prohlédli ji, a rekli jim: Jdete, shlédnete zemi. Kterížto když prišli na horu Efraim až do domu Míchova, prenocovali tam.
\par 3 Když pak byli blízko domu Míchova, poznali hlas toho mládence Levíty, a uchýlivše se tam, rekli jemu: Kdo te sem privedl? Co ty zde deláš? A co ty zde máš?
\par 4 Odpovedel jim: Toto mi a toto ucinil Mícha, a ze mzdy najal mne, abych byl jeho knezem.
\par 5 I rekli jemu: Porad se, prosíme, s Bohem, abychom vedeli, zdarí-li se nám cesta naše, kterouž jdeme.
\par 6 Odpovedel jim knez: Jdete v pokoji, Hospodint spravuje cestu vaši, po níž jdete.
\par 7 Tedy odešlo pet mužu tech, a prišli do Lais, a videli lid, kterýž tam byl, bezpecne bydlící, vedlé obyceje Sidonských v zahálce a bezpecnosti, a že nebylo, co by je kormoutiti melo v té zemi, ani kdo by dedicne ujíti chtel království. K tomu i od Sidonských vzdáleni byli, aniž spríznení jaké s kým meli.
\par 8 Když se pak navrátili k bratrím svým do Zaraha a Estaol, rekli jim bratrí jejich: Což vy?
\par 9 I odpovedeli: Vstante a táhneme na ne, nebo shlédli jsme tu zemi, a aj, velmi dobrá jest; a vy mlcíte? Nelenujtež se táhnouti, a vjíti k opanování té zeme.
\par 10 (Když prijdete, vejdete k lidu bezpecnému, a do zeme prostranné;) nebo dal ji Buh v ruku vaši, místo, v nemž není žádného nedostatku jakýchkoli vecí, kteréž na zemi býti mohou.
\par 11 Tedy vyšlo z celedi Dan odtud, totiž z Zaraha a Estaol, šest set mužu odených v odení válecné.
\par 12 A vytáhše, položili se u Kariatjeharim Judova; procež nazvali to místo Mahane Dan až do dnešního dne, a jest za Kariatjeharim.
\par 13 A odtud táhnouce na horu Efraim, prišli až k domu Míchovu.
\par 14 I mluvilo tech pet mužu, kteríž chodili k shlédnutí zeme Lais, a rekli bratrím svým: Víte-liž, že v domích techto jest efod a terafim, a rytina a slitina? Protož nyní vezte, co máte ciniti.
\par 15 A uchýlivše se tam, vešli do domu mládence Levíty v dome Míchove, a pozdravili ho pokojne.
\par 16 Ale šest set mužu odených v zbroj svou válecnou, kteríž byli z pokolení Dan, stáli prede dvermi.
\par 17 A šedše pet mužu, kteríž chodili k shlédnutí zeme, vešli tam a vzali rytinu a efod a terafim a slitinu; knez pak stál u vrat brány s šesti sty muži odenými v zbroji.
\par 18 A ti, kteríž vešli do domu Míchova, vzali rytinu, efod a terafim, a slitinu. I rekl jim knez: Což to deláte?
\par 19 Kteríž odpovedeli: Mlc, vlož ruku svou na ústa svá a pod s námi, a budeš nám za otce a za kneze. Což jest lépe tobe, knezem-li býti v dome jednoho cloveka, ci býti knezem pokolení a celedi v Izraeli?
\par 20 I zradovalo se srdce kneze, a vzav efod a terafim a rytinu, šel u prostred lidu toho.
\par 21 A obrátivše se odešli, a pustili napred deti a dobytek, a což meli dražšího.
\par 22 Když pak opodál byli od domu Míchova, tedy muži, kteríž bydlili v domích blízkých domu Míchova, shromáždili se a honili syny Dan.
\par 23 I volali za syny Dan. Kteríž ohlédše se, rekli Míchovi: Cožte, že jsi jich tolik shromáždil?
\par 24 Odpovedel: Bohy mé, kteréž jsem udelal, vzali jste, i kneze, a odcházíte. Což pak již budu míti? A ješte se ptáte: Cot jest?
\par 25 Jemuž odpovedeli synové Dan: Hlediž, at více neslyšíme hlasu tvého za sebou, sic jinác oborí se na vás muži hneviví, a ztratíš duši svou i duše domu svého.
\par 26 I brali se muži Dan cestou svou. A vida Mícha, že by silnejší byli nežli on, obrátiv se, šel do domu svého.
\par 27 Oni pak vzavše, což byl udelal Mícha, i kneze, kteréhož mel, pritáhli do Lais k lidu zahálivému a bezpecnému; i pobili je ostrostí mece, a mesto vypálili ohnem.
\par 28 A nebylo žádného, kdo by jim spomohl; nebo daleko byl Sidon, aniž meli spríznení s kterými lidmi. Mesto pak bylo v údolí, kteréž jest v Betrohob. A vystavevše zase mesto, bydlili v nem.
\par 29 A nazvali jméno mesta toho Dan, od jména otce svého, kterýž narozen byl Izraelovi, ješto prvé jméno mesta toho bylo Lais.
\par 30 Postavili pak sobe synové Dan tu rytinu, a Jonatan syn Gersonuv, syna Mojžíšova, on i synové jeho byli knežími v pokolení Dan, až do dne zajetí obyvatelu zeme.
\par 31 Vystavili tedy sobe tu rytinu, kterouž udelal Mícha, a byla tam po všecky dny,v nichž dum Boží byl v Sílo.

\chapter{19}

\par 1 Stalo se také toho casu, když krále nebylo v Izraeli, že muž nejaký Levíta, jsa pohostinu pri strane hory Efraimské, pojal sobe ženu ženinu z Betléma Judova.
\par 2 Kterážto ženina smilnila u neho. I odešla od neho do domu otce svého do Betléma Judova, a byla tam plné ctyri mesíce.
\par 3 Vstav pak muž její, šel za ní, aby namluve ji, zase ji privedl, maje s sebou mládence svého a dva osly. Tedy ona uvedla jej do domu otce svého. Kteréhož když uzrel otec té devky, zradoval se z príchodu jeho.
\par 4 I zdržel jej tchán jeho, otec té devky, tak že pozustal u neho za tri dni. Tu také jídali i píjeli i nocovali.
\par 5 Dne pak ctvrtého, když tím raneji vstali, vstal i on, aby odšel. Tedy rekl otec té devky k zeti svému: Posilni se kouskem chleba, a potom pujdete.
\par 6 Sedli tedy a pojedli oba spolu, a napili se. Potom rekl otec devky k muži: Posediž medle, nýbrž pobud pres noc, a bud mysli veselé.
\par 7 Když pak vstal ten muž, chteje predce jíti, mocí jej zdržel test jeho. A tak se vrátil a zustal tu pres noc.
\par 8 Potom dne pátého vstal tím raneji, aby se bral. I rekl otec té devky: Posiln se, prosím. I prodlili, až se den nachýlil, nebo jedli oba.
\par 9 Tedy vstal muž ten, aby šel, on i ženina jeho i mládenec jeho. I rekl mu tchán jeho, otec devky: Aj, již se den nachýlil k vecerou, medle zustante pres noc; aj, dokonává se den, pobud pres noc zde, a bud mysli veselé, a zítra tím raneji vypravíte se na cestu svou, a pujdeš k príbytku svému.
\par 10 On pak nechtel zustati pres noc, ale vstav, odšel, a prišel proti Jebus, jenž jest Jeruzalém, a s ním dva oslové s bremeny i ženina jeho.
\par 11 Když pak byli blízko Jebus, a den se velmi nachýlil, rekl mládenec pánu svému: Pod, prosím, obratme se do mesta toho Jebuzejského, abychom v nem prenocovali.
\par 12 Jemuž odpovedel pán jeho: Neobrátíme se do mesta cizozemcu, kteréž není synu Izraelských, ale pujdeme až do Gabaa.
\par 13 Rekl ješte mládenci svému: Pod, abychom prišli k nekterému z tech míst, a zustali pres noc v Gabaa aneb v Ráma.
\par 14 Pomíjejíce tedy, odešli, a zapadlo jim slunce blízko Gabaa, kteréž jest Beniaminských.
\par 15 I obrátili se tam, aby vejdouce, zustali pres noc v Gabaa. A když tam všel, posadil se na ulici mesta, proto že nebyl, kdo by je prijal do domu a dal jim nocleh.
\par 16 A aj, muž starý vracoval se od práce své s pole u vecer, kterýž také byl s hory Efraimovy, a bydlil pohostinu v Gabaa; ale lidé místa toho byli synové Jemini.
\par 17 A když pozdvihl ocí svých, uzrel muže toho pocestného na ulici mesta. I rekl jemu ten starec: Kam se béreš, a odkud jdeš?
\par 18 Jemuž odpovedel: Jdeme z Betléma Judova až k stranám hory Efraimovy, odkudž jsem; nebo jsem byl odšel do Betléma Judova. Jdut pak do domu Hospodinova, a není žádného, kdo by mne prijal do domu;
\par 19 Ješto mám i slámu a obrok pro osly své, tolikéž i chléb, ano i víno pro sebe a devku tvou, a mládence, kterýž jest s služebníkem tvým, tak že v nicemž nemáme nedostatku.
\par 20 I rekl muž ten starý: Mej ty pokoj. Cehot se koli nedostává, necht já to opatrím, ty toliko na ulici nezustávej pres noc.
\par 21 Tedy uvedl jej do domu svého, a obrok dal oslum; potom umyvše nohy své, jedli a pili.
\par 22 A když ocerstvili srdce své, aj, muži mesta toho, muži nešlechetní, obklícivše dum, tloukli na dvére a mluvili tomu muži starci, hospodári domu, rkouce: Vyved muže toho, kterýž všel do domu tvého, abychom ho poznali.
\par 23 K nimžto vyšed muž ten, hospodár domu, rekl jim: Nikoli, bratrí moji, necinte, prosím, zlého, ponevadž všel ten muž do domu mého, neprovodte nešlechetnosti té.
\par 24 Aj, dceru svou, kteráž pannou jest, a ženinu jeho, ty hned vyvedu, i ponížíte jich, aneb uciníte jim, což se vám za dobré vidí; jen muži tomu necinte veci té hanebné.
\par 25 Ale nechteli ho uposlechnouti muži ti. Pojav tedy muž ten ženinu svou, vyvedl ji k nim ven, i poznali ji, a zle jí požívali pres celou noc až do jitra; potom pustili ji, když pocínalo zasvitávati.
\par 26 V svitání pak prišla žena ta, a padši, ležela u dverí domu toho muže, kdež byl pán její, až se rozednilo.
\par 27 Když pak vstal pán její ráno, otevrev dvére domu, vycházel, aby se bral dále cestou svou. A aj, žena ta, ženina jeho, ležela u dverí domu, a ruce její byly na prahu.
\par 28 Jížto rekl: Vstan, a podme. A nic neodpovedela. Vzav tedy ji na osla, a vstav muž ten, odšel k místu svému.
\par 29 Když pak prišel do domu svého, vzal mec, a pochytiv ženinu svou, rozsekal ji s kostmi jejími na dvanácte kusu, a rozeslal ji po všech koncinách Izraelských.
\par 30 A bylo, že kdožkoli uzrel, pravil: Nikdy se nestalo ani vidíno bylo co podobného od toho casu, jakž vyšli synové Izraelští z zeme Egyptské, až do tohoto dne. Posudte toho pilne, poradte se a promluvte o to.

\chapter{20}

\par 1 I vyšli všickni synové Izraelští, a shromáždilo se všecko množství jednomyslne od Dan až do Bersabé, i zeme Galád, k Hospodinu do Masfa.
\par 2 Kdežto postavili se prední všeho lidu, všecka pokolení Izraelská v shromáždení lidu Božího, ctyrikrát sto tisíc lidu pešího válecného.
\par 3 (Uslyšeli pak synové Beniamin, že by sešli se synové Izraelští v Masfa.) I rekli synové Izraelští: Povezte, kterak se stala nešlechetnost ta?
\par 4 I odpovedev muž Levíta, manžel ženy zamordované, rekl: Do Gabaa kteréž jest Beniaminovo, prišel jsem s ženinou svou, abych tam prenocoval.
\par 5 Tedy povstavše proti mne muži Gabaa, obklícili mne v dome v noci, myslíce mne zamordovati, ženinu pak mou trápili, tak že umrela.
\par 6 Procež vzav ženinu svou, rozsekal jsem ji na kusy, a rozeslal jsem ji do všech krajin dedictví Izraelského; nebo nešlechetnosti a mrzkosti se dopustili v Izraeli.
\par 7 Aj, všickni vy synové Izraelští jste; považte toho mezi sebou, a radte se o to.
\par 8 A povstav všecken lid jednomyslne, rekli: Nenavrátí se žádný z nás do príbytku svého, aniž odejde kdo do domu svého.
\par 9 Ale nyní toto uciníme Gabaa, losujíce proti nemu.
\par 10 Vezmeme deset mužu ze sta po všech pokoleních Izraelských, a sto z tisíce, a tisíc z desíti tisícu, aby dodávali potravy lidu, kterýž by pritáhna do Gabaa Beniaminova, pomstil všech nešlechetností jeho, kterýchž se dopustilo v Izraeli.
\par 11 I sebrali se všickni muži Izraelští na to mesto, snesše se za jednoho cloveka.
\par 12 Poslala pak pokolení Izraelská muže do všech celedí synu Beniamin, rkouce: Jaký to zlý skutek stal se mezi vámi?
\par 13 Nyní tedy vydejte ty muže bezbožné, kteríž jsou v Gabaa, at je zbijeme a odejmeme zlé z Izraele. Ale nechteli Beniaminští slyšeti hlasu bratrí svých, synu Izraelských.
\par 14 Nýbrž shromáždili se synové Beniamin z mest svých do Gabaa, aby vytáhli k boji proti synum Izraelským.
\par 15 Toho dne nacteno jest synu Beniamin z mest jejich dvadceti šest tisíc mužu bojovných, krome obyvatelu Gabaa, jichž nacteno bylo šest set mužu vybraných.
\par 16 Mezi kterýmžto vším lidem bylo sedm set mužu vybraných, neužívajících pravé ruky své, z nichž každý z praku kamením házeli k vlasu, a nechybovali se.
\par 17 Mužu pak Izraelských nacteno jest krome Beniaminských ctyrikrát sto tisíc mužu bojovných; všickni tito byli muži udatní.
\par 18 Vstavše pak, brali se do domu Boha silného, a tázali se Boha, a rekli synové Izraelští: Kdo z nás pujde napred k boji proti synum Beniamin? I rekl Hospodin: Juda pujde napred.
\par 19 A tak ráno vstavše synové Izraelští, položili se proti Gabaa.
\par 20 I táhli muži Izraelští k boji proti synum Beniamin, a sšikovali se muži Izraelští k bitve proti Gabaa.
\par 21 Vyšedše pak synové Beniamin z Gabaa, porazili z Izraele toho dne dvamecítma tisíc mužu na zem.
\par 22 A posilnivše se muži lidu Izraelského, sporádali se zase k boji na míste, na kterémž se prvního dne zrídili.
\par 23 Prvé pak šli synové Izraelští, a plakali pred Hospodinem až do vecera. I tázali se Hospodina temi slovy: Pujdeme-li ješte k boji proti synum Beniamina bratra našeho? Odpovedel Hospodin: Jdete proti nim.
\par 24 Tedy potýkali se synové Izraelští s syny Beniamin druhého dne.
\par 25 A vyšedše synové Beniamin z Gabaa na ne druhého dne, porazili z synu Izraelských opet osmnácte tisíc mužu na zem, vše mužu bojovných.
\par 26 Protož vstoupili všickni synové Izraelští a všecken lid, a prišli do domu Boha silného. I plakali, usadivše se tam pred Hospodinem, a postili se toho dne až do vecera; obetovali též obeti zápalné a pokojné pred Hospodinem.
\par 27 I tázali se synové Izraelští Hospodina, (nebo tu byla truhla smlouvy Boží v tech dnech,
\par 28 A Fínes syn Eleazara, syna Aronova, stál pred ní v ten cas), rkouce: Pujdeme-li ješte k boji proti synum Beniamina bratra našeho, cili tak necháme? Odpovedel Hospodin: Jdete, nebo zítra dám je v ruku vaši.
\par 29 Tedy Izraelští zdelali zálohy proti Gabaa všudy vukol.
\par 30 I šli synové Izraelští proti synum Beniamin dne tretího, a sšikovali se proti Gabaa, jako prvé jednou i podruhé.
\par 31 Vyšedše pak synové Beniamin proti lidu, odtrhli se od mesta, a pocali bíti a mordovati lidu, jako prvé jednou i druhé po stezkách, (z nichž jedna šla k Bethel a druhá do Gabaa), i po poli, a  okolo tridcíti mužu z Izraele.
\par 32 A rekli synové Beniamin: Padají pred námi jako i prvé. Synové pak Izraelští rekli byli: Utíkejme, abychom je odtrhli od mesta až k stezkám.
\par 33 A v tom všickni synové Izraelští vstavše z místa svého, sšikovali se v Baltamar; zálohy také Izraelovy vyskocily z místa svého z trávníku Gabaa.
\par 34 Tedy vyšlo proti Gabaa deset tisíc mužu vybraných ze všeho Izraele, a bitva se rozmáhala; oni pak nevedeli o tom, že je potkati melo zlé.
\par 35 I porazil Hospodin Beniamina pred Izraelem, a zbili synové Izraelští z Beniaminských dne toho petmecítma tisíc a sto mužu, vše bojovných.
\par 36 A vidouce synové Beniamin, že by poraženi byli, (nebo muži Izraelští ustupovali z místa Beniaminským, ubezpecivše se na zálohy, kteréž zdelali proti Gabaa.
\par 37 Zálohy pak pospíšily a oborily se na Gabaa, a rozvlácne troubivše zálohy, zbily všecko mesto ostrostí mece.
\par 38 Meli pak muži Izraelští s zálohami jistý cas uložený, aby když by oni zapálili mesto,
\par 39 Obrátili se synové Izraelští k boji. Beniaminští pak pocali bíti a mordovati, a zbili z synu Izraelských okolo tridcíti mužu; nebo rekli: Jiste že padají pred námi jako v první bitve.
\par 40 Ohen pak pocal vzhuru jíti z mesta, a sloup dymový. A ohlédše se Beniaminští zpet, uzreli, an vstupuje ohen mesta k nebi.)
\par 41 A že muži Izraelští obrátili se, i zdešeni jsou muži Beniamin, nebo videli, že zahynutí jim nastává.
\par 42 I utíkali pred muži Izraelskými cestou ku poušti, a bojovníci postihali je, a kterí z mest vyšli, mordovali je mezi sebou.
\par 43 A tak obklícili Beniaminské, a honili i porazili je, od Manuha až naproti Gabaa k východu slunce.
\par 44 Tedy padlo Beniaminských osmnácte tisíc mužu; všickni ti byli muži silní.
\par 45 Tech pak, kteríž obrátivše se, utíkali na poušt k skále Remmon, zpaberovali po cestách pet tisíc mužu; potom honili je až k Gidom, a zbili z nich dva tisíce mužu.
\par 46 A tak bylo všech, kteríž padli v ten den z Beniaminských, petmecítma tisíc mužu bojovných, vše mužu silných.
\par 47 I obrátilo se na poušt a uteklo k skále Remmon šest set mužu, kteríž zustali v skále Remmon za ctyri mesíce.
\par 48 Potom muži Izraelští navrátili se k synum Beniamin, a zbili je ostrostí mece, tak lidi v mestech jako hovada, i všecko, což nalezeno bylo; také i všecka mesta, kteráž ješte pozustávala, ohnem vypálili.

\chapter{21}

\par 1 Nadto každý z mužu Izraelských prísahou se byli zavázali v Masfa, rkouce: Žádný z nás nedá dcery své Beniaminským za manželku.
\par 2 Protož všel lid do domu Boha silného, a sedeli tam až do vecera pred Bohem, a pozdvihše hlasu svého, plakali plácem velikým.
\par 3 A rekli: Proc Hospodine, Bože Izraelský, stalo se toto v Izraeli, aby dnes ubylo jedno pokolení z Izraele?
\par 4 Nazejtrí pak ráno vstal lid, a vzdelali tam oltár, a obetovali obeti zápalné a pokojné.
\par 5 Rekli pak synové Izraelští: Jest-li kdo, ješto neprišel do shromáždení tohoto ze všech pokolení Izraelských k Hospodinu? (Nebo se byli velice zaprisáhli proti tomu, kdož by neprišel k Hospodinu do Masfa, rkouce: Bez milosti at umre.
\par 6 Nebo litujíce synové Izraelští Beniamina bratra svého, rekli: Dnes vyhlazeno jest jedno pokolení z Izraele.
\par 7 Jakž tedy uciníme s temi ostatky, aby ženy meli, ponevadž jsme se prísahou zavázali skrze Hospodina, že jim nedáme dcer svých za manželky?)
\par 8 Rekli tedy: Jest-li kdo z pokolení Izraelských, ješto neprišel k Hospodinu do Masfa? A aj, neprišel byl žádný do vojska z Jábes Galád do shromáždení.
\par 9 Nebo když secten byl lid, a aj, nebylo tam žadného z obyvatelu Jábes Galád.
\par 10 Protož poslalo tam shromáždení to dvanácte tisíc mužu nejsilnejších, a prikázali jim, rkouce: Jdete a pobíte obyvatele Jábes Galád ostrostí mece, ženy i deti.
\par 11 Toto pak uciníte: Všecky mužského pohlaví, a každou ženu, kteráž muže poznala, zamordujete.
\par 12 Nalezli tedy mezi obyvateli Jábes Galád ctyri sta devecek panen, kteréž nepoznaly muže, a privedli je do vojska v Sílo, kteréž bylo v zemi Kananejské.
\par 13 Tedy poslalo všecko to shromáždení, a mluvili k synum Beniamin, kteríž byli v skále Remmon, a povolali jich v pokoji.
\par 14 Protož navrátili se Beniaminští toho casu. I dali jim ženy, kteréž živé zachovali z žen Jábes Galád, ale ani tak se jim nedostávalo jich.
\par 15 Lidu pak líto bylo Beniamina, proto že ucinil Hospodin mezeru v pokoleních Izraelských.
\par 16 Rekli tedy starší shromáždení toho: Jak uciníme s temi pozustalými, aby meli ženy? Nebo vyhlazeny jsou ženy z pokolení Beniamin.
\par 17 Rekli také: Dedictví Beniaminovo pozustalým náleží, aby nezahynulo pokolení z Izraele.
\par 18 My pak nemužeme jim dáti dcer svých za manželky; (nebo se byli prísahou zavázali synové Izraelští, rkouce: Zlorecený bud, kdož by dal manželku synum Beniamin.)
\par 19 Potom rekli: Aj, slavnost Hospodinova bývá v Sílo každého roku na míste, kteréž jest s pulnocní strany domu Boha silného, k východu slunce ceste, kterouž se chodí od domu Boha silného do Sichem, a Lebnu jest na poledne.
\par 20 Prikázali tedy synum Beniamin, rkouce: Jdete a skrejte se v vinicích.
\par 21 A šetrte, a aj, když vyjdou dcery Sílo plésati v houfích, tedy vyskocíte z vinic a pochytíte sobe každý manželku svou ze dcer Sílo, a odejdete do zeme Beniamin.
\par 22 Když pak prijdou otcové jejich aneb bratrí jejich, aby se pred námi soudili, tedy rekneme jim: Slitujte se nad námi místo nich, nebo v té válce nevzali jsme pro každého z nich manželky; také jste vy jim nedali jich, a tak nebudete nic vinni.
\par 23 Tedy ucinili tak synové Beniamin, a privedli sobe manželky vedlé poctu svého z tech plésajících, kteréž uchvátili; a odšedše, navrátili se k dedictví svému, a vzdelavše zase mesta svá, bydlili v nich.
\par 24 A tak odešli odtud synové Izraelští toho casu, jeden každý k svému pokolení a k celedi své; a vrátili se odtud jeden každý k dedictví svému.
\par 25 Tech dnu nebylo krále v Izraeli, ale každý, což se mu videlo, to cinil.

\end{document}