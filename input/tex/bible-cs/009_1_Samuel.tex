\begin{document}

\title{1 Samuelova}

\chapter{1}

\par 1 Byl muž nejaký z Ramataim Zofim, s hory Efraim, jehož jméno bylo Elkána, syn Jerochama, syna Elihu, syna Tohu, syna Zuf Efratejského.
\par 2 A ten mel dve ženy, jméno jedné Anna, a jméno druhé Penenna. Mela pak Penenna deti, ale Anna nemela detí.
\par 3 I chodíval muž ten z mesta svého každého roku, aby se klanel a obetoval Hospodinu zástupu, do Sílo, kdež byli dva synové Elí, Ofni a Fínes, kneží Hospodinovi.
\par 4 Když pak prišel den, v nemž obetoval Elkána, dal Penenne manželce své, a všechnem synum i dcerám jejím díly.
\par 5 Anne pak dal díl jeden výborný, nebo Annu miloval, ale Hospodin zavrel byl život její.
\par 6 Presto kormoutila ji také velmi protivnice její, toliko aby ji popouzela, proto že Hospodin zavrel byl život její.
\par 7 To když ciníval každého roku, a Anna též chodívala do domu Hospodinova, tak ji kormoutívala protivnice; ona pak plakávala a nic nejídala.
\par 8 Tedy rekl jí Elkána muž její: Anna, proc pláceš? A proc nejíš? Proc tak truchlí srdce tvé? Zdaliž já nejsem tobe lepší nežli deset synu?
\par 9 Vstala tedy Anna, když pojedli v Sílo a napili se; a Elí knez sedel na stolici u vereje chrámu Hospodinova.
\par 10 Ona pak jsuci v horkosti srdce, modlila se Hospodinu a plakala velmi.
\par 11 A ucinila slib, rkuci: Hospodine zástupu, jestliže vzhlédneš na trápení devky své a rozpomeneš se na mne, a nezapomeneš na devku svou, ale dáš služebnici své plod pohlaví mužského: tedy dám jej tobe, Hospodine, po všecky dny života jeho, a britva nevejde na hlavu jeho.
\par 12 I stalo se, když se dlouho modlila pred Hospodinem, že Elí pozor mel na ústa její.
\par 13 Ale Anna mluvila v srdci svém; toliko rtové její se hýbali, hlasu pak jejího nebylo slyšeti. I domníval se Elí, že by opilá byla.
\par 14 Protož rekl jí Elí: Dlouho-liž budeš opilá? Vystrízvej z vína svého.
\par 15 Odpovedela Anna, rkuci: Nikoli, pane muj, žena jsem ducha truchlivého, ani vína ani nápoje opojného jsem nepila, ale vylila jsem duši svou pred Hospodinem.
\par 16 Neprirovnávejž devky své k žene bezbožné, nebo z velikého myšlení a horkosti své mluvila jsem až dosavad.
\par 17 Jíž odpovedel Elí, rka: Jdiž u pokoji, a Buh Izraelský dejž tobe k prosbe tvé, zac jsi ho prosila.
\par 18 I rekla: Ó by nalezla devka tvá milost pred ocima tvýma! Tedy odšedši žena cestou svou, pojedla, a tvár její nebyla více smutná.
\par 19 I vstali velmi ráno, a poklonu ucinivše pred Hospodinem, navrátili se, a prišli do domu svého do Ramata. Poznal pak Elkána Annu manželku svou, a Hospodin rozpomenul se na ni.
\par 20 I stalo se po vyplnení dnu, jakž pocala Anna, že porodila syna, a nazvala jméno jeho Samuel; nebo rekla: Vyprosila jsem ho na Hospodinu.
\par 21 Šel pak muž ten Elkána se vší celedí svou, aby obetoval Hospodinu obet výrocní a slib svuj.
\par 22 Ale Anna nešla, nebo rekla muži svému: Až odchovám díte, tehdy povedu je, aby ukáže se pred Hospodinem, zustalo tam na veky.
\par 23 I rekl jí Elkána muž její: Ucin, cožt se dobrého vidí, zustan, dokudž neodchováš jeho. Ó by toliko utvrdil Hospodin slovo své! A tak zustala žena, a krmila syna svého, až jej i odchovala.
\par 24 Potom, když ho odchovala, vedla jej s sebou, se trmi volky a jednou efi mouky a nádobou vína, i uvedla jej do domu Hospodinova v Sílo; díte pak ješte bylo malé.
\par 25 Tedy zabili volka a privedli díte k Elí.
\par 26 Ona pak rekla: Poslyš mne, pane muj. Jako jest živa duše tvá, pane muj, já jsem žena ta, kteráž jsem stála tuto s tebou, modleci se Hospodinu.
\par 27 Za toto díte jsem prosila, a dal mi Hospodin k prosbe mé to, cehož jsem prosila od neho.
\par 28 Protož já také oddávám jej Hospodinu; po všecky dny, v nichž živ bude, oddanýt jest Hospodinu. A ucinil tu poklonu Hospodinu.

\chapter{2}

\par 1 I modlila se Anna a rekla: Zplésalo srdce mé v Hospodinu, a vyzdvižen jest roh muj v Hospodinu; rozšírila se ústa má proti neprátelum mým, nebo rozveselila jsem se v spasení tvém.
\par 2 Nenít žádný tak svatý jako Hospodin; nýbrž žádného není krome tebe, nenít žádný tak silný jako Buh náš.
\par 3 Nemluvtež již více hrde, a nevycházej z úst vašich slovo pyšné; nebo Buh silný vševedoucí jest Hospodin, a usilování jeho jemu nepochybují.
\par 4 Lucište i silní potríni jsou, a mdlí opásáni jsou silou.
\par 5 Sytí z chleba jednají se k dílu, a hladovití prestali lacneti, tak že neplodná porodila sedmero, a kteráž mnoho detí mela, zemdlena jest.
\par 6 Hospodin umrtvuje i obživuje, uvodí do pekla i vyvodí.
\par 7 Hospodin ochuzuje i zbohacuje, ponižuje i povyšuje.
\par 8 Nuzného vyzdvihuje z prachu, a z hnoje vyvyšuje chudého, aby je posadil s knížaty, a stolici slávy dedicne jim dal; nebo Hospodinovy jsou stežeje zeme, na nichž založil okršlek.
\par 9 Ont ostríhá noh svatých svých, ale bezbožní ve tme umlknou; nebo ne v síle záleží síla cloveka.
\par 10 Kteríž se protiví Hospodinu, setríni budou, na takové s nebe hrímati bude. Hospodin souditi bude konciny zeme, a dá sílu králi svému, a vyvýší roh pomazaného svého.
\par 11 I navrátil se Elkána do Ramaty, do domu svého, a pacholátko prisluhovalo Hospodinu pri knezi Elí.
\par 12 Synové pak Elí byli bezbožní, a neznali Hospodina.
\par 13 Nebo tech kneží obycej pri lidu byl: Kdokoli obetoval obet, pricházel knežský mládenec, když se maso varilo, maje v ruce své hák trízubý,
\par 14 A vrazil jej do nádoby neb do kotlíku, neb do pánve, aneb do hrnce, a cokoli zachytil hák, to sobe bral knez. Tak cinívali všemu lidu Izraelskému, kteríž tam pricházeli do Sílo.
\par 15 Nýbrž, prvé než tuk zapalovali, pricházel mládenec knežský a ríkal cloveku, kterýž obetoval: Dej masa, at upeku knezi, nebo nevezme od tebe masa vareného, ale surové.
\par 16 Jemuž odpovedel-li clovek ten: Necht jest prvé zapálen tuk, a potom vezmeš sobe, cehož žádá duše tvá, on ríkal: Nikoli, ale dej hned; jestliže pak nedáš, mocí vezmu.
\par 17 I byl to hrích mládencu tech pred Hospodinem velmi veliký; nebo pohrdali lidé obetmi Hospodinovými.
\par 18 Samuel pak prisluhoval pred Hospodinem, mládencek, odený jsa efodem lneným.
\par 19 A matka jeho delávala mu suknicku malou, a prinášela jemu každého roku, když pricházela s mužem svým k obetování obeti výrocní.
\par 20 I požehnal Elí Elkánovi a manželce jeho, rka: Dejž tobe Hospodin síme z ženy této za oddaného, kteréhož jsi vyžádal na Hospodinu. I odešli na místo své.
\par 21 Tedy navštívil Hospodin Annu, kteráž pocala a porodila tri syny a dve dcery. Ale mládencek Samuel rostl pred Hospodinem.
\par 22 Elí pak byl starý velmi a slyšel všecky veci, kteréž cinili synové jeho všemu Izraeli, a kterak obývali s ženami, kteréž prisluhovaly u dverí stánku úmluvy.
\par 23 I rekl jim: Proc ciníte takové veci? Nebo já slyším zlé slovo o vás ode všeho lidu tohoto.
\par 24 Nikoli, synové moji, nebot není dobrá povest, kterouž já slyším, že odvracíte lid Hospodinuv.
\par 25 Zhreší-li clovek proti cloveku, souditi ho bude soudce; pakli kdo zhreší proti Hospodinu, kdo se zasadí o neho? Ale neuposlechli hlasu otce svého, nebo je chtel zbíti Hospodin.
\par 26 Ale mládencek Samuel prospíval a rostl, a milý byl, jakož pred Hospodinem, tak i pred lidmi.
\par 27 I prišel muž Boží k Elí a rekl jemu: Takto praví Hospodin: Zdaliž jsem se patrne nezjevil domu otce tvého, když byli v Egypte, v dome Faraonove?
\par 28 A vyvolil jsem ho sobe ze všech pokolení Izraelských za kneze, aby obetoval na oltári mém, a aby kadil vonnými vecmi a nosil efod prede mnou; dal jsem také domu otce tvého všecky ohnivé obeti synu Izraelských.
\par 29 Proc pošlapáváte obet mou, i suchou obet mou, kterouž jsem prikázal v príbytku tomto? A víces ctil syny své nežli mne, abyste tyli prvotinami všech obetí suchých Izraele lidu mého.
\par 30 Protož praví Hospodin Buh Izraelský: Reklt jsem byl zajisté, dum tvuj a dum otce tvého prisluhovati mel prede mnou až na veky, ale nyní praví Hospodin: Odstup to ode mne; nebo tech, kteríž mne ctí, poctím, kteríž pak mnou pohrdají, v pohrdání prijdou.
\par 31 Aj, dnové pricházejí, že odetnu ráme tvé a ráme domu otce tvého, aby nebylo starce v dome tvém.
\par 32 A videti budeš protivníka v príbytku Páne, ve všem, cehož Buh štedre udíleti bude Izraelovi, aniž bude starce v dome tvém po všecky dny.
\par 33 A ackoli nevyhladím do konce muže od oltáre svého, abych trápil oci tvé a bolestí ssužoval duši tvou, všecko však množství domu tvého zemrou v veku vyspelém.
\par 34 A to mej za znamení, což prijde na dva syny tvé, Ofni a Fínesa: oba jednoho dne umrou.
\par 35 Vzbudím pak sobe kneze verného, kterýž vedlé srdce mého a vedlé duše mé vše konati bude, a vzdelám jemu dum stálý, aby prisluhoval pred pomazaným mým po všecky dny.
\par 36 A budet, že kdožkoli pozustane z domu tvého, prijde, aby se poklonil jemu, pro peníz stríbrný a pro skyvu chleba, a rekne: Pripust mne, prosím, k jedné tríde knežské, abych jedl chléb.

\chapter{3}

\par 1 Mládencek pak Samuel prisluhoval Hospodinu pri Elí, a rec Hospodinova byla vzácná v tech dnech, aniž bývalo videní zjevného.
\par 2 Stalo se pak jednoho dne, když Elí ležel na míste svém, (a již byl pocal scházeti na oci, a nemohl videti),
\par 3 Samuel také spal, a svetlo Boží ješte zhašeno nebylo v chráme Hospodinove, v nemž byla truhla Boží,
\par 4 Že zavolal Hospodin Samuele. Kterýž rekl: Ted jsem.
\par 5 I bežel k Elí a rekl: Ted jsem, nebo jsi mne volal. I odpovedel: Nevolal jsem, jdi zase spáti. Kterýž odšed, spal.
\par 6 Opet pak Hospodin zavolal Samuele. A vstav Samuel, šel k Elí a rekl: Ted jsem, nebo jsi mne volal. I odpovedel: Nevolalt jsem, synu muj; jdi zase, spi.
\par 7 Samuel pak ješte neznal Hospodina, a ješte nebyla mu zjevena rec Hospodinova.
\par 8 Tedy opet Hospodin zavolal Samuele po tretí. Kterýžto vstav, šel k Elí a rekl: Ted jsem, nebo jsi mne volal. Tedy srozumel Elí, že byl Hospodin volal mládence.
\par 9 I rekl Elí k Samuelovi: Jdi, spi, a bude-li te volati, rekneš: Mluv, Hospodine, nebo slyší služebník tvuj. Odšel tedy Samuel, a spal na míste svém.
\par 10 I prišel Hospodin, a stál a zavolal, jako i prvé, a rekl: Samueli, Samueli! Odpovedel Samuel: Mluv, nebo slyší služebník tvuj.
\par 11 I rekl Hospodin Samuelovi: Aj, já uciním vec takovou v Izraeli, kterouž kdokoli uslyší, zníti jemu bude v obou uších jeho.
\par 12 V ten den uvedu na Elí všecko to, což jsem mluvil proti domu jeho; pocnut i dokonám.
\par 13 A ukáži jemu, že já soudím dum jeho až na veky pro nepravost, o níž vedel; nebo znaje, že na se zlorecenství uvodí synové jeho, a však nezbránil jim.
\par 14 A protož jsem zaprisáhl domu Elí, že nebude vycištena nepravost domu Elí žádnou obetí, ani obetí suchou až na veky.
\par 15 I spal Samuel až do jitra, a otevrel dvére domu Hospodinova; ostýchal se pak Samuel oznámiti Elí toho videní.
\par 16 Tedy povolal Elí Samuele a rekl: Samueli, synu muj. Kterýž odpovedel: Ted jsem.
\par 17 I rekl: Jaká jest to rec, kterouž mluvil tobe? Netaj medle prede mnou. Toto ucin tobe Buh a toto pridej, jestliže co zatajíš prede mnou ze všech slov, kteráž mluvil tobe.
\par 18 A tak oznámil jemu Samuel všecka slova, a niceho nezatajil pred ním. A on rekl: Hospodint jest, necht uciní, což rácí.
\par 19 Rostl pak Samuel, a Hospodin byl s ním, tak že nedopustil padnouti žádnému slovu jeho na zem.
\par 20 Z cehož poznal všecken Izrael od Dan až do Bersabé, že Samuel jest verný prorok Hospodinuv.
\par 21 Nebo se jemu i potom ukazoval Hospodin v Sílo, jakož se byl prvé zjevil Hospodin Samuelovi v Sílo, skrze rec Hospodinovu.

\chapter{4}

\par 1 I stalo se vedlé reci Samuelovy všemu lidu Izraelskému. Nebo když vytáhl Izrael proti Filistinským k boji, a položili se pri Eben-Ezer, Filistinští pak položili se v Afeku;
\par 2 A když se sšikovali Filistinští proti Izraelovi, a již se potýkali: poražen jest Izrael od Filistinských, tak že jich zbito v té bitve na poli takmer ctyri tisíce mužu.
\par 3 I navrátil se lid do stanu, a rekli starší Izraelští: Proc nás dnes Hospodin porazil pred Filistinskými? Vezmeme k sobe z Sílo truhlu smlouvy Hospodinovy, at prijde mezi nás, a vysvobodí nás z rukou neprátel našich.
\par 4 Poslal tedy lid do Sílo, a vzali odtud truhlu smlouvy Hospodina zástupu, sedícího na cherubínech. Byli také tam dva synové Elí s truhlou smlouvy Boží, Ofni a Fínes.
\par 5 Když pak prinesena byla truhla smlouvy Hospodinovy do vojska, zkrikl všecken Izrael s velikým plésáním, až zeme vznela.
\par 6 Uslyševše pak Filistinští hluk plésání, rekli: Jaký jest to hlas výskání velikého tohoto v vojšte Hebrejském? I poznali, že truhla Hospodinova prišla do vojska.
\par 7 Protož báli se Filistinští, když praveno bylo: Prišel Buh do vojska jejich; a rekli: Beda nám, nebo nebylo prvé nic k tomu podobného.
\par 8 Beda nám! Kdo nás vysvobodí z ruky tech bohu silných? Tit jsou bohové, kteríž zbili Egypt všelikou ranou i na poušti.
\par 9 Posilnte se a budte muži, ó Filistinští, abyste nesloužili tem Hebrejským, jako oni sloužili vám; budtež tedy muži a bojujte.
\par 10 Bojovali tedy Filistinští, a poražen jest Izrael, a utíkali jeden každý do stanu svého. I byla ta porážka veliká velmi, nebo padlo z Izraele tridceti tisíc peších;
\par 11 Také truhla Boží vzata, a dva synové Elí zabiti, Ofni a Fínes.
\par 12 Beže pak jeden Beniaminský z bitvy, prišel do Sílo téhož dne, maje roucho roztržené a hlavu prstí posypanou.
\par 13 Prišel tedy, a aj, Elí sedel na stolici pri ceste, hlede, nebo srdce jeho lekalo se za truhlu Boží; a když všel muž ten, a oznámil v meste, kvílilo všecko mesto.
\par 14 A uslyšev Elí hlas kriku toho, rekl: Jaký jest to hlas hrmotu tohoto? Muž pak ten pospíchaje, pribehl, aby oznámil Elí.
\par 15 (A byl Elí v devadesáti a osmi letech; oci také jeho byly již pošly, a nemohl videti.)
\par 16 I rekl muž ten k Elí: Já jdu z bitvy, z bitvy zajisté utekl jsem dnes. I dí jemu on: Co se tam stalo, muj synu?
\par 17 Odpovedel ten posel a rekl: Utekl Izrael pred Filistinskými, také i porážka veliká stala se v lidu, ano i oba synové tvoji zabiti jsou, Ofni a Fínes, a truhla Boží jest vzata.
\par 18 I stalo se, že jakž jmenoval truhlu Boží, spadl Elí s stolice nazpet pri úhlu brány, a tak zlomiv šíji, umrel; nebo byl muž starý a težký. On soudil Izraele ctyridceti let.
\par 19 Nevesta pak jeho, manželka Fínesova, jsuc tehotná a blízká porodu, uslyševši také tu povest, že by truhla Boží vzata byla, a že umrel tchán její i muž její, sklonila se a porodila; nebo se oborily na ni úzkosti její.
\par 20 A v ten cas, když ona umírala, rekly, kteréž stály pri ní: Neboj se, však jsi porodila syna. Kteráž nic neodpovedela, aniž toho pripustila k srdci svému.
\par 21 I nazvala díte Ichabod, rkuci: Prestehovala se sláva z Izraele; proto že vzata byla truhla Boží, a umrel tchán i muž její.
\par 22 Protož rekla: Prestehovala se sláva z Izraele, nebo vzata jest truhla Boží.

\chapter{5}

\par 1 Filistinští pak vzali truhlu Boží, a zanesli ji z Eben-Ezer do Azotu.
\par 2 Vzali tedy Filistinští truhlu Boží, a vnesli ji do domu modly Dágon, a postavili ji vedlé Dágona.
\par 3 A když Azotští na zejtrí ráno vstali, aj, Dágon povalený ležel tvárí svou na zemi pred truhlou Hospodinovou. I vzali Dágona a postavili jej na místo jeho.
\par 4 A když opet ráno nazejtrí vstali, aj, Dágon, jakž upadl, ležel tvárí svou na zemi pred truhlou Hospodinovou; hlava pak Dágonova a obe dlane rukou jeho odražené byly na prahu, jen ho špalek zustal.
\par 5 Protož kneží Dágonovi a všickni, kteríž vcházejí do domu Dágonova, nešlapají na prah Dágonuv v Azotu až do tohoto dne.
\par 6 Tedy ztížena jest ruka Hospodinova nad Azotskými, a pohubila je, (nebo ranila je neduhy na zadku), Azot i konciny jeho.
\par 7 Vidouce pak muži Azotští, co se deje, rekli: Necht nezustává truhla Boha Izraelského u nás, nebo težká jest ruka jeho proti nám, a proti Dágonovi bohu našemu.
\par 8 I obeslali a shromáždili všecka knížata Filistinská k sobe, a rekli: Co uciníme s truhlou Boha Izraelského? Odpovedeli: Necht jest doprovozena do Gát truhla Boha Izraelského. I zprovodili tam truhlu Boha Izraelského.
\par 9 Stalo se pak, když ji vyprovázeli, že ruka Hospodinova prikrocila k mestu trápením velikým velmi, a ranila muže mesta od malého až do velikého, a nametalo se jim v tajných místech plno vredu.
\par 10 Z té príciny poslali truhlu Boží do Akaron. A když prišla truhla Boží do Akaron, zkrikli Akaronští, rkouce: Zprovodili k nám truhlu Boha Izraelského, aby nás pomorili i náš lid.
\par 11 Protož poslavše, shromáždili všecka knížata Filistinská, a rekli: Odešlete truhlu Boha Izraelského, at se navrátí na místo své, a nezmorí nás i lidu našeho. Nebo bylo trápení smrtelné po všem tom meste, a velmi težká byla ruka Boží tam.
\par 12 Muži pak, kteríž nezemreli, raneni byli neduhy na zadku, tak že krik mesta vstupoval až k nebi.

\chapter{6}

\par 1 Byla pak truhla Hospodinova v krajine Filistinské za sedm mesícu.
\par 2 Tedy Filistinští povolavše kneží a hadacu, rekli: Co uciníme s truhlou Hospodinovou? Oznamte nám, kterak bychom ji odeslali na místo její?
\par 3 Kteríž odpovedeli: Jestliže odešlete truhlu Boha Izraelského, neodsílejte jí prázdné, ale všelijak dejte jí obet za provinení; tehdy uzdraveni budete, a poznáte, proc ruka jeho neodešla od vás.
\par 4 I rekli: Jakáž jest to obet za hrích, kterouž jí dáti máme? Odpovedeli: Vedlé poctu knížat Filistinských pet zadku zlatých a pet myší zlatých; nebo rána jednostejná jest na všechnech, i na knížatech vašich.
\par 5 Udeláte tedy podobenství zadku svých a podobenství myší svých, kteréž pokazily zemi, i dáte slávu Bohu Izraelskému; snad pozlehcí ruky své nad vámi, i nad bohy vašimi a nad zemí vaší.
\par 6 A proc obtežujete srdce svá, jako jsou obtížili Egyptští a Farao srdce své? Zdaliž, když divné veci prokázal pri nich, nepropustili jich, a oni odešli?
\par 7 A protož udelejte vuz nový jeden a vezmete dve krávy otelené, na kteréž jha nebylo vzkládáno, a zapráhnete ty krávy do vozu, a zavrete telata jejich doma, aby za nimi nešla.
\par 8 A vezmouce truhlu Hospodinovu, vstavte ji na vuz; nádoby pak zlaté, kteréž jste dali jí za provinení, položte v škrince po boku jejím, a propustte ji, at odejde.
\par 9 A šetrte: Jestližet cestou k cíli svému pujde do Betsemes, ont jest nám zpusobil všecko toto zlé prenáramné; pakli nic, poznáme, že ne ruka jeho dotkla se nás, ale náhodou nám to prišlo.
\par 10 Ucinili tedy ti muži tak, a vzavše dve krávy otelené, zapráhli je do toho vozu, a telata jejich zavreli doma.
\par 11 A vstavili truhlu Hospodinovu na vuz, i škrinku i myši zlaté, a podobizny zadku svých.
\par 12 I šly uprímo krávy cestou, kteráž vede do Betsemes, silnicí jednou predce jdouce a ricíce, aniž se uchýlily na pravo aneb na levo. Knížata také Filistinská šla za nimi až ku pomezí Betsemes.
\par 13 Betsemští pak žali tehdáž pšenici v údolí, a pozdvihše ocí svých, uzreli truhlu; i veselili se, vidouce ji.
\par 14 A když prijel vuz na pole Jozue Betsemského, tu se zastavil. I byl tu kámen veliký. Tedy zsekavše dríví vozu toho i ty krávy, obetovali je v obet zápalnou Hospodinu.
\par 15 Levítové pak složili truhlu Hospodinovu i škrinku, kteráž byla pri ní, v níž byly nádoby zlaté, a postavili na ten kámen veliký. Muži pak Betsemští dodávali obetí zápalných, a obetovali obeti v ten den Hospodinu.
\par 16 Což videvše knížata Filistinská, navrátili se do Akaron téhož dne.
\par 17 Tito jsou pak zadkové zlatí, kteréž dali Filistinští za své provinení Hospodinu, za Azot jeden, za Gázu jeden, za Aškalon jeden, za Gát jeden, za Akaron jeden.
\par 18 Myši také zlaté, vedlé poctu všech mest Filistinských, za patero knížetství, od mesta hrazeného až do vsi nehrazené, a až k kameni tomu velikému, na nemž postavili truhlu Hospodinovu, kterýž jest až do tohoto dne na poli Jozue Betsemského.
\par 19 Pobil pak Hospodin z mužu Betsemských, kteríž hledeli do truhly Hospodinovy, pobil, pravím, z lidu padesát tisícu a sedmdesáte mužu. I kvílil lid, proto že ucinil Hospodin v lidu porážku velikou.
\par 20 Protož rekli muži Betsemští: Kdož bude moci ostáti pred Hospodinem Bohem svatým tímto? A k komu odejde od nás?
\par 21 I poslali posly k obyvatelum Kariatjeharim, rkouce: Vrátili zase Filistinští truhlu Hospodinovu; pridte, vezmete ji k sobe.

\chapter{7}

\par 1 Tedy prišli muži Kariatjeharim a vzali odtud truhlu Hospodinovu, a vnesli ji do domu Abinadabova na pahrbek, a Eleazara syna jeho posvetili, aby ostríhal truhly Hospodinovy.
\par 2 Stalo se pak, že od toho casu, jakž zustala truhla v Kariatjeharim, když bylo prebehlo mnoho dní, a byl již rok dvadcátý, teprv se roztoužil všecken dum Izraelský po Hospodinu.
\par 3 Nebo byl mluvil Samuel ke všemu domu Izraelskému, rka: Jestliže celým srdcem svým obracíte se k Hospodinu, odejmete bohy cizí z prostredku sebe i Astarota, a ustavte srdce své pri Hospodinu, a služte jemu samému, a vysvobodí vás z ruky Filistinských.
\par 4 A tak zavrhli synové Izraelští Bálim i Astarota, a sloužili Hospodinu samému.
\par 5 Byl také rekl Samuel: Shromaždte všecken lid Izraelský do Masfa, a modliti se budu za vás Hospodinu.
\par 6 I shromáždili se do Masfa, a vážíce vodu, vylévali pred Hospodinem, a postíce se v ten den, rekli tam: Zhrešili jsme proti Hospodinu. A soudil Samuel syny Izraelské v Masfa.
\par 7 Uslyšavše pak Filistinští, že synové Izraelští shromáždili se do Masfa, vytáhla knížata Filistinská proti Izraelovi. Což když uslyšeli synové Izraelští, báli se príchodu Filistinských.
\par 8 Procež rekli synové Izraelští Samuelovi: Neprestávej volati za nás k Hospodinu Bohu našemu, aby nás vysvobodil z ruky Filistinských.
\par 9 Vzav tedy Samuel beránka jednoho, kterýž ješte ssal, obetoval ho celého v obet zápalnou Hospodinu. I modlil se Samuel Hospodinu za Izraele, a uslyšel jej Hospodin.
\par 10 I bylo, že když Samuel obetoval obet zápalnou, Filistinští priblížili se, aby bojovali proti Izraelovi, ale zahrmel Hospodin hrímáním náramným v ten den nad Filistinskými, a potrel je, i poraženi jsou pred tvárí Izraele.
\par 11 Vytáhše pak muži Izraelští z Masfa, stihali Filistinské, a bili je až pod Betchar.
\par 12 Tehdy vzav Samuel kámen jeden, položil jej mezi Masfa a mezi Sen, a nazval jméno jeho Eben-Ezer; nebo rekl: Až potud pomáhal nám Hospodin.
\par 13 A tak sníženi jsou Filistinští, a netáhli více na pomezí Izraelské; a byla ruka Hospodinova proti Filistinským po všecky dny Samuelovy.
\par 14 Navrácena jsou také mesta Izraelovi, kteráž byli Filistinští odtrhli od Izraele, pocna od Akaron až do Gát; i pomezí jejich vysvobodil Izrael z ruky Filistinských. A byl pokoj mezi Izraelem a Amorejskými.
\par 15 I soudil Samuel Izraele po všecky dny života svého.
\par 16 A chode každého roku, obcházel Bethel a Galgala i Masfa, a soudil Izraele na všech tech místech.
\par 17 (Potom navracoval se do Ramata, nebo tam byl dum jeho, a tam soudil Izraele.) Tam také vzdelal oltár Hospodinu.

\chapter{8}

\par 1 Když se pak zstaral Samuel, ustanovil syny své za soudce v Izraeli.
\par 2 A bylo jméno syna jeho prvorozeného Joel, a jméno druhého Abia; ti byli soudcové v Bersabé.
\par 3 Nechodili pak synové jeho po cestách jeho, ale uchýlili se po lakomství, a berouce dary, prevraceli soud.
\par 4 Shromáždili se tedy všickni starší Izraelští, a prišli k Samuelovi do Ramata.
\par 5 A rekli jemu: Aj, tys se již zstaral, a synové tvoji nechodí po cestách tvých; protož nyní ustanov nám krále, aby soudil nás, jakož jest u všech národu.
\par 6 I nelíbila se ta rec Samuelovi, že rekli: Dej nám krále, aby nás soudil. Protož modlil se Samuel Hospodinu.
\par 7 Tedy rekl Hospodin Samuelovi: Uposlechni hlasu lidu ve všem, což mluví tobe; nebo ne tebout jsou pohrdli, ale mnou pohrdli, abych nekraloval nad nimi.
\par 8 Podlé všech skutku tech, kteréž cinili od onoho dne, v nemž jsem je vyvedl z Egypta až do tohoto dne, kdyžto opustili mne, a sloužili bohum cizím, takt oni ciní i tobe.
\par 9 Protož nyní uposlechni hlasu jejich, a však nejprv osvedc pilne pred nimi, a oznam jim obycej krále, kterýž nad nimi kralovati bude.
\par 10 I mluvil Samuel všecky reci Hospodinovy k lidu, kteríž krále žádali od neho.
\par 11 A rekl: Tento bude obycej krále, kterýž kralovati bude nad vámi: Bráti bude syny vaše, a dá je k vozum svým, a zdelá sobe z nich jezdce, a behati budou pred vozem jeho.
\par 12 Také ustanoví je sobe za hejtmany nad tisíci a za padesátníky, a aby jemu orali rolí jeho, a žali obilé jeho, též aby jemu delali nástroje válecné a prípravy k vozum jeho.
\par 13 Dcery také vaše bráti bude, aby delaly masti, a byly kucharky a pekarky.
\par 14 Nadto pole vaše a vinice vaše, i olivoví vaše nejvýbornejší pobére a rozdá služebníkum svým.
\par 15 Také z toho, což vsejete, a z vinic vašich desátky bráti bude, a dá komorníkum a služebníkum svým.
\par 16 Též služebníky vaše a devky vaše, a mládence vaše nejzpusobnejší, i osly vaše vezme, aby jimi delal dílo své.
\par 17 Z stád vašich desátky bráti bude, a vy budete jemu za služebníky.
\par 18 I budete volati v ten den prícinou krále vašeho, kteréhož byste sobe vyvolili, a nevyslyší vás Hospodin dne toho.
\par 19 Nechtel však lid uposlechnouti reci Samuelovy, a rekli: Nikoli, ale král bude nad námi,
\par 20 Abychom i my také byli, jako všickni jiní národové, a souditi nás bude král náš, a vycházeje pred námi, bojovati bude za nás.
\par 21 Vyslyšev tedy Samuel všecka slova lidu, oznámil je Hospodinu.
\par 22 I rekl Hospodin Samuelovi: Uposlechni hlasu jejich, a ustanov jim krále. Protož rekl Samuel mužum Izraelským: Jdetež jeden každý do mesta svého.

\chapter{9}

\par 1 Byl pak muž z pokolení Beniamin, jménem Cis, syn Abiele syna Seror, syna Bechorat, syna Afia, syna muže Jemini, muž udatný.
\par 2 Ten mel syna jménem Saule, mládence krásného, tak že žádného z synu Izraelských nebylo peknejšího nad nej; od ramene svého vzhuru prevyšoval všecken lid.
\par 3 Byly se pak ztratily oslice Cis, otce Saulova. I rekl Cis Saulovi synu svému: Vezmi i hned s sebou jednoho z služebníku, a vstana, jdi, hledej oslic.
\par 4 A tak šel pres horu Efraim, a prošel až do zeme Salisa, a nic nenalezli; prošli také zemi Sálim, a nic nenalezli. Ješte prešli i zemi Jemini, a nenalezli.
\par 5 Když pak prišli do zeme Zuf, rekl Saul služebníku svému, kterýž byl s ním: Pod, a navratme se, aby otec muj nechaje tech oslic, nemel starosti o nás.
\par 6 Kterýž rekl jemu: Aj, nyní muž Boží jest v meste tomto, a muž ten jest znamenitý; cožkoli praví, všecko se tak stává. Medle, jdeme tam, zdali by oznámil nám cestu naši, kterouž bychom jíti meli.
\par 7 Odpovedel Saul služebníku svému: Aj, pujdeme. Co pak prineseme muži tomu? Nebo jsme chléb vytrávili z pytlíku našich, ani daru nemáme, kterýž bychom prinesli muži Božímu. Což máme?
\par 8 I doložil služebník, odpovídaje Saulovi, a rekl: Hle, nalezl jsem u sebe ctvrt lotu stríbra; to dám muži Božímu, aby nám oznámil cestu naši.
\par 9 (Za starodávna v Izraeli tak ríkával každý, kdož jíti mel raditi se s Bohem: Podte, a pujdeme až k vidoucímu. Nebo ten, kterýž nyní slove prorok, za starodávna sloul vidoucí.)
\par 10 Rekl tedy Saul služebníku svému: Dobrá jest rec tvá, nu, jdemež. I šli do mesta, v kterémž byl muž Boží.
\par 11 A když vcházeli na horu mesta, potkali se s deveckami, vycházejícími vážit vody. I rekli jim: Jest-li zde vidoucí?
\par 12 Kteréž odpovedely jim a rekly: Jest. Hle, a on pred tebou, pospeš tedy; nebo dnes prišel do mesta, proto že obetuje dnes lid na té hore.
\par 13 Hned jakž vejdete do mesta, tedy naleznete ho, prvé než vstoupí na horu k jídlu. Lid zajisté nebude jísti, dokudž on neprijde, nebo on požehná obetí, a potom pozvaní jísti budou. A protož jdete, nebo v tuto chvíli naleznete ho.
\par 14 I šli do mesta. A když vcházeli do prostred mesta, aj, Samuel vycházel proti nim, aby vstoupil na tu horu.
\par 15 Hospodin pak zjevil Samuelovi o jeden den prvé, nežli prišel Saul, rka:
\par 16 V tuto chvíli zítra pošli k tobe muže z zeme Beniamin, kteréhož pomažeš, aby byl vudce lidu mého Izraelského, a vysvobodít lid muj z ruky Filistinských; vzhlédlt jsem zajisté na lid svuj, nebo prišlo volání jeho ke mne.
\par 17 A když Samuel uzrel Saule, rekl jemu Hospodin: Aj, ted ten muž, o kterémž pravil jsem tobe: tent spravovati bude lid muj.
\par 18 Tedy pristoupil Saul k Samuelovi v bráne, a rekl: Ukaž mi, prosím, kde jest tuto dum vidoucího?
\par 19 I odpovedel Samuel Saulovi: Já jsem vidoucí. Vstupiž prede mnou na horu, a budete dnes jísti se mnou; potom ráno propustím te, a cožkoli jest v srdci tvém, oznámím tobe.
\par 20 O oslice pak, kteréžt se ztratily dnes tretí den, nic se nestarej, nebo nalezeny jsou. Ale na koho žádost všeho Izraele? Zdali ne na tebe a na všecken dum otce tvého?
\par 21 I odpovedel Saul a rekl: Zdaliž já nejsem syn Jemini, z nejmenšího pokolení Izraelského? Ano i celed má jest nejchaternejší ze všech celedí pokolení Beniaminova. Procež jsi tedy ke mne mluvil taková slova?
\par 22 A pojav Samuel Saule i služebníka jeho, uvedl je do pokoje, a dal jim místo nejprednejší mezi pozvanými, jichž bylo okolo tridcíti mužu.
\par 23 I rekl Samuel kuchari: Dej sem ten díl, kterýž jsem dal tobe, o kterémžt jsem rekl: Schovej jej obzvláštne.
\par 24 Když tedy prinesl kuchar plece i s tím, což se ho prídrželo, položil Samuel pred Saule, a rekl: Ted, což pozustalo, vezmi sobe, jez; až k této chvíli zajisté zachováno jest to pro tebe, jakž jsem rekl: Lidu jsem pozval. I jedl Saul s Samuelem v ten den.
\par 25 A když sešli s hory do mesta, mluvil s Saulem na vrchní podlaze.
\par 26 Potom vstali velmi ráno. I stalo se, když záre vzcházela, že zavolal Samuel Saule na huru, rka: Vstan, a propustím te. Vstal tedy Saul, a vyšli oba ven, on i Samuel.
\par 27 A když pricházeli na konec mesta, rekl Samuel Saulovi: Rci služebníku, at jde napred, (i šel); ty pak pozastav se málo, ažt oznámím rec Boží.

\chapter{10}

\par 1 Tedy vzal Samuel nádobku oleje, a vylil na hlavu jeho, a políbil ho, i rekl: Aj, ted pomazal te Hospodin nad dedictvím svým za vudce.
\par 2 Když pujdeš dnes ode mne, nalezneš dva muže u hrobu Ráchel v koncinách Beniamin v Zelzachu. Kteríž reknou tobe: Nalezeny jsou oslice, jichž jsi chodil hledati. A aj, otec tvuj nechav péce o oslice, stará se o vás, prave: Kterak udelám o syna svého?
\par 3 A odejda odtud dále, prijdeš až k rovine Tábor. I potkají se s tebou tam tri muži vstupující k Bohu do Bethel, jeden nesa tri kozelce, druhý nesa tri pecníky chleba, a tretí nesa láhvici vína.
\par 4 Kterížto když te pozdraví, dadít dva chleby, kteréž prijmeš z rukou jejich.
\par 5 Potom prijdeš na pahrbek Boží, na kterémž jest stráž Filistinská. A když tam vejdeš do mesta, potká se s tebou zástup proroku sstupujících s hory, a pred nimi loutna, buben i píštalka a harfa, a oni prorokovati budou.
\par 6 I sstoupí na te Duch Hospodinuv, a prorokovati budeš s nimi, a promenen budeš v muže jiného.
\par 7 Když tedy zbehnou se tato znamení pri tobe, ucin, cožkoli najde ruka tvá, nebo Buh s tebou jest.
\par 8 Potom sstoupíš prede mnou do Galgala, a aj, já sstoupím k tobe, abych obetoval obeti zápalné, též abych obetoval obeti pokojné. Za sedm dní cekati budeš, až prijdu k tobe, a ukážit, co bys mel ciniti.
\par 9 A bylo, když se obrátil, aby šel od Samuele, že Buh promenil srdce jeho v jiné, a zbehla se všecka ta znamení dne toho.
\par 10 I prišli tam ku pahrbku, a aj, zástup proroku proti nemu, i sstoupil na nej Duch Boží, a prorokoval u prostred nich.
\par 11 Stalo se tedy, že všickni, kdož ho znali prvé, videli, an prorokuje s proroky. Procež mluvil každý, jeden k druhému: Což se to stalo synu Cis? Zdali také Saul mezi proroky?
\par 12 I odpovedel jeden odtud a rekl: A kdo jest otec jejich? Protož prišlo to v prísloví: Zdali také Saul mezi proroky?
\par 13 A prestav prorokovati, prišel na horu.
\par 14 Rekl pak strýc Sauluv jemu a k služebníku jeho: Kam jste chodili? Odpovedel: Hledati oslic; když jsme pak poznali, že jich není, prišli jsme k Samuelovi.
\par 15 I rekl strýc Sauluv: Povez mi medle, co vám rekl Samuel?
\par 16 Odpovedel Saul strýci svému: Oznámil nám místne, že by nalezeny byly oslice. Ale s strany království nic mu neoznámil, co mluvil Samuel.
\par 17 Svolal pak Samuel lid k Hospodinu do Masfa,
\par 18 A rekl synum Izraelským: Takto praví Hospodin Buh Izraelský: Já jsem vyvedl Izraele z Egypta, a vysvobodil jsem vás z ruky Egyptských, nýbrž z ruky všech království ssužujících vás.
\par 19 Ale vy dnes zavrhli jste Boha svého, kterýž sám vyproštuje vás ze všech zlých vecí vašich a z úzkostí vašich, a rekli jste jemu: Krále ustanov nad námi. Protož nyní postavtež se pred Hospodinem po pokoleních svých a po tisících svých.
\par 20 A když Samuel kázal pristupovati všechnem pokolením Izraelským, prišlo na pokolení Beniamin.
\par 21 Potom kázal pristupovati pokolení Beniamin po celedech jich, i prišlo na celed Matri, a prišlo na Saule, syna Cis. I hledali ho, a není nalezen.
\par 22 Protož ptali se opet Hospodina: Prijde-li ješte sem ten muž? I odpovedel Hospodin: Aj, skryl se mezi nádobím.
\par 23 Tedy beželi a vzali ho odtud. I postavil se u prostred lidu, a prevyšoval všecken lid od ramene svého vzhuru.
\par 24 I rekl Samuel všemu lidu: Vidíte-li, koho vyvolil Hospodin, žet mu není podobného ve všem lidu? Protož zkrikl všecken lid, a rekli: Živ bud král!
\par 25 I oznamoval Samuel lidu o správe království, a vepsal to do knihy, kterouž položil pred Hospodinem. Potom propustil Samuel všecken lid, jednoho každého do domu jeho.
\par 26 Také i Saul odšel do domu svého do Gabaa; a odešla s ním vojska, jichžto srdcí Buh se dotekl.
\par 27 Ale lidé nešlechetní rekli: Tento-liž nás vysvobodí? I pohrdali jím, ani mu pocty neprinesli. On pak cinil se neslyše.

\chapter{11}

\par 1 Tedy pritáhl Náhas Ammonitský, a položil se proti Jábes v Gálad. I rekli všickni muži Jábes k Náhasovi: Ucin s námi smlouvu, a budemet sloužiti.
\par 2 Jimž odpovedel Náhas Ammonitský: Na ten zpusob uciním s vámi smlouvu, jestliže každému z vás vyloupím oko pravé, a uvedu to pohanení na všecken Izrael.
\par 3 I rekli jemu starší Jábes: Propujc nám sedm dní, at rozešleme posly po všech koncinách Izraelských, a nebude-li žádného, kdo by nás vysvobodil, tedy vyjdeme k tobe.
\par 4 Když pak prišli ti poslové do Gabaa Saulova, a mluvili ta slova v uši lidu, pozdvihli všickni hlasu svého a plakali.
\par 5 A aj, Saul šel za voly s pole. I rekl Saul: Co je lidu, že pláce? I vypravovali jemu slova mužu Jábes.
\par 6 Tedy sstoupil Duch Boží na Saule, když uslyšel slova ta, a rozhnevala se prchlivost jeho náramne.
\par 7 A vzav pár volu, rozsekal je na kusy, a rozeslal po všech krajinách Izraelských po týchž poslích, rka: Kdo by koli nešel za Saulem a za Samuelem, tak se stane s voly jeho. I pripadl strach Hospodinuv na lid, a vyšli jako muž jeden.
\par 8 I sectl je v Bezeku; a bylo synu Izraelských trikrát sto tisíc, a mužu Juda tridceti tisícu.
\par 9 I rekli tem poslum, kteríž byli prišli: Takto povíte mužum Jábes v Galád: Zítra budete vysvobozeni, když slunce obejde. I prišli poslové, a oznámili mužum Jábes, kterížto zradovali se.
\par 10 Tedy rekli muži Jábes Ammonitským: Ráno vyjdeme k vám, abyste nám ucinili, cožkoli se vám za dobré videti bude.
\par 11 Nazejtrí pak rozdelil Saul lid na tri houfy; i vskocili do prostred vojska v cas jitrního bdení, a bili Ammonitské, až slunce dobre vzešlo; kteríž pak byli pozustali, rozprchli se, tak že nezustalo z nich ani dvou pospolu.
\par 12 I rekl lid Samuelovi: Kdo jest ten, jenž pravil: Zdali Saul kralovati bude nad námi? Vydejte ty muže, at je zbijeme.
\par 13 Ale Saul rekl: Nebudet dnes žádný zabit, ponevadž dnes ucinil Hospodin vysvobození v Izraeli.
\par 14 I rekl Samuel lidu: Pridte, a podme do Galgala, obnovíme tam království.
\par 15 Šel tedy všecken lid do Galgala, a ustanovili tam Saule za krále pred Hospodinem v Galgala, a obetovali také tam obeti pokojné pred Hospodinem. I veselil se tam Saul a všickni muži Izraelští velmi velice.

\chapter{12}

\par 1 Mluvil pak Samuel ke všemu Izraelovi: Aj, uposlechl jsem hlasu vašeho ve všem, což jste mluvili ke mne, a ustanovil jsem nad vámi krále.
\par 2 A nyní aj, král jde pred vámi, já pak zstaral jsem se a ošedivel, a synové moji, aj, mezi vámi jsou. Já také chodil jsem pred vámi od své mladosti až do dnešního dne.
\par 3 A však aj, ted jsem. Vydejte svedectví proti mne pred Hospodinem a pred pomazaným jeho, vzal-li jsem cího vola, aneb vzal-li jsem cího osla, utiskl-li jsem koho, ublížil-li jsem komu, aneb vzal-li jsem od koho úplatek, abych na nem prehlédl neco, a navrátím vám.
\par 4 Odpovedeli: Neutiskl jsi nás, aniž jsi ublížil nám, aniž jsi co vzal z ruky kterého cloveka.
\par 5 Rekl ješte jim: Svedek jest Hospodin proti vám, svedek jest i pomazaný jeho v tento den, že jste nenalezli pri mne nicehož. I rekl lid: Svedek jest.
\par 6 Tedy rekl Samuel lidu: Hospodin jest svedek, kterýž ucinil Mojžíše a Arona, a kterýž vyvedl otce vaše z zeme Egyptské.
\par 7 Protož nyní postavte se, at s vámi v soud vejdu pred Hospodinem o všecky spravedlivé skutky Hospodinovy, kteréž ucinil s vámi i s otci vašimi.
\par 8 Když byl všel Jákob do Egypta, volali otcové vaši k Hospodinu; i poslal Hospodin Mojžíše a Arona, kteríž vyvedli otce vaše z Egypta, a osadili je na tomto míste.
\par 9 A když se zapomenuli na Hospodina Boha svého, vydal je v ruku Zizary, knížete vojska Azor, a v ruku Filistinských, též v ruku krále Moábského, kteríž bojovali proti nim.
\par 10 Ale když volali k Hospodinu a rekli: Zhrešili jsme, nebo jsme opustili Hospodina, a sloužili jsme Bálim a Astarot, protož nyní vysvobod nás z ruky neprátel našich, a sloužiti budeme tobe:
\par 11 I poslal Hospodin Jerobále a Bedana a Jefte a Samuele, a vytrhl vás z ruky neprátel vašich okolních, a bydlili jste bezpecne.
\par 12 Potom vidouce, že Náhas král synu Ammon pritáhl proti vám, rekli jste mi: Nikoli, ale král bude kralovati nad námi; ješto Hospodin Buh váš jest králem vaším.
\par 13 Nyní tedy, aj, král, kteréhož jste zvolili, za nehož jste žádali, a aj, Hospodin ustanovil ho nad vámi králem.
\par 14 Budete-li se báti Hospodina, a jemu sloužiti a poslouchati hlasu jeho, a nebudete-li se zpecovati reci Hospodinove, tak i vy i král, kterýž kraluje nad vámi, ostojíte, jdouce za Hospodinem Bohem vaším.
\par 15 Pakli nebudete poslouchati hlasu Hospodinova, ale odporni budete reci jeho, bude ruka Hospodinova proti vám, jako i proti otcum vašim.
\par 16 Ješte se ted pozastavte, a vizte vec tuto velikou, kterouž uciní Hospodin pred ocima vašima.
\par 17 Zdaliž není dnes žen pšenicná? Volati budu k Hospodinu, i vydá hrímání a déšt, tak že poznati a videti musíte, jak jest to velmi zlá vec, kteréž jste se dopustili pred ocima Hospodinovýma, žádavše sobe krále.
\par 18 Protož volal Samuel k Hospodinu, a vydal Hospodin hrímání a déšt v ten den. I bál se všecken lid Hospodina velmi i Samuele.
\par 19 A rekl všecken lid Samuelovi: Modl se za služebníky své Hospodinu Bohu svému, abychom nezemreli; nebo jsme pridali ke všem hríchum našim i toto zlé, že jsme sobe žádali krále.
\par 20 I rekl Samuel lidu: Nebojte se. Ucinilit jste sic všecko to zlé, však proto neodstupujte zpet od Hospodina, ale služte Hospodinu celým srdcem svým.
\par 21 Neodstupujte, pravím, následujíce marnosti, totiž cizích bohu, kteríž nic neprospívají, aniž vysvobozují, nebo jsou marnost.
\par 22 Neopustít zajisté Hospodin lidu svého pro jméno své veliké; nebo zalíbilo se Hospodinu, aby vás sobe vzdelal v lid.
\par 23 Ode mne také odstup to, abych mel hrešiti proti Hospodinu, a prestávati modliti se za vás, nýbrž navoditi vás budu na cestu dobrou a prímou.
\par 24 A však bojte se Hospodina a služte jemu v pravde celým srdcem svým; nebo vidíte, jak veliké veci ucinil s vámi.
\par 25 Jestliže pak predce zle ciniti budete, i vy i král váš zahynete.

\chapter{13}

\par 1 Saul tedy prvního léta kralování svého, (kraloval pak dve léte nad Izraelem),
\par 2 Vybral sobe tri tisíce z Izraele. I bylo jich s Saulem dva tisíce v Michmas a na hore Bethel, a tisíc bylo s Jonatou v Gabaa Beniaminove; ostatek pak lidu rozpustil jednoho každého do príbytku jeho.
\par 3 I pobil Jonata stráž Filistinských, kterouž meli na pahrbku, a uslyšeli to Filistinští. Tedy Saul troubil v troubu po vší zemi, rka: At to slyší Hebrejští.
\par 4 A tak slyšel všecken Izrael, že bylo praveno: Pobil Saul stráž Filistinských, procež také zoškliven byl Izrael mezi Filistinskými. I svolán jest lid za Saulem do Galgala.
\par 5 Filistinští pak sebrali se k boji proti Izraelovi, tridceti tisíc vozu, a šest tisíc jezdcu, a lidu ve množství, jako jest písku na brehu morském. I vytáhli a položili se v Michmas na východ Betaven.
\par 6 A protož muži Izraelští vidouce, že jim úzko, (nebo byl ssoužen lid), skryl se lid v jeskyních a v ohradách, a v skalách a v horách, i v jamách.
\par 7 Hebrejští také prepravili se pres Jordán do zeme Gád a Galád. Saul pak ješte byl v Galgala, a všecken lid zastrašil se, jda za ním.
\par 8 I ocekával tu za sedm dní vedlé casu uloženého od Samuele; a když nepricházel Samuel do Galgala, rozešel se lid od neho.
\par 9 Tedy rekl Saul: Prineste ke mne obet zápalnou a obeti pokojné. I obetoval obeti zápalné.
\par 10 Když pak již dokonal obetování obeti zápalné, aj, Samuel pricházel, a Saul vyšel proti nemu, aby ho privítal.
\par 11 I rekl Samuel: Co jsi ucinil? Odpovedel Saul: Když jsem videl, že se lid rozchází ode mne, a ty nepricházíš k uloženému dni, a Filistinští byli shromáždeni v Michmas:
\par 12 I rekl jsem: Nyní pripadnou Filistinští na mne v Galgala, a tvári Hospodinove nemodlil jsem se. Takž jsem se opovážil a obetoval jsem obeti zápalné.
\par 13 Tedy rekl Samuel Saulovi: Bláznive jsi ucinil, nezachovals prikázaní Hospodina Boha svého, kteréž prikázal tobe; nebo nyní byl by utvrdil Hospodin království tvé nad Izraelem až na veky.
\par 14 Ale již nyní království tvé neostojí. Vyhledalt jest Hospodin sobe muže vedlé srdce svého, jemuž rozkázal Hospodin, aby byl vudce nad lidem jeho; nebo jsi nezachoval, cožt prikázal Hospodin.
\par 15 Vstav pak Samuel, vstoupil z Galgala do Gabaa Beniaminova. A Saul nacetl lidu, kterýž zustával pri nem, okolo šesti set mužu.
\par 16 Saul tedy a Jonata syn jeho, i lid, kterýž zustával s nimi, byli v Gabaa Beniaminove, Filistinští pak leželi v Michmas.
\par 17 I vyšli zhoubcové z vojska Filistinského na tré rozdelení. Houf jeden obrátil se k ceste Ofra, k zemi Sual;
\par 18 A houf druhý obrátil se na cestu Betoron; houf pak tretí pustil se cestou krajiny, kteráž patrí k údolí Seboim na poušt.
\par 19 Kováre pak žádného nenalézalo se ve vší zemi Izraelské; nebo byli rekli Filistinští: Aby sobe Hebrejští nenadelali mecu a kopí.
\par 20 Protož chodívali všickni Izraelští k Filistinským, aby ostril sobe jeden každý radlici svou, a motyku svou, sekeru svou i vidly své,
\par 21 Sic jinak byly šterbiny na radlicích, motykách, vidlách trírohých a sekerách; také i o zaostrení ostnu bývalo težko.
\par 22 I bylo, že v cas boje nenalézalo se mece ani kopí u žádného z lidu toho, kterýž byl s Saulem a s Jonatou, toliko u Saule a u Jonaty syna jeho.
\par 23 Vyšla pak stráž Filistinských k cestám Michmas.

\chapter{14}

\par 1 Stalo se pak jednoho dne, že rekl Jonata syn Sauluv služebníku, kterýž nosil zbroj jeho: Pod, pujdeme k stráži Filistinských, kteráž jest na oné strane. Ale otci svému toho nepovedel.
\par 2 Saul však zustával na kraji pahrbku pod jabloní zrnatou, kteráž byla v Migron; a lidu, kterýž byl s ním, bylo okolo šesti set mužu,
\par 3 Tolikéž Achiáš syn Achitobuv, bratra Ichabodova, syna Fínesova, syna Elí, kneze Hospodinova v Sílo, kterýž nosil efod. Ale lid nevedel, že by odšel Jonata.
\par 4 Mezi temi pak pruchody, jimiž pokoušel se Jonata prejíti k stráži Filistinských, byla skála príkrá k precházení s této strany, též skála príkrá k precházení s oné strany; jméno jedné Bóses, a jméno druhé Seneh.
\par 5 Jedna skála byla na pulnoci proti Michmas, a druhá na poledne proti Gabaa.
\par 6 I rekl Jonata služebníku, kterýž nosil zbroj jeho: Pod, prejdeme k stráži tech neobrezaných, snad bude Hospodin s námi; nebot není nesnadné Hospodinu zachovati ve mnoze aneb v mále.
\par 7 Odpovedel odenec jeho: Ucin, cožkoli jest v srdci tvém, obrat se, kam chceš; aj, budu s tebou podlé vule tvé.
\par 8 I rekl Jonata: Aj, my jdeme k mužum tem, a ukážeme se jim.
\par 9 Jestliže reknou nám takto: Pockejte, až prijdeme k vám, stujme na míste svém, a nechodme k nim.
\par 10 Pakli by rekli takto: Vstupte k nám, jdeme, nebo vydal je Hospodin v ruku naši. To zajisté nám bude za znamení.
\par 11 Ukázali se tedy oba dva stráži Filistinských. I rekli Filistinští: Hle, Hebrejští lezou z der, v nichž se byli skryli.
\par 12 I mluvili nekterí z stráže té k Jonatovi a k odenci jeho, a rekli: Vstupte k nám, a povíme vám neco. Procež rekl Jonata k odenci svému: Podiž za mnou, nebo je vydal Hospodin v ruku Izraele.
\par 13 A tak lezl ctvermo Jonata a odenec jeho za ním. I padali pred Jonatou, a odenec jeho mordoval je, jda za ním.
\par 14 A to byla porážka první, v níž zbil Jonata a odenec jeho okolo dvadcíti mužu, jako v pul honech rolí dvema volum s záprež.
\par 15 Protož byl strach v tom ležení a na tom poli, i na všem tom lidu; strážní i oni loupežníci desili se též, až se zeme trásla, nebo byla v strachu Božím.
\par 16 A vidouce strážní Saulovi v Gabaa Beniaminove, oznámili, jak množství to naruzno prchá, a vždy více se potírá.
\par 17 Saul pak rekl lidu, kterýž s ním byl: Vyhledejte i hned a zvezte, kdo jest z našich odšel. A když vyhledávali, hle, Jonaty nebylo a odence jeho.
\par 18 I rekl Saul Achiášovi: Postav sem truhlu Boží. (Truhla pak Boží toho casu byla s syny Izraelskými.)
\par 19 I stalo se, když ješte mluvil Saul k knezi, že hrmot, kterýž byl v vojšte Filistinských, více se rozcházel a rozmáhal. Protož rekl Saul knezi: Spust ruku svou.
\par 20 Shromáždili se tedy Saul i všecken lid, kterýž s ním byl, a prišli až k té bitve; a aj, byl mec jednoho proti druhému s hrmotem velmi velikým.
\par 21 Hebrejští pak nekterí byli s Filistinskými prvé, kteríž táhli s nimi polem sem i tam; i ti také obrátili se a stáli pri lidu Izraelském, kterýž byl s Saulem a s Jonatou.
\par 22 Všickni také muži Izraelští, kteríž se skryli na hore Efraim, když uslyšeli, že by utíkali Filistinští, honili je i oni v té bitve.
\par 23 I vysvobodil Hospodin toho dne Izraele. Boj pak protáhl se až do Betaven.
\par 24 A ackoli muži Izraelští utrápili se toho dne, však Saul zavázal lid s prísahou, rka: Zlorecený muž, kterýž by jedl chléb prvé, než bude vecer, a než se pomstím nad neprátely svými. A tak neokusil všecken lid chleba.
\par 25 Všecken pak lid té krajiny šli do lesa, kdež bylo hojnost medu po zemi.
\par 26 A když všel lid do lesa, videl tekoucí med; žádný však nepricinil k ústum svým ruky své, nebo se bál lid té prísahy.
\par 27 Ale Jonata neslyšev, že otec jeho zavazoval lid prísahou, vztáhl hul, kterouž mel v ruce své, a omocil konec její v plástu medu, a obrátil ruku svou k ústum svým; i osvítily se oci jeho.
\par 28 Odpovídaje pak jeden z lidu, rekl: Velikou prísahou zavázal otec tvuj lid, rka: Zlorecený muž, kterýž by jedl chléb dnes, ackoli zemdlel lid.
\par 29 Tedy rekl Jonata: Zkormoutil otec muj lid zeme. Pohledte, prosím, jak se osvítily oci mé, hned jakž jsem okusil malicko medu toho.
\par 30 Cím více kdyby se byl smele najedl dnes lid z loupeží neprátel svých, kterýchž dosáhl? Nebyla-liž by se nyní stala vetší porážka Filistinských?
\par 31 A tak bili toho dne Filistinské od Michmas až do Aialon; i ustal lid náramne.
\par 32 Protož obrátil se lid k loupeži, a nabravše ovcí a volu i telat, zbili je na zemi; i jedl lid se krví.
\par 33 I povedeli Saulovi, rkouce: Aj, lid hreší proti Hospodinu, jeda se krví. Kterýž rekl: Prestoupili jste prikázaní. Privaltež i hned ke mne kámen veliký.
\par 34 Opet rekl Saul: Rozejdete se mezi lid a rcete jim: Privedte ke mne jeden každý vola svého a jeden každý dobytce své, a bíte tuto a jezte, i nebudete hrešiti proti Hospodinu, jedouce se krví. Privedli tedy všecken lid jeden každý vola svého rukou svou té noci a zabíjeli tu.
\par 35 Vzdelal také Saul oltár Hospodinu; to nejprvnejší oltár udelal Hospodinu.
\par 36 Potom rekl Saul: Pustme se po Filistinských v noci, a budeme je loupiti až do jitra, aniž zustavujme z nich koho. Kteríž rekli: Cožt se koli vidí za dobré, ucin. Ale knez rekl: Pristupme sem k Bohu.
\par 37 I tázal se Saul Boha: Pustím-li se za Filistinskými? Dáš-li je v ruku Izraelovu? I neodpovedel mu v ten den.
\par 38 Protož rekl Saul: Pristupujte sem všecka knížata lidu, a vyzvezte a vyhledejte, kdo se jest dopustil dnes hríchu nejakého?
\par 39 Nebo živt jest Hospodin, kterýž vysvobozuje Izraele, že byt pak i na Jonatovi synu mém to bylo, smrtí umre. I neodpovedel jemu žádný ze všeho lidu.
\par 40 Rekl také všemu Izraelovi: Budte vy na jedné strane, já pak a Jonata syn muj budeme na druhé strane. Odpovedel lid Saulovi: Ucin, cožt se za dobré vidí.
\par 41 Protož rekl Saul Hospodinu Bohu Izraelskému: Ukaž spravedlive. I prišlo na Jonatu a Saule, lid pak z toho vyšel.
\par 42 I rekl Saul: Vrzte los mezi mnou a mezi Jonatou synem mým. A postižen jest Jonata.
\par 43 Rekl tedy Saul Jonatovi: Povez mi, co jsi ucinil? I povedel mu Jonata a rekl: Toliko jsem okusil malicko medu koncem holi, kterouž jsem mel v ruce své, a aj, proto-liž mám umríti?
\par 44 Odpovedel Saul: Toto ucin mi Buh a toto pridej, že smrtí umreš, Jonato.
\par 45 I rekl lid Saulovi: Což tedy umríti má Jonata, kterýž ucinil vysvobození toto veliké v Izraeli? Odstup to, živt jest Hospodin, že nespadne vlas s hlavy jeho na zemi, ponevadž s pomocí Boží ucinil to dnes. I vyprostil lid Jonatu, tak aby nebyl usmrcen.
\par 46 Tedy odtáhl Saul od Filistinských; Filistinští také navrátili se k místu svému.
\par 47 Saul pak uvázav se v království nad Izraelem, bojoval vukol se všemi neprátely svými, s Moábskými a s syny Ammon, a s Edomem, i s králi Soba, a s Filistinskými; a kamž se koli obracel, ukrutnost provodil.
\par 48 Sebrav také vojska, porazil Amalecha, a vysvobodil Izraele z ruky zhoubcu jeho.
\par 49 Byli pak synové Saulovi: Jonata a Jesui a Melchisua; a jména dvou dcer jeho, jméno prvorozené Merob, jméno pak mladší Míkol.
\par 50 A jméno manželky Saulovy Achinoam, dcera Achimaasova; jméno pak hejtmana vojska jeho Abner, syn Ner, strýce Saulova.
\par 51 Nebo Cis byl otec Sauluv, a Ner otec Abneruv, syn Abieluv.
\par 52 Byla pak válka veliká s Filistinskými po všecky dny Saulovy, protož kohožkoli Saul videl muže silného, a kohokoli udatného, bral ho k sobe.

\chapter{15}

\par 1 Rekl pak Samuel Saulovi: Hospodin poslal mne, abych te pomazal za krále nad lidem jeho, nad Izraelem, pozorujž tedy nyní hlasu slov Hospodinových.
\par 2 Takto praví Hospodin zástupu: Rozpomenul jsem se na to, co jest cinil Amalech Izraelovi, že se položil proti nemu na ceste, když se bral z Egypta.
\par 3 Protož i hned táhni a zkaz Amalecha, a zahladte jako proklaté všecko, což má. Neslitovávejž se nad ním, ale zahub od muže až do ženy, od malého až do toho, kterýž prsí požívá, od vola také až do ovce, a od velblouda až do osla.
\par 4 Sebral tedy Saul lid, a sectl je v Telaim, dvakrát sto tisíc peších, a deset tisíc mužu Judských.
\par 5 A pritáhl Saul až k mestu Amalechovu, aby bojoval v údolí jeho.
\par 6 I rekl Saul Cinejským: Jdete, oddelte se, vyjdete z prostredku Amalechitských, abych vás s nimi nezahladil; nebo vy jste ucinili milosrdenství se všemi syny Izraelskými, když šli z Egypta. A tak odšel Cinejský z prostredku Amalecha.
\par 7 I porazil Saul Amalecha od Hevilah, kudy se chodí do Sur, kteréž jest naproti Egyptu.
\par 8 Jal také Agaga krále Amalechitského živého, lid pak všecken vyhladil ostrostí mece.
\par 9 I zachoval Saul a lid jeho Agaga, a nejlepší bravy a skoty a krmný dobytek, a berany i všecko, což lepšího bylo, a nechteli vyhubiti jich; což pak bylo nicemného a churavého, to zahubili.
\par 10 A protož stalo se slovo Hospodinovo k Samuelovi, rkoucí:
\par 11 Žel mi, že jsem Saule ustanovil za krále, nebo odvrátil se ode mne, a slov mých nevyplnil. I rozhorlil se Samuel náramne a volal k Hospodinu celou noc.
\par 12 Vstav pak Samuel, šel vstríc Saulovi ráno. I oznámili Samuelovi, rkouce: Saul prišel na Karmel, a aj, pripravil sobe místo, a odtud hnuv se, táhl a sstoupil do Galgala.
\par 13 A když prišel Samuel k Saulovi, rekl jemu Saul: Požehnaný ty od Hospodina, vyplnil jsem slovo Hospodinovo.
\par 14 Samuel pak rekl: Jaké pak jest to becení ovcí tech v uších mých, a rvání volu, kteréž já slyším?
\par 15 Odpovedel Saul: Od Amalechitských prihnali je; nebo zachoval lid, což nejlepšího bylo z bravu a skotu, aby to obetoval Hospodinu Bohu tvému, ostatek pak jsme zahladili jako proklaté.
\par 16 I rekl Samuel Saulovi: Dopust, at oznámím tobe, co jest mi mluvil Hospodin noci této. Dí jemu: Oznam.
\par 17 Tedy rekl Samuel: Zdali jsi nebyl malický sám u sebe? A predce ucinen jsi hlavou pokolení Izraelských, a pomazal te Hospodin za krále nad Izraelem.
\par 18 A poslal te Hospodin na cestu a rekl tobe: Jdi, zahub jako proklaté hríšníky ty Amalechitské, a bojuj proti nim, dokudž byste nevyhladili jich.
\par 19 Procež jsi tedy neuposlechl hlasu Hospodinova, ale obrátil jsi se k loupeži, a ucinils zlou vec pred ocima Hospodinovýma?
\par 20 Odpovedel Saul Samuelovi: Však jsem uposlechl hlasu Hospodinova, a šel jsem cestou, kterouž poslal mne Hospodin, a privedl jsem Agaga krále Amalechitského, i Amalechitské jako proklaté vyhubil jsem.
\par 21 Ale lid vzal z loupeží bravy a skoty prední z vecí proklatých, k obetování Hospodinu Bohu tvému v Galgala.
\par 22 I rekl Samuel: Zdaliž líbost takovou má Hospodin v zápalích a v obetech, jako když se poslušenství koná hlasu Hospodinova? Aj, poslouchati lépe jest, nežli obetovati, a ku poslušenství státi, nežli tuk skopcu prinášeti.
\par 23 Nebo zpoura jest takový hrích jako carodejnictví, a prestoupiti prikázaní jako modlárství a obrazové. Ponevadž jsi pak zavrhl rec Hospodinovu, i on také zavrhl te, abys nebyl králem.
\par 24 Tedy rekl Saul Samuelovi: Zhrešil jsem, že jsem prestoupil rozkaz Hospodinuv a slova tvá, nebo jsem se bál lidu, a povolil jsem hlasu jejich.
\par 25 Protož nyní odpust, prosím, hrích muj, a navrat se se mnou, at se pomodlím Hospodinu.
\par 26 I rekl Samuel Saulovi: Nenavrátím se s tebou; nebo jsi zavrhl rec Hospodinovu, tebe také zavrhl Hospodin, abys nebyl králem nad Izraelem.
\par 27 A když se obrátil Samuel, aby odšel, Saul uchytil krídlo plášte jeho, i odtrhlo se.
\par 28 Tedy rekl jemu Samuel: Odtrhlt jest Hospodin království Izraelské dnes od tebe, a dal je bližnímu tvému, lepšímu, než jsi ty.
\par 29 Však proto vítez Izraelský klamati nebude, ani želeti; nebo není clovekem, aby mel ceho želeti.
\par 30 On pak rekl: Zhrešilt jsem, ale vždy mne cti, prosím, pred staršími lidu mého a pred Izraelem, a navrat se se mnou, abych se pomodlil Hospodinu Bohu tvému.
\par 31 I navrátiv se Samuel, šel za Saulem, a pomodlil se Saul Hospodinu.
\par 32 Rekl pak Samuel: Privedte ke mne Agaga, krále Amalechitského. I šel k nemu Agag nádherne; nebo rekl Agag: Jiste odešla horkost smrti.
\par 33 Ale Samuel rekl: Jakož uvedl sirobu na ženy mec tvuj, takt osirí matka tvá nad jiné ženy. I rozsekal Samuel Agaga na kusy pred Hospodinem v Galgala.
\par 34 Potom odšel Samuel do Ráma, Saul pak vstoupil do domu svého, do Gabaa Saulova.
\par 35 A již potom více Samuel nevidel Saule až do dne smrti své; však plakal Samuel Saule. Hospodin pak želel toho, že ucinil Saule králem nad Izraelem.

\chapter{16}

\par 1 Rekl pak Hospodin Samuelovi: I dokudž ty budeš plakati Saule, ponevadž jsem já ho zavrhl, aby nekraloval nad Izraelem? Napln roh svuj olejem a pod, pošli te k Izai Betlémskému; nebo jsem vybral sobe z synu jeho krále.
\par 2 I rekl jemu Samuel: Kterak mám jíti? Nebo Saul uslyše, zabije mne. Odpovedel Hospodin: Jalovici z stáda vezmeš s sebou, a díš: Prišel jsem, abych obetoval Hospodinu.
\par 3 A pozveš Izai k obeti, ját pak ukáži, co bys mel ciniti; i pomažeš mi toho, o kterémž já tobe povím.
\par 4 A tak ucinil Samuel, jakž mu byl mluvil Hospodin, a prišel do Betléma. A ulekše se starší mesta, vyšli proti nemu a rekli:Pokojný-li jest príchod tvuj?
\par 5 Odpovedel:Pokojný. Abych obetoval Hospodinu, prišel jsem. Posvette se, a podte k obeti se mnou. Posvetil také Izai a synu jeho, a pozval jich k obeti.
\par 6 Když pak prišli, vida Eliába, rekl:Jiste pred Hospodinem jest pomazaný jeho.
\par 7 Ale Hospodin rekl Samuelovi:Nehled na tvárnost jeho a na zrust postavy jeho, nebo jsem ho zavrhl; nebot nepatrím, nac patrí clovek. Clovek zajisté hledí na to, což jest pred ocima, ale Hospodin hledí k srdci.
\par 8 I povolal Izai Abinadaba, a kázal mu jíti pred Samuele. Kterýž rekl:Také ani toho nevyvolil Hospodin.
\par 9 Rozkázal též Izai jíti Sammovi. I rekl:Ani toho nevyvolil Hospodin.
\par 10 Takž rozkázal Izai jíti sedmi synum svým pred Samuele. Ale Samuel rekl Izai:Ani tech nevyvolil Hospodin.
\par 11 Potom rekl Samuel Izai:Jsou-li to již všickni synové? Odpovedel: Ještet zustává nejmladší, a aj, pase ovce. Tedy rekl Samuel Izai:Pošli a vezmi jej; nebo aniž sedneme za stul, dokudž on sem neprijde.
\par 12 I poslal a privedl ho. (Byl pak on ryšavý, krásných ocí a libého vzezrení.) I rekl Hospodin:Vstana, pomaž ho, nebo to jest ten.
\par 13 Protož vzal Samuel roh s olejem, a pomazal ho u prostred bratrí jeho. I odpocinul Duch Hospodinuv na Davidovi od toho dne i potom. A Samuel vstav, odšel do Ráma.
\par 14 Takž Duch Hospodinuv odšel od Saule, a nepokojil ho duch zlý od Hospodina.
\par 15 Služebníci pak Saulovi rekli jemu:Aj, ted duch Boží zlý nepokojí te.
\par 16 Necht, medle, rozkáže pán náš služebníkum svým, kteríž stojí pred tebou, at hledají muže, kterýž by umel hráti na harfu, aby, když by prišel na te duch Boží zlý, hral rukou svou, a tobe aby lehceji bylo.
\par 17 Tedy rekl Saul služebníkum svým:Medle, vyhledejte mi muže, kterýž by umel dobre hráti, a privedte ke mne.
\par 18 I odpovedel jeden z služebníku a rekl:Aj, videl jsem syna Izai Betlémského, kterýž umí hráti, muže udatného a bojovného, též správného a krásného, a jest s ním Hospodin.
\par 19 Protož poslal Saul posly k Izai, rka:Pošli mi Davida syna svého, kterýž jest pri stádu.
\par 20 Tedy Izai vzal osla, chléb a nádobu vína a kozelce jednoho, a poslal po Davidovi synu svém Saulovi.
\par 21 Když pak prišel David k Saulovi, stál pred ním; i zamiloval ho velmi, a ucinen jest jeho odencem.
\par 22 Potom poslal opet Saul k Izai, rka:Medle, necht David stojí prede mnou, nebot jest našel milost pred ocima mýma.
\par 23 I bývalo, že kdyžkoli napadal duch Boží Saule, David, bera harfu, hrával rukou svou; i míval Saul polehcení, a lépe mu bývalo, nebo ten duch zlý odstupoval od neho.

\chapter{17}

\par 1 I sebrali Filistinští vojska svá k boji, a shromáždili se u Socho, kteréž jest Judovo, a položili se mezi Socho a Azeka na pomezí Dammim.
\par 2 Ale Saul a muži Izraelští sebravše se, položili se v údolí Elah, a sšikovali se k bitve proti Filistinským.
\par 3 I stáli Filistinští na hore s strany jedné, a Izraelští stáli na hore s strany druhé, a údolí bylo mezi nimi.
\par 4 I vyšel muž bojovník z vojska Filistinského, Goliáš jménem, z Gát, zvýší šesti loket a dlani.
\par 5 Lebka pak ocelivá na hlave jeho, a v pancír brnený byl oblecen, kterýžto pancír vážil pet tisíc lotu oceli.
\par 6 Též plechovice ocelivé byly na nohách jeho, a pavéza ocelivá mezi rameny jeho.
\par 7 A drevo u kopí jeho jako vratidlo tkadlcovské, železo pak ostré kopí jeho vážilo šest set lotu železa; a ten, kterýž nosil bran jeho, šel pred ním.
\par 8 I postavil se, a volal na vojska Izraelská, rka k nim: Nac jste to vyšli, vojensky se sšikovavše? Zdaliž já nejsem Filistinský, a vy služebníci Saulovi? Vyberte z sebe muže, kterýž by sstoupil ke mne.
\par 9 Jestližet mi bude moci odolati a zabije mne, tedy budeme služebníci vaši; paklit já premohu jej a zabiji ho, budete vy služebníci naši, a sloužiti budete nám.
\par 10 Pravil také ten Filistinský: Já jsem dnes zhanel vojska Izraelská. Vydejtež mi muže, abychom se bili spolu.
\par 11 A když uslyšel Saul i všecken Izrael slova Filistinského taková, ulekli se a báli se velmi.
\par 12 David pak ten syn muže Efratejského z Betléma Judova, (jehož jméno Izai, kterýž mel osm synu, a již se byl sstaral za dnu Saule, priblíživ se k lidem sešlým.
\par 13 Tri také synové Izai starší šedše, táhli na vojnu s Saulem. Jména tech trí synu jeho, kteríž byli šli k boji: Eliáb prvorozený, a druhý po nem Abinadab, tretí pak Samma.
\par 14 Ale David byl nejmladší. A tak ti tri synové starší odešli s Saulem).
\par 15 David tedy odšel od Saule a navrátil se, aby pásl stádo otce svého v Betléme.
\par 16 I pricházel ten Filistinský ráno a vecer, a stavel se po ctyridceti dní.
\par 17 Tedy rekl Izai Davidovi synu svému: Vezmi i hned pro bratrí své efi pražmy této a deset chlebu techto, a bež do vojska k bratrím svým.
\par 18 A deset syrecku mladých techto doneseš hejtmanu, a navštíve bratrí své, pozdravíš jich, a základ jejich vyzdvihneš.
\par 19 Saul pak i oni, i všickni muži Izraelští byli v údolí Elah, bojujíce proti Filistinským.
\par 20 A tak vstav David tím raneji a zanechav stáda pri strážném, vzal to a šel, jakž mu byl prikázal Izai. I prišel až k šancum, a aj, vojsko vycházelo do šiku a kricelo k bitve.
\par 21 I sšikovali se Izraelští, ano i Filistinští, vojsko proti vojsku.
\par 22 Protož David zanechav bremene, kteréž s sebe složil u strážného pri bremeních, bežel do vojska; a když prišel, tázal se bratrí svých, jak se mají.
\par 23 A když on s nimi mluvil, aj, muž bojovník jménem Goliáš, Filistinský z Gát, vycházel z vojska Filistinských a mluvil jako i prvé; což slyšel i David.
\par 24 Všickni pak muži Izraelští, jakž uzreli toho muže, utíkali pred tvárí jeho, a báli se náramne.
\par 25 I mluvili Izraelští mezi sebou: Videli-li jste toho muže, kterýž vyšel? Nebo aby pohanení uvedl na Izraele, vyšel. Kdož by ho zabil, zbohatí ho král bohatstvím velikým, a dceru svou dá jemu, i dum otce jeho osvobodí v Izraeli.
\par 26 I mluvil David mužum, kteríž stáli u neho, a rekl: Co bude dáno muži tomu, kterýž by zabil toho Filistinského a odjal pohanení od Izraele? Nebo kdo jest Filistinský neobrezaný ten, že pohanení uvodí na vojsko Boha živého?
\par 27 Odpovedel jemu lid v táž slova, rka: To bude dáno muži, kterýž by ho zabil.
\par 28 A uslyšev Eliáb, bratr jeho nejstarší, že mluví s temi muži, rozhneval se Eliáb velmi na Davida, a rekl: Proc jsi sem prišel? A komus nechal kolikasi tech ovec na poušti? Známt já pýchu tvou a zlost srdce tvého, žes prišel dívati se bitve.
\par 29 I rekl David: Což jsem pak ucinil? Zdaž mi nebylo poruceno?
\par 30 Tedy šel od neho k jinému, jehož se tázal jako i prvé. I odpovedel jemu lid tak jako i prvé.
\par 31 A tak roznesla se slova, kteráž mluvil David, a oznámili je Saulovi. Kterýžto povolal ho.
\par 32 I rekl David Saulovi: Necht se neleká srdce cloveka pro neho; služebník tvuj pujde a bude se bíti s Filistinským tímto.
\par 33 Ale Saul rekl Davidovi: Nebudeš moci jíti proti Filistinskému tomu, abys se potýkal s ním; nebo mládencek jsi, on pak jest muž bojovný od mladosti své.
\par 34 Odpovedel David Saulovi: Služebník tvuj pastýrem byl stáda otce svého, a když pricházel lev aneb nedved, a bral dobytce z stáda,
\par 35 Já dostihal jsem ho a bil jsem jej, a vydíral jsem je z hrdla jeho. Pakli se na mne oboril, tedy ujma ho za celist, bil jsem jej, až jsem ho i zabil.
\par 36 I lva i nedveda zabil služebník tvuj; budet tedy Filistinský neobrezanec ten jako který z nich, nebo zhanel vojska Boha živého.
\par 37 Rekl také David: Hospodin, kterýž vytrhl mne z moci lva a z moci nedveda, ont mne vytrhne z ruky Filistinského tohoto. Tedy rekl Saul Davidovi: Jdi, a Hospodin budiž s tebou.
\par 38 I dal Saul obléci Davida v šaty své, a vstavil lebku ocelivou na hlavu jeho, a oblékl ho v pancír.
\par 39 Pripásal také David mec jeho na ty šaty jeho, a chtel jíti, ale že tomu nezvykl, protož rekl David Saulovi: Nemohut v tom jíti, nebo jsem neprivykl. I složil to David s sebe.
\par 40 A vzav hul svou do ruky své, vybral sobe pet kamenu hladkých z potoku, a vložil je do mošnicky pastýrské, kterouž mel, totiž do pytlíku, a prak svuj v ruce nesl, a priblížil se k Filistinskému.
\par 41 Bral se také i Filistinský, jda a približuje se k Davidovi, a muž ten, kterýž nesl bran jeho, šel pred ním.
\par 42 A když pohledel Filistinský a uzrel Davida, pohrdal jím, proto že byl mládencek, a ryšavý, a krásného vzezrení.
\par 43 I rekl Filistinský Davidovi: Což jsem pes, že jdeš proti mne s holí? A zlorecil Filistinský Davidovi skrze bohy své.
\par 44 Rekl také Filistinský Davidovi: Pod ke mne, a dámt telo tvé ptákum nebeským a šelmám zemským.
\par 45 Odpovedel David Filistinskému: Ty jdeš ke mne s mecem a s kopím a s pavézou, ale já k tobe jdu ve jménu Hospodina zástupu, Boha vojsk Izraelských, kterémuž jsi ty utrhal.
\par 46 Dnešního dne zavre te Hospodin v ruku mou, a zabiji te, a setnu hlavu tvou s tebe, a vydám tela vojska Filistinského dnes ptákum nebeským a šelmám zemským, a pozná všecka zeme, žet jest Buh v Izraeli.
\par 47 A zvít všecko shromáždení toto, že ne mecem ani kopím vysvobozuje Hospodin; (nebo Hospodinuv jest boj), protož vydát vás v ruce naše.
\par 48 Stalo se pak, že když vstal ten Filistinský, a šel, približuje se proti Davidovi, pospíšil i David a bežel proti Filistinskému, aby se s ním potýkal.
\par 49 V tom David vztáh ruku svou k mošnicce, vynal z ní kámen, kterýmž hodil z praku, a uderil Filistinského v celo jeho tak, že kámen uvázl v cele jeho. I padl tvárí svou na zem.
\par 50 A tak premohl David Filistinského prakem a kamenem, a uderiv Filistinského, zabil jej, ackoli David žádného mece v ruce nemel.
\par 51 A pribeh David, stál nad Filistinským. Potom pochytiv mec jeho, dobyl ho z pošvy a zabil jej, a stal jím hlavu jeho. To vidouce Filistinští, že umrel nejsilnejší jejich, utíkali.
\par 52 A protož vstavše muži Izraelští a Judští, zkrikli a honili Filistinské, až kde se vchází do údolí, a až k branám Akaron. I padali, raneni jsouce, Filistinští po ceste k Saraim, až do Gát a až do Akaron.
\par 53 A navrátivše se synové Izraelští od honení Filistinských, vzebrali tábor jejich.
\par 54 Potom David vzav hlavu toho Filistinského, prinesl ji do Jeruzaléma, a odení jeho složil v stanu svém.
\par 55 Tehdáž pak, když videl Saul Davida jdoucího proti tomu Filistinskému, rekl Abnerovi hejtmanu vojska: Abner, cí jest syn ten mládencek? Odpovedel Abner: Jako jest živa duše tvá, králi, že nevím.
\par 56 Ale král rekl: Zeptej ty se, cí jest syn mládenec ten.
\par 57 Když se pak vracoval David od zabití Filistinského toho, pojav ho Abner, privedl jej pred Saule, an drží hlavu Filistinského v ruce své.
\par 58 I rekl jemu Saul: Cí jsi syn, mládence? Odpovedel David: Syn služebníka tvého Izai Betlémského.

\chapter{18}

\par 1 I stalo se, že když prestal mluviti k Saulovi, duše Jonatova spojila se s duší Davidovou, tak že ho zamiloval Jonata jako sebe samého.
\par 2 A tak Saul vzal jej k sobe toho dne, a nedopustil mu navrátiti se do domu otce jeho.
\par 3 I ucinil Jonata s Davidem smlouvu, proto že ho miloval jako duši svou.
\par 4 A složiv Jonata s sebe plášt, kterýmž byl odín, dal jej Davidovi, i roucho své, až do mece svého, a až do lucište svého, i do pasu svého.
\par 5 Vycházel pak David, k cemuž ho koli posílal Saul, opatrne sobe pocínaje. I ustanovil ho Saul nad vojáky, a líbil se všemu lidu, též i služebníkum Saulovým.
\par 6 Stalo se pak, když se oni domu brali, a David též se navracoval od zabití Filistinského, že vyšly ženy z každého mesta Izraelského, zpívajíce a plésajíce, vstríc Saulovi králi s bubny, s veselím a s huslickami.
\par 7 A prozpevovaly jedny po druhých ženy ty, hrajíce, a rekly: Porazilt jest Saul svuj tisíc, ale David svých deset tisícu.
\par 8 I rozhneval se Saul náramne, nebo nelíbila se mu ta rec. Procež rekl: Dali Davidovi deset tisíc, a mne dali toliko tisíc. Co mu ješte pres to privlastní, lec království?
\par 9 Protož Saul zlobive hledel na Davida od toho dne i vždycky.
\par 10 Stalo se pak druhého dne, že duch Boží zlý napadl Saule, a prorokoval u prostred domu svého, a David hral rukou svou jako i jindy vždycky. Saul pak mel kopí v ruce své.
\par 11 I vyhodil Saul kopí,rka: Prohodím Davida až do steny. Ale David uhnul se jemu po dvakrát.
\par 12 A bál se Saul Davida, proto že Hospodin byl s ním, a od Saule odstoupil.
\par 13 Protož vybyl ho Saul od sebe, a ucinil jej sobe hejtmanem nad tisíci; kterýžto vycházel i vcházel pred lidem.
\par 14 David pak ve všech cestách svých opatrne sobe pocínal, nebo Hospodin byl s ním.
\par 15 A když to videl Saul, že sobe velmi opatrne pocíná, bál se ho.
\par 16 Ale všecken Izrael i Juda miloval Davida, nebo vycházel i vcházel pred nimi.
\par 17 Tedy rekl Saul Davidovi: Aj, dceru svou starší Merob dámt za manželku, toliko mi bud muž silný, a ved boje Hospodinovy. (Saul pak myslil: Necht neschází od mé ruky, ale od ruky Filistinských.)
\par 18 I rekl David Saulovi: Kdo jsem já, a jakýž jest rod muj a celed otce mého v Izraeli, abych byl zetem královým?
\par 19 I stalo se, že když již Merob dcera Saulova mela dána býti Davidovi, dána jest Adrielovi Molatitskému za manželku.
\par 20 Milovala pak Míkol dcera Saulova Davida; což když oznámili Saulovi, líbilo se to jemu.
\par 21 (Nebo rekl Saul: Dámt mu ji, aby mu byla osídlem, a aby proti nemu byla ruka Filistinských.) A tak rekl Saul Davidovi: Po této druhé budeš mi již zetem.
\par 22 I rozkázal Saul služebníkum svým: Mluvte k Davidovi tajne, rkouce: Aj, libuje te sobe král, a všickni služebníci jeho laskavi jsou na tebe; nyní tedy budiž zetem královým.
\par 23 A když mluvili služebníci Saulovi v uši Davidovy slova ta, odpovedel David: Zdaliž se vám malá vec zdá, býti zetem královým? A já jsem clovek chudý a opovržený.
\par 24 Tedy služebníci Saulovi oznámili jemu, rkouce: Taková slova mluvil David.
\par 25 I rekl Saul: Takto rcete Davidovi: Není žádostiv král vena, toliko sto obrízek Filistinských, aby byla pomsta nad neprátely královskými. (Saul pak to obmýšlel, aby David upadl v ruku Filistinských.)
\par 26 Protož oznámili služebníci jeho Davidovi slova ta, a líbilo se to Davidovi, aby byl zetem královým. Ješte se pak nebyli vyplnili dnové ti,
\par 27 Když vstav David, odšel, on i muži jeho, a zbil z Filistinských dve ste mužu, jejichž obrízky prinesl David, a z úplna je dali králi, aby byl zetem královým. I dal jemu Saul Míkol dceru svou za manželku.
\par 28 A vida Saul, nýbrž zkušené maje, že jest Hospodin s Davidem, a že Míkol dcera jeho miluje ho,
\par 29 Ješte tím více Saul obával se Davida. I byl Saul neprítelem Davidovým po všecky dny.
\par 30 Vtrhovali pak do zeme knížata Filistinská; i bývalo, že kdyžkoli vycházeli, opatrneji sobe pocínal David proti nim nade všecky služebníky Saulovy, procež i jméno jeho bylo velmi slavné.

\chapter{19}

\par 1 Mluvil pak Saul k Jonatovi synu svému a ke všechnem služebníkum svým, aby zamordovali Davida, ale Jonata syn Sauluv liboval sobe Davida velmi.
\par 2 I oznámil to Jonata Davidovi, rka: Usiluje Saul otec muj, aby te zabil; protož nyní šetr se, prosím, až do jitra, a usade se v skryte, schovej se.
\par 3 Ale ját vyjdu a státi budu pri boku otci svému na poli, kdež ty budeš, a budu mluviti o tobe s otcem svým, a cemuž vyrozumím, oznámím tobe.
\par 4 I mluvil Jonata o Davidovi dobre Saulovi otci svému, a rekl: Necht nehreší král proti služebníku svému Davidovi, nebt jest nic tobe neprovinil, nýbrž správa jeho jest tobe velmi užitecná.
\par 5 Nebo se opovážil života svého a zabil Filistinského, a ucinil Hospodin vysvobození veliké všemu Izraeli. Videls to a radoval jsi se. Procež bys tedy hrešil proti krvi nevinné, chteje zabiti Davida bez príciny?
\par 6 I uposlechl Saul reci Jonatovy a prisáhl Saul, rka: Živt jest Hospodin, žet nebude zabit.
\par 7 Tedy Jonata povolal Davida, a oznámil jemu Jonata všecka slova tato. A privedl Jonata Davida k Saulovi, i byl pred ním jako i prvé.
\par 8 Vznikla pak opet válka; a vytáh David, bojoval proti Filistinským, a porazil je porážkou velikou, a utekli pred ním.
\par 9 V tom duch Hospodinuv zlý napadl Saule, kterýž v dome svém sedel, maje kopí své v ruce své, a David hrál rukou pred ním.
\par 10 Ale Saul chtel prohoditi Davida kopím až do steny; kterýž uhnul se mu, a uderilo kopí v stenu. A tak David utekl a vynikl z nebezpecenství té noci.
\par 11 Potom poslal Saul posly k domu Davidovu, aby ho stráhli a zabili jej ráno. I oznámila to Davidovi Míkol manželka jeho, rkuci: Jestliže se neopatríš noci této, zítra zabit budeš.
\par 12 A protož spustila Míkol Davida oknem, kterýž odšed, utekl a vynikl z nebezpecenství.
\par 13 Vzala pak Míkol obraz a vložila na luže, a polštár kozí položila v hlavách jeho a prikryla šaty.
\par 14 Tedy Saul poslal posly, aby vzali Davida. I rekla: Jest nemocen.
\par 15 Opet poslal Saul posly, aby pohledeli na Davida, rka: Prineste ho na luži ke mne, at ho zabiji.
\par 16 A když prišli poslové, a aj, obraz ležel na luži a polštár kozí v hlavách jeho.
\par 17 I rekl Saul k Míkol: Proc jsi mne tak podvedla a pustilas neprítele mého, aby ušel? Odpovedela Míkol Saulovi: On mi rekl: Propust mne, sic jinak zabiji te.
\par 18 David tedy utíkaje, vynikl z nebezpecenství, a prišed k Samuelovi do Ráma, oznámil jemu všecko, co mu Saul ucinil. I odšel on i Samuel, a bydlili v Náiot.
\par 19 Oznámeno pak bylo Saulovi, rka: Aj, David jest v Náiot v Ráma.
\par 20 I poslal Saul posly, aby jali Davida. Kteríž když videli zástup proroku prorokujících, a Samuele jim predstaveného, an stojí, sstoupil také na posly Saulovy Duch Boží, a prorokovali i oni.
\par 21 To když oznámili Saulovi, poslal jiné posly, a prorokovali také i oni; a opet poslal Saul tretí posly, a i ti prorokovali.
\par 22 Šel tedy již sám do Ráma a prišel až k cisterni veliké, kteráž jest v Socho, a zeptal se, rka: Kde jest Samuel a David? I rekl nekdo: Aj, jsou v Náiot v Ráma.
\par 23 I šel tam do Náiot v Ráma; a sstoupil i na nej Duch Boží, a jda dále, prorokoval, až i prišel do Náiot v Ráma.
\par 24 Kdežto svlékl také sám roucho své a prorokoval i on pred Samuelem, a padna, ležel svlecený celý ten den a noc. Odkudž se ríká: Zdali také Saul mezi proroky?

\chapter{20}

\par 1 V tom utíkaje David z Náiot, kteréž jest v Ráma, prišel a mluvil pred Jonatou: Což jsem ucinil? Jaká jest nepravost má? A jaký jest hrích muj pred otcem tvým, že hledá bezživotí mého?
\par 2 Kterýž rekl jemu: Odstup to, neumreš. Aj, necinít otec muj niceho ani velikého ani malého, cehož by se mi nesveril. Jak by tedy otec muj tajil to prede mnou! Nenít toho.
\par 3 Nadto prisáhl také David, (to promluviv: Dobret ví otec tvuj, že jsi laskav na mne, protož myslí: Necht neví o tom Jonata, aby nemel zámutku). Nýbrž jiste, živt jest Hospodin, a živat jest duše tvá, že sotva jest krocej mezi mnou a mezi smrtí.
\par 4 Odpovedel Jonata Davidovi: Povez, cehokoli žádáš, a uciním tobe.
\par 5 I rekl David Jonatovi: Aj, zítra bude novmesíce, kdyžto já mám obycej sedati s králem k jídlu; protož propust mne, a skryji se na poli až do tretího vecera.
\par 6 Jestliže by se zvláštne na mne ptal otec tvuj, rekneš: Prosil mne velice David, aby sbehl do Betléma mesta svého; nebo obet výrocní tam míti má všecka jeho rodina.
\par 7 Rekne-lit: Dobre, pokoj služebníku tvému; paklit se rozzlobí, vez, žet se doplnila zlost jeho.
\par 8 A tak uciníš milosrdenství s služebníkem svým, ponevadž jsi v smlouvu Hospodinovu uvedl služebníka svého s sebou. Paklit jest na mne nepravost, zabí mne sám; nebo k otci svému proc bys mne vodil?
\par 9 I rekl Jonata: Odstup to od tebe; nebo zvím-lit to jistotne, že by se doplnila zlost otce mého, aby prišla na te, zdaliž neoznámím tobe toho?
\par 10 Rekl také David Jonatovi: Kdož mi oznámí, jestliže odpoví tobe otec tvuj neco tvrde?
\par 11 Odpovedel Jonata Davidovi: Pod, vyjdeme na pole. I vyšli oba na pole.
\par 12 Opet rekl Jonata Davidovi: Hospodin Buh Izraelský, (hned jakž porozumím na otci svém, okolo tohoto casu zítra neb pozejtrí, an bude dobre s Davidem, jestliže nepošli tehdáž k tobe, a neoznámím-lit),
\par 13 Toto ucin Hospodin Jonatovi a toto pridej. Paklit se bude líbiti otci mému uvésti zlé na tebe, takét i to zjevím tobe a propustím te; i pujdeš v pokoji, a Hospodin budiž s tebou, jakož byl s otcem mým.
\par 14 A zdaliž i ty, dokud jsem živ, neuciníš se mnou milosrdenství Hospodinova, ano bycht i umrel,
\par 15 Tak že neodvrátíš milosrdenství svého od domu mého až na veky, zvlášte tehdáž, když Hospodin vypléní neprátely Davidovy, jednoho každého se svrchku zeme.
\par 16 A tak ucinil Jonata smlouvu s domem Davidovým, rka: Vyhledávejž Hospodin toho z ruky neprátel Davidových.
\par 17 Ješte i prísahou zavázal Jonata v lásce odmenné k sobe Davida; nebo jakož miloval duši svou, tak jej miloval.
\par 18 I rekl mu Jonata: Zítra bude novmesíce, a bude se ptáti na tebe, když prázdné bude místo tvé.
\par 19 Do tretího tedy dne skrývaje se, sstoupíš rychle a prijdeš k tomu místu, na kterémžs se byl skryl, když se to jednalo, a pobudeš u kamene pocestných.
\par 20 A já tri strely vystrelím po strane k nemu, smeruje sobe k cíli.
\par 21 Potom hned pošli pachole a dím: Jdi, shledej strely. Jestliže proste reknu služebníku: Hle, strely za tebou blíže sem, prines je, tedy prid, nebo jest pokoj tobe, a nenít žádného nebezpecenství, živt jest Hospodin.
\par 22 Pakli takto reknu pacholeti: Hle, strely jsou pred tebou dále, tedy odejdi, nebo propustil te Hospodin.
\par 23 Reci pak této, kterouž jsme mluvili já a ty, aj, Hospodin svedek bude mezi mnou a mezi tebou až na veky.
\par 24 A tak skryl se David na poli. Byl pak novmesíce, i sedl král za stul k jídlu.
\par 25 A sedel král na stolici své, jakž obycej mel, na stolici u steny, ale Jonata povstal. Sedl také Abner podlé Saule, a místo Davidovo zustalo prázdné.
\par 26 A však toho dne Saul nic neríkal, nebo myslil: Neco se mu prihodilo, bud že cistý jest neb necistý.
\par 27 Stalo se pak nazejtrí, druhého dne novmesíce, že opet prázdné bylo místo Davidovo. I rekl Saul Jonatovi synu svému: Proc neprišel syn Izai ani vcera ani dnes k jídlu?
\par 28 Odpovedel Jonata Saulovi: Velice mne prosil David, aby šel do Betléma.
\par 29 A rekl: Odpust mne, prosím, nebo obet má míti rodina naše v meste, a bratr muj sám rozkázal mi prijíti; nyní tedy, nalezl-li jsem milost pred ocima tvýma, necht se odtrhnu, prosím, abych navštívil bratrí své. Tou prícinou neprišel k stolu královskému.
\par 30 I rozhneval se Saul náramne na Jonatu, a rekl jemu: Synu prevrácený a urputný, zdaliž nevím, že jsi zvolil sobe syna Izai k hanbe své, i k hanbe a lehkosti matky své?
\par 31 Nebo po všecky dny, v nichž bude živ syn Izai na zemi, nebudeš upevnen ty, ani království tvé. Protož hned pošli a prived jej ke mne, nebt jest hoden smrti.
\par 32 Odpovedel Jonata Saulovi otci svému, a rekl jemu: Proc má umríti? Což jest ucinil?
\par 33 I hodil Saul kopím na nej, aby ho zabil. Tedy seznav Jonata, že uložil otec jeho zabiti Davida,
\par 34 Vstal od stolu Jonata, rozpálen jsa hnevem, a nejedl toho druhého dne novumesíce pokrmu, nebo bolestil pro Davida, a že ho tak zlehcil otec jeho.
\par 35 Protož stalo se ráno, že vyšel Jonata na pole k casu uloženému Davidovi, a pachole malé s ním.
\par 36 Tedy rekl pacholeti svému: Bež a shledej strely, kteréž já vystrelím. I beželo pachole, a on strílel daleko pred nej.
\par 37 Když pak prišlo pachole až k cíli, k nemuž strílel Jonata, volal Jonata za pacholetem a rekl: Zdaliž není strely pred tebou tam dále?
\par 38 Opet volal Jonata za pacholetem: Rychle pospeš, nestuj. A tak sebravši pachole Jonatovo strely, vrátilo se k svému pánu.
\par 39 (Pachole pak nic nevedelo, toliko Jonata a David vedeli, co se jedná.)
\par 40 I dal Jonata bran svou pacholeti, kteréž s ním bylo, a rekl jemu: Jdi, dones do mesta.
\par 41 A když odešlo pachole, vstal David s strany polední, a padna na tvár svou k zemi, poklonil se trikrát; a políbivše jeden druhého plakali oba, až Jonata Davida pozdvihl.
\par 42 I rekl Jonata Davidovi: Jdiž u pokoji, a což jsme sobe oba prisáhli ve jménu Hospodinovu, rkouce: Hospodin budiž svedkem mezi mnou a tebou, i mezi semenem mým a mezi semenem tvým, necht trvá až na veky.
\par 43 A tak vstav David, odšel, Jonata pak navrátil se do mesta.

\chapter{21}

\par 1 Tedy prišel David do Nobe k Achimelechovi knezi. I ulek se Achimelech, vyšel vstríc Davidovi a rekl jemu: Cože to, že jsi sám, a není žádného s tebou?
\par 2 Odpovedel David Achimelechovi knezi: Král mi porucil nejakou vec, a rekl mi: At žádný nezvídá toho, proc te posílám, a cot jsem porucil. Služebníkum pak uložil jsem jisté místo.
\par 3 Protož nyní, co máš tu pred rukama, dej v ruku mou, asi pet chlebu, aneb což na hotove máš.
\par 4 Odpovedel knez Davidovi a rekl: Nemámt chleba obecného pred rukama, než toliko chléb svatý, však jestliže se toliko od žen zdrželi služebníci.
\par 5 Odpovedel David knezi a rekl: Jiste ženy vzdáleny byly od nás, jakož vcera tak i pred vcerejškem, když jsem vyšel; protož tela služebníku svatá jsou. Ac pak toto predsevzetí jest proti slušnosti, však i to dnes posveceno bude pro telo.
\par 6 A tak dal jemu knez chleby svaté; nebo nebylo tam chleba, jediné chlebové predložení, kteríž odloženi byli od tvári Hospodinovy, aby položeni byli chlebové teplí toho dne, když ti vzati byli.
\par 7 (Byl pak tu jeden z služebníku Saulových v týž den, kterýž se tam pozadržel pred Hospodinem, jehož jméno bylo Doeg Idumejský, nejprednejší mezi pastýri Saulovými.)
\par 8 I rekl David Achimelechovi: Nemáš-liž zde kopí aneb mece? Nebo ani mece svého ani brane své nevzal jsem v ruku svou, proto že rozkaz královský dotíral.
\par 9 Jemuž rekl knez: Mec Goliáše Filistinského, kteréhož jsi zabil v údolí Elah, aj, ten jest zde obvinutý v roucho za efodem. Jestliže jej chceš sobe vzíti, vezmi, nebo zde jiného není krome toho. I rekl David: Nenít pres ten, dejž mi jej.
\par 10 Tedy vstal David, a utekl toho dne pred Saulem, a prišel k Achisovi králi Gát.
\par 11 Služebníci pak Achisovi rekli jemu: Zdaliž tento není David, král zeme? Zdaliž neprozpevovali tomuto po houfích, rkouce: Porazil Saul svuj tisíc, David pak svých deset tisícu.
\par 12 I složil David ta slova v srdci svém, a bál se velmi Achisa krále Gát.
\par 13 Protož zmenil zpusob svuj pred ocima jejich, a bláznem se delal, jsa v rukou jejich; psal také po vratech u brány, a pouštel sliny po brade své.
\par 14 Tedy rekl Achis služebníkum svým: Hle, videvše cloveka blázna, procež jste ho ke mne privedli?
\par 15 Nedostává-liž se mi bláznu, že jste uvedli tohoto, aby bláznil prede mnou? Ten-liž má vjíti do mého domu?

\chapter{22}

\par 1 A tak odšel odtud David, skryl se v jeskyni Adulam. To když uslyšeli bratrí jeho i všecken dum otce jeho, sešli se tam k nemu.
\par 2 A shromáždili se k nemu, kterížkoli byli v ssoužení, a kterížkoli byli zadlužilí, a kterížkoli byli v horkosti ducha, a byl nad nimi knížetem, tak že jich bylo s ním okolo ctyr set mužu.
\par 3 Potom odšel odtud David do Masfa Moábského a rekl králi Moábskému: Prosím, necht vcházejí k vám i vycházejí otec muj a matka má, dokudž nezvím, co se mnou uciní Buh.
\par 4 I privedl je pred krále Moábského, a bydlili s ním po všecky dny, v nichž zustával David na tom hrade.
\par 5 Ale prorok Gád rekl Davidovi: Nebývejž déle na tom hrade, jdi, navrat se do zeme Judské. I odšel David, a prišel do lesa Haret.
\par 6 V tom uslyšel Saul, že by se zjevil David i muži, kteríž byli s ním. Saul pak bydlil v Gabaa pod hájem v Ráma, maje kopí své v rukou svých, a všickni služebníci jeho stáli pred ním.
\par 7 I rekl Saul služebníkum svým, kteríž stáli pred ním: Slyšte medle, synové Jemini: Všechnem-li vám dá syn Izai pole a vinice? Všecky-li vás postaví za správce nad tisíci a sty,
\par 8 Že jste se všickni spikli proti mne, aniž jest, kdo by mi oznámil? Již i syn muj ucinil smlouvu s synem Izai, a však žádný z vás nelituje mne, aniž mi kdo oznámí, že pozdvihl syn muj služebníka mého proti mne, aby mi zálohy strojil, jakož se to již deje.
\par 9 Odpovedel pak Doeg Idumejský, kterýž též stál s služebníky Saulovými, a rekl: Videl jsem syna Izai, an prišel do Nobe k Achimelechovi synu Achitobovu.
\par 10 Kterýž radil se o neho s Hospodinem a dal jemu potravy, také i mec Goliáše Filistinského dal jemu.
\par 11 I poslal král, aby zavolali Achimelecha syna Achitobova, kneze, i vší celedi otce jeho, totiž kneží, kteríž byli v Nobe. I prišli všickni pred krále.
\par 12 Tedy rekl Saul: Slyš nyní, synu Achitobuv. Kterýž odpovedel: Ej, pane muj.
\par 13 I dí k nemu Saul: Proc jste se spikli proti mne, ty a syn Izai, když jsi jemu dal chléb a mec, a radils se s Bohem o nej, aby povstal proti mne k strojení mi záloh, jakož se to již deje?
\par 14 Odpovídaje Achimelech králi, rekl: A kdo jest tak verný ze všech služebníku tvých jako David, kterýž i zetem královým jest, kterýž krácí v poslušenství tvém, a vzácný jest v dome tvém?
\par 15 Zdaliž jsem se nyní pocal tázati Boha o nej? Odstup to ode mne. Nescítejž král takové veci na služebníka svého, ani na koho z celedi otce mého, nebot neví služebník tvuj o nicemž o tom ani nejmenší veci.
\par 16 Ale král rekl: Smrtí umreš Achimelechu, ty i všecken dum otce tvého.
\par 17 I rekl král drabantum, kteríž stáli pred ním: Obratte se a zbíte kneží Hospodinovy, nebo i jejich ruka jest s Davidem; pres to vedouce, že utíká, nedali mi znáti. Služebníci však královští nechteli vztáhnouti rukou svých, ani se oboriti na kneží Hospodinovy.
\par 18 A protož rekl král Doegovi: Obrat ty se, a pobí ty kneží. Takž Doeg Idumejský obrátiv se, oboril se na kneží, a zbil toho dne osmdesáte a pet mužu, kteríž nosili efod lnený.
\par 19 Nobe také mesto knežské vyhubil ostrostí mece, od muže až do ženy, od malého až do požívajícího prsí, voly i osly, i dobytky pobil ostrostí mece.
\par 20 Jediný toliko syn Achimelechuv, syna Achitobova, jehož jméno bylo Abiatar, ušel a utekl k Davidovi.
\par 21 Tedy oznámil Abiatar Davidovi, že Saul zmordoval kneží Hospodinovy.
\par 22 I rekl David Abiatarovi: Vedelt jsem toho dne, když tam byl Doeg Idumejský, že jistotne oznámí Saulovi; ját jsem pricinu dal k zhubení všech duší domu otce tvého.
\par 23 Zustan u mne, neboj se; nebo kdož hledati bude bezživotí mého, hledati bude bezživotí tvého, ale ochránen budeš u mne.

\chapter{23}

\par 1 Tedy oznámili Davidovi, rkouce: Aj, Filistinští dobývají Cejly a loupí dvory.
\par 2 Protož tázal se David Hospodina, rka: Mám-li jíti a uderiti na ty Filistinské? I odpovedel Hospodin Davidovi: Jdi a porazíš Filistinské, i Cejlu vysvobodíš.
\par 3 Muži pak Davidovi rekli jemu: Aj, my zde v Judstvu bojíme se, cím více, když pujdeme k Cejle proti vojskum Filistinských.
\par 4 A tak opet David tázal se Hospodina. Jemuž odpovedel Hospodin a rekl: Vstana, vytáhni k Cejle, nebot dám Filistinské v ruce tvé.
\par 5 I táhl David a muži jeho k Cejle, a bojoval s Filistinskými, a zajal dobytky jejich. I porazil je ranou velikou, a tak vysvobodil David obyvatele Cejly.
\par 6 Stalo se pak, že když utíkal Abiatar syn Achimelechuv k Davidovi do Cejly, dostal se efod v ruce jeho.
\par 7 Potom oznámeno bylo Saulovi, že pritáhl David do Cejly. I rekl Saul: Dalt ho Buh v ruku mou, nebo zavrel se, všed do mesta hrazeného a zavritého.
\par 8 I svolal Saul všecken lid k boji, aby táhl k Cejle, a oblehl Davida i muže jeho.
\par 9 A zvedev David, že Saul tajne ukládá o nem zle, rekl Abiatarovi knezi: Vezmi na se efod.
\par 10 I rekl David: Hospodine, Bože Izraelský, za jistou vec slyšel služebník tvuj, že strojí Saul pritáhnouti k Cejle, aby zkazil mesto pro mne.
\par 11 Vydali-li by mne muži Cejly v ruce jeho? A pritáhl-li by Saul, jakž slyšel služebník tvuj? Hospodine, Bože Izraelský, oznam, prosím, služebníku svému. Odpovedel Hospodin: Pritáhl by.
\par 12 Rekl ješte David: Vydali-li by mne obyvatelé Cejly i muže mé v ruce Saulovy? Odpovedel Hospodin: Vydali by.
\par 13 Vstav tedy David a muži jeho, témer šest set mužu, vytáhli z Cejly a šli ustavicne, kamž jíti mohli. Saulovi pak povedíno, že ušel David z Cejly; i nechal tažení.
\par 14 Byl pak David na poušti v místech bezpecných, a bydlil na hore, na poušti Zif. A ackoli hledal ho Saul po všecky ty dny, však nevydal ho Buh v ruku jeho.
\par 15 Vida tedy David, že Saul vytáhl hledati bezživotí jeho, byl na poušti Zif v lese.
\par 16 Vstav pak Jonata syn Sauluv, prišel k Davidovi do lesa, a posilnil ruky jeho v Bohu.
\par 17 A rekl jemu: Neboj se, nebot nenalezne tebe ruka Saule otce mého, ale ty kralovati budeš nad Izraelem, a já budu druhý po tobe. Však i Saul otec muj zná to.
\par 18 I ucinili smlouvu oba pred Hospodinem; a David zustal v lese, Jonata pak navrátil se do domu svého.
\par 19 Tedy prišli Zifejští k Saulovi do Gabaa, rkouce: Zdaliž David nepokrývá se u nás v horách, v lese, na pahrbku Hachile, kterýž jest na pravé strane poušte.
\par 20 Protož nyní podlé vší žádosti duše své, ó králi, pritáhni steží, my se pak priciníme, abychom jej vydali v ruce královy.
\par 21 I rekl Saul: Požehnaní vy od Hospodina, že mne litujete.
\par 22 Jdetež medle, ujistte se tím ješte lépe, prezvezte a shlédnete místo jeho, kam se obrátí, kdo ho tam videl; nebo mi praveno, že divných chytrostí užívá.
\par 23 Vyšpehujtež tedy a vyzvezte všecky skrejše, v nichž se kryje, a navratte se ke mne s vecí jistou, i potáhnu s vámi, a jestliže v zemi jest, hledati ho budu ve všech tisících Judských.
\par 24 Kterížto vstavše, navrátili se do Zif pred Saulem; David pak a muži jeho byli na poušti Maon, na rovinách po pravé strane poušte.
\par 25 Nebo jakž vytáhl Saul s lidem svým hledati ho, oznámeno bylo to Davidovi, kterýž sstoupil s skály a bydlil na poušti Maon. O cemž Saul uslyšav, honil Davida po poušti Maon.
\par 26 A tak Saul táhl s jedné strany hory, David pak a muži jeho po druhé strane hory. A pospíšil David, aby mohl utéci od tvári Saulovy; nebo Saul s lidem svým obklicovali Davida i muže jeho, aby je zjímali.
\par 27 V tom prišel posel k Saulovi, rka: Pospeš a prid, nebo Filistinští vtrhli do zeme.
\par 28 A protož navrátil se Saul od honení Davida, táhl proti Filistinským. Procež nazvali to místo: Skála rozdelující.

\chapter{24}

\par 1 Odšed pak David odtud, bydlil v místech bezpecných Engadi.
\par 2 I stalo se, že když se navrátil Saul od honení Filistinských, oznámili jemu, rkouce: Hle, David jest na poušti Engadi.
\par 3 Tedy vzav Saul tri tisíce mužu vybraných ze všeho Izraele, odšel hledati Davida a mužu jeho na skalách kamsíku.
\par 4 I prišel k stájím ovcí blízko cesty, kdež byla jeskyne, do níž všel Saul na potrebu; David pak a muži jeho sedeli po stranách v té jeskyni.
\par 5 Protož rekli muži Davidovi jemu: Aj, tento jest den, o nemž tobe mluvil Hospodin, rka: Aj, já dám neprítele tvého v ruku tvou, abys mu ucinil, jakt se koli líbiti bude. Tedy vstav David, urezal tiše kus plášte Saulova.
\par 6 Potom pak padlo to težce na srdce Davidovi, že urezal krídlo plášte Saulova.
\par 7 Procež rekl mužum svým: Uchovejž mne Hospodin, abych to uciniti mel pánu mému, pomazanému Hospodinovu, abych vztáhnouti mel ruku svou na nej, ponevadž jest pomazaný Hospodinuv.
\par 8 A tak zabránil David mužum svým temi slovy, a nedal jim povstati proti Saulovi. Saul také vyšed z té jeskyne, bral se cestou svou.
\par 9 Potom vstal i David, a vyšed z jeskyne, volal za Saulem, rka: Pane muj králi! I ohlédl se Saul zpet, David pak sehnuv se tvárí k zemi, poklonil se jemu.
\par 10 A rekl David Saulovi: I proc posloucháš recí lidských, kteríž praví: Hle, David hledá tvého zlého.
\par 11 Aj, dnešního dne videti mohly oci tvé, že te byl vydal Hospodin dnes v ruku mou v jeskyni. A bylot mi receno, abych te zabil, ale šanoval jsem te; nebo rekl jsem: Nevztáhnut ruky své na pána svého, ponevadž jest pomazaný Hospodinuv.
\par 12 Nýbrž, otce muj, pohled a viz kus plášte svého v ruce mé, a žet jsem nechtel, odrezuje krídlo plášte tvého, zabiti tebe. Poznejž tedy a viz, žet není v úmysle mém nic zlého, ani jaké prevrácenosti, a žet jsem nezhrešil proti tobe; ty pak cíháš na duši mou, abys mi ji odjal.
\par 13 Sudiž Hospodin mezi mnou a mezi tebou, a pomstiž mne Hospodin nad tebou, ale ruka má nebude proti tobe.
\par 14 Jakož vzní ono prísloví starých: Od bezbožných vychází bezbožnost; protož nebudet ruka má proti tobe.
\par 15 Na koho to jen vytáhl král Izraelský? Koho to honíš? Psa mrtvého, blechu jednu.
\par 16 Ale budet Hospodin soudce; on necht rozsoudí mezi mnou a tebou, a necht pohledí a vyvede pri mou, a vysvobodí mne z ruky tvé.
\par 17 Když pak prestal David mluviti slov tech Saulovi, odpovedel Saul: Není-liž to hlas tvuj, synu muj Davide? A pozdvih Saul hlasu svého, plakal.
\par 18 A rekl Davidovi: Spravedlivejší jsi nežli já; nebo ty jsi mi odplatil se dobrým, já pak zlým tobe jsem se odplatil.
\par 19 Ty zajisté ukázal jsi dnes, že mi ciníš dobre; nebo ackoli mne byl Hospodin zavrel v ruce tvé, však jsi mne nezabil.
\par 20 Zdali kdo nalezna neprítele svého, propustí ho po ceste dobré? Ale Hospodin odplatiž tobe dobrým za to, co jsi mi dnešního dne ucinil.
\par 21 Protož nyní, (vím, že jistotne kralovati budeš, a že stálé bude v ruce tvé království Izraelské),
\par 22 Nyní, pravím, prisáhni mi skrze Hospodina, že nevypléníš semene mého po mne, a nevyhladíš jména mého z domu otce mého.
\par 23 A tak prisáhl David Saulovi. I odšel Saul do domu svého, David pak a muži jeho vstoupili na bezpecné místo.

\chapter{25}

\par 1 Mezi tím umrel Samuel. A shromáždil se všecken Izrael, i plakali ho, a pochovali jej v dome jeho v Ráma. Ale David vstav, šel na poušt Fáran.
\par 2 Clovek pak nejaký byl v Maon, kterýž statek svuj mel na Karmeli. A byl clovek ten možný velmi, nebo mel tri tisíce ovec a tisíc koz, a tehdáž rovne strihl ovce své na Karmeli.
\par 3 Muže pak toho jméno Nábal, a jméno ženy jeho Abigail. A byla žena ta opatrná a krásné tvári, ale muž její byl tvrdý a zlých povah, a byl z rodu Kálefova.
\par 4 Protož uslyšav David na poušti, že by Nábal strihl ovce své,
\par 5 Poslal deset služebníku, a rekl David tem služebníkum: Vstupte na Karmel, a jdete k Nábalovi, a pozdravte ho slovem mým prátelsky.
\par 6 A rcete jemu takto: Zdráv bud, a pokoj tobe, pokoj domu tvému, i všemu, což máš, pokoj.
\par 7 Slyšel jsem, že máš strižce, a pastýri tvoji bývali s námi; neucinili jsme jim žádné krivdy, aniž jim co zhynulo po všecky dny, v nichž byli na Karmeli.
\par 8 Ptej se služebníku svých, a povedí tobe. Nyní tedy necht naleznou mládenci milost pred ocima tvýma, nebo v den veselý prišli jsme; dej, prosím, což máš pred rukama, služebníkum svým a synu svému Davidovi.
\par 9 Prišedše tedy mládenci ti Davidovi, mluvili Nábalovi podlé všech slov techto jménem Davidovým, a umlkli.
\par 10 Odpovedel pak Nábal služebníkum Davidovým a rekl: Kdo jest David? A kdo syn Izai? Mnohote nyní služebníku, kteríž se odtrhují jeden každý od pána svého.
\par 11 Ano, vezmu já chléb svuj, a vodu svou a pokrmy své, kteréž jsem pripravil strižcum svým, a dám je tem lidem, kterýchž neznám, ani vím, odkud jsou?
\par 12 A obrátivše se služebníci Davidovi na cestu svou, navrátili se, a prišedše, oznámili jemu všecka slova ta.
\par 13 I rekl David mužum svým: Pripaš každý mec svuj. Kteríž když pripásali jeden každý mec svuj, pripásal také David mec svuj, a šlo za Davidem okolo ctyr set mužu, dve ste pak pozustalo u bremen.
\par 14 A v tom Abigail, žene Nábalove, oznámil mládenec jeden z služebníku, rka: Aj, poslal David posly s poušte, aby pozdravili pána našeho, ale on je ukrikal;
\par 15 Ješto muži ti prospešní nám byli velice, ani nám neucinili krivdy, aniž nám co zhynulo, když jsme bývali s nimi na poli.
\par 16 Místo zdi byli nám v noci i ve dne po všecky dny, dokudž jsme s nimi byli, pasouce stáda.
\par 17 Protož nyní pomysl a viz, co bys mela ciniti, nebot již zlé veci hotové jsou na pána našeho i na všecken dum jeho; on pak jest tak zlobivý, že s ním nelze ani mluviti.
\par 18 Tedy pospíšila Abigail, a vzala dve ste chlebu, a dve kožené láhvice vína, a pet ovcí pripravených, a pet mer pražmy, a sto sušených hroznu, a dve ste hrud fíku sušených, a vložila to na osly.
\par 19 I rekla služebníkum svým: Jdetež napred, a já za vámi pujdu. Ale muži svému Nábalovi neoznámila.
\par 20 I stalo se, že vsedši na osla, sjíždela po stráni s hory, a aj, David a muži jeho sstupovali proti ní, a potkala se s nimi.
\par 21 (David pak byl rekl: Jiste nadarmo jsem ostríhal všeho, což on mel na poušti, tak že nic nezahynulo ze všeho, což má; nebo mi se odplatil zlým za dobré.
\par 22 Toto ucin Buh neprátelum Davidovým a toto pridej, jestliže zanechám do jitra ze všeho, což má, až do toho, kterýž mocí na stenu.)
\par 23 Tedy uzrevši Abigail Davida, rychle ssedla s osla, a padla pred Davidem na tvár svou, a poklonila se až k zemi.
\par 24 A padši k nohám jeho, rekla: Na mne, pane muj, ta nepravost. Protož necht mluví, prosím, služebnice tvá v uši tvé, a vyslyš slova devky své.
\par 25 Necht se neobrací, prosím, pán muj myslí svou za mužem tím bezbožným Nábalem; nebo jakéž jest jméno jeho, takovýž jest. Nábal jméno jeho jest, a bláznovství jest pri nem. Já pak služebnice tvá nevidela jsem služebníku pána mého, kteréž jsi byl poslal.
\par 26 Protož nyní, pane muj, živt jest Hospodin, a živat jest duše tvá, že te tobe zbránil Hospodin vylíti krve, a abys nemstil sám sebe. A protož nyní budtež jako Nábal neprátelé tvoji, a ti, kteríž hledají zlého pánu mému.
\par 27 Ted pak dar tento, kterýž prinesla devka tvá pánu svému, necht jest dán služebníkum, kteríž chodí za pánem mým.
\par 28 Odpust, prosím, provinení devce své, nebot jiste vzdelá Hospodin pánu mému dum stálý, ponevadž boje Hospodinovy pán muj vede, a nic zlého se nenalézá pri tobe až posavad.
\par 29 A byt i povstal clovek, aby te stihal a hledal bezhrdlí tvého, budet však duše pána mého svázána v svazku živých u Hospodina Boha tvého, duši pak neprátel tvých jako z praku pryc zahodí.
\par 30 A když uciní Hospodin pánu mému dobre podlé toho všeho, jakž zaslíbil tobe, a prikážet, abys byl vudcím nad Izraelem:
\par 31 Tedy nebude to k zviklání ani k urážce srdce pánu mému, jako když by vylil krev bez príciny, aneb když by se mstil pán muj. Když tedy uciní dobre Hospodin pánu mému, rozpomen se na devku svou.
\par 32 I rekl David k Abigail: Požehnaný Hospodin Buh Izraelský, že te poslal dne tohoto mne v cestu;
\par 33 A požehnaná rec tvá, i ty požehnaná, že jsi zdržela mne dnes, abych nevylil krve a nemstil sám sebe.
\par 34 A jiste, jako živ jest Hospodin Buh Izraelský, kterýž mi zbránil, at bych zle necinil, že kdybys nebyla pospíšila a nevyšla mi v cestu, nebyl by zustal Nábalovi do jitra ani mocící na stenu.
\par 35 I prijal David z ruky její, což byla prinesla jemu, a rekl jí: Jdiž v pokoji do domu svého. Pohled, uslyšel jsem prosbu tvou, a vážil jsem sobe osoby tvé.
\par 36 A tak navrátila se Abigail k Nábalovi, an mel hody v dome svém jako hody královské; a srdce Nábalovo rozveselilo se bylo v nem, a byl opilý velmi. Procež ona neoznámila jemu nejmenšího slova až do jitra.
\par 37 Nazejtrí pak, když z vína vystrízlivel Nábal, tedy oznámila jemu žena jeho ty veci. I zmrtvelo v nem srdce jeho, a ucinen jest jako kámen.
\par 38 A když pominulo takmer deset dní, porazil Hospodin Nábale,i umrel.
\par 39 Uslyšev pak David, že by Nábal umrel, rekl: Požehnaný Hospodin, kterýž hodne pomstil pohanení mého nad Nábalem, a služebníka svého zdržel ode zlého, zlost pak Nábalovu shrnul Hospodin na hlavu jeho. Tedy poslal David a mluvil k Abigail, aby ji sobe vzal za manželku.
\par 40 I prišli služebníci Davidovi k Abigail na Karmel, a mluvili s ní, rkouce: David poslal nás k tobe, aby te vzal sobe za manželku.
\par 41 Kterážto vstavši, poklonila se na tvár až k zemi, rkuci: Aj, služebnice tvá za devku bude, aby umývala nohy služebníkum pána svého.
\par 42 Protož rychle vstavši Abigail a vsedši na osla svého, (pet pak devecek jejích šlo za ní), jela za posly Davidovými, a byla manželkou jeho.
\par 43 Též i Achinoam pojal David z Jezreel, a byly i tyto dve manželky jeho.
\par 44 Nebo Saul Míkol dceru svou, manželku Davidovu, dal byl Faltiovi, synu Lais, kterýž byl z Gallim.

\chapter{26}

\par 1 Opet prišli Zifejští k Saulovi do Gabaa, rkouce: Nevíš-liž, že se David kryje na pahrbku Hachila proti poušti?
\par 2 Protož povstal Saul a táhl na poušt Zif, a s ním tri tisíce mužu vybraných z Izraele, aby hledal Davida na poušti Zif.
\par 3 I položil se Saul na pahrbku Hachila, kterýž jest proti Jesimon pri ceste. David pak trvaje na poušti, srozumel, že Saul za ním pritáhl na poušt.
\par 4 Nebo poslav David špehére, vyzvedel jistotne, že Saul pritáhl.
\par 5 Tedy vstav David, šel k místu, na nemž se položil Saul s vojskem. I spatril David místo, na kterémž ležel Saul a Abner syn Neruv, hejtman vojska jeho. Saul pak spal, jsa vozy otocen, lid také ležení svá mel vukol neho.
\par 6 I mluvil David a rekl Achimelechovi Hetejskému a Abizai synu Sarvie, bratru Joábovu, rka: Kdo sstoupí se mnou k Saulovi do ležení? Odpovedel Abizai: Já sstoupím s tebou.
\par 7 A tak prišel David a Abizai k lidu v noci, a aj, Saul leže, spal, jsa vozy otocen, a kopí jeho vetknuté bylo v zemi u hlavy jeho, Abner pak i lid spali vukol neho.
\par 8 Tedy rekl Abizai Davidovi: Dalt Buh dnes neprítele tvého v ruku tvou. Protož nyní, medle necht jej probodnu pojednou kopím až do zeme, tak že nebude potrebí podruhé.
\par 9 Ale David rekl k Abizai: Nezabíjej ho; nebo kdo vztáhna ruku svou na pomazaného Hospodinova, byl by bez viny?
\par 10 Rekl také David: Živt jest Hospodin, lec Hospodin raní jej, aneb den jeho prijde, aby umrel, aneb na vojnu vytáhna, zahyne:
\par 11 Mne nedej Hospodin, abych vztáhnouti mel ruku svou na pomazaného Hospodinova. Ale nyní vezmi medle to kopí, kteréž jest u hlavy jeho, a tu cíši vodnou, a odejdeme.
\par 12 I vzal David kopí a cíši vodnou u hlavy Saulovy, a odešli, tak že žádný nevidel, ani nezvedel, ani neprocítil, ale všickni spali; nebo sen tvrdý Hospodinuv pripadl byl na ne.
\par 13 A prešed David na druhou stranu, postavil se na vrchu hory zdaleka; nebo bylo mezi nimi nemalé místo.
\par 14 I zavolal David na lid a na Abnera syna Ner, rka: Což se neozveš, Abner? Odpovídaje pak Abner, rekl: Kdo jsi ty, kterýž voláš na krále?
\par 15 I rekl David Abnerovi: Zdaliž ty nejsi muž? A kdo jest tobe rovný v Izraeli? Proc jsi tedy neostríhal krále, pána svého? Nebo prišel jeden z lidu, aby zabil krále, pána tvého.
\par 16 Nenít to dobre, co jsi ucinil. Živt jest Hospodin, že jste hodni smrti, proto že neostríháte pána svého, pomazaného Hospodinova. Ale nyní pohled, kde jest kopí královo a cíše vodná, kteráž byla u hlavy jeho.
\par 17 Tedy poznal Saul hlas Daviduv a rekl: Není-liž to hlas tvuj, synu muj Davide? Odpovedel David: Jest muj hlas, pane muj králi.
\par 18 Rekl také: Proc je to, že pán muj honí služebníka svého? Nebo co jsem ucinil? A co jest zlého v ruce mé?
\par 19 Protož nyní poslyš, prosím, pane muj králi, slov služebníka svého: Jestliže te Hospodin vzbudil proti mne, necht zachutná obet, pakli lidé, zlorecení jsou pred Hospodinem; nebo mne vyhnali dnes, abych nemohl obcovati dedictví Hospodinovu, jako by rekli: Jdi, služ bohum cizím.
\par 20 Ale již aspon necht není vylita krev má na zemi bez rozsouzení Hospodinova; nebo vytáhl král Izraelský hledati blechy jedné, rovne jako by honil koroptvu na horách.
\par 21 I rekl Saul: Zhrešilt jsem, navratiž se, synu muj Davide. Nebot nebudu více zle ciniti tobe, proto že jsi draze sobe vážil života mého dnešní den. Aj, bláznive jsem delal a bloudil prenáramne.
\par 22 A odpovídaje David, rekl: Ted hle kopí královo. Necht prijde nekdo z služebníku, a vezme je.
\par 23 Hospodin pak navratiž jednomu každému za spravedlnost jeho a vernost jeho. Dalte byl zajisté Hospodin tebe dnes v ruku mou, ale nechtelt jsem vztáhnouti ruky své na pomazaného Hospodinova.
\par 24 A protož jakož jsem já dnes sobe draze vážil života tvého, tak budiž draze vážen život muj pred Hospodinem, aby mne vysvobodil ze vší úzkosti.
\par 25 Tedy rekl Saul Davidovi: Požehnaný jsi, synu muj Davide. Tak cine, dokážeš ctnosti, a v tom se zmocnuje, zkvetneš. V tom odšel David cestou svou, Saul také navrátil se k místu svému.

\chapter{27}

\par 1 Rekl pak David v srdci svém: Když tedyž sejdu od ruky Saulovy, nic mi lepšího není, než abych naprosto utekl do zeme Filistinské. I pustí o mne Saul, a nebude mne více hledati v žádných koncinách Izraelských, a tak ujdu ruky jeho.
\par 2 Tedy vstav David, odebral se sám i tech šest set mužu, kteríž byli s ním, k Achisovi synu Maoch, králi Gát.
\par 3 I bydlil David s Achisem v Gát, on i muži jeho, jeden každý s celedí svou, David i dve ženy jeho, Achinoam Jezreelská, a Abigail Karmelská nekdy žena Nábalova.
\par 4 A když bylo oznámeno Saulovi, že utekl David do Gát, prestal ho hledati více.
\par 5 Rekl pak David Achisovi: Prosím, jestliže jsem nalezl milost pred ocima tvýma, at mi dají místo v nekterém meste krajiny této, abych tam bydlil; nebo proc má bydliti služebník tvuj s tebou v meste královském?
\par 6 I dal mu Achis v ten den Sicelech, odkudž Sicelech bylo králu Judských až do dne tohoto.
\par 7 Byl pak pocet dnu, v nichž bydlil David v krajine Filistinské, den a ctyri mesíce.
\par 8 I vycházel David s muži svými, vpády ciníce na Gessurské a Gerzitské a Amalechitské, (nebo ti bydlili v zemi té od starodávna), kudy se chodí pres Sur až do zeme Egyptské.
\par 9 A hubil David krajinu tu, nenechávaje živého muže ani ženy; bral také ovce i voly, i osly i velbloudy, i šaty, a navracoval se a pricházel k Achisovi.
\par 10 A když se ptal Achis: Kam jste dnes vpadli? odpovedel David: K strane polední Judove, a k strane polední Jerachmeelove, a k strane polední Cinejského.
\par 11 Neživil pak David ani muže ani ženy, aby koho privoditi mel do Gát; nebo myslil: Aby na nás nežalovali, rkouce: Tak ucinil David. A ten obycej jeho byl po všecky dny, dokudž zustával v krajine Filistinské.
\par 12 I veril Achis Davidovi, rka: Jižte se velice zošklivil lidu svému Izraelskému, protož budet mi za služebníka na veky.

\chapter{28}

\par 1 I stalo se za dnu tech, že Filistinští sebrali vojska svá k boji, aby bojovali s Izraelem. I rekl Achis Davidovi: Vez nepochybne, že potáhneš se mnou na vojnu, ty i muži tvoji.
\par 2 Odpovedel David Achisovi: Teprv ty poznáš, co uciní služebník tvuj. I rekl Achis Davidovi: Tout prícinou strážným života svého te ustanovím po všecky dny.
\par 3 (Samuel pak již byl umrel; procež plakal ho všecken Izrael, a pochovali jej v Ráma, totiž v meste jeho. A Saul byl vyplénil veštce a hadace z zeme.)
\par 4 Tedy shromáždivše se Filistinští, pritáhli a položili se u Sunem. Shromáždil i Saul všeho Izraele, a položili se v Gelboe.
\par 5 Vida pak Saul vojsko Filistinské, bál se, a uleklo se srdce jeho velmi.
\par 6 I dotazoval se Saul Hospodina, ale Hospodin neodpovídal jemu ani skrze sny, ani skrze urim, ani skrze proroky.
\par 7 Protož rekl Saul služebníkum svým: Pohledejte mi ženy mající ducha veštího, i pujdu k ní a poradím se skrze ni. Jemuž odpovedeli služebníci jeho: Aj, žena mající ducha veštího v Endor.
\par 8 Tedy zmeniv Saul odev, oblékl se v roucho jiné, a šel sám a dva muži s ním, a prišli k žene té v noci. I rekl: Medle hádej mi skrze ducha veštího, a zpusob to, at ke mne vyjde ten, kohož bych jmenoval tobe.
\par 9 Ale žena rekla jemu: Aj, ty víš, co ucinil Saul, kterak vyhladil veštce a hadace z zeme. Procež tedy ty pokládáš osídlo duši mé, abys mne o hrdlo pripravil?
\par 10 I prisáhl jí Saul skrze Hospodina, rka: Živt jest Hospodin, že neprijde na te trestání pro tu vec.
\par 11 Tedy rekla žena: Kohožt mám vyvésti? Kterýž rekl: Samuele mi vyved.
\par 12 A když uzrela žena Samuele, zkrikla hlasem velikým, a rekla žena Saulovi takto: Procež jsi mne oklamal! Nebo ty jsi Saul.
\par 13 I rekl jí král: Neboj se. Což jsi pak videla? Odpovedela žena Saulovi: Bohy jsem videla vystupující z zeme.
\par 14 Rekl jí opet: Jaký jest zpusob jeho? Odpovedela jemu: Muž starý vystupuje a jest odený pláštem. Tedy srozumel Saul, že by Samuel byl, a sehnuv se tvárí k zemi, poklonil se jemu.
\par 15 I rekl Samuel Saulovi: Proc mne nepokojíš, že jsi mne zavolati rozkázal? Odpovedel Saul: Úzkostmi sevrín jsem velice; nebo Filistinští bojují proti mne, a Buh odstoupil ode mne, a neodpovídá mi více, ani skrze proroky, ani skrze sny. Protož povolal jsem te, abys mi oznámil, co bych mel ciniti.
\par 16 I rekl Samuel: Proc tedy se mne dotazuješ, ponevadž Hospodin odstoupil od tebe, a jest s neprítelem tvým?
\par 17 Ucinilte zajisté jemu Hospodin, jakož mluvil skrze mne, a odtrhl Hospodin království od ruky tvé, a dal je bližnímu tvému, Davidovi.
\par 18 Nebo že jsi neuposlechl hlasu Hospodinova, a nevykonals hnevu prchlivosti jeho nad Amalechem, protož ucinil tobe to dnes Hospodin.
\par 19 Nadto vydá Hospodin i Izraele s tebou v ruku Filistinských, a zítra budeš ty i synové tvoji se mnou. I vojska Izraelská vydá Hospodin v ruku Filistinských.
\par 20 I padl Saul náhle tak, jak dlouhý byl, na zem, nebo se byl ulekl náramne slov Samuelových. K tomu ani síly v nem nebylo, nebo nic nejedl celý ten den a celou tu noc.
\par 21 Pristoupivši pak ta žena k Saulovi a uzrevši, že jest predešen náramne, rekla jemu: Aj, uposlechla devka tvá hlasu tvého, a opovážila jsem se života svého, že jsem uposlechla slov tvých, kteráž jsi mluvil ke mne.
\par 22 Nyní tedy uposlechni i ty, prosím, hlasu devky své, a položím pred tebe kousek chleba, abys jedl a posilil se, a tak jíti mohl cestou svou.
\par 23 Kterýž odeprel a rekl: Nebudut jísti. I prinutili ho služebníci jeho, ano i ta žena, tak že uposlechl hlasu jejich, a vstav s zeme, sedl na lužko.
\par 24 Mela pak ta žena tele tucné v dome, kteréž spešne zabila, a vzavši mouky, zadelala, a napekla z ní chlebu presných.
\par 25 Potom prinesla pred Saule a služebníky jeho, kteríž jedli, a vstavše v touž noc, odešli.

\chapter{29}

\par 1 A tak shromáždili Filistinští všecka vojska svá u Afeku, Izrael pak položil se u studnice, kteráž byla v Jezreel.
\par 2 I táhla knížata Filistinská po stu a po tisících, David pak a muži jeho táhli nazad s Achisem.
\par 3 Tedy rekla knížata Filistinská: K cemu jsou Židé tito? Odpovedel Achis knížatum Filistinským: Zdaliž toto není David služebník Saule, krále Izraelského, kterýž byl pri mne dnu techto, nýbrž techto let, a neshledal jsem na nem niceho ode dne, jakž odpadl od Saule, až do tohoto dne?
\par 4 I rozhnevala se na nej knížata Filistinská, a rekli jemu ta knížata Filistinská: Odešli zase muže toho, at se navrátí k místu svému, kteréž jsi mu ukázal, a necht netáhne s námi k boji, aby se nám nepostavil za neprítele v bitve. Nebo cím se zalíbiti muže pánu svému tento? Zdali ne hlavami mužu techto?
\par 5 Zdaliž tento není ten David, o kterémž zpívali v houfích plésajících, ríkajíce: Porazil Saul svuj tisíc, ale David svých deset tisícu?
\par 6 I povolal Achis Davida, a rekl jemu: Živt jest Hospodin, že jsi uprímý, a líbí mi se vycházení tvé i vcházení tvé se mnou do vojska. Nebo neshledal jsem na tobe nic zlého ode dne, v kterýž jsi prišel ke mne, až do dne tohoto, ale pred ocima knížat nejsi vzácný.
\par 7 Protož nyní navrat se a jdi v pokoji, a nebudeš težký v ocích knížat Filistinských.
\par 8 I rekl David Achisovi: Co jsem pak ucinil, a co jsi shledal na služebníku svém ode dne, v kterýž jsem pocal býti u tebe, až do tohoto dne, abych nešel a nebojoval proti neprátelum pána svého krále?
\par 9 A odpovídaje Achis, rekl Davidovi: Vímt, že jsi vzácný pred ocima mýma jako andel Boží, ale knížata Filistinská rekla: Necht netáhne s námi k boji.
\par 10 Nyní tedy vstan tím raneji a služebníci pána tvého, kteríž prišli s tebou, a vstanouce tím spíše ráno, hned jakž by zasvitávalo, odejdete.
\par 11 I vstal David, on i muži jeho, aby odšel tím raneji, a navrátil se do zeme Filistinské. Filistinští pak táhli do Jezreel.

\chapter{30}

\par 1 Byl pak, když se navrátil David a muži jeho do Sicelechu, den tretí, jakž Amalechitští byli vpád ucinili k strane polední i k Sicelechu, a vyhubili Sicelech, a vypálili jej.
\par 2 A zajali ženy, kteréž byly v nem. Nezabili žadného, ani malého ani velikého, ale szajímali a odešli cestou svou.
\par 3 A když prišel David a muži jeho k mestu, aj, vypáleno bylo ohnem, a ženy jejich, též synové a dcery jejich zajati byli.
\par 4 Tedy David i lid jeho, kterýž s ním byl, pozdvihše hlasu svého, plakali, až již více plakati nemohli.
\par 5 Obe také manželky Davidovy zajaty jsou, Achinoam Jezreelitská, a Abigail žena nekdy Nábale Karmelského.
\par 6 I ssoužen jest David náramne, nebo se smlouval lid, aby ho ukamenovali, (horkostí zajisté naplnena byla duše všeho lidu, jednoho každého pro syny jeho a pro dcery jeho). Však posilnil se David v Hospodinu Bohu svém.
\par 7 I rekl David Abiatarovi knezi, synu Achimelechovu: Medle, vezmi na sebe efod. I vzal Abiatar efod pro Davida.
\par 8 Tázal se pak David Hospodina, rka: Mám-li honiti lotríky ty? A dohoním-li se jich? I rekl jemu: Hon, nebo se jich jiste dohoníš, a své mocne vysvobodíš.
\par 9 A tak odšed David sám i tech šest set mužu, kteríž byli s ním, prišli až ku potoku Bezor; nekterí pak tu pozustali.
\par 10 I honil je David se ctyrmi sty mužu; nebo bylo pozustalo dve ste mužu, kteríž ustavše, nemohli prejíti potoka Bezor.
\par 11 A nalezše muže Egyptského v poli, privedli jej k Davidovi. I dali jemu chleba, aby pojedl; dali jemu také i vody píti.
\par 12 Dali jemu též kus hrudy fíku a dva hrozny suché. A tak pojedl a okrál zase, (nebo byl nic nejedl ani nepil tri dni a tri noci).
\par 13 Zatím rekl jemu David: Cí jsi ty? A odkud jsi? Kterýž odpovedel: Rodem jsem z Egypta, služebník muže Amalechitského, a opustil mne pán muj, proto že jsem stonal, dnes tretí den.
\par 14 Byli jsme zajisté vpád ucinili k strane polední Ceretejského, a v tu stranu, kteráž jest Judova, a ku poledni, kteráž jest Kálefova, a Sicelech jsme vypálili ohnem.
\par 15 Tedy rekl jemu David: Mohl-li bys dovésti mne k tem lotríkum? Kterýž rekl: Prisáhni mi skrze Boha, že mne nezabiješ, a že mne nevydáš v ruku pána mého, a privedu te na ty lotríky.
\par 16 I privedl ho. (A aj, byli se rozprostreli po vší té zemi, jedouce a pijíce a provyskujíce nade všemi koristmi tak velikými, kteréž pobrali z zeme Filistinské a z zeme Judovy.)
\par 17 Protož bil je David od vecera až do vecera druhého dne, aniž kdo z nich ušel, krome ctyr set mládencu, kteríž vsedše na velbloudy, utekli.
\par 18 A tak odjal David všecko, což byli pobrali Amalechitští; také obe ženy své vysvobodil David.
\par 19 A nezhynulo jim nic, ani malého ani velikého, i z synu i ze dcer, i z loupeže a ze všeho, což jim vzato bylo; všecko zase David privedl.
\par 20 Nadto szajímal David všecka stáda bravu i skotu, kteráž hnali pred dobytkem svým, a pravili: Totot jsou koristi Davidovy.
\par 21 Prišel pak David k tem dvema stum mužu, kteríž byli ustali, tak že nemohli jíti za Davidem, jimž byli kázali zustati pri potoku Bezor; tedy vyšli vstríc Davidovi a lidu, kterýž byl s ním. A pristoupiv David k tomu lidu, pozdravil jich prátelsky.
\par 22 Ale všickni, což jich koli bylo zlých a bezbožných mezi temi muži, kteríž chodili s Davidem, mluvili, rkouce: Ponevadž nešli s námi, nedáme jim z koristí, kteréž jsme odjali, toliko každému manželku jeho a syny jeho, aby vezmouce je, odešli.
\par 23 David pak rekl: Necinte tak, bratrí moji, s tím, což nám dal Hospodin, kterýž nás ostríhal a dal vojsko, jenž vytáhlo proti nám, v ruku naši.
\par 24 A kdož vás uposlechne v té veci? Nebo jakýž bude díl toho, kterýž vyšel k bitve, takovýž bude díl i toho, kterýž hlídal bremen; rovne se deliti budou.
\par 25 A tak bývalo od toho dne i potom, nebo to za právo a obycej uložil v Izraeli až do tohoto dne.
\par 26 A když prišel David do Sicelechu, poslal z tech koristí starším Juda, prátelum svým, rka: Ted máte dar z loupeží neprátel Hospodinových.
\par 27 Tem, kteríž byli v Bethel, a kteríž v Rámat ku poledni, a kteríž byli v Jeter;
\par 28 Též kteríž v Aroer, a kteríž v Sefama, a kteríž v Estemo;
\par 29 A kteríž v Rachal, a kteríž v mestech Jerachmeelových, a kteríž v mestech Cinejského;
\par 30 I tem, kteríž v Horma, a kteríž v Korasan, a kteríž v Atach;
\par 31 A kteríž byli v Hebronu, i po všech místech, na nichž býval David s lidem svým.

\chapter{31}

\par 1 A tak potýkali se Filistinští s Izraelem. Muži pak Izraelští utíkali pred Filistinskými a padli, zbiti jsouce na hore Gelboe.
\par 2 I stihali Filistinští Saule a syny jeho, a zabili Filistinští Jonatu a Abinadaba a Melchisua, syny Saulovy.
\par 3 Když se pak zsilila bitva proti Saulovi, trefili na nej strelci, muži s luky; i postrelen jest velmi od tech strelcu.
\par 4 I rekl Saul odenci svému: Vytrhni mec svuj a probodni mne jím, aby prijdouce ti neobrezanci, neprobodli mne, a neucinili sobe ze mne posmechu. Ale nechtel odenec jeho, nebo se bál velmi. A pochytiv Saul mec, nalehl na nej.
\par 5 Tedy vida odenec jeho, že umrel Saul, nalehl i on na mec svuj a umrel s ním.
\par 6 A tak umrel Saul a tri synové jeho, a odenec jeho, ano i všickni muži jeho v ten den spolu.
\par 7 Když pak uzreli synové Izraelští, kteríž bydlili za tím údolím, a kteríž bydlili za Jordánem, že zutíkali muži Izraelští, nadto že Saul i synové jeho zahynuli, opustivše mesta, také utíkali. I prišli Filistinští a bydlili v nich.
\par 8 A když bylo nazejtrí, prišli Filistinští, aby zloupili pobité; i nalezli Saule a tri syny jeho ležící na hore Gelboe.
\par 9 I stali hlavu jeho a svlékli odení jeho, a poslali po zemi Filistinské vukol, aby to ohlášeno bylo v chráme modl jejich i lidu.
\par 10 I složili odení jeho v chráme Astarot, telo pak jeho pribili na zdi Betsan.
\par 11 Tedy uslyšavše o tom obyvatelé Jábes Galád, co ucinili Filistinští Saulovi,
\par 12 Zdvihli se všickni muži silní, a jdouce celou noc, snali telo Saulovo i tela synu jeho se zdi Betsanské; a když se navrátili do Jábes, spálili je tam.
\par 13 A vzavše kosti jejich, pochovali je pod stromem v Jábes, a postili se sedm dní.

\end{document}