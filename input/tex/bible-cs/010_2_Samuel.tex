\begin{document}

\title{2 Samuel}

\chapter{1}

\par 1 Stalo se pak po smrti Saulove, když se navrátil David od porážky Amalechitských, že pobyl v Sicelechu za dva dni.
\par 2 A aj, dne tretího prišel jeden z vojska Saulova, maje roucho roztržené a prach na hlave své. Kterýž když prišel k Davidovi, padl na zemi a poklonil se.
\par 3 I rekl jemu David: Odkud jdeš? Jemuž odpovedel: Z vojska Izraelského utekl jsem.
\par 4 Opet rekl jemu David: Cože se stalo? Medle, povez mi. Kterýž odpovedel: To, že utekl lid z boje, a množství lidu padlo a zbito jest; též i Saul i Jonata syn jeho zbiti jsou.
\par 5 Rekl ješte David mládenci, kterýž mu to oznámil: Kterak ty víš, že umrel Saul i Jonata syn jeho?
\par 6 Odpovedel mládenec, kterýž to oznamoval jemu: Náhodou prišel jsem na horu Gelboe, a aj, Saul nalehl byl na kopí své, a vozové i jezdci postihali ho.
\par 7 Kterýžto ohlédna se zpátkem, uzrel mne a zavolal na mne. I rekl jsem: Aj, ted jsem.
\par 8 Tedy rekl mi: Kdo jsi ty? Odpovedel jsem jemu: Amalechitský jsem.
\par 9 I rekl mi: Pristup medle sem a zabí mne, nebo mne obklícila úzkost, a ješte všecka duše má jest ve mne.
\par 10 Protož stoje nad ním, zabil jsem ho, nebo jsem vedel, že nebude živ po svém pádu. A vzal jsem korunu, kteráž byla na hlave jeho, i záponu, kteráž byla na rameni jeho, a ted jsem to prinesl ku pánu svému.
\par 11 Tedy David uchytiv roucho své, roztrhl je; tolikéž i všickni muži, kteríž s ním byli.
\par 12 A nesouce smutek, plakali a postili se až do vecera pro Saule a pro Jonatu syna jeho, i pro lid Hospodinuv a pro dum Izraelský, že padli od mece.
\par 13 Rekl pak David mládenci, kterýž mu to oznámil: Odkud jsi ty? Odpovedel: Syn muže príchozího Amalechitského jsem.
\par 14 Opet mu rekl David: Kterak jsi smel vztáhnouti ruku svou, abys zahubil pomazaného Hospodinova?
\par 15 A zavolav David jednoho z mládencu, rekl jemu: Pristoupe, obor se na nej. Kterýžto uderil ho, tak že umrel.
\par 16 I rekl jemu David: Krev tvá budiž na hlavu tvou, nebot jsou ústa tvá svedcila na tebe, rkouce: Já jsem zabil pomazaného Hospodinova.
\par 17 Tedy naríkal David naríkáním tímto nad Saulem a nad Jonatou synem jeho,
\par 18 (Prikázav však, aby synové Judovi uceni byli stríleti z luku, jakož psáno v knize Uprímého.):
\par 19 Ó kráso Izraelská, na výsostech tvých zraneni, jakt jsou padli udatní!
\par 20 Neoznamujtež v Gát, ani toho ohlašujte na ulicích Aškalon, aby se neveselily dcery Filistinských, a neplésaly dcery neobrezaných.
\par 21 Ó hory Gelboe, ani rosa, ani déšt nespadej na vás, ani tu bud pole úrodné; nebo tam jest povržen štít udatných, štít Sauluv, jako by nebyl pomazán olejem.
\par 22 Od krve ranených a od tuku udatných lucište Jonatovo nikdy zpet neodskocilo, a mec Sauluv nenavracoval se prázdný.
\par 23 Saul a Jonata milí a utešení v živote svém, také pri smrti své nejsou rozlouceni. Nad orlice bystrejší, nad lvy silnejší byli.
\par 24 Dcery Izraelské, placte Saule, kterýž vás odíval cervcem dvakrát barveným rozkošne, kterýž dával ozdoby zlaté na roucha vaše.
\par 25 Ach, jakt jsou padli udatní u prostred boje? Jonata na výsostech tvých zabit jest.
\par 26 Velice jsem po tobe teskliv, bratre muj Jonato. Byl jsi mi príjemný náramne; vzácnejší u mne byla milost tvá nežli milost žen.
\par 27 Ach, jakt jsou padli udatní, a zahynula odení válecná.

\chapter{2}

\par 1 I stalo se potom, že se David tázal Hospodina, rka: Mám-li jíti do nekterého mesta Judského? Jemuž odpovedel Hospodin: Jdi. I rekl David: Kam mám jíti? Odpovedel: Do Hebronu.
\par 2 A protož bral se tam David, ano i obe manželky jeho, Achinoam Jezreelská, a Abigail žena nekdy Nábale Karmelského.
\par 3 Muže také své, kteríž s ním byli, pojal David, jednoho každého s celedí jeho, a bydlili v mestech Hebronských.
\par 4 I prišli muži Judští, a pomazali tam Davida za krále nad domem Judským. Oznámili také Davidovi, rkouce: Muži Jábes Galád, oni pochovali Saule.
\par 5 Tedy poslav David posly k mužum Jábes Galád, rekl jim: Požehnaní jste vy pred Hospodinem, že jste ucinili to milosrdenství pánu svému Saulovi, pochovavše ho.
\par 6 Protož nyní uciniž s vámi Hospodin milosrdenství a pravdu; ano i ját s vámi uciním milost, kteríž jste to ucinili.
\par 7 A tak tedy posilntež rukou svých a budtež statecní; nebo ac umrel pán váš Saul, však již mne pomazali dum Juduv za krále nad sebou.
\par 8 Abner pak syn Neruv, hejtman vojska Saulova, vzal Izbozeta syna Saulova a uvedl ho do Mahanaim.
\par 9 A ustavil ho králem nad Galád a nad Assur, a nad Jezreel, a nad Efraimem, a nad Beniaminem, i nade vším Izraelem.
\par 10 Ve ctyridcíti letech byl Izbozet syn Sauluv, když pocal kralovati nad Izraelem, a kraloval dve léte. (Toliko dum Juduv prídržel se Davida.
\par 11 A byl pocet dnu, v nichž byl David králem v Hebronu nad domem Judovým, sedm let a šest mesícu.)
\par 12 Potom vytáhl Abner syn Neruv, a služebníci Izbozeta syna Saulova z Mahanaim do Gabaon.
\par 13 Joáb také syn Sarvie a služebníci Davidovi vytáhše, potkali se s nimi práve u rybníka Gabaon. I pozustali tito u rybníka s strany jedné, oni pak u rybníka s druhé strany.
\par 14 Tedy rekl Abner Joábovi: Necht vystoupí nyní mládenci a pohrají pred námi. I rekl Joáb: Necht vystoupí.
\par 15 A tak vystoupili a vyšli v rovném poctu, dvanácte z Beniamina, z strany Izbozeta syna Saulova, a dvanácte z služebníku Davidových.
\par 16 Kterížto ujavše jeden každý za hlavu bližního svého, vrazil mec svuj v bok tovaryše svého, i padli spolu. Protož nazváno jest místo to Helkat Hassurim, a jest v Gabaon.
\par 17 I byla bitva velmi veliká v ten den, a poražen jest Abner i muži Izraelští od služebníku Davidových.
\par 18 Byli tu také tri synové Sarvie: Joáb, Abizai a Azael. Azael pak byl cerstvý na nohy své jako srna v poli.
\par 19 I honil Azael Abnera, a neuhnul se na pravo ani na levo, beže za Abnerem.
\par 20 Ohlédl se pak Abner zpátkem a rekl: Ty-li jsi Azael? Odpovedel: Jsem.
\par 21 Tedy rekl mu Abner: Uchyl se na pravo aneb na levo, a jmi sobe jednoho z mládencu tech, a vezmi sobe koristi jeho. Ale nechtel Azael uchýliti se od neho.
\par 22 Ješte znovu Abner rekl Azaelovi: Uchyl se ode mne, sic jinác prirazím te až k zemi, a jak bych smel pohledeti na Joába bratra tvého?
\par 23 Když pak nechtel ustoupiti, uhodil ho Abner kopím pod páté žebro, tak že vyniklo kopí hrbetem jeho; a padl tu na tom míste, na kterémž i umrel. A kdožkoli pricházeli k místu, na nemž padl Azael a umrel, zastavovali se.
\par 24 Ale Joáb a Abizai honili Abnera. Slunce pak již bylo zapadlo, když oni prišli ku pahrbku Amma, jenž jest naproti Giach, cestou k poušti Gabaon.
\par 25 Tedy sešli se synové Beniamin za Abnerem, a jsouce spolu v houfu, postavili se na vrchu pahrbku jednoho.
\par 26 Odkudž zavolal Abner na Joába, rka: Zdaliž bez prestání sžírati bude mec tvuj? Nevíš-liž, že horkost bývá naposledy? Dokudž tedy nerozkážeš lidu navrátiti se od honení bratrí svých?
\par 27 I rekl Joáb: Živt jest Buh, že kdybys byl nemluvil, hned ráno byl by odšel lid, jeden každý nechaje honení bratra svého.
\par 28 Tedy zatroubil Joáb v troubu, a zastavil se všecken lid, a nehonili více Izraele, aniž více bojovali.
\par 29 A tak Abner i lid jeho šli pres pole celou tu noc, a prepravili se pres Jordán, a prošedše všecku Betoron, prišli do Mahanaim.
\par 30 Ale Joáb navrátiv se od honení Abnera, shromáždil všecken lid, a nedostávalo se z služebníku Davidových devatenácti mužu a Azaele.
\par 31 Služebníci pak Davidovi zbili z Beniaminských a z mužu Abnerových tri sta a šedesáte mužu, kteríž tu zahynuli.
\par 32 A vzavše Azaele, pohrbili jej v hrobe otce jeho, kterýž byl v Betléme. Potom šli celou tu noc Joáb a muži jeho; i rozednilo se, když pricházeli do Hebronu.

\chapter{3}

\par 1 I trvala dlouho válka mezi domem Saulovým a domem Davidovým. David pak cím dále tím více se silil, ale dum Sauluv cím dále tím se více umenšoval.
\par 2 I zrodili se Davidovi synové v Hebronu, z nichž prvorozený jeho byl Amnon z Achinoam Jezreelské;
\par 3 A druhý po nem Cheleab z Abigail, ženy nekdy Nábale Karmelského; tretí pak Absolon, syn z Maachy dcery Tolmai krále Gessur;
\par 4 A ctvrtý Adoniáš syn Haggit, a pátý Sefatiáš syn Abitál;
\par 5 Šestý pak Jetram z Egly manželky Davidovy. Ti se zrodili Davidovi v Hebronu.
\par 6 I stalo se, když byla válka mezi domem Saulovým a mezi domem Davidovým, a Abner statecne zastával domu Saulova,
\par 7 (Mel pak byl Saul ženinu, jejíž jméno bylo Rizpa, dcera Aja), že rekl Izbozet Abnerovi: I proc jsi všel k ženine otce mého?
\par 8 Tedy rozhnevav se Abner velmi pro slova Izbozetova, rekl: I zdali já jsem psí hlava, kterýž jsem proti Judovi dnes ucinil milosrdenství s domem Saule otce tvého, s bratrími jeho i príbuznými jeho, a nevydal jsem te v ruku Davidovu, a však vyhledáváš na mne dnes nepravosti té ženy.
\par 9 Toto ucin Buh Abnerovi a toto mu pridej, jestliže nedopomohu k tomu Davidovi, jakož jemu prisáhl Hospodin,
\par 10 Aby preneseno bylo království od domu Saulova, a upevnen byl trun Daviduv nad Izraelem i nad Judou, od Dan až do Bersabé.
\par 11 A nemohl k tomu nic více odpovedíti Abnerovi, proto že se ho bál.
\par 12 A tak poslal Abner posly k Davidovi na míste svém, rka: Cí jest zeme? A aby rekli: Ucin smlouvu se mnou, a aj, ruka má bude s tebou, abych obrátil k tobe všecken Izrael.
\par 13 Jemuž odpovedel: Dobre. Ját uciním s tebou smlouvu, a však jedné veci od tebe žádám, totiž, abys nevidel tvári mé, lec prvé dáš privésti Míkol dceru Saulovu, když bys chtel prijíti, abys videl tvár mou.
\par 14 I poslal David posly k Izbozetovi synu Saulovu, aby rekli: Dej mi ženu mou Míkol, kterouž jsem sobe zasnoubil ve stu obrízek Filistinských.
\par 15 Poslav tedy Izbozet, vzal ji od muže Faltiele syna Lais.
\par 16 Šel pak s ní muž její, a jda za ní až do Bahurim, plakal. Tedy rekl jemu Abner: Jdi, navrat se zase. I navrátil se.
\par 17 Potom Abner ucinil rec k starším Izraelským, rka: Predešle stáli jste o to, aby David byl králem nad vámi.
\par 18 Protož nyní vykonejtež to; nebot jest Hospodin mluvil o Davidovi, rka: Skrze ruku Davida služebníka svého vysvobodím lid svuj Izraelský z ruky Filistinských, a z ruky všech neprátel jeho.
\par 19 To též mluvil Abner i k Beniaminským. Potom odšel Abner, aby mluvil k Davidovi v Hebronu všecko, což se za dobré videlo Izraelovi a všemu domu Beniamin.
\par 20 Když tedy prišel Abner k Davidovi do Hebronu a s ním dvadceti mužu, ucinil David Abnerovi i mužum, kteríž s ním byli, hody.
\par 21 I rekl Abner Davidovi: Vstanu a pujdu, abych shromáždil ku pánu svému králi všecken lid Izraelský, kteríž vejdou s tebou v smlouvu, a budeš kralovati nade všemi, jakož toho žádá duše tvá. I propustil David Abnera, kterýž odšel v pokoji.
\par 22 A aj, služebníci Davidovi a Joáb vraceli se z vojny, koristi veliké s sebou nesouce. Ale Abnera již nebylo s Davidem v Hebronu, nebo byl ho propustil, a již odšel v pokoji.
\par 23 Joáb tedy i všecko vojsko, kteréž bylo s ním, prišli tam. I oznámili Joábovi, rkouce: Byl zde Abner syn Neruv u krále, ale propustil jej, a odšel v pokoji.
\par 24 Protož všed Joáb k králi, rekl: Co jsi ucinil? Aj, prišel byl Abner k tobe; proc jsi ho pustil, aby zase odšel?
\par 25 Znáš Abnera syna Nerova. Proto, aby podvedl tebe, prišel, a aby vyšpehoval vycházení tvé i vcházení tvé, a zvedel všecko, co ty ciníš.
\par 26 Tedy vyšed Joáb od Davida, poslal posly za Abnerem, kteríž ho privedli zase od cisterny Sírach, o cemž David nic nevedel.
\par 27 A když se navrátil Abner do Hebronu, uvedl ho Joáb do prostred brány, aby s ním mluvil tiše. I uderil ho v páté žebro, a umrel pro krev Azaele bratra jeho.
\par 28 To když potom David uslyšel, rekl: Cist jsem já i království mé pred Hospodinem až na veky od krve Abnera syna Nerova.
\par 29 Nechat prijde na hlavu Joábovu i na všecken dum otce jeho, a necht není prázden dum Joábuv toho, jenž by trpel tok, aneb malomocného a na hul se podpírajícího, aneb padajícího od mece a nemajícího chleba.
\par 30 A tak Joáb a Abizai bratr jeho zamordovali Abnera, proto že byl zabil Azaele bratra jejich v Gabaon v bitve.
\par 31 I rekl David Joábovi a ke všemu lidu, kterýž byl s ním: Roztrhnete roucha svá, a prepašte se pytli, a placte pred Abnerem. Král pak David šel za márami.
\par 32 A když pochovávali Abnera v Hebronu, pozdvih král hlasu svého, plakal nad hrobem Abnerovým, plakal také všecken lid.
\par 33 V tom naríkaje král pro Abnera, rekl: Tak-liž jest mel umríti Abner, jako umírá nejaký nicemný clovek?
\par 34 Ruce tvé nebyly svázány a nohy tvé medenými poutami nebyly sevríny, ale padl jsi jako ten, kdož padá od lidí nešlechetných. Tedy ješte více všecken lid plakal nad ním.
\par 35 Potom prišel všecken lid, a meli k tomu Davida, aby jedl chléb ješte záhy. Ale David prisáhl, rka: Toto at mi uciní Buh a toto pridá, jestliže prvé, než slunce zapadne, okusím chleba aneb cehokoli jiného.
\par 36 Což když poznal všecken lid, líbilo se jim to; a všecko, což cinil král, líbilo se všemu lidu.
\par 37 I poznal všecken lid a všecken Izrael v ten den, že nepošlo to od krále, aby zabili Abnera syna Nerova.
\par 38 Rekl pak král služebníkum svým: Nevíte-liž, že kníže, a veliké, padlo dnes v Izraeli?
\par 39 A já ješte nyní mdlý jsem, jakožto pomazaný král, muži pak tito, synové Sarvie, jsou mi nepovolní. Odplatiž Hospodin tomu, kdož zle ciní, vedlé zlosti jeho.

\chapter{4}

\par 1 A když uslyšel Izbozet syn Sauluv, že umrel Abner v Hebronu, zemdlely ruce jeho, a všecken lid Izrael byl predešen.
\par 2 Mel pak syn Sauluv dva muže hejtmany nad dráby, jméno jednoho Baana, a jméno druhého Rechab, synové Remmona Berotského z synu Beniamin; nebo i Berot pocítá se v Beniaminovi.
\par 3 Utekli pak byli Berotští do Gittaim, a byli tam pohostinu až do toho dne.
\par 4 Mel také byl Jonata syn Sauluv syna chromého na nohy, (nebo když byl v peti letech a prišla povest o Saulovi a Jonatovi z Jezreel, vzavši ho chuva jeho, utíkala, a když pospíchala utíkajici, on upadl a okulhavel), jehož jméno Mifibozet.
\par 5 A odšedše synové Remmona Berotského, Rechab a Baana, vešli o poledni do domu Izbozetova. On pak spal na lužku poledním.
\par 6 A aj, když vešli až do domu, jako by bráti meli obilé, ranili ho v páté žebro, Rechab a Baana bratr jeho, a utekli.
\par 7 Nebo když byli vešli do domu, a on spal na lužku svém v pokojíku svém, kdež léhal, probodli jej a zabili, a stavše hlavu jeho, vzali ji, a šli cestou po pustinách celou tu noc.
\par 8 I prinesli hlavu Izbozetovu k Davidovi do Hebronu a rekli králi: Aj, hlava Izbozeta syna Saulova, neprítele tvého, kterýž hledal bezživotí tvého. Hle, pomstil dnes Hospodin pána mého krále nad Saulem i semenem jeho.
\par 9 Tedy odpovídaje David Rechabovi a Baanovi bratru jeho, synum Remmona Berotského, rekl jim: Živt jest Hospodin, kterýž vysvobodil duši mou ze všech úzkostí,
\par 10 Kdyžt jsem toho, kterýž mi oznámil, rka: Aj, Saul zahynul, (ješto se jemu zdálo, že veselé noviny zvestuje), vzal a zabil jsem ho v Sicelechu, jemuž se zdálo, že ho budu darovati za poselství:
\par 11 Cím pak více lidi bezbožné, kteríž zamordovali muže spravedlivého v dome jeho na ložci jeho? A nyní, zdaliž nebudu vyhledávati krve jeho z ruky vaší, a nevyhladím vás z zeme?
\par 12 I rozkázal David služebníkum, aby je zbili. I zutínali jim ruce i nohy jejich, a povesili u rybníka pri Hebronu. Hlavu pak Izbozetovu vzavše, pochovali v hrobe Abnerove v Hebronu.

\chapter{5}

\par 1 Tehdy prišla všecka pokolení Izraelská k Davidovi do Hebronu, a mluvili, rkouce: Aj, my kost tvá a telo tvé jsme.
\par 2 A predešlých casu, když byl Saul králem nad námi, ty jsi vyvodil i zase privodil lid Izraelský. A nadto rekl Hospodin tobe: Ty pásti budeš lid muj Izraelský, a ty budeš vývoda nad Izraelem.
\par 3 Prišli také všickni starší Izraelští k králi do Hebronu, a ucinil s nimi král David smlouvu v Hebronu pred Hospodinem. I pomazali Davida za krále nad Izraelem.
\par 4 Ve tridcíti letech byl David, když pocal kralovati, a kraloval ctyridceti let.
\par 5 V Hebronu kraloval nad Judou sedm let a šest mesícu, a v Jeruzaléme kraloval tridceti a tri léta nade vším Izraelem a Judou.
\par 6 Táhl pak král s lidem svým k Jeruzalému proti Jebuzejskému, obyvateli té zeme; kterýž mluvil Davidovi, rka: Nevejdeš sem, lec odejmeš slepé a kulhavé, pravíce: Nevejdet sem David.
\par 7 A však vzal David hrad Sion, tot jest mesto Davidovo.
\par 8 Nebo rekl David v ten den: Kdož by koli porazil Jebuzea, a zlezl by žlaby jeho, a pobil by ty slepé i chromé, kteréž má v nenávisti duše Davidova, knížetem bude. Protož pravili: Slepý a kulhavý nevejde do toho domu.
\par 9 I bydlil David na tom hrade, a nazval jej mestem Davidovým; nebo vystavel je David vukol od Mello až do vnitrku.
\par 10 A tak David cím dále tím více prospíval a rostl, Hospodin zajisté, Buh zástupu, byl s ním.
\par 11 Poslal také Chíram král Tyrský posly k Davidovi a dríví cedrového, i tesare, kameníky a zedníky umelé, kteríž vystaveli dum Davidovi.
\par 12 I poznal David, že ho potvrdil Hospodin za krále nad Izraelem, a že zvýšil království jeho pro lid svuj Izraelský.
\par 13 Nabral pak sobe David ješte více ženin i žen z Jeruzaléma, když byl prišel z Hebronu, a naplodilo se Davidovi ješte více synu a dcer.
\par 14 A tato jsou jména tech, kteríž se jemu zrodili v Jeruzaléme: Sammua, Sobab, Nátan a Šalomoun;
\par 15 Též Ibchar a Elisua, a Nefeg a Jafia;
\par 16 A Elisama a Eliada a Elifelet.
\par 17 Uslyšavše pak Filistinští, že pomazali Davida za krále nad Izraelem, vytáhli všickni Filistinští hledati ho. Což když zvedel David, sstoupil na místo hrazené.
\par 18 Protož Filistinští pritáhše, položili se v údolí Refaim.
\par 19 Tedy tázal se David Hospodina, rka: Potáhnu-li proti Filistinským? Vydáš-li je v ruku mou? Odpovedel Hospodin Davidovi: Táhni, nebot vydám jistotne Filistinské v ruku tvou.
\par 20 I pritáhl David do Balperazim, a porazil je tam, a rekl: Protrhlt jest Hospodin neprátely mé prede mnou, jako vody protrhují brehy. Protož nazval jméno místa toho Balperazim.
\par 21 Nebo zanechali tu rytin svých, kteréž pobral David i muži jeho.
\par 22 Opet znovu vytáhli Filistinští, a rozprostreli se v údolí Refaim.
\par 23 I tázal se David Hospodina. Kterýž odpovedel: Netáhni, ale obejda je po zadu, teprv dotreš na ne naproti moruším.
\par 24 A když uslyšíš, že šustí vrchové moruší, hneš se také; nebo tehdáž vyjde Hospodin pred tebou, aby zbil vojska Filistinských.
\par 25 I ucinil David tak, jakž mu prikázal Hospodin, a porazil Filistinské od Gabaa, až kudy se jde do Gázer.

\chapter{6}

\par 1 Tedy sebral opet David výborného lidu z Izraele tridceti tisíc.
\par 2 A vstav, šel David i všecken lid, kterýž byl s ním, z Bála Judova, aby prinesli odtud truhlu Boží, pri kteréž se vzývá jméno, jméno Hospodina zástupu, sedícího nad cherubíny.
\par 3 I vstavili truhlu Boží na nový vuz, vzavše ji z domu Abinadabova, kterýž byl na pahrbku. Uza pak a Achio, synové Abinadabovi, spravovali ten vuz nový.
\par 4 A tak vzali ji z domu Abinadabova, kterýž byl na pahrbku, jdouce s truhlou Boží; Achio pak šel pred truhlou.
\par 5 Ale David i všecken dum Izraelský hrali pred Hospodinem na všelijaké nástroje z dríví cedrového, totiž na harfy, loutny, bubny, huslicky, a na cymbály.
\par 6 A když prišli k humnu Náchonovu, vztáhl ruku svou Uza k truhle Boží a pozdržel jí, nebo uchýlili se volové.
\par 7 Protož rozhneval se Hospodin na Uzu, a zabil ho Buh pro neprozretelnost; i umrel tu u truhly Boží.
\par 8 Tedy zkormoutil se David, proto že se Hospodin tak prísne oboril na Uzu. I nazváno to místo Peres Uza až do tohoto dne.
\par 9 A boje se David Hospodina v ten den, rekl: Kterakž má vjíti ke mne truhla Hospodinova?
\par 10 Procež David nechtel prenésti k sobe truhly Hospodinovy do mesta svého, ale zpusobil to, aby se obrátila do domu Obededoma Gittejského.
\par 11 I pobyla truhla Hospodinova v dome Obededoma Gittejského za tri mesíce, a požehnal Hospodin Obededomovi i všemu domu jeho.
\par 12 V tom povedíno jest králi Davidovi, že požehnal Hospodin domu Obededomovu i všemu, což má, pro truhlu Boží. Tedy odšed David, prenesl truhlu Boží z domu Obededomova do mesta Davidova s veselím.
\par 13 A když poodešli ti, kteríž nesli truhlu Hospodinovu, na šest kroku, obetoval voly a tucný dobytek.
\par 14 David pak poskakoval ze vší síly pred Hospodinem, a byl oblecen David v efod lnený.
\par 15 A tak David i všecken dum Izraelský provázeli truhlu Hospodinovu s plésáním a zvukem trouby.
\par 16 Stalo se pak, když truhla Hospodinova vcházela do mesta Davidova, že Míkol dcera Saulova vyhlídala z okna, a viduci krále Davida plésajícího a poskakujícího pred Hospodinem, pohrdla jím v srdci svém.
\par 17 A když prinesli truhlu Hospodinovu, postavili ji na míste jejím u prostred stanu, kterýž jí byl David rozbil; a obetoval David pred Hospodinem obeti zápalné i pokojné.
\par 18 Zatím když prestal David obetovati obetí zápalných a pokojných, dal požehnání lidu ve jménu Hospodina zástupu.
\par 19 Dal také všemu lidu a všemu množství Izraelskému, od muže až do ženy, jednomu každému jeden pecník chleba a kus masa, a vína láhvici jednu. I odšel všecken lid, jeden každý do domu svého.
\par 20 Potom navracoval se David, aby dal požehnání domu svému. I vyšla Míkol dcera Saulova vstríc Davidovi, a rekla: Jak slavný byl dnes král Izraelský, kterýž se odkrýval dnes pred devkami služebníku svých, tak jako se odkrývá jeden z lehkomyslných!
\par 21 I rekl David k Míkol: Pred Hospodinem, (kterýž mne vyvolil nad otce tvého a nad všecken dum jeho, prikázav mi, abych byl vývodou lidu Hospodinova, totiž Izraele), plésal jsem a plésati budu pred Hospodinem.
\par 22 Anobrž cím se ješte více opovrhu nežli tuto, a ponížím se u sebe sám, tím i u tech devek, o nichž jsi mluvila, i u tech, pravím, slavnejší budu.
\par 23 Protož Míkol dcera Saulova nemela žádného plodu až do dne smrti své.

\chapter{7}

\par 1 I stalo se, že když král sedel v dome svém, a Hospodin jemu dal odpocinutí vukol prede všemi neprátely jeho,
\par 2 Rekl král Nátanovi proroku: Pohled medle, já bydlím v dome cedrovém, truhla pak Boží prebývá mezi kortýnami.
\par 3 I dí Nátan králi: Cožkoli jest v srdci tvém, jdi, ucin, nebo Hospodin s tebou jest.
\par 4 Potom té noci stalo se slovo Hospodinovo k Nátanovi, rkoucí:
\par 5 Jdi a povez služebníku mému Davidovi: Takto praví Hospodin: Zdaliž ty mi staveti budeš dum, v kterémž bych prebýval?
\par 6 Ponevadž jsem nebydlil v dome od toho dne, jakž jsem vyvedl syny Izraelské z Egypta, až do dne tohoto, ale precházel jsem s stánkem a príbytkem sem i tam.
\par 7 Nadto kudyž jsem koli chodil se všechnemi syny Izraelskými, zdali jsem jedno slovo rekl kterému z soudcu Izraelských, kterémuž jsem prikázal pásti lid svuj Izraelský, rka: Proc jste mi neustaveli domu cedrového?
\par 8 Protož nyní toto díš služebníku mému Davidovi: Takto praví Hospodin zástupu: Ját jsem te vzal z ovcince od stáda, abys byl vývodou lidu mého Izraelského.
\par 9 A býval jsem s tebou všudy, kamž jsi koli se obracel; všecky také neprátely tvé vyhladil jsem pred tvárí tvou, a ucinil jsem tobe jméno veliké, jako jméno vyvýšených na zemi.
\par 10 Zpusobím tolikéž místo lidu svému Izraelskému, a vštípím jej, a bydliti bude v míste svém a nepohne se více, aniž ho budou více trápiti lidé nešlechetní jako prvé,
\par 11 Hned od toho dne, jakž jsem ustanovil soudce nad lidem svým Izraelským, až jsem ted dal odpocinutí tobe prede všemi neprátely tvými. Presto oznamujet Hospodin, že on sám tobe dum vzdelá.
\par 12 Když se vyplní dnové tvoji, a usneš s otci svými, vzbudím síme tvé po tobe, kteréž vyjde z života tvého, a utvrdím království jeho.
\par 13 Ont ustaví dum jménu mému, a já utvrdím trun království jeho až na veky.
\par 14 Já budu jemu otcem, a on mi bude synem. Kterýž když zhreší, trestati ho budu metlou lidskou a ranami synu lidských.
\par 15 Ale milosrdenství mé nebude odjato od neho, tak jako jsem je odjal od Saule, kteréhož jsem zahnal od tvári tvé.
\par 16 A tak utvrzen bude dum tvuj a království tvé až na veky pred tebou, a trun tvuj bude stálý až na veky.
\par 17 Vedlé všech slov techto, a podlé všeho videní tohoto, tak mluvil Nátan Davidovi.
\par 18 Tedy všed král David, posadil se pred Hospodinem a rekl: Kdo jsem já, Panovníce Hospodine, a jaký jest dum muj, že jsi mne privedl v ta místa.
\par 19 Nýbrž ještet se to málo videlo, Panovníce Hospodine, že jsi také mluvil i o domu služebníka svého na dlouhé casy, ješto jest to povaha lidská, Panovníce Hospodine.
\par 20 Ale což již více má mluviti David pred tebou? Však ty znáš služebníka svého, Panovníce Hospodine.
\par 21 Pro slovo své a podlé srdce svého ciníš všecky tyto veci veliké, v známost je uvode služebníku svému.
\par 22 (Protož zveleben jsi Hospodine Bože, nebo není tobe rovného, anobrž není žádného Boha krome tebe), podlé toho všeho, jakž jsme slýchali ušima svýma.
\par 23 Kdo tedy jest jako lid tvuj, jako Izrael, který národ na zemi, jehož by Buh šel, aby jej sobe vykoupil za lid, a dobyl sobe jména, a ucinil tobe, ó Izraeli, tyto veliké veci a hrozné v zemi tvé, pred oblícejem lidu svého, kterýž jsi sobe vysvobodil z Egypta, z pohanstva bohu jejich?
\par 24 A vzdelal jsi sobe lid svuj Izraelský, sobe v lid až na veky, a protož, Hospodine, jsi jejich Bohem.
\par 25 Nyní tedy, Hospodine Bože, slovo to, jímž jsi se zamluvil služebníku svému a domu jeho, naplniž je až na veky, a ucin tak, jakž jsi mluvil,
\par 26 Tak aby zvelebováno bylo jméno tvé až na veky, když se ríkati bude: Hospodin zástupu jest Buh nad Izraelem, a dum služebníka tvého Davida aby byl stálý pred oblícejem tvým.
\par 27 Nebo ty, Hospodine zástupu, Bože Izraelský, zjevil jsi mne služebníku svému, rka: Dum ustavím tobe. Protož usoudil služebník tvuj v srdci svém, aby se modlil tobe modlitbou touto.
\par 28 A tak již, Panovníce Hospodine, ty sám jsi Buh, a slova tvá jsou pravda, jimiž jsi zaslíbil služebníku svému dobré veci tyto.
\par 29 Již tedy rac požehnati domu služebníka svého, aby trval na veky pred oblícejem tvým. Nebo ty jsi, Hospodine Bože, mluvil, že požehnáním tvým požehnán bude dum služebníka tvého na veky.

\chapter{8}

\par 1 Stalo se potom, že porazil David Filistinské, a zemdlil je; i vzal David Meteg Amma z ruky Filistinských.
\par 2 Porazil také Moábské a zmeril je provazcem, na zemi je rozprostíraje; a odmeril jich dva provazce k zbití a celý provazec k živení. I ucineni jsou Moábští Davidovi služebníci, a dávali jemu plat.
\par 3 Porazil též David Hadadezera syna Rohobova, krále Soba, když byl vytáhl, aby rozšíril konciny své až k rece Eufraten.
\par 4 A pobral mu David tisíc vozu a sedm set jezdcu, a dvadceti tisíc mužu peších, a zpodrezoval David žily všechnem konum vozníkum; toliko zanechal z nich ke stu vozum.
\par 5 Pritáhli pak byli Syrští od Damašku na pomoc Hadadezerovi králi Soba, ale David porazil z Syrských dvamecítma tisíc mužu.
\par 6 Tedy osadil David stráží Syrii Damašskou, a byli Syrští služebníci Davidovi, dávajíce plat; nebo zachovával Hospodin Davida, kamžkoli se obrátil.
\par 7 Pobral také David štíty zlaté, kteréž meli služebníci Hadadezerovi, a prinesl je do Jeruzaléma.
\par 8 Z Betach též a z Berot, mest Hadadezerových, nabral král David velmi mnoho medi.
\par 9 A když uslyšel Tohi král Emat, že porazil David všecko vojsko Hadadezerovo,
\par 10 Poslal Tohi Jorama syna svého k králi Davidovi, aby ho pozdravil prátelsky, a spolu s ním se radoval z toho, že štastne bojoval s Hadadezerem, a porazil ho; (nebo Hadadezer vedl válku proti Tohi). Kterýžto prinesl s sebou nádoby stríbrné, též nádoby zlaté a nádoby medené.
\par 11 Ty také obetoval král David Hospodinu s stríbrem a zlatem posveceným ze všech národu, kteréž podmanil,
\par 12 Totiž z Syrských a Moábských, též z synu Ammon a z Filistinských, i z Amalechitských a z koristí Hadadezera syna Rohobova, krále Soba.
\par 13 Zvelebil také David své jméno, když se navracoval od pobití osmnácti tisícu Syrských v údolí solnatém.
\par 14 Protož i nad Idumejskými stráž postavil, všecku krajinu Idumejskou stráží osadiv. I ucineni jsou všickni Idumejští služebníci Davidovi; nebo zachovával Hospodin Davida, kamž se koli obrátil.
\par 15 I kraloval David nade vším Izraelem, a cinil David soud a spravedlnost všemu lidu svému.
\par 16 Joáb pak syn Sarvie byl nad vojskem, a Jozafat syn Achiluduv kanclérem;
\par 17 Sádoch také syn Achitobuv a Achimelech syn Abiataruv knežími, a Saraiáš písarem.
\par 18 Benaiáš pak syn Joiaduv byl nad Cheretejskými a Peletejskými, a synové Davidovi knížaty.

\chapter{9}

\par 1 Tedy rekl David: Jest-li ješte kdo pozustalý z domu Saulova, abych jemu ucinil milosrdenství pro Jonatu?
\par 2 Byl pak služebník domu Saulova, jehož jméno bylo Síba. I zavolali ho k Davidovi. I rekl král jemu: Ty-li jsi Síba? Odpovedel: Jsem služebník tvuj.
\par 3 Rekl jemu král: Jest-liž ješte kdo z domu Saulova, abych jemu ucinil milosrdenství Boží? Odpovedel Síba králi: Ješte jest syn Jonatuv chromý na nohy.
\par 4 I rekl jemu král: Kdež jest? Odpovedel Síba králi: Tam jest v dome Machiry, syna Amielova v Lodebar.
\par 5 Protož poslav král David, vzal ho z domu Machiry, syna Amielova z Lodebar.
\par 6 A když prišel Mifibozet syn Jonaty, syna Saulova, k Davidovi, padl na tvár svou a poklonil se. I rekl David: Mifibozete. Kterýž odpovedel: Aj, služebník tvuj.
\par 7 Rekl jemu David: Neboj se, nebo jiste uciním s tebou milosrdenství pro Jonatu otce tvého, a navrátím tobe všecka pole Saule otce tvého, a ty jídati budeš za stolem mým vždycky.
\par 8 Kterýž pokloniv se, rekl: Co jest služebník tvuj, že jsi se ohlédl na psa mrtvého, jakýž jsem já?
\par 9 Zatím povolal král Síby, služebníka Saulova a rekl jemu: Cožkoli mel Saul i všecka celed jeho, to jsem dal synu pána tvého.
\par 10 Budeš tedy jemu spravovati rolí, ty i synové tvoji i služebníci tvoji, a snášeti budeš, aby mel syn pána tvého pokrm, kterýž by jedl, ale Mifibozet syn pána tvého jídati bude vždycky za stolem mým. (Mel pak Síba patnácte synu a dvadceti služebníku.)
\par 11 Odpovedel Síba králi: Vedlé všeho, jakž prikázal pán muj král služebníku svému, tak uciní služebník tvuj, ackoli i Mifibozet mohl by jídati za stolem mým jako jeden z synu královských.
\par 12 Mel také Mifibozet syna malého, jemuž jméno bylo Mícha; a všickni, kteríž bydlili v dome Síbove, byli za služebníky Mifibozetovi.
\par 13 A tak Mifibozet zustával v Jeruzaléme, proto že vždycky za stolem královským jídal; a byl kulhavý na obe noze.

\chapter{10}

\par 1 Stalo se také potom, že umrel král Ammonitský, a kraloval Chanun syn jeho po nem.
\par 2 I rekl David: Uciním milosrdenství s Chanunem synem Náhasovým, jakož otec jeho ucinil milosrdenství nade mnou. Tedy poslal David, aby ho potešil skrze služebníky své pro otce jeho. I prišli služebníci Davidovi do zeme Ammonitských.
\par 3 I rekla knížata Ammonitská k Chanunovi, pánu svému: Cožt se zdá, že David ciní poctivost otci tvému, že poslal k tobe, kteríž by te potešili? Zdaliž ne proto, aby shlédl mesto a vyšpehoval je, a potom je podvrátil, poslal David služebníky své k tobe?
\par 4 A tak Chanun vzav služebníky Davidovy, oholil každému pul brady, a zustrihoval roucha jejich až do polovice, totiž až do zadku jejich, a propustil je.
\par 5 To když oznámili Davidovi, poslal proti nim, (nebo muži ti zohaveni byli velice), a rekl jim král: Pobudte v Jerichu, dokudž neobrostou brady vaše, potom se navrátíte.
\par 6 Vidouce pak Ammonitští, že se zošklivili Davidovi, poslavše, najali ze mzdy z Syrie z domu Rohob, a z Syrie Soba dvadcet tisícu peších, a od krále Maacha tisíc mužu, a od Istoba dvanácte tisíc mužu.
\par 7 Což uslyšev David, poslal Joába se vším vojskem udatných.
\par 8 A tak vytáhše Ammonitští, šikovali se k boji u brány, Syrští také, Soba a Rohob, a Istob i Maacha zvlášte byli na poli.
\par 9 A protož vida Joáb proti sobe sšikovaný boj s predu i s zadu, vybrav nekteré ze všech výborných Izraelských, sšikoval je také proti Syrským.
\par 10 Ostatek pak lidu dal pod správu Abizai bratra svého, a sšikoval jej proti Ammonitským.
\par 11 A rekl: Jestliže Syrští budou silnejší mne, prispeješ mi na pomoc; jestliže pak Ammonitští silnejší budou tebe, také prispeji, abych pomohl tobe.
\par 12 Posiliž se, a budme udatní, bojujíce za lid náš a za mesta Boha našeho, Hospodin pak uciní, což se jemu dobre líbiti bude.
\par 13 I pristoupil Joáb s lidem svým k bitve proti Syrským, a oni utekli pred ním.
\par 14 Tedy Ammonitští vidouce, že utíkají Syrští, utekli i oni pred Abizai a vešli do mesta. I navrátil se Joáb od Ammonitských a prišel do Jeruzaléma.
\par 15 Ale Syrští vidouce, že by poraženi byli od Izraele, sebrali se vespolek.
\par 16 Poslal také Hadarezer a vyvedl Syrské, kteríž bydlejí za rekou, a prišli k Helam; Sobach pak, hejtman vojska Hadarezerova, vedl je.
\par 17 I oznámeno to Davidovi. Kterýžto shromáždiv všecken lid Izraelský, prepravil se pres Jordán, a pritáhli k Helam. I sšikovali se Syrští proti Davidovi, a bojovali proti nemu.
\par 18 Tedy utekli Syrští pred Izraelem, a porazil David z Syrských sedm set vozu a ctyridceti tisíc jezdcu. Sobacha také hejtmana vojska toho ranil, i umrel tu.
\par 19 Když pak videli všickni králové, kteríž byli pri Hadarezerovi, že jsou poraženi od Izraele, vešli v pokoj s Izraelem a sloužili jemu. A nesmeli již více Syrští táhnouti na pomoc Ammonitským.

\chapter{11}

\par 1 I stalo se po roce, když králové vyjíždívají na vojnu, že poslal David Joába a služebníky své s ním, též i všecken Izrael, aby hubili Ammonitské; i oblehli Rabba. David pak zustal v Jeruzaléme.
\par 2 Tedy stalo se k vecerou, když vstal David s ložce svého, procházeje se po paláci domu královského, že uzrel s paláce ženu, ana se myje; kterážto žena byla vzezrení velmi krásného.
\par 3 A poslav David, vyptal se o té žene a rekl: Není-liž to Betsabé dcera Eliamova, manželka Uriáše Hetejského?
\par 4 Opet poslal David posly a vzal ji. Kterážto když vešla k nemu, obýval s ní; (nebo se byla ocistila od necistoty své). Potom navrátila se do domu svého.
\par 5 I pocala žena ta. Protož poslavši, oznámila Davidovi, rkuci: Tehotná jsem.
\par 6 Poslav pak David, vzkázal Joábovi: Pošli ke mne Uriáše Hetejského. I poslal Joáb Uriáše k Davidovi.
\par 7 A když prišel Uriáš k nemu, otázal se ho, jak se má Joáb, a jak se má lid, a kterak se vede v vojšte?
\par 8 Rekl také David Uriášovi: Jdiž do domu svého, a umyj sobe nohy. I vyšel Uriáš z domu králova, a poslán za ním dar královský.
\par 9 Ale Uriáš spal u vrat domu královského se všemi služebníky pána svého, a nešel do domu svého.
\par 10 Což když oznámili Davidovi, rkouce: Nešel Uriáš do domu svého, rekl David Uriášovi: Však jsi z cesty prišel. Procež jsi nešel do domu svého?
\par 11 Odpovedel Uriáš Davidovi: Truhla Boží a Izrael i Juda zustávají v staních, a pán muj Joáb i služebníci pána mého na poli zustávají, já pak mám vjíti do domu svého, abych jedl a pil, a spal s manželkou svou? Jakož jsi živ ty, a jako jest živa duše tvá, žet toho neuciním.
\par 12 Tedy rekl David Uriášovi: Pobud zde ješte dnes, a zítra propustím tebe. I zustal Uriáš v Jeruzaléme toho dne i nazejtrí.
\par 13 V tom pozval ho David, aby jedl pred ním a pil, a opojil ho. A však on šel vecer spáti na ložce své s služebníky pána svého, a nešel do domu svého.
\par 14 A když bylo ráno, napsal David list Joábovi, kterýž poslal po Uriášovi.
\par 15 Psal pak v listu temi slovy: Postavte Uriáše proti bojovníkum nejsilnejším; mezi tím odstupte nazpet od neho, aby ranen jsa, umrel.
\par 16 I stalo se, když oblehl Joáb mesto, že postavil Uriáše proti místu, kdež vedel, že jsou nejsilnejší muži.
\par 17 A vyskocivše muži z mesta, bojovali proti Joábovi. I padli nekterí z lidu, z služebníku Davidových, také i Uriáš Hetejský zabit.
\par 18 Tedy poslav Joáb, oznámil Davidovi všecko, což se zbehlo v bitve.
\par 19 A prikázal tomu poslu, rka: Když vypravíš králi všecky veci, které se zbehly v bitve,
\par 20 Pri cemž jestliže se popudí prchlivost královská a rekne-lit: Proc jste pristoupili k mestu, bojujíce? Zdaliž jste nevedeli, že házejí se zdi?
\par 21 Kdo zabil Abimelecha syna Jerobosetova? Zdali ne žena, svrhši na nej kus žernovu se zdi, tak že umrel v Tébes? Proc jste pristupovali ke zdi? tedy díš: Také služebník tvuj Uriáš Hetejský umrel.
\par 22 A tak šel posel, a prišed, oznámil Davidovi všecko to, procež ho byl poslal Joáb.
\par 23 A rekl posel ten Davidovi: Zmocnili se nás muži ti, a vytrhli proti nám do pole, a honili jsem je až k bráne.
\par 24 V tom stríleli strelci na služebníky tvé se zdi, a zbiti jsou nekterí z služebníku králových, také i služebník tvuj Uriáš Hetejský zabit.
\par 25 Tedy rekl David poslu: Tak povez Joábovi: Nic te to nerozpakuj, že tak i jinak sehlcuje mec. Ztuž boj svuj proti mestu a zkaz je, a poteš drábu.
\par 26 Uslyševši pak manželka Uriášova, že umrel Uriáš muž její, plakala manžela svého.
\par 27 A když pominul plác, poslav David, vzal ji do domu svého, a mel ji za manželku; i porodila mu syna. Ale nelíbilo se to Hospodinu, co ucinil David.

\chapter{12}

\par 1 Protož poslal Hospodin Nátana k Davidovi. Kterýž prišed k nemu, rekl jemu: Dva muži byli v meste jednom, jeden bohatý a druhý chudý.
\par 2 Bohatý mel ovec i volu velmi mnoho,
\par 3 Chudý pak nemel nic, krome jednu ovecku malou, kterouž byl koupil a choval, až i odrostla pri nem a pri detech jeho tolikéž. Chléb jeho jedla a z cíše jeho pila, a v lunu jeho spávala, a tak byla mu jako za dceru.
\par 4 Když pak prišel pocestný k tomu bohatému cloveku, líto mu bylo vzíti z ovcí aneb z volu svých, což by pripravil pocestnému, kterýž k nemu prišel, ale vzav ovecku toho chudého cloveka, pripravil ji muži, kterýž byl prišel k nemu.
\par 5 Tedy rozhneval se David na muže toho náramne a rekl Nátanovi: Živt jest Hospodin, že hoden jest smrti muž, kterýž to ucinil.
\par 6 Ovecku také tu zaplatí ctvernásobne, proto že ucinil vec tu, a nelitoval ho.
\par 7 I rekl Nátan Davidovi: Ty jsi ten muž! Takto praví Hospodin Buh Izraelský: Já jsem te pomazal, abys králem byl nad Izraelem, a vytrhl jsem tebe z ruky Saulovy.
\par 8 A dal jsem tobe dum pána tvého, i ženy pána tvého v luno tvé, dalt jsem také dum Izraelský a Judský, a bylo-lit by to málo, byl bych pridal mnohem více.
\par 9 Procež jsi sobe zlehcil slovo Hospodinovo, cine to, což se nelíbí jemu? Uriáše Hetejského zabil jsi mecem, a manželku jeho pojals sobe za ženu, samého pak zamordoval jsi mecem Ammonitských.
\par 10 Protož nyní neodejdet mec z domu tvého až na veky, proto že jsi pohrdl mnou, a vzal jsi manželku Uriáše Hetejského, aby byla žena tvá.
\par 11 Takto praví Hospodin: Aj, já vzbudím proti tobe zlé z domu tvého, a vezma ženy tvé pred ocima tvýma, dám je bližnímu tvému, kterýž spáti bude s ženami tvými, an na to každý hledí.
\par 12 A ackoli ty ucinil jsi to tajne, já však uciním to zjevne prede vším Izraelem a všechnem známe.
\par 13 Tedy rekl David Nátanovi: Zhrešilt jsem Hospodinu. Zase rekl Nátan Davidovi: Tentýž Hospodin prenesl hrích tvuj, neumreš.
\par 14 Ale však, ponevadž jsi tou vecí dal prícinu neprátelum Hospodinovým, aby se rouhali, protož ten syn, kterýž se narodil tobe, jistotne umre.
\par 15 Potom odšel Nátan do domu svého. V tom Hospodin ranil díte, kteréž byla porodila manželka Uriášova Davidovi, tak že pochybili o nem.
\par 16 I modlil se David Bohu za to díte a postil se, a všed, ležel pres noc na zemi.
\par 17 A ackoli starší domu jeho prišli k nemu, aby ho zdvihli z zeme, on však nechtel, aniž s nimi co jedl.
\par 18 Stalo se pak dne sedmého, že umrelo díte. I nesmeli služebníci Davidovi toho oznámiti jemu, že by umrelo díte; nebo pravili: Aj, když ješte bylo díte živo, mluvili jsme jemu, a nechtel slyšeti hlasu našeho, což pak když jemu díme: Umrelo díte. Ovšem to tíže ponese.
\par 19 Ale vida David, an služebníci jeho k sobe šepcí, srozumel, že by umrelo díte. I rekl David služebníkum svým: Co již umrelo díte? Kteríž rekli: Umrelo.
\par 20 Tedy vstav David s zeme, umyl se a pomazal se, a zmenil roucho své, a všed do domu Hospodinova, pomodlil se. Potom vrátiv se do domu svého, rozkázal sobe dáti jísti a jedl.
\par 21 Služebníci pak jeho rekli jemu: Co jsi to ucinil? Pro díte, když živo bylo, postils se a plakal, ale jakž díte umrelo, vstal jsi a prijal jsi pokrm.
\par 22 Kterýžto rekl: Pokudž ješte díte živo bylo, postil jsem se a plakal, nebo jsem rekl: Kdo ví? Mužet se smilovati nade mnou Hospodin, aby živo bylo díte.
\par 23 Nyní pak, ponevadž umrelo, proc se mám postiti? Zdaliž budu moci zase je vzkrísiti? Ját pujdu k nemu, ale ono se nenavrátí ke mne.
\par 24 Potom David potešiv Betsabé manželky své, všel k ní a spal s ní. I porodila syna, a nazval jméno jeho Šalamoun, a Hospodin miloval jej.
\par 25 Protož poslal byl Nátana proroka, a nazval jméno jeho Jedidiah pro Hospodina.
\par 26 Bojoval pak Joáb proti Rabba Ammonitských, a vzal mesto královské.
\par 27 A poslav posly k Davidovi, rekl: Dobýval jsem Rabba a vzal jsem mesto vod.
\par 28 Protož nyní sebera ostatek lidu, polož se proti mestu a vezmi je, abych já nevzal mesta toho, a bylo by mne pripisováno vítezství nad ním.
\par 29 Tedy sebrav David všecken lid, vytáhl proti Rabba, kteréhož dobýval a vzal je.
\par 30 A snal korunu krále jejich s hlavy jeho, kteráž vážila centnér zlata, a bylo v ní kamení drahé, a vstavena byla na hlavu Davidovu. Vyvezl též koristi mesta velmi veliké.
\par 31 Lid pak, kterýž v nem byl, vyvedl a dal jej pod pily a pod brány železné a pod sekery železné, a vehnal je do peci cihelné. A tak cinil všechnem mestum Ammonitským. Potom navrátil se David se vším lidem do Jeruzaléma.

\chapter{13}

\par 1 I stalo se potom, že Absolon syn Daviduv mel sestru krásnou, jménem Támar; i zamiloval ji Amnon syn Daviduv.
\par 2 A tak se o to trápil Amnon, že i v nemoc upadl pro Támar sestru svou; nebo panna byla, a videl Amnon, že nesnadne bude jí moci co uciniti.
\par 3 Mel pak Amnon prítele, jehož jméno bylo Jonadab, syn Semmaa bratra Davidova, kterýžto Jonadab byl muž velmi chytrý.
\par 4 I rekl jemu: Proc tak chradneš, synu králuv, den ode dne? Neoznámíš-liž mi? I rekl mu Amnon: Támar sestru Absolona bratra svého miluji.
\par 5 Tedy rekl jemu Jonadab: Polož se na luže své a udelej se nemocným, a když prijde otec tvuj, aby te navštívil, díš jemu: Necht prijde, prosím, Támar sestra má a dá mi jísti, pripravíc pred ocima mýma pokrm, abych videl a jedl z ruky její.
\par 6 A tak složil se Amnon, delaje se nemocným. A když prišel král, aby ho navštívil, rekl Amnon králi: Necht prijde, prosím, Támar sestra má a pripraví pred ocima mýma asi dve krmicky, abych pojedl z ruky její.
\par 7 Protož poslal David k Támar do domu, rka: Jdi hned do domu Amnona bratra svého a priprav mu krmicku.
\par 8 I šla Támar do domu Amnona bratra svého; on pak ležel. A vzavši mouky, zadelala ji, a pripravivši krmicku pred ocima jeho, uvarila ji.
\par 9 Potom vzavši pánvici, vyložila pred nej, a on nechtel jísti. (I rekl Amnon: Spravte to, at vyjdou všickni ven. I vyšli od neho všickni.
\par 10 Rekl pak Amnon k Támar: Prines tu krmicku do pokojíka, abych pojedl z ruky tvé. A vzavši Támar krmicku, kterouž pripravila, prinesla ji pred Amnona bratra svého do pokojíka.)
\par 11 Ale když mu podávala, aby jedl, uchopil ji a rekl jí: Pod, lež se mnou, sestro má.
\par 12 Kterážto rekla jemu: Nikoli, bratre muj, necin mi násilí, nebo ne tak se má díti v Izraeli. Neprovod nešlechetnosti této.
\par 13 Nebo já na koho svedu pohanení své? Ty pak budeš jako jeden z nejnešlechetnejších v Izraeli. Radeji tedy mluv medle s králem, nebo neodepret mne tobe.
\par 14 Ale nechtel uposlechnouti hlasu jejího, nýbrž zmocniv se jí, ucinil jí násilí a ležel s ní.
\par 15 Potom vzal ji Amnon v nenávist velikou velmi, tak že vetší byla nenávist, kterouž nenávidel jí, než milost, kterouž ji miloval. I rekl jí Amnon: Vstan a jdi pryc.
\par 16 Kterážto odpovedela jemu: Za prícinou prevelmi zlé veci té, kterouž jsi pri mne spáchal, druhé horší se dopouštíš, že mne vyháníš. On pak nechtel jí slyšeti.
\par 17 Ale zavolav mládence, kterýž mu prisluhoval, rekl: Vyved hned tuto ode mne ven, a zamkni dvére po ní.
\par 18 (Mela pak na sobe sukni promenných barev, nebo v takových sukních chodívaly dcery královské panny.) A tak vyvedl ji ven služebník jeho a zamkl dvére po ní.
\par 19 Tedy posypala Támar hlavu svou popelem, a sukni promenných barev, kterouž mela na sobe, roztrhla; a vložila ruku na hlavu svou, a jduc, kricela s naríkáním.
\par 20 I rekl jí Absolon bratr její: Nebyl-liž Amnon bratr tvuj s tebou? Ale nyní, sestro má mlc; bratr tvuj jest, nepripouštej toho k srdci. A tak zustala Támar, jsuc opuštená, v dome Absolona bratra svého.
\par 21 A uslyšev král David o tech všech vecech, rozhneval se náramne.
\par 22 Absolon pak nic nemluvil s Amnonem, ani dobrého ani zlého; nebo nenávidel Absolon Amnona, proto že ucinil násilí Támar sestre jeho.
\par 23 I stalo se po celých dvou letech, když strihli ovce Absolonovi v Balazor, jenž jest podlé Efraim, že pozval Absolon všech synu královských.
\par 24 Nebo prišel Absolon k králi a rekl: Aj, nyní služebník tvuj má strižce; necht, prosím, jde král a služebníci jeho s služebníkem tvým.
\par 25 I rekl král Absolonovi: Necht, synu muj, necht nyní nechodíme všickni, abychom te neobtežovali. A ackoli nutkal ho, však nechtel jíti, ale požehnal mu.
\par 26 Rekl ješte Absolon: Necht aspon s námi jde, prosím, Amnon bratr muj. Odpovedel jemu král: Proc by s tebou šel?
\par 27 A když vždy dotíral Absolon, poslal s ním Amnona i všecky syny královské.
\par 28 Prikázal pak byl Absolon služebníkum svým, rka: Šetrte medle, když se rozveselí srdce Amnonovo vínem, a reknu vám: Bíte Amnona, tedy zabíte jej. Nebojte se nic, nebo zdaliž jsem já nerozkázal vám? Posilnte se a mejte se zmužile.
\par 29 I ucinili Amnonovi služebníci Absolonovi, jakž jim byl prikázal Absolon. Procež vstavše všickni synové královi, vsedli jeden každý na mezka svého a utekli.
\par 30 V tom když ješte byli na ceste, prišla taková povest k Davidovi: Pobil Absolon všecky syny královské, tak že z nich ani jednoho nezustalo.
\par 31 Tedy vstav král, roztrhl roucha svá a ležel na zemi; všickni také služebníci jeho stáli, roztrhše roucha.
\par 32 Ozval se pak Jonadab, syn Semmaa bratra Davidova, a rekl: Nepraviž toho, pane muj, jako by všecky mládence syny královy zbili, ale Amnon toliko zabit; nebo tak v úmysle Absolonove složeno bylo od toho dne, jakž on byl ucinil násilí Támar sestre jeho.
\par 33 Protož nechat nepripouští toho nyní pán muj král k srdci svému, mysle, že by všickni synové královi zbiti byli; nebo Amnon toliko umrel.
\par 34 Absolon pak utekl. Tedy pozdvih služebník hlásný ocí svých, uzrel, an mnoho lidu jde odtud, kudyž se chodilo k nemu cestou pod horami.
\par 35 I rekl Jonadab králi: Aj, synové královští prijíždejí. Vedlé reci služebníka tvého tak se stalo.
\par 36 A když prestal mluviti, aj, synové královští prišli, a pozdvihše hlasu svého, plakali; též také král i všickni služebníci jeho plakali plácem velmi velikým.
\par 37 Absolon pak utekl a ušel k Tolmai synu Amiudovu, králi Gessur. I plakal David syna svého po všecky ty dny.
\par 38 Absolon tedy utíkaje, prišel do Gessur, a byl tam tri léta.
\par 39 Potom žádal David vyjíti k Absolonovi, nebo již byl oželel smrti Amnonovy.

\chapter{14}

\par 1 Srozumev pak Joáb syn Sarvie, že by naklonilo se srdce královo k Absolonovi,
\par 2 Poslav do Tekoa, a povolav odtud ženy moudré, rekl jí: Medle, udelej se, jako bys zámutek mela, a oblec se, prosím, v roucho smutku, a nepomazuj se olejem, ale bud jako žena již za mnoho dní zámutek mající nad mrtvým.
\par 3 I pujdeš k králi a mluviti mu budeš vedlé reci této. A naucil ji Joáb, co by mela mluviti.
\par 4 Protož mluvila žena ta Tekoitská králi, padši na tvár svou k zemi, a poklonu ucinivši, rekla: Spomoz, ó králi.
\par 5 I rekl jí král: Což jest tobe? Odpovedela ona: Zajisté žena vdova jsem, a umrel mi muž muj.
\par 6 Mela pak služebnice tvá dva syny, kteríž svadili se spolu na poli, a když nebyl, kdo by je rozvadil, uderil jeden druhého a zabil ho.
\par 7 Aj, ted povstala všecka rodina proti služebnici tvé, a rekli: Vydej toho, jenž zabil bratra svého, at ho zabijeme za život bratra jeho, kteréhož zamordoval, nýbrž zahubíme i dedice. A tak uhasí jiskru mou, kteráž pozustala, aby nezanechali muži mému jména a ostatku na zemi.
\par 8 Tedy rekl král žene: Navrat se do domu svého, a ját porucím o tobe.
\par 9 I odpovedela žena Tekoitská králi: Necht jest, pane muj králi, na mne ta nepravost, a na dum otce mého, král pak a stolice jeho at jest bez viny.
\par 10 Rekl také král: Bude-li kdo mluviti co proti tobe, prived ho ke mne, a nedotknet se tebe více.
\par 11 Tedy ona rekla: Rozpomen se, prosím, králi, na Hospodina Boha svého, aby se nerozmnožili mstitelé krve k zhoube a nezahladili syna mého. I odpovedel: Živt jest Hospodin, žet nespadne vlas syna tvého na zemi.
\par 12 K tomu rekla žena: Necht promluví, prosím, služebnice tvá pánu svému králi slovo. Kterýžto odpovedel: Mluv.
\par 13 I rekla žena: Proc jsi tedy myslil podobnou vec proti lidu Božímu? Nebo mluví král rec tuto jako ten, kterýž sebe vinného ciní, ponevadž nechce zase povolati vyhnaného svého.
\par 14 Všicknit jsme zajisté nepochybne smrtelní a jako vody, kteréž rozlity jsouce po zemi, zase sebrány býti nemohou, aniž se Buh na necí osobu ohlédá; ovšemt i myšlení svá vynesl, aby vyhnaného nevyhánel od sebe.
\par 15 Nyní pak, že jsem prišla ku pánu svému králi mluviti reci tyto, prícina jest, že mne strašil lid. Protož rekla služebnice tvá: Budu nyní mluviti králi, snad naplní král žádost služebnice své.
\par 16 Nebot vyslyší král a vysvobodí služebnici svou z ruky muže, chtejícího vyhladiti mne i syna mého spolu z dedictví Božího.
\par 17 Rekla také služebnice tvá: Vždyt mi bude slovo pána mého krále k odtušení; (nebo jako andel Boží, tak jest pán muj král, když slyší dobré aneb zlé), a Hospodin Buh tvuj bude s tebou.
\par 18 A odpovídaje král, rekl žene: Medle, netaj prede mnou toho, nac se já tebe vzeptám. I rekla žena: Mluv, prosím, pane muj králi.
\par 19 Tedy rekl král: Není-liž Joáb jednatel všeho tohoto? I odpovedela žena a rekla: Jako jest živa duše tvá, pane muj králi, žet se nelze uchýliti na pravo aneb na levo ode všeho toho, což mluvil pán muj král; nebo služebník tvuj Joáb, ont jest mi rozkázal, a on naucil služebnici tvou všechnem slovum temto,
\par 20 A abych tak príkladne vedla rec tuto, zpusobil to služebník tvuj Joáb. Ale pán muj moudrý jest jako andel Boží, veda, což se koli deje na zemi.
\par 21 A protož rekl král Joábovi: Aj, již jsem to ucinil. Jdiž tedy, prived mládence Absolona.
\par 22 I padl Joáb na tvár svou k zemi, a pokloniv se, podekoval králi, a rekl Joáb: Dnes jest poznal služebník tvuj, že jsem nalezl milost pred ocima tvýma, pane muj králi, ponevadž jest král naplnil žádost služebníka svého.
\par 23 Tedy vstav Joáb, odšel do Gessur, a privedl Absolona do Jeruzaléma.
\par 24 I rekl král: Necht se navrátí do domu svého, ale tvári mé at nevidí. A tak navrátil se Absolon do domu svého, ale tvári královské nevidel.
\par 25 Nebylo pak žádného muže tak krásného jako Absolon ve všem Izraeli, aby takovou chválu mel. Od paty nohy jeho až do vrchu hlavy jeho nebylo na nem poškvrny.
\par 26 A když strihával vlasy hlavy své, (mel pak obycej každého roku je strihati, protože jej obtežovaly, i strihával je), tedy vážíval vlasy hlavy své, a bývalo jich dve ste lotu váhy obecné.
\par 27 Zrodili se pak Absolonovi tri synové a jedna dcera, jejíž jméno bylo Támar, kteráž byla žena vzezrení krásného.
\par 28 I byl Absolon v Jeruzaléme dve léte, a tvári královské nevidel.
\par 29 A protož poslal Absolon k Joábovi, chteje ho poslati k králi. Kterýžto nechtel prijíti k nemu. I poslal ješte podruhé, a nechtel prijíti.
\par 30 Tedy rekl služebníkum svým: Shlédnete dedinu Joábovu vedlé pole mého, kdežto má jecmen; jdete a spalte jej. I zapálili služebníci Absolonovi dedinu tu.
\par 31 A vstav Joáb, prišel k Absolonovi do domu jeho, a rekl jemu: Proc jsou služebníci tvoji zapálili dedinu mou?
\par 32 Odpovedel Absolon Joábovi: Aj, poslal jsem k tobe, rka: Prid sem, a pošli te k králi, abys rekl jemu: I proc jsem prišel z Gessur? Lépe mi bylo ješte tam zustati. Protož nyní nechat uzrím tvár královu. Paklit jest na mne nepravost, necht mne rozkáže zabiti.
\par 33 Tedy prišel Joáb k králi a oznámil to jemu. I povolal Absolona. Kterýž prišed k králi, poklonil se na tvár svou až k zemi pred ním. I políbil král Absolona.

\chapter{15}

\par 1 I stalo se potom, že sobe najednal Absolon vozu a koní a padesáte mužu, kteríž by behali pred ním.
\par 2 A vstávaje ráno Absolon, stával u cesty pri bráne, a každého muže, kterýž, maje pri, šel k králi pro rozsouzení, povolával Absolon, ríkaje: Z kterého jsi ty mesta? A když mu odpovedel: Z toho a z toho pokolení Izraelského jest služebník tvuj,
\par 3 Rekl jemu Absolon: Hle, tvá pre výborná jest a pravá, ale kdo by te vyslyšel, není žádného u krále.
\par 4 Ríkal také Absolon: Ó kdyby mne kdo ustanovil soudcím v zemi této, aby ke mne chodil každý, kdož by mel nesnáz a pri, a dopomáhal bych jemu k spravedlnosti.
\par 5 Ano i když nekdo pricházel, klaneje se jemu, vztáhl ruku svou, a uchope ho, políbil jej.
\par 6 Ciníval pak Absolon tím zpusobem všemu Izraelovi, kterýž pro rozsudek pricházel k králi; a tak obracel k sobe Absolon srdce mužu Izraelských.
\par 7 Stalo se také po ctyridcíti letech, že rekl Absolon k králi: Necht jdu, prosím, a splním slib svuj v Hebronu, který jsem ucinil Hospodinu.
\par 8 Nebo byl ucinil slib služebník tvuj, když jsem bydlil v Gessur v Syrii, rka: Jestliže mne zase kdy privede Hospodin do Jeruzaléma, tedy službu jemu uciním.
\par 9 Jemuž rekl král: Jdi u pokoji. Kterýžto vstav, odšel do Hebronu.
\par 10 I poslal Absolon špehére do všech pokolení Izraelských, aby rekli: Když uslyšíte zvuk trouby, rcete: Kralujet Absolon v Hebronu.
\par 11 S Absolonem pak šlo dve ste mužu z Jeruzaléma pozvaných, ale šli v uprímnosti své, aniž co o tom vedeli.
\par 12 Poslal také Absolon pro Achitofele Gilonského, rádce Davidova, aby prišel z mesta svého Gilo, když mel obetovati obeti. I stalo se spiknutí veliké, nebo lidu ustavicne pribývalo Absolonovi.
\par 13 V tom prišel posel k Davidovi, rka: Srdce Izraelských obrátilo se po Absolonovi.
\par 14 Tedy rekl David všechnem služebníkum svým, kteríž byli s ním v Jeruzaléme: Vstante a utecme, sic jinak nemohli bychom ujíti tvári Absolonovy. Pospešte, at odejdeme, aby se neuspíšil, a nezastihl nás, a neuvalil na nás zlého a nepohubil mesta ostrostí mece.
\par 15 I rekli služebníci královští králi: Což by koli zvolil pán náš král, ted jsme služebníci tvoji.
\par 16 A tak vyšel král se vší svou celedí pešky; toliko zanechal král desíti ženin, aby doma hlídaly.
\par 17 Když pak vyšel král a všecken lid pešky, postavili se na jednom míste zdaleka.
\par 18 I šli všickni služebníci pri nem; též všickni Cheretejští a všickni Peletejští, i všickni Gittejští, šest set mužu, kteríž prišli pešky z Gát, šli pred králem.
\par 19 Tedy rekl král k Ittai Gittejskému: Proc ty také jdeš s námi? Navrat se a zustan pri králi; nebo cizozemec jsi, kterýž se tudíž zase odbéreš k místu svému.
\par 20 Nedávnos prišel, a již bych dnes tebou pohnouti mel, abys s námi chodil? Ját zajisté pujdu, kdež mi se nahodí. Navratiž se a zprovod zase bratrí své, a budiž s tebou milosrdenství a pravda.
\par 21 Odpovídaje pak Ittai králi, rekl: Živt jest Hospodin a živt jest pán muj král, že na kterémkoli míste bude pán muj král, bud mrtev aneb živ, tut bude i služebník tvuj.
\par 22 I rekl David k Ittai: Podiž a prejdi. I prešel Ittai Gittejský a všickni muži jeho, i všecky dítky, kteréž byly s ním.
\par 23 Tedy plakala všecka zeme hlasem velikým, i všecken lid precházející. A tak prešel král potok Cedron; i všecken lid prešel proti ceste vedoucí na poušt.
\par 24 A aj, byl také Sádoch a všickni Levítové s ním, nesouce truhlu smlouvy Boží; i postavili truhlu Boží. Šel pak i Abiatar, a stál, dokudž všecken ten lid neprešel z mesta.
\par 25 I rekl král Sádochovi: Dones zase truhlu Boží do mesta. Jestližet najdu milost pred ocima Hospodinovýma, privedet mne zase, a ukážet mi ji i príbytek svuj.
\par 26 Paklit rekne takto: Nelíbíš mi se, aj, ted jsem, necht mi uciní, což se mu dobrého vidí.
\par 27 Rekl ješte král Sádochovi knezi: Zdaliž ty nejsi vidoucí? Navratiž se do mesta u pokoji, Achimaas pak syn tvuj, a Jonata syn Abiataruv, dva synové vaši, budou s vámi.
\par 28 Hle, já pobudu na rovinách poušte, dokudž neprijde poselství od vás, skrze než by mi oznámeno bylo.
\par 29 A tak donesl zase Sádoch a Abiatar truhlu Boží do Jeruzaléma, a zustali tam.
\par 30 Ale David šel vzhuru k vrchu hory Olivetské, vstupuje a pláce, maje prikrytou hlavu svou; šel pak bos. Všecken také lid, kterýž byl s ním, prikryl jeden každý hlavu svou, a šli, ustavicne placíce.
\par 31 Tedy povedíno Davidovi, že Achitofel jest s temi, kteríž se spikli s Absolonem. I rekl David: Zruš, prosím, ó Hospodine, radu Achitofelovu.
\par 32 I stalo se, jakž prišel David na vrch hory, aby se tam pomodlil Bohu, aj, potkal se s ním Chusai Architský, maje roztržené roucho své a hlavu prstí posypanou.
\par 33 I rekl jemu David: Pujdeš-li se mnou, budeš mi bremenem.
\par 34 Pakli se navrátíš do mesta a díš k Absolonovi: Služebníkem tvým budu, ó králi, služebníkem zajisté otce tvého byl jsem již zdávna, již pak budu služebníkem tvým, tedy zrušíš radu Achitofelovu.
\par 35 Však tam budou s tebou Sádoch a Abiatar kneží. Protož cožkoli uslyšíš z domu králova, oznámíš Sádochovi a Abaitarovi knežím.
\par 36 Aj, tam s nimi jsou dva synové jejich, Achimaas Sádochuv a Jonata Abiataruv, po nichž mi vzkážete, což byste koli uslyšeli.
\par 37 Šel tedy Chusai, prítel Daviduv do mesta; Absolon také prijel do Jeruzaléma.

\chapter{16}

\par 1 A když David sešel malicko s vrchu, aj Síba služebník Mifibozetuv vyšel proti nemu se dvema osly osedlanými, na nichž nesl dve ste chlebu, a sto hroznu suchých, a sto hrud fíku, a nádobu vína.
\par 2 I rekl král Síbovi: K cemu jsou tyto veci? Odpovedel Síba: Oslové pro celed královskou k jízde, chléb pak a fíky, aby jedli služebníci, a víno, aby se napil, kdož by ustal na poušti.
\par 3 Opet král rekl: Kdež jest pak syn pána tvého? Odpovedel Síba králi: Aj, zustal v Jeruzaléme, nebo rekl: Dnes navrátí mi dum Izraelský království otce mého.
\par 4 Rekl ješte král Síbovi: Aj, tvét jsou všecky veci, kteréž má Mifibozet. Jemuž rekl Síba s poklonou: Nechat vždycky tak nalézám milost pred ocima tvýma, pane muj králi.
\par 5 Tedy bral se král David do Bahurim, a aj, vyšel odtud muž z celedi domu Saulova, jehož jméno bylo Semei, syn Geruv, kterýžto vždy jda, zlorecil.
\par 6 Ano i kamením házel na Davida, a na všecky služebníky krále Davida, ackoli všecken lid a všickni udatní byli po pravici jeho i po levici jeho.
\par 7 A takto mluvil Semei, když mu zlorecil: Vyjdi, vyjdi, vražedlníce a nešlechetníce.
\par 8 Obrátilt jest na tebe Hospodin všelikou krev domu Saulova, na jehož jsi míste kraloval, a dal Hospodin království v ruku Absolona syna tvého. A aj, již se vidíš v svém neštestí, nebo jsi vražedlník.
\par 9 I rekl Abizai syn Sarvie králi: I proc zlorecí tento mrtvý pes pánu mému králi? Necht jdu medle a setnu mu hlavu.
\par 10 Ale král rekl: Co vám do toho, synové Sarvie, že zlorecí? Ponevadž Hospodin jemu rozkázal: Zlorec Davidovi, i kdož by smel ríci: Proc tak ciníš?
\par 11 Rekl ješte David k Abizai i ke všechnem služebníkum svým: Aj, syn muj, kterýž pošel z života mého, hledá bezživotí mého, cím více nyní tento syn Jemini? Nechte ho, at zlorecí, nebo jemu rozkázal Hospodin.
\par 12 Snad popatrí Hospodin na ssoužení mé, a odplatí mi Hospodin dobrým za zlorecenství jeho dnes.
\par 13 A tak šel David a muži jeho cestou. Semei také šel po stráni hory naproti nemu, a jda, zlorecil a házel kamením proti nemu a zmítal prachem.
\par 14 I prišel král a všecken lid, kterýž byl s ním, ustalý, a odpocinul tu.
\par 15 Absolon pak i všecken lid Izraelský prišli do Jeruzaléma, a Achitofel s ním.
\par 16 A když prišel Chusai Architský, prítel Daviduv k Absolonovi, rekl Chusai Absolonovi: Živ bud král, živ bud král.
\par 17 Tedy rekl Absolon k Chusai: Toliž jest vdecnost tvá k príteli tvému? Procež jsi nešel s prítelem svým?
\par 18 Odpovedel Chusai Absolonovi: Nikoli, ale kohož vyvolil Hospodin a lid tento i všickni muži Izraelští, toho budu a s tím zustanu.
\par 19 Presto komuž bych já sloužiti mel? Zdali ne synu jeho? Jakož jsem sloužil otci tvému, tak sloužiti budu tobe.
\par 20 Rekl pak Absolon Achitofelovi: Radtež, co máme ciniti?
\par 21 Odpovedel Achitofel Absolonovi: Vejdi k ženinám otce svého, kterýchž zanechal, aby hlídaly doma. I uslyší všecken Izrael, žes se zošklivil otci svému, a zsilí se ruce všech, kteríž jsou s tebou.
\par 22 A protož rozbili Absolonovi stan na paláci. I všel Absolon k ženinám otce svého pred ocima všeho Izraele.
\par 23 Rada pak Achitofelova, kterouž dával toho casu, byla, jako by se kdo doptával na rec Boží. Taková byla každá rada Achitofelova, jakož u Davida tak u Absolona.

\chapter{17}

\par 1 Rekl ješte Achitofel Absolonovi: Necht medle vyberu dvanácte tisíc mužu, abych vstana, honil Davida noci této.
\par 2 A dostihna ho, dokudž jest ustalý a zemdlené má ruce, predesím jej, a utece všecken lid, kterýž jest s ním, i zamorduji krále samého.
\par 3 A tak obrátím všecken lid k tobe; nebo jako bych je všecky obrátil, když zhyne ten muž, kteréhož ty hledáš. O lid ty se nic nestarej.
\par 4 I líbila se ta rec Absolonovi i všechnem starším Izraelským.
\par 5 A však rekl Absolon: Zavolej sem hned Chusai Architského, abychom slyšeli, co i on mluviti bude.
\par 6 Když pak prišel Chusai k Absolonovi, mluvil k nemu Absolon, rka: Takto mluvil Achitofel. Máme-li uciniti vedlé reci jeho, cili nic? Povez ty.
\par 7 Tedy rekl Chusai Absolonovi: Nenít dobrá rada, kterouž dal Achitofel nyní.
\par 8 Dále mluvil Chusai: Ty víš o otci svém i mužích jeho, že jsou udatní, k tomu zjitrení na mysli, jako nedvedice osirelá v poli. Nadto otec tvuj jest muž válecný, kterýž nebude nocovati s lidem.
\par 9 Nýbrž nyní spíše v záloze stojí v nejaké jeskyni, aneb v nekterém jiném míste. I stalo by se, jestliže by kterí padli z tech tvých pri zacátku, že každý, kdož by o tom uslyšel, rekl by: Stala se porážka v lidu, kterýž postoupil po Absolonovi.
\par 10 A tak by i silných reku srdce, kteréž jest jako srdce lva, škodlive osláblo; nebo ví všecken Izrael, že jest zmužilý otec tvuj, a silní ti, kteríž jsou s ním.
\par 11 Ale radím, abys, shromážde k sobe všecken Izrael od Dan až do Bersabé, kterýž v množství jest jako písek pri mori, ty sám životne vytáhl k boji.
\par 12 A tak potáhneme proti nemu, na kterém by koli míste nalezen býti mohl, a pripadneme na nej, jako padá rosa na zemi; i nepozustanet z neho, totiž ze všech mužu, kteríž jsou pri nem, ani jednoho.
\par 13 Jestliže by se pak shrnul do mesta, snesou všecken lid Izraelský k tomu mestu provazy, a vtrhneme je až do potoka, tak aby tam ani kaménka nebylo lze najíti.
\par 14 I rekl Absolon a všickni muži Izraelští: Lepšít jest rada Chusai Architského, než rada Achitofelova. Hospodin zajisté to zpusobil, aby zrušena byla rada Achitofelova, kteráž sic dobrá byla, aby tak uvedl Hospodin na Absolona zlé veci.
\par 15 Oznámil pak to Chusai Sádochovi a Abiatarovi knežím: Tak a tak radil Achitofel Absolonovi a starším Izraelským, ale já takto a takto jsem radil.
\par 16 Nyní tedy pošlete rychle a oznamte Davidovi, rkouce: Nezustavej pres noc na rovinách poušte, ale preprav se bez meškání, aby snad nebyl sehlcen král i všecken lid, kterýž jest s ním.
\par 17 Jonata pak a Achimaas stáli u studnice Rogel. I šla tam devecka a povedela jim, aby oni jdouce, oznámili to králi Davidovi; nebo nesmeli se ukázati aneb vjíti do mesta.
\par 18 A však uzrel je služebník, a oznámil Absolonovi. Procež odšedše oba rychle, vešli do domu muže v Bahurim, kterýž mel studnici na dvore svém. I spustili se do ní.
\par 19 A žena vzavši plachtu, rozestrela ji na vrch studnice, a nasypala na ni krup. A tak nebyla ta vec spatrína.
\par 20 Nebo když prišli služebníci Absolonovi k žene do domu a rekli: Kde jest Achimaas a Jonata? odpovedela jim žena: Prešlit jsou ten potok. Hledavše tedy a nic nenalezše, navrátili se do Jeruzaléma.
\par 21 A když oni odešli, tito vystoupivše z studnice, šli a oznámili králi Davidovi a rekli jemu: Vstante a prepravte se rychle pres vodu, nebo toto radil proti vám Achitofel.
\par 22 Protož vstav David a všecken lid, kterýž byl s ním, prepravili se pres Jordán, prvé než se rozednilo; a nezustal ani jeden, ješto by se neprepravil pres Jordán.
\par 23 Tedy Achitofel vida, že se nestalo vedlé rady jeho, osedlal osla, a vstav, odjel do domu svého, do mesta svého. A uciniv porízení v celedi své, obesil se a umrel; i pochován jest v hrobe otce svého.
\par 24 David pak byl již prišel do Mahanaim, když se Absolon prepravil pres Jordán, on i všecken lid Izraelský s ním.
\par 25 A tu ustanovil Absolon Amazu místo Joába nad vojskem. (Byl pak Amaza syn muže, jehož jméno bylo Jetra Izraelský, kterýž všel k Abigail dceri Náchas, sestre Sarvie matky Joábovy.)
\par 26 I položil se Izrael a Absolon v zemi Galád.
\par 27 Stalo se pak, když prišel David do Mahanaim, že Sobi syn Náchas z Rabbat synu Ammon, a Machir syn Amieluv z Lodebar, a Barzillai Galádský z Rogelim,
\par 28 Luže, cíše a nádoby hlinené, též pšenice, jecmene, mouky, krup, bobu, šocovice a pražmy,
\par 29 Ano i medu, másla a ovcí i syru kravských prinesli Davidovi a lidu, kterýž s ním byl, aby jedli. Nebo rekli: Lid ten jest hladovitý a ustalý, i žíznivý na té poušti.

\chapter{18}

\par 1 Sectl pak David lid, kterýž mel s sebou, a ustanovil nad nimi hejtmany a setníky.
\par 2 I uvedl David tretinu lidu v správu Joábovu, a tretinu v správu Abizai syna Sarvie, bratra Joábova, a tretinu v správu Ittai Gittejského. A rekl král lidu: I ját také potáhnu s vámi.
\par 3 Ale lid rekl: Nikoli nepotáhneš; nebo jestliže bychom i utíkali, nebudout na to velmi dbáti, aniž, by nás i polovici zbili, budou toho velmi vážiti. Ty sám zajisté jsi jako z nás deset tisícu, protož lépe bude, abys nám byl v meste ku pomoci.
\par 4 I rekl jim král: Což se vám za dobré vidí, uciním. Tedy stál král u brány, a všecken lid vycházel po stu a po tisíci.
\par 5 Prikázal pak král Joábovi a Abizai a Ittai, rka: Zacházejtež mi pekne s synem Absolonem. A všecken lid slyšel, když prikazoval král všechnem hejtmanum o Absolonovi.
\par 6 A tak vytáhl lid do pole proti lidu Izraelskému, a byla bitva v lese Efraim.
\par 7 I poražen jest tu lid Izraelský od služebníku Davidových, a stala se tu porážka veliká v ten den až do dvadceti tisícu.
\par 8 Nebo když ta bitva rozšírila se po vší krajine, více pohubil lidu les, nežli jich požral mec toho dne.
\par 9 Potkal se pak Absolon s služebníky Davidovými. Kterýžto Absolon jel na mezku, (i podšel mezek pod hustý dub veliký), a uvázl za hlavu v stromu tom, tak že visel mezi nebem a zemí. Ale mezek, kterýž pod ním byl, odbehl pryc.
\par 10 To uzrev jeden, oznámil Joábovi, rka: Hle, videl jsem Absolona, an visí na dube.
\par 11 I rekl Joáb muži, kterýž mu to oznámil: Aj, videls. Procež jsi ho tam nesrazil na zem? A já bylt bych povinen dáti deset lotu stríbra a pás rytírský jeden.
\par 12 Odpovedel muž ten Joábovi: A já, bych i odectených mel na ruce své tisíc lotu stríbra, nevztáhl bych ruky své na syna králova; nebo jsme slyšeli, kterak prikazoval král tobe a Abizai a Ittai, rka: Šetrte všickni syna mého Absolona;
\par 13 Lec bych se dopustiti chtel sám proti duši své nepravosti. Ale nebývát nic tajno pred králem, a ty bys sám proti mne stál.
\par 14 I rekl Joáb: Nebudut se tu meškati s tebou. Protož vzav tri kopí do ruky své, vrazil je do Absolona, an ješte živ byl na dube.
\par 15 A obskocivše Absolona deset služebníku, odencu Joábových, bili jej a zabili.
\par 16 V tom zatroubil Joáb v troubu. I vrátil se lid od honení Izraele; nebo zdržel Joáb lid.
\par 17 A vzavše Absolona, uvrhli jej v tom lese do jámy veliké, a nametali na nej hromadu kamení velmi velikou. Ale všecken Izrael zutíkali jeden každý do stanu svých.
\par 18 ( Absolon pak vzal byl a vyzdvihl sobe za života svého sloup v údolí královském; nebo byl rekl: Nemám syna, aby zustala pamet jména mého. Protož nazval ten sloup jménem svým, kterýž slove místo Absolonovo až do dnešního dne.)
\par 19 Tedy Achimaas syn Sádochuv rekl: Medle, necht bežím, abych zvestoval králi, že ho vysvobodil Hospodin z ruky neprátel jeho.
\par 20 I rekl jemu Joáb: Nebyl bys dnes dobrým poslem, ale oznámíš to jiného dne; dnes však neoznamuj, proto že syn králuv umrel.
\par 21 Potom rekl Joáb k Chuzi: Jdiž, zvestuj králi, co jsi videl. A pokloniv se Chuzi Joábovi, bežel.
\par 22 Mluvil pak ješte Achimaas syn Sádochuv, a rekl Joábovi: Bud, jak bud, medle, necht já také bežím za Chuzi. Odpovedel Joáb: Proc bys ty bežel, synu muj, když nemáš, co bys dobrého zvestoval?
\par 23 I rekl: Bud, jak bud, pobehnu. Odpovedel mu: Bež. A tak bežel Achimaas cestou prímejší a predbehl Chuzi.
\par 24 David pak sedel mezi dvema branami. I vyšel strážný na vrch brány na zed, kterýž pozdvih ocí svých, uzrel, a hle, muž bežel sám.
\par 25 Tedy volaje strážný, oznámil králi. I rekl král: Jest-lit sám, dobré poselství nese. A když ten šel predce a približoval se,
\par 26 Uzrel ješte strážný muže druhého bežícího. I zavolal strážný na branného a rekl: Hle, opet muž beží sám.Tedy rekl král: I tent v poselství beží.
\par 27 Rekl ješte strážný: Vidím beh prvního, jako beh Achimaasa syna Sádochova. I rekl král: Dobrýte to muž, protož s dobrým poselstvím jde.
\par 28 Tedy volaje Achimaas, rekl králi: Pokoj tobe. A pokloniv se králi tvárí svou k zemi, rekl: Požehnaný Hospodin Buh tvuj, kterýž podmanil muže ty, kteríž pozdvihli rukou svých proti pánu mému králi.
\par 29 I rekl král: Dobre-li se má syn muj Absolon? Odpovedel Achimaas: Videl jsem hluk veliký, když posílal služebníka králova Joáb, a mne služebníka tvého, ale nevím nic, co bylo.
\par 30 Jemuž rekl král: Odstup a stuj tamto. I odstoupil a stál.
\par 31 A v tom Chuzi prišed, rekl: Zvestuje se pánu mému králi, že vysvobodil te dnes Hospodin z ruky všech povstávajících proti tobe.
\par 32 Ale král rekl k Chuzi: Jest-liž živ syn muj Absolon? Odpovedel Chuzi: Necht jsou tak, jako syn králuv, neprátelé pána mého krále, i všickni, kteríž povstávají proti tobe ke zlému.
\par 33 I zarmoutil se král, a vstoupiv do horního pokoje na bráne, plakal a jda, mluvil takto: Synu muj Absolone, synu muj, synu muj Absolone! Ó kdybych byl umrel za tebe, Absolone synu muj, synu muj!

\chapter{19}

\par 1 I oznámeno jest Joábovi: Aj, král pláce a naríká pro Absolona.
\par 2 Procež obrátilo se to vysvobození toho dne v kvílení všemu lidu; nebo slyšel lid v ten den, že bylo praveno: Zámutek má král pro syna svého.
\par 3 A tak kradl se lid toho dne, vcházeje do mesta, jako se krade lid, když se stydí, utíkaje z boje.
\par 4 Král pak zakryl tvár svou, a kricel král hlasem velikým: Synu muj Absolone, Absolone synu muj, synu muj!
\par 5 Tedy všed Joáb k králi do domu, rekl: Zahanbil jsi dnes tvári všech služebníku svých, kteríž vysvobodili život tvuj dnes, a život synu tvých i dcer tvých, a život žen tvých i život ženin tvých,
\par 6 Miluje ty, kteríž te mají v nenávisti, a v nenávisti maje ty, kteríž te milují; nebos dokázal dnes, že sobe nevážíš hejtmanu a služebníku. Shledalt jsem to zajisté dnes, že kdyby byl Absolon živ zustal, bychom pak všickni dnes byli pobiti, tedy dobre by se tobe to líbilo.
\par 7 Protož nyní vstana, vyjdi, a mluv ochotne k služebníkum svým; nebo prisahámt skrze Hospodina, jestliže nevyjdeš, žet nezustane žádný s tebou této noci, a tot bude tobe horší, nežli všecky zlé veci, kteréž na tebe prišly od mladosti tvé až dosavad.
\par 8 A tak vstav král, posadil se v bráne. I povedíno bylo všemu lidu temito slovy: Aj, král sedí v bráne. I prišel všecken lid pred oblícej krále, ale lid Izraelský byl zutíkal jeden každý do svých stanu.
\par 9 Tedy všecken lid hádal se vespolek ve všech pokoleních Izraelských, a pravili: Král vytrhl nás z ruky neprátel našich, a tentýž vytrhl nás z ruky Filistinských, a ted nyní utekl z zeme pred Absolonem.
\par 10 Absolon pak, kteréhož jsme pomazali sobe, zahynul v boji. Nyní tedy, proc zanedbáváte privésti zase krále?
\par 11 Protož král David poslal k Sádochovi a Abiatarovi knežím, s temito slovy: Mluvte k starším Judským, rkouce: Proc máte býti poslední v uvedení zase krále do domu jeho? (Nebo rec všeho lidu Izraelského donesla se krále o uvedení jeho do domu jeho.)
\par 12 Bratrí moji jste, kost má a telo mé jste; proc tedy máte býti poslední v uvedení zase krále?
\par 13 A Amazovi také rcete: Zdaliž ty nejsi kost má a telo mé? Toto at mi uciní Buh a toto pridá, jestliže nebudeš hejtmanem vojska prede mnou po všecky dny na místo Joába.
\par 14 I naklonil srdce všech mužu Judských, jako muže jednoho, aby poslali k králi, rkouce: Navratiž se ty i všickni služebníci tvoji.
\par 15 Tedy navrátil se král, a prišel až k Jordánu. Lid pak Judský byl prišel do Galgala, aby se bral vstríc králi, a prevedl jej pres Jordán.
\par 16 Pospíšil také Semei syn Gery, syna Jemini, kterýž byl z Bahurim, a vyšel s lidem Judským vstríc králi Davidovi.
\par 17 A tisíc mužu bylo s ním z Beniamin. Síba také služebník domu Saulova, s patnácti syny svými, a dvadceti služebníku jeho s ním, prepravili se pres Jordán pred krále.
\par 18 I preplavili lodí, aby prevezli celed královskou a ucinili, což by se jemu líbilo. Semei pak syn Geruv padl pred králem, když se prepraviti mel pres Jordán.
\par 19 A rekl králi: Nepocítej mi pán muj nepravosti, a nezpomínej, co nepráve ucinil služebník tvuj toho dne, když vyšel pán muj král z Jeruzaléma, aby to mel skládati král v srdci svém.
\par 20 Nebot zná služebník tvuj, že zhrešil, a aj, prišel jsem dnes prvé, než kdo ze vší celedi Jozefovy, abych vyšel vstríc pánu svému králi.
\par 21 I odpovedel Abizai syn Sarvie a rekl: A což nebude zabit Semei, proto že zlorecil pomazanému Hospodinovu?
\par 22 Ale David rekl: Co vám do toho, synové Sarvie, že jste mi dnes odporní? Dnes-liž má zabit býti nekdo v Izraeli? Nebo zdaliž nevím, že dnes jsem králem nad Izraelem.
\par 23 Tedy rekl král k Semei: Neumreš. I prisáhl mu král.
\par 24 Mifibozet také vnuk Sauluv vyjel vstríc králi. (Neošetroval pak byl noh svých, ani brady nespravoval, ani šatu svých nepral od toho dne, jakž byl odšel král, až do dne, když se navrátil v pokoji.)
\par 25 A když prišel do Jeruzaléma, vyšel vstríc králi. I rekl jemu král: Proc jsi neodšel se mnou, Mifibozete?
\par 26 Kterýž odpovedel: Pane muj králi, služebník muj oklamal mne. Reklte byl zajisté služebník tvuj: Osedlám sobe osla, abych vsedna na nej, bral se s králem, proto že jest kulhavý služebník tvuj.
\par 27 I osocil služebníka tvého u pána mého krále, ale pán muj král jest jako andel Boží, uciniž tedy, cožt se dobrého vidí.
\par 28 Nebo všickni z domu otce mého byli jsme hodni smrti pred pánem mým králem, a však jsi zasadil služebníka svého mezi ty, kteríž jedí chléb stolu tvého. K cemuž bych více právo mel, a oc se více na krále domlouval?
\par 29 Jemuž rekl král: Proc šíríš rec svou? Vyrklt jsem. Ty a Síba rozdelte se statkem.
\par 30 Ješte rekl Mifibozet králi: Trebas necht všecko vezme, když se jen navrátil pán muj král v pokoji do domu svého.
\par 31 Ano i Barzillai Galádský vyšel z Rogelim, a prepravil se s králem pres Jordán, aby ho zprovodil za Jordán.
\par 32 Byl pak Barzillai velmi starý, maje osmdesáte let, kterýž opatroval krále stravou, když obýval v Mahanaim; nebo byl clovek bohatý velmi.
\par 33 I rekl král k Barzillai: Pod se mnou, a chovati te budu pri sobe v Jeruzaléme.
\par 34 Ale Barzillai odpovedel králi: Jacíž jsou dnové veku mého, abych šel s králem do Jeruzaléma?
\par 35 V osmdesáti letech jsem dnes. Zdaliž mohu rozeznati mezi dobrým a zlým? Zdaliž okušením rozezná služebník tvuj, co bych jedl a co bych pil? Zdaliž poslouchati mohu již hlasu zpeváku a zpevakyní? Proc by tedy služebník tvuj déle býti mel bremenem pánu svému králi?
\par 36 Malicko ješte pujde služebník tvuj za Jordán s králem; nebo proc by mi se takovou odplatou král odplacovati mel?
\par 37 Necht se navrátí, prosím, služebník tvuj, at umru v meste svém, kdež jest hrob otce mého a matky mé. Aj, služebník tvuj Chimham pujde se pánem mým králem, jemuž uciníš, cožt se dobrého videti bude.
\par 38 I rekl král: Dobre, necht jde se mnou Chimham, a uciním jemu, což se tobe za dobré videti bude; nadto cehožkoli požádáš ode mne, tot uciním.
\par 39 A když se prepravil všecken lid pres Jordán, král také prepravil se. I políbil král Barzillai a požehnal ho, kterýžto navrátil se na místo své.
\par 40 A tak bral se král do Galgala a Chimham s ním; všecken také lid Judský provázeli krále, též i polovice lidu Izraelského.
\par 41 A aj, všickni muži Izraelští prišedše k králi, rekli jemu: Proc jsou te ukradli bratrí naši, muži Juda, a prevedli krále a celed jeho pres Jordán, i všecky muže Davidovy s ním?
\par 42 I odpovedeli všickni muži Judští mužum Izraelským: Proto že král jest príbuzný náš. A proc se hneváte o to? Zdaliž nás za to král pokrmy opatroval? Zdaliž nám jaké dary dal?
\par 43 Odpovídajíce pak muži Izraelští mužum Judským, rekli: Deset dílu máme v králi, a protož i v Davidovi máme více nežli vy. Procež tedy málo jste nás sobe vážili? Zdaliž jsme my prvé o to nemluvili, abychom zase privedli krále svého? Ale tvrdší byla rec mužu Judských nad rec mužu Izraelských.

\chapter{20}

\par 1 Prišel pak tu náhodou clovek nešlechetný, jehož jméno bylo Seba, syn Bichri, muž Jemini. Ten zatroubil v troubu a rekl: Nemámet my dílu v Davidovi, ani dedictví v synu Izai;obrat se jeden každý k stanum svým, ó Izraeli.
\par 2 A tak všickni muži Izraelští odstoupili od Davida za Sebou synem Bichri, ale muži Judští prídrželi se krále svého od Jordánu až do Jeruzaléma.
\par 3 David pak prišel do domu svého do Jeruzaléma. A vzal král deset ženin, kterýchž byl nechal, aby hlídaly doma, a dal je pod stráž, a choval je, ale nevcházel k nim. A zustaly zavrené až do dne smrti své v vdovství.
\par 4 Potom rekl král Amazovi: Svolej mi muže Judské do tretího dne, ty také se tu postav.
\par 5 A tak odšel Amaza, aby svolal lid Judský, ale prodlil mimo urcitý cas, kterýž mu byl uložil.
\par 6 Protož rekl David Abizai: Již nyní hure nám ciniti bude Seba syn Bichri, nežli Absolon. Vezmi služebníky pána svého, a hon jej, aby sobe nenalezl mest hrazených, a tak ušel by nám s ocí.
\par 7 Tedy táhli za ním muži Joábovi, i Cheretejští, i Peletejští, a všickni udatní vytáhli z Jeruzaléma, aby honili Sebu syna Bichri.
\par 8 A když byli u kamene toho velikého, kterýž jest v Gabaon, Amaza potkal se s nimi. Joáb pak byl opásán po sukni, v kterouž byl oblecen, na níž také mel pripásaný mec k bedrám v pošve své, kterýž snadne vytrhnouti i zase vstrciti mohl.
\par 9 I rekl Joáb Amazovi: Dobre-li se máš, bratre muj? A ujal Joáb pravou rukou Amazu za bradu, aby ho políbil.
\par 10 Ale Amaza nešetril se mece, kterýž byl v ruce Joábove. I ranil ho jím pod páté žebro, a vykydl streva jeho na zem jednou ranou, a umrel. I honili Joáb a Abizai, bratr jeho, Sebu syna Bichri.
\par 11 Tedy stoje tu jeden podlé neho z služebníku Joábových, rekl: Kdokoli preje Joábovi, a kdokoli drží s Davidem, jdi za Joábem.
\par 12 Amaza pak válel se ve krvi prostred cesty. A vida onen, že se zastavoval všecken lid, odvlékl Amazu s cesty do pole, a uvrhl na nej roucho, vida, že každý, kdož šel mimo nej, zastavoval se.
\par 13 A když byl odvlecen z cesty, šel jeden každý za Joábem, aby honili Sebu syna Bichri.
\par 14 (Kterýžto prošel všecka pokolení Izraelská do Abel Betmaacha se všechnemi Berejskými, kteríž se byli shromáždili a šli za ním.)
\par 15 A pritáhše, oblehli jej v Abel Betmaacha, a udelali násyp proti mestu, kteréž se bránilo z bašt; všecken pak lid, kterýž byl s Joábem, usiloval podvrátiti zed.
\par 16 V tom zvolala jedna rozumná žena z mesta: Slyšte, slyšte! Rcete medle Joábovi: Pristup sem, a budu s tebou mluviti.
\par 17 Kterýž když k ní pristoupil, rekla žena: Ty-li jsi Joáb? Odpovedel: Jsem. I rekla jemu: Poslyš slov služebnice své. Odpovedel: Slyším.
\par 18 Protož mluvila, rkuci: Takt jsou rozmlouvali hned s pocátku, rkouce: Bez pochyby žet se ptáti budou Abelských, a tak se spraví.
\par 19 Ját jsem jedno z pokojných a verných Izraelských, ty pak usiluješ zkaziti mesto, a to ješte hlavní mesto v Izraeli. I procež usiluješ sehltiti dedictví Hospodinovo?
\par 20 Tedy odpovedel Joáb a rekl: Odstup, odstup to ode mne, abych sehltiti a zkaziti chtel.
\par 21 Nenít toho, ale muž z hory Efraim, jménem Seba, syn Bichri, pozdvihl ruky své proti králi Davidovi. Vydejte ho samého, a odtrhnemt od mesta. I rekla žena Joábovi: Hle, hlava jeho vyvržena bude tobe pres zed.
\par 22 A tak zjednala to žena ta u všeho lidu moudrostí svou, že stavše hlavu Seby syna Bichri, vyhodili ji Joábovi. Kterýžto když zatroubil v troubu, rozešli se od mesta jeden každý do stanu svých. Joáb také vrátil se do Jeruzaléma k králi.
\par 23 Byl pak Joáb predstaven všemu vojsku Izraelskému, a Banaiáš syn Joiaduv nad Cheretejskými a Peletejskými.
\par 24 Též Aduram byl nad platy, a Jozafat syn Achiluduv byl kanclírem,
\par 25 A Seiáš písarem, Sádoch pak a Abiatar byli knežími.
\par 26 Híra také Jairský byl knížetem Davidovým.

\chapter{21}

\par 1 Byl pak hlad za dnu Davidových tri léta porád. I hledal David tvári Hospodinovy. Jemuž rekl Hospodin: Pro Saule a pro dum jeho vražedlný, nebo zmordoval Gabaonitské.
\par 2 Tedy povolav král Gabaonitských, mluvil k nim. (Gabaonitští pak nebyli z synu Izraelských, ale z ostatku Amorejských, jimž ackoli byli synové Izraelští prísahou se zavázali, však Saul usiloval je vypléniti, v horlivosti své pro syny Izraelské a Judské.)
\par 3 A rekl David Gabaonitským: Což vám uciním a cím vás spokojím, abyste dobrorecili dedictví Hospodinovu?
\par 4 Odpovedeli jemu Gabaonitští: Není nám o stríbro ani o zlato ciniti s Saulem a domem jeho, ani o bezživotí nekoho z Izraele. I rekl: Cožkoli díte, uciním vám.
\par 5 Kteríž rekli králi: Muže toho, kterýž vyhubil nás, a kterýž ukládal o nás, vyhladíme, aby nezustal v žádných koncinách Izraelských.
\par 6 Necht jest nám dáno sedm mužu z jeho synu, a zvesíme je Hospodinu u Gabaa Saulova, kteréhož byl vyvolil Hospodin. I rekl král: Já dám.
\par 7 Odpustil pak král Mifibozetovi synu Jonaty, syna Saulova, pro prísahu Hospodinovu, kteráž byla mezi nimi, mezi Davidem a Jonatou synem Saulovým.
\par 8 Ale vzal král dva syny Rizpy dcery Aja, kteréž porodila Saulovi, Armona a Mifibozeta, a pet synu sestry Míkol dcery Saulovy, kteréž byla porodila Adrielovi synu Barzillai Molatitského.
\par 9 I vydal je v ruku Gabaonitských, kteréž zvešeli na hore pred Hospodinem. A tak kleslo tech sedm spolu, a zbiti jsou pri zacátku žne, když pocínali žíti jecmene.
\par 10 Rizpa pak, dcera Aja, vzavši žíni, prostrela ji sobe na skále pri zacátku žne, dokudž nepršel na ne déšt s nebe, a nedala ptákum nebeským sedati na ne ve dne, ani pricházeti zveri polní v noci.
\par 11 Tedy oznámeno jest Davidovi, co ucinila Rizpa dcera Aja, ženina Saulova.
\par 12 Protož odšed David, vzal kosti Saulovy a kosti Jonaty syna jeho od starších Jábes Galád, kteríž byli je ukradli na ulici Betsan, kdežto byli je zvešeli Filistinští v ten den, když porazili Filistinští Saule v Gelboe.
\par 13 Vzal, pravím, odtud kosti Saulovy a kosti Jonaty syna jeho; i kosti tech, kteríž zvešeni byli, sebrali.
\par 14 A pochovali je s kostmi Saulovými a Jonaty syna jeho, v zemi Beniamin, v Sela, v hrobe Cis otce jeho; a ucinili všecko, což byl prikázal král, a potom slitoval se Buh nad zemí.
\par 15 Vznikla pak opet válka Filistinských proti Izraelovi. I vytáhl David a služebníci jeho s ním, a bili se s Filistinskými, tak že ustal David.
\par 16 Tedy Izbibenob, kterýž byl z synu jednoho obra, (jehož kopí hrot vážil tri sta lotu oceli, a mel pripásaný k sobe mec nový), myslil zabiti Davida.
\par 17 Ale retoval ho Abizai syn Sarvie, a raniv toho Filistinského, zabil jej. Protož muži Davidovi prisáhli, rkouce jemu: Nepujdeš více s námi do boje, abys nezhasil svíce Izraelské.
\par 18 Opet byla válka v Gob s Filistinskými. Tehdáž Sibbechai Chusatský zabil Sáfu, kterýž byl z synu téhož obra.
\par 19 Byla ješte i jiná válka s Filistinskými v Gob, kdež Elchánan syn Járe Oregim Betlémský zabil bratra Goliáše Gittejského, u jehož kopí bylo drevo jako vratidlo tkadlcovské.
\par 20 Potom byla také válka v Gát. A byl tam muž veliké postavy, kterýž mel u rukou a u noh po šesti prstech, všech ctyrmecítma, a byl také syn toho obra.
\par 21 Ten když hanel Izraele, zabil ho Jonata, syn Semmaa bratra Davidova.
\par 22 Ti ctyri byli synové jednoho obra v Gát, kteríž padli od ruky Davidovy a od ruky služebníku jeho.

\chapter{22}

\par 1 Mluvil pak David Hospodinu slova písne této v ten den, když ho vysvobodil Hospodin z ruky všech neprátel jeho, i z ruky Saulovy.
\par 2 A rekl: Hospodin skála má a hrad muj, i vysvoboditel muj se mnou.
\par 3 Buh skála má, doufati budu v neho; štít muj a roh spasení mého, vyvýšení mé a útocište mé, spasitel muj, kterýž od násilí vysvobozuje mne.
\par 4 Chvály hodného vzýval jsem Hospodina, a od neprátel svých vysvobozen jsem.
\par 5 Nebo obklícily mne byly úzkosti smrti, a proudové bezbožných predesili mne.
\par 6 Bolesti smrtelné obstoupily mne, a osídla smrti zachvátila mne.
\par 7 V úzkosti své vzýval jsem Hospodina, a k Bohu svému volal jsem, i vyslyšel z chrámu svého hlas muj, a krik muj prišel v uši jeho.
\par 8 Tedy pohnula se a zatrásla zeme, základové nebes pohnuli se, a trásli se pro rozhnevání jeho.
\par 9 Dým vycházel z chrípí jeho, a ohen zžírající z úst jeho, od nehož se uhlí roznítilo.
\par 10 Nakloniv nebes, sstoupil, a mrákota byla pod nohami jeho.
\par 11 I vsedl na cherubín a letel, a spatrín jest na perí vetrovém.
\par 12 Položil temnosti vukol sebe jako stany, shrnutí vod, oblaky husté.
\par 13 Od blesku oblíceje jeho rozpálilo se uhlí reravé.
\par 14 Hrímal s nebes Hospodin, a Nejvyšší vydal zvuk svuj.
\par 15 Vystrelil i strely, kterýmiž je rozptýlil, a blýskání, jímž je porazil.
\par 16 I ukázaly se hlubiny morské, a odkryti jsou základové okršlku, pro zurivé kárání Hospodinovo, pro dmýchání vetru chrípí jeho.
\par 17 Poslav s výsosti, prijal mne, vytáhl mne z vod velikých.
\par 18 Vysvobodil mne od neprítele mého silného, od tech, kteríž mne nenávideli, ackoli silnejší mne byli.
\par 19 Predstihli mne v den trápení mého, ale Hospodin byl mi podpora.
\par 20 Kterýž vyvedl mne na prostranství, vysvobodil mne, nebo sobe oblíbil mne.
\par 21 Odplatil mi Hospodin podlé spravedlnosti mé, podlé cistoty rukou mých odplatil mi.
\par 22 Nebo jsem ostríhal cest Hospodinových, aniž jsem se bezbožne strhl Boha svého.
\par 23 Všickni zajisté soudové jeho jsou pred oblícejem mým, aniž jsem od kterých ustanovení jeho odstoupil.
\par 24 A tak byv dokonalý pred ním, šetril jsem, abych se nedopustil nepravosti.
\par 25 Protož odplatil mi Hospodin podlé spravedlnosti mé, vedlé cistoty mé pred ocima jeho.
\par 26 Ty, Pane, s milosrdným milosrdne nakládáš, a k uprímému upríme se máš.
\par 27 K sprostnému sprostne se ukazuješ, a s prevráceným prevrácene zacházíš.
\par 28 Lid pak ssoužený vysvobozuješ, ale pred vysokomyslnými oci své sklopuješ.
\par 29 Ty zajisté jsi svíce má, ó Hospodine. Hospodin jiste osvecuje temnosti mé.
\par 30 Nebo v tobe probehl jsem vojska, v Bohu svém preskocil jsem zed.
\par 31 Toho Boha silného cesta jest dokonalá, výmluvnosti Hospodinovy precištené; ont jest štít všech, kteríž doufají v neho.
\par 32 Nebo kdo jest Bohem krome Hospodina? A kdo jest skalou krome Boha našeho?
\par 33 Buh jest síla má i vojska mého, ont pusobí volnou cestu mou.
\par 34 Ciní nohy mé jako laní, a na vysokých místech mých postavuje mne.
\par 35 Cvicí ruce mé k boji, tak že lámi lucište ocelivá rukama svýma.
\par 36 Nebo dal mi štít spasení svého, a dobrotivost jeho zvelebila mne.
\par 37 Rozšíril kroky mé pode mnou, aby se nepodvrtly nohy mé.
\par 38 Honil jsem neprátely své a zahladil jsem je, aniž jsem se navrátil, dokudž jsem jich nevyplénil.
\par 39 Docela jsem je vyhubil a sprobodal jsem je, tak že nepovstanou; i padli pod nohy mé.
\par 40 Ty zajisté, Bože, prepásals mne udatností k boji, porazils pode mne ty, kteríž povstávají proti mne.
\par 41 Nýbrž dals mi šíji neprátel mých, tech, kteríž v nenávisti meli mne, a vyplénil jsem je.
\par 42 Ohlédali se, ale nebyl, kdo by vysvobodil, k Hospodinu, ale nevyslyšel jich.
\par 43 I potrel jsem je jako prach zeme, jako bláto na ulicích potlacil a rozptýlil jsem je.
\par 44 Ty jsi mne vytrhl z ruznic lidu mého, zachovals mne, abych byl za hlavu národum; lid neznámý mne sloužil.
\par 45 Cizozemci lhali mi, a jakž zaslechli, uposlechli mne.
\par 46 Cizozemci svadli, a trásli se i v ohradách svých.
\par 47 Živt jest Hospodin, a požehnaná skála má; protož at jest vyvyšován Buh, skála spasení mého,
\par 48 Buh silný, kterýž dává mi pomsty a podmanuje mi lidi.
\par 49 Vyvodíš mne z prostred neprátel mých, a nad povstávajícími proti mne vyvyšuješ mne, cloveka nepravého mne zbavuješ.
\par 50 Protož chváliti te budu, Hospodine, mezi národy, a jménu tvému žalmy zpívati budu.
\par 51 Ont jest hrad jistého spasení krále svého, a ten, kterýž ciní milosrdenství pomazanému svému Davidovi, i semeni jeho až na veky.

\chapter{23}

\par 1 Tato jsou pak poslední slova Davidova: Rekl David syn Izai, rekl, pravím, muž, kterýž jest velice zvýšený, pomazaný Boha Jákobova, a libý v zpevích Izraelských:
\par 2 Duch Hospodinuv mluvil skrze mne, a rec jeho jazykem mým vynesena.
\par 3 Rekl mi Buh Izraelský, mluvila skála Izraelská: Kdo panovati bude nad lidem, budet spravedlivý, panující v bázni Boží.
\par 4 Rovne jako bývá svetlo jitrní, když slunce vychází ráno bez oblaku, a jako mocí tepla a dešte roste bylina z zeme:
\par 5 Tak zajisté dum muj pred Bohem; nebo smlouvu vecnou ucinil se mnou, kteráž všelijak upevnena a ostríhána bude. A tot jest všecko mé spasení a všeliká útecha, že nic tak vzdelávati se nebude.
\par 6 Bezbožní pak všickni vypléneni budou jako trní, kteréž rukou bráno nebývá.
\par 7 Ale kdož by se ho chtel dotknouti, vezme železo a žerd, aneb ohnem docela spálí je tu, kdež zrostlo.
\par 8 Tato jsou jména nejudatnejších Davidových: Jošeb Bašebet Tachmonský, prední z vudcu, jehož rozkoš byla s kopím pojednou uderiti na osm set ku pobití jich.
\par 9 A po nem Eleazar syn Dodi, syna Achochi, mezi trmi silnými, kteríž byli s Davidem, když se opovážili proti Filistinským shromáždeným tam k boji, když již byli odtáhli muži Izraelští.
\par 10 Ten vstav, bil Filistinské, až ustala ruka jeho a ostála se pri meci. V ten den zajisté ucinil Hospodin vysvobození veliké, tak že se lid za ním navrátil, toliko aby koristi vzebral.
\par 11 Po nem pak byl Samma syn Age Hararský. Nebo když se byli shromáždili Filistinští v hromadu tu, kdež bylo díl rolí poseté šocovicí, a lid utekl pred Filistinskými:
\par 12 Tedy postavil se u prostred dílu toho, a vysvobodil jej, a porazil Filistinské. I ucinil Hospodin vysvobození veliké.
\par 13 Sstoupili také ti tri z tridcíti predních, a prišli ve žni k Davidovi do jeskyne Adulam, když vojsko Filistinské leželo v údolí Refaim.
\par 14 (Nebo David tehdáž byl v pevnosti své, a osazený lid Filistinských byl u Betléma.)
\par 15 Zechtelo se pak Davidovi vody, a rekl: Ó by mi nekdo dal píti vody z cisterny Betlémské, kteráž jest u brány!
\par 16 A protož probivše se ti tri udatní skrze vojsko Filistinských, navážili vody z cisterny Betlémské, kteráž byla u brány, kterouž nesli a donesli k Davidovi. On pak nechtel jí píti, ale obetoval ji Hospodinu,
\par 17 A rekl: Nedejž mi toho, ó Hospodine, abych to uciniti mel. Zdaliž to není krev mužu tech, kteríž šli, opováživše se života svého? I nechtel jí píti. To ucinili ti tri silní.
\par 18 Potom Abizai bratr Joábuv, syn Sarvie, byl prední mezi trmi, kterýž pozdvihl kopí svého proti trem stum, kteréž i pobil, a byl z tech trí nejslovoutnejší.
\par 19 Z tech trí on jsa nejslavnejší, byl knížetem jejich, a však onem trem nebyl rovný.
\par 20 Banaiáš také syn Joiaduv, syn muže udatného, velikých cinu, z Kabsael, ten porazil dva reky Moábské. Tentýž sstoupiv, zabil lva v jáme, když byl sníh.
\par 21 Ten také zabil muže Egyptského, muže ku podivení, a mel Egyptský v rukou kopí. I šel k nemu s holí, a vytrh kopí z ruky toho Egyptského, zabil ho kopím jeho.
\par 22 To ucinil Banaiáš syn Joiaduv, kterýž také slovoutný byl mezi temi trmi silnými.
\par 23 A ac byl mezi tridcíti nejslavnejší, však onem trem se nevrovnal. I ustanovil ho David nad drabanty svými.
\par 24 Azael také bratr Joábuv byl mezi tridcíti temito: Elchanan syn Doduv Betlémský,
\par 25 Samma Charodský, Elika Charodský,
\par 26 Chelez Faltický, Híra syn Ikeš Tekoitský,
\par 27 Abiezer Anatotský, Mebunnai Chusatský,
\par 28 Salmon Achochský, Maharai Netofatský,
\par 29 Cheleb syn Baany Netofatský, Ittai syn Ribai z Gabaa synu Beniamin,
\par 30 Banaiáš Faratonský, Haddai od potoku Gás,
\par 31 Abialbon Arbatský, Azmavet Barechumský,
\par 32 Eliachba Salbonský, z synu Jasen Jonata,
\par 33 Samma Hararský, Achiam syn Sárar Ararský,
\par 34 Elifelet syn Achasbai Machatského, Eliam syn Achitofeluv Gilonský,
\par 35 Chezrai Karmelský, Farai Arbitský,
\par 36 Igal syn Nátanuv z Soba, Báni Gádský,
\par 37 Zelek Ammonský, Nacharai Berotský, odenec Joába syna Sarvie,
\par 38 Híra Itrejský, Gareb Itrejský,
\par 39 Uriáš Hetejský. Všech tridceti a sedm.

\chapter{24}

\par 1 Tedy opet prchlivost Hospodinova popudila se proti Izraelovi; nebo byl ponukl satan Davida na ne, rka: Jdi, secti lid Izraelský a Judský.
\par 2 Protož rekl král Joábovi, hejtmanu vojska svého: Projdi i hned všecka pokolení Izraelská od Dan až k Bersabé, a sectete lid, abych vedel pocet jeho.
\par 3 Ale Joáb rekl králi: Pridejž Hospodin Buh tvuj k lidu, jakž ho koli mnoho, stokrát více, a to aby oci pána mého krále videly. Ale proc pán muj král jest toho tak žádostiv?
\par 4 A však premohla rec králova Joába i knížata vojska. Protož vyšel Joáb a knížata vojska od tvári královy, aby sectli lid Izraelský.
\par 5 A prepravivše se pres Jordán, položili se pri Aroer po pravé strane mesta, kteréž jest u prostred potoka Gád, a pri Jazer.
\par 6 Potom prišli do Galád a do zeme dolejší nové, odkudž šli do Dan Jáhan a do okolí Sidonského.
\par 7 Prišli také ku pevnosti Týru, a ke všem mestum Hevejských a Kananejských; odkudž šli na poledne zeme Judské do Bersabé.
\par 8 A když schodili všecku zemi, prišli po prebehnutí devíti mesícu a dvadcíti dní do Jeruzaléma.
\par 9 I dal Joáb pocet secteného lidu králi. Bylo pak lidu Izraelského osmkrát sto tisíc mužu silných, kteríž mohli vytrhnouti mec k bitve. Mužu také Judských petkrát sto tisíc.
\par 10 Potom padlo to težce na srdce Davidovi, když již sectl lid. I rekl David Hospodinu: Zhrešilt jsem težce, že jsem to ucinil, ale nyní, ó Hospodine, odejmi, prosím, nepravost služebníka svého, nebt jsem velmi bláznive ucinil.
\par 11 Když pak vstal David ráno, stalo se slovo Hospodinovo k Gádovi proroku, vidoucímu Davidovu, rkoucí:
\par 12 Jdi a rci Davidovi: Toto praví Hospodin: Trojít veci podávám, vyvol sobe jednu z nich, kteroužt bych ucinil.
\par 13 Protož prišed Gád k Davidovi, oznámil mu to, a rekl jemu: Chceš-li, aby prišel na tebe hlad za sedm let v zemi tvé? cili abys za tri mesíce utíkal pred neprátely svými, a oni aby te honili? cili aby za tri dni byl mor v zemi tvé? Pomysliž spešne, a viz, jakou mám odpoved dáti tomu, kterýž mne poslal.
\par 14 I rekl David k Gádovi: Úzkostmi sevrín jsem náramne; nechat, prosím, upadneme v ruku Hospodinovu, nebot jsou mnohá slitování jeho, jediné at v ruce lidské neupadám.
\par 15 A tak uvedl Hospodin mor na Izraele od jitra až do casu uloženého, a zemrelo jich z lidu od Dan až do Bersabé sedmdesáte tisíc mužu.
\par 16 A když vztáhl andel ruku svou na Jeruzalém, aby hubil jej, litoval Hospodin toho zlého, a rekl andelu, kterýž hubil lid: Dostit jest, již prestan. Andel pak Hospodinuv byl u humna Aravny Jebuzejského.
\par 17 I mluvil David Hospodinu, když uzrel andela, an bije lid, a rekl: Aj, já jsem zhrešil, a já jsem nepráve ucinil, ale tito, jsouce jako ovce, co ucinili? Necht jest, prosím, ruka tvá proti mne a proti domu otce mého.
\par 18 Potom navrátiv se Gád k Davidovi v ten den, rekl jemu: Vstup a vzdelej Hospodinu oltár na humne Aravny Jebuzejského.
\par 19 I šel David vedlé reci Gádovy, jakož byl prikázal Hospodin.
\par 20 A vyhlédaje Aravna, uzrel krále s služebníky jeho, an jdou k nemu; protož vyšed Aravna, poklonil se králi tvárí svou až k zemi.
\par 21 I rekl Aravna: Proc prišel pán muj král k služebníku svému? Odpovedel David: Abych koupil od tebe humno toto, na nemž bych vzdelal oltár Hospodinu, aby prestala rána tato v lidu.
\par 22 Opet rekl Aravna Davidovi: Necht vezme a obetuje pán muj král, což se mu za dobré vidí. Aj, ted volové k obeti zápalné, a smyky vozové i prípravy volu na drva.
\par 23 Všecky ty veci dával králík Aravna králi. Rekl ješte Aravna králi: Hospodin Buh tvuj zalibiž te sobe.
\par 24 Rekl pak král Aravnovi: Nikoli, ale radeji koupím od tebe, a zaplatím, aniž budu obetovati Hospodinu Bohu svému obeti zápalné darem dané. A tak koupil David humno a voly za padesáte lotu stríbra.
\par 25 A vzdelav tu David oltár Hospodinu, obetoval obeti zápalné a pokojné. I byl milostiv Hospodin zemi, a prestala zhouba v Izraeli.


\end{document}