\begin{document}

\title{1 Královská}

\chapter{1}

\par 1 Král pak David sstaral se a sešel vekem, a ackoli jej šaty prikrývali, však nemohl se zahrívati.
\par 2 I rekli jemu služebníci jeho: Necht pohledají pánu našemu králi devecky panny, kteráž by byla pri králi a opatrovala jej, a aby spala v lunu jeho, a zahrívala pána našeho krále.
\par 3 A protož hledajíce devecky krásné ve všech koncinách Izraelských, nalezli Abizag Sunamitskou, a privedli ji k králi.
\par 4 Kterážto devecka byla velmi krásná. I opatrovala krále a prisluhovala jemu, ale král jí nepoznal.
\par 5 Adoniáš pak syn Haggity pozdvihoval se, ríkaje: Ját budu kralovati. I najednal sobe vozu a jezdcu, a padesáte mužu, kteríž by behali pred ním.
\par 6 A nikdy ho nezarmoutil otec jeho, aby rekl: Proc jsi to ucinil? Nebo byl i on velmi krásné postavy, kteréhož porodila po Absolonovi.
\par 7 I mluvíval s Joábem synem Sarvie a s Abiatarem knezem, kteríž napomáhali Adoniášovi.
\par 8 Ale Sádoch knez a Banaiáš syn Joiaduv, též Nátan prorok a Simei i Rehi, a jiní prednejší, kteréž mel David, nebyli s Adoniášem.
\par 9 Tedy nabil Adoniáš ovcí a volu a krmného dobytka u kamene Zohelet, kterýž byl u studnice Rogel, a pozval všech bratrí svých synu královských, i všech mužu Judských, služebníku králových.
\par 10 Nátana však proroka a Banaiáše a jiných prednejších mužu, ani Šalomouna bratra svého nepozval.
\par 11 I mluvil Nátan k Betsabé matce Šalomounove, rka: Neslyšela-lis, že kraluje Adoniáš syn Haggity, a pán náš David nic neví o tom?
\par 12 Protož nyní pod, prosím, a dámt radu, kterouž vysvobodíš život svuj, i život syna svého Šalomouna.
\par 13 Jdiž, a vejda k králi Davidovi, rci jemu: Zdaliž jsi ty, pane muj králi, neprisáhl služebnici své, rka: Šalomoun syn tvuj kralovati bude po mne, a ont sedeti bude na stolici mé? Procež tedy kraluje Adoniáš?
\par 14 A aj, když ty ješte tam mluviti budeš s králem, prijdut za tebou a doplním rec tvou.
\par 15 A tak vešla Betsabé k králi do pokoje. Král pak již byl velmi starý, a Abizag Sunamitská prisluhovala králi.
\par 16 Tedy Betsabé naklonivši hlavy, poklonila se králi. Jížto rekl král: Což chceš?
\par 17 I odpovedela jemu: Pane muj, ty jsi prisáhl skrze Hospodina Boha svého služebnici své, rka: Šalomoun syn tvuj kralovati bude po mne, a on sedeti bude na stolici mé.
\par 18 A nyní, hle, Adoniáš kraluje, ty pak nyní, pane muj králi, nic nevíš o tom.
\par 19 Nebo nabiv volu a krmného dobytka a ovec množství, pozval všech synu královských, i Abiatara kneze a Joába hejtmana vojska, Šalomouna pak služebníka tvého nepozval.
\par 20 Ješto ty, pane muj králi, víš, že oci všeho Izraele na tebe jsou obráceny, abys jim oznámil, kdo by sedeti mel na stolici pána mého krále po tobe.
\par 21 Sic jinak bude to, když usne pán muj král s otci svými, že já a syn muj Šalomoun budeme jako hríšníci.
\par 22 A aj, když ješte ona mluvila s králem, prišel Nátan prorok.
\par 23 I oznámili králi, rkouce: Aj, ted Nátan prorok. A tak všel pred krále, a poklonil se jemu tvárí svou až k zemi.
\par 24 Zatím rekl Nátan: Pane muj králi, ty-lis povedel: Adoniáš kralovati bude po mne, a on sedeti bude na stolici mé?
\par 25 Nebo odšed dnes, nabil volu a krmného dobytka, i ovec množství, a sezval všecky syny královy a hejtmany vojska, i Abiatara kneze. A hle, oni jedí a pijí pred ním, a praví: Živ bud král Adoniáš.
\par 26 Mne pak služebníka tvého, a Sádocha kneze, a Banaiáše syna Joiadova, a Šalomouna služebníka tvého nepozval.
\par 27 Tak-liž jest ode pána mého krále narízeno? Mne jsi zajisté neoznámil služebníku svému, kdo bude sedeti na stolici pána mého krále po tobe?
\par 28 A odpovídaje král David, rekl: Zavolejte mi Betsabé. Kteráž všedši pred krále, stála pred ním.
\par 29 I prisáhl král, rka: Živt jest Hospodin, kterýž vysvobodil duši mou ze všelikých úzkostí,
\par 30 Že jakož jsem prisáhl tobe skrze Hospodina Boha Izraelského, rka, že syn tvuj Šalomoun kralovati bude po mne, a on sedeti bude na stolici mé místo mne, takt uciním dnes.
\par 31 A naklonivši hlavy Betsabé tvárí k zemi, poklonu ucinila králi a rekla: Živ bud pán muj král David na veky.
\par 32 Rekl ješte král David: Zavolejte ke mne Sádocha kneze a Nátana proroka, a Banaiáše syna Joiadova. Kterížto vešli pred krále.
\par 33 I rekl jim král: Vezmete s sebou služebníky pána svého, a vsadte Šalomouna syna mého na mezkyni mou, a vedte ji k Gihonu.
\par 34 I pomaže ho tam Sádoch knez a Nátan prorok za krále nad Izraelem, a troubiti budete troubou, a díte: Živ bud král Šalomoun.
\par 35 Potom vstoupíte zase, jdouce za ním. Kterýž prijda, sedeti bude na stolici mé, a on kralovati bude místo mne; nebt jsem jemu prikázal, aby byl vudcí nad Izraelem i nad Judou.
\par 36 A odpovídaje Banaiáš syn Joiaduv králi, rekl: Amen. Zpusobiž to Hospodin, Buh pána mého krále.
\par 37 A jakož byl Hospodin se pánem mým králem, tak budiž i s Šalomounem, a zvelebiž stolici jeho více, než stolici pána mého krále Davida.
\par 38 A tak šli dolu Sádoch knez a Nátan prorok, a Banaiáš syn Joiaduv, Cheretejští také i Peletejští, a vsadili Šalomouna na mezkyni krále Davida, a dovedli jej k Gihonu.
\par 39 Vzal také Sádoch knez roh oleje z stánku, a pomazal Šalomouna. Potom troubili trubou, a rekl všecken lid: Živ bud král Šalomoun.
\par 40 I vstoupil všecken lid za ním; kterýžto lid prozpevoval s plésáním, a veselil se radostí velikou, tak až se všudy rozléhalo od zvuku jejich.
\par 41 Což slyše Adoniáš i všickni pozvaní, kteríž byli s ním, (byli pak již pojedli), slyše také i Joáb zvuk trouby, rekl: Jaký to zvuk v meste hlucících?
\par 42 A když on ješte mluvil, aj, Jonata syn Abiatara kneze prišel, jemuž rekl Adoniáš: Pristup sem, nebo jsi muž udatný a dobré poselství neseš.
\par 43 Tedy odpovídaje Jonata, rekl Adoniášovi: Práve jižte pán náš král David ustanovil Šalomouna za krále.
\par 44 Nebo poslal s ním král Sádocha kneze a Nátana proroka a Banaiáše syna Joiadova, ano i Cheretejské a Peletejské, a vsadili jej na mezkyni královskou.
\par 45 I pomazali jej Sádoch knez a Nátan prorok za krále u Gihonu, a vstupovali odtud s radostí, tak že hlucelo mesto. Tot jest zvuk ten, kterýž jste slyšeli.
\par 46 Ano již i dosedl Šalomoun na stolici královskou.
\par 47 Nýbrž i služebníci královští prišli, aby dobrorecili pánu našemu, králi Davidovi, rkouce: Uciniž slavné Buh tvuj jméno Šalomounovo nad jméno tvé, a zvelebiž stolici jeho nad stolici tvou. I poklonil se král na ložci svém.
\par 48 Ano i král takto rekl: Požehnaný Hospodin Buh Izraelský, kterýž dal dnes sedícího na trunu mém, a abych na to ocima svýma hledel.
\par 49 I predešeni jsou, a vstavše všickni ti pozvaní, kteríž byli pri Adoniášovi, odešli jeden každý cestou svou.
\par 50 Adoniáš také boje se Šalomouna, vstav, odšel a chytil se rohu oltáre.
\par 51 I povedeli Šalomounovi, rkouce: Aj, Adoniáš bojí se krále Šalomouna, a hle, drží se rohu oltáre a praví: Necht mi prisáhne dnes král Šalomoun, že nezabije mne služebníka svého mecem.
\par 52 I rekl Šalomoun: Bude-li muž statecný, nespadnet vlas s neho na zem; paklit co zlého na nem nalezeno bude, umret.
\par 53 Poslal tedy král Šalomoun, aby jej odvedli od oltáre. Kterýž prišed, poklonil se králi Šalomounovi. Jemuž rekl Šalomoun: Jdiž do domu svého.

\chapter{2}

\par 1 Když se pak cas smrti Davidovy priblížil, dával prikázaní Šalomounovi synu svému, rka:
\par 2 Ját již odcházím podlé zpusobu všech lidí. Proto posilniž se a mej se zmužile,
\par 3 A ostríhej prikázaní Hospodina Boha svého, chode po cestách jeho, a zachovávaje ustanovení jeho, rozkázaní jeho, i soudy jeho i svedectví jeho, jakož psáno jest v zákone Mojžíšove, at by se štastne vedlo, což bys koli cinil, i všecko, k cemuž bys se obrátil,
\par 4 A aby naplnil Hospodin slovo své, kteréž mi zaslíbil, mluve: Budou-li ostríhati synové tvoji cesty své, chodíce prede mnou v pravde z celého srdce svého a z celé duše své, tak mluve: Že nebude vyplénen po tobe muž z stolice Izraelské.
\par 5 Také i tom víš, co jest mi ucinil Joáb syn Sarvie, totiž, co ucinil dvema hejtmanum vojsk Izraelských, Abnerovi synu Ner, a Amazovi synu Jeter, že je zamordoval, dopustiv se vraždy lstivé v cas pokoje, a zmazal krví zrádne vylitou pás svuj rytírský, kterýž mel na bedrách svých, a obuv svou, kterouž mel na nohách svých.
\par 6 Protož zachováš se podlé moudrosti své, a nedáš sstoupiti šedinám jeho v pokoji do hrobu.
\par 7 S syny pak Barzillai Galádského uciníš milosrdenství, tak aby jídali spolu s jinými pri stole tvém; nebo tak podobne ke mne prišli, když jsem utíkal pred Absolonem bratrem tvým.
\par 8 Hle, máš také v moci své Semei syna Gery Beniaminského z Bahurim, kterýž mi zlorecil zlorecením velikým v ten den, když jsem šel do Mahanaim, ackoli potom vyšel mi vstríc k Jordánu, jemuž jsem prisáhl skrze Hospodina, rka: Nezabiji te mecem.
\par 9 Nyní však neodpouštej jemu. A ponevadž jsi muž opatrný, víš, jak bys k nemu pristoupiti mel, abys uvedl šediny jeho se krví do hrobu.
\par 10 A tak usnul David s otci svými, a pohrben jest v meste Davidove.
\par 11 Dnu pak, v nichž kraloval David nad Izraelem, bylo ctyridceti let. V Hebronu kraloval sedm let, a v Jeruzaléme kraloval tridceti a tri léta.
\par 12 I dosedl Šalomoun na stolici Davida otce svého, a utvrzeno jest království jeho velmi.
\par 13 Tedy prišel Adoniáš syn Haggity k Betsabé matce Šalomounove. Kterážto rekla: Pokojný-li jest príchod tvuj? I odpovedel: Pokojný.
\par 14 Rekl dále: Mám k tobe rec. I rekla: Mluv.
\par 15 Tedy rekl: Ty víš, že mé bylo království, a na mne obrátili všickni Izraelští tvár svou, abych kraloval, ale preneseno jest království na bratra mého, nebo od Hospodina usouzeno mu bylo.
\par 16 Nyní pak jediné veci žádám od tebe, necht nejsem oslyšán. Kteráž rekla jemu: Mluv.
\par 17 A on rekl: Rci, prosím, Šalomounovi králi, (nebot neoslyší tebe), at mi dá Abizag Sunamitskou za manželku.
\par 18 Odpovedela Betsabé: Dobre, ját budu mluviti o tebe s králem.
\par 19 Protož vešla Betsabé k králi Šalomounovi, aby mluvila s ním o Adoniáše. I povstal král proti ní, a pokloniv se jí, posadil se na stolici své; rozkázal také postaviti stolici matce své. I posadila se po pravici jeho,
\par 20 A rekla: Jediné veci neveliké já žádám od tebe, necht nejsem oslyšána. I rekl jí král: Žádej, matko má, nebot neoslyším tebe.
\par 21 Kteráž rekla: Necht jest dána Abizag Sunamitská Adoniášovi bratru tvému za manželku.
\par 22 A odpovídaje král Šalomoun, rekl matce své: Proc ty jen žádáš za Abizag Sunamitskou Adoniášovi? Žádej jemu království, ponevadž on jest bratr muj starší nežli já, a má po sobe Abiatara kneze a Joába syna Sarvie.
\par 23 I prisáhl král Šalomoun skrze Hospodina, rka: Toto at mi uciní Buh a toto pridá, že sám proti sobe mluvil Adoniáš rec tuto.
\par 24 Protož nyní, živt jest Hospodin, kterýž mne utvrdil, a posadil na stolici Davida otce mého, a kterýž mi vzdelal dum, jakož byl mluvil, že dnes umríti musí Adoniáš.
\par 25 A tak poslal král Šalomoun Banaiáše syna Joiadova, kterýž uderil na nej, tak že umrel.
\par 26 Abiatarovi knezi také rekl král: Jdi do Anatot k rolí své, nebo jsi hoden smrti. Ale nyní nezabiji tebe, ponevadž jsi nosil truhlu Hospodinovu pred Davidem otcem mým, a snášel jsi všecka ta ssoužení, kteráž snášel otec muj.
\par 27 I svrhl Šalomoun Abiatara s knežství Hospodinova, aby naplnil rec Hospodinovu, kterouž byl mluvil proti domu Elí v Sílo.
\par 28 Donesla se pak povest ta Joába, (nebo Joáb postoupil po Adoniášovi, ackoli po Absolonovi byl nepostoupil). Protož utekl Joáb k stánku Hospodinovu, a chytil se rohu oltáre.
\par 29 I oznámeno králi Šalomounovi, že utekl Joáb k stánku Hospodinovu, a že jest u oltáre. Tedy poslal Šalomoun Banaiáše syna Joiadova, rka: Jdi, obor se na nej.
\par 30 A když prišel Banaiáš k stánku Hospodinovu a rekl jemu: Takto praví král: Vyjdi, odpovedel jemu: Nikoli, ale tuto umru. V tom oznámil zase Banaiáš králi rec tu, rka: Takto mluvil Joáb, a tak mi odpovedel.
\par 31 Na to rekl jemu král: Ucin, jakž mluvil, a uder na nej, a pochovej jej, a odejmeš krev nevinnou, kterouž vylil Joáb, ode mne i od domu otce mého.
\par 32 A navrátí Hospodin krev jeho na hlavu jeho; nebo oboriv se na dva muže spravedlivejší a lepší, než jest sám, zamordoval je mecem bez vedomí otce mého Davida, Abnera syna Ner, hejtmana vojska Izraelského, a Amazu syna Jeter, hejtmana vojska Judského.
\par 33 A tak navrátí se krev jejich na hlavu Joábovu, a na hlavu semene jeho na veky, Davidovi pak a semeni jeho, domu jeho a stolici jeho bud pokoj až na veky od Hospodina.
\par 34 A protož vyšed Banaiáš syn Joiaduv, oboril se na nej, a zabil jej. A pochován jest v dome svém na poušti.
\par 35 I ustanovil král Banaiáše syna Joiadova místo neho nad vojskem, a Sádocha kneze ustanovil král místo Abiatara.
\par 36 Zatím poslav král, povolal Semei, a rekl jemu: Ustavej sobe dum v Jeruzaléme, a tu zustávej a nevycházej odtud ani sem ani tam.
\par 37 Nebo v kterýkoli den vyjdeš, a prejdeš pres potok Cedron, vez jistotne, že umreš; krev tvá bude na hlavu tvou.
\par 38 I dí Semei k králi: Dobrát jest ta rec. Jakož mluvil pán muj král, takt uciní služebník tvuj. A tak bydlil Semei v Jeruzaléme po mnohé dny.
\par 39 Stalo se pak po trech letech, že utekli dva služebníci Semei k Achisovi synu Maachy, králi Gát. I povedeli to Semei, rkouce: Hle, služebníci tvoji jsou v Gát.
\par 40 Procež vstav Semei, osedlal osla svého, a jel do Gát k Achisovi, aby hledal služebníku svých. I navrátil se Semei, a privedl zase služebníky své z Gát.
\par 41 Povedíno pak Šalomounovi, že odšel Semei z Jeruzaléma do Gát, a že se navrátil.
\par 42 Tedy poslav král, povolal Semei a rekl jemu: Zdaliž jsem te prísahou nezavázal skrze Hospodina, a osvedcil jsem proti tobe, rka: V kterýkoli den vyjdeš, a sem i tam choditi budeš, vez jistotne, že smrtí umreš? A rekl jsi mi: Dobrá jest ta rec, kterouž jsem slyšel.
\par 43 Procež jsi tedy neostríhal prísahy Hospodinovy a prikázaní, kteréž jsem prikázal tobe?
\par 44 Rekl také král k Semei: Ty víš všecko zlé, kteréhož povedomo jest srdce tvé, co jsi ucinil Davidovi otci mému, ale Hospodin uvedl zase to tvé zlé na hlavu tvou.
\par 45 Král pak Šalomoun bude požehnaný, a stolice Davidova stálá pred Hospodinem až na veky.
\par 46 I prikázal král Banaiášovi synu Joiadovu, kterýž vyšed, oboril se na nej, tak že umrel. A tak utvrzeno jest království v ruce Šalomounove.

\chapter{3}

\par 1 Spríznil se pak Šalomoun s Faraonem králem Egyptským, nebo pojal dceru Faraonovu a uvedl ji do mesta Davidova, dokudž nedostavel domu svého, a domu Hospodinova i zdi Jeruzalémské vukol.
\par 2 Toliko lid obetoval na výsostech, proto že nebyl vystaven dum jménu Hospodinovu až do tech dnu.
\par 3 Nebo miloval Šalomoun Hospodina, chode v prikázaních Davida otce svého, a toliko na tech výsostech obetoval a kadil.
\par 4 Protož šel král do Gabaon, aby tam obetoval; ta výsost zajisté byla nejvetší. Tisíc obetí zápalných obetoval Šalomoun na tom oltári.
\par 5 Ukázal se pak Hospodin v Gabaon Šalomounovi ve snách noci té, a rekl Buh: Žádej zackoli, a dám tobe.
\par 6 I rekl Šalomoun: Ty jsi ucinil s služebníkem svým Davidem, otcem mým, milosrdenství veliké, když chodil pred tebou v pravde a v spravedlnosti, a v uprímnosti srdce stál pri tobe. Ovšem zachoval jsi jemu zvláštní toto milosrdenství, že jsi dal jemu syna, kterýž by sedel na stolici jeho, jakž se to dnes vidí.
\par 7 Ackoli pak nyní, ó Hospodine Bože muj, ty jsi ustanovil služebníka svého králem místo Davida otce mého, já však jsa velmi mladý, neumím vycházeti ani vcházeti.
\par 8 Služebník, pravím, tvuj jest u prostred lidu tvého, kterýž jsi vyvolil, lidu velikého, kterýž nemuže ani secten ani sepsán býti pro množství.
\par 9 Dejž tedy služebníku svému srdce rozumné, aby soudil lid tvuj, a aby rozeznal mezi dobrým a zlým; nebo kdo bude moci souditi tento lid tvuj tak mnohý?
\par 10 I líbilo se to Hospodinu, že žádal Šalomoun za tu vec.
\par 11 A rekl jemu Buh: Proto že jsi žádal veci takové, a neprosils sobe za dlouhý vek, aniž jsi žádal sobe bohatství, aniž jsi žádal bezživotí neprátel svých, ale žádal jsi sobe rozumnosti, abys slýchati umel rozepre,
\par 12 Aj, ucinil jsem vedlé reci tvé, aj, dalt jsem srdce moudré a rozumné, tak že rovného tobe nebylo pred tebou, ani po tobe aby nepovstal rovný tobe.
\par 13 K tomu i to, zacež jsi nežádal, dal jsem tobe, totiž bohatství a slávu, tak aby nebylo rovného tobe žádného mezi králi po všecky dny tvé.
\par 14 Pres to jestliže choditi budeš po cestách mých, ostríhaje ustanovení mých a prikázaní mých, jako chodil David otec tvuj, prodlím i dnu tvých.
\par 15 A když procítil Šalomoun, a aj, byl sen. I prišel do Jeruzaléma a stál pred truhlou smlouvy Hospodinovy, a obetoval obeti zápalné a obeti pokojné; udelal také hody všechnem služebníkum svým.
\par 16 Tedy prišly dve ženy hokyne k králi, a stály pred ním.
\par 17 I rekla jedna z tech žen: Prosím, pane muj, já a žena tato bydlíme v jednom dome. I porodila jsem u ní v témž dome.
\par 18 Potom stalo se dne tretího po porodu mém, že porodila také žena tato, a byly jsme spolu. Nebylo žádného cizího s námi v dome, krome nás dvou v témž dome.
\par 19 Umrel pak syn ženy této v noci, nebo speci, udávila ho.
\par 20 A vstavši o pul noci, vzala syna mého ode mne, když spala služebnice tvá, a položila jej v lunu svém, a syna svého mrtvého položila do luna mého.
\par 21 Ale když jsem vstala ráno, abych prikojila syna svého, a aj, mrtvý. Na kteréhož když jsem ráno pilneji pohledela, a aj, nebyl syn muj, kteréhož jsem porodila.
\par 22 I rekla žena druhá: Není tak, ale syn muj jest ten živý, a ten mrtvý jest syn tvuj. Ona pak rekla: Nikoli, ale syn tvuj jest ten mrtvý, a syn muj jest ten živý. A tak se hádaly pred králem.
\par 23 I rekl král: Tato praví: Ten živý jest syn muj, a syn tvuj jest ten mrtvý. Tato zase praví: Nenít tak, ale syn tvuj jest ten mrtvý, a syn muj jest ten živý.
\par 24 Protož rekl král: Prineste mi mec. I prinesli mec pred krále.
\par 25 Tedy rekl král: Rozetnete to díte živé na dvé, a dejte jednu polovici jedné, a polovici druhou druhé.
\par 26 Ale žena, jejíž syn byl ten, kterýž živ zustal, mluvila králi, (nebo pohnula se streva její nad synem jejím), a rekla: Prosím, pane muj, dejte jí nemluvnátko to živé, a nikoli nezabijejte ho. Druhá pak rekla: Necht není ani mne ani tobe, rozetnete.
\par 27 K cemuž odpovídaje král, rekl: Dejtež této díte to živé, a nikoli nezabijejte ho, onat jest matka jeho.
\par 28 Tedy uslyšavše všickni Izraelští soud tento, kterýž vynesl král, báli se krále; nebo videli, že moudrost Boží jest v srdci jeho k vykonávání soudu.

\chapter{4}

\par 1 A tak král Šalomoun byl králem nade vším Izraelem.
\par 2 Tato pak byla knížata jeho: Azariáš syn Sádochuv knížetem,
\par 3 Elichoref a Achiáš synové Sísovi byli písari, Jozafat syn Achiluduv kanclérem,
\par 4 A Banaiáš syn Joiaduv nad vojskem, Sádoch pak a Abiatar knežími,
\par 5 A Azariáš syn Nátanuv nad úredníky, a Zábud syn Nátanuv nejvyšší rada královská,
\par 6 A Achisar vládar domu, Adoniram pak syn Abdy nad vybraným lidem.
\par 7 Mel také Šalomoun dvanácte vládaru nade vším Izraelem, kteríž opatrovali potravou krále i dum jeho. Každého roku za mesíc jeden každý mel opatrovati krále.
\par 8 A tato jsou jména jejich: Syn Chur na hore Efraim;
\par 9 Syn Deker v Makaz a v Salbim, a v Betsemes a v Elon Betchanan;
\par 10 Syn Chesed v Arubot, jehož bylo Socho i všecka zeme Chefer;
\par 11 Syn Abinadabuv, jehož byly všecky konciny Dor, a mel Tafat dceru Šalomounovu za manželku;
\par 12 Baana syn Achiluduv, jehož byl Tanach a Mageddo, i všecken Betsan, kterýž jest vedlé Sartan pod Jezreelem, od Betsan až do Abelmehula a až za Jekmaam;
\par 13 Syn Geber v Rámot Galád, jehož byly vsi Jair, syna Manassesova, kteréž jsou v Galád, jehož byla krajina Argob, kteráž jest v Bázan, šedesáte mest velikých hrazených a závritých;
\par 14 Achinadab syn Iddo v Mahanaim;
\par 15 Achimaas v Neftalím, on také pojal Basemat dceru Šalomounovu za manželku;
\par 16 Baana syn Chusai v Asser a v Alot;
\par 17 Jozafat syn Paruach v Izachar;
\par 18 Semei syn Ela v Beniamin;
\par 19 Geber syn Uri v zemi Galád, v zemi Seona, krále Amorejského, a Oga krále Bázan; ten sám vládarem jedním predstaven byl té zemi.
\par 20 Tehdáž Juda a Izrael rozmnoženi jsouce jako písek pri mori v množství, jedli a pili, a veselili se.
\par 21 Nebo Šalomoun panoval nade všemi královstvími od reky až k zemi Filistinské, a až k koncinám Egyptským. I prinášeli dary a sloužili Šalomounovi po všecky dny života jeho.
\par 22 Vycházelo pak ku potrave Šalomounovi na každý den tridceti mer beli, a šedesáte mer mouky obecné,
\par 23 Deset volu krmných a dvadceti volu pastevných, a sto ovec, krome jelenu, srn, buvolu a ptactva vykrmeného.
\par 24 On zajisté panoval všudy s této strany reky od Tipsach až do Gázy nade všemi králi, kteríž byli pred rekou, a mel pokoj se všech stran vukol.
\par 25 I bydlil Juda a Izrael bezpecne, jeden každý pod svým vinným kmenem a pod svým fíkem, od Dan až do Bersabé, po všecky dny Šalomounovy.
\par 26 Mel také Šalomoun ctyridceti tisíc koní na stání k vozum svým, a dvanácte tisíc jízdných.
\par 27 A tak opatrovali ti úredníci krále Šalomouna i všecky, kteríž pricházeli k stolu krále Šalomouna, jeden každý za mesíc svuj, nedopouštejíce, aby v cem nedostatek býti mel.
\par 28 Jecmene také a slámy pro kone a pro mezky dodávali k tomu místu, kdež byl král, jeden každý, jakž mu uloženo bylo.
\par 29 Nadto dal Buh moudrost Šalomounovi a prozretelnost velikou náramne, a širokost mysli, jako jest písku na brehu morském.
\par 30 Nebo vetší byla moudrost Šalomounova, než moudrost všech národu východních, a než všeliká moudrost Egyptských.
\par 31 Nýbrž moudrejší byl nad všecky lidi, až i nad Etana Ezrachitského, též nad Hémana, a Kalkole i Darda, syny Máchol, a rozneslo se jméno jeho po všech národech vukol.
\par 32 Složil také tri tisíce prísloví, a písnicek jeho bylo tisíc a pet.
\par 33 Vypsal též i o stromích, pocna od cedru, kterýž jest na Libánu, až do mchu, kterýž roste na zdi; psal i o hovadech a o ptácích, a zemeplazích a o rybách.
\par 34 Protož pricházeli ze všech národu poslouchati moudrosti Šalomounovy, i ode všech králu zeme, kteríž slyšeli o moudrosti jeho.

\chapter{5}

\par 1 Poslal pak Chíram král Tyrský služebníky své k Šalomounovi, uslyšav, že ho pomazali za krále na místo otce jeho; nebo miloval Chíram Davida po všecky dny.
\par 2 Zase poslal Šalomoun k Chíramovi, rka:
\par 3 Ty víš, že David otec muj nemohl vystaveti domu jménu Hospodina Boha svého pro války, kteréž jej obklicovaly, dokudž Hospodin nepodložil neprátel pod nohy jeho.
\par 4 Ale nyní Hospodin Buh muj dal mi odpocinutí všudy vukol, není žádného protivníka, ani outoku nebezpecného.
\par 5 Z té príciny, aj, úmysl mám staveti dum jménu Hospodina Boha svého, jakož mluvil Hospodin Davidovi otci mému, rka: Syn tvuj, kteréhož posadím místo tebe na stolici tvé, ont vystaví dum ten jménu mému.
\par 6 A protož nyní rozkaž, at mi nasekají cedru na Libánu. Budou pak služebníci moji s služebníky tvými, a mzdu služebníku tvých dám tobe, tak jakž díš. Nebo sám víš, že mezi námi není žádného, kdož by umel sekati dríví, jako jsou Sidonští.
\par 7 Stalo se tedy, když uslyšel Chíram ta slova Šalomounova, že se zradoval náramne, a rekl: Požehnaný Hospodin budiž nyní,kterýž dal Davidovi syna moudrého nad tím lidem tak mnohým.
\par 8 I poslal Chíram k Šalomounovi, rka: Vyrozumel jsem, oc jsi poslal ke mne; ját uciním všecku vuli tvou z strany dríví cedrového i jedlového.
\par 9 Služebníci moji svezou je s Libánu až k mori, a dám je v vorích splaviti po mori až k místu tomu, kteréž mi ukážeš, a tu je složím; ty pak pobéreš je a naplníš také vuli mou, dodávaje potravy celedi mé.
\par 10 A tak dával Chíram Šalomounovi dríví cedrového a dríví jedlového, jakkoli mnoho chtel.
\par 11 Šalomoun také dával Chíramovi dvadceti tisíc mer pšenice ku pokrmu celedi jeho, a dvadceti tisíc mer oleje vytlaceného. To dával Šalomoun Chíramovi každého roku.
\par 12 Když tedy dal Hospodin moudrost Šalomounovi, jakož mu byl zaslíbil, a byl pokoj mezi Chíramem a mezi Šalomounem, tak že mezi sebou ucinili smlouvu:
\par 13 Rozkázal král Šalomoun vybírati osoby ze všeho Izraele, a bylo vybraných tridceti tisíc mužu.
\par 14 Z nichž posílal na Libán deset tisícu na každý mesíc, jedny po druhých. Jeden mesíc bývali na Libánu, a dva mesíce doma, Adoniram pak byl ustaven nad temi vybranými.
\par 15 Mel také Šalomoun sedmdesáte tisíc nosicu, a osmdesáte tisíc tech, kteríž tesali na hore,
\par 16 Krome predních vládaru Šalomounových, kterýchž bylo nad dílem tri tisíce a tri sta. Ti predstaveni byli lidem, kteríž delali.
\par 17 I prikázal král, aby navozili kamení velikého, kamení nákladného k založení toho domu, a kamení tesaného,
\par 18 Kteréž tesali kameníci Šalomounovi a kameníci Chíramovi a Giblictí. A tak pripravovali dríví i kamení k stavení domu toho.

\chapter{6}

\par 1 I stalo se léta ctyrstého osmdesátého po vyjití synu Izraelských z zeme Egyptské, že Šalomoun mesíce Ziv, (jenž jest mesíc druhý), ctvrtého léta kralování svého nad Izraelem, pocal staveti domu Hospodinova.
\par 2 Dum pak, kterýž stavel král Šalomoun Hospodinu, byl zdélí šedesáti loktu, a zšírí dvadcíti, a zvýší tridcíti loktu.
\par 3 Sínce také byla pred chrámem dvadcíti loktu zdélí vedlé širokosti domu, a desíti loktu zšírí pred domem.
\par 4 Udelal také v dome okna porozšírená vnitr, ale possoužená zevnitr.
\par 5 Vzdelal též pri zdi domu pavlace všudy vukol vedlé zdi domu, okolo chrámu a svatyne svatých, a nadelal pokojíku všudy vukol.
\par 6 Pavlace spodní širokost byla peti loket, prostrední širokost šesti loket, a tretí širokost sedmi loket; nebo byl zdelal ústupky ve zdi domu zevnitr všudy vukol, aby trámové nebyli vpouštíni do zdi domu.
\par 7 Když pak ten dum staven byl, z kamení hotového, tak privezeného staveli. Ani kladiva ani sekery, ani jakého nástroje železného nebylo slyšeti v dome, když byl staven.
\par 8 Dvére k prostredním pokojum byly po pravé strane domu, jimiž po šneku se chodilo k tem prostredním, a z prostredních k tretím.
\par 9 A tak vystavel dum ten, a dokonal jej, a prikryl jej krokvemi ohnutými a  prkny cedrovými.
\par 10 Vystavel i pavlace ty vukol všeho domu. Peti loket zvýši byla každá z nich, a pripojeny byly k domu trámy cedrovými.
\par 11 Stala se pak rec Hospodinova k Šalomounovi, rkoucí:
\par 12 Tot jest ten dum, kterýž ty stavíš. Jestliže budeš choditi v ustanoveních mých, a soudy mé vykonávati, a ostríhati všech prikázaní mých, chode v nich, tedy upevním slovo své s tebou, kteréž jsem mluvil Davidovi otci tvému.
\par 13 A budu bydliti u prostred synu Izraelských, a neopustím lidu svého Izraelského.
\par 14 A tak vystavel Šalomoun ten dum, a dokonal jej.
\par 15 A otafloval zdi domu vnitr prkny cedrovými, od podlahy domu až k stropu obložil drívím vnitr; položil také podlahu domu prkny jedlovými.
\par 16 Udelal také prehražení na dvadceti loket od jedné strany domu k druhé, z prken cedrových, od podlahy až do stropu. A tak udelal u vnitrku príbytek, aby byl svatyne svatých.
\par 17 Procež ctyridcíti loktu byl dum, jenž jest chrám prední.
\par 18 A na tom cedrovém domu vnitr otaflování byly rezby, nápodobné tykvím planým a kvetum otevreným. Všecko z cedru bylo, tak že ani kamene nebylo videti.
\par 19 Svatyni pak svatých v dome vnitr pripravil, aby tam postavena byla truhla smlouvy Hospodinovy.
\par 20 Kterážto svatyne svatých vnitr byla dvadcíti loktu zdélí, a dvadcíti loktu zšírí, též dvadcíti loktu zvýší, a obložil ji zlatem nejcistším. Oltár také cedrový obložil.
\par 21 Tak obložil Šalomoun príbytek ten zlatem nejcistším, a ucinil rozdelující stenu a provazy zlaté pred svatyní svatých, kterouž také obložil zlatem.
\par 22 Anobrž všecken dum obložil zlatem, žádné strany nepomíjeje; též všecken oltár, kterýž byl pred svatyní svatých, obložil zlatem.
\par 23 Udelal také v svatyni svatých dva cherubíny z dríví olivového desíti loket zvýší.
\par 24 A bylo na pet loket krídlo cherubína jedno, a na pet loket krídlo cherubína druhé; deset loket bylo od konce krídla jednoho až k konci krídla druhého.
\par 25 Tak na deset loket byl i cherubín druhý; míru jednostejnou a rezbu jednostejnou meli oba cherubínové.
\par 26 Vysokost cherubína jednoho byla desíti loket, a tolikéž cherubína druhého.
\par 27 I postavil ty cherubíny u prostred domu vnitrního, tak že roztáhli krídla svá, a dotýkalo se krídlo jednoho jedné steny, též krídlo cherubína druhého dotýkalo se druhé steny; krídla pak jejich u prostred domu spolu se dotýkala.
\par 28 Obložil také ty cherubíny zlatem.
\par 29 Presto i všecky steny domu vukol ozdobil rezbami cherubínu a palm a rozvitých kvetu, vnitr i zevnitr.
\par 30 I podlahu domu položil zlatem vnitr i zevnitr.
\par 31 Udelal též, kudy vcházeli do svatyne svatých, dvére z dríví olivového, a podvoje i s verejemi byly petihrané.
\par 32 Oboje pak ty dvére byly z dríví olivového, kteréž ozdobil rezbami cherubínu a palm a rozvitých kvetu, a obložil zlatem; potáhl také i cherubíny a palmy zlatem.
\par 33 Podobne udelal i u vrat chrámových dvére z dríví olivového ctverhrané.
\par 34 A oboje dvére byly z dríví jedlového. Na dve strany jedny dvére otvíraly se, též na dve strany druhé dvére se otvíraly.
\par 35 I vyrezal na nich cherubíny a palmy a rozvité kvety, a obložil zlatem taženým to, což bylo vyrezáno.
\par 36 Potom vystavel sín prostrední ze trí radu kamení tesaného, a jednoho radu dríví cedrového hoblovaného.
\par 37 Léta ctvrtého založen byl dum Hospodinuv, mesíce Ziv,
\par 38 A léta jedenáctého, mesíce Bul, (jenž jest mesíc osmý), dokonán jest dum se všemi prípravami svými, a se vším tím, což k nemu prináleželo; kterýž stavel sedm let.

\chapter{7}

\par 1 Potom stavel Šalomoun dum svuj za trinácte let, až jej všelijak dostavel.
\par 2 Vystavel též dum z lesu Libánského, sto loket zdélí a padesáte loket zšírí a tridceti loket zvýší na ctyrech radích sloupu cedrových, a trámové cedroví byli na tech sloupích.
\par 3 A prikryt byl cedrovím na hore na tech trámích, kteríž byli na ctyridcíti peti sloupích, po patnácti v každém radu.
\par 4 Oken také byly tri rady, okno proti oknu trmi rady.
\par 5 A všecky dvére i vereje ctverhrané byly, i okna, a zporádaná byla okna proti oknum trmi rady.
\par 6 Udelal i sínci na sloupích. Padesáti loket byla dlouhost její a tridcíti loket širokost její, a byla ta sínce napred, i sloupové i trámové její pred tím domem.
\par 7 Opet vzdelal jinou sínci, v níž byl trun, kdež soudil, totiž sínci soudnou, kteráž prikryta byla cedrovím od jedné podlahy až do druhé.
\par 8 Potom pri domu svém, v kterémž bydlil, udelal sín druhou za tou síní; dílo podobné onomu bylo. Udelal také dum dceri Faraonove, kterouž byl pojal Šalomoun, podobný té sínci.
\par 9 Všecko to bylo z kamení nákladného, vedlé míry tesaného a pilou rezaného, vnitr i zevnitr, a to od gruntu až k stropu, i zevnitr až k síni veliké.
\par 10 Základ také byl z kamení nákladného, z kamení velikého, z kamení na deset loket, a z kamení na osm loket.
\par 11 Tak i výš bylo kamení nákladné, príslušne vedlé míry tesané, a cedrové dsky.
\par 12 Sín také veliká mela všudy vukol trmi rady kamení tesané, a jedním radem dríví cedrové, podobne jako sín vnitrní domu Hospodinova, i sínce téhož domu.
\par 13 Poslav pak král Šalomoun, povolal Chírama z Týru,
\par 14 (Ten byl syn ženy vdovy z pokolení Neftalímova, jehož otec byl obyvatel Tyrský, remeslník díla z medi), proto že byl plný moudrosti a rozumnosti i umení k delání všelikého díla z medi. Kterýžto když prišel k králi Šalomounovi, delal všeliké dílo jeho.
\par 15 Nejprv sformoval dva sloupy medené. Osmnácti loket byla výsost sloupu jednoho, kterýž okolek mel dvanácti loktu. Tak i sloup druhý.
\par 16 Potom udelal dve makovice, kteréž by vstaveny byly na vrch tech sloupu, slité z medi. Peti loktu byla zvýší makovice jedna, a peti loktu zvýší makovice druhá.
\par 17 Mrežování také dílem pleteným, a šnury jako retízky udelal k tem makovicím, kteréž byly na vrchu sloupu, sedm na makovici jednu a sedm na makovici druhou.
\par 18 A udelav sloupy, pridal k mrežování jednomu dva rady jablek zrnatých vukol, aby makovice, kteréž byly na vrchu, prikrývaly. A tak ucinil makovici druhé.
\par 19 Na tech pak makovicích na vrchu sloupu, kteríž v sínci státi meli, bylo dílo kvetem liliovým na ctyri lokty.
\par 20 I mely makovice ty na dvou tech sloupích, jakož svrchu tak i proti vydutí pod zamrežováním jablka zrnatá, kterýchž bylo dve ste, dvema rady vukol, na jedné i na druhé makovici.
\par 21 A tak postavil ty sloupy v sínci chrámové. A postaviv sloup na pravé strane, nazval jméno jeho Jachin; a když postavil sloup na levé strane, nazval jméno jeho Boaz.
\par 22 A na vrchu sloupu tech bylo dílo liliové. I dokonáno jest dílo sloupu.
\par 23 Udelal také more slité, desíti loket od jednoho kraje k druhému, okrouhlé vukol, a pet loket byla vysokost jeho, a okolek jeho tridcíti loket vukol.
\par 24 Pod jehožto krajem pukly tykvím polním podobné všudy vukol, po desíti do lokte, obklicovaly more vukol; dva rady tykví litých bylo s ním slito.
\par 25 A stálo na dvanácti volích. Tri obráceni byli na pulnoci, a tri patrili k západu, tri zase postaveni byli ku poledni, a tri obráceni byli k východu, a more svrchu na nich stálo, ale všech jich zadkové byli pod morem.
\par 26 A bylo ztlouští na dlan. Kraj jeho byl, jakýž bývá u koflíku aneb kvetu liliového; dva tisíce tun v se bralo.
\par 27 Udelal také deset podstavku medených. Ctyr loket zdélí byl podstavek jeden každý, a ctyr loket zšírí, a trí loket zvýší.
\par 28 Bylo pak takové dílo každého podstavku; lištování bylo vukol na nich, kteréžto lištování bylo po krajích jeho.
\par 29 A na tom lištování, kteréž bylo po krajích, lvové, volové a cherubínové byli, a nad temi kraji byl sloupec svrchu, a pod lvy a voly obruba dílem taženým.
\par 30 Bylo také po ctyrech kolách medených pod každým podstavkem, a dsky medené, a na ctyrech úhlech svých meli raménka;tolikéž i pod každou medenicí byla ta raménka slitá pri každé obrube.
\par 31 Zrídlo pak jeho mezi obrubou po vrchu bylo zhloubí lokte, a však okrouhlé, remeslne jako i podstavek udelané, zšírí pul druhého lokte. Na kterémžto zrídle byly rezby, ale lištování jejich bylo cverhrané, neokrouhlé.
\par 32 A tak po ctyrech kolách bylo pod tím lištováním, a osy kol vycházely z podstavku; vysokost kola každého byla pul druhého lokte.
\par 33 Dílo tech kol bylo jako dílo kol u vozu; osy jejich i písty jejich, loukoti jejich i špice jejich, všecko bylo slité.
\par 34 Tolikéž ctyri raménka na ctyrech úhlech podstavku každého, z nehož ta raménka vynikala.
\par 35 Na vrchu pak podstavku byl sloupek na pul lokte zvýší, okrouhlý vukol, a na vrchu sloupku toho byli krajové vypuštení, a lištování na nich.
\par 36 I zdelal rezby na dskách po krajích jejich, a po lištování jejich, cherubíny, lvy a palmoví, jedno podlé druhého po obrube všudy vukol.
\par 37 Na ten zpusob udelal deset podstavku jednostejným slitím, na jednostejnou míru, a jednostejná rezba na všech byla.
\par 38 Udelal také deset umyvadel medených. Ctyridceti mer bralo v se jedno umyvadlo; ctyr loket bylo to každé umyvadlo, jedno umyvadlo na jednom podstavku, a tak na desíti podstavcích.
\par 39 I postavil podstavku pet po pravé strane domu, a pet po levé strane domu, more pak postavil na pravé strane domu, k východu na poledne.
\par 40 Nadelav tedy Chíram umyvadel, a lopat a kotlíku, dokonal všecko dílo, kteréž delal králi Šalomounovi k domu Hospodinovu:
\par 41 Dva sloupy a makovice okrouhlé, kteréž byly na vrchu dvou sloupu, a mrežování dvoje, aby prikrývalo ty dve makovice okrouhlé, kteréž byly na vrchu sloupu;
\par 42 A jablek zrnatých ctyri sta na dvojím mrežování; dva rady jablek zrnatých byly na mrežování jednom, aby prikrývaly dve makovice okrouhlé, kteréž byly na vrchu sloupu;
\par 43 Tolikéž podstavku deset, a umyvadel deset na podstavcích;
\par 44 A more jedno a volu dvanácte pod morem;
\par 45 Též hrnce a lopaty, a kotlíky. A všecko nádobí, kteréhož nadelal Chíram králi Šalomounovi do domu Hospodinova, bylo z medi pulerované.
\par 46 Ty veci na rovinách Jordánských slíval král v jilovaté zemi, mezi Sochot a Sartan.
\par 47 Zanechal pak Šalomoun vážení všeho nádobí, pro množství veliké není vyhledáváno váhy medi.
\par 48 Nadelal také Šalomoun všeho jiného nádobí do domu Hospodinova: Oltár zlatý a stul zlatý, na nemž byli chlebové predložení,
\par 49 A svícnu pet na pravé a pet na levé strane pred svatyní svatých z zlata nejcistšího, též kvety a lampy i uteradla z zlata,
\par 50 I báne a žaltáre, kotlíky, kadidlnice a nádoby k oharkum z zlata cistého, i panty zlaté ke dverím domu vnitrního, totiž svatyne svatých, a ke dverím chrámovým.
\par 51 A tak dokonáno jest všecko dílo, kteréž delal král Šalomoun k domu Hospodinovu, a vnesl tam Šalomoun veci posvecené od Davida otce svého, stríbro a zlato i nádobí, složiv je mezi poklady domu Hospodinova.

\chapter{8}

\par 1 Tedy shromáždil Šalomoun k sobe do Jeruzaléma starší lidu Izraelského, a všecky prední z pokolení, totiž knížata celedí otcovských s syny Izraelskými, aby prenesli truhlu smlouvy Hospodinovy z mesta Davidova, jenž jest Sion.
\par 2 I sešli se k králi Šalomounovi všickni muži Izraelští mesíce Etanim, v cas slavnosti; a ten mesíc jest sedmý.
\par 3 A když prišli všickni starší Izraelští, vzali kneží truhlu.
\par 4 I prenesli truhlu Hospodinovu a stánek úmluvy, i všecka nádobí posvátná, kteráž byla v stánku; a tak prenesli to kneží a Levítové.
\par 5 Král pak Šalomoun i všecko shromáždení Izraelské, kteréž se k nemu sešlo, obetovali s ním pred truhlou ovce i voly, kteríž ani popisováni ani pocítáni nebyli pro množství.
\par 6 I vnesli kneží truhlu smlouvy Hospodinovy na místo její, do vnitrního domu, totiž do svatyne svatých pod krídla cherubínu.
\par 7 Nebo cherubínové meli roztažená krídla nad místem truhly, a prikrývali cherubínové truhlu i sochory její svrchu.
\par 8 A povytáhli sochoru, tak že vidíni byli koncové jejich v svatyni k predku svatyne svatých, avšak vne nebylo jich videti; a byli tam až do tohoto dne.
\par 9 Nic nebylo v truhle krome dvou tabulí kamenných, kteréž tam složil Mojžíš na Orébe, když všel Hospodin v smlouvu s syny Izraelskými, a oni šli z zeme Egyptské.
\par 10 I stalo se, když vycházeli kneží z svatyne, že oblak naplnil dum Hospodinuv,
\par 11 Tak že nemohli kneží ostáti a sloužiti pro ten oblak; nebo sláva Hospodinova naplnila dum Hospodinuv.
\par 12 Tedy rekl Šalomoun: Hospodin rekl, že bude prebývati v mrákote.
\par 13 Jižt jsem vystavel dum k prebývání tvému, místo, v nemž bys prebýval na veky.
\par 14 A obrátiv král tvár svou, požehnání dával všemu shromáždení Izraelskému, všecko pak shromáždení Izraelské stálo.
\par 15 A rekl: Požehnaný Hospodin Buh Izraelský, kterýž mluvil ústy svými Davidovi otci mému, a skutecne naplnil to, rka:
\par 16 Od toho dne, kteréhož jsem vyvedl lid svuj Izraelský z Egypta, nevyvolil jsem mesta z žádného pokolení Izraelského k vystavení v nem domu, kdež by prebývalo jméno mé, ale Davida jsem vyvolil, aby spravoval lid muj Izraelský.
\par 17 Uložil te byl zajisté v srdci svém David otec muj staveti dum jménu Hospodina Boha Izraelského,
\par 18 Ale Hospodin rekl Davidovi otci mému: Ackoli jsi uložil v srdci svém staveti dum jménu mému, a dobre jsi ucinil, žes to myslil v srdci svém:
\par 19 A však ty nebudeš staveti toho domu, ale syn tvuj, kterýž vyjde z bedr tvých, on vystaví dum ten jménu mému.
\par 20 A tak splnil Hospodin slovo své, kteréž byl mluvil. Nebo jsem povstal na místo Davida otce svého, a dosedl jsem na stolici Izraelskou, jakož byl mluvil Hospodin, a ustavel jsem dum tento jménu Hospodina Boha Izraelského,
\par 21 A pripravil jsem tu místo truhle, v níž jest smlouva Hospodinova, kterouž ucinil s otci našimi, když je vyvedl z zeme Egyptské.
\par 22 Postavil se pak Šalomoun pred oltárem Hospodinovým, prede vším shromáždením Izraelským, a pozdvihl rukou svých k nebi,
\par 23 A rekl: Hospodine, Bože Izraelský, nenít podobného tobe Boha na nebi svrchu, ani na zemi dole, kterýž ostríháš smlouvy a milosrdenství služebníkum svým, chodícím pred tebou v celém srdci svém,
\par 24 Kterýž jsi splnil služebníku svému, Davidovi otci mému, to, co jsi mluvil jemu; nebo jsi mluvil ústy svými, a skutecne jsi to naplnil, jakož se dnes vidí.
\par 25 Nyní tedy, ó Hospodine, Bože Izraelský, naplniž služebníku svému, Davidovi otci mému, což jsi mluvil jemu, rka: Nebudet vyhlazen muž z rodu tvého od tvári mé, kterýž by sedel na stolici Izraelské, jestliže toliko ostríhati budou synové tvoji cesty své, aby chodili prede mnou, jakž jsi ty chodil prede mnou.
\par 26 Protož nyní, ó Bože Izraelský, prosím, necht jest upevneno slovo tvé, kteréž jsi mluvil služebníku svému, Davidovi otci mému.
\par 27 (Ac zdali v pravde bydliti bude Buh na zemi? Aj, nebesa, nýbrž nebesa nebes te neobsahují, mnohem méne dum tento, kterýž jsem vystavel.)
\par 28 A popatr k modlitbe služebníka svého a k úpení jeho, Hospodine, Bože muj, slyše volání a modlitbu, kterouž služebník tvuj modlí se dnes pred tebou,
\par 29 Aby oci tvé byly otevrené na dum tento v noci i ve dne, na místo toto, o nemž jsi rekl: Tut bude jméno mé, abys vyslýchal modlitbu, kterouž se modlívati bude služebník tvuj na míste tomto.
\par 30 Vyslýchejž tedy modlitbu služebníka svého i lidu svého Izraelského, kterouž se modlívati budou na míste tomto; ty vždy vyslýchej v míste prebývání svého na nebesích, a vyslýchaje, bud milostiv.
\par 31 Když by zhrešil clovek proti bližnímu svému, a nutil by ho k prísaze, tak že by prisahati musil, a prišla by ta prísaha pred oltár tvuj do domu tohoto:
\par 32 Ty vyslýchej na nebi a rozeznej, i rozsud služebníky své, mste nad bezbožným a obraceje skutky jeho na hlavu jeho, a ospravedlnuje spravedlivého, odplacuje mu vedlé spravedlnosti jeho.
\par 33 Když by poražen byl lid tvuj Izraelský od neprátel, proto že zhrešili proti tobe, jestliže by obrátíce se k tobe, vyznávali jméno tvé, a modléce se, ponížene prosili by tebe v dome tomto:
\par 34 Ty vyslýchej na nebi a odpust hrích lidu svému Izraelskému, a uved je zase do zeme, kterouž jsi dal otcum jejich.
\par 35 Když by zavríno bylo nebe, a nepršel déšt, proto že zhrešili proti tobe, a modléce se na míste tomto, vyznávali by se jménu tvému, a od hríchu svých odvrátili by se po tvém trestání:
\par 36 Ty vyslýchej na nebi, a odpust hrích služebníku svých a lidu svého Izraelského, vyuce je ceste výborné, po níž by chodili, a dej déšt na zemi svou, kterouž jsi dal lidu svému za dedictví.
\par 37 Byl-li by hlad na zemi, byl-li by mor, sucho, rez, kobylky, a brouci byli-li by; ssoužil-li by jej neprítel jeho v zemi obývání jeho, aneb jakákoli rána, jakákoli nemoc:
\par 38 Všelikou modlitbu a každé úpení, kteréž by cineno bylo od kteréhokoli cloveka, aneb ode všeho lidu tvého Izraelského, kdož by jen poznal ránu srdce svého, a pozdvihl by rukou svých v dome tomto,
\par 39 Ty vyslýchej na nebesích v míste prebývání svého, a slituj se i ucin a odplat jednomu každému vedlé všech cest jeho, kteréž ty znáš v srdci jeho, (nebo ty sám znáš srdce všech lidí),
\par 40 Aby se báli tebe po všecky dny, v nichž by živi byli na zemi, kterouž jsi dal otcum našim.
\par 41 Nýbrž také cizozemec, kterýž není z lidu tvého Izraelského, prišel-li by z zeme daleké pro jméno tvé,
\par 42 (Nebot uslyší o jménu tvém velikém a ruce tvé presilné, a rameni tvém vztaženém), prišel-li by tedy, a modlil by se v dome tomto:
\par 43 Ty vyslýchej na nebi v míste prebývání svého, a ucin všecko to, o cež volati bude k tobe cizozemec ten, atby poznali všickni národové zeme jméno tvé, a báli se tebe jako lid tvuj Izraelský, a aby vedeli, že jméno tvé vzýváno jest nad domem tímto, kterýž jsem vystavel.
\par 44 Když by vytáhl lid tvuj k boji proti nepríteli svému, cestou, kterouž bys poslal je, jestliže by se modlili Hospodinu naproti mestu tomuto, kteréž jsi vyvolil, a naproti domu, kterýž jsem vystavel jménu tvému:
\par 45 Tolikéž vyslýchej na nebi modlitbu jejich a úpení jejich, a vyved pri jejich.
\par 46 Když by zhrešili proti tobe, (jakož není cloveka, kterýž by nehrešil), a rozhnevaje se na ne, vydal bys je v moc nepríteli, tak že by je jaté vedli ti, kteríž by je zjímali, do zeme neprátelské, daleké neb blízké,
\par 47 A usmyslili by sobe v zemi té, do kteréž by zajati byli, a obrátíce se, modlili by se tobe v zemi tech, kteríž je zzajímali, rkouce: Zhrešilit jsme a nepráve jsme cinili, bezbožne jsme cinili,
\par 48 A tak navrátili by se k tobe celým srdcem svým a celou duší svou v zemi neprátel svých, kteríž je zzajímali, a modlili by se tobe naproti zemi své, kterouž jsi dal otcum jejich, naproti mestu tomu, kteréž jsi vyvolil, a domu, kterýž jsem vzdelal jménu tvému:
\par 49 Vyslýchejž tedy na nebesích, v míste prebývání svého, modlitbu jejich a úpení jejich, a vyved pri jejich,
\par 50 A odpust lidu svému, cožkoli zhrešil proti tobe, i všecka prestoupení jejich, jimiž prestoupili proti tobe, a naklon k nim lítostí ty, kteríž je zjímali, aby se smilovali nad nimi.
\par 51 Nebot jsou lid tvuj a dedictví tvé, kteréž jsi vyvedl z Egypta, z prostred peci železné.
\par 52 Nechat jsou tedy oci tvé otevrené k modlitbe služebníka tvého a k modlitbe lidu tvého Izraelského, abys je vyslýchal vždycky, když by te koli vzývali.
\par 53 Nebo jsi ty oddelil je sobe za dedictví ode všech národu zeme, jakož jsi mluvil skrze Mojžíše služebníka svého, když jsi vyvedl otce naše z Egypta, Panovníce Hospodine.
\par 54 Stalo se pak, když se prestal Šalomoun modliti Hospodinu vší modlitbou a prosbou touto, že vstal od oltáre Hospodinova, a prestal kleceti a pozdvihovati rukou svých k nebi.
\par 55 A stoje, dobrorecil všemu shromáždení Izraelskému hlasem velikým, rka:
\par 56 Požehnaný Hospodin, kterýž dal odpocinutí lidu svému Izraelskému vedlé všeho, což mluvil. Nepochybilo ani jedno slovo ze všelikého slova jeho dobrého, kteréž mluvil skrze služebníka svého Mojžíše.
\par 57 Budiž Hospodin Buh náš s námi, jako byl s otci našimi; nezamítejž nás, ani opouštej.
\par 58 Ale nakloniž srdce naše k sobe, abychom chodili po všech cestách jeho, a ostríhali prikázaní jeho, i ustanovení jeho, a soudu jeho, kteréž prikázal otcum našim.
\par 59 A at jsou má tato slova, kterýmiž jsem se modlil pred Hospodinem, blízko Hospodina Boha našeho dnem i nocí, tak aby vyvodil pri služebníka svého, a pri lidu svého Izraelského každého casu a dne,
\par 60 Atby poznali všickni národové zeme, že Hospodin sám jest Buh, a že není krome neho žádný.
\par 61 Budiž tedy srdce vaše celé k Hospodinu Bohu našemu, tak abyste chodili v ustanoveních jeho, ostríhajíce prikázaní jeho, tak jako dnešního dne.
\par 62 Král pak a s ním všecken Izrael obetovali obeti pred Hospodinem.
\par 63 A obetoval Šalomoun v obet pokojnou, kterouž obetoval Hospodinu, volu dvamecítma tisícu, a ovec sto dvadceti tisícu, a posvecovali domu Hospodinova král i všickni synové Izraelští.
\par 64 Téhož dne posvetil král prostredku síne, kteráž byla pred domem Hospodinovým; nebo obetoval tu obeti zápalné a obeti suché, a tuky pokojných obetí, proto že oltár medený, kterýž byl pred Hospodinem, byl malý, aniž mohly se na nem smestknati obeti zápalné a obeti suché, a tukové pokojných obetí.
\par 65 A tak držel Šalomoun toho casu slavnost, a všecken Izrael s ním, shromáždení veliké odtud, kudyž se chodí do Emat, až ku potoku Egyptskému, pred Hospodinem Bohem naším, za sedm dní a opet za sedm dní, to jest za ctrnácte dní.
\par 66 Dne pak osmého propustil lid. Kteríž požehnavše krále, odešli do obydlí svých, radujíce se a veselíce se v srdci ze všech dobrých vecí, kteréž ucinil Hospodin Davidovi služebníku svému a Izraelovi lidu svému.

\chapter{9}

\par 1 Stalo se pak, když dokonal Šalomoun stavení domu Hospodinova a domu královského, podlé vší líbosti své, jakž vykonati umínil,
\par 2 Že se ukázal Hospodin Šalomounovi podruhé, jakož se mu byl ukázal v Gabaon.
\par 3 I rekl jemu Hospodin: Vyslyšelt jsem modlitbu tvou a prosbu tvou, kteroužs se modlil prede mnou; posvetil jsem domu toho, kterýž jsi vystavel, aby prebývalo tam jméno mé až na veky, a byly tu oci mé i srdce mé po všecky dny.
\par 4 A ty jestliže choditi budeš prede mnou, jako chodil David otec tvuj, v dokonalosti srdce a v uprímnosti, cine všecko to, což jsem prikázal tobe, ustanovení mých i soudu mých ostríhaje:
\par 5 Utvrdím zajisté stolici království tvého nad Izraelem na veky, jakož jsem mluvil Davidovi otci tvému, rka: Nebudet odjat muž z rodu tvého od trunu Izraelského.
\par 6 Pakli se nazpet odvrátíte vy i synové vaši ode mne, a nebudete ostríhati prikázaní mých a ustanovení mých, kteráž jsem vydal vám, ale odejdouce, sloužiti budete bohum cizím a klaneti se jim:
\par 7 Vypléním docela Izraele se svrchku zeme, kterouž jsem jim dal, a dum tento, kteréhož jsem posvetil jménu svému, zavrhu od tvári své, i budet Izrael za prísloví a za rozprávku mezi všemi národy.
\par 8 Ano i dum tento, jakkoli bude slavný, kdokoli mimo nej pujde, užasne se a ckáti bude, i rekne: Proc tak ucinil Hospodin zemi této a domu tomuto?
\par 9 Tedy odpovedí: Proto že opustili Hospodina Boha svého, kterýž vyvedl otce jejich z zeme Egyptské, a chytili se bohu cizích, a klaneli se jim i sloužili jim, protož uvedl na ne Hospodin všecky tyto zlé veci.
\par 10 Stalo se také po prebehnutí dvadcíti let, v nichž vzdelal Šalomoun oba dva ty domy, dum Hospodinuv a dum královský,
\par 11 K cemuž byl Chíram král Tyrský daroval Šalomounovi hojne dríví cedrového a jedlového, i zlata vedlé vší vule jeho, že dal také král Šalomoun Chíramovi dvadceti mest v zemi Galilejské.
\par 12 I vyjel Chíram z Týru, aby videl ta mesta, kteráž mu daroval Šalomoun, ale nelíbila se jemu.
\par 13 Protož rekl: Jakáž jsou ta mesta, kteráž mi dáváš, bratre muj? I nazval je zemí Kabul až do tohoto dne.
\par 14 Nebo byl poslal Chíram králi sto a dvadceti centnéru zlata.
\par 15 Prícina pak platu, kterýž byl uložil král Šalomoun, byla, aby stavel dum Hospodinuv a dum svuj, a Mello i zdi Jeruzalémské, též Azor a Mageddo i Gázer.
\par 16 (Farao zajisté král Egyptský vytáh, dobyl Gázeru a vypálil je, Kananejské pak, kteríž byli v tom meste, pobil a dal mesto za veno dceri své, manželce Šalomounove.)
\par 17 A tak vystavel zase Šalomoun Gázer a Betoron dolní,
\par 18 Též Baalat a Tadmor na poušti v též zemi,
\par 19 I všecka mesta, v nichž Šalomoun mel své sklady, i mesta vozu, i mesta jezdcu, vše vedlé žádosti své, cožkoli chtel staveti v Jeruzaléme, a na Libánu i po vší zemi panování svého.
\par 20 Všecken také lid, kterýž byl pozustal z Amorejských, Hetejských, Ferezejských, Hevejských a Jebuzejských, kteríž nebyli z synu Izraelských,
\par 21 Totiž syny jejich, kteríž byli pozustali po nich v zemi, jichž nemohli synové Izraelští vyhladiti, uvedl Šalomoun pod plat a v službu až do tohoto dne.
\par 22 Z synu pak Izraelských žádného nepodrobil v službu Šalomoun, ale byli muži válecní, a služebníci jeho, a knížata jeho, vudcové jeho a úredníci nad vozy a jezdci jeho.
\par 23 Bylo tech predních vládaru, kteríž byli nad dílem Šalomounovým, pet set a padesát. Ti spravovali lidi, kteríž delali.
\par 24 Dcera pak Faraonova prestehovala se z mesta Davidova do domu svého, kterýž jí byl vystavel Šalomoun. Tehdáž také vystavel Mello.
\par 25 I obetoval Šalomoun každého roku trikrát obeti zápalné a pokojné na oltári, kterýž byl vzdelal Hospodinu, ale kadíval na tom, kterýž byl pred Hospodinem, když dokonal dum.
\par 26 Nadelal také král Šalomoun lodí velikých v Aziongaber, kteréž jest podlé Elat, na brehu more Rudého v zemi Idumejské.
\par 27 A poslal Chíram na tech lodech služebníky své, plavce umelé na mori, s služebníky Šalomounovými.
\par 28 Kteríž preplavivše se do Ofir, nabrali tam zlata ctyri sta a dvadceti centnéru, a privezli králi Šalomounovi.

\chapter{10}

\par 1 Uslyševši pak královna z Sáby povest o Šalomounovi a jménu Hospodinovu, prijela, aby zkusila jeho v pohádkách.
\par 2 A prijevši do Jeruzaléma s poctem velmi velikým, s velbloudy nesoucími vonné veci a zlata velmi mnoho i kamení drahého, prišla k Šalomounovi, a mluvila s ním o všecko, což mela v srdci svém.
\par 3 Jížto odpovedel Šalomoun na všecka slova její. Nebylo nic skrytého pred králem, nac by jí neodpovedel.
\par 4 Protož uzrevši královna z Sáby všecku moudrost Šalomounovu i dum, kterýž byl ustavel,
\par 5 Též pokrmy stolu jeho, i sedání a stávání služebníku prisluhujících jemu, i roucha jejich, šenkýre také jeho, i stupne, kterýmiž vstupoval k domu Hospodinovu, zdesila se náramne.
\par 6 A rekla králi: Pravát jest rec, kterouž jsem slyšela v zemi své, o vecech tvých a o moudrosti tvé,
\par 7 Ješto jsem nechtela veriti recem, až jsem prijela a uzrela ocima svýma. Ale aj, není mi praveno ani polovice; prevýšil jsi moudrostí a dobrotou povest tu, kterouž jsem slyšela.
\par 8 Blahoslavení muži tvoji, a blahoslavení služebníci tvoji, kteríž stojí pred tebou vždycky, a slyší moudrost tvou.
\par 9 Budiž Hospodin Buh tvuj požehnaný, kterýž te sobe oblíbil, aby te posadil na stolici Izraelské, proto že miluje Hospodin Izraele na veky, a ustanovil te králem, abys cinil soud a spravedlnost.
\par 10 I dala králi sto a dvadceti centnéru zlata, a vonných vecí velmi mnoho, i kamení drahého, aniž bylo kdy více privezeno takových vonných vecí tak mnoho, jako darovala královna z Sáby králi Šalomounovi.
\par 11 (Lodí také Chíramova, kteráž prinášela zlato z Ofir, privezla z Ofir dríví almugim velmi mnoho i kamení drahého.
\par 12 I nadelal král z toho dríví almugim zábradel k domu Hospodinovu, i k domu královu, též harf a louten zpevákum, aniž jest kdy privezeno takového dríví almugim, ani vidíno do dnešního dne.)
\par 13 Král také Šalomoun dal královne z Sáby vedlé vší vule její, cehož požádala, nad to, což jí Šalomoun daroval darem královským. Potom se navrátila do zeme své ona i služebníci její.
\par 14 Byla pak váha toho zlata, kteréž pricházelo Šalomounovi na každý rok, šest set šedesáte šest centnéru zlata,
\par 15 Krome toho, co pricházelo od kupcu a prodavacu vonných vecí, a ode všech králu Arabských i vývod zeme.
\par 16 A protož nadelal král Šalomoun dve ste štítu z zlata taženého, šest set zlatých dával na každý štít,
\par 17 A tri sta pavéz z taženého zlata, tri libry zlata dal na každou pavézu. I složil je král v dome lesu Libánského.
\par 18 Udelal také král stolici z kostí slonových velikou, a obložil ji zlatem ryzím.
\par 19 Šest stupnu bylo k té stolici, a vrch okrouhlý byl na stolici od zadní strany její, a zpolehadla rukám s obou stran té stolice, a dva lvové stáli u zpolehadel.
\par 20 A dvanácte lvu stálo tu na šesti stupních s obou stran. Nebylo nic takového ucineno v žádných královstvích.
\par 21 Nadto všecky nádoby krále Šalomouna, jichž ku pití užívali, byly zlaté, a všecky nádoby v dome lesu Libánského byly z zlata nejcistšího. Nic nebylo od stríbra, aniž ho sobe co vážili ve dnech Šalomounových.
\par 22 Nebo mel král lodí morské s lodími krále Chírama. Jednou ve trech letech vracely se ty lodí morské, prinášející zlato a stríbro, kosti slonové, a opice a pávy.
\par 23 I zveleben jest král Šalomoun nad všecky krále zemské v bohatství a v moudrosti.
\par 24 Procež všickni obyvatelé zeme žádostivi byli videti tvár Šalomounovu, aby slyšeli moudrost jeho, kterouž složil Buh v srdci jeho.
\par 25 A prinášeli jeden každý dary své, nádoby stríbrné a nádoby zlaté, roucha a zbroj, i vonné veci, kone a mezky, a to každého roku,
\par 26 Tak že nashromáždil Šalomoun vozu a jezdcu, a mel tisíc a ctyri sta vozu a dvanácte tisíc jezdcu, kteréž rozsadil v mestech vozu, a pri králi v Jeruzaléme.
\par 27 I složil král stríbra v Jeruzaléme jako kamení, a cedrového dríví jako planého fíkoví, kteréž roste v údolí u velikém množství.
\par 28 Privodili také Šalomounovi kone z Egypta a koupe rozlicné; nebo kupci královští brávali koupe rozlicné za slušnou mzdu,
\par 29 A vodívali sprež vozníku z Egypta za šest set lotu stríbra, kone pak jednoho za sto a padesáte, a tak všechnem králum Hetejským a králum Syrským oni dodávali.

\chapter{11}

\par 1 V tom král Šalomoun zamiloval ženy cizozemky mnohé, i dceru Faraonovu, i Moábské, Ammonitské, Idumejské, Sidonské a Hetejské,
\par 2 Z národu tech, kteréž zapovedel Hospodin synum Izraelským, rka: Nebudete se smešovati s nimi, aniž se oni budou smešovati s vámi, nebot by naklonili srdce vaše k bohum svým. K tem prilnul Šalomoun milostí,
\par 3 Tak že mel žen královen sedm set a ženin tri sta. I odvrátily ženy jeho srdce jeho.
\par 4 Stalo se tedy, že když se zstaral Šalomoun, ženy jeho naklonily srdce jeho k bohum cizím, tak že nebylo srdce jeho celé pri Hospodinu Bohu jeho, jako bylo srdce Davida otce jeho.
\par 5 Ale obrátil se Šalomoun k Astarot, bohyni Sidonské, a k Moloch, ohavnosti Ammonské.
\par 6 I cinil Šalomoun to, což se nelíbilo Hospodinu, a nenásledoval cele Hospodina, jako David otec jeho.
\par 7 Tedy vystavel Šalomoun výsost Chámosovi, ohavnosti Moábské, na hore, kteráž jest naproti Jeruzalému, a Molochovi, ohavnosti synu Ammon.
\par 8 A tak vzdelal všechnem ženám svým z cizího národu, kteréž kadily a obetovaly bohum svým.
\par 9 I rozhneval se Hospodin na Šalamouna, proto že se uchýlilo srdce jeho od Hospodina Boha Izraelského, kterýž se jemu byl ukázal po dvakrát,
\par 10 A zapovedel jemu tu vec, aby nechodil po bozích cizích. Ale neostríhal toho, což byl prikázal Hospodin.
\par 11 Protož rekl Hospodin Šalomounovi:Ponevadž se to nalezlo pri tobe, a neostríhal jsi smlouvy mé ani ustanovení mých, kterážt jsem prikázal, vez, že odtrhnu království toto od tebe, a dám je služebníku tvému.
\par 12 A však za dnu tvých neuciním toho pro Davida otce tvého, než z ruky syna tvého odtrhnu je.
\par 13 Všeho pak království neodtrhnu, ale pokolení jednoho zanechám synu tvému pro Davida služebníka svého a pro Jeruzalém, kterýž jsem vyvolil.
\par 14 A tak vzbudil Hospodin protivníka Šalomounovi, Adada Idumejského z semene královského, kterýž byl v zemi Idumejské.
\par 15 Nebo stalo se, když bojoval David proti Idumejským, a Joáb kníže vojska vytáhl, aby pochoval zmordované, a pobil všecky pohlaví mužského v zemi Idumejské,
\par 16 (Za šest zajisté mesícu byl tam Joáb se vším lidem Izraelským, dokudž nevyplénil všech pohlaví mužského v zemi Idumejské),
\par 17 Že tehdáž utekl Adad sám, a nekterí muži Idumejští z služebníku otce jeho s ním, aby šli do Egypta. Adad pak byl pachole neveliké.
\par 18 Kteríž jdouce z Madian, prišli do Fáran, a vzavše s sebou nekteré muže z Fáran, prišli do Egypta k Faraonovi, králi Egyptskému, kterýž dal jemu dum, i stravou opatril ho, dal jemu také i zemi.
\par 19 A tak nalezl Adad milost velikou pred Faraonem, tak že jemu dal za manželku sestru ženy své, sestru Tafnes královny.
\par 20 I porodila jemu sestra Tafnes syna Genubata, a odchovala ho Tafnes v domu Faraonovu. I byl Genubat v dome Faraonove mezi syny Faraonovými.
\par 21 Když pak uslyšel Adad v Egypte, že by usnul David s otci svými, a že umrel i Joáb kníže vojska, tedy rekl Adad Faraonovi: Propust mne, at jdu do zeme své.
\par 22 Jemuž rekl Farao: Cehožt se nedostává u mne, že chceš odjíti do zeme své? I rekl: Niceho, a však vždy mne propust.
\par 23 Vzbudil ješte Buh proti nemu protivníka, Rázona syna Eliadova, kterýž byl utekl od Hadarezera krále Soby, pána svého,
\par 24 A shromáždiv k sobe muže, byl knížetem roty, když je David hubil. Protož odšedše do Damašku, bydlili v nem a kralovali v Damašku.
\par 25 I byl protivníkem Izraelovým po všecky dny Šalomounovy; a to bylo nad to zlé, kteréž mu cinil Adad. I mel v ošklivosti Izraele, když kraloval v Syrii.
\par 26 Jeroboám také syn Nebatuv Efratejský z Sareda, (a jméno matky jeho ženy vdovy bylo Serua), služebník Šalomounuv, pozdvihl ruky proti králi.
\par 27 A tato byla prícina, pro kterouž pozdvihl ruky proti králi Šalomounovi: Že vystavev Šalomoun Mello, zavrel mezeru mesta Davidova otce svého.
\par 28 Jeroboám pak byl muž silný a udatný. Protož vida Šalomoun mládence, že by pracovitý byl, ustanovil ho nade všemi platy z domu Jozefa.
\par 29 I stalo se téhož casu, že když vyšel Jeroboám z Jeruzaléma, našel jej prorok Achiáš Silonský na ceste, jsa odín rouchem novým. A byli sami dva na poli.
\par 30 Tedy ujav Achiáš roucho nové, kteréž mel na sobe, roztrhal je na dvanácte kusu.
\par 31 A rekl Jeroboámovi: Vezmi sobe deset kusu; nebo takto praví Hospodin Buh Izraelský: Aj, já roztrhnu království z ruky Šalomounovy, a dám tobe desatero pokolení.
\par 32 Jedno toliko pokolení zustane jemu pro služebníka mého Davida, a pro mesto Jeruzalém, kteréž jsem vyvolil ze všech pokolení Izraelských,
\par 33 Proto že opustili mne, a klaneli se Astarot bohyni Sidonské, a Chámos bohu Moábskému, i Moloch bohu Ammonskému, a nechodili po cestách mých, aby cinili to, což mi se líbí, totiž ustanovení má a soudy mé, jako David otec jeho.
\par 34 A však neodejmu niceho z království z rukou jemu; nebo vudcím zanechám ho po všecky dny života jeho pro Davida služebníka svého, kteréhož jsem vyvolil, kterýž ostríhal prikázaní mých a ustanovení mých.
\par 35 Ale potom vezma království z ruky syna jeho, dám tobe z neho desatero pokolení,
\par 36 Synu pak jeho dám jedno pokolení, aby zustala svíce Davidovi služebníku mému po všecky dny prede mnou v meste Jeruzaléme, kteréž jsem sobe vyvolil, aby tam jméno mé prebývalo.
\par 37 A tak tebe vezmu, abys kraloval ve všech vecech, kterýchž by žádala duše tvá, a budeš králem nad Izraelem.
\par 38 Protož jestliže uposlechneš všeho toho, což prikáži tobe, a choditi budeš po cestách mých, a ciniti to, což mi se líbí, ostríhaje ustanovení mých a prikázaní mých, jako cinil David služebník muj: budu s tebou a vzdelám tobe dum stálý, jako jsem vzdelal Davidovi, a dám tobe lid Izraelský.
\par 39 Potrápímt zajisté semene Davidova pro tu vec, a však ne po všecky dny.
\par 40 Pro tu prícinu chtel Šalomoun zabiti Jeroboáma. Kterýž vstav, utekl do Egypta k Sesákovi králi Egyptskému, a byl v Egypte, dokudž neumrel Šalomoun.
\par 41 Jiné pak veci Šalomounovy, kteréž cinil, i moudrost jeho vypsány jsou v knize cinu Šalomounových.
\par 42 Dnu pak, v nichž kraloval Šalomoun v Jeruzaléme nade vším Izraelem, bylo ctyridceti let.
\par 43 I usnul Šalomoun s otci svými, a pochován jest v meste Davida otce svého. Kraloval pak Roboám syn jeho místo neho.

\chapter{12}

\par 1 Tedy pribral se Roboám do Sichem; nebo tam sešel se byl všecken Izrael, aby ho ustanovili za krále.
\par 2 Stalo se pak, když uslyšel Jeroboám syn Nebatuv, jsa ješte v Egypte, kamž byl utekl pred Šalomounem králem, (bydlel zajisté Jeroboám v Egypte),
\par 3 Že poslali a povolali ho; protož prišed Jeroboám i všecko shromáždení Izraelské, mluvili k Roboámovi, rkouce:
\par 4 Otec tvuj ztížil jho naše; protož nyní polehc služby otce svého tvrdé a bremena jeho težkého, kteréž vzložil na nás, a budeme tobe sloužiti.
\par 5 Kterýž rekl jim: Odejdete, a tretího dne navratte se ke mne. I odšel lid.
\par 6 Tedy radil se král Roboám s starci, kteríž stávali pred Šalomounem otcem jeho ješte za života jeho, rka: Kterak vy radíte, jakou odpoved mám dáti lidu tomuto?
\par 7 I odpovedeli jemu, rkouce: Jestliže dnes povolný budeš lidu tomuto a ochotne se jim ukážeš, a odpoved dávaje, mluviti budeš prívetive, budout služebníci tvoji po všecky dny.
\par 8 Ale on opustil radu starcu, kterouž dali jemu, a radil se s mládenci, kteríž s ním zrostli a stávali pred ním.
\par 9 A rekl k nim: Co vy radíte, jakou bychom dali odpoved lidu tomuto, kteríž mluvili ke mne, rkouce: Polehc bremene, kteréž vzložil otec tvuj na nás?
\par 10 I odpovedeli jemu mládenci, kteríž zrostli s ním, rkouce: Takto odpovíš lidu tomu, kteríž mluvili k tobe a rekli: Otec tvuj ztížil jho naše, ty pak polehc nám. Takto díš jim: Nejmenší prst muj tlustší jest nežli bedra otce mého.
\par 11 Nyní tedy otec muj težké bríme vložil na vás, já pak pridám bremene vašeho; otec muj trestal vás bicíky, ale já trestati vás budu bici uzlovatými.
\par 12 Prišel tedy Jeroboám i všecken lid k Roboámovi dne tretího, jakž byl rozkázal král, rka: Navratte se ke mne dne tretího.
\par 13 I odpovedel král lidu tvrde, opustiv radu starcu, kterouž dali jemu.
\par 14 A mluvil k nim vedlé rady mládencu, rka: Otec muj ztížil jho vaše, já pak pridám bremene vašeho; otec muj trestal vás bicíky, ale já trestati vás budu bici uzlovatými.
\par 15 I neuposlechl král lidu. Nebo byla prícina od Hospodina, aby se naplnila rec jeho, kterouž mluvil Hospodin skrze Achiáše Silonského k Jeroboámovi synu Nebatovu.
\par 16 Protož vida všecken Izrael, že by je král oslyšel, odpovedel lid králi v tato slova: Jakýž máme díl v Davidovi? Ani dedictví nemáme v synu Izai. K stanum svým, ó Izraeli! Nyní opatr dum svuj, Davide. Odšel tedy Izrael k stanum svým,
\par 17 Tak že nad syny Izraelskými toliko, kteríž bydlili v mestech Judských, kraloval Roboám.
\par 18 I poslal král Roboám Adurama, kterýž byl nad platy, a uházel ho všecken Izrael kamením až do smrti, címž král Roboám prinucen byl, aby vsedna na vuz, utekl do Jeruzaléma.
\par 19 A tak odstoupili synové Izraelští od domu Davidova až do dnešního dne.
\par 20 I stalo se, když uslyšeli všickni Izraelští, že by se navrátil Jeroboám, poslavše, povolali ho do shromáždení, a ustanovili ho králem nade vším Izraelem. Nezustávalo pri domu Davidovu než samo pokolení Judovo.
\par 21 Když pak prijel Roboám do Jeruzaléma, shromáždil všecken dum Judský a pokolení Beniamin, totiž sto a osmdesáte tisícu výborných bojovníku, aby bojovali proti domu Izraelskému, a aby zase privedeno bylo království k Roboámovi synu Šalomounovu.
\par 22 Tedy stala se rec Boží k Semaiášovi muži Božímu, rkoucí:
\par 23 Povez Roboámovi synu Šalomounovu, králi Judskému a všemu domu Judovu i Beniaminovu, a ostatku lidu temito slovy:
\par 24 Takto praví Hospodin: Netáhnete a nebojujte proti bratrím svým synum Izraelským. Navratte se jeden každý do domu svého, nebo ode mne stala se vec tato. I uposlechli rozkazu Hospodinova a navrátili se, aby odešli vedlé reci Hospodinovy.
\par 25 Potom vystavel Jeroboám Sichem na hore Efraim, a bydlil v nem, a vyšed odtud, vystavel Fanuel.
\par 26 Rekl pak Jeroboám v srdci svém: Tudíž by se navrátilo království toto k domu Davidovu,
\par 27 Když by chodíval lid tento k obetování obetí v dome Hospodinove do Jeruzaléma; i obrátilo by se srdce lidu tohoto ku pánu jeho Roboámovi králi Judskému, a tak zabijíce mne, navrátili by se k Roboámovi králi Judskému.
\par 28 Protož poradiv se král, udelal dvé telat zlatých a rekl lidu: Dosti jste již chodili do Jeruzaléma. Aj, ted bohové tvoji, ó Izraeli, kteríž te vyvedli z zeme Egyptské.
\par 29 I postavil jedno v Bethel, a druhé postavil v Dan.
\par 30 Kterážto vec byla prícinou k hrešení, nebo chodíval lid k jednomu z tech až do Dan.
\par 31 Vzdelal zajisté dum výsostí, a ustanovil kneží z lidu obecného, kteríž nebyli z synu Léví.
\par 32 Ustanovil také Jeroboám svátek mesíce osmého, v patnáctý den téhož mesíce, ku podobenství svátku, kterýž byl v Judstvu, a obetoval na oltári. Takž ucinil i v Bethel, obetuje telatum, kteréž byl udelal; také i v Bethel ustanovil kneží výsostí, kteréž byl zdelal.
\par 33 A obetoval na oltári, kterýž byl udelal v Bethel, v patnáctý den mesíce osmého, toho mesíce, kterýž byl sobe smyslil v srdci svém, a slavil svátek s syny Izraelskými, a pristoupil k oltári, aby kadil.

\chapter{13}

\par 1 A aj, muž Boží prišel z Judstva s slovem Hospodinovým do Bethel, tehdáž když Jeroboám, stoje u oltáre, kadil.
\par 2 I zvolal proti oltári slovem Hospodinovým, rka: Oltári, oltári, toto praví Hospodin: Hle, syn narodí se domu Davidovu, jménem Joziáš, kterýž obetovati bude na tobe kneží výsostí, jenž zapalují kadidlo na tobe; i kosti lidské páliti budou na tobe.
\par 3 I dal téhož dne znamení, rka: Totot jest znamení, že mluvil Hospodin: Aj, oltár roztrhne se a vysype se popel, kterýž jest na nem.
\par 4 V tom uslyšev král Jeroboám slovo muže Božího, že zvolal proti oltári v Bethel, vztáhl ruku svou od oltáre a rekl: Jmete ho. I uschla ruka jeho, kterouž vztáhl proti nemu, a nemohl ji zase pritáhnouti k sobe.
\par 5 Oltár také se roztrhl, a vysypal se popel z oltáre podlé znamení, kteréž byl predpovedel muž Boží slovem Hospodinovým.
\par 6 Protož mluvil král a rekl tomu muži Božímu: Pomodl se medle Hospodinu Bohu svému a pros za mne, abych mohl pritáhnouti ruku svou k sobe. I modlil se muž Boží Hospodinu, a pritáhl král ruku zase k sobe, a byla jako prvé.
\par 7 Tedy rekl král muži Božímu: Pod se mnou domu a posiln se, a dámt dar.
\par 8 I rekl muž Boží králi: Bys mi dal polovici domu svého, nešel bych s tebou, aniž bych jedl chleba, aniž bych pil vody na míste tomto.
\par 9 Nebo tak mi prikázal slovem svým Hospodin, rka: Nebudeš jísti chleba, ani píti vody, aniž se navrátíš tou cestou, kterouž jsi prišel.
\par 10 A tak odšel jinou cestou, a nenavrátil se tou cestou, kterouž byl prišel do Bethel.
\par 11 Prorok pak jeden starý bydlil v Bethel, jehož syn prišed, vypravoval mu všecky veci, kteréž ucinil muž Boží toho dne v Bethel; i slova, kteráž mluvil králi, vypravovali otci svému.
\par 12 Tedy rekl jim otec jejich: Kterou cestou šel? I ukázali mu synové jeho cestu, kterouž šel ten muž Boží, kterýž byl z Judstva prišel.
\par 13 Zatím rekl synum svým: Osedlejte mi osla. Kteríž osedlali mu osla, a on vsedl na nej.
\par 14 A jel za mužem Božím a nalezl jej, an sedí pod dubem. I rekl jemu: Ty-li jsi ten muž Boží, kterýž jsi prišel z Judstva? A on odpovedel: Jsem.
\par 15 Kterýž rekl jemu: Pod se mnou domu, a pojíš chleba.
\par 16 Ale on odpovedel: Nemohut se navrátiti, ani jíti s tebou, aniž budu jísti chleba, ani píti vody s tebou na míste tomto.
\par 17 Nebo stala se ke mne rec slovem Hospodinovým: Nebudeš tam jísti chleba, ani píti vody; nenavrátíš se, jda touž cestou, kterouž jsi šel.
\par 18 Jemuž odpovedel: Takét já prorok jsem jako i ty, a mluvil ke mne andel slovem Hospodinovým, rka: Navrat ho s sebou do domu svého, at pojí chleba a napije se vody. To pak sklamal jemu.
\par 19 Takž se navrátil s ním, a jedl chléb v dome jeho a pil vodu.
\par 20 Když pak oni sedeli za stolem, stalo se slovo Hospodinovo k proroku tomu, kterýž onoho byl zase navrátil.
\par 21 A zvolal na muže Božího, kterýž byl prišel z Judstva, rka: Toto praví Hospodin: Proto že jsi na odpor ucinil reci Hospodinove, a neostríhal jsi prikázaní, kteréžt vydal Hospodin Buh tvuj,
\par 22 Ale navrátils se a jedls chléb, a pils vodu na míste, o kterémžt rekl: Nebudeš jísti chleba, ani píti vody: nebudet pochováno telo tvé v hrobe otcu tvých.
\par 23 Tedy když pojedl chleba a napil se, osedlal mu osla, totiž tomu proroku, kteréhož byl navrátil.
\par 24 A když odšel, trefil na nej lev na ceste a udávil jej. I leželo telo jeho na ceste, a osel stál podlé neho; lev také stál vedlé tela mrtvého.
\par 25 A aj, muži nekterí jdouce tudy, uzreli telo mrtvé ležící na ceste, a lva, an stojí vedlé neho. Kteríž prišedše, povedeli v meste, v kterémž ten starý prorok bydlil.
\par 26 Což když uslyšel ten prorok, kterýž ho byl navrátil s cesty, rekl: Muž Boží jest, kterýž na odpor ucinil reci Hospodinove; protož vydal jej Hospodin lvu, kterýž potrev ho, udávil jej vedlé reci Hospodinovy, kterouž mluvil jemu.
\par 27 Zatím mluve k synum svým, rekl: Osedlejte mi osla. I osedlali.
\par 28 A odjev, nalezl mrtvé telo jeho ležící na ceste, a osla i lva, an stojí u toho tela; a nejedl lev tela toho, aniž co uškodil oslu.
\par 29 Protož vzav prorok telo muže Božího, vložil je na osla, a prinesl je. I prišel do mesta svého, aby ho plakal a pochoval.
\par 30 Položil pak telo jeho v hrobe svém, a plakali ho: Ach, bratre muj.
\par 31 A pochovav ho, mluvil k synum svým, rka: Když já umru, pochovejte mne v témž hrobe, v nemž muž Boží jest pochován, vedlé kostí jeho položte kosti mé.
\par 32 Nebot se jiste stane to, což ohlásil slovem Hospodinovým proti oltári, kterýž jest v Bethel, a proti všechnem domum výsostí, kteréž jsou v mestech Samarských.
\par 33 Po techto príbezích neodvrátil se Jeroboám od cesty své zlé, ale opet nadelal z lidu obecného kneží výsostí. Kdo jen chtel, posvetil ruky jeho, a ten byl knezem výsostí.
\par 34 I byla ta vec domu Jeroboámovu prícinou k hrešení, aby vyplénen a vyhlazen byl se svrchku zeme.

\chapter{14}

\par 1 Toho casu roznemohl se Abiáš, syn Jeroboámuv.
\par 2 I rekl Jerobám žene své: Vstan medle a zmen se, aby nepoznali, že jsi žena Jeroboámova, a jdi do Sílo. Hle, tam jest Achiáš prorok, kterýž mi byl predpovedel, že mám býti králem nad lidem tímto.
\par 3 A vezma v ruce své deset chlebu a kolácu, a láhvici medu, jdi k nemu; ont oznámí tobe, co se stane pacholeti tomuto.
\par 4 I ucinila tak žena Jeroboámova; nebo vstavši, šla do Sílo, a prišla do domu Achiášova. Ale Achiáš nemohl již hledeti, nebo pošly mu byly oci pro starost jeho.
\par 5 Hospodin pak rekl Achiášovi: Aj, žena Jeroboámova jde, aby se tebe neco zeptala o synu svém, proto že jest nemocen. Toto a toto budeš jí mluviti. Budet pak, že když prijde, ciniti se bude jinou.
\par 6 Protož uslyšev Achiáš šust noh jejích, když vcházela do dverí, rekl: Pod, ženo Jeroboámova, proc se jinou ciníš? Já zajisté poslán jsem k tobe s recí tvrdou.
\par 7 Jdi, povez Jeroboámovi: Toto praví Hospodin Buh Izraelský: Ponevadž jsem te vyvýšil z prostredku lidu, a postavil jsem te za vudce nad lidem svým Izraelským,
\par 8 A odtrhl jsem království od domu Davidova, a dal jsem je tobe, ty však nebyl jsi jako služebník muj David, kterýž ostríhal prikázaní mých, a následoval mne v celém srdci svém, cine toliko to, což jest pravého prede mnou,
\par 9 Ale zlost jsi páchal hojneji nade všecky, kteríž byli pred tebou; nebo odšed, ucinils sobe bohy cizí a slité, abys mne dráždil, mne pak zavrhl jsi za hrbet svuj:
\par 10 Protož aj, já uvedu zlé veci na dum Jeroboámuv, a vyhladím z Jeroboáma, i toho, jenž mocí na stenu, i zajatého i zanechaného v Izraeli, a vyvrhu ostatky domu Jeroboámova, jako hnuj vymítán bývá až docista.
\par 11 Toho, kdož z domu Jeroboámova umre v meste, psi žráti budou, a toho, kdož umre na poli, ptáci nebeští jísti budou; nebot jest mluvil Hospodin.
\par 12 Ty pak vstana, jdi do domu svého, a když vcházeti budeš do mesta, tehdy umre pachole.
\par 13 I budou ho plakati všecken lid Izraelský, a pochovají jej; nebo ten sám z domu Jeroboámova dostane se do hrobu, proto že o nem z domu Jeroboámova jest slovo dobré u Hospodina Boha Izraelského.
\par 14 Vyzdvihnet však sobe Hospodin krále nad Izraelem, kterýž vyhladí dum Jeroboámuv toho dne. Ale co? Nýbrž již vyzdvihl.
\par 15 A zaklátí Hospodin Izraelem tak, jako se klátí trtina u vodách, a vykorení Izraele z zeme výborné této, kterouž dal otcum jejich, a rozptýlí je daleko za reku, proto že sobe zdelali háje, popouzejíce Hospodina.
\par 16 A tak vydá Izraele pro hríchy Jeroboámovy, kterýž i sám hrešil, i v hrích uvodil Izraele.
\par 17 Tedy vstavši žena Jeroboámova, odešla a prišla do Tersa, a když vstupovala na prah domu, umrelo pachole.
\par 18 I pochovali je, a plakal ho všecken lid Izraelský vedlé reci Hospodinovy, kterouž mluvil skrze služebníka svého, Achiáše proroka.
\par 19 Jiné pak veci Jeroboámovy, jaké boje vedl a kterak kraloval, aj, sepsány jsou v knize o králích Izraelských.
\par 20 Všech dnu, v nichž kraloval Jeroboám, bylo dvamecítma let. I usnul s otci svými, a kraloval Nádab syn jeho místo neho.
\par 21 Roboám také syn Šalomounuv kraloval nad Judou. V jednom a ve ctyridcíti letech byl Roboám, když pocal kralovati, a sedmnácte let kraloval v Jeruzaléme meste, kteréž vyvolil Hospodin ze všech pokolení Izraelských, aby tam prebývalo jméno jeho. Bylo pak jméno matky jeho Naama Ammonitská.
\par 22 Cinil také lid Judský to, což zlého jest pred Hospodinem, a k zurivosti ho popudili hríchy svými, kterýmiž hrešili, více nežli otcové jejich všemi vecmi, kteréž cinili.
\par 23 Nebo i oni vystaveli sobe výsosti a sloupy, i háje na každém pahrbku vysokém a pod každým stromem zeleným.
\par 24 Presto byli ohyzdní sodomári v zemi té, ciníce vedlé všech ohavností pohanských, kteréž byl vyplénil Hospodin pred tvárí synu Izraelských.
\par 25 I stalo se léta pátého království Roboámova, že vytáhl Sesák král Egyptský proti Jeruzalému,
\par 26 A pobral poklady domu Hospodinova a poklady domu královského, všecko to pobral. Vzal také všecky pavézy zlaté, kterýchž byl nadelal Šalomoun.
\par 27 Místo kterýchž nadelal král Roboám pavéz medených, a porucil je úredníkum nad drabanty, kteríž ostríhali brány domu královského.
\par 28 A když král chodíval do domu Hospodinova, nosili je drabanti, a zase je prinášeli do pokoje drabantu.
\par 29 Jiné pak veci Roboámovy, a všecko, což cinil, zdaliž není zapsáno v knize o králích Judských.
\par 30 Ano i válka, kteráž byla mezi Roboámem a Jeroboámem po všecky dny.
\par 31 I usnul Roboám s otci svými, a pochován jest s nimi v meste Davidove. A jméno matky jeho bylo Naama Ammonitská. I kraloval Abiam syn jeho místo neho.

\chapter{15}

\par 1 Léta pak osmnáctého království Jeroboáma syna Nebatova, kraloval Abiam nad Judou.
\par 2 Tri léta kraloval v Jeruzaléme. A jméno matky jeho bylo Maacha, dcera Abissalomova.
\par 3 Ten chodil ve všech hríších otce svého, kteréž páchal pred oblícejem jeho; a nebylo srdce jeho celé pri Hospodinu Bohu jeho, jako srdce Davida otce jeho.
\par 4 A však pro Davida dal jemu Hospodin Buh jeho svíci v Jeruzaléme, vzbudiv syna jeho po nem, a utvrdiv Jeruzalém,
\par 5 Proto že David cinil to, což bylo dobrého pred Hospodinem, a neuchýlil se od žádné veci z tech, kteréž jemu byl prikázal, po všecky dny života svého, krome skutku pri Uriášovi Hetejském.
\par 6 A byl boj mezi Roboámem a Jeroboámem po všecky dny života jeho.
\par 7 Jiné pak veci Abiamovy, a všecko což cinil, popsáno jest v knize o králích Judských, ano i o boji mezi Abiamem a Jeroboámem.
\par 8 A když usnul Abiam s otci svými, pochovali ho v meste Davidove. I kraloval Aza syn jeho místo neho.
\par 9 A tak léta dvadcátého Jeroboáma, krále Izraelského, kraloval Aza nad Judou.
\par 10 Jedno a ctyridceti let kraloval v Jeruzaléme. A jméno matky jeho bylo Maacha, dcera Abissalomova.
\par 11 I cinil Aza, což dobrého bylo pred Hospodinem, jako David otec jeho.
\par 12 Nebo ohyzdné sodomáre vyplénil z zeme, a odjal všecky modly, kterýchž nadelali otcové jeho.
\par 13 Nadto i Maachu matku svou ssadil, aby nebyla královnou; nebo byla udelala hroznou modlu v háji. Protož podtal Aza modlu tu ohyzdnou, a spálil pri potoku Cedron.
\par 14 A ackoli výsosti nebyly zkaženy, a však srdce Azovo bylo celé pri Bohu po všecky dny jeho.
\par 15 Vnesl také ty veci, kteréž posveceny byly od otce jeho, i ty veci, kterýchž sám posvetil, do domu Hospodinova, stríbro a zlato i nádoby.
\par 16 Procež byla válka mezi Azou a Bázou, králem Izraelským, po všecky dny jejich.
\par 17 Nebo Báza král Izraelský vytáhl proti Judovi, a stavel Ráma, aby nedal žádnému vyjíti ani jíti k Azovi králi Judskému.
\par 18 Ale vzav Aza všecko stríbro i zlato, kteréž pozustalo v pokladích domu Hospodinova, a v pokladích domu královského, dal je v ruce služebníku svých, a poslal je král Aza k Benadadovi synu Tabremmon, syna Hezion, králi Syrskému, kterýž bydlil v Damašku, rka:
\par 19 Smlouva jest mezi mnou a mezi tebou, mezi otcem mým a mezi otcem tvým. Aj, ted posílám tobe dar, stríbro a zlato; jdi, zruš smlouvu svou s Bázou králem Izraelským, at odtrhne ode mne.
\par 20 I uposlechl Benadad krále Azy, a poslav knížata s vojsky svými proti mestum Izraelským, dobyl Jon a Dan, též Abelbetmaachy, i všeho Ceneretu, a vší zeme Neftalím.
\par 21 To když uslyšel Báza, prestal staveti Ráma, a zustal v Tersa.
\par 22 Tedy král Aza provolal po všem Judstvu, žádného nevymenuje. I pobrali to kamení z Ráma, i dríví jeho, z nehož stavel Báza, a vystavel z toho král Aza Gabaa Beniaminovo a Masfa.
\par 23 Jiné pak všecky veci Azovy, i všecka síla jeho a cokoli delal, i jaká mesta vystavel, zapsáno jest v knize o králích Judských. Jen že v veku starosti své byl nemocný na nohy.
\par 24 I usnul Aza s otci svými, a pochován jest s nimi v meste Davida otce svého. A kraloval místo neho Jozafat syn jeho.
\par 25 Nádab pak syn Jeroboámuv, pocal kralovati nad Izraelem léta druhého Azy, krále Judského, a kraloval nad Izraelem dve léte.
\par 26 A cinil to, což zlého jest pred ocima Hospodinovýma, chode po ceste otce svého, v hríších jeho, jimiž k hrešení privodil lid Izraelský.
\par 27 I cinil mu úklady Báza syn Achiášuv z domu Izachar, a porazil ho Báza u Gebbeton, kteréž bylo Filistinských. Nádab zajisté a všickni Izraelští oblehli byli Gebbeton.
\par 28 A tak zabil ho Báza léta tretího Azy, krále Judského, a kraloval místo neho.
\par 29 I stalo se, když pocal kralovati, že zahubil všecken dum Jeroboámuv, a nepozustavil žádné duše z Jeroboáma, až je i vyhladil vedlé reci Hospodinovy, kterouž byl mluvil skrze služebníka svého Achiáše Silonského,
\par 30 Pro hríchy Jeroboámovy, jimiž hrešil, a jimiž k hrešení privodil lid Izraelský, pro dráždení jeho, kterýmž dráždil Hospodina Boha Izraelského.
\par 31 Jiné pak veci Nádabovy a všecko, což cinil, zapsáno jest v knize o králích Izraelských.
\par 32 Byla tedy válka mezi Azou, a mezi Bázou králem Izraelským, po všecky dny jejich.
\par 33 Léta tretího Azy, krále Judského, kraloval Báza syn Achiášuv nade vším Izraelem v Tersa za ctyrmecítma let.
\par 34 A cinil to, což zlého jest pred Hospodinem, chode po ceste Jeroboámove a v hríších jeho, jimiž k hrešení privodil Izraele.

\chapter{16}

\par 1 Stala se pak rec Hospodinova k Jéhu, synu Chanani, proti Bázovi, rkoucí:
\par 2 Proto že jsem te vyzdvihl z prachu, a postavil za vudci lidu mého Izraelského, ty jsi pak chodil po ceste Jeroboámove, a privedls k hrešení lid muj Izraelský, aby mne popouzeli hríchy svými:
\par 3 Aj, já vyhladím potomky Bázovy a potomky domu jeho, a uciním domu tvému, jako domu Jeroboáma, syna Nebatova.
\par 4 Toho, kdož z rodiny Bázovy umre v meste, psi žráti budou, a kdož z nich umre na poli, ptáci nebeští jísti budou.
\par 5 Jiné pak veci Bázovy a všecko, což cinil, i síla jeho, o tom zapsáno jest v knize o králích Izraelských.
\par 6 Když pak usnul Báza s otci svými, pochován jest v Tersa, a Ela syn jeho kraloval místo neho.
\par 7 A tak skrze Jéhu syna Chanani, proroka, stala se rec Hospodinova proti Bázovi a proti domu jeho, i proti všemu zlému, kteréž cinil pred oblícejem Hospodinovým, popouzeje ho dílem rukou svých, že má býti podobný domu Jeroboámovu, a proto že jej zabil.
\par 8 Léta dvadcátého šestého Azy, krále Judského, kraloval Ela syn Bázuv nad Izraelem v Tersa dve léte.
\par 9 I zprotivil se jemu služebník jeho Zamri, hejtman nad polovicí vozu, když on v Tersa kvasil a opilý byl v dome Arsy, vládare mesta Tersa.
\par 10 V tom Zamri prišed, ranil ho, a zabil jej léta dvadcátého sedmého Azy krále Judského, a kraloval místo neho.
\par 11 Když pak kraloval a sedel na stolici jeho, pobil všecken dum Bázuv, i príbuzné jeho, i prátely jeho, nepozustaviv z neho ani mocícího na stenu.
\par 12 A tak vyhladil Zamri všecken dum Bázuv vedlé reci Hospodinovy, kterouž byl mluvil proti Bázovi skrze Jéhu proroka,
\par 13 Pro všecky hríchy Bázovy, i hríchy Ela syna jeho, kteríž hrešili, i v hríchy uvodili Izraele, popouzejíce Hospodina Boha Izraelského marnostmi svými.
\par 14 Jiné pak veci Ela a všecko, což cinil, vypsáno jest v knize o králích Izraelských.
\par 15 Léta dvadcátého sedmého Azy, krále Judského, kraloval Zamri v Tersa sedm dní, když lid vojenský ležel proti Gebbeton Filistinských.
\par 16 Nebo uslyšev lid, kterýž byl v ležení,takové veci, že by se Zamri zprotivil a že krále zabil, tedy všecken Izrael ustanovili sobe krále Amri, hejtmana nad vojskem Izraelským toho casu v vojšte.
\par 17 Protož táhl Amri a s ním všecken Izrael od Gebbeton, a oblehli Tersu.
\par 18 A když videl Zamri, že již mesto jest vzato, všed na palác domu královského, zapálil nad sebou dum královský, i umrel,
\par 19 Pro hríchy své, kterýmiž hrešil, cine, což zlého jest pred oblícejem Hospodinovým, a chode po ceste Jeroboámove a v hríších jeho, kteréž páchal, privozuje k hrešení lid Izraelský.
\par 20 Jiné pak veci Zamri i jeho úkladové, kteréž cinil, zapsáni jsou v knize o králích Izraelských.
\par 21 Tedy rozdelil se lid Izraelský na dvé. Polovice lidu postoupilo po Tebni synu Ginet, aby ho ucinili králem, a druhá polovice postoupila po Amri.
\par 22 Ale premohl lid, kterýž postoupil po Amri, lid ten, kterýž postoupil po Tebni synu Ginet. I umrel Tebni, a kraloval Amri.
\par 23 Léta tridcátého prvního Azy, krále Judského, kraloval Amri nad Izraelem dvanácte let. V Tersa kraloval šest let.
\par 24 I koupil horu Someron od Semera za dve hrivny stríbra, a když vystavel tu horu, nazval jméno mesta toho, kteréž vzdelal, od jména Semery, pána té hory, totiž Samarí.
\par 25 Cinil pak Amri to, což jest zlého pred oblícejem Hospodinovým; nýbrž horší veci cinil, než kdo ze všech, kteríž pred ním byli.
\par 26 Nebo chodil po všeliké ceste Jeroboáma syna Nebatova, a ve všech hríších jeho, kterýmiž privodil k hrešení Izraele, popouzeje Hospodina Boha Izraelského marnostmi svými.
\par 27 Jiné pak veci Amri i všecko, což cinil, i síla jeho, kterouž prokazoval, o tom zapsáno jest v knize o králích Izraelských.
\par 28 I usnul Amri s otci svými, a pochován jest v Samarí, a kraloval místo neho Achab syn jeho.
\par 29 Achab tedy syn Amri kraloval nad Izraelem léta tridcátého osmého Azy, krále Judského, a kraloval Achab syn Amri nad Izraelem v Samarí dvamecítma let.
\par 30 I cinil Achab syn Amri pred oblícejem Hospodinovým horší veci než kdo ze všech, kteríž pred ním byli.
\par 31 V tom stalo se, (nebo málo mu to bylo, že chodil v hríších Jeroboáma syna Nebatova), že sobe pojal ženu Jezábel dceru Etbál, krále Sidonského, a odšed, sloužil Bálovi a klanel se jemu.
\par 32 A vzdelal oltár Bálovi v chráme Bálove, kterýž byl ustavel v Samarí.
\par 33 Udelal také Achab i háj, a tak pricinil toho, cím by popouzel Hospodina Boha Izraelského, nade všecky jiné krále Izraelské, kteríž byli pred ním.
\par 34 Za dnu jeho Hiel Bethelský vystavel Jericho. V Abiramovi prvorozeném svém založil je, a v Segubovi nejmladším svém postavil brány jeho, vedlé reci Hospodinovy, kterouž byl mluvil skrze Jozue syna Nun.

\chapter{17}

\par 1 Tedy Eliáš Tesbitský, obyvatel Galádský, mluvil k Achabovi: Živt jest Hospodin Buh Izraelský, pred jehož oblícejem stojím, že nebude techto let rosy ani dešte, jediné vedlé reci mé.
\par 2 I stalo se slovo Hospodinovo k nemu, rkoucí:
\par 3 Odejdi odsud a obrat se k východu, a skrej se u potoka Karit, kterýž jest naproti Jordánu.
\par 4 A budeš z toho potoka píti, krkavcum pak jsem prikázal, aby te tam krmili.
\par 5 Kterýž odšed, ucinil, jakž mu prikázal Hospodin. Nebo odšed, usadil se pri potoku Karit, kterýž byl proti Jordánu.
\par 6 A krkavci prinášeli jemu chléb a maso každého jitra, též chléb a maso každého vecera, a z potoka pil.
\par 7 Tedy po prebehnutí dnu nekterých vysechl ten potok, nebo nebylo žádného dešte v té zemi.
\par 8 I stalo se slovo Hospodinovo k nemu, rkoucí:
\par 9 Vstan a jdi do Sarepty Sidonské, a bud tam. Aj, prikázal jsem žene vdove, aby te živila.
\par 10 Kterýž vstav, šel do Sarepty, a prišel k bráne mesta, a aj, žena vdova sbírala tu dríví. A zavolav jí, rekl: Medle, prines mi trochu vody v nádobe, abych se napil.
\par 11 A když šla, aby prinesla, zavolal jí zase a rekl: Medle, prines mi také kousek chleba v ruce své.
\par 12 I odpovedela: Živt jest Hospodin Buh tvuj, žet nemám žádného chleba, ani podpopelného, krome hrsti mouky v kbelíku a malicko oleje v nádobce, a aj, sbírám dve dreve, abych šla a pripravila to sobe a synu svému, abychom snedouce to, za tím zemríti musili.
\par 13 I rekl jí Eliáš: Neboj se. Jdi a ucin, jakž jsi rekla, a však udelej mi prvé z toho malý chléb podpopelný a prines mi; potom sobe a synu svému udeláš.
\par 14 Nebot toto praví Hospodin Buh Izraelský: Mouka z kbelíka toho nebude strávena, aniž oleje v nádobce té ubude až do toho dne, když dá Hospodin déšt na zemi.
\par 15 I šla a ucinila vedlé reci Eliášovy, a jedla ona i on i celed její až do tech dní.
\par 16 Z kbelíka toho mouka nebyla strávena, aniž oleje v nádobce ubylo, vedlé reci Hospodinovy, kterouž mluvil skrze Eliáše.
\par 17 A když to prebehlo, stalo se, že se roznemohl syn ženy, paní toho domu, a byla nemoc jeho velmi težká, tak že nezustalo v nem dýchání.
\par 18 Protož rekla Eliášovi: Co mne a tobe, muži Boží? Prišel jsi ke mne, abys mi pripomenul nepravost mou, a umoril syna mého?
\par 19 Kterýž odpovedel jí: Dej sem syna svého. A vzav ho z luna jejího, vnesl jej na sín, kdež sám prebýval, a položil ho na lužko své.
\par 20 Tedy volal k Hospodinu a rekl: Hospodine Bože muj, také-liž s tou vdovou, u kteréž pohostinu jsem, tak zle nakládáš, že jsi umoril syna jejího?
\par 21 A roztáh se nad dítetem po trikrát, zvolal k Hospodinu a rekl: Hospodine Bože muj, prosím, necht se navrátí duše dítete tohoto do vnitrností jeho.
\par 22 I vyslyšel Hospodin hlas Eliášuv, a navrátila se duše dítete do vnitrností jeho, i ožilo.
\par 23 A vzav Eliáš díte, snesl je z síne do domu, a dal je matce jeho. I rekl Eliáš: Pohled, syn tvuj živ jest.
\par 24 Tedy rekla žena Eliášovi: Jižt jsem nyní poznala, že jsi muž Boží, a že rec Hospodinova v ústech tvých jest pravá.

\chapter{18}

\par 1 Potom po mnohých dnech, po tretím tom léte, stalo se slovo Hospodinovo k Eliášovi, rkoucí: Jdi, ukaž se Achabovi, nebot dám déšt na zemi.
\par 2 Odšel tedy Eliáš, aby se ukázal Achabovi. Byl pak hlad veliký v Samarí.
\par 3 I povolal Achab Abdiáše, kterýž byl predstaven domu jeho. (Abdiáš pak bál se Hospodina velmi.
\par 4 Nebo když Jezábel mordovala proroky Hospodinovy, vzal Abdiáš sto proroku a skryl je po padesáti v jeskyni, a krmil je chlebem a vodou.)
\par 5 A rekl Achab Abdiášovi: Jdi skrze tu zemi ke všechnem studnicím vod, a ke všechnem potokum, zdali bychom kde nalezli trávu, abychom živé zachovali kone a mezky, a nezmorili dobytka.
\par 6 I rozdelili sobe zemi, kterouž by prošli. Achab šel jednou cestou sám, Abdiáš také šel cestou druhou sám.
\par 7 A když Abdiáš byl na ceste, aj, Eliáš potkal se s ním. Kterýž když ho poznal, padl na tvár svou a rekl: Nejsi-liž ty, pane muj, Eliáš?
\par 8 Odpovedel jemu: Jsem. Jdi, povez pánu svému: Aj, Eliáš prišel.
\par 9 Jemuž rekl: Což jsem zhrešil, že služebníka svého vydati chceš v ruku Achabovu, aby mne zamordoval?
\par 10 Živt jest Hospodin Buh tvuj, žet není národu ani království, kamž by neposlal pán muj hledati te, a když rekli, že tebe není, prísahy požádal od království a národu, že te nemohou nalezti.
\par 11 A ty nyní pravíš: Jdi, povez pánu svému: Aj, Eliáš prišel.
\par 12 I stalo by se, že jakž bych odšel od tebe, tedy duch Hospodinuv zanesl by te nevím kam, já pak jda, oznámil bych Achabovi, a když by te nenalezl, zabil by mne. Služebník zajisté tvuj bojí se Hospodina od mladosti své.
\par 13 Zdaliž není oznámeno pánu mému, co jsem ucinil, když Jezábel mordovala proroky Hospodinovy, že jsem skryl z proroku Hospodinových sto mužu, po padesáti mužích v jeskyni, a krmil jsem je chlebem a vodou?
\par 14 Ty pak nyní pravíš: Jdi, povez pánu svému: Aj, Eliáš prišel. A zamordujet mne.
\par 15 Odpovedel Eliáš: Živt jest Hospodin zástupu, pred jehož oblícejem stojím, žet se jemu dnes ukáži.
\par 16 A tak šel Abdiáš vstríc Achabovi, a oznámil jemu. Procež šel i Achab v cestu Eliášovi.
\par 17 A když uzrel Achab Eliáše, rekl Achab k nemu: Zdaliž ty nejsi ten, kterýž kormoutíš lid Izraelský?
\par 18 Kterýž odpovedel: Ját nekormoutím lidu Izraelského, ale ty a dum otce tvého, když opouštíte prikázaní Hospodinova a následujete Bálu.
\par 19 Protož nyní pošli a shromažd ke mne všecken lid Izraelský na horu Karmel, a proroku Bálových ctyri sta a padesáte, i proroku toho háje ctyri sta, kteríž jedí z stolu Jezábel.
\par 20 Obeslal tedy Achab všecky syny Izraelské, a shromáždil ty proroky na horu Karmel.
\par 21 A pristoupiv Eliáš ke všemu lidu, rekl: I dokudž kulhati budete na obe strane? Jestližet jest Hospodin Bohem, následujtež ho; pakli jest Bál, jdetež za ním. A neodpovedel jemu lid žádného slova.
\par 22 Opet rekl Eliáš lidu: Já sám toliko pozustal jsem prorok Hospodinuv, proroku pak Bálových jest ctyri sta a padesáte mužu.
\par 23 Necht jsou nám dáni dva volkové, a at vyberou sobe volka jednoho, kteréhož necht rozsekají na kusy a vkladou na dríví, ale ohne at nepodkládají. Ját také pripravím volka druhého, kteréhož vložím na dríví, a ohne nepodložím.
\par 24 Tedy vzývejte jméno bohu vašich, já pak vzývati budu jméno Hospodinovo, a budet to, že Buh, kterýž se ohlásí skrze ohen, ten jest Buh. A odpovedel všecken lid, rka: Dobrát jest rec tato.
\par 25 I rekl Eliáš prorokum Bálovým: Vyberte sobe volka jednoho a pripravte ho nejprvé, ponevadž jest vás více, a vzývejte jméno bohu vašich, ale ohne nepodkládejte.
\par 26 A tak vzali volka, kteréhož jim dal, a pripravili a vzývali jméno Bálovo od jitra až do poledne, ríkajíce: Ó Báli, uslyš nás. Ale nebylo hlasu, ani kdo by odpovedel. I skákali u oltáre, kterýž byli udelali.
\par 27 Když pak bylo poledne, posmíval se jim Eliáš a rekl: Kricte vysokým hlasem, ponevadž buh jest. Neb snad rozmlouvání má, neb jinou práci, neb jest na ceste, anebt spí, at procítí.
\par 28 Takž kriceli hlasem velikým, a bodli se vedlé obyceje svého nožíky a špicemi, tak až se krví polívali.
\par 29 I stalo se, když již bylo s poledne, že prorokovali až dotud, když se obetuje obet suchá, ale nebylo žádného se ohlášení, ani kdo by odpovedel, ani kdo by vyslyšel.
\par 30 Zatím rekl Eliáš všemu lidu: Pristuptež ke mne. I pristoupil všecken lid k nemu. Tedy opravil oltár Hospodinuv, kterýž byl zborený.
\par 31 Nebo vzal Eliáš dvanácte kamenu, (vedlé poctu pokolení synu Jákobových, k nemuž se byla stala rec Hospodinova, že Izrael bude jméno jeho),
\par 32 A vzdelal z tech kamenu oltár ve jménu Hospodinovu; udelal také struhu vukol oltáre zšírí, co by mohl dve míry obilé vsíti.
\par 33 Narovnal i dríví, a rozsekav volka na kusy, vkladl na dríví.
\par 34 A rekl: Naplnte ctyri stoudve vodou, a vylíte na obet zápalnou i na dríví. Rekl opet: Ucintež to po druhé. I ucinili po druhé.Rekl ješte: Po tretí ucinte. I ucinili po tretí,
\par 35 Tak že tekly vody okolo oltáre; také i struhu naplnila voda.
\par 36 Stalo se pak, když se obetuje obet suchá, pristoupil Eliáš prorok a rekl: Hospodine Bože Abrahamuv, Izákuv a Izraeluv, necht dnes poznají, že jsi ty Buh v Izraeli, a já služebník tvuj, a že jsem vedlé slova tvého cinil všecky veci tyto.
\par 37 Vyslyš mne, Hospodine, vyslyš mne, atby poznal lid tento, že jsi ty, Hospodine, Bohem, když bys obrátil srdce jejich zase.
\par 38 V tom spadl ohen Hospodinuv, a spálil obet zápalnou, dríví i kamení i prst; též vodu, kteráž byla v struze, vypil.
\par 39 Což když uzrel všecken ten lid, padli na tvári své a rekli: Hospodint jest Bohem, Hospodin jest Bohem.
\par 40 I rekl jim Eliáš: Zjímejte ty proroky Bálovy, žádný at z nich neujde. I zjímali je. Kteréž svedl Eliáš ku potoku Císon, a tam je zmordoval.
\par 41 Tedy rekl Eliáš Achabovi: Jdi, pojez a napí se, nebo aj, zvuk velikého dešte.
\par 42 I jel Achab, aby pojedl a napil se. Eliáš pak vstoupil na vrch Karmele, a rozprostrel se na zemi a sklonil tvár svou k kolenum svým.
\par 43 Potom rekl mládenci svému: Vystup nyní, a pohled tam k mori. Kterýž vystoupiv, pohledel a rekl: Nic není. Rekl opet: Vrat se po sedmkrát.
\par 44 I stalo se po sedmé, rekl: Aj, oblak malický jako dlan clovecí vystupuje z more. Ješte rekl: Jdi, rci Achabovi: Zapráhej a jed, aby te nezastihl déšt.
\par 45 Stalo se mezi tím, když se nebesa zamracila oblakem a vetrem, odkudž byl déšt veliký, že jel Achab a prišel do Jezreel.
\par 46 Ruka pak Hospodinova byla s Eliášem, tak že prepásav bedra svá, bežel pred Achabem, až prišel do Jezreel.

\chapter{19}

\par 1 Tedy oznámil Achab Jezábel všecko to, což ucinil Eliáš, a že naprosto všecky proroky její pobil mecem.
\par 2 A protož poslala Jezábel posla k Eliášovi, rkuci: Toto at mi uciní bohové a toto pridadí, jestliže v tuto hodinu zítra neuciním tobe, jako ty kterému z nich.
\par 3 Což když zvedel, vstana, odšel pro zachování života svého, a prišel do Bersabé, jenž jest v Judstvu, kdež nechal mládence svého.
\par 4 Sám pak šel predce po poušti cestou dne jednoho, a prišed, usadil se pod jedním jalovcem, a žádal sobe smrti a rekl: Jižt jest dosti, ó Hospodine, vezmi duši mou, nebt nejsem lepší otcu svých.
\par 5 I lehl a usnul pod tím jalovcem. A aj, v touž chvíli andel dotekl se ho a rekl jemu: Vstan, pojez.
\par 6 A když pohledel vukol, a aj, u hlavy jeho chléb na uhlí pecený a cíše vody. I pojedl a napil se, a lehl zase.
\par 7 Opet vrátiv se andel Hospodinuv po druhé, dotekl se ho a rekl: Vstan, pojez, nebo velmi dlouhou máš cestu pred sebou.
\par 8 A vstav, jedl a pil, a šel v síle pokrmu toho ctyridceti dní a ctyridceti nocí, až na horu Boží Oréb.
\par 9 Kdežto všed do jakési jeskyne, nocoval tam. A aj, rec Hospodinova k nemu, a rekl jemu: Co tu deláš, Eliáši?
\par 10 Kterýž odpovedel: Velice jsem horlil pro Hospodina Boha zástupu; nebo opustili smlouvu tvou synové Izraelští, oltáre tvé zborili a proroky tvé zmordovali mecem. I zustal jsem já sám, ted pak hledají života mého, aby mi jej odjali.
\par 11 Rekl Buh: Vyjdi a stuj na hore pred Hospodinem. A aj, Hospodin šel tudy, a vítr veliký a silný, podvracující hory a rozrážející skály pred Hospodinem, ale nebyl v tom vetru Hospodin. Za tím pak vetrem zeme tresení, ale nebyl v tom tresení Hospodin.
\par 12 A za tresením ohen, ale nebyl v ohni Hospodin. A za ohnem hlas tichý a temný.
\par 13 Což když uslyšel Eliáš, zavinul tvár svou pláštem svým, a vyšed, stál u dverí jeskyne, a aj, hlas k nemu, rkoucí: Co tu deláš, Eliáši?
\par 14 Kterýž odpovedel: Náramne horlil jsem pro Hospodina Boha zástupu; nebo opustili smlouvu tvou synové Izraelští, oltáre tvé zborili, a proroky tvé pomordovali mecem. I zustal jsem já sám, ted pak hledají života mého, aby mi jej odjali.
\par 15 Ale Hospodin rekl jemu: Jdi, navrat se cestou svou k Damašské poušti, kdežto prijda, pomažeš Hazaele za krále nad Syrií.
\par 16 Jéhu také syna Namsi pomažeš za krále nad Izraelem, a Elizea syna Safatova z Abelmehula pomažeš za proroka místo sebe.
\par 17 I stane se, že toho, kdož by ušel mece Hazaelova, zamorduje Jéhu, a toho, kdož by ušel mece Jéhu, zamorduje Elizeus.
\par 18 Zachovalt jsem pak v Izraeli sedm tisícu, jichžto všech kolena nesklánela se Bálovi, a jichžto všech ústa nelíbala ho.
\par 19 A tak odšed odtud, nalezl Elizea syna Safatova, an ore, a dvanáctero sprežení pred ním, sám také byl pri dvanáctém. A jda mimo nej Eliáš, uvrhl plášt svuj na nej.
\par 20 Kterýž zanechav volu, bežel za Eliášem a rekl: Prosím, necht políbím otce svého a matky své, i pujdu za tebou. Jemuž rekl: Jdi, vrat se zase, nebo vidíš, cot jsem ucinil.
\par 21 Navrátil se tedy od neho, a vzav pár volu, zabil je a drívím z pluhu uvaril maso jejich, kteréž dal lidu, i jedli. A vstav, šel za Eliášem a prisluhoval jemu.

\chapter{20}

\par 1 Tedy Benadad král Syrský, shromáždiv všecka vojska svá, a maje s sebou tridceti a dva krále, též kone i vozy, vytáhl a oblehl Samarí a dobýval ho.
\par 2 I poslal posly k Achabovi králi Izraelskému do mesta,
\par 3 A vzkázal jemu: Takto praví Benadad: Stríbro tvé a zlato tvé mé jest, též ženy tvé i synové tvoji nejzdárnejší moji jsou.
\par 4 I odpovedel král Izraelský a rekl: Vedlé reci tvé, pane muj králi, tvuj jsem i všecko, což mám.
\par 5 Opet navrátivše se poslové, rekli: Takto praví Benadad: Poslal jsem k tobe, atby rekli: Stríbro své a zlato své, ženy své i syny své dáš mi.
\par 6 Ale však o tomto case zítra pošli služebníky své k tobe, aby prehledali dum tvuj i domy služebníku tvých, a onit všecko, cožkoli máš nejlepšího, vezmouce v ruce své, poberou.
\par 7 A tak povolav král Izraelský všech starších zeme té, rekl: Posudte medle,a vizte, jakt ten zlého hledá; nebo poslal ke mne pro ženy mé a pro syny mé, též pro stríbro mé i pro zlato mé, a neodeprel jsem jemu.
\par 8 I rekli jemu všickni starší a všecken lid: Neposlouchejž ho, ani mu povoluj.
\par 9 Protož odpovedel poslum Benadadovým: Povezte pánu mému králi: Všecko to, o cež jsi poslal k služebníku svému prvé, uciním, ale této veci uciniti nemohu. Takž odešli poslové, a donesli mu tu odpoved.
\par 10 Ješte poslal k nemu Benadad a rekl: Toto at mi uciní bohové a toto at pridadí, dostane-li se prachu Samarského do hrstí všeho lidu, kterýž jest se mnou.
\par 11 Zase odpovedel král Izraelský a rekl: Povezte, at se nechlubí ten, kterýž se strojí do zbroje, jako ten, kterýž svlácí zbroj.
\par 12 I stalo se, když uslyšel tu rec, (nebo pil on i králové v staních), že rekl služebníkum svým: Pritrhnete. I pritrhli k mestu.
\par 13 A hle, prorok nejaký prišed k Achabovi králi Izraelskému, rekl: Toto praví Hospodin: Zdaliž jsi nevidel všeho množství tohoto velikého? Hle, já dám je dnes v ruku tvou, abys poznal, že já jsem Hospodin.
\par 14 A když rekl Achab: Skrze koho? odpovedel on: Tak praví Hospodin: Skrze služebníky knížat kraju. Rekl ješte: Kdo svede tu bitvu? Odpovedel: Ty.
\par 15 A protož sectl služebníky knížat kraju, jichž bylo dve ste tridceti a dva; po nich sectl i všecken lid všech synu Izraelských, sedm tisícu.
\par 16 I vytáhli o poledni. Benadad pak pil a ožral se v staních, on i tridceti a dva králové pomocníci jeho.
\par 17 A tak vytáhli služebníci knížat kraju nejprvé. I poslal Benadad, (když mu povedeli, rkouce: Muži vytáhli z Samarí),
\par 18 A rekl: Bud že vytáhli o pokoj, zjímejte je živé, bud že k bitve vytáhli, zjímejte je živé.
\par 19 Tak, pravím, vytáhli z mesta ti služebníci knížat kraju, a vojsko za nimi táhlo.
\par 20 A porazili jeden každý muže svého, tak že utíkali Syrští, Izraelští pak honili je. Ale Benadad král Syrský utekl na koni s jízdnými.
\par 21 Potom vytáhl král Izraelský, a pobil kone i vozy zkazil, a tak porazil Syrské ranou velikou.
\par 22 Opet prišel prorok ten k králi Izraelskému, a rekl jemu: Jdiž, zmužile se mej a vez i viz, co bys mel ciniti; nebo zase po roce král Syrský vytáhne proti tobe.
\par 23 Tedy služebníci krále Syrského rekli jemu: Bohové hor jsout bohové jejich, protož nás premohli; ale bojujme proti nim na rovinách, zvíš, nepremužeme-li jich.
\par 24 Protož ucin toto: Odbud tech králu, jednoho každého z místa jeho, a postav vývody místo nich.
\par 25 Ty pak secti sobe vojsko z svých jako vojsko onech, kteríž padli, a kone jako ony kone, a vozy jako ony vozy; i budeme bojovati proti nim na rovinách, a zvíš, nepremužeme-lit jich. I uposlechl hlasu jejich a ucinil tak.
\par 26 I stalo se po roce, že sectl Benadad Syrské a vytáhl do Afeku, aby bojoval proti Izraelovi.
\par 27 Izraelští také secteni jsou, a opatrivše se stravou, vytáhli jim v cestu. I položili se Izraelští proti nim jako dve malá stáda koz, Syrští pak prikryli tu zemi.
\par 28 V tom prišel týž muž Boží, a mluvil králi Izraelskému a rekl: Takto praví Hospodin: Proto že pravili Syrští: Bohem hor jest Hospodin a ne Bohem rovin, dám všecko množství toto veliké v ruce tvé, abyste vedeli, že já jsem Hospodin.
\par 29 A tak leželi oni proti nim za sedm dní; i stalo se dne sedmého, že svedli bitvu. I porazili synové Izraelští z Syrských sto tisíc peších dne jednoho.
\par 30 Jiní pak zutíkali do mesta Afeku, kdež padla zed na dvadceti a sedm tisíc mužu, kteríž byli pozustali. Ale Benadad utíkaje, prišel do mesta, do nejtajnejšího pokoje.
\par 31 I rekli jemu služebníci jeho: Aj, toto jsme slýchali, že králové domu Izraelského jsou králové milosrdní. Medle, necht vezmeme pytle na bedra svá a provazy na hlavy své, a vejdeme k králi Izraelskému; snad pri životu zachová tebe.
\par 32 A protož prepásali pytli bedra svá, a provazy vzali na hlavy své, a prišli k králi Izraelskému a rekli: Služebník tvuj Benadad praví: Prosím, necht jsem zachován pri životu. Kterýž rekl: Což jest ješte živ? Bratr te muj.
\par 33 Muži pak ti soudíc to za dobré znamení, rychle chytili ta slova od neho a rekli: Bratrt jest tvuj Benadad. I rekl: Jdete, privedte ho sem. Takž vyšel k nemu Benadad, i kázal mu vstoupiti na svuj vuz.
\par 34 Tedy rekl jemu Benadad: Mesta, kteráž otec muj pobral otci tvému, navrátímt, a ulice zdeláš sobe v Damašku, jako byl udelal otec muj v Samarí. I odpovedel: Já vedlé smlouvy této propustím te. A tak ucinil s ním smlouvu a propustil ho.
\par 35 Muž pak nejaký z synu prorockých rekl bližnímu svému z rozkazu Božího: Ubí mne medle. Ale muž ten nechtel ho bíti.
\par 36 Kterýž rekl jemu: Proto že jsi neuposlechl hlasu Hospodinova, aj ted, když pujdeš ode mne, udáví te lev. A když šel od neho, trefil na nej lev a udávil ho.
\par 37 Opet nalezl muže druhého, jemuž rekl: Medle, ubí mne. Kterýž ubil ho velice a ranil.
\par 38 I odšed prorok ten, postavil se králi v ceste, a zmenil se, zavesiv sobe oci.
\par 39 A když král pomíjel, zvolal na nej a rekl: Služebník tvuj vyšel do bitvy, a hle, jeden odšed, privedl ke mne muže a rekl: Ostríhej toho muže, jestliže se pak ztratí, budet život tvuj za život jeho, aneb centnér stríbra položíš.
\par 40 V tom když služebník tvuj toho i onoho hledel, tožt se ztratil. Procež rekl jemu král Izraelský: Takový bud soud tvuj, jakýž jsi sám vypovedel.
\par 41 Tedy rychle odhradil tvár svou, a poznal ho král Izraelský, že by z proroku byl.
\par 42 I rekl jemu: Takto praví Hospodin: Ponevadž jsi z ruky pustil muže k smrti odsouzeného, budet život tvuj za život jeho, a lid tvuj za lid jeho.
\par 43 Protož odjel král Izraelský do domu svého, smutný jsa a hnevaje se, a prišel do Samarí.

\chapter{21}

\par 1 Stalo se pak po tech vecech, mel Nábot Jezreelský vinici, kteráž byla v Jezreel podlé paláce Achaba krále Samarského.
\par 2 I mluvil Achab k Nábotovi, rka: Dej mi vinici svou, at ji mám místo zahrady k zelinám, ponevadž jest blízko podlé domu mého, a dámt za ni vinici lepší, než ta jest, aneb jestližet se vidí, dámt stríbra cenu její.
\par 3 Odpovedel Nábot Achabovi: Nedejž mi toho Hospodin, abych mel dáti dedictví otcu mých tobe.
\par 4 Tedy prišel Achab do domu svého, smutný jsa a hnevaje se pro rec, kterouž mluvil jemu Nábot Jezreelský, rka: Nedám tobe dedictví otcu svých. I lehl na lužko své, a odvrátiv tvár svou, ani chleba nejedl.
\par 5 V tom prišedši k nemu Jezábel žena jeho, rekla jemu: Proc tak smutný jest duch tvuj, že ani nejíš chleba?
\par 6 Kterýž odpovedel jí: Proto že jsem mluvil s Nábotem Jezreelským, a rekl jsem jemu: Dej mi vinici svou za peníze, aneb jestliže radeji chceš, dámt jinou vinici za ni. On pak odpovedel: Nedám tobe své vinice.
\par 7 I rekla mu Jezábel žena jeho: Takliž bys ty nyní spravoval království Izraelské? Vstan, pojez chleba a bud dobré mysli, já tobe dám vinici Nábota Jezreelského.
\par 8 Zatím napsala list jménem Achabovým, kterýž zapecetila jeho pecetí, a poslala ten list k starším a predním mesta jeho, spoluobyvatelum Nábotovým.
\par 9 Napsala pak v tom listu takto: Vyhlaste pust, a posadte Nábota mezi predními z lidu,
\par 10 A postavte dva muže nešlechetné proti nemu, kteríž by svedcili na nej, rkouce: Zlorecil jsi Bohu a králi. Potom vyvedte ho a ukamenujte jej, at umre.
\par 11 Ucinili tedy muži mesta toho, starší a prední, kteríž bydlili v meste jeho, jakž rozkázala jim Jezábel, tak jakž psáno bylo v listu, kterýž jim byla poslala.
\par 12 Vyhlásili pust, a posadili Nábota mezi predními z lidu.
\par 13 Potom prišli ti dva muži nešlechetní, a posadivše se naproti nemu, svedcili proti nemu muži ti nešlechetní, proti Nábotovi pred lidem, rkouce: Zlorecil Nábot Bohu a králi. I vyvedli ho za mesto a kamenovali jej, až umrel.
\par 14 A poslali k Jezábel, rkouce: Ukamenovánt jest Nábot a umrel.
\par 15 I stalo se, jakž uslyšela Jezábel, že by ukamenován byl Nábot, a že by umrel, rekla Achabovi: Vstan, vládni vinicí Nábota Jezreelského, kteréžt nechtel dáti za peníze; nebot není živ Nábot, ale umrel.
\par 16 A tak uslyšev Achab, že by umrel Nábot, vstal, aby šel do vinice Nábota Jezreelského, a aby ji ujal.
\par 17 Tedy stala se rec Hospodinova k Eliášovi Tesbitskému, rkoucí:
\par 18 Vstana, vyjdi vstríc Achabovi králi Izraelskému, kterýž bydlí v Samarí, a hle, jest na vinici Nábotove, do níž všel, aby ji ujal.
\par 19 A mluviti budeš k nemu temito slovy: Takto praví Hospodin: Zdaliž jsi nezabil, ano sobe i neprivlastnil? Mluv tedy k nemu, rka: Takto praví Hospodin: Proto že lízali psi krev Nábotovu, i tvou krev také lízati budou.
\par 20 I rekl Achab Eliášovi: Což jsi mne našel, nepríteli muj? Kterýž odpovedel: Našel, nebo jsi prodal se, abys cinil to, což zlého jest pred oblícejem Hospodinovým.
\par 21 Aj, já uvedu na tebe zlé, a odejmu potomky tvé, a vypléním z domu Achabova mocícího na stenu, i zajatého i zanechaného v Izraeli.
\par 22 A uciním s domem tvým jako s domem Jeroboáma syna Nebatova, a jako s domem Bázy syna Achiášova, pro zdráždení, kterýmž jsi mne popudil, a že jsi k hrešení privodil Izraele.
\par 23 Ano i proti Jezábel mluvil Hospodin, rka: Psi žráti budou Jezábel mezi zdmi Jezreelskými.
\par 24 Toho, kdož umre z domu Achabova v meste, psi žráti budou, a kdož umre na poli, ptáci nebeští jísti budou.
\par 25 Nebo nebylo podobného Achabovi, kterýž by se prodal, aby cinil to, což zlého jest pred oblícejem Hospodinovým, proto že ho ponoukala Jezábel žena jeho.
\par 26 Dopouštel se zajisté vecí velmi ohavných, následuje modl vedlé všeho toho, cehož se dopoušteli Amorejští, kteréž vyplénil Hospodin od tvári synu Izraelských.
\par 27 I stalo se, když uslyšel Achab slova tato, že roztrhl roucho své, a vzav žíni na telo své, postil se a léhal na pytli, a chodil krotce.
\par 28 Tedy stala se rec Hospodinova k Eliášovi Tesbitskému, rkoucí:
\par 29 Videl-lis, jak se ponížil Achab pred tvárí mou? Ponevadž se tak ponížil pred tvárí mou, neuvedu toho zlého za dnu jeho, ale za dnu syna jeho uvedu to zlé na dum jeho.

\chapter{22}

\par 1 Nebylo pak za tri léta války mezi Syrskými a Izraelskými.
\par 2 I stalo se léta tretího, prijel Jozafat král Judský k králi Izraelskému.
\par 3 Mluvil pak byl král Izraelský služebníkum svým: Víte-li, že naše bylo Rámot Galád, a my zanedbáváme vzíti ho zase z moci krále Syrského?
\par 4 Procež rekl Jozafatovi: Potáhneš-li se mnou na vojnu proti Rámot Galád? Odpovedel Jozafat králi Izraelskému: Jako jsem já, tak jsi ty, jako lid muj, tak lid tvuj, jako koni moji, tak koni tvoji.
\par 5 Rekl také Jozafat králi Izraelskému: Vzeptej se dnes medle na slovo Hospodinovo.
\par 6 I shromáždil král Izraelský proroku okolo ctyr set mužu a rekl jim: Mám-li táhnouti na vojnu proti Rámot Galád, ci tak nechati? I rekli: Táhni, nebo dá je Pán v ruku krále.
\par 7 Tedy rekl Jozafat: Což není zde již žádného proroka Hospodinova, abychom se ho zeptali?
\par 8 Na to rekl král Izraelský Jozafatovi: Ještet jest muž jeden, skrze nehož bychom se mohli poraditi s Hospodinem, ale já ho nenávidím, proto že mi dobrého neprorokuje, než zlé, Micheáš syn Jemluv. Ale Jozafat rekl: Necht tak nemluví král.
\par 9 Protož povolav král Izraelský komorníka jednoho, rekl: Prived sem rychle Micheáše syna Jemlova.
\par 10 (Mezi tím král Izraelský a Jozafat král Judský, jeden každý na stolici své, odení jsouce rouchem, sedeli v placu u vrat brány Samarské, a všickni proroci prorokovali pred nimi.
\par 11 Sedechiáš pak syn Kenanuv udelal sobe byl rohy železné a rekl: Takto praví Hospodin: Temito trkati budeš Syrské, dokudž nepohubíš jich.
\par 12 Takž podobne i jiní všickni proroci prorokovali, rkouce: Jed do Rámot Galád, a štastnet se povede; nebo dá je Hospodin v ruku královu.)
\par 13 V tom posel ten, kterýž šel, aby zavolal Micheáše, mluvil jemu, rka: Aj, nyní slova proroku jednemi ústy predpovídají dobré veci králi. Medle, bud rec tvá jako rec kterého z nich, a mluv dobré veci.
\par 14 Tedy rekl Micheáš: Živt jest Hospodin, že což mi koli rekne Hospodin, to mluviti budu.
\par 15 A když prišel k králi, rekl jemu král: Micheáši, máme-li jeti na vojnu proti Rámot Galád, ci tak nechati? Kterýž rekl jemu: Jed a štastnet se povede, nebo dá je Hospodin v ruku krále.
\par 16 I rekl jemu král: I kolikrátž te mám prísahou zavazovati, abys mi nemluvil než pravdu ve jménu Hospodinovu?
\par 17 Protož rekl: Videl jsem všecken lid Izraelský rozptýlený po horách jako ovce, kteréž nemají pastýre; nebo rekl Hospodin: Nemají pána tito, navrat se jeden každý do domu svého v pokoji.
\par 18 I rekl král Izraelský Jozafatovi: Zdaližt jsem nerekl, že mi nebude prorokovati dobrého, ale zlé?
\par 19 Rekl dále: Z té príciny slyš slovo Hospodinovo: Videl jsem Hospodina sedícího na stolici své, a všecko vojsko nebeské stojící po pravici jeho i po levici jeho.
\par 20 I rekl Hospodin: Kdo oklamá Achaba, aby vytáhl a padl u Rámot Galád? A když pravil ten toto, a jiný pravil jiné,
\par 21 Tožt vyšel jakýsi duch, a postaviv se pred Hospodinem, rekl: Já ho oklamám. Hospodin pak rekl jemu: Jakým zpusobem?
\par 22 Odpovedel: Vyjdu a budu duchem lživým v ústech všech proroku jeho. Kterýžto rekl: Oklamáš a dovedeš toho; vyjdiž a ucin tak.
\par 23 Protož, aj, jižte dal Hospodin ducha lživého v ústa všech proroku tvých techto, ješto však Hospodin mluvil zlé proti tobe.
\par 24 Tedy pristoupiv Sedechiáš syn Kenanuv, dal Micheášovi policek a rekl: Kudyže odšel duch Hospodinuv ode mne, aby mluvil tobe?
\par 25 Odpovedel Micheáš: Aj, uzríš v ten den, když vejdeš do nejtajnejšího pokoje, abys se skryl.
\par 26 I rekl král Izraelský: Jmi Micheáše, a doved ho k Amonovi knížeti mesta, a k Joasovi synu královu.
\par 27 A rekneš: Takto praví král: Dejte tohoto do žaláre, a dávejte mu jísti malicko chleba a malicko vody, dokudž se nenavrátím v pokoji.
\par 28 Ale Micheáš rekl: Jestliže ty se navrátíš v pokoji, tedyt nemluvil skrze mne Hospodin. Pres to rekl: Slyštež to všickni lidé!
\par 29 A tak táhl král Izraelský, a Jozafat král Judský proti Rámot Galád.
\par 30 I rekl král Izraelský Jozafatovi: Zmením já se, když pujdu k bitve, ale ty oblec se v roucho své. I zmenil se král Izraelský a jel k bitve.
\par 31 Král pak Syrský prikázal tridcíti dvema svým hejtmanum nad vozy, a rekl: Nebojujte proti malému ani proti velikému, než proti samému králi Izraelskému.
\par 32 I stalo se, když uzreli hejtmané nad vozy Jozafata, rekli: Jiste král Izraelský jest. I obrátili se proti nemu, aby bojovali. Tedy zkrikl Jozafat.
\par 33 V tom když uzreli hejtmané nad vozy, že on není král Izraelský, obrátili se od neho.
\par 34 Muž pak jeden strelil z lucište náhodou, a postrelil krále Izraelského, kdež se pancír spojuje. Procež rekl vozkovi svému: Obrat se a vyvez mne z vojska, nebo jsem nemocen.
\par 35 I rozmohla se bitva v ten den. Král pak stál na voze proti Syrským; potom umrel u vecer, a tekla krev z rány do vozu.
\par 36 I volal biric po vojšte, když již slunce zapadlo, rka: Jeden každý do mesta svého, a jeden každý do zeme své navrat se.
\par 37 Umrel tedy král a dovezen jest do Samarí, i pochovali ho v Samarí.
\par 38 A když pohrížen byl vuz v rybníku Samarském, strebali psi krev jeho, též když umývali zbroj jeho, vedlé reci Hospodinovy, kterouž byl mluvil.
\par 39 Jiné pak veci Achabovy, a cožkoli cinil, i jaký dum z kostí slonových vystavel, i všecka mesta, kteráž vzdelal, o tom zapsáno jest v knize o králích Izraelských.
\par 40 A tak usnul Achab s otci svými, a kraloval Ochoziáš syn jeho místo neho.
\par 41 Jozafat pak syn Azy pocal kralovati nad Judou léta ctvrtého Achaba, krále Izraelského.
\par 42 A byl Jozafat ve tridcíti peti letech, když pocal kralovati, a dvadceti pet let kraloval nad Jeruzalémem, jehož matky jméno bylo Azuba, dcera Silchi.
\par 43 I chodil po vší ceste Azy otce svého, aniž se od ní uchýlil, cine, což dobrého bylo pred oblícejem Hospodinovým.
\par 44 Toliko ponevadž výsostí nezkazili, ješte lid obetoval a kadil na tech výsostech.
\par 45 Všel také v pokoj král Jozafat s králem Izraelským.
\par 46 Jiné pak veci Jozafatovy, i síla jeho, kterouž prokazoval a kterouž bojoval, zapsány jsou v knize o králích Judských.
\par 47 Ten ostatky ohyzdných sodomáru, kteríž ješte pozustali za dnu Azy otce jeho, vyplénil z zeme.
\par 48 Tehdáž nebylo žádného krále v zemi Idumejské, hejtmana meli místo krále.
\par 49 I nadelal Jozafat lodí morských, aby jeli do Ofir pro zlato. Ale nejeli, nebo polámaly se lodí v Aziongaber.
\par 50 Rekl byl také Ochoziáš syn Achabuv Jozafatovi: Necht jedou služebníci moji s služebníky tvými na lodech. Ale Jozafat nechtel.
\par 51 I usnul Jozafat s otci svými, a pochován jest s otci svými v meste Davida otce svého; kraloval pak Joram syn jeho místo neho.
\par 52 Ochoziáš syn Achabuv pocal kralovati nad Izraelem v Samarí, léta sedmnáctého Jozafata krále Judského, a kraloval nad Izraelem dve léte.
\par 53 Nebo cinil zlé veci pred oblícejem Hospodinovým, a chodil po ceste otce svého a po ceste matky své, i po ceste Jeroboáma syna Nebatova, kterýž privodil k hrešení lid Izraelský.
\par 54 Sloužil také Bálovi a klanel se jemu, címž popudil k hnevu Hospodina Boha Izraelského vedlé toho všeho, což cinil otec jeho.

\end{document}