\begin{document}

\title{2 Královská}

\chapter{1}

\par 1 V tom zprotivil se Moáb Izraelovi po smrti Achabove.
\par 2 A Ochoziáš spadl skrze mríži paláce svého letního, kterýž byl v Samarí, a stonal. I poslal posly, rka jim: Jdete, pilne se ptejte Belzebuba, boha Akaron, povstanu-li z nemoci této?
\par 3 Ale andel Hospodinuv rekl Eliášovi Tesbitskému: Vstana, jdi vstríc poslum krále Samarského a mluv k nim: Zdali není Boha v Izraeli, že jdete dotazovati se Belzebuba, boha Akaron?
\par 4 Protož takto praví Hospodin: S ložce, na kterémžs se složil, nesejdeš, ale jistotne umreš. I odšel Eliáš.
\par 5 A když se navrátili poslové k nemu, rekl jim: Procež jste se zase vrátili?
\par 6 Odpovedeli jemu: Muž nejaký vyšel nám v cestu a rekl nám: Jdete, navratte se k králi, kterýž vás poslal, a rcete jemu: Takto praví Hospodin: Zdali není Boha v Izraeli, že ty posíláš tázati se Belzebuba, boha Akaron? Protož s ložce, na kterémžs se složil, nesejdeš, ale jistotne umreš.
\par 7 I rekl jim: Jaký jest zpusob muže toho, kterýž vám vstríc vyšel a mluvil vám ty reci?
\par 8 Kteríž odpovedeli jemu: Jest clovek chlupatý, a pasem koženým prepásán na bedrách svých. Tedy rekl: Eliáš Tesbitský jest.
\par 9 Protož poslal pro nej padesátníka s padesáti jeho. Kterýž šel k nemu, a aj, sedel na vrchu hory. I rekl jemu: Muži Boží, král rozkázal, abys sstoupil dolu.
\par 10 Odpovídaje Eliáš, rekl padesátníku: Jestliže jsem muž Boží, necht sstoupí ohen s nebe, a sžíre te i tvých padesát. A tak sstoupil ohen s nebe, a sežral jej i padesát jeho.
\par 11 Opet poslal k nemu král padesátníka jiného s padesáti jeho. Kterýž mluvil a rekl jemu: Muži Boží, takto rozkázal král: Rychle sstup dolu.
\par 12 I odpovedel Eliáš a rekl jim: Jestliže jsem muž Boží, necht sstoupí ohen s nebe, a sžíre te i tvých padesát. A tak sstoupil ohen Boží s nebe, a sežral jej i padesát jeho.
\par 13 Ješte poslal padesátníka tretího s padesáti jeho. Procež vstoupiv, prišel ten tretí padesátník, a klekl na kolena svá pred Eliášem, a prose ho pokorne, rekl jemu: Muži Boží,prosím, necht jest drahá duše má i duše služebníku tvých padesáti techto pred ocima tvýma.
\par 14 Aj, ohen sstoupiv s nebe, sežral dva první padesátníky s padesáti jejich, ale nyní necht jest drahá duše má pred ocima tvýma.
\par 15 Rekl pak andel Hospodinuv Eliášovi: Sstup s ním, neboj se nic tvári jeho. Tedy vstav, šel s ním k králi.
\par 16 A mluvil jemu: Takto praví Hospodin: Proto že jsi poslal posly tázati se Belzebuba, boha Akaron, (jako by nebylo Boha v Izraeli, abys se tázal slova jeho), protož s ložce, na kterémžs se složil, nesejdeš, ale jistotne umreš.
\par 17 I umrel podlé reci Hospodinovy, kterouž mluvil Eliáš, a kraloval Joram místo neho léta druhého Jehorama syna Jozafatova, krále Judského; nebo on nemel syna.
\par 18 O jiných pak vecech Ochoziášových, kteréž cinil, zapsáno jest v knize o králích Izraelských.

\chapter{2}

\par 1 Stalo se potom, když mel již vzíti Hospodin Eliáše u vichru do nebe, že vyšel Eliáš s Elizeem z Galgala.
\par 2 I rekl Eliáš Elizeovi: Medle, posed tuto, nebo Hospodin poslal mne až do Bethel. Jemuž rekl Elizeus: Živt jest Hospodin, a živat jest duše tvá, žet se tebe nespustím. I prišli do Bethel.
\par 3 Tedy vyšli synové proroctí, kteríž byli v Bethel, k Elizeovi a rekli jemu: Víš-liž, že dnes Hospodin vezme pána tvého od tebe? Kterýž odpovedel: A já vím, mlcte.
\par 4 Opet rekl jemu Eliáš: Elizee, medle posed tuto, nebo Hospodin poslal mne do Jericha. Kterýž odpovedel: Živt jest Hospodin, a živat jest duše tvá, žet se tebe nespustím. I prišli do Jericha.
\par 5 Pristoupivše pak synové proroctí, kteríž byli v Jerichu, k Elizeovi, rekli jemu: Víš-liž, že dnes vezme Hospodin pána tvého od tebe? I rekl: A já vím, mlcte.
\par 6 Rekl mu ješte Eliáš: Posed medle tuto, nebo Hospodin poslal mne k Jordánu. Kterýž odpovedel: Živt jest Hospodin, a živat jest duše tvá, žet se tebe nespustím. Šli tedy oba.
\par 7 Padesáte pak mužu z synu prorockých šli, a postavili se naproti zdaleka, ale oni oba zastavili se u Jordánu.
\par 8 A vzav Eliáš plášt svuj, svinul jej a uderil na vodu. I rozdelila se sem i tam, tak že prešli oba po suše.
\par 9 V tom když prešli, rekl Eliáš Elizeovi: Žádej sobe, co chceš, prvé než vzat budu od tebe. I rekl Elizeus: Necht jest, prosím, dvojnásobní díl ducha tvého na mne.
\par 10 Jemuž rekl: Nesnadnés veci požádal, a však jestliže uzríš mne, když budu vzat od tebe, stanet se tak; pakli nic, nestane se.
\par 11 Takž stalo se, že když predce jdouce, rozmlouvali, aj, vuz ohnivý a koni ohniví rozdelili je na ruzno. I vstoupil Eliáš u vichru do nebe.
\par 12 To vida Elizeus, volal: Otce muj, otce muj! Vozové Izraelští i jezdci jeho! A nevidel ho více. Potom uchopiv roucho své, roztrhl je na dva kusy.
\par 13 A zdvih plášt Eliášuv, kterýž byl spadl s neho, navrátil se a stál na brehu Jordánském.
\par 14 Tedy vzav plášt Eliášuv, kterýž byl spadl s neho, uderil na vodu a rekl: Kdež jest Hospodin Buh Eliášuv i on sám? A uderil, pravím, na vodu, kteráž rozstoupila se sem i tam. I prešel Elizeus.
\par 15 Což vidouce synové proroctí z Jericha, naproti stojíce, rekli: Odpocinul duch Eliášuv na Elizeovi. I šli proti nemu a poklonili se mu až k zemi.
\par 16 A rekli jemu: Aj, nyní jest s služebníky tvými padesáte mužu silných, medle necht jdou a hledají pána tvého. Snad ho zanesl duch Hospodinuv, a povrhl jej na nekteré hore aneb v nekterém údolí. Kterýž rekl: Neposílejte.
\par 17 Ale když na nej vždy dotírali, tak že mu to obtížné bylo, rekl: Pošlete. I poslali padesáte mužu, a hledajíce za tri dni, nenalezli ho.
\par 18 A když se k nemu navrátili, (on pak bydlil v Jerichu), rekl jim: Zdaliž jsem vám nerekl: Nechodte?
\par 19 Muži pak mesta toho rekli Elizeovi: Hle, nyní byt v meste tomto jest výborný, jakož, pane muj, vidíš, ale vody jsou zlé a zeme neúrodná.
\par 20 Tedy rekl: Prineste mi nádobu novou, a dejte do ní soli. I prinesli mu.
\par 21 A vyšed k pramenu tech vod, vsypal tam sul a rekl: Takto praví Hospodin: Uzdravuji vody tyto; nebudet více odtud smrti, ani nedochudcete.
\par 22 A tak uzdraveny jsou ty vody až do dnešního dne, vedlé reci Elizeovy, kterouž byl mluvil.
\par 23 Potom šel odtud do Bethel. A když šel cestou, pacholata malá vyšedše z mesta, posmívali se jemu, ríkajíce: Jdi lysý, jdi lysý!
\par 24 Kterýž ohlédna se, uzrel je a zlorecil jim ve jménu Hospodinovu. Protož vyskocivše dve nedvedice z lesa, roztrhaly z nich ctyridcatero a dvé detí.
\par 25 I šel odtud na horu Karmel, odkudž navrátil se do Samarí.

\chapter{3}

\par 1 Joram pak syn Achabuv pocal kralovati nad Izraelem v Samarí léta osmnáctého Jozafata krále Judského, a kraloval dvanácte let.
\par 2 A cinil to, což jest zlého pred ocima Hospodinovýma, ac ne tak jako otec jeho a jako matka jeho; nebo odjal modly Bál, kterýchž byl nadelal otec jeho.
\par 3 A však v hríších Jeroboáma syna Nebatova, kterýž k hrešení privodil Izraele, vždy vezel, a neodstoupil od nich.
\par 4 Mésa pak král Moábský mel hojnost dobytka, a dával králi Izraelskému sto tisíc beranu, a sto tisíc skopcu i s vlnou.
\par 5 I stalo se, když umrel Achab, že se zprotivil král Moábský králi Izraelskému.
\par 6 Tedy vytáhl v ten cas král Joram z Samarí, a sectl všecken Izrael.
\par 7 A když táhl, poslal k Jozafatovi králi Judskému, aby mu rekli: Král Moábský zprotivil mi se. Potáhneš-li se mnou proti Moábovi na vojnu? Odpovedel: Potáhnu. Jako jsem já, tak jsi ty, jako lid muj, tak lid tvuj, jakž koni moji, tak koni tvoji.
\par 8 Zatím rekl: Kterouž pak cestou potáhneme? Odpovedel: Cestou poušte Idumejské.
\par 9 A tak vytáhl král Izraelský a král Judský i král Idumejský. A když objíždeli cestou za sedm dní, nedostávalo se vody vojsku a hovadum jejich, kteráž meli s sebou.
\par 10 I rekl král Izraelský: Ach, beda! Nebo povolal Hospodin trí králu techto, aby je vydal v ruku Moábovu.
\par 11 Ale Jozafat rekl: Není-liž zde proroka Hospodinova, abychom se otázali Hospodina skrze neho? Odpovídaje pak jeden z služebníku krále Izraelského, rekl: Jestit zde Elizeus syn Safatuv, kterýž líval vodu na ruce Eliášovy.
\par 12 Tedy rekl Jozafat: U tohot jest slovo Hospodinovo. I šli k nemu, král Izraelský a Jozafat, i král Idumejský.
\par 13 I rekl Elizeus králi Izraelskému: Co mne do tebe? Jdi k prorokum otce svého a k prorokum matky své. Rekl jemu král Izraelský: Nikoli, nebo povolal Hospodin trí králu techto, aby je vydal v ruku Moábovu.
\par 14 K tomu rekl Elizeus: Živt jest Hospodin zástupu, pred jehož oblícejem stojím, bycht sobe nevážil Jozafata krále Judského, nepohledel bych na te, ani popatril.
\par 15 Ale nyní privedte mi toho, kterýž by umel hráti na harfu. A když on hral, byla nad ním ruka Hospodinova.
\par 16 I rekl: Takto praví Hospodin: Nadelej v tomto potoku množství dolu.
\par 17 Nebo toto dí Hospodin: Neuzríte vetru, aniž uzríte prívalu, však potok tento naplnen bude vodou, tak že píti budete i vy i množství vaše, i hovada vaše.
\par 18 A i to málo jest pred oblícejem Hospodinovým, nebo i Moábské dá v ruku vaši.
\par 19 A zkazíte všeliké mesto hrazené, i všeliké mesto výborné, též všecko stromoví dobré zporážíte, a všecky studnice vod zasypete, a všeliké pole dobré kamením priházíte.
\par 20 I stalo se ráno, když obetována bývá obet suchá, a aj, vody pricházely cestou od strany Idumejské, a naplnena jest zeme vodami.
\par 21 Všecken pak Moáb uslyšev, že by vytáhli králové, aby bojovali proti nim, svolali se všickni, od toho, kterýž se pasem opásati muže, a výše, a postavili se na pomezí.
\par 22 Potom ráno vstavše, když slunce vzešlo nad temi vodami, uzreli Moábští naproti ty vody rdející se jako krev.
\par 23 A rekli: Krev jest. Jiste žet jsou se pohubili ti králové, a zabil jeden každý bližního svého; protož nyní k loupežem, ó Moábští! A prišli až k ležení Izraelskému.
\par 24 Tedy povstavše Izraelští, porazili Moábské, kteríž utíkali pred nimi, a oni porazili je porážkou velikou, také i v jejich krajine.
\par 25 Nebo mesta jejich zborili, a na všeliké pole výborné házejíce jeden každý kamením svým, naplnili je, i všecky studnice vod zasypali, a všecko stromoví dobré zporáželi, tak že toliko nechali u Kirchareset zdi jeho. Protož shlukše se prakovníci, dobývali ho.
\par 26 A vida král Moábský, že jsou mu silní bojovníci ti, vzal s sebou sedm set mužu bojovných, chte se probiti skrze vojska krále Idumejského. Ale nemohli.
\par 27 Procež jav syna jeho prvorozeného, kterýž mel kralovati místo neho, obetoval jej v obet zápalnou na zdi. I stalo se rozhnevání veliké proti Izraelovi; protož odtrhše od neho, navrátili se do zeme své.

\chapter{4}

\par 1 Tedy žena jedna z manželek synu prorockých volala k Elizeovi, rkuci: Služebník tvuj, muž muj, umrel, ty pak víš, že služebník tvuj bál se Hospodina. A ted veritel prišel, aby vzal dva syny mé sobe za služebníky.
\par 2 Rekl jí Elizeus: Což chceš, at uciním? Oznam mi, co máš v dome? Ona rekla: Nemát služebnice tvá nic v dome, jediné báni oleje.
\par 3 I rekl: Jdi, vyžádej sobe nádob vne ode všech sousedu svých, nádob prázdných tím více.
\par 4 A vejda, zavri za sebou dvére i za syny svými, a nalévej do všech tech nádob, a kteráž bude plná, rozkážeš ji odstaviti.
\par 5 A tak odešla od neho a zavrela dvére za sebou a za syny svými. Oni podávali jí, a ona nalévala.
\par 6 I stalo se, že když naplnila ty nádoby, rekla synu svému: Podej mi ješte nádoby. Kterýž odpovedel: Již není více nádob. I prestal olej.
\par 7 Tedy ona prišedši, oznámila to muži Božímu. On pak rekl: Jdi, prodej ten olej, a spokoj veritele svého, ty pak s syny svými budeš se živiti z ostatku.
\par 8 Potom stalo se nekterého casu, že šel Elizeus skrze Sunem, kdež byla jedna žena vzácná, kteráž ho pozdržela, aby jedl u ní. A od toho casu chodívaje tudy, stavoval se tam, a jídal chléb.
\par 9 Nebo rekla byla muži svému: Aj, nyní vím, že ten muž Boží svatý jest, kterýž casto tudyto chodívá.
\par 10 Medle, udelejme pokojík malý, a postavme tam jemu ložce, stul, stolici a svícen, aby, když by koli k nám prišel, obrátil se tam.
\par 11 Nekterého tedy casu prišel tam, a všed do toho pokojíku, odpocinul tu.
\par 12 I rekl Gézi služebníku svému: Zavolej té Sunamitské. Kterýž zavolal jí. I postavila se pred ním.
\par 13 Pritom rekl jemu: Rci jí nyní: Hle, pecuješ a staráš se o nás všelijak. Co chceš, at uciním? Máš-li jakou potrebu pred králem aneb pred hejtmanem vojska? Kteráž rekla: U prostred lidu svého bydlím.
\par 14 Rekl on: Což tedy mám pro ni uciniti? Odpovedel Gézi: Hle, syna nemá, a muž její starý jest.
\par 15 Protož rekl: Zavolej jí. Takž jí zavolal, a ona stála u dverí.
\par 16 I rekl: V casu jistém vedlé casu života chovati budeš syna. Rekla ona: Nechtej, pane muj, muži Boží, nechtej se posmívati služebnici své.
\par 17 Zatím pocala žena a porodila syna v casu jistém vedlé casu života, kterýž predpovedel jí Elizeus.
\par 18 I rostlo díte. Stalo se pak nekterého casu, že vyšedši k otci svému k žencum,
\par 19 Reklo otci svému: Hlava má, hlava má! I rekl služebníku: Dones jej k matce jeho.
\par 20 Kterýž když ho vzal a prinesl k matce jeho, sedel na klíne jejím až do poledne a umrel.
\par 21 Tedy ona vstoupivši, položila jej na ložce muže Božího, a zavrevši ho, vyšla.
\par 22 Potom zavolala muže svého a rekla: Medle, pošli mi jednoho z služebníku a jednu oslici, at dobehnu až k muži Božímu, a zase se navrátím.
\par 23 Kterýž rekl: Proc chceš jíti k nemu? Dnes není novmesíce, ani sobota. Odpovedela ona: Mej o to pokoj.
\par 24 Osedlavši tedy oslici, rekla služebníku svému: Pobádej a bež, a nemeškej se pro mne v jízde, lec bych rozkázala tobe.
\par 25 A tak jela, až prijela k muži Božímu na horu Karmel. A když ji uzrel muž Boží zdaleka, rekl Gézi služebníku svému: Hle, Sunamitská tamto.
\par 26 Protož nyní bež jí v cestu a rci jí: Dobre-li se máš? Dobre-li se má muž tvuj? Dobre-li se má syn? Ona rekla: Dobre.
\par 27 V tom prišla k muži Božímu na horu, a chopila se noh jeho. I pristoupil Gézi, aby ji odehnal. Ale muž Boží rekl: Nechej jí; nebo v horkosti jest duše její, a Hospodin zatajil to prede mnou, aniž mi oznámil.
\par 28 Ona pak rekla: Zdaliž jsem žádala syna od pána mého? Zdaliž jsem nerekla, abys mne nešálil?
\par 29 Tedy rekl k Gézi: Prepaš bedra svá, a vezmi hul mou v ruku svou a jdi. Jestliže potkáš koho, nepozdravuj ho, a jestliže te kdo pozdraví, neodpovídej jemu, a prijda, polož hul mou na tvár dítete.
\par 30 I rekla matka dítete: Živt jest Hospodin a živat jest duše tvá, žet se tebe nespustím. Protož vstav, bral se za ní.
\par 31 Gézi pak byl predšel je, a položil hul na tvár dítete, ale nebylo hlasu, ani citelnosti. Procež vracuje se jemu vstríc, povedel mu, rka: Neprobudilo se díte.
\par 32 Všel tedy Elizeus do domu, a aj, díte mrtvé leželo na ložci jeho.
\par 33 A když všel tam, zavrel dvére pred obema, a modlil se Hospodinu.
\par 34 Zatím vstoupiv na lože, zpolehl na díte, vloživ ústa svá na ústa jeho, a oci své na oci jeho, a ruce své na ruce jeho, a rozprostrel se nad ním. I zahrelo se telo dítete.
\par 35 A odvrátiv se, procházel se po dome jednak sem jednak tam; potom vstoupiv, rozprostrel se opet nad ním. I kýchalo díte až do sedmikrát, a otevrelo to díte oci své.
\par 36 Tedy zavolav Gézi, rekl: Zavolej Sunamitské. I zavolal jí. A když prišla k nemu, rekl jí: Vezmiž syna svého.
\par 37 Kteráž jakž vešla, padla k nohám jeho a poklonila se k zemi. I vzala syna svého a vyšla.
\par 38 Potom Elizeus navrátil se do Galgala. Byl pak hlad v té zemi, a synové proroctí sedeli pred ním. I rekl mládenci svému: Pristav hrnec veliký, a navar kaše synum prorockým.
\par 39 Protož vyšel jeden na pole, aby nasbíral zelin, a nalezl rév polní, a nasbíral z neho tykví polních plnou sukni svou, a prišed, skrájel to do hrnce, aby navaril kaše; nebo neznali toho.
\par 40 I vylili mužum tem, aby jedli. Stalo se pak, když jedli tu kaši, že zkrikli a rekli: Smrt v hrnci, muži Boží! A nemohli jísti.
\par 41 Kterýžto rekl: Vezmete mouky. Kteréž nasypav do hrnce, rozkázal vyliti lidu. I jedli, aniž co zlého bylo v hrnci.
\par 42 V tom muž prišel z Balsalisa, kterýž prinesl muži Božímu chleby z prvotin, totiž dvadceti chlebu jecných, a klasu nových nevymnutých. I rekl: Dej lidu, at jedí.
\par 43 Odpovedel služebník jeho: Což to mám predložiti sto mužum? Opet rekl: Dej lidu, at pojedí. Nebo tak praví Hospodin: Jísti budou, a ješte zustane.
\par 44 A tak predložil jim, i jedli, a zustalo ješte vedlé reci Hospodinovy.

\chapter{5}

\par 1 Náman hejtman vojska krále Syrského, jsa muž veliký u pána svého, a osoba vzácná, skrze neho zajisté dal Hospodin vysvobození zemi Syrské, ten, pravím, muž tak udatný byl malomocný.
\par 2 Byli pak vyšli z zeme Syrské lotríkové, kteríž zajali z zeme Izraelské devecku malou, a ta sloužila manželce Námanove.
\par 3 Kteráž rekla paní své: Ó by pán muj dostal se k proroku, kterýž jest v Samarí, tedy on by ho uzdravil od malomocenství jeho.
\par 4 A on všed, oznámil pánu svému, rka: Takto a takto pravila devecka, kteráž jest z zeme Izraelské.
\par 5 Tedy rekl král Syrský: Jdi, vyprav se, a já pošli list králi Izraelskému. A odcházeje, vzal s sebou deset centnéru stríbra a šest tisícu zlatých, nad to desatero roucho promenné.
\par 6 I prinesl list králi Izraelskému v tato slova: Hned, jakž te dojde list tento, aj, poslal jsem k tobe Námana služebníka svého, abys ho uzdravil od malomocenství jeho.
\par 7 I stalo se, že když precetl král Izraelský ten list, roztrhl roucho své a rekl: Zdali jsem já Bohem, abych mohl umrtviti a obživiti, že tento poslal ke mne, abych uzdravil muže od malomocenství jeho? Nýbrž posudte, prosím, a pohledte, že príciny hledá proti mne.
\par 8 A když uslyšel Elizeus muž Boží, že roztrhl král Izraelský roucho své, poslal k králi, rka: Proc jsi roztrhl roucho své? Necht nyní prijde ke mne a zvít, že jest prorok v Izraeli.
\par 9 A tak pribral se Náman s jízdou svou a s vozy svými, a stál u dverí domu Elizeova.
\par 10 Poslal pak k nemu Elizeus posla, rka: Jdi a umej se sedmkrát v Jordáne, a uzdraveno bude telo tvé, a cist budeš.
\par 11 Tedy rozhnevav se Náman, bral se odtud a mluvil: Hle, rekl jsem u sebe, že jistotne vyjde, a stoje, vzývati bude jméno Hospodina Boha svého, a vznášeje ruku svou nad místem neduživým, tak uzdraví malomocenství mé.
\par 12 Zdali nejsou lepší Abana a Farfar, reky Damašské, nad všecky vody Izraelské? Zdaliž bych se nemohl v nich zmýti, abych cist byl? A obrátiv se, jel s hnevem.
\par 13 Ale pristoupivše služebníci jeho, mluvili k nemu a rekli: Otce muj, jakkoli velikou vec prorok ten rozkázal by tobe, což bys nemel uciniti toho? Cím tedy více, kdyžt rekl: Umej se, a budeš cist.
\par 14 A protož sstoupiv, pohrížil se v Jordáne sedmkrát vedlé reci muže Božího. I ucineno jest telo jeho jako telo dítete malého, a ocišten jest.
\par 15 A hned navrátil se k muži Božímu se vším komonstvem svým, a prišed, stál pred ním a rekl: Aj, již jsem poznal, žet není Boha na vší zemi, jediné v Izraeli. Protož nyní vezmi, prosím, dary tyto od služebníka svého.
\par 16 On pak rekl: Živt jest Hospodin, pred jehož oblícejem stojím, žet nic nevezmu. A ac ho nutil, aby vzal, však nikoli nechtel.
\par 17 I rekl Náman: Tedy nic? Ale necht jest dáno, prosím, mne služebníku tvému bríme zeme na dva mezky; nebot nebude více obetovati služebník tvuj zápalu aneb obetí bohum cizím, ale Hospodinu.
\par 18 V této však veci odpust Hospodin služebníku tvému: Když vchází pán muj do chrámu Remmon, aby se tam modlil, a on podpírá se na mou ruku, že i já skláním se v chráme Remmon. Toho sklánení mého v chráme Remmon necht neváží, prosím, Hospodin služebníku tvému pri té veci.
\par 19 I rekl jemu: Jdi v pokoji. A když odjel od neho nedaleko,
\par 20 V tom rekl Gézi služebník Elizea muže Božího: Hle, nedopustil pán muj Námanovi Syrskému tomuto, aby dáti mel z ruky své to, což privezl. Živt jest Hospodin, žet pobehnu za ním, a vezmu neco od neho.
\par 21 Protož bežel Gézi za Námanem. A vida Náman bežícího za sebou, rychle skocil s vozu vstríc jemu a rekl: Dobre-li se deje?
\par 22 A on odpovedel: Dobre. Pán muj poslal mne, atbych rekl: Aj, ted nyní prišli ke mne dva mládenci s hory Efraim z synu prorockých; dej jim medle centnér stríbra a dvoje roucho promenné.
\par 23 I rekl Náman: Radeji vezmi dva centnére. I prinutil ho. A svázav dva centnére stríbra do dvou pytlu, a dvoje roucho promenné, vložil to na dva pacholky své, kteríž nesli pred ním.
\par 24 A když prišel na vrch, vzal to z rukou jejich, a složil v jednom dome, propustiv muže ty, kterížto odešli.
\par 25 Sám pak všed, stál pred pánem svým. I rekl mu Elizeus: Odkudžto, Gézi? Odpovedel: Nechodil služebník tvuj nikam.
\par 26 Kterýž rekl jemu: Zdaliž srdce mé nebylo pri tom, když obrátil se muž s vozu svého vstríc tobe? Zdaliž cas byl bráti stríbro aneb roucho, aneb olivoví a vinice, aneb stáda a voly, neb služebníky aneb devky?
\par 27 Protož malomocenství Námanovo prichytí se tebe i semene tvého na veky. I vyšel od tvári jeho malomocný jako sníh.

\chapter{6}

\par 1 Rekli pak synové proroctí Elizeovi: Ej, ted místo toto, v nemž bydlíme s tebou, jest nám tesné.
\par 2 Medle, necht jdeme až k Jordánu, abychom vzali odtud jeden každý jedno drevo, a udeláme sobe tu místo, v nemž bychom bydlili. Jimž rekl: Jdete.
\par 3 I rekl jeden: Prosím, pod také s služebníky svými. Kterýž odpovedel: A já pujdu.
\par 4 Takž šel s nimi. A když prišli k Jordánu, sekali dríví.
\par 5 I stalo se, když jeden z nich podtínal drevo, že sekera spadla mu do vody. Tedy zkrikl a rekl: Ach, pane muj, a ta ješte byla vypujcená.
\par 6 Jemuž rekl muž Boží: Kamž jest upadla? I ukázal mu to místo. Kterýž utav drevo, uvrhl je tam, a ucinil, aby zplynula sekera.
\par 7 A rekl: Vezmi ji sobe. Kterýž vztáh ruku svou, vzal ji.
\par 8 Když pak král Syrský bojoval proti Izraelovi, a vešel v radu s služebníky svými, rka: Na tom a na tom míste položí se vojsko mé:
\par 9 Tedy poslal muž Boží k králi Izraelskému, rka: Viz, abys netáhl pres to místo, nebo tam Syrští jsou v zálohách.
\par 10 Protož posílal král Izraelský na to místo, o kterémž mu byl rekl muž Boží, a vystríhal ho, aby se ho šetril a to nejednou ani dvakrát.
\par 11 A tak zkormoutil se v srdci svém král Syrský pro tu vec, a svolav služebníky své, rekl jim: Proc mi neoznámíte, kdo z našich králi Izraelskému donáší?
\par 12 Jemuž rekl jeden z služebníku jeho: Nikoli, pane muj králi, ale Elizeus prorok, kterýž jest v Izraeli, oznamuje králi Izraelskému slova, kteráž ty mluvíš v nejtajnejším pokoji svém.
\par 13 Kterýž rekl: Jdete a vizte, kde jest, abych poslal a jal jej. I oznámeno jemu temito slovy: Hle, jest v Dotain.
\par 14 Protož poslal tam kone a vozy a vojsko veliké. Kteríž pritáhše v noci, oblehli mesto.
\par 15 Vstav pak ráno služebník muže Božího, vyšel, a aj, vojsko obklícilo mesto, koni i vozové. I rekl služebník ten jeho k nemu: Ach, pane muj, což budeme delati?
\par 16 Kterýž odpovedel: Neboj se, nebo mnohem více jich s námi jest, než s nimi.
\par 17 I modlil se Elizeus a rekl: Ó Hospodine, otevri, prosím, oci jeho, aby videl. A tak otevrel Hospodin oci služebníka toho, a videl, a aj, hora ta plná konu, a vozové ohniví okolo Elizea.
\par 18 A když neprátelé táhli k nemu, modlil se Elizeus Hospodinu a rekl: Poraz, prosím, národ tento slepotou. I porazil je slepotou vedlé reci Elizeovy.
\par 19 V tom rekl jim Elizeus: Nenít to ta cesta, ani to mesto. Podte za mnou, a dovedu vás k muži, kteréhož hledáte. Takž je vedl do Samarí.
\par 20 I stalo se, když vešli do Samarí, že rekl Elizeus: Ó Hospodine, otevri oci techto, at vidí. Tedy otevrel Hospodin oci jejich, a videli, že jsou u prostred Samarí.
\par 21 Rekl pak král Izraelský Elizeovi, když je uzrel: Mám-liž je zmordovati, otce muj?
\par 22 Odpovedel on: Nemorduj. Zdaliž jsi je zjímal mecem svým a lucištem svým, abys je zmordoval? Dej jim chleba a vody, at jedí a pijí, a navrátí se ku pánu svému.
\par 23 A tak pripravil jim hojnost velikou, a když pojedli a napili se, propustil je. Oni pak navrátili se ku pánu svému, aniž kdy více potom lotríkové Syrští vskakovali do zeme Izraelské.
\par 24 Stalo se potom, že shromáždil Benadad král Syrský všecka vojska svá, a pritáh, oblehl Samarí.
\par 25 Procež byl hlad veliký v Samarí; nebo aj, tak dlouho obleženo bylo, až hlava oslová byla za osmdesáte stríbrných, a ctvrtý díl míry káb trusu holubích za pet stríbrných.
\par 26 I prihodilo se, když král Izraelský šel po zdi, žena jedna zvolala k nemu, rkuci: Spomoz mi, pane muj králi.
\par 27 Kterýž rekl: Jestlit nespomuže Hospodin, odkud já mám pomoci tobe? Zdali z humna aneb z presu?
\par 28 Rekl jí ješte král: Cožte pak? Kteráž odpovedela: Žena tato rekla mi: Dej syna svého, abychom ho dnes snedly, zítra také sníme syna mého.
\par 29 I uvarily jsme syna mého, a snedly jsme jej. Potom druhého dne rekla jsem jí: Dej syna svého, abychom ho snedly. Ale ona skryla syna svého.
\par 30 A když uslyšel král slova ženy té, roztrhl roucha svá. (Když pak on šel po zdi, videl lid, an pytel byl na tele jeho zespod.)
\par 31 Protož rekl král: Toto mi ucin Buh a toto pridej, jestliže zustane hlava Elizea syna Safatova na nem dnes.
\par 32 (Elizeus pak sedel v dome svém, a starší s ním sedeli.) I poslal jednoho z prístojících svých, a prvé než prišel posel ten k nemu, již byl rekl starším: Nevíte-liž, že poslal ten syn vražedlníkuv, aby stal hlavu mou? Šetrtež, když by vcházel ten posel, zavrete dvére a odstrcte jej ode dverí. Zdaliž i dusání noh pána jeho není za ním?
\par 33 A když on ješte mluvil s nimi, hle, posel pricházel k nemu, a rekl: Aj, toto zlé jest od Hospodina, což mám déle cekati na nej?

\chapter{7}

\par 1 Rekl pak Elizeus: Slyšte slovo Hospodinovo. Toto praví Hospodin: O tomto casu zítra míra mouky belné bude za lot stríbra, a dve míry jecmene za lot stríbra v bráne Samarské.
\par 2 I odpovedel muži Božímu jeden z knížat, na jehož ruku král zpoléhal, a rekl: Kdyby otevrel Hospodin pruduchy nebeské, zdali by to býti mohlo? Kterýž rekl: Aj, uzríš ocima svýma, ale nebudeš jísti z toho.
\par 3 Byli pak u vrat brány ctyri muži malomocní, kteríž rekli jeden druhému: Co tu sedíme, až bychom zemríti musili?
\par 4 Díme-li: Podme do mesta, hlad jest v meste, i zemreme tam; pakli zde zustaneme, také zemreme. Protož nyní podte, a utecme do vojska Syrských. Budou-li nás živiti, zustaneme živi; pakli nás zbijí, též zemreme.
\par 5 Vstavše tedy v soumrak, aby šli k vojsku Syrských, prišli až k kraji ležení Syrského, a hle, nebylo tu žádného.
\par 6 Nebo Pán ucinil to, aby slyšelo vojsko Syrské hrmot vozu a zvuk konu, a tak hluk vojska velikého. I rekli jeden druhému: Hle, ze mzdy najal proti nám král Izraelský krále Hetejské a krále Egyptské, aby pripadli na nás.
\par 7 A vstavše, utekli v soumrak a nechali tu stanu svých, a konu svých i oslu svých, a ležení, tak jakž bylo, a utekli pro zachování života svého.
\par 8 Když tedy prišli malomocní ti na kraj ležení, všedše do jednoho stanu, jedli a pili, a pobravše v nem stríbro a zlato i roucha, odešli a schovali. Opet navrátivše se, vešli do jiného stanu a pobrali v nem, a odšedše, schovali.
\par 9 I rekli vespolek: Nedobre deláme. Den tento jest den dobrých novin, a my mlcíme. Budeme-li cekati až do svitání, postihne nás nepravost; protož nyní podte, vejdouce, oznamme to domu královskému.
\par 10 A prišedše, volali na branné mesta, a povedeli jim, rkouce: Prišli jsme do ležení Syrského, a aj, nebylo tam žádného, ani hlasu lidského, krome koní privázaných a oslu privázaných, a stanu, jakž prvé byli.
\par 11 I volal ten na jiné branné, a ti ohlásili to po všem dome královském.
\par 12 Vstav tedy král v noci, rekl služebníkum svým: Povím vám nyní, co jsou nám udelali Syrští. Vedí, že jsme hladovití, protož vyšli z ležení, aby se skryli v poli, rkouce: Když vyjdou z mesta, zjímáme je živé a vejdeme do mesta.
\par 13 K tomu odpovídaje jeden z služebníku jeho, rekl: Necht, prosím, vezmou pet koní z tech, kteríž pozustali v meste. (Hle, onit jsou jako i všecko množství Izraelské, jenž zustali v nem, hle, jsou jako všecko množství Izraelské, kteréž již hyne.) Ty pošleme a prezvíme.
\par 14 A tak vzali dva kone vozní, kteréž poslal král za vojskem Syrským, rka: Jdete a vizte.
\par 15 I jeli za nimi až k Jordánu, a aj, po vší té ceste plno bylo šatu a nádob, kteréž metali od sebe Syrští, splašeni jsouce. Tedy navrátivše se poslové, oznámili to králi.
\par 16 Protož vyšed lid, vzebral ležení Syrské. A byla míra mouky belné za lot, a dve míry jecmene za lot vedlé reci Hospodinovy.
\par 17 Král pak byl ustanovil to kníže, na jehož ruku zpoléhal, u brány. Kteréhož pošlapal lid v bráne, tak že umrel, vedlé reci muže Božího, kterouž mluvil, když král k nemu byl sešel.
\par 18 A stalo se tak, jakž mluvil muž Boží králi, rka: Dve míry jecmene budou za lot stríbra, a míra mouky belné za lot stríbra zítra o tomto casu v bráne Samarské.
\par 19 K cemuž bylo odpovedelo kníže muži Božímu, a reklo: Kdyby otevrel Hospodin pruduchy nebeské, zdali by to podlé reci této býti mohlo? Jemuž on rekl: Aj, ty uzríš ocima svýma, ale z toho jísti nebudeš.
\par 20 A tak se stalo jemu; nebo pošlapal ho lid v bráne, tak že umrel.

\chapter{8}

\par 1 Mluvil pak Elizeus k žene té, jejíhož byl syna vzkrísil, rka: Vstan a jdi, ty i celed tvá, a bud pohostinu, kdežkoli budeš moci byt míti; nebo zavolal Hospodin hladu, a prijdet na zemi za sedm let.
\par 2 Vstavši tedy žena, ucinila vedlé reci muže Božího, a odešla s celedí svou, a bydlila pohostinu v zemi Filistinských sedm let.
\par 3 I stalo se, že když pominulo tech sedm let, navrátila se ta žena z zeme Filistinské, a šla, aby prosila krále za dum svuj a za pole své.
\par 4 Mezi tím král mluvil s Gézi, služebníkem muže Božího, rka: Vypravuj mi medle o všech vecech velikých, kteréž cinil Elizeus.
\par 5 A když on vypravoval králi, kterak obživil mrtvého, aj žena, jejíhož syna vzkrísil, prosila krále za dum svuj a za pole své. I rekl Gézi: Pane muj králi, to jest ta žena, a tento syn její, kteréhož vzkrísil Elizeus.
\par 6 Tedy tázal se král ženy, kteráž vypravovala jemu. I porucil král komorníku jednomu, rka: At jsou navráceny všecky veci, kteréž byly její, i všecky užitky pole, ode dne, v nemž opustila zemi, až po dnes.
\par 7 Potom prišel Elizeus do Damašku, král pak Syrský Benadad stonal. I oznámeno jemu temito slovy: Prišel muž Boží sem.
\par 8 I rekl král Hazaelovi: Vezmi v ruce své dar, a jdi vstríc muži Božímu, a zeptej se Hospodina skrze neho, rka: Povstanu-li z nemoci této?
\par 9 A tak šel Hazael vstríc jemu, vzav s sebou dar, a všeliké veci výborné Damašské, bremena na ctyridcíti velbloudích, a prišed, stál pred ním a rekl: Syn tvuj Benadad, král Syrský, poslal mne k tobe, rka: Povstanu-li z nemoci této?
\par 10 Odpovedel mu Elizeus: Jdi, povez jemu: Zajisté mohl bys živ zustati. Ale Hospodin mi ukázal, že jistotne umre.
\par 11 V tom promeniv muž Boží tvár svou, ukázal se k nemu neochotne a plakal.
\par 12 Jemuž rekl Hazael: Proc pán muj pláce? Odpovedel: Proto že vím, co zlého ty uciníš synum Izraelským. Pevnosti jejich spálíš a mládence jejich zmorduješ mecem, a nemluvnátka jejich rozrážeti budeš, a tehotné jejich zroztínáš.
\par 13 I rekl Hazael: I což jest služebník tvuj pes, aby uciniti mohl vec tak velikou? Odpovedel Elizeus: Ukázal te mi Hospodin, že budeš králem Syrským.
\par 14 A odšed od Elizea, prišel ku pánu svému. Kterýž rekl jemu: Cožt rekl Elizeus? Odpovedel: Pravil mi, že bys mohl živ zustati.
\par 15 I stalo se nazejtrí, že vzal koberec, a namociv jej v vode, prostrel jej na tvár jeho. I umrel, a kraloval Hazael místo neho.
\par 16 Léta pak pátého Jorama, syna Achabova krále Izraelského, a Jozafata krále Judského, pocal kralovati Jehoram, syn Jozafatuv král Judský.
\par 17 Ve dvou a tridcíti letech byl, když pocal kralovati, a kraloval osm let v Jeruzaléme.
\par 18 A chodil po ceste králu Izraelských, tak jako cinil dum Achabuv; nebo mel dceru Achabovu za manželku. Takž cinil zlé veci pred ocima Hospodinovýma.
\par 19 Hospodin však nechtel zahladiti Judy pro Davida služebníka svého, jakž mu byl zaslíbil, že dá jemu svíci i synum jeho po všecky dny.
\par 20 Ve dnech jeho odstoupil Edom od království Judského, a ustanovili nad sebou krále.
\par 21 Tou prícinou táhl Jehoram do Seir, i všickni vozové s ním. A vstav v noci, porazil Idumejské, kteríž jej byli obklícili, i hejtmany vozu, a lid utekl do stanu svých.
\par 22 Však predce odstoupil Edom od království Judského až do tohoto dne. Takž odstoupilo i Lebno téhož casu.
\par 23 O jiných pak vecech Jehoramových, a o všem, cožkoli cinil, psáno jest v knize o králích Judských.
\par 24 I usnul Jehoram s otci svými, a pochován jest s otci svými v meste Davidove, kraloval pak Ochoziáš syn jeho místo neho.
\par 25 Léta dvanáctého Jorama syna Achabova, krále Izraelského, pocal kralovati Ochoziáš syn Jehorama, krále Judského.
\par 26 Ve dvamecítma letech byl Ochoziáš, když pocal kralovati, a kraloval jeden rok v Jeruzaléme. Jméno matky jeho bylo Atalia, dcera Amri krále Izraelského.
\par 27 A chodil cestou domu Achabova, cine, což jest zlého pred ocima Hospodinovýma, jako i dum Achabuv, nebo byl zetem domu Achabova.
\par 28 Procež vycházel s Joramem synem Achabovým, na vojnu proti Hazaelovi králi Syrskému, do Rámot Galád; ale porazili Syrští Jorama.
\par 29 A tak navrátil se král Joram, aby se hojil v Jezreel na rány, kterýmiž ho ranili Syrští v Ráma, když bojoval s Hazaelem králem Syrským. Ochoziáš pak syn Jehorama, krále Judského, prijel, aby navštívil Jorama syna Achabova v Jezreel; nebo tam nemocen byl.

\chapter{9}

\par 1 Prorok pak Elizeus povolal jednoho z synu prorockých, a rekl jemu: Prepaš bedra svá, a vezmi tuto nádobku oleje do ruky své, a jdi do Rámot Galád.
\par 2 A když tam prijdeš, uzríš tam Jéhu, syna Jozafata syna Namsi. K nemuž když prijdeš, vyvoláš ho z prostredku bratrí jeho, a uvedeš ho do nejtajnejšího pokojíku.
\par 3 A vezma nádobku oleje, vyleješ na hlavu jeho a díš: Toto praví Hospodin: Pomazuji tebe za krále nad Izraelem. A hned otevra dvére, utec a nemeškej se.
\par 4 I odšel mládenec ten služebník prorokuv do Rámot Galád.
\par 5 A když prišel, aj, knížata vojska sedeli. I rekl: Mámt neco povedíti, ó kníže. I rekl Jéhu: Komu ze všech nás? Odpovedel: Tobe, ó kníže.
\par 6 Vyvstav tedy, všel do vnitrku. I vylil olej na hlavu jeho a rekl jemu: Toto praví Hospodin, Buh Izraelský: Pomazuji tebe za krále nad lidem Hospodinovým, nad Izraelem.
\par 7 A zmorduješ dum Achaba pána svého; nebot pomstím krve služebníku svých proroku, a krve všech služebníku Hospodinových z ruky Jezábel.
\par 8 A tak zahyne všecken dum Achabuv, a vypléním z domu Achabova mocícího na stenu, a zajatého i zanechaného v Izraeli.
\par 9 Uciním zajisté domu Achabovu, jako domu Jeroboámovu syna Nebatova, a jako domu Bázy syna Achiášova.
\par 10 Jezábel pak sežerou psi na poli Jezreel, a nebude, kdo by ji pochoval. A rychle otevrev dvére, utekl.
\par 11 A když Jéhu vyšel k služebníkum pána svého, rekl mu jeden: Dobre-li se deje? Proc prišel ten blázen k tobe? Odpovedel jim: Vy znáte cloveka i rec jeho.
\par 12 I rekli: Klamt jest, povez nám medle. Kterýž rekl: Takto a takto mluvil mi, rka: Toto praví Hospodin: Pomazuji te za krále nad Izraelem.
\par 13 Tedy rychle vzali jeden každý roucho své, a prostreli pod nej na nejvyšším stupni, a troubíce v troubu, pravili: Kralujet Jéhu.
\par 14 Zatím spuntoval se Jéhu syn Jozafata, syna Namsi, proti Joramovi, (když Joram ostríhaje Rámot Galád spolu se vším Izraelem, pred Hazaelem králem Syrským,
\par 15 Navrátil se byl Joram král, aby se hojil v Jezreel na rány, kterýmiž ho ranili Syrští, když bojoval proti Hazaelovi králi Syrskému). A rekl Jéhu: Vidí-li se vám, necht nevychází žádný z mesta, kterýž by šel a oznámil to v Jezreel.
\par 16 I vsedl na vuz Jéhu a jel do Jezreel, nebo Joram ležel tam. Ochoziáš také král Judský prijel byl, aby navštívil Jorama.
\par 17 V tom strážný, kterýž stál na veži v Jezreel, když videl houf Jéhu pricházející, rekl: Vidím jakýsi houf. I rekl Joram: Povolej jízdného, a pošli vstríc jim, aby se otázal: Jest-li pokoj?
\par 18 I vyjel jízdný proti nim a rekl: Takto praví král: Jest-li pokoj? Odpovedel Jéhu: Co tobe do pokoje? Jed za mnou. Protož oznámil to strážný, rka: Prijel posel až k nim, ale nevracuje se.
\par 19 Ješte poslal druhého jízdného. Kterýž prijel k nim a rekl: Takto praví král: Jest-li pokoj? Odpovedel Jéhu: Co tobe do pokoje? Jed za mnou.
\par 20 Opet oznámil to strážný, rka: Prijel až k nim, ale nevrací se. Príjezd pak jest jako príjezd Jéhu syna Namsi; nebo vztekle jede.
\par 21 Tedy rekl Joram: Zapráhni. I zapráhl k vozu jeho. Takž vyjel Joram král Izraelský a Ochoziáš král Judský, každý na voze svém, a vyjevše proti Jéhu, potkali se s ním na poli Nábota Jezreelského.
\par 22 I stalo se, když uzrel Joram Jéhu, že rekl: Jest-liž pokoj, Jéhu? Odpovedel: Jaký pokoj, ponevadž ješte smilství Jezábel matky tvé a kouzelnictví její velmi mnohá trvají?
\par 23 Procež obrátiv se Joram, utíkal a rekl Ochoziášovi: Zrada, Ochoziáši!
\par 24 Ale Jéhu pochytiv lucište, postrelil Jorama mezi ramenem jeho, tak že strela pronikla srdce jeho. I padl na voze svém.
\par 25 Zatím rekl Jéhu Badakerovi, hejtmanu svému: Vezma, povrz jej na pole Nábota Jezreelského; nebo pamatuješ, když jsme já a ty jeli spolu za Achabem otcem jeho, že Hospodin vynesl proti nemu pohružku tuto, rka:
\par 26 Zajisté krve Nábotovy a krve synu jeho, kterouž jsem videl vcera, rekl Hospodin, pomstím na tobe na poli tomto. Hospodin to rekl, nyní tedy vezma, povrz jej na pole, vedlé reci Hospodinovy.
\par 27 Ochoziáš pak král Judský uzrev to, utíkal cestou k domu zahradnímu. Ale honil ho Jéhu a rekl: I toho zabíte na voze jeho. Takž ho ranili, když vyjíždel k Guru, kteréž jest podlé Jibleam. I utekl do Mageddo a tam umrel.
\par 28 Kteréhož vloživše na vuz služebníci jeho, vezli jej do Jeruzaléma, a pochovali jej v hrobe jeho s otci jeho v meste Davidove.
\par 29 Léta jedenáctého Jorama syna Achabova pocal kralovati Ochoziáš nad Judou.
\par 30 Potom Jéhu prijel do Jezreel. O cemž když uslyšela Jezábel, ulícila tvár svou, a ozdobila hlavu svou, a vyhlédala z okna.
\par 31 A když Jéhu jel skrze bránu, rekla: Jest-li pokoj, ó Zamri, mordéri pána svého?
\par 32 On pak pozdvih tvári své k oknu, rekl: Kdo drží se mnou, kdo? I vyhlédli na nej dva ci tri komorníci její.
\par 33 Jimž rekl: Svrzte ji. Kterížto svrhli ji. I pokropena jest krví její stena i koni; a tak pošlapal ji.
\par 34 A když prijel, jedl a pil, a rekl: Pohledte již na tu zlorecenou a pochovejte ji, nebt jest dcera královská.
\par 35 Protož odšedše, aby ji pochovali, nenalezli z ní než leb a nohy a dlane rukou.
\par 36 A navrátivše se, povedeli jemu. Kterýž rekl: Slovo Hospodinovo jest, kteréž mluvil skrze služebníka svého, Eliáše Tesbitského, rka: Na poli Jezreel žráti budou psi telo Jezábel.
\par 37 Budiž tedy telo Jezábel na poli Jezreel, jako hnuj na svrchku pole, tak aby žádný nerekl: Tato jest Jezábel.

\chapter{10}

\par 1 Mel pak Achab sedmdesáte synu v Samarí. I napsal Jéhu list, a poslal jej do Samarí k knížatum Jezreelským, starším, a kteríž chovali syny Achabovy, s tímto porucením:
\par 2 Hned jakž vás dojde list tento, ponevadž u vás jsou synové pána vašeho, jsou i vozové u vás, i koni, i mesto hrazené a zbroj,
\par 3 Vyberte nejlepšího a nejzpusobnejšího z synu pána svého, a posadte na stolici otce jeho, a bojujte za dum pána svého.
\par 4 Kteríž bojíce se náramne, pravili: Aj, dva králové neostáli pred ním, a kterak my ostojíme?
\par 5 A tak poslal ten, kterýž byl ustanoven nad domem, a kterýž ustanoven byl nad mestem, i starší, a kteríž chovali syny Achabovy, k Jéhu, rkouce: Služebníci tvoji jsme, a všecko, což nám rozkážeš, uciníme. Neustanovíme žádného krále; což se tobe dobre líbí, ucin.
\par 6 I napsal k nim list po druhé, rka: Jste-li moji a hlasu mého posloucháte-li, vezmouce hlavy všech synu pána svého, pridte ke mne zítra o tomto casu do Jezreel. (Synu pak králových bylo sedmdesáte mužu u nejprednejších v meste, kteríž chovali je.)
\par 7 Tedy stalo se, jakž jich došel list ten, že zjímali syny královské, a zbili sedmdesáte mužu, a vkladše hlavy jejich do košu, poslali je k nemu do Jezreel.
\par 8 A prišed posel, oznámil jemu, rka: Prinesli hlavy synu královských. A on rekl: Skladte je na dve hromady u vrat brány až do jitra.
\par 9 Potom ráno vyšed, postavil se a rekl všemu lidu: Spravedliví jste vy. Aj, já spuntoval jsem se proti pánu svému a zabil jsem jej; kdo by pobil tyto všecky?
\par 10 Veztež nyní, žet nepochybilo nižádné slovo Hospodinovo, kteréž mluvil Hospodin proti domu Achabovu, ale vykonal Hospodin to, což byl mluvil skrze služebníka svého Eliáše.
\par 11 A tak pobil Jéhu všecky, kteríž pozustali z domu Achabova v Jezreel, i všecky nejprednejší jeho, i známé jeho, i kneží jeho, tak že nezustalo z nich žádného živého.
\par 12 Potom vstav, odebral se a jel do Samarí, a již byl v Betekedu pastýru na ceste.
\par 13 I nalezl tam Jéhu bratrí Ochoziáše, krále Judského, a rekl: Kdo jste vy? Odpovedeli: Jsme bratrí Ochoziášovi a jdeme, abychom pozdravili synu králových a synu královny.
\par 14 Tedy rekl: Zjímejte je živé. I zjímali je živé, a pobili je u cisterny v Betekedu, ctyridceti a dva muže, a nezustal z nich žádný.
\par 15 Potom bera se odtud, nalezl Jonadaba syna Rechabova, kterýž se s ním potkal, a pozdravil ho. I rekl jemu: Jest-liž srdce tvé prímé, jako jest srdce mé s srdcem tvým? Odpovedel Jonadab: Jest, arci jest. Rekl Jéhu: Podejž mi ruky své. I podal mu ruky své. A on kázal mu vsednouti k sobe na vuz,
\par 16 A rekl: Pojed se mnou, a viz horlivost mou pro Hospodina. A tak vezli jej na voze jeho.
\par 17 Když pak prijel do Samarí, pobil všecky, kteríž byli pozustali z domu Achabova v Samarí, a vyhladil jej vedlé reci Hospodinovy, kterouž byl mluvil k Eliášovi.
\par 18 Zatím shromáždiv Jéhu všecken lid, rekl jim: Achab málo sloužil Bálovi, Jéhu bude mu více sloužiti.
\par 19 Protož nyní všecky proroky Bálovy, všecky služebníky jeho, a všecky kneží jeho svolejte ke mne, at žádný tam nezustává; nebo velikou obet budu obetovati Bálovi. Kdož by koli nebyl prítomen, nezustane živ. Ale Jéhu cinil to chytre, aby vyhladil ctitele Bálovy.
\par 20 Rekl také Jéhu: Zasvette svátek Bálovi. I prohlásili jej.
\par 21 Rozeslal zajisté Jéhu po vší zemi Izraelské. I sešli se všickni ctitelé Bálovi, tak že nezustalo ani jednoho, ješto by neprišel. A když vešli do domu Bálova, naplnen jest dum Báluv, od jednoho konce až do druhého.
\par 22 Tedy rekl tomu, kterýž vládl rouchem: Vynes roucha všechnem ctitelum Bálovým. I prinesl jim roucha.
\par 23 Potom všel i Jéhu a Jonadab syn Rechabuv do domu Bálova, a rekl ctitelum Bálovým: Vyhledejte a vizte, at není zde s vámi nekdo z ctitelu Hospodinových, krome samých ctitelu Bálových.
\par 24 A tak vešli, aby obetovali obeti a zápaly. Jéhu pak postavil sobe vne osmdesáte mužu, jimž byl rekl: Jestliže kdo utece z lidí tech, kteréž já uvozuji vám v ruce vaše, život váš za život jeho.
\par 25 I stalo se, když knez dokonal obetování zápalu, rekl Jéhu drabantum a hejtmanum: Vejdetež a zbíte je, at žádný neuchází. Kterížto pobili je ostrostí mece, a rozmetali tela jejich drabanti a hejtmané. Potom šli dále do každého mesta, kdež byl dum Báluv,
\par 26 A vymítajíce modly z domu Bálova, pálili je.
\par 27 Zkazili také modlu Bálovu, a zborivše dum jeho, nadelali z neho záchodu až do dnešního dne.
\par 28 A tak vyplénil Jéhu Bále z lidu Izraelského.
\par 29 A však proto od hríchu Jeroboáma syna Nebatova, kterýž privedl k hrešení Izraele, neodstoupil Jéhu, totiž od tech telat zlatých, kteráž byla v Bethel a v Dan.
\par 30 Tedy rekl Hospodin Jéhu: Ponevadž jsi snažne vykonal to, což dobrého jest pred ocima mýma, a všecky veci, kteréž jsem mel v srdci svém, ucinil jsi domu Achabovu, synové tvoji až do ctvrtého pokolení sedeti budou na stolici Izraelské.
\par 31 Ale Jéhu nebyl toho pilen, aby chodil v zákone Hospodina, Boha Izraelského, celým srdcem svým, aniž odstoupil od hríchu Jeroboáma, kterýž byl uvedl v hríchy lid Izraelský.
\par 32 Za tech dnu pocal Hospodin zmenšovati Izraele; nebo je porazil Hazael po všech koncinách Izraelských,
\par 33 Od Jordánu, proti východu slunce, všecku zemi Galád, Gádovu a Rubenovu i Manassesovu, od Aroer, kteréž jest pri potoku Arnon, tak Galád jako Bázan.
\par 34 O jiných vecech Jéhu, a cožkoli cinil, i o vší síle jeho, sepsáno jest v knize o králích Izraelských.
\par 35 I usnul Jéhu s otci svými, a pochovali jej v Samarí, a kraloval Joachaz syn jeho místo neho.
\par 36 Dnu pak, v nichž kraloval Jéhu nad lidem Izraelským v Samarí, bylo let osmmecítma.

\chapter{11}

\par 1 Atalia pak matka Ochoziášova viduci, že umrel syn její, vstavši, pomordovala všecko síme královské.
\par 2 Ale Jozaba dcera krále Jehorama, sestra Ochoziášova, vzala Joasa syna Ochoziášova, a ukradši ho z prostredku synu královských, kteríž mordováni byli, skryla jej i s chuvou jeho v pokoji, kdež luže byla. A tak skryli ho pred Atalií, a není zamordován.
\par 3 I byl s ní v dome Hospodinove tajne za šest let, v nichž Atalia kralovala nad zemí.
\par 4 Léta sedmého poslav Joiada, povolal setníku, hejtmanu a drabantu. I uvedl je k sobe do domu Hospodinova, a ucinil s nimi smlouvu, a zavázav je prísahou v dome Hospodinove, ukázal jim syna králova.
\par 5 A prikázal jim, rka: Tato jest vec, kterouž uciníte: Tretí díl vás, kteríž pricházíte v sobotu, a držíte stráž, bude pri dome králove,
\par 6 A díl tretí bude u brány Sur, a tretí díl bude u brány za drabanty, a budete stráž držeti, ostríhajíce domu tohoto pred outokem.
\par 7 Vy pak všickni, kteríž byste odjíti meli v sobotu, po vykonání povinnosti pri dome Hospodinove, ve dve pobocní stráže rozdelení, budte pri králi.
\par 8 A tak obstoupíte krále vukol, jeden každý majíce bran svou v rukou svých. Kdož by pak šel do šiku vašeho, at umre; a budete pri králi, když bude vycházeti i když bude vcházeti.
\par 9 Protož ucinili setníci ti všecky veci tak, jakž byl rozkázal Joiada knez, a vzavše jeden každý muže své, kteríž pricházeli v sobotu a kteríž odcházeli v sobotu, prišli k Joiadovi knezi.
\par 10 I dal knez setníkum kopí a pavézy, kteréž byly Davida krále, a kteréž byly v dome Hospodinove.
\par 11 Stáli pak ti drabanti, jeden každý držíce bran svou v ruce své, od pravé strany domu až do levé strany domu, proti oltári a proti domu pri králi vukol.
\par 12 Tedy vyvedl syna králova, a vstavil na nej korunu a ozdobu. I ustanovili jej králem a pomazali ho, a plésajíce rukama, ríkali: Živ bud král!
\par 13 V tom uslyševši Atalia hluk plésajícího lidu, vešla k lidu do domu Hospodinova.
\par 14 A když pohledela, a aj, král stál na míste vyšším, vedlé obyceje s knížaty, a trouby byly pred králem, a všecken lid zeme byl vesel, i troubili v trouby. Tedy roztrhla Atalia roucho své a zkrikla: Spiknutí, spiknutí!
\par 15 Protož rozkázal Joiada knez tem setníkum ustaveným nad vojskem, rka jim: Pustte ji prostredkem radu, a i toho, kdož by za ní šel, zabíte mecem. Nebo byl rekl knez: At není zabita v dome Hospodinove.
\par 16 I pustili ji. Ale když prišla na cestu, kudy koni vcházejí do domu královského, tu jest zabita.
\par 17 Tedy ucinil Joiada smlouvu mezi Hospodinem a mezi králem, i mezi lidem, aby byli lid Hospodinuv; též mezi králem a mezi lidem.
\par 18 Potom šel všecken lid zeme do domu Bálova, a zborili jej, i oltáre jeho a obrazy jeho v kusy stroskotali; Matana také kneze Bálova zabili pred oltári. Knez pak znovu narídil prisluhující v dome Hospodinove.
\par 19 A pojav ty setníky a hejtmany, i drabanty se vším lidem zeme, provázeli krále z domu Hospodinova, a prišli cestou k bráne drabantu do domu královského. Kdežto posadil se na stolici královské.
\par 20 I veselil se všecken lid zeme, a mesto se upokojilo. Atalii pak zabili mecem u domu královského.
\par 21 A byl Joas v sedmi letech, když pocal kralovati.

\chapter{12}

\par 1 Léta sedmého Jéhu pocal kralovati Joas, a kraloval ctyridceti let v Jeruzaléme. Jméno matky jeho bylo Sebia z Bersabé.
\par 2 I cinil Joas, což dobrého jest pred ocima Hospodinovýma po všecky dny své, pokudž ho vyucoval Joiada knez.
\par 3 A však výsosti nebyly zkaženy, ješte lid obetoval a kadil na tech výsostech.
\par 4 Rekl pak Joas knežím: Všecky peníze svaté, kteréž se vnášejí do domu Hospodinova, totiž peníze tech, kteríž jdou v pocet, a peníze ceny za osobu jednoho každého, a všecky peníze, kteréž by kdo dobrovolne uložil dáti do domu Hospodinova,
\par 5 Vezmou kneží k sobe, jeden každý od známého svého, a oni opraví zboreniny domu všudy, kdež by bylo zborení.
\par 6 Stalo se potom léta dvadcátého tretího krále Joasa, když ješte neopravili kneží zborenin chrámových,
\par 7 Že povolal král Joas Joiady kneze i jiných kneží, a rekl jim: Proc neopravujete zborenin chrámových? Protož nyní neprijímejte penez od známých svých, ale na zboreniny domu dávejte je.
\par 8 Jemuž povolivše kneží, nebrali penez od lidu, ani neopravovali domu.
\par 9 Protož Joiada knez vzav jednu truhlici, udelal díru v víku jejím, a postavil ji vedlé oltáre po pravé strane, kudy se vchází do domu Hospodinova. I postavili tu kneží ostríhající prahu i všech penez, kteréž vnášíny byly do domu Hospodinova.
\par 10 A když rozumeli, že by mnoho penez bylo v truhlici, tedy pricházel kanclér královský a knez nejvyšší, a scetše, schovávali ty peníze, kteréž se nalézaly v dome Hospodinove.
\par 11 Odkudž vydávali peníze hotové v ruce remeslníku postavených nad dílem domu Hospodinova, a ti obraceli je na tesare a delníky, kteríž opravovali dum Hospodinuv,
\par 12 Totiž na zedníky a kameníky, a k jednání dríví i tesaného kamení, aby opraveny byly zboreniny domu Hospodinova, i na všecko to, což obráceno melo býti na dum k oprave jeho.
\par 13 A však nebylo deláno do domu Hospodinova cíší stríbrných, žaltáru, kotlíku, trub a žádné nádoby zlaté, ani nádoby stríbrné z penez, kteréž prinášíny byly do domu Hospodinova,
\par 14 Ale tem, kteríž predstaveni byli nad dílem, dávali je, a opravovali na ne dum Hospodinuv.
\par 15 Aniž žádali poctu od mužu tech, jimž v ruce peníze dávali, aby platili delníkum; nebo verne delali.
\par 16 Penez za vinu, a penez za hríchy nebylo vnášíno do domu Hospodinova; knežím se dostávaly.
\par 17 Tedy vytáhl Hazael král Syrský, a bojoval proti Gát a dobyl ho. Potom obrátil Hazael tvár svou, aby táhl proti Jeruzalému.
\par 18 Protož pobral Joas král Judský všecky veci svaté, kterýchž byli nadali Jozafat a Jehoram a Ochoziáš, otcové jeho, králové Judští, i to, cehož sám posvetil, i všecko zlato, kteréž se nalezlo v pokladích domu Hospodinova a domu královského, a poslal k Hazaelovi králi Syrskému. I odtáhl od Jeruzaléma.
\par 19 O jiných pak cinech Joasových, a cožkoli cinil, psáno jest v knize o králích Judských.
\par 20 Potom povstavše služebníci jeho, spikli se spolu, a zabili Joasa v Betmillo, kudy se chodí do Silla,
\par 21 Totiž Jozachar syn Simatuv, a Jozabad syn Someruv. Ti služebníci jeho zabili ho, a umrel. I pochovali jej s otci jeho v meste Davidove, a kraloval Amaziáš syn jeho místo neho.

\chapter{13}

\par 1 Léta trimecítmého Joasa syna Ochoziáše, krále Judského, kraloval Joachaz syn Jéhu nad Izraelem v Samarí sedmnácte let.
\par 2 A cinil to, což jest zlého pred ocima Hospodinovýma; nebo následoval hríchu Jeroboáma syna Nebatova, kterýž privedl k hrešení Izraele, a neuchýlil se od nich.
\par 3 I rozhnevala se prchlivost Hospodinova na Izraele, a vydal je v ruku Hazaele krále Syrského, a v ruku Benadada syna Hazaelova po všecky ty dny.
\par 4 Ale když se modlil Joachaz Hospodinu, vyslyšel jej Hospodin. Videl zajisté trápení Izraelské, že je ssužoval král Syrský.
\par 5 Protož dal Hospodin Izraelovi vysvoboditele, a vyšli z ruky Syrských, i bydlili synové Izraelští v príbytcích svých jako kdy prvé.
\par 6 Však proto neodstoupili od hríchu domu Jeroboámova, kterýž privedl k hrešení Izraele, nýbrž v nich chodili, ano i háj ješte zustával v Samarí,
\par 7 Ackoli nezanechal Joachazovi lidu, krome padesáti jízdných a desíti vozu a desíti tisícu peších; nebo je pohubil král Syrský, a setrel je jako prach pri mlácení.
\par 8 O jiných pak cinech Joachazových, a cožkoli cinil, i o síle jeho zapsáno jest v knize o králích Izraelských.
\par 9 I usnul Joachaz s otci svými, a pochovali ho v Samarí. Kraloval pak Joas syn jeho místo neho.
\par 10 Léta tridcátého sedmého Joasa krále Judského kraloval Joas syn Joachazuv nad Izraelem v Samarí šestnácte let.
\par 11 A cinil to, což jest zlého pred ocima Hospodinovýma, neuchýliv se od žádných hríchu Jeroboáma syna Nebatova, kterýž privedl k hrešení Izraele, ale chodil v nich.
\par 12 O jiných pak cinech Joasových, i cožkoli cinil, i o síle jeho, kterouž bojoval proti Amaziášovi králi Judskému, zapsáno jest v knize o králích Izraelských.
\par 13 I usnul Joas s otci svými, a sedl Jeroboám na stolici jeho. I pochován jest Joas v Samarí s králi Izraelskými.
\par 14 Elizeus pak roznemohl se nemocí težkou, v kteréž i umrel. A prišel byl k nemu Joas král Izraelský, a pláce nad ním, rekl: Otce muj, otce muj, vozové Izraelští a jízdo jeho!
\par 15 Ale Elizeus rekl jemu: Vezmi lucište a strely. I vzav, prinesl k nemu lucište a strely.
\par 16 Rekl dále králi Izraelskému: Vezmi v ruku svou lucište. I vzal je v ruku svou. Vložil také Elizeus ruce své na ruce královy.
\par 17 A rekl: Otevri to okno k východu. A když otevrel, rekl Elizeus: Streliž. I strelil. Tedy rekl: Strela spasení Hospodinova a strela vysvobození proti Syrským; nebo porazíš Syrské v Afeku, až i do konce vyhladíš je.
\par 18 Opet rekl: Vezmi strely. I vzal. Tedy rekl králi Izraelskému: Strílej k zemi. I strelil po trikrát, potom tak nechal.
\par 19 Procež rozhnevav se na nej muž Boží, rekl: Mels petkrát neb šestkrát streliti, ješto bys byl porazil Syrské, ažbys je i do konce byl vyhladil; nyní pak jen po trikrát porazíš Syrské.
\par 20 Potom umrel Elizeus, a pochovali ho. Lotríkové pak Moábští vtrhli do zeme nastávajícího roku.
\par 21 I stalo se, když pochovávali jednoho, že uzrevše ty lotríky, uvrhli muže toho do hrobu Elizeova. Kterýžto muž, jakž tam padl, a dotekl se kostí Elizeových, ožil a vstal na nohy své.
\par 22 Hazael pak král Syrský, ssužoval Izraele po všecky dny Joachaza.
\par 23 A milostiv jsa jim Hospodin, slitoval se nad nimi, a popatril na ne, pro smlouvu svou s Abrahamem, s Izákem a s Jákobem, a nechtel jich zahladiti, aniž zavrhl jich od tvári své až do tohoto casu.
\par 24 I umrel Hazael král Syrský, a kraloval Benadad syn jeho místo neho.
\par 25 Protož Joas syn Joachazuv pobral zase mesta z ruky Benadada syna Hazaelova, kteráž byl vzal z ruky Joachaza otce jeho válecne; nebo po trikrát porazil ho Joas, a navrátil mesta Izraelská.

\chapter{14}

\par 1 Léta druhého Joasa syna Joachaza, krále Izraelského, kraloval Amaziáš syn Joasa, krále Judského.
\par 2 V petmecítma letech byl, když pocal kralovati, a kraloval dvadceti devet let v Jeruzaléme. Jméno matky jeho Joadan z Jeruzaléma.
\par 3 Ten cinil to, což dobrého jest pred ocima Hospodinovýma, ackoli ne tak jako David otec jeho. Všecko, což cinil Joas otec jeho, tak cinil.
\par 4 A však výsostí nezkazili, ješte lid obetoval a kadil na výsostech.
\par 5 I stalo se, když upevneno bylo království v ruce jeho, že pobil služebníky své, kteríž byli zabili krále otce jeho.
\par 6 Synu pak tech vražedlníku nepobil, jakož jest psáno v knize zákona Mojžíšova, kdež prikázal Hospodin, rka: Nebudout na hrdle trestáni otcové za syny, aniž synové trestáni budou na hrdle za otce, ale jeden každý za svuj hrích umre.
\par 7 On také porazil Idumejských v údolí solnatém deset tisícu, a dobyl Sela bojem. I nazval jméno jeho Jektehel až do tohoto dne.
\par 8 Tedy poslal Amaziáš posly k Joasovi synu Joachaza, syna Jéhu krále Izraelského, rka: Nuže, pohledme sobe v oci.
\par 9 Zase poslal Joas král Izraelský k Amaziášovi králi Judskému a rekl: Bodlák, kterýž byl na Libánu, poslal k cedru Libánskému, rka: Dej dceru svou synu mému za manželku. V tom šlo tudy zvíre polní, kteréž bylo na Libánu, a pošlapalo ten bodlák.
\par 10 Že jsi mocne porazil Idumejské, protož pozdvihlo te srdce tvé. Chlub se a sed v dome svém. I proc se máš plésti v neštestí, abys padl ty i Juda s tebou?
\par 11 Ale Amaziáš neuposlechl. Protož vytáhl Joas král Izraelský, a pohledeli sobe v oci, on s Amaziášem králem Judským u Betsemes Judova.
\par 12 I poražen jest Juda od Izraele, a utíkali jeden každý do príbytku svých.
\par 13 Amaziáše pak krále Judského, syna Joasa syna Ochoziášova, jal Joas král Izraelský u Betsemes, a pritáh do Jeruzaléma, zboril zed Jeruzalémskou, od brány Efraim až k bráne úhlu, na ctyri sta loktu.
\par 14 A pobral všecko zlato i stríbro, a všecky nádoby, kteréž se nalézaly v dome Hospodinove a v pokladích domu královského, také i mládence zastavené, a navrátil se do Samarí.
\par 15 O jiných pak cinech Joasa, kteréž cinil, i síle jeho, a kterak bojoval proti Amaziášovi králi Judskému, psáno jest v knize o králích Izraelských.
\par 16 I usnul Joas s otci svými, a pochován jest v Samarí s králi Izraelskými, i kraloval Jeroboám syn jeho místo neho.
\par 17 Živ pak byl Amaziáš syn Joasuv, král Judský, po smrti Joasa syna Joachaza, krále Izraelského, patnácte let.
\par 18 O jiných pak vecech Amaziášových psáno jest v knize o králích Judských.
\par 19 Potom spuntovali se proti nemu v Jeruzaléme, a on utekl do Lachis. Protož poslali za ním do Lachis, a zamordovali ho tam.
\par 20 Odkudž odnesli jej na koních, a pochován jest v Jeruzaléme s otci svými v meste Davidove.
\par 21 Tedy všecken lid Judský vzali Azariáše, kterýž byl v šestnácti letech, a ustanovili ho za krále na míste otce jeho Amaziáše.
\par 22 Ont jest vzdelal Elat a dobyl ho Judovi, když byl již usnul král s otci svými.
\par 23 Léta patnáctého Amaziáše syna Joasa, krále Judského, kraloval Jeroboám syn Joasa, krále Izraelského, v Samarí ctyridceti a jedno léto.
\par 24 A cinil to, což jest zlého pred ocima Hospodinovýma, neuchýliv se od žádných hríchu Jeroboáma syna Nebatova, kterýž k hrešení privedl Izraele.
\par 25 On zase dobyl koncin Izraelských od toho místa, kudy se jde do Emat, až k mori pustému, vedlé reci Hospodina Boha Izraelského, kterouž byl mluvil skrze služebníka svého Jonáše syna Amaty proroka, kterýž byl z Gethefer.
\par 26 Nebo videl Hospodin trápení Izraele težké náramne, a že jakož zajatý, tak i zanechaný s nic býti nemuže, aniž jest, kdo by spomohl Izraelovi.
\par 27 A nemluvil Hospodin, že by chtel zahladiti jméno Izraele, aby ho nebylo pod nebem; protož vysvobodil je skrze Jeroboáma syna Joasova.
\par 28 O jiných pak vecech Jeroboámových, a cožkoli cinil, i o síle jeho, a kterak bojoval, jak zase dobyl Damašku a Emat Judova Izraelovi, psáno jest v knize o králích Izraelských.
\par 29 I usnul Jeroboám s otci svými, s králi Izraelskými, a kraloval Zachariáš syn jeho místo neho.

\chapter{15}

\par 1 Léta sedmmecítmého Jeroboáma krále Izraelského kraloval Azariáš syn Amaziáše, krále Judského.
\par 2 V šestnácti letech byl, když pocal kralovati, a kraloval padesáte dve léte v Jeruzaléme. Jméno matky jeho bylo Jecholia z Jeruzaléma.
\par 3 Ten cinil, což dobrého jest pred ocima Hospodinovýma vedlé všech vecí, kteréž cinil Amaziáš otec jeho.
\par 4 A však výsostí nezkazili, ješte lid obetoval a kadil na výsostech.
\par 5 Ranil pak Hospodin krále, tak že byl malomocný až do dne smrti své, a bydlil v dome obzvláštním. Procež Jotam syn královský držel správu nad domem, soude lid zeme.
\par 6 O jiných pak cinech Azariášových, i cožkoli cinil, psáno jest v knize o králích Judských.
\par 7 I usnul Azariáš s otci svými, a pochovali jej s otci jeho v meste Davidove, a kraloval Jotam syn jeho místo neho.
\par 8 Léta tridcátého osmého Azariáše krále Judského kraloval Zachariáš syn Jeroboamuv nad Izraelem v Samarí šest mesícu.
\par 9 A cinil to, což jest zlého pred ocima Hospodinovýma, jakž cinívali otcové jeho, neuchýliv se od hríchu Jeroboáma syna Nebatova, kterýž privedl k hrešení Izraele.
\par 10 I spikl se proti nemu Sallum syn Jábes a bil ho pred lidem, i zabil jej a kraloval místo neho.
\par 11 O jiných pak vecech Zachariášových zapsáno jest v knize o králích Izraelských.
\par 12 Tot jest slovo Hospodinovo, kteréž mluvil k Jéhu, rka: Synové tvoji do ctvrtého kolena sedeti budou na stolici Izraelské. A tak se stalo.
\par 13 Sallum syn Jábesuv kraloval léta tridcátého devátého Uziáše krále Judského, a kraloval jeden mesíc v Samarí.
\par 14 Nebo pritáhl Manahem syn Gádi z Tersa, a prišed do Samarí, porazil Sallum syna Jábes v Samarí, a zabiv jej, kraloval na míste jeho.
\par 15 O jiných pak cinech Sallum i spiknutí jeho, kteréž ucinil, zapsáno jest v knize o králích Izraelských.
\par 16 Tehdy pohubil Manahem mesto Tipsach, a všecky, kteríž byli v nem, i všecky konciny jeho od Tersa, proto že mu ho neotevreli. I pomordoval je, a všecky tehotné jejich zroztínal.
\par 17 Léta tridcátého devátého Azariáše krále Judského kraloval Manahem syn Gádi nad Izraelem v Samarí za deset let.
\par 18 A cinil to, což jest zlého pred ocima Hospodinovýma, neodvrátiv se od hríchu Jeroboáma syna Nebatova, (kterýž privedl Izraele k hrešení), po všecky dny své.
\par 19 Když pak vytáhl Ful král Assyrský proti zemi té, dal Manahem Fulovi deset tisíc centnéru stríbra, aby mu byl nápomocen k utvrzení království v rukou jeho.
\par 20 I uložil Manahem dan na Izraele, na všecky bohaté, aby dávali králi Assyrskému jeden každý po padesáti lotech stríbra. Takž obrátiv se král Assyrský, nemeškal se více tu v zemi.
\par 21 O jiných pak cinech Manahemových, a cožkoli cinil, zapsáno jest v knize o králích Izraelských.
\par 22 I usnul Manahem s otci svými, a kraloval Pekachia syn jeho místo neho.
\par 23 Léta padesátého Azariáše, krále Judského, kraloval Pekachia syn Manahemuv nad Izraelem v Samarí dve léte.
\par 24 A cinil to, což jest zlého pred ocima Hospodinovýma, neuchýliv se od hríchu Jeroboáma syna Nebatova, kterýž k hrešení privedl Izraele.
\par 25 Tedy spuntoval se proti nemu Pekach syn Romeliuv, hejtman jeho, s Argobem i s Ariášem, a zabil jej v Samarí na paláci domu královského, maje s sebou padesáte mužu Galádských. A zabiv jej, kraloval místo neho.
\par 26 Jiní pak skutkové Pekachia a všecko, což cinil, zapsáno jest v knize o králích Izraelských.
\par 27 Léta padesátého druhého Azariáše krále Judského kraloval Pekach syn Romeliuv nad Izraelem v Samarí dvadceti let.
\par 28 A cinil to, což jest zlého pred ocima Hospodinovýma, aniž odstoupil od hríchu Jeroboáma syna Nebatova, kterýž k hrešení privedl Izraele.
\par 29 Za dnu Pekacha krále Izraelského pritáhl Tiglatfalazar král Assyrský, a vzal Jon a Abel, dum Maacha a Janoe, Kedes a Azor, i Galád a Galilei, i všecku zemi Neftalím, a prenesl obyvatele jejich do Assyrie.
\par 30 Tedy spuntoval se Ozee syn Ela, proti Pekachovi synu Romeliovu, a raniv ho, zabil jej, a kraloval na míste jeho léta dvadcátého Jotama syna Uziášova.
\par 31 O jiných pak cinech Pekachových, a cožkoli cinil, zapsáno jest v knize o králích Izraelských.
\par 32 Léta druhého Pekacha syna Romelia, krále Izraelského, kraloval Jotam syn Uziáše, krále Judského.
\par 33 V petmecítma letech byl, když pocal kralovati, a šestnácte let kraloval v Jeruzaléme. Jméno matky jeho Jerusa, dcera Sádochova.
\par 34 A cinil to, což pravého jest pred ocima Hospodinovýma. Všecko, jakž cinil Uziáš otec jeho, tak cinil.
\par 35 A však výsosti nebyly zkaženy, ješte lid obetoval a kadil na výsostech. On vystavel bránu horejší domu Hospodinova.
\par 36 O jiných pak cinech Jotamových, a cožkoli cinil, zapsáno jest v knize o králích Judských.
\par 37 Za dnu tech pocal Hospodin posílati na Judu Rezina krále Syrského a Pekacha syna Romeliova.
\par 38 I usnul Jotam s otci svými, a pochován jest s otci svými v meste Davida otce svého. I kraloval Achas syn jeho místo neho.

\chapter{16}

\par 1 Sedmnáctého léta Pekacha syna Romeliova kraloval Achas syn Jotama, krále Judského.
\par 2 Ve dvadcíti letech byl Achas, když kralovati pocal, a šestnácte let kraloval v Jeruzaléme, ale necinil toho, což pravého jest, pred Hospodinem Bohem svým, jako David otec jeho.
\par 3 Nýbrž chodil po ceste králu Izraelských; nadto i syna svého dal provésti skrze ohen vedlé ohavností pohanských, kteréž byl Hospodin vyplénil pred oblícejem synu Izraelských.
\par 4 Obetoval také a kadil na výsostech a na pahrbcích, i pod každým stromem zeleným.
\par 5 Tedy vytáhl Rezin král Syrský a Pekach syn Romeliášuv, král Izraelský, proti Jeruzalému k boji, a oblehli Achasa. Ale nemohli ho dobyti.
\par 6 (Toho casu Rezin král Syrský odtrhl zase mesto Elat k Syrii, a vyplénil Židy z Elot; Syrští pak prišedše do Elat, bydlili tam až do dnešního dne.)
\par 7 I poslal Achas posly k Tiglatfalazarovi králi Assyrskému, rka: Služebník tvuj a syn tvuj jsem, pritáhni a vysvobod mne z ruky krále Syrského, a z ruky krále Izraelského, kteríž povstali proti mne.
\par 8 A pobrav Achas stríbro a zlato, kteréž se nalézti mohlo v dome Hospodinove a v pokladích domu královského, poslal králi Assyrskému dar.
\par 9 Tedy povolil jemu král Assyrský, a pritáhl k Damašku a dobyl ho, a prenesl obyvatele jeho do Kir, Rezina pak zabil.
\par 10 I vypravil se král Achas vstríc Tiglatfalazarovi králi Assyrskému do Damašku. A uzrev král Achas oltár v Damašku, poslal k Uriášovi knezi podobenství toho oltáre, a formu jeho vedlé všelikého díla jeho.
\par 11 I vzdelal Uriáš knez oltár vedlé všeho toho, což byl poslal král Achas z Damašku. Tak ucinil knez Uriáš, prvé než se vrátil král Achas z Damašku.
\par 12 Když se pak navrátil král z Damašku, uzrev ten oltár pristoupil k nemu a obetoval na nem.
\par 13 A tak zapálil zápal svuj i suchou obet svou, a obetoval obet mokrou svou, a pokropil krví oltáre z pokojných obetí svých.
\par 14 Oltár pak medený, kterýž byl pred Hospodinem, prenesl z prední strany domu, aby nestál mezi oltárem jeho a mezi domem Hospodinovým, a postavil jej po boku oltáre svého k pulnoci.
\par 15 I prikázal král Achas knezi Uriášovi, rka: Na vetším oltári obetuj zápaly jitrní, a obet suchou vecerní, a obet zápalnou královskou s obetí suchou její, i obet zápalnou všeho lidu zeme, a obeti suché jejich, i obeti mokré jejich, a všelikou krví zápalu a všelikou krví obeti kropiti budeš na nej, oltár pak medený bude mi k doptávání se Boha.
\par 16 Tedy ucinil Uriáš knez všecko, jakž mu prikázal Achas.
\par 17 Osekal také král Achas prepásaní podstavku, a odjal od nich pánve, a more složil s volu medených, kteríž byli pod ním, a položil je na dlážení kamenné.
\par 18 Zastrení také sobotní, kteréž byli udelali v dome, a vcházení královské z zevnitr odjal od domu Hospodinova, boje se krále Assyrského.
\par 19 Jiní pak skutkové krále Achasa, kteréž cinil, zapsáni jsou v knize o králích Judských.
\par 20 I usnul Achas s otci svými, a pochován jest s nimi v meste Davidove, a kraloval Ezechiáš syn jeho místo neho.

\chapter{17}

\par 1 Léta dvanáctého Achasa krále Judského kraloval Ozee syn Ela v Samarí nad Izraelem devet let.
\par 2 A cinil to, což jest zlého pred ocima Hospodinovýma, ac ne tak jako jiní králové Izraelští, kteríž byli pred ním.
\par 3 Proti nemuž pritáhl Salmanazar král Assyrský. I ucinen jest Ozee služebníkem jeho, a dával mu plat.
\par 4 Shledal pak král Assyrský pri Ozee puntování; nebo poslal posly k Sua králi Egyptskému, a neposílal králi Assyrskému rocního platu. Protož oblehl ho král Assyrský, a svázaného dal do žaláre.
\par 5 I táhl král Assyrský po vší zemi; pritáhl také do Samarí, a ležel u ní tri léta.
\par 6 Léta pak devátého Ozee vzal král Assyrský Samarí, a prenesl Izraele do Assyrie, a osadil je v Chelach a v Chabor pri rece Gozan, a v mestech Médských.
\par 7 To se stalo, proto že hrešili synové Izraelští proti Hospodinu Bohu svému, kterýž je vyvedl z zeme Egyptské, aby nebyli pod mocí Faraona krále Egyptského, a ctili bohy cizí,
\par 8 Chodíce v ustanoveních pohanu, (kteréž byl vyvrhl Hospodin od tvári synu Izraelských), a králu Izraelských, kteráž narídili.
\par 9 Presto pokryte se meli synové Izraelští, ciníce to, což není dobrého pred Hospodinem Bohem svým, a vzdelali sobe výsosti ve všech mestech svých, od veže strážných až do mesta hrazeného.
\par 10 A nastaveli sobe obrazu i háju na všelikém pahrbku vysokém, a pod každým stromem zeleným,
\par 11 Zapalujíce tam vonné veci na všech výsostech, rovne jako pohané, kteréž vyhnal Hospodin od tvári jejich. A cinili veci nejhorší, popouzejíce Hospodina.
\par 12 A sloužili bohum necistým, o nichž byl Hospodin rekl jim, aby necinili toho.
\par 13 I osvedcoval Hospodin proti Izraelovi a proti Judovi skrze všecky proroky i všecky vidoucí, rka: Odvratte se od cest svých zlých, a ostríhejte prikázaní mých a ustanovení mých vedlé všeho zákona, kterýž jsem vydal otcum vašim, a kterýž jsem poslal k vám skrze služebníky své proroky.
\par 14 Však neposlechli, ale zatvrdili šíji svou, jako i otcové jejich šíje své, kteríž neverili v Hospodina Boha svého.
\par 15 A opovrhli ustanovení jeho i smlouvu jeho, kterouž ucinil s otci jejich, i osvedcování jeho, kterýmiž se jim osvedcoval, a odešli po marnosti, a marní ucineni jsou, následujíce pohanu, kteríž byli vukol nich, o nichž jim byl prikázal Hospodin, aby necinili jako oni.
\par 16 A opustivše všecka prikázaní Hospodina Boha svého, udelali sobe slitinu, dvé telat a háj, a klaneli se všemu vojsku nebeskému; sloužili také Bálovi.
\par 17 A vodili syny své a dcery své skrze ohen, a obírali se s hádáním a kouzly, a tak vydali se v cinení toho, což jest zlé pred ocima Hospodinovýma, popouzejíce ho.
\par 18 Protož rozhneval se Hospodin náramne na Izraele, a zahnal je od tvári své, nezanechav z nich nic, krome samého pokolení Judova.
\par 19 Ano i Juda neostríhal prikázaní Hospodina Boha svého, ale chodili v ustanoveních Izraelských, kteráž byli narídili.
\par 20 A protož opovrhl Hospodin všecko síme Izraelské a ssužoval je, a vydal je v ruku loupežníku, až je i zavrhl od tvári své.
\par 21 Nebo odtrhl Izraele od domu Davidova, a krále ustanovili Jeroboáma syna Nebatova, Jeroboám pak odvrátil Izraele od následování Hospodina, a privedl je k hrešení hríchem velikým.
\par 22 A chodili synové Izraelští ve všech hríších Jeroboámových, kteréž on cinil, a neodstoupili od nich,
\par 23 Až zavrhl Hospodin Izraele od tvári své, jakož byl mluvil skrze všecky služebníky své proroky. I vyhnán jest Izrael z zeme své do Assyrie až do tohoto dne.
\par 24 Potom král Assyrský privedl lidi z Babylona, a z Kut a z Ava, a z Emat a z Sefarvaim, a osadil je v mestech Samarských místo synu Izraelských, kteríž opanovavše Samarí, bydlili v mestech jejich.
\par 25 I stalo se, když tam bydliti pocali, a nesloužili Hospodinu, že poslal na ne Hospodin lvy, kteríž je dávili.
\par 26 Protož mluvili králi Assyrskému, rkouce: Národové ti, kteréž jsi prenesl a osadil v mestech Samarských, neznají obyceje Boha zeme té; protož poslal na ne lvy, kteríž je hubí, proto že neznají obyceje Boha zeme té.
\par 27 I prikázal král Assyrský, rka: Dovedte tam jednoho z tech kneží, kteréž jste odtud privedli, a odejdouce, necht tam bydlí, a ucí je obyceji Boha zeme té.
\par 28 Prišel tedy jeden z kneží, kteréž byli privedli z Samarí, a bydlil v Bethel, a ucil je, jak by meli sloužiti Hospodinu.
\par 29 A však nadelali sobe jeden každý národ bohu svých, kteréž staveli v dome výsostí, jichž byli nadelali Samarští, jeden každý národ v mestech svých, v nichž bydlili.
\par 30 Muži zajisté Babylonští udelali Sukkot Benot, muži pak Kutští udelali Nergale, a muži Ematští udelali Asima.
\par 31 Hevejští tolikéž udelali Nibchaz a Tartak, a Sefarvaimští pálili syny své ohnem Adramelechovi a Anamelechovi, bohum Sefarvaimským.
\par 32 A tak sloužili Hospodinu, nadelavše sobe z poctu svého kneží výsostí, kteríž prisluhovali jim v domích výsostí.
\par 33 Hospodina ctili, však predce bohum svým sloužili vedlé obyceje tech národu, odkudž prevedeni byli.
\par 34 A do dnes ciní vedlé obyceju starých; nebojí se Hospodina, a neciní vedlé ustanovení a úsudku jeho, ani vedlé zákona a prikázaní, kteráž vydal Hospodin synum Jákobovým, jehož nazval Izraelem.
\par 35 Ucinil také byl Hospodin s nimi smlouvu, a prikázal jim, rka: Nebudete ctíti bohu cizích, ani se jim klaneti, ani jim sloužiti, ani jim obetovati.
\par 36 Ale Hospodina, kterýž vás vyvedl z zeme Egyptské v síle veliké a v rameni vztaženém, toho ctíti budete, a jemu se klaneti, a jemu obetovati.
\par 37 Také ustanovení a úsudku, i zákona a prikázaní, kteráž napsal vám, ostríhati budete, plníce je po všecky dny, a nebudete ctíti bohu cizích.
\par 38 Nadto na smlouvu, kterouž jsem ucinil s vámi, nezapomínejte, aniž ctete bohu cizích.
\par 39 Ale Hospodina Boha vašeho se bojte; ont vás vysvobodí z ruky všech neprátel vašich.
\par 40 A však neposlechli, ale vedlé obyceje svého starého cinili.
\par 41 A tak ti národové ctili Hospodina, a však proto rytinám svým sloužili. Takž i synové jejich a synové synu jejich vedlé toho, což cinili otcové jejich, také ciní až do dnešního dne.

\chapter{18}

\par 1 Stalo se pak léta tretího Ozee syna Ela, krále Izraelského, kraloval Ezechiáš syn Achasa, krále Judského.
\par 2 V petmecítma letech byl, když pocal kralovati, a dvadceti devet let kraloval v Jeruzaléme. Jméno matky jeho bylo Abi, dcera Zachariášova.
\par 3 Ten cinil, což jest dobrého pred ocima Hospodinovýma, všecko tak, jakž cinil David otec jeho.
\par 4 On zkazil výsosti a stroskotal obrazy, a háje posekal, a ztloukl hada medeného, jehož byl udelal Mojžíš; nebo až do tech dnu synové Izraelští kadívali jemu, a nazval jej Nechustam.
\par 5 V Hospodinu Bohu Izraelském doufal, a nebylo po nem jemu rovného mezi všemi králi Judskými, i z tech, kteríž byli pred ním.
\par 6 Nebo se prídržel Hospodina, aniž se uchýlil od neho, a ostríhal prikázaní jeho,kteráž byl prikázal Hospodin Mojžíšovi.
\par 7 A byl Hospodin s ním; k cemuž se koli obrátil, štastne se mu vedlo. Zprotivil se pak králi Assyrskému, a nesloužil jemu.
\par 8 On porazil Filistinské až do Gázy a koncin jeho, od veže strážných až do mesta hrazeného.
\par 9 Stalo se pak léta ctvrtého krále Ezechiáše, (jenž byl sedmý rok Ozee syna Ela, krále Izraelského), vytáhl Salmanazar král Assyrský proti Samarí a oblehl ji.
\par 10 I vzali ji pri konci léta tretího; léta šestého Ezechiášova, (jenž byl rok devátý Ozee krále Izraelského), vzata jest Samarí.
\par 11 I zavedl král Assyrský Izraele do Assyrie, a osadil jej v Chelach a v Chabor pri potoku Gozan, a v mestech Médských,
\par 12 Proto že neposlouchali hlasu Hospodina Boha svého, ale prestupovali smlouvu jeho, i všecko to, což prikázal Mojžíš služebník Hospodinuv, tak že ani poslechnouti ani ciniti nechteli.
\par 13 Potom ctrnáctého léta krále Ezechiáše pritáhl Senacherib král Assyrský proti všechnem mestum Judským hrazeným, a zdobýval jich.
\par 14 Tedy poslal Ezechiáš král Judský k králi Assyrskému do Lachis, rka: Zhrešilt jsem, odtáhni ode mne; cožkoli na mne vzložíš, ponesu. I uložil král Assyrský Ezechiášovi králi Judskému tri sta centnéru stríbra a tridceti centnéru zlata.
\par 15 I dal Ezechiáš všecky peníze, kteréž jsou nalezeny v dome Hospodinove a v pokladích domu královského.
\par 16 Toho casu obloupal Ezechiáš dvére chrámu Hospodinova, i sloupy, kteréž byl obložil Ezechiáš král Judský, a dal je králi Assyrskému.
\par 17 A však poslal král Assyrský Tartana a Rabsara a Rabsace z Lachis k králi Ezechiášovi s vojskem velikým k Jeruzalému. Kteríž vytáhše, prijeli k Jeruzalému, a pritáhše, pritrhli a položili se u struhy rybníka horejšího, kteráž jest podlé cesty dlážené pri poli valchárovu.
\par 18 A když volali na krále, vyšel k nim Eliakim syn Helkiášuv, kterýž byl správce domu, a Sobna písar, a Joach syn Azafuv kanclér.
\par 19 I mluvil k nim Rabsace: Povezte medle Ezechiášovi: Toto praví král veliký, král Assyrský: Jakéž jest to doufání, na kterémž se zakládáš?
\par 20 Mluvils, (ale slova rtu svých), že jest rady i síly k válce dosti. V kohož tedy nyní doufáš, že mi se protivíš?
\par 21 Aj, nyní zpolehl jsi na hul trtinovou, a to ješte nalomenou, na Egypt, na niž zpodeprel-li by se kdo, pronikne ruku jeho a probodne ji. Takovýt jest Farao král Egyptský všechnem, kteríž doufají v neho.
\par 22 Pakli mi díte: V Hospodina Boha svého doufáme: zdaliž on není ten, jehož Ezechiáš poboril výsosti i oltáre, a prikázal Judovi i Jeruzalému, rka: Pred tímto oltárem klaneti se budete v Jeruzaléme.
\par 23 Nyní tedy, založ se medle se pánem mým, králem Assyrským, a dámt dva tisíce koní, budeš-li moci míti, kdo by na nich jeli?
\par 24 I jakž bys tedy odolati mohl jednomu knížeti z nejmenších služebníku pána mého, ackoli máš doufání v Egyptu pro vozy a jezdce?
\par 25 Presto, zdali bez Hospodina pritáhl jsem proti místu tomuto, abych je zkazil? Hospodin rekl mi: Táhni proti zemi té a zkaz ji.
\par 26 I rekl Eliakim syn Helkiášuv a Sobna a Joach Rabsacovi: Mluv medle k služebníkum svým Syrsky, však rozumíme, a nemluv k nám Židovsky pred lidem tímto, kterýž jest na zdech.
\par 27 I odpovedel jim Rabsace: Zdaliž ku pánu tvému a k tobe poslal mne pán muj, abych mluvil slova tato? Však k mužum tem, kteríž jsou na zdech, aby lejna svá jedli, a moc svuj spolu s vámi pili.
\par 28 A tak stoje Rabsace , volal hlasem velikým, Židovsky mluve a rka: Slyšte slovo krále velikého, krále Assyrského.
\par 29 Toto praví král: Necht vás nesvodí Ezechiáš, nebot nebude moci vás vysvoboditi z ruky mé.
\par 30 A necht vám nevelí Ezechiáš doufati v Hospodina, rka: Zajisté vysvobodí nás Hospodin, a nebudet dáno mesto toto v ruku krále Assyrského.
\par 31 Neposlouchejtež Ezechiáše, nebo takto praví král Assyrský: Ucinte se mnou smlouvu, a vyjdete ke mne, a jezte jeden každý z vinice své a z fíku svého, a píte jeden každý vodu z cisterny své,
\par 32 Dokudž neprijdu a neprenesu vás do zeme podobné zemi vaší, do zeme úrodné, zeme chleba a vinic, zeme olivoví, oleje a medu, a budete živi a nezemrete. Neposlouchejtež Ezechiáše, nebot vás svodí, rka: Hospodin vysvobodí nás.
\par 33 Zdaliž mohli vysvoboditi bohové národu jeden každý zemi svou z ruky krále Assyrského?
\par 34 Kdež jsou bohové Emat a Arfad? Kde jsou bohové Sefarvaim, Ana a Ava? Zdaliž jsou vysvobodili Samarí z ruky mé?
\par 35 Kterí jsou mezi všemi bohy tech zemí, ješto by vysvobodili zemi svou z ruky mé? Aby pak Hospodin vysvoboditi mohl Jeruzalém z ruky mé?
\par 36 Lid pak mlcel, a neodpovedel mu žádného slova; nebo takové bylo rozkázaní královo, rkoucí: Neodpovídejte jemu.
\par 37 Prišli tedy Eliakim syn Helkie, kterýž byl správcím domu, a Sobna písar, a Joach syn Azafuv kanclér k Ezechiášovi, majíce roucho roztržené, a oznámili jemu slova Rabsacova.

\chapter{19}

\par 1 To když uslyšel král Ezechiáš, roztrhl roucho své, a odev se žíní, všel do domu Hospodinova.
\par 2 I poslal Eliakima správce domu, a Sobnu písare, i starší z kneží, oblecené v žíne, k Izaiášovi proroku, synu Amosovu.
\par 3 Kteríž rekli jemu: Toto praví Ezechiáš: Den úzkosti a útržek i rouhání jest den tento; priblížilt se plod k vyjití, ale není síly pri rodicce.
\par 4 Ó by slyšel Hospodin Buh tvuj všecka slova Rabsacova, jehož poslal král Assyrský pán jeho, aby utrhal Bohu živému, aby pomstil Hospodin Buh tvuj tech slov, kteráž by slyšel. Protož pozdvihni modlitby své za tento ostatek, kterýž se nalézá.
\par 5 A tak prišli služebníci krále Ezechiáše k Izaiášovi.
\par 6 Jimž odpovedel Izaiáš: Toto povíte pánu svému: Takto praví Hospodin: Nestrachuj se slov tech, kteráž jsi slyšel, jimiž se mi rouhali služebníci krále Assyrského.
\par 7 Aj, já pustím nan vítr, aby uslyše povest, navrátil se do zeme své, a uciním to, že padne od mece v zemi své.
\par 8 Rabsace pak navrátiv se, nalezl krále Assyrského, an dobývá Lebna; nebo uslyšel, (procež odtrhl od Lachis),
\par 9 Uslyšel, pravím, o Tirhákovi králi Mourenínském, ano pravili: Aj, ted táhne, aby bojoval s tebou. I navrátil se, a však poslal jiné posly k Ezechiášovi s temito slovy:
\par 10 Takto povíte Ezechiášovi králi Judskému, rkouce: Necht tebe nesvodí Buh tvuj, v nemž ty doufáš, ríkaje: Nebudet dán Jeruzalém v ruku krále Assyrského.
\par 11 Aj, slyšels, co jsou cinili králové Assyrští všechnem zemím, pohubivše je, a ty bys mel býti vysvobozen?
\par 12 Zdaliž jsou je vysvobodili bohové tech národu, kteréž zahladili otcové moji, totiž Gozana, Charana, Resefa a syny Eden, kteríž byli v Telasar?
\par 13 Kde jest král Emat, a král Arfad, a král mesta Sefarvaim, Ana i Ava?
\par 14 Protož vzav Ezechiáš list z ruky poslu, precetl jej, a vstoupiv do domu Hospodinova, rozvinul jej Ezechiáš pred Hospodinem.
\par 15 A modlil se Ezechiáš pred Hospodinem, rka: Hospodine Bože Izraelský, kterýž sedíš nad cherubíny, ty jsi sám Buh všech království zeme, ty jsi ucinil nebe i zemi.
\par 16 Nakloniž, Hospodine, ucha svého a uslyš; otevri, Hospodine, oci své a pohled; slyš slova Senacheribova, kterýž poslal k cinení útržek Bohu živému.
\par 17 Takt jest, Hospodine, žet jsou zkazili králové Assyrští národy ty i zeme jejich.
\par 18 A uvrhli bohy jejich do ohne. Nebo nebyli bohové, ale dílo rukou lidských, drevo a kámen, protož zahladili je.
\par 19 A nyní, Hospodine Bože náš, vysvobod nás, prosím, z ruky jeho, at by poznala všecka království zeme, že jsi ty sám, Hospodine, Bohem.
\par 20 Tedy poslal Izaiáš syn Amosuv k Ezechiášovi, rka: Toto praví Hospodin Buh Izraelský: Zac jsi mi se modlil strany Senacheriba krále Assyrského, vyslyšel jsem te.
\par 21 Totot jest slovo, kteréž mluvil Hospodin o nem: Pohrdá tebou, a posmívá se tobe, králi, panna dcera Sionská, potrásá za tebou hlavou dcera Jeruzalémská.
\par 22 Kohož jsi zhanel? A komus se rouhal? Proti komu jsi povýšil hlasu a pozdvihls vzhuru ocí svých? Však proti svatému Izraelskému.
\par 23 Skrze posly své utrhal jsi Pánu, a rekl jsi: Ve množství vozu svých vytáhl jsem na hory vysoké, na stráne Libánské, a zpodtínám vysoké cedry jeho, i spanilé jedle jeho, a vejdu do nejdalších príbytku jeho, do lesu a výborných rolí jeho.
\par 24 Já jsem vykopal a pil jsem vody cizí, a vysušil jsem nohama svýma všecky potoky podmanených.
\par 25 Zdaliž jsi neslyšel, že již dávno jsem jej ucinil, a ode dnu starých jej sformoval? Což tedy nyní privedl bych jej k zkažení a v hromady rumu, jako jiná mesta hrazená?
\par 26 Jejichž obyvatelé mdlí byli, predešení a zahanbení, byvše jako bylina polní a zelina vzcházející, jako tráva na strechách, a jako osení rzí zkažené, prvé než by dorostlo obilí.
\par 27 Sedání pak tvé, vycházení tvé i vcházení tvé znám, i vzteklost tvou proti mne.
\par 28 Ponevadž jsi se rozzlobil proti mne, a tvé zpouzení prišlo v uši mé, protož vpustím udici svou v chrípe tvé, a udidla svá do úst tvých, a odvedu te zase tou cestou, kterouž jsi prišel.
\par 29 Toto pak mej, Ezechiáši, za znamení: Že jíte roku prvního to, což se samo rodí, též druhého roku, což samo vzchází, tretího teprv roku sejte a žnete, a štepujte vinice, a jezte ovoce z nich.
\par 30 Ostatek zajisté domu Judova, kterýž pozustal, vpustí zase koreny své hluboce, a vydá užitek nahoru.
\par 31 Nebo z Jeruzaléma vyjdou ostatkové, a ti, kteríž jsou zachováni, z hory Siona. Horlivost Hospodina zástupu uciní to.
\par 32 A protož toto praví Hospodin o králi Assyrském: Nevejdet do mesta tohoto, aniž sem strely vstrelí, aniž se ho zmocní pavézníci, aniž udelají u neho náspu.
\par 33 Cestou, kterouž pritáhl, zase navrátí se, a do mesta tohoto nevejde, praví Hospodin.
\par 34 Nebo chrániti budu mesta tohoto, abych je zachoval pro sebe a pro Davida služebníka svého.
\par 35 Tedy stalo se té noci, že vyšel andel Hospodinuv, a zbil v vojšte Assyrském sto osmdesáte pet tisícu. I vstali ráno, a aj, všickni mrtví.
\par 36 Procež odjel a utekl Senacherib král Assyrský, a navrátiv se, bydlil v Ninive.
\par 37 I stalo se, když se klanel v chráme Nizrocha boha svého, že Adramelech a Sarasar, synové jeho, zabili jej mecem, a sami utekli do zeme Ararat. I kraloval Esarchaddon syn jeho místo neho.

\chapter{20}

\par 1 V tech dnech roznemohl se Ezechiáš až k smrti. I prišel k nemu Izaiáš syn Amosuv prorok, a rekl jemu: Toto praví Hospodin: Zred dum svuj, nebo umreš, a nebudeš živ.
\par 2 I obrátil tvár svou k stene, a modlil se Hospodinu, rka:
\par 3 Prosím, ó Hospodine, rozpomen se nyní, že jsem stále chodil pred tebou v pravde a v srdci uprímém, a že jsem to cinil, což dobrého jest pred ocima tvýma. I plakal Ezechiáš plácem velikým.
\par 4 Ješte pak Izaiáš nebyl vyšel do pul síne, když se k nemu stalo slovo Hospodinovo, rkoucí:
\par 5 Navrat se a rci Ezechiášovi vudci lidu mého: Toto praví Hospodin Buh Davida otce tvého: Slyšelt jsem modlitbu tvou, a videl jsem slzy tvé; aj, já uzdravím te, tretího dne vstoupíš do domu Hospodinova.
\par 6 A pridám ke dnum tvým patnácte let, a z ruky krále Assyrského vysvobodím te, i mesto toto, a chrániti budu mesta tohoto, pro sebe a pro Davida služebníka svého.
\par 7 I rekl Izaiáš: Vezmete hrudu suchých fíku. Kterouž vzavše, priložili na vred, i uzdraven jest.
\par 8 Rekl pak Ezechiáš Izaiášovi: Jaké bude znamení toho, že mne uzdraví Hospodin, a že pujdu tretího dne do domu Hospodinova?
\par 9 Odpovedel Izaiáš: Toto bude tobe znamení od Hospodina, že Hospodin uciní vec tuto, kterouž mluvil: Chceš-li, aby postoupil dále stín o deset stupnu, aneb navrátil se zpátkem o deset stupnu?
\par 10 Odpovedel Ezechiáš: Snázet muže stín postoupiti dolu o deset stupnu. Nechci, ale necht zase postoupí stín zpátkem o deset stupnu.
\par 11 Volal tedy Izaiáš prorok k Hospodinu, a navrátil stín po stupních, po nichž sešel na hodinách slunecných Achasových, zpátkem o deset stupnu.
\par 12 Toho casu poslal Berodach Baladan syn Baladanuv, král Babylonský, list a dary Ezechiášovi; nebo slyšel, že nemocen byl Ezechiáš.
\par 13 I vyslyšel je Ezechiáš a ukázal jim všecky schrany klénotu svých, stríbra a zlata i vonných vecí, a olej nejvýbornejší, tolikéž dum zbroje své, a cožkoli mohlo nalezeno býti v pokladích jeho. Niceho nebylo, cehož by jim neukázal Ezechiáš v dome svém i ve všem panství svém.
\par 14 Protož prišel prorok Izaiáš k králi Ezechiášovi, a rekl jemu: Co pravili ti muži? A odkud prišli k tobe? Odpovedel Ezechiáš:Z zeme daleké prišli, z Babylona.
\par 15 I rekl: Co videli v dome tvém? Odpovedel Ezechiáš: Všecko, což jest v dome mém, videli. Niceho není v pokladích mých, cehož bych jim neukázal.
\par 16 Ale Izaiáš rekl Ezechiášovi: Slyšiž slovo Hospodinovo.
\par 17 Aj, dnové prijdou, v nichž odneseno bude do Babylona, cožkoli jest v dome tvém, a cožkoli nachovali otcové tvoji až do tohoto dne; nezustanet niceho, praví Hospodin.
\par 18 Syny také tvé, kteríž pojdou z tebe, kteréž ty zplodíš, poberou a budou komorníci pri dvoru krále Babylonského.
\par 19 Tedy rekl Ezechiáš Izaiášovi: Dobrét jest slovo Hospodinovo, kteréž jsi mluvil. Rekl ješte: Ovšem, žet jest dobré, jestliže pokoj a pravda bude za dnu mých.
\par 20 O jiných pak cinech Ezechiášových, i vší síle jeho, a kterak udelal rybník, a vodu po trubách uvedl do mesta, zapsáno jest v knize o králích Judských.
\par 21 I usnul Ezechiáš s otci svými, a kraloval Manasses syn jeho místo neho.

\chapter{21}

\par 1 Ve dvanácti letech byl Manasses, když pocal kralovati, a kraloval padesáte a pet let v Jeruzaléme. Jméno matky jeho bylo Chefziba.
\par 2 I cinil to, což jest zlého pred ocima Hospodinovýma, vedlé ohavností tech národu, kteréž Hospodin vyplénil pred syny Izraelskými.
\par 3 Nebo prevrátiv se, vzdelal zase výsosti, kteréž byl zkazil Ezechiáš otec jeho, a vystavel oltáre Bálovi, a vysadil háj, tak jako byl ucinil Achab král Izraelský. A klaneje se všemu vojsku nebeskému, ctil je.
\par 4 Vzdelal také oltáre v dome Hospodinove, o nemž byl rekl Hospodin: V Jeruzaléme položím jméno své.
\par 5 Nadto nadelal oltáru všemu vojsku nebeskému, v obou síních domu Hospodinova.
\par 6 Syna také svého provedl skrze ohen, a šetril casu, s hadacstvím zacházel, zaklinace a carodejníky zrídil, a velmi mnoho zlého cinil pred ocima Hospodinovýma, popouzeje ho.
\par 7 Postavil také rytinu háje, kterouž byl udelal v dome, o kterémž byl rekl Hospodin k Davidovi a Šalomounovi synu jeho: V dome tomto a v Jeruzaléme, kterýž jsem vyvolil ze všech pokolení Izraelských, položím jméno své na veky,
\par 8 Aniž více dopustím, aby se pohnula noha Izraele z zeme, kterouž jsem dal otcum jejich, jen toliko budou-li skutecne ostríhati všeho, což jsem jim prikázal, a všeho zákona, kterýž jim vydal Mojžíš služebník muj.
\par 9 Ale neuposlechli; nebo je svedl Manasses, tak že cinili horší veci, nežli ti národové, kteréž vyplénil Hospodin od tvári synu Izraelských,
\par 10 Ackoli mluvíval Hospodin skrze služebníky své proroky, rka:
\par 11 Proto že cinil Manasses král Judský ohavnosti tyto, a páchal horší veci nad všecko, což cinili Amorejští, kteríž byli pred ním, a že privedl i Judu k hrešení skrze ukydané bohy své,
\par 12 Protož rekl Hospodin Buh Izraelský: Aj, já uvedu zlé veci na Jeruzalém a na Judu, aby každému slyšícímu o tom znelo v obou uších jeho.
\par 13 Nebo vztáhnu na Jeruzalém šnuru Samarskou, a závaží domu Achabova, a vytru Jeruzalém, jako vytírá nekdo misku, a vytra, poklopí ji.
\par 14 A opustím ostatky dedictví svého, a vydám je v ruku neprátel jejich, i budou v loupež a v rozchvátání všechnem neprátelum svým,
\par 15 Proto že cinili to, což jest zlého pred ocima mýma, a popouzeli mne od toho dne, jakž vyšli otcové jejich z Egypta, až do dnešního dne.
\par 16 Ano i krve nevinné vylil Manasses velmi mnoho, tak že naplnil Jeruzalém od jednoho konce k druhému, krome hríchu svého, kterýmž privedl k hrešení Judu, aby cinili, což zlého jest pred ocima Hospodinovýma.
\par 17 O jiných pak cinech Manassesových, a cožkoli cinil, i hrích jeho, kterýmž zhrešil, zapsáno jest v knize o králích Judských.
\par 18 I usnul Manasses s otci svými, a pohrben jest v zahrade domu svého, v zahrade Uzy, a kraloval Amon syn jeho místo neho.
\par 19 Ve dvamecítma letech byl Amon, když pocal kralovati, a kraloval dve léte v Jeruzaléme. Jméno matky jeho bylo Mesullemet, dcera Charus z Jateba.
\par 20 I ten cinil to, což jest zlého pred ocima Hospodinovýma, jako cinil Manasses otec jeho.
\par 21 A chodil po vší ceste, po níž chodil otec jeho. Sloužil také ukydaným bohum, jimž sloužíval otec jeho, a klanel se jim,
\par 22 Opustiv Hospodina Boha otcu svých, aniž chodil po ceste Hospodinove.
\par 23 Spuntovali se pak služebníci Amonovi proti nemu, a zamordovali jej v dome jeho.
\par 24 Tedy pobil lid zeme všecky ty, kteríž se byli spuntovali proti králi Amonovi, a ustanovil lid zeme krále Joziáše syna jeho místo neho.
\par 25 O jiných pak cinech Amonových, kteréž cinil, zapsáno jest v knize o králích Judských.
\par 26 I pochoval ho lid v hrobe jeho v zahrade Uzy, a kraloval Joziáš syn jeho místo neho.

\chapter{22}

\par 1 V osmi letech byl Joziáš, když pocal kralovati, a kraloval jedno a tridceti let v Jeruzaléme. Jméno matky jeho bylo Jedida, dcera Adaiášova z Baskat.
\par 2 Ten cinil to, což pravého jest pred ocima Hospodinovýma, chode po vší ceste Davida otce svého, a neuchyluje se na pravo ani na levo.
\par 3 Stalo se pak osmnáctého léta krále Joziáše, že poslal král Safana syna Azaliášova, syna Mesullamova, písare, do domu Hospodinova, rka:
\par 4 Jdi k Helkiášovi knezi nejvyššímu, at sectou peníze, kteréž jsou vneseny do domu Hospodinova, kteréž sebrali strážní prahu od lidu.
\par 5 A at je dadí v ruce remeslníku predstavených nad dílem domu Hospodinova, aby je obraceli na delníky, kteríž delají dílo v dome Hospodinove, k opravení zborenin domu,
\par 6 Totiž na tesare a stavitele a zedníky, též aby jednali dríví a tesané kamení k opravování domu.
\par 7 A však at poctu neciní z penez, kteréž se dávají v ruce jejich; nebo verne delati budou.
\par 8 I rekl Helkiáš knez nejvyšší Safanovi písari: Knihu zákona nalezl jsem v dome Hospodinove. I dal Helkiáš knihu tu Safanovi, kterýž cetl v ní.
\par 9 Prišel pak Safan písar k králi, a oznamuje králi tu vec, rekl: Sebrali služebníci tvoji peníze, kteréž se nalezly v dome Páne, a dali je v ruce remeslníku predstavených nad dílem domu Hospodinova.
\par 10 Oznámil také Safan písar králi, rka: Knihu mi dal Helkiáš knez. I cetl ji Safan pred králem.
\par 11 A když slyšel král slova knihy zákona, roztrhl roucho své.
\par 12 A rozkázal král Helkiášovi knezi, a Achikamovi synu Safanovu, a Achborovi synu Michaia, a Safanovi písari, a Asaiášovi služebníku královskému,rka:
\par 13 Jdete, poradte se s Hospodinem o mne i o lid, a o všeho Judu z strany slov knihy této, kteráž jest nalezena; nebo veliký jest hnev Hospodinuv, kterýž rozpálen jest proti nám, proto že otcové naši neposlouchali slov knihy této, aby cinili všecko tak, jakž jest nám zapsáno.
\par 14 Tedy šli Helkiáš knez a Achikam, a Achbor, a Safan a Asaia k Chulde prorokyni, manželce Salluma, syna Tekue, syna Charchas, strážného nad rouchem; nebo bydlila v Jeruzaléme na druhé strane. A mluvili s ní.
\par 15 Kteráž rekla jim: Toto praví Hospodin Buh Izraelský: Povezte muži, kterýž vás poslal ke mne:
\par 16 Takto praví Hospodin: Aj, já uvedu zlé veci na místo toto a na obyvatele jeho, všecka slova knihy té, kterouž cetl král Judský,
\par 17 Proto že mne opustili a kadili bohum cizím, aby mne popouzeli všelikým dílem rukou svých. Z té príciny rozpálila se prchlivost má na místo toto, aniž bude uhašena.
\par 18 Králi pak Judskému, kterýž vás poslal, abyste se tázali Hospodina, takto povíte: Toto praví Hospodin Buh Izraelský o slovích tech, kteráž jsi slyšel:
\par 19 Ponevadž obmekceno jest srdce tvé, a ponížils se pred tvárí Hospodinovou, když jsi slyšel, které veci jsem mluvil proti místu tomuto, a proti obyvatelum jeho, že má prijíti v zpuštení a v zlorecení, a roztrhl jsi roucho své, a plakals prede mnou, i já také uslyšel jsem te, praví Hospodin.
\par 20 Protož, aj, já pripojím te k otcum tvým, a pochován budeš v hrobích svých v pokoji, aby nevidely oci tvé niceho z toho zlého, kteréž privedu na místo toto. I oznámili králi tu rec.

\chapter{23}

\par 1 Tedy poslav král, aby se shromáždili k nemu všickni starší Judští a Jeruzalémští,
\par 2 Vstoupil král do domu Hospodinova a všickni muži Judští, i všickni obyvatelé Jeruzalémští s ním, i kneží a proroci, a všecken lid od malého až do velikého, i cetl, aby všickni slyšeli všecka slova knihy smlouvy, kteráž byla nalezena v dome Hospodinove.
\par 3 Potom stoje král na míste vyšším, ucinil smlouvu pred Hospodinem, že bude následovati Hospodina, a ostríhati prikázaní jeho, i svedectví jehoa ustanovení jeho, vším srdcem svým a vší duší svou, a plniti slova smlouvy té, kteráž jsou zapsána v knize té. K kteréžto smlouve i všecken lid pristoupil.
\par 4 A prikázal král Helkiášovi knezi nejvyššímu a knežím nižším i strážným prahu, aby vymetali z chrámu Hospodinova všecky nádoby, kteréž udelány byly Bálovi a háji, i všemu vojsku nebeskému. Kterýž popáliv je vne za Jeruzalémem na poli Cedron, vnesl prach jejich do Bethel.
\par 5 Složil také kneží , kteréž byli ustanovili králové Judští, aby kadívali na výsostech v mestech Judských a vukol Jeruzaléma; takž podobne i ty, kteríž kadívali Bálovi, slunci, mesíci a planétám, i všemu vojsku nebeskému.
\par 6 Vyvezl také háj z domu Hospodinova ven z Jeruzaléma ku potoku Cedron, a spálil jej u potoka Cedron a setrel na prach; ten pak prach vysypal na hroby synu toho lidu.
\par 7 Domky také sodomáru hanebných zkazil, kteríž byli pri dome Hospodinove, v nichž ženy tkaly kortýny k háji.
\par 8 A kázal privésti všecky kneží z mest Judských, a poškvrnil výsostí, na nichž kadívali kneží, od Gabaa až do Bersabé. Zkazil také výsostí u bran, kteréž byly u vrat brány Jozue, knížete mesta, a byly po levé strane vcházejícímu do brány mestské.
\par 9 A však nepristupovali ti kneží výsostí k oltári Hospodinovu v Jeruzaléme, ale jídali chleby presné mezi bratrími svými.
\par 10 Poškvrnil také i Tofet, jenž jest v údolí syna Hinnom, aby více žádný nevodil syna svého aneb dcery své skrze ohen Molochovi.
\par 11 Zahladil také ty kone, kteréž byli postavili králové Judští slunci, kdež se vchází do domu Hospodinova, k domu Netanmelecha komorníka, kterýž byl v Parvarim; a vozy slunce spálil ohnem.
\par 12 Ano i oltáre, kteríž byli na vrchním paláci Achasovu, jichž byli nadelali králové Judští, a oltáre, jichž nadelal Manasses v obou síních domu Hospodinova, poboril král, a pospíšiv s nimi odtud, dal vysypati prach jejich do potoka Cedron.
\par 13 Výsosti také, kteréž byly pred Jeruzalémem, a kteréž byly po pravé strane hory Olivetské, jichž byl nadelal Šalomoun král Izraelský Astarotovi, ohavnosti Sidonských, a Chámosovi, ohavnosti Moábských, a Melchomovi, ohavnosti synu Ammon, poškvrnil král,
\par 14 A potrískal obrazy, háje posekal, a místa jejich naplnil kostmi lidskými.
\par 15 Nadto i oltár, jenž byl v Bethel, a výsost, kterouž udelal Jeroboám syn Nebatuv, kterýž privedl k hrešení Izraele, i ten oltár i výsost zkazil, a spáliv výsost, setrel ji na prach; spálil i háj.
\par 16 A obrátiv se Joziáš, uzrel hroby, kteríž tu na hore byli, a poslav, pobral kosti z tech hrobu a spálil je na tom oltári. A tak poškvrnil ho vedlé reci Hospodinovy, kterouž mluvil muž Boží ten, kterýž to predpovedel.
\par 17 I rekl: Jaký jest onenno nápis, kterýž vidím? Odpovedeli jemu muži mesta: Hrob muže Božího jest, kterýž prišed z Judstva, predpovedel tyto veci, kteréž jsi ucinil pri oltári v Bethel.
\par 18 Tedy rekl: Nechtež ho, aniž kdo hýbej kostmi jeho. I vysvobodili kosti jeho s kostmi proroka toho, kterýž byl prišel z Samarí.
\par 19 Též všecky domy výsostí, kteríž byli v mestech Samarských, jichž byli nadelali králové Izraelští, aby popouzeli Hospodina, zkazil Joziáš, a ucinil jim rovne tak, jakž byl ucinil v Bethel.
\par 20 Zbil také všecky kneží výsostí, kteríž tu byli, na oltárích, a pálil kosti lidské na nich. Potom navrátil se do Jeruzaléma.
\par 21 Prikázal pak král všemu lidu, rka: Slavte velikunoc Hospodinu Bohu svému, jakož psáno jest v knize smlouvy této.
\par 22 Nebo nebyla slavena taková velikanoc od casu soudcu, kteríž soudili Izraele, a po všecky dny králu Izraelských a králu Judských.
\par 23 Osmnáctého léta krále Joziáše slavena jest ta velikanoc Hospodinu v Jeruzaléme.
\par 24 Ano i veštce a hadace, obrazy i ukydané bohy, a všecky ty ohavnosti, což jich bylo videti v zemi Judské a v Jeruzaléme, vyplénil Joziáš, aby naplnil slova zákona zapsaná v knize, kterouž nalezl Helkiáš knez v dome Hospodinove.
\par 25 A nebylo jemu podobného krále pred ním, kterýž by obrátil se k Hospodinu celým srdcem svým, a celou duší svou, i všemi mocmi svými vedlé všeho zákona Mojžíšova, ani po nem nepovstal podobný jemu.
\par 26 A však neodvrátil se Hospodin od prchlivosti hnevu svého velikého, kterouž vzbuzen byl hnev jeho proti Judovi, pro všecka popouzení, kterýmiž popouzel ho Manasses.
\par 27 Protož rekl Hospodin: Také i Judu zavrhu od tvári své, jako jsem zavrhl Izraele, a opovrhu to mesto, kteréž jsem vyvolil, Jeruzalém, i ten dum, o nemž jsem byl rekl: Jméno mé tam bude.
\par 28 O jiných pak cinech Joziášových, a cožkoli cinil, zapsáno jest v knize o králích Judských.
\par 29 Za dnu jeho pritáhl Farao Nécho král Egyptský proti králi Assyrskému k rece Eufraten. I vytáhl král Joziáš proti nemu, a on zabil jej v Mageddo, když ho uzrel.
\par 30 Tedy služebníci jeho vloživše jej mrtvého na vuz, privezli ho z Mageddo do Jeruzaléma, a pochovali jej v hrobe jeho. I vzal lid zeme Joachaza syna Joziášova, a pomazali ho, i ustanovili králem na míste otce jeho.
\par 31 Ve trímecítma letech byl Joachaz, když pocal kralovati, a kraloval tri mesíce v Jeruzaléme. Jméno matky jeho bylo Chamutal, dcera Jeremiášova z Lebna.
\par 32 A cinil to, což jest zlého pred ocima Hospodinovýma, všecko tak, jakž cinili otcové jeho.
\par 33 I svázal ho Farao Nécho v Ribla, v zemi Emat, když kraloval v Jeruzaléme, a uložil dan na tu zemi, sto centnéru stríbra a centnér zlata.
\par 34 A ustanovil Farao Nécho za krále Eliakima syna Joziášova, na místo Joziáše otce jeho, a promenil jméno jeho, aby sloul Joakim. Ale Joachaza vzal, kterýž, když se dostal do Egypta, umrel tam.
\par 35 To pak zlato i stríbro dával Joakim Faraonovi; procež šacoval obyvatele zeme, aby mohl dáti stríbro k rozkázaní Faraonovu. Od jednoho každého vedlé toho, jakž byl šacován, bral stríbro i zlato od lidu zeme, aby dal Faraonovi Néchovi.
\par 36 V petmecítma letech byl Joakim, když pocal kralovati, a jedenácte let kraloval v Jeruzaléme. Jméno matky jeho bylo Zebuda, dcera Pedaiova z Ruma.
\par 37 I cinil to, což jest zlého pred ocima Hospodinovýma, podlé všeho, což cinili otcové jeho.

\chapter{24}

\par 1 Za dnu jeho pritáhl Nabuchodonozor král Babylonský, i ucinen jest Joakim služebníkem jeho za tri léta. Potom odvrátiv se, zprotivil se jemu.
\par 2 Protož poslal Hospodin na nej lotríky Kaldejské a lotríky Syrské, i lotríky Moábské a lotríky synu Ammon, poslal je, pravím, na Judu, aby ho zkazili vedlé reci Hospodinovy, kterouž byl mluvil skrze služebníky své proroky.
\par 3 A jiste podlé reci Hospodinovy to se dálo proti Judovi, aby jej zavrhl od tvári své pro hríchy Manassesovy, vedlé toho všeho, což byl cinil,
\par 4 Ano i pro krev nevinnou, kterouž vylil, naplniv Jeruzalém krví nevinnou, cehož nechtel Hospodin prominouti.
\par 5 O jiných pak vecech Joakimových, a cožkoli cinil, psáno jest v knize o králích Judských.
\par 6 A tak usnul Joakim s otci svými, a kraloval Joachin syn jeho místo neho.
\par 7 Král pak Egyptský nikdy více nevytáhl z zeme své; nebo král Babylonský pobral od reky Egyptské až k rece Eufraten všecko, cožkoli mel král Egyptský.
\par 8 V osmnácti letech byl Joachin, když pocal kralovati, a tri mesíce kraloval v Jeruzaléme. Jméno matky jeho Nechusta, dcera Elnatanova z Jeruzaléma.
\par 9 I cinil to, což jest zlého pred ocima Hospodinovýma, všecko tak, jakž byl cinil otec jeho.
\par 10 Toho casu vytáhli služebníci Nabuchodonozora krále Babylonského proti Jeruzalému, a obleženo jest mesto.
\par 11 Ano i Nabuchodonozor král Babylonský pritáhl proti mestu, když služebníci jeho leželi vukol neho.
\par 12 Tedy vyšel Joachin král Judský k králi Babylonskému, i s matkou svou i s služebníky, knížaty i komorníky svými, kteréhož vzal král Babylonský léta osmého kralování svého.
\par 13 A vynesl odtud všecky poklady domu Hospodinova, a poklady domu královského, a ztloukl všecky nádoby zlaté, jichž byl nadelal Šalomoun král Izraelský do domu Hospodinova, jakož byl mluvil Hospodin.
\par 14 Pri tom prenesl všecken Jeruzalém a všecka knížata i všecky muže udatné, deset tisícu zajatých, i všecky tesare a kováre. Žádného nezanechal, krome chaterných lidí zeme.
\par 15 Zavedl také i Joachina do Babylona, a matku jeho i ženy královské i komorníky jeho; silné také zeme zavedl do vezení z Jeruzaléma do Babylona.
\par 16 Všech také mužu udatných sedm tisícu, tesaru také a kováru tisíc, všecky též zmužilé bojovníky zavedl jaté král Babylonský do Babylona.
\par 17 Ustanovil pak král Babylonský Mataniáše, strýce jeho, králem místo neho, a zmenil mu jméno, aby sloul Sedechiáš.
\par 18 V jedenmecítma letech byl Sedechiáš, když pocal kralovati a jedenácte let kraloval v Jeruzaléme. Jméno matky jeho bylo Chamutal, dcera Jeremiášova z Lebna.
\par 19 I cinil to, což jest zlého pred ocima Hospodinovýma, všecko tak, jakž byl delal Joakim.
\par 20 Nebo se to dálo pro rozhnevání Hospodinovo proti Jeruzalému a Judovi, až je i zavrhl od tvári své. V tom opet zprotivil se Sedechiáš králi Babylonskému.

\chapter{25}

\par 1 Stalo se pak léta devátého kralování jeho, mesíce desátého, v desátý den téhož mesíce, že pritáhl Nabuchodonozor král Babylonský se vším vojskem svým k Jeruzalému, a položil se u neho, a vzdelali proti nemu hradbu vukol.
\par 2 A bylo mesto obleženo, až do jedenáctého léta krále Sedechiáše.
\par 3 V kterémžto, devátého dne ctvrtého mesíce, rozmohl se hlad v meste, a nemel chleba lid zeme.
\par 4 I protrženo jest mesto, a všickni muži bojovní utekli noci té skrze bránu mezi dvema zdmi u zahrady královské; Kaldejští pak leželi okolo mesta. Ušel také král cestou poušte.
\par 5 I honilo vojsko Kaldejské krále, a postihli ho na rovinách Jerišských, a všecko vojsko jeho rozprchlo se od neho.
\par 6 A tak javše krále, privedli jej k králi Babylonskému do Ribla, kdež ucinili o nem soud.
\par 7 Syny pak Sedechiášovy zmordovali pred ocima jeho. Potom Sedechiáše oslepili, a svázavše ho retezy ocelivými, zavedli jej do Babylona.
\par 8 Potom mesíce pátého, sedmý den téhož mesíce, léta devatenáctého kralování Nabuchodonozora krále Babylonského, pritáhl Nebuzardan hejtman nad žoldnéri, služebník krále Babylonského, do Jeruzaléma.
\par 9 A zapálil dum Hospodinuv i dum královský, i všecky domy v Jeruzaléme, a tak všecky domy veliké vypálil.
\par 10 Zdi také Jeruzalémské vukol poborilo všecko vojsko Kaldejské, kteréž bylo s tím hejtmanem nad žoldnéri.
\par 11 Ostatek pak lidu, kterýž byl zustal v meste, i pobehlce, kteríž se byli obrátili k králi Babylonskému, a jiný obecný lid, zavedl Nebuzardan hejtman nad žoldnéri.
\par 12 Toliko neco chaterného lidu zeme zanechal hejtman nad žoldnéri, aby byli vinari a oráci.
\par 13 Nadto sloupy medené, kteríž byli v dome Hospodinove, i podstavky, i more medené, kteréž bylo v dome Hospodinove, ztloukli Kaldejští, a med z nich odvezli do Babylona.
\par 14 Též hrnce, lopaty a nástroje hudebné, a kadidlnice i všecky nádoby medené, jimiž sloužili, pobrali.
\par 15 I nádoby k oharkum a kotlíky, a cokoli zlatého a stríbrného bylo, pobral hejtman nad žoldnéri,
\par 16 Sloupy dva, more jedno a podstavky, jichž byl nadelal Šalomoun do domu Hospodinova. Nebylo váhy medi všech tech nádob.
\par 17 Osmnácti loket byla výška sloupu jednoho, a makovice na nem medená, kterážto makovice trí loket zvýší byla, a mrežování i jablka zrnatá na té makovici vukol; všecko bylo medené. Takovýž byl i druhý sloup s mrežováním.
\par 18 Vzal také týž hejtman nad žoldnéri Saraiáše kneze predního, a Sofoniáše kneze nižšího, a tri strážné prahu.
\par 19 A z mesta vzal komorníka jednoho, kterýž byl hejtmanem nad muži bojovnými, a pet mužu z tech, jenž bývali pri králi, kteríž nalezeni byli v meste, a predního spisovatele vojska, kterýž popisoval vojsko z lidu zeme, a šedesáte mužu z lidu zeme, kteríž se nalezli v meste.
\par 20 Zjímav tedy je Nebuzardan hejtman nad žoldnéri, privedl je k králi Babylonskému do Ribla.
\par 21 I pobil je král Babylonský, a zmordoval je v Ribla, v zemi Emat, a tak zaveden jest Juda z zeme své.
\par 22 Lidu pak, kterýž zustal v zemi Judské, jehož byl zanechal Nabuchodonozor král Babylonský, predstavil Godoliáše syna Achikama, syna Safanova.
\par 23 I uslyšeli všickni hejtmané vojska, oni i lid jejich, že postavil za správce král Babylonský Godoliáše, a prišli k Godoliášovi do Masfa, totiž Izmael syn Netaniášuv, a Jochanan syn Kareachuv, a Saraiáš syn Tanchumeta Netofatského, a Jazaniáš syn Machatuv, oni i lid jejich.
\par 24 Tedy prisáhl jim Godoliáš i lidu jejich, a rekl jim: Nebojte se služby Kaldejských, zustante v zemi, a služte králi Babylonskému, a dobre vám bude.
\par 25 I stalo se mesíce sedmého, prišel Izmael syn Netaniáše, syna Elisamova, z semene královského, a deset mužu s ním. I zabili Godoliáše, a umrel; takž i Židy i Kaldejské, kteríž s ním byli v Masfa.
\par 26 Procež zdvih se všecken lid, od malého až do velikého, i hejtmané vojsk, ušli do Egypta; nebo se báli Kaldejských.
\par 27 Stalo se také léta tridcátého sedmého po zajetí Joachina krále Judského, dvanáctého mesíce, dvadcátého sedmého dne téhož mesíce, povýšil Evilmerodach král Babylonský toho léta, když pocal kralovati, Joachina krále Judského, pustiv ho z žaláre.
\par 28 A mluvil s ním dobrotive, i stolici jeho postavil nad stolice jiných králu, kteríž s ním byli v Babylone.
\par 29 Zmenil též roucho jeho, kteréž mel v žalári. I jídal vždycky pred ním po všecky dny života svého.
\par 30 Nebo vymerený pokrm ustavicne dáván byl jemu od krále, a to na každý den po všecky dny života jeho.

\end{document}