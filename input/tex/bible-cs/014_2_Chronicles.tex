\begin{document}

\title{2 Paralipomenon}

\chapter{1}

\par 1 Když se pak zmocnil Šalomoun syn Daviduv v království svém, a Hospodin Buh jeho byl s ním, a zvelebil ho náramne:
\par 2 Tedy rozkázal Šalomoun všemu Izraelovi i hejtmanum, setníkum i soudcum, i všechnem knížatum nade vším Izraelem, i prednejším v celedech otcovských.
\par 3 I bral se Šalomoun a všecko to shromáždení s ním na výsost, kteráž byla v Gabaon; nebo tam byl stánek shromáždení Božího, kterýž byl udelal Mojžíš služebník Hospodinuv na poušti.
\par 4 (Truhlu pak Boží privezl byl David z Kariatjeharim, pripraviv jí místo; nebo byl rozbil jí stan v Jeruzaléme.)
\par 5 A oltár medený, kterýž byl udelal Bezeleel syn Uri, syna Hur, byl tam pred stánkem Hospodinovým, kdež hledal ho Šalomoun i všecko to shromáždení.
\par 6 I obetoval tam Šalomoun pred Hospodinem na oltári medeném, kterýž byl pred stánkem úmluvy, a obetoval na nem tisíc zápalu.
\par 7 Té noci ukázal se Buh Šalomounovi, a rekl jemu: Žádej, zac chceš, a dám tobe.
\par 8 I rekl Šalomoun Bohu: Ty jsi ucinil otci mému Davidovi milosrdenství veliké, mne jsi též ustanovil králem místo neho.
\par 9 Již tedy, Hospodine Bože, budiž stálé slovo tvé mluvené s Davidem otcem mým; nebo ty jsi mne ustanovil za krále nad lidem tak mnohým, jako jest prachu zemského.
\par 10 Protož dej mi moudrost a umení, atbych vycházeti mohl pred lidem tímto i vcházeti. Nebo kdož by mohl souditi tento lid tvuj tak mnohý?
\par 11 Tedy odpovedel Buh Šalomounovi: Proto že bylo to v srdci tvém, a nežádal jsi bohatství, ani zboží, ani slávy, ani bezživotí tech, jenž tebe nenávidí, aniž jsi také za dlouhý vek žádal, ale žádal jsi sobe moudrosti a umení, abys soudil lid muj, nad nímž jsem te ustanovil za krále:
\par 12 Moudrost a umení dáno jest tobe, k cemužt pridám i bohatství a zboží, i slávy, tak že žádný z králu, kteríž byli pred tebou, nebyl tobe rovný, aniž bude po tobe takového.
\par 13 I navrátil se Šalomoun s výsosti, kteráž byla v Gabaon, do Jeruzaléma, od stánku úmluvy, a tak kraloval nad Izraelem.
\par 14 Nashromáždil pak Šalomoun vozu a jezdcu, a mel tisíc a ctyri sta vozu, a dvanácte tisíc jezdcu, kteréž rozsadil do mest vozu, a pri králi v Jeruzaléme.
\par 15 I složil král stríbra a zlata v Jeruzaléme jako kamení, a dríví cedrového jako planého fíkoví, kteréž roste v údolí u velikém množství.
\par 16 Privodili také Šalomounovi kone z Egypta a koupe rozlicné; nebo kupci královští brávali koupe rozlicné za slušnou mzdu.
\par 17 A vycházejíce, vodívali sprež vozníku z Egypta za šest set lotu stríbra, kone pak za pul druhého sta. A tak všechnem králum Hetejským i králum Syrským oni dodávali.

\chapter{2}

\par 1 A umíniv Šalomoun staveti dum jménu Hospodinovu, a dum svuj královský,
\par 2 Odectl Šalomoun sedmdesát tisíc nosicu a osmdesát tisíc tech, kteríž tesali na hore, a úredníku nad nimi tri tisíce a šest set.
\par 3 Poslal také Šalomoun k Chíramovi králi Tyrskému, rka: Jakž jsi se choval k Davidovi otci mému, posílaje mu dríví cedrové, aby sobe stavel dum k bydlení, tak cin i mne.
\par 4 Nebo aj, já staveti chci dum jménu Hospodina Boha svého, aby posvecen byl jemu k tomu, aby se pred ním kadilo vonnými vecmi, a k ustavicnému predkládání chlebu, i k zápalným obetem, ranním i vecerním, ve dny sobotní a novomesícné, i na slavnosti Hospodina Boha našeho, což v Izraeli na veky trvati má.
\par 5 Dum pak, kterýž staveti chci, veliký býti má; nebo Buh náš vetší jest nade všecky bohy.
\par 6 Ac kdo jest, ješto by mohl dum jemu vystaveti, ponevadž ho nebe i nebesa nebes obsáhnouti nemohou? Ano i já kdo jsem, abych jemu dum vystaveti mel, než toliko k tomu, aby se kadilo pred ním?
\par 7 Protož nyní pošli mi muže umelého, kterýž by umel delati na zlate, na stríbre, na medi, na železe, i z zlatohlavu a z cervce, a z postavce modrého, a kterýž by umel rezati rezby s jinými umelými, kteríž jsou u mne v Judstvu a v Jeruzaléme, kteréž zjednal David otec muj.
\par 8 Pres to pošli mi také dríví cedrového a jedlového, a algumim z Libánu; nebo vím, že služebníci tvoji umejí sekati dríví Libánské. A hle, služebníci moji budou s služebníky tvými,
\par 9 Aby mi pripravili dostatek dríví; nebo dum, kterýž já staveti chci, veliký býti má a slavný.
\par 10 A aj, dám na delníky, kteríž sekati mají dríví, pšenice semlené služebníkum tvým dvadcet tisíc mer, a dvadcet tisíc mer jecmene, a dvadcet tisíc lák vína, a dvadcet tisíc tun oleje.
\par 11 I odpovedel Chíram král Tyrský psáním, kteréž poslal k Šalomounovi: Jiste, žet miluje Hospodin lid svuj, protož te ustanovil nad nimi za krále.
\par 12 Rekl dále Chíram: Požehnaný Hospodin Buh Izraelský, kterýž ucinil nebe i zemi, a kterýž dal králi Davidovi syna moudrého, umelého, rozumného a opatrného, aby vystavel dum Hospodinu a dum svuj královský.
\par 13 Protož posílámt ted muže moudrého, umelého a opatrného, kterýž byl u Chírama otce mého,
\par 14 Syna jedné ženy ze dcer Dan, otce pak mel Tyrského, kterýž umí delati na zlate, na stríbre, na medi, železe, kamení a na dríví, i z šarlatu, z postavce modrého, z kmentu a z cervce, tolikéž rezati všelijaké rezby, a vymysliti všelijaké dílo, kteréž dáno mu bude s moudrými tvými, a s moudrými pána mého Davida otce tvého.
\par 15 Pšenice však toliko a jecmene, oleje a vína, což rekl pán muj, necht pošle služebníkum svým.
\par 16 My pak nasekáme dríví z Libánu, což ho koli bude potrebí tobe, a priplavíme je tobe v vorích po mori k Joppe, a ty dáš je voziti do Jeruzaléma.
\par 17 A tak sectl Šalomoun všecky cizozemce, kteríž byli v zemi Izraelské po sectení tom, kterýmž sectl je David otec jeho, a nalezeno jich sto a padesát tisíc, tri tisíce a šest set.
\par 18 I vybral z nich sedmdesát tisíc nosicu, a osmdesát tisíc tech, kteríž sekali na hore, tri pak tisíce a šest set úredníku, kteríž lid k dílu prídrželi.

\chapter{3}

\par 1 I zacal staveti Šalomoun domu Hospodinova v Jeruzaléme na hore Moria, kteráž byla ukázána Davidovi otci jeho, na míste, kteréž byl pripravil David, na humne Ornana Jebuzejského.
\par 2 A pocal staveti druhého mesíce, dne druhého, království svého léta ctvrtého.
\par 3 A toto jest vymerení Šalomounovo pri stavení domu Božího: Dlouhost loktu podlé první míry bylo šedesáti loket, a šír dvadcíti loket.
\par 4 A sín, kteráž byla v cele, jakž široký dum, byla na dvadceti loktu, vysokost pak sto a dvadcíti, a obložil ji vnitr zlatem cistým.
\par 5 Dum pak veliký opažil drívím jedlovým, kteréž obložil zlatem nejcistším, na nemž po vrchu dal nadelati palm a retízku.
\par 6 Prikryl také dum ten kamenem drahým ozdobne; zlato pak to bylo zlato Parvaimské.
\par 7 Obložil, pravím, dum, trámy, vereje i steny jeho, i dvére jeho zlatem, a vyryl cherubíny na stenách.
\par 8 Udelal i dum svatyne svatých, jehož dýlka byla jako šírka domu, dvadcíti loktu, a šírka loktu dvadcíti, a obložil ji zlatem výborným šesti sty centnéri.
\par 9 Hrebíkové též vážili padesát lotu zlata, ano i sínce obložil zlatem.
\par 10 Udelal také v dome svatyne svatých dva cherubíny dílem remeslným, a obložil je zlatem.
\par 11 Dlouhost krídel tech cherubínu byla na dvadceti loket. Krídlo jedno na pet loket, a dotýkalo se steny domu, a druhé krídlo na pet loket dotýkalo se krídla cherubína druhého.
\par 12 A tak krídlo cherubína jednoho na pet loket dotýkalo se steny domu, a krídlo druhé na pet loket dosahovalo krídla cherubína druhého.
\par 13 A tak krídla cherubínu tech roztažená byla na dvadceti loktu, a stáli na nohách svých, tvárí svou obráceni do domu.
\par 14 Udelal také i oponu z postavce modrého, z šarlatu, z cervce a z kmentu, a udelal na ní cherubíny.
\par 15 Udelal též pred domem sloupy dva na tridceti a pet loktu zvýší, a makovice, kteréž byly na každém svrchu, na pet loket.
\par 16 Zdelal též i retízky jako v svatyni svatých, a otocil je okolo makovic tech sloupu, a udelav jablek zrnatých sto, dal mezi retízky.
\par 17 A tak postavil ty sloupy pred chrámem, jeden po pravé a druhý po levé strane, a dal jméno tomu, kterýž byl po pravici Jachin, a jméno tomu, kterýž byl po levici, Boaz.

\chapter{4}

\par 1 Udelal také i oltár medený, zdélí dvadcíti loktu, a dvadcíti loktu zšírí, desíti pak loktu zvýší.
\par 2 Udelal též more slité, desíti loket od jednoho kraje k druhému, okrouhlé vukol, zvýší na pet loktu, a okolek jeho tridcíti loket vukol.
\par 3 Podobenství také volu pod ním, kterýchž všudy vukol bylo deset do lokte, obklicujících more vukol; a tak byly dva rady volu slitých, spolu s morem.
\par 4 A stálo na dvanácti volích. Tri obráceni byli na pulnoci, a tri patrili k západu, tri zase postaveni byli ku poledni, a tri obráceni byli k východu, a more svrchu na nich stálo, ale všech jich zadkové byli pod morem.
\par 5 A bylo ztlouští na dlan. Kraj jeho byl, jakýž bývá u koflíku aneb kvetu liliového, tri tisíce tun v se beroucí.
\par 6 Udelal také deset umyvadel, a postavil jich pet po pravé strane, a pet po levé, k obmývání z nich. Všecko, což se strojilo k zápalum, obmývali z nich, ale more, aby se z neho kneží umývali.
\par 7 Nadto udelal i svícnu zlatých deset vedlé slušnosti jejich, a postavil v chráme, pet po pravé strane a pet po levé.
\par 8 Udelal i stolu deset, kteréž postavil v chráme, pet po pravé a pet po levé strane. K tomu udelal cíší zlatých sto.
\par 9 Udelal potom sín knežskou, též sín velikou, a dvére u též síne okoval medí.
\par 10 More pak postavil na pravé strane k východu, naproti polední strane.
\par 11 Nadelal také Chíram hrncu, lopat a kotlíku, a dokonal Chíram dílo, kteréž byl delal králi Šalomounovi k domu Božímu.
\par 12 Dva sloupy a kruhy, makovice také na vrchu tech dvou sloupu, a mrežování dvoje, aby prikrývalo ty dve makovice okrouhlé, kteréž byly na vrchu sloupu,
\par 13 A jablek zrnatých ctyri sta na dvojím mrežování. Dvema rady jablka zrnatá byla na mrežování jednom, aby prikrývala ty dve makovice okrouhlé, kteréž byly na vrchu sloupu.
\par 14 Podstavky také zdelal, a umyvadla na tech podstavcích.
\par 15 More jedno, a volu dvanácte pod ním.
\par 16 Též hrnce a lopaty a vidlicky trírohé, a všecko nádobí jejich zdelal Chíram s otcem svým králi Šalomounovi k domu Hospodinovu z medi precisté.
\par 17 Na rovinách Jordánských sléval to král v zemi jilovaté, mezi Sochot a Saredata.
\par 18 A tak nadelal Šalomoun všech tech nádob velmi mnoho, tak že váhy té medi vyhledáváno nebylo.
\par 19 Zdelal také Šalomoun i jiná všelijaká nádobí k domu Božímu, jako oltár zlatý a stoly, na nichž kladeni byli chlebové predložení,
\par 20 Tolikéž svícny i lampy jejich z zlata nejcistšího, aby je rozsvecovali príslušne pred svatyní svatých,
\par 21 Kvety také a lampy i uteradla z zlata, a to bylo zlato nejvýbornejší.
\par 22 I žaltáre a kotlíky, kadidlnice a nádoby k oharkum z zlata cistého, k tomu i bránu domu, dvére vnitrní svatyne svatých, i dvére domu, totiž v chráme, byly z zlata.

\chapter{5}

\par 1 A tak dokonáno jest všecko dílo, kteréž delal Šalomoun k domu Hospodinovu, a vnesl tam Šalomoun veci posvecené od Davida otce svého, též stríbro a zlato, a všecko nádobí, složiv je mezi poklady domu Božího.
\par 2 Tedy shromáždil Šalomoun starší Izraelské, a všecky prední z pokolení, totiž knížata celedí otcovských s syny Izraelskými do Jeruzaléma, aby prenesli truhlu smlouvy Hospodinovy z mesta Davidova, jenž jest Sion.
\par 3 I shromáždili se k králi všickni muži Izraelští na slavnost, kteráž bývá mesíce sedmého.
\par 4 Když pak prišli všickni starší Izraelští, vzali Levítové truhlu,
\par 5 A nesli ji zhuru, též stánek úmluvy, i všecka nádobí posvátná, kteráž byla v stánku, prenesli to kneží a Levítové.
\par 6 Zatím král Šalomoun i všecko shromáždení Izraelské, kteréž se k nemu sešlo, obetovali pred truhlou ovce a voly, kteríž ani popisováni, ani vycítáni nebyli pro množství.
\par 7 A tak vnesli kneží truhlu smlouvy Hospodinovy na místo její, do vnitrního domu, do svatyne svatých, pod krídla cherubínu.
\par 8 Nebo cherubínové meli roztažená krídla nad místem truhly, a prikrývali cherubínové truhlu i sochory její svrchu.
\par 9 A povytáhli sochoru, tak že vidíni byli koncové jejich z truhly, k predku svatyne svatých, vne však nebylo jich videti. A byla tam až do tohoto dne.
\par 10 Nic nebylo v truhle, krome dvou tabulí, kteréž tam složil Mojžíš na Orébe tehdáž, když ucinil Hospodin smlouvu s syny Izraelskými, a oni vyšli z Egypta.
\par 11 I stalo se, když vycházeli kneží z svatyne, (nebo všickni kneží, kteríž se koli našli, byli se posvetili, aniž šetrili porádku.
\par 12 Tak i Levítové zpeváci všickni, kteríž byli pri Azafovi, Hémanovi a Jedutunovi, i synové jejich i bratrí jejich, odíni jsouce kmentem, stáli s cymbály a loutnami a harfami k východní strane oltáre, a s nimi kneží sto a dvadceti, troubících v trouby.
\par 13 Nebo meli ti, kteríž spolu troubili v trouby, a zpeváci vydávati jeden zvuk k chválení a oslavování Hospodina); a když povyšovali hlasu na trouby a cymbály i jiné nástroje hudebné, chválíce Hospodina a rkouce, že dobrý jest, a že na veky trvá milosrdenství jeho: tedy oblak naplnil dum ten, dum totiž Hospodinuv,
\par 14 Tak že nemohli kneží ostáti a sloužiti pro ten oblak; nebo sláva Hospodinova byla naplnila dum Boží.

\chapter{6}

\par 1 Tehdy rekl Šalomoun: Hospodin rekl, že bude prebývati v mrákote.
\par 2 Aj, jižt jsem vystavel tobe, Pane, dum k prebývání, a místo, v nemž bys prebýval na veky.
\par 3 A obrátiv král tvár svou, dával požehnání všemu shromáždení Izraelskému. (Všecko pak shromáždení Izraelské stálo.)
\par 4 A rekl: Požehnaný Hospodin Buh Izraelský, kterýž mluvil ústy svými Davidovi otci mému, a to ted skutecne naplnil, rka:
\par 5 Od toho dne, jakž jsem vyvedl lid svuj z zeme Egyptské, nevyvolil jsem mesta z žádného pokolení Izraelského k vystavení domu, kdež by prebývalo jméno mé, aniž jsem vyvolil kterého muže, aby byl vývodou nad lidem mým Izraelským.
\par 6 Ale vyvolil jsem Jeruzalém, aby tu prebývalo jméno mé, a vyvolil jsem Davida, aby byl nad lidem mým Izraelským.
\par 7 Uložil te byl zajisté David otec muj staveti dum jménu Hospodina Boha Izraelského.
\par 8 Ale Hospodin rekl Davidovi otci mému: Ackoli jsi uložil v srdci svém staveti dum jménu mému, a dobres ucinil, žes to myslil v srdci svém,
\par 9 A však ty nebudeš staveti toho domu, ale syn tvuj, kterýž vyjde z bedr tvých, on vystaví dum ten jménu mému.
\par 10 A tak splnil Hospodin slovo své, kteréž byl mluvil. Nebo jsem povstal na místo Davida otce svého, a dosedl jsem na stolici Izraelskou, jakož byl mluvil Hospodin, a ustavel jsem dum tento jménu Hospodina Boha Izraelského.
\par 11 A postavil jsem tam truhlu, v níž jest smlouva Hospodinova, kterouž ucinil s syny Izraelskými.
\par 12 I postavil se pred oltárem Hospodinovým, prede vším shromáždením Izraelským, a pozdvihl rukou svých.
\par 13 Udelal pak byl Šalomoun výstupek medený ku podobenství pánve, a postavil jej u prostred síne, peti loktu zdélí, a peti loket zšírí, a trí loket zvýší, i vstoupil na nej, a poklekl na kolena svá prede vším shromáždením Izraelským, a pozdvihl rukou svých k nebi,
\par 14 A rekl: Hospodine Bože Izraelský, nenít podobného tobe Boha na nebi ani na zemi, kterýž by ostríhal smlouvy a milosrdenství služebníkum svým, chodícím pred tebou v celém srdci svém,
\par 15 Kterýž jsi splnil služebníku svému, Davidovi otci mému, to, což jsi mluvil jemu. Jakž jsi mluvil ústy svými, tak jsi to skutecne naplnil, jakž se to dnes vidí.
\par 16 Nyní tedy, ó Hospodine Bože Izraelský, naplniž služebníku svému, Davidovi otci mému, což jsi mluvil jemu, rka: Nebudet vyhlazen muž z rodu tvého od tvári mé, aby nemel sedeti na stolici Izraelské, jestliže toliko ostríhati budou synové tvoji cesty své, chodíce v zákone mém, tak jakož jsi ty chodil prede mnou.
\par 17 Protož nyní, ó Hospodine Bože Izraelský, necht jest upevneno slovo tvé, kteréž jsi mluvil služebníku svému Davidovi.
\par 18 (Ac zdali v pravde bydliti bude Buh s clovekem na zemi? Aj, nebesa, nýbrž nebesa nebes neobsahují te, mnohem méne dum tento, kterýž jsem vystavel.)
\par 19 A popatr k modlitbe služebníka svého a k úpení jeho, Hospodine Bože muj, slyše volání a modlitbu, kterouž služebník tvuj modlí se pred tebou,
\par 20 Aby oci tvé byly otevrené na dum tento dnem i nocí, na místo toto, o nemž jsi mluvil, že tu prebývati bude jméno tvé, abys vyslýchal modlitbu, kterouž se modlívati bude služebník tvuj na míste tomto.
\par 21 Vyslýchejž tedy modlitbu služebníka svého, i lidu svého Izraelského, kterouž se modlívati budou na míste tomto, ty vždy vyslýchej z místa prebývání svého, s nebe, a vyslýchaje, bud milostiv.
\par 22 Když by zhrešil clovek proti bližnímu svému, a nutil by ho k prísaze, tak že by prisahati musil, a prišla by ta prísaha pred oltár do domu tohoto:
\par 23 Ty vyslýchej s nebe, a rozeznej i rozsud služebníky své, mste nad bezbožným, obraceje usilování jeho na hlavu jeho, a ospravedlnuje spravedlivého, odplacuje mu podlé spravedlnosti jeho.
\par 24 Tolikéž když by poražen byl lid tvuj Izraelský od neprátel, proto že zhrešili proti tobe, jestliže by obrátíce se, vyznávali jméno tvé, a modléce se, ponížene by prosili tebe v dome tomto:
\par 25 Ty vyslýchej s nebe, a odpust hrích lidu svému Izraelskému, a prived je zase do zeme, kterouž jsi jim dal i otcum jejich.
\par 26 Podobne když by zavríno bylo nebe, a nepršel by déšt, proto že zhrešili proti tobe, a modléce se na míste tomto, vyznávali by jméno tvé, a od hríchu svého by se odvrátili po tvém trestání:
\par 27 Ty vyslýchej na nebi, a odpust hrích služebníku svých a lidu svého Izraelského, vyucuje je ceste výborné, po níž by chodili, a dej déšt na zemi svou, kterouž jsi dal lidu svému za dedictví.
\par 28 Byl-li by hlad na zemi, byl-li by mor, sucho neb rez, kobylky neb brouci jestliže by byli, ssoužil-li by jej neprítel jeho v zemi obývání jeho, aneb jakákoli rána a jakákoli nemoc:
\par 29 Všelikou modlitbu a každé úpení, kteréž by pocházelo od kteréhokoli cloveka, aneb ode všeho lidu tvého Izraelského, když by jen poznajíce jeden každý ránu svou a bolest svou, pozdvihl by rukou svých v dome tomto,
\par 30 Ty vyslýchej s nebe, z místa prebývání svého, a slituj se, a odplat jednomu každému podlé skutku jeho, vedlé toho, jakž znáš srdce jeho, (ty zajisté sám znáš srdce lidská),
\par 31 Aby se báli tebe, a chodili po všech cestách tvých, po všecky dny, v nichž by živi byli na zemi, kterouž jsi dal otcum našim.
\par 32 Nýbrž také i cizozemec, kterýž není z lidu tvého Izraelského, prišel-li by z zeme daleké pro jméno tvé veliké a ruku tvou silnou, a ráme tvé vztažené, když by prišli a modlili se v dome tomto:
\par 33 Ty vyslýchej s nebe, z místa prebývání svého, a ucin všecko to, o cež volati bude k tobe cizozemec ten, aby poznali všickni národové zeme jméno tvé, a báli se tebe jako lid tvuj Izraelský, a aby poznali, že jméno tvé vzýváno jest nad domem tímto, kterýž jsem vystavel.
\par 34 Když by vytáhl lid tvuj k boji proti neprátelum svým cestou, kterouž bys je poslal, a modlili by se tobe naproti mestu tomuto, kteréž jsi vyvolil, a domu tomuto, kterýž jsem ustavel jménu tvému:
\par 35 Ty vyslýchej s nebe modlitbu a úpení jejich, a vyvod pri jejich.
\par 36 Když by zhrešili proti tobe, (jakož není cloveka, ješto by nehrešil), a rozhnevaje se na ne, vydal bys je v moc nepríteli, tak že by je jaté vedli ti, kteríž by je zjímali, do zeme daleké neb blízké;
\par 37 A usmyslili by sobe v zemi, do níž by zajati byli, a obrátíce se, modlili by se tobe v zemi zajetí svého, rkouce: Zhrešili jsme, prevrácene jsme cinili, a bezbožne jsme se chovali;
\par 38 A tak navrátili by se k tobe celým srdcem svým a celou duší svou v zemi zajetí svého, do kteréž by zajati byli, a modlili by se naproti zemi své, kterouž jsi dal otcum jejich, a naproti mestu, kteréž jsi vyvolil, a domu, kterýž jsem vzdelal jménu tvému:
\par 39 Vyslýchej s nebe, z místa príbytku svého, modlitbu jejich a úpení jejich, a vyvod pri jejich, a odpust lidu svému, kterýž by zhrešil proti tobe.
\par 40 Nyní tedy, Bože muj, necht jsou, prosím, oci tvé otevrené, a uši tvé naklonené k modlitbe na míste tomto.
\par 41 Aj, nyní povstan, ó Hospodine Bože, k odpocinutí svému, ty i truhla síly tvé; kneží tvoji, Hospodine Bože, necht jsou obleceni v spasení, a svatí tvoji at se veselí v dobrých vecech.
\par 42 Hospodine Bože, neodvracejž tvári od pomazaného svého, pamatuj na milosrdenství zaslíbená Davidovi služebníku svému.

\chapter{7}

\par 1 A když prestal Šalomoun modliti se, rychle ohen sstoupil s nebe, a sehltil zápal i jiné obeti, a sláva Hospodinova naplnila dum ten,
\par 2 Tak že nemohli kneží vjíti do domu Hospodinova, proto že naplnila sláva Hospodinova dum Hospodinuv.
\par 3 Všickni pak synové Izraelští videli, když sstupoval ohen a sláva Hospodinova na dum, a padše tvárí k zemi na dlážení, klaneli se a chválili Hospodina, že dobrý jest, a že na veky trvá milosrdenství jeho.
\par 4 Pri tom král a všecken lid obetovali obeti pred Hospodinem.
\par 5 Obetoval zajisté král Šalomoun obet dvamecítma tisíc volu, a ovec sto a dvadceti tisícu, když posvecovali domu Božího král i všecken lid.
\par 6 Ale kneží stáli pri svých úradích, též i Levítové s nástroji hudebnými Hospodinovými, kterýchž byl nadelal David král k oslavování Hospodina, (nebo na veky milosrdenství jeho), žalmem Davidovým, kterýž jim vydal. Jiní pak kneží troubili v trouby naproti nim, a všecken lid Izraelský stál.
\par 7 Posvetil také Šalomoun prostredku té síne, kteráž byla pred domem Hospodinovým; nebo obetoval tu obeti zápalné a tuky obetí, pokojných, proto že na oltári medeném, kterýž byl udelal Šalomoun, nemohli se smestknati zápalové a obeti suché i tukové jejich.
\par 8 I držel Šalomoun slavnost toho casu za sedm dní, a všecken Izrael s ním, shromáždení velmi veliké odtud, kudyž se vchází do Emat, až ku potoku Egyptskému.
\par 9 I svetili dne osmého svátek; nebo posvecení oltáre slavili za sedm dní, tolikéž slavnost tu za sedm dní.
\par 10 Trimecítmého pak dne mesíce sedmého propustil lid k príbytkum jejich s radostí a veselím srdce z toho, což dobrého ucinil Hospodin Davidovi a Šalomounovi a Izraelovi lidu svému.
\par 11 I dokonal Šalomoun dum Hospodinuv a dum královský, a všecko, cožkoli byl uložil v srdci svém, aby ucinil v dome Hospodinove a v dome svém, štastne se mu vedlo.
\par 12 V tom ukázal se Hospodin Šalomounovi v noci, a rekl jemu: Uslyšel jsem modlitbu tvou, a vyvolil jsem sobe to místo za dum obetí.
\par 13 Jestliže zavru nebe, tak že by nebylo dešte, a jestliže prikáži kobylkám, aby pohubily zemi, též jestliže pošli morovou ránu na lid svuj,
\par 14 A ponižujíce se lid muj, nad nímž jest vzýváno jméno mé, modlili by se, a hledali by tvári mé, a odvrátili by se od cest svých zlých: i já také vyslyším je s nebe, a odpustím hrích jejich, a uzdravím zemi jejich.
\par 15 Budout již i oci mé otevrené, a uši mé naklonené k modlitbe z místa tohoto.
\par 16 Nebo nyní vyvolil jsem a posvetil domu tohoto, aby tu prebývalo jméno mé až na veky, a aby tu byly oci mé a srdce mé po všecky dny.
\par 17 A ty budeš-li choditi prede mnou, jako chodil David otec tvuj, tak abys cinil všecko to, což jsem prikázal tobe, ustanovení i soudu mých ostríhaje:
\par 18 Utvrdím zajisté stolici království tvého, jakož jsem ucinil smlouvu s Davidem otcem tvým, rka: Nebude vyhlazen muž z rodu tvého, aby nepanoval nad Izraelem.
\par 19 Jestliže pak se odvrátíte a opustíte ustanovení má a prikázání má, kteráž jsem vám vydal, a odejdouce, sloužiti budete bohum cizím a klaneti se jim:
\par 20 Vypléním takové z zeme své, kterouž jsem jim dal, a dum tento, kteréhož jsem posvetil jménu svému, zavrhu od tvári své, a vydám jej v prísloví a v rozprávku mezi všemi národy.
\par 21 A tak dum ten, kterýž byl zvýšený každému jdoucímu mimo nej, bude k užasnutí, a dí: Proc tak ucinil Hospodin zemi této a domu tomuto?
\par 22 Tedy odpovedí: Proto že opustili Hospodina Boha otcu svých, kterýž je vyvedl z zeme Egyptské, a chopili se bohu cizích, a klanejíce se jim, sloužili jim, protož uvedl na ne všecky tyto zlé veci.

\chapter{8}

\par 1 Stalo se potom po prebehnutí dvadcíti let, v nichž stavel Šalomoun dum Hospodinuv a dum svuj,
\par 2 Že vystavel Šalomoun mesta, kteráž byl dal Chíram Šalomounovi, a osadil tam syny Izraelské.
\par 3 Zatím táhl Šalomoun do Emat Soby, a zmocnil se jí.
\par 4 I ustavel Tadmor na poušti, a všecka mesta skladu vystavel v Emat.
\par 5 Vystavel i Betoron horejší, též i Betoron dolejší, mesta ohrazená zdmi, branami i závorami.
\par 6 Ano i Baalat a všecka mesta, v nichž mel sklady Šalomoun, a všecka mesta vozu, i mesta jízdných, všecko vedlé žádosti své, cožkoli chtel staveti v Jeruzaléme a na Libánu, i po vší zemi panování svého.
\par 7 Všecken také lid, kterýž byl pozustal z Hetejských, a Amorejských a Ferezejských, a Hevejských a Jebuzejských, kteríž nebyli z Izraele,
\par 8 Z synu jejich, kteríž byli pozustali po nich v zemi té, jichž byli nevyhubili synové Izraelští, uvedl Šalomoun pod plat až do tohoto dne.
\par 9 Ale z synu Izraelských, jichž nepodrobil Šalomoun v službu pri díle svém, (nebo oni byli muži bojovní a prední knížata jeho, úredníci nad vozy a jezdci jeho),
\par 10 Z tech, pravím, bylo predních vládaru, kteréž mel král Šalomoun, dve ste a padesáte, kteríž panovali nad lidem.
\par 11 Dceru pak Faraonovu prestehoval Šalomoun z mesta Davidova do domu, kterýž jí byl vystavel. Nebo rekl: Nemohlat by bydliti manželka má v dome Davida krále Izraelského, nebo svatý jest, proto že vešla do neho truhla Hospodinova.
\par 12 Tedy obetoval Šalomoun zápaly Hospodinu na oltári Hospodinovu, kterýž byl vzdelal pred síní,
\par 13 Cokoli náležite každého dne obetováno býti melo podlé prikázaní Mojžíšova, ve dny sobotní, na novmesíce a na slavnosti, po trikrát do roka, na slavnost presnic, na slavnost téhodnu, a na slavnost stánku.
\par 14 Ustanovil také podlé narízení Davida otce svého porádky knežské k úradum jejich, a Levíty ku povinnostem jejich, aby chválili Boha, a prisluhovali pri knežích náležite každého dne, a vrátné v porádcích jejich u jedné každé brány; nebo tak byl rozkaz Davida muže Božího.
\par 15 Aniž se uchýlili od rozkázaní králova knežím a Levítum pri všeliké veci i pri pokladích.
\par 16 A když dostrojeno bylo všecko dílo Šalomounovo od toho dne, v nemž založen byl dum Hospodinuv, až do vystavení jeho, a tak dokonán byl dum Hospodinuv,
\par 17 Tedy jel Šalomoun do Aziongaber a do Elat pri brehu morském v zemi Idumejské.
\par 18 I poslal jemu Chíram po služebnících svých lodí, a služebníky umelé na mori, kteríž plavivše se s služebníky Šalomounovými do Ofir, nabrali odtud ctyri sta a padesáte centnéru zlata, a prinesli je králi Šalomounovi.

\chapter{9}

\par 1 Královna pak z Sáby uslyševši povest o Šalomounovi, prijela do Jeruzaléma, aby zkusila Šalomouna v pohádkách, s vojskem velmi velikým a s velbloudy, na nichž prinesla vonných vecí, a zlata velmi mnoho i kamení drahého. Prišla tedy k Šalomounovi, a mluvila s ním o všecko, což mela v srdci svém.
\par 2 Jížto odpovedel Šalomoun na všecka slova její. Nebylo nic skrytého pred Šalomounem, nac by jí neodpovedel.
\par 3 Protož uzrevši královna z Sáby moudrost Šalomounovu, a dum, kterýž byl ustavel,
\par 4 I pokrmy stolu jeho, též sedání a stávání služebníku jeho prisluhujících jemu, i roucha jejich, šenkýre také jeho a odev jejich, i stupne, kterýmiž vstupoval k domu Hospodinovu: zdesila se náramne,
\par 5 A rekla králi: Pravát jest rec, kterouž jsem slyšela v zemi své o vecech tvých a o moudrosti tvé.
\par 6 Však jsem nechtela veriti recem jejich, až jsem prijela a uzrela ocima svýma, a aj, není mi praveno ani polovice o velikosti moudrosti tvé. Prevýšil jsi povest tu, kterouž jsem slyšela.
\par 7 Blahoslavení muži tvoji, a blahoslavení služebníci tvoji tito, kteríž stojí pred tebou vždycky, a slyší moudrost tvou.
\par 8 Budiž Hospodin Buh tvuj požehnaný, kterýž te sobe oblíbil, aby te posadil na stolici své, abys byl králem na míste Hospodina Boha tvého. Proto že miluje Buh tvuj Izraele, aby jej utvrdil na veky, ustanovil te nad nimi králem, abys cinil soud a spravedlnost.
\par 9 I dala králi sto a dvadceti centnéru zlata, a vonných vecí velmi mnoho, i kamení drahého, aniž bylo kdy privezeno takových vonných vecí, jakéž darovala královna z Sáby Šalomounovi.
\par 10 K tomu také i služebníci Chíramovi, a služebníci Šalomounovi, kteríž byli privezli zlata z Ofir, privezli dríví algumim a kamení drahého.
\par 11 I nadelal král z toho dríví algumim stupnu k domu Hospodinovu i k domu královu, též harf a louten zpevákum, aniž kdy prvé vídány takové veci v zemi Judské.
\par 12 Král také Šalomoun dal královne z Sáby vedlé vší vule její, cehož požádala, krome toho, což byla prinesla k králi. Potom se navrátila a odjela do zeme své, ona i služebníci její.
\par 13 Byla pak váha toho zlata, kteréž pricházelo Šalomounovi na každý rok, šest set šedesáte a šest centnéru zlata,
\par 14 Krome toho, což kupci a prodavaci prinášeli, a všickni králové Arabští. A vývodové té zeme priváželi zlato a stríbro Šalomounovi.
\par 15 A protož nadelal král Šalomoun dve ste štítu z zlata taženého; šest set lotu zlata taženého dával na každý štít.
\par 16 A tri sta pavéz z zlata taženého; tri sta lotu zlata dal na každou pavézu. I složil je král v dome lesu Libánského.
\par 17 Udelal také král stolici z kostí slonových velikou, a obložil ji zlatem cistým.
\par 18 A bylo šest stupnu k té stolici, a podnože té stolice také byly z zlata, držící se stolice, ano i spolehadla rukám s obou stran, tu kdež se sedalo, a dva lvové stáli u spolehadel.
\par 19 Dvanácte též lvu stálo tu na šesti stupních s obou stran. Nebylo nic udeláno takového v žádném království.
\par 20 Nadto všecky nádoby krále Šalomouna, jichž ku pití užívali, byly zlaté, a všecky nádoby v dome lesu Libánského byly z zlata nejcistšího. Nic nebylo z stríbra, aniž ho sobe co vážili za dnu Šalomounových.
\par 21 Nebo mel král lodí, kteréž precházely pres more s služebníky Chíramovými. Jednou ve trech letech vracovaly se ty lodí morské, prinášející zlato a stríbro, kosti slonové a opice a pávy.
\par 22 I zveleben jest král Šalomoun nad všecky krále zemské v bohatství a v moudrosti.
\par 23 Procež všickni králové zeme žádostivi byli videti tvár Šalomounovu, aby slyšeli moudrost jeho, kterouž složil Buh v srdci jeho.
\par 24 Z nichž jeden každý prinášeli také dar svuj, nádoby stríbrné a nádoby zlaté, roucha a zbroj, i vonné veci, kone a mezky, každého roku,
\par 25 Tak že mel Šalomoun ctyri tisíce stájí koní a vozu, a dvanácte tisíc jezdcu, kteréž rozsadil v mestech vozu a pri králi v Jeruzaléme.
\par 26 I panoval nade všemi králi od reky Eufrates až k zemi Filistinské, a až k koncinám Egyptským.
\par 27 A složil král stríbra v Jeruzaléme jako kamení, a cedrového dríví jako planého fíkoví, kteréž roste v údolí u velikém množství.
\par 28 Privodili také Šalomounovi kone z Egypta i ze všech zemí.
\par 29 Jiné pak veci Šalomounovy, první i poslední, vypsány jsou v knize Nátana proroka, a v proroctví Achiáše Silonského, a u videních Jaaddy proroka o Jeroboámovi synu Nebatovu.
\par 30 A kraloval Šalomoun v Jeruzaléme nade vším Izraelem ctyridceti let.
\par 31 I usnul Šalomoun s otci svými, a pochovali jej v meste Davida otce jeho. Kraloval pak Roboám syn jeho místo neho.

\chapter{10}

\par 1 Tedy odšel Roboám do Sichem; nebo tam sešel se byl všecken Izrael, aby ho ustanovili za krále.
\par 2 Stalo se pak, když o tom uslyšel Jeroboám syn Nebatuv, jsa v Egypte, kamž byl utekl pred králem Šalomounem, navrátil se Jeroboám z Egypta.
\par 3 Nebo poslali a povolali ho. Tedy prišed Jeroboám i všecken Izrael, mluvili k Roboámovi, rkouce:
\par 4 Otec tvuj stížil jho naše, protož nyní polehc služby otce svého tvrdé a bremene jeho težkého, kteréž vložil na nás, a budeme tobe sloužiti.
\par 5 Kterýž rekl jim: Po trech dnech navratte se ke mne. I odšel lid.
\par 6 V tom radil se král Roboám s starci, kteríž stávali pred Šalomounem otcem jeho, ješte za života jeho, rka: Kterak vy radíte, jakou odpoved mám dáti lidu tomu?
\par 7 I odpovedeli jemu, rkouce: Jestliže se dobrotive ukážeš lidu tomuto a povolíš jim, a mluviti budeš prívetive, budou služebníci tvoji po všecky dny.
\par 8 Ale on opustil radu starcu, kterouž dali jemu, a radil se s mládenci, kteríž odrostli s ním, a stávali pred ním.
\par 9 A rekl jim: Co vy radíte, jakou máme dáti odpoved lidu tomuto, kteríž mluvili ke mne, rkouce: Polehc bremene, kteréž vložil otec tvuj na nás?
\par 10 Jemuž odpovedeli mládenci, kteríž zrostli s ním, rkouce: Takto odpovíš lidu tomu, kteríž mluvili k tobe a rekli: Otec tvuj stížil jho naše, ty pak polehc nám, takto díš jim: Nejmenší prst muj tlustší jest, nežli bedra otce mého.
\par 11 Nyní tedy, otec muj težké bríme vzložil na vás, já pak pridám bremene vašeho; otec muj trestal vás bicíky, ale já bici uzlovatými.
\par 12 Potom prišel Jeroboám i všecken lid k Roboámovi dne tretího, jakž byl vyrkl král, rka: Navratte se ke mne dne tretího.
\par 13 I odpovedel jim král tvrde. Nebo opustil král Roboám radu starcu,
\par 14 A mluvil k nim vedlé rady mládencu, rka: Otec muj stížil jho vaše, já pak k nemu pridám; otec muj trestal vás bicíky, ale já bici uzlovatými.
\par 15 A tak neuposlechl král lidu. (Prícina zajisté byla od Boha, aby naplnil Hospodin rec svou, kterouž mluvil skrze Achiáše Silonského k Jeroboámovi synu Nebatovu.)
\par 16 Protož vida všecken Izrael, že by je král oslyšel, odpovedel lid králi, rkouce: Jakýž máme díl v Davidovi? Ani dedictví nemáme v synu Izai. Jeden každý k stanum svým, ó Izraeli! Nyní opatr dum svuj, Davide. Odšel tedy všecken Izrael k stanum svým,
\par 17 Tak že nad syny Izraelskými toliko, kteríž bydlili v mestech Judských, kraloval Roboám.
\par 18 A když poslal král Roboám Adurama, kterýž byl nad platy, uházeli ho synové Izraelští kamením až do smrti, címž král Roboám prinucen byl, aby vsedna na vuz, utekl do Jeruzaléma.
\par 19 A tak odstoupili synové Izraelští od domu Davidova až do dnešního dne.

\chapter{11}

\par 1 Když pak prijel Roboám do Jeruzaléma, shromáždil dum Juduv a Beniaminuv, sto a osmdesáte tisíc výborných bojovníku, aby bojovali proti Izraelovi, a aby zase obráceno bylo království k Roboámovi.
\par 2 Tedy stala se rec Hospodinova k Semaiášovi, muži Božímu, rkoucí:
\par 3 Povez Roboámovi synu Šalomounovu, králi Judskému, a všemu lidu Izraelskému, kterýž jest v pokolení Judove a Beniaminove, rka:
\par 4 Takto praví Hospodin: Netáhnete a nebojujte proti bratrím svým, navratte se jeden každý do domu svého; nebo ode mne stala se vec tato. I uposlechli rozkazu Hospodinova, a navrátili se, aby netáhli proti Jeroboámovi.
\par 5 I bydlil Roboám v Jeruzaléme, a vzdelal mesta hrazená v Judstvu.
\par 6 A vzdelal Betlém, Etam. a Tekoe,
\par 7 Betsur, Socho a Adulam,
\par 8 Gát, Maresa a Zif,
\par 9 Adoraim, Lachis a Azeka,
\par 10 Zaraha, Aialon a Hebron, kteráž byla v pokolení Judove a Beniaminove mesta hrazená.
\par 11 A když upevnil mesta hrazená, osadil v nich knížata, a zdelal špižírny k potravám, k oleji a vínu.
\par 12 A v jednom každém meste složil štíty a kopí, a opatril ta mesta velmi dobre, a tak kraloval nad Judou a Beniaminem.
\par 13 Kneží také a Levítové, kteríž byli ve všem Izraeli, postavili se k nemu ze všech koncin svých.
\par 14 Opustili byli zajisté Levítové predmestí svá i vládarství svá, a odebrali se do Judstva a do Jeruzaléma, (proto, že je zavrhl Jeroboám a synové jeho, aby v úradu knežském nesloužili Hospodinu.
\par 15 A narídil sobe kneží k sloužení po výsostech dáblum a telatum, kterýchž byl nadelal).
\par 16 A za nimi ze všech pokolení Izraelských ti, kteríž oddali srdce své k hledání Hospodina Boha Izraelského, prišli do Jeruzaléma, aby obetovali Hospodinu Bohu otcu svých.
\par 17 A tak utvrdili království Judské, a zsilili Roboáma syna Šalomounova za tri léta, a po ta tri léta chodili po ceste Davidove a Šalomounove.
\par 18 Potom pojal sobe Roboám ženu, Mahalat, dceru Jerimota syna Davidova, a Abichail, dceru Eliaba syna Izai.
\par 19 Kteráž mu zplodila syny: Jeusa a Semariáše a Zahama.
\par 20 A po té pojal Maachu dceru Absolonovu, kteráž mu porodila Abiáše, Attaie, Zizu a Selomita.
\par 21 Ale miloval Roboám Maachu dceru Absolonovu, nade všecky ženy i ženiny své; nebo byl pojal žen osmnáct, a ženin šedesát, a zplodil dvadceti a osm synu a šedesáte dcer.
\par 22 Ustanovil pak Roboám Abiáše syna Maachy za kníže a vývodu mezi bratrími jeho; nebo myslil ho ustanoviti králem.
\par 23 A opatrnosti užívaje, rozsadil všecky jiné syny své po všech krajích Judových a Beniaminových, ve všech mestech ohrazených, i dal jim potravy hojne, a nabral jim mnoho žen.

\chapter{12}

\par 1 I stalo se, když utvrdil království Roboám a zmocnil je, opustil zákon Hospodinuv, i všecken Izrael s ním.
\par 2 Stalo se, pravím, léta pátého království Roboámova, že vytáhl Sesák král Egyptský proti Jeruzalému, (nebo byli zhrešili proti Hospodinu),
\par 3 S tisícem a dvema sty vozu, a s šedesáti tisíci jízdných, a nebylo poctu lidu, kterýž pritáhl s ním z Egypta, Lubimských, Sukkimských a Chussimských.
\par 4 A pobral mesta hrazená, kteráž byla v Judstvu, a pritáhl až k Jeruzalému.
\par 5 Tedy Semaiáš prorok prišel k Roboámovi a k knížatum Judským, kteríž se byli sebrali do Jeruzaléma, bojíce se Sesáka, a rekl jim: Toto praví Hospodin: Vy jste mne opustili, i já také opouštím vás v rukou Sesákových.
\par 6 I ponížili se knížata Izraelská i král, a rekli: Spravedlivýt jest Hospodin.
\par 7 A když videl Hospodin, že se ponížili, stalo se slovo Hospodinovo k Semaiášovi, rkoucí: Ponížilit jsou se, neshladím jich, ale dám jim tudíž vysvobození, aniž se vyleje prchlivost má na Jeruzalém skrze Sesáka.
\par 8 A však sloužiti jemu budou, aby zkusili, co jest to mne sloužiti, a sloužiti králum zemským.
\par 9 A tak vstoupiv Sesák král Egytský do Jeruzaléma, pobral poklady domu Hospodinova, i poklady domu královského, všecko to pobral. Vzal i pavézy zlaté, kterýchž byl nadelal Šalomoun.
\par 10 I nadelal král Roboám místo nich pavéz medených, a porucil je úredníkum nad drabanty, kteríž ostríhali brány domu královského.
\par 11 A když král chodíval do domu Hospodinova, pricházívali drabanti a nosili je; potom též zase prinášeli je do pokoje drabantu.
\par 12 Když se pak ponížil, odvrátila se od neho prchlivost Hospodinova, a neshladil ho do cela, tak jakož se k Judovi v dobrých vecech ohlašoval.
\par 13 Tedy zmocnil se král Roboám v Jeruzaléme a kraloval; nebo ve ctyridcíti a v jednom léte byl Roboám, když pocal kralovati, a sedmnácte let kraloval v Jeruzaléme, meste, kteréž vyvolil Hospodin ze všech pokolení Izraelských, aby tam prebývalo jméno jeho. Jméno matky jeho bylo Naama Ammonitská.
\par 14 Kterýž cinil zlé veci; nebo neustavil srdce svého, aby hledal Hospodina.
\par 15 Veci pak Roboámovy, první i poslední, zdaž zapsány nejsou v knihách Semaiáše proroka, a Iddo proroka, kdež se vycítá porádek rodu, ano i války mezi Roboámem a Jeroboámem, po všecky dny?
\par 16 I usnul Roboám s otci svými, a pochován jest v meste Davidove, a kraloval Abiáš syn jeho místo neho.

\chapter{13}

\par 1 Léta osmnáctého krále Jeroboáma kraloval Abiáš nad Judou.
\par 2 Tri léta kraloval v Jeruzaléme, (a jméno matky jeho Michaia, dcera Urielova z Gabaa), a byla válka mezi Abiášem a Jeroboámem.
\par 3 Procež sšikoval Abiáš vojsko udatných bojovníku, ctyrikrát sto tisíc mužu výborných, Jeroboám pak sšikoval se proti nemu, maje osmkrát sto tisíc mužu výborných, velmi udatných.
\par 4 I postavil se Abiáš na vrchu hory Semaraim, kteráž byla mezi horami Efraimskými, a rekl: Slyšte mne, Jeroboáme i všecken Izraeli.
\par 5 Zdaliž jste nemeli vedeti, že Hospodin Buh Izraelský dal království Davidovi nad Izraelem na veky, jemu i synum jeho smlouvou trvánlivou?
\par 6 Ale povstal Jeroboám syn Nebatuv, služebník Šalomouna syna Davidova, a zprotivil se pánu svému.
\par 7 A sebrali se k nemu lidé nevážní a bezbožní a zsilili se proti Roboámovi synu Šalomounovu, Roboám pak byv díte a srdce choulostivého, neodeprel jim zmužile.
\par 8 Tak vy nyní myslíte zsiliti se proti království Hospodinovu, kteréž jest v ruce synu Davidových, a jest vás veliké množství; máte také pri sobe telata zlatá, kterýchž vám nadelal Jeroboám, za bohy.
\par 9 Zdaliž jste nezavrhli kneží Hospodinových, synu Aronových a Levítu, a nadelali jste sobe kneží, tak jako národové jiných zemí? Kdožkoli prichází, aby posveceny byly ruce jeho volcetem mladým a sedmi skopci, hned je knezem tech, jenž nejsou bohové.
\par 10 Ale my jsme Hospodina Boha našeho, aniž jsme se ho spustili, kneží pak prisluhující Hospodinu jsou synové Aronovi a Levítové, kteríž konají práci svou,
\par 11 A pálí Hospodinu zápaly každého jitra a každého vecera, kadí také vonnými vecmi, zporádaní také chlebu na stole cistém, a svícen zlatý s lampami jeho spravují, aby horely každého vecera. A tak my ostríháme narízení Hospodina Boha svého, ale vy strhli jste se jeho.
\par 12 Protož aj, s námi jest Buh vudce náš, a kneží jeho a trouby zvucné, aby znely proti vám. Synové Izraelští, nebojujte s Hospodinem Bohem otcu svých, nebo nepovede se vám štastne.
\par 13 Mezi tím Jeroboám obvedl zálohy, aby po zadu na ne pripadli, a tak byli Izraelští pred Judskými, a zálohy ty byly jim pozadu.
\par 14 Tedy spatriv Juda, an válka jim s predu i s zadu, zvolali k Hospodinu, a kneží troubili v trouby.
\par 15 Ucinili také pokrik muži Judští. I stalo se v tom pokriku mužu Judských, ranil Buh Jeroboáma i všecken Izrael pred Abiášem a Judou.
\par 16 A tak utíkali synové Izraelští pred Judou, ale dal je Buh v ruku jejich.
\par 17 Nebo porazili je Abiáš a lid jeho ranou velikou, tak že padlo zbitých z Izraele petkrát sto tisíc mužu výborných.
\par 18 Protož sníženi jsou synové Izraelští v ten cas, a zmocnili se synové Judovi, nebo zpolehli na Hospodina Boha otcu svých.
\par 19 I honil Abiáš Jeroboáma, a vzal mu mesta, Bethel i vsi jeho, Jesana i vsi jeho, a Efron i vsi jeho.
\par 20 Aniž se mohl zase zsiliti více Jeroboám po všecky dny Abiášovy. I ranil jej Hospodin, tak že umrel.
\par 21 Ale Abiáš zmocnil se, kterýž byl sobe pojal žen ctrnáct, a zplodil dvamecítma synu a šestnácte dcer.
\par 22 Ostatek pak cinu Abiášových, a života jeho i recí jeho, sepsáno jest v knize proroka Iddo.

\chapter{14}

\par 1 Když pak usnul Abiáš s otci svými, a pochovali jej v meste Davidove, kraloval Aza syn jeho místo neho. Za jeho dnu v pokoji byla zeme deset let.
\par 2 I cinil Aza to, což se dobre líbilo Hospodinu Bohu jeho.
\par 3 Nebo zboril oltáre cizí i výsosti, a stroskotal obrazy jejich, a posekal háje jejich.
\par 4 A prikázal Judovi, aby hledali Hospodina Boha otcu svých, a ostríhali zákona a prikázaní jeho.
\par 5 Zkazil, pravím, po všech mestech Judských výsosti a slunecné obrazy, a bylo v pokoji království za casu jeho.
\par 6 Zatím vzdelal mesta hrazená v Judstvu, proto že v pokoji byla zeme, aniž jaká proti nemu válka povstala tech let; nebo Hospodin dal jemu odpocinutí.
\par 7 I rekl lidu Judskému: Vzdelejme ta mesta, a ohradme je zdmi a vežemi, branami i závorami, dokudž zeme jest v moci naší. Aj, že jsme hledali Hospodina Boha svého, hledali jsme ho, a dal nám odpocinutí odevšad. A tak staveli a štastne se jim zvedlo.
\par 8 Mel pak Aza vojsko tech, kteríž nosili štíty a kopí, z pokolení Judova trikrát sto tisícu, a z Beniaminova pavézníku a strelcu dve ste a osmdesáte tisícu. Všickni ti byli muži udatní.
\par 9 I vytáhl proti nim Zerach Mourenín, maje v vojšte desetkrát sto tisícu, a vozu tri sta, a pritáhl až k Maresa.
\par 10 Vytáhl též i Aza proti nemu. I sšikovali vojska v údolí Sefata u Maresa.
\par 11 Tedy volal Aza k Hospodinu Bohu svému, a rekl: Hospodine, nenít potrebí tobe velikého množství, když ty chceš pomoci mdlejším. Pomoziž nám, Hospodine Bože náš, nebot v tebe doufáme, a ve jménu tvém jdeme proti množství tomuto. Hospodine, ty jsi Buh náš; necht nemá moci proti tobe bídný clovek.
\par 12 I ranil Hospodin Moureníny pred Azou a pred lidem Judským, tak že utíkali Mourenínové.
\par 13 A honil je Aza i lid, kterýž byl s ním, až do Gerar. I padli Mourenínové, že se nijakž otaviti nemohli; nebo potríni jsou pred Hospodinem a pred vojskem jeho. I odnesli onino koristí velmi mnoho.
\par 14 Pohubili také všecka mesta vukol Gerar; strach zajisté Hospodinuv pripadl na ne. I vzebrali všecka mesta; nebo mnoho koristí v nich bylo.
\par 15 Též i obyvatele v staních pri dobytcích zbili, a zajavše ovec velmi mnoho a velbloudu, navrátili se do Jeruzaléma.

\chapter{15}

\par 1 Tehdy na Azariáše syna Odedova sstoupil duch Boží.
\par 2 Procež vyšed vstríc Azovi, rekl jemu: Slyšte mne, Azo i všecko pokolení Judovo a Beniaminovo. Hospodin byl s vámi, dokudž jste byli s ním, a budete-li ho hledati, najdete ho; pakli ho opustíte, opustít vás.
\par 3 Po mnohé zajisté dny Izrael jest bez pravého Boha a bez kneží, ucitelu i bez zákona,
\par 4 Ješto, kdyby se v úzkosti své k Hospodinu Bohu Izraelskému byli obrátili a hledali ho, bylit by ho nalezli.
\par 5 Ale casu techto není bezpecno vycházeti ani vcházeti; nebo nepokoj veliký jest mezi všemi obyvateli zeme,
\par 6 Tak že šlapá národ po národu, a mesto po mestu, proto že Buh kormoutí je všelijakými úzkostmi.
\par 7 Protož vy posilnte se, a neopouštejte rukou svých; nebo má mzdu práce vaše.
\par 8 A když slyšel Aza slova ta a proroctví Odeda proroka, posilil se, a vyplénil ohavnosti ze vší zeme Judské a Beniaminské, i z mest, kteráž byl vzal na hore Efraim, a obnovil oltár Hospodinuv, kterýž byl pred síní Hospodinovou.
\par 9 Potom shromáždil všecko pokolení Judovo i Beniaminovo, a príchozí s nimi z Efraima, Manassesa a Simeona; nebo jich bylo uteklo k nemu z lidu Izraelského velmi mnoho, vidouce, že Hospodin Buh jeho jest s ním.
\par 10 I sebrali se do Jeruzaléma tretího mesíce, patnáctého léta kralování Azova,
\par 11 A obetovali Hospodinu v ten den z koristí prihnaných, volu sedm set a ovcí sedm tisíc.
\par 12 A vešli v smlouvu, aby hledali Hospodina Boha otcu svých, z celého srdce svého a ze vší duše své,
\par 13 A kdož by koli nehledal Hospodina Boha Izraelského, aby byl usmrcen, bud malý neb veliký, bud muž neb žena.
\par 14 I prisáhli Hospodinu hlasem velikým, s zvukem na trouby i na pozouny.
\par 15 Všecken zajisté lid Judský radoval se z prísahy té; nebo celým srdcem svým prisáhli, a se vší ochotností hledali ho, a nalezli jej. I dal jim odpocinutí Hospodin se všech stran.
\par 16 Nadto i Maachu matku ssadil Aza král, aby nebyla královnou, proto že byla vzdelala v háji hroznou modlu. I podtal Aza modlu tu hroznou, a zdrobil i spálil ji pri potoku Cedron.
\par 17 Ackoli výsosti nebyly zkaženy v lidu Izraelském, srdce však Azovo bylo celé po všecky dny jeho.
\par 18 Vnesl také ty veci, kteréž posveceny byly od otce jeho, i to, cehož sám posvetil, do domu Božího, stríbro, zlato i nádoby.
\par 19 A nebylo války až do léta tridcátého pátého kralování Azova.

\chapter{16}

\par 1 Léta tridcátého šestého kralování Azova vytáhl Báza král Izraelský proti Judovi, a stavel Ráma, aby nedal vyjíti ani jíti k Azovi králi Judskému.
\par 2 Tedy vzav Aza stríbro i zlato z pokladu domu Hospodinova i domu královského, poslal k Benadadovi králi Syrskému, kterýž bydlil v Damašku, rka:
\par 3 Smlouva jest mezi mnou a mezi tebou, mezi otcem mým a mezi otcem tvým. Aj, posílámt ted stríbro a zlato, jdi, zruš smlouvu svou s Bázou králem Izraelským, at odtrhne ode mne.
\par 4 I uposlechl Benadad krále Azy, a poslal knížata s vojsky svými proti mestum Izraelským. I dobyli Jon a Dan, též Abelmaim i všech mest Neftalímových, v nichž meli sklady.
\par 5 To když uslyšel Báza, prestal staveti Ráma, a zanechal díla svého.
\par 6 V tom Aza král pojal všecken lid Judský, a pobrali kamení z Ráma i dríví, z nehož stavel Báza, a vystavel z neho Gabaa a Masfa.
\par 7 V ten pak cas prišel Chanani prorok k Azovi králi Judskému, a rekl jemu: Že jsi zpolehl na krále Syrského, a nezpolehl jsi na Hospodina Boha svého, proto ušlo vojsko krále Syrského z ruky tvé.
\par 8 Zdaliž jsou Chussimští a Lubimští nemeli vojsk velmi velikých, s vozy a jezdci náramne mnohými? A když jsi zpolehl na Hospodina, vydal je v ruku tvou.
\par 9 Oci zajisté Hospodinovy spatrují všecku zemi, aby dokazoval síly své pri tech, kteríž jsou k nemu srdce uprímého. Bláznive jsi ucinil v té veci, protož od toho casu budou proti tobe války.
\par 10 Tedy rozhnevav se Aza na proroka, dal ho do vezení; nebo se byl rozhneval na nej pro tu vec. A utiskal Aza nekteré z lidu toho casu.
\par 11 O jiných pak vecech Azových, prvních i posledních, zapsáno jest v knize o králích Judských a Izraelských.
\par 12 Potom léta tridcátého devátého kralování svého nemocen byl Aza na nohy své težkou nemocí, a však v nemoci své nehledal Boha, ale lékaru.
\par 13 A tak usnul Aza s otci svými, a umrel léta ctyridcátého prvního kralování svého.
\par 14 A pochovali jej v hrobe jeho, kterýž byl vytesal sobe v meste Davidove, a položili ho na lužku, kteréž byl naplnil vonnými vecmi a mastmi dílem apatykárským pripravenými. I pálili to jemu ohnem velmi velikým.

\chapter{17}

\par 1 I kraloval Jozafat syn jeho místo neho, a zmocnil se Izraele.
\par 2 I osadil lidem válecným všecka mesta hrazená v Judstvu, a postavil stráž v zemi Judské a v mestech Efraim, kteráž byl zdobýval Aza otec jeho.
\par 3 A byl Hospodin s Jozafatem; nebo chodil po cestách Davida otce svého prvních, aniž hledal modl.
\par 4 Ale Boha otce svého hledal, a v prikázaních jeho chodil, aniž následoval skutku lidu Izraelského.
\par 5 I utvrdil Hospodin království v ruce jeho, a dal všecken lid Judský dary Jozafatovi, tak že mel bohatství a slávy hojne.
\par 6 A nabyv udatného srdce k cestám Hospodinovým, zkazil presto také i výsosti a háje v Judstvu.
\par 7 Léta pak tretího království svého poslal knížata svá: Benchaile, Abdiáše, Zachariáše, Natanaele a Micheáše, aby ucili v mestech Judských.
\par 8 A s nimi Levíty: Semaiáše, Netaniáše, Zebadiáše, Azaele, Semiramota, Jonatana, Adoniáše, Tobiáše a Tobadoniáše, Levíty, a s nimi Elisama a Jehorama, kneží,
\par 9 Kteríž ucili v Judstvu, majíce s sebou knihu zákona Hospodinova. Chodili pak vukol po všech mestech Judských, vyucujíce lid.
\par 10 I byl strach Hospodinuv na všech královstvích zemí, kteréž byly vukol Judstva, tak že nebojovali proti Jozafatovi.
\par 11 Nýbrž i Filistinští prinášeli Jozafatovi dary a berni uloženou. Arabští také priháneli jemu dobytky, skopcu po sedmi tisících a sedmi stech, též kozlu po sedmi tisících a sedmi stech.
\par 12 I prospíval Jozafat, a rostl až na nejvyšší, a vystavel v Judstvu zámky a mesta k skladum.
\par 13 A práci mnohou vedl pri mestech Judských, muže pak válecné a slovoutné mel v Jeruzaléme.
\par 14 A tento jest pocet jich po domích otcu jejich: Z Judy knížata nad tisíci: Adna kníže, a s ním udatných rytíru trikrát sto tisíc.
\par 15 A po nem Jochanan kníže, a s ním dve ste a osmdesát tisíc.
\par 16 Po nem Amaziáš syn Zichruv, kterýž se dobrovolne byl oddal Hospodinu, a pri nem dvakrát sto tisíc mužu udatných.
\par 17 Z Beniamina pak muž udatný Eliada, a s ním lidu zbrojného s lucišti a pavézami dvakrát sto tisíc.
\par 18 A po nem Jozabad, a s ním sto a osmdesát tisíc zpusobných k boji.
\par 19 Ti sloužili králi, krome tech, kteréž byl osadil král po mestech hrazených po všem Judstvu.

\chapter{18}

\par 1 I mel Jozafat bohatství a slávu velmi velikou, a spríznil se s Achabem.
\par 2 Prijel pak po letech k Achabovi do Samarí, a nabil Achab ovcí a volu hojne pro nej a pro lid, kterýž byl s ním, a namlouval ho, aby táhl do Rámot Galád.
\par 3 Nebo rekl Achab král Izraelský Jozafatovi králi Judskému: Potáhneš-li se mnou do Rámot Galád? Odpovedel jemu: Jakž jsem já, tak jsi ty, jako lid tvuj, tak lid muj, a s tebout budu v tom boji.
\par 4 Rekl také Jozafat králi Izraelskému: Vzeptej se medle dnes na slovo Hospodinovo.
\par 5 I shromáždil král Izraelský proroku svých ctyri sta mužu, a rekl jim: Máme-li jeti do Rámot Galád na vojnu, cili tak nechati? I rekli: Táhni, nebo dá je Buh v ruku královu.
\par 6 Tedy rekl Jozafat: Což již není zde proroka Hospodinova žádného, abychom se ho zeptali?
\par 7 Odpovedel král Izraelský Jozafatovi: Ještet jest muž jeden, skrze nehož bys se mohl poraditi s Hospodinem, ale já ho nenávidím, proto že mi nikdy nic dobrého neprorokuje, ale vždycky zlé. Ten jest Micheáš syn Jemluv. Ale Jozafat rekl: Necht tak nemluví král.
\par 8 Protož povolav král Izraelský komorníka jednoho, rekl: Prived sem rychle Micheáše syna Jemlova.
\par 9 (Mezi tím král Izraelský a Jozafat král Judský sedeli jeden každý na stolici své, odíni jsouce rouchem. Sedeli pak v placu u vrat brány Samarské, a všickni proroci prorokovali pred nimi.
\par 10 Kdežto Sedechiáš syn Kenanuv udelal sobe byl i rohy železné, a rekl: Takto praví Hospodin: Temito trkati budeš Syrské, dokudž jich nepohubíš.
\par 11 Tak podobne všickni proroci prorokovali, rkouce: Jed do Rámot Galád, a štastnet se povede; nebo dá je Hospodin v ruku královu.)
\par 12 V tom posel ten, kterýž šel, aby zavolal Micheáše, mluvil jemu, rka: Aj, reci proroku tech jednostejné jsou a dobré pred králem. Medle, bud rec tvá, jako rec kterého z nich, a mluv dobré veci.
\par 13 Jemuž rekl Micheáš: Živt jest Hospodin, že cožkoli vyrkne Buh muj, to mluviti budu.
\par 14 A když prišel k králi, rekl jemu král: Micheáši, máme-li jeti proti Rámot Galád na vojnu, cili tak nechati? Kterýž rekl: Táhnete a štastnet se vám povede, i budou dáni v ruku vaši.
\par 15 Ješte rekl jemu král: I kolikrátž te mám prísahou zavazovati, abys mi nemluvil než pravdu ve jménu Hospodinovu?
\par 16 Protož rekl: Videl jsem všecken lid Izraelský rozptýlený po horách jako ovce, kteréž nemají pastýre; nebo rekl Hospodin: Nemají pána tito, navrat se jeden každý do domu svého v pokoji.
\par 17 I rekl král Izraelský Jozafatovi: Zdaližt jsem nerekl, že mi nebude prorokovati nic dobrého, ale zlé?
\par 18 Rekl dále: Protož slyšte slovo Hospodinovo: Videl jsem Hospodina sedícího na trunu svém, a všecko vojsko nebeské stojící po pravici jeho i po levici jeho.
\par 19 I rekl Hospodin: Kdo oklamá Achaba krále Izraelského, aby vytáhl a padl u Rámot Galád? A když pravil ten toto, a jiný pravil jiné,
\par 20 Tožt vyšel duch, a postaviv se pred Hospodinem, rekl: Já ho oklamám. Hospodin pak rekl jemu: Jakým zpusobem?
\par 21 Odpovedel: Vyjdu a budu duchem lživým v ústech všech proroku jeho. Kterýžto rekl: Oklamáš a dovedeš toho; jdiž a ucin tak.
\par 22 Protož aj, jižte dal Hospodin ducha lživého v ústa proroku tvých techto, ješto však Hospodin mluvil zlé proti tobe.
\par 23 Tedy pristoupiv Sedechiáš syn Kenanuv, dal polícek Micheášovi a rekl: Kterouž jest cestou odšel duch Hospodinuv ode mne, aby mluvil tobe?
\par 24 Odpovedel Micheáš: Aj, uzríš v ten den, když vejdeš do nejtajnejšího pokoje, abys se skryl.
\par 25 I rekl král Izraelský: Jmete Micheáše, a dovedte ho k Amonovi hejtmanu mesta, a k Joasovi synu královu.
\par 26 A díte: Takto praví král: Dejte tohoto do žaláre a dávejte mu jísti malicko chleba a malicko vody, dokudž se nenavrátím v pokoji.
\par 27 Ale Micháš rekl: Jestliže se ty navrátíš v pokoji, tedyt nemluvil Hospodin skrze mne. Presto rekl: Slyštež to všickni lidé.
\par 28 A tak táhl král Izraelský a Jozafat král Judský proti Rámot Galád.
\par 29 I rekl král Izraelský Jozafatovi: Zmením já se, když pujdu k bitve, ale ty oblec se v roucho své. I zmenil se král Izraelský, a jeli k bitve.
\par 30 Král pak Syrský prikázal byl hejtmanum nad vozy svými, rka: Nebojujte proti malému ani proti velikému, než proti samému králi Izraelskému.
\par 31 I stalo se, když hejtmané nad vozy uzreli Jozafata, rekli: Král Izraelský jest. Takž se obrátili proti nemu, aby bojovali. Tedy zkrikl Jozafat, a Hospodin spomohl jemu, a odvrátil je Buh od neho.
\par 32 Nebo uzrevše hejtmané nad vozy, že on není král Izraelský, obrátili se od neho.
\par 33 Muž pak strelil z lucište náhodou, a postrelil krále Izraelského, kdež se pancír spojuje. Procež rekl vozkovi: Obrat se, a vyvez mne z vojska; nebo jsem nemocen.
\par 34 I rozmohla se bitva v ten den. Král pak Izraelský stál na voze proti Syrským až do vecera, i umrel v západ slunce.

\chapter{19}

\par 1 Když se pak navracoval Jozafat král Judský do domu svého v pokoji do Jeruzaléma,
\par 2 Vyšel jemu vstríc Jéhu syn Chananuv, prorok, a rekl králi Jozafatovi: Zdaliž jsi bezbožnému mel pomáhati, a ty, jenž nenávidí Hospodina, milovati? Z té príciny proti tobe jest hnev Hospodinuv.
\par 3 Ale však nalezly se veci dobré pri tobe, že jsi vysekal háje z zeme, a nastrojil srdce své, abys hledal Boha.
\par 4 I bydlil Jozafat v Jeruzaléme, a zase projel lid od Bersabé až k hore Efraim, a navrátil je k Hospodinu Bohu otcu jejich.
\par 5 A ustanovil soudce v zemi po všech mestech Judských hrazených, v jednom každém meste.
\par 6 Tedy rekl soudcum: Vizte, jak co ciníte; nebo nevedete soudu za cloveka, ale za Hospodina, kterýž vám prítomen jest pri vykonávání soudu.
\par 7 A protož budiž bázen Hospodinova pri vás. Ostríhejte toho a cinte tak, nebot není u Hospodina Boha našeho nepravosti, tak aby šetriti mel osob, aneb prijímati dary.
\par 8 Tak i v Jeruzaléme ustanovil Jozafat nekteré z Levítu a z kneží, a z knížat otcovských, celedí Izraelských, k soudu Hospodinovu a k rozeprem tech, kteríž by se obraceli o to do Jeruzaléma.
\par 9 A prikázal jim, rka: Takž delejte v bázni Hospodinove, u víre a v srdci uprímém.
\par 10 A pri všeliké rozepri, kteráž by prišla pred vás od bratrí vašich, kteríž bydlí v mestech svých, bud mezi krví a krví, mezi zákonem a prikázaním, ustanoveními a soudy, napomenete jich, aby nehrešili proti Hospodinu, tak aby neprišla prchlivost na vás, ani na bratrí vaše. Tak cinte, a neubehnete v hrích.
\par 11 A aj, Amariáš, knez nejvyšší, mezi vámi bude ve všech vecech Hospodinových, a Zebadiáš syn Izmaeluv, vývoda domu Judova, ve všeliké veci královské. Máte také Levíty správce mezi sebou; posilntež se a zmužile sobe pocínejte, a budet Hospodin s tím, kdož bude dobrý.

\chapter{20}

\par 1 I stalo se potom, že pritáhli synové Moáb a synové Ammon, a s nimi nekterí od Ammonitských, proti Jozafatovi na vojnu.
\par 2 A prišedše, oznámili Jozafatovi, rkouce: Pritáhlo proti tobe množství veliké z zámorí, z zeme Syrské, a aj, jsou v Hasesontamar, jenž jest Engadi.
\par 3 I ulekl se, a obrátil Jozafat tvár svou k hledání Hospodina, a vyhlásil pust všemu lidu Judskému.
\par 4 A tak shromáždil se lid Judský, aby hledali Hospodina. Také i ze všech mest Judských sešli se hledati Hospodina.
\par 5 Tedy stál Jozafat v shromáždení Judském a Jeruzalémském, v dome Hospodinove pred síní novou,
\par 6 A rekl: Hospodine, Bože otcu našich, zdaliž ty sám nejsi Bohem na nebi? Zdaliž ty nepanuješ nade všemi královstvími národu? Zdaliž v ruce tvé není síly a moci, tak že není, kdo by se mohl postaviti proti tobe?
\par 7 Zdaliž jsi ty, Bože náš, nevyhnal obyvatelu zeme této pred tvárí lidu svého Izraelského, a dal jsi ji semeni Abrahama, milovníka svého na veky?
\par 8 Kterížto bydlili v ní, a vzdelali tobe v ní svatyni, jménu tvému, rkouce:
\par 9 Jestliže by na nás prišly zlé veci, mec pomsty, bud morová rána, bud hlad, postavíme se pred tímto domem a pred tebou, (ponevadž jméno tvé jest v dome tomto), a budeme volati k tobe v úzkostech svých, i vyslyšíš a vysvobodíš.
\par 10 A nyní, aj, synové Ammon a Moáb, a hora Seir, skrze než jsi nedopustil jíti Izraelovi, když se brali z zeme Egyptské, ale uhnuli se od nich, a nepohubili jich,
\par 11 Aj hle, oni odplacejí se nám, pritáhše, aby nás vyhnali z dedictví tvého, kteréž jsi právem dedicným dal nám.
\par 12 Bože náš, zdali jich souditi nebudeš? V nást zajisté není žádné síly proti množství tomuto velikému, kteréž táhne proti nám, aniž my víme, co bychom ciniti meli, toliko na te patrí oci naše.
\par 13 Všecken také lid Judský stáli pred Hospodinem, též i dítky jich, ženy i synové jejich.
\par 14 Jachaziel pak syn Zachariáše syna Benaiášova, syna Jehielova, syna Mataniášova, Levíta z synu Azafových, nadšen jsa duchem Hospodinovým, u prostred toho shromáždení,
\par 15 Rekl: Pozorujte všecken Judo a obyvatelé Jeruzalémští, i ty králi Jozafate. Takto vám praví Hospodin: Nebojte se vy, ani se lekejte množství tohoto velikého; nebo ne váš bude boj, ale Boží.
\par 16 Zítra vytáhnete proti nim, aj, oni potáhnou po stráni Ziz, a naleznete je pri konci údolí naproti poušti Jeruel.
\par 17 Nebudete vy bojovati tuto. Postavte se, stujte a vizte vysvobození Hospodinovo pri sobe, ó Judo a Jeruzaléme. Nebojte se, aniž se strachujte; zítra vyjdete proti nim, a Hospodin bude s vámi.
\par 18 I sklonil se Jozafat tvárí k zemi, a všecken lid Judský i obyvatelé Jeruzalémští padli pred Hospodinem, klanejíce se Hospodinu.
\par 19 Vstali pak Levítové z synu Kahat a z synu Chóre, aby chválili Hospodina Boha Izraelského hlasem velikým a vysokým.
\par 20 Potom vstavše ráno, vytáhli na poušt Tekoe. A když vycházeli, stál Jozafat a rekl: Slyšte mne, Judo a obyvatelé Jeruzalémští. Verte v Hospodina Boha svého, a stane se vám verne; verte prorokum jeho, a štastne se vám povede.
\par 21 A tak poradiv se s lidem, postavil zpeváky Hospodinu, aby chválili okrasu svatosti, a když by vycházeli sšikovaní k boji, aby oni napred šli a ríkali: Oslavujte Hospodina, nebo na veky milosrdenství jeho.
\par 22 Tu chvíli pak, když oni zacali zpev a chválení, obrátil Hospodin ty, kteríž byli v zálohách, na syny Ammon, Moáb a obyvatele hory Seir, ješto však byli pritáhli proti Judovi, a tak sami se bili.
\par 23 Nebo povstali synové Ammon a Moábští proti obyvatelum hory Seir, aby zmordovali a shladili je. A když dokonali boj proti obyvatelum hory Seir, pomáhali sobe a hubili jedni druhé.
\par 24 Mezi tím lid Judský pritáhl k stráži, kteráž jest na poušti, a uzreli to množství, a aj, mrtví leží na zemi, aniž kdo ušel.
\par 25 Procež pristoupil Jozafat s lidem svým, aby rozebrali loupeže jejich, a nalezli u nich hojnost zboží i klénotu na telích zbitých. I rozbitovali toho mezi sebou, tak že unesti nemohli; za tri dni delili ty loupeže, proto že jich mnoho bylo.
\par 26 V den pak ctvrtý shromáždili se do údolí Beracha, a že tu dobrorecili Hospodinu, protož nazvali jméno místa toho údolí Beracha až do dnešního dne.
\par 27 Zatím obrátili se všickni muži Judští a Jeruzalémští, a Jozafat pred nimi, aby se navrátili do Jeruzaléma s veselím; nebo byl obveselil je Hospodin nad neprátely jejich.
\par 28 I vešli do Jeruzaléma s loutnami a harfami a s trubami do domu Hospodinova.
\par 29 Tedy pripadl strach Boží na všecka království zemská, když uslyšeli, že Hospodin bojoval proti neprátelum lidu Izraelského.
\par 30 A tak v pokoji bylo království Jozafatovo; nebo odpocinutí dal jemu Buh jeho odevšad.
\par 31 Kraloval pak Jozafat nad Judou. Ve tridcíti peti letech byl, když kralovati zacal, a petmecítma let kraloval v Jeruzaléme. Jméno matky jeho Azuba, dcera Silchi.
\par 32 A chodil po ceste Azy otce svého, aniž se uchýlil od ní, cine to, což pravého jest pred ocima Hospodinovýma.
\par 33 A však výsosti nebyly zkaženy, nebo ješte lid byl nenastrojil srdce svého k Bohu otcu svých.
\par 34 O jiných pak vecech Jozafatových,prvních i posledních, sepsáno jest v knihách Jéhu syna Chanani, kterémuž bylo poruceno, aby to vložil do knihy o králích Izraelských.
\par 35 Potom stovaryšil se Jozafat král Judský s Ochoziášem králem Izraelským, kterýž sobe bezbožne pocínal.
\par 36 Stovaryšil se pak s ním proto, aby nadelal lodí, kteréž by precházely pres more. I nadelali lodí v Aziongaber.
\par 37 Protož prorokoval Eliezer syn Dodavahuv z Maresa proti Jozafatovi, rka: Jakž jsi se stovaryšil s Ochoziášem, roztrhl Hospodin skutky tvé. I stroskotány jsou lodí, a tak nemohly se doplaviti pres more.

\chapter{21}

\par 1 Potom usnul Jozafat s otci svými, a pochován jest s nimi v meste Davidove. I kraloval Jehoram syn jeho místo neho.
\par 2 Mel pak bratrí, syny Jozafatovy, Azariáše, Jechiele, Zachariáše, Azariáše, Michaele a Sefatiáše. Všickni ti byli synové Jozafata krále Izraelského.
\par 3 Kterýmž byl dal otec jejich daru mnoho, stríbra a zlata, a vecí drahých, s mesty hrazenými v Judstvu, království pak dal Jehoramovi, proto že on byl prvorozený.
\par 4 I uvázal se Jehoram v království otce svého, a zmocniv se, zmordoval všecky bratrí své mecem, ano i nekteré z knížat Izraelských.
\par 5 Ve dvou a tridcíti letech byl Jehoram, když pocal kralovati, a osm let kraloval v Jeruzaléme.
\par 6 A chodil po ceste králu Izraelských, tak jako cinil dum Achabuv; nebo dceru Achabovu mel za manželku, a cinil zlé veci pred ocima Hospodinovýma.
\par 7 Hospodin však nechtel zahladiti domu Davidova, pro smlouvu, kterouž byl ucinil s Davidem, a ponevadž byl rekl, že dá jemu svíci i synum jeho po všecky dny.
\par 8 Ve dnech jeho odstoupili Idumejští, aby nebyli poddáni Judovi, a ustanovili nad sebou krále.
\par 9 Procež pritáhl Jehoram s knížaty svými, i se všemi vozy svými, a vstav v noci porazil Idumejské, kteríž jej byli obklícili, i hejtmany vozu jeho.
\par 10 Však predce vytáhli se Idumejští z manství Judova až do tohoto dne. Téhož casu zprotivilo se i Lebno, aby nebylo pod jeho mocí, proto že opustil Hospodina Boha otcu svých.
\par 11 Presto nadelal výsostí po horách Judských, a uvedl v smilství obyvatele Jeruzalémské, nýbrž dostrcil, jako i Judské.
\par 12 I prišlo k nemu psání od Eliáše proroka, rkoucího: Toto praví Hospodin Buh Davida otce tvého: Proto že jsi nechodil po cestách Jozafata otce svého, a po cestách Azy krále Judského,
\par 13 Ale chodil jsi po ceste králu Izraelských, a uvedl jsi v smilství Judu i obyvatele Jeruzalémské, tak jako jest v smilství uvedl dum Achabuv Izraele, nadto i bratrí své, rodinu otce svého, lepší než jsi sám, zmordoval jsi:
\par 14 Aj, Hospodin uvede ránu velikou na lid tvuj a na syny tvé, na ženy tvé a na všecko jmení tvé,
\par 15 Ty pak nemocen budeš težce nemocí strev svých, až z tebe vyjdou streva tvá pro nemoc rozmáhající se den ode dne.
\par 16 A tak vzbudil Hospodin proti Jehoramovi ducha Filistinských a Arabských, kteríž jsou pri koncinách Chussimských.
\par 17 Kteríž vytáhše proti zemi Judské, vtrhli do ní, a rozchvátali všecko jmení, kteréž se nalezlo v dome královském, ano i syny jeho a ženy jeho, tak že mu nezustalo žádného syna krome Joachaza, nejmladšího z synu jeho.
\par 18 Potom pak po všem ranil jej Hospodin na strevách jeho nemocí nezhojitelnou.
\par 19 A když se to den po dni rozmáhalo, a již cas vycházel prebehnutí dvou let, vyšla streva jeho pro nemoc jeho, i umrel na hrozné bolesti. A nepálil mu lid jeho vonných vecí, jako pálívali otcum jeho.
\par 20 Ve tridcíti a ve dvou letech byl, když kralovati pocal, a osm let kraloval v Jeruzaléme, a sešel tak, že po nem netoužili. A však jej pochovali v meste Davidove, ale ne v hrobích královských.

\chapter{22}

\par 1 Tedy ustanovili krále obyvatelé Jeruzalémští Ochoziáše, syna jeho nejmladšího, místo neho; nebo všecky starší byl pobil houf lotríku, kteríž pritáhli s Arabskými do ležení. A tak kraloval Ochoziáš syn Jehorama krále Judského.
\par 2 Ve ctyridcíti a dvou letech byl Ochoziáš, když kralovati pocal, a jediný rok kraloval v Jeruzaléme. Jméno pak matky jeho Atalia dcera Amri.
\par 3 I on také chodil po cestách domu Achabova, nebo matka jeho nabádala jej k nepobožnosti.
\par 4 Protož cinil zlé veci pred ocima Hospodinovýma, jako dum Achabuv; ty on mel zajisté rádce po smrti otce svého, k zahynutí svému.
\par 5 Nebo radou jejich se spravoval, a táhl s Joramem synem Achabovým, králem Izraelským, na vojnu proti Hazaelovi králi Syrskému do Rámot Galád, kdežto ranili Syrští Jorama.
\par 6 Kterýž navrátil se, aby se hojil v Jezreel; nebo mel rány, kteréž mu ucinili v Ráma, když bojoval s Hazaelem králem Syrským. V tom Azariáš syn Jehorama krále Judského sstoupil, aby navštívil Jorama syna Achabova v Jezreel, kterýž tam nemocen byl.
\par 7 Bylo pak to ku pádu od Boha Ochoziášovi, že prišel k Joramovi. Nebo jakž prišel, vyjel s Joramem proti Jéhu synu Namsi, kteréhož byl pomazal Hospodin, aby vyplénil dum Achabuv.
\par 8 Tedy stalo se, když mstil Jéhu nad domem Achabovým, že našel knížata Judská, a syny bratrí Ochoziášových, kteríž sloužili Ochoziášovi, i zmordoval je.
\par 9 Potom hledajíce i Ochoziáše, jali jej, an se pokrýval v Samarí, a privedše ho k Jéhu, zabili jej a pochovali; nebo rekli: Synt jest Jozafatuv, kterýž hledal Hospodina celým srdcem svým. A tak nebylo žádného více z domu Ochoziášova, kterýž by mel moc k obdržení království.
\par 10 Procež Atalia matka Ochoziášova viduci, že umrel syn její, vstavši, pomordovala všecko síme královské domu Judského.
\par 11 Ale Jozabat dcera králova vzala Joasa syna Ochoziášova, a tajne ho uchvátivši z prostredku synu královských, kteríž mordováni byli, schovala jej s chuvou jeho v pokoji, kdež luže byla. I skryla ho Jozabat dcera krále Jorama, manželka Joiady kneze, (nebo ona byla sestra Ochoziášova), pred Atalií, aby ho nezamordovala.
\par 12 I byl s nimi v dome Božím, skryt jsa za šest let, v nichž Atalia kralovala v té zemi.

\chapter{23}

\par 1 Léta pak sedmého posilnil se Joiada, a pojal s sebou v smlouvu setníky, Azariáše syna Jerochamova, a Izmaele syna Jochananova, a Azariáše syna Obédova, a Maaseiáše syna Adaiášova, a Elizafata syna Zichri,
\par 2 Kteríž obcházejíce zemi Judskou, shromáždili Levíty ze všech mest Judských, a prední z otcovských celedí Izraelských, a prišli do Jeruzaléma.
\par 3 I ucinilo všecko to shromáždení smlouvu v dome Božím s králem. Byl pak jim rekl Joiada: Aj, syn králuv kralovati bude, jakož mluvil Hospodin o synech Davidových.
\par 4 Tatot jest vec, kterouž uciníte: Tretí díl vás, kteríž pricházíte v sobotu z kneží a z Levítu, budou vrátnými.
\par 5 Tretí pak díl bude pri dome královu, a tretí díl u brány prední, ale všecken lid bude v síních domu Hospodinova.
\par 6 Aniž kdo vcházej do domu Hospodinova krome kneží a tech Levítu, kteríž konají služby. Ti at vcházejí, nebo svatí jsou, všecken pak lid at drží stráž Hospodinovu.
\par 7 I obstoupí Levítové krále vukol, jeden každý maje bran svou v rukou svých, a kdož by koli všel do domu, at umre. A budte pri králi, když bude vcházeti, i když bude vycházeti.
\par 8 Tedy ucinili Levítové a všecken Juda všecko, cožkoli prikázal Joiada knez, a vzali jeden každý muže své, kteríž pricházeli v sobotu, a kteríž odcházeli v sobotu; nebo byl nepropustil Joiada knez žádné strídy.
\par 9 I dal Joiada knez setníkum kopí a pavézy i štíty, kteríž byli krále Davida, jenž byli v dome Božím.
\par 10 A postavil všecken lid, (a jeden každý mel bran v ruce své), od pravé strany domu až do levé strany domu, proti oltári a proti domu, pri králi všudy vukol.
\par 11 A tak vyvedli syna králova, a vstavili na nej korunu královskou a ozdobu. I ustanovili jej králem, a pomazali ho Joiada a synové jeho, a ríkali: Živ bud král!
\par 12 V tom uslyševši Atalia hrmot sbíhajícího se lidu, a chválícího krále, vešla k lidu do domu Hospodinova.
\par 13 A když pohledela, a aj, král stál na míste vyšším, kudyž se vchází, a knížata i trubaci podlé krále, a všecken lid zeme veselící se, a zvucící na trouby, a zpeváci s nástroji hudebnými, a ti, kteríž predcili v zpívání. Tedy roztrhla Atalia roucho své, a rekla: Spiknutí, spiknutí!
\par 14 Ale Joiada knez rozkázal vyjíti setníkum, hejtmanum vojska, rka jim: Vyvedte ji prostredkem radu, a kdož by za ní šel, at jest zabit mecem. Nebo byl rekl knez: Nezabijejte jí v dome Hospodinove.
\par 15 A když se jí rozstoupili s obou stran, šla, kudy se chodí k bráne konské k domu královskému, a tu ji zabili.
\par 16 Tedy Joiada ucinil smlouvu mezi Hospodinem a mezi vším lidem, i mezi králem, aby byli lid Hospodinuv.
\par 17 Potom šel všecken lid do domu Bálova a zborili jej i oltáre jeho, a obrazy stroskotali. Matana pak kneze Bálova zabili pred oltári.
\par 18 I uvedl zase Joiada úredníky domu Hospodinova pod moc kneží a Levítu, kteréž byl rozdelil David v dome Hospodinove, aby obetovali zápaly Hospodinu, (jakož psáno jest v zákone Mojžíšove), s veselím a zpíváním podlé narízení Davidova.
\par 19 Postavil také vrátné u vrat domu Hospodinova, aby tam nevcházel poškvrnený jakoužkoli vecí.
\par 20 A pojav setníky a znamenitejší, a ty, kteríž správu drželi nad lidem, i všecken lid zeme, provázel krále z domu Hospodinova. I šli branou horejší do domu královského, a posadili krále na stolici královské.
\par 21 I veselil se všecken lid zeme, a mesto se upokojilo, jakž Atalii zabili mecem.

\chapter{24}

\par 1 V sedmi letech byl Joas, když pocal kralovati, a ctyridceti let kraloval v Jeruzaléme. Jméno pak matky jeho Sebia z Bersabé.
\par 2 A cinil Joas to, což pravého bylo pred ocima Hospodinovýma po všecky dny kneze Joiady.
\par 3 Mezi tím vzal mu Joiada dve žene, i plodil syny i dcery.
\par 4 Potom pak uložil v srdci Joas, aby opravil dum Hospodinuv.
\par 5 I shromáždil kneží a Levíty, a rekl jim: Vyjdete do mest Judských, a vybírejte ode všeho Izraele peníze, na opravu domu Boha vašeho na každý rok, vy pak pospešte s tou vecí. Ale nepospíchali Levítové.
\par 6 Protož povolav král Joiady, predního kneze, rekl jemu: Proc neprídržíš Levítu, aby snesli z Judstva a z Jeruzaléma sbírku Mojžíše služebníka Hospodinova a shromáždení všeho Izraele k stánku svedectví?
\par 7 Nebo Atalia bezbožná a synové její vlámali se do domu Božího, a všecky veci posvecené domu Hospodinova obrátili na modly.
\par 8 A tak porucil král, aby udelali jednu truhlu, a postavili ji u brány domu Hospodinova vne.
\par 9 I dali provolati v Judstvu a v Jeruzaléme, aby snesli Hospodinu sbírku Mojžíše služebníka Božího, uloženou na Izraele na poušti.
\par 10 Tedy veselila se všecka knížata i všecken lid, a prinášejíce, metali do truhly, až se zporádali.
\par 11 Bývalo pak to, že když prinášeli truhlu k úredníkum královským skrze ruce Levítu, (nebo když videli, že mnoho jest penez, tedy pricházel kanclír královský a úredník nejvyššího kneze), aby vyprázdnili truhlu, potom ji zase donášeli a postavovali na místo její. A tak cinívali den po dni, a sebrali penez velmi mnoho.
\par 12 Kteréž vydával král a Joiada správcum nad dílem domu Hospodinova, a oni najímali kameníky a remeslníky k opravování domu Hospodinova, ano i kováre a kotláre k utvrzení domu Hospodinova.
\par 13 Takž delali delníci, a spraveno jest dílo to skrze ruce jejich, tak že privedli zase dum Boží k zpusobu jeho, a upevnili jej.
\par 14 A když dokonali, prinesli pred krále a Joiadu ostatek penez. I dal nadelati z nich nádob do domu Hospodinova, nádob k službe a obetování, a kadidlnic, i jiných nádob zlatých a stríbrných. A tak obetovávali obeti zápalné v dome Hospodinove ustavicne, po všecky dny Joiadovy.
\par 15 Potom sstaral se Joiada, pln jsa dnu, a umrel. Ve stu a ve tridcíti letech byl, když umrel.
\par 16 I pochovali ho v meste Davidove s králi, proto že cinil, což dobrého jest Izraelovi, Bohu i domu jeho.
\par 17 Po smrti pak Joiadove prišla knížata Judská, a padše, klaneli se králi. Tedy uposlechl jich král.
\par 18 Procež opustivše dum Hospodina Boha otcu svých, sloužili hájum a modlám. I rozhneval se Buh na lid Judský a na Jeruzalém pro ten hrích jejich.
\par 19 I posílal k nim proroky, aby je zase obrátili k Hospodinu. Kterížto svedcili jim, oni však neuposlechli.
\par 20 Nýbrž, když Duch Boží vzbudil Zachariáše syna Joiady kneze, (kterýž vystoupiv pred lid, mluvil jim: Takto praví Buh: Proc prestupujete prikázaní Hospodinova? Nepovedet se vám štastne. Proto že jste opustili Hospodina, i on také opustí vás);
\par 21 Oni spikše se proti nemu, ukamenovali jej z rozkazu králova v síni domu Hospodinova.
\par 22 (Aniž se rozpomenul Joas král na milosrdenství, kteréž byl ucinil jemu Joiada otec jeho, ale zamordoval syna jeho.) Kterýž když umíral, rekl: Necht popatrí Hospodin, a vyhledá to.
\par 23 I stalo se po roce, vytáhlo proti nemu vojsko Syrské a pritáhlo proti Judovi a Jeruzalému, a vyhladili z lidu všecka knížata jejich, a všecky loupeže jich poslali králi Damašskému.
\par 24 Nebo ac v malém poctu lidu bylo pritáhlo vojsko Syrské, však Hospodin dal v ruku jejich vojsko velmi veliké, proto že opustili Hospodina Boha otcu svých. Ano i samého Joasa trápili.
\par 25 A jakž ti odešli od neho, (opustivše jej v težkých nemocech), spikli se proti nemu služebníci jeho pro krev synu Joiady kneze, a zamordovali jej na loži jeho. I umrel. Tedy pochovali jej v meste Davidove, ale nepochovali ho v hrobích královských.
\par 26 A tito jsou, kteríž se spikli proti nemu: Zabad syn Simaty Ammonitského, a Jozabad syn Simrity Moábského.
\par 27 O synech pak jeho, a o veliké dani od neho uložené, i o stavení domu Božího, to vše sepsáno jest v knize královské. Kraloval pak Amaziáš syn jeho místo neho.

\chapter{25}

\par 1 V petmecítma letech kralovati pocal Amaziáš, a kraloval dvadceti devet let v Jeruzaléme. Jméno matky jeho Joadan z Jeruzaléma.
\par 2 A cinil to, což pravého bylo pred ocima Hospodinovýma, ale však ne srdcem uprímým.
\par 3 I stalo se, když bylo upevneno království jeho, že zmordoval služebníky své, kteríž zabili krále otce jeho.
\par 4 A však synu jejich nekázal mordovati, ale ucinil tak, jakž psáno jest v zákone, v knize Mojžíšove, kdež prikázal Hospodin, rka: Nebudou na hrdle trestáni otcové za syny, aniž synové trestáni budou na hrdle za otce, ale jeden každý za svuj hrích umre.
\par 5 Zatím shromáždil Amaziáš lid Judský, a ustanovil po domích otcovských celedí jejich hejtmany a setníky, po všem Judstvu a pokolení Beniaminovu, a sectl je od dvadcítiletých a výše, a našel jich trikrát sto tisíc vybraných mužu bojovných, kopidlníku a pavézníku.
\par 6 Najal také ze mzdy z Izraele sto tisíc mužu udatných ze sta hriven stríbra.
\par 7 Muž pak Boží prišel k nemu, rka: Ó králi, necht netáhne s tebou vojsko Izraelské; nebo Hospodin není s Izraelem, a všechnemi syny Efraimovými.
\par 8 Ale táhni ty. Uciniž tak, a posiln se k boji; jinak porazí te Buh pred neprátely, nebot muže Buh i pomoci i poraziti.
\par 9 Tedy rekl Amaziáš muži Božímu: Jakž pak udelám s stem hriven stríbra, kteréž jsem dal vojsku Izraelskému? Odpovedel muž Boží: Mát Hospodin mnohem více nad to, což by dal tobe.
\par 10 A tak oddelil Amaziáš vojsko to, kteréž bylo pritáhlo k nemu z Efraim, aby odešli na místo své. Procež rozhnevali se náramne na Judské a navrátili se k místu svému s velikým hnevem.
\par 11 Mezi tím Amaziáš posiliv se, vedl lid svuj, a táhl do údolí solnatého, a porazil synu Seir deset tisícu.
\par 12 Deset také tisícu živých jali synové Judští, a vedli na vrch skály, a sházeli je s vrchu té skály, tak že se všickni zrozráželi.
\par 13 Vojsko pak to, kteréž zase odeslal Amaziáš, aby netáhlo s ním na vojnu, vtrhlo do mest Judských od Samarí až do Betoron, a pobivše z nich tri tisíce, nabrali loupeží mnoho.
\par 14 I stalo se, když se vracel Amaziáš od porážky Idumejských, že privezl s sebou bohy synu Seir, a vyzdvihl je sobe za bohy, a sklánel se pred nimi, a kadil jim.
\par 15 A protož rozhneval se náramne Hospodin na Amaziáše, a poslal k nemu proroka,kterýž jemu rekl: Proc hledáš bohu lidu toho, kteríž nevytrhli lidu svého z ruky tvé?
\par 16 Stalo se pak, když on k nemu mluvil, rekl mu král: Co te postavili, abys byl rádcím královským? Prestaniž, nechceš-li bit býti. A tak prestal prorok, však rekl: Seznávám, že te usoudil Buh zahladiti, ponevadž uciniv to, neuposlechl jsi rady mé.
\par 17 Tedy poradiv se Amaziáš král Judský, poslal k Joasovi synu Joachaza syna Jéhu, králi Izraelskému, rka: Nuže, pohledme sobe v oci.
\par 18 I poslal Joas král Izraelský k Amaziášovi králi Judskému, a rekl: Bodlák, kterýž byl na Libánu, poslal k cedru Libánskému, rka: Dej dceru svou synu mému za manželku. V tom šlo tudy zvíre polní, kteréž bylo na Libánu, a pošlapalo ten bodlák.
\par 19 Myslil jsi: Aj, porazil jsem Idumejské, protož pozdvihlo te srdce tvé, tak že se tím honosíš. Medle, zustan v dome svém. Proc se máš zapletati v takové zlé, až bys padl i ty i Juda s tebou?
\par 20 Ale neuposlechl Amaziáš; nebo od Boha to bylo, aby je vydal v ruku onechno, proto že hledali bohu Idumejských.
\par 21 Takž vytáhl Joas král Izraelský, a pohledeli sobe v oci s Amaziášem králem Judským v Betsemes, kteréž jest Judovo.
\par 22 I poražen jest Juda pred Izraelem, a utíkali jeden každý do príbytku svých.
\par 23 Amaziáše pak krále Judského, syna Joasova, jenž byl syn Joachazuv, jal Joas král Izraelský u Betsemes, a privedl jej do Jeruzaléma, a zboril zdi Jeruzalémské odbrány Efraim až k bráne úhlu na ctyri sta loktu.
\par 24 A pobrav všecko zlato a stríbro i všecko nádobí, kteréž se nalezlo v dome Božím u Obededoma, a v pokladích domu královského, také i ty, kteríž byli v zástave, navrátil se do Samarí.
\par 25 Byl pak živ Amaziáš syn Joasuv, král Judský, po smrti Joasa syna Joachaza, krále Izraelského, patnácte let.
\par 26 Ale o jiných cinech Amaziášových, prvních i posledních, vypsáno jest v knize o králích Judských a Izraelských.
\par 27 Od toho zajisté casu, jakž se spustil Amaziáš Hospodina, spikli se jedovate proti nemu v Jeruzaléme. A když utekl do Lachis, poslali za ním do Lachis, a zabili jej tam.
\par 28 A prinesše ho na koních, pochovali jej s otci jeho v meste Judove.

\chapter{26}

\par 1 Tedy všecken lid Judský vzali Uziáše, kterýž byl v šestnácti letech, a ustanovili jej za krále na míste otce jeho Amaziáše.
\par 2 Ont jest vzdelal Elot, a dobyl ho zase Judovi, když již umrel král s otci svými.
\par 3 V šestnácti letech byl Uziáš, když kralovati pocal, a kraloval padesáte a dve létev Jeruzaléme. Jméno pak matky jeho Jecholia z Jeruzaléma.
\par 4 Ten cinil to, což pravého bylo pred ocima Hospodinovýma, podlé všeho, což byl cinil Amaziáš otec jeho.
\par 5 A hledal s pilností Boha ve dnech Zachariáše, rozumejícího videní Božímu, a po ty dny, v nichž hledal Hospodina, štastný prospech dal jemu Buh.
\par 6 Nebo vytáh, bojoval proti Filistinským, a proboril zed mesta Gát, a zed mesta Jabne, i zed mesta Azotu, a vystavel mesta okolo Azotu, i v zemi Filistinské.
\par 7 Buh zajisté pomáhal jemu proti Filistinským a proti Arabským, kteríž bydlili v Gurbal, i proti Maonitským.
\par 8 I dávali Ammonitští dary Uziášovi, a rozneslo se jméno jeho až do Egypta; nebo zsilil se na nejvyšší.
\par 9 A vzdelal Uziáš veže v Jeruzaléme u brány úhlové, a u brány údolí, a u Mikzoa, i upevnil je.
\par 10 Vystavel také veže na poušti, a vykopal studnic mnoho, proto že mel stád mnoho, jakož pri údolí tak i na rovinách, oráce tolikéž a vinare po horách i na místech úrodných; nebo laskav byl na rolí.
\par 11 Mel také Uziáš vojsko bojovníku, vycházejících k boji po houfích v jistém poctu, jakž vycteni byli od Jehiele kanclíre, a Maaseiáše úredníka, uvedené pod správu Chananiášovi knížeti královskému.
\par 12 Všecken pocet knížat celedí otcovských, mužu udatných, dva tisíce a šest set.
\par 13 A pod spravou jejich lidu válecného trikrát sto tisíc, sedm tisíc a pet set bojovníku udatných, aby pomáhali králi proti nepríteli.
\par 14 Pripravil pak Uziáš všemu tomu vojsku pavézy, kopí, lebky, pancíre, lucište i kamení prakové.
\par 15 Nadelal také v Jeruzaléme vtipne vymyšlených nástroju válecných, aby byli na vežech a na úhlech k strílení strelami a kamením velikým. I rozneslo se jméno jeho daleko, proto že divné pomoci mel, až se i zmocnil.
\par 16 Ale když se zmocnil, pozdvihlo se srdce jeho k zahynutí jeho, a zhrešil proti Hospodinu Bohu svému; nebo všel do chrámu Hospodinova, aby kadil na oltári, na nemž se kadívalo.
\par 17 I všel za ním Azariáš knez, a s ním jiných kneží Hospodinových osmdesáte, mužu udatných.
\par 18 Kteríž postavili se proti Uziášovi králi, a mluvili jemu: Ne tobe, Uziáši, náleží kaditi Hospodinu, ale knežím, synum Aronovým, kteríž posveceni jsou, aby kadili. Vyjdiž z svatyne, nebo jsi zhrešil, aniž to bude tobe ke cti pred Hospodinem Bohem.
\par 19 Procež rozhneval se Uziáš, (v ruce pak své mel kadidlnici, aby kadil). A když se spouzel na kneží, ukázalo se malomocenství na cele jeho pred knežími v dome Hospodinove u oltáre, na nemž se kadilo.
\par 20 A pohledev na nej Azariáš, nejvyšší knez, a všickni kneží, a aj, byl malomocný na cele svém. Protož rychle jej vyvedli ven, nýbrž i sám se nutil vyjíti, proto že ho ranil Hospodin.
\par 21 A tak byl Uziáš král malomocný až do dne smrti své, a bydlil v dome obzvláštním, jsa malomocný; nebo byl vyobcován z domu Hospodinova. Mezi tím Jotam syn jeho byl nad domem královským, soude lid zeme.
\par 22 O jiných pak vecech Uziášových, prvních i posledních, psal Izaiáš prorok, syn Amosuv.
\par 23 I usnul Uziáš s otci svými, a pochovali jej s otci jeho na poli hrobu královských; nebo rekli: Malomocný jest. I kraloval Jotam syn jeho místo neho.

\chapter{27}

\par 1 V petmecítma letech byl Jotam, když pocal kralovati, a šestnáct let kraloval v Jeruzaléme. Jméno matky jeho Jerusa dcera Sádochova.
\par 2 Kterýž cinil to, což pravého bylo pred ocima Hospodinovýma, podlé všeho, což cinil Uziáš otec jeho, krome že nevšel do chrámu Hospodinova, a lid ješte porušený byl.
\par 3 On ustavel bránu domu Hospodinova horejší, a na zdi Ofel mnoho stavel.
\par 4 Nadto vystavel i mesta na horách Judských, a v lesích zdelal zámky a veže.
\par 5 On také bojoval s králem synu Ammon, a zmocnil se jich. I dali mu synové Ammon toho roku sto centnéru stríbra, a deset tisíc mer pšenice, a jecmene deset tisícu. Tolikéž dali mu synové Ammon i léta druhého i tretího.
\par 6 A tak zsilil se Jotam; nebo nastrojil cesty své pred Hospodinem Bohem svým.
\par 7 O jiných pak vecech Jotamových, a o všech válkách i cestách jeho, zapsáno jest v knize o králích Izraelských a Judských.
\par 8 V petmecítma letech byl, když kralovati zacal, a šestnáct let kraloval v Jeruzaléme.
\par 9 I usnul Jotam s otci svými, a pochovali jej v meste Davidove, a kraloval Achas syn jeho místo neho.

\chapter{28}

\par 1 Ve dvadcíti letech byl Achas, když pocal kralovati, a šestnácte let kraloval v Jeruzaléme, a necinil toho, což pravého bylo pred ocima Hospodinovýma, jako David otec jeho.
\par 2 Ale chodil po cestách králu Izraelských, nýbrž nadelal také slitin modlárských.
\par 3 Presto i sám kadíval v údolí Benhinnom, a pálil syny své ohnem podlé ohavností pohanu, kteréž vyhnal Hospodin pred syny Izraelskými.
\par 4 A obetoval i kadíval na výsostech a na pahrbcích, i pod každým stromem zeleným.
\par 5 Protož dal jej Hospodin Buh jeho v ruku krále Syrského. Kteríž porazivše jej, zajali odtud množství veliké, a privedli je do Damašku. Nad to i v ruku krále Izraelského dán jest, kterýž porazil jej porážkou velikou.
\par 6 Nebo pomordoval Pekach syn Romeliášuv v Judstvu sto a dvadcet tisícu dne jednoho, vše mužu udatných, proto že opustili Hospodina Boha otcu svých.
\par 7 Zabil také Zichri, rytír Efraimský Maaseiáše syna králova, a Azrikama, správce domu jeho, a Elkána, druhého po králi.
\par 8 Nadto zajali též synové Izraelští z bratrí svých dvakrát sto tisíc žen, synu a dcer, ano i koristí mnoho nabrali od nich, a vezli loupež do Samarí.
\par 9 Byl pak tu prorok Hospodinuv, jménem Oded, kterýž vyšed proti vojsku pricházejícímu do Samarí, rekl jim: Hle, rozhnevav se Hospodin Buh otcu vašich na Judské, vydal je v ruku vaši, a vy jste je pomordovali v prchlivosti, kteráž až k nebi dosáhla.
\par 10 A ješte Judské a Jeruzalémské myslíte sobe podrobiti za pacholky a devky. Zdaliž i vy, i vy, jste bez hríchu proti Hospodinu Bohu vašemu?
\par 11 A protož nyní poslechnete mne, a dovedte zase zajaté, kteréž jste zajali z bratrí svých; nebo jinak hnev prchlivosti Hospodinovy bude proti vám.
\par 12 Tedy povstali nekterí z predních synu Efraimových: Azariáš syn Jochananuv, Berechiáš syn Mesillemotuv, Ezechiáš syn Sallumuv, a Amaza syn Chadlaiuv proti tem, kteríž se vracovali z té vojny.
\par 13 A rekli jim: Neprivedete sem tech zajatých; nebo vinu proti Hospodinu myslíte na nás uvesti, privetšujíce hríchu našich a provinení našich. Veliký te zajisté hrích náš a prchlivost hnevu proti Izraelovi.
\par 14 Takž nechali vojáci tech zajatých i koristí svých pred knížaty a vším shromáždením.
\par 15 A pricinivše se muži nekterí ze jména znamenaní, vzali ty zajaté, a všecky, kterížkoli z nich nebyli odení, priodíli je z tech koristí. A když je zobláceli a zobouvali, nakrmili i napojili, ano i pomazali, a zprovodili na oslích, každého nemocného z nich, a dovedli ho do Jericha mesta palmového k bratrím jejich, potom navrátili se do Samarí.
\par 16 Toho casu poslal král Achas k králum Assyrským, aby mu pomoc dali.
\par 17 Nebo ješte pritáhli i Idumejští, a zbili nekteré z Judských, a jiné zajali.
\par 18 Nadto i Filistinští ucinili vpád do mest, na rovinách k strane polední Judstva, a vzali Betsemes a Aialon, a Gederot a Socho i vsi jeho, a Tamna i vsi jeho, a Gimzo i vsi jeho, a bydlili v nich.
\par 19 Tak zajisté ponižoval Hospodin Judy pro Achasa krále Izraelského, proto že odvrátil Judu, aby naprosto prevrácene cinil proti Hospodinu.
\par 20 A tak pritáhl k nemu Tiglatfalazar král Assyrský, a více ho ssužoval, nežli mu pomáhal.
\par 21 A ackoli Achas vzal z domu Hospodinova a domu královského, i od knížat, a dal králi Assyrskému, však nedal jemu pomoci.
\par 22 V který pak koli cas byl ssužován, tím vetší prevrácenost páchal proti Hospodinu. Takový byl král Achas.
\par 23 Nebo obetoval bohum Damašským, kteríž ho porazili, a rekl: Ponevadž bohové králu Syrských pomáhají jim, tem obetovati budu, aby mne pomáhali. Ale oni byli ku pádu jemu, ano i všemu Izraeli.
\par 24 Procež shromáždiv Achas nádoby domu Božího, osekal je, a zamkl dvére domu Hospodinova, a nadelal sobe oltáru pri každém úhlu v Jeruzaléme.
\par 25 Ano i v každém meste Judském zdelal výsosti, aby kadil bohum cizím, a tak popouzel k hnevu Hospodina Boha otcu svých.
\par 26 O jiných pak vecech jeho, a o všech cestách jeho, prvních i posledních zapsáno jest v knize králu Judských a Izraelských.
\par 27 I usnul Achas s otci svými, a pochovali jej v meste Jeruzaléme; nebo nevložili ho do hrobu králu Izraelských. I kraloval Ezechiáš syn jeho místo neho.

\chapter{29}

\par 1 Ezechiáš kralovati pocal, když byl v petmecítma letech, a dvadceti devet let kraloval v Jeruzaléme. Jméno matky jeho bylo Abia dcera Zachariášova.
\par 2 A cinil to, což pravého bylo pred ocima Hospodinovýma, podlé všech vecí, kteréž cinil David otec jeho.
\par 3 On prvního léta kralování svého mesíce prvního otevrel dvére domu Hospodinova, a opravil je.
\par 4 Uvedl také kneží a Levíty, shromáždiv je do ulice východní,
\par 5 A rekl jim: Slyšte mne, Levítové. Nyní posvette se, posvette také i domu Hospodina Boha otcu vašich, a vyneste necistotu z svatyne.
\par 6 Nebot zhrešili otcové naši, a cinili zlé veci pred ocima Hospodina Boha našeho, opouštejíce jej, a odvracujíce tvár svou od stánku Hospodinova, obracejíce se hrbetem k nemu.
\par 7 Zavreli také dvére síne, a zhasili lampy, a kadidlem nekadili, aniž zápalu obetovali v svatyni Bohu Izraelskému.
\par 8 Protož rozhneval se Hospodin na Judu a na Jeruzalém, a vydal je v posmýkání, zpuštení a ku podivení, jakož sami ocima svýma vidíte.
\par 9 A aj, padli otcové naši od mece, a synové naši i dcery naše a manželky naše zajaty jsou pro tu prícinu.
\par 10 Nyní tedy umínil jsem uciniti smlouvu s Hospodinem Bohem Izraelským, aby odvrátil od nás hnev prchlivosti své.
\par 11 Synové moji, nebludtež již. Vást jest zajisté vyvolil Hospodin, abyste stojíce pred ním, sloužili jemu, a byli služebníci jeho, a kadili.
\par 12 Tedy povstali Levítové tito: Machat syn Amazai, a Joel syn Azariášuv z synu Kahat; z synu pak Merari: Cis syn Abdi, a Azariáš syn Jehalleeluv; a z Gersonitských: Joach syn Zimma, a Eden syn Joachuv;
\par 13 Též z synu Elizafanových: Simri a Jehiel; z synu pak Azafových: Zachariáš a Mataniáš;
\par 14 A z synu Hémanových: Jechiel a Simei; z synu pak Jedutunových: Semaiáš a Uziel.
\par 15 I shromáždili bratrí své, kteríž posvetivše se, prišli podlé rozkázaní králova a slov Hospodinových, aby vycistili dum Hospodinuv.
\par 16 A všedše kneží do domu Hospodinova, aby jej vycistili, vynesli všelikou necistotu, kterouž našli v chráme Hospodinove, do síne domu Hospodinova. Levítové pak probravše to, vynesli ven ku potoku Cedron.
\par 17 Zacali pak prvního dne mesíce prvního posvecovati, a dne osmého téhož mesíce vešli do síne Hospodinovy, a posvecovali domu Hospodinova za osm dní, a dne šestnáctého mesíce prvního dokonali.
\par 18 Tedy vešli k Ezechiášovi králi, a rekli: Vycistili jsme všecken dum Hospodinuv, i oltár zápalu, i všecky nádoby jeho, i stul predložení a všecky nádoby jeho.
\par 19 Všecky také nádoby, kteréž byl zavrhl král Achas za kralování svého, když prevrácene cinil, pripravili jsme a posvetili, a aj, jsou pred oltárem Hospodinovým.
\par 20 Potom vstav ráno Ezechiáš král, shromáždil úredníky mesta, a vstoupil do domu Hospodinova.
\par 21 A privedli volku sedm, a skopcu sedm, a beránku sedm, a kozlu sedm za hrích, za království, a za svatyni, a za Judu; a rozkázal synum Aronovým knežím, aby obetovali na oltári Hospodinovu.
\par 22 Takž zbili voly ty, a kneží berouce krev jejich, kropili na oltár. Zbili též i skopce a pokropovali krví jejich oltáre; zbili také i beránky, a krví jejich kropili na oltár.
\par 23 Privedli i kozly k obeti za hrích pred krále i shromáždení, kterížto vložili ruce své na ne.
\par 24 I zbili je kneží a vycistili krví jejich oltár k ocištení všeho Izraele; nebo za všecken lid Izraelský rozkázal král obetovati zápal a obeti za hrích.
\par 25 Postavil zase i Levíty v dome Hospodinove s cymbály, loutnami a harfami podlé rozkázaní Davidova, a Gáda proroka královského, a Nátana proroka; nebo od Hospodina bylo to prikázáno skrze proroky jeho.
\par 26 A tak stáli Levítové s nástroji Davidovými a kneží s trubami.
\par 27 I rozkázal Ezechiáš, aby obetovali zápal na oltári. A když se zacala obet zápalná, zacal se zpev Hospodinuv a troubení i zpevové na nástroje Davida, krále Izraelského.
\par 28 Všecko pak shromáždení klanelo se, když zpeváci zpívali, a trubaci troubili, a to vše, až se dokonal zápal.
\par 29 A když dokonali obetování zápalu, sklonili se král a všickni, kteríž byli s ním, a poklonu ucinili.
\par 30 Tedy prikázal Ezechiáš král i knížata Levítum, aby chválili Hospodina slovy Davidovými a Azafa proroka. I chválili s velikým veselím, a sklánejíce se, poklonu cinili.
\par 31 Potom mluvil Ezechiáš, rka: Nyní jste posvetili rukou svých Hospodinu. Pristupte, a privedte obeti vítezné a obeti chvály do domu Hospodinova. Protož privodilo shromáždení to obeti vítezné a obeti chvály, ano i každý z srdce ochotného k zápalným obetem,
\par 32 Tak že pocet obetí zápalných, kteréž privedlo shromáždení to, byl volu sedmdesáte, skopcu sto, beránku dve ste, všecko to na obet zápalnou Hospodinu.
\par 33 Jiných pak vecí posvecených bylo volu šest set, a ovec tri tisíce.
\par 34 Kneží však bylo málo, tak že nemohli postaciti vytahovati z koží všech obetí zápalných. Procež pomáhali jim bratrí jejich Levítové, až i dokonali dílo to, a dokudž se neposvetili jiní kneží; nebo Levítové byli hotovejší ku posvecení se, nežli kneží.
\par 35 K tomu k zápalum bylo množství veliké tuku z obetí pokojných a obetí mokrých, krome jiných zápalu. A tak vykonána byla služba domu Hospodinova.
\par 36 A veselil se Ezechiáš i všecken lid, že Buh byl nastrojil lid, tak aby se ta vec rychle spravila.

\chapter{30}

\par 1 Potom poslal Ezechiáš ke všemu lidu Izraelskému a Judskému, psal také listy k Efraimovi a Manassesovi, aby prišli do domu Hospodinova do Jeruzaléma, a slavili velikunoc Hospodinu Bohu Izraelskému.
\par 2 Snesli se pak byli na tom král i knížata jeho, i všecko shromáždení v Jeruzaléme, aby slavili velikunoc mesíce druhého.
\par 3 Nebo nemohli slaviti toho casu, proto že kneží posvecených se nedostávalo, aniž se lid byl shromáždil do Jeruzaléma.
\par 4 I videlo se to za dobré králi i všemu množství,
\par 5 A usoudili, aby provolati dali po všem Izraeli, od Bersabé až do Dan, aby prišli k slavení velikonoci Hospodinu Bohu Izraelskému do Jeruzaléma; nebo již byli dávno neslavili tak, jakž psáno jest.
\par 6 Protož šli poslové s listy od krále a od knížat jeho po vší zemi Izraelské a Judské s rozkazem královským tímto: Synové Izraelští, obratte se k Hospodinu Bohu Abrahamovu, Izákovu a Izraelovu, i obrátí se k ostatkum, kteríž jste koli znikli ruky králu Assyrských.
\par 7 A nebývejte jako otcové vaši a bratrí vaši, kteríž prevrácene cinili proti Hospodinu Bohu otcu svých, procež dal je v zpuštení, jakž sami vidíte.
\par 8 Nyní tedy nezatvrzujte šijí svých tak, jako otcové vaši. Podejte ruky Hospodinu, a podte do svatyne jeho, kteréž posvetil na veky, a služte Hospodinu Bohu svému, a odvrátí se od vás hnev prchlivosti jeho.
\par 9 Nebo obrátíte-li se k Hospodinu, bratrí vaši i synové vaši milosrdenství obdrží u tech, kteríž je zjímali, tak že se navrátí do zeme této. Milosrdný zajisté a dobrotivý jest Hospodin Buh váš, aniž odvrátí tvári od vás, jestliže se obrátíte k nemu.
\par 10 Když pak ti poslové chodili z mesta do mesta po kraji Efraim a Manasse až do Zabulon, posmívali se a utrhali jim.
\par 11 Ale však nekterí z Asser, Manasses a Zabulon, poníživše se, prišli do Jeruzaléma.
\par 12 V Judstvu tolikéž byla ruka Boží, tak že jim dal srdce jedno k vykonání rozkazu králova a knížat, podlé slova Hospodinova.
\par 13 I sešlo se do Jeruzaléma lidu mnoho, aby drželi slavnost presnic mesíce druhého, shromáždení velmi veliké.
\par 14 Tedy povstavše, poborili oltáre, kteríž byli v Jeruzaléme; též i všecky oltáre, na kterýchž se kadívalo, rozborili a vmetali do potoka Cedron.
\par 15 Potom pak obetovali beránka velikonocního, ctrnáctého dne mesíce druhého. Kneží pak a Levítové zastydevše se, posvetili se, a privedli obeti do domu Hospodinova.
\par 16 A stáli v radu svém vedlé povinnosti své podlé zákona Mojžíše muže Božího, a kneží kropili krví, berouce ji z rukou Levítu.
\par 17 Nebo mnozí byli v shromáždení, kteríž se neposvetili. Protož Levítové zabíjeli beránky velikonocní za každého necistého, k tomu, aby jich posvetili Hospodinu.
\par 18 Veliký zajisté díl toho lidu, množství z Efraima, Manassesa, Izachara a Zabulona, neocistili se, a však jedli beránka velikonocního jinak než psáno jest. Ale modlil se Ezechiáš za ne, rka: Dobrotivý Hospodin ocistiž toho,
\par 19 Kdožkoli celým srdcem svým se ustavil, aby hledal Boha, Hospodina Boha otcu svých, byt pak i ocišten nebyl vedlé ocištování svatého.
\par 20 I vyslyšel Hospodin Ezechiáše, a uzdravil lid.
\par 21 A tak drželi synové Izraelští, kteríž byli v Jeruzaléme, slavnost presnic za sedm dní s veselím velikým, a chválili Hospodina každého dne Levítové a kneží na nástrojích sílu Hospodinovu.
\par 22 Mluvil pak byl Ezechiáš prívetive ke všechnem Levítum vyucujícím známosti pravé Hospodina. I jedli obeti slavnosti té za sedm dní, obetujíce obeti pokojné, a oslavujíce Hospodina Boha otcu svých.
\par 23 Tedy sneslo se na tom všecko shromáždení, aby ucinili tolikéž za druhých sedm dní. Takž slavili jiných sedm dní s radostí.
\par 24 Nebo Ezechiáš král Judský dal shromáždení tomu tisíc volku, a sedm tisíc ovec. Knížata také dala shromáždení volku tisíc, a ovec deset tisícu. A posvetilo se kneží velmi mnoho.
\par 25 A tak veselilo se všecko shromáždení Judské, i kneží a Levítové a všecko shromáždení, kteréž prišlo z Izraele, i príchozí, kteríž byli prišli z zeme Izraelské, i obyvatelé Judští.
\par 26 A bylo veselí veliké v Jeruzaléme; nebo ode dnu Šalomouna syna Davida, krále Izraelského, nic takového nebylo v Jeruzaléme.
\par 27 Potom pak vstali kneží i Levítové, a dali požehnání lidu. I vyslyšán jest hlas jejich, a prišla modlitba jejich k príbytku svatosti Hospodinovy v nebe.

\chapter{31}

\par 1 A když se to všecko dokonalo, vyšed všecken lid Izraelský, což ho koli bylo v mestech Judských, stroskotali modly a posekali háje, a poborili výsosti a oltáre po vší zemi Judské a Beniamin, i v Efraim a Manasses, až vše dokonali. Potom navrátili se všickni synové Izraelští, jeden každý k vládarství svému do mesta svého.
\par 2 Ezechiáš pak zase zrídil trídy kneží a Levítu podlé tríd jejich, jednoho každého podlé povinnosti prisluhování jeho, kneží a Levíty k obetem zápalným a pokojným, aby prisluhovali, a oslavovali i chválili Hospodina v branách vojska jeho.
\par 3 Též i cástka z statku královského dávána k obetování zápalu, k zápalum jitrním i vecerním, též k zápalum sobotním, a na novmesíce i na slavnosti výrocní, jakož psáno jest v zákone Hospodinove.
\par 4 Prikázal také lidu prebývajícímu v Jeruzaléme, dávati díl knežím a Levítum, aby tím ochotnejší byli v zákone Hospodinove.
\par 5 A jakž vyšel tento rozkaz, snesli synové Izraelští mnoho prvotin obilé, mstu, oleje, ovoce palmového i všech úrod polních, a desátek ze všech vecí hojný prinesli.
\par 6 Synové také Izraelští a Judští, kteríž bydlili v mestech Judských, i oni desátky z skotu a bravu, a desátky svaté, posvecené Hospodinu Bohu svému, snášeli a skládali po hromadách.
\par 7 Tretího mesíce pocali zakládati tech hromad, a mesíce sedmého dokonali.
\par 8 Prišed pak Ezechiáš s knížaty, a vidouce hromady ty, dobrorecili Hospodinu a lidu jeho Izraelskému.
\par 9 I tázal se Ezechiáš kneží a Levítu o tech hromadách.
\par 10 Jemuž odpovedel Azariáš knez nejvyšší z domu Sádochova, rka: Jakž pocali tech obetí prinášeti do domu Hospodinova, jedli jsme a nasyceni jsme, a ješte zustává hojne; nebo Hospodin požehnal lidu svému, že toho tak mnoho pozustalo.
\par 11 Tedy rozkázal Ezechiáš nadelati špižírní pri domu Hospodinovu. I nadelali,
\par 12 A snášeli tam verne obeti a desátky, i veci posvecené, a nad tím byl za správce Konaniáš Levíta, a Simei bratr jeho, druhý.
\par 13 Jechiel pak a Azaziáš , Nachat a Azael, a Jerimot a Jozabad, Eliel, Izmachiáš, Machat a Benaiáš, privzati za úredníky od Konaniáše a Simei bratra jeho, podlé porucení Ezechiáše krále, a Azariáše knížete domu Božího.
\par 14 Chóre pak syn Imny, Levíta, vrátný brány východní, byl nad tím, což dobrovolne obetovali Bohu, aby rozdeloval obeti Hospodinovy a veci svatosvaté.
\par 15 Jemuž ku pomoci byli Eden, Miniamin, Jesua, Semaiáš, Amariáš a Sechaniáš po mestech knežských, muži hodnoverní, aby rozdelovali bratrím svým díly, jakž velikému, tak malému,
\par 16 (Mimo ty, kteríž z pokolení jejich byli pohlaví mužského, od tríletých a výše), každému vcházejícímu do domu Hospodinova ku povinnostem denním, podlé úradu jejich, a podlé služby jich i podlé trídy jejich,
\par 17 A tem, kteríž pocteni byli v rodine knežské po celedech otcu jejich, i Levítum, ode dvadcítiletého a výše, podlé služby jich v trídách jejich,
\par 18 Též i rodine jejich, na všecky malické jich, ženy jejich, syny jejich i dcery jejich, a všemu množství; nebo doverne posvetili se v svatosti.
\par 19 Synum také Aronovým knežím v predmestích mest jejich po všech mestech ti muži, kteríž zejména zaznamenáni jsou, dávali díly každému pohlaví mužského z kneží i každému z rodiny Levítu.
\par 20 A tak ucinil Ezechiáš ve všem Judstvu, a cinil, což dobrého, prímého a pravého jest pred Hospodinem Bohem svým.
\par 21 A ve všelikém díle, kteréžkoli zacal pri službe domu Božího, a v zákone i v prikázaní, hledaje Boha svého, celým srdcem svým to cinil, a štastne se mu vedlo.

\chapter{32}

\par 1 Po tech vecech a stálém narízení jejich, pritáh Senacherib král Assyrský, vtrhl do Judstva, a položil se proti mestum hrazeným, a uložil jich zdobývati sobe.
\par 2 Vida pak Ezechiáš, že pritáhl Senacherib, a že tvár jeho obrácena jest k boji proti Jeruzalému,
\par 3 Uradil se s knížaty a rytíri svými, aby zasypali vody studnic, kteréž byly vne za mestem. I pomáhali jemu.
\par 4 Nebo shromáždilo se lidu množství, a zasypali všecky studnice i potok rozvodnující se u prostred zeme, rkouce: Proc prijdouce králové Assyrští, mají najíti vody tak mnoho?
\par 5 A posiliv se, vystavel všecku zed zborenou, a zopravoval veže, a vne zed druhou. Upevnil i Mello mesta Davidova, k tomu také nadelal brane velmi mnoho i pavéz.
\par 6 Zrídil též hejtmany válecné nad lidem, a shromáždil je k sobe do ulice u brány mestské, a mluvil jim prívetive, rka:
\par 7 Posilnte se a zmužile sobe pocínejte, nebojte se, ani strachujte tvári krále Assyrského, ani všeho množství, kteréž jest s ním; nebo vetší jest s námi, než s ním.
\par 8 S nímt jest ráme cloveka, s námi pak jest Hospodin Buh náš, ku pomoci naší a k bojování za nás. I zpolehl lid na slova Ezechiáše krále Judského.
\par 9 Potom poslal Senacherib král Assyrský služebníky své do Jeruzaléma, (sám pak ležel u Lachis, a všecko království jeho bylo s ním), k Ezechiášovi králi Judskému, i ke všemu lidu Judskému, kterýž byl v Jeruzaléme, rka:
\par 10 Takto praví Senacherib král Assyrský: V cem vy doufáte, že zustáváte v ohrade v Jeruzaléme?
\par 11 Zdaliž Ezechiáš nenavodí vás, aby vás zmoril hladem a žízní, prave: Hospodin Buh náš vytrhne nás z ruky krále Assyrského?
\par 12 Zdaliž jest sám Ezechiáš nepoboril výsostí jeho a oltáru jeho, a prikázal Judovi a obyvatelum Jeruzalémským, rka: Pred jedním oltárem klaneti se budete, a na nem kaditi.
\par 13 Nevíte-liž, co jsem ucinil já i otcové moji všechnem národum zemí? Zdaliž jak mohli bohové národu a zemí vytrhnouti zeme své z ruky mé?
\par 14 Kdo byl mezi všemi bohy národu tech, kteréž jsou vyplénili otcové moji, kterýž by mohl vytrhnouti lid svuj z ruky mé? Aby pak mohl Buh váš vytrhnouti vás z ruky mé?
\par 15 Protož tedy necht vás nesvodí Ezechiáš, ani vás namlouvá, aniž mu verte. Kdyžte nemohl žádný buh všech národu a království vytrhnouti lidu svého z ruky mé, jako i z ruky otcu mých, nadtot ovšem bohové vaši nevytrhnou vás z ruky mé.
\par 16 Pres to ješte mluvili služebníci jeho i proti Hospodinu Bohu, i proti Ezechiášovi služebníku jeho.
\par 17 Psal také listy, rouhaje se Hospodinu Bohu Izraelskému, a mluve proti nemu, rka: Jakož bohové národu zemských nevytrhli lidu svého z ruky mé, tak nevytrhne Buh Ezechiášuv lidu svého z ruky mé.
\par 18 Kriceli pak hlasem velikým Židovsky proti lidu Jeruzalémskému, kterýž byl na zdi, aby strach na ne pustili a predesili je, aby tak vzali mesto.
\par 19 A tak mluvili o Bohu Jeruzalémském, jako o jiných bozích národu zeme, dílu rukou lidských.
\par 20 Tedy modlil se Ezechiáš král, a Izaiáš prorok syn Amosuv z príciny té, a volali k nebi.
\par 21 I poslal Hospodin andela, kterýž vyhladil každého udatného i vývodu i kníže v vojšte krále Assyrského, tak že se s hanbou velikou navrátil do zeme své. A když všel do chrámu boha svého, ti, kteríž vyšli z života jeho, zamordovali ho tam mecem.
\par 22 A tak vysvobodil Hospodin Ezechiáše a obyvatele Jeruzalémské z ruky Senacheriba krále Assyrského, a z ruky všech, a provázel je všudy vukol.
\par 23 Tedy mnozí prinášeli obeti Hospodinu do Jeruzaléma, ano i dary drahé Ezechiášovi králi Judskému, tak že potom vznešen jest u všech národu.
\par 24 V tech dnech roznemohl se Ezechiáš až k smrti, i modlil se Hospodinu. Kterýž promluvil k nemu, a ukázal mu zázrak.
\par 25 Ale Ezechiáš nebyl vdecen dobrodiní sobe ucineného, nebo pozdvihlo se srdce jeho. Procež povstala proti nemu prchlivost, i proti Judovi a Jeruzalému.
\par 26 Ale když se pokoril Ezechiáš pro to pozdvižení srdce svého i s obyvateli Jeruzalémskými, neprišla na ne prchlivost Hospodinova za dnu Ezechiášových.
\par 27 Mel pak Ezechiáš bohatství a slávu velmi velikou; nebo nashromáždil sobe pokladu stríbra a zlata i kamení drahého a vonných vecí, i pavéz i všelijakých klénotu.
\par 28 A mel špižírny pro úrody obilí, mstu, oleje, i stáje pro všeliká hovada a chlévy pro dobytek.
\par 29 Mesta také zdelal sobe, a mel bravu a skotu množství; nebo Buh dal jemu zboží náramne veliké.
\par 30 Tentýž Ezechiáš zasypal tok vody Gihonu horejší, a prímo vedl jej dolu k západní strane mesta Davidova, a štastne se vedlo Ezechiášovi ve všech skutcích jeho.
\par 31 Toliko pri poselství knížat Babylonských poslaných k nemu, aby se vyptali na zázrak, kterýž se byl stal v zemi, opustil ho Buh, aby ho zkusil, aby známé bylo všecko, co bylo v srdci jeho.
\par 32 Jiné pak veci Ezechiášovy i pobožnost jeho zapsány jsou v proroctví Izaiáše proroka syna Amosova, a v knize o králích Judských a Izraelských.
\par 33 I usnul Ezechiáš s otci svými, a pochovali jej výše nad hroby potomku Davidových, a ucinili jemu poctivost pri smrti jeho všecken Juda i obyvatelé Jeruzalémští. A kraloval Manasses syn jeho místo neho.

\chapter{33}

\par 1 Ve dvanácti letech byl Manasses, když pocal kralovati, a padesáte pet let kraloval v Jeruzaléme.
\par 2 Cinil pak to, což jest zlého pred ocima Hospodinovýma, vedlé ohavností tech národu, kteréž Hospodin vyhnal pred syny Izraelskými.
\par 3 Nebo vzdelal zase výsosti, kteréž byl rozboril Ezechiáš otec jeho, a vystavel oltáre Bálum, a vysadil háje, a klaneje se všemu vojsku nebeskému, sloužil jim.
\par 4 Vzdelal také oltáre v dome Hospodinove, o nemž byl rekl Hospodin: V Jeruzaléme bude jméno mé na veky.
\par 5 Vzdelal, pravím, oltáre všemu vojsku nebeskému ve dvou síních domu Hospodinova.
\par 6 Presto dal provoditi syny své skrze ohen v údolí Benhinnom, a šetril casu, s hadacstvím a s kouzly se obíral, a narídil zaklinace a carodejníky, a mnoho zlého páchal pred ocima Hospodinovýma, popouzeje ho.
\par 7 Postavil také obraz rytý, kterýž byl udelal, v dome Božím, o kterémž byl rekl Buh Davidovi a Šalomounovi synu jeho: V dome tomto a v Jeruzaléme, kterýž jsem vyvolil ze všech pokolení Izraelských, oslavím jméno své na veky.
\par 8 Aniž více pohnu nohou lidu Izraelského z zeme, kterouž jsem oddelil otcum vašim, jen toliko budou-li šetriti, aby plnili všecko to, což jsem jim prikázal, všecken zákon, ustanovení a soudy skrze Mojžíše vydané.
\par 9 Ale Manasses uvedl v blud Judské i obyvatele Jeruzalémské, tak že cinili horší veci nežli ti národové, kteréž vyplénil Hospodin pred tvárí synu Izraelských.
\par 10 A ackoli mluvil Hospodin k Manassesovi a k lidu jeho, oni však nepozorovali.
\par 11 Procež privedl na ne Hospodin hejtmany vojska krále Assyrského, kteríž jali Manassesa v trní, a svázavše ho dvema retezy ocelivými, dovedli jej do Babylona.
\par 12 Tam pak jsa sevrín, modlil se Hospodinu Bohu svému, a ponižoval se velmi pred oblícejem Boha otcu svých,
\par 13 A modlil se jemu. I naklonil se k nemu, a vyslyšel modlitbu jeho, a uvedl jej zase do Jeruzaléma do království jeho. Tehdy poznal Manasses, že sám Hospodin jest Bohem.
\par 14 A potom vystavel zed zevnitrní mesta Davidova k západní strane Gihonu potoku, až kudy se chodí k bráne rybné, a obehnal Ofel, a vyhnal ji velmi vysoko. Rozsadil také hejtmany vojska po všech mestech hrazených v Judstvu.
\par 15 Vymetal také bohy cizí a rytinu z domu Hospodinova, a všecky oltáre, kterýchž byl nadelal na hore domu Hospodinova i v Jeruzaléme, a vyházel za mesto.
\par 16 Opravil zase i oltár Hospodinuv, a obetoval na nem obeti pokojné a díkcinení, a prikázal Judským, aby sloužili Hospodinu Bohu Izraelskému.
\par 17 A však vždy ješte lid obetoval na výsostech, ale toliko Hospodinu Bohu svému.
\par 18 Jiné pak veci Manassesovy, i modlitba jeho k Bohu jeho, a slova proroku, kteríž mluvívali k nemu ve jménu Hospodina Boha Izraelského, to vše zapsáno v knize o králích Izraelských.
\par 19 Modlitba pak jeho i to, že vyslyšán jest, a každý hrích jeho, i prestoupení jeho, i místa, na kterýchž byl postavil výsosti, a zdelal háje a rytiny, ješte prvé než se pokoril, to vše zapsáno jest v knihách Chozai.
\par 20 I usnul Manasses s otci svými, a pochovali jej v dome jeho, a kraloval Amon syn jeho místo neho.
\par 21 Ve dvamecítma letech byl Amon, když pocal kralovati, a dve léte kraloval v Jeruzaléme.
\par 22 I cinil to, což jest zlého pred ocima Hospodinovýma, tak jako byl cinil Manasses otec jeho; nebo všechnem rytinám, kterýchž byl nadelal Manasses otec jeho, obetoval Amon a sloužil jim.
\par 23 Aniž se ponížil pred Hospodinem, jako se ponížil Manasses otec jeho, nýbrž on Amon mnohem více hrešil.
\par 24 Spuntovali se pak proti nemu služebníci jeho, a zamordovali jej v dome jeho.
\par 25 Tedy pobil lid zeme všecky ty, kteríž se byli spuntovali proti králi Amonovi, a ustanovil lid zeme krále Joziáše syna jeho místo neho.

\chapter{34}

\par 1 V osmi letech byl Joziáš, když kralovati pocal, a kraloval jedno a tridceti let v Jeruzaléme.
\par 2 Ten cinil to, což jest pravého pred ocima Hospodinovýma, chode po cestách Davida otce svého, a neuchýlil se na pravo ani na levo.
\par 3 Léta zajisté osmého kralování svého, když ješte mládencek byl, pocal hledati Boha Davida otce svého, dvanáctého pak léta pocal vycištovati Judy a Jeruzaléma od výsostí, háju, rytin a slitin.
\par 4 Nebo u prítomnosti jeho rozborili oltáre Bálu, i obrazy slunecné, kteríž byli na nich, zpodtínal. Též i háje a rytiny, i slitiny zdrobil a setrel, a rozsypal po hrobích tech, kteríž jim obetovávali.
\par 5 Kosti pak kneží popálil na oltárích jejich, a vycistil Judu a Jeruzalém,
\par 6 Též i mesta v pokolení Manasses, Efraim a Simeon, až i Neftalím i pustiny jejich vukol.
\par 7 A tak zkazil oltáre a háje, a rytiny stroskotal na kusy, a všecky obrazy slunecné vysekal po vší zemi Izraelské. Potom navrátil se do Jeruzaléma.
\par 8 Léta pak osmnáctého kralování svého, když vycistil zemi a dum Boží, poslal Safana syna Azaliášova, a Maaseiáše úredníka mesta, aJoacha syna Joachazova kanclíre, aby opraven byl dum Hospodina Boha jeho.
\par 9 Kteríž prišedše k Helkiášovi knezi nejvyššímu, dali peníze snesené do domu Božího, kteréž byli Levítové vrátní vybrali od Manassesa a Efraima, i ode všech ostatku Izraelských, i od všeho Judy a Beniamina, a navrátili se do Jeruzaléma.
\par 10 Dali je pak v ruce správcu toho díla, kteríž predstaveni byli nad domem Hospodinovým, a ti vydávali je delníkum, jenž delali v dome Hospodinove, opravujíce a utvrzujíce jej.
\par 11 I dali tesarum a stavitelum, aby najednali kamení tesaného, a dríví k vazbám i k trámum domu, kteréž byli zborili králové Judští.
\par 12 Muži pak ti verne se meli pri té práci, nad nimiž ustanoveni Jachat a Abdiáš Levítové z synu Merari, Zachariáš a Mesullam z synu Kahat, aby s pilností prídrželi k práci. A z tech Levítu každý umel hráti na nástroje muzické.
\par 13 Ale nad nosici, i kteríž prídrželi delníky ke všeliké práci, byli z Levítu písari, úredníci a vrátní.
\par 14 A když vynášeli peníze, kteréž byli sneseny do domu Hospodinova, našel Helkiáš knez knihu zákona Hospodinova vydaného skrze Mojžíše.
\par 15 I rekl Helkiáš k Safanovi písari: Knihu zákona nalezl jsem v dome Hospodinove. I dal Helkiáš tu knihu Safanovi.
\par 16 Safan pak prinesl tu knihu králi, a pri tom jemu oznámil, rka: Všecko, cožkoli bylo poruceno služebníkum tvým, delají.
\par 17 Nebo sebravše peníze, což se jich koli nalezlo v dome Hospodinove, vydali je v ruce úredníkum a delníkum.
\par 18 Oznámil také Safan písar králi, rka: Knihu mi dal Helkiáš knez. I cetl v ní Safan pred králem.
\par 19 Procež když slyšel král slova zákona, roztrhl roucho své.
\par 20 A rozkázal král Helkiášovi, též Achikamovi synu Safanovu, a Abdonovi synu Míchovu, a Safanovi písari, a Asaiášovi služebníku královskému, rka:
\par 21 Jdete, poradte se s Hospodinem o mne i o pozustalý lid Izraelský a Judský, z strany slov knihy této, kteráž jest nalezena; nebo veliký jest hnev Hospodinuv, kterýž jest vylit na nás, proto že neostríhali otcové naši slova Hospodinova, aby cinili všecko, což psáno jest v knize této.
\par 22 Tedy šli, Helkiáš a kteríž byli pri králi, k Chulde prorokyni, manželce Salluma syna Tekue, syna Chasrova, strážného nad rouchem, nebo bydlila v Jeruzaléme na druhé strane, a mluvili k ní ty veci.
\par 23 Kteráž rekla jim: Toto praví Hospodin Buh Izraelský: Povezte muži tomu, kterýž poslal vás ke mne,
\par 24 Takto praví Hospodin: Aj, já uvedu zlé veci na toto místo a na obyvatele jeho, všecka zlorecení napsaná v knize té, kterouž ctli pred králem Judským,
\par 25 Proto že opustili mne a kadili bohum cizím, aby mne popouzeli všemi skutky rukou svých. Z té príciny vylita bude prchlivost má na toto místo, aniž bude uhašena.
\par 26 Králi pak Judskému, kterýž poslal vás, abyste se tázali Hospodina, takto povíte: Toto praví Hospodin Buh Izraelský o slovích tech, kteráž jsi slyšel:
\par 27 Ponevadž obmekceno jest srdce tvé, a ponížil jsi se pred tvárí Boží, kdyžs slyšel slova jeho proti místu tomuto a proti obyvatelum jeho, a ponižuje se prede mnou, roztrhl jsi roucho své, a plakals prede mnou, i já vyslyšel jsem te, praví Hospodin.
\par 28 Aj, já pripojím te k otcum tvým, a pochován budeš v hrobích svých v pokoji, aby nevidely oci tvé niceho z toho zlého, kteréž já uvedu na místo toto a na obyvatele jeho. I oznámili králi tu rec.
\par 29 Tedy poslav král, shromáždil všecky starší Judské a Jeruzalémské.
\par 30 A vstoupil král do domu Hospodinova, a všickni muži Judští i obyvatelé Jeruzalémští, kneží, Levítové a všecken lid od velikého až do malého, i cetl, aby všickni slyšeli všecka slova knihy smlouvy, kteráž byla nalezena v dome Hospodinove.
\par 31 Potom stoje král na míste svém, ucinil smlouvu pred Hospodinem, že bude následovati Hospodina, a ostríhati prikázaní jeho i svedectví jeho a ustanovení jeho z celého srdce svého a ze vší duše své, a plniti slova smlouvy té, kteráž jsou v té knize zapsána.
\par 32 I rozkázal, aby k témuž každý stál, kdož by koli nalezen byl v Jeruzaléme a v Beniaminovi. I cinili obyvatelé Jeruzalémští podlé smlouvy Boží, Boha otcu svých.
\par 33 Tehdáž také vyprázdnil Joziáš všecky ohavnosti ze všech zemí synu Izraelských, a prídržel všecky, kterížkoli byli v Izraeli, k tomu, aby sloužili Hospodinu Bohu svému. Po všecky dny jeho neodstoupili od následování Hospodina Boha otcu svých.

\chapter{35}

\par 1 Slavil také Joziáš v Jeruzaléme velikunoc Hospodinu, a zabili beránka velikonocního ctrnáctého dne mesíce prvního.
\par 2 A postaviv kneží v strážech jejich, utvrdil je v prisluhování domu Hospodinova.
\par 3 I rekl Levítum, kteríž pripravovali za všecken lid Izraelský veci svaté Hospodinu: Vneste truhlu svatosti do domu, kterýž ustavel Šalomoun syn Daviduv, král Izraelský. Nebudet vám bremenem na ramenou. A tak nyní služte Hospodinu Bohu svému a lidu jeho Izraelskému.
\par 4 A pripravte se po celedech otcu svých, po trídách svých, tak jakž narídil David král Izraelský, a podlé narízení Šalomouna syna jeho.
\par 5 A stujte v svatyni podlé rozdílu celedí otcovských, bratrí národu svého, a podlé rozdelení každé celedi otcovské Levítu.
\par 6 A tak zabíte beránka, a posvette se, a pripravte bratrím svým, chovajíce se vedlé slova Hospodinova vydaného skrze Mojžíše.
\par 7 Daroval pak Joziáš všemu množství z stáda beránku a kozelcu, vše k obetem velikonocním, podlé toho všeho, jakž postacovalo, v poctu tridcet tisícu, a skotu tri tisíce, to vše z statku královského.
\par 8 Knížata jeho také k lidu, knežím i Levítum štedre se ukázali: Helkiáš, Zachariáš a Jechiel, prední v dome Božím, dali knežím k obetem velikonocním dva tisíce a šest set bravu a skotu tri sta.
\par 9 Konaniáš pak s Semaiášem a Natanaelem bratrími svými, též Chasabiáš, Jehiel, Jozabad, prední z Levítu, darovali jiným Levítum k obetem velikonocním pet tisíc bravu a skotu pet set.
\par 10 A když pripraveno bylo všecko k službe, stáli kneží na míste svém, a Levítové v trídách svých, podlé rozkazu královského.
\par 11 I zabíjeli beránky velikonocní, kneží pak kropili krví, berouce z rukou jejich, a Levítové vytahovali z koží.
\par 12 Tedy oddelili zápal, aby dali lidu podlé rozdelení domu celedí otcovských, tak aby obetováno bylo Hospodinu, jakž psáno jest v knize Mojžíšove. Tolikéž i s voly ucinili.
\par 13 I pekli beránky velikonocní ohnem podlé obyceje, ale jiné veci posvecené varili v hrncích, v kotlících a v pánvicích, a rozdávali rychle všemu lidu.
\par 14 Potom pak pripravili sobe a knežím. Nebo kneží, synové Aronovi, zápaly a tuky až do noci obetovali, protož Levítové pripravili sobe i knežím, synum Aronovým.
\par 15 Též i zpeváci, synové Azafovi, stáli na míste svém podlé rozkázaní Davidova Azafovi, Hémanovi i Jedutunovi, proroku královskému, vydaného. Vrátní také pri jedné každé bráne nemeli odcházeti od své povinnosti, procež bratrí jejich Levítové jiní strojili jim.
\par 16 A tak nastrojena byla všecka služba Hospodinova v ten den, v slavení hodu beránka a obetování obeti zápalné na oltári Hospodinovu, podlé rozkázaní krále Joziáše.
\par 17 I slavili synové Izraelští, což se jich našlo, hod beránka v ten cas a slavnost presnic za sedm dní.
\par 18 Nebyla pak slavena velikanoc té podobná v Izraeli ode dnu Samuele proroka, aniž který z králu Izraelských slavil hodu beránka podobného tomu, kterýž slavil Joziáš s knežími a Levíty, i se vším lidem Judským a Izraelským, což se ho našlo, i s obyvateli Jeruzalémskými.
\par 19 Osmnáctého léta kralování Joziášova slaven byl ten hod beránka.
\par 20 Potom po všem, když již spravil Joziáš dum Boží, pritáhl Nécho král Egyptský, aby bojoval proti Charkemis podlé Eufraten, Joziáš pak vytáhl proti nemu.
\par 21 Kterýž ac poslal k nemu posly, rka: Já nic nemám s tebou ciniti, králi Judský. Ne proti tobe, slyš ty, dnes táhnu, ale proti domu, kterýž se mnou bojuje, kamž mi rozkázal Buh, abych pospíšil. Nebojuj s Bohem, kterýž se mnou jest, at te neshladí:
\par 22 Joziáš však neodvrátil tvári své od neho, ale aby bojoval s ním, zmenil odev svuj, aniž uposlechl slov Néchových, pošlých z úst Božích. A tak pritáhl, aby se s ním potýkal na poli Mageddo.
\par 23 I postrelili strelci krále Joziáše. Tedy rekl král služebníkum svým: Odvezte mne, nebot jsem velmi nemocen.
\par 24 Takž prenesli jej služebníci jeho z toho vozu, a vložili ho na druhý vuz, kterýž mel, a dovezli jej do Jeruzaléma. I umrel a pochován jest v hrobích otcu svých. I plakal všecken Juda a Jeruzalém nad Joziášem.
\par 25 Naríkal také i Jeremiáš nad Joziášem, a zpívávali všickni zpeváci a zpevakyne v naríkáních svých o Joziášovi až do dnešního dne, kteráž uvedli v obycej Izraeli. A ta jsou zapsána v pláci Jeremiášovu.
\par 26 Jiné pak veci Joziášovy, i pobožnost jeho, jakž napsáno jest v zákone Hospodinove,
\par 27 I skutkové jeho první i poslední, to vše zapsáno jest v knize o králích Izraelských a Judských.

\chapter{36}

\par 1 Tedy vzal lid zeme Joachaza syna Joziášova, a ustanovili ho králem na míste otce jeho v Jeruzaléme.
\par 2 Ve trimecítma letech byl Joachaz, když pocal kralovati, a tri mesíce kraloval v Jeruzaléme.
\par 3 Nebo vzal jej král Egyptský z Jeruzaléma, a uložil na zemi pokutu sto centnéru stríbra a centnér zlata.
\par 4 A ustanovil král Egyptský Eliakima bratra jeho nad Judou a Jeruzalémem za krále, a promenil mu jméno jeho, aby sloul Joakim. Joachaza pak bratra jeho vzal Nécho, a zavedl ho do Egypta.
\par 5 V petmecítma letech byl Joakim, když pocal kralovati, a jedenáct let kraloval v Jeruzaléme. A když cinil to, což jest zlého pred ocima Hospodina Boha svého,
\par 6 Pritáhl proti nemu Nabuchodonozor král Babylonský, a svázal jej retezy ocelivými, aby jej zavedl do Babylona.
\par 7 Nádoby také domu Hospodinova zavezl Nabuchodonozor do Babylona, a dal je do chrámu svého v Babylone.
\par 8 Jiné pak veci Joakimovy i ohavnosti jeho, kteréž páchal, i což se nalézalo pri nem, to vše zapsáno jest v knize o králích Izraelských a Judských. I kraloval Joachin syn jeho místo neho.
\par 9 V osmi letech byl Joachin, když kralovati pocal, a kraloval tri mesíce a deset dní v Jeruzaléme, a cinil to, což bylo zlého pred ocima Hospodinovýma.
\par 10 Potom pak po roce poslal král Nabuchodonozor, a dal ho zavésti do Babylona s klénoty domu Hospodinova, a ustanovil králem Sedechiáše, príbuzného jeho, nad Judou a Jeruzalémem.
\par 11 V jedenmecítma letech byl Sedechiáš, když pocal kralovati, a jedenáct let kraloval v Jeruzaléme.
\par 12 I cinil to, což jest zlého pred ocima Hospodina Boha svého, aniž se pokoril pred Jeremiášem prorokem mluvícím rec Hospodinovu.
\par 13 Nýbrž i Nabuchodonozorovi králi se zprotivil, jenž ho byl prísahou zavázal skrze Boha, a zatvrdiv šíji svou, urputne uložil v srdci svém, aby se nenavracoval k Hospodinu Bohu Izraelskému.
\par 14 Ano i všecka knížata, kneží i lid v mnohá prestoupení se vydali, podlé všech ohavností pohanských, a poškvrnili domu Hospodinova, jehož byl posvetil v Jeruzaléme.
\par 15 A když posílal Hospodin Buh otcu jejich k nim posly své, ráno vstávaje a posílaje, proto že se prívetive mel k lidu svému a k príbytku svému:
\par 16 Posmívali se poslum Božím, a pohrdali slovy jeho, a za svudce meli proroky jeho, až se rozpálila prchlivost Hospodinova na lid jeho, tak aby nebylo žádného ulécení.
\par 17 I privedl na ne krále Kaldejského, kterýž zmordoval mládence jejich mecem v dome svatyne jejich, a neodpustil mládenci ani panne, starci ani kmeti. Všecky dal v ruku jeho.
\par 18 K tomu i všecky nádoby domu Božího, veliké i malé, i poklady domu Hospodinova, i poklady královské i knížat jeho, všecko zavezl do Babylona.
\par 19 A vypálili dum Boží, a zborili zed Jeruzalémskou, též i všecky paláce v nem popálili, ano i všelijaké klénoty drahé v nem zkazili.
\par 20 A což jich pozustalo po meci, to prevedl do Babylona, a byli služebníci jeho i synu jeho, dokudž nekraloval král Perský,
\par 21 Aby se naplnila rec Hospodinova skrze ústa Jeremiášova, dokudž zeme nevykonala sobot svých; po všecky dny, pokudž byla pustá, odpocívala, až se vyplnilo sedmdesáte let.
\par 22 Potom léta prvního Cýra krále Perského, aby se naplnila rec Hospodinova skrze ústa Jeremiášova, vzbudil Hospodin ducha Cýrova krále Perského. Kterýž dal provolati po všem království svém, ano také i rozepsal, rka:
\par 23 Toto praví Cýrus král Perský: Všecka království zeme dal mi Hospodin Buh nebeský, a ten mi porucil, abych mu vystavel dum v Jeruzaléme, kterýž jest v Judstvu. Kdo jest mezi vámi ze všeho lidu jeho, Hospodin Buh jeho budiž s ním, a at jde.

\end{document}