\begin{document}

\title{Ezra}

\chapter{1}

\par 1 Léta prvního Cýra krále Perského, aby se naplnila rec Hospodinova skrze ústa Jeremiášova, vzbudil Hospodin ducha Cýrova krále Perského, aby dal provolati po všem království svém, ano také i rozepsal, rka:
\par 2 Toto praví Cýrus král Perský: Všecka království zeme dal mi Hospodin Buh nebeský, a on mi porucil, abych mu vystavel dum v Jeruzaléme, kterýž jest v Judstvu.
\par 3 Kdo jest mezi vámi ze všeho lidu jeho, budiž Buh jeho s ním, a nechat jde do Jeruzaléma, kterýž jest v Judstvu, a staví dum Hospodina Boha Izraelského, toho Boha, kterýž jest v Jeruzaléme.
\par 4 Kdož by pak zustal na kterémkoli míste, kdež jest pohostinu, lidé místa toho at mu pomohou stríbrem a zlatem, statkem i hovady, mimo obeti dobrovolné k domu Božímu, kterýž jest v Jeruzaléme.
\par 5 I povstali prední z celedí otcu z pokolení Judova a Beniaminova, kneží také a Levítové i každý, cíhož ducha vzbudil Buh, aby šli k stavení domu Hospodinova, kterýž jest v Jeruzaléme.
\par 6 Jimž všickni ti, kteríž bydlili okolo nich, pomáhali nádobami stríbrnými a zlatými, statkem i hovady, a vecmi drahými, mimo všecko, což dobrovolne obetováno bylo.
\par 7 Král také Cýrus vydal nádoby domu Hospodinova, kteréž byl pobral Nabuchodonozor z Jeruzaléma, a dal je byl do domu Boha svého.
\par 8 Vydal je pak Cýrus král Perský skrze Mitridata správce nad poklady, kterýž je vyctl Sesbazarovi knížeti Judskému.
\par 9 A tento jest pocet jejich: Medenic zlatých tridceti, medenic stríbrných tisíc, nožu devetmecítma.
\par 10 Koflíku zlatých tridceti, koflíku stríbrných prostejších ctyri sta a deset, nádob jiných na tisíce.
\par 11 Všech nádob zlatých i stríbrných pet tisíc a ctyri sta. Všecko to odnesl Sesbazar, když se stehoval lid zajatý z Babylona do Jeruzaléma.

\chapter{2}

\par 1 Tito pak jsou lidé té krajiny, kteríž se z zajetí a prestehování toho brali, jakž je byl prestehoval Nabuchodonozor král Babylonský do Babylona, a navrátili se do Jeruzaléma a do Judstva, jeden každý do mesta svého,
\par 2 Kteríž prišli s Zorobábelem, s Jesua, s Nehemiášem, Saraiášem, Reelaiášem, Mardocheem, Bilsanem, Misparem, Bigvajem, Rechumem a Baanou. Pocet mužu z lidu Izraelského:
\par 3 Synu Farosových dva tisíce, sto sedmdesáte dva.
\par 4 Synu Sefatiášových tri sta sedmdesáte dva.
\par 5 Synu Arachových sedm set sedmdesát pet.
\par 6 Synu Pachat Moábových, synu Jesue a Joábových dva tisíce, osm set a dvanácte.
\par 7 Synu Elamových tisíc, dve ste padesáte ctyri.
\par 8 Synu Zattuových devet set ctyridceti pet.
\par 9 Synu Zakkai sedm set a šedesát.
\par 10 Synu Báni šest set ctyridceti dva.
\par 11 Synu Bebai šest set trimecítma.
\par 12 Synu Azgadových tisíc, dve ste dvamecítma.
\par 13 Synu Adonikamových šest set šedesáte šest.
\par 14 Synu Bigvai dva tisíce, padesáte šest.
\par 15 Synu Adinových ctyri sta padesáte ctyri.
\par 16 Synu Aterových z Ezechiáše devadesát osm.
\par 17 Synu Bezai tri sta trimecítma.
\par 18 Synu Jorahových sto a dvanácte.
\par 19 Synu Chasumových dve ste trimecítma.
\par 20 Synu Gibbarových devadesáte pet.
\par 21 Synu Betlémských sto trimecítma.
\par 22 Mužu Netofatských padesáte šest.
\par 23 Mužu Anatotských sto osmmecítma.
\par 24 Synu Azmavetských ctyridceti dva.
\par 25 Synu Kariatarimských, Kafirských a Berotských sedm set ctyridceti a tri.
\par 26 Synu Ráma a Gabaa šest set jedenmecítma.
\par 27 Mužu Michmas sto dvamecítma.
\par 28 Mužu z Bethel a Hai dve ste trimecítma.
\par 29 Synu z Nébo padesáte dva.
\par 30 Synu Magbisových sto padesáte šest.
\par 31 Synu Elama druhého tisíc, dve ste padesáte ctyri.
\par 32 Synu Charimových tri sta dvadceti.
\par 33 Synu Lodových, Chadidových a Onových sedm set dvadceti pet.
\par 34 Synu Jerecho tri sta ctyridceti pet.
\par 35 Synu Senaa tri tisíce, šest set a tridceti.
\par 36 Kneží, synu Jedaiášových z domu Jesua, devet set sedmdesáte tri.
\par 37 Synu Immerových tisíc, padesáte dva.
\par 38 Synu Paschurových tisíc, dve ste ctyridceti sedm.
\par 39 Synu Charimových tisíc a sedmnácte.
\par 40 Levítu, synu Jesua a Kadmiele, synu Hodaviášových, sedmdesáte ctyri.
\par 41 Zpeváku, synu Azafových, sto dvadceti osm.
\par 42 Synu vrátných, synu Sallumových, synu Aterových, synu Talmonových, synu Akkubových, synu Chatita, synu Sobai, všech sto tridceti devet.
\par 43 Netinejských, synu Zicha, synu Chasufa, synu Tabbaot,
\par 44 Synu Keros, synu Siaha, synu Fádon,
\par 45 Synu Lebana, synu Chagaba, synu Akkub,
\par 46 Synu Chagab, synu Samlai, synu Chanan,
\par 47 Synu Giddel, synu Gachar, synu Reaia,
\par 48 Synu Rezin, synu Nekoda, synu Gazam,
\par 49 Synu Uza, synu Paseach, synu Besai,
\par 50 Synu Asna, synu Meunim, synu Nefusim,
\par 51 Synu Bakbuk, synu Chakufa, synu Charchur,
\par 52 Synu Bazlut, synu Mechida, synu Charsa,
\par 53 Synu Barkos, synu Sisera, synu Tamach,
\par 54 Synu Neziach, synu Chatifa,
\par 55 Synu služebníku Šalomounových, synu Sotai, synu Soferet, synu Feruda,
\par 56 Synu Jaala, synu Darkon, synu Giddel,
\par 57 Synu Sefatiášových, synu Chattil, synu Pocheret Hazebaim, synu Ami,
\par 58 Všech Netinejských a synu služebníku Šalomounových tri sta devadesáte dva.
\par 59 Tito také byli, kteríž šli z Telmelach, Telcharsa, Cherub, Addan a Immer, ale nemohli ukázati domu otcu svých a semene svého, že by z Izraele byli:
\par 60 Synu Delaiášových, synu Tobiášových, synu Nekodových šest set padesáte dva.
\par 61 A z synu knežských synové Chabaiášovi, synové Kózovi, synové Barzillai, kterýž pojav sobe ze dcer Barzillai Galádského manželku, nazván jest jménem jejich.
\par 62 Ti vyhledávali zapsání o sobe, chtíce prokázati rod svuj, ale nenašlo se. Protož zbaveni jsou knežství.
\par 63 A zapovedel jim Tirsata, aby nejedli z vecí svatosvatých, dokudž by nestál knez s urim a thumim.
\par 64 Všeho toho shromáždení pospolu ctyridceti a dva tisíce, tri sta šedesáte,
\par 65 Krome služebníku jejich, a devek jejich, jichž bylo sedm tisíc, tri sta tridceti sedm. A mezi nimi bylo zpeváku a zpevakyní dve ste.
\par 66 Koní jejich sedm set tridceti šest, mezku jejich dve ste ctyridceti pet.
\par 67 Velbloudu jejich ctyri sta tridceti pet, oslu šest tisíc, sedm set a dvadceti.
\par 68 Z knížat pak celedí otcovských, nekterí, když prišli k domu Hospodinovu, kterýž byl v Jeruzaléme, dobrovolne se oddavše, aby staveli dum Boží na gruntech jeho,
\par 69 Vedlé možnosti své dali náklad k dílu: Zlata jeden a šedesáte tisíc drachem, stríbra pak pet tisíc liber, a sukní knežských sto.
\par 70 A tak osadili se kneží i Levítové a nekterí z lidu, i zpeváci i vrátní a Netinejští v mestech svých, i všecken Izrael v mestech svých.

\chapter{3}

\par 1 Když pak nastal mesíc sedmý, a byli synové Izraelští v mestech, shromáždil se lid jednomyslne do Jeruzaléma.
\par 2 Tedy vstav Jesua syn Jozadakuv s bratrími svými knežími, a Zorobábel syn Salatieluv s bratrími svými, vzdelali oltár Boha Izraelského, aby obetovali na nem obeti zápalné, jakož psáno jest v zákone Mojžíše muže Božího.
\par 3 A když postavili oltár ten na základích jeho, ackoli se obávali národu jiných zemí, však obetovali na nem obeti zápalné Hospodinu, obeti zápalné ráno i vecer.
\par 4 Drželi také slavnost stánku, jakož psáno jest, obetujíce zápal na každý den, vedlé poctu a vedlé obyceje každého dne,
\par 5 A potom obet zápalnou ustavicnou, i na novmesíce i na každou slavnost Hospodinu posvecenou, i od každého dobrovolne obetujícího dobrovolnou obet Hospodinu.
\par 6 Od prvního dne toho mesíce sedmého pocali obetovati zápalu Hospodinu, ackoli chrám Hospodinuv ješte nebyl založen.
\par 7 I dali peníze kameníkum a remeslníkum, též potravy a nápoje i oleje Sidonským a Tyrským, aby vezli dríví cedrové z Libánu k mori Joppen, podlé povolení jim Cýra krále Perského.
\par 8 Léta pak druhého po jejich se navrácení k domu Božímu do Jeruzaléma, mesíce druhého, zacali Zorobábel syn Salatieluv, a Jesua syn Jozadakuv, i jiní bratrí jejich kneží a Levítové a všickni, kteríž byli prišli z toho zajetí do Jeruzaléma, a ustanovili Levíty od dvadcítiletých a výše, aby prihlédali k dílu domu Hospodinova.
\par 9 A tak postaven jest Jesua, synové jeho i bratrí jeho, Kadmiel i synové jeho, synové Judovi spolu, aby prihlédali k dílu pri dome Božím, potomci Chenadadovi, synové jejich i bratrí jejich Levítové.
\par 10 A když zakládali grunty stavitelé chrámu Hospodinova, postavili kneží zoblácené s trubami, a Levíty, syny Azafovy s cymbály, aby chválili Hospodina vedlé narízení Davida krále Izraelského.
\par 11 I prozpevovali jedni po druhých, chválíce a oslavujíce Hospodina, nebo dobrý jest, a že na veky trvá milosrdenství jeho nad Izraelem. Všecken také lid prokrikoval hlasem velikým, chválíce Hospodina, proto že založen byl dum Hospodinuv.
\par 12 Mnozí pak z kneží a z Levítu i z knížat celedí otcovských, starci, kteríž byli videli prvnejší dum, když zakládali tento dum pred ocima jejich, plakali hlasem velikým. Mnozí nazpet prokrikovali s radostí hlasem velikým,
\par 13 Tak že lid nemohl rozeznati hlasu prokrikování radostného od hlasu placícího lidu; nebo lid ten prokrikoval hlasem velikým, a hlas ten slyšán byl daleko.

\chapter{4}

\par 1 Uslyšavše pak neprátelé Judovi a Beniaminovi, že by ti, kteríž prestehováni byli, staveli chrám Hospodinu Bohu Izraelskému,
\par 2 Pristoupili k Zorobábelovi a k knížatum celedí otcovských, a rekli jim: Budeme s vámi staveti; nebo jako i vy hledati budeme Boha vašeho, jemuž i obeti obetujeme ode dnu Esarchaddona krále Assyrského, kterýž nás sem uvedl.
\par 3 Tedy rekl jim Zorobábel a Jesua i jiná knížata celedí otcovských z Izraele: Ne vám, ale nám náleží staveti dum Bohu našemu; nebo my sami staveti budeme Hospodinu Bohu Izraelskému, jakž prikázal nám král Cýrus, král Perský.
\par 4 A však lid té krajiny zemdléval ruce lidu Judského, a odhrožovali je, aby nestaveli.
\par 5 Anobrž i najímali proti nim rádce, aby rušili rady jejich, po všecky dny Cýra krále Perského, až do kralování Daria krále Perského.
\par 6 Nebo když kraloval Asverus, pri zacátku kralování jeho sepsali žalobu proti obyvatelum Judským a Jeruzalémským.
\par 7 (Tak jako za dnu Artaxerxa psal Bislam, Mitridates, Tabel a jiní tovaryši jeho k Artaxerxovi králi Perskému.) Písmo pak listu toho psáno bylo Syrsky, i vykládáno Syrsky.
\par 8 Rechum totiž kanclér a Simsai písar napsali jeden list proti Jeruzalému Artaxerxovi králi, takový:
\par 9 Rechum kanclér a Simsai písar i jiní tovaryši jejich, Dinaiští a Afarsatchaiští, Tarpelaiští, Afarzaiští, Arkevaiští, Babylonští, Susanechaiští, Dehavejští a Elmaiští,
\par 10 I jiní národové, kteréž byl prevedl Asnapar veliký a slavný, a rozsadil v mestech Samarských, a jiní za rekou, i Cheenetští,
\par 11 (Tento jest prípis listu, kterýž poslali k Artaxerxovi králi), služebníci tvoji, lidé za rekou a Cheenetští.
\par 12 Známo bud králi, že Židé, kteríž se vrátili od tebe, prišedše k nám do Jeruzaléma, mesto odporné a škodlivé stavejí, i zdi delají, a základy spojují.
\par 13 Protož nyní bud vedomo králi, bude-li to mesto vystaveno, a zdi dodelány, platut, cla a úroku dávati nebudou, a tak komore královské újma bude.
\par 14 Nyní tedy ponevadž dobrodiní paláce užíváme, na obnažování krále neslušelo se nám dívati. Tou prícinou poslali jsme a oznámili to králi,
\par 15 Aby dal hledati v knihách kronik otcu svých, a najdeš v nich, i zvíš, že mesto to jest mesto odporné a škodlivé králum i krajinám, a že se v nem puntovávají od starodávna, procež to mesto prvé zkaženo bylo.
\par 16 Nadto známot ciníme králi, že bude-li to mesto vystaveno, a zdi dodelány, tedy vládarství za rekou míti nebudeš.
\par 17 Tedy odeslal odpoved král Rechumovi kancléri a Simsaiovi písari i jiným tovaryšum jejich, kteríž bydlili v Samarí, a jiným za rekou v Selam i v Cheet:
\par 18 Psání, kteréž jste k nám poslali, zjevne cteno jest prede mnou.
\par 19 Protož rozkázal jsem, aby hledali. I nalezli, že to mesto zdávna povstává proti králum, a zprotivování i puntování bývají v nem.
\par 20 Nadto i králové mocní že bývali v Jeruzaléme, a panovali nade vším, co jest za rekou, jimž platové, cla a úrok dáván býval.
\par 21 Protož nyní prikažte, at jest zastaveno mužum tem, aby to mesto nebylo staveno, dokudž by ode mne poruceno nebylo.
\par 22 Hledtež pak, abyste se v té veci nemýlili, a at skrze to nezroste neco zlého na škodu králum.
\par 23 Když pak ten prípis listu Artaxerxa krále cten byl pred Rechumem a Simsaiem písarem a tovaryši jejich, odešli rychle do Jeruzaléma k Židum, a zastavili jim mocí a silou.
\par 24 A tak pretrženo jest dílo domu Božího, kterýž byl v Jeruzaléme, a stálo tak až do druhého léta kralování Daria krále Perského.

\chapter{5}

\par 1 Toho casu prorokoval Aggeus prorok a Zachariáš syn Iddo, proroci, Židum, kteríž byli v Judstvu a v Jeruzaléme, ve jménu Boha Izraelského mluvíce k nim.
\par 2 Tedy povstavše Zorobábel syn Salatieluv, a Jesua syn Jozadakuv, pocali zase staveti domu Božího, kterýž jest v Jeruzaléme, a byli s nimi proroci Boží, pomáhajíce jim.
\par 3 Téhož casu prišel k nim Tattenai, vývoda za rekou, a Setarbozenai, i tovaryši jejich, kteríž takto k nim rekli: Kdo vám porucil dum tento staveti a zdi tyto delati?
\par 4 Tedy jsme jim rekli takto, ano i ty muže, kterí to stavení delali, jmenovali.
\par 5 Nad staršími pak Židovskými byla ochrana Boha jejich, tak že neprekazili jim, dokudž ta vec neprišla pred Daria, jehož tehdáž odpoved prinesli o té veci.
\par 6 Prípis listu, kterýž poslal Tattenai vývoda za rekou, a Setarbozenai s tovaryši svými, i Afarsechaiští, kteríž byli za rekou, k Dariovi králi.
\par 7 List poslali jemu, a takto bylo psáno v nem: Dariovi králi pokoj všeliký.
\par 8 Známo bud králi, že jsme prišli do Judské krajiny k domu Boha velikého. Kterýžto stavejí kamením velikým, a dríví kladou do sten, a dílo to spešne se staví, a darí se v rukou jejich.
\par 9 Tedy otázali jsme se tech starších, a takto jsme jim rekli: Kdo vám porucil staveti dum tento, a zdi tyto delati?
\par 10 Ano i na jména jejich ptali jsme se jich, abychom oznámili tobe, a napsali jména mužu tech, kteríž jsou prední mezi nimi.
\par 11 Temito pak slovy nám odpovedeli, rkouce: My jsme služebníci Boha nebe a zeme, a stavíme dum, kterýž byl ustaven prvé pred mnohými lety, jejž byl veliký král Izraelský stavel i dokonal.
\par 12 Ale potom, když popudili otcové naši Boha nebeského, vydal je v ruku Nabuchodonozora krále v Babylone, Kaldejského, kterýž dum tento zboril, a lid prevedl do Babylona.
\par 13 A však léta prvního Cýra krále Babylonského, Cýrus král rozkázal, aby tento Boží dum byl staven.
\par 14 Nadto i nádoby domu Božího zlaté a stríbrné, kteréž byl Nabuchodonozor vynesl z chrámu, jenž byl v Jeruzaléme, a vnesl je do chrámu Babylonského, i ty vynesl Cýrus král z chrámu Babylonského, a dány jsou Sesbazarovi, jehož byl vývodou ustanovil.
\par 15 A porucil mu, rka: Nádoby tyto vezma, odejdi a slož je v chráme, kterýž jest v Jeruzaléme, a dum Boží at stavejí na míste jeho.
\par 16 Tedy Sesbazar ten prišed, založil grunty domu Božího, kterýž jest v Jeruzaléme, a od toho casu až po dnes staví se, a ješte není dokonán.
\par 17 Nyní tedy jestli se za dobré králi vidí, necht se pohledá mezi poklady královskými, kteríž jsou tam v Babylone, jest-li tak, že by Cýrus král porucil, aby staven byl dum Boží tento, kterýž jest v Jeruzaléme. Potom vuli královskou necht nám pošle o té veci.

\chapter{6}

\par 1 Tedy král Darius rozkázal, aby hledali v bibliotéce, mezi poklady složenými v Babylone.
\par 2 I nalezena jest v Achmeta na hrade, kterýž jest v Médské krajine, jedna kniha, a takto zapsána byla v ní pamet:
\par 3 Léta prvního Cýra krále Cýrus král rozkázal: Dum Boží, kterýž byl v Jeruzaléme, at jest v meste staven na míste, kdež se obetují obeti, a grunty jeho at vzhuru ženou, zvýší loktu šedesáti, a zšírí loktu šedesáti,
\par 4 Trmi rady z kamení velikého, a rad z dríví nového, náklad pak z domu královského bude dáván.
\par 5 Nadto nádobí domu Božího zlaté i stríbrné, kteréž Nabuchodonozor byl pobral z chrámu, jenž byl v Jeruzaléme, a byl je prenesl do Babylona, at zase navrátí, aby se dostalo do chrámu Jeruzalémského na místo své, a složeno bylo v dome Božím.
\par 6 Protož nyní, ty Tattenai, vývodo za rekou, Setarbozenai s tovaryši svými, i Afarsechaiští, kteríž jste za rekou, ustupte odtud.
\par 7 Nechte jich pri díle toho domu Božího. Vudce Židovský a starší jejich nechat ten dum Boží stavejí na míste jeho.
\par 8 Ode mne také poruceno jest o tom, co byste meli ciniti s staršími Židovskými pri stavení toho domu Božího, totiž, aby z zboží královského, z duchodu, jenž jsou za rekou, bez meškání náklad dáván byl mužum tem, aby dílo nemelo prekážky.
\par 9 A to, cehož by bylo potrebí, bud volku aneb skopcu i beránku k zápalným obetem Boha nebeského, obilé, soli, vína i oleje, jakž by rozkázali kneží Jeruzalémští, necht se jim dává na každý den, a to beze všeho podvodu,
\par 10 Aby meli z ceho obetovati vuni príjemnou Bohu nebeskému, a aby se modlili za život krále i za syny jeho.
\par 11 Nadto ode mne jest rozkázáno: Kdož by koli zrušil prikázaní toto, aby vyboreno bylo z domu jeho drevo, a zdviženo bylo, na nemž by obešen byl, a dum jeho at jest hnojištem pro takovou vec.
\par 12 Buh pak, jehož jméno tam prebývá, zkaziž všelikého krále i lid, kterýž by vztáhl ruku svou k zmenení toho, a zkáze toho domu Božího, jenž jest v Jeruzaléme. Já Darius prikazuji sám, bez meškání at se stane.
\par 13 Tedy Tattenai, vývoda za rekou, a Setarbozenai s tovaryši svými vedlé toho, jakž rozkázal Darius král, tak ucinili bez meškání.
\par 14 I staveli starší Židovští, a štastne se jim vedlo vedlé proroctví Aggea proroka, a Zachariáše syna Iddova. Staveli tedy a dokonali z rozkázaní Boha Izraelského, a z rozkázaní Cýra a Daria, a Artaxerxa krále Perského.
\par 15 A dokonán jest ten dum k tretímu dni mesíce Adar, a ten byl rok šestý kralování Daria krále.
\par 16 Tedy synové Izraelští, kneží a Levítové, i jiní, kteríž byli prišli z prevedení, posvetili toho domu Božího s radostí.
\par 17 A obetovali pri posvecení toho Božího domu sto telat, skopcu dve ste, beránku ctyri sta, a kozlu k obeti za hrích za všecken Izrael dvanáct, vedlé poctu pokolení Izraelského.
\par 18 I postavili kneží v trídách jejich, a Levíty v porádcích jejich pri službe Boží v Jeruzaléme, jakož psáno jest v knize Mojžíšove.
\par 19 Slavili také ti, kteríž se vrátili z prestehování, velikunoc ctrnáctého dne mesíce prvního.
\par 20 Nebo ocistili se byli kneží a Levítové jednomyslne. Všickni cistí byli, a obetovali beránka velikonocního za všecky, jenž zajati byli, i za bratrí své kneží, i sami za sebe.
\par 21 A jedli synové Izraelští, kteríž se byli navrátili z prestehování, i každý, kdož oddeliv se od poškvrnení pohanu zeme, pripojil se k nim, aby hledal Hospodina Boha Izraelského.
\par 22 Drželi také slavnost presnic za sedm dní s veselím, proto že je rozveselil Hospodin, a obrátil srdce krále Assyrského k nim, aby jich posilnil v díle domu Božího, Boha Izraelského.

\chapter{7}

\par 1 Po tech pak vecech, za kralování Artaxerxa krále Perského, Ezdráš syn Saraiáše, syna Azariášova, syna Helkiášova,
\par 2 Syna Sallumova, syna Sádochova, syna Achitobova,
\par 3 Syna Amariášova, syna Azariášova, syna Meraiotova,
\par 4 Syna Zerachiášova, syna Uzi, syna Bukki,
\par 5 Syna Abisuova, syna Fínesova, syna Eleazarova, syna Aronova, kneze nejvyššího,
\par 6 Tento Ezdráš vyšel z Babylona, a byl clovek zbehlý v zákone Mojžíšove, kterýž dal Hospodin Buh Izraelský, a dal mu král podlé toho, jakž ruka Hospodina Boha jeho byla s ním, všecko, zac ho koli žádal.
\par 7 Vyšli také synové Izraelští a kneží, i Levítové a zpeváci, vrátní a Netinejští do Jeruzaléma, léta sedmého Artaxerxa krále.
\par 8 A prišel do Jeruzaléma pátého mesíce. Tent jest rok sedmý krále Daria.
\par 9 Prvního zajisté dne mesíce prvního vyšel z Babylona, a prvního dne mesíce pátého prišel do Jeruzaléma s pomocí Boha svého.
\par 10 Nebo Ezdráš byl uložil v srdci svém, aby zpytoval zákon Hospodinuv i plnil jej, a aby ucil lid Izraelský ustanovením a soudum.
\par 11 Tento pak jest prípis listu, kterýž dal král Artaxerxes Ezdrášovi knezi umelému v zákone, zbehlému v tech vecech, kteréž prikázal Hospodin, a v ustanoveních jeho v Izraeli:
\par 12 Artaxerxes, král nad králi, Ezdrášovi knezi umelému v zákone Boha nebeského, muži zachovalému i Cheenetským.
\par 13 Ode mne jest prikázáno, kdož by koli v království mém z lidu Izraelského, a z kneží jeho i z Levítu, dobrovolne jíti chtel s tebou do Jeruzaléma, aby šel.
\par 14 Ponevadž jsi od krále a sedmi rad jeho poslán, abys dohlédal k Judstvu a k Jeruzalému podlé zákona Boha svého, kterýž máš v ruce své,
\par 15 A abys donesl stríbro a zlato, kteréž král a rady jeho dobrovolne obetovali Bohu Izraelskému, jehož príbytek jest v Jeruzaléme,
\par 16 A všecko stríbro a zlato, kteréhož bys dostal ve vší krajine Babylonské u tech, kteríž by z lidu dobrovolne co obetovati chteli, i s knežími, kteríž by dobrovolne obetovali k domu Boha svého, kterýž jest v Jeruzaléme,
\par 17 Abys rychle nakoupil za to stríbro telat, skopcu, beránku s suchými i mokrými obetmi jejich, a obetoval je na oltári domu Boha vašeho v Jeruzaléme.
\par 18 A což se koli tobe a bratrím tvým za dobré videti bude, s ostatkem stríbra a zlata uciniti, vedlé vule Boha vašeho ucinte.
\par 19 Nádoby pak, kteréžt jsou dány k službe domu Boha tvého, navrat pred Bohem v Jeruzaléme,
\par 20 I jiné veci prináležející k domu Boha tvého. A což by bylo potrebí dáti, dáš z komory královské.
\par 21 A já, já Artaxerxes král porucil jsem všechnem výbercím, kteríž jste za rekou, aby všecko, cehož by koli žádal od vás Ezdráš knez, ucitel zákona Boha nebeského, rychle se stalo,
\par 22 Až do sta centnéru stríbra, a až do sta mer pšenice, a až do sta sudu vína, a až do sta tun oleje, a soli bez míry.
\par 23 Což by koli bylo z rozkazu Boha nebeského, necht rychle spraví k domu Boha nebeského. Nebo proc má býti prchlivost jeho proti království královu i synum jeho?
\par 24 Také vám oznamujeme, aby na žádného z kneží a Levítu, zpeváku, vrátných, Netinejských a služebníku v dome Boha toho, platu, cla a úroku žádný úredník nevzkládal.
\par 25 Presto, ty Ezdráši, podlé moudrosti Boha svého, kterouž jsi obdaren, narídíš soudce a rádce, kteríž by soudili všecken lid, jenž jest za rekou, ze všech, kteríž povedomi jsou zákona Boha tvého. A kdo by neumel, budete uciti.
\par 26 Kdož by pak koli neplnil zákona Boha tvého a zákona králova, at se i hned soud vynese o nem, bud k smrti, budto k vypovedení jeho, neb aby na statku pokutován byl, aneb vezením trestán.
\par 27 Požehnaný Hospodin Buh otcu našich, kterýž dal to srdce královo, aby zvelebil dum Hospodinuv, kterýž jest v Jeruzaléme,
\par 28 A naklonil ke mne milosrdenstvím krále i rad jeho, i všech mocných knížat královských. Protož já posilnen jsa rukou Hospodina Boha svého nade mnou, shromáždil jsem z lidu Izraelského prednejší, kteríž by šli se mnou.

\chapter{8}

\par 1 Tito jsou pak prednejší po celedech svých otcovských, a rod tech, kteríž vyšli se mnou za kralování Artaxerxa krále z Babylona:
\par 2 Z synu Fínesových Gersom, z synu Itamarových Daniel, z synu Davidových Chattus.
\par 3 Z synu Sechaniášových, jenž byl z synu Farosových, Zachariáš, a s ním pocet mužu sto a padesáte.
\par 4 Z synu Pachat Moábových Eliehoenai syn Zerachiášuv, a s ním dve ste mužu.
\par 5 Z synu Sechaniášových syn Jachazieluv, a s ním tri sta mužu.
\par 6 Z synu Adinových Ebed syn Jonatanuv, a s ním padesát mužu.
\par 7 Z synu pak Elamových Izaiáš syn Ataliášuv, a s ním sedmdesát mužu.
\par 8 Z synu Sefatiášových Zebadiáš syn Michaeluv, a s ním osmdesáte mužu.
\par 9 Z synu Joábových Abdiáš syn Jechieluv, a s ním dve ste a osmnácte mužu.
\par 10 Z synu Selomitových syn Josifiášuv, a s ním sto a šedesáte mužu.
\par 11 Z synu Bebai Zachariáš syn Bebai, a s ním osmmecítma mužu.
\par 12 Z synu Azgadových Jochanan syn Hakatanuv, a s ním sto a deset mužu.
\par 13 Z synu Adonikamových poslednejších, jichž jména jsou tato: Elifelet, Jehiel, Semaiáš, a s ním šedesáte mužu.
\par 14 Z synu Bigvai Utai a Zabbud, a s nimi sedmdesáte mužu.
\par 15 Shromáždil jsem je pak u potoku, kterýž vpadá do Ahavy, a leželi jsme tu tri dni. Potom prehlédal jsem lid a kneží, a z synu Léví nenašel jsem tu žádného.
\par 16 Protož poslal jsem Eliezera, Ariele, Semaiáše, Elnatana, Jariba, Elnatana, Nátana, Zachariáše a Mesullama, prednejší, a Joiariba a Elnatana, muže ucené,
\par 17 A rozkázal jsem jim k Iddovi, knížeti v Chasifia míste, a naucil sem je, jak by meli mluviti k Iddovi, Achivovi a Netinejským v Chasifia míste, aby nám privedli služebníky domu Boha našeho.
\par 18 I privedli nám s pomocí Boží muže rozumného z synu Moholi, syna Léví, syna Izraelova, a Serebiáše s syny jeho, a bratrí jeho osmnácte,
\par 19 A Chasabiáše, a s ním Izaiáše z synu Merari, bratrí jeho a synu jejich dvadceti.
\par 20 Z Netinejských pak, kteréž byl zrídil David a knížata k službe Levítum, dve ste a dvadceti Netinejských. A ti všickni ze jména vycteni byli.
\par 21 Tedy vyhlásil jsem tu pust u reky Ahava, abychom se ponižovali pred Bohem svým, a hledali od neho cesty prímé sobe a dítkám svým, i všemu jmení našemu.
\par 22 Nebo stydel jsem se žádati od krále vojska a jízdných, aby nás bránili pred neprátely na ceste; nebo jsme byli pravili králi, rkouce: Ruka Boha našeho jest nade všemi, kteríž ho hledají upríme, ale moc a prchlivost jeho proti všechnem, kteríž ho opouštejí.
\par 23 A když jsme se postili a hledali v tom Boha svého, tedy vyslyšel nás.
\par 24 Oddelil jsem pak prednejších kneží dvanácte: Serebiáše, Chasabiáše, a s nimi z bratrí jejich deset.
\par 25 I odvážil jsem jim stríbro a zlato, a to nádobí, obet domu Boha našeho, kterouž obetovali král i rady jeho, i knížata jeho a všecken lid Izraelský, což se ho našlo.
\par 26 Odvážil jsem, pravím, do rukou jejich stríbra šest set centnéru a padesát, a nádobí stríbrného sto centnéru, a zlata sto centnéru,
\par 27 A koflíku zlatých dvadceti, každý v tisíc drachem, a dve nádoby z mosazi nejlepší, tak vzácné jako zlato.
\par 28 Potom jsem jim rekl: Vy jste posveceni Hospodinu, i tyto nádoby posveceny jsou, a to stríbro i zlato jest obet dobrovolná Hospodinu Bohu otcu vašich.
\par 29 Pilni toho budte a ostríhejte, až to i odvážíte pred knežími prednejšími a Levíty, i predními z celedí otcovských z Izraele v Jeruzaléme v pokojích domu Hospodinova.
\par 30 A tak prijali kneží a Levítové váhu stríbra, zlata a nádobí, aby donesli do Jeruzaléma, do domu Boha našeho.
\par 31 Zatím hnuli jsme se od reky Ahava, dvanáctého dne mesíce prvního, abychom se brali do Jeruzaléma, a ruka Boha našeho byla s námi, a vytrhla nás z ruky neprátel a úkladníku na ceste.
\par 32 I prišli jsme do Jeruzaléma, a pobyli jsme tu tri dni.
\par 33 Ctvrtého pak dne odváženo jest stríbro a zlato, a nádobí to v dome Boha našeho k ruce Meremota syna Uriáše kneze, s nímž byl Eleazar syn Fínesuv, a pomocníci jejich Jozabad syn Jesua, a Noadiáš syn Binnui, Levítové,
\par 34 Vše v poctu a váze, a zapsána jest všecka ta váha toho casu.
\par 35 Vrátivše se pak z zajetí ti, kteríž byli prestehováni, obetovali zápaly Bohu Izraelskému, volku dvanáct za všecken lid Izraelský, skopcu devadesát a šest, beránku sedmdesát a sedm, kozlu za hrích dvanáct, vše v obet zápalnou Hospodinu.
\par 36 I dali výpovedi královy vládarum královským i vývodám za rekou. Kteríž pomocni byli lidu i domu Božímu.

\chapter{9}

\par 1 A když se to vykonalo, pristoupili ke mne knížata, rkouce: Neoddelil se lid Izraelský, ani kneží a Levítové od národu zemí, ale ciní podlé ohavností Kananejských, Hetejských, Ferezejských, Jebuzejských, Ammonitských, Moábských, Egyptských a Amorejských.
\par 2 Nebo nabrali sobe a synum svým dcer jejich, a smísili se síme svaté s národy zemí, a knížata a vrchnost první byla v tom prestoupení.
\par 3 Kteroužto vec když jsem uslyšel, roztrhl jsem roucho své i plášt, a trhal jsem vlasy s hlavy své i z brady, a sedel jsem zdešený.
\par 4 I shromáždili se ke mne všickni, tresoucí se pred recmi Boha Izraelského pro prestoupení lidu prestehovaného, já pak sedel jsem zdešený, až do obeti vecerní.
\par 5 Ale v cas obeti vecerní vstal jsem od trápení svého, maje na sobe roucho roztržené i plášt svuj, a klekl jsem na kolena svá, rozprostíraje ruce své k Hospodinu Bohu svému.
\par 6 A rekl jsem: Bože muj, stydím se a hanbím pozdvihnouti, Bože muj, tvári své k tobe; nebo nepravosti naše rozmnožily se nad hlavou, a provinení naše vzrostlo až k nebi.
\par 7 Ode dnu otcu našich u veliké jsme vine až do tohoto dne, a pro nepravosti naše vydáni jsme my, králové naši i kneží naši v ruku králu zemí pod mec, v zajetí a v loupež, a v zahanbení tvári, tak jakž se to nyní deje.
\par 8 Ted pak rychle stala se nám milost od Hospodina Boha našeho, že zanechal nám ostatku, a dal nám obydlí na míste svatém, aby osvítil oci naše Buh náš, a dal nám malické povydchnutí od služby naší.
\par 9 Nebo ac jsme byli služebníci, však nenechal nás Buh náš v porobe naší, ale naklonil k nám milostí krále Perské, dav nám život, abychom vyzdvihli dum Boha našeho, a zase obnovili pustiny jeho, nýbrž ohradil nás v Judstvu a v Jeruzaléme.
\par 10 Nyní tedy což díme, ó Bože náš, po tech vecech, ponevadž jsme opustili prikázaní tvá,
\par 11 Kteráž jsi vydal skrze služebníky své proroky, rka: Zeme ta, do kteréž jdete, abyste jí dedicne vládli, jest zeme necistá, pro necistotu národu tech zemí, pro ohavnosti jejich, kterýmiž ji naplnili všudy naskrze v necistote své.
\par 12 Protož nyní dcer vašich nedávejte synum jejich, a dcer jejich neberte synum vašim, a nehledejte pokoje jejich a dobrého jejich, až na veky, abyste se zmocnili, a jedli dobré veci zeme, a k dedicnému vládarství zanechali ji synum svým až na veky.
\par 13 Po všech pak tech vecech, kteréž na nás prišly pro zlé skutky naše, a pro veliké naše provinení, ponevadž ty, Bože náš, netrestal jsi nás podlé nepravostí našich, a dal jsi nám vysvobození takové,
\par 14 Opet-liž bychom rušiti meli tvá prikázaní, a prízniti se s národy temito ohavnými? Zdaliž bys se zurive nehneval na nás, až bys nás do konce vyhladil, tak že by žádný nezustal a neušel?
\par 15 Hospodine Bože Izraelský, ty jsi spravedlivý; nebo jsme pozustali ostatkové, jakž se to vidí dnešního dne. Aj, my jsme pred tebou s provinením svým, ac bychom nemeli postavovati se pred tvárí tvou pro veci takové.

\chapter{10}

\par 1 Když se pak Ezdráš, padna pred domem Božím, tak modlil a vyznával s plácem, sešel se k nemu z lidu Izraelského zástup velmi veliký mužu i žen i detí. A když plakal lid plácem velikým,
\par 2 Tedy mluvil Sechaniáš syn Jechieluv z synu Elamových, a rekl Ezdrášovi: Myt jsme zhrešili proti Bohu svému, že jsme pojímali ženy cizozemky z národu zemí, a však vždy Izrael muže míti nadeji pri té veci.
\par 3 Nyní tedy vejdeme v smlouvu s Bohem svým, zapudíce všecky ženy i syny jejich podlé rady Páne a tech, jenž se tresou pred prikázaním Boha našeho, a tak podlé zákona at se stane.
\par 4 Vstan, nebo na tobe jest ta vec, a my budeme pri tobe. Posiln se a ucin tak.
\par 5 I vstal Ezdráš a zavázal prísahou prednejší kneží a Levíty a všecken lid Izraelský, aby tak ucinili. I prisáhli.
\par 6 A vstav Ezdráš od domu Božího, odšel do pokojíka Jochananova, syna Eliasibova, i všel tam, a nejedl chleba, ani vody nepil; nebo zámutek mel pro prestoupení tech, jenž se prestehovali.
\par 7 Zatím dali provolati v Judstvu a v Jeruzaléme všechnem prestehovaným, aby se shromáždili do Jeruzaléma,
\par 8 Kdo by pak koli neprišel ve trech dnech, podlé rady knížat a starších, aby všecken statek svuj propadl, a sám odloucen byl od shromáždení prestehovaných.
\par 9 A protož shromáždili se všickni muži Judští i Beniamin do Jeruzaléma ke dni tretímu, dvadcátého dne mesíce, (a ten mesíc byl devátý). I sedel všecken lid na ulici domu Božího, tresouce se pro tu vec i pro déšt.
\par 10 Tedy vyvstal Ezdráš knez a rekl k nim: Vy jste zhrešili, že jste pojímali ženy cizozemky, abyste pridali k provinení lidu Izraelského.
\par 11 Protož již vyznejte se Hospodinu Bohu otcu svých, a cinte vuli jeho, a oddelte se od národu cizích, i od žen cizozemek.
\par 12 I odpovedelo všecko to shromáždení, a rekli hlasem velikým: Podlé slova tvého povinni jsme tak uciniti.
\par 13 Ale lidu mnoho jest, a prška, a nemužeme vne státi. K tomu není ta práce jednoho dne ani dvou, nebo mnoho jest nás, kteríž jsme v tom prestoupili.
\par 14 Necht jsou postavena, prosíme, knížata naše ze všeho shromáždení, a kdož koli jest v mestech našich, kterýž pojal ženy cizozemky, at prijde v cas uložený, a s nimi starší z jednoho každého mesta i soudcové jejich, až bychom tak odvrátili hnev prchlivosti Boha našeho od sebe pro tu vec.
\par 15 Takž Jonatan syn Azaheluv, a Jachziáš syn Tekue, postaveni byli nad tím, Mesullam pak a Sabbetai, Levítové, pomáhali jim.
\par 16 Tedy ucinili tak pri tech, jenž prestehováni byli. I oddeleni jsou Ezdráš knez a muži prední celedí otcovských po domích otcu svých, všickni ti ze jména, a zasedli prvního dne mesíce desátého, aby to vyhledali.
\par 17 Což i konali pri všech mužích, kteríž byli pojali ženy cizozemky, až do prvního dne prvního mesíce.
\par 18 Našli se pak z synu knežských, jenž zpojímali ženy cizozemky tito: Z synu Jesua syna Jozadakova, a z bratrí jeho: Maaseiáš, Eliezer, Jarib a Gedaliáš.
\par 19 Ale povolili, aby zapudili ženy své. A ti, kteríž provinili, obetovali skopce z stáda za vinu svou.
\par 20 A z synu Immer: Chanani a Zebadiáš.
\par 21 Z synu Charim: Maaseiáš, Eliah, Semaiáš, Jechiel a Uziáš.
\par 22 Z synu Paschur: Elioenai, Maaseiáš, Izmael, Natanael, Jozabad a Elasa.
\par 23 A z Levítu: Jozabad, Simei, Kelaiáš, (jenž jest Kelita), Petachiáš, Juda a Eliezer.
\par 24 Z zpeváku pak Eliasib, a z vrátných Sallum, Telem a Uri.
\par 25 A z lidu Izraelského, z synu Faresových: Ramiáš, Jeziáš, Malkiáš, Miamin, Eleazar, Malkiáš a Benaiáš.
\par 26 Z synu Elamových: Mataniáš, Zachariáš, Jechiel, Abdi, Jeremot a Eliah.
\par 27 Z synu Zattu: Elioenai, Eliasib, Mataniáš, Jeremot, Zabad, a Aziza.
\par 28 Též z synu Bebai: Jochanan, Chananiáš, Zabbai, Atlai.
\par 29 Z synu Báni: Mesullam, Malluch, Adaiáš, Jasub, Seal, Jeramot.
\par 30 Z synu Pachat Moábových: Adna a Chélal, Benaiáš, Maaseiáš, Mataniáš, Bezaleel, Binnui a Manasse.
\par 31 Z synu Charimových: Eliezer, Isiáš, Malkiáš, Semaiáš, Simeon,
\par 32 Beniamin, Malluch, Semariáš.
\par 33 Z synu Chasumových: Mattenai, Mattata, Zabad, Elifelet, Jeremai, Manasses, Simei.
\par 34 Z synu Báni: Maadai, Amram, Uel,
\par 35 Benaiáš, Bediáš, Keluhu,
\par 36 Vaniáš, Meremot, Eliasib,
\par 37 Mattaniáš, Mattenai, Jaasav,
\par 38 Báni, Binnui, Simei,
\par 39 Selemiáš, Nátan a Adaiáš,
\par 40 Machnadbai, Sasai, Sarai,
\par 41 Azarel, Selemiáš, Semariáš,
\par 42 Sallum, Amariáš a Jozef.
\par 43 Z synu Nébových: Jehiel, Mattitiáš, Zabad, Zebina, Jaddav, Joel a Benaiáš.
\par 44 Ti všickni pojali byli ženy cizozemky, a byly z tech žen nekteré, že i deti zplodily.

\end{document}