\begin{document}

\title{Nehemjáš}

\chapter{1}

\par 1 Slova Nehemiáše syna Chachaliášova. I stalo se mesíce Kislev léta dvadcátého, když jsem byl na hrade Susan,
\par 2 Že prišel Chanani, jeden z bratrí mých, s nekterými muži z Judstva. Kterýchžto vzeptal jsem se na Židy, na ostatky pozustalé z zajetí a na Jeruzalém.
\par 3 I rekli mi: Ostatkové ti, kteríž pozustali z zajetí tam v té krajine, jsou u velikém nátisku a v pohanení, nadto i zed Jeruzalémská rozborena jest, a brány jeho ohnem zkaženy.
\par 4 Kterážto slova když jsem uslyšel, sedna plakal jsem a kvílil za nekolik dní, a postil jsem se, i modlil pred Bohem nebeským.
\par 5 A rekl jsem: Prosím, Hospodine Bože nebeský, silný, veliký a hrozný, ostríhající smlouvy a milosrdenství tem, jenž te milují, a ostríhají prikázaní tvých,
\par 6 Budiž, prosím, ucho tvé naklonené, a oci tvé otevrené, abys slyšel modlitbu služebníka svého, kterouž já se modlím pred tebou nyní dnem i nocí za syny Izraelské, služebníky tvé, a vyznávám hríchy synu Izraelských, jimiž jsme hrešili proti tobe. Já také i dum otce mého hrešili jsme.
\par 7 Zproneverili jsme se tobe, a neostríhali jsme prikázaní a ustanovení a soudu, kteréž jsi vydal skrze Mojžíše služebníka svého.
\par 8 Budiž pametliv, prosím, na slovo, kteréž jsi prikázal Mojžíšovi služebníku svému, rka: Vy prestoupíte, já pak rozptýlím vás mezi národy.
\par 9 Ale když se obrátíte ke mne, a budete ostríhati prikázaní mých, a plniti je, by pak nekterí z vás až na kraj sveta zahnáni byli, i odtud shromáždím je, a privedu je zase na místo, kteréž jsem vyvolil, aby tam prebývalo jméno mé.
\par 10 Však tito jsou služebníci tvoji a lid tvuj, kteréž jsi vykoupil mocí svou prevelikou, a rukou svou presilnou.
\par 11 Prosím, ó Hospodine, necht jest naklonené ucho tvé k modlitbe služebníka tvého, a k modlitbe služebníku tvých, kteríž žádají báti se jména tvého, a dej štastný prospech, prosím, služebníku svému dnes, naklone k nemu lítostí cloveka toho. Já pak byl jsem šenkérem královským.

\chapter{2}

\par 1 Tedy stalo se mesíce Nísan léta dvadcátého Artaxerxa krále, když víno stálo pred ním, že vzav víno, podal jsem ho králi. Nebýval jsem pak smuten pred ním.
\par 2 Procež mi rekl král: Proc oblícej tvuj smutný jest, ponevadž nestuneš? Jiného není, než sevrení srdce. Procež ulekl jsem se velmi velice.
\par 3 A rekl jsem králi: Král na veky bud živ. Kterak nemá býti smutný oblícej muj, když mesto to, kdež jsou hrobové otcu mých, zpušteno jest, a brány jeho ohnem zkaženy?
\par 4 Opet mi rekl král: Ceho žádáš? Mezi tím modlil jsem se Bohu nebeskému.
\par 5 Potom rekl jsem králi: Zdá-lit se za dobré králi, a jestliže má lásku služebník tvuj u tebe, žádám, abys mne poslal do Judstva, do mesta, kdež jsou hrobové otcu mých, abych je zase vystavel.
\par 6 Ješte mi rekl král (královna pak sedela podlé neho): Dlouho-li budeš na té ceste, a kdy se zas navrátíš? I líbilo se to králi, a propustil mne, hned jakž jsem mu oznámil jistý cas.
\par 7 Zatím rekl jsem králi: Vidí-li se za dobré králi, necht mi dadí listy k vývodám za rekou, aby mne provedli, až bych prišel do Judstva,
\par 8 Tolikéž psání k Azafovi, vládari lesu královského, aby mi dal dríví na trámy, k branám paláce, pri dome Božím, a ke zdem mestským, a k domu, do nehož bych vjíti mohl. I dal mi král vedlé štedré ruky Boha mého ke mne.
\par 9 Tedy prišed k vývodám za rekou, dal jsem jim psání královské. Poslal pak byl se mnou král hejtmany vojska a jezdce.
\par 10 To když uslyšel Sanballat Choronský, a Tobiáš služebník Ammonitský, velmi je to mrzelo, že prišel clovek, kterýž by obmýšlel dobré synu Izraelských.
\par 11 A tak prišed do Jeruzaléma, pobyl jsem tam za tri dni.
\par 12 V noci pak vstal jsem, a kolikosi mužu se mnou, a neoznámil jsem žádnému, co mi dal Buh muj v srdce, abych cinil v Jeruzaléme, aniž jsem mel hovada jakého s sebou, krome hovádka, na nemž jsem jel.
\par 13 I vyjel jsem branou pri údolí v noci k studnici drakové a k bráne hnojné, a ohledoval jsem zdí Jeruzalémských, kteréž byly poborené, a bran jeho zkažených ohnem.
\par 14 Odtud jsem jel k bráne studnicné, a k rybníku královskému, kdež nebylo brodu hovádku, na nemž jsem jel, aby prejíti mohlo.
\par 15 Procež bral jsem se zhuru podlé potoka v noci, a ohledoval jsem zdi. Odkudž vraceje se, vjel jsem branou pri údolí, a tak jsem se navrátil.
\par 16 Ale knížata nic nevedeli, kam jsem jezdil, a co jsem cinil; nebo jsem Židum, ani knežím, ani prednejším, ani knížatum, ani jiným úredníkum až do té chvíle neoznámil.
\par 17 Protož rekl jsem jim: Vy vidíte, v jakém jsme zlém, a Jeruzalém zpuštený, a brány jeho ohnem zkaženy. Podtež, a stavejme zed Jeruzalémskou, abychom nebyli více v pohanení.
\par 18 A když jsem jim oznámil, že pomoc Boha mého jest se mnou, ano i slova králova, kteráž ke mne mluvil, teprv rekli: Pricinmež se a stavejme. I posilnili rukou svých k dobrému.
\par 19 Což když uslyšel Sanballat Choronský, a Tobiáš služebník Ammonitský, a Gesem Arabský, posmívali se, a utrhali nám, pravíce: Což to deláte? Co se králi protivíte?
\par 20 Jimž odpovídaje, rekl jsem: Buh nebeský, tent nám dá prospech, a my služebníci jeho priciníce se, staveti budeme, vy pak nemáte žádného dílu, ani práva, ani památky v Jeruzaléme.

\chapter{3}

\par 1 Tedy povstal Eliasib, knez nejvyšší, a príbuzní jeho kneží, a staveli bránu bravnou, (tit jsou ji vystaveli, a zavesili vrata její, až k veži Mea vystaveli ji), až k veži Chananeel.
\par 2 A podlé neho staveli muži Jerecha, též podlé neho stavel Zakur syn Imruv.
\par 3 Bránu pak rybnou staveli synové Senaa. Ti položili trámy její, a vstavili vrata její s zámky i závorami jejími.
\par 4 Podlé tech také opravoval Meremot syn Uriáše, syna Kózova, a podlé nich opravoval Mesullam syn Berechiáše, syna Mesezabelova, a podlé tech opravoval Sádoch syn Baanuv.
\par 5 A podlé nich opravovali Tekoitští. Ale ti, kteríž byli znamenitejší z nich, nepodklonili šíje své k dílu pána svého.
\par 6 Bránu pak starou opravovali Joiada syn Paseachuv, a Mesullam syn Besodiášuv. Ti položili trámy její a vstavili vrata její s zámky a závorami jejími.
\par 7 Podlé nich opravoval Melatiáš Gabaonitský, a Jádon Meronotský, muži z Gabaon a z Masfa, až k stolici knížecí z této strany reky.
\par 8 Podlé nich pak opravoval Uziel syn Charhaiášuv s zlatníky, a podlé neho opravoval Chananiáš, syn apatekáruv. A nechali Jeruzaléma až do zdi široké.
\par 9 Dále podlé nich opravoval Refaiáš syn Churuv, hejtman nad polovicí kraje Jeruzalémského.
\par 10 A podlé nich opravoval Jedaiáš Charumafuv proti domu svému. Podlé nehož opravoval Chattus syn Chasabneiášuv.
\par 11 Druhý pak díl opravoval Malkiáš syn Charimuv, a Chasub syn Pachat Moábuv, a veži Tannurim.
\par 12 Podlé nehož opravoval Sallum syn Lochesuv, hejtman nad polovicí kraje Jeruzalémského, se dcerami svými.
\par 13 Bránu pri údolí opravil Chanun s obyvateli Zanoe. Oni ji staveli, a vstavili vrata její s zámky i závorami jejími, a zdi na tisíc loket až do brány hnojné.
\par 14 Bránu pak hnojnou opravil Malkiáš syn Rechabuv, hejtman kraje Betkarem. On ji ustavel, a vstavil vrata s zámky i závorami jejími.
\par 15 Bránu pak studnice opravoval Sallun syn Kolchozuv, hejtman kraje Masfa. On ji vystavel a prikryl ji, a vstavil vrata její s zámky i závorami jejími, a zed rybníka Selach, od zahrady královské až k stupnum sstoupajícím z mesta Davidova.
\par 16 Za ním opravoval Nehemiáš syn Azbukuv, hejtman nad polovici kraje Betsur, až naproti hrobum Davidovým, a až k rybníku udelanému, až k domu silných.
\par 17 Za ním opravovali Levítové, Rechum syn Báni, podlé nehož opravoval Chasabiáš, hejtman nad polovicí kraje Ceily s krajem svým.
\par 18 Za ním opravovali bratrí jejich, Bavai syn Chenadaduv, hejtman nad polovicí kraje Ceily.
\par 19 Podlé neho pak opravoval Ezer syn Jesua, hejtman Masfa, díl druhý naproti, kudyž se chodí k skladu zbroje Mikzoa.
\par 20 Za ním rozhorliv se, opravoval Báruch syn Zabbai, díl druhý od Mikzoa až ke dverum domu Eliasiba, kneze nejvyššího.
\par 21 Za ním opravoval Meremot syn Uriáše, syna Kózi, díl druhý ode dverí domu Eliasibova až do konce domu jeho.
\par 22 Za ním opravovali kneží, kteríž bydlili v rovine.
\par 23 Za ním opravoval Beniamin a Chasub proti domum svým. Za ním opravoval Azariáš syn Maaseiáše, syna Ananiášova vedlé domu svého.
\par 24 Za ním opravoval Binnui syn Chenadaduv díl druhý, od domu Azariášova až do Mikzoa a až k úhlu.
\par 25 Pálal syn Uzai proti Mikzoa a veži vysoké, kteráž vyhlédala z domu králova, jenž byla v placu u žaláre. Za ním Pedaiáš syn Farosuv.
\par 26 Netinejští pak, jenž bydlili v Ofel, až naproti bráne vodné k východu, a veži vysoké.
\par 27 Za ním opravovali Tekoitští díl druhý, naproti veži veliké a vysoké, až ke zdi pri Ofel.
\par 28 Od brány konské opravovali kneží, jeden každý naproti domu svému.
\par 29 Za tím opravoval Sádoch syn Immeruv naproti domu svému. Za ním pak opravoval Semaiáš syn Sechaniášuv, strážný brány východní.
\par 30 Za ním opravoval Chananiáš syn Selemiášuv, a Chanun syn Zalafuv šestý, díl druhý. Za ním opravoval Mesullam syn Berechiášuv proti pokoji svému.
\par 31 Za ním opravoval Malkiáš syn zlatníkuv až k domu Netinejských a kupcu, naproti bráne Mifkad, až do paláce úhlového.
\par 32 A mezi palácem úhlovým až do brány bravné opravovali zlatníci a kupci.

\chapter{4}

\par 1 I stalo se, když uslyšel Sanballat, že stavíme zed, rozpálil se hnevem, a rozzlobiv se velmi, posmíval se Židum.
\par 2 Nebo mluvil pred bratrími svými a pred vojskem Samarským, rka: Co ti bídní Židé berou pred sebe? Tak-liž mají býti zanecháni? Což budou obetovati? Za jeden-liž den to dodelati mají? Také-liž i kamení z hromad rumu krísiti budou, kteréž spáleno jest?
\par 3 Tobiáš pak Ammonitský podlé neho rekl: Nechat stavejí, však liška pribehna, prorazí zed jejich kamennou.
\par 4 Slyš, ó Bože náš, že jsme v pohrdání, a obrat pohanení jejich na hlavu jejich, a dej je v loupež v zemi, do níž by zajati byli.
\par 5 A neprikrývej nepravosti jejich, a hrích jejich od tvári tvé at není shlazen, proto že jsou te popouzeli v stavitelích.
\par 6 A tak staveli jsme tu zed, a spojili všecku až do polu, a mel lid srdce k tomu dílu.
\par 7 Což když uslyšel Sanballat, a Tobiáš a Arabští, Ammonitští i Azotští, že by na dýl pribývalo zdi Jeruzalémské, a že se již byly pocaly mezery zavírati, rozpálili se hnevem velice.
\par 8 Protož spuntovali se všickni spolecne, aby táhli k boji proti Jeruzalému a aby jemu prekážku ucinili.
\par 9 (My pak modlili jsem se Bohu svému, a postavili jsme stráž proti nim ve dne i v noci, bojíce se jich.
\par 10 Nebo rekli Judští: Zemdlelat jest síla nosicu, a rumu ješte mnoho jest, neodolalit bychom jim, stavejíce zed.)
\par 11 Anobrž i rekli neprátelé naši: Nezvedít, ani spatrí, až vpadneme mezi ne, a pomordujeme je, aneb zastavíme to dílo.
\par 12 Když pak prišli Židé, kteríž s nimi bydlili, a pravili nám na desetkrát: Mejte pozor na všecka místa, kudyž se chodí k nám:
\par 13 Tedy postavil jsem na dolních místech za zdí, i na místech príkrých, a osadil jsem lidem po celedech s meci, s kopími a lucišti jejich.
\par 14 A když jsem to spatril, vstana, rekl jsem k prednejším a knížatum i k jinému lidu: Nebojte se jich, ale na Pána velikého a hrozného pamatujte, a bojujte za bratrí své, za syny své a dcery své, za manželky své a domy své.
\par 15 I stalo se, když uslyšeli neprátelé naši, že jest nám to oznámeno, tožt Buh rozptýlil radu jejich, a my navrátili jsme se všickni ke zdem, každý k dílu svému.
\par 16 Ale však od toho dne polovice služebníku mých delali, a polovice jich držela kopí, pavézy a lucište, a pancíre, a knížata stála za vší celedí Judskou.
\par 17 Ti také, kteríž delali na zdi, i nosici bremen, i nakladaci, každý jednou rukou delal, a v druhé držel bran.
\par 18 Z tech pak, kteríž staveli, mel jeden každý mec svuj pripásaný na bedrách svých, a tak staveli, a trubac stál podlé mne.
\par 19 Nebo jsem rekl k prednejším a knížatum i k jinému lidu: Dílo veliké a široké jest, a my poruznu jsme na zdi, podál jeden od druhého.
\par 20 Na kterém byste koli míste uslyšeli hlas trouby, tu se shromaždte k nám. Buh náš bude bojovati za nás.
\par 21 A tak my delali jsme dílo, a polovice jich držela kopí od svitání až do soumraku.
\par 22 Tehdáž také rekl jsem lidu: Každý s služebníkem svým nocuj u prostred Jeruzaléma, at je máme k stráži v noci, a ve dne k dílu.
\par 23 Procež i já, i bratrí moji, i služebníci moji, i strážní, kteríž chodí za mnou, nebudeme svláceti odevu svého. Žádný ho nesloží, lec u vody.

\chapter{5}

\par 1 Byl pak pokrik veliký lidu i žen jejich na bratrí jich Židy.
\par 2 Nebo nekterí pravili: Synu a dcer máme tak mnoho, že za ne obilé jednáme, abychom jísti a živi býti mohli.
\par 3 Jiní opet pravili: Pole svá i vinice své, a domy své zzastavovati musíme, abychom obilé jednati mohli v hladu tomto.
\par 4 Jiní ješte pravili: Musíme vypujciti penez, abychom dali plat králi, na svá pole i vinice své,
\par 5 Ješto aj, jakož telo bratrí našich, tak tela naše, jakož synové jejich, tak i synové naši. A však my musíme podrobovati syny své a dcery své v službu, a nekteré již ze dcer našich podrobeny jsou, a nemužeme s nic býti, ponevadž pole naše a vinice naše drží jiní.
\par 6 Protož rozhneval jsem se velmi, když jsem slyšel krik jejich a slova taková.
\par 7 I uložil jsem v srdci svém, abych domlouval prednejším a knížatum, rka jim: Vy jste ti, jenž obtežujete jeden každý bratra svého. I svolal jsem proti nim shromáždení veliké.
\par 8 A rekl jsem jim: My vyplacujeme bratrí své Židy, kteríž prodáni byli pohanum, podlé možnosti naší. Což vy zase prodávati máte bratrí vaše, anobrž což je sobe prodávati budete? Kterížto umlkli a nenalezli odpovedi.
\par 9 Rekl jsem dále: Není to dobre, což deláte. Zdali v bázni Boha našeho nemáte choditi radeji než v pohanení pohanu, neprátel našich?
\par 10 I já také s bratrími svými a s služebníky svými mohl bych bráti od nich peníze aneb obilé, a však odpustme jim medle ten dluh.
\par 11 Navratte jim, prosím, ješte dnes pole jejich, vinice jejich, zahrady olivové jejich i domy jejich, i ten stý díl penez, obilé, vína i oleje, kterýž od nich bérete.
\par 12 Odpovedeli: Navrátíme, aniž ceho od nich vyhledávati budeme; tak uciníme, jakž ty pravíš. Tedy svolav kneží, zavázal jsem je prísahou, aby tak ucinili.
\par 13 Vytrásl jsem také podolek svuj, a rekl jsem: Tak vytres Buh každého muže, kdož by nenaplnil slova tohoto, z domu jeho, z úsilí jeho, a tak bud vytresený a prázdný. I reklo všecko shromáždení: Amen, a chválili Hospodina. I ucinil lid tak.
\par 14 Anobrž také ode dne, v nemž jsem postaven, abych byl vývodou jejich v zemi Judské, od léta dvadcátého až do léta tridcátého druhého Artaxerxa krále, za dvanácte let, ani já ani bratrí moji pokrmu knížecího jsme nejedli,
\par 15 Ješto vývodové prvnejší, kteríž byli prede mnou, obtežovali lid, berouce od nich chléb a víno mimo ctyridceti lotu stríbra. Nadto i služebníci jejich ssužovali lid, cehož jsem já necinil, boje se Boha.
\par 16 Alebrž také i pri opravování zdi pracoval jsem, aniž jsme skupovali rolí, ano i všickni služebníci moji byli tu shromáždeni k dílu.
\par 17 Presto Židé a knížat pul druhého sta osob, a kteríž pricházeli k nám z národu okolních, jídali u stolu mého.
\par 18 Procež strojívalo se toho na každý den jeden vul, šest ovec výborných, též i ptáci byli mi strojeni, a v jednom z desíti dnu všelijakého vína dávalo se dosti. Však s tím se vším pokrmu knížecího nežádal jsem, nebo težká poroba vzložena byla na lid ten.
\par 19 Budiž pametliv na mne, Bože muj, k dobrému, což jsem pak koli cinil pri lidu tomto.

\chapter{6}

\par 1 I stalo se, když uslyšel Sanballat a Tobiáš, a Gesem Arabský i jiní neprátelé naši, že bych vystavel zed, a že nezustalo v ní mezery, ackoli jsem ješte až do toho casu nevstavil vrat do bran,
\par 2 Že poslal Sanballat a Gesem ke mne, rka: Prid, a rozmluvíme spolu ve vsi na rovinách Ono. Ale oni mé zlé obmýšleli.
\par 3 Takž jsem poslal k nim posly, rka: Dílo veliké delám, protož nemohu odjíti. Což se má meškati dílo, když bych je opuste, k vám sjíti mel?
\par 4 I posílali ke mne na týž zpusob ctyrikrát, a odpovedel jsem jim týmiž slovy.
\par 5 Potom poslal ke mne Sanballat na tentýž zpusob po páté služebníka svého s listem otevreným.
\par 6 V kterémž bylo psáno: Slyší se mezi národy, jakž Gasmu praví, že ty a Židé myslíte se zprotiviti, a že ty proto stavíš zed, abys králem jejich byl, jakž se to dokoná.
\par 7 Také žes i proroky postavil, aby hlásali o tobe v Jeruzaléme, rkouce: Král v Judstvu. Nyní tedy uslyší král ty veci. Protož prid, a poradíme se spolu.
\par 8 Tedy poslal jsem k nemu, rka: Nenít toho nic, což ty pravíš, ale sám sobe to vymýšlíš.
\par 9 Nebo všickni ti nás ustrašiti se snažovali, myslíce: Oslábnou ruce jejich pri díle, a nedokoná se to. Ale však ty, ó Bože, posiln rukou mých.
\par 10 A když jsem všel do domu Semaiáše syna Delaiášova, syna Mehetabelova,kterýž se byl zavrel, rekl mi: Sejdeme se do domu Božího, do vnitrku chrámu, a zavreme dvére chrámové; nebo prijdou, chtíc te zamordovati. A to v noci prijdou, aby te zamordovali.
\par 11 Jemuž jsem rekl: Takový-liž by muž, jako jsem já, utíkati mel? Aneb kdo jest tak, jako jsem já, ješto by vejda do chrámu, živ byl? Nevejdut.
\par 12 I poznal jsem, a aj, Buh neposlal ho, ale proroctví to mluvil proti mne, že ho Tobiáš a Sanballat byli ze mzdy najali.
\par 13 Proto pak ze mzdy byl najat, abych já ustrašen jsa, ucinil to a zhrešil, aby mi to bylo u nich k zlé povesti, címž by mi utrhali.
\par 14 Budiž pametliv, muj Bože, na Tobiáše a Sanballata, podlé tech skutku jejich, i na Noadii prorokyni, a na jiné proroky, kteríž strašili mne.
\par 15 A tak dodelána jest zed ta petmecítmého dne mesíce Elul v padesáti a dvou dnech.
\par 16 To když uslyšeli všickni neprátelé naši, a videli všickni národové, kteríž byli vukol nás, ulekli se velmi; nebo poznali, že od Boha našeho pusobeno bylo dílo to.
\par 17 Také v tech dnech i listy casto posílali prednejší Judští k Tobiášovi, tolikéž od Tobiáše docházely k nim.
\par 18 Nebo mnozí v Judstvu meli s ním prísahu, proto že byl zetem Sechaniáše syna Arachova, a Jochanan syn jeho pojal byl dceru Mesullama syna Berechiášova.
\par 19 K tomu i dobré ciny jeho vypravovali prede mnou, a reci mé vynášeli k nemu. Listy pak posílal Tobiáš, aby mne ustrašil.

\chapter{7}

\par 1 A když byla dostavena zed, a zavesil jsem vrata, a ustanoveni byli vrátní i zpeváci i Levítové,
\par 2 Porucil jsem Chananovi bratru svému, a Chananiášovi hejtmanu hradu Jeruzalémského, (proto že on byl muž verný a bohabojný nad mnohé),
\par 3 A rekl jsem jim: Necht nebývají otvírány brány Jeruzalémské, až obejde slunce, a když ti, jenž tu stávají, zavrou brány, vy ohledejte. A tak postavil jsem stráž z obyvatelu Jeruzalémských, každého v stráži jeho, a každého naproti domu jeho.
\par 4 Mesto pak to bylo široké a veliké, ale lidu málo v ohrade jeho, a domové nebyli vystaveni.
\par 5 Protož dal mi to Buh muj v srdce mé, že jsem shromáždil prednejší, a knížata i lid, aby byli vycteni podlé porádku rodu. I nalezl jsem knihu o rodu tech, kteríž se byli prvé prestehovali, a našel jsem v ní napsáno:
\par 6 Tito jsou lidé té krajiny, kteríž šli z zajetí a prestehování toho, jakž je byl prestehoval Nabuchodonozor král Babylonský, a navrátili se do Jeruzaléma a do Judstva, jeden každý do mesta svého.
\par 7 Kteríž prišli s Zorobábelem, s Jesua, s Nehemiášem, Azariášem, Raamiášem, Nachamanem, Mardocheem, Bilsanem, Misperetem, Bigvajem, Nechumem, Baanou, pocet mužu z lidu Izraelského:
\par 8 Synu Farosových dva tisíce, sto sedmdesát dva;
\par 9 Synu Sefatiášových tri sta sedmdesát dva;
\par 10 Synu Arachových šest set padesát dva;
\par 11 Synu Pachat Moábových, synu Jesua a Joábových, dva tisíce, osm set a osmnáct;
\par 12 Synu Elamových tisíc, dve ste padesát ctyri;
\par 13 Synu Zattuových osm set ctyridceti pet;
\par 14 Synu Zakkai sedm set a šedesát;
\par 15 Synu Binnui šest set ctyridceti osm;
\par 16 Synu Bebai šest set dvadceti osm;
\par 17 Synu Azgadových dva tisíce, tri sta dvamecítma;
\par 18 Synu Adonikamových šest set šedesáte sedm;
\par 19 Synu Bigvai dva tisíce, šedesáte sedm;
\par 20 Synu Adinových šest set padesát pet;
\par 21 Synu Aterových z Ezechiáše devadesát osm;
\par 22 Synu Chasumových tri sta dvadceti osm;
\par 23 Synu Bezai tri sta dvadceti ctyri;
\par 24 Synu Charifových sto a dvanáct;
\par 25 Synu Gabaonitských devadesát pet;
\par 26 Mužu Betlémských a Netofatských sto osmdesát osm;
\par 27 Mužu Anatotských sto dvadceti osm;
\par 28 Mužu Betazmavetských ctyridceti dva;
\par 29 Mužu Kariatjeharimských, Kafirských a Berotských sedm set ctyridceti a tri;
\par 30 Mužu Ráma a Gabaa šest set dvadceti jeden;
\par 31 Mužu Michmas sto dvadceti dva;
\par 32 Mužu z Bethel a Hai sto dvadceti tri;
\par 33 Mužu z Nébo druhého padesáte dva;
\par 34 Synu Elama druhého tisíc, dve ste padesát ctyri;
\par 35 Synu Charimových tri sta dvadceti;
\par 36 Synu Jerecho tri sta ctyridceti pet;
\par 37 Synu Lodových, Chadidových a Onových sedm set dvadceti jeden;
\par 38 Synu Senaa tri tisíce, devet set a tridceti.
\par 39 Kneží: Synu Jedaiášových z domu Jesua devet set sedmdesát tri;
\par 40 Synu Immerových tisíc, padesát dva;
\par 41 Synu Paschurových tisíc, dve ste ctyridceti sedm;
\par 42 Synu Charimových tisíc a sedmnáct.
\par 43 Levítu: Synu Jesua a Kadmiele, synu Hodevášových sedmdesát ctyri.
\par 44 Zpeváku: Synu Azafových sto ctyridceti osm.
\par 45 Vrátných: Synu Sallumových, synu Aterových, synu Talmonových, synu Akkubových, synu Chatita, synu Sobai, sto tridceti osm.
\par 46 Netinejských: Synu Zicha, synu Chasufa, synu Tabbaot,
\par 47 Synu Keros, synu Sia, synu Fadon,
\par 48 Synu Lebana, synu Chagaba, synu Salmai,
\par 49 Synu Chanan, synu Giddel, synu Gachar,
\par 50 Synu Reaia, synu Rezin, synu Nekoda,
\par 51 Synu Gazam, synu Uza, synu Paseach,
\par 52 Synu Besai, synu Meunim, synu Nefisesim,
\par 53 Synu Bakbuk, synu Chakufa, synu Charchur,
\par 54 Synu Bazlit, synu Mechida, synu Charsa,
\par 55 Synu Barkos, synu Sisera, synu Tamach,
\par 56 Synu Neziach, synu Chatifa,
\par 57 Synu služebníku Šalomounových, synu Sotai, synu Soferet, synu Ferida,
\par 58 Synu Jaala, synu Darkon, synu Giddel,
\par 59 Synu Sefatiášových, synu Chattil, synu Pocheret Hazebaim, synu Amon,
\par 60 Všech Netinejských a synu služebníku Šalomounových tri sta devadesát dva.
\par 61 Tito také byli, kteríž vyšli z Telmelach a Telcharsa: Cherub, Addon a Immer. Ale nemohli prokázati rodu otcu svých a semene svého, že by z Izraele byli.
\par 62 Synu Delaiášových, synu Tobiášových, synu Nekodových šest set ctyridceti dva.
\par 63 A z kneží: Synové Chabaiášovi, synové Kózovi, synové Barzillai toho, kterýž pojav sobe z dcer Barzillai Galádského manželku, nazván jest jménem jejich.
\par 64 Ti vyhledávali jeden každý zapsání o sobe, chtíce prokázati rod svuj, ale nenašlo se. A protož zbaveni jsou knežství.
\par 65 A zapovedel jim Tirsata, aby nejedli z vecí svatosvatých, dokudž by nestál knez s urim a tumim.
\par 66 Všeho toho shromáždení pospolu ctyridceti a dva tisíce, tri sta a šedesát,
\par 67 Krome služebníku jejich a devek jejich, jichž bylo sedm tisíc, tri sta tridceti sedm. A mezi nimi bylo zpeváku a zpevakyní dve ste ctyridceti pet.
\par 68 Koní jejich sedm set tridceti šest, mezku jejich dve ste ctyridceti pet,
\par 69 Velbloudu ctyri sta tridceti pet, oslu šest tisíc, sedm set a dvadceti.
\par 70 Tehdy nekterí z knížat celedí otcovských dávali ku potrebám. Tirsata dal na poklad tisíc drachem zlata, bání padesát, sukní knežských pet set a tridceti.
\par 71 Knížata také celedí otcovských dali na poklad ku potrebám dvadceti tisíc drachem zlata, a stríbra liber dva tisíce a dve ste.
\par 72 Což pak dali jiní z lidu, bylo zlata dvadcet tisíc drachem, a stríbra dva tisíce liber, a sukní knežských šedesát sedm.
\par 73 A tak osadili se kneží a Levítové, a vrátní i zpeváci, lid a Netinejští, i všecken Izrael v mestech svých. I nastal mesíc sedmý, a synové Izraelští byli v mestech svých.

\chapter{8}

\par 1 I shromáždil se všecken lid jednomyslne do ulice, kteráž jest proti bráne vodné, a rekli Ezdrášovi uciteli, aby prinesl knihu zákona Mojžíšova, kterýž vydal Hospodin lidu Izraelskému.
\par 2 Tedy prinesl Ezdráš knez zákon pred to shromáždení mužu i žen i všech, kteríž by rozumne poslouchati mohli, prvního dne mesíce sedmého.
\par 3 I cetl v nem na té ulici, kteráž jest proti bráne vodné, od jitra až do poledne, pri prítomnosti mužu i žen, i kterížkoli rozumeti mohli. A uši všeho lidu obráceny byly k knize zákona.
\par 4 Stál pak Ezdráš ucitel na kazatelnici drevené, kterouž byli udelali k té veci, a stál podlé neho Mattitiáš, Sema, Anaiáš, Uriáš, Helkiáš a Maaseiáš, po pravé ruce jeho, po levé pak Pedaiáš, Misael, Malkiáš, Chasum, Chasbaddana, Zachariáš a Mesullam.
\par 5 Otevrel tedy Ezdráš knihu pred ocima všeho lidu, (nebo výše stál než všecken lid). Kterouž jakž otevrel, povstal všecken lid.
\par 6 I dobrorecil Ezdráš Hospodinu Bohu velikému, a všecken lid odpovídal: Amen, amen, pozdvihujíce rukou svých, a sklonujíce hlavy, poklonu ucinili Hospodinu tvárí k zemi.
\par 7 Tak i Jesua, Báni, Serebiáš, Jamin, Akkub, Sabbetai, Hodiáš, Maaseiáš, Kelita, Azariáš, Jozabad, Chanan, Pelaiáš, a Levítové vyucovali lid zákonu. Lid pak byl na míste svém.
\par 8 Nebo ctli v té knize v zákone Božím srozumitelne, a vykládajíce smysl, vysvetlovali to, což ctli.
\par 9 Potom rekl Nehemiáš, jinak Tirsata, a Ezdráš knez, ucitel, a Levítové, kteríž lid vyucovali, všemu lidu: Den tento posvecený jest Hospodinu Bohu vašemu, nekveltež, ani placte. (Nebo plakal všecken lid, když slyšeli slova zákona.)
\par 10 Ale jdete, rekl jim, a jezte tucné veci, a píte sladké, sdílejíce se s temi, kteríž nemají nic pripraveno. Den zajisté tento svatý jest Pánu našemu, protož nermuttež se, nýbrž radost Hospodinova budiž síla vaše.
\par 11 A když tak Levítové pospokojili všeho lidu, mluvíce: Mlcte, nebo den svatý jest, a nermutte se,
\par 12 Odšel všecken lid, aby jedli a pili, a aby se sdíleli. I veselili se velmi, proto že srozumeli slovum tem, kteráž jim v známost uvedli.
\par 13 Potom nazejtrí sešla se knížata celedí otcovských ze všeho lidu, kneží i Levítové k Ezdrášovi uciteli, aby vyrozumeli slovum zákona.
\par 14 Našli pak napsáno v zákone, že prikázal Hospodin skrze Mojžíše, aby bydlili synové Izraelští v stáncích na slavnost mesíce sedmého,
\par 15 A aby dali prohlásiti a provolati po všech mestech svých i v Jeruzaléme, rkouce: Vyjdete na hory, a prineste ratolestí olivových a ratolestí dríví borového, a ratolestí myrtových, i ratolestí palmových, a ratolestí dríví hustého, abyste nadelali stánku, tak jakž psáno jest.
\par 16 Protož vyšed lid, prinesli a nadelali sobe stánku, jeden každý na streše své, i v síních svých, i v síních domu Božího, i v ulici brány vodné, i v ulici brány Efraim.
\par 17 A tak nadelali stánku všecko shromáždení tech, jenž se vrátili z zajetí, a bydlili v nich, (ackoli necinili tak synové Izraelští ode dnu Jozue syna Nun, až do dne toho). I byla radost velmi veliká.
\par 18 Cetl pak v knize zákona Božího na každý den, od prvního dne až do posledního, a drželi slavnost za sedm dní. Osmého pak dne byl svátek podlé obyceje.

\chapter{9}

\par 1 Potom dvadcátého ctvrtého dne téhož mesíce shromáždili se synové Izraelští, a postíce se v žíních, posypali se prstí,
\par 2 (Oddelilo se pak bylo síme Izraelské ode všech cizozemcu), a stojíce, vyznávali hríchy své i nepravosti otcu svých.
\par 3 I stáli na místech svých, a ctli v knize zákona Hospodina Boha svého ctyrikrát za den, a ctyrikrát vyznávali a klaneli se Hospodinu Bohu svému.
\par 4 Za tím vystoupivše na výstupek Levítský Jesua a Báni, Kadmiel, Sebaniáš, Bunni, Serebiáš, Báni a Chenani, volali hlasem velikým k Hospodinu Bohu svému.
\par 5 A rekli Levítové ti, Jesua, Kadmiel, Báni, Chasabniáš, Serebiáš, Hodiáš, Sebaniáš a Petachiáš: Vstante, dobrorecte Hospodinu Bohu svému, od veku až na veky, a at dobrorecí slavnému jménu jeho, a vyššímu nad každé dobrorecení i chválu.
\par 6 Ty jsi, Hospodine, sám ten jediný, ty jsi ucinil nebesa, nebesa nebes i všecko vojsko jejich, zemi i všecko, což jest na ní, more i všecko, což jest v nich, obživuješ také všecko, ano i vojska nebeská pred tebou se sklánejí.
\par 7 Ty jsi, Hospodine, Buh ten, kterýž jsi vyvolil Abrama, a vyvedl jej z Ur Kaldejských, a dal jsi jemu jméno Abraham.
\par 8 A nalezl jsi srdce jeho verné pred sebou, a ucinil jsi s ním smlouvu, že dáš zemi Kananejského, Hetejského, Amorejského, Ferezejského, Jebuzejského a Gergezejského, že dáš ji semeni jeho, a naplnils slova svá, nebo spravedlivý jsi ty.
\par 9 Popatril jsi zajisté na trápení otcu našich v Egypte, a krik jejich vyslyšel jsi pri mori Rudém.
\par 10 A ukazovals znamení a zázraky na Faraonovi i na všech služebnících jeho, i na všem lidu zeme jeho; nebo vedel jsi, že jsou pýchu provodili nad nimi. Címž jsi dobyl sobe jména, jakž se to podnes vidí.
\par 11 Nadto i more jsi pred nimi rozdelil, tak že prešli prostredkem more po suše, a ty, jenž je stihali, uvrhl jsi do hlubin, jako kámen do vody veliké.
\par 12 A sloupem oblakovým vodils je ve dne, a sloupem ohnivým v noci, osvecuje jim cestu, kudy by jíti meli.
\par 13 Potom jsi sstoupil na horu Sinai, a mluvil jsi s nimi s nebe, a vydal jsi jim soudy prímé a zákony pravé, ustanovení a prikázaní dobrá.
\par 14 Též i sobotu svou svatou známu jsi jim ucinil, a prikázaní, ustanovení i zákon vydals jim skrze Mojžíše, služebníka svého.
\par 15 Také i chléb s nebe dal jsi jim v hladu jejich, a vodu z skály vyvedl jsi jim v žízni jejich, a rozkázal jsi jim, aby šli a dedicne vládli zemí, kterouž jsi zdvihna ruku svou, prisáhl dáti jim.
\par 16 Oni pak a otcové naši pyšne sobe pocínali, a zatvrdivše šíji svou, neposlouchali prikázaní tvých.
\par 17 Nýbrž hned nechteli slyšeti, aniž se rozpomenuli na divné ciny tvé, kteréž jsi pusobil pri nich, a zatvrdivše šíji svou, ustavovali sobe vudce, chtíce se navrátiti k porobení svému z zarputilosti své. Ty však, Bože, snadný k odpuštení, milostivý a lítostivý, dlouho shovívající a hojný v milosrdenství, neopustils jich.
\par 18 Ano i tehdáž, když sobe udelali tele slité a rekli: Tito jsou bohové tvoji, kteríž te vyvedli z Egypta, a dopustili se velikého rouhání.
\par 19 Ty však pro svá mnohá slitování neopustil jsi jich na poušti. Sloup oblakový neodcházel od nich ve dne, veda je po ceste, ani sloup ohnivý v noci, osvecuje je a cestu, po níž jíti meli.
\par 20 Nadto Ducha svého dobrého dal jsi k vyucování jich, manny své také neodjals od úst jejich, a vodu dal jsi jim v žízni jejich.
\par 21 A tak za ctyridceti let krmil jsi je na poušti. V nicemž nedostatku nemeli, odev jejich nezvetšel, a nohy jejich se neodhnetly.
\par 22 Potom dal jsi jim království a národy, kteréž jsi rozehnal do koutu, tak že dedicne obdrželi zemi Seonovu, a zemi krále Ezebon, i zemi Oga krále Bázan.
\par 23 Syny pak jejich rozmnožil jsi jako hvezdy nebeské, a uvedl jsi je do zeme, o kteréž jsi byl rekl otcum jejich, že do ní vejdou, aby jí vládli.
\par 24 Nebo všedše synové, dedicne obdrželi zemi tu, když jsi snížil pred nimi obyvatele té zeme Kananejské, a dals je v ruku jejich, i krále jejich, i národy té zeme, aby s nimi nakládali podlé vule své.
\par 25 A tak vzali mesta hrazená i pole úrodná, a dedicne ujali domy plné všeho dobrého, studnice vykopané, vinice a olivoví, i stromoví ovoce nesoucí velmi mnohé. I jedli, a nasyceni jsouce, vytyli, a v dobrodiní tvém hojném rozkoší oplývali.
\par 26 Když pak popouzejíce te, zprotivili se tobe, zavrhše zákon tvuj za hrbet svuj, a proroky tvé zmordovali, kteríž jim osvedcovali, aby je obrátili k tobe, a dopoušteli se velikého rouhání,
\par 27 Dával jsi je v ruku neprátel jejich, kteríž je ssužovali. A když v cas ssoužení svého volali k tobe, tys je s nebe vyslýchal, a podlé mnohých slitování svých dával jsi jim vysvoboditele, kteríž je vysvobozovali z ruky neprátel jejich.
\par 28 Mezitím, jakž jen oddechnutí meli, zase znovu cinili zlé pred tebou, a protož pouštel jsi je v ruku neprátel jejich, aby panovali nad nimi. Když se pak opet obrátili, a kriceli k tobe, tys je s nebe vyslýchal a vysvobozoval podlé slitování svých po mnohé casy.
\par 29 A napomínals jich, abys je obrátil k zákonu svému, ale oni pyšne sobe pocínali, a neposlouchali prikázaní tvých, a proti soudum tvým hrešili, v nichžto, cinil-li by je clovek, byl by živ. Nýbrž plece svého uchylujíce, šíji svou zatvrzovali, a neposlouchali.
\par 30 A však shovíval jsi jim po mnohá léta, osvedcuje jim duchem svým skrze proroky své, a když neposlouchali, dal jsi je v ruku národum zemí.
\par 31 Ale pro slitování svá mnohá nedals jim do konce zahynouti, aniž jsi jich opustil, proto že jsi Buh milostivý a lítostivý.
\par 32 Nyní tedy, ó Bože náš, silný, veliký, mocný a hrozný, kterýž ostríháš smlouvy a milosrdenství, necht to není u tebe za málo, že ty všecky težkosti na nás prišly, na krále naše, knížata naše, kneží naše, proroky naše i na otce naše, a na všecken lid tvuj, hned ode dnu králu Assyrských, až do tohoto dne,
\par 33 Ackoli ty jsi spravedlivý ve všech tech vecech, kteréž prišly na nás. Nebo jsi spravedlive to ucinil, ale my jsme bezbožne cinili.
\par 34 I králové naši, knížata naše, kneží naši i otcové naši neplnili zákona tvého, aniž šetrili prikázaní tvých a svedectví tvých, jimiž se jim osvedcoval.
\par 35 Nebo oni v království svém a v dobrodiní tvém hojném, kteréž jsi jim ukazoval, a v zemi široké a úrodné, kterouž jsi jim dal, nesloužili tobe, aniž se odvrátili od cinu svých zlých.
\par 36 Aj, my jsme dnes manové, a to v zemi, kterouž jsi dal otcum našim, aby jedli ovoce její a dobré veci její, aj, jsme v ní manové.
\par 37 Jižt úrody své vydává v hojnosti králum, kteréž jsi postavil nad námi pro hríchy naše, a onit i nad tely našimi se potrásají, i nad hovady našimi podlé vule své, tak že jsme u veliké úzkosti.
\par 38 Se vším však tím ciníme smlouvu nepohnutelnou, i zapisujeme, kteréž potvrzují knížata naše, Levítové naši, i kneží naši.

\chapter{10}

\par 1 Kteríž pak potvrzovali, byli tito: Nehemiáš Tirsata, syn Chachaliášuv, a Sedechiáš,
\par 2 Saraiáš, Azariáš, Jeremiáš,
\par 3 Paschur, Amariáš, Malkiáš,
\par 4 Chattus, Sebaniáš, Malluch,
\par 5 Charim, Meremot, Abdiáš,
\par 6 Daniel, Ginneton, Báruch,
\par 7 Mesullam, Abiáš, Miamin,
\par 8 Maaseiáš, Bilkai, Semaiáš. To kneží.
\par 9 Levítové pak: Jesua syn Azaniášuv, Binnui z synu Chenadad, Kadmiel.
\par 10 Bratrí pak jejich: Sebaniáš, Hodiáš, Kelita, Pelaiáš, Chanan,
\par 11 Mícha, Rechob, Chasabiáš,
\par 12 Zakur, Serebiáš, Sebaniáš,
\par 13 Hodiáš, Báni, Beninu.
\par 14 Prední z lidu: Paros, Pachat Moáb, Elam, Zattu, Báni,
\par 15 Bunni, Azgad, Bebai,
\par 16 Adoniáš, Bigvai, Adin,
\par 17 Ater, Ezechiáš, Azur,
\par 18 Hodiáš, Chasum, Bezai,
\par 19 Charif, Anatot, Nebai,
\par 20 Magpias, Mesullam, Chezir,
\par 21 Mesezabel, Sádoch, Jaddua,
\par 22 Pelatiáš, Chanan, Anaiáš,
\par 23 Ozeáš, Chananiáš, Chasub,
\par 24 Loches, Pilcha, Sobek,
\par 25 Rechum, Chasabna, Maaseiáš,
\par 26 Achiáš, Chanan, Anan,
\par 27 Malluch, Charim, Baana.
\par 28 Tak i jiní z lidu, kneží, Levítu, vrátných, zpeváku, Netinejských i všickni, kteríž se oddelili od národu zemí k zákonu Božímu, ženy jejich, synové jejich i dcery jejich, každý umelý a rozumný,
\par 29 Chopivše se téhož s bratrími svými a temi, kteríž byli prednejší, pristupovali, prokletím a prísahou se zavazujíce: Že budeme choditi v zákone Božím, kterýž vydán jest skrze Mojžíše služebníka Božího, a ostríhati i plniti všecka prikázaní Hospodina Pána našeho, i soudy jeho, i ustanovení jeho;
\par 30 Také že nebudeme dávati dcer svých národum zemí, ani dcer jejich bráti synum svým,
\par 31 Ani od cizozemcu, kteríž by nám prinášeli jaké koupe a jakékoli potravy v den sobotní na prodaj, prijímati v sobotu aneb v svátecní den, a že necháme rolí sedmého léta, i všelikého dobývání dluhu.
\par 32 I to narídili jsme mezi sebou, abychom dávali tretí díl lotu na každý rok k službe domu Boha našeho,
\par 33 Na chleby zporádaní, i obet ustavicnou, i na zápal ustavicný k sobotám, novmesícum a slavnostem, i na veci svaté, i obeti za hríchy k ocištení Izraele, i na všecko prisluhování domu Boha našeho.
\par 34 Metali jsme i losy s strany kneží, Levítu i lidu, prícinou dríví nošení, aby ho dodávali do domu Boha našeho, celedem otcu našich casy vymerenými, rok po roku, aby horelo na oltári Hospodina Boha našeho, jakž psáno jest v zákone.
\par 35 Také, že chceme prinášeti prvotiny zeme své, i prvotiny všelikého ovoce každého stromu, rok po roku do domu Hospodinova.
\par 36 K tomu i prvorozené syny své, i hovádka svá, (jakož psáno jest v zákone), i prvorozené z skotu i bravu svých, že prinášeti budeme do domu Boha svého, knežím prisluhujícím v dome Boha našeho.
\par 37 A prvotiny testa svého i obetí svých, i ovoce všelijakého stromu, mstu i oleje nového, aby prinášeli knežím do pokoju domu Boha našeho, a desátky zeme naší Levítum. A Levítové desátky ty vybírati budou po všech mestech, v nichž pracovati budeme.
\par 38 Bude pak pri tom knez, syn Aronuv, s Levíty, když Levítové ty desátky vyberou, a Levítové sami vnesou desátek z desátku do domu Boha našeho, do pokoju v dome skladu.
\par 39 Nebo do tech pokoju donášeti budou synové Izraelští i synové Léví obeti obilé, mstu a oleje nového, (kdež jsou nádoby svatyne), i kneží prisluhující, vrátní i zpeváci, abychom neopoušteli domu Boha svého.

\chapter{11}

\par 1 I prebývala knížata lidu v Jeruzaléme. Jiný pak lid metali losy, aby vzali desátého cloveka k bydlení v Jeruzaléme meste svatém, ale devet dílu v jiných mestech,
\par 2 Ackoli dobrorecil lid všechnem mužum tem, kteríž se sami dobrovolne podali k bydlení v Jeruzaléme.
\par 3 A tito jsou prednejší té krajiny, kteríž se osadili v Jeruzaléme. (V jiných pak mestech Judských bydlili jeden každý v vládarství svém, po mestech svých, lid Izraelský, kneží a Levítové, i Netinejští, též i synové služebníku Šalomounových.)
\par 4 A tak v Jeruzaléme bydlili nekterí z synu Judových i z synu Beniaminových. Z Judových: Ataiáš syn Uziáše, syna Zachariášova, syna Amariášova, syna Sefatiášova, syna Mahalaleelova z synu Fáresových.
\par 5 Též Maaseiáš syn Bárucha, syna Kolchozy, syna Chazaiášova, syna Adaiášova, syna Joiaribova, syna Zachariášova, syna Silonského.
\par 6 Všech synu Fáresových, bydlících v Jeruzaléme, ctyri sta šedesáte osm, mužu udatných.
\par 7 Synové Beniaminovi tito: Sallu syn Mesullama, syna Joedova, syna Pedaiášova, syna Kolaiášova, syna Maaseiášova, syna Itielova, syna Izaiášova.
\par 8 A po nem Gabai, Sallai. Všech devet set dvadceti osm.
\par 9 Joel pak syn Zichri byl jim predstaven, a Juda syn Senua nad mestem druhý po nem.
\par 10 Z kneží: Jedaiáš syn Joiaribuv a Jachin.
\par 11 Saraiáš syn Helkiáše, syna Mesullamova, syna Sádochova, syna Meraiotova, syna Achitobova, prední v dome Božím.
\par 12 Bratrí pak jejich, prisluhujících v tom dome, osm set dvadceti dva. A Adaiáš syn Jerochama, syna Pelaliášova, syna Amzi, syna Zachariášova, syna Paschurova, syna Malkiášova,
\par 13 A bratrí jeho, knížat celedí otcovských, dve ste ctyridceti dva. A Amassai syn Azarele, syna Achzai, syna Mesillemotova, syna Immerova.
\par 14 Bratrí pak jejich, mužu udatných, sto dvadceti osm, jimž predstaven byl Zabdiel syn Gedolimuv.
\par 15 Z Levítu tito: Semaiáš syn Chasuby, syna Azrikamova, syna Chasabiášova, syna Bunni.
\par 16 A Sabbetai s Jozabadem byli nad dílem pri dome Božím vne, z predních Levítu.
\par 17 A Mataniáš syn Míchy, syna Zabdi, syna Azafova, prední, zacínal chvály a modlitbu. A Bakbukiáš druhý z bratrí jeho, a Abda syn Sammua, syna Galalova, syna Jedutunova.
\par 18 Všech Levítu v meste svatém dve ste osmdesát a ctyri.
\par 19 Z vrátných: Akkub, Talmon a bratrí jejich, strážní u bran, sto sedmdesát a dva.
\par 20 Ostatek pak lidu Izraelského, kneží a Levítu, bydlili ve všech mestech Judských, jeden každý v dedictví svém.
\par 21 Ale Netinejští bydlili v Ofel, Zicha pak a Gispa byli predstaveni Netinejským.
\par 22 Predstavený pak Levítum v Jeruzaléme byl Uzi syn Báni, syna Chasabiášova, syna Mataniášova, syna Míchova z synu Azafových, jenž byli zpeváci pri službe domu Božího.
\par 23 Nebo porucení královské bylo o nich, a stálé odmerení pro zpeváky na každý den.
\par 24 A Petachiáš syn Mesezabeluv, z synu Záry syna Judova, místo královské držící v každém jednání k lidu.
\par 25 Ve vsech pak s poli jejich, z synu Judových bydlili v Kariatarbe a v vesnicích jeho, v Dibon a vesnicích jeho, v Jekabsael i ve vsech jeho,
\par 26 A v Jesua, v Molada, v Betfelet,
\par 27 A v Azarsual, v Bersabé i v vesnicích jeho.
\par 28 A v Sicelechu, v Mechona, i v vesnicích jeho,
\par 29 V Enremmon, v Zaraha, v Jarmut,
\par 30 V Zanoe, v Adulam i ve vsech jeho, v Lachis s poli jeho, v Azeka a vesnicích jeho. A tak bydlili od Bersabé až do Gehinnom.
\par 31 Synové pak Beniaminovi z Gabaa v Michmas, v Aia, v Bethel i v vesnicích jeho,
\par 32 V Anatot, v Nobe, v Anania,
\par 33 V Azor, v Ráma, v Gittaim,
\par 34 V Chadid, v Seboim, v Neballat,
\par 35 V Lod, v Ono a v údolí remeslníku.
\par 36 Z Levítu pak nekterí bydlili v dílích Judských a Beniaminských.

\chapter{12}

\par 1 Tito pak jsou kneží a Levítové, kteríž se byli navrátili s Zorobábelem synem Salatielovým, a s Jesua: Saraiáš, Jeremiáš, Ezdráš,
\par 2 Amariáš, Malluch, Chattus,
\par 3 Sechaniáš, Rechum, Meremot,
\par 4 Iddo, Ginnetoi, Abiáš,
\par 5 Miamin, Maadiáš, Bilga,
\par 6 Semaiáš, Joiarib, Jedaiáš,
\par 7 Sallu, Amok, Helkiáš, Jedaiáš. Ti byli prednejší z kneží a bratrí svých za casu Jesua.
\par 8 Levítové pak: Jesua, Binnui, Kadmiel, Serebiáš, Juda; Mattaniáš, postavený nad zpevy chvalitebnými s bratrími svými.
\par 9 A Bakbukiáš a Unni, bratrí jejich, byli naproti nim v porádcích svých.
\par 10 Jesua pak zplodil Joiakima, a Joiakim zplodil Eliasiba, Eliasib pak zplodil Joiadu.
\par 11 A Joiada zplodil Jonatana, Jonatan pak zplodil Jaddua.
\par 12 Za casu pak Joiakima byli prední kneží z celedí otcovských: Z Saraiášovy Meraiáš, z Jeremiášovy Chananiáš,
\par 13 Z Ezdrášovy Mesullam, z Amariášovy Jochanan,
\par 14 Z Melichovy Jonatan, z Sebaniášovy Jozef,
\par 15 Z Charimovy Adna, z Meraiotovy Chelkai,
\par 16 Z Iddovy Zachariáš, z Ginnetonovy Mesullam,
\par 17 Z Abiášovy Zichri, z Miniaminovy a z Moadiášovy Piltai,
\par 18 Z Bilgovy Sammua, z Semaiášovy Jonatan,
\par 19 Z Joiaribovy Mattenai, z Jedaiášovy Uzi,
\par 20 Z Sallaiovy Kallai, z Amokovy Heber,
\par 21 Z Helkiášovy Chasabiáš, z Jedaiášovy Natanael.
\par 22 Levítové pak prednejší z celedí otcovských za dnu Eliasiba, Joiady, Jochanana a Jaddua, zapsáni jsou až do kralování Daria Perského.
\par 23 Synové, pravím, Léví, prední v celedech otcovských, zapsáni jsou v knize Paralipomenon, až do casu Jochanana syna Eliasibova.
\par 24 Potom prední Levítové: Chasabiáš, Serebiáš, a Jesua syn Kadmieluv, a bratrí jejich naproti nim k chválení a oslavování Boha, podlé narízení Davida muže Božího, trída proti tríde.
\par 25 Mataniáš, Bakbukiáš, Abdiáš, Mesullam, Talmon, Akkub, držící stráž vrátných pri domu pokladu u bran.
\par 26 Ti byli za casu Joiakima , syna Jesua, syna Jozadakova, a za casu Nehemiáše vudce, a Ezdráše kneze a ucitele.
\par 27 Ku posvecování pak zdí Jeruzalémských shlédávali Levíty ze všech míst jejich, aby je privedli do Jeruzaléma, aby vykonali posvecení a veselí, a to s oslavováním a zpevy, cymbály, loutnami a harfami.
\par 28 Protož shromáždeni jsou synové zpeváku, i z rovin okolo Jeruzaléma, i ze vsí Netofatských,
\par 29 Též z domu Galgal, a z polí Gaba i Azmavet; nebo vsi staveli sobe zpeváci okolo Jeruzaléma.
\par 30 A ocistivše se kneží a Levítové, ocistili také lid, brány i zed.
\par 31 Za tím rozkázal jsem vstoupiti knížatum Judským na zed, a postavil jsem dva houfy veliké oslavujících, z nichž jedni šli na pravo, od horní strany zdi k bráne hnojné.
\par 32 A za temi šel Hosaiáš a polovice knížat Judských,
\par 33 Též Azariáš, Ezdráš, a Mesullam,
\par 34 Juda, Beniamin, Semaiáš a Jeremiáš.
\par 35 Potom za syny knežskými s trubami Zachariáš syn Jonatana, syna Semaiášova, syna Mattaniášova, syna Michaiášova, syna Zakurova, syna Azafova.
\par 36 A bratrí jeho: Semaiáš, Azarel, Milalai, Gilalai, Maai, Natanael a Juda, Chanani s nástroji hudebnými Davida muže Božího, Ezdráš pak ucitel pred nimi.
\par 37 Potom k bráne u studnice, kteráž naproti nim byla, vstupovali po stupních mesta Davidova, kudy se chodí na zed, a ode zdi pri dome Davidove, až k bráne vodné k východu.
\par 38 Houf pak druhý oslavujících bral se naproti onemno, a já za nimi, a polovice lidu po zdi od veže Tannurim až ke zdi široké,
\par 39 A od brány Efraim k bráne staré, a k bráne rybné, a veži Chananeel, a veži Mea, až k bráne bravné. I zastavili se v bráne stráže.
\par 40 Potom zastavili se oba houfové oslavujících v dome Božím, i já, a polovice knížat se mnou.
\par 41 Ano i kneží: Eliakim, Maaseiáš, Miniamin, Michaiáš, Elioenai, Zachariáš, Chananiáš, s trubami,
\par 42 A Maaseiáš, Semaiáš, Eleazar, Uzi, Jochanan, Malkiáš, Elam a Ezer. Zpeváci pak zvucne zpívali s Izrachiášem predstaveným svým.
\par 43 Obetovali také v ten den obeti veliké, a veselili se; nebo Buh obveselil je veselím velikým. Ano i ženy a deti veselily se, tak že bylo slyšáno veselí Jeruzaléma opodál.
\par 44 Mezi tím zrízeni jsou v ten den muži nad komorami k pokladum a k obetem, i k prvotinám a k desátkum, aby shromaždovali do nich s polí mestských díly, zákonem vymerené knežím a Levítum; nebo veselil se Juda z kneží a Levítu prístojících,
\par 45 Kteríž držeti meli stráž Boha svého, a stráž ocištování, a zpeváku i vrátných, podlé narízení Davidova a Šalomouna syna jeho.
\par 46 Nebo za casu Davidova a Azafova od starodávna prední zpeváci k zpívání, chválení a oslavování Boha stáli pred ním.
\par 47 Procež všecken Izrael za dnu Zorobábele a za casu Nehemiáše dávali díly pro zpeváky a vrátné, na každý den stálé odmerení, a odvodili je Levítum, Levítové pak dávali synum Aronovým.

\chapter{13}

\par 1 V ten den cteno jest v knize Mojžíšove, tak že lid slyšeti mohl. I nalezeno v ní napsáno, že nemá vjíti Ammonitský a Moábský do shromáždení Božího až na veky,
\par 2 Proto že nevyšli proti synum Izraelským s chlebem a s vodou, ale najali ze mzdy proti nim Baláma, aby jim zlorecil, ackoli obrátil Buh náš to zlorecení v požehnání.
\par 3 Stalo se pak, když slyšeli zákon, že odmísili všecky primíšené od Izraele.
\par 4 Ale pred tím Eliasib knez, správce komory domu Boha našeho, spríznil se s Tobiášem.
\par 5 Kterémuž udelal pokoj veliký, kdež prvé skládali dary, kadidlo a nádoby, a desátky z obilé, mstu a oleje nového, narízené Levítum a zpevákum i vrátným, též i obet knežím.
\par 6 Ale když se to vše dálo, nebyl jsem v Jeruzaléme. Nebo léta tridcátého druhého Artaxerxa krále Babylonského prišel jsem k králi, a po prebehnutí let, vyžádán jsem na králi.
\par 7 Když jsem pak prišel do Jeruzaléma, srozumel jsem tomu zlému, co ucinil Eliasib pro Tobiáše, udelav mu pokoj v síních domu Božího.
\par 8 Což mi se velmi nelíbilo. Protož vyházel jsem všecko nádobí domu Tobiášova ven z toho pokoje.
\par 9 A rozkázal jsem ocistiti ty pokoje. I vnesl jsem tam zase nádoby domu Božího, dary a kadidlo.
\par 10 Potom dovedev se, že dílové Levítum nebyli dáváni, a že rozbehli se jeden každý na svou rolí, Levítové i zpeváci vedoucí práci,
\par 11 Protož domlouval jsem se na starší, rka: Proc jest opušten dum Boží? A shromáždiv je, postavil jsem je na míste jejich.
\par 12 A všecken Juda prinášeli desátky obilé, mstu a oleje nového do skladu.
\par 13 A ustanovil jsem úredníky nad sklady: Selemiáše kneze, a Sádocha ucitele, a Pedaiáše z Levítu, pridav k nim Chanana syna Zakurova, syna Mattaniášova; nebo za verné jmíni byli. A ti meli rozdelovati bratrím svým.
\par 14 Budiž pametliv na mne, Bože muj, pro to, a nevyhlazuj dobrodiní mých, kteráž jsem prokázal k domu Boha svého i k službám jeho.
\par 15 V tech dnech videl jsem v Judstvu, ani tlací presem v sobotu, a prinášejí snopy, kteréž nakládali na osly, též víno, hrozny, fíky i všeliká bremena, a snášejí v den sobotní do Jeruzaléma. I domlouval jsem jim v ten den, když prodávali potravu.
\par 16 Tyrští také, kteríž bydlili v nem, nosili ryby i všelijaké koupe, a prodávali v sobotu synum Juda, a to v Jeruzaléme.
\par 17 Protož jsem domlouval starším Judským, a rekl jsem jim: Jaká jest to nepravost, kterouž ciníte, poškvrnujíce dne sobotního?
\par 18 Zdaliž jsou tak necinili otcové vaši? Procež Buh náš privedl na nás všecko toto zlé, i na toto mesto, a vy pridáváte hnevivosti na Izraele, poškvrnujíce soboty.
\par 19 Když tedy byly v stínu brány Jeruzalémské pred sobotou, rozkázal jsem zavríti brány, a rozkázal jsem, aby jich neotvírali až po sobote. K tomu i z služebníku svých nekteré postavil jsem v bráne, aby náklad nebyl vezen do mesta v sobotu.
\par 20 Protož zustali kupci a prodavaci všelijakých vecí prodajných vne pred Jeruzalémem, jednou i podruhé.
\par 21 I osvedcil jsem se jim, rka jim: Proc zustáváte pres noc naproti zdi? Uciníte-li to více, vztáhnu ruku na vás. Od té chvíle nepricházeli v sobotu.
\par 22 Rozkázal jsem pak Levítum, aby se ocistili, a prijdouce, ostríhali bran, a svetili den sobotní. Také i v tom pamatuj na mne, Bože muj, a bud mi milostiv podlé množství milosrdenství svého.
\par 23 V tech dnech spatril jsem také Židy, kteríž byli pojali ženy Azotské, Ammonitské a Moábské,
\par 24 An synové jejich mluvili od polu Azotsky, a neumeli mluviti Židovsky, ale podlé jazyku každého toho lidu.
\par 25 Protož domlouval jsem jim, a zlorecil jsem jim, a nekteré z nich bil jsem a rval, prísahou je zavazuje skrze Boha, rka: Budete-li dávati dcery své synum jejich, aneb bráti dcery jejich synum svým neb sobe, zlorecení budete.
\par 26 Zdaliž tím nezhrešil Šalomoun král Izraelský? Ješto ve mnohých národech nebylo krále jemu podobného, kterýž byl milý Bohu svému, tak že ustanovil jej Buh králem nade vším Izraelem, však i toho k hríchu privedly ženy cizozemky.
\par 27 A vám zdali povolíme, abyste se dopoušteli všeho toho zlého velikého, a prestupovali proti Bohu svému, pojímajíce ženy cizozemky?
\par 28 Z synu pak Joiady syna Eliasiba, kneze nejvyššího, jeden byl zet Sanballata Choronského, kteréhož jsem zahnal od sebe.
\par 29 Budiž pametliv na to, Bože muj, proti tem, kteríž poškvrnují knežství, a smlouvy knežské i Levítské.
\par 30 Protož jsem je vycistil od všelikého cizozemce, a ustanovil jsem zase trídy knežím a Levítum, jednomu každému v práci jeho,
\par 31 I nošení dríví k obetem casy uloženými, též i prvotin. Budiž pametliv na mne, Bože muj, k mému dobrému.

\end{document}