\begin{document}

\title{Esther}

\chapter{1}

\par 1 Stalo se pak za casu krále Asvera, (to j\par ten Asverus, jenž kraloval od Indie až k Mourenínské zemi nad sto dvadcíti a sedmi krajinami),
\par 2 Že toho casu, když sedel král Asverus na stolici království svého, jenž byla v Susan, meste královském,
\par 3 Léta tretího kralování svého, ucinil u sebe hody všechnem knížatum svým a služebníkum svým, nejznamenitejším Perským a Médským, hejtmanum a vládarum nad krajinami,
\par 4 Ukazuje bohatství, slávu království svého a cest, i ozdobu dustojnosti své za mnoho dnu, totiž za sto a osmdesáte dnu.
\par 5 (A když se vyplnili dnové ti, ucinil král všemu lidu, což ho koli bylo v Susan meste královském, od nejvetšího až do nejmenšího, hody za sedm dní na paláci v zahrade pri dome královském.)
\par 6 Též calouny bílé, zelené a z postavce modrého, zavešené na provázcích kmentových a šarlatových u kroužku stríbrných, na sloupích mramorových, lužka zlatá a stríbrná na podlaze porfyretové a mramorové, pariové a socharetové.
\par 7 Nápoj pak dávali v nádobách zlatých, a to v nádobách jiných a jiných, i vína královského v hojnosti, jakž slušelo na krále.
\par 8 Ale ku pití, podlé narízení, žádný nenutil. Nebo tak porucil král všechnem správcum domu svého, aby cinili podlé vule jednoho každého.
\par 9 Také i královna Vasti ucinila hody ženám v dome královském krále Asvera.
\par 10 Dne pak sedmého, když se podveselil král vínem, rozkázal Mehumanovi, Biztovi, Charbonovi, Bigtovi a Abagtovi, Zetarovi a Karkasovi, sedmi komorníkum, kteríž sloužili pred oblícejem krále Asvera,
\par 11 Aby privedli Vasti královnu pred oblícej krále v korune královské, aby okázal národum i knížatum krásu její; nebo velmi krásná byla.
\par 12 Ale odeprela královna Vasti prijíti k rozkazu královskému, skrze ty komorníky vzkázanému. Procež král rozhneval se velmi a rozpálil se hnevem sám v sobe.
\par 13 I rekl král mudrcum znajícím casy, (nebo tak každé veci podával král na všecky zbehlé v právích a soudech),
\par 14 A nejbližšímu sebe Charsenovi, Setarovi, Admatovi, Tarsisovi, Meresovi, Marsenovi, Memuchanovi, sedmi vývodám Perským a Médským, jenž vídali tvár královskou, a sedali první po králi:
\par 15 Co se má podlé práva státi s královnou Vasti, proto že nevyplnila rozkazu krále Asvera, stalého skrze komorníky?
\par 16 Tedy rekl Memuchan pred králem i knížaty: Ne proti samému králi zavinila královna Vasti, ale proti všechnem knížatum, a proti všechnem národum všech krajin Asvera krále.
\par 17 Nebo když se donese to, co ucinila královna, všech žen, zlehcí sobe muže své a reknou: An král Asverus rozkázal privésti královnu Vasti pred oblícej svuj, a však neprišla.
\par 18 Nýbrž ješte tohoto dne budou to mluviti knežny Perské a Médské, (kteréž slyšely, co ucinila královna), všechnem knížatum královským, i naplodí se hojne pýchy a zpoury.
\par 19 Jestliže se tedy králi za dobré vidí, necht se stane výpoved královská od oblíceje jeho, a necht j\par vepsána mezi práva Perská a Médská, kteráž by nemohla zmenena býti, že nechtela prijíti Vasti pred oblícej krále Asvera, procež království její dá král jiné, lepší nežli ona.
\par 20 Tak když uslyší výpoved královskou, kterouž vyhlásiti dá po všem království svém, jakkoli veliké jest, všecky ženy v poctivosti míti budou manžely své, od nejvetšího až do nejmenšího.
\par 21 I líbila se ta rada králi i knížatum, a ucinil král podlé rady Memuchanovy.
\par 22 A rozeslal listy do všech krajin královských, do jedné každé krajiny písmem jejím, a do každého národu jazykem jeho, aby každý muž byl pánem domu svého. Což oznámil každý hejtman lidu jazykem jeho.

\chapter{2}

\par 1 Ty veci když se zbehly, a když se upokojila prchlivost krále Asvera, zpomenul na Vasti a na to, což byla ucinila, též i na to, jaká se výpoved stala proti ní.
\par 2 I rekli mládenci královští, služebníci jeho: Necht hledají králi mladic, panen krásných.
\par 3 A necht zrídí král úredníky ve všech krajinách království svého, kteríž by shromáždili všecky mladice, panny krásné, do Susan mesta královského, do domu ženského, pod stráž Hegai komorníka královského, strážce žen, a at jim vydává okrasy jejich.
\par 4 A mladice, kteráž by se zalíbila králi, aby kralovala místo Vasti. I líbila se ta vec králi, a ucinil tak.
\par 5 Byl pak jeden Žid v Susan, meste královském, jménem Mardocheus syn Jairuv, syna Simei, syna Cis, z pokolení Beniamin.
\par 6 A ten byl prestehován z Jeruzaléma s prestehovanými, kteríž prestehováni byli s Jekoniášem králem Judským, kteréhož prestehoval Nabuchodonozor král Babylonský.
\par 7 Ten choval Hadassu, jinak Esteru, dceru strýce svého, proto že nemela otce ani matky. A devecka ta byla pekné postavy a krásné tvári, kterouž po smrti otce a matky její vzal sobe Mardocheus za dceru.
\par 8 Stalo se tedy, když vyšlo slovo královské a rozkaz jeho, a když shromáždováno bylo panen mnoho do Susan, mesta královského, pod stráž Hegai, že i Ester vzata byla do domu královského pod stráž Hegai, strážného nad ženami.
\par 9 I líbila se jemu ta mladice, a našla milost u neho. Procež i hned dal jí okrasu její a díl její, a sedm panen zpusobných z domu královského; k tomu opatrení její i jejích panen promenil v lepší v dome ženském.
\par 10 Neoznámila pak Ester lidu svého, ani rodiny své; nebo prikázal jí byl Mardocheus, aby nepravila.
\par 11 Ale Mardocheus každého dne chodíval pred síní domu ženského, aby zvedel, jak se má Ester, a co se s ní deje.
\par 12 Když pak prišel jistý cas jedné každé panny, aby vešla pred krále Asvera, když se vyplnilo pri ní podlé práva žen, za dvanácte mesícu, (nebo za tolik dnu ozdobovaly se, š\par mesícu olejem mirrovým, a š\par mesícu vecmi vonnými a ozdobami ženskými),
\par 13 A tak pricházela panna pred krále. Cožkoli rekla, dávalo se jí, aby s tím šla z domu žen až k domu královskému.
\par 14 U vecer vcházela k králi, a ráno zase odcházela do druhého domu ženského, pod stráž Saasgazy, komorníka královského, strážce ženin. Nepricházela více k králi, ale jestliže se líbila králi, povolána bývala ze jména.
\par 15 Tedy když prišel cas jistý Estery, dcery Abichaile, strýce Mardocheova, kterouž byl vzal sobe za dceru, aby vešla k králi, nežádala niceho, než což rekl Hegai komorník královský, strážce žen. I líbila se Ester všechnem, kteríž ji videli.
\par 16 A tak vzata j\par Ester k králi Asverovi do domu jeho královského, mesíce desátého, (jenž j\par mesíc Tebet), léta sedmého kralování jeho.
\par 17 I zamiloval král Ester nade všecky jiné ženy, a nalezla milost a lásku u neho nade všecky panny, tak že vstavil korunu královskou na hlavu její, a ucinil ji královnou místo Vasti.
\par 18 K tomu také ucinil král hody veliké všechnem knížatum svým a služebníkum svým, totiž hody Estery, a dal odpocinutí krajinám, a daroval je tak, jakž slušelo na krále.
\par 19 A když podruhé shromaždovány byly panny, a Mardocheus sedel u brány královské,
\par 20 Ester pak neoznámila byla rodiny své a národu svého, jakž jí byl prikázal Mardocheus; nebo rozkázaní Mardocheovo Ester cinila, jako i když chována byla u neho:
\par 21 V tech dnech, (když Mardocheus sedal u brány královské), rozhneval se Bigtan a Teres, dva komorníci královští z strážných prahu, a smýšleli o to, aby vztáhli ruce na krále Asvera.
\par 22 Cehož dovedev se Mardocheus, oznámil to Ester královne, Ester pak oznámila králi jménem Mardocheovým.
\par 23 A když bylo toho vyhledáváno, našlo se tak. I obešeni jsou ti oba na šibenici, a zapsáno j\par to do kronik pred králem.

\chapter{3}

\par 1 Po tech vecech zvelebil král Asverus Amana syna Hammedatova Agagského, a vyvýšil ho, tak že vyzdvihl stolici jeho nade všecka jiná knížata, kteríž byli pri nem.
\par 2 A všickni služebníci královští, kteríž vcházeli do brány královské, klaneli se a padali pred Amanem; nebo tak prikázal o nem král. Ale Mardocheus se neklanel, ani padal.
\par 3 Protož rekli služebníci královští, kteríž byli v bráne královské, Mardocheovi: Proc prestupuješ prikázaní královské?
\par 4 I stalo se, když o to s ním mluvívali každého dne, a neposlechl jich, že to oznámili Amanovi, aby videli, bude-li stálý v svých slovích Mardocheus; nebo oznámil jim, že j\par Žid.
\par 5 Vida pak Aman, že se Mardocheus neklaní, ani padá pred ním, naplnen j\par Aman prchlivostí.
\par 6 Ale za malou vec sobe položil, vztáhnouti ruku na Mardochea samého, (nebo byli oznámili jemu, z kterého by lidu byl Mardocheus). Protož smýšlel Aman, aby zahubil národ Mardocheuv, totiž všecky Židy, kteríž byli ve všem království Asverovu.
\par 7 Takž mesíce prvního, totiž mesíce Nísan, léta dvanáctého kralování Asverova, rozkázal uvrci pur, totiž los, pred sebou ode dne ke dni, a od mesíce až do mesíce dvanáctého, jenž j\par mesíc Adar.
\par 8 Nebo byl rekl Aman králi Asverovi: J\par lid jakýsi rozptýlený a roztroušený mezi lidem ve všech krajinách království tvého, jejichž práva rozdílná jsou ode všech národu, práv pak královských neostríhají. Protož králi není užitecné, nechati jich.
\par 9 Jestliže se králi za dobré vidí, necht se napíše, aby je zahladili, a já deset tisíc centnéru stríbra odvážím do rukou predstaveným té práci, aby je vnesli do komory královské.
\par 10 Tedy král snav prsten svuj s ruky své, dal jej Amanovi synu Hammedatovu Agagskému, nepríteli Židovskému.
\par 11 A rekl král Amanovi: Stríbro to daruji tobe i lid ten, abys naložil s ním, jakt se koli líbí.
\par 12 Protož povoláni jsou písari královští téhož mesíce prvního trináctého dne, a psáno j\par všecko tak, jakž prikázal Aman, k knížatum královským i vývodám, kteríž byli v jedné každé krajine, i hejtmanum jednoho každého národu, každé krajine vedlé písma jejího, a každému národu vedlé jazyku jeho; jménem krále Asvera psáno, a zapeceteno prstenem královským.
\par 13 I jsou posláni listové po poslích do všech krajin královských, aby hubili, mordovali a vyhladili všecky Židy, od mladého až do starce, deti i ženy, jednoho dne, totiž trináctého, mesíce dvanáctého, (jenž j\par mesíc Adar), a loupeže jejich aby rozbitovali.
\par 14 Summa toho psání byla: Aby vyhlášeno bylo v jedné každé krajine a oznámeno všechnem národum, aby hotovi byli ke dni tomu.
\par 15 Tedy vyjeli poslové steží s porucením královským, a vyhlášeno j\par to v Susan, meste královském. Král pak a Aman sedeli, kvasíce, ale meštané Susan zkormouceni byli.

\chapter{4}

\par 1 : Mardocheus pak zvedev všecko to, což se bylo stalo, roztrhl roucho své, a vzal na sebe žíni a popel, a jda po meste, kricel krikem velikým a žalostným.
\par 2 A prišel až pred bránu královskou; nebo žádný nemel vcházeti do brány královské v odevu žíneném.
\par 3 V každé také krajine i míste, kamžkoli slovo královské a výpoved jeho došla, kvílení veliké bylo od Židu, a pust, plác i úpení, a žíni s popelem podestreli sobe mnozí.
\par 4 Procež prišedše devecky Estery a komorníci její, oznámili jí to. I zarmoutila se královna velmi, a poslala šaty, aby oblékli Mardochea, a aby vzali žíni jeho od neho. Ale on jich neprijal.
\par 5 Tedy zavolavši Ester Hatacha, jednoho z komorníku královských, kteréhož jí byl dal za služebníka, porucila mu o Mardocheovi, aby prezvedel, co a proc by to bylo.
\par 6 Takž vyšel Hatach k Mardocheovi na ulici mesta, kteráž j\par pred branou královskou.
\par 7 I oznámil jemu Mardocheus všecko, co se mu prihodilo, i o té summe penez, kterouž pripovedel Aman dáti do komory královské proti Židum, k vyhlazení jich.
\par 8 K tomu i prípis psané výpovedi, kteráž vyhlášena byla v Susan, aby zhubeni byli, dal jemu, aby ukázal Ester a oznámil jí, a aby jí rozkázal jíti k králi, a milosti hledati u neho, a prositi pred tvárí jeho za lid svuj.
\par 9 Tedy prišel Hatach, a oznámil Ester slova Mardocheova.
\par 10 I rekla Ester Hatachovi, a porucila mu oznámiti Mardocheovi:
\par 11 Všickni služebníci královští i lid krajin královských vedí, že kterýž by koli muž neb žena všel pred krále do síne vnitrní, nejsa povolán, jedno právo o nem jest, aby hrdlo propadl, krome toho, k komuž by král vztáhl berlu zlatou, že živ zustane. Já pak nebyla jsem povolána, abych vešla k králi již tridceti dnu.
\par 12 Ale když oznámili Mardocheovi slova Estery,
\par 13 Rekl Mardocheus, aby zase oznámeno bylo Estere: Nemysl sobe, abys zachována býti mohla v dome královském ze všech Židu.
\par 14 Nebo jestliže se umlcíš v tento cas, oddechnutí a vysvobození Židum prijde odjinud, ty pak a dum otce tvého zahynete. A kdo ví, ne pro tento-lis cas prišla k tomu království?
\par 15 I rekla Ester, aby zase oznámili Mardocheovi:
\par 16 Jdi a shromažd všecky Židy, což j\par jich v Susan, a postte se za mne, a nejezte ani píte za tri dni, v noci ani ve dne. Já podobne, i panny mé postiti se budou, a teprv vejdu k králi, což však neobycejné jest, a jestližet zahynu, necht zahynu.
\par 17 Tedy šel Mardocheus, a ucinil všecko tak, jakž mu porucila Ester.

\chapter{5}

\par 1 Stalo se potom dne tretího, oblékši se Ester v roucho královské, postavila se v síni vnitrní domu královského, naproti pokoji královskému. Král pak sedel na stolici své královské, v pokoji královském naproti dverím domu.
\par 2 Stalo se, pravím, když uzrel král Ester královnu, ana stojí v síni, že milostí jsa k ní naklonen, vztáhl král k Ester berlu zlatou, kterouž držel v ruce své. Tedy pristoupivši Ester, dotkla se konce berly.
\par 3 I rekl jí král: Co j\par tobe, královno Ester? A jaká j\par prosba tvá? Až do polovice království dáno bude tobe.
\par 4 Odpovedela Ester: Jestliže se králi za dobré vidí, nechat prijde král s Amanem dnes na hody, kteréž jsem jemu pripravila.
\par 5 I rekl král: Zavolejte rychle Amana, at naplní žádost Estery. A tak prišel král s Amanem na hody, kteréž byla pripravila Ester.
\par 6 Potom rekl král k Ester, napiv se vína. Jaká j\par žádost tvá? A budet dáno. Aneb jaká j\par prosba tvá? Bys pak žádala až do polovice království, stanet se.
\par 7 I odpovedela Ester: Žádost má a prosba má jest:
\par 8 Nalezla-li jsem milost u krále, a jestliže se králi za dobré vidí povoliti žádosti mé, a naplniti prosbu mou, aby ješte prišel král i Aman na hody, kteréž jim pripravím, a zítra uciním podlé slova královského.
\par 9 A tak vyšel Aman dne toho, vesel jsa a dobré mysli. Ale když videl Aman Mardochea v bráne královské, že ani nepovstal, ani se nehnul pred ním, naplnen j\par Aman hnevem proti Mardocheovi.
\par 10 A však zdržel se Aman, až prišel do domu svého, a poslav, povolal prátel svých a Zeres ženy své.
\par 11 I vypravoval jim Aman o sláve bohatství svého, i o množství synu svých, i o všem, címž ho zvelebil král, a jak ho vyvýšil nad knížata i služebníky královské.
\par 12 A doložil Aman: Nadto nepozvala Ester královna s králem na hody, kteréž byla pripravila jen mne, a ješte i k zejtrí pozván jsem od ní s králem.
\par 13 Ale všecko to nic mi neprospívá, pokudžkoli vídám Mardochea, toho Žida, sedati u brány královské.
\par 14 Rekla jemu Zeres žena jeho, i všickni prátelé jeho: Necht udelají šibenici zvýši padesáti loket, a ráno rci králi, aby na ní obesili Mardochea, a vejdi s králem na hody vesele. I líbila se ta rada Amanovi, a rozkázal postaviti šibenici.

\chapter{6}

\par 1 Té noci král nemoha spáti, rozkázal prinésti knihy pametné, kroniky, kteréž cteny byly pred králem.
\par 2 I našli zapsáno, že povedel Mardocheus na Bigtana a Teresa, dva komorníky královské z tech, jenž ostríhali prahu, že ukládali vztáhnouti ruku na krále Asvera.
\par 3 Tedy rekl král: Cím j\par pocten, neb jak zveleben Mardocheus za to? Odpovedeli služebníci královští, dvorané jeho: Není jemu nic dáno.
\par 4 I rekl král: Kdo j\par v síni? (Aman pak byl prišel do síne domu královského zevnitrní, mluviti s králem, aby obešen byl Mardocheus na šibenici, kterouž jemu pripravil.)
\par 5 Odpovedeli králi služebníci jeho: Aj, Aman stojí v síni. Rekl král: Necht vejde sem.
\par 6 Takž všel Aman. Jemuž rekl král: Co sluší uciniti muži tomu, kteréhož by chtel král ctíti? (Aman rekl sám v sobe: Kohož by jiného chtel král více ctíti, nežli mne?)
\par 7 Odpovedel Aman králi: Muži tomu,jehož král ctíti chce,
\par 8 At prinesou roucho královské, do kteréhož se král oblácí, a privedou kone, na kterémž jezdí král, a vstaví korunu královskou na hlavu jeho.
\par 9 A dadouce roucho to i kone toho v ruku nekterého z nejznamenitejších knížat královských, at oblekou muže toho, jehož král chce ctíti, a at jej vodí na koni po ulici mesta, a volají pred ním: Tak se má státi muži tomu, jehož král ctíti chce.
\par 10 Tedy rekl král Amanovi: Pospeš, vezmi to roucho a kone, jakž jsi rekl, a ucin to Mardocheovi Židu, kterýž sedí v bráne královské. Nepomíjejž niceho ze všeho toho, což jsi mluvil.
\par 11 Protož vzav Aman roucho i kone, oblékl Mardochea, a vedl jej na koni po ulici mesta, volaje pred ním: Tak se má státi muži tomu, jehož král ctíti chce.
\par 12 Potom navrátil se Mardocheus k bráne královské. Aman pak rychle pospíšil do domu svého, smuten jsa, s zakrytou hlavou.
\par 13 I vypravoval Aman Zeres žene své a všechnem prátelum svým všecko, což se mu prihodilo. I rekli jemu mudrci jeho i Zeres žena jeho: Ponevadž z národu Židovského j\par Mardocheus, pred jehož oblícejem pocal jsi klesati, neodoláš jemu, ale jiste padneš pred oblícejem jeho.
\par 14 A když oni ješte mluvili s ním, komorníci královští prišli, a rychle vzali Amana na hody, kteréž pripravila Ester.

\chapter{7}

\par 1 A tak prišel král i Aman, aby hodovali s Ester královnou.
\par 2 I rekl král k Ester opet druhého dne, napiv se vína: Jaká j\par žádost tvá, Ester královno? A budet dáno. Aneb jaká j\par prosba tvá? Až do polovice království stanet se.
\par 3 Tedy odpovedela Ester královna a rekla: Jestliže jsem nalezla milost pred ocima tvýma, ó králi, a jestliže se králi za dobré vidí, necht mi j\par darován život muj k mé žádosti, a národu mému k prosbe mé.
\par 4 Nebo prodáni jsme, já i národ muj, abychom zbiti, pomordováni a vyhlazeni byli, ješto kdybychom za služebníky a devky prodáni byli, mlcela bych, ac by i tak ten neprítel nijakž nemohl nahraditi králi té škody.
\par 5 Opet odpovídaje král Asverus, rekl Ester královne: I kdož j\par to ten, a kde j\par ten, jehož srdce tak j\par naduté, aby to cinil?
\par 6 I rekla Ester: Muž protivník a neprítel nejhorší j\par Aman tento. Takž Aman zhrozil se pred oblícejem krále a královny.
\par 7 Tedy král vstal v prchlivosti své od pití vína, a vyšel na zahradu pri paláci. Aman pak pozustal tu, aby prosil za život svuj Estery královny; nebo videl, že j\par o nem zle uloženo od krále.
\par 8 Potom král navrátil se z zahrady palácu do domu, kdež pil víno, a Aman padl na lužko, na kterémž sedela Ester. I rekl král: Což ješte násilé uciniti chce královne u mne v dome? A hned jakž slovo to vyšlo z úst krále, tvár Amanovu zakryli.
\par 9 Mezi tím rekl Charbona, jeden z komorníku, pred králem: Aj, ješte šibenice, kterouž pripravil Aman Mardocheovi, kterýž mluvil králi k dobrému, stojí pri dome Amanove, zvýší padesáti loket. I rekl král: Obestež ho na ní.
\par 10 Tedy obesili Amana na té šibenici, kterouž byl pripravil Mardocheovi. A tak prchlivost královská ukojila se.

\chapter{8}

\par 1 Téhož dne dal král Asverus Ester královne dum Amana neprítele Židovského, a Mardocheus prišel pred krále, (nebo oznámila Ester, co by jí on byl).
\par 2 Kdežto král snav prsten svuj, kterýž vzal od Amana, dal jej Mardocheovi. Ester pak ustanovila Mardochea nad domem Amanovým.
\par 3 Potom ješte Ester mluvila pred králem, padši k nohám jeho, a s plácem pokorne ho prosila, aby zrušil nešlechetnost Amana Agagského a jeho úklady, kteréž smyslil proti Židum.
\par 4 (Tedy vztáhl král k Estere berlu zlatou, a Ester vstavši, postavila se pred králem.)
\par 5 A rekla: Jestliže se králi za dobré vidí, a nalezla-li jsem milost pred tvárí jeho, a jestliže se ta vec králi slušná býti vidí, a já jsem-li príjemná pred ocima jeho: necht j\par psáno, aby byli zrušeni listové ti, a tak úkladové Amana syna Hammedatova Agagského, kteréž rozepsal, aby vyhladili Židy, což j\par jich ve všech krajinách královských.
\par 6 Nebo jak bych se mohla dívati na to zlé, kteréž by potkalo lid muj? A jak bych mohla hledeti na zhoubu rodiny své?
\par 7 I rekl král Asverus Ester královne a Mardocheovi Židu: Ej, dum Amanuv dal jsem Estere, jeho pak obesili na šibenici, proto že vztáhnouti chtel ruku svou na Židy.
\par 8 Vy tedy pište Židum, jakž se vám za dobré zdá, jménem královským, a zapecette prstenem královským. (Nebo což se píše jménem krále, a zapecetí prstenem královským, nemuže zpátkem jíti.)
\par 9 Takž svolali písare královské v ten cas mesíce tretího, jenž j\par mesíc Siban, dvadcátého tretího dne téhož mesíce, a psáno j\par všecko tak, jakž prikázal Mardocheus, k Židum a knížatum, i vývodám a hejtmanum krajin, kteréž jsou od Indie až do zeme Mourenínské, sto dvadceti sedm krajin, do každé krajiny písmem jejím, každému národu jazykem jeho, též i Židum písmem jejich a jazykem jejich.
\par 10 A když napsal jménem krále Asvera a zapecetil prstenem královským, rozeslal listy po poslích, kteríž jezdívali na koních rychlých a mezcích mladých:
\par 11 Že j\par povolil král Židum, kterížby v kterémkoli meste byli, aby shromáždíce se, zastávali života svého, a aby hubili, mordovali a plénili všecka vojska národu i krajiny, útok cinících na ne, na deti jejich i ženy jejich, a koristi jejich aby rozbitovali.
\par 12 Jednoho a téhož dne ve všech krajinách krále Asvera, totiž trináctého, mesíce dvanáctého, jenž j\par mesíc Adar.
\par 13 Summa toho psání: Aby vyhlášeno bylo v jedné každé krajine, a oznámeno všechnem národum, aby Židé byli hotovi ke dni tomu ku pomste nad neprátely svými.
\par 14 Tedy poslové, kteríž jezdívali na koních prudkých a na mezcích, vyjeli snažným a rychlým behem s porucením královským, a vyhlášeno j\par to v Susan, meste královském.
\par 15 Mardocheus pak vycházel od oblíceje královského v rouše královském z postavce modrého a bílého, v korune zlaté veliké a v plášti kmentovém a šarlatovém, a mesto Susan plésalo a veselilo se.
\par 16 Nebo Židum vzešlo svetlo a radost, i veselé a sláva.
\par 17 Ano i v každé krajine i v každém meste, na kteréžkoli místo porucení královské a výpoved jeho došla, veselé a radost meli Židé, hody a dobrou vuli, a mnozí z národu jiných pristupovali k Židum; nebo pripadl na ne strach Židovský.

\chapter{9}

\par 1 Potom dvanáctého mesíce, (jenž j\par mesíc Adar), trináctého dne téhož mesíce, když prišel cas porucení královského a výpovedi jeho, aby se vyplnila, v ten den, v kterýž se nadáli neprátelé Židovští, že budou panovati nad nimi, stalo se na odpor, že panovali Židé nad temi, kteríž je v nenávisti meli.
\par 2 Nebo se byli shromáždili Židé v mestech svých, po všech krajinách krále Asvera, aby vztáhli ruku na ty, kteríž hledali jejich zlého. A žádný pred nimi neostál, nebo pripadl strach jejich na všecky národy.
\par 3 A všickni hejtmané krajin, i knížata a vývodové, i správcové díla královského v poctivosti meli Židy; nebo strach Mardocheuv na ne pripadl.
\par 4 Byl zajisté Mardocheus veliký v dome královském, a rozcházela se pov\par o nem po všech krajinách; nebo muž ten Mardocheus vždy více rostl.
\par 5 A tak zbili Židé všecky neprátely své, mecem hubíce, mordujíce a vyhlazujíce je, a nakládajíce s temi, kteríž je v nenávisti meli, podlé líbosti své.
\par 6 Ano i v Susan meste královském zmordovali Židé a vyhladili pet set mužu.
\par 7 A Parsandata, Dalfona i Aspata,
\par 8 A Porata, Adalia i Aridata,
\par 9 I Parimasta, Arisai i Aridai a Vajezata,
\par 10 Deset synu Amana syna Hammedatova, neprítele Židovského, zmordovali. Ale k loupeži nevztáhli ruky své.
\par 11 Dne toho, když se donesl krále pocet zmordovaných v Susan meste královském,
\par 12 Rekl král Estere královne: V Susan meste královském zmordovali Židé a vyhladili pet set mužu, a deset synu Amanových. Co pak ucinili v jiných krajinách královských? Již tedy jaká j\par žádost tvá? A dánot bude. Aneb která prosba tvá ješte? A stanet se.
\par 13 Odpovedela Ester: jestliže se králi za dobré vidí, necht j\par dopušteno ješte zítra Židum, kteríž jsou v Susan, uciniti podlé výpovedi dnešní, a deset synu Amanových zvešeti na šibenici.
\par 14 I prikázal král, aby se tak stalo. Tedy vyhlášena j\par výpoved v Susan, a tak deset synu Amanových zvešeli.
\par 15 A shromáždivše se Židé, kteríž byli v Susan, také i ctrnáctého dne mesíce Adar, zmordovali v Susan tri sta mužu. Ale k loupeži nevztáhli ruky své.
\par 16 Jiní také Židé, kteríž byli v krajinách královských, shromáždivše se, a zastávajíce života svého, tak odpocinuli od neprátel svých. Zmordovali pak tech, jenž je v nenávisti meli, sedmdesáte pet tisíc. Ale k loupeži nevztáhli ruky své.
\par 17 Stalo se to dne trináctého mesíce Adar. I odpocinuli ctrnáctého dne téhož mesíce, a ucinili sobe v ten den hody a veselé.
\par 18 Ale Židé, kteríž byli v Susan, shromáždili se trináctého dne téhož mesíce a též ctrnáctého, a odpocinuli patnáctého, a ucinili sobe na ten den hody a veselé.
\par 19 Protož Židé, kteríž bydlí ve vsech a v mesteckách nehrazených, svetí den ctrnáctý mesíce Adar, majíce veselé, hody a dobrou vuli, a posílajíce cástky pokrmu jedni druhým.
\par 20 Nebo rozepsal Mardocheus ty veci, a rozeslal listy ke všem Židum, kteríž byli ve všech krajinách krále Asvera, blízkým i dalekým,
\par 21 Ustavuje jim, aby slavili den ctrnáctý mesíce Adar, a den patnáctý téhož mesíce každého roku,
\par 22 Podlé dnu tech, v nichž odpocinuli Židé od neprátel svých, a mesíce toho, kterýž se jim obrátil z zámutku v radost, a z kvílení v dobrou vuli, aby ty dni slavili, hodujíce a veselíce se, a posílajíce cástky pokrmu jeden druhému, i dary chudým.
\par 23 I prijali to všickni Židé, že budou ciniti to, což zacali, a což jim psal Mardocheus:
\par 24 Jak Aman syn Hammedatuv Agagský, protivník všech Židu, ukládal o Židech, aby je vyhubil a uvrhl pur, totiž los, k setrení a zahlazení jejich.
\par 25 Ale jak ona vešla pred oblícej krále, porucil král v listech, aby obráceni byli úkladové jeho nešlechetní, kteréž vymyslil proti Židum, na hlavu jeho, a aby ho obesili i syny jeho na šibenici.
\par 26 I nazvali ty dny Purim, totiž losu, od jména toho pur. A tak z príciny všech slov listu toho, a což videli pri tom, i což prišlo k nim,
\par 27 Ustavili a prijali Židé na sebe i na síme své, i na všecky pripojené k sobe, aby to nepominulo, že budou slaviti ty dva dni podlé vypsání jejich, a podlé urcitého casu jejich každého roku.
\par 28 A že ti dnové budou pametní a slavní v každém veku, rodine, krajine a meste; k tomu, že ti dnové Purim nepominou z prostredku Židu, a památka jejich neprestane u potomku jejich.
\par 29 Psala také Ester královna, dcera Abichailova, i Mardocheus Žid pro vetší upevnení psání z strany dnu Purim po druhé.
\par 30 A on rozeslal to psání ke všem Židum, do sta dvadcíti sedmi krajin království Asverova, vzkazuje jim pozdravení,
\par 31 Aby tuze drželi dny ty Purim v urcité casy jich, jakž je narídil jim Mardocheus Žid a Ester královna, a jakž to prijali na sebe a na síme své, na pamet postu a kriku jejich.
\par 32 A tak výpoved Estery potvrdila ustanovení dnu Purim, což zapsáno j\par v knize této.

\chapter{10}

\par 1 Potom uložil král Asverus dan na zemi i na ostrovy morské.
\par 2 Všickni pak cinové síly jeho a moci jeho, i vypsání dustojnosti Mardocheovy, kterouž ho zvelebil král, to vše zapsáno j\par v knize o králích Médských a Perských.
\par 3 Nebo Mardocheus Žid byl druhý po králi Asverovi, a veliký u Židu, i vzácný u všeho množství bratrí svých, pecuje o dobré lidu svého, a zpusobuje pokoj všemu semeni svému.

\end{document}