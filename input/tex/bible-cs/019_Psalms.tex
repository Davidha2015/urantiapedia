\begin{document}

\title{Žalmy}

\chapter{1}

\par 1 Blahoslavený ten muž, kterýž nechodí po rade bezbožných, a na ceste hríšníku nestojí, a na stolici posmevacu nesedá.
\par 2 Ale v zákone Hospodinove jest líbost jeho, a v zákone jeho premýšlí dnem i nocí.
\par 3 Nebo bude jako strom štípený pri tekutých vodách, kterýž ovoce své vydává casem svým, jehožto list nevadne, a cožkoli ciniti bude, štastne mu se povede.
\par 4 Ne takt budou bezbožní, ale jako plevy, kteréž rozmítá vítr.
\par 5 A protož neostojí bezbožní na soudu, ani hríšníci v shromáždení spravedlivých.
\par 6 Nebot zná Hospodin cestu spravedlivých, ale cesta bezbožných zahyne.

\chapter{2}

\par 1 Proc se bourí národové, a lidé daremné veci premyšlují?
\par 2 Sstupují se králové zemští, a knížata se spolu radí proti Hospodinu, a proti pomazanému jeho,
\par 3 Ríkajíce: Roztrhejme svazky jejich, a zavrzme od sebe provazy jejich.
\par 4 Ale ten, jenž prebývá v nebesích, smeje se, Pán posmívá se jim.
\par 5 Tehdáž mluviti bude k nim v hneve svém, a v prchlivosti své predesí je, rka:
\par 6 Ját jsem ustanovil krále svého nad Sionem, horou svatou mou.
\par 7 Vypravovati budu úsudek. Hospodin rekl ke mne: Syn muj ty jsi, já dnes zplodil jsem te.
\par 8 Požádej mne, a dámt národy, dedictví tvé, a konciny zeme, vládarství tvé.
\par 9 Roztluceš je prutem železným, a jako nádobu hrncírskou roztríštíš je.
\par 10 A protož, králové, nyní srozumejte, vyucujte se, soudcové zemští.
\par 11 Služte Hospodinu v bázni, a veselte se s tresením.
\par 12 Líbejte syna, aby se nerozhneval, a zhynuli byste na ceste, jakž by se jen málo zapálil hnev jeho. Blahoslavení jsou všickni, kteríž doufají v neho.

\chapter{3}

\par 1 Žalm Daviduv, když utíkal pred Absolonem synem svým.
\par 2 Hospodine, jakt jsou mnozí neprátelé moji! Mnozí povstávají proti mne.
\par 3 Mnozí mluví o duši mé: Nemát tento žádné pomoci v Bohu.
\par 4 Ale ty, Hospodine, jsi štítem vukol mne, slávou mou, a kterýž povyšuješ hlavy mé.
\par 5 Hlasem svým volal jsem k Hospodinu, a vyslyšel mne s hory svaté své. Sélah.
\par 6 Já jsem lehl, a spal jsem, i zas procítil; nebo mne zdržoval Hospodin.
\par 7 Nebudut se báti mnoha tisícu lidí, kteríž se vukol kladou proti mne.
\par 8 Povstaniž, Hospodine, zachovej mne, Bože muj, kterýž jsi zbil všech neprátel mých líce, a zuby bezbožníku zvyrážel.
\par 9 Tvét, ó Hospodine, jest spasení, a nad lidem tvým požehnání tvé. Sélah.

\chapter{4}

\par 1 Prednímu zpeváku na neginot, žalm Daviduv.
\par 2 Když volám, vyslyš mne, Bože spravedlnosti mé. Ty jsi mi v úzkosti prostranství zpusoboval, smiluj se nade mnou, a vyslyš modlitbu mou.
\par 3 Synové lidští, dokudž sláva má v potupe bude? Dlouho-liž marnost milovati a lži hledati budete? Sélah.
\par 4 Vezte, žet jest oddelil Hospodin sobe milého. Vyslyšít mne Hospodin, když k nemu volati budu.
\par 5 Uleknetež se a nehrešte, premyšlujte o tom v srdci svém, na ložci svém, a umlknete. Sélah.
\par 6 Obetujte obeti spravedlnosti, a doufejte v Hospodina.
\par 7 Mnozí ríkají: Ó bychom videti mohli dobré veci. Hospodine, pozdvihni nad námi jasného oblíceje svého,
\par 8 I zpusobíš radost v srdci mém vetší, než oni mívají, když obilé a víno jejich se obrodí.
\par 9 Ját u pokoji i lehnu i spáti budu; nebo ty, Hospodine, sám zpusobíš mi bydlení bezpecné.

\chapter{5}

\par 1 Prednímu zpeváku na nechilot, žalm Daviduv.
\par 2 Slova má slyš, Hospodine, porozumej tužebnému úpení mému.
\par 3 Pozoruj hlasu volání mého, králi muj a Bože muj; nebo se tobe modlím.
\par 4 Hospodine, v jitre vyslyšíš hlas muj, v jitre predložím tobe žádost, a šetriti budu.
\par 5 Nebo ty, ó Bože silný, neoblibuješ bezbožnosti, nemá místa u tebe nešlechetník.
\par 6 Neostojí blázniví pred ocima tvýma, v nenávisti máš všecky cinitele nepravosti.
\par 7 Zatratíš mluvící lež. Cloveka ukrutného a lstivého v ohavnosti má Hospodin.
\par 8 Já pak ve množství milosrdenství tvého vejdu do domu tvého, klaneti se budu k svatému chrámu tvému v bázni tvé.
\par 9 Hospodine, proved mne v spravedlnosti své, pro ty, jenž mne strehou; spravuj prede mnou cestu svou.
\par 10 Nebot není v ústech jejich žádné uprímnosti, vnitrnosti jejich plné nešlechetnosti, hrob otevrený hrdlo jejich, jazykem svým lahodne mluví.
\par 11 Zkaz je, ó Bože, nechat padnou od rad svých; pro množství nešlechetností jejich rozptyl je, ponevadž odporní jsou tobe.
\par 12 A at se všickni v te doufající radují, na veky at plésají, když je zastírati budeš; at se v tobe veselí, kterížkoli milují jméno tvé.
\par 13 Nebo ty, Hospodine, požehnáš spravedlivému, a jako štítem prívetivostí svou vukol zastreš jej.

\chapter{6}

\par 1 Prednímu kantoru na neginot, k nízkému zpevu, žalm Daviduv.
\par 2 Hospodine, netresci mne v hneve svém, ani v prchlivosti své kárej mne.
\par 3 Smiluj se nade mnou, Hospodine, nebot jsem zemdlený; uzdrav mne, Hospodine, nebo ztrnuly kosti mé.
\par 4 Ano i duše má predešena jest náramne, ty pak, Hospodine, až dokavad?
\par 5 Navratiž se, Hospodine, a vytrhni duši mou; spomoz mi pro milosrdenství své.
\par 6 Nebo mrtví nezpomínají na tebe, a v hrobe kdo te bude oslavovati?
\par 7 Ustávám v úpení svém, ložce své každé noci svlažuji, slzami svými postel svou smácím.
\par 8 Sškvrkla se zámutkem tvár má, sstarala se prícinou všech neprátel mých.
\par 9 Odstuptež ode mne všickni cinitelé nepravosti; nebot jest vyslyšel Hospodin hlas pláce mého.
\par 10 Vyslyšel Hospodin pokornou modlitbu mou, Hospodin modlitbu mou prijal.
\par 11 Nechažt se zastydí a predesí zrejme všickni neprátelé moji, nechažt jsou zpet obráceni a rychle zahanbeni.

\chapter{7}

\par 1 Osvedcení neviny Davidovy, o cemž zpíval Hospodinu z príciny slov Chusi syna Jeminova.
\par 2 Hospodine Bože muj, v tobet doufám, vysvobod mne ode všech protivníku mých, a vytrhni mne,
\par 3 Aby neuchvátil jako lev duše mé, a neroztrhal, když by nebyl, kdo by vysvobodil.
\par 4 Hospodine Bože muj, ucinil-li jsem to, jest-li nepravost pri mne,
\par 5 Cinil-li jsem zle tomu, kdož se ke mne pokojne choval, (nýbrž spomáhal jsem protivícímu se mi bez príciny),
\par 6 Nechat stihá neprítel duši mou, i popadne, a pošlapá na zemi život muj, a slávu mou v prach uvede. Sélah.
\par 7 Povstan, Hospodine, v hneve svém, vyvyš se proti vzteklostem mých neprátel, a procit ke mne, nebo jsi soud narídil.
\par 8 I shrne se k tobe shromáždení lidí; pro ne tedy u výsost navrat se zase.
\par 9 Hospodin souditi bude lidi. Sudiž mne, Hospodine, podlé spravedlnosti mé, a podlé nevinnosti mé, kteráž pri mne jest.
\par 10 Ó by již k skoncení prišla nešlechetnost bezbožných, spravedlivého pak abys utvrdil ty, kterýž zkušuješ srdce a ledví, Bože spravedlivý.
\par 11 Buh jest štít muj, kterýž spaseny ciní uprímé srdcem.
\par 12 Buh jest soudce spravedlivý, Buh silný hnevá se na bezbožného každý den.
\par 13 Neobrátí-li se, naostrít mec svuj; lucište své natáhl, a nameril je.
\par 14 Pripravil sobe i zbroj smrtelnou, strely své proti škudcím pristrojil.
\par 15 Aj, rodí nepravost; nebo pocav težkou bolest, urodí lež.
\par 16 Jámu kopal, i vykopal ji, ale padne do dolu, kterýž pristrojil.
\par 17 Obrátít se usilování jeho na hlavu jeho, a na vrch hlavy jeho nepravost jeho sstoupí.
\par 18 Slaviti budu Hospodina podlé spravedlnosti jeho, a žalmy zpívati jménu Hospodina nejvyššího.

\chapter{8}

\par 1 Prednímu zpeváku na gittejský nástroj, žalm Daviduv.
\par 2 Hospodine Pane náš, jak dustojné jest jméno tvé na vší zemi! Nebo jsi vyvýšil slávu svou nad nebesa.
\par 3 Z úst nemluvnátek a tech, jenž prsí požívají, mocne dokazuješ síly z príciny svých neprátel, abys prítrž ucinil protivníku a vymstívajícímu se.
\par 4 Když spatruji nebesa tvá, dílo prstu tvých, mesíc a hvezdy, kteréž jsi tak upevnil, ríkám:
\par 5 Co jest clovek, že jsi nan pametliv, a syn cloveka, že jej navštevuješ?
\par 6 Nebo ucinil jsi ho málo menšího andelu, slávou a ctí korunoval jsi jej.
\par 7 Pánem jsi ho ucinil nad dílem rukou svých, všecko jsi podložil pod nohy jeho:
\par 8 Ovce i voly všecky, také i zver polní,
\par 9 Ptactvo nebeské, i ryby morské, a cožkoli chodí stezkami morskými.
\par 10 Hospodine Pane náš, jak dustojné jest jméno tvé na vší zemi!

\chapter{9}

\par 1 Prednímu zpeváku na al mutlabben, žalm Daviduv.
\par 2 Oslavovati te budu, Hospodine, celým srdcem svým, vypravovati budu všecky divné skutky tvé.
\par 3 Radovati a veseliti se budu v tobe, žalmy zpívati budu jménu tvému, ó Nejvyšší,
\par 4 Proto že neprátelé moji jsou nazpet obráceni, že klesli, a zahynuli od tvári tvé.
\par 5 Nebo jsi vyvedl soud muj a pri mou, posadils se na stolici soudce spravedlivý.
\par 6 Ohromil jsi národy, zatratils bezbožníka, jméno jejich vyhladil jsi na vecné veky.
\par 7 Ó nepríteli, již-li jsou dokonány zhouby tvé na veky? Již-li jsi mesta podvrátil? Zahynula památka jejich s nimi.
\par 8 Ale Hospodin na veky kraluje, pripravil k soudu trun svuj.
\par 9 Ont soudí sám okršlek v spravedlnosti, a výpoved ciní národum v pravosti.
\par 10 Hospodin zajisté jest útocište chudého, útocište v cas ssoužení.
\par 11 I budou v tebe doufati, kteríž znají jméno tvé; nebo neopouštíš hledajících te, Hospodine.
\par 12 Žalmy zpívejte Hospodinu, prebývajícímu na Sionu, zvestujte mezi národy skutky jeho.
\par 13 Nebo on vyhledává krve, rozpomíná se na ni, aniž se zapomíná na krik utištených.
\par 14 Smiluj se nade mnou, Hospodine, viz ssoužení mé od tech, kteríž mne nenávidí, ty, kterýž mne vyzdvihuješ z bran smrti,
\par 15 Abych vypravoval všecky chvály tvé v branách dcery Sionské, a veselil se v spasení tvém.
\par 16 Pohríženit jsou národové v jáme, kterouž udelali; v osídle, kteréž polékli, uvázla noha jejich.
\par 17 Prišelt jest v známost Hospodin, pomstu uciniv; v díle rukou svých zapletl se bezbožník. Higgaion Sélah.
\par 18 Obráceni budte bezbožníci do pekla, všickni národové, kteríž se zapomínají nad Bohem.
\par 19 Nebot nebude dán chudý u vecné zapomenutí; ocekávání ssoužených nezahynet na veky.
\par 20 Povstaniž, Hospodine, at se nesilí clovek; národové budte souzeni pred tebou.
\par 21 Pust na ne strach, ó Hospodine, at tomu porozumejí národové, že smrtelní jsou. Sélah.

\chapter{10}

\par 1 Proc, ó Hospodine, stojíš zdaleka, a skrýváš se v cas ssoužení?
\par 2 Z pychu bezbožník protivenství ciní chudému. Ó by jati byli v zlých radách, kteréž vymýšlejí.
\par 3 Nebot se honosí bezbožník v líbostech života svého, a lakomý sobe pochlebuje, a Hospodina popouzí.
\par 4 Bezbožník pro pýchu, kterouž na sobe prokazuje, nedbá na nic; všecka myšlení jeho jsou, že není Boha.
\par 5 Dobre mu se darí na cestách jeho všelikého casu, soudové tvoji vzdáleni jsou od neho, i na všecky neprátely své fouká,
\par 6 Ríkaje v srdci svém: Nepohnut se od národu až do pronárodu, nebo nebojím se zlého.
\par 7 Ústa jeho plná jsou zlorecenství, i chytrosti a lsti; pod jazykem jeho trápení a starost.
\par 8 Sedí v zálohách ve vsech a v skrýších, aby zamordoval nevinného; ocima svýma po chudém špehuje.
\par 9 Cíhá v skryte jako lev v jeskyni své, cíhá, aby pochytil chudého, uchvacujet jej, a táhne pod sítku svou.
\par 10 Pripadá a stuluje se, dokudž by nevpadlo v silné pazoury jeho shromáždení chudých.
\par 11 Ríká v srdci svém: Zapomenult jest Buh silný, skryl tvár svou, nepohledít na veky.
\par 12 Povstaniž, Hospodine Bože silný, vznes ruku svou, nezapomínejž se nad chudými.
\par 13 Proc má bezbožník Boha popouzeti, ríkaje v srdci svém, že toho vyhledávati nebudeš?
\par 14 Díváš se do casu, nebo ty nátisk a bolest spatruješ, abys jim odplatil rukou svou; na tebet se spouští chudý, sirotku ty jsi spomocník.
\par 15 Potri ráme bezbožného, a vyhledej nepravosti zlostného, tak aby neostál.
\par 16 Hospodin jest králem vecných veku, národové z zeme své hynou.
\par 17 Žádost ponížených vyslýcháš, Hospodine, utvrzuješ srdce jejich, ucha svého k nim naklonuješ,
\par 18 Abys soud cinil sirotku a ssouženému, tak aby jich nessužoval více clovek bídný a zemský.

\chapter{11}

\par 1 Prednímu zpeváku, žalm Daviduv.
\par 2 Hospodina doufám, kterakž tedy ríkáte duši mé: Ulet s hory své jako ptáce?
\par 3 Nebo aj, bezbožníci napínají lucište, prikládají šípy své na tetivo, aby stríleli skryte na uprímé srdcem.
\par 4 Ale temi usilováními zkaženi budou; nebo spravedlivý co ucinil?
\par 5 Hospodin jest v chráme svatém svém, trun Hospodinuv v nebi jest; oci jeho hledí, vícka jeho zkušují synu lidských.
\par 6 Hospodin zkušuje spravedlivého, bezbožníka pak a milujícího nepravost nenávidí duše jeho.
\par 7 Dštíti bude na bezbožníky uhlím reravým, ohnem a sirou, a duch vichrice bude cástka kalicha jejich.
\par 8 Nebo Hospodin spravedlivý jest, spravedlnost miluje, na uprímého oci jeho patrí.

\chapter{12}

\par 1 Prednímu kantoru k nízkému zpevu, žalm Daviduv.
\par 2 Spomoz, ó Hospodine; nebo se již nenalézá milosrdného, a vyhynuli verní z synu lidských.
\par 3 Lež mluví jeden každý s bližním svým, rty úlisnými z srdce dvojitého reci vynášejí.
\par 4 Ó by vyplénil Hospodin všeliké rty úlisné, a jazyk velikomluvný,
\par 5 Kteríž ríkají: Jazykem svým premužeme, mámet ústa svá s sebou, kdo jest pánem naším?
\par 6 Pro zhoubu chudých, pro úpení nuzných jižt povstanu, praví Hospodin, v bezpecnosti postavím toho, na nejž poleceno bylo.
\par 7 Výmluvnosti Hospodinovy jsou výmluvnosti cisté, jako stríbro v hlinené peci prehnané a sedmkrát zprubované.
\par 8 Ty, Hospodine, jim spomáhati budeš, a ostríhati každého od národu tohoto až na veky.
\par 9 Vukol a vukol bezbožní se protulují, když takoví nicemní vyvýšeni bývají mezi syny lidskými.

\chapter{13}

\par 1 Prednímu zpeváku, žalm Daviduv.
\par 2 Až dokud, Hospodine? Což se na veky zapomeneš na mne? Dokudž tvár svou skrývati budeš prede mnou?
\par 3 Dokud rady vyhledávati budu v mysli své, a den ode dne svírati se v srdci svém? Až dokud se zpínati bude neprítel muj nade mnou?
\par 4 Vzhlédni, vyslyš mne, Hospodine Bože muj, osvet oci mé, abych neusnul snem smrti,
\par 5 A aby nerekl neprítel muj: Svítezil jsem nad ním, a neprátelé moji aby neplésali, jestliže bych se poklesl.
\par 6 Ját zajisté v milosrdenství tvém doufám, plésati bude srdce mé v spasení tvém;
\par 6 Ját zajisté v milosrdenství tvém doufám, plésati bude srdce mé v spasení tvém;

\chapter{14}

\par 1 Prednímu zpeváku, písen Davidova. Ríká blázen v srdci svém: Není Boha. Porušeni jsou, a ohavní v snažnostech; není, kdo by cinil dobré.
\par 2 Hospodin s nebe popatril na syny lidské, aby videl, byl-li by kdo rozumný a hledající Boha.
\par 3 Všickni se odvrátili, naporád neužitecní ucineni jsou; není, kdo by cinil dobré, není ani jednoho.
\par 4 Zdaliž nevedí všickni cinitelé nepravosti, že zžírají lid muj, jako by chléb jedli? Hospodina pak nevzývají.
\par 5 Tehdáž se náramne strašiti budou; nebo Buh jest v rodine spravedlivého.
\par 6 Radu chudého potupujete, ale Hospodin jest nadeje jeho.
\par 7 Ó by z Siona dáno bylo spasení Izraelovi. Když Hospodin zase privede zajaté lidu svého, plésati bude Jákob, a veseliti se Izrael.

\chapter{15}

\par 1 Žalm Daviduv. Hospodine, kdo bude prebývati v stánku tvém? Kdo bydliti bude na hore svaté tvé?
\par 2 Ten, kdož chodí v uprímnosti, a ciní spravedlnost, a mluví pravdu z srdce svého.
\par 3 Kdož neutrhá jazykem svým, bližnímu svému neciní zlého, a potupy neuvodí na bližního svého.
\par 4 Ten, pred jehož ocima v nevážnosti jest zavržený, v poctivosti pak bojící se Hospodina; a prisáhl-li by i se škodou, však toho nemení.
\par 5 Kdož penez svých nedává na lichvu, a daru proti nevinnému nebére. Kdož tyto veci ciní, nepohnet se na veky.

\chapter{16}

\par 1 Zlatý zpev Daviduv. Ostríhej mne, Bože silný, nebot v tebe doufám.
\par 2 Rciž, duše má, Hospodinu: Ty jsi Pán muj, dobrota má nic tobe neprospeje.
\par 3 Ale svatým, kteríž jsou na zemi, a znamenitým, v nichž všecka líbost má.
\par 4 Rozmnožují bolesti své ti, kteríž k cizímu bohu chvátají; neokusím obetí jejich ze krve, aniž vezmu jména jejich ve rty své.
\par 5 Hospodin jest cástka dílu mého a kalicha mého; ty zdržuješ los muj.
\par 6 Provazcové padli mi na místech veselých, a dedictví rozkošné dostalo se mi.
\par 7 Dobroreciti budu Hospodinu, kterýž mi radí; nebo i v noci vyucují mne ledví má.
\par 8 Predstavuji Hospodina pred oblícej svuj vždycky, a kdyžt jest mi po pravici, nikoli se nepohnu.
\par 9 Z té príciny rozveselilo se srdce mé, a zplésala sláva má, ano i telo mé v bezpecnosti prebývati bude.
\par 10 Nebo nenecháš duše mé v pekle, aniž dopustíš svatému svému videti porušení.
\par 11 Známou uciníš mi cestu života; sytost hojného veselí jest pred oblícejem tvým, a dokonalé utešení po pravici tvé až na veky.

\chapter{17}

\par 1 Modlitba Davidova. Vyslyš, Hospodine, spravedlnost, a pozoruj volání mého; naklon uší k modlitbe mé, kteráž jest beze vší rtu ošemetnosti.
\par 2 Od tvári tvé vyjdiž soud muj, oci tvé nechat patrí na uprímnost.
\par 3 Zkusils srdce mého, navštívils je v noci; ohnem jsi mne zpruboval, aniž jsi co shledal; to, což myslím, nepredstihá úst mých.
\par 4 Z strany pak skutku lidských, já podlé slova rtu tvých vystríhal jsem se stezky zhoubce.
\par 5 Zdržuj kroky mé na cestách svých, aby se neuchylovaly nohy mé.
\par 6 Já volám k tobe, nebo vyslýcháš mne, ó Bože silný; naklon ke mne ucha svého, a slyš rec mou.
\par 7 Prokaž milosrdenství svá, nadeji majících ochránce pred temi, kteríž povstávají proti pravici tvé.
\par 8 Ostríhej mne jako zrítelnice oka, v stínu krídel svých skrej mne,
\par 9 Od tvári bezbožných tech, kteríž mne hubí, od neprátel mých úhlavních obklicujících mne,
\par 10 Kteríž tukem svým zarostli, mluví pyšne ústy svými.
\par 11 Jižt i kroky naše predstihají, oci své obrácené mají, aby nás porazili na zem.
\par 12 Každý z nich podoben jest lvu žádostivému loupeže, a lvíceti sedícímu v skrýši.
\par 13 Povstaniž, Hospodine, predejdi tvári jeho, sehni jej, a vytrhni duši mou od bezbožníka mecem svým,
\par 14 Rukou svou od lidí, ó Hospodine, od lidí svetských, jichžto oddíl jest v tomto živote, a jejichž bricho ty z špižírny své naplnuješ. Címž i synové jejich nasyceni bývají, a ostatku zanechávají malickým svým.
\par 15 Já pak v spravedlnosti spatrovati budu tvár tvou; nasycen budu obrazem tvým, když procítím.

\chapter{18}

\par 1 Prednímu zpeváku, služebníka Hospodinova Davida, kterýž mluvil Hospodinu slova písne této v ten den, v nemž ho vysvobodil Hospodin z ruky všech neprátel jeho i z ruky Saulovy, a rekl:
\par 2 Z vnitrnosti srdce miluji te, Hospodine, sílo má.
\par 3 Hospodin skála má a hrad muj, i vysvoboditel muj, Buh silný muj, skála má, v nemž nadeji skládám, štít muj a roh spasení mého, mé útocište.
\par 4 Chvály hodného vzýval jsem Hospodina, a od neprátel svých byl jsem vyprošten.
\par 5 Obklícilyt mne byly bolesti smrti, a proudové nešlechetných predesili mne.
\par 6 Bylyt jsou mne obklícily bolesti hrobu, osídla smrti zachvátila mne.
\par 7 V úzkosti své vzýval jsem Hospodina, a k Bohu svému volal jsem, i vyslyšel z chrámu svého hlas muj, a volání mé pred oblícejem jeho prišlo v uši jeho.
\par 8 Tehdy pohnula se a zatrásla zeme, základové hor pohnuli se, a trásli se pro rozhnevání jeho.
\par 9 Dým vystupoval z chrípí jeho, a ohen zžírající z úst jeho, od nehož se uhlí rozpálilo.
\par 10 Nakloniv nebes, sstoupil, a mrákota byla pod nohami jeho.
\par 11 A sede na cherubínu, letel, letel na perí vetrovém.
\par 12 Udelal sobe z temností skrýši,vukol sebe stánek svuj z temných vod a hustých oblaku.
\par 13 Od blesku pred ním oblakové jeho rozehnáni jsou, krupobití i uhlí reravé.
\par 14 I hrímal na nebi Hospodin, a Nejvyšší vydal zvuk svuj, i krupobití a uhlí reravé.
\par 15 Vystrelil i strely své, a rozptýlil je, a blýskáním castým porazil je.
\par 16 I ukázaly se hlubiny vod, a odkryti jsou základové okršlku pro žehrání tvé, Hospodine, pro dchnutí vetru chrípí tvých.
\par 17 Poslav s výsosti, uchopil mne, vytáhl mne z velikých vod.
\par 18 Vytrhl mne od neprítele mého silného, a od tech, kteríž mne nenávideli, ackoli silnejší mne byli.
\par 19 Predstihlit jsou mne byli v den trápení mého, ale Hospodin byl mi podpora.
\par 20 Kterýž vyvedl mne na prostranno, vytrhl mne, nebo mne sobe oblíbil.
\par 21 Odplatil mi Hospodin podlé spravedlnosti mé, podlé cistoty rukou mých nahradil mi.
\par 22 Nebo jsem ostríhal cest Hospodinových, aniž jsem se bezbožne strhl Boha svého.
\par 23 Všickni zajisté soudové jeho jsou prede mnou, a ustanovení jeho neodložil jsem od sebe.
\par 24 Nýbrž upríme choval jsem se k nemu, a vystríhal jsem se nepravosti své.
\par 25 Protož odplatil mi Hospodin podlé spravedlnosti mé, podlé cistoty rukou mých, kteráž jest pred ocima jeho.
\par 26 Ty, Pane, s milosrdným milosrdne nakládáš, a k cloveku uprímému upríme se máš.
\par 27 K sprostnému sprostne se ukazuješ, a s prevráceným prevrácene zacházíš.
\par 28 Lid pak ssoužený vysvobozuješ, a oci vysoké snižuješ.
\par 29 Ty zajisté rozsvecuješ svíci mou; Hospodin Buh muj osvecuje temnosti mé.
\par 30 Nebo v tobe probehl jsem vojsko, a v Bohu svém preskocil jsem i zed.
\par 31 Boha tohoto silného cesta jest dokonalá, výmluvnosti Hospodinovy jsou precištené, ont jest štít všech, kteríž v neho doufají.
\par 32 Nebo kdo jest Bohem krome Hospodina, a kdo skalou krome Boha našeho?
\par 33 Ten Buh silný prepasuje mne udatností, a ciní dokonalou cestu mou.
\par 34 Ciní nohy mé jako laní, a na vysokostech mých postavuje mne.
\par 35 Ucí ruce mé boji, tak že lámi i lucište ocelivé rukama svýma.
\par 36 Tys mi také dodal štítu spasení svého, a pravice tvá podpírala mne, a dobrotivost tvá mne zvelebila.
\par 37 Rozšíril jsi krokum mým místo pode mnou, a nepodvrtly se nohy mé.
\par 38 Honil jsem neprátely své, a postihl jsem je, aniž jsem se navrátil, až jsem je vyhubil.
\par 39 Tak jsem je zranil, že nemohli povstati, padše pod nohy mé.
\par 40 Ty jsi zajisté mne prepásal udatností k boji, povstávající proti mne sehnul jsi pode mne.
\par 41 Nýbrž dals mi šíji neprátel mých, abych ty, kteríž mne nenávideli, vyplénil.
\par 42 Volalit jsou, ale nebylo spomocníka k Hospodinu, ale nevyslyšel jich.
\par 43 I potrel jsem je jako prach u povetrí, jako bláto na ulicích rozšlapal jsem je.
\par 44 Ty jsi mne vyprostil z ruznic lidu, a postavils mne v hlavu národum; lid, kteréhož jsem neznal, sloužil mi.
\par 45 Jakž jen zaslechli, uposlechli mne, cizozemci lhali mi.
\par 46 Cizozemci svadli a trásli se v ohradách svých.
\par 47 Živt jest Hospodin, a požehnaná skála má; protož bud vyvyšován Buh spasení mého,
\par 48 Buh silný, kterýž mi pomsty poroucí, a podmanuje mi lidi.
\par 49 Ty jsi vysvoboditel muj z moci neprátel mých, také i nad povstávající proti mne vyvýšils mne, a od cloveka ukrutného vyprostils mne.
\par 50 A protož chváliti te budu mezi národy, ó Hospodine, a jménu tvému žalmy prozpevovati,
\par 51 Kterýž tak dustojne vysvobozuješ krále svého, a ciníš milosrdenství pomazanému svému Davidovi, i semeni jeho až na veky.

\chapter{19}

\par 1 Prednímu z kantoru, žalm Daviduv.
\par 2 Nebesa vypravují slávu Boha silného, a dílo rukou jeho obloha zvestuje.
\par 3 Den po dni vynáší rec, a noc po noci ukazuje umení.
\par 4 Nenít reci ani slov, kdež by nemohl slyšán býti hlas jejich.
\par 5 Po vší zemi rozchází se zpráva jejich, a až do koncin okršlku slova jich, slunci pak rozbil stánek na nich.
\par 6 Kteréž jako ženich vychází z pokoje svého, veselí se jako udatný rek, cestou bežeti maje.
\par 7 Od koncin nebes východ jeho, a obcházení jeho až zase do koncin jejich, a niceho není, což by se ukryti mohlo pred horkostí jeho.
\par 8 Zákon Hospodinuv jest dokonalý, ocerstvující duši, Hospodinovo svedectví pravé, moudrost dávající neumelým.
\par 9 Rozkazové Hospodinovi prímí, obveselující srdce, prikázaní Hospodinovo cisté, osvecující oci.
\par 10 Bázen Hospodinova cistá, zustávající na veky, soudové Hospodinovi praví, a k tomu i spravedliví.
\par 11 Mnohem žádostivejší jsou než zlato, a než mnoho ryzího zlata, sladší než med a stred z plástu.
\par 12 Služebník tvuj zajisté jimi osvecován bývá, a kdož jich ostríhá, užitek hojný má.
\par 13 Ale poblouzením kdo vyrozumí? Protož i od tajných ocist mne.
\par 14 I od zúmyslných zdržuj služebníka svého, aby nade mnou nepanovali, a tehdyt dokonalý budu, a ocištený od prestoupení velikého.
\par 15 Ó at jsou slova úst mých tobe príjemná, i premyšlování srdce mého pred tebou, Hospodine skálo má, a vykupiteli muj.

\chapter{20}

\par 1 Prednímu zpeváku, žalm Daviduv.
\par 2 Vyslyšiž te Hospodin v den ssoužení, k zvýšení te prived jméno Boha Jákobova.
\par 3 Sešliž tobe pomoc z svatyne, a z Siona utvrzuj te.
\par 4 Rozpomeniž se na všecky obeti tvé, a zápaly tvé v popel obrat. Sélah.
\par 5 Dejž tobe vše podlé srdce tvého, a všelikou radu tvou vypln.
\par 6 I budeme prozpevovati o spasení tvém, a ve jménu Boha našeho korouhve vyzdvihneme; naplniž Hospodin všecky prosby tvé.
\par 7 Nynít jsme poznali, že Hospodin zachoval svého pomazaného, a že jej vyslyšel s nebe svatého svého; nebo v jeho presilné pravici jest spasení.
\par 8 Tito v vozích, jiní v koních doufají, ale my jméno Hospodina Boha našeho sobe pripomínáme.
\par 9 A protož oni sehnuti jsou, a padli, ale my povstali jsme, a zmužile stojíme.
\par 10 Hospodine, zachovávejž nás, i král at slyší nás, když k nemu volati budeme.

\chapter{21}

\par 1 Prednímu z kantoru, žalm Daviduv.
\par 2 Hospodine, v síle tvé raduje se král, a v spasení tvém veselí se prenáramne.
\par 3 Žádost srdce jeho dal jsi jemu, a prosbe rtu jeho neodeprel jsi. Sélah.
\par 4 Predšel jsi jej zajisté hojným požehnáním, vstavil jsi na hlavu jeho korunu z ryzího zlata.
\par 5 Života požádal od tebe, a dal jsi mu prodlení dnu na veky veku.
\par 6 Veliká jest sláva jeho v spasení tvém, dustojností a krásou priodel jsi jej.
\par 7 Nebo jsi jej vystavil za príklad hojného požehnání až na veky, rozveselil jsi jej radostí oblíceje svého.
\par 8 A ponevadž král doufá v Hospodina, a v milosrdenství Nejvyššího, nepohnet se.
\par 9 Najdet ruka tvá všecky neprátely své, dosáhne pravice tvá tech, kteríž te nenávidí.
\par 10 Uvržeš je jako do peci ohnivé v cas rozhnevání svého; Hospodin v prchlivosti své sehltí je, a ohen sžíre je.
\par 11 Pléme jejich z zeme vyhladíš, a síme jejich z synu lidských,
\par 12 Nebo jsou proti tobe ukládali zlost, myslili na nešlechetnost, ac ji dovesti nemohli.
\par 13 Protož je vystavíš za cíl, na tetiva svá prikládati budeš proti tvári jejich.
\par 14 Zjeviž se, ó Hospodine, v síle své, a budemet zpívati a oslavovati udatnost tvou.

\chapter{22}

\par 1 Prednímu zpeváku k casu jitrnímu, žalm Daviduv.
\par 2 Bože muj, Bože muj, procež jsi mne opustil? Vzdálils se od spasení mého a od slov naríkání mého.
\par 3 Bože muj, pres celý den volám, a neslyšíš, i v noci, a nemohu se utajiti.
\par 4 Ty zajisté jsi svatý, zustávající vždycky k veliké chvále Izraelovi.
\par 5 V tobet doufali otcové naši, doufali, a vysvobozovals je.
\par 6 K tobe volávali, a spomáhals jim; v tobe doufali, a nebývali zahanbeni.
\par 7 Já pak cerv jsem, a ne clovek, útržka lidská a povrhel vubec.
\par 8 Všickni, kteríž mne vidí, posmívají se mi, ošklebují se, a hlavami potrásají, ríkajíce:
\par 9 Spustilte se na Hospodina, necht ho vysvobodí; nechat jej vytrhne, ponevadž se mu v nem zalíbilo.
\par 10 Ješto ty jsi, kterýž jsi mne vyvedl z života, ustaviv mne v doufání pri prsích matky mé.
\par 11 Na tebet jsem uvržen od narození svého, od života matky mé Buh muj ty jsi.
\par 12 Nevzdalujž se ode mne, nebo ssoužení blízké jest, a nemám spomocníka.
\par 13 Obklicujít mne býkové mnozí, silní volové z Bázan obstupují mne.
\par 14 Otvírají na mne ústa svá, jako lev rozsapávající a rvoucí.
\par 15 Jako voda rozplynul jsem se, a rozstoupily se všecky kosti mé, a srdce mé jako vosk rozpustilo se u prostred vnitrností mých.
\par 16 Vyprahla jako strepina síla má, a jazyk muj prilnul k dásním mým, anobrž v prachu smrti položils mne.
\par 17 Nebo psi obskocili mne, rota zlostníku oblehla mne, zprobijeli ruce mé i nohy mé.
\par 18 Mohl bych scísti všecky kosti své, oni pak hledí na mne, a dívají se mi.
\par 19 Delí mezi sebou roucha má, a o muj odev mecí los.
\par 20 Ale ty, Hospodine, nevzdalujž se, sílo má, prispej k spomožení mému.
\par 21 Vychvat od mece duši mou, a z moci psu jedinkou mou.
\par 22 Zachovej mne od úst lva, a od rohu jednorožcových vyprost mne.
\par 23 I budu vypravovati bratrím svým o jménu tvém, u prostred shromáždení chváliti te budu, rka:
\par 24 Kteríž se bojíte Hospodina, chvalte jej, všecko síme Jákobovo ctete jej, a boj se ho všecka rodino Izraelova.
\par 25 Nebo nepohrdá, ani se odvrací od trápení ztrápeného, aniž skrývá tvári své od neho, nýbrž když k nemu volá, vyslýchá jej.
\par 26 O tobe chvála má v shromáždení velikém, sliby své plniti budu pred temi, kteríž se bojí tebe.
\par 27 Jísti budou tiší a nasyceni budou, chváliti budou Hospodina ti, kteríž ho hledají, živo bude srdce vaše na veky.
\par 28 Rozpomenou a obrátí se k Hospodinu všecky konciny zeme, a skláneti se budou pred ním všecky celedi národu.
\par 29 Nebo Hospodinovo jest království, a ont panuje nad národy.
\par 30 Jísti budou a skláneti se pred ním všickni tucní zeme, jemu se klaneti budou všickni sstupující do prachu, a kteríž duše své nemohou pri životu zachovati.
\par 31 Síme jejich sloužiti mu bude, a pricteno bude ku Pánu v každém veku.
\par 32 Prijdout, a lidu, kterýž z nich vyjde, vypravovati budou spravedlnost jeho; nebo ji skutkem vykonal.

\chapter{23}

\par 1 Žalm Daviduv. Hospodin jest muj pastýr, nebudu míti nedostatku.
\par 2 Na pastvách zelených pase mne, k vodám tichým mne privodí.
\par 3 Duši mou ocerstvuje, vodí mne po stezkách spravedlnosti pro jméno své.
\par 4 Byt mi se dostalo jíti pres údolí stínu smrti, nebudut se báti zlého, nebo ty se mnou jsi; prut tvuj a hul tvá, tot mne potešuje.
\par 5 Strojíš stul pred oblícejem mým naproti mým neprátelum, pomazuješ olejem hlavy mé, kalich muj naléváš, až oplývá.
\par 6 Nadto i dobrota a milosrdenství následovati mne budou po všecky dny života mého, a prebývati budu v dome Hospodinove za dlouhé casy.

\chapter{24}

\par 1 Žalm Daviduv. Hospodinova jest zeme, a plnost její, okršlek zeme, i ti, kteríž obývají na nem.
\par 2 Nebo on ji na mori založil, a na rekách upevnil ji.
\par 3 Kdo vstoupí na horu Hospodinovu? A kdo stane na míste svatém jeho?
\par 4 Ten, kdož jest rukou nevinných, a srdce cistého, kdož neobrací duše své k marnosti, a neprisahá lstive.
\par 5 Ten prijme požehnání od Hospodina, a spravedlnost od Boha spasitele svého.
\par 6 Tot jest národ hledajících jeho, hledajících tvári tvé, ó Bože Jákobuv. Sélah.
\par 7 Pozdvihnetež, ó brány, svrchku svých, pozdvihnetež se vrata vecná, aby vjíti mohl král slávy.
\par 8 Kdož jest to ten král slávy? Hospodin silný a mocný, Hospodin udatný válecník.
\par 9 Pozdvihnetež, ó brány, svrchku svých, pozdvihnete se vrata vecná, aby vjíti mohl král slávy.
\par 10 Kdož jest to ten král slávy? Hospodin zástupu, ont jest král slávy. Sélah.

\chapter{25}

\par 1 Žalm Daviduv. K tobet, Hospodine, duše své pozdvihuji.
\par 2 Bože muj, v tobet nadeji skládám, necht nejsem zahanben, aby se neradovali neprátelé moji nade mnou.
\par 3 A takt i všickni, kteríž na te ocekávají, zahanbeni nebudou; zahanbeni budou, kteríž se prevrácene mají bez príciny.
\par 4 Cesty své, Hospodine, uved mi v známost, a stezkám svým vyuc mne.
\par 5 Dejž, at chodím v pravde tvé, a poucuj mne; nebo ty jsi Buh spasitel muj, na tebet ocekávám dne každého.
\par 6 Rozpomen se na slitování svá, Hospodine, a na milosrdenství svá, kteráž jsou od veku.
\par 7 Hríchu mladosti mé a prestoupení mých nezpomínej, ale podlé milosrdenství svého pametliv bud na mne pro dobrotu svou, Hospodine.
\par 8 Dobrý a prímý jest Hospodin, a protož vyucuje hríšníky ceste své.
\par 9 Pusobí to, aby tiší chodili v soudu, a vyucuje tiché ceste své.
\par 10 Všecky stezky Hospodinovy jsou milosrdenství a pravda tem, kteríž ostríhají smlouvy jeho a svedectví jeho.
\par 11 Pro jméno své, Hospodine, odpust nepravost mou, nebot jest veliká.
\par 12 Který jest clovek, ješto se bojí Hospodina, jehož vyucuje, kterou by cestu vyvoliti mel?
\par 13 Duše jeho v dobrém prebývati bude, a síme jeho dedicne obdrží zemi.
\par 14 Tajemství Hospodinovo zjevné jest tem, kteríž se jeho bojí, a v známost jim uvodí smlouvu svou.
\par 15 Oci mé vždycky patrí k Hospodinu, on zajisté z leci vyvodí nohy mé.
\par 16 Popatriž na mne, a smiluj se nade mnou, nebot jsem opuštený a strápený.
\par 17 Ssoužení srdce mého rozmnožují se, z úzkostí mých vyved mne.
\par 18 Viz trápení mé a bídu mou, a odpust všecky hríchy mé.
\par 19 Viz neprátely mé, jak mnozí jsou, a nenávistí nešlechetnou nenávidí mne.
\par 20 Ostríhej duše mé, a vytrhni mne, at nejsem zahanben, nebot v tebe doufám.
\par 21 Sprostnost a uprímnost nechat mne ostríhají, nebo na te ocekávám.
\par 22 Vykup, ó Bože, Izraele ze všelijakých úzkostí jeho.

\chapter{26}

\par 1 Žalm Daviduv. Sud mne, Hospodine, nebo já v uprímnosti své chodím, a v te Hospodina doufám, nepohnut se.
\par 2 Zprubujž mne, Hospodine, a zkus mne, prepal ledví má i srdce mé.
\par 3 Milosrdenství tvé zajisté pred ocima mýma jest, a chodím stále v pravde tvé.
\par 4 S lidmi marnými nesedám, a s pokrytci v spolek nevcházím.
\par 5 V nenávisti mám shromáždení zlostníku, a s bezbožnými se neusazuji.
\par 6 Umývám v nevinnosti ruce své, postavuji se pri oltári tvém, Hospodine,
\par 7 Abych te hlasite chválil, a vypravoval všecky divné skutky tvé.
\par 8 Hospodine, ját miluji obydlí domu tvého, a místo príbytku slávy tvé.
\par 9 Nezahrnujž s hríšnými duše mé, a s lidmi vražedlnými života mého,
\par 10 V jejichž rukou jest nešlechetnost, a pravice jejich vzátku plná.
\par 11 Já pak v uprímnosti své chodím, vykupiž mne, a smiluj se nade mnou.
\par 12 Noha má stojí na rovine, v shromáždeních svatých dobroreciti budu Hospodinu.

\chapter{27}

\par 1 Daviduv. Hospodin svetlo mé a spasení mé, kohož se budu báti? Hospodin síla života mého, kohož se budu strašiti?
\par 2 Útok ucinivše na mne zlostníci, k sežrání tela mého, protivníci moji a neprátelé moji, sami se potkli a padli.
\par 3 Protož byt i stany své proti mne rozbili, nebude se lekati srdce mé; byt se pozdvihla proti mne i válka, na tot se já spouštím.
\par 4 Jedné veci žádal jsem od Hospodina, tét vždy hledati budu: Abych prebýval v dome Hospodinove po všecky dny života svého,a spatroval okrasu Hospodinovu, a zpytoval v chráme jeho.
\par 5 Nebo tu mne ukryje v stánku svém, v den zlý schová mne v skrýši stanu svého, a na skálu vyzdvihne mne.
\par 6 A tak vyvýšena bude hlava má nad neprátely mými, kteríž mne obklícili; i budu obetovati v stánku jeho obeti plésání, prozpevovati a chvály vzdávati budu Hospodinu.
\par 7 Slyš mne, Hospodine, hlasem volajícího, a smiluj se nade mnou, i vyslyš mne.
\par 8 O tobe premýšlí srdce mé, že velíš, rka: Hledejte tvári mé, a protož tvári tvé, Hospodine, hledati budu.
\par 9 Neskrývejž tvári své prede mnou, aniž zamítej v hneve služebníka svého; spomožení mé býval jsi, neopouštej mne, aniž se mne zhoštuj, Bože spasení mého.
\par 10 Ackoli otec muj a matka má mne opustili, Hospodin však mne k sobe privine.
\par 11 Vyuc mne, Hospodine, ceste své, a ved mne po stezce prímé pro ty, jenž mne strehou.
\par 12 Nevydávejž mne líbosti protivníku mých, nebot by ostáli proti mne svedkové lživí, i ten, jenž dýše ukrutností.
\par 13 Bycht neveril, že užívati budu dobroty Hospodinovy v zemi živých, nikoli bych neostál.
\par 14 Ocekávejž na Hospodina, posiln se, a ont posilní srdce tvého; protož ocekávej na Hospodina.

\chapter{28}

\par 1 Daviduv. K tobet, Hospodine, volám, skálo má, neodmlcujž mi se, abych, neozval-li bys mi se, nebyl podobný ucinen tem, kteríž sstupují do hrobu.
\par 2 Vyslýchej hlas pokorných modliteb mých, kdyžkoli k tobe volám, když pozdvihuji rukou svých k svatyni svatosti tvé.
\par 3 Nezahrnuj mne s bezbožníky, a s temi, kteríž páší nepravost, kteríž mluvívají pokoj s bližními svými, ale prevrácenost jest v srdcích jejich.
\par 4 Dejž jim podlé skutku jejich, a podlé zlosti nešlechetností jejich; podlé práce rukou jejich dej jim, zaplat jim zaslouženou mzdu jejich.
\par 5 Nebo nechtí mysli priložiti k skutkum Hospodinovým, a k dílu rukou jeho; protož podvrátí je, a nebude jich vzdelávati.
\par 6 Požehnaný Hospodin, nebo vyslyšel hlas pokorných modliteb mých.
\par 7 Hospodin jest síla má a štít muj, v nemt jest složilo nadeji srdce mé, a dána mi pomoc; protož se veselí srdce mé, a písnickou svou oslavovati jej budu.
\par 8 Hospodin jest síla svých, a síla hojného spasení pomazaného svého on jest.
\par 9 Spas lid svuj, Hospodine, a požehnej dedictví svému, a pas je, i vyvyš je až na veky.

\chapter{29}

\par 1 Žalm Daviduv. Vzdejte Hospodinu, velikomocní, vzdejte Hospodinu cest a sílu.
\par 2 Vzdejte Hospodinu slávu jména jeho, sklánejte se Hospodinu v ozdobe svatosti.
\par 3 Hlas Hospodinuv nad vodami, Buh silný slávy hrímání vzbuzuje, Hospodin to ciní nad vodami mnohými.
\par 4 Hlas Hospodinuv prichází s mocí, hlas Hospodinuv s velebností.
\par 5 Hlas Hospodinuv láme cedry, rozrážít Hospodin cedry Libánské.
\par 6 A ciní, aby skákali jako telata, Libán a Sirion, jako mladý jednorožec.
\par 7 Hlas Hospodinuv rozkresává plamen ohne.
\par 8 Hlas Hospodinuv k bolesti privodí poušt, k bolesti privodí Hospodin poušt Kádes.
\par 9 Hlas Hospodinuv to ciní, že lane plodu pozbývají, obnažuje i lesy, ale v chráme svém všecku svou slávu vypravuje.
\par 10 Hospodin nad potopou sedel, a budet sedeti Hospodin, jsa králem i na veky.
\par 11 Hospodin silou lid svuj darí, Hospodin požehná lidu svému v pokoji.

\chapter{30}

\par 1 Žalm písne, pri posvecení domu Davidova.
\par 2 Vyvyšovati te budu, Hospodine, nebo jsi vyvýšil mne, aniž jsi obradoval neprátel mých nade mnou.
\par 3 Hospodine Bože muj, k tobet jsem volal, a uzdravil jsi mne.
\par 4 Hospodine, vyvedl jsi z pekla duši mou, obživil jsi mne, abych s jinými nesstoupil do hrobu.
\par 5 Žalmy zpívejte Hospodinu svatí jeho, a oslavujte památku svatosti jeho.
\par 6 Nebo na kraticko trvá v hneve svém, všecken pak život v dobré líbeznosti své; z vecera potrvá plác, ale z jitra navrátí se prozpevování.
\par 7 I ját jsem rekl, když mi se štastne vedlo: Nepohnu se na veky.
\par 8 Nebo ty, Hospodine, podlé dobre líbezné vule své silou upevnil jsi horu mou, ale jakž jsi skryl tvár svou, byl jsem prestrašen.
\par 9 I volal jsem k tobe, Hospodine, Pánu pokorne jsem se modlil, rka:
\par 10 Jaký bude užitek z mé krve, jestliže sstoupím do jámy? Zdaliž te prach oslavovati bude? Zdaliž zvestovati bude pravdu tvou?
\par 11 Vyslyšiž, Hospodine, a smiluj se nade mnou, Hospodine, budiž muj spomocník.
\par 12 I obrátil jsi mi plác muj v plésání, odvázal jsi pytel muj, a prepásals mne veselím.
\par 13 Protož tobe žalmy zpívati bude jazyk muj, a nebude mlceti. Hospodine Bože muj, na veky te oslavovati budu.

\chapter{31}

\par 1 Prednímu zpeváku, žalm Daviduv.
\par 2 V tebe, Hospodine, doufám, nedejž mi zahanbenu býti na veky, pro spravedlnost svou vysvobod mne.
\par 3 Naklon ke mne ucha svého, rychle vytrhni mne; budiž mi pevnou skalou a domem ohraženým, abys mne zachoval.
\par 4 Nebo skála má a hrad muj ty jsi, protož pro jméno své ved i doved mne.
\par 5 Vyved mne z leci, kterouž polékli na mne; nebo síla má ty jsi.
\par 6 V ruce tvé poroucím ducha svého, nebo jsi mne vykoupil, Hospodine, Bože silný a verný.
\par 7 Nenávidím tech, kteríž následují pouhých marností, nebo já v Hospodinu nadeji skládám.
\par 8 Plésati a radovati se budu v milosrdenství tvém, že jsi vzezrel na mé trápení, a poznal jsi v ssoužení duši mou.
\par 9 Aniž jsi mne zavrel v ruce neprítele, ale postavil jsi na širokosti nohy mé.
\par 10 Smiluj se nade mnou, Hospodine, nebo jsem ssoužen, tak že usvadla zámutkem tvár má, duše má, i život muj.
\par 11 Žalostí zajisté zhynulo zdraví mé, a léta má od úpení, zemdlena bídou mou síla má, a kosti mé vyprahly.
\par 12 U všech neprátel svých jsem v pohanení, a nejvíce u sousedu, známým pak svým jsem strašidlem; kteríž mne vídají vne, utíkají prede mnou.
\par 13 Vyšel jsem z pameti tak, jako mrtvý, ucinen jsem jako nádoba rozražená.
\par 14 Nebo slýchám utrhání mnohých, strach odevšad, když se proti mne spolu puntují, lstive premýšlejíce, jak by odjali duši mou.
\par 15 Ale já v tobe nadeji skládám, Hospodine; rekl jsem: Buh muj jsi ty.
\par 16 V rukou tvých jsou casové moji, vytrhni mne z ruky neprátel mých a tech, kteríž mne stihají.
\par 17 Osvet tvár svou nad služebníkem svým, zachovej mne pro milosrdenství své.
\par 18 Hospodine, at nejsem zahanben, nebo jsem te vzýval; nechat jsou zahanbeni bezbožníci, a skroceni v pekle.
\par 19 Onemejte rtové lživí, kteríž mluví proti spravedlivému tvrde, pyšne a s potupou.
\par 20 Ó jak veliká jest dobrotivost tvá, kterouž jsi odložil tem, jenž se bojí tebe, a kterouž jsi ciníval doufajícím v tebe pred syny lidskými.
\par 21 Ty je skrýváš v skrýši oblíceje svého pred vysokomyslností cloveka, skrýváš je jako v stanu pred jazyky svárlivými.
\par 22 Požehnaný bud Hospodin, nebo prokázal ke mne divné milosrdenství své jako v meste ohraženém.
\par 23 Já zajisté když jsem pospíchal, rekl jsem: Zavržent jsem od ocí tvých, ale ty jsi vyslyšel hlas pokorných modliteb mých, když jsem k tobe volal.
\par 24 Milujtež Hospodina všickni svatí jeho, nebot ostríhá verících Hospodin, a též odplací vrchovate tomu, kdož pýchu provodí.
\par 25 Zmužile sobe cinte, (a posilní Buh srdce vašeho), všickni, kteríž nadeji máte v Hospodinu.

\chapter{32}

\par 1 Žalm Daviduv vyucující. Blahoslavený jest ten, jemuž odpušteno prestoupení, a jehož hrích prikryt jest.
\par 2 Blahoslavený clovek, jemuž nepocítá Hospodin nepravosti, a v jehož duchu lsti není.
\par 3 Já když jsem mlcel, prahly kosti mé v úpení mém každého dne.
\par 4 Nebo dnem i nocí obtížena byla nade mnou ruka tvá, tak že prirozená vlhkost má obrátila se v sucho letní. Sélah.
\par 5 Protož hrích svuj oznámil jsem tobe, a nepravosti své jsem neukryl. Rekl jsem: Vyznám na sebe Hospodinu prestoupení svá, a ty jsi odpustil nepravost hríchu mého. Sélah.
\par 6 Za to se tobe bude modliti každý svatý, v casu príhodném k nalezení tebe; procež vody mnohé v rozvodnení k nemu nedosáhnou.
\par 7 Ty jsi skrýše má, od ssoužení zachováš mne, a plésáním vítezným obdaríš. Sélah.
\par 8 Já tobe k srozumení posloužím, a vyucím te ceste, po níž bys choditi mel; dámt radu, oci své na te obráte.
\par 9 Nebývejte jako kun a jako mezek, kteríž rozumu nemají, jejichž ústa uzdou a udidly sevríti musíš, aby tobe neškodili.
\par 10 Mnohé bolesti jsou bezbožníka, ale toho, jenž nadeji skládá v Hospodinu, milosrdenství obklící.
\par 11 Radujte se v Hospodinu, a plésejte spravedliví, a prozpevujte všickni srdce uprímého.

\chapter{33}

\par 1 Veselte se spravedliví v Hospodinu, na uprímét prísluší chválení.
\par 2 Oslavujte Hospodina na harfe, na loutne, a na nástroji o desíti strunách, žalmy zpívejte jemu.
\par 3 Zpívejte jemu písen novou, a hudte dobre a zvucne.
\par 4 Nebo pravé jest slovo Hospodinovo, a všeliké dílo jeho stálé.
\par 5 Milujet spravedlnost a soud, milosrdenství Hospodinova plná jest zeme.
\par 6 Slovem Hospodinovým nebesa ucinena jsou, a duchem úst jeho všecko vojsko jejich.
\par 7 Ont shrnul jako na hromadu vody morské, a složil na poklad propasti.
\par 8 Boj se Hospodina všecka zeme, destež se pred ním všickni obyvatelé okršlku zemského.
\par 9 Nebo on rekl, a stalo se, on rozkázal, a postavilo se.
\par 10 Hospodin ruší rady národu, a v nic obrací premyšlování lidská.
\par 11 Rada pak Hospodinova na veky trvá, myšlení srdce jeho od národu do pronárodu.
\par 12 Blahoslavený národ, kteréhož Hospodin jest Bohem jeho, lid ten, kterýž sobe on vyvolil za dedictví.
\par 13 Hospodin patre s nebe, vidí všecky syny lidské,
\par 14 Z príbytku trunu svého dohlédá ke všechnem obyvatelum zeme.
\par 15 Ten, kterýž stvoril srdce jednoho každého z nich, spatruje všecky skutky jejich.
\par 16 Nebývá král zachován skrze mnohý zástup, ani udatný rek vysvobozen skrze velikou moc svou.
\par 17 Oklamavatelný jest kun k spomožení, aniž ve množství síly své vytrhuje.
\par 18 Aj, oci Hospodinovy patrí na ty, kteríž se ho bojí, a na ty, kteríž ocekávají milosrdenství jeho,
\par 19 Aby vyprostil od smrti duše jejich, a živil je v cas hladu.
\par 20 Duše naše ocekává na Hospodina, on jest spomožení naše, a pavéza naše.
\par 21 V nem zajisté rozveselí se srdce naše, nebo ve jménu jeho svatém nadeji skládáme.
\par 22 Budiž milosrdenství tvé nad námi, Hospodine, jakož nadeji máme v tobe.

\chapter{34}

\par 1 Daviduv, když promenil oblícej svuj pred Abimelechem; procež jsa od neho vyhnán, odšel.
\par 2 Dobroreciti budu Hospodinu každého casu, vždycky chvála jeho v ústech mých.
\par 3 V Hospodinu chlubiti se bude duše má, což uslyšíc tiší, budou se veseliti.
\par 4 Zvelebujtež se mnou Hospodina, a jméno jeho spolecne vyvyšujme.
\par 5 Hledal jsem Hospodina, a vyslyšel mne, a ze všech prístrachu mých vytrhl mne.
\par 6 Procež k nemu patriti budou, a sbíhati se, a nebudou zahanbeny tvári jejich, ale rkou:
\par 7 Tento chudý volal a Hospodin vyslyšel, i ze všech úzkostí jeho vysvobodil jej.
\par 8 Vojensky se klade andel Hospodinuv okolo tech, kteríž se ho bojí, a zastává jich.
\par 9 Okuste a vizte, jak dobrý jest Hospodin. Blahoslavený clovek, kterýž doufá v neho.
\par 10 Bojtež se Hospodina svatí jeho; nebot nemívají nedostatku ti, kdož se ho bojí.
\par 11 Lvícátka nedostatek a hlad trpívají, ale ti, kteríž hledají Hospodina, nemívají nedostatku ve všem dobrém.
\par 12 Podtež, dítky, poslouchejte mne, bázni Hospodinove vyucovati vás budu.
\par 13 Který clovek žádostiv jest života, a miluje dny, aby užíval dobrých vecí?
\par 14 Zdržuj jazyk svuj od zlého, a rty své od mluvení lsti.
\par 15 Odstup od zlého, a cin dobré, hledej pokoje, a stíhej jej.
\par 16 Oci Hospodinovy obrácené jsou k spravedlivým, a uši jeho k volání jejich:
\par 17 Ale zurivý oblícej Hospodinuv proti tem, kteríž páší zlé veci, aby vyplénil z zeme památku jejich.
\par 18 Volají-li spravedliví, Hospodin vyslýchá, a ze všech jejich úzkostí je vytrhuje.
\par 19 Nebo blízko jest Hospodin tem, kteríž jsou srdce skroušeného, a potreným v duchu spomáhá.
\par 20 Mnohé úzkosti jsou spravedlivého, ale Hospodin ze všech jej vytrhuje.
\par 21 Ont ostríhá všech kostí jeho, žádná z nich nebývá zlámána.
\par 22 Bezbožníka zahubí zlost, a ti, kteríž nenávidí spravedlivého, zkaženi budou.
\par 23 Služebníku pak svých duše vykoupí Hospodin, a nebudou zkaženi, kteríž doufají v neho.

\chapter{35}

\par 1 Žalm Daviduv. Sud se, Hospodine, s temi, kteríž se se mnou soudí; bojuj proti tem, kteríž proti mne bojují.
\par 2 Pochyt štít a pavézu, a povstan mi ku pomoci.
\par 3 Vezmi i kopí, a vyjdi vstríc tem, kteríž táhnou proti mne. Rciž duši mé: Spasení tvé ját jsem.
\par 4 Nechat se zahanbí a zapýrí ti, kteríž hledají duše mé; zpet at jsou obráceni a zahanbeni ti, kteríž mi zlé obmýšlejí.
\par 5 At jsou jako plevy pred vetrem, a andel Hospodinuv rozptylujž je.
\par 6 Cesta jejich budiž temná a plzká, a andel Hospodinuv stihej je.
\par 7 Nebo jsou bez príciny polékli v jáme osídlo své, bez príciny vykopali jámu duši mé.
\par 8 Pripadniž na ne setrení, jehož se nenadáli, a sít jejich, kterouž ukryli, at je uloví; s hrmotem at do ní vpadnou.
\par 9 Duše má pak at se veselí v Hospodinu, a at raduje se v spasení jeho.
\par 10 A tut všecky kosti mé reknou: Hospodine, kdo jest podobný tobe, ješto vytrhuješ ztrápeného z moci toho, kterýž nad nej silnejší jest, tolikéž chudého a nuzného od toho, kterýž ho násilne loupí?
\par 11 Povstávají svedkové lživí, a na to, o cemž nevím, dotazují se mne.
\par 12 Zlým za dobré mi se odplacují, duše mé zbaviti mne chtíce,
\par 13 Ježto já v nemoci jejich pytlem jsem se priodíval, duši svou postem trápil, a sám u sebe za ne casto se modlil.
\par 14 Jako k príteli, jako k bratru vlastnímu jsem chodíval; jakožto ten, kterýž po matce kvílí, smutek nesa, tak jsem se ponižoval.
\par 15 Ale oni z mého zlého radovali se, a rotili se; shromaždovali se proti mne i ti nejnevážnejší, o cemž jsem nevedel; utrhali mi, a nemlceli.
\par 16 S pokrytci, posmevaci, fatkári škripeli na mne zuby svými.
\par 17 Pane, dlouho-liž se dívati budeš? Vytrhni duši mou od zhouby jejich, od lvu jedinkou mou.
\par 18 I budu te oslavovati v shromáždení velikém, ve množství lidu tebe chváliti budu.
\par 19 Nechažt se nade mnou neradují ti, kteríž bezprávne ke mne se neprátelsky mají; ti, kteríž mne nenávidí bez príciny, at nemhourají ocima.
\par 20 Nebot nemluví ku pokoji, ale proti pokojným na zemi slova lstivá vymýšlejí.
\par 21 Anobrž rozdírají proti mne ústa svá, a ríkají: Hahá, hahá, jižt vidí oko naše.
\par 22 Vidíš ty to, Hospodine, neodmlcujž se, Pane, nevzdalujž se ode mne.
\par 23 Probudiž se a procit k soudu mému, Bože muj a Pane muj, k obhájení pre mé.
\par 24 Sud mne podlé spravedlnosti své, Hospodine Bože muj, at se neradují nade mnou.
\par 25 At neríkají v srdci svém: Mehodek duši naší; at neríkají: Sehltili jsme jej.
\par 26 Ale at se zahanbí a zapýrí všickni radující se mému zlému, v stud a hanbu at jsou obleceni ti, kteríž se zpínají proti mne.
\par 27 Ti pak, kteríž mi prejí mé spravedlnosti, at plésají, a radují se, a at ríkají vždycky: Veleslaven budiž Hospodin, kterýž preje pokoje služebníku svému.
\par 28 I muj jazyk ohlašovati bude spravedlnost tvou, a na každý den chválu tvou.

\chapter{36}

\par 1 Prednímu z kantoru, služebníka Hospodinova Davida.
\par 2 Prevrácenost bezbožníka pojištuje u vnitrnosti srdce mého, že není žádné bázne Boží pred ocima jeho.
\par 3 Nebo mu ona pochlebuje pred ocima jeho, aby vykonal nepravost svou až do zošklivení.
\par 4 Slova úst jeho jsou nepravá a lstivá, prestal srozumívati, aby dobre cinil.
\par 5 Nepravost smýšlí i na ložci svém, ustavuje se na ceste nedobré, zlého se nevaruje.
\par 6 Hospodine, až do nebes milosrdenství tvé, pravda tvá až do nejvyšších oblaku.
\par 7 Spravedlnost tvá jako nejvyšší hory, soudové tvoji jako hlubokost nesmírná; lidi i hovada sám zachováváš, Hospodine.
\par 8 Jak prevelmi drahé jest milosrdenství tvé, Bože, a protož synové lidští v stínu krídel tvých doufají.
\par 9 Tucností domu tvého rozvlažováni bývají, a potokem rozkoší svých napájíš je.
\par 10 Nebo u tebe jest studnice života, a v svetle tvém svetlo vidíme.
\par 11 Rozprostri milosrdenství své na ty, kteríž tebe znají, a spravedlnost tvou na uprímé srdcem.
\par 12 Nechažt nedotírá na mne noha pyšných, a ruka bezbožníku at mne nezavozuje.
\par 13 Tam, kdež padají cinitelé nepravosti, poraženi bývají, a nemohou povstati.

\chapter{37}

\par 1 Žalm Daviduv. Nehnevej se prícinou zlostníku, nechtej závideti tem, kteríž páší nepravost.
\par 2 Nebo jako tráva v náhle podtati budou, a jako zelená bylina uvadnou.
\par 3 Doufej v Hospodina, a
\par 4 Teš se v Hospodinu, a dá tobe žádosti srdce tvého.
\par 5 Uval na Hospodina cestu svou, a slož v nem nadeji, ont zajisté všecko spraví.
\par 6 A vyvedet spravedlnost tvou jako svetlo, a nevinu tvou jako poledne.
\par 7 Mlcelive se mej k Hospodinu, a ocekávej na nej peclive. Nekormut se prícinou toho, jemuž se darí na ceste jeho, prícinou cloveka, kterýž provodí, cožkoli umyslil.
\par 8 Pust mimo sebe hnev, a zanech prchlivosti; nezpouzej se tak, abys zle ciniti chtel.
\par 9 Nebo zlostníci vypléneni budou, ale ti, kteríž ocekávají na Hospodina, dedicne zemí vládnouti budou.
\par 10 Po malé chvíli zajisté, ant bezbožníka nebude, a pohledíš na místo jeho, ant ho již není.
\par 11 Ale tiší dedicne obdrží zemi, a rozkoš míti budou ve množství pokoje.
\par 12 Zle myslí bezbožník o spravedlivém, a škripí na nej zuby svými,
\par 13 Ale Hospodin smeje se jemu; nebo vidí, že se približuje den jeho.
\par 14 Vytrhujít bezbožníci mec, a natahují lucište své, aby porazili chudého a nuzného, aby hubili ty, kteríž jsou ctného obcování;
\par 15 Ale mec jejich vejde v jejich srdce, a lucište jejich budou polámána.
\par 16 Lepší jest málo, což má spravedlivý, než veliká bohatství bezbožníku mnohých.
\par 17 Nebo ramena bezbožných polámána budou, spravedlivé pak zdržuje Hospodin.
\par 18 Znát Hospodin dny uprímých, protož dedictví jejich na veky zustane.
\par 19 Nebudout zahanbeni v cas zlý, a ve dnech hladu nasyceni budou;
\par 20 Ale bezbožníci zahynou, a neprátelé Hospodinovi, jak tuk beranu s dymem mizí, tak zmizejí.
\par 21 Vypujcuje bezbožník, a nemá co oplatiti, ale spravedlivý milost ciní, a rozdává.
\par 22 Nebo požehnaní ode Pána zemí vládnouti budou, ale zlorecení od neho budou vypléneni.
\par 23 Krokové cloveka spravedlivého od Hospodina spravováni bývají, a cestu jeho libuje.
\par 24 Jestliže by upadl, neurazí se; nebo Hospodin drží jej za ruku jeho.
\par 25 Mlad jsem byl, a sstaral jsem se, a nevidel jsem spravedlivého opušteného, ani semene jeho žebrati chleba.
\par 26 Každého dne milost ciní, i pujcuje, a však síme jeho jest v požehnání.
\par 27 Odstup od zlého, a cin dobré, a bydliti budeš na veky.
\par 28 Nebo Hospodin miluje soud, a neopouští svatých svých, na veky v stráži jeho budou; síme pak bezbožníku bude vypléneno.
\par 29 Ale spravedliví ujmou zemi dedicne, a na veky v ní prebývati budou.
\par 30 Ústa spravedlivého mluví moudrost, a jazyk jeho vynáší soud.
\par 31 Zákon Boha jeho jest v srdci jeho, procež nepodvrtnou se nohy jeho.
\par 32 Špehujet bezbožník po spravedlivém, a hledá ho zahubiti;
\par 33 Ale Hospodin ho nenechá v ruce jeho, aniž ho dopustí potupiti, když by souzen byl.
\par 34 Ocekávejž tedy na Hospodina, a ostríhej cesty jeho, a on te povýší, abys dedicne obdržel zemi, z níž že vykoreneni budou bezbožníci, uhlédáš.
\par 35 Videl jsem bezbožníka hrozné síly, an se rozložil jako zelený samorostlý strom.
\par 36 Ale tudíž pominul, a aj nebylo ho; nebo hledal jsem ho, a není nalezen.
\par 37 Pozor mej na pobožného, a viz uprímého, žet takového cloveka poslední veci jsou potešené,
\par 38 Prestupníci pak že tolikéž vyhlazeni budou, a bezbožníci naposledy vytati.
\par 39 Ale spasení spravedlivých jest od Hospodina, ont jest síla jejich v casu ssoužení.
\par 40 Spomáhát jim Hospodin, a je vytrhuje, vytrhuje je od bezbožníku, a zachovává je; nebo doufají v neho.

\chapter{38}

\par 1 Žalm Daviduv k pripomínání.
\par 2 Hospodine, v prchlivosti své netresci mne, ani v hneve svém kárej mne.
\par 3 Nebo strely tvé uvázly ve mne, a ruka tvá na mne težce dolehla.
\par 4 Nic není celého v tele mém pro tvou hnevivost, nemají pokoje kosti mé pro hrích muj.
\par 5 Nebo nepravosti mé vzešly nad hlavu mou, jako bríme težké nemožné jsou mi k unesení.
\par 6 Zahnojily se, a kyší rány mé pro bláznovství mé.
\par 7 Pohrbený a sklícený jsem náramne, každého dne v smutku chodím.
\par 8 Nebo ledví má plná jsou mrzkosti, a nic není celého v tele mém.
\par 9 Zemdlen jsem a potrín prevelice, rvu pro úzkost srdce svého.
\par 10 Pane, pred tebou jest všecka žádost má, a vzdychání mé není pred tebou skryto.
\par 11 Srdce mé zmítá se, opustila mne síla má, i to svetlo ocí mých není se mnou.
\par 12 Ti, kteríž mne milovali, a tovaryši moji, štítí se ran mých, a príbuzní moji zdaleka stojí.
\par 13 Ti pak, kteríž stojí o bezživotí mé, osídla lécejí, a kteríž mého zlého hledají, mluví prevrácene, a pres celý den lest a chytrost smýšlejí.
\par 14 Ale já jako hluchý neslyším, a jako nemý, kterýž neotvírá úst svých;
\par 15 Tak jsem, jako clovek, kterýž neslyší, a v jehož ústech není žádného odporu.
\par 16 Nebo na te, Hospodine, ocekávám, ty za mne odpovíš, Pane Bože muj.
\par 17 Nebo jsem rekl: At se neradují ze mne, poklesla-li by se noha má, at se pyšne nepozdvihují nade mnou,
\par 18 Ponevadž k snášení bíd hotov jsem, anobrž bolest má vždycky jest prede mnou.
\par 19 A že nepravosti své vyznávám, a pro hrích svuj tesklím.
\par 20 Procež neprátelé moji veseli jsouce, silí se, a rozmnožují se ti, kteríž mne bez príciny nenávidí.
\par 21 A zlým za dobré mi se odplacujíce, ciní mi protivenství, proto že dobrého následuji.
\par 22 Neopouštejž mne, Hospodine Bože muj, nevzdalujž se ode mne.
\par 23 Prispej k spomožení mému, Pane spasení mého.

\chapter{39}

\par 1 Prednímu zpeváku Jedutunovi, žalm Daviduv.
\par 2 Rekl jsem: Ostríhati budu cest svých, abych nezhrešil jazykem svým; pojmu v uzdu ústa svá, dokudž bude bezbožník prede mnou.
\par 3 Mlcením byl jsem k nemému podobný, umlcel jsem se i spravedlivého odporu, ale bolest má více zbourena jest.
\par 4 Horelo ve mne srdce mé, roznícen jest ohen v premyšlování mém, tak že jsem mluvil jazykem svým, rka:
\par 5 Dej mi znáti, Hospodine, konec života mého, a odmerení dnu mých jaké jest, abych vedel, jak dlouho trvati mám.
\par 6 Aj, na dlan odmeril jsi mi dnu, a vek muj jest jako nic pred tebou, a jiste žet není než pouhá marnost každý clovek, jakkoli pevne stojící. Sélah.
\par 7 Jiste tak pomíjí clovek jako stín, nadarmo zajisté kvaltuje se; shromažduje, a neví, kdo to pobére.
\par 8 Nacež bych tedy nyní ocekával, Pane? Ocekávání mé jest na tebe.
\par 9 A protož ode všech prestoupení mých vysvobod mne, za posmech bláznu nevystavuj mne.
\par 10 Onemel jsem, a neotevrel úst svých, proto že jsi ty ucinil to.
\par 11 Odejmi ode mne metlu svou, nebo od švihání ruky tvé docela zhynul jsem.
\par 12 Ty, když žehráním pro nepravost tresceš cloveka, hned jako mol k zetlení privodíš zdárnost jeho; marnost zajisté jest všeliký clovek. Sélah.
\par 13 Vyslyšiž modlitbu mou, Hospodine, a volání mé prijmi v uši své; neodmlcujž se kvílení mému, nebo jsem príchozí a podruh u tebe, jako i všickni otcové moji.
\par 14 Ponechej mne, at se posilím, prvé než bych se odebral, a již zde více nebyl.

\chapter{40}

\par 1 Prednímu zpeváku, Daviduv žalm.
\par 2 Žádostive ocekával jsem na Hospodina, i naklonil se ke mne, a vyslyšel mé volání.
\par 3 A vytáhl mne z cisterny hlucící, i z bláta bahnivého, a postavil na skále nohy mé, a kroky mé utvrdil.
\par 4 A tak vložil v ústa má písen novou, chválu Bohu našemu, což když uhlédají mnozí, i báti se, i doufání skládati budou v Hospodinu.
\par 5 Blahoslavený ten clovek, kterýž skládá v Hospodinu svou nadeji, a neohlédá se na pyšné, ani na ty, kteríž se ke lži uchylují.
\par 6 Mnohé veci ciníš ty, Hospodine Bože muj, a divní jsou skutkové tvoji i myšlení tvá o nás; není, kdo by je porád vycísti mohl pred tebou. Já chtel-li bych je vymluviti a vypraviti, mnohem více jich jest, nežli vypraveno býti muže.
\par 7 Obeti a daru neoblíbils, ale uši jsi mi otevrel; zápalu a obeti za hrích nežádal jsi.
\par 8 Tehdy rekl jsem: Aj, jdut, jakož v knihách psáno jest o mne.
\par 9 Abych cinil vuli tvou, Bože muj, líbost mám; nebo zákon tvuj jest u prostred vnitrností mých.
\par 10 Ohlašoval jsem spravedlnost v shromáždení velikém; aj, rtu svých že jsem nezdržoval, ty znáš, Hospodine.
\par 11 Spravedlnosti tvé neukryl jsem u prostred srdce svého, pravdu tvou a spasení tvé vypravoval jsem, nezatajil jsem milosrdenství tvého a pravdy tvé v shromáždení velikém.
\par 12 Ty pak, Hospodine, nevzdaluj slitování svých ode mne; milosrdenství tvé a pravda tvá vždycky at mne ostríhají.
\par 13 Nebot jsou mne obklícily zlé veci, jimž poctu není; dostihly mne mé nepravosti, tak že prohlédnouti nemohu; rozmnožily se nad pocet vlasu hlavy mé, a srdce mé opustilo mne.
\par 14 Raciž ty mne, Hospodine, vysvoboditi; Hospodine, pospešiž ku pomoci mé.
\par 15 Zahanbeni budte, a zapyrte se všickni, kteríž hledají duše mé, aby ji zahladili; zpet obráceni a v potupu dáni budte, kteríž líbost mají v neštestí mém.
\par 16 Prijdiž na ne spuštení za to, že mne k hanbe privésti usilují, ríkajíce: Hahá, hahá.
\par 17 Ale at radují a veselí se v tobe všickni hledající tebe, a milující spasení tvé at ríkají vždycky: Veleslaven budiž Hospodin.
\par 18 Já pak ackoli chudý a nuzný jsem, Pán však pecuje o mne. Pomoc má a vysvoboditel muj ty jsi. Bože muj, neprodlévejž.

\chapter{41}

\par 1 Prednímu zpeváku, žalm Daviduv.
\par 2 Blahoslavený, kdož prozretelný soud ciní o chudém; v den zlý vysvobodí jej Hospodin.
\par 3 Hospodin ho ostreže, a obživí jej; blažený bude na zemi, aniž ho vydá líbosti neprátel jeho.
\par 4 Hospodin ho na ložci ve mdlobe posilí, všecko ležení jeho v nemoci jeho promení.
\par 5 Já rekl jsem: Hospodine, smiluj se nade mnou, uzdrav duši mou, nebo jsem tobe zhrešil.
\par 6 Neprátelé moji mluvili zle o mne, rkouce: Skoro-liž umrel, a zahyne jméno jeho?
\par 7 A jestliže kdo z nich prichází, aby mne navštívil, pochlebenství mluví; srdce jeho sbírá sobe nepravost, a vyjda ven, roznáší ji.
\par 8 Sšeptávají se spolu proti mne všickni, kteríž mne nenávidí, a pricítají mi zlé veci, ríkajíce:
\par 9 Pomsta pro nešlechetnost prichytila se ho, a kdyžte se složil nepovstanet zase.
\par 10 Také i ten, s nímž jsem byl v prátelství, jemuž jsem se doveroval, a kterýž jídal chléb muj, pozdvihl paty proti mne.
\par 11 Ale ty, Hospodine, smiluj se nade mnou, a pozdvihni mne, a odplatím jim;
\par 12 Abych odtud poznal, že mne sobe libuješ, když by se neradoval nade mnou neprítel muj.
\par 13 Mne pak v uprímnosti mé zachováš, a postavíš pred oblícejem svým na veky.
\par 14 Požehnaný Hospodin Buh Izraelský, od veku až na veky, Amen i Amen.

\chapter{42}

\par 1 Prednímu zpeváku z synu Chóre, žalm vyucující.
\par 2 Jakož jelen rve, dychte po tekutých vodách, tak duše má rve k tobe, ó Bože.
\par 3 Žízní duše má Boha, Boha živého, a ríká: Skoro-liž pujdu, a ukáži se pred oblícejem Božím?
\par 4 Slzy mé jsou mi místo chleba dnem i nocí, když mi ríkají každého dne: Kdež jest Buh tvuj?
\par 5 Na to když se rozpomínám, témer duši svou sám v sobe vylévám, že jsem chodíval s mnohými, a ubírával jsem se s nimi do domu Božího s hlasitým zpíváním, a díkcinením v zástupu plésajících.
\par 6 Proc jsi smutná, duše má, a proc se kormoutíš? Poseckej na Boha, nebot ješte vyznávati jej budu, i hojné spasení tvári jeho.
\par 7 Muj Bože, jak tesklí duše má! Protož se na te rozpomínám v krajine Jordánské a Hermonské, na hore Mitsar.
\par 8 Propast propasti se ozývá k hlucení trub tvých, všecka vlnobití tvá a rozvodnení tvá na mne se svalila.
\par 9 Verím však, žet mi udelí ve dne Hospodin milosrdenství svého, a v noci písnicka jeho se mnou, a modlitba má k Bohu života mého.
\par 10 Dím k Bohu silnému, skále své: Procež jsi zapomenul se nade mnou? Proc pro ssoužení od neprítele v smutku mám choditi?
\par 11 Jako rána v kostech mých jest to, když mi utrhají neprátelé moji, ríkajíce mi každého dne: Kdež jest Buh tvuj?
\par 12 Proc jsi smutná, duše má, a proc se kormoutíš ve mne? Poseckej na Boha, nebot ješte vyznávati jej budu; ont jest hojné spasení tvári mé a Buh muj.

\chapter{43}

\par 1 Sud mne, ó Bože, a zasad se o mou pri; od národu nemilosrdného, a od cloveka lstivého a nepravého vytrhni mne.
\par 2 Nebo ty jsi Buh mé síly. Proc jsi mne zapudil? Proc pro ssoužení od neprítele v smutku mám ustavicne choditi?
\par 3 Sešliž svetlo své a pravdu svou, to at mne vodí a zprovodí na horu svatosti tvé a do príbytku tvých,
\par 4 Abych pristoupil k oltári Božímu, k Bohu radostného plésání mého, a budu te oslavovati na harfe, ó Bože, Bože muj.
\par 5 Proc jsi smutná, duše má, a proc se kormoutíš ve mne? Poseckej na Boha, nebot ješte vyznávati jej budu; ont jest hojné spasení tvári mé a Buh muj.

\chapter{44}

\par 1 Prednímu zpeváku z synu Chóre, vyucující.
\par 2 Bože, ušima svýma slýchali jsme, a otcové naši vypravovali nám o skutcích, kteréž jsi ciníval za dnu jejich, za dnu starodávních.
\par 3 Tys sám rukou svou vyhnal pohany, a vštípil jsi je; potrel jsi lidi, a je jsi rozplodil.
\par 4 Nebot jsou neopanovali zeme mecem svým, aniž jim ráme jejich spomohlo, ale pravice tvá a ráme tvé, a svetlost oblíceje tvého, proto že jsi je zamiloval.
\par 5 Ty jsi sám král muj, ó Bože, udílejž hojného spasení Jákobova.
\par 6 V tobet jsme protivníky naše potírali, a ve jménu tvém pošlapávali jsme povstávající proti nám.
\par 7 Nebot jsem v lucišti svém nadeje neskládal, aniž mne kdy obránil mec muj.
\par 8 Ale ty jsi nás vysvobozoval od neprátel našich, a kteríž nás nenávidí, ty jsi zahanboval.
\par 9 A protož chválíme te Boha na každý den, a jméno tvé ustavicne oslavujeme. Sélah.
\par 10 Ale nyní jsi nás zahnal i zahanbil, a nevycházíš s vojsky našimi.
\par 11 Obrátil jsi nás nazpet, a ti, kteríž nás nenávidí, rozchvátali mezi sebou jmení naše.
\par 12 Oddal jsi nás, jako ovce k snedení, i mezi pohany rozptýlil jsi nás.
\par 13 Prodal jsi lid svuj bez penez, a nenadsadils mzdy jejich.
\par 14 Vydal jsi nás k utrhání sousedum našim, ku posmechu a ku potupe tem, kteríž jsou vukol nás.
\par 15 Uvedl jsi nás v prísloví mezi národy, a mezi lidmi, aby nad námi hlavou zmítáno bylo.
\par 16 Na každý den stydeti se musím, a hanba tvári mé prikrývá mne,
\par 17 A to z príciny reci utrhajícího a hanejícího, z príciny neprítele a vymstívajícího se.
\par 18 Všecko to prišlo na nás, a však jsme se nezapomenuli na te, aniž jsme zrušili smlouvy tvé.
\par 19 Neobrátilo se nazpet srdce naše, aniž se uchýlil krok náš od stezky tvé,
\par 20 Ackoli jsi nás byl potrel na míste draku, a prikryl jsi nás stínem smrti.
\par 21 Kdybychom se byli zapomenuli na jméno Boha svého, a pozdvihli rukou svých k bohu cizímu,
\par 22 Zdaliž by toho Buh byl nevyhledával? Nebo on zná skrytosti srdce.
\par 23 Anobrž pro tebe mordováni býváme každého dne, jmíni jsme jako ovce k zabití oddané.
\par 24 Procitiž, proc spíš, ó Pane? Probudiž se, a nezahánej nás na veky.
\par 25 I procež tvár svou skrýváš, a zapomínáš se na trápení a ssoužení naše?
\par 26 Nebote se již sklonila až k prachu duše naše, prilnul k zemi život náš.
\par 27 Povstaniž k našemu spomožení, a vykup nás pro své milosrdenství.

\chapter{45}

\par 1 Prednímu kantoru z synu Chóre, na šošannim, vyucující. Písen o lásce.
\par 2 Vyneslo srdce mé slovo dobré, vypravovati budu písne své o králi, jazyk muj jako péro hbitého písare.
\par 3 Krásnejší jsi nad všecky syny lidské, rozlita jest i milost ve rtech tvých, proto že jest tobe požehnal Buh až na veky.
\par 4 Pripaš mec svuj na bedra, ó reku udatný, prokaž dustojnost a slávu svou.
\par 5 A v té sláve své štastne vyjíždej s slovem pravdy, tichosti a spravedlnosti, a dokáže pravice tvá hrozných vecí.
\par 6 Strely tvé jsou ostré, padati budou od nich pred tebou národové, proniknou až k srdci neprátel královských.
\par 7 Trun tvuj, ó Bože, jest vecný a stálý, berla království tvého jestit berla nejuprímejší.
\par 8 Miluješ spravedlnost, a nenávidíš bezbožnosti, protož pomazal te, Bože, Buh tvuj olejem veselé nad úcastníky tvé.
\par 9 Mirra, aloe a kassia, všecka roucha tvá voní z palácu, z kostí slonových vzdelaných, nad ty, jenž te obveselují.
\par 10 Dcery králu jsou mezi vzácnými tvými, prístojít i manželka tobe po pravici v ryzím zlate.
\par 11 Slyšiž, dcerko, a viz, a naklon ucha svého, a zapomen na lid svuj a na dum otce svého.
\par 12 I zalíbí se králi tvá krása; ont jest zajisté Pán tvuj, protož sklánej se pred ním.
\par 13 Tut i Tyrští s dary, pred oblícejem tvým koriti se budou bohatí národové.
\par 14 Všecka slavná jest dcera královská u vnitrku, roucho zlatem vytkávané jest odev její.
\par 15 V rouše krumpovaném privedena bude králi, i panny za ní, družicky její, privedeny budou k tobe.
\par 16 Privedeny budou s radostí velikou a plésáním, a vejdou na palác královský.
\par 17 Místo otcu svých budeš míti syny své, kteréž postavíš za knížata po vší zemi.
\par 18 V pamet uvoditi budu jméno tvé po všecky veky, procež oslavovati te budou národové na veky veku.

\chapter{46}

\par 1 Prednímu kantoru z synu Chóre, písen na alamot.
\par 2 Buh jest naše útocište i síla, ve všelikém ssoužení pomoc vždycky hotová.
\par 3 A protož nebudeme se báti, byt se pak i zeme podvrátila, a zprevracely se hory do prostred more.
\par 4 Byt i jecely, a kormoutily se vody jeho, a hory se rozrážely od dutí jeho. Sélah.
\par 5 Potok a pramenové jeho obveselují mesto Boží, nejsvetejší z príbytku Nejvyššího.
\par 6 Buh jest u prostred neho, nepohnet se; prispejet jemu Buh na pomoc hned v jitre.
\par 7 Když hluceli národové, a pohnula se království, vydal hlas svuj, a rozplynula se zeme.
\par 8 Hospodin zástupu jest s námi, hradem vysokým jest nám Buh Jákobuv. Sélah.
\par 9 Podte, vizte skutky Hospodinovy, jakýcht jest pustin nadelal na zemi.
\par 10 Prítrž ciní bojum až do koncin zeme, lucište láme, kopí posekává, a vozy spaluje ohnem,
\par 11 Mluve: Upokojtež se, a vezte, žet jsem já Buh, kterýž vyvýšen budu mezi národy, vyvýšen budu na zemi.
\par 12 Hospodin zástupu jest s námi, hradem vysokým jest nám Buh Jákobuv. Sélah.

\chapter{47}

\par 1 Prednímu zpeváku z synu Chóre, žalm.
\par 2 Všickni národové plésejte rukama, trubte Bohu s hlasitým prozpevováním.
\par 3 Nebo Hospodin nejvyšší, hrozný, jest král veliký nade vší zemi.
\par 4 Uvozuje lidi v moc naši, a národy pod nohy naše.
\par 5 Oddelil nám za dedictví naše slávu Jákobovu, kteréhož miloval. Sélah.
\par 6 Vstoupil Buh s troubením, Hospodin s zvukem trouby.
\par 7 Žalmy zpívejte Bohu, zpívejte; zpívejte žalmy králi našemu, zpívejte.
\par 8 Nebo král vší zeme Buh jest, zpívejte žalmy rozumne.
\par 9 Kralujet Buh nad národy, Buh sedí na trunu svém svatém.
\par 10 Knížata národu pripojili se k lidu Boha Abrahamova; nebo pavézy zeme Boží jsou, procež on náramne vyvýšen jest.

\chapter{48}

\par 1 Písen žalmu synu Chóre.
\par 2 Veliký jest Hospodin, a prevelmi chvalitebný v meste Boha našeho, na hore svatosti své.
\par 3 Ozdoba krajiny, útecha vší zeme jestit hora Sion, k strane pulnocní, mesto krále velikého.
\par 4 Buh na palácích jeho, a znají ho býti vysokým hradem.
\par 5 Nebo aj, králové když se shromáždili a spolu táhli,
\par 6 Sami to uzrevše, velmi se divili, a predešeni byvše, náhle utíkali.
\par 7 Tut jest je strach popadl, a bolest jako ženu rodící.
\par 8 Vetrem východním rozrážíš lodí Tarské.
\par 9 Jakž jsme slýchali, tak jsme spatrili, v meste Hospodina zástupu, v meste Boha našeho. Buh upevní je až na veky.
\par 10 Rozjímáme, ó Bože, milosrdenství tvé u prostred chrámu tvého.
\par 11 Jakož jméno tvé, Bože, tak i chvála tvá až do koncin zeme; pravice tvá zajisté plná jest spravedlnosti.
\par 12 Raduj se, horo Sione, plésejte, dcery Judské, z príciny soudu Božích.
\par 13 Obejdete Sion, a obstupte jej, sectete veže jeho.
\par 14 Priložte mysl svou k ohrade, popatrte na paláce jeho, abyste umeli vypravovati veku potomnímu,
\par 15 Že tento Buh jest Buh náš na vecné veky, a že on vudce náš bude až do smrti.

\chapter{49}

\par 1 Prednímu kantoru z synu Chóre, žalm.
\par 2 Slyšte to všickni národové, pozorujte všickni obyvatelé zemští.
\par 3 Tak z lidu obecného, jako z povýšených, tak bohatý, jako chudý.
\par 4 Ústa má mluviti budou moudrost, a premyšlování srdce mého rozumnost.
\par 5 Nakloním k prísloví ucha svého, a pri harfe vykládati budu prípovídku svou.
\par 6 I proc se báti mám ve dnech zlých, aby nepravost tech, kteríž mi na paty šlapají, mne obklíciti mela?
\par 7 Kteríž doufají v svá zboží, a množstvím bohatství svého se chlubí.
\par 8 Žádný bratra svého nijakž vykoupiti nemuže, ani Bohu za nej dáti mzdy vyplacení,
\par 9 (Nebot by velmi drahé musilo býti vyplacení duše jejich, protož nedovedet toho na veky),
\par 10 Aby živ byl vecne, a nevidel porušení.
\par 11 Nebo se vídá, že i moudrí umírají, blázen a hovadný clovek zaroven hynou, zboží svého i cizím zanechávajíce.
\par 12 Myšlení jejich jest, že domové jejich vecní jsou, a príbytkové jejich od národu do pronárodu; procež je po krajinách nazývají jmény svými.
\par 13 Ale clovek v sláve netrvá, jsa podobný hovadum, kteráž hynou.
\par 14 Taková snažnost jejich jest bláznovstvím pri nich, však potomci jejich ústy svými to schvalují. Sélah.
\par 15 Jako hovada v pekle skladeni budou, smrt je žráti bude, ale uprímí panovati budou nad nimi v jitre; zpusob pak onechno aby zvetšel, z príbytku svého octnou se v hrobe.
\par 16 Ale Buh vykoupí duši mou z moci pekla, když mne prijme. Sélah.
\par 17 Neboj se, když by nekdo zbohatl, a když by se rozmnožila sláva domu jeho.
\par 18 Pri smrti zajisté niceho nevezme, aniž sstoupí za ním sláva jeho.
\par 19 Actkoli duši své, pokudž jest živ, lahodí; k tomu chválí jej i jiní, když sobe ciste povoluje:
\par 20 A však musí se odebrati za vekem otcu svých, a na veky svetla neuzrí.
\par 21 Summou: Clovek jsa ve cti, neusrozumí-li sobe, bývá ucinen podobný hovadum, kteráž hynou.

\chapter{50}

\par 1 Žalm Azafovi. Buh silný, Buh Hospodin mluvil, a privolal zemi od východu slunce i od západu jeho.
\par 2 Z Siona v dokonalé kráse Buh zastkvel se.
\par 3 Béret se Buh náš, a nebude mlceti; ohen pred ním vše zžírati bude, a vukol neho vichrice náramná.
\par 4 Zavolal nebes s hury i zeme, aby soudil lid svuj, rka:
\par 5 Shromaždte mi svaté mé, kteríž smlouvu se mnou ucinili pri obetech.
\par 6 I budou vypravovati nebesa spravedlnost jeho; nebo sám Buh soudce jest. Sélah.
\par 7 Slyš, lide muj, a budut mluviti, Izraeli, a budut tebou osvedcovati. Já zajisté Buh, Buh tvuj jsem.
\par 8 Nechci te obvinovati z príciny obetí tvých, ani zápalu tvých, že by vždycky prede mnou nebyli.
\par 9 Nevezmut z domu tvého volka, ani z chlévu tvých kozlu.
\par 10 Nebo má jest všecka zver lesní, i hovada na tisíci horách.
\par 11 Já znám všecko ptactvo po horách, a zver polní pred sebou mám.
\par 12 Zlacním-li, nic tobe o to nedím; nebo muj jest okršlek zemský i plnost jeho.
\par 13 Zdaliž jídám maso z volu, a pijím krev kozlovou?
\par 14 Obetuj Bohu obet chvály, a pln Nejvyššímu své sliby;
\par 15 A vzývej mne v den ssoužení, vytrhnu te, a ty mne budeš slaviti.
\par 16 Sic jinak bezbožníku praví Buh: Což tobe do toho, že ty vypravuješ ustanovení má, a béreš smlouvu mou v ústa svá,
\par 17 Ponevadž jsi vzal v nenávist kázen, a zavrhl jsi za sebe slova má.
\par 18 Vidíš-li zlodeje, hned s ním bežíš, a s cizoložníky díl svuj máš.
\par 19 Ústa svá pouštíš ke zlému, a jazyk tvuj skládá lest.
\par 20 Usazuješ se, a mluvíš proti bratru svému, a na syna matky své lehkost uvodíš.
\par 21 To jsi cinil, a já mlcel jsem. Mel-liž jsi ty se domnívati, že já tobe podobný budu? Nýbrž trestati te budu, a predstavímt to pred oci tvé.
\par 22 Srozumejtež tomu již aspon vy, kteríž se zapomínáte na Boha, abych snad nepochytil, a nebyl by, kdo by vytrhl.
\par 23 Kdož obetuje obet chvály, tent mne uctí, a tomu, kdož napravuje cestu svou, ukáži spasení Boží.

\chapter{51}

\par 1 Prednímu z kantoru, žalm Daviduv,
\par 2 Když k nemu prišel Nátan prorok, po jeho vjití k Betsabé.
\par 3 Smiluj se nade mnou, Bože, podlé milosrdenství svého, podlé množství slitování svých shlad prestoupení má.
\par 4 Dokonale obmej mne od nepravosti mé, a od hríchu mého ocist mne.
\par 5 Nebo já znám prestoupení svá, a hrích muj prede mnou jest ustavicne.
\par 6 Tobe, tobe samému, zhrešil jsem, a zlého se pred ocima tvýma dopustil, abys spravedlivý zustal v recech svých, a bez úhony v soudech svých.
\par 7 Aj, v nepravosti zplozen jsem, a v hríchu pocala mne matka má.
\par 8 Aj, ty libuješ pravdu u vnitrnostech, nadto skrytou moudrost zjevil jsi mi.
\par 9 Vycist mne yzopem, a ocišten budu, umej mne, a nad sníh belejší budu.
\par 10 Dej mi slyšeti radost a potešení, tak at zpléší kosti mé, kteréž jsi potrel.
\par 11 Odvrat tvár svou prísnou od hríšností mých, a vymaž všecky nepravosti mé.
\par 12 Srdce cisté stvor mi, ó Bože, a ducha prímého obnov u vnitrnostech mých.
\par 13 Nezamítej mne od tvári své, a Ducha svatého svého neodjímej ode mne.
\par 14 Navrat mi radost spasení svého, a duchem dobrovolným utvrd mne.
\par 15 I budu vyucovati prestupníky cestám tvým, aby hríšníci k tobe se obraceli.
\par 16 Vytrhni mne z pomsty pro vylití krve, ó Bože, Bože spasiteli muj, a budet s veselím prozpevovati jazyk muj o spravedlnosti tvé.
\par 17 Pane, rty mé otevri, i budou ústa má zvestovati chválu tvou.
\par 18 Nebo neoblíbil bys obeti, bycht ji i dal, aniž bys zápalu prijal.
\par 19 Obeti Boží duch skroušený; srdcem skroušeným a potrebným, Bože, nezhrzíš.
\par 20 Dobrotive nakládej z milosti své s Sionem, vzdelej zdi Jeruzalémské.
\par 21 A tehdáž sobe zalíbíš obeti spravedlnosti, zápaly a pálení celých obetí, tehdážt voly na oltári tvém obetovati budou.

\chapter{52}

\par 1 Prednímu z kantoru, vyucující žalm Daviduv.
\par 2 Když prišel Doeg Idumejský, a zvestoval Saulovi, a povedel mu, že David všel do domu Achimelechova.
\par 3 Proc se chlubíš nešlechetností, ty mocný? Milosrdenstvít Boha silného trvá každého dne.
\par 4 Težkosti obmýšlí jazyk tvuj, tak jako britva nabroušená lest provodí.
\par 5 Miluješ zlé více než dobré, radeji lež mluvíš než spravedlnost. Sélah.
\par 6 Miluješ všelijaké reci k sehlcení, a jazyk ošemetný.
\par 7 I tebet Buh silný zkazí na veky, pochytí te, a vytrhne te z stánku, a vykorení z zeme živých. Sélah.
\par 8 Což spravedliví vidouce, budou se báti a jemu posmívati:
\par 9 Aj, tot jest ten clovek, kterýž neskládal v Bohu síly své, ale doufaje ve množství bohatství svých, zmocnoval se v zlosti své.
\par 10 Já pak budu jako oliva zelenající se v dome Božím; nebot jsem nadeji složil v milosrdenství Božím na veky veku.
\par 11 Oslavovati te budu, Pane, na veky, že jsi to ucinil, a poshovím na jméno tvé, nebot jest vzácné pred oblícejem svatých tvých.

\chapter{53}

\par 1 Prednímu z kantoru na machalat, vyucující žalm Daviduv.
\par 2 Ríká blázen v srdci svém: Není Boha. Porušeni jsou, a ohavnou páší nepravost; není, kdo by cinil dobré.
\par 3 Buh s nebe popatril na syny lidské, aby videl, byl-li by kdo rozumný a hledající Boha.
\par 4 Aj, každý z nich nazpet odšel, naporád neužitecní ucineni jsou; není, kdo by cinil dobré, není ani jednoho.
\par 5 Zdaliž nevedí všickni cinitelé nepravosti, lid muj sžírajíce, jako by chléb jedli? Boha pak nevzývají.
\par 6 Tut se náramne strašiti budou, kdež není strachu. Buh zajisté rozptýlí kosti tech, kteríž te vojensky oblehli; zahanbíš je, nebo Buh pohrdl jimi.
\par 7 Ó by z Siona dáno bylo spasení Izraelovi. Když Buh zase privede zajaté lidu svého, plésati bude Jákob, a veseliti se Izrael.

\chapter{54}

\par 1 Prednímu z kantoru na neginot, vyucující žalm Daviduv,
\par 2 Když prišli Zifejští, a rekli Saulovi: Nevíš-liž, že se David pokrývá u nás?
\par 3 Bože, spasena mne ucin pro jméno své, a v moci své ved pri mou.
\par 4 Bože, slyš modlitbu mou, pozoruj reci úst mých.
\par 5 Nebot jsou cizí povstali proti mne, a ukrutníci hledají duše mé, nepredstavujíce sobe Boha pred oblícej svuj. Sélah.
\par 6 Aj, Buht jest spomocník muj; Pán s temi jest, kteríž jsou podpurcové života mého.
\par 7 Odplat zlým neprátelum mým, v pravde své vyplen je, ó Pane.
\par 8 I budut dobrovolne obeti obetovati, slaviti budu jméno tvé, Hospodine, proto že jest dobré.
\par 9 Nebo z ssoužení všelikého vytrhl mne Buh, nýbrž i pomstu nad neprátely mými videlo oko mé.

\chapter{55}

\par 1 Prednímu kantoru na neginot, vyucující žalm Daviduv.
\par 2 Slyš, ó Bože, modlitbu mou, a neskrývej se pred prosbou mou.
\par 3 Pozoruj a vyslyš mne, nebot naríkám v úpení svém, a kormoutím se,
\par 4 A to pro krik neprítele, pro nátisk bezbožníka; nebot scítají na mne lživé veci, a s vzteklostí se proti mne postavují.
\par 5 Srdce mé bolestí ve mne, a strachové smrti pripadli na mne.
\par 6 Bázen a strach prišel na mne, a hruza prikvacila mne.
\par 7 I rekl jsem: Ó bych mel krídla jako holubice, zaletel bych a poodpocinul.
\par 8 Aj, daleko bych se vzdálil, a prebýval bych na poušti. Sélah.
\par 9 Pospíšil bych ujíti vetru prudkému a vichrici.
\par 10 Zkaz je, ó Pane, zmet jazyk jejich, nebot jsem spatril bezpráví a rozbroj v meste.
\par 11 Dnem i nocí ty veci je obklicují po zdech jeho, a v prostredku jeho jest nepravost a prevrácenost.
\par 12 Težkosti jsou u prostred neho, aniž vychází chytrost a lest z ulic jeho.
\par 13 Nebo ne nejaký neprítel útržky mi cinil, sic jinak snesl bych to; ani ten, kdož mne nenávidí, pozdvihl se proti mne, nebo skryl bych se pred ním:
\par 14 Ale ty, clovece mne rovný, vudce muj a domácí muj;
\par 15 Ješto jsme spolu mile tajné rady držívali, a do domu Božího spolecne chodívali.
\par 16 Ó by je smrt náhle prikvacila, tak aby za živa sstoupiti musili do pekla; nebo jest nešlechetnost v príbytcích jejich a u prostred nich.
\par 17 Já pak k Bohu volati budu, a Hospodin vysvobodí mne.
\par 18 U vecer, i ráno, též o poledni modliti se, a nezbedne volati budu, až i vyslyší hlas muj.
\par 19 Vykoupít duši mou, tak aby v pokoji byla pred válkou proti mne; nebo veliké množství bylo jich pri mne.
\par 20 Vyslyšít Buh silný, a je ssouží, (nebot sedí od vecnosti, Sélah), proto že nenapravují, aniž se bojí Boha.
\par 21 Vztáhl ruce své na ty, kteríž s ním pokoj meli, a zrušil smlouvu svou.
\par 22 Libejší než máslo byla slova úst jeho, ale v srdci boj; mekcejší nad olej reci jeho, a však byly jako mecové.
\par 23 Uvrz na Hospodina bríme své, a ont opatrovati te bude, aniž dopustí, aby na veky pohnut byl spravedlivý.
\par 24 Ale onyno, ty Bože, svedeš do jámy zatracení; lidé zajisté vražedlní a lstiví nedojdou polovice dnu svých, já pak v tebe doufati budu.

\chapter{56}

\par 1 Prednímu z kantoru, o nemé holubici v místech vzdálených, zlatý žalm Daviduv, když ho jali Filistinští v Gát.
\par 2 Smiluj se nade mnou, ó Bože, nebo mne sehltiti chce clovek; každého dne boj veda, ssužuje mne.
\par 3 Sehltiti mne usilují na každý den moji neprátelé; jiste žet jest mnoho válcících proti mne, ó Nejvyšší.
\par 4 Kteréhokoli dne strach mne obklicuje, v tebe doufám.
\par 5 Boha chváliti budu z slova jeho, v Boha doufati budu, aniž se budu báti, aby mi co mohlo uciniti telo.
\par 6 Na každý den slova má prevracejí, proti mne jsou všecka myšlení jejich ke zlému.
\par 7 Spolu se scházejí, skrývají se, a šlepejí mých šetrí, cíhajíce na duši mou.
\par 8 Za nešlechetnost-liž zniknou pomsty? V prchlivosti, ó Bože, smeceš lidi ty.
\par 9 Ty má utíkání v poctu máš, schovej slzy mé do láhvice své, a což bys jich v poctu nemel?
\par 10 A tehdyt obráceni budou zpet neprátelé moji v ten den, když volati budu; tot vím, že Buh pri mne stojí.
\par 11 Ját budu Boha chváliti z slova, Hospodina oslavovati budu z slova jeho.
\par 12 V Boha doufám, nebudu se báti, aby mi co uciniti mohl clovek.
\par 13 Tobe jsem, Bože, ucinil sliby, a protož tobe vzdám chvály.
\par 14 Nebo jsi vytrhl z smrti duši mou, a nohy mé od poklesnutí, tak abych stále chodil pred Bohem v svetle živých.

\chapter{57}

\par 1 Prednímu kantoru, jako: Nevyhlazuj, zlatý žalm Daviduv, když utekl pred Saulem do jeskyne.
\par 2 Smiluj se nade mnou, ó Bože, smiluj se nade mnou; nebot v tebe doufá duše má, a v stínu krídel tvých schráním se, až prejde ssoužení.
\par 3 Volati budu k Bohu nejvyššímu, k Bohu silnému, kterýž dokonává za mne.
\par 4 Ont pošle s nebe, a zachová mne od potupy usilujícího mne sehltiti. Sélah. Pošle Buh milosrdenství své a pravdu svou.
\par 5 Duše má jest u prostred lvu, bydlím mezi palici, mezi lidmi, jejichž zubové kopí a strely, a jazyk jejich jako ostrý mec.
\par 6 Vyvyšiž se nad nebesa, ó Bože, a nade všecku zemi sláva tvá.
\par 7 Tenata roztáhli nohám mým, sklícili duši mou, vykopali prede mnou jámu, ale sami upadli do ní. Sélah.
\par 8 Hotovo jest srdce mé, Bože, hotovo jest srdce mé, zpívati a oslavovati te budu.
\par 9 Probud se, slávo má, probud se, loutno a harfo, když v svitání povstávám.
\par 10 Slaviti te budu mezi lidmi, Pane, žalmy prozpevovati tobe budu mezi národy.
\par 11 Nebo veliké jest až k nebi milosrdenství tvé, a až k nejvyšším oblakum pravda tvá.
\par 12 Vyvyšiž se nad nebesa, ó Bože, a nade všecku zemi sláva tvá.

\chapter{58}

\par 1 Prednímu z kantoru, jako: Nevyhlazuj, Daviduv žalm zlatý.
\par 2 Práve-liž vy, ó shromáždení, spravedlnost vypovídáte? Upríme-liž soudy ciníte, vy synové lidští?
\par 3 Anobrž radeji nepravosti v srdci ukládáte, a násilí rukou svých v zemi této odvažujete.
\par 4 Uchýlili se bezbožníci hned od narození, pobloudili hned od života matky, mluvíce lež.
\par 5 Jed v sobe mají jako jedovatý had, jako lítý had hluchý, kterýž zacpává ucho své,
\par 6 Aby neslyšel hlasu zaklinacu, a carodejníka v cárích vycviceného.
\par 7 Ó Bože, potri jim zuby v ústech jejich, strenovní zuby lvícat tech polámej, Hospodine.
\par 8 Necht se rozplynou jako voda, a zmizejí; at jsou jako ten, kterýž napíná luk, jehož však strely se lámí,
\par 9 Jako hlemejžd, kterýž tratí se a mizí, jako nedochudce ženy, ješto nespatrilo slunce.
\par 10 Prvé než lidé pocítí trní jejich a bodláku, hned za živa zapálením jako vichricí zachváceni budou.
\par 11 I bude se veseliti spravedlivý, když uzrí pomstu, nohy své umyje ve krvi bezbožníka.
\par 12 Ano dí každý: V pravde, žet má užitek spravedlivý, jiste, žet jest Buh soudce na zemi.

\chapter{59}

\par 1 Prednímu z kantoru, jako: Nevyhlazujž, Daviduv žalm zlatý, když poslal Saul, aby strehouc domu, zabili ho.
\par 2 Vytrhni mne od neprátel mých, Bože muj; pred temi, kteríž povstávají proti mne, bezpecna mne ucin.
\par 3 Vytrhni mne od tech, kteríž páší nepravost, a od mužu vražedlných zachovej mne.
\par 4 Neb aj, zálohy ciní duši mé, sbírají se proti mne mocní, bez mého provinení a bez hríchu mého, ó Hospodine.
\par 5 Beze vší mé nepravosti sbíhají se, a strojí; povstaniž mne vstríc, a popatr.
\par 6 Sám ty, Hospodine Bože zástupu, Bože Izraelský, procit, abys navštívil všecky národy, aniž se smilovávej nad kterým z tech prevrácencu nešlechetných. Sélah.
\par 7 Navracejí se k vecerou, štekají jako psi, a behají okolo mesta.
\par 8 Aj, cot vynášejí ústy svými! Mecové jsou ve rtech jejich, nebo ríkají: Zdaliž kdo slyší?
\par 9 Ale ty, Hospodine, smeješ se jim, posmíváš se všechnem národum.
\par 10 Když on moc provozuje, na tebe pozor míti budu, nebo ty, Bože, jsi hrad muj vysoký.
\par 11 Buh mne milosrdný predejdet mne, Buh dá mi videti pomstu nad neprátely mými.
\par 12 Nezbijej jich, aby nezapomnel lid muj, ale zmítej jimi mocí svou, a sházej je, pavézo naše, ó Pane.
\par 13 Hrích úst svých, slova rtu svých, (postiženi jsouce v pýše své, pro prokletí a chradnutí), at vypravují.
\par 14 Zahlad v prchlivosti, zahlad je, at jich není, at poznají, že Buh panuje v Jákobovi i do koncin zeme. Sélah.
\par 15 I nechažt se pak zase navracejí k vecerou, štekají jako psi, a behají okolo mesta.
\par 16 Nechat tekají, a potravy hledají, však hladovití jsouce, uložiti se musejí.
\par 17 Já pak zpívati budu o síle tvé, hned z jitra hlasite slaviti budu milosrdenství tvé, nebo jsi byl hrad muj vysoký, a útocište v den ssoužení mého.
\par 18 A posilen jsa, žalmy tobe zpívati budu; nebo jsi Buh, vysoký hrad muj, Buh mne milosrdný.

\chapter{60}

\par 1 Prednímu z kantoru na šušan eduth, zlatý žalm Daviduv, k vyucování,
\par 2 Když válku vedl proti Syrii Naharaim, a proti Syrii Soba, kdyžto navrátil se Joáb, pobiv Idumejských v údolí slaném dvanácte tisícu.
\par 3 Bože, zavrhl jsi byl nás, roztrhls nás a hnevals se, navratiž se zase k nám.
\par 4 Zatrásl jsi byl zemí a roztrhls ji, uzdraviž rozsedliny její, nebot se chveje.
\par 5 Ukazoval jsi lidu svému tvrdé veci, napájels nás vínem zkormoucení.
\par 6 Ale nyní dal jsi tem, kteríž se tebe bojí, korouhev, aby ji vyzdvihli pro pravdu tvou. Sélah.
\par 7 At jsou vysvobozeni milí tvoji, zachovávejž jich pravicí svou, a vyslyš mne.
\par 8 Buh mluvil skrze svatost svou, veseliti se budu, budu deliti Sichem, a údolí Sochot rozmerím.
\par 9 Mujt jest Galád, muj i Manasses, a Efraim síla hlavy mé, Juda ucitel muj.
\par 10 Moáb medenice k umývání mému, na Edoma uvrhu obuv svou, proti mne, Palestino, trub.
\par 11 Kdo mne uvede do mesta ohraženého? Kdo mne zprovodí až do Idumee?
\par 12 Zdali ne ty, ó Bože, kterýž jsi nás byl zavrhl, a nevycházels, Bože, s vojsky našimi?
\par 13 Udeliž nám pomoci pred neprítelem, nebo marná jest pomoc lidská.
\par 14 V Bohu udatne sobe pocínati budeme, a ont pošlapá neprátely naše.

\chapter{61}

\par 1 Prednímu z kantoru na neginot, Daviduv.
\par 2 Slyš, ó Bože, volání mé, pozoruj modlitby mé.
\par 3 Od konce zeme v sevrení srdce svého k tobe volám, na skálu nade mne vyšší uvediž mne.
\par 4 Nebo jsi býval mé útocište, a pevná veže pred tvárí neprítele.
\par 5 Budut bydliti v stánku tvém na veky, schráním se v skrýši krídel tvých. Sélah.
\par 6 Ty jsi zajisté, Bože, vyslyšel žádosti mé, dal jsi dedictví bojícím se jména tvého.
\par 7 Ke dnum krále více dnu pridej, at jsou léta jeho od národu do pronárodu,
\par 8 At bydlí na veky pred tvárí Boží; milosrdenství a pravdu nastroj, at ho ostríhají.
\par 9 A tak žalmy zpívati budu jménu tvému na veky, a sliby své plniti budu den po dni.

\chapter{62}

\par 1 Prednímu z kantoru Jedutunovi, žalm Daviduv.
\par 2 Vždy predce k Bohu má se mlcelive duše má, od nehot jest spasení mé.
\par 3 Vždyt predce on jest skála má, mé spasení, vysoký hrad muj, nepohnut se škodlive.
\par 4 Až dokud zlé obmýšleti budete proti cloveku? Všickni vy zahubeni budete, jako zed navážená a stena nachýlená jste.
\par 5 Však nic méne radí se o to, jak by jej odstrcili, aby nebyl vyvýšen; lež oblibují, ústy svými dobrorecí, ale u vnitrnosti své proklínají. Sélah.
\par 6 Vždy predce mej se k Bohu mlcelive, duše má, nebo od neho jest ocekávání mé.
\par 7 Ont jest zajisté skála má, mé spasení, vysoký hrad muj, nepohnut se.
\par 8 V Bohu jest spasení mé a sláva má; skála síly mé, doufání mé v Bohu jest.
\par 9 Nadeji v nem skládejte všelikého casu, ó lidé, vylévejte pred oblícejem jeho srdce vaše, Buh útocište naše. Sélah.
\par 10 Jiste žet jsou marnost synové lidští,a synové mocných lživí. Budou-li spolu na váhu vloženi, lehcejší budou nežli marnost.
\par 11 Nedoufejtež v utiskování, ani v loupeži, a nebývejte marní; statku pribývalo-li by, neprikládejte srdce.
\par 12 Jednou mluvil Buh, dvakrát jsem to slyšel, že Boží jest moc,
\par 13 A že tvé, Pane, jest milosrdenství, a že ty odplatíš jednomu každému podlé skutku jeho.

\chapter{63}

\par 1 Žalm Daviduv, když byl na poušti Judské.
\par 2 Bože, Buh silný muj ty jsi, tebet hned v jitre hledám, tebe žízní duše má, po tobe touží telo mé, v zemi žíznivé a vyprahlé, v níž není vody,
\par 3 Abych te v svatyni tvé spatroval, a videl sílu tvou a slávu tvou,
\par 4 (Nebot jest lepší milosrdenství tvé,nežli život), aby te chválili rtové moji,
\par 5 A tak abych tobe dobrorecil, pokudž jsem živ, a ve jménu tvém pozdvihoval rukou svých.
\par 6 Jako tukem a sádlem sytila by se tu duše má, a s radostným rtu prozpevováním chválila by te ústa má.
\par 7 Jiste žet na te pametliv jsem i na ložci svém, každého bdení nocního premýšlím o tobe.
\par 8 Nebo jsi mi býval ku pomoci, protož v stínu krídel tvých prozpevovati budu.
\par 9 Prilnula duše má k tobe, pravice tvá zdržuje mne.
\par 10 Procež ti, kteríž hledají pádu duše mé, sami vejdou do nejvetší hlubokosti zeme.
\par 11 Zabijí každého z nich ostrostí mece, i budou liškám za podíl.
\par 12 Král pak veseliti se bude v Bohu, i každý, kdož skrze neho prisahá, chlubiti se bude; nebo ústa mluvících lež zacpána budou.

\chapter{64}

\par 1 Prednímu z kantoru, žalm Daviduv.
\par 2 Slyš, ó Bože, hlas muj, když naríkám, pred hruzou neprítele ostríhej života mého.
\par 3 Skrej mne pred úklady zlostníku, pred zbourením tech, kteríž páší nepravost,
\par 4 Kteríž naostrili jako mec jazyk svuj, namerili strelu svou, rec prehorkou.
\par 5 Aby stríleli z skrýší na uprímého; nenadále nan vystrelují, aniž se koho bojí.
\par 6 Zatvrzují se ve zlém, smlouvají se, jak by poléci mohli osídla, a ríkají: Kdo je spatrí?
\par 7 Vyhledávají snažne nešlechetnosti, hyneme od ran prelstivých; takt vnitrnost a srdce cloveka hluboké jest.
\par 8 Ale jakž Buh vystrelí na ne prudkou strelu, poraženi budou.
\par 9 A ku pádu je privede vlastní jazyk jejich; vzdálí se jich každý, kdož by je videl.
\par 10 I budou se báti všickni lidé, a ohlašovati skutek Boží, a k srozumívání dílu jeho prícinu dadí.
\par 11 Spravedlivý pak veseliti se bude v Hospodinu, a v nem doufati, anobrž chlubiti se budou všickni, kteríž jsou uprímého srdce.

\chapter{65}

\par 1 Prednímu z kantoru žalm, Davidova písen.
\par 2 Na tobe prestávati sluší, ó Bože, tebe na Sionu chváliti, a tu tobe slib vyplnovati.
\par 3 Tu k tobe modlitbu vyslýchajícímu všeliké telo pricházeti bude.
\par 4 Mnohé nepravosti, kteréž se zmocnily nás, a prestoupení naše ty ocistíš.
\par 5 Blahoslavený, kohož vyvoluješ, a privodíš, aby obýval v síncích tvých. Tut nasyceni budeme dobrým domu tvého, v svatyni chrámu tvého.
\par 6 Predivné veci podlé spravedlnosti nám mluvíš, Bože spasení našeho, nadeje všech koncin zeme i more dalekého,
\par 7 Kterýž upevnuješ hory mocí svou,silou jsa prepásán,
\par 8 Kterýž skrocuješ zvuk more, jecení vlnobití jeho, i bourení se národu,
\par 9 Tak že se báti musejí obyvatelé koncin zázraku tvých, kteréž nastáváním jitra a vecera k plésání privodíš.
\par 10 Navštevuješ zemi a svlažuješ ji, hojne ji obohacuješ. Potok Boží naplnován bývá vodami, i nastrojuješ obilé jejich, když ji tak spravuješ.
\par 11 Záhony její svlažuješ, brázdy její snižuješ, dešti ji obmekcuješ, a zrostlinám jejím požehnání dáváš.
\par 12 Korunuješ rok dobrotivostí svou, a šlepeje tvé kropí tucností;
\par 13 Skropují pastvište po pustinách, tak že i pahrbkové plésáním prepasováni bývají.
\par 14 Priodíny bývají roviny dobytkem, a údolí pristrína obilím, tak že radostí prokrikují, a prozpevují.

\chapter{66}

\par 1 Prednímu z kantoru, písen žalmu. Plésej Bohu všecka zeme.
\par 2 Zpívejte žalmy k sláve jména jeho, ohlašujte slávu a chválu jeho.
\par 3 Rcete Bohu: Jak hrozný jsi v skutcích svých! Pro velikost síly tvé lháti budou tobe neprátelé tvoji.
\par 4 Všecka zeme skláneti se tobe a prozpevovati bude, žalmy zpívati bude jménu tvému. Sélah.
\par 5 Podte a vizte skutky Boží, jak hrozný jest v správe pri synech lidských.
\par 6 Obrátil more v suchost, reku prešli nohou po suše, tut jsme se veselili v nem.
\par 7 Panuje v síle své nade vším svetem, oci jeho spatrují národy, zpurní nebudou míti zniku. Sélah.
\par 8 Dobrorecte národové Bohu našemu, a ohlašujte hlas chvály jeho.
\par 9 Zachoval pri životu duši naši, aniž dopustil, aby se poklesla noha naše.
\par 10 Nebo jsi nás zpruboval, ó Bože, precistil jsi nás, tak jako precišteno bývá stríbro.
\par 11 Uvedl jsi nás byl do leci, krute jsi bedra naše ssoužil,
\par 12 Vsadils cloveka na hlavu naši, vešli jsme byli do ohne i do vody, a však jsi nás vyvedl do rozvlažení.
\par 13 A protož vejdu do domu tvého s zápalnými obetmi, a plniti tobe budu sliby své,
\par 14 Kteréž vyrkli rtové moji, a vynesla ústa má, když jsem byl v ssoužení.
\par 15 Zápaly tucných beranu obetovati budu tobe s kadením, volu i kozlu nastrojím tobe. Sélah.
\par 16 Podte, slyšte, a vypravovati budu, kteríž se koli bojíte Boha, co jest ucinil duši mé.
\par 17 Ústy svými k nemu jsem volal, a vyvyšoval jsem ho jazykem svým.
\par 18 Bycht byl patril k nepravosti srdcem svým, nebyl by vyslyšel Pán.
\par 19 Ale vyslyšelt Buh, a pozoroval hlasu modlitby mé.
\par 20 Požehnaný Buh, kterýž neodstrcil modlitby mé, a milosrdenství svého ode mne neodjal.

\chapter{67}

\par 1 Prednejšímu z kantoru na neginot, žalm k zpívání.
\par 2 Bože, smiluj se nad námi, a požehnej nám, zasvet oblícej svuj nad námi, Sélah,
\par 3 Tak aby poznali na zemi cestu tvou, mezi všemi národy spasení tvé.
\par 4 I budou te oslavovati národové, ó Bože, oslavovati te budou všickni lidé.
\par 5 Veseliti se a prozpevovati budou národové; nebo ty souditi budeš lidi v pravosti, a národy spravovati budeš na zemi. Sélah.
\par 6 I budou te oslavovati národové, ó Bože, oslavovati te budou všickni lidé.
\par 7 Zeme také vydá úrodu svou. Požehnání svého udeliž nám Buh, Buh náš;
\par 8 Požehnej nás Buh, a bojtež se jeho všecky konciny zeme.

\chapter{68}

\par 1 Prednímu z kantoru, Daviduv žalm k zpívání.
\par 2 Povstane Buh, a rozprchnou se neprátelé jeho, a utekou od tvári jeho ti, kteríž ho mají v nenávisti.
\par 3 Jakož rozehnán bývá dým, tak je rozženeš; jakož se rozplývá vosk pred ohnem, tak bezbožní zahynou pred tvárí Boží.
\par 4 Spravedliví pak veselíce se, poskakovati budou pred Bohem, a plésati budou radostí.
\par 5 Prozpevujte Bohu, žalmy zpívejte jménu jeho, vyrovnejte cestu tomu, kterýž se vznáší na oblacích. Hospodin jest jméno jeho, plésejtež pred ním.
\par 6 Otec jest sirotku a ochránce vdov, Buh v príbytku svatém svém.
\par 7 Buh samotné rozmnožuje v domy, vyvodí vezne z okovu, zpurní pak bydliti musejí v zemi vyprahlé.
\par 8 Bože, když jsi predcházel lid svuj, když jsi krácel po poušti, Sélah,
\par 9 Zeme se trásla, též i nebesa rozplývala se pred tvárí Boží, i ta hora Sinai pred prítomností Boží, Boha Izraelského.
\par 10 Deštem štedrosti hojné skropoval jsi, Bože, dedictví své, a když zemdlívalo, ty jsi je zase ocerstvoval.
\par 11 Zástupové tvoji prebývají v nem, kteréžs ty nastrojil dobrotivostí svou pro chudého, ó Bože.
\par 12 Pán dal slovo své, i tech, kteréž potešene zvestovaly, zástup veliký, rkoucích:
\par 13 Králové s vojsky utíkali, utíkali, a doma hlídající delily koristi.
\par 14 Ackoli jste mezi kotly ležeti musili, však jste jako holubice, mající krídla postríbrená, a brky z ryzího zlata.
\par 15 Když Všemohoucí rozptýlí krále v této zemi, zbelíš jako sníh na hore Salmon.
\par 16 Hore veliké, hore v Bázan, hore pahrbkovaté, hore v Bázan.
\par 17 Procež vyskakujete, hory pahrbkovaté? Na tétot hore zalíbilo se Bohu prebývati,jiste žet na ní Hospodin na veky prebývati bude.
\par 18 Vozu Božích jest dvadceti tisícu, mnoho tisícu andelu, Pán pak mezi nimi jako na Sinai v svatyni prebývá.
\par 19 Vstoupil jsi na výsost, jaté jsi vedl vezne, vzal jsi dary pro lidi. I nejzpurnejší k prebývání s námi, Hospodine Bože, privozuješ.
\par 20 Požehnaný Pán, každého dne nás osýpá dary svými, Buh silný spasení našeho. Sélah.
\par 21 Ont jest Buh silný náš, Buh silný k hojnému spasení. Hospodin Pán z smrti vyvodí.
\par 22 Raní zajisté Buh hlavu neprátel svých, a vrch hlavy vlasatý chodícího v hríších svých.
\par 23 Reklte Pán: Zaset vyvedu své, jako z Bázan, zase vyvedu z hlubokosti morské.
\par 24 A protož noha tvá zbrocena bude ve krvi, i jazyk psu tvých krví neprátelskou.
\par 25 Spatrili slavné jití tvé, Bože, jití silného Boha mého a krále mého v svatyni.
\par 26 Napred šli zpeváci, z zadu hrající na nástroje hudebné, u prostred pak devecky bubnující.
\par 27 V shromáždeních dobrorecte Bohu Pánu, kteríž jste z národu Izraelského.
\par 28 Tu at jest Beniamin malický, kterýž je opanoval, tu knížata z Judy a houfové jejich, knížata z Zabulona, i knížata z Neftalíma.
\par 29 Obdaril te Buh tvuj silou. Potvrdiž, Bože, což jsi mezi námi vzdelal,
\par 30 Z chrámu svého nad Jeruzalémem, do nehož tobe prinášeti budou králové dary.
\par 31 Zahub zástup kopidlníku, sebrání mocných vudcu i lidu bujného, pyšne vykracující s kusy stríbra; rozptyl lidi žádostivé válek.
\par 32 Prijdout nejvzácnejší z Egypta, Mourenínská zeme rychle vztáhne ruku svou k Bohu.
\par 33 Království zeme zpívejte Bohu, žalmy zpívejte Pánu, Sélah,
\par 34 Tomu, kterýž se vznáší nad nebem nebes starodávních; aj, vydává hlas svuj, hlas premocný.
\par 35 Dejte cest síly Bohu, jehož dustojnost nad Izraelem, a velikomocnost jeho na oblacích.
\par 36 Prehrozný jsi, ó Bože, z svatých príbytku svých. Buh silný Izraelský, ont dává moc a sílu lidu svému, Buh požehnaný.

\chapter{69}

\par 1 Prednímu z kantoru na šošannim, žalm Daviduv.
\par 2 Vysvobod mne, ó Bože, nebot jsou dosáhly vody až k duši mé.
\par 3 Pohrížen jsem v hlubokém bahne, v nemž dna není; všel jsem do hlubokosti vod, jejichž proud zachvátil mne.
\par 4 Ustal jsem, volaje, vyschlo hrdlo mé, zemdlely oci mé od ohlídání se na te Boha svého.
\par 5 Více jest tech, kteríž mne nenávidí bez príciny, než vlasu hlavy mé; zmocnili se ti, kteríž mne vyhladiti usilují, a jsou neprátelé moji bez mého provinení; to, cehož jsem nevydrel, nahražovati jsem musil.
\par 6 Bože, ty znáš sám nemoudrost mou, a výstupkové moji nejsou skryti pred tebou.
\par 7 Necht nebývají zahanbeni prícinou mou ti, kteríž na te ocekávají, Pane, Hospodine zástupu; necht nepricházejí skrze mne k hanbe ti, kteríž te hledají, ó Bože Izraelský.
\par 8 Nebot pro tebe snáším pohanení, a stud prikryl tvár mou.
\par 9 Cizí ucinen jsem bratrím svým, a cizozemec synum matky své,
\par 10 Proto že horlivost domu tvého snedla mne, a hanení hanejících te na mne pripadla.
\par 11 Když jsem plakal, postem trápiv duši svou, bylo mi to ku potupe obráceno.
\par 12 Když jsem bral na se pytel místo roucha, tehdy jsem jim byl za prísloví.
\par 13 Pomlouvali mne, sedíce v bráne, a písnickou byl jsem tem, kteríž pili víno.
\par 14 Já pak modlitbu svou k tobe odsílám, Hospodine, cast jest dobré líbeznosti tvé. Ó Bože, vedlé množství milosrdenství svého vyslyš mne, pro pravdu svou spasitelnou.
\par 15 Vytrhni mne z bláta, abych nebyl pohrížen; necht jsem vytržen od tech, kteríž mne nenávidí, jako z hlubokostí vod,
\par 16 Aby mne nezachvátili proudové vod, a nesehltila hlubina, ani se nade mnou zavrela prohlubne.
\par 17 Vyslyšiž mne, Hospodine, nebot jest dobré milosrdenství tvé; vedlé množství slitování svých vzhlédniž na mne.
\par 18 A neskrývej tvári své od služebníka svého, nebot mám úzkost; rychle vyslyš mne.
\par 19 Približ se k duši mé, a vyprost ji; pro neprátely mé vykup mne.
\par 20 Ty znáš pohanení mé, a zahanbení mé, i potupu mou, pred tebout jsou všickni neprátelé moji.
\par 21 Pohanení potrelo srdce mé, procež jsem byl v žalosti. Ocekával jsem, zdali by mne kdo politoval, ale žádného nebylo, zdali by kdo potešiti chteli, ale nedockal jsem.
\par 22 Nýbrž místo pokrmu poskytli mi žluci, a v žízni mé napájeli mne octem.
\par 23 Budiž jim stul jejich pred nimi za osídlo, a pokojný zpusob jejich místo síti.
\par 24 At se zatmí oci jejich, aby videti nemohli, a bedra jejich k stálému prived zemdlení.
\par 25 Vylí na ne rozhnevání své, a prchlivost hnevu tvého at je zachvátí.
\par 26 Budiž príbytek jejich pustý, v staních jejich kdo by obýval, at není žádného.
\par 27 Nebo se tomu, jehož jsi ty zbil, protiví, a o bolesti zranených tvých rozmlouvají.
\par 28 Prilož nepravost k nepravosti jejich, a at nepricházejí k spravedlnosti tvé.
\par 29 Necht jsou vymazáni z knihy živých, a s spravedlivými at nejsou zapsáni.
\par 30 Já pak ztrápený jsem, a bolestí sevrený, ale spasení tvé, ó Bože, na míste bezpecném postaví mne.
\par 31 I budut chváliti jméno Boží s prozpevováním, a velebiti je s dekováním.
\par 32 A bude to príjemnejší Hospodinu nežli vul, neb volek rohatý s rozdelenými kopyty.
\par 33 To když uhlédají tiší, radovati se budou, hledajíce Boha, a ožive srdce jejich.
\par 34 Nebot vyslýchá chudé Hospodin, a vezni svými nezhrzí.
\par 35 Chvaltež ho nebesa a zeme, more i všeliký hmyz jejich.
\par 36 Buht zajisté zachová Sion, a vzdelá mesta Judská, i budou tu bydliti, a zemi tu dedicne obdrží.
\par 37 Tolikéž i síme služebníku jeho dedicne jí vládnouti budou, a milující jméno jeho v ní prebývati.

\chapter{70}

\par 1 Prednímu z kantoru, žalm Daviduv, k pripomínání.
\par 2 Bože, abys mne vytrhl, Hospodine, abys mi spomohl, pospeš.
\par 3 Zahanbeni budte, a zapyrte se, kteríž hledají duše mé; zpet obráceni a v potupu dáni budte, kteríž se kochají v neštestí mém.
\par 4 Zpet obráceni budte za to, že mne k hanbe privesti usilují ti, kteríž na mne povolávají: Hahá, hahá.
\par 5 Ale at se radují a veselí v tobe všickni ti, kteríž te hledají, a ti, kteríž milují spasení tvé, at ríkají vždycky: Veleslaven budiž Buh náš.
\par 6 Já pak chudý a nuzný jsem, ó Bože, pospešiž ke mne; pomoc má, a vysvoboditel muj ty jsi, neprodlévejž, Hospodine.

\chapter{71}

\par 1 V tebet, Hospodine, doufám, necht nejsem zahanben na veky.
\par 2 Vedlé spravedlnosti své vytrhni mne, a vyprost mne; naklon ke mne ucha svého, a spas mne.
\par 3 Budiž mi skalou obydlí, na niž bych ustavicne utíkal; prikázal jsi ostríhati mne, nebo skála má i pevnost má ty jsi.
\par 4 Bože muj, vytrhni mne z ruky bezbožníka, z ruky prevráceného a násilníka.
\par 5 Nebo ty jsi má nadeje, Pane; Hospodine, v tebet doufám od své mladosti.
\par 6 Na tebe jsem zpolehl hned od života, z bricha matky mé ty jsi mne vyvedl, v tobe jest chvála má vždycky.
\par 7 Jako zázrak byl jsem mnohým, a však ty jsi mé silné doufání.
\par 8 Ó at jsou naplnena ústa má chválením tebe, pres celý den slavením tebe.
\par 9 Nezamítejž mne v veku starosti; když zhyne síla má, neopouštejž mne.
\par 10 Nebo mluvili neprátelé moji proti mne, a ti, jenž strehou duše mé, radili se spolu,
\par 11 Pravíce: Buh jej opustil, honte a popadnete jej, nebo kdo by ho vytrhl, není žádného.
\par 12 Bože, nevzdalujž se ode mne, Bože muj, prispejž mi na pomoc.
\par 13 Necht jsou zahanbeni, a zhynou protivníci duše mé; prikryti budte lehkostí a hanbou, kteríž hledají pádu mého.
\par 14 Já pak ustavicne cekati, a vždy víc a víc te chváliti budu.
\par 15 Ústa má budou vypravovati spravedlnost tvou, každého dne spasení tvé, ackoli mu poctu nevím.
\par 16 Pristoupe k všelijaké moci Panovníka Hospodina, pripomínati budu tvou vlastní spravedlnost.
\par 17 Bože, ucinil jsi mne od mladosti mé, a až po dnes vypravuji o divných cinech tvých.
\par 18 Protož také i do starosti a šedin, Bože, neopouštej mne, až v známost uvedu ráme tvé tomuto veku, a všechnem potomkum sílu tvou.
\par 19 Nebo spravedlnost tvá, Bože, vyvýšená jest, provodíš zajisté veci veliké. Bože, kdo jest podobný tobe?
\par 20 Kterýž ac jsi mi dal okusiti úzkostí velikých a hrozných, však zase k životu navrátíš mne, a z propastí zeme zase mne vyzdvihneš.
\par 21 Rozmnožíš dustojnost mou, a zase utešíš mne.
\par 22 I ját také budu te slaviti na nástroji hudebném, i pravdu tvou, Bože muj; žalmy tobe zpívati budu na harfe, ó svatý Izraelský.
\par 23 Plésati budou rtové moji, když žalmy zpívati budu tobe, i duše má, kterouž jsi vykoupil.
\par 24 Nadto i jazyk muj každý den vypravovati bude spravedlnost tvou; nebo se zastydeti a zahanbiti musili ti, jenž mého pádu hledali.

\chapter{72}

\par 1 Šalomounovi. Bože, soudy své králi dej, a spravedlnost svou synu královu,
\par 2 Aby soudil lid tvuj v spravedlnosti a chudé tvé v pravosti.
\par 3 Hory prinesou pokoj lidu i pahrbkové v spravedlnosti.
\par 4 Souditi bude chudé z lidu, a vysvobodí syny nuzného, násilníka pak potre.
\par 5 Báti se budou tebe, dokudž slunce a mesíc trvati bude, od národu až do pronárodu.
\par 6 Jako když sstupuje déšt na prisecenou trávu, a jako tiší déštové skrápející zemi:
\par 7 Tak zkvete ve dnech jeho spravedlivý, a bude hojnost pokoje, dokud mesíce stává.
\par 8 Panovati bude od more až k mori, a od reky až do koncin zeme.
\par 9 Pred ním skláneti se budou obyvatelé pustin vyprahlých, a neprátelé jeho prach lízati budou.
\par 10 Králové pri mori a z ostrovu pocty mu prinesou, králové Šebejští a Sabejští dary obetovati budou.
\par 11 Nadto klaneti se jemu budou všickni králové, všickni národové jemu sloužiti budou.
\par 12 Nebo vytrhne nuzného, volajícího a nátisk trpícího, kterýž nemá spomocníka.
\par 13 Smiluje se nad bídným a potrebným, a duše nuzných spasí.
\par 14 Od lsti a násilí vysvobodí duši jejich; nebot jest drahá krev jejich pred ocima jeho.
\par 15 Budet dlouhoveký, a dávati mu budou zlato Arabské, a ustavne za nej se modliti, na každý den jemu dobroreciti budou.
\par 16 Když se vrže hrst obilí do zeme na vrchu hor, klátiti se budou jako Libán klasové jeho, a kvésti budou meštané jako byliny zeme.
\par 17 Jméno jeho bude na veky; dokudž slunce trvá, dediti bude jméno jeho. A požehnání sobe dávajíce v nem všickni národové, budou ho blahoslaviti.
\par 18 Požehnaný Hospodin Buh, Buh Izraelský, kterýž sám ciní divné veci,
\par 19 A požehnané jméno slávy jeho na veky. Budiž také naplnena slávou jeho všecka zeme, Amen i Amen.
\par 20 Skonávají se modlitby Davidovy, syna Izai.

\chapter{73}

\par 1 Žalm Azafovi. Jiste žet jest Buh dobrý Izraelovi, tem, kteríž jsou cistého srdce.
\par 2 Ale nohy mé témer se byly ušinuly, o málo, že by byli sklouzli krokové moji,
\par 3 Když jsem horlil proti bláznivým, vida štestí nešlechetných.
\par 4 Nebo nebývají vázáni až k smrti, ale zustává v cele síla jejich.
\par 5 V práci lidské nejsou, a s lidmi trestáni nebývají.
\par 6 Protož otoceni jsou pýchou jako halží, a ukrutností jako rouchem ozdobným priodíni.
\par 7 Vysedlo tukem oko jejich; majíce hojnost nad pomyšlení srdce,
\par 8 Rozpustilí jsou, a mluví zlostne, o nátisku velmi pyšne mluví.
\par 9 Stavejí proti nebi ústa svá, a jazyk jejich po zemi se vozí.
\par 10 A protož na to prichází lid jeho, když se jim vody až do vrchu nalívá,
\par 11 Že ríkají: Jakt má o tom vedeti Buh silný? Aneb zdaž jest to známé Nejvyššímu?
\par 12 Nebo aj, ti bezbožní jsouce, mají pokoj v svete, a dosahují zboží.
\par 13 Nadarmo tedy v cistote chovám srdce své, a v nevinnosti ruce své umývám.
\par 14 Ponevadž každý den trestán bývám, a kázen prichází na mne každého jitra.
\par 15 Reknu-li: Vypravovati budu veci takové, hle, rodina synu tvých dí, že jsem jim kriv.
\par 16 Chtel jsem to rozumem vystihnouti, ale videlo mi se pracno.
\par 17 Až jsem všel do svatyní Boha silného, tu jsem srozumel poslední veci jejich.
\par 18 Jiste že jsi je na místech plzkých postavil, a uvržeš je v spustliny.
\par 19 Aj, jakt pricházejí na spuštení jako v okamžení! Mizejí a hynou hruzami,
\par 20 Jako snové tomu, kdož procítí; Pane, když je probudíš, obraz ten jejich za nic položíš.
\par 21 Když zhorklo srdce mé, a ledví má bodena byla,
\par 22 Nesmyslný jsem byl, aniž jsem co znal, jako hovádko byl jsem pred tebou.
\par 23 A však vždycky jsem byl s tebou, nebo jsi mne ujal za mou pravici.
\par 24 Podlé rady své ved mne, a potom v slávu prijmeš mne.
\par 25 Kohož bych mel na nebi? A mimo tebe v žádném líbosti nemám na zemi.
\par 26 Ac telo i srdce mé hyne, skála srdce mého, a díl muj Buh jest na veky.
\par 27 Nebo aj, ti, kteríž se vzdalují tebe, zahynou; vytínáš ty, kteríž cizoloží odcházením od tebe.
\par 28 Ale mne nejlépe jest prídržeti se Boha; procež skládám v Panovníku Hospodinu doufání své, abych vypravoval všecky skutky jeho.

\chapter{74}

\par 1 Vyucující, Azafuv. Proc, ó Bože, nás tak do konce zamítáš? Proc roznícena jest prchlivost tvá proti stádci pastvy tvé?
\par 2 Rozpomen se na shromáždení své, jehož jsi od starodávna dobyl a vykoupil, na proutek dedictví svého, na Sion horu tuto, na níž prebýváš.
\par 3 Prispejž k hrozným pustinám. Jak všecko pohubil neprítel v svatyni!
\par 4 Rvali neprátelé tvoji u prostred shromáždení tvých, a na znamení toho zanechali množství korouhví svých.
\par 5 Za hrdinu jmín byl ten, kterýž co nejvýše zdvihl sekeru, roubaje vazbu dríví jeho.
\par 6 A nyní již rezby jeho naporád sekerami a palicemi otloukají.
\par 7 Uvrhli ohen do svatyne tvé, na zem zrítivše, poškvrnili príbytku jména tvého.
\par 8 Rekli v srdci svém: Vyhubme je naporád. Takž vypálili všecky stánky Boha silného v zemi.
\par 9 Znamení svých nevidíme, jižt není proroka, aniž jest mezi námi, kdo by vedel, dokud to stane.
\par 10 I dokudž, ó Bože, útržky ciniti bude odpurce? A neprítel ustavicne-liž rouhati se bude jménu tvému?
\par 11 Proc zdržuješ ruku svou, a pravice své z luna svého nevzneseš?
\par 12 Však jsi ty, Bože, král muj od starodávna, pusobíš hojné spasení u prostred zeme.
\par 13 Ty silou svou rozdelil jsi more, a potrels hlavy draku u vodách.
\par 14 Ty jsi potrel hlavu Leviatanovi, dal jsi jej za pokrm lidu na poušti.
\par 15 Ty jsi otevrel vrchovište a potoky, ty jsi osušil i reky prudké.
\par 16 Tvujt jest den, tvá jest také i noc, svetlo i slunce ty jsi ucinil.
\par 17 Ty jsi založil všecky konciny zeme, léto i zimy ty jsi sformoval.
\par 18 Rozpomeniž se na to, že útržky cinil ten odpurce Hospodinu, a lid bláznivý jak se jménu tvému rouhal.
\par 19 Nevydávejž té zberi duše hrdlicky své,na stádce chudých svých nezapomínej se na veky.
\par 20 Ohlédni se na smlouvu; nebo plní jsou i nejtmavejší koutové zeme peleší ukrutnosti.
\par 21 Nechažt bídní neodcházejí s hanbou, chudý a nuzný at chválí jméno tvé.
\par 22 Povstaniž, ó Bože, a ved pri svou, rozpomen se na pohanení, kteréžt se deje od nesmyslných na každý den.
\par 23 Nezapomínej se na vykrikování svých neprátel, a na hluk proti tobe povstávajících, kterýž se silí ustavicne.

\chapter{75}

\par 1 Prednímu z kantoru, jako: Nezahlazuj, žalm Azafuv, a písen.
\par 2 Oslavujeme te, Bože, oslavujeme; nebo že jest blízké jméno tvé, vypravují to predivní skutkové tvoji.
\par 3 Když prijde cas uložený, já práve souditi budu.
\par 4 Rozplynula se zeme i všickni obyvatelé její, já utvrdím sloupy její. Sélah.
\par 5 Dím bláznum: Nebláznete, a bezbožným: Nevyzdvihujte rohu.
\par 6 Nevyzdvihujte proti Nejvyššímu rohu svých, aniž mluvte tvrdošijne.
\par 7 Nebo ne od východu, ani západu, ani od poušte prichází zvýšení;
\par 8 Ale Buh soudce jednoho ponižuje, a druhého povyšuje.
\par 9 Kalich zajisté jest v rukou Hospodinových, a to vína kalného plný nalitý, z nehož nalévati bude, tak že i kvasnice jeho vyváží, a vypijí všickni bezbožníci zeme.
\par 10 Já pak zvestovati budu skutky Páne na vecnost, žalmy zpívati budu Bohu Jákobovu.
\par 11 A všecky rohy bezbožníku srážím, rohové pak spravedlivého vyvýšeni budou.

\chapter{76}

\par 1 Prednímu z kantoru na neginot, žalm Azafuv a písen.
\par 2 Znám jest Buh v Judstvu, a v Izraeli veliké jméno jeho.
\par 3 V Sálem jest stánek jeho, a obydlé jeho na Sionu.
\par 4 Tamt jest polámal ohnivé šípy lucišt, pavézu a mec, i válku. Sélah.
\par 5 Slavný jsi ucinen a dustojný horami loupeže.
\par 6 V loupež dáni jsou udatní srdcem, zesnuli snem svým, aniž nalezly zmužilé hrdiny síly v rukou svých.
\par 7 Od žehrání tvého, ó Bože Jákobuv, i vuz i kun tvrde zesnuli.
\par 8 Ty jsi, ty velmi hrozný, a kdo jest, ješto by pred tebou ostál v rozhnevání tvém?
\par 9 Když s nebe dáváš slyšeti výpoved svou, zeme se bojí a tichne,
\par 10 Když povstává k soudu Buh, aby zachoval všecky pokorné na zemi. Sélah.
\par 11 Zajisté i hnev cloveka chváliti te musí, a ostatek zurivosti skrotíš.
\par 12 Sliby cinte a plnte Hospodinu Bohu vašemu; kterížkoli jste vukol neho, prinášejte dary Prehroznému.
\par 13 Ont odjímá ducha knížatum, a k hruze jest králum zemským.

\chapter{77}

\par 1 Prednímu kantoru z potomku Jedutunových, s Azafem, žalm.
\par 2 Hlas muj k Bohu, když volám, hlas muj k Bohu, aby ucha naklonil ke mne.
\par 3 V den ssoužení svého Pána hledal jsem, v noci ruce své rozprostíral jsem bez prestání, a nedala se potešiti duše má.
\par 4 Na Boha zpomínal jsem a kormoutil se, premyšloval jsem, a úzkostmi svírán byl duch muj. Sélah.
\par 5 Zdržoval jsi oci mé, aby bdely; potrín jsem byl, aniž jsem mluviti mohl.
\par 6 I pricházeli mi na pamet dnové predešlí, a léta dávní.
\par 7 Rozpomínal jsem se v noci na zpevy své, v srdci svém premyšloval jsem, a zpytoval to duch muj, prave:
\par 8 Zdali na veky zažene Buh? Nikdy-liž již více lásky neukáže?
\par 9 Zdali do konce prestane milosrdenství jeho? A konec vezme slovo od pokolení až do pokolení?
\par 10 Zdali se zapomnel smilovávati Buh silný? Zdaž zadržel v hneve milosrdenství svá? Sélah.
\par 11 I rekl jsem: Tot jest má smrt. Ale ucinít promenu pravice Nejvyššího.
\par 12 Rozpomínati se budu na skutky Hospodinovy, a pripomínati sobe divné ciny tvé, od starodávna.
\par 13 A premyšlovati o všelikém díle tvém, a o skutcích tvých mluviti.
\par 14 Bože, svatá jest cesta tvá. Kdo jest silný, veliký, jako Buh?
\par 15 Ty jsi ten Buh silný, jenž ciníš divné veci; uvedl jsi v známost mezi národy sílu svou.
\par 16 Vysvobodil jsi ramenem lid svuj, syny Jákobovy a Jozefovy. Sélah.
\par 17 Videlyt jsou te vody, Bože, videly te vody, a zstrašily se; pohnuly se také i hlubiny.
\par 18 Vydali povodne oblakové, vydala hrmot nebesa, ano i kamenícko tvé skákalo.
\par 19 Vznelo hrímání tvé po obloze, blýskání osvecovalo okršlek zemský, pohybovala se a trásla zeme.
\par 20 Skrze more byla cesta tvá, a stezky tvé skrze vody veliké, a však šlepejí tvých nebylo znáti.
\par 21 Vedl jsi jako stádo lid svuj skrze Mojžíše a Arona.

\chapter{78}

\par 1 Vyucující, Azafovi. Pozoruj, lide muj, zákona mého, naklonte uší svých k slovum úst mých.
\par 2 Otevru v podobenství ústa svá, vypravovati budu prípovídky starobylé.
\par 3 Co jsme slýchali i poznali, a co nám otcové naši vypravovali,
\par 4 Nezatajíme toho pred syny jejich, kteríž budoucím potomkum svým vypravovati budou chvály Hospodinovy, ano i moc jeho a divné skutky jeho, kteréž cinil.
\par 5 Nebot jest vyzdvihl svedectví v Jákobovi, a zákon vydal v Izraeli, což prikázal otcum našim, aby v známost uvodili synum svým,
\par 6 Aby to poznal vek potomní, synové, kteríž se zroditi meli, a ti povstanouce, aby vypravovali dítkám svým,
\par 7 Aby pokládali v Bohu nadeji svou, a nezapomínali se na skutky Boha silného, ale ostríhali prikázaní jeho,
\par 8 Aby nebývali jako otcové jejich, pokolení zpurné a protivné, národ, kterýž nenapravil srdce svého, a nebyl verný Bohu silnému duch jeho.
\par 9 Jako synové Efraim zbrojní, ac umele z lucište stríleli, však v cas boje zpet se obrátili,
\par 10 Nebo neostríhali smlouvy Boží, a v zákone jeho zpecovali se choditi.
\par 11 Zapomenuli se na ciny jeho, a na divné skutky jeho, kteréž jim ukázal.
\par 12 Pred otci jejich cinil divy v zemi Egyptské, na poli Soan.
\par 13 Rozdelil more, a prevedl je; ucinil, aby stály vody jako hromada.
\par 14 Vedl je ve dne v oblace, a každé noci v jasném ohni.
\par 15 Protrhl skály na poušti, a napájel je jako z propastí velikých.
\par 16 Vyvedl potoky z skály, a ucinil, aby vody tekly jako reky.
\par 17 A však vždy pricíneli hríchu proti nemu, a popouzeli Nejvyššího na poušti.
\par 18 A pokoušeli Boha silného v srdci svém, žádajíce pokrmu podlé líbosti své.
\par 19 A mluvili proti Bohu, rkouce: Zdaliž bude moci Buh silný pripraviti stul na této poušti?
\par 20 Aj, uderilt jest v skálu, a tekly vody, a reky se rozvodnily. Zdali také bude moci dáti chleba? Zdali nastrojí masa lidu svému?
\par 21 A protož uslyšav Hospodin, rozhneval se, a ohen zažžen jest proti Jákobovi, a prchlivost vstoupila na Izraele,
\par 22 Proto že se nedoverili Bohu, a nemeli nadeje v spasení jeho,
\par 23 Ackoli rozkázal oblakum shury, a pruduchy nebeské otevrel,
\par 24 A dštil na ne mannou ku pokrmu, a obilé nebeské dával jim.
\par 25 Chléb mocných jedl clovek, seslal jim pokrmu do sytosti.
\par 26 Obrátil vítr východní u povetrí, a privedl mocí svou vítr polední.
\par 27 I dštil na ne masem jako prachem, a ptactvem pernatým jako pískem morským.
\par 28 Spustil je do prostred vojska jejich, a všudy vukol stanu jejich.
\par 29 I jedli, a nasyceni jsou hojne, a dal jim to, cehož žádali.
\par 30 Ješte nevyplnili žádosti své, ješte pokrm byl v ústech jejich,
\par 31 A v tom prchlivost Boží pripadla na ne, a zbil tucné jejich, a prední Izraelské porazil.
\par 32 S tím se vším vždy ješte hrešili, a neverili predivným skutkum jeho.
\par 33 A protož dopustil na ne, že marne skonali dny své, a léta svá s chvátáním.
\par 34 Když je hubil, jestliže ho hledali, a zase k Bohu silnému hned na úsvite se navraceli,
\par 35 Rozpomínajíce se na to, že Buh byl skála jejich, a Buh silný nejvyšší vykupitel jejich:
\par 36 (Ackoli mu s pochlebenstvím mluvili ústy svými, a jazykem svým lhali jemu.
\par 37 A srdce jejich nebylo uprímé pred ním, aniž se verne meli v smlouve jeho),
\par 38 On jsa milosrdný, odpouštel nepravosti jejich, a nezahladil jich; castokrát odvracel hnev svuj, a nevzbuzoval vší zurivosti své.
\par 39 Nebo pamatoval, že jsou telo, vítr, kterýž odchází, a nenavracuje se zase.
\par 40 Kolikrát jsou ho dráždili na poušti, a k bolesti privodili na pustinách.
\par 41 Týž i týž navracujíce se, pokoušeli Boha silného, a svatému Izraelskému cíle vymerovali.
\par 42 Nepamatovali na moc jeho, a na ten den, v kterémž je vysvobodil z ssoužení,
\par 43 Když cinil v Egypte znamení svá, a zázraky své na poli Soan,
\par 44 Když obrátil v krev reky a potoky jejich, tak že jich píti nemohli.
\par 45 Dopustil na ne smesici žížal, aby je žraly, a žáby, aby je hubily.
\par 46 A dal chroustum úrody jejich, a úsilí jejich kobylkám.
\par 47 Stloukl krupami réví jejich, a stromy fíkové jejich ledem.
\par 48 Vydal krupobití na hovada jejich, a na dobytek jejich uhlí reravé.
\par 49 Poslal na ne prchlivost hnevu svého, rozpálení, zurivost i ssoužení, dopustiv na ne andely zlé.
\par 50 Uprostrannil stezku prchlivosti své, neuchoval od smrti duše jejich, ano i na hovada jejich mor dopustil.
\par 51 A pobil všecko prvorozené v Egypte, prvotiny síly v staních Chamových.
\par 52 Ale lid svuj vyvedl jako ovce, a vodil se s nimi jako s stádem po poušti.
\par 53 Vodil je v bezpecnosti, tak že nestrašili, neprátely pak jejich prikrylo more,
\par 54 Až je privedl ku pomezí svatosti své, na horu tu, kteréž dobyla pravice jeho.
\par 55 Vyhnav pred tvárí jejich národy, zpusobil to, aby jim na provazec dedictví jejich prišli, a aby prebývala v staních jejich pokolení Izraelská.
\par 56 Však vždy predce pokoušeli a dráždili Boha nejvyššího, a svedectví jeho neostríhali.
\par 57 Ale zpet odšedše, prevrácene cinili, jako i predkové jejich; uchýlili se jako mylné lucište.
\par 58 Nebo popouzeli ho výsostmi svými, a rytinami svými k horlení privedli jej.
\par 59 Slyšel Buh, a rozhneval se, a u velikou ošklivost vzal Izraele,
\par 60 Tak že opustiv príbytek v Sílo, stánek, kterýž postavil mezi lidmi,
\par 61 Vydal v zajetí sílu svou, a slávu svou v ruce neprítele.
\par 62 Dal pod mec lid svuj, a na dedictví své se rozhneval.
\par 63 Mládence jeho sežral ohen, a panny jeho nebyly chváleny.
\par 64 Kneží jejich od mece padli, a vdovy jejich neplakaly.
\par 65 Potom pak procítil Pán jako ze sna, jako silný rek, kterýž po víne sobe vykrikuje.
\par 66 A ranil neprátely své po zadu, a u vecné pohanení je vydal.
\par 67 Ackoli pak pohrdl stánkem Jozefovým, a pokolení Efraimova nevyvolil,
\par 68 Však vyvolil pokolení Judovo, horu Sion, kterouž zamiloval.
\par 69 A vystavel sobe, jako hrad vysoký, svatyni svou, jako zemi, kterouž utvrdil na veky.
\par 70 A vyvolil Davida služebníka svého, vzav jej od chlévu stáda.
\par 71 Když chodil za ovcemi brezími, zavedl jej, aby pásl Jákoba, lid jeho, a Izraele, dedictví jeho.
\par 72 Kterýž pásl je v uprímnosti srdce svého, a zvláštní opatrností rukou svých vodil je.

\chapter{79}

\par 1 Žalm Azafovi. Bože, vtrhli pohané do dedictví tvého, poškvrnili chrámu svatosti tvé, obrátili Jeruzalém v hromady.
\par 2 Dali tela mrtvá služebníku tvých za pokrm ptákum nebeským, tela svatých tvých šelmám zemským.
\par 3 Vylili krev jejich jako vodu okolo Jeruzaléma, a nebyl, kdo by je pochovával.
\par 4 Vydáni jsme v pohanení sousedum našim, v posmech a žert tem, kteríž jsou vukol nás.
\par 5 Až dokud, ó Hospodine? Na veky-liž se hnevati budeš, a horeti bude jako ohen horlení tvé?
\par 6 Vylí hnev svuj na národy, kteríž te neznají, a na království, kteráž jména tvého nevzývají.
\par 7 Nebot jsou sežrali Jákoba, a obydlí jeho v poustku obrátili.
\par 8 Nezpomínejž nám drevních nepravostí našich, rychle at predejdou nás milosrdenství tvá, nebot jsme velmi znuzeni.
\par 9 Pomoz nám, ó Bože spasení našeho, pro slávu jména svého; vytrhni nás, a bud milostiv hríchum našim pro jméno své.
\par 10 Proc mají ríkati pohané: Kdež jest Buh jejich? Budiž znám mezi pohany, pred ocima našima, skrze pomstu krve služebníku svých, kteráž jest vylita.
\par 11 Vstupiž pred oblícej tvuj lkání veznu, a podlé velikosti síly své zanechej ostatku k smrti oddaných.
\par 12 A odplat sousedum našim sedmernásobne do luna jejich za pohanení, kteréž jsou tobe cinili, ó Pane.
\par 13 My pak, lid tvuj a ovce pastvy tvé, slaviti te budeme na veky; od národu do pronárodu vypravovati budeme chválu tvou.

\chapter{80}

\par 1 Prednímu z kantoru na šošannim, žalm svedectví, Azafovi.

\chapter{80}

\par 1 Prednímu z kantoru na šošannim, žalm svedectví, Azafovi.
\par 3 Pred Efraimem, Beniaminem a Manasse vzbud moc svou, a prispej k spasení našemu.
\par 4 Ó Bože, navrat nás, a dejž, at nám svítí oblícej tvuj, a spaseni budeme.
\par 5 Hospodine Bože zástupu, dokudž se prísne staveti budeš k modlitbám lidu svého?
\par 6 Nakrmil jsi je chlebem pláce, a dals jim vypiti slz míru velikou.
\par 7 Postavils nás k sváru sousedum našim, a neprátelé naši aby sobe posmech z nás cinili.
\par 8 Ó Bože zástupu, navrat nás, a dej, at nám svítí oblícej tvuj, a spaseni budeme.
\par 9 Ty jsi kmen vinný z Egypta prenesl, vyhnal jsi pohany, a vsadils jej.
\par 10 Uprázdnil jsi mu, a ucinils, aby se vkorenil, i zemi naplnil.
\par 11 Prikryty jsou hory stínem jeho, a réví jeho jako nejvyšší cedrové.
\par 12 Vypustil ratolesti své až k mori, a až do reky rozvody své.
\par 13 I procež jsi rozboril hradbu vinice, tak aby každý, kdož by tudy šel, trhati z ní mohl?
\par 14 Vepr divoký zryl ji, a zver polní spásla ji.
\par 15 Ó Bože zástupu, obrat se již, popatr s nebe, viz a navštev kmen vinný tento,
\par 16 Vinici tu, kterouž štípila pravice tvá, a mladistvé réví, kteréž jsi zmocnil sobe.
\par 17 Popálenat jest ohnem a poroubána, od žehrání oblíceje tvého hyne.
\par 18 Budiž ruka tvá nad mužem pravice tvé, nad synem cloveka, kteréhož jsi zmocnil sobe,
\par 19 A neodstoupímet od tebe; zachovej nás pri životu, at jméno tvé vzýváme.
\par 20 Hospodine Bože zástupu, navratiž nás zase, a dej, at nám svítí oblícej tvuj, a spaseni budeme.

\chapter{81}

\par 1 Prednímu z kantoru na gittit, Azafovi.
\par 2 Plésejte Bohu, síle naší, prokrikujte Bohu Jákobovu.
\par 3 Vezmete žaltár, pridejte buben, harfu libou a loutnu.
\par 4 Trubte trubou na novmesíce, v uložený cas, v den slavnosti naší.
\par 5 Nebo tot jest ustavení v Izraeli, rád Boha Jákobova.
\par 6 Na svedectví v Jozefovi vyzdvihl jej, když byl vyšel proti zemi Egyptské, kdež jsme jazyk neznámý slýchati musili.
\par 7 Osvobodil jsem, dí Buh, od bremene rameno jeho, a ruce jeho nádob zednických zprošteny byly.
\par 8 V ssoužení tom, když jsi volal, vytrhl jsem te, vyslyšel jsem te z skrýše hromu, zkušoval jsem te pri vodách sváru. Sélah.
\par 9 Reklt jsem: Slyš, lide muj, a osvedcím se tobe, ó Izraeli, budeš-li mne poslouchati,
\par 10 A nebude-li mezi vámi Boha jiného, a nebudeš-li se klaneti bohu cizímu.
\par 11 Já jsem Hospodin Buh tvuj, kterýž jsem te vyvedl z zeme Egyptské, otevri jen ústa svá, a naplnímt je.
\par 12 Ale neuposlechl lid muj hlasu mého, a Izrael neprestal na mne,
\par 13 A protož pustil jsem je v žádost srdce jejich, i chodili po radách svých.
\par 14 Ó byt mne byl lid muj poslouchal, a Izrael po cestách mých chodil,
\par 15 Tudíž bych já byl neprátely jejich snížil, a na protivníky jejich obrátil ruku svou.
\par 16 A ti, kteríž v nenávisti mají Hospodina, úlisne by se jim poddávati musili, i byl by cas jejich až na veky.
\par 17 A krmil bych je byl jádrem pšenice, a medem z skály sytil bych je.

\chapter{82}

\par 1 Žalm Azafuv. Buh stojí v shromáždení Božím, u prostred bohu soud cine, a dí:
\par 2 Dokudž souditi budete nespravedlive, a osoby nešlechetných prijímati? Sélah.
\par 3 Zastávejte bídného a sirotka, utišteného a chudého spravedliva vyhlašujte.
\par 4 Vytrhnete bídného a nuzného, z ruky nešlechetných vytrhnete ho.
\par 5 Ale nevedí nic, nerozumejí nic; ve tmách ustavne chodí, až se proto všickni základové zeme pohybují.
\par 6 Reklt jsem já byl: Bohové jste, a synové Nejvyššího vy všickni;
\par 7 A však jako i jiní lidé zemrete, a jako jeden z knížat padnete.
\par 8 Povstaniž, ó Bože, sud zemi; nebo ty dedicne vládneš všemi národy.

\chapter{83}

\par 1 Písen a žalm Azafuv.

\chapter{83}

\par 1 Písen a žalm Azafuv.
\par 3 Nebo aj, neprátelé tvoji se bourí, a ti, kteríž te v nenávisti mají, pozdvihují hlavy.
\par 4 Chytre tajné rady proti lidu tvému skládají, a radí se proti tem, kteréž ty skrýváš,
\par 5 Ríkajíce: Podte, a vyhladme je, at nejsou národem, tak aby ani zpomínáno nebylo více jména Izraelova.
\par 6 Jednomyslnet se na tom spolu snesli, i smlouvou se proti tobe zavázali,
\par 7 Stánkové Idumejští a Izmaelitští, Moábští a Agarenští,
\par 8 Gebálští a Ammonitští, a Amalechitští, Filistinští s obyvateli Tyrskými.
\par 9 Ano i Assyrští spojili se s nimi, jsouce na ruku synum Lotovým. Sélah.
\par 10 Uciniž jim jako Madianským, jako Zizarovi, a jako Jabínovi pri potoku Císon,
\par 11 Kteríž jsou do konce vyhlazeni v Endor, a ucineni hnuj zeme.
\par 12 Nalož s nimi a s vudci jejich jako s Gorébem, jako s Zébem, jako s Zebahem, a jako s Salmunou, se všemi knížaty jejich.
\par 13 Nebot jsou rekli: Uvažme se dedicne v príbytky Boží.
\par 14 Muj Bože, ucin to, at jsou jako chumelice, a jako stéblo pred vetrem.
\par 15 Jakož ohen spaluje les, a jako plamen zapaluje hory,
\par 16 Tak ty je vichricí svou stihej, a bourí svou ohrom je.
\par 17 Napln tváre jejich zahanbením, tak aby hledali jména tvého, Hospodine.
\par 18 Nechat se hanbí a desí na vecné casy, a at potupu nesou a zahynou.
\par 19 A tak at poznají, že ty, kterýž sám jméno máš Hospodin, jsi nejvyšší nade vší zemí.

\chapter{84}

\par 1 Prednímu z kantoru na gittit, synu Chóre, žalm.
\par 2 Jak jsou milí príbytkové tvoji, Hospodine zástupu!
\par 3 Žádostiva jest a velice touží duše má po síncích Hospodinových; srdce mé, i telo mé pléše k Bohu živému.
\par 4 Ano i ten vrabec nalezl sobe místo a vlaštovice hnízdo, v nemž by schránila mladé své, pri oltárích tvých, Hospodine zástupu, králi muj a Bože muj.
\par 5 Blahoslavení, kteríž prebývají v dome tvém, tit tebe na veky chváliti budou. Sélah.
\par 6 Blahoslavený clovek, jehož síla jest Hospodin, a v jejichž srdci jsou stezky kroku jejich,
\par 7 Ti, kteríž jdouce pres údolí moruší, za studnici jej sobe pokládají, na než i déšt požehnání prichází.
\par 8 Berou se houf za houfem, a ukazují se pred Bohem na Sionu.
\par 9 Hospodine Bože zástupu, vyslyš modlitbu mou, pozoruj, ó Bože Jákobuv. Sélah.
\par 10 Pavézo naše, popatr, ó Bože, a viz tvár pomazaného svého.
\par 11 Nebo lepší jest den v síncích tvých, než jinde tisíc; zvolil jsem sobe radeji u prahu sedeti v dome Boha svého, nežli prebývati v stáncích bezbožníku.
\par 12 Nebo Hospodin Buh jest slunce a pavéza; tut milosti i slávy udílí Hospodin, aniž odepre ceho dobrého chodícím v uprímnosti.
\par 13 Hospodine zástupu, blahoslavený clovek, kterýž nadeji skládá v tobe.

\chapter{85}

\par 1 Prednímu z kantoru, synu Chóre, žalm.
\par 2 Laskaves se, Hospodine, nekdy ukazoval k zemi své, privedls zase z vezení Jákoba.
\par 3 Odpustil jsi nepravost lidu svého, prikryls všeliký hrích jejich. Sélah.
\par 4 Zdržels všecken hnev svuj, odvrátils od zurivosti prchlivost svou.
\par 5 Navratiž se zase k nám, ó Bože spasení našeho, a ucin prítrž hnevu svému proti nám.
\par 6 Zdaliž na veky hnevati se budeš na nás? A protáhneš zurivost svou od národu do pronárodu?
\par 7 Zdaliž ty obráte se, neobživíš nás, tak aby se lid tvuj veselil v tobe?
\par 8 Ukaž nám, Hospodine, milosrdenství své, a spasení své dej nám.
\par 9 Ale poslechnu, co ríká Buh ten silný, Hospodin. Jiste žet mluví pokoj k lidu svému, a k svatým svým, než aby se nenavracovali zase k bláznovství.
\par 10 Zajisté žet jest blízké tem, kteríž se ho bojí, spasení jeho, a prebývati bude sláva v zemi naší.
\par 11 Milosrdenství a víra potkají se spolu, spravedlnost a pokoj dadí sobe políbení.
\par 12 Víra z zeme puciti se bude, a spravedlnost s nebe vyhlédati.
\par 13 Dát také Hospodin i casné dobré, tak že zeme naše vydá úrody své.
\par 14 Zpusobí to, aby spravedlnost pred ním šla, když obrátí k ceste nohy své.

\chapter{86}

\par 1 Modlitba Davidova. Naklon, Hospodine, ucha svého, a vyslyš mne, nebot jsem chudý a nuzný.
\par 2 Ostríhejž duše mé, nebot jsem ten, jehož miluješ; zachovej služebníka svého, ty Bože muj, v tobe nadeji majícího.
\par 3 Smiluj se nade mnou, Hospodine, k tobet zajisté každého dne volám.
\par 4 Poteš duše služebníka svého, nebot k tobe, ó Pane, duše své pozdvihuji.
\par 5 Nebo ty jsi, Pane, dobrotivý a lítostivý, a hojný v milosrdenství ke všechnem, kteríž te vzývají.
\par 6 Slyš, Hospodine, modlitbu mou, a pozoruj hlasu žádostí mých.
\par 7 V den ssoužení svého vzývám te, nebo mne vyslýcháš.
\par 8 Nenít žádného tobe podobného mezi bohy, ó Pane, a není takových skutku, jako jsou tvoji.
\par 9 Všickni národové, kteréž jsi ucinil, pricházejíce, skláneti se budou pred tebou, Pane, a ctíti jméno tvé.
\par 10 Nebo jsi ty veliký, a ciníš divné veci, ty jsi Buh sám.
\par 11 Vyuc mne, Hospodine, ceste své, abych chodil v pravde tvé; ustav srdce mé v bázni jména svého.
\par 12 I budu te oslavovati, Pane Bože muj, z celého srdce svého a ctíti jméno tvé na veky,
\par 13 Ponevadž milosrdenství tvé veliké jest nade mnou, a vytrhls duši mou z jámy nejhlubší.
\par 14 Ó Bože, povstalit jsou pyšní proti mne, a rota násilníku hledají bezživotí mého, ti, kteríž te sobe nepredstavují.
\par 15 Ale ty Pane, Bože silný, lítostivý a milostivý, shovívající a hojný v milosrdenství, i pravdomluvný,
\par 16 Vzhlédniž na mne, a smiluj se nade mnou, obdar silou svou služebníka svého, a zachovávej syna devky své.
\par 17 Prokaž ke mne znamení dobrotivosti, tak aby vidouce to ti, kteríž mne nenávidí, zahanbeni byli, že jsi ty mi, Hospodine, spomohl, a mne potešil.

\chapter{87}

\par 1 Synum Chóre, žalm a písen. Základ svuj na horách svatých.
\par 2 Milujet Hospodin, totiž brány Sionské, nade všecky príbytky Jákobovy.
\par 3 Preslavnét jsou to veci, kteréž se o tobe hlásají, ó mesto Boží. Sélah.
\par 4 Pripomínati budu Egypt a Babylon pred svými známými, ano i Filistinské a Tyrské i Moureníny, že se tu každý z nich narodil.
\par 5 An i o Sionu praveno bude: Ten i onen jest rodem z neho, sám pak Nejvyšší utvrdí jej.
\par 6 Sectet Hospodin pri popisu národy, a dí, že tento se tu narodil. Sélah.
\par 7 Tou prícinou zpívají s plésáním o tobe všecky moci života mého.

\chapter{88}

\par 1 Písen a žalm synu Chóre, prednímu zpeváku na machalat k zpívání, vyucující, složený od Hémana Ezrachitského.
\par 2 Hospodine, Bože spasení mého, ve dne i v noci k tobe volám.
\par 3 Vstupiž pred oblícej tvuj modlitba má, naklon ucha svého k volání mému.
\par 4 Nebot jest naplnena trápeními duše má, a život muj až k hrobu se priblížil.
\par 5 Pocten jsem mezi ty, kteríž se dostávají do jámy; pripodobnen jsem cloveku beze vší síly.
\par 6 Mezi mrtvé jsem odložen, jako zmordovaní ležící v hrobe, na než nezpomínáš více, kteríž od ruky tvé vyhlazeni jsou.
\par 7 Spustils mne do jámy nejzpodnejší, do nejtemnejšího a nejhlubšího místa.
\par 8 Dolehla na mne prchlivost tvá, a vším vlnobitím svým prikvacil jsi mne. Sélah.
\par 9 Daleko jsi vzdálil mé známé ode mne, jimž jsi mne velice zošklivil, a tak jsem sevrín, že mi nelze nijakž vyjíti.
\par 10 Zrak muj hyne trápením; na každý den vzývám te, Hospodine, ruce své pred tebou rozprostíraje.
\par 11 Zdali pred mrtvými uciníš zázrak? Aneb vstanou-liž mrtví, aby te oslavovali?Sélah.
\par 12 I zdali bude ohlašováno v hrobe milosrdenství tvé, a pravda tvá v zahynutí?
\par 13 Zdaliž v známost prichází ve tmách div tvuj, a spravedlnost tvá v zemi zapomenutí?
\par 14 Já pak, Hospodine, k tobe volám, a každého jitra predchází te modlitba má.
\par 15 Procež, ó Hospodine, zamítáš mne, a tvár svou skrýváš prede mnou?
\par 16 Ztrápený jsem, jako hned maje umríti od násilí; snáším hruzy tvé, a desím se.
\par 17 Hnev tvuj prísný na mne se oboril, a hruzy tvé krute sevrely mne.
\par 18 Obklicují mne jako voda, na každý den obstupují mne hromadne.
\par 19 Vzdálil jsi ode mne prítele a tovaryše, a známým svým jsem ve tme.

\chapter{89}

\par 1 Vyucující, složený od Etana Ezrachitského.
\par 2 O milosrdenstvích Hospodinových na veky zpívati budu, od národu do pronárodu zvestovati budu pravdu tvou ústy svými.
\par 3 Nebo jsem rekl: Na veky milosrdenství vzdelávati se bude, na nebi utvrdíš pravdu svou, o nížs rekl:
\par 4 Ucinil jsem smlouvu s vyvoleným svým, prisáhl jsem Davidovi služebníku svému,
\par 5 Že až na veky utvrdím síme tvé, a vzdelám od národu do národu trun tvuj. Sélah.
\par 6 Protož oslavují nebesa div tvuj, Hospodine, i pravdu tvou v shromáždení svatých.
\par 7 Nebo kdo na nebi prirovnán býti muže Hospodinu? Kdo jest podobný Hospodinu mezi syny silných?
\par 8 Buh i v shromáždení svatých strašlivý jest náramne, a hrozný nade všecky vukol neho.
\par 9 Hospodine Bože zástupu, kdo jest jako ty, silný Hospodin? Nebo pravda tvá tobe prístojí všudy vukol.
\par 10 Ty panuješ nad dutím more; když se zdvihají vlny jeho, ty je skrocuješ.
\par 11 Ty jsi jako raneného potrel Egypt, a silným ramenem svým rozptýlil jsi neprátely své.
\par 12 Tvát jsou nebesa, tvá také i zeme, okršlek i plnost jeho ty jsi založil.
\par 13 Pulnocní i polední strana, kteréž jsi ty stvoril, i Tábor a Hermon o tvém jménu zpívají.
\par 14 Tvé ráme jest premocné, silná ruka tvá, a vyvýšená pravice tvá.
\par 15 Spravedlnost a soud jsou základem trunu tvého, milosrdenství a pravda predcházejí tvár tvou.
\par 16 Blahoslavený lid, kterýž zná zvuk tvuj; tit, Hospodine, v svetle oblíceje tvého choditi budou.
\par 17 Ve jménu tvém plésati budou každého dne, a v spravedlnosti tvé vyvýší se.
\par 18 Nebo sláva síly jejich ty jsi, a z milosti tvé k zvýšení prijde roh náš.
\par 19 Nebo štít náš jest Hospodinuv, a svatého Izraelského král náš.
\par 20 Tehdy mluve u videní k svatému svému, rekl jsi: Složil jsem pomoc v reku udatném, zvýšil jsem vybraného z lidu.
\par 21 Nalezl jsem Davida služebníka svého, olejem svým svatým pomazal jsem ho.
\par 22 A protož budet s ním stále ruka má, ano i ramenem svým posilovati ho budu.
\par 23 Nebudet ho moci nuziti neprítel, ani clovek nešlechetný trápiti.
\par 24 Nebo potru pred tvárí jeho protivníky jeho, a ty, kteríž ho nenávidí, porazím.
\par 25 Nadto pravda má a milosrdenství mé s ním bude, a ve jménu mém vyvýšen bude roh jeho.
\par 26 A vložím na more ruku jeho, a na reky pravici jeho.
\par 27 On volaje ke mne, dí: Ty jsi otec muj, Buh silný muj a skála spasení mého.
\par 28 Já také za prvorozeného vystavím jej, a za vyššího králu zemských.
\par 29 Na veky zachovám jemu milosrdenství své, a smlouva s ním stálá bude.
\par 30 Uciním i to, aby na veky trvalo síme jeho, a trun jeho jako dnové nebes.
\par 31 Jestliže by pak synové jeho opustili zákon muj, a v soudech mých nechodili,
\par 32 Jestliže by ustanovení mých poškvrnili, a prikázaní mých neostríhali:
\par 33 Tedy navštívím metlou prestoupení jejich, a trestáním nepravost jejich,
\par 34 Ale milosrdenství svého neodejmu od neho, aniž klamati budu proti pravde své.
\par 35 Nepoškvrnímt smlouvy své, a toho, což vyšlo z úst mých, nezmením.
\par 36 Jednou jsem prisáhl skrze svatost svou, nesklamámt Davidovi,
\par 37 Že síme jeho na veky bude, a trun jeho jako slunce prede mnou,
\par 38 Jako mesíc utvrzeno bude na veky, a jako svedkové na obloze hodnoverní.
\par 39 Ale ty jsi jej zavrhl a potupil, rozhnevals se na pomazaného svého.
\par 40 Zavrhl jsi smlouvu s služebníkem svým, povrhls korunu jeho na zem.
\par 41 Roztrhal jsi všecky ohrady jeho, a bašty jeho jsi rozválel.
\par 42 Derou jej všickni, kteríž tudy jdou; jest ku posmechu i sousedum svým.
\par 43 Vyvýšil jsi pravici protivníku jeho, obveselils všecky neprátely jeho.
\par 44 Ztupils i ostrí mece jeho, aniž jsi dal jemu, aby ostáti mohl v boji.
\par 45 Ucinils prítrž okrase jeho, a trun jeho svrhl jsi na zem.
\par 46 Ukrátil jsi dnu mladosti jeho, a hanbous jej priodíl. Sélah.
\par 47 Až dokud, Hospodine? Na veky-liž se skrývati budeš? Tak-liž horeti bude jako ohen prchlivost tvá?
\par 48 Rozpomeniž se na mne, jak kratický jest vek muj. Zdaliž jsi pak nadarmo stvoril všecky syny lidské?
\par 49 Kdo z lidí muže tak živ býti, aby neokusil smrti? Kdo vytrhne život svuj z hrobu? Sélah.
\par 50 Kdež jsou milosrdenství tvá první, ó Pane? Prísahut jsi ucinil Davidovi, v pravde své.
\par 51 Pamatuj, Pane, na útržky cinené služebníkum tvým, a jak jsem já nosil v lune svém potupu ode všech nejmocnejších národu,
\par 52 Jak jsou utrhali neprátelé tvoji, Hospodine, jak jsou utrhali šlepejím pomazaného tvého.
\par 53 Budiž pochválen Hospodin na veky, Amen i Amen.

\chapter{90}

\par 1 Modlitba Mojžíše, muže Božího. Pane, ty jsi býval príbytek náš od národu do pronárodu.
\par 2 Prvé, než hory stvoreny byly, nežlis sformoval zemi, a okršlek sveta, ano hned od veku a až na veky, ty jsi Buh silný.
\par 3 Ty privodíš cloveka na to, aby setrín byl, ríkaje: Navrattež se zase, synové lidští.
\par 4 Nebo by tisíc let pretrval, jest to pred ocima tvýma jako den vcerejší, a bdení nocní.
\par 5 Povodní zachvacuješ je; jsou sen, a jako bylina hned v jitre pomíjející.
\par 6 Toho jitra, kteréhož vykvetne, mení se, u vecer pak jsuc podtata, usychá.
\par 7 Ale my hyneme od hnevu tvého, a prochlivostí tvou jsme zdešeni.
\par 8 Nebo jsi položil nepravosti naše pred sebe, a tajnosti naše na svetlo oblíceje svého.
\par 9 Procež všickni dnové naši v náhle prebíhají pro tvé rozhnevání; k skoncení let svých docházíme jako rec.
\par 10 Všech dnu let našich jest let sedmdesáte, aneb jest-li kdo silnejšího prirození, osmdesát let, a i to, což nejzdárnejšího v nich, jest práce a bída, a když to pomine, tožt ihned rychle zaletíme.
\par 11 Ale kdo jest, ješto by znal prísnost hnevu tvého, a ostýchal se zurivosti tvé?
\par 12 Nauciž nás pocítati dnu našich, abychom uvodili moudrost v srdce.
\par 13 Navrat se zase, Hospodine, až dokud prodléváš? Mejž lítost nad služebníky svými.
\par 14 Nasyt nás hned v jitre svým milosrdenstvím, tak abychom prozpevovati, a veseliti se mohli po všecky dny naše.
\par 15 Obveseliž nás podlé dnu, v nichž jsi nás ssužoval, a let, v nichž jsme okoušeli zlého.
\par 16 Budiž zrejmé pri služebnících tvých dílo tvé, a okrasa tvá pri synech jejich.
\par 17 Budiž nám prítomná i ochotnost Hospodina Boha našeho, a díla rukou našich potvrd mezi námi, díla, pravím, rukou našich potvrd.

\chapter{91}

\par 1 Ten, kdož v skrýši Nejvyššího prebývá, v stínu Všemohoucího odpocívati bude.
\par 2 Dím Hospodinu: Útocište mé a hrad muj, Buh muj, v nemž nadeji skládati budu.
\par 3 Ont zajisté vysvobodí te z osídla lovce, a od nejjedovatejšího nakažení morního.
\par 4 Brky svými prikryje te, a pod krídly jeho bezpecen budeš; místo štítu a pavézy budeš míti pravdu jeho.
\par 5 Nebudeš se báti prístrachu nocního, ani strely létající ve dne.
\par 6 Ani nakažení morního, vlekoucího se v mrákote, ani povetrí morního, v polední cas hubícího.
\par 7 Padne jich po boku tvém tisíc, a deset tisícu po pravici tvé, ale k tobe se to nepriblíží.
\par 8 Ocima toliko svýma to spatríš, a odplate bezbožných se podíváš.
\par 9 Ponevadž jsi Hospodina, kterýž útocište mé jest, a Nejvyššího za svuj príbytek položil,
\par 10 Neprihodí se tobe nic zlého, aniž se priblíží jaká rána k stánku tvému.
\par 11 Nebo andelum svým prikázal o tobe, aby te ostríhali na všech cestách tvých.
\par 12 Na rukou ponesou te, abys neurazil o kámen nohy své.
\par 13 Po lvu a bazališku choditi budeš, a pošlapáš lvíce i draka.
\par 14 Ponevadž mne, dí Buh, zamiloval, vysvobodím jej, a vyvýším; nebo poznal jméno mé.
\par 15 Vzývati mne bude, a vyslyším jej; já s ním budu v ssoužení, vytrhnu a oslavím jej.
\par 16 Dlouhostí dnu jej nasytím, a ukáži jemu spasení své.

\chapter{92}

\par 1 Žalm a písen, ke dni sobotnímu.
\par 2 Dobré jest oslavovati Hospodina, a žalmy zpívati jménu tvému, ó Nejvyšší,
\par 3 Zvestovati každé jitro milosrdenství tvé, a pravdu tvou každé noci,
\par 4 Pri nástroji o desíti strunách, pri loutne, a pri harfe s písnickou.
\par 5 Nebo jsi mne rozveselil, Hospodine, skutky svými, o skutcích rukou tvých zpívati budu.
\par 6 Jak velicí jsou skutkové tvoji, Hospodine! Velmi hluboká jsou myšlení tvá.
\par 7 Clovek hovadný nezná toho, aniž blázen rozumí tomu,
\par 8 Že vyrostají bezbožní jako bylina, a kvetou všickni cinitelé nepravosti, aby vyhlazeni byli na veky.
\par 9 Ty pak, ó Nejvyšší, že na veky jsi Hospodin.
\par 10 Nebo aj, neprátelé tvoji, Hospodine, nebo aj, neprátelé tvoji zahynou; rozptýleni budou všickni cinitelé nepravosti.
\par 11 Muj pak roh vyzdvihneš jako jednorožcuv, pokropen budu olejem novým.
\par 12 I podívá se oko mé na ty, jenž mne špehují, a o tech nešlechetnících, jenž proti mne povstávají, ušima svýma uslyším.
\par 13 Spravedlivý jako palma kvésti bude, a jako cedr na Libánu rozloží se.
\par 14 Štípení v dome Hospodinove v síncích Boha našeho kvésti budou.
\par 15 Ješte i v šedinách ovoce ponesou, spanilí a zelení budou,
\par 16 Aby to zvestováno bylo, že prímý jest Hospodin, skála má, a že nepravosti žádné pri nem není.

\chapter{93}

\par 1 Hospodin kraluje, v dustojnost se oblékl, oblékl se Hospodin v sílu, a prepásal se; také okršlek zeme upevnil, aby se nepohnul.
\par 2 Utvrzent jest trun tvuj prede všemi casy, od vecnosti ty jsi.
\par 3 Pozdvihují se reky, ó Hospodine, pozdvihují reky zvuku svého, pozdvihují reky vlnobití svých.
\par 4 Nad zvuk mnohých vod, nad sílu vln morských mnohem silnejší jest na výsostech Hospodin.
\par 5 Svedectví tvá jsou velmi jistá, domu tvému ušlechtilá svatost, Hospodine, až na veky.

\chapter{94}

\par 1 Bože silný pomst, Hospodine, Bože silný pomst, zastkvej se.
\par 2 Zdvihni se, ó soudce vší zeme, a dej odplatu pyšným.
\par 3 Až dokud bezbožní, Hospodine, až dokud bezbožní budou plésati,
\par 4 Žváti a hrde mluviti, honosíce se, všickni cinitelé nepravosti?
\par 5 Lid tvuj, Hospodine, potírati a dedictví tvé bedovati?
\par 6 Vdovy a príchozí mordovati, a sirotky hubiti,
\par 7 Ríkajíce: Nehledít na to Hospodin, aniž tomu rozumí Buh Jákobuv?
\par 8 Rozumejte, ó vy hovadní v lidu, a vy blázni, kdy srozumíte?
\par 9 Zdali ten, jenž ucinil ucho, neslyší? A kterýž stvoril oko, zdali nespatrí?
\par 10 Zdali ten, jenž tresce národy, nebude kárati, kterýž ucí lidi umení?
\par 11 Hospodint zná myšlení lidská, že jsou pouhá marnost.
\par 12 Blahoslavený jest ten muž, kteréhož ty cvicíš, Hospodine, a z zákona svého jej vyucuješ.
\par 13 Abys mu zpusobil pokoj pred casy zlými, až by za tím vykopána byla bezbožníku jáma.
\par 14 Nebot neopustí Hospodin lidu svého, a dedictví svého nezanechá,
\par 15 Ale až k spravedlnosti navrátí se soud, a za ním všickni uprímého srdce.
\par 16 Kdož by se byl o mne zasadil proti zlostníkum? Kdo by se byl za mne postavil proti tem, jenž páší nepravost?
\par 17 Kdyby mi Hospodin nebyl ku pomoci, tudíž by se byla octla duše má v mlcení.
\par 18 Již jsem byl rekl: Klesla noha má, ale milosrdenství tvé, ó Hospodine, zdrželo mne.
\par 19 Ve množství premyšlování mých u vnitrnosti mé, tvá potešování obveselovala duši mou.
\par 20 Zdaliž se k tobe pritovaryší stolice prevráceností tech, jenž vynášejí nátisk mimo spravedlnost,
\par 21 Jenž se shlukují proti duši spravedlivého, a krev nevinnou odsuzují?
\par 22 Ale Hospodin jest mým hradem vysokým, a Buh muj skalou útocište mého.
\par 23 Ont obrátí na ne nepravost jejich, a zlostí jejich zahladí je, zahladí je Hospodin Buh náš.

\chapter{95}

\par 1 Podte, zpívejme Hospodinu, prokrikujme skále spasení našeho.
\par 2 Predejdeme oblícej jeho s díkcinením, žalmy prozpevujme jemu.
\par 3 Nebo Hospodin jest Buh veliký, a král veliký nade všecky bohy,
\par 4 V jehož rukou základové zeme, a vrchové hor jeho jsou.
\par 5 Jehož jest i more, nebo on je ucinil, i zeme, kterouž ruce jeho sformovaly.
\par 6 Podte, sklánejme se, a padneme pred ním, klekejme pred Hospodinem stvoritelem naším.
\par 7 Ont jest zajisté Buh náš, a my jsme lid pastvy jeho, a stádo rukou jeho. Dnes uslyšíte-li hlas jeho,
\par 8 Nezatvrzujte srdce svého, jako pri popuzení, a v den pokušení na poušti,
\par 9 Kdežto pokoušeli mne otcové vaši, zkusilit jsou mne, a videli skutky mé.
\par 10 Za ctyridceti let mel jsem nesnáz s národem tím, a rekl jsem: Lid tento bloudí srdcem, a nepoznali cest mých.
\par 11 Jimž jsem prisáhl v hneve svém, že nevejdou v odpocinutí mé.

\chapter{96}

\par 1 Zpívejte Hospodinu písen novou, zpívej Hospodinu všecka zeme.
\par 2 Zpívejte Hospodinu, dobrorecte jménu jeho, zvestujte den po dni spasení jeho.
\par 3 Vypravujte mezi národy slávu jeho, mezi všemi lidmi divy jeho.
\par 4 Nebo veliký Hospodin, a vší chvály hodný, i hrozný jest nade všecky bohy.
\par 5 Všickni zajisté bohové národu jsou modly, ale Hospodin nebesa ucinil.
\par 6 Sláva a dustojnost pred ním, síla i okrasa v svatyni jeho.
\par 7 Vzdejte Hospodinu celedi národu, vzdejte Hospodinu cest i moc.
\par 8 Vzdejte Hospodinu cest jména jeho, prineste dary, a vejdete do síncí jeho.
\par 9 Sklánejte se Hospodinu v okrase svatoti, boj se oblíceje jeho všecka zeme.
\par 10 Rcete mezi pohany: Hospodin kraluje, a že i okršlek zemský upevnen bude, tak aby se nepohnul, a že souditi bude lidi spravedlive.
\par 11 Rozveseltež se nebesa, a plésej zeme, zvuc more, i což v nem jest.
\par 12 Plésej pole a vše, což na nem, tehdáž at prozpevuje všecko dríví lesní,
\par 13 Pred tvárí Hospodina; nebot se bére, bére se zajisté, aby soudil zemi. Budet souditi okršlek sveta v spravedlnosti, a národy v pravde své.

\chapter{97}

\par 1 Hospodin kraluje, plésej zeme, a vesel se ostrovu všecko množství.
\par 2 Oblak a mrákota jest vukol neho, spravedlnost a soud základ trunu jeho.
\par 3 Ohen predchází jej, a zapaluje vukol neprátely jeho.
\par 4 Zasvecujít se po okršlku sveta blýskání jeho; to viduc zeme, desí se.
\par 5 Hory jako vosk rozplývají se pred oblícejem Hospodina, pred oblícejem Panovníka vší zeme.
\par 6 Nebesa vypravují o jeho spravedlnosti, a slávu jeho spatrují všickni národové.
\par 7 Zastydte se všickni, kteríž sloužíte rytinám, kteríž se chlubíte modlami; sklánejte se pred ním všickni bohové.
\par 8 To uslyše Sion, rozveselí se, a zpléší dcery Judské z príciny soudu tvých, Hospodine.
\par 9 Nebo ty, Hospodine, jsi nejvyšší na vší zemi, a velice jsi vyvýšený nade všecky bohy.
\par 10 Vy, kteríž milujete Hospodina, mejte v nenávisti to, což zlého jest; ont ostríhá duší svatých svých, a z ruky bezbožníku je vytrhuje.
\par 11 Svetlo vsáto jest spravedlivým, a radost tem, kteríž jsou uprímého srdce.
\par 12 Veselte se, spravedliví v Hospodinu, a oslavujte památku svatosti jeho.

\chapter{98}

\par 1 Žalm. Zpívejte Hospodinu písen novou, nebot jest divné veci ucinil; spomohla mu pravice jeho, a ráme svatosti jeho.
\par 2 V známost uvedl Hospodin spasení své, pred ocima národu zjevil spravedlnost svou.
\par 3 Rozpomenul se na milosrdenství své, a na pravdu svou k domu Izraelskému; všecky konciny zeme vidí spasení Boha našeho.
\par 4 Prokrikuj Hospodinu všecka zeme; zvuk vydejte, prozpevujte, a žalmy zpívejte.
\par 5 Žalmy zpívejte Hospodinu na citare, k citare i hlasem prizpevujte.
\par 6 Trubami a zvucnými pozouny hlas vydejte pred králem Hospodinem.
\par 7 Zvuc more i to, což v nem jest, okršlek sveta i ti, kteríž na nem bydlí.
\par 8 Reky rukama plésejte, spolu i hory prozpevujte,
\par 9 Pred Hospodinem; nebot se bére, aby soudil zemi. Budet souditi okršlek sveta v spravedlnosti, a národy v pravosti.

\chapter{99}

\par 1 Hospodin kraluje, užasnete se národové; sedí nad cherubíny, pohniž se zeme.
\par 2 Hospodin na Sionu veliký, a vyvýšený jest nade všecky lidi.
\par 3 Oslavujte jméno tvé veliké a hrozné, nebo svaté jest.
\par 4 Moc zajisté králova miluje soud; ty jsi ustanovil práva, soud a spravedlnost v Jákobovi ty konáš.
\par 5 Vyvyšujte Hospodina Boha našeho, a sklánejte se u podnoží noh jeho, svatýt jest.
\par 6 Mojžíš a Aron mezi knežími jeho, a Samuel mezi vzývajícími jméno jeho; volávali k Hospodinu, a on je vyslýchal.
\par 7 V sloupu oblakovém mluvíval k nim; kterížto když ostríhali svedectví jeho, i ustanovení jim vydal.
\par 8 Hospodine Bože náš, tys je vyslýchal, Bože, bývals jim milostiv, i když jsi je trestal pro výstupky jejich.
\par 9 Vyvyšujte Hospodina Boha našeho, a sklánejte se na hore svaté jeho; nebot jest svatý Hospodin Buh náš.

\chapter{100}

\par 1 Žalm k díku cinení. Prokrikuj Hospodinu všecka zeme.
\par 2 Služte Hospodinu s veselím, predstupte pred oblícej jeho s prozpevováním.
\par 3 Vezte, že Hospodin jest Buh; on ucinil nás, a ne my sami sebe, abychom byli lid jeho, a ovce pastvy jeho.
\par 4 Vcházejte do bran jeho s díkcinením, a do síní jeho s chvalami; oslavujte jej, a dobrorecte jménu jeho.
\par 5 Nebo dobrý jest Hospodin, na veky milosrdenství jeho, a od národu až do pronárodu pravda jeho.

\chapter{101}

\par 1 Žalm Daviduv. O milosrdenství a soudu zpívati budu, tobe, ó Hospodine, žalmy budu zpívati.
\par 2 Opatrne se míti budu na ceste prímé, až prijdeš ke mne; choditi budu ustavicne v uprímnosti srdce svého i v dome svém.
\par 3 Nepredstavímt sobe pred oci veci nešlechetné; skutek uchylujících se v nenávisti mám, neprichytít se mne.
\par 4 Srdce prevrácené odstoupí ode mne, zlého nebudu oblibovati.
\par 5 Škodícího jazykem bližnímu svému tajne, tohot vytnu; ocí vysokých a mysli naduté nikoli nebudu moci trpeti.
\par 6 Oci mé na pravdomluvné v zemi, aby sedali se mnou; kdož chodí po ceste uprímé, tent mi sloužiti bude.
\par 7 Nebude bydliti v dome mém cinící lest, a mluvící lež nebude míti místa u mne.
\par 8 Každého jitra pléniti budu všecky nešlechetné z zeme, abych tak vyplénil z mesta Hospodinova všecky, kdož páší nepravost.

\chapter{102}

\par 1 Modlitba chudého, když sevrín jsa, pred Hospodinem vylévá žádosti své.
\par 2 Hospodine, slyš modlitbu mou, a volání mé prijdiž k tobe.
\par 3 Neskrývej tvári své prede mnou, v den ssoužení mého naklon ke mne ucha svého; když k tobe volám, rychle vyslyš mne.
\par 4 Nebo mizejí jako dým dnové moji, a kosti mé jako ohnište vypáleny jsou.
\par 5 Poraženo jest jako bylina, a usvadlo srdce mé, tak že jsem chleba svého jísti zapomenul.
\par 6 Od hlasu lkání mého prilnuly kosti mé k kuži mé.
\par 7 Podobný jsem ucinen pelikánu na poušti, jsem jako výr na pustinách.
\par 8 Bdím, a jsem jako vrabec osamelý na streše.
\par 9 Každý den utrhají mi neprátelé moji, a posmevaci moji proklínají mnou.
\par 10 Nebo jídám popel jako chléb, a k nápoji svému slz primešuji,
\par 11 Pro rozhnevání tvé a zažžený hnev tvuj; nebo zdvihna mne, hodils mnou.
\par 12 Dnové moji jsou jako stín nachýlený, a já jako tráva usvadl jsem.
\par 13 Ale ty, Hospodine, na veky zustáváš, a památka tvá od národu až do pronárodu.
\par 14 Ty povstana, smiluješ se nad Sionem, nebo cas jest uciniti milost jemu, a cas uložený prišel.
\par 15 Nebo líbost mají služebníci tvoji v kamení jeho, a nad prachem jeho slitují se,
\par 16 Aby se báli pohané jména Hospodinova, a všickni králové zeme slávy tvé,
\par 17 Když by Hospodin vzdelal Sion, a ukázal se v sláve své,
\par 18 Když by popatril k modlitbe poníženého lidu, nepohrdaje modlitbou jejich.
\par 19 Budet to zapsáno pro budoucí potomky, a lid, kterýž má stvoren býti, chváliti bude Hospodina,
\par 20 Že shlédl s výsosti svatosti své. Hospodin s nebe na zemi že popatril,
\par 21 Aby vyslyšel vzdychání veznu, a rozvázal ty, kteríž již k smrti oddání byli,
\par 22 Aby vypravovali na Sionu jméno Hospodinovo, a chválu jeho v Jeruzaléme,
\par 23 Když se spolu shromáždí národové a království, aby sloužili Hospodinu.
\par 24 Ztrápilt jest na ceste sílu mou, ukrátil dnu mých,
\par 25 Až jsem rekl: Muj Bože, neber mne u prostred dnu mých; od národu zajisté až do pronárodu jsou léta tvá,
\par 26 I prvé nežlis založil zemi, a dílo rukou svých, nebesa.
\par 27 Onat pominou, ty pak zustáváš; všecky ty veci jako roucho zvetšejí, jako odev zmeníš je, a zmeneny budou.
\par 28 Ty pak jsi tentýž, a léta tvá nikdy neprestanou.
\par 29 Synové služebníku tvých bydliti budou, a síme jejich zmocní se pred tebou.

\chapter{103}

\par 1 Daviduv. Dobrorec duše má Hospodinu, a všecky vnitrnosti mé jménu svatému jeho.
\par 2 Dobrorec duše má Hospodinu, a nezapomínej se na všecka dobrodiní jeho,
\par 3 Kterýž odpouští tobe všecky nepravosti, kterýž uzdravuje všecky nemoci tvé,
\par 4 Kterýž vysvobozuje od zahynutí život tvuj, kterýž te korunuje milosrdenstvím a mnohým slitováním,
\par 5 Kterýž nasycuje dobrými vecmi ústa tvá, tak že se obnovuje jako orlice mladost tvá.
\par 6 Ciní, což spravedlivého jest, Hospodin, a soudy všechnem utišteným.
\par 7 Známé ucinil Mojžíšovi cesty své, synum Izraelským skutky své.
\par 8 Lítostivý a milostivý jest Hospodin, dlouhoshovívající a mnohého milosrdenství.
\par 9 Nebudet ustavicne žehrati, ani na veky hnevu držeti.
\par 10 Ne podlé hríchu našich nakládá s námi, ani vedlé nepravostí našich odplacuje nám.
\par 11 Nebo jakož jsou vysoko nebesa nad zemí, tak jest vyvýšené milosrdenství jeho nad temi, kteríž se ho bojí.
\par 12 A jak daleko jest východ od západu, tak daleko vzdálil od nás prestoupení naše.
\par 13 Jakož se slitovává otec nad dítkami,tak se slitovává Hospodin nad temi, kteríž se hobojí.
\par 14 Ont zajisté zná slepení naše, v pameti má, že prach jsme.
\par 15 Dnové cloveka jsou jako tráva, a jako kvet polní, tak kvete.
\par 16 Jakž vítr na nej povane, ant ho není, aniž ho již více pozná místo jeho.
\par 17 Milosrdenství pak Hospodinovo od veku až na veky nad temi, kteríž se ho bojí, a spravedlnost jeho nad syny synu,
\par 18 Kteríž ostríhají smlouvy jeho, a pamatují na prikázaní jeho, aby je cinili.
\par 19 Hospodin na nebesích utvrdil trun svuj, a kralování jeho nade vším panuje.
\par 20 Dobrorecte Hospodinu andelé jeho, kteríž jste mocní v síle, a ciníte slovo jeho, poslušní jsouc hlasu slova jeho.
\par 21 Dobrorecte Hospodinu všickni zástupové jeho, služebníci jeho, kteríž ciníte vuli jeho.
\par 22 Dobrorecte Hospodinu všickni skutkové jeho, na všech místech panování jeho. Dobrorec duše má Hospodinu.

\chapter{104}

\par 1 Dobrorec duše má Hospodinu. Hospodine Bože muj, velmi jsi veliký, velebnost a krásu jsi oblékl.
\par 2 Priodels se svetlem jako rouchem, roztáhls nebesa jako kortýnu.
\par 3 Kterýž sklenul na vodách paláce své, kterýž užívá hustých oblaku místo vozu, a vznáší se na perí vetrovém.
\par 4 Kterýž ciní posly své duchy, služebníky své ohen plápolající.
\par 5 Založil zemi na sloupích jejich, tak že se nepohne na veky veku.
\par 6 Propastí jako rouchem byl jsi ji priodel, i nad horami stály vody.
\par 7 K žehrání tvému rozbehly se, pred hrmotem hromu tvého pospíšily,
\par 8 (Vystoupily hory, snížilo se údolí), na místo, kteréž jsi jim založil.
\par 9 Meze jsi položil, aby jich neprestupovaly, ani se navracovaly k prikrývání zeme.
\par 10 Kterýž vypouštíš potoky pres údolé, aby tekli mezi horami,
\par 11 A nápoj dávali všechnem živocichum polním. Tut uhašují oslové divocí žízen svou.
\par 12 Pri nich hnízdí se ptactvo nebeské, a z prostred ratolestí hlas svuj vydává.
\par 13 Kterýž svlažuješ hory z výsostí svých, aby ovocem cinu tvých sytila se zeme.
\par 14 Dáváš, aby rostla tráva dobytku, a bylina ku potrebe cloveku, abys tak vyvodil chléb z zeme,
\par 15 A víno, jenž obveseluje srdce cloveka. Ciní, aby se stkvela tvár od oleje, ano i pokrmem zdržuje život lidský.
\par 16 Nasyceno bývá i dríví Hospodinovo, cedrové Libánští, kteréž štípil.
\par 17 Na nichž se ptáci hnízdí, i cáp príbytek svuj má na jedlí.
\par 18 Hory vysoké jsou kamsíku, skály útocište králíku.
\par 19 Ucinil mesíc k jistým casum, a slunce zná západ svuj.
\par 20 Uvodíš tmu, a bývá noc, v níž vybíhají všickni živocichové lesní:
\par 21 Lvícata rvoucí po loupeži, aby hledali od Boha silného pokrmu svého.
\par 22 Když slunce vychází, zase shromaždují se, a v doupatech svých se ukládají.
\par 23 Clovek vychází ku práci své, a k dílu svému až do vecera.
\par 24 Jak mnozí a velicí jsou skutkové tvoji, Hospodine! Všeckys je moudre ucinil, plná jest zeme bohatství tvého.
\par 25 V mori pak velikém a preširokém, tamt jsou hmyzové nescíslní, a živocichové malí i velicí.
\par 26 Tut bárky precházejí i velryb, kteréhož jsi stvoril, aby v nem hrál.
\par 27 Všecko to na te ocekává, abys jim dával pokrm casem svým.
\par 28 Když jim dáváš, sbírají; když otvíráš ruku svou, nasyceni bývají dobrými vecmi.
\par 29 Když skrýváš tvár svou, rmoutí se; když odjímáš ducha jejich, hynou, a v prach svuj se navracejí.
\par 30 Vysíláš ducha svého, a zase stvoreni bývají, a obnovuješ tvár zeme.
\par 31 Budiž sláva Hospodinova na veky, rozveselujž se Hospodin v skutcích svých.
\par 32 On když pohledí na zemi, anat se trese; když se dotkne hor, ant se kourí.
\par 33 Zpívati budu Hospodinu, dokudž jsem živ; žalmy Bohu svému zpívati budu, pokudž mne stává.
\par 34 Libé bude premyšlování mé o nem, ját rozveselím se v Hospodinu.
\par 35 Ó by hríšníci vyhynuli z zeme, a bezbožných aby již nebylo. Dobrorec duše má Hospodinu. Halelujah.

\chapter{105}

\par 1 Oslavujte Hospodina, ohlašujte jméno jeho, oznamujte mezi národy skutky jeho.
\par 2 Zpívejte jemu, žalmy prozpevujte jemu, rozmlouvejte o všech divných skutcích jeho.
\par 3 Chlubte se jménem svatým jeho; vesel se srdce tech, kteríž hledají Hospodina.
\par 4 Hledejte Hospodina a síly jeho, hledejte tvári jeho ustavicne.
\par 5 Rozpomínejte se na divné skutky jeho, kteréž cinil, na zázraky jeho a na soudy úst jeho,
\par 6 Síme Abrahamovo, služebníka jeho, synové Jákobovi, vyvolení jeho.
\par 7 Ont jest Hospodin Buh náš, na vší zemi soudové jeho.
\par 8 Pamatuje vecne na smlouvu svou, na slovo, kteréž prikázal až do tisíce pokolení,
\par 9 Kteréž upevnil s Abrahamem, a na prísahu svou ucinenou Izákovi.
\par 10 Nebo ji utvrdil Jákobovi za ustanovení, Izraelovi za smlouvu vecnou,
\par 11 Prave: Tobe dám zemi Kananejskou za podíl dedictví vašeho,
\par 12 Ješto jich byl malý pocet, malý pocet, a ješte v ní byli pohostinu.
\par 13 Precházeli zajisté z národu do národu, a z království k jinému lidu.
\par 14 Nedopustil žádnému ublížiti jim, ano i krále pro ne trestal, rka:
\par 15 Nedotýkejte se pomazaných mých, a prorokum mým necinte nic zlého.
\par 16 Když privolav hlad na zemi, všecku hul chleba polámal,
\par 17 Poslal pred nimi muže znamenitého, jenž za služebníka prodán byl, totiž Jozefa.
\par 18 Jehož nohy sevreli pouty, železa podniknouti musil,
\par 19 Až do toho casu, když se zmínka stala o nem; rec Hospodinova zkusila ho.
\par 20 Poslav král, propustiti ho rozkázal, panovník lidu svobodna ho ucinil.
\par 21 Ustanovil ho pánem domu svého, a panovníkem všeho vládarství svého,
\par 22 Aby vládl i knížaty jeho podlé své líbosti, a starce jeho vyucoval moudrosti.
\par 23 Potom všel Izrael do Egypta, a Jákob pohostinu byl v zemi Chamove.
\par 24 Kdež rozmnožil Buh lid svuj náramne, a ucinil, aby silnejší byl nad neprátely své.
\par 25 Zmenil mysl techto, aby v nenávisti meli lid jeho, a aby ukládali lest o služebnících jeho.
\par 26 I poslal Mojžíše slouhu svého, a Arona, kteréhož vyvolil.
\par 27 Kteríž predložili jim slova znamení jeho a zázraku v zemi Chamove.
\par 28 Poslal tmu, a zatmelo se, aniž odporná byla slovu jeho.
\par 29 Obrátil vody jejich v krev, a zmoril ryby v nich.
\par 30 Vydala zeme jejich množství žab, i v pokoleních králu jejich.
\par 31 Rekl, i prišla smesice žížal, a stenice na všecky konciny jejich.
\par 32 Dal místo dešte krupobití, ohen horící na zemi jejich,
\par 33 Tak že potloukl réví jejich i fíkoví jejich, a zprerážel dríví v krajine jejich.
\par 34 Rekl, i prišly kobylky a chroustu nescíslné množství.
\par 35 I sežrali všelikou bylinu v krajine jejich, a pojedli úrody zeme jejich.
\par 36 Nadto pobil všecko prvorozené v zemi jejich, pocátek všeliké síly jejich.
\par 37 Tedy vyvedl své s stríbrem a zlatem, aniž byl v pokoleních jejich, ješto by se poklesl.
\par 38 Veselili se Egyptští, když tito vycházeli; nebo byl pripadl na ne strach Izraelských.
\par 39 Roztáhl oblak k zastírání jich, a ohen k osvecování noci.
\par 40 K žádosti privedl krepelky, a chlebem nebeským sytil je.
\par 41 Otevrel skálu, i tekly vody, a odcházely pres vyprahlá místa jako reka.
\par 42 Nebo pametliv byl na slovo svatosti své, k Abrahamovi služebníku svému mluvené.
\par 43 Protož vyvedl lid svuj s radostí, s prozpevováním vyvolené své.
\par 44 A dal jim zeme pohanu, a tak úsilí národu dedicne obdrželi,
\par 45 Aby zachovávali ustanovení jeho, a práv jeho ostríhali. Halelujah.

\chapter{106}

\par 1 Halelujah. Oslavujte Hospodina, nebo dobrý jest, nebo na veky milosrdenství jeho.
\par 2 Kdo muže vymluviti nesmírnou moc Hospodinovu, a vypraviti všecku chválu jeho?
\par 3 Blahoslavení, kteríž ostríhají soudu, a ciní spravedlnost každého casu.
\par 4 Pamatuj na mne, Hospodine, pro milost k lidu svému, navštev mne spasením svým,
\par 5 Abych užíval dobrých vecí s vyvolenými tvými, a veselil se veselím národu tvého, a chlubil se spolu s dedictvím tvým.
\par 6 Zhrešili jsme i s otci svými, nepráve jsme cinili, a bezbožnost páchali.
\par 7 Otcové naši v Egypte nerozumeli predivným skutkum tvým, aniž pamatovali na množství milosrdenství tvých, ale odporni byli pri mori, pri mori Rudém.
\par 8 A však vysvobodil je pro jméno své, aby v známost uvedl moc svou.
\par 9 Nebo primluvil mori Rudému, a vyschlo; i provedl je skrze hlubiny, jako po poušti.
\par 10 A tak zachoval je od ruky toho, jenž jich nenávidel, a vyprostil je z ruky neprítele.
\par 11 V tom prikryly vody ty, kteríž je ssužovali, nezustalo ani jednoho z nich.
\par 12 A ackoli verili slovum jeho, a zpívali chválu jeho,
\par 13 Však rychle zapomenuli na skutky jeho, a necekali na radu jeho;
\par 14 Ale jati jsouce žádostí na poušti, pokoušeli Boha silného na pustinách.
\par 15 I dal jim, cehož se jim chtelo, a však dopustil hubenost na život jejich.
\par 16 Potom, když horlili proti Mojžíšovi v vojšte, a Aronovi, svatému Hospodinovu,
\par 17 Otevrevši se zeme, požrela Dátana, a prikryla zber Abironovu.
\par 18 A roznícen byl ohen na rotu jejich, plamen spálil bezbožníky.
\par 19 Udelali i tele na Orébe, a skláneli se slitine.
\par 20 A zmenivše slávu svou v podobiznu vola, jenž jí trávu,
\par 21 Zapomneli na Boha silného, spasitele svého, kterýž cinil veliké veci v Egypte.
\par 22 A predivné v zemi Chamove, prehrozné pri mori Rudém.
\par 23 Procež rekl, že je vypléní, kdyby se byl Mojžíš, vyvolený jeho, nepostavil v té mezere pred ním, a neodvrátil prchlivosti jeho, aby nehubil.
\par 24 Za tím zošklivili sobe zemi žádanou, neveríce slovu jeho.
\par 25 A repcíce v staních svých, neposlouchali hlasu Hospodinova.
\par 26 A protož pozdvihl ruky své proti nim, aby je rozmetal po poušti,
\par 27 A aby rozptýlil síme jejich mezi pohany, a rozehnal je do zemí.
\par 28 Spráhli se také byli s modlou Belfegor, a jedli obeti mrch.
\par 29 A tak dráždili Boha skutky svými, až se na ne oborila rána,
\par 30 Až se postavil Fínes, a pomstu vykonal, i pretržena jest rána ta.
\par 31 Což jest mu pocteno za spravedlnost od národu do pronárodu, a až na veky.
\par 32 Opet ho byli popudili pri vodách sváru, až se i Mojžíšovi zle stalo pro ne.
\par 33 Nebo k horkosti privedli ducha jeho, a pronesl ji rty svými.
\par 34 K tomu nevyplénili ani národu tech, o kterýchž jim byl Hospodin porucil,
\par 35 Ale smešujíce se s temi národy, naucili se skutkum jejich,
\par 36 A sloužili modlám jejich, kteréž jim byly osídlem.
\par 37 Obetovali zajisté syny své a dcery své dáblum.
\par 38 A vylili krev nevinnou, krev synu svých a dcer svých, kteréž obetovali trapidlum Kananejským, tak že poškvrnena jest zeme temi vraždami.
\par 39 I zmazali se skutky svými, a smilnili ciny svými.
\par 40 Protož rozpáliv se v prchlivosti Hospodin na lid svuj, v ošklivost vzal dedictví své.
\par 41 A vydal je v ruce pohanu. I panovali nad nimi ti, jenž je v nenávisti meli,
\par 42 A utiskali je neprátelé jejich, tak že sníženi jsou pod ruku jejich.
\par 43 Mnohokrát je vysvobozoval, oni však popouzeli ho radou svou, procež potlaceni jsou pro nepravost svou.
\par 44 A však patril na úzkost jejich, a slyšel krik jejich.
\par 45 Nebo se rozpomenul na smlouvu svou s nimi, a želel toho podlé množství milosrdenství svých,
\par 46 Tak že naklonil k nim lítostí všecky, kteríž je u vezení drželi.
\par 47 Zachovej nás, Hospodine Bože náš, a shromažd nás z tech pohanu, abychom slavili jméno tvé svaté, a chlubili se v chvále tvé.
\par 48 Požehnaný Hospodin Buh Izraelský od veku až na veky. Na to rciž všecken lid: Amen, Halelujah.

\chapter{107}

\par 1 Oslavujte Hospodina, nebo jest dobrý, nebo na veky milosrdenství jeho.
\par 2 Necht o tom vypravují ti, kteríž jsou vykoupeni skrze Hospodina, jak je on vykoupil z ruky tech, kteríž je ssužovali,
\par 3 A shromáždil je z zemí, od východu a od západu, od pulnoci i od more.
\par 4 Bloudili po poušti, po cestách pustých, mesta k prebývání nenacházejíce.
\par 5 Hladovití a žízniví byli, až v nich svadla duše jejich.
\par 6 Když volali k Hospodinu v ssoužení svém, z úzkostí jejich vytrhl je,
\par 7 A vedl je po ceste prímé, aby prišli do mesta k bydlení.
\par 8 Nechat oslavují pred Hospodinem milosrdenství jeho, a divné skutky jeho pred syny lidskými,
\par 9 Ponevadž napájí duši žíznivou, a duši hladovitou naplnuje dobrými vecmi.
\par 10 Kteríž sedí ve tme a v stínu smrti, sevríni jsouce bídou i železy,
\par 11 Protože odporni byli recem Boha silného, a radou Nejvyššího pohrdli.
\par 12 Procež ponížil bídou srdce jejich, padli, a nebylo pomocníka.
\par 13 Když volají k Hospodinu v ssoužení svém, z úzkostí je vysvobozuje.
\par 14 Vyvodí je z temností a stínu smrti, a svazky jejich trhá.
\par 15 Nechat oslavují pred Hospodinem milosrdenství jeho, a divné skutky jeho pred syny lidskými,
\par 16 Ponevadž láme brány medené, a závory železné posekává.
\par 17 Blázni pro cestu prevrácenosti své, a pro nepravosti své v trápení bývají.
\par 18 Oškliví se jim všeliký pokrm, až se i k branám smrti približují.
\par 19 Když volají k Hospodinu v ssoužení svém, z úzkostí jejich je vysvobozuje.
\par 20 Posílá slovo své, a uzdravuje je, a vysvobozuje je z hrobu.
\par 21 Nechat oslavují pred Hospodinem milosrdenství jeho, a divné skutky jeho pred syny lidskými,
\par 22 A obetujíce obeti chvály, at vypravují skutky jeho s prozpevováním.
\par 23 Kterí se plaví po mori na lodech, pracujíce na velikých vodách,
\par 24 Tit vídají skutky Hospodinovy, a divy jeho v hlubokosti.
\par 25 Jakž jen dí, hned se strhne vítr bourlivý, a dme vlny morské.
\par 26 Vznášejí se k nebi, sstupují do propasti, duše jejich v nebezpecenství rozplývá se.
\par 27 Motají se a naklonují jako opilý, a všecko umení jejich mizí.
\par 28 Když volají k Hospodinu v ssoužení svém, z úzkostí jejich je vysvobozuje.
\par 29 Promenuje bouri v utišení, tak že umlkne vlnobití jejich.
\par 30 I veselí se, že utichlo; a tak privodí je k brehu žádostivému.
\par 31 Nechat oslavují pred Hospodinem milosrdenství jeho, a divné skutky jeho pred syny lidskými.
\par 32 Necht ho vyvyšují v shromáždení lidu, a v rade starcu chválí jej.
\par 33 Obrací reky v poušt, a prameny vod v suchost,
\par 34 Zemi úrodnou v slatinnou, pro zlost obyvatelu jejích.
\par 35 Pustiny obrací v jezera, a zemi vyprahlou v prameny vod.
\par 36 I osazuje na ní hladovité, aby staveli mesta k bydlení.
\par 37 Kteríž osívají pole, a delají vinice, a shromaždují sobe užitek úrody.
\par 38 Takt on jim žehná, že se rozmnožují velmi, a dobytka jejich neumenšuje.
\par 39 A nekdy pak umenšeni a sníženi bývají ukrutenstvím, bídou a truchlostí,
\par 40 Když vylévá pohrdání na knížata, dopoušteje, aby bloudili po poušti bezcestné.
\par 41 Ont vyzdvihuje nuzného z trápení, a rozmnožuje rodinu jako stádo.
\par 42 Necht to spatrují uprímí, a rozveselí se, ale všeliká nepravost at zacpá ústa svá.
\par 43 Ale kdo jest tak moudrý, aby toho šetril, a vyrozumíval mnohému milosrdenství Hospodinovu?

\chapter{108}

\par 1 Písen a žalm Daviduv.
\par 2 Hotovo jest srdce mé, Bože, zpívati a oslavovati te budu, také i sláva má.
\par 3 Probud se loutno a harfo, když v svitání povstávám.
\par 4 Slaviti te budu mezi lidmi, Hospodine, a tobe žalmy prozpevovati mezi národy.
\par 5 Nebo nad nebesa vetší jest milosrdenství tvé, a až k nejvyšším oblakum pravda tvá.
\par 6 Vyvyšiž se nad nebesa, ó Bože, a nade všecku zemi sláva tvá.
\par 7 At jsou vysvobozeni milí tvoji, zachovávejž je pravicí svou, a vyslyš mne.
\par 8 Buh mluvil skrze svatost svou; veseliti se budu, že budu deliti Sichem, a údolí Sochot že rozmerím.
\par 9 Mujt jest Galád, muj i Manasses, a Efraim síla hlavy mé, Juda ucitel muj.
\par 10 Moáb medenice k umývání mému, na Edoma uvrhu obuv svou, proti Palestine troubiti budu.
\par 11 Kdo mne uvede do mesta hrazeného? Kdo mne zprovodí až do Idumejské zeme?
\par 12 Zdali ne ty, ó Bože, kterýž jsi nás byl zavrhl, a nevycházels, ó Bože, s vojsky našimi?
\par 13 Udeliž nám pomoci proti nepríteli, nebo marná jest pomoc lidská.
\par 14 V Bohu udatne sobe pocínati budeme, a on pošlapá neprátely naše.

\chapter{109}

\par 1 Prednímu zpeváku, žalm Daviduv. Ó Bože chvály mé, necin se neslyše.
\par 2 Nebo ústa nešlechetného a ústa lstivá proti mne se otevrela, mluvili proti mne jazykem lživým.
\par 3 A slovy jizlivými osocili mne, válcí proti mne beze vší príciny.
\par 4 Protivili mi se za mé milování, ješto jsem se za ne modlíval.
\par 5 Odplacují se mi zlým za dobré, a nenávistí za milování mé.
\par 6 Postav nad ním bezbožníka, a protivník at mu stojí po pravici.
\par 7 Když pred soudem stane, at zustane za nešlechetného, a prosba jeho budiž jemu k hríchu.
\par 8 Budiž dnu jeho málo, a úrad jeho vezmi jiný.
\par 9 Budtež deti jeho sirotci, a žena jeho vdovou.
\par 10 Budtež behouni a tuláci synové jeho, žebrete, vyhnáni jsouce z poustek svých.
\par 11 Pritáhni k sobe lichevník všecko, cožkoli má, a úsilé jeho rozchvátejte cizí.
\par 12 Nebudiž, kdo by mu chtel milosrdenství prokázati, aniž bud, kdo by se smiloval nad sirotky jeho.
\par 13 Potomci jeho z koren vytati budte, v druhém kolenu vyhlazeno bud jméno jejich.
\par 14 Prijdiž na pamet nepravost predku jeho pred Hospodinem, a hrích matky jeho nebud shlazen.
\par 15 Budtež pred Hospodinem ustavicne, až by vyhladil z zeme památku jejich,
\par 16 Proto že nepamatoval, aby cinil milosrdenství, ale protivenství cinil cloveku chudému a nuznému a sevrenému bolestí srdce, aby jej zamordoval.
\par 17 Ponevadž miloval zlorecení, nechat prijde na nej; nemel líbosti v požehnání, nechat se vzdálí od neho.
\par 18 A tak budiž oblecen v zlorecenství jako v svuj odev, a at vejde do vnitrností jeho jako voda, a jako olej do kostí jeho.
\par 19 Budiž jemu to jako plášt k priodívání, a jako pás k ustavicnému opasování.
\par 20 Taková mzda prijdiž mým protivníkum od Hospodina, a mluvícím zlé veci proti duši mé.
\par 21 Ty pak, Hospodine Pane, nalož se mnou laskave pro jméno své, nebo dobré jest milosrdenství tvé; vytrhni mne.
\par 22 Jsemt zajisté chudý a nuzný, a srdce mé raneno jest u vnitrnostech mých.
\par 23 Jako stín, když se nachyluje, ucházeti musím; honí se za mnou jako za kobylkou.
\par 24 Kolena má klesají postem, a telo mé vyschlo z tucnosti.
\par 25 Nadto jsem jim za posmech; když mne uhlédají, potrásají hlavami svými.
\par 26 Spomoz mi, ó Hospodine Bože muj, zachovej mne podlé milosrdenství svého,
\par 27 Tak aby poznati mohli, že jest to ruka tvá, a že jsi ty, Hospodine, ucinil to.
\par 28 Necht oni jakkoli zlorecí, ty dobrorec; kteríž povstali, necht se zastydí, aby se veselil služebník tvuj.
\par 29 Budtež obleceni protivníci moji v zahanbení, a necht se odejí jako pláštem hanbou svou.
\par 30 Slaviti budu Hospodina velice ústy svými, a u prostred mnohých chváliti jej budu,
\par 31 Proto že stojí po pravici nuznému, aby ho zachoval od tech, kteríž odsuzují život jeho.

\chapter{110}

\par 1 Daviduv žalm. Rekl Hospodin Pánu mému: Sed na pravici mé, dokudž nepoložím neprátel tvých za podnože noh tvých.
\par 2 Berlu moci tvé vyšle Hospodin z Siona, rka: Panuj u prostred neprátel svých.
\par 3 Lid tvuj dobrovolný v den boje tvého v ozdobe svatosti, z života hned v svitání jako rosa plod tvuj bude.
\par 4 Prisáhl Hospodin, a nebude želeti toho, rka: Ty jsi knez na veky podlé rádu Melchisedechova.
\par 5 Pán po pravici tvé potre v den hnevu svého krále.
\par 6 Soud ciniti bude mezi národy, porážku hroznou uciní, potre i hlavu panující nad mnohými krajinami.
\par 7 Z potoka na ceste píti bude, a protož povýší hlavy.

\chapter{111}

\par 1 Halelujah. Slaviti budu Hospodina z celého srdce, v rade prímých i v shromáždení;
\par 2 Veliké skutky Hospodinovy, a patrné všechnem, kteríž v nich líbost mají;
\par 3 Slavné a prekrásné dílo jeho, a spravedlnost jeho zustávající na veky.
\par 4 Památku zpusobil predivnými skutky svými milostivý a milosrdný Hospodin.
\par 5 Pokrm dal tem, kteríž se ho bojí, pametliv jsa vecne na smlouvu svou.
\par 6 Mocné skutky své oznámil lidu svému, dav jim dedictví pohanu.
\par 7 Skutkové rukou jeho pravda a soud, a nepohnutelní všickni rozkazové jeho.
\par 8 Upevnení na vecnou vecnost; ucineni jsou v pravde a v pravosti.
\par 9 Vykoupení poslav lidu svému, prikázal na veky smlouvu svou; svaté a hrozné jest jméno jeho.
\par 10 Pocátek moudrosti jest bázen Hospodina; rozumu výborného nabývají všickni, kteríž ciní ty veci; chvála jeho zustává na veky.

\chapter{112}

\par 1 Halelujah. Blahoslavený muž, kterýž se bojí Hospodina, a v prikázaních jeho má velikou líbost.
\par 2 Mocné na zemi bude síme jeho, rodina uprímých požehnání dojde.
\par 3 Zboží a bohatství v dome jeho, a spravedlnost jeho zustává na veky.
\par 4 Vzchází ve tmách svetlo uprímým, milostivý jest, milosrdný a spravedlivý.
\par 5 Dobrý clovek slitovává se i pujcuje, a rídí své veci s soudem.
\par 6 Nebo nepohne se na veky, v pameti vecné bude spravedlivý.
\par 7 Slyše zlé noviny, nebojí se; stálé jest srdce jeho, a doufá v Hospodina.
\par 8 Utvrzené srdce jeho nebojí se, až i uzrí pomstu na svých neprátelích.
\par 9 Rozdeluje štedre, a dává nuzným; spravedlnost jeho zustává na veky, roh jeho bude vyvýšen v sláve.
\par 10 Bezbožný vida to, zlobiti se, zuby svými škripeti a schnouti bude; žádost bezbožníku zahyne.

\chapter{113}

\par 1 Halelujah. Chvalte služebníci Hospodinovi, chvalte jméno Hospodinovo.
\par 2 Budiž jméno Hospodinovo požehnáno od tohoto casu až na veky.
\par 3 Od východu slunce až do západu jeho chváleno bud jméno Hospodinovo.
\par 4 Vyvýšent jest nade všecky národy Hospodin, a nad nebesa sláva jeho.
\par 5 Kdo jest rovný Hospodinu Bohu našemu, kterýž vysoko bydlí?
\par 6 Kterýž snižuje se, aby všecko spatroval, což jest na nebi i na zemi.
\par 7 Vyzdvihuje z prachu nuzného, a z hnoje vyvyšuje chudého,
\par 8 Aby jej posadil s knížaty, s knížaty lidu svého.
\par 9 Kterýž vzdelává neplodnou v celed, a matku veselící se z dítek. Halelujah.

\chapter{114}

\par 1 Když vycházel Izrael z Egypta, a rodina Jákobova z národu jazyka cizího,
\par 2 Byl Juda posvecením jeho, Izrael panováním jeho.
\par 3 To když videlo more, uteklo, Jordán nazpet se obrátil.
\par 4 Hory poskakovaly jako skopci, pahrbkové jako jehnata.
\par 5 Cot bylo, ó more, že jsi utíkalo? Jordáne, že jsi nazpet se obrátil?
\par 6 Ó hory, že jste poskakovaly jako skopci, pahrbkové jako jehnata?
\par 7 Pro prítomnost Panovníka trásla jsem se já zeme, pro prítomnost Boha Jákobova,
\par 8 Kterýž obrací i tu skálu v jezero vod, a škremen v studnici vod.

\chapter{115}

\par 1 Ne nám, Hospodine, ne nám, ale jménu svému dej cest, pro milosrdenství své a pro pravdu svou.
\par 2 Proc mají ríkati pohané: Kdež jest nyní Buh jejich?
\par 3 Ješto Buh náš jest na nebi, cine všecko, což se mu líbí.
\par 4 Modly pak jejich jsou stríbro a zlato, dílo rukou lidských.
\par 5 Ústa mají, a nemluví, oci mají, a nevidí.
\par 6 Uši mají, a neslyší, nos mají, a necijí.
\par 7 Ruce mají, a nemakají, nohy mají, a nechodí, aniž volati mohou hrdlem svým.
\par 8 Nechat jsou jim podobni, kteríž je delají, a kdožkoli v nich doufají.
\par 9 Izraeli, doufej v Hospodina, nebo spomocníkem a štítem takových on jest.
\par 10 Dome Aronuv, doufej v Hospodina, spomocníkem a štítem takových on jest.
\par 11 Kteríž se bojíte Hospodina, doufejte v Hospodina, spomocníkem a štítem takových on jest.
\par 12 Hospodin rozpomena se na nás, požehná; požehná domu Izraelovu, požehná i domu Aronovu.
\par 13 Požehná bojícím se Hospodina, malým, i velikým.
\par 14 Rozmnoží Hospodin vás, vás i syny vaše.
\par 15 Požehnaní vy od Hospodina, kterýž ucinil nebesa i zemi.
\par 16 Nebesa jsou nebesa Hospodinova, zemi pak dal synum lidským.
\par 17 Ne mrtví chváliti budou Hospodina, ani kdo ze všech tech, kteríž sstupují do místa mlcení,
\par 18 Ale my dobroreciti budeme Hospodinu od tohoto casu až na veky. Halelujah.

\chapter{116}

\par 1 Miluji Hospodina, proto že vyslýchá hlas muj a pokorné modlitby mé.
\par 2 Nebo naklonil ucha svého ke mne, když jsem ho vzýval ve dnech svých.
\par 3 Obklícilyt mne byly bolesti smrti, a úzkosti hrobu potkaly mne; sevrení a truchlost prišla na mne.
\par 4 I vzýval jsem jméno Hospodinovo, rka: Prosím, ó Hospodine, vysvobod duši mou.
\par 5 Milostivý Hospodin a spravedlivý, Buh náš lítostivý.
\par 6 Ostríhá sprostných Hospodin; znuzen jsem byl, a spomohl mi.
\par 7 Navratiž se, duše má, do odpocinutí svého, ponevadž Hospodin jest dobrodince tvuj.
\par 8 Nebo jsi vytrhl duši mou od smrti, oci mé od slz, nohu mou od poklesnutí.
\par 9 Ustavicne choditi budu pred Hospodinem v zemi živých.
\par 10 Uveril jsem, protož i mluvil jsem, ackoli jsem byl velmi ztrápený.
\par 11 Já jsem byl rekl v pospíchání: Všeliký clovek jest lhár.
\par 12 Cím se odplatím Hospodinu za všecka dobrodiní jeho mne ucinená?
\par 13 Kalich mnohého spasení vezmu, a jméno Hospodinovo vzývati budu.
\par 14 Sliby své Hospodinu splním, ted prede vším lidem jeho.
\par 15 Drahá jest pred ocima Hospodinovýma smrt svatých jeho.
\par 16 Ó Hospodine, že jsem služebník tvuj, služebník, pravím, tvuj, syn devky tvé, rozvázal jsi svazky mé.
\par 17 Tobe obetovati budu obet díku cinení, a jméno Hospodinovo vzývati budu.
\par 18 Sliby své Hospodinu splním, ted prede vším lidem jeho,
\par 19 V síncích domu Hospodinova, u prostred tebe, Jeruzaléme. Halelujah.

\chapter{117}

\par 1 Chvalte Hospodina všickni národové, velebtež ho všickni lidé.
\par 2 Nebot jest rozšíreno nad námi milosrdenství jeho, a pravda Hospodinova na veky. Halelujah.

\chapter{118}

\par 1 Oslavujte Hospodina, nebo jest dobrý, nebo na veky trvá milosrdenství jeho.
\par 2 Rciž nyní, Izraeli, že na veky milosrdenství jeho.
\par 3 Rciž nyní, dome Aronuv, že na veky milosrdenství jeho.
\par 4 Rcetež nyní bojící se Hospodina, že na veky milosrdenství jeho.
\par 5 V úzkosti vzýval jsem Hospodina, a vyslyšev, uprostrannil mi Hospodin.
\par 6 Hospodin se mnou, nebudu se báti. Co mi muže uciniti clovek?
\par 7 Hospodin se mnou jest mezi pomocníky mými, procež já podívám se tem, kteríž mne mají v nenávisti.
\par 8 Lépe jest doufati v Hospodina, než nadeji skládati v cloveku.
\par 9 Lépe jest doufati v Hospodina, nežli nadeji skládati v knížatech.
\par 10 Všickni národové obklícili mne, ale ve jménu Hospodinovu vyplénil jsem je.
\par 11 Mnohokrát obklícili mne, ale ve jménu Hospodinovu vyplénil jsem je.
\par 12 Ssuli se na mne jako vcely, však zhasli jako ohen z trní: nebo ve jménu Hospodinovu vyplénil jsem je.
\par 13 Velmi jsi ztuha na mne dotíral, abych padl, ale Hospodin spomohl mi.
\par 14 Síla má a písnicka má jest Hospodin, on byl muj vysvoboditel.
\par 15 Hlas prokrikování a spasení v staních spravedlivých. Pravice Hospodinova dokázala síly,
\par 16 Pravice Hospodinova vyvýšila se, pravice Hospodinova dokázala síly.
\par 17 Neumrut, ale živ budu, abych vypravoval skutky Hospodinovy.
\par 18 Trestalte mne prísne Hospodin, ale smrti mne nevydal.
\par 19 Otevretež mi brány spravedlnosti,a vejda do nich, oslavovati budu Hospodina.
\par 20 Tat jest brána Hospodinova, kterouž spravedliví vcházejí.
\par 21 Tut já te oslavovati budu, nebo jsi mne vyslyšel, a byls muj vysvoboditel.
\par 22 Kámen, kterýž zavrhli stavitelé, ucinen jest v hlavu úhelní.
\par 23 Od Hospodina stalo se to, a jest divné pred ocima našima.
\par 24 Tentot jest den, kterýž ucinil Hospodin, a protož radujme se a veselme se v nem.
\par 25 Prosím, Hospodine, zachovávejž již; prosím, Hospodine, dávej již štastný prospech.
\par 26 Požehnaný, jenž se bére ve jménu Hospodinovu; dobrorecíme vám z domu Hospodinova.
\par 27 Buh silný Hospodin, ont se zasvítil nám, važte beránky až k rohum oltáre.
\par 28 Buh silný muj ty jsi, protož slaviti te budu, Bože muj, vyvyšovati te budu.
\par 29 Oslavujtež Hospodina, nebot jest dobrý, nebo na veky milosrdenství jeho.

\chapter{119}

\par 1 Aleph. Blahoslavení ti, kteríž jsou ctného obcování, kteríž chodí v zákone Hospodinove.
\par 2 Blahoslavení, kteríž ostríhají svedectví jeho, a kteríž ho celým srdcem hledají.
\par 3 Neciní zajisté nepravosti, ale krácejí po cestách jeho.
\par 4 Ty jsi prikázal, aby pilne bylo ostríháno rozkazu tvých.
\par 5 Ó by spraveny byly cesty mé k ostríhání ustanovení tvých.
\par 6 Tehdážt nebudu zahanben, když budu patriti na všecka prikázaní tvá.
\par 7 Oslavovati te budu v uprímnosti srdce, když se vyucovati budu právum spravedlnosti tvé.
\par 8 Ustanovení tvých budu ostríhati s pilností, toliko neopouštej mne.
\par 9 Beth. Jakým zpusobem ocistí mládenec stezku svou? Takovým, aby se choval vedlé slova tvého.
\par 10 Celým srdcem svým hledám tebe, nedopouštejž mi blouditi od prikázaní tvých.
\par 11 V srdci svém skládám rec tvou, abych nehrešil proti tobe.
\par 12 Ty chvály hodný Hospodine, vyuc mne ustanovením svým.
\par 13 Rty svými vypravuji o všech soudech úst tvých.
\par 14 Z cesty svedectví tvých raduji se více, než z nejvetšího zboží.
\par 15 O prikázaních tvých premyšluji, a patrím na stezky tvé.
\par 16 V ustanoveních tvých se kochám, aniž se zapomínám na slovo tvé.
\par 17 Gimel Tu milost ucin s služebníkem svým, abych, dokudž jsem živ, ostríhal slova tvého.
\par 18 Otevri oci mé, abych spatroval divné veci z zákona tvého.
\par 19 Príchozí jsem na tom svete, neukrývejž prede mnou prikázaní svých.
\par 20 Umdlévá duše má pro žádost soudu tvých všelikého casu.
\par 21 Vyhlazuješ pyšné, zlorecené, kteríž bloudí od prikázaní tvých.
\par 22 Odejmi ode mne útržku a potupu, nebot ostríhám svedectví tvých.
\par 23 Také i knížata se zasazují, a mluví proti mne, služebník pak tvuj premýšlí o ustanoveních tvých.
\par 24 Svedectví tvá zajisté jsou mé rozkoše a moji rádcové.
\par 25 Daleth Prilnula k prachu duše má; obživiž mne podlé slova svého.
\par 26 Cesty své predložilt jsem, a vyslýchals mne; vyuc mne ustanovením svým.
\par 27 Ceste rozkazu tvých dej at vyrozumívám, a at premýšlím o divných skutcích tvých.
\par 28 Rozplývá se zámutkem duše má, ocerstviž mne podlé slova svého.
\par 29 Cestu lživou odvrat ode mne, a zákon svuj z milosti dej mi.
\par 30 Cestu pravou jsem vyvolil, soudy tvé sobe predkládám.
\par 31 Svedectví tvých se prídržím, Hospodine, nedejž mi zahanbenu býti.
\par 32 Cestou rozkazu tvých pobehnu, když ty rozšíríš srdce mé.
\par 33 He Vyuc mne, Hospodine, ceste ustanovení svých, kteréž bych ostríhal do konce.
\par 34 Dej mi ten rozum, at šetrím zákona tvého, a at ho ostríhám celým srdcem.
\par 35 Dej, at chodím cestou prikázaní tvých; nebo v tom svou rozkoš skládám.
\par 36 Naklon srdce mého k svedectvím svým, a ne k lakomství.
\par 37 Odvrat oci mé, at nehledí marnosti; na ceste své obživ mne.
\par 38 Potvrd služebníku svému reci své, kterýž se oddal k službe tvé.
\par 39 Odvrat ode mne pohanení, jehož se bojím; nebo soudové tvoji dobrí jsou.
\par 40 Aj, toužím po rozkázaních tvých; dej, at jsem živ v spravedlnosti tvé.
\par 41 Vav Ó at se priblíží ke mne milosrdenství tvá, Hospodine, a spasení tvé vedlé reci tvé,
\par 42 Tak abych odpovedíti umel utrhaci svému skutkem, že doufání skládám v slovu tvém.
\par 43 A nevynímej z úst mých slova nejpravejšího; nebot na soudy tvé ocekávám.
\par 44 I budu ostríhati zákona tvého ustavicne, od veku až na veky,
\par 45 A bez prestání choditi na širokosti, nebot jsem se dotázal rozkazu tvých.
\par 46 Nýbrž mluviti budu o svedectvích tvých i pred králi, a nebudu se hanbiti.
\par 47 Nebo rozkoš svou skládám v prikázaních tvých, kteráž jsem zamiloval.
\par 48 Pricinímt i ruce své k prikázaním tvým, kteráž miluji, a premýšleti budu o ustanoveních tvých.
\par 49 Zajin Rozpomen se na slovo k služebníku svému, kterýmž jsi mne ubezpecil.
\par 50 Tot jest má útecha v ssoužení mém, že mne slovo tvé obživuje.
\par 51 Pyšní mi se velmi posmívají, však od zákona tvého se neuchyluji.
\par 52 Nebot se rozpomínám na soudy tvé vecné, Hospodine, kterýmiž se potešuji.
\par 53 Desím se nad bezbožnými, kteríž opouštejí zákon tvuj.
\par 54 Ustanovení tvá jsou mé písnicky na míste mého putování.
\par 55 Rozpomínám se i v noci na jméno tvé, Hospodine, a ostríhám zákona tvého.
\par 56 Tot mám odtud, abych ostríhal rozkazu tvých.
\par 57 Cheth Díl muj, rekl jsem, Hospodine, ostríhati výmluvnosti tvé.
\par 58 Modlívám se milosti tvé v celém srdci: Smiluj se nade mnou podlé slova svého.
\par 59 Rozvážil jsem na mysli cesty své, a obrátil jsem nohy své k tvým svedectvím.
\par 60 Pospíchámt a neodkládám ostríhati rozkazu tvých.
\par 61 Rota bezbožníku zloupila mne, na zákon tvuj se nezapomínám.
\par 62 O pulnoci vstávám, abych te oslavoval v soudech spravedlnosti tvé.
\par 63 Úcastník jsem všech, kteríž se bojí tebe, a tech, kteríž ostríhají prikázaní tvých.
\par 64 Milosrdenství tvého, Hospodine, plná jest zeme, ustanovením svým vyuc mne.
\par 65 Teth Dobrotive jsi nakládal s služebníkem svým, Hospodine, podlé slova svého.
\par 66 Pravému soudu a umení vyuc mne, nebo jsem prikázaním tvým uveril.
\par 67 Prvé než jsem snížen byl, bloudil jsem, ale nyní výmluvnosti tvé ostríhám.
\par 68 Dobrý jsi ty a dobrotivý, vyuc mne ustanovením svým.
\par 69 Složilit jsou lež proti mne pyšní, ale já celým srdcem ostríhám prikázaní tvých.
\par 70 Zbridlo jako tuk srdce jejich, já zákonem tvým se potešuji.
\par 71 K dobrémut jest mi to, že jsem pobyl v trápení, abych se naucil ustanovením tvým.
\par 72 Za lepší sobe pokládám zákon úst tvých, nežli na tisíce zlata a stríbra.
\par 73 Jod Ruce tvé ucinily a sformovaly mne, dej mi ten rozum, abych se naucil prikázaním tvým,
\par 74 Tak aby bojící se tebe mne vidouce, radovali se, že na slovo tvé ocekávám.
\par 75 Seznávámt, Hospodine, že jsou spravedliví soudové tvoji, a že jsi mne hodne potrestal.
\par 76 Nechat jest již zrejmé milosrdenství tvé ku potešení mému, podlé reci tvé mluvené služebníku tvému.
\par 77 Pridtež na mne slitování tvá, abych živ býti mohl; nebo zákon tvuj rozkoš má jest.
\par 78 Zahanbeni budte pyšní, proto že lstive chteli mne podvrátiti, já pak premyšluji o prikázaních tvých.
\par 79 Obrattež se ke mne, kteríž se bojí tebe, a kteríž znají svedectví tvá.
\par 80 Budiž srdce mé uprímé pri ustanoveních tvých, tak abych nebyl zahanben.
\par 81 Kaph Touží duše má po spasení tvém, na slovo tvé ocekávám.
\par 82 Hynou i oci mé žádostí výmluvností tvých, když ríkám: Skoro-liž mne potešíš?
\par 83 Ackoli jsem jako nádoba kožená v dymu, na ustanovení tvá však jsem nezapomenul.
\par 84 Mnoho-liž bude dní služebníka tvého? Skoro-liž soud vykonáš nad temi, kteríž mi protivenství ciní?
\par 85 Vykopali mi pyšní jámy, kterážto vec není podlé zákona tvého.
\par 86 Všecka prikázaní tvá jsou pravda; bez príciny mi se protiví, spomoziž mi.
\par 87 Témert jsou mne již v nic obrátili na zemi, já jsem však neopustil prikázaní tvých.
\par 88 Podlé milosrdenství svého obživ mne, abych ostríhal svedectví úst tvých.
\par 89 Lamed Na veky, ó Hospodine, slovo tvé trvánlivé jest v nebesích.
\par 90 Od národu do pronárodu pravda tvá, utvrdil jsi zemi, a tak stojí.
\par 91 Vedlé úsudku tvých stojí to vše do dnešního dne, všecko to zajisté jsou služebníci tvoji.
\par 92 Byt zákon tvuj nebyl mé potešení, dávno bych byl zahynul v svém trápení.
\par 93 Na veky se nezapomenu na rozkazy tvé; jimi zajisté obživil jsi mne.
\par 94 Tvujt jsem já, zachovávejž mne; nebo prikázaní tvá zpytuji.
\par 95 Ocekávajít na mne bezbožní, aby mne zahubili, já pak svedectví tvá rozvažuji.
\par 96 Každé veci dokonalé vidím skoncení; rozkaz tvuj jest preširoký náramne.
\par 97 Mem Ó jak miluji zákon tvuj, tak že každého dne on jest mé premyšlování.
\par 98 Nad neprátely mé moudrejšího mne ciníš prikázaními svými; nebo mám je ustavicne pred sebou.
\par 99 Nade všecky své ucitele rozumnejší jsem ucinen; nebo svedectví tvá jsou má premyšlování.
\par 100 I nad starce opatrnejší jsem, nebo prikázaní tvých ostríhám.
\par 101 Od každé cesty zlé zdržuji nohy své, abych ostríhal slova tvého.
\par 102 Od soudu tvých se neodvracuji, proto že ty mne vyucuješ.
\par 103 Ó jak jsou sladké dásním mým výmluvnosti tvé, nad med ústum mým.
\par 104 Z prikázaní tvých rozumnosti jsem nabyl, a protož všeliké cesty bludné nenávidím.
\par 105 Nun Svíce nohám mým jest slovo tvé, a svetlo stezce mé.
\par 106 Prisáhl jsem, což i splním, že chci ostríhati soudu spravedlnosti tvé.
\par 107 Ztrápenýt jsem prenáramne, Hospodine, obživiž mne vedlé slova svého.
\par 108 Dobrovolné obeti úst mých, žádám, oblib, Hospodine, a právum svým vyuc mne.
\par 109 Duše má jest v ustavicném nebezpecenství, a však na zákon tvuj se nezapomínám.
\par 110 Polékli jsou mi bezbožní osídlo, ale já od rozkazu tvých se neodvracím.
\par 111 Za dedictví vecné ujal jsem svedectví tvá, nebot jsou radost srdce mého.
\par 112 Naklonil jsem srdce svého k vykonávání ustanovení tvých ustavicne, až i do konce.
\par 113 Samech Výmyslku nenávidím, zákon pak tvuj miluji.
\par 114 Skrýše má a pavéza má ty jsi, na slovo tvé ocekávám.
\par 115 Odstuptež ode mne nešlechetníci, abych ostríhal prikázaní Boha svého.
\par 116 Zdržujž mne podlé slova svého, tak abych živ byl, a nezahanbuj mne v mém ocekávání.
\par 117 Posiluj mne, abych zachován byl, a patril k ustanovením tvým ustavicne.
\par 118 Potlacuješ všecky ty, kteríž odstupují od ustanovení tvých; nebot jest lživá opatrnost jejich.
\par 119 Jako trusku odmítáš všecky bezbožníky zeme, a protož miluji svedectví tvá.
\par 120 Desí se strachem pred tebou telo mé; nebo soudu tvých bojím se.
\par 121 Ajin Ciním soud a spravedlnost, nevydávejž mne mým násilníkum.
\par 122 Zastup sám služebníka svého k dobrému, tak aby mne pyšní nepotlacili.
\par 123 Oci mé hynou cekáním na spasení tvé, a na výmluvnost spravedlnosti tvé.
\par 124 Nalož s služebníkem svým vedlé milosrdenství svého, a ustanovením svým vyuc mne.
\par 125 Služebník tvuj jsem já, dejž mi rozumnost, abych umel svedectví tvá.
\par 126 Cast jest, abys se pricinil, Hospodine; zrušili zákon tvuj.
\par 127 Z té príciny miluji prikázaní tvá více nežli zlato, i to, kteréž jest nejlepší.
\par 128 A proto, že všecky rozkazy tvé o všech vecech pravé býti poznávám, všeliké stezky bludné nenávidím.
\par 129 Pe Predivnát jsou svedectví tvá, a protož jich ostríhá duše má.
\par 130 Zacátek ucení tvého osvecuje, a vyucuje sprostné rozumnosti.
\par 131 Ústa svá otvírám, a dychtím, nebo prikázaní tvých jsem žádostiv.
\par 132 Popatriž na mne, a smiluj se nade mnou podlé práva tech, kteríž milují jméno tvé.
\par 133 Kroky mé utvrzuj v slovu svém, a nedej, aby nade mnou panovati mela jaká nepravost.
\par 134 Vysvobod mne z nátisku lidských, abych ostríhal rozkazu tvých.
\par 135 Zasvet tvár svou nad služebníkem svým, a ustanovením svým vyuc mne.
\par 136 Potuckové vod vyplývají z ocí mých prícinou tech, kteríž neostríhají zákona tvého.
\par 137 Tsade Spravedlivý jsi, Hospodine, a uprímý v soudech svých.
\par 138 Ty jsi vydal spravedlivá svedectví svá, a vší víry hodná.
\par 139 Až svadnu, tak horlím, že se zapomínají na slovo tvé neprátelé moji.
\par 140 Zprubovanát jest rec tvá dokonale, tou prícinou ji miluje služebník tvuj.
\par 141 Malický a opvržený jsem já, však na rozkazy tvé se nezapomínám.
\par 142 Spravedlnost tvá jest spravedlnost vecná, a zákon tvuj pravda.
\par 143 Ssoužení a nátisk mne stihají, prikázaní tvá jsou mé rozkoše.
\par 144 Spravedlnost svedectví tvých trvá na veky; dej mi z ní rozumnosti nabýti, tak abych živ býti mohl.
\par 145 Koph Z celého srdce volám, vyslyšiš mne, ó Hospodine, abych ostríhal ustanovení tvých.
\par 146 K tobe volám, vysvobod mne, abych šetril svedectví tvých.
\par 147 Predstihám svitání a volám, na tvét slovo ocekávám.
\par 148 Predstihají oci mé bdení proto, abych premýšlel o výmluvnostech tvých.
\par 149 Hlas muj slyš podlé svého milosrdenství, Hospodine, podlé soudu svých obživ mne.
\par 150 Približují se následovníci nešlechetnosti, ti, kteríž se od zákona tvého vzdálili.
\par 151 Ty blíže jsi, Hospodine; nebo všecka prikázaní tvá jsou pravda.
\par 152 Jižt to dávno vím o svedectvích tvých, že jsi je stvrdil až na veky.
\par 153 Reš Popatriž na mé trápení, a vysvobod mne; nebot se na zákon tvuj nezapomínám.
\par 154 Zasad se o mou pri, a ochran mne; pro rec svou obživ mne.
\par 155 Dalekot jest od bezbožných spasení, nebo nedotazují se na ustanovení tvá.
\par 156 Slitování tvá mnohá jsou, Hospodine; podlé soudu svých obživ mne.
\par 157 Jakžkoli jsou mnozí protivníci moji a neprátelé moji, však od svedectví tvých se neuchyluji.
\par 158 Videl jsem ty, kteríž se prevrácene meli, velmi to težce nesa, že reci tvé neostríhali.
\par 159 Popatriž, žet rozkazy tvé miluji, Hospodine; podlé milosrdenství svého obživ mne.
\par 160 Nejprednejší vec slova tvého jest pravda, a na veky trvá všeliký úsudek spravedlnosti tvé.
\par 161 Šin Knížata mi se protiví bez príciny, však slova tvého desí se srdce mé.
\par 162 Já raduji se z reci tvé tak jako ten, kterýž dochází hojné koristi.
\par 163 Falše pak nenávidím, a jí v ohavnosti mám; zákon tvuj miluji.
\par 164 Sedmkrát za den chválím te z soudu tvých spravedlivých.
\par 165 Pokoj mnohý tem, kteríž milují zákon tvuj, a nemají žádné urážky.
\par 166 Ocekávám na spasení tvé, Hospodine,a prikázaní tvá vykonávám.
\par 167 Ostríhá duše má svedectví tvých, nebo je velice miluji.
\par 168 Ostríhám rozkazu tvých a svedectví tvých; nebo všecky cesty mé jsou pred tebou.
\par 169 Thav Predstupiž úpení mé pred oblícej tvuj, Hospodine, a podlé slova svého udel mi rozumnosti.
\par 170 Vejdiž pokorná prosba má pred tvár tvou, a vedlé reci své vytrhni mne.
\par 171 I vynesou rtové moji chválu, když ty mne vyucíš ustanovením svým.
\par 172 Zpívati bude i jazyk muj slovo tvé, a že všecka prikázaní tvá jsou spravedlnost.
\par 173 Budiž mi ku pomoci ruka tvá; nebot jsem sobe zvolil prikázaní tvá.
\par 174 Toužím po spasení tvém, Hospodine,a zákon tvuj jest rozkoš má.
\par 175 Živa bude duše má, a bude te chváliti,a soudové tvoji budou mi na pomoc.
\par 176 Bloudím jako ovce ztracená, hledejž služebníka svého, nebot se na prikázaní tvá nezapomínám.

\chapter{120}

\par 1 Písen stupnu. K Hospodinu v ssoužení svém volal jsem,a vyslyšel mne.
\par 2 Hospodine, vysvobod duši mou od rtu lživých, a od jazyka lstivého.
\par 3 Cot prospeje, aneb cot pridá jazyk lstivý,
\par 4 Podobný k strelám preostrým silného,a k uhlí jalovcovému?
\par 5 Beda mne, že pohostinu býti musím v Mešech, a prebývati v saláších Cedarských.
\par 6 Dlouho bydlí duše má mezi temi, kteríž nenávidí pokoje.
\par 7 Já ku pokoji, ale když mluvím, oni k boji.

\chapter{121}

\par 1 Písen stupnu. Pozdvihuji ocí svých k horám, odkudž by mi prišla pomoc.
\par 2 Pomoc má jest od Hospodina, kterýž ucinil nebe i zemi.
\par 3 Nedopustít, aby se pohnouti mela noha tvá, nedrímet strážný tvuj.
\par 4 Aj, nedrímet, ovšem nespí ten, kterýž ostríhá Izraele.
\par 5 Hospodin strážce tvuj, Hospodin zastínení tvé tobe po pravici.
\par 6 Nebudet bíti na te slunce ve dne, ani mesíc v noci.
\par 7 Hospodin te ostríhati bude ode všeho zlého, ostríhati bude duše tvé.
\par 8 Hospodin ostríhati te bude, když vycházeti i vcházeti budeš, od tohoto casu až na veky.

\chapter{122}

\par 1 Písen stupnu, Davidova. Veselím se z toho, že mi ríkáno bývá: Podme do domu Hospodinova,
\par 2 A že se postavují nohy naše v branách tvých, ó Jeruzaléme.
\par 3 Jižte Jeruzalém ušlechtile vystaven, a jako v mesto k sobe vespolek pripojen.
\par 4 Do nehož vstupují pokolení, pokolení Hospodinova, k svedectví Izraelovu, aby oslavovali jméno Hospodinovo.
\par 5 Nebo tamt jsou postaveny stolice soudu, stolice domu Davidova.
\par 6 Žádejtež pokoje Jeruzalému, rkouce: Dejž se pokojne tem, kteríž te milují.
\par 7 Budiž pokoj v predhradí tvém, a upokojení na palácích tvých.
\par 8 Pro bratrí své a prátely své žádati budu pokoje tobe.
\par 9 Pro dum Hospodina Boha našeho budu tvého dobrého hledati.

\chapter{123}

\par 1 Písen stupnu. K tobet pozdvihuji ocí svých, ó ty, kterýž na nebesích prebýváš.
\par 2 Aj hle, jakož oci služebníku k rukám pánu jejich, jakož oci devky k rukám paní její: tak oci naše k Hospodinu Bohu našemu, až by se smiloval nad námi.
\par 3 Smiluj se nad námi, Hospodine, smiluj se nad námi, nebot jsme již príliš potupou nasyceni.
\par 4 Jižt jest príliš nasycena duše naše posmíšky bezbožných, a potupou pyšných.

\chapter{124}

\par 1 Písen stupnu, Davidova. Byt Hospodina s námi nebylo, rciž nyní, Izraeli,
\par 2 Byt Hospodina s námi nebylo, když lidé povstali proti nám:
\par 3 Tehdáž by nás byli za živa sehltili v rozpálení hnevu svého proti nám;
\par 4 Tehdáž by nás byly prikvacily vody, proud zachvátil by byl duši naši;
\par 5 Tehdáž zachvátily by byly duši naši ty vody zduté.
\par 6 Požehnaný Hospodin, kterýž nás nevydal v loupež zubum jejich.
\par 7 Duše naše jako ptáce znikla osídla ptácníku; osídlo se ztrhalo, i vynikli jsme.
\par 8 Pomoc naše jest ve jménu Hospodinovu, kterýž ucinil nebe i zemi.

\chapter{125}

\par 1 Písen stupnu. Ti, kteríž doufají v Hospodina, podobni jsou k hore Sionu, kteráž se nepohybuje, ale na veky zustává.
\par 2 Okolo Jeruzaléma jsou hory, Hospodin jest vukol lidu svého, od tohoto casu až na veky.
\par 3 Nebot nebude státi sceptrum bezbožníku nad losem spravedlivých, aby nevztáhli spravedliví k nepravosti rukou svých.
\par 4 Dobre ucin, Hospodine, dobrým, a tem, kteríž jsou uprímého srdce.
\par 5 Ty pak, kteríž se uchylují k cestám svým krivým, zapudiž Hospodin s ciniteli nepravosti. Pokoj prijdiž na Izraele.

\chapter{126}

\par 1 Písen stupnu. Když zase vedl Hospodin zajaté Sionské, zdálo se nám to jako ve snách.
\par 2 Tehdážt byla plná radosti ústa naše,a jazyk náš plésání; tehdáž pravili mezi národy: Veliké veci s nimi ucinil Hospodin.
\par 3 Ucinilt jest s námi veliké veci Hospodin, a protož veselili jsme se.
\par 4 Uvediž zase, ó Hospodine, zajaté naše, tak jako potoky na vyprahlou krajinu.
\par 5 Ti, kteríž se slzami rozsívali, s prozpevováním žíti budou.
\par 6 Sem i tam chodící lid s plácem rozsívá drahé síme, ale potom prijda, s plésáním snášeti bude snopy své.

\chapter{127}

\par 1 Písen stupnu, Šalomounova. Nebude-li Hospodin staveti domu, nadarmo usilují ti, kteríž stavejí jej; nebude-li Hospodin ostríhati mesta, nadarmo bdí strážný.
\par 2 Daremnét jest vám ráno vstávati, dlouho sedati, a jísti chléb bolesti, ponevadž Buh dává milému svému i sen.
\par 3 Aj, dedictví od Hospodina jsou dítky, a plod života jest mzda.
\par 4 Jako strely v ruce udatného, tak jsou dítky zdárné.
\par 5 Blahoslavený muž, kterýž by jimi naplnil toul svuj; nebudout zahanbeni, když v rozepri budou s neprátely v branách.

\chapter{128}

\par 1 Písen stupnu. Blahoslavený každý, kdo se bojí Hospodina, a chodí po cestách jeho.
\par 2 Nebo z práce rukou svých živnost míti budeš, blahoslavený budeš, a štastnet se povede.
\par 3 Manželka tvá jako vinný kmen plodný po bocích domu tvého, dítky tvé jako mladistvé olivoví vukol stolu tvého.
\par 4 Aj, takovét bude míti požehnání muž bojící se Hospodina.
\par 5 Požehnání tobe udeliž Hospodin z Siona, a ty spatruj dobré veci Jeruzaléma po všecky dny života svého;
\par 6 A viz syny synu svých, a pokoj nad Izraelem.

\chapter{129}

\par 1 Písen stupnu. Velicet jsou mne ssužovali hned od mladosti mé, rciž nyní Izraeli,
\par 2 Velicet jsou mne ssužovali hned od mladosti mé, a však mne nepremohli.
\par 3 Po hrbete mém orali oráci, a dlouhé proháneli brázdy své.
\par 4 Ale Hospodin jsa spravedlivý, zpretínal prostranky bezbožných.
\par 5 Zahanbeni a zpet obráceni budou všickni, kteríž nenávidí Siona.
\par 6 Budou jako tráva na strechách, kteráž prvé než odrostá, usychá.
\par 7 Z níž nemuže hrsti své naplniti žnec, ani nárucí svého ten, kterýž váže snopy.
\par 8 Aniž reknou tudy jdoucí: Požehnání Hospodinovo budiž s vámi, aneb: Dobrorecíme vám ve jménu Hospodinovu.

\chapter{130}

\par 1 Písen stupnu. Z hlubokosti volám k tobe, Hospodine.
\par 2 Pane, vyslyš hlas muj, naklon uší svých k hlasu pokorných proseb mých.
\par 3 Budeš-li nepravosti šetriti, Hospodine Pane, kdo ostojí?
\par 4 Ale u tebe jest odpuštení, tak aby uctivost k tobe zachována byla.
\par 5 Ocekávám na Hospodina, ocekává duše má, a ješte ocekává na slovo jeho.
\par 6 Duše má ceká Pána, víc než ponocní svitání, kteríž ponocují až do jitra.
\par 7 Ocekávejž, Izraeli, na Hospodina; nebo u Hospodina jest milosrdenství, a hojné u neho vykoupení.
\par 8 Ont zajisté vykoupí Izraele ze všech nepravostí jeho.

\chapter{131}

\par 1 Písen stupnu, Davidova. Hospodine, nepozdvihlote se srdce mé, ani se povýšily oci mé, aniž jsem se vydal v veci veliké, aneb vyšší nad to, než mi náleží.
\par 2 Zdali jsem nepoložil a neupokojil duše své, jako díte ostavené od matky své? Ostavenému podobná byla ve mne duše má.
\par 3 Doufej, ó Izraeli, v Hospodina, od tohoto casu až na veky.

\chapter{132}

\par 1 Písen stupnu. Pametliv bud, Hospodine, na Davida i na všecka trápení jeho,
\par 2 Jak se prísahou zavázal Hospodinu, a slib ucinil Nejmocnejšímu Jákobovu, rka:
\par 3 Jiste že nevejdu do stánku domu svého, a nevstoupím na postel ložce svého,
\par 4 Aniž dám ocím svým usnouti, ani víckám svým zdrímati,
\par 5 Dokudž nenajdu místa Hospodinu, k príbytkum Nejmocnejšímu Jákobovu.
\par 6 Aj, uslyšavše o ní, že byla v kraji Efratském, našli jsme ji na polích Jaharských.
\par 7 Vejdemet již do príbytku jeho, a skláneti se budeme u podnoží noh jeho.
\par 8 Povstaniž, Hospodine, a vejdi do odpocinutí svého, ty i truhla velikomocnosti tvé.
\par 9 Kneží tvoji at se zoblácejí v spravedlnost, a svatí tvoji at vesele prozpevují.
\par 10 Pro Davida služebníka svého neodvracejž tvári pomazaného svého.
\par 11 Ucinilt jest Hospodin pravdomluvnou prísahu Davidovi, aniž se od ní uchýlí, rka: Z plodu života tvého posadím na trun tvuj.
\par 12 Budou-li ostríhati synové tvoji smlouvy mé a svedectví mých, kterýmž je vyucovati budu, také i synové jejich až na veky sedeti budou na stolici tvé.
\par 13 Nebot jest vyvolil Hospodin Sion, oblíbil jej sobe za svuj príbytek, rka:
\par 14 Tot bude obydlí mé až na veky, tut prebývati budu, nebo jsem sobe to oblíbil.
\par 15 Potravu jeho hojným požehnáním rozmnožím, chudé jeho chlebem nasytím,
\par 16 A kneží jeho v spasení zoblácím, a svatí jeho vesele prozpevovati budou.
\par 17 Tut zpusobím, aby zkvetl roh Daviduv; pripravím svíci pomazanému svému.
\par 18 Neprátely jeho v hanbu zoblácím, nad ním pak kvésti bude koruna jeho.

\chapter{133}

\par 1 Písen stupnu, Davidova. Aj, jak dobré a jak utešené, když bratrí v jednomyslnosti prebývají!
\par 2 Jako mast výborná na hlave, sstupující na bradu, bradu Aronovu, tekoucí až i na podolek roucha jeho.
\par 3 A jako rosa Hermon, kteráž sstupuje na hory Sionské. Nebo tu udílí Hospodin požehnání i života až na veky.

\chapter{134}

\par 1 Písen stupnu. Ej nuž dobrorecte Hospodinu všickni služebníci Hospodinovi, kteríž stáváte v dome Hospodinove každé noci.
\par 2 Pozdvihujte rukou svých k svatyni, a dobrorecte Hospodinu, ríkajíce:
\par 3 Požehnejž tobe Hospodin z Siona, kterýž ucinil nebe i zemi.

\chapter{135}

\par 1 Halelujah. Chvalte jméno Hospodinovo, chvalte služebníci Hospodinovi,
\par 2 Kteríž stáváte v dome Hospodinove, v síncích domu Boha našeho.
\par 3 Chvalte Hospodina, nebo jest dobrý Hospodin; žalmy zpívejte jménu jeho, nebo rozkošné jest.
\par 4 Jákoba zajisté sobe vyvolil Hospodin, a Izraele za svuj lid zvláštní.
\par 5 Ját jsem jiste seznal, že veliký jest Hospodin, a Pán náš nade všecky bohy.
\par 6 Cožkoli chce Hospodin, to ciní na nebi i na zemi, v mori i ve všech propastech.
\par 7 Kterýž zpusobuje to, že páry vystupují od kraju zeme; blýskání s deštem privodí, a vyvodí vítr z pokladu svých.
\par 8 Kterýž zbil prvorozené v Egypte, od cloveka až do hovada.
\par 9 Poslal znamení a zázraky u prostred tebe, Egypte, na Faraona i na všecky služebníky jeho.
\par 10 Kterýž pobil národy mnohé, a zbil krále mocné,
\par 11 Seona krále Amorejského, a Oga krále Bázan, i všecka království Kananejská.
\par 12 A dal zemi jejich v dedictví, v dedictví Izraelovi lidu svému.
\par 13 Hospodine, jméno tvé na veky, Hospodine, památka tvá od národu až do pronárodu.
\par 14 Souditi zajisté bude Hospodin lid svuj, a služebníkum svým bude milostiv.
\par 15 Ale modly pohanské stríbro a zlato, dílo rukou lidských,
\par 16 Ústa mají a nemluví, oci mají a nevidí.
\par 17 Uši mají a neslyší, nýbrž ani ducha není v ústech jejich.
\par 18 Budtež jim podobní, kteríž je delají, a kdožkoli nadeji svou v nich skládají.
\par 19 Dome Izraelský, dobrorecte Hospodinu; dome Aronuv, dobrorecte Hospodinu.
\par 20 Dome Léví, dobrorecte Hospodinu; kteríž se bojíte Hospodina, dobrorecte Hospodinu.
\par 21 Požehnaný Hospodin z Siona, kterýž prebývá v Jeruzaléme. Halelujah.

\chapter{136}

\par 1 Oslavujte Hospodina, nebo jest dobrý, nebo vecné jest milosrdenství jeho.
\par 2 Oslavujte Boha bohu, nebo jest vecné milosrdenství jeho.
\par 3 Oslavujte Pána pánu, nebo jest vecné milosrdenství jeho.
\par 4 Toho, kterýž sám ciní divy veliké, nebo jest vecné milosrdenství jeho.
\par 5 Kterýž ucinil nebesa moudre, nebo jest vecné milosrdenství jeho.
\par 6 Kterýž roztáhl zemi na vodách, nebo jest vecné milosrdenství jeho.
\par 7 Kterýž ucinil svetla veliká, nebo jest vecné milosrdenství jeho.
\par 8 Slunce, aby panovalo ve dne, nebo jest vecné milosrdenství jeho.
\par 9 Mesíc a hvezdy, aby panovaly v noci, nebo jest vecné milosrdenství jeho.
\par 10 Kterýž ranil Egyptské v prvorozených jejich, nebo jest vecné milosrdenství jeho.
\par 11 A vyvedl Izraele z prostredku jejich, nebo jest vecné milosrdenství jeho.
\par 12 V ruce silné a v rameni vztaženém, nebo jest vecné milosrdenství jeho.
\par 13 Kterýž rozdelil more Rudé na díly, nebo jest vecné milosrdenství jeho.
\par 14 A prevedl Izraele prostredkem jeho, nebo jest vecné milosrdenství jeho.
\par 15 A uvrhl Faraona s vojskem jeho do more Rudého, nebo jest vecné milosrdenství jeho.
\par 16 Kterýž vedl lid svuj pres poušt, nebo jest vecné milosrdenství jeho.
\par 17 Kterýž pobil krále veliké, nebo jest vecné milosrdenství jeho.
\par 18 A zbil krále znamenité, nebo jest vecné milosrdenství jeho.
\par 19 Seona krále Amorejského, nebo jest vecné milosrdenství jeho.
\par 20 Též Oga krále Bázan, nebo jest vecné milosrdenství jeho.
\par 21 A dal zemi jejich v dedictví, nebo jest vecné milosrdenství jeho.
\par 22 V dedictví Izraelovi, služebníku svému, nebo jest vecné milosrdenství jeho.
\par 23 Kterýž v snížení našem pamatuje na nás, nebo jest vecné milosrdenství jeho.
\par 24 A vytrhl nás z neprátel našich, nebo jest vecné milosrdenství jeho.
\par 25 Kterýž dává pokrm všelikému telu, nebo jest vecné milosrdenství jeho.
\par 26 Oslavujte Boha silného nebes, nebot jest vecné milosrdenství jeho.

\chapter{137}

\par 1 Pri rekách Babylonských tam jsme sedávali, a plakávali, rozpomínajíce se na Sion.
\par 2 Na vrbí v té zemi zavešovali jsme citary své.
\par 3 A když se tam dotazovali nás ti, kteríž nás zajali, na slova písnicky, (ješto jsme zavesili byli veselí), ríkajíce: Zpívejte nám nekterou písen Sionskou:
\par 4 Kterakž bychom meli zpívati písen Hospodinovu v zemi cizozemcu?
\par 5 Jestliže se zapomenu na tebe, ó Jeruzaléme, zapomeniž i pravice má.
\par 6 Prilniž i jazyk muj k dásním mým, nebudu-li se rozpomínati na tebe, jestliže v samém Jeruzaléme nebudu míti svého nejvetšího potešení.
\par 7 Rozpomen se, Hospodine, na Idumejské, a na den Jeruzaléma, kteríž pravili: Rozborte, rozborte až do základu v nem.
\par 8 Ó dcero Babylonská, zkažena býti máš. Blahoslavený ten, kdož odplatí tobe za to, což jsi nám zlého ucinila.
\par 9 Blahoslavený, kdož pochytí dítky tvé a o skálu je rozrážeti bude.

\chapter{138}

\par 1 Daviduv. Oslavovati te budu, Pane, celým srdcem svým, a pred mocnými žalmy tobe zpívati.
\par 2 Skláneti se budu k chrámu svatému tvému, a oslavovati jméno tvé pro milosrdenství tvé a pro pravdu tvou; nebo jsi zvelebil nade všecko jméno své a slovo své.
\par 3 Kteréhokoli dne vzýval jsem te, vyslyšels mne, a obdarils silou duši mou.
\par 4 Oslavovati te budou, Hospodine, i všickni králové zeme, když uslyší rec úst tvých.
\par 5 A zpívati budou o cestách Hospodinových, a že veliká jest sláva Hospodinova,
\par 6 A ac vyvýšený jest Hospodin, však že na poníženého patrí, a vysokomyslného zdaleka zná.
\par 7 Bych pak chodil u prostred ssoužení, obživíš mne; proti vzteklosti neprátel mých vztáhneš ruku svou, a tak vysvobodí mne pravice tvá.
\par 8 Hospodin dokoná za mne; nebo milosrdenství tvé, Hospodine, na veky, aniž díla rukou svých kdy opustíš.

\chapter{139}

\par 1 Prednímu zpeváku, žalm Daviduv. Hospodine, ty jsi mne zkusil a seznal.
\par 2 Ty znáš sednutí mé i povstání mé, rozumíš myšlení mému zdaleka.
\par 3 Chození mé i ležení mé ty obsahuješ, a všech mých cest svedom jsi.
\par 4 Než ješte mám na jazyku slovo, aj, Hospodine, ty to všecko víš.
\par 5 Z zadu i z predu obklícils mne, a vzložils na mne ruku svou.
\par 6 Divnejší jest umení tvé nad muj vtip; vysoké jest, nemohu k nemu.
\par 7 Kamž bych zašel od ducha tvého? Aneb kam bych pred tvárí tvou utekl?
\par 8 Jestliže bych vstoupil na nebe, tam jsi ty; pakli bych sobe ustlal v hrobe, aj, prítomen jsi.
\par 9 Vzal-li bych krídla záre jitrní, abych bydlil pri nejdalším mori:
\par 10 I tamt by mne ruka tvá provedla, a pravice tvá by mne popadla.
\par 11 Dím-li pak: Aspon tmy, jako v soumrak, prikryjí mne, ale i noc jest svetlem vukol mne.
\par 12 Aniž ty tmy pred tebou ukryti mohou, anobrž noc jako den tobe svítí, rovne tma jako svetlo.
\par 13 Ty zajisté v moci máš ledví má, priodel jsi mne v živote matky mé.
\par 14 Oslavuji te, proto že se hrozným a divným skutkum tvým divím, a duše má zná je výborne.
\par 15 Nenít ukryta žádná kost má pred tebou, jakž jsem ucinen v skryte, a remeslne složen, v nejhlubších místech zeme.
\par 16 Trupel muj videly oci tvé, v knihu tvou všickni oudové jeho zapsáni jsou, i dnové,v nichž formováni byli, když ješte žádného z nich nebylo.
\par 17 Protož u mne ó jak drahá jsou myšlení tvá, Bože silný, a jak jest jich nescíslná summa!
\par 18 Chtel-li bych je scísti, více jest jich než písku; procítím-li, a já jsem vždy s tebou.
\par 19 Zabil-li bys, ó Bože, bezbožníka, tehdážt by muži vražedlní odstoupili ode mne,
\par 20 Kteríž mluví proti tobe nešlechetne; marne vyvyšují neprátely tvé.
\par 21 Zdaliž tech, kteríž te v nenávisti mají, ó Hospodine, v nenávisti nemám? A ti, kteríž proti tobe povstávají, zdaž mne nemrzejí?
\par 22 Úhlavní nenávistí jich nenávidím, a mám je za neprátely.
\par 23 Vyzpytuj mne, Bože silný, a poznej srdce mé; zkus mne, a poznej myšlení má.
\par 24 A popatr, chodím-lit já cestou odpornou tobe, a ved mne cestou vecnou.

\chapter{140}

\par 1 Prednímu z kantoru, žalm Daviduv.
\par 2 Vysvobod mne, Hospodine, od cloveka zlého, a od muže ukrutného ostríhej mne,
\par 3 Kteríž myslí zlé veci v srdci, a na každý den sbírají se k válce.
\par 4 Naostrují jazyk svuj jako had, jed lítého hada jest ve rtech jejich. Sélah.
\par 5 Ostríhej mne, Hospodine, od rukou bezbožníka, od muže ukrutného zachovej mne, kteríž myslí podraziti nohy mé.
\par 6 Polékli pyšní na mne osídlo a provazy, roztáhli teneta u cesty, a lécky své mi položili. Sélah.
\par 7 Rekl jsem Hospodinu: Buh silný muj jsi, pozoruj, Hospodine, hlasu pokorných modliteb mých.
\par 8 Hospodine Pane, sílo spasení mého, kterýž prikrýváš hlavu mou v cas boje,
\par 9 Nedávej, Hospodine, bezbožnému, cehož žádostiv jest, ani predsevzetí zlého vykonati jemu dopouštej, aby se nepovýšil. Sélah.
\par 10 Vudce tech, jenž obklicují mne, nepravost rtu jejich at prikryje.
\par 11 Padej na ne uhlé reravé, a na ohen uvrz je, do jam hlubokých, aby nemohli povstati.
\par 12 Clovek utrhac nebude upevnen na zemi, a muž ukrutný, zlostí polapen jsa, padne.
\par 13 Vím, žet se Hospodin zasadí o pri chudého, a pomstí nuzných.
\par 14 A tak spravedliví slaviti budou jméno tvé, a uprímí prebývati pred oblícejem tvým.

\chapter{141}

\par 1 Žalm Daviduv. Hospodine, k tobet volám, pospeš ke mne; pozoruj hlasu mého, když tebe vzývám.
\par 2 Budiž príjemná modlitba má, jako kadení pred oblícejem tvým, pozdvižení rukou mých, jako obet vecerní.
\par 3 Polož, Hospodine, stráž ústum mým, ostríhej dverí rtu mých.
\par 4 Nedopouštej srdci mému uchýliti se ke zlé veci, k cinení skutku bezbožných, s muži cinícími nepravost, a abych nebyl prelouzen líbostmi jejich.
\par 5 Necht mne bije spravedlivý, prijmu to za dobrodiní, a necht tresce mne, bude mi to olej nejcistší, kterýž neprorazí hlavy mé, ale ještet modlitba má platná bude proti zlosti jejich.
\par 6 Smetáni jsou do míst skalnatých soudcové jejich, aby slyšeli slova má, nebo jsou libá.
\par 7 Jako když nekdo roubá a štípá dríví na zemi, tak se rozletují kosti naše až k ústum hrobovým.
\par 8 Ale k tobet, Hospodine Pane, oci mé; v tebe doufám, nevylévej duše mé.
\par 9 Zachovej mne od osídla, kteréž mi roztáhli, a od sítek cinících nepravost.
\par 10 Necht padnou hromadne do sítek svých bezbožní, a já zatím prejdu.

\chapter{142}

\par 1 Vyucující Daviduv, když byl v jeskyni, modlitba jeho.
\par 2 Hlasem svým k Hospodinu volám, hlasem svým Hospodinu pokorne se modlím.
\par 3 Vylévám pred oblícejem jeho žádost svou, a ssoužení své pred ním oznamuji.
\par 4 Když se úzkostmi svírá ve mne duch muj, ty znáš stezku mou; na ceste, po kteréžkoli chodím, osídlo mi ukryli.
\par 5 Ohlédám-li se na pravo, a patrím, není, kdo by mne znáti chtel; zhynulo útocište mé, není, kdo by se ujal o život muj.
\par 6 K tobe volám, Hospodine, ríkaje: Ty jsi doufání mé a díl muj v zemi živých.
\par 7 Pozorujž volání mého, nebot jsem zemdlen prenáramne; vysvobod mne od tech, jenž stihají mne, nebo jsou silnejší nežli já.
\par 8 Vyved z žaláre duši mou, abych oslavoval jméno tvé; obstoupí mne spravedliví, když mi dobrodiní uciníš.

\chapter{143}

\par 1 Žalm Daviduv. Hospodine, slyš modlitbu mou, pozoruj pokorné prosby mé; pro pravdu svou vyslyš mne, i pro spravedlnost svou.
\par 2 A nevcházej v soud s služebníkem svým, nebot by nebyl spravedliv pred tebou nižádný živý.
\par 3 Nebo stihá neprítel duši mou, potírá až k zemi život muj; na to mne privodí, abych bydlil v mrákote, jako ti, kteríž již dávno zemreli,
\par 4 Tak že se svírá úzkostmi duch muj ve mne, u vnitrnosti mé hyne srdce mé.
\par 5 Rozpomínaje se na dny predešlé, a rozvažuje všecky skutky tvé, a dílo rukou tvých rozjímaje,
\par 6 Vztahuji ruce své k tobe, duše má jako zeme vyprahlá žádá tebe. Sélah.
\par 7 Pospešiž a vyslyš mne, Hospodine, hyne duch muj; neukrývejž tvári své prede mnou, nebot jsem podobný tem, kteríž sstupují do hrobu.
\par 8 Ucin to, at v jitre slyším milosrdenství tvé, nebot v tobe nadeji mám; oznam mi cestu, po kteréž bych choditi mel, nebot k tobe pozdvihuji duše své.
\par 9 Vytrhni mne z neprátel mých, Hospodine, u tebet se skrývám.
\par 10 Nauc mne ciniti vule tvé, nebo ty jsi Buh muj; duch tvuj dobrý vediž mne jako po rovné zemi.
\par 11 Pro jméno své, Hospodine, obživ mne, pro spravedlnost svou vyved z úzkosti duši mou.
\par 12 A pro milosrdenství své vyplen neprátely mé, a vyhlad všecky, kteríž trápí duši mou; nebo já jsem služebník tvuj.

\chapter{144}

\par 1 Daviduv. Požehnaný Hospodin skála má, kterýž ucí ruce mé boji, a prsty mé bitve.
\par 2 Milosrdenství mé a hrad muj, útocište mé, vysvoboditel muj, a štít muj, protož v nehot já doufám; ont mi podmanuje lidi.
\par 3 Hospodine, co jest clovek, že se znáš k nemu, a syn cloveka, že ho sobe tak vážíš?
\par 4 Clovek marnosti podobný jest, dnové jeho jako stín pomíjející.
\par 5 Hospodine, naklon svých nebes a sstup, dotkni se hor, a kouriti se budou.
\par 6 Sešli hromobití a rozptyl je, vypust strely své a poraz je.
\par 7 Vztáhni ruku svou s výsosti, vysvobod mne, a vytrhni mne z vod mnohých, z ruky cizozemcu.
\par 8 Jejichž ústa mluví marnost, a pravice jejich jest pravice lživá.
\par 9 Bože, písen novou zpívati budu tobe na loutne, a na desíti strunách žalmy tobe prozpevovati,
\par 10 Dávajícímu vítezství králum, a vysvobozujícímu Davida, služebníka svého od mece vražedlného.
\par 11 Vysvobod mne, a vytrhni mne z ruky cizozemcu, jejichž ústa mluví marnost, a pravice jejich pravice lživá.
\par 12 Aby synové naši byli jako štípkové zdárne rostoucí v mladosti své, a dcery naše jako úhelní kamenové, tesaní ku podobenství chrámu.
\par 13 Špižírny naše plné at vydávají všelijaké potravy; dobytek náš at rodí na tisíce, a na deset tisícu v stájích našich.
\par 14 Volové naši at jsou vytylí; at není vpádu ani zajetí, ani naríkání na ulicích našich.
\par 15 Blahoslavený lid, jemuž se tak deje, blahoslavený ten lid, jehož Hospodin Bohem jest.

\chapter{145}

\par 1 Chvalitebná písen Davidova. Vyvyšovati te budu, Bože muj králi, a dobroreciti jménu tvému na veky veku.
\par 2 Na každý den dobroreciti budu tobe, a chváliti jméno tvé na veky veku.
\par 3 Hospodin veliký jest, a vší chvály hodný, a velikost jeho nemuž vystižena býti.
\par 4 Rodina rodine vychvalovati bude skutky tvé, a predivnou moc tvou zvestovati.
\par 5 O sláve a kráse velebnosti tvé, i o vecech tvých predivných mluviti budu.
\par 6 A moc prehrozných skutku tvých rozhlašovati budou; i já dustojnost tvou budu vypravovati.
\par 7 Pamet mnohé dobroty tvé hlásati budou, a o spravedlnosti tvé zpívati, rkouce:
\par 8 Milostivý a lítostivý jest Hospodin, dlouhoshovívající a velikého milosrdenství.
\par 9 Dobrotivý Hospodin všechnem, a slitování jeho nade všecky skutky jeho.
\par 10 Oslavujtež tebe, Hospodine, všickni skutkové tvoji, a svatí tvoji tobe dobrorecte.
\par 11 Slávu království tvého at vypravují, a o síle tvé mluví,
\par 12 Aby v známost uvedli synum lidským moci jeho, a slávu i ozdobu království jeho.
\par 13 Království tvé jest království všech veku, a panování tvé nad jedním každým pokolením.
\par 14 Zdržujet Hospodin všecky padající, a pozdvihuje všechnech sklícených.
\par 15 Oci všechnech v tebe doufají, a ty dáváš jim pokrm jejich v cas príhodný.
\par 16 Otvíráš ruku svou, a nasycuješ každý živocich podlé dobre líbezné vule své.
\par 17 Spravedlivý jest Hospodin ve všech cestách svých, a milosrdný ve všech skutcích svých.
\par 18 Blízký jest Hospodin všechnem, kteríž ho vzývají, všechnem, kteríž ho vzývají v pravde.
\par 19 Vuli tech, kteríž se ho bojí, ciní, a krik jejich slyší, a spomáhá jim.
\par 20 Ostríhá Hospodin všech, kdož jej milují, ale všecky bezbožné zatratí.
\par 21 Chválu Hospodinovu vypravovati budou ústa má, a dobroreciti bude všeliké telo jménu svatému jeho od veku až na veky.

\chapter{146}

\par 1 Halelujah. Chval, duše má, Hospodina.
\par 2 Chváliti budu Hospodina, dokud jsem živ, žalmy zpívati Bohu svému, dokud mne stává.
\par 3 Nedoufejtež v knížatech, v synech lidských, v nichž není vysvobození.
\par 4 Vychází duch jejich, navracují se do zeme své, v tentýž den mizejí myšlení jejich.
\par 5 Blahoslavený ten, jehož spomocník jest Buh silný Jákobuv, jehož nadeje jest v Hospodinu Bohu jeho,
\par 6 Kterýž ucinil nebe, zemi, more, i vše, což v nich jest, kterýž ostríhá pravdy až na veky,
\par 7 Kterýž ciní soud utišteným, dává chléb lacným. Hospodin vysvobozuje vezne.
\par 8 Hospodin otvírá oci slepých, Hospodin pozdvihuje snížených, Hospodin miluje spravedlivé.
\par 9 Hospodin ostríhá príchozích, sirotku a vdove pomáhá, ale cestu bezbožných podvrací.
\par 10 Kralovati bude Hospodin na veky, Buh tvuj, ó Sione, od národu až do pronárodu. Halelujah.

\chapter{147}

\par 1 Chvalte Hospodina, nebo dobré jest zpívati žalmy Bohu našemu, nebo rozkošné jest, a ozdobná jest chvála.
\par 2 Stavitel Jeruzaléma Hospodin, rozptýlený lid Izraelský shromažduje,
\par 3 Kterýž uzdravuje skroušené srdcem, a uvazuje bolesti jejich,
\par 4 Kterýž scítá pocet hvezd, a každé z nich ze jména povolává.
\par 5 Velikýt jest Pán náš, a nesmírný v síle; rozumnosti jeho není poctu.
\par 6 Pozdvihuje pokorných Hospodin, ale bezbožné snižuje až k zemi.
\par 7 Zpívejte Hospodinu s díkcinením, zpívejte žalmy Bohu našemu na citare,
\par 8 Kterýž zastírá nebesa hustými oblaky, nastrojuje zemi déšt, a vyvodí trávu na horách.
\par 9 Kterýž dává hovadum potravu jejich, i mladým krkavcum, kteríž volají k nemu.
\par 10 Nemát v síle kone zalíbení, aniž se kochá v lejtkách muže udatného.
\par 11 Líbost má Hospodin v tech, kteríž se ho bojí, a kteríž doufají v milosrdenství jeho.
\par 12 Chval, Jeruzaléme, Hospodina, chval Boha svého, Sione.
\par 13 Nebo on utvrzuje závory bran tvých, požehnání udílí synum tvým u prostred tebe.
\par 14 On pusobí v koncinách tvých pokoj, a belí pšenicnou nasycuje te.
\par 15 On když vysílá na zemi rozkaz svuj, velmi rychle k vykonání beží slovo jeho.
\par 16 Ont dává sníh jako vlnu, jíním jako popelem posýpá.
\par 17 Hází ledem svým jako skyvami; pred zimou jeho kdo ostojí?
\par 18 Vysílaje slovo své, rozpouští je; hned jakž povane vetrem svým, ant tekou vody.
\par 19 Zvestuje slovo své Jákobovi, ustanovení svá a soudy své Izraelovi.
\par 20 Neucinilt tak žádnému národu, a protož soudu jeho nepoznali. Halelujah.

\chapter{148}

\par 1 Halelujah. Chvalte hospodina stvorení nebeská, chvaltež ho na výsostech.
\par 2 Chvalte jej všickni andelé jeho, chvalte jej všickni zástupové jeho.
\par 3 Chvalte jej slunce i mesíc, chvalte jej všecky jasné hvezdy.
\par 4 Chvalte jej nebesa nebes, i vody, kteréž jsou nad nebem tímto.
\par 5 Chvalte jméno Hospodinovo všecky veci, kteréž, jakž on rekl, pojednou stvoreny jsou.
\par 6 A utvrdil je na vecné veky, uložil cíle, z nichž by nevykracovaly.
\par 7 Chvalte Hospodina tvorové zemští, velrybové a všecky propasti,
\par 8 Ohen a krupobití, sníh i pára, vítr bourlivý, vykonávající rozkaz jeho,
\par 9 I hory a všickni pahrbkové, stromoví ovoce nesoucí, i všickni cedrové,
\par 10 Zver divoká i všeliká hovada, zemeplazové i ptactvo létavé,
\par 11 Králové zemští i všickni národové, knížata i všickni soudcové zeme,
\par 12 Mládenci, též i panny, starci s dítkami,
\par 13 Chvalte jméno Hospodinovo; nebo vyvýšeno jest jméno jeho samého, a sláva jeho nade všecku zemi i nebe.
\par 14 A vyzdvihl roh lidu svého, chválu všech svatých jeho, synu Izraelských, lidu s ním spojeného. Halelujah.

\chapter{149}

\par 1 Halelujah. Zpívejte Hospodinu písen novou, chválu jeho v shromáždení svatých.
\par 2 Vesel se Izrael v tom, kterýž ho ucinil, synové Sionští plésejte v králi svém.
\par 3 Chvalte jméno jeho na píštalu, na buben a na citaru prozpevujte jemu.
\par 4 Nebo zalíbilo se Hospodinu v lidu jeho; ont ozdobuje pokorné spasením.
\par 5 Plésati budou svatí v Boží sláve, a zpívati v pokojích svých.
\par 6 Oslavování Boha silného bude ve rtech jejich, a mec na obe strane ostrý v rukou jejich,
\par 7 K vykonávání pomsty nad pohany, a k strestání národu,
\par 8 K svazování králu jejich retezy, a šlechticu jejich pouty železnými,
\par 9 K nakládání s nimi podlé práva zapsaného, k sláve všechnem svatým jeho. Halelujah.

\chapter{150}

\par 1 Halelujah. Chvalte Boha silného pro svatost jeho, chvalte jej pro rozšírení síly jeho.
\par 2 Chvalte jej ze všelijaké moci jeho, chvalte jej podlé veliké dustojnosti jeho.
\par 3 Chvalte jej zvukem trouby, chvalte jej na loutnu a citaru.
\par 4 Chvalte jej na buben a píštalu, chvalte jej na husle a varhany.
\par 5 Chvalte jej na cymbály hlasité, chvalte jej na cymbály zvucné.
\par 6 Všeliký duch chval Hospodina. Halelujah.

\end{document}