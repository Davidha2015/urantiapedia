\begin{document}

\title{Přísloví}

\chapter{1}

\par 1 Prísloví Šalomouna syna Davidova, krále Izraelského,
\par 2 Ku poznání moudrosti a cvicení, k vyrozumívání recem rozumnosti,
\par 3 K dosažení vycvicení v opatrnosti, spravedlnosti, soudu a toho, což pravého jest,
\par 4 Aby dána byla hloupým dumyslnost, mládenecku umení a prozretelnost.
\par 5 Když poslouchati bude moudrý, pribude mu umení, a rozumný bude vtipnejší,
\par 6 K srozumení podobenství, a výmluvnosti reci moudrých a pohádkám jejich.
\par 7 Bázen Hospodinova jest pocátek umení, moudrostí a cvicením pohrdají blázni.
\par 8 Poslouchej, synu muj, cvicení otce svého, a neopouštej naucení matky své.
\par 9 Nebot to pridá príjemnosti hlave tvé, a bude zlatým retezem hrdlu tvému.
\par 10 Synu muj, jestliže by te namlouvali hríšníci, neprivoluj.
\par 11 Jestliže by rekli: Pod s námi, úklady cinme krvi, skryjeme se proti nevinnému bez ostýchání se;
\par 12 Sehltíme je jako hrob za živa, a v cele jako ty, jenž sstupují do jámy;
\par 13 Všelijakého drahého zboží dosáhneme, naplníme domy své loupeží;
\par 14 Vrz los svuj mezi nás, mešec jeden budeme míti všickni:
\par 15 Synu muj, nevycházej na cestu s nimi, zdrž nohu svou od stezky jejich;
\par 16 Nebo nohy jejich ke zlému beží, a pospíchají k vylévání krve.
\par 17 Jiste, že jakož nadarmo roztažena bývá sít pred ocima jakéhokoli ptactva,
\par 18 Tak tito proti krvi své ukládají, skrývají se proti dušem svým.
\par 19 Takovét jsou cesty každého dychtícího po zisku, duši pána svého uchvacuje.
\par 20 Moudrost vne volá, na ulicech vydává hlas svuj.
\par 21 V nejvetším hluku volá, u vrat brány, v meste, a výmluvnosti své vypravuje, rka:
\par 22 Až dokud hloupí milovati budete hloupost, a posmevaci posmech sobe libovati, a blázni nenávideti umení?
\par 23 Obrattež se k domlouvání mému. Hle, vynáším vám ducha svého, a v známost vám uvodím slova svá.
\par 24 Ponevadž jsem volala, a odpírali jste; vztahovala jsem ruku svou, a nebyl, kdo by pozoroval,
\par 25 Anobrž strhli jste se všeliké rady mé, a trestání mého jste neoblíbili:
\par 26 Procež i já v bíde vaší smáti se budu, posmívati se budu, když prijde to, cehož se bojíte,
\par 27 Když prijde jako hrozné zpuštení to, cehož se bojíte, a bída vaše jako boure nastane, když prijde na vás trápení a ssoužení.
\par 28 Tehdy volati budou ke mne, a nevyslyším; ráno hledati mne budou, a nenaleznou mne.
\par 29 Proto že nenávideli umení, a bázne Hospodinovy nevyvolili,
\par 30 Aniž povolili rade mé, ale pohrdali všelikým domlouváním mým.
\par 31 Protož jísti budou ovoce skutku svých, a radami svými nasyceni budou.
\par 32 Nebo pokoj hloupých zmorduje je, a štestí bláznu zahubí je.
\par 33 Ale kdož mne poslouchá, bydliti bude bezpecne, pokoj maje pred strachem zlých vecí.

\chapter{2}

\par 1 Synu muj, prijmeš-li slova má, a prikázaní má schováš-li u sebe;
\par 2 Nastavíš-li moudrosti ucha svého, a nakloníš-li srdce svého k opatrnosti;
\par 3 Ovšem, jestliže na rozumnost zavoláš, a na opatrnost zvoláš-li;
\par 4 Budeš-li jí hledati jako stríbra, a jako pokladu pilne vyhledávati jí:
\par 5 Tehdy porozumíš bázni Hospodinove, a známosti Boží nabudeš;
\par 6 Nebo Hospodin dává moudrost, z úst jeho umení a opatrnost.
\par 7 Chová uprímým dlouhovekosti, pavézou jest chodícím v sprostnosti,
\par 8 Ostríhaje stezek soudu; on cesty svatých svých ostríhá.
\par 9 Tehdy porozumíš spravedlnosti a soudu, a uprímosti i všeliké ceste dobré,
\par 10 Když vejde moudrost v srdce tvé, a umení duši tvé se zalíbí.
\par 11 Prozretelnost ostríhati bude tebe, a opatrnost zachová te,
\par 12 Vysvobozujíc te od cesty zlé, a od lidí mluvících veci prevrácené,
\par 13 Kteríž opouštejí stezky prímé, aby chodili po cestách tmavých,
\par 14 Kteríž se veselí ze zlého cinení, plésají v prevrácenostech nejhorších,
\par 15 Jejichž stezky krivolaké jsou, anobrž zmotaní jsou na cestách svých;
\par 16 Vysvobozujíc te i od ženy postranní, od cizí, kteráž recmi svými lahodí,
\par 17 Kteráž opouští vudce mladosti své, a na smlouvu Boha svého se zapomíná;
\par 18 K smrti se zajisté nachyluje dum její, a k mrtvým stezky její;
\par 19 Kterížkoli vcházejí k ní, nenavracují se zase, aniž trefují na cestu života;
\par 20 Abys chodil po ceste dobrých, a stezek spravedlivých abys ostríhal.
\par 21 Nebo uprímí bydliti budou v zemi, a pobožní zustanou v ní;
\par 22 Bezbožní pak z zeme vytati budou, a prestupníci vykoreneni budou z ní.

\chapter{3}

\par 1 Synu muj, na ucení mé nezapomínej, ale prikázaní mých nechat ostríhá srdce tvé.
\par 2 Dlouhosti zajisté dnu, i let života i pokoje pridadí tobe.
\par 3 Milosrdenství a pravda necht neopouštejí te, privaž je k hrdlu svému, napiš je na tabuli srdce svého,
\par 4 A nalezneš milost a prospech výborný pred Bohem i lidmi.
\par 5 Doufej v Hospodina celým srdcem svým, na rozumnost pak svou nezpoléhej.
\par 6 Na všech cestách svých snažuj se jej poznávati, a ont spravovati bude stezky tvé.
\par 7 Nebývej moudrý sám u sebe; boj se Hospodina, a odstup od zlého.
\par 8 Tot bude zdraví životu tvému, a rozvlažení kostem tvým.
\par 9 Cti Hospodina z statku svého, a z nejprednejších vecí všech úrod svých,
\par 10 A naplneny budou stodoly tvé hojností, a presové tvoji mstem oplývati budou.
\par 11 Kázne Hospodinovy, synu muj, nezamítej, aniž sobe oškliv domlouvání jeho.
\par 12 Nebo kohož miluje Hospodin, tresce, a to jako otec syna, jejž libuje.
\par 13 Blahoslavený clovek nalézající moudrost, a clovek vynášející opatrnost.
\par 14 Lépet jest zajisté težeti jí, nežli težeti stríbrem, anobrž nad výborné zlato užitek její.
\par 15 Dražší jest než drahé kamení, a všecky nejžádostivejší veci tvé nevrovnají se jí.
\par 16 Dlouhost dnu v pravici její, a v levici její bohatství a sláva.
\par 17 Cesty její cesty utešené, a všecky stezky její pokojné.
\par 18 Stromem života jest tem, kteríž jí dosahují, a kteríž ji mají, blahoslavení jsou.
\par 19 Hospodin moudrostí založil zemi, utvrdil nebesa opatrností.
\par 20 Umením jeho propasti protrhují se, a oblakové vydávají rosu.
\par 21 Synu muj, necht neodcházejí ty veci od ocí tvých, ostríhej zdravého naucení a prozretelnosti.
\par 22 I budet to životem duši tvé, a ozdobou hrdlu tvému.
\par 23 Tehdy choditi budeš bezpecne cestou svou, a v nohu svou neurazíš se.
\par 24 Když lehneš, nebudeš se strašiti, ale odpocívati budeš, a bude libý sen tvuj.
\par 25 Nelekneš se strachu náhlého, ani zpuštení bezbožníku, když prijde.
\par 26 Nebo Hospodin bude doufání tvé, a ostríhati bude nohy tvé, abys nebyl lapen.
\par 27 Nezadržuj dobrodiní potrebujícím, když s to býti mužeš, abys je cinil.
\par 28 Neríkej bližnímu svému: Odejdi, potom navrat se, a zítrat dám, maje to u sebe.
\par 29 Neukládej proti bližnímu svému zlého, kterýž s tebou doverne bydlí.
\par 30 Nevad se s clovekem bez príciny, jestližet neucinil zlého.
\par 31 Nechtej závideti muži dráci, aniž zvoluj které cesty jeho.
\par 32 Nebo ohavností jest Hospodinu prevrácenec, ale s uprímými tajemství jeho.
\par 33 Zlorecení Hospodinovo jest v dome bezbožníka, ale príbytku spravedlivých žehná:
\par 34 Ponevadž posmevacum on se posmívá, pokorným pak dává milost.
\par 35 Slávu moudrí dedicne obdrží, ale blázny hubí pohanení.

\chapter{4}

\par 1 Poslouchejte, synové, ucení otcova, a pozorujte, abyste poznali rozumnost.
\par 2 Nebo naucení dobré dávám vám, neopouštejtež zákona mého.
\par 3 Když jsem byl syn u otce svého mladický, a jediný pri matce své,
\par 4 On vyucoval mne a ríkal mi: At se chopí výmluvností mých srdce tvé, ostríhej prikázaní mých, a živ budeš.
\par 5 Nabud moudrosti, nabud rozumnosti; nezapomínej, ani se uchyluj od recí úst mých.
\par 6 Neopouštejž jí, a bude te ostríhati; miluj ji, a zachová te.
\par 7 Predne moudrosti, moudrosti nabývej, a za všecko jmení své zjednej rozumnost.
\par 8 Vyvyšuj ji, a zvýšít te; poctí te, když ji prijmeš.
\par 9 Pridá hlave tvé príjemnosti, korunou krásnou obdarí te.
\par 10 Slyš, synu muj, a prijmi reci mé, a tak rozmnoží se léta života tvého.
\par 11 Ceste moudrosti ucím te, vedu te stezkami prímými.
\par 12 Když choditi budeš, nebude ssoužen krok tvuj, a pobehneš-li, neustrcíš se.
\par 13 Chopiž se ucení, nepouštej, ostríhej ho, nebo ono jest život tvuj.
\par 14 Na stezku bezbožných nevcházej, a nekrácej cestou zlostníku.
\par 15 Opust ji, nechod po ní, uchyl se od ní, a pomin jí.
\par 16 Nebot nespí, lec zlost provedou; anobrž zahánín bývá sen jejich, dokudž ku pádu neprivodí,
\par 17 Proto že jedí chléb bezbožnosti, a víno loupeží pijí.
\par 18 Ale stezka spravedlivých jako svetlo jasné, kteréž rozmáhá se, a svítí až do pravého dne.
\par 19 Cesta pak bezbožných jako mrákota; nevedí, na cem se ustrciti mohou.
\par 20 Synu muj, slov mých pozoruj, k recem mým naklon ucha svého.
\par 21 Nechat neodcházejí od ocí tvých, ostríhej jich u prostred srdce svého.
\par 22 Nebo životem jsou tem, kteríž je nalézají, i všemu telu jejich lékarstvím.
\par 23 Prede vším, cehož se stríci sluší, ostríhej srdce svého, nebo z neho pochází život.
\par 24 Odlož od sebe prevrácenost úst, a zlost rtu vzdal od sebe.
\par 25 Oci tvé at k dobrým vecem patrí, a vícka tvá at príme hledí pred tebou.
\par 26 Zvaž stezku noh svých, a všecky cesty tvé at jsou spraveny.
\par 27 Neuchyluj se na pravo ani na levo, odvrat nohu svou od zlého.

\chapter{5}

\par 1 Synu muj, pozoruj moudrosti mé, k opatrnosti mé naklon ucha svého,
\par 2 Abys ostríhal prozretelnosti, a rtové tvoji šetrili umení.
\par 3 Nebo rtové cizí ženy strdí tekou, a mekcejší nad olej ústa její.
\par 4 Poslední pak veci její horké jsou jako pelynek, ostré jako mec na obe strane ostrý.
\par 5 Nohy její sstupují k smrti, krokové její hrob uchvacují.
\par 6 Stezku života snad bys zvážiti chtel? Vrtkét jsou cesty její, neseznáš.
\par 7 Protož, synové, poslechnete mne, a neodstupujte od recí úst mých.
\par 8 Vzdal od ní cestu svou, a nepribližuj se ke dverím domu jejího,
\par 9 Abys snad nedal jiným slávy své, a let svých ukrutnému,
\par 10 Aby se nenasytili cizí úsilím tvým, a práce tvá nezustala v dome cizím.
\par 11 I rval bys naposledy, když bys zhubil telo své a cerstvost svou,
\par 12 A rekl bys: Jak jsem nenávidel cvicení, a domlouváním pohrdalo srdce mé,
\par 13 A neposlouchal jsem hlasu vyucujících mne, a k ucitelum svým nenaklonil jsem ucha svého!
\par 14 O málo, že jsem nevlezl ve všecko zlé u prostred shromáždení a zástupu.
\par 15 Pí vodu z cisterny své, a prameny z prostredku vrchovište svého.
\par 16 Necht se rozlévají studnice tvé ven, a potuckové vod na ulice.
\par 17 Mej je sám sobe, a ne cizí s tebou.
\par 18 Budiž požehnaný pramen tvuj, a vesel se z manželky mladosti své.
\par 19 Lane milostné a srny utešené; prsy její at te opojují všelikého casu, v milování jejím kochej se ustavicne.
\par 20 Nebo proc bys se kochal, synu muj, v cizí, a objímal život postranní,
\par 21 Ponevadž pred ocima Hospodinovýma jsou cesty cloveka, a on všecky stezky jeho váží?
\par 22 Nepravosti vlastní jímají bezbožníka takového, a v provazích hríchu svého uvázne.
\par 23 Takovýt umre, proto že neprijímal cvicení, a ve množství bláznovství svého blouditi bude.

\chapter{6}

\par 1 Synu muj, slíbil-lis za prítele svého, podal-lis cizímu ruky své,
\par 2 Zapleten jsi slovy úst svých, jat jsi recmi úst svých.
\par 3 Uciniž tedy toto, synu muj, a vyprost se, ponevadžs se dostal v ruku prítele svého. Jdi, pokor se, a probud prítele svého.
\par 4 Nedej usnouti ocím svým, a zdrímati víckám svým.
\par 5 Vydri se jako srna z ruky, a jako pták z ruky cižebníka.
\par 6 Jdi k mravenci, lenochu, shlédni cesty jeho, a nabud moudrosti.
\par 7 Kterýž nemaje vudce, ani správce, ani pána,
\par 8 Pripravuje v léte pokrm svuj, shromažduje ve žni potravu svou.
\par 9 Dokudž lenochu ležeti budeš? Skoro-liž vstaneš ze sna svého?
\par 10 Malicko pospíš, malicko zdrímeš, malicko složíš ruce, abys poležel,
\par 11 V tom prijde jako pocestný chudoba tvá, a nouze tvá jako muž zbrojný.
\par 12 Clovek nešlechetný, muž nepravý chodí v prevrácenosti úst.
\par 13 Mhourá ocima svýma, mluví nohama svýma, ukazuje prsty svými.
\par 14 Prevrácenost všeliká jest v srdci jeho, smýšlí zlé všelikého casu, sváry rozsívá.
\par 15 A protož v náhle prijde bída jeho, rychle setrín bude, a nebudet ulécení.
\par 16 Techto šesti vecí nenávidí Hospodin, a sedmá ohavností jest duši jeho:
\par 17 Ocí vysokých, jazyka lživého, a rukou vylévajících krev nevinnou,
\par 18 Srdce, kteréž ukládá myšlení nepravá, noh kvapných bežeti ke zlému,
\par 19 Svedka lživého, mluvícího lež, a toho, jenž rozsívá ruznice mezi bratrími.
\par 20 Ostríhejž, synu muj, prikázaní otce svého, a neopouštej naucení matky své.
\par 21 Privazuj je k srdci svému ustavicne, a k hrdlu svému je pripínej.
\par 22 Kamžkoli pujdeš, ono te zprovodí, když spáti budeš, bude te ostríhati, a když procítíš, bude s tebou rozmlouvati,
\par 23 (Nebo prikázaní jest svíce, a naucení svetlo, a cesta života jsou domlouvání vyucující),
\par 24 Aby te ostríhalo od ženy zlé, od úlisnosti jazyka ženy cizí.
\par 25 Nežádejž krásy její v srdci svém, a nechat te nejímá vícky svými.
\par 26 Nebo prícinou ženy cizoložné zchudl bys až do kusu chleba, anobrž žena cizoložná drahou duši ulovuje.
\par 27 Muže-liž kdo skrýti ohen v klíne svém, aby roucho jeho se nepropálilo?
\par 28 Muže-liž kdo choditi po uhlí reravém, aby nohy jeho se neopálily?
\par 29 Tak kdož vchází k žene bližního svého, nebudet bez viny, kdož by se jí koli dotkl.
\par 30 Neuvozují potupy na zlodeje, jestliže by ukradl, aby nasytil život svuj, když lacní,
\par 31 Ac postižen jsa, navracuje to sedmernásobne, vším statkem domu svého nahražuje:
\par 32 Ale cizoložící s ženou blázen jest; kdož hubí duši svou, tent to ciní;
\par 33 Trápení a lehkosti dochází, a útržka jeho nebývá shlazena.
\par 34 Nebo zurivý jest hnev muže, a neodpouštít v den pomsty.
\par 35 Neohlídá se na žádnou záplatu, aniž prijímá, by i množství daru dával.

\chapter{7}

\par 1 Synu muj, ostríhej recí mých, a prikázaní má schovej u sebe.
\par 2 Ostríhej prikázaní mých, a živ budeš, a naucení mého jako zrítelnice ocí svých.
\par 3 Privaž je na prsty své, napiš je na tabuli srdce svého.
\par 4 Rci moudrosti: Sestra má jsi ty, a rozumnost prítelkyní jmenuj,
\par 5 Aby te ostríhala od ženy cizí, od postranní, jenž recmi svými lahodí.
\par 6 Nebo z okna domu svého okénkem vyhlédaje,
\par 7 Videl jsem mezi hloupými, spatril jsem mezi mládeží mládence bláznivého.
\par 8 Kterýž šel po ulici vedlé úhlu jejího, a cestou k domu jejímu krácel,
\par 9 V soumrak, u vecer dne, ve tmách nocních a v mrákote.
\par 10 A aj, žena potkala ho v ozdobe nevestcí a chytrého srdce,
\par 11 Štebetná a opovážlivá, v dome jejím nezustávají nohy její,
\par 12 Jednak vne, jednak na ulici u každého úhlu úklady cinící.
\par 13 I chopila jej, a políbila ho, a opovrhši stud, rekla jemu:
\par 14 Obeti pokojné jsou u mne, dnes splnila jsem slib svuj.
\par 15 Protož vyšla jsem vstríc tobe, abych pilne hledala tvári tvé, i nalezla jsem te.
\par 16 Koberci jsem obestrela luže své, s rezbami a prosteradly Egyptskými,
\par 17 Vykadila jsem pokojík svuj mirrou a aloe a skoricí.
\par 18 Pod, opojujme se milostí až do jitra, obveselíme se v milosti.
\par 19 Nebo není muže doma, odšel na cestu dalekou.
\par 20 Pytlík penez vzal s sebou, v jistý den vrátí se do domu svého.
\par 21 I naklonila ho mnohými recmi svými, a lahodností rtu svých prinutila jej.
\par 22 Šel za ní hned, jako vul k zabití chodívá, a jako blázen v pouta, jimiž by trestán byl.
\par 23 Dokudž nepronikla strela jater jeho, pospíchal jako pták k osídlu, neveda, že ono bezživotí jeho jest.
\par 24 Protož nyní, synové, slyšte mne, a pozorujte recí úst mých.
\par 25 Neuchyluj se k cestám jejím srdce tvé, aniž se toulej po stezkách jejích.
\par 26 Nebo mnohé zranivši, porazila, a silní všickni zmordováni jsou od ní.
\par 27 Cesty pekelné dum její, vedoucí do skrýší smrti.

\chapter{8}

\par 1 Zdaliž moudrost nevolá, a rozumnost nevydává hlasu svého?
\par 2 Na vrchu vysokých míst, u cesty, na rozcestí stojí,
\par 3 U bran, kudy se chodí do mesta, a kudy se chodí dvermi, volá, rkuci:
\par 4 Na vást, ó muži, volám, a hlas muj jest k synum lidským.
\par 5 Poucte se hloupí opatrnosti, a blázni srozumejte srdcem.
\par 6 Poslouchejtež, nebo znamenité veci mluviti budu, a otevrení rtu mých pouhou pravdu.
\par 7 Jiste žet pravdu zvestují ústa má, a ohavností jest rtum mým bezbožnost.
\par 8 Spravedlivé jsou všecky reci úst mých, není v nich nic krivého ani prevráceného.
\par 9 Všecky pravé jsou rozumejícímu, a prímé tem, kteríž nalézají umení.
\par 10 Prijmetež cvicení mé radeji než stríbro, a umení radeji než zlato nejvýbornejší.
\par 11 Nebo lepší jest moudrost než drahé kamení, tak že jakékoli veci žádostivé vrovnati se jí nemohou.
\par 12 Já moudrost bydlím s opatrností, a umení pravé prozretelnosti prítomné mám.
\par 13 Bázen Hospodinova jest v nenávisti míti zlé, pýchy a vysokomyslnosti, i cesty zlé a úst prevrácených nenávidím.
\par 14 Má jest rada i štastný prospech, ját jsem rozumnost, a má jest síla.
\par 15 Skrze mne králové kralují, a knížata ustanovují veci spravedlivé.
\par 16 Skrze mne knížata panují, páni i všickni soudcové zemští.
\par 17 Já milující mne miluji, a kteríž mne pilne hledají, nalézají mne.
\par 18 Bohatství a sláva pri mne jest, zboží trvánlivé i spravedlnost.
\par 19 Lepší jest ovoce mé než nejlepší zlato, i než ryzí, a užitek muj než stríbro výborné.
\par 20 Stezkou spravedlnosti vodím, prostredkem stezek soudu,
\par 21 Abych tem, kteríž mne milují, pridedila zboží vecné, a poklady jejich naplnila.
\par 22 Hospodin mel mne pri pocátku cesty své, pred skutky svými, prede všemi casy.
\par 23 Pred veky ustanovena jsem, pred pocátkem, prvé než byla zeme.
\par 24 Když ješte nebylo propasti, zplozena jsem, když ješte nebylo studnic oplývajících vodami.
\par 25 Prvé než hory založeny byly, než byli pahrbkové, zplozena jsem;
\par 26 Ješte byl neucinil zeme a rovin, ani zacátku prachu okršlku zemského.
\par 27 Když pripravoval nebesa, byla jsem tu, když vymeroval okrouhlost nad propastí;
\par 28 Když upevnoval oblaky u výsosti, když utvrzoval studnice propasti;
\par 29 Když ukládal mori cíl jeho, a vodám, aby neprestupovaly rozkázaní jeho, když vymeroval základy zeme:
\par 30 Tehdáž byla jsem od neho pestována, a byla jsem jeho potešení na každý den, anobrž hrám pred ním každého casu;
\par 31 Hrám i na okršlku zeme jeho, a rozkoše mé s syny lidskými.
\par 32 A tak tedy, synové, poslechnete mne, nebo blahoslavení jsou ostríhající cest mých.
\par 33 Poslouchejte cvicení, a nabudte rozumu, a nerozpakujte se.
\par 34 Blahoslavený clovek, kterýž mne slýchá, bde u dverí mých na každý den, šetre verejí dverí mých.
\par 35 Nebo kdož mne nalézá, nalézá život, a dosahuje lásky od Hospodina.
\par 36 Ale kdož hreší proti mne, ukrutenství provodí nad duší svou; všickni, kteríž mne nenávidí, milují smrt.

\chapter{9}

\par 1 Moudrost vystavela dum svuj, vytesavši sloupu svých sedm.
\par 2 Zbila dobytek svuj, smísila víno své, stul také svuj pripravila.
\par 3 A poslavši devecky své, volá na vrchu nejvyšších míst v meste:
\par 4 Kdožkoli jest hloupý, uchyl se sem. Až i bláznivým ríká:
\par 5 Podte, jezte chléb muj, a píte víno, kteréž jsem smísila.
\par 6 Opustte hloupost a živi budte, a chodte cestou rozumnosti.
\par 7 Kdo tresce posmevace, dochází hanby, a kdo primlouvá bezbožnému, pohanení.
\par 8 Nedomlouvej posmevaci, aby te nevzal v nenávist; primlouvej moudrému, a bude te milovati.
\par 9 Ucin to moudrému, a bude moudrejší; pouc spravedlivého, a bude umelejší.
\par 10 Pocátek moudrosti jest bázen Hospodinova, a umení svatých rozumnost.
\par 11 Nebo skrze mne rozmnoží se dnové tvoji, a pridánot bude let života.
\par 12 Budeš-li moudrý, sobe moudrý budeš; pakli posmevac, sám vytrpíš.
\par 13 Žena bláznivá štebetná, nesmyslná, a nic neumí.
\par 14 A sedí u dverí domu svého na stolici, na místech vysokých v meste,
\par 15 Aby volala jdoucích cestou, kteríž prímo jdou stezkami svými, rkuci:
\par 16 Kdo jest hloupý, uchyl se sem. A bláznivému ríká:
\par 17 Voda kradená sladší jest, a chléb pokoutní chutnejší.
\par 18 Ale neví hlupec, že mrtví jsou tam, a v hlubokém hrobe ti, kterýchž pozvala.

\chapter{10}

\par 1 Syn moudrý obveseluje otce, ale syn bláznivý zámutkem jest matce své.
\par 2 Neprospívají pokladové bezbožne nabytí, ale spravedlnost vytrhuje od smrti.
\par 3 Nedopustí lacneti Hospodin duši spravedlivého, statek pak bezbožných rozptýlí.
\par 4 K nouzi privodí ruka lstivá, ruka pak pracovitých zbohacuje.
\par 5 Kdo shromažduje v léte, jest syn rozumný; kdož vyspává ve žni, jest syn, kterýž hanbu ciní.
\par 6 Požehnání jest nad hlavou spravedlivého, ale ústa bezbožných prikrývají ukrutnost.
\par 7 Památka spravedlivého požehnaná, ale jméno bezbožných smrdí.
\par 8 Moudré srdce prijímá prikázaní, ale blázen od rtu svých padne.
\par 9 Kdo chodí upríme, chodí doufanlive; kdož pak prevrací cesty své, vyjeven bude.
\par 10 Kdo mhourá okem, uvodí nesnáz; a kdož jest bláznivých rtu, padne.
\par 11 Pramen života jsou ústa spravedlivého, ale ústa bezbožných prikrývají ukrutnost.
\par 12 Nenávist vzbuzuje sváry, ale láska prikrývá všecka prestoupení.
\par 13 Ve rtech rozumného nalézá se moudrost, ale kyj na hrbete blázna.
\par 14 Moudrí skrývají umení, úst pak blázna blízké jest setrení.
\par 15 Zboží bohatého jest mesto pevné jeho, ale nouze jest chudých setrení.
\par 16 Práce spravedlivého jest k životu, nábytek pak bezbožných jest k hríchu.
\par 17 Stezkou života jde, kdož prijímá trestání; ale kdož pohrdá domlouváním, bloudí.
\par 18 Kdož prikrývá nenávist rty lživými, i kdož uvodí v lehkost, ten blázen jest.
\par 19 Mnohé mluvení nebývá bez hríchu, kdož pak zdržuje rty své, opatrný jest.
\par 20 Stríbro výborné jest jazyk spravedlivého, ale srdce bezbožných za nic nestojí.
\par 21 Rtové spravedlivého pasou mnohé, blázni pak \par bláznovství umírají.
\par 22 Požehnání Hospodinovo zbohacuje, a to beze všeho trápení.
\par 23 Za žert jest bláznu ciniti nešlechetnost, ale muž rozumný moudrosti se drží.
\par 24 Ceho se bojí bezbožný, to prichází na nej; ale cehož žádají spravedliví, dává Buh.
\par 25 Jakož pomíjí vichrice, tak nestane bezbožníka, spravedlivý pak jest základ stálý.
\par 26 Jako ocet zubum, a jako dým ocima, tak jest lenivý tem, kteríž jej posílají.
\par 27 Bázen Hospodinova pridává dnu, léta pak bezbožných ukrácena bývají.
\par 28 Ocekávání spravedlivých jest potešení, nadeje pak bezbožných zahyne.
\par 29 Silou jest uprímému cesta Hospodinova, a strachem tem, kteríž ciní nepravost.
\par 30 Spravedlivý na veky se nepohne, bezbožní pak nebudou bydliti v zemi.
\par 31 Ústa spravedlivého vynášejí moudrost, ale jazyk prevrácený vytat bude.
\par 32 Rtové spravedlivého znají, což jest Bohu libého, ústa pak bezbožných prevrácené veci.

\chapter{11}

\par 1 Váha falešná ohavností jest Hospodinu, ale závaží pravé líbí se jemu.
\par 2 Za pýchou prichází zahanbení, ale pri pokorných jest moudrost.
\par 3 Sprostnost uprímých vodí je, prevrácenost pak prestupníku zatracuje je.
\par 4 Neprospívát bohatství v den hnevu, ale spravedlnost vytrhuje z smrti.
\par 5 Spravedlnost uprímého spravuje cestu jeho, ale \par bezbožnost svou padá bezbožný.
\par 6 Spravedlnost uprímých vytrhuje je, ale prestupníci v zlosti zjímáni bývají.
\par 7 Když umírá clovek bezbožný, hyne nadeje, i ocekávání rekovských cinu mizí.
\par 8 Spravedlivý z úzkosti bývá vysvobozen, bezbožný pak prichází na místo jeho.
\par 9 Pokrytec ústy kazí bližního svého, ale spravedliví umením vytrženi bývají.
\par 10 Z štestí spravedlivých veselí se mesto, když pak hynou bezbožní, bývá prozpevování.
\par 11 Požehnáním spravedlivých zvýšeno bývá mesto, ústy pak bezbožných vyvráceno.
\par 12 Pohrdá bližním svým blázen, ale muž rozumný mlcí.
\par 13 Utrhac toulaje se, pronáší tajnost, verný pak clovek tají vec.
\par 14 Kdež není dostatecné rady, padá lid, ale spomožení jest ve množství rádcu.
\par 15 Velmi sobe škodí, kdož slibuje za cizího, ješto ten, kdož nenávidí rukojemství, bezpecen jest.
\par 16 Žena šlechetná má cest, a ukrutní mají zboží.
\par 17 Clovek úcinný dobre ciní životu svému, ale ukrutný kormoutí telo své.
\par 18 Bezbožný delá dílo falešné, ale kdož rozsívá spravedlnost, má mzdu jistou.
\par 19 Tak spravedlivý rozsívá k životu, a kdož následuje zlého, k smrti své.
\par 20 Ohavností jsou Hospodinu prevrácení srdcem, ale ctného obcování líbí se jemu.
\par 21 Zlý, by sobe i na pomoc privzal, neujde pomsty, síme pak spravedlivých uchází toho.
\par 22 Zápona zlatá na pysku svine jest žena pekná bez rozumu.
\par 23 Žádost spravedlivých jest toliko dobrých vecí, ale ocekávání bezbožných hnev.
\par 24 Mnohý rozdává štedre, a však pribývá mu více; jiný skoupe drží nad slušnost, ale k chudobe.
\par 25 Clovek štedrý bývá bohatší, a kdož svlažuje, také sám bude zavlažen.
\par 26 Kdo zadržuje obilí, zlorecí mu lid; ale požehnání na hlave toho, kdož je prodává.
\par 27 Kdo pilne hledá dobrého, nalézá prízen; kdož pak hledá zlého, potká jej.
\par 28 Kdo doufá v bohatství své, ten spadne, ale spravedliví jako ratolest zkvetnou.
\par 29 Kdo kormoutí dum svuj, za dedictví bude míti vítr, a blázen sloužiti musí moudrému.
\par 30 Ovoce spravedlivého jest strom života, a kdož vyucuje duše, jest moudrý.
\par 31 Aj, spravedlivému na zemi odplacováno bývá, cím více bezbožnému a hríšníku?

\chapter{12}

\par 1 Kdo miluje cvicení, miluje umení; kdož pak nenávidí domlouvání, nemoudrý jest.
\par 2 Dobrý nalézá lásku u Hospodina, ale muže nešlechetného potupí Buh.
\par 3 Nebývá trvánlivý clovek v bezbožnosti, koren pak spravedlivých nepohne se.
\par 4 Žena statecná jest koruna muže svého, ale jako hnis v kostech jeho ta, kteráž k hanbe privodí.
\par 5 Myšlení spravedlivých jsou pravá, rady pak bezbožných lstivé.
\par 6 Slova bezbožných úklady ciní krvi, ústa pak spravedlivých vytrhují je.
\par 7 Vyvráceni bývají bezbožní tak, aby jich nebylo, ale dum spravedlivých ostojí.
\par 8 Podlé toho, jakž rozumný jest, chválen bývá muž, prevráceného pak srdce bude v pohrdání.
\par 9 Lepší jest nevzácný, kterýž má služebníka, nežli ten, kterýž sobe slavne pocíná, a nemá chleba.
\par 10 Pecuje spravedlivý o život hovádka svého, srdce pak bezbožných ukrutné jest.
\par 11 Kdo delá zemi svou, nasycen bývá chlebem; ale kdož následuje zahalecu, blázen jest.
\par 12 Žádostiv jest bezbožný obrany proti zlému, ale koren spravedlivých zpusobuje ji.
\par 13 Do prestoupení rtu zapletá se zlostník, ale spravedlivý vychází z ssoužení.
\par 14 Z ovoce úst každý nasycen bude dobrým, a odplatu za skutky cloveka dá jemu Buh.
\par 15 Cesta blázna prímá se zdá jemu, ale kdo poslouchá rady, moudrý jest.
\par 16 Hnev blázna v tentýž den poznán bývá, ale opatrný hanbu skrývá.
\par 17 Kdož mluví pravdu, ohlašuje spravedlnost, svedek pak falešný lest.
\par 18 Nekdo vynáší reci podobné meci probodujícímu, ale jazyk moudrých jest lékarství.
\par 19 Rtové pravdomluvní utvrzeni budou na veky, ale na kraticko jazyk lživý.
\par 20 V srdci tech, kteríž zlé obmýšlejí, bývá lest, v tech pak, kteríž radí ku pokoji, veselí.
\par 21 Nepotká spravedlivého žádná težkost, bezbožní pak naplneni budou zlým.
\par 22 Ohavností jsou Hospodinu rtové lživí, ale ti, jenž ciní pravdu, líbí se jemu.
\par 23 Clovek opatrný tají umení, ale srdce bláznu vyvolává bláznovství.
\par 24 Ruka pracovitých panovati bude, lstivá pak musí dávati plat.
\par 25 Starost v srdci cloveka snižuje ji, ale vec dobrá obveseluje ji.
\par 26 Vzácnejší jest nad bližního svého spravedlivý, cesta pak bezbožných svodí je.
\par 27 Nebude péci fortelný, což ulovil, ale clovek bedlivý statku drahého nabude.
\par 28 Na stezce spravedlnosti jest život, a cesta stezky její nesmrtelná jest.

\chapter{13}

\par 1 Syn moudrý prijímá cvicení otcovo, ale posmevac neposlouchá domlouvání.
\par 2 Z ovoce úst každý jísti bude dobré, ale duše prevrácených nátisky.
\par 3 Kdo ostríhá úst svých, ostríhá duše své; kdo rozdírá rty své, setrení na nej prijde.
\par 4 Žádá, a nic nemá duše lenivého, duše pak pracovitých zbohatne.
\par 5 Slova lživého nenávidí spravedlivý, bezbožníka pak v ošklivost uvodí a zahanbuje.
\par 6 Spravedlnost ostríhá príme chodícího po ceste, bezbožnost pak vyvrací hríšníka.
\par 7 Nekdo bohatým se delaje, nemá nic: zase nekdo delaje se chudým, má však statku mnoho.
\par 8 Výplata života cloveku jest bohatství jeho, ale chudý neslyší domlouvání.
\par 9 Svetlo spravedlivých rozsvetluje se, svíce pak bezbožných zhasne.
\par 10 Samou toliko pýchou pusobí clovek svár, ale pri tech, jenž užívají rady, jest moudrost.
\par 11 Statek zle dobytý umenšovati se bude, kdož pak shromažduje rukou, privetší ho.
\par 12 Ocekávání dlouhé zemdlívá srdce, ale žádost splnená jest strom života.
\par 13 Kdož pohrdá slovem Božím, sám sobe škodí; ale kdož se bojí prikázaní, odplaceno mu bude.
\par 14 Naucení moudrého jest pramen života, k vyhýbání se osídlum smrti.
\par 15 Rozum dobrý dává milost, cesta pak prevrácených jest tvrdá.
\par 16 Každý dumyslný delá umele, ale blázen rozprostírá bláznovství.
\par 17 Posel bezbožný upadá v neštestí, jednatel pak verný jest lékarství.
\par 18 Chudoba a lehkost potká toho, jenž se vytahuje z kázne; ale kdož ostríhá naucení, zveleben bude.
\par 19 Žádost naplnená sladká jest duši, ale ohavnost jest bláznum odstoupiti od zlého.
\par 20 Kdo chodí s moudrými, bude moudrý; ale kdo tovaryší s blázny, setrín bude.
\par 21 Hríšníky stihá neštestí, ale spravedlivým odplatí Buh dobrým.
\par 22 Dobrý zanechává dedictví vnukum, ale zboží hríšného zachováno bývá spravedlivému.
\par 23 Hojnost jest pokrmu na rolí chudých, nekdo pak hyne skrze nerozšafnost.
\par 24 Kdo zdržuje metlu svou, nenávidí syna svého; ale kdož ho miluje, za casu jej tresce.
\par 25 Spravedlivý jí až do nasycení duše své, bricho pak bezbožných nedostatek trpí.

\chapter{14}

\par 1 Moudrá žena vzdelává dum svuj, bláznice pak rukama svýma borí jej.
\par 2 Kdo chodí v uprímnosti své, bojí se Hospodina, ale prevrácený v cestách svých pohrdá jím.
\par 3 V ústech blázna jest hul pýchy, rtové pak moudrých ostríhají jich.
\par 4 Když není volu, prázdné jsou jesle, ale hojná úroda jest v síle volu.
\par 5 Svedek verný neklamá, ale svedek falešný mluví lež.
\par 6 Hledá posmevac moudrosti, a nenalézá, rozumnému pak umení snadné jest.
\par 7 Odejdi od muže bláznivého, když neseznáš pri nem rtu umení.
\par 8 Moudrost opatrného jest, aby rozumel ceste své, bláznovství pak bláznu ke lsti.
\par 9 Blázen prikrývá hrích, ale mezi uprímými dobrá vule.
\par 10 Srdce ví o horkosti duše své, a k veselí jeho neprimísí se cizí.
\par 11 Dum bezbožných vyhlazen bude, ale stánek uprímých zkvetne.
\par 12 Cesta zdá se prímá cloveku, a však dokonání její jest cesta k smrti.
\par 13 Také i v smíchu bolí srdce, a cíl veselí jest zámutek.
\par 14 Cestami svými nasytí se prevrácený srdcem, ale muž dobrý štítí se jeho.
\par 15 Hloupý verí každému slovu, ale opatrný šetrí kroku svého.
\par 16 Moudrý bojí se a odstupuje od zlého, ale blázen dotre a smelý jest.
\par 17 Náhlý se dopouští bláznovství, a muž myšlení zlých v nenávisti bývá.
\par 18 Dedicne vládnou hlupci bláznovstvím, ale opatrní bývají korunováni umením.
\par 19 Sklánejí se zlí pred dobrými, a bezbožní u bran spravedlivého.
\par 20 Také i príteli svému v nenávisti bývá chudý, ale milovníci bohatého mnozí jsou.
\par 21 Pohrdá bližním svým hríšník, ale kdož se slitovává nad chudými, blahoslavený jest.
\par 22 Zajisté žet bloudí, kteríž ukládají zlé; ale milosrdenství a pravda tem, kteríž smýšlejí dobré.
\par 23 Všeliké práce bývá zisk, ale slovo rtu jest jen k nouzi.
\par 24 Koruna moudrých jest bohatství jejich, bláznovství pak bláznivých bláznovstvím.
\par 25 Vysvobozuje duše svedek pravdomluvný, ale lstivý mluví lež.
\par 26 V bázni Hospodinove jestit doufání silné, kterýž synum svým útocištem bude.
\par 27 Bázen Hospodinova jest pramen života, k vyhýbání se osídlum smrti.
\par 28 Ve množství lidu jest sláva krále, ale v nedostatku lidu zahynutí vudce.
\par 29 Zpozdilý k hnevu hojne má rozumu,ale náhlý pronáší bláznovství.
\par 30 Život tela jest srdce zdravé, ale hnis v kostech jest závist.
\par 31 Kdo utiská chudého, útržku ciní Uciniteli jeho; ale ctí jej, kdož se slitovává nad chudým.
\par 32 \par zlost svou odstrcen bývá bezbožný, ale nadeji má i pri smrti své spravedlivý.
\par 33 V srdci rozumného odpocívá moudrost, co pak jest u vnitrnosti bláznu, nezatají se.
\par 34 Spravedlnost zvyšuje národ, ale hrích jest ku pohanení národum.
\par 35 Laskav bývá král na služebníka rozumného, ale hneviv na toho, kterýž hanbu ciní.

\chapter{15}

\par 1 Odpoved mekká odvracuje hnev, ale rec zpurná vzbuzuje prchlivost.
\par 2 Jazyk moudrých ozdobuje umení, ale ústa bláznu vylévají bláznovství.
\par 3 Na všelikém míste oci Hospodinovy spatrují zlé i dobré.
\par 4 Zdravý jazyk jest strom života, prevrácenost pak z neho ztroskotání od vetru.
\par 5 Blázen pohrdá cvicením otce svého, ale kdož ostríhá naucení, opatrnosti nabude.
\par 6 V dome spravedlivého jest hojnost veliká, ale v úrode bezbožného zmatek.
\par 7 Rtové moudrých rozsívají umení, srdce pak bláznu ne tak.
\par 8 Obet bezbožných ohavností jest Hospodinu, ale modlitba uprímých líbí se jemu.
\par 9 Ohavností jest Hospodinu cesta bezbožného, toho pak, kdož následuje spravedlnosti, miluje.
\par 10 Trestání prísné opouštejícímu cestu, a kdož nenávidí domlouvání, umre.
\par 11 Peklo i zatracení jest pred Hospodinem, cím více srdce synu lidských?
\par 12 Nemiluje posmevac toho, kterýž ho tresce, aniž k moudrým pristoupí.
\par 13 Srdce veselé obveseluje tvár, ale \par žalost srdce duch zkormoucen bývá.
\par 14 Srdce rozumného hledá umení, ale ústa bláznu pasou se bláznovstvím.
\par 15 Všickni dnové chudého zlí jsou, ale dobromyslného hody ustavicné.
\par 16 Lepší jest malicko s bázní Hospodinovou než poklad veliký s nepokojem.
\par 17 Lepší jest krme z zelí, kdež jest láska, nežli z krmného vola, kdež jest nenávist.
\par 18 Muž hnevivý vzbuzuje sváry, ale zpozdilý k hnevu upokojuje svadu.
\par 19 Cesta lenivého jest jako plot z trní, ale stezka uprímých jest vydlážená.
\par 20 Syn moudrý obveseluje otce, bláznivý pak clovek pohrdá matkou svou.
\par 21 Bláznovství jest veselím bláznu, ale clovek rozumný uprímo kráceti smeruje.
\par 22 Kdež není rady, zmarena bývají usilování, ale množství rádcu ostojí.
\par 23 Vesel bývá clovek z odpovedi úst svých; nebo slovo v cas príhodný ó jak jest dobré!
\par 24 Cesta života vysoko jest rozumnému proto, aby se uchýlil od pekla dole.
\par 25 Dum pyšných vyvrací Hospodin, meze pak vdovy upevnuje.
\par 26 Ohavností jsou Hospodinu myšlení zlého, ale cistých reci vzácné.
\par 27 Kdož dychtí po lakomství, kormoutí dum svuj; ale kdož nenávidí daru, živ bude.
\par 28 Srdce spravedlivého premyšluje, co má mluviti, ale ústa bezbožných vylévají všelijakou zlost.
\par 29 Vzdálen jest Hospodin od bezbožných, ale modlitbu spravedlivých vyslýchá.
\par 30 To, což se zraku naskýtá, obveseluje srdce; povest dobrá tukem naplnuje kosti.
\par 31 Ucho, kteréž poslouchá trestání života, u prostred moudrých bydliti bude.
\par 32 Kdo se vyhýbá cvicení, zanedbává duše své; ale kdož prijímá domlouvání, má rozum.
\par 33 Bázen Hospodinova jest cvicení se moudrosti, a slávu predchází ponížení.

\chapter{16}

\par 1 Pri cloveku bývá sporádání myšlení, ale od Hospodina jest rec jazyka.
\par 2 Všecky cesty cloveka cisté se jemu zdají, ale kterýž zpytuje duchy, Hospodin jest.
\par 3 Uval na Hospodina ciny své, a budou upevnena predsevzetí tvá.
\par 4 Hospodin všecko ucinil \par sebe samého, také i bezbožného ke dni zlému.
\par 5 Ohavností jest Hospodinu každý pyšného srdce; by sobe na pomoc i jiné privzal, neujde pomsty.
\par 6 Milosrdenstvím a pravdou ocištena bývá nepravost, a v bázni Hospodinove uchází se zlého.
\par 7 Když se líbí Hospodinu cesty cloveka, také i neprátely jeho spokojuje k nemu.
\par 8 Lepší jest malicko s spravedlností, než množství duchodu nespravedlivých.
\par 9 Srdce cloveka premýšlí o ceste své, ale Hospodin spravuje kroky jeho.
\par 10 Rozhodnutí jest ve rtech královských, v soudu neuchylují se ústa jeho.
\par 11 Váha a závaží jsou úsudek Hospodinuv, a všecka závaží v pytlíku jeho narízení.
\par 12 Ohavností jest králum ciniti bezbožne; nebo spravedlností upevnován bývá trun.
\par 13 Rtové spravedliví líbezní jsou králum, a ty, kteríž upríme mluví, milují.
\par 14 Rozhnevání královo jistý posel smrti, ale muž moudrý ukrotí je.
\par 15 V jasné tvári královské jest život, a prívetivost jeho jako oblak s deštem jarním.
\par 16 Mnohem lépe jest nabyti moudrosti než zlata nejcistšího, a nabyti rozumnosti lépe než stríbra.
\par 17 Cesta uprímých jest odstoupiti od zlého; ostríhá duše své ten, kdož ostríhá cesty své.
\par 18 Pred setrením bývá pýcha, a pred pádem pozdvižení ducha.
\par 19 Lépe jest poníženého duchu býti s pokornými, než deliti korist s pyšnými.
\par 20 Ten, kdož pozoruje slova, nalézá dobré; a kdož doufá v Hospodina, blahoslavený jest.
\par 21 Ten, kdož jest moudrého srdce, slove rozumný, a sladkost rtu pridává naucení.
\par 22 Rozumnost tem, kdož ji mají, jest pramen života, ale umení bláznu jest bláznovství.
\par 23 Srdce moudrého rozumne spravuje ústa svá, tak že rty svými pridává naucení.
\par 24 Plást medu jsou reci utešené, sladkost duši, a lékarství kostem.
\par 25 Cesta zdá se prímá cloveku, ale dokonání její jistá cesta smrti.
\par 26 Clovek pracovitý pracuje sobe, nebo ponoukají ho ústa jeho.
\par 27 Muž nešlechetný vykopává zlé, v jehožto rtech jako ohen spalující.
\par 28 Muž prevrácený rozsívá sváry, a klevetník rozlucuje prátely.
\par 29 Muž ukrutný preluzuje bližního svého, a uvodí jej na cestu nedobrou.
\par 30 Zamhuruje oci své, smýšleje veci prevrácené, a zmítaje pysky svými, vykonává zlé.
\par 31 Koruna ozdobná jsou šediny na ceste spravedlnosti se nalézající.
\par 32 Lepší jest zpozdilý k hnevu než silný rek, a kdož panuje nad myslí svou nežli ten, kterýž dobyl mesta.
\par 33 Do klínu umítán bývá los, ale od Hospodina všecko rízení jeho.

\chapter{17}

\par 1 Lepší jest kus chleba suchého s pokojem,nežli dum plný nabitých hovad s svárem.
\par 2 Služebník rozumný panovati bude nad synem, kterýž jest k hanbe, a mezi bratrími deliti bude dedictví.
\par 3 Teglík stríbra a pec zlata zkušuje, ale srdcí Hospodin.
\par 4 Zlý clovek pozoruje recí nepravých, a lhár poslouchá jazyka prevráceného.
\par 5 Kdo se posmívá chudému, útržku ciní Uciniteli jeho; a kdo se z bídy raduje, nebude bez pomsty.
\par 6 Koruna starcu jsou vnukové, a ozdoba synu otcové jejich.
\par 7 Nesluší na blázna reci znamenité, ovšem na kníže rec lživá.
\par 8 Jako kámen drahý, tak bývá vzácný dar pred ocima toho, kdož jej bére; k cemukoli smeruje, darí se jemu.
\par 9 Kdo prikrývá prestoupení, hledá lásky; ale kdo obnovuje vec, rozlucuje prátely.
\par 10 Více se chápá rozumného jedno domluvení, nežli by blázna stokrát ubil.
\par 11 Zpurný toliko zlého hledá, procež prísný posel na nej poslán bývá.
\par 12 Lépe cloveku potkati se s nedvedicí osiralou, nežli s bláznem v bláznovství jeho.
\par 13 Kdo odplacuje zlým za dobré, neodejdet zlé z domu jeho.
\par 14 Zacátek svady jest, jako když kdo protrhuje vodu; protož prvé než by se zsilil svár, prestan.
\par 15 Kdož ospravedlnuje nepravého, i kdož odsuzuje spravedlivého, ohavností jsou Hospodinu oba jednostejne.
\par 16 K cemu jest zboží v ruce blázna, když k nabytí moudrosti rozumu nemá?
\par 17 Všelikého casu miluje, kdož jest prítelem, a bratr v ssoužení ukáže se.
\par 18 Clovek bláznivý ruku dávaje, ciní slib pred prítelem svým.
\par 19 Kdož miluje svadu, miluje hrích; a kdo vyvyšuje ústa svá, hledá potrení.
\par 20 Prevrácené srdce nenalézá toho, což jest dobrého; a kdož má vrtký jazyk, upadá v težkost.
\par 21 Kdo zplodil blázna, k zámutku svému zplodil jej, aniž se bude radovati otec nemoudrého.
\par 22 Srdce veselé ocerstvuje jako lékarství, ale duch zkormoucený vysušuje kosti.
\par 23 Bezbožný tajne bére dar, aby prevrátil stezky soudu.
\par 24 Na oblíceji rozumného vidí se moudrost, ale oci blázna tekají až na konec zeme.
\par 25 K žalosti jest otci svému syn blázen, a k horkosti rodicce své.
\par 26 Jiste že pokutovati spravedlivého není dobré, tolikéž, aby knížata bíti meli \par uprímost.
\par 27 Zdržuje reci své muž umelý; drahého ducha jest muž rozumný.
\par 28 Také i blázen, mlce, za moudrého jmín bývá, a zacpávaje rty své, za rozumného.

\chapter{18}

\par 1 Svémyslný hledá toho, což se jemu líbí, a ve všelijakou vec plete se.
\par 2 Nezalibuje sobe blázen v rozumnosti, ale v tom, což zjevuje srdce jeho.
\par 3 Když prijde bezbožný, prichází také pohrdání, a s lehkomyslným útržka.
\par 4 Slova úst muže vody hluboké, potok rozvodnilý pramen moudrosti.
\par 5 Prijímati osobu bezbožného není dobré, abys prevrátil spravedlivého v soudu.
\par 6 Rtové blázna smerují k svade, a ústa jeho bití se domluví.
\par 7 Ústa blázna k setrení jemu, a rtové jeho osídlem duši jeho.
\par 8 Slova utrhace jsou jako ubitých, ale však sstupují do vnitrností života.
\par 9 Také ten, kdož jest nedbalý v práci své, bratr jest mrhace.
\par 10 Veže pevná jest jméno Hospodinovo; k nemu se utece spravedlivý, a bude povýšen.
\par 11 Zboží bohatého jest mesto pevné jeho, a jako zed vysoká v mysli jeho.
\par 12 Pred setrením vyvyšuje se srdce cloveka, ale pred povýšením bývá ponížení.
\par 13 Kdož odpovídá neco, prvé než vyslyší, pocítá se to za bláznovství jemu a za lehkost.
\par 14 Duch muže snáší nemoc svou, ducha pak zkormouceného kdo snese?
\par 15 Srdce rozumného dosahuje umení, a ucho moudrých hledá umení.
\par 16 Dar cloveka uprostrannuje jemu, a pred oblícej mocných privodí jej.
\par 17 Spravedlivý zdá se ten, kdož jest první v své pri, ale když prichází bližní jeho, tedy stihá jej.
\par 18 Los pokojí svady, a mezi silnými rozeznává.
\par 19 Bratr krivdou uražený tvrdší jest než mesto nedobyté, a svárové jsou jako závora u hradu.
\par 20 Ovocem úst jednoho každého nasyceno bývá bricho jeho, úrodou rtu svých nasycen bude.
\par 21 Smrt i život jest v moci jazyka, a ten, kdož jej miluje, bude jísti ovoce jeho.
\par 22 Kdo nalezl manželku, nalezl vec dobrou, a navážil lásky od Hospodina.
\par 23 Ponížene mluví chudý, ale bohatý odpovídá tvrde.
\par 24 Ten, kdož má prátely, má se míti prátelsky, ponevadž prítel bývá vlastnejší než bratr.

\chapter{19}

\par 1 Lepší jest chudý, jenž chodí v uprímnosti své, nežli prevrácený ve rtech svých, kterýž jest blázen.
\par 2 Jiste že bez umení duši není dobre, a kdož jest kvapných noh, hreší.
\par 3 Bláznovství cloveka prevrací cestu jeho, ackoli proti Hospodinu zpouzí se srdce jeho.
\par 4 Statek pridává prátel množství, ale chudý od prítele svého odloucen bývá.
\par 5 Svedek falešný nebude bez pomsty, a kdož mluví lež, neutece.
\par 6 Mnozí pokorí se pred knížetem, a každý jest prítel muži štedrému.
\par 7 Všickni bratrí chudého v nenávisti jej mají; cím více prátelé jeho vzdalují se od neho! Když volá za nimi, není jich.
\par 8 Ten, kdož miluje duši svou, nabývá moudrosti, a ostríhá opatrnosti, aby nalezl dobré.
\par 9 Svedek falešný nebude bez pomsty, a kdož mluví lež, zahyne.
\par 10 Nesluší na blázna rozkoš, a ovšem, aby služebník nad knížaty panoval.
\par 11 Rozum cloveka zdržuje hnev jeho, a cest jeho jest prominouti provinení.
\par 12 Prchlivost královská jako rvání mladého lva jest, a ochotnost jeho jako rosa na bylinu.
\par 13 Trápení otci svému jest syn bláznivý, a ustavicné kapání žena svárlivá.
\par 14 Dum a statek jest po rodicích, ale od Hospodina manželka rozumná.
\par 15 Lenost privodí tvrdý sen, a duše váhavá lacneti bude.
\par 16 Ten, kdož ostríhá prikázaní, ostríhá duše své; ale kdož pohrdá cestami svými, zahyne.
\par 17 Kdo udeluje chudému, pujcuje Hospodinu, a ont za dobrodiní jeho odplatí jemu.
\par 18 Tresci syna svého, dokudž jest o nem nadeje, a k zahynutí jeho neodpouštej jemu duše tvá.
\par 19 Veliký hnev ukazuj, odpoušteje trestání, proto že ponevadž odpouštíš, potom více trestati budeš.
\par 20 Poslouchej rady, a prijímej kázen, abys vždy nekdy moudrý byl.
\par 21 Mnozí úmyslové jsou v srdci cloveka, ale uložení Hospodinovo tot ostojí.
\par 22 Žádaná vec cloveku jest dobre ciniti jiným, ale pocestnejší jest chudý než muž lživý.
\par 23 Bázen Hospodinova k životu. Takový jsa nasycen, bydlí, aniž neštestím navštíven bývá.
\par 24 Lenivý schovává ruku svou za nadra, ani k ústum svým jí nevztáhne.
\par 25 Ubí posmevace, at se hlupec dovtípí; a potresci rozumného, at porozumí umení.
\par 26 Syn, kterýž hanbu a lehkost ciní, hubí otce, a zahání matku.
\par 27 Prestan, synu muj, poslouchati ucení, kteréž od recí rozumných odvozuje.
\par 28 Svedek nešlechetný posmívá se soudu, a ústa bezbožných prikrývají nepravost.
\par 29 Nebo na posmevace hotoví jsou nálezové, a rány na hrbet bláznu.

\chapter{20}

\par 1 Víno ciní posmevace, a nápoj opojný nepokojného; procež každý, kdož se kochá v nem, nebývá moudrý.
\par 2 Hruza královská jako rvání mladého lva; kdož ho rozhnevá, hreší proti životu svému.
\par 3 Prestati od sváru jest to každému ku poctivosti, ale kdožkoli se do nich zapletá, blázen jest.
\par 4 Lenoch neore \par zimu, procež žebrati bude ve žni, ale nadarmo.
\par 5 Rada v srdci muže voda hluboká, muž však rozumný dosáhne jí.
\par 6 Vetší díl lidí honosí se úcinností svou, ale v pravde takového kdo nalezne?
\par 7 Spravedlivý ustavicne chodí v uprímnosti své; blažení synové jeho po nem.
\par 8 Král sede na soudné stolici, rozhání ocima svýma všecko zlé.
\par 9 Kdo muže ríci: Ocistil jsem srdce své? Cist jsem od hríchu svého?
\par 10 Závaží rozdílná a míra rozdílná, obé to ohavností jest Hospodinu.
\par 11 Po skutcích svých poznáno bývá také i pachole, jest-li uprímé a pravé dílo jeho.
\par 12 Ucho, kteréž slyší, a oko, kteréž vidí, obé to ucinil Hospodin.
\par 13 Nemiluj snu, abys nezchudl, otevri oci své, a nasytíš se chlebem.
\par 14 Zlé, zlé, ríká ten, kdož kupuje, a odejda, tedy se chlubí.
\par 15 Zlato a množství perel, a nejdražší klínot jsou rtové umelí.
\par 16 Vezmi roucho toho, kterýž slíbil za cizího, a kdo za cizozemku, základ jeho.
\par 17 Chutný jest nekomu chléb falše, ale potom ústa jeho pískem naplnena bývají.
\par 18 Myšlení radou upevnuj, a s opatrnou radou ved boj.
\par 19 Kdo vynáší tajnost, chodí neupríme, procež k lahodícímu rty svými neprimešuj se.
\par 20 Kdo zlorecí otci svému neb matce své, zhasne svíce jeho v temných mrákotách.
\par 21 Dedictví rychle z pocátku nabytému naposledy nebývá dobroreceno,
\par 22 Neríkej: Odplatím se zlým; ocekávej na Hospodina, a vysvobodí te.
\par 23 Ohavností jsou Hospodinu závaží rozdílná, a váhy falešné neoblibuje.
\par 24 Od Hospodina jsou krokové muže, ale clovek jak vyrozumívá ceste jeho?
\par 25 Osídlo jest cloveku pohltiti vec posvecenou, a po slibu zase toho vyhledávati.
\par 26 Král moudrý rozptyluje bezbožné, a uvodí na ne pomstu.
\par 27 Duše cloveka jest svíce Hospodinova, kteráž zpytuje všecky vnitrnosti srdecné.
\par 28 Milosrdenství a pravda ostríhají krále, a milosrdenstvím podpírá se trun jeho.
\par 29 Ozdoba mládencu jest síla jejich, a okrasa starcu šediny.
\par 30 Modriny ran jsou lékarství pri zlém, a bití vnitrnostem života.

\chapter{21}

\par 1 Jako potuckové vod jest srdce královo v ruce Hospodinove; kamžkoli chce, naklonuje ho.
\par 2 Všeliká cesta cloveka prímá se zdá jemu, ale kterýž zpytuje srdce, Hospodin jest.
\par 3 Vykonávati spravedlnost a soud více se líbí Hospodinu nežli obet.
\par 4 Vysokost ocí, širokost srdce, a orání bezbožných jest hríchem.
\par 5 Myšlení bedlivého všelijak ku prospechu pricházejí, ale každého toho, kdož kvapný jest, toliko k nouzi.
\par 6 Pokladové jazykem lživým shromáždení jsou marnost pomíjející hledajících smrti.
\par 7 Zhouba, kterouž ciní bezbožníci, bydliti bude u nich; nebo se zpecují ciniti soudu.
\par 8 Muž, jehož cesta prevrácená jest, cizí jest, cistého pak dílo prímé jest.
\par 9 Lépe jest bydliti v koute na streše,nežli s ženou svárlivou v dome spolecném.
\par 10 Duše bezbožného žádá zlého, ani prítel jeho jemu príjemný nebývá.
\par 11 Posmevac když bývá trestán, hloupý bývá moudrejší; a když se umele nakládá s moudrým, prijímá umení.
\par 12 Vyucuje Buh spravedlivého na dome bezbožného, kterýž vyvrací bezbožné \par zlost.
\par 13 Kdo zacpává ucho své k volání chudého, i on sám volati bude, a nebude vyslyšán.
\par 14 Dar skrytý ukrocuje prchlivost, a pocta v klíne hnev prudký.
\par 15 Radostí jest spravedlivému ciniti soud, ale hruzou cinitelum nepravosti.
\par 16 Clovek bloudící z cesty rozumnosti v shromáždení mrtvých odpocívati bude.
\par 17 Muž milující veselost nuzníkem bývá, a kdož miluje víno a masti, nezbohatne.
\par 18 Výplatou za spravedlivého bude bezbožný, a za uprímé ošemetný.
\par 19 Lépe jest bydliti v zemi pusté než s ženou svárlivou a zlostnou.
\par 20 Poklad žádostivý a olej jest v príbytku moudrého, bláznivý pak clovek zžírá jej.
\par 21 Kdo snažne následuje spravedlnosti a milosrdenství, nalézá život, spravedlnost i slávu.
\par 22 Do mesta silných vchází moudrý, a borí pevnost doufání jeho.
\par 23 Kdo ostríhá úst svých a jazyka svého, ostríhá od úzkosti duše své.
\par 24 Hrdého a pyšného jméno jest posmevac, kterýž vše s neochotností a pýchou delá.
\par 25 Žádost lenivého zabijí jej, nebo nechtí ruce jeho delati.
\par 26 Každého dne žádostí horí, spravedlivý pak dává a neskoupí se.
\par 27 Obet bezbožných ohavností jest, ovšem pak jestliže by ji s nešlechetností obetovali.
\par 28 Svedek lživý zahyne, ale muž, kterýž co slyší, stále mluviti bude.
\par 29 Muž bezbožný zatvrzuje tvár svou, uprímý pak merí cestu svou.
\par 30 Není žádné moudrosti, ani opatrnosti, ani rady proti Hospodinu.
\par 31 Kun strojen bývá ke dni boje, ale Hospodinovo jest vysvobození.

\chapter{22}

\par 1 Vzácnejší jest jméno dobré než bohatství veliké, a prízen lepší než stríbro a zlato.
\par 2 Bohatý a chudý potkávají se, ucinitel obou jest Hospodin.
\par 3 Opatrný vida zlé, vyhne se, ale hloupí predce jdouce, težkosti docházejí.
\par 4 Pokory a bázne Hospodinovy odplata jest bohatství a sláva i život.
\par 5 Trní a osídla jsou na ceste prevráceného; kdož ostríhá duše své, vzdálí se od nich.
\par 6 Vyucuj mladého podlé zpusobu cesty jeho; nebo když se i zstará, neuchýlí se od ní.
\par 7 Bohatý nad chudými panuje, a vypujcující bývá služebníkem toho, jenž pujcuje.
\par 8 Kdo rozsívá nepravost, žíti bude trápení; prut zajisté prchlivosti jeho prestane.
\par 9 Oko dobrotivé, onot požehnáno bude; nebo udílí z chleba svého chudému.
\par 10 Vyvrz posmevace, a odejdet svada, anobrž prestane svár a lehkost.
\par 11 Kdo miluje cistotu srdce, a v cích rtech jest príjemnost, takového král prítelem bývá.
\par 12 Oci Hospodinovy ostríhají umení, ale snažnosti ošemetného prevrací.
\par 13 Ríká lenoch: Lev jest vne, naprostred ulic byl bych zabit.
\par 14 Jáma hluboká ústa postranních; ten, na kohož se hnevá Hospodin, vpadne tam.
\par 15 Bláznovství privázáno jest k srdci mladého, ale metla kázne vzdálí je od neho.
\par 16 Kdo utiská nuzného, aby rozmnožil své, a dává bohatému, jistotne bude v nouzi.
\par 17 Naklon ucha svého, a slyš slova moudrých, a mysl svou prilož k ucení mému.
\par 18 Nebo to bude utešenou vecí, jestliže je složíš v srdci svém, budou-li spolu nastrojena ve rtech tvých.
\par 19 Aby bylo v Hospodinu doufání tvé, oznamujit to dnes. I ty také ostríhej toho.
\par 20 Zdaližt jsem nenapsal znamenitých vecí z strany rad a umení,
\par 21 Atbych v známost uvedl jistotu recí pravých, tak abys vynášeti mohl slova pravdy tem, kteríž by k tobe poslali?
\par 22 Nelup nuzného, proto že nuzný jest, aniž potírej chudého v bráne.
\par 23 Nebo Hospodin povede pri jejich, a vydre duši tem, kteríž vydírají jim.
\par 24 Nebývej prítelem hnevivého, a s mužem prchlivým neobcuj,
\par 25 Abys se nenaucil stezkám jeho, a nevložil osídla na duši svou.
\par 26 Nebývej mezi rukojmemi, mezi slibujícími za dluhy.
\par 27 Nemáš-li, cím bys zaplatil, proc má kdo bráti luže tvé pod tebou?
\par 28 Neprenášej mezníku starodávního, kterýž ucinili otcové tvoji.
\par 29 Vídáš-li, že muž snažný v díle svém pred králi stává? Nestává pred nepatrnými.

\chapter{23}

\par 1 Když sedneš k jídlu se pánem, pilne šetr, co jest pred tebou.
\par 2 Jinak vrazil bys nuž do hrdla svého, byl-li bys lakotný.
\par 3 Nežádej lahudek jeho, nebo jsou pokrm oklamavatelný.
\par 4 Neusiluj, abys zbohatl; od opatrnosti své prestan.
\par 5 K bohatství-liž bys obrátil oci své? Ponevadž v náhle mizí; nebo sobe zdelalo krídla podobná orlicím, a zaletuje k nebi.
\par 6 Nejez chleba cloveka závistivého, a nežádej lahudek jeho.
\par 7 Nebo jak on sobe tebe váží v mysli své, tak ty pokrmu toho. Dít: Jez a pí, ale srdce jeho není s tebou.
\par 8 Skyvu svou, kterouž jsi snedl, vyvrátíš, a zmaríš slova svá utešená.
\par 9 Pred bláznem nemluv, nebo pohrdne opatrností recí tvých.
\par 10 Neprenášej mezníku starodávního, a na pole sirotku nevcházej.
\par 11 Silnýt jest zajisté ochránce jejich; ont povede pri jejich proti tobe.
\par 12 Zaved k ucení mysl svou, a uši své k recem umení.
\par 13 Neodjímej od mladého kázne; nebo umrskáš-li jej metlou, neumret.
\par 14 Ty metlou jej mrskávej, a tak duši jeho z pekla vytrhneš.
\par 15 Synu muj, bude-li moudré srdce tvé, veseliti se bude srdce mé všelijak ve mne;
\par 16 A plésati budou ledví má, když mluviti budou rtové tvoji pravé veci.
\par 17 Necht nezávidí srdce tvé hríšníku, ale radeji chod v bázni Hospodinove celý den.
\par 18 Nebo ponevadž jest odplata, nadeje tvá nebude podtata.
\par 19 Slyš ty, synu muj, a bud moudrý, a naprav na cestu srdce své.
\par 20 Nebývej mezi pijány vína, ani mezi žráci masa.
\par 21 Nebo opilec a žrác zchudne, a ospánlivost v hadry oblácí.
\par 22 Poslouchej otce svého, kterýž te zplodil, aniž pohrdej matkou svou, když se zstará.
\par 23 Pravdy nabud, a neprodávej jí, též moudrosti, umení a rozumnosti.
\par 24 Náramne bývá potešen otec spravedlivého, a ten, kdož zplodil moudrého, veselí se z neho.
\par 25 Nechat se tedy veselí otec tvuj a matka tvá, a at pléše rodicka tvá.
\par 26 Dej mi, synu muj, srdce své, a oci tvé cest mých at ostríhají.
\par 27 Nebo nevestka jest jáma hluboká, a studnice tesná žena cizí.
\par 28 Onat také jako loupežník úklady ciní, a zoufalce na svete rozmnožuje.
\par 29 Komu beda? komu ouvech? komu svady? komu krik? komu rány darmo? komu cervenost ocí?
\par 30 Tem, kteríž se zdržují na víne; tem, kteríž chodí, aby vyhledali strojené víno.
\par 31 Nehled na víno rdící se, že vydává v koflíku zári svou, a prímo vyskakuje.
\par 32 Naposledy jako had uštípne, a jako štír uštkne.
\par 33 Oci tvé hledeti budou na cizí, a srdce tvé mluviti bude prevrácené veci,
\par 34 A budeš jako ten, kterýž spí u prostred more, a jako ten, kterýž spí na vrchu sloupu bárky.
\par 35 Díš: Zbili mne, a nestonal jsem, tloukli mne, a necil jsem; když procítím, dám se zase v to.

\chapter{24}

\par 1 Nenásleduj lidí zlých, aniž žádej bývati s nimi.
\par 2 Nebo o zhoube premýšlí srdce jejich, a rtové jejich o trápení mluví.
\par 3 Moudrostí vzdelán bývá dum, a rozumností upevnen.
\par 4 Skrze umení zajisté pokojové naplneni bývají všelijakým zbožím drahým a utešeným.
\par 5 Muž moudrý jest silný, a muž umelý pridává síly.
\par 6 Nebo skrze rady opatrné svedeš bitvu, a vysvobození skrze množství rádcu.
\par 7 Vysoké jsou bláznu moudrosti; v bráne neotevre úst svých.
\par 8 Kdo myslí zle ciniti, toho nešlechetným nazovou.
\par 9 Zlé myšlení blázna jest hrích, a ohavnost lidská posmevac.
\par 10 Budeš-li se lenovati ve dni ssoužení, špatná bude síla tvá.
\par 11 Vytrhuj jaté k smrti; nebo od tech, ješto se chýlí k zabití, což bys se zdržel?
\par 12 Díš-li: Aj, nevedeli jsme o tom: zdaliž ten, jenž zpytuje srdce, nerozumí, a ten, kterýž jest strážce duše tvé, nezná, a neodplatí každému podlé skutku jeho?
\par 13 Synu muj, jez med, nebo dobrý jest, a plást sladký dásním tvým.
\par 14 Tak umení moudrosti duši tvé. Jestliže ji najdeš, onat bude mzda, a nadeje tvá nebude vytata.
\par 15 Neciniž úkladu, ó bezbožníce, príbytku spravedlivého, a nekaz odpocinutí jeho.
\par 16 Nebo ac sedmkrát padá spravedlivý, však zase povstává, bezbožníci pak padají ve zlém.
\par 17 Když by padl neprítel tvuj, neraduj se, a když by klesl, nechat nepléše srdce tvé,
\par 18 Aby snad nepopatril Hospodin, a nelíbilo by se to jemu, a odvrátil by od neho hnev svuj.
\par 19 Nehnevej se prícinou zlostníku, aniž následuj bezbožných.
\par 20 Nebo zlý nebude míti odplaty; svíce bezbožných zhasne.
\par 21 Boj se Hospodina, synu muj, i krále, a k neustavicným se neprimešuj.
\par 22 Nebo v náhle nastane bída jejich, a pomstu obou tech kdo zná?
\par 23 Také i toto moudrým náleží: Prijímati osobu v soudu není dobré.
\par 24 Toho, kdož ríká bezbožnému: Spravedlivý jsi, klnouti budou lidé, a v ošklivost jej vezmou národové.
\par 25 Ale kteríž kárají, budou potešeni, a prijde na ne požehnání dobrého.
\par 26 Bude líbati rty toho, kdož mluví slova pravá.
\par 27 Nastroj vne dílo své, a sprav je sobe na poli; potom také vystavíš dum svuj.
\par 28 Nebývej svedkem všetecným proti bližnímu svému, aniž lahodne namlouvej rty svými.
\par 29 Neríkej: Jakž mi ucinil, tak mu uciním; odplatím muži tomu podlé skutku jeho.
\par 30 Pres pole muže lenivého šel jsem, a pres vinici cloveka nemoudrého,
\par 31 A aj, porostlo všudy trním, prikryly všecko koprivy, a ohrada kamenná její byla zborená.
\par 32 A vida to, posoudil jsem toho; vida, vzal jsem to k výstraze.
\par 33 Malicko pospíš, malicko zdrímeš, malicko složíš ruce, abys poležel,
\par 34 V tom prijde jako pocestný chudoba tvá, a nouze tvá jako muž zbrojný.

\chapter{25}

\par 1 Jaké i tato jsou prísloví Šalomounova, kteráž shromáždili muži Ezechiáše, krále Judského:
\par 2 Sláva Boží jest skrývati vec, ale sláva králu zpytovati vec.
\par 3 Vysokosti nebes, a hlubokosti zeme, a srdce králu není žádného vystižení.
\par 4 Jako když bys odjal trusku od stríbra, ukáže se slevaci nádoba cistá:
\par 5 Tak když odejmeš bezbožného od oblíceje králova, tedy utvrzen bude v spravedlnosti trun jeho.
\par 6 Nestavej se za znamenitého pred králem, a na míste velikých nestuj.
\par 7 Nebo lépe jest, atby receno bylo: Vstup sem, nežli abys snížen byl pred knížetem; což vídávají oci tvé.
\par 8 Nevcházej v svár kvapne, tak abys naposledy neceho se nedopustil, kdyby te zahanbil bližní tvuj.
\par 9 Srovnej pri svou s bližním svým, a tajné veci jiného nevyjevuj,
\par 10 Atby lehkosti neucinil ten, kdož by to slyšel, až by i zlá povest tvá nemohla jíti nazpet.
\par 11 Jablka zlatá s rezbami stríbrnými jest slovo propovedené prípadne.
\par 12 Náušnice zlatá a ozdoba z ryzího zlata jest trestatel moudrý u toho, jenž poslouchá.
\par 13 Jako studenost snežná v cas žne, tak jest posel verný tem, kteríž jej posílají; nebo duši pánu svých ocerstvuje.
\par 14 Jako oblakové a vítr bez dešte, tak clovek, kterýž se chlubí darem lživým.
\par 15 Snášelivostí naklonen bývá vývoda, a jazyk mekký láme kosti.
\par 16 Nalezneš-li med, jez, pokudž by dosti bylo tobe, abys snad nasycen jsa jím, nevyvrátil ho.
\par 17 Zdržuj nohu svou od domu bližního svého, aby syt jsa tebe, nemel te v nenávisti.
\par 18 Kladivo a mec a strela ostrá jest každý, kdož mluví falešné svedectví proti bližnímu svému.
\par 19 Zub vylomený a noha vytknutá jest doufání v prevráceném v den úzkosti.
\par 20 Jako ten, kdož svlácí odev v cas zimy, a ocet lije k sanitru, tak kdož zpívá písnicky srdci smutnému.
\par 21 Jestliže by lacnel ten, jenž te nenávidí, nakrm jej chlebem, a žíznil-li by, napoj jej vodou.
\par 22 Nebo uhlí reravé shromáždíš na hlavu jeho, a Hospodin odplatí tobe.
\par 23 Vítr pulnocní zplozuje déšt, a tvár hnevivá jazyk tajne utrhající.
\par 24 Lépe jest bydliti v koute na streše,nežli s ženou svárlivou v dome spolecném.
\par 25 Voda studená duši ustalé jest novina dobrá z zeme daleké.
\par 26 Studnice nohami zakalená a pramen zkažený jest spravedlivý z místa svého pred bezbožným vystrcený.
\par 27 Jísti mnoho medu není dobre; tak zpytování slávy jejich není slavné.
\par 28 Mesto rozborené beze zdi jest muž, kterýž nemá moci nad duchem svým.

\chapter{26}

\par 1 Jako sníh v léte, a jako déšt ve žni, tak nepripadá na blázna cest.
\par 2 Jako vrabec prenáší se, a vlaštovice létá, tak zlorecení bez príciny nedojde.
\par 3 Bic na kone, uzda na osla, a kyj na hrbet blázna.
\par 4 Neodpovídej bláznu podlé bláznovství jeho, abys i ty jemu nebyl podobný.
\par 5 Odpovez bláznu podlé bláznovství jeho, aby sám u sebe nebyl moudrý.
\par 6 Jako by nohy osekal, bezpráví se dopouští ten, kdož sveruje poselství bláznu.
\par 7 Jakož nejednostejní jsou hnátové kulhavého, tak rec v ústech bláznu.
\par 8 Jako vložiti kámen do praku, tak jest, když kdo ctí blázna.
\par 9 Trn, kterýž se dostává do rukou opilého, jest prísloví v ústech bláznu.
\par 10 Veliký pán stvoril všecko, a dává odplatu bláznu, i odmenu prestupníkum.
\par 11 Jakož pes navracuje se k vývratku svému, tak blázen opetuje bláznovství své.
\par 12 Spatril-li bys cloveka, an jest moudrý sám u sebe, nadeje o bláznu lepší jest než o takovém.
\par 13 Ríká lenoch: Lev lítý jest na ceste, lev jest v ulici.
\par 14 Dvére se obracejí na stežejích svých, a lenoch na luži svém.
\par 15 Schovává lenivý ruku svou za nadra; težko mu vztáhnouti ji k ústum svým.
\par 16 Moudrejší jest lenivý u sebe sám, nežli sedm odpovídajících s soudem.
\par 17 Psa za uši lapá, kdož odcházeje, hnevá se ne v své pri.
\par 18 Jako nesmyslný vypouští jiskry a šípy smrtelné,
\par 19 Tak jest každý, kdož oklamává bližního, a ríká: Zdaž jsem nežertoval?
\par 20 Když není drev, hasne ohen; tak když nebude klevetníka, utichne svár.
\par 21 Uhel mrtvý k roznícení, a drva k ohni, tak clovek svárlivý k roznícení svady.
\par 22 Slova utrhace jako ubitých, ale však sstupují do vnitrností života.
\par 23 Stríbrná truska roztažená po strepe jsou rtové protivní a srdce zlé.
\par 24 Rty svými za jiného se staví ten, jenž nenávidí, ale u vnitrnosti své skládá lest.
\par 25 Když se ochotný ukáže recí svou, never mu; nebo sedmera ohavnost jest v srdci jeho.
\par 26 Prikrývána bývá nenávist chytre, ale zlost její zjevena bývá v shromáždení.
\par 27 Kdo jámu kopá, do ní upadá, a kdo valí kámen, na nej se obrací.
\par 28 Clovek jazyka ošemetného v nenávisti má ponížené, a ústy úlisnými zpusobuje pád.

\chapter{27}

\par 1 Nechlub se dnem zítrejším, nebo nevíš, cot ten den prinese.
\par 2 Nechat te chválí jiní, a ne ústa tvá, cizí, a ne rtové tvoji.
\par 3 Tíž má kamen, a váhu písek, ale hnev blázna težší jest nad to obé.
\par 4 Ukrutnáte vec hnev a prudká prchlivost, ale kdo ostojí pred závistí?
\par 5 Lepší jest domlouvání zjevné, než milování tajné.
\par 6 Bezpecnejší rány od prítele, než lahodná líbání nenávidícího.
\par 7 Duše sytá pohrdá i medem, ale duši lacné každá horkost sladká.
\par 8 Jako pták zaletuje od hnízda svého, tak muž odchází od místa svého.
\par 9 Mast a kadení obveseluje srdce; tak sladkost prítele víc než rada vlastní.
\par 10 Prítele svého a prítele otce svého neopouštej, a do domu bratra svého nechod v cas bídy své; lepšít jest soused blízký, než bratr daleký.
\par 11 Bud moudrý, synu muj, a obvesel srdce mé, at mám co odpovedíti tomu, kdož mi utrhá.
\par 12 Opatrný vida zlé, vyhne se, ale hloupí predce jdouce, težkosti docházejí.
\par 13 Vezmi roucho toho, kterýž slíbil za cizího, a od toho, kdo za cizozemku slíbil, základ jeho.
\par 14 Tomu, kdož dobrorecí príteli svému hlasem velikým, ráno vstávaje, za zlorecení pocteno bude.
\par 15 Kapání ustavicné v cas prívalu, a žena svárlivá rovní jsou sobe;
\par 16 Kdož ji schovává, schovává vítr, a jako mast v pravici voneti bude.
\par 17 Železo železem se ostrí; tak muž zostruje tvár prítele svého.
\par 18 Kdo ostríhá fíku, jídá ovoce jeho; tak kdo ostríhá pána svého, pocten bude.
\par 19 Jakož u vode tvár proti tvári se ukazuje, tak srdce cloveka cloveku.
\par 20 Propast a zahynutí nebývají nasyceni, tak oci cloveka nasytiti se nemohou.
\par 21 Teglík stríbra a pec zlata zkušuje, tak cloveka povest chvály jeho.
\par 22 Bys blázna i v stupe mezi krupami píchem zopíchal, neodejde od neho bláznovství jeho.
\par 23 Pilne prihlídej k dobytku svému, pecuj o stáda svá.
\par 24 Nebo ne na veky trvá bohatství, ani koruna do pronárodu.
\par 25 Když zroste tráva, a ukazuje se bylina, tehdáž at se shromažduje seno s hor.
\par 26 Beránkové budou k odevu tvému, a záplata pole kozelci.
\par 27 Nadto dostatek mléka kozího ku pokrmu tvému, ku pokrmu domu tvého, a živnosti devek tvých.

\chapter{28}

\par 1 Utíkají bezbožní, ac jich žádný nehoní, ale spravedliví jako mladý lev smelí jsou.
\par 2 \par prestoupení zeme mnoho knížat jejích, ale \par cloveka rozumného a umelého trvánlivé bývá panování.
\par 3 Muž chudý, kterýž utiská nuzné, podoben jest prívalu zachvacujícímu, za címž nebývá chleba.
\par 4 Kterí opouštejí zákon, chválí bezbožného, ale kteríž ostríhají zákona, velmi jsou jim na odpor.
\par 5 Lidé zlí nesrozumívají soudu, ti pak, kteríž hledají Hospodina, rozumejí všemu.
\par 6 Lepší jest chudý, kterýž chodí v uprímnosti své, než prevrácený na kterékoli ceste, ackoli jest bohatý.
\par 7 Kdo ostríhá zákona, jest syn rozumný; kdož pak s žráci tovaryší, hanbu ciní otci svému.
\par 8 Kdo rozmnožuje statek svuj lichvou a úrokem, shromažduje tomu, kdož by jej z milosti chudým rozdeloval.
\par 9 Kdo odvrací ucho své, aby neslyšel zákona, i modlitba jeho jest ohavností.
\par 10 Kdo zavodí uprímé na cestu zlou, do jámy své sám vpadne, ale uprímí dedicne obdrží dobré.
\par 11 Moudrý jest u sebe sám muž bohatý, ale chudý rozumný vystihá jej.
\par 12 Když plésají spravedliví, velmi to pekne sluší; ale když povstávají bezbožní, vyhledáván bývá clovek.
\par 13 Kdo prikrývá prestoupení svá, nepovede se jemu štastne; ale kdož je vyznává a opouští, milosrdenství dujde.
\par 14 Blahoslavený clovek, kterýž se strachuje vždycky; ale kdož zatvrzuje srdce své, upadne ve zlé.
\par 15 Lev rvoucí a nedved hladovitý jest panovník bezbožný nad lidem nuzným.
\par 16 Kníže bez rozumu bývá veliký drác, ale kdož v nenávisti má mrzký zisk, prodlí dnu.
\par 17 Cloveka, kterýž násilí ciní krvi lidské, ani nad jamou, když utíká, žádný ho nezadrží.
\par 18 Kdo chodí upríme, zachován bude, prevrácený pak na kterékoli ceste padne pojednou.
\par 19 Kdo delá zemi svou, nasycen bývá chlebem; ale kdož následuje zahalecu, nasycen bývá chudobou.
\par 20 Muž verný prisporí požehnání, ale kdož chvátá zbohatnouti, nebývá bez viny.
\par 21 Prijímati osobu není dobré; nebo mnohý \par kus chleba nepráve ciní.
\par 22 Clovek závistivý chvátá k statku, nic neveda, že nouze na nej prijde.
\par 23 Kdo domlouvá cloveku, potom spíše milost nalézá nežli ten, kterýž lahodí jazykem.
\par 24 Kdo loupí otce svého a matku svou, a ríká, že to není žádný hrích, tovaryš jest vražedlníka.
\par 25 Vysokomyslný vzbuzuje svár, ale kdo doufá v Hospodina, hojnost míti bude.
\par 26 Kdo doufá v srdce své, blázen jest; ale kdož chodí moudre, pomuže sobe.
\par 27 Kdo dává chudému, nebude míti žádného nedostatku; kdož pak zakrývá oci své, bude míti množství zlorecení.
\par 28 Když povstávají bezbožní, skrývá se clovek; ale když hynou, rozmnožují se spravedliví.

\chapter{29}

\par 1 Clovek, kterýž casto kárán bývaje, zatvrzuje šíji, rychle potrín bude, tak že neprospeje žádné lékarství.
\par 2 Když se množí spravedliví, veselí se lid; ale když panuje bezbožník, vzdychá lid.
\par 3 Muž, kterýž miluje moudrost, obveseluje otce svého; ale kdož se pritovaryšuje k nevestkám, mrhá statek.
\par 4 Král soudem upevnuje zemi, muž pak, kterýž bére dary, borí ji.
\par 5 Clovek, kterýž pochlebuje príteli svému, rozprostírá sít pred nohama jeho.
\par 6 Výstupek bezbožného jest jemu osídlem, spravedlivý pak prozpevuje a veselí se.
\par 7 Spravedlivý vyrozumívá pri nuzných, ale bezbožník nemá s to rozumnosti ani umení.
\par 8 Muži posmevaci zavozují mesto, ale moudrí odvracují hnev.
\par 9 Muž moudrý, kterýž se nesnadní s mužem bláznivým, bud že se pohne, bud že se smeje, nemá pokoje.
\par 10 Vražedlníci v nenávisti mají uprímého, ale uprímí pecují o duši jeho.
\par 11 Všecken duch svuj vypouští blázen, ale moudrý na potom zdržuje jej.
\par 12 Pána toho, kterýž rád poslouchá slov lživých, všickni služebníci jsou bezbožní.
\par 13 Chudý a drác potkávají se, obou dvou však oci osvecuje Hospodin.
\par 14 Krále toho, kterýž soudí práve nuzné, trun na veky bývá utvrzen.
\par 15 Metla a kárání dává moudrost, ale díte sobe volné k hanbe privodí matku svou.
\par 16 Když se rozmnožují bezbožní, rozmnožuje se prevrácenost, a však spravedliví spatrují pád jejich.
\par 17 Tresci syna svého, a prineset odpocinutí, a zpusobí rozkoš duši tvé.
\par 18 Když nebývá videní, rozptýlen bývá lid; kdož pak ostríhá zákona, blahoslavený jest.
\par 19 Slovy nebývá napraven služebník; nebo rozumeje, však neodpoví.
\par 20 Spatril-li bys cloveka, an jest kvapný v vecech svých, lepší jest nadeje o bláznu, než o takovém.
\par 21 Kdo rozkošne chová z detinství služebníka svého, naposledy bude syn.
\par 22 Clovek hnevivý vzbuzuje svár, a prchlivý mnoho hreší.
\par 23 Pýcha cloveka snižuje jej, ale chudý duchem dosahuje slávy.
\par 24 Kdo má spolek s zlodejem, v nenávisti má duši svou; zlorecení slyší, však neoznámí.
\par 25 Strašlivý clovek klade sobe osídlo, ale kdo doufá v Hospodina, bývá povýšen.
\par 26 Mnozí hledají tvári pánu, ješto od Hospodina jest soud jednoho každého.
\par 27 Ohavností spravedlivým jest muž nepravý, ohavností pak bezbožnému, kdož upríme krácí.

\chapter{30}

\par 1 Slova Agura, syna Jáke. Sepsání recí muže toho k Itielovi, k Itielovi a Uchalovi.
\par 2 Jiste žet jsem hloupejší nad jiné, tak že rozumnosti cloveka obecného nemám,
\par 3 Aniž jsem se naucil moudrosti, a umení svatých neumím.
\par 4 Kdo vstoupil v nebe, i sstoupil? Kdo sebral vítr do hrstí svých? Kdo shrnul vody v roucho své? Kdo upevnil všecky konciny zeme? Které jméno jeho, a jaké jméno syna jeho, víš-li?
\par 5 Všeliká výmluvnost Boží precištená jest; ont jest štít doufajících v neho.
\par 6 Nepridávej k slovum jeho, aby te nekáral, a byl bys ve lži postižen.
\par 7 Dvou vecí žádám od tebe, neoslýchejž mne, prvé než umru:
\par 8 Marnost a slovo lživé vzdal ode mne, chudoby neb bohatství nedávej mi, živ mne pokrmem vedlé potreby mé,
\par 9 Abych snad nasycen jsa, te nezaprel, a nerekl: Kdo jest Hospodin? a abych zchudna, nekradl, a nebral naprázdno jména Hospodina Boha svého.
\par 10 Nesoc na služebníka pred pánem jeho, atby nezlorecil, a ty abys nehrešil.
\par 11 Jest pokolení, kteréž otci svému zlorecí, a matce své nedobrorecí.
\par 12 Jest pokolení cisté samo u sebe, ackoli od necistot svých není obmyto.
\par 13 Jest pokolení, jehož vysoké jsou oci, a vícka jeho jsou vyzdvižená.
\par 14 Jest pokolení, jehož zubové jsou mecové, a trenovní zubové jeho nožové, k zžírání chudých na zemi a nuzných na svete.
\par 15 Pijavice má dve dcery ríkající: Dej, dej. Tri veci nebývají nasyceny, anobrž ctyry, kteréž nikdy nereknou: Dosti:
\par 16 Peklo a život neplodné, zeme též nebývá nasycena vodou, a ohen neríká: Dosti.
\par 17 Oko, kteréž se posmívá otci, a pohrdá poslušenstvím matky, vyklubí krkavci potocní, aneb snedí je orlicata.
\par 18 Tri tyto veci skryty jsou prede mnou, nýbrž ctyry, kterýchž neznám:
\par 19 Cesty orlice v povetrí, cesty hada na skále, cesty lodí u prostred more, a cesty muže pri panne.
\par 20 Takováž jest cesta ženy cizoložné: Jí, a utre ústa svá, a dí: Nepáchala jsem nepravosti.
\par 21 Pode trmi vecmi pohybuje se zeme, anobrž pod ctyrmi, jichž nemuž snésti:
\par 22 Pod služebníkem, když kraluje, a bláznem, když se nasytí chleba;
\par 23 Pod omrzalou, když se vdá, a devkou, když dedickou bývá paní své.
\par 24 Ctyry tyto veci jsou malé na zemi, a však jsou moudrejší nad mudrce:
\par 25 Mravenci, lid nesilný, kteríž však pripravují v léte pokrm svuj;
\par 26 Králíkové, lid nesilný, kteríž však stavejí v skále dum svuj;
\par 27 Krále nemají kobylky, a však vycházejí po houfích všecky;
\par 28 Pavouk rukama delá, a bývá na palácích královských.
\par 29 Tri tyto veci udatne vykracují, anobrž ctyry, kteréž zmužile chodí:
\par 30 Lev nejsilnejší mezi zvíraty, kterýž neustupuje pred žádným;
\par 31 Prepásaný na bedrách kun neb kozel, a král, proti nemuž žádný nepovstává.
\par 32 Jestliže jsi bláznil, vynášeje se, a myslil-lis zle, ruku na ústa polož.
\par 33 Kdo tluce smetanu, stlouká máslo, a stiskání nosu vyvodí krev, tak popouzení k hnevu vyvodí svár.

\chapter{31}

\par 1 Slova proroctví Lemuele krále, kterýmž vyucovala jej matka jeho.
\par 2 Co dím, synu muj, co, synu života mého? Co, rku, dím, synu slibu mých?
\par 3 Nedávej ženám síly své, ani cest svých tem, kteréž k zahynutí privodí krále.
\par 4 Ne králum, ó Lemueli, ne králum náleží píti víno, a ne pánum žádost nápoje opojného,
\par 5 Aby pije, nezapomnel na ustanovení, a nezmenil pre všech lidí ssoužených.
\par 6 Dejte nápoj opojný hynoucímu, a víno tem, kteríž jsou truchlivého ducha,
\par 7 At se napije, a zapomene na chudobu svou, a na trápení své nezpomíná více.
\par 8 Otevri ústa svá za nemého, v pri všech oddaných k smrti,
\par 9 Otevri, rku, ústa svá, sud spravedlive, a ved pri chudého a nuzného.
\par 10 Ženu statecnou kdo nalezne? Nebo daleko nad perly cena její.
\par 11 Doveruje se jí srdce muže jejího; nebo tu koristí nebude nedostatku.
\par 12 Dobre ciní jemu a ne zle, po všecky dny života svého.
\par 13 Hledá pilne vlny a lnu, a delá štastne rukama svýma.
\par 14 Jest podobná lodi kupecké, zdaleka priváží pokrm svuj.
\par 15 Kterážto velmi ráno vstávajíc, dává pokrm celedi své, a podíl náležitý devkám svým.
\par 16 Rozsuzuje pole, a ujímá je; z výdelku rukou svých štepuje i vinici.
\par 17 Prepasuje silou bedra svá, a zsiluje ramena svá.
\par 18 Zakouší, jak jest užitecné zamestknání její; ani v noci nehasne svíce její.
\par 19 Rukama svýma sahá k kuželi, a prsty svými drží vreteno.
\par 20 Ruku svou otvírá chudému, a ruce své vztahuje k nuznému.
\par 21 Nebojí se za celed svou v cas snehu; nebo všecka celed její oblácí se v roucho dvojnásobní.
\par 22 Koberce delá sobe z kmentu, a z zlatohlavu jest odev její.
\par 23 Patrný jest v branách manžel její, když sedá s staršími zeme.
\par 24 Plátno drahé delá, a prodává; též i pasy prodává kupci.
\par 25 Síla a krása odev její, nestará se o casy potomní.
\par 26 Ústa svá otvírá k moudrosti, a naucení dobrotivosti v jazyku jejím.
\par 27 Spatruje obcování celedi své, a chleba zahálky nejí.
\par 28 Povstanouce synové její, blahoslaví ji; manžel její také chválí ji,
\par 29 Ríkaje: Mnohé ženy statecne sobe pocínaly, ty pak prevyšuješ je všecky.
\par 30 Oklamavatelná jest príjemnost a marná krása; žena, kteráž se bojí Hospodina, tat chválena bude.
\par 31 Dejtež takové z ovoce rukou jejích, a nechat ji chválí v branách skutkové její.

\end{document}