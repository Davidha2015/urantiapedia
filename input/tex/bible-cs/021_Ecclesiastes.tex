\begin{document}

\title{Ecclesiastes}

\chapter{1}

\par 1 Slova kazatele syna Davidova, krále v Jeruzaléme.
\par 2 Marnost nad marnostmi, rekl kazatel, marnost nad marnostmi, a všecko marnost.
\par 3 Jaký užitek má clovek ze všelijaké práce své, kterouž vede pod sluncem?
\par 4 Vek pomíjí, a jiný vek nastává, ackoli zeme na veky trvá.
\par 5 Vychází slunce, i zapadá slunce, a k místu svému chvátá, kdež vychází.
\par 6 Jde ku poledni, a obrací se na pulnoci, sem i tam se toce, chodí vítr, a okolky svými navracuje se vítr.
\par 7 Všecky reky jdou do more, a však se more nepreplnuje; do místa, do nehož tekou reky, navracují se, aby zase odtud vycházely.
\par 8 Všecky veci jsou plné zaneprázdnení, aniž muže clovek vymluviti; nenasytí se oko hledením, aniž se naplní ucho slyšením.
\par 9 Což bylo, jest to, což býti má; a což se nyní deje, jest to, což se díti bude; aniž jest co nového pod sluncem.
\par 10 Jest-liž jaká vec, o níž by ríci mohl: Pohled, tot jest cosi nového? Ano již to bylo pred veky, kteríž byli pred námi.
\par 11 Není pameti prvních vecí, aniž také potomních, kteréž budou, památka zustane u tech, jenž potom nastanou.
\par 12 Já kazatel byl jsem králem nad Izraelem v Jeruzaléme,
\par 13 A priložil jsem mysl svou k tomu, jak bych vyhledati a vystihnouti mohl rozumností svou všecko to, což se deje pod nebem. (Takové bídné zamestknání dal Buh synum lidským, aby se jím bedovali.)
\par 14 Videl jsem všecky skutky, dející se pod sluncem, a aj, všecko jest marnost a trápení ducha.
\par 15 Což krivého jest, nemuže se zprímiti, a nedostatkové nemohou secteni býti.
\par 16 Protož tak jsem myslil v srdci svém, rka: Aj, já zvelebil jsem a rozšíril moudrost nade všecky, kteríž byli prede mnou v Jeruzaléme, a srdce mé dosáhlo množství moudrosti a umení.
\par 17 I priložil jsem mysl svou, abych poznal moudrost a umení, nemoudrost i bláznovství, ale shledal jsem, že i to jest trápení ducha.
\par 18 Nebo kde jest mnoho moudrosti, tu mnoho hnevu; a kdož rozmnožuje umení, rozmnožuje bolest.

\chapter{2}

\par 1 Rekl jsem opet srdci svému: Nuže nyní zkusím te v veselí, užívejž tedy dobrých vecí. A hle, i to marnost.
\par 2 Smíchu jsem rekl: Blázníš, a veselí: Co to deláš?
\par 3 Premyšloval jsem v srdci svém, abych povoloval u víne telu svému, srdce však své spravuje moudrostí, a prídržel se bláznovství dotud, až bych zkusil, co by lepšího bylo synum lidským, aby cinili pod nebem v poctu dnu života svého.
\par 4 Veliké jsem skutky cinil, vystavel jsem sobe domy, štípil jsem sobe vinice.
\par 5 Vzdelal jsem sobe zahrady a štepnice, a štípil jsem v nich stromy všelijakého ovoce.
\par 6 Nadelal jsem sobe rybníku, abych svlažoval jimi les plodící dríví.
\par 7 Najednal jsem sobe služebníku a devek, a mel jsem celed v dome svém; k tomu i stáda skotu a bravu veliká mel jsem nade všecky, kteríž byli prede mnou v Jeruzaléme.
\par 8 Nahromáždil jsem sobe také stríbra a zlata a klínotu od králu a krajin; zpusobil jsem sobe zpeváky a zpevakyne i jiné rozkoše synu lidských a nástroje muzické rozlicné.
\par 9 A tak velikým jsem ucinen, a zrostl jsem nade všecky, kteríž prede mnou byli v Jeruzaléme; nadto moudrost má zustávala pri mne.
\par 10 A cehožkoli žádaly oci mé, nezbránil jsem jim, aniž jsem zbranoval srdci svému jakého veselí; srdce mé zajisté veselilo se ze vší práce mé, a to byl podíl muj ze vší práce mé.
\par 11 Ale jakž jsem se ohlédl na všecky skutky své, kteréž cinily ruce mé, a na práci úsilne vedenou, a aj, všecko marnost a trápení ducha, a že nic není užitecného pod sluncem.
\par 12 Procež obrátil jsem se, abych spatroval moudrost a nemoudrost, i bláznovství. (Nebo co by clovek spravil, chteje následovati krále? To, což již jiní spravili.)
\par 13 I videl jsem, že jest užitecnejší moudrost než bláznovství, tak jako jest užitecnejší svetlo nežli temnost.
\par 14 Moudrý má oci v hlave své, blázen pak ve tmách chodí; a však poznal jsem, že jednostejné príhody všechnem se priházejí.
\par 15 Protož jsem rekl v srdci svém: Málit mi se tak díti, jako se deje bláznu, procež jsem tedy moudrostí predcil? A tak rekl jsem v srdci svém: I to jest marnost.
\par 16 Nebo není památka moudrého jako i blázna na veky, proto že to, což nyní jest, ve dnech budoucích všecko v zapomenutí prichází, a že jakož umírá moudrý, tak i blázen.
\par 17 Procež mrzí mne tento život; nebo semi nelíbí nic, což se deje pod sluncem, ponevadž všecky veci jsou marnost a trápení ducha.
\par 18 Ano mrzí mne i všecka práce má, kterouž jsem vedl pod sluncem proto že jí zanechati musím cloveku, kterýž bude po mne.
\par 19 A kdo ví, bude-li moudrý, ci blázen? A však panovati bude nade vší prací mou,kterouž jsem vedl, a v níž jsem moudrý byl pod sluncem. A i to marnost.
\par 20 I prišel jsem na to, abych pochybil v srdci svém o vší práci, kterouž jsem konal, a v níž jsem moudrý byl pod sluncem.
\par 21 Mnohý zajisté clovek pracuje moudre, umele a spravedlive, a však jinému, kterýž nepracoval o tom, nechává toho za podíl jemu, ješto i to jest marnost a bídná vec.
\par 22 Nebo co má clovek ze vší práce své a z kvaltování srdce svého, kteréž snáší pod sluncem,
\par 23 Ponevadž všickni dnové jeho bolestní, a zamestknání jeho hnev, tak že ani v noci neodpocívá srdce jeho? A i to marnost.
\par 24 Zdaliž není to chvalitebné pri cloveku, jísti a píti, a uciniti životu svému pohodlí z práce své? Ac i to také videl jsem, že z ruky Boží pochází.
\par 25 Nebo kdož by jísti a užívati mel toho nežli já?
\par 26 Cloveku zajisté, kterýž se líbí jemu, dává moudrost, umení a veselí; hríšníku pak dává trápení, aby shromaždoval a hrnul, cehož by zanechal tomu, kterýž se líbí Bohu. I to také jest marnost a trápení ducha.

\chapter{3}

\par 1 Všeliká vec má jistý cas, a každé predsevzetí pod nebem svou chvíli.
\par 2 Jest cas rození i cas umírání, cas sázení a cas vykopání, což vsazeno bývá;
\par 3 Cas mordování a cas hojení, cas borení a cas stavení;
\par 4 Cas pláce a cas smíchu, cas smutku a cas proskakování;
\par 5 Cas rozmítání kamení a cas shromaždování kamení, cas objímání a cas vzdálení se od objímání;
\par 6 Cas hledání a cas ztracení, cas chování a cas zavržení;
\par 7 Cas roztrhování a cas sšívání, cas mlcení a cas mluvení;
\par 8 Cas milování a cas nenávidení, cas boje a cas pokoje.
\par 9 Co tedy má ten, kdo práci vede, z toho, o cemž pracuje?
\par 10 Videl jsem zamestknání, kteréž dal Buh synum lidským, aby se jím trápili.
\par 11 Sám všecko ciní ušlechtile casem svým, nýbrž i žádost sveta dal v srdce jejich, aby nestihal clovek díla toho, kteréž delá Buh, ani pocátku ani konce.
\par 12 Odtud seznávám, že nic lepšího nemají, než aby se veselili, a cinili dobre v živote svém,
\par 13 Ac i to, když všeliký clovek jí a pije, a užívá dobrých vecí ze všelijaké práce své, jest dar Boží.
\par 14 Znám, že cožkoli ciní Buh, to trvá na veky; nemuže se k tomu nic pridati, ani od toho co odjíti. A ciní to Buh, aby se báli oblíceje jeho.
\par 15 To, což bylo, i nyní jest, a což bude, již bylo; nebo Buh obnovuje to, což pominulo.
\par 16 Presto videl jsem ješte pod sluncem na míste soudu bezbožnost, a na míste spravedlnosti nespravedlnost.
\par 17 I rekl jsem v srdci svém: Budet Buh spravedlivého i bezbožného souditi; nebo tam bude cas každému predsevzetí i každému skutku.
\par 18 Rekl jsem v srdci svém o zpusobu synu lidských, že jim ukázal Buh, aby videli, že jsou podobni hovadum.
\par 19 Prípadnost synu lidských a prípadnost hovad jest prípadnost jednostejná. Jakož umírá ono, tak umírá i on, a dýchání jednostejné všickni mají, aniž co napred má clovek pred hovadem; nebo všecko jest marnost.
\par 20 Obé to jde k místu jednomu; obé jest z prachu, obé také zase navracuje se do prachu.
\par 21 Kdo to zná, že duch synu lidských vstupuje zhuru, a duch hovadí že sstupuje pod zemi?
\par 22 Protož spatril jsem, že nic není lepšího, než veseliti se cloveku v skutcích svých, ponevadž to jest podíl jeho. Nebo kdo jej k tomu privede, aby poznati mohl to, což jest budoucího po nem?

\chapter{4}

\par 1 Opet obrátiv se, i videl jsem všeliká ssoužení, kteráž se dejí pod sluncem, a aj, slzy krivdu trpících, ješto nemají potešitele, ani moci k vyjití z ruky tech, kteríž je ssužují, a nemají potešitele.
\par 2 Protož já chválil jsem mrtvé, kteríž již zemreli, více nežli živé, kteríž jsou živi až po dnes.
\par 3 Nýbrž nad oba tyto štastnejší jest ten, kterýž ješte nebyl, a nevidel skutku zlého, dejícího se pod sluncem.
\par 4 Nebo spatril jsem všelikou práci a každé dobré dílo, že jest k závisti jednech druhým. I to také jest marnost a trápení ducha.
\par 5 Blázen skládá ruce své, a jí maso své, ríkaje:
\par 6 Lepší jest plná hrst s odpocinutím,nežli prehršlí plné s prací a trápením ducha.
\par 7 Opet obrátiv se, videl jsem jinou marnost pod sluncem:
\par 8 Jest samotný nekdo, nemaje žádného, ani syna, ani bratra, a však není konce všeliké práci jeho, ani oci jeho nemohou se nasytiti bohatství. Nepomyslí: Komu já pracuji, tak že i životu svému ujímám pohodlí? I to také jest marnost a bídné zaneprázdnení.
\par 9 Lépet jest dvema než jednomu; mají zajisté dobrý užitek z práce své.
\par 10 Nebo padne-li který z nich, druhý pozdvihne tovaryše svého. Beda tedy samotnému, když by padl; nebo nemá druhého, aby ho pozdvihl.
\par 11 Také budou-li dva spolu ležeti, zahrejí se, ale jeden jak se zahreje?
\par 12 Ovšem, jestliže by se kdo jednoho zmocniti chtel, dva postaví se proti nemu; ano trojnásobní provázek nesnadne se pretrhne.
\par 13 Lepší jest díte chudé a moudré, než král starý a blázen, kterýž neumí již ani napomenutí prijímati,
\par 14 Ackoli z žaláre vychází, aby kraloval, nýbrž i v království svém muže na chudobu prijíti.
\par 15 Videl jsem všecky živé, kteríž chodí pod sluncem, ani se prídrželi pacholete, potomka onoho, kterýž mel kralovati místo neho.
\par 16 Nebývalo konce té vrtkosti všeho lidu, jakž toho, kterýž byl pred nimi, takž ani potomci nebudou se tešiti z neho. Protož i to jest marnost a trápení ducha.

\chapter{5}

\par 1 Ostríhej nohy své, když jdeš do domu Božího, a bud hotovejší k slyšení nežli k dávání obetí bláznu; nebo oni neznají toho, že zle ciní.
\par 2 Nebývej rychlý k mluvení, ani srdce tvé kvapné k vynášení slova pred oblícejem Božím, ponevadž Buh jest na nebi, a ty na zemi; protož necht jsou slova tvá nemnohá.
\par 3 Nebo jakož prichází sen z velikého pracování, tak hlas blázna z množství slov.
\par 4 Když bys ucinil slib Bohu, neprodlévej ho splniti, nebo nemá líbosti v blázních. Cožkoli slíbíš, spln.
\par 5 Lépe jest, abys nesliboval, než abys slibe, neplnil.
\par 6 Nedopouštej ústum svým, aby k hríchu privodila telo tvé, aniž ríkej pred andelem, že to jest poblouzení. Proc máš hnevati Boha recí svou, kterýž by na zkázu privedl dílo rukou tvých?
\par 7 Nebo kdež jest mnoho snu, tu i marnosti a slova mnohá, ale ty Boha se boj.
\par 8 Jestliže bys nátisk chudého a zadržení soudu a spravedlnosti spatril v krajine, nediv se té veci; nebo vyšší vysokého šetrí, a ješte vyšší nad nimi.
\par 9 Zemský pak obchod u všech prední místo má; i král rolí slouží.
\par 10 Kdo miluje peníze, nenasytí se penezi; a kdo miluje hojnost, nebude míti užitku. I to jest marnost.
\par 11 Kde jest mnoho statku, mnoho bývá i tech, kteríž jedí jej. Jakýž tedy má užitek pán jeho? Jediné, že ocima svýma hledí na nej.
\par 12 Sladký jest sen pracovitému, jez on málo neb mnoho, ale sytost bohatého nedopouští mu spáti.
\par 13 Jest prebídná vec, kterouž jsem videl pod sluncem: Bohatství nachované tomu, kdož je má, k jeho zlému.
\par 14 Nebo hyne bohatství takové pro zlou správu; syn, kteréhož zplodí, nebude míti v ruce své niceho.
\par 15 Jakž vyšel z života matky své nahý, tak zase odchází, jakž prišel, aniž ceho odnáší z práce své, což by vzal v ruku svou.
\par 16 A tot jest také prebídná vec, že rovne, jakž prišel, tak odjíti musí. Protož jaký užitek míti bude toho, že pracoval u vítr?
\par 17 K tomu, že po všecky dny své v temnostech jídal, s mnohým zurením, nemocí a hnevem?
\par 18 Aj, tot jsem já spatril, že dobrá a cistá jest vec jísti a píti a užívati pohodlí ze vší práce své, kterouž kdo vede pod sluncem v poctu dnu života svého, kteréž dal jemu Buh, nebo to jest podíl jeho;
\par 19 A že kterémukoli cloveku dal Buh bohatství a zboží, a dopustil, aby užíval jich, a bral díl svuj, a veselil se z práce své, to jest dar Boží.
\par 20 Nebo nebude mnoho pamatovati na dny života svého, proto že Buh jemu preje veselí srdce jeho.

\chapter{6}

\par 1 Jest bídná vec, kterouž jsem videl pod sluncem, a lidem obycejná:
\par 2 Kterému cloveku dal Buh bohatství a zboží i slávu, tak že nemá nedostatku duše jeho v nicemž, cehokoli žádá, a však nedopouští mu Buh užívati tech vecí, ale jiný leckdos sžíre to, a tot jest marnost a bídná vec.
\par 3 Zplodil-li by kdo sto synu, a byl by živ mnoho let, jakkoli rozmnoženi jsou dnové let jeho, nebyl-li život jeho nasycen dobrými vecmi, a nemel by ani pohrbu, pravím, že štastnejší jest nedochudce nežli on.
\par 4 Nebo ono v zmarení pricházeje, do temností odchází, a jméno jeho temnostmi prikryto bývá.
\par 5 Nýbrž ani slunce nevídá, aniž ceho poznává, a tak odpocinutí má lepší nežli onen.
\par 6 A byt pak byl živ dva tisíce let, a pohodlí by neužil, zdaliž k jednomu místu všickni neodcházejí?
\par 7 Všecka práce cloveka jest pro ústa jeho, a však duše jeho nemuže se nasytiti.
\par 8 Nebo co má více moudrý nežli blázen? A co chudý, kterýž se umí chovati mezi lidmi?
\par 9 Lépe jest videti nežli žádati, ale i to jest marnost a trápení ducha.
\par 10 Címžkoli jest, dávno jest tím nazván, a známé bylo, že clovek býti mel, a že se nebude moci souditi s silnejším, nežli jest sám.
\par 11 A ponevadž predsevzetí mnohá rozmnožují marnost, co na tom má clovek?
\par 12 Nebo kdo ví, co by bylo dobrého cloveku v tomto živote, v poctu dnu marného života jeho, kteríž pomíjejí jako stín? Aneb kdo oznámí cloveku, co se díti bude po nem pod sluncem?

\chapter{7}

\par 1 Lepší jest jméno dobré nežli mast výborná, a den smrti než den narození cloveka.
\par 2 Lépe jest jíti do domu zámutku, nežli jíti do domu hodování, pro dokonání každého cloveka, a kdož jest živ, složí to v srdci svém.
\par 3 Lepší jest horlení nežli smích; nebo zurivá tvár polepšuje srdce.
\par 4 Srdce moudrých v dome zámutku, ale srdce bláznu v dome veselí.
\par 5 Lépe jest slyšeti žehrání moudrého,nežli aby nekdo poslouchal písne bláznu.
\par 6 Nebo jako praštení trní pod hrncem, tak smích blázna. A i to jest marnost.
\par 7 Ssužování zajisté k bláznovství privodí moudrého, a dar oslepuje srdce.
\par 8 Lepší jest skoncení veci nežli pocátek její; lepší jest dlouho cekající nežli vysokomyslný.
\par 9 Nebud kvapný v duchu svém k hnevu; nebo hnev v lunu bláznu odpocívá.
\par 10 Neríkej: Cím jest to, že dnové první lepší byli nežli tito? Nebo bys se nemoudre na to vytazoval.
\par 11 Dobrá jest moudrost s statkem, a velmi užitecná tem, kteríž vidí slunce;
\par 12 Nebo v stínu moudrosti a v stínu stríbra odpocívají. A však prednejší jest umení moudrosti, prináší život tem, kdož ji mají.
\par 13 Hled na skutky Boží. Nebo kdo muže zprímiti to, což on zkrivil?
\par 14 V den dobrý užívej dobrých vecí, a v den zlý bud bedliv; nebo i to naproti onomu ucinil Buh z té príciny, aby nenalezl clovek po nem niceho.
\par 15 Všecko to videl jsem za dnu marnosti své: Bývá spravedlivý, kterýž hyne s spravedlností svou; tolikéž bývá bezbožný, kterýž dlouho živ jest v zlosti své.
\par 16 Nebývej príliš spravedlivý, aniž bud príliš moudrý. Proc máš na zkázu pricházeti?
\par 17 Nebud príliš starostlivý, aniž bývej bláznem. Proc máš umírati dríve casu svého?
\par 18 Dobrét jest, abys se onoho prídržel, a tohoto se nespouštel; nebo kdo se bojí Boha, ujde všeho toho.
\par 19 Moudrost posiluje moudrého nad desatero knížat, kteríž jsou v meste.
\par 20 Není zajisté cloveka spravedlivého na zemi, kterýž by cinil dobre a nehrešil.
\par 21 Také ne ke všechnem slovum, kteráž mluví lidé, prikládej mysli své, ponevadž nemáš dbáti, by i služebník tvuj zlorecil tobe.
\par 22 Nebot ví srdce tvé, že jsi i ty castokrát zlorecil jiným.
\par 23 Všeho toho zkusil jsem moudrostí, a rekl jsem: Budu moudrým, ale moudrost vzdálila se ode mne.
\par 24 Což pak vzdálené a velmi hluboké jest, kdož to najíti muže?
\par 25 Všecko jsem prebehl myslí svou, abych poznal a vyhledal, i vynalezl moudrost a rozumnost, a abych poznal bezbožnost, bláznovství a nemoudrost i nesmyslnost.
\par 26 I našel jsem vec horcejší nad smrt, ženu, jejíž srdce tenata, a ruce její okovy. Kdož se líbí Bohu, zachován bývá od ní, ale hríšník bývá od ní jat.
\par 27 Pohled, to jsem shledal, (praví kazatel), jedno proti druhému staveje, abych nalezl umení,
\par 28 Ceho pak presto hledala duše má, však jsem nenalezl: Muže jednoho z tisíce našel jsem, ale ženy mezi tolika jsem nenalezl.
\par 29 Obzvláštne pohled i na to, což jsem nalezl: Že ucinil Buh cloveka dobrého, ale oni následovali smyšlínek rozlicných.
\par 30 Kdo se muže vrovnati moudrému, a kdo muže vykládati všelikou vec?

\chapter{8}

\par 1 Moudrost cloveka osvecuje oblícej jeho, a nestydatost tvári jeho promenuje.
\par 2 Ját radím: Výpovedi královské ostríhej, a však podlé prísahy Boží.
\par 3 Nepospíchej odjíti od tvári jeho, aniž trvej v zpoure; nebo cožt by koli chtel, ucinil by.
\par 4 Nebo kde slovo královské, tu i moc jeho, a kdo dí jemu: Co deláš?
\par 5 Kdo ostríhá prikázaní, nezví o nicem zlém. I cas i príciny zná srdce moudrého.
\par 6 Nebo všeliké predsevzetí má cas a príciny. Ale i to neštestí veliké cloveka se prídrží,
\par 7 Že neví, co budoucího jest. Nebo jak se stane, kdo mu oznámí?
\par 8 Není cloveka, kterýž by moci mel nad životem, aby zadržel duši, aniž má moc nade dnem smrti; nemá ani brane v tom boji, aniž vysvobodí bezbožnost bezbožného.
\par 9 Ve všem tom videl jsem, priloživ mysl svou ke všemu tomu, co se deje pod sluncem, že casem panuje clovek nad clovekem k jeho zlému.
\par 10 A tehdáž videl jsem bezbožné pohrbené, že se zase navrátili, ale kteríž z místa svatého odešli, v zapomenutí dáni jsou v meste tom, v kterémž dobre cinili. I to také jest marnost.
\par 11 Nebo že ne i hned ortel dochází pro skutek zlý, protož vroucí jest k tomu srdce synu lidských, aby cinili zlé veci.
\par 12 A ackoli hríšník ciní zle na stokrát, a vždy se mu odkládá, já však vím, že dobre bude bojícím se Boha, kteríž se bojí oblíceje jeho.
\par 13 S bezbožným pak nedobre se díti bude, aniž se prodlí dnové jeho; minou jako stín, proto že jest bez bázne pred oblícejem Božím.
\par 14 Jest marnost, kteráž se deje na zemi: Že bývají spravedliví, jimž se však vede, jako by cinili skutky bezbožných; zase bývají bezbožní, kterýmž se však vede, jako by cinili skutky spravedlivých. Protož jsem rekl: I to také jest marnost.
\par 15 A tak chválil jsem veselí, proto že nic nemá lepšího clovek pod sluncem, jediné aby jedl a pil, a veselil se, a že to pozustává jemu z práce jeho ve dnech života jeho, kteréž dal jemu Buh pod sluncem.
\par 16 A ac jsem se vydal srdcem svým na to, abych moudrost vystihnouti mohl, a vyrozumeti bíde, kteráž bývá na zemi, pro kterouž ani ve dne ani v noci nespí,
\par 17 A však videl jsem pri každém skutku Božím, že nemuže clovek vystihnouti skutku dejícího se pod sluncem. O cež pracuje clovek, vyhledati chteje, ale nenalézá; nýbrž byt i myslil moudrý, že se doví, nebude moci nic najíti.

\chapter{9}

\par 1 Všecko to zajisté rozvažoval jsem v srdci svém, abych vysvetlil všecko to, že spravedliví a moudrí, i skutkové jejich jsou v rukou Božích. Jakož milosti, tak ani nenávisti nezná clovek ze všech vecí, kteréž jsou pred oblícejem jeho.
\par 2 Všecko se deje jednostejne pri všech; jedna a táž prípadnost jest spravedlivého jako bezbožného, dobrého a cistého jako necistého, obetujícího jako toho, kterýž neobetuje, tak dobrého jako hríšníka, prisahajícího jako toho, kterýž se prísahy bojí.
\par 3 A tot jest prebídná vec mezi vším tím, což se deje pod sluncem, že prípadnost jednostejná jest všechnech, ovšem pak že srdce synu lidských plné jest zlého, a že bláznovství prídrží se srdce jejich, pokudž živi jsou, potom pak umírají,
\par 4 Ackoli ten, kterýž tovaryší se všechnemi živými, má nadeji, an psu živému lépe jest nežli lvu mrtvému.
\par 5 Nebo živí vedí, že umríti mají, mrtví pak nevedí nic, aniž více mají odplaty, proto že v zapomenutí prišla památka jejich.
\par 6 Anobrž i milování jejich, i nenávist jejich, i závist jejich zahynula, a již více nemají dílu na veky v žádné veci, kteráž se deje pod sluncem.
\par 7 Nuže tedy jez s radostí chléb svuj, a pí s veselou myslí víno své, nebo již oblibuje Buh skutky tvé.
\par 8 Každého casu at jest roucho tvé bílé, a oleje na hlave tvé necht není nedostatku.
\par 9 Živ bud s manželkou, kterouž jsi zamiloval, po všecky dny života marnosti své, kterýžt dán pod sluncem po všecky dny marnosti tvé; nebo to jest podíl tvuj v živote tomto a pri práci tvé, kterouž vedeš pod sluncem.
\par 10 Všecko, což by pred se vzala ruka tvá k cinení, podlé možnosti své konej; nebo není práce ani dumyslu ani umení ani moudrosti v hrobe, do nehož se béreš.
\par 11 A obrátiv se, spatril jsem pod sluncem, že nezáleží beh na rychlých, ani boj na udatných, nýbrž ani živnost na moudrých, ani bohatství na opatrných, ani prízen na umelých, ale podlé casu a príhody prihází se všechnem.
\par 12 Nebo tak nezná clovek casu svého jako ryby, kteréž loveny bývají sítí škodlivou, a jako ptáci polapeni bývají osídlem; tak zlapáni bývají synové lidští v cas zlý, když na ne pripadá v náhle.
\par 13 Také i tuto moudrost videl jsem pod sluncem, kteráž za velikou byla u mne:
\par 14 Bylo mesto malé, a v nem lidí málo, k nemuž pritáhl král mocný, a obehnav je, zdelal proti nemu náspy veliké.
\par 15 I nalezen jest v nem muž chudý moudrý, kterýž vysvobodil to mesto moudrostí svou, ackoli žádný nevzpomenul na muže toho chudého.
\par 16 Protož rekl jsem já: Lepší jest moudrost než síla, ackoli moudrost chudého toho byla v pohrdání, a slov jeho neposlouchali.
\par 17 Slov moudrých pokojne poslouchati sluší, radeji než kriku panujícího mezi blázny.
\par 18 Lepší jest moudrost než nástrojové válecní, ale nemoudrý jeden kazí mnoho dobrého.

\chapter{10}

\par 1 Muchy mrtvé nasmrazují a nakažují mast apatekárskou; tak pro moudrost a slávu vzácného malicko bláznovství zohyžduje.
\par 2 Srdce moudrého jest po pravici jeho, ale srdce blázna po levici jeho.
\par 3 I tehdáž, když blázen cestou jde, srdce jeho nedostatek trpí; nebo všechnem znáti dává, že blázen jest.
\par 4 Jestliže by duch toho, jenž panuje, povstal proti tobe, neopouštej místa svého; nebo krotkost prítrž ciní hríchum velikým.
\par 5 Jest zlá vec, kterouž jsem videl pod sluncem, totiž neprozretelnost, kteráž pochází od vrchnosti,
\par 6 Že blázen postaven bývá v dustojnosti veliké, a bohatí že v nízkosti sedávají.
\par 7 Videl jsem služebníky na koních, knížata pak, ana chodí pešky jako služebníci.
\par 8 Kdo kopá jámu, upadá do ní; a kdo borí plot, uštkne jej had.
\par 9 Kdo prenáší kamení, urazí se jím; a kdo štípá dríví, nebezpecenství bude míti od neho.
\par 10 Jestliže se ztupí železo, a nenabrousí-li ostrí jeho, tedy síly priciniti musí; ale mnohem lépe muže to spraviti moudrost.
\par 11 Uštkne-li had, než by zaklet byl, nic neprospejí slova zaklinace.
\par 12 Slova úst moudrého jsou príjemná, ale rtové blázna sehlcují jej.
\par 13 Pocátek slov úst jeho jest nemoudrost, a ostatek mluvení jeho pouhé bláznovství.
\par 14 Nebo blázen mnoho mluví, ješto neví clovek ten, co budoucího jest. To zajisté, co bude po nem, kdo mu oznámí?
\par 15 Práce bláznu k ustání je privodí, nebo neumí ani do mesta trefiti.
\par 16 Beda tobe, zeme, když král tvuj díte jest, a knížata tvá ráno hodují.
\par 17 Blahoslavená jsi ty zeme, když král tvuj jest syn šlechetných, a knížata tvá, když cas jest, jídají pro posilnení, a ne pro opilství.
\par 18 Ano pro lenost schází krov, a pro opuštení rukou kapává do domu.
\par 19 Pro obveselení strojívají hody, a víno obveseluje život, peníze pak ke všemu dopomáhají.
\par 20 Ani sám u sebe králi nezlorec, ani v skrýších pokoje svého nezlorec mocnejšímu; nebo pták nebeský donesl by hlas ten, a to, což krídla má, vyjevilo by rec tvou.

\chapter{11}

\par 1 Pouštej chléb svuj po vode, nebo po mnohých dnech najdeš jej.
\par 2 Dej cástku sedmi aneb i osmi, nebo nevíš, co zlého bude na zemi.
\par 3 Když se naplnují oblakové, déšt na zem vydávají; a když padá drevo na poledne aneb na pulnoci, na kteréž místo padá to drevo, tu zustává.
\par 4 Kdo šetrí vetru, nebude síti; a kdo hledí na husté oblaky, nebude žíti.
\par 5 Jakož ty nevíš, která jest cesta vetru, a jak rostou kosti v živote tehotné, tak neznáš díla Božího, kterýž ciní všecko.
\par 6 Hned z jitra rozsívej síme své, a u vecer nedávej odpocinutí ruce své; nebo ty nevíš, co jest lepšího, to-li ci ono, cili obé jednostejne dobré jest.
\par 7 Príjemnét jest zajisté svetlo, a milá vec ocima videti slunce;
\par 8 A však, by mnoho let živ jsa clovek, ve všech tech veselil by se, tedy rozpomenul-li by se na dny temnosti, jak jich mnoho bude, cožkoli prebehlo, pocte za marnost.
\par 9 Radujž se tedy, mládence, v mladosti své, a necht te obveseluje srdce tvé ve dnech mladosti tvé, a chod po cestách srdce svého, a podlé žádosti ocí svých, než vez, že te s tím se vším privede Buh na soud.
\par 10 Anobrž odejmi hnev od srdce svého, a odvrat zlost od tela svého; nebo detinství a mladost jest marnost.

\chapter{12}

\par 1 A pamatuj na stvoritele svého ve dnech mladosti své, prvé než nastanou dnové zlí, a priblíží se léta, o nichž díš: Nemám v nich zalíbení;
\par 2 Prvé než se zatmí slunce a svetlo, a mesíc i hvezdy, a navrátí se hustí oblakové po dešti;
\par 3 V ten den, v kterémž se trísti budou strážní domu, a nakriví se muži silní, a ustanou melící, proto že jich málo bude, a zatmí se ti, kteríž vyhlédají z oken,
\par 4 A zavríny budou dvére od ulice s slabým zvukem mlení, a povstanou k hlasu ptacímu, a prestanou všecky slibnosti zpevu;
\par 5 Ano i vysokosti báti se budou, a úrazu na ceste, a kvésti bude mandlový strom, tak že i kobylka težká bude, a poruší se žádost; nebo bére se clovek do domu vecného, a choditi budou po ulici kvílící;
\par 6 Prvé než se pretrhne provaz stríbrný, a než se rozrazí cíše zlatá, a roztríští se vederce nad vrchovištem, a roztrhne se kolo nad studnicí,
\par 7 A navrátí se prach do zeme, jakž prvé byl, duch pak navrátí se k Bohu, kterýž jej dal.
\par 8 Marnost nad marnostmi, rekl kazatel, a všecko marnost.
\par 9 Cím pak byl kazatel moudrejší, tím více vyucoval lid umení, a rozvažoval, zpytoval, i složil množství prísloví.
\par 10 Snažovalte se kazatel vyhledati veci nejžádostivejší, a napsal, což pravého jest, a slova verná.
\par 11 Slova moudrých podobná ostnum a hrebíkum vbitým, slova skladatelu, kteráž jsou vydána od pastýre jednoho.
\par 12 A tak tedy jimi, synu muj, hojne dosti osvícen býti mužeš. Delání knih mnohých žádného konce není, a císti mnoho jest zemdlení tela.
\par 13 Summa všeho, což jsi slyšel: Boha se boj, a prikázaní jeho ostríhej, nebo na tom všecko cloveku záleží.
\par 14 Ponevadž všeliký skutek Buh privede na soud, i každou vec tajnou, budto dobrou, budto zlou.

\end{document}