\begin{document}

\title{Isaiah}

\chapter{1}

\par 1 Videní Izaiáše syna Amosova, kteréž videl o Judovi a Jeruzalému, za dnu Uziáše, Jotama, Achasa a Ezechiáše, králu Judských.
\par 2 Slyšte nebesa, a ušima pozoruj zeme, nebo Hospodin mluví: Syny jsem vychoval a vyvýšil, oni pak strhli se mne.
\par 3 Vul zná hospodáre svého, a osel jesle pánu svých; Izrael nezná, lid muj nesrozumívá.
\par 4 Ach, národe hríšný, lide obtížený nepravostí, síme zlostníku, synové nešlechetní, opustili Hospodina, pohrdli svatým Izraelským, odvrátili se zpet.
\par 5 Proc, cím více biti býváte, tím více se odvracujete? Všecka hlava jest neduživá, a všecko srdce zemdlené.
\par 6 Od zpodku nohy až do vrchu hlavy není na nem místa celého, jen rána a zsinalost, a zbití zahnojené, aniž se vytlacuje, ani uvazuje, ani olejem zmekcuje.
\par 7 Zeme vaše spustla, mesta vaše vypálena ohnem; zemi vaši pred vámi cizozemci zžírají, a v poušt obracejí, tak jakž vše kazí cizozemci.
\par 8 I zustala dcera Sionská jako boudka na vinici, jako chaloupka v zahrade tykevné, a jako mesto zkažené.
\par 9 Byt nám byl Hospodin zástupu jakkoli malicka ostatku nezanechal, byli bychom jako Sodoma, byli bychom Gomore podobni.
\par 10 Slyšte slovo Hospodinovo, knížata Sodomská, ušima pozorujte zákona Boha našeho, lide Gomorský:
\par 11 K cemu jest mi množství obetí vašich? dí Hospodin. Syt jsem zápalných obetí skopcu a tuku krmných hovad, a krve volku a beránku a kozlu nejsem žádostiv.
\par 12 Že pricházíte, abyste se ukazovali prede mnou, kdož toho z ruky vaší hledal, abyste šlapali síne mé?
\par 13 Neprinášejte více obeti oklamání. Kadení v ohavnosti mám, novmesícu a sobot a svolávání nemohu trpeti, (nepravost jest), ani shromáždení.
\par 14 Novmesícu vašich a slavností vašich nenávidí duše má; jsou mi bremenem, ustal jsem, nesa je.
\par 15 Protož, když rozprostíráte ruce vaše, skrývám oci své pred vámi, a když množíte modlitbu, neslyším; ruce vaše krve plné jsou.
\par 16 Umejte se, ocistte se, odvrzte zlost skutku vašich od ocí mých, prestante zle ciniti.
\par 17 Ucte se dobre ciniti, hledejte soudu, pozdvihnete potlaceného, dopomozte k spravedlnosti sirotku, zastante vdovy.
\par 18 Podtež nu, a poukažme sobe, praví Hospodin: Budou-li hríchové vaši jako cervec dvakrát barvený, jako sníh zbelejí; budou-li cervení jako šarlat, jako vlna budou.
\par 19 Budete-li povolní a poslušní, dobré veci zeme jísti budete.
\par 20 Pakli nebudete povolní, ale zpurní, od mece sežráni budete; nebo ústa Hospodinova mluvila.
\par 21 Jak te nevestkou ucineno to mesto verné, plné soudu! Spravedlnost prebývala v nem, nyní pak vražedlníci.
\par 22 Stríbro tvé obrátilo se v trusky, víno tvé smíšeno s vodou.
\par 23 Knížata tvá zpurná a tovaryši zlodeju, jeden každý z nich miluje dary, a dychtí po úplatcích; sirotku k spravedlnosti nedopomáhají, a pre vdovy pred ne neprichází.
\par 24 Protož dí Pán, Hospodin zástupu, silný Izraelský: Aj, ját se poteším nad protivníky svými, a vymstím se nad neprátely svými,
\par 25 Když zase obrátím ruku svou na te, až prepálím docista trusky tvé, a odejmu všecken cín tvuj,
\par 26 A obnovím soudce tvé tak jako na pocátku, a rádce tvé jako s prvu. A tu potom slouti budeš mestem spravedlnosti, mestem verným.
\par 27 Sion v soudu vykoupen bude, a kteríž zase uvedeni budou do neho, v spravedlnosti.
\par 28 Setrení pak prestupníku a nešlechetných v náhle prijde, a kteríž opouštejí Hospodina, docela zahynou.
\par 29 Nebo zahanbeni budete pro háje, po nichž jste toužili, a zastydíte se pro zahrady, kteréž jste sobe zvolili.
\par 30 Budete zajisté jako dub, s nehož lístí prší, a jako zahrada, v níž vody není.
\par 31 I bude nejsilnejší jako koudel, a ucinitel jeho jako jiskra; i bude to obé horeti spolu, a nebude žádného, ješto by uhasiti mohl.

\chapter{2}

\par 1 Slovo, kteréž videl Izaiáš syn Amosuv o Judovi a Jeruzalému.
\par 2 I stane se v posledních dnech, že utvrzena bude hora domu Hospodinova na vrchu hor, a vyvýšena nad pahrbky, i pohrnou se k ní všickni národové.
\par 3 A pujdou lidé mnozí, ríkajíce: Podte, a vstupme na horu Hospodinovu, do domu Boha Jákobova, a bude nás vyucovati cestám svým, i budeme choditi po stezkách jeho. Nebo z Siona vyjde zákon, a slovo Hospodinovo z Jeruzaléma.
\par 4 Ont bude souditi mezi národy, a trestati bude lidi mnohé. I zkují mece své v motyky, a oštípy své v srpy. Nepozdvihne národ proti národu mece, a nebudou se více uciti boji.
\par 5 Dome Jákobuv, podtež, a chodme v svetle Hospodinove.
\par 6 Ale ty jsi opustil lid svuj, dum Jákobuv, proto že jsou naplneni ohavnostmi národu východních, a jsou poverní jako Filistinští, a smyšlínky cizozemcu že sobe libují.
\par 7 K tomu naplnena jest zeme jejich stríbrem a zlatem, tak že není konce pokladum jejich; naplnena jest zeme jejich i konmi, vozum pak jejich poctu není.
\par 8 Naplnena jest také zeme jejich modlami; klanejí se dílu rukou svých, kteréž ucinili prstové jejich.
\par 9 Klaní se obecný clovek, a ponižuje se i prední muž; protož neodpouštej jim.
\par 10 Vejdi do skály, a skrej se v prachu pred hruzou Hospodinovou, pred slávou dustojnosti jeho.
\par 11 Tut oci vysoké cloveka sníženy budou, a sklonena bude vysokost lidská, ale Hospodin sám vyvýšen bude v ten den.
\par 12 Nebo den Hospodina zástupu se blíží na každého pyšného a zpínajícího se, i na každého vyvýšeného, i bude ponížen,
\par 13 I na všecky cedry Libánské vysoké a vyvýšené, i na všecky duby Bázanské,
\par 14 I na všecky hory vysoké, i na všecky pahrbky vyvýšené,
\par 15 I na všelikou veži vysokou, i na všelikou zed pevnou,
\par 16 I na všecky lodí morské, i na všecka malování slibná.
\par 17 A sehnuta bude pýcha cloveka, a snížena bude vysokost lidská, ale vyvýšen bude Hospodin sám v ten den,
\par 18 Modly pak docela vymizejí.
\par 19 Tehdy pujdou do jeskyní skal a do roklí zeme, pred hruzou Hospodinovou a slávou dustojnosti jeho, když povstane, aby potrel zemi.
\par 20 V ten den zavrže clovek modly své stríbrné a modly své zlaté, kterýchž mu nadelali, aby se klanel, totiž krtum a netopýrum.
\par 21 I vejde do slují skal a do vysedlin jejich pred hruzou Hospodinovou, a pred slávou dustojnosti jeho, když povstane, aby potrel zemi.
\par 22 Prestantež doufati v cloveku, jehož dýchání v chrípích jeho jest. Nebo zac má jmín býti?

\chapter{3}

\par 1 Nebo aj, Pán, Hospodin zástupu, odejme od Jeruzaléma a Judy hul a podporu, všelijakou hul chleba a všelikou podporu vody,
\par 2 Silného i muže válecného, soudce i proroka, mudrce i starce,
\par 3 Padesátníka i pocestného, i rádci, i vtipného remeslníka, i výmluvného,
\par 4 A dám jim deti za knížata; deti, pravím, panovati budou nad nimi.
\par 5 I bude ssužovati v lidu jeden druhého, a bližní bližního svého; zpurne se postaví díte proti starci, a chaterný proti vzácnému.
\par 6 Procež se chopí jeden každý bratra svého z domu otce svého, a dí: Máš odev, knížetem naším budeš, a pád tento zdrž rukou svou.
\par 7 Ale on prisáhne v ten den, rka: Nebudut vázati tech ran, nebo v dome mém není chleba ani odevu; neustanovujtež mne knížetem lidu.
\par 8 Nebo se oboril Jeruzalém, a Juda padl, proto že jazyk jejich a skutkové jejich jsou proti Hospodinu, k dráždení ocí slávy jeho.
\par 9 Nestydatost tvári jejich svedcí proti nim; hrích zajisté svuj jako Sodomští ohlašují, a netají. Beda duši jejich, nebo sami na sebe uvodí zlé.
\par 10 Rcete spravedlivému: Dobre bude; nebo ovoce skutku svých jísti bude.
\par 11 Ale beda bezbožnému, zle bude; nebo odplata rukou jeho dána jemu bude.
\par 12 Lidu mého knížata jsou deti, a ženy panují nad ním. Lide muj, kteríž te vodí, svodí te, a cestu stezek tvých ukrývají.
\par 13 Stojít Hospodin k rozsudku, stojí, pravím, k rozsudku s lidmi.
\par 14 Hospodin k soudu prijde proti starším lidu svého a knížatum jejich, a dí: Vy jste pohubili vinici mou, loupež chudého jest v domích vašich.
\par 15 Proc vy nuzíte lid muj, a tváre chudých zahanbujete? praví Pán, Hospodin zástupu.
\par 16 I dí Hospodin: Proto že se pozdvihují dcery Sionské, a chodí s vytaženým krkem, a pasou ocima, protulujíce se, a zdrobna krácejíce, i nohama svýma lákají,
\par 17 Protož okydne Pán prašivinou vrch hlavy dcer Sionských, a Hospodin hanbu jejich obnaží.
\par 18 V ten den odejme Pán okrasu tech nástrah, totiž paucníky a halže,
\par 19 Jablka zlatá a spinadla a cepce,
\par 20 Biréty a zápony, tkanice, punty a náušnice,
\par 21 Prsteny a nácelníky,
\par 22 Promenná roucha, i pláštky, i roušky, i vacky,
\par 23 I zrcadla, i cechlíky, i vence, i šlojíre.
\par 24 A budet místo vonných vecí hnis, a místo pasu roztržení, a místo strojení kaderí lysina, a místo širokého podolku bude prepásání pytlem, obhorení pak místo krásy.
\par 25 Muži tvoji od mece padnou, a silní tvoji v boji.
\par 26 I budou plakati a kvíliti brány jeho, a spustlý na zemi sedeti bude.

\chapter{4}

\par 1 I chopí se sedm žen muže jednoho v ten den, a reknou: Chléb svuj jísti budeme, a rouchem svým se odívati, toliko at po tobe se jmenujeme; odejmi pohanení naše.
\par 2 V ten den bude výstrelek Hospodinuv ušlechtilý a slavný, a plod zeme výtecný a krásný, totiž ti, kteríž zachováni budou z Izraele.
\par 3 I stane se, že kdož bude zanechán na Sionu, a pozustaven bude v Jeruzaléme, svatý slouti bude, každý, kdož jest zapsán k životu v Jeruzaléme,
\par 4 Když Pán smyje necistotu dcer Sionských, a krev Jeruzaléma vyplákne z neho v duchu soudu a v duchu horlivosti.
\par 5 A stvorí Hospodin nad každým obydlím hory Sion, a nad každým shromáždením jejím oblak ve dne, a dým a blesk plápolajícího ohne v noci; nebo nad všecku slávu bude zastrení.
\par 6 A bude stánkem k zastenování ve dne pred horkem, a za útocište a skrýši pred prívalem a deštem.

\chapter{5}

\par 1 Zpívati již budu milému svému písen milého svého o vinici jeho: Vinici má milý muj na vrchu úrodném.
\par 2 Kterouž ohradil, a kamení z ní vybral, a vysadil ji vinným kmenem výborným, a ustavel veži u prostred ní, také i pres udelal v ní, a ocekával, aby nesla hrozny, ale vydala plané víno.
\par 3 Nyní tedy, obyvatelé Jeruzalémští a muži Judští, sudte medle mezi mnou a vinicí mou.
\par 4 Což ješte cineno býti melo vinici mé, ješto bych jí neucinil? Proc, když jsem ocekával, aby nesla hrozny, plodila plané víno?
\par 5 A protož oznámím vám, co já uciním vinici své: Odejmu plot její, a prijde na spuštení; rozborím hradbu její, i prijde na pošlapání.
\par 6 Zapustím ji, nebude rezána, ani kopána, i vzroste na ní bodlácí a trní; oblakum také zapovím, aby nevydávali více na ni dešte.
\par 7 Vinice zajisté Hospodina zástupu dum Izraelský jest, a muži Judští révové milí jemu. Ocekával pak soudu, a aj, nátisk; spravedlnosti, a aj, krik.
\par 8 Beda vám, kteríž pripojujete dum k domu, a pole s polem spojujete, tak že místa jiným není, jako byste sami rozsazeni byli k prebývání u prostred zeme.
\par 9 V uši mé rekl Hospodin zástupu: Jiste žet domové mnozí zpustnou, velicí a krásní bez obyvatele budou.
\par 10 Nadto deset dílcu vinicných vydá láhvici jednu, a semena chomer vydá efi.
\par 11 Beda tem, kteríž ráno vstávajíce, chodí po opilství, a trvají pri tom do vecera, až je víno i rozpaluje.
\par 12 A harfa, loutna a buben, a píštalka,a víno bývá na hodech jejich; na skutky pak Hospodinovy nehledí, a díla rukou jeho nespatrují.
\par 13 Protož v zajetí pujde lid muj, nebo jest bez umení; a slavní jeho budou hladovití, a množství jeho žízní usvadne.
\par 14 Procež rozšírilo i peklo hrdlo své, a rozedrelo nad míru ústa svá, i sstoupí do neho slavní její a množství její, i hluk její, i ti, kteríž se veselí v ní.
\par 15 A tak sehnut bude clovek, a ponížen muž, a oci pyšných sníženy budou,
\par 16 Hospodin pak zástupu vyvýšen bude v soudu, a Buh silný a svatý ukáže se svatý v spravedlnosti.
\par 17 I pásti se budou beránkové podlé obyceje svého, a ostatky tech tucných, navrátíce se, jísti budou.
\par 18 Beda tem, kteríž táhnou nepravost za provazy marnosti, a jako provazem u vozu hrích;
\par 19 Kteríž ríkají: Nechat pospíší, a nemešká dílem svým, abychom videli, a nechat se priblíží, a prijde rada toho Svatého Izraelského, abychom zvedeli.
\par 20 Beda tem, kteríž ríkají zlému dobré, a dobrému zlé, kladouce tmu za svetlo, a svetlo za tmu, pokládajíce horké za sladké, a sladké za horké.
\par 21 Beda tem, kteríž jsou moudrí sami u sebe, a vedlé zdání svého opatrní.
\par 22 Beda tem, kteríž jsou silní ku pití vína, a muži udatní k smíšení nápoje opojného;
\par 23 Kteríž ospravedlnují bezbožného pro dary, spravedlnost pak spravedlivých odjímají od nich.
\par 24 Z té príciny, jakož plamen ohne zžírá strnište, a plevy plamen v nic obrací, tak koren jejich bude jako shnilina, a kvet jejich jako prach vzejde; nebo zavrhli zákon Hospodina zástupu, a recí svatého Izraelského pohrdli.
\par 25 Procež rozpáliv se prchlivostí Hospodin na lid svuj, a vztáh ruku svou na nej, porazil jej, tak že se hory zatrásly, a tela mrtvá jejich jako hnuj u prostred ulic. V tom však ve všem neodvrátila se prchlivost jeho, ale predce ruka jeho jest vztažená.
\par 26 Nebo vyzdvihne korouhev národu dalekému, a zahvízdne nan od koncin zeme, a aj, rychle a prudce prijde.
\par 27 Žádného ustalého ani klesajícího nebude mezi nimi; nebude drímati ani spáti, aniž se rozepne pás bedr jeho, aniž se strhá remen obuví jeho.
\par 28 Strely jeho ostré, a všecka lucište jeho natažená; kopyta konu jeho jako škremen souzena budou, a kola jeho jako vichrice.
\par 29 Rvání jeho jako rvání lva, a rváti bude jako lvícata; a mumlati bude, a pochytí loupež a utece, aniž bude, kdo by vydrel.
\par 30 A zvuceti bude nad ním v ten den, jako zvucí more. Tehdy pohledíme na zemi, a aj, mrákota a úzkost; nebo se i svetlo zatmí pri pohubení jejím.

\chapter{6}

\par 1 Léta, kteréhož umrel král Uziáš, videl jsem Pána sedícího na trunu vysokém a vyzdviženém, a podolek jeho naplnoval chrám.
\par 2 Serafínové stáli nad ním. Šest krídel mel každý z nich: dvema zakrýval tvár svou, a dvema prikrýval nohy své, a dvema létal.
\par 3 A volal jeden k druhému, ríkaje: Svatý, svatý, svatý Hospodin zástupu, plná jest všecka zeme slávy jeho.
\par 4 A pohnuly se podvoje verejí od hlasu volajícího, a dum plný byl dymu.
\par 5 I rekl jsem: Beda mne, jižt zahynu, proto že jsem clovek poškvrnené rty maje, k tomu u prostred lidu rty poškvrnené majícího bydlím, a že krále Hospodina zástupu videly oci mé.
\par 6 I priletel ke mne jeden z serafínu, maje v ruce své uhel reravý, kleštemi vzatý z oltáre,
\par 7 A dotekl se úst mých, a rekl: Aj hle,dotekl se uhel tento úst tvých; nebo odešla nepravost tvá, a hrích tvuj shlazen jest.
\par 8 Potom slyšel jsem hlas Pána rkoucího: Koho pošli? A kdo nám pujde? I rekl jsem: Aj já, pošli mne.
\par 9 On pak rekl: Jdi, a rci lidu tomu: Slyšte slyšíce, a nerozumejte, a hledte hledíce, a nepoznávejte.
\par 10 Zatvrd srdce lidu toho, a uši jeho zacpej, a oci jeho zavri, aby nevidel ocima svýma, a ušima svýma neslyšel, a srdcem svým nerozumel, a neobrátil se, a nebyl uzdraven.
\par 11 A když jsem rekl: Až dokud, Pane? i odpovedel: Dokudž nezpustnou mesta, tak aby nebylo žádného obyvatele, a domové, aby nebylo v nich žádného cloveka, a zeme docela nezpustne,
\par 12 A nevzdálí Hospodin všelikého cloveka, a nebude dokonalého zpuštení u prostred zeme;
\par 13 Dokudž ješte v ní nebude desateré zhouby, a teprv zkažena bude. Ale jakož ono jilmoví, a jako doubí onoho náspu podporou jest, tak síme svaté jest podpora její.

\chapter{7}

\par 1 I stalo se za dnu Achasa syna Jotamova, syna Uziáše, krále Judského, že pritáhl Rezin král Syrský, a Pekach syn Romeliáše, krále Izraelského, k Jeruzalému, aby bojoval proti nemu, ale nemohl ho dobyti.
\par 2 I oznámeno jest domu Davidovu v tato slova: Spikla se zeme Syrská s Efraimem. Procež pohnulo se srdce jeho, i srdce lidu jeho, tak jako se pohybuje dríví v lese od vetru.
\par 3 Tedy rekl Hospodin Izaiášovi: Vyjdi nyní vstríc Achasovi, ty a Sear Jašub syn tvuj, až na konec struhy rybníka horejšího, k silnici pole valchárova,
\par 4 A díš jemu: Šetr se, abys se nekormoutil. Neboj se, a srdce tvé nechat se nedesí dvou ostatku hlavní tech kourících se, pred rozpáleným hnevem Rezinovým s Syrskými, a syna Romeliášova,
\par 5 Proto že zlou radu složili proti tobe Syrský, Efraim a syn Romeliášuv, rkouce:
\par 6 Táhneme proti zemi Judské, a vyležme ji, a odtrhneme ji k sobe, a ustavme krále u prostred ní syna Tabealova.
\par 7 Toto praví Panovník Hospodin: Nestanet se a nebude toho.
\par 8 Nebo hlava Syrské zeme jest Damašek, a hlava Damašku Rezin, a po šedesáti peti letech potrín bude Efraim, tak že nebude lidem.
\par 9 Mezi tím hlava Efraimova Samarí, a hlava Samarí syn Romeliášuv. Jestliže neveríte, jiste že neostojíte.
\par 10 I mluvil ješte Hospodin k Achasovi, rka:
\par 11 Požádej sobe znamení od Hospodina Boha svého, bud dole hluboko, aneb na hore vysoko.
\par 12 I rekl Achas: Nebudu prositi, aniž budu pokoušeti Hospodina.
\par 13 Tedy rekl prorok: Slyšte nyní, dome Daviduv, ješte-liž jest vám málo, lidem býti k obtížení, že i Bohu mému k obtížení jste?
\par 14 Protož sám Pán dá vám znamení: Aj, panna pocne, a porodí syna, a nazuve jméno jeho Immanuel.
\par 15 Máslo a med jísti bude, až by umel zavrci zlé, a vyvoliti dobré.
\par 16 Nýbrž prvé než bude umeti díte to zavrci zlé a vyvoliti dobré, opuštena bude zeme, kteréž nenávidíš pro dva krále její.
\par 17 Na tebe pak privede Hospodin, a na lid tvuj, i na dum otce tvého dny, jakýchž nebylo ode dne, v nemž odstoupil Efraim od Judy, a to skrze krále Assyrského.
\par 18 Nebo stane se v ten den, že pošepce Hospodin muchám, kteréž jsou pri nejdalších rekách Egyptských, a vcelám, kteréž jsou v zemi Assyrské.
\par 19 I prijdou, a usadí se všickni ti v údolích pustých a v derách skalních, i na všech chrastinách i na všech stromích užitecných.
\par 20 V ten den oholí Pán tou britvou najatou, (skrze ty, kteríž za rekou jsou, skrze krále Assyrského), hlavu a vlasy noh, ano také i bradu do cista sholí.
\par 21 I bude tehdáž, že sotva clovek zachová kravicku neb dve ovce,
\par 22 Avšak pro množství mléka, kteréhož nadojí, jísti bude máslo. Máslo zajisté a med jísti bude, kdožkoli v zemi bude zanechán.
\par 23 Bude také v ten cas, že každé místo, na nemž jest tisíc kmenu vinných za tisíc stríbrných, trním a hložím poroste.
\par 24 S strelami a lucištem tudy jíti musí; hložím zajisté a trním zaroste všecka zeme.
\par 25 Všecky pak hory, kteréž motykou kopány býti mohou, nebudou se báti hloží a trní; nebo budou za pastvište volum, a bravum ku pošlapání.

\chapter{8}

\par 1 I rekl mi Hospodin: Vezmi sobe knihu velikou, a napiš na ní písmem lidským: K rychlé koristi pospíchá loupežník.
\par 2 I vzal jsem sobe za svedky verné Uriáše kneze, a Zachariáše syna Jeberechiášova.
\par 3 V tom pristoupil jsem k prorokyni, kteráž pocala, a porodila syna. I rekl mi Hospodin: Dej mu jméno: K rychlé koristi pospíchá loupežník.
\par 4 Nebo prvé než bude umeti díte to volati: Otce muj, matko má, odejme zboží Damašské a loupeže Samarské lid krále Assyrského.
\par 5 I to ješte mluvil Hospodin ke mne, rka:
\par 6 Ponevadž pohrdl lid ten vodami Siloe tiše tekoucími, raduje se z Rezina a syna Romeliášova,
\par 7 Protož aj, Pán uvede na ne vody, reky násilné a mnohé, totiž krále Assyrského, a všecku slávu jeho, tak že vystoupí ze všech toku svých, a pujde nad všecky brehy své.
\par 8 Pujde i pres Judu, rozleje se a rozejde, až k hrdlu dosáhne, a roztažená krídla jeho naplní širokost zeme tvé, ó Immanueli.
\par 9 Puntujtež se lidé, však potríni budete, (nýbrž pozorujte všickni v daleké zemi), prepašte se, však potríni budete, prepašte se, však potríni budete.
\par 10 Vejdete v radu, a zrušena bude, mluvte slovo, a neostojít; nebo s námi jest Buh silný.
\par 11 Tak zajisté mluvil Hospodin ke mne, ujav mne za ruku, a dav mi výstrahu, abych nechodil cestou lidu tohoto, rka:
\par 12 Neríkejte: Spuntování, když lid ten praví: Spuntování; aniž se jako oni strachujte, nerci-li, abyste se desiti meli.
\par 13 Hospodina zástupu samého posvecujte; on budiž bázen vaše i strach váš.
\par 14 A budet vám i svatyní, kamenem pak urážky a skalou pádu obema domum Izraelským, osídlem a léckou i obyvatelum Jeruzalémským.
\par 15 I urazí se o to mnozí, a padnou, a potríni budou, aneb zapletouce se, popadeni budou.
\par 16 Zavaž osvedcení, zapecet zákon mezi ucedlníky mými.
\par 17 Procež ocekávati budu na Hospodina, kterýž skryl tvár svou od domu Jákobova; na nej, pravím, cekati budu.
\par 18 Aj, já a dítky, kteréž mi dal Hospodin, na znamení a zázraky v Izraeli od Hospodina zástupu, kterýž prebývá na hore Sion.
\par 19 Jestliže by vám pak rekli: Dotazujte se na hadacích a veštcích, kteríž šepcí a šveholí, rcete: Nemá-liž se lid na Bohu svém dotazovati? K mrtvým-liž místo živých má se utíkati?
\par 20 K zákonu a svedectví! Pakli nechtí, nechat mluví vedlé slova toho, v nemž není žádné záre,
\par 21 Až by každý toulati se musil, zbedovaný jsa a hladovitý. I stane se, že se bude, hladovitý jsa, sám v sobe zlobiti, a zloreciti králi svému a Bohu svému, bud že zhuru pohledí,
\par 22 Bud že na zemi popatrí, a aj, všudy ssoužení a tma, mrákota, bída i nátisk v temnostech.

\chapter{9}

\par 1 A však ne tak obklící mrákota té zeme, kteráž ssoužena bude, jako když se ponejprv neprítel dotkl zeme Zabulon a zeme Neftalím, ani jako potom, když více obtíží, naproti mori, pri Jordánu Galilei lidnou.
\par 2 Nebo lid tento chode v temnostech, uzrí svetlo veliké, a bydlícím v zemi stínu smrti svetlo zastkví se.
\par 3 Rozmnožil jsi tento národ, ale nezvelicils veselé. A však veseliti se budou pred tebou, tak jako se veselí ve žni, jako se radují, když delí koristi,
\par 4 Když jho bremene jeho a prut ramene jeho, hul násilníka jeho polámeš, jako za dnu Madianských,
\par 5 Kdyžto všickni bojovníci predešeni, a roucha ve krvi zbrocena, ano což horeti mohlo, i ohnem spáleno.
\par 6 Nebo díte narodilo se nám, syn dán jest nám, i bude knížetství na rameni jeho, a nazváno bude jméno jeho: Predivný, Rádce, Buh silný, Rek udatný, Otec vecnosti, Kníže pokoje.
\par 7 K rozmnožování pak toho knížetství a pokoje, jemuž nebude konce, sedne na stolici Davidove, a na království jeho, až je i v rád uvede, a utvrdí v soudu a v spravedlnosti, od tohoto casu až na veky. Horlivost Hospodina zástupu to uciní.
\par 8 Slovo poslal Pán Jákobovi, a padlo v Izraeli.
\par 9 A zvít všecken lid, Efraim, i obyvatelé Samarští, kteríž v pýše a vysokomyslnosti srdce ríkají:
\par 10 Padly cihly, ale my tesaným kamenem staveti budeme; planí fíkové podtati jsou, a my to v cedry smeníme.
\par 11 Ale zvýší Hospodin protivníky Rezinovy nad nej, a neprátely jeho svolá,
\par 12 Syrské po predu, a Filistinské po zadu. I budou žráti Izraele celými ústy, aniž ve všem tom odvrátí se prchlivost jeho, ale ruka jeho predce bude vztažená.
\par 13 Protože se lid ten nenavrací k tomu, kterýž jej bije, a Hospodina zástupu nehledají,
\par 14 Protož odetne Hospodin od Izraele hlavu i ocas, ratolest i sítí jednoho dne.
\par 15 (Starec a vzácný clovek, ont jest hlava, prorok pak, kterýž ucí lži, ont jest ocas.)
\par 16 Nebo vudcové lidu tohoto jsou svudcové, a kteríž se jim vésti dadí, zhynuli.
\par 17 Protož z mládencu jeho nepoteší se Pán, a nad sirotky a vdovami jeho neslituje se; nebo všickni jsou pokrytci a zlocinci, a každá ústa mluví nešlechetnost. Aniž ve všem tom odvrátí se prchlivost jeho, ale predce ruka jeho bude vztažená.
\par 18 Nebo roznícena jsouc jako ohen bezbožnost, bodlácí a trní pálí, potom zapálí i houšte lesu; procež rozptýleni budou jako dým u povetrí.
\par 19 Pro hnev Hospodina zástupu zatmí se zeme, a ten lid bude jako pokrm ohne. Žádný ani bratra svého šanovati nebude,
\par 20 Ale krájeje sobe po pravé strane, však lacneti bude, a zžíraje po levé, však nenasytí se. Jeden každý maso ramene svého žráti bude,
\par 21 Manasses Efraima a Efraim Manessesa, oba pak spolu proti Judovi budou. Ve všem tom však neodvrátí se prchlivost jeho, ale predce ruka jeho bude vztažená.

\chapter{10}

\par 1 Beda tem, kteríž ustanovují práva nepravá, a spisovatelum, kteríž težkosti spisují,
\par 2 Aby odstrkovali nuzné od soudu, a vydírali spravedlnost chudých lidu mého, vdovy aby byly korist jejich, a sirotky aby loupili.
\par 3 I což uciníte v den navštívení a zpuštení, kteréž zdaleka prijde? K komu se o pomoc utecete? A kde zanecháte slávy své,
\par 4 By nemusila skloniti se mezi vezni, a mezi zbitými klesnouti? Ve všem tom neodvrátí se prchlivost jeho, ale predce ruka jeho bude vztažená.
\par 5 Beda Assurovi, metle hnevu mého, ackoli hul rozhnevání mého jest v rukou jeho,
\par 6 A na národ ošemetný pošli jej, a o lidu hnevu mého prikáži jemu, aby smele bral koristi, a loupil bez milosti, a položil jej v pošlapání jako bláto na ulicích.
\par 7 Ale on ne tak se bude domnívati, ani srdce jeho tak mysliti bude; nebo srdce jeho jest hubiti a pléniti národy mnohé.
\par 8 Nebo rekne: Zdaliž knížata má nejsou také i králové?
\par 9 Zdaliž jako Charkemis není Chalno? Zdali není jako Arfad Emat? Zdali není jako Damašek Samarí?
\par 10 Jakož nalezla ruka má království bohu, ješto rytiny jejich byly nad Jeruzalémských a Samarských,
\par 11 Zdaliž jako jsem ucinil Samarí a modlám jeho, tak neuciním Jeruzalému a obrazum jeho?
\par 12 I stanet se, když dokoná Pán všecko dílo své na hore Sion a v Jeruzaléme, že navštívím ovoce pyšného srdce krále Assyrského, a nádhernost vysokých ocí jeho.
\par 13 Nebo rekne: V síle ruky své to jsem vykonal, a v moudrosti své; nebo jsem rozumný byl, a odjal jsem meze národu, a poklady jejich jsem vzebral, a strhl jsem dolu, jako mocný, obyvatele.
\par 14 Anobrž jako hnízdo nalezla ruka má zboží národu, a jako zbírána bývají vejce opuštená, tak všecku zemi já jsem sebral, aniž byl, kdo by krídlem hnul, aneb otevrel ústa a siptel.
\par 15 Zdaliž se bude sekera velebiti nad toho, kdož ní seká? Zdaliž se honositi bude pila nad toho, kdož ní tre? Jako by se zpínala metla proti tomu, kdož by ji zdvihl; jako by se chlubila hul, že není drevem.
\par 16 Protož pošle Pán, Hospodin zástupu, na vytylé jeho vyzáblost, a po zpodku slávu jeho prudce zapálí, jako silný ohen.
\par 17 Nebo svetlo Izraelovo bude ohnem, a Svatý jeho plamenem, i spálí a sžíre trní i bodlácí jeho jednoho dne.
\par 18 Též spanilost lesu jeho, i úrodných polí jeho, od duše až do tela, všecko vyhubí. I stane se, že predešený jsa, bude utíkati.
\par 19 A pozustalého dríví lesu jeho malý pocet bude, tak že by je mohlo díte popsati.
\par 20 I stane se v ten den, nebudou více ostatkové Izraelští a pozustalí z domu Jákobova zpoléhati na toho, kdož je tepe, ale zpoléhati budou na Hospodina, Svatého Izraelského v pravde.
\par 21 Ostatkové obrátí se, ostatkové Jákobovi k Bohu silnému, reku udatnému.
\par 22 Nebo byt bylo lidu tvého, Izraeli, jako písku morského, ostatkové jeho obrátí se. Pohubení uložené rozhojní spravedlnost.
\par 23 Pohubení, pravím, a to jisté, Pán, Hospodin zástupu, uciní u prostred vší této zeme.
\par 24 Protož takto praví Pán, Hospodin zástupu: Neboj se Assyrského, lide muj, kterýž prebýváš na Sionu. Prutem umrská te, a holí svou opráhne na tebe na ceste Egyptské.
\par 25 Po malickém zajisté casu dokoná se hnev a prchlivost má k vyhlazení jich.
\par 26 Nebo vzbudí na nej Hospodin zástupu bic, jako porážku Madianských na skále Goréb, a jakož pozdvihl holi své na more, tak jí pozdvihne na nej, na ceste Egyptské.
\par 27 I stane se v ten den, že složeno bude bríme jeho s ramene tvého, a jho jeho s šíje tvé, nýbrž zkaženo bude jho od prítomnosti pomazaného.
\par 28 Pritáhne do Aiat, prejde pres Migron, v Michmas složí nádobí svá.
\par 29 Projdou pruchod, v Gabaa budou míti hospodu k prenocování; ulekne se Ráma, Gabaa Saulovo utece.
\par 30 Naríkej hlasem svým, mesto Gallim, at se slyší v Lais: Ach, ubohá Anatot.
\par 31 Pohne se Madmena, obyvatelé Gábim schopí se.
\par 32 Ješte téhož dne zastave se v Nobe, pohrozí rukou svou hore dcery Sionské, pahrbku Jeruzalémskému.
\par 33 Aj, Panovník Hospodin zástupu oklestí vší silou ratolesti, ty pak, kteríž jsou vysokého zrostu, podetne; i budou vysocí sníženi.
\par 34 Vyseká též houšt lesu sekerou, i Libán od velikomocného padne.

\chapter{11}

\par 1 Ale vyjdet proutek z parezu Izai, a výstrelek z korenu jeho vyroste, a ovoce ponese.
\par 2 Na nemž odpocine Duch Hospodinuv, Duch moudrosti a rozumnosti, Duch rady a síly. Duch umení a bázne Hospodinovy.
\par 3 A bude stižitelný v bázni Hospodinove, a nebudet podlé videní ocí svých souditi, ani podlé slyšení uší svých trestati.
\par 4 Ale souditi bude chudé podlé spravedlnosti, a v pravosti trestati tiché v zemi. Bíti zajisté bude zemi holí úst svých, a duchem rtu svých zabije bezbožného.
\par 5 Nebo spravedlnost bude pasem bedr jeho, a pravda prepásaním ledví jeho.
\par 6 I bude bydliti vlk s beránkem a pardus s kozlátkem ležeti; tolikéž tele a lvíce i krmný dobytek spolu budou, a malé pacholátko je povede.
\par 7 Také i kráva a nedvedice spolu pásti se budou, a plod jejich spolu ležeti, lev pak jako vul plevy jísti bude.
\par 8 A lítý had nad derou pohrávati bude s dítetem prsí požívajícím, a to, kteréž ostaveno jest, smele sáhne rukou svou do díry bazališkovy.
\par 9 Neuškodí, aniž zahubí na vší mé hore svaté; nebo zeme naplnena bude známostí Hospodina, tak jako vodami more naplneno jest.
\par 10 A budet v ten den, že na koren Izai, kterýž stane za korouhev národum, pohané pilne ptáti se budou; nebo odpocívání jeho bude slavné.
\par 11 I bude v ten den, že priciní Pán po druhé ruky své, aby shledal ostatky lidu svého, což jich zanecháno bude od Assura, a od Egypta, a od Patros, a od Chus, a od Elam, a od Sinear, a od Emat, a od ostrovu morských.
\par 12 A vyzdvihne korouhev mezi pohany, a sbére zahnané Izraelské, a rozptýlené Judovy shromáždí ode ctyr stran zeme.
\par 13 I prestane nenávist Efraimova, a neprátelé Judovi vyhlazeni budou. Efraim nebude nenávideti Judy, a Juda nebude ssužovati Efraima.
\par 14 Ale vletí na rameno Filistinských k západu, a spolu loupiti budou národy východní. Na Idumejské a Moábské sáhnou rukou svou, a synové Ammon poslouchati jich budou.
\par 15 Zkazí též Hospodin zátoku more Egyptského, a vztáhne ruku svou na reku v prudkosti vetru svého, a rozrazí ji na sedm potucku, a uciní, aby je v obuvi prejíti mohli.
\par 16 I bude silnice ostatkum lidu toho, kterýž zanechán bude od Assyrských, jako byla Izraelovi v ten den, když vycházel z zeme Egyptské.

\chapter{12}

\par 1 I díš v ten den: Oslavovati te budu, Hospodine, proto že byv hneviv na mne, odvrátil jsi prchlivost svou, a utešil jsi mne.
\par 2 Aj, Buh silný spasení mé, doufati budu, a nebudu se strašiti; nebo síla má a písen a spasení mé jest Buh Hospodin.
\par 3 I budete vážiti vody s radostí z studnic toho spasení,
\par 4 A reknete v ten den: Oslavujte Hospodina, vzývejte jméno jeho, známé cinte mezi lidmi skutky jeho, pripomínejte, že vyvýšené jest jméno jeho.
\par 5 Žalmy zpívejte Hospodinu, nebo veliké veci ucinil; a to známé bud po vší zemi.
\par 6 Prokrikni a zpívej, obyvatelkyne Sionská, nebo veliký jest u prostred tebe Svatý Izraelský.

\chapter{13}

\par 1 Bríme Babylona, kteréž videl Izaiáš syn Amosuv.
\par 2 Na hore vysoké vyzdvihnete korouhev, povyšte hlasu k nim, dejte náveští rukou, at vejdou do bran knížecích.
\par 3 Já jsem prikázal posveceným svým, povolal jsem také i udatných reku svých k vykonání hnevu svého, veselících se z vyvýšení mého.
\par 4 Hlas množství na horách, jakožto lidu mnohého, hlas a zvuk království a národu shromáždených: Hospodin zástupu sbírá vojsko k válce.
\par 5 Táhnou z zeme daleké od koncin nebes. Hospodin a osudí prchlivosti jeho, aby poplénil všecku zemi.
\par 6 Kvelte, nebo blízko jest den Hospodinuv, jako zpuštení od Všemohoucího prijde.
\par 7 A protož všeliké ruce oslábnou, a všeliké srdce cloveka rozplyne se.
\par 8 I budou predešeni, svírání a bolesti je zachvátí, jako rodicka stonati budou; každý nad bližním svým užasne se, tváre jejich k plameni podobné budou.
\par 9 Aj, den Hospodinuv prichází prísný, a zurivost a rozpálení hnevu, aby obrátil tu zemi v poušt, a hríšníky její z ní vyhladil.
\par 10 Nebo hvezdy nebeské a planéty jejich nedopustí svítiti svetlu svému; zatmí se slunce pri vycházení svém, a mesíc nevydá svetla svého.
\par 11 A navštívím na okršlku zeme zlost, a na bezbožných nepravost jejich; a káži prestati pýše pyšných, a vysokomyslnost tyranu snížím.
\par 12 Zpusobím to, že dražší bude clovek nad zlato cisté, clovek, pravím, nad zlato z Ofir.
\par 13 Z té príciny zatresu nebesy, a pohne se zeme z místa svého, v prchlivosti Hospodina zástupu, a ve dni rozpálení hnevu jeho.
\par 14 I bude jako srna zplašená, a jako stádo, když není, kdo by je shromáždil; jeden každý k lidu svému se obrátí, a každý do zeme své utece.
\par 15 Kdožkoli nalezen bude, bude proboden, a kteríž by se koli shlukli, od mece padnou.
\par 16 Nadto i dítky jejich rozrážíny budou pred ocima jejich, domové jejich zloupeni, a ženy jejich poškvrneny.
\par 17 Aj, já vzbudím proti nim Médské, kteríž sobe stríbra nebudou vážiti, a v zlate nebudou se kochati,
\par 18 Ale z luku dítky rozrážeti, aniž se nad plodem života slitují, aniž synum odpustí oko jejich.
\par 19 I budet Babylon, nekdy ozdoba království a okrasa dustojnosti Kaldejské, podobný podvrácené Sodome a Gomore.
\par 20 Nebudou se v nem osazovati na veky, ani bydliti od pokolení až do pokolení; aniž rozbije tam stanu svého Arab, ani pastýri tam odpocívati budou.
\par 21 Ale lítá zver tam odpocívati bude, a domové jejich šelmami naplneni budou; bydliti budou tam i sovy, a príšery tam skákati budou.
\par 22 Ozývati se také budou sobe hrozné potvory na palácích jejich, a draci na hradích rozkošných. A blízkot jest, že prijde cas jeho, a dnové jeho prodlévati nebudou.

\chapter{14}

\par 1 Nebo slituje se Hospodin nad Jákobem, a vyvolí zase Izraele, a dá jim odpocinutí v zemi jejich; a pripojí se k nim cizozemec, a prídržeti se budou domu Jákobova.
\par 2 Nebo pojmou ty národy, a privedou je k místu svému, i uvedou je v dedictví dum Izraelský v zemi Hospodinove, za služebníky a za devky; a jímati budou ty, kteríž je zjímali, a panovati budou nad násilníky svými.
\par 3 I stanet se v ten den, v nemž tobe odpocinutí dá Hospodin od težkosti tvé a strachu tvého, a od poroby težké, v kterouž jsi byl podroben,
\par 4 Že uživeš prísloví tohoto o králi Babylonském, a rekneš: Aj, jak prestal násilník! Prestalo dychtení po zlate.
\par 5 Potrískal Hospodin hul bezbožných, prut panujících,
\par 6 Mrskajícího lidi v prchlivosti mrskáním ustavicným, panujícího v hneve nad národy, kteríž ssužováni bývali bez lítosti.
\par 7 Odpocívá, jest v pokoji všecka zeme, zvucne prozpevují.
\par 8 I jedloví veselí se nad tebou, i cedroví Libánské, rkouce: Jakž jsi klesl, nepovstal, kdo by nás podtínal.
\par 9 I peklo zespod zbourilo se pro tebe, k vyjití vstríc pricházejícímu tobe vzbudilo pro te mrtvé, všecka knížata zeme; kázalo vyvstati z stolic jejich i všechnem králum národu.
\par 10 Všickni tito odpovídajíce, mluví tobe: Což ty také jsi zemdlen jako i my, a nám podobný ucinen?
\par 11 Svrženat jest do pekla pýcha tvá, i zvuk hudebných nástroju tvých; moli tobe podestláno, a cervi te prikrývají.
\par 12 Jakž to, že jsi spadl s nebe, ó lucifere v jitre vycházející? Poražen jsi až na zem, ještos zemdlíval národy.
\par 13 Však jsi ty ríkával v srdci svém: Vstoupím do nebe, nad hvezdy Boha silného vyvýším stolici svou, a posadím se na hore shromáždení k strane pulnocní.
\par 14 Vstoupím nad výsosti oblaku, budu rovný Nejvyššímu.
\par 15 A ty pak stržen jsi až do pekla, pryc na stranu do jámy.
\par 16 Ti, kdož te uzrí, za tebou se ohlédati, a tebe spatrovati budou, ríkajíce: To-liž jest ten muž, kterýž nepokojil zemi, a pohyboval královstvími,
\par 17 Obracel jako v pustinu okršlek zeme, a mesta jeho boril, veznu svých nepropouštel domu?
\par 18 Všickni králové národu, což jich koli bylo, pochováni slavne doma jeden každý z nich;
\par 19 Ty pak zavržen jsi od hrobu svého jako ratolest ohyzdná, a roucho zbitých, ukrutne zranených, kteríž se dostávají do jámy mezi kamení, a jako mrcha pošlapaná.
\par 20 Nebudeš k onemno v pohrbu priúcastnen, nebo jsi poplénil zemi svou, lid svuj jsi pomordoval; nebudet pripomínáno na veky síme zlostníku.
\par 21 Pripravte se k zmordování synu jeho pro nepravosti otcu jejich, aby nepovstali, a dedicne neujali zeme, a nenaplnili svrchku okršlku zemského mesty.
\par 22 Nebo povstanu proti nim, praví Hospodin zástupu, a zahladím jméno Babylona i ostatky syna i vnuka, praví Hospodin.
\par 23 A obrátím jej v dedictví bukacu, a v jezera vod, a vymetu jej pometlem zahynutí, praví Hospodin zástupu.
\par 24 Prisáhl Hospodin zástupu, rka: Jiste že jakž jsem myslil, tak bude, a jakž jsem uložil, stane se,
\par 25 Že potru Assyrského v zemi své, a na horách svých pošlapám jej, a odejde z nich jho jeho, bríme také jeho s ramene jejich snato bude.
\par 26 Tot jest ta rada, kteráž zavrína jest o vší té zemi, a to jest ta ruka vztažená proti všechnem tem národum.
\par 27 Ponevadž pak Hospodin zástupu usoudil, kdo to tedy zruší? A ruku jeho vztaženou kdo odvrátí?
\par 28 Léta kteréhož umrel král Achas, stalo se proroctví toto:
\par 29 Neraduj se všecka ty zeme Filistinská, že zlámán jest prut toho, kterýž te mrskal; nebo z plemene hadího vyjde bazališkus, jehož plod bude drak ohnivý létající.
\par 30 I budou se pásti prvorození chudých, a nuzní bezpecne odpocívati budou; koren pak tvuj umorím hladem, a ostatky tvé zmorduje.
\par 31 Kvel, ó bráno, kric mesto, již jsi rozplynula se všecka ty zeme Filistinská; nebo od pulnoci ohen prijde aniž bude, kdo by stranil z obcí jeho.
\par 32 Co pak odpovedí poslové národu? To, že Hospodin upevnil Sion, v nemž útocište mají chudí z lidu jeho.

\chapter{15}

\par 1 Bríme Moábských. Když v noci Ar Moábské popléneno a zkaženo bude, když i Kir Moábské v noci popléneno a zkaženo bude,
\par 2 Vstoupí do Baít, a do Dibon a do Bamot s plácem, nad Nébo a nad Medaba Moáb kvíliti bude, na všech hlavách jeho bude lysina, a každá brada oholena bude.
\par 3 Na ulicích jeho prepáší se žíní, na strechách jeho i na ryncích jeho každý kvíliti bude, s plácem se vraceje.
\par 4 A kriceti bude Ezebon a Eleale, až v Jasa slyšán bude hlas jejich, nýbrž i zbrojní Moábští kriceti budou. Duše každého z nich žalostiti bude, a rekne:
\par 5 Srdce mé rve nad Moábem a pevnostmi jeho, až slyšeti v Ségor, jako jalovice tríletá; nebo cestou Luchitskou s plácem pujde, a kudyž se chodí k Choronaim, krik hrozný vydávati budou,
\par 6 Proto že vody Nimrim vymizejí, že uschne bylina, usvadne tráva, aniž co zeleného bude.
\par 7 A protož zboží nachované a statky jejich odnesou ku potoku Arabim.
\par 8 Nebo krik obejde vukol meze Moábské, až do Eglaim kvílení jeho, a až do Beer Elim kvílení jeho,
\par 9 Ponevadž i vody Dimon naplneny budou krví. Pridám zajisté Dimonu prídavku, a pošli na ty, kteríž ujdou z Moábských, lvy, i na pozustalé v té zemi.

\chapter{16}

\par 1 Pošlete beránky panovníku zeme, pocnouc od Sela až do poušte, k hore dcery Sionské.
\par 2 Sic jinak bude Moáb jako pták místa nemající, a s hnízda sehnaný; budou dcery Moábské pri brodech Arnon.
\par 3 Svolej radu, ucin soud, priprav stín svuj u prostred poledne jako noc; skrej vyhnané, místa nemajícího nevyzrazuj.
\par 4 Nechat u tebe pobudou vyhnaní moji, ó Moábe, bud skrýší jejich pred zhoubcím; nebo prestane násilník, prestane zhoubce, pošlapávající vyhlazen bude z zeme.
\par 5 A upevnen bude milosrdenstvím trun, a sedeti bude na nem stále v stánku Davidovu ten, kterýž by soudil a vyhledával soudu a pospíchal k spravedlnosti.
\par 6 Ale slýchalit jsme o pýše Moábove, že velmi pyšný jest, o pýše jeho, a chloube jeho i spouzení se jeho, ale neprijdout k vykonání myšlénky jeho.
\par 7 Protož kvíliti bude Moáb pred Moábem, jeden každý kvíliti bude; nad grunty Kirchareset úpeti budete, a ríkati: Jižt jsou zkaženi.
\par 8 Nýbrž i réví Ezebon usvadlo, i vinní kmenové Sibma. Páni národu potreli výborné réví jeho, kteréž až do Jazer dosahalo, a bylo se rozšírilo pri poušti; rozvodové jeho rozložili se, a dosahali až za more.
\par 9 Protož pláci pro plác Jazerských a pro vinici Sibma; svlažuji te slzami svými, ó Ezebon a Eleale, nebo provyskování nad ovocem tvým letním a nad žní tvou kleslo.
\par 10 A prestalo veselé a plésání nad polem úrodným, na vinicích se nezpívá, ani prokrikuje, vína v presích netlací ten, kterýž tlacívá. Takž i já provyskování prestávám.
\par 11 Protože streva má nad Moábem jako harfa znejí, a vnitrnosti mé pro Kircheres.
\par 12 I stane se, když zrejmé bude, an ustává Moáb nad výsostmi, že vejde do svatyne své, aby se modlil, však nic nespraví.
\par 13 Tot jest to slovo, kteréž mluvil Hospodin o Moábovi již dávno.
\par 14 Nyní pak praví Hospodin, rka: Po trech letech, jakáž jsou léta nájemníka, v potupu uvedena bude sláva Moábova se vším množstvím velikým, tak že ostatkové jeho budou skrovní, premaliccí a mdlí.

\chapter{17}

\par 1 Bríme Damašku. Aj, Damašek prestane býti mestem, a bude hromadou rumu.
\par 2 Zpuštená mesta Aroer pro stáda budou, a odpocívati tam budou, a nebude žádného, kdo by je strašil.
\par 3 I bude odjata pevnost od Efraima, a království od Damašku i ostatku Syrských; jako i sláva synu Izraelských na nic prijdou, praví Hospodin zástupu.
\par 4 I bude v ten den, že opadne sláva Jákobova, a tuk tela jeho vymizí.
\par 5 Nebo bude Assur jako ten, jenž shromažduje ve žni obilé, a ráme jeho žne klasy, anobrž bude jako ten, jenž zbírá klasy v údolí Refaim.
\par 6 Paberkové však zanecháni v nem budou, jako po ocesání olivy dve neb tri olivky na vrchu vetve, a ctyry neb pet na ratolestech jejích plodistvých, praví Hospodin Buh Izraelský.
\par 7 V ten den patriti bude clovek k Uciniteli svému, a oci jeho k Svatému Izraelskému hledeti budou.
\par 8 A nebude patriti k oltárum, dílu rukou svých; ani k tomu, což ucinili prstové jeho, hledeti bude, ani k hájum, ani k obrazum slunecným.
\par 9 V ten den budou mesta síly jeho opuštena jako chrastinka a ružtka, kteráž opuštena budou od synu Izraelských, i spustneš, ó zeme.
\par 10 Nebos se zapomnela na Boha spasení svého, a na skálu síly své nezpomenulas. Protož ackoli štepy rozkošné štepuješ, a kmen vinný prespolní sázíš,
\par 11 V cas štepování tvého štípí, aby rostlo, opatruješ, nýbrž téhož jitra o to, což seješ, aby se pucilo, pecuješ: v den však užitku odejde žen, na žalost tvou pretežkou.
\par 12 Beda množství lidí mnohých, kteríž jako zvuk morský jecí, a hlucícím národum, kteríž jako zvuk vod násilných hlucí,
\par 13 Národum jako zvuk vod mnohých zvucícím; nebo je Buh okrikne. Procež daleko utíkati budou, a honeni budou jako plevy po vrších od vetru, a jako chumel od vichrice.
\par 14 Nebo u vecer aj, predešení, a než jitro prijde, ant ho není. Tent jest podíl tech, kteríž nás potlacují, a los tech, kteríž nás loupí.

\chapter{18}

\par 1 Beda zemi zastenující se krídly, kteráž jest pri rekách zeme Mourenínské,
\par 2 Posílající po mori posly, v nástrojích z sítí po vodách, rka: Jdete, poslové rychlí, k národu rozptýlenému a zloupenému, k lidu hroznému zdávna i posavad, k národu všelijak potlacenému, jehož zemi reky roztrhaly.
\par 3 Všickni obyvatelé sveta, a prebývající na zemi, když bude korouhev vyzdvižena na horách, uzríte, a když troubiti se bude trubou, uslyšíte.
\par 4 Nebo takto praví Hospodin ke mne: Spokojímt se, a podívám z príbytku svého, a budu jako teplo vysušující po dešti, a jako oblak deštový v cas horké žne.
\par 5 Pred vinobraním zajisté, když se vypucí pupenec, a kvet vydá hrozen trpký, ješte rostoucí, tedy podreže révícko noži, a rozvody odejme a zpodtíná.
\par 6 I budou zanecháni všickni spolu ptactvu z hor, a šelmám zemským, a bude na nich pres léto ptactvo, a všeliké šelmy zemské na nich pres zimu zustanou.
\par 7 V ten cas prinesen bude dar Hospodinu zástupu (od lidu rozptýleného a zloupeného, od lidu hrozného zdávna i posavad, národu všelijak potlaceného, jehožto zemi reky rozchvátaly), k místu jména Hospodina zástupu, hore Sion.

\chapter{19}

\par 1 Bríme Egyptských. Aj, Hospodin bére se na oblaku lehkém, a pritáhne na Egypt. I pohnou se modly Egyptské pred tvárí jeho, a srdce Egyptských rozplyne se u prostred neho.
\par 2 Nebo spustím Egyptské s Egyptskými, tak že bojovati budou jeden každý proti bratru svému, a prítel proti príteli svému, mesto proti mestu, království proti království.
\par 3 A na nic prijde duch Egyptských u prostred neho, a radu jeho sehltím. I raditi se budou modl a kouzedlníku a zaklinacu a hadacu.
\par 4 Dám zajisté Egypt v ruku pánu ukrutných, a král prísný panovati bude nad nimi, praví Pán, Hospodin zástupu.
\par 5 A vymizejí vody z more, i reka osákne a vyschne.
\par 6 I vzdálí se reky, opadnou a vyschnou potokové Egyptští, trtí i rákosí usvadne.
\par 7 Tráva okolo potoka a pri pramenu potoka, i vše, což se seje pri potoku, uschne, zmizí a ztratí se.
\par 8 I budou žalostiti rybári, a kvíliti všickni, kteríž mecí do potoka udici; a kteríž rozstírají síti na vody, na nouzi prijdou.
\par 9 Zahanbeni budou i ti, kteríž delají veci lnené a hedbávné, a kteríž tkají kment.
\par 10 Nebo síti jeho budou zkaženy, i všickni delající rybníky pro ryby.
\par 11 Jiste žet jsou blázni knížata Soan, moudrých rádcu Faraonových rada zhlupela. Jakž ríkati mužete Faraonovi: Syn moudrých já jsem, syn králu starožitných?
\par 12 Kdež jsou, kde ti moudrí tvoji? Nechat oznámí nyní tobe, vedí-li, co uložil Hospodin zástupu o Egyptu.
\par 13 Zbláznila se knížata Soan, podvedena jsou knížata Nof, svedli Egypt prednejší v pokolení jeho.
\par 14 Hospodin pustil mezi ne ducha závrativého, i spraví to, že zbloudí Egypt pri všelikém predsevzetí svém, tak jako bloudí ožralec pri vývratku svém.
\par 15 Aniž bude dílo v Egypte, kteréž by ucinila hlava neb ocas, ratolest aneb sítí.
\par 16 V ten den bude Egypt podobný ženám; nebo strašiti se a desiti bude pred zdvižením ruky Hospodina zástupu, kterouž on zdvihne proti nemu.
\par 17 A budet zeme Judská Egyptu k hruzi; každý, kdož zpomene na ni, strašiti se bude, pro radu Hospodina zástupu, kterouž zavrel o nem.
\par 18 V ten den bude pet mest v zemi Egyptské, mluvících jazykem Kananejským, a prisahajících skrze Hospodina zástupu, jedno pak nazváno bude mesto zpuštení.
\par 19 V ten den bude oltár Hospodinuv u prostred zeme Egyptské, a sloup pri pomezí jejím Hospodinu;
\par 20 Bude, pravím, na znamení a na svedectví Hospodinu zástupu v zemi Egyptské. A když volati budou k Hospodinu prícinou tech, kteríž by je ssužovali, tedy pošle jim spasitele a kníže, i vysvobodí je.
\par 21 I bude známý Hospodin Egyptským; nebo poznají Egyptští Hospodina v ten den, a ctíti jej budou obetmi a dary, a ciniti budou sliby Hospodinu, i plniti.
\par 22 A tak bíti bude Hospodin Egypt, aby zbije, uzdravil jej; nebo obrátí se k Hospodinu, a on je vyslyší a uzdraví.
\par 23 V ten den bude silnice z Egypta do Assyrie, i budou choditi Assyrští do Egypta, a Egyptští do Assyrie, a sloužiti budou Egyptští s Assyrskými Hospodinu.
\par 24 V ten den bude Izrael Egyptským a Assyrským jako tretí z nich, i budou požehnaní u prostred zeme.
\par 25 Nebo požehná jim Hospodin zástupu, rka: Požehnaný lid muj Egyptský, a dílo rukou mých Assur, i dedictví mé Izrael.

\chapter{20}

\par 1 Léta, kteréhož pritáhl Tartan do Azotu, poslán jsa od Sargona, krále Assyrského, a když bojoval proti Azotu, a vzal jej,
\par 2 Casu toho mluvil Hospodin skrze Izaiáše syna Amosova, rka: Jdi a slož žíni z bedr svých, a obuv svou zzuj s noh svých. I ucinil tak, a chodil nahý a bosý.
\par 3 I rekl Hospodin: Jakož chodí služebník muj Izaiáš nahý a bosý, na znamení a zázrak, tretího roku Egypta a Mourenínské zeme,
\par 4 Tak povede král Assyrský jaté Egyptské a jaté Mourenínské, mladé i staré, nahé a bosé, s obnaženými zadky, k hanbe Egyptských.
\par 5 I užasnou se a zahanbí nad Moureníny, útocištem svým, a nad Egyptskými, chloubou svou.
\par 6 Tedy rekne obyvatel ostrovu tohoto v ten den: Aj hle, tot naše útocište, k nemuž jsme se utíkali o pomoc, abychom vysvobozeni byli z moci krále Assyrského. Jakž bychom my tedy ušli?

\chapter{21}

\par 1 Bríme pustého more. Jako vichrice na poledne se žene, tak prijde z poušte, z zeme hrozné.
\par 2 Videní tvrdé jest mi ukázáno. Nešlechetník nešlechetnost páše, a zhoubce hubí. Pritáhniž Elame, Médský oblehni. Všelikému úpení jeho prestati rozkáži.
\par 3 Z té príciny naplnena jsou bedra má bolestí, úzkosti postihly mne jako úzkosti rodicky; sklícen jsem, slyše to, strnul jsem, vida to.
\par 4 Zkormoutilo se srdce mé, hruza predesila mne, noc mých rozkoší obrátila se mi v strach.
\par 5 Pristroj na stul, necht stráž drží strážný, jez, pí. Vstante knížata, mažte pavézy.
\par 6 Nebo tak rekl ke mne Pán: Jdi, postav strážného, kterýž by to, což uhlédá, oznámil.
\par 7 I videl vozy, a dvema rady jízdu, vozy, kteréž oslové, a vozy, kteréž velbloudové táhli; šetril zajisté pilne s velikou bedlivostí.
\par 8 A volal jako lev: Ját, Pane muj, stojím na stráži ustavicne ve dne, nýbrž na stráži své já stávám v každickou noc.
\par 9 (A aj, v tom prijeli na vozích muži a dvema rady jízda.) Zvolal tedy a rekl: Padl, padl Babylon, a všecky rytiny bohu jeho o zem roztrískány.
\par 10 Mét jest humno, a obilé humna mého. Což jsem slyšel od Hospodina zástupu, Boha Izraelského, oznámil jsem vám.
\par 11 Bríme Dumy. Slyším hlas z Seir: Strážný, co bylo v noci? Strážný, co se stalo v noci?
\par 12 Rekl strážný: Prišlo jitro, a tolikéž noc; chcete-li hledati, hledejte. Navratte se, pridte.
\par 13 Bríme na Arabii. Po lesích v Arabii nocleh mívati budete, ó pocestní Dedanských.
\par 14 Obyvatelé zeme Tema nechat vynesou vody vstríc žíznivému; s chlebem jeho nechat vyjdou proti utíkajícímu.
\par 15 Nebo pred mecem utíkati budou, pred mecem vytaženým, pred lucištem nataženým, pred težkostí boje.
\par 16 Tak zajisté rekl Pán ke mne: Že po roce, jakýž jest rok nájemníka, prestane všecka sláva Cedar,
\par 17 A pozustalý pocet strelcu udatných synu Cedar zmenšen bude; nebo Hospodin Buh Izraelský to mluvil.

\chapter{22}

\par 1 Bríme údolí videní. Cožt se stalo, že jsi vystoupilo všecko na strechy,
\par 2 Mesto plné hrmotu a hluku, mesto veselící se? Zbití tvoji nejsou zbiti mecem, ani zhynuli v boji.
\par 3 Všecka knížata tvá rozprchla se naporád, od strelcu svázána jsou. Což jich koli nalezeno jest v tobe, naporád svázáni jsou, zdaleka utíkají.
\par 4 Protož jsem rekl: Ponechejte mne, at horekuji s plácem, a neusilujte mne tešiti nad poplénením dcerky lidu mého.
\par 5 Nebo jest den ssoužení, a pošlapání, a v mysli sevrení ode Pána, Hospodina zástupu, v údolí videní, den borení zdi, a kriku k horám.
\par 6 Bylte zajisté Elam pochytil toul s vozy lidu vojenského, a Kir ukázal pavézu.
\par 7 I stalo se, že nejvýbornejší údolí tvá naplnena byla vozy, a vojáci silne položili se u brány,
\par 8 A odkryto bylo zastrení Judovo; však obrátilo jsi zretel v ten den k zbrojné komore.
\par 9 I k zboreninám mesta Davidova dohlédli jste, nebo mnohé byly, a shromáždili jste vody rybníka dolního.
\par 10 Domy též Jeruzalémské sectli jste, i poborili, abyste utvrdili zed.
\par 11 Udelali jste také stav mezi dvema zdmi pro vody rybníka starého, aniž jste popatrili k Uciniteli jeho, a toho, kdo jej vzdelal od starodávna, nevideli jste.
\par 12 Nadto když volal Pán, Hospodin zástupu, v ten den k pláci a k kvílení, a k lysine a k prepásání se žíní.
\par 13 A aj, radost a veselí vaše zabijeti voly, a bíti ovce, jísti maso, a píti víno, a ríkati: Jezme, píme, nebo zítra zemreme.
\par 14 Ale známét jest to v uších mých, praví Hospodin zástupu. Protož nikoli vám nebude odpuštena ta nepravost, až i zemrete, praví Pán, Hospodin zástupu.
\par 15 Takto praví Pán, Hospodin zástupu: Jdi, vejdi k Sochitskému tomu, k Sobnovi správci domu, a rekni:
\par 16 Co ty zde máš? A koho zde máš, že jsi vytesal sobe zde hrob? Vytesals sobe na vysokém míste hrob svuj, a vystavels na skále príbytek svuj.
\par 17 Aj, Hospodin, kterýž te pristrel, jakž na muže znamenitého náleží, a kterýž te výborne priodel,
\par 18 Prudce te zakulí jako kuli do zeme všelijak prostranné. Tam umreš, tam i vozové slávy tvé, ó ohyzdo domu Pána svého.
\par 19 A tak seženu te s místa tvého, a s úradu tvého svrhu te.
\par 20 I stane se v ten den, že povolám služebníka svého Eliakima syna Helkiášova,
\par 21 A obleku jej v sukni tvou, a pasem tvým potvrdím ho, panování tvé také dám v ruku jeho. I bude za otce obyvatelum Jeruzalémským a domu Judovu,
\par 22 A vložím klíc domu Davidova na rameno jeho. Když otevre, žádný nezavre, a když zavre, žádný neotevre.
\par 23 A vbiji jej jako hrebík v míste pevném, a bude stolicí slávy domu otce svého.
\par 24 I zavesí na nem synové a dcery všecku slávu domu otce jeho, všecko nádobí, i to nejmenší, od nádobí, z nehož se pije, až do všech nádob vinných.
\par 25 V ten den, praví Hospodin zástupu, pohne se hrebík, kterýž vbit byl v míste pevném, a vytat bude, a spadne, odtato bude i bríme, kteréž jest na nem; nebo Hospodin mluvil.

\chapter{23}

\par 1 Bríme Týru. Kvelte lodí morské; nebo poplénen jest, tak že není ani domu. Aniž kdo prichází z zeme Citim; známé ucineno jest jim.
\par 2 Umlknetež obyvatelé ostrovu, kterýž kupci Sidonští, plavíce se pres more, naplnovali,
\par 3 A jehož úrody na velikých vodách, síme Sichor, žen jeho, užitek potoka, a kterýž byl skladem národu.
\par 4 Zastyd se, Sidone, nebo praví more, pevnost morská, rkuc: Nepracuji ku porodu, a nerodím, a nevychovávám mládencu, aniž odchovávám panen.
\par 5 Jakož nad povestí o Egyptu, tak žalostiti budou nad povestí o Týru.
\par 6 Prepravte se do Tarsu, kvelte obyvatelé ostrovu.
\par 7 To-liž by se vždycky veselilo? Jestit starožitnost jeho ode dnu starodávních, ale zavedout je daleko nohy jeho.
\par 8 Kdo že to usoudil proti Týru, korunujícímu jiné, jehož kupci jsou jako knížata, a kramári jeho znamenití v zemi?
\par 9 Hospodin zástupu usoudil to, aby zohavil pýchu všelikého slavného, a za nic položil všecky znamenité zeme.
\par 10 Navrat se do zeme své jako reka, ó dcero Tarská, nenít tam více ani pasu.
\par 11 Vztáhl ruku svou na more, pohnul královstvími; Hospodin prikázal o tržišti, aby zkaženy byly pevnosti jeho.
\par 12 A rekl: Nebudeš se více veseliti, trpec nátisk, panno, dcero Sidonská. Povstan, ber se do Citim, ale i tam nebudeš míti odpocinutí.
\par 13 Aj, zeme Kaldejská, ten lid nebyl lidem. Assur vzdelal ji pro obyvatele pustin, vystaveli veže jejich, vzdelali paláce její, tak že i Assyrii vyvrátil.
\par 14 Kvelte lodí morské, nebo zpuštena jest pevnost vaše.
\par 15 I stane se v ten den, že v zapomenutí bude Týrus za sedmdesáte let, jako za vek krále jednoho. Do skonání sedmdesáti let bude míti Týrus jako písnicku nevestky.
\par 16 Vezmi harfu, obcházej mesto, ó nevestko v zapomenutí daná, hrej dobre, zpívej dlouho, abys v pamet uvedena byla.
\par 17 I bude po dokonání sedmdesáti let, že navštíví Hospodin Týr, ale on navrátí se zase k nevestcí mzde své, a smilniti bude se všemi královstvími zeme na okršlku sveta.
\par 18 Však kupectví jeho a mzda jeho svatá bude Hospodinu. Nebude na poklad skládána, ani schovávána, ale pro ty, kteríž prebývají pred Hospodinem, bude kupectví jeho, aby jedli do sytosti, a meli roucho dobré.

\chapter{24}

\par 1 Aj, Hospodin vyprázdní zemi, a pustou uciní; promení zajisté zpusob její, a rozptýlí obyvatele její.
\par 2 I budet jakož lid tak kníže, jakož služebník tak pán jeho, jakož devka tak paní její, jakož kupující tak prodávající, jakož pujcující tak vypujcující, jakož lichevník tak ten, jenž lichvu dává.
\par 3 Náramne vyprázdena bude zeme, a velice zloupena; nebo Hospodin mluvil slovo toto.
\par 4 Kvíliti bude a padne zeme, zemdlí a padne okršlek zeme, zemdlejí vysocí národové zemští,
\par 5 Proto že i ta zeme poškvrnena jest pod obyvateli svými; nebo prestoupili zákony, zmenili ustanovení, zrušili smlouvu vecnou.
\par 6 Protož prokletí zžíre zemi, a vypléneni budou obyvatelé její; protož horeti budou obyvatelé zeme, a pozustane lidí malicko.
\par 7 Žalostiti bude mest, usvadne vinný kmen, úpeti budou všickni veselého srdce.
\par 8 Odpocine radost bubnu, prestane hluk veselících se, utichne veselí harfy.
\par 9 Nebudou píti vína s prozpevováním, zhorkne opojný nápoj pitelum jeho.
\par 10 Potríno bude mesto marnosti; zavrín bude každý dum, aby do neho nechodili.
\par 11 Naríkání bude na ulicích pro víno, zatemneno bude všeliké veselí, odstehuje se radost zeme.
\par 12 Zustane v meste poušt, i brány zboreny budou.
\par 13 Nebo tak bude u prostred zeme, u prostred národu, jako ocesání olivy, jako paberkování, když se dokoná vinobraní.
\par 14 Tit pozdvihnou hlasu svého, prozpevovati budou v dustojnosti Hospodinove, prokrikovati budou i pri mori.
\par 15 Protož v údolích oslavujte Hospodina, na ostrovích morských jméno Hospodina Boha Izraelského.
\par 16 Od koncin zeme slyšíme písnicky o sláve spravedlivého. Ale já rekl jsem: Zchuravel jsem, zchuravel jsem. Ach, na mé hore, že nešlechetní nešlechetnost provodí, nešlechetnost, pravím, že tak nestydate páší.
\par 17 Hruza a jáma a osídlo te ocekává, ó obyvateli zeme.
\par 18 I stane se, že kdož utece pred povestí strachu, upadne do jámy, a kdož vyleze z jámy, v osídle uvázne; nebo pruduchové s výsosti otevríni budou, a zatresou se základové zeme.
\par 19 Velmi se rozstoupí zeme, velice se rozpadne zeme, náramne pohybovati se bude zeme.
\par 20 Motaje, motati se bude zeme jako opilý, a prenešena bude jako chaloupka; nebo težce na ni dolehne nepravost její, i padne tak, že nepovstane více.
\par 21 I stane se v ten den, navštíví Hospodin vojsko vysoké na výsosti, též i krále zemské na zemi.
\par 22 Kterížto shromáždeni budou, tak jako shromaždováni bývají veznové do žaláre, a zavríni budou u vezení; po mnohých, pravím, dnech navštíveni budou.
\par 23 I zahanbí se mesíc, a zastydí slunce, když kralovati bude Hospodin zástupu na hore Sion, a v Jeruzaléme, a pred starci svými slavne.

\chapter{25}

\par 1 Hospodine, ty jsi Buh muj, vyvyšovati te budu, a oslavovati budu jméno tvé; nebo jsi ucinil predivné veci. Rady tvé zdávna uložené jsou verná pravda.
\par 2 Nebo jsi obrátil mesto v hromadu, mesto hrazené v zríceninu, paláce cizozemcu, aby nebyli mestem, a na veky aby nebyli zase staveni.
\par 3 Protož ctíti te budou lid silný, mesta národu hrozných báti se tebe budou.
\par 4 Nebo jsi byl hradem chudému, hradem nuznému v úzkosti jeho, útocištem pred povodní, zastínením pred horkem; (nebo vzteklost ukrutníku podvrátila by zed).
\par 5 Hlucení cizozemcu jsi pretrhl jako horkost v sucho, horkost stínem oblaku; zhouba ukrutných pretržena.
\par 6 I uciní Hospodin zástupu všechnem národum na hore této hody z vecí tucných, hody z vína vystálého, z vecí tucných, mozk v sobe majících, z vína vystálého a ucišteného.
\par 7 A zkazí na hore této zastrení, kteréž zastírá všecky lidi, a prikrytí, jímž prikryti jsou všickni národové.
\par 8 Sehltí i smrt u vítezství, a setre Panovník Hospodin slzu s všeliké tvári, a pohanení lidu svého odejme ze vší zeme; (nebo Hospodin mluvil).
\par 9 Procež rekne v ten den: Aj, Buh náš tento jest, ocekávalit jsme na nej, a vysvobodil nás. Ont jest Hospodin, jehož jsme ocekávali; plésati a veseliti se budeme v spasení jeho.
\par 10 Nebo odpocine ruka Hospodinova na hore této, a mlácen bude Moáb na míste svém, jako vymlacována bývá pleva do hnoje.
\par 11 A roztáhnet ruce své u prostred neho, jako roztahuje ten, kterýž plyne k plování, a poníží pýchy jeho rameny rukou svých.
\par 12 A tak pevnost i výsost zdí tvých sehne, poníží a srazí na zem až do prachu.

\chapter{26}

\par 1 V ten den zpívána bude písen tato v zemi Judské: Mesto máme pevné, sám Buh spasením obdaril zdi a valy jeho.
\par 2 Otevrete brány, at vejde národ spravedlivý, ostríhající všeliké pravdy.
\par 3 Cloveka spoléhajícího na te ostríháš v pokoji; v pokoji, nebo v tebe doufá.
\par 4 Doufejtež v Hospodina až na veky; nebo v Hospodinu, v Hospodinu jest skála vecná.
\par 5 Ale obyvatele vysokých míst snižuje, mesta vyvýšeného ponižuje, ponižuje ho až k zemi, sráží je až do prachu.
\par 6 Pošlapává je noha, nohy chudého, krokové nuzných.
\par 7 Cesta spravedlivého jest uprímá; stezku spravedlivého vyrovnáváš.
\par 8 Také na ceste soudu tvých, Hospodine, ocekáváme na te; ke jménu tvému a k rozpomínání se na te patrí žádost duše.
\par 9 Duše má touží po tobe v noci, nýbrž i duchem svým ve mne ráno te hledám. Nebo když soudové tvoji dejí se na zemi, obyvatelé okršlku zemského ucí se spravedlnosti.
\par 10 Když se milost ciní bezbožnému, neucí se spravedlnosti; v zemi pravosti nepráve ciní, a nehledí na dustojnost Hospodinovu.
\par 11 Hospodine, ackoli vyvýšena jest ruka tvá, však toho nevidí. Uzrít a zahanbeni budou, závidíce lidu tvému; nadto i ohnem ty neprátely své sehltíš.
\par 12 Nám, Hospodine, zpusobíš pokoj; nebo i všecko, cožkoli se dálo pri nás, delal jsi pro dobré naše.
\par 13 Hospodine, Bože náš, panovalit jsou nad námi páni jiní než ty, (my však toliko v tebe doufajíce, rozpomínali jsme se na jméno tvé).
\par 14 Ale již zemrevše, neoživout, mrtví jsouc, nevstanout, proto že jsi je navštívil, a vyplénil, i zahladil všecku památku jejich.
\par 15 Rozmnožil jsi národ, ó Hospodine, rozmnožil jsi národ, a oslaven jsi, ac jsi jej byl vzdálil do všech koncin zeme.
\par 16 Hospodine, v úzkosti hledali tebe, vylévali prosby, když jsi je trestával.
\par 17 Jako tehotná, blížíc se ku porodu, svírá se, kricí v bolestech svých, tak jsme byli pred tvárí tvou, Hospodine.
\par 18 Pocali jsme, svírali jsme se, jako bychom rodili vítr: však jsme žádného vysvobození nezpusobili zemi, aniž padli obyvatelé okršlku zemského.
\par 19 Oživout mrtví tvoji, tela mrtvá má vstanou. Procitte a prozpevujte, obyvatelé prachu. Nebo rosa tvá jako rosa na bylinách, ale bezbožné k zemi zporážíš.
\par 20 Ej lide muj, vejdi do pokoju svých, a zavri dvére své za sebou; schovej se na malickou chvilku, dokudž neprejde hnev.
\par 21 Nebo aj, Hospodin bére se z místa svého, aby navštívil nepravost na obyvatelích zeme, a odkryje zeme zbité své, a nebude prikrývati více zmordovaných svých.

\chapter{27}

\par 1 V ten den navštíví Hospodin mecem svým prísným, velikým a mocným Leviatana, hada dlouhého, a Leviatana, hada stocilého, a zabije draka, kterýž jest v mori.
\par 2 V ten den o vinici výborné víno vydávající zpívejte.
\par 3 Já Hospodin, kterýž ji ostríhám, každé chvilky budu ji svlažovati, a aby jí nekdo neuškodil, v noci i ve dne ji ostríhati.
\par 4 Prchlivosti pri mne žádné není. Kdož mi dá bodlák a trn, abych proti ní válcil, a spálil ji docela?
\par 5 Zdali sváže sílu mou, aby ucinil se mnou pokoj, aby pravím, ucinil se mnou pokoj?
\par 6 Vždyt na to prijde, že se vkorení Jákob, zkvetne a zroste Izrael, a naplní okršlek zemský ovocem.
\par 7 Nebo zdaliž jej tak ubil, jako ubil neprítele jeho? Zdali jej zamordoval, jako jsou jiní zmordováni od neho?
\par 8 Vedlé míry trestal jej i tehdáž, když jej zavesti dal, ješto neprítele zachvátil vetrem svým tuhým východním.
\par 9 A protož tím zpusobem ocištena bude nepravost Jákobova, a tot bude veliký užitek, že odejme hrích jeho, když rozmece všecko kamení oltáre jako kamení vápenné rozsypané, neostojí hájové, ani slunecní obrazové;
\par 10 Když mesto hrazené zpustne, a bude príbytkem zavrženým a opušteným jako poušt, tam pásti se bude tele, a tam léhati, a pokazí docela výstrelky jeho.
\par 11 Když pocne zráti žen jeho, potrína bude; ženy prijdouce, zapálí ji. Nebo ten lid nemá žádného rozumu; protož neslituje se nad ním ucinitel jeho, a stvoritel jeho neuciní jemu milosti.
\par 12 I stane se v ten den, když pomstu uvoditi bude Hospodin od toku reky až do potoka Egyptského, že vy, synové Izraelští, po jednom sebráni budete.
\par 13 Stane se také v ten den, že troubeno bude trubou velikou, i prijdou, kteríž byli zahynuli v zemi Assyrské, a zahnáni byli do zeme Egyptské, a klaneti se budou Hospodinu na hore svaté v Jeruzaléme.

\chapter{28}

\par 1 Beda korune pýchy, ožralcum Efraimským, kvetu nestálému v kráse a sláve své, tem,kteríž jsou pri vrchu údolí velmi úrodného, a ztupeným od vína.
\par 2 Aj, silný a mocný Páne jako príval s krupobitím, jako povetrí vyvracející, jako povoden vod prudkých a rozvodnilých prudce až k zemi porazí.
\par 3 Nohami pošlapána bude koruna pýchy, ožralci Efraimští.
\par 4 Tehdáž stane se, že kvet ten nestálý v kráse a sláve své tech, kteríž jsou pri vrchu údolí velmi úrodného, bude jako ranní ovoce, prvé než léto bývá; kteréž vida nekdo, nepustil by ho z ruky, až by je snedl.
\par 5 V ten den bude Hospodin zástupu korunou ozdoby, a korunou okrasy ostatkum lidu svého,
\par 6 A duchem soudu sedícímu na soudu, a silou tem, kteríž zapuzují válku až k bráne.
\par 7 Ale i ti od vína bloudí, a od opojného nápoje se potácejí. Kníže i prorok bloudí, preplnujíce se nápojem opojným, pohlceni jsou od vína, potácejí se od nápoje opojného, bloudí u videní, chybují v soudu.
\par 8 Nebo všickni stolové plní jsou vývratku a lejn, tak že žádného místa cistého není.
\par 9 Kohož by vyucoval umení? A komu by posloužil, aby vyrozumel naucení? Zdali ostaveným od mléka, odtrženým od prsí?
\par 10 Ponevadž meli naucení za naucením, naucení za naucením, správu za správou, správu za správou, trošku odtud, trošku od onud.
\par 11 A však jako by neznámou recí a cizím jazykem mluvil lidu tomuto,
\par 12 Kdyžto jim rekl: Totot jest odpocinutí, zpusobte odpocinutí ustalému, tot jest, pravím, odpocinutí. Ale nechteli slyšeti.
\par 13 I bude jim slovo Hospodinovo naucení za naucením, naucení za naucením, správa za správou, správa za správou, troška odtud, troška od onud; k tomu aby šli a padajíce nazpet, setríni byli, a zapleteni jsouce, aby polapeni byli.
\par 14 Protož slyšte slovo Hospodinovo, muži posmevaci, panující nad lidem tímto, kterýž jest v Jeruzaléme:
\par 15 Proto že ríkáte: Ucinili jsme smlouvu s smrtí, a s peklem máme srozumení, pomsta rozvodnilá, ac precházeti bude, neprijde na nás, jakžkoli jsme položili svod za útocište své, a pod falší jsme se ukryli:
\par 16 Z té príciny takto praví Panovník Hospodin: Aj, já zakládám na Sionu kámen, kámen zkušený, úhelný drahý, základ pevný; kdo verí, nebudet kvapiti.
\par 17 A vykonám soud podlé pravidla, a spravedlnost podlé závaží, i zamete to omylné útocište krupobití, a skrýši povoden zatopí.
\par 18 A tak zrušena bude smlouva vašes smrtí, a srozumení vaše s peklem neostojí; a když precházeti bude pomsta rozvodnilá, budete od ní pošlapáni.
\par 19 Jakž jen pocne precházeti, zachvátí vás; každého zajisté jitra precházeti bude, ve dne i v noci. I stane se, že sám strach tomu, což jste slýchali, k srozumení poslouží,
\par 20 Zvlášt když bude tak krátké luže, že se nebude lze stáhnouti, a prikrýti úzké, by se i skrcil.
\par 21 Nebo jako na hore Perazim povstane Hospodin, jako v údolí Gabaon hnevati se bude, aby delal dílo své, neobycejné dílo své, aby vykonal skutek svuj, neobycejný skutek svuj.
\par 22 A protož nebudtež již posmevaci, aby se nezadrhla osídla vaše; nebo o zkažení, a to jistém, vší zeme slyšel jsem ode Pána, Hospodina zástupu.
\par 23 Nastavte uší, a slyšte hlas muj; pozorujte, a poslechnete reci mé.
\par 24 Zdaliž každého dne ore orác, aby sel, prohání brázdy, a vlácí rolí svou?
\par 25 Zdali když srovná svrchek její, nerozsívá viky, a nerozmítá kmínu a neseje pšenice prední a jecmene výborného, i špaldy v míste príhodném?
\par 26 Nebo ucí jej rozšafnosti, Buh jeho vyucuje jej.
\par 27 Nebývát pak okovaným smykem mlácena vika, aniž kolem vozním po kmínu se vukol jezdí; nebo holí vytlouká se vika, a kmín prutem.
\par 28 Pšenice mlácena bývá; však i té ne vždycky mlátiti bude, aniž ji potre kolem vozu svého, ani o zuby jeho rozdrobí.
\par 29 I to od Hospodina zástupu vyšlo, kterýž jest divný v rade, a veleslavný v skutku.

\chapter{29}

\par 1 Beda Arieli, Arieli mestu, v kterémž bydlil David. Pridejte rok po roku, nechat zarezují beránky.
\par 2 Však predce ssoužím Ariele. I nastane žalost a zámutek, nebo mi bude jako Ariel.
\par 3 Položím se zajisté vukol proti tobe vojensky, a ssoužím te bez lítosti, a vzdelám proti tobe šance.
\par 4 Tehdy sníženo jsuc, z zeme mluviti budeš, a z prachu šeptati bude rec tvá; bude, pravím, jako hadace z zeme hlas tvuj, a z prachu rec tvá sipteti.
\par 5 Nebo jako prášku drobného bude množství neprátel tvých, a jako plev létajících množství ukrutníku, a stane se to hned v okamžení.
\par 6 Od Hospodina zástupu navštíveno bude hromem a zeme tresením, a zvukem velikým, vichricí a bourí a plamenem ohne sžírajícího.
\par 7 I budet jako zdání videní nocního množství všech národu bojujících proti Arieli, a všech válcících proti nemu a pevnostem jeho, a ssužujících jej.
\par 8 Bude, pravím, jako když se lacnému ve snách zdá, an jí, ale když procítí, prázdný jest život jeho; a jako když se žíznivému ve snách zdá, an pije, a když procítí, žíznivým zustává, a duše jeho vždy žádá: tak bude množství všech národu, bojujících proti hore Sion.
\par 9 Jak zpozdilí jste, ješto byste se meli užasnouti; rozkoš provodíte, ješto byste meli na pomoc volati. Zpili se, ale ne vínem; potácejí se, ale ne od nápoje opojného.
\par 10 Nebo naplnil vás Hospodin duchem chropotu, a zavrel oci vaše; proroku i knížat vašich nejopatrnejších oci zastrel.
\par 11 Protož jest vám všeliké videní podobné slovum knihy zapecetené, kterouž dadí-li tomu, kterýž zná písmo, rkouce: Cti ji medle, i rekne: Nemohu, nebo zapecetená jest.
\par 12 Pakli dadí knihu tomu, kterýž nezná písma, rkouce: Cti ji medle, tedy dí: Neznám písma.
\par 13 Nebo praví Pán: Proto že lid tento približuje se ústy svými, a rty svými ctí mne, srdce pak své vzdaluje a bázen jejich, již se mne bojí, jest z prikázaní lidských pošlá:
\par 14 Z té príciny, aj, já také divne zajdu s lidem tímto, divne, pravím, a zázracne. I zahyne moudrost moudrých jeho, a opatrnost opatrných jeho vymizí.
\par 15 Beda tem, kteríž hluboko pred Hospodinem skrývají radu, jejichž každý skutek deje se v temnostech, a ríkají: Kdo nás vidí? A kdo nás šetrí?
\par 16 Prevrácená myšlení vaše zdali nejsou podobná hline hrncírove? Zdali ríká dílo o delníku svém: Neucinil mne? A úcinek ríká-liž o uciniteli svém: Nerozumel?
\par 17 Zdaliž po malickém a kratickém casu neobrátí se Libán v pole, a pole za les nebude pocteno?
\par 18 I uslyší v ten den hluší slova knihy, a z mrákoty a tmy oci slepých prohlédnou.
\par 19 Ale tiší rozveselí se náramne v Hospodinu, a chudí lidé v Svatém Izraelském plésati budou,
\par 20 Kdyžto prestane ukrutník, a zahyne posmevac, a všickni, kteríž jsou pilni marnosti, vypléneni budou,
\par 21 Kteríž obvinují z hríchu cloveka pro slovo, a na toho, kterýž je tresce, v bráne lécejí, a pro nic utiskují spravedlivého.
\par 22 Protož takto dí o domu Jákobovu Hospodin, kterýž vykoupil Abrahama: Nebudet již zahanben Jákob, aniž více tvár jeho zbledne.
\par 23 Nebo když uzrí syny své, dílo rukou mých u prostred sebe, an posvecují jména mého, tedy posvecovati budou Svatého Jákobova, a k bázni Boha Izraelského sloužiti,
\par 24 Aby bloudící duchem nabyli rozumnosti, a reptáci naucili se umení.

\chapter{30}

\par 1 Beda synum zpurným, dí Hospodin, skládajícím radu, kteráž není ze mne, a prikrývajícím ji prikrytím, ale ne z ducha mého, aby hrích k hríchu pridávali;
\par 2 Kteríž chodí a sstupují do Egypta, nedotazujíce se úst mých, aby se zmocnovali v síle Faraonove, a doufali v stínu Egyptském.
\par 3 Nebo síla Faraonova bude vám k hanbe, a to odpocívání v stínu Egyptském k lehkosti,
\par 4 Proto že knížata jeho byli v Soan, a poslové jeho do Chanes chodili.
\par 5 Všeckyt k zahanbení privede skrze lid, kterýž jim nic neprospeje, aniž bude ku pomoci, ani k užitku, ale k hanbe toliko a k útržce.
\par 6 Bríme hovad poledních v zemi nátisku a ssoužení, odkudž lev a lvíce, ješterka a drak ohnivý létající, odnesou na hrbete hovádek bohatství svá, a na hrbu velbloudu poklady své k lidu, kterýž jim nic neprospeje.
\par 7 Nebo Egyptští nadarmo a na prázdno pomáhati budou. Procež ohlašuji to, že by síla jejich byla s pokojem sedeti.
\par 8 Nyní jdi, napiš to na tabuli pred ocima jejich, a na knize vyrej to, aby to zustávalo do nejposlednejšího dne, a až na veky veku,
\par 9 Že lid tento zpurný jest, synové lhári, synové, kteríž nechtí poslouchati zákona Hospodinova;
\par 10 Kteríž ríkají vidoucím: Nemívejte videní, a prorokum: Neprorokujte nám toho, což pravého jest; mluvte nám pochlebenství, prorokujte oklamání.
\par 11 Sejdete s cesty, svozujte od stezky, nechat se vzdálí od tvárí naší Svatý Izraelský.
\par 12 Protož takto praví Svatý Izraelský: Proto že pohrdáte slovem tím, a doufáte ve lsti a v prevrácenosti, a spoléháte na ni:
\par 13 Z té príciny bude vám tato nepravost jako zed tržená padající, a vydutí na zdi vysoké, jejíž brzké a náhlé bývá oborení.
\par 14 A rozrazí ji, jako rozrážejí nádobu hrncírskou rozbitou; neodpustít, tak že nebude nalezena po rozražení jejím ani strepina k nabrání ohne z ohnište, anebo k nabrání vody z louže.
\par 15 Nebo tak rekl Panovník Hospodin, Svatý Izraelský: Obrátíte-li se, a spokojíte-li se, zachováni budete. V utišení se a v doufání bude síla vaše. Ale nechcete.
\par 16 Nýbrž ríkáte: Nikoli, ale na koních uteceme. Protož utíkati budete. Na rychlých ujedeme. Ale rychlejší budou stihající vás.
\par 17 Jeden tisíc pred okriknutím jednoho, a pred okriknutím peti utíkati budete, až (jestliže však vás co pozustane), budete zanecháni jako okleštené drevo na vrchu hory, a jako korouhev na pahrbku.
\par 18 Protot pak shovívá Hospodin, milost vám cine, a protot se vyvýší, aby se smiloval nad vámi; nebo Hospodin jest Buh spravedlivý. Blahoslavení všickni, kteríž ocekávají na nej.
\par 19 Lid zajisté na Sionu a v Jeruzaléme bydliti bude. Nikoli plakati nebudeš; k hlasu volání tvého bude všelijak milost ciniti s tebou. Hned jakž uslyší, ohlásít se.
\par 20 A ackoli Pán dá vám chleba úzkosti a vody ssoužení, však nebudou více odjati tobe ucitelé tvoji, ale ocima svýma vídati budeš ucitele své,
\par 21 A ušima svýma slýchati slovo tobe po zadu rkoucích: Tot jest ta cesta, chodte po ní, bud že byste se na pravo neb na levo uchýlili.
\par 22 Tedy zavržete obestrení rytin svých stríbrných, a odev slitin svých zlatých; odloucíš je jako nemoc svou trpící, rka jim: Táhnete tam.
\par 23 Dát i déšt na rozsívání tvé, kterýmž bys osíval zemi, a chléb z úrody zeme, kterýž bude jadrný a zdárný; v ten den pásti se bude i dobytek tvuj na pastvišti širokém.
\par 24 Volové také i oslové, delající zemi, píci cistou jísti budou, kteráž opálkou a vejeckou vycištena bývá.
\par 25 Budou také na všeliké hore vysoké, a na všelikém pahrbku vyvýšeném pramenové a potokové vod, v den porážky veliké, když padnou veže.
\par 26 Bude i svetlo mesíce jako svetlo slunce, svetlo pak slunce bude sedmernásobní, jako svetlo sedmi dnu, v den, v kterýž uváže Hospodin zlámání lidu svého, a ránu zbití jeho uzdraví.
\par 27 Aj, jméno Hospodinovo prichází z daleka, jehožto hnev horící a težká pomsta; rtové jeho naplneni jsou prchlivostí, a jazyk jeho jako ohen sžírající.
\par 28 Duch pak jeho jako potok rozvodnilý, kterýž až do hrdla dosáhne, aby tríbil národy, až by v nic obráceni byli, a uzdou svíral celisti národu.
\par 29 I budete zpívati, jako když se v noci zasvecuje slavnost, a veseliti se srdecne, jako ten, kterýž jde s píštalkou, bera se na horu Hospodinovu, k skále Izraelove,
\par 30 Když dá slyšeti Hospodin hlas dustojnosti své, a ukáže vztaženou ruku svou s hnevem prchlivosti a plamenem ohne sžírajícího, an vše rozráží i prívalem i kamenným krupobitím.
\par 31 Hlasem zajisté Hospodinovým potrín bude Assur, kterýž jiné kyjem bijíval.
\par 32 Ale stane se, že každé uderení holí, kterouž doloží na nej Hospodin, silne dolehne; s bubny a harfami a bitvou veselou bojovati bude proti nemu.
\par 33 Nebo pripraveno jest již dávno peklo, také i samému králi pripraveno jest. Hluboké a široké je ucinil, hranic jeho, ohne a dríví mnoho; dmýchání Hospodinovo jako potoksiry je zapaluje.

\chapter{31}

\par 1 Beda tem, kteríž se utíkají do Egypta o pomoc, a v koních zpoléhají, a doufají v vozích, že jich mnoho, a v jezdcích, že jich množství veliké, a nepatrí k Svatému Izraelskému, aniž Hospodina hledají,
\par 2 Ještot i on jest moudrý. Protož uvede pomstu, a nezmenít slov svých, ale povstane proti domu zlostníku a proti pomoci tech, kteríž páší nepravost.
\par 3 Egyptští pak jsou lidé, a ne Buh silný, a koni jejich telo, a ne duch. A protož jakž jen Hospodin vztáhne ruku svou, padne i pomocník, padne i ten, jemuž byl ku pomoci; a tak jednostejne všickni ti na nic prijdou.
\par 4 Nebo tak rekl Hospodin ke mne: Jako když lev rve aneb lvíce nad loupeží svou, proti nemuž byl-li by svolán houf pastýru, kriku jejich se nedesí, aniž se pro hluk jejich korí: tak sstoupí Hospodin zástupu, aby bojoval o horu Sion, a o pahrbek její.
\par 5 Jako ptáci létajíce, tak hájiti bude Hospodin zástupu Jeruzaléma, anobrž obhajuje vysvobodí, pomíjeje zachová.
\par 6 Navrattež se k tomu, od nehož hluboko zabredli synové Izraelští.
\par 7 Nebo v ten den zavržete jeden každý modly své stríbrné a modly své zlaté, kterýchž vám nadelaly ruce vaše, abyste hrešili.
\par 8 I padne Assur od mece ne muže, a mec ne cloveka zžíre jej; a utíkati bude pred mecem, a nejudatnejší jeho pod plat uvedeni budou.
\par 9 A tak skála jeho pro strach zmizí, a knížata jeho korouhve desiti se budou, praví Hospodin, jehož jest ohen na Sionu a pec v Jeruzaléme.

\chapter{32}

\par 1 Aj, v spravedlnosti kralovati bude král, a knížata v soudu panovati budou.
\par 2 Nebo bude muž ten jako skrýše pred vetrem, a schrana pred prívalem, jako potokové vod na míste suchém, jako stín skály veliké v zemi vyprahlé.
\par 3 A oci vidoucích nebudou blíkati, a uši slyšících pozorovati budou.
\par 4 Procež srdce bláznu nabude umení, a jazyk zajikavých prostranne a svetle mluviti bude.
\par 5 Nebudet více nazýván nešlechetný šlechetným, a skrbný nebude slouti štedrým.
\par 6 Proto že nešlechetný o nešlechetnosti mluví, a srdce jeho skládá nepravost, jak by provodil ošemetnost, a mluvil proti Hospodinu scestné veci, jak by znuzil duši lacného, a nápoj žíznivému odjal.
\par 7 Také i usilování skrbného jsou škodlivá; nebo nešlechetnosti obmýšlí, jak by k záhube privedl ponížené slovy lživými, a mluvil proti nuznému pred soudem.
\par 8 Ješto šlechetný obmýšlí šlechetné veci, a takovýt pri tom, což šlechetného jest, státi bude.
\par 9 Ženy lhostejné, vstante, slyšte hlas muj; dcery bezpecne sobe pocínající, ušima pozorujte reci mé.
\par 10 Za mnohé dny a léta vichrovány budete, ó vy v bezpecnosti bydlící; nebo prestane vinobraní, a klizení úrod neprijde.
\par 11 Trestež se strachem, ó lhostejné, pohnetež se, bezpecne sobe pocínající; svlecte se, a obnažte se, a prepašte se po bedrách.
\par 12 Kvílíce nad prsy, nad poli výbornými a nad kmeny úrodnými.
\par 13 Na zemi lidu mého trní a hloží vzejde, anobrž na všech domích veselých a meste plésajícím.
\par 14 Nebo rozkošný palác opušten bude, hluk mesta prestane, hrad vysoký a veže obráceny budou v jeskyne na vecnost, k radosti divokým oslum, a ku pastvišti stádum.
\par 15 Dokudž nebude vylit na nás duch s výsosti, a nebude obrácena poušt v pole úrodné, a pole úrodné za les pocítáno.
\par 16 I bude na poušti soud bydliti, a spravedlnost na poli úrodném prebývati.
\par 17 A zjeví se skutek spravedlnosti, pokoj, ovoce, pravím, spravedlnosti, pokoj a bezpecnost až na veky.
\par 18 Nebo bydliti bude lid muj v obydlí pokojném, totiž v príbytcích nejbezpecnejších a v odpocívání nejpokojnejším,
\par 19 Byt pak i krupobití spadlo na les, a velmi sníženo bylo mesto.
\par 20 Blaze vám, kteríž sejete na všelikých místech úrodných, vypouštejíc tam vola i osla.

\chapter{33}

\par 1 Beda tobe, zhoubce, ješto sám nebýváš huben, a kterýž neverne deláš, ješto tobe necinili neverne. Když prestaneš býti zhoubcím, pohuben budeš; když prestaneš neverne ciniti, nevernet ciniti budou.
\par 2 Hospodine, ucin nám milost, na tebet ocekáváme; budiž ramenem svých každého jitra, a vysvobozením naším v cas ssoužení.
\par 3 Pred zvukem hrmotu rozprchnou se národové, pred vyvýšením tvým budou rozptýleni pohané.
\par 4 A sebrána bude loupež vaše, tak jako sbíráni bývají chroustové; jako pripadají kobylky, tak pripadnou na ni.
\par 5 Vyvýšít se Hospodin, nebo na výsosti prebývá, a naplní Sion soudem a spravedlností.
\par 6 I bude upevnením casu tvých, silou i hojným spasením; moudrost a umení, a bázen Hospodinova poklad tvuj.
\par 7 Aj, rekové jejich naríkali vne, jednatelé pokoje horce plakali.
\par 8 Zpustly silnice, prestali choditi cestou; zrušil prímerí, nevážil sobe mest, za nic položil sobe cloveka.
\par 9 Kvílila a zemdlela zeme, stydeti se musil Libán a usvadl; Sáron ucinen jako poušt, Bázan pak a Karmel oklácen.
\par 10 Jižt povstanu, praví Hospodin, již vyvýšen, již vyzdvižen budu.
\par 11 Pocnouce slámu, porodíte strnište; ohen dýchání vašeho sžíre vás.
\par 12 I budou národové vypálené vápno, trní podtaté, ohnem spáleni budou.
\par 13 Slyšte dalecí, co jsem ucinil, a poznejte blízcí sílu mou.
\par 14 Zdesili se na Sionu hríšníci, podjala hruza pokrytce, rkoucí: Kdož by z nás mohl ostáti pred ohnem sžírajícím? Kdož by z nás mohl ostáti pred plamenem vecným?
\par 15 Ten, kterýž chodí v spravedlnosti, a mluví pravé veci, kterýž pohrdá ziskem z útisku, kterýž otrásá ruce své, aby daru neprijímal, kterýž zacpává uši své, aby neslyšel rady o vražde, a zavírá oci své, aby se na zlé nedíval:
\par 16 Ten na vysokých místech prebývati bude, hradové na skalách útocište jeho, tomu chléb dán bude, vody jeho stálé budou.
\par 17 Krále v okrase jeho uzrí oci tvé, spatrí i zemi dalekou.
\par 18 Srdce tvé premyšlovati bude o strachu, rka: Kdež jest písar, kde výbercí, kde spisovatel velikých domu?
\par 19 Lidu ukrutného neuhledáš, lidu hluboké reci, jíž bys neslýchal, a jazyku cizího, jemuž bys nerozumel.
\par 20 Patr na Sion, mesto slavností našich, oci tvé nechat hledí na Jeruzalém, obydlí pokojné, stánek, kterýž nebude prenešen, kolíkové jeho na veky se nepohnou, a žádný provaz jeho se neztrhá;
\par 21 Proto že velikomocný Hospodin jest nám na míste tom rekami toku širokých, po nemž nepujde lodí s vesly, aniž bárka veliká po nem precházeti bude.
\par 22 Nebo Hospodin jest soudce náš, Hospodin ustanovitel práv našich, Hospodin král náš, ont spasí nás.
\par 23 Osláblit jsou provazové tvoji, aniž budou moci utvrditi sloupu bárky své, ani roztáhnouti plachty, ant již rozdelena bude korist loupeže mnohé; i chromí rozchvátají korist.
\par 24 Aniž dí kdo z obyvatelu: Nemocen jsem. Lid osedlý v nem zprošten bude nepravosti.

\chapter{34}

\par 1 Pristuptež národové, abyste slyšeli, a lidé pozorujte. Nechat slyší zeme i plnost její, okršlek zemský i všeliký plod jeho.
\par 2 Proto že hnev Hospodinuv jest proti všechnem národum, a prchlivost proti všemu vojsku jejich: vyhubí je jako proklaté, a vydá je k zabití.
\par 3 I budou povrženi zbití jejich, a z tel mrtvých jejich vzejde smrad, a rozplynou se hory od krve jejich.
\par 4 Chradnouti bude i všecko vojsko nebeské, a svinuta budou nebesa jako kniha, a všecko vojsko jejich sprchne, jako prší list s vinného kmene, a jako nezralé ovoce s fíku.
\par 5 Nebo opojen jest na nebi mec muj; na Idumejské sstoupí, a na lid, na nejž jsem klatbu vydal, aby trestán byl.
\par 6 Mec Hospodinuv bude plný krve, umastí se tukem a krví beranu a kozlu, tukem ledvin skopových; obet zajisté bude míti Hospodin v Bozra, a zabijení veliké v zemi Idumejské.
\par 7 Sstoupí s nimi i jednorožcové a volcata s voly, i opije se zeme jejich krví, a prach jejich tukem se omastí.
\par 8 Nebo den pomsty Hospodinovy, léto odplacování se, aby mšteno bylo Siona.
\par 9 A obráceni budou potokové její v smolu, a prach její v siru, a zeme její obrátí se v smolu horící.
\par 10 V noci ani ve dne neuhasne, na veky vystupovati bude dým její, od národu až do pronárodu pustá zustane, na veky veku nebude, kdo by šel pres ni.
\par 11 Ale osednou ji pelikán a výr, kalous také a krkavec budou bydliti v ní, a roztáhne po ní šnuru zahanbení a závaží marnosti.
\par 12 Šlechticu jejích volati budou k království, ale nebude tam žádného, nebo všecka knížata její zhynou.
\par 13 A vzroste na palácích jejích trní, koprivy a bodlácí na hradích jejích, a bude príbytkem draku a obydlím sov.
\par 14 Tam se budou potkávati spolu zver s ptactvem, a príšera jedna druhé se ozývati; tam toliko nocní preluda se usadí, a odpocinutí sobe nalezne.
\par 15 Tam se zhnízdí sup, a škreceti bude, a když vysedí, shromáždí je pod stín svuj; tam také shledají se lunáci jeden s druhým.
\par 16 Hledejte v knize Hospodinove, a ctete. Ani jedno z tech nechybí, a jeden každý bez své druže nebude; nebo to ústa Páne prikázala, a duch jeho shromáždí je.
\par 17 Ont zajisté vrže jim losy, a ruka jeho jim rozdelí ji provazcem; až na veky dedicne ji osednou, od národu až do pronárodu v ní prebývati budou.

\chapter{35}

\par 1 Veseliti se budou z toho poušt a pustina, plésati, pravím, bude poušt, a zkvetne jako ruže.
\par 2 Ušlechtile zkvetne, ano i radostne plésati bude s prozpevováním. Sláva Libánská dána jí bude, okrasa Karmelská a Sáronská. Tyt veci uzrí slávu Hospodinovu, dustojnost Boha našeho.
\par 3 Posilntež rukou opuštených, a kolena klesající utvrdte.
\par 4 Rcete tem, kteríž jsou bázlivého srdce: Posilnte se, nebojte se. Aj, Buh váš s pomstou prijde, s odplatou Buh sám prijde, a spasí vás.
\par 5 Tehdáž otevrou se oci slepých, otevrou se též i uši hluchých.
\par 6 Tehdáž poskocí kulhavý jako jelen, a jazyk nemého prozpevovati bude; nebo se vyprýští vody na poušti, a potokové na pustinách.
\par 7 A obrátí se místo vyprahlé v jezero, a žíznivé v prameny vod; v doupatech draku, v pelešech jejich tráva, trtí a sítí.
\par 8 Bude také tam silnice a cesta, kteráž cestou svatou slouti bude. Nepujde po ní necistý, ale bude samých techto; tou cestou jdoucí i nejhloupejší nezbloudí.
\par 9 Nebude tam lva, a lítá zver nebude choditi po ní, aniž tam nalezena bude, ale pujdou po ní ti, jenž budou vysvobozeni.
\par 10 Vykoupení, pravím, Hospodinovi navrátí se, a prijdou na Sion s prozpevováním, a veselé vecné bude na hlave jejich; radosti a veselé dojdou, zámutek pak a úpení utece od nich.

\chapter{36}

\par 1 Stalo se pak ctrnáctého léta kralování Ezechiášova, pritáhl Senacherib král Assyrský proti všechnem mestum Judským hrazeným, a zdobýval jich.
\par 2 I poslal král Assyrský Rabsaka z Lachis do Jeruzaléma k králi Ezechiášovi s vojskem velikým. Kterýž se postavil u struhy rybníka horejšího, pri silnici pole valchárova.
\par 3 Tedy vyšel k nemu Eliakim syn Helkiášuv, kterýž byl správce domu, a Sobna písar, a Joach syn Azafuv, kanclér.
\par 4 I mluvil k nim Rabsaces: Povezte medle Ezechiášovi: Toto praví král veliký, král Assyrský: Jakéž jest to doufání, na kterémž se zakládáš?
\par 5 Rekl jsem: Jiste žet jest vec daremní; radyt jest a síly k válce potrebí. A protož v koho doufáš, že mi se protivíš?
\par 6 Aj, spolehl jsi na hul trtiny té nalomené, na Egypt, na niž zpodeprel-li by se kdo, pronikne ruku jeho, a probodne ji. Takovýt jest Farao král Egyptský všechnem, kteríž v nem doufají.
\par 7 Pakli mi díš: V Hospodinu Bohu svém doufáme: zdaliž on není ten, jehož poboril Ezechiáš výsosti i oltáre, a prikázal Judovi a Jeruzalému, rka: Pred tímto oltárem klaneti se budete?
\par 8 Ale nu, potkej se medle se pánem mým králem Assyrským. Pridámt ješte dva tisíce koní, mužeš-li jen míti, kdo by na nich jeli.
\par 9 Jakž tedy odoláš jednomu knížeti z nejmenších služebníku pána mého, ackoli máš doufání v Egyptu pro vozy a jezdce?
\par 10 Presto, zdali jsem bez Hospodina pritáhl do zeme této, abych ji zkazil? Hospodin rekl mi: Táhni na tu zemi, a zkaz ji.
\par 11 I rekl Eliakim a Sobna a Joach Rabsakovi: Mluv medle k služebníkum svým Syrsky, však rozumíme, a nemluv k nám Židovsky pred lidem tímto, kterýž jest na zdech.
\par 12 I odpovedel Rabsaces: Zdaliž ku pánu tvému a k tobe poslal mne pán muj, abych mluvil slova tato? Však k mužum tem, kteríž jsou na zdech, aby lejna svá jedli, a moc svuj spolu s vámi pili.
\par 13 A tak stoje Rabsaces, volal hlasem velikým Židovsky, a rekl: Slyšte slova krále velikého, krále Assyrského:
\par 14 Toto praví král: Necht vás nesvodí Ezechiáš, nebot nebude moci vyprostiti vás.
\par 15 A necht vám nevelí Ezechiáš doufati v Hospodina, rka: Zajisté vysvobodí nás Hospodin, a nebudet dáno mesto toto v ruku krále Assyrského.
\par 16 Neposlouchejte Ezechiáše. Nebo takto praví král Assyrský: Ucinte mi to k líbosti, a vyjdete ke mne, i bude moci jísti jeden každý z vinice své, a jeden každý z fíku svého, a píti jeden každý vodu z cisterny své,
\par 17 Dokudž neprijdu, a nepoberu vás do zeme podobné zemi vaší, do zeme úrodné, zeme chleba a vinic.
\par 18 Necht vás nesvodí Ezechiáš, rka: Hospodin vysvobodí nás. Zdaliž mohli vysvoboditi bohové národu jeden každý zemi svou z ruky krále Assyrského?
\par 19 Kde jsou bohové Emat a Arfad? Kde jsou bohové Sefarvaim? Zdaliž jsou vysvobodili i Samarí z ruky mé?
\par 20 Kterí jsou mezi všemi bohy tech zemí, ješto by vysvobodili zemi svou z ruky mé? Aby pak Hospodin mel vysvoboditi Jeruzalém z ruky mé?
\par 21 Oni pak mlceli, a neodpovedeli jemu slova. Nebo takové bylo rozkázaní královo, rkoucí: Neodpovídejte jemu.
\par 22 I prišel Eliakim syn Helkiášuv, kterýž byl správcím domu, a Sobna písar, a Joach syn Azafuv, kanclér, k Ezechiášovi, majíce roucha roztržená, a oznámili jemu slova Rabsakova.

\chapter{37}

\par 1 I stalo se, když to uslyšel král Ezechiáš, že roztrhl roucho své, a odel se žíní, a všel do domu Hospodinova.
\par 2 I poslal Eliakima správce domu, a Sobnu písare, a starší z kneží, oblecené v žíne k Izaiášovi proroku, synu Amosovu.
\par 3 Kteríž rekli jemu: Toto praví Ezechiáš: Den úzkosti a útržky i rouhání jest den tento, proto že se priblížil plod k vyjití, ale není síly ku porodu.
\par 4 Ó by slyšel Hospodin Buh tvuj slova Rabsakova, jehož poslal král Assyrský pán jeho, aby utrhal Bohu živému, a pomstil Hospodin Buh tvuj tech slov, kteráž slyšel. Protož pozdvihni modlitby za tento ostatek lidu, kterýž se nalézá.
\par 5 I prišli služebníci krále Ezechiáše k Izaiášovi.
\par 6 Jimž odpovedel Izaiáš: Toto povíte pánu svému: Takto praví Hospodin: Nestrachuj se slov tech, kteráž jsi slyšel, jimiž se mne rouhali služebníci krále Assyrského.
\par 7 Aj, já pustím nan vítr, aby uslyše povest, navrátil se do zeme své, a uciním to, že padne od mece v zemi své.
\par 8 Navrátiv se pak Rabsaces, nalezl krále Assyrského, an dobývá Lebna. Nebo uslyšel, (procež odtrhl od Lachis),
\par 9 Uslyšel, pravím, o Tirhákovi králi Mourenínském, ano pravili: Táhne, aby bojoval s tebou. A však uslyšev to, vždy poslal posly své k Ezechiášovi s temito slovy:
\par 10 Takto povíte Ezechiášovi králi Judskému, rkouce: Necht tebe nesvodí Buh tvuj, v nemž ty doufáš, ríkaje: Nebude dán Jeruzalém v ruku krále Assyrského.
\par 11 Aj, slyšels, co jsou ucinili králové Assyrští všechnem zemím, pohubivše je, a ty bys mel býti vysvobozen?
\par 12 Zdaliž jsou je vysvobodili bohové tech národu, kteréž zahladili predkové moji, Gozana, Charana, Resefa a syny z Eden, kteríž byli v Telasar?
\par 13 Kde jest král Emat, a král Arfad, a král mesta Sefarvaim, Ana i Ava?
\par 14 Protož vzav Ezechiáš list z ruky poslu, prectl jej a vstoupiv do domu Hospodinova, rozvinul jej Ezechiáš pred Hospodinem.
\par 15 A modlil se Ezechiáš Hospodinu, rka:
\par 16 Hospodine zástupu, Bože Izraelský, kterýž sedíš nad cherubíny, ty jsi sám Buh všech království zeme, ty jsi ucinil nebe i zemi.
\par 17 Nakloniž, Hospodine, ucha svého a uslyš; otevri, Hospodine, oci své a pohled; slyš, pravím, všecka slova Senacheribova, kterýž poslal k cinení útržek Bohu živému.
\par 18 Takt jest, Hospodine, žet jsou pohubili králové Assyrští všecky ty krajiny i zemi jejich,
\par 19 A uvrhli bohy jejich do ohne; nebo nebyli bohové, ale dílo ruku lidských, drevo a kámen, protož zahladili je.
\par 20 A nyní, Hospodine Bože náš, vysvobod nás z ruky jeho, atby poznala všecka království zeme, že jsi ty Hospodin sám.
\par 21 Tedy poslal Izaiáš syn Amosuv k Ezechiášovi, rka: Toto praví Hospodin Buh Izraelský: Zac jsi mi se modlil strany Senacheriba krále Assyrského,
\par 22 Totot jest slovo, kteréž mluvil Hospodin o nem: Pohrdá tebou, a posmívá se tobe panna, dcera Sionská, potrásá hlavou za tebou dcera Jeruzalémská.
\par 23 Kohož jsi zhanel? A komus se rouhal? A proti komus povýšil hlasu, a pozdvihl zhuru ocí svých? Proti Svatému Izraelskému.
\par 24 Skrze služebníky své utrhal jsi Pánu, a rekl jsi: Ve množství vozu svých já jsem vytáhl na hory vysoké, na stráne Libánské, a zpodtínám vysoké cedry jeho, i spanilé jedle jeho, a vejdu na nejvyšší kraj jeho do lesu a výborných rolí jeho.
\par 25 Já jsem vykopal a pil vody; nebo jsem vysušil nohama svýma všecky potoky míst obležených.
\par 26 Zdalis neslyšel, že jsem již to dávno ucinil, a ode dnu starých to sformoval? Nyní pak k tomu privozuji, aby v poušt a v hromady rumu mesta hrazená obrácena byla,
\par 27 A jejich obyvatelé ruce oslablé majíc, predešeni a zahanbeni jsouc, byli jako bylina polní, a zelina vzcházející, jako tráva na strechách, a osení rzí zkažené, prvé než by dorostlo.
\par 28 Sedání pak tvé, a vycházení tvé i vcházení tvé znám, i vzteklost tvou proti sobe.
\par 29 Ponevadž ty se vztekáš proti mne, a tvé zpouzení prišlo v uši mé, protož vpustím udici svou v chrípe tvé, a udidla svá v ústa tvá, a odvedu te zase tou cestou, kterouž jsi prišel.
\par 30 A toto mej za znamení: Budete jísti roku prvního, co samo od sebe zroste, též druhého roku, což se samo od sebe zrodí, tretího pak roku budete síti a žíti a štepovati vinice, a jísti ovoce jejich.
\par 31 Ostatek zajisté domu Judova, kterýž pozustal, vpustí zase koreny své hluboce, a vydá užitek zhuru.
\par 32 Nebo z Jeruzaléma vyjdou ostatkové, a zachovaní z hory Siona. Horlivost Hospodina zástupu uciní to.
\par 33 A protož toto praví Hospodin o králi Assyrském: Nevejdet do mesta tohoto, aniž sem strely vstrelí, aniž na ne dotrou pavézníci, aniž udelají u neho náspu.
\par 34 Cestou, kterouž pritáhl, zase navrátí se, a do mesta tohoto nevejde, praví Hospodin.
\par 35 Nebo chrániti budu mesta tohoto, abych je zachoval pro sebe a pro Davida služebníka svého.
\par 36 Tedy vyšel andel Hospodinuv, a zbil v vojšte Assyrském sto osmdesáte a pet tisícu. I vstali velmi ráno, a aj, všickni mrtví.
\par 37 A tak odjel, anobrž utekl, a navrátil se Senacherib král Assyrský, a bydlil v Ninive.
\par 38 I stalo se, když se klanel v chráme Nizrocha boha svého, že Adramelech a Sarasar, synové jeho, zabili jej mecem, a utekli do zeme Ararat. I kraloval Esarchaddon syn jeho místo neho.

\chapter{38}

\par 1 V tech dnech roznemohl se Ezechiáš až k smrti. I prišel k nemu Izaiáš syn Amosuv, prorok, a rekl jemu: Toto praví Hospodin: Zred dum svuj, nebo umreš, a nebudeš živ.
\par 2 I obrátil Ezechiáš tvár svou k stene, a modlil se Hospodinu.
\par 3 A rekl: Prosím, ó Hospodine, rozpomen se nyní, že jsem stále chodil pred tebou v pravde a v srdci uprímém, a že jsem to cinil, což dobrého jest pred ocima tvýma. I plakal Ezechiáš plácem velikým.
\par 4 Tedy stalo se slovo Hospodinovo k Izaiášovi, rkoucí:
\par 5 Jdi a rci Ezechiášovi: Toto praví Hospodin Buh Davida otce tvého: Slyšelt jsem modlitbu tvou, videl jsem slzy tvé; aj, já pridám ke dnum tvým patnácte let.
\par 6 A z ruky krále Assyrského vysvobodím te i mesto toto, a chrániti budu mesta tohoto.
\par 7 A toto budeš míti znamení od Hospodina, že Hospodin uciní vec tuto, kterouž mluvil:
\par 8 Aj, já navrátím zpátkem stín po stupních, po nichž sešel na hodinách slunecných Achasových, o deset stupnu. I navrátilo se slunce o deset stupnu, po týchž stupních, po nichž bylo sešlo.
\par 9 Zapsání Ezechiáše krále Judského, když nemocen byl, a ozdravel po nemoci své:
\par 10 Ját jsem byl rekl v prestrižení dnu svých, že vejdu do bran hrobu, zbaven budu ostatku let svých.
\par 11 Rekl jsem byl, že neuzrím Hospodina, Hospodina v zemi živých, nebudu vídati cloveka více mezi prebývajícími na svete.
\par 12 Prebývání mé pomíjí, a stehuje se ode mne jako stánek pastýrský; prestrihl jsem jako tkadlec život svuj, od trísní odreže mne. Dnes dríve než noc prijde, uciníš mi konec.
\par 13 Predkládal jsem sobe v jitre, že jako lev tak potre všecky kosti mé, dnes dríve než noc prijde, že mi uciníš konec.
\par 14 Jako reráb a vlaštovice pištel jsem, lkal jsem jako holubice, oci mé zhuru pozdvižené byly. Pane, násilé trpím, ó prodliž mi života.
\par 15 Ale cot mám více mluviti? I predpovedel mi, i ucinil, že živ pobudu mimo všecka léta svá po horkosti duše své.
\par 16 Pane, kdo po nich i v nich živi budou, všechnem znám bude život dýchání mého, žes mi zdraví navrátil, a mne pri životu zachoval.
\par 17 Aj, v cas pokoje potkala mne byla horkost nejhorcejší, ale tobe zalíbilo se vytrhnouti duši mou z propasti zkažení, proto že jsi zavrhl za hrbet svuj všecky hríchy mé.
\par 18 Nebo ne hrob oslavuje tebe, ani smrt te chválí, aniž ocekávají ti, jenž do jámy sstupují, pravdy tvé.
\par 19 Ale živý, živý, tent oslavovati bude tebe, jako já dnes, a otec synum v známost uvodí pravdu tvou.
\par 20 Hospodin vysvobodil mne, a protož písne mé zpívati budeme po všecky dny života našeho v dome Hospodinove.
\par 21 Rekl pak byl Izaiáš: Nechat vezmou hrudu suchých fíku, a priloží na vred, a zdráv bude.
\par 22 I rekl byl Ezechiáš: Jaké jest znamení, že vstoupím do domu Hospodinova?

\chapter{39}

\par 1 Toho casu poslal Merodach Baladan syn Baladanuv, král Babylonský, list a dary Ezechiášovi, když uslyšel, že nemocen byv, zase ozdravel.
\par 2 I zradoval se z toho Ezechiáš, a ukázal jim dum klénotu svých, stríbra a zlata a vonných vecí, a olej nejvýbornejší, tolikéž dum zbroje své, a cožkoli mohlo nalezeno býti v pokladích jeho. Niceho nebylo, cehož by jim neukázal Ezechiáš v dome svém i ve všem panství svém.
\par 3 V tom prišel prorok Izaiáš k králi Ezechiášovi, a rekl jemu: Co pravili ti muži? A odkud prišli k tobe? I odpovedel Ezechiáš: Z zeme daleké prišli ke mne, z Babylona.
\par 4 Rekl ješte: Co jsou videli v dome tvém? Odpovedel Ezechiáš: Všecko, což jest v dome mém, videli. Niceho není v pokladích mých, cehož bych jim neukázal.
\par 5 Tedy rekl Izaiáš Ezechiášovi: Slyšiž slovo Hospodina zástupu:
\par 6 Aj, dnové prijdou, že odneseno bude do Babylona, cožkoli jest v dome tvém, a cožkoli nachovali otcové tvoji, až do tohoto dne; nezustanet niceho, (praví Hospodin).
\par 7 I syny tvé také, kteríž pojdou z tebe,kteréž zplodíš, poberou, a budou komorníci pri dvore krále Babylonského.
\par 8 Tedy rekl Ezechiáš Izaiášovi: Dobrét jest slovo Hospodinovo, kteréž jsi mluvil. (A doložil): Proto že pokoj a pravda bude za dnu mých.

\chapter{40}

\par 1 Potešujte, potešujte lidu mého, dí Buh váš.
\par 2 Mluvte k srdci Jeruzaléma, a ohlašujte jemu, že se již doplnil cas uložený jeho, že jest odpuštena nepravost jeho, a že vzal z ruky Hospodinovy dvojnásobne za všecky hríchy své.
\par 3 Hlas volajícího: Pripravtež na poušti cestu Hospodinovu, prímou ucinte na pustine stezku Boha našeho.
\par 4 Každé údolí at jest vyvýšeno, a všeliká hora i pahrbek at jest snížen; což jest krivého, at jest prímé, a místa nerovná at jsou rovinou.
\par 5 Nebo se zjeví sláva Hospodinova, a uzrí všeliké telo spolu, že ústa Hospodinova mluvila.
\par 6 Hlas rkoucího: Volej. I rekl: Co mám volati? To, že všeliké telo jest tráva, a všeliká vzácnost jeho jako kvet polní.
\par 7 Usychá tráva, kvet prší, jakž vítr Hospodinuv povane na nej. V pravdet jsou lidé ta tráva.
\par 8 Usychá tráva, kvet prší, ale slovo Boha našeho zustává na veky.
\par 9 Na horu vysokou vystup sobe, Sione, zvestovateli vecí potešených, povyš mocne hlasu svého, Jeruzaléme, zvestovateli vecí potešených, povyš, aniž se boj. Rci mestum Judským: Aj, Buh váš.
\par 10 Aj, Panovník Hospodin proti silnému prijde, a ráme jeho panovati bude nad ním;aj, mzda jeho s ním, a dílo jeho pred ním.
\par 11 Jako pastýr stádo své pásti bude, do nárucí svého shromáždí jehnátka, a v klíne svém je ponese, brezí pak poznenáhlu povede.
\par 12 Kdo zmeril hrstí svou vody, a nebesa pídí rozmeril? A kdo zmeril merou prach zeme, a zvážil na váze hory, a pahrbky na závaží?
\par 13 Kdo vystihl ducha Hospodinova, a rádcím jeho byl, že by mu oznámil?
\par 14 S kým se radil, že by mu pridal srozumení, a naucil jej stezce soudu, a vyucil jej umení, a cestu všelijaké rozumnosti jemu v známost uvedl?
\par 15 Aj, národové jako krupe od okova, a jako prášek na vážkách se pocítají, ostrovy jako nejmenší vec zachvacuje.
\par 16 Ani Libán nepostacil by k zanícení ohne, a živocichové jeho nepostacili by k zápalné obeti.
\par 17 Všickni národové jsou jako nic pred ním, za nic a za marnost pokládají se u neho.
\par 18 K komu tedy pripodobníte Boha silného? A jaké podobenství prirovnáte jemu?
\par 19 Jakžkoli rytinu lící remeslník, a zlatník zlatem ji potahuje, a retízky stríbrné k ní slévá;
\par 20 A ten, kterýž pro chudobu nemá co obetovati, drevo, kteréž by nepráchnivelo, vybírá, a remeslníka umelého sobe hledá k pristrojení rytiny, aby se nepohnula.
\par 21 Zdaliž nevíte? Zdaliž neslýcháte? Zdaliž se vám nezvestuje od pocátku? Zdaliž nesrozumíváte z základu zeme?
\par 22 Ten, kterýž sedí nad okršlkem zeme, jejížto obyvatelé jako kobylky, kterýž rozprostrel jako kortýnu nebesa, a roztáhl je jako stánek k prebývání,
\par 23 Ont privodí knížata na nic, soudce zemské jako nic rozptyluje,
\par 24 Tak že nebývají štípeni ani sáti, aniž korene pouští do zeme parez jejich. Nebo jakž jen zavane na ne, hned usychají, a vicher jako plevu zanáší je.
\par 25 K komu tedy pripodobníte mne, abych podobný byl jemu, praví Svatý?
\par 26 Pozdvihnete zhuru ocí svých, a vizte, kdo to stvoril? Kdo vyvodí v poctu vojsko jejich, a všeho toho zejména povolává? Vedlé množství síly a veliké moci ani jedno z nich nehyne.
\par 27 Procež tedy ríkáš, Jákobe, a mluvíš, Izraeli: Skrytat jest cesta má pred Hospodinem, a pre má pred Boha mého neprichází?
\par 28 Zdaliž nevíš, zdaž jsi neslýchal, že Buh vecný Hospodin, kterýž stvoril konciny zeme, neustává ani zemdlívá, a že vystižena býti nemuže moudrost jeho?
\par 29 On dává ustalému sílu, a tomu, ješto žádné síly nemá, moci hojne udílí.
\par 30 Ustává a umdlévá mládež, a mládenci težce klesají,
\par 31 Ale ti, jenž ocekávají na Hospodina, nabývají nové síly. Vznášejí se perím jako orlice; beží, a však neumdlévají, chodí, a neustávají.

\chapter{41}

\par 1 Umlknetež prede mnou ostrovové, a národové nechat se ssilí, nechat pristoupí, a tu at mluví. Pristupmež spolu k soudu.
\par 2 Kdo vzbudil od východu spravedlivého, aby ho následoval? Kdo jemu podmanil národy, a zpusobil, aby nad králi panoval, vydav je jako prach meci jeho, a jako plevy rozptýlené lucišti jeho?
\par 3 Sháneje se s nimi, prošel pokojne cestou, po níž nohama svýma nechodíval.
\par 4 Kdo to spravil a ucinil, povolávaje rodin od pocátku? Já Hospodin, první i poslední, já sám.
\par 5 Videli ostrovové, a ulekli se, konciny zeme predesily se, shromáždili se a sešli.
\par 6 Jeden druhému pomáhal, a bratru svému ríkal: Posiln se.
\par 7 A tak posilnoval tesar zlatníka, vyhlazujícího kladivo tlucením na nákovadlí, ríkaje: K sletování toto dobré jest. I utvrdil to hrebíky, aby se nepohnulo.
\par 8 Ale ty, Izraeli služebníce muj, ty Jákobe, kteréhož jsem vyvolil, síme Abrahama, prítele mého,
\par 9 Ty, kteréhož jsem vychvátil od koncin zeme, nýbrž pominuv prednejších jejich, povolal jsem te, rka tobe: Služebník muj jsi, vyvolil jsem te, aniž jsem zavrhl tebe.
\par 10 Nebojž se, nebo jsem já s tebou; nestrachujž se, nebo já jsem Buh tvuj. Posilním te, a pomáhati budu tobe, a podpírati te budu pravicí spravedlnosti své.
\par 11 Aj, zastydí se, a zahanbeni budou všickni, kteríž se zlobí proti tobe; v nic obráceni budou, a zahynou ti, kteríž tobe odporují.
\par 12 Hledal-li bys jich, nenalezneš jich. Ti, kteríž se s tebou nesnadní, v nic obráceni budou, a na nic prijdou ti, jenž s tebou válcí.
\par 13 Nebo já, Hospodin Buh tvuj, ujal jsem te za tvou pravici, a pravímt: Neboj se, já tobe pomáhati budu.
\par 14 Neboj se, cervícku Jákobuv, hrstko Izraelova, já spomáhati budu tobe, praví Hospodin, a vykupitel tvuj, Svatý Izraelský.
\par 15 Aj, ucinil jsem te jako smyk i s zuby novými po obou stranách; pomlátíš hory, a zetreš je, a s pahrbky jako s plevami naložíš.
\par 16 Preveješ je, v tom je vítr zachvátí, a vicher rozptýlí, ty pak plésati budeš v Hospodinu, v Svatém Izraelském chlubiti se budeš.
\par 17 Chudé a nuzné, kteríž hledají vody, an jí není, jejichž jazyk žízní prahne, já Hospodin vyslyším, já Buh Izraelský neopustím jich.
\par 18 Vyvedu na vysokých místech reky, a u prostred rovin studnice; obrátím poušt v jezero vod, a zemi vyprahlou v prameny vod.
\par 19 Nasadím na poušti cedru, výborných cedru, myrtoví, a olivoví; vysadím poušt jedlovím, jilmem, též i pušpanem,
\par 20 Aby videli, a poznali, a rozvažujíce, srozumeli, že to ruka Hospodinova ucinila, a Svatý Izraelský že to stvoril.
\par 21 Vedte pri svou, praví Hospodin, ukažte mocné duvody své, dí král Jákobuv.
\par 22 Nechat pristoupí, a necht nám oznámí to, což se státi má. Veci prvé stalé oznamte, abychom povážili v srdci svém, a poznali cíl jejich, aneb aspon co jest budoucího, povezte nám.
\par 23 Oznamte, co se budoucne státi má, a poznáme, že bohové jste. Pakli, ucinte neco dobrého neb zlého, abychom se desili, když bychom to videli spolu.
\par 24 Aj, vy naprosto nic nejste, a dílo vaše tolikéž nic není; protož ohavnýt jest, kdo vás zvoluje.
\par 25 Vzbudím od pulnoci, ten pritáhne, od východu slunce, ten svolá ve jménu mém, a oborí se na knížata jako na bláto, a pošlapá je jako hrncír hlinu.
\par 26 Kdo oznámí od pocátku, abychom vedeli, aneb od starodávna, abychom rekli: Práv jest? Naprosto žádného není, kdo by oznámil, aniž jest kdo, ješto by se dal slyšeti, aneb ješto by slyšel reci vaše.
\par 27 Já jsem první, kterýž Sionu predpovídám: Aj, aj, ted jsou; a Jeruzalému: Zvestovatele potešených vecí dám.
\par 28 Nebo vidím, že není žádného,není žádného mezi nimi rozumného; ac se jich otazuji, však neodpovídají slova.
\par 29 Aj, všickni ti jsou marnost, za nic nestojí dílo jejich, vítr a daremní vec jsou slitiny jejich.

\chapter{42}

\par 1 Aj, služebník muj, na kteréhož se zpodepru, vyvolený muj, jehož libuje duše má. Ducha svého dám jemu, ont soud národum vynášeti bude.
\par 2 Nebude kriceti, ani se vyvyšovati, ani slyšán bude vne hlas jeho.
\par 3 Trtiny nalomené nedolomí, a lnu kourícího se neuhasí, ale soud podlé pravdy vynášeti bude.
\par 4 Nebude neochotný, ani prísný, dokudž soudu na zemi nevykoná, a ucení jeho ostrovové ocekávati budou.
\par 5 Tak praví Buh silný Hospodin, kterýž stvoril nebesa, a roztáhl je, kterýž rozšíril zemi, i to, což z ní pochází, kterýž dává dýchání lidu na ní, a ducha tem, jenž chodí po ní.
\par 6 Já Hospodin povolal jsem te v spravedlnosti, a ujal jsem te za ruku tvou; protož ostríhati te budu, a dám te v smlouvu lidu, a za svetlo národum,
\par 7 Abys otvíral oci slepé, a vyvodil z žaláre vezne, a z vezení ty, kteríž sedí ve tmách.
\par 8 Já jsem Hospodin, tot jest jméno mé, a slávy své jinému nedám, ani chvály své rytinám.
\par 9 Aj, prvnejší veci prišly, a i nové predpovídaje, dríve než se zacnou, dám o nich slyšeti vám.
\par 10 Zpívejte Hospodinu písen novou, chvála jeho jest od koncin zeme, kteríž se plavíte po mori, i všecko, což v nem jest, ostrovové i obyvatelé jejich.
\par 11 Pozdvihnete hlasu pustiny i mesta její, i vsi, v nichž bydlí Cedar, prokrikujte obyvatelé skal, s vrchu hor volejte.
\par 12 Vzdejte slávu Hospodinu, a chválu jeho na ostrovích zvestujte.
\par 13 Hospodin jako silný rek vyjde, jako muž válecný rozhorlí se, troubiti, anobrž i prokrikovati bude, a proti neprátelum svým zmužile sobe pocínati, rka:
\par 14 Mlcel jsem dosti dlouho, cinil jsem se neslyše, zdržoval jsem se, ale již jako pracující ku porodu kriceti budu, pohubím a sehltím vše pojednou.
\par 15 V pustinu obrátím hory i pahrbky, a všelikou bylinu jejich usuším, a obrátím reky v ostrovy, a jezera vysuším.
\par 16 I povedu slepé po ceste, kteréž neznali, a po stezkách, kterýchž neumeli, provedu je; obrátím pred nimi tmu v svetlo, a co nerovného, v rovinu. Tot jest, což jim uciním, a neopustím jich.
\par 17 Obrátí se zpet, zahanbeni budou ti, kteríž doufají v rytinu, kteríž ríkají slitinám: Vy jste bohové naši.
\par 18 Ó hluší, slyštež, a vy slepí, prohlédnete, abyste videli.
\par 19 Kdo jest to slepý, jediné služebník muj? A hluchý, než posel muj, kteréhož posílám? Kdo slepý tak jako dokonalý? Slepý, pravím, jako služebník Hospodinuv?
\par 20 Hlede na mnohé veci, však nesrozumívá; otevrené maje uši, však neslyší.
\par 21 Melt jest Hospodin líbost v nem pro spravedlnost svou, zvelebil jej zákonem, a slavného ucinil.
\par 22 Že pak lid tento obloupený jest a potlacený, jehožto mládence, což jich koli, jímají a do žaláru skrývají, že jsou dáni v loupež, aniž jest, kdo by je vytrhl, v rozchvátání, aniž jest, kdo by rekl: Navrat zase,
\par 23 Kdo z vás ušima pozoruje toho, srozumívá tomu, aby se bedliveji chtel míti napotom?
\par 24 Kdo vydal v potlacení Jákoba, a Izraele loupežníkum? Zdali ne Hospodin, proti nemuž jsme zhrešili? Nebo nechteli po cestách jeho choditi, aniž poslouchali zákona jeho.
\par 25 A protož vylil na nej s prchlivostí hnev svuj, a násilé boje, a zapálil jej vukol, a však nepoznal toho. Zapálil jej, pravím, a však nepripustil toho k srdci.

\chapter{43}

\par 1 Ale nyní takto praví Hospodin stvoritel tvuj, ó Jákobe, a ucinitel tvuj, ó Izraeli: Neboj se, nebo vykoupil jsem te, a povolal jsem te jménem tvým. Muj jsi ty.
\par 2 Když pujdeš pres vody, s tebou budu, pakli pres reky, neprikvací te; pujdeš-li pres ohen, nespálíš se, aniž plamen chytí se tebe.
\par 3 Nebo já Hospodin Buh tvuj, Svatý Izraelský, jsem spasitel tvuj. Dal jsem na výplatu za tebe Egypt, zemi Mourenínskou a Sábu místo tebe.
\par 4 Hned jakž jsi drahým ucinen pred ocima mýma, zveleben jsi, a já jsem te miloval; protož dal jsem lidi za tebe, a národy za život tvuj.
\par 5 Nebojž se, nebo já s tebou jsem. Od východu zase privedu síme tvé, a od západu shromáždím te.
\par 6 Dím pulnocní strane: Navrat, a polední: Nezbranuj. Prived zase syny mé zdaleka, a dcery mé od koncin zeme,
\par 7 Každého toho, jenž se nazývá jménem mým, a kteréhož jsem k sláve své stvoril, jejž jsem sformoval, a kteréhož jsem ucinil.
\par 8 Vyved lid slepý, kterýž již má oci, a hluché, kteríž již mají uši.
\par 9 Všickni národové nechat se spolu shromáždí, a sberou se lidé. Kdo jest mezi nimi, ješto by to zvestoval, a to, což se predne státi má, aby oznámil nám? Necht vystaví svedky své, a spravedlivi budou, aneb at slyší, a reknou: Pravdat jest.
\par 10 Vy svedkové moji jste, praví Hospodin, a služebník muj, kteréhož jsem vyvolil, tak že mužete vedeti, a mne veriti, i rozumeti, že já jsem, a že prede mnou nebyl sformován Buh silný, aniž po mne bude.
\par 11 Já, já jsem Hospodin, a žádného není krome mne spasitele.
\par 12 Já oznamuji, i vysvobozuji, jakž predpovídám, a ne nekdo mezi vámi z cizích bohu, a vy mi toho svedkové jste, praví Hospodin, že já Buh silný jsem.
\par 13 Ješte prvé nežli den byl, já jsem, a není žádného, kdož by vytrhl z ruky mé. Když co delám, kdo ji odvrátí?
\par 14 Takto praví Hospodin vykupitel váš, Svatý Izraelský: Pro vás pošli do Babylona, a sházím závory všecky, i Kaldejské s lodimi veselými jejich.
\par 15 Já jsem Hospodin svatý váš, stvoritel Izraele, král váš.
\par 16 Takto praví Hospodin, kterýž zpusobuje na mori cestu, a na prudkých vodách stezku,
\par 17 Kterýž vyvodí vozy a kone, vojsko i sílu, ciní, že v náhle padají, až i povstati nemohou, hasnou, jako knot hasne:
\par 18 Nezpomínejte na první veci, a na starodávní se neohlédejte.
\par 19 Aj, já uciním vec novou, a tudíž se zjeví. Zdaliž o tom nezvíte? Nadto zpusobím na poušti cestu, a na pustinách reky.
\par 20 I slaviti mne bude zver polní, drakové i sovy, že jsem vyvedl na poušti vody a reky na pustinách, abych dal nápoj lidu svému, vyvolenému svému.
\par 21 Lid, kterýž nastrojím sobe, chválu mou vypravovati bude,
\par 22 Ponevadž jsi mne nevzýval, ó Jákobe, nýbrž stesklot se se mnou, ó Izraeli.
\par 23 Neprivedl jsi mi hovádka k zápalum svým, a obetmi svými neuctils mne; nenutil jsem te, abys mi sloužil obetmi suchými, aniž jsem te tím obtežoval, abys mi kadil.
\par 24 Nekoupil jsi mi za peníze vonných vecí, ani tukem obetí svých zavlažil jsi mne, ale zamestknal jsi mne hríchy svými, a obtížils mne nepravostmi svými.
\par 25 Já, já sám shlazuji prestoupení tvá pro sebe, a na hríchy tvé nezpomínám.
\par 26 Prived mi ku pameti, sudme se spolu; oznam ty, podlé ceho bys mohl spravedliv býti.
\par 27 Otec tvuj první zhrešil, a ucitelé tvoji prestoupili proti mne.
\par 28 A protož smeci knížata z míst svatých, a vydám v prokletí Jákoba, a Izraele v pohanení.

\chapter{44}

\par 1 A však nyní slyš, Jákobe, služebníce muj, a Izraeli, kteréhož jsem vyvolil.
\par 2 Toto dí Hospodin, kterýž te ucinil, a sformoval hned od života matky, a spomáhá tobe: Neboj se, služebníce muj, Jákobe, a uprímý, kteréhož jsem vyvolil.
\par 3 Nebo vyleji vody na žíznivého, a potoky na vyprahlost; vyleji Ducha svého na síme tvé, a požehnání své na potomky tvé.
\par 4 I porostou jako mezi bylinami, jako vrbí vedlé tekutých vod.
\par 5 Tento dí: Hospodinuv já jsem, a onen nazuve se jménem Jákobovým, a jiný zapíše se rukou svou Hospodinu, a jménem Izraelským jmenovati se bude.
\par 6 Takto praví Hospodin, král Izraeluv a vykupitel jeho, Hospodin zástupu: Já jsem první, a já poslední, a krome mne není žádného Boha.
\par 7 Nebo kdo tak jako já ohlašuje a oznamuje to, aneb sporádá mi to hned od toho casu, jakž jsem rozsadil lidi na svete? A kdo to, což prijíti má, oznámiti jim mohou?
\par 8 Nebojtež se, ani lekejte. Zdali hned zdávna nepovedel jsem tobe a neoznámil, cehož vy sami mne svedkové jste? Zdaliž jest Buh krome mne? Nenít jiste žádné skály, ját o žádné nevím.
\par 9 Ti, kteríž formují rytiny, všickni nic nejsou; tolikéž ty milostné jejich nic neprospívají. Cehož ony sobe samy svedectvím jsou; nic nevidí, aniž ceho znají, aby se stydeti mohly.
\par 10 Kdo formuje Boha silného a rytinu slévá, k nicemu se to nehodí.
\par 11 Aj, všickni, i ti, kteríž se k nemu priúcastnují, zahanbeni budou, ovšem remeslníci ti nad jiné lidi; byt se pak všickni shromáždili a postavili, strašiti se musejí, a spolu zahanbeni budou.
\par 12 Kovár pochyte železo, delá pri uhlí, a kladivy formuje modlu. Když ji delá vší silou svou, ješte k tomu lacní až do zemdlení, aniž se vody napije, až i ustává.
\par 13 Tesar roztáhna šnuru, znamenává ji hrudkou, spravuje ji úhelnicemi, a kružidlem rozmeruje ji, až ji udelá tvárnost muže mající a podobnost krásy cloveka, aby sedela doma.
\par 14 Nasekaje sobe cedru, vezme také cypriš a dub aneb to, kteréž jest nejcelistvejší mezi drívím lesním; i javor štepuje, a déšt jej k zrostu privozuje.
\par 15 I bývá cloveku k topení; nebo vezma z neho, zhrívá se. Roznecuje také ohen, aby napekl chleba. Mimo to udelá sobe boha, a klaní se jemu; udelá z neho rytinu, a kleká pred ní.
\par 16 Cástku jeho pálí ohnem, pri druhé cástce jeho maso jí, pece peceni, a nasycen bývá. Zhrívá se také, a ríká: Aha, zhrel jsem se, videl jsem ohen.
\par 17 Z ostatku pak jeho udelá boha, rytinu svou, pred níž kleká, a klaní se, a modlí se jí, rka: Vysvobod mne, nebo Buh silný muj jsi.
\par 18 Neznají ani soudí, proto že zaslepil oci jejich, aby nevideli, a srdce jejich, aby nerozumeli.
\par 19 Aniž považují toho v srdci svém. Není umení, není rozumu, aby kdo rekl: Díl z neho spálil jsem ohnem, a pri uhlí jeho napekl jsem chleba, pekl jsem maso, a jedl jsem, a mám z ostatku jeho ohavnost udelati, a pred špalkem dreveným klekati?
\par 20 Popelem se pase takový, srdce svedené sklonuje jej, aby nemohl osvoboditi duše své, ani ríci: Není-liž omylu v predsevzetí mém?
\par 21 Pamatujž na to, Jákobe a Izraeli, proto že jsi ty služebník muj. Já jsem te sformoval, služebník muj jsi, Izraeli, nebudeš u mne v zapomenutí.
\par 22 Zahladím jako hustý oblak prestoupení tvá, a jako mrákotu hríchy tvé; navratiž se ke mne, nebo jsem te vykoupil.
\par 23 Prozpevujte nebesa, nebo Hospodin to ucinil; zvucte nižiny zeme, zvucne prozpevujte hory, les i všeliké dríví v nem, nebo vykoupil Hospodin Jákoba, a v Izraeli sebe oslavil.
\par 24 Takto praví Hospodin vykupitel tvuj, a ten, kterýž te sformoval hned od života matky: Já Hospodin ciním všecko, roztahuji nebesa sám, rozprostírám zemi mocí svou.
\par 25 Rozptyluji znamení lháru, a z hadacu blázny delám; obracím moudré nazpet, a umení jejich v bláznovství.
\par 26 Potvrzuji slova služebníka svého, a radu poslu svých vykonávám. Kterýž dím o Jeruzalému: Bydleno bude v nem, a o mestech Judských: Vystavena budou, nebo pustiny jejich vzdelám.
\par 27 Kterýž dím hlubine: Vyschni, nebo potoky tvé vysuším.
\par 28 Kterýž dím o Cýrovi: Pastýr muj, nebo všelikou vuli mou vykoná, a rekne Jeruzalému: Zase vystaven bud, a chrámu: Založen bud.

\chapter{45}

\par 1 Takto praví Hospodin pomazanému svému Cýrovi, jehož pravici zmocním, a národy pred ním porazím, a bedra králu rozpáši, a zotvírám pred ním vrata, a brány nebudou zavírány:
\par 2 Já pred tebou pujdu, a cesty krivé zprímím, vrata medená potru, a závory železné posekám.
\par 3 A dám tobe poklady skryté, a klénoty schované, abys poznal, že já jsem Hospodin Buh Izraelský, kterýž te ze jména volám.
\par 4 Pro služebníka svého Jákoba a Izraele vyvoleného svého jmenoval jsem te jménem tvým, i príjmím tvým, ackoli mne neznáš.
\par 5 Já jsem Hospodin, a není žádného více, krome mne není žádného Boha. Prepásal jsem te, ackoli mne neznáš,
\par 6 Aby poznali od východu slunce i od západu, že není žádného krome mne. Ját jsem Hospodin, a není, žádného více.
\par 7 Kterýž formuji svetlo, a tvorím tmu, pusobím pokoj, a tvorím zlé, já Hospodin ciním to všecko.
\par 8 Rosu dejte nebesa s hury, a nejvyšší oblakové dštete spravedlnost; otevri se zeme, a at vzejde spasení, a spravedlnost at spolu vykvete. Já Hospodin zpusobím to.
\par 9 Beda tomu, kdož se v odpory dává s tím, jenž jej sformoval, jsa strep jako jiné strepiny hlinené. Zdaliž dí hlina hrncíri svému: Což deláš? Dílo tvé zajisté nicemné jest.
\par 10 Beda tomu, kterýž ríká otci: Co zplodíš? A žene: Co porodíš?
\par 11 Takto praví Hospodin, Svatý Izraelský, a kterýž jej sformoval: Budoucí-liž veci na mne vyzvídati chcete, o synech mých a díle rukou mých mne vymerovati?
\par 12 Já jsem ucinil zemi, a cloveka na ní stvoril; já jsem, jehož ruce roztáhly nebesa, a všemu vojsku jejich rozkazuji.
\par 13 Já vzbudím jej v spravedlnosti, a všecky cesty jeho zprímím. Ont vzdelá mesto mé, a zajaté mé propustí, ne ze mzdy, ani pro dar, praví Hospodin zástupu.
\par 14 Takto praví Hospodin: Práce Egyptská, a kupectví Mourenínská, a Sabejští, muži veliké postavy, k tobe prijdou, a tvoji budou. Za tebou se poberou, v poutech pujdou, tobe se klaneti, a tobe se koriti budou, ríkajíce: Toliko u tebe jest Buh silný, a nenít žádného více krome toho Boha.
\par 15 (Jiste ty jsi Buh silný, skrývající se, Buh Izraelský, spasitel.)
\par 16 Všickni onino se zastydí, a zahanbeni budou, spolu odejdou s hanbou cinitelé obrazu;
\par 17 Ale Izrael spasen bude skrze Hospodina spasením vecným. Nebudete zahanbeni, ani v lehkost uvedeni na veky veku.
\par 18 Nebo tak praví Hospodin stvoritel nebes, (ten Buh, kterýž sformoval zemi a ucinil ji, kterýž utvrdil ji, ne na prázdno stvoril ji, k bydlení sformoval ji): Já jsem Hospodin, a není žádného více.
\par 19 Nemluvím tajne v míste zemském tmavém, neríkám semeni Jákobovu nadarmo: Hledejte mne. Já Hospodin mluvím spravedlnost, a zvestuji veci pravé.
\par 20 Shromaždte se a pridte, približte se spolu vy, kteríž jste pozustali mezi pohany. Nic neznají ti, kteríž se s drevem rytiny své nosí; nebo se modlí bohu, kterýž nemuže vysvoboditi.
\par 21 Oznamte a privedte i jiné, a nechat spolu v radu vejdou, a ukáží, kdo to od starodávna predpovedel, a hned zdávna oznámil? Zdali ne já Hospodin? Nebot není žádného jiného Boha krome mne, není Boha silného, spravedlivého, a spasitele žádného krome mne.
\par 22 Obrattež zretel ke mne, abyste spaseny byly všecky konciny zeme; nebo já jsem Buh silný, a není žádného více.
\par 23 Skrze sebe prisáhl jsem, vyšlo z úst mých slovo spravedlnosti, kteréž nepujde na zpet: Že se mne skláneti bude všeliké koleno, a prisahati každý jazyk,
\par 24 Ríkaje: Toliko v Hospodinu mám všelijakou spravedlnost a sílu, a až k samému prijde; ale zahanbeni budou všickni, kteríž se koli zlobí proti nemu.
\par 25 V Hospodinu ospravedlneni budou, a chlubiti se všecko síme Izraelovo.

\chapter{46}

\par 1 Klesl Bél, padl Nébo, modly jejich octnou se na hovadech a na dobytku. Tím zajisté, což vy nosíváte, budou náramne obtížena až do ustání.
\par 2 Klesly, padly spolu, aniž budou moci retovati bremene, nýbrž i oni sami v zajetí odejdou.
\par 3 Slyšte mne, dome Jákobuv, a všickni ostatkové domu Izraelova, kteréž pestuji hned od života, kteréž nosím hned od narození:
\par 4 Až i do starosti já sám, nýbrž až do šedin já ponesu; já jsem vás ucinil, a já nositi budu, já, pravím, ponesu a vysvobodím.
\par 5 K komu mne pripodobníte a prirovnáte, aneb podobna uciníte, abychom sobe podobní byli?
\par 6 Ti, kteríž marne vynakládají zlato z mešce, a stríbro na vážkách váží, najímají ze mzdy zlatníka, aby udelal z neho boha, pred nímž padají a sklánejí se.
\par 7 Nosí jej na rameni, pestují se s ním, a stavejí ho na míste jeho, i stojí, z místa svého se nehýbaje. Volá-li kdo k nemu, neozývá se, aniž jej z úzkosti jeho vysvobozuje.
\par 8 Pamatujtež na to, a zastydte se; pripustte to, ó zproneverilí, k srdci.
\par 9 Rozpomente se na první veci od veku stalé, nebo já jsem Buh silný, a není žádného více Boha, aniž jest mne podobného.
\par 10 Kterýž oznamuji pri pocátku dokonání, a hned zdaleka to, což se ješte nestalo; reknu-li co, rada má se koná, a vše, což mi se líbí, ciním.
\par 11 Kterýž zavolám od východu ptáka, z zeme daleké toho, kterýž by vykonal uložení mé. Rekl jsem, a dovedu toho, umínil jsem, a vykonám to.
\par 12 Slyšte mne, vy urputného srdce, kteríž jste dalecí od spravedlnosti.
\par 13 Ját zpusobím, aby se priblížila spravedlnost má. Nebudet prodlévati, aniž spasení mé bude meškati; nebo složím v Sionu spasení, a v Izraeli slávu svou.

\chapter{47}

\par 1 Sstup a sed v prachu, panno dcero Babylonská, sed na zemi, a ne na trunu, dcero Kaldejská; nebo nebudou te více nazývati milostnou a rozkošnou.
\par 2 Chyt se žernovu, a mel mouku; odkrej kadere své, obnaž nohy, odkrej hnáty, bred pres reky.
\par 3 Odkryta bude hanba tvá, a ukáže se mrzkost tvá. Mstíti budu, a nedám sobe žádnému prekaziti,
\par 4 Praví vykupitel náš, jehož jméno jest Hospodin zástupu, Svatý Izraelský.
\par 5 Sediž mlce, a vejdi do tmy, dcero Kaldejská; nebo nebudou te více nazývati paní království.
\par 6 Rozhneval jsem se na lid svuj, v lehkost jsem uvedl dedictví své, a vydal jsem je v ruku tvou, a neprokázalas k nim milosrdenství. Starce jsi obtížila velmi jhem svým,
\par 7 A ríkalas: Na veky budu paní, a nikdy jsi nesložila tech vecí v srdci svém, aniž jsi pamatovala na cíl jeho.
\par 8 Protož nyní slyšiž toto, ó rozkošná, (kteráž sedíš bezpecne, a ríkáš v srdci svém: Já jsem, a není krome mne žádné; nebudut vdovou, aniž zvím o sirobe),
\par 9 Že obé to prijde na te pojednou dne jednoho, i siroba i vdovství. Všecko zúplna prijde na te, i na množství kouzlu tvých, a na velikou moc cáru tvých.
\par 10 Nebo doufáš v zlost svou, a ríkáš: Žádný mne nevidí. Moudrost tvá a umení tvé, to te prevrátilo, abys ríkala v srdci svém: Já jsem, a není krome mne žádné.
\par 11 A protož prijde na te zlé, jehož východu neznáš, a pripadne na te bída, kteréž nebudeš moci se odžehnati, a prijde na te pojednou hrozné zpuštení, než zvíš.
\par 12 Postav se nyní s cáry svými, a s množstvím kouzlu svých, jimiž jsi se zamestknávala od mladosti své, budeš-li moci co prospeti, aneb snad zmocniti se.
\par 13 Ustáváš s množstvím rad svých. Nechat se nyní postaví hvezdári, kteríž spatrují hvezdy, a oznamují na každý mesíc, a vysvobodí te z toho, což prijíti má na te.
\par 14 Aj, jako pleva jsou, ohen popálí je, nevychvátí ani sami sebe z prudkosti plamene; žádného uhlí nezustane k zhrívání se, ani ohne, aby se mohlo posedeti u neho.
\par 15 Takt se stane i kupcum tvým, jimiž jsi se zamestknávala od mladosti své. Jeden každý svou stranou pujde, aniž bude, kdo by te vysvobodil.

\chapter{48}

\par 1 Slyštež to, dome Jákobuv, kteríž se nazýváte jménem Izraelovým, a z vod Judových jste pošli, kteríž prisaháte ve jménu Hospodinovu, a Boha Izraelského pripomínáte, však ne v pravde, ani v spravedlnosti,
\par 2 Ackoli od mesta svatého se jmenujete, a na Boha Izraelského, jehož jméno jest Hospodin zástupu, zpoléháte.
\par 3 Predešlé veci zdávna jsem oznamoval, a což vyšlo z úst mých, i což jsem ohlašoval, brzce jsem ciníval, a stávalo se.
\par 4 Vedel jsem, že jsi zatvrdilý, a houžev železná šíje tvá, a celo tvé ocelivé.
\par 5 A protožt jsem oznamoval z dávna, prvé než pricházelo, ohlašovalt jsem, abys neríkal: Modla má ucinila ty veci, a rytina má neb slitina má prikázala to.
\par 6 Slýchals o tom, pohlediž na to na všecko, vy pak, nebudete-liž toho oznamovati? Již nyní ohlašujit nové a tajné veci, o nichž jsi ty nic nevedel.
\par 7 Nyní stvoreny jsou, a ne predešlého casu, o nichž jsi pred tímto dnem nic neslyšel, abys nerekl: Aj, vedel jsem o tom.
\par 8 Anobrž aniž jsi slyšel, ani vedel, aniž se to tehdáž doneslo ucha tvého; nebo jsem vedel, že sobe velmi nevážne pocínati budeš, a že jsi prevrácenec hned od života matky.
\par 9 Pro jméno své poshovím s prchlivostí svou, a pro chválu svou poukrotím hnevu proti tobe, abych te nevyplénil.
\par 10 Aj, prepálím te, ackoli ne jako stríbro, preberu te v peci ssoužení.
\par 11 Pro sebe, pro sebe uciním to. Nebo jakž by mohlo v lehkost vydáno býti? Slávy své zajisté jinému nedám.
\par 12 Slyš mne, Jákobe a Izraeli, povolaný muj: Já jsem, já první, já jsem i poslední.
\par 13 Má zajisté ruka založila zemi, a pravice má dlaní rozmerila nebesa; povolal jsem jich, a hned se postavily.
\par 14 Shromaždte se vy všickni, a slyšte. Kdo z nich oznámil tyto veci: Hospodin miluje jej, ont vykoná vuli jeho proti Babylonu, a ráme jeho proti Kaldejským?
\par 15 Já, já mluvil jsem, protož povolám ho; privedu jej, a štastnou bude míti cestu svou.
\par 16 Pristupte ke mne, slyšte to: Nemluvíval jsem z pocátku v skryte; od toho casu, v kterémž se to dálo, prítomen jsem byl. A nyní Panovník Hospodin poslal mne a duch jeho.
\par 17 Toto praví Hospodin vykupitel tvuj, Svatý Izraelský: Já Hospodin Buh tvuj ucím te, abys prospech bral, a vodím te po ceste, po kteréž bys chodil.
\par 18 Ó kdybys byl šetril prikázaní mých, bylt by jako potok pokoj tvuj, a spravedlnost tvá jako vlny morské.
\par 19 A bylo by jako písku semene tvého, a plodu života tvého jako šterku jeho, aniž by vytato, ani vyhlazeno bylo jméno jeho pred oblícejem mým.
\par 20 Vyjdete z Babylona, utecte od Kaldejských, hlasem zvucným zvestujte, ohlašujte to, rozneste to až do koncin zeme. Rcete: Vykoupil Hospodin služebníka svého Jákoba.
\par 21 Nebudout žízniti, když je po pustinách povede, vody z skály vyvede jim; nebo rozetne skálu, aby tekly vody.
\par 22 Nemajít žádného pokoje, praví Hospodin, bezbožní.

\chapter{49}

\par 1 Poslouchejte mne ostrovové, a pozorujte národové dalecí: Hospodin hned z života povolal mne, od života matky mé v pamet uvedl jméno mé,
\par 2 A ucinil ústa má podobná meci ostrému. V stínu ruky své skryl mne, a uciniv ze mne strelu vypulerovanou, v toule svém schoval mne.
\par 3 A rekl mi: Služebník muj jsi, v Izraeli skrze tebe oslaven budu.
\par 4 Já pak rekl jsem: Nadarmo jsem pracoval, daremne a marne sílu svou jsem strávil. Ale však soud muj jestit u Hospodina, a práce má u Boha mého.
\par 5 A nyní dí Hospodin, kterýž mne sformoval hned od života za služebníka svého, abych zase privedl k nemu Jákoba; (byt pak i nebyl sebrán Izrael, slávu však mám pred ocima Hospodinovýma; nebo Buh muj jest síla má);
\par 6 I to rekl Hospodin: Málot by to bylo, abys mi byl služebníkem ku pozdvižení pokolení Jákobových, a k navrácení ostatku Izraelských; protož dal jsem te za svetlo pohanum, abys byl spasení mé až do koncin zeme.
\par 7 Toto praví Hospodin vykupitel Izraeluv, Svatý jeho, tomu, jímž pohrdá každý, a jehož sobe oškliví národové, služebníku panujících: Králové, vidouce te, povstanou, a knížata klaneti se budou pro Hospodina, kterýž verný jest, Svatého Izraelského, jenž te vyvolil.
\par 8 Toto praví Hospodin: V cas milosti vyslyším te, a ve dni spasení spomohu tobe. Nadto ostríhati te budu, a dám te v smlouvu lidu, abys utvrdil zemi, a v dedictví uvedl dedictví zpuštená;
\par 9 Abys rekl veznum: Vyjdete, tem, kteríž jsou ve tmách: Zjevte se. I budou se pásti podlé cest, a na všech místech vysokých bude pastva jejich.
\par 10 Nebudou lacneti ani žízniti, nebude na ne bíti horko ani slunce; nebo slitovník jejich zprovodí je, a podlé pramenu vod povede je.
\par 11 Pres to zpusobím na všech horách svých cesty, a silnice mé vyvýšeny budou.
\par 12 Aj, tito zdaleka prijdou, aj, onino od pulnoci a od more, a jiní z zeme Sinim.
\par 13 Prozpevujte nebesa, a plésej zeme, a zvucne prokrikujte hory; nebot jest potešil Hospodin lidu svého, a nad chudými svými slitoval se.
\par 14 Ale rekl Sion: Opustilte mne Hospodin, a Pán zapomenul se na mne.
\par 15 I zdaliž se muže zapomenouti žena nad nemluvnátkem svým, aby se neslitovala nad plodem života svého? A byt se pak ony zapomnely, já však nezapomenu se na te.
\par 16 Aj, na dlaních vyryl jsem te, zdi tvé jsou vždycky prede mnou.
\par 17 Pospíšít k tobe synové tvoji, ti pak, kteríž te borili a kazili, odejdou od tebe.
\par 18 Pozdvihni vukol ocí svých, a pohled, všickni ti shromáždíce se, prijdou k tobe. Živt jsem já, praví Hospodin, že se jimi všemi jako okrasou priodeješ, a otocíš se jimi jako halží nevesta,
\par 19 Proto že pustiny tvé, a poušte tvé, a zboreniny zeme tvé že tehdáž tesné budou, prícinou obyvatelu, když vzdáleni budou ti, kteríž te zžírali;
\par 20 Tak že reknou pred tebou synové siroby tvé: Tesné mi jest toto místo, ustup mi, abych bydliti mohl.
\par 21 I díš v srdci svém: Kdo mi naplodil techto? Nebo jsem já byla osirelá a osamelá, sem i tam precházející a odcházející. Tyto, pravím, kdo vychoval? Aj, pozustala jsem byla sama jediná. Kdež tito byli?
\par 22 Toto praví Panovník Hospodin: Aj, já pozdvihnu k národum ruky své, a k lidem vyzdvihnu korouhev svou, aby prinesli syny tvé na rukou, a dcery tvé aby na plecech neseny byly.
\par 23 I budou králové pestounové tvoji, a královny jejich chovacky tvé. Tvárí k zemi skláneti se budou pred tebou, a prach noh tvých lízati budou, a poznáš, že jsem já Hospodin, a že nebývají zahanbeni, kteríž na mne ocekávají.
\par 24 Ale díš: Zdaliž odjato bude reku udatnému to, což uchvátil? A zdaž zajatý lid spravedlivého vyprošten bude?
\par 25 Anobrž tak praví Hospodin: I zajatý lid reku udatnému odjat bude, a to, což uchvátil násilník, vyprošteno bude; nebo s tím, kterýž se s tebou nesnadní, já se nesnadniti budu, a syny tvé já vysvobodím.
\par 26 A nakrmím ty, kteríž te utiskují, vlastním masem jejich, a jako mstem krví svou se zpijí. I poznát všeliké telo, že já Hospodin jsem spasitel tvuj, a vykupitel tvuj Buh silný Jákobuv.

\chapter{50}

\par 1 Takto praví Hospodin: Kdež jest lístek zapuzení matky vaší, kterýmž jsem ji propustil? Aneb kdo jest z veritelu mých, jemuž jsem vás prodal? Aj, nepravostmi svými prodali jste sebe, a pro prevrácenosti vaše propuštena jest matka vaše.
\par 2 Proc, když pricházím, není žádného, když volám, žádný se neozývá? Zdaliž jest naprosto ukrácena ruka má, aby nemohla vykoupiti? A žádné-liž není ve mne moci k vysvobození? Aj, žehráním svým vysušuji more, obracím reky v poušt, až se smrazují ryby jejich, proto že nebývá vody, a mrou žízní.
\par 3 Oblácím nebesa v smutek, a žíni dávám jim za odev.
\par 4 Panovník Hospodin dal mi jazyk umelý, abych umel príhodne ustalému mluviti slova. Probuzuje každého jitra, probuzuje mi uši, abych slyšel, tak jako pilne se ucící.
\par 5 Panovník Hospodin otvírá mi uši, a já se nepostavuji zpurne, aniž se nazpet odvracím.
\par 6 Tela svého nastavuji bijícím, a líce svého rvoucím mne, tvári své neskrývám od pohanení a plvání.
\par 7 Nebo Panovník Hospodin spomáhá mi, procež nebývám zahanben. Pro touž prícinu nastavuji tvári své jako škremene; nebo vím, že nebudu zahanben.
\par 8 Blízkot jest ten, kterýž mne ospravedlnuje. Kdož se nesnadniti bude se mnou? Postavme se spolu; kdo jest odpurce muj, necht pristoupí ke mne.
\par 9 Aj, Panovník Hospodin spomáhati mi bude. Kdož jest, ješto by mne potupil? Aj, všickni takoví jako roucho zvetšejí, mol sžíre je.
\par 10 Kdo jest mezi vámi, ješto se bojí Hospodina, poslouchej hlasu služebníka jeho. Kdo jest, ješto chodí v temnostech, a nemá žádného svetla, doufej ve jméno Hospodinovo, a zpolehni na Boha svého.
\par 11 Aj, vy všickni, kteríž zanecujete ohen, a jiskrami se prepasujete, chodtež v blesku ohne svého, a v jiskrách, kteréž jste roznítili. Od ruky mé toto se vám stane, že v bolesti ležeti budete.

\chapter{51}

\par 1 Poslouchejte mne, následovníci spravedlnosti, kteríž hledáte Hospodina, pohledte na skálu, odkudž vytati jste, a na hlubokost jámy, z níž vykopáni jste.
\par 2 Pohledte na Abrahama otce vašeho, a na Sáru, kteráž vás porodila, že jsem jej jediného povolal, a požehnal jsem jemu, i rozmnožil jej.
\par 3 Nebo poteší Hospodin Siona, poteší všech pustin jeho, a uciní poušt jeho prerozkošnou, a pustinu jeho podobnou zahrade Hospodinove. Radost a veselí bude nalezeno v nem, díkcinení, a hlas žalmu zpívání.
\par 4 Pozorujte mne, lide muj, a rodino má, nastavte mi uší; nebo zákon ode mne vyjde, a soud svuj za svetlo národum vystavím.
\par 5 Blízko jest spravedlnost má, vyjdet spasení mé, a ramena má národy souditi budou. Na mnet ostrovové cekají, a po mém rameni touží.
\par 6 Pozdvihnete k nebi ocí svých, a popatrte na zem dolu. Nebesa zajisté jako dým zmizejí, a zeme jako roucho zvetší, a obyvatelé její též podobne zemrou: ale spasení mé na veky zustane, a spravedlnost má nezahyne.
\par 7 Poslouchejte mne, kteríž znáte spravedlnost, lide, v jehož srdci jest zákon muj; nebojte se útržky lidí bídných, a hanení jejich nedeste se.
\par 8 Nebo jako roucho zžíre je mol, a jako vlnu zžíre je cerv, ale spravedlnost má na veky zustane, a spasení mé od národu do pronárodu.
\par 9 Probud se, probud se, oblec se v sílu, ó ráme Hospodinovo, probud se, jako za dnu starodávních a národu predešlých. Zdaliž ty nejsi to, kteréžs poplénilo Egypt, a ranilo draka?
\par 10 Zdaliž ty nejsi to, kteréžs vysušilo more, vody propasti veliké, kteréžs obrátilo hlubiny morské v cestu, aby prešli ti, jenž byli vysvobozeni?
\par 11 A tak ti, kteréž vykoupil Hospodin, at se navrátí, a prijdou na Sion s prozpevováním, a veselé vecné at jest na hlave jejich; radosti a veselé at dojdou, zámutek pak a úpení at utekou.
\par 12 Já, já jsem utešitel váš. Jakáž jsi ty, že se bojíš cloveka smrtelného, a syna cloveka tráve podobného?
\par 13 Že se zapomínáš na Hospodina ucinitele svého, kterýž roztáhl nebesa, a založil zemi, a že se desíš ustavicne každého dne prchlivosti ssužujícího, jakž se jen postrojí, aby hubil? Ale kdež jest pak ta prchlivost toho,kterýž ssužuje?
\par 14 Pospíšít, aby zajatý propušten byl; nebot neumre v jáme, aniž bude míti jaký nedostatek chleba svého.
\par 15 Ját jsem zajisté Hospodin Buh tvuj, kterýž rozdeluje more, jehož vlny zvuk vydávají. Hospodin zástupu jest jméno mé,
\par 16 Kterýž jsem vložil slova svá v ústa tvá, a stínem ruky své prikryl jsem te, abys štípil nebesa, a založil zemi, a rekl Sionu: Lid muj jsi ty.
\par 17 Probud se, probud, povstan, Jeruzaléme, kterýž jsi pil z ruky Hospodinovy kalich prchlivosti jeho, kvasnice z kalicha hruzy vypil jsi, i vyvážil.
\par 18 Žádný jí nevedl ze všech synu, kterýchž naplodila, a žádný jí neujal za ruku její ze všech synu, kteréž vychovala.
\par 19 Toto dvé te potkalo, (kdo te politoval?) zpuštení a setrení, hlad a mec. Kdo te potešoval?
\par 20 Synové tvoji omráceni jsouc, leží na rozcestí všech ulic, jako buvol v leci, plni jsouce prchlivosti Hospodinovy, žehrání Boha tvého.
\par 21 A protož slyš nyní toto, ztrápená a opilá, ale ne vínem:
\par 22 Takto praví Pán tvuj, Hospodin a Buh tvuj, vedoucí pri lidu svého: Aj, beru z ruky tvé kalich hruzy, i kvasnice kalicha prchlivosti mé, nebudeš ho píti více.
\par 23 Ale dám jej do ruky tech, jenž te ssužují, kteríž ríkali duši tvé: Sehni se, at pres te prejdeme, jimž jsi podkládala jako zemi hrbet svuj, a jako ulici precházejícím.

\chapter{52}

\par 1 Probud se, probud se, oblec se v sílu svou, Sione, oblec se v roucho okrasy své, ó Jeruzaléme, mesto svaté; nebot nebude již více na te dotírati neobrezaný a necistý.
\par 2 Otres se z prachu, povstan, posad se,Jeruzaléme; dobud se z okovu hrdla svého, ó jatá dcerko Sionská.
\par 3 Takto zajisté dí Hospodin: Darmo jste sebe prodali, protož bez penez budete vykoupeni.
\par 4 Nebo takto praví Panovník Hospodin: Do Egypta sstoupil lid muj predešle, aby tam byl pohostinu, ale Assur bez príciny jej ssužuje.
\par 5 Nyní tedy což mám ciniti? praví Hospodin. Ponevadž jest lid muj zajat darmo, a panovníci jeho k úpení jej privodí, praví Hospodin. Nad to ustavicne každého dne jménu mému útržka se ciní.
\par 6 Protož poznát lid muj jméno mé, protož poznát, pravím, v ten den, že já tentýž, kterýž mluvím, aj, prítomen budu.
\par 7 Ó jak krásné na horách nohy toho, ješto potešené veci zvestuje, a ohlašuje pokoj, toho, ješto zvestuje dobré, ješto káže spasení, a mluví k Sionu: Kralujet Buh tvuj.
\par 8 Strážní tvoji hlasu, hlasu pozdvihnou, a spolu prokrikovati budou; nebot okem v oko uzrí, že Hospodin zase privede Sion.
\par 9 Zvucte, prozpevujte spolu pustiny Jeruzalémské; nebot jest potešil Hospodin lidu svého, vykoupil Jeruzalém.
\par 10 Ohrnul Hospodin ráme svatosti své pred ocima všech národu, aby videly všecky konciny zeme spasení Boha našeho.
\par 11 Odejdete, odejdete, vyjdete z Babylona, necistého se nedotýkejte; vyjdete z prostredku jeho, ocistte se vy, kteríž nosíte nádobí Hospodinovo.
\par 12 Nebo ne s chvátáním vyjdete, aniž s utíkáním pujdete; predcházeti zajisté bude vás Hospodin, a zbére vás Buh Izraelský.
\par 13 Aj, služebníku mému štastne se povede, vyvýšen, vznešen a zveleben bude velmi.
\par 14 A jakož mnozí se nad ním užasnou, že tak zohavena jest nad jiné lidi osoba jeho, zpusob jeho nad syny lidské:
\par 15 Tak zase skropí národy mnohé, i králové pred ním zacpají ústa svá, proto že což jim nebylo vypravováno, to uzrí, a tomu, o cemž neslýchali, porozumejí.

\chapter{53}

\par 1 Kdo uveril kázaní našemu? A ráme Hospodinovo komu jest zjeveno?
\par 2 Nebo pred ním vyrostl jako proutek, a jako koren z zeme vyprahlé, nemaje podoby ani krásy. Videlit jsme jej, ale nic nebylo videti toho, proc bychom ho žádostivi byli.
\par 3 Nejpohrdanejší zajisté a nejopovrženejší byl z lidí, muž bolestí, a kterýž zkusil nemocí, a jako ukrývající tvár svou; nejpohrdanejší, procež jsme ho za nic nevážili.
\par 4 Ještote on nemoci naše vzal, a bolesti naše vlastní on nesl, my však domnívali jsme se, že jest ranen, a ubit od Boha, i strápen.
\par 5 On pak ranen jest pro prestoupení naše, potrín pro nepravosti naše; kázen pokoje našeho na nej vzložena, a zsinalostí jeho lékarství nám zpusobeno.
\par 6 Všickni my jako ovce zbloudili jsme, jeden každý na cestu svou obrátili jsme se, a Hospodin uvalil na nej nepravosti všech nás.
\par 7 Pokutován jest i strápen, však neotevrel úst svých. Jako beránek k zabití veden byl, a jako ovce pred temi, kdož ji strihou, onemel, aniž otevrel úst svých.
\par 8 Z úzkosti a z soudu vynat jest, a protož rod jeho kdo vypraví, ackoli vytat jest z zeme živých, a zranen pro prestoupení lidu mého?
\par 9 Kterýžto vydal bezbožným hrob jeho, a bohatému, aby byl usmrcen, ješto však nepravosti neucinil, aniž jest nalezena lest v ústech jeho.
\par 10 Takte se líbilo Hospodinu jej stírati, a nemocí trápiti, aby polože duši svou v obet za hrích,videl síme své, byl dlouhoveký, a to, což se líbí Hospodinu, skrze neho štastne konáno bylo.
\par 11 Z práce duše své uzrí užitek, jímž nasycen bude. Známostí svou ospravedlní spravedlivý služebník muj mnohé; nebo nepravosti jejich on sám ponese.
\par 12 A protož dám jemu díl pro mnohé, aby s nescíslnými delil se o korist, proto že vylil na smrt duši svou, a s prestupníky pocten jest. Ont sám nesl hrích mnohých, a prestupníku zástupcím byl.

\chapter{54}

\par 1 Prozpevuj,neplodná, kteráž nerodíš, zvucne prozpevuj a prokrikni, kteráž ku porodu nepracuješ, nebo více bude synu opuštené, nežli synu té, kteráž má muže, praví Hospodin.
\par 2 Rozšir místo stanu svého, a calounu príbytku svých roztáhnouti nezbranuj; natáhni i provazu svých, a kolíky své utvrd.
\par 3 Nebo na pravo i na levo se rozmužeš, a síme tvé národy dedicne vládnouti bude, a mesta pustá osadí.
\par 4 Nebojž se, nebo nebudeš zahanbena, aniž se zapyruj, nebo nebudeš v potupu uvedena; nýbrž na potupu mladosti své zapomeneš, a na pohanení vdovství svého nezpomeneš více.
\par 5 Nebo manželem tvým jest Ucinitel tvuj, jehož jméno Hospodin zástupu, a vykupitel tvuj Svatý Izraelský Bohem vší zeme slouti bude.
\par 6 Nebo jako ženy propuštené a v duchu sevrené povolá te Hospodin, a jako ženy mladice, když v pohrdnutí budeš, praví Buh tvuj.
\par 7 Na malickou chvilku poopustil jsem te, ale v slitování prevelikém shromáždím te.
\par 8 V malickém hneve skryl jsem tvár svou na malicko pred tebou, ale v milosrdenství vecném slituji se nad tebou, praví vykupitel tvuj Hospodin.
\par 9 Nebot jest to u mne, co pri potope Noé. Jakož jsem prisáhl, že se nebudou více rozlévati vody Noé po zemi, tak jsem prisáhl, že se nerozhnevám na te, aniž tobe prísne domlouvati budu.
\par 10 A byt se i hory pohybovaly, a pahrbkové ustupovali, milosrdenství mé však od tebe neodstoupí, a smlouva pokoje mého se nepohne, praví slitovník tvuj Hospodin.
\par 11 Ó ssoužená, vichricí zmítaná, potešení zbavená, aj, já položím na karbunkulích kamení tvé, a založím te na zafirích.
\par 12 A vzdelám z krištálu skla tvá, a brány tvé z kamení trpytícího se, i všecka pomezí tvá z kamení drahého.
\par 13 Synové pak tvoji všickni vyucení budou od Hospodina, a hojnost pokoje budou míti synové tvoji.
\par 14 Na spravedlnosti upevnena budeš. Vzdálíš se od ssoužení, protož se ho nebudeš báti, a od setrení, nebo nepriblíží se k tobe.
\par 15 Aj, budout nejedni bydliti s tebou, kteríž nejsou moji, ale kdož by bydleje s tebou, byl proti tobe, padne.
\par 16 Aj, já stvoril jsem kováre dýmajícího pri ohni na uhlí, a vynášejícího nádobí k dílu svému, já také stvoril jsem zhoubce, aby hubil.
\par 17 Žádný nástroj proti tobe udelaný nepodarí se, a každý jazyk, povstávající proti tobe na soudu, potupíš. Tot jest dedictví služebníku Hospodinových, a spravedlnost jejich ode mne, praví Hospodin.

\chapter{55}

\par 1 Ej, všickni žízniví, podte k vodám, i vy,kteríž nemáte žádných penez. Podte, kupujte a jezte, podte, pravím, kupujte bez penez a bez záplaty víno a mléko.
\par 2 Proc vynakládáte peníze ne za chléb, a práci svou za to, což nenasycuje? Poslechnete mne radeji, a jezte to, což jest dobrého, a necht se kochá v tuku duše vaše.
\par 3 Naklonte ucha svého, a podte ke mne, poslechnete, a budet živa duše vaše; uciním zajisté s vámi smlouvu vecnou, milosrdenství Davidova prepevná.
\par 4 Aj, za svedka národum dal jsem jej, za vudce a ucitele národum.
\par 5 Aj, národu, k nemužs se neznal, povoláš, a národové, kteríž te neznali, k tobe se sbehnou, pro Hospodina Boha tvého, a Svatého Izraelského, nebo te oslaví.
\par 6 Hledejte Hospodina, pokudž muže nalezen býti; vzývejte ho, pokudž blízko jest.
\par 7 Opust bezbožný cestu svou, a clovek nepravý myšlení svá, a necht se navrátí k Hospodinu, i slitujet se nad ním, a k Bohu našemu, nebt jest hojný k odpuštení.
\par 8 Nejsout zajisté myšlení má jako myšlení vaše, ani cesty vaše jako cesty mé, praví Hospodin.
\par 9 Ale jakož vyšší jsou nebesa než zeme, tak prevyšují cesty mé cesty vaše, a myšlení má myšlení vaše.
\par 10 Nebo jakož prší déšt neb sníh s nebe, a zase se tam nenavracuje, ale napájí zemi, a ciní ji plodistvou a úrodnou, tak že vydává síme rozsívajícímu, a chléb jedoucímu,
\par 11 Tak bude slovo mé, kteréž vyjde z úst mých. Nenavrátí se ke mne prázdné, ale uciní to, což mi se líbí, a prospešne to vykoná, k cemuž je posílám.
\par 12 A protož s veselím vyjdete, a v pokoji sprovozeni budete. Hory i pahrbkové zvucne naproti vám prozpevovati budou, a všecko dríví polní rukama plésati bude.
\par 13 Místo chrastiny vzejde jedlé, a místo hloží vyroste myrtus, a bude to Hospodinu k sláve, na znamení vecné, kteréž nebude vyhlazeno.

\chapter{56}

\par 1 Toto praví Hospodin: Ostríhejte soudu, a cinte spravedlnost; nebo brzo spasení mé prijde, a spravedlnost má zjevena bude.
\par 2 Blahoslavený clovek, kterýž ciní to, a syn cloveka, kterýž se prídrží toho, ostríhaje soboty, aby jí nepoškvrnoval, a ostríhaje ruky své, aby nic zlého neucinila.
\par 3 Necht tedy nemluví cizozemec, kterýž se pripojuje k Hospodinu, ríkaje: Jiste odloucil mne Hospodin od lidu svého. Též at neríká kleštenec: Aj, já jsem drevo suché.
\par 4 Nebo toto praví Hospodin o kleštencích, kteríž by ostríhali sobot mých, a zvolili to, což mi se líbí, a drželi smlouvu mou:
\par 5 Že dám jim v dome svém a mezi zdmi svými místo, a jméno lepší nežli synu a dcer. Jméno vecné dám jim, kteréž nebude vyhlazeno.
\par 6 Cizozemce pak, kteríž by se pripojili k Hospodinu, aby sloužili jemu, a milovali jméno Hospodinovo, jsouce u neho za služebníky, všecky ostríhající soboty, aby jí nepoškvrnovali, a držící smlouvu mou,
\par 7 Ty privedu k hore svatosti své, a obveselím je v dome svém modlitebném. Zápalové jejich a obeti jejich príjemné mi budou na oltári mém; nebo dum muj dum modlitby slouti bude u všech národu.
\par 8 Pravít Panovník Hospodin, kterýž shromažduje rozehnané Izraelovy: Ještet shromáždím k nemu a k shromáždeným jeho.
\par 9 Všecka zvírata polní podte žráti, i všecka zvírata lesní.
\par 10 Strážní jeho jsou slepí, všickni naporád nic neznají, všickni jsou psi nemí, aniž mohou štekati; jsou ospalci, leží, milujíce drímotu.
\par 11 Nadto jsou psi obžerní, nevedí, kdy jsou syti; procež sami se pasou. Neumejí uciti, všickni k cestám svým patrí, jeden každý k zisku svému po své strane.
\par 12 Podte, naberu vína, a opojíme se nápojem opojným, a bude rovne zítrejší jako dnešní den, nýbrž vetší a mnohem hojnejší.

\chapter{57}

\par 1 Spravedlivý hyne, a žádný nepripouští toho k srdci, a muži pobožní odcházejí, a žádný nerozvažuje toho, že pred príchodem zlého vychvácen bývá spravedlivý,
\par 2 Že dochází pokoje, a odpocívá na ložci svém, kdožkoli chodí v uprímosti své.
\par 3 Ale vy pristupte sem, synové kouzedlnice, síme cizoložníka a smilnice.
\par 4 Komu se to s takovou chutí posmíváte? Proti komu rozdíráte ústa, vyplazujete jazyk? Zdaliž nejste synové nerádní, síme postranní?
\par 5 Kteríž smilníte v hájích pod každým drevem zeleným, zabíjejíce syny své pri potocích, pod vysokými skalami.
\par 6 Mezi hladkými kameny potocními jest díl tvuj. Tit jsou, ti los tvuj, na než také vyléváš mokrou obet, obetuješ suchou obet. V tech-liž bych vecech se kochal?
\par 7 Na hore vysoké a vyvýšené stavíš lože své, a tam vstupuješ k obetování obeti.
\par 8 Pametné pak znamení své za dvére a za vereje stavíš, když ode mne, odkryvši se, vstupuješ, a rozširuješ lože své, ciníc je prostrannejší víc než pohané; miluješ lože jejich, kdež místo oblíbíš.
\par 9 Chodíš i k králi s olejem, a s mnohými vonnými mastmi svými; posíláš zajisté posly své daleko, a ponižuješ se až do hrobu.
\par 10 Pro množství cest svých ustáváš, aniž ríkáš: Daremnét jest to. Nalezla jsi sobe zber ku pomoci, protož neželíš práce.
\par 11 A kohož jsi se desila a bála, že jsi klamala, a na mne se nerozpomínala, ani pripustila k srdci svému? Zdali proto, že jsem já mlcel, a to zdávna, nebojíš se mne?
\par 12 Já tvou spravedlnost oznámím, a skutky tvé, kterížt nic neprospejí.
\par 13 Když kriceti budeš, nechat te vysvobodí zber tvá. Ano pak všecky je zanese vítr, a zachvátí marnost, ale kdož doufá ve mne, vládnouti bude zemí, a dedicne obdrží horu svatou mou.
\par 14 Nebo receno bude: Vyrovnejte, vyrovnejte, spravte cestu, odklidte prekážky z cesty lidu mého.
\par 15 Nebo takto dí ten dustojný a vyvýšený, kterýž u vecnosti prebývá, jehož jméno jest Svatý: Na výsosti a v míste svatém bydlím, ano i s tím, kterýž jest skroušeného a poníženého ducha prebývám, obživuje ducha ponížených, obživuje také srdce skroušených.
\par 16 Nebudut se zajisté na veky nesnadniti, aniž se budu vecne hnevati, nebot by duch pred oblícejem mým zmizel, i dchnutí, kteréž jsem já ucinil.
\par 17 Pro nepravost lakomství jeho rozhneval jsem se, a ubil jsem jej; skryl jsem se a rozhneval proto, že odvrátiv se, odšel cestou srdce svého.
\par 18 Vidím cesty jeho, a však uzdravím jej; zprovodím jej, a jemu potešení navrátím, i tem, kteríž kvílí s ním.
\par 19 Stvorím ovoce rtu, hojný pokoj, dalekému jako blízkému, praví Hospodin, a tak uzdravím jej.
\par 20 Bezbožní pak budou jako more zbourené, když se spokojiti nemuže, a jehož vody vymítají necistotu a bláto.
\par 21 Nemajít žádného pokoje, praví Buh muj, bezbožní.

\chapter{58}

\par 1 Volej vším hrdlem, nezadržuj, jako trouba povýš hlasu svého, a oznam lidu mému prevrácenost jejich, a domu Jákobovu hríchy jejich.
\par 2 Jakkoli každého dne mne hledají, a znáti cesty mé jsou chtiví, jako by byli národ, kterýž spravedlnost ciní, a soudu Boha svého neopouští. Dotazují se mne na soudy spravedlnosti, blízcí Boha býti chtejí,
\par 3 A ríkají: Proc se postíváme, ponevadž nepatríš? Trápíváme duše své, a nechceš vedeti o tom? Aj, v den postu vašeho líbost provodíte, a ke všem robotám svým prísne doháníte.
\par 4 Aj, k sváru a ruznici se postíváte, a abyste bili pestí nemilostive; nepostíte se tak dnu tech, aby slyšán byl na výsosti hlas váš.
\par 5 Zdaliž to jest takový pust, jakýž oblibuji, a den, v nemž by trápil clovek duši svou? Zdali, aby svesil jako trtina hlavu svou, a podstíral žíni a popel? To-liž nazuveš postem a dnem vzácným Hospodinu?
\par 6 Není-liž toto pust, kterýž oblibuji: Rozvázati svazky bezbožnosti; roztrhnouti snopky obtežující, a potrené propustiti svobodné, a tak všelijaké jho abyste roztrhli?
\par 7 Není-liž: Abys lámal lacnému chléb svuj, a chudé vypovedené abys uvedl do domu? Videl-li bys nahého, abys jej priodel, a pred telem svým abys se neskrýval.
\par 8 Tehdáž se vyrazí jako jitrní záre svetlo tvé, a zdraví tvé rychle zkvetne; predcházeti te zajisté bude spravedlnost tvá, a sláva Hospodinova zbére te.
\par 9 Tehdy volati budeš, a Hospodin vyslyší te; zavoláš, a reknet: Ted jsem. Jestliže vyvržeš z prostred sebe jho, a vztahování prstu, a mluvení nepravostí,
\par 10 A vyleješ-li lacnému duši svou, a strápenou duši nasytíš-li: vzejde v temnostech svetlo tvé, a mrákota tvá bude jako poledne.
\par 11 Nebo povede te Hospodin ustavicne, a nasytí i v náramné sucho duši tvou, a kosti tvé tukem naplní. I budeš jako zahrada svlažená, a jako pramen vod, jehož vody nevysychají.
\par 12 A vzdelají od tebe zplození pustiny starodávní; základy od národu do pronárodu vyzdvihneš. I slouti budeš vzdelavatel zboreniny, a napravovatel stezek k bydlení.
\par 13 Jestliže odvrátíš od soboty nohu svou, abys nevykonával líbosti své v den svatý muj, anobrž nazuveš-li sobotu rozkoší, a svatou Hospodinu slavnou, a budeš-li ji slaviti tak, abys necinil cest svých, ani vykonával, co by se líbilo, ani nemluvil slova:
\par 14 Tehdy rozkoš míti budeš v Hospodinu, a uvedu te na vysoká místa zeme, a zpusobím to, abys užíval dedictví Jákoba otce svého; nebo ústa Hospodinova mluvila.

\chapter{59}

\par 1 Aj, nenít ukrácena ruka Hospodinova, aby nemohla zachovati, aniž jest obtíženo ucho jeho, aby nemohlo slyšeti.
\par 2 Ale nepravosti vaše rozloucily vás s Bohem vaším, a hríchové vaši to zpusobili, že skryl tvár pred vámi, aby neslyšel.
\par 3 Nebo ruce vaše jsou poškvrnené krví, a prstové vaši nepravostí; rtové vaši mluví lež, jazyk váš vynáší prevrácenost.
\par 4 Není žádného, ješto by se zasadil o spravedlnost, aniž jest kdo, ješto by zastával pravdy. Doufají v marnost, a mluví daremné veci; pocínají nátisk, a rodí nepravost.
\par 5 Vejce bazališková vysedeli, a plátna pavoukového natkali. Kdož by jedl vejce jejich, umre; pakli je roztlací, vynikne ješterka.
\par 6 Plátna jejich nehodí se na roucho, aniž se odejí dílem svým; skutkové jejich skutkové nepravosti, a dílo ukrutnosti jest v rukou jejich.
\par 7 Nohy jejich k zlému beží, a pospíchají k vylévání krve nevinné. Myšlení jejich jsou myšlení nepravá, zpuštení a setrení jest na cestách jejich.
\par 8 Cesty pokoje neznají, a není žádné spravedlnosti v šlepejích jejich. Stezky své prevracejí tajne; kdožkoli po nich chodí, nemívá pokoje.
\par 9 Protož vzdálil se od nás soud, a nedochází nás spravedlnost. Cekáme-li na svetlo, aj, tma, pakli na blesk, v mrákotách chodíme.
\par 10 Makáme jako slepí stenu, a jako bychom žádných ocí nemeli, šámáme. Urážíme se o poledni jako v soumrak, u veliké hojnosti podobni jsme mrtvým.
\par 11 Mumleme všickni my jako nedvedi, a jako holubice ustavicne lkáme. Ocekáváme na soud, ale není ho, na vysvobození, ale daleké jest od nás.
\par 12 Nebo rozmnožena jsou prestoupení naše pred tebou, a hríchové naši svedcí proti nám, ponevadž prestoupení naše jsou pri nás, i nepravosti naše. Známet to,
\par 13 Že jsme se zproneverili, a lhali Hospodinu, a odvrátili se od následování Boha svého, že jsme mluvili o nátisku a odvrácení, že jsme ukládali a vynášeli z srdce slova lživá,
\par 14 Tak že odvrácen jest nazpet soud, a spravedlnost zdaleka stojí; nebo klesla na ulici pravda, a pravost nemá pruchodu.
\par 15 Nýbrž zhynula pravda, a ten, kdož se uchyluje od zlého, loupeži bývá vydán; což vidí Hospodin, a nelíbí se to jemu, že není žádného soudu.
\par 16 Když tedy videl, že není žádného muže, až se užasl, že není žádného prostredníka. A protož vysvobození jemu zpusobilo ráme jeho, a spravedlnost jeho sama jej zpodeprela.
\par 17 Nebo oblékl spravedlnost jako pancír, a lebka spasení na hlave jeho. Oblékl se v roucho pomsty jako v sukni, a odel se horlivostí jako pláštem,
\par 18 Aby podlé skutku, aby podlé nich odplacel prchlivostí protivníkum svým, odmenu neprátelum svým, i ostrovum odplatu aby dával.
\par 19 I budou se báti na západ jména Hospodinova, a na východu slunce slávy jeho, když se privalí jako reka neprítel, jejž duch Hospodinuv prec zažene.
\par 20 Nebot prijde k Sionu vykupitel, a k tem, kteríž se odvracují od prestoupení v Jákobovi, praví Hospodin.
\par 21 Tatot pak bude smlouva má s nimi, praví Hospodin: Duch muj, kterýž jest v tobe, a slova má, kteráž jsem vložil v ústa tvá, neodejdout od úst tvých, ani od úst semene tvého, ani od úst potomku semene tvého, praví Hospodin, od tohoto casu až na veky.

\chapter{60}

\par 1 Povstaniž, zastkvej se, ponevadž prišlo svetlo tvé, a sláva Hospodinova vzešla nad tebou.
\par 2 Nebo aj, tmy prikryjí zemi, a mrákota národy, ale nad tebou vzejde Hospodin, a sláva jeho nad tebou vidína bude.
\par 3 I budou choditi národové v svetle tvém, a králové v blesku, jenž vzejde nad tebou.
\par 4 Pozdvihni vukol ocí svých, a popatr. Všickni tito shromáždíce se, k tobe se poberou, synové tvoji zdaleka prijdou, a dcery tvé pri boku tvém chovány budou.
\par 5 Tehdáž uzríš to, a rozveselíš se, tehdáž podiví se, a rozšírí se srdce tvé; nebo se obrátí k tobe množství morské, síla pohanu prijde k tobe.
\par 6 Stádo velbloudu prikryje te, a dromedári Madianští a Efejští, všickni ti z Sáby prijdou, zlato a kadidlo prinesou, a chvály Hospodinovy zvestovati budou.
\par 7 Všecka stáda Cedarská shromáždí se k tobe, skopcové Nabajotští prisluhovati budou tobe, a obetováni jsouce na mém oltári, príjemní budou. A takt dum okrasy své ozdobím.
\par 8 I díš: Kdo jsou ti, kteríž se jako hustý oblak sletují, a jako holubice k derám svým?
\par 9 Na mnet zajisté ostrovové ocekávají, a lodí morské hned zdávna, aby privedli syny tvé zdaleka, též stríbro své a zlato své s sebou, k sláve Hospodina, Boha tvého a Svatého Izraelského; nebo te oslaví.
\par 10 I vystavejí cizozemci zdi tvé, a králové jejich prisluhovati budou tobe, když v prchlivosti své ubiji te, a v dobré líbeznosti své slituji se nad tebou.
\par 11 A otevríny budou brány tvé ustavicne, ve dne ani v noci nebudou zavírány, aby privedli k tobe sílu pohanu, i králové jejich aby privedeni byli.
\par 12 Národ zajisté ten a království, kteréž by nesloužilo tobe, zahyne; národové, pravím, ti docela pohubeni budou.
\par 13 Sláva Libánská prijde k tobe, jedle, jilm, též i pušpan k ozdobe místa svatyne mé, abych místo noh svých oslavil.
\par 14 Také prijdou k tobe s ponížením synové tech, kteríž te trápili, a klaneti se budou k zpodku noh tvých, kterížkoli pohrdali tebou, a nazývati te budou mestem Hospodinovým, Sionem Svatého Izraelského.
\par 15 Místo toho, že jsi byla opuštená a v nenávisti, tak že žádný skrze te nechodil, zpusobímt dustojnost vecnou, a veselí od národu do pronárodu.
\par 16 Nebo ssáti budeš mléko národu, a prsy králu ssáti budeš; i poznáš, že jsem já Hospodin vysvoboditel tvuj, a vykupitel tvuj silný Jákobuv.
\par 17 Místo medi dodávati budu zlata, a místo železa dodávati budu stríbra, a místo dríví medi, a místo kamení železa, a predstavímt správce pokojné a úredníky spravedlivé.
\par 18 Nebude více slyšáno o bezpraví v zemi tvé, o zpuštení a zhoube na hranicích tvých, ale hlásati budeš spasení na zdech svých, a v branách svých chválu.
\par 19 Nebudeš míti více slunce za svetlo denní, a blesk mesíce nebude te osvecovati, ale budet Hospodin svetlem tvým vecným, a Buh tvuj okrasou tvou.
\par 20 Nezajdet více slunce tvé, a mesíc tvuj neschová se, nebo Hospodin bude svetlem tvým vecným, a tak dokonáni budou dnové smutku tvého.
\par 21 Lid také tvuj, kteríž by koli byli spravedliví, na veky dedicne obdrží zemi, výstrelek štípení mého, dílo rukou mých, abych v nem oslavován byl.
\par 22 Samotný rozmnoží se v tisíce, a nejšpatnejší v národ nescíslný, já Hospodin casem svým brzo zpusobím to.

\chapter{61}

\par 1 Duch Panovníka Hospodina jest nade mnou, proto že pomazal mne Hospodin, abych kázal evangelium tichým. Poslal mne, abych uvázal rány skroušených srdcem, abych vyhlásil jatým svobodu, a veznum otevrení žaláre,
\par 2 Abych vyhlásil léto milostivé Hospodinovo, a den pomsty Boha našeho, abych tešil všecky kvílící,
\par 3 Abych zpusobil radost kvílícím Sionským, a dal jim okrasu místo popela, olej veselé místo smutku, odev chvály místo ducha sevreného. I nazvání budou stromové spravedlnosti, štípení Hospodinovo, abych oslavován byl.
\par 4 Tedy vzdelají pustiny starodávní, poušte staré spraví, a obnoví mesta zpuštená, pustá po mnohé národy.
\par 5 Nebo postaví se cizozemci, a pásti budou stáda vaše, a synové cizozemcu oráci vaši a vinari vaši budou.
\par 6 Vy pak kneží Hospodinovi nazváni budete, služebníci Boha našeho slouti budete, zboží pohanu užívati budete, a v sláve jejich zvýšeni budete.
\par 7 Za dvojnásobní zahanbení vaše a pohanení prozpevovati budete, z podílu jejich a v zemi jejich dvojnásobní dedictví obdržíte, a tak veselé vecné míti budete.
\par 8 Já zajisté Hospodin miluji soud, a nenávidím loupeže pri obeti, a protož zpusobím, aby skutkové jejich dáli se v pravde, a smlouvu vecnou s nimi uciním.
\par 9 I vejdet v známost mezi pohany síme jejich, a potomci jejich u prostred národu. Všickni, kteríž je uzrí, poznají je, že jsou síme, jemuž požehnal Hospodin.
\par 10 Velice se budu radovati v Hospodinu, a plésati bude duše má v Bohu mém; nebo mne oblékl v roucho spasení, a pláštem spravedlnosti priodel mne jako ženicha, kterýž se strojí ozdobne, a jako nevestu okrašlující se ozdobami svými.
\par 11 Nebo jakož zeme vydává zrostlinu svou, a jakož zahrada síme své vyvodí, tak Panovník Hospodin vyvede spravedlnost a chválu prede všemi národy.

\chapter{62}

\par 1 Pro Sion nebudu mlceti, a pro Jeruzalém neupokojím se, dokudž nevyjde jako blesk spravedlnost jeho, a spasení jeho jako pochodne horeti nebude.
\par 2 I uzrí národové spravedlnost tvou, a všickni králové slávu tvou, a nazovou te jménem novým, kteréž ústa Hospodinova vyrknou.
\par 3 Nadto budeš korunou ozdobnou v ruce Hospodinove, a korunou královskou v ruce Boha svého.
\par 4 Nebudeš více slouti opuštená, a zeme tvá nebude více slouti pustinou, ale ty nazývána budeš rozkoší, a zeme tvá vdanou; nebo rozkoš míti bude Hospodin v tobe, a zeme tvá bude vdaná.
\par 5 Nebo jakož pojímá mládenec pannu, tak te sobe pojmou synové tvoji, a jakou má radost ženich z nevesty, tak radovati se bude z tebe Buh tvuj.
\par 6 Na zdech tvých, Jeruzaléme, postavím strážné, kteríž pres celý den i pres celou noc nikdy nebudou mlceti. Kteríž tedy pripomínáte Hospodina, nemlctež,
\par 7 A nedávejte jemu pokoje, dokudž neutvrdí, a dokudž nezpusobí, aby Jeruzalém byl slavný na zemi.
\par 8 Prisáhlte Hospodin skrze pravici svou, a skrze ráme síly své, rka: Nikoli nedám obilí tvého více za pokrm neprátelum tvým, aniž píti budou cizozemci vína tvého, o nemž jsi pracoval.
\par 9 Ale ti, kteríž je shromaždují, budou je jísti, a chváliti Hospodina, a kteríž je zbírají, budou je píti v síncích svatosti mé.
\par 10 Vejdetež, vejdetež branami, spravte cestu lidu, vyrovnejte, vyrovnejte silnici, vyberte kamení, vyzdvihnete korouhev k národum.
\par 11 Aj, Hospodin rozkáže provolati až do koncin zeme: Rcetež dceri Sionské: Aj, Spasitel tvuj bére se, aj, mzda jeho s ním, a dílo jeho pred ním.
\par 12 I nazovou syny tvé lidem svatým, vykoupenými Hospodinovými, ty pak slouti budeš mestem vzácným a neopušteným.

\chapter{63}

\par 1 Kdož jest to, ješto se bére z Edom, v ubroceném rouše z Bozra, ten ozdobený rouchem svým, kráceje u velikosti síly své? Ját jsem, kterýž mluvím spravedlive, dostatecný k vysvobození.
\par 2 Proc jest cervené roucho tvé, a odev tvuj jako toho, kterýž tlací v presu?
\par 3 Pres jsem tlacil sám, aniž kdo z lidí byl se mnou. Tlacil jsem neprátely v hneve svém, a pošlapal jsem je v prchlivosti své, až stríkala krev i nejsilnejších jejich na roucho mé, a tak všecken odev svuj zkálel jsem.
\par 4 Den zajisté pomsty v srdci mém, a léto, v nemž mají vykoupeni býti moji, prišlo.
\par 5 Když jsem pak videl, že není žádného spomocníka, až jsem se užasl, že žádného nebylo, kdo by podpíral. A protož mi vysvobození zpusobilo ráme mé, a prchlivost má, ta mne podeprela.
\par 6 I pošlapal jsem národy v hneve svém, a opojil jsem je prchlivostí svou, a porazil jsem na zem nejsilnejší reky jejich.
\par 7 Milosrdenství Hospodinova pripomínati budu, a chvály Hospodinovy ze všeho, což ucinil nám Hospodin, i množství dobroty, kteréž dokazoval k domu Izraelskému z veliké lítosti své, a z velikého milosrdenství svého.
\par 8 Nebo rekl: Vždyt jsou lidem mým, jsou synové, neucinít mi neverne. A protož byl jejich spasitelem.
\par 9 Ve všelikém ssoužení jejich i on mel ssoužení, a andel prístojící jemu vysvobozval je. Z milování svého a z lítosti své on sám vykoupil je, a pestoval je, i nosil je po všecky dny veku.
\par 10 Ale oni zpurní byli, a zarmucovali Ducha svatého jeho; procež obrátil se jim v neprítele, a sám bojoval proti nim.
\par 11 I rozpomínal se lid jeho na dny starodávní, i na Mojžíše: Kdež jest ten, kterýž je vyvedl z more s pastýrem stáda svého? Kde jest ten, kterýž položil u prostred neho Ducha svatého svého?
\par 12 Kterýž je vedl ramenem velebnosti své po pravici Mojžíšove, kterýž rozdelil vody pred nimi, aby sobe zpusobil jméno vecné,
\par 13 Kterýž je provedl skrze hlubiny jako kone po poušti, ani se nepoklesli.
\par 14 Jako když hovádko do údolí sstupuje,tak Duch Hospodinuv poznenáhlu vedl z nich každého. Tak jsi vedl lid svuj, abys sobe zpusobil jméno slavné.
\par 15 Popatriž s nebe, a pohled z príbytku svatosti své a okrasy své. Kdež jest horlivost tvá a veliká síla tvá? Kde množství milosrdenství tvých a slitování tvých? Mne-liž se zadržovati budou?
\par 16 Ty jsi zajisté otec náš; nebo Abraham nic neví o nás, a Izrael nezná nás. Ty jsi, Hospodine, otec náš, vykupitel náš, tot jest od vecnosti jméno tvé.
\par 17 Procež jsi nám dal zblouditi, Hospodine, od cest svých, zatvrdil jsi srdce naše, abychom se nebáli tebe? Navratiž se zase pro služebníky své, pokolení dedictví svého.
\par 18 Nejšpatnejší vládne lidem svatosti tvé, neprátelé naši pošlapali svatyni tvou.
\par 19 My tvoji jsme od veku; nad nimi jsi nikdy nepanoval, aniž nad nimi jméno tvé vzýváno jest.

\chapter{64}

\par 1 Ó bys protrhl nebesa a sstoupil, aby se od prítomnosti tvé hory rozplynouti musily,
\par 2 (Jako od rozníceného ohne rozpouštejícího voda vre), abys v známost uvedl jméno své neprátelum svým, a aby se pred tvárí tvou národové trásli;
\par 3 Jako když jsi cinil hrozné veci, jichž jsme se nenadáli, sstoupil jsi, pred oblícejem tvým hory se rozplývaly;
\par 4 Cehož se od veku neslýchalo, a ušima nepochopilo, oko nevídalo Boha krome tebe, aby tak cinil tomu, kterýž nan ocekává.
\par 5 Vyšel jsi vstríc tomu, kdož ochotne ciní spravedlnost, a na cestách tvých na te se rozpomínali. Aj, ty rozhnevals se, proto že jsme hrešili na nich ustavicne, a však zachováni budeme,
\par 6 Ackoli jsme jako necistý my všickni, a jako roucho ohyzdné všecky spravedlnosti naše. Procež pršíme jako list my všickni, a nepravosti naše jako vítr zachvacují nás.
\par 7 Nadto není žádného, ješto by vzýval jméno tvé, a probudil se k tomu, aby se chopil tebe, aspon když jsi skryl tvár svou pred námi, a zpusobil to, abychom mizeli pro nepravosti naše.
\par 8 Ale již, ó Hospodine, ty jsi otec náš, my hlina, ty pak ucinitel náš, a tak jsme všickni dílo ruky tvé.
\par 9 Nehnevejž se tak velmi, Hospodine, aniž se na veky rozpomínej na nepravost. Ó vzhlédniž, prosíme, všickni my lid tvuj jsme.
\par 10 Mesta svatosti tvé obrácena jsou v poušt, Sion v poušt, i Jeruzalém v pustinu obrácen.
\par 11 Dum svatosti naší a okrasy naší, v kterémž te chválívali otcové naši, ohnem zkažen, a cožkoli jsme meli nejvzácnejšího, jest popléneno.
\par 12 I zdaliž pro ty veci, Hospodine, se zdržíš? Mlceti a nás tak velmi trápiti budeš?

\chapter{65}

\par 1 Dal jsem se najíti tem, kteríž se na mne neptávali, nalezen jsem od tech, kteríž mne nehledali, a národu, kterýž se nenazýval jménem mým, rekl jsem: Ted jsem, ted jsem.
\par 2 Rozprostíral jsem ruce své na každý den k lidu zpurnému, kteríž chodí cestou nedobrou za myšlénkami svými,
\par 3 K lidu, kteríž zjevne popouzejí mne ustavicne, obetujíce v zahradách, a kadíce na cihlách,
\par 4 Kteríž sedají pri hrobích, a pri svých modlách nocují, kteríž jedí maso svinské, a polívku necistého z nádob svých,
\par 5 Ríkajíce: Táhni prec, nepristupuj ke mne, nebo svetejší jsem nežli ty. Tit jsou dymem v chrípích mých a ohnem horícím pres celý den.
\par 6 Aj, zapsáno jest to prede mnou: Nebudut mlceti, nýbrž nahradím a odplatím do luna jejich,
\par 7 Za nepravosti vaše, a spolu za nepravosti otcu vašich, praví Hospodin, kteríž kadívali na horách, a na pahrbcích v lehkost uvodili mne; procež odmerím za dílo jejich predešlé do luna jejich.
\par 8 Takto praví Hospodin: Jako nekdo nalezna víno v hroznu, i rekl by: Nekaz ho, proto že požehnání jest v nem, tak i já uciním pro služebníky své, že nevyhladím všech techto.
\par 9 Nebo vyvedu z Jákoba síme, a z Judy toho, kterýž by dedicne obdržel hory mé, i budou ji dedicne držeti vyvolení moji, a služebníci moji bydliti budou tam.
\par 10 Lid pak muj, kteríž by mne hledali,budou míti Sáron za pastvište ovcím, a údolí Achor za odpocivadlo skotum.
\par 11 Ale vás, kteríž opouštíte Hospodina, kteríž se zapomínáte na horu svatosti mé, kteríž strojíte vojsku tomu stul, a kteríž vykonáváte tomu poctu obeti,
\par 12 Vás, pravím, sectu pod mec, tak že všickni vy k zabití na kolena padati budete, proto že, když jsem volal, neohlásili jste se, mluvil jsem, a neslyšeli jste, ale cinili jste to, což zlého jest pred ocima mýma, a to, cehož neoblibuji, vyvolili jste.
\par 13 A protož takto dí Panovník Hospodin: Aj, služebníci moji jísti budou, vy pak hlad trpeti budete: aj, služebníci moji píti budou, vy pak žízniti budete; aj, služebníci moji veseliti se budou, vy pak zahanbeni budete.
\par 14 Aj, služebníci moji prozpevovati budou pro radost srdce, vy pak kriceti budete pro bolest srdce, a pro setrení ducha kvíliti budete,
\par 15 A zanecháte jména svého k proklínání vyvoleným mým, když vás pomorduje Panovník Hospodin, služebníky pak své nazuve jménem jiným.
\par 16 Ten, kterýž bude sobe požehnání dávati na zemi, požehnání dávati sobe bude v Bohu pravém, a kdož prisahati bude na zemi, prisahati bude skrze Boha pravého; v zapomenutí zajisté dána budou ta ssoužení první, a budou skryta od ocí mých.
\par 17 Nebo aj, já stvorím nebesa nová a zemi novou, a nebudou pripomínány první veci, aniž vstoupí na srdce.
\par 18 Anobrž radujte se a veselte se na veky veku z toho, což já stvorím; nebo aj, já stvorím Jeruzalém k plésání, a lid jeho k radosti.
\par 19 I já plésati budu v Jeruzaléme, a radovati se v lidu svém, aniž se více bude slýchati v nem hlasu pláce, aneb hlasu kriku.
\par 20 Nebude tam více žádného v veku detinském, ani starce, kterýž by dnu svých nevyplnil; nebo díte ve stu letech umre, hríšníku pak, by došel i sta let, zloreceno bude.
\par 21 Nastavejí též domu, a bydliti budou v nich, i vinice štepovati budou, a jísti budou ovoce jejich.
\par 22 Nebudou staveti tak, aby jiný bydlil, nebudou štepovati, aby jiný jedl; nebo jako dnové dreva budou dnové lidu mého, a díla rukou svých do zvetšení užívati budou vyvolení moji.
\par 23 Nebudout pracovati nadarmo, aniž ploditi budou k strachu; nebo budou síme požehnaných od Hospodina, i potomkové jejich s nimi.
\par 24 Nadto stane se, že prvé než volati budou, já se ohlásím; ješte mluviti budou, a já vyslyším.
\par 25 Vlk s beránkem budou se pásti spolu, a lev jako vul bude jísti plevy, hadu pak za pokrm bude prach. Neuškodít, aniž zahubí na vší mé hore svaté, praví Hospodin.

\chapter{66}

\par 1 Takto praví Hospodin: Nebe jest mi stolice, a zeme podnože noh mých. Kdež ten dum bude, kterýž mi vzdeláte? Aneb kde bude místo odpocívání mého?
\par 2 Nebo všecko to ruka má ucinila, a jí stojí všecko, praví Hospodin. I však na toho patrím, kdož jest chudý a skroušeného ducha, a trese se pred slovem mým.
\par 3 Sic jinak ten,kdož zabijí vola, zabil cloveka; kdo zabijí hovádko, psa stal; kdo obetuje obet suchou, krev svinskou obetoval; kdo kadí kadidlem, dary dával modle. To oni vyvolili na cestách svých, proto že, v ohavnostech svých duše jejich se kochá.
\par 4 I ját také vyvolím za nešlechetnosti jejich, a to, cehož se strachují, na ne uvedu, proto že, když jsem volal, žádný se neohlásil, když jsem mluvil, neslyšeli, ale cinili to, což zlého jest pred ocima mýma, a to, cehož neoblibuji, vyvolili.
\par 5 Slyšte slovo Hospodinovo, kteríž se tresete pred slovem jeho: Ríkávají bratrí vaši, v nenávisti majíce vás, a vypovídajíce vás pro jméno mé: Necht se zjeví sláva Hospodinova. Ukážet se zajisté ku potešení vašemu, ale oni zahanbeni budou.
\par 6 Hlas hrmotu z mesta, hlas z chrámu, hlas Hospodinuv, an odplatu dává neprátelum svým.
\par 7 Prvé než pracovala ku porodu, porodila; prvé než prišla na ni bolest, porodila pacholátko.
\par 8 Kdo slýchal co takového? Kdo vídal co podobného? Zdaliž muže zpusobeno býti, aby zeme zplodila lid dne jednoho? Zdaliž zplozen bývá národ pojednou? Ale Sion jen pocal pracovati ku porodu, a porodil syny své.
\par 9 Což bych já, kterýž otvírám život matky, neplodil? praví Hospodin. Což bych já, kterýž ciním to, aby rodily, zavrín byl? praví Buh tvuj.
\par 10 Veselte se s Jeruzalémem, a plésejte v nem všickni, kteríž jej milujete. Radujte se s ním velice, kteríž jste koli kvílili nad ním,
\par 11 Proto že ssáti budete, a sytiti se z prs potešení jeho, ssáti budete, a rozkoš míti v blesku slávy jeho.
\par 12 Nebo takto praví Hospodin: Aj, já obrátím na nej jako reku pokoj, a jako potok rozvodnilý slávu národu. I budete ssáti, na rukou pestováni a na klíne rozkošne chováni budete.
\par 13 Jako ten, kteréhož matka jeho teší, tak já vás tešiti budu, a tak v Jeruzaléme potešováni budete.
\par 14 Uzríte zajisté, a radovati se bude srdce vaše, a kosti vaše jako bylinka zkvetnou. I seznána bude ruka Hospodinova pri služebnících jeho, a prchlivost proti neprátelum jeho.
\par 15 Nebo aj, Hospodin v ohni prijde, a jako vichrice budou vozové jeho, aby vypustil v prchlivosti hnev svuj, a žehrání své v plameni ohne.
\par 16 Hospodin, pravím, ohnem mstíti bude, a mecem svým nad všelikým telem, tak že mnoho bude zbitých od Hospodina.
\par 17 I ti, kteríž se posvecují a ocištují v zahradách, jeden po druhém zjevne, kteríž jedí maso svinské, a vec ohavnou i myši, též konec vezmou, praví Hospodin.
\par 18 Nebo já, když skutkové a myšlení jejich prijdou, shromáždím všecky národy a jazyky. I prijdou, a uzrí slávu mou.
\par 19 A položím na ne znamení, a pošli z nich, kteríž zachováni budou, k národum do Tarsu, Pul a Lud, jenž natahují lucište, do Tubal a Javan, na ostrovy daleké, kteríž neslýchali povesti o mne, aniž vídali slávy mé. I budou zvestovati slávu mou mezi národy.
\par 20 A privedou všecky bratrí vaše ze všech národu za dar Hospodinu, na koních a na vozích, a na nuších a na mezcích, a na dromedárích, na horu svatosti mé do Jeruzaléma, praví Hospodin, tak jako prinášejí synové Izraelští dar v nádobe cisté do domu Hospodinova.
\par 21 A z techt také vezmu za kneží a za Levíty, praví Hospodin.
\par 22 Nebo jakož ta nebesa nová, a zeme ta nová, kterouž já uciním, stane prede mnou, praví Hospodin, tak stane síme vaše a jméno vaše.
\par 23 I stane se, že od novmesíce do novmesíce, od soboty do soboty pricházeti bude všeliké telo, aby se klanelo prede mnou, praví Hospodin.
\par 24 A vyjdouce, uzrí tela mrtvá lidí tech, kteríž se mi zproneverili; nebo cerv jejich neumre, a ohen jejich neuhasne. I budout v ošklivosti všelikému telu.

\end{document}