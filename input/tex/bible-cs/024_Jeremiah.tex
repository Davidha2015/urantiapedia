\begin{document}

\title{Jeremiah}

\chapter{1}

\par 1 Slova Jeremiáše syna Helkiášova, z kneží, kteríž byli v Anatot, v zemi Beniamin,
\par 2 K nemuž se stalo slovo Hospodinovo za dnu Joziáše syna Amonova, krále Judského, trináctého léta kralování jeho.
\par 3 Byl i za dnu Joakima syna Joziášova, krále Judského, až do vyplnení jedenáctého léta Sedechiáše syna Joziášova, krále Judského, až do zajetí Jeruzaléma mesíce pátého.
\par 4 Stalo se, pravím, slovo Hospodinovo ke mne, rkoucí:
\par 5 Dríve než jsem te sformoval v živote, znal jsem tebe, a dríve nežlis vyšel z života, posvetil jsem te, za proroka národum dal jsem tebe.
\par 6 I rekl jsem: Ach, Panovníce Hospodine, aj, neumím mluviti, nebo díte jsem.
\par 7 Ale Hospodin rekl mi: Neríkej, díte jsem, nýbrž k cemuž te koli pošli, jdi, a vše, cožt prikáži, mluv.
\par 8 Neboj se jich, nebot jsem s tebou, abych te vysvobozoval, dí Hospodin.
\par 9 A vztáhna ruku svou Hospodin, dotekl se úst mých, a rekl mi Hospodin: Aj, vložil jsem slova svá v ústa tvá.
\par 10 Hle, ustanovuji te dnešního dne nad národy a nad královstvími, abys plénil a kazil, a hubil a boril, abys stavel a štepoval.
\par 11 Potom se stalo slovo Hospodinovo ke mne, rkoucí: Co vidíš, Jeremiáši? I rekl jsem: Prut mandlový vidím.
\par 12 Tedy rekl mi Hospodin: Dobre vidíš; nebo pospíchám já s slovem svým, abych je vykonal.
\par 13 Opet stalo se slovo Hospodinovo ke mne po druhé, rkoucí: Co vidíš? I rekl jsem: Vidím hrnec, an vre, a prední strana jeho k strane pulnocní.
\par 14 Tedy rekl mi Hospodin: Od pulnoci privalí se to zlé na všecky obyvatele této zeme.
\par 15 Nebo aj, já svolám všecky rodiny království pulnocních, dí Hospodin, aby pritáhnouce, postavili jeden každý stolici svou v branách Jeruzalémských, a pri všech zdech jeho vukol, a pri všech mestech Judských.
\par 16 A tak vypovím úsudky své proti nim, pro všelikou nešlechetnost tech, kteríž opustili mne, a kadili bohum cizím, a skláneli se dílu rukou svých.
\par 17 Protož ty prepaš bedra svá, a vstana, mluv k nim, cožkoli já prikazuji tobe. Nelekej se jich, abych te nepotrel pred oblícejem jejich.
\par 18 Nebo aj, já postavuji te dnes jako mesto hrazené, a jako sloup železný, a jako zdi medené proti vší této zemi, proti králum Judským, proti knížatum jejím, proti knežím jejím, a lidu zeme této.
\par 19 Kteríž bojovati budou proti tobe, ale neodolají proti tobe. Nebo já jsem s tebou, praví Hospodin, abych te vysvobozoval.

\chapter{2}

\par 1 I stalo se slovo Hospodinovo ke mne, rkoucí:
\par 2 Jdi a volej, tak aby slyšel Jeruzalém, rka: Takto praví Hospodin: Rozpomínám se na te pro milosrdenství mladosti tvé, a pro lásku snetí tvého, když jsi za mnou chodila po poušti v zemi, kteráž nebývá osívána.
\par 3 Tehdáž svatost Hospodinova byl Izrael, prvotiny úrod jeho. Všickni, kteríž jej zžírali, obvineni byli; zlé veci na ne prišly, praví Hospodin.
\par 4 Slyšte slovo Hospodinovo, dome Jákobuv, a všecky celedi domu Izraelského.
\par 5 Takto praví Hospodin: Jakou shledali otcové vaši pri mne nepravost, že se vzdálili ode mne, a chodíce za marností, marní ucineni jsou,
\par 6 Tak že ani nerekli: Kde jest Hospodin, kterýž nás vyvedl z zeme Egyptské, kterýž nás vodil po poušti, po zemi pusté a strašlivé, po zemi vyprahlé a stínu smrti, po zemi, skrze niž nechodil žádný, a kdež žádný clovek nebydlil?
\par 7 Nýbrž, když jsem vás uvedl do zeme úrodné, abyste jedli ovoce její i dobré veci její, všedše tam, poškvrnili jste zeme mé, a dedictví mé zohavili jste.
\par 8 Kneží nerekli: Kde jest Hospodin? a ti, kteríž se obírají s zákonem, nepoznali mne, pastýri pak odstoupili ode mne, a proroci prorokovali skrze Bále, a za vecmi neužitecnými chodili.
\par 9 Procež vždy nesnáz mám s vámi, praví Hospodin, i s syny synu vašich nesnáz míti musím.
\par 10 Projdete ale ostrovy Citim, a pohledte, i do Cedar pošlete, a pošetrte pilne, a pohledte, stalo-li se co takového.
\par 11 Zdali zmenil který národ bohy, ackoli nejsou bohové? Lid pak muj zmenil slávu svou v vec neužitecnou.
\par 12 Užasnete se nebesa nad tím, a deste se, chradnete velmi, praví Hospodin.
\par 13 Nebo dvojí zlost spáchal lid muj: Mne opustili pramen vod živých, aby sobe vykopali cisterny, cisterny deravé, kteréž nedrží vody.
\par 14 Zdali otrok jest Izrael? Zdali man doma zplozený? Procež vydán jest v loupež?
\par 15 Lvícata rvou na nej, a vydávají hlas svuj, a obracejí zemi jeho v pustinu; mesta jeho vypálena jsou, tak že není žádného obyvatele.
\par 16 Obyvatelé také Nof a Tachpanes pasou na vrchu hlavy tvé.
\par 17 Zdaliž toho sobe nepusobíš, opouštejíc Hospodina Boha svého v ten cas, když te vodí po ceste své?
\par 18 A nyní co tobe do cesty Egyptské, že piješ vodu z Níle? Aneb co tobe do cesty Assyrské, že piješ vodu z reky?
\par 19 Trestati te bude zlost tvá, a odvrácení tvá domlouvati budou tobe. Poznejž tedy a viz, že zlá a horká vec jest, že opouštíš Hospodina Boha svého, a není bázne mé pri tobe, dí Panovník Hospodin zástupu.
\par 20 Ackoli dávno polámal jsem jho tvé, potrhal jsem to, cím jsi svázána byla, a reklas: Nebudut sloužiti modlám, však po každém pahrbku vysokém, a pod každým drevem zeleným touláš se, ó nevestko.
\par 21 Ješto jsem já te vysadil vinným kmenem výborným, všecku naporád semenem cistotným, i kterakž jsi mi promenila se v plané réví cizího kmene?
\par 22 Nebo bys ty se pak umyla sanitrem, a mnoho na sebe mýdla vypotrebovala, predcet patrná jest nepravost tvá pred oblícejem mým, praví Panovník Hospodin.
\par 23 Kterakž mužeš ríci: Nepoškvrnovala jsem se, za Báli jsem nechodila? Pohled na cestu svou v tomto údolí, poznej, cos cinila, dromedárko rychlá, kteráž znamení necháváš na cestách svých.
\par 24 Jsi divoká oslice, zvyklá na poušti, kteráž podlé líbosti duše své hltá vítr, když se jí prícina dá. Kdo jí prekážku uciní? Všickni ti, kteríž jí hledají, nepotrebí se jim kvaltovati, naleznout ji v mesíci jejím.
\par 25 Dí-lit kdo: Zdržuj nohu svou, aby bosá nebyla, a hrdlo své od žízne, tedy ríkáš: To nic, nikoli; nebo jsem zamilovala cizí, a za nimi choditi budu.
\par 26 Jakož k hanbe prichází zlodej, když postižen bývá, tak zahanben bude dum Izraelský, oni, králové jejich, knížata jejich, a kneží jejich, i proroci jejich,
\par 27 Kteríž ríkají drevu: Otec muj jsi, a kameni: Ty jsi mne zplodil. Nebo se hrbetem ke mne obracejí a ne tvárí, ale v cas trápení svého ríkají: Vstan a vysvobod nás.
\par 28 I kdež jsou bohové tvoji, kterýchž jsi nadelal sobe? Necht vstanou, budou-li te moci vysvoboditi v cas trápení tvého, ponevadž podlé poctu mest svých máš bohy své, ó Judo.
\par 29 Co se vaditi budete se mnou? Vy všickni odstoupili jste ode mne, dí Hospodin.
\par 30 Nadarmo jsem bil syny vaše, kázne neprijali; sežral mec váš proroky vaše jako lev, kterýž dáví.
\par 31 Ó národe, vy posudte slova Hospodinova, zdali jsem byl pouští Izraelovi, zdali zemí tmavou? Proc ríká lid muj: Panujeme, neprijdeme více k tobe?
\par 32 Zdali se zapomíná panna na ozdoby své, a nevesta na tkanice své? Lid pak muj zapomnel se na mne za dny nescíslné.
\par 33 Proc zastáváš cesty své, hledajíc toho, což miluješ? Procež i jiné nešlechetnice ucíš cestám svým.
\par 34 Nad to, na podolcích tvých nalézá se krev duší chudých nevinných. Nenesnadne nalézám to, nebo videti to na tech všech podolcích.
\par 35 A vždy ríkáš: Ponevadž nevinná jsem, jiste odvrácena jest prchlivost jeho ode mne. Aj, já v soud vejdu s tebou, proto že pravíš: Nehrešila jsem.
\par 36 Proc tak beháš, promenujíc cestu svou? Jakož jsi zahanbena od Assyrských, tak i od Egyptských zahanbena budeš.
\par 37 Také odtud vyjdeš, a ruce tvé budou nad hlavou tvou; nebo zamítá Hospodin troštování tvá, a nepovedet se štastne v nich.

\chapter{3}

\par 1 Dí dále: Propustil-li by muž ženu svou, a ona odejduc od neho, vdala by se za jiného muže, zdaliž se navrátí k ní více? Zdaliž by hrozne nebyla poškvrnena zeme ta? Ale ty, ac jsi smilnila s milovníky mnohými, a však navratiž se ke mne, dí Hospodin.
\par 2 Pozdvihni ocí svých k vysokým místum, a pohled, kdes necizoložila? Na cestách usazovalas se jim jako Arab na poušti, a poškvrnila jsi zeme smilstvím svým a nešlechetností svou.
\par 3 A ackoli zadržáni jsou podzimní deštové, a dešte jarního nebývalo, však celo ženy nevestky majíc, nechtelas se stydeti.
\par 4 Zdali od nynejšího casu volati budeš ke mne: Otce muj, ty jsi vudce mladosti mé?
\par 5 Zdaliž Buh držeti bude hnev na vecnost? Zdali chovati jej bude na veky? Aj, mluvíš i pášeš zlé veci, jakž jen mužeš.
\par 6 Tedy rekl mi Hospodin za dnu Joziáše krále: Videl-lis, co cinila zpurná dcera Izraelská? Chodívala na každou horu vysokou, i pod každé drevo zelené, a tam smilnila.
\par 7 A ackoli jsem rekl, když ty všecky veci ona cinila: Necht se navrátí ke mne, však se nenavrátila. Nacež hledela zproneverilá sestra její, dcera Judská.
\par 8 Procež videlo mi se pro ty všecky príciny, ponevadž cizoložila zpurná dcera Izraelská, propustiti ji, a dáti jí lístek zapuzení jejího. Však se vždy neulekla zproneverilá sestra její, dcera Judská, ale šedši, smilnila také sama.
\par 9 I stalo se, že hanebným smilstvím svým poškvrnila zeme; nebo cizoložila s kamenem i s drevem.
\par 10 A však s tím se vším neobrátila se ke mne zproneverilá sestra její, dcera Judská, celým srdcem svým, ale pokryte, praví Hospodin.
\par 11 Protož rekl Hospodin ke mne: Ospravedlnila duši svou zpurná dcera Izraelská, více nežli zproneverilá Judská.
\par 12 Jdi a volej slovy temito ku pulnoci, a rci: Navrat se, zpurná dcero Izraelská, dí Hospodin, a neoborí se tvár má zurivá na vás; nebo já dobrotivý jsem, dí Hospodin, aniž držím hnevu na vecnost.
\par 13 Jen toliko poznej nepravost svou, že jsi od Hospodina Boha svého odstoupila, a sem i tam behala cestami svými k cizím pod každé drevo zelené, a hlasu mého neposlouchali jste, dí Hospodin.
\par 14 Navratte se synové zpurní, dí Hospodin; nebo já jsem manžel váš, a prijmu vás, jednoho z mesta, a dva z celedi, abych vás uvedl na Sion.
\par 15 Kdežto dám vám pastýre podlé srdce svého, kteríž pásti vás budou umele a rozumne.
\par 16 I stane se, když se rozmnožíte a rozplodíte v této zemi za dnu tech, dí Hospodin, že nebudou ríkati více: Truhla smlouvy Hospodinovy, aniž jim vstoupí na srdce, aniž zpomenou na ni, ani k ní choditi, aniž bude více u vážnosti.
\par 17 V ten cas nazývati budou Jeruzalém stolicí Hospodinovou, a shromáždí se tam všickni národové ke jménu Hospodinovu do Jeruzaléma, a nebudou choditi více podlé zdání srdce svého zlého.
\par 18 V tech dnech prijdou dum Judský s domem Izraelským, a priberou se spolu z zeme pulnocní do zeme, kterouž jsem v dedictví uvedl otcum vašim;
\par 19 Ac já pravím: Kterakž bych te pocísti mohl mezi syny, a dáti tobe zemi žádostivou, dedictví slavné zástupu pohanských, lec abys mne vzýval, ríkaje: Otce muj, a od následování mne abys se neodvracel?
\par 20 Ponevadž jakož žena zproneveruje se manželu svému, tak jste se zproneverili mne, dome Izraelský, dí Hospodin.
\par 21 Hlas po místech vysokých bud slyšán, plác pokorné modlitby synu Izraelských. Nebo prevrátivše cesty své, zapomneli se na Hospodina Boha svého, rkoucího:
\par 22 Navratte se, synové zpurní, a uzdravím odvrácení vaše. Rcete: Aj, my jdeme k tobe, nebo ty, Hospodine, jsi Buh náš.
\par 23 Práve marné jest v pahrbcích a v množství hor doufání; zajisté v Hospodinu Bohu našem jest spasení Izraelovo.
\par 24 Nebo ohavnost ta zžírala práci otcu našich od detinství našeho, bravy jejich i skoty jejich, syny jejich i dcery jejich.
\par 25 Ležíme v hanbe své, a prikrývá nás pohanení naše, že jsme proti Hospodinu Bohu svému hrešili, my i otcové naši, od detinství svého až do dne tohoto, a neposlouchali jsme hlasu Hospodina Boha svého.

\chapter{4}

\par 1 Budeš-li se míti navrátiti, Izraeli, dí Hospodin, ke mne se navrat. Nebo odejmeš-li ohavnosti své od tvári mé a nebudeš-li se toulati,
\par 2 A budeš-li prisahati práve, náležite a spravedlive, ríkaje: Živt jest Hospodin,tedy požehnání dávati sobe v nem budou národové, a v nem se chlubiti.
\par 3 Nebo takto praví Hospodin mužum Judským a Jeruzalémským: Zorte sobe ouhor, a nerozsívejte do trní.
\par 4 Obrežte se Hospodinu, a odejmete neobrízky srdce vašeho, muži Judští a obyvatelé Jeruzalémští, aby nevyšla jako ohen prchlivost má, a nehorela, tak že by nebyl kdo uhasiti, pro nešlechetnost predsevzetí vašich.
\par 5 Oznamte v Judstvu, a v Jeruzaléme ohlaste, a rcete: Trubte trubou v zemi, svolejte a sberte lid, a rcete: Shromaždte se, a vejdeme do mest hrazených.
\par 6 Vyzdvihnete korouhev na Sionu, zmužile se mejte, nepostávejte; nebo já zlé veci uvedu od pulnoci, a potrení veliké.
\par 7 Vychází lev z houšte své, a ten, kterýž hubí národy, vyšed z místa svého, táhne, aby obrátil zemi tvou v pustinu, a mesta tvá aby zborena byla, tak aby nebylo žádného obyvatele.
\par 8 Protož prepašte se žínemi, kvelte a naríkejte; nebo není odvrácen hnev prchlivosti Hospodinovy od nás.
\par 9 Stane se zajisté v ten den, dí Hospodin, že zhyne srdce královo a srdce knížat, užasnou se i kneží, a proroci diviti se budou.
\par 10 I rekl jsem: Ach, Panovníce Hospodine, jiste že jsi velice podvedl lid tento, i Jeruzalém, ríkaje: Pokoj míti budete, a však pronikl mec až k duši.
\par 11 V ten cas receno bude lidu tomuto i Jeruzalému: Vítr tuhý z míst vysokých na poušti jde uprímo na lid muj, ne aby prevíval, ani precištoval.
\par 12 Vítr silnejší než oni prijde mi, nyní já také vypovím jim úsudky.
\par 13 Aj, jako oblakové vystupuje, a jako vicher vozové jeho, rychlejší jsou nežli orlice koni jeho. Beda nám, nebo popléneni jsme.
\par 14 Obmej od nešlechetnosti srdce své, Jeruzaléme, abys vysvobozen byl. Dokudž zustávati budou u prostred tebe myšlení marnosti tvé?
\par 15 Nebo hlas oznamujícího prichází od Dan, a toho, kterýž ohlašuje nepravost, s hory Efraim.
\par 16 Pripomínejte temto národum, aj, ohlašujte Jeruzalémským, že strážní táhnou z zeme daleké, a vydávají proti mestum Judským hlas svuj.
\par 17 Jako ti, kteríž hlídají polí, položí se proti nemu vukol; nebo jest mi odporný, dí Hospodin.
\par 18 Cesta tvá a skutkové tvoji to zpusobili tobe; tot nešlechetnost tvá, žet to horké jest, a že dosahá až do srdce tvého.
\par 19 Ó, streva má, streva má, bolest trpím, ó osrdí mé, kormoutí se ve mne srdce mé, nemohut mlceti. Nebo hlas trouby slyšíš, duše má, a prokrikování vojenské.
\par 20 Potrení za potrením provolává se, poplénena zajisté bude všecka zeme, náhle popléneni budou stanové moji, v okamžení kortýny mé.
\par 21 Až dokud vídati budu korouhev, slýchati hlas trouby?
\par 22 Nebo bláznivý lid muj nezná mne, synové nemoudrí a nerozumní jsou. Moudrí jsou k cinení zlého, ale ciniti dobre neumejí.
\par 23 Hledím-li na zemi, a aj, neslicná jest a prázdná; pakli na nebe, není na nem žádného svetla.
\par 24 Hledím-li na hory, a aj, tresou se, a všickni pahrbkové pohybují se.
\par 25 Hledím-li, a aj, není žádného cloveka, a všeliké ptactvo nebeské zaletelo.
\par 26 Hledím-li, a aj, pole úrodné jest pouští, a všecka mesta jeho zborena jsou od Hospodina a od hnevu prchlivosti jeho.
\par 27 Nebo takto praví Hospodin: Spustne všecka zeme, a však konce ješte neuciním.
\par 28 Kvíliti bude nad tím zeme, a zasmuší se na hore nebe, proto že jsem mluvil, co jsem myslil, a nelituji, aniž se odvrátím od toho.
\par 29 Pred hrmotem jezdcu a tech, kteríž strílejí z lucište, utece všecko mesto. Vejdou do hustých oblaku, a na skálí vylezou; všecka mesta opuštena budou, a žádný nebude bydliti v nich.
\par 30 Ty pak pohubena jsuc, což ciniti budeš? Ackoli oblácíš se v šarlat, ackoli se ozdobuješ ozdobou zlatou, ackoli lícíš tvár svou lícidlem, darmo se okrašluješ. Pohrdají tebou frejíri, bezživotí tvého hledají.
\par 31 Nebo slyším hlas jako rodicky, svírání jako té, kteráž po nejprvé ku porodu pracuje, hlas dcery Sionské, ustavicne vzdychající, a lomící rukama svýma, ríkající: Beda mne nyní, nebo ustala duše má pro vrahy.

\chapter{5}

\par 1 Zbehejte ulice Jeruzalémské, pohledte nyní, a zvezte, a hledejte v ulicích jeho, naleznete-li muže, jest-li kdo, ješto by cinil soud, a vyhledával toho, což pravého jest, a odpustím jemu.
\par 2 Ale i když ríkají: Živt jest Hospodin, takovým zpusobem krive prisahají.
\par 3 Ó Hospodine, zdaliž oci tvé nepatrí na pravdu? Biješ je, ale necítí bolesti; hubíš je, ale nechtí prijímati kázne. Tvrdší jsou tváre jejich než skála, nechtí se navrátiti.
\par 4 I rekl jsem já: Snad tito prostí jsou, nerozumne sobe pocínají; nebo nejsou povedomi cesty Hospodinovy, soudu Boha svého.
\par 5 Pujdu aspon k prednejším, a mluviti budu k nim; nebo oni povedomi jsou cesty Hospodinovy, soudu Boha svého. Ale i ti spolu polámali jho, roztrhali svazky.
\par 6 Protož je podáví lev z lesa, vlk vecerní pohubí je, pardus cíhati bude u mest jejich. Kdožkoli vyjde z nich, roztrhán bude; nebo mnohá jsou prestoupení jejich, rozmohly se prevrácenosti jejich.
\par 7 Kdež jest to, procež bych mel odpustiti tobe? Synové tvoji opouštejí mne, a prisahají skrze ty, kteríž nejsou bohové. Jakž jsem jen nasytil je, hned cizoloží, a do domu nevestky houfem se valí.
\par 8 Ráno vstávajíce, jsou jako koni vytylí; každý k žene bližního svého rehce.
\par 9 Zdaliž pro takové veci nemám navštíviti? dí Hospodin. A zdali nad národem takovým nemá mstíti duše má?
\par 10 Vstupte na zdi jeho, a zkazte je, a neprestávejte; svrzte štítky zdí jeho, nebo nejsou Hospodinovy.
\par 11 Velice zajisté zproneverili se mi dum Izraelský a dum Judský, dí Hospodin.
\par 12 Scítali klam na Hospodina, a ríkali: Není tak, nikoli neprijde na nás zlé, a mece ani hladu nepocítíme.
\par 13 Ti pak proroci pominou s vetrem, a žádného slova není v nich. Takt se stane jim.
\par 14 Protož takto praví Hospodin Buh zástupu: Proto že tak mluvíte, aj, já zpusobím, aby slova tvá v ústech tvých byla jako ohen, a lid tento drívím, kteréž on zžíre.
\par 15 Aj, já privedu na vás národ zdaleka, ó dome Izraelský, dí Hospodin, národ silný, národ starodávní, národ, jehož jazyka nebudeš umeti, ani rozumeti, co mluví.
\par 16 Jehož toul jako hrob otevrený, všickni jsou silní.
\par 17 A vytráví obilé tvé a chléb tvuj, požerou syny tvé a dcery tvé, pojí bravy tvé i skot tvuj, pojí vinné kmeny tvé i fíkoví tvé, mesta tvá hrazená, v nichž ty doufáš, znuzí mecem.
\par 18 A však ani tech casu, dí Hospodin, neuciním s vámi konce.
\par 19 Nebo stane se, když reknete: Proc nám ciní Hospodin Buh náš všecko toto? že rekneš jim: Jakož jste opustili mne, a sloužili bohum cizozemcu v zemi své, tak sloužiti budete cizozemcum v zemi ne své.
\par 20 Oznamtež to v dome Jákobove, a rozhlaste v Judstvu, rkouce:
\par 21 Slyštež nyní toto, lide bláznivý a nesmyslný, kteríž oci mají a nevidí, kteríž uši mají a neslyší:
\par 22 Což se mne nebudete báti? dí Hospodin. Což pred oblícejem mým nebudete se trásti? Kterýž jsem položil písek za cíl mori ustanovením vecným, jehož neprekracuje. Ackoli zmítají se, však neodolají, ackoli zvucí vlnobití jeho, však ho neprecházejí.
\par 23 Ale lid tento má srdce zarputilé a zpurné, odstoupili a odešli.
\par 24 Ani nerekli v srdci svém: Bojme se již Hospodina Boha našeho, kterýž dává déšt jarní i podzimní casem svým, téhodnu narízených ke žni ostríhá nám.
\par 25 Nepravostit vaše prekážku ciní tem vecem, a hríchové vaši pripravují vás o to dobré.
\par 26 Nebo nalézají se v lidu mém bezbožníci; streže jako cižebníci, kteríž lécejí, stavejí osídla, lidi lapají.
\par 27 Jako klece plná ptáku, tak domové jejich plní jsou lsti. Protož zrostli a zbohatli,
\par 28 Vytylí jsou, lsknou se, nadto umejí se vyhýbati bídám. Pre nesoudí, ani pre sirotka, a však štastne se jim vede, ackoli k spravedlnosti chudým nedopomáhají.
\par 29 Zdaliž pro takové veci nemám navštíviti jich? dí Hospodin. Zdali nad národem takovým nemá mstíti duše má?
\par 30 Vec užasnutí hodná a hrozná deje se v zemi této.
\par 31 Proroci prorokují lžive, a kneží panují skrze ne, a lid muj miluje to. Ceho byste pak neucinili naposledy?

\chapter{6}

\par 1 Shromaždte se, synové Beniaminovi, z prostredku Jeruzaléma, a v Tekoa trubte trubou, a nad Betkarem vyzdvihnete korouhev; nebo videti zlé od pulnoci, a potrení veliké.
\par 2 Panne krásné a rozkošné pripodobnil jsem byl dceru Sionskou.
\par 3 Ale pritáhnou k ní pastýri s stády svými, rozbijí proti ní stany vukol, spase každý místo své.
\par 4 Vyzdvihnete proti ní válku, vstante a pritrhneme o poledni. Beda nám, že pomíjí den, že se roztáhli stínové vecerní.
\par 5 Vstante a pritrhneme v noci, a zkazme paláce její.
\par 6 Takto zajisté praví Hospodin zástupu: Nasekejte dríví, a zdelejte proti Jeruzalému náspy. Tot jest to mesto, kteréž navštíveno býti musí; což ho koli, jen nátisk jest u prostred neho.
\par 7 Jakož studnice vypryštuje vodu svou, tak ono vypryštuje zlost svou. Nátisk a zhoubu slyšeti v nem pred oblícejem mým ustavicne, bolest i bití.
\par 8 Usmysl sobe, ó Jeruzaléme, aby se neodloucila duše má od tebe, abych te neobrátil v pustinu, v zemi nebydlitelnou.
\par 9 Takto praví Hospodin zástupu: Jiste paberovati budou jako vinný kmen ostatek Izraele, ríkajíce: Sahej rukou svou jako ten,kterýž víno zbírá do putny.
\par 10 Komuž mluviti budu, a kým osvedcovati, aby slyšeli? Aj, neobrezané jsou uši jejich, tak že nemohou pozorovati; aj, slovo Hospodinovo mají v posmechu, a nemají líbosti v nem.
\par 11 Protož plný jsem prchlivosti Hospodinovy, ustal jsem, drže ji v sobe. Vylita bude i na malické vne, spolu i na shromáždení mládencu, ovšem pak muž s ženou jat bude, starec s kmetem.
\par 12 A dostanou se domové jejich jiným, též pole i ženy, když vztáhnu ruku svou na obyvatele této zeme, dí Hospodin.
\par 13 Od nejmenšího zajisté z nich, až do nejvetšího z nich, všickni naporád vydali se v lakomství, anobrž od proroka až do kneze všickni naporád provodí faleš.
\par 14 A hojí potrení dcery lidu mého povrchu, ríkajíce: Pokoj, pokoj, ješto není žádného pokoje.
\par 15 Stydeli-liž se pak co proto, že ohavnost páchali? Aniž se lid co stydel, aniž jich proroci k zahanbení privesti umeli. Protož padnou mezi padajícími; v cas, v nemž je navštívím, klesnou, praví Hospodin.
\par 16 Když takto ríkával Hospodin: Zastavte se na cestách, a pohledte, a vyptejte se na stezky staré, která jest cesta dobrá, i chodte po ní, a naleznete odpocinutí duši své, tedy ríkávali: Nebudeme choditi.
\par 17 Když jsem pak ustanovil nad vámi strážné, rka: Mejtež pozor na zvuk trouby, tedy ríkávali: Nebudeme pozorovati.
\par 18 Protož slyšte, ó národové, a poznej, ó shromáždení, co se deje mezi nimi.
\par 19 Slyš, ó zeme: Aj, já uvedu zlé na lid tento, ovoce myšlení jejich, proto že nepozorují slov mých, ani zákona mého, ale jím pohrdají.
\par 20 K cemuž mi kadidlo z Sáby prichází, a vonná trtina výborná z zeme daleké? Zápalu vašich nelibuji sobe, aniž obeti vaše jsou mi príjemné.
\par 21 Protož takto praví Hospodin: Aj, já nakladu lidu tomuto úrazu, a zurážejí se o ne otcové, tolikéž i synové, soused i bližní jeho, a zahynou.
\par 22 Takto praví Hospodin: Aj, lid pritáhne z zeme pulnocní, a národ veliký povstane od koncin zeme.
\par 23 Lucište i kopí pochytí, každý ukrutný bude, a neslitují se. Hlas jejich jako more zvuceti bude, a na koních jezditi budou, zšikovaní jako muž k boji, proti tobe, ó dcero Sionská.
\par 24 Jakž uslyšíme povest o nem, opadnou ruce naše; ssoužení zachvátí nás, a bolest jako rodicku.
\par 25 Nevycházejte na pole, a na cestu nechodte; nebo mec neprítele a strach jest vukol.
\par 26 Ó dcero lidu mého, prepaš se žíní, a válej se v popele. Vydej se v kvílení, jako po synu jednorozeném, v kvílení prehorké; nebo náhle pritáhne zhoubce na nás.
\par 27 Dal jsem te za veži v lidu tvém, a za baštu, abys spatroval a zkušoval cesty jejich.
\par 28 Všickni jsou z zarputilých nejzarputilejší, chodí jako utrhac, jsou ocel a železo, všickni naporád zhoubcové jsou.
\par 29 Prahnou mechy, od ohne mizí olovo, nadarmo ustavicne prepaluje zlatník; nebo zlé veci nemohou býti oddeleny.
\par 30 Stríbrem falešným nazovou je, nebo Hospodin zavrhl je.

\chapter{7}

\par 1 Slovo, kteréž se stalo k Jeremiášovi od Hospodina, rkoucí:
\par 2 Postav se v brane domu Hospodinova, a ohlašuj tam slovo toto, a rci: Slyšte slovo Hospodinovo všickni Judští, kteríž vcházíte do bran techto, abyste se klaneli Hospodinu.
\par 3 Takto praví Hospodin zástupu, Buh Izraelský: Polepšte cest svých i predsevzetí svých, a zpusobím to, abyste bydlili na míste tomto.
\par 4 Neskládejte nadeje své v slovích lživých, ríkajíce: Chrám Hospodinuv, chrám Hospodinuv, chrám Hospodinuv jest.
\par 5 Ale jestliže všelijak polepšíte cest svých, a predsevzetí svých, jestliže spravedlive soud konati budete mezi mužem a mezi bližním jeho;
\par 6 Prichozího, sirotka a vdovy neutisknete, a krve nevinné nevylejete na míste tomto, a za bohy cizími nebudete-li choditi k svému zlému:
\par 7 Tedy zpusobím, abyste bydlili na míste tomto, v zemi, kterouž jsem dal otcum vašim, od veku až na veky.
\par 8 Aj, vy skládáte nadeji svou v slovích lživých, kteráž neprospívají.
\par 9 Zdaliž kradouce, mordujíce a cizoložíce i krive prisahajíce a kadíce Bálovi; též chodíce za bohy cizími, jichž neznáte,
\par 10 Predce choditi a postavovati se budete pred oblícejem mým v dome tomto, kterýž nazván jest od jména mého, a ríkati: Vyprošteni jsme, abyste páchali ty všecky ohavnosti?
\par 11 Což peleší lotrovskou jest dum tento pred ocima vašima, kterýž nazván jest od jména mého? Aj, takét já vidím, dí Hospodin.
\par 12 Ale jdete aspon na místo mé, kteréž bylo v Sílo, kdež jsem byl zpusobil príbytek jménu svému z pocátku, a vizte, co jsem ucinil jemu pro nešlechetnost lidu svého Izraelského.
\par 13 Protož nyní, že ciníte všecky skutky tyto, dí Hospodin, a když mluvím k vám, ráno vstávaje, a to ustavicne, tedy neposloucháte, a když volám na vás, tedy neozýváte se:
\par 14 Protož uciním domu tomuto, kterýž nazván jest od jména mého, v nemž vy doufáte, i místu tomuto, kteréž jsem dal vám a otcum vašim, jako jsem ucinil Sílo;
\par 15 A zavrhu vás od tvári své, jako jsem zavrhl bratrí vaše, všecko síme Efraimovo.
\par 16 Ty tedy nemodl se za lid tento, aniž pozdvihuj za ne hlasu a modlitby, aniž se primlouvej ke mne; nebo te nikoli nevyslyším.
\par 17 Zdaliž sám nevidíš, co oni ciní v mestech Judských a po ulicích Jeruzalémských?
\par 18 Synové zbírají dríví, a otcové zanecují ohen, ženy pak zadelávají testo, aby pekly koláce tvoru nebeskému, a obetovali obeti mokré bohum cizím, aby mne popouzeli.
\par 19 Zdaliž to proti mne jest, že mne oni popouzejí? dí Hospodin. Zdali není proti nim k zahanbení tvári jejich?
\par 20 Procež takto praví Panovník Hospodin: Aj, hnev muj a prchlivost má vylita bude na místo toto, na lidi i na hovada, i na dríví polní, i na úrody zeme, a horeti bude, tak že neuhasne.
\par 21 Takto praví Hospodin zástupu, Buh Izraelský: Zápaly své pridejte k obetem svým, a jezte maso.
\par 22 Nebo jsem nemluvil s otci vašimi, aniž jsem prikázal jim v ten den, v kterýž jsem je vyvedl z zeme Egyptské, o zápalích a obetech.
\par 23 Ale toto prikázal jsem jim, rka: Poslouchejte hlasu mého, a budu vaším Bohem, a vy budete mým lidem, a chodte po vší ceste, kterouž jsem vám prikázal, aby vám dobre bylo.
\par 24 Však neposlechli, aniž naklonili ucha svého, ale chodili po radách, a podlé zdání srdce svého zlého. Obrátili se ke mne hrbetem a ne tvárí svou.
\par 25 Od toho casu, jakž vyšli otcové vaši z zeme Egyptské, až do tohoto dne posílal jsem k vám všecky služebníky své proroky, každý den ráno vstávaje, a to ustavicne.
\par 26 Však neposlechli mne, aniž naklonili ucha svého, ale zatvrdivše šíji svou, hure cinili nežli otcové jejich.
\par 27 Když jim mluvíš všecka slova tato, ani tebe neposlouchají, a když voláš na ne, neohlašují se tobe.
\par 28 Protož rciž jim: Tento jest národ, kteríž neposlouchají hlasu Hospodina Boha svého, aniž prijímají naucení. Zhynula pravda, a vymizela z úst jejich.
\par 29 Ohol vlasy své a zavrz, a naríkej hlasem na místech vysokých; nebo zavrhl Hospodin a opustil rodinu, na kterouž se velmi hnevá.
\par 30 Cinili zajisté synové Judovi, což zlého jest pred ocima mýma, dí Hospodin. Nastaveli ohavností svých v dome tom, kterýž nazván jest od jména mého, aby poškvrnili jej.
\par 31 Nadto vzdelali výsosti Tofet, kteréž jest v údolí syna Hinnom, aby pálili syny své i dcery své ohnem, cehož jsem neprikázal, aniž vstoupilo na srdce mé.
\par 32 Protož aj, dnové jdou, dí Hospodin, kdyžto nebude slouti více Tofet, ani údolí syna Hinnom, ale údolí mordu, a pochovávati budou v Tofet, nebo nebude dostávati místa.
\par 33 I budou mrtvá tela lidu tohoto za pokrm ptactvu nebeskému a šelmám zemským, a nebude žádného, kdo by odstrašil.
\par 34 Zpusobím také, aby prestal v mestech Judských a v ulicích Jeruzalémských hlas radosti a hlas veselé, hlas ženicha a hlas nevesty; nebo pustinou ucinena bude zeme.

\chapter{8}

\par 1 V ten cas, dí Hospodin, vyberou kosti králu Judských, i kosti knížat jejich, i kosti kneží, i kosti proroku, ano i kosti obyvatelu Jeruzalémských z hrobu jejich,
\par 2 A rozmecí je proti slunci a mesíci, i proti všemu vojsku nebeskému, kteréž milují, a kterýmž slouží, a za kterýmiž chodí, a kterýchž hledají, a kterýmž se klanejí. Nebudou sebrány, ani pochovány, budou místo hnoje na svrchku zeme.
\par 3 I bude žádostivejší smrt nežli život všechnem ostatkum tem, kteríž pozustanou z rodiny této nešlechetné na všech místech, kdež by koli pozustali, tam kamž je zaženu, dí Hospodin zástupu.
\par 4 Protož díš jim: Takto praví Hospodin: Tak-liž padli, aby nemohli povstati? Tak-liž se odvrátil, aby se nemohl zase navrátiti?
\par 5 Proc se odvrátil lid tento Jeruzalemský odvrácením vecným? Chytají se lsti, a nechtí se navrátiti.
\par 6 Pozoroval jsem a poslouchal. Nemluví, což pravého jest; není, kdo by želel zlosti své, ríkaje: Což jsem ucinil? Každý obrácen jest k behu svému jako kun, kterýž prudce beží k boji.
\par 7 Ješto cáp u povetrí zná narízené casy své, a hrdlicka i reráb i vlaštovice šetrí casu príletu svého, lid pak muj nezná soudu Hospodinova.
\par 8 Jakž mužete ríci: Moudrí jsme a zákon Hospodinuv máme? Aj, jiste nadarmo delá písar péro, nadarmo jsou v zákone zbehlí.
\par 9 Kohož zahanbili ti moudrí? Kdo jsou predešeni a jati? Aj, slovem Hospodinovým pohrdají, jakáž tedy jest moudrost jejich?
\par 10 A protož dám ženy jejich jiným, pole jejich tem, kteríž by je opanovali; nebo od nejmenšího až do nejvetšího všickni naporád vydali se v lakomství, od proroka až do kneze všickni naporád provodí faleš.
\par 11 Nebo hojí potrení dcery lidu mého po vrchu, ríkajíce: Pokoj, pokoj, ješto není žádného pokoje.
\par 12 Jsou-liž zahanbováni, proto že ohavnost páchali? Ba ani se co ustydli, ani zahanbiti umeli. Protož padnou, když oni padati budou; v cas, v nemž je navštívím, klesnou, praví Hospodin.
\par 13 Do konce vykorením je, dí Hospodin. Nebude žádného hroznu na vinném kmenu, ani žádných fíku na fíku, ano i list sprchne, a což dám jim, odjato bude.
\par 14 Proc my tu sedíme? Shromaždte se, a vejdeme do mest hrazených, a odpocineme tam. Ale Hospodin Buh náš káže odpocívati nám, když vodou jedovatou napojí nás, proto že jsme hrešili proti Hospodinu.
\par 15 Cekej pokoje, ale nic dobrého, casu uzdravení, ale aj, hruza.
\par 16 Od Dan slyšeti frkání koní jeho, od hlasu prokrikování silných jeho všecka zeme se trese, kteríž táhnou, aby zžírali zemi i všecko, což jest na ní, mesto i ty, kteríž bydlejí v nem.
\par 17 Nebo aj, já pošli na vás hady nejjedovatejší, proti nimž nic neprospívá zaklínání, a štípati vás budou, dí Hospodin.
\par 18 Srdce mé ve mne, kteréž by mne melo obcerstvovati v zármutku, mdlé jest.
\par 19 Aj, hlas kriku dcery lidu mého z zeme velmi daleké: Zdali Hospodina není na Sionu? Zdali krále jeho není na nem? Proc mne popouzeli rytinami svými, marnostmi cizozemcu?
\par 20 Pominula žen, dokonalo se léto, a my nejsme vyprošteni.
\par 21 Pro potrení dcery lidu mého potrín jsem, smutek nesu, užasnutí podjalo mne.
\par 22 Což není žádného lékarství v Galád? Což není žádného lékare tam? Proc tedy není zhojena dcera lidu mého?

\chapter{9}

\par 1 Ó kdo mi to dá, aby hlava má byla vodou, a oci mé pramenem slzí, abych dnem i nocí oplakati mohl zmordovaných dcerky lidu svého.
\par 2 Ó kdo mne postaví na poušti v hospode pocestných, abych opustil lid svuj, a odešel od nich; nebo všickni jsou cizoložníci, zber zproneverilých,
\par 3 A natahují jazyka svého ke lži jako lucište své. Zmocnili se na zemi, ale ne k pravde; nebo ze zlého ve zlé jdou, a mne neznají, praví Hospodin.
\par 4 Každý strez se bližního svého, a ne každému bratru se doveruj; nebo každý bratr hledí všelijak podtrhnouti, a každý bližní jako utrhac chodí.
\par 5 A jeden každý bližního svého oklamává, a pravdy nemluví; ucí jazyk svuj mluviti lež, nepráve ciníce, ustávají.
\par 6 Tvuj byt jest u prostred lidu prelstivého; pro lest nechtejí mne poznati, dí Hospodin.
\par 7 Z té príciny takto praví Hospodin zástupu: Aj, já preháneje, pruboval jsem je. Jakž tedy již naložiti mám s dcerou lidu svého?
\par 8 Strela zabíjející jest jazyk jejich, lest mluví. Ústy svými pokojne s bližním svým mluví, ale v srdci svém skládá úklady své.
\par 9 Zdali pro takové veci nemám jich navštíviti? dí Hospodin. Zdaliž nad národem takovým nemá mstíti duše má?
\par 10 Pro tyto hory dám se v plác a v naríkání, a pro pastvište, kteráž jsou na poušti, v kvílení; nebo popálena budou, tak že nebude žádného, kdo by skrze ne šel, aniž bude slyšán hlas dobytka. Od ptactva nebeského až do hovada všecko se odbére a odejde.
\par 11 A obrátím Jeruzalém v hromady, príbytek draku, a mesta Judská obrátím v pustinu, tak že nebude obyvatele.
\par 12 Kdo jest ten muž moudrý, ješto by rozumel tomu? A k komu mluvila ústa Hospodinova, ješto by oznamoval to, proc zahynouti má tato zeme, a vypálena býti jako poušt, tak aby nebylo, kdo by skrze ni šel?
\par 13 Nebo praví Hospodin: Proto že opustili zákon muj, kterýž jsem jim predložil, a neposlouchali hlasu mého, aniž chodili za ním,
\par 14 Ale chodili za myšlénkami srdce svého a za Báli, cemuž je naucili otcové jejich:
\par 15 Protož takto praví Hospodin zástupu, Buh Izraelský: Aj, já nakrmím je, totiž lid tento, pelynkem, a napojím je vodou jedovatou.
\par 16 Nebo rozptýlím je mezi národy,kterýchž neznali oni, ani otcové jejich, a posílati budu za nimi mec, až je do konce vyhladím.
\par 17 Takto praví Hospodin zástupu: Pilne považte, a svolejte ty, kteréž naríkávají, at prijdou, a k tem, kteréž jsou vycvicené, pošlete, aby prišly.
\par 18 Necht pospíší, a dadí se nad námi v naríkání, aby slzy tekly z ocí našich, a vícka naše oplývala vodou.
\par 19 Hlas zajisté naríkání slyšeti z Siona: Jak jsme pohubeni! Stydíme se náramne, že jsme ztratili zemi, že borí príbytky naše.
\par 20 Anobrž slyšte, ženy, slovo Hospodinovo, a necht prijme ucho vaše slovo úst jeho, abyste ucily dcerky své naríkání, a jedna každá tovaryšku svou kvílení.
\par 21 Nebo vlezla smrt okny našimi, vešla na paláce naše, aby vyhubila deti z rynku a mládence z ulic.
\par 22 (Mluv i to: Takto dí Hospodin:) A padla mrtvá tela lidská jako hnuj po poli, a jako snopové za žencem, a není žádného, kdo by pochoval.
\par 23 Takto praví Hospodin: Nechlub se moudrý v moudrosti své, ani se chlub silný v síle své, aniž se chlub bohatý v bohatství svém.
\par 24 Ale v tom necht se chlubí, kdo se chlubí, že rozumí a zná mne, že já jsem Hospodin, kterýž ciním milosrdenství, soud i spravedlnost na zemi; nebo v tech vecech líbost mám, dí Hospodin.
\par 25 Aj, dnové jdou, praví Hospodin, v nichž navštívím každého, obrezaného i neobrezaného,
\par 26 Egyptské i Judské, a Idumejské i Ammonitské, a Moábské, i všecky, kteríž v nejzadnejším koute bydlí na poušti. Nebo ti všickni národové jsou neobrezaní, tolikéž všecken dum Izraelský jest neobrezaného srdce.

\chapter{10}

\par 1 Slyšte slovo toto, kteréž mluví k vám Hospodin, ó dome Izraelský.
\par 2 Takto praví Hospodin: Ceste pohanu neucte se, aniž se znamení nebeských deste, nebot se jich desí pohané.
\par 3 Ustanovení zajisté tech národu jsou pouhá marnost. Nebo setna drevo sekerou v lese, dílo rukou remeslníka,
\par 4 Stríbrem a zlatem ozdobí je, hrebíky a kladivy utvrzují je, aby se neviklalo.
\par 5 Jsou jako palmový špalek tvrdý, ani nemluví; nošeni býti musejí, nebo choditi nemohou. Nebojtež se jich, nebo zle uciniti nemohou, aniž také dobre uciniti mohou.
\par 6 Z nichž není žádného tobe podobného, ó Hospodine; veliký jsi, i jméno tvé veliké jest v moci.
\par 7 Kdož by se nebál tebe, králi národu? Na tebet zajisté to sluší, ponevadž mezi všemi mudrci národu, i ve všem království jejich nikdá nebylo podobného tobe.
\par 8 A však ze spolka zhlupeli, a blázni jsou; z dreva uciti se jest pouhá marnost.
\par 9 Stríbro tažené z zámorí privážíno bývá, a zlato z Ufaz, dílo remeslníka a rukou zlatníka; z postavce modrého a šarlatový jest odev jejich, všecko to jest dílo umelých.
\par 10 Ale Hospodin jest Buh pravý, jest Buh živý a král vecný, pred jehož prchlivostí zeme se trese, aniž mohou snésti národové rozhnevání jeho.
\par 11 (Takto ríkejte jim: Bohové ti, kteríž nebe ani zeme neucinili, necht zahynou z zeme, a necht jich není pod nebem.)
\par 12 Kterýž ucinil zemi mocí svou, kterýž utvrdil okršlek sveta moudrostí svou, a opatrností svou roztáhl nebesa.
\par 13 Kterýžto když vydává hlas, jecení vod bývá na nebi, a kterýž pusobí to, aby vystupovaly páry od kraje zeme, blýskání s deštem privodí, a vyvodí vítr z pokladu svých.
\par 14 Tak zhlupel každý clovek, že nezná toho, že zahanben bývá každý zlatník pro rytinu; nebo slitina jeho jest faleš, a není ducha v nich.
\par 15 Marnost jsou a dílo podvodu; v cas, v nemž je navštívím, zahynou.
\par 16 Nenít podobný temto díl Jákobuv, nebo on jest stvoritel všeho; Izrael tolikéž jest pokolení dedictví toho, jehož jméno jest Hospodin zástupu.
\par 17 Sber z zeme koupi svou, ty kteráž bydlíš v pevnosti této.
\par 18 Nebo takto praví Hospodin: Aj, já vyhodím z praku obyvatele zeme této pojednou, a ssoužím je, aby shledali toto:
\par 19 Beda mne pro setrení mé, prebolestná jest rána má, ješto jsem já byl rekl: Jiste tuto nemoc budu moci snésti.
\par 20 Stan muj poplénen jest, a všickni provazové moji potrháni jsou. Synové moji odebrali se ode mne, a není žádného; není žádného, kdo by více rozbíjel stan muj, a roztáhl kortýny mé.
\par 21 Nebo zhlupeli pastýri, a Hospodina se nedotazovali; protož nevede se jim štastne, a všecko stádo pastvy jejich rozptýleno jest.
\par 22 Aj, povest jistá prichází, a pohnutí veliké z zeme pulnocní, aby obrácena byla mesta Judská v pustinu a v príbytek draku.
\par 23 Vím, Hospodine, že není v moci cloveka cesta jeho, aniž jest v moci muže toho, kterýž chodí, aby spravoval krok svuj.
\par 24 Kárej mne, Hospodine, však milostive, ne v hneve svém, abys nesetrel mne.
\par 25 Vylí hnev svuj na ty národy, kteríž tebe neznají, a na rodiny, kteréž jména tvého nevzývají; nebo zžírají Jákoba, a tak zžírají jej, aby jej všeho sežrali, a obydlí jeho v poustku obrátili.

\chapter{11}

\par 1 Slovo, kteréž se stalo k Jeremiášovi od Hospodina, rkoucí:
\par 2 Slyšte slova smlouvy této, kteráž byste mluvili mužum Judským, a obyvatelum Jeruzalémským,
\par 3 A rci jim: Takto praví Hospodin Buh Izraelský: Zlorecený ten clovek, kterýž by neposlechl slov smlouvy této,
\par 4 Kterouž jsem vydal otcum vašim tehdáž, když jsem je vyvedl z zeme Egyptské, z peci železné, rka: Poslouchejte hlasu mého, a cinte to všecko, tak jakž prikazuji vám, i budete lidem mým, a já budu Bohem vaším,
\par 5 Abych splnil prísahu, kterouž jsem ucinil otcum vašim, že jim dám zemi oplývající mlékem a strdí, jakž dnešní den jest. Jemuž odpovedev, rekl jsem: Amen, Hospodine.
\par 6 Potom rekl mi Hospodin: Ohlašuj všecka slova tato po mestech Judských, a po ulicích Jeruzalémských, rka: Slyšte slova smlouvy této, a cinte je.
\par 7 Nebo castokrát osvedcoval jsem se otcum vašim, od toho dne, jakž jsem je vyvedl z zeme Egyptské, až do dne tohoto; ráno privstávaje a osvedcuje se, ríkával jsem: Poslouchejte hlasu mého.
\par 8 Ale neposlouchali, aniž naklonili ucha svého, nýbrž chodil jeden každý po zdání srdce svého zlého. Procež uvedl jsem na ne všecka slova smlouvy této, kterouž jsem prikázal plniti, ale neplnili.
\par 9 Tehdy rekl mi Hospodin: Nalézá se spiknutí mezi muži Judskými, a mezi obyvateli Jeruzalémskými.
\par 10 Obrátili se k nepravostem otcu svých starých, kteríž nechteli poslouchati slov mých. Tolikéž tito chodí za bohy cizími, sloužíce jim; dum Izraelský a dum Judský zrušili smlouvu mou, kterouž jsem ucinil s otci jejich.
\par 11 Protož takto praví Hospodin: Aj, já uvedu na ne zlé, z nehož nebudou moci vyjíti. By pak volali ke mne, nevyslyším jich.
\par 12 I pujdou mesta Judská i obyvatelé Jeruzalémští, a budou volati k bohum tem, kterýmž kadí, ale nikoli nevysvobodí jich v cas bídy jejich,
\par 13 Ackoli podlé poctu mest máš bohy své, ó Judo, a podlé poctu ulic Jeruzalémských nastaveli jste oltáru ohavnosti té, oltáru, na nichž byste kadili Bálovi.
\par 14 Protož ty nemodl se za lid tento, aniž pozdvihuj za ne hlasu a modlitby; nebot nikoli nevyslyším jich v ten cas, když by volali ke mne prícinou svého zlého.
\par 15 Co jest milému mému do mého domu, ponevadž nestydate páše nešlechetnosti s mnohými, a obeti svaté odešly od tebe, a že v zlosti své pléšeš?
\par 16 Bylte Hospodin nazval jméno tvé olivou zelenající se, peknou pro ovoce ušlechtilé, ale s zvukem boure veliké zapálí ji s hury, když polámí ratolesti její.
\par 17 Nebo Hospodin zástupu, kterýž te byl štípil, vyrkl zlé proti tobe, pro nešlechetnost domu Izraelského a domu Judského, kterouž mezi sebou páchali, aby mne popouzeli, kadíce Bálovi.
\par 18 Hospodin zajisté oznámil mi, i dovedel jsem se. Tehdáž jsi mi ukázal predsevzetí jejich,
\par 19 Když jsem já byl jako beránek a volcek, kterýž veden bývá k zabití. Nebo nevedel jsem, by proti mne rady skládali: Zkazme strom s ovocem jeho, a vyhladme jej z zeme živých, aby jméno jeho nebylo pripomínáno více.
\par 20 Ale ó Hospodine zástupu, soudce spravedlivý, kterýž zkušuješ ledví i srdce, necht se podívám na pomstu tvou nad nimi; nebo jsem tobe zjevil pri svou.
\par 21 Protož takto praví Hospodin o Anatotských, kteríž hledají bezživotí tvého, ríkajíce: Neprorokuj ve jménu Hospodinovu, abys neumrel v ruce naší:
\par 22 Protož takto praví Hospodin zástupu: Aj, já navštívím je. Mládenci zbiti budou mecem, synové jejich i dcery jejich zemrou hladem,
\par 23 A nebudou míti potomku, když uvedu zlé na Anatotské, casu toho, v nemž je navštívím.

\chapter{12}

\par 1 Spravedlivý zustaneš, Hospodine, povedu-li odpor proti tobe, a však o soudech tvých mluviti budu s tebou. Proc se ceste bezbožníku štastne vede? Mají pokoj všickni, kteríž se pyšne zproneverili.
\par 2 Štepuješ je, ano i vkorenují se; rostou, ano i ovoce nesou ti, jejichžto úst blízko jsi, ale daleko od ledví jejich.
\par 3 Ale ty, Hospodine, znáš mne, prohlédáš mne, a zkusils srdce mého, že s tebou jest, onyno pak táhneš jako ovce k zabíjení, a pripravuješ je ke dni zabití.
\par 4 Dokudž by žalostila zeme, a bylina všeho pole svadla pro zlost prebývajících v ní, a hynulo každé hovado i ptactvo? Nebo ríkají: Nevidít Buh skoncení našeho.
\par 5 Ponevadž tebe s pešími bežícího k ustání privodí, kterakž bys tedy stacil pri koních? A ponevadž v zemi pokojné, jíž jsi se doveril, ustáváš, což pak spravíš pri tom zdutém Jordánu?
\par 6 Nebo i bratrí tvoji i dum otce tvého zproneverili se tobe, a ti také povolávají za tebou plnými ústy. Never jim, byt pak mluvili s tebou prátelsky.
\par 7 Opustil jsem dum svuj, zavrhl jsem dedictví své, dal jsem to, což velice milovala duše má, v ruce neprátel jeho.
\par 8 Ucineno jest mi dedictví mé podobné lvu v lese, vydává proti mne hlas svuj, procež ho nenávidím.
\par 9 A což ptákem dravým jest mi dedictví mé? Což ptactvo vukol bude proti nemu? Jdetež nu, shromaždte se všecka zvírata polní, sejdete se k jídlu.
\par 10 Mnozí pastýri zkazí vinici mou, pošlapají podíl muj, podíl mne velmi milý obrátí v poušt nejhroznejší,
\par 11 Obrátí jej v pustinu. Kvíliti bude, spustošen jsa ode mne; spustne všecka tato zeme, nebo není žádného, kdo by to v srdci skládal.
\par 12 Na všecka místa vysoká po poušti potáhnou zhoubcové, mec zajisté Hospodinuv zžíre od jednoho kraje zeme až do druhého, nebude míti pokoje žádné telo.
\par 13 Nasejí pšenice, ale trní žíti budou; bolestne to ponesou, že užitku nevezmou, a stydeti se budou za úrody své pro prchlivost hnevu Hospodinova.
\par 14 Takto praví Hospodin o všech mých sousedech zlých, jenž se dotýkají dedictví, kteréž jsem uvedl v dedictví lidu svému Izraelskému: Aj, já vypléním je z zeme jejich, když dum Judský vypléním z prostredku jejich.
\par 15 Stane se však, když je vypléním, že se navrátím a smiluji se nad nimi, a privedu zase jednoho každého z nich k dedictví jeho, a jednoho každého do zeme jeho.
\par 16 Stane se také, jestliže by se pilne ucili cestám lidu mého, a prisahali by ve jménu mém, ríkajíce: Živt jest Hospodin, jakž oni ucívali lid muj prisahati skrze Bále, že vzdeláni budou u prostred lidu mého.
\par 17 Jestliže by pak neposlouchali, tedy pléniti budu národ ten ustavicne a hubiti, dí Hospodin.

\chapter{13}

\par 1 Takto rekl Hospodin ke mne: Jdi a zjednej sobe pás lnený, a opaš jím bedra svá, do vody pak nedávej ho.
\par 2 Tedy zjednal jsem ten pás podlé slova Hospodinova, a opásal jsem bedra svá.
\par 3 Potom stalo se slovo Hospodinovo ke mne podruhé, rkoucí:
\par 4 Vezmi ten pás, kterýž jsi zjednal, kterýž jest na bedrách tvých, a vstana, jdi k Eufrates, a skrej jej tam do díry skalní.
\par 5 I šel jsem, a skryl jsem jej u Eufrates, jakž mi byl prikázal Hospodin.
\par 6 Stalo se pak po prebehnutí dnu mnohých, že rekl Hospodin ke mne: Vstana, jdi k Eufrates, a vezmi odtud ten pás, kterýžt jsem prikázal skrýti tam.
\par 7 I šel jsem k Eufrates, a vykopav, vzal jsem ten pás z místa toho, kdež jsem jej byl skryl. A aj, zkažený byl ten pás, aniž se k cemu hodil.
\par 8 Tehdy stalo se slovo Hospodinovo ke mne, rkoucí:
\par 9 Takto praví Hospodin: Takt zkazím pýchu Judských i pýchu Jeruzalémských velikou.
\par 10 Lidu toho prenešlechetného, kteríž nechtí poslouchati slov mých, kteríž chodí podlé zdání srdce svého, a chodí za bohy cizími, sloužíce jim, a klanejíce se jim. I bude podoben pasu tomu, kterýž se nehodí k nicemu.
\par 11 Nebo jakož se drží pás na bedrách muže, tak jsem byl pripojil k sobe všecken dum Izraelský i všecken dum Judský, dí Hospodin, aby byli lidem mým, a to k sláve, a k chvále, i k ozdobe, ale nebyli poslušni.
\par 12 Protož rci jim slovo toto: Takto praví Hospodin Buh Izraelský: Všeliká nádoba vinná naplnována bývá vínem. Kdyžt pak reknou: Zdaliž nevíme dobre, že všeliká nádoba vinná naplnována bývá vínem?
\par 13 I díš jim: Takto praví Hospodin: Aj, já naplním všecky obyvatele zeme této i krále, kteríž sedí místo Davida na stolici jeho, i kneží i proroky, a tolikéž všecky obyvatele Jeruzalémské opilstvím,
\par 14 A rozrazím jednoho o druhého, jakož otce, tak také syny, dí Hospodin. Nebudu šanovati, aniž odpustím, aniž se smiluji, abych zkaziti jich nemel.
\par 15 Poslouchejte a ušima pozorujte, nepovyšujte se, nebot Hospodin mluví.
\par 16 Dejte Hospodinu Bohu svému cest, drív než by tmu uvedl, a dríve nežli by se zurážely nohy vaše o hory tmavé. I cekali byste svetla, ale obrátil by je v stín smrti, promenil by je v mrákotu.
\par 17 Jestliže pak toho neuposlechnete, v skrýších plakati bude duše má pro pýchu vaši, a náramne kvíliti bude. Potekou, pravím, z ocí mých slzy, nebo zajato bude stádce Hospodinovo.
\par 18 Rci králi i královne: Sedte na zemi; nebo odjata bude prednost vaše, koruna ozdoby vaší.
\par 19 Mesta polední uzavírána budou, tak že nebude žádného, kdo by otevríti mohl. Zastehováno bude všecko Judstvo, zastehováno bude docela.
\par 20 Pozdvihnete ocí svých, a vizte ty, kteríž táhnou od pulnoci. Kdež jest to stádo, kteréžt dáno bylo, stádce ozdoby tvé?
\par 21 Co díš, když te navštíví, ještos ty naucila je, aby byli nad tebou vudcové prední? Zdaliž bolesti tebe nezachvátí jako ženu rodící?
\par 22 Díš-li v srdci svém: Proc by mne to potkati melo? Pro množství nepravosti tvé odkryti budou podolkové tvoji, násilne odjata bude obuv tvá.
\par 23 Muže-li zmeniti Mourenín kuži svou, aneb pardus perestost svou, také vy budete moci dobre ciniti, naucivše se zle ciniti.
\par 24 Protož rozptýlím je, jako vítr poušte rozptyluje plevy.
\par 25 Ten bude los tvuj a díl odmerený tobe ode mne, praví Hospodin, proto žes se zapomnela nade mnou, a úfalas v lež.
\par 26 A tak i já také odkryji podolek tvuj nad hlavu tvou, aby spatrína byla hanba tvá,
\par 27 Cizoložství tvá, a rehtání tvá, nešlechetná smilství tvá, na pahrbcích i na poli. Videlt jsem ty ohavnosti tvé; beda tobe, Jeruzaléme. A což se ješte neocistíš? I až dokud pak?

\chapter{14}

\par 1 Slovo, kteréž se stalo k Jeremiášovi o suchu.
\par 2 Kvíliti bude zeme Judská, a brány její zemdlejí, smutek ponesou na zemi, a naríkání Jeruzaléma vzejde.
\par 3 Také nejznamenitejší z nich rozsílati budou i nejšpatnejší své pro vodu. Prijdouce k cisternám, a nenaleznouce vody, navrátí se s nádobami svými prázdnými, hanbíce a stydíce se; protož prikryjí hlavu svou.
\par 4 I oráci stydíce se, prikryjí hlavu svou prícinou zeme vyprahlé, proto že dešte nebude na zemi.
\par 5 Anobrž i lan na poli, což porodí, opustí; nebo mladistvé trávy nebude.
\par 6 A divocí oslové stojíce na vysokých místech, hltati budou vítr jako draci; prehledí se oci jejich, nebo nebude žádné trávy.
\par 7 Ó Hospodine, ponevadž nepravosti naše svedcí proti nám, slituj se pro jméno své. Nebo mnohá jsou odvrácení naše, tobet jsme zhrešili.
\par 8 Ó nadeje Izraelova, vysvoboditeli jeho v cas ssoužení, proc býti máš jako príchozí v této zemi, a jako pocestný stavující se na noclehu?
\par 9 Proc se ukazuješ jako muž ustalý, jako silný, kterýž nemuže vysvoboditi? Všaks ty u prostred nás, Hospodine, a jméno tvé nad námi vzýváno jest; neopouštejž nás.
\par 10 Takto praví Hospodin o lidu tomto: Tak milují toulky, noh svých nezdržují, až Hospodin nemá v nich líbosti, a nyní zpomíná nepravost jejich, a navštevuje hríchy jejich.
\par 11 Potom rekl ke mne Hospodin: Nemodl se za lid tento k dobrému.
\par 12 Když se postiti budou, já nikoli nevyslyším volání jejich, a když obetovati budou obet zápalnou a suchou, já nikoli neoblíbím sobe tech vecí, ale mecem a hladem a morem já do konce zhubím je.
\par 13 Tedy rekl jsem: Ach, Panovníce Hospodine, aj, tito proroci ríkají jim: Neuzríte mece, a hlad neprijde na vás, ale pokoj pravý dám vám na míste tomto.
\par 14 I rekl Hospodin ke mne: Lež prorokují ti proroci ve jménu mém. Neposlalt jsem jich, aniž jsem prikázal jim, anobrž aniž jsem mluvil k nim. Videní lživé a hádání, a marné veci i lest srdce svého oni prorokují vám.
\par 15 Protož takto praví Hospodin o prorocích, kteríž prorokují ve jménu mém, ješto jsem já jich neposlal, a kteríž ríkají: Mece ani hladu nebude v zemi této: Mecem a hladem i ti sami proroci zhynou.
\par 16 Lid pak ten, jemuž oni prorokují, rozmetán bude po ulicích Jeruzalémských hladem a mecem, aniž bude, kdo by je pochovával, je, manželky jejich, a syny jejich, a dcery jejich. Tak vyleji na ne nešlechetnost jejich.
\par 17 Protož rciž jim slovo toto: Z ocí mých tekou slzy dnem i nocí bez prestání; nebo potrína bude velmi velice panna dcera lidu mého ranou náramne bolestnou.
\par 18 Vyjdu-li na pole, aj, tam zbití mecem; pakli vejdu do mesta, aj, tam nemocní hladem. Nebo jakož prorok tak knez obcházejíce, kupcí zemí, a lidé toho neznají.
\par 19 Zdaliž do konce zamítáš Judu? Zdali Sion oškliví sobe duše tvá? Proc nás biješ, tak abychom již nebyli uzdraveni? Cekáme-li pokoje, a aj, nic dobrého pakli casu uzdravení, a aj, hruza.
\par 20 Poznávámet, Hospodine, bezbožnost svou i nepravost otcu svých, že jsme hrešili proti tobe.
\par 21 Nezamítejž pro jméno své, nezlehcuj stolice slávy své; rozpomen se, neruš smlouvy své s námi.
\par 22 Zdaliž jsou mezi marnostmi pohanskými ti, kteríž by déšt dávali? A zdaliž nebesa dávají prívaly? Zdaliž ty nejsi sám, Hospodine, Buh náš? Protož na tebet ocekáváme, nebo ty pusobíš všecko to.

\chapter{15}

\par 1 Tedy rekl Hospodin ke mne: Byt se postavil Mojžíš i Samuel pred oblícejem mým, nikoli srdce nemohu míti k lidu tomuto. Pust je ode mne, a necht jdou pryc.
\par 2 A reknou-lit: Kam bychom šli? tedy díš jim: Takto praví Hospodin: Kdo k smrti, na smrt, a kdo k meci, pod mec, a kdo k hladu, k hladu, a kdo k zajetí, do zajetí.
\par 3 Predstavím zajisté jim to ctvero, dí Hospodin: Mec k zmordování, a psy, aby je rozsmýkali, a ptactvo nebeské i zver zemskou, aby je sežrala a zkazila.
\par 4 A musejí se smýkati po všech královstvích zeme prícinou Manassesa syna Ezechiášova, krále Judského, pro ty veci, kteréž páchal v Jeruzaléme.
\par 5 Nebo kdo by se slitoval nad tebou, Jeruzaléme? A kdo by te politoval? Aneb kdo by prišel, aby se zeptal, jakt se vede?
\par 6 Tys opustil mne, dí Hospodin, odšels nazpet; protož vztáhnu ruku svou na tebe, abych te zkazil. Ustal jsem, želeje.
\par 7 Protož pretríbím je vejeckou skrze brány zeme této, na sirobu privedu a zkazím lid svuj, nebo se od cest svých nenavracují.
\par 8 Vetší bude pocet vdov jeho než písku morského. Privedu na ne, na matky, na mládence zhoubce i v poledne; zpusobím to, aby náhle pripadli na to mesto, i budou zdešeni.
\par 9 Zemdlí i ta, kteráž rodívala po sedmerém; vypustí duši svou, zapadne jí slunce ješte ve dne, hanbiti a stydeti se bude. Ostatek pak jich vydám pod mec pred oblícejem neprátel jejich, dí Hospodin.
\par 10 Beda mne, matko má, že jsi mne porodila, muže sváru a muže ruznice vší zemi. Nepujcoval jsem jim, aniž mi oni pujcovali, a každý mi zlorecí.
\par 11 I rekl Hospodin: Zdaliž tobe, kterýž pozustaneš, nebude dobre? Zdaliž nebudu tvým zástupcím u neprítele v cas trápení a v cas ssoužení?
\par 12 Zdaliž železo poláme pulnocní železo a ocel?
\par 13 Jmení tvé, ó Judo, i poklady tvé vydám v rozchvátání darmo, po všech koncinách tvých, a to pro všelijaké hríchy tvé.
\par 14 A zpusobím to, že musíš jíti s neprátely svými do zeme cizí, když ohen zanícený v prchlivosti mé na vás pálati bude.
\par 15 Ty znáš mne, Hospodine, rozpomen se na mne, a navštev mne, a pomsti mne nad temi, kteríž dotírají na mne. Shovívaje jim, nezachvacuj mne; vez, že snáším pro tebe pohanení.
\par 16 Když se naskytly reci tvé, snedl jsem je, a mel jsem slovo tvé za radost a potešení srdce svého, ponevadž jsi ty mne povolal sám, ó Hospodine Bože zástupu.
\par 17 Nesedám v rade posmevacu, aniž pléši; pro prísnost tvou samotný sedám, nebo prchlivostí naplnils mne.
\par 18 Proc má býti bolest má vecná, a rána má smrtelná, kteráž se nechce zhojiti? Proc mi býti máš naprosto jako oklamavatelný, jako vody nestálé?
\par 19 Protož takto praví Hospodin: Jestliže se obrátíš, také te zase obrátím, abys stál pred oblícejem mým; oddelíš-li vec drahou od nicemné, jako ústa má budeš. Necht se oni obrátí k tobe, ty pak neobracej se k nim.
\par 20 Nebo jsem te postavil proti lidu tomuto jako zed medenou pevnou. Kteríž bojovati budou proti tobe, ale neodolají tobe; nebo já jsem s tebou, abych te vysvobozoval a vytrhoval, dí Hospodin.
\par 21 Vytrhnu te zajisté z rukou nešlechetníku, a vykoupím te z ruky násilníku.

\chapter{16}

\par 1 I stalo se slovo Hospodinovo ke mne, rkoucí:
\par 2 Nepojímej sobe ženy, aniž mej synu neb dcer na míste tomto.
\par 3 Nebo takto praví Hospodin o synech i dcerách zplozených v míste tomto, a o matkách jejich, kteréž zrodily je, i o jejich otcích, kteríž zplodili je v zemi této:
\par 4 Smrtmi prebolestnými pomrou, nebudou oplakáni, ani pochováni; místo hnoje na svrchku zeme budou, a mecem i hladem do konce zhubeni budou. I budou mrtvá tela jejich za pokrm ptactvu nebeskému a šelmám zemským.
\par 5 Nebo takto praví Hospodin: Nevcházej do domu smutku, aniž chod kvíliti, aniž jich lituj; odjal jsem zajisté pokoj svuj od lidu tohoto, dobrotivost i slitování, dí Hospodin.
\par 6 Když pomrou velicí i malí v zemi této, nebudou pochováni, aniž kvíliti budou nad nimi, aniž se zreží, aniž sobe lysiny zdelají pro ne.
\par 7 Aniž jim dadí jísti, aby v truchlosti potešovali jich nad mrtvým, aniž jich napojí z cíše potešení po otci jejich neb matce jejich.
\par 8 Tolikéž do domu hodování nechod, abys sedati mel s nimi pri jídle a pití.
\par 9 Nebo takto praví Hospodin zástupu, Buh Izraelský: Aj, já zpusobím, aby nebývalo na míste tomto pred ocima vašima a za dnu vašich hlasu radosti, ani hlasu veselé, hlasu ženicha, ani hlasu nevesty.
\par 10 Když pak oznámíš lidu tomuto všecka slova tato, a reknou-lit: Proc vyrkl Hospodin proti nám všecko zlé veliké toto, a jaká jest nepravost naše, aneb jaký hrích náš, jímž jsme hrešili proti Hospodinu Bohu svému?
\par 11 Tedy rci jim: Proto že opustili mne otcové vaši, dí Hospodin, a chodíce za bohy cizími, sloužili jim, a klaneli se jim, mne pak opustili, a zákona mého neostríhali.
\par 12 Vy pak mnohem jste hure cinili nežli otcové vaši; nebo aj, vy chodíte jeden každý podlé zdání srdce svého zlého, neposlouchajíce mne.
\par 13 Procež hodím vámi z zeme této do zeme, o níž nevíte vy, ani otcové vaši, a sloužiti budete tam bohum cizím dnem i nocí, dokudž neuciním vám milosti.
\par 14 Protož aj, dnové jdou, dí Hospodin, v nichž nebude ríkáno více: Živt jest Hospodin, kterýž vyvedl syny Izraelské z zeme Egyptské,
\par 15 Ale: Živt jest Hospodin, kterýž vyvedl syny Izraelské z zeme pulnocní a ze všech zemí, do nichž je byl rozehnal, když je zase privedu do zeme jejich, kterouž jsem dal otcum jejich.
\par 16 Aj, já pošli k rybárum mnohým, dí Hospodin, aby je vylovili; potom pošli i k mnohým lovcum, aby je zlapali na všeliké hore, a na všelikém pahrbku, i v derách skalních.
\par 17 Hledím na všecky cesty jejich, nejsout tajné prede mnou, aniž jest skryta nepravost jejich pred ocima mýma.
\par 18 I odplatím jim prvé dvojnásobne za nepravost jejich a hrích jejich, proto že zemi mnou poškvrnili tely mrtvými ohyzdnými, a ohavnostmi svými naplnili dedictví mé.
\par 19 Hospodine, sílo má a hrade muj, i útocište mé v den ssoužení, k tobet prijdou národové od koncin zeme, a reknou: Jiste žet se falše drželi otcové naši, marnosti a práve neužitecných vecí.
\par 20 Zdaliž udelá sobe clovek bohy, ponevadž sami nejsou bohové?
\par 21 Protož aj, já zpusobím to, aby poznali té chvíle, zpusobím, aby poznali ruku mou i moc mou, a zvedít, že jméno mé jest Hospodin.

\chapter{17}

\par 1 Hrích Juduv napsán jest pérem železným, rafijí kamene pretvrdého, vyryt jest na tabuli srdce jejich, a na rozích oltáru vašich,
\par 2 Tak když zpomínají synové jejich na oltáre jejich i háje jejich, pod drívím zeleným, na pahrbcích vysokých.
\par 3 Ó horo, s tím polem jmení tvé i všecky poklady tvé vydám v rozchvátání, pro hrích výsostí tvých, ve všech koncinách tvých.
\par 4 A ty musíš lhutu dáti z strany sebe dedictví svému, kteréž jsem byl dal tobe, a podrobím te v službu neprátelum tvým v zemi, o níž nevíš; nebo jste ohen zanítili v prchlivosti mé, kterýž až na veky horeti bude.
\par 5 Takto praví Hospodin: Zlorecený ten muž, kterýž doufá v cloveka, a kterýž klade telo za ráme své, od Hospodina pak odstupuje srdce jeho.
\par 6 Nebo bude podobný vresu na pustine, kterýž necítí, když co prichází dobrého, ale bývá na vyprahlých místech na poušti v zemi slatinné, a v níž se nebydlí.
\par 7 Požehnaný ten muž, kterýž doufá v Hospodina, a jehož nadeje jest Hospodin.
\par 8 Nebo podobný bude stromu štípenému pri vodách, a pri potoku pouštejícímu koreny své, kterýž necítí, když prichází vedro, ale list jeho bývá zelený, a v rok suchý nestará se, aniž prestává nésti ovoce.
\par 9 Nejlstivejší jest srdce nade všecko, a nejprevrácenejší. Kdo vyrozumí jemu?
\par 10 Já Hospodin, kterýž zpytuji srdce, a zkušuji ledví, tak abych odplatil jednomu každému podlé cesty jeho, podlé ovoce skutku jeho.
\par 11 Koroptva škrecí, ale nevysedí. Tak kdož dobývá statku však s krivdou, v polovici dnu svých musí opustiti jej, a naposledy bude bláznem.
\par 12 Místo svatyne naší, stolice slavná Nejvyššího, vecne trvá.
\par 13 Ó nadeje Izraelova, Hospodine, všickni, kteríž te opouštejí, necht jsou zahanbeni. Kárání má v zemi této necht jsou zapsána; nebo opustili pramen vod živých, Hospodina.
\par 14 Uzdrav mne, Hospodine, a zdráv budu; vysvobod mne, a vysvobozen budu, ty jsi zajisté chvála má.
\par 15 Aj, oni ríkají mi: Kdež jest to, což predpovídal Hospodin? Necht již prijde.
\par 16 Ješto jsem já se nevetrel, abych pastýrem byl tvým, a dne bolesti nebylt jsem žádostiv, ty víš. Cožkoli vyšlo z rtu mých, pred oblícejem tvým jest.
\par 17 Nebudiž mi k strachu, útocište mé jsi v cas trápení.
\par 18 Necht jsou zahanbeni, kteríž mne stihají, já pak at nejsem zahanben; necht se oni desí, já pak at se nedesím. Uved na ne den trápení, a dvojím setrením setri je.
\par 19 Takto rekl Hospodin ke mne: Jdi a postav se v bráne lidu tohoto, skrze kterouž chodívají králové Judští, a skrze kterouž vycházívají, anobrž ve všech branách Jeruzalémských,
\par 20 A rci jim: Slyšte slovo Hospodinovo, králové Judští, i všecken Judo, a všickni obyvatelé Jeruzaléma, kteríž chodíváte skrze brány tyto:
\par 21 Takto praví Hospodin: Vystríhejte se s pilností, abyste nenosili bremen v den sobotní, ani vnášeli skrze brány Jeruzalémské.
\par 22 Ani nevynášejte bremen z domu svých v den sobotní, a žádného díla nedelejte, ale svette den sobotní, jakž jsem prikázal otcum vašim.
\par 23 (Však neuposlechli, aniž naklonili ucha svého, ale zatvrdili šíji svou, neposlouchajíce a neprijímajíce naucení.)
\par 24 Stane se zajisté, jestliže s ochotností mne uposlechnete, dí Hospodin, abyste nenosili bremen skrze brány mesta tohoto v den sobotní, ale svetili den sobotní, nedelajíce v nem žádného díla,
\par 25 Že poberou se skrze brány mesta tohoto králové i knížata sedící na stolici Davidove, jezdíce na vozích i na koních, oni i knížata jejich, muži Judští a obyvatelé Jeruzalémští, a státi bude toto mesto až na vecnost.
\par 26 I budou pricházeti z mest Judských a z okolí Jeruzaléma, jakož z zeme Beniaminovy, tak z roviny, i z té hory, i od poledne, nesouce zápal, a obet i dar s kadidlem, také i díku cinení nesouce do domu Hospodinova.
\par 27 Jestliže pak neuposlechnete mne, abyste svetili den sobotní, a nenosili bremen, chodíce skrze brány Jeruzalémské v den sobotní, tedy zanítím ohen v branách jeho, kterýžto zžíre paláce Jeruzalémské, a neuhasne.

\chapter{18}

\par 1 Slovo, kteréž se stalo k Jeremiášovi od Hospodina, rkoucí:
\par 2 Vstan a sejdi do domu hrncírova, a tam zpusobím to, abys slyšel slova má.
\par 3 I sešel jsem do domu hrncírova, a aj, on delal dílo na kruzích.
\par 4 Když se pak zkazila nádoba v ruce hrncírove, kterouž on delal z hliny, tehdy zase udelal z ní nádobu jinou, jakouž se dobre líbilo hrncíri udelati.
\par 5 I stalo se slovo Hospodinovo ke mne, rkoucí:
\par 6 Zdaliž jako hrncír tento nemohl bych nakládati s vámi, ó dome Izraelský? dí Hospodin. Aj, jakož hlina v ruce hrncíre, tak jste vy v ruce mé, ó dome Izraelský.
\par 7 Mluvil-li bych proti národu a proti království, že je v okamžení vypléním a zkazím, i vyhubím,
\par 8 Však odvrátil-li by se národ ten od nešlechetnosti své, proti nemuž bych mluvil: i já litoval bych toho zlého, kteréž jsem myslil uciniti jemu.
\par 9 Zase mluvil-li bych o národu a o království, že je v okamžení vzdelám a vštípím,
\par 10 Však cinil-li by, což zlého jest pred ocima mýma, neposlouchaje hlasu mého: i já litoval bych dobrodiní toho, kteréž bych rekl uciniti jemu.
\par 11 Protož nyní rci mužum Judským i obyvatelum Jeruzalémským, rka: Takto praví Hospodin: Aj, já strojím na vás zlou vec, a obrátím na vás pohromu; navrattež se již jeden každý od cesty své zlé, a polepšte cest svých i predsevzetí svých.
\par 12 Kterížto rekli: To nic, nebo za myšlénkami svými pujdeme, a jeden každý zdání srdce svého nešlechetného vykonávati budeme.
\par 13 Protož takto praví Hospodin: Vyptejte se nyní mezi pohany, slýchal-li kdo takové veci? Mrzkosti veliké dopustila se panna Izraelská.
\par 14 Zdaliž kdo pohrdá cerstvou vodou Libánskou z skály? Zdaž pohrdají vodami studenými odjinud bežícími?
\par 15 Lid pak muj zapomenuvše se na mne, kadí marnosti. Nebo k úrazu je privodí na cestách jejich, na stezkách starobylých, chodíce stezkami cesty neprotrené,
\par 16 Tak abych musil obrátiti zemi jejich v poušt na odivu vecnou; každý, kdož by šel skrze ni, aby se užasl, a pokynul hlavou svou.
\par 17 Vetrem východním rozptýlím je pred neprítelem; hrbetem a ne tvárí pohledím na ne v cas bídy jejich.
\par 18 I rekli: Podte a vymyslme proti Jeremiášovi nejakou chytrost; nebot nezhyne zákon od kneze, ani rada od moudrého, ani slovo od proroka. Podte a zarazme jej jazykem, a nemejme pozoru na žádná slova jeho.
\par 19 Pozoruj mne, Hospodine, a slyš hlas tech, kteríž se vadí se mnou.
\par 20 Zdaliž má odplacováno býti za dobré zlým, že mi jámu kopají? Rozpomen se, že jsem se postavoval pred oblícejem tvým, abych se primlouval k jejich dobrému, a odvrátil prchlivost tvou od nich.
\par 21 Protož dopust na syny jejich hlad, a zpusob to, at jsou násilne zmordování mecem, a necht jsou ženy jejich osirelé a ovdovelé, a muži jejich at jsou ukrutne zmordováni, a mládenci jejich zbiti mecem v boji.
\par 22 Necht jest slýchati krik z domu jejich, když privedeš na ne vojsko náhle. Nebo vykopali jámu, aby popadli mne, a osídla polékli nohám mým;
\par 23 Ješto ty, Hospodine, povedom jsi vší rady jejich o mém usmrcení. Nebud milostiv nepravosti jejich, a hríchu jejich pred tvárí svou neshlazuj, ale necht jsou k úrazu dostrceni pred oblícejem tvým, a v cas prchlivosti své s nimi zacházej.

\chapter{19}

\par 1 Takto rekl Hospodin: Jdi a zjednej báni záhrdlitou od hrncíre, hlinenou, a pojma nekteré z starších lidu a z starších kneží,
\par 2 Vejdi do údolí Benhinnom, kteréž jest u vrat brány východní, a ohlašuj tam slova ta, kteráž mluviti budu tobe.
\par 3 A rci: Slyšte slovo Hospodinovo, králové Judští i obyvatelé Jeruzalémští: Takto praví Hospodin zástupu, Buh Izraelský: Aj, já uvedu bídu na místo toto, o kteréž kdokoli uslyší, zníti mu bude v uších jeho,
\par 4 Proto že mne opustili, a poškvrnili místa tohoto, kadíce na nem bohum cizím, jichž neznali oni, ani otcové jejich, ani králové Judští, a naplnili toto místo krví nevinných.
\par 5 Vzdelali také výsosti Bálovi, aby pálili syny své ohnem v zápaly Bálovi, cehož jsem neprikázal, aniž jsem o tom mluvil, nýbrž ani nevstoupilo na mé srdce.
\par 6 Protož aj, dnové jdou, dí Hospodin, v nichž nebude slouti více toto místo Tofet, ani údolí Benhinnom, ale údolí mordu.
\par 7 Nebo v nic obrátím radu Judovu i Jeruzalémských v míste tomto, zpusobe to, aby padli od mece pred neprátely svými, a od ruky tech, kteríž hledají bezživotí jejich, i dám mrtvá tela jejich za pokrm ptactvu nebeskému a šelmám zemským.
\par 8 Obrátím také mesto toto v poušt na odivu. Každý, kdožkoli pujde mimo ne, užasne se, a ckáti bude pro všelijaké rány jeho.
\par 9 A zpusobím to, že žráti budou maso synu svých a maso dcer svých, tolikéž jeden každý maso bližního svého žráti bude, v obležení a v ssoužení, kterýmž ssouží je neprátelé jejich, a ti, kteríž hledají bezživotí jejich.
\par 10 Potom roztluc tu záhrdlitou báni pred ocima tech lidí, kteríž pujdou s tebou,
\par 11 A rci jim: Takto praví Hospodin zástupu: Tak potluku lid tento i mesto toto, jako ten, kdož rozráží nádobu hrncírskou, kteráž nemuže opravena býti více, a v Tofet pochovávati budou, proto že nebude žádného místa ku pohrbu.
\par 12 Tak uciním místu tomuto, dí Hospodin, i obyvatelum jeho, a naložím s mestem tímto tak jako s Tofet.
\par 13 Nebo budou domové Jeruzalémských i domové králu Judských tak jako toto místo Tofet zanecišteni se všechnemi domy temi, na jejichž strechách kadili všemu vojsku nebeskému, a obetovali obeti mokré bohum cizím.
\par 14 Tedy navrátiv se Jeremiáš z Tofet, kamž jej byl poslal Hospodin, aby prorokoval tam, postavil se v sínci domu Hospodinova, a rekl ke všemu lidu:
\par 15 Takto praví Hospodin zástupu, Buh Izraelský: Aj, já uvedu na mesto toto i na všecka mesta jeho všecko to zlé, kteréž jsem vyrkl proti nemu; nebo zatvrdili šíji svou, aby neposlouchali slov mých.

\chapter{20}

\par 1 Tedy slyšev Paschur syn Immeruv, knez, kterýž byl prední správce v dome Hospodinove, Jeremiáše prorokujícího o tech vecech,
\par 2 Ubil Paschur Jeremiáše proroka, a dal jej do vezení v bráne Beniaminove horejší, kteráž byla pri dome Hospodinove.
\par 3 Stalo se pak nazejtrí, když vyvedl Paschur Jeremiáše z vezení, že rekl jemu Jeremiáš: Nenazval Hospodin jména tvého Paschur, ale Magor missabib.
\par 4 Nebo takto praví Hospodin: Aj, já pustím na tebe strach, na tebe i na všecky prátely tvé, kteríž padnou od mece neprátel svých, nacež oci tvé hledeti budou, když všecken lid Judský vydám v ruku krále Babylonského, kterýž zavede je do Babylona, a mecem je pobije.
\par 5 Vydám i všelijaké bohatství mesta tohoto, a všecko úsilé jeho, i všelijakou vec drahou jeho, i všecky poklady králu Judských vydám v ruku neprátel jejich, a rozchvátají je, i poberou je, a dovezou je do Babylona.
\par 6 Ty pak Paschur i všickni, kteríž bydlí v dome tvém, pujdete do zajetí, a do Babylona se dostaneš, a tam umreš, i tam pochován budeš ty i všickni milující tebe, jimž jsi prorokoval lžive.
\par 7 Namlouvals mne, Hospodine, a dalt jsem se premluviti; silnejšís byl nežli já, protož zmocnils se mne. Jsem v posmechu každý den, každý se mi posmívá.
\par 8 Nebo jakž jsem zacal mluviti, úpím, pro ukrutenství a zhoubu kricím; slovo zajisté Hospodinovo jest mi ku potupe a ku posmechu každého dne.
\par 9 I rekl jsem: Nebudut ho pripomínati, ani mluviti více ve jménu jeho. Ale jest v srdci mém jako ohen horící, zavrený v kostech mých, jehož snažuje se zdržeti, však nemohu,
\par 10 Ackoli slýchám utrhání mnohých, i Magor missabiba, ríkajících: Povezte neco na nej, a oznámíme to králi. Všickni, kteríž by meli býti prátelé moji, cíhají na poklesnutí mé, ríkajíce: Snad nekde podveden bude, a zmocníme se ho, a pomstíme se nad ním.
\par 11 Ale Hospodin jest se mnou jakožto rek udatný, protož ti, kteríž mne stihají, zurážejí se, a neodolají; stydeti se budou náramne, nebo se jim štastne nezvede, aniž potupa vecná v zapomenutí dána bude.
\par 12 Protož ó Hospodine zástupu, kterýž zkušuješ spravedlivého, kterýž spatruješ ledví a srdce, necht se podívám na pomstu tvou nad nimi, tobe zajisté zjevil jsem pri svou.
\par 13 Zpívejte Hospodinu, chvalte Hospodina, že vytrhl duši nuzného z ruky nešlechetných.
\par 14 Zlorecený ten den, v nemžto zplozen jsem, den, v nemž porodila mne matka má, at není požehnaný.
\par 15 Zlorecený ten muž, kterýž zvestoval otci mému, chteje zvláštne obradovati jej, rka: Narodilot se díte pohlaví mužského.
\par 16 A necht jest ten muž podobný mestum, kteráž podvrátil Hospodin, a neželel; nebo slyšel krik v jitre, a provyskování v cas polední.
\par 17 Ó že mne neusmrtil od života, ješto by mi matka má byla hrobem mým, a život její vecne tehotný.
\par 18 Proc jsem jen z života vyšel, abych okoušel težkosti a zámutku, a aby stráveni byli v pohanení dnové moji?

\chapter{21}

\par 1 Slovo, kteréž se stalo k Jeremiášovi od Hospodina, když poslal k nemu král Sedechiáš Paschura syna Malkiášova a Sofoniáše syna Maaseiášova, kneze, aby rekli:
\par 2 Porad se medle o nás s Hospodinem, nebo Nabuchodonozor král Babylonský bojuje proti nám, zdali by snad naložil Hospodin s námi podlé všech divných skutku svých, aby on odtáhl od nás.
\par 3 Tedy rekl Jeremiáš k nim: Tak rcete Sedechiášovi:
\par 4 Takto praví Hospodin Buh Izraelský: Aj, já odvrátím nástroje válecné, kteréž jsou v rukou vašich, jimiž vy bojujete proti králi Babylonskému a Kaldejským, kteríž oblehli vás vne za zdí, a shromáždím je do prostred mesta tohoto.
\par 5 Bojovati zajisté budu já proti vám rukou vztaženou a ramenem silným, a to v hneve a v rozpálení i v prchlivosti veliké.
\par 6 A raním obyvatele mesta tohoto, tak že lidé i hovada morem velikým pomrou.
\par 7 Potom pak (dí Hospodin), dám Sedechiáše krále Judského a služebníky jeho i lid, totiž ty, kteríž pozustanou v meste tomto po moru, po meci a po hladu, v ruku Nabuchodonozora krále Babylonského, a v ruku neprátel jejich, a tak v ruku hledajících bezživotí jejich. Kterýžto je zbije ostrostí mece, neodpustí jim, aniž jich šanovati bude, aniž se smiluje.
\par 8 Protož rci lidu tomuto: Takto praví Hospodin: Aj, já kladu pred vás cestu života i cestu smrti.
\par 9 Kdokoli zustane v meste tomto, zahyne od mece, neb hladem, neb morem, ale kdož vyjde a poddá se Kaldejským, kteríž oblehli vás, jistotne živ zustane, a bude míti život svuj místo koristi.
\par 10 Nebo postavil jsem zurivou tvár svou proti mestu tomuto k zlému, a ne k dobrému, dí Hospodin. V ruku krále Babylonského vydáno bude, i vypálí je ohnem.
\par 11 Domu pak krále Judského rci: Slyšte slovo Hospodinovo,
\par 12 Ó dome Daviduv, takto praví Hospodin: Držívejte každého jitra soud, a vychvacujte obloupeného z ruky násilníka, aby nevyšla jako ohen prchlivost má, a nehorela, tak že by nebylo žádného, kdo by uhasiti mohl, pro nešlechetnost predsevzetí vašich.
\par 13 Aj já, dí Hospodin, na tebe, kteráž prebýváš v údolí tomto, skálo roviny této, kteríž ríkáte: Kdo by pritáhl na nás, aneb kdo by všel do príbytku našich?
\par 14 Nebo trestati vás budu podlé skutku vašich, dí Hospodin, a zanítím ohen v lese tvém, kterýž zžíre všecko vukol neho.

\chapter{22}

\par 1 Takto rekl Hospodin: Sejdi do domu krále Judského, a mluv tam slovo toto,
\par 2 A rci: Slyš slovo Hospodinovo, králi Judský, kterýž sedíš na stolici Davidove, ty i služebníci tvoji i lid tvuj, kteríž chodíte skrze brány tyto.
\par 3 Takto praví Hospodin: Konejte soud a spravedlnost, a vychvacujte obloupeného z ruky násilníka; príchozímu tolikéž, sirotku, ani vdove necinte krivdy, aniž jich utiskujte, a krve nevinné nevylévejte na míste tomto.
\par 4 Nebo budete-li to pilne vykonávati, jiste že poberou se skrze brány domu tohoto králové, sedící místo Davida na stolici jeho, jezdíce na vozích, neb na koních, král s služebníky svými i s lidem svým.
\par 5 Jestliže pak neuposlechnete slov techto, skrze sebe prisahám, dí Hospodin, že poušt bude dum tento.
\par 6 Nebo takto praví Hospodin o domu krále Judského: Byl jsi mi jako Galád a vrch Libánský, ale obrátím te jistotne v poušt jako mesta, v nichž se bydliti nemuže.
\par 7 A pristrojím na tebe zhoubce, jednoho každého se zbrojí jeho, kterížto zpodtínají nejvýbornejší cedry tvé, a vmecí na ohen.
\par 8 A když pujdou národové mnozí mimo mesto toto, a rekne jeden druhému: Proc tak ucinil Hospodin mestu tomuto velikému?
\par 9 Tedy reknou: Proto že opustili smlouvu Hospodina Boha svého, a klaneli se bohum cizím a sloužili jim.
\par 10 Neplactež mrtvého, aniž ho litujte, ale ustavicne placte prícinou toho, kterýž odchází; nebot se nenavrátí více, aby pohledel na zemi, v níž se narodil.
\par 11 Nebo takto praví Hospodin o Sallumovi synu Joziáše, krále Judského, kterýž kraluje místo Joziáše otce svého: Když vyjde z místa tohoto, nenavrátí se sem více.
\par 12 Ale tam v tom míste, kamž jej zastehují, umre, a tak zeme této neuzrí více.
\par 13 Beda tomu, kdož staví dum svuj s útiskem, a paláce své s krivdou, kterýž bližního svého v službu podrobuje darmo, mzdy pak jeho nedává jemu;
\par 14 Kterýž ríká: Vystavím sobe dum veliký a paláce prostranné, kterýž prolamuje sobe okna, a tafluje cedrovím, a maluje barvou.
\par 15 Zdaliž kralovati budeš, že se pleteš v to cedrové stavení? Otec tvuj zdaliž nejídal a nepíjel? Když konal soud a spravedlnost, tedy dobre bylo jemu.
\par 16 Když soudíval pri chudého a nuzného, tedy dobre bylo. Zdaliž mne to není známé? dí Hospodin.
\par 17 Ale oci tvé i srdce tvé nehledí než lakomství tvého, a abys krev nevinnou proléval, a nátisk a krivdu cinil.
\par 18 Protož takto praví Hospodin o Joakimovi synu Joziáše, krále Judského: Nebudout ho kvíliti: Ach, bratre muj, aneb ach, sestro. Nebudou ho kvíliti: Ach, pane, aneb ach, kdež dustojnost jeho?
\par 19 Pohrbem oslicím pohrben bude, vyvlecen a vyvržen jsa za brány Jeruzalémské.
\par 20 Vstup na Libán a kric, i na hore Bázan vydej hlas svuj; kric také pres brody, když potríni budou všickni milovníci tvoji.
\par 21 Mluvíval jsem s tebou v nejvetším štestí tvém, ríkávalas: Nebudut poslouchati. Tat jest cesta tvá od detinství tvého; neuposlechlas zajisté hlasu mého.
\par 22 Všecky pastýre tvé zpase vítr, a milovníci tvoji v zajetí pujdou. Tehdáž jiste hanbiti a stydeti se budeš za všelikou nešlechetnost svou.
\par 23 Ó ty, kteráž jsi usadila se na Libánu, jenž se hnízdíš na cedroví, jak milostná budeš, když na te prijdou svírání a bolest jako rodicky!
\par 24 Živt jsem já, dí Hospodin, že byt pak byl Koniáš syn Joakima, krále Judského, prstenem pecetním na mé ruce pravé, však te i odtud strhnu.
\par 25 A vydám te v ruku tech, kteríž hledají bezživotí tvého, a v ruku tech, jejichž ty se oblíceje lekáš, totiž v ruku Nabuchodonozora krále Babylonského, a v ruku Kaldejských.
\par 26 A hodím tebou i matkou tvou, kteráž te porodila, do zeme cizí, tam, kdež jste se nezrodili, a tam zemrete.
\par 27 Do zeme pak, po níž toužiti budete, abyste se navrátili tam, tam se nenavrátíte.
\par 28 Zdaliž modlou nicemnou, kteráž roztrískána bývá, bude muž tento Koniáš? Zdali nádobou, v níž není žádné líbosti? Proc by vyházíni byli on i síme jeho, a uvrženi do zeme, o níž nevedí?
\par 29 Ó zeme, zeme, zeme, slyš slovo Hospodinovo.
\par 30 Takto praví Hospodin: Zapište to, že muž tento bez detí bude, a že se jemu nepovede štastne za dnu jeho. Anobrž nepovede se štastne i tomu muži, kterýž by z semene jeho sedel na stolici Davidove, a panoval ješte nad Judou.

\chapter{23}

\par 1 Beda pastýrum hubícím a rozptylujícím stádce pastvy mé, dí Hospodin.
\par 2 Protož takto praví Hospodin Buh Izraelský o pastýrích, kteríž pasou lid muj: Vy rozptylujete ovce mé, anobrž rozháníte je, a nenavštevujete jich; aj, já navštívím vás pro nešlechetnost predsevzetí vašich, dí Hospodin.
\par 3 Ostatek pak ovcí svých já shromáždím ze všech zemí, do nichž jsem je rozehnal, a privedu je zase do ovcincu jejich, kdežto ploditi a množiti se budou.
\par 4 Nadto ustanovím nad nimi pastýre, kteríž by je pásli, aby se nebály více, ani strachovaly, ani hynuly, dí Hospodin.
\par 5 Aj, dnové jdou, dí Hospodin, v nichž vzbudím Davidovi výstrelek spravedlivý, i kralovati bude král, a štastne se jemu povede; soud zajisté a spravedlnost na zemi konati bude.
\par 6 Za dnu jeho spasen bude Juda, a Izrael bydliti bude bezpecne, a tot jest jméno jeho, kterýmž ho nazývati bude: Hospodin spravedlnost naše.
\par 7 Protož aj, dnové jdou, dí Hospodin,v nichž nebude ríkáno více: Živt jest Hospodin, kterýž vyvedl syny Izraelské z zeme Egyptské,
\par 8 Ale: Živt jest Hospodin, kterýž vyvedl a kterýž zprovodil síme domu Izraelského z zeme pulnocní i ze všech zemí, do nichž jsem byl je rozehnal, když se osadí v zemi své.
\par 9 Prícinou proroku potríno jest srdce mé ve mne, pohnuly se všecky kosti mé; jsem jako clovek opilý, a jako muž, kteréhož rozešlo víno, pro Hospodina a pro slova svatosti jeho.
\par 10 Nebo cizoložníku plná jest tato zeme, a prícinou krivých prísah kvílí zeme, usvadla pastviska na poušti; jest zajisté utiskování techto nešlechetné, a moc jejich nepravá.
\par 11 Nebo jakož prorok, tak knez pokrytství páchají. Také v dome svém nacházím nešlechetnost jejich, dí Hospodin.
\par 12 Procež budou míti cestu svou podobnou plzkosti v mrákote, na níž postrceni budou a padnou, když uvedu na ne bídu v cas navštívení jejich, dí Hospodin.
\par 13 Pri prorocích zajisté Samarských videl jsem nesmyslnost; prorokovali skrze Bále, a svodili lid muj Izraelský.
\par 14 Ale pri prorocích Jeruzalémských vidím hroznou vec, že cizoložíce a se lží se obcházejíce, posilnují také rukou nešlechetníku, aby se neobrátil žádný od nešlechetnosti své. Mám všecky za podobné Sodome, a obyvatele jeho za podobné Gomore.
\par 15 Protož takto praví Hospodin zástupu o prorocích techto: Aj, já nakrmím je pelynkem, a napojím je vodami jedovatými; nebo od proroku Jeruzalémských vyšla poškvrna na všecku tuto zemi.
\par 16 Takto praví Hospodin zástupu: Neposlouchejtež slov tech proroku, jenž prorokují vám, prázdných vás zanechávajíce. Videní srdce svého mluví, ne z úst Hospodinových.
\par 17 Ustavicne ríkají tem, kteríž mnou pohrdají: Pravil Hospodin: Pokoj míti budete, a každému chodícímu podlé zdání srdce svého ríkají: Neprijdet na vás nic zlého.
\par 18 Nebo kdož jest stál v rade Hospodinove, a videl neb slyšel slovo jeho? Kdo pozoroval slova jeho, neb vyslechl je?
\par 19 Aj, vichrice Hospodinova s prchlivostí vyjde, a to vichrice trvající; nad hlavou nešlechetných trvati bude.
\par 20 Neodvrátít se hnev Hospodinuv, dokudž neuciní a nevykoná úmyslu srdce svého. A tehdáž porozumíte tomu cele,
\par 21 Žet jsem neposílal tech proroku, ale sami beželi, že jsem nemluvil k nim, a však oni prorokovali.
\par 22 Nebo byt byli stáli v rade mé, jiste že by byli ohlašovali slova má lidu mému, a bylit by je odvraceli od cesty jejich zlé, a od nešlechetnosti predsevzetí jejich.
\par 23 Zdaliž jsem já Buh jen z blízka? dí Hospodin. A nejsem Buh i z daleka?
\par 24 Zdaž se ukryje kdo v skrýších, abych já ho neshlédl? dí Hospodin. Zdaliž nebe i zeme já nenaplnuji? dí Hospodin.
\par 25 Slýchávámt, co ríkají ti proroci, kteríž prorokují lež ve jménu mém, ríkajíce: Mel jsem sen, mel jsem sen.
\par 26 I dokudž to bude? Zdaliž v srdci tech proroku, kteríž prorokují, není lež? Anobrž jsou proroci lsti srdce svého,
\par 27 Kteríž obmýšlejí to, jak by vyrazili z pameti lidu mému jméno mé sny svými, kteréž vypravují jeden každý bližnímu svému, jako se zapomneli otcové jejich na jméno mé za prícinou Bále.
\par 28 Prorok, kterýž má sen, necht vypravuje sen, ale kterýž má slovo mé, necht mluví slovo mé práve. Co jest té pleve do pšenice? dí Hospodin.
\par 29 Zdaliž není slovo mé takové jako ohen, dí Hospodin, a jako kladivo rozrážející skálu?
\par 30 Protož aj já, dí Hospodin, proti tem prorokum, kteríž ukrádají slova má jeden každý pred bližním svým.
\par 31 Aj já, dí Hospodin, proti tem prorokum, kteríž chlubne mluví, ríkajíce: Praví Hospodin.
\par 32 Aj já, dí Hospodin, proti tem, kteríž prorokují sny lživé, a vypravujíce je, svodí lid muj lžmi svými a žvavostí svou, ješto jsem já jich neposlal, aniž jsem jim prikázal. Procež naprosto nic neprospívají lidu tomuto, dí Hospodin.
\par 33 Protož, když by se tázal tebe lid tento, neb nekterý prorok neb knez, rka: Jaké jest bríme Hospodinovo? tedy rci jim: Jaké bríme? I to: Opustím vás, dí Hospodin.
\par 34 Nebo proroka a kneze toho i lid ten, kterýž by rekl: Bríme Hospodinovo, jiste trestati budu muže toho i dum jeho.
\par 35 Ale takto ríkejte jeden každý bližnímu svému a jeden každý bratru svému: Co odpovedel Hospodin? aneb: Co mluvil Hospodin?
\par 36 Bremene pak Hospodinova nepripomínejte více, sic by bremenem bylo jednomu každému slovo jeho, když byste prevraceli slova Boha živého, Hospodina zástupu, Boha našeho.
\par 37 Takto ríkati budeš proroku: Cot odpovedel Hospodin? aneb: Co mluvil Hospodin?
\par 38 Ale ponevadž ríkáte: Bríme Hospodinovo, tedy takto praví Hospodin: Ponevadž ríkáte slovo to: Bríme Hospodinovo, ješto jsem posílal k vám, ríkaje: Neríkejte: Bríme Hospodinovo,
\par 39 Protož aj, já jiste zapomenu se na vás do konce, a zavrhu vás i to mesto, kteréž jsem byl dal vám i otcum vašim, od tvári své,
\par 40 A uvedu na vás pohanení vecné i potupu vecnou, kteráž neprijde v zapomenutí.

\chapter{24}

\par 1 Ukázal mi Hospodin, a aj, dva košové fíku postaveni byli pred chrámem Hospodinovým, když byl prestehoval Nabuchodonozor král Babylonský Jekoniáše syna Joakimova, krále Judského, a knížata Judská, i tesare a kováre z Jeruzaléma, a privedl je do Babylona.
\par 2 Jeden koš byl fíku velmi dobrých, jacíž bývají fíkové ranní, druhý pak koš fíku velmi zlých, jakýchž nelze jísti pro trpkost.
\par 3 Tedy rekl mi Hospodin: Co vidíš, Jeremiáši? I rekl jsem: Fíky, dobré fíky, a to velmi dobré, zlé pak, a to velmi zlé, jichž nelze jísti pro trpkost.
\par 4 I stalo se slovo Hospodinovo ke mne, rkoucí:
\par 5 Takto praví Hospodin, Buh Izraelský: Jako fíkové tito dobrí, tak mne príjemní budou zajatí Judští, kteréž jsem zaslal z místa tohoto do zeme Kaldejské k dobrému.
\par 6 Obrátím zajisté oci své k nim k dobrému, a privedu je zase do zeme této, kdežto vzdelám je, a nezkazím, štípím je, a nevypléním.
\par 7 Nebo dám jim srdce, aby znali mne, že já jsem Hospodin. I budou mým lidem, a já budu jejich Bohem, když se obrátí ke mne celým srdcem svým.
\par 8 Naodpor, jako fíky zlé, kterýchž nelze jísti pro trpkost, tak zavrhu (tot zajisté praví Hospodin), Sedechiáše krále Judského s knížaty jeho, a ostatek Jeruzalémských pozustalých v zemi této, i ty, kteríž bydlí v zemi Egyptské.
\par 9 Vydám je, pravím, v posmýkání k zlému po všech královstvích zeme, v pohanení a v prísloví, v rozprávku a v proklínání po všech tech místech, kamž je rozženu.
\par 10 A budu posílati na ne mec, hlad a mor, dokudž by do konce vyhlazeni nebyli z zeme, kterouž jsem byl dal jim i otcum jejich.

\chapter{25}

\par 1 Slovo, kteréž se stalo k Jeremiášovi proti všemu lidu Judskému, léta ctvrtého Joakima syna Joziášova, krále Judského, (jenž jest první rok Nabuchodonozora krále Babylonského),
\par 2 Kteréž mluvil Jeremiáš prorok ke všemu lidu Judskému, i ke všechnem obyvatelum Jeruzalémským, rka:
\par 3 Od trináctého léta Joziáše syna Amonova, krále Judského, až do tohoto dne, po techto trimecítma let, bývalo slovo Hospodinovo ke mne, kteréž jsem vám mluvíval, ráno privstávaje, a to ustavicne, ale neposlouchali jste.
\par 4 Posílal také Hospodin k vám všecky slouhy své proroky, ráno privstávaje, a to ustavicne, (jichžto neposlouchali jste, aniž jste naklonili ucha svého, abyste slyšeli).
\par 5 Kteríž ríkali: Navrattež se již jeden každý z cesty své zlé a od nešlechetnosti predsevzetí svých, a tak osazujte se v té zemi, kterouž dal Hospodin vám i otcum vašim od veku až na veky.
\par 6 A nechodte za bohy cizími, abyste sloužiti meli jim, aniž se jim klanejte, aniž hnevejte mne dílem rukou svých, a neucinímt vám zle.
\par 7 Ale neposlouchali jste mne, dí Hospodin, abyste jen hnevali mne dílem rukou svých k svému zlému.
\par 8 Protož takto praví Hospodin zástupu: Proto že jste neuposlechli slov mých,
\par 9 Aj, já pošli, a pojmu všecky národy pulnocní, dí Hospodin, i k Nabuchodonozorovi králi Babylonskému, služebníku svému, a privedu je na zemi tuto i na obyvatele její, i na všecky národy tyto okolní, kteréž jako proklaté vyhladím, a zpusobím to, aby byli k užasnutí, na odivu, a poustkou vecnou.
\par 10 Také zpusobím to, aby zahynul jim hlas radosti i hlas veselé, hlas ženicha i hlas nevesty, hluk žernovu i svetlo svíce.
\par 11 I bude všecka zeme tato pustinou a pouští, a sloužiti budou národové tito králi Babylonskému sedmdesáte let.
\par 12 Potom pak po vyplnení sedmdesáti let trestati budu na králi Babylonském a na tom národu, dí Hospodin, nepravost jejich, totiž na zemi Kaldejské, tak že obrátím ji v pustiny vecné.
\par 13 A uvedu na zemi tu všecka slova svá, kteráž jsem mluvil o ní, všecko, což psáno jest v knize této, cožkoli prorokoval Jeremiáš o všech národech.
\par 14 Když je v službu podrobí, i jiné národy mnohé a krále veliké, tehdáž odplatím jim podlé skutku jejich a podlé cinu rukou jejich.
\par 15 Nebo takto mi rekl Hospodin, Buh Izraelský: Vezmi kalich vína prchlivosti této z ruky mé, a napájej jím všecky ty národy, k kterýmž já pošli tebe,
\par 16 Aby pili a potáceli se, anobrž bláznili prícinou mece, kterýž já pošli mezi ne.
\par 17 I vzal jsem kalich z ruky Hospodinovy, a napájel jsem všecky ty národy, k nimž mne poslal Hospodin,
\par 18 Jeruzalémské i mesta zeme Judské, a krále její i knížata její, abych je oddal v pustinu a v zpuštení, na odivu, i k proklínání tohoto dnešního dne,
\par 19 Faraona krále Egyptského i služebníky jeho, i knížata jeho, i všecken lid jeho,
\par 20 I všecku tu smesici, totiž všecky krále zeme Uz, všecky také krále zeme Filistinské, i Aškalon, i Gázy, i Akaron, i ostatek Azotu,
\par 21 Idumejské, i Moábské, i syny Ammon,
\par 22 I všecky krále Tyrské, i všecky krále Sidonské, i krále krajiny té, kteráž jest pri mori,
\par 23 Dedana a Temu, a Buzu i všecky prebývající v koutech nejzadnejších,
\par 24 I všecky krále Arabské, i všecky krále té smesice, kteríž bydlí na poušti,
\par 25 Všecky také krále Zamritské, i všecky krále Elamitské, též všecky krále Médské,
\par 26 Anobrž všecky krále pulnocní, blízké i daleké, jednoho jako druhého, všecka také království zeme, kterážkoli jsou na svrchku zeme. Král pak Sesák píti bude po nich.
\par 27 A rci jim: Takto praví Hospodin zástupu, Buh Izraelský: Pítež a opojte se, anobrž vyvracejte z sebe, a padejte, tak abyste nepovstali pro mec, kterýž já pošli mezi vás.
\par 28 Jestliže by pak nechteli vzíti kalichu z ruky tvé, aby pili, tedy díš jim: Takto praví Hospodin zástupu: Konecne že píti musíte.
\par 29 Nebo ponevadž na to mesto, kteréž nazváno jest od jména mého, já zacínám uvozovati zlé veci, a vy abyste bez hodné pomsty byli? Nebudete bez pomsty, nebo já zavolám mece na všecky obyvatele té zeme, dí Hospodin zástupu.
\par 30 Protož ty prorokuj proti nim všecka slova tato, a rci jim: Hospodin s výsosti rváti bude, a z príbytku svatosti své vydá hlas svuj, mocne rváti bude z obydlí svého. Krik ponoukajících se, jako presovníku, rozléhati se bude proti všechnem obyvatelum té zeme,
\par 31 I prujde hrmot až do konce zeme. Nebo rozepri má Hospodin s temi národy, v soud vchází sám se všelikým telem; bezbožníky vydá pod mec, dí Hospodin.
\par 32 Takto praví Hospodin zástupu: Aj, bída pujde z národu na národ, a vichrice veliká strhne se od koncin zeme.
\par 33 I budou zbiti od Hospodina v ten cas od konce zeme až do konce zeme; nebudou oplakáni, ani sklizeni, ani pochováni, místo hnoje na svrchku zeme budou.
\par 34 Kvelte pastýri a kricte, anobrž válejte se v popele, vy nejznamenitejší toho stáda; nebo naplnili se dnové vaši, abyste zbiti byli, a abyste rozptýleni byli, i budete padati jako nádoba drahá.
\par 35 I zahyne útocište pastýrum a utíkání nejznamenitejším toho stáda.
\par 36 Hlas žalostný pastýru a kvílení nejznamenitejších toho stáda; nebo zkazí Hospodin pastvu jejich.
\par 37 Zkažena budou i pastviska pokoj mající, pro prchlivost hnevu Hospodinova,
\par 38 Jako lev opustí jeskyni svou; nebo prijde zeme jejich na spuštení, pro prchlivost zhoubce a pro prchlivost hnevu jeho.

\chapter{26}

\par 1 Na pocátku kralování Joakima syna Joziášova, krále Judského, stalo se slovo toto od Hospodina, rkoucí:
\par 2 Takto praví Hospodin: Postav se v sínci domu Hospodinova, a mluv ke všechnem mestum Judským, pricházejícím klaneti se v dome Hospodinove, všecka slova, kteráž tobe prikazuji mluviti k nim, neujímejž slova,
\par 3 Zdali by aspon uposlechli, a odvrátili se jeden každý od cesty své zlé, abych litoval zlého kteréž myslím uciniti jim pro nešlechetnost predsevzetí jejich.
\par 4 Rciž tedy jim: Takto praví Hospodin: Neuposlechnete-li mne, abyste chodili v zákone mém, kterýž jsem predložil vám,
\par 5 Poslouchajíce slov služebníku mých proroku, kteréž já posílám k vám, jakož jste, když jsem je, ráno privstávaje posílal, neposlouchali:
\par 6 Jiste žet naložím s domem tímto jako s Sílo, a mesto toto vydám v proklínání všechnem národum zeme.
\par 7 Slyšeli pak kneží a proroci, i všecken lid Jeremiáše mluvícího slova ta v domu Hospodinovu.
\par 8 I stalo se, že hned, jakž prestal Jeremiáš mluviti všeho, cožkoli prikázal Hospodin mluviti ke všemu lidu, jali jej ti kneží a proroci i všecken lid ten, rkouce: Smrtí umreš.
\par 9 Proc jsi prorokoval ve jménu Hospodinovu, rka: Stane se jako Sílo domu tomuto, a mesto toto tak spustne, že nebude v nem žádného obyvatele? Shromaždoval se pak všecken lid k Jeremiášovi do domu Hospodinova.
\par 10 Tedy uslyšavše knížata Judská ty veci, prišli z domu královského do domu Hospodinova, a posadili se u dverí brány Hospodinovy nové.
\par 11 I rekli kneží a proroci tem knížatum a všemu lidu, rkouce:Hoden jest smrti muž tento; nebo prorokoval proti mestu tomuto, jakž jste slyšeli v své uši.
\par 12 Tedy promluvil Jeremiáš ke všechnem knížatum tem i ke všemu lidu, rka: Hospodin poslal mne, abych prorokoval o domu tomto i o meste tomto všecky ty veci, kteréž jste slyšeli.
\par 13 Protož nyní polepšte cest svých a predsevzetí svých, a poslouchejte hlasu Hospodina Boha svého, i bude litovati Hospodin toho zlého, kteréž vyrkl proti vám.
\par 14 Já pak aj, v rukou vašich jsem, ucinte mi, což se vám za dobré a spravedlivé vidí.
\par 15 Ale však jistotne vezte, usmrtíte-li mne, že krev nevinnou na sebe uvedete, i na mesto toto, i na obyvatele jeho; nebo v pravde poslal mne Hospodin k vám, abych mluvil v uši vaše všecka slova tato.
\par 16 I rekli knížata i všecken lid knežím a tem prorokum: Nemát nikoli muž tento odsuzován býti na smrt, ponevadž ve jménu Hospodina Boha našeho mluvil nám.
\par 17 Tedy povstali nekterí z starších té zeme, a promluvili ke všemu shromáždení lidu, rkouce:
\par 18 Micheáš Moraštický prorokoval za casu Ezechiáše krále Judského, a pravil všemu lidu Judskému, rka: Takto praví Hospodin zástupu: Sion jako pole orán bude, a Jeruzalém jako hromady, hora pak domu tohoto jako lesové vysocí.
\par 19 Zdaliž hned proto usmrtil jej Ezechiáš král Judský a všecken Juda? Zdaliž neulekl se Hospodina, a nemodlil se Hospodinu? I litoval Hospodin toho zlého, kteréž vyrkl proti nim. Protož my ciníme velmi zlou vec proti dušem svým.
\par 20 A byl také muž prorokující ve jménu Hospodinovu, Uriáš syn Semaiášuv z Kariatjeharim, kterýž prorokoval o meste tomto i o zemi této v táž všecka slova jako Jeremiáš.
\par 21 A když uslyšel král Joakim a všickni udatní jeho, i všecka knížata slova jeho, hned usiloval král usmrtiti jej. O cemž uslyšev Uriáš, bál se, a utíkaje, prišel do Egypta.
\par 22 Ale poslal král Joakim nekteré do Egypta, Elnatana syna Achborova i jiné s ním do Egypta.
\par 23 Kteríž vyvedše Uriáše z Egypta, privedli jej k králi Joakimovi. I zabil jej mecem, a vhodil telo jeho do hrobu lidu obecného.
\par 24 A však ruka Achikamova syna Safanova byla pri Jeremiášovi, aby ho nevydával v ruku lidu k usmrcení jeho.

\chapter{27}

\par 1 Na pocátku kralování Joakima syna Joziášova, krále Judského, stalo se slovo toto k Jeremiášovi od Hospodina, rkoucí:
\par 2 Takto rekl Hospodin ke mne: Zdelej sobe obojecky a jha, a dej je na šíji svou.
\par 3 Potom je pošli k králi Idumejskému, a k králi Moábskému, a k králi synu Ammonových, a k králi Tyrskému, a k králi Sidonskému po tech poslích, kteríž prijedou do Jeruzaléma k Sedechiášovi králi Judskému.
\par 4 A prikaž jim, at pánum svým reknou: Takto praví Hospodin zástupu, Buh Izraelský: Tak rcete pánum svým:
\par 5 Já jsem ucinil zemi, cloveka i hovada, kterážkoli jsou na svrchku zeme, mocí svou velikou a ramenem svým vztaženým. Protož dávám ji, komuž se mi dobre líbí.
\par 6 Jako nyní já dal jsem všecky zeme tyto v ruku Nabuchodonozora krále Babylonského, služebníka svého, ano i živocichy polní dal jsem jemu, aby sloužili jemu.
\par 7 Protož budout sloužiti jemu všickni ti národové, i synu jeho, i synu syna jeho, dokudž by neprišel cas zeme jeho i jeho samého, když v službu podrobí jej sobe národové znamenití a králove velicí.
\par 8 Stane se pak, že národ ten i království to, kteréž by nesloužilo jemu, Nabuchodonozorovi králi Babylonskému, a kterýž by nepoddal šíje své pod jho krále Babylonského, mecem a hladem i morem navštívím národ ten, dí Hospodin, dokudž bych do konce nevyplénil jich rukou jeho.
\par 9 Protož vy neposlouchejte proroku svých, ani hadacu svých, ani snu svých, ani planetáru svých, ani kouzedlníku svých, kteríž mluvívají k vám, ríkajíce: Nebudete sloužiti králi Babylonskému.
\par 10 Nebo oni vám lež prorokují, abych vzdálil vás od zeme vaší, a vyhnal vás, abyste zahynuli.
\par 11 Národu pak, kterýž skloní šíji svou pod jho krále Babylonského a sloužiti bude jemu, toho zajisté nechám v zemi jeho, dí Hospodin, aby delal ji, a bydlil v ní.
\par 12 Sedechiášovi také, králi Judskému, mluvil jsem naskrze ta všecka slova, rka: Sklonte šíje své pod jho krále Babylonského, a služte jemu i lidu jeho, a budte živi.
\par 13 Proc máte zahynouti, ty i lid tvuj, mecem, hladem a morem, jakž mluvil Hospodin o národu, kterýž by nesloužil králi Babylonskému?
\par 14 Neposlouchejtež tedy slov proroku tech, kteríž mluvíce k vám, ríkají: Nebudete sloužiti králi Babylonskému. Nebo oni vám lež prorokují.
\par 15 Neposlalt jsem jich zajisté, dí Hospodin, a však oni prorokují ve jménu mém lžive, abych zahnal vás, kdež byste zahynuli vy i ti proroci, kteríž prorokují vám.
\par 16 Knežím také i všemu lidu tomu mluvil jsem, rka: Takto praví Hospodin: Neposlouchejte slov proroku svých, kteríž prorokují vám, ríkajíce: Aj, nádobí domu Hospodinova navrácena budou z Babylona již brzo. Nebot lež oni prorokují vám.
\par 17 Neposlouchejtež jich, služte králi Babylonskému a živi budte. Proc má býti toto mesto pouští?
\par 18 Jestliže pak oni jsou proroci, a jestliže slovo Hospodinovo jest v nich, necht se medle primluví k Hospodinu zástupu, at nádobí to, pozustávající v dome Hospodinove a v dome krále Judského a v Jeruzaléme, nedostává se do Babylona.
\par 19 Nebo takto praví Hospodin zástupu o tech sloupích, a o tom mori, a o tech podstavcích, i o ostatku nádobí pozustávajícím v meste tomto,
\par 20 Kteréhož nepobral Nabuchodozor král Babylonský, když prestehoval Jekoniáše syna Joakimova, krále Judského, z Jeruzaléma do Babylona, a všecky nejprednejší Judské i Jeruzalémské,
\par 21 Takto zajisté dí Hospodin zástupu, Buh Izraelský, o tech nádobách, pozustávajících v dome Hospodinove a v dome krále Judského i v Jeruzaléme:
\par 22 Do Babylona zavezeny budou, a tam budou až do dne toho, v nemž je navštívím, dí Hospodin, a rozkáži je privezti, a zase navrátím je na místo toto.

\chapter{28}

\par 1 Stalo se pak léta toho, na pocátku kralování Sedechiáše krále Judského, léta totiž ctvrtého, mesíce pátého, mluvil ke mne Chananiáš syn Azuruv, prorok, kterýž byl z Gabaon, v dome Hospodinove, pred ocima kneží i všeho lidu, rka:
\par 2 Takto praví Hospodin zástupu, Buh Izraelský, rka: Polámal jsem jho krále Babylonského.
\par 3 Po dvou letech já navrátím zase na místo toto všecka nádobí domu Hospodinova, kteráž pobral Nabuchodonozor král Babylonský z místa tohoto, a zavezl je do Babylona.
\par 4 Jekoniáše také syna Joakimova, krále Judského, i všecky zajaté Judské, kteríž se dostali do Babylona, já zase privedu na místo toto, dí Hospodin; nebo polámi jho krále Babylonského.
\par 5 Tedy rekl Jeremiáš prorok Chananiášovi proroku tomu pred ocima kneží a pred ocima všeho lidu, kteríž stáli v dome Hospodinove,
\par 6 Rekl, pravím, Jeremiáš prorok: Amen, uciniž tak Hospodin. Potvrdiž Hospodin slov tvých, kteráž jsi prorokoval o navrácení nádobí domu Hospodinova, a všech zajatých z Babylona na místo toto.
\par 7 Ale však poslechni medle slova tohoto, kteréž já mluvím pri tvé prítomnosti a pri prítomnosti všeho tohoto lidu.
\par 8 Proroci, kteríž bývali prede mnou i pred tebou od veku, ti prorokovali proti zemím znamenitým a proti královstvím velikým o válce a o ssoužení a o moru.
\par 9 Prorok ten, kterýž prorokuje o pokoji, když dojde slovo toho proroka, ten prorok znám bývá, že jej poslal Hospodin v pravde.
\par 10 Tedy snal Chananiáš prorok to jho z šíje Jeremiáše proroka, a polámal je.
\par 11 A mluvil Chananiáš pred ocima všeho lidu, rka: Takto praví Hospodin: Tak polámi jho Nabuchodonozora krále Babylonského po dvou letech z šíje všech národu. I pocal jíti Jeremiáš prorok cestou svou.
\par 12 Ale stalo se slovo Hospodinovo k Jeremiášovi, když polámal Chananiáš prorok to jho z šíje Jeremiáše proroka, rkoucí:
\par 13 Jdi a mluv k Chananiášovi, rka: Takto praví Hospodin: Jha drevená jsi polámal, protož zdelej místo nich jha železná.
\par 14 Nebo takto praví Hospodin zástupu, Buh Izraelský: Jho železné vložím na šíji všech národu techto, aby sloužili Naduchodonozorovi králi Babylonskému, a budout sloužiti jemu. Také i živocichy polní dám jemu.
\par 15 Zatím rekl Jeremiáš prorok Chananiášovi proroku: Slyšiž nyní, Chananiáši: Neposlal tebe Hospodin, ale ty jsi k tomu privedl, aby lid tento ve lži skládal doufání.
\par 16 Protož takto praví Hospodin: Aj, já sklidím te se svrchku zeme, tento rok ty umreš; nebo jsi mluvil to, címž bys odvrátil lid od Hospodina.
\par 17 I umrel Chananiáš prorok roku toho mesíce sedmého.

\chapter{29}

\par 1 Tato jsou slova listu, kterýž poslal Jeremiáš prorok z Jeruzaléma k ostatku starších zajatých a k knežím i k prorokum i ke všemu lidu, kterýž prestehoval Nabuchodonozor z Jeruzaléma do Babylona,
\par 2 Když vyšel Jekoniáš král a královna, i komorníci, knížata Judská i Jeruzalémská, tolikéž tesari i kovári z Jeruzaléma,
\par 3 Po Elasovi synu Safanovu, a Gemariášovi synu Helkiášovu, (kteréž byl poslal Sedechiáš král Judský k králi Babylonskému do Babylona), rka:
\par 4 Takto praví Hospodin zástupu, Buh Izraelský, všechnem zajatým, kteréž jsem prestehoval z Jeruzaléma do Babylona:
\par 5 Stavejte domy, a osazujte se, štepujte také štepnice, a jezte ovoce jejich.
\par 6 Pojímejte ženy, a plodte syny i dcery, dávejte také synum svým ženy, dcery též své dávejte za muže, at rodí syny a dcery; množte se tam, a neberte umenšení.
\par 7 A hledejte pokoje mesta toho, do kteréhož jsem zastehoval vás, a modlívejte se za ne Hospodinu; nebo v pokoji jeho budete míti pokoj.
\par 8 Takto zajisté praví Hospodin zástupu, Buh Izraelský: Necht vás nesvodí proroci vaši, kteríž jsou mezi vámi, ani hadaci vaši, a nespravujte se sny svými, jichž vy prícinou jste, aby je mívali.
\par 9 Nebo vám oni lžive prorokují ve jménu mém; neposlalt jsem jich, dí Hospodin.
\par 10 Takto zajisté praví Hospodin: Že jakž se jen vyplní Babylonu sedmdesáte let, navštívím vás, a potvrdím vám slova svého výborného o navrácení vás na místo toto.
\par 11 Nebo já nejlépe znám myšlení, kteráž myslím o vás, dí Hospodin, myšlení o pokoji, a ne o trápení, abych ucinil vašemu ocekávání konec prežádostivý.
\par 12 Když mne vzývati budete, a pujdete, a modliti se mne budete, tedy vyslyším vás.
\par 13 A hledajíce mne, naleznete, když mne hledati budete celým srdcem svým.
\par 14 Dám se zajisté nalezti vám, dí Hospodin, a privedu zase zajaté vaše, a shromáždím vás ze všech národu i ze všech míst, kamžkoli jsem zahnal vás, dí Hospodin, a uvedu vás zase na místo toto, odkudž jsem vás zastehoval,
\par 15 Když reknete: Vzbuzovalt nám Hospodin proroky v Babylone.
\par 16 Nebo takto praví Hospodin o králi sedícím na stolici Davidove, a o všem lidu obývajícím v meste tomto, bratrích vašich, kteríž nevyšli s vámi v tom zajetí,
\par 17 Takto dí Hospodin zástupu: Aj, já pošli na ne mec, hlad a mor, a naložím s nimi jako s fíky trpkými, kterýchž nelze jísti pro trpkost.
\par 18 Nebo stihati je budu mecem, hladem i morem, a vydám je ku posmýkání po všech královstvích zeme, k proklínání, a k užasnutí, anobrž na odivu, a k utrhání mezi všemi národy, tam kdež je vypudím,
\par 19 Proto že neposlouchají slov mých, dí Hospodin, když posílám k nim služebníky své proroky, ráno privstávaje, a to ustavicne, a neposlouchali jste, dí Hospodin.
\par 20 Protož slyštež vy slovo Hospodinovo, všickni zajatí, kteréž jsem vyslal z Jeruzaléma do Babylona.
\par 21 Takto praví Hospodin zástupu, Buh Izraelský, o Achabovi synu Kolaiášovu, a o Sedechiášovi synu Maaseiášovu, kteríž prorokují vám ve jménu mém lež: Aj, já vydám je v ruku Nabuchodonozora krále Babylonského, aby je zbil pred ocima vašima.
\par 22 I bude vzato na nich klnutí mezi všecky zajaté Judské, kteríž jsou v Babylone, aby ríkali: Necht nakládá Hospodin s tebou, jako s Sedechiášem a jako s Echabem, kteréž upekl král Babylonský na ohni,
\par 23 Proto že páchali nešlechetnost v Izraeli, cizoložíce s ženami bližních svých, a mluvíce slovo ve jménu mém lžive, cehož jsem neprikázal jim. Já pak o tom vím, a jsem toho i svedkem, dí Hospodin.
\par 24 Semaiášovi Nechelamitskému mluv, rka:
\par 25 Takto praví Hospodin zástupu, Buh Izraelský, rka: Proto že jsi poslal jménem svým listy ke všemu lidu, kterýž jest v Jeruzaléme, a Sofoniášovi synu Maaseiášovu knezi i ke všechnem knežím, rka:
\par 26 Hospodin dal te za kneze na místo Joiady kneze, abyste pozor meli v dome Hospodinove na každého muže pošetilého a vystavujícího se za proroka, a abys dal takového do žaláre a do klady.
\par 27 Procež jsi pak nyní neokrikl Jeremiáše Anatotského, kterýž se vám vystavuje za proroka?
\par 28 Nebo posílal k nám do Babylona, rka: Protáhnet se to dlouho, stavejte domy, a osazujte se, štepujte také štepnice, a jezte ovoce jejich.
\par 29 Nebo Sofoniáš knez cetl ten list pred Jeremiášem prorokem.
\par 30 I stalo se slovo Hospodinovo k Jeremiášovi, rkoucí:
\par 31 Pošli ke všechnem zajatým, rka: Takto praví Hospodin o Semaiášovi Nechelamitském: Proto že prorokuje vám Semaiáš, ješto jsem já ho neposlal, a privodí vás k tomu, abyste doufání skládali ve lži,
\par 32 Protož takto praví Hospodin: Aj, já trestati budu Semaiáše Nechelamitského i síme jeho. Nebude míti žádného, kdo by bydlil u prostred lidu tohoto, aniž uzrí toho dobrého, kteréž já uciním lidu svému, dí Hospodin; nebo mluvil to, címž by odvrátil lid od Hospodina.

\chapter{30}

\par 1 Slovo, kteréž se stalo k Jeremiášovi od Hospodina, rkoucí:
\par 2 Takto praví Hospodin, Buh Izraelský, rka: Spiš sobe do knihy všecka slova, kterážt jsem mluvil.
\par 3 Aj, dnové zajisté jdou, dí Hospodin, privedu zase zajaté lidu svého Izraelského i Judského, praví Hospodin, a uvedu je do zeme, kterouž jsem byl dal otcum jejich, a dedicne ji obdrží,
\par 4 Tatot pak jsou slova, kteráž mluvil Hospodin o Izraelovi a Judovi:
\par 5 Takto zajisté praví Hospodin: Hlas predešení a hruzy slyšíme, a že není žádného pokoje.
\par 6 Ptejte se nyní, a vizte rodívá-li samec. Procež tedy vidím, an každý muž rukama svýma drží se za bedra svá jako rodicka, a obrácené všechnech oblíceje v zsinalost?
\par 7 Ach, nebo veliký jest den tento, tak že nebylo žádného jemu podobného. Ale jakt koli cas jest ssoužení Jákobova, predcet z neho vysvobozen bude.
\par 8 Stane se zajisté v ten den, dí Hospodin zástupu, že polámi jho jeho z šíje tvé, a svazky tvé potrhám, i nebudou ho více v službu podrobovati cizozemci.
\par 9 Ale sloužiti budou Hospodinu Bohu svému, a Davidovi králi svému, kteréhož jim vzbudím.
\par 10 Protož ty neboj se, služebníce muj Jákobe, dí Hospodin, aniž se strachuj, ó Izraeli, nebo aj, já vysvobodím te zdaleka, i síme tvé z zeme zajetí jejich. I navrátí se Jákob, aby odpocíval, a pokoj mel, a nebude žádného, kdo by jej predesil.
\par 11 Nebot já s tebou jsem, dí Hospodin, abych te vysvobodil, když uciním konec všechnem národum, mezi kteréž te rozptýlím. Tobe však neuciním konce, ale budu te trestati v soudu, ackoli te bez trestání naprosto nenechám.
\par 12 Takto zajisté praví Hospodin: Pretežké bude potrení tvé, prebolestná rána tvá.
\par 13 Nebude, kdo by prisoudil pri tvou k ulécení; lékarství platného žádného míti nebudeš.
\par 14 Všickni, kteríž te milují, zapomenou se nad tebou, aniž te navštíví, když te raním ranou neprítele, a trestáním prísným pro mnohou nepravost tvou a nescíslné hríchy tvé.
\par 15 Proc kricíš nad svým potrením a težkou bolestí svou? Pro mnohou nepravost tvou a nescíslné hríchy tvé to ciním tobe.
\par 16 A však všickni, kteríž zžírají tebe, sežráni budou, a všickni, kteríž utiskají tebe, všickni, pravím, do zajetí pujdou, a kteríž te pošlapávají, pošlapáni budou, a všecky, kteríž te loupí, v loupež vydám,
\par 17 Tehdáž když tobe navrátím zdraví, a na rány tvé zhojím te, dí Hospodin, proto že zahnanou nazývali tebe, ríkajíce: Tato jest Sion, není žádného, kdo by ji navštívil.
\par 18 Takto praví Hospodin: Aj, já zase privedu zajaté stánku Jákobových, a nad príbytky jeho slituji se, i budet zase vzdeláno mesto na prvním míste svém, a palác podlé zpusobu svého vystaven.
\par 19 A bude pocházeti od nich díku cinení a hlas veselících se; nebo je rozmnožím, a nebudou umenšení bráti, zvelebím je, a nebudou sníženi.
\par 20 A budou synové jeho tak jako i prvé, a shromáždení jeho prede mnou utvrzeno bude, trestati pak budu všecky, kteríž jej ssužují.
\par 21 A povstane z neho nejdustojnejší jeho, a panovník jeho z prostredku jeho vyjde, kterémuž rozkáži priblížiti se, aby predstoupil prede mne. Nebo kdo jest ten, ješto by slíbil za sebe, že predstoupí prede mne? dí Hospodin.
\par 22 I budete mým lidem, a já budu vaším Bohem.
\par 23 Aj, vicher Hospodinuv s prchlivostí vyjde, vicher trvající nad hlavou nešlechetných trvati bude.
\par 24 Neodvrátít se prchlivost hnevu Hospodinova, dokudž neuciní toho, a dokudž nevykoná úmyslu srdce svého. Tehdáž porozumíte tomu.

\chapter{31}

\par 1 Toho casu, dí Hospodin, budu Bohem všech celedí Izraelových, a oni budou mým lidem.
\par 2 Takto praví Hospodin: Lid z tech pozustalých po meci nalezl milost na poušti, když jsem chodil pred ním, abych odpocinutí zpusobil Izraelovi.
\par 3 Za starodávnat se mi ukazoval Hospodin. I však milováním vecným miluji te, procež ustavicne ciním tobe milosrdenství.
\par 4 Ješte vždy vzdelávati te budu, a vzdelána budeš, panno Izraelská; ješte se obveselovati budeš bubny svými, a vycházeti s houfem plésajících.
\par 5 Ješte štepovati budeš vinice v horách Samarských, štepovati budou štepari i jísti.
\par 6 Nebo nastane den, v nemž volati budou strážní na hore Efraimove: Vstante a vstupme na Sion k Hospodinu Bohu svému.
\par 7 Takto zajisté praví Hospodin: Prozpevujte Jákobovi o vecech veselých, prokrikujte zjevne pred temi národy; dejte se slyšeti, chválu vzdávejte, a rcete: Vysvobod, Hospodine, ostatek lidu svého Izraelského.
\par 8 Aj, já privedu je z zeme pulnocní, a shromáždím je ze všech stran zeme, s nimi spolu slepého i kulhavého, tehotnou i rodící; shromáždení veliké sem se navrátí.
\par 9 Kteréžto s plácem jdoucí a s pokornými modlitbami zase privedu, a povedu je podlé tekutých vod cestou prímou, na níž by se nepoklesli; nebot jsem Izraeluv otec, a Efraim jest prvorozený muj.
\par 10 Slyšte slovo Hospodinovo, ó národové, a zvestujte na ostrovích dalekých, a rcete: Ten, kterýž rozptýlil Izraele, shromáždí jej, a ostríhati ho bude jako pastýr stáda svého.
\par 11 Nebot vykoupil Hospodin Jákoba, protož vysvobodí jej z ruky toho, kterýž silnejší jest nad nej.
\par 12 I prijdou, a prozpevovati budou na výsosti Siona, a pohrnou se k dobrote Hospodinove s obilím, a s vínem, a s olejem, a s plodem skotu a bravu, duše pak jejich bude podobná zahrade svlažené, a nebudout se rmoutiti více.
\par 13 Tehdáž veseliti se bude panna s plésáním, a mládenci i starci spolu; obrátím zajisté kvílení jejich v radost, a poteším jich, a obveselím je po zámutku jejich.
\par 14 Rozvlažím i duši kneží tukem, a lid muj dobroty mé nasytí se, dí Hospodin.
\par 15 Takto praví Hospodin: Hlas v Ráma slyšán jest, naríkání a plác prehorký. Ráchel placeci synu svých, nedala se potešiti po synech svých, proto že žádného není.
\par 16 Takto praví Hospodin: Zdrž hlas svuj od pláce, a oci své od slz, nebo budeš míti mzdu za práci svou, praví Hospodin, že se navrátí z zeme neprátelské.
\par 17 Jest, pravím, cáka, žet se potom, dí Hospodin, navrátí synové do svého kraje.
\par 18 V pravde slyším Efraima, an sobe stýšte, prave: Trestals mne, abych strestán byl jako telátko neupráhané; obrat mne, abych obrácen byl, ty jsi zajisté, Hospodine, Buh muj.
\par 19 Nebo po obrácení svém pokání ciniti budu, a když mi k známosti sebe poslouženo bude, uderím se v bedra. Stydímt se, anobrž i pýrím, že snáším útržku detinství svého.
\par 20 Zdali Efraim jest mým synem milým, aneb díte velmi milostné? A však jakž jsem mluvil proti nemu, ustavicne se vždy na nej rozpomínám. Procež pohybují se vnitrnosti mé prícinou jeho; jiste žet se slituji nad ním, dí Hospodin.
\par 21 Nastavej sobe pametných znamení, naklad sobe hromad kamení, pamatuj na tu silnici a cestu, kterouž jsi šla; navrat se, panno Izraelská, navrat se k mestum svým temto.
\par 22 Dokudž se toulati budeš, ó dcero zpurná? Nebot uciní Hospodin novou vec na zemi: Žena bude vukol obcházeti muže.
\par 23 Takto praví Hospodin zástupu, Buh Izraelský: Ještet ríkati budou slovo toto v zemi Judove a v mestech jeho, když zase privedu zajaté jejich: Požehnejž tobe Hospodin, ó príbytku spravedlnosti, horo svatosti.
\par 24 Nebo osazovati se budou v zemi Judské, ve všech mestech jeho spolu oráci, a kteríž zacházejí s stádem.
\par 25 Rozvlažím zajisté duši ustalou, a všelikou duši truchlou nasytím.
\par 26 V tom jsem procítil, a ohlédl se, a ten muj sen byl mi vdecný.
\par 27 Aj, dnové jdou, dí Hospodin, v nichžto oseji dum Izraelský a dum Judský semenem lidským a semenem hovad.
\par 28 I stane se, že jakož jsem se snažoval, abych je plénil, a boril, a kazil a hubil, a trápil, tak se snažím, abych je vzdelal a rozsazoval, dí Hospodin.
\par 29 Tech casu nebudou ríkati více: Otcové jedli hrozen trpký, procež zubové synu laskominy mají,
\par 30 Nýbrž radeji: Jeden každý pro nepravost svou umre. Každého cloveka, kterýž by jedl hrozen trpký, laskominy míti budou zubové jeho.
\par 31 Aj, dnové jdou, dí Hospodin, v nichž uciním s domem Izraelským a s domem Judským smlouvu novou.
\par 32 Ne takovou smlouvu, jakouž jsem ucinil s otci jejich v ten den, v kterýž jsem je ujal za ruku jejich, abych je vyvedl z zeme Egyptské. Kteroužto smlouvu mou oni zrušili, a já abych zustati mel manželem jejich? dí Hospodin.
\par 33 Ale tatot jest smlouva, kterouž uciním s domem Izraelským po techto dnech, dí Hospodin: Dám zákon svuj do vnitrnosti jejich, a na srdci jejich napíši jej; i budu Bohem jejich, a oni budou mým lidem.
\par 34 A nebudou uciti více jeden každý bližního svého, a jeden každý bratra svého, ríkajíce: Poznejte Hospodina. Všickni zajisté naporád znáti mne budou, od nejmenšího z nich až do nejvetšího z nich, dí Hospodin; milostiv zajisté budu nepravosti jejich, a na hrích jejich nezpomenu více.
\par 35 Takto praví Hospodin, kterýž dává slunce za svetlo ve dne, zrízení mesíce a hvezd za svetlo v noci, kterýž rozdeluje more, a zvucí vlnobití jeho, jehož jméno jest Hospodin zástupu:
\par 36 Jestliže se pohnou ta narízení pred oblícejem mým, dí Hospodin, takét síme Izraelovo prestane býti národem pred oblícejem mým po všecky dny.
\par 37 Takto praví Hospodin: Budou-li moci býti zmerena nebesa zhuru, a vyhledáni základové zeme dolu, takét já zavrhu docela síme Izraelovo pro všecko to, což cinili, dí Hospodin.
\par 38 Aj, dnové jdou, dí Hospodin, v nichž vystaveno bude mesto toto Hospodinu, od veže Chananeel až k bráne úhlu.
\par 39 A pujde ješte šnura merící naproti ní, ku pahrbku Gareb, a pritocí se k Gou.
\par 40 A všecko údolí tel mrtvých a popela, i všecko to pole až ku potoku Cedron, až k úhlu brány východní konské posvecené bude Hospodinu; nebudet pléneno, ani kaženo více na veky.

\chapter{32}

\par 1 Slovo, kteréž se stalo k Jeremiášovi od Hospodina léta desátého Sedechiáše krále Judského, kterýžto rok jest osmnáctý Nabuchodonozoruv.
\par 2 (Bylo pak tehdáž vojsko krále Babylonského oblehlo Jeruzalém, a Jeremiáš prorok byl zavrín v síni stráže, kteráž byla v dome krále Judského.
\par 3 Nebo dal jej byl vsaditi Sedechiáš král Judský, rka: Proc ty prorokuješ, práve: Takto praví Hospodin: Aj, já dám mesto toto v ruku krále Babylonského, aby je vzal.
\par 4 Sedechiáš také král Judský neznikne ruky Kaldejských, ale jistotne vydán bude v ruku krále Babylonského, a budou mluviti ústa jeho s ústy jeho, a oci jeho uzrí oci jeho.
\par 5 Nýbrž do Babylona zavede Sedechiáše, aby tam byl, až ho navštívím, dí Hospodin. Ponevadž bojujete s Kaldejskými, nepovede se vám štastne.)
\par 6 Tehdy rekl Jeremiáš: Stalo se slovo Hospodinovo ke mne, rkoucí:
\par 7 Aj, Chanameel syn Salluma, strýce tvého, jde k tobe, atby rekl: Kup sobe dvur muj, kterýž jest v Anatot; nebo tobe právem príbuznosti náleží koupiti jej.
\par 8 Když pak prišel ke mne Chanameel syn strýce mého, podlé slova Hospodinova, do síne stráže, a rekl ke mne: Kup medle dvur muj, kterýž jest v Anatot, jenž jest v zemi Beniaminove; nebo tvuj jest právem dedicným, a právem príbuznosti, kupiž jej sobe: tedy porozumev, že jest to slovo Hospodinovo,
\par 9 I koupil jsem od Chanameele syna strýce svého ten dvur, kterýž jest v Anatot, a odvážil jsem jemu penez sedmnácte lotu stríbra.
\par 10 A zapsav to do cedule, zapecetil jsem, a osvedcil jsem svedky, odváživ peníze na váze.
\par 11 Potom vzal jsem, podlé prikázaní a ustanovení, cedule té koupe zapecetenou i otevrenou,
\par 12 A dal jsem ceduli té koupe Báruchovi synu Neriášovu, synu Maaseiášovu, pred ocima Chanameele strýce svého, a pred ocima svedku, kteríž se podepsali v ceduli té koupe, pred ocima všech Judských, kteríž se byli posadili v síni stráže.
\par 13 A prikázal jsem Báruchovi pred ocima jejich, rka:
\par 14 Takto praví Hospodin zástupu, Buh Izraelský: Vezmi cedule tyto, tuto ceduli té koupe, jakož zapecetenou, tak ceduli otevrenou tuto, a vlož je do nádoby hlinené, aby trvaly za mnohá léta.
\par 15 Nebo takto praví Hospodin zástupu, Buh Izraelský: Ještet kupováni budou domové, a rolí, i vinice v zemi této.
\par 16 Potom modlil jsem se Hospodinu, když jsem dal ceduli té koupe Báruchovi synu Neriášovu, rka:
\par 17 Ach, Panovníce Hospodine, aj, ty jsi ucinil nebe i zemi mocí svou velikou a ramenem svým vztaženým, nemužet skryta býti pred tebou žádná vec.
\par 18 Ciníš milosrdenství nad tisíci, a odplacíš za nepravost otcu do luna synu jejich po nich, Buh veliký, silný, mocný, jehož jméno Hospodin zástupu,
\par 19 Veliký v rade a znamenitý v správe, ponevadž oci tvé otevrené jsou na všecky cesty synu lidských, abys odplatil jednomu každému podlé cest jeho, a podlé ovoce predsevzetí jeho.
\par 20 Kterýž jsi cinil divy a zázraky v zemi Egyptské až do tohoto dne, jakož v Izraelovi tak mezi jinými lidmi, a dobyls sobe jména, jakéž jest po dnešní den.
\par 21 Nebo vyvedl jsi lid svuj Izraelský z zeme Egyptské v divích a v zázracích, a rukou silnou, a ramenem vztaženým, a v strachu velikém,
\par 22 A dals jim zemi tuto, kterouž jsi prisáhl dáti otcum jejich, zemi oplývající mlékem a strdí.
\par 23 Ale že všedše do ní, a dedicne ujavše ji, neposlouchali hlasu tvého, a v zákone tvém nechodili, všeho, což jsi koli prikázal jim ciniti, necinili, protož zpusobils to, aby je potkalo všecko toto zlé.
\par 24 Aj, strelci pritáhli na mesto, aby je vzali, a mesto skrze mec a hlad a mor dáno jest v ruku Kaldejských, bojujících proti nemu; a tak, což jsi koli promluvil, deje se, jakž to sám vidíš.
\par 25 Ty pak pravíš mi, Panovníce Hospodine: Zjednej sobe dvur tento za peníze, a osvedc to svedky, ano již mesto toto dáno jest v ruku Kaldejských.
\par 26 I stalo se slovo Hospodinovo k Jeremiášovi, rkoucí:
\par 27 Aj, já jsem Hospodin Buh všelikého tela, zdaliž prede mnou muže býti skryta která vec?
\par 28 Protož takto praví Hospodin: Aj, já dávám toto mesto v ruku Kaldejských a v ruku Nabuchodonozora krále Babylonského, aby je vzal.
\par 29 A vejdouce Kaldejští, kteríž bojují proti mestu tomuto, zapálí ohnem toto mesto, a vypálí je, i domy ty, na jejichž strechách kadívali Bálovi, a obetovali obeti mokré bohum cizím, aby mne hnevali.
\par 30 Nebo synové Izraelští a synové Judští od detinství svého jen toliko ciní to, což jest zlého pred ocima mýma; synové, pravím, Izraelští jen toliko hnevají mne dílem rukou svých, dí Hospodin.
\par 31 Toto zajisté mesto od toho dne, jakž je vystaveli, až do tohoto dne k hnevu mému a prchlivosti mé popouzí mne, tak že je sklidím od tvári své,
\par 32 Pro všelikou nešlechetnost synu Izraelských a synu Judských, kterouž páchali, aby mne k hnevu popouzeli, oni, králové jejich, knížata jejich, kneží jejich i proroci jejich, jakož muži Judští, tak obyvatelé Jeruzalémští,
\par 33 Obracejíce se ke mne hrbetem, a ne tvárí. A když je ucím, ráno privstávaje, a to ustavicne, však nikoli neposlouchají, aby prijímali naucení.
\par 34 Nadto nastaveli ohavností svých v dome tom, kterýž nazván jest od jména mého, aby jej zanecistili.
\par 35 Nastaveli, pravím, výsostí Bálovi, kteréž jsou v údolí Benhinnom, aby vodili syny své a dcery své Molochovi, ješto jsem jim toho neprikázal, aniž vstoupilo na srdce mé, aby páchati meli tu ohavnost, a Judu k hrešení privodili.
\par 36 A nyní z príciny té takto dí Hospodin. Buh Izraelský, o meste tomto, o kterémž vy ríkáte: Dánot jest v ruku krále Babylonského skrze mec a hlad a mor:
\par 37 Aj, já shromáždím je ze všech zemí, do nichž jsem je rozehnal v hneve svém, a v rozpálení svém, i v prchlivosti veliké, a privedu je zase na toto místo, a zpusobím to, aby bydlili bezpecne.
\par 38 I budou lidem mým, a já budu jejich Bohem.
\par 39 Dám zajisté jim srdce jedno a cestu jednu, aby se mne báli po všecky dny, tak aby jim dobre bylo i synum jejich po nich.
\par 40 A uciním s nimi smlouvu vecnou, že se neodvrátím od nich, abych jim nemel dobre ciniti; nadto bázen svou dám v srdce jejich, aby neodstupovali ode mne.
\par 41 I veseliti se budu z nich, dobre jim cine, když je štípím v zemi této pevne, celým srdcem svým a vší duší svou.
\par 42 Nebo takto praví Hospodin: Jakož jsem uvedl všecko toto zlé veliké na lid tento, tak uvedu na ne všecko to dobré, o nemž jim mluvím.
\par 43 Tehdáž kupováno bude pole v zemi této, o níž vy ríkáte: Pustá jest, tak že není v ní žádného cloveka ani hovada, dánat jest v ruku Kaldejských.
\par 44 Pole za peníze kupovati budou, a zapisovati do cedulí, a zpecetíce, svedky osvedcovati v zemi Beniaminove a vukol Jeruzaléma v mestech Judských, jakož v mestech, kteráž jsou pri horách, tak v mestech na rovinách, a v mestech poledních, když zase privedu zajaté jejich, dí Hospodin.

\chapter{33}

\par 1 Potom stalo se slovo Hospodinovo k Jeremiášovi po druhé, když ješte zavrín byl v síni stráže, rkoucí:
\par 2 Takto praví Hospodin, kterýž uciní to, Hospodin, kterýž sformuje to, potvrdí toho, Hospodin jméno jeho:
\par 3 Volej ke mne, a ohlásímt se, a oznámímt veci veliké a tajné, o nichž nevíš.
\par 4 Nebo takto praví Hospodin, Buh Izraelský, o domích mesta tohoto, a o domích králu Judských, kteríž zkaženi býti mají berany válecnými a mecem:
\par 5 Potáhnout k boji proti Kaldejským, ale aby naplnili tyto domy mrtvými tely lidskými, kteréž zbiji v hneve svém a v prchlivosti své, pro jejichž všelikou nešlechetnost skryl jsem tvár svou od mesta tohoto.
\par 6 Aj, já zopravuji je a vzdelám, a uzdravím obyvatele, a zjevím jim hojnost pokoje, a to stálého.
\par 7 Nebo privedu zase zajaté Judské a zajaté Izraelské, a vzdelám je jako prvé,
\par 8 A ocistím je od všeliké nepravosti jejich, kterouž hrešili proti mne, a odpustím všecky nepravosti jejich, kterýmiž hrešili proti mne, a jimiž zproneverovali se mne.
\par 9 A tot mi bude k jménu, k radosti, k chvále, a k zvelebení mezi všemi národy zeme, kteríž uslyší o všem tom dobrém, kteréž já jim uciním, a desíce se, trásti se budou nade vším tím dobrým a nade vším pokojem tím, kterýž já jim zpusobím.
\par 10 Takto praví Hospodin: Na tomto míste, o kterémž vy ríkáte: Popléneno jest, tak že není ani cloveka ani žádného hovada v mestech Judských a na ulicích Jeruzalémských zpustlých, tak že není žádného cloveka, ani žádného obyvatele, ani žádného hovada,
\par 11 Ještet bude slýchán hlas radosti a hlas veselé, hlas ženicha a hlas nevesty, hlas rkoucích: Oslavujte Hospodina zástupu, nebo dobrý jest Hospodin, nebo na veky milosrdenství jeho, a obetujících díkcinení v dome Hospodinove, když zase privedu zajaté zeme této jako na pocátku, praví Hospodin.
\par 12 Takto praví Hospodin zástupu: Na míste tomto popléneném, tak že není žádného cloveka ani hovada, i ve všech mestech jeho bude ješte obydlé pastýru, kdež by chovali stáda.
\par 13 V mestech pri horách, v mestech na rovinách a v mestech v strane polední, tolikéž v zemi Beniaminove a vukol Jeruzaléma, i v mestech Judských, ješte procházívati budou stáda skrze ruce pocítajícího, praví Hospodin.
\par 14 Aj, dnové jdou, dí Hospodin, v nichž vykonám slovo to výborné, kteréž jsem mluvil o domu Izraelovu a o domu Judovu.
\par 15 V tech dnech a za casu toho zpusobím to, aby zrostl Davidovi výstrelek spravedlivý, kterýž konati bude soud a spravedlnost na zemi.
\par 16 V tech dnech spasen bude Juda, a Jeruzalém bydliti bude bezpecne, a tot jest, což jemu privolá Hospodin, spravedlnost naše.
\par 17 Nebo takto praví Hospodin: Nebudet vyplénen muž z rodu Davidova, ješto by nesedel na stolici domu Judského.
\par 18 Z kneží také Levítských nebude vyplénen muž od tvári mé, ješto by neobetoval zápalu, a zapaloval suchou obet, a obetoval obet po všecky dny.
\par 19 Potom stalo se slovo Hospodinovo k Jeremiášovi, rkoucí:
\par 20 Takto praví Hospodin: Jestliže budete moci zrušiti smlouvu mou se dnem, a smlouvu mou s nocí, aby nebývalo dne ani noci casem svým:
\par 21 Takét smlouva má zrušena bude s Davidem služebníkem mým, aby nemel syna, kterýž by kraloval na stolici jeho, a s Levítskými knežími, aby nebyli služebníky mými.
\par 22 A jakož nemuže secteno býti vojsko nebeské, ani zmeren býti písek morský, tak rozmnožím síme Davida služebníka svého, a Levítu mne prisluhujících.
\par 23 Opet stalo se slovo Hospodinovo k Jeremiášovi, rkoucí:
\par 24 Což nesoudíš, co lid tento mluví, ríkaje: Že dvojí celed, kterouž byl vyvolil Hospodin, již ji zavrhl, a lidem mým že pohrdají, jako by nebyl více národem pred oblícejem jejich.
\par 25 Takto praví Hospodin: Nebude-lit smlouva má se dnem a nocí, a ustanovení nebes i zeme zdržáno,
\par 26 Také síme Jákobovo a Davida služebníka svého zavrhu, abych nebral z semene jeho tech, kteríž by panovati meli nad semenem Abrahamovým, Izákovým a Jákobovým, když zase privedu zajaté jejich, a smiluji se nad nimi.

\chapter{34}

\par 1 Slovo,kteréž se stalo k Jeremiášovi od Hospodina, (když Nabuchodonozor král Babylonský, a všecko vojsko jeho, i všecka království zeme pod moc jeho prináležející, všickni také národové bojovali proti Jeruzalému a proti všechnem mestum jeho), rkoucí:
\par 2 Takto praví Hospodin Buh Izraelský: Jdi a rci Sedechiášovi králi Judskému, rci, pravím, jemu: Takto dí Hospodin: Aj, já dám toto mesto v ruku krále Babylonského, aby je vypálil ohnem.
\par 3 I ty neznikneš ruky jeho, ale jistotne jat a v ruku jeho vydán budeš, a oci tvé uzrí oci krále Babylonského, i ústa jeho s ústy tvými mluviti budou, a do Babylona se dostaneš.
\par 4 A však slyš slovo Hospodinovo, Sedechiáši králi Judský: Takto praví Hospodin o tobe: Neumreš od mece,
\par 5 V pokoji umreš. A jakož pálívali otcum tvým, králum predešlým, kteríž byli pred tebou, tak páliti budou tobe, a ríkajíce: Ach, pane, kvíliti budou nad tebou; nebo slovo toto já mluvil jsem, dí Hospodin.
\par 6 I mluvil Jeremiáš prorok Sedechiášovi, králi Judskému, všecka slova ta v Jeruzaléme,
\par 7 Když vojsko krále Babylonského bojovalo proti Jeruzalému a proti všechnem mestum Judským ostatním, proti Lachis a proti Azeku; nebo ta byla pozustala z mest Judských mesta hrazená.
\par 8 Slovo, kteréž se stalo k Jeremiášovi od Hospodina, když ucinil král Sedechiáš smlouvu se vším lidem, kterýž byl v Jeruzaléme, o vyhlášení jim svobody,
\par 9 Aby propustil jeden každý služebníka svého a jeden každý devku svou, Hebrejského neb Hebrejskou, svobodné, aby nepodroboval sobe v službu Žida, bratra svého, nižádný.
\par 10 Tedy uposlechla všecka knížata a všecken lid, kterýž byl v smlouvu všel, aby propustil jeden každý služebníka svého a jeden každý devku svou svobodné, aby nepodroboval jich v službu více; uposlechli, pravím, a propustili.
\par 11 Potom pak rozmyslivše se, zase pobrali služebníky své a devky své, kteréž byli propustili svobodné, a podrobili je sobe za služebníky a devky.
\par 12 I stalo se slovo Hospodinovo k Jeremiášovi od Hospodina, rkoucí:
\par 13 Takto praví Hospodin Buh Izraelský: Já jsem ucinil smlouvu s otci vašimi toho dne, když jsem je vyvedl z zeme Egyptské, z domu služebníku, rka:
\par 14 Po vyplnení sedmi let propouštívejte jeden každý bratra svého Žida, kterýž by prodán byl tobe, a sloužilt by šest let, propust, pravím, jej svobodného od sebe. Ale neuposlechli otcové vaši mne, aniž naklonili ucha svého.
\par 15 Vy zajisté usmyslivše sobe dnes, ucinili jste to, což jest spravedlivého pred ocima mýma, že jste vyhlásili svobodu jeden každý bližnímu svému, ucinivše smlouvu pred oblícejem mým v dome tom, kterýž nazván jest od jména mého.
\par 16 Ale zase zpácivše se, zlehcili jste jméno mé, že jste vzali zase jeden každý služebníka svého a jeden každý devku svou, kteréž jste byli propustili svobodné podlé žádosti jejich, a podrobili jste je, aby byli vaši služebníci a devky.
\par 17 Protož takto praví Hospodin: Vy neuposlechli jste mne, abyste vyhlásili svobodu jeden každý bratru svému, a jeden každý bližnímu svému, aj, já vyhlašuji proti vám svobodu, dí Hospodin, meci, moru a hladu, a vydám vás ku posmýkání po všech královstvích zeme.
\par 18 Vydám zajisté ty lidi, jenž prestoupili smlouvu mou, kteríž nevykonali slov smlouvy té, kterouž ucinili pred oblícejem mým, když tele roztali na dvé, a prošli mezi díly jeho,
\par 19 Totiž knížata Judská a knížata Jeruzalémská, komorníci, a kneží, i všecken lid té zeme, kteríž prošli mezi díly toho telete.
\par 20 Vydám je, pravím, v ruku neprátel jejich, a v ruku hledajících bezživotí jejich, i budou tela mrtvá jejich za pokrm ptactvu nebeskému, a šelmám zemským.
\par 21 Sedechiáše také krále Judského, i knížata jeho vydám v ruku neprátel jejich, a v ruku hledajících bezživotí jejich, v ruku, pravím, vojska krále Babylonského, kteríž odtáhli od vás.
\par 22 Aj, já prikáži, dí Hospodin, a privedu je zase na mesto toto, aby bojovali proti nemu, a vezmouce je, vypálili je ohnem; mesta také Judská obrátím v poušt, tak že nebude žádného obyvatele.

\chapter{35}

\par 1 Slovo, kteréž se stalo k Jeremiášovi od Hospodina za dnu Joakima syna Joziášova, krále Judského, rkoucí:
\par 2 Jdi mezi Rechabitské, a promluve s nimi, doved je k domu Hospodinovu do jednoho z pokoju, a dej jim píti vína.
\par 3 Tedy pojav Jazaniáše syna Jeremiášova, syna Chabaciniášova, a bratrí jeho i všecky syny jeho se vší rodinou Rechabitských,
\par 4 Dovedl jsem je k domu Hospodinovu do pokoje synu Chanana syna Igdaliášova, muže Božího, kterýž byl pri pokoji knížat, jenž byl nad pokojem Maaseiáše syna Sallumova, ostríhajícího prahu.
\par 5 Potom postave pred syny domu Rechabitských koflíky plné vína a cíše, i rekl jsem jim: Píte víno.
\par 6 Kteríž rekli: Nepíjíme vína. Nebo Jonadab syn Rechabuv, otec náš, zapovedel nám, rka: Nepíjejte vína, vy, ani synové vaši na veky.
\par 7 A domu nestavejte, ani semene nerozsívejte, vinice také neštepujte, ani mívejte, ale v staních prebývejte po všecky dny vaše, abyste živi byli mnoho dnu na tvári zeme, v níž pohostinu jste.
\par 8 Protož uposlechli jsme hlasu Jonadaba syna Rechabova, otce našeho, ve všem, což prikázal nám, abychom nepili vína po všecky dny své, my, manželky naše, synové naši, i dcery naše,
\par 9 Abychom nestaveli domu k bydlení svému, a vinice, ani rolí, ani nic osátého nemívali.
\par 10 Ale abychom bydlili v staních. Uposlechli jsme, pravím, a deláme všecko, jakž prikázal nám Jonadab otec náš.
\par 11 Stalo se pak, když vtrhl Nabuchodonozor král Babylonský do zeme, že jsme rekli: Podte, a ujdeme do Jeruzaléma pred vojskem Kaldejským, a pred vojskem Syrským. Takž bydlíme v Jeruzaléme.
\par 12 Tedy stalo se slovo Hospodinovo k Jeremiášovi, rkoucí:
\par 13 Takto praví Hospodin zástupu, Buh Izraelský: Jdi a rci mužum Judským a obyvatelum Jeruzalémským: Což neprijmete naucení, abyste poslouchali slov mých? dí Hospodin.
\par 14 K vykonání prichází všeliké slovo Jonadaba syna Rechabova, kterýž prikázal synum svým, aby nepili vína. Nepili zajisté až do tohoto dne, nebo poslouchají prikázaní otce svého. Já pak mluvím k vám, ráno privstávaje, a to ustavicne, a však neposloucháte mne.
\par 15 Nadto posílám k vám všecky služebníky své proroky, ráno privstávaje, a to ustavicne, ríkaje: Navratte se již jeden každý z cesty své zlé, a polepšte predsevzetí svých, a nechodte za bohy cizími, sloužíce jim, a tak prebývejte v zemi této, kterouž jsem dal vám i otcum vašim: však nenaklonujete uší svých, aniž mne posloucháte.
\par 16 Ješto synové Jonadabovi syna Rechabova plní prikázaní otce svého, kteréž prikázal jim, lid pak tento neposlouchají mne.
\par 17 Protož takto praví Hospodin Buh zástupu, Buh Izraelský: Aj, já uvedu na Judu a na všecky obyvatele Jeruzalémské všecko to zlé, kteréž jsem vyrkl proti nim, proto že jsem mluvíval k nim, a neposlouchali, a volával jsem na ne, ale neohlásili se.
\par 18 Rodine pak Rechabitských rekl Jeremiáš: Takto praví Hospodin zástupu, Buh Izraelský: Protože posloucháte prikázaní Jonadaba otce vašeho, a ostríháte všech prikázaní jeho, anobrž deláte všecko, jakž prikázal vám,
\par 19 Protož takto praví Hospodin zástupu, Buh Izraelský: Nebudet vyplénen muž z rodu Jonadabova syna Rechabova, ješto by nestál pred oblícejem mým po všecky dny.

\chapter{36}

\par 1 Stalo se pak léta ctvrtého Joakima syna Joziášova, krále Judského, stalo se slovo toto k Jeremiášovi od Hospodina, rkoucí:
\par 2 Vezmi sobe knihu, a zapiš do ní všecka slova, kteráž jsem mluvil tobe proti Izraelovi, a proti Judovi, i všechnem národum od toho dne, jakž jsem mluvíval s tebou, ode dnu Joziášových, až do dne tohoto,
\par 3 Zdali aspon uslyšíce dum Judský o všem tom zlém, kteréž já myslím uciniti jim, navrátí se jeden každý z cesty své zlé, abych odpustil nepravost jejich a hrích jejich.
\par 4 Protož zavolal Jeremiáš Bárucha syna Neriášova, i sepsal Báruch do knihy z úst Jeremiášových všecka slova Hospodinova, kteráž mluvil jemu.
\par 5 Potom prikázal Jeremiáš Báruchovi, rka: Já zápoved maje, nemohu jíti do domu Hospodinova.
\par 6 Protož jdi ty, a cti v knize této, což jsi napsal z úst mých, slova Hospodinova, pri prítomnosti lidu v dome Hospodinove, v den postu, též také pri prítomnosti všech Judských, kteríž by se z mest svých sešli, cti je,
\par 7 Zda by snad ponížene a pokorne modléce se pred Hospodinem, i navrátili by se jeden každý z cesty své zlé; nebot jest veliký hnev a prchlivost, v níž mluvil Hospodin proti lidu tomuto.
\par 8 I ucinil Báruch syn Neriášuv všecko, jakž prikázal jemu Jeremiáš prorok, cta v knize té slova Hospodinova v dome Hospodinove.
\par 9 Stalo se pak léta pátého za Joakima syna Joziášova, krále Judského, mesíce devátého, že vyhlásili pust pred Hospodinem všemu lidu v Jeruzaléme, a všemu lidu, kteríž se byli sešli z mest Judských do Jeruzaléma.
\par 10 I cetl Báruch z knihy slova Jeremiášova v dome Hospodinove, v pokoji Gemariáše syna Safanova, písare, na síni horejší, u dverí brány domu Hospodinova nové, pri prítomnosti všeho lidu.
\par 11 Když pak vyslyšel Micheáš syn Gemariášuv, syna Safanova, všecka slova Hospodinova z té knihy,
\par 12 Hned šel do domu královského, do pokoje kanclérova, a aj, tam všecka knížata sedela: Elisama ten kanclér, a Delaiáš syn Semaiášuv, a Elnatan syn Achboruv, a Gemariáš syn Safanuv, a Sedechiáš syn Chananiášuv, i všecka knížata.
\par 13 A oznámil jim Micheáš všecka slova, kteráž slyšel, když cetl Báruch v knize pri prítomnosti lidu.
\par 14 Protož poslala všecka knížata k Báruchovi Judu syna Netaniášova, syna Selemiášova, syna Chuzova, aby rekl: Tu knihu, v níž jsi cetl pri prítomnosti lidu, vezmi do ruky své a pod. I vzal Báruch syn Neriášuv tu knihu do ruky své, a prišel k nim.
\par 15 Kteríž rekli jemu: Sed medle a cti ji pred námi. I cetl Báruch pred nimi.
\par 16 Stalo se pak, když uslyšeli všecka slova ta, že predešeni byli všickni, a rekli Báruchovi: Jistotne oznámíme králi všecka slova tato.
\par 17 Potom otázali se Bárucha, rkouce: Oznam již nám, jak jsi všecka slova ta sepsal z úst jeho?
\par 18 Jimž rekl Báruch: Z úst svých pravil mi všecka slova tato, a já psal jsem na knihách cernidlem.
\par 19 Tedy rekla ta knížata Báruchovi: Jdi schovej se ty i Jeremiáš, at žádný neví, kde jste.
\par 20 Potom vešli k králi na sín, když tu knihu schovati dali v pokoji Elisama kanclére, a oznámili králi všecka slova ta.
\par 21 I poslal král Judu, aby vzal tu knihu. Kterýžto vzal ji z pokoje Elisama kanclére, a cetl ji Juda pri prítomnosti krále, a pri prítomnosti všech knížat stojících pred králem.
\par 22 Král pak sedel v dome, v nemž v zime býval, mesíce devátého, a na ohništi pred ním horelo.
\par 23 Tedy stalo se, jakž jen prectl Juda tri listy neb ctyri, že porezal ji škriptorálem, a házel na ohen, kterýž byl na ohništi, až shorela všecka kniha ta ohnem tím, kterýž byl na ohništi.
\par 24 Ale neulekli se, aniž roztrhl roucha svá král a všickni služebníci jeho, kteríž slyšeli všecka slova ta.
\par 25 Nýbrž ješte když Elnatan a Delaiáš a Gemariáš primlouvali se k králi, aby nepálil té knihy, tedy neuposlechl jich.
\par 26 Ale prikázal král Jerachmeelovi synu královu, a Saraiášovi synu Azrielovu, a Selemiášovi synu Abdeelovu, aby jali Bárucha písare a Jeremiáše proroka. Ale skryl je Hospodin.
\par 27 Stalo se pak slovo Hospodinovo k Jeremiášovi, když spálil král tu knihu a slova, kteráž byl sepsal Báruch z úst Jeremiášových, rkoucí:
\par 28 Vezmi sobe zase knihu jinou, a napiš na ní všecka slova první, kteráž byla v té knize prvnejší, kterouž spálil Joakim král Judský.
\par 29 O Joakimovi pak králi Judském rci: Takto praví Hospodin: Ty jsi spálil knihu tuto, prave: Proc jsi psal v ní, rka: Jistotne pritáhne král Babylonský, a zkazí zemi tuto, a vyhladí z ní lidi i hovada?
\par 30 Protož takto praví Hospodin o Joakimovi králi Judském: Nebude míti, kdo by sedel na stolici Davidove, a telo jeho mrtvé bude vyvrženo na horko ve dne a na mráz v noci.
\par 31 Nebo trestati budu na nem, a na semeni jeho, i na služebnících jeho nepravost jejich, a uvedu na ne, a na obyvatele Jeruzalémské, i na muže Judské všecko to zlé, o kterémž jsem mluvíval jim, a neposlouchali.
\par 32 I vzal Jeremiáš knihu jinou, a dal ji Báruchovi synu Neriášovu, písari, kterýž sepsal do ní z úst Jeremiášových všecka slova té knihy, kterouž byl spálil Joakim král Judský ohnem. A ješte pridáno jest k tem slovum mnoho tem podobných.

\chapter{37}

\par 1 Potom kraloval král Sedechiáš syn Joziášuv místo Koniáše syna Joakimova, kteréhož ustanovil králem Nabuchodonozor král Babylonský v zemi Judské.
\par 2 Ale neposlouchal on, ani služebníci jeho, ani lid té zeme slov Hospodinových, kteráž mluvil skrze Jeremiáše proroka.
\par 3 Ac byl poslal Sedechiáš Jehuchale syna Selemiášova a Sofoniáše syna Maaseiášova, kneze, k Jeremiášovi proroku, aby rekli: Modl se medle za nás Hospodinu Bohu našemu.
\par 4 Nebo Jeremiáš ješte bydlil svobodne u prostred lidu, aniž ho ješte byli dali do žaláre.
\par 5 A vojsko Faraonovo bylo vytáhlo z Egypta; (nebo uslyšavše Kaldejští, kteríž oblehli byli Jeruzalém, povest o nich, odtrhli od Jeruzaléma).
\par 6 Stalo se pak slovo Hospodinovo k Jeremiášovi proroku, rkoucí:
\par 7 Toto praví Hospodin Buh Izraelský: Takto rcete králi Judskému, kterýž vás poslal ke mne, abyste se radili se mnou: Aj, vojsko Faraonovo, kteréž potáhne vám na pomoc, navrátí se zase do zeme své Egyptské.
\par 8 Kaldejští pak navrátí se zase, a bojovati budou proti mestu tomuto, a vezmouce je, vypálí je ohnem.
\par 9 Takto praví Hospodin: Nesvodte sami sebe, ríkajíce: Konecne odtrhnou od nás Kaldejští, nebot neodtrhnou.
\par 10 Nýbrž byste pobili všecko vojsko Kaldejských bojujících s vámi, tak že by pozustali z nich toliko ranení, tit z stanu svých povstanou, a toto mesto ohnem vypálí.
\par 11 Stalo se pak, když odtrhlo vojsko Kaldejské od Jeruzaléma pred vojskem Faraonovým,
\par 12 Že vycházel Jeremiáš z Jeruzaléma, jíti chteje do zeme Beniaminovy, aby tak vynikl z prostredku lidu.
\par 13 Když pak byl v bráne Beniaminské, byl tu hejtman nad stráží, jménem Jiriáš syn Selemiáše, syna Chananiášova, kterýž jal Jeremiáše proroka, rka: K Kaldejským ty ustupuješ.
\par 14 Jemuž rekl Jeremiáš: Není pravda, neustupujit já k Kaldejským. Ale nechtel ho slyšeti, nýbrž jal Jiriáš Jeremiáše, a dovedl jej k knížatum.
\par 15 Tedy rozhnevavše se knížata na Jeremiáše, ubili jej, a dali jej do vezení, do domu Jonatana písare; nebo z neho byli udelali žalár.
\par 16 Když pak vešel Jeremiáš do té jámy a do sklípku jejích, a sedel tam Jeremiáš mnoho dní.
\par 17 Teprv poslav král Sedechiáš, vzal jej, a tázal se ho král v dome svém tajne, rka: Stalo-li se slovo od Hospodina? Jemuž rekl Jeremiáš: Stalo. Potom rekl: V ruku krále Babylonského vydán budeš.
\par 18 Pri tom rekl Jeremiáš králi Sedechiášovi: Cot jsem zavinil aneb služebníkum tvým, aneb lidu tomuto, že jste mne dali do žaláre tohoto?
\par 19 A kdež jsou proroci vaši, kteríž prorokují vám, ríkajíce: Nepritáhnet král Babylonský na vás, ani na zemi tuto?
\par 20 Nyní tedy slyš, žádám, pane muj králi, necht, prosím, místo má pred tebou pokorná prosba má; nedopouštej mne zase voditi do domu Jonatana písare, abych tam neumrel.
\par 21 I prikázal král Sedechiáš, aby vsadili Jeremiáše do síne stráže, a dávali jemu pecník chleba na den z ulice pekaru, dokudž by nebyl vytráven všecken chléb v meste. A tak sedel Jeremiáš v síni stráže.

\chapter{38}

\par 1 Slyšel pak Sefatiáš syn Matanuv, a Gedaliáš syn Paschuruv, a Juchal syn Selemiášuv, a Paschur syn Malkiášuv slova, kteráž Jeremiáš mluvil ke všemu lidu, rka:
\par 2 Takto praví Hospodin: Kdo by zustal v meste tomto, zahyne mecem, hladem aneb morem, ale kdož by vyšel k Kaldejským, že bude živ, a že bude míti život svuj místo koristi, a živ zustane.
\par 3 Takto praví Hospodin: Jistotne vydáno bude mesto toto v ruku vojska krále Babylonského, a vezme je.
\par 4 Protož rekla ta knížata králi: Necht jest usmrcen muž ten, ponevadž zemdlívá ruce mužu bojovných, pozustalých v meste tomto, i ruce všeho lidu, mluve jim slova taková; nebo muž ten nikoli neobmýšlí pokoje lidu tomuto, ale zlé.
\par 5 Tedy rekl král Sedechiáš: Aj, v ruce vaší jest, nebot král zhola nic nemuže proti vám.
\par 6 I vzali Jeremiáše, kterýž byl v síni stráže, a uvrhli jej do jámy Malkiášovy, syna králova, a spustili Jeremiáše po provazích. V té pak jáme nebylo nic vody, ale bláto, tak že Jeremiáš tonul v tom bláte.
\par 7 Ale jakž uslyšel Ebedmelech Mourenín, dvoran, kterýž byl v dome královském, že dali Jeremiáše do té jámy, (král pak sedel v bráne Beniaminské),
\par 8 Hned vyšel Ebedmelech z domu královského, a mluvil s králem, rka:
\par 9 Pane muj, králi, zle ucinili muži tito všecko, což ucinili Jeremiášovi proroku, že jej uvrhli do té jámy; nebot by byl umrel i na prvním míste hladem, ponevadž již není žádného chleba v meste.
\par 10 Protož porucil král Ebedmelechovi Mourenínu, rka: Vezmi s sebou odsud tridceti mužu, a vytáhni Jeremiáše proroka z té jámy, prvé než by umrel.
\par 11 Tedy vzal Ebedmelech ty muže s sebou, a všel do domu královského pod pokladnici, a nabral starých hadru strhaných, hadru, pravím, zkažených, kteréž spustil k Jeremiášovi do té jámy po provazích.
\par 12 A rekl Ebedmelech Mourenín Jeremiášovi: Nu, podlož ty staré, strhané hadry a zkažené pod paže rukou svých s provazy. I ucinil tak Jeremiáš.
\par 13 Takž vytáhli Jeremiáše po provazích, a dobyli jej z té jámy. I sedel Jeremiáš v síni stráže.
\par 14 Potom poslav král Sedechiáš, vzal Jeremiáše proroka k sobe do tretího pruchodu, kterýž byl pri domu Hospodinovu, a rekl král Jeremiášovi: Zeptám se tebe na neco, netaj prede mnou nicehož.
\par 15 I rekl Jeremiáš Sedechiášovi: Oznámím-lit, zdaliž mne konecne neusmrtíš? A poradím-lit, neuposlechneš mne.
\par 16 Tedy prisáhl král Sedechiáš Jeremiášovi tajne, rka: Živt jest Hospodin, kterýž ucinil nám život tento, že te neusmrtím, aniž te vydám v ruku mužu tech, kteríž hledají bezživotí tvého.
\par 17 I rekl Jeremiáš Sedechiášovi: Takto praví Hospodin Buh zástupu, Buh Izraelský: Jestliže dobrovolne vyjdeš k knížatum krále Babylonského, i duše tvá živa bude, i mesto toto nebude vypáleno ohnem, a tak živ zustaneš ty i dum tvuj.
\par 18 Jestliže pak nevyjdeš k knížatum krále Babylonského, jiste že vydáno bude mesto toto v ruku Kaldejských, a vypálí je ohnem, ano i ty neznikneš ruky jejich.
\par 19 Tedy rekl král Sedechiáš Jeremiášovi: Velmi se bojím Židu, kteríž ustoupili k Kaldejským, aby mne snad nevydali v ruku jejich, i ucinili by sobe ze mne posmech.
\par 20 Ale Jeremiáš rekl: Nevydadí. Uposlechni, prosím, hlasu Hospodinova, o kterémž já mluvím tobe, a bude dobre tobe, i duše tvá živa bude.
\par 21 Jestliže pak nebudeš chtíti vyjíti, toto jest slovo to, kteréž mi ukázal Hospodin,
\par 22 Že aj, všecky ženy, kteréž pozustaly v dome krále Judského, privedeny budou knížatum krále Babylonského, a samyt ríkati budou: Nabádalit jsou te, a obdrželi na tobe ti, kteríž te troštovali pokojem; uvázlyt v bahne nohy tvé, a nazpet obráceny.
\par 23 Všecky také manželky tvé i syny tvé dovedou k Kaldejským, i ty sám neznikneš ruky jejich, ale rukou krále Babylonského jat budeš, a mesto toto vypálíš ohnem,
\par 24 Tedy rekl Sedechiáš Jeremiášovi: Žádný at neví o vecech techto, abys neumrel.
\par 25 Pakli uslyšíce knížata, že jsem mluvil s tebou, prišli by k tobe, a reklit by: Oznam medle nám, cos mluvil s králem, netaj pred námi, a neusmrtíme te, a co mluvil s tebou král?
\par 26 Tedy rci jim: Predkládal jsem poníženou a pokornou prosbu svou pred krále, aby mne nedal zase voditi do domu Jonatanova, abych tam neumrel.
\par 27 I sešla se všecka knížata k Jeremiášovi, aby se ho tázali. Kterýžto oznámil jim podlé toho všeho, jakž prikázal král. Takž mlckem odešli od neho, když nebylo slyšeti o té veci.
\par 28 Jeremiáš pak sedel v síni stráže až do toho dne, v nemž dobyt jest Jeruzalém, kdežto byl, když dobýván byl Jeruzalém.

\chapter{39}

\par 1 Léta devátého Sedechiáše krále Judského, mesíce desátého, pritáhl Nabuchodonozor král Babylonský se vším vojskem svým k Jeruzalému, a oblehli jej.
\par 2 Jedenáctého pak léta Sedechiášova, mesíce ctvrtého, devátého dne mesíce, prolomeno jest mesto.
\par 3 I vešla tam všecka knížata krále Babylonského, a posadila se v bráne prostrední: Nergalšaretser, Samgarnebu, Sarsechim, Rabsaris, Nergalšaretser, Rabmag, a všecka jiná knížata krále Babylonského.
\par 4 Stalo se pak, když je uzrel Sedechiáš král Judský, a že všickni muži bojovní utekli, a vyšli v noci z mesta cestou zahrady královské skrze bránu mezi dvema zdmi, ušel také cestou poušte.
\par 5 I honilo je vojsko Kaldejské, a postihli Sedechiáše na rovinách Jerišských, a jali jej a privedli ho k Nabuchodonozorovi králi Babylonskému do Ribla, do zeme Emat. Kdežto ucinil soud.
\par 6 Nebo zmordoval král Babylonský syny Sedechiášovy v Ribla pred ocima jeho, i všecky nejprednejší Judské zmordoval král Babylonský.
\par 7 Oci pak Sedechiášovy oslepil, a svázav ho retezy ocelivými, dovedl jej do Babylona.
\par 8 Dum také královský i domy toho lidu vypálili Kaldejští ohnem, a zdi Jeruzalémské poborili.
\par 9 Ostatek pak lidu, kterýž byl zustal v meste, i pobehlce, kteríž byli ustoupili k nemu, a jiný lid pozustalý zavedl Nebuzardan, hejtman nad žoldnéri, do Babylona.
\par 10 Toliko nejchaternejších z lidu, kteríž nikdež nemeli nic, zanechal Nebuzardan, hejtman nad žoldnéri, v zemi Judské, jimž rozdal vinice a rolí v ten den.
\par 11 Prikázal pak Nabuchodonozor král Babylonský o Jeremiášovi Nebuzardanovi, hejtmanu nad žoldnéri, rka:
\par 12 Vezmi jej, a pilne ho opatruj, aniž jemu cin co zlého, ale jakž tobe rekne, tak nalož s ním.
\par 13 Protož poslav Nebuzardan, hejtman nad žoldnéri, a Nebušazban, Rabsaris, a Nergalšaretser, Rabmag, a všickni hejtmané krále Babylonského,
\par 14 Poslavše, pravím, vzali Jeremiáše z síne stráže, a dali jej Godoliášovi synu Achikamovu, syna Safanova, aby jej domu dovedl. Takž bydlil u prostred lidu.
\par 15 Stalo se pak k Jeremiášovi slovo Hospodinovo, když ješte byl zavrín v síni stráže, rkoucí:
\par 16 Jdi a rci Ebedmelechovi Mourenínu, rka: Takto praví Hospodin zástupu, Buh Izraelský: Aj, já zpusobím to, aby došla slova má na mesto toto ke zlému a ne k dobrému, a splní se pred oblícejem tvým v ten den.
\par 17 Ale vysvobodím te v ten den, dí Hospodin, aniž budeš vydán v ruku mužu tech, jejichž oblíceje se lekáš.
\par 18 Nebo jistotne vychvátím te, abys od mece nepadl, a budeš míti život svuj místo koristi, proto že jsi nadeji složil ve mne, dí Hospodin.

\chapter{40}

\par 1 Slovo to, kteréž se stalo k Jeremiášovi od Hospodina, když jej propustil Nebuzardan, hejtman nad žoldnéri, z Ráma, vzav jej, když byl svázaný retezy u prostred všech zajatých Jeruzalémských a Judských, zajatých do Babylona.
\par 2 Tedy vzal hejtman nad žoldnéri Jeremiáše, a rekl jemu: Hospodin Buh tvuj byl vyrkl zlé toto proti místu tomuto.
\par 3 Protož je uvedl a ucinil Hospodin, jakž mluvil; nebo jste hrešili proti Hospodinu, a neposlouchali jste hlasu jeho, procež stala se vám vec tato.
\par 4 Již tedy, aj, rozvazuji te dnes z retezu tech, kteríž jsou na rukou tvých. Vidí-lit se za dobré jíti se mnou do Babylona, pod, a budu te pilne opatrovati; jestližet se pak nevidí za dobré jíti se mnou do Babylona, nech tak. Aj, všecka tato zeme jest pred tebou; kamžt se za dobré a slušné vidí jíti, tam jdi.
\par 5 Pakli, (ponevadž se nenavracuje), obrat se k Godoliášovi synu Achikamovu, syna Safanova, kteréhož ustanovil král Babylonský nad mesty Judskými, a zustan s ním u prostred lidu, aneb kamžt se koli dobre líbí jíti, jdi. I dal jemu hejtman nad žoldnéri na cestu, a dar, a propustil jej.
\par 6 Takž prišel Jeremiáš k Godoliášovi synu Achikamovu do Masfa, a bydlil s ním u prostred lidu pozustalého v zemi.
\par 7 Uslyšeli pak všickni hejtmané vojsk, kteríž byli na poli, oni i všecken lid jejich, že ustanovil král Babylonský Godoliáše syna Achikamova nad tou zemí, a že jemu porucil muže a ženy i deti, a to z nejchaternejších té zeme, z tech kteríž nebyli zajati do Babylona.
\par 8 Protož prišli k Godoliášovi do Masfa, totiž Izmael syn Netaniášuv, též Jochanan a Jonatan, synové Kareachovi, a Saraiáš syn Tanchumetuv, a synové Efai Netofatského, a Jazaniáš syn Machatuv, oni i lid jejich.
\par 9 Tedy prisáhl jim Godoliáš syn Achikamuv, syna Safanova, i lidu jejich, rka: Nebojte se služby Kaldejských, zustante v zemi, a služte králi Babylonskému, a dobre vám bude.
\par 10 Nebo aj, já bydlím v Masfa, abych sloužil Kaldejským, kteríž pricházejí k nám, vy pak slízejte víno a letní ovoce i olej, a skládejte do nádob svých, a zustante v mestech svých, kteráž držíte.
\par 11 Tak i všickni Judští, kteríž byli u Moábských a mezi Ammonitskými, a mezi Idumejskými, a kteríž byli ve všech zemích, uslyšavše, že by pozustavil král Babylonský ostatek Judských, a že by ustanovil nad nimi Godoliáše syna Achikamova, syna Safanova,
\par 12 Navrátili se všickni Judští ze všech míst, kamž rozehnaní byli, a prišli do zeme Judské k Godoliášovi do Masfa. I nazbírali vína a letního ovoce velmi mnoho.
\par 13 Jochanan pak syn Kareachuv, a všecka knížata vojsk, kteráž byla v poli, prišli k Godoliášovi do Masfa,
\par 14 A rekli jemu: Víš-li co o tom, že Baalis král Ammonitský poslal Izmaele syna Netaniášova, aby te zabil? Ale neveril jim Godoliáš syn Achikamuv.
\par 15 Nadto Jochanan syn Kareachuv rekl Godoliášovi tajne v Masfa, rka: Necht jdu medle, a zabiji Izmaele syna Netaniášova, však žádný nezví. Proc má zabiti tebe, a mají rozptýleni býti všickni Judští, kteríž shromáždeni jsou k tobe, a zahynouti ostatek Judských?
\par 16 Ale rekl Godoliáš syn Achikamuv Jochananovi synu Kareachovu: Necin toho, nebo lež ty mluvíš o Izmaelovi.

\chapter{41}

\par 1 Stalo se pak mesíce sedmého, že prišel Izmael syn Netaniášuv, syna Elisamova z semene královského, a hejtmané královští, totiž deset mužu s ním, k Godoliášovi synu Achikamovu do Masfa, a jedli tam chléb spolu v Masfa.
\par 2 Potom vstav Izmael syn Netaniášuv, a deset mužu, kteríž byli s ním, zabili Godoliáše syna Achikamova, syna Safanova, mecem; zamordoval, pravím, toho, kteréhož ustanovil král Babylonský nad tou zemí.
\par 3 Všecky také Židy, kteríž byli s ním, s Godoliášem, v Masfa, i ty Kaldejské, kteríž postiženi byli tam, muže bojovné, pobil Izmael.
\par 4 Stalo se pak druhého dne, jakž zamordoval Godoliáše, (o cemž žádný nezvedel),
\par 5 Že prišli nekterí z Sichem, z Sílo a z Samarí, mužu osmdesáte, oholivše bradu, a roztrhše roucha, a režíce se, kterížto suchou obet a kadidlo v rukou svých meli, aby je nesli do domu Hospodinova.
\par 6 Tedy Izmael syn Netaniášuv vyšel jim vstríc z Masfa, ustavicne jda a place. Stalo se pak, když je potkal, že rekl jim: Podte k Godoliášovi synu Achikamovu.
\par 7 Ale stalo se, když prišli do prostred mesta, že pobiv je Izmael syn Netaniášuv, vmetal je do prostred jámy, on i muži, kteríž s ním byli.
\par 8 Deset pak mužu nalezlo se mezi nimi, kteríž rekli Izmaelovi: Nemorduj nás; nebo máme sklady skryté v poli, pšenice a jecmene, též oleje a medu. I nechal tak, a nemordoval jich u prostred bratrí jejich.
\par 9 Jáma pak, do níž Izmael vmetal k Godoliášovi všecka tela mužu tech, kteréž pobil, ta jest, kterouž udelal král Aza, boje se Bázy krále Izraelského. Kterouž naplnil Izmael syn Netaniášuv zbitými.
\par 10 A zajal Izmael všecken ostatek lidu, kteríž byli v Masfa, dcery královské a všecken lid, kteríž byli pozustali v Masfa, nad nimiž byl ustanovil Nebuzardan, hejtman nad žoldnéri, Godoliáše syna Achikamova. Kteréžto zajav Izmael syn Netaniášuv, šel, aby se dopravil k Ammonitským.
\par 11 V tom uslyšel Jochanan syn Kareachuv, i všickni hejtmané tech vojsk, kteríž s ním byli, o všem tom zlém, kteréž ucinil Izmael syn Netaniášuv.
\par 12 I vzali všecken svuj lid, a táhli k boji proti Izmaelovi synu Netaniášovu, jehož nalezli pri vodách velikých, kteréž jsou v Gabaon.
\par 13 Stalo se pak, že když uzrel všecken lid, kterýž byl s Izmaelem, Jochanana syna Kareachova a všecka knížata vojsk, kteríž s ním byli, zradovali se.
\par 14 A obrátiv se všecken ten lid, kterýž byl zajal Izmael z Masfa, navrátil se zase, a prišel k Jochananovi synu Kareachovu.
\par 15 Izmael pak syn Netaniášuv ušel pred Jochananem s osmi muži, a prišel k Ammonitským.
\par 16 Protož vzal Jochanan syn Kareachuv, a všecka knížata vojsk, kteríž s ním byli, všecken ostatek lidu, kterýž zase privedl od Izmaele syna Netaniášova z Masfa, když zabil Godoliáše syna Achikamova, muže bojovné, též ženy a deti i komorníky, kteréž zase privedl z Gabaon,
\par 17 A odšedše, pobyli v hospode Chimhamove, kteráž jest u Betléma, aby jdouce, vešli do Egypta,
\par 18 Pred Kaldejskými. Nebo báli se jich, proto že zabil Izmael syn Netaniášuv Godoláše syna Achikamova, kteréhož byl ustanovil král Babylonský nad tou zemí.

\chapter{42}

\par 1 Potom pristoupili všecka knížata vojsk, i Jochanan syn Kareachuv i Jazaniáš syn Hosaiášuv, i všecken lid, od nejmenšího až do nejvetšího,
\par 2 A rekli Jeremiášovi proroku: Uslyš medle nás v ponížené prosbe naší, a modl se za nás Hospodinu Bohu svému, za všecken ostatek tento, nebo nás malicko zustalo z mnohých, jakž oci tvé nás vidí,
\par 3 At nám oznámí Hospodin Buh tvuj cestu, po níž bychom jíti, a co ciniti meli.
\par 4 Jimž rekl Jeremiáš prorok: Uslyšel jsem. Aj, já modliti se budu Hospodinu Bohu vašemu podlé slov vašich, a cožkoli vám odpoví Hospodin, oznámím vám; nezatajím pred vámi slova.
\par 5 Oni zase rekli Jeremiášovi: Necht jest Hospodin mezi námi svedkem pravým a verným, jestliže podlé každého slova, pro než poslal te Hospodin Buh tvuj k nám, tak se chovati nebudeme.
\par 6 Bud dobré aneb zlé, hlasu Hospodina Boha našeho, pro nejž te vysíláme k nemu, uposlechneme, aby nám dobre bylo, když uposlechneme hlasu Hospodina Boha našeho.
\par 7 Stalo se pak po prebehnutí desíti dnu, když se stalo slovo Hospodinovo k Jeremiášovi,
\par 8 Že povolal Jochanana syna Kareachova, a všech knížat vojsk, kteríž s ním byli, i všeho lidu, od nejmenšího až do nejvetšího,
\par 9 A rekl jim: Takto praví Hospodin Buh Izraelský, k nemuž jste mne poslali, abych rozprostíral poníženou prosbu vaši pred oblícejem jeho:
\par 10 Jestliže navrátíce se, zustanete v zemi této, zajisté že vzdelám vás, a nezkazím, anobrž vštípím vás, a nevypléním; nebot lituji toho zlého, kteréž jsem ucinil vám.
\par 11 Nebojtež se krále Babylonského, jehož se bojíte, nebojte se ho, dí Hospodin; nebot s vámi jsem, abych vás vysvobozoval, a vytrhoval vás z ruky jeho.
\par 12 Nadto zpusobím vám milost, aby se slitoval nad vámi, a dal se vám navrátiti do zeme vaší.
\par 13 Ale reknete-li: Nezustaneme v zemi této, neposlouchajíce hlasu Hospodina boha svého,
\par 14 A ríkajíce: Nikoli, ale do zeme Egyptské vejdeme, kdež neuzríme boje, ani zvuku trouby neuslyšíme, a chleba lacneti nebudeme, procež tam se osadíme:
\par 15 Protož nyní slyštež slovo Hospodinovo, ostatkové Judští: Takto praví Hospodin zástupu, Buh Izraelský: Jestliže vy zarputile na tom zustanete, abyste vešli do Egypta, a vejdete-li, abyste tam pobyli,
\par 16 Jiste stane se to, že mec, kteréhož se bojíte, tam v zemi Egyptské vás postihne, a hlad, jehož se obáváte, prijde na vás v Egypte, a tam pomrete.
\par 17 Tak se stane všechnem tem mužum, kteríž uložili predce jíti do Egypta, aby tam byli pohostinu, že zhynou mecem, hladem a morem, a nezustane z nich žádného, aniž kdo znikne toho zlého, kteréž já uvedu na ne.
\par 18 Nebo takto praví Hospodin zástupu, Buh Izraelský: Jakož vylit jest hnev muj a prchlivost má na obyvatele Jeruzalémské, tak vylita bude prchlivost má na vás, když vejdete do Egypta, a budete k proklínání, a k užasnutí, a k zlorecení, a za útržku; nadto neuzríte více místa tohoto.
\par 19 K vámt mluví Hospodin, ó ostatkové Judští: Nevcházejte do Egypta. Jistotne vezte, (nebot se vám dnes osvedcuji),
\par 20 Ponevadž jste se neupríme ke mne meli v myšleních svých, poslavše mne k Hospodinu Bohu vašemu, rkouce: Modl se za nás Hospodinu Bohu našemu, a všecko, jakžt koli dí Hospodin Buh náš, tak nám oznam, a uciníme,
\par 21 Když pak oznamuji vám dnes, však neposloucháte hlasu Hospodina Boha vašeho hned v nicemž, procež mne k vám poslal:
\par 22 Protož pravím, vezte jistotne, že mecem, hladem a morem pomrete v tom míste, kamž se vám zachtelo jíti, abyste tam byli pohostinu.

\chapter{43}

\par 1 Stalo se pak, když prestal Jeremiáš mluviti ke všemu lidu všech slov Hospodina Boha jejich, s nimiž poslal jej Hospodin Buh jejich k nim, všech, pravím, slov tech,
\par 2 Že rekl Azariáš syn Hosaiášuv, a Jochanan syn Kareachuv, i všickni muži ti pyšní, rkouce Jeremiášovi: Lež ty mluvíš, neposlalt tebe Hospodin Buh náš, abys rekl: Nevcházejte do Egypta, abyste tam byli pohostinu.
\par 3 Ale Báruch syn Neriášuv ponouká te proti nám, aby vydal nás v ruku Kaldejských k usmrcení nás, aneb zajetí nás do Babylona.
\par 4 I neuposlechl Jochanan syn Kareachuv, i všecka knížata vojsk, též i všecken lid hlasu Hospodinova, aby zustal v zemi Judské.
\par 5 Ale Jochanan syn Kareachuv, a všecka knížata vojsk vzali všecken ostatek Judských, kteríž se byli navrátili ze všech národu, kamž byli rozehnáni, k obývání v zemi Judské,
\par 6 Muže i ženy, též dítky a dcery královské, i všelikou duši, kterýchž Nebuzardan, hejtman nad žoldnéri, s Godoliášem synem Achikamovým, syna Safanova, zanechal, i s Jeremiášem prorokem a Báruchem synem Neriášovým,
\par 7 A šli do zeme Egyptské, (nebo neuposlechli hlasu Hospodinova), a prišli až do Tachpanches.
\par 8 Stalo se pak slovo Hospodinovo k Jeremiášovi v Tachpanches, rkoucí:
\par 9 Naber do ruky své kamení velikého, a schovej je v hline v cihelni, kteráž jest u brány domu Faraonova v Tachpanches, pred ocima mužu Judských.
\par 10 A rci jim: Takto praví Hospodin zástupu, Buh Izraelský: Aj, já pošli, a vezmu Nabuchodonozora krále babylonského, služebníka svého, a postavím stolici jeho na tomto kamení, kteréž jsem schoval, i rozbije majestát svuj na nem.
\par 11 Nebo pritáhne a popléní zemi Egyptskou: Kteríž k smrti, k smrti, a kteríž k zajetí, do zajetí, a kteríž pod mec,
\par 12 A zanítím ohen v domích bohu Egyptských, aby je popálil, a zajal je. A priodeje se zemí Egyptskou, tak jako se odívá pastýr rouchem svým, a vyjde odtud s pokojem,
\par 13 Když poláme sochy Betsemes, kteréž jest v zemi Egyptské, a domy bohu Egyptských popálí ohnem.

\chapter{44}

\par 1 Slovo, kteréž se stalo k Jeremiášovi proti všechnem Judským bydlícím v zemi Egyptské, kteríž bydlili v Magdol a v Tachpanches, a v Nof, a v zemi Patros, rkoucí:
\par 2 Takto praví Hospodin zástupu, Buh Izraelský: Vy jste videli všecko to zlé, kteréž jsem uvedl na Jeruzalém a na všecka mesta Judská, že aj, pustá jsou podnes, tak že v nich není žádného obyvatele,
\par 3 Pro nešlechetnost jejich, kterouž páchali, aby jen popouzeli mne, chodíce kaditi a sloužiti bohum cizím, jichž neznali sami, vy, ani otcové vaši,
\par 4 Ješto posílal jsem k vám všecky služebníky své proroky, ráno privstávaje, a to ustavicne, ríkaje: Necinte medle této veci ohavné, kteréž nenávidím.
\par 5 Ale neuposlechli, aniž naklonili ucha svého, aby se navrátili od nešlechetnosti své, a aby nekadili bohum cizím.
\par 6 Protož vylita jest prchlivost má i hnev muj, a rozpálil se v mestech Judských i po ulicích Jeruzalémských, tak že obrácena jsou v poušt a v pustinu, jakž videti v dnešní den.
\par 7 Nyní pak takto praví Hospodin Buh zástupu, Buh Izraelský: Proc vy ciníte zlé preveliké proti dušem svým, abyste vyplénili z sebe muže i ženu, díte i prsí požívajícího z prostredku Judy, abyste nepozustavili sobe ostatku,
\par 8 Popouzejíce mne dílem ruku svých, kadíce bohum cizím v zemi Egyptské, do níž jste vešli, abyste tam byli pohostinu, abyste vyplénili sebe, a byli k zlorecení a za útržku u všech národu zeme?
\par 9 Zdali jste zapomenuli na nešlechetnosti otcu svých, a na nešlechetnosti králu Judských, a na nešlechetnosti manželek jejich, a na nešlechetnosti své, i na nešlechetnosti manželek svých, kteréž páchali v zemi Judské a po ulicích Jeruzalémských?
\par 10 Nekorili se až do tohoto dne, aniž se báli, aniž chodili v zákone mém a v ustanoveních mých, kteráž predkládám vám jako i otcum vašim.
\par 11 Protož takto praví Hospodin zástupu, Buh Izraelský: Aj, já obracím tvár svou proti vám k zlému, totiž abych vyplénil všecky Judské.
\par 12 Zhubím zajisté ostatek Judských, kteríž svévolne vešli do zeme Egyptské, aby tam byli pohostinu, tak že zhynou docela všickni v zemi Egyptské. Padnou od mece, hladem docela zhynou, od nejmenšího až do nejvetšího, mecem a hladem pomrou; nadto budou k proklínání a k užasnutí, a k zlorecení a za útržku.
\par 13 Nebo navštívím ty, kteríž bydlí v zemi Egyptské, jako jsem navštívil Jeruzalém mecem, hladem a morem,
\par 14 Tak že nebude, kdo by ušel, neb pozustal z ostatku Judských, kteríž prišli do zeme Egyptské, aby tam byli pohostinu, aby se zase navrátiti mohli do zeme Judské. Címž se však troštují, že se zase navrátí, aby prebývali tam, ale nenavrátít se, než ti, kteríž sami ujdou.
\par 15 Tedy odpovedeli Jeremiášovi všickni ti muži, vedevše, že kadívaly manželky jejich bohum cizím, ty všecky ženy, jichž stál veliký zástup, i všecken lid prebývající v zemi Egyptské v Patros, rkouce:
\par 16 V té veci, o kteréž jsi nám mluvil ve jménu Hospodinovu, neuposlechneme tebe.
\par 17 Ale dosti ciniti chceme každému slovu, kteréž by pošlo z úst našich, kadíce tvoru nebeskému, a obetujíce jemu obeti mokré, jakž jsme cinívali my i otcové naši, králové naši i knížata naše po mestech Judských a po ulicích Jeruzalémských; nebo nasyceni jsme bývali chlebem, a bývali jsme veseli, zlého pak neokoušeli jsme.
\par 18 Ale jakž jsme prestali kaditi tvoru nebeskému, a obetovati jemu obeti mokré, nedostatek trpíme ve všem, a mecem i hladem hyneme.
\par 19 Že pak kadíme tvoru nebeskému, a obetujeme jemu obeti mokré, zdaliž bez znamenitých mužu našich peceme jemu koláce, službu jemu konajíce, a obetujíce jemu obeti mokré?
\par 20 Tedy rekl Jeremiáš všemu lidu, mužum i ženám a všemu lidu, kteríž jemu to odpovedeli, rka:
\par 21 Zdaliž na kadidlo, jímž jste kadívali po mestech Judských a po ulicích Jeruzalémských vy i otcové vaši, králové vaši i knížata vaše i lid zeme, nerozpomenul se Hospodin, a zdaž jím to nepohnulo,
\par 22 Tak že nemohl Hospodin více snášeti nešlechetnosti predsevzetí vašich a ohavností, kteréž jste cinili? Procež obrácena jest zeme vaše v poušt a v pustinu a v zlorecenství, tak že není v ní žádného obyvatele, jakž dnešní den jest.
\par 23 Proto že jste kadívali, a že jste hrešili proti Hospodinu, a neposlouchali jste hlasu Hospodinova, a tak v zákone jeho a v ustanoveních jeho ani v svedectvích jeho nechodili jste, protož potkalo vás zlé toto, jakž videti dnešní den.
\par 24 Nadto rekl Jeremiáš všemu tomu lidu i všechnem tem ženám: Slyšte slovo Hospodinovo všickni Judští, kteríž jste v zemi Egyptské:
\par 25 Takto praví Hospodin zástupu, Buh Izraelský, rka: Vy i ženy vaše ústy svými jste mluvili, a rukami doplnili, ríkajíce: Konecnet budeme plniti sliby své, kteréž jsme slíbili, kadíce tvoru nebeskému, a obetujíce jemu obeti mokré, a tak vší snažností vyplnujete sliby vaše, a všelijak slibum vašim dosti ciníte.
\par 26 Protož slyšte slovo Hospodinovo všickni Judští, kteríž bydlíte v zemi Egyptské: Aj, já prisahám skrze jméno své veliké, praví Hospodin, že nebude více vzýváno jméno mé ústy žádného muže Judského po vší zemi Egyptské, kterýž by rekl: Živt jest Panovník Hospodin.
\par 27 Aj, já bdíti budu nad nimi k zlému a ne k dobrému, i budou pléneni všickni muži Judští, kteríž jsou v zemi Egyptské, mecem a hladem, dokudž by docela nezhynuli.
\par 28 Kteríž pak ujdou mece, navrátí se z zeme Egyptské do zeme Judské, lidu malý pocet. I zvedít všickni ostatkové Judští, kteríž prišli do zeme Egyptské, aby tam byli pohostinu, cí slovo ostojí, mé-li cili jejich.
\par 29 A toto mejte za znamení, dí Hospodin, že já trestati budu vás v míste tomto, abyste vedeli, že jistotne ostojí slova má proti vám k zlému.
\par 30 Takto praví Hospodin: Aj, já vydám Faraona Chofra krále Egyptského v ruku neprátel jeho a v ruku hledajících bezživotí jeho, jako jsem vydal Sedechiáše krále Judského v ruku Nabuchodonozora krále Babylonského, neprítele jeho, a hledajícího bezživotí jeho.

\chapter{45}

\par 1 Slovo, kteréž promluvil Jeremiáš prorok k Báruchovi synu Neriášovu, když spisoval slova tato do knihy z úst Jeremiášových, léta ctvrtého Joakima syna Joziášova, krále Judského, rka:
\par 2 Takto praví Hospodin Buh Izraelský o tobe, ó Báruchu.
\par 3 Rekls: Jižte mi beda, nebo priciní Hospodin zámutku k bolesti mé. Ustávám v úpení svém, a odpocinutí nenalézám.
\par 4 Takto rci jemu: Takto praví Hospodin: Aj, což jsem vystavel, já borím, a což jsem vštípil, já pléním, totiž všecku zemi tuto.
\par 5 A ty bys hledal sobe velikých vecí? Nehledej. Nebo aj, já uvedu bídu na všeliké telo, dí Hospodin, ale dám tobe život tvuj místo koristi na všech místech, kamž pujdeš.

\chapter{46}

\par 1 Slovo Hospodinovo, kteréž se stalo k Jeremiášovi proroku proti národum temto:
\par 2 O Egyptských. Proti vojsku Faraona Néchy krále Egyptského, (jenž bylo pri rece Eufrates u Charkemis, kteréž zbil Nabuchodonozor král Babylonský), léta ctvrtého Joakima syna Joziášova, krále Judského:
\par 3 Pripravtež štít a pavézu, a jdete k boji.
\par 4 Zapráhejte kone, a vsedejte jezdci, postavte se s lebkami, vytírejte kopí, zoblácejte se v pancíre.
\par 5 Proc vidím tyto zdešené, zpet obrácené, a nejsilnejší jejich potrené, anobrž prudce utíkající, tak že se ani neohlédnou? Strach jest vukol, dí Hospodin,
\par 6 Aby neutekl cerstvý, a neušel silný, aby na pulnoci o breh reky Eufrates zavadili a padli.
\par 7 Kdo se to valí jako potok, jako reky, jejichž vody sebou zmítají?
\par 8 Egypt jako potok se valí a jako reky, když se vody hýbají, tak prave: Potáhnu, prikryji zemi, zkazím mesto i prebývající v nem.
\par 9 Krepctež, ó koni, a bežte s hrmotem, ó vozové; necht vytáhnou i silní, Mourenínové a Putští, kteríž užívají pavézy, i Ludimští, kteríž užívají a natahují lucište.
\par 10 A však ten den Panovníka Hospodina zástupu bude den pomsty, aby uvedl pomstu na neprátely své, kteréžto zžíre mec a nasytí se, anobrž opojí se krví jejich; nebo obet Panovníka Hospodina zástupu bude v zemi pulnocní u reky Eufrates.
\par 11 Vyprav se do Galád, a naber masti, panno dcero Egyptská, nadarmo však užíváš mnohého lékarství, nebo ty zhojena býti nemužeš.
\par 12 Slyšít národové o zlehcení tvém, a naríkání tvého plná jest zeme, proto že silný na silném se ustrkuje, tak že spolu zaroven padají.
\par 13 Slovo, kteréž mluvil Hospodin k Jeremiášovi proroku o pritažení Nabuchodonozora krále Babylonského, aby zkazil zemi Egyptskou.
\par 14 Oznamte v Egypte, a ohlaste v Magdol, rozhlaste také v Nof, i v Tachpanches rcete: Postuj a priprav se, ale však zžíre mec všecko, což vukol tebe jest.
\par 15 Proc poražen jest každý z nejsilnejších tvých? Nemuže ostáti, proto že Hospodin pudí jej.
\par 16 Mnohot bude tech, kteríž klesnou a padnou jeden na druhého, a reknou: Vstan, a navratme se k lidu svému, a do zeme, v níž jsme se zrodili, pred mecem toho zhoubce.
\par 17 Budou kriceti tam: Faraona krále Egyptského není než sám chrest, jižt pominul volný jeho cas.
\par 18 Živt jsem já, (dí král ten, jehož jméno Hospodin zástupu), že jakž jest Tábor mezi horami, a jako Karmel pri mori, tak toto prijde.
\par 19 Pripraviž sobe to, co bys s sebou vystehovati mela, ó ty, kteráž jsi usadila se, dcero Egyptská; nebot Nof v poušt obrácen a vypálen bude, tak že nebude tam žádného obyvatele.
\par 20 Velmi pekná jalovice jest Egypt, ale zabití její od pulnoci jistotne prijde.
\par 21 Ano i nájemníci jeho u prostred neho jsou jako telata vytylá, ale takét i ti obrátíce se, utekou. Neostojí, když den bídy jejich prijde na ne, v cas navštívení jejich.
\par 22 Hlas jeho jako hadí sipteti bude, když s vojskem pritáhnou, a s sekerami prijdou na nej jako ti, kteríž sekají dríví.
\par 23 Posekají les jeho, dí Hospodin, ackoli mu konce není; více jest jich než kobylek, aniž mají poctu.
\par 24 Zahanbenat bude dcera Egyptská, vydána bude v ruku lidu pulnocního.
\par 25 Praví Hospodin zástupu, Buh Izraelský: Aj, já navštívím lidné mesto No, též Faraona i Egypt s bohy jeho a králíky jeho, Faraona, pravím, a ty, kteríž v nem doufají.
\par 26 A vydám je v ruku tech, kteríž hledají bezživotí jejich, totiž v ruku Nabuchodonozora krále Babylonského, a v ruku služebníku jeho; potom však bydleno bude v nem, jako za dnu starodávních, dí Hospodin.
\par 27 Ale ty neboj se, služebníce muj Jákobe, aniž se strachuj, ó Izraeli; nebo aj, já vysvobodím te zdaleka, i síme tvé z zeme zajetí jejich. I navrátí se Jákob, aby odpocíval a pokoj mel, a aby nebylo žádného, kdo by predesil.
\par 28 Ty, pravím, neboj se, služebníce muj Jákobe, dí Hospodin; nebot jsem s tebou. Uciním zajisté konec všechnem národum, mezi kteréž te zaženu, tobe pak neuciním konce, ale budu te trestati v soudu, ackoli te naprosto bez trestání nenechám.

\chapter{47}

\par 1 Slovo Hospodinovo, kteréž se stalo k Jeremiášovi proroku proti Filistinským, prvé než dobyl Farao Gázy.
\par 2 Takto praví Hospodin: Aj, vody vystupují od pulnoci, a obrátí se v potok rozvodnilý, tak že zatopí zemi, i cožkoli jest na ní, mesto i ty, kteríž bydlí v nem; procež kriceti budou lidé, a kvíliti všeliký obyvatel té zeme;
\par 3 Pro zvuk dusání kopyt silných jeho, pro hrmot vozu jeho, a hrcení kol jeho, neohlédnou se otcové na syny, majíce opuštené ruce;
\par 4 Pro ten den, kterýž prijíti má, aby pohubil všecky Filistinské, aby zahladil Týr a Sidon, i všelikou pozustávající pomoc, když hubiti bude Hospodin Filistinské, ostatek krajiny Kaftor.
\par 5 Prijde lysina na Gázu, vyplénen bude Aškalon i ostatek údolí jejich. Dokudž se rezati budeš?
\par 6 Ach, meci Hospodinuv, dokudž se nespokojíš? Navrat se do pošvy své, utiš se, a zastav se.
\par 7 I jakž by se spokojil? Však Hospodin prikázal jemu. Proti Aškalon a proti brehu morskému, tam postavil jej.

\chapter{48}

\par 1 Proti Moábovi. Takto praví Hospodin zástupu, Buh Izraelský: Beda mestu Nébo, nebot popléneno bude; zahanbeno a vzato bude Kariataim, zahanbeno bude Misgab a desiti se bude.
\par 2 Nebudet míti více žádné pochvaly Moáb z Ezebon, obmýšlejí proti nemu zlé: Podte, a vyhladme je z národu. I ty, ó Madmen, vypléneno budeš, pujde za tebou mec.
\par 3 Hlas žalostný z Choronaim: Ó poplénení a potrení veliké.
\par 4 Potrín bude Moáb, slyšán bude krik malických jeho.
\par 5 Proto že na ceste Luchitské ustavicný bude plác, a že kudyž se chodí k Choronaim, neprátelé krik hrozný slyšeti budou:
\par 6 Utecte, vysvobodte život svuj, a budte jako vres na poušti.
\par 7 Nebo proto, že doufáš v statku svém a v pokladích svých, také ty jat budeš; i pujde Chámos do zajetí, kneží jeho, též i knížata jeho.
\par 8 Pritáhne zajisté zhoubce na každé mesto, aniž ho které mesto znikne; zahyne i údolí, a zahlazena bude rovina, jakž praví Hospodin.
\par 9 Dejte brky Moábovi, at rychle uletí; nebo mesta jeho v poušt obrácena budou, tak že nebude žádného obyvatele v nich.
\par 10 Zlorecený ten, kdož delá dílo Hospodinovo lstive, a zlorecený ten, kdož zdržuje mec svuj od krve.
\par 11 Melt jest pokoj Moáb od detinství svého, a usadil se na kvasnicích svých, aniž býval prelíván z nádoby do nádoby, totiž do zajetí nechodíval; procež zustala v nem chut jeho, a vune jeho není promenena.
\par 12 Protož aj, dnové jdou, dí Hospodin, že pošli na nej ty, kteríž vpády ciní, a zajmou jej, sudy pak jeho vyprázdní, a nádoby jeho roztríští.
\par 13 I zahanben bude Moáb od Chámos, jako zahanbeni jsou dum Izraelský od Bethel nadeje své.
\par 14 Kterakž ríkáte: Silní jsme a muži statecní k boji?
\par 15 Pohuben bude Moáb a z mest svých vyjde, a nejvýbornejší mládenci jeho pujdou k zabití, dí král, jehož jméno Hospodin zástupu.
\par 16 Blízkot jest bída Moábova, aby prišla, a trápení jeho pospíchá velice.
\par 17 Litujte ho všickni okolní jeho, a všickni, kteríž víte o jménu jeho, rcete: Jak polámána jest hul nejpevnejší, prut nejozdobnejší?
\par 18 Sejdi z slávy své, a sed v žíni, ó ty, kterážs se usadila, dcero Dibonská; nebot zhoubce Moábuv pritáhne proti tobe, zkazí ohrady tvé.
\par 19 Postav se na ceste, a ohlédni se pilne, obyvatelkyne Aroer; vyptej se toho, kdož utíká, a té, kteráž uchází, rci: Co se deje?
\par 20 Stydí se Moáb, že jest potrín. Kvelte a kricte, oznamte u Arnon, že hubí Moába.
\par 21 Nebo soud prišel na zemi té roviny, na Holon a na Jasa a na Mefat,
\par 22 I na Dibon a Nébo, a na Betdiblataim,
\par 23 I na Kariataim a na Betgamul, a na Betmeon,
\par 24 A na Kariot, a na Bozru i na všecka mesta zeme Moábské, daleká i blízká.
\par 25 Odtat bude roh Moábuv, a ráme jeho zlámáno bude, dí Hospodin.
\par 26 Opojte jej, ponevadž se proti Hospodinu zpínal, až by se válel Moáb v vývratku svém, a byl za smích i on také.
\par 27 Nebo zdali v posmechu nebyl u tebe Izrael? Zdali mezi zlodeji zastižen jest, že jakž jen promluvíš o nem, poskakuješ?
\par 28 Zanechte mest, a prebývejte v skále, ó obyvatelé Moábští, a budte podobni holubici hnízdící se daleko v rozsedlinách.
\par 29 Slýchalit jsme o pýše Moábove, že velmi pyšný jest, o vysokomyslnosti jeho a pýše jeho, i o nadutosti jeho, a povyšování se srdce jeho.
\par 30 Známt já, dí Hospodin, vzteklost jeho, ale špatnát jest ona k tomu; lži jeho nedovedout toho.
\par 31 Protož nad Moábskými kvílím, a nade vším Moábem kricím, pro obyvatele Kircheres úpí srdce.
\par 32 Více než plakáno bylo Jazerských, placi tebe, ó vinný kmene Sibma. Rozvodové tvoji dostanout se za more, až k mori Jazerskému dosáhnou; na letní ovoce tvé a na vinobraní tvé zhoubce pripadne.
\par 33 I prestane veselé a plésání nad polem úrodným v zemi Moábské, a vínu z presu prítrž uciním; nebudou ho tlacívati s prokrikováním. Prokrikování nebudet prokrikováním.
\par 34 Více kriceti budou než Ezebonští, až do Eleale, až do Jasa vydávati budou hlas svuj, od Ségor až do Choronaim jako jalovice tríletá; nebo také i vody Nimrim vymizejí.
\par 35 A tak uciním, dí Hospodin, prítrž Moábovi, obetujícímu na výsosti a kadícímu bohum jeho.
\par 36 Protož srdce mé nad Moábem jako píštalky zníti bude, srdce mé i nad obyvateli Kircheres jako píštalky zníti bude, proto že i zboží nashromáždená se v nic obrátí.
\par 37 Nebo na každé hlave bude lysina, a každá brada oholena, na všech rukou rezání, a na bedrách žíne.
\par 38 Na všech strechách Moábových i po ulicích jeho všudy jen kvílení; nebo rozrazím Moába jako nádobu neužitecnou, dí Hospodin.
\par 39 Kvíliti budou: Jakt jest potrín, jak se obrátil zpet Moáb s hanbou! A jest Moáb v posmechu a k predešení všechnem, kteríž jsou vukol neho.
\par 40 Nebo takto praví Hospodin: Aj, jako orlice priletí a roztáhne krídla svá na Moába.
\par 41 Vzato bude Kariot, i pevnosti pobrány budou, a bude srdce silných Moábských v ten den podobné srdci ženy svírající se.
\par 42 I vyhlazen bude Moáb z lidu, proto že se proti Hospodinu zpínal.
\par 43 Strach a jáma a osídlo nad tebou, ó obyvateli Moábský, praví Hospodin.
\par 44 Kdo utece pred strachem, vpadne do jámy, a kdo vyleze z jámy, osídlem lapen bude; nebo uvedu na nej, totiž na Moába, rok navštívení jejich, dí Hospodin.
\par 45 Stávalit jsou v stínu Ezebon ti, kteríž utíkávali pred násilím, ale ohen vyjde z Ezebon, a plamen z prostred Seon, a zžíre kout Moábuv a vrch hlavy tech, kteríž jen bourí.
\par 46 Beda tobe, Moábe, zahynet lid Chámosuv; nebo pobráni budou synové tvoji do zajetí, i dcery tvé do zajetí.
\par 47 A však zase privedu zajaté Moábské v posledních dnech, dí Hospodin. Až potud soud o Moábovi.

\chapter{49}

\par 1 Proti Ammonitským. Takto praví Hospodin: Což nemá žádných synu Izrael? Což dedice žádného nemá? Proc dedicne opanoval král jejich Gádu, a lid jeho v mestech tohoto bydlí?
\par 2 Protož aj, dnové jdou, dí Hospodin, že zpusobím, aby slyšáno bylo proti Rabba Ammonitských troubení válecné, a aby bylo obráceno v hromadu rumu, a jiná mesta jeho ohnem vypálena. I opanuje Izrael ty, kteríž jej byli opanovali, praví Hospodin.
\par 3 Kvel Ezebon, když popléneno bude Hai, kricte, ó dcery Rabba, prepašte se žínemi, placte, a bežte pres ploty; nebo král váš do zajetí pujde, kneží jeho i knížata jeho spolu.
\par 4 Což se chlubíš údolími, když oplývá údolí tvé, ó dcero zpurná, kteráž doufáš v pokladech svých, ríkajíc: Kdo by táhl na mne?
\par 5 Aj, já uvedu na tebe strach, praví Panovník Hospodin zástupu, ze všeho vukolí tvého, jímž rozehnáni budete všickni, a nebude žádného, kdo by shromáždil toulající se.
\par 6 A však potom zase privedu zajaté Ammonitské, dí Hospodin.
\par 7 Proti Idumejským. Takto praví Hospodin zástupu: Což není více moudrosti v Teman? Zahynula rada od rozumných? Zmarena moudrost jejich?
\par 8 Utíkejte, obrate se, a hluboko se schovejte, obyvatelé Dedan; nebo bídu uvedu na Ezau v cas navštívení jeho.
\par 9 Kdyby ti, kteríž zbírají víno, prišli na tebe, zdaž by nepozustavili paberku? Pakli zlodeji v noci, zdaž by škodili více nad potrebu svou?
\par 10 Ale já obnažím Ezau, zodkrývám skrýše jeho, tak že se nebude moci ukryti. Pohubenot bude síme jeho i bratrí jeho i sousedé jeho, tak že nebude naprosto, kdo by rekl:
\par 11 Zanech sirotku svých, já živiti je budu, a vdov svých mne se doverte.
\par 12 Takto zajisté praví Hospodin: Aj, ti, kteríž nemají žádného práva píti kalichu tohoto, predce pijí, ty pak sám abys toho naprosto prázen byl? Nebudeš prázen, ale jistotne píti budeš.
\par 13 Nebo skrze sebe prisahám, dí Hospodin, že pustinou, útržkou, pouští a prokletím bude Bozra, a všecka mesta jeho budou pouští vecnou.
\par 14 Povest slyšel jsem od Hospodina, že posel k národum poslán jest: Shromaždte se, a táhnete proti nemu; nuže, vstante k boji.
\par 15 Nebo aj, zpusobím to, abys byl za nejšpatnejšího mezi národy, v nevážnosti mezi lidmi.
\par 16 To, že jsi hrozný, zklamá te, i pýcha srdce tvého, ó ty, kterýž bydlíš v rozsedlinách skalních, kterýž se držíš vysokých pahrbku. Bys pak vysoko udelal hnízdo své jako orlice, i odtud te strhnu, dí Hospodin.
\par 17 I bude zeme Idumejská pustinou. Každý, kdož pujde skrze ni, užasne se, a diviti se bude nade všemi ranami jejími;
\par 18 Jako podvrácení Sodomy a Gomory a sousedu jejich, praví Hospodin. Neosadí se tam žádný, aniž bydliti bude v ní syn cloveka.
\par 19 Aj, jako lev vystupuje i více než zdutí Jordána proti príbytku Nejsilnejšího, a však v okamžení zaženu jej z této zeme, a toho, kterýž jest vyvolený, ustanovím nad ní. Nebo kdo jest mne rovný? A kdo mi složí rok? A kdo jest ten pastýr, kterýž by se postavil proti mne?
\par 20 Protož slyšte radu Hospodinovu, kterouž zavrel o Idumejských, a to, což myslil proti obyvatelum Temanským: Zajisté žet je vyvlekou nejmenší tohoto stáda, zajisté že je popléní i príbytky jejich.
\par 21 Od hrmotu pádu jejich trásti se bude ta zeme, hlas a krik jejich slyšán bude u more Rudého.
\par 22 Aj, jako orlice pritáhne a priletí, a roztáhne krídla svá na Bozru, i bude srdce silných Idumejských v ten den podobné srdci ženy svírající se.
\par 23 Proti Damašku. Zastydí se Emat i Arfad, nebo novinu zlou uslyší, a užasnou se, tak že se i more zkormoutí, aniž se bude moci upokojiti.
\par 24 Oslábne Damašek, obrátí se k utíkání, a hruza podejme jej, svírání a bolesti zachvátí jej jako rodicku.
\par 25 Ale rkou: Jakž by nemelo ostáti mesto slovoutné, mesto radosti mé?
\par 26 Protož padnou mládenci jeho na ulicích jeho, a všickni muži bojovní vypléneni budou v ten den, dí Hospodin zástupu.
\par 27 A zanítím ohen ve zdi Damašské, kterýž do konce zkazí paláce Benadadovy.
\par 28 Proti Cedar a královstvím Azor, kteráž pohubiti má Nabuchodonozor král Babylonský. Takto praví Hospodin: Vstante, táhnete proti Cedar, a vyplente národy východní.
\par 29 Stany jejich i stáda jejich vezmou, kortýny jejich se vším nádobím jejich, i velbloudy jejich poberou sobe, a volati budou na ne: Strach jest vukol.
\par 30 Utecte, rozprchnete se rychle, skrejte se hluboce, obyvatelé Azor, dí Hospodin, nebot Nabuchodonozor král Babylonský složil proti vám radu, a vymyslil proti vám chytrost.
\par 31 Vstante, táhnete proti národu upokojenému, kterýž sedí bezpecne, praví Hospodin. Nemá ani vrat ani závory, a samotní bydlejí.
\par 32 Budou zajisté velbloudi jejich v loupež, a množství dobytku jejich v korist, a rozptýlím na všeliký vítr ty, kteríž i v nejzadnejších koutech bydlejí, a ze všech stran uvedu bídu na ne, dí Hospodin.
\par 33 I bude Azor príbytkem draku, pustinou až na veky; neosadí se tam žádný, aniž bude bydleti v nem syn cloveka.
\par 34 Slovo Hospodinovo, kteréž se stalo k Jeremiášovi proroku proti Elamitským, na pocátku kralování Sedechiáše krále Judského, rkoucí:
\par 35 Takto praví Hospodin zástupu: Aj, já polámi lucište Elamitských, nejvetší sílu jejich.
\par 36 Uvedu zajisté na Elamitské ctyri vetry ode ctyr stran sveta, a rozptýlím je na všecky ty vetry, tak že nebude národu, do nehož by se nedostal nekdo z vyhnaných Elamitských.
\par 37 A predesím Elamitské pred neprátely jejich a pred temi, kteríž hledají bezživotí jejich. Uvedu, pravím, na ne zlé, prchlivost hnevu svého, dí Hospodin, a budu posílati za nimi mec, dokudž jim konce neuciním.
\par 38 I postavím stolici svou mezi Elamitskými, a vypléním odtud krále i knížata, praví Hospodin.
\par 39 A však stane se v posledních dnech, že zase privedu zajaté Elamitské, dí Hospodin.

\chapter{50}

\par 1 Slovo, kteréž mluvil Hospodin proti Babylonu a proti zemi Kaldejské skrze Jeremiáše proroka:
\par 2 Oznamujte mezi národy a rozhlašujte, zdvihnete korouhev, rozhlašujte, netajte, rcete: Vzat bude Babylon, zahanben bude Bél, potrín bude Merodach, zahanbeny budou modly jeho, potríni budou ukydaní bohové jeho.
\par 3 Nebo pritáhne na nej národ od pulnoci, kterýž obrátí zemi jeho v pustinu, tak že nebude obyvatele v ní. Od cloveka až do hovada vystehují se, odejdou.
\par 4 V tech dnech a toho casu, dí Hospodin, prijdou synové Izraelští, oni i synové Judští spolu; placíce, ochotne pujdou, a Hospodina Boha svého hledati budou.
\par 5 Na cestu k Sionu ptáti se budou, a obrátíce se tam, reknou: Podte a pripojte se k Hospodinu smlouvou vecnou, nepricházející v zapomenutí.
\par 6 Ovce hynoucí jsou lid muj, pastýri jejich pusobí to, aby bloudily, a po horách se toulaly, s hury na pahrbek chodily, zapomenuvše na príbytky své.
\par 7 Všickni, kteríž je nalézají, zžírají je, a neprátelé jejich ríkají: Nebudeme nic vinni, proto že hreší proti Hospodinu. Príbytek spravedlnosti a otcu jejich nadeje jest Hospodin.
\par 8 Vystehujte se z prostredku Babylona, a z zeme Kaldejské vyjdete, a budte jako kozlové pred stádem.
\par 9 Nebo aj, já vzbudím a privedu na Babylon shromáždení národu velikých z zeme pulnocní, kterížto sšikují se proti nemu, i bude dobyt odtud. Kterýchžto strely jsou jako silného, jenž sirobu uvodí; žádnát se nenavrátí na prázdno.
\par 10 I bude zeme Kaldejských v loupež; všickni, kteríž ji loupiti budou, nasytí se, dí Hospodin.
\par 11 Proto že se veselíte, proto že pléšete, ó dráci dedictví mého, proto že jste zbujneli jako jalovice vytylá, a provyskujete jako rekové,
\par 12 Zahanbena bude matka vaše velice, a zapýrí se rodicka vaše: Aj, nejzadnejší z národu, poušt, zeme vyprahlá a pustina.
\par 13 Pro prchlivost Hospodinovu nebude v ní bydleno, ale velmi spustne všecko. Každý, kdož pujde mimo Babylon, užasne se, a diviti se bude nade všemi ranami jeho.
\par 14 Sšikujte se proti Babylonu vukol všickni, kteríž natahujete lucište, strílejte proti nemu, nelitujte strely; nebo hrešil proti Hospodinu.
\par 15 Kricte proti nemu vukol: Poddal se, padli základové jeho, poboreny jsou zdi jeho. Nebo pomsta Hospodinova jest, uvedte pomstu na nej; jakž ciníval, ucinte jemu.
\par 16 Vyplente rozsevace z Babylona, i držícího srp v cas žne; pred mecem hubícím každý necht se k lidu svému obrátí, a každý do zeme své necht utece.
\par 17 Hovádko zahnané jest Izrael, kteréž lvové splašili. Nejprvé zžíral je král Assyrský, tento pak poslednejší, Nabuchodonozor král Babylonský, kosti jeho potrel.
\par 18 Protož toto praví Hospodin zástupu, Buh Izraelský: Aj, já navštívím krále Babylonského i zemi jeho, jako jsem navštívil krále Assyrského.
\par 19 A privedu zase Izraele do príbytku jeho, aby se pásl na Karmeli a Bázan, a na hore Efraim, a v Galád aby se sytila duše jeho.
\par 20 V tech dnech a toho casu, dí Hospodin, byla-li by vyhledávána nepravost Izraelova, nebude žádné, a hríchové Judovi, však nebudou nalezení; nebo odpustím tem, kteréž pozustavím.
\par 21 Proti té zemi zpurných táhni, a proti obyvatelum pomsty; zhub je a zahlad jako proklaté i utíkající, dí Hospodin. Uciniž, pravím, všecko, jakž prikazuji tobe,
\par 22 At jest hluk boje v té zemi a potrení veliké.
\par 23 Jakž by posekáno a polámáno býti mohlo kladivo vší zeme? Jak by k užasnutí Babylon býti mohl mezi národy?
\par 24 Polékl jsem na te, ó Babylone, procež vzat budeš, než zvíš. Nalezen, ano i polapen budeš, proto že jsi smel potýkati se s Hospodinem.
\par 25 Otevrel Hospodin poklad svuj, a vynesl nástroje hnevu svého; nebo dílo toto jest Panovníka Hospodina zástupu v zemi Kaldejské.
\par 26 Pritáhnete na ni od konce zeme, zotvírejte obilnice její, šlapejte po ní jako po stozích, a zahladte ji jako proklatou, tak aby z ní niceho nepozustalo.
\par 27 Zbíte mecem všecky volky její, necht sstoupí k zabití; beda jim, když prijde den jejich, cas navštívení jejich.
\par 28 Hlas utíkajících a ucházejících z zeme babylonské, aby oznámili na Sionu pomstu Hospodina Boha našeho, pomštení chrámu jeho.
\par 29 Shromaždte proti babylonu nejudatnejší, všickni natahující lucište, položte se proti nemu vukol, at nelze jemu ujíti. Odplatte jemu podlé skutku jeho, všecko, jakž delával, ucinte jemu; nebo proti Hospodinu pýchal, proti Svatému Izraelskému.
\par 30 Protož padnou mládenci jeho na ulicích jeho, a všickni muži bojovní jeho vypléneni budou v ten den, dí Hospodin.
\par 31 Aj, já jsem proti tobe, ó pýcho, praví Panovník Hospodin zástupu; nebot prišel den tvuj, cas, abych te navštívil.
\par 32 Poklesne se zajisté ten pyšný a padne, a nebude žádného, kdo by jej zdvihl; a zanítím ohen v mestech jeho, kterýžto zžíre všecka vukolí jeho.
\par 33 Takto praví Hospodin zástupu: Utišteni jsou synové Izraelští, i s syny Judskými, a všickni, kteríž je zjímali, drží je, nechtí propustiti jich.
\par 34 Ale vykupitel jejich silný, jehož jméno jest Hospodin zástupu, jistotne povede pri jejich, aby pokoj zpusobil této zemi, a pohnul obyvateli Babylonskými.
\par 35 Mec na Kaldejské, dí Hospodin, a na obyvatele Babylonské, i na knížata jeho i na mudrce jeho.
\par 36 Mec na lháre, aby se zbláznili, mec na silné jeho, aby potríni byli.
\par 37 Mec na kone jeho a na vozy jeho, i na všecku tu smesici, kteráž jest u prostred neho, aby byli jako ženy; mec na poklady jeho, aby rozchvátáni byli.
\par 38 Sucho na vody jeho, aby vyschly; nebo zeme plná jest rytin, a pri modlách bláznívají.
\par 39 Protož bydliti budou tam šelmy s hroznými potvorami, bydliti budou v ní i mladé sovy; a nebude tam bydleno na veky, ani prebýváno od národu až do pronárodu.
\par 40 Podobná bude k podvrácení hroznému Sodomy a Gomory i sousedu jejich, dí Hospodin; neosadí se tam žádný, aniž bydliti bude v ní syn cloveka.
\par 41 Aj, lid pritáhne od pulnoci, a národ veliký, i králové znamenití, vzbuzeni jsouce od stran zeme.
\par 42 Lucište a kopí pochytí, ukrutní budou, a neslitují se; hlas jejich jako more zvuceti bude, a na koních pojedou, sšikovaní jako muž udatný k boji proti tobe, ó dcero Babylonská.
\par 43 Král Babylonský jakž uslyší povest o nich, opadnou ruce jeho, úzkost zachvátí jej, bolest jako rodicku.
\par 44 Aj, jako lev vystupuje, více než zdutí Jordána proti príbytku Nejsilnejšího, a však v okamžení zaženu jej z této zeme, a toho, kterýž jest vyvolený, ustanovím nad ní. Nebo kdo jest mne rovný? A kdo mi složí rok? A kdo jest ten pastýr, kterýž by se postavil proti mne?
\par 45 Protož slyšte radu Hospodinovu, kterouž zavrel o Babylonu, a to, což myslil proti zemi Kaldejské: Zajisté žet je vyvlekou nejmenší tohoto stáda, zajisté že je popléní i príbytek jejich.
\par 46 Od zvuku pri dobývání Babylona trásti se bude ta zeme, a krik mezi národy slyšán bude.

\chapter{51}

\par 1 Takto praví Hospodin: Aj, já vzbudím proti Babylonu a proti tem, kteríž bydlejí u prostred povstávajících proti mne, vítr hubící.
\par 2 A pošli na Babylon ty, jenž vejí, kterížto prevívati budou jej, a vyprázdní zemi jeho, když budou proti nemu vukol v den bídy.
\par 3 Natahujícímu, kterýž silne natahuje lucište své, a krácí v pancíri svém, dím: Neslitovávejtež se nad mládenci jeho, zahladte jako proklaté všecko vojsko jeho.
\par 4 At padnou zbití v zemi Kaldejské, a probodnutí na ulicích jeho.
\par 5 Nebo není opušten Izrael a Juda od Boha svého, od Hospodina zástupu, ackoli zeme jejich plná jest provinení proti Svatému Izraelskému.
\par 6 Utecte z prostredku babylona, a zachovejte jeden každý život svuj, abyste nebyli vypléneni v nepravosti jeho. Nebo cas bude pomsty Hospodinovy, sám odplatu dá jemu.
\par 7 Bylte koflíkem zlatým Babylon v ruce Hospodinove, opojujícím všecku zemi; víno jeho pili národové, protož se zbláznili národové.
\par 8 Ale v náhle padne Babylon, a potrín bude. Kvelte nad ním, naberte masti pro bolesti jeho, snad bude moci zhojen býti.
\par 9 Hojili jsme babylon, ale není zhojen, opustme jej, a podme jeden každý do zeme své; nebo až k nebi dosahá soud jeho, a vznesen jest až k nejvyšším oblakum.
\par 10 Vyvedl Hospodin pri naši; podte, a vypravujme na Sionu dílo Hospodina Boha našeho.
\par 11 Vytrete strely, shledejte, což nejvíc mužete, pavéz, Hospodin vzbudil ducha králu Médských, proto že proti Babylonu usouzení jeho jest, aby jej zkazil; pomsta zajisté jest Hospodinova, pomštení chrámu jeho.
\par 12 Na zdech babylonských zdvihnete korouhev, osadte stráž, postavte strážné, pripravte zálohy; nebo i myslil Hospodin, i uciní, což rekl proti obyvatelum Babylonským.
\par 13 Ó ty, kterýž bydlíš pri vodách velikých, ó kterýž máš množství pokladu, prišlot skoncení tvé, cíl lakomství tvého.
\par 14 Prisáhlt jest Hospodin zástupu skrze samého sebe: Jistotne naplním te lidmi jako brouky, kteríž by prokrikovali nad tebou radostne:
\par 15 Ten, kterýž ucinil zemi mocí svou, kterýž utvrdil okršlek sveta moudrostí svou, a opatrností svou roztáhl nebesa,
\par 16 Kterýžto, když vydává hlas, jecí vody na nebi, a kterýž pusobí to, aby vystupovaly páry od kraje zeme, blýskání s deštem privodí, a vyvodí vítr z pokladu svých.
\par 17 Tak zhlupel každý clovek, že nezná toho, že zahanben bývá každý zlatník od rytiny; nebo faleš jest slitina jeho, a není ducha v nich.
\par 18 Marnost jsou a dílo podvodu; v cas, v nemž navštíveni budou, zahynou.
\par 19 Nenít jim podobný díl Jákobuv; nebo on jest stvoritelem všeho, a cástka dedictví jeho, Hospodin zástupu jest jméno jeho.
\par 20 Ty jsi mým kladivem rozrážejícím, nástroji válecnými, abych rozrážel skrze tebe národy, a kazil skrze tebe království,
\par 21 Abych rozrážel skrze tebe kone s jezdcem jeho, abych rozrážel skrze tebe vuz s tím, kdož jezdí na nem.
\par 22 Abych rozrážel skrze tebe muže i ženu, abych rozrážel skrze tebe starého i díte, abych rozrážel skrze tebe mládence i pannu,
\par 23 Abych rozrážel skrze tebe pastýre s stádem jeho, abych rozrážel skrze tebe oráce s sprežením jeho, abych rozrážel skrze tebe vývody a knížata.
\par 24 Ale jižt odplatím Babylonu i všechnem obyvatelum Kaldejským za všecko bezpráví jejich, kteréž cinili Sionu pred ocima vašima, dí Hospodin.
\par 25 Aj, já jsem proti tobe, ó horo, kteráž hubíš, dí Hospodin, kteráž hubíš všecku zemi; a vztáhna ruku svou na tebe, svalím te z tech skal, a obrátím te v horu spálenou.
\par 26 A nevezmou z tebe kamene k úhlu, ani kamene k základum; nebo pustinou vecnou budeš, praví Hospodin.
\par 27 Vyzdvihnete korouhev v zemi, trubte trubou mezi národy, pripravte proti nemu národy, svolejte proti nemu království Ararat, Minni, Ascenez, ustanovte hejtmana proti nemu, privedte koní jako brouku pocet nescíslný.
\par 28 Pripravte proti Babylonu národy, krále zeme Médské, vývody její a všecka knížata její, i všecku zemi panování jejich.
\par 29 I bude se trásti zeme a bolestiti, když vykonáváno bude usouzení Hospodinovo proti Babylonu, aby obrátil zemi Babylonskou v pustinu, v níž by se žádný neosazoval.
\par 30 Prestanou silní Babylonští bojovati, sedeti budou v ohradách, zhyne síla jejich, budou jako ženy, zapálí príbytky jejich, polámány budou závory jejich.
\par 31 Pošta jedna druhou potká, a posel posla, aby oznámeno bylo králi Babylonskému, že vzato jest mesto jeho na kraji,
\par 32 A že brodové vzati, i jezera vypálena ohnem, a muži bojovní predešeni.
\par 33 Nebo takto praví Hospodin zástupu, Buh Izraelský: Dcera Babylonská jest jako humno, jehož nabíjení cas; ješte malicko, a prijde cas žne její.
\par 34 Vyjídá mne, potírá mne Nabuchodonozor král Babylonský, vystavuje mne nádobu prázdnou, požírá mne jako drak, naplnuje brich svuj rozkošemi mými, vyhání mne.
\par 35 Násilé, kteréž se mne a mému telu deje, prijdiž na Babylon, praví obyvatelkyne Sionská, a krev má na obyvatele zeme Kaldejské, praví Jeruzalém.
\par 36 Protož takto praví Hospodin: Aj, já vyvedu pri tvou, a pomstím te; nebo vysuším more jeho, vysuším i vrchovište jeho.
\par 37 I bude Babylon v hromady, v príbytek draku, v užasnutí a ckání, tak že nebude v nem žádného obyvatele.
\par 38 Jako lvové spolu rváti budou, skuceti budou jako lvícata.
\par 39 Když se rozpálí, uciním jim hody, a tak je opojím, že zkriknouti a snem vecným zesnouti musejí, tak aby neprocítili, dí Hospodin.
\par 40 Povedu je jako berany k zabití, jako skopce s kozly.
\par 41 Jakž by dobyt býti mohl Sesák? Jakž by vzata býti mohla chvála vší zeme? Jakž by prijíti mohl na spuštení Babylon mezi národy?
\par 42 Vystoupí proti Babylonu more, množstvím vlnobití jeho prikryt bude.
\par 43 Mesta jeho budou pustinou, zemí vyprahlou a pustou, zemí, v jejíchž mestech neosadí se žádný, aniž projde skrze ne syn cloveka.
\par 44 Navštívím také Béle v Babylone, a vytrhnu, což sehltil, z úst jeho, i nepohrnou se k nemu více národové; také i zdi Babylonské padnou.
\par 45 Vyjdete z prostredku jeho, lide muj, a vysvobodte jeden každý duši svou od prchlivosti hnevu Hospodinova.
\par 46 A nebudtež choulostivého srdce, aniž se bojte povesti, kteráž slyšána bude v té zemi, když prijde tohoto roku povest, a potom druhého roku povest, i ukrutenství v zemi, a pán na pána.
\par 47 Protož aj, dnové prijdou, že navštívím rytiny Babylonské, a všecka zeme jeho zahanbena bude, i všickni zbití jeho padnou u prostred neho.
\par 48 I budou prozpevovati nad Babylonem nebesa i zeme, a cožkoli v nich jest, když na nej od pulnoci pritáhnou ti zhoubcové, dí Hospodin.
\par 49 Ano i Babylon padnouti musí, ó zbití Izraelovi, i s Babylonem padnou zbití vší zeme jeho.
\par 50 Ó kdož jste znikli mece, jdete, nezastavujte se; zpomínejte, daleko jsouce, na Hospodina, a Jeruzalém necht vstupuje na srdce vaše.
\par 51 Rcete: Stydímet se, že slýcháme útržku, hanbí se tváre naše, že cizozemci chodí do svatyní domu Hospodinova.
\par 52 Protož aj, dnové prijdou, dí Hospodin, že navštívím rytiny jeho, a po vší zemi jeho stonati bude zranený.
\par 53 Byt pak vstoupil Babylon na nebe, a byt pak ohradil velmi vysokou pevnost svou, pritáhnou ode mne na nej zhoubcové, praví Hospodin.
\par 54 Zvuk kriku z Babylona, a potrení veliké z zeme Kaldejské.
\par 55 Nebo Hospodin popléní Babylon, a vyhladí z neho hrmot veliký, byt pak zvucela vlnobití jejich jako vody mnohé, vydáván byl hluk hlasu jejich,
\par 56 Když pritáhne na nej, na Babylon zhoubce. I budou jati silní jeho, potríno bude lucište jejich; nebo Buh silný odplatí, Hospodin vrchovate odplatí.
\par 57 Opojím knížata jeho i moudré jeho, vývody jeho i znamenitejší jeho i silné jeho, aby zesnuli snem vecným, a neprocítili, dí král, jehož jméno jest Hospodin zástupu.
\par 58 Takto praví Hospodin zástupu: Oboje zed Babylonská velmi široká do gruntu zborena bude, a brány jeho vysoké ohnem spáleny budou, a tak nadarmo pracovati budou lidé a národové pri ohni, až ustanou.
\par 59 Slovo, kteréž prikázal Jeremiáš prorok Saraiášovi synu Neriášovu, synu Maaseiášovu, když se vypravil od Sedechiáše krále Judského do Babylona, léta ctvtého kralování jeho, (byl pak Saraiáš kníže Menuchské),
\par 60 Když sepsal Jeremiáš všecko zlé, kteréž prijíti melo na Babylon, do knihy jedné, všecka ta slova, kteráž jsou psána proti Babylonu.
\par 61 I rekl Jeremiáš Saraiášovi: Když prijdeš do Babylona, a uzríš jej, tedy cti všecka ta slova.
\par 62 A rci: Ó Hospodine, ty jsi mluvil o míste tomto, že je zkazíš, tak že nebude v nem obyvatele, od cloveka až do hovada, ale že hrozne zpušteno na veky bude.
\par 63 Když pak do konce precteš knihu tuto, privaž k ní kámen, a hod ji do prostred Eufrates,
\par 64 A rci: Tak potopen bude Babylon, a nepovstane z toho zlého, kteréž já uvedu na nej, ackoli ustávati budou. Až potud slova Jeremiášova.

\chapter{52}

\par 1 V jedenmecítma letech byl Sedechiáš, když pocal kralovati, a jedenácte let kraloval v Jeruzaléme. Jméno matky jeho bylo Chamutal, dcera Jeremiášova z Lebna.
\par 2 I cinil to, což jest zlého pred ocima Hospodinovýma, všecko tak, jakž byl delal Joakim.
\par 3 Nebo se to dálo pro rozhnevání Hospodinovo proti Jeruzalému a Judovi, až je i zavrhl od tvári své. V tom zprotivil se Sedechiáš králi Babylonskému.
\par 4 Stalo se pak léta devátého kralování jeho, mesíce desátého, v desátý den téhož mesíce, že pritáhl Nabuchodonozor král Babylonský se vším vojskem svým k Jeruzalému, a položili se u neho, a vzdelali proti nemu hradbu vukol.
\par 5 A bylo mesto obleženo až do jedenáctého léta krále Sedechiáše.
\par 6 V kterémžto, mesíce ctvrtého, devátého dne téhož mesíce, rozmohl se hlad v meste, a nemel chleba lid zeme.
\par 7 I prolomeny jsou zdi mestské, a všickni muži bojovní utekli, a vyšli z mesta v noci skrze bránu mezi dvema zdmi, u zahrady královské, (Kaldejští pak leželi okolo mesta), a ušli cestou poušte.
\par 8 I honilo vojsko Kaldejské krále, a postihli Sedechiáše na rovinách Jerišských, a všecko vojsko jeho rozprchlo se od neho.
\par 9 A tak javše krále, privedli ho k králi Babylonskému do Ribla v zemi Emat, kdežto ucinil o nem soud.
\par 10 I zmordoval král Babylonský syny Sedechiášovy pred ocima jeho, ano i všecka knížata Judská zmordoval v Ribla.
\par 11 Oci pak Sedechiášovy oslepil, a svázav ho retezy ocelivými král Babylonský, dal jej dovésti do Babylona, a dal jej do vezení až do dne smrti jeho.
\par 12 Potom mesíce pátého, desátého dne téhož mesíce, léta devatenáctého kralování Nabuchodonozora krále Babylonského, pritáhl Nebuzardan, hejtman nad žoldnéri, kterýž sloužíval králi Babylonskému, do Jeruzaléma.
\par 13 A zapálil dum Hospodinuv, i dum královský, i všecky domy v Jeruzaléme, a tak všecky domy veliké vypálil ohnem.
\par 14 Všecky také zdi Jeruzalémské vukol poborilo všecko vojsko Kaldejské, kteréž bylo s tím hejtmanem nad žoldnéri.
\par 15 Pritom z chaterného lidu, totiž ostatek lidu, kterýž byl zustal v meste, i pobehlce, kteríž byli ustoupili k králi Babylonskému, a jiný obecný lid zavedl Nebuzardan, hejtman nad žoldnéri.
\par 16 Toliko neco chaterného lidu zeme zanechal Nebuzardan, hejtman nad žoldnéri, aby byli vinari a oráci.
\par 17 Nadto sloupy medené, kteríž byli v dome Hospodinove, i podstavky i more medené, kteréž bylo v dome Hospodinove, ztloukli Kaldejští, a odvezli všecku med z nich do Babylona.
\par 18 Též hrnce. lopaty a nástroje hudebné, a kotlíky a kadidlnice, i všecky nádoby medené, jimiž sloužili, pobrali.
\par 19 I medenice a nádoby k oharkum, s kotlíky a hrnci, a svícny, a kadidlnice a koflíky, a cožkoli zlatého a stríbrného bylo, pobral hejtman nad žoldnéri;
\par 20 Sloupy dva, more jedno, a volu dvanáct medených, kteríž byli pod podstavky, jichž byl nadelal král Šalomoun do domu Hospodinova. Nebylo váhy medi všech tech nádob.
\par 21 Nebo sloupu tech, osmnácti loket byla výška sloupu jednoho, kterýž okolek mel dvanácti loktu, ztlouští pak ctyr prstu byl dutý.
\par 22 A makovice na nem medená, a makovice jedné výška peti loket, a mrežování i jablka zrnatá na té makovici vukol; všecko bylo medené. Takovýž byl i sloup druhý s zrnatými jablky.
\par 23 A bylo jablek zrnatých devadesát a šest po každé strane; všudy jablek zrnatých bylo po stu na mrežování vukol.
\par 24 Vzal také hejtman nad žoldnéri Saraiáše kneze predního, a Sofoniáše kneze nižšího, a tri strážné prahu.
\par 25 A z mesta vzal komorníka jednoho, kterýž byl hejtmanem nad muži bojovnými, a sedm mužu, jenž bývali pri králi, kteríž nalezeni byli v meste, a predního spisovatele vojska, kterýž popisoval vojsko z lidu zeme, a šedesáte mužu z lidu zeme, kteríž nalezeni byli v meste.
\par 26 Zjímav tedy je Nebuzardan, hejtman nad žoldnéri, privedl je k králi Babylonskému do Ribla.
\par 27 I pobil je král Babylonský, a zmordoval je v Ribla v zemi Emat, a tak zaveden jest Juda z zeme své.
\par 28 Tent jest lid, kterýž zavedl Nabuchodonozor léta sedmého, Judských tri tisíce a trimecítma.
\par 29 Léta osmnáctého Nabuchodonozora zavedl z Jeruzaléma duší osm set tridceti a dve.
\par 30 Léta trimecítmého Nabuchodonozora zavedl Nebuzardan, hejtman nad žoldnéri, Judské, duší sedm set ctyridceti a pet, všech duší ctyry tisíce a šest set.
\par 31 Stalo se také tridcátého sedmého léta po zajetí Joachina krále Judského, dvanáctého mesíce, dvadcátého pátého dne téhož mesíce, povýšil Evilmerodach král Babylonský toho léta, když pocal kralovati, Joachina krále Judského, pustiv ho z žaláre.
\par 32 A mluvil s ním dobrotive, i stolici jeho postavil nad stolice králu, kteríž s ním byli v Babylone.
\par 33 Zmenil též roucho jeho, kteréž mel v žalári, a jídal chléb pred ním vždycky, po všecky dny života svého.
\par 34 Nebo vymerený pokrm ustavicne dáván byl jemu od krále Babylonského, a to na každý den, až do dne smrti jeho, po všecky dny života jeho.

\end{document}