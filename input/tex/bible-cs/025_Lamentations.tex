\begin{document}

\title{Pláč (Žalozpěvů)}

\chapter{1}

\par 1 Ach, mesto tak lidné, jakt jest samotné zustalo, a ucineno jako vdova! Nejznamenitejší mezi národy, prední mezi krajinami pod plat uvedeno.
\par 2 Ustavicne pláce v noci, a slzy jeho na lících jeho, ze všech milovníku svých nemá žádného potešitele; všickni prátelé jeho neverne se k nemu mají, obrátili se mu v neprátely.
\par 3 Zastehoval se Juda, proto že byl trápen a u veliké porobe, však osadiv se mezi pohany, nenalézá odpocinutí; všickni, kteríž jej honí, postihají jej v tesne.
\par 4 Cesty Siona kvílí, že žádný neprichází k slavnosti. Všecky brány jeho zpustly, kneží jeho vzdychají, panny jeho smutné jsou, on pak sám pln jest horkosti.
\par 5 Neprátelé jeho jsou hlavou, odpurcum jeho štastne se vede; nebo jej Hospodin zarmoutil pro množství prestoupení jeho. Maliccí jeho odešli do zajetí pred oblícejem neprítele.
\par 6 A tak odnata od dcery Sionské všecka okrasa její. Knížata její jsou podobná jelenum, kteríž nenalézají pastvy, a ucházejí bez moci pred tím, kdož je honí.
\par 7 Rozpomínát se dcera Jeruzalémská ve dnech trápení svého a kvílení svého na všecka svá utešení, kteráž mívala ode dnu starodávních, když padá lid její od ruky neprítele, nemajíc žádného, kdo by ji retoval. Protivnícit se jí dívajíce, posmívají se klesnutí jejímu.
\par 8 Težce hrešila dcera Jeruzalémská, protož jako necistá odloucena jest. Všickni, kteríž ji v poctivosti mívali, neváží jí sobe, proto že vidí nahotu její; ona pak vzdychá, obrácena jsuci zpet.
\par 9 Necistota její na podolcích jejích; nepamatovala na skoncení své, protož patrne klesá, nemajíc žádného, kdo by ji potešil. Popatriž, Hospodine, na trápení mé, nebot se vyvýšil neprítel.
\par 10 Sáhl rukou svou neprítel na všecky drahé veci její; nebo musí se dívati pohanum, an chodí do svatyne její, o cemž jsi byl prikázal, aby tobe nevcházeli do shromáždení.
\par 11 Všecken lid její vzdychajíce, hledají chleba, vynakládají nejdražší veci své za pokrm k ocerstvení života. Vzezriž, Hospodine, a popatriž, nebot jsem v nevážnosti.
\par 12 Nic-liž vám do toho, ó všickni, kteríž tudyto jdete? Pohledte a vizte, jest-li bolest podobná bolesti mé, kteráž jest mi ucinena, jak mne zámutkem naplnil Hospodin v den prchlivosti hnevu svého.
\par 13 Seslal s výsosti ohen do kostí mých, kterýž opanoval je; roztáhl sít nohám mým, obrátil mne zpet, obrátil mne v pustinu, celý den neduživá jsem.
\par 14 Tuze svázáno jest rukou jeho jho prestoupení mých, tuze spletené houžve pripadly na hrdlo mé, porazilo sílu mou; vydal mne Pán v ruku neprátel, nemohut povstati.
\par 15 Pošlapal Pán všecky mé silné u prostred mne, svolal proti mne zástupy, aby potrel mládence mé, tlacil Pán presem pannu dceru Judskou.
\par 16 Pro tyt veci já pláci, z ocí mých, z ocí mých tekou vody, a že jest vzdálen ode mne potešitel, kterýž by ocerstvil duši mou; synové moji jsou pohubeni, nebo ssilil se neprítel.
\par 17 Rozprostírá dcera Sionská ruce své, nemá žádného, kdo by ji potešil; vzbudilte Hospodin proti Jákobovi všudy vukol neho neprátely jeho, mezi nimiž jest dcera Jeruzalémská jako pro necistotu oddelená.
\par 18 Spravedlivý jest Hospodin, nebot jsem na odpor cinila ústum jeho. Slyšte medle všickni lidé, a vizte bolest mou; panny mé i mládenci moji odebrali se do zajetí.
\par 19 Volala jsem na milovníky své, oni oklamali mne; kneží moji a starci moji v meste pomreli, hledajíce pokrmu, aby posilnili života svého.
\par 20 Popatriž, ó Hospodine, nebot mi úzko; vnitrnosti mé zkormouceny jsou, srdce mé svadne ve mne, proto že jsem na odpor velice cinila. Vne mec na sirobu privodí, v dome pouhá smrt.
\par 21 Slýchajít, že já vzdychám, ale není žádného, kdo by mne potešil. Všickni neprátelé moji slyšíce o mých bídách, radují se, že jsi to ucinil, a privedl den predohlášený, ale budout mne podobní.
\par 22 Necht prijde všecka nešlechetnost jejich pred oblícej tvuj, a ucin jim, jakož jsi ucinil mne pro všecka prestoupení má; nebot jsou mnohá úpení má, a srdce mé neduživé.

\chapter{2}

\par 1 Jak hustým oblakem prchlivosti své prikryl Pán dceru Sionskou! Shodil s nebe na zem slávu Izraelovu, aniž se rozpomenul na podnože noh svých v den prchlivosti své.
\par 2 Sehltil Pán beze vší lítosti všecky príbytky Jákobovy, zboril v prchlivosti své ohrady dcery Judské, uderil jimi o zem, v potupu uvedl království i knížata její.
\par 3 Odtal v rozpálení hnevu všecken roh Izraeluv, odvrátil zpet pravici svou od neprítele, a rozpáliv se proti Jákobovi jako ohen plápolající, pálí do cela vukol.
\par 4 Natáhl lucište své jako neprítel, postavil pravici svou jako protivník, i zbil všecky nejzdarilejší z lidu, a vylil do stánku dcery Sionské jako ohen prchlivost svou.
\par 5 Ucinen jest Pán podobný nepríteli, sehltil Izraele, sehltil všecky paláce jeho, zkazil ohrady jeho, a rozmnožil v lidu Judském zámutek a žalost.
\par 6 Mocí zajisté odtrhl jako od zahrady plot svuj, zkazil stánek svuj, v zapomenutí uvedl Hospodin na Sionu slavnost a sobotu, a v prchlivosti hnevu svého zavrhl krále i kneze.
\par 7 Zavrhl Pán oltár svuj, v ošklivost vzal svatyni svou, vydal v ruku neprítele zdi a paláce Sionské; kriceli v dome Hospodinove jako v den slavnosti.
\par 8 Uložilte Hospodin zkaziti zed dcery Sionské, roztáhl šnuru, a neodvrátil ruky své od zhouby; procež val i zed kvílí, a spolu mdlejí.
\par 9 Poraženy jsou na zem brány její, zkazil a polámal závory její; král její i knížata její mezi pohany. Není ani zákona, proroci také její nemívají videní od Hospodina.
\par 10 Starší dcery Sionské usadivše se na zemi, mlcí, posýpají prachem hlavy své, a prepasují se žínemi panny Jeruzalémské, svešují k zemi hlavy své.
\par 11 Zhynuly od slz oci mé, zkormoutily se vnitrnosti mé, a vykydla se na zem játra má, pro potrení dcery lidu mého, když i nemluvnátka a prsí požívající na ulicích mesta se svírají.
\par 12 A ríkají matkám svým: Kdež jest obilé a víno? když se jako zranený svírají po ulicích mesta, a vypouštejí duše své na klíne matek svých.
\par 13 Kohot za svedka privedu? Koho pripodobním k tobe, ó dcero Jeruzalémská? Koho tobe prirovnám, abych te potešil, panno dcero Sionská? Nebo veliké jest jako more potrení tvé. Kdož by te zhojiti mohl?
\par 14 Proroci tvoji predpovídali tobe lživé a nicemné veci, a neodkrývali nepravosti tvé, aby odvrátili zajetí tvé, ale predpovídali tobe težkosti, oklamání a vyhnání.
\par 15 Všickni, kteríž jdou cestou, tleskají nad tebou rukama, diví se a potrásají hlavou svou za tebou, dcero Jeruzalémská, ríkajíce: To-li jest to mesto, o nemž ríkávali, že jest nejkrásnejší a utešením vší zeme?
\par 16 Všickni neprátelé tvoji rozdírají na tebe ústa svá, hvízdají a škripí zuby, ríkajíce: Sehltme ji. Totot jest jiste ten den, jehož jsme ocekávali; jižte nastal, vidíme.
\par 17 Ucinil Hospodin to, což byl uložil, splnil rec svou, kterouž prikazoval ode dnu starodávních, boril bez lítosti, a obveselil nad tebou neprítele, povýšil rohu protivníku tvých.
\par 18 Vykrikovalo srdce jejich proti Pánu. Ó ty zdi dcery Sionské, vylévej jako potok slzy dnem i nocí, nedávej sobe odpocinutí, aniž se spokojuj zrítelnice oka tvého.
\par 19 Vstan, kric v noci, pri pocátku bdení, vylévej jako vodu srdce své pred oblícejem Páne; pozdvihuj k nemu rukou svých za život dítek svých svírajících se hladem, na rohu všech ulic, a rci:
\par 20 Pohled, Hospodine, a popatr, komu jsi tak kdy ucinil? Zdaliž jídají ženy plod svuj, nemluvnátka rozkošná? Zdaliž mordován býti má v svatyni Páne knez a prorok?
\par 21 Leží na zemi po ulicích mladý i starý, panny mé i mládenci moji padli od mece, zmordoval jsi je, a zbil v den prchlivosti své bez lítosti.
\par 22 Svolal jsi jako ke dni slavnosti z vukolí ty, jichž se velice straším, a nebylo v den prchlivosti Hospodinovy, kdo by ušel neb pozustal. Kteréž jsem na rukou pestovala a vychovala, ty neprítel muj do konce zhubil.

\chapter{3}

\par 1 Já jsem muž okoušející trápení od metly rozhnevání Božího.
\par 2 Zahnal mne, a uvedl do tmy a ne k svetlu.
\par 3 Toliko proti mne se postavuje, a obrací ruku svou pres celý den.
\par 4 Uvedl sešlost na telo mé a kuži mou, a polámal kosti mé.
\par 5 Zastavel mne a obklícil preodpornou horkostí.
\par 6 Postavil mne v tmavých místech jako ty, kteríž již dávno zemreli.
\par 7 Ohradil mne, abych nevyšel; obtížil ocelivý retez muj.
\par 8 A jakžkoli volám a kricím, zacpává uši pred mou modlitbou.
\par 9 Ohradil cesty mé tesaným kamenem, a stezky mé zmátl.
\par 10 Jest nedved cíhající na mne, lev v skrejších.
\par 11 Cesty mé stocil, anobrž roztrhal mne, a na to mne privedl, abych byl pustý.
\par 12 Natáhl lucište své, a vystavil mne za cíl strelám.
\par 13 Postrelil ledví má strelami toulu svého.
\par 14 Jsem v posmechu se vším lidem svým, a písnickou jejich pres celý den.
\par 15 Sytí mne horkostmi, opojuje mne pelynkem.
\par 16 Nadto potrel o kamenícko zuby mé, vrazil mne do popela.
\par 17 Tak jsi vzdálil, ó Bože, duši mou od pokoje, až zapomínám na pohodlí,
\par 18 A ríkám: Zahynulate síla má i nadeje má, kterouž jsem mel v Hospodinu.
\par 19 A však duše má rozvažujíc trápení svá a plác svuj, pelynek a žluc,
\par 20 Rozvažujíc to ustavicne, ponižuje se ve mne.
\par 21 A privode sobe to ku pameti, (nadeji mám),
\par 22 Že veliké jest milosrdenství Hospodinovo, když jsme do konce nevyhynuli. Neprestávajít zajisté slitování jeho,
\par 23 Ale nová jsou každého jitra; preveliká jest pravda tvá.
\par 24 Díl muj jest Hospodin, ríká duše má; protož nadeji mám v nem.
\par 25 Dobrý jest Hospodin tem, jenž ocekávají na nej, duši té, kteráž ho hledá.
\par 26 Dobré jest trpelive ocekávajícímu na spasení Hospodinovo.
\par 27 Dobré jest muži tomu, kterýž by nosil jho od detinství svého,
\par 28 Kterýž by pak byl opušten, trpelive se má v tom, což na nej vloženo,
\par 29 Dávaje do prachu ústa svá, až by se ukázala nadeje,
\par 30 Nastavuje líce tomu, kdož jej bije, a syte se potupou.
\par 31 Nebot nezamítá Pán na vecnost;
\par 32 Nýbrž ackoli zarmucuje, však slitovává se podlé množství milosrdenství svého.
\par 33 Netrápít zajisté z srdce svého, aniž zarmucuje synu lidských.
\par 34 Aby kdo potíral nohama svýma všecky vezne v zemi,
\par 35 Aby nespravedlive soudil muže pred oblícejem Nejvyššího,
\par 36 Aby prevracel cloveka v pri jeho, Pán nelibuje.
\par 37 Kdo jest, ješto když rekl, stalo se neco, a Pán neprikázal?
\par 38 Z úst Nejvyššího zdali nepochází zlé i dobré?
\par 39 Proc by tedy sobe stýskal clovek živý, muž nad kázní za hríchy své?
\par 40 Zpytujme radeji a ohledujme cest našich, a navratme se až k Hospodinu.
\par 41 Pozdvihujme srdcí i rukou svých k Bohu silnému v nebe.
\par 42 Myt jsme se zproneverili, a zpurní jsme byli, protož ty neodpouštíš.
\par 43 Obestrels se hnevem a stiháš nás, morduješ a nešanuješ.
\par 44 Obestrels se oblakem, aby nemohla proniknouti k tobe modlitba.
\par 45 Za smeti a povrhel položil jsi nás u prostred národu techto.
\par 46 Rozdírají na nás ústa svá všickni neprátelé naši.
\par 47 Strach a jáma potkala nás, zpuštení a setrení.
\par 48 Potokové vod tekou z ocí mých pro potrení dcery lidu mého.
\par 49 Oci mé slzí bez prestání, proto že není žádného odtušení,
\par 50 Ažby popatril a shlédl Hospodin s nebe.
\par 51 Oci mé rmoutí duši mou pro všecky dcery mesta mého.
\par 52 Lovilit jsou mne ustavicne, jako ptáce, neprátelé moji bez príciny.
\par 53 Uvrhli do jámy život muj, a primetali mne kamením.
\par 54 Rozvodnily se vody nad hlavou mou, rekl jsem: Jižte po mne.
\par 55 Vzývám jméno tvé, ó Hospodine, z jámy nejhlubší.
\par 56 Hlas muj vyslýchával jsi; nezacpávejž ucha svého pred vzdycháním mým a voláním mým.
\par 57 V ten den, v nemž jsem te vzýval, pricházeje, ríkávals: Neboj se.
\par 58 Pane, zasazuje se o pri duše mé, vysvobozoval jsi život muj.
\par 59 Vidíš, ó Hospodine, prevrácenost, kteráž se mne deje, dopomoziž mi k spravedlnosti.
\par 60 Vidíš všecko vymstívání se jejich, všecky úklady jejich proti mne.
\par 61 Slýcháš utrhání jejich, ó Hospodine, i všecky obmysly jejich proti mne,
\par 62 Reci povstávajících proti mne, a premyšlování jejich proti mne pres celý den.
\par 63 Pohled, jak pri sedání jejich i povstání jejich jsem písnickou jejich.
\par 64 Dej jim odplatu, Hospodine, podlé díla rukou jejich.
\par 65 Dej jim zatvrdilé srdce a prokletí své na ne.
\par 66 Stihej v prchlivosti, a vyhlad je, at nejsou pod nebem tvým.

\chapter{4}

\par 1 Jak te zašlo zlato, zmenilo se ryzí zlato nejvýbornejší, rozmetáno jest kamení svaté sem i tam po všech ulicích.
\par 2 Synové Sionští nejdražší, kteríž ceneni býti meli za zlato nejcištší, jakt jsou pocteni za nádoby hlinené, dílo rukou hrncíre!
\par 3 An draci vynímajíce prsy, krmí mladé své, dcera pak lidu mého prícinou ukrutníka podobná jest sovám na poušti.
\par 4 Jazyk prsí požívajícího prilnul žízní k dásním jeho; deti prosí za chléb, ale není žádného, kdo by lámal jim.
\par 5 Ti, kteríž jídali rozkošné krme, hynou na ulicích; kteríž chováni byli v šarlate, octli se v hnoji.
\par 6 Vetší jest trestání dcery lidu mého, než pomsta Sodomy, kteráž podvrácena jest jako v okamžení, aniž trvaly pri ní rány.
\par 7 Cistší byli Nazareové její než sníh, belejší než mléko, rdela se tela jejich více než drahé kamení, jako by z zafiru vytesáni byli.
\par 8 Ale již vzezrení jejich temnejší jest než cernost, nemohou poznáni býti na ulicích; prischla kuže jejich k kostem jejich, prahne, jest jako drevo.
\par 9 Lépe se stalo tem, jenž zbiti jsou mecem, nežli kteríž mrou hladem, (oni zajisté zhynuli, probodeni byvše), pro nedostatek úrod polních.
\par 10 Ruce žen lítostivých varily syny své, aby jim byli za pokrm v potrení dcery lidu mého.
\par 11 Všelijak vypustil Hospodin prchlivost svou, vylévá zurivost hnevu svého, a zapálil ohen na Sionu, kterýž sežral základy jeho.
\par 12 Králové zeme i všickni obyvatelé okršlku sveta nikoli by neuverili, by mel byl vjíti protivník a neprítel do bran Jeruzalémských,
\par 13 Pro hríchy proroku jeho a nepravosti kneží jeho, vylévajících u prostred neho krev spravedlivých.
\par 14 Toulali se jako slepí po ulicích, kálejíce se ve krvi, kteréž nemohli se než dotýkati odevy svými.
\par 15 Volali na ne: Ustupujte necistí, ustupujte, ustupujte, nedotýkejte se. Právet jsou ustoupili, anobrž sem i tam se rozlezli, až mezi národy ríkají: Nebudout více míti vlastního bydlení.
\par 16 Tvár hnevivá Hospodinova rozptýlila je, aniž na ne více popatrí; neprátelé kneží nešanují, starcum milosti neciní.
\par 17 A vždy ješte až do ustání ocí svých hledíme o pomoc sobe neprospešnou, zrení majíce k národu, nemohoucímu vysvoboditi.
\par 18 Šlakují kroky naše, tak že ani po ulicích našich choditi nemužeme; priblížilo se skoncení naše, doplnili se dnové naši, prišlo zajisté skoncení naše.
\par 19 Rychlejší jsou ti, kteríž nás stihají, než orlice nebeské; po horách stihají nás, na poušti zálohy nám zdelali.
\par 20 Dýchání chrípí našich, totiž pomazaný Hospodinuv, lapen jest v jamách jejich, o nemž jsme ríkali: V stínu jeho živi budeme mezi národy.
\par 21 Raduj se a vesel se, dcero Idumejská, kteráž jsi se usadila v zemi Uz. Takét k tobe prijde kalich, opiješ se a obnažíš se.
\par 22 Ó dcero Sionská, když vykonána bude kázen nepravosti tvé, nenechá te déle v zajetí tvém. Ale tvou nepravost, ó dcero Idumejská, trestati bude a odkryje hríchy tvé.

\chapter{5}

\par 1 Rozpomen se, Hospodine, co se nám deje; popatr a viz pohanení naše.
\par 2 Dedictví naše obráceno jest k cizím, domové naši k cizozemcum.
\par 3 Sirotci jsme a bez otce, matky naše jsou jako vdovy.
\par 4 Vody své za peníze pijeme, dríví naše za záplatu prichází.
\par 5 Na hrdle svém protivenství snášíme, pracujeme, nedopouští se nám odpocinouti.
\par 6 Egyptským podáváme ruky i Assyrským, abychom nasyceni byli chlebem.
\par 7 Otcové naši hrešili, není jich, my pak trestáni po nich neseme.
\par 8 Služebníci panují nad námi; není žádného, kdo by vytrhl z ruky jejich.
\par 9 S opovážením se života svého hledáme chleba svého, pro strach mece i na poušti.
\par 10 Kuže naše jako pec zcernaly od náramného hladu.
\par 11 Ženám na Sionu i pannám v mestech Judských násilé ciní.
\par 12 Knížata rukou jejich zvešena jsou, osoby starých nemají v poctivosti.
\par 13 Mládence k žernovu berou, a pacholata pod drívím klesají.
\par 14 Starci sedati v branách prestali a mládenci od zpevu svých.
\par 15 Prestala radost srdce našeho, obrátilo se v kvílení plésání naše.
\par 16 Spadla koruna s hlavy naší; beda nám již, že jsme hrešili.
\par 17 Protot jest mdlé srdce naše, pro tyt veci zatmely se oci naše,
\par 18 Pro horu Sion, že zpuštena jest; lišky chodí po ní.
\par 19 Ty Hospodine, na veky zustáváš, a stolice tvá od národu do pronárodu.
\par 20 Proc se zapomínáš na veky na nás, a opouštíš nás za tak dlouhé casy?
\par 21 Obrat nás, ó Hospodine, k sobe,a obráceni budeme; obnov dny naše, jakž byly za starodávna.
\par 22 Nebo zdali všelijak zavržeš nás, a hnevati se budeš na nás velice?

\end{document}