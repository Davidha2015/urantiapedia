\begin{document}

\title{Ezechiel}

\chapter{1}

\par 1 Stalo se pak tridcátého léta, ctvrtého mesíce, dne pátého, když jsem byl mezi zajatými u reky Chebar, že otevrína byla nebesa, a videl jsem videní Boží.
\par 2 Pátého dne téhož mesíce, pátého léta zajetí krále Joachina,
\par 3 V pravde stalo se slovo Hospodinovo k Ezechielovi knezi, synu Buzi, v zemi Kaldejské u reky Chebar, a byla nad ním ruka Hospodinova.
\par 4 I videl jsem, a aj, vítr tuhý pricházel od pulnoci, a oblak veliký, a ohen plápolající, a okolo neho byl blesk, a z prostredku jeho jako nejaká velmi prudká svetlost, z prostredku toho ohne.
\par 5 Z prostredku jeho také ukázalo se podobenství ctyr zvírat, jejichž takový byl zpusob: Podobenství cloveka meli.
\par 6 A po ctyrech tvárích jedno každé, a po ctyrech krídlích jedno každé melo.
\par 7 Jejichž nohy nohy prímé, ale zpodek noh jejich jako zpodek nohy telecí, a blyštely se podobne jako ocel pulerovaná.
\par 8 Ruce pak lidské pod krídly jejich, po ctyrech stranách jejich, a tvári jejich i krídla jejich na ctyrech tech stranách.
\par 9 Spojena byla krídla jejich jednoho s druhým. Neobracela se, když šla; jedno každé prímo na svou stranu šlo.
\par 10 Podobenství pak tvárí jejich s predu tvár lidská, a tvár lvová po pravé strane každého z nich; tvár pak volovou po levé strane všech ctvero, též tvár orlicí s zadu melo všech ctvero z nich.
\par 11 A tvári jejich i krídla jejich pozdvižena byla zhuru. Každé zvíre dve krídla pojilo s krídly dvema druhého, dvema pak prikrývala tela svá.
\par 12 A každé prímo na svou stranu šlo. Kamkoli ukazoval duch, aby šla, tam šla, neuchylovala se, když chodila.
\par 13 Podobnost také tech zvírat na pohledení byla jako uhlí reravého, na pohledení jako pochodne. Kterýžto ohen ustavicne chodil mezi zvíraty, a ten ohen mel blesk, a z téhož ohne vycházelo blýskání.
\par 14 Také ta zvírata behala, a navracovala se jako prudké blýskání.
\par 15 A když jsem hledel na ta zvírata, a aj, kolo jedno bylo na zemi pri zvíratech u ctyr tvárí jednoho každého z nich.
\par 16 Na pohledení byla kola, a udelání jich jako barva tarsis, a podobnost jednostejnou mela všecka ta kola, a byla na pohledení i udelání jejich, jako by bylo kolo u prostred kola.
\par 17 Na ctyri strany své jíti majíce, chodila, a neuchylovala se, když šla.
\par 18 A loukoti své, i vysokost mela, že hruza z nich šla, a šínové jejich vukol všech ctyr kol plní byli ocí.
\par 19 Když pak chodila zvírata, chodila kola podlé nich, a když se vznášela zvírata vzhuru od zeme, vznášela se i kola.
\par 20 Kdekoli chtel Duch, aby šla, tam šla; kde Duch chtel jíti, i kola vznášela se naproti nim, nebo duch zvírat byl v kolách.
\par 21 Když ona šla, šla, a když ona stála, stála, a když se vznášela od zeme, vznášela se také kola s nimi, nebo duch zvírat byl v kolách.
\par 22 Podobenství pak oblohy bylo nad hlavami zvírat jako podobenství krištálu roztaženého nad hlavami jejich svrchu.
\par 23 A pod oblohou krídla jejich pozdvižená byla, jedno pripojené k druhému. Každé melo dve, jimiž se prikrývalo, každé, pravím, melo dve, jimiž prikrývalo telo své.
\par 24 I slyšel jsem zvuk krídel jejich jako zvuk vod mnohých, jako zvuk Všemohoucího, když chodila, zvuk hluku jako zvuk vojska. Když pak stála, spustila krídla svá.
\par 25 Byl také zvuk svrchu nad oblohou, kteráž byla nad hlavou jejich, když stála a spustila krídla svá.
\par 26 Svrchu pak na obloze, kteráž byla nad hlavou jejich, bylo podobenství trunu, na pohledení jako kámen zafirový, a nad podobenstvím trunu na nem svrchu, na pohledení jako tvárnost cloveka.
\par 27 I videl jsem na pohledení jako velmi prudkou svetlost, a u vnitrku jejím vukol na pohledení jako ohen, od bedr jeho vzhuru; od bedr pak jeho dolu videl jsem na pohledení jako ohen, a blesk vukol neho.
\par 28 Na pohledení jako duha, kteráž bývá na oblace v cas dešte, takový na pohledení byl blesk vukol. To bylo videní podobenství slávy Hospodinovy. Kteréžto videv, padl jsem na tvár svou, a slyšel jsem hlas mluvícího.

\chapter{2}

\par 1 Kterýž rekl ke mne: Synu clovecí, postav se na nohy své, at mluvím s tebou.
\par 2 I vstoupil do mne duch, když promluvil ke mne, a postavil mne na nohy mé, a slyšel jsem, an mluví ke mne.
\par 3 Kterýž rekl mi: Synu clovecí, já te posílám k synum Izraelským, k národum zpurným, kteríž zpurne se postavovali proti mne; oni i otcové jejich zproneverovali se mi, až práve do tohoto dne.
\par 4 K tech, pravím, synum nestydaté tvári a zatvrdilého srdce já posílám te, a díš k nim: Takto praví Panovník Hospodin,
\par 5 Již oni slyšte neb nechte: Že dum zpurný jsou. At vedí, že prorok byl u prostred nich.
\par 6 Ty pak synu clovecí, neboj se jich, aniž se boj slov jejich, že zpurní a jako trní jsou proti tobe, a že mezi štíry bydlíš. Slov jejich neboj se, a tvári jejich se nestrachuj, proto že dum zpurný jsou.
\par 7 Ale mluv slova má k nim, již oni slyšte neb nechte: Že zpurní jsou.
\par 8 Ty pak synu clovecí, slyš, co já pravím tobe: Nebud zpurný jako ten dum zpurný. Otevri ústa svá, a snez, co já tobe dám.
\par 9 I videl jsem, a aj, ruka vztažena byla ke mne, a aj, v ní svinutá kniha.
\par 10 Kteroužto rozvinul prede mnou, a byla popsaná s predu i z zadu, a bylo v ní psáno naríkání, kvílení a beda.

\chapter{3}

\par 1 Tedy rekl mi: Synu clovecí, což pred tebou jest, snez, snez knihu tuto, a jdi, mluv k domu Izraelskému.
\par 2 I otevrel jsem ústa svá, a dal mi snísti knihu tu,
\par 3 Rka ke mne: Synu clovecí, nakrm bricho své, a streva svá napln knihou touto, kteroužt dávám. I snedl jsem, a byla v ústech mých jako med sladká.
\par 4 Za tím rekl mi: Synu clovecí, jdiž k domu Izraelskému, a mluv k nim slovy mými.
\par 5 Nebo nebudeš poslán k lidu hluboké reci a nesnadného jazyka, ale k domu Izraelskému.
\par 6 Ne k národum mnohým hluboké reci a nesnadného jazyka, jejichž bys slovum nerozumel, ješto, kdybych te k nim poslal, uposlechli by tebe.
\par 7 Ale dum Izraelský nebudou te chtíti poslouchati, ponevadž nechtí poslouchati mne; nebo všecken dum Izraelský jest tvrdocelný a zatvrdilého srdce.
\par 8 Ale ucinil jsem tvár tvou tvrdou proti tvári jejich, a celo tvé tvrdé proti celu jejich.
\par 9 Jako kámen pretvrdý, pevnejší než skálu ucinil jsem celo tvé; nebojž se jich, aniž se strachuj tvári jejich, proto že dum zpurný jsou.
\par 10 I rekl ke mne: Synu clovecí, všecka slova má, kterážt mluviti budu, prijmi v srdce své, a ušima svýma slyš.
\par 11 A jdi k zajatým, k synum lidu svého, a mluv k nim, a rci jim: Takto praví Panovník Hospodin, již oni slyšte neb nechte.
\par 12 Tehdy odnesl mne Duch, a slyšel jsem za sebou hlas hrmotu velikého: Požehnaná sláva Hospodinova z místa svého.
\par 13 A hlas krídel tech zvírat, kteráž se vespolek dotýkala, a hlas kol naproti nim, a hlas hrmotu velikého.
\par 14 Duch pak odnesl mne, a vzal mne, a odšel jsem truchliv, v hneve ducha svého, ale ruka Hospodinova nade mnou silnejší byla.
\par 15 I prišel jsem k zajatým do Telabib, bydlícím pri rece Chebar, a sedel jsem, kdež oni bydlili. Sedel jsem, pravím, sedm dní u prostred nich s užasnutím.
\par 16 I stalo se po dokonání sedmi dnu, že se stalo slovo Hospodinovo ke mne, rkoucí:
\par 17 Synu clovecí, strážným jsem te postavil nad domem Izraelským, abys slyše slovo z úst mých, napomínal jich ode mne.
\par 18 Když bych já rekl bezbožnému: Smrtí umreš, a nenapomenul bys ho, ani nemluvil, abys ho odvedl od cesty jeho bezbožné, proto abys ho pri životu zachoval: ten bezbožný pro nepravost svou umre, ale krve jeho z ruky tvé vyhledám.
\par 19 Paklibys ty napomenul bezbožného, a neodvrátil by se od bezbožnosti své, a od cesty své bezbožné, ont pro nepravost svou umre, ale ty duši svou vysvobodíš.
\par 20 Odvrátil-li by se pak spravedlivý od spravedlnosti své, a cinil by nepravost, a já bych položil urážku pred nej, a tak by umrel, ty pak bys ho nenapomenul: pro hrícht svuj umre, aniž na pamet prijde která spravedlnost jeho, kterouž cinil, ale krve jeho z ruky tvé vyhledám.
\par 21 Pakli bys ty napomenul spravedlivého, aby nehrešil spravedlivý, a on by nehrešil, jiste že bude živ; nebo napomenut byl. Ty také duši svou vysvobodíš.
\par 22 I byla tam nade mnou ruka Hospodinova, kterýžto rekl mi: Vstan, jdi do tohoto údolí, a tam mluviti budu s tebou.
\par 23 A tak vstav, šel jsem do toho údolí, a aj, sláva Hospodinova stála tam, jako sláva, kterouž jsem videl u reky Chebar. I padl jsem na tvár svou.
\par 24 Tehdy vstoupil do mne Duch, a postaviv mne na nohy, mluvil ke mne, a rekl mi: Jdiž, zavri se v dome svém.
\par 25 Nebo na te, synu clovecí, aj, dadí na te provazy, a sváží te jimi, a nebudeš moci vyjíti mezi ne.
\par 26 A já uciním, aby jazyk tvuj prilnul k dásním tvým, a abys onemel, a nebyl jim mužem domlouvajícím, protože dum zpurný jsou.
\par 27 Ale když mluviti budu s tebou, otevru ústa tvá, a díš jim: Takto praví Panovník Hospodin: Kdo slyšeti chce, necht slyší, a kdo nechce, necht nechá, že dum zpurný jsou.

\chapter{4}

\par 1 Ty pak synu clovecí, vezmi sobe cihlu, a polože ji pred sebe, vyrej na ní mesto Jeruzalém.
\par 2 A postav na ní obležení, a vzdelaje na ní šance, vysyp na ní násyp, a polož na ní vojska, a postav na ní berany válecné vukol.
\par 3 Potom vezmi sobe pánev železnou, a polož ji místo zdi železné, mezi tebou a mezi mestem, a zatvrd tvár svou proti nemu, at jest obleženo, a oblehneš je. Tot bude znamením domu Izraelskému.
\par 4 Ty pak lehni na levý bok svuj, a vlož na nej nepravost domu Izraelského. Podlé poctu dnu, v nemž ležeti budeš na nem, poneseš nepravost jejich.
\par 5 A já dávám tobe léta nepravosti jejich v poctu dnu, tri sta a devadesáte dnu, v nichž poneseš nepravost domu Izraelského.
\par 6 Když je pak vyplníš, budeš ležeti na pravém boku podruhé, a poneseš nepravost domu Judova ctyridceti dnu. Den za rok, den za rok dávám tobe.
\par 7 K obležení, pravím, Jeruzaléma zatvrd tvár svou, ohrna ruku svou, a prorokuje proti nemu.
\par 8 A aj, dávám na te provazy, abys se neobracel z boku jednoho na druhý, dokudž nevyplníš dnu obležení svého.
\par 9 Protož ty vezmi sobe pšenice a jecmene, též bobu, šocovice, i prosa, a špaldy, a dej to do jedné nádoby, abys sobe nastrojil z toho pokrmu podlé poctu dnu, v nichž ležeti budeš na boku svém. Za tri sta a devadesáte dnu jísti jej budeš.
\par 10 Pokrmu pak tvého, kterýž jísti budeš, váha bude dvadceti lotu na den. Od casu až do casu jísti jej budeš.
\par 11 Vodu také na míru píti budeš, šestý díl hin; od casu do casu píti budeš.
\par 12 Podpopelný pak chléb jecný, kterýž jísti budeš, ten lejny necistoty lidské pec pred ocima jejich.
\par 13 I rekl Hospodin: Tak budou jísti synové Izraelští chléb svuj necistý pro pohany, kteréž tam shromáždím.
\par 14 Tedy rekl jsem: Ach, Panovníce Hospodine, aj, duše má není poškvrnena mrchami, a udáveného nejedl jsem od detinství svého až podnes, aniž vešlo v ústa má maso ohavné.
\par 15 Kterýž rekl mi: Aj, dávámt kravince místo lejn lidských, abys sobe jimi napekl chleba.
\par 16 Za tím rekl mi: Synu clovecí, aj, já zlámi hul chleba v Jeruzaléme, tak že jísti budou chléb na váhu, a to s zámutkem, a vodu na míru píti, a to s predešením,
\par 17 Aby nedostatek majíce v chlebe a v vode, desili se jeden každý z nich, a svadli pro nepravost svou.

\chapter{5}

\par 1 Potom ty synu clovecí, vezmi sobe nuž ostrý, totiž britvu holicu, vezmi jej sobe, a ohol ním hlavu i bradu svou. Potom vezma sobe váhu, rozdel to.
\par 2 Tretinu ohnem spal u prostred mesta, když se vyplní dnové obležení; zatím vezma druhou tretinu, posekej mecem okolo neho; ostatní pak tretinu rozptyl u vítr. Nebo mecem dobytým budu je stihati.
\par 3 A však odejmi odtud neco málo, a zavaž do krídel svých.
\par 4 A i z tech ješte vezma, uvrz je do prostred ohne, a spal je ohnem, odkudž vyjde ohen na všecken dum Izraelský.
\par 5 Takto praví Panovník Hospodin: Tento Jeruzalém, kterýž jsem postavil u prostred pohanu, a vukol otocil krajinami,
\par 6 Zmenil soudy mé v bezbožnost více než pohané, a ustanovení má více než jiné zeme, kteréž jsou vukol neho; nebo soudy mými pohrdli, a v ustanoveních mých nechodili.
\par 7 Protož takto praví Panovník Hospodin: Proto že mnohem více než pohané, kteríž jsou vukol vás, v ustanoveních mých nechodili jste, a soudu mých necinili jste, nýbrž ani tak jako pohané, kteríž jsou vukol vás, soudu nekonali jste:
\par 8 Protož takto praví Panovník Hospodin: Aj, já na tebe, aj já, a vykonám u prostred tebe soudy pred ocima pohanu.
\par 9 Nebo uciním pri tobe to, cehož jsem prv neucinil, a cehož podobne neuciním více pro všecky ohavnosti tvé,
\par 10 Tak že otcové jísti budou syny u prostred tebe, a synové jísti budou otce své, a vykonám proti tobe soudy, a rozptýlím všecky ostatky tvé na všecky strany.
\par 11 Protož živt jsem já, praví Panovník Hospodin, že ponevadž jsi ty svatyne mé poškvrnil všelikými mrzkostmi svými, a všelikými ohavnostmi svými, i já také zlehcím tebe, a neodpustít oko mé, a nikoli se neslituji.
\par 12 Tretina tebe morem zemre a hladem zhyne u prostred tebe, a tretina druhá mecem padne vukol tebe, ostatní pak tretinu na všecky stany rozptýlím, a mecem dobytým stihati je budu.
\par 13 A tak do konce vylit bude hnev muj, a dotru prchlivostí svou na ne, i poteším se. I zvedít, že já Hospodin mluvil jsem v horlivosti své, když vykonám prchlivost svou na nich,
\par 14 A obrátím te v poušt, a dám te v útržku mezi národy, kteríž jsou vukol tebe, pred ocima každého tudy jdoucího.
\par 15 A tak budeš k útržce, posmechu, k hroznému príkladu a k užasnutí národum, kteríž jsou vukol tebe, tehdáž když vykonám proti tobe soudy v hneve a v prchlivosti a v žehrání zurivém. Já Hospodin mluvil jsem.
\par 16 Tehdáš když vystrelím jízlivé strely hladu k záhube vaší, kteréž vystrelím, abych vás vyhubil, a hlad shromážde proti vám, zlámi vám hul chleba.
\par 17 Pošli zajisté na vás hlad a zver lítou, kteráž te na sirobu privede; i mor a krev prijde na tebe, když uvedu na te mec. Já Hospodin mluvil jsem.

\chapter{6}

\par 1 I stalo se ke mne slovo Hospodinovo, rkoucí:
\par 2 Synu clovecí, obrat tvár svou proti horám Izraelským, a prorokuj proti nim,
\par 3 A rci: Hory Izraelské, slyšte slovo Panovníka Hospodina: Takto praví Panovník Hospodin horám a pahrbkum, prudkým potokum a údolím: Aj, já, já uvedu na vás mec, a zkazím výsosti vaše.
\par 4 A tak zpustnou oltárové vaši, a ztroskotáni budou slunecní obrazové vaši, a rozmeci zbité vaše pred ukydanými bohy vašimi.
\par 5 Povrhu také mrtvá tela synu Izraelských pred ukydanými bohy jejich, a rozptýlím kosti vaše okolo oltáru vašich.
\par 6 Kdekoli bydliti budete, mesta zpuštena budou, a výsosti zpustnou. Procež popléneni budou a zpustnou i oltárové vaši, potroskotáni budou a prestanou ukydaní bohové vaši, a slunecní obrazové vaši zpodtínáni, a tak vyhlazena budou díla vaše.
\par 7 I padnou zbití u prostred vás, a zvíte, že já jsem Hospodin.
\par 8 A pozustavím nekteré z vás, kteríž by ušli mece, mezi pohany, když rozptýleni budete do zemí.
\par 9 I budou se rozpomínati na mne, kteríž z vás zachováni budou mezi národy, mezi než budou zajati, že jsem kormoucen byl srdcem jejich smilným, kteréž odstoupilo ode mne, a ocima jejich, kteréž smilníce, chodily za ukydanými bohy svými, a tak sami se býti hodné ošklivosti seznají pro nešlechetnosti, kteréž páchali ve všech ohavnostech svých.
\par 10 A zvedí, že já jsem Hospodin, a že jsem ne nadarmo mluvil, že na ne uvedu zlé toto.
\par 11 Takto praví Panovník Hospodin: Tleskni rukou svou, a dupni nohou svou, a rci: Nastojte na dum Izraelský, že pro všecky ohavnosti nejhorší mecem, hladem a morem padnouti mají.
\par 12 Ten, kdož daleko bude, morem umre, a kdo blízko, mecem padne, ostatní pak a obležený hladem umre, a tak docela vyleji prchlivost svou na ne.
\par 13 A zvíte, že já jsem Hospodin, když budou zbití jejich u prostred ukydaných bohu jejich vukol oltáru jejich, na všelikém pahrbku vysokém, na všech vrších hor, a pod všelikým drevem zeleným, i pod všelikým dubem hustým, na kterémkoli míste obetovávali vuni libou všelikým ukydaným bohum svým.
\par 14 Nebo vztáhnu ruku svou na ne, a uciním zemi tuto zpustlou, zpustlejší než poušt Diblat, po všech obydlích jejich. I zvedí, že já jsem Hospodin.

\chapter{7}

\par 1 Potom stalo se slovo Hospodinovo ke mne, rkoucí:
\par 2 Slyš, synu clovecí: Takto praví Panovník Hospodin o zemi Izraelské: Ach, skoncení, prišlo skoncení její na ctyri strany zeme.
\par 3 Jižte skoncení nastalo tobe; nebo vypustím hnev svuj na te, a budu te souditi podlé cest tvých, a uvrhu na te všecky ohavnosti tvé.
\par 4 Neodpustí zajisté oko mé tobe, aniž se slituji, ale cesty tvé na tebe uvrhu, a ohavnosti tvé u prostred tebe budou. I zvíte, že já jsem Hospodin.
\par 5 Takto praví Panovník Hospodin: Bída jedna, aj hle, prichází bída.
\par 6 Skoncení prichází, prichází skoncení, procítil na te, aj, pricházít.
\par 7 Pricházít to jitro na tebe, obyvateli zeme, pricházít ten cas, približuje se ten den hrmotu, a ne hlas rozléhající se po horách.
\par 8 Již tudíž vyliji prchlivost svou na te, a docela vypustím hnev svuj na tebe, a budu te souditi podlé cest tvých, a uvrhu na te všecky ohavnosti tvé.
\par 9 Neodpustít zajisté oko mé, aniž se slituji, ale podlé cest tvých zaplatím tobe, a budou ohavnosti tvé u prostred tebe. A tak zvíte, že já jsem Hospodin, ten, kterýž biji.
\par 10 Aj, ted jest ten den, aj, pricházeje,prišlo to jitro, zkvetl prut, vypucila se z neho pýcha,
\par 11 Ukrutnost vzrostla v prut bezbožnosti. Nezustanet z nich nic, ani z množství jejich, ani z hluku jejich, aniž bude naríkáno nad nimi.
\par 12 Pricházít ten cas, blízko jest ten den. Kdo koupí, nebude se veseliti, a kdo prodá, nic toho nebude litovati; nebo prchlivost prijde na všecko množství její.
\par 13 Ten zajisté, kdož prodal, k veci prodané nepujde nazpet, by pak ješte byl mezi živými život jejich, ponevadž videní na všecko množství její nenavrátí se, a žádný v nepravosti života svého nezmocní se.
\par 14 Troubiti budou v troubu, a pripraví všecko, ale nebude žádného, kdo by šel k boji; nebo prchlivost má oborí se na všecko množství její.
\par 15 Mec bude vne, mor pak a hlad doma; kdo bude na poli, mecem zabit bude, toho pak, kdož bude v meste, hlad a mor zahubí.
\par 16 Kteríž pak z nich utekou, ti na horách, jako holubice v údolí, všickni lkáti budou, jeden každý pro nepravost svou.
\par 17 Všeliké ruce klesnou, a všeliká kolena rozplynou se jako voda.
\par 18 I prepáší se žínemi, a hruza je prikryje, a na všeliké tvári bude stud, a na všech hlavách jejich lysina.
\par 19 Stríbro své po ulicích rozházejí, a zlato jejich bude jako necistota; stríbro jejich a zlato jejich nebude jich moci vysvoboditi v den prchlivosti Hospodinovy. Nenasytí se, a strev svých nenaplní, proto že nepravost jejich jest jim k urážce,
\par 20 A že v sláve okrasy své, v té, kterouž k dustojnosti postavil Buh, obrazu ohavností svých a mrzkostí svých nadelali. Protož obrátil jsem jim ji v necistotu.
\par 21 Nebo vydám ji v ruku cizozemcu k rozchvátání, a bezbožným na zemi za loupež, kteríž poškvrní jí.
\par 22 Odvrátím též tvár svou od nich, i poškvrní svatyne mé, a vejdou do ní, boríce a poškvrnujíce jí.
\par 23 Udelej retez, nebo zeme plná jest soudu ukrutných, a mesto plné jest nátisku.
\par 24 Protož privedu nejhorší z pohanu, aby dedicne vládli domy jejich; i uciním prítrž pýše silných, a poškvrneny budou, kteríž jich posvecují.
\par 25 Zkažení prišlo, protož hledati budou pokoje, ale žádného nebude.
\par 26 Bída za bídou prijde, a novina bude za novinou, i budou hledati videní od proroka, ale zákon zahyne od kneze, a rada od starcu.
\par 27 Král ustavicne kvíliti bude, a kníže oblece se v smutek, a ruce lidu v zemi predešeny budou. Podlé cesty jejich uciním jim, a podlé soudu jejich souditi je budu. I zvedí, že já jsem Hospodin.

\chapter{8}

\par 1 I stalo se léta šestého, v pátý den šestého mesíce, že jsem sedel v dome svém, a starší Judští sedeli prede mnou. I pripadla na mne tu ruka Panovníka Hospodina.
\par 2 A videl jsem, a aj, podobenství na pohledení jako ohen. Od bedr jeho dolu ohen, od bedr pak jeho vzhuru na pohledení jako blesk, na pohledení jako nejaká velmi prudká svetlost.
\par 3 Tedy vztáh podobenství ruky, vzal mne za kštici hlavy mé, a vyzdvihl mne Duch mezi nebe a mezi zemi, a uvedl mne do Jeruzaléma u videních Božích, k vratum brány vnitrní, kteráž patrí na pulnoci, kdež byla stolice modly k horlivosti a k zurivosti popouzející.
\par 4 A aj, sláva Boha Izraelského byla tam na pohledení jako ta, kterouž jsem videl v údolí.
\par 5 I rekl mi: Synu clovecí, pozdvihni nyní ocí svých k ceste na pulnoci. Tedy pozdvihl jsem oci svých k ceste na pulnoci, a aj, na pulnoci u brány oltárové ta modla horlení, práve kudyž se vchází.
\par 6 V tom rekl mi: Synu clovecí, vidíš-liž ty, co tito ciní, ohavnosti tak veliké, kteréž ciní dum Izraelský tuto, tak že se vzdáliti musím od svatyne své? Ale obráte se, uzríš ješte vetší ohavnosti.
\par 7 I privedl mne ke dverum síne, kdež jsem uzrel, a aj, díra jedna byla v stene.
\par 8 A rekl mi: Synu clovecí, kopej medle tu stenu. I kopal jsem stenu, a aj, dvére jedny.
\par 9 Tedy rekl mi: Vejdi, a viz ohavnosti tyto nejhorší, kteréž oni ciní zde.
\par 10 Protož všed, uzrel jsem, a aj, všeliké podobenství zemeplazu a hovad ohyzdných, i všech ukydaných bohu domu Izraelského vyryto bylo na stene vukol a vukol.
\par 11 A sedmdesáte mužu z starších domu Izraelského, s Jazaniášem synem Safanovým, stojícím u prostred nich, stáli pred nimi, maje každý kadidlnici svou v ruce své, tak že hustý oblak kadení vzhuru vstupoval.
\par 12 I rekl mi: Videl-lis, synu clovecí, co starší domu Izraelského ciní ve tme, jeden každý v pokojích svých malovaných? Nebo ríkají: Nikoli na nás nepatrí Hospodin, opustil Hospodin zemi.
\par 13 Dále mi rekl: Obráte se, uzríš ješte vetší ohavnosti, kteréž oni ciní.
\par 14 I privedl mne k vratum brány domu Hospodinova, kteráž jest na pulnoci, a aj, ženy sedely tam, placíce Tammuze.
\par 15 I rekl mi: Videl-lis, synu clovecí? Obráte se, uzríš ješte vetší ohavnosti nad tyto.
\par 16 Tedy uvedl mne do síne domu Hospodinova vnitrní, a aj, u vrat chrámu Hospodinova, mezi síncí a oltárem bylo okolo petmecítma mužu, jejichž záda byla k chrámu Hospodinovu, tvári pak jejich k východu, kteríž klaneli se proti východu slunce.
\par 17 I rekl mi: Videl-lis, synu clovecí? Zdali lehká vec jest domu Judovu, aby cinili ohavnosti tyto, kteréž ciní zde? Nebo naplnivše zemi nátiskem,obrátili se, aby mne popouzeli, a aj, pricinejí ratolest vinnou k nosum svým.
\par 18 Protož i já také uciním podlé prchlivosti; neslitujet se oko mé, aniž se smiluji. I budou volati v uši mé hlasem velikým, a nevyslyším jich.

\chapter{9}

\par 1 Potom zavolal hlasem velikým, tak že jsem slyšel, rka: Pristupte hejtmané k tomuto mestu, a jeden každý s zbrojí svou hubící v ruce své.
\par 2 A aj, šest mužu prišlo cestou k bráne horejší, kteráž patrí na pulnoci, maje každý zbroj svou rozrážející v ruce své. Muž pak jeden byl u prostred nich, odený rouchem lneným, a kalamár písarský pri bedrách jeho; a prišedše, stáli u oltáre medeného.
\par 3 Sláva pak Boha Izraelského sstoupila byla s cherubína, na kterémž byla, k prahu domu, a zvolala na muže toho odeného rouchem lneným, pri jehož bedrách byl kalamár písarský.
\par 4 I rekl jemu Hospodin: Prejdi prostredkem mesta, prostredkem Jeruzaléma, a znamenej znamením na celích muže ty, kteríž vzdychají a naríkají nade všemi ohavnostmi dejícími se u prostred neho.
\par 5 Onemno pak rekl tak, že jsem slyšel: Projdete skrze mesto za ním, a bíte; neodpouštejž oko vaše, aniž se slitovávejte.
\par 6 Starce, mládence i pannu, malické i ženy mordujte do vyhubení, ale ke všelikému muži, na nemž by bylo znamení, nepristupujte, a od svatyne mé pocnete. Takž zacali od mužu tech starších, kteríž byli pred chrámem.
\par 7 (Nebo jim byl rekl: Poškvrnte domu, a naplnte síne zbitými. Jdetež.) A vyšedše, bili v meste.
\par 8 I stalo se, když je bili, a já pozustal, že jsem padl na tvár svou, a zvolal jsem, rka: Ach, Panovníce Hospodine, zdaliž zahubíš všecken ostatek Izraelský, vylévaje prchlivost svou na Jeruzalém?
\par 9 I rekl mi: Nepravost domu Izraelského a Judského veliká jest velmi velice, a naplnena jest zeme mordy, a mesto plné jest prevrácencu. Nebo ríkali: Opustil Hospodin zemi tuto, a Hospodin nikoli nevidí nás.
\par 10 Protož já také cestu jejich na hlavu jejich obrátím; neodpustí oko mé, aniž se slituji.
\par 11 A aj, muž odený rouchem lneným, pri jehož bedrách byl kalamár, oznámil to, rka: Ucinil jsem, jakž jsi mi rozkázal.

\chapter{10}

\par 1 I videl jsem, a aj, na obloze, kteráž byla nad hlavou cherubínu, jako kámen zafirový na pohledení, jako podobenství trunu se ukázalo nad nimi.
\par 2 Tedy promluviv k muži tomu odenému rouchem lneným, rekl: Vejdi do prostred kol pod cherubíny, a napln hrsti své uhlím reravým z prostredku cherubínu, a roztrus po meste. Kterýžto všel pred ocima mýma.
\par 3 (Cherubínové pak stáli po pravé strane domu, když vcházel muž ten, a oblak hustý naplnil sín vnitrní.
\par 4 Nebo když se byla zdvihla sláva Hospodinova s cherubínu k prahu domu, tedy naplnen byl dum tím hustým oblakem, a sín naplnena byla bleskem slávy Hospodinovy.
\par 5 Zvuk také krídel cherubínu slyšán byl až k té síni zevnitrní jako hlas Boha silného, všemohoucího, když mluví.)
\par 6 I stalo se, když prikázal muži tomu odenému rouchem lneným, rka: Vezmi ohne z prostredku kol, z prostredku cherubínu, že všel a postavil se podlé kol.
\par 7 Tedy vztáhl cherubín jeden ruku svou z prostredku cherubínu k ohni tomu, kterýž byl u prostred cherubínu, a vzav, dal do hrsti toho odeného rouchem lneným. Kterýžto vzal a vyšel.
\par 8 Nebo ukazovalo se na cherubíních podobenství ruky lidské pod krídly jejich.
\par 9 I videl jsem, a aj, ctyri kola podlé cherubínu, jedno kolo podlé jednoho každého cherubína, podobenství pak kol jako barva kamene tarsis.
\par 10 A na pohledení mela podobnost jednostejnou ta kola, jako by bylo kolo u prostred kola.
\par 11 Když chodili, na ctyri strany jejich chodili. Neuchylovali se, když šli, ale k tomu místu, kamž se obracel vudce, za ním šli. Neuchylovali se, když šli.
\par 12 Všecko také telo jejich i hrbetové jejich, i ruce jejich, i krídla jejich, též i kola plná byla ocí vukol jich samých ctyr i kol jejich.
\par 13 Kola pak ta nazval okršlkem, jakž slyšely uši mé.
\par 14 Ctyri tvári melo každé. Tvár první tvár cherubínová, tvár pak druhá tvár clovecí, a tretí tvár lvová, ctvrtá pak tvár orlicí.
\par 15 I zdvihli se cherubínové. To jsou ta zvírata, kteráž jsem videl u reky Chebar.
\par 16 Když pak šli cherubínové, šla i kola podlé nich, a když vznášeli cherubínové krídla svá, aby se pozdvihli od zeme, neuchylovala se také kola od nich.
\par 17 Když stáli oni, stála, a když se vyzdvihovali, vyzdvihovala se s nimi; nebo duch zvírat byl v nich.
\par 18 I odešla sláva Hospodinova od prahu domu, a stála nad cherubíny,
\par 19 Hned jakž pozdvihli cherubínové krídel svých a vznesli se od zeme, pred ocima mýma odcházejíce, a kola naproti nim, a stála u vrat brány domu Hospodinova východní, a sláva Boha Izraelského svrchu nad nimi.
\par 20 To jsou ta zvírata, kteráž jsem videl pod Bohem Izraelským u reky Chebar, a poznal jsem, že cherubínové byli.
\par 21 Po ctyrech tvárích mel jeden každý, a po ctyrech krídlích jeden každý; podobenství také rukou lidských pod krídly jejich.
\par 22 Podobenství pak tvárí jejich bylo jako tvárí, kteréž jsem byl videl u reky Chebar, oblícej jejich i oni sami. Jeden každý prímo k své strane chodil.

\chapter{11}

\par 1 I vznesl mne Duch, a privedl mne k bráne východní domu Hospodinova, kteráž patrí na východ, a aj, ve vratech brány té petmecítma mužu, mezi kterýmiž jsem videl Jazaniáše syna Azurova, a Pelatiáše syna Banaiášova, knížata lidu.
\par 2 Tedy rekl mi: Synu clovecí, to jsou ti muži, kteríž smýšlejí nepravost, a skládají radu zlou v meste tomto,
\par 3 Ríkajíce: Nestavejme domu blízko, sic mesto bude hrnec a my maso.
\par 4 Protož prorokuj proti nim, prorokuj, synu clovecí.
\par 5 I sstoupil na mne Duch Hospodinuv, a rekl mi: Rci: Takto dí Hospodin: Tak ríkáte, dome Izraelský. Nebo což vám koli vstupuje na mysl, o tom já vím.
\par 6 Veliké množství zmordovali jste v meste tomto, a naplnili jste ulice jeho zbitými.
\par 7 Protož takto praví Panovník Hospodin: Ti, kteríž jsou zbiti od vás, kteréž jste skladli u prostred neho, onit jsou maso, mesto pak hrnec, ale vás vyvedu z prostredku jeho.
\par 8 Báli jste se mece, ale mec uvedu na vás, praví Panovník Hospodin.
\par 9 A vyvedu vás z prostredku jeho, a dám vás v ruku cizích, a vykonám nad vámi soudy.
\par 10 Mecem padnete, na pomezí Izraelském souditi vás budu, a zvíte, že já jsem Hospodin.
\par 11 Mesto nebude vám hrnec, aniž vy budete u prostred neho maso, na pomezí Izraelském souditi vás budu.
\par 12 I zvíte, že já jsem Hospodin, ponevadž jste v ustanoveních mých nechodili, a soudu mých necinili, a podlé soudu tech národu, kteríž jsou vukol vás, cinili jste.
\par 13 Stalo se pak, když jsem prorokoval, že Pelatiáš syn Banaiášuv umrel. Procež padl jsem na tvár svou, a zvolal jsem hlasem velikým, rka: Ach, Panovníce Hospodine, skonání, ciníš ostatkum Izraelským.
\par 14 Tedy stalo se slovo Hospodinovo ke mne, rkoucí:
\par 15 Synu clovecí, bratrí tvoji, bratrí tvoji, príbuzní tvoji, a všecken dum Izraelský, všecken dum, kterýmž ríkali obyvatelé Jeruzalémští: Daleko zajdete od Hospodina, nám jest dána zeme tato v dedictví.
\par 16 Protož rci: Takto praví Panovník Hospodin: Ackoli daleko zahnal jsem je mezi národy, a ackoli rozptýlil jsem je do zemí, však budu jim svatyní i za ten malý cas v zemích tech, do kterýchž prijdou.
\par 17 Protož rci: Takto praví Panovník Hospodin: Shromáždím vás z národu a zberu vás z zemí, do kterýchž rozptýleni jste, a dám vám zemi Izraelskou.
\par 18 I vejdou tam, a vyvrhou všecky mrzkosti její, i všecky ohavnosti její z ní.
\par 19 Nebo dám jim srdce jedno, a Ducha nového dám do vnitrností vašich, a odejmu srdce kamenné z tela jejich, a dám jim srdce masité,
\par 20 Aby v ustanoveních mých chodili, a soudu mých ostríhali, a cinili je. I budou lidem mým, a já budu jejich Bohem.
\par 21 Kterýchž pak srdce chodilo by po žádostech mrzkostí svých a ohavností svých, tech cestu na hlavu jejich obrátím, praví Panovník Hospodin.
\par 22 Tedy vznesli cherubínové krídla svá i kola s nimi, sláva pak Boha Izraelského nad nimi svrchu.
\par 23 I odešla sláva Hospodinova z prostredku mesta, a stála na hore, kteráž jest na východ mestu.
\par 24 Duch pak vznesl mne, a zase mne privedl do zeme Kaldejské k zajatým, u videní skrze Ducha Božího. I odešlo ode mne videní, kteréž jsem videl.
\par 25 A mluvil jsem k zajatým všecky ty veci Hospodinovy, kteréž mi ukázal.

\chapter{12}

\par 1 I stalo se slovo Hospodinovo ke mne, rkoucí:
\par 2 Synu clovecí, u prostred domu zpurného ty bydlíš, kteríž mají oci, aby videli, však nevidí; uši mají, aby slyšeli, však neslyší, proto že dum zpurný jsou.
\par 3 Protož ty, synu clovecí, priprav sobe to, s cím bys se stehoval, a stehuj se ve dne pred ocima jejich. Prestehuješ se pak z místa svého na místo jiné pred ocima jejich, zdaby aspon videli; nebo dum zpurný jsou.
\par 4 Vynesa pak své veci, jakožto ty, s nimiž se stehovati máš ve dne pred ocima jejich, vyjdi u vecer pred ocima jejich, jako ti, kteríž se stehují.
\par 5 Pred ocima jejich prokopej sobe zed, a vynes skrze ni.
\par 6 Pred ocima jejich na rameni nes, po tme vynes, tvár svou prikrej, a nehled na zemi; nebo za zázrak dal jsem te domu Izraelskému.
\par 7 I ucinil jsem tak, jakž rozkázáno bylo. Veci své vynesl jsem, jakožto ty, s nimiž bych se stehoval ve dne, u vecer pak prokopal jsem sobe zed rukou; po tme jsem je vynesl, na rameni nesa pred ocima jejich.
\par 8 Opet stalo se slovo Hospodinovo ke mne ráno, rkoucí:
\par 9 Synu clovecí, zdaližt rekli dum Izraelský, dum ten zpurný: Co ty deláš?
\par 10 Rciž jim: Takto praví Panovník Hospodin: Na kníže v Jeruzaléme vztahuje se bríme toto a na všecken dum Izraelský, kteríž jsou u prostred neho.
\par 11 Rciž jim: Já jsem zázrakem vaším. Jakož jsem cinil, tak se stane jim, postehují se a v zajetí pujdou.
\par 12 A kníže, kteréž jest u prostred nich, na rameni ponese po tme a vyjde. Zed prokopají, aby jej vyvedli skrze ni; tvár svou zakryje, tak že nebude videti okem svým zeme.
\par 13 Nebo roztáhnu sít svou na nej, a polapen bude do vrše mé, a zavedu jej do Babylona, zeme Kaldejské, kteréž neuzrí, a tam umre.
\par 14 Všecky také, kteríž jsou vukol neho na pomoc jemu, i všecky houfy jeho rozptýlím na všecky strany, a mecem dobytým budu je stihati.
\par 15 I zvedí, že já jsem Hospodin, když je rozptýlím mezi národy, a rozženu je po krajinách.
\par 16 Pozustavím pak z nich muže nemnohé po meci, po hladu a po moru, aby vypravovali všecky ohavnosti své mezi národy, kamž se dostanou, i zvedí, že já jsem Hospodin.
\par 17 Opet stalo se slovo Hospodinovo ke mne, rkoucí:
\par 18 Synu clovecí, chléb svuj s strachem jez, a vodu svou s tresením a s zámutkem pí,
\par 19 A rci lidu zeme této: Takto praví Panovník Hospodin o obyvatelích Jeruzalémských, o zemi Izraelské: Chléb svuj s zámutkem jísti budou, a vodu svou s predešením píti, aby obloupena byla zeme jeho z hojnosti své, pro nátisk všech prebývajících v ní.
\par 20 Mesta také, v nichž bydlejí, zpustnou, a zeme pustá bude, a tak zvíte, že já jsem Hospodin.
\par 21 Opet stalo se slovo Hospodinovo ke mne, rkoucí:
\par 22 Synu clovecí, jaké to máte prísloví o zemi Izraelské, ríkajíce: Prodlí se dnové, aneb zahyne všeliké videní?
\par 23 Protož rci jim: Takto praví Panovník Hospodin: Uciním, aby prestalo prísloví toto, aniž užívati budou prísloví toho více v Izraeli. Rci jim: Nýbrž priblížili se dnové ti a splnení všelikého videní.
\par 24 Nebo nebude více žádného videní marného, a hádání pochlebníka u prostred domu Izraelského,
\par 25 Proto že já Hospodin mluviti budu, a kterékoli slovo promluvím, stane se. Neprodlít se dlouho, ale za dnu vašich, dome zpurný, mluviti budu slovo, a naplním je, praví Panovník Hospodin.
\par 26 I stalo se slovo Hospodinovo ke mne, rkoucí:
\par 27 Synu clovecí, aj, dum Izraelský ríkají: Videní to, kteréž vidí tento, ke dnum mnohým patrí, a na dlouhé casy tento prorokuje.
\par 28 Protož rci jim: Takto praví Panovník Hospodin: Neprodlít se dlouho všeliké slovo mé, ale slovo, kteréž mluviti budu, stane se, praví Panovník Hospodin.

\chapter{13}

\par 1 I stalo se slovo Hospodinovo ke mne, rkoucí:
\par 2 Synu clovecí, prorokuj proti prorokum Izraelským, kteríž prorokují, a rci prorokujícím z srdce svého: Slyšte slovo Hospodinovo:
\par 3 Takto praví Panovník Hospodin: Beda prorokum bláznivým, kteríž následují ducha svého, ješto však nic nevideli.
\par 4 Podobni jsou liškám na pustinách proroci tvoji, Izraeli.
\par 5 Nevstupujete k mezerám, aniž deláte hradby okolo domu Izraelského, aby ostáti mohl v boji v den Hospodinuv.
\par 6 Vídají marnost a hádání lživé. Ríkají: Praví Hospodin, ješto jich neposlal Hospodin, a troštují lidi, aby jen utvrdili slovo své.
\par 7 Zdaliž videní marného nevídáte, a hádání lživého nevypravujete? A ríkáte: Praví Hospodin, ješto jsem já nemluvil.
\par 8 Procež takto praví Panovník Hospodin: Proto že mluvíte marnost, a vídáte lež, protož aj, já jsem proti vám, praví Panovník Hospodin.
\par 9 Nebo bude ruka má proti prorokum, kteríž vídají marnost a hádají lež. V losu lidu mého nebudou, a v popisu domu Izraelského nebudou zapsáni, aniž do zeme Izraelské vejdou. I zvíte, že já jsem Panovník Hospodin,
\par 10 Proto, proto že v blud uvedli lid muj, ríkajíce: Pokoj, ješto nebylo žádného pokoje. Jeden zajisté ustavel stenu hlinenou, a jiní obmítali ji vápnem nicemným.
\par 11 Rciž tem, kteríž obmítají vápnem nicemným: I však to padne. Prijde príval rozvodnilý, a vy, kamenové krupobití velikého, spadnete, a vítr bourlivý roztrhne ji.
\par 12 A aj, když padne ta stena, zdaliž vám nebude receno: Kdež jest to obmítání, jímž jste obmítali?
\par 13 Protož takto praví Panovník Hospodin: Roztrhnu ji, pravím, vetrem bourlivým v prchlivosti své, a príval rozvodnilý v hneve mém prijde, a kamení krupobití velikého v prchlivosti mé k vyhlazení jí.
\par 14 Nebo rozborím stenu tu, kterouž jste obmetali vápnem nicemným, a porazím ji na zem, tak že odkryt bude grunt její. I padne, a zkaženi budete u prostred ní, a zvíte, že já jsem Hospodin.
\par 15 A tak vykonaje prchlivost svou na té stene, a na tech, kteríž ji obmítají vápnem nicemným, dím vám: Není více té steny, není ani tech, kteríž ji obmítali,
\par 16 Totiž proroku Izraelských, kteríž prorokují Jeruzalému, a ohlašují jemu videní pokoje, ješto není žádného pokoje, praví Panovník Hospodin.
\par 17 Ty pak, synu clovecí, obrat tvár svou k dcerám lidu svého, kteréž prorokují z srdce svého, a prorokuj i proti nim.
\par 18 A rci: Takto praví Panovník Hospodin: Beda tem, kteréž šijí polštáríky pod všeliké lokty rukou lidu mého, a delají kukly na hlavu všeliké postavy, aby lovily duše. Zdaliž loviti máte duše lidu mého, abyste se živiti mohly?
\par 19 Nebo zlehcujete mne u lidu mého pro hrst jecmene a pro kus chleba, umrtvujíce duše, kteréž neumrou, a obživujíce duše, kteréž živy nebudou, lhouce lidu mému, kteríž poslouchají lži.
\par 20 Protož takto praví Panovník Hospodin: Aj, já dám se v polštáríky vaše, jimiž vy lovíte tam duše, abyste je oklamaly. Nebo strhnu je s ramenou vašich, a propustím duše, duše, kteréž vy lovíte, abyste je oklamaly.
\par 21 I strhnu kukly vaše, a vytrhnu lid svuj z ruky vaší, tak abyste jich nemohly více loviti, i zvíte, že já jsem Hospodin;
\par 22 Proto že kormoutíte srdce spravedlivého lžmi, ješto jsem já ho nekormoutil, a posilujete rukou bezbožného, aby se neodvrátil od zlé cesty své, obživujíce jej.
\par 23 Protož nebudete vídati marnosti, a s hádáním nebudete se obírati více; nebo vytrhnu lid svuj z ruky vaší, i zvíte, že já jsem Hospodin.

\chapter{14}

\par 1 Potom prišedše ke mne muži z starších Izraelských, sedeli prede mnou.
\par 2 I stalo se slovo Hospodinovo ke mne, rkoucí:
\par 3 Synu clovecí, muži tito složili ukydané bohy své v srdci svém, a nepravost, kteráž jim k urážce jest, položili pred tvári své. Zdaliž se upríme radí se mnou?
\par 4 Protož mluv jim a rci jim: Takto praví Panovník Hospodin: Kdo by koli z domu Izraelského složil ukydané bohy své v srdci svém, a nepravost, kteráž mu k urážce jest, položil pred tvár svou, a prišel by k proroku: já Hospodin odpovídati budu tomu, kterýž prišel, o množství ukydaných bohu jeho,
\par 5 Abych polapil dum Izraelský v srdci jejich, že se odvrátili ode mne k ukydaným bohum svým všickni naporád.
\par 6 Protož rci domu Izraelskému: Takto praví Panovník Hospodin: Obratte se a odvratte od ukydaných bohu vašich, a ode všech ohavností vašich odvratte tvár svou.
\par 7 Nebo kdož by koli z domu Izraelského i z pohostinných, kteríž jsou pohostinu v Izraeli, odvrátil se od následování mne, a složil by ukydané bohy své v srdci svém, a nepravost, kteráž mu k urážce jest, položil by pred tvár svou a prišel by k proroku, aby se mne tázal skrze neho: já Hospodin odpovím jemu o sobe,
\par 8 A obrátím tvár svou hnevivou proti muži tomu, a dám jej za znamení a za prísloví, a vytnu jej z prostred lidu svého, i zvíte, že já jsem Hospodin.
\par 9 Prorok pak, dal-li by se privábiti, aby mluvil slovo, já Hospodin privábil jsem proroka toho. Než vztáhnut ruku svou na nej, a vyhladím jej z prostred lidu svého Izraelského.
\par 10 A tak ponesou nepravost svou. Jakáž pokuta na toho, kdož by se tázal, takováž pokuta na proroka bude,
\par 11 Aby nebloudili více dum Izraelský ode mne, a nepoškvrnovali se více žádnými prevrácenostmi svými, aby byli lidem mým, a já abych byl jejich Bohem, praví Panovník Hospodin.
\par 12 Opet stalo se slovo Hospodinovo ke mne, rkoucí:
\par 13 Synu clovecí, když by zeme zhrešila proti mne, dopouštejíc se prestoupení, tehdy vztáhl-li bych ruku svou na ni, a zlámal jí hul chleba, a poslal bych na ni hlad, a vyhubil z ní lidi i hovada:
\par 14 By pak byli u prostred ní tito tri muži, Noé, Daniel a Job, oni v spravedlnosti své vysvobodili by sami sebe, praví Panovník Hospodin.
\par 15 Pakli bych zver lítou uvedl na zemi, tak že by ji na sirobu privedla, a byla by pustá, aniž by kdo pres ni jíti mohl pro zver:
\par 16 Živt jsem já, praví Panovník Hospodin, že byt tri muži tito u prostred ní byli, nikoli by nevysvobodili synu ani dcer. Oni by sami vysvobozeni byli, zeme pak byla by pustá.
\par 17 Aneb mec uvedl-li bych na zemi tu, a rekl bych meci: Projdi skrz zemi tu, abych vyhubil z ní lidi i hovada:
\par 18 Živt jsem já, praví Panovník Hospodin, že byt pak tri muži tito byli u prostred ní, nikoli by nevysvobodili synu ani dcer, ale oni sami by vysvobozeni byli.
\par 19 Aneb mor poslal-li bych na zemi tu, a vylil prchlivost svou na ni k zhoube, aby vyhlazeni byli z ní lidé i hovada:
\par 20 Živt jsem já, praví Panovník Hospodin, že byt pak Noé, Daniel a Job u prostred ní byli, nikoli by ani syna ani dcery nevysvobodili. Oni v spravedlnosti své vysvobodili by sami sebe.
\par 21 Nýbrž takto praví Panovník Hospodin: Bych pak ctyri pokuty své zlé, mec a hlad, zver lítou a mor poslal na Jeruzalém, abych vyhubil z neho lidi i hovada,
\par 22 A aj, pozustali-li by v nem, kteríž by toho ušli, a vyvedeni byli, synové neb dcery: aj, i oni musejí jíti k vám, a uzríte cestu jejich a skutky jejich, i potešíte se nad tím zlým, kteréž uvedu na Jeruzalém, nade vším, což uvedu na nej.
\par 23 A tak poteší vás, když uzríte cestu jejich a skutky jejich. I zvíte, že jsem ne nadarmo ucinil všecko to, což jsem ucinil pri nem, praví Panovník Hospodin.

\chapter{15}

\par 1 Tedy stalo se slovo Hospodinovo ke mne, rkoucí:
\par 2 Synu clovecí, co jest drevo révové proti všelijakému drevu, aneb proti ratolestem dríví lesního?
\par 3 Zdaliž vzato bude z neho drevo k udelání neceho? Zdaliž udelají z neho hrebík k zavešování na nem všelijaké nádoby?
\par 4 Aj, na ohen dává se k sežrání. Když oba konce jeho sežere ohen, a prostredek jeho obhorí, zdaž se k cemu hoditi muže?
\par 5 Aj, když byl celý, nic nemohlo býti z neho udeláno, ovšem když jej ohen sežral, a shorel, k nicemu se více hoditi nebude.
\par 6 Protož takto praví Panovník Hospodin: Jakož jest drevo révové mezi drívím lesním, kteréž jsem oddal ohni k sežrání, tak jsem oddal obyvatele Jeruzalémské.
\par 7 Nebo postavím tvár svou hnevivou proti nim. Z ohne jednoho vyjdou, a ohen druhý zžíre je. I zvíte, že já jsem Hospodin, když obrátím tvár svou hnevivou proti nim.
\par 8 A obrátím zemi tuto v poušt, proto že se prestoupení dopoušteli, praví Panovník Hospodin.

\chapter{16}

\par 1 Opet stalo se slovo Hospodinovo ke mne, rkoucí:
\par 2 Synu clovecí, oznam Jeruzalému ohavnosti jeho,
\par 3 A rci: Takto praví Panovník Hospodin dceri Jeruzalémské: Obcování tvé a rod tvuj jest z zeme Kananejské, otec tvuj jest Amorejský, a matka tvá Hetejská.
\par 4 Narození pak tvé: V den, v nemž jsi se narodila, nebyl prirezán pupek tvuj, a vodou nebylas obmyta, abys ošetrena byla, aniž jsi byla solí posolena, ani plénkami obvinuta.
\par 5 Neslitovalo se nad tebou oko, atby v jednom z tech vecí posloužilo, maje lítost nad tebou, ale bylas povržena na svrchku pole, proto že jsi ošklivá byla v den, v kterémž jsi narodila se.
\par 6 A jda mimo tebe, a vida te ku potlacení vydanou ve krvi tvé, rekl jsem tobe: Ve krvi své živa bud. Rekl jsem, pravím, tobe: Ve krvi své živa bud.
\par 7 Rozmnožil jsem te jako z rostliny polní, i rozmnožena jsi a zvelicena, a prišlas k nejvetší ozdobe. Prsy tvé oduly se, a vlasy tvé zrostly, ac jsi byla nahá a odkrytá.
\par 8 Protož jda mimo te, a vida te, an aj, cas tvuj cas milování, vztáhl jsem krídlo své na te, a prikryl jsem nahotu tvou, a prísahou zavázav se tobe, všel jsem v smlouvu s tebou, praví Panovník Hospodin, a tak jsi má ucinena.
\par 9 I umyl jsem te vodou, a splákl jsem krev tvou s tebe, a pomazal jsem te olejem.
\par 10 Nadto priodel jsem te rouchem krumpovaným, a obul jsem te v drahé strevíce, a opásal jsem te kmentem, a priodel jsem te rouchem hedbávným.
\par 11 Ozdobil jsem te také ozdobou, a dal jsem náramky na ruce tvé, a tocenici na hrdlo tvé.
\par 12 Dalt jsem ozdobu i na celo tvé, a náušnice na uši tvé, a korunu krásnou na hlavu tvou.
\par 13 A tak bylas ozdobena zlatem a stríbrem, a odev tvuj byl kment a roucho hedbávné, a promenných barev roucho. Bel a med a olej jídala jsi, a krásná jsi ucinena velmi velice, a štastnet se vedlo v království,
\par 14 Tak že se rozešla povest o tobe mezi národy pro krásu tvou; nebo dokonalá byla, pro slávu mou, kterouž jsem byl vložil na tebe, praví Panovník Hospodin.
\par 15 Ale úfalas v krásu svou, a smilnilas pricinou povestí své; nebo jsi páchala smilství s každým mimo tebe jdoucím. Každý snadne užil krásy tvé.
\par 16 A vzavši z roucha svého, nadelalas sobe výsostí rozlicných barev, a páchalas smilství pri nich, jemuž podobné nikdy neprijde, aniž kdy potom bude.
\par 17 Nad to, vzavši prípravy ozdoby své z zlata mého a stríbra mého, kteréžt jsem byl dal, nadelalas sobe obrazu pohlaví mužského, a smilnilas s nimi.
\par 18 Vzalas také roucha svá krumpovaná, a priodílas je, olej muj i kadidlo mé kladlas pred nimi.
\par 19 Ano i chléb muj, kterýžt jsem byl dal, bel a olej i med, jímž jsem te krmil, kladlas pred nimi u vuni príjemnou, a bylo tak, praví Panovník Hospodin.
\par 20 Bralas i syny své a dcery své, kteréž jsi mne zplodila, a obetovalas jim k spálení. Cožt se ješte zdálo málo, taková smilství tvá,
\par 21 Žes i syny mé zabíjela a dávalas je, aby je provodili jim?
\par 22 K tomu ve všech ohavnostech svých a smilstvích svých nerozpomenulas se na dny mladosti své, když jsi byla nahá a odkrytá, ku potlacení vydaná ve krvi své.
\par 23 Ale stalo se pres všecku tuto nešlechetnost tvou, (beda, beda tobe), praví Panovník Hospodin,
\par 24 Že jsi vystavela sobe i vysoké místo, a vzdelalas sobe výsost v každé ulici.
\par 25 Pri všelikém rozcestí udelalas výsost svou, a zohavilas krásu svou, roztahujíc nohy své každému tudy jdoucímu, a príliš jsi smilnila.
\par 26 Nebo smilnila jsi s syny Egyptskými, sousedy svými velikého tela, a príliš jsi smilnila, abys mne k hnevu popouzela.
\par 27 Protož aj, vztáhl jsem ruku svou na tebe, a ujal jsem vymereného pokrmu tvého, a vydal jsem te k líbosti nenávidících te dcer Filistinských, kteréžto stydely se za cestu tvou nešlechetnou.
\par 28 Smilnilas též s syny Assyrskými, proto že jsi nemohla nasytiti se, a smilnivši s nimi, aniž jsi tak se nasytila.
\par 29 A tak príliš jsi smilnila v zemi Kananejské s Kaldejskými, a aniž jsi tak se nasytila.
\par 30 Jakt jest zmámeno srdce tvé, praví Panovník Hospodin, ponevadž se dopouštíš všech techto skutku ženy nevestky prenestydaté,
\par 31 Staveje sobe vysoké místo na rozcestí všeliké silnice, a výsost sobe stroje i v každé ulici. Nýbrž pohrdaje darem, nejsi ani jako nevestka,
\par 32 A žena cizoložná, kteráž místo muže svého povoluje cizím.
\par 33 Všechnem nevestkám dávají mzdu, ale ty dávalas mzdu svou všechnem frejírum svým, a darovalas je, aby vcházeli k tobe odevšad pro smilství tvá.
\par 34 A tak jest pri tobe naproti obyceji tech žen pri smilstvích tvých, ponevadž te k smilství nehledají, nýbrž ty dáváš dary, a ne tobe dar dáván bývá. A tot jest naopak.
\par 35 Protož ó nevestko, slyš slovo Hospodinovo:
\par 36 Takto praví Panovník Hospodin: Proto že vylita jest mrzkost tvá, a odkrývána byla nahota tvá pri smilstvích tvých s frejíri tvými, a se všemi ukydanými bohy ohavností tvých, též pro krev synu tvých, kteréž jsi dala jim:
\par 37 Proto aj, já shromáždím všecky frejíre tvé, s nimiž jsi obcovala, a všecky, kteréž jsi milovala, se všechnemi, jichž jsi nenávidela, a shromážde je proti tobe odevšad, odkryji nahotu tvou pred nimi, aby videli všecku nahotu tvou.
\par 38 A budu te souditi soudem cizoložnic, a tech, kteríž krev vylévají, a oddám te k smrti, kteráž prijde na te z prchlivosti a horlení.
\par 39 Nebo vydám te v ruku jejich. I rozborí vysoké místo tvé, a rozválejí výsosti tvé, a svlekouce te z roucha tvého, poberou prípravy ozdoby tvé, a nechají te nahé a odkryté.
\par 40 I privedou proti tobe shromáždení, a uházejí te kamením, a probodnou te meci svými.
\par 41 Popálí také domy tvé ohnem, a vykonají na tobe pomstu pred ocima žen mnohých, a tak prítrž uciním tvému smilství, a aniž budeš dávati daru více.
\par 42 A tak odpocinet sobe hnev muj na tobe, a horlení mé odejde od tebe, abych upokojil se a nehneval se více,
\par 43 Proto žes se nerozpomenula na dny mladosti své, ale postavovalas se proti mne ve všem tom. Aj hle, já také cestu tvou na hlavu tvou obrátil jsem, praví Panovník Hospodin, tak že nebudeš páchati nešlechetnosti, ani kterých ohavností svých.
\par 44 Aj, kdožkoli užívá prísloví, o tobe užive prísloví, rka: Jakáž matka, takáž dcera její.
\par 45 Dcera matky své jsi, té, kteráž sobe zošklivila muže svého a dítky své, a sestra obou sestr svých jsi, kteréž zošklivily sobe muže své a dítky své. Matka vaše jest Hetejská, a otec váš Amorejský.
\par 46 Sestra pak tvá starší, kteráž sedí po levici tvé, jest Samarí a dcery její, a sestra tvá mladší, kteráž sedí po pravici tvé, jest Sodoma a dcery její.
\par 47 Nýbrž aniž jsi po cestách jejich chodila, ani podlé ohavností jejich cinila, zošklivivši sobe jako vec špatnou, procež pokazilas se více než ony na všech cestách svých.
\par 48 Živt jsem já, praví Panovník Hospodin, že Sodoma sestra tvá i dcery její necinily, jako jsi ty cinila s dcerami svými.
\par 49 Aj, tatot byla nepravost Sodomy sestry tvé: Pýcha, sytost chleba a hojnost pokoje. To ona majíc i dcery její, ruky však chudého a nuzného neposilnovala.
\par 50 Ale pozdvihše se, páchaly ohavnost prede mnou; protož sklidil jsem je, jakž mi se videlo.
\par 51 Samarí také ani polovice hríchu tvých nenahrešila. Nebo jsi rozhojnila ohavnosti své nad ne, a tak jsi spravedlivejší býti ukázala sestry své všemi ohavnostmi svými, kteréž jsi páchala.
\par 52 Nesiž i ty také potupu svou, kterouž jsi prisoudila sestrám svým, pro hríchy své, kteréž jsi ohavne páchala více než ony. Spravedlivejšít byly než ty. I ty, pravím, styd se, a nes potupu svou, ponevadž spravedlivejší býti ukazuješ sestry své.
\par 53 Privedu-li zase zajaté jejich, totiž zajaté Sodomy a dcer jejich, též zajaté Samarí a dcer jejich, takét zajetí zajatých tvých u prostred nich,
\par 54 Proto, abys musila nésti potupu svou a hanbiti se za všecko, což jsi páchala, jsuc jejich potešením.
\par 55 Jestližet sestry tvé, Sodoma a dcery její, navrátí se k prvnímu zpusobu svému, též Samarí a dcery její navrátí-li se k prvnímu zpusobu svému: i ty také s dcerami svými navrátíte se k prvnímu zpusobu svému.
\par 56 Ponevadž nebyla Sodoma sestra tvá povestí v ústech tvých v den zvýšení tvého,
\par 57 Prvé než zjevena byla zlost tvá, jako za casu útržky od dcer Syrských a všech, kteríž jsou vukol nich, dcer Filistinských, kteréž te hubily se všech stran:
\par 58 Nešlechetnost svou a ohavnosti své poneseš, praví Hospodin.
\par 59 Nebo takto praví Panovník Hospodin: Tak uciním tobe, jakž jsi ucinila, když jsi pohrdla prísahou, a zrušila smlouvu.
\par 60 A však rozpomenu se na smlouvu svou s tebou ve dnech mladosti tvé, potvrdím, pravím, tobe smlouvy vecné.
\par 61 I rozpomeneš se na cesty své, a hanbiti se budeš, když prijmeš sestry své starší, nežli jsi ty, i mladší, nežli jsi ty, a dám je tobe za dcery, ale ne podlé smlouvy tvé.
\par 62 A tak utvrdím smlouvu svou s tebou, i zvíš, že já jsem Hospodin,
\par 63 Abys se rozpomenula a stydela, a nemohla více úst otevríti pro hanbu svou, když te ocistím ode všeho, což jsi cinila, praví Panovník Hospodin.

\chapter{17}

\par 1 Opet stalo se slovo Hospodinovo ke mne, rkoucí:
\par 2 Synu clovecí, vydej pohádku, a predlož podobenství o domu Izraelském,
\par 3 A rci: Takto praví Panovník Hospodin: Orlice veliká, velikých krídel a dlouhých brku, plná perí, strakatá, priletevši na Libán, vzala vrch cedru.
\par 4 Vrch mladistvých ratolestí jeho ulomila, a prenesla jej do zeme kupecké; v meste kupeckém položila jej.
\par 5 Potom vzavši z semene té zeme, vsadila je v poli úrodném, a vsadila je velmi opatrne pri vodách mnohých.
\par 6 Kteréžto bylo by vzešlo, a bylo by révem bujným, jakžkoli nízké postavy, a byly by patrily ratolesti jeho k ní, a korenové jeho poddáni byli by jí, a tak bylo by kmenem vinným, kterýž by vydal byl ratolesti, a vypustil rozvody.
\par 7 Ale byla orlice jedna veliká velikých krídel a vyperená, a aj, ten kmen vinný pripjal koreny své k ní, a ratolesti své vztáhl k ní, aby svlažovala jej z brázd štípení svého,
\par 8 Ješto v poli dobrém, pri vodách mnohých štípen byl, aby vypustil ratolesti, a nesl ovoce, a byl kmenem slavným.
\par 9 Rci: Takto praví Panovník Hospodin: Zdaliž se podarí? Zdaliž korenu jeho nevytrhá, a ovoce jeho neotrhá a neusuší? Zdaž všech ratolestí vyrostlých z neho neusuší? Zdaliž s velikou silou a s mnohým lidem nevyhladí ho z korenu jeho?
\par 10 Aj, jakžkoli štípen, zdaliž se podarí? Zdaliž, jakž se ho dotkne vítr východní, do konce neuschne? Pri brázdách, pri nichž se ujal, zdaž neuschne?
\par 11 Za tím stalo se slovo Hospodinovo ke mne, rkoucí:
\par 12 Rci nyní domu zpurnému: Nevíte-liž, co je toto? Rci: Aj, pritáhl král Babylonský do Jeruzaléma, a vzal krále jeho i knížata jeho, a zavedl je s sebou do Babylona.
\par 13 Vzal také z semene královského, a uciniv s ním smlouvu, prísahou jej zavázal, a silné zeme té pobral,
\par 14 Aby bylo království snížené, proto aby se nepozdvihovalo, aby ostríhaje smlouvy jeho, tak stálo.
\par 15 Ale zprotivil se jemu, poslav posly své do Egypta, aby jemu podal koní a lidu mnohého. Zdaž se mu to podarí? Zdaž pomsty ujde ten, kdož tak ciní? Ten kdož ruší smlouvu, zdaliž pomsty ujde?
\par 16 Živt jsem já, praví Panovník Hospodin, že v míste krále toho, kterýž jej králem ucinil, jehož prísahou pohrdl, a jehož smlouvu zrušil, u neho v Babylone umre.
\par 17 Aniž mu Farao s vojskem velikým a s zástupem mnohým co napomuže v boji, když vysype násyp, a vzdelá šance, aby zahubil množství lidí,
\par 18 Ponevadž pohrdl prísahou, zrušiv smlouvu. Neb aj, podal ruky své, a však všecko toto ciní. Neujdet pomsty.
\par 19 Protož takto praví Panovník Hospodin: Živt jsem já, že prísahu svou, kterouž pohrdl, a smlouvu svou, kterouž zrušil, jistotne obrátím na hlavu jeho.
\par 20 Nebo roztáhnu na nej sít svou, a polapen bude do vrše mé, i zavedu jej do Babylona, a souditi se s ním budu tam pro prestoupení jeho, kteréhož se dopustil proti mne.
\par 21 Všickni též, kteríž utekli od neho se všemi houfy jeho, od mece padnou, ostatní pak na všecky strany rozprostríni budou. I zvíte, že já Hospodin mluvil jsem.
\par 22 Takto praví Panovník Hospodin: A však vezmu z vrchu cedru toho vysokého a vsadím, z vrchu mladistvých ratolestí jeho mladou vetvicku ulomím, a štípím na hore vysoké a vyvýšené.
\par 23 Na hore vysoké Izraelské štípím ji, i vypustí ratolesti, a ponese ovoce, a tak ucinena bude cedrem slavným, a bude bydliti pod ním všeliké ptactvo; všecko, což krídla má, v stínu ratolestí jeho bydliti bude.
\par 24 A tak zvedí všecka dríví polní, že já Hospodin snížil jsem drevo vysoké, a povýšil jsem dreva nízkého; usušil jsem strom zelený, a zpusobil to, aby zkvetl strom suchý. Já Hospodin mluvil jsem to i uciním.

\chapter{18}

\par 1 Opet stalo se slovo Hospodinovo ke mne, rkoucí:
\par 2 Což jest vám, že užíváte prísloví tohoto o zemi Izraelské, ríkajíce: Otcové jedli hrozen trpký, a zubové synu laskominy mají?
\par 3 Živt jsem já, praví Panovník Hospodin, že nebudete moci více užívati prísloví tohoto v Izraeli.
\par 4 Aj, všecky duše mé jsou, jakož duše otcova, tak i duše synova mé jsou. Duše, kteráž hreší, ta umre.
\par 5 Nebo byl-li by nekdo spravedlivý, a cinil by soud a spravedlnost;
\par 6 Na horách by nejídal, a ocí svých nepozdvihoval k ukydaným bohum domu Izraelského, a manželky bližního svého by nepoškvrnil, a k žene pro necistotu oddelené nepristoupil;
\par 7 Kterýž by žádného neutiskal, základ dlužníku svému by navracoval, cizího mocí nebral, chleba svého by lacnému udílel, a nahého priodíval rouchem;
\par 8 Na lichvu by nedával, a úroku nebral, od nepravosti ruku svou by odvracoval, soud pravý mezi jedním i druhým by cinil;
\par 9 V ustanoveních mých by chodil, a soudu mých ostríhal, cine, což pravého jest: spravedlivý ten jiste žet živ bude, praví Panovník Hospodin.
\par 10 Zplodil-li by pak syna lotra, prolevace krve, kterýž by címkoli z tech vecí škodil bratru,
\par 11 Onoho pak všeho necinil by, anobrž i na horách by jídal, a ženy bližního svého by poškvrnil;
\par 12 Chudého a nuzného by utiskl, cizí veci mocí vzal, základu by nenavrátil, a k ukydaným bohum ocí svých by pozdvihoval, ohavnost provodil,
\par 13 Na lichvu by dával, a úrok bral: zdaž bude živ? Nebude živ. Ponevadž všecky ohavnosti tyto cinil, jistotne umre, krev jeho prijde na nej.
\par 14 A aj, zplodil-li by syna, kterýž by spatril všecky hríchy otce svého, kteréž cinil, a vida, necinil by tak;
\par 15 Na horách by nejídal, a ocí svých nepozdvihoval k ukydaným bohum domu Izraelského, manželky bližního svého by nepoškvrnil,
\par 16 A aniž by koho utiskal, základu by nezadržoval, cizího mocí nebral, chleba svého lacnému by udílel, a nahého rouchem by priodíl;
\par 17 Od chudého by zdržel ruku svou, lichvy a úroku by nebral, soudy mé cinil, v ustanoveních mých by chodil: tent neumre pro nepravost otce svého, jiste živ bude.
\par 18 Otec pak jeho, proto že se bezpráví dopouštel, cizí veci bratru mocí bral, a to, což není dobré, cinil u prostred lidu svého: protož aj, umre pro nepravost svou.
\par 19 Ale ríkáte: Jak by to bylo? Zdaž nenese syn nepravosti otcovy? Když syn ciní soud a spravedlnost, všech ustanovení mých ostríhá a ciní je, jiste žet živ bude.
\par 20 Duše, kteráž hreší, ta umre. Syn neponese nepravosti otcovy, aniž otec ponese nepravosti synovy; spravedlnost spravedlivého pri nem zustane, též bezbožnost bezbožného na nej pripadne.
\par 21 Pakli byse bezbožný odvrátil ode všech hríchu svých, kteréž cinil, a ostríhal by všech ustanovení mých, a cinil by soud a spravedlnost, jiste živ bude a neumre.
\par 22 Žádná prestoupení jeho, jichž se dopustil, nebudou jemu pripomínána; v spravedlnosti své, kterouž by cinil, živ bude.
\par 23 Zdaliž jakou líbost mám, když umírá bezbožný? dí Panovník Hospodin. Zdali ne radeji když se odvrací od cest svých, aby živ byl?
\par 24 Pakli by se odvrátil spravedlivý od spravedlnosti své, a cinil by nepravost, cine podlé všech ohavností, kteréž ciní bezbožný, takový-liž by živ byl? Na žádné spravedlnosti jeho, kteréž cinil, nebude pamatováno. Pro prestoupení své, jehož se dopouštel, a pro hrích svuj, kterýž páchal, pro tyt veci umre.
\par 25 Že pak ríkáte: Není pravá cesta Páne, poslyštež nyní, ó dome Izraelský: Zdali má cesta není pravá? Zdali nejsou cesty vaše nepravé?
\par 26 Když by se odvrátil spravedlivý od spravedlnosti své, a cine nepravost, v tom by umrel, pro nepravost svou, kterouž cinil, umre.
\par 27 A když by se odvrátil bezbožný od bezbožnosti své, kterouž cinil, a cinil by soud a spravedlnost, tent duši svou zachová pri životu.
\par 28 Nebo prohlédl, a odvrátil, se ode všech prestoupení svých, jichž se dopouštel; jiste žet živ bude a neumre.
\par 29 A však vždy ríká dum Izraelský: Není pravá cesta Páne. Zdali mé cesty nepravé jsou, ó dome Izraelský? Zdaliž nejsou cesty vaše nepravé?
\par 30 A protož každého z vás podlé cest jeho souditi budu, ó dome Izraelský, dí Panovník Hospodin. Navrattež se a odvratte ode všech prestoupení svých, aby vám nebyla k úrazu nepravost.
\par 31 Odvrzte od sebe všecka prestoupení vaše, jichž jste se dopoušteli, a ucinte sobe srdce nové a ducha nového. I procež mrete, ó dome Izraelský?
\par 32 Však nemám líbosti v smrti toho, jenž umírá, dí Panovník Hospodin. Obratte se tedy, a živi budte.

\chapter{19}

\par 1 Ty pak vydej se v naríkání nad knížaty Izraelskými.
\par 2 A rci: Co byla matka tvá? Lvice mezi lvy odpocívající, a u prostred dravých lvu vychovávala lvícátka svá.
\par 3 A když odchovala jedno z lvícat svých, udelal se z neho dravý lev, tak že nauciv se bráti loupeže, žrával lidi.
\par 4 To když uslyšeli národové, v jáme jejich polapen jest, a doveden v retezích do zeme Egyptské.
\par 5 To viduc lvice, že ocekávaná zhynula jí nadeje její, vzavši jedno z lvícat svých, ucinila z neho silného lva;
\par 6 Kterýž ustavicne chode mezi lvy, udelal se dravým lvem, a nauciv se bráti loupeže, žrával lidi.
\par 7 Poboril i pusté paláce jejich, a mesta jejich v poušt uvedl, tak že spustla zeme, i což v ní bylo, od hlasu rvání jeho.
\par 8 I polékli na nej národové z okolních krajin, a rozestreli na nej tenata svá, a do jámy jejich lapen jest.
\par 9 I dali jej do klece v retezích, a dopravili ho k králi Babylonskému, a uvedli jej do vezení nejtežšího, aby nebyl slýchán hlas jeho více po horách Izraelských.
\par 10 Matka tvá v cas pokoje tvého jako vinný kmen pri vodách štípený; plodistvý a rozkladitý byl pro hojnost vod.
\par 11 A mel pruty mocné k berlám panovníku, zrust pak jeho vyvýšil se nad prostredek hustého vetvoví, tak že patrný byl pro svou vysokost a pro množství ratolestí svých.
\par 12 Ale vytržen jsa v prchlivosti, na zemi povržen jest, a vítr východní usušil ovoce jeho; vylomily se a uschly ratolesti silné jeho, ohen sežral je.
\par 13 A nyní štípen jest na poušti, v zemi vyprahlé a žíznivé.
\par 14 Nadto vyšed ohen z prutu ratolestí jeho, sežral ovoce jeho, tak že není na nem prutu mocného k berle panovníka. Tot jest naríkání, a budet v naríkání.

\chapter{20}

\par 1 Tedy stalo se léta sedmého, a dne desátého, pátého mesíce, prišli nekterí z starších Izraelských raditi se s Hospodinem, a posadili se prede mnou.
\par 2 Tehdy stalo se slovo Hospodinovo ke mne, rkoucí:
\par 3 Synu clovecí, mluv k starším Izraelským a rci jim: Takto praví Panovník Hospodin: Zdali, abyste se se mnou radili, vy pricházíte? Živ jsem já, že vy se neradíte se mnou, dí Panovník Hospodin.
\par 4 Zdali jich zastávati budeš? Zdali zastávati budeš, synu clovecí? Oznam jim ohavnosti otcu jejich,
\par 5 A rci jim: Takto praví Panovník Hospodin: V ten den, v kterýž jsem zvolil Izraele, zdvihl jsem ruku svou semeni domu Jákobova, a v známost jsem se uvedl jim v zemi Egyptské; zdvihlt jsem jim ruku svou, rka: Já jsem Hospodin Buh váš.
\par 6 V ten den zdvihl jsem jim ruku svou, že je vyvedu z zeme Egyptské do zeme, kterouž jsem jim obhlédl, tekoucí mlékem a strdí, jenž jest okrasa všech jiných zemí.
\par 7 A rekl jsem jim: Jeden každý ohavnosti ocí svých zavrzte, a ukydanými bohy Egyptskými se nepoškvrnujte, nebo já jsem Hospodin Buh váš.
\par 8 Ale zpurne se postavovali proti mne, aniž mne chteli slyšeti, aniž kdo ohavnosti ocí svých zavrhl, a ukydaných bohu Egyptských nezanechali.Protož jsem rekl: Vyleji prchlivost svou na ne, a vyplním hnev svuj na nich u prostred zeme Egyptské.
\par 9 A však ucinil jsem pro jméno své, aby nebylo zlehceno pred ocima tech národu, mezi nimiž byli, pred jejichž ocima jsem se jim v známost uvedl, že je chci vyvesti z zeme Egyptské.
\par 10 Takž jsem je vyvedl z zeme Egyptské, a privedl jsem je na poušt,
\par 11 A dal jsem jim ustanovení svá, a soudy své v známost jsem jim uvedl, ješto cinil-li by je kdo, jiste že by živ byl skrze ne.
\par 12 Nadto i soboty své vydal jsem jim, aby byly na znamení mezi mnou a mezi nimi, aby znali, že já Hospodin jsem posvetitel jejich.
\par 13 Ale dum Izraelský zpurne se postavovali proti mne na poušti, v ustanoveních mých nechodili, a soudy mými pohrdli, ješto cinil-li by je kdo, jiste že by živ byl skrze ne; též soboty mé poškvrnili náramne. Procež jsem rekl: Že vyleji prchlivost svou na ne na poušti, abych je docela vyhladil.
\par 14 Ale ucinil jsem pro jméno své, aby nebylo zlehceno pred ocima tech národu, pred jejichž ocima jsem je vyvedl.
\par 15 A však i já také prisáhl jsem jim na té poušti, že jich neuvedu do zeme, kterouž jsem byl dal, tekoucí mlékem a strdí, jenž jest okrasa všech jiných zemí,
\par 16 Proto že soudy mými pohrdli, a v ustanoveních mých nechodili, a soboty mé poškvrnili, a že za ukydanými bohy jejich srdce jejich chodí.
\par 17 Však odpustilo jim oko mé, tak že jsem jich nezahladil, a neucinil jim konce na poušti.
\par 18 A rekl jsem synum jejich na též poušti: V ustanoveních otcu svých nechodte, a soudu jejich neostríhejte, a ukydanými bohy jejich se nepoškvrnujte.
\par 19 Já jsem Hospodin Buh váš, v ustanoveních mých chodte, a soudu mých ostríhejte, a cinte je.
\par 20 Též soboty mé svette, i budou na znamení mezi mnou a vámi, aby známé bylo, že já jsem Hospodin Buh váš.
\par 21 Ale zpurne se meli ke mne ti synové, v ustanoveních mých nechodili, a soudu mých nešetrili, aby je cinili, (ješto cinil-li by je kdo, jiste že by živ byl skrze ne), i soboty mé poškvrnili. I rekl jsem: Že vyleji prchlivost svou na ne, abych vyplnil hnev svuj na nich na té poušti.
\par 22 Ale odvrátil jsem zase ruku svou; což jsem ucinil pro jméno své, aby nebylo zlehceno pred ocima tech národu, pred jejichž ocima jsem je vyvedl.
\par 23 Také i tem prisáhl jsem na poušti, že je rozptýlím mezi pohany, a že je rozženu po krajinách,
\par 24 Proto že soudu mých necinili, a ustanoveními mými pohrdli, a soboty mé poškvrnili, a že k ukydaným bohum otcu svých zrení meli.
\par 25 Procež já také dal jsem jim ustanovení nedobrá, a soudy, skrze než nebudou živi.
\par 26 A poškvrnil jsem jich s jejich dary, (proto že provodili všecko, což otvírá život), abych je zkazil, aby poznali, že já jsem Hospodin.
\par 27 Protož mluv k domu Izraelskému, synu clovecí, a rci jim: Takto praví Panovník Hospodin: Ješte i tímto rouhali se mi otcové vaši, dopouštejíce se proti mne prestoupení,
\par 28 Že, když jsem je uvedl do zeme, o kteréž jsem prisáhl, že jim ji dám, kdež spatrili který pahrbek vysoký, aneb které drevo husté, hned tu obetovali obeti své, a tu dávali popouzející dary své, tu kladli i vuni svou príjemnou, a tu obetovali mokré obeti své.
\par 29 A ackoli jsem ríkal jim: Což jest ta výsost, kamž vy chodíváte? ale ona slove výsostí až do tohoto dne.
\par 30 Protož rci domu Izraelskému: Takto praví Panovník Hospodin: Cestou-liž otcu vašich vy se máte poškvrnovati, a s ohavnostmi jejich máte smilniti?
\par 31 Též prinášejíce dary své, vodíce syny své skrze ohen, máte-liž se poškvrnovati pri všech ukydaných bozích vašich až do tohoto dne, a ode mne vždy rady hledati, ó dome Izraelský? Živt jsem já, dí Panovník Hospodin, že vy se nebudete mne více raditi.
\par 32 To pak, což jste sobe v mysli uložili, nikoli se nestane, že ríkáte: Budeme jako jiní národové, jako celedi jiných zemí, sloužíce drevu a kameni.
\par 33 Živ jsem já, praví Panovník Hospodin, že rukou silnou a ramenem vztaženým, a prchlivostí vylitou kralovati budu nad vámi.
\par 34 Vyvedu vás zajisté z národu, a shromáždím vás z zemí, do nichž jste rozptýleni, rukou silnou a ramenem vztaženým, i prchlivostí vylitou.
\par 35 A vode vás po poušti tech národu, souditi se budu s vámi tam tvárí v tvár.
\par 36 Tak jako jsem se soudil s otci vašimi na poušti zeme Egyptské, tak se budu souditi s vámi, praví Panovník Hospodin.
\par 37 A proženu vás pod hul, abych vás uvedl do závazku smlouvy.
\par 38 Ale ty, jenž se mi zprotivili a zproneverili, vymísím z vás; z zeme, v níž pohostinu jsou, vyvedu je, do zeme však Izraelské nevejdou. I zvíte, že já jsem Hospodin.
\par 39 Vy tedy, ó dome Izraelský, takto praví Panovník Hospodin: Jdetež, služte každý ukydaným bohum svým i napotom, ponevadž neposloucháte mne, a jména mého svatého nepoškvrnujte více dary svými, a ukydanými bohy svými.
\par 40 Nebo na hore mé svaté, na hore vysoké Izraelské, dí Panovník Hospodin, tam mi sloužiti budou všecken dum Izraelský, což jich koli bude v té zemi. Tam je laskave prijmu, a tam vyhledávati budu od vás obetí vzhuru pozdvižení i prvotin daru vašich, se všemi svatými vecmi vašimi.
\par 41 S vuní libou laskave vás prijmu, když vás vyvedu z národu, a shromáždím vás z tech zemí, do nichž jste rozptýleni byli, a tak posvecen budu v vás pred ocima tech národu.
\par 42 I poznáte, že já jsem Hospodin, když vás uvedu do zeme Izraelské, do zeme té, o kteréž jsem prisáhl, že ji dám otcum vašim.
\par 43 A tu se rozpomenete na cesty své a na všecky ciny své, jimiž jste se poškvrnovali, tak že sami se býti hodné ošklivosti uznáte, pro všecky nešlechetnosti vaše, kteréž jste cinívali.
\par 44 Tu poznáte, že já jsem Hospodin, když vám to uciním pro jméno své, ne podlé cest vašich zlých, ani podlé cinu vašich porušených, dome Izraelský, praví Panovník Hospodin.
\par 45 I stalo se slovo Hospodinovo ke mne, rkoucí:
\par 46 Synu clovecí, obrat tvár svou k strane polední, a vypust jako rosu na poledne, a prorokuj proti lesu toho pole, kteréž jest na poledne.
\par 47 A rci lesu polednímu: Slyš slovo Hospodinovo: Takto praví Panovník Hospodin: Aj, já zanítím v tobe ohen, kterýž sežere v tobe každé drevo zelené i každé drevo suché; nezhasnet plamen preprudký, a budou jím opáleny všecky tváre od poledne až na pulnoci.
\par 48 I uzrí všeliké telo, že jsem já Hospodin zapálil jej; nezhasnet.
\par 49 I rekl jsem: Ach, Panovníce Hospodine, oni mi ríkají: Však tento v prísloví nám toliko mluví.

\chapter{21}

\par 1 Opet stalo se slovo Hospodinovo ke mne, rkoucí:
\par 2 Synu clovecí, obrat tvár svou k Jeruzalému, a vypust jako rosu proti místum svatým, a prorokuj proti zemi Izraelské.
\par 3 A rci zemi Izraelské: Takto praví Hospodin: Aj, já jsem proti tobe, a vytáhnu mec svuj z pošvy jeho, a vypléním z tebe spravedlivého i bezbožného.
\par 4 Proto, abych vyplénil z tebe spravedlivého i bezbožného, proto vyjde mec muj z pošvy své proti všelikému telu, od poledne až na pulnoci.
\par 5 I zvít všeliké telo, že jsem já Hospodin vytáhl mec svuj z pošvy jeho; nenavrátít se zase více.
\par 6 Ty pak synu clovecí, vzdychej, jako bys zlámaná mel bedra, a to s horekováním vzdychej pred ocima jejich.
\par 7 I stane se, žet reknou: Nad cím ty vzdycháš? Tedy rekneš: Nad povestí, kteráž prichází, k níž rozplyne se každé srdce, a každé ruce klesnou, a všeliký duch skormoutí se, a každá kolena rozplynou se jako voda. Aj, pricházít a deje se, praví Panovník Hospodin.
\par 8 Opet stalo se slovo Hospodinovo ke mne, rkoucí:
\par 9 Synu clovecí, prorokuj a rci: Takto praví Hospodin: Rci: Mec, mec nabroušen, také i vycišten jest.
\par 10 Aby zabíjel k zabití oddané, nabroušen jest; aby se blyštel, vycišten jest. Radovati-liž se budeme, když prut syna mého pohrdá každým drevem?
\par 11 Dalt jej vycistiti, aby v ruku vzat byl; jestit nabroušený mec, jest i vycištený, aby dán byl do ruky mordujícího.
\par 12 Kric a kvel, synu clovecí, proto že ten bude proti lidu mému, tentýž proti všechnem knížatum Izraelským; uvrženi budou na mec s lidem mým, protož bí se v bedra.
\par 13 Když jsem je trestával, co bylo? Nemám-liž metly hubící již priciniti? dí Hospodin zástupu.
\par 14 Ty tedy synu clovecí, prorokuj a tleskej rukama; nebo po druhé i po tretí prijde mec, mec mordujících, ten mec mordujících bez lítosti, pronikajících i do pokoju jejich.
\par 15 Tak aby se rozplynulo srdce, a rozmnoženi byli úrazové, v každé bráne jejich postavím ostrí mece. Ach, vycištent jest, aby se blyštel, zaostren, aby zabíjel.
\par 16 Shlukni se a pust se na pravo i na levo, kamžkoli a nackoli se tobe nahodí.
\par 17 I ját také tleskati budu rukama svýma, a doložím prchlivost svou. Já Hospodin mluvil jsem.
\par 18 V tom stalo se slovo Hospodinovo ke mne, rkoucí:
\par 19 Ty pak synu clovecí, predlož sobe dve cesty, kudy by jíti mel mec krále Babylonského. Z zeme jedné at vycházejí obe dve, a na rozcestí vyber tu k mestu, tu vyber.
\par 20 Ukaž cestu, kudy by jíti mel mec, k Rabbat-li synu Ammon, cili k Judstvu, na Jeruzalémské pevnosti,
\par 21 Proto že stane král Babylonský na rozcestí, na pocátku dvou cest, obíraje se s hádáním, vycistí strely, doptávati se bude modl, hledeti bude do jater.
\par 22 Po pravé ruce jeho hádání ukáže Jeruzalém, aby sšikoval hejtmany, kteríž by ponoukali k mordování, a pozdvihovali hlasu s prokrikováním, aby pristavili berany válecné proti branám, aby vysypán byl násyp, a vzdelání byli šancové.
\par 23 I budou to míti za hádání marné pred ocima svýma ti, jenž se zavázali prísahami; a tot privede na pamet nepravost, kterouž by popadeni byli.
\par 24 Protož takto praví Panovník Hospodin: Proto že ku pameti privodíte nepravost svou, a odkrývá se nevera vaše, tak že jsou patrní hríchové vaši ve všech cinech vašich, proto že na pamet pricházíte, rukou tou popadeni budete.
\par 25 Ty pak necistý bezbožníce, kníže Izraelské, jehož den prichází, a cas skonání nepravosti,
\par 26 Takto praví Panovník Hospodin: Sejmi tu cepici, a svrz tu korunu, kteráž nikdy již taková nebude; toho, kterýž na snížení prišel, povyš, a vyvýšeného poniž.
\par 27 Zmotanou, zmotanou, zmotanou uciním ji, (cehož prvé nebývalo), až prijde ten, jenž má právo, kteréž jsem jemu dal.
\par 28 Ty pak synu clovecí, prorokuj a rci: Takto praví Panovník Hospodin o synech Ammon, i o pohanení jejich, rci, pravím: Mec, mec dobyt jest, k zabíjení vycišten jest, aby hubil všecko, a aby se blyštel.
\par 29 A ackoli predpovídají tobe marné veci, a hádají tobe lež, aby te priložili k hrdlum zbitých bezbožníku, jejichž den prichází a cas skonání nepravosti:
\par 30 Schovej mec do pošvy jeho. Na míste, na kterémž jsi zplozena, v zemi prebývání tvého, budu te souditi.
\par 31 A vyleji na te rozhnevání své, ohnem prchlivosti své na te dmýchati budu, a dám te v ruku lidí vzteklých, remeslníku všecko kazících.
\par 32 Budeš ohni k sežrání, krev tvá bude u prostred zeme, nebudeš pripomínána; nebot jsem já Hospodin mluvil.

\chapter{22}

\par 1 Opet stalo se slovo Hospodinovo ke mne, rkoucí:
\par 2 Ty pak synu clovecí, zastával-li bys, zastával-liž bys toho mesta vražedlného? Radeji mu oznam všecky ohavnosti jeho,
\par 3 A rci: Takto praví Panovník Hospodin: Pricházít cas mesta toho, jenž prolévá krev u prostred sebe, a delá ukydané bohy proti sobe, aby se poškvrnovalo.
\par 4 Ty krví svou, kterouž jsi prolilo, zavinivší, a ukydanými bohy svými, jichž jsi nadelalo, sebe poškvrnivší, to jsi zpusobilo, že se priblížili dnové tvoji, a prišlo jsi k letum svým. Protož vydám te v pohanení národum, a ku posmechu všechnem zemím.
\par 5 Blízké i daleké od tebe budou se tobe posmívati, ó zlopovestné a ruznic plné.
\par 6 Aj, knížata Izraelská jeden každý vší silou na to se vydali, aby krev v tobe prolévali.
\par 7 Otce i matku zlehcují v tobe, pohostinnému ciní nátisk u prostred tebe, sirotka a vdovu utiskují v tobe.
\par 8 Svatými vecmi mými zhrdáš, a sobot mých poškvrnuješ.
\par 9 Utrhaci jsou v tobe, aby prolévali krev, a na horách jídají v tobe; nešlechetnost páší u prostred tebe.
\par 10 Nahotu otce syn odkrývá v tobe, a necisté v oddelení ponižují v tobe.
\par 11 Jiný pak s ženou bližního svého páše ohavnost, a jiný s nevestou svou poškvrnuje se nešlechetností, a jiný sestry své, dcery otce svého, ponižuje v tobe.
\par 12 Dar berou v tobe, aby krev prolili; lichvu a úrok béreš, a zisku hledáš s útiskem bližního svého, na mne se pak zapomínáš, dí Panovník Hospodin.
\par 13 Protož aj, já tleskl jsem rukama svýma nad ziskem tvým, jehož dobýváš, i nad tou krví, kteráž byla u prostred tebe.
\par 14 Zdali ostojí srdce tvé? Zdaž odolají ruce tvé dnum, v nichž já budu zacházeti s tebou? Já Hospodin mluvil jsem i uciním.
\par 15 Nebo rozptýlím te mezi pohany, a rozženu te po krajinách, a do konce vyprázdním necistotu tvou z tebe.
\par 16 A zavrženo jsuc pred ocima pohanu, poznáš, že já jsem Hospodin.
\par 17 Potom stalo se slovo Hospodinovo ke mne, rkoucí:
\par 18 Synu clovecí, obrátili se mi dum Izraelský v trusku, všickni naporád jsou med, cín, železo a olovo u prostred peci, trusky stríbra jsou.
\par 19 Procež takto praví Panovník Hospodin: Proto že jste vy všickni obrátili se v trusky, protož aj, já shromáždím vás do Jeruzaléma.
\par 20 Jakž se shromažduje stríbro a med, železo, olovo i cín do prostred peci, k rozdmýchání ohne vukol neho a k rozpouštení: tak shromáždím v hneve a v prchlivosti své, a slože, rozpoušteti vás budu.
\par 21 Sberu vás, pravím, a rozdmýchám okolo vás ohen prchlivosti své, i rozpustíte se u prostred mesta.
\par 22 Jakž se stríbro rozpouští v peci, tak se rozpustíte u prostred neho, i zvíte, že já Hospodin vylil jsem prchlivost svou na vás.
\par 23 Ješte stalo se slovo Hospodinovo ke mne, rkoucí:
\par 24 Synu clovecí, rci: Ty zeme jsi necistá, nebudeš deštem svlažena v den rozhnevání.
\par 25 Spiknutí proroku jejích u prostred ní, jsou podobni lvu rvoucímu, uchvacujícímu loupež, duše žerou, bohatství a veci drahé berou, a ciní mnoho vdov u prostred ní.
\par 26 Kneží její natahují zákona mého, a svaté veci mé poškvrnují, mezi svatým a poškvrneným rozdílu neciní, a mezi necistým a cistým nerozeznávají. Nadto od sobot mých skrývají oci své, tak že zlehcován bývám mezi nimi.
\par 27 Knížata její u prostred ní jsou jako vlci uchvacující loupež, vylévajíce krev, hubíce duše, aby sháneli mrzký zisk.
\par 28 Proroci pak jejich obmítají jim vápnem nicemným, predpovídají marné veci, a hádají jim lež, ríkajíce: Takto praví Panovník Hospodin, ješto Hospodin nemluvil.
\par 29 Lid této zeme ciní nátisk, a cizí veci mocí bére, chudému a nuznému ubližují, a pohostinného utiskují nespravedlive.
\par 30 Hledaje pak nekoho z nich, kterýž by udelal hradbu, a postavil se v mezere pred tvárí mou za tuto zemi, abych jí nezkazil, žádného nenacházím.
\par 31 Protož vyleji na ne rozhnevání své, ohnem prchlivosti své konec jim uciním, cestu jejich jim na hlavu obrátím, praví Panovník Hospodin.

\chapter{23}

\par 1 Opet stalo se slovo Hospodinovo ke mne, rkoucí:
\par 2 Synu clovecí, dve ženy, dcery jedné matere byly.
\par 3 Ty smilnily v Egypte, v mladosti své smilnily. Tam jsou mackány prsy jejich, tam opiplány prsy panenství jejich.
\par 4 Jména pak jejich: vetší Ahola, sestry pak její Aholiba. Tyt jsou byly mé, a rozplodily syny a dcery. Jména, pravím, jejich jsou: Samarí Ahola, Jeruzaléma pak Aholiba.
\par 5 Ale Ahola maje mne, smilnila a freju hledela s milovníky svými, s Assyrskými blízkými,
\par 6 Odenými postavcem modrým, s vývodami a knížaty, i všemi naporád mládenci krásnými, a s jezdci jezdícími na koních.
\par 7 Vydala se, pravím, v smilství svá s nimi, se všemi nejprednejšími syny Assyrskými, a se všemi, jimž frejovala, a poškvrnila se všemi ukydanými bohy jejich.
\par 8 A tak smilství svých Egyptských nenechala; nebo ji zléhali v mladosti její, a oni mackali prsy panenství jejího, a vylili smilství svá na ni.
\par 9 Protož dal jsem ji v ruku frejíru jejích, v ruku Assyrských, jimž frejovala.
\par 10 Onit jsou odkryli nahotu její, syny i dcery její pobrali, ji pak samu mecem zamordovali. Takž vzata jest na slovo od jiných žen, když soudy vykonali pri ní.
\par 11 To videla sestra její Aholiba, však mnohem více než onano frejovala, a smilství její vetší byla než smilství sestry její.
\par 12 S Assyrskými freju hledela, s vývodami a knížaty blízkými, odenými nádherne, s jezdci jezdícími na koních, a všemi mládenci krásnými.
\par 13 I videl jsem, že se poškvrnila, a že cesta jednostejná jest obou dvou.
\par 14 Ale tato ješte to pricinila k smilstvím svým, že viduc muže vyryté na stene, obrazy Kaldejských vymalované barvou,
\par 15 Prepásané pasem po bedrách jejich, a klobouky barevné na hlavách jejich, a že jsou všickni na pohledení jako hejtmané, podobní synum Babylonským v Kaldejské zemi, jejichž ona vlast jest,
\par 16 I zahorela k nim z pohledení ocima svýma, a vyslala posly k nim do zeme Kaldejské.
\par 17 Tedy vešli k ní Babylonští na luže nepoctivé, a poškvrnili ji smilstvím svým. A když se poškvrnila s nimi, odloucila se duše její od nich.
\par 18 A odkryla smilství svá, odkryla též nahotu svou, i odloucila se duše má od ní, tak jako se odloucila duše má od sestry její.
\par 19 Nebo rozmnožila smilství svá, rozpomínajíc se na dny mladosti své, v nichž smilnila v zemi Egyptské,
\par 20 A frejovala s kubenári jejich, jejichž telo jest jako telo oslu, a tok jejich jako tok konský.
\par 21 A tak jsi zase navrátila se k nešlechetnosti mladosti své, když mackali Egyptští prsy tvé z príciny prsu mladosti tvé.
\par 22 Protož ó Aholiba, takto praví Panovník Hospodin: Aj, já vzbudím frejíre tvé proti tobe, ty, od nichž se odloucila duše tvá, a privedu je na te odevšad,
\par 23 Babylonské a všecky Kaldejské, Pekodské, a Šohejské, i Kohejské, všecky syny Assyrské s nimi, mládence krásné, vývody a knížata všecka, hejtmany a slovoutné, všecky jezdící na koních.
\par 24 A pritáhnou na te na vozích železných a prikrytých, i kárách, a to s zberí národu, s pavézami a štíty i lebkami, položí se proti tobe vukol, i predložím jim právo, aby te soudili soudy svými.
\par 25 Vyleji zajisté horlení své na tebe, tak že naloží s tebou prchlive, nos tvuj i uši tvé odejmou, a ostatek tebe mecem padne. Ti syny tvé i dcery tvé poberou, a ostatek tebe spáleno bude ohnem.
\par 26 A vyvlekou te z roucha tvého, a rozberou šperky okrasy tvé.
\par 27 A tak prítrž uciním pri tobe nešlechetnosti tvé, i smilství tvému z zeme Egyptské vzatému, a nepozdvihneš ocí svých k nim, a na Egypt nezpomeneš více.
\par 28 Nebo takto praví Panovník Hospodin: Aj, já dám tebe v ruku tech, kterýchž nenávidíš, v ruku tech, od nichž se odloucila duše tvá,
\par 29 I budou nakládati s tebou podlé nenávisti, a poberou všecko úsilé tvé, a nechají te nahé a obnažené. A tak bude zrejmá nahota smilství tvého a nešlechetnosti tvé, smilství, pravím, tvého.
\par 30 Což vše uciní tobe proto, že jsi smilnila, následujíc pohanu, proto že jsi poškvrnila se ukydanými bohy jejich.
\par 31 Cestou sestry své chodila jsi, protož dám kalich její v ruku tvou.
\par 32 Takto praví Panovník Hospodin: Kalich sestry své píti budeš hluboký a široký; budet sporý, tak že smích a žert budou míti z tebe.
\par 33 Opilstvím a zámutkem naplnena budeš, kalichem pustiny a zpuštení, kalichem sestry své Samarí.
\par 34 I vypiješ jej a vyvážíš, a než jej polámeš, snáze prsy své roztrháš; nebot jsem já mluvil, praví Panovník Hospodin.
\par 35 Protož takto praví Panovník Hospodin: Z té príciny, že jsi zapomenula na mne, a zavrhlas mne za hrbet svuj, i ty také vezmi za svou nešlechetnost, a za smilství svá.
\par 36 I rekl Hospodin ke mne: Synu clovecí, budeš-liž zastávati Ahole neb Aholiby? Nýbrž oznam jim ohavnosti jejich,
\par 37 Že cizoložily, a krev jest na rukou jejich, a s ukydanými bohy svými cizoložily. Také i syny své, kteréž mne zplodily, vodily jim, aby sežráni byli.
\par 38 Ješte i toto cinily mi, že zanecištovaly svatyni mou v tentýž den, a sobot mých poškvrnovaly.
\par 39 Nebo obetovavše syny své ukydaným bohum svým, vcházely do svatyne mé v tentýž den, aby ji poškvrnily. Aj hle, takt jsou cinívaly u prostred mého domu.
\par 40 Nadto, že vysílaly k mužum, jenž by prišli zdaleka, kteríž, jakž posel vyslán k nim, aj, hned pricházívali. Jimž jsi se umývala, a tvár svou lícila, a okrašlovalas se okrasou.
\par 41 A usazovalas se na loži slavném, pred nímž stul pripravený byl, na než jsi i kadidlo mé i masti mé vynakládala.
\par 42 Když pak hlas toho množství poutichl, tedy i k mužum z obecného lidu vysílaly, jenž bývali privozováni ožralí z poušte. I dávali náramky na ruce jejich, i koruny ozdobné na hlavy jejich.
\par 43 A ackoli jsem se domlouval na cizoložství té lotryne, a že oni jednak s jednou, jednak s druhou smilství provodí,
\par 44 A že každý z nich vchází k ní, tak jako nekdo vchází k žene nevestce: však vždy vcházeli k Ahole a Aholibe, ženám prenešlechetným.
\par 45 Protož muži spravedliví, tit je souditi budou soudem cizoložných a soudem tech, jenž vylévaly krev, proto že cizoložily, a krev jest na rukou jejich.
\par 46 Nebo takto praví Panovník Hospodin: Privedu na ne vojsko, a dám je v posmýkání i v loupež.
\par 47 I uhází je to shromáždení kamením, a poseká je meci svými; syny jejich i dcery jejich pomordují, a domy jejich ohnem popálí.
\par 48 A tak prítrž uciním nešlechetnosti v zemi této, i budou se tím káti všecky ženy, a nedopustí se nešlechetnosti podobné vaší.
\par 49 Nebo vzložena bude na vás nešlechetnost vaše, a ponesete hríchy ukydaných bohu svých. I zvíte, že já jsem Panovník Hospodin.

\chapter{24}

\par 1 Opet stalo se slovo Hospodinovo ke mne léta devátého, mesíce desátého, desátého dne téhož mesíce, rkoucí:
\par 2 Synu clovecí, napiš sobe jméno tohoto dne, vlastne dnešního dne tohoto: nebo oblehl král Babylonský Jeruzalém práve tohoto dnešního dne.
\par 3 A predlož tomu domu zpurnému podobenství, rka k nim: Takto praví Panovník Hospodin: Pristav tento hrnec, pristav a nalej také do neho vody.
\par 4 A sebera kusy náležité do neho, každý kus dobrý, stehno i plece, a nejlepšími kostmi napln jej.
\par 5 Privezmi i nejlepších bravu, a udelej ohen z kostí pod ním, zpusob, at to vre, ažby kypelo, at se i kosti jeho rozvarí v nem.
\par 6 Protož takto praví Panovník Hospodin: Beda mestu tomu vražedlnému, hrnci, v nemž zustává pripálenina jeho, z nehož, pravím, pripálenina jeho nevychází. Po kusích, po kusích vytahuj z neho, nepadnet na nej los.
\par 7 Nebo krev jest u prostred neho. Na vysedlou skálu vystavilo ji; nevylilo jí na zemi, aby ji prach prikryl.
\par 8 I já zaníte prchlivost k vykonání pomsty, vystavím krev na vysedlou skálu, aby nebyla prikryta.
\par 9 Protož takto praví Panovník Hospodin: Beda mestu vražedlnému, i já udelám veliký ohen,
\par 10 Prikládaje dríví, roznecuje ohen, v nic obraceje maso, a korene korením, tak že i kosti spáleny budou.
\par 11 A postavím ten hrnec na uhlí jeho prázdný, aby se zhrela i rozpálila med jeho, ažby se vyvarila u prostred neho necistota jeho, a vyprázdnila pripálenina jeho.
\par 12 Klamy svými bylo mi težké, protož nevyjde z neho množství šumu jeho; do ohne musí šum jeho.
\par 13 V tvé necistote jest nešlechetnost, proto že jsem te ocištoval, však nejsi ocišteno. Nebudeš více ocištováno od necistoty své, až i doložím prchlivost svou na tebe.
\par 14 Já Hospodin mluvil jsem, dojdet, a uciním to; neustoupímt, aniž se slituji, ani želeti budu. Podlé cest tvých a cinu tvých budou te souditi, praví Panovník Hospodin.
\par 15 Opet stalo se slovo Hospodinovo ke mne, rkoucí:
\par 16 Synu clovecí, aj, já odejmu od tebe žádost ocí tvých v náhle, však nekvel ani plac, a necht nevycházejí slzy tvé.
\par 17 Stonati prestan, smutku, jakž bývá nad mrtvým, nenes, klobouk svuj vstav na sebe, a strevíce své obuj na nohy své, a nezastírej brady své, aniž pokrmu cího jez.
\par 18 Což když jsem povedel lidu ráno, tedy umrela žena má u vecer. I ucinil jsem na ráno, jakž mi rozkázáno bylo.
\par 19 I rekl ke mne lid: Což nám neoznámíš, co tyto veci nám znamenají, kteréž ciníš?
\par 20 Tedy rekl jsem jim: Slovo Hospodinovo stalo se ke mne, rkoucí:
\par 21 Rci domu Izraelskému: Takto praví Panovník Hospodin: Aj, já poškvrním svatyne své, vyvýšenosti síly vaší, žádosti ocí vašich a toho, cehož šanuje duše vaše. Též synové vaši i dcery vaše, kterýchž jste zanechali, mecem padnou.
\par 22 I budete tak ciniti, jakž já ciním. Brady nezastrete, aniž cího pokrmu jísti budete.
\par 23 A majíce klobouky své na hlavách svých a strevíce na nohách svých, nebudete kvíliti ani plakati, ale svadnouce pro nepravosti své, úpeti budete jeden s druhým.
\par 24 Nebo jest vám Ezechiel zázrakem. Všecko, což on ciní, budete ciniti, a když to prijde, tedy zvíte, že já jsem Panovník Hospodin.
\par 25 Ty pak synu clovecí, zdali v ten den, když já odejmu od nich sílu jejich, veselé okrasy jejich, žádost ocí jejich, a to, po cemž touží duše jejich, syny jejich i dcery jejich,
\par 26 Zdali v ten den prijde k tobe ten, kdož utece, vypravuje tu novinu?
\par 27 V ten den otevrou se ústa tvá pri prítomnosti toho, kterýž ušel, i budeš mluviti, a nebudeš více nemým. Takž jim budeš zázrakem, i zvedí, že já jsem Hospodin.

\chapter{25}

\par 1 I stalo se slovo Hospodinovo ke mne, rkoucí:
\par 2 Synu clovecí, obrat tvár svou proti synum Ammon, a prorokuj proti nim.
\par 3 A rci synum Ammon: Slyšte slovo Panovníka Hospodina: Takto praví Panovník Hospodin: Proto že jsi nad svatyní mou, když poškvrnena byla, ríkal: To dobre to, a nad zemí Izraelskou, když zpuštena byla, a nad domem Judským, když šel v zajetí,
\par 4 Protož aj, já dám te národum východním za dedictví, i vzdelají sobe hrady v tobe, a vystavejí v tobe príbytky své. Tit budou jísti ovoce tvé, a ti budou píti mléko tvé.
\par 5 A dám Rabbu za obydlé velbloudum, a mesta synu Ammon za odpocivadlo stádum, i zvíte, že já jsem Hospodin.
\par 6 Nebo takto praví Panovník Hospodin: Proto že jsi tleskal rukou, a dupal nohou, a veselil se srdecne, že jsi všelijak loupil zemi Izraelskou,
\par 7 Protož aj, já vztáhnu ruku svou na tebe, a vydám te v loupež národum, a vypléním te z lidí, a vyhubím te z zemí, i zahladím te. Tu zvíš, že já jsem Hospodin.
\par 8 Takto praví Panovník Hospodin: Z té príciny, že ríkal Moáb a Seir: Hle, podobný jest všechnem jiným národum dum Judský,
\par 9 Protož aj, já otevru bok Moábských, (hned od Arim, od mest jejich na pomezí jejich, rozkošnou zemi Betjesimotských, Balmeonských i Kariataimských),
\par 10 Národum východním s zemí synu Ammon; nebo jsem ji dal v dedictví, tak aby nebylo zpomínáno na syny Ammon mezi národy.
\par 11 A tak i nad Moábem soudy vykonám, i zvedí, že já jsem Hospodin.
\par 12 Takto praví Panovník Hospodin: Proto že Idumejští nenáležite se vymstívajíce, ukrutne se meli k domu Judskému, a tak uvodili na se vinu velikou, vymstívajíce se na nich,
\par 13 Protož takto dí Panovník Hospodin: I na Idumea vztáhnu ruku svou, a vypléním z neho lidi i hovada, a obrátím jej v pustinu. Hned od Teman až do Dedan mecem padati budou.
\par 14 A tak uvedu pomstu svou na Idumejské skrze ruku lidu mého Izraelského, a naloží s Idumejskými podlé hnevu mého a podlé prchlivosti mé, i poznají pomstu mou, praví Panovník Hospodin.
\par 15 Takto praví Panovník Hospodin: Proto že se Filistinští ukrutne meli z príciny pomsty, nenáležite se vymstívajíce, loupíce zlostne, a zhoubu uvodíce z nenávisti starodávní,
\par 16 Protož takto praví Panovník Hospodin: Aj, já vztáhnu ruku svou na Filistinské, a vypléním Ceretejské, a zkazím ostatek prístavu morského.
\par 17 A tak vykonám pri nich pomsty veliké káraními zurivými, a zvedí, že já jsem Hospodin, když uvedu pomstu svou na ne.

\chapter{26}

\par 1 Bylo pak jedenáctého léta, prvního dne mesíce, že se stalo slovo Hospodinovo ke mne, rkoucí:
\par 2 Synu clovecí, proto že Týrus o Jeruzalému ríká: Dobre se stalo, že jest potríno mesto bran velmi lidných, obrací se ke mne, naplnen budu, kdyžte zpušteno,
\par 3 Protož takto praví Panovník Hospodin: Aj, já proti tobe, ó Týre, a privedu na te národy mnohé, tak jako bych privedl more s vlnami jeho.
\par 4 I zkazí zdi Týru, a zborí veže jeho; vymetu také z neho prach jeho, a obrátím jej v skálu vysedlou,
\par 5 Tak že budou vysušovati síti u prostred more. Nebo jsem já mluvil, praví Panovník Hospodin, protož bude v loupež národum.
\par 6 Dcery pak jeho, kteréž na poli budou, mecem zmordovány budou, i zvedí, že já jsem Hospodin.
\par 7 Nebo takto praví Panovník Hospodin: Aj, já privedu na Týr Nabuchodonozora krále Babylonského od pulnoci, krále nad králi, s konmi a s vozy, i s jezdci i s vojskem a s lidem mnohým.
\par 8 Dcery tvé na poli mecem zmorduje, a vzdelá proti tobe šance, a vysype proti tobe násyp, a postaví proti tobe pavézníky.
\par 9 I strelbu zasadí proti zdem tvým, a veže tvé poborí nosatci svými.
\par 10 Od množství koní jeho prikryje te prach jejich; od hrmotu jezdcu a kár i vozu zatresou se zdi tvé, když on vcházeti bude do bran tvých, jako do pruchodu mesta proboreného.
\par 11 Kopyty koní svých pošlapá všecky ulice tvé, lid tvuj mecem pomorduje, a sloupové pametní síly tvé na zem padnou.
\par 12 I rozberou zboží tvá, a rozchvátají kupectví tvá, a rozválejí zdi tvé, i domy tvé rozkošné poborí, a kamení tvé i dríví tvé, i prach tvuj do vody vmecí.
\par 13 A tak prítrž uciním hluku zpevu tvých, a zvuku citar tvých aby nebylo slýcháno více.
\par 14 A obrátím te v skálu vysedlou, budeš k vysušování sítí, nebudeš vystaven více; nebo já Hospodin mluvil jsem, praví Panovník Hospodin.
\par 15 Takto praví Panovník Hospodin Týru: Zdaliž od hrmotu padání tvého, když stonati budou zranení, když ukrutný mord bude u prostred tebe, nepohnou se ostrovové?
\par 16 A vyvstanou z stolic svých všecka knížata pomorská, a složí z sebe plášte své, i roucha svá krumpovaná svlekou; v hruzu se oblekou, na zemi sedeti budou, a tresouce se každé chvíle, trnouti budou nad tebou.
\par 17 I vydadí se nad tebou v naríkání, a reknou tobe: Jak jsi zahynulo, ó mesto, v nemž bydleno bylo pro more, mesto slovoutné, ješto bylo pevné na mori, ono i s obyvateli svými, kteríž poušteli strach svuj na všecky obyvatele jeho!
\par 18 Tehdáž trásti se budou ostrovové v den pádu tvého; predešeni, pravím, budou ostrovové, kteríž jsou na mori, nad zahynutím tvým.
\par 19 Nebo tak praví Panovník Hospodin: Když te uciním mestem zpušteným jako mesta, v nichž se nebydlí, když uvedu na te hlubinu, tak že te prikryjí vody mnohé,
\par 20 Když uciním, že sstoupíš s sstupujícími do jámy k lidu dávnímu, a posadím te v nejnižších stranách zeme, na pustinách starodávních s temi, jenž sstupují do jámy, aby nebylo bydleno v tobe: prokáži slávu v zemi živých.
\par 21 Nebo uciním to, že budeš k náramné hruze, když te nestane, a bys pak bylo hledáno, abys nebylo na veky nalezeno, praví Panovník Hospodin.

\chapter{27}

\par 1 I stalo se slovo Hospodinovo ke mne, rkoucí:
\par 2 Ty pak synu clovecí, vydej se nad Týrem v naríkání,
\par 3 A rci Týru, jenž sede tu, kdež se na more pouštejí, kupectví provodí s národy na ostrovích mnohých: Takto praví Panovník Hospodin: Ó Týre, ty jsi ríkal: Já jsem nejkrásnejší.
\par 4 U prostred more byly hranice tvé, stavitelé tvoji dokonale te ozdobovali.
\par 5 Z jedloví z Sanir delávali všecka taflování tvá, cedry z Libánu brávali k delání sloupu tvých.
\par 6 Z dubu Bázanských delávali vesla tvá, a lavicky tvé delávali z kostí slonových, a z pušpanu z ostrovu Citejských.
\par 7 Kment s krumpováním Egyptským býval plátno tvé, z nehož jsi plachty míval; modrý postavec a šarlat z ostrovu Elisa prikrýval te.
\par 8 Obyvatelé Sidonští i Arvadští bývali plavci tvoji, moudrí tvoji v tobe, ó Týre, ti bývali správcové tvoji.
\par 9 Starší Gebalští a moudrí jejich opravovali v tobe zboreniny tvé; všecky lodí morské i plavci jejich bývali v tobe, smenujíce s tebou kupectví.
\par 10 Perští a Ludští i Putští bývali v vojšte tvém bojovníci tvoji, pavézu a lebku zavešovali v tobe. Tit jsou pridávali tobe ozdoby.
\par 11 Arvadští s vojskem tvým na zdech tvých vukol, též Gamadští na vežech tvých bývali, štíty své zavešovali na zdech tvých vukol. Tit jsou te zvláštne ozdobovali.
\par 12 Zámorští kupci tvoji v množství všelijakého zboží, v stríbre, železe, cínu i olove kupcili na jarmarcích tvých.
\par 13 Javan, Tubal a Mešech, kupci tvoji, lidi a nádobí medené dávali za smenu tobe.
\par 14 Z domu Togarma v koních a jezdcích i mezcích kupcili na jarmarcích tvých.
\par 15 Synové Dedanovi kupci tvoji, a ostrovové mnozí prekupníci byli koupí tvých, tobe k ruce; rohy, kosti slonové i dríví hebénové smenovali za mzdu tvou.
\par 16 Syrští kupci tvoji pro množství vecí tvých remeslne udelaných, v karbunkulích, šarlatu, krumpování i kmentu, a korálích a krištálích kupcívali na jarmarcích tvých.
\par 17 Judští i zeme Izraelská kupci tvoji, pšenici Mennitskou i Fenickou, a med, olej i kadidlo dávali za smenu tobe.
\par 18 Damašští kupci tvoji, pro množství vecí tvých remeslne udelaných, kupcili ve množství všelijakého zboží, ve víne Chelbonském a vne belostkvoucí.
\par 19 Též Dan i Javan chodíce na jarmarky tvé, kupcili, a železo pulerované, kassii i trtinu vonnou tobe smenovali.
\par 20 Dedan kupcíval v tobe v suknech drahých k vozum.
\par 21 Arabští i všecka knížata Cedarská kupcívali tobe k ruce v beranech a skopcích i kozlích, tím kupcívali v tobe.
\par 22 Kupci Sabejští i Ragmejští bývali kupci tvoji ve všelijakých nejprednejších vonných vecech, i ve všelijakém kamení drahém i zlate, kupcívali na jarmarcích tvých.
\par 23 Cháran a Kanne i Eden, kupci Sabejští, Assur i Kilmad kupcíval v tobe.
\par 24 Ti bývali kupci tvoji na jarmarcích tvých s nejvýbornejšími vecmi, s štouckami postavce modrého, a s krumpováním i s klénoty drahých vecí, kteréž se provazy svazují a zavírají do cedru.
\par 25 Lodí morské predek mely v kupectví tvém. Summou, naplneno jsi i zvelebeno náramne u prostred more.
\par 26 Na vodu velikou zavezli te ti, kteríž te vesly táhli; vítr východní potríská te u prostred more.
\par 27 Zboží tvá i jarmarkové tvoji, kupectví tvá, plavci tvoji a správcové tvoji, i ti, kteríž opravovali zboreniny tvé, a smenovali s tebou kupectví, a všickni muži válecní tvoji, kteríž byli v tobe, i všecko shromáždení tvé, kteréž bylo u prostred tebe, padnou do hlubokosti morské v den pádu tvého.
\par 28 Od hrmotu kriku správcu tvých zbourí se i vlnobití.
\par 29 I vystoupí z lodí svých všickni ti, kteríž táhnou veslem, plavci i všickni správcové lodí morských na zemi stanou,
\par 30 A hlasem velikým nad tebou naríkati a žalostne kriceti budou, a sypouce prach na hlavy své, v popele se váleti.
\par 31 Nadto zdelajíce prícinou tvou lysiny,prepáší se žínemi, a kvílením horkým nad tebou s žalostí srdecnou plakati budou.
\par 32 Vydadí se, pravím, nad tebou s horekováním svým v naríkání, a budou naríkati nad tebou: Které mesto podobné Týru, zahlazenému u prostred more?
\par 33 Když vycházely koupe tvé z more, nasycovalo jsi národy mnohé; množstvím zboží svého i kupectví svých zbohacovalos krále zemské.
\par 34 Ale když ztroskotáno budeš od more v hlubokých vodách, kupectví tvé i všecko shromáždení tvé u prostred tebe klesne.
\par 35 Všickni obyvatelé ostrovu ztrnou nad tebou, a králové jejich ohromeni jsouce, zhrozí se náramne.
\par 36 Kupci mezi národy ckáti budou nad tebou; k hruze veliké budeš, a nebude te na veky.

\chapter{28}

\par 1 I stalo se slovo Hospodinovo ke mne, rkoucí:
\par 2 Synu clovecí, rci vývodovi Tyrskému: Takto praví Panovník Hospodin: Proto že se vyvyšuje srdce tvé, a ríkáš: Buh silný jsem, na stolici Boží sedím u prostred more, ješto jsi clovek a ne Buh silný, ackoli sobe privlastnuješ srdce podobné srdci Božímu,
\par 3 Aj hle, moudrejší jsi nad Daniele, žádná vec tajná není pred tebou ukrytá;
\par 4 Moudrosti svou a rozumností svou nashromáždils sobe zboží, a nahrnuls zlata a stríbra do pokladu svých;
\par 5 Velikou moudrostí svou v kupectví svém rozmnožil jsi zboží svá, a tak pozdvihlo se srdce tvé zbožím tvým;
\par 6 Z té príciny takto praví Panovník Hospodin: Proto že sobe privlastnuješ srdce podobné srdci Božímu,
\par 7 Protož aj, já privedu na te cizozemce, nejukrutnejší národy, kteríž vytrhnouce mece své na krásu moudrosti tvé, zabijí jasnost tvou.
\par 8 Do jámy spustí te, a umreš smrtí hroznou u prostred more.
\par 9 Budeš-liž tu ješte ríkati pred oblícejem mordére svého: Bohem jsem, ponevadž jsi clovek a ne Buh silný, jsa v ruce toho, jenž te mordovati bude?
\par 10 Smrtí neobrezancu umreš od ruky cizozemcu; nebo já mluvil jsem, praví Panovník Hospodin.
\par 11 I stalo se slovo Hospodinovo ke mne, rkoucí:
\par 12 Synu clovecí, vydej se v naríkání nad králem Tyrským, a rci jemu: Takto praví Panovník Hospodin: Ty, jenž zapecetuješ summy, plný moudrosti a nejkrásnejší,
\par 13 V Eden, zahrade Boží, byl jsi, všelijaké drahé kamení prikrývalo te, sardius, topazius, jaspis, tarsis, onychin, beryl, zafir, karbunkulus a smaragd i zlato; nástrojové bubnu tvých a píštal tvých v tobe, hned jakžs se narodil, pripraveni jsou.
\par 14 Ty jsi cherubem od pomazání. Jakž jsem te za ochránce predstavil, na hore svaté Boží jsi byl, u prostred kamení ohnivého ustavicne jsi chodil.
\par 15 Byl jsi dokonalý na cestách svých, hned jakžs se narodil, až se našla nepravost pri tobe.
\par 16 Pro množství kupectví tvého u prostred tebe plno jest bezpráví, a velmi jsi prohrešil. Procež zahubím te, a vyhladím z hory Boží, ó cherube ochránce, z prostred kamenu ohnivých.
\par 17 Pozdvihlo se srdce tvé slávou tvou, k zlému jsi užíval moudrosti své prícinou jasnosti své. Porazím te na zemi, pred oblícej králu povrhu te, aby se dívali na tebe.
\par 18 Pro množství nepravostí tvých, a pro nespravedlnosti kupectví tvého poškvrnil jsi svatyne své. Protož vyvedu ohen z prostred tebe, kterýž te sžíre, a obrátí te v popel na zemi pred ocima všech na te hledících.
\par 19 Všickni, kdož te znali mezi národy, ztrnou nad tebou; k hruze veliké budeš, a nebude te na veky.
\par 20 I stalo se slovo Hospodinovo ke mne, rkoucí:
\par 21 Synu clovecí, obrat tvár svou proti Sidonu, a prorokuj proti nemu,
\par 22 A rci: Takto praví Panovník Hospodin: Aj, já proti tobe jsem, ó Sidone, a budu oslaven u prostred tebe. I zvedí, že já jsem Hospodin, když vykonám pri nem soudy, a posvecen budu v nem.
\par 23 Pošli zajisté na nej mor, a krev na ulice jeho, a padati budou zranení u prostred neho od mece na všecky strany, i zvedí, že já jsem Hospodin.
\par 24 A tak nebude více míti dum Izraelský trnu urážejícího a bodláku bodoucího ze všech okolních pohrdajících jimi, a zvedí, že já jsem Panovník Hospodin.
\par 25 Takto praví Panovník Hospodin: Když shromáždím dum Izraelský z národu, mezi než rozptýleni jsou, a posvecen budu v nich pred ocima pohanu, a bydliti budou v zemi své, kterouž jsem byl dal služebníku svému Jákobovi:
\par 26 Tehdy bydliti budou v ní bezpecne, a nastavejí domu, a štípí vinice. Bydliti, pravím, budou bezpecne, když vykonám soudy, pri všech zhoubcích jejich vukol nich, i zvedí, že já jsem Hospodin Buh jejich.

\chapter{29}

\par 1 Léta desátého, desátého mesíce, dvanáctého dne téhož mesíce, stalo se slovo Hospodinovo ke mne, rkoucí:
\par 2 Synu clovecí, obrat tvár svou proti Faraonovi králi Egyptskému, a prorokuj proti nemu i proti všemu Egyptu.
\par 3 Mluv a rci: Takto praví Panovník Hospodin: Aj, já proti tobe jsem, ó Faraone králi Egyptský, draku veliký, kterýž ležíš u prostred rek svých, ješto ríkáš: Já mám reku svou, já zajisté jsem ji privedl sobe.
\par 4 Protož dám udici do celistí tvých, a uciním, že zváznou ryby rek tvých na tvých šupinách; i vyvleku te z prostredku rek tvých,i všecky ryby rek tvých na šupinách tvých zvázlé.
\par 5 A nechám te na poušti, tebe i všech ryb rek tvých. Na svrchku pole padneš, nebudeš sebrán ani shromážden; šelmám zemským i ptactvu nebeskému dám te k sežrání.
\par 6 I zvedí všickni obyvatelé Egyptští, že já jsem Hospodin, proto že jste holí trtinovou domu Izraelskému.
\par 7 Když se chytají tebe rukou, lámeš se, a roztínáš jim všecko rameno; a když se na te zpodpírají, ztroskotána býváš, ackoli nastavuješ jim všech bedr.
\par 8 Protož takto praví Panovník Hospodin: Aj, já privedu na te mec, a vypléním z tebe lidi i hovada.
\par 9 I bude zeme Egyptská pustinou a pouští, i zvedí, že já jsem Hospodin, proto že ríkal: Reka má jest, já zajisté jsem ji privedl.
\par 10 Protož aj, já budu proti tobe i proti rece tvé, a obrátím zemi Egyptskou v pustiny, v poušt prehroznou od veže Sevéne až do pomezí Mourenínského.
\par 11 Nepujdet pres ni noha cloveka, ani noha dobytcete pujde pres ni, a nebude v ní bydleno za ctyridceti let.
\par 12 A tak uvedu zemi Egyptskou v pustinu nad jiné zeme pusté, a mesta její nad jiná mesta zpuštená pustá budou za ctyridceti let, když rozptýlím Egyptské mezi národy, a rozženu je do rozlicných zemí.
\par 13 A však takto praví Panovník Hospodin: Po skonání ctyridceti let shromáždím Egyptské z národu, kamž rozptýleni budou.
\par 14 A privedu zase zajaté Egyptské, a uvedu je zase do zeme Patros, do zeme prebývání jejich, i budou tam královstvím sníženým.
\par 15 Mimo jiná království bude sníženejší, aniž se bude vynášeti více nad jiné národy; zmenším je zajisté, aby nepanovali nad národy.
\par 16 I nebude více domu Izraelskému doufáním, kteréž by na pamet privodilo nepravost, když by se obraceli za nimi; nebo zvedí, že já jsem Panovník Hospodin.
\par 17 Potom bylo dvadcátého sedmého léta, prvního mesíce, prvního dne, že se stalo slovo Hospodinovo ke mne, rkoucí:
\par 18 Synu clovecí, Nabuchodonozor král Babylonský podrobil vojsko své v službu velikou proti Týru. Každá hlava oblezla, a každé rameno odríno jest, mzdy pak nemá, on ani vojsko jeho, z Týru za tu službu, kterouž sloužili proti nemu.
\par 19 Z té príciny takto praví Panovník Hospodin: Aj, já dávám Nabuchodonozorovi králi Babylonskému zemi Egyptskou, aby pobral zboží její, a rozebral loupeže její, i rozchvátal koristi její, aby melo mzdu vojsko jeho.
\par 20 Za práci jejich, kterouž mi sloužili, dám jim zemi Egyptskou, proto že mne pracovali, praví Panovník Hospodin.
\par 21 V ten den rozkáži vypuciti se rohu domu Izraelského. Tobe též zpusobím, že otevreš ústa u prostred nich, i zvedí, že já jsem Hospodin.

\chapter{30}

\par 1 Opet se stalo slovo Hospodinovo ke mne, rkoucí:
\par 2 Synu clovecí, prorokuj a rci: Takto praví Panovník Hospodin: Kvelte: Ach, nastojte na tento den.
\par 3 Nebo blízko jest den, blízko jest, pravím, den Hospodinuv, den mrákoty, cas národu bude.
\par 4 A prijde mec do Egypta, a bude pretežká bolest v Mourenínské zemi, když padati budou zbití v Egypte, a poberou zboží jeho, a zboreni budou základové jeho.
\par 5 Mourenínové a Putští i Ludští i všelijaká smesice, též Kubští i obyvatelé zeme smlouvy s nimi mecem padnou.
\par 6 Takt praví Hospodin, že padnou podpurcové Egypta, a snížena bude vyvýšenost síly jeho; od veže Sevéne mecem padati budou v ní, praví Panovník Hospodin.
\par 7 I budou v pustinu obráceni nad jiné zeme pusté, a mesta jejich nad jiná mesta pustá budou.
\par 8 I zvedí, že já jsem Hospodin, když zapálím ohen v Egypte, a potríni budou všickni pomocníci jeho.
\par 9 V ten den vyjdou poslové od tvári mé na lodech, aby prestrašili Mourenínskou zemi ubezpecenou, i budou míti bolest pretežkou, jakáž byla ve dni Egypta; nebo aj, pricházít.
\par 10 Takto praví panovník Hospodin: Uciním zajisté konec množství Egyptskému skrze ruku Nabuchodonozora krále Babylonského.
\par 11 On i lid jeho s ním, nejukrutnejší národové privedeni budou, aby zkazili tu zemi; nebo vytrhnou mece své na Egypt, a naplní tu zemi zbitými.
\par 12 A obráte reky v sucho, prodám tu zemi v ruku nešlechetných, a tak v pustinu uvedu zemi, i což v ní jest, skrze ruku cizozemcu. Já Hospodin mluvil jsem.
\par 13 Takto praví Panovník Hospodin: Zkazím i ukydané bohy, a konec uciním modlám v Nof, a knížete z zeme Egyptské nebude více, když pustím strach na zemi Egyptskou.
\par 14 Nebo pohubím Patros, a zapálím ohen v Soan, a vykonám soudy v No.
\par 15 Vyleji prchlivost svou i na Sin, pevnost Egyptskou, a vypléním množství No.
\par 16 Když zapálím ohen v Egypte, velikou bolest bude míti Sin, a No bude roztrháno, Nof pak neprátely bude míti ve dne.
\par 17 Mládenci On a Bubastští mecem padnou, panny pak v zajetí pujdou.
\par 18 A v Tachpanches zatmí se den, když tam polámi závory Egypta, a prítrž se stane v nem vyvýšenosti síly jeho. Mrákota jej prikryje, dcery pak jeho v zajetí pujdou.
\par 19 A tak vykonám soudy pri Egyptu, i zvedí, že já jsem Hospodin.
\par 20 Opet bylo jedenáctého léta, prvního mesíce, sedmého dne, že se stalo slovo Hospodinovo ke mne, rkoucí:
\par 21 Synu clovecí, ráme Faraona krále Egyptského zlámal jsem, a aj, nebudet uvázáno, ani pricineno lékarství, aniž priložen bude šat pro obvázání jeho a posilnení jeho k držení mece.
\par 22 Protož takto praví panovník Hospodin: Aj, já jsem proti Faraonovi králi Egyptskému, a polámi ramena jeho, i sílu jeho, i budet zlámané, a vyrazím mec z ruky jeho.
\par 23 A rozptýlím Egyptské mezi národy, a rozženu je do zemí.
\par 24 Posilním zajisté ramen krále Babylonského, a dám mec svuj v ruku jeho, i polámi ramena Faraonova, tak že stonati bude pred ním, jakž stonává smrtelne ranený.
\par 25 Posilním, pravím, ramen krále Babylonského, ramena pak Faraonova klesnou. I zvedí, že já jsem Hospodin, když dám mec svuj v ruku krále Babylonského, aby jej vztáhl na zemi Egyptskou.
\par 26 A tak rozptýlím Egyptské mezi národy, a rozženu je do zemí, i zvedí, že já jsem Hospodin.

\chapter{31}

\par 1 Bylo pak jedenáctého léta, tretího mesíce, prvního dne téhož mesíce, že se stalo slovo Hospodinovo ke mne, rkoucí:
\par 2 Synu clovecí, rci Faraonovi králi Egyptskému i množství jeho: K komu jsi podoben v své velikosti?
\par 3 Aj, Assur byl jako cedr na Libánu, pekných ratolestí, a vetvovím zastenující, a vysokého zrustu, jehož vrchové byli mezi hustými vetvemi.
\par 4 Vody k zrustu privedly jej, propast jej vyvýšila, potoky jejími opušten byl vukol kmen jeho, ješto jen praménky své vypouštela na všecka dríví polní.
\par 5 Takž se vyvýšil zrust jeho nade všecka dríví polní, a rozmnožili se výstrelkové jeho, a pro hojnost vod roztáhly se ratolesti jeho, kteréž vypustil.
\par 6 Na ratolestech jeho hnízdilo se všelijaké ptactvo nebeské, a pod vetvemi jeho rodili se všelijací živocichové polní, a v stínu jeho sedali všickni národové velicí.
\par 7 I byl ušlechtilý pro svou velikost, a pro dlouhost vetví svých; nebo koren jeho byl pri vodách mnohých.
\par 8 Cedrové v zahrade Boží neprikryli ho, jedle nevrovnaly se ratolestem jeho, a stromové kaštanoví nebyli podobni vetvem jeho. Žádné drevo v zahrade Boží nebylo rovné jemu v kráse své.
\par 9 Ozdobil jsem jej množstvím vetvoví jeho, tak že mu závidela všecka dríví Eden, kteráž byla v zahrade Boží.
\par 10 Protož takto praví Panovník Hospodin: Proto že vysoce vyrostl, a vypustil vrch svuj mezi husté vetvoví, a pozdvihlo se srdce jeho prícinou vysokosti jeho:
\par 11 Protož vydal jsem jej v ruku nejsilnejšího z národu, aby s ním prísne nakládal; pro bezbožnost jeho vyhnal jsem jej.
\par 12 A tak vytali jej cizozemci, nejukrutnejší národové, a nechali ho tu. Po horách i po všech údolích opadly vetve jeho, a slomeny jsou ratolesti jeho na všecky prudké potoky té zeme. Procež vystoupili z stínu jeho všickni národové zeme, a opustili jej.
\par 13 Na nemž padlém bydlí všelijaké ptactvo nebeské, a na ratolestech jeho jsou všelijací živocichové polní,
\par 14 Proto aby se nevyvyšovalo v zrostu svém žádné dríví pri vodách, a aby nevypouštelo vrchu svých mezi hustými vetvemi, a nevypínalo se nad jiné vysokostí svou žádné drevo zapojené vodami, proto že všickni ti oddáni jsou k smrti, dolu do zeme mezi syny lidské s temi, kteríž sstupují do jámy.
\par 15 Takto praví Panovník Hospodin: Toho dne, v kterýž on sstoupil do hrobu, privedl jsem k kvílení, a prikryl jsem prícinou jeho propast, a zadržel jsem potoky její, aby se zastavily vody mnohé, a ucinil jsem, aby smutek nesl prícinou jeho Libán, a všecko dríví polní prícinou jeho aby umdlelo.
\par 16 Od hrmotu pádu jeho ucinil jsem, že se trásli národové, když jsem jej svedl do hrobu s temi, kteríž sstupují do jámy. Nad címž se potešila na zemi dole všecka dríví Eden, což výborného a dobrého jest na Libánu, vše což zapojeného jest vodou.
\par 17 I ti s ním sstoupili do hrobu k tem, kteríž jsou zbiti mecem, i ráme jeho, i kteríž sedali v stínu jeho u prostred národu.
\par 18 Kterému z stromu Eden podoben jsi tak v sláve a velikosti? Však svržen budeš s drívím Eden dolu na zem, mezi neobrezanci s zbitými mecem lehneš. Tot jest Farao i všecko množství jeho, praví Panovník Hospodin.

\chapter{32}

\par 1 Opet bylo dvanáctého léta, dvanáctého mesíce, prvního dne téhož mesíce, že se stalo slovo Hospodinovo ke mne, rkoucí:
\par 2 Synu clovecí, vydej se v naríkání nad Faraonem králem Egyptským, a rci jemu: Lvu mladému mezi národy podoben jsi, a jsi jako velryb v mori, když procházeje se v potocích svých, kalíš vodu nohama svýma, a kormoutíš potoky její.
\par 3 Takto praví Panovník Hospodin: Rozestrut na te sít svou skrze shromáždení národu mnohých, kteríž te vytáhnou nevodem mým.
\par 4 I nechám te na zemi, povrhu te na svrchku pole, a uciním, že na tobe prebývati bude všelijaké ptactvo nebeské, a nasytím tebou živocichy vší zeme.
\par 5 A rozmeci maso tvé po horách, a naplním údolí vysokostí tvou.
\par 6 A napojím zemi, v níž ploveš, krví tvou až do hor, tak že i potokové naplneni budou tebou.
\par 7 V tom, když te zhasím, zakryji nebesa, a zasmušilé uciním hvezdy jejich; slunce mrákotou zastru, a mesíc nebude svítiti svetlem svým.
\par 8 Všecka svetla jasná na nebesích zasmušilá uciním prícinou tvou, a uvedu tmu na zemi tvou, praví Panovník Hospodin.
\par 9 Nadto zkormoutím srdce národu mnohých, když zpusobím, aby došla povest o potrení tvém mezi národy, do zemí, jichž jsi neznal.
\par 10 Uciním, pravím, že trnouti budou nad tebou národové mnozí, a králové jejich hroziti se prícinou tvou velice, když šermovati budu mecem svým pred tvárí jejich. Budou se zajisté lekati každé chvilky, každý sám za sebe v den pádu tvého.
\par 11 Nebo takto praví Panovník Hospodin: Mec krále Babylonského prijde na te.
\par 12 Meci udatných porazím množství tvé, nejukrutnejších ze všech národu; tit zkazí pýchu Egypta, a zahlazeno bude všecko množství jeho.
\par 13 Zahladím i všecka hovada jeho, kteráž jsou pri vodách mnohých, tak že jich nezakalí noha clovecí více, aniž jich kaliti budou kopyta hovad.
\par 14 Tut uciním, že se usadí vody jejich, a potokové jejich že jako olej pujdou, praví Panovník Hospodin,
\par 15 Když obrátím zemi Egyptskou v poušt prehroznou, v zemi prázdnou toho, což prvé v ní bylo, a když zbiji v ní všecky obyvatele. I zvedí, že já jsem Hospodin.
\par 16 Tot jest naríkání, jímž naríkati budou. Tak dcery národu naríkati budou, tak nad Egyptem i nade vším jeho množstvím naríkati budou, dí Panovník Hospodin.
\par 17 Potom bylo dvanáctého léta, patnáctého dne téhož mesíce, že se stalo slovo Hospodinovo ke mne, rkoucí:
\par 18 Synu clovecí, naríkej nad množstvím Egypta, a snes jej i dcery národu tech slavných do zpodních míst zeme k tem, kteríž sstupují do jámy.
\par 19 A rci: Nad kohož bys utešenejší byl? Sstupiž a lež s neobrezanci.
\par 20 Mezi zbitými mecem padnou, meci vydán jest, vlectež jej i všecko množství jeho.
\par 21 Budout k nemu mluviti hrdiny s jeho pomocníky z prostred hrobu, kdež neobrezanci mecem zbití sstoupivše, leží.
\par 22 Tam jest Assur i všecka zber jeho, jehož hrobové jsou vukol tohoto. Všickni ti byvše zbiti, padli od mece.
\par 23 Jehož hrobové jsou po stranách jámy, aby byla zber jeho vukol hrobu tohoto. Všickni ti byvše zbiti, padli od mece, kteríž pouštívali strach v zemi živých.
\par 24 Tam Elam i všecko množství jeho vukol hrobu tohoto. Všickni ti neobrezanci byvše zbiti, padli od mece, a sstoupili do zpodních míst zeme, kteríž pouštívali strach svuj v zemi živých. Jižt nesou potupu svou s temi, kteríž sstupují do jámy.
\par 25 Mezi zbitými postavili jemu lože, i všemu množství jeho, vukol nehož jsou hrobové tohoto. Všickni ti neobrezanci zbiti mecem, nebo pouštín býval strach jejich v zemi živých. Jižt nesou potupu svou s temi, jenž sstupují do jámy, mezi zbitými položeni jsouce.
\par 26 Tam Mešech, Tubal i všecko množství jeho, a vukol neho hrobové tohoto. Všickni ti neobrezanci zbiti mecem, nebo pouštívali strach svuj v zemi živých.
\par 27 Actkoli ti ješte nelehli s hrdinami, kteríž padli z neobrezancu, kteríž sstoupili do hrobu s zbrojí svou vojenskou, a podložili mece své pod hlavy své, a však dujdet nepravost jejich na kosti jejich; nebo strach hrdin byl v zemi živých.
\par 28 I ty mezi neobrezanci potrín budeš, a lehneš s zbitými mecem.
\par 29 Tam Edom, králové jeho, i všecka knížata jeho, kteríž položeni jsou i s svou mocí s zbitými mecem. I ti s neobrezanci lehnou a s temi, kteríž sstupují do jámy.
\par 30 Tam knížata pulnocní strany všickni naporád, i všickni Sidonští, kteríž sstoupí k zbitým, s strachem svým, za svou moc stydíce se, a ležeti budou ti neobrezanci s zbitými mecem, a ponesou potupu svou s temi, kteríž sstupují do jámy.
\par 31 Ty uhlédaje Farao, poteší se nade vším množstvím svým, Farao i všecko vojsko jeho, zbiti jsouce mecem, dí Panovník Hospodin.
\par 32 Nebo pustím strach svuj v zemi živých, a položen bude mezi neobrezanci s zbitými mecem Farao i všecko množství jeho, praví Panovník Hospodin.

\chapter{33}

\par 1 Opet stalo se slovo Hospodinovo ke mne, rkoucí:
\par 2 Synu clovecí, mluv k synum lidu svého a rci jim: Když uvedu na zemi nekterou mec, jestliže vezme lid té zeme muže jednoho od koncin svých, a ustanoví jej sobe za strážného,
\par 3 A ten vida mec pricházející na tu zemi, troubil-li by na troubu a napomínal lidu,
\par 4 A slyše nekdo zvuk trouby, však by se nedal napomenouti, a v tom prijda mec, shladil by jej: krev jeho na hlavu jeho bude.
\par 5 Nebo slyšel hlas trouby, však nedal se napomenouti; krev jeho na nem zustane.Byt se byl napomenouti dal, duši svou byl by vysvobodil.
\par 6 Pakli strážný vida, an mec prichází, však by nezatroubil na troubu, a lid by nebyl napomenut, a prijda mec, zachvátil by nekoho z nich: ten pro nepravost svou zachvácen bude, ale krve jeho z ruky strážného toho vyhledávati budu.
\par 7 Tebe pak synu clovecí, tebe jsem strážným ustanovil nad domem Izraelským, abys slyše z úst mých slovo, napomínal jich ode mne.
\par 8 Když bych já rekl bezbožnému: Bezbožníce, smrtí umreš, a nemluvil bys, vystríhaje bezbožného od cesty jeho: ten bezbožný pro nepravost svou umre, ale krve jeho z ruky tvé vyhledávati budu.
\par 9 Pakli bys ty vystríhal bezbožného od cesty jeho, tak aby se od ní odvrátil, a však neodvrátil by se od cesty své: ont pro nepravost svou umre, ale ty duši svou vysvobodíš.
\par 10 Protož ty synu clovecí, rci domu Izraelskému: Takto mluvíte, ríkajíce: Proto že prestoupení naše a hríchové naši jsou na nás, a my v nich svadneme, i jakž bychom živi byli?
\par 11 Rci jim: Živt jsem já, dí Panovník Hospodin, žet nemám líbosti v smrti bezbožného, ale aby se odvrátil bezbožný od cesty své a živ byl. Odvrattež se, odvratte od cest svých zlých. I proc mríti máte, ó dome Izraelský?
\par 12 Ty tedy synu clovecí, rci synum lidu svého: Spravedlnost spravedlivého nevytrhne ho v den prestoupení jeho, aniž bezbožný v své bezbožnosti padne, v kterýž by se den odvrátil od bezbožnosti své; tolikéž spravedlivý nebude moci živ býti v ní, v kterýž by den zhrešil.
\par 13 Jestliže dím spravedlivému: Jiste živ budeš, a on doufaje v spravedlnost svou, cinil by nepravost: žádná spravedlnost jeho neprijde na pamet, ale pro tu nepravost svou, kterouž cinil, umre.
\par 14 Zase reknu-li bezbožnému: Smrtí umreš, však odvrátí-li se od hríchu svého, a ciniti bude soud a spravedlnost,
\par 15 Což v zástave jest, navrátí-li bezbožný, což vydrel, nahradí-li, v ustanoveních života bude-li choditi, necine nepravosti: jiste že bude živ, neumre.
\par 16 Žádní hríchové jeho, jimiž hrešil, nebudou mu zpomínáni; soud a spravedlnost cinil, jiste že bude živ.
\par 17 A vždy ríkají synové lidu tvého: Není pravá cesta Páne, ješto jejich cesta není pravá.
\par 18 Když by se odvrátil spravedlivý od spravedlnosti své, a cinil by nepravost, umret pro ty veci.
\par 19 Ale když by se odvrátil bezbožný od bezbožnosti své, a cinil by soud a spravedlnost, podlé tech vecí živ bude.
\par 20 A predce ríkáte: Není pravá cesta Páne. Každého z vás podlé cest jeho souditi budu, ó dome Izraelský.
\par 21 Stalo se pak dvanáctého léta, desátého mesíce, pátého dne téhož mesíce od zajetí našeho, že prišel ke mne jeden, kterýž ušel z Jeruzaléma, rka: Dobyto jest mesto.
\par 22 Ruka pak Hospodinova byla pri mne u vecer pred tím, než prišel ten, kterýž utekl, a otevrela ústa má, až i ke mne prišel ráno; otevrela, pravím, ústa má, abych nebyl nemým déle.
\par 23 I stalo se slovo Hospodinovo ke mne, rkoucí:
\par 24 Synu clovecí, obyvatelé pustin techto v zemi Izraelské mluví, rkouce: Jedinký byl Abraham, a dedicne držel zemi tuto, nás pak mnoho jest; námt dána jest zeme tato v dedictví.
\par 25 Protož rci jim: Takto praví Panovník Hospodin: Se krví jídáte, a ocí svých pozdvihujete k ukydaným modlám svým, i krev vyléváte, a chteli byste zemí touto dedicne vládnouti?
\par 26 Stojíte na meci svém, pášete ohavnost, a každý ženy bližního svého poškvrnujete, a chteli byste zemí touto dedicne vládnouti?
\par 27 Takto mluv k nim: Takto praví Panovník Hospodin: Živt jsem já, že ti, kteríž jsou na pustinách, mecem padnou, a kdo na poli, toho zveri dám k sežrání, kdo pak na hradích neb v jeskyních, morem zemrou.
\par 28 I obrátím zemi tu v hroznou poušt a prestane vyvýšenost moci její, a zpustnou hory Izraelské, tak že nebude žádného, kdo by šel pres ne.
\par 29 I zvedí, že já jsem Hospodin, když obrátím zemi tu v hroznou poušt pro všecky ohavnosti jejich, kteréž páchali.
\par 30 Ty pak synu clovecí, slyš, synové lidu tvého casto mluvívají o tobe, za stenami i ve dverích domu, a ríkají jeden druhému a každý bratru svému, rka: Podte medle a poslechnete, jaké slovo vyšlo od Hospodina.
\par 31 I scházejí se k tobe, tak jako se schází lid, a sedají pred tebou lid muj, a poslouchají slov tvých, ale neciní jich. A ackoli je sobe ústy svými libují, však za mrzkým ziskem svým srdce jejich odchází.
\par 32 A aj, ty jsi jim jako zpev libý pekného zvuku a dobre vznející. Slyšít zajisté slova tvá, ale žádný jich neciní.
\par 33 Než když to prijde, (aj, pricházít), tedy zvedí, že prorok byl u prostred nich.

\chapter{34}

\par 1 I stalo se slovo Hospodinovo ke mne, rkoucí:
\par 2 Synu clovecí, prorokuj proti pastýrum Izraelským. Prorokuj a rci jim, tem pastýrum: Takto praví Panovník Hospodin: Beda pastýrum Izraelským, kteríž pasou sami sebe. Zdaliž pastýri nemají stáda pásti?
\par 3 Tuk jídáte, a vlnou se odíváte, což tucného, zabijíte, stáda však nepasete.
\par 4 Neduživých neposilujete, a nemocné nehojíte, a zlámané neuvazujete, a zaplašené zase neprivodíte, a zahynulé nehledáte, ale prísne a tvrde panujete nad nimi,
\par 5 Tak že rozptýleny jsou, nemajíce pastýre, a rozptýleny jsouce, jsou za pokrm všelijaké zveri polní.
\par 6 Bloudí stádo mé po všech horách, a na každém pahrbku vysokém, nýbrž po vší zeme širokosti rozptýleny jsou ovce stáda mého, a není žádného, kdo by se po nich ptal, ani žádného, kdo by jich hledal.
\par 7 Protož ó pastýri, slyšte slovo Hospodinovo:
\par 8 Živt jsem já, praví Panovník Hospodin, zajisté proto že stádo mé bývá v loupež, a ovce stáda mého bývají k sežrání všelijaké zveri polní, nemajíce žádného pastýre, aniž se ptají pastýri moji po stádu mém, ale pasou pastýri sami sebe, stáda pak mého nepasou:
\par 9 Protož vy pastýri, slyšte slovo Hospodinovo:
\par 10 Takto praví Panovník Hospodin: Aj, já jsem proti pastýrum tem, a budu vyhledávati stáda mého z ruky jejich, a zastavím jim pasení stáda, aby nepásli více ti pastýri samých sebe. Vytrhnu zajisté ovce své z úst jejich, aby jim nebyly za pokrm.
\par 11 Nebo takto praví Panovník Hospodin: Aj já, já ptáti se budu po ovcích svých a shledávati je.
\par 12 Jakož shledává pastýr stádo své tehdáž, když bývá u prostred ovec svých rozptýlených: tak shledávati budu stádo své, a vytrhnu je ze všech míst, kamž v den oblaku a mrákoty rozptýleny byly.
\par 13 A vyvedu je z národu, a shromáždím je z zemí, a uvedu je do zeme jejich, a pásti je budu na horách Izraelských, pri potocích i na všech místech k bydlení príhodných v zemi té.
\par 14 Na pastve dobré pásti je budu, a na horách vysokých Izraelských bude ovcinec jejich. Tamt léhati budou v ovcinci veselém, a pastvou tucnou pásti se budou na horách Izraelských.
\par 15 Já pásti budu stádo své, a já zpusobím to, že odpocívati budou, praví Panovník Hospodin.
\par 16 Zahynulé hledati budu, a zaplašenou zase privedu, a polámanou uvíži, a nemocné posilím, tucnou pak a silnou zahladím; nebo je pásti budu v soudu.
\par 17 Vy pak, stádo mé, slyšte: Takto praví Panovník Hospodin: Aj, já soudím mezi dobytcetem a dobytcetem, mezi skopci a kozly.
\par 18 Což jest vám málo pastvou dobrou se pásti, že ješte ostatek pastvy vaší pošlapáváte nohama svýma? a ucištenou vodu píti, že ostatek nohama svýma kalíte,
\par 19 Tak aby ovce mé tím, což vy nohama pošlapáte, se pásti, a kal noh vašich píti musily?
\par 20 Protož takto praví Panovník Hospodin k nim: Aj já, já souditi budu mezi dobytcetem tucným a mezi dobytcetem hubeným,
\par 21 Proto že boky i plecemi strkáte, a rohy svými trkáte všecky neduživé, tak že je vyháníte i ven.
\par 22 Protož vysvobodím stádo své, aby nebylo více v loupež, a souditi budu mezi dobytcetem a dobytcetem.
\par 23 A vzbudím nad nimi pastýre jednoho, kterýž je pásti bude, služebníka svého Davida. Tent je pásti bude, a ten bude jejich pastýrem.
\par 24 Já pak Hospodin budu jejich Bohem, a služebník muj David knížetem u prostred nich. Já Hospodin mluvil jsem.
\par 25 A ucine s nimi smlouvu pokoje, zpusobím, že prestane zver zlá na zemi; i budou bydleti na poušti bezpecne, a spáti i po lesích.
\par 26 K tomu obdarím je i okolí pahrbku svého požehnáním, a ssílati budu déšt casem svým; deštové požehnání budou bývati;
\par 27 Tak že vydá drevo polní ovoce své a zeme vydá úrodu svou; i budou v zemi své bezpecní, a zvedí, že já jsem Hospodin, když polámi závory jha jejich, a vytrhnu je z ruky tech, jenž je v službu podrobují.
\par 28 I nebudou více loupeží národum, a zver zemská nebude jich žráti, ale bydliti budou bezpecne, aniž jich kdo prestraší.
\par 29 Nadto vzbudím jim výstrelek k sláve, a nebudou více mríti hladem v té zemi, aniž ponesou více potupy od pohanu.
\par 30 I zvedí, že já Hospodin Buh jejich jsem s nimi, a oni lid muj, dum Izraelský, praví Panovník Hospodin.
\par 31 Vy pak ovce mé, ovce pastvy mé, jste vy lidé, a já Buh váš, praví Panovník Hospodin.

\chapter{35}

\par 1 Opet stalo se slovo Hospodinovo ke mne, rkoucí:
\par 2 Synu clovecí, obrat tvár svou proti hore Seir, a prorokuj proti ní.
\par 3 A rci jí: Takto praví Panovník Hospodin: Aj, já jsem proti tobe, horo Seir, a vztáhnu ruku svou na tebe, a obrátím te v hroznou poušt.
\par 4 Mesta tvá v poušt obrátím, a ty budeš pustinou, i zvíš, že já jsem Hospodin,
\par 5 Proto že máš neprátelství vecné, a rozptyluješ syny Izraelské mecem v cas bídy jejich, v cas dokonání nepravosti.
\par 6 Protož živt jsem já, praví Panovník Hospodin, že k zabití pripravím te, a krev te stihati bude. Ponevadž krve vylévání v nenávisti nemáš, také te krev stihati bude.
\par 7 Obrátím, pravím, horu Seir v hroznou poušt, a vypléním z ní jdoucího pres ni i vracujícího se.
\par 8 A naplním hory její zbitými jejími; na pahrbcích tvých i v údolích tvých, i pri všech potocích tvých zbití mecem padati budou.
\par 9 V pustiny vecné obrátím te, tak že mesta tvá nebudou zase vzdelána, i zvíte, že já jsem Hospodin.
\par 10 Proto že ríkáš: Tito dva národové a tyto dve zeme mé budou, a budeme dedicne vládnouti tou, v níž Hospodin prebýval,
\par 11 Protož živt jsem já, praví Panovník Hospodin, žet uciním podlé hnevu tvého a podlé závisti tvé, kterouž jsi prokázala z nenávisti své proti nim. I budu poznán od nich, když te budu souditi.
\par 12 A zvíš, že já Hospodin slyšel jsem všecka hanení tvá, kteráž jsi mluvila o horách Izraelských, rkuci: Zpuštenyt jsou, námt jsou dány k sežrání.
\par 13 Nebo jste se honosili proti mne ústy svými, a množili proti mne slova svá, což jsem sám slyšel.
\par 14 Takto praví Panovník Hospodin: Jakž se veselí všecka ta zeme, tak pustinu uciním z tebe.
\par 15 Jakž se veselíš nad dedictvím domu Izraelského, proto že zpustl, tak uciním i tobe. Pustinou budeš, ó horo Seir, i všecka zeme Idumejská naskrze, i zvedít, že já jsem Hospodin.

\chapter{36}

\par 1 Ty synu clovecí, prorokuj i o horách Izraelských a rci: Hory Izraelské, slyšte slovo Hospodinovo.
\par 2 Takto praví Panovník Hospodin: Proto že ríká ten neprítel o vás: Aj, to dobre, takét i výsosti vecné v dedictví se nám dostanou,
\par 3 Protož prorokuj a rci: Takto praví Panovník Hospodin: Proto, proto že poplénili a sehltili vás vukol, abyste byli za dedictví ostatku národu, a vydáni jste v pomluvu a v zlou povest lidem,
\par 4 Protož hory Izraelské, slyšte slovo Panovníka Hospodina: Takto praví Panovník Hospodin horám i pahrbkum, potokum i údolím, pustinám hrozným i mestum opušteným, jenž jsou za loupež a posmech ostatku národu okolních,
\par 5 Protož takto dí Panovník Hospodin: jiste že v ohni horlivosti své mluviti budu proti ostatku tech národu, i proti vší zemi Idumejské, kteríž sobe osobili zemi mou za dedictví, veselíce se z celého srdce, a pléníce s chutí, aby sídlo vyhnaných z neho bylo v loupež.
\par 6 Protož prorokuj o zemi Izraelské, a mluv k horám a pahrbkum, potokum i k údolím: Takto praví Panovník Hospodin: Aj, já v horlivosti své a v prchlivosti své mluvím, proto že pohanení od národu snášíte.
\par 7 Protož takto praví Panovník Hospodin: já zdvihna ruku svou, prisahám, že ti národové, kteríž jsou vukol vás, pohanení své nésti musejí.
\par 8 Vy pak hory Izraelské, ratolesti své vypoušteti, a ovoce své prinášeti budete lidu mému Izraelskému, když se priblíží a prijdou.
\par 9 Nebo aj, já jsem s vámi, a patrím na vás, abyste delány a osívány byly.
\par 10 A rozmnožím v vás lidi, všecken dum Izraelský, jakýžkoli jest, i budou se osazovati mesta, a pustiny vzdelávati.
\par 11 Rozmnožím, pravím, v vás lidi i hovada, a budou se množiti a ploditi, i uciním, že bydliti budete jako za predešlých let vašich, cine vám dobre, více než v prvotinách vašich. A zvíte, že já jsem Hospodin.
\par 12 Nebo uvedu do vás lidi, lid svuj Izraelský, a budou tebou dedicne vládnouti; a budeš jim za dedictví, aniž kdy více siroby na ne uvedeš.
\par 13 Takto praví Panovník Hospodin: Proto že ríkají vám, že jsi ty ta zeme, kteráž zžíráš lidi, a sirotky ciníš z národu svých,
\par 14 Protož nebudeš více zžírati lidi, a národu svých více k sirobe privoditi, praví panovník Hospodin.
\par 15 Aniž dopustím, aby více slýcháno bylo v tobe potupy národu, aniž útržky lidské snášeti budeš více, ani ku pádu privoditi více národu svých, praví Panovník Hospodin.
\par 16 I stalo se slovo Hospodinovo ke mne, rkoucí:
\par 17 Synu clovecí, dum Izraelský, bydlíce v zemi své, poškvrnili jí cestou svou a skutky svými; podobná necistote ženy pro necistotu oddelené byla cesta jejich pred ocima mýma.
\par 18 Procež vylil jsem prchlivost svou na ne, proto že krev vylévali na zemi, a ukydanými bohy svými poškvrnili jí.
\par 19 I rozptýlil jsem je po národech, tak že rozplašeni jsou po zemích; podlé cesty jejich a podlé skutku jejich soudil jsem je.
\par 20 Nýbrž odšedše mezi národy, kamž prišli, poškvrnili jména svatosti mé, když ríkáno o nich: Že jsou lid Hospodinuv, a že z jeho zeme vyšli.
\par 21 Ale slitoval jsem se pro jméno svatosti své, kteréhož poškvrnili dum Izraelský mezi národy, kamž prišli.
\par 22 Protož rci domu Izraelskému: Takto praví Panovník Hospodin: Ne pro vás já to uciním, ó dome Izraelský, ale pro jméno svatosti své, kteréhož jste poškvrnili mezi národy, kamž jste prišli,
\par 23 Abych posvetil jména svého velikého, kteréž bylo poškvrneno mezi národy, kteréhož jste poškvrnili u prostred nich, aby poznali národové, že já jsem Hospodin, praví Panovník Hospodin, když posvecen budu v vás pred ocima jejich.
\par 24 Nebo poberu vás z národu, a shromáždím vás ze všech zemí, a uvedu vás do zeme vaší.
\par 25 A pokropím vás vodou cistou, a cisti budete; ode všech poškvrn vašich, i ode všech ukydaných bohu vašich ocistím vás.
\par 26 A dám vám srdce nové, a ducha nového dám do vnitrností vašich, a odejma srdce kamenné z tela vašeho, dám vám srdce masité.
\par 27 Ducha svého, pravím, dám do vnitrností vašich, a uciním, abyste v ustanoveních mých chodili, a soudu mých ostríhali a cinili je.
\par 28 I budete bydliti v zemi, kterouž jsem byl dal otcum vašim, a budete lidem mým, a já budu vaším Bohem.
\par 29 Nebo vysvobodím vás ze všelijakých poškvrn vašich, a privolám obilé, a rozmnožím je, a nedopustím na vás hladu.
\par 30 Rozmnožím i ovoce stromu a úrody polní, tak že neponesete více potupy hladu mezi národy.
\par 31 I rozpomenete se na zlé cesty vaše a na skutky vaše nedobré, tak že sami se býti hodné ošklivosti uznáte pro nepravosti vaše a pro ohavnosti vaše.
\par 32 Ne pro vás já to uciním, praví Panovník Hospodin, známo vám bud. Zahanbetež se a zastydte za cesty své, dome Izraelský.
\par 33 Takto praví Panovník Hospodin: V ten den, v kterémž vás ocistím ode všech nepravostí vašich, osadím mesta, a vzdelány budou pustiny.
\par 34 A tak zeme pustá bude delána, kteráž prvé byla pouští pred ocima každého tudy jdoucího.
\par 35 I reknou: Zeme tato zpuštená jest jako zahrada Eden, ano i mesta pustá, zkažená a zborená jsou ohrazena a osazena.
\par 36 I zvedí národové, kteríž pozustanou vukol vás, že já Hospodin vystavel jsem zboreniny, a vysadil pustiny. Já Hospodin mluvil jsem i uciním.
\par 37 Takto praví Panovník Hospodin: Ješte toho hledati bude pri mne dum Izraelský, abych je rozmnožil. Rozmnožím je lidmi jako stády.
\par 38 Jako stádo k obetem, jako stádo Jeruzalémské pri slavnostech jeho, tak mesta pustá budou plná stád lidí, i zvedí, že já jsem Hospodin.

\chapter{37}

\par 1 Byla nade mnou ruka Hospodinova, a vyvedl mne Hospodin v duchu, a postavil mne u prostred údolí, kteréž bylo plné kostí.
\par 2 I provedl mne skrze ne vukol a vukol, a aj, bylo jich velmi mnoho v tom údolí, a aj, byly velmi suché.
\par 3 I rekl mi: Synu clovecí, mohly-li by ožiti kosti tyto? I rekl jsem: Panovníce Hospodine, ty víš.
\par 4 V tom rekl mi: Prorokuj o tech kostech a rci jim: Kosti suché, slyšte slovo Hospodinovo.
\par 5 Toto praví Panovník Hospodin kostem temto: Aj, já uvedu do vás ducha, abyste ožily.
\par 6 A dám na vás žily, a uciním, že zroste na vás maso, a otáhnu vás koží; dám, pravím, do vás ducha, abyste ožily, i zvíte, že já jsem Hospodin.
\par 7 Tedy prorokoval jsem tak, jakž mi rozkázáno bylo. I stal se zvuk, když jsem já prorokoval, a aj, hrmot, když se približovaly kosti jedna k druhé.
\par 8 I videl jsem, a aj, žily a maso na nich se ukázalo, i koží potaženy byly po vrchu, ale ducha žádného nebylo v nich.
\par 9 I rekl mi: Prorokuj k duchu, prorokuj, synu clovecí, a rci duchu: Takto praví Panovník Hospodin: Ode ctyr vetru prid, duchu, a vej na tyto zmordované, at oživou.
\par 10 Tedy prorokoval jsem, jakž mi rozkázal. I všel do nich duch, a ožili, a postavili se na nohách svých, zástup velmi veliký.
\par 11 I rekl mi: Synu clovecí, kosti tyto jsou všecken dum Izraelský. Aj, ríkají: Uschly kosti naše, a zhynula cáka naše, jižt jest po nás.
\par 12 Protož prorokuj a rci jim: Takto praví Panovník Hospodin: Aj, já otevru hroby vaše, a vyvedu vás z hrobu vašich, lide muj, a uvedu vás do zeme Izraelské.
\par 13 I zvíte, že já jsem Hospodin, když otevru hroby vaše, a vyvedu vás z hrobu vašich, lide muj.
\par 14 A dám Ducha svého do vás, abyste ožili, a osadím vás v zemi vaší, i zvíte, že já Hospodin mluvím i ciním, dí Hospodin.
\par 15 Opet stalo se slovo Hospodinovo ke mne, rkoucí:
\par 16 Ty pak synu clovecí, vezmi sobe drevo jedno, a napiš na nem: Judovi, a synum Izraelským, tovaryšum jeho. Zatím vezma drevo druhé, napiš na nem: Jozefovi drevo Efraimovo, a všechnem synum Izraelským, tovaryšum jeho.
\par 17 I spojž je sobe jedno k druhému v jedno drevo, aby byla jako jedno v ruce tvé.
\par 18 A když mluviti budou k tobe synové lidu tvého, rkouce: Což nám neoznámíš, co ty veci znamenají:
\par 19 Mluv k nim: Takto praví panovník Hospodin: Aj, já vezmu drevo Jozefovo, kteréž jest v ruce Efraimove, a pokolení Izraelských, tovaryšu jeho, a priložím je s ním k drevu Judovu, a uciním je drevem jedním, i budou jedno v ruce mé.
\par 20 A když budou ta dreva, na kterýchž jsi psal, v ruce tvé pred ocima jejich,
\par 21 Tedy mluv k nim: Takto praví Panovník Hospodin: Aj, já vezmu syny Izraelské z prostred národu tech, kamž odešli, a shromážde je odevšad, uvedu je do zeme jejich.
\par 22 A uciním je, aby byli národem jedním v té zemi na horách Izraelských, a král jeden bude nad nimi nade všemi králem. A nebudou více dva národové, aniž se již více deliti budou na dvoje království.
\par 23 Aniž se budou poškvrnovati více ukydanými bohy svými, a ohavnostmi svými, ani jakými prestoupeními svými, i vysvobodím je ze všech obydlí jejich, v nichž hrešili, a ocistím je. I budou mým lidem, a já budu jejich Bohem.
\par 24 A služebník muj David bude králem nad nimi, a pastýre jednoho všickni míti budou, aby v soudech mých chodili, a ustanovení mých ostríhali, i cinili je.
\par 25 I budou bydliti v té zemi, kterouž jsem byl dal služebníku svému Jákobovi, v níž bydlili otcové vaši. Budou, pravím, v ní bydliti oni i synové jejich, i synové synu jejich až na veky, a David služebník muj knížetem jejich bude na veky.
\par 26 Nadto uciním s nimi smlouvu pokoje; smlouva vecná bude s nimi. A rozsadím je, i rozmnožím je, a postavím svatyni svou u prostred nich na veky.
\par 27 Bude i príbytek muj mezi nimi, a budu jejich Bohem, a oni budou lidem mým.
\par 28 I zvedí národové, že jsem já Hospodin, jenž posvecuji Izraele, když bude svatyne má u prostred nich na veky.

\chapter{38}

\par 1 Opet stalo se slovo Hospodinovo ke mne, rkoucí:
\par 2 Synu clovecí, obrat tvár svou proti Gogovi, zeme Magog, knížeti a hlave v Mešech a Tubal, a prorokuj proti nemu,
\par 3 A rci: Takto praví Panovník Hospodin: Aj, já budu proti tobe, Gogu, kníže a hlavo v Mešech a Tubal.
\par 4 A odvedu te zpet, dada udice do celistí tvých, když te vyvedu i všecko vojsko tvé, kone i jezdce, všecky oblecené v celou zbroj, zástup veliký s pavézami a štíty, všecky ty, kteríž užívají mece:
\par 5 Perské, Moureníny i Putské s nimi, všecky ty s štíty a lebkami,
\par 6 Gomera i všecky houfy jeho, dum Togarmy od stran pulnocních, i všecky houfy jeho, národy mnohé s tebou.
\par 7 Budiž hotov, a priprav se, ty i všecko shromáždení tvé, tech, kteríž se k tobe sebrali, a bud strážcím jejich.
\par 8 Po mnohých dnech navštíven budeš, v potomních letech pritáhneš na lid vysvobozený od mece, a shromáždený z národu mnohých na hory Izraelské, kteréž byly pustinou ustavicne, když oni z národu jsouce vyvedeni, budou bydliti bezpecne všickni.
\par 9 V tom pritáhneš a prijdeš jako boure, budeš jako oblak prikrývající zemi, ty i všickni houfové tvoji, i národové mnozí s tebou.
\par 10 Takto praví Panovník Hospodin: I stane se v ten den, že vstoupí mnohé veci na srdce tvé, a budeš mysliti myšlení zlé.
\par 11 A díš: Potáhnu na zemi, v níž jsou vsi, pritáhnu na ty, jenž pokoje užívají, bydlíce bezpecne, na všecky, kteríž bydlejí b\par všech zdí, a nemají žádné závory ani bran,
\par 12 Abych vzebral koristi, a rozchvátal loupeže, obraceje ruku svou proti pouštem již osazeným, a proti lidu zase shromáždenému z národu, zacházejícímu s dobytkem a jmením, bydlejícím u prostred zeme.
\par 13 Sába a Dedan, a kupci morští, i všecka lvícata jeho reknou tobe: K rozebrání-liž koristí ty se béreš? K rozchvátání-li loupeže shromáždil jsi množství své, abys bral stríbro a zlato, abys nabral dobytka i zboží, a nashromáždil loupeže veliké?
\par 14 Protož synu clovecí, prorokuj a rci Gogovi: Takto praví Panovník Hospodin: Zdaliž toho dne, když bude lid muj Izraelský bydliti bezpecne, nezvíš,
\par 15 Kdyžto pritáhnouce z místa svého od stran pulnocních, ty i národové mnozí s tebou, sedíce na koních všickni, shromáždení veliké a vojsko znamenité,
\par 16 Potáhneš na lid muj Izraelský jako oblak, abys prikryl tu zemi? V potomních dnech stane se, že te privedu na zemi svou, aby mne poznali národové, když posvecen budu v tobe pred ocima jejich, ó Gogu.
\par 17 Takto praví Panovník Hospodin: Zdaliž ty nejsi ten, o kterémž jsem mluvil za dnu starodávních skrze služebníky své, proroky Izraelské, kteríž prorokovali za tech dnu a let, že te privedu na ne?
\par 18 A však stane se v ten den, v den, v kterýž pritáhne Gog na zemi Izraelskou, praví Panovník Hospodin, že povstane prchlivost má s hnevem mým,
\par 19 A v rozhorlení svém, v ohni prchlivosti své mluviti budu. Jiste že v ten den bude pohnutí veliké v zemi Izraelské,
\par 20 Tak že se pohnou pred tvárí mou ryby morské i ptactvo nebeské, a zver polní i všeliký zemeplaz plazící se po zemi, i všickni lidé, kteríž jsou na svrchku zeme. I rozválejí se hory, a padnou výsosti, i každá zed na zem upadne.
\par 21 Nebo zavolám proti nemu po všech horách mých mece, praví Panovník Hospodin; mec každého proti bratru jeho bude.
\par 22 A vykonám pri nem soud morem a krve prolitím a prívalem rozvodnilým, a kamením krupobití velikého, ohnem a sirou dštíti budu na nej i na houfy jeho, a na národy mnohé, kteríž jsou s ním.
\par 23 A tak zveleben, a posvecen, a v známost uveden budu pred ocima národu mnohých, a zvedí, že já jsem Hospodin.

\chapter{39}

\par 1 Ty synu clovecí, prorokuj ješte proti Gogovi a rci: Takto praví Panovník Hospodin: Aj, já povstanu proti tobe, ó Gogu, kníže a hlavo v Mešech a Tubal.
\par 2 A odvedu te zpet, navštíve te šesti ranami, když te vzbudím, abys pritáhl od stran pulnocních, a privedu te na hory Izraelské.
\par 3 Nebo vyrazím lucište tvé z ruky tvé levé, a strely z ruky tvé pravé vyvrhu.
\par 4 Na horách Izraelských padneš ty i všickni houfové tvoji i národové, kteríž budou s tebou; ptákum a všemu ptactvu krídla majícímu i zveri polní dám te k sežrání.
\par 5 Na svrchku pole padneš; nebot jsem já mluvil, praví Panovník Hospodin.
\par 6 Vypustím také ohen na Magoga i na ty, kteríž prebývají na ostrovích bezpecne, i zvedít, že já jsem Hospodin.
\par 7 A jméno svatosti své uvedu v známost u prostred lidu svého Izraelského. Nedopustím, pravím, více poškvrnovati jména svatosti své, i zvedít národové, že já jsem Hospodin Svatý v Izraeli.
\par 8 Aj, prijdet a stane se to, praví Panovník Hospodin, téhož dne, o kterémž jsem mluvil.
\par 9 Tehdy vyjdou obyvatelé mest Izraelských, a zapálíce, popálí zbroj a štíty i pavézy, lucište i strely, drevce i kopí, a budou je páliti ohnem sedm let.
\par 10 Aniž nositi budou dríví s pole, ani sekati v lesích, proto že zbrojí zanecovati budou ohen, když zloupí ty, kteríž je loupívali, a mocí poberou tem, kteríž jim mocí brávali, praví Panovník Hospodin.
\par 11 I stane se v ten den, že dám Gogovi místo ku pohrbu tam v Izraeli, údolí, kudyž se jde k východní strane k mori, kteréžto zacpá ústa tudy jdoucích. I pohrbí tam Goga i všecko množství jeho, a nazovou je údolí množství Gogova.
\par 12 Nebo pochovávati je budou dum Izraelský za sem mesícu proto, aby vycistili zemi.
\par 13 A tak pohrbí je všecken lid té zeme, a bude jim to ku poctivosti v den, v kterýž oslaven budu, praví Panovník Hospodin.
\par 14 Oddelít pak muže statecné, kteríž by procházeli tu zemi, proto aby pochovávali ty, kteríž by pozustali na svrchku zeme, aby ji vycistili. Po vyjití sedmi mesícu prehledávati zacnou.
\par 15 A ti procházejíce, choditi budou po zemi, a když uzrí kosti clovecí, vzdelají pri nich znamení pametné, aby je pohrbili hrobari v údolí množství Gogova.
\par 16 Nýbrž to množství jeho bude k sláve i mestu, když vycistí tu zemi.
\par 17 Ty pak synu clovecí, takto praví Panovník Hospodin: Rci ke všelijakým ptákum krídla majícím i ke všelijaké zveri polní: Shromaždte se a pridte, zberte se odevšad k obetem mým, kterýchž já nabiji vám,obetí velikých na horách Izraelských, a budete jísti maso a píti krev.
\par 18 Maso silných reku jísti budete, a krev knížat zemských píti, skopcu, beranu a kozlu, volku, všecko tucných Bázanských.
\par 19 A budete jísti tuk do sytosti, a píti krev do opití z obetí mých, kterýchž vám nabiji.
\par 20 A nasytíte se z stolu mého konmi i jezdci, silnými reky i všemi muži válecnými, dí Panovník Hospodin.
\par 21 A tak zjevím slávu svou mezi národy, aby videli všickni národové soud muj, kterýž jsem vykonal, i ruku mou, kterouž jsem doložil na ne.
\par 22 I zvít dum Izraelský, že já jsem Hospodin Buh jejich, od tohoto dne i potom.
\par 23 Zvedí také i národové, že pro svou nepravost zajat jest dum Izraelský, proto že se dopoušteli prestoupení proti mne; procež jsem skryl tvár svou pred nimi, a vydal jsem je v ruku protivníku jejich, aby padli mecem všickni naporád.
\par 24 Podlé necistoty jejich a podlé zproneverení se jejich ucinil jsem jim, a skryl jsem tvár svou pred nimi.
\par 25 Z té príciny takto praví Panovník Hospodin: Jižt zase privedu zajaté Jákobovy, a smiluji se nade vším domem Izraelským a horliti budu prícinou jména svatosti své,
\par 26 Ac ponesou potupu svou a všecko prestoupení své, kteréhož se dopustili proti mne, když bydlili v zemi své bezpecne, aniž byl, kdo by je prestrašil.
\par 27 A však je zase privedu z národu, a shromáždím je z zemí neprátel jejich, a posvecen budu v nich pred ocima národu mnohých.
\par 28 A tak poznají, že já jsem Hospodin Buh jejich, když zaveda je do národu, shromáždím je do zeme jejich, a nepozustavím tam žádného z nich.
\par 29 Aniž skryji více tvári své pred nimi, když vyleji Ducha svého na dum Izraelský, praví Panovník Hospodin.

\chapter{40}

\par 1 Dvadcátého pátého léta od zajetí našeho, na pocátku roku, desátého dne mesíce, ctrnáctého léta po dobytí mesta, práve v tentýž den byla nade mnou ruka Hospodinova, a uvedl mne tam.
\par 2 U videních Božích privedl mne do zeme Izraelské, a postavil mne na hore velmi vysoké, pri níž bylo jako stavení mesta, ku poledni.
\par 3 I uvedl mne tam, a aj, muž, kterýž na pohledení byl jako med, maje šnuru lnenou v ruce své, a prut k rozmerování, a ten stál v bráne.
\par 4 I mluvil ke mne ten muž: Synu clovecí, viz ocima svýma, a ušima svýma slyš, a prilož srdce své ke všemu, což já ukáži tobe; nebo proto atbych ukazoval, priveden jsi sem. Ty pak oznámíš všecko domu Izraelskému, což vidíš.
\par 5 A aj, zed zevnitr pri domu vukol a vukol, a v ruce muže toho prut míry, šesti loket, (loket o dlan delší než obecní). Zmeril šírku toho stavení jednoho prutu, a výšku jednoho prutu.
\par 6 A všed do brány, kteráž byla naproti východu, šel na horu po stupních jejích, i zmeril prah brány jednoho prutu zšírí, a prah druhý jednoho prutu zšírí.
\par 7 A pokojíky jednoho prutu zdélí, a jednoho prutu zšírí, mezi pokojíky pak pet loket, a prah brány podlé sínce brány vnitr prutu jednoho.
\par 8 I zmeril sínci brány vnitr na jeden prut.
\par 9 Zmeril také sínci brány osmi loket, a vereje její dvou loket, totiž sínce brány vnitr.
\par 10 I pokojíky brány východní, tri s jedné a tri s druhé strany. Jedné míry byly všecky tri, a míra jednostejná verejí po obou stranách.
\par 11 Zmeril též i širokost dverí té brány desíti loket, a dlouhost též brány trinácti loket,
\par 12 A prístreší pred pokojíky jednoho lokte, a na jeden loket prístreší s této strany, pokojíky pak šesti loket s jedné, a šesti loket s druhé strany.
\par 13 A tak zmeril bránu od strechy pokojíka do strechy druhého, širokost petmecítma loket, dvére naproti dverím.
\par 14 A udelal vereje šedesáti loket, a každé vereje síne i brány vukol a vukol jedna míra.
\par 15 Od predku pak brány, kudyž se vchází do predku sínce brány vnitrní, bylo padesáte loket.
\par 16 A byla okna possoužená v pokojících i nad verejemi jejich, do vnitrku brány vukol a vukol, takž i pri klenutí, a na okních vukol a vukol do vnitrku, a na verejích byly palmy.
\par 17 Potom mne uvedl do síne zevnitrní, a aj, komurky a puda udelaná pri síni vukol a vukol. Tridceti komurek na té pude bylo.
\par 18 Ta pak puda po strane tech bran, naproti dlouhosti bran, puda nižší byla.
\par 19 Zmeril také šírku od predku brány dolejší až k predku síne vnitrní, zevnitr na sto loket k východu a pulnoci.
\par 20 Též bránu, kteráž byla na pulnoci pri síni zevnitrní, zmeril na dél i na šír.
\par 21 (Jejíž pokojíkové tri s jedné a tri s druhé strany, i vereje její i sínce její byly podlé míry brány prvnejší.) Padesáti loket dlouhost její, širokost pak petmecítma loket.
\par 22 Okna také její i sínce její i palmy její byly podlé míry brány té, kteráž byla k východu, a vstupovalo se k ní po sedmi stupních, pred nimiž sínce její byly.
\par 23 A byla ta brána u síne vnitrní naproti bráne pulnocní a východní; i odmeril od brány k bráne sto loket.
\par 24 Potom mne vedl ku poledni, a aj, brána ku poledni; i zmeril vereje její, i sínce její podlé tech mer,
\par 25 (A okna v ní i sínce její vukol a vukol podobná oknum onem), padesáti loket zdélí a petmecítma loket zšírí.
\par 26 A k vstupování k ní stupnu sedm, a sínce její pred nimi; též palmy pri ní, jedna s jedné a druhá s druhé strany pri verejích jejích.
\par 27 Též bránu síne vnitrní ku poledni zmeril od brány k bráne, na poledne, sto loket.
\par 28 I uvedl mne do síne vnitrní branou polední, a zmeril tu bránu polední podlé týchž mer,
\par 29 I pokojíky její i vereje její, i sínce její podlé týchž mer, i okna její i sínce její vukol a vukol, padesáti loket na dél, a na šír petmecítma loket.
\par 30 A sínce vukol a vukol zšírí petmecítma loket, a zdélí padesáti loket.
\par 31 A sínce její jako sín zevnitrní, i palmy pri verejích jejích, též stupnu osm k vstupování k ní.
\par 32 Uvedl mne také do síne vnitrní k východu, i zmeril bránu tu podlé týchž mer,
\par 33 Též pokojíky její i vereje její, i sínce její, podlé týchž mer, i okna její i sínce její vukol a vukol, zdélí padesáti loket, zšírí pak petmecítma loket.
\par 34 Též sínce její pri síni zevnitrní, a palmy pri verejích s obou stran, a osm stupnu k vstupování k ní.
\par 35 Privedl mne též k bráne pulnocní, a zmeril ji podlé týchž mer,
\par 36 Pokojíky její, vereje její i sínce její i okna její vukol a vukol, zdélí padesáti loket, a zšírí petmecítma loket.
\par 37 I vereje její pri síni zevnitrní a palmy pri verejích s obou stran, též osm stupnu k vstupování k ní,
\par 38 I komurky a dvére její pri verejích bran, kdež obmývali obeti zápalné.
\par 39 A v sínci brány byli dva štokové s jedné strany, a dva štokové s druhé strany, aby na nich zabíjeli obeti zápalné, a za hrích i vinu.
\par 40 A po boku, kudyž se tam vstupuje pri dverích brány pulnocní, dva štokové; tolikéž po boku druhém sínce též brány dva štokové.
\par 41 Ctyri štokové s jedné, a ctyri štokové s druhé strany po boku brány, osm štoku, na nichž zabíjeli.
\par 42 Ctyri pak štokové k zápalu byli z kamení tesaného, zdélí puldruhého lokte, a zšírí puldruhého lokte, a zvýší lokte jednoho, na nichž nechávali nádobí, kterýmž zabíjeli k zápalum a obetem.
\par 43 A kotliska ztlouští na jednu dlan pripravená v dome vukol a vukol, na štocích pak maso vecí obetovaných.
\par 44 Potom zevnitr, pri bráne vnitrní, komurky zpeváku v síni vnitrní, kteráž byla po boku brány pulnocní, a ty byly na poledne; jedna pri boku brány východní byla na pulnoci.
\par 45 I mluvil ke mne: Tyto komurky na poledne jsou kneží stráž držících nad domem;
\par 46 Ty pak komurky na pulnoci jsou kneží stráž držících nad oltárem, totiž synu Sádochových, kteríž pristupují z synu Léví k Hospodinu, aby sloužili jemu.
\par 47 I zmeril tu sín ctverhranou zdélí sto loket, a zšírí sto loket, i oltár pred domem.
\par 48 I privedl mne k sínci domu, a zmeril vereje té sínce, peti loket s jedné a peti loket s druhé strany, širokost pak brány byla trí loket s jedné a trí loket s druhé strany.
\par 49 Dlouhost sínce dvadcíti loket, širokost pak jedenácti loket i s stupni, po nichž se vstupovalo k ní; sloupové pak byli pri verejích, jeden s jedné, a druhý s druhé strany.

\chapter{41}

\par 1 Opet privedl mne k chrámu, a zmeril vereje, šesti loket zšírí s jedné strany, a šesti loket zšírí s druhé strany, podlé širokosti stánku.
\par 2 A širokost dverí desíti loket, boky pak dverí peti loket s jedné a peti loket s druhé strany. Zmeril také i dlouhost jejich ctyridcíti loket, širokost pak dvadcíti loket.
\par 3 Prišel také do vnitrku, a zmeril vereje dverí dvou loket, a dvére šesti loket, širokost pak dverí sedmi loket.
\par 4 Zmeril také dlouhost svatyne dvadcíti loket, a širokost dvadcíti loket v chráme, a rekl mi: Tato jest svatyne svatých.
\par 5 Zmeril též zed domu šesti loket, a širokost pavlace ctyr loket, vukol a vukol okolo domu.
\par 6 Ty pak pavlace, pavlac nad pavlací, byly tri, a tridcíti noh zdélí, a scházely se pri zdi domu vespolek, tak že se pavlace vukol a vukol držely, a nedržely se na zdi domu.
\par 7 Nebo se rozširovala vukol více a více, svrchu pro pavlace, kteréž byly okolo domu, od vrchu až dolu, vukol a vukol domu, ponevadž nejširší dum byl na hore, a tak nejnižší porozširovala se k vrchu pro prostrední.
\par 8 Tak podobne spatril jsem pri domu pavlace, i nejvyšší vukol a vukol, jejichž pudy zouplna odmereny byly šesti loket k výstupkum.
\par 9 Širokost zdi, kteráž byla pri pavlacích zevnitr, peti loket byla, i plac pavlací, kteréž byly pri domu.
\par 10 Mezi nimiž a komurkami byla širokost dvadcíti loket okolo domu vukol a vukol.
\par 11 A dvére pavlací byly k placu, dvére jedny na pulnoci, a druhé dvére na poledne, a širokost placu byla peti loket vukol a vukol.
\par 12 Stavení pak, kteréž bylo pred príhradkem v úhlu k západu, širokost byla sedmdesáti loket, a zed téhož stavení peti loket zšírí vukol a vukol, a zdélí devadesáti loket.
\par 13 Potom zmeril dum, zdélí sto loket, totiž príhradek i stavení, a zdi jeho zdélí sto loket.
\par 14 Též širokost predku domu i príhradku k východu sto loket.
\par 15 Zmeril i dlouhost stavení pred príhradkem, kteréž bylo za ním, též i paláce jeho s jedné i s druhé strany, a bylo sto loket; též chrám vnitr i sínce s síní,
\par 16 Prahy i okna possoužená, i paláce vukol po trech stranách jejich, naproti prahu taflování drevené vukol a vukol, i od zeme až do oken, též i okna otaflovaná,
\par 17 Od svrchku dverí až do vnitrní i zevnitrní strany domu, i všecku zed vukol a vukol vnitr i zevnitr zmerené.
\par 18 Kteréž taflování bylo udeláno s cherubíny a palmami, a to vše palma mezi cherubínem a cherubínem. A cherubín mel dve tváre.
\par 19 Totiž tvár lidskou naproti palme s jedné strany, a tvár lvícete naproti palme s druhé strany. Tak udeláno bylo ve všem domu vukol a vukol.
\par 20 Od zeme až do vrchu dverí cherubínové a palmy zdelány byly i na zdi chrámu.
\par 21 Chrámu vereje byly ctverhrané, a predek svatyne podobný jemu.
\par 22 Oltár drevený trí loket zvýší, zdélí pak dvou loket s úhly svými, jehož dlouhost i pobocnice jeho drevené byly. I mluvil ke mne: Tento jest stul, kterýž stojí pred Hospodinem.
\par 23 A dvojnásobní dvére byly u chrámu i u svatyne,
\par 24 A dvojnásobní dvére ve vratech, totiž dvojnásobní dvére obracející se, dvojnásobní ve vratech jednech, a dvojnásobní dvére v druhých.
\par 25 Byli pak udeláni na nich, na tech dverích chrámu, cherubínové a palmy, tak jakž udeláni byli na stenách, trámové také drevení byli pred síncí vne.
\par 26 Též na oknech possoužených byly palmy s obou stran po bocích sínce, i na pavlacích domu toho i trámích.

\chapter{42}

\par 1 Potom mne vyvedl k síni zevnitrní, kteráž byla na pulnoci, a privedl mne k tem komurkám, kteréž byly pred príhradkem, a kteréž byly naproti stavení na pulnoci,
\par 2 Jehož délka pri dverích pulnocních na pohledení byla sto loket, a šírka padesát loket.
\par 3 Naproti síni vnitrní, kteráž mela dvadceti loket, a naproti dlážení, kteréž bylo v síni zevnitrní, byl palác naproti paláci trmi rady.
\par 4 Pred komurkami pak byli pláckové desíti loket zšírí vnitr, cesta k nim lokte jednoho, a dvére jejich na pulnoci.
\par 5 Komurky pak nejvyšší byly užší, proto že palácové byli širší než ony, nežli nejzpodnejší a prostrední stavení.
\par 6 Nebo o trojím ponebí bylo, ale nemelo žádných sloupu, jacíž byli sloupové síní; protož užší bylo nežli dolejší i než prostrední od zeme.
\par 7 Ohrady pak, kteráž byla vne naproti komurkám k síni zevnitrní, pred komurkami, dlouhost byla padesáti loket.
\par 8 Nebo dlouhost komurek, kteréž byly v síni zevnitrní, byla padesáti loket, pred chrámem pak sto loket.
\par 9 Pod temi pak komurkami bylo vcházení od východu, skrze kteréž by se vcházelo do nich z síne té zevnitrní.
\par 10 Na šír ohrady té síne k východu pred príhradkem i pred stavením byly komurky.
\par 11 Cesta pak pred nimi byla podobná ceste komurek, kteréž byly na pulnoci. Jakáž byla dlouhost jejich, tak širokost jejich, a všecka vycházení jejich i dvére jejich podobná onemno.
\par 12 A dvére tech komurek, kteréž byly na poledne, podobné byly dverím pri zacátku cesty, cesty pred ohradou prímou k východu, kudyž se vchází do nich.
\par 13 I rekl mi: Komurky na pulnoci a komurky na poledne, kteréž jsou pred príhradkem, jsou komurky svaté, kdežto jídají kneží, kteríž pristupují k Hospodinu, veci nejsvetejší. Tam nechávati budou vecí nejsvetejších a obetí suchých, též za hrích a za vinu; nebo to místo svaté jest.
\par 14 Když tam vejdou kneží, tedy nevyjdou z svatyne do síne zevnitrní, lec tam nechají roucha svého, v nemž prisluhovali; nebo svaté jest. A oblekou roucha jiná, když budou míti pristupovati k tomu, což se dotýce lidu.
\par 15 A dokonav rozmerování domu vnitrního, i vedl mne k bráne, kteráž byla na východ, a zmeril jej vukol a vukol.
\par 16 Zmeril stranu východní prutem, pet set loket prutových tím prutem vukol.
\par 17 Zmeril stranu pulnocní tím prutem, pet set prutu vukol.
\par 18 Zmeril stranu polední tím prutem, pet set prutu.
\par 19 A obrátiv se k strane západní, nameril tím prutem pet set prutu.
\par 20 Na ctyri strany zmeril to, totiž zed vukol a vukol, zdélí peti set, a zšírí peti set, aby delila svaté místo od obecného.

\chapter{43}

\par 1 Potom vedl mne k bráne, kterážto brána patrila k východu.
\par 2 A aj, sláva Boha Izraelského pricházela od východu, jejíž zvuk byl jako zvuk vod mnohých, a zeme svítila se od slávy jeho.
\par 3 A podobné bylo to videní, kteréž jsem videl, práve tomu videní, kteréž jsem byl videl, když jsem šel, abych kazil mesto, videní, pravím, podobná videní onomu, kteréž jsem videl pri rece Chebar. I padl jsem na tvár svou.
\par 4 A když sláva Boží vcházela do domu, cestou brány patrící k východu,
\par 5 Tedy pojal mne Duch, a uvedl mne do síne vnitrní, a aj, dum plný byl slávy Hospodinovy.
\par 6 I slyšel jsem, an mluví ke mne z domu, a muž stál podlé mne.
\par 7 I rekl mi: Synu clovecí, místo stolice mé a místo šlepejí noh mých, kdežto bydliti budu u prostred synu Izraelských na veky, a nebudou poškvrnovati více dum Izraelský jména svatosti mé, oni ani králové jejich smilstvím svým a mrtvými tely králu svých, ani výsostmi svými,
\par 8 Když kladli prah svuj podlé prahu mého, a vereji svou podlé vereje mé, a stenu mezi mnou a mezi sebou, a tak poškvrnovali jména svatosti mé ohavnostmi svými, kteréž páchali, procež jsem je sehltil v hneve svém.
\par 9 Ale nyní vzdálí smilství svá i mrtvá tela králu svých, ode mne, a budu bydliti u prostred nich na veky.
\par 10 Ty synu clovecí, oznam domu Izraelskému o tomto domu, a necht se zahanbí pro nepravosti své, a at zmerí všecko naskrze.
\par 11 A když se hanbiti budou za všecko, což páchali, zpusob domu i formu jeho, i vycházení jeho, též vcházení jeho, i všecky zpusoby jeho, všecka ustanovení jeho, všecky, pravím, zpusoby jeho i všecky zákony jeho v známost jim uved, a napiš pred ocima jejich, at ostríhají všeho zpusobu jeho i všech ustanovení jeho a ciní je.
\par 12 Tento jest zákon toho domu: Na vrchu hory všecko obmezení jeho vukol a vukol, nejsvetejšít jest. Aj, ten jest zákon toho domu.
\par 13 Tyto pak jsou míry oltáre na též lokty, o dlan delší: Predne zpodek lokte zvýší a lokte zšírí, obruba pak jeho pri kraji jeho vukol pídi jedné. Takový jest výstupek oltáre,
\par 14 Totiž od zpodku pri zemi až do prepásaní dolejšího dva lokty, širokost pak lokte jednoho, a od prepásání menšího až do prepásaní vetšího ctyri lokty, širokost též na loket.
\par 15 Ale sám oltár at jest ctyr loket, a z oltáre zhuru ctyri rohové.
\par 16 Oltár pak dvanácti loket zdélí a dvanácti zšírí, ctyrhranatý po ctyrech stranách svých.
\par 17 Prepásaní pak ctrnácti loket zdélí a ctrnácti zšírí po ctyrech stranách jeho, a obruba vukol neho na pul lokte, a zpodek pri nem na loket vukol, a stupnové jeho naproti východu.
\par 18 I rekl ke mne: Synu clovecí, takto praví Panovník Hospodin: Ta jsou ustanovení oltáre v den, v kterýž bude udelán, k obetování na nem zápalu a kropení na nej krví.
\par 19 Nebo dáš knežím Levítským, kteríž jsou z semene Sádochova, kteríž pristupují ke mne, dí Panovník Hospodin, aby mi sloužili, volka mladého za hrích.
\par 20 A nabera krve jeho, dáš na ctyri rohy jeho, i na ctyri úhly toho prepásaní, i na obrubu vukol, a tak jej ocistíš i vycistíš.
\par 21 A vezmeš toho volka za hrích, i spálí jej na míste uloženém v tom dome, vne pred svatyní.
\par 22 V den pak druhý obetovati budeš kozla bez poškvrny za hrích, a ocistí oltár, tak jakž vycistili volkem.
\par 23 A když dokonáš ocištování, obetuj volka mladého bez vady, a skopce z stáda bez poškvrny.
\par 24 Kteréž když obetovati budeš pred Hospodinem, uvrhou kneží na ne soli, a budou je obetovati v zápal Hospodinu.
\par 25 Po sedm dní obetuj kozla za hrích, na každý den; též i volka mladého a skopce z stáda bez poškvrny obetovati budou.
\par 26 Sedm dní ocištovati budou oltár, a vycistí jej, a posvetí ruky své.
\par 27 A když vyplní ty dny, osmého dne i potom obetovati budou kneží na oltári zápaly vaše, a pokojné obeti vaše, i prijmu vás laskave, praví Panovník Hospodin.

\chapter{44}

\par 1 Tedy privedl mne zase cestou k bráne svatyne zevnitrní, kteráž patrí k východu, a ta byla zavrená.
\par 2 I rekl ke mne Hospodin: Brána tato zavrená bude, nebudet otvírána, aniž kdo bude vcházeti skrze ni. Nebo Hospodin Buh Izraelský všel skrze ni; protož budet zavrená.
\par 3 Knížecí jest, kníže samo v ní sedati bude, aby jídalo chléb pred Hospodinem. Cestou síne této brány vejde, a cestou její vyjde.
\par 4 I vedl mne cestou k bráne pulnocní, k prední strane domu, i videl jsem, a aj, naplnila sláva Hospodinova dum Hospodinuv. I padl jsem na tvár svou.
\par 5 I rekl ke mne Hospodin: Synu clovecí, prilož srdce své, a viz ocima svýma, i ušima svýma slyš, cožkoli já mluvím tobe o všech ustanoveních domu Hospodinova, i o všech zákonech jeho. Prilož, pravím, srdce své k vcházení do domu, i ke všemu vycházení z svatyne,
\par 6 A rci zpurnému domu Izraelskému: Takto praví Panovník Hospodin: Dosti mejte na všech ohavnostech svých, ó dome Izraelský,
\par 7 Že jste uvodili cizozemce neobrezaného srdce a neobrezaného tela, aby bývali v svatyni mé, a poškvrnovali i domu mého, když jste obetovali chléb muj, tuk i krev, ješto oni sic rušili smlouvu mou, mimo všecky ohavnosti vaše,
\par 8 A nedrželi jste stráže nad svatými vecmi mými, ale postavili jste strážné na stráži mé, v svatyni mé místo sebe.
\par 9 Takto praví Panovník Hospodin: Žádný cizozemec neobrezaného srdce a neobrezaného tela nevejde do svatyne mé, ze všech cizozemcu, kteríž mezi syny Izraelskými jsou.
\par 10 Nýbrž i Levítové, kteríž se vzdalovali ode mne, když bloudil Izrael, kteríž zbloudili ode mne za ukydanými bohy svými, ponesou nepravost svou.
\par 11 Nebo budou v svatyni mé za služebníky, v povinnostech pri branách domu, a za slouhy pri domu. Oni budou zabíjeti obeti zápalné i obeti lidu, a oni stávati budou pred nimi k sloužení jim,
\par 12 Proto že prisluhovali jim pred ukydanými bohy jejich, a byli domu Izraelskému prícinou pádu v nepravost. Procež prisáhl jsem jim, praví Panovník Hospodin, že ponesou nepravost svou.
\par 13 Aniž pristoupí ke mne, aby mi knežský úrad konali, ovšem aby pristupovati meli k kterým svatým vecem mým, neb k nejsvetejším, ale ponesou pohanení své i ohavnosti své, kteréž páchali.
\par 14 Protož postavím je za strážné u domu, ke vší službe jeho i ke všemu, což cineno býti má v nem.
\par 15 Kneží pak Levítští, synové Sádochovi, kteríž drželi stráž nad svatyní mou, když zbloudili synové Izraelští ode mne, ti budou pristupovati ke mne, aby mi prisluhovali, a státi pred tvárí mou, obetujíce mi tuk i krev, praví Panovník Hospodin.
\par 16 Ti pricházeti budou k svatyni mé, a ti pristupovati k stolu mému, aby mi sloužili a drželi stráž mou.
\par 17 I stane se, když budou míti vcházeti do bran síne vnitrní, že roucha lnená oblekou, aniž na se vezmou vlneného, když by služby konati meli v branách síne vnitrní i u vnitrku.
\par 18 Klobouky lnené míti budou na hlave své, a košilky lnené at mají na bedrách svých, a neprepasují se nicímž, což by pot vyvodilo.
\par 19 A majíce vyjíti do síne zevnitrní, do síne zevnitrní k lidu, svlekou roucha svá, v kterýchž služby konali, a nechají jich v komurkách svatyne, a oblekou roucha jiná, a nebudou posvecovati lidu rouchem svým.
\par 20 Aniž hlavy své holiti budou, ani vlasu nositi, ale slušne ostríhají vlasy své.
\par 21 Vína též nebude píti žádný z kneží, když budou míti vcházeti do síne vnitrní.
\par 22 Vdovy také aneb zahnané nebudou sobe pojímati za manželky, ale panny z semene domu Izraelského, aneb vdovu, kteráž by ovdovela po knezi, pojíti mohou.
\par 23 A lid muj budou rozdílu uciti mezi svatým a nesvatým, též mezi necistým a cistým at je ucí rozeznávati.
\par 24 V rozepri pak at se postavují k souzení, a podlé soudu mého soudí ji. Zákonu mých i ustanovení mých pri všech slavnostech mých at ostríhají, a soboty mé svetí.
\par 25 K mrtvému pak cloveku nepujde, aby se poškvrniti mel, lec pri otci neb pri materi, též pri synu a pri dceri, pri bratru a sestre, kteráž nebyla za mužem, muže se poškvrniti.
\par 26 Po ocištení pak jeho (sedm dní odectou jemu),
\par 27 V ten den, v kterýž vejde do svatyne, do síne vnitrní, aby služby konal v svatyni, obetovati bude za hrích svuj, praví Panovník Hospodin.
\par 28 Dedictví pak jejich toto: Já jsem dedictví jejich, protož vládarství nedávejte jim v Izraeli; já jsem vládarství jejich.
\par 29 Obeti suché a za hrích i vinu, to oni jísti budou, i všelijaká vec, oddána Bohu v Izraeli, jejich bude.
\par 30 I prední veci všech prvotin ze všeho, i každá obet zhuru pozdvižení všeliké veci, ze všech obetí zhuru pozdvižení vašich, knežské bude. I prvotiny testa vašeho dávati budete knezi, aby odpocinulo požehnání v dome tvém.
\par 31 Žádné mrchy a udáveného, ani z ptactva ani z hovad nebudou jídati kneží.

\chapter{45}

\par 1 Když pak ujmete zemi v dedictví, obetovati budete obet Hospodinu, díl svatý té zeme, zdélí petmecítma tisíc loket, zšírí pak deset tisíc, a budet svatý po všem pomezí svém vukol.
\par 2 Z nehož bude místo svaté pet set zdélí, a pet set zšírí, ctyrhrané vukol, a at má padesáte loket prostranství vukol.
\par 3 Z toho pak odmerení odmeríš dýlku petmecítma tisíc loket, a šírku deset tisíc, aby na nem byla svatyne, i svatyne svatých.
\par 4 Díl ten zeme svatý jest. Kneží služebníku pri svatyni býti má, tech, kteríž pristupují, aby prisluhovali Hospodinu, aby meli místo pro domy i místo svaté pro svatyni.
\par 5 Tech pak petmecítma tisíc loket dýlka a deset tisíc šírka at jest také Levítum, služebníkum domu, jim k držení dvadceti komurek.
\par 6 Místo pak k vystavení na nem mesta oddelíte pet tisíc loket na šír, a na dýl petmecítma tisíc, naproti obeti místa svatého; bude pro všecken dum Izraelský.
\par 7 Knížeti pak s obou stran té obeti místa svatého, i položení mesta pred obeti místa svatého, a pred položením mesta od strany západní díl k západu, a po strane východní díl k východu, dlouhost pak naproti každému z tech dílu od pomezí západního ku pomezí východnímu.
\par 8 To bude míti za vládarství v Izraeli, a nebudou více utiskati knížata má lidu mého, ale rozdadí zemi domu Izraelskému po pokoleních jejich.
\par 9 Takto praví Panovník Hospodin: Dostite již, ó knížata Izraelská, nátisk a zhoubu odložte, a soud a spravedlnost konejte, sejmete težké roboty vaše s lidu mého, praví Panovník Hospodin.
\par 10 Váhu spravedlivou a efi spravedlivou i bát spravedlivý míti budete.
\par 11 Efi i bát v jednu míru at jest, aby chomer bral v se deset bátu, efi pak desátý díl chomeru; podlé chomeru at jest míra.
\par 12 Lot dvadceti penez; dvadceti lotu, petmecítma lotu, a patnácte lotu libra bude vám.
\par 13 Tato pak obet pozdvižení bude, kterouž obetovati budete: Šestý díl efi z chomeru pšenice, též šestý díl efi dáte z chomeru jecmene.
\par 14 Narízení pak o oleji (bát jest míra oleje): Desátý díl bátu z míry chomeru, desíti bátu; nebo deset bátu jest chomer.
\par 15 A dobytce jedno ze dvou set bravu z dobrých pastvišt Izraelských, k obeti suché a zápalné a k obetem pokojným, k ocištení vás, praví Panovník Hospodin.
\par 16 Všecken lid té zeme, i s knížetem Izraelským zavázán bude k té obeti zhuru pozdvižení.
\par 17 Nebo kníže povinen bude zápaly, a suché i mokré obeti, v svátky a na novmesíce, i v soboty, na všecky slavnosti domu Izraelského; on obetovati bude za hrích, i obet suchou i zápalnou, i obeti pokojné, aby se ocištení dálo za dum Izraelský.
\par 18 Takto praví Panovník Hospodin: Prvního mesíce, prvního dne vezmeš volka mladého bez poškvrny, kterýmž ocistíš svatyni.
\par 19 I nabére knez krve té obeti za hrích, a pomaže verejí domu, a ctyr rohu toho prepásání na oltári, i vereje brány síne vnitrní.
\par 20 Takž také uciní sedmého dne téhož mesíce, za každého pobloudilého i za hloupého. Tak ocistíte dum.
\par 21 Prvního mesíce, ctrnáctého dne, budete míti Fáze, svátek sedmi dnu, chlebové presní jísti se budou.
\par 22 A bude obetovati kníže v ten den za sebe i za všecken lid té zeme volka za hrích.
\par 23 A po sedm dní svátku obetovati bude zápal Hospodinu, sedm volku a sedm skopcu bez poškvrny na den, po tech sedm dní, a za hrích kozla na den.
\par 24 A obet suchou, efi na volka a efi na skopce, pripraví též oleje hin na efi.
\par 25 Sedmého mesíce, patnáctého dne, v svátek tolikéž obetovati bude po sedm dní, jakož za hrích, tak zápal, tak obet suchou i olej.

\chapter{46}

\par 1 Takto praví Panovník Hospodin: Brána síne vnitrní, kteráž patrí k východu, bude zavrená po šest dní všedních, v den pak sobotní otevrína bude; též v den novmesíce otvírána bude.
\par 2 I prijde kníže cestou síne brány zevnitr, a postaví se u vereje té brány, a budou obetovati kneží obet zápalnou jeho, i obeti pokojné jeho, a poklone se na prahu brány, potom vyjde. Brána pak nebude zavírána do vecera,
\par 3 Aby se klanel lid zeme té u dverí brány té ve dny sobotní, i na novmesíce pred Hospodinem.
\par 4 Obet pak zápalná, kterouž obetovati má kníže Hospodinu v den sobotní, šest beránku bez vady a skopec bez poškvrny,
\par 5 A obet suchá, efi na skopce, i na beránky obet suchá, podlé toho, jakž nadeleno, a oleje hin na efi.
\par 6 Ke dni pak novmesíce at jest volek mladý bez poškvrny, a šest beránku i skopec bez poškvrny.
\par 7 Též at obetuje efi obeti suché pri volku, a efi pri skopci i pri beráncích, sec bude moci býti, a oleje hin na efi.
\par 8 Kníže pak vcházeje, cestou sínce též brány pujde, a cestou její odejde.
\par 9 Ale když vcházeti bude lid zeme té pred Hospodina na slavnosti, ten kdož vejde cestou brány pulnocní, aby se klanel, vyjde cestou brány polední; a ten kdož vejde cestou brány polední, vyjde cestou brány pulnocní. Nenavrátí se cestou té brány, kterouž všel, ale naproti ní vyjde.
\par 10 A když oni vcházeti budou, kníže mezi nimi vcházeti bude, a když odcházeti budou, odejde.
\par 11 Též na svátky i na slavnosti at jest suchá obet, efi na volka a efi na skopce, a na beránky, což nadeleno, a oleje hin na efi.
\par 12 Bude-li pak obetovati kníže obet dobrovolnou, zápal aneb obeti pokojné dobrovolne Hospodinu, tedy at jest mu otevrína brána, kteráž k východu patrí, a at obetuje zápal svuj aneb pokojné obeti své, tak jakž obetuje v den sobotní. Potom odejde, a brána bude zavrína po odchodu jeho.
\par 13 K tomu beránka rocního bez poškvrny obetovati bude v zápal každý den Hospodinu; každého jitra beránka obetovati bude.
\par 14 Též suchou obet priciní k nemu, každého jitra šestý díl efi, též oleje tretinu hin k skropení mouky belné, suchou obet Hospodinu, narízením vecným ustavicne.
\par 15 A tak budou obetovati beránka i obet suchou, i olej každého jitra, zápal ustavicný.
\par 16 Taktot praví Panovník Hospodin: Dá-li kníže dar nekomu z synu svých, dedictvít jeho jest, synu jeho bud, k vládarství jejich dedicnému.
\par 17 Jestliže pak dá dar z dedictví svého nekterému z služebníku svých, také bude jeho až do léta svobodného, kdyžto navrátí se knížeti tomu; však dedictví jeho budou míti synové jeho.
\par 18 Aniž bude bráti kníže z dedictví lidu, z vládarství jejich je vytiskuje; z svého vládarství dedictví dá synum svým, aby nebyl rozptylován lid muj žádný z vládarství svého.
\par 19 Potom vedl mne pruchodem, kterýž jest po strane brány, k knežím do komurek svatých, kteréž patrily na pulnoci, a aj, tu bylo místo po dvou bocích k západu.
\par 20 I rekl mi: Toto jest místo, kdež varí kneží obeti za vinu a za hrích, kdež smaží obeti suché, aby nevynášeli do síne zevnitrní ku posvecování lidu.
\par 21 Vyvedl mne též do síne zevnitrní, a vodil mne po ctyrech koutech síne, a aj, sín byla v každém rohu té síne.
\par 22 Ve ctyrech úhlech té síne byly síne s komíny, ctyridcíti loket zdélí a trídcíti zšírí; míra jednostejná tech ctyr síní nárožních.
\par 23 A v tech ctyrech byly kuchynky vukol, též ohnište zdelána v tech kuchynkách vukol.
\par 24 I rekl mi: Ta jsou místa tech, kteríž varí, kdežto varí služebníci domu obeti lidu.

\chapter{47}

\par 1 Potom privedl mne zase ke dverím domu, a aj, vody vycházely od spodku prahu domu na východ; nebo prední strana domu k východu byla, a vody scházely pozpodu po pravé strane domu, po strane polední oltáre.
\par 2 Odtud mne vyvedl cestou brány pulnocní, a obvedl mne cestou zevnitrní k bráne zevnitrní, cestou, kteráž patrí k východu, a aj, vody vyplývaly po pravé strane.
\par 3 Když pak vycházel ten muž k východu, v jehož rukou míra, i nameril tisíc loket, a provedl mne skrze vodu,vodu do kutku.
\par 4 Potom nameriv tisíc, provedl mne skrze vodu, vodu do kolenou; ješte nameriv tisíc, provedl mne vodou do pasu.
\par 5 Opet když nameril tisíc, byl potok, kteréhož jsem nemohl prebrísti; nebo vyzdvihly se vody, vody, pres než by se musilo plynouti, potok, kterýž by nemohl prebreden býti.
\par 6 Tedy rekl mi: Videl-lis, synu clovecí? I vedl mne, a posadil mne na breh toho potoka.
\par 7 Když jsem se pak obrátil, aj, na brehu toho potoka bylo dríví velmi veliké po obou stranách.
\par 8 I rekl mi: Vody tyto vycházejí od Galilee první, a sstupujíce po rovine, vejdou do more, a když do more vpadnou, opraví se vody.
\par 9 I stane se, že každý živocich, kterýž se plazí, všudy, kamžkoli prijdou potokové, ožive, a bude ryb velmi mnoho, proto že když prijdou tam tyto vody, ocerstvejí, a živy budou všudy, kdežkoli dojde tento potok.
\par 10 Stane se i to, že se postaví podlé neho rybári od Engadi až do studnice Eglaim; budou tu rozstírány síti. Podlé rozlicnosti své bude ryb jejich, jako ryb more velikého, velmi mnoho.
\par 11 Bahna a louže jeho, kteréž se neopraví, soli oddány budou.
\par 12 Pri potoku pak poroste na brehu jeho po obou stranách všelijaké dríví ovoce nesoucí, jehož list neprší, aniž ovoce jeho prestává, v mesících svých nese prvotiny; nebo vody jeho z svatyne vycházejí, protož ovoce jeho jest ku pokrmu, a lístí jeho k lékarství.
\par 13 Takto praví Panovník Hospodin: Totot jest obmezení, v nemž sobe dedicne privlastníte zemi po pokoleních Izraelských dvanácti; Jozefovi dva provazcové.
\par 14 Dedicne, pravím, jí vládnouti budete, jeden rovne jako druhý, o níž prisáhl jsem, že ji dám otcum vašim. I pripadne vám zeme tato v dedictví.
\par 15 Toto jest tedy pomezí té zeme: K strane pulnocní od more velikého cestou Chetlonu, kudyž se vchází do Sedad,
\par 16 Emat, Berota, Sibraim, kteríž jsou mezi pomezím Damašským, a mezi pomezím Emat, vsi prostrední, kteréž jsou pri pomezí Chavrón.
\par 17 A tak bude pomezí od more Azar Enon, pomezí Damašek, a pulnocní strana na pulnoci, a pomezí Emat. A to jest strana pulnocní.
\par 18 Strana pak východní mezi Chavrón a mezi Damaškem, a mezi Galád, a mezi zemí Izraelskou pri Jordánu; od toho pomezí pri mori východním meriti budete. A tot jest strana východní.
\par 19 Strana pak polední na poledne, od Támar až k vodám sváru v Kádes, od potoka až k mori velikému. A to jest strana polední na poledne.
\par 20 Strana pak západní more veliké, od pomezí až naproti, kudyž se vchází do Emat. Ta jest strana západní.
\par 21 A tak rozdelíte zemi tuto sobe po pokoleních Izraelských.
\par 22 I stane se, že když ji rozmeríte, bude vám v dedictví i príchozím, kteríž by pohostinu byli mezi vámi, kteríž by zplodili syny mezi vámi; nebo budou vám jako tu zrodilí mezi syny Izraelskými, s vámit ujmou dedictví mezi pokoleními Izraelskými.
\par 23 Protož necht jest v tom pokolení príchozí, u nehož pohostinu jest. Tu dedictví dáte jemu, praví Panovník Hospodin.

\chapter{48}

\par 1 Tato jsou pak jména pokolení: V koncinách na pulnocní stranu podlé cesty Chetlon, kudyž se vchází do Emat, Azar Enan, ku pomezí Damašskému na pulnocní stranu, podlé Emat, od východní strany až do západní, osadí se pokolení jedno, totiž Dan,
\par 2 A pri pomezí Dan, od strany východní až k strane západní jedno, totiž Asser,
\par 3 A pri pomezí Asser, od strany východní až do strany západní jedno, totiž Neftalím,
\par 4 A pri pomezí Neftalím, od strany východní až do strany západní jedno, totiž Manasses,
\par 5 A pri pomezí Manasses, od strany východní až do strany západní jedno, totiž Efraim,
\par 6 A pri pomezí Efraim, od strany východní až k strane západní jedno, totiž Ruben,
\par 7 A pri pomezí Ruben, od strany východní až k strane západní jedno, totiž Juda.
\par 8 A pri pomezí Juda, od strany východní až k strane západní bude obet, kterouž obetovati budete, petmecítma tisíc loket zšírí, zdélí pak zaroven s jedním z jiných dílu, od strany východní až k strane západní, a bude svatyne u prostred neho.
\par 9 Ta obet, kterouž obetovati máte Hospodinu, bude zdélí petmecítma tisíc loket, zšírí pak deset tisíc.
\par 10 Temto pak se dostane ta obet svatá,totiž knežím, na pulnoci petmecítma tisíc loket, k západu pak zšírí desíti tisíc, a na východ zšírí desíti tisíc, na poledne též zdélí petmecítma tisíc, a bude svatyne Hospodinova u prostred neho,
\par 11 Knežím, posvecenému každému z synu Sádochových, kteríž drží stráž mou, kteríž nebloudili, když bloudili synové Izraelští, jako bloudili Levítové.
\par 12 I bude díl jejich obetovaný z obeti té zeme, vec nejsvetejší pri pomezí Levítu.
\par 13 Levítu pak díl bude naproti pomezí knežskému, petmecítma tisíc loket zdélí, a zšírí deset tisíc; každá dlouhost petmecítma tisíc, a širokost deset tisíc.
\par 14 A nebudou ho uprodávati, ani smenovati, ani prenášeti prvotin zeme, proto že jest posvecená Hospodinu.
\par 15 Pet pak tisíc loket pozustalých na šír, proti tem petmecítma tisícum, bude místo obecné, pro mesto k bydlení a k predmestí, i bude mesto u prostred neho.
\par 16 Tyto pak jsou míry jeho: Strana pulnocní na ctyri tisíce a pet set loket, též strana polední na ctyri tisíce a pet set, od strany též východní ctyri tisíce a pet set, takž strana západní na ctyri tisíce a pet set.
\par 17 Bude i predmestí pri meste k pulnoci na dve ste a padesáte loket, a ku poledni na dve ste a padesáte, takž na východ na dve ste a padesáte, též k západu na dve ste a padesáte.
\par 18 Ostatek pak na dél, naproti obeti svaté, deset tisíc loket k východu, a deset tisíc k západu; a z toho, což bude naproti té obeti svaté, budou míti duchody ku pokrmu služebníci mesta.
\par 19 A ti služebníci mesta sloužiti budou Izraelovi ze všech pokolení Izraelských.
\par 20 Všecku tuto obet, petmecítma tisíc loket, podlé tech petmecítma tisíc, ctverhranou obetovati budete v obet svatou k vládarství mestu.
\par 21 Což pak pozustane, knížeti, s obou stran obeti svaté a vládarství mesta, pred temi petmecítma tisíci loket obeti, až ku pomezí východnímu, a od západu proti týmž petmecítma tisíc loket, podlé pomezí západního naproti tem dílum, knížeti bude. A to bude obet svatá, a svatyne domu u prostred neho.
\par 22 Od vládarství pak Levítu a od vládarství mesta, u prostred toho, což jest knížecího, mezi pomezím Judovým a mezi pomezím Beniaminovým, knížecí bude.
\par 23 Ostatní pak pokolení, od strany východní až k strane západní, osadí se pokolení jedno, totiž Beniamin.
\par 24 A pri pomezí Beniamin, od strany východní až k strane západní jedno, totiž Simeon,
\par 25 A pri pomezí Simeon, od strany východní až k strane západní jedno, totiž Izachar,
\par 26 A pri pomezí Izachar, od strany východní až k strane západní jedno, totiž Zabulon,
\par 27 A pri pomezí Zabulon, od strany východní až k strane západní jedno, totiž Gád.
\par 28 A pri pomezí Gád, k strane polední na poledne, tu bude pomezí od Támar až k vodám sváru v Kádes, ku potoku pri mori velikém.
\par 29 Tot jest ta zeme, kterouž ujmete hned od potoka, po pokoleních Izraelských, a ti dílové jejich, praví Panovník Hospodin.
\par 30 Tato pak jsou vymezení mesta: Od strany pulnocní ctyr tisíc a pet set loket míra.
\par 31 Brány pak mesta podlé jmen pokolení Izraelských, brány tri na pulnoci, brána Rubenova jedna, brána Judova jedna, brána Léví jedna.
\par 32 A od strany východní ctyr tisíc a pet set, a brány tri, totiž brána Jozefova jedna, brána Beniaminova jedna, brána Danova jedna.
\par 33 Též od strany polední ctyr tisíc a pet set loket míra, a brány tri, brána Simeonova jedna, brána Izacharova jedna, brána Zabulonova jedna.
\par 34 Od strany západní ctyr tisíc a pet set, brány jejich tri, brána Gádova jedna, brána Asserova jedna, brána Neftalímova jedna.
\par 35 Okolek osmnácti tisíc loket, jméno pak mesta od dnešního dne bude: Hospodin tam prebývá.

\end{document}