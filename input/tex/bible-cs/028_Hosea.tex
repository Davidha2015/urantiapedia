\begin{document}

\title{Hosea}

\chapter{1}

\par 1 Slovo Hospodinovo, kteréž se stalo k Ozeášovi synu Bérovu za dnu Uziáše, Jotama, Achasa, Ezechiáše, králu Judských, a za dnu Jeroboáma syna Joasova, krále Izraelského.
\par 2 Když Hospodin zacal mluviti k Ozeášovi, rekl Hospodin Ozeášovi: Jdi, pojmi sobe ženu smilnou, a deti z smilstva; nebo nestydate smilneci tato zeme, odvrátila se od Hospodina.
\par 3 A tak šel a pojal Gomeru, dceru Diblaimskou, kterážto pocala a porodila jemu syna.
\par 4 Tedy rekl jemu Hospodin: Nazoviž jméno jeho Jezreel; nebo po malém casu já vyhledávati budu krve Jezreel na domu Jéhu, a prestati káži království domu toho.
\par 5 I stane se v ten den, že polámi lucište Izraelovo v údolí Jezreel.
\par 6 Opet pocala znovu a porodila dceru. I rekl jemu: Nazov jméno její Lorucháma; nebo již více neslituji se nad domem Izraelským, abych jim co prominouti mel.
\par 7 Ale nad domem Judským se slituji, a vysvobodím je skrze Hospodina Boha jejich; nebo nevysvobodím jich lucištem a mecem, ani bojem, konmi neb jezdci.
\par 8 Potom ostavivši Loruchámu, opet pocala a porodila syna.
\par 9 I rekl: Nazov jméno jeho Loammi; nebo vy nejste lid muj, a já také nebudu váš.
\par 10 A však bude pocet synu Izraelských jako písku morského, kterýž ani zmeren, ani secten býti nemuže. A stane se, že místo toho, kdež receno jim bylo: Nejste vy lid muj, receno jim bude: Synové Boha silného a živého jste.
\par 11 I budou shromáždeni synové Judští a synové Izraelští spolu, a ustanovíce nad sebou hlavu jednu, vyjdou z této zeme, ackoli veliký bude den Jezreel.

\chapter{2}

\par 1 Rcete bratrím vašim: Ó lide muj, a sestrám vašim: Ó milosrdenství došlá.
\par 2 Odpor vedte proti matce vaší, dokažte, že ona není manželka má, a že já nejsem muž její, lec odvaruje smilství svých od tvári své, a cizoložství svých z prostred prsí svých,
\par 3 Abych jí nesvlékl do naha, a nepostavil jí tak, jakž byla v den narození svého, a ucine ji podobnou poušti, a obráte ji jako v zemi vyprahlou, umoril bych ji žízní.
\par 4 Neslitoval bych se ani nad syny jejími, proto že jsou synové z smilstva.
\par 5 Nebo smilní matka jejich, hanebnost páše rodicka jejich; ríká zajisté: Pujdu za frejíri svými, kteríž mi dodávají chleba mého, vody mé, vlny mé, lnu mého, oleje mého i nápoju mých.
\par 6 A protož aj, já opletu cestu její trním, a ohradím hradbou, aby stezek svých nalezti nemohla.
\par 7 Tehdy behati bude za frejíri svými, a však nedostihne jich, hledati jich bude, ale nenalezne. I dí: Ej nu, již se navrátím k manželu svému prvnímu, proto že mi lépe tehdáž bylo než nyní.
\par 8 Nebo ona nezná toho, že jsem já dával jí obilé, a mest a olej, anobrž rozmnožoval stríbro i zlato, kteréž vynakládají na Bále.
\par 9 Protož poberu zase obilé své v cas jeho, i mest svuj v jistý cas jeho, a odejmu jí vlnu svou i len svuj k priodívání nahoty její,
\par 10 A tak v brzce odkryji mrzkost její pred ocima frejíru jejích, a žádný jí nevytrhne z ruky mé.
\par 11 A uciním prítrž vší radosti její, svátkum jejím, novomesícum jejím i sobotám jejím, a všechnem slavnostem jejím.
\par 12 Pohubím také révoví její a fíkoví její, proto že ríká: Ty veci jsou mzda má, kterouž mi dali frejíri moji; a obrátím je v les, a sžerou je živocichové polní.
\par 13 A budu na ní vyhledávati dnu Bálu,v nichž jim kadí, a ozdobeci se náušnicemi svými a záponami svými, chodí za frejíri svými, na mne se pak zapomíná, praví Hospodin.
\par 14 Protož aj, já namluvím ji, když ji uvedu na poušt; nebo mluviti budu k srdci jejímu.
\par 15 A dám jí vinice její od téhož místa, i údolé Achor místo dverí nadeje, i bude tam zpívati jako za dnu mladosti své, totiž jako tehdáž, když vycházela z zeme Egyptské.
\par 16 I stane se v ten den, dí Hospodin, že volati budeš: Muži muj, a nebudeš mne volati více: Báli muj.
\par 17 Nebo vyprázdním jména Bálu z úst tvých, aniž pripomínáni budou více v jménu svém.
\par 18 A uciním pro tebe smlouvu v ten den s živocichy polními, a s ptactvem nebeským i s zemeplazy, lucište pak a mec polámi, i válku odejmu z zeme, a zpusobím to, aby bydleli bezpecne.
\par 19 I zasnoubím te sobe na vecnost, zasnoubím te sobe, pravím, v spravedlnosti a v soudu a v dobrotivosti a v hojném milosrdenství.
\par 20 Zasnoubím te sobe také u víre, abys poznala Hospodina.
\par 21 I stane se v ten den, že vyslýchati budu, dí Hospodin, vyslýchati budu nebesa, a ona vyslyší zemi.
\par 22 Zeme pak vyslyší obilé, i mest, i olej, a ty veci vyslyší Jezreele.
\par 23 Nebo ji rozseji sobe na zemi, a smiluji se nad Loruchámou, Loammi pak reknu: Lid muj jsi ty, a on dí: Bože muj.

\chapter{3}

\par 1 Opet rekl mi Hospodin: Ješte jdi, a zamiluj ženu, milou frejíri a cizoložnou, tak jako miluje Hospodin syny Izraelské, ackoli oni hledí k bohum cizím, a milují káde vína.
\par 2 Tedy zjednal jsem ji sobe z patnácti stríbrných a z puldruhého chomeru jecmene.
\par 3 A rekl jsem jí: Za mnoho dnu sed mi, nesmilni, aniž se vdávej za muže, a já také prícinou tvou.
\par 4 Nebo za mnohé dny budou synové Izraelští bez krále, bez knížete, bez obeti, bez modly, bez efodu a terafim.
\par 5 Potom pak obrátí se synové Izraelští, a hledati budou Hospodina Boha svého i Davida krále svého; a predešeni jsouce, pobehnou k Hospodinu a k dobrote jeho v posledních casích.

\chapter{4}

\par 1 Slyšte slovo Hospodinovo, ó synové Izraelští, nebot má rozepri Hospodin s obyvateli zeme této, proto že není žádné vernosti, ani žádného milosrdenství, ani žádné známosti Boží v této zemi.
\par 2 Proklínání a lži a vraždy, a zlodejství i cizoložství na vrch zrostlo, a vražda vraždu postihá.
\par 3 Protož kvíliti bude tato zeme, a umdlí všecko, což v ní prebývá, živocichové polní i ptactvo nebeské, ano i ryby morské zhynou.
\par 4 A však žádný jim nedomlouvej, aniž jich kdo tresci; nebo lid tvuj podobni jsou tem, kteríž se vadí s knezem.
\par 5 Protož ve dne padneš, padne také i prorok s tebou v noci, zahladím i matku tvou.
\par 6 Vyhlazen bude lid muj pro neumení. Ponevadž jsi ty pohrdl umením, i tebou pohrdnu, abys mi knežství nekonal; a že jsi zapomnel na zákon Boha svého, já také zapomenu se na syny tvé.
\par 7 Cím se více rozmohli, tím více hrešili proti mne; slávu jejich v pohanení smením.
\par 8 Obeti za hrích lidu mého jedí, protož k nepravosti jejich duše své pozdvihují.
\par 9 Procež stane se jakž lidu tak knezi. Nebo vyhledávati budu na nem cest jeho, a skutky jeho jemu vrátím.
\par 10 I budou jísti, a však se nenasytí, smilniti budou, ale nerozmnoží se; nebo nechtejí pozoru míti na Hospodina.
\par 11 Smilství a víno a mest odjímá srdce.
\par 12 Lid muj dreva svého se dotazuje, a hul jeho oznamuje jemu; nebo je duch smilství v blud uvodí, aby smilnili, odcházejíce od Boha svého.
\par 13 Na vrších hor obetují, a na pahrbcích kadí, pod doubím a topolím a jilmovím, nebo jest príhodný stín jejich; protož smilní dcery vaše, a nevesty vaše cizoloží.
\par 14 Nevyhledával-liž bych na dcerách vašich, že smilní, a na nevestách vašich, že cizoloží, že tito s nevestkami se oddelují, a s ženkami obetují? Anobrž lid, kterýž sobe nesrozumívá, padne.
\par 15 Jestliže smilníš ty Izraeli, nechažt nehreší Juda. Protož nechodtež do Galgala, aniž vstupujte do Betaven, aniž prisahejte: Živt jest Hospodin.
\par 16 Nebo jako jalovice tvrdošijná tvrdošijný jest Izrael, jižt je pásti bude Hospodin jako beránka na prostranne.
\par 17 Efraim stovaryšil se s modlami, nechej ho.
\par 18 Zpurné je ciní nápoj jejich, velice smilní, milují: Dejte. Ochráncové jeho jsou ohyzda.
\par 19 Zachvátí je vítr krídly svými, i budou zahanbeni pro své obeti.

\chapter{5}

\par 1 Slyštež to, ó kneží, a pozorujte, dome Izraelský, i dome královský, poslouchejte, nebo proti vám soud tento jest, proto že jste osídlo v Masfa, a sítka rozestrená na vrchu Tábor.
\par 2 Nýbrž k zabíjení uchylujíce se, pripadají k zemi, ale já ztresci každého z nich.
\par 3 Známt já Efraima, a Izrael není ukryt prede mnou; nebo nyní smilníš, Efraime, poškvrnuje se Izrael.
\par 4 Nemají se k tomu, aby se obrátili k Bohu svému, proto že duch smilství mezi nimi jest, Hospodina pak znáti nechtejí,
\par 5 Tak že hrdost Izraelova svedcí vuci proti nemu; protož Izrael i Efraim padnou pro nepravost svou, padne také i Juda s nimi.
\par 6 S stády bravu a skotu svých pujdou hledati Hospodina, však nenaleznou; vzdálilte se od nich.
\par 7 Hospodinu se zproneverili, nebo syny cizí zplodili; jižt je zžíre mesíc i s jmením jejich.
\par 8 Trubte trubou v Gabaa, a na pozoun v Ráma; kricte v Betaven: Po tobe, ó Beniamine.
\par 9 Efraim zpušten bude v den kázne, v nemž po pokoleních Izraelských uvedu v známost pravdu.
\par 10 Knížata Judská jsou podobná tem, kteríž prenášejí mezník; vyleji na ne jako vodu prchlivost svou.
\par 11 Utišten jest Efraim, potrín soudem, však sobe libuje choditi za rozkazem.
\par 12 Protož i já byl jsem jako mol Efraimovi, a jako hnis domu Judovu.
\par 13 Procež vida Efraim neduh svuj, a Juda nežit svuj, utekl se Efraim k Assurovi, a poslal k králi, kterýž by se o nej zasadil. Ale on nebude moci zhojiti vás, ani uzdraviti vás od nežitu.
\par 14 Nebo já jsem jako lítý lev Efraimovi, a jako lvíce domu Judovu; já, já uchvátím a ujdu, vezmu, a žádný nevytrhne.
\par 15 Odejda, navrátím se na místo své, až se vinni dadí, a hledati budou tvári mé.

\chapter{6}

\par 1 V úzkosti své ráno hledati mne budou: Podte, a navratme se k Hospodinu; nebo on uchvátil a zhojí nás, ubil a uvíže rány naše.
\par 2 Obživí nás po dvou dnech, dne tretího vzkrísí nás, a budeme živi pred oblícejem jeho,
\par 3 Tak abychom znajíce Hospodina, více poznávati se snažovali; nebo jako jitrní svitání jest vycházení jeho, a prijde nám jako déšt jarní a podzimní na zemi.
\par 4 Což mám ciniti s tebou, ó Efraime? Což mám ciniti s tebou, ó Judo, ano vaše dobrota jest jako oblak ranní, a jako rosa jitrní pomíjející?
\par 5 Protož otesával jsem skrze proroky, zbil jsem je recmi úst svých, aby soudu tvých svetlo vzešlo.
\par 6 Nebo milosrdenství oblibuji a ne obet, a známost Boha více než zápaly.
\par 7 Ale oni smlouvu mou jako lidskou prestoupili, a tu se mi zproneverili.
\par 8 Galád mesto cinitelu nepravosti, plné šlepejí krvavých.
\par 9 Rota pak knežstva jsou jako lotri, kteríž na nekoho cekají na ceste, kudyž se jde do Sichem; nebo zúmyslnou nešlechetnost páší.
\par 10 V dome Izraelském vidím hroznou vec: Tam smilstvím Efraimovým poškvrnuje se Izrael.
\par 11 Ano i u tebe, ó Judo, vsadil rouby, když jsem já zase vedl zajatý lid svuj.

\chapter{7}

\par 1 Když lécím Izraele, tedy zjevuje se nepravost Efraimova a zlosti Samarské; nebo provodí faleš. Vnitr zlodejství, a vne provozují loupežnictví.
\par 2 Aniž na to pomýšlejí v srdci svém, že na všecku nešlechetnost jejich pamatují; již je obklicují skutkové jejich, a pred mým oblícejem jsou.
\par 3 Nešlechetností svou obveselují krále, a klamy svými knížata.
\par 4 Všickni naporád cizoloží, podobni jsouce peci zanícené od pekare, kterýž prestává bdíti, jen ažby zadelané testo zkynulo.
\par 5 V den krále našeho k nemoci jej privodí knížata láhvicí vína; vztahuje ruku svou s posmevaci.
\par 6 Nebo priložili k úkladum svým srdce své podobné peci; celou noc spí pekar jejich, v jitre horí jako plamen ohne.
\par 7 Všickni naporád rozpáleni jsou jako pec, a zžírají soudce své; všickni králové jejich padají, aniž kdo z nich volá ke mne.
\par 8 Efraim s národy smísil se, Efraim bude chléb podpopelný neobrácený.
\par 9 Cizozemci zžírají sílu jeho, ackoli on toho nezná; i šedinami prokvítaje, však vždy toho nezná.
\par 10 A ackoli pýcha Izraelova svedcí vuci proti nemu, však se nenavracují k Hospodinu Bohu svému, aniž ho hledají s tím se vším.
\par 11 A Efraim jest jako holubice hloupá bez srdce; k Egyptskému králi volají, k Assyrskému se utíkají.
\par 12 Když odejdou, roztáhnu na ne sítku svou, a jako ptactvo nebeské pritrhnu je; kárati je budu tak, jakž slýcháno bylo o tom v shromáždení jejich.
\par 13 Beda jim, že jsou pobehli mne. Zpuštení na ne, proto že se mi zproneverili, ješto jsem já je vykoupil, ale oni mluvili proti mne lži.
\par 14 Aniž volají ke mne z srdce svého, když kvílí na ložcích svých, a když pro obilé a mest shromaždujíce se, obracejí se ke mne,
\par 15 Ješto já potrestav, posiloval jsem ramen jejich, ale oni proti mne zlé vymýšlejí.
\par 16 Navracujít se, ale ne k Nejvyššímu, jsou jako lucište omylné, padají od mece knížata jejich, od rozhnevání jazyka jejich, což jim ku posmechu jest v zemi Egyptské.

\chapter{8}

\par 1 Pricine k ústum svým troubu, rci: Aj, letí jako orlice na dum Hospodinuv, proto že prestoupili smlouvu mou, a proti zákonu mému prevrácene cinili.
\par 2 Budout sic volati: Bože muj, znát tebe Izrael,
\par 3 Ale opustilt jest Izrael dobré, neprítel jej stihati bude.
\par 4 Onit ustanovují krále, ale beze mne; knížata vyzdvihují, k nimž já se neznám; z stríbra i z zlata svého ciní sobe modly k svému zkažení.
\par 5 Opustít je tele tvé, ó Samarí, když se zažhne prchlivost má na ne. Až dokudž nebudou moci ostríhati nevinnosti?
\par 6 Však i ono jest od Izraele, remeslník je udelal, a nenít Bohem; nebo drtiny budou z toho telete Samarského.
\par 7 Ponevadž vetru rozsívají, takét vichrici žíti budou; ani stébla žádného míti nebudou; úroda nevydá mouky, a by pak i vydala, cizozemci to sehltí.
\par 8 Sehlcen bude Izrael, tudíž budou mezi pohany jako nádoba, v níž není žádné líbosti.
\par 9 Proto že se oni utíkají k Assurovi, oslu divokému, kterýž jest toliko sám svuj, a že Efraim sobe najímá milovníky,
\par 10 A že posílali dary mezi pohany: protož tudíž je zberu i já, anobrž jižt jsou okusili neceho pro bríme krále knížat.
\par 11 Nebo vzdelal Efraim mnoho oltáru k hrešení, mát oltáre k hrešení.
\par 12 Vypsal jsem jemu znamenité veci v zákone svém, ale neváží sobe rovne jako cizí veci.
\par 13 Z obetí daru mých obetují maso a jedí, Hospodin neoblibuje jich. Jižt zpomene na nepravost jejich, a vyhledávati bude hríchy jejich, že se oni do Egypta navracují,
\par 14 A že se zapomenul Izrael na Ucinitele svého, a nastavel chrámu, a Juda vzdelal mnoho mest hrazených. Protož pošli ohen na mesta jeho, a zžíre paláce jeho.

\chapter{9}

\par 1 Neraduj se, Izraeli, s plésáním jako jiní národové, že smilníš, odcházeje od Boha svého, a miluješ mzdu po všech obilnicích.
\par 2 Obilnice ani pres nebude jich pásti, a mest pochybí jim.
\par 3 Aniž budou bydliti v zemi Hospodinove, ale navrátí se Efraim do Egypta, a v Assyrii veci necisté jísti budou.
\par 4 Nebudou obetovati Hospodinu vína, aniž príjemné jemu bude. Obeti jejich budou jako chléb kvílících, z nehož kdož by koli jedli, poškvrnili by se, protože chléb jejich pro mrtvé jejich nemá pricházeti do domu Hospodinova.
\par 5 Co pak ciniti budete v den slavnosti, a v den svátku Hospodinova?
\par 6 Nebo aj, zahynou skrze poplénení, Egypt zbére je, a pohrbí je Memfis; nejrozkošnejšími schranami stríbra jejich kopriva dedicne vládnouti bude, a bodlácí v domích jejich.
\par 7 Prijdou dnové navštívení, prijdou dnové odplacení, poznají Izraelští, že ten prorok jest blázen šílený a clovek nicemný, pro množství nepravosti tvé a velikou nenávist tvou.
\par 8 Prorok, kterýž stráž drží nad Efraimem spolu s Bohem mým, jest osídlem ciharským na všech cestách svých, nenávist jest v dome Boha jeho.
\par 9 Hlubokot jsou zabredli a porušili se, tak jako za dnu Gabaa; zpomenet na nepravost jejich, a vyhledávati bude hríchy jejich.
\par 10 Jako hrozny na poušti nalezl jsem byl Izraele, jako ranní fíky v prvotinách jejich popatril jsem na otce vaše; oni odešli za Belfegor, a oddali se té ohavnosti, protož budout ohavní, tak jakž se jim líbilo.
\par 11 Efraim jako pták zaletí, i sláva jejich od narození a od života, nýbrž od pocetí.
\par 12 A byt pak i odchovali syny své, však je zbavím veku zmužilého; nýbrž i jim beda, když já se od nich odvrátím.
\par 13 Efraim, jakýž vidím Týr, vštípen jest v obydlí, a však Efraim vyvede k mordéri syny své.
\par 14 Dej jim, Hospodine; co bys dal? Dej jim život neplodný a prsy vyschlé.
\par 15 Vrch zlosti jejich jest v Galgala, protož i tam jich nenávidím. Pro zlost skutku jejich vyženu je z domu svého, aniž jich více budu milovati; všecka knížata jejich jsou zpurná.
\par 16 Bit bude Efraim, koren jejich uschne, ovoce neprinesou, a byt pak zplodili, tedy zmorím nejmilejší života jejich.
\par 17 Pohrdne jimi Buh muj, nebo nechtí ho poslouchati, i budou tuláci mezi pohany.

\chapter{10}

\par 1 Izrael jest vinný kmen prázdný, ovoce skládá sobe. Cím více mívá ovoce svého, tím více rozmnožuje oltáre, a cím lepší jest zeme jeho, tím více vzdelává obrazy.
\par 2 Klade díly srdce jejich, procež vinni jsou. Ont poborí oltáre jejich, popléní obrazy jejich,
\par 3 Ponevadž i ríkají: Nemáme žádného krále, nýbrž aniž se bojíme Hospodina, a král co by nám ucinil?
\par 4 Mluví slova, klnouce se lžive, když ciní smlouvu, a soud podobný jedu roste na záhonech polí mých.
\par 5 Z príciny jalovic Betavenských desiti se budou obyvatelé Samarští, když kvíliti bude nad nimi lid jejich i kneží jejich, (kteríž prícinou jejich nyní pléší), proto že sláva jejich zastehuje se od nich.
\par 6 Ano i sám lid do Assyrie zaveden bude v dar králi, kterýž obhájce býti mel; Efraim hanbu ponese, a Izrael stydeti se bude za své predsevzetí.
\par 7 Vyhlazen bude král Samarský jako pena na svrchku vody.
\par 8 Vypléneny budou také výsosti Avenu, hrích Izraelských, trní a hloží zroste na oltárích jejich. I dejí horám: Prikrejte nás, a pahrbkum: Padnete na nás.
\par 9 Ode dnu Gabaa hrešil jsi, Izraeli. Tamt jsou ostáli, nepostihla jich v Gabaa válka proti nešlechetným.
\par 10 A protož podlé líbosti své svíži je; nebo sberou se na ne národové k svázání jich, pro dvojí nepravost jejich.
\par 11 Nebo Efraim jest jalovicka, kteráž byla vyucována; mlatbu miluje, ackoli jsem já nastupoval na tucný krk její, abych k jízde užíval Efraima, Juda aby oral, a Jákob vlácil. A ríkal jsem:
\par 12 Rozsívejte sobe k spravedlnosti, žnete k milosrdenství, orte sobe ouhor, ponevadž cas jest k hledání Hospodina, ažby prišel a dštil vám spravedlností.
\par 13 Ale orali jste bezbožnost, žali jste nepravost, jedli jste ovoce lži; nebo doufáš v cestu svou a ve množství reku svých.
\par 14 Protož povstane rozbroj mezi lidem tvým, i každá pevnost tvá zpuštena bude, tak jako zpustil Salman Bet Arbel v den boje; matky s syny rozrážíny budou.
\par 15 Aj, tot vám zpusobí Bethel pro prílišnou nešlechetnost vaši; na svitání docela vyhlazen bude král Izraelský.

\chapter{11}

\par 1 Když dítetem byl Izrael, miloval jsem jej, a z Egypta povolal jsem syna svého.
\par 2 Volali jich, oni tím více ucházeli pred nimi, Bálum obetovali, a rytinám kadili,
\par 3 Ješto jsem já na nohy stavel Efraima, on pak bral je na lokty své; aniž znáti chteli, že jsem já je uzdravoval.
\par 4 Potahoval jsem jich provázky lidskými, provazy milování, a cinil jsem jim tak jako ti, kteríž pozdvihují jha na celistech hovádka, podávaje potravy jemu.
\par 5 Nenavrátít se do zeme Egyptské, ale Assur bude králem jeho, proto že se nechteli obrátiti.
\par 6 Nadto bude trvati mec v mestech jeho, a zkazí závory jeho, a sžíre je pro rady jejich.
\par 7 Nebo lid muj ustrnul na odvrácení se ode mne, a ac ho k Nejvyššímu volají, však žádný ho neoslavuje.
\par 8 Jakž bych te vydal, ó Efraime? Jakž bych te vydal, ó Izraeli? Kterak bych te položil jako Adamu, podvrátil jako Seboim? Zkormouceno jest ve mne srdce mé, ano i streva slitování mých pohnula se.
\par 9 Nevykonámt prchlivosti hnevu svého, nezkazím více Efraima; nebo jsem já Buh silný, a ne clovek, u prostred tebe svatý, aniž pritáhnu na mesto.
\par 10 I pujdou za Hospodinem rvoucím jako lev; on zajisté rváti bude, tak že s strachem pribehnou synové od more.
\par 11 S strachem pobehnou jako ptactvo z Egypta, a jako holubice z zeme Assyrské, i osadím je v domích jejich, dí Hospodin.
\par 12 Obklícili mne Efraimští lží, a dum Izraelský lstí, když ješte Juda panoval s Bohem silným, a s svatými verný byl.

\chapter{12}

\par 1 Efraim se vetrem pase, a vítr východní honí, každého dne lež a zhoubu množí; nebo smlouvu s Assurem ciní, a masti do Egypta donášejí se.
\par 2 I s Judou má soud Hospodin; procež navštíve Jákoba podlé cest jeho, podlé snažností jeho odplatí jemu.
\par 3 V živote za patu držel bratra svého, a silou svou knížetsky se potýkal s Bohem.
\par 4 Knížetsky, pravím, potýkal se s andelem, a premohl; plakal a pokorne ho prosil; v Bethel jej nalezl, a tam s námi mluvil.
\par 5 Tot jest Hospodin Buh zástupu, pametné jeho jest Hospodin.
\par 6 Protož ty k Bohu svému se obrat, milosrdenství a soudu ostríhej, a ocekávej na Boha svého ustavicne.
\par 7 Kramárem jest, v jehož ruce jsou vážky falešné, rád utiskuje,
\par 8 A ríká Efraim: Však jsem zbohatl, dobyl jsem sobe zboží; ve všech mých pracech nenajdou mi nepravosti, jenž by hríchem byla.
\par 9 Já pak Hospodin jsa Bohem tvým od vyjití z zeme Egyptské, ješte-liž bych te sedeti nechal v stáncích jako ve dnech svátecních?
\par 10 A mluve skrze proroky, já abych videní mnoho ukazoval, a skrze proroky podobenství predkládal?
\par 11 Zdali toliko v Gálád byla nepravost a marnost? I v Galgala voly obetují, pres to i oltáru jejich jest jako kopcu na záhonech polí mých.
\par 12 Tamto utekl byl Jákob z krajiny Syrské, kdež sloužil Izrael pro ženu, a pro ženu byl pastýrem;
\par 13 Sem pak skrze proroka privedl Hospodin Izraele z Egypta, též skrze proroka ostríhán jest.
\par 14 Vzbudilt Efraim hnev prehorký, a protož krev jeho na nej se vztáhne, a potupu svou vrátí jemu Pán jeho.

\chapter{13}

\par 1 Když mluvíval Efraim, býval strach; vznešený byl v Izraeli, ale prohrešiv pri Bálovi, tožt umrel.
\par 2 A i nyní ješte predce hreší. Nebo delají sobe a slévají z stríbra svého podlé rozumu svého strašidla, ješto všecko to není než dílo remeslníku, o cemž ríkají: Lidé, kteríž obetují, telata at líbají.
\par 3 Protož budou jako oblak ranní a jako rosa jitrní, kteráž odchází, jako plevy vichricí zachvácené z humna, a jako dým z komínu,
\par 4 Ješto já jsem Hospodin Buh tvuj od vyjití z zeme Egyptské, a Boha krome mne nepoznal jsi, aniž jest jiný vysvoboditel krome mne.
\par 5 Ját jsem te poznal na poušti v zemi velmi vyprahlé.
\par 6 Dobré pastvy mevše, nasyceni jsou, ale když se nasytili, pozdvihlo se srdce jejich, protož zapomenuli na mne.
\par 7 Procež budu jim jako lítý lev, jako pardus vedlé cesty cíhati budu.
\par 8 Potkám se s nimi jako nedved osirelý, a roztrhám všecko srdce jejich, a sežeru je tam jako lev, jako zver divoká roztrhující je.
\par 9 Z tebet jest zhouba tvá, ó Izraeli, ješto ve mne všecka pomoc tvá.
\par 10 Kdež jest král tvuj? Kdež jest? Necht te zachová ve všech mestech tvých. Aneb soudcové tvoji, o nichž jsi rekl: Dej mi krále a knížata?
\par 11 Dal jsem tobe krále v hneve svém, a odjal jsem v prchlivosti své.
\par 12 Svázánat jest nepravost Efraimova, schován jest hrích jeho.
\par 13 Bolesti rodicky prijdou na nej; jest syn nemoudrý, sic jinak nezustával by tak dlouho v živote matky.
\par 14 Z ruky hrobu vyplatím je, od smrti vykoupím je; budu zhoubcím tvým, ó smrti, budu zkažením tvým, ó hrobe, želení skryto bude od ocí mých.
\par 15 Nebo on mezi bratrími ovoce ponese, ac prv prijde vítr východní, vítr Hospodinuv od poušte vstupující, a vysuší studnice jeho, osuší i vrchovište jeho; onenno rozchvátá poklad všech nejdražších klénotu.
\par 16 Zpuštena bude Samarí, proto že se protivila Bohu svému; od mece padnou, dítky jejich zrozrážíny budou, a tehotné ženy jejich zroztínány.

\chapter{14}

\par 1 Obratiž se, ó Izraeli, cele k Hospodinu Bohu svému, nebo jsi padl prícinou nepravosti své.
\par 2 Vezmete s sebou slova, a obratte se k Hospodinu, a rcete jemu: Sejmi všelikou nepravost, a dej to, což dobrého jest, a budemet se odplaceti volky rtu našich.
\par 3 Assurt nemuže zachovati nás, na koních nepojedeme, aniž díme více dílu rukou našich: Buh náš; nebo v tobe smilování nalézá sirotek.
\par 4 Uzdravím odvrácení jejich, budu je milovati dobrovolne; nebo odvrácen bude hnev muj od nich.
\par 5 Budu jako rosa Izraelovi, zkvetne jako lilium, a hluboce vpustí koreny své jako Libán.
\par 6 Rozloží se ratoléstky jeho, a bude jako oliva okrasa jeho, a vune jeho jako Libánská.
\par 7 Ti, kteríž by sedeli pod stínem jeho, navrátí se, oživou jako obilé, a puciti se budou jako kmen vinný, jehož památka bude jako vína Libánského.
\par 8 Efraime, což jest mi již do modl? Já vyslýchati, a patriti budu na te; já jsem jako jedle zelenající se, ze mnet ovoce tvé jest.
\par 9 Kdo jest moudrý, porozumej temto vecem, a rozumný poznej je; nebo prímé jsou cesty Hospodinovy, a spravedliví choditi budou po nich, prestupníci pak na nich padnou.

\end{document}