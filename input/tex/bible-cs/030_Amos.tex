\begin{document}

\title{Ámos}

\chapter{1}

\par 1 Slova Amosova, (kterýž byl mezi pastýri) z Tekoa, kteráž videl o Izraelovi za dnu Uziáše krále Judského, a za dnu Jeroboáma syna Joasova, krále Izraelského, dve léte pred zeme tresením.
\par 2 I rekl: Hospodin rváti bude z Siona, a z Jeruzaléma vydá hlas svuj, i budou kvíliti salášové pastýru, a vyschnou pole nejvýbornejší.
\par 3 Takto praví Hospodin: Pro troji nešlechetnost Damašku, ovšem pro ctveru neodpustím jemu, proto že mlátili Galáda cepami okovanými.
\par 4 Ale pošli ohen na dum Hazaeluv, kterýžto zžíre paláce Benadadovy.
\par 5 I polámi závoru Damašku, a vypléním obyvatele z údolí Aven, a toho, kterýž drží berlu, z domu Eden, i pujde v zajetí lid Syrský do Kir, dí Hospodin.
\par 6 Takto praví Hospodin: Pro troji nešlechetnost Gázy, ovšem pro ctveru neslituji se nad ním, proto že je zajímajíce, v zajetí vecné podrobovali Idumejským.
\par 7 Ale pošli ohen na zed Gázy, kterýžto zžíre paláce její,
\par 8 A vypléním obyvatele z Azotu, i toho, kterýž drží berlu, z Aškalon, a obrátím ruku svou proti Akaron, i zahyne ostatek Filistinských, praví Panovník Hospodin.
\par 9 Takto praví Hospodin: Pro troji nešlechetnost Týru, ovšem pro ctveru neodpustím jemu, proto že je v zajetí vecné podrobili Idumejským, a nepamatovali na smlouvu bratrskou.
\par 10 Ale pošli ohen na zed Tyrskou, kterýžto zžíre paláce jeho.
\par 11 Takto praví Hospodin: Pro troji nešlechetnost Edoma, ovšem pro ctveru neslituji se nad ním, proto že udusiv v sobe všecku lítostivost, stihá mecem bratra svého, a hnev jeho ustavicne rozsapává, anobrž vzteklost jeho špehuje bez prestání.
\par 12 Ale pošli ohen na Teman, kterýžto zžíre paláce v Bozra.
\par 13 Toto praví Hospodin: Pro troji nešlechetnost synu Ammon, ovšem pro ctveru neodpustím jemu, proto že roztínali tehotné Galádské, jen aby rozširovali pomezí své.
\par 14 Ale zanítím ohen na zdi Rabba, kterýžto zžíre paláce její, s troubením v den boje a s bourí v den vichrice.
\par 15 I pujde král jejich v zajetí, on i knížata jeho s ním, praví Hospodin.

\chapter{2}

\par 1 Takto praví Hospodin: Pro troji nešlechetnost Moábovu, ovšem pro ctveru neodpustím jemu, proto že spálil kosti krále Idumejského na vápno.
\par 2 Ale pošli ohen na Moába, kterýžto zžíre paláce Kariot, i umre s hlukem Moáb, s krikem a s hlasem trouby.
\par 3 A vypléním soudce jeho, i všecka knížata jeho zmorduji s ním, praví Hospodin.
\par 4 Takto praví Hospodin: Pro troji nešlechetnost Judovu, ovšem pro ctveru neslituji se nad ním, proto že oni pohrdají zákonem Hospodinovým, a ustanovení jeho neostríhají, a svodí se lžmi svými, jichž následovali otcové jejich.
\par 5 Ale pošli ohen na Judu, kterýžto zžíre paláce Jeruzalémské.
\par 6 Takto praví Hospodin: Pro troji nešlechetnost Izraelovu, ovšem pro ctveru neodpustím jemu, proto že prodávají spravedlivého za peníze, a nuzného za pár strevícu.
\par 7 Kteríž dychtí, aby chudé s prstí smísili, a cestu tichých prevracejí; nadto syn i otec vcházejí k jedné a též mladici, aby poškvrnili jména svatosti mé.
\par 8 A na odevu zastaveném klanejí se pri každém oltári, a víno pokutovaných pijí v dome bohu svých,
\par 9 Ješto jsem já vyhladil Amorejského od tvári jejich, jehož vysokost byla jako vysokost cedru. Ackoli pevne stál jako dub, však jsem zkazil svrchu ovoce jeho, pospodu pak koreny jeho.
\par 10 A vás já jsem vyvedl z zeme Egyptské, a vyvedl jsem vás po poušti ctyridceti let, abyste dedicne vládli zemí Amorejského.
\par 11 A vzbuzoval jsem z synu vašich proroky, a z mládencu vašich Nazarejské. Zdaliž není tak, ó synové Izraelští? praví Hospodin.
\par 12 Ale vy jste napájeli Nazarejské vínem, a prorokum jste zapovídali, rkouce: Neprorokujte.
\par 13 Aj, já tlaciti budu zemi vaši tak, jako vuz težký tlací snopy.
\par 14 I zahyne utíkání od rychlého, silný též neužive síly své, a udatný nevysvobodí života svého.
\par 15 A ten, kterýž se chápá lucište, neostojí, a cerstvý na nohy své neutece, a ten, kterýž jezdí na koni, nevysvobodí života svého.
\par 16 Ale i zmužilého srdce mezi nejudatnejšími nahý utíkati bude v ten den, praví Hospodin.

\chapter{3}

\par 1 Slyšte slovo to, kteréž mluví Hospodin proti vám, synové Izraelští, proti vší té rodine, kterouž jsem vyvedl z zeme Egyptské, rka:
\par 2 Toliko vás samy poznal jsem ze všeliké rodiny zeme, protož trestati vás budu ze všech nepravostí vašich.
\par 3 Zdaliž pujdou dva spolu, lec by se snesli?
\par 4 Zdaliž zarve lev v lese, když by nebylo žádné loupeže? Vydá-liž lvícek hlas svuj z peleše své, kdyby lapiti nemel?
\par 5 Padne-liž ptáce do osídla na zem, když by žádné lécky nebylo? Bude-liž zdviženo osídlo z zeme, když by nic neuvázlo?
\par 6 Zdaliž když se troubí trubou v meste, lid s strachem se nezbíhá? Zdaž když se má státi v meste co zlého, Hospodin toho známa neciní?
\par 7 Necinít zajisté Panovník Hospodin niceho, lec by zjevil tajemství své služebníkum svým prorokum.
\par 8 Lev rve, kdož by se nebál? Panovník Hospodin velí, kdož by neprorokoval?
\par 9 Rozhlaste po palácích v Azotu, a po palácích v zemi Egyptské, a rcete: Sberte se na hory Samarí, a vizte znepokojení veliká u prostred neho a nátisk trpící v nem,
\par 10 A že neumejí delati upríme, dí Hospodin. Poklady skládají z nátisku a loupeže na palácích svých.
\par 11 Protož takto praví Panovník Hospodin: Aj, neprítel, a to na zemi tuto vukol, a tent odejme od tebe sílu tvou, i budou rozchvátáni palácové tvoji.
\par 12 Takto praví Hospodin: Jako když vytrhne pastýr z úst lva dva hnáty aneb kus ucha, tak vytrženi budou synové Izraelští, sedící v Samarí lhostejne na postelích, a na ložcích rozkošných.
\par 13 Slyšte a osvedcte v dome Jákobove, dí Panovník Hospodin, Buh zástupu,
\par 14 Že v ten den, když Izraele trestati budu pro prestoupení jeho, navštívím také oltáre v Bethel, a odtati budou rohové oltáre, tak že spadnou na zem.
\par 15 A uderím domem zimním o dum letní, i zahynou domové z kostí slonových, a konec vezmou domové velicí, praví Hospodin.

\chapter{4}

\par 1 Slyšte slovo toto, ó krávy Bázanské, kteréž jste na horách Samarských, kteréž nátisk ciníte chudým, a potíráte nuzné, kteréž ríkáte pánum jejich: Prineste, at pijeme.
\par 2 Prisáhl Panovník Hospodin skrze svatost svou, že aj, dnové jdou na vás, v nichž vezme vás na háky, a potomky vaše na udice rybárské.
\par 3 I vyjdete mezerami, každá tak, jakž stojí, a budete rozhazovati, což na palácích, dí Hospodin.
\par 4 Jdetež do Bethel, a budtež pobehlci Galgala, rozmnožujte prevrácenost, a prinášejte každého jitra obeti své, tretího roku desátky své.
\par 5 A pálíce obet chvály z kvašených vecí, provolejte i obeti dobrovolné, a rozhlaste, ponevadž se vám tak líbí, ó synové Izraelští, dí Panovník Hospodin.
\par 6 A ackoli já dal jsem vám cistotu zubu po všech mestech vašich, totiž nedostatek chleba na všech místech vašich, a však jste se ke mne neobrátili, dí Hospodin.
\par 7 Já také zadržel jsem vám déšt, když ješte tri mesícové byli do žne, a dštil jsem na jedno mesto, a na druhé mesto jsem nedštil; jeden díl deštem svlažen byl, a ten díl, na kterýž nepršelo, uschl.
\par 8 A toulali se dve i tri mesta k jednomu mestu, aby se napili vody, aniž se napiti mohli, a však jste se neobrátili ke mne, dí Hospodin.
\par 9 Bil jsem vás suchem a rzí; hojnost, kterouž prinášely zahrady vaše a vinice vaše, i fíkoví vaše, i olivoví vaše, pojedly housenky, a však jste se neobrátili ke mne, dí Hospodin.
\par 10 Poslal jsem na vás mor tak jako na Egypt, zbil jsem mecem mládence vaše, v zajetí jsem vydal kone vaše, a ucinil jsem, že vstupoval smrad vojsk vašich i v chrípe vaše, a však jste se neobrátili ke mne, dí Hospodin.
\par 11 Podvrátil jsem vás, jako podvrátil Buh Sodomu a Gomoru, tak že jste byli jako hlavne vychvácená z ohne, vždy však neobrátili jste se ke mne, dí Hospodin.
\par 12 A protož takt uciním, ó Izraeli, a ponevadž takt uciniti míním, pripraviž se vstríc Bohu svému, ó Izraeli.
\par 13 Nebo aj, on jest sformovatel hor a stvoritel vetru, a oznamuje cloveku, jaké by bylo jeho myšlení; ciní z záre jitrní tmu, a šlapá po vysokostech zeme. Hospodin Buh zástupu jest jméno jeho.

\chapter{5}

\par 1 Slyšte slovo toto, kteréž já vynáším proti vám, totiž naríkání nad domem Izraelským.
\par 2 Padnet, aniž více povstane panna Izraelská; opuštena bude v zemi své, nebude žádného, kdo by jí pozdvihl.
\par 3 Nebo takto praví Panovník Hospodin: V meste, z kteréhož vycházelo tisíc, zustane sto, v tom pak, z kteréhož vycházelo sto, zustane deset domu Izraelskému.
\par 4 Nebo takto praví Hospodin domu Izraelskému: Hledejte mne a živi budte.
\par 5 A nehledejte Bethel, aniž chodte do Galgala, a do Bersabé se nesmýkejte; nebo Galgal jistotne se prestehuje, a Bethel prijde na nic.
\par 6 Hledejte Hospodina, (a živi budte, aby nepronikl jako ohen domu Jozefova, a nesehltil, a nebylo by žádného, kdo by uhasil Bethelské,
\par 7 Kteríž promenují v pelynek soud, a spravedlnosti na zemi zanechávají),
\par 8 Toho, kterýž ucinil Kurátka i Oriona, kterýž promenuje stín smrti v jitro, a den v temnosti nocní, kterýž privolává vody morské, a vylévá je na svrchek zeme, jehož jméno jest Hospodin,
\par 9 Kterýž ocerstvuje zemdleného proti silnému, tak že zemdlený do pevnosti vchází.
\par 10 Nenávidí trescícího v bráne, a toho, kdož mluví veci pravé, v ohavnosti mají.
\par 11 A protož, proto že loupíte chudého, a bríme obilé bérete od neho, domu z tesaného kamení nastaveli jste, ale nebudete bydliti v nich; vinic výborných naštepovali jste, ale nebudete píti vína z nich.
\par 12 Nebo já vím o mnohých nešlechetnostech vašich a velikých hríších vašich, že trápíte spravedlivého, berouce poctu, a nuzných pri v bráne prevracíte.
\par 13 Protož rozumný v ten cas mlceti musí, nebo cas ten zlý jest.
\par 14 Hledejte dobrého a ne zlého, abyste živi byli, a budet tak Hospodin Buh zástupu s vámi, jakž pravíte.
\par 15 Mejte v nenávisti zlé, a milujte dobré, a ustanovte v bráne soud; snad Hospodin Buh zástupu milost uciní ostatkum Jozefovým.
\par 16 Protož takto praví Panovník Hospodin, Buh zástupu: Po všech ulicích bude kvílení, a na všecky strany zkriknou: Ouvech, ouvech, a povolají oráce k pláci a kvílení s temi, kteríž umejí naríkati.
\par 17 Nýbrž i po všech vinicích bude kvílení, když projdu prostredkem tebe, dí Hospodin.
\par 18 Beda tem, kteríž žádají dne Hospodinova. K cemuž jest vám ten den Hospodinuv, ponevadž jest tmy a ne svetla?
\par 19 Jako když by nekdo utíkal pred lvem, potkal by se s ním nedved; aneb když by všel do domu, a zpolehna rukou svou na stenu, uštkl by ho had.
\par 20 Zdali není tmy a ne svetla den Hospodinuv, v nemž není blesku, ale mrákota?
\par 21 Nenávidím, zavrhl jsem svátky vaše, aniž sobe chutnám slavností vašich.
\par 22 Nebo budete-li mi obetovati zápaly a suché obeti vaše, neoblíbím jich, a na pokojné obeti krmného dobytka vašeho nepopatrím.
\par 23 Odejmi ode mne hluk písní svých, ani hudby louten vašich nechci poslouchati.
\par 24 Ale povalí se jako voda soud, a spravedlnost jako potok silný.
\par 25 Zdali jste mne obeti a dary obetovali na poušti za ctyridceti let, dome Izraelský?
\par 26 Nýbrž nosili jste stánek Melecha vašeho a Kijuna, obrazy vaše, hvezdu boha vašeho, kteréžto veci sami jste sobe zdelali.
\par 27 Protož prestehuji vás dále než Damašské, praví Hospodin, jehož jméno jest Buh zástupu.

\chapter{6}

\par 1 Beda pokoj majícím na Sionu, a doufajícím v horu Samarskou, kteréžto hory jsou slovoutné mimo jiné u tech národu, k nimž se scházejí dum Izraelský.
\par 2 Projdete až do Chalne, a odtud jdete do Emat veliké, a sstupte do Gát Filistinských, a shlédnete, jsou-li která království lepší nežli tato? Jest-li vetší jejich kraj nežli kraj váš?
\par 3 Kteríž smyslíte, že jest daleko den bídy, a pristavujete stolici nátisku.
\par 4 Kteríž léhají na ložcích slonových, a rozkošne sobe pocínají na postelích svých, kteríž jídají berany z stáda a telata nejtucnejší,
\par 5 Prizpevujíce k loutne jako David, vymýšlejíce sobe nástroje muzické.
\par 6 Kteríž pijí z bání vinných, a drahými mastmi se maží, aniž jsou citelni bolesti pro potrení Jozefovo.
\par 7 Procež jižt pujdou v zajetí v prvním houfu stehujících se, a tak nastane žalost tem, kteríž sobe rozkošne pocínají.
\par 8 Prisáhl Panovník Hospodin skrze sebe samého, dí Hospodin Buh zástupu: V ohavnosti mám pýchu Jákobovu, i palácu jeho nenávidím, procež vydám mesto i vše, což v nem jest.
\par 9 I stane se, že pozustane-li deset osob v dome jednom, i ti zemrou.
\par 10 A vezme nekoho strýc jeho, aneb ten, kterýž mu strojí pohreb, aby vynesl kosti z domu, a dí tomu, kdož jest v pokoji domu: Jest-liž ješte s tebou kdo? I dí: Není žádného. I rekne: Mlc, proto že nepripomínali jména Hospodinova.
\par 11 Nebo aj, Hospodin prikáže, a bíti bude na dum veliký prívaly, a na dum menší rozsedlinami.
\par 12 Zdaliž koni bežeti mohou po skále? Zdaliž ji kdo orati muže voly? Nebo jste promenili soud v jed, a ovoce spravedlnosti v pelynek,
\par 13 Vy, kteríž se veselíte, ano není z ceho, ríkajíce: Zdaliž jsme svou silou nevzali sobe rohu?
\par 14 Ale aj, já vzbudím proti vám, ó dome Izraelský, praví Hospodin Buh zástupu, národ, kterýž vás ssouží, odtud, kudy se jde do Emat, až do potoka roviny.

\chapter{7}

\par 1 Toto mi ukázal Panovník Hospodin, že aj, formoval kobylky, když nejprvé pocala rusti otava, když aj, otava byla po královském posecení.
\par 2 I stalo se, když snedly byliny zemské, že jsem rekl: Panovníce Hospodine, odpustiž, prosím. Kdož zustane Jákobovi? Nebot ho malicko jest.
\par 3 I želel Hospodin toho. Nestanet se, rekl Hospodin.
\par 4 Tedy ukázal mi Panovník Hospodin, a aj, Panovník Hospodin volal, že pri svou povede ohnem. A spáliv propast velikou, spálil i díl.
\par 5 Já pak rekl jsem: Panovníce Hospodine, prestaniž, prosím. Kdož zustane Jákobovi? Nebot ho malicko jest.
\par 6 I želel Hospodin toho. A ani toho se nestane, rekl Panovník Hospodin.
\par 7 Potom ukázal mi, a aj, Pán stál na zdi podlé pravidla vzdelané, v jehož ruce bylo pravidlo.
\par 8 I rekl mi Hospodin: Co vidíš, Amose? Rekl jsem: Pravidlo. I rekl Pán: Aj, já položím pravidlo u prostred lidu svého Izraelského, nebudut již více promíjeti jemu.
\par 9 Nebo zpušteny budou výsosti Izákovy, a svatyne Izraelovy zpustnou, tehdáž, když povstanu proti domu Jeroboámovu s mecem.
\par 10 Tedy poslal Amaziáš knez Bethelský k Jeroboámovi králi Izraelskému, rka: Spuntoval se proti tobe Amos u prostred domu Izraelského, ta zeme nemohla by snésti všech slov jeho.
\par 11 Nebo takto praví Amos: Od mece umre Jeroboám, a Izrael jistotne prestehován bude z zeme své.
\par 12 Potom rekl Amaziáš Amosovi: Ó vidoucí, ujdi, radímt, utec do zeme Judské, a jez tam chléb, a tam prorokuj.
\par 13 Ale v Bethel již více neprorokuj, nebo ono svatyne králova i dum královský jest.
\par 14 Tedy odpovídaje Amos, rekl Amaziášovi: Nebylt jsem já prorokem, ano ani synem prorockým, ale byl jsem skotákem, a cesával jsem plané fíky.
\par 15 Ale Hospodin mne vzal, když jsem chodil za stádem, a rekl mi Hospodin: Jdi, prorokuj lidu mému Izraelskému.
\par 16 Nyní tedy slyšiž slovo Hospodinovo. Ty pravíš: Neprorokuj v Izraeli, a nekaž v dome Izákove.
\par 17 Protož takto praví Hospodin: Žena tvá cizoložiti bude v meste, synové pak tvoji i dcery tvé od mece padnou; a zeme tvá provazcem delena bude, a ty v zemi necisté umreš, Izrael pak jistotne prestehován bude z zeme své.

\chapter{8}

\par 1 Také mi ukázal Panovník Hospodin, a aj, byl koš ovoce letního.
\par 2 A rekl: Co ty vidíš, Amose? I rekl jsem: Koš ovoce letního. Opet mi rekl Hospodin: Prišelte konec lidu mému Izraelskému, nebudut již více promíjeti jemu.
\par 3 Procež kvíliti budou zpevové chrámoví v ten den, praví Panovník Hospodin. Množství mrtvých, mlce, namece na všelijaké místo.
\par 4 Slyštež to vy, kteríž sehlcujete chudého, abyste vyhladili nuzné z zeme,
\par 5 Ríkajíce: Skoro-liž pomine novmesíce, abychom prodávali obilé, a sobota, abychom otevreli obilnice, abychom ujímali efi, a privetšovali váhy, a faleš provodili vážkami falešnými,
\par 6 Kupujíce za peníze nuzné, a chudého za pár strevícu, nadto abychom plevy obilné prodávali?
\par 7 Prisáhl Hospodin skrze dustojnost Jákobovu: Žet se nezapomenu na veky na všecky skutky jejich.
\par 8 Nad tím-liž by se netrásla i zeme, a nekvílil by každý, kdož prebývá na ní? Proto-liž by nemela vystoupiti všecka jako potok, a zachvácena i zatopena býti jako potokem Egyptským?
\par 9 Anobrž stane se v ten den, praví Panovník Hospodin, uciním, že slunce zajde o poledni, a uvedu tmy na zemi v jasný den.
\par 10 A promením svátky vaše v kvílení, a všecky zpevy vaše v naríkání, a zpusobím to, že bude na každých bedrách žíne, a na každé hlave lysina, a bude v zemi této kvílení jako nad jednorozeným, a poslední veci její jako den horkosti.
\par 11 Aj, dnové jdou, dí Panovník Hospodin, že pošli hlad na zemi, ne hlad chleba, ani žízen vody, ale slyšení slov Hospodinových,
\par 12 Tak že toulati se budou od more až k mori, a od pulnoci až na východ behati, hledajíce slova Hospodinova, však nenajdou.
\par 13 V ten cas umdlévati budou panny krásné, ano i mládenci tou žízní,
\par 14 Kteríž prisahají skrze ohavnost Samarskou, a ríkají: Živt jest Buh tvuj, ó Dan, a živa jest cesta Bersabé. I padnou, a nepovstanou více.

\chapter{9}

\par 1 Videl jsem Pána, an se postavil na oltári a rekl: Uder v makovici, až se zatresou ty vereje, a rozetni je všecky od vrchu jejich, ostatek pak mecem zmorduji. Neutecet žádný z nich, aniž kdo z nich bude, ješto by toho znikl.
\par 2 Byt se pak do zeme zakopali, i odtud by je ruka má vzala; pakli by vstoupili do nebe, i odtud bych je strhl.
\par 3 A jestliže by se schovali na vrchu Karmele, vyhledám a vezmu je odtud; pakli by se skryli pred ocima mýma na dne more, prikáži hadu, aby je i odtud vyhryzl.
\par 4 Paklit by šli v zajetí pred neprátely svými, i tam prikáži meci, aby je zmordoval; obrátím zajisté oko své proti nim k zlému, a ne k dobrému.
\par 5 Nebo Panovník Hospodin zástupu když se jen dotkne zeme, rozplývá se, a kvílí všickni prebývající na ní, a vystupuje všecka jako potok, a zatopena bývá jako potokem Egyptským,
\par 6 Kterýž vzdelal na nebesích paláce své, a zástup svuj na zemi sšikoval, kterýž muže zavolati vody morské, a vyliti ji na svrchek zeme, jehož jméno jest Hospodin.
\par 7 Zdaliž nejste podobni synum Mourenínu prede mnou, ó synové Izraelovi? dí Hospodin. Zdaliž jsem Izraele nevyvedl z zeme Egyptské, tak jako Filistinské z Kaftor, a Syrské z Kir?
\par 8 Aj, oci Panovníka Hospodina proti království tomuto hrešícímu, abych je vyhladil se svrchku zeme, a však nevyhladím docela domu Jákobova, dí Hospodin.
\par 9 Nebo aj, já prikázal jsem, a budu zmítati domem Izraelským mezi všemi národy, tak jako zmítáno bývá na rícici, tak že ani kamenícko nepropadne na zem.
\par 10 Mecem zbiti budou všickni hríšníci z lidu mého, kteríž ríkají: Nepriblížít se, aniž potká nás to zlé.
\par 11 V ten den zdvihnu stánek Daviduv, kterýž klesá, a zahradím mezery jeho, a zboreniny jeho opravím, a vzdelám jej jako za dnu starodávních,
\par 12 Aby dedili s ostatky Idumejských a se všechnemi národy, nad kterýmiž jest vzýváno jméno mé, dí Hospodin, kterýž to uciní.
\par 13 Aj, dnové jdou, dí Hospodin, že postihati bude orác žence, a ten, kdož tlací hrozny, rozsevace, hory pak dštíti budou mstem, a všickni pahrbkové oplývati.
\par 14 Privedu také zase zajatý lid svuj Izraelský, a vystavejí mesta zpuštená, aby v nich bydlili; a budou štepovati vinice, a píti víno jejich, nadelají i zahrad, a jísti budou ovoce jejich.
\par 15 A tak je štípím v zemi jejich, že nebudou vykoreneni více z zeme své, kterouž jsem dal jim, praví Hospodin Buh tvuj.

\end{document}