\begin{document}

\title{Micah}

\chapter{1}

\par 1 Slovo Hospodinovo, kteréž se stalo k Micheášovi Moraštickému za dnu Jotama, Achasa a Ezechiáše králu Judských, kteréž u videní slyšel o Samarí a Jeruzalému.
\par 2 Slyšte všickni lidé naporád, pozoruj zeme, i což na ní jest, a necht jest Panovník Hospodin proti vám svedkem, Panovník z chrámu svatosti své.
\par 3 Nebo aj, Hospodin vyjde z místa svého, a sstoupe, šlapati bude po vysokostech zeme.
\par 4 I budou se rozplývati hory pod ním, a údolé se roztrhovati, tak jako vosk od ohne, a jako vody mající spád dolu,
\par 5 A to všecko pro Jákobovo zproneverení, a pro hríchy domu Izraelského. Kdo jest prícina zproneverení Jákobova? Zdali ne Samarí? A kdo výsostí Judských: Zdali ne Jeruzalém?
\par 6 Protož obrátím Samarí v hromadu rumu, k štípení vinic, a svalím do údolí kamení její, i základy její odkryji.
\par 7 A všecky rytiny její ztluceny budou, i všickni darové její ohnem spáleni, a všecky modly její obrátím v pustinu. Nebo ze mzdy nevestcí toho nashromáždila, protož se zase ke mzde nevestcí to navrátí.
\par 8 Nad címž kvíliti a naríkati budu, chode svlecený a nahý, vydám se v naríkání jako drakové, a v kvílení jako mladé sovy,
\par 9 Proto že zneduživela od ran svých, a že prišlo to až k Judovi, dosáhlo až k bráne lidu mého, až do Jeruzaléma.
\par 10 Neoznamujtež v Gát, aniž hned placte; v dome Ofra v prachu se válej.
\par 11 Ty, kteráž bydlíš v Safir, zajdi, obnaženou majíc hanbu. Nevyjdet ta, kteráž bydlí v Zaanan pro kvílení v Betezel, od vás maje živnost svou.
\par 12 Bude, pravím, bolestiti pro dobré veci obyvatelkyne Marót, proto že sstoupí zlé od Hospodina až do brány Jeruzalémské.
\par 13 Zapráhni do vozu rychlé kone, obyvatelkyne Lachis, kteráž jsi puvod hrícha dceri Sionské; nebo v tobe nalezena jsou prestoupení Izraelova.
\par 14 Protož pošleš dary své s Morešet v Gát; domové Achzib zklamají krále Izraelské.
\par 15 Ó obyvatelkyne Maresa, i tobet tudíž privedu dedice; až do Adulam prijde, k sláve Izraelské.
\par 16 Uciniž sobe lysinu, a ohol se pro syny rozkoší svých; rozšir lysinu svou jako orlice, nebo stehují se od tebe.

\chapter{2}

\par 1 Beda tem, kteríž vymýšlejí nepravost a ukládají zlé na ložcích svých, a na úsvite ráno vykonávají je, když jest v moci rukou jejich.
\par 2 Požádají polí, i vydírají, takž i domu, a odjímají; a tak provozují moc nad mužem i domem jeho, nad jedním každým i dedictvím jeho.
\par 3 Protož takto praví Hospodin: Aj, já myslím proti celedi té vec zlou, odkudž nevynmete hrdel vašich, aniž budete choditi vysokomyslne; nebo cas zlý bude.
\par 4 V ten den užívati budou o vás prísloví, a naríkati budou naríkáním žalostným, rkouce: Do cela zpušteni jsme, podíl lidu mého promenil. Jakte mi jej odjal, a vzav pole naše, rozdelil!
\par 5 Protož nebudeš míti, kdožt by vztáhl provazec na los v shromáždení Hospodinovu.
\par 6 Neprorokujte. Budou prorokovati, ale nebudou temto prorokovati, a neponesou pohanení.
\par 7 Ó kteráž sloveš domem Jákobovým, zdali do tesna vehnán býti má duch Hospodinuv? To-liž by skutkové jeho byli? Zdaliž slova má nejsou dobrá tomu, kdož upríme chodí?
\par 8 Vcera byl lidem mým, již jako neprítel povstává. Majíce odev, plášt strhujete s tech, kteríž chodí bezpecne, vyhýbajíce se bitve.
\par 9 Manželky lidu mého vyháníte z domu rozkoší jejich, od dítek jejich odjímáte slávu mou na veky.
\par 10 Vstante a odejdete, nebot tato není sídlem pro necistotu. Ztratí vás, a to ztracením jistým.
\par 11 Jestliže za proroka se vydávaje a lež mluve, ríká: Budu tobe prorokovati o víne aneb o nápoji opojujícím, takový bývá prorokem lidu tohoto.
\par 12 Jistotne zberu te, Jákobe, docela, jistotne shromáždím ostatky Izraele, a seženu je v hromadu jako ovce v Bozra, jako stádo do prostred ovcince jeho, i vzejde hluk od lidu.
\par 13 Vstoupí ten, kterýž prolamovati bude pred nimi. Prolomí, a projdou bránu, a vyjdou skrze ni; ano i král jejich pujde pred nimi, a Hospodin na špici jejich.

\chapter{3}

\par 1 Protož pravím: Slyštež již prední v Jákobovi, a vudcové domu Izraelského: Zdaliž vy nemáte povedomi býti soudu?
\par 2 Nenávidí dobrého a milují zlé, sdírají s lidu kuži jejich, a maso jejich s kostí jejich.
\par 3 A jedí maso lidu mého, a kuži jejich s nich svlácejí, i kosti jejich rozlamují, a rozdelují jako do hrnce, a jako maso do kotlíku.
\par 4 Tehdy volati budou k Hospodinu, a nevyslyší jich, ale skryje tvár svou pred nimi v ten cas, tak jakž oni vykonávali zlá predsevzetí svá.
\par 5 Takto praví Hospodin o tech prorocích, kteríž v blud uvodí lid muj, a hryzouce zuby svými, vyhlašují pokoj, a proti tomu, kdož by jim nic do úst nedal, válku vyzdvihují.
\par 6 A protož obrátí se vám videní v noc, a predpovídání vaše v tmu; nebo zajde slunce tem prorokum, a zatmí se jim den.
\par 7 I budou se hanbiti ti vidoucí, a stydeti ti veštci, a zastrou bradu svou všickni naporád, proto že nebude žádné odpovedi Boží.
\par 8 Ale já naplnen jsem silou Ducha Hospodinova, a soudem i udatností, abych oznámil Jákobovi zproneverení jeho, a Izraelovi hrích jeho.
\par 9 Slyštež již toto prední v dome Jákobove, a vudcové domu Izraelského, kteríž v ošklivosti mají soud, a cožkoli jest pravého, prevracejí:
\par 10 Každý vzdelává Sion vraždami, a Jeruzalém nepravostí.
\par 11 Jehož prední podlé daru soudí, a kneží jeho ze mzdy ucí, a proroci jeho z penez hádají, a však na Hospodina zpoléhají, ríkajíce: Zdaliž Hospodina není u prostred nás? Neprijdet na nás nic zlého.
\par 12 A protož vaší prícinou Sion jako pole orán bude, a Jeruzalém v hromady obrácen bude, a hora domu toho v vysoké lesy.

\chapter{4}

\par 1 Ale stane se v posledních dnech, že utvrzena bude hora domu Hospodinova na vrchu hor, a vyvýšena nad pahrbky, i pohrnou se k ní národové.
\par 2 A pujdou lidé mnozí, ríkajíce: Podte, a vstupme na horu Hospodinovu, totiž do domu Boha Jákobova, a bude nás vyucovati cestám svým, i budeme choditi po stezkách jeho. Nebo z Siona vyjde zákon, a slovo Hospodinovo z Jeruzaléma.
\par 3 Ont bude souditi mezi národy mnohými, a trestati bude národy silné za dlouhé casy. I skují mece své v motyky, a oštípy své v srpy. Nepozdvihne národ proti národu mece, a nebudou se více uciti boji.
\par 4 Ale sedeti bude každý pod vinným kmenem svým, a pod fíkovím svým, a nebude žádného, kdo by prestrašil; nebo ústa Hospodina zástupu mluvila.
\par 5 Všickni zajisté národové choditi budou jeden každý ve jménu boha svého, ale my choditi budeme ve jménu Hospodina Boha našeho na veky veku.
\par 6 V ten den, dí Hospodin, zberu zase kulhavou, a zahnanou shromáždím, i tu, kteréž jsem zle cinil.
\par 7 I dám té kulhavé potomky, a pryc zahnané národ silný, a bude kralovati Hospodin nad nimi na hore Sion od tohoto casu až na veky.
\par 8 A tak ty veže bravná, bašto dcery Sionské, až k tobe prijde, prijde, pravím, panování první, a království k dceri Jeruzalémské.
\par 9 Procež nyní tak velice kricíš? Zdaliž není žádného krále v tobe? Zdali rádce tvuj zahynul, že te zachvátila bolest jako rodicku?
\par 10 Pracujž ku porodu a úpej, dcero Sionská, jako rodicka; nebo již vyjdeš z mesta, a budeš bydliti na poli, a prijdeš až do Babylona. Tam vytržena budeš, tam te vykoupí Hospodin z ruky neprátel tvých.
\par 11 Sbírajít se nyní sic proti tobe národové mnozí, ríkající: Necht jest poškvrnen Sion, a necht se podívají na to oci naše.
\par 12 Však oni neznají myšlení Hospodinových, aniž rozumejí rade jeho, že je shromažduje jako snopy na humno.
\par 13 Vstaniž a mlat, dcero Sionská; nebo rok tvuj uciním železný, a kopyta tvá uciním ocelivá. I zetreš národy mnohé, a posvetím Hospodinu jmení jejich, a zboží jejich Pánu vší zeme.

\chapter{5}

\par 1 Sberiž se nyní po houfích, ó záškodnice, oblehni nás, nechat bijí holí v líce soudce Izraelského.
\par 2 A ty Betléme Efrata, jakžkoli jsi nejmenší mezi tisíci Judskými, z tebe mi vyjde ten, kterýž má býti Panovníkem v Izraeli, a jehož východové jsou od starodávna, ode dnu vecných.
\par 3 Protož, ac je vydá v rozptýlení, ažby ta, kteráž rodí, porodila, však ostatek bratrí jeho navrátí se s syny Izraelskými.
\par 4 I stane a pásti bude v síle Hospodinove, a u velebnosti jména Hospodina Boha svého. I budou bydliti, nebo již velikomocný bude až do koncin zeme.
\par 5 I budet takový pokoj, že když Assur pritáhne do zeme naší, a šlapati bude po palácích našich, tedy postavíme proti nemu sedm pastýru, a osmero knížat z lidu,
\par 6 Kteríž spasou zemi Assyrskou mecem, a zemi Nimrodovu v pomezích jejích. A tak vytrhne od Assura, když pritáhne do zeme naší, a šlapati bude po našem pomezí.
\par 7 A protož ostatkové Jákobovi u prostred národu mnohých budou jako rosa od Hospodina, jako tiší deštové skrápející bylinu, jichž neocekává od žádného, aniž ceká od synu lidských.
\par 8 Budou též ostatkové Jákobovi mezi pohany, a u prostred národu mnohých, jako lev mezi zverí divokou, jako mladý lev mezi stády ovec. Kterýžto když jde, a pošlapává, i lapá, není žádného, kdo by vytrhl.
\par 9 Vyvýšit se ruka tvá nad tvými neprátely, a všickni protivníci tvoji vypléneni budou.
\par 10 I stane se v ten den, dí Hospodin, že vypléním kone tvé z prostredku tvého, a zkazím vozy tvé.
\par 11 A vypléním mesta zeme tvé, a rozborím všecky pevnosti tvé.
\par 12 Vypléním též kouzly z tebe, a planetáru nebude v tobe.
\par 13 Zahladím i rytiny tvé, i obrazy tvé z prostredku tvého, a nebudeš se klaneti více dílu rukou svých.
\par 14 Vykorením i háje tvé z prostredku tvého, a zkazím neprátely tvé.
\par 15 A tak v hnevu a v prchlivosti vykonám pomstu nad temi národy, kteríž nebyli poslušni.

\chapter{6}

\par 1 Slyštež nyní, co praví Hospodin: Vstan, sud se s temito horami, a necht slyší pahrbkové hlas tvuj.
\par 2 Slyštež hory rozepri Hospodinovu, i nejpevnejší základové zeme; nebo má rozepri Hospodin s lidem svým, a proti Izraelovi odpor povede.
\par 3 Lide muj, cožt jsem ucinil? A cím jsem te obtežoval? Vydej svedectví proti mne.
\par 4 Ješto jsem te vyvedl z zeme Egyptské, a z domu služebníku vykoupil jsem te, a poslal jsem pred tvárí tvou Mojžíše, Arona a Marii.
\par 5 Lide muj, rozpomen se nyní, jakou radu skládal Balák král Moábský, a co jemu odpovídal Balám syn Beoruv, od Setim až do Galgala, abys poznal hojnou spravedlnost Hospodinovu.
\par 6 Címž predejdu Hospodina? Skloniti-liž se mám pred Bohem nejvyšším? Predejdu-liž ho zápaly, volky rocními?
\par 7 Zalíbí-liž sobe Hospodin v tisících skopcu, v mnohokrát desíti tisících potoku oleje? Dám-liž prvorozeného svého za prestoupení své, plod života svého za hrích duše své?
\par 8 Oznámilte tobe, ó clovece, co jest dobrého, i cehož Hospodin vyhledává od tebe, jediné, abys cinil soud, a miloval milosrdenství, a pokorne chodil s Bohem svým.
\par 9 Hlas Hospodinuv na mesto volá (ale sám rozumný spatruje jméno tvé,): Slyštež o metle, a kdo ji uložil.
\par 10 Ješte-liž jsou v dome bezbožného pokladové nespravedliví, a míra nespravedlivá a ohavná?
\par 11 Zdaliž ospravedlniti mám vážky nepravé, a v pytlíku kamení falešné?
\par 12 Bohatí jeho plní jsou nátisku, a obyvatelé jeho mluví lež, a jazyk jejich lstivý jest v ústech jejich.
\par 13 Procež i já také nemoc dopustím, bíti a pléniti te budu pro hríchy tvé.
\par 14 Ty budeš jísti, a nenasytíš se, a snížení tvé bude u prostred tebe. Vyneseš zajisté, ale neodneseš, a co vyneseš, vydám pod mec.
\par 15 Ty budeš síti, ale nebudeš žíti; ty tlaciti budeš olivky, ale nebudeš se pomazovati olejem, i mest, ale nebudeš píti vína.
\par 16 Nebo snažne ostríhá ustanovení Amri, i každého skutku domu Achabova, a spravujete se radami jejich, tak abych te vydal v zpuštení, a obyvatele jeho v posmech. A protož pohanení lidu mého ponesete.

\chapter{7}

\par 1 Beda mne, že jsem jako paberek úrod letních, jako paberkové po vinobraní. Není žádného hroznu k jídlu, prvotiny z ovoce žádá duše má.
\par 2 Zahynul pobožný z zeme této, a uprímého mezi lidmi není žádného; všickni naporád o vylití krve úklady ciní, jeden každý bratra svého loví sítí.
\par 3 Co zlého obema rukama páchají, to aby za dobré pocteno bylo. Kníže žádá, a soudce z úplatku soudí, a kdož veliký jest, ten mluví prevrácenost duše své, a v hromadu ji pletou.
\par 4 Nejlepší z nich jest jako bodlák, nejuprímejší prevyšuje trní. Pricházít den strážných tvých, navštívení tvé prichází; jižt nastane zpletení jejich.
\par 5 Nedoverujte se príteli, nedoufejte v vudce; pred tou, jenž leží v lunu tvém, ostríhej dverí úst svých.
\par 6 Nebo syn v lehkost uvodí otce, dcera povstává proti materi své, nevesta proti svegruši své, a neprátelé jednoho každého jsou vlastní jeho.
\par 7 Protož já na Hospodina vyhlédati budu, ocekávati budu na Boha spasení svého, vyslyšít mne Buh muj.
\par 8 Neraduj se ze mne, neprítelkyne má. Upadla-lit jsem, povstanu; sedím-lit v temnostech, svítí mi Hospodin.
\par 9 Zurivost Hospodinovu ponesu, nebo jsem proti nemu zhrešila, až se vždy zasadí o mou pri, a mne zastane. Vyvedet mne na svetlo, budu videti spravedlnost jeho.
\par 10 Uzrít to neprítelkyne má, a prikryje ji hanba, ješto mi ríká: Kdež jest Hospodin Buh tvuj? Oci mé podívají se na ni; jižt bude rozšlapána jako bláto na ulicích.
\par 11 Toho dne, v nemž vystaveny budou hradby tvé, toho dne daleko se rozejde výpoved.
\par 12 Toho dne k tobe pricházeti budou i z Assyrské zeme až do pevností, a od pevností až k rece, a od more k mori, a od hory k hore.
\par 13 A však zeme tato zpuštena bude pro obyvatele své, pro ovoce cinu jejich.
\par 14 Pasiž lid svuj berlou svou, stádce dedictví svého, kteréž prebývá osamelé v lese u prostred polí, at spasou Bázan a Galád jako za dnu starodávních,
\par 15 Jako za dnu v nichž jsi vyšel z zeme Egyptské. Ukáži jemu divné veci.
\par 16 Což vidouce národové, stydeti se budou za všecku sílu svou; vloží ruku na ústa, a uši jejich ohlechnou.
\par 17 Lízati budou prach jako had, a jako hadové zemští s tresením polezou z der svých; k Hospodinu Bohu našemu, predešeni jsouce, pobehnou, a báti se tebe budou.
\par 18 Kdo jest Buh silný podobný tobe, kterýž by snímal nepravost a promíjel prestoupení ostatkum dedictví svého, kterýž by nedržel na veky hnevu svého, proto že líbost má v slitování se?
\par 19 Navráte se, slituje se nad námi, podmaní nepravosti naše; nýbrž uvržeš do hlubin morských všecky hríchy naše.
\par 20 Pravdomluvným se ukážeš Jákobovi, milosrdným Abrahamovi, jakož jsi prisáhl otcum našim ode dnu starodávních.


\end{document}