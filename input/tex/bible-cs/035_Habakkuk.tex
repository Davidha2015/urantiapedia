\begin{document}

\title{Abakuk}

\chapter{1}

\par 1 Bríme, kteréž u videní videl Abakuk prorok.
\par 2 Až dokud, ó Hospodine, kriceti budu, a nevyslyšíš? Volati budu k tobe pro nátisk, a nespomužeš?
\par 3 Proc dopouštíš, abych hledeti musil na nepravost, a na bezpráví se dívati, též na zhoubu a na nátisk proti mne? A vždy jest, kdož svár a ruznici vzbuzuje.
\par 4 Odkudž se deje opuštení zákona, a nedochází nikdá soud. Nebo bezbožný obklicuje spravedlivého, procež vychází soud prevrácený.
\par 5 Pohledte na národy a popatrte, nýbrž s velikým podivením se užasnete, proto že delám dílo ve dnech vašich, o nemž když vypravováno bude, neuveríte.
\par 6 Nebo aj, já vzbudím Kaldejské, národ lítý a rychlý, kterýž zjezdí všecku širokost zeme, aby dedicne opanoval príbytky jiných,
\par 7 Strašlivý a hrozný, od nehož soud jeho i vyvýšenost jeho vyjde.
\par 8 Koni jeho rychlejší budou než rysové, a lítejší nad vlky vecerní; veliké množství bude jezdcu jeho, kterížto jezdci jeho zdaleka prijedou, a priletí jako orlice, kteráž chvátá na pastvu.
\par 9 Každý z nich k utiskování prijde, obrátíce tvári své k východu, když seberou jako písek zajaté.
\par 10 Tentýž i králum se posmívati bude, a knížata smích jemu budou; tentýž každé pevnosti smáti se bude, a zdelaje náspy, dobude jí.
\par 11 Tehdy promení se duch jeho, a prestoupí i zaviní, mysle, že ta moc jeho boha jeho jest.
\par 12 Hospodine Bože muj, Svatý muj, zdaliž ty nejsi od vecnosti? Myt nezemreme. Hospodine, k soudu postavil jsi jej, ty, ó skálo, k trestání nastrojil jsi ho.
\par 13 Cistét jsou tvé oci, tak že na zlé veci hledeti, a na bezpráví se dívati nemužeš. Procež prehlídati máš nešlechetníkum a mlceti, ponevadž bezbožník sehlcuje spravedlivejšího, než sám jest.
\par 14 A zanechávati lidí jako ryb morských, jako zemeplazu, kterýž nemá pána?
\par 15 Všecky naporád udicí vytahuje, zatahuje je sítí svou, a zahrnuje je nevodem svým, protož veselí se a pléše.
\par 16 Protož síti své obetuje, a kadí nevodu svému; nebo skrze ne ztucnel díl jeho, a strava jeho zlepšena.
\par 17 S tím-liž se vším predce zatahovati má sít svou, a ustavicne národy mordovati bez lítosti?

\chapter{2}

\par 1 Na stráži své státi budu, a postavím se na bašte, vyhlédaje, abych videti mohl, co mluviti bude ke mne Buh, a co bych odpovídati mel po svém trestání.
\par 2 I odpovedel mi Hospodin, a rekl: Napiš videní, a to zretelne na dskách, aby je prebehl ctenár,
\par 3 Proto že ješte do jistého casu bude videní, a smele mluviti bude až do konce, a nesklamát. Jestliže by pak poprodlilo, poseckej na ne; nebot jistotne dojde, aniž bude meškati.
\par 4 Aj ten, kdož se zpíná, tohot duše není uprímá v nem, ale spravedlivý z víry své živ bude.
\par 5 Ovšem pak více opilec, nešlechetnost a pýchu provode, neostojí v príbytku; kterýž rozširuje jako peklo duši svou, jest jako smrt, kteráž se nemuže nasytiti, byt pak shromáždil k sobe všecky národy, a shrnul k sobe všecky lidi.
\par 6 Zdaliž všickni ti proti nemu prísloví nevynesou, a svetlých slov i pohádek o nem? A nereknou-liž: Beda tomu, kterýž rozmnožuje veci ne své, (až dokud pak?) a obtežuje se hustým blátem?
\par 7 Zdaliž nepovstanou rychle, kteríž by te hryzli, a neprocítí, kteríž by tebou smýkali? A budeš u nich v ustavicném potlacení.
\par 8 Proto že jsi ty zloupil národy mnohé, zloupí te všickni ostatkové národu, pro krev lidskou, a nátisk zeme a mesta, i všech, kteríž prebývají v nem.
\par 9 Beda tomu, kdož lakome hledá mrzkého zisku domu svému, aby postavil na míste vysokém hnízdo své, a tak znikl nebezpecenství.
\par 10 Uradils se k hanbe domu svému, abys plénil národy mnohé, a zhrešil sobe samému.
\par 11 Nebo kamení ze zdi kriceti bude, a suk z dreva posvedcovati bude toho.
\par 12 Beda tomu, kterýž staví mesto krví, a utvrzuje mesto nepravostí.
\par 13 Aj, zdaliž to není od Hospodina zástupu, že, o cemž pracují lidé a národové až do ustání nadarmo, ohen zkazí.
\par 14 Nebo naplnena bude zeme známostí slávy Hospodinovy, jako vody naplnují more.
\par 15 Beda tomu, kterýž napájí bližního svého, pricineje nádoby své, tak aby jej opojil, a díval se na jeho nahotu.
\par 16 Sytíš se potupou, proto že jsi slavný. Píti budeš i ty, a obnažen budeš; obejdet k tobe kalich pravice Hospodinovy, a vývratek mrzutý na slávu tvou.
\par 17 Nebo nátisk Libánu a zhouba zveri, kteráž ji desila, prikryje te, pro krev lidskou, a nátisk zeme a mesta, i všech, kteríž prebývají v nem.
\par 18 Co prospívá rytina, že ji vyryl remeslník její? Slitina i ucitel lži, že doufá ucinitel v úcinek svuj, delaje modly nemé?
\par 19 Beda tomu, kterýž ríká drevu: Procit, a kameni nemému: Probud se. On-liž by uciti mohl? Pohled na nej. Obložent jest zlatem a stríbrem, ale není v nem žádného ducha.
\par 20 Hospodin pak v chráme svatosti své jest, umlkniž pred oblícejem jeho všecka zeme.

\chapter{3}

\par 1 Modlitba Abakuka proroka podlé Šigejonót:
\par 2 Ó Hospodine, uslyšev pohružku tvou, ulekl jsem se. Hospodine, dílo své u prostred let pri životu zachovej, u prostred let známé ucin, v hneve na milosrdenství se rozpomen.
\par 3 Když se Buh bral od poledne, a Svatý s hory Fáran, Sélah, slávu jeho prikryla nebesa, a zeme byla plná chvály jeho.
\par 4 Blesk byl jako svetlo, rohy po bocích svých mel, a tu skryta byla síla jeho.
\par 5 Pred tvárí jeho šlo morní nakažení, a uhlí reravé šlo pred nohama jeho.
\par 6 Zastavil se, a zmeril zemi; pohledel, a rozptýlil národy. Zrozrážíny jsou hory vecné, sklonili se pahrbkové vecní, cesty jeho jsou vecné.
\par 7 Videl jsem, že stanové Chusan jsou pouhá marnost, a trásli se kobercové zeme Madianské.
\par 8 Zdaliž se na reky, ó Hospodine, zdaliž se na reky rozpálil hnev tvuj? Zdali proti mori rozhnevání tvé, když jsi jel na koních svých a na vozích svých spasitelných?
\par 9 Patrne jest zjeveno lucište tvé pro prísahy pokolením lidu tvého stalé, Sélah. Reky zeme jsi rozdelil,
\par 10 Videly te hory, trásly se, povoden vod ustoupila; vydala propast hlas svuj, hlubina rukou svých pozdvihla.
\par 11 Slunce a mesíc v obydlí svém zastavil se, pri svetle strely tvé létaly, pri blesku stkvoucí kopí tvé.
\par 12 V hneve šlapal jsi zemi, v prchlivosti mlátil jsi pohany.
\par 13 Vyšel jsi k vysvobození lidu svého, k vysvobození s pomazaným svým; srazil jsi hlavu s domu bezbožníka až do hrdla, obnaživ základ. Sélah.
\par 14 Holemi jeho probodl jsi hlavu vsí jeho, když se bourili jako vichrice k rozptýlení mému, plésali, jako by sežrati meli chudého v skryte.
\par 15 Bral jsi se po mori na koních svých, skrze hromadu vod mnohých.
\par 16 Slyšel jsem, a zatráslo se bricho mé, k hlasu tomu drkotali rtové moji, kosti mé práchnively, a všecken jsem se trásl, že se mám upokojiti v den ssoužení, když pritáhne na lid, aby jej válecne hubil.
\par 17 Byt pak fík nekvetl, a nebylo úrody na vinicích; byt i ovoce olivy pochybilo, a rolí neprinesla užitky; a od ovcince odrezován byl brav, a nebylo žádného skotu v chlévích:
\par 18 Já však v Hospodinu veseliti se budu, plésati budu v Bohu spasení svého.
\par 19 Hospodin Panovník jest síla má, kterýž ciní nohy mé jako laní, a na vysokých místech mých cestu mi zpusobuje. Prednímu zpeváku na muj neginot.

\end{document}