\begin{document}

\title{Matouš}

\chapter{1}

\par 1 Kniha (o) rodu Ježíše Krista syna Davidova, syna Abrahamova.
\par 2 Abraham zplodil Izáka. Izák pak zplodil Jákoba. Jákob zplodil Judu a bratrí jeho.
\par 3 Judas pak zplodil Fáresa a Záru z Támar. Fáres pak zplodil Ezroma. Ezrom zplodil Arama.
\par 4 Aram pak zplodil Aminadaba. Aminadab pak zplodil Názona. Názon zplodil Salmona.
\par 5 Salmon zplodil Bóza z Raab. A Bóz zplodil Obéda z Rut. Obéd pak zplodil Jesse.
\par 6 Jesse zplodil Davida krále. David pak král zplodil Šalomouna, z té, kteráž nekdy byla žena Uriášova.
\par 7 Šalomoun zplodil Roboáma. Roboám zplodil Abiáše. Abiáš zplodil Azu.
\par 8 Aza zplodil Jozafata. Jozafat zplodil Joráma. Jorám zplodil Oziáše.
\par 9 Oziáš pak zplodil Joátama. Joátam pak zplodil Achasa. Achas zplodil Ezechiáše.
\par 10 Ezechiáš pak zplodil Manasesa. A Manases zplodil Amona. Amon pak zplodil Joziáše.
\par 11 Joziáš pak zplodil Jekoniáše a bratrí jeho v zajetí Babylonském.
\par 12 A po zajetí Babylonském Jekoniáš zplodil Salatiele. Salatiel pak zplodil Zorobábele.
\par 13 A Zorobábel zplodil Abiuda. Abiud pak zplodil Eliachima. Eliachim zplodil Azora.
\par 14 Azor potom zplodil Sádocha. Sádoch zplodil Achima. Achim pak zplodil Eliuda.
\par 15 Eliud zplodil Eleazara. Eleazar zplodil Mátana. Mátan zplodil Jákoba.
\par 16 Jákob pak zplodil Jozefa, muže Marie, z nížto narodil se JEŽÍŠ, jenž slove Kristus.
\par 17 A tak všech rodu od Abrahama až do Davida bylo rodu ctrnácte. A od Davida až do zajetí Babylonského rodu ctrnácte. A od zajetí Babylonského až do Krista rodu ctrnácte.
\par 18 Jezukristovo pak narození takto se stalo: Když matka jeho Maria snoubena byla Jozefovi, prve než se sešli, nalezena jest tehotná z Ducha svatého.
\par 19 Ale Jozef muž její spravedlivý jsa, a nechtev jí v lehkost uvésti, chtel ji tajne propustiti.
\par 20 Když pak on o tom premyšloval, aj, andel Páne ve snách ukázal se jemu, rka: Jozefe synu Daviduv, neboj se prijíti Marie manželky své; nebo což v ní jest pocato, z Ducha svatého jest.
\par 21 Porodít pak syna, a nazuveš jméno jeho Ježíš; ont zajisté vysvobodí lid svuj od hríchu jejich.
\par 22 Toto pak všecko stalo se, aby se naplnilo, což povedíno bylo ode Pána skrze proroka, rkoucího:
\par 23 Aj, panna tehotná bude, a porodí syna, a nazuveš jméno jeho Emmanuel, jenž se vykládá: S námi Buh.
\par 24 Procítiv pak Jozef ze sna, ucinil, jakož mu prikázal andel Páne, a prijal manželku svou.
\par 25 Ale nepoznal jí, až i porodila Syna svého prvorozeného, a nazvala jméno jeho Ježíš.

\chapter{2}

\par 1 Když se pak narodil Ježíš v Betléme Judove, za dnu Herodesa krále, aj, mudrci od východu slunce prijeli do Jeruzaléma,
\par 2 Rkouce: Kde jest ten, kterýž se narodil Král Židovský? Nebo videli jsme hvezdu jeho na východu slunce, a prijeli jsme klaneti se jemu.
\par 3 To uslyšev Herodes král, zarmoutil se i všecken Jeruzalém s ním.
\par 4 A svolav všecky prední kneží a ucitele lidu, tázal se jich, kde by se Kristus mel naroditi.
\par 5 Oni pak rekli jemu: V Betléme Judove. Nebo tak jest psáno skrze proroka:
\par 6 A ty Betléme, zeme Judská, nikoli nejsi nejmenší mezi knížaty Judskými; nebot z tebe vyjde Vývoda, kterýž pásti bude lid muj Izraelský.
\par 7 Tedy Herodes tajne povolav mudrcu, pilne se jich ptal, kterého by se jim casu hvezda ukázala.
\par 8 A když je propouštel do Betléma, rekl: Jdouce, ptejte se pilne na to díte, a když naleznete, zvestujtež mi, at i já prijda, pokloním se jemu.
\par 9 Oni pak vyslyševše krále, jeli. A aj, hvezda, kterouž byli videli na východu slunce, predcházela je, až i prišedši, stála nad domem, kdež bylo díte.
\par 10 A uzrevše hvezdu, zradovali se radostí velmi velikou.
\par 11 I všedše do domu, nalezli díte s Marijí matkou jeho, a padše, klaneli se jemu; a otevrevše poklady své, obetovali jemu dary, zlato a kadidlo a mirru.
\par 12 A od Boha napomenuti jsouce ve snách, aby se nenavraceli k Herodesovi, jinou cestou navrátili se do krajiny své.
\par 13 Když pak oni odjeli, aj, andel Páne ukázal se Jozefovi ve snách, rka: Vstana, vezmi díte i matku jeho, a utec do Egypta, a bud tam, dokavadž nepovím tobe; nebot bude Herodes hledati díte, aby je zahubil.
\par 14 Kterýžto vstav hned v noci, vzal díte i matku jeho, a odšel do Egypta.
\par 15 A byl tam až do smrti Herodesovy, aby se naplnilo povedení Páne skrze proroka, rkoucího: Z Egypta povolal jsem Syna svého.
\par 16 Tehdy Herodes uzrev, že by oklamán byl od mudrcu, rozhneval se náramne, a poslav služebníky své, zmordoval všecky dítky, kteréž byly v Betléme i ve všech koncinách jeho, od dvouletých a níže, podle casu, na kterýž se byl pilne vyptal od mudrcu.
\par 17 Tehdy naplneno jest to povedení Jeremiáše proroka, rkoucího:
\par 18 Hlas v Ráma slyšán jest, naríkání a plác a kvílení mnohé; Ráchel placící synu svých, a nedala se potešiti, protože jich není.
\par 19 Když pak umrel Herodes, aj, andel Páne ukázal se Jozefovi ve snách v Egypte,
\par 20 Rka: Vstana, vezmi díte i matku jeho, a jdiž do zeme Izraelské; nebot jsou zemreli ti, jenž hledali bezživotí dítete.
\par 21 Kterýžto vstav, vzal díte i matku jeho, a prišel do zeme Izraelské.
\par 22 A uslyšev, že by Archelaus kraloval v Judstvu místo Herodesa otce svého, obával se tam jíti; a napomenut jsa od Boha ve snách, obrátil se do krajin Galilejských.
\par 23 A prišed, bydlil v meste, jenž slove Nazarét, aby se naplnilo, což povedíno bylo skrze proroky, že Nazaretský slouti bude.

\chapter{3}

\par 1 V tech pak dnech prišel Jan Krtitel, káže na poušti Judské,
\par 2 A rka: Pokání cinte, nebo priblížilo se království nebeské.
\par 3 Totot jest zajisté ten Predchudce predpovedený od Izaiáše proroka, rkoucího: Hlas volajícího na poušti: Pripravujte cestu Páne, prímé cinte stezky jeho.
\par 4 Mel pak Jan roucho z srstí velbloudových a pás kožený okolo bedr svých, a pokrm jeho byl kobylky a med lesní.
\par 5 Tedy vycházeli k nemu z Jeruzaléma a ze všeho Judstva, i ze vší krajiny ležící pri Jordánu,
\par 6 A krteni byli od neho v Jordáne, vyznávajíce hríchy své.
\par 7 Uzrev pak mnohé z farizeu a z saduceu, že jdou k jeho krtu, rekl jim: Pokolení ještercí, i kdo vám ukázal, kterak byste utéci meli budoucího hnevu?
\par 8 Protož cinte ovoce hodné pokání.
\par 9 A nedomnívejte se, že mužete ríkati sami u sebe: Otce máme Abrahama. Nebot pravím vám, že by mohl Buh z kamení tohoto vzbuditi syny Abrahamovi.
\par 10 A jižt jest i sekera k korenu stromu priložena. Každý zajisté strom, kterýž nenese ovoce dobrého, vytat a na ohen uvržen bývá.
\par 11 Já krtím vás vodou ku pokání, ten pak, kterýž po mne prichází, jestit mocnejší nežli já, jehožto nejsem hoden obuvi nositi. Ont vás krtíti bude Duchem svatým a ohnem.
\par 12 Jehožto vejecka v ruce jeho, a vycistít humno své, a shromáždí pšenici svou do obilnice, ale plevy páliti bude ohnem neuhasitelným.
\par 13 Tehdy prišel Ježíš od Galilee k Jordánu k Janovi, aby také pokrten byl od neho.
\par 14 Ale Jan zbranoval mu, rka: Mne jest potrebí, abych od tebe pokrten byl, a ty pak jdeš ke mne?
\par 15 A odpovídaje Ježíš, dí jemu: Dopust tak; nebot tak sluší na nás, abychom plnili všelikou spravedlnost. Tedy dopustil jemu.
\par 16 A pokrten jsa Ježíš, vystoupil ihned z vody; a aj, otevrína jsou mu nebesa, a videl Ducha Božího, sstupujícího jako holubici a pricházejícího na nej.
\par 17 A aj, zavznel hlas s nebe rkoucí: Tentot jest ten muj milý Syn, v nemž mi se dobre zalíbilo.

\chapter{4}

\par 1 Tehdy Ježíš veden jest na poušt od Ducha, aby pokoušín byl od dábla.
\par 2 A postiv se ctyridceti dnu a ctyridceti nocí, potom zlacnel.
\par 3 A pristoupiv k nemu pokušitel, rekl: Jsi-li Syn Boží, rciž, at kamení toto chlebové jsou.
\par 4 On pak odpovídaje, rekl: Psánot jest: Ne samým chlebem živ bude clovek, ale každým slovem vycházejícím skrze ústa Boží.
\par 5 Tedy pojal jej dábel do svatého mesta a postavil ho na vrchu chrámu.
\par 6 A rekl mu: Jsi-li Syn Boží, spustiž se dolu; nebo psánot jest, že andelum svým prikázal o tobe, a na ruce uchopí tebe, abys nekde o kámen nohy své neurazil.
\par 7 I rekl mu Ježíš: Zase psáno jest: Nebudeš pokoušeti Pána Boha svého.
\par 8 Opet pojal ho dábel na horu vysokou velmi, a ukázal mu všecka království sveta i slávu jejich, a rekl jemu:
\par 9 Toto všecko tobe dám, jestliže padna, budeš mi se klaneti.
\par 10 Tedy dí mu Ježíš: Odejdiž, satane; nebot jest psáno: Pánu Bohu svému klaneti se budeš a jemu samému sloužiti budeš.
\par 11 Tedy opustil ho dábel, a aj, andelé pristoupili a sloužili jemu.
\par 12 A když uslyšel Ježíš, že by Jan vsazen byl do žaláre, odšel do Galilee.
\par 13 A opustiv Nazarét, prišed, bydlil v Kafarnaum za morem, v krajinách Zabulon a Neftalím,
\par 14 Aby se naplnilo povedení skrze Izaiáše proroka, rkoucího:
\par 15 Zeme Zabulon a Neftalím pri mori za Jordánem, Galilea pohanská,
\par 16 Lid, kterýž bydlil v temnostech, videl svetlo veliké, a sedícím v krajine a stínu smrti, svetlo vzešlo jim.
\par 17 Od toho casu pocal Ježíš kázati a praviti: Pokání cinte; nebot se priblížilo království nebeské.
\par 18 A chode Ježíš podle more Galilejského, uzrel dva bratry, Šimona, kterýž slove Petr, a Ondreje bratra jeho, ani pouštejí sít do more, (nebo byli rybári.)
\par 19 I dí jim: Pojdte za mnou, a uciním vás rybáre lidí.
\par 20 A oni hned opustivše síti, šli za ním.
\par 21 A poodšed odtud, uzrel jiné dva bratry, Jakuba syna Zebedeova, a Jana bratra jeho, na lodí s Zebedeem otcem jejich, ani tvrdí síti své. I povolal jich.
\par 22 A oni hned opustivše lodí a otce svého, šli za ním.
\par 23 I procházel Ježíš všecku Galilei, uce v shromáždeních jejich a káže evangelium království a uzdravuje všelikou nemoc i všeliký neduh v lidu.
\par 24 A rozešla se o nem povest po vší Syrii. I privedli k nemu všecky nemocné, rozlicnými neduhy a trápeními poražené, i dábelníky, i námesicníky, i šlakem poražené; a uzdravoval je.
\par 25 I šli za ním zástupové mnozí z Galilee a z krajiny Desíti mest, i z Jeruzaléma i z Judstva i z Zajordání.

\chapter{5}

\par 1 Vida pak Ježíš zástupy, vstoupil na horu; a když se posadil, pristoupili k nemu ucedlníci jeho.
\par 2 I otevrev ústa svá, ucil je, rka:
\par 3 Blahoslavení chudí duchem, nebo jejich jest království nebeské.
\par 4 Blahoslavení lkající, nebo oni potešeni budou.
\par 5 Blahoslavení tiší, nebo oni dedictví obdrží na zemi.
\par 6 Blahoslavení, kteríž lacnejí a žíznejí spravedlnosti, nebo oni nasyceni budou.
\par 7 Blahoslavení milosrdní, nebo oni milosrdenství dujdou.
\par 8 Blahoslavení cistého srdce, nebo oni Boha videti budou.
\par 9 Blahoslavení pokojní, nebo oni synové Boží slouti budou.
\par 10 Blahoslavení, kteríž protivenství trpí pro spravedlnost, neb jejich jest království nebeské.
\par 11 Blahoslavení jste, když vám zloreciti budou lidé a protivenství ciniti, a mluviti všecko zlé o vás, lhouce, pro mne.
\par 12 Radujte se a veselte se, nebo odplata vaše hojná jest v nebesích. Takt zajisté protivili se prorokum, kteríž byli pred vámi.
\par 13 Vy jste sul zeme. Jestliže sul zmarena bude, cím bude osolena? K nicemuž se nehodí více, než aby byla ven vyvržena a od lidí potlacena.
\par 14 Vy jste svetlo sveta. Nemužet mesto na hore ležící skryto býti.
\par 15 Aniž rozsvecují svíce a stavejí ji pod kbelec, ale na svícen; i svítí všechnem, kteríž v domu jsou.
\par 16 Tak svet svetlo vaše pred lidmi, at vidí skutky vaše dobré, a slaví Otce vašeho, jenž jest v nebesích.
\par 17 Nedomnívejte se, že bych prišel rušiti Zákon anebo Proroky. Neprišelt jsem rušiti, ale naplniti.
\par 18 Amen zajisté pravím vám: Dokudž nepomine nebe i zeme, jediná literka aneb jediný punktík nepomine z Zákona, až se všecky veci stanou.
\par 19 Protož zrušil-li by kdo jedno z prikázání techto nejmenších a ucil by tak lidi, nejmenší slouti bude v království nebeském. Kdož by pak koli cinil i jiné ucil, ten veliký slouti bude v království nebeském.
\par 20 Nebo pravím vám: Nebude-lit hojnejší spravedlivost vaše nežli zákoníku a farizeu, nikoli nevejdete do království nebeského.
\par 21 Slyšeli jste, že receno jest od starých: Nezabiješ. Pakli by kdo zabil, povinen bude státi k soudu.
\par 22 Ale ját pravím vám: Že každý, kdož se hnevá na bratra svého bez príciny, povinen k soudu státi. Kdož by pak rekl bratru svému: Rácha, povinen bude pred radou státi; a kdož by rekl: Blázne, povinen bude pekelný ohen trpeti.
\par 23 Protož obetoval-li bys dar svuj na oltár, a tu bys se rozpomenul, že bratr tvuj má neco proti tobe,
\par 24 Nechejž tu daru svého pred oltárem a jdi, prve smir se s bratrem svým, a potom prijda, obetuj dar svuj.
\par 25 Vejdi v dobrou vuli s protivníkem svým rychle, dokudž jsi s ním na ceste, at by snad nedal tebe protivník tvuj soudci, a soudce dal by te služebníku, a byl bys uvržen do žaláre.
\par 26 Amen pravím tobe: Nevyjdeš odtud nikoli, dokudž i toho posledního halére nenavrátíš.
\par 27 Slyšeli jste, že receno jest od starých: Nezcizoložíš.
\par 28 Ale ját pravím vám: Že každý, kdož by pohledel na ženu ku požádání jí, již zcizoložil s ní v srdci svém.
\par 29 Jestliže pak oko tvé pravé horší te, vylupiž je a vrz od sebe; nebt jest užitecneji tobe, aby radeji zahynul jeden úd tvuj, nežli by celé telo tvé uvrženo bylo do ohne pekelného.
\par 30 A pakli ruka tvá pravá horší te, utniž ji a vrz od sebe; nebo užitecneji jest tobe, aby zahynul radeji jeden úd tvuj, nežli by všecko telo tvé uvrženo bylo do pekelného ohne.
\par 31 Též receno jest: Kdož by koli propustil manželku svou, aby jí dal lístek rozloucení.
\par 32 Ját pak pravím vám: Že kdožkoli propustil by manželku svou, krome príciny cizoložstva, uvodí ji v cizoložstvo, a kdož propuštenou pojme, cizoloží.
\par 33 Opet slyšeli jste, že receno jest od starých: Nebudeš krive prisahati, ale splníš Pánu prísahy své.
\par 34 Ale ját pravím vám: Abyste neprisahali všelijak, ani skrze nebe, nebo stolice Boží jest,
\par 35 Ani skrze zemi, nebo podnož jeho jest, ani skrze Jeruzalém, nebo mesto velikého Krále jest.
\par 36 Ani skrze hlavu svou budeš prisahati, nebo nemužeš jednoho vlasu uciniti bílého aneb cerného.
\par 37 Ale bud rec vaše: Jiste, jiste; nikoli, nikoli. Což pak nad to více jest, to od toho zlého jest.
\par 38 Slyšeli jste, že receno jest: Oko za oko, a zub za zub.
\par 39 Ját pak pravím vám: Abyste neodpírali zlému. Ale uderí-li te kdo v pravé líce tvé, nasad jemu i druhého.
\par 40 A tomu, kdož se s tebou chce souditi a sukni tvou vzíti, nech mu i plášte.
\par 41 A nutil-li by te kdo jíti s sebou míli jednu, jdi s ním dve.
\par 42 A prosícímu tebe dej, a od toho, kdo by chtel vypujciti od tebe, neodvracuj se.
\par 43 Slyšeli jste, že receno jest: Milovati budeš bližního svého, a nenávideti budeš neprítele svého.
\par 44 Ale ját vám pravím: Milujte neprátely vaše, dobrorecte tem, kteríž vás proklínají, a dobre cinte nenávidícím vás, a modlte se za neprátely a protivníky vaše,
\par 45 Abyste byli synové Otce vašeho, jenž jest v nebesích; ješto slunci svému velí vzchoditi na dobré i na zlé, a déšt dává na spravedlivé i na nespravedlivé.
\par 46 Nebo milujete-li ty, jenž vás milují, jakou odplatu míti budete? Zdaliž i publikáni téhož neciní?
\par 47 A budete-li pozdravovati toliko bratrí svých, což více nad jiné ciníte? Však i publikáni to ciní.
\par 48 Budtež vy tedy dokonalí, jako i Otec váš nebeský dokonalý jest.

\chapter{6}

\par 1 Pilne se varujte, abyste almužny vaší nedávali pred lidmi, proto abyste byli vidíni od nich, jinak nebudete míti odplaty u Otce vašeho, jenž jest v nebesích.
\par 2 Protož když dáváš almužnu, netrub pred sebou, jako pokrytci ciní v školách a na ulicech, aby chváleni byli od lidí. Amen pravím vám, majít odplatu svou.
\par 3 Ale ty když dáváš almužnu, tak cin, at neví levice tvá, co ciní pravice tvá,
\par 4 Aby almužna tvá byla v skryte, a Otec tvuj, kterýž vidí v skryte, odplatí tobe zjevne.
\par 5 A když bys se chtel modliti, nebývejž jako pokrytci, kteríž obycej mají, v školách a na úhlech rynku stojíce, modliti se, aby byli vidíni od lidí. Amen pravím vám, žet mají odplatu svou.
\par 6 Ale ty když bys se modliti chtel, vejdi do pokojíka svého, a zavra dvere své, modliž se Otci svému, jenž jest v skryte, a Otec tvuj, kterýž vidí v skryte, odplatí tobe zjevne.
\par 7 Modléce se pak, nebudtež marnomluvní jako pohané; nebo se domnívají, že mnohomluvností svou to zpusobí, aby byli uslyšáni.
\par 8 Neprirovnávejtež se tedy jim, nebot ví Otec váš, ceho jest vám potrebí, prve nežli byste vy ho prosili.
\par 9 A protož vy takto se modlte: Otce náš, jenž jsi v nebesích, posvet se jméno tvé.
\par 10 Prijd království tvé. Bud vule tvá jako v nebi tak i na zemi.
\par 11 Chléb náš vezdejší dej nám dnes.
\par 12 A odpust nám viny naše, jakož i my odpouštíme vinníkum našim.
\par 13 I neuvod nás v pokušení, ale zbav nás od zlého. Nebo tvé jest království, i moc, i sláva, na veky, Amen.
\par 14 Nebo budete-li odpoušteti lidem viny jejich, odpustít i vám nebeský Otec váš.
\par 15 Jestliže pak neodpustíte lidem vin jejich, aniž Otec váš odpustí vám hríchu vašich.
\par 16 Když byste se pak postili, nebývejtež jako pokrytci zasmušilí; nebot pošmurují tvárí svých, aby vedomé bylo lidem, že se postí. Amen pravím vám, vzalit jsou odplatu svou.
\par 17 Ty pak, když se postíš, pomaž hlavy své a tvár svou umej,
\par 18 Aby nebylo zjevné lidem, že se postíš, ale Otci tvému, kterýž jest v skryte. A Otec tvuj, kterýž vidí v skrytosti, odplatí tobe zjevne.
\par 19 Neskládejte sobe pokladu na zemi, kdežto mol a rez kazí, a kdež zlodeji vykopávají a kradou.
\par 20 Ale skládejte sobe poklady v nebi, kdežto ani mol ani rez kazí, a kdežto zlodeji nevykopávají ani kradou.
\par 21 Nebo kdežt jest poklad váš, tut jest i srdce vaše.
\par 22 Svíce tela jestit oko; jestliže oko tvé sprostné bylo by, všecko telo tvé svetlé bude.
\par 23 Paklit by oko tvé bylo nešlechetné, všecko telo tvé tmavé bude. Protož jestliže svetlo, kteréž jest v tobe, jest tma, co pak sama tma, jaká bude?
\par 24 Žádný nemuže dvema pánum sloužiti. Nebo zajisté jednoho nenávideti bude, a druhého milovati, aneb jednoho prídržeti se bude, a druhým pohrdne. Nemužte Bohu sloužiti i mamone.
\par 25 Protož pravím vám: Nepecujte o život váš, co byste jedli a co pili, ani o telo vaše, cím byste je odívali. Zdaliž není život více nežli pokrm, a telo více nežli odev?
\par 26 Hledte na ptactvo nebeské, žet nesejí ani žnou, ani shromaždují do stodol, avšak Otec váš nebeský krmí je. I zdaliž vy jich mnohem neprevyšujete?
\par 27 A kdo z vás peclive mysle, muže pridati ku postave své loket jeden?
\par 28 A o odev proc pecujete? Poucte se na kvítí polním, kterak roste, nepracuje ani prede.
\par 29 A aj, pravím vám, že ani Šalomoun ve vší sláve své tak odín nebyl, jako jedno z nich.
\par 30 Ponevadž tedy trávu polní, ješto dnes jest, a zítra do peci bývá vložena, Buh tak odívá, i zdaliž mnohem více vám toho neciní, ó malé víry?
\par 31 Nepecujtež tedy, ríkajíce: Co budeme jísti? anebo: Co budeme píti? anebo: Cím se budeme odívati?
\par 32 Nebo toho všeho pohané hledají. Vít zajisté Otec váš nebeský, že toho všeho potrebujete.
\par 33 Ale hledejte vy nejprv království Božího a spravedlnosti jeho, a toto vše bude vám pridáno.
\par 34 Protož nepecujte o zítrejší den, nebo zítrejší den pecovati bude o své veci. Dostit má den na svém trápení.

\chapter{7}

\par 1 Nesudtež, abyste nebyli souzeni.
\par 2 Nebo jakým soudem soudíte, takovýmž budete souzeni, a jakouž merou meríte, takovouž bude vám zase odmereno.
\par 3 Kterakž pak vidíš mrvu v oku bratra svého, a v oku svém brevna necítíš?
\par 4 Aneb kterak díš bratru svému: Nech, at vyvrhu mrvu z oka tvého, a aj, brevno jest v oku tvém?
\par 5 Pokrytce, vyvrz nejprv brevno z oka svého, a tehdy prohlédneš, abys vynal mrvu z oka bratra tvého.
\par 6 Nedávejte svatého psum, aniž mecte perel svých pred svine, at by snad nepotlacily jich nohama svýma, a psi obrátíce se, aby neroztrhaly vás.
\par 7 Proste, a dánot bude vám; hledejte, a naleznete; tlucte, a bude vám otevríno.
\par 8 Nebo každý, kdož prosí, bére; a kdož hledá, nalézá; a tomu, jenž tluce, bude otevríno.
\par 9 Nebo který z vás jest clovek, kteréhož kdyby prosil syn jeho za chléb, zdali kamene podá jemu?
\par 10 A prosil-li by za rybu, zdali hada podá jemu?
\par 11 Ponevadž tedy vy, jsouce zlí, umíte dobré dary dávati synum vašim, cím více Otec váš, jenž jest v nebesích, dá dobré veci tem, kteríž ho prosí?
\par 12 A protož všecko, což byste chteli, aby vám lidé cinili, to i vy cinte jim; tot zajisté jest Zákon i Proroci.
\par 13 Vcházejte tesnou branou; nebo prostranná brána a široká cesta jest, kteráž vede k zahynutí, a mnoho jest tech, kteríž vcházejí skrze ni.
\par 14 Nebo tesná jest brána a úzká cesta, kteráž vede k životu, a málo jest nalézajících ji.
\par 15 Pilne se pak varujte falešných proroku, kteríž pricházejí k vám v rouše ovcím, ale vnitr jsou vlci hltaví.
\par 16 Po ovocích jejich poznáte je. Zdaliž sbírají z trní hrozny, aneb z bodlácí fíky?
\par 17 Takt každý strom dobrý ovoce dobré nese, zlý pak strom zlé ovoce nese.
\par 18 Nemužet dobrý strom zlého ovoce nésti, ani strom zlý ovoce dobrého vydávati.
\par 19 Všeliký strom, kterýž nenese ovoce dobrého, vytat a na ohen uvržen bývá.
\par 20 A tak tedy po ovocích jejich poznáte je.
\par 21 Ne každý, kdož mi ríká: Pane, Pane, vejde do království nebeského, ale ten, kdož ciní vuli Otce mého, kterýž v nebesích jest.
\par 22 Mnozít mi dejí v onen den: Pane, Pane, zdaliž jsme ve jménu tvém neprorokovali, a ve jménu tvém dáblu nevymítali, a v tvém jménu zdaliž jsme divu mnohých necinili?
\par 23 A tehdyt jim vyznám, že jsem vás nikdy neznal. Odejdete ode mne, cinitelé nepravosti.
\par 24 A protož každého, kdož slyší slova má tato a zachovává je, pripodobním muži moudrému, kterýž ustavel dum svuj na skále.
\par 25 I spadl príval, a prišly reky, a váli vetrové, a oborili se na ten dum, a nepadl; nebo založen byl na skále.
\par 26 A každý, kdož slyší slova má tato, a neplní jich, pripodobnen bude muži bláznu, kterýž ustavel dum svuj na písku.
\par 27 I spadl príval, a prišly reky, a váli vetrové, a oborili se na ten dum, i padl, a byl pád jeho veliký.
\par 28 I stalo se, když dokonal Ježíš reci tyto, že se prevelmi divili zástupové ucení jeho.
\par 29 Nebo ucil je jako moc maje, a ne jako zákoníci.

\chapter{8}

\par 1 A když sstupoval s hory, šli za ním zástupové mnozí.
\par 2 A aj, malomocný prišed, klanel se jemu, rka: Pane, kdybys jen chtel, mužeš mne ocistiti.
\par 3 I vztáh Ježíš ruku, dotekl se ho, rka: Chci, bud cist. A hned ocišteno jest malomocenství jeho.
\par 4 I dí mu Ježíš: Viziž, abys žádnému nepravil. Ale jdi, a ukaž se knezi, a obetuj dar, kterýž prikázal Mojžíš, na svedectví jim.
\par 5 A když vcházel Ježíš do Kafarnaum, pristoupil k nemu setník, prose ho,
\par 6 A rka: Pane, služebník muj leží doma šlakem poražený, velmi se trápe.
\par 7 I dí mu Ježíš: Já prijdu a uzdravím ho.
\par 8 A odpovídaje setník, rekl: Pane, nejsemt hoden, abys všel pod strechu mou, ale toliko rci slovo, a uzdraven bude služebník muj.
\par 9 Nebo i já jsem clovek moci poddaný, maje pod sebou žoldnére, avšak dím-li tomuto: Jdi, tedy jde, a jinému: Prijd, a prijde, a služebníku svému: Ucin toto, a uciní.
\par 10 Tedy uslyšev to Ježíš, podivil se, a jdoucím za sebou rekl: Amen pravím vám, ani v Izraeli tak veliké víry jsem nenalezl.
\par 11 Pravím pak vám, žet prijdou mnozí od východu i od západu, a stoliti budou s Abrahamem, s Izákem a s Jákobem v království nebeském,
\par 12 Ale synové království vyvrženi budou do temností zevnitrních. Tamt bude plác a škripení zubu.
\par 13 I rekl Ježíš setníkovi: Jdiž, a jakžs uveril, stan se tobe. I uzdraven jest služebník jeho v tu hodinu.
\par 14 A prišed Ježíš do domu Petrova, uzrel svegruši jeho, ana leží a má zimnici.
\par 15 I dotekl se ruky její, a hned prestala jí zimnice. I vstala a posluhovala jim.
\par 16 A když byl vecer, privedli k nemu mnohé, kteríž dábelství meli, a on vymítal duchy zlé slovem, a všecky, kteríž se zle meli, uzdravil,
\par 17 Aby se naplnilo povedení skrze Izaiáše proroka, rkoucího: Ont jest vzal na se mdloby naše, a neduhy naše nesl.
\par 18 Vida pak Ježíš zástupy mnohé okolo sebe, kázal preplaviti se na druhou stranu.
\par 19 A pristoupiv jeden zákoník, rekl jemu: Mistre, pujdu za tebou, kamžkoli pujdeš.
\par 20 I dí mu Ježíš: Lišky doupata mají, a ptactvo nebeské hnízda, ale Syn cloveka nemá, kde by hlavu sklonil.
\par 21 Jiný pak z ucedlníku jeho rekl jemu: Pane, dopust mi prve odjíti a pochovati otce mého.
\par 22 Ale Ježíš rekl jemu: Pojd za mnou, a nech, at tam mrtví pochovávají mrtvé své.
\par 23 A když vstoupil na lodí, vstoupili za ním i ucedlníci jeho.
\par 24 A aj, boure veliká stala se jest na mori, takže vlny prikrývaly lodí. On pak spal.
\par 25 A pristoupivše ucedlníci jeho, zbudili jej, rkouce: Pane, zachovej nás, hynemet.
\par 26 I dí jim: Proc se bojíte, ó malé víry? Tedy vstav, primluvil vetrum a mori, i stalo se utišení veliké.
\par 27 Lidé pak divili se, rkouce: Kteraký jest tento, že ho i vetrové i more poslouchají?
\par 28 A když se preplavil na druhou stranu do krajiny Gergezenských, potkali se s ním dva dábelníci z hrobu vyšlí, ukrutní náramne, takže pro ne žádný nemohl tou cestou choditi.
\par 29 A aj, zkrikli, rkouce: Co je nám po tobe, Ježíši, Synu Boží? Prišel jsi sem pred casem trápiti nás.
\par 30 A bylo opodál od nich stádo veliké vepru, pasoucích se.
\par 31 Dáblové pak prosili ho, rkouce: Ponevadž nás vymítáš, dopustiž nám vjíti do toho stáda vepru.
\par 32 I rekl jim: Jdete. A oni vyšedše, vešli do stáda tech vepru. A aj, hnalo se všecko stádo tech vepru s vrchu dolu do more, i ztonuli v vodách.
\par 33 Pastýri pak utekli. A prišedše do mesta, vypravovali to všecko, i o tech dábelnících.
\par 34 A aj, všecko mesto vyšlo v cestu Ježíšovi, a uzrevše ho, prosili, aby šel z koncin jejich.

\chapter{9}

\par 1 A vstoupiv na lodí, preplavil se, a prišel do mesta svého.
\par 2 A aj, prinesli mu šlakem poraženého, ležícího na loži. A videv Ježíš víru jejich, dí šlakem poraženému: Doufej, synu, odpuštenit jsou tobe hríchové tvoji.
\par 3 A aj, nekterí z zákoníku rekli sami v sobe: Tento se rouhá.
\par 4 A videv Ježíš myšlení jejich, rekl: Proc vy myslíte zlé veci v srdcích vašich?
\par 5 Nebo co jest snáze ríci, to-li: Odpouštejí se tobe hríchové? cili ríci: Vstan a chod?
\par 6 Ale abyste vedeli, žet má moc Syn cloveka na zemi odpoušteti hríchy, tedy dí šlakem poraženému: Vstan, vezmi lože své, a jdi do domu svého.
\par 7 Tedy vstal a odšel do domu svého.
\par 8 A vidouce to zástupové, divili se a velebili Boha, kterýž dal takovou moc lidem.
\par 9 A jda odtud Ježíš, uzrel cloveka sedícího na cle, jménem Matouše. I dí mu: Pojd za mnou. A on vstav, šel za ním.
\par 10 I stalo se, když sedel za stolem v domu jeho, a aj, mnozí celní a hríšníci prišedše, stolili s Ježíšem a s ucedlníky jeho.
\par 11 A vidouce to farizeové, rekli ucedlníkum jeho: Proc s celnými a hríšníky jí Mistr váš?
\par 12 Ježíš pak uslyšev to, rekl jim: Nepotrebujít zdraví lékare, ale nemocní.
\par 13 Jdete vy radeji a ucte se, co jest to: Milosrdenství chci a ne obeti. Nebo neprišel jsem volati spravedlivých, ale hríšných ku pokání.
\par 14 Tehdy pristoupili k nemu ucedlníci Janovi, rkouce: Proc my a farizeové postíme se casto, ucedlníci pak tvoji se nepostí?
\par 15 I rekl jim Ježíš: Zdaliž mohou synové Ženichovi rmoutiti se, dokudž s nimi jest Ženich? Ale prijdou dnové, když bude od nich odjat Ženich, a tehdyt se budou postiti.
\par 16 Žádný zajisté neprišívá záplaty sukna nového k rouchu vetchému; nebo ta záplata jeho odtrhla by ješte nejaký díl od roucha, a tak vetší by díra byla.
\par 17 Aniž lejí vína nového do nádob starých; sic jinak rozpuknou se sudové, a víno se vyleje, a sudové se zkazí. Ale víno nové lejí do nových nádob, a bývá obé zachováno.
\par 18 A když on toto k nim mluvil, aj, kníže jedno pristoupilo a klanelo se jemu, rka: Pane dcera má nyní umrela. Ale pojd, vlož na ni ruku svou, a budet živa.
\par 19 A vstav Ježíš, šel za ním, i ucedlníci jeho.
\par 20 (A aj, žena, kteráž nemocí svou trápena byla ode dvanácti let, pristoupivši pozadu, dotkla se podolka roucha jeho.
\par 21 Nebo rekla sama v sobe: Dotknu-li se jen toliko roucha jeho, uzdravena budu.
\par 22 Ježíš pak obrátiv se a uzrev ji, rekl: Doufej, dcero, víra tvá te uzdravila. A zdráva ucinena jest žena od té chvíle.)
\par 23 Prišed pak Ježíš do domu knížete, a uzrev tu trubace a zástup hlucící,
\par 24 Rekl jim: Odejdetež; nebt neumrela devecka, ale spí. I posmívali se jemu.
\par 25 A když byl vyhnán zástup, všed tam, ujal ji za ruku její; i vstala jest devecka.
\par 26 A roznesla se povest ta po vší té zemi.
\par 27 A když šel odtud Ježíš, šli za ním dva slepí, volajíce a rkouce: Smiluj se nad námi, Synu Daviduv.
\par 28 A když všel do domu, pristoupili k nemu ti slepí. I dí jim Ježíš: Veríte-li, že to mohu uciniti? Rekli jemu: Ovšem, Pane.
\par 29 Tedy dotekl se ocí jejich, rka: Podle víry vaší staniž se vám.
\par 30 I otevríny jsou oci jejich. Zapovedel jim pak tuze Ježíš, rka: Viztež, at nižádný o tom nezví.
\par 31 Ale oni vyšedše, rozhlásali jej po vší té zemi.
\par 32 A když oni vycházeli, aj, privedli mu cloveka nemého, majícího dábelství.
\par 33 A když vyvrhl dábelství, mluvil jest nemý. I divili se zástupové, rkouce: Že nikdy se nic takového neukázalo v lidu Izraelském.
\par 34 Farizeové pak pravili: Mocí knížete dábelského vymítá dábly.
\par 35 I obcházel Ježíš všecka mesta i mestecka, uce v školách jejich a káže evangelium království, a uzdravuje všelikou nemoc i všeliký neduh v lidu.
\par 36 A když hledel na zástupy, slitovalo se mu jich, že byli tak opušteni a rozptýleni jako ovce, nemajíce pastýre.
\par 37 Tedy dí ucedlníkum svým: Žen zajisté jest mnohá, ale delníku málo.
\par 38 Protož proste Pána žni, at vypudí delníky na žen svou.

\chapter{10}

\par 1 A svolav k sobe dvanácte ucedlníku svých, dal jim moc nad duchy necistými, aby je vymítali, a aby uzdravovali všelikou nemoc, i všeliký neduh.
\par 2 Dvanácti pak apoštolu jména jsou tato: První Šimon, jenž slove Petr, a Ondrej bratr jeho, Jakub Zebedeuv a Jan bratr jeho,
\par 3 Filip a Bartolomej, Tomáš a Matouš, jenž byl celný, Jakub Alfeuv a Lebbeus, prijmím Thaddeus,
\par 4 Šimon Kananitský a Jidáš Iškariotský, kterýž i zradil ho.
\par 5 Tech dvanácte poslal Ježíš, prikazuje jim, rka: Na cestu pohanu nechodte, a do mest Samaritánských nevcházejte.
\par 6 Ale radeji jdete k ovcem zahynulým z domu Izraelského.
\par 7 Jdouce pak, kažte, rkouce: Že se priblížilo království nebeské.
\par 8 Nemocné uzdravujte, malomocné cistte, mrtvé kreste, dábelství vymítejte; darmo jste vzali, darmo dejte.
\par 9 Neshromaždujte zlata ani stríbra, ani penez v opascích vašich mívejte,
\par 10 Ani mošny na ceste, ani dvou sukní, ani obuvi, ani hulky; hodent jest zajisté delník pokrmu svého.
\par 11 A do kteréhožkoli mesta neb mestecka vešli byste, vzeptejte se, kdo by v nem hodný byl, a tu pobudte, až byste i vyšli odtud.
\par 12 A vcházejíce do domu, pozdravtež ho.
\par 13 A jestližet bude dum ten hodný, pokoj váš prijdiž nan; paklit by nebyl hodný, pokoj váš navratiž se k vám.
\par 14 A kdožkoli neprijal by vás, a neuposlechl by recí vašich, vyjdouce ven z domu neb z mesta toho, vyraztež prach z noh vašich.
\par 15 Amen pravím vám: Lehceji bude zemi Sodomských a Gomorských v den soudný nežli mestu tomu.
\par 16 Aj, já posílám vás jako ovce mezi vlky; protož budte opatrní jako hadové, a sprostní jako holubice.
\par 17 Vystríhejtež se pak lidí; nebt vás vydávati budou do snemu, a v školách svých budou vás bicovati.
\par 18 Ano i pred vladare i pred krále vedeni budete pro mne, na svedectví jim, i tem národum.
\par 19 Kdyžt pak vás vydadí, nebudtež pecliví, kterak aneb co byste mluvili; dánot bude zajisté vám v tu hodinu, co budete míti mluviti.
\par 20 Nebo ne vy jste, jenž mluvíte, ale duch Otce vašeho, jenž mluví v vás.
\par 21 Vydát pak bratr bratra na smrt, i otec syna, a povstanout dítky proti rodicum, a zmordují je.
\par 22 A budete v nenávisti všechnem pro jméno mé, ale kdož setrvá až do konce, tent spasen bude.
\par 23 Když se pak vám budou protiviti v tom meste, utecte do jiného. Amen zajisté pravím vám, nezchodíte mest Izraelských, ažt prijde Syn cloveka.
\par 24 Nenít ucedlník nad mistra, ani služebník nad pána svého.
\par 25 Dostit jest ucedlníku, aby byl jako mistr jeho, a služebník jako pán jeho. Ponevadž jsou hospodáre Belzebubem nazývali, cím pak více domácí jeho?
\par 26 Protož nebojte se jich; nebt není nic skrytého, což by nemelo býti zjeveno, ani co tajného, ješto by nemelo zvedíno býti.
\par 27 Což vám pravím ve tmách, pravte na svetle, a co v uši slyšíte, hlásejte na domích.
\par 28 A nebojte se tech, kteríž zabíjejí telo, ale duše nemohou zabíti; než bojte se radeji toho, kterýž muže i duši i telo zatratiti v pekelném ohni.
\par 29 Zdaliž neprodávají dvou vrabcu za malý peníz? a jeden z nich nepadá na zem bez vule Otce vašeho.
\par 30 Vaši pak i vlasové na hlave všickni secteni jsou.
\par 31 Protož nebojte se, mnohých vrabcu dražší jste vy.
\par 32 Kdožkoli tedy vyzná mne pred lidmi, vyznámt i já jej pred Otcem svým, jenž jest v nebesích.
\par 33 Ale kdož by mne zaprel pred lidmi, zaprímt ho i já pred Otcem svým, kterýž jest v nebesích.
\par 34 Nedomnívejte se, že bych prišel pokoj dáti na zemi. Neprišelt jsem, abych pokoj uvedl, ale mec.
\par 35 Prišelt jsem zajisté, abych rozdelil cloveka proti otci jeho, a dceru proti materi její, a nevestu proti svegruši její.
\par 36 A neprátelé cloveka budou domácí jeho.
\par 37 Kdo miluje otce neb matku více nežli mne, nenít mne hoden; a kdož miluje syna nebo dceru více nežli mne, nenít mne hoden.
\par 38 A kdož nebére kríže svého a nenásleduje mne, nenít mne hoden.
\par 39 Kdož nalezne duši svou, ztratít ji; a kdo by ztratil duši svou pro mne, naleznet ji.
\par 40 Kdož vás prijímá, mnet prijímá; a kdo mne prijímá, prijímát toho, kterýž mne poslal.
\par 41 Kdo prijímá proroka ve jménu proroka, odplatu proroka vezme; a kdož prijímá spravedlivého ve jménu spravedlivého, odplatu spravedlivého vezme.
\par 42 A kdož by koli dal jednomu z techto nejmenších cíši vody studené k nápoji, toliko ve jménu ucedlníka, zajisté pravím vám, neztratít odplaty své.

\chapter{11}

\par 1 I stalo se, když dokonal Ježíš reci své, kteréž mluvil, prikázání dávaje dvanácti ucedlníkum svým, bral se odtud, aby ucil a kázal v mestech jejich.
\par 2 Jan pak v vezení uslyšev o skutcích Kristových, poslal dva z ucedlníku svých,
\par 3 A rekl jemu: Ty-li jsi ten, kterýž prijíti má, cili jiného cekati máme?
\par 4 I odpovídaje Ježíš, rekl jim: Jdouce, zvestujtež Janovi, co slyšíte a vidíte:
\par 5 Slepí vidí, a kulhaví chodí, malomocní se cistí, a hluší slyší, mrtví z mrtvých vstávají, chudým pak evangelium se zvestuje.
\par 6 A blahoslavený jest, kdož se nehorší na mne.
\par 7 A když oni odešli, pocal Ježíš praviti zástupum o Janovi: Co jste vyšli na poušt videti? Zdali trtinu vetrem se klátící?
\par 8 Aneb co jste vyšli videti? Zda cloveka mekkým rouchem odeného? Aj, kteríž se mekkým rouchem odívají, v domích královských jsou.
\par 9 Aneb co jste vyšli videti? Proroka-li? Jiste pravím vám, i více nežli proroka.
\par 10 Tentot jest zajisté, o nemž psáno: Aj, já posílám andela svého pred tvárí tvou, kterýžto pripraví cestu tvou pred tebou.
\par 11 Amen pravím vám, mezi syny ženskými nepovstal vetší nad Jana Krtitele; ale kdo jest menší v království nebeském, jestit vetší nežli on.
\par 12 Ode dnu pak Jana Krtitele až dosavad království nebeské násilí trpí, a ti, kteríž násilí ciní, uchvacujít je.
\par 13 Nebo všickni Proroci i Zákon až do Jana prorokovali.
\par 14 A chcete-li prijmouti: Ont jest Eliáš, kterýž prijíti mel.
\par 15 Kdo má uši k slyšení, slyš.
\par 16 Ale k komu pripodobním pokolení toto? Podobno jest detem, sedícím na ryncích, a kteríž na tovaryše své volají,
\par 17 A ríkají: Pískali jsme vám, a neskákali jste; žalostne jsme naríkali, a neplakali jste.
\par 18 Prišel zajisté Jan, nejeda ani pije, a oni rkou: Dábelství má.
\par 19 Prišel Syn cloveka, jeda a pije, a oni rkou: Aj, clovek žrác a pijan vína, prítel publikánu a hríšníku. Ale ospravedlnena jest moudrost od synu svých.
\par 20 Tehdy pocal primlouvati mestum, v nichžto cineni jsou jeho mnozí divové, že pokání necinili.
\par 21 Beda tobe Korozaim, beda tobe Betsaido. Nebo kdyby v Týru a Sidonu byli cineni divové ti, kteríž jsou cineni v vás, dávno by byli v žíni a v popele pokání cinili.
\par 22 Nýbrž pravím vám, že Týru a Sidonu lehceji bude v den soudný nežli vám.
\par 23 A ty Kafarnaum, kteréž jsi až k nebi vyvýšeno, až do pekla sstrceno budeš. Nebo kdyby v Sodome cineni byli divové ti, kteríž jsou cineni v tobe, bylit by zustali až do dnešního dne.
\par 24 Ano více pravím vám, že zemi Sodomských lehceji bude v den soudný nežli tobe.
\par 25 V ten cas odpovedev Ježíš, rekl: Chválím te, Otce, Pane nebe i zeme, že jsi skryl tyto veci pred moudrými a opatrnými, a zjevil jsi je malickým.
\par 26 Jiste, Otce, že se tak líbilo pred tebou.
\par 27 Všecky veci dány jsou mi od Otce mého, a žádnýt nezná Syna, jediné Otec, aniž Otce kdo zná, jediné Syn, a komuž by chtel Syn zjeviti.
\par 28 Pojdtež ke mne všickni, kteríž pracujete a obtíženi jste, a já vám odpocinutí dám.
\par 29 Vezmete jho mé na se, a ucte se ode mne, nebot jsem tichý a pokorný srdcem, a naleznete odpocinutí dušem vašim.
\par 30 Jho mé zajisté jestit rozkošné, a bríme mé lehké.

\chapter{12}

\par 1 V ten cas šel Ježíš v den svátecní skrze obilí, a ucedlníci jeho lacni jsouce, pocali vymínati klasy a jísti.
\par 2 Farizeové pak vidouce to, rekli jemu: Hle, ucedlníci tvoji ciní to, cehož nesluší ciniti v den svátecní.
\par 3 On pak rekl jim: Co jste nectli, co jest ucinil David, když lacnel, on i ti, kteríž s ním byli?
\par 4 Kterak všel do domu Božího a chleby posvátné jedl, kterýchžto jemu neslušelo jísti, ani tem, kteríž s ním byli, než toliko samým knežím?
\par 5 Anebo zdali jste nectli v Zákone, že kneží ve dny svátecní v chráme svátek ruší, a jsou bez hríchu?
\par 6 Ale pravímt vám, žet vetší jest tuto nežli chrám.
\par 7 Než kdybyste vedeli, co je to: Milosrdenství chci a ne obeti, neodsuzovali byste nevinných.
\par 8 Syn zajisté cloveka jestit pánem i dne svátecního.
\par 9 A poodšed odtud Ježíš, prišel do školy jejich.
\par 10 A aj, byl tu clovek, maje ruku uschlou. I tázali se ho, rkouce: Sluší-li v den svátecní uzdravovati? aby jej obžalovali.
\par 11 On pak dí jim: Který bude z vás clovek, ješto by mel ovci jednu, a kdyby ta upadla do jámy v den svátecní, i zdaliž dosáhna nevytáhne jí?
\par 12 A cím lepší jest clovek nežli ovce? A protož slušít v den svátecní dobre ciniti.
\par 13 Tedy rekl cloveku tomu: Vztáhni ruku svou. I vztáhl, a ucinena jest zdravá jako i druhá.
\par 14 Farizeové pak vyšedše, drželi radu proti nemu, kterak by jej zahladili.
\par 15 A veda to Ježíš, šel odtud. I šli za ním zástupové mnozí, a uzdravil je všecky.
\par 16 A s pohružkou prikázal jim, aby ho nezjevovali,
\par 17 Aby se naplnilo povedení skrze Izaiáše proroka, rkoucího:
\par 18 Aj, služebník muj, kteréhož jsem vyvolil, milý muj, v nemž se dobre zalíbilo duši mé. Položím Ducha svého na nej, a soud národum zvestovati bude.
\par 19 Nebude se vaditi, ani kriceti, ani kdo na ulicích uslyší hlas jeho.
\par 20 Trtiny nalomené nedolomí, a lnu kourícího se neuhasí, až i vypoví soud k vítezství.
\par 21 A ve jménu jeho národové doufati budou.
\par 22 Tedy podán jemu dábelstvím posedlý, slepý a nemý. I uzdravil jej, takže ten byv slepý a nemý, i mluvil i videl.
\par 23 I divili se všickni zástupové a pravili: Není-liž tento Syn Daviduv?
\par 24 Ale farizeové to uslyševše, rekli: Tento nevymítá dáblu než Belzebubem, knížetem dábelským.
\par 25 Ježíš pak znaje myšlení jejich, dí jim: Každé království rozdelené samo v sobe zpustne, a každé mesto neb dum proti sobe rozdelený nestane.
\par 26 A jestližet satan satana vymítá, proti sobe rozdelen jest. Kterak tedy stane království jeho?
\par 27 A vymítám-lit já dábly v Belzebubu, synové vaši v kom vymítají? Protož oni soudcové vaši budou.
\par 28 Paklit já Duchem Božím dábly vymítám, jiste prišlo jest mezi vás království Boží.
\par 29 Aneb kterak kdo muže do domu silného reka vjíti a jeho nádobí pobrati, lec by prve svázal toho silného, a teprvt by dum jeho obloupiti mohl?
\par 30 Kdož není se mnou, proti mne jest; a kdo neshromažduje se mnou, rozptylujet.
\par 31 Protož pravím vám: Všeliký hrích i rouhání bude lidem odpušteno, ale rouhání proti Duchu svatému nebude odpušteno lidem.
\par 32 A kdyby kdo rekl slovo proti Synu cloveka, bude jemu odpušteno, ale kdož by mluvil proti Duchu svatému, nebude jemu odpušteno, ani na tomto svete, ani na budoucím.
\par 33 A protož nebo cinte strom dobrý, a ovoce jeho dobré; anebo cinte strom zlý, a ovoce jeho zlé. Nebot po ovoci strom bývá poznán.
\par 34 Pokolení ještercí, kterakž mužete dobré veci mluviti, jsouce zlí? Nebo z hojnosti srdce ústa mluví.
\par 35 Dobrý clovek z dobrého pokladu srdce vynáší dobré, a zlý clovek ze zlého pokladu vynáší zlé.
\par 36 Ale pravím vám, že z každého slova prázdného, kteréž mluviti budou lidé, vydadí pocet v den soudný.
\par 37 Nebo z slov svých spravedliv budeš ucinen, a z recí tvých budeš odsouzen.
\par 38 Tehdy odpovedeli nekterí z zákoníku a farizeu, rkouce: Mistre, chceme od tebe znamení videti.
\par 39 On pak odpovídaje, dí jim: Pokolení zlé a cizoložné znamení hledá, a znamení nebude jemu dáno, jediné znamení Jonáše proroka.
\par 40 Nebo jakož byl Jonáš v briše velryba tri dni a tri noci, takt bude Syn cloveka v srdci zeme tri dni a tri noci.
\par 41 Muži Ninivitští stanou na soudu s pokolením tímto, a odsoudí je, protože pokání cinili k Jonášovu kázání, a aj, vícet jest nežli Jonáš tuto.
\par 42 Královna od poledne povstane k soudu s pokolením tímto, a odsoudí je; nebo prijela od koncin zeme, aby slyšela moudrost Šalomounovu, a aj, vícet jest tuto nežli Šalomoun.
\par 43 Když pak necistý duch vyjde od cloveka, chodí po místech suchých, hledaje odpocinutí, ale nenalézaje, dí:
\par 44 Navrátím se do domu svého, odkudž jsem vyšel. A prijda, nalezne prázdný, vycištený a ozdobený.
\par 45 Tedy jde a vezme s sebou sedm jiných duchu horších, a vejdouce, prebývají tam, i bývají poslední veci cloveka toho horší nežli první. Takt bude i tomuto zlému pokolení.
\par 46 A když on ješte mluvil k zástupum, aj, matka a bratrí jeho stáli vne, žádajíce s ním promluviti.
\par 47 I rekl jemu jeden: Aj, matka tvá i bratrí tvoji stojí vne, chtíce s tebou mluviti.
\par 48 On pak odpovídaje, rekl tomu, kterýž k nemu byl promluvil: Kdo jest matka má? A kdo jsou bratrí moji?
\par 49 A vztáhna ruku svou na ucedlníky své, i rekl: Aj, matka má i bratrí moji.
\par 50 Nebo kdož by cinil vuli Otce mého nebeského, tent jest bratr muj, i sestra má, i matka má.

\chapter{13}

\par 1 A v ten den vyšed Ježíš z domu, sedl podle more.
\par 2 I sešli se k nemu zástupové mnozí, takže vstoupiv na lodí, sedel, všecken pak zástup stál na brehu.
\par 3 I mluvil jim mnoho v podobenstvích, rka: Aj, vyšel rozsevac, aby rozsíval.
\par 4 A když on rozsíval, nekterá seménka padla podle cesty, a prileteli ptáci, i szobali je.
\par 5 Jiná pak padla na místa skalnatá, kdežto nemela mnoho zeme; a rychle vzešla, protože nemela hlubokosti zeme.
\par 6 Ale když slunce vzešlo, uvadla, a že nemela korene, uschla.
\par 7 Jiná pak padla v trní; i vzrostlo trní, a udusilo je.
\par 8 A jiná padla v zemi dobrou; i vydalo užitek, nekteré stý, jiné šedesátý a jiné tridcátý.
\par 9 Kdo má uši k slyšení, slyš.
\par 10 Tedy pristoupivše ucedlníci, rekli jemu: Proc jim v podobenstvích mluvíš?
\par 11 On pak odpovedev, rekl jim: Nebo vám dáno jest znáti tajemství království nebeského, ale jim není dáno.
\par 12 (Nebo kdož má, bude jemu dáno a rozhojnít se; ale kdož nemá, i to, což má, bude od neho odjato.)
\par 13 Protot v podobenstvích mluvím jim, že vidouce nevidí, a slyšíce neslyší, ani rozumejí.
\par 14 A plní se na nich proroctví Izaiáše, rkoucí: Ušima uslyšíte, ale nesrozumíte; a hledíce, hledeti budete, ale neuzríte.
\par 15 Nebo ztucnelo jest srdce lidu tohoto, a ušima težce slyšeli a oci své zamhourili, aby snad nekdy neuzreli ocima a ušima neslyšeli a srdcem nesrozumeli, a neobrátili se, a já abych jich neuzdravil.
\par 16 Ale oci vaše blahoslavené jsou, že vidí, i uši vaše, že slyší.
\par 17 Amen zajisté pravím vám, že mnozí proroci a spravedliví žádali videti to, což vy vidíte, a nevideli, a slyšeti to, což vy slyšíte, a neslyšeli.
\par 18 Vy tedy slyšte podobenství rozsevace.
\par 19 Každý, kdož slyší slovo království a nerozumí, prichází ten zlý a uchvacuje to, což jest vsáto v srdce jeho. To jest ten, kterýž podle cesty vsát jest.
\par 20 Ale v skalnatou zemi vsátý, ten jest, kterýž slyší slovo, a hned je s radostí prijímá.
\par 21 Než nemá v sobe korene, ale jest casný, a když prichází soužení nebo protivenství pro slovo, hned se horší.
\par 22 Ale mezi trní vsátý, ten jest, kterýž slyší slovo Boží, ale pecování tohoto sveta a oklamání zboží udušuje slovo, i bývá ucineno bez užitku.
\par 23 V dobrou pak zemi vsátý, ten jest, kterýž slyší slovo a rozumí, i ovoce nese a vydává, nekteré zajisté stý, a jiné šedesátý, jiné pak tridcátý.
\par 24 Jiné podobenství predložil jim, rka: Podobno jest království nebeské cloveku, rozsívajícímu dobré semeno na poli svém.
\par 25 Když pak lidé zesnuli, prišel neprítel jeho a nasál koukole mezi pšenici a odšel.
\par 26 A když vzrostla bylina a užitek prinesla, tedy ukázal se i koukol.
\par 27 I pristoupivše služebníci hospodáre toho, rekli jemu: Pane, všaks dobrého semene nasál na poli svém, kdeže se pak vzal koukol?
\par 28 A on rekl jim: Neprítel clovek to ucinil. Služebníci pak rekli mu: Chceš-liž, tedy pujdeme a vytrháme jej?
\par 29 On pak odpovedel: Nikoli, abyste trhajíce koukol, spolu s ním nevytrhali pšenice.
\par 30 Nechte, at obé spolu roste až do žni. A v cas žni dím žencum: Vytrhejte nejprv koukol a svažte jej v snopky k spálení, ale pšenici shromaždte do stodoly mé.
\par 31 Jiné podobenství predložil jim, rka: Podobno jest království nebeské zrnu horcicnému, kteréž vzav clovek, vsál na poli svém.
\par 32 Kteréžto zajisté nejmenší jest mezi všemi semeny, když pak vzroste, vetší jest všech bylin, a bývá strom, takže ptactvo nebeské priletíce, hnízda sobe delají na ratolestech jeho.
\par 33 Jiné podobenství mluvil jim, rka: Podobno jest království nebeské kvasu, kterýž vzavši žena, zadelala ve trech mericích mouky, až by zkysalo všecko.
\par 34 Toto všecko mluvil Ježíš v podobenstvích k zástupum, a bez podobenství nemluvil jim,
\par 35 Aby se naplnilo povedení skrze proroka, rkoucího: Otevru v podobenstvích ústa svá, vypravovati budu skryté veci od založení sveta.
\par 36 Tedy rozpustiv zástupy, šel do domu Ježíš. I pristoupili k nemu ucedlníci jeho, rkouce: Vylož nám podobenství o koukoli toho pole.
\par 37 On pak odpovídaje, rekl jim: Rozsevac dobrého semene jestit Syn cloveka.
\par 38 A pole jest tento svet, dobré pak síme jsout synové království, ale koukol jsou synové toho zlostníka.
\par 39 A neprítel, kterýž jej rozsívá, jestit dábel, ale žen jest skonání sveta, a ženci jsou andelé.
\par 40 Protož jakož vytrhávají koukol a ohnem spalují, takt bude pri skonání sveta tohoto.
\par 41 Pošle Syn cloveka andely své, i vyberout z království jeho všecka pohoršení, i ty, kteríž ciní nepravost,
\par 42 A uvrhout je do peci ohnivé. Tamt bude plác a škripení zubu.
\par 43 A tehdážt spravedliví stkvíti se budou jako slunce v království Otce svého. Kdo má uši k slyšení, slyš.
\par 44 Opet podobno jest království nebeské pokladu skrytému v poli, kterýž nalezna clovek, skrývá, a radostí pro nej odejde a prodá všecko, což má, a koupí pole to.
\par 45 Opet podobno jest království nebeské cloveku kupci, hledajícímu dobrých perel.
\par 46 Kterýž nalezna jednu velmi drahou perlu, odšel a prodal všecko, což mel, a koupil ji.
\par 47 Opet podobno jest království nebeské vrši puštené do more a ze všelikého plodu shromaždující;
\par 48 Kteroužto, když naplnena byla, vytáhše na breh a sedíce, vybírali, což dobrého bylo, v nádoby své, a což bylo zlého, prec zamítali.
\par 49 Takt bude pri skonání sveta. Vyjdou andelé a oddelí zlé z prostredku spravedlivých,
\par 50 A uvrhou je do peci ohnivé. Tamt bude plác a škripení zubu.
\par 51 Potom dí jim Ježíš: Srozumeli-li jste tomuto všemu? Rekli jemu: I ovšem, Pane.
\par 52 On pak rekl jim: Protož každý ucitel umelý v království nebeském podoben jest cloveku hospodári, kterýž vynáší z pokladu svého nové i staré veci.
\par 53 I stalo se, když dokonal Ježíš podobenství tato, bral se odtud.
\par 54 A prišed do vlasti své, ucil je v školách jejich, takže se velmi divili, rkouce: Odkud má tento moudrost tuto a moc tuto?
\par 55 Zdaliž tento není syn tesaruv? a zdaliž matka jeho neslove Maria a bratrí jeho Jakub a Jozes a Šimon a Judas?
\par 56 A sestry jeho zdaliž také všecky u nás nejsou? Odkudž tedy má tyto všecky veci?
\par 57 I zhoršili se na nem. A Ježíš rekl: Není prorok beze cti, než v své vlasti a v domu svém.
\par 58 I neucinil tu mnoho divu, pro neveru jejich.

\chapter{14}

\par 1 V tom case uslyšel Herodes ctvrták povest o Ježíšovi.
\par 2 I rekl služebníkum svým: To jest Jan Krtitel. Ont jest vstal z mrtvých, a protož se divové dejí skrze neho.
\par 3 Nebo Herodes byl jal Jana a svázal jej a dal do žaláre pro Herodiadu manželku Filipa bratra svého.
\par 4 Nebo byl rekl jemu Jan: Nesluší tobe míti jí.
\par 5 A chtev zabíti jej, bál se lidu; nebo za proroka jej meli.
\par 6 Když pak slaven byl den narození Herodesova, tancovala dcera Herodiady uprostred hodovníku, i líbilo se to Herodesovi,
\par 7 Tak že s prísahou zaslíbil jí dáti, zac by ho prosila.
\par 8 A ona jsuci prve navedena od matere své, rekla: Dej mi zde na míse hlavu Jana Krtitele.
\par 9 I zarmoutil se král, ale pro prísahu a pro ty, kteríž spolu s ním stolili, rozkázal jí dáti.
\par 10 A poslav kata, stal Jana v žalári.
\par 11 I prinesena jest hlava jeho na míse, a dána devecce. A ona nesla ji materi své.
\par 12 A prišedše ucedlníci jeho, vzali telo jeho a pochovali je; a šedše, povedeli to Ježíšovi.
\par 13 A uslyšev to Ježíš, plavil se odtud na lodicce na místo pusté soukromí. A uslyševše o tom zástupové, šli za ním pešky z mest.
\par 14 A vyšed Ježíš, uzrel zástup mnohý. I slitovalo mu se jich, a uzdravoval nemocné jejich.
\par 15 A když bylo k vecerou, pristoupili k nemu ucedlníci jeho, rkouce: Pusté jest místo toto, a cas již pominul. Rozpust zástupy, at jdouce do mestecek, nakoupí sobe pokrmu.
\par 16 Ježíš pak rekl jim: Není potrebí odcházeti, dejte vy jim jísti.
\par 17 A oni rkou jemu: Nemáme zde, než pet chlebu a dve rybe.
\par 18 Kterýžto dí jim: Prinestež mi je sem.
\par 19 A rozkázav zástupu posaditi se na tráve a vzav pet chlebu a dve rybe, vzhléd v nebe, požehnal, a lámaje, dal ucedlníkum chleby, a ucedlníci zástupum.
\par 20 I jedli všickni a nasyceni jsou. I sebrali pozustalých drobtu, dvanácte košu plných.
\par 21 Tech pak, kteríž jedli, bylo okolo peti tisícu mužu, krome žen a detí.
\par 22 A ihned prinutil ucedlníky své, aby vstoupili na lodí a predešli jej za more, dokudž by nerozpustil zástupu.
\par 23 A rozpustiv zástupy, vstoupil na horu soukromí, aby se modlil. A když byl vecer, sám byl tam.
\par 24 Lodí pak již byla uprostred more, zmítající se vlnami, nebo byl vítr odporný jim.
\par 25 Pri ctvrtém pak bdení nocním bral se k nim Ježíš, jda po mori.
\par 26 A vidouce jej ucedlníci po mori jdoucího, zarmoutili se, rkouce: Obluda jest. A strachem kriceli.
\par 27 Ale ihned Ježíš promluvil k nim, rka: Doufejtež, ját jsem, nebojte se.
\par 28 I odpovedev Petr, rekl: Pane, jsi-li ty, rozkažiž mi k sobe prijíti po vode.
\par 29 A on rekl: Pojd. A vystoupiv Petr z lodí, šel po vode, aby prišel k Ježíšovi.
\par 30 Ale vida vítr tuhý, bál se. A pocav tonouti, zkrikl, rka: Pane, pomoz mi.
\par 31 A ihned Ježíš vztáh ruku, ujal jej a rekl jemu: Malé víry, procežs pochyboval?
\par 32 A jakž oni vstoupili na lodí, prestal vítr.
\par 33 Ti pak, kteríž na lodí byli, pristoupivše, klaneli se jemu, rkouce: Jiste Syn Boží jsi.
\par 34 A preplavivše se, prišli do zeme Genezaretské.
\par 35 A poznavše jej muži místa toho, rozeslali po vší té krajine vukol, a shromáždili k nemu všecky neduživé.
\par 36 A prosili ho, aby se aspon podolka roucha jeho dotkli. A kterížkoli dotkli se, uzdraveni jsou.

\chapter{15}

\par 1 A v tom pristoupí k Ježíšovi Jeruzalémští zákoníci a farizeové, rkouce:
\par 2 Proc ucedlníci tvoji prestupují ustanovení starších? Nebo neumývají rukou svých, když mají jísti chléb.
\par 3 A on odpovídaje, rekl jim: Procež i vy prestupujete prikázání Boží pro ustanovení vaše?
\par 4 Nebo prikázal Buh, rka: Cti otce svého i matku, a kdož by zlorecil otci neb materi, smrtí at umre.
\par 5 Ale vy pravíte: Kdož by koli rekl otci neb materi: Dar ode mne obetovaný, tobe prospeje, by pak i neuctil otce svého neb matere své, bez viny bude.
\par 6 A takž zrušili jste prikázání Boží pro své ustanovení.
\par 7 Pokrytci, dobre prorokoval o vás Izaiáš, rka:
\par 8 Približuje se ke mne lid tento ústy svými a rty mne ctí, ale srdce jejich daleko jest ode mne.
\par 9 Nadarmot mne ctí, ucíce ucení, jenž jsou prikázání lidská.
\par 10 A svolav zástup, rekl jim: Slyšte a rozumejte.
\par 11 Ne to, což vchází v ústa, poskvrnuje cloveka, ale což z úst pochází, tot poskvrnuje cloveka.
\par 12 Tehdy pristoupivše ucedlníci jeho, rekli mu: Víš-li, že farizeové, slyševše tu rec, zhoršili se?
\par 13 A on odpovídaje, rekl: Všeliké štípení, jehož neštípil Otec muj nebeský, vykoreneno bude.
\par 14 Nechte jich, vudcovét jsou slepí slepých, a povede-li slepý slepého, oba v jámu upadnou.
\par 15 I odpovedev Petr, rekl jemu: Vylož nám to podobenství.
\par 16 Ježíš pak rekl: Ješte i vy bez rozumu jste?
\par 17 Nerozumíte-liž, že všecko, což v ústa vchází, do bricha jde a vypouští se ven?
\par 18 Ale které veci z úst pocházejí, z srdce jdou, a tyt poskvrnují cloveka.
\par 19 Z srdcet zajisté vycházejí zlá myšlení, vraždy, cizoložstva, smilstva, krádeže, krivá svedectví, rouhání.
\par 20 Tyt jsou veci poskvrnující cloveka. Ale neumytýma rukama jísti, tot neposkvrnuje cloveka.
\par 21 A vyšed odtud Ježíš, bral se do krajin Tyrských a Sidonských.
\par 22 A aj, žena Kananejská z koncin tech vyšedši, volala za ním, rkuci: Smiluj se nade mnou, Pane, synu Daviduv. Dceru mou hrozne trápí dábelství.
\par 23 Kterýžto neodpovedel jí slova. I pristoupivše ucedlníci jeho, prosili ho, rkouce: Propust ji, nebot volá za námi.
\par 24 On pak odpovedev, rekl: Nejsem poslán než k ovcem zahynulým z domu Izraelského.
\par 25 Ale ona pristoupivši, klanela se jemu, rkuci: Pane, pomoz mi.
\par 26 On pak odpovedev, rekl: Není slušné vzíti chléb synu a vrci štenatum.
\par 27 A ona rekla: Takt jest, Pane. Avšak štenata jedí drobty, kteríž padají z stolu pánu jejich.
\par 28 Tedy odpovídaje Ježíš, rekl jí: Ó ženo, veliká jest víra tvá. Staniž se tobe, jakž chceš. I uzdravena jest dcera její v tu hodinu.
\par 29 A odšed odtud Ježíš, šel podle more Galilejského, a vstoupiv na horu, posadil se tam.
\par 30 I prišli k nemu zástupové mnozí, majíce s sebou kulhavé, slepé, nemé, polámané a jiné mnohé. I kladli je k nohám Ježíšovým, a on uzdravil je,
\par 31 Takže se zástupové divili, vidouce, ano nemí mluví, polámaní zdraví jsou, kulhaví chodí, slepí vidí. I velebili Boha Izraelského.
\par 32 Ježíš pak svolav ucedlníky své, rekl: Líto mi zástupu, ješto již tri dni trvají se mnou a nemají, co by jedli; a rozpustiti jich lacných nechci, aby nezhynuli na ceste.
\par 33 I rekli mu ucedlníci jeho: I kde bychom vzali tolik chleba na této poušti, abychom takový zástup nasytili?
\par 34 I rekl jim Ježíš: Kolik chlebu máte? A oni rkou: Sedm a málo rybicek.
\par 35 I rozkázal zástupum, aby se posadili na zemi.
\par 36 A vzav tech sedm chlebu a ryby, uciniv díky, lámal a dal ucedlníkum svým, a ucedlníci zástupu.
\par 37 I jedli všickni a nasyceni jsou. A sebrali, což zbylo drobtu, sedm košu plných.
\par 38 Bylo pak tech, kteríž jedli, ctyri tisíce mužu krome žen a detí.
\par 39 A rozpustiv zástupy, vstoupil na lodí. I prišel do krajiny Magdala.

\chapter{16}

\par 1 I pristoupili k nemu farizeové a saduceové, a pokoušejíce, prosili ho, aby jim znamení s nebe ukázal.
\par 2 On pak odpovídaje, rekl jim: Když bývá vecer, ríkáte: Jasno bude, nebo se cervená nebe.
\par 3 A ráno: Dnes bude necas, nebo se cervená zasmušilé nebe. Pokrytci, zpusob zajisté nebe rozsouditi umíte, znamení pak casu nemužete souditi?
\par 4 Národ zlý a cizoložný znamení hledá, ale znamení jemu nebude dáno, než toliko znamení Jonáše proroka. A opustiv je, odšel.
\par 5 A preplavivše se ucedlníci jeho pres more, zapomenuli vzíti chleba.
\par 6 Ježíš pak rekl jim: Hledte a varujte se kvasu farizejského a saducejského.
\par 7 Oni pak rozjímali mezi sebou, rkouce: Nevzali jsme chleba.
\par 8 A znaje to Ježíš, rekl jim: Co to rozjímáte mezi sebou, ó malé víry, že jste chlebu nevzali?
\par 9 Ješte-liž nerozumíte, ani pamatujete na pet chlebu, jimiž nasyceno bylo pet tisícu lidu, a kolik jste košu drobtu sebrali?
\par 10 Ani na sedm chlebu, jimiž nasyceno bylo ctyri tísíce lidí, a kolik jste košu drobtu sebrali?
\par 11 I kterakž pak nerozumíte, že ne o chlebu mluvil jsem vám, prave: Varujte se od kvasu farizejského a saducejského?
\par 12 Tedy srozumeli, že nerekl, aby se varovali od kvasu chleba, ale od ucení farizeu a saduceu.
\par 13 Prišed pak Ježíš do krajin Cesaree Filipovy, otázal se ucedlníku svých, rka: Kým mne praví lidé býti, mne Syna cloveka?
\par 14 A oni rekli: Nekterí Janem Krtitelem, a jiní Eliášem, jiní pak Jeremiášem, aneb jedním z proroku.
\par 15 I dí jim: Vy pak kým mne býti pravíte?
\par 16 I odpovedev Šimon Petr, rekl: Ty jsi Kristus, Syn Boha živého.
\par 17 A odpovídaje Ježíš, rekl mu: Blahoslavený jsi Šimone, synu Jonášuv; nebo telo a krev nezjevilo tobe toho, ale Otec muj, kterýž jest v nebesích.
\par 18 I ját pravím tobe, že jsi ty Petr, a na tét skále vzdelám církev svou, a brány pekelné nepremohou jí.
\par 19 A tobe dám klíce království nebeského. A což bys koli svázal na zemi, budet svázáno i na nebi; a což bys koli rozvázal na zemi, budet rozvázáno i na nebi.
\par 20 Tedy prikázal ucedlníkum svým, aby žádnému nepravili, že by on byl ten Ježíš Kristus.
\par 21 A od té chvíle pocal Ježíš oznamovati ucedlníkum svým, že musí jíti do Jeruzaléma, a mnoho trpeti od starších a predních kneží a od zákonníku, a zabit býti, a tretího dne z mrtvých vstáti.
\par 22 I odved ho Petr na stranu, pocal mu primlouvati, rka: Odstup to od tebe, Pane, nestanet se tobe toho.
\par 23 Kterýžto obrátiv se, rekl Petrovi: Jdiž za mnou, satane, ku pohoršení jsi mi; nebo nechápáš tech vecí, kteréž jsou Boží, ale kteréž jsou lidské.
\par 24 Tedy rekl Ježíš ucedlníkum svým: Chce-li kdo za mnou prijíti, zapriž sebe sám, a vezmi kríž svuj, a následujž mne.
\par 25 Nebo kdož by chtel duši svou zachovati, ztratít ji; kdož by pak ztratil duši svou pro mne, naleznet ji.
\par 26 Nebo co jest platno cloveku, by pak všecken svet získal, a své duši uškodil? Aneb kterou dá clovek odmenu za duši svou?
\par 27 Syn zajisté cloveka prijde v sláve Otce svého s andely svými, a tehdážt odplatí jednomu každému podle skutku jeho.
\par 28 Amen pravím vám: Jsou nekterí z stojících tuto, kteríž neokusí smrti, až i uzrí Syna cloveka, pricházejícího v království svém.

\chapter{17}

\par 1 A po šesti dnech pojal Ježíš Petra a Jakuba a Jana bratra jeho, i uvedl je na horu vysokou soukromí,
\par 2 A promenil se pred nimi. I zastkvela se tvár jeho jako slunce, a roucho jeho ucineno bílé jako svetlo.
\par 3 A aj, ukázali se jim Mojžíš a Eliáš, rozmlouvající s ním.
\par 4 A odpovedev Petr, rekl Ježíšovi: Pane, dobré jest nám tuto býti. Chceš-li, udelejme tuto tri stánky, tobe jeden a Mojžíšovi jeden a Eliášovi jeden.
\par 5 Když pak on ješte mluvil, aj, oblak svetlý zastínil je. A aj, zavznel hlas z oblaku rkoucí: Toto jest ten muj milý Syn, v nemž mi se dobre zalíbilo, toho poslouchejte.
\par 6 To uslyšavše ucedlníci, padli na tvári své a báli se velmi.
\par 7 A pristoupiv Ježíš, dotekl se jich, rka jim: Vstante, nebojte se.
\par 8 A pozdvihše ocí svých, žádného nevideli, než samého Ježíše.
\par 9 Když pak sstupovali s hory, prikázal jim Ježíš, rka: Žádnému nepravte tohoto videní, dokudž by Syn cloveka nevstal z mrtvých.
\par 10 I otázali se ho ucedlníci jeho, rkouce: Což to pak zákoníci praví, že má Eliáš prve prijíti?
\par 11 A Ježíš odpovídaje, rekl jim: Eliáš zajisté prijde prve a napraví všecky veci.
\par 12 Ale pravím vám, že Eliáš již prišel, avšak nepoznali ho, ale ucinili mu, což chteli. Takt i Syn cloveka trpeti bude od nich.
\par 13 Tedy srozumeli ucedlníci, že jim to praví o Janovi Krtiteli.
\par 14 A když prišli k zástupu, pristoupil k nemu clovek jeden, a poklekl pred ním na kolena,
\par 15 A rekl: Pane, smiluj se nad synem mým, nebo námesicník jest, a bídne se trápí. Casto zajisté padá do ohne a castokrát do vody.
\par 16 I privedl jsem ho ucedlníkum tvým, ale nemohli ho uzdraviti.
\par 17 Odpovídaje pak Ježíš, rekl: Ó národe neverný a prevrácený, dokud budu s vámi? Dokudž vás trpeti budu? Privedte jej sem ke mne.
\par 18 I pohrozil jemu Ježíš. I vyšlo od neho dábelství a uzdraven jest mládenec v tu hodinu.
\par 19 Tedy pristoupivše ucedlníci k Ježíšovi soukromí, rekli jemu: Proc jsme my ho nemohli vyvrci?
\par 20 Rekl jim Ježíš: Pro neveru vaši. Amen zajisté pravím vám: Budete-li míti víru, jako jest zrno horcicné, díte hore této: Jdi odsud tam, a pujde, a nebudet vám nic nemožného.
\par 21 Toto pak pokolení nevychází, jediné skrze modlitbu a pust.
\par 22 A když byli v Galileji, rekl jim Ježíš: Syn cloveka bude zrazen v ruce lidí bezbožných.
\par 23 A zabijít jej, a tretího dne z mrtvých vstane. I zarmoutili se náramne.
\par 24 A když prišli do Kafarnaum, pristoupili, kteríž plat vybírali, ku Petrovi a rekli: Což mistr váš nedává platu?
\par 25 A on rekl: Dává. A když všel do domu, predšel jej Ježíš recí, rka: Co se tobe zdá, Šimone? Králové zemští od kterých berou dan anebo plat, od synu-li svých, cili od cizích?
\par 26 Odpovedel jemu Petr: Od cizích. I dí mu Ježíš: Tedy synové jsou svobodní?
\par 27 Ale abychom jich nepohoršili, jda k mori, vrz udici, a tu rybu, kteráž nejprve uvázne, vezmi, a otevra ústa její, nalezneš groš. Ten vezma, dej jim za mne i za sebe.

\chapter{18}

\par 1 V ten cas pristoupili ucedlníci k Ježíšovi, rkouce: Kdo pak jest vetší v království nebeském?
\par 2 A zavolav Ježíš pacholete, postavil je uprostred nich,
\par 3 A rekl: Amen pravím vám: Neobrátíte-li se a nebudete-li jako pacholátka, nikoli nevejdete do království nebeského.
\par 4 Protož kdož by se koli ponížil jako pacholátko toto, tent jest vetší v království nebeském.
\par 5 A kdož by koli prijal pacholátko takové ve jménu mém, mnet prijímá.
\par 6 Kdo by pak pohoršil jednoho z malických techto verících ve mne, lépe by jemu bylo, aby zavešen byl žernov oslicí na hrdlo jeho, a pohrížen byl do hlubokosti morské.
\par 7 Beda svetu pro pohoršení. Ackoli musí to býti, aby pricházela pohoršení, ale však beda cloveku, skrze nehož prichází pohoršení.
\par 8 Protož jestliže ruka tvá anebo noha tvá pohoršuje te, utniž ji a vrz od sebe. Lépe jest tobe do života vjíti kulhavému anebo bezrukému, nežli dve ruce aneb dve noze majícímu uvrženu býti do vecného ohne.
\par 9 A pakli oko tvé pohoršuje tebe, vylup je a vrz od sebe. Lépe jest tobe jednookému do života vjíti, nežli obe oci majícímu uvrženu býti do pekelného ohne.
\par 10 Viztež, abyste nepotupovali ani jednoho z malických techto. Nebot pravím vám, že andelé jejich v nebesích vždycky vidí tvár Otce mého, kterýž v nebesích jest.
\par 11 Nebo prišel Syn cloveka, aby spasil to, což bylo zahynulo.
\par 12 Co se vám zdá? Kdyby nekterý clovek mel sto ovec, a zbloudila by jedna z nich, zdaliž nenechá devadesáti devíti, a jda na hory, nehledá té pobloudilé?
\par 13 A nahodí-lit mu se nalézti ji, amen pravím vám, že se radovati bude nad ní více, než nad devadesáti devíti nepobloudilými.
\par 14 Takt není vule pred Otcem vaším, kterýž jest v nebesích, aby zhynul jeden z malických techto.
\par 15 Zhrešil-li by pak proti tobe bratr tvuj, jdi a potresci ho mezi sebou a jím samým. Uposlechl-li by tebe, získal jsi bratra svého.
\par 16 Jestliže by pak neuposlechl, prijmi k sobe jednoho anebo dva, aby v ústech dvou nebo trí svedku stálo každé slovo.
\par 17 Paklit by jich neuposlechl, povez církvi. Jestliže pak i církve neuposlechne, budiž tobe jako pohan a publikán.
\par 18 Amen pravím vám: Cožkoli svížete na zemi, budet svázáno i na nebi; a cožkoli rozvížete na zemi, budet rozvázáno i na nebi.
\par 19 Opet pravím vám: Jestliže by dva z vás svolili se na zemi o všelikou vec, za kterouž by koli prosili, stanet se jim od Otce mého nebeského.
\par 20 Nebo kdežkoli shromáždí se dva nebo tri ve jménu mém, tut jsem já uprostred nich.
\par 21 Tedy pristoupiv k nemu Petr, rekl: Pane, kolikrát zhreší proti mne bratr muj, a odpustím jemu? Do sedmilikrát?
\par 22 I dí mu Ježíš: Nepravím tobe až do sedmikrát, ale až do sedmdesátikrát sedmkrát.
\par 23 A protož podobno jest království nebeské cloveku králi, kterýž chtel pocet klásti s služebníky svými.
\par 24 A když pocal poctu klásti, podán mu jeden, kterýž byl dlužen deset tisícu hriven.
\par 25 A když nemel cím zaplatiti, kázal jej pán jeho prodati, i ženu jeho i deti i všecko, což mel, a zaplatiti.
\par 26 Tedy padna služebník ten, prosil ho, rka: Pane, poshovej mi, a všeckot zaplatím tobe.
\par 27 I slitovav se pán nad služebníkem tím, propustil ho a dluh jemu odpustil.
\par 28 Vyšed pak služebník ten, nalezl jednoho z spoluslužebníku svých, kterýž mu byl dlužen sto penez, a chopiv se ho, hrdloval se s ním, rka: Zaplat mi, cos dlužen.
\par 29 Tedy padna spoluslužebník ten k nohám jeho, prosil ho, rka: Poshovej mi, a všeckot zaplatím tobe.
\par 30 On pak nechtel, ale odšed, dal jej do žaláre, dokudž by nezaplatil dluhu.
\par 31 Tedy vidouce spoluslužebníci, co se dálo, zarmoutili se velmi; a šedše, povedeli pánu svému všecko, co se bylo stalo.
\par 32 Tehdy povolav ho pán jeho, dí mu: Služebníce zlý, všecken ten tvuj dluh odpustil jsem tobe, nebs mne prosil.
\par 33 Zdaližs i ty nemel se smilovati nad spoluslužebníkem svým, jako i já smiloval jsem se nad tebou?
\par 34 I rozhnevav se pán jeho, dal jej katum, dokudž by nezaplatil všeho, což mu byl dlužen.
\par 35 Takt i Otec muj nebeský uciní vám, jestliže neodpustíte jeden každý bratru svému z srdcí vašich jejich provinení.

\chapter{19}

\par 1 I stalo se, když dokonal Ježíš reci tyto, bral se z Galilee, a prišel do koncin Judských za Jordán.
\par 2 I šli za ním zástupové mnozí, a uzdravil je tam.
\par 3 I pristoupili k nemu farizeové, pokoušejíce ho a rkouce jemu: Sluší-li cloveku propustiti ženu svou z kterékoli príciny?
\par 4 On pak odpovídaje, rekl jim: Což jste nectli, že ten, jenž stvoril cloveka s pocátku, muže a ženu ucinil je?
\par 5 A rekl: Protož opustí clovek otce i matku, a pripojí se k manželce své, i budou dva jedno telo.
\par 6 A tak již nejsou dva, ale jedno telo. A protož, což jest Buh spojil, clovek nerozlucuj.
\par 7 Rekli jemu: Procež tedy Mojžíš rozkázal dáti list zapuzení a propustiti jí?
\par 8 Dí jim: Mojžíš pro tvrdost srdce vašeho dopustil vám propoušteti manželky vaše, ale s pocátku nebylo tak.
\par 9 Protož pravím vám: Že kdožkoli propustil by manželku svou, (lec pro smilství) a jinou pojme, cizoloží, a kdož propuštenou pojme, také cizoloží.
\par 10 Rekli jemu ucedlníci jeho: Ponevadž jest taková pre s manželkou, není dobré ženiti se.
\par 11 On pak rekl jim: Ne všicknit chápají slova toho, ale ti toliko, jimž jest dáno.
\par 12 Jsout zajisté panicové, kteríž se tak z života matky zrodili; a jsout panicové, kteríž ucineni jsou od lidí; a jsou panicové, kteríž se sami v panictví oddali pro království nebeské. Kdo muže pochopiti, pochop.
\par 13 Tehdy prineseny jsou k nemu dítky, aby na ne ruce vzkládal a modlil se za ne. Ucedlníci pak primlouvali jim.
\par 14 Ale Ježíš rekl jim: Nechte dítek a nebrante jim jíti ke mne; nebo takovýcht jest království nebeské.
\par 15 A po vzkládání na ne rukou odebral se odtud.
\par 16 A aj, jeden pristoupiv, rekl jemu: Mistre dobrý, co dobrého budu ciniti, abych mel život vecný?
\par 17 A on rekl jemu: Co mne nazýváš dobrým? Žádný není dobrý, než jediný, totiž Buh. Chceš-li pak vjíti do života, ostríhej prikázání.
\par 18 Dí jemu: Kterých? A Ježíš rekl mu: Nezabiješ, nezcizoložíš, nepokradeš, nepromluvíš krivého svedectví,
\par 19 Cti otce svého i matku, a milovati budeš bližního svého jako sebe samého.
\par 20 Dí jemu mládenec: Všeho toho ostríhal jsem od své mladosti. Cehož mi se ješte nedostává?
\par 21 Rekl mu Ježíš: Chceš-li dokonalým býti, jdiž a prodej statek svuj, a rozdej chudým, a budeš míti poklad v nebi, a pojd, následuj mne.
\par 22 Uslyšev pak mládenec tu rec, odšel, smuten jsa; nebo mel statku mnoho.
\par 23 Tedy Ježíš rekl ucedlníkum svým: Amen pravím vám, že bohatý nesnadne vejde do království nebeského.
\par 24 A opet pravím vám: Snázet jest velbloudu skrze ucho jehly projíti, nežli bohatému vjíti do království Božího.
\par 25 A uslyšavše to ucedlníci jeho, i užasli se velmi, rkouce: I kdož tedy muže spasen býti?
\par 26 A pohledev na ne Ježíš, rekl jim: U lidít jest to nemožné, ale u Boha všecko jest možné.
\par 27 Tehdy odpovedev Petr, rekl mu: Aj, my opustili jsme všecky veci, a šli jsme za tebou. Což pak nám bude dáno za to?
\par 28 A Ježíš rekl jim: Amen pravím vám, že vy, kteríž jste následovali mne, v druhém narození, když se posadí Syn cloveka na trunu velebnosti své, sednete i vy na dvanácti stolicích, soudíce dvanáctero pokolení Izraelské.
\par 29 A každý, kdož opustil by domy, nebo bratry, neb sestry, neb otce, neb matku, nebo manželku, nebo syny, nebo pole pro jméno mé, stokrát více vezme, a život vecný dedicne obdrží.
\par 30 Mnozí pak první budou poslední, a poslední první.

\chapter{20}

\par 1 Nebo podobno jest království nebeské cloveku hospodári, kterýž vyšel na úsvite, aby najal delníky na vinici svou.
\par 2 Smluviv pak s delníky z peníze denního, odeslal je na vinici svou.
\par 3 A vyšed okolo hodiny tretí, uzrel jiné, ani stojí na trhu zahálejíce.
\par 4 I rekl jim: Jdetež i vy na vinici mou, a co bude spravedlivého, dám vám.
\par 5 A oni šli. Opet vyšed pri šesté a deváté hodine, ucinil též.
\par 6 Pri jedenácté pak hodine vyšed, nalezl jiné, ani stojí zahálejíce. I rekl jim: Procež tu stojíte, celý den zahálejíce?
\par 7 Rkou jemu: Nebo nižádný nás nenajal. Dí jim: Jdetež i vy na vinici mou, a což by bylo spravedlivého, vezmete.
\par 8 Vecer pak rekl pán vinice šafári svému: Zavolej delníku a zaplat jim, pocna od posledních až do prvních.
\par 9 A prišedše ti, kteríž byli pri jedenácté hodine najati, vzali jeden každý po penízi.
\par 10 Prišedše pak první, domnívali se, že by více meli vzíti; ale vzali i oni jeden každý po penízi.
\par 11 A vzavše, reptali proti hospodári, rkouce:
\par 12 Tito poslední jednu hodinu toliko delali, a rovné jsi je nám ucinil, kteríž jsme nesli bríme dne i horko.
\par 13 On pak odpovídaje jednomu z nich, rekl: Príteli, neciním tobe krivdy; však jsi z peníze denního smluvil se mnou.
\par 14 Vezmiž, což tvého jest, a jdi predce. Já pak chci tomuto poslednímu dáti jako i tobe.
\par 15 Zdaliž mi nesluší v mém uciniti, což chci? Cili oko tvé nešlechetné jest, že já dobrý jsem?
\par 16 Takt budou poslední první, a první poslední; nebo mnoho jest povolaných, ale málo vyvolených.
\par 17 A vstupuje Ježíš do Jeruzaléma, pojal dvanácte ucedlníku svých soukromí na ceste, i rekl jim:
\par 18 Aj, vstupujeme do Jeruzaléma, a Syn cloveka vydán bude predním knežím a zákoníkum, a odsoudí ho na smrt.
\par 19 A vydadít jej pohanum ku posmívání a k bicování a ukrižování; a tretího dne z mrtvých vstane.
\par 20 Tedy pristoupila k nemu matka synu Zebedeových s syny svými, klanející se a prosecí neco od neho.
\par 21 Kterýžto rekl jí: Co chceš? Rekla jemu: Rci, at tito dva synové moji sednou, jeden na pravici tvé a druhý na levici, v království tvém.
\par 22 Odpovídaje pak Ježíš, rekl: Nevíte, zac prosíte. Mužete-li píti kalich, kterýž já píti budu, a krtem, jímž já se krtím, krteni býti? Rekli jemu: Mužeme.
\par 23 Dí jim: Kalich zajisté muj píti budete, a krtem, jímž já se krtím, pokrteni budete, ale sedeti na pravici mé a na levici mé, nenít mé dáti vám, ale dáno bude tem, kterýmž pripraveno jest od Otce mého.
\par 24 A uslyšavše to deset ucedlníku Páne, rozhnevali se na ty dva bratry.
\par 25 Ale Ježíš svolav je, rekl: Víte, že knížata národu panují nad svými, a kteríž velicí jsou, moci užívají nad nimi.
\par 26 Ne tak bude mezi vámi; ale kdožkoli chtel by mezi vámi býti velikým, budiž služebník váš.
\par 27 A kdož by koli mezi vámi chtel býti první, budiž váš služebník;
\par 28 Jako i Syn cloveka neprišel, aby jemu slouženo bylo, ale aby on sloužil a aby dal život svuj na vykoupení za mnohé.
\par 29 A když vycházeli z Jericho, šel za ním zástup veliký.
\par 30 A aj, dva slepí sedící u cesty, uslyševše, že by Ježíš tudy šel, zvolali, rkouce: Smiluj se nad námi, Pane, synu Daviduv.
\par 31 Zástup pak primlouval jim, aby mlceli. Oni pak více volali, rkouce: Smiluj se nad námi, Pane, synu Daviduv.
\par 32 I zastaviv se Ježíš, zavolal jich, a rekl: Co chcete, abych vám ucinil?
\par 33 Rkou jemu: Pane, at se otevrou oci naše.
\par 34 I slitovav se nad nimi Ježíš, dotekl se ocí jejich, a ihned prohlédly oci jejich. A oni šli za ním.

\chapter{21}

\par 1 A když se priblížili k Jeruzalému, a prišli do Betfage k hore Olivetské, tedy Ježíš poslal dva ucedlníky své,
\par 2 Rka jim: Jdetež do mestecka, kteréž proti vám jest, a hned naleznete oslici privázanou a oslátko s ní. Odvežtež je a privedte ke mne.
\par 3 A rekl-lit by kdo co vám, rcete, že Pán jich potrebuje, a hnedt propustí je.
\par 4 Toto se pak všecko stalo, aby se naplnilo povedení skrze proroka, rkoucího:
\par 5 Povezte dceri Sionské: Aj, král tvuj bére se tobe tichý, a sede na oslici, a na oslátku té oslice jhu podrobené.
\par 6 I jdouce ucedlníci, a ucinivše tak, jakož jim prikázal Ježíš,
\par 7 Privedli oslici i oslátko, a vložili na ne roucha svá, a jej navrchu posadili.
\par 8 Mnohý pak zástup stlali roucha svá na ceste, jiní pak ratolesti z dríví sekali a metali na cestu.
\par 9 A zástupové, kteríž napred šli, i ti, kteríž nazad byli, volali, rkouce: Aj syn Daviduv, Spasitel. Požehnaný, jenž se bére ve jménu Páne; spasiž nás ty, kterýž jsi na výsostech.
\par 10 A když vjel do Jeruzaléma, zbourilo se všecko mesto, rkouce: I kdo jest tento?
\par 11 Zástupové pak pravili: Toto jest ten Ježíš, prorok od Nazarétu Galilejského.
\par 12 I všel Ježíš do chrámu Božího, a vymítal všecky prodavace a kupce z chrámu, a stoly penezomencu a stolice prodávajících holubice prevracel,
\par 13 A rekl jim: Psánot jest: Dum muj dum modlitby slouti bude, ale vy ucinili jste jej peleší lotrovskou.
\par 14 I pristoupili k nemu slepí a kulhaví v chráme, i uzdravil je.
\par 15 Vidouce pak prední kneží a zákoníci divy, kteréž cinil, a dítky, any volají v chráme a praví : Aj syn Daviduv, Spasitel, rozhnevali se.
\par 16 A rekli jemu: Slyšíš-liž, co tito praví? Ježíš pak rekl jim: I ovšem. Nikdá-liž jste nectli, že z úst nemluvnátek a tech, jenž prsí požívají, dokonal jsi chválu?
\par 17 A opustiv je, šel ven z mesta do Betany a tu zustal.
\par 18 Ráno pak navracuje se do mesta, zlacnel.
\par 19 A vida jeden fíkový strom podle cesty, šel k nemu, a nic na nem nenalezl, než listí toliko. I dí tomu stromu: Nikdy více nerod se z tebe ovoce na veky. I usechl jest hned fík ten.
\par 20 A vidouce to ucedlníci, divili se, rkouce: Kterak jest ihned usechl fík ten!
\par 21 I odpovedev Ježíš, rekl jim: Amen pravím vám: Budete-li míti víru, a nebudete-li pochybovati, netoliko to, co se stalo fíkovému drevu, uciníte, a kdybyste i této hore rekli: Zdvihni se a vrz sebou do more, stanet se.
\par 22 A všecko, zac byste koli prosili na modlitbe, veríce, vezmete.
\par 23 A když prišel do chrámu, pristoupili k nemu prední kneží a starší lidu, když ucil, rkouce: Jakou mocí tyto veci ciníš? A kdo jest tobe tu moc dal?
\par 24 Odpovídaje pak Ježíš, rekl jim: Otížit se i já vás na jednu vec, kterouž povíte-li mi, i já vám povím, jakou mocí tyto veci ciním.
\par 25 Krest Januv odkud jest byl? S nebe-li, cili z lidí? A oni rozvažovali mezi sebou, rkouce: Díme-li: S nebe, dít nám: Proc jste pak neverili jemu?
\par 26 Pakli díme: Z lidí, bojíme se zástupu. Nebo všickni meli Jana za proroka.
\par 27 I odpovídajíce Ježíšovi, rekli: Nevíme. Rekl jim i on: Aniž já vám povím, jakou mocí tyto veci ciním.
\par 28 Ale co se vám zdá? Clovek jeden mel dva syny. A pristoupiv k prvnímu, rekl: Synu, jdi na vinici mou dnes a delej.
\par 29 A on odpovedev, rekl: Nechci. A potom usmysliv sobe, šel.
\par 30 I pristoupiv k druhému, rekl jemu též. A on odpovedev, rekl: Jdu, pane. A nešel.
\par 31 Který z tech dvou ucinil vuli otcovu? Rekli jemu: První. Dí jim Ježíš: Amen pravím vám, že publikáni a nevestky predcházejí vás do království Božího.
\par 32 Nebo prišel k vám Jan cestou spravedlnosti, a neuverili jste mu, ale publikáni a nevestky uverili jemu. Vy pak videvše to, aniž jste potom usmyslili sobe, abyste verili jemu.
\par 33 Jiné podobenství slyšte: Byl jeden hospodár, kterýž vzdelal vinici, a opletl ji plotem, a vkopal v ní pres, a ustavel veži, i pronajal ji vinarum, a odšel pryc pres pole.
\par 34 A když se priblížil cas ovoce, poslal služebníky své k vinarum, aby vzali užitky její.
\par 35 Vinari pak zjímavše služebníky jeho, jiného zmrskali, jiného zabili, a jiného ukamenovali.
\par 36 Opet poslal jiných služebníku více nežli prve. I ucinili jim též.
\par 37 Naposledy pak poslal k nim syna svého, rka: Ostýchati se budou syna mého.
\par 38 Vinari pak uzrevše syna jeho, rekli mezi sebou: Tentot jest dedic; pojdte, zabijme jej, a uvažme se v dedictví jeho.
\par 39 I chytivše ho, vyvrhli jej ven z vinice a zabili.
\par 40 Protož když prijde pán vinice, co uciní vinarum tem?
\par 41 Rekli jemu: Zlé zle zatratí, a vinici svou pronajme jiným vinarum, kteríž budou vydávati jemu užitek casy svými.
\par 42 Rekl jim Ježíš: Nikdy-li jste nectli v Písmích: Kámen, kterýž jsou zavrhli delníci, ten ucinen jest v hlavu úhlovou? Ode Pána stalo se toto, a jest divné pred ocima našima.
\par 43 Protož pravím vám, že bude odjato od vás království Boží, a bude dáno lidu cinícímu užitky jeho.
\par 44 A kdož by padl na ten kámen, rozrazít se; a na kohož upadne, setret jej.
\par 45 A slyšavše prední kneží podobenství jeho, porozumeli, že by o nich mluvil.
\par 46 I hledajíce ho jíti, báli se zástupu; neb ho meli za proroka.

\chapter{22}

\par 1 I odpovídaje Ježíš, mluvil jim opet v podobenstvích, rka:
\par 2 Podobno jest království nebeské cloveku králi, kterýž ucinil svadbu synu svému.
\par 3 I poslal služebníky své, aby povolali pozvaných na svadbu; a oni nechteli prijíti.
\par 4 Opet poslal jiné služebníky, rka: Povezte pozvaným: Aj, obed muj pripravil jsem, volové moji a krmný dobytek zbit jest, a všecko hotovo. Pojdtež na svadbu.
\par 5 Ale oni nedbavše na to, odešli, jiný do vsi své a jiný po kupectví svém.
\par 6 Jiní pak zjímavše služebníky jeho a posmech jim ucinivše, zmordovali.
\par 7 A uslyšav to král, rozhneval se; a poslav vojska svá, zhubil vražedníky ty a mesto jejich zapálil.
\par 8 Tedy rekl služebníkum svým: Svadba zajisté hotova jest, ale ti, kteríž pozváni byli, nebyli hodni.
\par 9 Protož jdete na rozcestí, a kteréžkoli naleznete, zovtež na svadbu.
\par 10 I vyšedše služebníci ti na cesty, shromáždili všecky, kteréžkoli nalezli, zlé i dobré. A naplnena jest svadba hodovníky.
\par 11 Tedy všed král, aby pohledel na hodovníky, uzrel tam cloveka neodeného rouchem svadebním.
\par 12 I rekl jemu: Príteli, kteraks ty sem všel, nemaje roucha svadebního? A on onemel.
\par 13 Tedy rekl král služebníkum: Svížíce ruce jeho i nohy, vezmete ho, a uvrztež jej do temností zevnitrních. Tamt bude plác a škripení zubu.
\par 14 Nebo mnoho jest povolaných, ale málo vyvolených.
\par 15 Tedy odšedše farizeové, radili se, jak by polapili jej v reci.
\par 16 I poslali k nemu ucedlníky své s herodiány, rkouce: Mistre, víme, že pravdomluvný jsi a ceste Boží v pravde ucíš a nedbáš na žádného; nebo nepatríš na osobu lidskou.
\par 17 Protož povez nám, co se tobe zdá: Sluší-li dan dáti císari, cili nic?
\par 18 Znaje pak Ježíš zlost jejich, rekl: Co mne pokoušíte, pokrytci?
\par 19 Ukažte mi peníz dane. A oni podali mu peníze.
\par 20 I rekl jim: Cí jest tento obraz a svrchu napsání?
\par 21 Rekli mu: Císaruv. Tedy dí jim: Dejtež, co jest císarova, císari, a co jest Božího, Bohu.
\par 22 To uslyšavše, divili se, a opustivše jej, odešli.
\par 23 V ten den prišli k nemu saduceové, kteríž praví, že není z mrtvých vstání. I otázali se ho,
\par 24 Rkouce: Mistre, Mojžíš povedel: Umrel-li by kdo, nemaje detí, aby bratr jeho právem švagrovství pojal ženu jeho a vzbudil síme bratru svému.
\par 25 I bylo u nás sedm bratru. První pojav ženu, umrel, a nemaje semene, zustavil ženu svou bratru svému.
\par 26 Takž podobne i druhý, i tretí, až do sedmého.
\par 27 Nejposléze pak po všech umrela i žena.
\par 28 Protož pri vzkríšení kterého z tech sedmi bude žena? Nebo všickni ji meli.
\par 29 I odpovedev Ježíš, rekl jim: Bloudíte, neznajíce Písem ani moci Boží.
\par 30 Však pri vzkríšení ani se nebudou ženiti ani vdávati, ale budou jako andelé Boží v nebi.
\par 31 O vzkríšení pak mrtvých zdaliž jste nectli, co jest vám povedíno od Boha, kterýž takto dí:
\par 32 Já jsem Buh Abrahamuv a Buh Izákuv a Buh Jákobuv; a Buht není Buh mrtvých, ale živých.
\par 33 A slyševše to zástupové, divili se ucení jeho.
\par 34 Farizeové pak uslyšavše, že by k mlcení privedl saducejské, sešli se v jedno.
\par 35 I otázal se ho jeden z nich zákoník nejaký, pokoušeje ho, a rka:
\par 36 Mistre, které jest prikázání veliké v Zákone?
\par 37 I rekl mu Ježíš: Milovati budeš Pána Boha svého z celého srdce svého a ze vší duše své a ze vší mysli své.
\par 38 To jest prední a veliké prikázání.
\par 39 Druhé pak jest podobné tomu: Milovati budeš bližního svého jako sebe samého.
\par 40 Na tech dvou prikázáních všecken Zákon záleží i Proroci.
\par 41 A když se sešli farizeové, otázal se jich Ježíš,
\par 42 Rka: Co se vám zdá o Kristu? Cí jest syn? Rkou jemu: Daviduv.
\par 43 Dí jim: Kterakž pak David v Duchu nazývá ho Pánem, rka:
\par 44 Rekl Pán Pánu mému: Sed na pravici mé, dokavadž nepodložím neprátel tvých, aby byli za podnože noh tvých?
\par 45 Ponevadž tedy David Pánem ho nazývá, i kterakž syn jeho jest?
\par 46 A nižádný nemohl jemu odpovedíti slova, aniž se odvážil kdo více od toho dne jeho se nac tázati.

\chapter{23}

\par 1 Tedy Ježíš mluvil zástupum a ucedlníkum svým,
\par 2 Rka: Na stolici Mojžíšove posadili se zákoníci a farizeové.
\par 3 Protož všecko, což by koli rozkázali vám zachovávati, zachovávejte a cinte, ale podle skutku jejich necinte; nebot praví, a neciní.
\par 4 Svazujít zajisté bremena težká a nesnesitelná, a vzkládají na ramena lidská, ale prstem svým nechtí jimi ani pohnouti.
\par 5 A všeckyt ty své skutky ciní, aby byli vidíni od lidí. Rozširují zajisté nápisy své a veliké delají podolky pláštu svých,
\par 6 A milují prední místa na vecerích, a první stolice v školách,
\par 7 A pozdravování na trhu, a aby byli nazýváni od lidí: Mistri, mistri.
\par 8 Ale vy nebývejte nazýváni mistri; nebo jeden jest Mistr váš, totiž Kristus, vy pak všickni bratrí jste.
\par 9 A otce nenazývejte sobe na zemi; nebo jeden jest Otec váš, kterýž jest v nebesích.
\par 10 Ani se nazývejte vudcové; nebo jeden jest vudce váš Kristus.
\par 11 Ale kdo z vás vetší jest, budet služebníkem vaším.
\par 12 Nebo kdož by se sám povyšoval, bude ponížen; a kdož by se ponížil, bude povýšen.
\par 13 Ale beda vám, zákoníci a farizeové pokrytci, že zavíráte království nebeské pred lidmi; nebo sami tam nevcházíte, ani tem, jenž by vjíti chteli, vcházeti dopouštíte.
\par 14 Beda vám, zákoníci a farizeové pokrytci, že zžíráte domy vdovské, za prícinou dlouhého modlení; protož težší soud ponesete.
\par 15 Beda vám, zákoníci a farizeové pokrytci, že obcházíte more i zemi, abyste ucinili jednoho novoverce, a když bude ucinen, uciníte jej syna zatracení, dvakrát více, nežli jste sami.
\par 16 Beda vám, vudcové slepí, kteríž ríkáte: Prisáhl-li by kdo skrze chrám, to nic není; ale kdo by prisáhl skrze zlato chrámové, povinent jest prísaze dosti ciniti.
\par 17 Blázni a slepci; nebo co jest vetšího, zlato-li, cili chrám, kterýž posvecuje zlata?
\par 18 A prisáhl-li by kdo skrze oltár, nic není; ale kdo by prisáhl skrze ten dar, kterýž jest na nem, povinen jest.
\par 19 Blázni a slepci, i co jest vetšího, dar-li, cili oltár, kterýž posvecuje daru?
\par 20 A protož kdokoli prisahá skrze oltár, prisahá skrze nej, i skrze to všecko, což na nem jest .
\par 21 A kdož prisahá skrze chrám, prisahá skrze nej, i skrze toho, kterýž prebývá v nem.
\par 22 A kdož prisahá skrze nebe, prisahá skrze trun Boží, i skrze toho, kterýž na nem sedí.
\par 23 Beda vám, zákoníci a farizeové pokrytci, že dáváte desátky z máty a z kopru a z kmínu, a opouštíte to, což težšího jest v Zákone, totiž soud a milosrdenství a vernost. Tyto veci meli jste ciniti a onech neopoušteti.
\par 24 Vudcové slepí, kteríž cedíte komára, velblouda pak požíráte.
\par 25 Beda vám, zákoníci a farizeové pokrytci, že cistíte po vrchu konvice a mísy, a vnitr plno jest loupeže a nestredmosti.
\par 26 Farizee slepce, vycist prve to, což vnitr jest v konvi a v míse, aby i to, což jest zevnitr, bylo cisto.
\par 27 Beda vám, zákoníci a farizeové pokrytci, nebo jste se pripodobnili hrobum zbíleným, kteríž ac se zdadí zevnitr krásní, ale vnitr jsou plní kostí umrlcích i vší necistoty.
\par 28 Tak i vy zevnitr zajisté zdáte se lidem spravedliví, ale vnitr plní jste pokrytství a nepravosti.
\par 29 Beda vám, zákoníci a farizeové pokrytci, nebot vzdeláváte hroby proroku a ozdobujete hroby spravedlivých,
\par 30 A ríkáte: Kdybychom byli za dnu otcu našich, nebyli bychom úcastníci jejich ve krvi proroku.
\par 31 Protož osvedcujete sami proti sobe, že jste synové tech, kteríž proroky zmordovali.
\par 32 I vy také naplnte míru otcu svých.
\par 33 Hadové, pléme ještercí, i jakž byste ušli odsudku do pekelného ohne?
\par 34 Protož aj, já posílám k vám proroky a moudré a ucitele, a vy z tech nekteré zmordujete a ukrižujete, a nekteré z nich bicovati budete v školách vašich, a budete je honiti z mesta do mesta,
\par 35 Aby prišla na vás všeliká krev spravedlivá, vylitá na zemi, od krve Abele spravedlivého, až do krve Zachariáše syna Barachiášova, kteréhož jste zabili mezi chrámem a oltárem.
\par 36 Amen pravím vám: Prijdou tyto všecky veci na pokolení toto.
\par 37 Jeruzaléme, Jeruzaléme, mordéri proroku, a kterýž kamenuješ ty, jenž byli k tobe posíláni, kolikrát jsem chtel shromážditi dítky tvé, tak jako slepice shromažduje kurátka svá pod krídla, a nechteli jste.
\par 38 Aj, zanechávát se vám dum váš pustý.
\par 39 Nebot pravím vám, že mne již více nikoli neuzríte od této chvíle, až i díte: Požehnaný, jenž se bére ve jménu Páne.

\chapter{24}

\par 1 A vyšed Ježíš, bral se z chrámu; i pristoupili ucedlníci jeho, aby ukázali jemu stavení chrámové.
\par 2 Ježíš pak rekl jim: Vidíte-liž tyto všecky veci? Amen pravím vám: Nebude zustaven tuto kámen na kameni, kterýž by nebyl zboren.
\par 3 A když se posadil na hore Olivetské, pristoupili k nemu ucedlníci jeho soukromí, rkouce: Povez nám, kdy to bude, a která znamení budou príchodu tvého a skonání sveta?
\par 4 I odpovedev Ježíš, rekl jim: Vizte, aby vás žádný nesvedl.
\par 5 Nebo mnozí prijdou ve jménu mém, rkouce: Ját jsem Kristus, a svedout mnohé.
\par 6 Budete slyšeti zajisté boje a povesti boju. Hledtež, abyste se nekormoutili; nebo musí to všecko býti; ale ne ihned bude konec.
\par 7 Nebo povstane národ proti národu a království proti království, a budou morové a hladové a zemetresení po místech.
\par 8 Ale tyto všecky veci jsou pocátkové bolestí.
\par 9 A tehdy vy budete souženi, a budou vás mordovati, a budete v nenávisti u všech národu pro jméno mé.
\par 10 A tehdyt se zhorší mnozí a vespolek se budou zrazovati a nenávideti.
\par 11 A mnozí falešní proroci povstanou, a svedou mnohé.
\par 12 A že rozmnožena bude nepravost, ustydnet láska mnohých.
\par 13 Ale kdož by setrval až do konce, tent spasen bude.
\par 14 A budet kázáno toto evangelium království po všem svete, na svedectví všem národum, a tehdážt prijde skonání.
\par 15 Protož když uzríte ohavnost zpuštení, predpovedenou od Daniele proroka, ana stojí na míste svatém, (kdo cte, rozumej,)
\par 16 Tehdáž ti, kteríž by byli v Judstvu, necht utekou k horám.
\par 17 A kdo na streše, nesstupuj dolu, aby neco vzal z domu svého.
\par 18 A kdo na poli, nevracuj se zase, aby vzal roucha svá.
\par 19 Beda pak tehotným a tem, kteréž kojí, v tech dnech.
\par 20 Protož modlte se, aby utíkání vaše nebylo v zime anebo v svátek.
\par 21 Nebo bude tehdáž soužení veliké, jakéž nebylo od pocátku sveta až dosavad, aniž kdy potom bude.
\par 22 A byt nebyli ukráceni dnové ti, nebyl by spasen nižádný clovek. Ale pro vyvolené ukráceni budou dnové ti.
\par 23 Tehdy rekl-li by vám kdo: Aj, tutot jest Kristus, anebo tamto, neverte.
\par 24 Nebo povstanou falešní Kristové a falešní proroci, a ciniti budou divy veliké a zázraky, tak aby v blud uvedli, (by možné bylo,) také i vyvolené.
\par 25 Aj, predpovedel jsem vám.
\par 26 Protož reknou-lit vám: Aj, na poušti jest, nevycházejte. Aj, v skrýších, neverte.
\par 27 Neb jakož blesk vychází od východu slunce, a ukazuje se až na západ, takt bude i príchod Syna cloveka.
\par 28 Nebot kdežkoli bude telo, tut se sletí i orlice.
\par 29 A hned po soužení, kteréž bude tech dnu, slunce se zatmí a mesíc nedá svetla svého a hvezdy budou padati s nebe a moci nebeské budou se pohybovati.
\par 30 A tehdyt se ukáže znamení Syna cloveka na nebi, a tut budou kvíliti všecka pokolení zeme, a uzrít Syna cloveka pricházejícího na oblacích nebeských s mocí a slavou velikou.
\par 31 Kterýž pošle andely své s hlasem velikým trouby, a shromáždít vyvolené jeho ode ctyr vetru, od koncin nebes až do koncin jejich.
\par 32 Od stromu pak fíkového naucte se podobenství: Když již ratolest jeho odmladne a listí se pucí, porozumíváte, že blízko jest léto.
\par 33 Takéž i vy, když uzreli byste toto všecko, vezte, žet blízko jest a ve dverích království Boží.
\par 34 Amen pravím vám, že nepomine vek tento, až se tyto všecky veci stanou.
\par 35 Nebe a zeme pominou, ale slova má nepominou.
\par 36 O tom pak dni a hodine té nižádný neví, ani andelé nebeští, jediné sám Otec muj.
\par 37 Ale jakož bylo za dnu Noé, takt bude i príchod Syna cloveka.
\par 38 Nebo jakož jsou za dnu tech pred potopou žrali a pili, ženili se a vdávaly se, až do toho dne, když Noé všel do korábu,
\par 39 A nezvedeli, až prišla potopa, a zachvátila všecky, takt bude i príští Syna cloveka.
\par 40 Tehdyt dva budou na poli; jeden bude vzat, a druhý zanechán.
\par 41 Dve budou ve mlýne pri žernovu; jedna bude vzata, a druhá zanechána.
\par 42 Bdetež tedy, ponevadž nevíte, v kterou hodinu Pán váš prijíti má.
\par 43 Toto pak vezte, že byt vedel hospodár, v které by bdení zlodej mel prijíti, bdel by zajisté, a nedalt by podkopati domu svého.
\par 44 Protož i vy budte hotovi; nebo v tu hodinu, v kterouž se nenadejete, Syn cloveka prijde.
\par 45 Kdot tedy jest služebník verný a opatrný, kteréhož ustanovil pán jeho nad celedí svou, aby jim dával pokrm v cas?
\par 46 Blahoslavený služebník ten, kteréhož, prijda pán jeho, nalezl by, an tak ciní.
\par 47 Amen pravím vám, že nade vším statkem svým ustanoví jej.
\par 48 Jestliže by pak rekl zlý služebník ten v srdci svém: Prodlévá pán muj prijíti,
\par 49 I pocal by bíti spoluslužebníky, jísti a píti s opilci,
\par 50 Prijdet pán služebníka toho v den, v kterýž se nenadeje, a v hodinu, v kterouž neví.
\par 51 I oddelít jej, a díl jeho položí s pokrytci. Tamt bude plác a škripení zubu.

\chapter{25}

\par 1 Tehdy podobno bude království nebeské desíti pannám, kteréžto vzavše lampy své, vyšly proti Ženichovi.
\par 2 Pet pak z nich bylo opatrných, a pet bláznivých.
\par 3 Ty bláznivé vzavše lampy své, nevzaly s sebou oleje.
\par 4 Opatrné pak vzaly olej v nádobkách svých s lampami svými.
\par 5 A když prodléval Ženich, zdrímaly se všecky a zesnuly.
\par 6 O pulnoci pak stal se krik: Aj, Ženich jde, vyjdete proti nemu.
\par 7 Tedy vstaly všecky panny ty, a ozdobily lampy své.
\par 8 Bláznivé pak opatrným rekly: Udelte nám oleje vašeho, nebo lampy naše hasnou.
\par 9 I odpovedely opatrné, rkouce: Bojíme se, že by se snad ani nám i vám nedostalo, jdete radeji k prodavacum a kupte sobe.
\par 10 A když odešly kupovati, prišel Ženich, a které hotovy byly, vešly s ním na svadbu, i zavríny jsou dvere.
\par 11 Potom pak prišly i ty druhé panny, rkouce: Pane, pane, otevri nám.
\par 12 A on odpovedev, rekl: Amen, pravím vám, neznámt vás.
\par 13 Bdetež tedy; neb nevíte dne ani hodiny, v kterou Syn cloveka prijde.
\par 14 Neb tak se díti bude, jako když clovek jeden, jda na cestu, povolal služebníku svých a porucil jim statek svuj.
\par 15 I dal jednomu pet hriven, jinému pak dve, a jinému jednu, každému podle možnosti jeho, i odšel hned.
\par 16 Odšed pak ten, kterýž vzal pet hriven, težel jimi, i vydelal jiných pet hriven.
\par 17 Též i ten, kterýž vzal dve, získal jiné dve.
\par 18 Ale ten, kterýž vzal jednu, odšed, zakopal ji v zemi, a skryl peníze pána svého.
\par 19 Po mnohém pak casu prišel pán služebníku tech, i cinil pocet s nimi.
\par 20 A pristoupiv ten, kterýž byl pet hriven vzal, podal jiných peti hriven, rka: Pane, pet hriven dal jsi mi, aj, jiných pet hriven získal jsem jimi.
\par 21 I rekl mu pán jeho: To dobre, služebníce dobrý a verný, nad málem byl jsi verný, nad mnohem tebe ustanovím. Vejdiž v radost pána svého.
\par 22 Pristoupiv pak ten, kterýž byl dve hrivne vzal, dí: Pane, dve hrivne jsi mi dal, aj, jiné dve jimi získal jsem.
\par 23 Rekl mu pán jeho: To dobre, služebníce dobrý a verný, nad málem byl jsi verný, nad mnohem tebe ustanovím. Vejdiž v radost pána svého.
\par 24 Pristoupiv pak i ten, kterýž vzal jednu hrivnu, rekl: Pane, vedel jsem, že jsi ty clovek prísný, žna, kde jsi nerozsíval, a sbíraje, kde jsi nerozsypal,
\par 25 I boje se, odšel jsem a skryl hrivnu tvou v zemi. Aj, ted máš, což tvého jest.
\par 26 A odpovídaje pán jeho, rekl mu: Služebníce zlý a lenivý, vedel jsi, že žnu, kdež jsem nerozsíval, a sbírám, kdež jsem nerozsypal,
\par 27 Protož mel jsi ty peníze mé dáti penezomencum, a já prijda, vzal byl bych, což jest mého, s požitkem.
\par 28 Nu vezmetež od neho tu hrivnu, a dejte tomu, kterýž má deset hriven.
\par 29 (Nebo každému majícímu bude dáno, a budet hojne míti, od nemajícího pak i to, což má, budet odjato.)
\par 30 A toho neužitecného služebníka uvrztež do temností zevnitrních. Tamt bude plác a škripení zubu.
\par 31 A když prijde Syn cloveka v sláve své, a všickni svatí andelé s ním, tedy se posadí na trunu velebnosti své,
\par 32 A shromáždeni budou pred nej všickni národové. I rozdelí je na ruzno, jedny od druhých, tak jako pastýr oddeluje ovce od kozlu.
\par 33 A postavít ovce zajisté na pravici své, kozly pak na levici.
\par 34 Tedy dí Král tem, kteríž na pravici jeho budou: Pojdte požehnaní Otce mého, dedicne vládnete královstvím, vám pripraveným od ustanovení sveta.
\par 35 Nebo jsem lacnel, a dali jste mi jísti; žíznil jsem, a dali jste mi píti; hostem jsem býval, a prijali jste mne;
\par 36 Nah jsem byl, a priodeli jste mne; nemocen jsem byl, a navštívili jste mne; v žalári jsem sedel, a pricházeli jste ke mne.
\par 37 Tedy odpovedí jemu spravedliví, rkouce: Pane, kdy jsme te vídali lacného, a krmili jsme tebe, žíznivého, a dávalit jsme nápoj?
\par 38 Aneb kdy jsme te videli hoste, a prijali jsme tebe, anebo nahého, a priodeli jsme tebe?
\par 39 Aneb kdy jsme te videli nemocného, aneb v žalári, a pricházeli jsme k tobe?
\par 40 A odpovídaje Král, dí jim: Amen pravím vám: Cožkoli jste cinili jednomu z bratrí techto mých nejmenších, mne jste ucinili.
\par 41 Potom dí i tem, kteríž na levici budou: Jdete ode mne zlorecení do ohne vecného, kterýž jest pripraven dáblu i andelum jeho.
\par 42 Nebot jsem lacnel, a nedali jste mi jísti; žíznil jsem, a nedali jste mi píti;
\par 43 Hostem jsem byl, a neprijali jste mne; nah, a neodívali jste mne; nemocen a v žalári jsem byl, a nenavštívili jste mne.
\par 44 Tedy odpovedí jemu i oni, rkouce: Pane, kdy jsme tebe vídali lacného, neb žíznivého, aneb hoste, nebo nahého, neb nemocného, aneb v žalári, a neposloužili jsme tobe?
\par 45 Tedy odpoví jim, rka: Amen pravím vám: Cehož jste koli necinili jednomu z nejmenších techto, mne jste necinili.
\par 46 I pujdou tito do trápení vecného, ale spravedliví do života vecného.

\chapter{26}

\par 1 I stalo se, když dokonal Ježíš reci tyto všecky, rekl ucedlníkum svým:
\par 2 Víte, že po dvou dnech velikanoc bude a Syn cloveka zrazen bude, aby byl ukrižován.
\par 3 Tedy sešli se prední kneží a zákoníci, i starší lidu na sín nejvyššího kneze, kterýž sloul Kaifáš.
\par 4 A radili se, jak by Ježíše lstive jali a zamordovali.
\par 5 Ale pravili: Ne v den svátecní, aby snad nebyl rozbroj v lidu.
\par 6 Když pak byl Ježíš v Betany, v domu Šimona malomocného,
\par 7 Pristoupila k nemu žena, mající nádobu alabastrovou masti drahé, i vylila ji na hlavu jeho, když sedel za stolem.
\par 8 A vidouce to ucedlníci jeho, rozhnevali se, rkouce: I k cemu jest ztráta tato?
\par 9 Neb mohla tato mast prodána býti za mnoho, a dáno býti chudým.
\par 10 A znaje to Ježíš, dí jim: Proc za zlé máte této žene? Dobrý zajisté skutek ucinila nade mnou.
\par 11 Nebo chudé vždycky máte s sebou, ale mne ne vždycky míti budete.
\par 12 Vylivši zajisté tato mast tuto na mé telo, ku pohrebu mému to ucinila.
\par 13 Amen pravím vám: Kdežkoli kázáno bude evangelium toto po všem svete, takét i to bude praveno, co ucinila tato, na památku její.
\par 14 Tedy odšed k predním knežím, jeden ze dvanácti, kterýž sloul Jidáš Iškariotský,
\par 15 Rekl jim: Co mi chcete dáti, a já vám ho zradím? A oni uložili jemu dáti tridceti stríbrných.
\par 16 A od té chvíle hledal príhodného casu, aby ho zradil.
\par 17 Prvního pak dne presnic, pristoupili k Ježíšovi ucedlníci, rkouce jemu: Kde chceš, at pripravíme tobe, abys jedl beránka?
\par 18 On pak rekl: Jdete tam k jednomu do mesta, a rcete jemu: Vzkázalt Mistr: Cas muj blízko jest, u tebet jísti budu beránka s ucedlníky svými.
\par 19 I ucinili ucedlníci tak, jakož jim porucil Ježíš, a pripravili beránka.
\par 20 A když byl vecer, posadil se za stul se dvanácti.
\par 21 A když jedli, rekl jim: Amen pravím vám, že jeden z vás mne zradí.
\par 22 I zarmoutivše se velmi, pocali každý z nich ríci jemu: Zdali já jsem, Pane?
\par 23 On pak odpovídaje, rekl: Kdo omácí se mnou rukou v míse, tent mne zradí.
\par 24 Synt zajisté cloveka jde, jakož psáno o nem, ale beda cloveku tomu, skrze nehož Syn cloveka zrazen bude. Dobré by bylo jemu, by se byl nenarodil clovek ten.
\par 25 Odpovídaje pak Jidáš, kterýž ho zrazoval, dí: Zdali já jsem, Mistre? Rekl jemu: Ty jsi rekl.
\par 26 A když oni jedli, vzav Ježíš chléb a dobroreciv, lámal, a dal ucedlníkum, a rekl: Vezmete, jezte, to jest telo mé.
\par 27 A vzav kalich, a díky ciniv, dal jim, rka: Pijte z toho všickni.
\par 28 Neb to jest krev má nové smlouvy, kteráž za mnohé vylévá se na odpuštení hríchu.
\par 29 Ale pravímt vám, žet nebudu píti již více z tohoto plodu vinného korene, až do onoho dne, když jej píti budu s vámi nový v království Otce mého.
\par 30 A sezpívavše písnicku, vyšli na horu Olivetskou.
\par 31 Tedy dí jim Ježíš: Všickni vy zhoršíte se nade mnou této noci. Nebo psáno jest: Bíti budu pastýre, a rozprchnout se ovce stáda.
\par 32 Ale když z mrtvých vstanu, predejdu vás do Galilee.
\par 33 Odpovídaje pak Petr, rekl jemu: Byt se pak všickni zhoršili nad tebou, ját se nikdy nezhorším.
\par 34 Dí mu Ježíš: Amen pravím tobe, že této noci, prve než kohout zazpívá, trikrát mne zapríš.
\par 35 Rekl jemu Petr: Bycht pak mel také s tebou umríti, nezaprím tebe. Takž podobne i všickni ucedlníci pravili.
\par 36 Tedy prišel s nimi Ježíš na místo, kteréž sloulo Getsemany. I dí ucedlníkum: Posedtež tuto, ažt odejda, pomodlím se tamto.
\par 37 A pojav s sebou Petra a dva syny Zebedeovy, pocal se rmoutiti a teskliv býti.
\par 38 Tedy rekl jim: Smutnát jest duše má až k smrti. Pozustantež tuto a bdete se mnou.
\par 39 A poodšed malicko, padl na tvár svou, modle se a rka: Otce muj, jest-li možné, necht odejde ode mne kalich tento. Avšak ne jakž já chci, ale jakž ty chceš.
\par 40 I prišel k ucedlníkum, a nalezl je, ani spí. I rekl Petrovi: Tak-liž jste nemohli jediné hodiny bdíti se mnou?
\par 41 Bdetež a modlte se, abyste nevešli v pokušení. Ducht zajisté hotov jest, ale telo nemocno.
\par 42 Opet po druhé odšed, modlil se, rka: Otce muj, nemuže-lit tento kalich minouti mne, než abych jej pil, staniž se vule tvá.
\par 43 I prišed k nim, nalezl je, a oni zase spí; nebo byly oci jejich obtíženy.
\par 44 A nechav jich, opet odšel, a modlil se po tretí, touž rec ríkaje.
\par 45 Tedy prišel k ucedlníkum svým, a rekl jim: Spetež již a odpocívejte. Aj, priblížila se hodina, a Syna cloveka zrazují v ruce hríšných.
\par 46 Vstantež, pojdme. Aj, priblížil se ten, jenž mne zrazuje.
\par 47 A když on ješte mluvil, aj, Jidáš, jeden ze dvanácti, prišel, a s ním zástup mnohý s meci a s kyjmi, poslaných od predních kneží a starších lidu.
\par 48 Ten pak, jenž jej zrazoval, dal jim znamení, rka: Kteréhožtkoli políbím, ten jest; držtež jej.
\par 49 A hned pristoupiv k Ježíšovi, rekl: Zdráv bud, Mistre, a políbil jej.
\par 50 I rekl jemu Ježíš: Príteli, nac jsi prišel? Tedy pristoupili a ruce vztáhli na Ježíše a jali ho.
\par 51 A aj, jeden z tech, kteríž byli s Ježíšem, vztáh ruku, vytrhl mec svuj; a uderiv služebníka nejvyššího kneze, utal ucho jeho.
\par 52 Tedy dí jemu Ježíš: Obrat mec svuj v místo jeho; nebo všickni, kteríž mec berou, od mece zahynou.
\par 53 Zdaliž mníš, že bych nyní nemohl prositi Otce svého, a vydal by mi více nežli dvanácte houfu andelu?
\par 54 Ale kterak by se pak naplnila Písma, kteráž svedcí, že tak musí býti?
\par 55 V tu hodinu rekl Ježíš k zástupum: Jako na lotra vyšli jste s meci a s kyjmi jímati mne. Na každý den sedával jsem u vás, uce v chráme, a nejali jste mne.
\par 56 Ale toto se všecko stalo, aby se naplnila Písma prorocká. Tedy ucedlníci všickni opustivše ho, utekli.
\par 57 A oni javše Ježíše, vedli ho k Kaifášovi nejvyššímu knezi, kdežto zákoníci a starší byli se sešli.
\par 58 Ale Petr šel za ním zdaleka, až do síne nejvyššího kneze. A všed vnitr, sedel s služebníky, aby videl všeho toho konec.
\par 59 Prední pak kneží a starší a všecka ta rada hledali falešného svedectví proti Ježíšovi, aby jej na smrt vydali,
\par 60 I nenalezli. A ackoli mnozí falešní svedkové pristupovali, však nenalézali. Naposledy pak prišli dva falešní svedkové,
\par 61 A rekli: Tento jest povedel: Mohu zboriti chrám Boží a ve trech dnech zase jej ustaveti.
\par 62 A povstav nejvyšší knez, rekl jemu: Nic neodpovídáš? Což pak tito proti tobe svedcí?
\par 63 Ale Ježíš mlcel. I odpovídaje nejvyšší knez, rekl k nemu: Zaklínám te skrze Boha živého, abys nám povedel, jsi-li ty Kristus Syn Boží?
\par 64 Dí mu Ježíš: Ty jsi rekl. Ale však pravím vám: Od toho casu uzríte Syna cloveka sedícího na pravici moci Boží a pricházejícího na oblacích nebeských.
\par 65 Tedy nejvyšší knez roztrhl roucho své, a rekl: Rouhal se. Což ješte potrebujeme svedku? Aj, nyní jste slyšeli rouhání jeho.
\par 66 Co se vám zdá? A oni odpovídajíce, rekli: Hodent jest smrti.
\par 67 Tedy plili na tvár jeho a pohlavkovali jej; jiní pak hulkami jej bili,
\par 68 Ríkajíce: Hádej nám, Kriste, kdo jest ten, kterýž tebe uderil?
\par 69 Ale Petr sedel vne v síni. I pristoupila k nemu jedna devecka, rkuci: I ty jsi byl s Ježíšem tím Galilejským.
\par 70 On pak zaprel prede všemi, rka: Nevím, co pravíš.
\par 71 A když vycházel ze dverí, uzrela jej jiná devecka. I rekla tem, kteríž tu byli: I tento byl s Ježíšem tím Nazaretským.
\par 72 A on opet zaprel s prísahou, rka: Neznám toho cloveka.
\par 73 A po malé chvíli pristoupili blíže, kteríž tu stáli, i rekli Petrovi: Jiste i ty z nich jsi, neb i rec tvá známa tebe ciní.
\par 74 Tedy pocal se proklínati a prisahati, rka: Neznám toho cloveka. A hned kohout zazpíval.
\par 75 I rozpomenul se Petr na slovo Ježíšovo, kterýž jemu byl rekl: Že prve než kohout zazpívá, trikrát mne zapríš. A vyšed ven, plakal horce.

\chapter{27}

\par 1 A když bylo ráno, vešli v radu všickni prední kneží a starší lidu proti Ježíšovi, aby jej na smrt vydali.
\par 2 I svázavše jej, vedli, a vydali ho Pontskému Pilátovi hejtmanu.
\par 3 Tedy vida Jidáš, zrádce jeho, že by odsouzen byl, želeje toho, navrátil zase tridceti stríbrných predním knežím a starším,
\par 4 Rka: Zhrešil jsem, zradiv krev nevinnou. Oni pak rekli: Co nám do toho? Ty viz.
\par 5 A on povrh ty stríbrné v chráme, odšel pryc, a odšed, obesil se.
\par 6 A prední kneží vzavše peníze, rekli: Neslušít jich vložiti do pokladnice, nebo mzda krve jest.
\par 7 A poradivše se, koupili za ne pole to hrncírovo, ku pohrebu poutníku.
\par 8 Protož nazváno jest pole to pole krve, až do dnešního dne.
\par 9 A tehdy naplnilo se povedení skrze Jeremiáše proroka rkoucího: A vzali tridceti stríbrných, mzdu ceneného, kterýž šacován byl od synu Izraelských,
\par 10 A dali je za pole hrncírovo, jakož mi ustanovil Pán.
\par 11 Ježíš pak stál pred vladarem. A otázal se ho vladar, rka: Ty-li jsi ten král Židovský? Rekl jemu Ježíš: Ty pravíš.
\par 12 A když na nej prední kneží a starší žalovali, nic neodpovedel.
\par 13 Tedy dí mu Pilát: Neslyšíš-li, kteraké veci proti tobe svedcí?
\par 14 Ale on neodpovedel jemu k žádnému slovu, takže se vladar tomu velmi divil.
\par 15 Mel pak obycej vladar v svátek propustiti lidu vezne jednoho, kteréhož by chteli.
\par 16 I meli v ten cas vezne jednoho znamenitého, kterýž sloul Barabbáš.
\par 17 Protož když se lidé sešli, rekl: Kterého chcete, at vám propustím? Barabbáše-li, cili Ježíše, jenž slove Kristus?
\par 18 Nebot vedel, že jej z závisti vydali.
\par 19 A když sedel na soudné stolici, poslala k nemu žena jeho, rkuci: Nic nemej ciniti s spravedlivým tímto, nebo jsem mnoho trpela dnes ve snách pro neho.
\par 20 Ale prední kneží a starší navedli lid, aby prosili za Barabbáše, Ježíše pak aby zahubili.
\par 21 I odpovedev vladar, rekl jim: Kterého chcete ze dvou, at vám propustím? A oni rekli: Barabbáše.
\par 22 Dí jim Pilát: Co pak uciním s Ježíšem, jenž slove Kristus? Rekli mu všickni: Ukrižován bud.
\par 23 Vladar pak rekl: I což jest zlého ucinil? Oni pak více volali, rkouce: Ukrižován bud.
\par 24 A vida Pilát, že by nic neprospel, ale že by vetší rozbroj byl, vzav vodu, umyl ruce pred lidem, rka: Cist jsem já od krve spravedlivého tohoto. Vy vizte.
\par 25 A odpovedev všecken lid, rekl: Krev jeho na nás i na naše syny.
\par 26 Tedy propustil jim Barabbáše, ale Ježíše zbicovav, vydal, aby byl ukrižován.
\par 27 Tedy žoldnéri hejtmanovi, vzavše Ježíše do radného domu, shromáždili k nemu všecku svou rotu.
\par 28 A svlékše jej, priodíli ho pláštem brunátným.
\par 29 A spletše korunu z trní, vstavili na hlavu jeho, a dali trtinu v pravou ruku jeho, a klekajíce pred ním, posmívali se jemu, rkouce: Zdráv bud, ó králi Židovský.
\par 30 A plijíce na neho, brali trtinu a bili jej v hlavu.
\par 31 A když se mu naposmívali, svlékli s neho plášt, a oblékli jej v roucho jeho. I vedli ho, aby byl ukrižován.
\par 32 A vyšedše, nalezli cloveka Cyrenenského, jménem Šimona. Toho prinutili, aby nesl kríž jeho.
\par 33 I prišedše na místo, kteréž slove Golgata, to jest popravné místo,
\par 34 Dali mu píti octa, smíšeného se žlucí. A okusiv ho, nechtel píti.
\par 35 Ukrižovavše pak jej, rozdelili roucha jeho, mecíce o ne los, aby se naplnilo povedení proroka, rkoucího: Rozdelili sobe roucho mé, a o muj odev metali los.
\par 36 A sedíce, ostríhali ho tu.
\par 37 I vstavili nad hlavu jeho vinu jeho napsanou: Totot jest Ježíš, ten král Židovský.
\par 38 I ukrižováni jsou s ním dva lotri, jeden na pravici a druhý na levici.
\par 39 Ti pak, kteríž chodili tudy, rouhali mu se, ukrivujíce hlav svých,
\par 40 A ríkajíce: Hej, ty jako rušíš chrám Boží a ve trech dnech jej zase vzdeláváš, pomoziž sám sobe. Jsi-li Syn Boží, sestupiž s kríže.
\par 41 Tak podobne i prední kneží posmívajíce se s zákoníky a staršími, pravili:
\par 42 Jiným pomáhal, sám sobe nemuž pomoci. Jestliže jest král Židovský, nechat nyní sstoupí s kríže, a uveríme jemu.
\par 43 Doufalt v Boha, nechat ho nyní vysvobodí, chce-lit mu; nebo pravil: Syn Boží jsem.
\par 44 Takž také i lotri, kteríž byli s ním ukrižováni, utrhali jemu.
\par 45 Od šesté pak hodiny tma se stala po vší té zemi až do hodiny deváté.
\par 46 A pri hodine deváté zvolal Ježíš hlasem velikým, rka: Eli, Eli, lama zabachtani? To jest: Bože muj, Bože muj, proc jsi mne opustil?
\par 47 A nekterí z tech, jenž tu stáli, slyšíce, pravili, že Eliáše volá tento.
\par 48 A hned jeden z nich bežev, vzal houbu, naplnil ji octem a vloživ na trest, dával jemu píti.
\par 49 Ale jiní pravili: Nech tak, pohledíme, prijde-li Eliáš, aby ho vysvobodil.
\par 50 Ježíš pak opet volaje hlasem velikým, vypustil duši.
\par 51 A aj, opona chrámová roztrhla se na dvé, od vrchu až dolu, a zeme se trásla a skálé se pukalo,
\par 52 A hrobové se otvírali, a mnohá tela zesnulých svatých vstala jsou.
\par 53 A vyšedše z hrobu, po vzkríšení jeho prišli do svatého mesta a ukázali se mnohým.
\par 54 Tedy centurio a ti, kteríž s ním byli, ostríhajíce Ježíše, vidouce zemetresení a to, co se dálo, báli se velmi, rkouce: Jiste Syn Boží byl tento.
\par 55 Byly také tu ženy mnohé, zdaleka se dívajíce, kteréž byly prišly za Ježíšem od Galilee, posluhujíce jemu,
\par 56 Mezi nimiž byla Maria Magdaléna a Maria, matka Jakubova a Jozesova, a matka synu Zebedeových.
\par 57 A když byl vecer, prišel jeden clovek bohatý od Arimatie, jménem Jozef, kterýž také byl ucedlník Ježíšuv.
\par 58 Ten pristoupil ku Pilátovi a prosil za telo Ježíšovo. Tedy Pilát rozkázal dáti telo.
\par 59 A vzav telo Ježíšovo Jozef, obvinul je v plátno cisté,
\par 60 A vložil do svého nového hrobu, kterýž byl vytesal v skále; a privaliv kámen veliký ke dverum hrobovým, odšel.
\par 61 A byla tu Maria Magdaléna a druhá Maria, sedíce naproti hrobu.
\par 62 Druhého pak dne, kterýž byl po velikém pátku, sešli se prední kneží a farizeové ku Pilátovi,
\par 63 Rkouce: Pane, rozpomenuli jsme se, že ten svudce rekl, ješte živ jsa: Po trech dnech vstanu.
\par 64 Rozkažiž tedy ostríhati hrobu až do tretího dne, at by snad ucedlníci jeho, prijdouce v noci, neukradli ho, a rekli by lidu: Vstalt jest z mrtvých. I budet poslední blud horší nežli první.
\par 65 Rekl jim Pilát: Máte stráž; jdete, ostríhejte, jakž víte.
\par 66 A oni šedše, osadili hrob strážnými, zapecetivše kámen.

\chapter{28}

\par 1 Na skonání pak soboty, když již svitalo na první den toho téhodne, prišla Maria Magdaléna a druhá Maria, aby pohledely na hrob.
\par 2 A aj, zemetresení stalo se veliké. Nebo andel Páne sstoupiv s nebe a pristoupiv, odvalil kámen ode dverí hrobových, a posadil se na nem.
\par 3 A byl oblicej jeho jako blesk, a roucho jeho bílé jako sníh.
\par 4 A pro strach jeho zdesili se strážní a ucineni jsou jako mrtví.
\par 5 I odpovedev andel, rekl ženám: Nebojte se vy, nebot vím, že Ježíše ukrižovaného hledáte.
\par 6 Nenít ho tuto; nebo vstalt jest, jakož predpovedel. Pojdte, a vizte místo, kdež ležel Pán.
\par 7 A rychle jdouce, povezte ucedlníkum jeho, že vstal z mrtvých. A aj, predchází vás do Galilee, tam jej uzríte. Aj, povedel jsem vám.
\par 8 I vyšedše rychle z hrobu s bázní a s radostí velikou, bežely, aby ucedlníkum jeho zvestovaly.
\par 9 Když pak šly zvestovati ucedlníkum jeho, aj, Ježíš potkal se s nimi, rka: Zdrávy budte. A ony pristoupivše, chopily se noh jeho, a klanely se jemu.
\par 10 Tedy dí jim Ježíš: Nebojtež se. Jdete, zvestujte bratrím mým, at jdou do Galilee, a tamt mne uzrí.
\par 11 Když pak ony odešly, aj, nekterí z stráže prišedše do mesta, oznámili predním knežím všecko, co se stalo.
\par 12 Kterížto shromáždivše se s staršími a uradivše se, mnoho penez dali žoldnérum,
\par 13 Rkouce: Pravte, že ucedlníci jeho nocne prišedše, ukradli jej, když jsme my spali.
\par 14 A uslyší-lit o tom hejtman, myt ho spokojíme a vás bezpecny uciníme.
\par 15 A oni vzavše peníze, ucinili, jakž nauceni byli. I rozhlášeno jest slovo to u Židu až do dnešního dne.
\par 16 Jedenácte pak ucedlníku šli do Galilee na horu, kdežto jim byl uložil Ježíš.
\par 17 A uzrevše ho, klaneli se jemu. Ale nekterí pochybovali.
\par 18 A pristoupiv Ježíš, mluvil jim, rka: Dána jest mi všeliká moc na nebi i na zemi.
\par 19 Protož jdouce, ucte všecky národy, krtíce je ve jméno Otce i Syna i Ducha svatého,
\par 20 Ucíce je zachovávati všecko, což jsem koli prikázal vám. A aj, já s vámi jsem po všecky dny, až do skonání sveta. Amen.


\end{document}