\begin{document}

\title{Mark}

\chapter{1}

\par 1 Pocátek evangelium Ježíše Krista, Syna Božího;
\par 2 Jakož psáno jest v Prorocích: Aj, já posílám andela svého pred tvárí tvou, kterýž pripraví cestu tvou pred tebou.
\par 3 Hlas volajícího na poušti: Pripravujte cestu Páne, prímé cinte stezky jeho.
\par 4 Krtil Jan na poušti a kázal krest pokání na odpuštení hríchu.
\par 5 I vycházeli k nemu ze vší krajiny Židovské i Jeruzalémští, a krtili se od neho všickni v Jordáne rece, vyznávajíce hríchy své.
\par 6 Byl pak Jan odín srstmi velbloudovými, a pás kožený na bedrách jeho, a jídal kobylky a med lesní.
\par 7 A kázal, rka: Za mnou jde silnejší mne, kteréhožto nejsem hoden, sehna se, rozvázati reménka u obuvi jeho.
\par 8 Já zajisté krtil jsem vás vodou, ale ont vás krtíti bude Duchem svatým.
\par 9 I stalo se v tech dnech, prišel Ježíš z Nazarétu Galilejského, a pokrten jest v Jordáne od Jana.
\par 10 A hned vystoupe z vody, uzrel nebesa otevrená a Ducha jakožto holubici, sstupujícího na nej.
\par 11 A hlas stal se s nebe: Ty jsi ten muj milý Syn, v nemž mi se dobre zalíbilo.
\par 12 A ihned ho Duch vypudil na poušt.
\par 13 I byl tam na poušti ctyridceti dnu, a pokoušín byl od satana; a byl s zverí, a andelé prisluhovali jemu.
\par 14 Když pak byl vsazen Jan do žaláre, prišel Ježíš do Galilee, zvestuje evangelium království Božího,
\par 15 Prave: Že se naplnil cas, a priblížilo se království Boží. Cinte pokání, a verte evangelium.
\par 16 A chode podle more Galilejského, uzrel Šimona a Ondreje bratra jeho, ani pouštejí síti do more, nebo rybári byli.
\par 17 I rekl jim Ježíš: Pojdte za mnou, a uciním vás rybáre lidí.
\par 18 A oni hned opustivše síti své, šli za ním.
\par 19 A pošed odtud malicko, uzrel Jakuba Zebedeova, a Jana bratra jeho, kteríž také byli na lodí tvrdíce síti své;
\par 20 A hned povolal jich. A oni opustivše otce svého Zebedea na lodí s pacholky, šli za ním.
\par 21 I vešli do Kafarnaum. A hned v sobotu šel Ježíš do školy, a ucil.
\par 22 I divili se náramne ucení jeho; nebo ucil je, jako moc maje, a ne jako zákoníci.
\par 23 I byl v škole jejich clovek, posedlý duchem necistým. I zvolal,
\par 24 Rka: Ale což jest tobe do nás, Ježíši Nazaretský? Prišel jsi zatratiti nás; znám te, kdo jsi, a vím, že jsi ten svatý Boží.
\par 25 I primluvil mu Ježíš, rka: Umlkniž a vyjdi z neho.
\par 26 I polomcovav jím duch necistý a krice hlasem velikým, vyšel z neho.
\par 27 I lekli se všickni, takže se tázali mezi sebou, rkouce: I co jest toto? Jakéž jest toto nové ucení, že tento mocne duchum necistým rozkazuje, a poslouchají ho?
\par 28 I roznesla se povest o nem hned po vší krajine Galilejské.
\par 29 A hned vyšedše ze školy, prišli do domu Šimonova a Ondrejova s Jakubem a s Janem.
\par 30 Šimonova pak svegruše ležela, majíc zimnici. A hned jemu povedeli o ní.
\par 31 A pristoupiv, pozdvihl jí, ujav ji za ruku její, a hned prestala jí zimnice. I posluhovala jim.
\par 32 Vecer pak již pri západu slunce, nosili k nemu všecky nemocné i dábelníky.
\par 33 A bylo se všecko mesto sbehlo ke dverum.
\par 34 I uzdravoval mnohé ztrápené rozlicnými neduhy, a dábelství mnohá vymítal, a nedopustil mluviti dáblum; nebo znali ho.
\par 35 A prede dnem velmi ráno vstav Ježíš, vyšel, a šel na pusté místo, a tam se modlil.
\par 36 I šel za ním Šimon i ti, kteríž s ním byli.
\par 37 A když jej nalezli, rekli jemu: Všickni te hledají.
\par 38 I dí jim: Pojdmež do okolních mestecek, abych i tam kázal. Nebo na to jsem prišel.
\par 39 I kázal v školách jejich ve vší Galileji, a dábelství vymítal.
\par 40 Tedy prišel k nemu malomocný, prose ho, a klekna pred ním, rekl jemu: Pane, chceš-li, mužeš mne ocistiti.
\par 41 Ježíš pak slitovav se, vztáhl ruku, dotekl se ho a rekl jemu: Chci, bud cist.
\par 42 A když to rekl, hned odstoupilo od neho malomocenství, a ocišten jest.
\par 43 I pohroziv mu, hned ho odbyl,
\par 44 A rekl mu: Viziž, abys nižádnému nic nepravil. Ale jdi, ukaž se knezi, a obetuj za ocištení své to, což prikázal Mojžíš, na svedectví jim.
\par 45 On pak vyšed, pocal vypravovati mnoho a ohlašovati tu vec, takže již nemohl Ježíš do mesta zjevne vjíti, ale vne na místech pustých byl. I scházeli se k nemu odevšad.

\chapter{2}

\par 1 A opet všel do Kafarnaum po nekolika dnech. I uslyšáno jest, že by doma byl.
\par 2 A hned sešlo se jich množství, takže již nemohli ani ke dverum. I mluvil jim slovo.
\par 3 Tedy prijdou k nemu nekterí, nesouce šlakem poraženého, kterýžto ode ctyr nesen byl.
\par 4 A když k nemu nemohli pro zástupy, loupali strechu, kdež byl Ježíš, a proborivše pudu, spustili po provazích dolu ložce, na nemž ležel šlakem poražený.
\par 5 A vida Ježíš víru jejich, dí šlakem poraženému: Synu, odpouštejí se tobe hríchové tvoji.
\par 6 A byli tu nekterí z zákoníku, sedíce a myslíce v srdcích svých:
\par 7 Co tento tak mluví rouhave? Kdo muž odpustiti hríchy, jediné sám Buh?
\par 8 To hned poznav Ježíš duchem svým, že by tak premyšlovali sami v sobe, rekl jim: Proc o tom premyšlujete v srdcích svých?
\par 9 Co jest snáze ríci šlakem poraženému: Odpouštejí se tobe hríchové, cili ríci: Vstan a vezmi lože své a chod?
\par 10 Ale abyste vedeli, že Syn cloveka má moc na zemi odpoušteti hríchy, dí šlakem poraženému:
\par 11 Tobet pravím: Vstan, a vezmi lože své, a jdi do domu svého.
\par 12 I vstal hned, a vzav lože své prede všemi, odšel, takže se desili všickni, a chválili Boha, rkouce: Nikdy jsme toho nevideli.
\par 13 I vyšel opet k mori, a všecken zástup pricházel k nemu, i ucil je.
\par 14 A pomíjeje Ježíš, uzrel Léví syna Alfeova, sedícího na cle. I dí jemu: Pojd za mnou. A on vstav, šel za ním.
\par 15 I stalo se, když sedel za stolem v domu jeho, že i publikáni mnozí a hríšníci sedeli spolu s Ježíšem a s ucedlníky jeho; neb mnoho jich bylo, a šlo za ním.
\par 16 Zákoníci pak a farizeové vidouce, že jedl s publikány a s hríšníky, rekli ucedlníkum jeho: Což jest toho, že s publikány a hríšníky jí a pije Mistr váš?
\par 17 To uslyšav Ježíš, dí jim: Nepotrebují zdraví lékare, ale nemocní. Neprišelt jsem volati spravedlivých, ale hríšných ku pokání.
\par 18 Ucedlníci pak Janovi a farizejští postívali se. I prišli a rekli jemu: Proc ucedlníci Janovi a farizejští postí se, a tvoji ucedlníci se nepostí?
\par 19 I rekl jim Ježíš: Kterakž mohou synové Ženichovi postiti se, když jest s nimi Ženich? Dokavadž mají s sebou Ženicha, nemohout se postiti.
\par 20 Ale prijdout dnové, když od nich odjat bude Ženich, a tehdáž se budou postiti v tech dnech.
\par 21 Ano nižádný záplaty sukna nového neprišívá k rouchu starému; jinak odtrhne ta záplata nová od starého ješte neco, i bývá vetší díra.
\par 22 A žádný nevlévá vína nového do nádob starých; jinak rozpucí nové víno nádoby, a tak víno se vyleje, a nádoby se pokazí. Ale víno nové má lito býti do nádob nových.
\par 23 I stalo se, že šel Ježíš v sobotu skrze obilí, i pocali ucedlníci jeho jdouce vymínati klasy.
\par 24 Tedy farizeové rekli jemu: Pohled, cot ciní ucedlníci tvoji, cehož nesluší ciniti v sobotu.
\par 25 I rekl jim: Nikdy-liž jste nectli, co ucinil David, když nouze byla, a lacnel, on i ti, kteríž s ním byli?
\par 26 Kterak všel do domu Božího za Abiatara nejvyššího kneze, a jedl chleby posvátné, (jichžto neslušelo jísti než samým knežím,) a dal i tem, kteríž s ním byli?
\par 27 I pravil jim: Sobota pro cloveka ucinena jest, a ne clovek pro sobotu.
\par 28 Protož Syn cloveka jest pánem také i soboty.

\chapter{3}

\par 1 I všel opet do školy, a byl tu clovek, maje ruku uschlou.
\par 2 I šetrili ho, uzdraví-li jej v sobotu, aby ho obžalovali.
\par 3 I rekl tomu cloveku, kterýž mel uschlou ruku: Vstan a pojd sem do prostredku.
\par 4 I dí jim: Sluší-li v sobotu dobre ciniti, cili zle, život zachovati, cili zamordovati? Ale oni mlceli.
\par 5 A pohledev na ne vukol hnevive, zarmoutiv se nad tvrdostí srdce jejich, rekl cloveku: Vztáhni ruku svou. I vztáhl, a ucinena jest ruka jeho zdravá, jako i druhá.
\par 6 A vyšedše farizeové, hned s herodiány radu ucinili proti Ježíšovi, kterak by ho zahubili.
\par 7 Ježíš pak s ucedlníky svými poodšel k mori, a veliké množství od Galilee šlo za ním, i z Judstva,
\par 8 I od Jeruzaléma, i od Idumee, i z Zajordání; i ti, kteríž byli okolo Týru a Sidonu, množství veliké, slyšíce, kteraké veci ciní, prišli k nemu.
\par 9 I rozkázal ucedlníkum svým, aby lodicku ustavicne nahotove meli, pro zástup, aby ho tak netiskli.
\par 10 Nebo mnohé uzdravoval, takže nan padali, aby se ho dotýkali, kterížkoli meli jaké neduhy.
\par 11 A duchové necistí, jakž ho zazreli, padali pred ním a kriceli, rkouce: Ty jsi Syn Boží.
\par 12 A on velmi jim primlouval, aby ho nezjevovali.
\par 13 I vstoupil na horu, a povolal k sobe tech, kterýchž se jemu videlo; i prišli k nemu.
\par 14 I ustanovil jich dvanácte, aby s ním byli, aby je poslal kázati,
\par 15 A aby meli moc uzdravovati nemoci a vymítati dábelství:
\par 16 A nejprve Šimona, jemuž dal jméno Petr,
\par 17 A Jakuba Zebedeova, a Jana bratra Jakubova, (a dal jim jméno Boanerges, to jest synové hromovi,)
\par 18 A Ondreje, a Filipa, a Bartolomeje, a Matouše, a Tomáše, a Jakuba Alfeova, a Taddea, a Šimona Kananejského,
\par 19 A Jidáše Iškariotského, kterýž i zradil jej. I šli s ním domu.
\par 20 A vtom opet sšel se zástup, takže nemohli ani chleba pojísti.
\par 21 A slyšavše o tom príbuzní jeho, prišli, aby jej vzali; nebo pravili, že by se smyslem pominul.
\par 22 Zákoníci pak, kteríž byli prišli od Jeruzaléma, pravili, že Belzebuba má a že v knížeti dábelském vymítá dábly.
\par 23 A povolav jich, mluvil k nim v podobenstvích: Kterak muže satan satana vymítati?
\par 24 A jestliže království v sobe se rozdvojí, nemuže státi království to.
\par 25 A rozdvojí-li se dum proti sobe, nebude moci dum ten státi.
\par 26 Tak jestliže jest povstal satan sám proti sobe, a rozdvojen jest, nemuže státi, ale konec bére.
\par 27 Nižádný nemuže nádobí silného reka, vejda do domu jeho, rozebrati, lec by prve silného toho svázal; a tehdyt dum jeho zloupí.
\par 28 Amen pravím vám, že všickni hríchové odpušteni budou synum lidským, i rouhání, jímž by se rouhali,
\par 29 Ale kdo by se rouhal proti Duchu svatému, nemá odpuštení na veky, ale hoden jest vecného odsouzení.
\par 30 Nebo pravili: Že ducha necistého má.
\par 31 Tedy prišla matka jeho a bratrí, a stojíce vne, poslali k nemu, aby ho vyvolali.
\par 32 A sedel okolo neho zástup. I rekli jemu: Aj, matka tvá a bratrí tvoji vne hledají tebe.
\par 33 Ale on odpovedel jim, rka: Kdo jest matka má a bratrí moji?
\par 34 A obezrev ucedlníky vukol sedící, rekl: Aj, matka má a bratrí moji.
\par 35 Nebo kdož by koli cinil vuli Boží, tent jest bratr muj, i sestra, i matka má.

\chapter{4}

\par 1 A opet pocal Ježíš uciti u more. I shromáždil se k nemu zástup mnohý, takže vstoupiv na lodí, sedel na mori, a všecken zástup byl na zemi podle more.
\par 2 I ucil je mnohým vecem v podobenstvích, a pravil jim v ucení svém:
\par 3 Slyšte. Aj, vyšel rozsevac, aby rozsíval.
\par 4 I stalo se v tom rozsívání, že jedno padlo podle cesty, a priletelo ptactvo nebeské, i szobali je.
\par 5 A jiné padlo na místo skalnaté, kdežto nemelo mnoho zeme, a hned vzešlo; neb nemelo hlubokosti zeme.
\par 6 A když vyšlo slunce, uvadlo, a protože nemelo korene, uschlo.
\par 7 A jiné padlo mezi trní; i zrostlo trní, a udusilo je. I nevydalo užitku.
\par 8 Jiné pak padlo v zemi dobrou, a dalo užitek vzhuru vstupující a rostoucí; prineslo zajisté jedno tridcátý, a jiné šedesátý, a jiné pak stý.
\par 9 I pravil jim: Kdo má uši k slyšení, slyš.
\par 10 A když pak byl sám, tázali se ho ti, kteríž pri nem byli, se dvanácti, na to podobenství.
\par 11 I rekl jim: Vámt jest dáno, znáti tajemství království Božího, ale tem, kteríž jsou vne, v podobenství všecko se deje,
\par 12 Aby hledíce, hledeli, a neuzreli, a slyšíce, slyšeli, a nesrozumeli, aby se snad neobrátili, a byli by jim odpušteni hríchové.
\par 13 I dí jim: Neznáte podobenství tohoto? A kterakž pak jiná všecka podobenství poznáte?
\par 14 Rozsevac, ten slovo rozsívá.
\par 15 Titot pak jsou, ješto podle cesty síme prijímají, kdežto se rozsívá slovo, kteréž když oni slyší, ihned prichází satan a vynímá slovo, kteréž vsáto jest v srdcích jejich.
\par 16 A tak podobne ti, kteríž jako skalnatá zeme posáti jsou, kterížto jakž uslyší slovo, hned s radostí prijímají je.
\par 17 Než nemají korene v sobe, ale jsou casní; potom když vznikne soužení a protivenství pro slovo Boží, hned se horší.
\par 18 A tito jsou, jenž mezi trní posáti jsou, kteríž ac slovo slyší,
\par 19 Ale pecování tohoto sveta a oklamání zboží, a jiné žádosti zlé k tomu pristupující, udušují slovo, takže bez užitku bývá.
\par 20 Titot pak jsou, jenž v zemi dobrou síme prijali, kteríž slyší slovo Boží, a prijímají, a užitek prinášejí, jedno tridcátý, a jiné šedesátý, a jiné stý.
\par 21 Dále pravil jim: Zdali rozsvícena bývá svíce, aby postavena byla pod nádobu nebo pod postel? Však aby na svícen vstavena byla.
\par 22 Nebo nic není skrytého, co by nebylo zjeveno; aniž jest co tak ukrytého, aby najevo nevyšlo.
\par 23 Jestliže kdo má uši k slyšení, slyš.
\par 24 I mluvil k nim: Vizte, co slyšíte. Kterou merou budete meriti, tout vám bude odmereno, a pridáno bude vám poslouchajícím.
\par 25 Nebo kdožt má, tomu bude dáno; a kdo nemá, i to, což má, bude od neho odjato.
\par 26 I pravil jim: Tak jest království Boží, jako kdyby clovek uvrhl síme v zemi.
\par 27 A spal by, a vstával by ve dne i v noci, a semeno by vzešlo a vzrostlo, jakž on neví.
\par 28 Nebo sama od sebe zeme plodí, nejprv bylinu, potom klas, potom plné obilé v klasu.
\par 29 A když sezrá úroda, ihned priciní srp; nebot jest nastala žen.
\par 30 I rekl opet: K cemu pripodobníme království Boží? Aneb kterému podobenství je prirovnáme?
\par 31 Jest jako zrno horcicné, kteréžto, když vsáto bývá v zemi, menší jest ze všech semen, kteráž jsou na zemi.
\par 32 Ale když vsáto bývá, roste, a bývá vetší než všecky byliny, a cinít ratolesti veliké, takže pod stínem jeho mohou sobe ptáci nebeští hnízda delati.
\par 33 A takovými mnohými podobenstvími mluvil jim slovo, jakž mohli slyšeti.
\par 34 A bez podobenství nemluvil jim, ucedlníkum pak svým soukromí vykládal všecko.
\par 35 I rekl jim v ten den, když již bylo vecer: Plavme se na druhou stranu.
\par 36 A nechavše zástupu, pojali jej, tak jakž byl na lodicce. Ale i jiné lodicky byly s ním.
\par 37 Tedy stala se boure veliká od vetru, až se vlny na lodí valily, takže se již naplnovala lodí.
\par 38 A on zzadu na lodí spal na podušce. I zbudili jej, a rekli jemu: Mistre, což pak nic nedbáš, že hyneme?
\par 39 I probudiv se, primluvil vetru a rekl mori: Umlkni a upokoj se. I prestal vítr, a stalo se utišení veliké.
\par 40 I rekl jim: Proc se tak bojíte? Což ješte nemáte víry?
\par 41 I báli se bázní velikou, a pravili jeden k druhému: Hle kdo jest tento, že i vítr i more poslouchají jeho?

\chapter{5}

\par 1 Tedy preplavili se pres more do krajiny Gadarenských.
\par 2 A jakž vyšel z lodí, hned se s ním potkal clovek z hrobu, maje ducha necistého.
\par 3 Kterýž bydlil v hrobích, a aniž ho kdo již mohl retezy svázati,
\par 4 Nebo casto jsa pouty a retezy okován, polámal retezy a pouta roztrhal, a žádný nemohl ho zkrotiti.
\par 5 A vždycky ve dne i v noci na horách a v hrobích byl, krice a tepa se kamením.
\par 6 Uzrev pak Ježíše zdaleka, bežel a poklonil se jemu,
\par 7 A krice hlasem velikým, rekl: Co jest tobe do mne, Ježíši, Synu Boha nejvyššího? Zaklínám te skrze Boha, abys mne netrápil.
\par 8 (Nebo pravil jemu: Vyjdiž, duchu necistý, z cloveka tohoto.)
\par 9 I otázal se ho: Jakt ríkají? A on odpovídaje, rekl: Množství jméno mé jest, neb jest nás mnoho.
\par 10 I prosil ho velmi, aby jich nevyhánel z té krajiny.
\par 11 Bylo pak tu pri horách stádo vepru veliké pasoucích se.
\par 12 I prosili ho všickni ti dáblové, rkouce: Pust nás do vepru, at do nich vejdeme.
\par 13 I povolil jim hned Ježíš. A vyšedše duchové necistí, vešli do vepru. I beželo to stádo s vrchu dolu do more, (a bylo jich ke dvema tisícum,) i ztonuli v mori.
\par 14 Ti pak, kteríž ty vepre pásli, utekli a oznámili to v meste i ve vsech. I vyšli lidé, aby videli, co je se to stalo.
\par 15 I prišli k Ježíšovi, a uzreli toho, kterýž byl trápen od dábelství, an sedí, odín jsa a maje zdravý rozum, toho totiž, kterýž mel tmu dáblu. I báli se.
\par 16 A kteríž to videli, vypravovali jim, kterak se stalo tomu, kterýž mel dábelství, i o veprích.
\par 17 Tedy pocali ho prositi, aby odšel z krajin jejich.
\par 18 A když vstoupil na lodí, prosil ho ten, kterýž trápen byl od dábelství, aby byl s ním.
\par 19 Ježíš pak nedopustil mu, ale rekl jemu: Jdi k svým do domu svého, a zvestuj jim, kterak jest veliké veci ucinil tobe Hospodin, a slitoval se nad tebou.
\par 20 I odšel, a pocal ohlašovati v krajine Desíti mest, kterak veliké veci ucinil mu Ježíš. I divili se všickni.
\par 21 A když se preplavil Ježíš na lodí zase na druhou stranu, sšel se k nemu zástup mnohý. A on byl u more.
\par 22 A aj, prišel jeden z knížat školy Židovské, jménem Jairus, a uzrev jej, padl k nohám jeho,
\par 23 A velmi ho prosil, rka: Dcerka má skonává. Prosím, pojd, vlož na ni ruce, aby uzdravena byla, a budet živa.
\par 24 I šel s ním, a zástup mnohý šel za ním, i tiskli jej.
\par 25 (Tedy žena jedna, kteráž tok krve mela dvanácte let,
\par 26 A mnoho byla trápena od mnohých lékaru, a vynaložila všecken statek svuj, a nic jí bylo neprospelo, ale vždy se hure mela,
\par 27 Uslyšavši o Ježíšovi, prišla v zástupu pozadu, a dotkla se roucha jeho.
\par 28 Neb rekla byla: Dotknu-li se jen roucha jeho, uzdravena budu.
\par 29 A hned prestal jest krvotok její, a pocítila na tele, že by uzdravena byla od neduhu svého.
\par 30 A hned Ježíš poznav sám v sobe, že jest moc vyšla z neho k uzdravení, obrátiv se v zástupu, rekl: Kdo se dotekl roucha mého?
\par 31 I rekli mu ucedlníci jeho: Vidíš, že te zástup tiskne, a pravíš: Kdo se mne dotekl?
\par 32 I hledel vukol, aby ji uzrel, která jest to ucinila.
\par 33 Ta pak žena s bázní a s tresením, veduci, co se stalo pri ní, pristoupila a padla pred ním, a povedela mu všecku pravdu.
\par 34 On pak rekl jí: Dcero, víra tvá te uzdravila, jdiž u pokoji, a bud zproštena od trápení svého.)
\par 35 A když on ješte mluvil, prišli nekterí z domu knížete školy, rkouce: Dcera tvá umrela, proc již zamestknáváš Mistra?
\par 36 Ježíš pak, hned jakž uslyšel to, což oni mluvili, rekl knížeti školy: Neboj se, toliko ver.
\par 37 I nedal žádnému za sebou jíti, jediné Petrovi, Jakubovi a Janovi, bratru Jakubovu.
\par 38 I prišel do domu knížete školy, a videl tam hluk, ano plací a kvílí velmi.
\par 39 I všed tam, rekl jim: Co se bouríte a placete? Neumrelat jest devecka, ale spí.
\par 40 I posmívali se jemu. On pak vyhnav všecky, pojal toliko otce a matku devecky, a ty, kteríž s ním byli, i všel tam, kdež devecka ležela.
\par 41 A vzav ruku devecky, rekl jí: Talitha kumi, jenž se vykládá: Devecko, (tobet pravím,) vstan.
\par 42 A hned vstala devecka, a chodila; nebo byla ve dvanácti letech. I zdesili se divením prevelikým.
\par 43 A prikázal jim pilne, aby žádný o tom nezvedel. I rozkázal jí dáti jísti.

\chapter{6}

\par 1 I vyšel odtud a prišel do vlasti své, a šli za ním ucedlníci jeho.
\par 2 A když bylo v sobotu, pocal uciti v škole, a mnozí slyšíce, divili se, rkouce: Odkud tento má tyto veci? A jaká jest to moudrost, kteráž jest jemu dána, že i takové moci dejí se skrze ruce jeho?
\par 3 Zdaliž tento není tesar, syn Marie, bratr Jakubuv a Jozesuv a Juduv a Šimonuv? A zdaliž nejsou i sestry jeho zde u nás? I zhoršili se na nem.
\par 4 I rekl jim Ježíš: Není prorok beze cti, jediné v vlasti své a v rodine své a v domu svém.
\par 5 I nemohl tu znamení žádného uciniti, jediné málo nemocných, vzkládaje na ne ruce, uzdravil.
\par 6 I podivil se jejich nevere, a obcházel vukol po mesteckách, uce.
\par 7 A svolav dvanácte, pocal je posílati po dvou a dvou, a dal jim moc nad duchy necistými.
\par 8 A prikázal jim, aby nicehož nebrali na cestu, jediné toliko hul, ani mošny, ani chleba, ani na pase penez,
\par 9 Ale jen obuté míti nohy v strevíce, a aby neobláceli dvou sukní.
\par 10 A pravil jim: Kdežkoli vešli byste do domu, tu ostante, dokudž nevyšli byste odtud.
\par 11 A kdož by koli vás neprijali, ani vás slyšeli, vyjdouce odtud, vyrazte prach z noh vašich na svedectví jim. Amen pravím vám: Lehceji bude Sodomským a Gomorským v den soudný nežli mestu tomu.
\par 12 Tedy oni vyšedše, kázali, aby pokání cinili.
\par 13 A dábelství mnohá vymítali, a mazali olejem mnohé nemocné, a uzdravovali je.
\par 14 A uslyšev o tom Herodes král, (neb zjevné ucineno bylo jméno jeho,) i pravil, že Jan Krtitel vstal z mrtvých, a protož se dejí divové skrze neho.
\par 15 Jiní pak pravili, že jest Eliáš; a jiní pravili, že jest prorok, aneb jako jeden z proroku.
\par 16 To uslyšev Herodes, rekl: Kteréhož jsem já stal, Jana, tent jest. Onte z mrtvých vstal.
\par 17 Ten zajisté Herodes byl poslal a jal Jana a vsadil jej do žaláre pro Herodiadu manželku Filipa bratra svého, že ji byl za manželku pojal.
\par 18 Nebo pravil Jan Herodesovi: Neslušít tobe míti manželky bratra svého.
\par 19 Herodias pak lest skládala proti nemu, a chtela jej o hrdlo pripraviti, ale nemohla.
\par 20 Nebo Herodes ostýchal se Jana, veda jej býti muže spravedlivého a svatého. I šetril ho, a slýchaje jej, mnoho i cinil, a rád ho poslouchal.
\par 21 A když prišel den príhodný, že Herodes, pamatuje den svého narození, ucinil veceri knížatum svým a hejtmanum a predním mužum z Galilee,
\par 22 A dcera té Herodiady tam vešla a tancovala, zalíbilo se Herodesovi i spoluhodovníkum, i rekl král devecce: Pros mne, zac chceš, a dámt.
\par 23 I prisáhl jí: Že zackoli prositi budeš, dám tobe, by pak bylo až do polovice království mého.
\par 24 Ona pak vyšedši, rekla materi své: Zac budu prositi? A ona rekla: Za hlavu Jana Krtitele.
\par 25 A všedši hned s chvátáním k králi, prosila ho, rkuci: Chci, abys mi dal hned na míse hlavu Jana Krtitele.
\par 26 Král pak zarmoutiv se velmi, pro prísahu a pro spoluhodovníky nechtel jí oslyšeti.
\par 27 I poslav hned kata, rozkázal prinésti hlavu Janovu.
\par 28 A on odšed, stal jej v žalári, a prinesl hlavu jeho na míse, a dal ji devecce, a devecka dala materi své.
\par 29 To uslyšavše ucedlníci jeho, prišli a vzali telo jeho, a pochovali je v hrobe.
\par 30 Tedy sšedše se apoštolé k Ježíšovi, zvestovali jemu všecko, i to, co cinili, i co ucili.
\par 31 I rekl jim: Pojdte vy sami obzvláštne na pusté místo, a odpocinte malicko. Nebo bylo množství tech, kteríž pricházeli a odcházeli, takže jsou ani k jídlu chvíle nemeli.
\par 32 I plavili se až na pusté místo soukromí.
\par 33 A vidouce je zástupové, že jdou pryc, poznali jej mnozí. I sbehli se tam ze všech mest pešky, a predešli je, a shromáždili se k nemu.
\par 34 Tedy vyšed Ježíš, uzrel zástup mnohý, a slitovalo mu se jich, že byli jako ovce, nemající pastýre. I pocal je uciti mnohým vecem.
\par 35 A když se již prodlilo, pristoupivše k nemu ucedlníci jeho, rekli: Pustét jest toto místo, a již se prodlilo,
\par 36 Rozpust je, at jdouce do okolních vesnic a mestecek, nakoupí sobe chleba; nebo nemají, co by jedli.
\par 37 On pak odpovedev, rekl jim: Dejte vy jim jísti. I rkou jemu: Co tedy, jdouce koupíme za dve ste grošu chleba, a dáme jim jísti?
\par 38 I dí jim: Kolik chlebu máte? Jdete a zvezte. A když zvedeli, rekli: Pet, a dve rybe.
\par 39 I rozkázal jim, aby se kázali posaditi všechnem po houfích na zelené tráve.
\par 40 I usadili se rozdílne, místy po stu a místy po padesáti.
\par 41 A vzav tech pet chlebu a ty dve rybe, popatriv do nebe, dobrorecil, i lámal chleby, a dal ucedlníkum svým, aby kladli pred ne. A dve rybe rozdelil též mezi všecky.
\par 42 I jedli všickni, a nasyceni jsou.
\par 43 Potom sebrali drobtu dvanácte košu plných, i z ryb.
\par 44 A bylo tech, kteríž jedli ty chleby, okolo pet tisícu mužu.
\par 45 A hned prinutil ucedlníky své vstoupiti na lodí, aby jej predešli pres more do Betsaidy, až by on rozpustil zástup.
\par 46 A rozpustiv je, šel na horu, aby se modlil.
\par 47 A když bylo vecer, byla lodí uprostred more, a on sám na zemi.
\par 48 A videl je, a oni se s težkostí plavili; (nebo byl vítr odporný jim.) A pri ctvrtém bdení nocním prišel k nim, chode po mori, a chtel je pominouti.
\par 49 Oni pak uzrevše jej, an chodí po mori, domnívali se, že by obluda byla, i zkrikli.
\par 50 (Nebo jej všickni videli, a zstrašili se.) A hned promluvil k nim a rekl jim: Doufejtež, ját jsem, nebojte se.
\par 51 I vstoupil k nim na lodí, a utišil se vítr; a oni náramne sami v sobe se desili a divili.
\par 52 Nebo nerozumeli byli, co se stalo pri chlebích; bylo zajisté srdce jejich zhrublo.
\par 53 A když se preplavili, prišli do zeme Genezaretské, a tu lodí pristavili.
\par 54 A když vyšli z lodí, hned jej poznali.
\par 55 A behajíce po vší krajine té, pocali na ložcích k nemu nositi nemocné, kdežkoli zvedeli o nem, že by byl.
\par 56 A kamžkoli vcházel do mestecek neb do mest nebo do vsí, na ulicech kladli neduživé, a prosili ho, aby se aspon podolka roucha jeho dotkli. A kolikož jich koli se jeho dotkli, uzdraveni byli.

\chapter{7}

\par 1 I sešli se k nemu farizeové a nekterí z zákoníku, kteríž byli prišli z Jeruzaléma.
\par 2 A uzrevše nekteré z ucedlníku jeho obecnýma rukama (to jest neumytýma) jísti chleby, reptali o to.
\par 3 Nebo farizeové i všickni Židé nejedí, lec by ruce umyli, držíce ustanovení starších.
\par 4 A z trhu prijdouce nejedí, lec se umyjí. A jiné mnohé veci jsou, kteréž prijali, aby zachovávali, jako umývání koflíku, žejdlíku a medenic i stolu.
\par 5 Potom otázali se ho farizeové a zákoníci: Proc ucedlníci tvoji nezachovávají ustanovení starších, ale neumytýma rukama jedí chléb?
\par 6 On pak odpovedev, rekl jim: Dobre o vás pokrytcích prorokoval Izaiáš, jakož psáno jest: Lid tento rty mne ctí, srdce pak jejich daleko jest ode mne.
\par 7 Ale nadarmot mne ctí, ucíce ucení, kterážto nejsou než ustanovení lidská.
\par 8 Nebo opustivše prikázání Boží, držíte ustanovení lidská, totiž umývání žejdlíku a koflíku; a jiné mnohé veci tem podobné ciníte.
\par 9 I pravil jim: Ciste vy rušíte prikázání Boží, abyste ustanovení své zachovali.
\par 10 Nebo Mojžíš povedel: Cti otce svého i matku svou, a kdož by zlorecil otci nebo materi, at smrtí umre.
\par 11 Ale vy pravíte: Rekl-li by clovek otci neb materi: Korban, to jest, dar, kterýžkoli jest ode mne, tobet prospeje,
\par 12 A nedopustíte mu nic více uciniti otci svému nebo materi své,
\par 13 Rušíce prikázání Boží ustanoveními vašimi, kteráž jste ustanovili. A mnohé tem podobné veci ciníte.
\par 14 I svolav všecken zástup, pravil jim: Slyšte mne všickni a rozumejte.
\par 15 Nic není z zevnitrku vcházejícího do cloveka, což by jej poskvrniti mohlo; ale to, což pochází z neho, tot jest, což poskvrnuje cloveka.
\par 16 Má-li kdo uši k slyšení, slyš.
\par 17 A když všel do domu od zástupu, tázali se ho ucedlníci jeho o tom podobenství.
\par 18 I rekl jim: Tak jste i vy nerozumní? Což nerozumíte, že všecko, což z zevnitrku do cloveka vchází, nemuže ho poskvrniti?
\par 19 Nebo nevchází v srdce jeho, ale v bricho, a potom ven vychází, cisteci všeliké pokrmy.
\par 20 Ale pravil, že to, což pochází z cloveka, to poskvrnuje cloveka.
\par 21 Nebo z vnitrku z srdce lidského zlá myšlení pocházejí, cizoložstva, smilstva, vraždy,
\par 22 Krádeže, lakomství, nešlechetnosti, lest, nestydatost, oko zlé, rouhání, pýcha, bláznovství.
\par 23 Všecky tyto zlé veci pocházejí z vnitrku a poskvrnují cloveka.
\par 24 A vstav odtud, odšel do koncin Týru a Sidonu, a všed do domu, nechtel, aby kdo o nem vedel, ale nemohl se tajiti.
\par 25 Nebo uslyševši o nem žena, jejížto dcerka mela ducha necistého, prišla a padla k nohám jeho.
\par 26 (Byla pak ta žena pohanka, Syrofenitská rodem.) I prosila ho, aby dábelství vyvrhl z její dcery.
\par 27 Ale Ježíš rekl jí: Nechat se prve nasytí synové; nebt není slušné vzíti chléb synu a vrci štenatum.
\par 28 A ona odpovedela a rekla mu: Ovšem, Pane, nebo štenátka jedí pod stolem drobty synu.
\par 29 I rekl jí: Pro tu rec jdi, vyšlot jest dábelství z tvé dcery.
\par 30 I odšedši do domu svého, nalezla devecku, ana leží na loži, a dábelství z ní vyšlo.
\par 31 Tedy odšed zase z koncin Tyrských a Sidonských, prišel k mori Galilejskému, prostredkem krajin Desíti mest.
\par 32 I privedli jemu hluchého a nemého, a prosili ho, aby na nej ruku vzložil.
\par 33 A pojav jej soukromí ven z zástupu, vložil prsty své v uši jeho, a plinuv, dotekl se jazyka jeho.
\par 34 A vzezrev k nebi, vzdechl, a rekl jemu: Effeta, to jest, otevri se.
\par 35 A hned otevríny jsou uši jeho, a rozvázán jest svazek jazyka jeho, i mluvil práve.
\par 36 I prikázal jim, aby žádnému nepravili. Ale jakžkoli on jim prikazoval, predce oni mnohem více ohlašovali.
\par 37 A prevelmi se divili, rkouce: Dobre všecky veci ucinil. I hluchým rozkázal slyšeti, i nemým mluviti.

\chapter{8}

\par 1 V tech dnech když opet velmi veliký zástup byl s ním, a nemeli, co by jedli, svolav Ježíš ucedlníky své, rekl jim:
\par 2 Lítost mám nad zástupy; nebo již tri dni trvají se mnou a nemají, co by jedli.
\par 3 A rozpustím-li je lacné do domu jejich, zhynou na ceste; nebo nekterí z nich zdaleka prišli.
\par 4 Odpovedeli mu ucedlníci jeho: I odkud bude moci kdo tyto nakrmiti chleby zde na poušti?
\par 5 I otázal se jich: Kolik chlebu máte? A oni rekli: Sedm.
\par 6 I kázal zástupu posaditi se na zemi. A vzav sedm chlebu, díky uciniv, lámal a dával ucedlníkum svým, aby predkládali. I kladli pred zástup.
\par 7 A meli také rybicek malicko. Jichž požehnav, kázal i ty pred ne klásti.
\par 8 I jedli a nasyceni jsou; a sebrali, což pozustalo drobtu, sedm košu.
\par 9 Tech pak, kteríž jedli, bylo okolo ctyr tisícu. I rozpustil je.
\par 10 Potom hned vstoupiv na lodí s ucedlníky svými, preplavil se do krajin Dalmanutských.
\par 11 I vyšli farizeové a pocali se s ním hádati, hledajíce od neho znamení s nebe, pokoušejíce ho.
\par 12 A on vzdech duchem svým, dí: Co pokolení toto znamení hledá? Amen pravím vám: Nebude dáno znamení pokolení tomuto.
\par 13 A opustiv je, vstoupil zase na lodí, i plavil se pres more.
\par 14 I zapomenuli s sebou vzíti chlebu, a nemeli než jeden chléb s sebou na lodí.
\par 15 Tedy prikazoval jim, rka: Vizte a pilne se šetrte kvasu farizejského a kvasu Herodesova.
\par 16 I premyšlovali, rkouce jeden k druhému: Chleba nemáme.
\par 17 A znaje to Ježíš, rekl jim: Co premyšlujete o tom, že chleba nemáte? Ješte neznáte, ani rozumíte? Ješte máte oslepené srdce vaše?
\par 18 Oci majíce, nevidíte? A uši majíce, neslyšíte? A nepomníte,
\par 19 Že jsem pet chlebu lámal mezi pet tisícu? A kolik jste plných košu drobtu sebrali? Rekli jemu: Dvanácte.
\par 20 A když také sedm chlebu lámal jsem mezi ctyri tisíce, kolik jste plných košu drobtu vzali? I rkou jemu: Sedm.
\par 21 I rekl jim: Kterakž tedy ješte nerozumíte?
\par 22 I prišel do Betsaidy, a privedli k nemu slepého, prosíce ho, aby se ho dotekl.
\par 23 I ujav slepého za ruku, vyvedl jej ven z mestecka, a plinuv na oci jeho a vloživ na nej ruce, otázal se ho, videl-li by co.
\par 24 A on pohledev, rekl: Znamenám lidi; nebo vidím, že chodí jako stromové.
\par 25 Potom opet vložil ruce na oci jeho, a kázal mu hledeti. I uzdraven jest, takže i zdaleka jasne videl všecky.
\par 26 I odeslal jej do domu jeho, rka: Aniž do toho mestecka chod, aniž komu z mestecka co o tom prav.
\par 27 Tedy vyšel Ježíš a ucedlníci jeho do mestecek Cesaree Filipovy. A na ceste tázal se ucedlníku svých, rka jim: Kým mne praví býti lidé?
\par 28 Kterížto odpovedeli: Janem Krtitelem, a jiní Eliášem, jiní pak jedním z proroku.
\par 29 Tedy on rekl jim: Vy pak kým mne býti pravíte? Odpovedev Petr, rekl jemu: Ty jsi Kristus.
\par 30 I prikázal jim, aby toho o nem žádnému nepravili.
\par 31 I pocal uciti je, že Syn cloveka musí mnoho trpeti, a potupen býti od starších a predních kneží a zákoníku, a zabit býti, a ve trech dnech z mrtvých vstáti.
\par 32 Zjevne to slovo mluvil. A chytiv jej Petr, pocal mu domlouvati.
\par 33 Kterýžto obrátiv se a pohledev na ucedlníky své, primluvil Petrovi, rka: Jdiž za mnou, satane; nebo nechápáš, co jest Božího, ale co lidského.
\par 34 A svolav zástup s ucedlníky svými, rekl jim: Chce-li kdo za mnou prijíti, zapri sebe sám, a vezmi kríž svuj, a následujž mne.
\par 35 Nebo chtel-li by kdo duši svou zachovati, ztratít ji; pakli by kdo ztratil duši svou pro mne a pro evangelium, tent ji zachová.
\par 36 Nebo co prospeje cloveku, by všecken svet získal, a své duši škodu ucinil?
\par 37 Aneb jakou dá clovek odmenu za duši svou?
\par 38 Nebo kdož by se koli za mne stydel a za má slova v tomto pokolení cizoložném a hríšném, i Syn cloveka stydeti se bude za nej, když prijde v sláve Otce svého s andely svatými.

\chapter{9}

\par 1 I pravil jim: Amen pravím vám, žet jsou nekterí z stojících tuto, kteríž neokusí smrti, až i uzrí království Boží pricházející v moci.
\par 2 A po šesti dnech pojal Ježíš Petra a Jakuba a Jana, i uvedl je na horu vysokou soukromí samy, a promenil se pred nimi.
\par 3 A ucineno jest roucho jeho stkvoucí a bílé velmi jako sníh, ješto tak bílého žádný belic na zemi uciniti nemuže.
\par 4 I uzreli Eliáše s Mojžíšem, ani s Ježíšem mluví.
\par 5 A odpovedev Petr, rekl k Ježíšovi: Mistre, dobrét jest nám tuto býti. Protož udelejme tri stánky, tobe jeden, Mojžíšovi jeden a Eliášovi jeden.
\par 6 Nebo nevedel, co mluví; byli zajisté prestrašeni.
\par 7 I stal se oblak zastenující je, a prišel hlas z oblaku, rkoucí: Tentot jest ten Syn muj milý, jeho poslouchejte.
\par 8 A hned obezrevše se, žádného víc nevideli než samého Ježíše s sebou.
\par 9 A když sstupovali s hory, prikázal jim, aby toho žádnému nevypravovali, co videli, než až Syn cloveka z mrtvých vstane.
\par 10 I zachovali tu vec u sebe, tížíce mezi sebou, co by to bylo z mrtvých vstáti?
\par 11 I otázali se ho, rkouce: Což pak zákoníci praví, že Eliáš musí prijíti prve?
\par 12 On pak odpovedev, rekl jim: Eliáš prijda nejprve, napraví všecky veci, a jakož psáno jest o Synu cloveka, že má mnoho trpeti a za nic položen býti.
\par 13 Ale pravím vám, že Eliáš již prišel, a ucinili mu, což jsou chteli, jakož psáno jest o nem.
\par 14 Tedy prišed k ucedlníkum, uzrel zástup veliký okolo nich a zákoníky, an se hádají s nimi.
\par 15 A hned všecken zástup uzrev jej, ulekli se; a sbehše se, privítali ho.
\par 16 I otázal se zákoníku: Co se hádáte spolu?
\par 17 A odpovídaje jeden z zástupu, rekl: Mistre, privedl jsem syna svého k tobe, kterýž má ducha nemého.
\par 18 Ten kdyžkoli jej pochopí, lomcuje jím, a on se sliní, a škripí zubami, a svadne. I rekl jsem ucedlníkum tvým, aby jej vyvrhli, a nemohli.
\par 19 A on odpovídaje jemu, rekl: Ó národe neverný! Ale dokudž s vámi budu? A dokudž vás trpeti budu? Privedte jej ke mne.
\par 20 I privedli ho k nemu. A jakž jej uzrel, hned jím duch lomcoval; a padna na zemi, válel se a slinil.
\par 21 I otázal se otce jeho: Dávno-li se jemu to stalo? A on rekl: Hned od detinství.
\par 22 A casto jím metal i na ohen i do vody, aby jej zahubil. Ale mužeš-li co, spomoz nám, slituje se nad námi.
\par 23 A Ježíš rekl jemu: Mužeš-li tomu veriti; všeckot jest možné verícímu.
\par 24 A ihned zvolav otec mládence toho s slzami, rekl: Verím, Pane, spomoz nedovere mé.
\par 25 Uzrev pak Ježíš, že se zástup sbíhá, primluvil duchu tomu necistému, rka jemu: Hluchý a nemý duše, já tobe prikazuji, vyjdi z neho, a nevcházej více do neho.
\par 26 Tedy krice a velmi jím lomcuje, vyšel z neho. I ucinen jest clovek ten jako mrtvý, takže mnozí pravili, že umrel.
\par 27 Ale Ježíš ujav jej za ruku, pozdvihl ho, a on vstal.
\par 28 A když všel do domu, ucedlníci jeho otázali ho soukromí: Procež jsme my ho nemohli vyvrci?
\par 29 I rekl jim: Toto pokolení nijakž nemuž vyhnáno býti, jediné modlitbou a postem.
\par 30 A jdouce odtud, šli skrze Galilei, a nechtel, aby kdo o tom vedel.
\par 31 Nebo ucil ucedlníky své a pravil jim: Že Syn cloveka dán bude v ruce lidské, a zamordujít jej, ale zamordován jsa, tretí den z mrtvých vstane.
\par 32 Oni pak nesrozumeli tomu povedení, a ostýchali se ho otázati.
\par 33 I prišel do Kafarnaum, a v dome byv, otázal se jich: Co jste na ceste mezi sebou rozjímali?
\par 34 A oni mlceli. Nebo hádali se byli na ceste mezi sebou, kdo by z nich byl vetší.
\par 35 A posadiv se, zavolal dvanácti, a dí jim: Chce-li kdo první býti, budet všech nejposlednejší a všech služebník.
\par 36 A vzav pacholátko, postavil je uprostred nich, a vzav je na lokty své, rekl jim:
\par 37 Kdož by koli jedno z takových dítek prijal ve jménu mém, mnet prijímá; a kdož by mne koli prijal, ne mnet prijímá, ale toho, kterýž mne poslal.
\par 38 I odpovedel mu Jan, rka: Mistre, videli jsme tam jednoho, an ve jménu tvém dábly vymítá, kterýž nechodí s námi; i bránili jsme mu, protože s námi nechodí.
\par 39 Ježíš pak rekl: Nebrantež mu. Nebt není žádného, kterýž by divy cinil ve jménu mém, at by mohl snadne zle mluviti o mne.
\par 40 Nebo kdož není proti nám, s námit jest.
\par 41 Kdož by koli zajisté dal vám píti cíši vody ve jménu mém, protože jste Kristovi, amen pravím vám, neztratít nikoli odplaty své.
\par 42 A kdožt by koli pohoršil jednoho z techto malických, verících ve mne, mnohem by lépe mu bylo, aby byl zavešen na hrdlo jeho žernov mlýnský a vržen byl do more.
\par 43 A horšila-li by te ruka tvá, utni ji. Lépet jest tobe bezrukému vjíti do života, radeji nežli obe ruce majícímu jíti do pekla, v ohen neuhasitelný,
\par 44 Kdež cerv jejich neumírá a ohen nehasne.
\par 45 A pakli noha tvá horšila by te, utniž ji. Lépet jest tobe kulhavému vjíti do života, nežli obe noze majícímu uvrženu býti do pekla, v ohen neuhasitelný,
\par 46 Kdežto cerv jejich neumírá a ohen nehasne.
\par 47 Pakli by te oko tvé horšilo, vylup je. Lépet jest tobe jednookému vjíti do království Božího, nežli obe oci majícímu uvrženu býti do ohne pekelného,
\par 48 Kdežto cerv jejich neumírá a ohen nehasne.
\par 49 Nebo každý clovek ohnem bude solen, a všeliká obet solí bude osolena.
\par 50 Dobrát jest sul. Pakli sul bude neslaná, cím ji osolíte? Mejte sul v sobe sami, a pokoj mejte mezi sebou.

\chapter{10}

\par 1 A vstav odtud, prišel do koncin Judských skrze krajinu za Jordánem ležící. I sešli se k nemu zase zástupové, a jakž obycej mel, opet je ucil.
\par 2 Tedy pristoupivše farizeové, otázali se ho: Sluší-li muži ženu propustiti? pokoušejíce ho.
\par 3 On pak odpovídaje, rekl jim: Co vám prikázal Mojžíš?
\par 4 Kterížto rekli: Mojžíš dopustil lístek zapuzení napsati a propustiti.
\par 5 I odpovedev Ježíš, rekl jim: Pro tvrdost srdce vašeho napsal vám Mojžíš to prikázání.
\par 6 Ale od pocátku stvorení muže a ženu ucinil je Buh.
\par 7 Protot opustí clovek otce svého i matku, a prídržeti se bude ženy své.
\par 8 I budou dva jedno telo. A tak již nejsou dva, ale jedno telo.
\par 9 Protož což Buh spojil, clovek nerozlucuj.
\par 10 A v domu opet ucedlníci jeho otázali se ho o též veci.
\par 11 I dí jim: Kdož by koli propustil manželku svou a jinou pojal, cizoloží a hreší proti ní.
\par 12 A jestliže by žena propustila muže svého a za jiného se vdala, cizoloží.
\par 13 Tedy prinášeli k nemu dítky, aby se jich dotýkal. Ale ucedlníci primlouvali tem, kteríž je nesli.
\par 14 To videv Ježíš, nelibe to nesl, a rekl jim: Nechtež dítek jíti ke mne a nebrantež jim, nebo takovýcht jest království Boží.
\par 15 Amen pravím vám: Kdož by koli neprijal království Božího jako díte, nikolit do neho nevejde.
\par 16 A bera je na lokty své a vzkládaje na ne ruce, požehnání jim dával.
\par 17 Potom když vyšel na cestu, pribehl jeden, a poklekna pred ním, otázal se ho, rka: Mistre dobrý, co uciním, abych života vecného dedicne došel?
\par 18 I rekl mu Ježíš: Co mne nazýváš dobrým? Žádný není dobrý, než sám toliko Buh.
\par 19 Prikázání umíš: Nezcizoložíš, nezabiješ, neukradneš, nevydáš falešného svedectví, neoklamáš, cti otce svého i matku.
\par 20 A on odpovedev, rekl jemu: Mistre, toho všeho jsem ostríhal od své mladosti.
\par 21 Tedy Ježíš pohledev na nej, zamiloval ho, a rekl mu: Jednohot se nedostává. Jdi, a cožkoli máš, prodej, a dej chudým, a budeš míti poklad v nebi; a pojd, následuj mne, vezma kríž svuj.
\par 22 On pak zarmoutiv se pro to slovo, odšel, truchliv jsa; nebo mel mnohá zboží.
\par 23 A pohledev vukol Ježíš, dí ucedlníkum svým: Aj jak nesnadne ti, jenž statky mají, vejdou do království Božího.
\par 24 Tedy ucedlníci užasli se nad recmi jeho. Ježíš pak zase odpovedev, dí jim: Synáckové, kterak nesnadné jest doufajícím v statek do království Božího vjíti.
\par 25 Snáze jest velbloudu skrze jehelní ucho projíti, nežli bohatému vjíti do království Božího.
\par 26 Oni pak více se desili, rkouce mezi sebou: I kdož muže spasen býti?
\par 27 A pohledev na ne Ježíš, dí: U lidít jest nemožné, ale ne u Boha; nebo u Boha všecko možné jest.
\par 28 I pocal Petr mluviti k nemu: Aj, my opustili jsme všecko a šli jsme za tebou.
\par 29 Odpovedev pak Ježíš, rekl: Amen pravím vám, žádného není, ješto by opustil dum, neb bratrí, nebo sestry, neb otce, nebo matku, nebo manželku, nebo syny, nebo rolí pro mne a pro evangelium,
\par 30 Aby nevzal stokrát tolik nyní v casu tomto domu a bratru a sestr a matek a synu a rolí s protivenstvím, a v budoucím veku život vecný.
\par 31 Mnozít zajisté byvše první, budou poslední, a poslední první.
\par 32 Byli pak na ceste, jdouce do Jeruzaléma, a Ježíš šel napred. I byli predešeni, a jdouce za ním, báli se. Tedy pojav Ježíš opet dvanácte, pocal jim praviti, co se jemu má státi,
\par 33 Rka: Aj, vstupujeme do Jeruzaléma, a Syn cloveka vydán bude predním knežím a zákoníkum, i odsoudí jej na smrt, a vydadí jej pohanum.
\par 34 Kterížto posmívati se budou jemu, a ubicují ho, a uplijí a zabijí jej, ale tretího dne z mrtvých vstane.
\par 35 Tedy pristoupili k nemu Jakub a Jan, synové Zebedeovi, rkouce: Mistre, chceme, zac bychom koli prosili tebe, abys ucinil nám.
\par 36 On pak rekl jim: Co chcete, abych vám ucinil?
\par 37 I rekli jemu: Dej nám, abychom jeden na pravici tvé a druhý na levici tvé sedeli v sláve tvé.
\par 38 Ježíš pak rekl jim: Nevíte, zac prosíte. Mužete-li píti kalich, kterýž já piji, a krtíti se krtem, kterýmž já se krtím?
\par 39 A oni rekli jemu: Mužeme. A Ježíš rekl jim: Kalich zajisté, kterýž já piji, píti budete, a krtem, kterýmž já se krtím, krteni budete,
\par 40 Ale sedeti na pravici mé, nebo na levici mé, nenít má vec dáti vám, ale dánot bude, kterýmž pripraveno jest.
\par 41 A uslyšavše to jiných deset, pocali se hnevati na Jakuba a na Jana.
\par 42 Ale Ježíš povolav jich, rekl jim: Víte, že ti, kteríž sobe zalibují vládnouti nad národy, panujít nad nimi; a kteríž velicí u nich jsou, moc provozují nad nimi.
\par 43 Ne takt bude mezi vámi. Ale kdožkoli chtel by mezi vámi býti veliký, budiž váš služebník.
\par 44 A kdožkoli z vás chtel by býti prední, budiž služebník všech.
\par 45 Nebo i Syn cloveka neprišel, aby mu sloužili, ale aby on sloužil, a aby dal duši svou na vykoupení za mnohé.
\par 46 Tedy prišli do Jericho, a když vycházel on z Jericha, i ucedlníci jeho a zástup mnohý, Timeuv syn, Bartimeus slepý, sedel podle cesty, žebre.
\par 47 A když uslyšel, že by to byl Ježíš Nazaretský, pocal volati a ríci: Ježíši, synu Daviduv, smiluj se nade mnou.
\par 48 I primlouvali mu mnozí, aby mlcel. Ale on mnohem více volal: Synu Daviduv, smiluj se nade mnou.
\par 49 Tedy zastaviv se Ježíš, kázal ho zavolati. I zavolali toho slepého, rkouce jemu: Dobré mysli bud, vstan, volá te.
\par 50 On pak povrh plášt svuj, a zchopiv se, šel k Ježíšovi.
\par 51 I odpovedev Ježíš, dí jemu: Co chceš, at uciním? A slepý rekl jemu: Mistre, at vidím.
\par 52 Tedy Ježíš rekl mu: Jdi, víra tvá te uzdravila. A on hned prohlédl, a šel cestou za Ježíšem.

\chapter{11}

\par 1 A když se priblížili k Jeruzalému a Betfagi i Betany pri hore Olivetské, poslal dva z ucedlníku svých,
\par 2 A rekl jim: Jdete do hrádku, kterýž proti vám jest, a hned vejdouce tam, naleznete oslátko privázané, na kterémž ješte nižádný z lidí nesedel. Odvížíce, privedte ke mne.
\par 3 A rekl-lit by vám kdo: Co to ciníte? rcete: Že ho Pán potrebuje. A hned je propustí sem.
\par 4 I odešli, a nalezli oslátko privázané vne u dverí na rozcestí. I odvázali je.
\par 5 Tedy nekterí z tech, kteríž tu stáli, rekli jim: Co ciníte, odvazujíce oslátko?
\par 6 Oni pak rekli jim, jakož byl prikázal Ježíš. I nechali jich.
\par 7 Protož privedli oslátko k Ježíšovi, a vložili na ne roucha svá. I vsedl na ne.
\par 8 Mnozí pak stlali roucha svá na ceste, a jiní ratolesti sekali z stromu, a metali na cestu.
\par 9 A kteríž napred šli, i ti, kteríž za ním šli, volali, rkouce: Spas nás. Požehnaný, jenž se bére ve jménu Páne.
\par 10 Požehnané, kteréž jest prišlo ve jménu Páne, království otce našeho Davida! Spas nás na výsostech.
\par 11 I všel do Jeruzaléma Ježíš, i do chrámu. A spatriv tu všecko, když již byla vecerní hodina, vyšel do Betany se dvanácti.
\par 12 A druhého dne, když vycházel z Betany, zlacnel.
\par 13 A uzrev zdaleka fík, an má listí, šel, zda by co nalezl na nem. A když prišel k nemu, nic nenalezl krome listí; nebo nebyl cas fíku.
\par 14 Tedy odpovedev Ježíš, rekl jemu: Již více na veky nižádný z tebe ovoce nejez. A slyšeli to ucedlníci jeho.
\par 15 I prišli do Jeruzaléma. A všed Ježíš do chrámu, pocal vymítati ty, jenž prodávali a kupovali v chráme, a stoly penezomencu a stolice prodávajících holuby prevracel.
\par 16 A nedopustil, aby kdo jakou nádobu nesl skrze chrám.
\par 17 I ucil je, rka jim: Zdaliž není psáno, že dum muj dum modlitby slouti bude u všech národu? Vy pak ucinili jste jej peleší lotru.
\par 18 Slyšeli pak to zákoníci i prední kneží, a hledali, kterak by jej zahubili; nebo se ho báli, protože všecken zástup divil se ucení jeho.
\par 19 A když byl vecer, vyšel Ježíš z mesta.
\par 20 A ucedlníci ráno jdouce, uzreli fík, an usechl hned z korene.
\par 21 Tedy zpomenuv Petr, rekl jemu: Mistre, aj fík, kterémuž jsi zlorecil, usechl.
\par 22 I odpovedev Ježíš, rekl jim: Mejte víru Boží.
\par 23 Nebo amen pravím vám, že kdož by koli rekl hore této: Zdvihni se a vrz sebou do more, a nepochyboval by v srdci svém, ale veril by, že se stane, cožkoli dí, budet jemu tak, což by koli rekl.
\par 24 Protož pravím vám: Zacež byste koli, modléce se, prosili, verte, že vezmete, a stanet se vám.
\par 25 A když se postavíte k modlení, odpouštejte, máte-li co proti komu, aby i Otec váš nebeský odpustil vám hríchy vaše.
\par 26 Nebo jestliže vy neodpustíte, ani Otec váš, kterýž v nebesích jest, odpustí vám hríchu vašich.
\par 27 I prišli zase do Jeruzaléma. A když on chodil v chráme, pristoupili k nemu prední kneží a zákoníci a starší.
\par 28 I rekli jemu: Jakou mocí to ciníš? A kdo jest tobe dal tu moc, abys tyto veci cinil?
\par 29 Tedy odpovídaje Ježíš, rekl jim: Otížit se i já vás na jednu vec. Odpovezte mi, a povím vám, jakou mocí to ciním.
\par 30 Krest Januv s nebe-li byl, cili z lidí? Odpovezte mi.
\par 31 I rozvažovali to sami mezi sebou, rkouce: Díme-li: S nebe, dít nám: Procež jste tedy neuverili jemu?
\par 32 Pakli díme: Z lidí, bojíme se lidu. Nebo všickni o Janovi smyslili, že jest práve byl prorok.
\par 33 I odpovedevše, rekli Ježíšovi: Nevíme. A Ježíš odpovídaje, rekl jim: Aniž já vám povím, jakou mocí to ciním.

\chapter{12}

\par 1 Tedy pocal jim mluviti v podobenstvích: Vinici štípil jeden clovek, a opletl ji plotem, a vkopal pres, a ustavel veži, a najal ji vinarum, i odšel na cestu.
\par 2 A v cas užitku poslal k vinarum služebníka, aby od vinaru vzal ovoce z vinice.
\par 3 Oni pak javše jej, zmrskali ho a odeslali prázdného.
\par 4 I poslal k nim zase jiného služebníka. I toho též kamenovavše, ranili v hlavu a odeslali zohaveného.
\par 5 I poslal opet jiného. I toho zabili, a mnoho jiných, z nichž nekteré zmrskali a jiné zmordovali.
\par 6 Ješte pak maje jediného syna nejmilejšího, i toho poslal k nim naposledy, rka: Ostýchati se budou syna mého.
\par 7 Ale vinari rekli jedni k druhým: Tentot jest dedic; pojdte, zabijme jej, a budet naše dedictví.
\par 8 Tedy javše jej, zabili ho a vyvrhli ven z vinice.
\par 9 Což tedy uciní pán vinice? Prijde, a zatratí vinare ty, a dá vinici jiným.
\par 10 Zdaliž jste písma toho nectli? Kámen, kterýž zavrhli delníci, ten ucinen jest hlavou úhlovou.
\par 11 Ode Pána stalo se toto, a jest divné pred ocima našima.
\par 12 I hledali ho jíti, ale báli se zástupu; nebo poznali, že podobenství to proti nim povedel. A nechavše ho, odešli pryc.
\par 13 Potom poslali k nemu nekteré z farizeu a herodiánu, aby jej polapili v reci.
\par 14 Kterížto prišedše, rekli jemu: Mistre, víme, že jsi pravdomluvný, a nedbáš na žádného; nebo nepatríš na osobu lidskou, ale v pravde ceste Boží ucíš. Sluší-li dan dávati císari, cili nic? Dáme-liž, cili nedáme?
\par 15 On pak znaje pokrytství jejich, rekl jim: Co mne pokoušíte? Prineste mi peníz, at pohledím.
\par 16 A oni podali mu. Tedy rekl jim: Cí jest tento obraz a nápis? A oni rekli mu: Císaruv.
\par 17 I odpovedev Ježíš, rekl jim: Dávejtež tedy, což jest císarova, císari, a což jest Božího, Bohu. I podivili se tomu.
\par 18 Potom prišli k nemu saduceové, kteríž praví, že není z mrtvých vstání. I otázali se ho, rkouce:
\par 19 Mistre, Mojžíš nám napsal: Kdyby cí bratr umrel, a ostavil po sobe manželku, a synu by nemel, aby bratr jeho pojal manželku jeho a vzbudil síme bratru svému.
\par 20 I bylo sedm bratru. A první pojav ženu, umrel, neostaviv semene.
\par 21 A druhý pojav ji, také umrel, a aniž ten ostavil semene. A tretí tolikéž.
\par 22 A tak ji pojalo všech sedm, a nezustavili po sobe semene. Nejposléze pak po všech umrela i žena.
\par 23 Protož pri vzkríšení, když z mrtvých vstanou, cí z tech bude manželka? Neb jich sedm melo ji za manželku.
\par 24 A odpovídaje Ježíš, rekl jim: Zdaliž ne proto bloudíte, že neznáte Písem ani moci Boží?
\par 25 Nebo když vstanou z mrtvých, nebudou se ženiti ani vdávati, ale budou jako andelé nebeští.
\par 26 O mrtvých pak, že mají vstáti, zdaliž jste nectli v knihách Mojžíšových, kterak ve kri promluvil k nemu Buh, rka: Já jsem Buh Abrahamuv a Buh Izákuv a Buh Jákobuv?
\par 27 Nenít Buh mrtvých, ale Buh živých. Protož vy velmi bloudíte.
\par 28 Tedy pristoupil k nemu jeden z zákoníku, slyšev je hádající se, a vida, že jim dobre odpovedel, otázal se ho, které by bylo prikázání první ze všech.
\par 29 A Ježíš odpovedel jemu, že první ze všech prikázání jest: Slyš, Izraeli, Pán Buh náš Pán jeden jest.
\par 30 Protož milovati budeš Pána Boha svého ze všeho srdce svého, a ze vší duše své, a ze vší mysli své, i ze všech mocí svých. To jest první prikázání.
\par 31 Druhé pak jest podobné tomu: Milovati budeš bližního svého jako sebe samého. Vetšího prikázání jiného nad tato není.
\par 32 I rekl jemu ten zákoník: Mistre, dobre jsi vpravde povedel. Nebo jeden jest Buh, a není jiného krome neho;
\par 33 A milovati ho ze všeho srdce, a ze vší mysli, a ze vší duše, i ze všech mocí, a milovati bližního jako sebe samého, tot jest vetší nade všecky zápalné i vítezné obeti.
\par 34 A videv Ježíš, že by moudre odpovedel, dí jemu: Nejsi daleko od království Božího. A žádný více neodvážil se ho o nic tázati.
\par 35 I odpovídaje Ježíš, rekl, uce v chráme: Kterak praví zákoníci, že Kristus jest syn Daviduv?
\par 36 Nebo David praví v Duchu svatém: Rekl Pán Pánu mému: Sed na pravici mé, ažt i položím neprátely tvé podnože noh tvých.
\par 37 Ponevadž sám David nazývá jej Pánem, kterakž syn jeho jest? A mnohý zástup rád ho poslouchal.
\par 38 I mluvil jim v ucení svém: Varujte se zákoníku, kteríž rádi v krásném rouše chodí, a chtejí pozdravováni býti na trhu,
\par 39 A na predních stolicích sedeti v školách, a prední místa míti na vecerech,
\par 40 Kterížto zžírají domy vdovské pod zámyslem dlouhých modliteb. Tit vezmou soud težší.
\par 41 A posadiv se Ježíš proti pokladnici, díval se, kterak zástup metal peníze do pokladnice. A mnozí bohatí metali mnoho.
\par 42 A prišedši jedna chudá vdova, i vrhla dva šarty, jenž jest ctvrtá cástka peníze tehdejšího.
\par 43 I svolav ucedlníky své, dí jim: Amen pravím vám, že tato chudá vdova více uvrhla, než tito všickni, kteríž metali do pokladnice.
\par 44 Nebo všickni z toho, což jim zbývalo, metali, ale tato z své chudoby, všecko, což mela, uvrhla, všecku živnost svou.

\chapter{13}

\par 1 A když vycházel z chrámu, dí jemu jeden z ucedlníku jeho: Mistre, pohled, kteraké kamení a jaké jest toto stavení!
\par 2 Tedy Ježíš odpovídaje, rekl jemu: Vidíš toto tak veliké stavení? Nebudet ostaven kámen na kameni, kterýž by nebyl zboren.
\par 3 A když se posadil na hore Olivetské proti chrámu, otázali se jeho obzvláštne Petr, Jakub a Jan a Ondrej, rkouce:
\par 4 Povez nám, kdy to bude? A které znamení, když se toto všecko bude plniti?
\par 5 Ježíš pak odpovídaje jim, pocal praviti: Vizte, aby vás nekdo nesvedl.
\par 6 Nebot mnozí prijdou ve jménu mém, rkouce: Já jsem Kristus, a mnohét svedou.
\par 7 Když pak uslyšíte boje a povest o válkách, nestrachujte se; nebo musí to býti, ale ne ihned konec.
\par 8 Povstanet zajisté národ proti národu a království proti království, a bude zemetresení po místech, a hladové i bourky.
\par 9 A tot budou pocátkové bolesti. Vy pak šetrte se. Nebo vydávati vás budou na snemy a do shromáždení; budete biti, a pred vladari a králi stanete pro mne, na svedectví jim.
\par 10 Ale ve všech národech nejprv musí býti kázáno evangelium.
\par 11 Když pak vás povedou vyzrazujíce, nestarejte se, co byste mluvili, aniž o to peclive premyšlujte, ale což vám bude dáno v tu hodinu, to mluvte; nebo nejste vy, jenž mluvíte, ale Duch svatý.
\par 12 Vydát pak bratr bratra na smrt a otec syna, a povstanou deti proti rodicum, a budou je mordovati.
\par 13 A budete v nenávisti všechnem pro jméno mé. Ale kdož setrvá až do konce, tent spasen bude.
\par 14 Když pak uzríte ohavnost zpuštení, o kteréž povedíno jest skrze Daniele proroka, ana stojí, kdež by státi nemela, (kdo cte, rozumej,) tehdáž ti, kdož jsou v Židovstvu, at utekou na hory.
\par 15 A kdož na streše jest, nesstupuj do domu, ani vcházej, aby co vzal z domu svého.
\par 16 A kdo na poli, nevracuj se zase, aby vzal roucho své.
\par 17 Beda pak tehotným a tem, kteréž krmí v tech dnech.
\par 18 Protož modlte se, aby utíkání vaše nebylo v zime.
\par 19 Nebot budou ti dnové plní takového soužení, jakéhož nebylo od pocátku stvorení, kteréž Buh stvoril, až dosavad, aniž potom bude.
\par 20 A byt neukrátil Pán tech dnu, nebyl by spasen žádný clovek. Ale pro vyvolené, kteréž vyvolil, ukrátil tech dnu.
\par 21 A tehdáž rekl-li by vám kdo: Aj ted jest Kristus, aneb, aj tamto, neverte.
\par 22 Nebot povstanou falešní Kristové a falešní proroci, a budou ciniti divy a zázraky k svedení, by možné bylo, také i vyvolených.
\par 23 Vy pak šetrte se. Aj, predpovedel jsem vám všecko.
\par 24 V tech pak dnech, po soužení tom, slunce se zatmí a mesíc nedá svetla svého.
\par 25 A hvezdy nebeské budou padati, a moci, které jsou na nebi, pohnou se.
\par 26 A tehdážt uzrí Syna cloveka, an se bére v oblacích s mocí velikou a s slavou.
\par 27 I tehdyt pošle andely své, a shromáždí vyvolené své ode ctyr vetru, od koncin zeme až do koncin nebe.
\par 28 Od fíku pak ucte se podobenství: Když již ratolest jeho odmladne a vypucí se listí, znáte, že blízko jest léto.
\par 29 Takž i vy, když uzríte, ano se tyto veci dejí, vezte, že blízko jest a ve dverích království Boží.
\par 30 Amen pravím vám, žet nepomine pokolení toto, až se tyto všecky veci stanou.
\par 31 Nebe a zeme pominou, ale slova má nepominou.
\par 32 Ale o tom dni a hodine žádný neví, ani andelé, jenž jsou v nebesích, ani Syn, jediné sám Otec.
\par 33 Vizte, bdete a modlte se; nebo nevíte, kdy bude ten cas.
\par 34 Syn cloveka zajisté jest jako clovek, kterýž daleko odšel, opustiv dum svuj, a poruciv služebníkum svým vladarství, a jednomu každému práci jeho, vrátnému prikázal, aby bdel.
\par 35 Protož bdete; nebo nevíte, kdy Pán domu prijde, u vecer-li, cili o pulnoci, cili když kohouti zpívají, cili ráno;
\par 36 Aby snad prijda v nenadále, nenalezl vás, a vy spíte.
\par 37 A cožt vám pravím, všechnemt pravím: Bdete.

\chapter{14}

\par 1 Po dvou pak dnech byl hod beránka a presnic; i hledali prední kneží a zákoníci, kterak by jej lstive jmouce, zamordovali.
\par 2 Ale pravili: Ne v svátek, aby snad nebyl rozbroj v lidu.
\par 3 A když byl v Betany, v domu Šimona malomocného, a sedel za stolem, prišla žena, mající nádobu alabastrovou masti velmi drahé, z nardového korení. A rozbivši nádobu, vylila ji na hlavu jeho.
\par 4 I hnevali se nekterí mezi sebou, rkouce: I proc ztráta masti této stala se?
\par 5 Nebo mohlo jest toto prodáno býti dráže než za tri sta penez, a dáno býti chudým. I škripeli zubami na ni.
\par 6 Ale Ježíš rekl: Nechtež jí. Proc ji rmoutíte? Dobrýt jest skutek ucinila nade mnou.
\par 7 Však chudé máte vždycky s sebou, a když budete chtíti, mužete jim dobre ciniti, ale mne ne vždy míti budete.
\par 8 Ona což mohla, to ucinila; predešlat jest, aby tela mého pomazala ku pohrebu.
\par 9 Amen pravím vám: Kdežkoli bude kázáno toto evangelium po všem svete, také i to, což ucinila tato, bude vypravováno na památku její.
\par 10 Tedy Jidáš Iškariotský, jeden ze dvanácti, odšel k predním knežím, aby ho jim zradil.
\par 11 Oni pak uslyševše to, zradovali se, a slíbili mu peníze dáti. I hledal, kterak by ho príhodne zradil.
\par 12 Prvního pak dne presnic, když velikonocní beránek zabíjín býval, rkou jemu ucedlníci jeho: Kde chceš, at jdouce, pripravíme, abys jedl beránka?
\par 13 I poslal dva z ucedlníku svých, a rekl jim: Jdete do mesta, a potkát vás clovek dcbán vody nesa. Jdetež za ním.
\par 14 A kamžkoli vejde, rcete k hospodári domu toho: Mistrt praví: Kde jest veceradlo, v nemž bych jedl beránka s ucedlníky svými?
\par 15 A on vám ukáže veceradlo veliké, podlážené a pripravené. Tu nám pripravte.
\par 16 I odešli ucedlníci jeho, a prišli do mesta, a nalezli tak, jakož jim byl povedel. I pripravili beránka.
\par 17 Když pak byl vecer, prišel se dvanácti.
\par 18 A když sedeli za stolem a jedli, rekl Ježíš: Amen pravím vám, že jeden z vás mne zradí, kterýž jí se mnou.
\par 19 A oni pocali se rmoutiti a praviti jemu jeden každý obzvláštne: Zdali já jsem? A jiný: Zdali já?
\par 20 On pak odpovedev, rekl jim: Jeden ze dvanácti, kterýž omácí se mnou v míse.
\par 21 Syn zajisté cloveka jde, jakož jest psáno o nem, ale beda cloveku tomu, skrze nehož Syn cloveka bude zrazen. Dobré by bylo jemu, aby se byl nenarodil clovek ten.
\par 22 A když oni jedli, vzav Ježíš chléb, a dobroreciv, lámal a dával jim, rka: Vezmete, jezte, to jest telo mé.
\par 23 A vzav kalich, a díky uciniv, dal jim. A pili z neho všickni.
\par 24 I rekl jim: To jest krev má nového Zákona, kteráž se za mnohé vylévá.
\par 25 Amen pravím vám, žet již více nebudu píti z plodu vinného korene, až do onoho dne, když jej píti budu nový v království Božím.
\par 26 A sezpívavše písnicku, vyšli na horu Olivetskou.
\par 27 Potom rekl jim Ježíš: Všickni vy zhoršíte se nade mnou této noci. Nebo psáno jest: Bíti budu pastýre, a rozprchnou se ovce.
\par 28 Ale když z mrtvých vstanu, predejdut vás do Galilee.
\par 29 Tedy Petr rekl jemu: Byt se pak všickni zhoršili, ale já nic.
\par 30 Rekl jemu Ježíš: Amen pravím tobe, že dnes této noci, prve než kohout po dvakrát zazpívá, trikrát mne zapríš.
\par 31 On pak mnohem více mluvil: Bycht pak mel s tebou i umríti, nezaprímt tebe. A takž také i všickni mluvili.
\par 32 I prišli na místo, kterémuž jméno Getsemany. Tedy rekl ucedlníkum svým: Sedtež tuto, ažt se pomodlím.
\par 33 A pojav s sebou Petra a Jakuba a Jana, pocal se lekati a velmi teskliv býti.
\par 34 I dí jim: Smutnát jest duše má až k smrti. Pocekejtež tuto a bdete.
\par 35 A poodšed malicko, padl na zemi a modlil se, aby, bylo-li by možné, odešla od neho hodina ta.
\par 36 I rekl: Abba, Otce, všecko jest možné tobe. Prenes kalich tento ode mne, ale však ne, což já chci, ale co ty.
\par 37 I prišel k ucedlníkum a nalezl je, ani spí. I rekl Petrovi: Šimone, spíš? Nemohl-lis jediné hodiny bdíti?
\par 38 Bdete a modlte se, abyste nevešli v pokušení. Ducht zajisté hotov jest, ale telo nemocno.
\par 39 A opet odšed, modlil se, táž slova mluve.
\par 40 A navrátiv se, nalezl je, ani opet spí; nebo oci jejich byly obtíženy; aniž vedeli, co by jemu odpovedeli.
\par 41 I prišel po tretí, a rekl jim: Spetež již a odpocívejte; dostit jest. Prišla hodina; aj, Syna cloveka zrazují v ruce hríšných.
\par 42 Vstante, pojdme. Aj, kterýž mne zrazuje, blízkot jest.
\par 43 A hned, když on ješte mluvil, prišel Jidáš, jenž byl jeden ze dvanácti, a s ním zástup veliký s meci a s kyjmi, poslaných od predních kneží a od zákoníku a starších.
\par 44 Zrádce pak byl jim dal znamení, rka: Kteréhožkoli políbím, tent jest, jmetež ho a vedte opatrne.
\par 45 A prišed, hned pristoupiv k nemu, rekl: Mistre, Mistre, a políbil ho.
\par 46 Tedy oni vztáhli nan ruce své a jali jej.
\par 47 Jeden pak z tech, kteríž tu okolo stáli, vytrh mec, uderil služebníka nejvyššího kneze, a utal jemu ucho.
\par 48 I odpovedev Ježíš, rekl jim: Jako na lotra vyšli jste s meci a s kyjmi, abyste mne jali?
\par 49 Na každý den býval jsem u vás, uce v chráme, a nejali jste mne. Ale tot se deje, aby se naplnila písma.
\par 50 Tedy ucedlníci opustivše jej, všickni utekli.
\par 51 Jeden pak mládencek šel za ním, odín jsa rouchem lneným po nahém tele. I popadli jej mládenci.
\par 52 On pak opustiv roucho, nahý utekl od nich.
\par 53 I privedli Ježíše k nejvyššímu knezi, a sešli se k nemu všickni prední kneží i starší i zákoníci.
\par 54 Petr pak šel za ním zdaleka až na dvur nejvyššího kneze; i sedel s služebníky, zhrívaje se u ohne.
\par 55 Ale nejvyšší knez i všecka ta rada hledali proti Ježíšovi svedectví, aby jej na smrt vydali, avšak nenalezli.
\par 56 Nebo ac mnozí krivé svedectví mluvili proti nemu, ale svedectví jejich nebyla jednostejná.
\par 57 Tedy nekterí povstavše, krivé svedectví dávali proti nemu, rkouce:
\par 58 My jsme slyšeli tohoto, že rekl: Já zborím chrám tento rukou udelaný, a ve trech dnech jiný ne rukou udelaný postavím.
\par 59 Ale ani to jejich svedectví nebylo jednostejné.
\par 60 Tedy povstav nejvyšší knez uprostred, otázal se Ježíše, rka: Neodpovídáš nicehož, což tito na tebe svedcí?
\par 61 Ale on mlcel, a nic neodpovedel. Opet nejvyšší knez otázal se ho a rekl jemu: Jsi-liž ty Kristus, Syn Boha Požehnaného?
\par 62 A Ježíš rekl: Ját jsem, a uzríte Syna cloveka, an sedí na pravici moci Boží, a prichází s oblaky nebeskými.
\par 63 Tedy nejvyšší knez roztrh roucho své, rekl: I což ješte potrebujeme svedku?
\par 64 Slyšeli jste rouhání. Co se vám zdá? Oni pak všickni odsoudili jej, že jest hoden smrti.
\par 65 I pocali nekterí nan plvati, a tvár jeho zakrývati, a polickovati, a ríkati jemu: Prorokuj nám. A služebníci kyji jej bili.
\par 66 A když byl Petr v síni dole, prišla jedna z devecek nejvyššího kneze.
\par 67 A uzrevši Petra, an se ohrívá, a popatrivši nan, dí: I ty s Ježíšem Nazaretským byl jsi.
\par 68 Ale on zaprel, rka: Aniž vím, ani rozumím, co ty pravíš. I vyšel ven pred sín, a kohout zazpíval.
\par 69 Tedy devecka, uzrevši jej opet, pocala praviti tem, kteríž tu okolo stáli, že tento z nich jest.
\par 70 A on opet zaprel. A po malé chvíli opet ti, kteríž tu stáli, rekli Petrovi: Jiste z nich jsi, nebo i Galilejský jsi, i rec tvá podobná jest.
\par 71 On pak pocal se proklínati a prisahati, prave: Neznám cloveka toho, o nemž vy pravíte.
\par 72 A hned po druhé kohout zazpíval. I rozpomenul se Petr na slovo, kteréž byl rekl jemu Ježíš: Že prve než kohout dvakrát zazpívá, trikrát mne zapríš. A vyšed, plakal.

\chapter{15}

\par 1 A hned ráno uradivše se prední kneží s staršími a s zákoníky i se vším shromáždením, svázavše Ježíše, vedli jej a dali Pilátovi.
\par 2 I otázal se ho Pilát: Ty-liž jsi král Židovský? A on odpovedev, rekl jemu: Ty pravíš.
\par 3 I žalovali na nej prední kneží mnoho. On pak nic neodpovídal.
\par 4 Tedy Pilát otázal se ho opet, rka: Nic neodpovídáš? Hle, jak mnoho proti tobe svedcí.
\par 5 Ale Ježíš predce nic neodpovedel, takže se podivil Pilát.
\par 6 Ve svátek pak propouštíval jim jednoho z veznu, za kteréhož by prosili.
\par 7 I byl jeden, kterýž sloul Barabbáš, jenž s svárlivými byl v vezení, kteríž v svade vraždu byli spáchali.
\par 8 A zvolav zástup, pocal prositi, aby ucinil, jakož jim vždycky ciníval.
\par 9 Pilát pak odpovedel jim, rka: Chcete-li, propustím vám krále Židovského?
\par 10 (Nebo vedel, že jsou jej z závisti vydali prední kneží.)
\par 11 Ale prední kneží ponukli zástupu, aby jim radeji propustil Barabbáše.
\par 12 A Pilát odpovedev, rekl jim zase: Což pak chcete, at uciním tomu, kteréhož králem Židovským nazýváte?
\par 13 A oni opet zvolali: Ukrižuj ho.
\par 14 A Pilát pravil jim: I což jest zlého ucinil? Oni pak více volali: Ukrižuj ho.
\par 15 Tedy Pilát, chte lidu dosti uciniti, pustil jim Barabbáše, a dal jim Ježíše ubicovaného, aby byl ukrižován.
\par 16 Žoldnéri pak uvedli jej vnitr do síne, do radného domu, a svolali všecku sber.
\par 17 I oblékli jej v šarlat, a korunu spletše z trní, vložili nan.
\par 18 I pocali ho pozdravovati, rkouce: Zdráv bud, králi Židovský.
\par 19 A bili hlavu jeho trtinou, a plvali na nej, a sklánejíce kolena, klaneli se jemu.
\par 20 A když se jemu naposmívali, svlékli s neho šarlat, a oblékli jej v roucho jeho vlastní. I vedli jej, aby ho ukrižovali.
\par 21 I prinutili nejakého Šimona Cyrenenského, pomíjejícího je, (kterýž šel z pole, otce Alexandrova a Rufova,) aby vzal kríž jeho.
\par 22 I vedli jej až na místo Golgota, to jest, (vyložil-li by,) popravné místo.
\par 23 I dávali mu píti víno s mirrou, ale on neprijal ho.
\par 24 A ukrižovavše jej, rozdelili roucha jeho, mecíce o ne los, kdo by co vzíti mel.
\par 25 A byla hodina tretí, když ho ukrižovali.
\par 26 A byl nápis viny jeho napsán temi slovy: Král Židovský.
\par 27 Ukrižovali také s ním dva lotry: jednoho na pravici a druhého na levici jeho.
\par 28 I naplneno jest písmo, rkoucí: A s nepravými pocten jest.
\par 29 A kteríž tudy chodili mimo nej, rouhali se jemu, potrásajíce hlavami svými, a ríkajíce: Hahá, kterýž rušíš chrám Boží, a ve trech dnech jej zase vzdeláváš,
\par 30 Spomoz sobe samému, a sstup s kríže.
\par 31 Též i prední kneží posmívajíce se, jeden k druhému s zákoníky pravili: Jinýmt jest pomáhal, sám sobe pomoci nemuže.
\par 32 Kristus král Izraelský, nechažt nyní sstoupí s kríže, at uzríme a uveríme. A i ti, kteríž s ním ukrižováni byli, útržku mu cinili.
\par 33 A když byla hodina šestá, stala se tma po vší zemi až do hodiny deváté.
\par 34 A v hodinu devátou zvolal Ježíš hlasem velikým, rka: Elói, Elói, lama zabachtani? jenž se vykládá: Bože muj, Bože muj, procs mne opustil?
\par 35 A nekterí z okolo stojících, slyševše to, pravili: Hle, Eliáše volá.
\par 36 A bežev jeden, naplnil houbu octem a vloživ na trest, dával jemu píti, rka: Ponechte, uzríme, prijde-li Eliáš, aby jej složil.
\par 37 Ježíš pak zvolav hlasem velikým, pustil duši.
\par 38 A opona v chráme roztrhla se na dvé, od vrchu až dolu.
\par 39 Videv pak to centurio, kterýž naproti stál, že tak volaje, vypustil duši, rekl: Jiste clovek tento Syn Boží byl.
\par 40 Byly pak tu i ženy, zdaleka se dívajíce, mezi nimiž byla Maria Magdaléna, a Maria Jakuba menšího, a Jozesova máte, a Salome.
\par 41 Kteréž, když ješte byl v Galilei, chodily za ním a posluhovaly jemu, i jiné mnohé, kteréž byly s ním vstoupily do Jeruzaléma.
\par 42 A když již byl vecer, (že byl den pripravování, to jest pred sobotou,)
\par 43 Prišed Jozef z Arimatie, pocestná osoba úradná, kterýž také ocekával království Božího, smele všel ku Pilátovi a prosil za telo Ježíšovo.
\par 44 Pilát pak podivil se, již-li by umrel. A povolav centuriona, otázal se ho, dávno-li je umrel.
\par 45 A zvedev od centuriona, dal telo Jozefovi.
\par 46 A Jozef koupiv plátna, a složiv ho s kríže, obvinul v plátno, i položil do hrobu, kterýž byl vytesán z skály, a privalil kámen ke dverum hrobovým.
\par 47 Ale Maria Magdaléna a Maria Jozesova dívaly se, kde by byl položen.

\chapter{16}

\par 1 A když pominula sobota, Maria Magdaléna a Maria Jakubova a Salome nakoupily vonných vecí, aby prijdouce, pomazaly Ježíše.
\par 2 A velmi ráno vyšedše první den po sobote, prišly k hrobu, an již slunce vzešlo.
\par 3 I pravily vespolek: Kdo nám odvalí kámen ode dverí hrobových?
\par 4 (A vzhlédše, uzrely odvalený kámen.) Byl zajisté veliký velmi.
\par 5 A všedše do hrobu, uzrely mládence, an sedí na pravici, odeného rouchem bílým. I ulekly se.
\par 6 Kterýžto rekl jim: Nebojte se. Ježíše hledáte Nazaretského ukrižovaného. Vstalt jest, nenít ho tuto; aj, místo, kdež jej byli položili.
\par 7 Ale jdete, povezte ucedlníkum jeho i Petrovi, žet vás predejde do Galilee. Tam jej uzríte, jakož jest povedel vám.
\par 8 A ony vyšedše rychle, utekly od hrobu; nebo prišel na ne strach a hruza. A aniž komu co rekly, nebo se bály.
\par 9 Vstav pak Ježíš z mrtvých ráno v nedeli, ukázal se nejprv Mariji Magdaléne, z nížto byl vyvrhl sedm dáblu.
\par 10 Ona pak šedši, zvestovala tem, kteríž s ním bývali, lkajícím a placícím.
\par 11 A oni slyšavše, že by živ byl a vidín od ní, neverili.
\par 12 Potom pak dvema z nich jdoucím ukázal se v jiné zpusobe, když šli pres pole.
\par 13 A ti šedše, povedeli jiným. Ani tem neverili.
\par 14 Nejposléze sedícím spolu jedenácti ukázal se, a trestal nedoveru jejich a tvrdost srdce, že tem, kteríž jej videli vzkríšeného, neverili.
\par 15 A rekl jim: Jdouce po všem svete, kažte evangelium všemu stvorení.
\par 16 Kdož uverí a pokrtí se, spasen bude; kdož pak neuverí, budet zatracen.
\par 17 Znamení pak ti, kteríž uverí, tato míti budou: Ve jménu mém dábly budou vymítati, jazyky novými mluviti.
\par 18 Hady bráti; a jestliže by co jedovatého pili, neuškodít jim; na nemocné ruce vzkládati budou, a dobre se míti budou.
\par 19 Když pak jim odmluvil Pán, vzhuru vzat jest do nebe, a sedí na pravici Boží.
\par 20 A oni šedše, kázali všudy, a Pán jim pomáhal, a slov jejich potvrzoval cinením divu.


\end{document}