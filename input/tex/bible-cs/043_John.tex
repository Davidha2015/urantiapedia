\begin{document}

\title{Jan}

\chapter{1}

\par 1 Na pocátku bylo Slovo, a to Slovo bylo u Boha, a to Slovo byl Buh.
\par 2 To bylo na pocátku u Boha.
\par 3 Všecky veci skrze ne ucineny jsou, a bez neho nic není ucineno, což ucineno jest.
\par 4 V nem život byl, a život byl svetlo lidí.
\par 5 A to Svetlo v temnostech svítí, ale tmy ho neobsáhly.
\par 6 Byl clovek poslaný od Boha, jemuž jméno bylo Jan.
\par 7 Ten prišel na svedectví, aby svedcil o tom Svetle, aby všickni uverili skrze neho.
\par 8 Nebyl on to Svetlo, ale poslán byl, aby svedectví vydával o tom Svetle.
\par 9 Tent byl to pravé Svetlo, jenž osvecuje každého cloveka pricházejícího na svet.
\par 10 Na svete byl, a svet skrze neho ucinen jest, a svet ho nepoznal.
\par 11 Do svého vlastního prišel, a vlastní jeho neprijali ho.
\par 12 Kteríž pak koli prijali jej, dal jim moc syny Božími býti, totiž tem, kteríž verí ve jméno jeho,
\par 13 Kterížto ne ze krví, ani z vule tela, ani z vule muže, ale z Boha zplozeni jsou.
\par 14 A Slovo to telo ucineno jest, a prebývalo mezi námi, (a videli jsme slávu jeho, slávu jakožto jednorozeného od Otce,) plné milosti a pravdy.
\par 15 Jan svedectví vydával o nem, a volal, rka: Tentot jest, o nemž jsem pravil, že po mne prišed, predšel mne; nebo prednejší jest nežli já.
\par 16 A z plnosti jeho my všickni vzali jsme, a to milost za milost.
\par 17 Nebo zákon skrze Mojžíše dán jest, ale milost a pravda skrze Ježíše Krista stala se jest.
\par 18 Boha žádný nikdy nevidel, ale jednorozený Syn, kterýž jest v lunu Otce, ont jest nám vypravil.
\par 19 A totot jest svedectví Janovo, když poslali Židé z Jeruzaléma kneží a Levíty, aby se ho otázali: Ty kdo jsi?
\par 20 I vyznal a nezaprel, a vyznal: Že já nejsem ten Kristus.
\par 21 I otázali se ho: Což pak? Eliáš jsi ty? I rekl: Nejsem. Tedy jsi ten Prorok? Odpovedel: Nejsem.
\par 22 I rekli jemu: Kdožs pak? At odpoved dáme tem, kteríž nás poslali. Co pravíš sám o sobe?
\par 23 Rekl: Já jsem hlas volajícího na poušti: Spravte cestu Páne, jakož povedel Izaiáš prorok.
\par 24 Ti pak, kteríž byli posláni, z farizeu byli.
\par 25 I otázali se ho a rekli jemu: Procež tedy krtíš, ponevadž ty nejsi Kristus, ani Eliáš, ani Prorok?
\par 26 Odpovedel jim Jan, rka: Já krtím vodou, ale uprostred vás stojí, jehož vy neznáte.
\par 27 Ten ackoli po mne prišel, však predšel mne, u jehožto obuvi já nejsem hoden rozvázati reménka.
\par 28 Toto v Betabare stalo se za Jordánem, kdežto Jan krtil.
\par 29 Druhého pak dne uzrel Jan Ježíše, an jde k nemu. I dí: Aj, Beránek Boží, kterýž snímá hrích sveta.
\par 30 Tentot jest, o kterémž jsem já pravil, že za mnou jde muž, kterýž mne predšel; nebo prednejší jest nežli já.
\par 31 A já jsem ho neznal, ale aby zjeven byl lidu Izraelskému, proto jsem já prišel, krte vodou.
\par 32 A svedectví vydal Jan, rka: Videl sem Ducha sstupujícího jako holubice s nebe, a zustal na nem.
\par 33 A já jsem ho neznal, ale kterýž mne poslal krtíti vodou, ten mi rekl: Nad kýmž uzríš Ducha sstupujícího a zustávajícího na nem, tent jest, kterýž krtí Duchem svatým.
\par 34 A já jsem videl, a svedectví jsem vydal, že on jest ten Syn Boží.
\par 35 Druhého pak dne opet stál Jan, a z ucedlníku jeho dva,
\par 36 A uzrev Ježíše, an se prochází, rekl: Aj, Beránek Boží.
\par 37 I slyšeli ho dva ucedlníci mluvícího, a šli za Ježíšem.
\par 38 I obrátiv se Ježíš, a uzrev je, ani jdou za ním, dí jim: Co hledáte? A oni rekli jemu: Rabbi, (jenž se vykládá: Mistre,) kde bydlíš?
\par 39 Dí jim: Pojdte a vizte. I šli, aby videli, kde by bydlil, a zustali u neho ten den; neb bylo již okolo desáté hodiny.
\par 40 Byl pak Ondrej, bratr Šimona Petra, jeden ze dvou, kteríž byli to slyšeli od Jana, a šli za ním.
\par 41 I nalezl ten první bratra svého vlastního Šimona, a rekl mu: Nalezli jsme Mesiáše, jenž se vykládá Kristus.
\par 42 I privedl jej k Ježíšovi. A pohledev nan Ježíš, dí: Ty jsi Šimon, syn Jonášuv, ty slouti budeš Céfas, jenž se vykládá Petr.
\par 43 Na druhý pak den Ježíš chtel vyjíti do Galilee, i nalezl Filipa, a rekl jemu: Pojd za mnou.
\par 44 A byl Filip z Betsaidy, mesta Ondrejova a Petrova.
\par 45 Nalezl také Filip Natanaele. I dí jemu: O kterémž psal Mojžíš v Zákone a Proroci, nalezli jsme Ježíše, syna Jozefova z Nazaréta.
\par 46 I rekl jemu Natanael: A muže z Nazaréta co dobrého býti? Rekl jemu Filip: Pojd a viz.
\par 47 Vida Ježíš Natanaele, an jde k nemu, i dí o nem: Aj, práve Izraelitský, v nemžto lsti není.
\par 48 Rekl mu Natanael: Jakž ty mne znáš? Odpovedel Ježíš a rekl jemu: Prve nežli te Filip zavolal, kdyžs byl pod fíkem, videl jsem tebe.
\par 49 Odpovedel Natanael a rekl jemu: Mistre, ty jsi Syn Boží, ty jsi ten Král Izraelský.
\par 50 Odpovedel Ježíš a rekl jemu: Žet jsem rekl: Videl jsem tebe pod fíkem, veríš? Vetší veci nad tyto uzríš.
\par 51 I dí mu: Amen, amen pravím vám: Od tohoto casu uzríte nebe otevrené, a andely Boží vstupující a sstupující na Syna cloveka.

\chapter{2}

\par 1 Tretího dne stala se svadba v Káni Galilejské, a byla matka Ježíšova tam.
\par 2 A pozván jest také Ježíš i ucedlníci jeho na svadbu.
\par 3 Když se pak nedostalo vína, rekla matka Ježíšova k nemu: Vína nemají.
\par 4 Dí jí Ježíš: Co mne a tobe ženo? Ješte neprišla hodina má.
\par 5 Dí matka jeho k služebníkum: Což by koli vám rekl, ucinte.
\par 6 I bylo tu kamenných stoudví šest postaveno, podle obyceje ocištování Židovského, beroucí v sebe jedna každá dve nebo tri míry.
\par 7 Rekl jim Ježíš: Naplnte ty stoudve vodou. I naplnili je až do vrchu.
\par 8 I dí jim: Nalévejtež již, a neste vrchnímu správci svadby. I nesli.
\par 9 A jakž okusil vrchní správce svadby vody vínem ucinené, (nevedel pak, odkud by bylo, ale služebníci vedeli, kteríž vážili vodu,) povolal ženicha ten vrchní správce,
\par 10 A rekl mu: Každý clovek nejprve dobré víno dává, a když by se hojne napili, tehdy to, kteréž horší jest. Ale ty zachoval jsi víno dobré až dosavad.
\par 11 To ucinil Ježíš pocátek divu v Káni Galilejské, a zjevil slávu svou. I uverili v neho ucedlníci jeho.
\par 12 Potom sstoupil do Kafarnaum, on i matka jeho, i bratrí jeho, i ucedlníci jeho, a pobyli tam nemnoho dní;
\par 13 Nebo blízko byla velikanoc Židovská. I vstoupil Ježíš do Jeruzaléma.
\par 14 A nalezl v chráme, ano prodávají voly a ovce i holubice, a penezomence sedící.
\par 15 A udelav bic z provázku, všecky vyhnal z chrámu, i ovce i voly, a penezomencum rozsypal peníze, a stoly zprevracel.
\par 16 A tem, kteríž holuby prodávali, rekl: Odnestež tyto veci odsud, a necinte domu Otce mého domem kupeckým.
\par 17 I rozpomenuli se ucedlníci jeho, že psáno jest: Horlivost domu tvého snedla mne.
\par 18 Tedy odpovedeli Židé a rekli jemu: Jaké znamení toho nám ukážeš, že tyto veci ciníš?
\par 19 Odpovedel Ježíš a rekl jim: Zrušte chrám tento, a ve trech dnech zase vzdelám jej.
\par 20 I rekli Židé: Ctyridceti a šest let delán jest chrám tento, a ty ve trech dnech vzdeláš jej?
\par 21 Ale on pravil o chrámu tela svého.
\par 22 A protož když z mrtvých vstal, rozpomenuli se ucedlníci jeho, že jim to byl povedel. I uverili Písmu a slovu, kteréž povedel Ježíš.
\par 23 A když byl v Jeruzaléme na velikunoc v den svátecní, mnozí uverili ve jméno jeho, vidouce divy jeho, kteréž cinil.
\par 24 Ale Ježíš nesveril sebe samého jim, protože on znal všecky.
\par 25 Aniž potreboval, aby jemu kdo svedectví vydával o cloveku; neb on vedel, co by bylo v cloveku.

\chapter{3}

\par 1 Byl pak clovek z farizeu, jménem Nikodém, kníže Židovské.
\par 2 Ten prišel k Ježíšovi v noci, a rekl jemu: Mistre, víme, že jsi od Boha prišel Mistr; nebo žádný nemuže tech divu ciniti, kteréž ty ciníš, lec by Buh byl s ním.
\par 3 Odpovedel Ježíš a rekl jemu: Amen, amen pravím tobe: Nenarodí-li se kdo znovu, nemuž videti království Božího.
\par 4 Rekl jemu Nikodém: Kterak muž clovek naroditi se, starý jsa? Zdali muže opet v život matky své vjíti a naroditi se?
\par 5 Odpovedel Ježíš: Amen, amen pravím tobe: Nenarodí-li se kdo z vody a z Ducha svatého, nemuž vjíti do království Božího.
\par 6 Což se narodilo z tela, telo jest, a což se narodilo z Ducha, duch jest.
\par 7 Nediviž se, že jsem rekl tobe: Musíte se znovu zroditi.
\par 8 Vítr kde chce veje, a hlas jeho slyšíš, ale nevíš, odkud prichází, a kam jde. Takt jest každý, kdož se z Ducha narodil.
\par 9 Odpovedel Nikodém a rekl jemu: Kterak mohou tyto veci býti?
\par 10 Odpovedel Ježíš a rekl jemu: Ty jsi mistr v Izraeli, a toho neznáš?
\par 11 Amen, amen pravím tobe: Že což víme, mluvíme, a což jsme videli, svedcíme, ale svedectví našeho neprijímáte.
\par 12 Ponevadž zemské veci mluvil jsem vám, a neveríte, kterak, budu-li vám praviti nebeské, uveríte?
\par 13 Nebo žádný nevstoupil v nebe, než ten, jenž sstoupil s nebe, Syn cloveka, kterýž jest v nebi.
\par 14 A jakož jest Mojžíš povýšil hada na poušti, takt musí povýšen býti Syn cloveka,
\par 15 Aby každý, kdož verí v neho, nezahynul, ale mel život vecný.
\par 16 Nebo tak Buh miloval svet, že Syna svého jednorozeného dal, aby každý, kdož verí v neho, nezahynul, ale mel život vecný.
\par 17 Nebot jest neposlal Buh Syna svého na svet, aby odsoudil svet, ale aby spasen byl svet skrze neho.
\par 18 Kdož verí v neho, nebude odsouzen, ale kdož neverí, jižt jest odsouzen; nebo neuveril ve jméno jednorozeného Syna Božího.
\par 19 Toto pak jest ten soud, že Svetlo prišlo na svet, ale milovali lidé více tmu nežli Svetlo; nebo skutkové jejich byli zlí.
\par 20 Každý zajisté, kdož zle ciní, nenávidí svetla, a nejde k svetlu, aby nebyli trestáni skutkové jeho.
\par 21 Ale kdož ciní pravdu, jde k svetlu, aby zjeveni byli skutkové jeho, že v Bohu ucineni jsou.
\par 22 Potom prišel Ježíš i ucedlníci jeho do zeme Judské, a tu prebýval s nimi, a krtil.
\par 23 A Jan také krtil v Enon, blízko Sálim, nebo byly tam vody mnohé. I pricházeli mnozí, a krtili se.
\par 24 Nebo ješte Jan nebyl vsazen do žaláre.
\par 25 Tedy vznikla otázka mezi Židy a nekterými z ucedlníku Janových o ocištování.
\par 26 I prišli k Janovi a rekli jemu: Mistre, ten, kterýž byl s tebou za Jordánem, jemužs ty svedectví vydal, aj, on krtí, a všickni jdou k nemu.
\par 27 Odpovedel Jan a rekl: Nemužt clovek vzíti nicehož, lec by jemu dáno bylo s nebe.
\par 28 Vy sami svedkové jste mi, že jsem povedel: Nejsem já Kristus, ale že jsem poslán pred ním.
\par 29 Kdož má nevestu, ženicht jest, prítel pak ženicha, jenž stojí a slyší ho, radostí raduje se pro hlas ženicha. Protož ta radost má naplnena jest.
\par 30 Ont musí rusti, já pak menšiti se.
\par 31 Kdož jest shury prišel, nade všeckyt jest; kdožt jest z zeme, zemskýt jest, a zemské veci mluví. Ale ten, jenž s nebe prišel, nade všecky jest.
\par 32 A což videl a slyšel, tot svedcí, ale svedectví jeho žádný neprijímá.
\par 33 Kdož pak prijímá svedectví jeho, zpecetil jest to, že Buh pravdomluvný jest.
\par 34 Nebo ten, kteréhož Buh poslal, slovo Boží mluví; nebo jemu ne v míru dává Buh ducha.
\par 35 Otec miluje Syna a všecko dal v ruku jeho.
\par 36 Kdož verí v Syna, má život vecný; ale kdožt jest neverící Synu, neuzrít života, ale hnev Boží zustává na nem.

\chapter{4}

\par 1 A jakž poznal Pán, že jsou slyšeli farizeové, že by Ježíš více ucedlníku cinil a krtil nežli Jan,
\par 2 (Ackoli Ježíš sám nekrtil, ale ucedlníci jeho,)
\par 3 Opustil Judstvo a odšel opet do Galilee.
\par 4 Musil pak jíti skrze Samarí.
\par 5 I prišel k mestu Samarskému, kteréž slove Sichar, vedle popluží, kteréž byl dal Jákob Jozefovi, synu svému.
\par 6 Byla pak tu studnice Jákobova. Protož ustav na ceste Ježíš, posadil se tak na studnici. A bylo již okolo šesté hodiny.
\par 7 I prišla žena z Samarí vážiti vody. Kteréžto rekl Ježíš: Dej mi píti.
\par 8 (Nebo ucedlníci jeho byli odešli do mesta, aby nakoupili pokrmu.)
\par 9 I rekla jemu žena ta Samaritánka: Kterakž ty, jsa Žid, žádáš ode mne nápoje od ženy Samaritánky? (Nebo neobcují Židé s Samaritány.)
\par 10 Odpovedel Ježíš a rekl jí: Kdybys znala ten dar Boží, a vedela, kdo jest, kterýž praví tobe: Dej mi píti, ty bys prosila jeho, a dalt by tobe vody živé.
\par 11 I dí jemu žena: Pane, aniž máš, cím bys navážil, a studnice jest hluboká. Odkudž tedy máš tu vodu živou?
\par 12 Zdaliž jsi ty vetší nežli otec náš Jákob, kterýž nám dal tuto studnici, a sám z ní pil, i synové jeho, i dobytek jeho?
\par 13 Odpovedel Ježíš a rekl jí: Každý, kdož pije vodu tuto, žízniti bude opet.
\par 14 Ale kdož by se napil vody té, kterouž já dám jemu, nežíznil by na veky, ale voda ta, kterouž já dám jemu, bude v nem studnicí vody prýštící se k životu vecnému.
\par 15 Rekla jemu žena: Pane, dej mi té vody, at bych nežíznila, ani chodila sem vážiti.
\par 16 Rekl jí Ježíš: Jdi, zavolej muže svého, a prijd sem.
\par 17 Odpovedela žena a rekla: Nemám muže. Dí jí Ježíš: Dobres rekla: Nemám muže.
\par 18 Nebos pet mužu mela, a nyní kteréhož máš, není tvuj muž. To jsi pravdu povedela.
\par 19 Rekla jemu žena: Pane, vidím, že jsi ty prorok.
\par 20 Otcové naši na této hore modlívali se, a vy pravíte, že v Jeruzaléme jest místo, kdežto náleží se modliti.
\par 21 Dí jí Ježíš: Ženo, ver mi, žet jde hodina, kdyžto ani na této hore, ani v Jeruzaléme nebudete se modliti Otci.
\par 22 Vy se modlíte, a nevíte, cemu; my se pak modlíme, cemuž víme, nebo spasení z Židu jest.
\par 23 Ale jdet hodina, a nynít jest, kdyžto praví modlitebníci modliti se budou Otci v duchu a v pravde. Nebot takových Otec hledá, aby se modlili jemu.
\par 24 Buh duch jest, a ti, kteríž se jemu modlí, v duchu a v pravde musejí se modliti.
\par 25 Dí jemu žena: Vím, že Mesiáš prijde, jenž slove Kristus. Ten, když prijde, oznámí nám všecko.
\par 26 Dí jí Ježíš: Ját jsem, kterýž mluvím s tebou.
\par 27 A v tom prišli ucedlníci jeho, i divili se, že by s ženou mluvil, avšak žádný jemu nerekl: Nac se jí ptáš, aneb proc mluvíš s ní?
\par 28 I nechala tu žena vedra svého, a šla do mesta, a rekla tem lidem:
\par 29 Pojdte, vizte cloveka, kterýž povedel mi všecko, což jsem koli cinila. Není-li on ale Kristus?
\par 30 Tedy vyšli z mesta, a prišli k nemu.
\par 31 Mezi tím pak prosili ho ucedlníci, rkouce: Mistre, pojez.
\par 32 A on rekl jim: Ját mám pokrm k jísti, kteréhož vy nevíte.
\par 33 Ucedlníci pak mluvili vespolek: Zdali jemu kdo prinesl jísti?
\par 34 Dí jim Ježíš: Mujt pokrm jest, abych cinil vuli toho, jenž mne poslal, a dokonal dílo jeho.
\par 35 Však vy pravíte, že ješte ctyri mesícové jsou, a žen prijde. Aj, pravím vám: Pozdvihnete ocí vašich, a patrte na krajiny, žet se již belejí ke žni.
\par 36 Kdož pak žne, odplatut bére, a shromažduje užitek k životu vecnému, aby i ten, kdož rozsívá, spolu se radoval, i kdo žne.
\par 37 Nebo i v tom pravé jest slovo, žet jiný jest, jenž rozsívá, a jiný, kterýž žne.
\par 38 Ját jsem vás poslal žíti, o cemž jste vy nepracovali. Jinít jsou pracovali, a vy jste v jejich práce vešli.
\par 39 Z mesta pak toho mnozí z Samaritánu uverili v neho, pro rec ženy, svedcící: Že mi povedel všecko, což jsem cinila.
\par 40 A když k nemu prišli Samaritáni, prosili ho, aby s nimi zustal. I pobyl tu za dva dni.
\par 41 A mnohem jich více uverilo pro rec jeho.
\par 42 A žene té rekli: Že již ne pro tvé vypravování veríme; nebo sami jsme slyšeli, a víme, že tento jest práve Spasitel sveta, Kristus.
\par 43 Po dvou pak dnech vyšel odtud, a šel do Galilee.
\par 44 Nebo sám Ježíš byl svedectví vydal, že prorok v vlasti své v vážnosti není.
\par 45 A když prišel do Galilee, prijali jej Galilejští, všecko videvše, co cinil v Jeruzaléme v svátek; nebo i oni byli prišli ke dni svátecnímu.
\par 46 Tedy opet prišel Ježíš do Káne Galilejské, kdežto ucinil byl z vody víno. I byl jeden královský služebník v Kafarnaum, jehožto syn nemocen byl.
\par 47 Ten uslyšev, že by Ježíš prišel z Judstva do Galilee, šel k nemu, a prosil ho, aby sstoupil a uzdravil syna jeho; nebo pocínal umírati.
\par 48 I rekl k nemu Ježíš: Neuzríte-li divu a zázraku, neuveríte.
\par 49 Dí jemu ten královský služebník: Pane, pojd prve, nežli umre syn muj.
\par 50 Dí jemu Ježíš: Jdi, syn tvuj živ jest. I uveril clovek reci, kterouž mluvil k nemu Ježíš, a šel.
\par 51 Když pak on již šel, potkali se s ním služebníci jeho a zvestovali mu, rkouce: Syn tvuj živ jest.
\par 52 Tedy otázal se jich na hodinu, v kterou by se lépe mel. I rekli jemu: Vcera v hodinu sedmou prestala mu zimnice.
\par 53 Tedy poznal otec, že práve v tu hodinu to se stalo, v kterouž rekl byl jemu Ježíš: Syn tvuj živ jest. I uveril on i dum jeho všecken.
\par 54 To opet druhý div ucinil Ježíš, prišed z Judstva do Galilee.

\chapter{5}

\par 1 Potom byl svátek Židovský, i šel Ježíš do Jeruzaléma.
\par 2 Byl pak v Jeruzaléme u brány bravné rybník, kterýž slove Židovsky Bethesda, patero prístreší maje.
\par 3 Kdežto leželo množství veliké neduživých, slepých, kulhavých, suchých, ocekávajících hnutí vody.
\par 4 Nebo andel Páne jistým casem sstupoval do rybníka, a kormoutil vodu. A kdož tam nejprve sstoupil po tom zkormoucení vody, uzdraven býval, od kterékoli nemoci trápen byl.
\par 5 I byl tu clovek jeden, kterýž osm a tridceti let nemocen byl.
\par 6 Toho uzrev Ježíš ležícího, a poznav, že jest již dávno nemocen, dí jemu: Chceš-li zdráv býti?
\par 7 Odpovedel mu nemocný: Pane, nemám cloveka, kterýž by, když se zkormoutí voda, uvrhl mne do rybníka, nebo když já jdu, jiný prede mnou již vstupuje.
\par 8 Dí jemu Ježíš: Vstan, vezmi lože své a chod.
\par 9 A hned zdráv jest ucinen clovek ten, a vzav lože své, i chodil. A byla sobota v ten den.
\par 10 Tedy rekli Židé tomu uzdravenému: Sobota jest, neslušít tobe lože nositi.
\par 11 Odpovedel jim: Ten, kterýž mne uzdravil, ont mi rekl: Vezmi lože své a chod.
\par 12 I otázali se ho: Kdo jest ten clovek, kterýž tobe rekl: Vezmi lože své a chod?
\par 13 Ten pak uzdravený nevedel, kdo by byl. Nebo Ježíš byl poodšel od zástupu shromáždeného na tom míste.
\par 14 Potom pak nalezl jej Ježíš v chráme, a rekl jemu: Aj, zdráv jsi ucinen; nikoli víc nehreš, at by se neco horšího neprihodilo.
\par 15 Odšel ten clovek, a povedel Židum, že by Ježíš byl ten, kterýž ho zdravého ucinil.
\par 16 A protož protivili se Židé Ježíšovi, a hledali ho zabíti, že to ucinil v sobotu.
\par 17 Ježíš pak odpovedel jim: Otec muj až dosavad delá, i ját delám.
\par 18 Tedy Židé ješte víc proto hledali ho zamordovati, že by netoliko rušil sobotu, ale že Otce svého pravil býti Boha, rovného se cine Bohu.
\par 19 I odpovedel Ježíš a rekl jim: Amen, amen pravím vám: Nemužet Syn sám od sebe nic ciniti, jediné což vidí, an Otec ciní. Nebo cožkoli on ciní, tot i Syn též podobne ciní.
\par 20 Otec zajisté miluje Syna, a ukazuje mu všecko, což sám ciní; a vetší nad to ukáže jemu skutky tak, abyste vy se divili.
\par 21 Nebo jakož Otec krísí mrtvé a obživuje, tak i Syn, kteréž chce, obživuje.
\par 22 Aniž zajisté Otec soudí koho, ale všecken soud dal Synu,
\par 23 Aby všickni ctili Syna, tak jakž Otce ctí. Kdo nectí Syna, nectí ani Otce, kterýž ho poslal.
\par 24 Amen, amen pravím vám: Že kdož slovo mé slyší, a verí tomu, jenž mne poslal, mát život vecný, a na soud neprijde, ale prešelt jest z smrti do života.
\par 25 Amen, amen pravím vám: Že prijde hodina, a nynít jest, kdyžto mrtví uslyší hlas Syna Božího, a kteríž uslyší, živi budou.
\par 26 Nebo jakož Otec má život sám v sobe, tak jest dal i Synu, aby mel život v samém sobe.
\par 27 A dal jemu moc i soud ciniti, nebo Syn cloveka jest.
\par 28 Nedivtež se tomu; nebot prijde hodina, v kteroužto všickni, kteríž v hrobích jsou, uslyší hlas jeho.
\par 29 A pujdou ti, kteríž dobré veci cinili, na vzkríšení života, ale ti, kteríž zlé veci cinili, na vzkríšení soudu.
\par 30 Nemohut já sám od sebe nic ciniti. Ale jakžt slyším, takt soudím, a soud muj spravedlivý jest. Nebo nehledám vule své, ale vule toho, jenž mne poslal, Otcovy.
\par 31 Vydám-lit já svedectví sám o sobe, svedectví mé není pravé.
\par 32 Jinýt jest, kterýž svedectví vydává o mne, a vím, že pravé jest svedectví, kteréž vydává o mne.
\par 33 Vy jste byli poslali k Janovi, a on svedectví vydal pravde.
\par 34 Ale ját neprijímám svedectví od cloveka, než totot pravím, abyste vy spaseni byli.
\par 35 Ont jest byl svíce horící a svítící, vy pak chteli jste na cas poradovati se v svetle jeho.
\par 36 Ale já mám vetší svedectví, nežli Janovo. Nebo skutkové, kteréž mi dal Otec, abych je vykonal, tit skutkové, kteréž já ciním, svedcí o mne, že jest mne Otec poslal.
\par 37 A kterýž mne poslal, Otec, ont jest svedectví vydal o mne, jehož jste vy hlasu nikdy neslyšeli, ani tvári jeho videli.
\par 38 A slova jeho nemáte v sobe zustávajícího. Nebo kteréhož jest on poslal, tomu vy neveríte.
\par 39 Ptejte se na Písma; nebo vy domníváte se v nich vecný život míti, a tat svedectví vydávají o mne.
\par 40 A nechcete prijíti ke mne, abyste život meli.
\par 41 Chvály od lidí ját neprijímám.
\par 42 Ale poznal jsem vás, že milování Božího nemáte v sobe.
\par 43 Já jsem prišel ve jménu Otce svého, a neprijímáte mne. Kdyby jiný prišel ve jménu svém, toho prijmete.
\par 44 Kterak vy mužete veriti, chvály jedni od druhých hledajíce, ponevadž chvály, kteráž jest od samého Boha, nehledáte?
\par 45 Nedomnívejte se, bycht já na vás žalovati mel pred Otcem. Jestit, kdo by žaloval na vás, Mojžíš, v nemž vy nadeji máte.
\par 46 Nebo kdybyste verili Mojžíšovi, verili byste i mne; nebt jest on o mne psal.
\par 47 Ale ponevadž jeho písmum neveríte, i kterak slovum mým uveríte?

\chapter{6}

\par 1 Potom odšel Ježíš za more Galilejské, jenž jest Tiberiadské.
\par 2 A šel za ním zástup veliký; nebo videli divy jeho, kteréž cinil nad nemocnými.
\par 3 I všel na horu Ježíš, a tam sedel s ucedlníky svými.
\par 4 Byla pak blízko velikanoc, svátek Židovský.
\par 5 Tedy pozdvih ocí Ježíš a videv, že zástup veliký jde k nemu, dí k Filipovi: Kde nakoupíme chlebu, aby pojedli tito?
\par 6 (Ale to rekl pokoušeje ho; nebo on sám vedel, co by mel uciniti.)
\par 7 Odpovedel jemu Filip: Za dve ste penez chlebu nepostací jim, aby jeden každý z nich neco malicko vzal.
\par 8 Dí jemu jeden z ucedlníku jeho, Ondrej, bratr Šimona Petra:
\par 9 Jestit mládencek jeden zde, kterýž má pet chlebu jecných a dve rybicky. Ale cot jest to mezi tak mnohé?
\par 10 I rekl Ježíš: Rozkažtež lidu, at se usadí. A bylo trávy mnoho na tom míste. I posadilo se mužu v poctu okolo peti tisícu.
\par 11 Tedy Ježíš vzal ty chleby, a díky uciniv, rozdával ucedlníkum, ucedlníci pak sedícím; též podobne z tech rybicek, jakž jsou mnoho chteli.
\par 12 A když byli nasyceni, rekl ucedlníkum svým: Sberte ty drobty, kteríž zustali, at nezhynou.
\par 13 I sebrali a naplnili dvanácte košu drobtu z peti chlebu jecných, kteríž pozustali po tech, jenž jedli.
\par 14 Ti pak lidé, uzrevše ten div, kterýž ucinil Ježíš, pravili: Tento jest jiste prorok, kterýž mel prijíti na svet.
\par 15 Tedy Ježíš veda, že by meli prijíti a chytiti jej, aby ho ucinili králem, ušel na horu opet sám jediný.
\par 16 Když pak byl vecer, sstoupili ucedlníci jeho k mori.
\par 17 A vstoupivše na lodí, plavili se pres more do Kafarnaum. A bylo již tma, a neprišel byl Ježíš k nim.
\par 18 More pak dutím velikého vetru zdvihalo se.
\par 19 A odplavivše se honu jako petmecítma nebo tridceti, uzreli Ježíše, an chodí po mori a približuje se k lodí. I báli se.
\par 20 On pak rekl jim: Ját jsem, nebojte se.
\par 21 I vzali ho na lodí ochotne, a hned pribehla k zemi, do kteréž se plavili.
\par 22 Nazejtrí pak zástup, kterýž byl za morem, videv, že jiné lodicky nebylo, než ta jedna, na kterouž byli vstoupili ucedlníci jeho, a že Ježíš nebyl všel s ucedlníky svými na lodí, ale sami ucedlníci jeho byli se plavili,
\par 23 (Jiné pak lodí byly priplouly od Tiberiady k tomu místu blízko, kdežto byli jedli chléb, když díky ucinil Pán,)
\par 24 Když tedy uzrel zástup, že Ježíše tu není, ani ucedlníku jeho, vstoupili i oni na lodí, a prijeli do Kafarnaum, hledajíce Ježíše.
\par 25 A nalezše ho za morem, rekli jemu: Mistre, kdy jsi sem prišel?
\par 26 Odpovedel jim Ježíš a rekl: Amen, amen pravím vám: Hledáte mne, ne protože jste divy videli, ale že jste jedli chleby a nasyceni jste.
\par 27 Pracujte ne o pokrm, kterýž hyne, ale o ten pokrm, kterýž zustává k životu vecnému, kterýž Syn cloveka dá vám. Nebo tohot jest potvrdil Buh Otec.
\par 28 Tedy rekli jemu: Co budeme ciniti, abychom delali dílo Boží?
\par 29 Odpovedel Ježíš a rekl jim: Totot jest to dílo Boží, abyste verili v toho, kteréhož on poslal.
\par 30 I rekli jemu: Jakéž pak ty znamení ciníš, abychom videli a verili tobe? Co deláš?
\par 31 Otcové naši jedli mannu na poušti, jakož psáno jest: Chléb s nebe dal jim jísti.
\par 32 Tedy rekl Ježíš: Amen, amen pravím vám: Ne Mojžíš dal vám chléb s nebe, ale Otec muj vám dává ten chléb s nebe pravý.
\par 33 Nebo chléb Boží ten jest, kterýž sstupuje s nebe a dává život svetu.
\par 34 A oni rekli jemu: Pane, dávejž nám chléb ten vždycky.
\par 35 I rekl jim Ježíš: Ját jsem ten chléb života. Kdož prichází ke mne, nebude nikoli lacneti, a kdož verí ve mne, nebude žízniti nikdy.
\par 36 Ale povedel jsem vám, anobrž videli jste mne, a neveríte.
\par 37 Všecko, což mi dává Otec, ke mne prijde, a toho, kdož ke mne prijde, nevyvrhu ven.
\par 38 Nebo jsem sstoupil s nebe, ne abych cinil vuli svou, ale vuli toho, jenž mne poslal.
\par 39 Tatot jest pak vule toho, kterýž mne poslal, Otcova, abych všecko, což mi dal, neztratil toho, ale vzkrísil to v nejposlednejší den.
\par 40 A tatot jest vule toho, kterýž mne poslal, aby každý, kdož vidí Syna a verí v neho, mel život vecný. A ját jej vzkrísím v den nejposlednejší.
\par 41 I reptali Židé na neho, že rekl: Já jsem chléb, kterýž jsem s nebe sstoupil.
\par 42 A pravili: Zdaliž tento není Ježíš, syn Jozefuv, jehož my otce i matku známe? Kterak pak dí tento: S nebe jsem sstoupil?
\par 43 Tedy odpovedel Ježíš a rekl jim: Nerepcete vespolek.
\par 44 Však žádný nemuž prijíti ke mne, jediné lec Otec muj, kterýž mne poslal, pritrhl by jej; a ját jej vzkrísím v den nejposlednejší.
\par 45 Psáno jest v Prorocích: A budou všickni uceni od Boha. Protož každý, kdož slyšel od Otce a naucil se, jdet ke mne.
\par 46 Ne že by kdo videl Otce, jediné ten, kterýž jest od Boha, tent jest videl Otce.
\par 47 Amen, amen pravím vám: Kdož verí ve mne, mát život vecný.
\par 48 Ját jsem ten chléb života.
\par 49 Otcové vaši jedli mannu na poušti, a zemreli.
\par 50 Totot jest chléb ten s nebe sstupující. Kdožt by koli jej jedl, neumret.
\par 51 Ját jsem ten chléb živý, jenž jsem s nebe sstoupil. Bude-li kdo jísti ten chléb, živ bude na veky. A chléb, kterýž já dám, telo mé jest, kteréž já dám za život sveta.
\par 52 Tedy hádali se Židé vespolek, rkouce: Kterak tento muže dáti nám telo své jísti?
\par 53 I rekl jim Ježíš: Amen, amen pravím vám: Nebudete-li jísti tela Syna cloveka a píti krve jeho, nemáte života v sobe.
\par 54 Kdož jí mé telo a pije mou krev, mát život vecný, a ját jej vzkrísím v nejposlednejší den.
\par 55 Nebo telo mé práve jest pokrm, a krev má práve jest nápoj.
\par 56 Kdo jí mé telo a pije mou krev, ve mne prebývá a já v nem.
\par 57 Jakož mne poslal ten živý Otec, a já živ jsem skrze Otce, tak kdož jí mne, i on živ bude skrze mne.
\par 58 Totot jest ten chléb, kterýž s nebe sstoupil. Ne jako otcové vaši jedli mannu, a zemreli. Kdož jí chléb tento, živt bude na veky.
\par 59 Toto mluvil Ježíš v škole, uce v Kafarnaum.
\par 60 Tedy mnozí z ucedlníku jeho, slyševše to, rekli: Tvrdát jest toto rec. Kdo ji muže slyšeti?
\par 61 Ale veda Ježíš sám v sobe, že by proto reptali ucedlníci jeho, rekl jim: To vás uráží?
\par 62 Co pak, kdybyste uzreli Syna cloveka, an vstupuje, kdež prve byl?
\par 63 Duch jest, jenž obživuje, telot nic neprospívá. Slova, kteráž já mluvím vám, Duch a život jsou.
\par 64 Ale jsout nekterí z vás, ješto neverí. Nebo vedel Ježíš od pocátku, kdo by byli neverící, a kdo by ho mel zraditi.
\par 65 I pravil: Protož jsem vám rekl, že žádný nemuže prijíti ke mne, lec by dáno bylo od Otce mého.
\par 66 A z toho mnozí z ucedlníku jeho odešli zpet, a nechodili s ním více.
\par 67 Tedy rekl Ježíš ke dvanácti: Zdali i vy chcete odjíti?
\par 68 I odpovedel jemu Šimon Petr: Pane, k komu pujdeme? A ty slova vecného života máš.
\par 69 A my jsme uverili, a poznali, že jsi ty Kristus, Syn Boha živého.
\par 70 Odpovedel jim Ježíš: Však jsem já vás dvanácte vyvolil, a jeden z vás dábel jest.
\par 71 A to rekl o Jidášovi synu Šimona Iškariotského; nebo ten jej mel zraditi, byv jeden ze dvanácti.

\chapter{7}

\par 1 Potom pak chodil Ježíš po Galilei; nebo nechtel býti v Judstvu, protože ho hledali Židé zabíti.
\par 2 A byl blízko svátek Židovský, památka stánku.
\par 3 Tedy rekli jemu bratrí jeho: Vyjdi odsud, a jdi do Judstva, at i ucedlníci tvoji vidí skutky tvé, kteréž ciníš.
\par 4 Nižádný zajisté v skryte nic nedelá, kdož chce vidín býti. Protož ty, ciníš-li takové veci, zjeviž se svetu.
\par 5 Nebo ani bratrí jeho neverili v neho.
\par 6 I dí jim Ježíš: Cas muj ješte neprišel, ale cas váš vždycky jest hotov.
\par 7 Nemužet vás svet nenávideti, ale mnet nenávidí; nebo já svedectví vydávám o nem, že skutkové jeho zlí jsou.
\par 8 Vy jdete k svátku tomuto. Ját ješte nepujdu k svátku tomuto, nebo cas muj ješte se nenaplnil.
\par 9 To povedev jim, zustal v Galilei.
\par 10 A když odešli bratrí jeho, tedy i on šel k svátku, ne zjevne, ale jako ukryte.
\par 11 Židé pak hledali ho v svátek, a pravili: Kde jest onen?
\par 12 A mnoho recí bylo o nem v zástupu. Nebo nekterí pravili, že dobrý jest, a jiní pravili: Není, ale svodí zástup.
\par 13 Žádný však o nem nemluvil zjevne pro bázen Židu.
\par 14 Když pak již polovici svátku se vykonalo, vstoupil Ježíš do chrámu, a ucil.
\par 15 I divili se Židé, rkouce: Kterak tento Písmo umí, neuciv se?
\par 16 Odpovedel jim Ježíš a rekl: Mé ucení nenít mé, ale toho, jenž mne poslal.
\par 17 Bude-li kdo chtíti vuli jeho ciniti, tent bude umeti rozeznati, jest-li to ucení z Boha, cili mluvím já sám od sebe.
\par 18 Kdot sám od sebe mluví, chvály své vlastní hledá, ale kdož hledá chvály toho, kterýž ho poslal, tent pravdomluvný jest, a nepravosti v nem není.
\par 19 Však Mojžíš dal vám Zákon? a žádný z vás neplní Zákona? Proc mne hledáte zamordovati?
\par 20 Odpovedel zástup a rekl: Dábelství máš. Kdo te hledá zamordovati?
\par 21 Odpovedel Ježíš a rekl jim: Jeden skutek ucinil jsem, a všickni se tomu divíte.
\par 22 Mojžíš vydal vám obrízku, (ne že by z Mojžíše byla, ale z otcu,) a v sobotu obrezujete cloveka.
\par 23 Ponevadž clovek obrízku prijímá i v sobotu, aby nebyl rušen Zákon Mojžíšuv, proc pak  se na mne hneváte, že jsem celého cloveka uzdravil v sobotu?
\par 24 Nesudte podle osoby, ale spravedlivý soud sudte.
\par 25 Tedy nekterí z Jeruzalémských pravili: Zdaliž toto není ten, kteréhož hledají zabíti?
\par 26 A aj, svobodne mluví, a nic mu neríkají. Zdali jsou již práve poznali knížata, že tento jest práve Kristus?
\par 27 Ale o tomto víme, odkud jest, Kristus pak když prijde, žádný nebude vedeti, odkud by byl.
\par 28 I volal Ježíš v chráme, uce a rka: I mne znáte, i odkud jsem, víte. A všakt jsem neprišel sám od sebe, ale jestit pravdomluvný, kterýž mne poslal, jehož vy neznáte.
\par 29 Ale já znám jej, nebo od neho jsem, a on mne poslal.
\par 30 I hledali ho jíti, ale žádný nevztáhl ruky na neho, nebo ješte byla neprišla hodina jeho.
\par 31 Z zástupu pak mnozí uverili v nej, a pravili: Kristus když prijde, zdali více divu ciniti bude nad ty, kteréž tento ciní?
\par 32 Slyšeli pak farizeové zástup, an o nem takové veci rozmlouvá, i poslali farizeové a prední kneží služebníky, aby jej jali.
\par 33 Tedy rekl jim Ježíš: Ješte malický cas jsem s vámi, potom odejdu k tomu, jenž mne poslal.
\par 34 Hledati mne budete, a nenaleznete, a kdež já budu, vy tam nemužte prijíti.
\par 35 I rekli Židé k sobe vespolek: Kam tento pujde, že my ho nenalezneme? Zdali v rozptýlení pohanu pujde, a bude uciti pohany?
\par 36 Jaká jest to rec, kterouž promluvil: Hledati mne budete, a nenaleznete, a kdež já budu, vy nemužete prijíti?
\par 37 V poslední pak den ten veliký svátku toho, stál Ježíš a volal, rka: Žízní-li kdo, pojd ke mne, a napij se.
\par 38 Kdož verí ve mne, jakož dí Písmo, reky z života jeho poplynou vody živé.
\par 39 (A to rekl o Duchu svatém, kteréhož meli prijíti verící v neho; nebo ješte nebyl dán Duch svatý, protože ješte Ježíš nebyl oslaven.)
\par 40 Tedy mnozí z zástupu uslyševše tu rec, pravili: Tentot jest práve prorok.
\par 41 Jiní pravili: Tentot jest Kristus. Ale nekterí pravili: Zdaliž od Galilee prijde Kristus?
\par 42 Zdaž nedí písmo, že z semene Davidova a z Betléma mestecka, kdež býval David, prijíti má Kristus?
\par 43 A tak ruznice v zástupu stala se pro nej.
\par 44 Nekterí pak z nich chteli ho jíti, ale žádný nevztáhl ruky na nej.
\par 45 Tedy prišli služebníci k predním knežím a k farizeum, i rekli jim oni: Proc jste ho neprivedli?
\par 46 Odpovedeli služebníci: Nikdy tak clovek nemluvil, jako tento clovek.
\par 47 I odpovedeli jim farizeové: Zdali i vy jste svedeni?
\par 48 Zdaliž kdo z knížat uveril v neho anebo z farizeu?
\par 49 Než zástup ten, kterýž nezná Zákona. Zlorecenít jsou.
\par 50 I dí k nim Nikodém, ten, jenž byl prišel k nemu v noci, kterýž byl jeden z nich:
\par 51 Zdali Zákon náš soudí cloveka, lec prve uslyší od neho a zví, co by cinil?
\par 52 Odpovedeli a rekli jemu: Zdali i ty Galilejský jsi? Ptej se, žet žádný prorok od Galilee nepovstal.
\par 53 I šel jeden každý do domu svého.

\chapter{8}

\par 1 Ježíš pak odšel na horu Olivetskou.
\par 2 Potom na úsvite zase prišel do chrámu, a všecken lid sšel se k nemu. A on posadiv se, ucil je.
\par 3 I privedli k nemu zákoníci a farizeové ženu v cizoložstvu popadenou; a postavivše ji v prostredku,
\par 4 Rekli jemu: Mistre, tato žena jest postižena pri skutku, když cizoložila.
\par 5 A v Zákone Mojžíš prikázal nám takové kamenovati. Ty pak co pravíš?
\par 6 A to rekli, pokoušejíce ho, aby jej mohli obžalovati. Ježíš pak skloniv se dolu, prstem psal na zemi.
\par 7 A když se neprestávali otazovati jeho, zdvihl se a rekl jim: Kdo jest z vás bez hríchu, nejprv hod na ni kamenem.
\par 8 A opet schýliv se, psal na zemi.
\par 9 A oni uslyševše to a v svedomích svých obvineni jsouce, jeden po druhém odcházeli, pocavše od starších až do posledních. I zustal tu Ježíš sám, a žena uprostred stojeci.
\par 10 A pozdvih se Ježíš a žádného nevidev, než ženu, rekl jí: Ženo, kde jsou ti, kteríž na tebe žalovali? Žádný-li te neodsoudil?
\par 11 Kterážto rekla: Žádný, Pane. I rekl jí Ježíš: Aniž já tebe odsuzuji. Jdiž a nehreš více.
\par 12 Tedy Ježíš opet jim mluvil, rka: Já jsem Svetlo sveta. Kdož mne následuje, nebudet choditi v temnostech, ale budet míti Svetlo života.
\par 13 I rekli jemu farizeové: Ty sám o sobe svedectví vydáváš, svedectví tvé není pravé.
\par 14 Odpovedel Ježíš a rekl jim: Ackoli já svedectví vydávám sám o sobe, však pravé jest svedectví mé; nebo vím, odkud jsem prišel a kam jdu. Ale vy nevíte, odkud jsem prišel, anebo kam jdu.
\par 15 Vy podle tela soudíte, já nesoudím žádného.
\par 16 A bycht pak i soudil já, soud muj jestit pravý; nebo nejsem sám, ale jsem já a ten, kterýž mne poslal, Otec.
\par 17 A v Zákone vašem psáno jest: Že dvou cloveku svedectví pravé jest.
\par 18 Ját svedectví vydávám sám o sobe, a svedectví vydává o mne ten, kterýž mne poslal, Otec.
\par 19 Tedy rekli jemu: Kdež jest ten tvuj Otec? Odpovedel Ježíš: Aniž mne znáte, ani Otce mého. Kdybyste mne znali, i Otce mého znali byste.
\par 20 Tato slova mluvil Ježíš u pokladnice, uce v chráme, a žádný ho nejal, nebo ješte byla neprišla hodina jeho.
\par 21 I rekl jim opet Ježíš: Já jdu, a hledati budete mne, a v hríchu vašem zemrete. Kam já jdu, vy nemužete prijíti.
\par 22 I pravili Židé: Zdali se sám zabije, že praví: Kam já jdu, vy nemužete prijíti?
\par 23 I rekl jim: Vy z dulu jste, já s hury jsem. Vy jste z tohoto sveta, já nejsem z sveta tohoto.
\par 24 Protož jsem rekl vám: Že zemrete v hríších svých. Nebo jestliže nebudete veriti, že já jsem, zemrete v hríších vašich.
\par 25 I rekli jemu: Kdo jsi ty? I rekl jim Ježíš: To, což hned s pocátku pravím vám.
\par 26 Mnohot mám o vás mluviti a souditi, ale ten, kterýž mne poslal, pravdomluvný jest, a já, což jsem slyšel od neho, to mluvím na svete.
\par 27 A oni neporozumeli, že by o Bohu Otci pravil jim.
\par 28 Protož rekl jim Ježíš: Když povýšíte Syna cloveka, tehdy poznáte, že já jsem. A sám od sebe nic neciním, ale jakž mne naucil Otec muj, takt mluvím.
\par 29 A ten, kterýž mne poslal, se mnout jest, a neopustil mne samého Otec; nebo což jest jemu libého, to já ciním vždycky.
\par 30 Ty veci když mluvil, mnozí uverili v neho.
\par 31 Tedy rekl Ježíš tem Židum, kteríž uverili jemu: Jestliže vy zustanete v reci mé, práve ucedlníci moji budete.
\par 32 A poznáte pravdu, a pravda vás vysvobodí.
\par 33 I odpovedeli jemu: Síme Abrahamovo jsme, a žádnému jsme nesloužili nikdy. I kterakž ty díš: Že svobodní budete?
\par 34 Odpovedel jim Ježíš: Amen, amen pravím vám: Že každý, kdož ciní hrích, služebník jest hrícha.
\par 35 A služebník nezustává v domu na veky; ale Syn zustává na veky.
\par 36 Protož jestližet vás vysvobodí Syn, práve svobodní budete.
\par 37 Vím, že jste síme Abrahamovo, ale hledáte mne zabíti; nebo rec má nemá místa u vás.
\par 38 Já, což jsem videl u Otce svého, to mluvím; a i vy, co jste videli u otce vašeho, to ciníte.
\par 39 Odpovedeli a rekli jemu: Otec náš jest Abraham. Dí jim Ježíš: Kdybyste synové Abrahamovi byli, cinili byste skutky Abrahamovy.
\par 40 Ale nyní hledáte mne zabíti, cloveka toho, kterýž jsem pravdu mluvil vám, kterouž jsem slyšel od Boha. Tohot jest Abraham necinil.
\par 41 Vy ciníte skutky otce svého. I rekli jemu: Myt z smilstva nejsme zplozeni, jednohot Otce máme, totiž Boha.
\par 42 Tedy rekl jim Ježíš: Byt Buh Otec váš byl, milovali byste mne. Nebo já jsem z Boha pošel, a prišel jsem; aniž jsem sám od sebe prišel, ale on mne poslal.
\par 43 Proc mluvení mého nechápáte? Protože hned slyšeti nemužete reci mé.
\par 44 Vy z otce dábla jste, a žádosti otce vašeho chcete ciniti. On byl vražedník od pocátku, a v pravde nestál; nebo pravdy v nem není. Když mluví lež, z svého vlastního mluví; nebo lhár jest a otec lži.
\par 45 Já pak že pravdu pravím, neveríte mi.
\par 46 Kdo z vás bude mne obvinovati z hríchu? A ponevadž pravdu pravím, proc vy mi neveríte?
\par 47 Kdo z Boha jest, slova Boží slyší; protož vy neslyšíte, že z Boha nejste.
\par 48 Tedy odpovedeli Židé a rekli jemu: Zdaliž my dobre nepravíme, že jsi ty Samaritán, a dábelství máš?
\par 49 Odpovedel Ježíš: Ját dábelství nemám, ale ctím Otce svého; než vy jste mne neuctili.
\par 50 Ját pak nehledám chvály své; jestit, kdo hledá a soudí.
\par 51 Amen, amen pravím vám: Bude-li kdo zachovávati slovo mé, smrti neuzrí na veky.
\par 52 Tedy rekli mu Židé: Nyní jsme poznali, že dábelství máš. Abraham umrel i proroci, a ty pravíš: Bude-li kdo zachovávati rec mou, smrti neokusí na veky.
\par 53 Zdali jsi ty vetší otce našeho Abrahama, kterýž umrel? I proroci zemreli jsou. Kým ty se ciníš?
\par 54 Odpovedel Ježíš: Chválím-lit já se sám, chvála má nic není. Jestit kterýž mne chválí, Otec muj, o nemž vy pravíte, že Buh váš jest.
\par 55 A nepoznali jste ho, ale já jej znám. A kdybych rekl, že ho neznám, byl bych podobný vám, lhár. Ale známt jej, a rec jeho zachovávám.
\par 56 Abraham, otec váš, veselil se, aby videl den muj, i videl, a radoval se.
\par 57 Tedy rekli jemu Židé: Padesáti let ješte nemáš, a Abrahama jsi videl?
\par 58 Rekl jim Ježíš: Amen, amen pravím vám: Prve nežli Abraham byl, já jsem.
\par 59 I zchápali Židé kamení, aby házeli na nej. Ježíš pak skryl se, a prošed skrze ne, vyšel z chrámu, a tak jich znikl.

\chapter{9}

\par 1 Pomíjeje Ježíš, uzrel cloveka slepého od narození.
\par 2 I otázali se ucedlníci jeho, rkouce: Mistre, kdo jest zhrešil, tento-li, cili rodicové jeho, že se slepý narodil?
\par 3 Odpovedel Ježíš: Ani tento nezhrešil, ani rodicové jeho, ale aby zjeveni byli skutkové Boží na nem.
\par 4 Ját musím delati dílo toho, kterýž mne poslal, dokudž den jest. Pricházít noc, kdyžto žádný nebude moci delati.
\par 5 Dokudž jsem na svete, Svetlo jsem sveta.
\par 6 To povedev, plinul na zemi, a ucinil bláto z sliny, i pomazal tím blátem oci slepého.
\par 7 A rekl jemu: Jdi, umej se v rybníku Siloe, jenž se vykládá: Poslaný. A on šel a umyl se, i prišel, vida.
\par 8 Sousedé pak a ti, kteríž jej prve vídali slepého, rekli: Však tento jest, kterýž sedával a žebral?
\par 9 Jiní pravili, že ten jest, a jiní, že jest podoben k nemu. Ale on pravil: Já jsem.
\par 10 Tedy rekli jemu: Kterak jsou otevríny oci tvé?
\par 11 On odpovedel a rekl: Clovek ten, kterýž slove Ježíš, bláto ucinil a pomazal ocí mých, a rekl mi: Jdi k rybníku Siloe, a umej se. I odšed a umyv se, prohlédl jsem.
\par 12 I rekli jemu: Kde jest ten clovek? Rekl: Nevím.
\par 13 Tedy privedli toho, kterýž nekdy byl slepý, k farizeum.
\par 14 Byla pak sobota, když Ježíš ucinil bláto a otevrel oci jeho.
\par 15 I tázali se ho opet i farizeové, kterak by prozrel. On pak rekl jim: Bláto položil mi na oci, a umyl jsem se, i vidím.
\par 16 Tedy nekterí z farizeu rekli: Tento clovek není z Boha, nebo neostríhá soboty. Jiní pravili: Kterak muže clovek hríšný takové divy ciniti? I byla ruznice mezi nimi.
\par 17 Tedy rekli opet slepému: Co ty o nem pravíš, že otevrel oci tvé? A on rekl: Že prorok jest.
\par 18 I neverili Židé o nem, by slepý byl a prozrel, až povolali rodicu toho, kterýž byl prozrel.
\par 19 A otázali se jich, rkouce: Jest-li ten syn váš, o kterémž vy pravíte, že by se slepý narodil? Kterakž pak nyní vidí?
\par 20 Odpovedeli jim rodicové jeho a rekli: Vímet, že tento jest syn náš a že se slepý narodil.
\par 21 Ale kterak nyní vidí, nevíme; aneb kdo jest otevrel oci jeho, myt nevíme. Však má léta, ptejte se ho, on sám za sebe mluviti bude.
\par 22 Tak mluvili rodicové jeho, že se báli Židu; nebo již tak byli uložili Židé, kdož by ho koli vyznal Kristem, aby byl vyobcován ze školy.
\par 23 Protož rekli rodicové jeho: Mát léta, ptejte se jeho.
\par 24 I zavolali po druhé cloveka toho, kterýž býval slepý, a rekli jemu: Vzdej chválu Bohu. My víme, že clovek ten hríšník jest.
\par 25 I odpovedel on a rekl: Jest-li hríšník, nevím, než to vím, že byv slepý, již nyní vidím.
\par 26 I rekli jemu opet: Cot ucinil? Kterak otevrel oci tvé?
\par 27 Odpovedel jim: Již jsem vám povedel, a neslyšeli jste? Což opet chcete slyšeti? Zdaliž i vy chcete ucedlníci jeho býti?
\par 28 I zlorecili jemu a rekli: Budiž ty sám ucedlníkem jeho, ale myt jsme Mojžíšovi ucedlníci.
\par 29 My víme, že Mojžíšovi mluvil Buh, tento pak nevíme, odkud jest.
\par 30 Odpovedel ten clovek a rekl jim: Tot jest jiste divná vec, že vy nevíte, odkud jest, a otevrel oci mé.
\par 31 Víme pak, že Buh hríšníku neslyší, ale kdo by byl ctitel Boží a vuli jeho cinil by, toho slyší.
\par 32 Od veku není slýcháno, aby kdo otevrel oci slepého tak narozeného.
\par 33 Byt tento nebyl od Boha, nemohlt by nic uciniti.
\par 34 Odpovedeli a rekli jemu: Ty jsi všecken se v hríších narodil, a ty náš ucíš? I vyhnali jej ven.
\par 35 Uslyšel pak Ježíš, že jsou jej vyhnali ven. A když jej nalezl, rekl jemu: Veríš-liž ty v Syna Božího?
\par 36 Odpovedel on a rekl: I kdož jest, Pane, abych veril v neho?
\par 37 I rekl jemu Ježíš: I videl jsi ho, a kterýž mluví s tebou, ont jest.
\par 38 A on rekl: Verím, Pane, a klanel se jemu.
\par 39 I rekl jemu Ježíš: Na soud prišel jsem já na tento svet, aby ti, kteríž nevidí, videli, a ti, jenž vidí, aby slepí byli.
\par 40 I slyšeli to nekterí z farizeu, kteríž s ním byli, a rekli jemu: Zdali i my slepí jsme?
\par 41 Rekl jim Ježíš: Byste slepí byli, hríchu byste nemeli; ale nyní pravíte: Vidíme, protož hrích váš zustává.

\chapter{10}

\par 1 Amen, amen pravím vám: Kdož nevchází dvermi do ovcince ovcí, ale vchází jinudy, ten zlodej jest a lotr.
\par 2 Ale kdož vchází dvermi, pastýr jest ovcí.
\par 3 Tomut vrátný otvírá, a ovce hlas jeho slyší, a on svých vlastních ovec ze jména povolává, a vyvodí je.
\par 4 A jakž ovce své vlastní ven vypustí, pred nimi jde, a ovce jdou za ním; nebo znají hlas jeho.
\par 5 Ale cizího nikoli následovati nebudou, ale utekou od neho; nebo neznají hlasu cizích.
\par 6 To prísloví povedel jim Ježíš, ale oni nevedeli, co by to bylo, což jim mluvil.
\par 7 Tedy opet rekl jim Ježíš: Amen, amen pravím vám: Že já jsem dvere ovcí.
\par 8 Všickni, kolikož jich koli prede mnou prišlo, zlodeji jsou a lotri, ale neslyšely jich ovce.
\par 9 Já jsem dvere. Skrze mne všel-li by kdo, spasen bude, a vejde i vyjde, a pastvu nalezne.
\par 10 Zlodej neprichází, jediné aby kradl a mordoval a hubil; já jsem prišel, aby život mely, a hojne aby mely.
\par 11 Já jsem ten pastýr dobrý. Dobrý pastýr duši svou pokládá za ovce.
\par 12 Ale nájemník a ten, kterýž není pastýr, jehož nejsou ovce vlastní, vida vlka, an jde, i opouští ovce i utíká, a vlk lapá a rozhání ovce.
\par 13 Nájemník pak utíká; nebo nájemník jest, a nemá péce o ovce.
\par 14 Já jsem ten dobrý pastýr, a známt ovce své, a znajít mne mé.
\par 15 Jakož mne zná Otec, tak i já znám Otce, a duši svou pokládám za ovce.
\par 16 A mámt i jiné ovce, kteréž nejsou z tohoto ovcince. I tyt musím privésti, a hlas muj slyšeti budou. A budet jeden ovcinec a jeden pastýr.
\par 17 Protož mne Otec miluje, že já pokládám duši svou, abych ji zase vzal.
\par 18 Nižádnýt jí nebére ode mne, ale já pokládám ji sám od sebe. Mám moc položiti ji, a mám moc zase vzíti ji. To prikázání vzal jsem od Otce svého.
\par 19 Tedy stala se opet ruznice mezi Židy pro ty reci.
\par 20 A pravili mnozí z nich: Dábelství má a blázní. Co ho posloucháte?
\par 21 Jiní pravili: Tato slova nejsou dábelství majícího. Zdaliž dábelství muže slepých oci otvírati?
\par 22 I bylo posvícení v Jeruzaléme, a zima byla.
\par 23 I procházel se Ježíš v chráme po sínci Šalomounove.
\par 24 Tedy obstoupili jej Židé, a rekli jemu: Dokudž duši naši držíš? Jestliže jsi ty Kristus, povez nám zjevne.
\par 25 Odpovedel jim Ježíš: Povedel jsem vám, a neveríte. Skutkové, kteréž já ciním ve jménu Otce svého, tit svedectví vydávají o mne.
\par 26 Ale vy neveríte, nebo nejste z ovcí mých, jakož jsem vám povedel.
\par 27 Nebo ovce mé hlas muj slyší, a já je znám, a následujít mne.
\par 28 A ját život vecný dávám jim, a nezahynout na veky, aniž jich kdo vytrhne z ruky mé.
\par 29 Otec muj, kterýž mi je dal, vetšít jest nade všecky, a žádnýt jich nemuže vytrhnouti z ruky Otce mého.
\par 30 Já a Otec jedno jsme.
\par 31 Tedy zchápali opet kamení Židé, aby jej kamenovali.
\par 32 Odpovedel jim Ježíš: Mnohé dobré skutky ukázal jsem vám od Otce svého. Pro který z tech skutku kamenujete mne?
\par 33 Odpovedeli jemu Židé, rkouce: Pro dobrý skutek tebe nekamenujeme, ale pro rouhání, totiž že ty, clovek jsa, deláš se Bohem.
\par 34 Odpovedel jim Ježíš: Však psáno jest v Zákone vašem: Já jsem rekl: Bohové jste.
\par 35 Ponevadž ty nazval bohy, k nimžto rec Boží stala se, a nemužet zrušeno býti Písmo,
\par 36 Kterakž tedy o mne, kteréhož posvetil Otec a poslal na svet, vy pravíte, že se rouhám, že jsem rekl: Syn Boží jsem?
\par 37 Neciním-lit skutku Otce svého, neverte mi.
\par 38 Paklit ciním, tedy byste pak mne neverili, aspon skutkum verte, abyste poznali a verili, že Otec ve mne jest, a já v nem.
\par 39 Tedy opet hledali ho jíti, ale on vyšel z rukou jejich.
\par 40 I odšel opet za Jordán na to místo, kdež nejprv Jan krtil, a pozustal tam.
\par 41 I prišli k nemu mnozí, a pravili: Jan zajisté žádného divu neucinil, ale všecko, cožkoli mluvil Jan o tomto, pravé bylo.
\par 42 A mnozí tam uverili v neho.

\chapter{11}

\par 1 Byl pak nemocen clovek nejaký jménem Lazar z Betany, totiž z mestecka Marie a Marty, sestry její.
\par 2 (A to byla ta Maria, kteráž pomazala Pána mastí a vytrela nohy jeho vlasy svými, jejížto bratr Lazar byl nemocen.)
\par 3 Tedy poslaly k nemu sestry jeho, rkouce: Pane, aj, ten, kteréhož miluješ, nemocen jest.
\par 4 A uslyšav to Ježíš, rekl: Nemoc ta nenít k smrti, ale pro slávu Boží, aby oslaven byl Syn Boží skrze ni.
\par 5 Miloval pak Ježíš Martu i sestru její i Lazara.
\par 6 A jakž uslyšel, že by nemocen byl, i pozustal za dva dni na tom míste, kdež byl.
\par 7 Potom pak dí ucedlníkum: Pojdme zase do Judstva.
\par 8 Rekli jemu ucedlníci: Mistre, nyní hledali te Židé kamenovati, a ty zase tam chceš jíti?
\par 9 Odpovedel Ježíš: Však dvanácte hodin za den jest. Chodí-li kdo ve dne, neurazí se; nebo svetlo tohoto sveta vidí.
\par 10 Paklit by kdo chodil v noci, urazít se; nebo svetla není v nem.
\par 11 To povedel, a potom dí jim: Lazar, prítel náš, spí, ale jdut, abych jej ze sna probudil.
\par 12 I rekli ucedlníci jeho: Pane, spí-lit, zdráv bude.
\par 13 Ale Ježíš rekl o smrti jeho, oni pak domnívali se, že by o spání sna mluvil.
\par 14 Tedy rekl jim Ježíš zjevne: Lazar umrel.
\par 15 A raduji se pro vás, že jsem tam nebyl, abyste verili. Ale pojdme k nemu.
\par 16 I rekl Tomáš, kterýž slove Didymus, spoluucedlníkum: Pojdme i my, abychom zemreli s ním.
\par 17 Tedy prišel Ježíš, i nalezl ho již ctyri dni v hrobe pochovaného.
\par 18 Byla pak Betany blízko od Jeruzaléma, okolo honu patnácte.
\par 19 Mnozí pak z Židu byli prišli k Marte a Mariji, aby je tešili pro smrt bratra jejich.
\par 20 Tedy Marta, jakž uslyšela, že Ježíš jde, vyšla proti nemu, ale Maria doma sedela.
\par 21 I rekla Marta k Ježíšovi: Pane, kdybys ty byl zde, bratr muj byl by neumrel.
\par 22 Ale i nynít vím, že cožkoli požádal bys od Boha, dá tobe Buh.
\par 23 Dí jí Ježíš: Vstanet bratr tvuj.
\par 24 Rekla jemu Marta: Vím, že vstane pri vzkríšení v den nejposlednejší.
\par 25 Rekl jí Ježíš: Já jsem vzkríšení i život. Kdo verí ve mne, byt pak i umrel, živ bude.
\par 26 A každý, kdož jest živ, a verí ve mne, neumret na veky. Veríš-li tomu?
\par 27 Rekla jemu: Ovšem, Pane, já jsem uverila, že jsi ty Kristus, Syn Boží, kterýž mel prijíti na svet.
\par 28 A když to povedela, odešla a zavolala tajne Marie, sestry své, rkuci: Mistr zde jest, a volá tebe.
\par 29 Ona jakž to uslyšela, vstala rychle, a šla k nemu.
\par 30 (Ješte pak byl Ježíš neprišel do mestecka, ale byl na tom míste, kdež vyšla byla proti nemu Marta.)
\par 31 Tedy Židé, kteríž s ní byli v dome a tešili ji, videvše Mariji, že jest rychle vstala a vyšla, šli za ní, rkouce: Jde k hrobu, aby tam plakala.
\par 32 Ale Maria, když tam prišla, kdež byl Ježíš, uzrevši jej, padla k nohám jeho, a rekla jemu: Pane, bys ty byl zde, bratr muj byl by neumrel.
\par 33 Ježíš pak jakž uzrel, ana pláce, i Židy, kteríž byli s ní prišli, ani plací, zastonal duchem, a zkormoutil se.
\par 34 A rekl: Kdež jste jej položili? Rkou jemu: Pane, pojd a pohled.
\par 35 I zaplakal Ježíš.
\par 36 Tedy rekli Židé: Aj, kterak ho miloval!
\par 37 Nekterí pak z nich rekli: Nemohl-liž jest tento, kterýž otevrel oci slepého, uciniti i toho, aby tento neumrel?
\par 38 Ježíš pak opet zastonav sám v sobe, prišel k hrobu. Byla pak jeskyne, a kámen byl svrchu položen na ni.
\par 39 I dí Ježíš: Zdvihnete kámen. Rekla jemu Marta, sestra toho mrtvého: Pane, jižt smrdí; nebo ctyri dni v hrobe jest.
\par 40 Dí jí Ježíš: Všakt jsem rekl, že budeš-li veriti, uzríš slávu Boží.
\par 41 Tedy zdvihli kámen, kdež byl mrtvý pochován. Ježíš pak pozdvihl vzhuru ocí a rekl: Otce, dekuji tobe, že jsi mne slyšel.
\par 42 Já zajisté vím, že ty mne vždycky slyšíš, ale pro zástup, kterýž okolo stojí, rekl jsem, aby verili, že jsi ty mne poslal.
\par 43 A to povedev, zavolal hlasem velikým: Lazare, pojd ven!
\par 44 I vyšel, kterýž byl umrel, maje svázané ruce i nohy rouchami, a tvár jeho šatem byla obvinuta. Rekl jim Ježíš: Rozvežtež jej, a nechte, at odejde.
\par 45 Tedy mnozí z Židu, kteríž byli prišli k Mariji, videvše, co jest ucinil Ježíš, uverili v neho.
\par 46 Nekterí pak z nich odešli k farizeum a povedeli jim, co jest ucinil Ježíš.
\par 47 I sešli se prední kneží a farizeové v radu, a pravili: Co ciníme? Tento clovek divy mnohé ciní.
\par 48 Necháme-li ho tak, všickni uverí v neho, i prijdou Rímané, a odejmou místo naše i lid.
\par 49 Jeden pak z nich, jménem Kaifáš, nejvyšším knezem jsa toho léta, rekl jim: Vy nic nevíte,
\par 50 Aniž co o tom premyšlujete, že jest užitecné nám, aby jeden clovek umrel za lid, a ne, aby všecken tento národ zahynul.
\par 51 Toho pak nerekl sám od sebe, ale nejvyšším knezem byv léta toho, prorokoval, že jest mel Ježíš umríti za tento národ,
\par 52 A netoliko za tento národ, ale také, aby syny Boží rozptýlené shromáždil v jedno.
\par 53 Protož od toho dne spolu se o to radili, aby jej zabili.
\par 54 Ježíš pak již nechodil zjevne mezi Židy, ale odšel odtud do krajiny, kteráž byla blízko poušte, do mesta, jenž slove Efraim, a tu bydlil s ucedlníky svými.
\par 55 Byla pak blízko velikanoc Židovská. I šli mnozí do Jeruzaléma z krajiny té pred velikonocí, aby se ocistili.
\par 56 I hledali Ježíše, a rozmlouvali vespolek, v chráme stojíce: Co se vám zdá, že neprišel k svátku?
\par 57 Vydali pak byli prední kneží a farizeové mandát, jestliže by kdo zvedel, kde by byl, aby povedel, aby jej jali.

\chapter{12}

\par 1 Tedy Ježíš šestý den pred velikonocí prišel do Betany, kdežto byl Lazar, ten kterýž byl umrel, jehož vzkrísil z mrtvých.
\par 2 I pripravili jemu tu veceri, a Marta posluhovala, Lazar pak byl jeden z stolících s nimi.
\par 3 Maria pak vzavši libru masti drahé z nardu výborného, pomazala noh Ježíšových, a vytrela vlasy svými nohy jeho. I naplnen jest dum vuní té masti.
\par 4 Tedy rekl jeden z ucedlníku jeho, Jidáš, syn Šimona Iškariotského, kterýž jej mel zraditi:
\par 5 Proc tato mast není prodána za tri sta penez, a není dáno chudým?
\par 6 To pak rekl, ne že by mel péci o chudé, ale že zlodej byl, a mešec mel, a to, což do neho kladeno bylo, nosil.
\par 7 Tedy rekl Ježíš: Nech jí, ke dni pohrebu mého zachovala to.
\par 8 Chudé zajisté vždycky máte s sebou, ale mne ne vždycky míti budete.
\par 9 Zvedel pak zástup veliký z Židu o nem, že by tu byl. I prišli tam, ne pro Ježíše toliko, ale také, aby Lazara videli, kteréhož byl vzkrísil z mrtvých.
\par 10 Radili se pak prední kneží, aby i Lazara zamordovali.
\par 11 Nebo mnozí z Židu odcházeli pro neho, a uverili v Ježíše.
\par 12 Potom nazejtrí mnohý zástup, kterýž byl prišel k svátku velikonocnímu, když uslyšeli, že Ježíš jde do Jeruzaléma,
\par 13 Nabrali ratolestí palmových, a vyšli proti nemu, a volali: Spas nás! Požehnaný, jenž se bére ve jménu Páne, Král Izraelský.
\par 14 I dostav Ježíš oslátka, vsedl na ne, jakož psáno jest:
\par 15 Neboj se, dcero Sionská, aj, Král tvuj bére se, na oslátku sede.
\par 16 Tomu pak nesrozumeli ucedlníci jeho zprvu, ale když oslaven byl Ježíš, tedy se rozpomenuli, že jest to psáno bylo o nem a že jemu to ucinili.
\par 17 Vydával pak o nem svedectví zástup, kterýž byl s ním, že Lazara povolal z hrobu a vzkrísil jej z mrtvých.
\par 18 Protož i v cestu vyšel jemu zástup, když slyšeli, že by ten div ucinil.
\par 19 Tedy farizeové pravili mezi sebou: Vidíte, že nic neprospíváte? Aj, všecken svet postoupil po nem.
\par 20 Byli pak nekterí Rekové z tech, kteríž pricházívali, aby se modlili v svátek.
\par 21 Ti také pristoupili k Filipovi, kterýž byl od Betsaidy Galilejské, a prosili ho, rkouce: Pane, chteli bychom Ježíše videti.
\par 22 Prišel Filip a povedel Ondrejovi, Ondrej pak a Filip povedeli Ježíšovi.
\par 23 A Ježíš odpovedel jim, rka: Prišlat jest hodina, aby oslaven byl Syn cloveka.
\par 24 Amen, amen pravím vám: Zrno pšenicné padna v zemi, neumre-li, onot samo zustane, a paklit umre, mnohý užitek prinese.
\par 25 Kdož miluje duši svou, ztratít ji; a kdož nenávidí duše své na tomto svete, k životu vecnému ostríhá jí.
\par 26 Slouží-li mi kdo, následujž mne; a kdež jsem já, tut i muj služebník bude. A bude-li mi kdo sloužiti, poctít ho Otec muj.
\par 27 Nyní duše má zkormoucena jest, a což dím? Otce, vysvobod mne z této hodiny. Ale proto jsem prišel k hodine této.
\par 28 Otce, oslaviž jméno své. Tedy prišel hlas s nebe rkoucí: I oslavil jsem, i ješte oslavím.
\par 29 Ten pak zástup, kterýž tu stál a to slyšel, pravil: Zahrmelo. Jiní pravili: Andel k nemu mluvil.
\par 30 Odpovedel Ježíš a rekl: Ne pro mnet hlas tento se stal, ale pro vás.
\par 31 Nynít jest soud sveta tohoto, nyní kníže sveta tohoto vyvrženo bude ven.
\par 32 A já budu-lit povýšen od zeme, všecky potáhnu k sobe.
\par 33 (To pak povedel, znamenaje, kterou by smrtí mel umríti.)
\par 34 Odpovedel jemu zástup: My jsme slyšeli z Zákona, že Kristus zustává na veky, a kterakž ty pravíš, že musí býti povýšen Syn cloveka? Kdo jest to Syn cloveka?
\par 35 Tedy rekl jim Ježíš: Ješte na malý cas Svetlo s vámi jest. Chodte, dokud Svetlo máte, at vás tma nezachvátí; nebo kdo chodí ve tmách, neví, kam jde.
\par 36 Dokud Svetlo máte, verte v Svetlo, abyste synové Svetla byli. Toto povedel Ježíš, a odšed, skryl se pred nimi.
\par 37 A ackoli tak mnohá znamení cinil pred nimi, však jsou neuverili v neho,
\par 38 Aby se naplnila rec Izaiáše proroka, kterouž povedel: Pane, kdo uveril kázaní našemu a ráme Páne komu jest zjeveno?
\par 39 Ale protot jsou nemohli veriti, neb opet Izaiáš rekl:
\par 40 Oslepil oci jejich a zatvrdil srdce jejich, aby ocima nevideli a srdcem nerozumeli a neobrátili se, abych jich neuzdravil.
\par 41 To povedel Izaiáš, když videl slávu jeho a mluvil o nem.
\par 42 A ackoli mnozí z knížat uverili v neho, však pro farizee nevyznávali ho, aby ze školy nebyli vyobcováni.
\par 43 Nebo milovali slávu lidskou více než slávu Boží.
\par 44 Ježíš pak zvolal a rekl: Kdo verí ve mne, ne ve mnet verí, ale v toho, jenž mne poslal.
\par 45 A kdož vidí mne, vidí toho, kterýž mne poslal.
\par 46 Já Svetlo na svet jsem prišel, aby žádný, kdož verí ve mne, ve tme nezustal.
\par 47 A slyšel-lit by kdo slova má, a neveril by, ját ho nesoudím; nebo neprišel jsem, abych soudil svet, ale abych spasen ucinil svet.
\par 48 Kdož mnou pohrdá a neprijímá slov mých, mát, kdo by jej soudil. Slova, kteráž jsem mluvil, tat jej souditi budou v nejposlednejší den.
\par 49 Nebo já sám od sebe jsem nemluvil, ale ten, jenž mne poslal, Otec, on mi prikázaní dal, co bych mel praviti a mluviti.
\par 50 A vím, že prikázání jeho jest život vecný. A protož, což já mluvím, jakž mi povedel Otec, takt mluvím.

\chapter{13}

\par 1 Pred svátkem pak velikonocním, veda Ježíš, že prišla hodina jeho, aby šel z tohoto sveta k Otci, milovav své, kteríž byli na svete, až do konce je miloval.
\par 2 A když vecereli, (a dábel již byl vnukl v srdce Jidáše Šimona Iškariotského, aby jej zradil,)
\par 3 Veda Ježíš, že jemu Otec všecko v ruce dal a že od Boha vyšel a k Bohu jde,
\par 4 Vstal od vecere, a složil roucho své, a vzav rouchu, prepásal se.
\par 5 Potom nalil vody do medenice, i pocal umývati nohy ucedlníku a vytírati rouchou, kterouž byl prepásán.
\par 6 Tedy prišel k Šimonovi Petrovi. A on rekl jemu: Pane, ty mi chceš nohy mýti?
\par 7 Odpovedel Ježíš a rekl jemu: Co já ciním, ty nevíš nyní, ale potom zvíš.
\par 8 Dí jemu Petr: Nebudeš ty mýti noh mých na veky. Odpovedel jemu Ježíš: Neumyji-lit tebe, nebudeš míti dílu se mnou.
\par 9 Dí jemu Šimon Petr: Pane, netoliko nohy mé, ale i ruce i hlavu.
\par 10 Rekl jemu Ježíš: Kdož jest umyt, nepotrebuje, než aby nohy umyl; nebo cist jest všecken. I vy cistí jste, ale ne všickni.
\par 11 Nebo vedel o zrádci svém; protož rekl: Ne všickni cistí jste.
\par 12 Když pak umyl nohy jejich a vzal na sebe roucho své, posadiv se za stul zase, rekl jim: Víte-liž, co jsem ucinil vám?
\par 13 Vy nazýváte mne Mistrem a Pánem, a dobre pravíte, jsemt zajisté.
\par 14 Ponevadž tedy já umyl jsem nohy vaše, Pán a Mistr, i vy máte jeden druhému nohy umývati.
\par 15 Príklad zajisté dal jsem vám, abyste, jakož jsem já ucinil vám, i vy též cinili.
\par 16 Amen, amen pravím vám: Není služebník vetší pána svého, ani posel vetší nežli ten, kdož jej poslal.
\par 17 Znáte-li tyto veci, blahoslavení jste, budete-li je ciniti.
\par 18 Ne o všecht vás pravím. Já vím, které jsem vyvolil, ale musí to býti, aby se naplnilo písmo: Ten, jenž jí chléb se mnou, pozdvihl proti mne paty své.
\par 19 Nyní pravím vám, prve než by se to stalo, abyste, když se stane, uverili, že já jsem.
\par 20 Amen, amen pravím vám: Kdo prijímá toho, kohož bych já poslal, mnet prijímá; a kdož mne prijímá, prijímá toho, kterýž mne poslal.
\par 21 A to povedev Ježíš, zkormoutil se v duchu, a osvedcil, rka: Amen, amen pravím vám, že jeden z vás mne zradí.
\par 22 Tedy ucedlníci vzhlédali na sebe vespolek, pochybujíce, o kom by to pravil.
\par 23 Byl pak jeden z ucedlníku Ježíšových, kterýž zpolehl na klíne jeho, jehož miloval Ježíš.
\par 24 Protož tomu náveští dal Šimon Petr, aby se zeptal, kdo by to byl, o nemž praví?
\par 25 A on odpocívaje na prsech Ježíšových, rekl jemu: Pane, kdo jest?
\par 26 Odpovedel Ježíš: Ten jest, komuž já omocené skyvy chleba podám. A omociv skyvu chleba, podal Jidášovi, synu Šimona Iškariotského.
\par 27 A hned po vzetí toho chleba vstoupil do neho satan. I rekl jemu Ježíš: Co ciníš, cin spešne.
\par 28 Tomu pak žádný z prísedících nerozumel, k cemu by jemu to rekl.
\par 29 Nebo nekterí se domnívali, ponevadž Jidáš mešec mel, že by jemu rekl Ježíš: Nakup tech vecí, kterýchž jest nám potrebí k svátku, anebo aby neco chudým dal.
\par 30 A on vzav skyvu chleba, hned vyšel. A byla již noc.
\par 31 Když pak on vyšel, dí Ježíš: Nynít oslaven jest Syn cloveka, a Buh oslaven jest v nem.
\par 32 A ponevadž Buh oslaven jest v nem, i Buh oslaví jej sám v sobe, a to hned oslaví jej.
\par 33 Synáckové, ješte malicko s vámi jsem. Hledati mne budete, a jakož jsem rekl Židum: Kam já jdu, vy nemužete prijíti, tak i vám pravím nyní.
\par 34 Prikázání nové dávám vám, abyste se milovali vespolek; jakož já miloval jsem vás, tak abyste i vy milovali jeden druhého.
\par 35 Po tomt poznají všickni, že jste moji ucedlníci, budete-li míti lásku jedni k druhým.
\par 36 Dí jemu Šimon Petr: Pane, kam jdeš? Odpovedel mu Ježíš: Kam já jdu, nemužeš ty nyní jíti za mnou, ale pujdeš potom.
\par 37 Dí jemu Petr: Pane, proc bych nemohl nyní jíti za tebou? A já duši svou za tebe položím.
\par 38 Odpovedel jemu Ježíš: Duši svou za mne položíš? Amen pravím tobe: Nezazpívát kohout, až mne trikrát zapríš.

\chapter{14}

\par 1 Nermutiž se srdce vaše. Veríte v Boha, i ve mne verte.
\par 2 V domu Otce mého príbytkové mnozí jsou. Byt nebylo tak, povedel bych vám.
\par 3 Jdut, abych vám pripravil místo. A odejdu-lit, a pripravím vám místo, zaset prijdu, a poberu vás k sobe samému, abyste, kde jsem já, i vy byli.
\par 4 A kam já jdu, víte, i cestu víte.
\par 5 Dí jemu Tomáš: Pane, nevíme, kam jdeš. A kterak mužeme cestu vedeti?
\par 6 Dí jemu Ježíš: Já jsem cesta, i pravda, i život. Žádný neprichází k Otci než skrze mne.
\par 7 Byste znali mne, také i Otce mého znali byste; a již nyní jej znáte, a videli jste ho.
\par 8 Rekl jemu Filip: Pane, ukaž nám Otce, a dostit jest nám.
\par 9 Dí jemu Ježíš: Tak dlouhý cas s vámi jsem, a nepoznal jsi mne? Filipe, kdož vidí mne, vidí Otce, a kterak ty pravíš: Ukaž nám Otce?
\par 10 A což neveríš, že já v Otci a Otec ve mne jest? Slova, kteráž já mluvím vám, sám od sebe nemluvím, ale Otec ve mne prebývaje, ont ciní skutky.
\par 11 Vertež mi, že jsem já v Otci a Otec ve mne; nebo aspon pro samy skutky verte mi.
\par 12 Amen, amen pravím vám: Kdož verí ve mne, skutky, kteréž já ciním, i on ciniti bude, a vetší nad ty ciniti bude. Nebo já jdu k Otci svému.
\par 13 A jestliže byste co prosili ve jménu mém, tot uciním, aby oslaven byl Otec v Synu.
\par 14 Budete-li zac prositi ve jménu mém, ját uciním.
\par 15 Milujete-li mne, prikázání mých ostríhejte.
\par 16 A ját prositi budu Otce, a jiného Utešitele dá vám, aby s vámi zustal na veky,
\par 17 Ducha pravdy, jehož svet nemuže prijíti. Nebo nevidí ho, aniž ho zná, ale vy znáte jej, nebt u vás prebývá a v vás bude.
\par 18 Neopustímt vás sirotku, prijdut k vám.
\par 19 Ješte malicko, a svet mne již neuzrí, ale vy uzríte mne; nebo já živ jsem, a i vy živi budete.
\par 20 V ten den vy poznáte, že já jsem v Otci svém, a vy ve mne, a já v vás.
\par 21 Kdož by mel prikázaní má a ostríhal jich, ont jest ten, kterýž mne miluje. A kdož mne miluje, milován bude od Otce mého, a ját jej budu milovati a zjevím jemu samého sebe.
\par 22 Rekl jemu Judas, ne onen Iškariotský: Pane, jakž jest to, že sebe nám zjeviti chceš, a ne svetu?
\par 23 Odpovedel Ježíš a rekl jemu: Miluje-li mne kdo, slova mého ostríhati bude, a Otec muj bude jej milovati, a k nemu prijdeme, a príbytek u neho uciníme.
\par 24 Kdož pak nemiluje mne, slov mých neostríhá; a slovo, kteréž slyšíte, nenít mé, ale toho, kterýž mne poslal, Otcovo.
\par 25 Toto mluvil jsem vám, u vás prebývaje.
\par 26 Utešitel pak, ten Duch svatý, kteréhož pošle Otec ve jménu mém, ont vás naucí všemu a pripomenet vám všecko, což jsem koli mluvil vám.
\par 27 Pokoj zustavuji vám, pokoj muj dávám vám; ne jako svet dává, já dávám vám. Nermutiž se srdce vaše, ani strachuj.
\par 28 Slyšeli jste, že já rekl jsem vám: Jdu, a zase prijdu k vám. Kdybyste mne milovali, radovali byste se jiste, že jsem rekl: Jdu k Otci; nebo Otec vetší mne jest.
\par 29 A nyní povedel jsem vám, prve nežli by se stalo, abyste, když se stane, uverili.
\par 30 Již nemnoho mluviti budu s vámi; nebot jde Kníže tohoto sveta, ale nemát nic na mne.
\par 31 Ale aby poznal svet, že miluji Otce, a jakož mi prikázání dal Otec, tak ciním. Vstante, pojdme odtud.

\chapter{15}

\par 1 Já jsem ten vinný kmen pravý, a Otec muj vinar jest
\par 2 Každou ratolest, kteráž ve mne nenese ovoce, odrezuje, a každou, kteráž nese ovoce, cistí, aby hojnejší ovoce nesla.
\par 3 Již vy cisti jste pro rec, kterouž jsem mluvil vám.
\par 4 Zustantež ve mne, a já v vás. Jakož ratolest nemuže nésti ovoce sama od sebe, nezustala-li by pri kmenu, takž ani vy, lec zustanete ve mne.
\par 5 Já jsem vinný kmen a vy ratolesti. Kdo zustává ve mne, a já v nem, ten nese ovoce mnohé; nebo beze mne nic nemužete uciniti.
\par 6 Nezustal-li by kdo ve mne, vyvržen bude ven jako ratolest, a uschnet, a sberout ty ratolesti a na ohen uvrhou a shorít.
\par 7 Zustanete-li ve mne, a slova má zustanou-lit v vás, což byste koli chteli, proste, a stanet se vám.
\par 8 V tomt bývá oslaven Otec muj, když ovoce nesete hojné, a takt budete moji ucedlníci.
\par 9 Jakož miloval mne Otec, tak i já miloval jsem vás. Zustantež v milování mém.
\par 10 Budete-li zachovávati prikázaní má, zustanete v mém milování, jakož i já prikázání Otce svého zachoval jsem, i zustávám v jeho milování.
\par 11 Toto mluvil jsem vám, aby radost má zustala v vás, a radost vaše aby byla plná.
\par 12 Totot jest prikázání mé, abyste se milovali vespolek, jako i já miloval jsem vás.
\par 13 Vetšího milování nad to žádný nemá, než aby duši svou položil za prátely své.
\par 14 Vy prátelé moji jste, uciníte-li to, což já prikazuji vám.
\par 15 Již vás nebudu více nazývati služebníky, nebo služebník neví, co by cinil pán jeho. Ale vás jsem nazval práteli, nebo všecko, což jsem koli slyšel od Otce svého, oznámil jsem vám.
\par 16 Ne vy jste mne vyvolili, ale já jsem vás vyvolil a postavil, abyste šli a ovoce prinesli, a ovoce vaše aby zustalo, aby zac byste koli prosili Otce ve jménu mém, dal vám.
\par 17 To prikazuji vám, abyste se milovali vespolek.
\par 18 Jestližet vás svet nenávidí, víte, žet jest mne prve než vás v nenávisti mel.
\par 19 Byste byli z sveta, svet, což jest jeho, miloval by; že pak nejste z sveta, ale já z sveta vyvolil jsem vás, protož vás svet nenávidí.
\par 20 Pamatujte na rec mou, kterouž jsem já mluvil vám: Nenít služebník vetší nežli pán jeho. Ponevadž se mne protivili, i vámt se protiviti budou; ponevadž jsou reci mé šetrili, i vaší šetriti budou.
\par 21 Ale toto všecko uciní vám pro jméno mé; nebo neznají toho, jenž mne poslal.
\par 22 Kdybych byl neprišel a nemluvil jim, hríchu by nemeli; ale nyní výmluvy nemají z hríchu svého.
\par 23 Kdož mne nenávidí, i Otcet mého nenávidí.
\par 24 Bych byl skutku necinil mezi nimi, jichžto žádný jiný necinil, hríchu by nemeli; ale nyní jsou i videli, i nenávideli, i mne i Otce mého.
\par 25 Ale musilo tak býti, aby se naplnila rec, kteráž v Zákone jejich napsána jest: Že v nenávisti meli mne bez príciny.
\par 26 Když pak prijde Utešitel, kteréhož já pošli vám od Otce, Duch pravdy, kterýž od Otce pochází, tent svedectví vydávati bude o mne.
\par 27 Ano i vy svedectví vydávati budete, nebo od pocátku se mnou jste.

\chapter{16}

\par 1 Toto mluvil jsem vám, abyste se nezhoršili.
\par 2 Vypovedít vás ze škol, ano prijdet cas, že všeliký, kdož vás mordovati bude, domnívati se bude, že tím Bohu slouží.
\par 3 A tot uciní vám proto, že nepoznali Otce ani mne.
\par 4 Ale toto mluvil jsem vám, abyste, když prijde ten cas, rozpomenuli se na to, že jsem já to predpovedel vám. Tohot jsem vám s pocátku nemluvil, neb jsem byl s vámi.
\par 5 Nyní pak jdu k tomu, kterýž mne poslal, a žádný z vás neptá se mne: Kam jdeš?
\par 6 Ale že jsem vám tyto veci mluvil, zámutek naplnil srdce vaše.
\par 7 Já pak pravdu pravím vám, že jest vám užitecné, abych já odšel. Nebo neodejdu-lit, Utešitel neprijde k vám; a paklit odejdu, pošli ho k vám.
\par 8 A ont prijda, obvinovati bude svet z hríchu, a z spravedlnosti, a z soudu.
\par 9 Z hríchu zajisté, že neverí ve mne;
\par 10 A z spravedlnosti, že jdu k Otci, a již více neuzríte mne;
\par 11 Z soudu pak, že Kníže tohoto sveta již jest odsouzeno.
\par 12 Ještet bych mel mnoho mluviti vám, ale nemužete snésti nyní.
\par 13 Ale když prijde ten Duch pravdy, uvedet vás ve všelikou pravdu. Nebo nebude mluviti sám od sebe, ale cožkoli uslyší, tot mluviti bude; ano i budoucí veci zvestovati bude vám.
\par 14 Ont mne oslaví; nebo z mého vezme, a zvestuje vám.
\par 15 Všecko, cožkoli má Otec, mé jest. Protož jsem rekl, že z mého vezme, a zvestuje vám.
\par 16 Malicko, a neuzríte mne, a opet malicko, a uzríte mne; nebo já jdu k Otci.
\par 17 I rekli nekterí z ucedlníku jeho mezi sebou: Co jest to, že praví nám: Malicko, a neuzríte mne, a opet malicko, a uzríte mne, a že já jdu k Otci?
\par 18 Protož pravili: Co jest to, že praví: Malicko? Nevíme, co praví.
\par 19 I poznal Ježíš, že se ho chteli otázati. I rekl jim: O tom tížete mezi sebou, že jsem rekl: Malicko, a neuzríte mne, a opet malicko, a uzríte mne?
\par 20 Amen, amen pravím vám, že plakati a kvíliti budete vy, ale svet se bude radovati; vy pak se budete rmoutiti, ale zámutek váš obrátít se v radost.
\par 21 Žena, když rodí, zámutek má, nebo prišla hodina její; ale když porodí díte, již nepamatuje na soužení, pro radost, že se narodil clovek na svet.
\par 22 Protož i vy zámutek máte nyní, ale opet uzrím vás, a radovati se bude srdce vaše, a radosti vaší žádný neodejme od vás.
\par 23 A v ten den nebudete se mne tázati o nicemž. Amen, amen pravím vám: Že zac byste koli prosili Otce ve jménu mém, dát vám.
\par 24 Až dosavad za nic jste neprosili ve jménu mém. Prostež, a vezmete, aby radost vaše doplnena byla.
\par 25 Toto v príslovích mluvil jsem vám; prijdet hodina, když již ne v príslovích budu mluviti vám, ale zjevne o Otci svém zvestovati budu vám.
\par 26 V ten den ve jménu mém prositi budete, a nepravímt vám, že já budu prositi Otce za vás.
\par 27 Nebo sám Otec miluje vás, proto že jste vy mne milovali, a uverili, že jsem já od Boha vyšel.
\par 28 Vyšelt jsem od Otce, a prišel jsem na svet; a opet opouštím svet, a jdu k Otci.
\par 29 Rkou jemu ucedlníci jeho: Aj, nyní zjevne mluvíš, a prísloví žádného nepravíš.
\par 30 Nyní víme, že víš všecko, a nepotrebuješ, aby se kdo tebe tázal. Skrze to veríme, že jsi od Boha prišel.
\par 31 Odpovedel jim Ježíš: Nyní veríte.
\par 32 Aj, prijdet hodina, anobrž již prišla, že se rozprchnete jeden každý k svému, a mne samého necháte. Ale nejsemt sám, nebo Otec se mnou jest.
\par 33 Tyto veci mluvil jsem vám, abyste ve mne pokoj meli. Na svete soužení míti budete, ale doufejtež, ját jsem premohl svet.

\chapter{17}

\par 1 To povedev Ježíš, i pozdvihl ocí svých k nebi a rekl: Otce, prišlat jest hodina, oslaviž Syna svého, aby i Syn tvuj oslavil tebe;
\par 2 Jakož jsi dal jemu moc nad každým clovekem, aby tem všechnem, kteréž jsi dal jemu, on život vecný dal.
\par 3 Totot jest pak vecný život, aby poznali tebe samého pravého Boha, a kteréhož jsi poslal, Ježíše Krista.
\par 4 Ját jsem oslavil tebe na zemi; dílo jsem vykonal, kteréž jsi mi dal, abych cinil.
\par 5 A nyní oslaviž ty mne, Otce, u sebe samého, slávou, kterouž jsem mel u tebe, prve nežli svet byl.
\par 6 Oznámil jsem jméno tvé lidem, kteréž jsi mi dal z sveta. Tvojit jsou byli, a mne jsi je dal, a rec tvou zachovali.
\par 7 A nyní poznali, že všecky veci, kteréž jsi mi dal, od tebe jsou.
\par 8 Nebo slova, kteráž jsi mi dal, dal jsem jim; a oni prijali, a poznali vpravde, že jsem od tebe vyšel, a uverili, že jsi ty mne poslal.
\par 9 Já za ne prosím, ne za svet prosím, ale za ty, kteréž jsi mi dal, nebo tvoji jsou.
\par 10 A všecky veci mé tvé jsou, a tvé mé jsou, a já oslaven jsem v nich.
\par 11 Již pak více nejsem na svete, ale oni jsou na svete, a já k tobe jdu. Otce svatý, ostríhejž jich ve jménu svém, kteréž jsi mi dal, at by byli jedno jako i my.
\par 12 Dokudž jsem s nimi byl na svete, já jsem jich ostríhal ve jménu tvém. Kteréž jsi mi dal, zachoval jsem, a žádný z nich nezahynul, než syn zatracení, aby #se Písmo naplnilo.
\par 13 Ale nyní k tobe jdu, a toto mluvím na svete, aby meli radost mou plnou v sobe.
\par 14 Já jsem jim dal slovo tvé, a svet jich nenávidel; nebo nejsou z sveta, jako i já nejsem z sveta.
\par 15 Neprosímt, abys je vzal z sveta, ale abys jich zachoval od zlého.
\par 16 Z svetat nejsou, jakož i já nejsem z sveta.
\par 17 Posvetiž jich v pravde své, slovo tvé pravda jest.
\par 18 Jakož jsi mne poslal na svet, i já jsem je poslal na svet.
\par 19 A já posvecuji sebe samého za ne, aby i oni posveceni byli v pravde.
\par 20 Ne za tytot pak toliko prosím, ale i za ty, kteríž skrze slovo jejich mají uveriti ve mne,
\par 21 Aby všickni jedno byli, jako ty, Otce, ve mne, a já v tobe, aby i oni v nás jedno byli, aby uveril svet, že jsi ty mne poslal.
\par 22 A já slávu, kterouž jsi mi dal, dal jsem jim, aby byli jedno, jakož i my jedno jsme.
\par 23 Já v nich, a ty ve mne, aby dokonáni byli v jedno, a aby poznal svet, že jsi ty mne poslal, a že jsi je miloval, jakožs i mne miloval.
\par 24 Otce, kteréž jsi mi dal, chcit, kdež jsem já, aby i oni byli se mnou, aby hledeli na slávu mou, kteroužs mi dal; nebo jsi mne miloval pred ustanovením sveta.
\par 25 Otce spravedlivý, tebet jest svet nepoznal, ale já jsem tebe poznal, a i tito poznali, že jsi ty mne poslal.
\par 26 A známét jsem jim ucinil jméno tvé, a ješte známo uciním, aby to milování, kterýmž jsi mne miloval, bylo v nich, a i já v nich.

\chapter{18}

\par 1 To když povedel Ježíš, vyšel s ucedlníky svými pres potok Cedron, kdež byla zahrada; do kteréžto všel on i ucedlníci jeho.
\par 2 Vedel pak i Jidáš, zrádce jeho, to místo; nebo casto chodíval tam Ježíš s ucedlníky svými.
\par 3 Protož Jidáš pojav s sebou zástup, a od predních kneží a farizeu služebníky, prišel tam s lucernami a s pochodnemi a s zbrojí.
\par 4 Ježíš pak veda všecko, což prijíti melo na nej, vyšed proti nim, rekl jim: Koho hledáte?
\par 5 Odpovedeli jemu: Ježíše Nazaretského. Rekl jim Ježíš: Ját jsem. Stál pak s nimi i Jidáš, zrádce jeho.
\par 6 A jakž rekl jim: Já jsem, postoupili nazpet, a padli na zem.
\par 7 I otázal se jich opet: Koho hledáte? A oni rekli: Ježíše Nazaretského.
\par 8 Odpovedel Ježíš: Povedel jsem vám, že já jsem. Ponevadž tedy mne hledáte, nechtež techto, at odejdou.
\par 9 Aby se naplnila rec, kterouž byl povedel: Že které jsi mi dal, neztratil jsem z nich žádného.
\par 10 Tedy Šimon Petr, maje mec, vytrhl jej a uderil služebníka nejvyššího kneze a utal mu ucho jeho pravé. A bylo jméno služebníka toho Malchus.
\par 11 I rekl Ježíš Petrovi: Schovej mec svuj do pošvy. Což nemám píti kalicha, kterýž mi dal Otec?
\par 12 Tedy zástup a hejtman a služebníci Židovští jali Ježíše, a svázali jej.
\par 13 A vedli ho k Annášovi nejprve; nebo byl test Kaifášuv, kterýž byl nejvyšším knezem toho léta.
\par 14 Kaifáš pak byl ten, kterýž byl radu dal Židum, že by užitecné bylo, aby clovek jeden umrel za lid.
\par 15 Šel pak za Ježíšem Šimon Petr a jiný ucedlník. A ten ucedlník byl znám nejvyššímu knezi, i všel s Ježíšem do síne nejvyššího kneze.
\par 16 Ale Petr stál u dverí vne. I vyšel ten druhý ucedlník, kterýž byl znám nejvyššímu knezi, a promluvil s vrátnou, i uvedl tam Petra.
\par 17 Tedy rekla Petrovi devecka vrátná: Nejsi-liž i ty z ucedlníku cloveka toho? Rekl on: Nejsem.
\par 18 Stáli pak tu služebníci a pacholci, kteríž ohen udelali, nebo zima bylo, i zhrívali se. A byl s nimi také i Petr, stoje tu a zhrívaje se.
\par 19 Tedy nejvyšší knez tázal se Ježíše o ucedlnících jeho a o ucení jeho.
\par 20 Odpovedel jemu Ježíš: Já zjevne mluvil jsem svetu, já vždycky ucíval jsem v škole a v chráme, kdežto se odevšad Židé scházejí, a tajne jsem nic nemluvil,
\par 21 Co se mne ptáš? Ptej se tech, kteríž mne slýchali, co jsem jim mluvil. Aj, tit vedí, co jsem já mluvil.
\par 22 A když on to povedel, jeden z služebníku, stoje tu, dal policek Ježíšovi, rka: Tak-liž odpovídáš nejvyššímu knezi?
\par 23 Odpovedel mu Ježíš: Mluvil-li jsem zle, svedectví vydej o zlém; pakli dobre, proc mne tepeš?
\par 24 I poslal jej Annáš svázaného k Kaifášovi nejvyššímu knezi.
\par 25 Stál pak Šimon Petr, a zhríval se. Tedy rekli jemu: Nejsi-liž i ty z ucedlníku jeho? Zaprel on a rekl: Nejsem.
\par 26 Dí jemu jeden z služebníku nejvyššího kneze, príbuzný toho, kterémuž Petr utal ucho: Zdaž jsem já tebe nevidel s ním v zahrade?
\par 27 Tedy Petr opet zaprel. A hned kohout zazpíval.
\par 28 I vedli Ježíše od Kaifáše do radného domu, a bylo ráno. Oni pak nevešli do radného domu, aby se neposkvrnili, ale aby jedli beránka.
\par 29 Tedy vyšel k nim Pilát ven a rekl: Jakou žalobu vedete proti cloveku tomuto?
\par 30 Odpovedeli a rekli jemu: Byt tento nebyl zlocinec, nedali bychom ho tobe.
\par 31 I rekl jim Pilát: Vezmete vy jej, a podle Zákona vašeho sudte ho. I rekli mu Židé: Námt nesluší zabíti žádného.
\par 32 Aby se rec Ježíšova naplnila, kterouž rekl, znamenaje, kterou by smrtí mel umríti.
\par 33 Tedy Pilát všel opet do radného domu, i povolal Ježíše a rekl jemu: Ty-li jsi Král Židovský?
\par 34 Odpovedel Ježíš: Sám-li od sebe to pravíš, cili jiní tobe povedeli o mne?
\par 35 Odpovedel Pilát: Zdaliž jsem já Žid? Národ tvuj a prední kneží dali mi tebe. Co jsi ucinil?
\par 36 Odpovedel Ježíš: Království mé není z tohoto sveta. Byt z tohoto sveta bylo království mé, služebníci moji bránili by mne, abych nebyl vydán Židum. Ale nyní mé království není odsud.
\par 37 I rekl jemu Pilát: Tedy král jsi ty? Dí Ježíš: Ty pravíš, že já král jsem. Ját jsem se k tomu narodil, a proto jsem na svet prišel, abych svedectví vydal pravde. Každý, kdož jest z pravdy, slyší hlas muj.
\par 38 Dí jemu Pilát: Co jest pravda? A když to rekl, opet vyšel k Židum, a dí jim: Já na nem žádné viny nenalézám.
\par 39 Ale jest obycej váš, abych vám propustil jednoho vezne na velikunoc. Chcete-liž tedy, at vám propustím Krále Židovského?
\par 40 I zkrikli opet všickni, rkouce: Ne toho, ale Barabbáše. Byl pak Barabbáš lotr.

\chapter{19}

\par 1 Tedy vzal Pilát Ježíše, a zbicoval jej.
\par 2 A žoldnéri zpletše korunu z trní, vstavili na hlavu jeho, a pláštem šarlatovým priodeli jej.
\par 3 A ríkali: Zdráv bud, Králi Židovský. A dávali jemu policky.
\par 4 I vyšel opet ven Pilát, a rekl jim: Aj, vyvedu jej vám ven, abyste poznali, žet na nem žádné viny nenalézám.
\par 5 Tedy vyšel Ježíš ven, nesa trnovou korunu a plášt šarlatový. I rekl jim Pilát: Aj, clovek.
\par 6 A jakž jej uzreli prední kneží a služebníci jejich, zkrikli rkouce: Ukrižuj, ukrižuj ho. Dí jim Pilát: Vezmetež vy jej a ukrižujte, nebo já na nem viny nenalézám.
\par 7 Odpovedeli jemu Židé: My Zákon máme, a podle Zákona našeho mát umríti, nebo Synem Božím se cinil.
\par 8 A když Pilát uslyšel tuto rec, více se obával.
\par 9 I všel do radného domu zase, a rekl Ježíšovi: Odkud jsi ty? Ale Ježíš nedal jemu odpovedi.
\par 10 Tedy rekl jemu Pilát: Nemluvíš se mnou? Nevíš-liž, že mám moc ukrižovati te a moc mám propustiti tebe?
\par 11 Odpovedel Ježíš: Nemel bys nade mnou moci nižádné, byt nebylo dáno s hury; protož, kdot jest mne tobe vydal, vetšít hrích má.
\par 12 Od té chvíle hledal Pilát propustiti ho. Ale Židé volali rkouce: Propustíš-li tohoto, nejsi prítel císaruv; nebo každý, kdož se králem ciní, protiví se císari.
\par 13 Tedy Pilát uslyšev tu rec, vyvedl ven Ježíše, a sedl na soudné stolici na míste, kteréž slove Litostrotos, a Židovsky Gabbata.
\par 14 A byl den pripravování pred velikonocí, okolo šesté hodiny. I rekl Židum: Aj, král váš.
\par 15 Oni pak zkrikli: Vezmi, vezmi a ukrižuj jej. Rekl jim Pilát: Krále vašeho ukrižuji? Odpovedeli prední kneží: Nemámet krále, než císare.
\par 16 Tedy vydal jim ho, aby byl ukrižován. I pojali Ježíše a vedli jej ven.
\par 17 A on nesa kríž svuj, šel až na místo, kteréž slove popravné, a Židovsky Golgota.
\par 18 Kdežto ukrižovali ho, a s ním jiné dva s obou stran, a v prostredku Ježíše.
\par 19 Napsal pak i nápis Pilát, a vstavil na kríž. A bylo napsáno: Ježíš Nazaretský, Král Židovský.
\par 20 Ten pak nápis mnozí z Židu ctli; nebo blízko mesta bylo to místo, kdež ukrižován byl Ježíš. A bylo psáno Židovsky, Recky a Latine.
\par 21 Tedy prední kneží Židovští rekli Pilátovi: Nepiš: Král Židovský, ale že on rekl: Král Židovský jsem.
\par 22 Odpovedel Pilát: Co jsem psal, psal jsem.
\par 23 Žoldnéri pak, když Ježíše ukrižovali, vzali roucha jeho, a ucinili ctyri díly, každému rytíri díl jeden, vzali také i sukni, kterážto sukne byla nesšívaná, ale odvrchu všecka naskrze setkaná.
\par 24 I rekli mezi sebou: Neroztrhujme jí, ale losujme o ni, cí bude. Aby se naplnilo písmo, rkoucí: Rozdelili sobe roucho mé, a o muj odev metali los. A žoldnéri zajisté tak ucinili.
\par 25 Stály pak blízko kríže Ježíšova matka jeho a sestra matky jeho, Maria, manželka Kleofášova, a Maria Magdaléna.
\par 26 Tedy Ježíš uzrev matku a ucedlníka tu stojícího, kteréhož miloval, rekl k matce své: Ženo, aj, syn tvuj.
\par 27 Potom rekl ucedlníkovi: Aj, matka tvá. A od té hodiny prijal ji ucedlník ten k sobe.
\par 28 Potom veda Ježíš, že již všecko jiné dokonáno jest, aby se naplnilo písmo, rekl: Žízním.
\par 29 Byla pak tu postavena nádoba plná octa. Tedy oni naplnili houbu octem, a obloživše yzopem, podali k ústum jeho.
\par 30 A když okusil Ježíš octa, rekl: Dokonánot jest. A nakloniv hlavy, ducha Otci porucil.
\par 31 Židé pak, aby nezustala na kríži tela na sobotu, ponevadž byl den pripravování, (byl zajisté veliký ten den sobotní,) prosili Piláta, aby zlámáni byli hnátové jejich a aby byli složeni.
\par 32 I prišli žoldnéri, a prvnímu zajisté zlámali hnáty, i druhému, kterýž ukrižován byl s ním.
\par 33 Ale k Ježíšovi prišedše, jakž uzreli jej již mrtvého, nelámali hnátu jeho.
\par 34 Ale jeden z žoldnéru bok jeho kopím otevrel, a hned vyšla krev a voda.
\par 35 A ten, jenž videl, svedectví vydal, a pravé jest svedectví jeho; ont ví, že pravé veci praví, abyste i vy verili.
\par 36 Stalo se pak to, aby se naplnilo Písmo: Kost jeho žádná nebude zlámána.
\par 37 A opet jiné Písmo dí: Uzrít, v koho jsou bodli.
\par 38 Potom pak prosil Piláta Jozef z Arimatie, (kterýž byl ucedlník Ježíšuv, ale tajný, pro strach Židovský,) aby snal telo Ježíšovo. I dopustil Pilát. A on prišed, i snal telo Ježíšovo.
\par 39 Prišel pak i Nikodém, (kterýž byl prve prišel k Ježíšovi v noci,) nesa smíšení mirry a aloes okolo sta liber.
\par 40 Tedy vzali telo Ježíšovo, a obvinuli je prosteradly s vonnými vecmi, jakž obycej jest Židum se pochovávati.
\par 41 A byla na tom míste, kdež ukrižován byl, zahrada, a v zahrade hrob nový, v nemžto ješte žádný nebyl pochován.
\par 42 Protož tu pro den pripravování Židovský, že blízko byl hrob, položili Ježíše.

\chapter{20}

\par 1 První pak den po sobote Maria Magdaléna prišla ráno k hrobu, když ješte tma bylo. I uzrela kámen odvalený od hrobu.
\par 2 I bežela odtud a prišla k Šimonovi Petrovi a k jinému ucedlníku, jehož miloval Ježíš, a rekla jim: Vzali Pána z hrobu, a nevíme, kde jsou jej položili.
\par 3 Tedy vyšel Petr a jiný ucedlník, a šli k hrobu.
\par 4 I beželi oba spolu. Ale ten druhý ucedlník predbehl Petra, a prišel prve k hrobu.
\par 5 A nachýliv se, uzrel prosteradla položená, ale však tam nevšel.
\par 6 Tedy prišel Šimon Petr, za ním jda, a všel do hrobu. I uzrel prosteradla položená,
\par 7 A rouchu, kteráž byla na hlave jeho, ne s prosteradly položenou, ale obzvláštne svinutou na jednom míste.
\par 8 Potom všel i ten druhý ucedlník, kterýž byl prve prišel k hrobu, i uzrel a uveril.
\par 9 Nebo ješte neznali Písma, že mel Kristus z mrtvých vstáti.
\par 10 I odešli zase ti ucedlníci tam, kdež prve byli.
\par 11 Ale Maria stála u hrobu vne, placici. A když plakala, naklonila se do hrobu.
\par 12 A uzrela dva andely v bílém rouše sedící, jednoho u hlavy a druhého u noh, tu kdež bylo položeno telo Ježíšovo.
\par 13 Kterížto rekli jí: Ženo, co pláceš? I dí jim: Vzali Pána mého, a nevím, kde ho položili.
\par 14 To když rekla, obrátila se zpátkem, a uzrela Ježíše, an stojí, ale nevedela, by Ježíš byl.
\par 15 Dí jí Ježíš: Ženo, co pláceš? Koho hledáš? Ona domnívajici se, že by zahradník byl, rekla jemu: Pane, vzal-lis ty jej, povez mi, kdes ho položil, at já jej vezmu.
\par 16 Rekl jí Ježíš: Maria. Obrátivši se ona, rekla jemu: Rabbóni, jenž se vykládá: Mistre.
\par 17 Dí jí Ježíš: Nedotýkejž se mne; neb jsem ješte nevstoupil k Otci svému. Ale jdiž k bratrím mým, a povez jim: Vstupuji k Otci svému, a k Otci vašemu, k Bohu svému, a k Bohu vašemu.
\par 18 I prišla Maria Magdaléna, zvestujici ucedlníkum, že by videla Pána a že jí to povedel.
\par 19 Když pak byl vecer toho dne, kterýž jest první po sobote, a dvere byly zavríny, kdež byli ucedlníci shromáždeni, pro strach Židovský, prišel Ježíš, a stál u prostred, a rekl jim: Pokoj vám.
\par 20 A to povedev, ukázal jim ruce i bok svuj. I zradovali se ucedlníci, vidouce Pána.
\par 21 Tedy rekl jim opet: Pokoj vám. Jakož mne poslal Otec, tak i já posílám vás.
\par 22 To povedev, dechl, a rekl jim: Prijmete Ducha svatého.
\par 23 Kterýmžkoli odpustili byste hríchy, odpouštejít se jim; a kterýmžkoli zadrželi byste je, zadržánit jsou.
\par 24 Tomáš pak jeden ze dvanácti, jenž sloul Didymus, nebyl s nimi, když byl prišel Ježíš.
\par 25 I rekli jemu jiní ucedlníci: Videli jsme Pána. A on rekl jim: Lec uzrím v rukou jeho bodení hrebu, a vpustím prst svuj v místo hrebu, a ruku svou vložím v bok jeho, nikoli neuverím.
\par 26 A po osmi dnech opet ucedlníci jeho byli vnitr, a Tomáš s nimi. Prišel Ježíš, a dvere byly zavríny, i stál uprostred a rekl: Pokoj vám.
\par 27 Potom rekl k Tomášovi: Vložiž prst svuj sem, a viz ruce mé, a vztáhni ruku svou, a vpust v bok muj, a nebudiž neverící, ale verící.
\par 28 I odpovedel Tomáš a rekl jemu: Pán muj a Buh muj.
\par 29 Dí jemu Ježíš: Žes mne videl, Tomáši, uveril jsi. Blahoslavení, kteríž nevideli, a uverili.
\par 30 Mnohé zajisté i jiné divy cinil Ježíš pred oblicejem ucedlníku svých, kteréž nejsou psány v knize této.
\par 31 Ale toto psáno jest, abyste verili, že Ježíš jest Kristus, Syn Boží, a abyste veríce, život vecný meli ve jménu jeho.

\chapter{21}

\par 1 Potom opet zjevil se Ježíš ucedlníkum u more Tiberiadského. A zjevil se takto:
\par 2 Byli spolu Šimon Petr a Tomáš, jenž sloul Didymus, a Natanael, jenž byl z Kány Galilejské, a synové Zebedeovi, a jiní z ucedlníku jeho dva.
\par 3 Dí jim Šimon Petr: Pujdu ryb loviti. Rekli jemu: Pujdeme i my s tebou. I šli, a vstoupili na lodí hned; a té noci nic nepopadli.
\par 4 A když bylo již ráno, stál Ježíš na brehu. Nevedeli však ucedlníci, by Ježíš byl.
\par 5 Tedy dí jim Ježíš: Dítky, máte-li jakou krmicku? Odpovedeli jemu: Nemáme.
\par 6 On pak rekl jim: Zavrztež sít na pravou stranu lodí, a naleznete. I zavrhli sít a hned nemohli jí táhnouti pro množství ryb.
\par 7 I rekl ucedlník ten, kteréhož miloval Ježíš, Petrovi: Pán jest. A Šimon Petr, jakž uslyšel, že Pán jest, opásal se po košili, (nebo byl nah,) a pustil se do more.
\par 8 Jiní také ucedlníci na lodí plavili se, (nebo nedaleko byli od brehu, asi okolo dvou set loket,) táhnouce sít plnou ryb.
\par 9 A jakž vystoupili na breh, uzreli reravé uhlí a rybu svrchu položenou a chléb.
\par 10 Rekl jim Ježíš: Prineste z ryb, kterýchž jste nalapali nyní.
\par 11 Vstoupil pak Šimon Petr a vytáhl sít na zem, plnou ryb velikých, jichž bylo sto padesáte a tri. A ackoli jich tak mnoho bylo, však neztrhala se sít.
\par 12 Rekl jim Ježíš: Pojdte, obedujte. Žádný pak z ucedlníku neodvážil se ho otázati: Ty kdo jsi? vedouce, že Pán jest.
\par 13 I prišel Ježíš, a vzal chléb, a dával jim, i rybu též.
\par 14 To již po tretí ukázal se Ježíš ucedlníkum svým, vstav z mrtvých.
\par 15 A když poobedvali, rekl Ježíš Šimonovi Petrovi: Šimone, synu Jonášuv, miluješ-li mne více nežli tito? Rekl jemu: Ovšem, Pane, ty víš, že te miluji. Dí jemu: Pasiž beránky mé.
\par 16 Rekl jemu opet po druhé: Šimone Jonášuv, miluješ-li mne? Rekl jemu: Ovšem, Pane, ty víš, že te miluji. Dí jemu: Pasiž ovce mé.
\par 17 Rekl jemu po tretí: Šimone Jonášuv, miluješ-li mne? I zarmoutil se Petr proto, že jemu rekl po tretí: Miluješ-li mne? A odpovedel jemu: Pane, ty znáš všecko, ty víš, že te miluji. Rekl jemu Ježíš: Pasiž ovce mé.
\par 18 Amen, amen pravím tobe: Když jsi byl mladší, opasovals se a chodíval jsi, kams chtel; ale když se zstaráš, ztáhneš ruce své, a jiný te opáše, a povede, kamž ty nechceš.
\par 19 To pak povedel, znamenaje, kterou by smrtí mel oslaviti Boha. A to povedev, rekl jemu: Pojd za mnou.
\par 20 I obrátiv se Petr, uzrel toho ucedlníka, kteréhož miloval Ježíš, an jde za ním, kterýž i odpocíval za vecerí na prsech jeho, a byl rekl: Pane, kdo jest ten, kterýž te zradí?
\par 21 Toho videv Petr, dí k Ježíšovi: Pane, co pak tento?
\par 22 Rekl jemu Ježíš: Chci-li ho nechati, dokudž neprijdu, co tobe po tom? Ty pojd za mnou.
\par 23 I vyšla rec ta mezi bratrí, že by ucedlník ten nemel umríti. A nerekl byl jemu Ježíš, že by nemel umríti, ale rekl: Chci-li ho nechati, dokudž neprijdu, co tobe po tom?
\par 24 Tot jest ucedlník ten, kterýž svedectví vydává o techto vecech, a napsal toto, a víme, že pravé jest svedectví jeho.
\par 25 Jestit pak i jiných mnoho vecí, kteréž cinil Ježíš, kteréž kdyby mely všecky, každá obzvláštne, psány býti, mám za to, že by ten svet nemohl prijíti tech knih, kteréž by napsány byly. Amen.


\end{document}