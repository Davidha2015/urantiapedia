\begin{document}

\title{Acts}

\chapter{1}

\par 1 První zajisté knihu sepsalt jsem, ó Teofile, o všech vecech, kteréž zacal Ježíš ciniti a uciti,
\par 2 Až do toho dne, v kterémžto dav prikázání apoštolum, kteréž byl skrze Ducha svatého vyvolil, vzhuru vzat jest.
\par 3 Kterýmžto i zjevoval sebe samého živého po svém umucení ve mnohých jistých duvodích, za ctyridceti dnu ukazuje se jim a mluve o království Božím.
\par 4 A shromáždiv se s nimi, prikázal jim, aby z Jeruzaléma neodcházeli, ale aby ocekávali zaslíbení Otcova, o kterémž jste prý slyšeli ode mne.
\par 5 Nebo Jan zajisté krtil vodou, ale vy pokrteni budete Duchem svatým po nemnohých techto dnech.
\par 6 Oni pak sšedše se, otázali se ho, rkouce: Pane, v tomto-li casu napravíš království Izraelské?
\par 7 I rekl jim: Nenít vaše vec znáti casy anebo príhodnosti casu, kteréžto Otec v moci své položil.
\par 8 Ale prijmete moc Ducha svatého, pricházejícího na vás, a budete mi svedkové, i v Jeruzaléme, i ve všem Judstvu, i v Samarí, a až do posledních koncin zeme.
\par 9 A to povedev, ani na to hledí, vzhuru vyzdvižen jest, a oblak vzal jej od ocí jejich.
\par 10 A když za ním v nebe jdoucím pilne hledeli, aj, dva muži postavili se podle nich v rouše bílém,
\par 11 A rekli: Muži Galilejští, co stojíte, hledíce do nebe? Tento Ježíš, kterýž vzhuru vzat jest od vás do nebe, takt prijde, jakž jste spatrili zpusob jeho jdoucího do nebe.
\par 12 Tedy navrátili se do Jeruzaléma od hory, jenž slove Olivetská, kteráž jest blízko Jeruzaléma, vzdálí cesty jednoho dne svátecního.
\par 13 A když prišli domu, vstoupili do vrchního príbytku domu, kdež prebývali, i Petr i Jakub, i Jan a Ondrej, Filip a Tomáš, Bartolomej a Matouš, Jakub Alfeuv a Šimon Zelótes a Judas bratr Jakubuv.
\par 14 Ti všickni byli trvajíce jednomyslne na modlitbe a pokorné prosbe s ženami a s Marijí, matkou Ježíšovou, i s bratrími jeho.
\par 15 V tech pak dnech povstav Petr uprostred ucedlníku, rekl (a byl zástup lidí spolu shromáždených okolo sta a dvadcíti):
\par 16 Muži bratrí, musilo se naplniti Písmo to, kteréž predpovedel Duch svatý skrze ústa Davidova o Jidášovi, kterýž byl vudce tech, jenž jímali Ježíše.
\par 17 Nebo byl pricten k nám, a byl došel losu prisluhování tohoto.
\par 18 Ten zajisté obdržel pole ze mzdy nepravosti, a obesiv se, rozpukl se na dvé, i vykydla se všecka streva jeho.
\par 19 A to známé jest ucineno všechnem prebývajícím v Jeruzaléme, takže jest nazváno pole to vlastním jazykem jejich Akeldama, to jest pole krve.
\par 20 Psáno jest zajisté v knihách Žalmu: Budiž príbytek jeho pustý, a nebud, kdo by prebýval v nem, a opet: Biskupství jeho vezmi jiný.
\par 21 Protož musít to býti, aby jeden z tech mužu, kteríž jsou s námi bývali po všecken cas, v nemž prebýval mezi námi Pán Ježíš,
\par 22 Pocav od krtu Janova až do dne toho, v kterémžto vzhuru vzat jest od nás, byl svedkem spolu s námi vzkríšení jeho.
\par 23 Tedy postavili dva, Jozefa, jenž sloul Barsabáš, kterýž mel príjmí Justus, a Mateje.
\par 24 A modléce se, rekli: Ty, Pane, všech srdcí zpytateli, ukažiž, kterého jsi vyvolil z techto dvou,
\par 25 Aby prijal los prisluhování tohoto a apoštolství, z nehož jest vypadl Jidáš, aby odšel na místo své.
\par 26 I dali jim losy. Spadl pak los na Mateje, i pripojen jest z spolecného snešení k jedenácti apoštolum.

\chapter{2}

\par 1 A když prišel den padesátý, byli všickni spolu na jednom míste.
\par 2 I stal se rychle zvuk s nebe, jako pricházejícího vetru prudkého, a naplnil všecken dum, kdež sedeli.
\par 3 I ukázali se jim rozdelení jazykové jako ohen, kterýžto posadil se na každém z nich.
\par 4 I naplneni jsou všickni Duchem svatým, a pocali mluviti jinými jazyky, jakž ten Duch dával jim vymlouvati.
\par 5 Byli pak v Jeruzaléme prebývající Židé, muži nábožní, ze všelikého národu, kterýž pod nebem jest.
\par 6 A když se stal ten hlas, sešlo se množství a užasli se toho, že je slyšel jeden každý, ani mluví prirozeným jazykem jeho.
\par 7 I desili se všickni a divili se, rkouce jedni k druhým: Aj, zdaliž nejsou tito všickni, kteríž mluví, Galilejští?
\par 8 A kterak my je slyšíme jeden každý z nás mluviti jazykem naším, v kterémž jsme se zrodili?
\par 9 Partští, a Medští, a Elamitští, a kteríž prebýváme v Mezopotamii, v Židovstvu a v Kappadocii, v Pontu a v Azii,
\par 10 V Frygii a v Pamfylii, v Egypte a v krajinách Libye, kteráž jest vedle Cyrénu, a hosté Rímané, Židé, i vnove na víru obrácení,
\par 11 Kretští i Arabští, slyšíme je, ani mluví jazyky našimi veliké veci Boží.
\par 12 I desili se všickni a divili se, jeden k druhému rkouce: I což toto bude?
\par 13 Jiní pak posmívajíce se, pravili: Mstem se zpili tito.
\par 14 A stoje Petr s jedenácti, pozdvihl hlasu svého a promluvil k nim: Muži Židé a všickni, kteríž bydlíte v Jeruzaléme, toto vám známo bud, a ušima pozorujte slov mých.
\par 15 Jiste nejsout tito, jakož vy se domníváte, zpilí, ponevadž jest teprv tretí hodina na den.
\par 16 Ale totot jest, což jest predpovedíno skrze proroka Joele:
\par 17 A budet v posledních dnech, (dí Buh,) vyleji z Ducha mého na všeliké telo, a prorokovati budou synové vaši, i dcery vaše, a mládenci vaši videní vídati budou, a starci vaši sny míti budou.
\par 18 A zajisté na služebníky své a na služebnice své v tech dnech vyleji z Ducha mého, a budou prorokovati.
\par 19 A ukáži zázraky na nebi svrchu a znamení na zemi dole, krev a ohen a páru dýmovou.
\par 20 Slunce obrátí se v temnost a mesíc v krev, prve než prijde den Páne veliký a zjevný.
\par 21 A stanet se, že každý, kdožkoli vzýval by jméno Páne, spasen bude.
\par 22 Muži Izraelští, slyšte slova tato: Ježíše toho Nazaretského, muže od Boha zveliceného mezi vámi mocmi a zázraky a znameními, kteréž cinil skrze neho Buh uprostred vás, jakož i vy sami víte,
\par 23 Toho, pravím, vydaného, z uložené rady a predzvedení Božího vzavše a skrze ruce nešlechetných ukrižovavše, zamordovali jste.
\par 24 Jehožto Buh vzkrísil, zprostiv ho bolestí smrti, jakož nebylo možné jemu držánu býti od ní.
\par 25 Nebo David praví o nem: Spatroval jsem Pána pred sebou vždycky; nebo jest mi po pravici, abych se nepohnul.
\par 26 Protož rozveselilo se srdce mé, a zplésal jazyk muj, nýbrž i telo mé odpocine v nadeji.
\par 27 Nebo nenecháš duše mé v pekle, aniž dáš videti svatému svému porušení.
\par 28 Známé jsi mi ucinil cesty života, a naplníš mne utešením pred oblicejem svým.
\par 29 Muži bratrí, sluší smele mluviti k vám o patriarchovi Davidovi, že i umrel, i pochován jest, i hrob jeho jest u nás až do dnešního dne.
\par 30 Prorok tedy byv a vedev, že prísahou zavázal se jemu Buh, že z plodu ledví jeho podle tela vzbudí Krista a posadí na stolici jeho,
\par 31 To predzvedev, mluvil o vzkríšení Kristovu, že není opuštena duše jeho v pekle, ani telo jeho videlo porušení.
\par 32 Toho Ježíše vzkrísil Buh, jehožto my všickni svedkové jsme.
\par 33 Protož pravicí Boží jsa zvýšen, a vzav zaslíbení Ducha svatého od Otce, vylil to, což vy nyní vidíte a slyšíte.
\par 34 Nebot jest David nevstoupil v nebe, ale on praví: Rekl Pán Pánu mému: Sed na pravici mé,
\par 35 Dokavadž nepoložím neprátel tvých, aby byli podnože noh tvých.
\par 36 Protož veziž jiste všecken dum Izraelský, žet jest Buh i Pánem ho ucinil i Kristem, toho Ježíše, kteréhož jste vy ukrižovali.
\par 37 To slyševše, zkormouceni jsou v srdci svém, a rekli ku Petrovi a k jiným apoštolum: Což máme ciniti, muži bratrí?
\par 38 Tedy Petr rekl k nim: Pokání cinte, a pokrti se jeden každý z vás ve jménu Ježíše Krista na odpuštení hríchu a prijmete dar Ducha svatého.
\par 39 Vámt jest zajisté zaslíbení stalo se a synum vašim, i všechnem, kteríž daleko jsou, kterýchžkoli povolal by Pán Buh náš.
\par 40 A jinými slovy mnohými osvedcoval a napomínal jich, rka: Oddelte se od pokolení toho zlého.
\par 41 Tedy ti, kteríž ochotne prijali slova jeho, pokrteni jsou, a pripojilo se k nim ten den duší okolo trí tisícu.
\par 42 I zustávali v ucení apoštolském, a v spolecnosti, a v lámání chleba, a na modlitbách.
\par 43 I prišla na všelikou duši bázen, a mnozí divové a zázrakové dáli se skrze apoštoly.
\par 44 Všickni pak verící byli pospolu, a meli všecky veci obecné.
\par 45 A prodávali vladarství a statky, a delili mezi všecky, jakž komu potrebí bylo.
\par 46 A na každý den trvajíce jednomyslne v chráme, a lámajíce po domích chléb, prijímali pokrm s potešením a sprostností srdce,
\par 47 Chválíce Boha a milost majíce u všeho lidu. Pán pak pridával církvi na každý den tech, kteríž by spaseni byli.

\chapter{3}

\par 1 Petr pak a Jan spolu vstupovali do chrámu v hodinu modlitebnou devátou.
\par 2 A muž nejaký, chromý tak narozený z života matky své, nesen byl, kteréhož sázeli na každý den u dverí chrámových, kteréž slouly Krásné, aby prosil za almužnu tech, kteríž vcházeli do chrámu.
\par 3 Ten uzrev Petra a Jana, ani vcházeti meli do chrámu, prosil jich, aby mu almužnu dali.
\par 4 I pohledev nan Petr s Janem, rekl: Hled na nás.
\par 5 A on pilne pohledel na ne, nadeje se, že neco vezme od nich.
\par 6 Tedy rekl Petr: Stríbra a zlata nemám, ale což mám, to tobe dám. Ve jménu Ježíše Krista Nazaretského vstan a chod.
\par 7 I ujav jej za ruku jeho pravou, pozdvihl ho, a hned utvrzeny jsou nohy jeho i kloubové.
\par 8 A zchytiv se, stál, a chodil, a všel s nimi do chrámu, chode, a poskakuje, a chvále Boha.
\par 9 A videl jej všecken lid, an chodí a chválí Boha.
\par 10 I poznali ho, že jest ten, kterýž na almužne sedával u dverí Krásných chrámových. I naplneni jsou strachem a dešením nad tím, což se stalo jemu.
\par 11 A když se ten uzdravený prídržel Petra a Jana, sbehl se k nim všecken lid do sínce, kteráž sloula Šalomounova, predešen jsa.
\par 12 To videv Petr, promluvil k lidu: Muži Izraelští, co se divíte tomuto? Anebo co na nás tak pilne hledíte, jako bychom my svou mocí aneb nábožností ucinili to, aby tento chodil?
\par 13 Buh Abrahamuv a Izákuv a Jákobuv, Buh otcu našich, oslavil Syna svého Ježíše, kteréhož jste vy vydali a odepreli se pred tvárí Pilátovou, kterýž ho soudil býti hodného propuštení.
\par 14 Vy pak svatého a spravedlivého odepreli jste se a prosili jste za muže vražedníka, aby vám byl dán.
\par 15 Ale dárce života zamordovali jste, kteréhož Buh vzkrísil z mrtvých; cehož my svedkové jsme.
\par 16 A skrze víru ve jméno jeho, tohoto, kteréhož vy vidíte a znáte, utvrdilo jest jméno jeho a víra, kteráž jest skrze neho, dala jemu celé zdraví toto pred oblicejem všech vás.
\par 17 Ale nyní, bratrí, vím, že jste to z nevedomí ucinili, jako i knížata vaše.
\par 18 Buh pak to, což predzvestoval skrze ústa všech proroku, že mel Kristus trpeti, tak jest naplnil.
\par 19 Protož cinte pokání, a obratte se, aby byli shlazeni hríchové vaši, když by prišli casové rozvlažení od tvári Páne,
\par 20 A poslal by toho, kterýž vám kázán jest, Ježíše Krista.
\par 21 Kteréhož zajisté musí prijíti nebesa, až do casu napravení všech vecí; což byl predpovedel Buh skrze ústa svých svatých proroku od veku.
\par 22 Mojžíš zajisté otcum rekl, že Proroka vám vzbudí Pán Buh váš z bratrí vašich jako mne, jehož poslouchati budete ve všem, cožkoli bude mluviti vám.
\par 23 A stanet se, že každá duše, kteráž by neposlouchala toho Proroka, vyhlazena bude z lidu mého.
\par 24 Ano i všickni proroci od Samuele a potomních, kterížkoli mluvili, také jsou o techto dnech predzvestovali.
\par 25 Vy jste synové proroku a synové úmluvy, kterouž ucinil Buh s otci našimi, rka k Abrahamovi: V semeni tvém požehnány budou všecky celedi zeme.
\par 26 Vám nejprve Buh, vzbudiv Syna svého Ježíše, poslal ho dobrorecícího vám, aby se jeden každý z vás odvrátil od nepravostí svých.

\chapter{4}

\par 1 A když oni mluvili k lidu, prišli kneží a úredník chrámu a saduceové,
\par 2 Težce to nesouce, že lid ucili a zvestovali ve jménu Ježíše vzkríšení z mrtvých.
\par 3 I vztáhli na ne ruce a vsadili je do žaláre až do jitra, neb již byl vecer.
\par 4 Mnozí pak z tech, kteríž slyšeli slovo Boží, uverili. I ucinen jest pocet mužu okolo peti tisícu.
\par 5 Stalo se pak nazejtrí, sešli se knížata jejich, a starší, a zákoníci v Jeruzaléme,
\par 6 A Annáš nejvyšší knez, a Kaifáš, a Jan, a Alexander, a kterížkoli byli z pokolení nejvyššího kneze.
\par 7 I postavivše je mezi sebou, otázali se jich: Jakou mocí aneb v kterém jménu ucinili jste to vy?
\par 8 Tedy Petr, jsa pln Ducha svatého, rekl jim: Knížata lidu a starší Izraelští,
\par 9 Ponevadž my dnes k soudu jsme privedeni pro dobrodiní cloveku nemocnému ucinené, kterak by on zdráv ucinen byl:
\par 10 Známo bud všechnem vám i všemu lidu Izraelskému, že ve jménu Ježíše Krista Nazaretského, kteréhož jste vy ukrižovali, jehož Buh vzkrísil z mrtvých, skrze toho jméno tento stojí pred vámi zdravý.
\par 11 Tot jest ten kámen za nic položený od vás delníku, kterýž ucinen jest v hlavu úhelní.
\par 12 A nenít v žádném jiném spasení; nebot není jiného jména pod nebem daného lidem, skrze kteréž bychom mohli spaseni býti.
\par 13 I vidouce takovou udatnost a smelost v mluvení Petrovu a Janovu, a shledavše, že jsou lidé neucení a prostí, divili se, a poznali je, že s Ježíšem bývali.
\par 14 Cloveka také toho vidouce, an stojí s nimi, kterýž byl uzdraven, nemeli co mluviti proti nim.
\par 15 I rozkázavše jim vystoupiti z rady, rozmlouvali vespolek,
\par 16 Rkouce: Co uciníme lidem temto? Nebo že jest zjevný zázrak stal se skrze ne, všem prebývajícím v Jeruzaléme známé jest, aniž mužeme toho zapríti.
\par 17 Ale aby se to více nerozhlašovalo v lidu, s pohružkou prikažme jim, aby více v tom jménu žádnému z lidí nemluvili.
\par 18 I povolavše jich, prikázali jim, aby nikoli nemluvili, ani ucili ve jménu Ježíšovu.
\par 19 Tedy Petr a Jan odpovídajíce, rekli jim: Jest-li to spravedlivé pred oblicejem Božím, abychom vás více poslouchali než Boha, sudte.
\par 20 Nebt nemužeme nemluviti toho, co jsme videli a slyšeli.
\par 21 A oni pohrozivše jim, propustili je, nenalezše na nich príciny trestání, pro lid; nebo všickni velebili Boha z toho, co se bylo stalo.
\par 22 Byl zajisté v letech více než ve ctyridcíti clovek ten, pri kterémž se byl stal zázrak ten uzdravení.
\par 23 A jsouce propušteni, prišli k svým a povedeli jim, co k nim prední kneží a starší mluvili.
\par 24 Kteríž uslyševše to, jednomyslne pozdvihli hlasu k Bohu a rekli: Hospodine, ty jsi Buh, kterýž jsi ucinil nebe i zemi, i more i všecko, což v nich jest,
\par 25 Kterýž jsi skrze ústa Davida, služebníka svého, rekl: Proc jsou se bourili národové a lidé myslili marné veci?
\par 26 Postavili se králové zemští, a knížata sešla se vespolek proti Pánu a proti Pomazanému jeho.
\par 27 Právet jsou se jiste sešli proti svatému Synu tvému Ježíšovi, kteréhož jsi pomazal, Herodes a Pontský Pilát, s pohany a lidem Izraelským,
\par 28 Aby ucinili to, což ruka tvá a rada tvá preduložila, aby se stalo.
\par 29 A nyní, Pane, pohlediž na pohružky jejich a dejž služebníkum svým mluviti slovo tvé svobodne a smele,
\par 30 Vztahuje ruku svou k uzdravování a k cinení divu a zázraku, skrze jméno svatého Syna tvého Ježíše.
\par 31 A když se oni modlili, zatráslo se to místo, na kterémž byli shromáždeni, a naplneni jsou všickni Duchem svatým, a mluvili slovo Boží smele a svobodne.
\par 32 Toho pak množství verících bylo jedno srdce a jedna duše. Aniž kdo co z tech vecí, kteréž mel, svým vlastním býti pravil, ale meli všecky veci obecné.
\par 33 A mocí velikou vydávali apoštolé svedectví o vzkríšení Pána Ježíše, a milost veliká prítomná byla všechnem jim.
\par 34 A žádný mezi nimi nebyl nuzný; nebo kterížkoli meli pole nebo domy, prodávajíce, prinášeli peníze, za kteréž prodávali,
\par 35 A kladli pred nohy apoštolské. I rozdelováno bylo jednomu každému, jakž komu potrebí bylo.
\par 36 Jozes pak, kterýž príjmí mel od apoštolu Barnabáš, (což se vykládá syn utešení,) z pokolení Levítského, z Cypru rodem,
\par 37 Mev pole, prodal je, a prinesl peníze, a položil k nohám apoštolským.

\chapter{5}

\par 1 Muž pak jeden, jménem Ananiáš, s Zafirou, manželkou svou, prodal statek.
\par 2 A lstive neco tech penez ujal s vedomím manželky své, a prinesa díl nejaký, položil k nohám apoštolským.
\par 3 I rekl Petr: Ananiáši, proc naplnil satan srdce tvé lstí, tak abys lhal Duchu svatému a lstive ujal cástku penez za to pole?
\par 4 Zdaliž nebylo tvé, kdybys ho byl sobe nechal? A když bylo prodáno, v moci tvé bylo. I proc jsi tuto vec složil v srdci svém? Neselhal jsi lidem, ale Bohu.
\par 5 Tedy uslyšav Ananiáš tato slova, padna, zdechl. I spadla bázen veliká na všecky, kteríž to slyšeli.
\par 6 A vstavše mládenci, vzali jej, a vynesše ven, pochovali.
\par 7 I stalo se po chvíli, jako po trech hodinách, že i jeho žena, neveduci, co se bylo stalo, prišla.
\par 8 I rekl k ní Petr: Povez mi, za toliko-li jste pole své prodali? A ona rekla: Ano, za tolik.
\par 9 Tedy dí jí Petr: I procež jste se smluvili, abyste pokoušeli Ducha Páne? Aj, nohy tech, kteríž pochovali muže tvého, prede dvermi jsou, a vynesout také i tebe.
\par 10 I padla hned pred nohy jeho, a zdechla. A všedše mládenci, nalezli ji mrtvou; i vynesše, pochovali podle muže jejího.
\par 11 I byla bázen veliká po vší církvi, i mezi všemi, kteríž to slyšeli.
\par 12 Skrze ruce pak apoštolu dáli se divové a zázrakové velicí v lidu, (A bývali všickni jednomyslne v sínci Šalomounove.
\par 13 Jiný pak žádný neodvážil se pripojiti k nim, ale velebil je lid.
\par 14 A vždy více se rozmáhalo množství verících Pánu, mužu i také žen.)
\par 15 Takže i na ulice vynášeli nemocné, a kladli na ložcích a na nosidlách, aby, když by šel Petr, aspon stín jeho zastínil na nekteré z nich.
\par 16 Scházelo se pak množství z okolních mest do Jeruzaléma, nesouce nemocné a trápené od duchu necistých, a uzdravováni byli všickni.
\par 17 Tedy povstav nejvyšší knez a všickni, kteríž byli s ním, (jenž byli saducejské sekty,) naplneni jsou závistí.
\par 18 I zjímali apoštoly, a vsázeli je do žaláre obecného.
\par 19 Ale andel Páne v noci otevrev dvere u žaláre, vyvedl je ven a rekl:
\par 20 Jdete, a postavíce se, mluvte lidu v chráme všecka slova života tohoto.
\par 21 To oni uslyševše, vešli na úsvite do chrámu a ucili. Tedy prišed nejvyšší knez a ti, kteríž s ním byli, svolali radu a všecky starší synu Izraelských, i poslali do žaláre, aby byli privedeni.
\par 22 A služebníci prišedše, nenalezli jich v žalári. A navrátivše se, vypravovali,
\par 23 Rkouce: Žalár zajisté nalezli jsme zavrený se vší pilností a strážné vne stojící u dverí, ale otevrevše dvere, žádného jsme tam nenalezli.
\par 24 A když uslyšeli reci tyto i nejvyšší knez i úredník chrámu i jiní prední kneží, nerozumeli, co by se to stalo.
\par 25 A prišed kdosi, povedel jim, rka: Aj, muži, kteréž jste vsázeli do žaláre, v chráme stojí a ucí lid.
\par 26 Tedy šel tam úredník s služebníky, a privedl je bez násilé; nebo se báli lidu, aby nebyli ukamenováni.
\par 27 A privedše je, postavili je v rade. I otázal se jich nejvyšší knez,
\par 28 Rka: Zdaliž jsme vám prísne neprikázali, abyste neucili v tom jménu? A aj, naplnili jste Jeruzalém ucením svým, a chcete na nás uvésti krev cloveka toho.
\par 29 Odpovedev pak Petr a apoštolé, rekli: Více sluší poslouchati Boha než lidí.
\par 30 Buh otcu našich vzkrísil Ježíše, kteréhož jste vy zamordovali, povesivše na dreve.
\par 31 Toho jest Buh, jakožto Knížete a Spasitele, povýšil pravicí svou, aby bylo dáno lidu Izraelskému pokání a odpuštení hríchu.
\par 32 A my jsme svedkové toho všeho, což mluvíme, ano i Duch svatý, kteréhož dal Buh tem, jenž jsou poslušni jeho.
\par 33 Oni pak slyševše to, rozzlobili se, a radili se o to, kterak by je vyhladili.
\par 34 Tedy povstav v rade jeden farizeus, jménem Gamaliel, Zákona ucitel, vzácný muž u všeho lidu, rozkázal, aby na malou chvíli ven vyvedli apoštoly.
\par 35 I rekl jim: Muži Izraelští, pilne se rozmyslte pri techto lidech, co máte ciniti.
\par 36 Nebo pred temito casy byl povstal Teudas, prave se také býti necím velikým, jehož se prídrželo mužu okolo ctyr set; kterýžto již zahynul, i všickni, kteríž pristoupili k nemu, rozptýleni jsou a v nic obráceni.
\par 37 Po nem pak povstal Judas Galilejský za dnu popisu, a mnoho lidu po sobe obrátil. Ale i ten zahynul, a všickni, kterížkoli pristoupili k nemu, rozptýleni jsou.
\par 38 A protož nyní pravím vám: Dejte pokoj temto lidem, a nechte jich. Nebo jestližet jest z lidí rada tato anebo dílo toto, rozprchnet se;
\par 39 Paklit jest z Boha, nebudete moci toho zkaziti; abyste snad i Bohu odporní nalezeni nebyli.
\par 40 I povolili jemu. A povolavše apoštolu, a zmrskavše je, prikázali, aby více nemluvili ve jménu Ježíšovu. I propustili je.
\par 41 Oni pak šli z toho jejich shromáždení, radujíce se, že jsou hodni ucineni trpeti protivenství pro jméno Pána Ježíše.
\par 42 Na každý pak den neprestávali v chráme i po domích uciti a zvestovati Ježíše Krista.

\chapter{6}

\par 1 A v tech dnech, když se rozmnožovali ucedlníci, stalo se reptání Reku proti Židum proto, že by zanedbávány byly v prisluhování vezdejším vdovy jejich.
\par 2 Tedy dvanácte apoštolu, svolavše množství ucedlníku, rekli: Není slušné, abychom my, opustíce slovo Boží, prisluhovali stolum.
\par 3 Protož, bratrí, vyberte z sebe mužu sedm dobropovestných, plných Ducha svatého a moudrosti, jimž bychom porucili tu práci.
\par 4 My pak modlitby a služby slova Páne pilni budeme.
\par 5 I líbila se ta rec všemu množství. I vyvolili Štepána, muže plného víry a Ducha svatého, a Filipa, a Prochora, a Nikánora, a Timona, a Parména, a Mikuláše Antiochenského, k víre vnove obráceného.
\par 6 Ty postavili pred oblicejem apoštolu, kterížto pomodlivše se, vzkládali na ne ruce.
\par 7 I rostlo jest slovo Boží, a rozmáhal se pocet ucedlníku v Jeruzaléme velmi. Mnohý také zástup kneží poslouchal víry.
\par 8 Štepán pak, jsa plný víry a moci, cinil divy a zázraky veliké v lidu.
\par 9 I povstali nekterí z školy, kteráž sloula Libertinských, a Cyrenenských, a Alexandrinských, a tech, kteríž byli z Cilicie a Azie, hádajíce se s Štepánem.
\par 10 A nemohli odolati moudrosti a Duchu Páne, kterýž mluvil.
\par 11 Tedy lstive nastrojili muže, kteríž rekli: My jsme jej slyšeli mluviti slova rouhavá proti Mojžíšovi a proti Bohu.
\par 12 A tak zbourili lid a starší i zákoníky, a oborivše se na nej, chytili jej, a vedli do rady.
\par 13 I vystavili falešné svedky, kteríž rekli: Clovek tento neprestává mluviti slov rouhavých proti místu tomuto svatému i proti Zákonu.
\par 14 Nebo jsme slyšeli jej, an praví: Že ten Ježíš Nazaretský zkazí místo toto, a promení ustanovení, kteráž nám vydal Mojžíš.
\par 15 A pilne patríce na nej všickni, kteríž sedeli v rade, videli tvár jeho, jako tvár andela.

\chapter{7}

\par 1 Tedy rekl nejvyšší knez: Jest-liž to tak?
\par 2 A on rekl: Muži bratrí a otcové, slyšte. Buh slávy ukázal se otci našemu Abrahamovi, když byl v Mezopotamii, prve než bydlil v Cháran.
\par 3 A rekl k nemu: Vyjdi z zeme své a z príbuznosti své, a pojd do zeme, kterouž ukáži tobe.
\par 4 Tedy vyšel z zeme Kaldejské a bydlil v Cháran. A odtud, když umrel otec jeho, prestehoval jej do zeme této, v kteréžto vy nyní bydlíte.
\par 5 A nedal jemu dedictví v ní, ani šlépeje nožné, ac byl jemu ji slíbil dáti k vladarství, i semeni jeho po nem, když ješte nemel dedice.
\par 6 Mluvil pak jemu Buh takto: Budet síme tvé pohostinu v zemi cizí, a bude v službu podrobeno, a zle s ním budou nakládati za ctyri sta let.
\par 7 Ale národ ten, jemuž sloužiti budou, já souditi budu, pravít Buh. A potom zase vyjdou, a sloužiti mi budou na tomto míste.
\par 8 I vydal jemu smlouvu obrízky. A tak on zplodil Izáka, a obrezal jej osmého dne, a Izák zplodil Jákoba, a Jákob
\par 9 A patriarchové v nenávisti mevše Jozefa, prodali jej do Egypta. Ale Buh byl s ním,
\par 10 A vysvobodil ho ze všech úzkostí jeho, a dal jemu milost a moudrost pred tvárí faraona, krále Egyptského, takže ho ucinil úredníkem nad Egyptem a nade vším domem svým.
\par 11 Potom prišel hlad na všecku zemi Egyptskou i Kananejskou, a soužení veliké, aniž meli pokrmu otcové naši.
\par 12 A uslyšev Jákob, že by obilé bylo v Egypte, poslal tam otce naše nejprve.
\par 13 A když je poslal po druhé, poznán jest Jozef od bratrí svých, a zjevena jest rodina Jozefova faraonovi.
\par 14 Tedy poslav Jozef posly, pristehoval otce svého Jákoba, i všecku rodinu svou v osobách sedmdesáti a peti.
\par 15 I vstoupil Jákob do Egypta, a tam umrel on i otcové naši.
\par 16 I preneseni jsou do Sichem, a pochováni v hrobe, kterýž byl koupil Abraham za stríbro od synu Emorových, otce Sichemova.
\par 17 Když se pak približoval cas zaslíbení, o kterémž byl prisáhl Buh Abrahamovi, rostl lid a množil se v Egypte,
\par 18 Až vtom povstal jiný král, kterýž neznal Jozefa.
\par 19 Ten lstive nakládaje s pokolením naším, trápil otce naše, takže musili vyhazovati nemluvnátka svá, aby se nerozplozovali.
\par 20 V tom casu narodil se Mojžíš, a byl velmi krásný, kterýžto chován jest za tri mesíce v domu otce svého.
\par 21 A když vyložen byl na reku, vzala jej dcera faraonova, a vychovala jej sobe za syna.
\par 22 I vyucen jest Mojžíš vší moudrosti Egyptské, a byl mocný v recech i v skutcích.
\par 23 A když jemu bylo ctyridceti let, vstoupilo na srdce jeho, aby navštívil bratrí své, syny Izraelské.
\par 24 A uzrev jednoho, an bezpráví trpí, zastal ho a pomstil toho, kterýž bezpráví trpel, zabiv Egyptského.
\par 25 Domníval se zajisté, že bratrí jeho rozumejí tomu, že skrze ruku jeho chce jim dáti Buh vysvobození, ale oni nerozumeli.
\par 26 Druhého pak dne ukázal se jim, když se vadili, i chtel je v pokoj uvésti, rka: Muži, bratrí jste, i proc krivdu ciníte sobe vespolek?
\par 27 Ten pak, kterýž cinil krivdu bližnímu svému, odehnal ho, rka: Kdo te ustanovil knížetem a soudcím nad námi?
\par 28 Což ty mne chceš zamordovati, jako jsi vcera zabil Egyptského?
\par 29 I utekl Mojžíš pro ta slova a bydlil pohostinu v zemi Madianské, a tam zplodil dva syny.
\par 30 A když se vyplnilo let ctyridceti, ukázal se jemu na poušti hory Sinai andel Páne, v plameni ohne ve kri.
\par 31 A Mojžíš uzrev to, divil se tomu videní. A když blíže pristoupil, aby to pilneji spatril, stal se k nemu hlas Páne:
\par 32 Ját jsem Buh otcu tvých, Buh Abrahamuv a Buh Izákuv a Buh Jákobuv. I zhroziv se Mojžíš, neodvážil se patriti.
\par 33 I rekl jemu Pán: Zzuj obuv s noh svých; nebo místo, na kterémž stojíš, zeme svatá jest.
\par 34 Videl jsem, videl trápení lidu svého, kterýž jest v Egypte, a vzdychání jejich uslyšel jsem a sstoupil jsem, abych je vysvobodil. Protož nyní pojd, pošli te do Egypta.
\par 35 Toho Mojžíše, kteréhož se odepreli, rkouce: Kdo te ustanovil knížetem a soudcí? tohot jest Buh kníže a vysvoboditele poslal, skrze ruku andela, kterýž se jemu ukázal ve kri.
\par 36 A ten je vyvedl, cine divy a zázraky v zemi Egyptské a na mori Cerveném, i na poušti za ctyridceti let.
\par 37 Tot jest ten Mojžíš, kterýž rekl synum Izraelským: Proroka vám vzbudí Pán Buh váš z bratrí vašich, podobne jako mne, toho poslouchejte.
\par 38 Ont jest, kterýž byl mezi lidem na poušti s andelem, kterýž mluvíval k nemu na hore Sinai, i s otci našimi, kterýž prijal slova živá, aby je nám vydal.
\par 39 Jehož nechteli poslušni býti otcové naši, ale zavrhli jej, a odvrátili se srdci svými do Egypta,
\par 40 Rkouce k Aronovi: Ucin nám bohy, kteríž by šli pred námi; nebo Mojžíšovi tomu, kterýž nás vyvedl z zeme Egyptské, nevíme, co se prihodilo.
\par 41 I udelali v tech dnech tele, a obetovali obeti modle, a veselili se v díle rukou svých.
\par 42 I odvrátil se od nich Buh, a vydal je, aby sloužili vojsku nebeskému, jakož napsáno jest v knihách Prorockých: Zdaliž jste mi obeti aneb dary obetovali za ctyridceti let na poušti, dome Izraelský?
\par 43 Nýbrž nosili jste stánek modly Moloch, a hvezdu boha vašeho Remfan, ta podobenství, kteráž jste zdelali sobe, abyste se jim klaneli. Protož prestehuji vás za Babylon.
\par 44 Stánek svedectví meli jsou otcové naši na poušti, jakož byl narídil ten, jenž rekl Mojžíšovi, aby jej udelal, podle zpusobu toho, kterýž byl videl.
\par 45 Kterýžto prijavše otcové naši, vnesli jej s Jozue tam, kdež bylo prve vladarství pohanu, kteréž vyhnal Buh od tvári otcu našich, až do dnu Davida.
\par 46 Jenž nalezl milost pred oblicejem Božím, a prosil, aby nalezl stánek Bohu Jákobovu.
\par 47 Šalomoun pak udelal jemu dum.
\par 48 Ale Nejvyšší nebydlí v domích rukou udelaných, jakož dí prorok:
\par 49 Nebe jest mi stolice a zeme podnož noh mých, i jakýž mi tedy dum udeláte? praví Pán. Anebo jaké jest místo odpocívání mého?
\par 50 Zdaliž ruka má všeho toho neucinila?
\par 51 Tvrdošijní a neobrezaného srdce i uší, vy jste se vždycky Duchu svatému protivili, jakož otcové vaši, takž i vy.
\par 52 Kterému z proroku otcové vaši se neprotivili? Zmordovali zajisté ty, jenž predzvestovali príchod spravedlivého tohoto, jehožto vy nyní zrádci a vražedníci jste.
\par 53 Kteríž jste vzali Zákon pusobením andelským, a neostríhali jste ho.
\par 54 Tedy slyšíce to, rozzlobili se v srdcích svých a škripeli zuby na neho.
\par 55 On pak pln jsa Ducha svatého, pohledev do nebe, uzrel slávu Boží a Ježíše stojícího na pravici Boží.
\par 56 I rekl: Aj, vidím nebesa otevrená a Syna cloveka stojícího na pravici Boží.
\par 57 A oni zkrikše hlasem velikým, zacpali uši své, a oborili se jednomyslne na nej.
\par 58 A vyvedše jej z mesta, kamenovali ho. A svedkové složili roucha svá u noh mládence, kterýž sloul Saul.
\par 59 I kamenovali Štepána vzývajícího Boha a rkoucího: Pane Ježíši, prijmi ducha mého.
\par 60 A poklek na kolena, zvolal hlasem velikým: Pane, nepokládej jim toho za hrích. A to povedev, usnul v Pánu.

\chapter{8}

\par 1 Saul pak také privolil k usmrcení jeho. I prišlo v ten cas veliké protivenství na církev, kteráž byla v Jeruzaléme, a všickni se rozprchli po krajinách Judských a Samarských, krome apoštolu.
\par 2 I pochovali Štepána muži pobožní, a plakali velmi nad ním.
\par 3 Saul pak hubil církev, do domu vcházeje, a jímaje muže i ženy, dával je do žaláre.
\par 4 Ti pak, kteríž se byli rozprchli, chodili, kážíce slovo Boží.
\par 5 A Filip všed do mesta Samarí, kázal jim Krista.
\par 6 I pozorovali zástupové s pilností jednomyslne toho, což se pravilo od Filipa, slyšíce a vidouce divy, kteréž cinil.
\par 7 Nebo duchové necistí z mnohých, kteríž je meli, kricíce hlasem velikým, vycházeli, a mnozí šlakem poražení a kulhaví uzdraveni jsou.
\par 8 A stala se radost veliká v tom meste.
\par 9 Muž pak nejaký, jménem Šimon, pred tím v tom meste cáry provodil, a lid Samarský mámil, prave se býti nejakým velikým.
\par 10 Na nehož pozor meli všickni, od nejmenšího až do nejvetšího, ríkajíce: Tentot jest Boží moc veliká.
\par 11 Pozor pak meli na neho, protože je za mnohý cas mámil svými cáry.
\par 12 A když, uverivše Filipovi zvestujícímu o království Božím a o jménu Ježíše Krista, krtili se muži i ženy,
\par 13 Tedy i ten Šimon také uveril, a pokrten byv, prídržel se Filipa, a vida zázraky a moci veliké cinené, desil se.
\par 14 Uslyšavše pak v Jeruzaléme apoštolé, že by Samarí prijala slovo Boží, poslali k nim Petra a Jana.
\par 15 Kteríž prišedše k nim, modlili se za ne, aby prijali Ducha svatého.
\par 16 (Nebo ješte byl na žádného z nich nesstoupil, ale pokrteni toliko byli ve jménu Pána Ježíše.)
\par 17 Tedy vzkládali na ne ruce, a oni prijali Ducha svatého.
\par 18 I uzrev Šimon, že skrze vzkládání rukou apoštolských dává se Duch svatý, prinesl jim peníze,
\par 19 Rka: Dejte i mne tu moc, at, na kohož bych koli vzložil ruce, prijme Ducha svatého.
\par 20 I rekl k nemu Petr: Peníze tvé budtež s tebou na zatracení, protože jsi se domníval, že by dar Boží mohl býti zjednán za peníze.
\par 21 Nemáš dílu ani losu v této veci; nebo srdce tvé není uprímé pred Bohem.
\par 22 Protož cin pokání z této své nešlechetnosti, a pros Boha, zda by odpušteno bylo tobe to myšlení srdce tvého.
\par 23 Nebo v žluci horkosti a v svazku nepravosti tebe býti vidím.
\par 24 I odpovedev Šimon, rekl: Modltež vy se za mne Pánu, aby na mne neprišlo neco z tech vecí, kteréž jste mluvili.
\par 25 Oni pak osvedcovavše a mluvivše slovo Páne, navrátili se do Jeruzaléma, a ve mnohých mesteckách Samaritánských kázali evangelium.
\par 26 V tom andel Páne mluvil k Filipovi, rka: Vstan a jdi ku polední strane na cestu, kteráž vede od Jeruzaléma do mesta Gázy, kteréž jest pusté.
\par 27 A on vstav, i šel. A aj, muž Mourenín, kleštenec, komorník královny Mourenínské Kandáces, kterýž vládl všemi poklady jejími, a byl prijel do Jeruzaléma, aby se modlil,
\par 28 Již se navracoval zase, sede na voze svém, a cetl Izaiáše proroka.
\par 29 I rekl Duch k Filipovi: Pristup a privin se k vozu tomu.
\par 30 A pribeh Filip, slyšel jej, an cte Izaiáše proroka. I rekl: Rozumíš-liž, co cteš?
\par 31 A on rekl: Kterakž bych mohl rozumeti, lec by mi kdo vyložil? I prosil Filipa, aby vstoupil na vuz, a sedel s ním.
\par 32 Místo pak toho Písma, kteréž cetl, toto bylo: Jako ovce k zabití veden jest, a jako beránek nemý pred tím, kdož jej striže, tak neotevrel úst svých.
\par 33 V ponížení jeho odsouzení jeho vyhlazeno jest, rod pak jeho kdo vypraví, ackoli zahlazen byl z zeme život jeho?
\par 34 A odpovídaje komorník Filipovi, dí: Prosím tebe, o kom toto mluví prorok? Sám-li o sobe, cili o nekom jiném?
\par 35 Tedy otevrev Filip ústa svá, a pocav od toho Písma, zvestoval jemu Ježíše.
\par 36 A když jeli cestou, prijeli k jedné vode. I rekl komorník: Aj, ted voda. Proc nemám býti pokrten?
\par 37 I rekl Filip: Veríš-li celým srdcem, slušít. A on odpovedev, rekl: Verím, že Ježíš Kristus jest Syn Boží.
\par 38 I rozkázal státi vozu, a sstoupili oba do vody, i Filip i komorník. I pokrtil ho.
\par 39 A když vystoupili z vody, Duch Páne pochopil Filipa, a nevidel ho více komorník; i jel cestou svou, raduje se.
\par 40 Filip pak nalezen jest v Azotu; a chode, kázal evangelium všem mestum, až prišel do Cesaree.

\chapter{9}

\par 1 Saul pak ješte dychte po pohružkách a po mordu proti ucedlníkum Páne, šel k nejvyššímu knezi,
\par 2 A vyžádal od neho listy do Damašku do škol, nalezl-li by tam té cesty které muže nebo ženy, aby svázané privedl do Jeruzaléma.
\par 3 A když byl na ceste, stalo se, že již približoval se k Damašku. Tedy pojednou rychle obklícilo jej svetlo s nebe.
\par 4 A padna na zem, uslyšel hlas rkoucí: Saule, Saule, proc mi se protivíš?
\par 5 A on rekl: I kdo jsi, Pane? A Pán rekl: Já jsem Ježíš, jemuž ty se protivíš. Tvrdot jest tobe proti ostnum se zpecovati.
\par 6 A on tresa se a boje se, rekl: Pane, co chceš, abych cinil? A Pán k nemu: Vstan a jdi do mesta, a bude tobe povedíno, co bys ty mel ciniti.
\par 7 Ti pak muži, kteríž šli za ním, stáli, ohromeni jsouce, hlas zajisté slyšíce, ale žádného nevidouce.
\par 8 I vstal Saul z zeme, a otevrev oci své, nic nevidel. Tedy pojavše ho za ruce, uvedli jej do Damašku.
\par 9 I byl tu za tri dni nevida, a nejedl nic, ani nepil.
\par 10 Byl pak jeden ucedlník apoštolský v Damašku, jménem Ananiáš. I rekl k nemu Pán u videní: Ananiáši. A on rekl: Aj, já, Pane.
\par 11 A Pán k nemu: Vstan a jdi do ulice, kteráž slove Prímá, a hledej v dome Judove Saule, jménem Tarsenského. Nebo aj, modlí se,
\par 12 A videl u videní muže, Ananiáše jménem, an jde k nemu, a vzkládá nan ruku, aby zrak prijal.
\par 13 I odpovedel Ananiáš: Pane, slyšel jsem od mnohých o tom muži, kterak mnoho zlého cinil svatým tvým v Jeruzaléme.
\par 14 A i zdet má moc od predních kneží, aby jímal všecky, kteríž vzývají jméno tvé.
\par 15 I rekl jemu Pán: Jdi, nebot on jest má nádoba vyvolená, aby nosil jméno mé pred pohany i krále, i pred syny Izraelské.
\par 16 Ját zajisté ukáži jemu, kterak on mnoho musí trpeti pro jméno mé.
\par 17 I šel Ananiáš, a všel do toho domu, a vloživ ruce nan, rekl: Saule bratre, Pán Ježíš poslal mne, kterýž se ukázal tobe na ceste, po níž jsi šel, abys zrak prijal a naplnen byl Duchem svatým.
\par 18 A hned spadly s ocí jeho jako lupiny, a on prohlédl pojednou; a vstav, pokrten jest.
\par 19 A prijav pokrm, posilnil se. I zustal Saul s ucedlníky, kteríž byli v Damašku, za nekoliko dní.
\par 20 A hned v školách kázal Krista, prave, že on jest Syn Boží.
\par 21 I divili se náramne všickni, kteríž jej slyšeli, a pravili: Zdaliž toto není ten, jenž hubil v Jeruzaléme ty, kteríž vzývali jméno toto, a sem na to prišel, aby je svázané vedl k predním knežím?
\par 22 Saul pak mnohem více se zmocnoval a zahanboval Židy, kteríž byli v Damašku, potvrzuje toho, že ten jest Kristus.
\par 23 A když prebehlo drahne dnu, radu mezi sebou na tom zavreli Židé, aby jej zabili.
\par 24 Ale zvedel Saul o tech úkladech jejich. Ostríhali také i bran ve dne i v noci, aby jej zahubili.
\par 25 Ale ucedlníci v noci vzavše ho, spustili jej po provaze pres zed v koši.
\par 26 Prišed pak Saul do Jeruzaléma, pokoušel se pritovaryšiti k ucedlníkum, ale báli se ho všickni, neveríce, by byl ucedlníkem.
\par 27 Barnabáš pak prijav jej, vedl ho k apoštolum, a vypravoval jim, kterak na ceste videl Pána, a že mluvil s ním, a kterak v Damašku svobodne mluvil ve jménu Ježíše.
\par 28 I byl s nimi prebývaje v Jeruzaléme,
\par 29 A smele mluve ve jménu Pána Ježíše, a hádal se s Reky; oni pak usilovali ho zabíti.
\par 30 To zvedevše bratrí, dovedli ho do Cesaree, a poslali jej do Tarsu.
\par 31 A tak sborové po všem Judstvu a Galilei i Samarí meli pokoj, vzdelávajíce se, a chodíce v bázni Páne, a rozhojnovali se potešením Ducha svatého.
\par 32 Stalo se pak, že Petr, když procházel všecky sbory, prišel také k svatým, kteríž byli v Lydde.
\par 33 I nalezl tu cloveka jednoho, jménem Eneáše, již od osmi let na loži ležícího, kterýž byl šlakem poražený.
\par 34 I rekl mu Petr: Eneáši, uzdravujet tebe Ježíš Kristus; vstan a ustel sobe. A hned vstal.
\par 35 I videli jej všickni, kteríž bydlili v Lydde a v Sárone, kteríž se obrátili ku Pánu.
\par 36 Byla pak jedna ucedlnice v Joppen, jménem Tabita, což se vykládá Dorkas. Ta byla plná skutku dobrých a almužen, kteréž cinila.
\par 37 I stalo se v tech dnech, že roznemohši se, umrela. Kteroužto umyvše, položili na sín vrchní.
\par 38 A že byla blízko Lydda od Joppen, uslyšavše ucedlníci, že by tam byl Petr, poslali k nemu dva muže, prosíce ho, aby sobe neobtežoval prijíti až k nim.
\par 39 Tedy vstav Petr, šel s nimi. A když prišel, vedli jej na sín. I obstoupily ho všecky vdovy, placíce a ukazujíce sukne a plášte, kteréž jim delala, dokudž s nimi byla, Dorkas.
\par 40 I rozkázav všechnem vyjíti ven Petr, poklek na kolena, modlil se, a obrátiv se k tomu telu, rekl: Tabito, vstan. A ona otevrela oci své, a uzrevši Petra, posadila se.
\par 41 A podav jí ruky Petr, pozdvihl jí; a povolav svatých a vdov, ukázal jim ji živou.
\par 42 I rozhlášeno jest to po všem meste Joppen, a uverili mnozí v Pána.
\par 43 I stalo se, že za mnohé dni pozustal Petr v Joppen u nejakého Šimona koželuha.

\chapter{10}

\par 1 Muž pak nejaký byl v Cesarii, jménem Kornelius, setník, z zástupu, kterýž sloul Vlaský,
\par 2 Nábožný a bohobojný se vším domem svým, cine almužny mnohé lidu,
\par 3 A modle se Bohu vždycky. Ten videl u videní zretelne, jako v hodinu devátou na den, andela Božího, an všel k nemu, a rekl jemu: Kornéli.
\par 4 A on pilne popatril nan, a zstrašiv se, rekl: Co chceš, Pane? I rekl jemu: Modlitby tvé a almužny tvé vstoupily na pamet pred tvárí Boží.
\par 5 Protož nyní pošli muže nekteré do Joppen, a povolej Šimona, kterýž má príjmí Petr.
\par 6 Tent hospodu má u nejakého Šimona koželuha, kterýž má dum u more. Ont poví tobe, co bys mel ciniti.
\par 7 A když odšel andel, kterýž mluvil Korneliovi, zavolal dvou služebníku svých, a rytíre pobožného z tech, kteríž vždycky pri nem byli,
\par 8 A povedev jim všecko to, poslal je do Joppen.
\par 9 Nazejtrí pak, když oni šli a približovali se k mestu, všel Petr nahoru, aby se modlil, v hodinu šestou.
\par 10 A potom velice zlacnev, chtel pojísti. Když pak oni strojili, pripadlo na nej mysli vytržení.
\par 11 I uzrel nebe otevrené a sstupující k sobe nádobu jakous jako prosteradlo veliké, za ctyri rohy uvázané, ano se spouští na zem,
\par 12 Na nemž byla všeliká zemská hovada ctvernohá, a zvírata, a zemeplazové, i ptactvo nebeské.
\par 13 I stal se hlas k nemu: Vstan, Petre, bij a jez.
\par 14 I rekl Petr: Nikoli, Pane, nebt jsem nikdy nejedl nic obecného anebo necistého.
\par 15 Tedy opet po druhé stal se hlas k nemu: Cožt jest Buh ocistil, nemej ty toho za necisté.
\par 16 A to se stalo po trikrát. I vzato jest zase prosteradlo do nebe.
\par 17 A když Petr sám u sebe rozjímal, co by znamenalo videní to, kteréž videl, aj muži ti, kteríž posláni byli od Kornelia, ptajíce se na dum Šimonuv, stáli prede dvermi.
\par 18 A zavolavše kohosi, tázali se: Má-li zde hospodu Šimon, kterýž má príjmí Petr?
\par 19 A když Petr premyšloval o tom videní, rekl jemu Duch: Aj, muži tri hledají tebe.
\par 20 Protož vstana, sejdi dolu a jdi s nimi, nic nepochybuje; nebot jsem já je poslal.
\par 21 Tedy sšed Petr k mužum, jenž posláni byli k nemu od Kornelia, rekl: Aj, ját jsem ten, kteréhož hledáte. Jaká jest prícina, pro niž jste prišli?
\par 22 Oni pak rekli: Kornelius setník, muž spravedlivý a bohabojící, i svedectví dobré mající ode všeho národu Židovského, u videní napomenut jest od andela svatého, aby povolal tebe do domu svého a slyšel rec od tebe.
\par 23 Tedy zavolav jich do domu, prijal je do hospody. Druhého pak dne Petr šel s nimi, a nekterí bratrí z Joppen šli s ním.
\par 24 A nazejtrí prišli do Cesaree. Kornelius pak ocekával jich, svolav príbuzné své a prátely blízké.
\par 25 I stalo se, když vcházel Petr, vyšel proti nemu Kornelius, a padna k nohám jeho, klanel se mu.
\par 26 Ale Petr pozdvihl ho, rka: Vstan, všakt i já také clovek jsem.
\par 27 A rozmlouvaje s ním, všel do domu, a nalezl mnoho tech, kteríž se byli sešli.
\par 28 I rekl k nim: Vy víte, že neslušné jest muži Židu pripojiti se aneb pristoupiti k cizozemci, ale mne ukázal Buh, abych žádného cloveka nepravil obecným neb necistým býti.
\par 29 Protož bez odporu prišel jsem, povolán jsa. I ptám se vás, pro kterou prícinu poslali jste pro mne?
\par 30 A Kornelius rekl: Prede ctyrmi dny postil jsem se až do této hodiny, a v hodinu devátou modlil jsem se v domu svém. A aj, postavil se prede mnou muž v rouše belostkvoucím,
\par 31 A rekl: Kornéli, uslyšánat jest modlitba tvá a almužny tvé jsout v pameti pred tvárí Boží.
\par 32 Protož pošli do Joppen a povolej Šimona, kterýž slove Petr. Tent má hospodu v domu Šimona koželuha u more; on prijda, bude mluviti tobe.
\par 33 Já pak hned jsem poslal k tobe, a ty jsi dobre ucinil, žes prišel. Nyní tedy my všickni pred oblicejem Božím hotovi jsme slyšeti všecko, což jest koli prikázáno tobe od Boha.
\par 34 Tedy Petr otevrev ústa, rekl: V pravde jsem shledal, že Buh není prijimac osob.
\par 35 Ale v každém národu, kdož se ho bojí a ciní spravedlnost, príjemný jest jemu;
\par 36 Jakž to oznámil synum Izraelským, zvestuje pokoj skrze Ježíše Krista, jenž jest Pánem všeho.
\par 37 O cemž i vy sami víte, co se dálo po všem Židovstvu, pocna od Galilee, po krtu, kterýž kázal Jan:
\par 38 Kterak Ježíše od Nazaréta pomazal Buh Duchem svatým a mocí; kterýžto chodil, dobre cine, a uzdravuje všecky posedlé od dábla; nebo Buh s ním byl.
\par 39 A my jsme svedkové všeho toho, což jest cinil v krajine Judské a v Jeruzaléme. Kteréhožto zamordovali jsou, povesivše na dreve.
\par 40 Toho Buh vzkrísil tretího dne, a zpusobil to, aby zjeven byl,
\par 41 Ne všemu lidu, ale svedkum prve k tomu zrízeným od Boha, nám, kteríž jsme s ním jedli a pili po jeho z mrtvých vstání.
\par 42 A prikázal nám kázati lidu a svedciti, že on jest ten ustanovený od Boha soudce živých i mrtvých.
\par 43 Jemut všickni proroci svedectví vydávají, že odpuštení hríchu vezme skrze jméno jeho všeliký, kdožkoli uveril by v neho.
\par 44 A když ješte Petr mluvil slova tato, sstoupil Duch svatý na všecky, kteríž poslouchali slova Božího.
\par 45 I užasli se ti, jenž z obrezaných verící byli, kteríž byli prišli s Petrem, že i na pohany dar Ducha svatého jest vylit.
\par 46 Nebo slyšeli je, ani mluví jazyky rozlicnými, a velebí Boha. Tedy odpovedel Petr:
\par 47 Zdali muže kdo zabrániti vody, aby tito nebyli pokrteni, kteríž Ducha svatého prijali jako i my?
\par 48 A rozkázal je pokrtíti ve jménu Páne. I prosili ho, aby u nich pobyl za nekterý den.

\chapter{11}

\par 1 Uslyšeli pak apoštolé a bratrí, kteríž byli v Judstvu, že by i pohané prijali slovo Boží.
\par 2 A když prišel Petr do Jeruzaléma, domlouvali se nan ti, kteríž byli z obrezaných,
\par 3 Rkouce: K mužum neobrezaným všel jsi, a jedl jsi s nimi.
\par 4 Tedy zacav Petr, vypravoval jim porád, rka:
\par 5 Byl jsem v meste Joppen, modle se. I videl jsem u vytržení mysli jsa videní, nádobu nejakou sstupující jako prosteradlo veliké, za ctyri rohy privázané, ano se spouští s nebe, a prišlo až ke mne.
\par 6 V kteréž pohledev pilne, spatril jsem hovada zemská ctvernohá, i zvírata, a zemeplazy, i ptactvo nebeské.
\par 7 Slyšel jsem také i hlas ke mne rkoucí: Vstan, Petre, bij a jez.
\par 8 I rekl jsem: Nikoli, Pane, nebo nic obecného aneb necistého nikdy nevcházelo v ústa má.
\par 9 I odpovedel mi hlas po druhé s nebe, rka: Co jest Buh ocistil, nemej ty toho za necisté.
\par 10 A to se stalo po trikrát. I vtrženo jest zase to všecko do nebe.
\par 11 A aj, hned té chvíle tri muži stáli u domu, v kterémž jsem byl, posláni jsouce ke mne z Cesaree.
\par 12 I rekl mi Duch, abych šel s nimi, nic nepochybuje. A šlo se mnou i techto šest bratrí, a vešli jsme do domu muže jednoho.
\par 13 Kterýžto vypravoval nám, kterak videl andela v domu svém, an se pred ním postavil a rekl jemu: Pošli do Joppen muže nekteré a povolej Šimona, kterýž slove Petr.
\par 14 Ont tobe bude mluviti slova, skrze než spasen budeš ty i všecken tvuj dum.
\par 15 Když jsem pak já mluviti zacal, sstoupil Duch svatý na ne jako i na nás na pocátku.
\par 16 I rozpomenul jsem se na slovo Páne, kteréž byl povedel: Jan zajisté krtil vodou, ale vy pokrteni budete Duchem svatým.
\par 17 Ponevadž tedy jednostejný dar dal jim Buh jako i nám, kteríž uverili v Pána Ježíše Krista, i kdož jsem já byl, abych mohl zabrániti Bohu?
\par 18 To uslyšavše, spokojili se a slavili Boha, rkouce: Tedy i pohanum Buh pokání dal k životu.
\par 19 Ti pak, kteríž se byli rozprchli prícinou soužení, kteréž se bylo stalo pro Štepána, prišli až do Fenicen a Cypru a do Antiochie, žádnému nemluvíce slova Božího než samým toliko Židum.
\par 20 A byli nekterí z nich muži z Cypru a nekterí Cyrenenští, kterížto prišedše do Antiochie, mluvili Rekum, zvestujíce jim Pána Ježíše.
\par 21 A byla ruka Páne s nimi, a veliký pocet verících obrátil se ku Pánu.
\par 22 I prišla povest o tom k církvi, kteráž byla v Jeruzaléme. I poslali Barnabáše, aby šel až do Antiochie.
\par 23 Kterýžto prišed tam, a uzrev milost Boží, zradoval se, a napomínal všech, aby v úmyslu srdce svého trvali v Pánu.
\par 24 Nebo byl muž dobrý, a plný Ducha svatého a víry. I pribyl jich tu veliký zástup Pánu.
\par 25 Tedy šel odtud Barnabáš do Tarsu hledati Saule, a nalezna jej, privedl ho do Antiochie.
\par 26 I byli pres celý rok pri tom sboru, a ucili zástup veliký, takže nejprv tu v Antiochii ucedlníci nazváni jsou krestané.
\par 27 V tech pak dnech prišli z Jeruzaléma proroci do Antiochie.
\par 28 I povstav jeden z nich, jménem Agabus, oznamoval, ponuknut jsa skrze Ducha, že bude hlad veliký po všem okršlku zemském. Kterýžto i stal se za císare Klaudia.
\par 29 Tedy ucedlníci, jeden každý podle možnosti své, umínili poslati neco ku pomoci bratrím prebývajícím v Judstvu.
\par 30 Což i ucinili, poslavše k starším skrze ruce Barnabáše a Saule.

\chapter{12}

\par 1 A pri tom casu dal se v to Herodes král, aby sužoval nekteré z církve.
\par 2 I zamordoval Jakuba, bratra Janova, mecem.
\par 3 A vida, že se to líbilo Židum, umínil jíti i Petra. (A byli dnové presnic.)
\par 4 Kteréhož jav, do žaláre vsadil, poruciv jej šestnácti žoldnérum k ostríhání, chteje po velikonoci vyvésti jej lidu.
\par 5 I byl Petr ostríhán v žalári, modlitba pak ustavicná k Bohu dála se za nej od církve.
\par 6 A když jej již vyvésti mel Herodes, té noci spal Petr mezi dvema žoldnéri, svázán jsa dvema retezy, a strážní prede dvermi ostríhali žaláre.
\par 7 A aj, andel Páne postavil se, a svetlo se zastkvelo v žalári; a uderiv Petra v bok, zbudil ho, rka: Vstan rychle. I spadli retezové s rukou jeho.
\par 8 Tedy rekl andel k nemu: Opaš se a podvaž obuv svou. To když ucinil, rekl jemu: Odej se pláštem svým, a pojd za mnou.
\par 9 I vyšed, bral se za ním, a nevedel, by to pravé bylo, co se dálo skrze andela, ale domníval se, že by videní videl.
\par 10 A prošedše skrze první i druhou stráž, prišli k bráne železné, kteráž vede do mesta, a ta se jim hned sama otevrela. A všedše skrze ni, prešli ulici jednu, a hned odšel andel od neho.
\par 11 Tedy Petr prišed sám k sobe, rekl: Nyní práve vím, že poslal Pán andela svého, a vytrhl mne z ruky Herodesovy, a ze všeho ocekávání lidu Židovského.
\par 12 A pováživ toho, šel k domu Marie, matky Janovy, kterýž príjmí mel Marek, kdež se jich bylo mnoho sešlo, a modlili se.
\par 13 A když Petr potloukl na dvere, vyšla devecka, aby poslechla, jménem Ródé.
\par 14 A poznavši hlas Petruv, pro radost neotevrela dverí, ale vbehši, zvestovala, že Petr stojí u dverí.
\par 15 A oni rekli jí: I co blázníš? Ona pak potvrzovala, že tak jest. Tedy oni rekli: Andel jeho jest.
\par 16 Ale Petr predce tloukl. A otevrevše dvere, uzreli jej, i ulekli se.
\par 17 A pokynuv na ne rukou, aby mlceli, vypravoval jim, kterak jest jej Pán vyvedl z žaláre, a potom rekl: Poveztež to Jakubovi a bratrím. A vyšed, bral se na jiné místo.
\par 18 A když byl den, stal se rozbroj nemalý mezi žoldnéri o to, co se stalo pri Petrovi.
\par 19 Herodes pak ptaje se na nej, a nenalezna, vytazovav se na strážných, kázal je pryc vésti; a odebrav se z Judstva do Cesaree, prebýval tam.
\par 20 A v ten cas Herodes rozzlobil se proti Tyrským a Sidonským. Kterížto jednomyslne prišli k nemu, a namluvivše sobe Blasta, predního komorníka královského, žádali za pokoj, protože jejich krajiny potravu mely z zemí královských.
\par 21 V uložený pak den Herodes, obleka se v královské roucho, a posadiv se na soudné stolici, ucinil k nim rec.
\par 22 I zvolal lid, rka: Boží jest toto hlas, a ne lidský.
\par 23 A ihned ranil jej andel Páne, protože nevzdal slávy Bohu; a rozlez se cervy, umrel.
\par 24 A slovo Páne rostlo a rozmáhalo se.
\par 25 Barnabáš pak a Saul navrátili se z Jeruzaléma, vykonavše službu, pojavše s sebou i Jana, kterýž príjmí mel Marek.

\chapter{13}

\par 1 Byli pak v církvi, kteráž byla v Antiochii, proroci a ucitelé, jako Barnabáš a Šimon, kterýž mel príjmí Cerný, a Lucius Cyrenenský, a Manahen, kterýž byl spolu vychován s Herodesem tetrarchou, a Saul.
\par 2 A když oni služby Páne konali a postili se, dí jim Duch svatý: Oddelte mi Barnabáše a Saule k dílu, k kterémuž jsem jich povolal.
\par 3 Tedy postíce se, a modlíce se, a vzkládajíce na ne ruce, propustili je.
\par 4 A oni posláni jsouce od Ducha svatého, prišli do Seleucie, a odtud plavili se do Cypru.
\par 5 A prišedše do Salaminy, kázali slovo Boží v školách Židovských; a meli s sebou i Jana k službe.
\par 6 A když zchodili ten ostrov až do Páfu, nalezli tu jakéhos carodejníka, falešného proroka Žida, jemuž jméno bylo Barjezus,
\par 7 Kterýž byl u znamenitého vladare Sergia Pavla, muže opatrného. Ten povolav Barnabáše a Saule, žádal od nich slyšeti slovo Boží.
\par 8 Ale protivil se jim Elymas, totiž carodejník ten, (nebo se tak vykládá jméno jeho,) usiluje odvrátiti vladare od víry.
\par 9 Tedy Saul, kterýž slove i Pavel, naplnen jsa Duchem svatým, pilne pohledev na nej,
\par 10 Rekl: Ó plný vší lsti a vší nešlechetnosti, synu dábluv, a nepríteli vší spravedlnosti, což neprestaneš prevraceti cest Páne prímých?
\par 11 A aj, nyní ruka Páne nad tebou, a budeš slepý, nevida slunce až do casu. A pojednou pripadla na nej mrákota a tma, a jda vukol, hledal, kdo by ho za ruku vedl.
\par 12 Tehdy vladar uzrev, co se stalo, uveril, dive se ucení Páne.
\par 13 A pustivše se od Páfu Pavel a ti, kteríž s ním byli, prišli do mesta Pergen v krajine Pamfylii. Jan pak odšed od nich, vrátil se do Jeruzaléma.
\par 14 Oni pak šedše z Pergen, prišli do Antiochie Pisidické, a všedše do školy v den sobotní, posadili se.
\par 15 A když bylo po ctení Zákona a Proroku, poslali k nim knížata školy té rkouce: Muži bratrí, máte-li úmysl jaké napomenutí uciniti k lidu, mluvte.
\par 16 Tedy Pavel povstav a rukou, aby mlceli, pokynuv, rekl: Muži Izraelští, a kteríž se bojíte Boha, slyšte.
\par 17 Buh lidu tohoto Izraelského vyvolil otce naše a lidu povýšil, když byl pohostinu v zemi Egyptské, a v rameni vztaženém vyvedl je z ní.
\par 18 A za cas ctyridceti let snášel jejich obyceje na poušti.
\par 19 A zahladiv sedm národu v zemi Kanán, rozdelil losem mezi ne zemi jejich.
\par 20 A potom, témer za ctyri sta a padesáte let, dával jim soudce až do Samuele proroka.
\par 21 A vtom žádali za krále, i dal jim Buh Saule, syna Cis, muže z pokolení Beniaminova, za ctyridceti let.
\par 22 A když toho zavrhl, vzbudil jim Davida krále, kterémužto svedectví dávaje, rekl: Nalezl jsem Davida, syna Jesse, muže podle srdce svého, kterýž bude ciniti všecku vuli mou.
\par 23 Z jehožto semene Buh podle zaslíbení vzbudil lidu Izraelskému Spasitele Ježíše,
\par 24 Pred jehožto príštím kázal Jan krest pokání všemu lidu Izraelskému.
\par 25 A když Jan dokonával beh svuj, pravil: Kteréhož se mne domníváte býti, nejsem já ten, ale aj, prijdet po mne tak dustojný, ješto já nejsem hoden rozvázati obuvi noh jeho.
\par 26 Muži bratrí, synové rodu Abrahamova, a kteríž mezi vámi jsou bojící se Boha, vám slovo spasení tohoto posláno jest.
\par 27 Nebo ti, kteríž prebývají v Jeruzaléme, a knížata jejich, toho Ježíše neznajíce, odsoudili, a tak hlasy Prorocké, kteríž se na každou sobotu ctou, naplnili.
\par 28 A nižádné viny hodné smrti na nem nenalezše, aby zamordován byl, Piláta prosili.
\par 29 A když dokonali všecko, což o nem psáno bylo, složen jsa s dreva, do hrobu jest položen.
\par 30 Buh pak vzkrísil jej z mrtvých.
\par 31 Kterýžto vidín jest po mnohé dni od tech, jenž spolu s ním byli prišli z Galilee do Jeruzaléma, kterížto jsou svedkové jeho k lidu.
\par 32 A my vám zvestujeme to zaslíbení, které se stalo otcum, že jest je již Buh naplnil nám synum jejich, vzkrísiv Ježíše;
\par 33 Jakož i v druhém Žalmu napsáno jest: Syn muj jsi ty, já dnes zplodil jsem tebe.
\par 34 A že jej z mrtvých vzkrísil, aby se již více nenavracoval v porušení, takto o tom rekl: Dám vám svaté veci Davidovy verné.
\par 35 Protož i v jiném Žalmu dí: Nedáš svatému svému videti porušení.
\par 36 David zajisté za svého veku poslouživ vuli Boží, usnul, a priložen jest k otcum svým, a videl porušení,
\par 37 Ale ten, kteréhož Buh vzkrísil, nevidel porušení.
\par 38 Protož známo vám bud, muži bratrí, že skrze toho zvestuje se vám odpuštení hríchu,
\par 39 A to ode všech, od kterýchž jste nemohli skrze Zákon Mojžíšuv ospravedlneni býti, skrze tohoto každý, kdož verí, bývá ospravedlnen.
\par 40 Protož vizte, at na vás neprijde to, což jest v Prorocích povedíno:
\par 41 Vizte potupníci, a podivte se, a na nic prijdte; nebo já dílo delám za dnu vašich, dílo to, o kterémž vy neuveríte, kdyby je vám kdo vypravoval.
\par 42 A když vycházeli ze školy Židovské, prosili jich pohané, aby jim v druhou sobotu mluvili táž slova.
\par 43 A když bylo rozpušteno shromáždení, šlo mnoho Židu a nábožných lidí znovu na víru obrácených za Pavlem a Barnabášem, kterížto promlouvajíce k nim, radili jim, aby trvali v milosti Boží.
\par 44 V druhou pak sobotu témer všecko mesto sešlo se k slyšení slova Božího.
\par 45 A Židé vidouce zástupy, naplneni jsou závistí, a odporovali tomu, co bylo praveno od Pavla, protivíce a rouhajíce se.
\par 46 Tedy svobodne Pavel a Barnabáš rekli: Vámt jest melo nejprv mluveno býti slovo Boží, ale ponevadž je zamítáte, a za nehodné sebe soudíte vecného života, aj, obracíme se ku pohanum.
\par 47 Nebot jest nám tak prikázal Pán, rka: Položil jsem tebe svetlo pohanum, tak abys ty byl spasení až do koncin zeme.
\par 48 A slyšíce to pohané, zradovali se a velebili slovo Páne; a uverili všickni, což jich koli bylo predzrízeno k životu vecnému.
\par 49 I rozhlašovalo se slovo Páne po vší krajine.
\par 50 Tedy Židé zbourili ženy nábožné a poctivé a prední meštany, a vzbudili protivenství proti Pavlovi a Barnabášovi, i vyhnali je z koncin svých.
\par 51 A oni vyrazivše prach z noh svých na ne, prišli do Ikonie.
\par 52 Ucedlníci pak naplneni byli radostí a Duchem svatým.

\chapter{14}

\par 1 I stalo se v Ikonii, že vešli spolu do školy Židovské, a mluvili slovo Boží, takže jest uverilo i Židu i Reku veliké množství.
\par 2 Ale kteríž z Židu nepovolní byli, ti zbourili a zdráždili mysli pohanu proti bratrím.
\par 3 I byli tu za dlouhý cas, smele a svobodne mluvíce v Pánu, kterýž svedectví vydával slovu milosti své, a pusobil to, aby se dáli divové a zázrakové skrze ruce jejich.
\par 4 I rozdelilo se množství mesta, a jedni byli s Židy, a jiní s apoštoly.
\par 5 A když se oborili na apoštoly i pohané i Židé s knížaty svými, aby je utiskli a kamenovali,
\par 6 Oni srozumevše tomu, utekli do mest Lykaonitských, do Lystry a do Derben, a do toho okolí,
\par 7 A tu kázali evangelium.
\par 8 Muž pak nejaký v Lystre, nemocný na nohy, sedával, chromý jsa hned z života matky své, kterýž nikdy nechodil.
\par 9 Ten poslouchal Pavla mluvícího. Kterýžto pohledev nan, a vida, an verí, že uzdraven bude,
\par 10 Rekl velikým hlasem: Postav se príme na nohách svých. I zchopil se a chodil.
\par 11 Zástupové pak videvše, co ucinil Pavel, pozdvihli hlasu svého, Lykaonitsky rkouce: Bohové pripodobnivše se lidem, sstoupili k nám.
\par 12 I nazvali Barnabáše Jupiterem a Pavla Merkuriášem; nebo on mluvil slovo Boží.
\par 13 Tedy knez Jupitera, modly té, kteráž byla pred mestem jejich, privedl býky a prinesl vence pred bránu, a chtel s lidem obeti obetovati.
\par 14 To když uslyšeli apoštolé, Barnabáš a Pavel, roztrhše sukne své, vybehli k zástupum, kricíce,
\par 15 A rkouce: Muži, což to ciníte? Však i my lidé jsme, týmž bídám jako i vy poddaní, kteríž vás napomínáme, abyste se obrátili od techto marností k Bohu živému, kterýž ucinil nebe i zemi, i more, i všecko, což v nich jest.
\par 16 Kterýžto za predešlých veku všech pohanu nechával, aby chodili po cestách svých,
\par 17 Ackoli proto nenechal sebe bez osvedcení, dobre cine, dávaje nám s nebe déšt a casy úrodné, naplnuje pokrmem a potešením srdce naše.
\par 18 A to mluvíce, sotva spokojili zástupy, aby jim neobetovali.
\par 19 A vtom prišli od Antiochie a Ikonie nejací Židé, kterížto navedše zástupy, a ukamenovavše Pavla, vytáhli jej pred mesto, domnívajíce se, že umrel.
\par 20 A když jej obstoupili ucedlníci, vstal a všel do mesta, a nazejtrí odšel s Barnabášem do Derben.
\par 21 A kázavše evangelium mestu tomu, a ucedlníku mnoho získavše, navrátili se do Lystry a do Ikonie a do Antiochie,
\par 22 Potvrzujíce duší ucedlníku, a napomínajíce jich, aby trvali u víre, a pravíce, že musíme skrze mnohá soužení vjíti do království Božího.
\par 23 A zrídivše jim, podle daných hlasu, starší po církvech, a modlivše se s postem, porucili je Pánu, v kteréhož jsou uverili.
\par 24 A prošedše Pisidii, prišli do Pamfylie,
\par 25 A mluvivše slovo Boží v Pergen, šli do mesta Attalie.
\par 26 A odtud plavili se do Antiochie, odkudž poruceni byli milosti Boží k dílu, kteréž jsou vykonali.
\par 27 A když tam prišli a shromáždili církev, vypravovali jim, kteraké veci Buh skrze ne ucinil a že otevrel pohanum dvere víry.
\par 28 I byli tu za dlouhý cas s ucedlníky.

\chapter{15}

\par 1 Prišedše pak nekterí z Židovstva, ucili bratrí: Že nebudete-li se obrezovati podle obyceje Mojžíšova, nebudete moci spaseni býti.
\par 2 A když se stala mezi nimi ruznice, a nemalou hádku Pavel a Barnabáš s nimi mel, i zustali na tom, aby Pavel a Barnabáš a nekterí jiní z nich šli k apoštolum a starším do Jeruzaléma o tu otázku.
\par 3 Tedy oni jsouce vyprovozeni od církve, šli skrze Fenicen a Samarí, vypravujíce o obrácení pohanu na víru, i zpusobili radost velikou všem bratrím.
\par 4 A když se dostali do Jeruzaléma, prijati jsou od církve a od apoštolu a starších. I zvestovali jim, kteraké veci cinil skrze ne Buh.
\par 5 A že povstali nekterí z sekty farizejské, kteríž byli uverili, pravíce, že musejí obrezováni býti, a potom aby jim bylo prikázáno zachovávati zákon Mojžíšuv.
\par 6 Tedy sešli se apoštolé a starší, aby toho povážili.
\par 7 A když mnohé vyhledávání toho bylo, povstav Petr, rekl k nim: Muži bratrí, vy víte, že od dávních dnu mezi námi Buh vyvolil mne, aby skrze ústa má slyšeli pohané slovo evangelium a uverili.
\par 8 A ten, jenž zpytuje srdce, Buh, svedectví jim vydal, dav jim Ducha svatého, jako i nám.
\par 9 A neucinil rozdílu mezi nimi a námi, verou ocistiv srdce jejich.
\par 10 Protož nyní, proc pokoušíte Boha, chtíce vzložiti na hrdlo ucedlníku jho, kteréhož ani otcové naši, ani my nésti jsme nemohli?
\par 11 Ale skrze milost Pána Ježíše Krista veríme, že spaseni budeme, rovne jako i oni.
\par 12 I mlcelo všecko to množství, a poslouchali Barnabáše a Pavla, vypravujících, kteraké divy a zázraky cinil Buh skrze ne mezi pohany.
\par 13 A když oni tak umlkli, odpovedel Jakub, rka: Muži bratrí, slyšte mne.
\par 14 Šimon ted vypravoval, kterak Buh nejprve popatril na pohany, aby z nich prijal lid jménu svému.
\par 15 A s tím se srovnávají i reci prorocké, jakož psáno jest:
\par 16 Potom se navrátím, a vzdelám zase stánek Daviduv, kterýž byl klesl, a zboreniny jeho zase vzdelám, a vyzdvihnu jej,
\par 17 Tak aby ti ostatkové toho lidu hledali Pána, i všickni pohané, nad kterýmiž jest vzýváno jméno mé, dí Pán, jenž ciní tyto všecky veci.
\par 18 Známát jsou Bohu od veku všecka díla jeho.
\par 19 Protož já tak soudím, aby nebyli kormouceni ti, kteríž se z pohanu obracejí k Bohu,
\par 20 Ale aby jim napsáno bylo, at se zdržují od poskvrn modl, a smilstva, a toho, což jest udáveného a od krve.
\par 21 Nebo Mojžíš od dávních veku má po všech mestech, kdo by jej kázal v školách, ponevadž na každou sobotu cítán bývá.
\par 22 Tehdy videlo se apoštolum a starším se vší církví, aby vyvolili z sebe muže a poslali je do Antiochie s Pavlem a Barnabášem: Judu, kterýž sloul Barsabáš, a Sílu, muže znamenité mezi bratrími,
\par 23 Napsavše po nich toto: Apoštolé a starší i všickni bratrí tem, kteríž jsou v Antiochii a v Syrii a v Cilicii bratrím, kteríž jsou z pohanu, pozdravení vzkazují.
\par 24 Ponevadž jsme slyšeli, že nekterí vyšedše od nás, zkormoutili vás, recmi svými zemdlévajíce duše vaše, pravíce, že se máte obrezovati a Zákon zachovávati, jimž jsme toho neporucili:
\par 25 I videlo se nám shromáždeným jednomyslne, abychom vyvolili muže nekteré a poslali k vám s nejmilejšími našimi bratrími, Barnabášem a Pavlem,
\par 26 Lidmi temi, kteríž vydali duše své pro jméno Pána našeho Ježíše Krista.
\par 27 Protož poslali jsme Judu a Sílu, a tit i ústne povedí vám totéž.
\par 28 Videlo se zajisté Duchu svatému i nám, žádného více na vás bremene nevzkládati, krome techto vecí potrebných,
\par 29 Totiž abyste se zdržovali od obetovaného modlám, a od krve, a od udáveného, a od smilstva. Od tech vecí budete-li se ostríhati, dobre budete ciniti. Mejte se dobre.
\par 30 Tedy oni propušteni jsouce, prišli do Antiochie, a shromáždivše množství, dali jim list.
\par 31 Kterýžto ctouce, radovali se z toho potešení jich.
\par 32 Judas pak a Sílas, byvše i oni proroci, širokou recí napomínali bratrí a potvrzovali jich.
\par 33 A pobyvše tu za nekterý cas, propušteni jsou od bratrí v pokoji zase k apoštolum.
\par 34 Ale Sílovi se videlo, aby tu zustal.
\par 35 Tolikéž Pavel i Barnabáš pobyli v Antiochii, ucíce a zvestujíce slovo Páne, i s mnohými jinými.
\par 36 Po nekolika pak dnech rekl Pavel k Barnabášovi: Vracujíce se, navštevme bratrí naše po všech mestech, v kterýchž jsme kázali slovo Páne, a prezvíme, kterak se mají.
\par 37 Tedy Barnabáš radil, aby pojali s sebou i Jana, kterýž príjmí mel Marek.
\par 38 Ale Pavlovi se nezdálo pojíti toho s sebou, kterýž byl odšel od nich z Pamfylie, aniž šel s nimi ku práci.
\par 39 I vznikl mezi nimi tuhý odpor, takže se rozešli ruzno. Barnabáš pak pojav s sebou Marka, plavil se do Cypru.
\par 40 A Pavel pripojiv k sobe Sílu, odšel, porucen jsa milosti Boží od bratrí.
\par 41 I chodil po Syrii a Cilicii, potvrzuje církví.

\chapter{16}

\par 1 Prišel pak do Derben a do Lystry, aj, ucedlník jeden tu byl, jménem Timoteus, syn jedné ženy Židovky verící, ale otce mel Reka.
\par 2 Tomu svedectví dobré vydávali ti, jenž byli v Lystre a v Ikonii bratrí.
\par 3 Toho sobe oblíbil Pavel, aby s ním šel. I pojav ho, obrezal jej, pro Židy, kteríž byli v tech místech; nebo vedeli všickni, že otec jeho byl Rek.
\par 4 A když chodili po mestech, vydávali jim k ostríhání ustanovení zrízená od apoštolu a od starších, kteríž byli v Jeruzaléme.
\par 5 A tak církve utvrzovaly se u víre a rozmáhaly se v poctu na každý den.
\par 6 A prošedše Frygii i Galatskou krajinu, když jim zabráneno od Ducha svatého, aby nemluvili slova Božího v Azii,
\par 7 Prišedše do Myzie, pokoušeli se jíti do Bitynie. Ale nedal jim Duch Ježíšuv.
\par 8 Tedy pominuvše Myzii, šli do Troady.
\par 9 I ukázalo se Pavlovi v noci videní, jako by muž nejaký Macedonský stál, a prosil ho, rka: Prijda do Macedonie, pomoz nám.
\par 10 A jakž to videní videl, ihned jsme usilovali o to, abychom šli do Macedonie, tím ujišteni jsouce, že jest nás povolal Pán, abychom jim kázali evangelium.
\par 11 Protož pustivše se od Troady, prímým behem priplavili jsme se do Samotracie a nazejtrí do Neapolis,
\par 12 A odtud do Filippis, kteréžto jest první mesto krajiny Macedonské, obyvateli cizími osazené. I zustali jsme v tom meste za nekolik dní.
\par 13 V den pak sobotní vyšli jsme ven za mesto k rece, kdež býval obycej modliti se. A usadivše se, mluvili jsme ženám, kteréž se tu byly sešly.
\par 14 Jedna pak žena, jménem Lydia, kteráž šarlaty prodávala v meste Tyatirských, bohabojící, poslouchala nás. Jejížto srdce otevrel Pán, aby to pilne rozsuzovala, co se od Pavla pravilo.
\par 15 A když pokrtena byla i dum její, prosila, rkuci: Ponevadž jste mne soudili vernou Pánu býti, vejdetež do domu mého, a pobudte u mne. I prinutila nás.
\par 16 I stalo se, když jsme šli k modlitbe, že devecka nejaká, mající ducha veštího, potkala se s námi, kteráž veliký užitek prinášela pánum svým budoucích vecí predpovídáním.
\par 17 Ta šedši za Pavlem a za námi, volala, rkuci: Tito lidé jsou služebníci Boha nejvyššího, kterížto zvestují nám cestu spasení.
\par 18 A to cinívala za mnoho dní. Pavel pak težce to nesa, a obrátiv se, rekl duchu tomu: Prikazuji tobe ve jménu Ježíše Krista, abys vyšel od ní. I vyšel hned té chvíle.
\par 19 A videvše páni její, že jest odešla nadeje zisku jejich, chytivše Pavla a Sílu, vedli je na rynk pred úrad.
\par 20 A postavivše je pred úredníky, rekli: Tito lidé bourí mesto naše, jsouce Židé,
\par 21 A zvestují obyceje, kterýchž nám nesluší prijíti ani zachovávati, ponevadž jsme my Rímané.
\par 22 I povstala obec proti nim. A úredníci, roztrhše sukne jejich, kázali je metlami mrskati.
\par 23 A množství ran jim davše, vsadili je do žaláre, prikázavše strážnému žaláre, aby jich pilne ostríhal.
\par 24 Tedy on takové maje prikázání, vsadil je do nejhlubšího žaláre, a nohy jejich sevrel kladou.
\par 25 O pulnoci pak Pavel a Sílas modléce se, zpívali písnicky o Bohu, takže je slyšeli i jiní veznové.
\par 26 A vtom pojednou zeme tresení stalo se veliké, až se pohnuli gruntové žaláre, a hned se jim všecky dvere otevrely a všech okovové spadli.
\par 27 I procítiv strážný žaláre a vida dvere žaláre otevrené, vytrhl mec, aby se zabil, domnívaje se, že veznové utekli.
\par 28 I zkrikl nan Pavel hlasem velikým, rka: Necin sobe nic zlého, však jsme ted všickni.
\par 29 A požádav svetla, vbehl k nim, a tresa se, padl k nohám Pavlovým a Sílovým.
\par 30 I vyved je ven, rekl: Páni, co já mám ciniti, abych spasen byl?
\par 31 A oni rekli: Ver v Pána Ježíše a budeš spasen ty i dum tvuj.
\par 32 I mluvili jemu slovo Páne, i všechnem, kteríž byli v domu jeho.
\par 33 A on pojav je v tu hodinu v noci, umyl rány jejich. I pokrten jest hned on i všecka celed jeho.
\par 34 A když je uvedl do domu svého, pripravil jim stul, a veselil se, že jest se vším domem svým uveril Bohu.
\par 35 A když bylo ve dne, poslali úredníci služebníky, rkouce: Vypust ty lidi.
\par 36 I oznámil strážný žaláre slova ta Pavlovi, prave: Že poslali úredníci, abyste byli propušteni. Protož nyní vyjdouce, jdetež v pokoji.
\par 37 Ale Pavel rekl k nim: Zmrskavše nás zjevne a bez vyslyšení, lidi Rímany, vsázeli do žaláre, a již nyní nás chtejí tajne vyhnati? Nikoli, ale nechat sami prijdou, a vyvedou nás.
\par 38 Tedy povedeli úredníkum služebníci slova ta. I báli se, uslyšavše, že by Rímané byli.
\par 39 A prišedše, prosili jich; a vyvedše je, žádali jich, aby šli z mesta.
\par 40 I vyšedše z žaláre, vešli k Lydii, a uzrevše bratrí, potešili jich, i šli odtud.

\chapter{17}

\par 1 A prošedše mesto Amfipolim a Apollonii, prišli do Tessaloniky, kdež byla Židovská škola.
\par 2 Tedy Pavel podle obyceje svého všel k nim, a po tri soboty kázal jim z Písem,
\par 3 Otvíraje a predkládaje to, že mel Kristus trpeti a z mrtvých vstáti, a že ten jest Kristus Ježíš, kteréhož já zvestuji vám.
\par 4 I uverili nekterí z nich, a pripojili se Pavlovi a Sílovi, i Reku nábožných veliké množství, i žen znamenitých nemálo.
\par 5 Ale zažženi jsouce závistí Židé pravde nepovolní, a privzavše k sobe nekteré lehkomyslné a zlé lidi, shlukše se, zbourili mesto, a útok ucinivše na dum Jázonuv, hledali jich, aby je vyvedli pred lid.
\par 6 A nenalezše jich, táhli Jázona a nekteré bratrí k starším mesta, kricíce: Tito, kteríž všecken svet bourí, ti sem také prišli,
\par 7 Kteréž prijal Jázon. A ti všickni proti ustanovení císarskému ciní, pravíce býti králem jiného, jménem Ježíše.
\par 8 A tak zbourili obec i starší mesta, kteríž od nich to slyšeli.
\par 9 Ale oni prijavše dosti ucinení od Jázona a jiných, propustili je.
\par 10 Bratrí pak hned v noci vypustili i Pavla i Sílu do Berie. Kterížto prišedše tam, vešli do školy Židovské.
\par 11 A ti byli udatnejší nežli Tessalonitští, kterížto prijali slovo Boží se vší chtivostí, na každý den rozvažujíce Písma, tak-li by ty veci byly, jakž kázal Pavel.
\par 12 A tak mnozí z nich uverili, i Recké ženy poctivé i mužu nemálo.
\par 13 A když zvedeli Židé v Tessalonice, že by i v Berii kázáno bylo slovo Boží od Pavla, prišli také tam, bouríce zástupy.
\par 14 Ale hned bratrí vypustili Pavla, aby šel jako k mori; Sílas pak a Timoteus zustali tu.
\par 15 Ti pak, kteríž provodili Pavla, dovedli ho až do Atén. A vzavše porucení k Sílovi a k Timoteovi, aby prišli k nemu, což nejspíše mohou, šli zase.
\par 16 A když Pavel cekal jich v Aténách, roznecoval se v nem duch jeho, vida to mesto oddané býti modloslužbe.
\par 17 I rozmlouval s Židy a nábožnými lidmi v škole, ano i na rynku, po všecky dni, s temi, kteríž se koli nahodili.
\par 18 Tedy nekterí z epikureu a stoických mudráku hádali se s ním. A nekterí rekli: I co tento žvác chce povedíti? Jiní pak pravili: Zdá se býti nejakých cizích bohu zvestovatel. Neb jim o Ježíšovi a o z mrtvých vstání vypravoval.
\par 19 I popadše jej, vedli ho do Areopágu, a rekli jemu: Mužeme-li vedeti, jaké jest to ucení nové, kteréž vypravuješ?
\par 20 Nebo nové jakési veci vkládáš v uši naše, protož chceme vedeti, co by to bylo.
\par 21 (Aténští zajisté všickni, i ti, kteríž tu byli hosté, k nicemuž jinému tak hotovi nebyli, než praviti neb slyšeti neco nového.)
\par 22 Tedy Pavel, stoje uprostred Areopágu, rekl: Muži Aténští, vidím vás býti všelijak príliš nábožné lidi.
\par 23 Nebo procházeje se a spatruje náboženství vaše, nalezl jsem také oltár, na kterémž napsáno jest: Neznámému Bohu. Protož kteréhož vy ctíte neznajíce, tohot já zvestuji vám.
\par 24 Buh ten, kterýž stvoril svet i všecko, což jest na nem, ten jsa Pánem nebe i zeme, nebydlí v chrámích rukou udelaných,
\par 25 Aniž bývá cten lidskýma rukama, jako by neceho potreboval, ponevadž on dává všechnem život i dýchání i všecko.
\par 26 A ucinil z jedné krve všecko lidské pokolení, aby prebývalo na tvári vší zeme, vymeriv jim uložené casy a cíle prebývání jejich,
\par 27 Aby hledali Boha, zda by snad makajíce, mohli nalézti jej, ackoli není daleko od jednoho každého z nás.
\par 28 Nebo jím živi jsme, a hýbeme se, i trváme, jakož i nekterí z vašich poetu povedeli: Že i rodina jeho jsme.
\par 29 Rodina tedy Boží jsouce, nemámet se domnívati, že by Buh zlatu neb stríbru neb kamenu, remeslem anebo duvtipem lidským vyrytému, byl podoben.
\par 30 Nebo casy této neznámosti prehlídaje Buh, již nyní zvestuje lidem všechnem všudy, aby pokání cinili,
\par 31 Protože uložil den, v kterémžto souditi bude všecken svet v spravedlnosti skrze muže, kteréhož jest k tomu vystavil, slouže k víre o tom všechnem, vzkríšením jeho z mrtvých.
\par 32 Uslyševše pak o vzkríšení z mrtvých, nekterí se posmívali, a nekterí rekli: Budeme te slyšeti o tom po druhé.
\par 33 A tak Pavel vyšel z prostredku jejich.
\par 34 Nekterí pak muži, prídržíce se ho, uverili, mezi kterýmiž byl i Dionyzius Areopagitský, i žena, jménem Damaris, a jiní s nimi.

\chapter{18}

\par 1 Potom pak Pavel vyšed z Atén, prišel do Korintu.
\par 2 A nalezl jednoho Žida, jménem Akvilu, jenž byl rodem z Pontu, kterýž nedávno byl prišel z Vlach, i s Priscillou manželku svou, (protože byl rozkázal Klaudius, aby všickni Židé z Ríma vyšli), i privinul se k nim.
\par 3 A že byl téhož remesla jako oni, bydlil u nich a delal; a bylo remeslo jejich stany delati.
\par 4 I hádal se v škole na každou sobotu a k získání privodil i Židy i Reky.
\par 5 A když prišli z Macedonie Sílas a Timoteus, roznecoval se v duchu Pavel, osvedcuje Židum, že Ježíš jest Kristus.
\par 6 A když jemu oni odporovali a rouhali se, vyraziv prach z roucha svého, rekl k nim: Krev vaše budiž na hlavu vaši. Já cist jsa, hned pujdu ku pohanum.
\par 7 A jda odtud, všel do domu cloveka jednoho, jménem Justa, ctitele Božího, kteréhož dum byl u samé školy.
\par 8 Krispus pak, kníže školy, uveril Pánu se vším domem svým, a mnozí z Korintských slyšíce, uverili a krteni byli.
\par 9 I rekl Pán v noci u videní Pavlovi: Neboj se, ale mluv a nemlc.
\par 10 Nebt já s tebou jsem, a žádnýt nesáhne na tebe, at by zle ucinil; nebo mnoho mám lidu v tomto meste.
\par 11 I byl tu pul druhého léta, káže jim slovo Boží.
\par 12 Když pak Gallio vladarem byl v Achaii, povstali jednomyslne Židé proti Pavlovi, a privedli jej pred soudnou stolici,
\par 13 Pravíce: Tento navodí lidi, aby proti Zákonu ctili Boha.
\par 14 A když Pavel mel již otevríti ústa, rekl Gallio k Židum: Ó Židé, jestliže by co nepravého stalo se, nebo nešlechetnost nejaká, slušne bych vás vyslyšel.
\par 15 Paklit jsou jaké hádky o slovích a o jméních a Zákonu vašem, vy sami k tomu prihlédnete. Ját toho soudce býti nechci.
\par 16 I odehnal je od soudné stolice.
\par 17 Tedy Rekové všickni, uchopivše Sostena, kníže školy Židovské, bili jej tu pred soudnou stolicí, a Gallio na to nic nedbal.
\par 18 Pavel pak, pobyv tam ješte za mnoho dní, i rozžehnav se s bratrími, plavil se do Syrie, a s ním spolu Priscilla a Akvila, oholiv hlavu v Cenchreis; nebo byl ucinil slib.
\par 19 I prišel do Efezu a nechal jich tu; sám pak všed do školy, hádal se s Židy.
\par 20 A když ho prosili, aby tu déle pobyl u nich, nepovolil.
\par 21 Ale požehnav jich, rekl: Musím já jistotne svátek ten, kterýž nastává, v Jeruzaléme slaviti, ale navrátím se k vám zase, bude-li vule Boží. I bral se z Efezu.
\par 22 A prišed do Cesaree, vstoupil do Jeruzaléma, a pozdraviv církve, odtud šel do Antiochie.
\par 23 A pobyv tu za nekterý cas, odšel a procházel porád Galatskou krajinu a Frygii, potvrzuje všech ucedlníku.
\par 24 Žid pak nejaký, jménem Apollo, rodem z Alexandrie, muž výmluvný, prišel do Efezu, ucený v Písme.
\par 25 Ten byl pocátecne naucen ceste Páne, a jsa vroucího ducha, horlive mluvil a ucil pilne tem vecem, kteréž jsou Páne, znaje toliko krest Januv.
\par 26 A ten pocal smele a svobodne mluviti v škole. Kteréhož slyševše Priscilla a Akvila, prijali ho k sobe a dokonaleji vypravovali jemu o ceste Boží.
\par 27 A když chtel jíti do Achaie, bratrí napomenuvše ho, psali ucedlníkum, aby jej prijali. Kterýžto když tam prišel, mnoho prospel tem, kteríž uverili skrze milost Boží.
\par 28 Nebo náramne premáhal Židy, zjevne prede všemi jim toho dokazuje z Písem, že Ježíš jest Kristus.

\chapter{19}

\par 1 I stalo se, když Apollo byl v Korintu, že Pavel prošed vrchní krajiny, prišel do Efezu, a nalezna tu nekteré ucedlníky,
\par 2 Rekl k nim: Prijali-li jste Ducha svatého, uverivše? A oni rekli jemu: Ba, aniž jsme slýchali, jest-li Duch svatý.
\par 3 On pak rekl jim: Nacež tedy pokrteni jste? A oni rekli: Krteni jsme krtem Janovým.
\par 4 I rekl Pavel: Jant zajisté krtil krtem pokání, prave lidu, aby v toho, kterýž mel po nem prijíti, verili, to jest v Krista Ježíše.
\par 5 Kteríž pak toho uposlechli, pokrteni jsou ve jméno Pána Ježíše.
\par 6 A když vzkládal na tyto ruce Pavel, sstoupil Duch svatý na ne, i mluvili jazyky rozlicnými, a prorokovali.
\par 7 A bylo všech spolu okolo dvanácti mužu.
\par 8 Tedy Pavel všed do školy, smele a svobodne mluvil za tri mesíce, hádaje se a uce o království Božím.
\par 9 A když se nekterí zatvrdili a nepovolovali, zle mluvíce o ceste Boží prede vším množstvím, odstoupiv od nich, oddelil ucedlníky, na každý den kázání cine v škole nejakého Tyranna.
\par 10 A to se dálo za dve léte, takže všickni, kteríž prebývali v Azii, poslouchali slova Pána Ježíše, i Židé i Rekové.
\par 11 A divy veliké cinil Buh skrze ruce Pavlovy,
\par 12 Takže i šátky a fertochy od jeho tela na nemocné nosívali, a odstupovaly od nich nemoci, a duchové necistí vycházeli z nich.
\par 13 Tedy pokusili se nekterí z Židu tuláku, kteríž se s zaklínáním obírali, vzývati jméno Pána Ježíše nad temi, jenž meli duchy necisté, ríkajíce: Zaklínáme vás skrze Ježíše, kteréhož káže Pavel.
\par 14 A bylo jich sedm synu jednoho Žida, jménem Scevy, predního kneze, kteríž to cinili.
\par 15 Tedy odpovedev duch zlý, rekl: Ježíše znám, a o Pavlovi vím, ale vy kdo jste?
\par 16 A oboriv se na ne clovek ten, v kterémž byl duch zlý, a opanovav je, zmocnil se jich, takže nazí a zranení vybehli z domu toho.
\par 17 A to známo ucineno jest všechnem Židum i Rekum bydlícím v Efezu, a spadla bázen na ne na všecky. I oslaveno jest jméno Pána Ježíše.
\par 18 Mnozí pak z verících pricházeli, vyznávajíce se a zjevujíce skutky své.
\par 19 Mnozí také z tech, kteríž se s marnými umeními obírali, snesše knihy o tech vecech, spálili je prede všemi; a pocetše cenu jejich, shledali toho padesáte tisícu penez.
\par 20 Tak jest mocne rostlo slovo Páne a zmocnovalo se.
\par 21 A když se to všecko dokonalo, uložil Pavel v duchu svém, aby projda Macedonii a Achaii, šel do Jeruzaléma, rka: Když pobudu tam, musímt také i na Rím pohledeti.
\par 22 I poslav dva z tech, jenž mu prisluhovali, do Macedonie, Timotea a Erasta, sám pozustal v Azii do casu.
\par 23 Tedy stala se v ten cas nemalá bourka pro cestu Boží.
\par 24 Nebo zlatník jeden, jménem Demetrius, kterýž delával chrámy stríbrné modly Diány, nemalý zisk privodil remeslníkum.
\par 25 Kteréž svolav, i ty, kteríž byli k tem podobných vecí delníci, rekl jim: Muži, víte, že z tohoto remesla jest živnost naše.
\par 26 A vidíte i slyšíte, že netoliko v Efezu, ale témer po vší Azii tento Pavel svedl a odvrátil veliké množství lidu, prave: Že to nejsou bohové, kteríž jsou rukama udelaní.
\par 27 A protož strach jest, netoliko aby se nám v živnosti naší prítrž nestala, ale také i veliké bohyne Diány chrám aby za nic nebyl jmín, a aby neprišlo k zkáze dustojenství její, kteroužto všecka Azia i všecken sveta okršlek ctí.
\par 28 To uslyšavše a naplneni jsouce hnevem, zkrikli rkouce: Veliká jest Diána Efezských.
\par 29 I naplneno jest všecko mesto rozbrojem, a valili se všickni spolu na plac, pochopivše Gáia a Aristarcha, Macedonské, tovaryše cesty Pavlovy.
\par 30 Pavlovi pak, když chtel jíti k lidu, nedopustili ucedlníci.
\par 31 Ano i nekterí z predních mužu Azianských, kteríž jemu práli, poslavše k nemu, prosili ho, aby se nedával do toho hluku.
\par 32 A jedni tak, jiní jinak kriceli; nebo byla obec zbourena, a mnozí nevedeli, proc jsou se sbehli.
\par 33 Z zástupu pak nekterí Alexandra nejakého táhli k mluvení, kteréhož i Židé k tomu nutili. Alexander pak pokynuv rukou, chtel zprávu dáti lidu.
\par 34 Ale jakž poznali, že jest Žid, ihned se stal jednostejný všech hlas, jako za dve hodine volajících: Veliká jest Diána Efezských.
\par 35 A když písar pokojil zástup, rekl: Muži Efezští, i kdož z lidí jest, ješto by nevedel, že mesto Efezské slouží veliké bohyni Diáne a od Jupitera spadlému obrazu?
\par 36 A ponevadž tomu odpíráno býti nemuže, slušné jest, abyste se upokojili a nic kvapne necinili.
\par 37 Privedli jste zajisté lidi tyto, kteríž nejsou ani svatokrádce ani ruhaci bohyne vaší.
\par 38 Jestliže pak Demetrius a ti, kteríž jsou s ním remeslníci, mají s kým jakou pri, však bývá obecný soud, a jsou k tomu konšelé. Nechat jedni druhé viní.
\par 39 Pakli ceho jiného hledáte, i tot muž v porádném svolání obce skoncováno býti.
\par 40 Nebo strach jest, abychom nedošli nesnáze pro tu bourku dnešní, ponevadž žádné príciny není, kterouž bychom mohli predložiti, proc jsme se tuto sbehli. A to povedev, rozpustil lid.

\chapter{20}

\par 1 Když pak prestala ta bourka, povolav Pavel ucedlníku a požehnav jich, vyšel odtud, aby se bral do Macedonie.
\par 2 A když prošel krajiny ty, a napomenutí jim uciniv mnohými recmi, prišel do zeme Recké,
\par 3 V kteréžto pobyv za tri mesíce, (kdežto Židé ucinili jemu zálohy,) když se mel plaviti do Syrie, umínil navrátiti se skrze Macedonii.
\par 4 I šel s ním Sópater Berienský až do Azie, a z Tessalonicenských Aristarchus a Sekundus a Gáius Derbeus a Timoteus, z Azianských pak Tychikus a Trofimus.
\par 5 Ti všickni šedše napred, dockali nás v Troade.
\par 6 My pak plavili jsme se po velikonoci z Filippis, a prišli jsme k nim do Troady v peti dnech, a tu jsme pobyli za sedm dní.
\par 7 Tedy první den po sobote, když se ucedlníci sešli k lámání chleba, Pavel mluvil k nim, maje nazejtrí jíti pryc, i prodlil recí až do pulnoci.
\par 8 A bylo mnoho svetel tu na té síni, kdež byli shromáždeni.
\par 9 Jeden pak mládenec, jménem Eutychus, sede na okne, jsa obtížen hlubokým snem, když tak dlouho Pavel kázal, spe, spadl s tretího ponebí dolu, a vzat jest mrtvý.
\par 10 I sstoupiv dolu Pavel, zpolehl na nej, a objav jej, rekl: Nermuttež se, však duše jeho v nem jest.
\par 11 A vstoupiv zase na sín, lámal chléb a jedl, a kázání jim uciniv dlouho až do svitání, tak odšel pryc.
\par 12 I privedli toho mládence živého, a byli nad tím velice potešeni.
\par 13 My pak vstoupivše na lodí, plavili jsme se do Asson, odtud majíce k sobe vzíti Pavla; neb jest byl tak porucil, maje sám jíti po zemi.
\par 14 A když se k nám pripojil v Asson, vzavše jej, prijeli jsme do Mitylénu.
\par 15 A odtud plavíce se, druhý den byli jsme proti Chium, a tretího dne pristavili jsme bárku k Sámu, a pobyvše v Trogyllí, nazejtrí prišli jsme do Milétu.
\par 16 Nebo Pavel byl umínil pominouti Efez, aby se nemeškal v Azii; nebo pospíchal, by možné bylo, aby byl o letnicích v Jeruzaléme.
\par 17 Tedy z Milétu poslav do Efezu, povolal k sobe starších církve.
\par 18 Kterížto když prišli k nemu, rekl jim: Vy víte od prvního dne, v kterýžto prišel jsem do Azie, kterak jsem po všecken ten cas s vámi byl,
\par 19 Slouže Pánu se vší pokorou i s mnohými slzami a pokušeními, kteráž na mne pricházela z úkladu Židovských.
\par 20 Kterak jsem nicehož nepominul, což by vám užitecného bylo, abych vám toho neoznámil, ale ucil jsem vás vubec zjevne i po domích,
\par 21 Svedectví vydávaje i Židum i Rekum o pokání k Bohu a o víre v Pána našeho Ježíše Krista.
\par 22 A aj, nyní já sevrín jsa duchem, beru se do Jeruzaléma, neveda, co mi se v nem má státi,
\par 23 Než že Duch svatý po mestech, kudyž jsem šel, osvedcuje mi, prave, že vezení a soužení mne ocekávají.
\par 24 Však já nic na to nedbám, aniž jest mi tak drahá duše má, jen abych beh svuj s radostí vykonal a prisluhování, kteréž jsem prijal od Pána Ježíše, k osvedcování evangelium milosti Boží.
\par 25 A aj, já nyní vím, že již více neuzríte tvári mé vy všickni, mezi kterýmiž jsem chodil, káže o království Božím.
\par 26 Protož osvedcujit vám dnešní den, žet jsem cist od krve všech.
\par 27 Nebt jsem neobmeškal zvestovati vám všeliké rady Boží.
\par 28 Budtež tedy sebe pilni i všeho stáda, v nemžto Duch svatý ustanovil vás biskupy, abyste pásli církev Boží, kteréž sobe dobyl svou vlastní krví.
\par 29 Nebo já to jistotne vím, že po mém odjití vejdou mezi vás vlci hltaví, kteríž nebudou odpoušteti stádu.
\par 30 A z vás samých povstanou muži, jenž budou mluviti prevrácené veci, aby obrátili ucedlníky po sobe.
\par 31 Protož bdete, v pameti majíce, že jsem po tri léta neprestával dnem i nocí s plácem napomínati jednoho každého z vás.
\par 32 A již nyní, bratrí, poroucím vás Bohu a slovu milosti jeho, kterýžto mocen jest vzdelati vás, a dáti vám dedictví mezi všemi posvecenými.
\par 33 Stríbra nebo zlata neb roucha nežádal jsem od žádného.
\par 34 Nýbrž sami víte, že toho, cehož mi kdy potrebí bylo, i tem, kteríž jsou se mnou, dobývaly ruce tyto.
\par 35 Vše ukázal jsem vám, že tak pracujíce, musíme snášeti mdlé, a pamatovati na slova Pána Ježíše; nebt on rekl: Blahoslaveneji jest dáti nežli bráti.
\par 36 A to povedev, klekna na kolena svá, modlil se s nimi se všemi.
\par 37 I stal se plác veliký ode všech, a padajíce na hrdlo Pavlovo, líbali jej,
\par 38 Rmoutíce se nejvíce nad tím slovem, kteréž rekl, že by již více nemeli tvári jeho videti. I provodili jej až k lodí.

\chapter{21}

\par 1 Když jsme se pak plavili, rozloucivše se s nimi, prímým behem prijeli jsme do Koum, a druhý den do Ródu, a odtud do Patary.
\par 2 I nalezše tu lodí, kteráž mela plouti do Fenicen, a vstoupivše na ni, plavili jsme se.
\par 3 A když se nám pocal ukazovati Cyprus, nechavše ho na levé strane, plavili jsme se do Syrie, a priplavili jsme se do Týru; nebo tu meli složiti náklad z lodí.
\par 4 A nalezše tu ucedlníky, pobyli jsme tam za sedm dní, kterížto pravili Pavlovi skrze Ducha svatého, aby nechodil do Jeruzaléma.
\par 5 A když jsme my vyplnili ty dni, vyšedše, brali jsme se pryc, a sprovodili nás všickni s ženami i s detmi až za mesto. A poklekše na kolena na brehu, pomodlili jsme se.
\par 6 A když jsme se vespolek rozžehnali, vstoupili jsme na lodí, a oni se vrátili domu.
\par 7 My pak preplavivše se od Týru, dostali jsme se až do Ptolemaidy, a pozdravivše tu bratrí, pobyli jsme u nich jeden den.
\par 8 A nazejtrí vyšedše Pavel a my, jenž jsme s ním byli, prišli jsme do Cesaree, a všedše do domu Filipa evangelisty, (kterýž byl jeden z onech sedmi,) pobyli jsme u neho.
\par 9 A ten mel ctyri dcery panny, prorokyne.
\par 10 A když jsme tu pobyli za drahne dní, prišel prorok nejaký z Judstva, jménem Agabus.
\par 11 Ten když k nám prišel, vzal pás Pavluv, a svázav sobe ruce i nohy, rekl: Totot praví Duch svatý: Muže toho, jehož jest pás tento, tak sváží Židé v Jeruzaléme a vydadí v ruce pohanum.
\par 12 A jakž jsme to uslyšeli, prosili jsme ho i my i ti, kteríž byli v tom míste, aby nechodil do Jeruzaléma.
\par 13 Tedy odpovedel Pavel: I co ciníte, placíce a trápíce srdce mé? Však já netoliko svázán býti, ale i umríti hotov jsem v Jeruzaléme pro jméno Pána Ježíše.
\par 14 A když nechtel povoliti, dali jsme tomu pokoj, rkouce: Stan se vule Páne.
\par 15 Po tech pak dnech pripravivše se, brali jsme se do Jeruzaléma.
\par 16 A šli s námi i ucedlníci nekterí z Cesaree, vedouce s sebou nejakého Mnázona z Cypru, starého ucedlníka, u nehož bychom hospodu meli.
\par 17 A když jsme prišli do Jeruzaléma, vdecne nás prijali bratrí.
\par 18 Druhého pak dne všel Pavel s námi k Jakubovi, a tu se byli všickni starší sešli.
\par 19 Jichžto pozdraviv, vypravoval jim všecko, což Buh skrze službu jeho cinil mezi pohany.
\par 20 A oni slyšavše to, velebili Pána, a rekli jemu: Vidíš, bratre, kterak jest mnoho tisícu Židu verících, a ti všickni jsou horliví milovníci Zákona.
\par 21 Ale o tobet mají zprávu, že bys ty vedl od Zákona Mojžíšova všecky ty Židy, kteríž jsou mezi pohany, prave, že nemají obrezovati synu svých, ani zachovávati obyceju Zákona.
\par 22 Což tedy ciniti? Musít zajisté shromáždeno býti všecko množství, nebot uslyší o tobe, že jsi prišel.
\par 23 Uciniž tedy toto, cožt povíme: Mámet tu ctyri muže, kteríž mají slib na sobe.
\par 24 Ty prijma k sobe, posvet se s nimi, i náklad ucin s nimi, aby oholili hlavy své. A takt zvedí všickni, že to, což slyšeli o tobe, nic není, ale že i sám ty chodíš, ostríhaje Zákona.
\par 25 Z strany pak tech, kteríž z pohanu uverili, my jsme psali, usoudivše, aby tohoto niceho nešetrili, toliko aby se varovali modlám obetovaného, a krve, a udáveného, a smilstva.
\par 26 Tedy Pavel, prijav k sobe ty muže, na druhý den posvetiv se s nimi, všel do chrámu, a vypravoval o vyplnení dní toho posvecení, až i obetována jest obet za jednoho každého z nich.
\par 27 A když se vyplniti melo dní sedm, Židé nekterí z Azie, uzrevše jej v chráme, zbourili všecken lid a vztáhli nan ruce,
\par 28 Kricíce: Muži Izraelští, pomozte! Totot jest ten clovek, kterýž proti lidu i Zákonu i místu tomuto všecky všudy ucí, a k tomu i pohany uvedl do chrámu, a poskvrnil svatého tohoto místa.
\par 29 Nebo byli videli prve Trofima Efezského s ním v meste, kteréhož domnívali se, že by Pavel do chrámu uvedl.
\par 30 Takž se zbourilo všecko mesto, a sbehli se lidé, a uchopivše Pavla, táhli jej ven z chrámu. A hned zavríny jsou dvere.
\par 31 A když jej chteli zamordovati, povedíno hejtmanu vojska, že se bourí všecken Jeruzalém.
\par 32 Kterýžto hned pojav s sebou žoldnére a setníky, pribehl na ne. A oni uzrevše hejtmana a žoldnére, prestali bíti Pavla.
\par 33 Tedy pristoupiv hejtman, dosáhl ho, a rozkázal jej svázati dvema retezy, a ptal se, kdo jest a co ucinil.
\par 34 V zástupu pak jedni tak, jiní jinak kriceli. A nemoha nic jistého zvedeti pro hluk, rozkázal jej vésti do vojska.
\par 35 A když prišel k stupnum, nahodilo se, že nesen byl od žoldnéru pro násilé lidu.
\par 36 Nebo šlo za ním množství lidu, kricíce: Zahlad jej!
\par 37 A když mel již uveden býti do vojska Pavel, rekl hejtmanu: Mohu-li co promluviti k tobe? Kterýž rekl: Umíš Recky?
\par 38 Nejsi-liž ty ten Egyptský, kterýž jsi pred temito dny byl bourku ucinil, a vyvedls na poušt ctyri tisíce lotru?
\par 39 I rekl Pavel: Ját jsem clovek Žid Tarsenský, neposledního mesta Cilické zeme obyvatel; protož prosím tebe, dopust mi promluviti k lidu.
\par 40 A když mu on dopustil, Pavel stoje na stupních, pokynul rukou na lid. A když se veliké mlcení stalo, mluvil k nim Židovsky, rka:

\chapter{22}

\par 1 Muži bratrí a otcové, poslechnete této mé omluvy, kterouž vám nyní predložím.
\par 2 (Uslyševše pak, že by k nim mluvil Židovským jazykem, tím radeji mlceli.) I rekl:
\par 3 Já zajisté jsem muž Žid, narozený v Tarsu meste Cilickém, ale vychován jsem v tomto meste u noh Gamalielových, vyucený s pilností podle Zákona otcovského, horlivý milovník Boha, jakož i vy všickni podnes jste.
\par 4 Kterýž jsem se této ceste protivil až k smrti, svazuje a dávaje do žaláre i muže i ženy,
\par 5 Jakož i nejvyšší knez svedek mi toho jest, i všickni starší. Od nichž i listy k bratrím vzav, šel jsem do Damašku, abych i ty, kteríž tam byli, svázané privedl do Jeruzaléma, aby byli trápeni.
\par 6 I stalo se, když jsem se bral cestou a približoval k Damašku, okolo poledne, že pojednou rychle s nebe obklícilo mne svetlo veliké.
\par 7 I padl jsem na zem, a slyšel jsem hlas, an mi dí: Saule, Saule, proc mi se protivíš?
\par 8 A já odpovedel jsem: Kdo jsi, Pane? I rekl ke mne: Ját jsem Ježíš Nazaretský, kterémuž ty se protivíš.
\par 9 Ti pak, kteríž se mnou byli, svetlo zajisté videli a prestrašeni jsou, ale hlasu toho, kterýž se mnou mluvil, neslyšeli.
\par 10 I rekl jsem: Pane, což mám ciniti? A Pán rekl ke mne: Vstana, jdiž do Damašku, a tut tobe bude povedíno všecko, což jest uloženo, abys ty cinil.
\par 11 A že jsem byl oslnul pro jasnost svetla toho, za ruce jsa veden od tech, kteríž se mnou byli, prišel jsem do Damašku.
\par 12 Ananiáš pak nejaký, muž pobožný podle zákona, svedectví maje ode všech prebývajících v Damašku Židu,
\par 13 Prišel ke mne, a stoje, rekl mi: Saule, bratre, prohlédni. A já hned v tu hodinu pohledel jsem na nej.
\par 14 I rekl mi: Buh otcu našich vyvolil te, abys poznal vuli jeho, a uzrel Spravedlivého tohoto, a abys slyšel hlas z úst jeho.
\par 15 Nebo svedkem jemu budeš u všech lidí tech vecí, kteréž jsi videl a slyšel.
\par 16 A protož nyní co prodléváš? Vstana, pokrti se, a obmej hríchy své, vzývaje jméno Páne.
\par 17 Stalo se pak, když jsem se navrátil do Jeruzaléma a modlil jsem se v chráme, že jsem byl u vytržení mysli.
\par 18 I videl jsem jej, an dí ke mne: Pospeš a vyjdi rychle z Jeruzaléma, nebot neprijmou svedectví tvého o mne.
\par 19 A já rekl jsem: Pane, onit vedí, že jsem já do žaláre dával a bil jsem v školách ty, kteríž verili v tebe.
\par 20 A když vylévali krev Štepána, svedka tvého, já také jsem tu stál, a privolil jsem k usmrcení jeho, a ostríhal jsem roucha tech, kteríž jej mordovali.
\par 21 A Pán rekl ke mne: Jdi, nebot já ku pohanum daleko pošli tebe.
\par 22 I poslouchali ho až do toho slova. A tu hned pozdvihli hlasu svého, rkouce: Zahlad z zeme takového, nebot nesluší jemu živu býti.
\par 23 A když oni kriceli, a metali s sebe roucha, a prachem házeli v povetrí,
\par 24 Rozkázal jej hejtman uvésti do vojska, a kázal jej bici mrskati, aby zvedel, pro kterou prícinu na nej tak kricí.
\par 25 A když jej svázali remením, rekl Pavel setníkovi, jenž tu stál: Sluší-liž vám cloveka Rímana a neodsouzeného mrskati?
\par 26 To uslyšav setník, pristoupe k hejtmanu, povedel jemu, rka: Viz, co chceš ciniti; nebo clovek tento jest Ríman.
\par 27 A pristoupiv hejtman, rekl mu: Povez mi, jsi-li ty Ríman? A on rekl: A já jsem.
\par 28 I odpovedel hejtman: Já jsem za veliké peníze toho meštanství dosáhl. Pavel pak rekl: Ale já jsem se i narodil Ríman.
\par 29 Tedy ihned odstoupili od neho ti, kteríž jej meli trápiti. Ano i hejtman bál se, když zvedel, že jest Ríman, a že jej byl kázal svázati.
\par 30 Nazejtrí pak, chteje zvedeti jistotu, z ceho by jej vinili Židé, propustil jej z pout, a rozkázal, aby se sešli prední kneží i všecka rada jejich. I vyvedl Pavla, a postavil ho pred nimi.

\chapter{23}

\par 1 Tedy Pavel, pohledev pilne na to shromáždení, rekl: Muži bratrí, já všelijak s dobrým svedomím sloužil jsem Bohu až do dnešního dne.
\par 2 Tedy nejvyšší knez Ananiáš kázal tem, kteríž tu stáli, aby jej bili v ústa.
\par 3 Pavel pak rekl jemu: Budet tebe bíti Buh, steno zbílená. A ty sedíš, soude mne vedle Zákona, a proti Zákonu velíš mne bíti.
\par 4 A nekterí stojíce tu rekli: Nejvyššímu knezi Božímu zlorecíš?
\par 5 I rekl Pavel: Nevedelt jsem, bratrí, byt nejvyšším knezem byl; psánot jest zajisté: Knížeti lidu svého nebudeš zloreciti.
\par 6 A veda Pavel, že tu byla jedna strana saduceu a druhá farizeu, zvolal v rade: Muži bratrí, já jsem farizeus, syn farizeuv; pro nadeji a z mrtvých vstání já tuto k soudu stojím.
\par 7 A když on to promluvil, stal se rozbroj mezi farizei a saducei, a rozdvojilo se množství.
\par 8 Nebo saduceové tak praví, že není vzkríšení, ani andela, ani ducha, ale farizeové obé to vyznávají.
\par 9 I stal se krik veliký. A povstavše ucitelé strany farizejské, zastávali ho, rkouce: Nic jsme zlého nenalezli na tomto cloveku; protož bud že mluvil jemu duch neb andel, nebojujme s Bohem.
\par 10 A když veliký rozbroj vznikl, obávaje se hejtman, aby Pavel nebyl od nich roztrhán, rozkázal žoldnérum sjíti dolu a vychvátiti ho z prostredku jejich a vésti do vojska.
\par 11 V druhou pak noc ukázav se jemu Pán, rekl: Budiž stálý, Pavle; nebo jakož jsi svedcil o mne v Jeruzaléme, tak musíš svedciti i v Ríme.
\par 12 A když byl den, sšedše se nekterí z Židu, zaprisáhli se s klatbou, rkouce, že nebudou jísti ani píti, až zabijí Pavla.
\par 13 A bylo jich více než ctyridceti, kteríž se byli tak spikli.
\par 14 Kterížto pristoupivše k predním knežím a k starším, rekli: Prokletím prokleli jsme se, že neokusíme nicehož, dokudž nezabijeme Pavla.
\par 15 Protož vy nyní dejte vedeti hejtmanu, z jednostejného vší rady svolení, aby jej zítra k vám privedl, jako byste neco jistšího chteli zvedeti o jeho vecech; my pak, prve nežlit se k vám priblíží, hotovi jsme jej zabíti.
\par 16 Slyšev pak o tech úkladech, syn sestry Pavlovy odšel, a všel do vojska a povedel Pavlovi.
\par 17 Tedy zavolav Pavel k sobe jednoho z setníku, rekl: Doved mládence tohoto k hejtmanu; nebot má jemu neco povedíti.
\par 18 A on pojav jej, vedl k hejtmanu a rekl jemu: Vezen Pavel zavolav mne, prosil, abych tohoto mládence privedl k tobe, že by mel neco mluvit s tebou.
\par 19 I vzav jej hejtman za ruku jeho, a odstoupiv s ním soukromí, otázal se ho: Co jest to, ješto mi máš oznámiti?
\par 20 Tedy on rekl: Že jsou Židé uložili prositi tebe, abys k nim zítra do rady uvedl Pavla, jako by neco jistšího chteli vyzvedeti o nem.
\par 21 Ale ty nepovoluj jim; nebot úklady ciní jemu více než ctyridceti mužu z nich, kteríž se zaprisáhli s klatbou, že nebudou ani jísti, ani píti, až jej zabijí. A jižt jsou hotovi, cekajíce na odpoved od tebe.
\par 22 Tedy hejtman propustil toho mládence, prikázav: Abys žádnému nepravil, žes mi to oznámil.
\par 23 A zavolav dvou setníku, rekl jim: Pripravte žoldnéru dve ste, aby šli až do Cesaree, a jezdcu sedmdesáte, a drabantu dve ste k tretí hodine na noc.
\par 24 I hovada privedte, aby vsadíce na ne Pavla, ve zdraví jej dovedli k Felixovi vladari;
\par 25 Napsav jemu také i list v tento rozum:
\par 26 Klaudius Lyziáš výbornému vladari Felixovi vzkazuje pozdravení.
\par 27 Muže tohoto javše Židé, hned zamordovati meli. Kteréhožto já, prispev s houfem žoldnéru, vydrel jsem, zvedev, že jest Ríman.
\par 28 A chteje zvedeti, z ceho by jej vinili, uvedl jsem ho do rady jejich.
\par 29 I shledal jsem, že na nej žalují o nejaké otázky Zákona jejich a že nemá žádné viny, pro kterouž by byl hoden smrti neb vezení.
\par 30 A když mi povedíno o úkladech, kteréž jsou o nem skládali Židé, ihned jsem jej poslal k tobe, prikázav také i žalobníkum jeho, aby, což mají proti nemu, oznámili pred tebou. Mej se dobre.
\par 31 Tedy žoldnéri, jakž jim poruceno bylo, pojavše Pavla, privedli jej v noci do Antipatridy.
\par 32 A nazejtrí, nechavše jízdných, aby s ním jeli, vrátili se do vojska.
\par 33 Oni pak prijevše do Cesaree, dali list vladari, a Pavla také postavili pred ním.
\par 34 A precta list vladar, i otázal se ho, z které by krajiny byl. A zvedev, že jest z Cilicie,
\par 35 Rekl: Budu te slyšeti, když tvoji žalobníci také prijdou. I rozkázal ho ostríhati v dome Herodesove.

\chapter{24}

\par 1 Po peti pak dnech sstoupil nejvyšší knez Ananiáš s staršími a s nejakým Tertullem recníkem; kterížto postavili se pred vladarem proti Pavlovi.
\par 2 A když povolán byl, pocal nan žalovati Tertullus, rka:
\par 3 Kterak mnohý pokoj zpusoben jest nám skrze tebe a mnohé veci v tomto národu výborne se dejí skrze tvou opatrnost, to my i všelijak i všudy se vším dekováním vyznáváme, výborný Felix.
\par 4 Ale abych te déle nezamestnával, prosím, vyslyšiž nás malicko podle obyceje prívetivosti své.
\par 5 Nalezli jsme zajisté cloveka tohoto nešlechetného, a vzbuzujícího ruznice mezi všemi Židy po všem svete, a vudci té sekty nazaretské;
\par 6 Jenž také i o to se pokoušel, aby chrámu poskvrnil; a kteréhožto javše, vedle Zákona našeho chteli jsme souditi.
\par 7 Ale prišed k tomu hejtman Lyziáš s mocí velikou, vzal ho z rukou našich,
\par 8 Rozkázav, aby žalobníci jeho šli k tobe. Od nehožto ty sám budeš moci, vyptaje se, zvedeti o všem o tom, z ceho my jej viníme.
\par 9 A k tomu se primluvili i Židé, pravíce, že to tak jest.
\par 10 Tedy Pavel odpovedel, když mu náveští dal vladar, aby mluvil: Od mnohých let veda tebe býti soudcím národu tomuto predstaveným, s lepší myslí k tomu, což se mne dotýce, odpovídati budu,
\par 11 Ponevadž ty mužeš to vedeti, že není tomu dní více než dvanácte, jakž jsem prišel do Jeruzaléma, abych se modlil.
\par 12 A aniž jsou mne nalezli v chráme s nekým se hádajícího, aneb cinícího roty v zástupu, ani v školách, ani v meste,
\par 13 Aniž toho prokázati mohou, což na mne žalují.
\par 14 Ale totot já pred tebou vyznávám, že podle té cesty, kterouž oni nazývají kacírstvím, tak sloužím Bohu otcu svých, vere všemu, cožkoli psáno jest v Zákone a v Prorocích,
\par 15 Maje nadeji v Bohu, že bude, jehož i oni cekají, vzkríšení z mrtvých, i spravedlivých i nespravedlivých.
\par 16 A tak se chovati hledím, abych mel dobré svedomí bez úrazu pred Bohem i pred lidmi vždycky
\par 17 Po mnohých pak letech prišel jsem, almužny nesa národu svému a obeti.
\par 18 Pri cemž mne nalezli v chráme ocišteného, ne s zástupem, ani s bourkou, nekterí Židé z Azie,
\par 19 Kteríž by meli tuto také pred tebou státi a žalovati, meli-li by co proti mne.
\par 20 Anebo nechat tito sami povedí, nalezli-li jsou na mne jakou nepravost, když jsem stál v rade,
\par 21 Lec to jedno promluvení, že jsem zavolal, stoje mezi nimi: Pro vzkríšení z mrtvých já k soudu potažen jsem dnes od vás.
\par 22 A vyslyšav to Felix, odložil jim, až by o té ceste neco místnejšího vyzvedel, rka: Až hejtman Lyziáš sem prijede, posoudím té pre vaší.
\par 23 I porucil setníkovi, aby Pavla ostríhal a polehcil mu vezení a nebránil žádnému z prátel jeho posluhovati jemu anebo navštevovati ho.
\par 24 Po nekolika pak dnech prišed Felix s Druzillou, manželkou svou, kteráž byla Židovka, zavolal Pavla, a slyšel od neho o víre v Krista.
\par 25 A když on vypravoval o spravedlnosti a o zdrželivosti a o budoucím soudu, ulekl se Felix, a odpovedel: Nyní odejdi, a v cas príhodný zavolám te.
\par 26 Nadál se pak, že jemu Pavel dá nejaké peníze, aby jej propustil, procež i tím casteji, povolávaje ho, mluvíval s ním.
\par 27 Po dvou pak letech mel po sobe námestka Felix, Festa Porcia, a chteje se zalíbiti Židum Felix, nechal Pavla v vezení.

\chapter{25}

\par 1 Tedy Festus vladarství ujav, po trech dnech prijel z Cesaree do Jeruzaléma.
\par 2 I oznámili jemu nejvyšší knez a prednejší z Židu o Pavlovi, a prosili ho,
\par 3 Žádajíce té milosti proti nemu, aby jej kázal privésti do Jeruzaléma, zálohy ucinivše jemu, aby zabit byl na ceste.
\par 4 A Festus odpovedel, že má Pavel ostríhán býti v Cesaree, a on sám že tudíž tam prijede.
\par 5 Protož (rekl), kteríž z vás mohou, necht tam také se vypraví spolu se mnou, a jest-li jaká vina na tom muži, nechat nan žalují.
\par 6 A pobyv mezi nimi nic více než deset dní, jel do Cesaree. A druhého dne posadiv se na soudné stolici, kázal Pavla privésti.
\par 7 Kterýžto když byl priveden, obstoupili jej ti, jenž byli prišli z Jeruzaléma, Židé, mnohé a težké žaloby proti Pavlovi vedouce, kterýchž nemohli dovésti,
\par 8 Nebo Pavel pri všem mírnou zprávu dával, že ani proti Zákonu Židovskému, ani proti chrámu, ani proti císari nic neprovinil.
\par 9 Ale Festus, chteje se Židum zalíbiti, odpovedev, rekl Pavlovi: Chceš-li jíti do Jeruzaléma a tam o to souzen býti prede mnou?
\par 10 I rekl Pavel: Pred stolicí císarovou chci státi a tam souzen býti. Židum jsem nic neublížil, jakož i ty dobre to víš.
\par 11 Nebo jestližet jim v cem ubližuji, aneb neco smrti hodného jsem spáchal, neodpírámt umríti; a paklit nic toho pri mne není, z cehož mne oni viní, žádnýt mne jim nemuže dáti. K císari se odvolávám.
\par 12 Tedy Festus promluviv s radou, odpovedel: K císaris se odvolal? K císari pujdeš.
\par 13 A po nekolika dnech král Agrippa a Bernice prijeli do Cesaree, aby pozdravili Festa.
\par 14 A když tu za mnoho dní pobyli, oznámil Festus králi o Pavlove pri, rka: Muž jeden ostaven jest od Felixa v vezení,
\par 15 O kterémž, když jsem byl v Jeruzaléme, oznámili mi prední kneží a starší Židovští, žádajíce na nej ortele.
\par 16 Kterýmž jsem odpovedel, že není obycej Rímanum vydati cloveka na smrt, prve nežli by ten, na kohož se žaloba deje, prítomné mel žalobníky a volnost k odpovídání na to, z cehož by byl obvinován.
\par 17 A protož když se byli sem sešli, hned beze všeho meškání, druhý den posadiv se na soudné stolici, rozkázal jsem privésti toho muže.
\par 18 Jehožto žalobníci tu stojíce, z niceho takového nevinili ho, cehož jsem já se domníval.
\par 19 Ale o nejaké otázky pri tom svém náboženství meli s ním nesnáz, a o jakémsi Ježíšovi mrtvém, o kterémž jistil Pavel, že jest živ.
\par 20 Já pak maje tu pri v pochybnosti, rekl jsem jemu, chtel-li by jíti do Jeruzaléma, a tam o ty veci souzen býti.
\par 21 A když se on odvolal, aby byl chován k soudu Augustovu, kázal jsem ho hlídati, až bych jej poslal k císari.
\par 22 Tedy Agrippa rekl k Festovi: Chtelt bych i já rád cloveka toho slyšeti. A on rekl: Zítra ho uslyšíš.
\par 23 Nazejtrí pak, když prišel Agrippa a Bernice s velikou slavou, a vešli na sín s hejtmany a s lidmi nejznamenitejšími mesta toho, k rozkázání Festovu priveden jest Pavel.
\par 24 I rekl Festus: Králi Agrippo a všickni, kteríž jste tuto s námi, vidíte tohoto, za nejžto všecko množství Židu prosili mne, i v Jeruzaléme i zde, kricíce, že takový nemá více živ býti.
\par 25 Já pak shledav to, že nic hodného smrti neucinil, však když se sám k Augustovi odvolal, umínil jsem jej tam poslati.
\par 26 O nemž, co bych jistého napsal pánu svému, nevím. Protož jsem jej ted privedl pred vás, a zvlášte pred tebe, králi Agrippo, abych vyptaje se, vedel, co psáti.
\par 27 Nebo zdá mi se to neslušné býti poslati vezne a pre jeho neoznámiti.

\chapter{26}

\par 1 Potom Agrippa rekl Pavlovi: Dopouštít se, abys sám za sebe promluvil. Tedy Pavel vztáh ruku, mluvil:
\par 2 Na všecky ty veci, z kterýchž mne viní Židé, králi Agrippo, za štastného se pocítám, že dnes pred tebou odpovídati mám,
\par 3 A zvlášte proto, ponevadž jsi ty povedom všech tech, kteríž u Židu jsou, obyceju a otázek. Protož prosím tebe, vyslyšiž mne trpelive.
\par 4 O životu mém hned od mladosti mé, jaký byl od pocátku v národu mém v Jeruzaléme, vedí všickni Židé,
\par 5 Mevše mne prve zdávna v dobré známosti, (kdyby chteli svedectví vydati,) kterak vedle nejjistší sekty v našem náboženství byl jsem živ farizeus.
\par 6 A nyní pro nadeji toho zaslíbení, kteréž se stalo otcum našim od Boha, ted stojím, soudu jsa poddán,
\par 7 K kterémužto zaslíbení dvanáctero pokolení naše, sloužíce Bohu ustavicne dnem i nocí, nadeji má, že prijde; pro kteroužto nadeji žalují na mne Židé, ó králi Agrippo.
\par 8 A tak-liž se to od vás za nepodobné k víre soudí, že Buh krísí mrtvé?
\par 9 Act i mne se zdálo, že by mi náležité bylo, proti jménu Ježíše Nazaretského mnoho odporného ciniti,
\par 10 Jakož jsem i cinil v Jeruzaléme, a mnohé z svatých já jsem do žaláru dával, vzav moc od predních kneží. A když meli mordováni býti, já jsem pomáhal ortele vynášeti.
\par 11 A po všech školách casto trápe je, prinucoval jsem, aby se rouhali; a náramne rozlítiv se na ne, protivil jsem se jim, až i do jiných mest na to jsem jezdil.
\par 12 A v tom, když jsem šel do Damašku, s mocí a s porucením predních kneží,
\par 13 O poledni na ceste, ó králi, uzrel jsem, ano svetlo s nebe, jasnejší nad blesk slunecný, obklícilo mne, i ty, kteríž se mnou šli.
\par 14 A když jsme my všickni na zem padli, slyšel jsem hlas mluvící ke mne a rkoucí Židovským jazykem: Saule, Saule, proc mi se protivíš? Tvrdot jest tobe proti ostnum se zpecovati.
\par 15 A já rekl: I kdo jsi, Pane? A on rekl: Já jsem Ježíš, kterémuž ty se protivíš.
\par 16 Ale vstan a stuj na nohách svých; nebo protot jsem se tobe ukázal, at bych tebe ucinil služebníkem a svedkem i tech vecí, kteréž jsi videl, i tech, v kterýchžto ukazovati se budu tobe,
\par 17 Vysvobozuje tebe z lidu tohoto, i z pohanu, k nimž te nyní posílám,
\par 18 Otvírati oci jejich, aby se obrátili od temností k svetlu a z moci dábelské k Bohu, aby tak hríchu odpuštení a díl s posvecenými vzali skrze víru, kteráž jest ve mne.
\par 19 A protož, ó králi Agrippo, nebyl jsem neverící nebeskému videní.
\par 20 Ale hned nejprv tem, kteríž jsou v Damašku a v Jeruzaléme, i po vší krajine Judské, potom i pohanum zvestoval jsem evangelium, aby pokání cinili a obrátili se k Bohu, skutky hodné pokání ciníce.
\par 21 A pro tu prícinu Židé javše mne v chráme, pokoušeli se rukama svýma zamordovati.
\par 22 Ale s pomocí Boží ješte až do dnešního dne stojím, vydávaje svedectví i malému i velikému, nic jiného nevypravuje, než to, což jsou Proroci a Mojžíš zvestovali, že se melo státi:
\par 23 Totiž, že mel Kristus trpeti, a z mrtvých vstana první, zvestovati svetlo lidu tomuto i pohanum.
\par 24 To když od sebe promluvil Pavel, Festus hlasem velikým rekl: Blázníš, Pavle. Mnohé tvé umení k bláznovství tebe privodí.
\par 25 On pak rekl: Nebláznímt, výborný Feste, ale slova pravdy a stredmosti mluvím.
\par 26 Vít zajisté o tech vecech král, pred kterýmž smele a svobodne mluvím; nebo mám za to, žet ho nic z techto vecí tajno není, ponevadž se toto nedálo pokoutne.
\par 27 Veríš-li, králi Agrippo, Prorokum? Vím, že veríš.
\par 28 Tedy Agrippa rekl Pavlovi: Témer bys mne k tomu naklonil, abych byl krestanem.
\par 29 A Pavel rekl: Žádalt bych od Boha, abyste i ponekud i z cela, netoliko ty, ale všickni, kteríž slyší mne dnes, byli takoví, jakýž já jsem, krome okovu techto.
\par 30 A když to Pavel promluvil, vstal král, i vladar a Bernice, i ti, kteríž s nimi sedeli.
\par 31 A odšedše na stranu, mluvili spolu, rkouce: Nic hodného smrti neb vezení neciní clovek tento.
\par 32 Agrippa pak Festovi rekl: Mohl propušten býti clovek tento, kdyby se byl neodvolal k císari.

\chapter{27}

\par 1 A když bylo již usouzeno, abychom my se plavili do Vlach, porucen jest i Pavel i nekterí jiní veznové setníku, jménem Juliovi, kterýž byl nad houfem Augustovým.
\par 2 Tedy všedše na lodí Adramyttenskou, abychom se plavili podle krajin Azie, pustili jsme se na more. A byl s námi Aristarchus Macedonský z Tessaloniky.
\par 3 Druhý pak den priplavili jsme se k Sidonu. A tu Julius prívetive se maje ku Pavlovi, dopustil mu, aby jda k prátelum, u nich mel pohodlí.
\par 4 A berouce se odtud, plavili jsme se podle Cypru, protože byl vítr odporný nám.
\par 5 A tak preplavivše se pres more Cilické a Pamfylické, prišli jsme do mesta Myry, kteréž jest v krajine, jenž slove Lycia.
\par 6 A tu našed setník bárku Alexandrinskou, kteráž mela plouti do Vlach, uvedl nás na ni.
\par 7 A když jsme za mnoho dní znenáhla se plavili a sotva priplouli proti Gnidum, a vítr nám bránil priblížiti se k zemi, i podjeli jsme Krétu podle Salmóny.
\par 8 A sotva ji pomíjeti mohše, prijeli jsme na jedno místo, kteréž slove Pekný breh, od kteréhožto nedaleko bylo mesto Lasea.
\par 9 Když pak drahný cas prešel, a již bylo nebezpecné plavení, (neb již byl i pust pominul,) napomínal jich Pavel,
\par 10 Rka k nim: Muži, vidím, že s velikým ublížením a s mnohou škodou netoliko nákladu a lodí, ale i životu našich toto plavení bude.
\par 11 Ale setník více veril správci lodí a marinári, nežli tomu, co Pavel pravil.
\par 12 A když nebylo tu príhodného prístavu, kdež by pobyli pres zimu, mnozí tak radili, aby se predce pustili odtud, zda by jak mohli, prijedouce do Fenicen, pres zimu tu pobýti na brehu Kréty, kterýž leží k vetru nešpornímu a k vetru více než západnímu.
\par 13 A když pocal vítr víti od poledne, majíce za to, že se budou umínenou cestou držeti, i nahodilo se jim, že jeli blízko Kréty.
\par 14 Po neveliké pak chvíli zdvihl se proti nim vítr bourlivý pulnocní, kterýž slove Euroklydon.
\par 15 A když lodí zachvácena byla a nemohla odolati proti vetru, pustivše ji po vetru, tak jsme se vezli.
\par 16 A pribehše pod jeden ostrov neveliký, kterýž slove Klauda, sotva jsme mohli obdržeti clun u bárky.
\par 17 Kterýžto zdvihše, pomoci užívali, podpásavše bárku; a bojíce se, aby neuhodili na místo nebezpecné, spustivše clun, tak se plavili.
\par 18 A když boure vichrová námi velmi zmítala, na druhý den, což bylo v lodí nákladu, metali ven.
\par 19 A tretí den i to nádobí bárce potrebné svýma rukama vyházeli jsme.
\par 20 A když ani slunce se neukázalo ani hvezdy za mnoho dní, a boure vždy vetší nastávala, již byla všecka nadeje o vysvobození našem odjata.
\par 21 A když jsme již byli hladem velmi ztrápeni, tedy stoje Pavel uprostred nich, rekl: Meli jste zajisté, ó muži, uposlechnouce mne, nehýbati se od Kréty, a tak uvarovati se nebezpecenství tohoto a škody
\par 22 A i nyní vás napomínám, abyste dobré mysli byli; nebot nezahyne žádný z vás, krome bárky samé.
\par 23 Nebo této noci ukázal mi se andel Boha toho, jehož já jsem a kterémuž sloužím,
\par 24 Rka: Neboj se, Pavle, pred císarem máš státi, a aj, dalt jest tobe Buh všecky, kteríž se plaví s tebou.
\par 25 Protož budte dobré mysli, muži; nebot já verím Bohu, žet se tak stane, jakž jest mi mluveno.
\par 26 Mámet se pak dostati na nejaký ostrov.
\par 27 A když již byla ctrnáctá noc, a my se plavili po mori Adriatickém, okolo pulnoci, domnívali se plavci, že by se jim okazovala krajina nejaká.
\par 28 Kterížto spustivše do vody olovnici, nalezli hlubokost dvadcíti loktu; a odjevše odtud malicko, opet spustivše olovnici, nalezli hlubokost patnácti loktu.
\par 29 A bojíce se, aby na místa skalnatá neuhodili, spustivše z bárky ctyri kotve, žádali, aby den byl.
\par 30 Chteli pak marinári utéci z bárky, pustivše clun do more, pod zámyslem, jako by chteli od predku lodí kotve roztahovati,
\par 31 I rekl Pavel setníkovi a žoldnérum: Nezustanou-li tito na lodí, vy nebudete moci zachováni býti.
\par 32 Tedy žoldnéri utínali provazy u clunu, a pustili jej, aby pryc plynul.
\par 33 A když již dnelo, napomínal Pavel všech, aby pojedli, rka: Již jest tomu dnes ctrnáctý den, jakž ocekávajíce, trváte lacní, nic nejedouce.
\par 34 Protož prosím vás, abyste pojedli pro zachování vašeho zdraví; neb žádného z vás vlas s hlavy nespadne.
\par 35 A to povedev, vezma chléb, díky vzdával Bohu prede všemi, a rozlomiv, pocal jísti.
\par 36 A tak potešeni byvše všickni, pojedli i oni.
\par 37 Bylo pak nás všech osob na lodí dve ste sedmdesáte a šest.
\par 38 A nasyceni jsouce pokrmem, oblehcovali bárku, vysýpajíce pšenici do more.
\par 39 A když byl den, nemohli zeme videti, než okrídlí nejaké znamenali, ano má breh, k nemuž myslili, kdyby jak mohli pristáti s lodí.
\par 40 A vytáhše kotvy, pustili se po mori, rozpustivše také provazy pravidl; a zdvihše plachtu k vetru, táhli se k brehu.
\par 41 Ale když trefili na to místo, kdež se dvoje more schází, tu se zastavila lodí. A prední konec lodí uváznutý stál, nehýbaje se, zadní pak konec lámal se násilím vln.
\par 42 Tedy žoldnéri radili setníkovi, aby vezne zmordovali, aby jim nekterý vyplyna, neutekl.
\par 43 Ale setník chteje zachovati Pavla, nedal toho uciniti. I rozkázal tem, kteríž mohli plynouti, aby se pustili nejprv do more a vyplynuli na zem,
\par 44 Jiní pak aby na dskách plynuli a nekterí na tech kusích lodí. I takž se stalo, že všickni zdraví vyšli na zemi.

\chapter{28}

\par 1 A tak zachováni jsouce, teprv poznali, že ostrov ten sloul Melita.
\par 2 Lidé pak toho ostrova velikou prívetivost k nám ukázali. Nebo zapálivše hranici drev, prijali nás všecky, pro déšt, kterýž v tu chvíli byl, a pro zimu.
\par 3 A když Pavel sebral drahne roždí a kladl na ohen, ješterka, utíkajíc pred horkem, pripjala se k ruce jeho.
\par 4 A když uzreli lidé toho ostrovu, ano ješterka visí u ruky jeho, rekli jedni k druhým: Jiste clovek tento jest vražedlník; neb ac z more vyšel, však pomsta Boží nedá jemu živu býti.
\par 5 Ale on strásl tu ješterku do ohne, a nic se jemu zlého nestalo.
\par 6 Oni pak ocekávali, že otece, aneb padna, pojednou umre. A když toho dlouho cekali a videli, že se mu nic zlého nestalo, zmenivše myšlení své, pravili, že jest on buh.
\par 7 Na tech pak místech mel popluží kníže toho ostrova, jménem Publius, kterýžto prijav nás k sobe, za tri dni prátelsky u sebe v hospode choval.
\par 8 I prihodilo se, že otec toho Publia ležel, maje zimnici a cervenou nemoc. K nemuž všed Pavel, a pomodliv se, vzložil na nej ruce a uzdravil jej.
\par 9 A když se to stalo, tedy i jiní, kteríž na ostrove tom nemocní byli, pristupovali, a byli uzdravováni.
\par 10 Ctili nás pak velice, a když jsme se meli pryc plaviti, nakladli nám na lodí toho, cehož potrebí bylo.
\par 11 A po trech mesících plavili jsme se na bárce Alexandrinské, kteráž tu byla na tom ostrove pres zimu, majici za erb Kastora a Polluxa.
\par 12 A když jsme prijeli do Syrakusis, zustali jsme tu za tri dni.
\par 13 A odtud okolo plavíce se, prišli jsme do Regium; a po jednom dni, když vál vítr od poledne, druhý den prijeli jsme do Puteolos.
\par 14 Kdežto nalezli jsme bratrí, kteríž nás prosili, abychom pobyli u nich za sedm dní. A tak jsme šli k Rímu.
\par 15 Odkudž, když o nás uslyšeli bratrí, vyšli proti nám až na rynk Appiuv a ke Trem krcmám. Kteréžto uzrev Pavel, podekoval Bohu a pocal býti dobré mysli.
\par 16 A když jsme prišli do Ríma, setník dal vezne v moc hejtmanu vojska, ale Pavlovi dopušteno, aby sám bydlil s žoldnérem, kterýž ho ostríhal.
\par 17 I stalo se po trech dnech, svolal Pavel muže prední z Židu. A když se sešli, rekl k nim: Muži bratrí, já nic neuciniv proti lidu ani proti obycejum otcovským, jat jsa, z Jeruzaléma vydán jsem v ruce Rímanum,
\par 18 Kteríž vyslyševše mne, chteli mne propustiti, protože žádné viny hodné smrti na mne nebylo.
\par 19 Ale když Židé tomu odpírali, prinucen jsem odvolati se k císari, ne jako bych mel co národ svuj žalovati.
\par 20 A protož z té príciny povolal jsem vás, žádostiv jsa videti vás a s vámi promluviti; nebo pro nadeji lidu Izraelského retezem tímto svázán jsem.
\par 21 A oni rekli jemu: Myt jsme žádného psaní nemeli o tobe z Židovstva, aniž kdo z bratrí prijda, vypravoval nám, aneb mluvil co zlého o tobe.
\par 22 Ale žádámet od tebe slyšeti, jak smyslíš; nebo víme o té sekte, že se jí všudy odpírá.
\par 23 A když jemu uložili den, sešlo se jich mnoho do hospody k nemu, jimžto s osvedcováním vypravoval o království Božím, slouže jim k víre o Ježíšovi z Zákona Mojžíšova a Proroku, od jitra až do vecera.
\par 24 A nekterí uverili tomu, což Pavel vypravoval, nekterí pak neverili.
\par 25 A tak nejsouce mezi sebou svorní, rozešli se, k nimžto promluvil Pavel toto jedno slovo: Jiste žet jest dobre Duch svatý skrze proroka Izaiáše mluvil k otcum našim,
\par 26 Rka: Jdi k lidu tomuto a rciž jim: Sluchem slyšeti budete, a nesrozumíte, a hledíce hledeti budete, a neuzríte.
\par 27 Nebo zhrublo srdce lidu tohoto, a ušima težce slyšeli, a oci své zamhourili, aby snad nevideli ocima, a ušima neslyšeli, a srdcem nerozumeli, a neobrátili se, abych jich neuzdravil.
\par 28 Budiž vám tedy známo, že jest pohanum posláno toto spasení Boží, a onit slyšeti budou.
\par 29 A když on to propovedel, odešli Židé, majíce mezi sebou mnohé hádky.
\par 30 Pavel pak trval za celé dve léte v hospode své, a prijímal všecky, kteríž pricházeli k nemu,
\par 31 Káže o království Božím a uce tem vecem, kteréž jsou o Pánu Ježíši Kristu, se vší doufanlivostí bez prekážky.


\end{document}