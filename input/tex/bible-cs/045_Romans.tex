\begin{document}

\title{Romans}

\chapter{1}

\par 1 Pavel, služebník Jezukristuv, povolaný apoštol, oddelený k kázaní evangelium Božího,
\par 2 (Kteréžto zdávna zaslíbil skrze proroky své v Písmích svatých,)
\par 3 O Synu jeho, (zplozeném z semene Davidova s strany tela,
\par 4 Kterýž prokázán jest býti Synem Božím mocne, podle Ducha posvecení, skrze z mrtvých vstání,) totiž o Ježíši Kristu, Pánu našem,
\par 5 Skrze nehožto jsme prijali milost a apoštolství ku poslušenství víry mezi všemi národy pro jméno jeho,
\par 6 Z nichžto i vy jste povolaní Ježíše Krista,
\par 7 Všechnem, kteríž jste v Ríme, milým Božím, povolaným svatým: Milost vám a pokoj od Boha Otce našeho a Pána Ježíše Krista.
\par 8 Nejprve pak díky ciním Bohu svému skrze Ježíše Krista ze všech vás, že víra vaše rozhlašuje se po všem svete.
\par 9 Svedek mi jest zajisté Buh, kterémužto sloužím duchem mým v evangelium Syna jeho, žet bez prestání zmínku o vás ciním,
\par 10 Vždycky na svých modlitbách žádaje, abych aspon nekdy mohl štastne k vám, byla-li by vule Boží, prijíti.
\par 11 Nebot velice žádám videti vás, abych vám udelil cástku nejakou milosti duchovní ku potvrzení vašemu,
\par 12 To jest, abych spolu s vámi potešen byl, skrze spolecnou i vaši i mou víru.
\par 13 Nechcit pak, bratrí, abyste nevedeli, žet jsem mnohokrát již uložil prijíti k vám, (ale prekážky jsem mel až dosavad,) abych nejaký užitek také i mezi vámi mel, jako i mezi jinými národy.
\par 14 Nebo Rekum i kterýmkoli jiným národum, i moudrým i nemoudrým, dlužník jsem.
\par 15 A tak, pokudž na mne jest, hotov jsem i vám, kteríž v Ríme jste, zvestovati evangelium.
\par 16 Nebot se nestydím za evangelium Kristovo; moc zajisté Boží jest k spasení každému verícímu, Židu predne, potom i Reku.
\par 17 Nebo spravedlnost Boží zjevuje se skrze ne z víry u víru, jakož psáno jest: Spravedlivý pak z víry živ bude.
\par 18 Zjevuje se zajisté hnev Boží s nebe proti každé bezbožnosti a nepravosti lidí, pravdu Boží v nepravosti zadržujících.
\par 19 Nebo což poznáno býti muže o Bohu, známé jest jim ucineno, Buh zajisté zjevil jim.
\par 20 Nebo neviditelné veci jeho mohou vidíny býti, když z skutku pri stvorení sveta stalých rozumem pochopeny bývají, totiž ta jeho vecná moc a Božství, tak aby již byli bez výmluvy,
\par 21 Protože poznavše Boha, nectili jako Boha, ani jemu dekovali, ale marní ucineni jsou v myšleních svých, a zatmíno jest nemoudré srdce jejich.
\par 22 Mevše se za moudré, blázni ucineni jsou.
\par 23 Nebo smenili slávu neporušitelného Boha v slávu podobenství obraza porušitelného cloveka, i ptactva, i hovad ctvernohých, i zemeplazu.
\par 24 Protož i Buh vydal je v žádosti srdce jejich, v necistotu, aby zprznili tela svá vespolek,
\par 25 Jakožto ty, kteríž jsou smenili pravdu Boží za lež, a ctili i sloužili stvorení radeji nežli Stvoriteli, kterýž jest požehnaný na veky. Amen.
\par 26 Protož je vydal Buh v žádosti ohavné. Nebo i ženy jejich zmenily prirozené sebe užívání v to, kteréž jest proti prirození.
\par 27 A podobne i mužské pohlaví, opustivše prirozené užívání ženy, rozpálili se v žádostech svých jedni k druhým, mužské pohlaví vespolek mrzkost pášíce, a tak spravedlivou mzdu, kteráž na jejich blud slušela, sami na sebe uvodíce.
\par 28 A jakož sobe málo vážili známosti Boha, takž také Buh vydal je v prevrácený smysl, aby cinili to, což nesluší,
\par 29 Jsouce naplneni vší nepravostí, zlostí, smilstvem, nešlechetností, lakomstvím, plní závisti, vraždy, sváru, lsti, zlých obyceju,
\par 30 Utrhaci, pomluvaci, Boha nenávidící, hanliví, pyšní, chlubní, nalezaci zlých vecí, rodicu neposlušní,
\par 31 Nemoudrí, smluv nezdrželiví, beze vší lítosti, neukojitelní a nemilosrdní.
\par 32 Kterížto vedouce o tom právu Božím, že ti, kteríž takové veci ciní, hodni jsou smrti, avšak netoliko ty veci ciní, ale i jiným též cinícím nakládají.

\chapter{2}

\par 1 Protož nemužeš se vymluviti, ó clovece každý, potupuje jiného. Nebo tím, že jiného potupuješ, sám sebe odsuzuješ, ponevadž totéž ciníš, což na jiném tupíš.
\par 2 Vímet zajisté, že soud Boží jest podle pravdy proti tem, kteríž takové veci ciní.
\par 3 Zdali se domníváš, ó clovece, jenž soudíš ty, kdož takové veci ciní, a sám totéž cine, že ty ujdeš soudu Božího?
\par 4 Cili bohatstvím dobrotivosti jeho a snášelivosti i dlouhocekání pohrdáš, neveda, že dobrotivost Boží ku pokání tebe vede?
\par 5 Ale podle tvrdosti své a srdce nekajícího shromažduješ sobe hnev ke dni hnevu a zjevení spravedlivého soudu Božího,
\par 6 Kterýž odplatí jednomu každému podle skutku jeho,
\par 7 Tem zajisté, kteríž trvajíce v dobrém skutku, slávy a cti a nesmrtelnosti hledají, životem vecným,
\par 8 Tem pak, kteríž jsou svárliví a pravde nepovolují, ale povolují nepravosti, prchlivostí a hnevem,
\par 9 Trápením a úzkostí, a to každé duši cloveka cinícího zlé, i Žida predne, a též i Reka.
\par 10 Ale slávu a cest a pokoj dá každému, kdož ciní dobré, i Židu predne, a též i Reku.
\par 11 Nebot není prijímání osob u Boha.
\par 12 Kterížkoli zajisté bez Zákona hrešili, bez Zákona i zahynou; a kterížkoli pod Zákonem byvše hrešili, skrze Zákon odsouzeni budou,
\par 13 (Nebo ne ti, jenž slyší Zákon, spravedlivi jsou pred Bohem, ale cinitelé Zákona spravedlivi budou.
\par 14 Nebo když pohané Zákona nemajíce, od prirození ciní to, což prikazuje Zákon, takoví Zákona nemajíce, sami sobe zákonem jsou,
\par 15 Kterížto ukazují dílo Zákona napsané na srdcích svých, když jim to osvedcuje svedomí jejich i myšlení, kteráž se vespolek obvinují, anebo také vymlouvají.)
\par 16 V ten den, kdyžto souditi bude Buh tajné veci lidské, podle evangelium mého skrze Jezukrista.
\par 17 Aj, ty sloveš Žid, a spoléháš na Zákon, a chlubíš se Bohem,
\par 18 A znáš vuli jeho, a rozeznáváš, co sluší neb nesluší, naucen jsa z Zákona,
\par 19 A za to máš, že jsi ty vudcím slepých, svetlem tech, kteríž jsou ve tme,
\par 20 Reditelem nemoudrých, ucitelem nemluvnat, majícím formu umení a pravdy v Zákone.
\par 21 Kterakž tedy jiného uce, sám sebe neucíš? Vyhlašuje, že nemá kradeno býti, sám kradeš?
\par 22 Prave: Nezcizoložíš, a cizoložství pácháš? V ohavnosti maje modly, svatokrádeže se dopouštíš?
\par 23 Zákonem se chlube, prestupováním Zákona Bohu neúctu ciníš?
\par 24 Nebo jméno Boží pro vás v porouhání jest mezi pohany, jakož psáno jest.
\par 25 Obrezánít zajisté prospeje, budeš-li Zákon plniti; pakli budeš prestupitelem Zákona, obrezání tvé ucineno jest neobrezáním.
\par 26 A protož jestližet by neobrízka ostríhala práv Zákona, zdaliž nebude poctena neobrízka jejich za obrízku?
\par 27 A odsoudí ti, kteríž jsou z prirození neobrízka, zachovávajíce Zákon, tebe, kterýž pod literou a obrízkou prestupník jsi Zákona?
\par 28 Nebo ne ten jest pravý Žid, kterýž jest zjevne Židem; aniž to jest pravé obrezání, kteréž bývá zjevne na tele;
\par 29 Ale ten jest pravý Žid, kterýžto vnitr jest Židem, a to jest pravé obrezání, kteréž jest srdecné v duchu, a ne podle litery; jehožto chvála ne z lidí jest, ale z Boha.

\chapter{3}

\par 1 Což tedy má více Žid nežli pohan? Aneb jaký jest užitek obrízky?
\par 2 Mnohý všelikterak. Prední zajisté ten, že jest jim sveren Zákon Boží.
\par 3 Nebo což jest do toho, jestliže byli nekterí z nich neverní? Zdaliž nevera jejich vernost Boží vyprázdní?
\par 4 Nikoli, nýbrž budiž Buh pravdomluvný, ale každý clovek lhár, jakož psáno jest: Aby ospravedlnen byl v recech svých, a premohl, když by soudil.
\par 5 Ale jestližet pak nepravost naše spravedlnost Boží zvelebuje, což díme? Zdali nespravedlivý jest Buh, jenž uvodí hnev? (Po lidskut pravím.)
\par 6 Nikoli, sic jinak kterakž by Buh soudil svet?
\par 7 Nebo jestližet pravda Boží mou lží rozmohla se k sláve jeho, i procež pak já jako hríšník bývám souzen?
\par 8 A ne radeji (jakž o nás zle mluví a jakož nekterí praví, že bychom ríkali,): Cinme zlé veci, aby prišly dobré? Jichžto spravedlivé jest odsouzení.
\par 9 Což tedy? My prevyšujeme pohany? Nikoli, nebo jsme již prve dokázali toho, že jsou, i Židé i Rekové, všickni pod hríchem,
\par 10 Jakož psáno jest: Že není spravedlivého ani jednoho.
\par 11 Není rozumného, není, kdo by hledal Boha.
\par 12 Všickni se uchýlili, spolu neužitecní ucineni jsou; není, kdo by cinil dobré, není ani jednoho.
\par 13 Hrob otevrený hrdlo jejich, jazyky svými lstive mluvili, jed lítých hadu pod rty jejich.
\par 14 Kterýchžto ústa plná jsou zlorecení a horkosti.
\par 15 Nohy jejich rychlé k vylévání krve.
\par 16 Setrení a bída na cestách jejich.
\par 17 A cesty pokoje nepoznali.
\par 18 Není bázne Boží pred ocima jejich.
\par 19 Víme pak, že cožkoli Zákon mluví, tem, kteríž jsou pod Zákonem, mluví, aby všeliká ústa zacpána byla a aby vinen byl všecken svet Bohu.
\par 20 Protož z skutku Zákona nebude ospravedlnen žádný clovek pred oblicejem jeho; nebo skrze Zákon prichází poznání hrícha.
\par 21 Ale nyní bez Zákona spravedlnost Boží zjevena jest, osvedcená Zákonem i Proroky,
\par 22 Spravedlnost totiž Boží, skrze víru Ježíše Krista, ke všem a na všecky verící.
\par 23 Nebot není rozdílu. Všicknit zajisté zhrešili, a nemají slávy Boží.
\par 24 Spravedlivi pak ucineni bývají darmo, milostí jeho, skrze vykoupení, kteréž se stalo v Kristu Ježíši,
\par 25 Jehožto Buh vydal za smírci, skrze víru ve krvi jeho, k ukázání spravedlnosti své, skrze odpuštení predešlých hríchu,
\par 26 V shovívání Božím, k dokázání spravedlnosti své v nynejším casu, k tomu, aby on spravedlivým byl a ospravedlnujícím toho, jenž jest z víry Ježíšovy.
\par 27 Kdež jest tedy chlouba tvá? Vyprázdnena jest. Skrze který zákon? Skutku-li? Nikoli, ale skrze zákon víry.
\par 28 Protož za to máme, že clovek bývá spravedliv ucinen verou bez skutku Zákona.
\par 29 Zdaliž jest toliko Buh Židu? Zdali také není i pohanu? Ba, jiste i pohanu,
\par 30 Ponevadž jeden jest Buh, kterýž ospravedlnuje obrízku z víry, a neobrízku skrze víru.
\par 31 Což tedy Zákon vyprazdnujeme skrze víru? Nikoli, nýbrž Zákon tvrdíme.

\chapter{4}

\par 1 Což tedy díme, ceho jest došel podle tela Abraham otec náš?
\par 2 Nebo byl-lit Abraham z skutku spravedliv ucinen, mát se cím chlubiti, ale ne u Boha.
\par 3 Nebo co praví Písmo? Uveril pak Abraham Bohu, i pocteno jemu za spravedlnost.
\par 4 Kdožt skutky ciní, tomut odplata nebývá poctena podle milosti, ale podle dluhu.
\par 5 Tomu pak, kdož neciní skutku, ale verí v toho, kterýž spravedlivého ciní bezbožníka, bývá poctena víra jeho za spravedlnost,
\par 6 Jakož i David vypravuje blahoslavenství cloveka, jemuž Buh privlastnuje spravedlnost bez skutku, rka:
\par 7 Blahoslavení, jichžto odpušteny jsou nepravosti a jejichžto prikryti jsou hríchové.
\par 8 Blahoslavený muž, kterémuž Pán nepocítá hríchu.
\par 9 Blahoslavenství tedy toto k obrízce-li se vztahuje toliko, cili také k neobrízce? Nebo pravíme, že Abrahamovi víra jest poctena za spravedlnost.
\par 10 Kterak pak jest poctena? Zdali když byl obrezán, cili pred obrezáním? Ne v obrízce, ale pred obrezáním.
\par 11 A potom znamení, totiž obrízku, prijal za znamení spravedlnosti víry, kteráž byla pred obrezáním, na to, aby byl otcem všech verících v neobrízce, aby i jim prictena byla spravedlnost,
\par 12 A aby byl otcem obrízky, tech totiž, kteríž ne z obrízky toliko jsou, ale kteríž krácejí šlépejemi víry otce našeho Abrahama, kteráž byla pred obrezáním.
\par 13 Nebo ne skrze Zákon stalo se zaslíbení Abrahamovi, aneb semeni jeho, aby byl dedicem sveta, ale skrze spravedlnost víry.
\par 14 Nebo jestližet toliko ti, kteríž jsou z Zákona, dedicové jsou, zmarena jest víra a zrušeno zaslíbení.
\par 15 Zákon zajisté hnev pusobí; nebo kdež není Zákona, tu ani prestoupení.
\par 16 A protož z víry jde dedictví, aby šlo podle milosti, proto aby pevné bylo zaslíbení všemu semeni, netoliko tomu, kteréž z Zákona jest, ale i tomu, jenž jest z víry Abrahamovy, kterýž jest otec všech nás,
\par 17 (Jakož psáno jest: Otcem mnohých národu postavil jsem tebe,) pred oblicejem Boha, jemuž uveril, kterýž obživuje mrtvé a povolává i tech vecí, jichž není, jako by byly.
\par 18 Kterýžto Abraham v nadeji proti nadeji uveril, že bude otcem mnohých národu, podle toho povedení: Takt bude síme tvé, jakožto hvezdy nebeské a jako písek morský.
\par 19 A nezemdlev u víre, nepatril na své telo již umrtvené, ješto témer ve stu letech byl, ani na život Sáry také již umrtvený.
\par 20 Ale o zaslíbení Božím nepochyboval z nedovery, nýbrž posilnil se verou, dav chválu Bohu,
\par 21 Jsa tím jist, že cožkoli zaslíbil, mocen jest i uciniti.
\par 22 A protož pocteno jest jemu to za spravedlnost.
\par 23 Jestit pak to napsáno ne pro nej toliko, že jemu to pocteno bylo za spravedlnost,
\par 24 Ale i pro nás, kterýmžto bude pocteno za spravedlnost, verícím v toho, kterýž vzkrísil Ježíše Pána našeho z mrtvých,
\par 25 Kterýž vydán jest na smrt pro hríchy naše a vstal z mrtvých pro ospravedlnení naše.

\chapter{5}

\par 1 Ospravedlneni tedy jsouce z víry, pokoj máme s Bohem skrze Pána našeho Jezukrista,
\par 2 Skrze nehož i prístup meli jsme verou k milosti této, kterouž stojíme. A chlubíme se nadejí slávy Boží.
\par 3 A ne toliko nadejí, ale také chlubíme se souženími, vedouce, že soužení trpelivost pusobí,
\par 4 A trpelivost zkušení, zkušení pak nadeji,
\par 5 A nadejet nezahanbuje. Nebo láska Boží rozlita jest v srdcích našich skrze Ducha svatého, kterýž dán jest nám.
\par 6 Kristus zajisté, když jsme my ješte mdlí byli, v cas príhodný za bezbožné umrel.
\par 7 Ješto sotva kdo za spravedlivého umre, ac za dobréhot by nekdo snad i umríti smel.
\par 8 Dokazujet pak Buh lásky své k nám; nebo když jsme ješte hríšníci byli, Kristus umrel za nás.
\par 9 Cím tedy více nyní již ospravedlneni jsouce krví jeho, spaseni budeme skrze neho od hnevu.
\par 10 Ponevadž byvše neprátelé, smíreni jsme s Bohem skrze smrt Syna jeho, nadtot již smírení spaseni budeme skrze život jeho.
\par 11 A ne toliko to, ale chlubíme se také i Bohem, skrze Pána našeho Jezukrista, skrze nehož nyní smírení jsme došli.
\par 12 A protož jakož skrze jednoho cloveka hrích na svet všel a skrze hrích smrt, a tak na všecky lidi smrt prišla, v nemž všickni zhrešili.
\par 13 Nebo až do Zákona hrích byl na svete, ale hrích se nepocítá, když Zákona není.
\par 14 Kralovala pak smrt od Adama až do Mojžíše také i nad temi, kteríž nehrešili ku podobenství prestoupení Adamova, kterýž jest figura toho budoucího Adama.
\par 15 Ale ne jako hrích, tak i milost. Nebo ponevadž onoho pádem jednoho mnoho jich zemrelo, mnohemt více milost Boží a dar z milosti toho jednoho cloveka Jezukrista na mnohé rozlit jest.
\par 16 Avšak ne jako skrze jednoho, kterýž zhrešil, tak zase milost. Nebo vina z jednoho pádu privedla všecky k odsouzení, ale milost z mnohých hríchu privodí k ospravedlnení.
\par 17 Nebo ponevadž pro pád jeden smrt kralovala pro jednoho, mnohemt více, kteríž by rozhojnenou milost a dar spravedlnosti prijali, v živote novém kralovati budou skrze jednoho Jezukrista.
\par 18 A tak tedy, jakž skrze pád jeden všickni lidé prišli k odsouzení, tak i skrze ospravedlnení jednoho všickni lidé mohou prijíti k ospravedlnení života.
\par 19 Neb jakož skrze neposlušenství jednoho cloveka ucineno jest mnoho hríšných, tak i skrze poslušenství jednoho spravedlivi ucineni budou mnozí.
\par 20 Zákon pak vkrocil mezi to, aby se rozhojnil hrích, a když se rozhojnil hrích, tedy ješte více rozhojnila se milost,
\par 21 Aby jakož jest kraloval hrích k smrti, tak aby i milost kralovala skrze spravedlnost k životu vecnému, skrze Jezukrista Pána našeho.

\chapter{6}

\par 1 Což tedy díme? Snad zustaneme v hríchu, aby se milost rozhojnila?
\par 2 Nikoli. Kteríž jsme zemreli hríchu, kterakž ješte živi budeme v nem?
\par 3 Zdaliž nevíte, že kterížkoli pokrteni jsme v Krista Ježíše, v smrt jeho pokrteni jsme?
\par 4 Pohrbeni jsme tedy s ním skrze krest v smrt, abychom, jakož z mrtvých vstal Kristus k sláve Otce, tak i my v novote života chodili.
\par 5 Nebo ponevadž jsme v nej vštípeni pripodobnením smrti jeho, tedy i vzkríšením jemu pripodobneni budeme,
\par 6 To vedouce, že starý clovek náš s ním spolu ukrižován jest, aby bylo umrtveno telo hrícha, abychom již potom nesloužili hríchu.
\par 7 Nebo kdožt umrel, ospravedlnen jest od hríchu.
\par 8 Jestližet jsme pak zemreli s Kristem, verímet, že spolu s ním také živi budeme,
\par 9 Vedouce, že Kristus vstav z mrtvých, již více neumírá, smrt nad ním již více nepanuje.
\par 10 Nebo že jest umrel, hríchu umrel jednou; že pak jest živ, živ jest Bohu.
\par 11 Tak i vy za to mejte, že jste zemreli zajisté hríchu, ale živi jste Bohu v Kristu Ježíši, Pánu našem.
\par 12 Nepanujž tedy hrích v smrtelném tele vašem, tak abyste povolovali jemu v žádostech jeho.
\par 13 Aniž vydávejte údu svých za odení nepravosti kterémukoli hríchu, ale vydávejte se k sloužení Bohu, jakožto vstavše z mrtvých a jsouce živi, a údy své vydávejte za odení spravedlnosti Bohu.
\par 14 Nebo hrích nebude panovati nad vámi; nejste zajisté pod Zákonem, ale pod milostí.
\par 15 Což tedy? Hrešiti budeme, když nejsme pod Zákonem, ale pod milostí? Nikoli.
\par 16 Zdaliž nevíte, že komuž se vydáváte za služebníky ku poslušenství, toho jste služebníci, kohož posloucháte, budto hríchu k smrti, budto poslušenství k spravedlnosti?
\par 17 Ale díka Bohu, že byvše služebníci hrícha, uposlechli jste z srdce zpusobu ucení toho, v kteréž uvedeni jste.
\par 18 Vysvobozeni jsouce pak od hríchu, ucineni jste služebníci spravedlnosti.
\par 19 Po lidsku pravím, pro mdlobu tela vašeho: Jakž jste vydávali údy vaše v službu necistote a nepravosti k tomu, abyste cinili nepravost, tak již nyní vydávejte údy vaše v službu spravedlnosti ku posvecení.
\par 20 Nebo když jste byli služebníci hrícha, cizí jste byli od spravedlnosti.
\par 21 Jaký jste pak užitek meli tehdáž toho, zacež se nyní stydíte? Konec zajisté tech vecí jest smrt.
\par 22 Nyní pak vysvobozeni jsouce od hríchu a podmaneni v službu Bohu, máte užitek váš ku posvecení, cíl pak život vecný.
\par 23 Nebo odplata za hrích jest smrt, ale milost Boží jest život vecný v Kristu Ježíši, Pánu našem.

\chapter{7}

\par 1 Zdaliž nevíte, bratrí, (nebo povedomým Zákona mluvím) že Zákon panuje nad clovekem, dokudž živ jest clovek?
\par 2 Nebo žena, kteráž za mužem jest, živému muži privázána jest zákonem; pakli by umrel muž její, rozvázána jest od zákona muže.
\par 3 A protož dokudž jest živ muž její, slouti bude cizoložnice, bude-li s jiným mužem; paklit by muž její umrel, jižt jest svobodna od zákona toho, takže již nebude cizoložnice, bude-li s jiným mužem.
\par 4 Takž, bratrí moji, i vy umrtveni jste Zákonu skrze telo Kristovo, abyste byli jiného, totiž toho, kterýž z mrtvých vstal, abychom ovoce nesli Bohu.
\par 5 Nebo když jsme byli v tele, žádosti hríchu prícinou Zákona vzbuzené moc svou provodily v údech našich k nesení ovoce ne Bohu, ale smrti.
\par 6 Nyní pak osvobozeni jsme od Zákona, když umrel ten, v nemž jsme držáni byli, tak abychom již sloužili v novote ducha, a ne v vetchosti litery.
\par 7 Což tedy díme? Že Zákon jest hríchem? Nikoli; nýbrž hríchu jsem nepoznal, než skrze Zákon. Nebo i o žádosti byl bych nevedel, aby hríchem byla, by byl Zákon nerekl: Nepožádáš.
\par 8 Ale prícinu vzav hrích skrze prikázaní, zplodil ve mne všelikou žádost. Bez Zákona zajisté hrích mrtev jest.
\par 9 Ját pak byl jsem živ nekdy bez Zákona, ale když prišlo prikázání, hrích ožil,
\par 10 a já umrel. I shledáno jest, že to prikázání, kteréž melo mi býti k životu, že jest mi k smrti.
\par 11 Nebo hrích, vzav prícinu skrze to prikázání, podvedl mne, a skrze ne i zabil.
\par 12 A tak Zákon zajisté svatý, a prikázání svaté i spravedlivé a dobré jest.
\par 13 Tedy to dobré ucineno jest mi smrt? Nikoli, ale hrích, kterýž aby se okázal býti hríchem, skrze to dobré zplodil mi smrt, aby tak byl príliš velmi hrešící hrích skrze prikázání.
\par 14 Víme zajisté, že Zákon jest duchovní, ale já jsem telesný, prodaný hríchu.
\par 15 Nebo toho, což ciním, neoblibuji; nebo ne, což chci, to ciním, ale, což v nenávisti mám, to ciním.
\par 16 Jestližet pak, což nechci, to ciním, tedy povoluji Zákonu, že jest dobrý.
\par 17 A tak již ne já to ciním, ale ten, kterýž prebývá ve mne, hrích.
\par 18 Vímt zajisté, že neprebývá ve mne, (to jest v tele mém), dobré. Nebo chtení hotové mám, ale vykonati dobrého, tohot nenalézám.
\par 19 Nebo neciním toho dobrého, což chci, ale ciním to zlé, cehož nechci.
\par 20 A ponevadž pak, cehož já nechci, to ciním, tedyt již ne já ciním to, ale ten, kterýž prebývá ve mne, hrích.
\par 21 Nalézám tedy takový pri sobe zákon, když chci ciniti dobré, že se mne prídrží zlé.
\par 22 Nebo zvláštní libost mám v Zákone Božím podle vnitrního cloveka;
\par 23 Ale vidím jiný zákon v údech svých, odporující zákonu mysli mé a jímající mne, tak abych byl vezen zákona hrícha, kterýž jest v údech mých.
\par 24 Bídný já clovek! Kdo mne vysvobodí z toho tela smrti?
\par 25 Ale dekujit Bohu skrze Jezukrista Pána našeho. A takžt já sloužím myslí Zákonu Božímu, ale telem zákonu hrícha.

\chapter{8}

\par 1 A protož nenít již žádného odsouzení tem, kteríž jsou v Kristu Ježíši, totiž nechodícím podle tela, ale podle Ducha.
\par 2 Nebo zákon Ducha života v Kristu Ježíši, vysvobodil mne od zákona hrícha a smrti.
\par 3 Nebo sec nemohl býti Zákon, byv mdlý pro telo, Buh poslav Syna svého v podobnosti tela hrícha, a to prícinou hrícha, odsoudil hrích na tele,
\par 4 Aby spravedlnost Zákona vyplnena byla v nás, kteríž nechodíme podle tela, ale podle Ducha.
\par 5 Ti zajisté, kteríž jsou podle tela živi, chutnají to, což jest tela, ale ti, kteríž jsou živi podle Ducha, oblibují to, což jest Ducha.
\par 6 Nebo smýšlení tela jest smrt, smýšlení pak Ducha život a pokoj,
\par 7 Protože smýšlení tela jest neprátelské Bohu; nebo Zákonu Božímu není poddáno, a aniž hned muže býti.
\par 8 Protož ti, kteríž jsou v tele, Bohu se líbiti nemohou.
\par 9 Vy pak nejste v tele, ale v Duchu, ponevadž Duch Boží prebývá v vás. Jestližet pak kdo Ducha Kristova nemá, tent není jeho.
\par 10 A jest-lit Kristus v vás, tedy ac telo umrtveno jest pro hrích, však duch živ jest pro spravedlnost.
\par 11 Jestližet pak Duch toho, kterýž vzkrísil Ježíše z mrtvých, prebývá v vás, ten, kterýž vzkrísil Krista z mrtvých, obživí i smrtelná tela vaše, pro prebývajícího Ducha jeho v vás.
\par 12 A takž tedy, bratrí, dlužnícit jsme ne telu, abychom podle tela živi byli.
\par 13 Nebo budete-li podle tela živi, zemrete; pakli byste Duchem skutky tela mrtvili, živi budete.
\par 14 Nebo kterížkoli Duchem Božím vedeni bývají, ti jsou synové Boží.
\par 15 Neprijali jste zajisté Ducha služby opet k bázni, ale prijali jste Ducha synovství, v nemžto voláme Abba, totiž Otce.
\par 16 A tent Duch osvedcuje duchu našemu, že jsme synové Boží.
\par 17 A jestliže synové, tedy i dedicové, dedicové zajisté Boží, spolu pak dedicové Kristovi, však tak, jestliže spolu s ním trpíme, abychom spolu i oslaveni byli.
\par 18 Nebo tak za to mám, že nejsou rovná utrpení nynejší oné budoucí sláve, kteráž se zjeviti má na nás.
\par 19 Nebo peclivé ocekávání všeho stvorení ocekává žádostivého zjevení synu Božích.
\par 20 Marnosti zajisté poddáno jest stvorení, nechte, ale pro toho, kterýž je poddal,
\par 21 V nadeji, že i ono vysvobozeno bude od služby porušení a privedeno v svobodu slávy synu Božích.
\par 22 Nebo víme, že všecko stvorení spolu lká a spolu ku porodu pracuje až posavad,
\par 23 A netoliko ono, ale i my, prvotiny Ducha mající, i myt také sami v sobe lkáme, zvolení synu Božích ocekávajíce, a tak vykoupení tela našeho.
\par 24 Nebo nadejí spaseni jsme. Nadeje pak, kteráž se vidí, není nadeje. Nebo což kdo vidí, proc by se toho nadál?
\par 25 Pakli cehož nevidíme, toho se nadejeme, tedy toho skrze trpelivost ocekáváme.
\par 26 Ano také i Duch svatý pomocen jest mdlobám našim. Nebo zac bychom se meli modliti, jakž by náleželo, nevíme, ale ten Duch prosí za nás lkáními nevypravitelnými.
\par 27 Ten pak, kterýž jest zpytatel srdcí, zná, jaký by byl smysl Ducha, že podle Boha prosí za svaté.
\par 28 Vímet pak, že milujícím Boha všecky veci napomáhají k dobrému, totiž tem, kteríž podle uložení jeho povoláni jsou.
\par 29 Nebo kteréž predzvedel, ty i predzrídil, aby byli pripodobneni obrazu Syna jeho, aby tak on byl prvorozený mezi mnohými bratrími.
\par 30 Kteréž pak predzrídil, tech i povolal, a kterýchž povolal, ty i ospravedlnil, a kteréž ospravedlnil, ty i oslavil.
\par 31 Což tedy díme k tomu? Kdyžt jest Buh s námi, i kdo proti nám?
\par 32 Kterýž ani vlastnímu Synu svému neodpustil, ale za nás za všecky vydal jej, i kterakž by tedy nám s ním všech vecí nedal?
\par 33 Kdo bude žalovati na vyvolené Boží? Buh jest, jenž ospravedlnuje.
\par 34 Kdo jest, ješto by je odsoudil? Kristus jest, kterýž umrel za ne, nýbrž i z mrtvých vstal, a kterýž i na pravici Boží jest, kterýž také i oroduje za nás.
\par 35 A protož kdo nás odloucí od lásky Kristovy? Zdali zarmoucení, aneb úzkost, nebo protivenství? Zdali hlad, cili nahota? Zdali nebezpecenství, cili mec?
\par 36 Jakož psáno jest: Pro tebe mrtveni býváme celý den, jmíni jsme jako ovce oddané k zabití.
\par 37 Ale v tom ve všem udatne vítezíme, skrze toho, kterýž nás zamiloval.
\par 38 Jist jsem zajisté, že ani smrt, ani život, ani andelé, ani knížatstvo, ani mocnosti, ani nastávající veci, ani budoucí,
\par 39 Ani vysokost, ani hlubokost, ani kterékoli jiné stvorení, nebude moci nás odlouciti od lásky Boží, kteráž jest v Kristu Ježíši, Pánu našem.

\chapter{9}

\par 1 Pravdut pravím v Kristu a neklamámt, cemuž i svedomí mé svedectví vydává v Duchu svatém,
\par 2 Žet mám veliký zámutek a ustavicnou bolest v srdci svém.
\par 3 Nebo žádal bych já sám zavrženým býti od Krista místo bratrí svých, totiž príbuzných svých podle tela.
\par 4 Kterížto jsou Izraelští, jejichžto jest prijetí za syny, i sláva, i smlouvy, i Zákona dání, i služba Boží, i zaslíbení.
\par 5 Jejichž jsou i otcové, a ti, z nichžto jest Kristus podle tela, kterýž jest nade všecky Buh požehnaný na veky. Amen.
\par 6 Avšak nemuže zmareno býti slovo Boží. Nebo ne všickni, kteríž jsou z Izraele, Izraelští jsou.
\par 7 Aniž proto, že jsou síme Abrahamovo, hned také všickni jsou synové, ale v Izákovi nazváno bude tvé síme.
\par 8 To jest, ne všickni ti, jenž jsou synové tela, jsou také synové Boží, ale kteríž jsou synové Božího zaslíbení, ti se pocítají za síme.
\par 9 Nebo toto jest slovo zaslíbení: V týž cas prijdu, a Sára bude míti syna.
\par 10 A netoliko to, ale i Rebeka z jednoho pocavši, totiž z Izáka, otce našeho, toho duvodem jest.
\par 11 Nebo ješte pred narozením obou tech synu, a prve nežli co dobrého nebo zlého ucinili, aby uložení Boží, kteréžto jest podle vyvolení, a tak ne z skutku, ale z toho, jenž povolává, pevné bylo,
\par 12 Receno jest jí: Vetší sloužiti bude menšímu,
\par 13 Jakož psáno jest: Jákoba jsem miloval, ale Ezau nenávidel jsem.
\par 14 I což tedy díme? Zdali nespravedlnost jest u Boha? Nikoli.
\par 15 Nebo Mojžíšovi dí: Smiluji se nad tím, komuž milost uciním, a slituji se nad tím, nad kýmž se slituji.
\par 16 A tak tedy nenít vyvolování na tom, jenž chce, ani na tom, jenž beží, ale na Bohu, jenž se smilovává.
\par 17 Nebo dí Písmo faraonovi: Proto jsem vzbudil tebe, abych na tobe ukázal moc svou a aby rozhlášeno bylo jméno mé po vší zemi.
\par 18 A tak tedy nad kýmž chce, smilovává se, a koho chce, zatvrzuje.
\par 19 Ale díš mi: I procež se pak hnevá? Nebo vuli jeho kdo odeprel?
\par 20 Ale ó clovece, kdo jsi ty, že tak odpovídáš Bohu? Zdaž hrnec dí hrncíri: Procs mne tak udelal?
\par 21 Zdaliž hrncír nemá moci nad hlinou, aby z jednostejného truple udelal jednu nádobu ke cti a jinou ku potupe?
\par 22 Což pak divného, že Buh, chteje ukázati hnev a oznámiti moc svou, snášel ve mnohé trpelivosti nádoby hnevu, pripravené k zahynutí.
\par 23 A takž také, aby známé ucinil bohatství slávy své pri nádobách milosrdenství, kteréž pripravil k sláve.
\par 24 Kterýchžto i povolal, totiž nás, netoliko z Židu, ale také i z pohanu,
\par 25 Jakož i skrze Ozé dí: Nazovu nelid muj lidem mým, a nemilou nazovu milou.
\par 26 A budet na tom míste, kdež receno bylo jim: Nejste vy lid muj, tut nazváni budou synové Boha živého.
\par 27 Izaiáš pak volá nad Izraelem, rka: Byt pak byl pocet synu Izraelských jako písek morský, ostatkové toliko spaseni budou.
\par 28 Nebo pohubení uciní spravedlivé, a to jisté, pohubení zajisté uciní Pán na zemi, a to jisté.
\par 29 A jakož prve povedel Izaiáš: Byt byl Pán zástupu nepozustavil nám semene, jako Sodoma ucineni bychom byli, a Gomore byli bychom podobni.
\par 30 Což tedy díme? Že pohané, kteríž nenásledovali spravedlnosti, dosáhli spravedlnosti, a to spravedlnosti té, kteráž jest z víry;
\par 31 Izrael pak následovav zákona spravedlnosti, k zákonu spravedlnosti neprišel.
\par 32 Proc? Nebo ne z víry, ale jako z skutku Zákona jí hledali. Urazili se zajisté o kámen urážky,
\par 33 Jakož psáno jest: Aj, kladu na Sionu kámen urážky a skálu pohoršení, a každý, kdož uverí v nej, nebude zahanben.

\chapter{10}

\par 1 Bratrí, príchylnost zajisté s zvláštní libostí srdce mého jestit k Izraelovi, i modlitba za nej k Bohu, aby spasen byl.
\par 2 Nebot jim svedectví vydávám, žet horlivost Boží mají, ale ne podle umení.
\par 3 Nebo neznajíce Boží spravedlnosti, a svou vlastní spravedlnost hledajíce vystaviti, spravedlnosti Boží nebyli poddáni.
\par 4 Nebo konec Zákona jest Kristus k ospravedlnení všelikému verícímu.
\par 5 Nebo Mojžíš píše o spravedlnosti, kteráž jest z Zákona, prave: Který by koli clovek cinil ty veci, živ bude v nich.
\par 6 Ta pak spravedlnost, kteráž jest z víry, takto praví: Neríkej v srdci svém: Kdo vstoupí na nebe? To jest Krista s výsosti svésti.
\par 7 Aneb kdo sstoupí do propasti? To jest Krista z mrtvých vzbuditi.
\par 8 Ale co dí spravedlnost z víry? Blízko tebe jestit slovo, v ústech tvých a v srdci tvém. A tot jest slovo to víry, kteréž kážeme,
\par 9 Totiž, vyznáš-li ústy svými Pána Ježíše a srdcem svým uveríš-li, že jej Buh vzkrísil z mrtvých, spasen budeš.
\par 10 Nebo srdcem se verí k spravedlnosti, ale ústy vyznání deje se k spasení.
\par 11 Nebo dí Písmo: Všeliký, kdož verí v nej, nebude zahanben.
\par 12 Nenít zajisté rozdílu mezi Židem a Rekem; nebo tentýž Pán všech, bohatý jest ke všem vzývajícím jej.
\par 13 Každý zajisté, kdožkoli vzýval by jméno Páne, spasen bude.
\par 14 Ale kterak budou vzývati toho, v kteréhož neuverili? A kterak uverí tomu, o nemž neslyšeli? A kterak uslyší bez kazatele?
\par 15 A kterak kázati budou, jestliže nebudou posláni? Jakož psáno jest: Aj, jak krásné nohy zvestujících pokoj, zvestujících dobré veci.
\par 16 Ale ne všickni uposlechli evangelium. Nebo Izaiáš praví: Pane, kdo uveril kázání našemu?
\par 17 Tedy víra z slyšení, a slyšení skrze slovo Boží.
\par 18 Ale pravímt: Zdaliž jsou neslyšeli? Anobrž po vší zemi rozšel se zvuk jejich a až do koncin okršlku zeme slova jejich.
\par 19 Ale pravím: Zdaliž nepoznal Izrael toho? Ano první z nich Mojžíš rekl: Já k závisti vás privedu skrze národ ten, kteréhož nemáte za lid muj; skrze lid nemoudrý k hnevu popudím vás.
\par 20 A Izaiáš smele dí: Nalezen jsem od tech, kteríž mne nehledali; zjeven jsem tem, kteríž se na mne neptali.
\par 21 Ale proti lidu Izraelskému dí: Pres celý den roztahoval jsem ruce své k lidu nepovolnému a protivnému.

\chapter{11}

\par 1 Protož pravím: Zdali jest Buh zavrhl lid svuj? Nikoli; nebo i já Izraelský jsem, z semene Abrahamova, z pokolení Beniaminova.
\par 2 Nezavrhlt jest Buh lidu svého, kterýž predzvedel. Zdali nevíte, co Písmo praví o Eliášovi? Kterak se modlí Bohu proti lidu Izraelskému rka:
\par 3 Pane, proroky tvé zmordovali a oltáre tvé rozkopali, já pak zustal jsem sám, a i mét duše hledají.
\par 4 Ale co jemu dí odpoved Boží? Pozustavil jsem sobe sedm tisícu mužu, kteríž neskláneli kolen pred Bálem.
\par 5 Takt i nyní ostatkové podle vyvolení jdoucího z pouhé milosti Boží zustali,
\par 6 A ponevadž z milosti, tedy ne z skutku, sic jinak milost již by nebyla milost. Pakli z skutku, tedy již není milost, jinak skutek nebyl by skutek.
\par 7 Což tedy? Ceho hledá Izrael, toho jest nedošel, ale vyvolení došli toho, jiní pak zatvrzeni jsou,
\par 8 (Jakož psáno jest: Dal jim Buh ducha zkormoucení, oci, aby nevideli, a uši, aby neslyšeli,) až do dnešního dne.
\par 9 A David dí: Budiž jim stul jejich osidlem a pastmi a pohoršením i spravedlivým odplacením.
\par 10 Zatmetež se oci jejich, at nevidí, a hrbet jejich vždycky shýbej.
\par 11 A z toho pravím: Tak-liž jsou pak Židé klesli, aby docela padli? Nikoli, ale jejich klesnutím spasení priblížilo se pohanum, aby je tak Buh k závidení privedl.
\par 12 A ponevadž pak jejich pád jest bohatství sveta a zmenšení jejich jest bohatství pohanu, cím více plnost jich?
\par 13 Vámt zajisté pravím pohanum, jelikož jsem já apoštol pohanský, prisluhování mé oslavuji,
\par 14 Zda bych jak k závidení vzbuditi mohl ty, jenž jsou telo mé, a k spasení privésti aspon nekteré z nich.
\par 15 Nebo kdyžt zavržení jich jest smírení sveta, co pak bude zase jich prijetí, než život z mrtvých?
\par 16 Ponevadž prvotiny svaté jsou, takét svaté jest i testo; a jest-lit koren svatý, tedy i ratolesti.
\par 17 Žet jsou pak nekteré ratolesti vylomeny, a ty, byv planou olivou, vštípen jsi místo nich a ucinen jsi úcastník korene i tucnosti olivy.
\par 18 Proto ty se nechlub proti ratolestem. Pakli se chlubíš, vez, že ne ty koren neseš, ale koren tebe.
\par 19 Pakli díš: Vylomeny jsou ratolesti, abych já byl vštípen,
\par 20 Dobre pravíš. Pro neveru vylomeny jsou, ale ty verou stojíš. Nebudiž vysokomyslný, ale boj se.
\par 21 Nebo ponevadž Buh ratolestem prirozeným neodpustil, vez, žet by ani tobe neodpustil.
\par 22 A protož viz dobrotivost i zurivost Boží. K tem zajisté, kteríž padli, zurivost, ale k tobe dobrotivost, ac budeš-li trvati v dobrote. Sic jinak i ty vytat budeš.
\par 23 Ano i oni, jestliže nezustanou v nevere, zase vštípeni budou. Mocent jest zajisté Buh zase vštípiti je.
\par 24 Nebo ponevadž ty vytat jsi z prirozené plané olivy a proti prirození vštípen jsi v dobrou olivu, cím více pak ti, kteríž podle prirození jsou z dobré olivy, vštípeni budou v svou vlastní olivu.
\par 25 Nebot nechci, bratrí, abyste nevedeli tohoto tajemství, (abyste nebyli sami u sebe moudrí,) že zatvrdilost tato zcástky prihodila se Izraelovi, dokudž by nevešla plnost pohanu.
\par 26 A takt všecken Izrael spasen bude, jakož psáno jest: Prijde z Siona vysvoboditel a odvrátít bezbožnosti od Jákoba.
\par 27 A tatot bude smlouva má s nimi, když shladím hríchy jejich.
\par 28 A tak s strany evangelium jsout neprátelé pro vás, ale podle vyvolení jsou milí pro otce svaté.
\par 29 Daru zajisté svých a povolání Buh nelituje.
\par 30 Nebo jakož i vy nekdy jste nebyli poslušni Boha, ale nyní milosrdenství jste došli pro jejich neveru,
\par 31 Tak i oni nyní neuposlechli, aby pro ucinené vám milosrdenství i oni také milosrdenství dosáhli.
\par 32 Zavrel zajisté Buh všecky v nevere, aby se nade všemi smiloval.
\par 33 Ó hlubokosti bohatství i moudrosti i umení Božího! Jak jsou nezpytatelní soudové jeho a nevystižitelné cesty jeho!
\par 34 Nebo kdo jest poznal mysl Páne? Aneb kdo jemu radil?
\par 35 Nebo kdo prve dal jemu, a budet mu odplaceno?
\par 36 Nebo z neho a skrze neho a v nem jsou všecky veci, jemuž sláva na veky. Amen.

\chapter{12}

\par 1 Protož prosím vás, bratrí, skrze milosrdenství Boží, abyste vydávali tela svá v obet živou, svatou, Bohu libou, rozumnou službu vaši.
\par 2 A nepripodobnujte se svetu tomuto, ale promentež se obnovením mysli vaší, tak abyste zkusili, jaká by byla vule Boží dobrá, libá a dokonalá.
\par 3 Nebot pravím (skrze milost, kteráž dána jest mi,) každému z tech, jenž jsou mezi vámi, aby ne více smyslil, než sluší smysliti, ale aby smyslil v stredmosti, tak jakž jednomu každému Buh udelil míru víry.
\par 4 Nebo jakož v jednom tele mnohé údy máme, ale nemají všickni údové jednostejného díla,
\par 5 Tak mnozí jedno telo jsme v Kristu, a obzvláštne jedni druhých údové.
\par 6 Ale majíce obdarování rozdílná podle milosti, kteráž dána jest nám, budto proroctví, kteréž at jest podle pravidla víry;
\par 7 Budto úrad, v bedlivém prisluhování; budto ten, jenž ucí, v vyucování.
\par 8 Též kdo napomíná, v napomínání; ten, jenž rozdává, dávej v uprímnosti; kdož jiným predložen jest, konej úrad svuj s pilností; kdo milosrdenství ciní, s ochotností.
\par 9 Milování bud bez pokrytství; v ošklivosti mejte zlé, pripojeni jsouce k dobrému.
\par 10 Láskou bratrskou jedni k druhým nakloneni jsouce, uctivostí se vespolek predcházejte,
\par 11 V pracech neleniví, duchem vroucí, príhodnosti casu šetrící,
\par 12 Nadejí se radující, v souženích trpeliví, na modlitbe ustavicní,
\par 13 V potrebách s svatými se sdelující, prívetivosti k hostem následující.
\par 14 Dobrorecte protivníkum vašim, dobrorecte, pravím, a nezlorecte.
\par 15 Radujte s radujícími, a placte s placícími.
\par 16 Budte vespolek jednomyslní, ne vysoce o sobe smýšlejíce, ale k nízkým se naklonujíce.
\par 17 Nebudte opatrní sami u sebe. Žádnému zlého za zlé neodplacujte, opatrujíce dobré prede všemi lidmi,
\par 18 Jestliže jest možné, pokudž na vás jest, se všemi lidmi pokoj majíce,
\par 19 Ne sami sebe mstíce, nejmilejší, ale dejte místo hnevu; nebo psáno jest: Mne pomsta, já odplatím, praví Pán.
\par 20 A protož lacní-li neprítel tvuj, nakrm jej, a žízní-li, dej mu píti. Nebo to ucine, uhlí reravé shrneš na hlavu jeho.
\par 21 Nedej se premoci zlému, ale premáhej v dobrém zlé.

\chapter{13}

\par 1 Každá duše vrchnostem povýšeným poddána bud. Nebot není vrchnosti, jediné od Boha, a kteréž vrchnosti jsou, ty od Boha zrízené jsou.
\par 2 A protož, kdož se vrchnosti protiví, Božímu zrízení se protiví; kteríž se pak protiví, tit sobe odsouzení dobudou.
\par 3 Nebo vrchnosti nejsou k strachu dobre cinícím, ale zle cinícím. Protož chceš-li se nebáti vrchnosti, cin dobre, a budeš míti chválu od ní.
\par 4 Boží zajisté služebník jest, tobe k dobrému. Pakli bys zle cinil, boj se; nebot ne nadarmo nese mec. Boží zajisté služebník jest, mstitel zurivý nad tím, kdož zle ciní.
\par 5 A protož musejít vrchnostem všickni poddáni býti, netoliko pro hnev, ale i pro svedomí.
\par 6 Nebo proto i dan dáváte, ponevadž služebníci Boží jsou, pilnou práci o to vedouce.
\par 7 Každému tedy což jste povinni, dávejte. Komu dan, tomu dan; komu clo, tomu clo; komu bázen, tomu bázen; komu cest, tomu cest.
\par 8 Žádnému nebývejte nic dlužni, než to, abyste se vespolek milovali. Nebo kdož miluje bližního, Zákon naplnil,
\par 9 Ponevadž to prikázání: Nesesmilníš, nezabiješ, neukradneš, nepromluvíš krivého svedectví, nepožádáš, a jest-li které jiné prikázání, v tomto slovu se zavírá: Milovati budeš bližního svého jako sebe samého.
\par 10 Láska bližnímu zle neuciní, a protož plnost Zákona jestit láska.
\par 11 A zvlášte pak vidouce takovou príhodnost, žet jest se nám již cas ze sna probuditi. (Nynít zajisté blíže nás jest spasení, nežli když jsme uverili.)
\par 12 Noc pominula, ale den se priblížil. Odvrzmež tedy skutky temnosti, a oblecme se v odení svetla.
\par 13 Jakožto ve dne poctive chodme, ne v hodování a v opilství, ne v smilstvích a v chlipnostech, ne v sváru a v závisti,
\par 14 Ale oblecte se v Pána Jezukrista, a nepecujte o telo k vyplnování žádostí jeho.

\chapter{14}

\par 1 Mdlého pak u víre prijímejte, ne k hádkám o nepotrebných otázkách.
\par 2 Nebo nekdo verí, že muže jísti všecko; jiný pak u víre mdlý jsa, jí zelinu.
\par 3 Ten, kdož jí, nepokládej sobe za nic toho, kdož nejí; a kdo nejí, toho nesud, kdož jí. Nebo Buh prijal jej.
\par 4 Ty kdo jsi, abys soudil cizího služebníka? Však Pánu svému stojí, anebo padá. Stanet pak; mocen jest zajisté Buh utvrditi jej.
\par 5 Nebo nekdo rozsuzuje mezi dnem a dnem, a nekdo soudí každý den jednostejný býti. Jeden každý v svém smyslu ujišten bud.
\par 6 Kdož dnu šetrí, Pánu šetrí; a kdo nešetrí, Pánu nešetrí. A kdo jí, Pánu jí, nebo dekuje Bohu; a kdož nejí, Pánu nejí, a dekuje Bohu.
\par 7 Žádný zajisté z nás není sám sobe živ, a žádný sobe sám neumírá.
\par 8 Nebo budto že jsme živi, Pánu živi jsme; budto že mreme, Pánu mreme. A tak bud že jsme živi, bud že umíráme, Páne jsme.
\par 9 Na tot jest zajisté Kristus i umrel, i z mrtvých vstal, i ožil, aby nad živými i nad mrtvými panoval.
\par 10 Ty pak proc odsuzuješ bratra svého? Anebo také ty proc za nic pokládáš bratra svého? Však všickni staneme pred stolicí Kristovou.
\par 11 Psáno jest zajisté: Živt jsem já, praví Pán, žet prede mnou bude klekati každé koleno, a každý jazyk vyznávati bude Boha.
\par 12 A takt jeden každý z nás sám za sebe pocet vydávati bude Bohu.
\par 13 Nesudmež tedy více jedni druhých, ale toto radeji rozsuzujte, jak byste nekladli úrazu nebo pohoršení bratru.
\par 14 Vím a v tom ujišten jsem v Pánu Ježíši, žet nic necistého není samo z sebe; než tomu, kdož tak soudí, že by necisté bylo, jemut necisté jest.
\par 15 Ale bývá-lit rmoucen bratr tvuj pro pokrm, již nechodíš podle lásky. Hlediž, abys k zahynutí neprivedl pokrmem svým toho, za kteréhož Kristus umrel.
\par 16 Nebudiž tedy v porouhání dáno dobré vaše.
\par 17 Království zajisté Boží není pokrm a nápoj, ale spravedlnost a pokoj a radost v Duchu svatém.
\par 18 Nebo kdož v tom slouží Kristu, milý jest Bohu a lidem príjemný.
\par 19 Protož následujme toho, což by sloužilo ku pokoji a k vzdelání vespolek.
\par 20 Nekaziž pro pokrm díla Božího. Všecko zajisté cisté jest, ale zlé jest cloveku, kterýž jí s pohoršením.
\par 21 Dobré jest nejísti masa a nepíti vína, ani cehokoli toho, na cemž se uráží bratr tvuj, nebo horší, anebo zemdlívá.
\par 22 Ty víru máš? Mejž ji sám u sebe pred Bohem. Blahoslavený, kdož nesoudí sebe samého v tom, což oblibuje.
\par 23 Ale kdož pak rozpakuje se, kdyby jedl, odsouzen jest, nebo ne z víry jí. A cožkoli není z víry, hrích jest.

\chapter{15}

\par 1 Povinnit jsme pak my silní mdloby nemocných snášeti, a ne sami sobe se líbiti.
\par 2 Ale jeden každý z nás bližnímu se lib k dobrému pro vzdelání.
\par 3 Nebo i Kristus ne sám se sobe líbil, ale jakož psáno jest: Hanení hanejících tebe pripadla jsou na mne.
\par 4 Nebo kteréžkoli veci napsány jsou, k našemu naucení napsány jsou, abychom skrze trpelivost a potešení Písem nadeji meli.
\par 5 Buh pak dárce trpelivosti a potešení dejž vám jednomyslným býti vespolek podle Jezukrista,
\par 6 Abyste jednomyslne jednemi ústy oslavovali Boha a Otce Pána našeho Jezukrista.
\par 7 Protož prijímejte se vespolek, jakož i Kristus prijal nás v slávu Boží.
\par 8 Nebot pravím vám, že Kristus Ježíš byl služebníkem obrízky pro Boží pravost, aby potvrzeni byli slibové otcum ucinení,
\par 9 A aby pohané z milosrdenství slavili Boha, jakož psáno jest: Protož vyznávati tebe budu mezi pohany, a jménu tvému plésati budu.
\par 10 A opet dí: Veselte se pohané s lidem jeho.
\par 11 A opet: Chvalte Hospodina všickni národové, a velebtež ho všickni lidé.
\par 12 A opet Izaiáš dí: Budet koren Jesse, a ten, jenž povstane, panovati nad pohany; v nemt pohané doufati budou.
\par 13 Buh pak nadeje naplnujž vás všelikou radostí a pokojem u víre, tak abyste se rozhojnili v nadeji skrze moc Ducha svatého.
\par 14 Vím zajisté, bratrí moji, i já také o vás, že i vy jste plní dobroty, naplneni jsouce všelikou známostí, takže se i napomínati mužete vespolek.
\par 15 Ale však proto psal jsem vám, bratrí, ponekud smele, jako ku pameti privode vám, podle milosti, kteráž jest mi dána od Boha,
\par 16 K tomu, abych byl služebníkem Ježíše Krista mezi pohany, slouže evangelium Božímu, aby byla obet pohanu vzácná, posvecena jsuci skrze Ducha svatého.
\par 17 Mám se tedy cím chlubiti v Kristu Ježíši, v Božích vecech.
\par 18 Nebot bych se neodvážil mluviti toho, cehož by skrze mne neucinil Kristus, k tomu, aby ku poslušenství privedeni byli pohané slovem i skutkem,
\par 19 V moci divu a zázraku, v síle Ducha Božího, takže jsem od Jeruzaléma vukol až k Ilyrické zemi naplnil evangelium Kristovým,
\par 20 A to tak žádostiv jsa kázati evangelium, kdež ani jmenován nebyl Kristus, abych na cizí základ nestavel,
\par 21 Ale jakož psáno jest: Kterýmž není zvestováno o nem, uzrí, a ti, jenž neslýchali, srozumejí.
\par 22 A tímt jest mi mnohokrát prekaženo prijíti k vám.
\par 23 Nyní pak nemaje již více místa v techto krajinách a žádost maje prijíti k vám od mnoha let,
\par 24 Kdyžkoli pujdu do Hišpanie, prijdu k vám. Mámt zajisté nadeji, že tudy jda, uzrím vás, a že vy mne provodíte tam, avšak až bych prve u vás ponekud pobyl.
\par 25 Nyní pak beru se do Jeruzaléma, službu cine svatým.
\par 26 Nebo za dobré se videlo Macedonským a Achaiským, aby sbírku nejakou ucinili na chudé svaté, kteríž jsou v Jeruzaléme.
\par 27 Takt sobe to oblíbili, a také povinni jsou jim to. Nebo ponevadž duchovních vecí jejich byli úcastni pohané, povinnit jsou jim také sloužiti telesnými.
\par 28 A protož když to vykonám a jim odvedu užitek ten, pujdut skrze vás do Hišpanie.
\par 29 A vímt, že prijda k vám, v hojnosti požehnání evangelium Kristova prijdu.
\par 30 Prosímt pak vás, bratrí, skrze Pána našeho Jezukrista a skrze lásku Ducha svatého, abyste spolu se mnou modlili se za mne Bohu snažne,
\par 31 Abych vysvobozen byl od protivníku, kteríž jsou v Judstvu a aby služba tato má príjemná byla svatým v Jeruzaléme,
\par 32 Abych k vám bohdá s radostí prišel, a s vámi poodpocinul.
\par 33 Buh pak pokoje budiž se všemi vámi. Amen.

\chapter{16}

\par 1 Poroucímt pak vám Fében, sestru naši, služebnici církve Cenchrenské,
\par 2 Abyste ji prijali v Pánu, tak jakž sluší na svaté, a abyste jí pomocni byli, jestliže by vás v cem potrebovala. Nebo i ona mnohým hostem ochotne posluhovala, až i mne také.
\par 3 Pozdravte Priscilly a Akvila, pomocníku mých v Kristu Ježíši,
\par 4 Kteríž pro muj život svých vlastních hrdel nasadili, jimžto ne já sám toliko dekuji, ale i všecky církve pohanské,
\par 5 I domácího jejich shromáždení. Pozdravte mého milého Epéneta, kterýž jest prvotiny Achaie v Kristu.
\par 6 Pozdravte Marie, kteráž mnoho práce mela s námi.
\par 7 Pozdravte Andronika a Junia, príbuzných mých a spoluveznu mých, kteríž jsou vzácní u apoštolu a kteríž prede mnou byli v Kristu Ježíši.
\par 8 Pozdravte Amplia mne v Pánu milého.
\par 9 Pozdravte Urbana, pomocníka našeho v Kristu, a Stachyna mého milého.
\par 10 Pozdravte Apella zkušeného v Kristu. Pozdravte tech, kteríž jsou z domu Aristobulova.
\par 11 Pozdravte Herodiona, príbuzného mého. Pozdravte tech, kterí jsou z celedi Narciškovy verící v Pána.
\par 12 Pozdravte Tryfény a Tryfózy, kteréž práci vedou v Pánu. Pozdravte Persidy milé, kteráž mnoho pracovala v Pánu.
\par 13 Pozdravte Rufa, zvláštního v Pánu, a matky jeho i mé.
\par 14 Pozdravte Asynkrita, Flegonta, Hermy, Patroba, Merkuria, i jiných bratrí, kteríž jsou s nimi.
\par 15 Pozdravte Filologa i Julie, Nerea a sestry jeho, i Olympa i všech svatých, kteríž jsou s nimi.
\par 16 Pozdravte sebe vespolek v políbení svatém. Pozdravujít vás církve Kristovy.
\par 17 Prosímt pak vás, bratrí, abyste šetrili tech, kteríž ruznice a pohoršení ciní, naodpor ucení tomu, kterémuž jste vy se naucili, a varujte se jich.
\par 18 Nebo takoví Pánu našemu Ježíši Kristu neslouží, ale svému brichu; a skrze lahodné reci a pochlebenství svodí srdce prostých.
\par 19 Nebo vaše poslušenství všechnech došlo. A protož se raduji z vás. Než chcit, abyste byli moudrí k dobrému, a prostí k zlému.
\par 20 Buh pak pokoje potre satana pod nohy vaše brzo. Milost Pána našeho Jezukrista budiž s vámi. Amen.
\par 21 Pozdravují vás Timoteus, pomocník muj, a Lucius, a Jázon, a #Sozipater, príbuzní moji.
\par 22 Pozdravuji vás v Pánu já Tercius, kterýž jsem psal tento list.
\par 23 Pozdravuje vás Gáius, hospodár muj i vší církve. Pozdravuje vás Erastus, duchodní písar mestský, a Kvartus bratr.
\par 24 Milost Pána našeho Jezukrista se všemi vámi. Amen.
\par 25 Tomu pak, jenž muže vás utvrditi podle evangelium mého a kázání Ježíše Krista, podle zjevení tajemství od casu vecných skrytého,
\par 26 Nyní pak zjeveného i skrze Písma prorocká, podle porucení vecného Boha, ku poslušenství víry všechnem národum oznámeného,
\par 27 Tomu, pravím, samému moudrému Bohu sláva skrze Jezukrista na veky. Amen List tento k Rímanum psán jest z Korintu, a poslán po Fében, služebnici sboru Cenchrenského.


\end{document}