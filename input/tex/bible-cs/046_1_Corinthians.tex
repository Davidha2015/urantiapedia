\begin{document}

\title{1 Korintským}

\chapter{1}

\par 1 Pavel, povolaný apoštol Ježíše Krista, skrze vuli Boží, a bratr Sostenes,
\par 2 Církvi Boží, kteráž jest v Korintu, posveceným v Kristu Ježíši, povolaným svatým, spolu se všemi, kteríž vzývají jméno Pána našeho Jezukrista na všelikém míste, i jejich i našem:
\par 3 Milost vám a pokoj od Boha Otce našeho a Pána Jezukrista.
\par 4 Dekuji Bohu svému vždycky za vás pro tu milost Boží, kteráž dána jest vám v Kristu Ježíši,
\par 5 Že ve všem obohaceni jste v nem, v každém slovu a ve všelikém umení,
\par 6 Jakož svedectví Kristovo upevneno jest mezi vámi,
\par 7 Takže nemáte žádného nedostatku ve všeliké milosti, ocekávajíce zjevení Pána našeho Jezukrista,
\par 8 Kterýžto i utvrdí vás až do konce bez úhony ke dni príští Pána našeho Jezukrista.
\par 9 Vernýt jest Buh, skrze nehož povoláni jste k úcastenství Syna jeho Jezukrista, Pána našeho.
\par 10 Prosímt vás pak, bratrí, skrze jméno Pána našeho Jezukrista, abyste jednostejne mluvili všickni a aby nebylo mezi vámi roztržek, ale budte spojeni jednostejnou myslí a jednostejným smyslem.
\par 11 Nebo oznámeno jest mi o vás, bratrí moji, od nekterých z celedi Chloe, že by mezi vámi byly ruznice.
\par 12 Míním pak toto, že jeden každý z vás ríká: Já jsem Pavluv, já Apolluv, já Petruv, já pak Kristuv.
\par 13 Zdali rozdelen jest Kristus? Zdali Pavel ukrižován jest za vás? Anebo zdali jste ve jménu Pavlovu pokrteni byli?
\par 14 Dekuji Bohu, že jsem žádného z vás nekrtil, než Krispa a Gáia,
\par 15 Aby nekdo nerekl, že jsem ve jméno své krtil.
\par 16 Krtilt jsem také i Štepánovu celed. Více nevím, abych koho jiného krtil.
\par 17 Nebo neposlal mne Kristus krtíti, ale evangelium kázati, ne v moudrosti reci, aby nebyl vyprázdnen kríž Kristuv.
\par 18 Nebo slovo kríže tem, kteríž hynou, bláznovstvím jest, ale nám, kteríž spasení dosahujeme, moc Boží jest.
\par 19 Nebo psáno jest: Zahladím moudrost moudrých, a opatrnost opatrných zavrhu.
\par 20 Kde jest moudrý? A kde ucený? A kde chytrák tohoto sveta? Zdaliž Buh neobrátil moudrosti tohoto sveta v bláznovství?
\par 21 Nebo když v moudrosti Boží svet nepoznal skrze moudrost Boha, zalíbilo se Bohu skrze bláznové kázaní spasiti verící,
\par 22 Ponevadž i Židé zázraku žádají, i Rekové hledají moudrosti.
\par 23 Myt pak kážeme Krista ukrižovaného, Židum zajisté pohoršení, a Rekum bláznovství,
\par 24 Ale povolaným, i Židum i Rekum, Krista, Boží moc a Boží moudrost.
\par 25 Nebo to bláznovství Boží jest moudrejší nežli lidé a mdloba Boží jest silnejší než lidé.
\par 26 Vidíte zajisté povolání vaše, bratrí, že nemnozí moudrí podle tela, nemnozí mocní, nemnozí urození;
\par 27 Ale což bláznivého jest u sveta, to sobe vyvolil Buh, aby zahanbil moudré, a to, což jest u sveta mdlé, Buh vyvolil, aby zahanbil silné.
\par 28 A neurozené u sveta a za nic položené vyvolil Buh, ano hned, kteréž nejsou, aby ty veci, kteréž jsou, zkazil,
\par 29 Proto aby se nechlubilo pred oblicejem jeho žádné telo.
\par 30 Vy pak jste z neho v Kristu Ježíši, kterýž ucinen jest nám moudrost od Boha, i spravedlnost, i posvecení, i #vykoupení,
\par 31 Aby se tak dálo, jakož jest napsáno: Kdo se chlubí, v Pánu se chlub.

\chapter{2}

\par 1 I já prišed k vám, bratrí, neprišel jsem s dustojností reci nebo moudrostí, zvestuje vám svedectví Boží.
\par 2 Nebo tak jsem usoudil nic jiného neumeti mezi vámi, nežli Ježíše Krista, a to ješte toho ukrižovaného.
\par 3 A byl jsem já u vás v mdlobe, a v bázni, i v strachu mnohém.
\par 4 A rec má a kázaní mé nebylo v slibných lidské moudrosti recech, ale v dokázání Ducha svatého a moci,
\par 5 Aby víra vaše nebyla založena v moudrosti lidské, ale v moci Boží.
\par 6 Moudrost pak mluvíme mezi dokonalými, ale moudrost ne tohoto sveta, ani knížat sveta tohoto, jenž hynou.
\par 7 Ale mluvíme moudrost Boží v tajemství, kterážto skryta jest, kterouž Buh preduložil pred veky k sláve naší,
\par 8 Jížto žádný z knížat sveta tohoto nepoznal. Nebo kdyby byli poznali, nebylit by Pána slávy ukrižovali.
\par 9 Ale kážeme, jakož psáno jest: Cehož oko nevídalo, ani ucho slýchalo, ani na srdce lidské vstoupilo, co jest pripravil Buh tem, kteríž jej milují.
\par 10 Nám pak Buh zjevil skrze Ducha svého. Nebo Duch zpytuje všecky veci, i hlubokosti Božské.
\par 11 Nebo kdo z lidí ví, co jest v cloveku, jediné duch cloveka, kterýž jest v nem? Takt i Božích vecí nezná žádný, jediné Duch Boží.
\par 12 My pak neprijali jsme ducha sveta, ale Ducha toho, kterýž jest z Boha, abychom vedeli, které veci od Boha darovány jsou nám.
\par 13 O nichž i mluvíme ne temi slovy, jimž lidská moudrost ucí, ale kterýmž vyucuje Duch svatý, duchovním to, což duchovního jest, privlastnujíce.
\par 14 Ale telesný clovek nechápá tech vecí, kteréž jsou Ducha Božího; nebo jsou jemu bláznovství, aniž jich muže poznati, protože ony duchovne mají rozsuzovány býti.
\par 15 Ale duchovní clovek rozsuzujet všecko, sám pak od žádného nebývá souzen.
\par 16 Nebo kdo jest poznal mysl Páne? A kdo jej bude uciti? My pak mysl Kristovu máme.

\chapter{3}

\par 1 A já, bratrí, nemohl jsem vám mluviti jako duchovním, ale jako telesným, jako malickým v Kristu.
\par 2 Mlékem jsem vás živil, a ne pokrmem; nebo jste ješte nemohli pokrmu tvrdších užívati, ano i nyní ješte nemužete.
\par 3 Ješte zajisté telesní jste. Ponevadžt jest mezi vámi nenávist, svárové a ruznice, zdaž ješte telesní nejste? a tak podle cloveka chodíte.
\par 4 Nebo když nekdo ríká: Ját jsem Pavluv, jiný pak: Já Apolluv, zdaliž nejste telesní?
\par 5 Nebo kdo jest Pavel, a kdo jest Apollo, než služebníci, skrze než jste uverili, a jakž jednomu každému dal Pán?
\par 6 Ját jsem štípil, Apollo zaléval, ale Buh dal zrust.
\par 7 A protož ani ten, kdož štepuje, nic není, ani ten, jenž zalévá, ale ten, kterýž zrust dává, Buh.
\par 8 Ten pak, kdož štepuje, a ten, kdož zalévá, jedno jsou, avšak jeden každý vlastní odplatu vezme podle své práce.
\par 9 Božít jsme zajisté pomocníci, Boží rolí, Boží vzdelání jste.
\par 10 Já podle milosti Boží mne dané, jako moudrý stavitel, základ jsem založil, jiný pak na nem staví. Ale jeden každý viz, jak na nem staví.
\par 11 Nebo základu jiného žádný položiti nemuž, mimo ten, kterýž položen jest, jenž jest Ježíš Kristus.
\par 12 Staví-lit pak kdo na ten základ zlato, stríbro, kamení drahé, dríví, seno, strnište,
\par 13 Jednohot každého dílo zjeveno bude. Den zajisté to všecko okáže; nebo v ohni zjeví se, a jednoho každého dílo, jaké by bylo, ohen zprubuje.
\par 14 Zustane-lit cí dílo, kteréž na nem stavel, vezme odplatu.
\par 15 Paklit cí dílo shorí, tent vezme škodu, ale sám spasen bude, avšak tak jako skrze ohen.
\par 16 Zdaliž nevíte, že chrám Boží jste, a Duch Boží v vás prebývá?
\par 17 Jestližet kdo chrámu Božího poskvrnuje, tohot zatratí Buh; nebo chrám Boží svatý jest, jenž jste vy.
\par 18 Žádný sám sebe nesvod. Zdá-li se komu z vás, že jest moudrý na tomto svete, budiž bláznem, aby byl ucinen moudrým.
\par 19 Moudrost zajisté sveta tohoto bláznovství jest u Boha. Nebo psáno jest: Kterýž zlapá moudré v chytrosti jejich.
\par 20 A opet: Znát Pán premyšlování moudrých, že jsou marná.
\par 21 A tak nechlubiž se žádný lidmi; nebo všecky veci vaše jsou.
\par 22 Budto Pavel, budto Apollo, budto Petr, budto svet, budto život, budto smrt, budto prítomné veci, budto budoucí, všecko jest vaše,
\par 23 Vy pak Kristovi, a Kristus Boží.

\chapter{4}

\par 1 Tak o nás smýšlej clovek, jako o služebnících Kristových a šafárích tajemství Božích.
\par 2 Dále pak vyhledává se pri šafárích toho, aby každý z nich verný nalezen byl.
\par 3 Mne pak to za nejmenší vec jest, abych od vás souzen byl, aneb od lidského soudu; nýbrž aniž sám sebe soudím.
\par 4 Nebo ackoli do sebe nic bezbožného nevím, však ne skrze to jsem spravedliv; nebo ten, ješto mne soudí, Pán jest.
\par 5 Protož nesudtež nic pred casem, až by prišel Pán, kterýž i osvítí to, což skrytého jest ve tme, a zjeví rady srdcí. A tehdážt bude míti chválu jeden každý od Boha.
\par 6 Tyto pak veci, bratrí moji, v podobenství obrátil jsem na sebe a na Apollo, pro vás, abyste se na nás ucili nad to, což psáno jest, výše nesmýšleti, a abyste jeden pro druhého nenadýmali se proti nekomu.
\par 7 Nebo kdož te soudí? A co máš, ješto bys nevzal? A když jsi vzal, proc se chlubíš, jako bys nevzal?
\par 8 Již jste nasyceni, již jste zbohatli, bez nás kralujete. Ale ó byste kralovali, abychom i my také spolu s vámi kralovali.
\par 9 Za to mám jiste, že nás Buh apoštoly poslední okázal jako k smrti oddané; nebo ucineni jsme divadlo tomuto svetu, i andelum, i lidem.
\par 10 My blázni pro Krista, ale vy opatrní v Kristu; my mdlí, vy pak silní; vy slavní, ale my opovržení.
\par 11 Až do tohoto casu i lacníme, i žízníme, i nahotu trpíme, i polickováni býváme, i místa nemáme,
\par 12 A pracujeme, delajíce rukama vlastníma; uhaneni jsouce, dobrorecíme; protivenství trpíce, mile snášíme.
\par 13 Když se nám rouhají, modlíme se za ne; jako smeti tohoto sveta ucineni jsme, a jako povrhel u všech, až posavad.
\par 14 Ne proto, abych vás zahanbil, píši toto, ale jako svých milých synu napomínám.
\par 15 Nebo byste pak deset tisíc pestounu meli v Kristu, však proto nemnoho máte otcu. Nebo v Kristu Ježíši skrze evangelium já jsem vás zplodil.
\par 16 Protož prosím vás, budtež následovníci moji.
\par 17 Pro tu prícinu poslal jsem vám Timotea, kterýžto jest syn muj milý a verný v Pánu. Tent vám pripomínati bude, které jsou cesty mé v Kristu, jakž všudy v každé církvi ucím.
\par 18 Rovne jako bych nemel k vám prijíti, tak se naduli nekterí.
\par 19 Ale prijdut k vám brzo, bude-li Pán chtíti, a poznám ne rec tech nadutých, ale moc.
\par 20 Nebot nezáleží v reci království Boží, ale v moci.
\par 21 Co chcete? S metlou-li abych prišel k vám, cili s láskou, a s duchem tichosti?

\chapter{5}

\par 1 Naprosto se slyší, že by mezi vámi bylo smilstvo, a to takové smilstvo, jakéž se ani mezi pohany nejmenuje, totiž aby nekdo mel manželku otce svého.
\par 2 A vy nadutí jste, a nermoutíte se radeji, aby vyvržen byl z prostredku vás ten, kdož takový skutek spáchal.
\par 3 Já zajisté, ac vzdálený telem, ale prítomný duchem, již jsem to usoudil, jako bych prítomen byl, abyste toho, kterýž to tak spáchal,
\par 4 Ve jménu Pána našeho Jezukrista sejdouce se spolu, i s mým duchem, s mocí Pána našeho Jezukrista,
\par 5 Vydali takového satanu k zahubení tela, aby duch spasen byl v den Pána Ježíše.
\par 6 Nenít dobrá chlouba vaše. Zdaliž nevíte, že malicko kvasu všecko testo nakvašuje?
\par 7 Vycisttež tedy starý kvas, abyste byli nové zadelání, jakož pak jste nenakvašeni. Nebot jest Beránek náš velikonocní za nás obetován, Kristus.
\par 8 A protož hodujmež ne v kvasu starém, ani v kvasu zlosti a nešlechetnosti, ale v presnicích uprímosti a pravdy.
\par 9 Psal jsem vám v listu, abyste se nesmešovali s smilníky.
\par 10 Ale ne všelikterak s smilníky tohoto sveta, neb s lakomci, nebo s dráci, aneb s modlári, sic jinak musili byste z tohoto sveta vyjíti.
\par 11 Nyní pak psal jsem vám, abyste se nesmešovali s takovými, kdyby kdo, maje jméno bratr, byl smilník, neb lakomec, neb modlár, neb zlolejce, neb opilec, neb drác. S takovým ani nejezte.
\par 12 Nebo proc já mám i ty, kteríž jsou vne, souditi? Však ty, kteríž jsou vnitr, vy soudíte?
\par 13 Ty pak, kteríž jsou vne, Buh soudí. Vyvrztež tedy toho zlého sami z sebe.

\chapter{6}

\par 1 Smí nekdo z vás, maje pri s druhým, souditi se pred nepravými, a ne radeji pred svatými?
\par 2 Nevíte-liž, že svatí svet souditi budou? I ponevadž od vás souzen býti má svet, kterakž tedy nehodni jste tech nejmenších vecí rozsuzovati?
\par 3 Zdaliž nevíte, že andely souditi budeme? Co pak tyto casné veci?
\par 4 Protož když byste meli míti soud o tyto casné veci, temi, kteríž nejzadnejší jsou v církvi, soud osadte.
\par 5 K zahanbenít vašemu to pravím. Tak-liž není mezi vámi moudrého ani jednoho, kterýž by mohl rozsouditi mezi bratrem a bratrem svým?
\par 6 Ale bratr s bratrem soudí se, a to pred neverícími?
\par 7 Již tedy konecne nedostatek mezi vámi jest, že soudy máte mezi sebou. Proc radeji krivdy netrpíte? Proc radeji škody nebérete?
\par 8 Nýbrž vy krivdu ciníte, a k škode privodíte, a to bratrí své.
\par 9 Zdali nevíte, že nespravedliví dedictví království Božího nedosáhnou? Nemylte se, však ani smilníci, ani modlári, ani cizoložníci, ani zženštilí, ani samcoložníci,
\par 10 Ani zlodeji, ani lakomci, ani opilci, ani zlolejci, ani dráci, dedictví království Božího nedujdou.
\par 11 A takoví jste nekterí byli, ale obmyti jste, ale posveceni jste, ale ospravedlneni jste ve jménu Pána Jezukrista a skrze Ducha Boha našeho.
\par 12 Všecko mi sluší, ale ne všecko prospívá; všecko mi sluší, ale ját pod žádné té veci moc poddán nebudu.
\par 13 Pokrmové brichu náležejí, a bricho pokrmum; Buh pak i pokrmy i bricho zkazí. Ale telo ne smilstvu oddáno býti má, ale Pánu, a Pán telu.
\par 14 Buh pak i Pána Ježíše vzkrísil, i nás také vzkrísí mocí svou.
\par 15 Nevíte-liž, že tela vaše jsou údové Kristovi? Což tedy vezma údy Kristovy, uciním je údy nevestky? Odstup to.
\par 16 Zdaliž nevíte, že kdož se pripojuje k nevestce, jedno telo jest s ní? Nebo dí Písmo: Budou dva jedno telo.
\par 17 Ten pak, jenž se pripojuje Pánu, jeden duch jest s ním.
\par 18 Utíkejte smilstva. Všeliký hrích, kterýžkoli ucinil by clovek, krome tela jest, ale kdož smilní, ten proti svému vlastnímu telu hreší.
\par 19 Zdaliž nevíte, že telo vaše jest chrám Ducha svatého, jenž prebývá v vás, kteréhož máte od Boha, a nejste sami svoji?
\par 20 Nebo koupeni jste za velikou mzdu. Oslavujtež tedy Boha telem vaším i duchem vaším, kteréžto veci Boží jsou.

\chapter{7}

\par 1 O cemž jste mi pak psali, k tomut vám toto odpovídám: Dobrét by bylo cloveku ženy se nedotýkati.
\par 2 Ale pro uvarování se smilstva, jeden každý manželku svou mej a jedna každá mej muže svého.
\par 3 Muž k žene povinnou prívetivost okazuj, a tak podobne i žena muži.
\par 4 Žena vlastního tela svého v moci nemá, ale muž; též podobne i muž tela svého vlastního v moci nemá, ale žena.
\par 5 Nezbavujte jeden druhého, lec by to bylo z spolecného svolení na cas, abyste se uprázdnili ku postu a k modlitbe; a potom zase k témuž se navratte, aby vás nepokoušel satan pro nezdrželivost vaši.
\par 6 Ale tot pravím podle dopuštení, ne podle rozkazu.
\par 7 Nebo chtel bych, aby všickni lidé tak byli jako já, ale jeden každý svuj vlastní dar od Boha má, jeden tak a jiný jinak.
\par 8 Pravím pak neženatým a vdovám: Dobré jest jim, aby tak zustali jako i já.
\par 9 Paklit se nemohou zdržeti, nechažt v stav manželský vstoupí; nebo lépe jest v stav manželský vstoupiti nežli páliti se.
\par 10 Vdaným pak prikazuji ne já, ale Pán, rka: Manželka od muže neodcházej.
\par 11 Paklit by odešla, zustaniž nevdaná, anebo smir se s mužem svým. Tolikéž muž nepropouštej ženy.
\par 12 Jiným pak pravím já, a ne Pán: Má-li který bratr manželku neverící, a ta povoluje býti s ním, nepropouštejž jí.
\par 13 A má-li která žena muže neverícího, a on chce býti s ní, nepropouštej ho.
\par 14 Posvecent jest zajisté neverící muž pro ženu verící, a žena neverící posvecena jest pro muže; sic jinak deti vaši necistí by byli, ale nyní svatí jsou.
\par 15 Paklit neverící odjíti chce, nechat jde. Nenít manem bratr neb sestra v takové veci, ale ku pokoji povolal nás Buh.
\par 16 A kterak ty víš, ženo, získáš-li muže svého? Anebo co ty víš, muži, získáš-li ženu?
\par 17 Ale jakž jednomu každému odmeril Buh a jakž jednoho každého povolal Pán, tak chod. A takt ve všech církvech rídím.
\par 18 Obrezaný nekdo povolán jest? Neuvodiž na sebe neobrezání; pakli kdo v neobrízce povolán, neobrezuj se.
\par 19 Nebo obrízka nic není, a neobrízka také nic není, ale zachovávání prikázání Božích.
\par 20 Jeden každý v povolání tom, jímž povolán jest, zustávej.
\par 21 Služebníkem byv, povolán jsi? Nedbej na to. Pakli bys mohl býti svobodný, radeji toho užívej.
\par 22 Nebo kdož jest v Pánu povolán, byv služebníkem, osvobozený jest Páne. Též podobne kdož jest povolán byv svobodný, služebník jest Kristuv.
\par 23 Za mzdu koupeni jste, nebudtež služebníci lidští.
\par 24 Jeden každý, jakž povolán jest, bratrí, v tom zustávej pred Bohem.
\par 25 O pannách pak prikázání Páne nemám, ale však radu dávám, jakožto ten, jemuž z milosrdenství svého Pán dal verným býti.
\par 26 Za tot pak mám, že jest to dobré pro nastávající potrebu, totiž že jest dobré cloveku tak býti.
\par 27 Privázán-lis k žene, nehledej rozvázání. Jsi-li prost od ženy, nehledej ženy.
\par 28 Pakli bys ses i oženil, nezhrešils, a vdala-li by se panna, nezhrešila; ale trápení tela míti budou takoví; ját pak vám odpouštím.
\par 29 Ale totot vám pravím, bratrí, protože cas ostatní jest ukrácený: náležít tedy, aby i ti, kteríž mají ženy, byli, jako by jich nemeli.
\par 30 A kteríž plací, jako by neplakali, a kteríž se radují, jako by se neradovali, a kteríž kupují, jako by nevládli,
\par 31 A kteríž užívají tohoto sveta, jako by neužívali. Nebot pomíjí zpusob tohoto sveta.
\par 32 A já bych rád chtel, abyste vy bez pecování byli. Nebo kdo ženy nemá, pecuje o to, což jest Páne, kterak by se líbil Pánu.
\par 33 Ale kdo se oženil, pecuje o veci tohoto sveta, kterak by se líbil žene.
\par 34 Rozdílnét jsou jiste žena a panna. Nevdaná pecuje o to, což jest Páne, aby byla svatá i telem i duchem, ale vdaná pecuje o veci sveta, kterak by se líbila muži.
\par 35 Totot pak pravím proto, abych vám to, což jest užitecnejšího, ukázal, ne abych na vás osidlo uvrhl, ale abyste slušne a prípadne Pána se prídrželi bez všeliké roztržitosti.
\par 36 Pakli kdo za neslušnou vec své panne pokládá pomíjení casu k vdání, a tak by se státi melo, ucin, jakžkoli chce, nezhreší. Nechažt ji vdá.
\par 37 Ale kdož jest se pevne ustavil v srdci svém, a není mu toho potreba, ale v moci má vlastní vuli svou, a to uložil v srdci svém, aby choval pannu svou, dobre ciní.
\par 38 A tak i ten, kdož vdává pannu svou, dobre ciní, ale kdo nevdává, lépe ciní.
\par 39 Žena privázána jest k manželství zákonem dotud, dokudž její muž živ jest. Pakli by umrel muž její, svobodná jest; muž se vdáti, za kohož chce, toliko v Pánu.
\par 40 Ale blahoslavenejší jest, zustala-li by tak, podle mého soudu. Mámt pak za to, žet i já mám Ducha Božího.

\chapter{8}

\par 1 O tech pak vecech, kteréž modlám obetovány bývají, víme, že všickni známost máme. A známost nadýmá, ale láska vzdelává.
\par 2 Zdá-li se pak komu, že neco umí, ješte nic nepoznal, tak jakž by mel znáti.
\par 3 Ale jestliže kdo miluje Boha, tent jest vyucen od neho.
\par 4 A protož o pokrmích, kteríž se modlám obetují, toto dím: Víme, že modla na svete nic není a že není jiného žádného Boha nežli jeden.
\par 5 Nebo ackoli jsou nekterí, ješto slovou bohové, i na nebi i na zemi, (jakož jsou mnozí bohové a páni mnozí,)
\par 6 Ale my máme jediného Boha Otce, z nehož všecko, a my v nem, a jednoho Pána Ježíše Krista, skrze nehož všecko, i my skrze neho.
\par 7 Ale ne ve všecht jest to umení. Nebo nekterí se zlým svedomím pro modlu až dosavad jako modlám obetované jedí, a svedomí jejich, jsuci mdlé, poskvrnuje se.
\par 8 Necinít pak nás pokrm vzácných Bohu. Nebo budeme-li jísti, nic tím lepší nebudeme, a nebudeme-li jísti, nic horší nebudeme.
\par 9 Ale vizte, at by snad ta vaše moc nebyla k urážce mdlým.
\par 10 Nebo uzrí-li kdo tebe, majícího známost, a ty sedíš pri pokrmu modlám obetovaném, zdaliž svedomí toho, kterýž jest mdlý, nebude privedeno k tomu, aby také jedl modlám obetované?
\par 11 I zahynet bratr mdlý, (pro tvé to vedení), za kteréhož Kristus umrel.
\par 12 A tak hrešíce proti bratrím, a urážejíce svedomí jejich mdlé, proti Kristu hrešíte.
\par 13 A protož jestližet pohoršuje pokrm bližního mého, nebudu jísti masa na veky, abych nezhoršil bratra svého.

\chapter{9}

\par 1 Zdaliž nejsem apoštol? Zdaliž nejsem svobodný? Zdaliž jsem Jezukrista Pána našeho nevidel? Zdaliž vy nejste práce má v Pánu?
\par 2 Bycht pak jiným nebyl apoštol, tedy vám jsem. Nebo pecet mého apoštolství vy jste v Pánu.
\par 3 Odpoved má pred temi, jenž mne soudí, ta jest:
\par 4 Zdaliž nemáme moci jísti a píti?
\par 5 Zdaliž nemáme moci sestry ženy pri sobe míti, jako i jiní apoštolé, i bratrí Páne, i Petr?
\par 6 Zdaliž sám já a Barnabáš nemáme moci telesných prací zanechati?
\par 7 I kdo bojuje kdy na svuj náklad? Kdo štepuje vinici a jejího ovoce nejí? Anebo kdo pase stádo a mléka od stáda nejí?
\par 8 Zdali podle cloveka to pravím? Zdaliž i Zákon toho nepraví?
\par 9 Nebo v Zákone Mojžíšove psáno jest: Nezavížeš úst volu mlátícímu. I zdali Buh tak o voly pecuje?
\par 10 Cili naprosto pro nás to praví? Pro nást jiste to napsáno jest. Nebo kdo ore, v nadeji orati má; a kdo mlátí v nadeji, nadeje své má úcasten býti.
\par 11 Ponevadž jsme my vám duchovní veci rozsívali, tak-liž jest pak to veliká vec, jestliže bychom my vaše casné veci žali?
\par 12 Kdyžt jiní práva svého k vám užívají, proc ne radeji my? Avšak neužívali jsme práva toho, ale všecko snášíme, abychom žádné prekážky neucinili evangelium Kristovu.
\par 13 Zdaliž nevíte, že ti, kteríž o svatých vecech pracují, z svatých vecí jedí, a kteríž oltári prístojí, s oltárem spolu díl mají?
\par 14 Tak jest i Pán narídil tem, kteríž evangelium zvestují, aby z evangelium živi byli.
\par 15 Ját jsem však niceho toho neužíval. Aniž jsem toho proto psal, aby se to pri mne tak dálo, anot by mi mnohem lépe bylo umríti, nežli aby kdo chválu mou vyprázdnil.
\par 16 Nebo káži-li evangelium, nemám se cím chlubiti, ponevadž jsem to povinen; ale beda by mne bylo, kdybych nekázal.
\par 17 Jestližet pak dobrovolne to ciním, mámt odplatu; pakli bezdeky, úradt jest mi sveren.
\par 18 Jakouž tedy mám odplatu? abych evangelium káže, bez nákladu býti evangelium Kristovo uložil, proto abych zle nepožíval práva svého pri evangelium.
\par 19 Svoboden zajisté jsa ode všech, všechnem sebe samého v službu jsem vydal, abych mnohé získal.
\par 20 A ucinen jsem Židum jako Žid, abych Židy získal; tem, kteríž pod Zákonem jsou, jako bych pod Zákonem byl, abych ty, kteríž pod Zákonem jsou, získal.
\par 21 Tem, kteríž jsou bez Zákona, jako bych bez Zákona byl, (a nejsa bez Zákona Bohu, ale jsa v Zákone Kristu,) abych získal ty, jenž jsou bez Zákona.
\par 22 Ucinen jsem mdlým jako mdlý, abych mdlé získal. Všechnem všecko jsem ucinen, abych vždy nekteré k spasení privedl.
\par 23 A tot ciním pro evangelium, abych úcastník jeho byl.
\par 24 Zdaliž nevíte, že ti, kteríž v závod beží, všickni zajisté beží, ale jeden bére základ? Tak bežte, abyste základu dosáhli.
\par 25 A všeliký, kdož bojuje, ve všem jest zdrželivý. A oni zajisté, aby porušitelnou korunu vzali, jsou zdrželiví, ale my neporušitelnou.
\par 26 Protož já tak bežím, ne jako v nejistotu, tak bojuji, ne jako vítr rozrážeje,
\par 27 Ale podmanuji telo své a v službu podrobuji, abych snad jiným káže, sám nebyl nešlechetný.

\chapter{10}

\par 1 Nechcit pak, abyste nevedeli, bratrí, že otcové naši všickni pod oblakem byli, a všickni more prešli,
\par 2 A všickni v Mojžíše pokrteni jsou v oblace a v mori,
\par 3 A všickni týž pokrm duchovní jedli,
\par 4 A všickni týž nápoj duchovní pili. Pili zajisté z duchovní skály, kteráž za nimi šla; a ta skála byl Kristus.
\par 5 Ale ne ve mnohých z nich zalíbilo se Bohu, nebo zhynuli na poušti.
\par 6 Ty pak veci za príklad nám býti mají k tomu, abychom nebyli žádostivi zlého, jako i oni žádali.
\par 7 Protož nebudte modlári, jako nekterí z nich, jakož psáno jest: Posadil se lid, aby jedl a pil, a vstali, aby hrali.
\par 8 Aniž smilneme, jako nekterí z nich smilnili, a padlo jich jeden den trimecítma tisícu.
\par 9 Ani pokoušejme Krista, jako nekterí z nich pokoušeli, a od hadu zhynuli.
\par 10 Ani repcete, jako i nekterí z nich reptali, a zhynuli od záhubce.
\par 11 Toto pak všecko u figure dálo se jim, a napsáno jest k napomenutí našemu, kteríž jsme již na konci sveta.
\par 12 A protož kdo se domnívá, že stojí, hlediž, aby nepadl.
\par 13 Pokušení vás nezachvátilo, než lidské. Ale vernýt jest Buh, kterýž nedopustí vás pokoušeti nad vaši možnost, ale zpusobít s pokušením také i vysvobození, abyste mohli snésti.
\par 14 Protož, moji milí bratrí, utíkejtež modlárství.
\par 15 Jakožto opatrným mluvím. Vy sudte, co pravím.
\par 16 Kalich dobrorecení, kterémuž dobrorecíme, zdaliž není spolecnost krve Kristovy? A chléb, kterýž lámeme, zdaliž není spolecnost tela Kristova?
\par 17 Nebo jeden chléb, jedno telo mnozí jsme; všickni zajisté z jednoho chleba jíme.
\par 18 Pohledte na Izraele podle tela. Zdaliž ti, kteríž jedí obeti, nejsou úcastníci oltáre?
\par 19 Což pak tedy dím? Že modla jest neco? Anebo že modlám obetované neco jest? Nikoli.
\par 20 Ale toto pravím, že, což obetují pohané, dáblum obetují, a ne Bohu. Nechtelt bych pak, abyste vy byli úcastníci dáblu.
\par 21 Nebo nemužete kalicha Páne píti a kalicha dáblu; nemužete úcastníci býti stolu Páne a stolu dáblu.
\par 22 Cili k hnevu popouzíme Pána? Zdali silnejší jsme nežli on?
\par 23 Všecko mi sluší, ale ne všecko jest užitecné; všecko mi sluší, ale ne všecko vzdelává.
\par 24 Žádný nehledej svých vecí, ale jeden každý toho, což jest bližního.
\par 25 Všecko, což se v masných krámích prodává, jezte, nic se nevyptávajíce pro svedomí.
\par 26 Nebo Pánet jest zeme i plnost její.
\par 27 Pozval-lit by vás pak kdo z neverících k stolu, a chcete jíti, vše, cožkoli bylo by vám predloženo, jezte, nic se nevyptávajíce pro svedomí.
\par 28 Pakli by vám nekdo rekl: Toto jest modlám obetované, nejezte pro toho, jenž oznámil, a pro svedomí. Páne zajisté jest zeme i plnost její.
\par 29 Svedomí pak pravím ne tvé, ale toho druhého. Nebo proc by mela svoboda má potupena býti od cizího svedomí?
\par 30 A ponevadž já s díku cinením požívám, proc mi se rouhají prícinou toho, z cehož já díky ciním?
\par 31 Protož budto že jíte, nebo pijete, anebo cožkoli ciníte, všecko k sláve Boží cinte.
\par 32 Bez úrazu budte i Židum i Rekum i církvi Boží,
\par 33 Jakož i já ve všem líbím se všechnem, nehledaje v tom svého užitku, ale mnohých, aby spaseni byli.

\chapter{11}

\par 1 Následovníci moji budte, jako i já Kristuv.
\par 2 Chválímt pak vás, bratrí, že všecky veci mé v pameti máte, a jakž jsem vydal vám ustanovení, tak je zachováváte.
\par 3 Chcit pak, abyste vedeli, že všelikého muže hlava jest Kristus, a hlava ženy muž, hlava pak Kristova Buh.
\par 4 Každý muž, modle se aneb prorokuje s prikrytou hlavou, ohyžduje hlavu svou.
\par 5 Každá pak žena, modleci se anebo prorokujici s neprikrytou hlavou, ohyžduje hlavu svou; nebo jednostejná vec jest, jako by se oholila.
\par 6 Nebo nezavíjí-lit žena hlavy své, nechažt se také ostríhá. Pakli jest mrzká vec žene oholiti se neb ostríhati, nechažt se zavíjí.
\par 7 Mužt nemá zavíjeti hlavy své, obraz a sláva Boží jsa, ale žena sláva mužova jest.
\par 8 Nebo není muž z ženy, ale žena z muže.
\par 9 Není zajisté muž stvoren pro ženu, ale žena pro muže.
\par 10 Protož mát žena míti obestrení na hlave pro andely.
\par 11 Avšak ani muž bez ženy, ani žena bez muže, v Pánu.
\par 12 Nebo jakož žena jest z muže, tak i muž skrze ženu, všecky pak veci z Boha.
\par 13 Vy sami mezi sebou sudte, sluší-li se žene s neprikrytou hlavou modliti Bohu.
\par 14 Zdaliž vás i samo prirození neucí, žet jest ohyzda muži míti dlouhé vlasy?
\par 15 Ale žene míti dlouhé vlasy poctivé jest; nebo vlasové k zastírání dány jsou jí.
\par 16 Jestliže pak komu se vidí neustupným býti, myt takového obyceje nemáme, ani církev Boží.
\par 17 Toto pak predkládaje, nechválím toho, že ne k lepšímu, ale k horšímu se scházíte.
\par 18 Nejprve zajisté, když se scházíte do shromáždení, slyším, že jsou roztržky mezi vámi, a ponekud tomu verím.
\par 19 Nebot musejí i kacírstva mezi vámi býti, aby práve zbožní zjeveni byli mezi vámi.
\par 20 A tak když se scházíte vespolek, jižt to není veceri Páne jísti,
\par 21 Ponevadž jeden každý nejprv veceri svou prijímá v jedení, a tu nekdo lacní, a jiný se prepil.
\par 22 A což pak domu nemáte k jedení a ku pití? Cili církev Boží tupíte, a zahanbujete ty, kteríž nemají hojnosti pokrmu? Což vám dím? Chváliti budu vás? V tom jiste nechválím.
\par 23 Já zajisté prijal jsem ode Pána, což i vydal jsem vám, že Pán Ježíš v tu noc, v kterouž zrazen jest, vzal chléb,
\par 24 A díky ciniv, lámal a rekl: Vezmete, jezte, to jest telo mé, kteréž se za vás láme. To cinte na mou památku.
\par 25 Takž i kalich, když povecerel, rka: Tento kalich jest ta nová smlouva v mé krvi. To cinte, kolikrátkoli píti budete, na mou památku.
\par 26 Nebo kolikrátž byste koli jedli chléb tento a z kalicha toho pili, smrt Páne zvestujte, dokavadž neprijde.
\par 27 A protož kdokoli jedl by chléb tento a pil z kalicha Páne nehodne, vinen bude telem a krví Páne.
\par 28 Zkusiž tedy sám sebe clovek, a tak chléb ten jez, a z toho kalicha pí.
\par 29 Nebo kdož jí a pije nehodne, odsouzení sobe jí a pije, nerozsuzuje tela Páne.
\par 30 Protož mezi vámi jsou mnozí mdlí a nemocní, a spí mnozí,
\par 31 Ješto kdybychom se sami rozsuzovali, nebyli bychom souzeni.
\par 32 Ale když býváme souzeni, ode Pána býváme poucováni, abychom s svetem nebyli odsouzeni.
\par 33 A tak, bratrí moji, když se scházíte k jedení, jedni na druhé cekávejte.
\par 34 Pakli kdo lacní, doma jez, abyste se nescházeli k odsouzení. Jiné pak veci, když prijdu, zrídím.

\chapter{12}

\par 1 O duchovních pak darích, bratrí, nechci, abyste nevedeli.
\par 2 Víte, že jste byli pohané, kteríž k modlám nemým, jakž jste bývali vedeni, tak jste chodili.
\par 3 Protož známot vám ciním, že žádný v Duchu Božím mluve, nezlorecí Pánu Ježíši, a žádný nemuže ríci Pán Ježíš, jediné v Duchu svatém.
\par 4 Rozdílnít pak darové jsou, ale tentýž Duch,
\par 5 A rozdílná jsou prisluhování, ale tentýž Pán,
\par 6 A rozdílné jsou moci, ale tentýž Buh, jenžto pusobí všecko ve všech.
\par 7 Jednomu pak každému dáno bývá zjevení Ducha k užitku.
\par 8 Nebo nekomu dána bývá skrze Ducha rec moudrosti, jinému pak rec umení podle téhož Ducha,
\par 9 Jinému víra v témž Duchu, jinému darové uzdravování v jednostejném Duchu,
\par 10 Nekomu divu cinení, jinému proroctví, jinému rozeznání duchu, jinému rozlicnost jazyku, jinému vykládání jazyku.
\par 11 Ale to vše pusobí jeden a týž Duch, rozdeluje jednomu každému obzvláštne, jakž rácí.
\par 12 Nebo jakož telo jedno jest a mnoho má údu, ale všickni ti jednoho tela údové, mnozí jsouce, však jedno telo jsou: tak i Kristus.
\par 13 Skrze jednoho zajisté Ducha my všickni v jedno telo pokrteni jsme, budto Židé, budto Rekové, budto služebníci, nebo svobodní, a všickni v jeden duch zapojeni jsme.
\par 14 Nebo telo není jeden úd, ale mnozí.
\par 15 Dí-li noha: Ponevadž nejsem rukou, nejsem z tela, zdaliž proto není z tela?
\par 16 A dí-li ucho: Když nejsem oko, nejsem z tela, zdaliž proto není z tela?
\par 17 Jestliže všecko telo jest oko, kde pak bude sluch? Pakli všecko telo jest sluch, kde povonení?
\par 18 Ale zrídil Buh údy jeden každý z nich v tele, tak jakž jest on chtel.
\par 19 Nebo kdyby byli všickni údové jeden úd, kde by bylo telo?
\par 20 Ale nyní mnozí údové jsou, však jedno telo.
\par 21 A tak nemužt oko ríci ruce: Nepotrebí mi tebe, anebo opet hlava nohám: Nepotrebuji vás.
\par 22 Nýbrž mnohem více údové, kteríž se zdadí nejmdlejší v tele býti, potrební jsou.
\par 23 A kteréž máme za nejméne ctihodné údy v tele, ty vetší ctí pristíráme; a nezdobní údové naši hojnejší ozdobu mají,
\par 24 Ozdobní pak údové naši toho nepotrebují. Ale Buh tak zpusobil telo, poslednejšímu dav hojnejší poctu,
\par 25 Aby nebyla nesvornost v tele, ale aby údové jedni o druhé vespolek pecovali
\par 26 A protož jestliže trpí co jeden úd, spolu s ním trpí všickni údové; pakli jest v sláve jeden úd, radují se spolu s ním všickni údové.
\par 27 Vy pak jste telo Kristovo, a údové z cástky.
\par 28 A nekteré zajisté postavil Buh v církvi nejprv apoštoly, druhé proroky, tretí ucitele, potom moci, potom ty, kterí mají dary uzdravování, pomocníky, správce jiných, rozlicnost jazyku mající.
\par 29 Zdaliž jsou všickni apoštolé? Zdali všickni proroci? Zdali všickni ucitelé? Zdali všickni divy ciní?
\par 30 Zdali všickni mají dary k uzdravování? Zdali všickni jazyky rozlicnými mluví? Zdali všickni vykládají?
\par 31 Snažujtež se pak dojíti daru lepších, a ještet vyšší cestu vám ukáži.

\chapter{13}

\par 1 Bych jazyky lidskými mluvil i andelskými, a lásky kdybych nemel, ucinen jsem jako med zvucící anebo zvonec znející.
\par 2 A bycht mel proroctví, a znal všecka tajemství, i všelikého umení došel, a kdybych mel tak velikou víru, že bych hory prenášel, lásky pak kdybych nemel, nic nejsem.
\par 3 A kdybych vynaložil na pokrmy chudých všecken statek svuj, a bych vydal telo své k spálení, a lásky bych jen nemel, nic mi to neprospívá.
\par 4 Láska trpelivá jest, dobrotivá jest, láska nezávidí, láska není všetecná, nenadýmá se.
\par 5 V nic neslušného se nevydává, nehledá svých vecí, nezpouzí se, neobmýšlí zlého.
\par 6 Neraduje se z nepravosti, ale spolu raduje se pravde.
\par 7 Všecko snáší, všemu verí, všeho se nadeje, všeho trpelive ceká.
\par 8 Láska nikdy nevypadá, ackoli proroctví prestanou, i jazykové utichnou, i ucení v nic prijde.
\par 9 Z cástky zajisté poznáváme a z cástky prorokujeme.
\par 10 Ale jakžt by prišlo dokonalé, tehdyt to, což jest z cástky, vyhlazeno bude.
\par 11 Dokudž jsem byl díte, mluvil jsem jako díte, myslil jsem jako díte, smýšlel jsem jako díte, ale když jsem ucinen muž, opustil jsem detinské veci.
\par 12 Nyní zajisté vidíme v zrcadle a skrze podobenství, ale tehdáž tvárí v tvár. Nyní poznávám z cástky, ale tehdy poznám, tak jakž i známostí obdaren budu.
\par 13 Nyní pak zustává víra, nadeje, láska, to tré, ale nejvetší z nich jestit láska.

\chapter{14}

\par 1 Následujtež tedy lásky, horlive žádejte duchovních vecí, nejvíce však, abyste prorokovali.
\par 2 Nebo ten, jenž mluví cizím jazykem, ne lidem mluví, ale Bohu; nebo žádný neposlouchá, ale duchem vypravuje tajemství.
\par 3 Kdož pak prorokuje, lidem mluví vzdelání, i napomínání, i potešení.
\par 4 Kdož mluví cizím jazykem, sám sebe vzdelává, ale kdož prorokuje, tent církev vzdelává.
\par 5 Chtelt bych pak, abyste všickni jazyky rozlicnými mluvili, ale však radeji, abyste prorokovali. Nebo vetší jest ten, jenž prorokuje, nežli ten, kdož jazyky cizími mluví, lec by také to, což mluví, vykládal, aby se vzdelávala církev.
\par 6 A protož, bratrí, prišel-li bych k vám, jazyky cizími mluve, což vám prospeji, nebudu-lit vám mluviti, bud v zjevení neb v umení, bud v proroctví neb v ucení?
\par 7 Podobne jako i bezdušné veci, vydávající zvuk, jako píštalka nebo harfa, kdyby rozdílného zvuku nevydávaly, kterak by vedíno bylo, co se píská, anebo na harfu hrá?
\par 8 Ano trouba vydala-li by nejistý hlas, kdož se bude strojiti k boji?
\par 9 Tak i vy, nevydali-li byste jazykem svým srozumitelných slov, kterak bude rozumíno, co se mluví? Budete jen u vítr mluviti.
\par 10 Tak mnoho, (jakž vidíme,) rozdílu hlasu jest na svete, a nic není bez hlasu.
\par 11 Protož nebudu-lit znáti moci hlasu, budu tomu, kterýž mluví, cizozemec, a ten, jenž mluví, bude mi také cizozemec.
\par 12 Tak i vy, ponevadž jste horliví milovníci duchovních vecí, toho hledejte, abyste se k vzdelání církve rozhojnili.
\par 13 A protož, kdož mluví jazykem cizím, modl se, aby mohl vykládati.
\par 14 Nebo budu-li se modliti cizím jazykem, duch muj se toliko modlí, ale mysl má bez užitku jest.
\par 15 Což tedy jest? Modliti se budu duchem, a modliti se budu i myslí; plésati budu duchem a plésati budu i myslí.
\par 16 Nebo kdybys ty dobrorecil Bohu duchem, kterakž ten, jenž prostý clovek jest, k tvému dobrorecení rekne Amen, ponevadž neví, co pravíš?
\par 17 Nebo ac ty dobre díky ciníš, ale jiný se nevzdelává.
\par 18 Dekuji Bohu svému, že více nežli vy všickni jazyky cizími mluvím.
\par 19 Ale v sboru radeji bych chtel pet slov srozumitelne promluviti, abych také jiných poucil, nežli deset tisícu slov jazykem neznámým.
\par 20 Bratrí, nebudte deti v smyslu, ale zlostí budte deti, smyslem pak budte dospelí.
\par 21 Psáno jest v Zákone: Že rozlicnými jazyky a cizími rty budu mluviti lidu tomuto, a anižt tak mne slyšeti budou, praví Pán.
\par 22 A tak jazykové jsou za div ne tem, jenž verí, ale neverícím, proroctví pak ne neverícím, ale verícím.
\par 23 A protož když by se sešla všecka církev spolu, a všickni by jazyky cizími mluvili, a vešli by tam i neucení neb neverící, zdaliž nereknou, že blázníte?
\par 24 Ale kdyby všickni prorokovali, a všel by tam mezi ne nekdo neverící nebo neucený, premáhán by byl ode všech a souzen ode všech.
\par 25 A tak tajnosti srdce jeho zjeveny budou, a on padna na tvár, klaneti se bude Bohu, vyznávaje, že jiste Buh jest mezi vámi.
\par 26 Což tedy bratrí? Když se scházíte, jeden každý z vás písen má, ucení má, cizí jazyk má, zjevení má, vykládání má, všecko to budiž k vzdelání.
\par 27 Budto že by kdo jazykem cizím mluvil, at se to deje skrze dva neb nejvíce tri, a to jeden po druhém, a jeden at vykládá.
\par 28 Pakli by nebylo vykladace, nechat mlcí v shromáždení, než sobe sám nechažt mluví a Bohu.
\par 29 Proroci pak dva nebo tri at mluví, a jiní necht rozsuzují.
\par 30 Paklit by jinému tu prísedícímu zjeveno bylo, první mlc.
\par 31 Nebo mužete všickni, jeden po druhém, prorokovati, aby se všickni ucili a všickni se potešovali.
\par 32 Duchovét pak proroku prorokum poddáni jsou.
\par 33 Nebo nenít Buh puvod ruznice, ale pokoje, jakož i ve všech shromáždeních svatých ucím.
\par 34 Ženy vaše v shromáždeních at mlcí, nebo nedopouští se jim mluviti, ale aby poddány byly, jakž i Zákon praví.
\par 35 Pakli se chtí cemu nauciti, doma mužu svých nechat se ptají. Nebo mrzká vec jest ženám mluviti v shromáždení.
\par 36 Zdaliž jest od vás slovo Boží pošlo? Zdali k samým vám prišlo?
\par 37 Zdá-li se sobe kdo býti prorokem nebo duchovním, nechažt pozná, co vám píši, žet jsou prikázání Páne.
\par 38 Pakli kdo neví, nevez.
\par 39 A takž, bratrí, o to se snažte, abyste prorokovali, a jazyky cizími mluviti nezbranujte.
\par 40 Všecko slušne a podle rádu at se deje.

\chapter{15}

\par 1 Známot vám pak ciním, bratrí, evangelium, kteréž jsem zvestoval vám, kteréž jste i prijali, v nemž i stojíte,
\par 2 Skrze kteréž i spasení bérete, kterak kázal jsem vám, pamatujete-li, lec byste nadarmo uverili.
\par 3 Vydal jsem zajisté vám nejprve to, což jsem i vzal, že Kristus umrel za hríchy naše podle Písem,
\par 4 A že jest pohrben a že vstal z mrtvých tretího dne podle Písem.
\par 5 A že vidín jest od Petra, potom od dvanácti.
\par 6 Potom vidín více než od peti set bratrí spolu, z nichžto mnozí ješte živi jsou až dosavad, a nekterí již zesnuli.
\par 7 Potom vidín jest od Jakuba, potom ode všech apoštolu.
\par 8 Nejposléze pak ze všech, jakožto nedochudceti, ukázal se i mne.
\par 9 Nebo já jsem nejmenší z apoštolu, kterýž nejsem hoden slouti apoštol, protože jsem se protivil církvi Boží.
\par 10 Ale milostí Boží jsem to, což jsem, a milost jeho mne ucinená daremná nebyla, ale hojneji nežli oni všickni pracoval jsem, avšak ne já, ale milost Boží, kteráž se mnou jest.
\par 11 Protož i já i oni tak kážeme, a tak jste uverili.
\par 12 Ponevadž se pak káže o Kristu, že jest z mrtvých vstal, kterakž nekterí mezi vámi praví, že by nebylo z mrtvých vstání?
\par 13 Nebo není-lit z mrtvých vstání, anižt jest Kristus z mrtvých vstal.
\par 14 A nevstal-lit jest z mrtvých Kristus, tedyt jest daremné kázaní naše, a daremnát jest i víra vaše.
\par 15 A byli bychom nalezeni i kriví svedkové Boží; nebo vydali jsme svedectví o Bohu, že vzkrísil z mrtvých Krista. Kteréhož nevzkrísil, (totiž) jestliže mrtví z mrtvých nevstávají.
\par 16 Nebo jestližet mrtví z mrtvých nevstávají, anižt jest Kristus vstal.
\par 17 A nevstal-lit jest z mrtvých Kristus, marná jest víra vaše, ješte jste v hríších vašich.
\par 18 A takt i ti, kteríž zesnuli v Kristu, zahynuli.
\par 19 Jestližet pak v tomto živote toliko nadeji máme v Kristu, nejbídnejší jsme ze všech lidí.
\par 20 Ale vstalt jest z mrtvých Kristus, prvotiny tech, jenž zesnuli.
\par 21 Nebo ponevadž skrze cloveka smrt, také i skrze cloveka vzkríšení z mrtvých.
\par 22 Nebo jakož v Adamovi všickni umírají, tak i skrze Krista všickni obživeni budou.
\par 23 Ale jeden každý v svém porádku: Prvotiny Kristus, potom ti, kteríž jsou Kristovi, v príští jeho.
\par 24 Potom bude konec, když vzdá království Bohu a Otci, když vyprázdní všeliké knížatstvo, i všelikou vrchnost i moc.
\par 25 Nebo musít to býti, aby on kraloval, dokudž nepoloží všech neprátel pod nohy jeho.
\par 26 Nejposlednejší pak neprítel zahlazen bude smrt.
\par 27 Nebo všecky veci poddal pod nohy jeho. Když pak praví, že všecky veci poddány jsou jemu, zjevnét jest, že krome toho, kterýž jemu poddal všecko.
\par 28 A když poddáno jemu bude všecko, tedy i on Syn poddá se tomu, kterýž jemu poddati má všecko, aby byl Buh všecko ve všech.
\par 29 Sic jinak co ciní ti, kteríž se krtí za mrtvé? Nevstávají-lit mrtví z mrtvých, i proc se krtí za mrtvé?
\par 30 Proc i my nebezpecenství trpíme každé hodiny?
\par 31 Na každý den umírám, skrze slávu naši, kterouž mám v Kristu Ježíši Pánu našem.
\par 32 Jestližet jsem obycejem jiných lidí bojoval s šelmami v Efezu, co mi to prospeje, nevstanou-lit mrtví? Tedy jezme a pijme, nebo zítra zemreme.
\par 33 Nedejte se svoditi. Porušujít dobré obyceje zlá rozmlouvání.
\par 34 Procitte k konání spravedlnosti, a nehrešte. Nebo nekterí ješte neznají Boha; k zahanbení vašemu toto mluvím.
\par 35 Ale rekne nekdo: Kterakž pak vstanou mrtví? A v jakém tele prijdou?
\par 36 Ó nemoudrý, však to, což rozsíváš, nebývá obživeno, lec umre.
\par 37 A což rozsíváš, ne to telo, kteréž potom zroste, rozsíváš, ale holé zrno, jaké se trefí, pšenice nebo kteréžkoli jiné.
\par 38 Buh pak dává jemu telo, jakž rácí, a jednomu každému z tech semen jeho vlastní telo.
\par 39 Ne každé telo jest jednostejné telo, ale jiné zajisté telo lidské, jiné telo hovadí, jiné pak rybí, a jiné ptací.
\par 40 A jsou tela nebeská, a jsou tela zemská, ale jinát jest sláva nebeských, a jiná zemských,
\par 41 Jiná sláva slunce, a jiná mesíce, a jiná sláva hvezd; nebo hvezda od hvezdy delí se v sláve.
\par 42 Tak i vzkríšení z mrtvých. Rozsívá se telo porušitelné, vstane neporušitelné;
\par 43 Rozsívá se neslicné, vstane slavné; rozsívá se nemocné, vstane mocné;
\par 44 Rozsívá se telo telesné, vstane telo duchovní. Jest telo telesné, jest také i duchovní telo.
\par 45 Takt i psáno jest: Ucinen jest první clovek Adam v duši živou, ale poslední Adam v ducha obživujícího.
\par 46 Však ne nejprve duchovní, ale telesné, potom pak duchovní.
\par 47 První clovek byl z zeme zemský, druhý clovek jest sám Pán s nebe.
\par 48 Jakýž jest ten zemský, takoví jsou i zemští, a jakýž ten nebeský, takovíž budou také i nebeští.
\par 49 A jakož jsme nesli obraz zemského, takt poneseme obraz i nebeského.
\par 50 Totot pak pravím, bratrí: Že telo a krev království Božího dedictví nedosáhnou, aniž porušitelnost neporušitelnosti dedicne obdrží.
\par 51 Aj, tajemství vám pravím: Ne všickni zajisté zesneme, ale všickni promeneni budeme, hned pojednou, v okamžení, k zatroubení poslednímu.
\par 52 Nebot zatroubí, a mrtví vstanou neporušitelní, a my promeneni budeme.
\par 53 Musí zajisté toto porušitelné telo obléci neporušitelnost, a smrtelné toto obléci nesmrtelnost.
\par 54 A když porušitelné toto telo oblece neporušitelnost, a smrtelné toto oblece nesmrtelnost, tehdy se naplní rec, kteráž napsána jest: Pohlcena jest smrt v vítezství.
\par 55 Kde jest, ó smrti, osten tvuj? Kde jest, ó peklo, vítezství tvé?
\par 56 Osten pak smrti jestit hrích, a moc hrícha jest Zákon.
\par 57 Ale Bohu díka, kterýž dal nám vítezství skrze Pána našeho Jezukrista.
\par 58 Protož, bratrí moji milí, stálí budte a nepohnutelní, rozhojnujíce se v díle Páne vždycky, vedouce, že práce vaše není daremná v Pánu.

\chapter{16}

\par 1 O sbírce pak na svaté, jakž jsem narídil v církvích Galatských, tak i vy cinte.
\par 2 V každou nedeli jeden každý z vás sám u sebe slož, schovaje podle možnosti, aby ne teprv, když k vám prijdu, sbírky se dály.
\par 3 Když pak prijdu, kterékoli osoby schválíte skrze listy, tyt pošli, aby donesli tuto milost vaši do Jeruzaléma.
\par 4 Paklit by bylo potrebí, abych i já šel, pujdout se mnou.
\par 5 Prijdut pak k vám, když Macedonií projdu; (nebo Macedonií míním projíti.)
\par 6 Ale u vást snad poostanu, aneb i pres zimu pobudu, abyste vy mne doprovodili, kamž bych koli šel.
\par 7 Nechci zajisté s vámi se nyní toliko na zastavení shledati, ale nadeji se, že za nejaký cas pobudu u vás, bude-li Pán chtíti.
\par 8 Zustanut pak v Efezu až do letnic.
\par 9 Nebo otevríny jsou mi tu veliké a mocné dvere, a protivníku mnoho.
\par 10 Prišel-lit by pak k vám Timoteus, hledtež, aby bezpecne byl u vás; nebot dílo Boží delá jako i já.
\par 11 Protož necht jím žádný nepohrdá, ale vyprovodte jej v pokoji, at prijde ke mne; nebot na nej cekám s bratrími.
\par 12 O Apollovi pak bratru oznamuji vám, že jsem ho velmi prosil, aby šel k vám s bratrími. Ale nikoli nebyla vule jeho, aby nyní prišel, než prijdet, když bude míti cas príhodný.
\par 13 Bdete, stujte u víre, zmužile sobe cinte a budtež silní.
\par 14 Všecky veci vaše at se dejí v lásce.
\par 15 Prosímt vás pak, bratrí, znáte celed Štepánovu, že jsou oni prvotiny Achaie, a že jsou se v službu svatým vydali,
\par 16 Abyste i vy poddáni byli takovým, i všelikému pomáhajícímu a pracujícímu.
\par 17 Tešímt se pak z príchodu Štepána a Fortunáta a Achaika; nebo nedostatek pro vaši neprítomnost oni doplnili.
\par 18 Potešili zajisté mého ducha i vašeho. Protož znejtež takové.
\par 19 Pozdravujít vás sborové, kteríž jsou v Azii. Pozdravujít vás v Pánu velice Akvila a Priscilla, s církví tou, kteráž jest v domu jejich.
\par 20 Pozdravují vás všickni bratrí. Pozdravte sebe vespolek v políbení svatém.
\par 21 Pozdravení vlastní rukou Pavlovou.
\par 22 Jestliže kdo nemiluje Pána Jezukrista, budiž proklatý: Maran atha.
\par 23 Milost Pána Jezukrista budiž s vámi.
\par 24 I láska má v Kristu Ježíši se všemi vámi. Amen.


\end{document}