\begin{document}

\title{Galatians}

\chapter{1}

\par 1 Pavel apoštol, (ne od lidí, ani skrze cloveka, ale skrze Jezukrista, a Boha Otce, kterýž jej vzkrísil z mrtvých,)
\par 2 I ti, kteríž jsou se mnou, všickni bratrí sborum Galatským:
\par 3 Milost vám a pokoj od Boha Otce a Pána našeho Jezukrista,
\par 4 Kterýž vydal sebe samého za hríchy naše, aby nás vytrhl z tohoto prítomného veku zlého, podle vule Boha a Otce našeho,
\par 5 Jemuž bud sláva na veky veku. Amen.
\par 6 Divím se tomu, že tak rychle od toho, kterýž vás povolal v milost Kristovu, uchýlili jste se k jinému evangelium,
\par 7 Kteréž není jiné, ale jsou nekterí, ješto vás kormoutí a prevrátiti chtejí evangelium Kristovo.
\par 8 Ale bychom pak i my neb andel s nebe kázal vám mimo to, což jsme vám kázali, prokletý bud.
\par 9 Jakož jsme již povedeli, a ješte znovu pravím: Jestliže by vám kdo co jiného kázal mimo to, což jste prijali, prokletý bud.
\par 10 Nebo lidské-liž veci, cili Boží predkládám? Zdaliž lidem se líbiti hledám? Kdybych se zajisté ješte lidem zaliboval, služebník Kristuv bych nebyl.
\par 11 Oznamujit pak vám, bratrí, že evangelium to, kteréž kázáno jest ode mne, není podle cloveka.
\par 12 Nebo aniž jsem já ho prijal od cloveka, ani se naucil, ale skrze zjevení Ježíše Krista.
\par 13 Slýchali jste zajisté o mém obcování nekdejším v Židovstvu, kterak jsem se velice protivil církvi Boží a hubil jsem ji,
\par 14 A že jsem prospíval v Židovstvu nad mnohé mne rovné v pokolení mém, byv velmi horlivý milovník otcovských ustanovení.
\par 15 Ale když se zalíbilo Bohu, kterýž mne byl oddelil z života matky mé a povolal skrze milost svou,
\par 16 Zjeviti Syna svého mne, abych jej kázal mezi pohany, hned jsem se neporadil s telem a krví;
\par 17 Aniž jsem se vrátil do Jeruzaléma k tem, jenž prve byli apoštolé nežli já, ale šel jsem do Arabie, prišel jsem pak zase do Damašku.
\par 18 Potom po trech letech navrátil jsem se do Jeruzaléma, abych navštívil Petra, a pobyl jsem u neho patnácte dní.
\par 19 Jiného pak z apoštolu žádného jsem nevidel než Jakuba, bratra Páne.
\par 20 Cožt pak píši vám, aj, pred Bohem osvedcuji, žet neklamám.
\par 21 Potom prišel jsem do krajin Syrských a Cilických.
\par 22 A nebyl jsem známý osobou sborum Židovským, kteríž byli v Kristu,
\par 23 Než toliko slýchali o mne: Že ten, kterýž se nám nekdy protivil, již nyní káže víru, kterouž nekdy hubil.
\par 24 A slavili ve mne Boha.

\chapter{2}

\par 1 Potom po ctrnácti letech opet vstoupil jsem do Jeruzaléma s Barnabášem, pojav s sebou i Tita.
\par 2 Vstoupil jsem pak podle zjevení, a vypravoval jsem jim evangelium, kteréž káži mezi pohany, a zvlášte pak znamenitejším, abych snad nadarmo nebežel nyní i prve.
\par 3 Ale ani Titus, kterýž se mnou byl, pohan byv, nebyl prinucen obrezati se,
\par 4 Totiž pro podešlé falešné bratrí, kteríž se byli vloudili k vyšpehování svobody naší, kterouž máme v Kristu Ježíši, aby nás v službu podrobili.
\par 5 Kterýmžto ani na chvilku neustoupili jsme a nepoddali se, aby pravda evangelium zustala u vás.
\par 6 Od tech pak, kteríž se zdadí neco býti, nic mi není pridáno, ac jací jsou nekdy byli, mne potom nic není, Buht osoby cloveka neprijímá; ti, pravím, kteríž se neco zdadí býti, nic mi nepridali.
\par 7 Nýbrž naodpor, když uzreli, že jest mi svereno evangelium, abych je kázal neobrezaným, jako i Petrovi mezi Židy,
\par 8 (Nebo ten, kterýž mocný byl skrze Petra z strany apoštolství mezi Židy, byl mocný i skrze mne mezi pohany,)
\par 9 A poznavše milost mne danou, Jakub a Petr a Jan, kteríž se zdadí sloupové býti, podali svých pravic mne a Barnabášovi na tovaryšství, abychom my mezi pohany a oni mezi Židy kázali.
\par 10 Toliko napomenuli, abychom na chudé pamatovali, což jsem se i ciniti snažoval.
\par 11 A když byl prišel Petr do Antiochie, zjevne jsem jemu odeprel; hoden zajisté byl trestání.
\par 12 Nebo prve nežli jsou prišli nekterí od Jakuba, jídal s pohany, a když prišli, ucházel a oddeloval se od nich, boje se tech, kteríž byli z Židovstva.
\par 13 A spolu s ním v tom pokrytství byli i jiní Židé, takže i Barnabáš uveden byl v jejich pokrytství.
\par 14 Ale já uzrev, že nesprostne chodí v pravde evangelium, rekl jsem Petrovi prede všemi: Ponevadž ty jsa Žid, pohansky živ jsi, a ne Židovsky, proc pohany k Židovskému zpusobu vedeš?
\par 15 My prirození Židé, a ne pohané hríšní,
\par 16 Vedouce, že nebývá clovek ospravedlnen z skutku Zákona, ale skrze víru v Jezukrista, i my v Krista uverili jsme, abychom ospravedlneni byli z víry Kristovy, a ne z skutku Zákona, protože nebude ospravedlnen z skutku Zákona žádný clovek.
\par 17 Jestliže pak hledajíce ospravedlneni býti v Kristu, nalézáme se i my hríšníci, tedy jest Kristus služebník hrícha? Nikoli.
\par 18 Nebo budu-li to, což jsem zboril, opet zase vzdelávati, prestupníkem sebe ciním.
\par 19 Já zajisté skrze Zákon Zákonu umrel jsem, abych živ byl Bohu.
\par 20 S Kristem ukrižován jsem. Živt jsem pak již ne já, ale živ jest ve mne Kristus. Že pak nyní živ jsem v tele, u víre Syna Božího živ jsem, kterýžto zamiloval mne a vydal sebe samého za mne.
\par 21 Nepohrdámt tou milostí Boží. Nebo jestližet jest z Zákona spravedlnost, tedyt jest Kristus nadarmo umrel.

\chapter{3}

\par 1 Ó nemoudrí Galatští, kdož jest vás tak zmámil, abyste nebyli povolni pravde, kterýmž pred ocima Ježíš Kristus prve byl vypsán a mezi vámi ukrižován?
\par 2 Toto bych jen rád chtel zvedeti od vás, z skutku-li Zákona Ducha svatého jste prijali, cili z slyšení víry?
\par 3 Tak nemoudrí jste? Pocavše Duchem, již nyní telem konáte?
\par 4 Tak mnoho trpeli jste nadarmo? A ješte nadarmo-li.
\par 5 Ten tedy, kterýž vám dává Ducha svatého a ciní divy mezi vámi, z skutku-li Zákona to ciní, cili z slyšení víry?
\par 6 Jako Abraham uveril Bohu, a pocteno jemu to k spravedlnosti.
\par 7 A tak vidíte, že ti, kteríž jsou z víry, ti jsou synové Abrahamovi.
\par 8 Predzvedevši pak Písmo, že z víry ospravedlnuje pohany Buh, predpovedelo Abrahamovi: Že v tobe budou požehnáni všickni národové.
\par 9 A tak ti, kteríž jsou z víry, docházejí požehnání s verným Abrahamem.
\par 10 Kteríž pak koli z skutku Zákona jsou, pod zlorecenstvím jsou. Nebo psáno jest: Zlorecený každý, kdož nezustává ve všem, což jest psáno v knihách Zákona, aby to plnil.
\par 11 A že z Zákona nebývá žádný ospravedlnen pred Bohem, zjevné jest, nebo spravedlivý z víry živ bude.
\par 12 Zákon pak není z víry, ale ten clovek, kterýž by plnil ta prikázání, živ bude v nich.
\par 13 Ale vykoupilt jest nás Kristus z zlorecenství Zákona, ucinen byv pro nás zlorecenstvím, (nebo psáno jest: Zlorecený každý, kdož visí na dreve),
\par 14 Aby na pohany požehnání Abrahamovo prišlo v Kristu Ježíši a abychom zaslíbení Ducha svatého prijali skrze víru.
\par 15 Bratrí, po lidsku pravím: Však utvrzené nekterého cloveka smlouvy žádný neruší, aniž k ní kdo neco pridává.
\par 16 Abrahamovi pak ucinena jsou zaslíbení, i semeni jeho. Nedí: A semenum, jako by o mnohých mluvil, ale jako o jednom: A semenu tvému, jenž jest Kristus.
\par 17 Totot pak pravím: Že smlouvy prve od Boha stvrzené, vztahující se k Kristu, Zákon, kterýž po ctyrech stech a po tridcíti letech zacal se, nevyprazdnuje, tak aby slib Boží v nic obrátil.
\par 18 Nebo jestližet z Zákona pochází dedictví, tedy již ne z zaslíbení. Ale Abrahamovi skrze zaslíbení daroval jest Buh dedictví.
\par 19 Což pak Zákon? Pro prestupování ustanoven jest, dokudž by neprišlo to síme, jemuž se stalo zaslíbení, zpusobený skrze andely v ruce prostredníka.
\par 20 Ale prostredník není jednoho, Buh pak jeden jest.
\par 21 Tedy Zákon jest proti slibum Božím? Odstup to. Nebo kdyby byl Zákon dán, kterýž by mohl obživiti, jiste z Zákona byla by spravedlnost.
\par 22 Ale zavrelo Písmo všecky pod hrích, aby zaslíbení z víry Jezukristovy dáno bylo verícím.
\par 23 Prve pak, nežli prišla víra, pod Zákonem byli jsme ostríháni, zavríni jsouce k té víre, kteráž potom mela zjevena býti.
\par 24 A tak Zákon pestounem naším byl k Kristu, abychom z víry ospravedlneni byli.
\par 25 Ale když prišla víra, již nejsme pod pestounem.
\par 26 Všickni zajisté vy synové Boží jste v Kristu Ježíši skrze víru.
\par 27 Nebo kterížkoli v Krista pokrteni jste, Krista jste oblékli.
\par 28 Nenít ani Žid, ani Rek, ani slouha, ani svobodný, ani muž, ani žena. Nebo všickni vy jedno jste v Kristu Ježíši.
\par 29 A když Kristovi jste, tedy síme Abrahamovo jste, a podle zaslíbení dedicové.

\chapter{4}

\par 1 Pravímt pak, že pokudž dedic malický jest, nic není rozdílný od služebníka, jsa pánem všeho,
\par 2 Ale pod ochráncemi a správcemi jest až do casu uloženého od otce.
\par 3 Tak i my, když jsme byli maliccí, pod živly sveta byli jsme v službu podrobeni.
\par 4 Ale když prišla plnost casu, poslal Buh Syna svého ucineného z ženy, ucineného pod Zákonem,
\par 5 Aby ty, kteríž pod Zákonem byli, vykoupil, abychom právo synu prijali.
\par 6 A že jste synové, protož poslal Buh Ducha Syna svého v srdce vaše, volajícího: Abba, totiž Otce.
\par 7 A tak již nejsi slouha, ale syn, a ponevadž syn, tedy i dedic Boží skrze Krista.
\par 8 Ale tehdáž, nemevše známosti Boha, sloužili jste tem, kteríž z prirození nejsou bohové.
\par 9 Nyní pak, znajíce Boha, nýbrž poznáni jsouce od Boha, kterakž se zpátkem zase obracíte k mdlým a bídným živlum, jimž opet znovu chcete sloužiti?
\par 10 Dnu šetríte, a mesícu, a casu, i let.
\par 11 Bojím se za vás, abych snad nadarmo nepracoval mezi vámi.
\par 12 Budte jako já, neb i já jsem jako vy, bratrí, prosím vás. Nic jste mi neublížili.
\par 13 Neb víte, že s mdlobou tela kázal jsem vám evangelium ponejprve.
\par 14 A pokušení mé, prišlé na telo mé, nebylo u vás málo váženo, aniž jste pohrdli, ale jako andela Božího prijali jste mne, jako Krista Ježíše.
\par 15 Jaké tehdy bylo blahoslavenství vaše? Svedectvít vám zajisté dávám, že, kdyby to možné bylo, oci vaše vyloupíce, byli byste mi dali.
\par 16 Což tedy ucinen jsem vaším neprítelem, pravdu vám prave?
\par 17 Milujít vás nekterí nedobre, nýbrž odstrciti vás chtejí, abyste vy je milovali.
\par 18 Slušnét jest pak horlive milovati v dobrém vždycky, a ne jen toliko tehdáž, když jsem prítomen vám.
\par 19 Synáckové moji, (kteréž opet rodím, až by Kristus zformován byl v vás),
\par 20 Chtel bych pak prítomen vám býti nyní a promeniti hlas svuj; nebo v pochybnosti jsem o vás.
\par 21 Povezte mi, kteríž pod Zákonem chcete býti, nepozorujete-liž Zákona?
\par 22 Nebo psáno jest: Že Abraham mel dva syny, jednoho z služebnice, a druhého z svobodné.
\par 23 Ale ten z služebnice podle tela se narodil, tento pak z svobodné podle zaslíbení.
\par 24 Kteréžto veci u figure se staly. Nebo tot jsou ti dva Zákonové, jeden s hory Sinai, k manství zplozující, a tent jest jako Agar.
\par 25 Agar zajisté jest hora Sinai v Arabii. Dobret se pak k ní trefuje nynejší Jeruzalém, nebo v službu podroben jest s syny svými.
\par 26 Ale ten svrchní Jeruzalém svobodný jest, kterýž jest matka všech nás.
\par 27 Nebo psáno jest: Vesel se neplodná, kteráž nerodíš, vykrikni a zvolej, kteráž nepracuješ ku porodu; nebo ta opuštená mnoho má synu, více nežli ta, kteráž má muže.
\par 28 Myt jsme tedy, ó bratrí, tak jako Izák, synové zaslíbení.
\par 29 Ale jakož tehdáž ten podle tela zplozený protivil se tomu, kterýž byl zplozen podle Ducha, tak se deje i nyní.
\par 30 Než co praví Písmo? Vyvrz služebnici i syna jejího; nebo nebudet dedicem syn služebnice s synem svobodné.
\par 31 A tak, bratrí, nejsmet synové služebnice, ale svobodné.

\chapter{5}

\par 1 Protož v svobode, kterouž Kristus nás osvobodil, stujte a nezapletejtež se zase v jho služebnosti.
\par 2 Aj, já Pavel pravím vám, že budete-li se obrezovati, Kristus vám nic neprospeje.
\par 3 A zase osvedcuji všelikému cloveku, kterýž se obrezuje, že jest povinen všecken Zákon naplniti.
\par 4 Odcizili jste se Krista, kterížkoli v Zákone ospravedlneni býti hledáte; vypadli jste z milosti.
\par 5 My zajisté duchem z víry, nadeje spravedlnosti ocekáváme.
\par 6 Nebo v Kristu Ježíši ani obrízka neprospívá, ani neobrízka, ale víra skrze lásku delající.
\par 7 Beželi jste dobre. Kdo jest vám prekazil, abyste pravdy neposlouchali?
\par 8 Nenít ta rada z toho, kterýž vás volá.
\par 9 Malicko kvasu všecko testo nakvašuje.
\par 10 Ját mám nadeji o vás v Pánu, že nic jiného smýšleti nebudete; ale ten, jenž kormoutí vás, trpeti bude soud, kdožt jest on koli.
\par 11 Já pak, bratrí, káži-li také obrízku, i procež protivenství trpím? Tedy jest vyprázdneno pohoršení kríže.
\par 12 Ó by radeji odrezáni byli, kteríž vás nepokojí.
\par 13 Nebo vy v svobodu povoláni jste, bratrí, toliko abyste té svobody nepokládali sobe za zámysl povolování telu, ale skrze lásku posluhujte sobe vespolek.
\par 14 Nebo všecken Zákon v jednom slovu se zavírá, totiž v tom: Milovati budeš bližního svého jako sebe samého.
\par 15 Pakli se vespolek koušete a žerete, hledtež, abyste jedni od druhých zkaženi nebyli.
\par 16 Ale pravímt: Duchem chodte, a žádosti tela nevykonáte.
\par 17 Nebo telo žádá proti Duchu, a Duch proti telu. Ty pak veci jsou sobe vespolek odporné, tak abyste ne hned, což byste chteli, to cinili.
\par 18 Jestliže pak Duchem vedeni býváte, nejste pod Zákonem.
\par 19 Zjevnít jsou pak skutkové tela, jenž jsou: Cizoložstvo, smilstvo, necistota, chlipnost,
\par 20 Modloslužba, carování, neprátelství, svárové, nenávisti, hnevové, dráždení, ruznice, sekty,
\par 21 Závisti, vraždy, opilství, hodování, a tem podobné veci. Kteréžto kdožkoli ciní, tot vám predpovídám, jakož jsem i prve pravil, že království Božího dedicové nebudou.
\par 22 Ovoce pak Ducha jestit: Láska, radost, pokoj, tichost, dobrotivost, dobrota, vernost, krotkost, stredmost.
\par 23 Proti takovýmt není Zákon.
\par 24 Nebo kteríž jsou Kristovi, tit jsou telo své ukrižovali s vášnemi a s žádostmi.
\par 25 Jsme-lit tedy Duchem živi, Duchem i chodme.
\par 26 Nebývejme marné chvály žádostivi, jedni druhých popouzejíce a jedni druhým závidíce.

\chapter{6}

\par 1 Bratrí, byl-li by zachvácen clovek v nejakém pádu, vy duchovní napravte takového v duchu tichosti, prohlédaje každý sám k sobe, abys snad i ty nebyl pokoušín.
\par 2 Jedni druhých bremena neste a tak plnte zákon Kristuv.
\par 3 Nebo zdá-li se komu, že by neco byl, nic nejsa, takového vlastní mysl jeho svodí.
\par 4 Ale díla svého zkus jeden každý, a takt sám v sobe chválu míti bude, a ne v jiném.
\par 5 Neb jeden každý své bríme ponese.
\par 6 Sdílejž se pak ten, kterýž naucení prijímá v slovu Páne, s tím, od kohož naucení bére, vším svým statkem.
\par 7 Nemylte se, Buht nebude oklamán; nebo cožkoli rozsíval by clovek, tot bude i žíti.
\par 8 A kdož rozsívá telu svému, z tela žíti bude porušení; ale kdož rozsívá Duchu, z Duchat žíti bude život vecný.
\par 9 Ciníce pak dobre, neoblevujme; nebo casem svým budeme žíti, neustávajíce.
\par 10 A protož dokudž cas máme, cinme dobre všechnem, a zvlášte nejvíce pak domácím víry
\par 11 Hle, jaký jsem vám list napsal svou rukou.
\par 12 Ti, kteríž chtejí tvární býti podle tela, nutí vás, abyste se obrezovali, jediné aby protivenství pro kríž Kristuv netrpeli.
\par 13 Nebo ani sami ti, kteríž se obrezují, nezachovávají Zákona, ale chtí, abyste se obrezovali proto, aby se telem vaším chlubili.
\par 14 Ale ode mne odstup to, at abych se v cem chlubil, jediné v kríži Pána našeho Jezukrista, skrze nehož jest mi svet ukrižován, a já svetu.
\par 15 Nebo v Kristu Ježíši ani obrízka nic neplatí, ani neobrízka, ale nové stvorení.
\par 16 A kteríkoli tohoto pravidla následují, pokoj prijde na ne a milosrdenství, i na lid Boží Izraelský.
\par 17 Dále pak žádný mi necin zamestknání, já zajisté jízvy Pána Ježíše nosím na tele svém.
\par 18 Milost Pána Ježíše Krista budiž s duchem vaším, bratrí. Amen.


\end{document}