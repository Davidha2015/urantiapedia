\begin{document}

\title{Ephesians}

\chapter{1}

\par 1 Pavel apoštol Jezukristuv, skrze vuli Boží, svatým, kterí jsou v Efezu, a verným v Kristu Ježíši:
\par 2 Milost vám a pokoj od Boha Otce našeho a Pána Jezukrista.
\par 3 Požehnaný Buh a Otec Pána našeho Jezukrista, kterýž požehnal nám všelikým požehnáním duchovním v nebeských vecech v Kristu.
\par 4 Jakož vyvolil nás v nem pred ustanovením sveta k tomu, abychom byli svatí a neposkvrnení pred oblicejem jeho v lásce,
\par 5 Predzrídiv nás k zvolení za syny skrze Ježíše Krista v sebe, podle dobre libé vule své,
\par 6 K chvále slávy milosti své, kterouž vzácné nás ucinil v milém Synu svém.
\par 7 V nemžto máme vykoupení skrze krev jeho, totiž odpuštení hríchu, podle bohatství milosti jeho,
\par 8 Kterouž rozhojnil k nám ve vší moudrosti a opatrnosti,
\par 9 Oznámiv nám tajemství vule své podle dobré libosti své, kteroužto libost preduložil byl sám v sobe,
\par 10 Aby v dokonání plnosti casu v jedno shromáždil všecko v Kristu, budto nebeské veci, budto zemské.
\par 11 V nem, pravím, v kterémžto i k losu pripušteni jsme, predzrízeni byvše, podle preduložení toho, jenž všecko pusobí podle rady vule své,
\par 12 Abychom tak byli k chvále slávy jeho my, kteríž jsme prve nadeji meli v Kristu.
\par 13 V kterémžto i vy nadeji máte, slyševše slovo pravdy, totiž evangelium spasení vašeho, v nemžto i uverivše, znamenáni jste Duchem zaslíbení Svatým,
\par 14 Kterýž jest závdavek dedictví našeho, na vykoupení toho, což jím dobyto jest, k chvále slávy jeho.
\par 15 Protož i já, slyšev o vaší víre v Pánu Ježíši, a o lásce ke všem svatým,
\par 16 Neprestávám díku ciniti z vás, zmínku cine o vás na modlitbách svých:
\par 17 Aby Buh Pána našeho Jezukrista, Otec slávy, dal vám Ducha moudrosti a zjevení, v poznání jeho,
\par 18 A tak osvícené oci mysli vaší, abyste vedeli, která by byla nadeje povolání jeho a jaké bohatství slávy dedictví jeho v svatých,
\par 19 A kterak jest prevýšená velikost moci jeho k nám verícím podle pusobení mocnosti síly jeho,
\par 20 Kteréž dokázal na Kristu, vzkrísiv jej z mrtvých a posadiv na pravici své na nebesích,
\par 21 Vysoce nade všecko knížatstvo, i mocnosti, i moci, i panstvo, i nad každé jméno, kteréž se jmenuje, netoliko v veku tomto, ale i v budoucím.
\par 22 A všecko poddal pod nohy jeho, a jej dal hlavu nade všecko církvi,
\par 23 Kterážto jest telo jeho a plnost všecko ve všech naplnujícího.

\chapter{2}

\par 1 I vás obživil mrtvé vinami a hríchy,
\par 2 V nichž jste nekdy chodili podle obyceje sveta tohoto a podle knížete mocného v povetrí, ducha toho, kterýž nyní delá v synech zpoury.
\par 3 Mezi nimižto i my všickni žili jsme nekdy v žádostech tela našeho, cinivše to, což se líbilo telu a mysli, a byli jsme z prirození synové hnevu jako i jiní.
\par 4 Ale Buh, bohatý jsa v milosrdenství pro velikou lásku svou, kteroužto zamiloval nás,
\par 5 Když jsme my ješte mrtví byli v hríších, obživil nás spolu s Kristem, jehož milostí jste spaseni,
\par 6 A spolu s ním vzkrísil, i posadil na nebesích v Kristu Ježíši,
\par 7 Aby ukázal v veku budoucím neprevýšené bohatství milosti své, z dobroty své k nám v Kristu Ježíši.
\par 8 Nebo milostí spaseni jste skrze víru, a to ne sami z sebe: dart jest to Boží,
\par 9 Ne z skutku, aby se nekdo nechlubil.
\par 10 Jsme zajisté jeho dílo, jsouce stvoreni v Kristu Ježíši k skutkum dobrým, kteréž Buh pripravil, abychom v nich chodili.
\par 11 Protož pamatujte, že vy byli nekdy pohané podle tela, kteríž jste slouli neobrízka od tech, kteríž slouli obrízka na tele, kteráž se pusobí rukama,
\par 12 A že jste byli za onoho casu bez Krista, odcizeni od spolecnosti Izraele, a cizí od úmluv zaslíbení, nadeje nemající, a bez Boha na svete.
\par 13 Ale nyní v Kristu Ježíši vy, kteríž jste nekdy byli dalecí, blízcí ucineni jste skrze krev Kristovu.
\par 14 Nebo ont jest pokoj náš, kterýž ucinil oboje jedno, zboriv hradbu delící na ruzno,
\par 15 A neprátelství, totiž Zákon prikázání, záležející v ustanoveních, vyprázdniv skrze telo své, aby dvoje vzdelal v samém sobe v jednoho nového cloveka, tak cine pokoj,
\par 16 A v mír uvode oboje v jednom tele Bohu skrze kríž, vyhladiv neprátelství skrze nej.
\par 17 A prišed, zvestoval pokoj vám, dalekým i blízkým.
\par 18 Nebot skrze neho obojí máme prístup v jednom Duchu k Otci.
\par 19 Aj, již tedy nejste hosté a príchozí, ale spolumeštané svatých a domácí Boží,
\par 20 Vzdelaní na základ apoštolský a prorocký, kdežto jest gruntovní úhelný kámen sám Ježíš Kristus,
\par 21 Na nemžto všecko stavení príslušne vzdelané roste v chrám svatý v Pánu;
\par 22 Na kterémžto i vy spolu vzdeláváte se v príbytek Boží, v Duchu svatém.

\chapter{3}

\par 1 Pro tu vec já Pavel, jsem vezen Krista Ježíše pro vás pohany,
\par 2 Jestliže však jste slyšeli o milosti Boží, kteréž jest mi udeleno k prisluhování vám,
\par 3 Že skrze zjevení oznámil mi tajemství, (jakož jsem vám prve psal krátce;
\par 4 Z cehož mužete, ctouce, porozumeti známosti mé v tajemství Kristovu;)
\par 5 Kteréž za jiných veku nebylo známo synum lidským, tak jako nyní zjeveno jest svatým apoštolum jeho a prorokum skrze Ducha,
\par 6 Totiž že by meli býti pohané spoludedicové a jednotelní, i spoluúcastníci zaslíbení jeho v Kristu skrze evangelium,
\par 7 Kteréhož já ucinen jsem slouha z daru milosti Boží mne dané, podle pusobení moci jeho.
\par 8 Mne nejmenšímu ze všech svatých dána jest milost ta, abych mezi pohany zvestoval nestihlá bohatství Kristova,
\par 9 A vysvetlil všechnem, kteraké by bylo obcování tajemství skrytého od veku v Bohu, kterýž všecko stvoril skrze Ježíše Krista,
\par 10 Aby nyní oznámena byla knížatstvu a mocem na nebesích skrze církev rozlicná moudrost Boží,
\par 11 Podle preduložení vecného, kteréž jest uložil v Kristu Ježíši Pánu našem,
\par 12 V nemžto máme smelost a prístup s doufáním skrze víru jeho.
\par 13 Protož prosím, abyste nehynuli v cas soužení mých pro vás, kteráž jsou sláva vaše.
\par 14 A pro tu prícinu klekám na kolena svá pred Otcem Pána našeho Jezukrista,
\par 15 Z nehožto všeliká rodina na nebi i na zemi jmenuje se,
\par 16 Aby vám dal, podle bohatství slávy své, mocí posilnenu býti skrze Ducha svého na vnitrním cloveku,
\par 17 Aby Kristus skrze víru prebýval v srdcích vašich,
\par 18 Abyste v lásce vkoreneni a založeni jsouce, mohli stihnouti se všemi svatými, kteraká by byla širokost, a dlouhost, a hlubokost, a vysokost,
\par 19 A poznati prenesmírnou lásku Kristovu, abyste tak naplneni byli ve všelikou plnost Boží.
\par 20 Tomu pak, kterýž mocen jest nade všecko uciniti mnohem hojneji, nežli my prosíme aneb rozumíme, podle moci té, kterouž delá v nás,
\par 21 Tomu, pravím, bud sláva v církvi svaté skrze Krista Ježíše po všecky veky veku. Amen.

\chapter{4}

\par 1 Protož prosímt vás já vezen v Pánu, abyste hodne chodili, jakž sluší na povolání vaše, kterýmž povoláni jste,
\par 2 Se vší pokorou, tichostí, i s snášelivostí, snášejíce se vespolek v lásce,
\par 3 Usilujíce zachovávati jednotu Ducha v svazku pokoje.
\par 4 Jedno jest telo, a jeden Duch, jakož i povoláni jste v jedné nadeji povolání vašeho.
\par 5 Jeden Pán, jedna víra, jeden krest,
\par 6 Jeden Buh a Otec všech, kterýž jest nade všecko, a skrze všecko, i ve všech vás.
\par 7 Jednomu pak každému z nás dána jest milost podle míry obdarování Kristova.
\par 8 Protož dí Písmo: Vstoupiv na výsost, jaté vedl vezne, a dal dary lidem.
\par 9 Ale to, že vstoupil, co jest, jediné že i sstoupil prve do nejnižších stran zeme?
\par 10 Ten pak, jenž sstoupil, ont jest, kterýž i vstoupil vysoko nade všecka nebesa, aby naplnil všecko.
\par 11 A ont jest dal nekteré zajisté apoštoly, nekteré pak proroky, jiné evangelisty, jiné pastýre a ucitele,
\par 12 Pro sporádání svatých, k dílu služebnosti, k vzdelání tela Kristova,
\par 13 Až bychom se sbehli všickni v jednotu víry a známosti Syna Božího, v muže dokonalého, v míru postavy plného veku Kristova,
\par 14 Abychom již více nebyli deti, zmítající se a tocící každým vetrem ucení v neustavicnosti lidské, v chytrosti k oklamávání lstivému;
\par 15 Ale pravdu ciníce v lásce, rostme v nej všelikterak, v toho, kterýž jest hlava, totiž Kristus,
\par 16 Z kteréhožto všecko telo príslušne spojené a svázané po všech kloubích prisluhování, podle vnitrní moci v míru jednoho každého údu, zrust tela ciní, k vzdelání svému v lásce.
\par 17 A protož totot pravím a osvedcuji v Pánu, abyste již více nechodili, jako i jiní pohané chodí, v marnosti mysli své,
\par 18 Zatemnení v rozumu, odcizeni jsouce od života Božího, pro neznámost, kteráž jest v nich z zatvrzení srdce jejich.
\par 19 Kterížto zoufavše sobe, vydali se v nestydatost, aby všelikou necistotu páchali s chtivostí.
\par 20 Ale vy ne tak jste se vyucili následovati Krista,
\par 21 Ac jestliže jste ho slyšeli a o nem byli vyuceni, jakž ta jest pravda v Ježíšovi,
\par 22 Totiž, složiti ono první obcování podle starého cloveka, rušícího se, podle žádostí oklamávajících,
\par 23 Obnoviti se pak duchem mysli vaší,
\par 24 A obléci nového cloveka, podle Boha stvoreného, v spravedlnosti a v svatosti pravdy.
\par 25 Protož složíce lež, mluvtež pravdu jeden každý s bližním svým; nebo jsme vespolek údové.
\par 26 Hnevejte se, a nehrešte; slunce nezapadej na hnevivost vaši.
\par 27 Nedávejte místa dáblu.
\par 28 Kdo kradl, již více nekrad, ale radeji pracuj, delaje rukama svýma, což dobrého jest, aby mel z ceho udeliti nuznému.
\par 29 Žádná rec mrzutá nevycházej z úst vašich, ale at jest každé promluvení dobré k vzdelání užitecnému, aby dalo milost posluchacum.
\par 30 A nezarmucujte Ducha svatého Božího, kterýmžto znamenáni jste ke dni vykoupení.
\par 31 Všeliká horkost, a rozzlobení se, i hnev, i krik, i rouhání bud odjato od vás, se vší zlostí,
\par 32 Ale budte k sobe vespolek dobrotiví, milosrdní, odpouštejíce sobe vespolek, jakož i Buh v Kristu odpustil vám.

\chapter{5}

\par 1 Budtež tedy následovníci Boží, jakožto synové milí.
\par 2 A chodtež v lásce, jakož i Kristus miloval nás, a vydal sebe samého za nás, dar a obet Bohu u vuni rozkošnou.
\par 3 Smilstvo pak a všeliká necistota, neb lakomství, aniž jmenováno bud mezi vámi, jakož sluší na svaté,
\par 4 A tolikéž mrzkost, ani bláznové mluvení, ani šprýmování, kteréžto veci jsou nenáležité, ale radeji at jest díku cinení.
\par 5 Víte zajisté o tom, že žádný smilník aneb necistý, ani lakomec, (jenž jest modloslužebník,) nemá dedictví v království Kristove a Božím.
\par 6 Žádný vás nesvod marnými recmi; nebo pro takové veci prichází hnev Boží na syny nepoddané.
\par 7 Nebývejtež tedy úcastníci jejich.
\par 8 Byli jste zajisté nekdy temnosti, ale již nyní jste svetlo v Pánu. Jakožto synové svetla chodte,
\par 9 (Nebo ovoce Ducha záleží ve vší dobrote, a spravedlnosti, a v pravde,)
\par 10 O to stojíce, což by se dobre líbilo Pánu.
\par 11 A neobcujte s skutky neužitecnými tmy, ale radeji je trescete.
\par 12 Nebo což se tajne deje od nich, mrzko jest o tom i mluviti.
\par 13 Všecko pak, což bývá trestáno, od svetla bývá zjeveno; nebo to, což všecko zjevuje, svetlo jest.
\par 14 Protož praví: Probud se ty, jenž spíš, a vstan z mrtvých, a zasvítít se tobe Kristus.
\par 15 Viztež tedy, kterak byste opatrne chodili, ne jako nemoudrí, ale jako moudrí,
\par 16 Vykupujíce cas; nebo dnové zlí jsou.
\par 17 Protož nebývejte neopatrní, ale rozumející, která by byla vule Páne.
\par 18 A neopíjejte se vínem, v nemžto jest prostopášnost, ale naplneni budte Duchem svatým,
\par 19 Mluvíce sobe vespolek v žalmích, a v chvalách, a v písnickách duchovních, zpívajíce a plésajíce v srdcích vašich Pánu,
\par 20 Díky ciníce vždycky ze všeho ve jménu Pána našeho Jezukrista Bohu a Otci,
\par 21 Poddáni jsouce jedni druhým v bázni Boží.
\par 22 Ženy mužum svým poddány budte jako Pánu.
\par 23 Nebo muž jest hlava ženy, jako i Kristus jest hlava církve, a ont jest Spasitel tela.
\par 24 A tak jakož církev poddána jest Kristu, tak i ženy mužum svým at jsou poddány ve všem.
\par 25 Muži milujte ženy své, jako i Kristus miloval církev, a vydal sebe samého za ni,
\par 26 Aby ji posvetil, ocistiv ji obmytím vody v slovu,
\par 27 Aby ji sobe postavil slavnou církev, nemající poskvrny, ani vrásky, neb cokoli takového, ale aby byla svatá a bez úhony.
\par 28 Takt jsou povinni muži milovati ženy své jako svá vlastní tela. Kdo miluje ženu svou, sebet samého miluje.
\par 29 Žádný zajisté nikdy tela svého nemel v nenávisti, ale krmí a chová je, jakožto i Pán církev.
\par 30 Nebot jsme údové tela jeho, z masa jeho a z kostí jeho.
\par 31 A protot opustí clovek otce i matku, a pripojí se k manželce své, i budout dva jedno telo.
\par 32 Tajemství toto veliké jest, ale já pravím o Kristu a o církvi.
\par 33 Avšak i vy, jeden každý z vás, manželku svou tak jako sám sebe miluj. Žena pak at se bojí muže svého.

\chapter{6}

\par 1 Dítky poslouchejte rodicu svých v Pánu; nebot jest to spravedlivé.
\par 2 Cti otce svého i matku, (tot jest prikázaní první s zaslíbením,)
\par 3 Aby dobre bylo tobe a abys byl dlouhoveký na zemi.
\par 4 A vy otcové nepopouzejte k hnevu dítek vašich, ale vychovávejte je v cvicení a v napomínání Páne.
\par 5 Služebníci, budte poslušni pánu telesných s bázní a s strachem, v sprostnosti srdce vašeho, jako Krista,
\par 6 Ne na oko sloužíce, jako ti, jenž se lidem líbiti usilují, ale jako služebníci Kristovi, ciníce vuli Boží z té duše,
\par 7 S dobrou myslí sloužíce, jakožto Pánu, a ne lidem,
\par 8 Vedouce, že, což by koli jeden každý ucinil dobrého, za to odplatu vzíti má ode Pána, budto služebník, budto svobodný.
\par 9 A vy páni též se tak mejte k nim, zanechajíce pohružky, vedouce, že i vy také máte Pána v nebesích, a prijímání osob není u neho.
\par 10 Dále pak, bratrí moji, posilnte se v Pánu a v moci síly jeho.
\par 11 Oblecte se v celé odení Boží, abyste mohli státi proti útokum dábelským.
\par 12 Nebot není bojování naše proti telu a krvi, ale proti knížatstvu, proti mocnostem, proti sveta pánum temností veku tohoto, proti duchovním zlostem, kteréž jsou vysoko.
\par 13 A protož vezmete celé odení Boží, abyste mohli odolati v den zlý, a všecko vykonajíce, státi.
\par 14 Stujtež tedy, majíce podpásaná bedra vaše pravdou, a obleceni jsouce v pancír spravedlnosti,
\par 15 A obuté majíce nohy v hotovost k evangelium pokoje,
\par 16 A zvlášte pak vezmouce štít víry, kterýmž byste mohli všecky šípy ohnivé nešlechetníka toho uhasiti.
\par 17 A lebku spasení vezmete, i mec Ducha, jenž jest slovo Boží,
\par 18 Všelikou modlitbou a prosbou modléce se každého casu v Duchu, a v tom bedlivi jsouce se vší ustavicností a prošením za všecky svaté,
\par 19 I za mne, aby mi dána byla rec v otevrení úst mých svobodne a smele, abych oznamoval tajemství evangelium svatého,
\par 20 Pro nežto úrad svuj konám v retezu tomto, abych tak v nem svobodne a smele mluvil, jakž mne mluviti náleží.
\par 21 A abyste i vy vedeli, co se se mnou deje a co ciním, všecko vám to oznámí Tychikus, milý bratr a verný služebník v Pánu,
\par 22 Jehož jsem pro to samo k vám poslal, abyste vedeli o našich vecech a aby potešil srdcí vašich.
\par 23 Pokoj budiž vám bratrím, a láska s verou, od Boha Otce a Pána Jezukrista.
\par 24 Milost Boží budiž se všemi milujícími Pána našeho Jezukrista v neporušitelnosti. Amen. List tento k Efezským psán byl z Ríma po Tychikovi.


\end{document}