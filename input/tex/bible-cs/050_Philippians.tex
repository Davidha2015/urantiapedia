\begin{document}

\title{Philippians}

\chapter{1}

\par 1 Pavel a Timoteus, služebníci Ježíše Krista, všechnem svatým v Kristu Ježíši, kteríž jsou v meste Filippis, s biskupy a s jáhny:
\par 2 Milost vám a pokoj od Boha Otce našeho a Pána Jezukrista.
\par 3 Dekuji Bohu mému, když se koli na vás rozpomenu,
\par 4 (Vždycky pri každé modlitbe mé, s radostí za všecky za vás prosbu cine,)
\par 5 Z vašeho úcastenství v evangelium, hned od prvního dne až posavad,
\par 6 Jist jsa tím, že ten, kterýž zacal v vás dílo dobré, dokoná až do dne Ježíše Krista,
\par 7 Jakož jest mi spravedlivé tak smysliti o všech vás proto, že vás v srdci mám i v vezení mém, a v obranování i v utvrzování evangelium, vás, pravím, všecky úcastníky milosti mne dané.
\par 8 Svedek mi jest zajisté Buh, kterak po všech po vás toužím v srdci Ježíše Krista.
\par 9 A za to se modlím, aby láska vaše ješte více a více se rozhojnovala v známosti a ve všelikém smyslu,
\par 10 K tomu, abyste zkušením rozeznati mohli užitecné veci od neužitecných, tak abyste byli uprímí a bez úrazu, až ke dni Kristovu,
\par 11 Naplneni jsouce ovocem spravedlnosti, kteréž nesete skrze Jezukrista, k sláve a k chvále Boží.
\par 12 Chcit pak, bratrí, abyste vedeli, že to pokušení, kteréž mne obklícilo, k vetšímu prospechu evangelium prišlo,
\par 13 Takže vezení mé pro Krista rozhlášeno jest po všem císarském dvore, i jinde všudy.
\par 14 A mnozí z bratrí v Pánu, spolehše na vezení mé, hojnejší smelost mají bez strachu mluviti slovo Boží.
\par 15 Nekterí zajisté jen z závisti a navzdoru, nekterí pak také z oblíbení sobe toho Krista káží.
\par 16 Ti pak, kteríž navzdoru Krista zvestují, ne v cistote, domnívají se, že mi k vezení mému soužení pridadí;
\par 17 Kterí pak z lásky, ti vedí, že jsem k obrane evangelium postaven.
\par 18 Ale což pak o to? Nýbrž jakýmkoli zpusobem, bud v samé tvárnosti, bud v pravde Kristus se zvestuje, i z tohot se raduji, a ješte radovati budu.
\par 19 Nebot vím, že mi to prijde k spasení skrze vaši modlitbu a pomoc Ducha Jezukristova,
\par 20 Podle peclivého ocekávání a nadeje mé, že v nicemž nebudu zahanben, ale ve vší doufanlivé smelosti, jako i prve vždycky, tak i nyní veleben bude Kristus na tele mém, budto skrze život, budto skrze smrt.
\par 21 Mne zajisté Kristus i v živote i v smrti ziskem jest.
\par 22 Jest-li mi pak prospešneji živu býti v tele pro práci, tedy nevím, co bych vyvolil.
\par 23 K obémut se k tomu naklonuji, žádost maje umríti, a býti s Kristem, což by mnohem lépe bylo,
\par 24 Ale pozustati ješte v tele potrebneji jest pro vás.
\par 25 Nacež spoléhaje, vím, že pobudu a s vámi se všemi spolu pozustanu k vašemu prospechu a k radosti víry,
\par 26 Aby se vaše ze mne chlouba v Kristu Ježíši rozhojnila, skrze mou opet vám zase prítomnost.
\par 27 Toliko, jakž sluší na úcastníky evangelium Kristova, se chovejte, abych, budto prijda k vám a vida vás, budto vzdálen jsa, slyšel o vás, že stojíte v jednom duchu, jednomyslne pracujíce u víre evangelium,
\par 28 A v nicemž se nestrachujte protivníku, což jest jim jistým znamením zahynutí, vám pak spasení, a to od Boha.
\par 29 Nebo vám jest to z milosti dáno pro Krista, abyste netoliko v neho verili, ale také pro nej i trpeli,
\par 30 Týž boj majíce, jakýž jste pri mne i videli, i nyní o mne slyšíte.

\chapter{2}

\par 1 Protož jest-li jaké potešení v Kristu, jest-li které utešení lásky, jest-li která spolecnost Ducha svatého, jsou-li která milosrdenství a slitování,
\par 2 Naplnte radost mou v tom, abyste jednostejného smyslu byli, jednostejnou lásku majíce, jednodušní jsouce, jednostejne smýšlejíce,
\par 3 Nic necinte skrze svár anebo marnou chválu, ale v pokore jedni druhé za dustojnejší sebe majíce.
\par 4 Nehledejte jeden každý svých vecí, ale každý také toho, což jest jiných.
\par 5 Budiž tedy tatáž mysl pri vás, jakáž byla pri Kristu Ježíši,
\par 6 Kterýž jsa v zpusobu Božím, nepoložil sobe toho za loupež rovný býti Bohu,
\par 7 Ale samého sebe zmaril, zpusob služebníka prijav, podobný lidem ucinen.
\par 8 A v zpusobu nalezen jako clovek, ponížil se, poslušný jsa ucinen až do smrti, a to do smrti kríže.
\par 9 Protož i Buh povýšil ho nade vše a dal jemu jméno, kteréž jest nad každé jméno,
\par 10 Aby ve jménu Ježíše každé koleno klekalo, tech, kteríž jsou na nebesích, a tech, jenž jsou na zemi, i tech, jenž jsou pod zemí,
\par 11 A každý jazyk aby vyznával, že Ježíš Kristus jest Pánem v sláve Boha Otce.
\par 12 A tak, moji milí, jakož jste vždycky poslušni byli, netoliko v prítomnosti mé, ale nyní mnohem více, když jsem vzdálen od vás, s bázní a s tresením spasení své konejte.
\par 13 Buh zajisté jest, kterýž pusobí v vás i chtení i skutecné cinení, podle dobre libé vule své.
\par 14 Všecko pak cinte bez reptání a bez pochybování,
\par 15 Abyste byli bez úhony, a uprímí synové Boží, bez obvinení uprostred národu zlého a prevráceného; mezi kterýmižto svette jakožto svetla na svete,
\par 16 Slovo života zachovávajíce, k chloube mé v den Kristuv, aby bylo vidíno, že jsem ne nadarmo bežel, ani nadarmo pracoval.
\par 17 A bycht pak i obetován byl pro obet a službu víre vaší, raduji se a spolu raduji se se všemi vámi.
\par 18 A též i vy radujte se a spolu radujte se se mnou.
\par 19 Mámt pak nadeji v Kristu Ježíši, že Timotea brzy pošli vám, abych i já pokojné mysli byl, zveda, kterak vy se máte.
\par 20 Nebo žádného tak jednomyslného nemám, kterýž by tak vlastne o vaše veci pecoval.
\par 21 Všickni zajisté svých vecí hledají, a ne tech, kteréž jsou Krista Ježíše.
\par 22 Ale jej zkušeného býti víte, podle toho, že jakož syn s otcem se mnou prisluhoval v evangelium.
\par 23 Tohot hle, nadeji mám, že pošli, jakž jen porozumím, co se bude díti se mnou.
\par 24 Mámt pak nadeji v Pánu, že i sám brzo k vám prijdu.
\par 25 Ale zdálo se mi za potrebné Epafrodita, bratra a pomocníka a spolurytíre mého, vašeho pak apoštola i služebníka, v potrebe mé poslati k vám,
\par 26 Ponevadž takovou mel žádost vás všecky videti, a velmi težek nad tím byl, že jste o nem slyšeli, že by byl nemocen.
\par 27 A bylt jest jiste nemocen, i blízek smrti, ale Buh se nad ním smiloval, a ne nad ním toliko, ale i nade mnou, abych zámutku na zámutek nemel.
\par 28 Protož tím spešneji poslal jsem ho k vám, abyste, vidouce jej zase, zradovali se, a já abych byl bez zámutku.
\par 29 Protož prijmetež jej v Pánu se vší radostí, a mejte takové v uctivosti.
\par 30 Nebot jest pro dílo Kristovo až k smrti se priblížil, opováživ se života svého, jedné aby doplnil to, v cemž jste vy meli nedostatek pri posloužení mne.

\chapter{3}

\par 1 Dále pak, bratrí moji, radujte se v Pánu. Jednostejných vecí vám psáti mne se jiste nestýšte, vám pak to bezpecné jest.
\par 2 Vizte psy, vizte zlé delníky, vizte roztržku.
\par 3 Nebo myt jsme obrízka, kteríž duchem sloužíme Bohu, a chlubíme se v Kristu Ježíši, a nedoufáme v tele,
\par 4 Ackoli i já mohl bych doufati v tele. Zdá-lit se komu jinému, že by mohl doufati v tele, já mnohem více,
\par 5 Obrezán jsa osmého dne, jsa z rodu Izraelského, pokolení Beniaminova, Žid z Židu, podle zákona farizeus,
\par 6 A z strany horlivosti protivník církve, z strany pak spravedlnosti zákonní jsa bez úhony.
\par 7 Ale to, což mi bylo jako zisk, položil jsem sobe pro Krista za škodu.
\par 8 Nýbrž i všecky veci pokládám škodou býti pro vyvýšenost známosti Krista Ježíše Pána mého, pro nejž jsem to všecko ztratil, a mám to jako za smetí, jedné abych Krista získal,
\par 9 A v nem nalezen byl nemající své spravedlnosti, kteráž jest z Zákona, ale tu, kteráž jest z víry Kristovy, spravedlnost, kterážto jest z Boha a u víre záleží,
\par 10 Abych tak poznal jej, a divnou moc vzkríšení jeho, a spolecnost utrpení jeho, pripodobnuje se k smrti jeho,
\par 11 Zda bych tak prišel k vzkríšení z mrtvých.
\par 12 Ne že bych již dosáhl, aneb již dokonalým byl, ale snažne bežím, zda bych i dostihnouti mohl, nacež uchvácen jsem od Krista Ježíše.
\par 13 Bratrí, ját nemám za to, že bych již dosáhl.
\par 14 Ale to jedno ciním, na ty veci, kteréž jsou za mnou, zapomínaje, k tem pak, kteréž jsou prede mnou, úsilne chvátaje, k cíli bežím, k odplate svrchovaného povolání Božího v Kristu Ježíši.
\par 15 Protož kterížkoli jsme dokonalí, to smýšlejme. A pakli v cem jinak smyslíte, i tot vám Buh zjeví.
\par 16 Ale k cemuž jsme již prišli, v tom pri jednostejném zustávejme pravidle a jednostejne smysleme.
\par 17 Spolunásledovníci moji budte, bratrí, a šetrte tech, kteríž tak chodí, jakož máte príklad na nás.
\par 18 Nebot mnozí chodí, o nichž jsem castokrát pravil vám a nyní s plácem pravím, žet jsou neprátelé kríže Kristova,
\par 19 Jichžto konec jest zahynutí, jejichžto buh jest bricho, a sláva jejich v mrzkostech jejich, kteríž o zemské veci stojí.
\par 20 Ale naše meštanství jestit v nebesích, odkudž i Spasitele ocekáváme Pána Jezukrista,
\par 21 Kterýž promení to telo naše ponížené, tak aby bylo podobné k telu slávy jeho, podle mocnosti té, kteroužto on mocen jest i všecky veci podmaniti sobe.

\chapter{4}

\par 1 A tak, bratrí moji milí a prežádoucí, jenž jste radost a koruna má, tak stujte v Pánu, milí.
\par 2 Evody prosím, i Syntychény prosím, aby jednostejne smyslily v Pánu.
\par 3 Ano i tebe prosím, tovaryši muj vlastní, budiž jim pomocen, kteréžto v evangelium spolu se mnou pracovaly, i spolu s Klimentem a s jinými pomocníky mými, jejichžto jména napsána jsou v knize života.
\par 4 Radujte se v Pánu vždycky; opet pravím, radujte se.
\par 5 Stredmost vaše známa bud všechnem lidem. Pán blízko.
\par 6 O nic nebudte pecliví, ale ve všech vecech skrze modlitbu a poníženou žádost s díku cinením prosby vaše známy budte Bohu.
\par 7 A pokoj Boží, kterýž prevyšuje všeliký rozum, hájiti bude srdcí vašich i smyslu vašich v Kristu Ježíši.
\par 8 Dále pak, bratrí, kterékoli veci jsou pravé, kterékoli poctivé, kterékoli spravedlivé, kterékoli cisté, kterékoli milé, kterékoli dobropovestné, jest-li která ctnost, a jest-li která chvála, o tech vecech premyšlujte.
\par 9 Kterýmž jste se i naucili, je i prijali, a slyšeli i videli na mne. Ty veci cinte, a Buh pokoje budet s vámi.
\par 10 Zradoval jsem se v Pánu velice z toho, že již opet zase rozzelenala se péce vaše o mne. Nacež bezpochyby i prve myslili jste, ale nedostalo se vám príhodného casu.
\par 11 Ne proto, že bych jakou nouzi mel, toto pravím; nebo já naucil jsem se dosti míti na tom, což mám.
\par 12 Umímt i snížen býti, umím také i hojnost míti; všudy a ve všech vecech pocvicen jsem, i nasycen býti i lacneti, i hojnost míti i nouzi trpeti.
\par 13 Všecko mohu v Kristu, kterýž mne posiluje.
\par 14 Avšak dobre jste ucinili, úcastni byvše mého soužení.
\par 15 Víte pak i vy, Filipenští, že pri pocátku evangelium, když jsem šel z Macedonie, žádný sbor neudelil mi z strany dání a vzetí, než vy sami.
\par 16 Ano i do Tesaloniky jednou i po druhé, cehož jsem potreboval, poslali jste mi.
\par 17 Ne protože bych hledal daru, ale hledám užitku hojného k vašemu prospechu.
\par 18 Prijalt jsem pak všecko, a hojnet mám, naplnent jsem již, vzav od Epafrodita to, což posláno bylo od vás, k vuni sladkosti, obet vzácnou a libou Bohu.
\par 19 Buh pak muj naplnít všelikou potrebu vaši podle bohatství svého slavne v Kristu Ježíši.
\par 20 Bohu pak a Otci našemu sláva na veky veku. Amen.
\par 21 Pozdravtež všech svatých v Kristu Ježíši. Pozdravujít vás bratrí, kteríž jsou se mnou.
\par 22 Pozdravujít vás všickni svatí, zvlášte pak ti, kteríž jsou z domu císarova.
\par 23 Milost Pána našeho Jezukrista se všemi vámi. Amen. List tento k Filipenským psán jest z Ríma po Epafroditovi.


\end{document}