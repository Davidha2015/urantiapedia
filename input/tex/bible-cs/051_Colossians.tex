\begin{document}

\title{Colossians}

\chapter{1}

\par 1 Pavel apoštol Ježíše Krista, skrze vuli Boží, a Timoteus bratr,
\par 2 Tem, kteríž jsou v meste Kolossis, svatým a verným bratrím v Kristu: Milost vám a pokoj od Boha Otce našeho a Pána Ježíše Krista.
\par 3 Díky ciníme Bohu a Otci Pána našeho Jezukrista, vždycky za vás modléce se,
\par 4 Slyšavše o víre vaší v Kristu Ježíši a o lásce ke všechnem svatým,
\par 5 Pro nadeji složenou vám v nebesích, o níž jste prve slyšeli v slovu pravdy, jenž jest evangelium.
\par 6 Kteréž jest prišlo k vám, jako i na všecken svet, a ovoce nese, jako i u vás, od toho dne, v kterémž jste slyšeli a poznali milost Boží v pravde,
\par 7 Jakož jste se i naucili od Epafry, milého spoluslužebníka našeho, kterýž jest verný k vám služebník Kristuv,
\par 8 Kterýž také oznámil nám lásku vaši v Duchu.
\par 9 Protož i my, od toho dne, jakž jsme to uslyšeli, neprestáváme modliti se za vás a žádati, abyste naplneni byli známostí vule jeho ve vší moudrosti a rozumnosti duchovní,
\par 10 Abyste chodili hodne Pánu ke vší jeho líbeznosti, v každém skutku dobrém, ovoce vydávajíce a rostouce v známosti Boží,
\par 11 Všelikou mocí zmocneni jsouce, podle síly slávy jeho, ke vší trpelivosti a dobrotivosti s radostí,
\par 12 Díky ciníce Otci, kterýž hodné nás ucinil úcastnosti losu svatých v svetle,
\par 13 Kterýž vytrhl nás z moci temnosti a prenesl do království milého Syna svého,
\par 14 V nemžto máme vykoupení skrze krev jeho, totiž odpuštení hríchu,
\par 15 Kterýž jest obraz Boha neviditelného a prvorozený všeho stvorení.
\par 16 Nebo skrze neho stvoreny jsou všecky veci, kteréž jsou na nebi i na zemi, viditelné i neviditelné, budto trunové nebo panstva, budto knížatstva nebo mocnosti; všecko skrze neho a pro neho stvoreno jest.
\par 17 A on jest prede vším a všecko jím stojí.
\par 18 A ont jest hlava tela církve, kterýž jest pocátek a prvorozený z mrtvých, aby tak on ve všem prvotnost držel,
\par 19 Ponevadž se zalíbilo Otci, aby v nem všecka plnost prebývala,
\par 20 A skrze neho aby smíril s sebou všecko, v pokoj uvode skrze krev kríže jeho, skrze nej, pravím, budto ty veci, kteréž jsou na zemi, bud ty, kteréž jsou na nebi.
\par 21 A vás také nekdy odcizené a neprátely, v mysli vaší obrácené k skutkum zlým, nyní již smíril,
\par 22 Telem svým skrze smrt, aby vás postavil svaté, a neposkvrnené, a bez úhony pred oblicejem svým,
\par 23 Však jestliže zustáváte u víre založení a pevní, a neuchylujete se od nadeje evangelium, kteréž jste slyšeli, jenž jest kázáno všemu stvorení, kteréž jest pod nebem, jehožto já Pavel ucinen jsem služebník;
\par 24 Kterýž nyní raduji se z utrpení mých, kteráž snáším pro vás, a doplnuji ostatky soužení Kristových na tele svém za jeho telo, jenž jest církev,
\par 25 Jejížto ucinen jsem já služebník, tak jakž mi to sveril Buh na to, abych vám sloužil, a tak naplnil slovo Boží,
\par 26 Jenž jest tajemství skryté od veku a národu, nyní pak zjevené svatým jeho.
\par 27 Jimžto Buh rácil známo uciniti, kteraké by bylo bohatství slavného tajemství tohoto mezi pohany, jenž jest prebývání Krista v vás, kterýž jest nadeje slávy,
\par 28 Kteréhož my zvestujeme, napomínajíce všelikého cloveka a ucíce všelikého cloveka ve vší moudrosti, abychom postavili každého cloveka dokonalého v Kristu Ježíši.
\par 29 O cež i pracuji, bojuje podle té jeho mocnosti, kteráž delá ve mne dílo své mocne.

\chapter{2}

\par 1 Chcit zajisté, abyste vedeli, kterakou nesnáz mám o vás a o ty, kteríž jsou v Laodicii a kterížkoli nevideli tvári mé podle tela,
\par 2 Aby potešena byla srdce jejich, spojená v lásce, a to ke všemu bohatství prejistého smyslu, ku poznání tajemství Boha i Otce i Krista Pána,
\par 3 V nemžto jsou skryti všickni pokladové moudrosti a známosti.
\par 4 A totot pravím proto, aby vás žádný, falešne dovode, neoklamal podobnou k pravde recí.
\par 5 Nebo ackoli vzdálen jsem telem, však duchem s vámi jsem, raduje se, a vida rád váš a utvrzení té víry vaší v Krista.
\par 6 Protož jakož jste prijali Krista Ježíše Pána, tak v nem chodte,
\par 7 Vkorenení a vzdelaní na nem, a utvrzení u víre, jakž jste nauceni, rozhojnujíce se v ní s díku cinením.
\par 8 Hledtež, at by vás nekdo neobloupil moudrostí sveta a marným zklamáním, uce podle ustanovení lidských, podle živlu sveta, a ne podle Krista.
\par 9 Nebo v nem prebývá všecka plnost Božství telesne.
\par 10 A vy v nem jste doplneni, kterýžto jest hlava všeho knížatstva i mocnosti,
\par 11 V nemžto i obrezáni jste obrezáním ne rukou ucineným, svlékše telo hríchu v obrezání Kristovu,
\par 12 Pohrbeni jsouce s ním ve krtu, skrze kterýžto i spolu s ním z mrtvých vstali jste skrze víru, jíž jste došli z mocnosti Boží, kterýž vzkrísil jej z mrtvých.
\par 13 A vás, ješte mrtvé v hríších a v neobrízce tela vašeho, spolu s ním obživil, odpustiv vám všecky hríchy,
\par 14 A smazav proti nám celící zápis, jenž záležel v ustanoveních, a byl odporný nám, i vyzdvihl jej z prostredku, pribiv jej k kríži;
\par 15 A obloupiv knížatstva i moci, vedl je na odivu zjevne, triumf slaviv nad nimi skrze nej.
\par 16 Protož žádný vás nesud pro pokrm, aneb pro nápoj, anebo z strany nekterého svátku, nebo novumesíce, anebo sobot,
\par 17 Kteréžto veci jsou stín budoucích, ale to, od cehož se stín delal, jestit telo Kristovo.
\par 18 Nižádný vás nepredchvacuj svémyslne pokorou a náboženstvím andelským, v to, cehož nevidel, vysokomyslne se vydávaje, marne se nadýmaje smyslem tela svého,
\par 19 A nedrže se hlavy, od níž všecko telo po kloubích a svazích vzdelané a spojené roste Božím zrustem.
\par 20 Ponevadž tedy zemrevše s Kristem osvobozeni jste od živlu sveta, procež tak jako byste živi byli svetu, temi ceremoniemi dáte se obtežovati?
\par 21 Když ríkáte: Nedotýkej se, ani okoušej, aniž se s tím obírej!
\par 22 Kteréžto všecky veci samým užíváním jich pricházejí k zkáze, a tot vše není než podle prikázání a ucení lidských.
\par 23 Actkoli pak ty veci mají tvárnost moudrosti, v zpusobu té nábožnosti pošmourné a v povrchní pokore i v neodpouštení telu, avšak nejsou v žádné platnosti, když prináležejí toliko k nasycení tela.

\chapter{3}

\par 1 Protož povstali-li jste s Kristem, vrchních vecí hledejte, kdež Kristus na pravici Boží sedí.
\par 2 O svrchní veci pecujte, ne o zemské.
\par 3 Nebo zemreli jste, a život váš skryt jest s Kristem v Bohu.
\par 4 Když se pak ukáže Kristus, život náš, tehdy i vy ukážete se s ním v sláve.
\par 5 Protož mrtvete údy své zemské, smilstvo, necistotu, chlípnost, žádost zlou, i lakomství, jenž jest modlám sloužení.
\par 6 Pro kteréžto veci prichází hnev Boží na syny zpurné.
\par 7 V kterýchžto hríších i vy nekdy chodili jste, když jste živi byli v nich.
\par 8 Ale již nyní složtež i vy všecko to, hnev, prchlivost, zlobivost, zlorecení i mrzkomluvnost zapudte od úst vašich.
\par 9 Nelžete jedni na druhé, když jste svlékli s sebe starého cloveka s skutky jeho,
\par 10 A oblékli toho nového, obnovujícího se k známosti jasné, podle obrazu toho, jenž jej stvoril,
\par 11 Kdežto není Rek a Žid, obrízka a neobrízka, cizozemec a Scýta, služebník a svobodný ale všecko a ve všech Kristus.
\par 12 Protož oblectež se jako vyvolení Boží, svatí, a milí, v srdce lítostivé, v dobrotivost, nízké o sobe smýšlení, krotkost, trpelivost,
\par 13 Snášejíce jeden druhého a odpouštejíce sobe vespolek, mel-li by kdo proti komu jakou žalobu; jako i Kristus odpustil vám, tak i vy.
\par 14 Nadto pak nade všecko obleceni budte v lásku, kteráž jest svazek dokonalosti.
\par 15 A pokoj Boží víteziž v srdcích vašich, k nemuž i povoláni jste v jedno telo; a budtež vdecni.
\par 16 Slovo Kristovo prebývejž v vás bohate se vší moudrostí; ucíce a napomínajíce sebe vespolek Žalmy, a zpevy, a písnickami duchovními, s milostí zpívajíce v srdci vašem Pánu.
\par 17 A všecko, cožkoli ciníte v slovu nebo v skutku, všecko cinte ve jménu Pána Ježíše, díky ciníce Bohu a Otci skrze neho.
\par 18 Ženy poddány budte mužum svým tak, jakž sluší, v Pánu.
\par 19 Muži milujte ženy své, a nemejtež se prísne k nim.
\par 20 Dítky poslouchejte rodicu ve všem; nebo to jest dobre libé Pánu.
\par 21 Otcové nepopouzejte k hnevivosti dítek svých, aby sobe nezoufaly.
\par 22 Služebníci poslušni budte ve všem pánu svých telesných, ne na oko toliko sloužíce, jako ti, jenž se usilují lidem líbiti, ale v sprostnosti srdce, bojíce se Boha.
\par 23 A všecko, což byste koli cinili, z té duše cinte, jakožto Pánu, a ne lidem,
\par 24 Vedouce, že ode Pána vzíti máte odplatu vecného dedictví; nebo Pánu Kristu sloužíte.
\par 25 Kdož by pak nepravost páchal, odmenu své nepravosti vezme. A nenít prijímání osob u Boha.

\chapter{4}

\par 1 Páni spravedlive a slušne s služebníky svými nakládejte, vedouce, že i vy Pána máte v nebesích.
\par 2 Na modlitbe budtež ustavicní, bdíce v tom s díku cinením,
\par 3 Modléce se spolu i za nás, aby Buh otevrel nám dvere slova, k mluvení o tajemství Kristovu, pro než i v vezení jsem,
\par 4 Abych je zjevoval, tak jakž mi náleží mluviti.
\par 5 Chodtež v moudrosti pred temi, kteríž jsou vne, cas kupujíce.
\par 6 Rec vaše vždycky budiž príjemná, ozdobená solí, tak abyste vedeli, kterak byste meli jednomu každému odpovedíti.
\par 7 O vecech mých o všech oznámí vám Tychikus, bratr milý, a verný slouha Kristuv a spoluslužebník v Pánu;
\par 8 Kteréhož jsem poslal k vám naschvál, aby zvedel, co se deje u vás, a potešil srdcí vašich,
\par 9 S Onezimem, verným a milým bratrem, kterýž jest tam od vás. Tit vám všecko oznámí, co se deje u nás.
\par 10 Pozdravuje vás Aristarchus, spoluvezen muj, a Marek, sestrenec Barnabášuv, (o kterémž jsem vám porucil, prišel-li by k vám, prijmetež jej;)
\par 11 A Jezus, kterýž slove Justus, kterížto jsou Židé. Ti toliko jsou pomocníci moji v kázání o království Božím; tit mi byli ku potešení.
\par 12 Pozdravuje vás Epafras, kterýž od vás jest, slouha Kristuv, kterýž vždycky úsilne pracuje na modlitbách za vás, abyste stáli dokonalí a plní ve vší vuli Boží.
\par 13 Nebo svedectví jemu vydávám, žet vás velmi horlive miluje, a též i ty, kteríž jsou v Laodicii, i kteríž jsou v Hierapoli.
\par 14 Pozdravuje vás Lukáš, lékar, bratr milý, a Démas.
\par 15 Pozdravte bratrí Laodicenských, i Nymfy, i té církve, kteráž jest v domu jeho.
\par 16 A když bude precten u vás tento list, spravtež to, at jest i v Laodicenském sboru cten; a ten, kterýž jest psán z Laodicie, i vy také prectete,
\par 17 A rcete Archippovi: Viz, abys služebnost, kterouž jsi prijal od Pána, vyplnil.
\par 18 Pozdravení mou rukou Pavlovou. Pamatujtež na mé vezení. Milost Boží budiž s vámi. Amen. List tento psán k Kolossenským z Ríma po Tychikovi a Onezimovi.


\end{document}