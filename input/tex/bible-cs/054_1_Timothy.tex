\begin{document}

\title{1 Timoteovi}

\chapter{1}

\par 1 Pavel, apoštol Ježíše Krista, podle zrízení Boha spasitele našeho a pána Jezukrista, nadeje naší,
\par 2 Timoteovi, vlastnímu synu u víre: Milost, milosrdenství a pokoj od Boha Otce našeho a od Krista Ježíše Pána našeho.
\par 3 Jakož jsem prosil tebe, abys pozustal v Efezu, když jsem šel do Macedonie, viziž, abys prikázal nekterým jiného ucení neuciti,
\par 4 Ani oblibovati básní a vypravování rodu, cemuž konce není, odkudž jen hádky pocházejí, více nežli vzdelání Boží, kteréž jest u víre,
\par 5 Ješto konec prikázání jest láska z srdce cistého a z svedomí dobrého a z víry neošemetné.
\par 6 Od cehož nekterí jako od cíle pobloudivše, uchýlili se k marnomluvnosti,
\par 7 Chtíce býti ucitelé Zákona, a nerozumejíce, ani co mluví, ani ceho zastávají.
\par 8 Víme pak, že dobrý jest Zákon, když by ho kdo náležite užíval,
\par 9 Toto znaje, že spravedlivému není uložen Zákon, ale nepravým a nepoddaným, bezbožným a hríšníkum, nešlechetným a necistým, mordérum otcu svých a matek, vražedlníkum,
\par 10 Smilníkum, samcoložníkum, tem, kteríž lidi kradou, lhárum, krivým prísežníkum, a jest-li co jiného, ješto by bylo naodpor zdravému ucení,
\par 11 Jenž jest podle slavného evangelium blahoslaveného Boha, kteréžto mne svereno jest.
\par 12 Protož dekuji tomu, kterýž mne zmocnil, totiž Kristu Ježíši Pánu našemu, že mne za tak verného soudil, aby mne v službe té postavil,
\par 13 Ješto jsem prve byl ruhac, a protivník, a ukrutník. Ale milosrdenství jsem došel; neb jsem to z neznámosti cinil v nevere.
\par 14 Rozhojnila se pak náramne milost Pána našeho, s verou a s milováním, kteréž jest v Kristu Ježíši.
\par 15 Vernát jest tato rec a všelijak oblíbení hodná, že Kristus Ježíš prišel na svet, aby hríšné spasil, z nichžto já první jsem.
\par 16 Ale proto milosrdenství jsem došel, aby na mne prvním dokázal Ježíš Kristus všeliké dobrotivosti, ku príkladu tem, kteríž mají uveriti v neho k životu vecnému.
\par 17 Protož Králi veku nesmrtelnému, neviditelnému, samému moudrému Bohu budiž cest i sláva na veky veku. Amen.
\par 18 Totot prikázání poroucím tobe, synu Timotee, totiž abys podle predešlých o tobe proroctví bojoval v tom dobrý boj,
\par 19 Maje víru a dobré svedomí, kteréžto nekterí potrativše, zhynuli u víre.
\par 20 Z nichžto jest Hymeneus a Alexander, kteréž jsem vydal satanu, aby trestáni jsouce, ucili se nerouhati.

\chapter{2}

\par 1 Napomínámt pak, aby predevším cineny bývaly pokorné modlitby, prosby, svaté žádosti a díku cinení za všelijaké lidi,
\par 2 Za krále i za všecky v moci postavené, abychom tichý a pokojný život vedli ve vší zbožnosti a šlechetnosti.
\par 3 Nebo tot jest dobré a vzácné pred spasitelem naším Bohem,
\par 4 Kterýž chce, aby všelijací lidé spaseni byli a k známosti pravdy prišli.
\par 5 Jedent jest zajisté Buh, jeden také i prostredník Boží a lidský, clovek Kristus Ježíš,
\par 6 Kterýžto dal sebe samého mzdu na vykoupení za všecky, na osvedcení casem svým.
\par 7 K cemuž postaven jsem já za kazatele a apoštola, (pravdut pravím v Kristu a neklamámt,) za ucitele pohanu u víre a pravde.
\par 8 Protož chtel bych, aby se modlili muži na všelikém míste, pozdvihujíce cistých rukou, bez hnevu a bez roztržitosti.
\par 9 Takž také i ženy aby se odevem slušným s stydlivostí a s stredmostí ozdobovaly, ne strojením a krtaltováním sobe vlasu, neb zlatem, anebo perlami, anebo drahým rouchem,
\par 10 Ale (tak, jakž sluší ženám, kteréž dokazují pri sobe zbožnosti,) dobrými skutky.
\par 11 Žena at se ucí mlceci, ve všeliké poddanosti.
\par 12 Nebo žene nedopouštím uciti, ani vládnouti nad mužem, ale aby byla v mlcení.
\par 13 Adam zajisté prve jest stvoren, potom Eva.
\par 14 A Adam nebyl sveden, ale žena svedena jsuci, prícinou prestoupení byla.
\par 15 Ale však spasena bude v plození detí, jestliže by zustala u víre, a v lásce, a v posvecení svém, s stredmostí.

\chapter{3}

\par 1 Vernát jest tato rec, žádá-li kdo biskupství žet výborné práce žádá.
\par 2 Ale musít biskup býti bez úhony, jedné manželky muž, bedlivý, stredmý, vážný, k hostem prívetivý, zpusobný k ucení.
\par 3 Ne pijan vína, ne bitec, ani mrzkého zisku žádostivý, ale mírný, ne svárlivý, ne lakomec,
\par 4 Kterýž by dum svuj dobre spravoval, a dítky své mel v poddanosti se vší šlechetností.
\par 5 (Nebo jestliže kdo domu svého spraviti neumí, kterak o církev Páne pecovati bude?)
\par 6 Ne novák, aby snad nadut jsa, neupadl v potupení dáblovo.
\par 7 A musít také i svedectví dobré míti od tech, kteríž jsou vne, aby neupadl v pohanení a v osidlo dáblovo.
\par 8 Takž podobne jáhnové musejí býti poctiví, ne dvojího jazyku, ne mnoho vína pijící, ne žádostiví mrzkého zisku,
\par 9 Mající tajemství víry v svedomí cistém.
\par 10 A ti také at jsou nejprv zkušeni, a tak at prisluhují, jsouce bez úhony.
\par 11 Též i manželky jejich musejí býti šlechetné, neutrhavé, stredmé, ve všem verné.
\par 12 Jáhnové budtež jedné manželky muži, kteríž by své dítky dobre spravovali i své domy.
\par 13 Nebo kteríž by dobre prisluhovali, dobrého stupne sobe dobudou, a mnohé doufanlivosti u víre, kteráž jest v Kristu Ježíši.
\par 14 Totot píši tobe, maje nadeji, že brzo prijdu k tobe.
\par 15 Paklit prodlím, abys vedel, kterak máš v domu Božím chovati se, jenž jest církev Boha živého, sloup a utvrzení pravdy.
\par 16 A v pravde velikét jest tajemství zbožnosti, že Buh zjeven jest v tele, ospravedlnen v Duchu, ukázal se andelum, kázán jest pohanum, uvereno jemu na svete, vzhuru prijat jest ve slávu.

\chapter{4}

\par 1 Duch pak svetle praví, že v posledních casích odvrátí se nekterí od víry, poslouchajíce duchu bludných a ucení dábelských,
\par 2 V pokrytství lež mluvících, a cejchované majících svedomí své,
\par 3 Zbranujících ženiti se, prikazujících zdržovati se od pokrmu, kteréž Buh stvoril k užívání s díkcinením verícím a tem, jenž poznali pravdu.
\par 4 Nebo všeliké stvorení Boží dobré jest, a nic nemá zamítáno býti, což se s díku cinením prijímá.
\par 5 Posvecuje se zajisté skrze slovo Boží a modlitbu.
\par 6 Toto predkládaje bratrím, budeš dobrý služebník Jezukristuv, vykrmený slovy víry a pravého ucení, kteréhož jsi následoval.
\par 7 Svetské pak a babské básne zavrz, ale cvic se v zbožnosti.
\par 8 Nebo telesné cvicení malého jest užitku, ale zbožnost ke všemu jest užitecná, a má i nynejšího i budoucího života zaslíbení.
\par 9 Vernát jest tato rec a hodná, aby všelijak oblíbena byla.
\par 10 Proto zajisté i pracujeme, i pohanení neseme, že nadeji máme v Bohu živém, kterýž jest spasitel všech lidí, a zvlášte verících.
\par 11 Ty veci predkládej a uc.
\par 12 Nižádný mladostí tvou nepohrdej, ale bud príkladem verných v reci, v lásce, v duchu, u víre, v cistote.
\par 13 Dokudž k tobe neprijdu, budiž pilen cítání, a napomínání, i ucení.
\par 14 Nezanedbávej daru, kterýž jest v tobe, jenž jest dán skrze proroctví s vzkládáním rukou starších na te.
\par 15 O tom premyšluj, v tom bud ustavicne, aby prospech tvuj zjevný byl všechnem.
\par 16 Budiž sebe pilen i ucení, a v tom trvej; nebo to cine, i samého sebe spasíš, i ty, kteríž tebe poslouchají.

\chapter{5}

\par 1 Staršího zurive netresci, ale napomínej jako otce, mladších jako bratrí,
\par 2 Starých žen jako matek, mladic jako sestr, ve vší cistote.
\par 3 Vdovy mej v uctivosti, kteréž pravé vdovy jsou.
\par 4 Pakli která vdova syny nebo vnuky má, nechažt se oni ucí predne k svému domu pobožnosti dokazovati, a zase rodicum se odplacovati; nebot jest to chvalitebné a vzácné pred oblicejem Božím.
\par 5 Kterážt pak práve vdova jest a osamela, mát nadeji v Bohu, a trvát na modlitbách a svatých žádostech dnem i nocí.
\par 6 Ale rozkošná, ta živa jsuci, již umrela.
\par 7 Protož jim to prikaž, at jsou bez úhony.
\par 8 Jestliže pak kdo o své, a zvlášte o domácí péce nemá, zaprelt jest víry, a jest horší nežli neverící.
\par 9 Vdova bud vyvolená, kteráž by nemela méne šedesáti let, kteráž byla jednoho muže manželka,
\par 10 O níž by svedectví bylo, že dobré skutky cinila; a jestliže dítky své zbožne vychovala, do domu pocestné prijímala, jestliže svatým nohy umývala, jestliže bídným posluhovala, jestliže každého skutku dobrého pilná byla.
\par 11 Ale mladých vdov neprijímej; nebo když, nedbajíce na Krista, v chlipnost se vydadí, teprv se vdávati chtejí,
\par 12 Jsouce již hodné odsouzení, protože první víru zrušily.
\par 13 Nýbrž také i zahálejíce, ucí se choditi po domích; a netoliko nedelné, ale jsou i klevetné a všetecné, mluvíce, což nesluší.
\par 14 Protož chci, aby se mladší vdávaly, deti rodily, hospodyne byly, a tak žádné príciny protivníku nedávaly ku pomlouvání.
\par 15 Nebo již se nekteré uchýlily zpet po satanovi.
\par 16 Protož má-lit kdo verící neb která verící vdovy, opatrujž je, a nebud obtežována církev, aby tem, kteréž práve vdovy jsou, postacilo.
\par 17 Predložení, kteríž dobre spravují, dvojí cti hodni jmíni budte, zvlášte ti, kteríž pracují v slovu Božím a v ucení.
\par 18 Nebo praví Písmo: Volu mlátícímu nezavížeš úst. A hodent jest delník své mzdy.
\par 19 Proti staršímu žaloby neprijímej, lec pode dvema nebo trmi svedky.
\par 20 Ty pak, kteríž hreší, prede všemi tresci, aby i jiní bázen meli.
\par 21 Osvedcujit pred oblicejem Božím a Pána Jezukrista, i vyvolených andelu jeho, abys techto vecí ostríhal bez prijímání osob, nic necine podle náchylnosti.
\par 22 Nevzkládej rukou rychle na nižádného, a nepriúcastnuj se hríchum cizím. Sebe samého ostríhej v cistote.
\par 23 Nepij již více vody, ale vína skrovne užívej, pro svuj žaludek a casté nemoci své.
\par 24 Nekterých lidí hríchové prve zjevní jsou, predcházející soud, nekterých pak i následují.
\par 25 A takž také i skutkové dobrí prve zjevní jsou. Což pak jest jinak, ukrýti se nemuže.

\chapter{6}

\par 1 Kterížkoli pod jhem jsou služebníci, at mají pány své za hodné vší uctivosti, aby jméno Boží a ucení jeho nebylo rouháno.
\par 2 Kteríž pak mají pány verící, nechažt jich sobe nezlehcují, protože jsou bratrími, ale tím radeji slouží, protože jsou verící a milí, dobrodiní Božího úcastníci. Tomu vyucuj a napomínej.
\par 3 Jestližet pak kdo jinak ucí, a nepovoluje zdravým recem Pána našeho Jezukrista, a tomu ucení, kteréž jest podle zbožnosti,
\par 4 Takovýt jest nadutý, nic neumeje, ale nemoudrost provodí pri otázkách a hádkách o slova, z kterýchžto pochází závist, svár, rouhání, zlá domnení,
\par 5 Marné hádky lidí na mysli porušených a pravdy zbavených, domnívajích se, že by zbožnost byla zisk telesný. Takových se varuj.
\par 6 Jestit pak zisk veliký zbožnost, s takovou myslí, kteráž na tom, což má, prestati umí.
\par 7 Nic jsme zajisté neprinesli na tento svet, bezpochyby že také nic odnésti nemužeme.
\par 8 Ale majíce pokrm a odev, na tom prestaneme.
\par 9 Kteríž pak chtí zbohatnouti, upadají v pokušení, a v osidlo, a v žádosti mnohé nerozumné a škodlivé, kteréž pohrižují lidi v zahynutí a v zatracení.
\par 10 Koren zajisté všeho zlého jestit milování penez, kterýchžto nekterí žádostivi jsouce, pobloudili od víry a sami se naplnili bolestmi mnohými.
\par 11 Ale ty, ó clovece Boží, takových vecí utíkej, následuj pak spravedlnosti, zbožnosti, víry, lásky, trpelivosti, tichosti.
\par 12 Bojuj ten dobrý boj víry, dosáhni vecného života, k nemuž i povolán jsi, a vyznals dobré vyznání pred mnohými svedky.
\par 13 Prikazujit tobe pred Bohem, kterýž všecko obživuje, a pred Kristem Ježíšem, kterýž osvedcil pred Pontským Pilátem dobré vyznání,
\par 14 Abys ostríhal prikázání tohoto, chovaje se bez poskvrny a bez úhony, až do zjevení se Pána našeho Ježíše Krista,
\par 15 Kteréž casem svým ukáže ten blahoslavený a sám mocný, Král kralujících a Pán panujících,
\par 16 Kterýž sám má nesmrtelnost, a prebývá v svetle nepristupitelném, jehož žádný z lidí nevidel, aniž videti muže, kterémužto cest a síla vecná. Amen.
\par 17 Bohatým tohoto sveta prikazuj, at nejsou vysokomyslní a at nedoufají v nejistém zboží, ale v Bohu živém, kterýž dává nám hojnost všeho ku požívání.
\par 18 A at dobre ciní, a bohatnou v dobrých skutcích, a snadní at jsou k udílení, i prívetiví,
\par 19 Tak sobe skládajíce základ dobrý k casu budoucímu, aby dosáhli vecného života.
\par 20 Ó Timotee, cožt jest svereno, ostríhej, utíkaje bezbožných daremních kriku a odporu falešne nazvaného umení,
\par 21 Kterýmž nekterí chlubíce se, z strany víry pobloudili od cíle. Milost Boží s tebou. Amen. K Timoteovi první list psán byl z Laodicie, mesta hlavního Frygie Pakacianské.


\end{document}