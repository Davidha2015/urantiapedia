\begin{document}

\title{2 Timothy}

\chapter{1}

\par 1 Pavel, apoštol Ježíše Krista, skrze vuli Boží, podle zaslíbení života, kterýž jest v Kristu Ježíši,
\par 2 Timoteovi milému synu: Milost, milosrdenství a pokoj od Boha Otce a Krista Ježíše Pána našeho.
\par 3 Díky ciním Bohu, jemuž sloužím, jako i predkové moji, v svedomí cistém, z toho, že mám na te ustavicnou pamet na svých modlitbách ve dne i v noci,
\par 4 Žádaje videti tebe, zpomínaje na tvé slzy, abych radostí naplnen byl,
\par 5 Rozpomínaje se na tu víru, kteráž v tobe jest bez pokrytství, kteráž byla nejprv v bábe tvé Loide, a v matce tvé Eunice, a tak smyslím, že i v tobe.
\par 6 Pro kteroužto prícinu napomínám tebe, abys roznecoval v sobe dar Boží, kterýžt jest dán skrze vzkládání rukou mých.
\par 7 Nebo nedal nám Buh ducha bázne, ale moci, a milování, a mysli zpusobné.
\par 8 Protož nestyd se za svedectví Pána našeho, ani za mne, vezne jeho, ale citedlen bud úzkostí pricházejících pro evangelium podle moci Boží,
\par 9 Kterýž spasil nás, a povolal povoláním svatým, ne podle skutku našich, ale podle uložení svého a milosti nám dané v Kristu Ježíši pred casy veku,
\par 10 Nyní pak teprv zjevené skrze príští Spasitele našeho Jezukrista, kterýž zahladil smrt, život pak na svetlo vyvedl, i nesmrtelnost skrze evangelium,
\par 11 Jehožto já ustanoven jsem kazatelem, a apoštolem, i ucitelem pohanu.
\par 12 A pro tu prícinu toto všecko trpím, ale nestydímt se za to; nebo vím, komu jsem uveril, a jist jsem tím, že mocen jest toho, což jsem u neho složil, ostríhati až do onoho dne.
\par 13 Mejž jistý príklad zdravých recí, kteréž jsi slýchal ode mne, u víre a v lásce, kteráž jest v Kristu Ježíši.
\par 14 Výborného toho pokladu ostríhej, skrze Ducha svatého prebývajícího v nás.
\par 15 Víš snad o tom, že se odvrátili ode mne všickni, kteríž jsou v Azii, z nichžto jest Fygellus a Hermogenes.
\par 16 Dejž milosrdenství Pán domu Oneziforovu; nebo casto mi cinil pohodlí, aniž se stydel za retezy mé.
\par 17 Nýbrž prišed do Ríma, pilne mne hledal, a nalezl.
\par 18 Dejž jemu Pán nalézti milosrdenství u Pána v onen den. A jak mi mnoho posluhoval v Efezu, ty výborne víš.

\chapter{2}

\par 1 Protož ty, synu muj, zmocniž se v milosti, kteráž jest v Kristu Ježíši.
\par 2 A což jsi slyšel ode mne pred mnohými svedky, sverujž to lidem verným, kteríž by zpusobní byli i jiné uciti.
\par 3 A tak ty snášej protivenství, jako ctný rytír Ježíše Krista.
\par 4 Žádný, kdož ryteruje, neplete se v obecné živnosti, aby se svému hejtmanu líbil.
\par 5 A jestliže by kdo i bojoval, nebudet korunován, lec by rádne bojoval.
\par 6 Pracovati musí i orác, prve nežli užitku okusí.
\par 7 Rozumej, cot pravím, a dejž tobe Pán ve všem smysl pravý.
\par 8 Pamatujž na to, že Ježíš Kristus vstal z mrtvých, jenž jest z semene Davidova, podle evangelium mého.
\par 9 V kterémžto protivenství trpím, až i vezení, jako bych zlocinec byl, ale slovo Boží není u vezení.
\par 10 Protož všecko to snáším pro vyvolené Boží, aby i oni spasení došli, kteréžto jest v Kristu Ježíši, s slavou vecnou.
\par 11 Verná jest tato rec. Nebo jestližet jsme s ním zemreli, tedy také spolu s ním živi budeme.
\par 12 A trpíme-lit, budeme také spolu s ním kralovati; pakli ho zapíráme, i ont nás zapre.
\par 13 A jsme-lit neverní, ont zustává verný; zapríti sám sebe nemuže.
\par 14 Tyto veci pripomínej, s osvedcováním pred oblicejem Páne, a at se o slova nevadí, nebo to k nicemu není užitecné, ale jest ku podvrácení posluchacu.
\par 15 Pilne se snažuj vydati sebe Bohu milého delníka, za nejž by se nebylo proc stydeti, a kterýž by práve slovo pravdy rozdeloval.
\par 16 Bezbožných pak tech kriku daremních varuj se, nebot velmi rozmnožují bezbožnost,
\par 17 A rec jejich jako rak rozjídá se. Z nichžto jest Hymeneus a Filétus,
\par 18 Kteríž pri pravde pobloudili od cíle, pravíce, že by se již stalo vzkríšení, a prevracejí víru nekterých.
\par 19 Ale pevný základ Boží stojí, maje znamení toto: Znát Pán ty, kteríž jsou jeho, a opet: Odstup od nepravosti každý, kdož vzývá jméno Kristovo.
\par 20 V domu pak velikém netoliko jsou nádoby zlaté a stríbrné, ale také drevené i hlinené, a nekteré zajisté ke cti, nekteré pak ku potupe.
\par 21 Protož jestliže by se kdo ocistil od tech vecí, bude nádobou ke cti, posvecenou, a užitecnou Pánu, ke všelikému skutku dobrému hotovou.
\par 22 Mládencích pak žádostí utíkej, ale následuj spravedlnosti, víry, lásky, pokoje, s temi, kteríž vzývají Pána z srdce cistého.
\par 23 Bláznivých pak a nevzdelavatelných otázek varuj se, veda, že plodí sváry.
\par 24 Na služebníka pak Božího nesluší vaditi se, ale aby byl prívetivý ke všem, zpusobný k ucení, trpelivý,
\par 25 Kterýž by v tichosti vyucoval ty, jenž se pravde protiví, zda by nekdy dal jim Buh pokání ku poznání pravdy,
\par 26 Aby sami k sobe prijdouce, dobyli se z osidla dáblova, od nehož jsou zjímáni k vykonávání jeho vule.

\chapter{3}

\par 1 Toto pak vez, že v posledních dnech nastanou casové nebezpecní.
\par 2 Nebo nastanou lidé sami sebe milující, peníze milující, chlubní, pyšní, zlolejci, rodicu neposlušní, nevdecní, bezbožní,
\par 3 Nelítostiví, smluv nezdrželiví, utrhaci, nestredmí, plaší, kterýmž nic dobrého milo není,
\par 4 Zrádci, prívažciví, nadutí, rozkoší milovníci více nežli Boha,
\par 5 Mající zpusob pobožnosti, ale moci její zapírajíce, a od takových se odvracuj.
\par 6 Nebo z tech jsou i ti, kteríž nacházejí do domu, a jímajíce, vodí ženky obtížené hríchy, jenž vedeny bývají rozlicnými žádostmi,
\par 7 Kteréž vždycky se ucí, ale nikdy ku poznání pravdy prijíti nemohou.
\par 8 Jakož zajisté Jannes a Jambres zprotivili se Mojžíšovi, tak i tito protiví se pravde, lidé na mysli porušení a pri víre spletení.
\par 9 Ale nebudout více prospívati. Nebo nemoudrost jejich zjevná bude všechnem, jako i onechno byla.
\par 10 Ale ty jsi došel mého ucení, zpusobu života mého, úmyslu, víry, snášelivosti, lásky, trpelivosti,
\par 11 Protivenství, utrpení, kteráž na mne prišla v Antiochii, v Ikonii, a v Lystre; kterážto protivenství snášel jsem, ale ze všech vysvobodil mne Pán.
\par 12 A takž i všickni, kteríž chtejí zbožne živi býti v Kristu Ježíši, protivenství míti budou.
\par 13 Zlí pak lidé a svudcové prospívati budou ze zlého v horší, i jiné v blud uvodíce, i sami bludem pojati jsouce.
\par 14 Ale ty zustávej v tom, cemužs se naucil a cožt jest svereno, veda, od kohos se naucil.
\par 15 A že hned od detinství svatá Písma znáš, kteráž te mohou moudrého uciniti k spasení skrze víru, kteráž jest v Kristu Ježíši.
\par 16 Všeliké zajisté Písmo od Boha jest vdechnuté, a užitecné k ucení, k trestání, k napravování, k správe, kteráž náleží k spravedlnosti,
\par 17 Aby byl dokonalý clovek Boží, ke všelikému skutku dobrému hotový.

\chapter{4}

\par 1 Protož já osvedcuji pred oblicejem Božím a Pána Jezukrista, kterýž má souditi živé i mrtvé v den zjevení svého a království svého,
\par 2 Kaž slovo Boží, ponoukej vcas nebo nevcas, tresci, žehri, napomínej, ve vší tichosti a ucení.
\par 3 Nebo prijde cas, že zdravého ucení nebudou prijímati, ale majíce svrablavé uši, podle svých vlastních žádostí shromaždovati sami sobe budou ucitele.
\par 4 A odvrátít uši od pravdy, a k básnem obrátí.
\par 5 Ale ty ve všem bud bedliv, protivenství snášej, dílo kazatele konej, dokazuj toho dostatecne, že jsi verný Kristuv služebník.
\par 6 Neb já se již k tomu blížím, abych obetován byl, a cas rozdelení mého nastává.
\par 7 Boj výborný bojoval jsem, beh jsem dokonal, víru jsem zachoval.
\par 8 Již za tím odložena jest mi koruna spravedlnosti, kterouž dá mi v onen den Pán, ten spravedlivý soudce, a netoliko mne, ale i všechnem tem, kteríž milují príští jeho.
\par 9 Pricin se k tomu, abys ke mne brzo prišel.
\par 10 Nebo Démas mne opustil, zamilovav tento svet, a šel do Tessaloniky, Krescens do Galacie, Titus do Dalmacie.
\par 11 Sám toliko Lukáš se mnou jest. Marka vezmi s sebou; nebo jest mi velmi potrebný k službe.
\par 12 Tychikat jsem poslal do Efezu.
\par 13 Truhlicku, kteréž jsem nechal v Troade u Karpa, když pujdeš, prines s sebou, i knihy, zvlášte pergamén.
\par 14 Alexander kotlár mnoho mi zlého zpusobil; odplatiž jemu Pán podle skutku jeho.
\par 15 Kteréhož i ty se vystríhej; nebo velmi se protivil recem našim.
\par 16 Pri prvním mém odpovídání žádný se mnou nebyl, ale všickni mne opustili. Nebudiž jim to pocítáno za hrích.
\par 17 Pán pak byl se mnou a posilnil mne, aby skrze mne utvrzeno bylo kázání o Kristu, a aby je slyšeli všickni národové. I vytržen jsem byl z úst lva.
\par 18 A vysvobodít mne Pán od každého skutku zlého a zachová k království svému nebeskému, jemuž sláva na veky veku. Amen.
\par 19 Pozdrav Prišky a Akvile, i Oneziforova domu.
\par 20 Erastus zustal v Korintu, Trofima pak nechal jsem v Milétu nemocného.
\par 21 Pospeš pred zimou prijíti ke mne. Pozdravuje tebe Eubulus a Pudens a Línus a Klaudia, i všickni bratrí.
\par 22 Pán Ježíš Kristus budiž s duchem tvým. Milost Boží s vámi. Amen. List tento druhý psán jest z Ríma k Timoteovi, (kterýž první v Efezu za biskupa skrze vzkládání rukou zrízen byl,) když Pavel opet po druhé se stavel pred císarem Neronem.


\end{document}