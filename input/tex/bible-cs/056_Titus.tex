\begin{document}

\title{Titovi}

\chapter{1}

\par 1 Pavel, služebník Boží, apoštol pak Ježíše Krista, podle víry vyvolených Božích a známosti pravdy, kteráž jest podle zbožnosti,
\par 2 K nadeji života vecného, kterýž zaslíbil pred casy veku ten, jenž nikdy neklamá, Buh, zjevil pak casy svými,
\par 3 Totiž to slovo své, skrze kázání mne sverené, podle zrízení Spasitele našeho Boha, Titovi, vlastnímu synu u víre obecné:
\par 4 Milost, milosrdenství a pokoj od Boha Otce a Pána Jezukrista Spasitele našeho.
\par 5 Z té príciny zanechal jsem tebe v Kréte, abys to, cehož tam ješte potrebí, spravil a ustanovil po mestech starší, jakož i já pri tobe jsem zrídil:
\par 6 Jest-li kdo bez úhony, jedné manželky muž, dítky maje verící, na kteréž by nemohlo touženo býti, že by byli bujní, anebo nepoddaní.
\par 7 Nebot biskup má býti bez úhony, jako Boží šafár, ne svémyslný, ne hnevivý, ne pijan vína, ne bijce, ne žádostivý mrzkého zisku,
\par 8 Ale prívetivý k hostem, dobrotivý, opatrný, spravedlivý, svatý, zdrželivý,
\par 9 Pilne se prídržící verné reci v ucení Božím, aby mohl i napomínati ucením zdravým, i ty, kteríž odpírají, premáhati.
\par 10 Nebot jsou mnozí nepoddaní, marnomluvní, i svudcové myslí lidských, zvlášte ti, jenž jsou z obrízky.
\par 11 Jimž musejí ústa zacpána býti; kteríž celé domy prevracejí, ucíce neslušným vecem, pro mrzký zisk.
\par 12 Rekl jeden z nich, vlastní jejich prorok, že Kretenští jsou vždycky lhári, zlá hovada, bricha lenivá.
\par 13 Svedectví to pravé jest. A protož tresciž je prísne, at jsou zdraví u víre,
\par 14 Nešetríce Židovských básní, a prikázání lidí tech, jenž se odvracují od pravdy.
\par 15 Všecko zajisté cisté jest cistým, poskvrneným pak a neverícím nic není cistého, ale poskvrnená jest i mysl jejich i svedomí.
\par 16 Vypravují o tom, že Boha znají, ale skutky svými toho zapírají, ohavní jsouce, a nepoddaní, a ke všelikému skutku dobrému nehodní.

\chapter{2}

\par 1 Ty pak mluv, což sluší na zdravé ucení.
\par 2 Starci at jsou strízliví, vážní, opatrní, zdraví u víre, v lásce, v snášelivosti.
\par 3 Tak také i staré ženy at chodí zbožne, v odevu príslušném a at nejsou prevrhlé, ani mnoho vína pijící, dobrým vecem ucící,
\par 4 Aby mladice vyucovaly, kterak by muže své i dítky rádne milovaly,
\par 5 A byly šlechetné, cistotné, netoulavé, dobrotivé, mužum svým poddané, aby nebylo dáno v porouhání slovo Boží.
\par 6 Takž i mládencu napomínej k stredmosti,
\par 7 Ve všech vecech sebe samého vydávaje za príklad dobrých skutku, a zachovávaje v ucení celost, vážnost,
\par 8 Slovo zdravé bez úhony, aby ten, jenž by se protivil, zastydeti se musil, nemaje co zlého mluviti o vás.
\par 9 Služebníky uc, at jsou poddáni pánum svým, ve všem jim se líbíce, neodmlouvajíce,
\par 10 Neokrádajíce, ale ve všem vernosti pravé dokazujíce, aby ucení Spasitele našeho Boha ve všem ozdobovali.
\par 11 Zjevilat se jest zajisté ta milost Boží spasitelná všechnem lidem,
\par 12 Vyucující nás, abychom odreknouce se bezbožnosti a svetských žádostí, strízlive, a spravedlive, a zbožne živi byli na tomto svete,
\par 13 Ocekávajíce té blahoslavené nadeje a príští slávy velikého Boha a Spasitele našeho Jezukrista,
\par 14 Kterýž dal sebe samého za nás, aby nás vykoupil od všeliké nepravosti, a ocistil sobe samému lid zvláštní, horlive následovný dobrých skutku.
\par 15 Toto mluv, a napomínej, a tresci mocne. Žádný tebou nepohrzej.

\chapter{3}

\par 1 Napomínej jich, at jsou knížatum a mocnostem poddáni, jich poslušni, a at jsou k každému skutku dobrému hotovi.
\par 2 Žádnému at se nerouhají, nejsou svárliví, ale prívetiví, dokazujíce všeliké tichosti ke všem lidem.
\par 3 Bylit jsme zajisté i my nekdy nesmyslní, tvrdošijní, bloudící, sloužíce žádostem a rozkošem rozlicným, v zlosti a v závisti bydlíce, ohyzdní, vespolek se nenávidíce.
\par 4 Ale když se zjevila dobrota a láska k lidem Spasitele našeho Boha,
\par 5 Ne z skutku spravedlnosti, kteréž bychom my cinili, ale podle milosrdenství svého spasil nás, skrze obmytí druhého narození, a obnovení Ducha svatého,
\par 6 Kteréhož vylil na nás hojne, skrze Jezukrista Spasitele našeho,
\par 7 Abychom, ospravedlneni jsouce milostí jeho, byli dedicové v nadeji života vecného.
\par 8 Vernát jest rec tato, a chcit, abys tech vecí potvrzoval, at se snaží v dobrých skutcích predciti všickni, kteríž uverili Bohu. A tot jsou ty veci dobré, i lidem užitecné.
\par 9 Nemoudré pak otázky, a vycítání rodu, a sváry, a hádky o veci zákonní zastavuj; nebt jsou neužitecné a marné.
\par 10 Cloveka kacíre po jednom neb druhém napomínání vyvrz,
\par 11 Veda, že takový jest prevrácený, a hreší, svým vlastním soudem jsa odsouzen.
\par 12 Když pošli k tobe Artemana aneb Tychika, snaž se prijíti ke mne do Nikopolim; neb jsem umínil tu pres zimu pobýti.
\par 13 Zéna, uceného v Zákone, a Apollo s pilností vyprovod, at v nicemž nemají nedostatku.
\par 14 A nechažt se také ucí i naši v dobrých skutcích predciti, a zvlášte, kdež jsou toho potreby, aby nebyli neužitecní.
\par 15 Pozdravují te, kteríž jsou se mnou, všickni. Pozdraviž tech, kteríž nás milují u víre. Milost Boží budiž se všemi vámi. Amen. K Titovi, kterýž první biskup církve Kretenské skrze vzkládání rukou zrízen byl, psán z Nikopoli mesta Macedonského.


\end{document}