\begin{document}

\title{Židům}

\chapter{1}

\par 1 Castokrát a rozlicnými zpusoby mluvíval nekdy Buh otcum skrze proroky, v techto pak posledních dnech mluvil nám skrze Syna svého,
\par 2 Kteréhož ustanovil dedicem všeho, skrze nehož i veky ucinil.
\par 3 Kterýžto jsa blesk slávy, a obraz osoby jeho, a zdržuje všecko slovem mocnosti své, ocištení hríchu našich skrze sebe samého uciniv, posadil se na pravici velebnosti na výsostech,
\par 4 Tím dustojnejší nad andely ucinen, cím vyvýšenejší nad ne jméno dedicne obdržel.
\par 5 Nebo kterému kdy z andelu rekl: Syn muj jsi ty, já dnes zplodil jsem tebe? A opet: Já budu jemu Otcem, a on mi bude Synem?
\par 6 A opet, když uvodí prvorozeného na okršlek zeme, dí: A klanejte se jemu všickni andelé Boží.
\par 7 A o andelích zajisté dí: Kterýž ciní andely své duchy, a služebníky své plamen ohne;
\par 8 Ale k Synu dí: Stolice tvá, ó Bože, trvá na veky veku, berla pravosti jestit berla království tvého.
\par 9 Miloval jsi spravedlnost, a nenávidel jsi nepravosti, protož pomazal tebe, ó Bože, Buh tvuj olejem veselé nad spoluúcastníky tvé.
\par 10 A opet:Ty, Pane, na pocátku založil jsi zemi, a díla rukou tvých jsout nebesa.
\par 11 Onat pominou, ty pak zustáváš; a všecka jako roucho zvetšejí,
\par 12 A jako odev svineš je, i budout zmenena. Ale ty jsi vždycky tentýž, a léta tvá nikdy neprestanou.
\par 13 A kterému kdy z andelu rekl: Sed na pravici mé, dokavadž nepoložím neprátel tvých za podnože noh tvých?
\par 14 Zdaliž všickni nejsou služební duchové, kteríž posíláni bývají k službe pro ty, jenž mají dedicne obdržeti spasení?

\chapter{2}

\par 1 Protož musímet my tím snažneji šetriti toho, což jsme slýchali, aby nám to nevymizelo.
\par 2 Nebo ponevadž skrze andely mluvené slovo bylo pevné, a každé prestoupení a neposlušenství vzalo spravedlivou odmenu pomsty,
\par 3 Kterakž my uteceme, takového zanedbávajíce spasení? Kteréžto nejprvé zacalo vypravováno býti skrze samého Pána, od tech pak, kteríž Pána slýchali, nám utvrzeno jest.
\par 4 Cemuž i Buh svedectví vydával skrze divy a zázraky, a rozlicné moci, i podelování Duchem svatým, podle vule své.
\par 5 Nebo nepoddal andelum okršlku zeme budoucího, o kterémž mluvíme.
\par 6 Osvedcilt jest pak na jednom míste jeden, rka: Co jest clovek, že nan pomníš, aneb syn cloveka, že na nej patríš.
\par 7 Malickos jej menšího andelu ucinil, slavou a ctí korunoval jsi ho, a ustanovils jej nad dílem rukou svých.
\par 8 Všecko jsi podmanil pod nohy jeho. A kdyžt jest jemu všecko poddal, tedy niceho nezanechal nepodmaneného jemu. Ackoli nyní ješte nevidíme, aby jemu všecko poddáno bylo.
\par 9 Ale toho malicko nižšího andelu, vidíme Ježíše, pro utrpení smrti slávou a ctí korunovaného, aby z milosti Boží za všecky okusil smrti.
\par 10 Slušelot zajisté na toho, pro kteréhož jest všecko, a skrze kteréhož jest všecko, aby mnohé syny priveda k sláve, vudce spasení jejich skrze utrpení dokonalého ucinil.
\par 11 Nebo i ten, jenž posvecuje, i ti, kteríž posveceni bývají, z jednoho jsou všickni. Pro kteroužto prícinu nestydí se jich nazývati bratrími,
\par 12 Rka: Zvestovati budu jméno tvé bratrím svým, uprostred shromáždení prozpevovati budu tobe.
\par 13 A opet: Já budu v nem doufati. A opet: Aj, já a dítky, kteréž dal mi Buh.
\par 14 Ponevadž tedy dítky úcastnost mají tela a krve, i on též podobne úcasten jest jich, aby skrze smrt zahladil toho, kterýž má vládarství smrti, to jest dábla,
\par 15 A abyvysvobodil ty, kterížto bázní smrti po všecken cas života svéhopodrobeni byli v službu.
\par 16 Nebot neprijal andelu, ale síme Abrahamovo prijal.
\par 17 A protož ve všem pripodobnen býti mel bratrím, aby milosrdný byl a verný nejvyšší knez v tom, což by u Boha k ocištení hríchu lidu jednáno býti melo.
\par 18 Nebo že jest i sám trpel, pokoušín byv, muže také pokušení trpícím spomáhati.

\chapter{3}

\par 1 A protož, bratrí svatí, povolání nebeského úcastníci, spatrujte apoštola a nejvyššího kneze vyznání našeho, Krista Ježíše,
\par 2 Verného tomu, kdož jej ustanovil, jako i Mojžíš byl verný ve všem dome jeho.
\par 3 Tím vetší zajisté slávy tento nad Mojžíše jest hoden, cím vetší má cest stavitel nežli sám dum.
\par 4 Nebo všeliký dum ustaven bývá od nekoho, ten pak, kdož všecky tyto veci ustavel, Buh jest.
\par 5 A Mojžíš zajisté verný byl v celém dome jeho, jako služebník, na osvedcení toho, což potom melo mluveno býti.
\par 6 Ale Kristus, jakožto Syn, vládne nad domem svým. Kterýžto dum my jsme, jestliže tu svobodnou doufanlivost, a tu chloubu nadeje až do konce pevnou zachováme.
\par 7 Protož jakž praví Duch svatý: Dnes, uslyšeli-li byste hlas jeho,
\par 8 Nezatvrzujtež srdcí svých, jako pri onom popouzení Boha v den pokušení toho na poušti;
\par 9 Kdežto pokoušeli mne otcové vaši, zkusilit jsou mne, a videli skutky mé po ctyridceti let.
\par 10 Protož hneviv jsem byl na pokolení to, a rekl jsem: Tito vždycky bloudí srdcem, a nepoznávají cest mých.
\par 11 Takže jsem prisáhl v hneve svém, že nevejdou v odpocinutí mé.
\par 12 Viztež, bratrí, aby snad v nekom z vás nebylo srdce zlé, a neverné, kteréž by odstupovalo od Boha živého.
\par 13 Ale napomínejte se vespolek po všecky dny, dokavadž se dnes jmenuje, aby nekdo nebyl zatvrzen oklamáním hrícha.
\par 14 Úcastníci zajisté Krista ucineni jsme, jestliže však ten pocátek podstaty až do konce pevný zachováme.
\par 15 Protož dokudž se ríká: Dnes, uslyšeli-li byste hlas jeho, nezatvrzujte srdcí svých, jako pri onom popouzení Boha.
\par 16 Nebo nekterí slyševše, popouzeli ho, ale ne všickni, jenž vyšli z Egypta skrze Mojžíše.
\par 17 Na které se pak hneval ctyridceti let? Zdali ne na ty, kteríž hrešili, jejichžto tela padla na poušti?
\par 18 A kterým zaprisáhl, že nevejdou do odpocinutí jeho? Však tem, kteríž byli neposlušní.
\par 19 A vidíme, že jsou nemohli vjíti pro neveru.

\chapter{4}

\par 1 Bojmež se tedy, aby snad opuste zaslíbení o vjití do odpocinutí jeho, neopozdil se nekdo z vás.
\par 2 Nebo i nám zvestováno jest, jako i onemno, ale neprospela jim rec slyšená, nepripojená k víre, když slyšeli.
\par 3 Nebot vcházíme v odpocinutí my, kteríž jsme uverili, jakož rekl: Protož jsem prisáhl v hneve svém, žet nevejdou v odpocinutí mé, ackoli dávno odpocinul Buh, hned po vykonání skutku od ustanovení sveta.
\par 4 Nebo povedel na jednom míste o sedmém dni takto: I odpocinul Buh dne sedmého ode všech skutku svých.
\par 5 A tuto zase: Že nevejdou v odpocinutí mé.
\par 6 A ponevadž vždy na tom jest, že nekterí mají vjíti do neho, a ti, kterýmž prvé zvestováno jest, nevešli pro svou neveru,
\par 7 Opet ukládá den jakýsi, Dnes, prave skrze Davida, po takovém casu, jakož receno jest, Dnes uslyšíte-li hlas jeho, nezatvrzujte srdcí svých.
\par 8 Nebo byt byl Jozue v odpocinutí je uvedl, nebylt by potom mluvil o jiném dni.
\par 9 A protož zustávát svátek lidu Božímu.
\par 10 Neb kdožkoli všel v odpocinutí jeho, takét i on odpocinul od skutku svých, jako i Buh od svých.
\par 11 Snažmež se tedy vjíti do toho odpocinutí, aby nekdo neupadl v týž príklad nedovery.
\par 12 Živát jest zajisté rec Boží a mocná, a pronikavejší nad všeliký mec na obe strane ostrý, a dosahujet až do rozdelení i duše i ducha i kloubu i mozku v kostech, a rozeznává myšlení i mínení srdce.
\par 13 A nenít žádného stvorení, kteréž by nebylo zjevné pred oblicejem jeho, nýbrž všecky veci jsou nahé a odkryté ocima toho, o kterémž jest rec naše.
\par 14 Protož majíce nejvyššího kneze velikého, kterýžto pronikl nebesa, Ježíše Syna Božího, držmež vyznání naše.
\par 15 Nebo nemáme nejvyššího kneže, kterýž by nemohl citedlen býti mdlob našich, ale zkušeného ve všem nám podobne, krome hríchu.
\par 16 Pristupmež tedy smele s doufáním k trunu milosti, abychom dosáhli milosrdenství, a milost nalezli ku pomoci v cas príhodný.

\chapter{5}

\par 1 Všeliký zajisté nejvyšší knez z lidu vzatý za lidi bývá postaven v tech vecech, kteréž u Boha mají jednány býti, totiž aby obetoval i dary i obeti za hríchy,
\par 2 Kterýž by mohl, jakž sluší, lítost míti nad neznajícími a bloudícími, jsa i sám obklícen nemocí.
\par 3 A pro ni povinen jest, jakož za lid, tak i za sebe samého obetovati obeti za hríchy.
\par 4 A aniž kdo sobe sám té cti osobuje, ale ten, kterýž by byl povolán od Boha, jako i Aron.
\par 5 Tak i Kristus ne sám sobe té cti osobil, aby byl nejvyšším knezem, ale ten, kterýž rekl jemu: Syn muj jsi ty, já dnes zplodil jsem tebe.
\par 6 Jakž i jinde praví: Ty jsi knez na veky podle rádu Melchisedechova.
\par 7 Kterýž za dnu tela svého modlitby a ponížené prosby k tomu, kterýž ho mohl zachovati od smrti, s krikem velikým a slzami obetoval, a uslyšán jest i vysvobozen z toho, cehož se strašil.
\par 8 A ackoli byl Syn Boží, z toho však, což strpel, naucil se poslušenství.
\par 9 A tak dokonalý jsa, ucinen jest všechnem sebe poslušným puvodem spasení vecného,
\par 10 Nazván jsa od Boha nejvyšším knezem podlé rádu Melchisedechova.
\par 11 O kterémž mnoho by se melo mluviti, a to nesnadných veci k vypravení, ale vy jste nezpusobných uší.
\par 12 Nebo mevše býti v tak dlouhém casu mistri, opet potrebujete uceni býti prvním pocátkum výmluvností Božích, a ucineni jste mléka potrebující, a ne pokrmu hrubšího.
\par 13 Kdožkoli zajisté mléka se drží, nechápá slova spravedlnosti; (nebo nemluvne jest).
\par 14 Ale dokonalých jest hrubý pokrm, totiž tech, kteríž pro zvyklost mají smysly zpusobné k rozeznání dobrého i zlého.

\chapter{6}

\par 1 Protož opustíce rec pocátku Kristova, k dokonalosti se nesme, ne opet zakládajíce gruntu pokání z skutku mrtvých, a víry v Boha,
\par 2 Krtu ucení, vzkládání rukou, a vzkríšení z mrtvých, i soudu vecného.
\par 3 A tot uciníme, dopustí-li Buh.
\par 4 Nebo nemožné jest jednou již osvíceným, kteríž i zakusili daru nebeského, a úcastníci ucineni byli Ducha svatého,
\par 5 Okusili také dobrého Božího slova, a moci veka budoucího,
\par 6 Kdyby padli, zase obnoviti se ku pokání, jakožto tem, kteríž opet sobe znovu križují Syna Božího, a v porouhání vydávají.
\par 7 Zeme zajisté, kteráž casto na sebe pricházející déšt pije, a rodí bylinu príhodnou tem, od kterýchž bývá delána, dochází požehnání od Boha.
\par 8 Ale vydávající trní a bodláky zavržená jest, a blízká zlorecení, jejížto konec bývá spálení.
\par 9 My pak, nejmilejší, nadejemet se o vás lepších vecí, a náležejících k spasení, ac pak koli tak mluvíme.
\par 10 Nebot není nespravedlivý Buh, aby se zapomenul na práci vaši a na pracovitou lásku, kteréž jste dokazovali ke jménu jeho, slouživše svatým, a ješte sloužíce.
\par 11 Žádámet pak, aby jeden každý z vás až do konce prokazoval tu opravdovou pilnost k nabytí plné jistoty nadeje,
\par 12 Tak abyste nebyli líní, ale následovníci tech, kteríž skrze víru a snášelivost obdrželi dedictví zaslíbené.
\par 13 Abrahamovi zajisté zaslíbení cine Buh, když nemel skrze koho vetšího prisáhnouti, prisáhl skrze sebe samého,
\par 14 Rka: Jiste požehnám velmi tobe, a velice rozmnožím tebe.
\par 15 A tak trpelive ocekávaje, dosáhl zaslíbení.
\par 16 Lidé zajisté skrze vetšího, nežli jsou sami, prisahají, a všeliké rozepre mezi nimi konec jest, když bývá potvrzena prísahou.
\par 17 A takž Buh, chteje dostatecne ukázati dedicum zaslíbení svých nepromenitelnost rady své, vložil mezi to prísahu,
\par 18 Abychom skrze ty dve veci nepohnutelné, (v nichž nemožné jest, aby Buh klamal), meli prepevné potešení, my, kteríž jsme se utekli k obdržení predložené nadeje.
\par 19 Kteroutož máme jako kotvu duše, i bezpecnou i pevnou, a vcházející až do vnitrku za oponu,
\par 20 Kdežto predchudce náš pro nás všel Ježíš, jsa ucinen podle rádu Melchisedechova nejvyšším knezem na veky.

\chapter{7}

\par 1 Nebo ten Melchisedech byl král Sálem, knez Boha nejvyššího, kterýž vyšel v cestu Abrahamovi, navracujícímu se od pobití králu, a dal jemu požehnání.
\par 2 Kterémužto Abraham i desátek dal ze všeho. Kterýž nejprvé vykládá se král spravedlnosti, potom pak i král Sálem, to jest král pokoje,
\par 3 Bez otce, bez matky, bez rodu, ani pocátku dnu, ani skonání života nemaje, ale pripodobnen jsa Synu Božímu, zustává knezem vecne.
\par 4 Pohledtež tedy, kteraký ten byl, jemuž i desátky z koristí dal Abraham patriarcha.
\par 5 A ješto ti, kteríž jsou z synu Léví knežství prijímající, prikázaní mají desátky bráti od lidu podle Zákona, to jest od bratrí svých, ackoli pošlých z bedr Abrahamových,
\par 6 Tento pak, jehož rod není pocten mezi nimi, desátky vzal od Abrahama, a tomu, kterýž mel zaslíbení, požehnání dal.
\par 7 A jiste beze všeho odporu menší od vetšího požehnání bére.
\par 8 A tuto desátky berou smrtelní lidé, ale tamto ten, o nemž se vysvedcuje, že jest živ.
\par 9 A at tak dím, i sám Léví, kterýž desátky bére, v Abrahamovi desátky dal.
\par 10 Nebo ješte v bedrách otce byl, když vyšel proti nemu Melchisedech.
\par 11 A protož byla-lit dokonalost spasení skrze Levítské knežství, (nebo za toho knežství vydán jest lidu Zákon,) jakáž toho byla potreba, aby jiný knez podle rádu Melchisedechova povstal, a nebyl již více podle rádu Aronova jmenován?
\par 12 A ponevadž jest knežství preneseno, musilot také i Zákona prenesení býti.
\par 13 Nebo ten, o kterémž se to praví, jiného jest pokolení, z kteréhožto žádný pri oltári v službe nebyl.
\par 14 Zjevné jest zajisté, že z pokolení Judova pošel Pán náš, o kterémžto pokolení nic z strany knežství nemluvil Mojžíš.
\par 15 Nýbrž hojneji to zjevné jest i z toho, že povstal jiný knez podle rádu Melchisedechova,
\par 16 Kterýžto ucinen jest knezem ne podle zákona prikázaní telesného, ale podle moci života neporušitelného.
\par 17 Nebo svedcí Písmo, rka: Ty jsi knez na veky podle rádu Melchisedechova.
\par 18 Stalo se zajisté složení onoho predešlého prikázaní, protože bylo mdlé a neužitecné.
\par 19 Nebo nicehož k dokonalosti neprivedl Zákon, ale na místo jeho uvedena lepší nadeje, skrze niž približujeme se k Bohu.
\par 20 A to tím lepší, že ne bez prísahy.
\par 21 Nebo onino bez prísahy knežími ucineni bývali, tento pak s prísahou, skrze toho, kterýž rekl k nemu: Prisáhl Pán, a nebudet toho litovati: Ty jsi knez na veky podle rádu Melchisedechova.
\par 22 Takt lepší smlouvy prostredníkem ucinen jest Ježíš.
\par 23 A také onino mnozí bývali kneží, protože smrt bránila jim vždycky trvati;
\par 24 Ale tento ponevadž zustává na veky, vecné má knežství.
\par 25 A protož i dokonale spasiti muže všecky pristupující skrze nej k Bohu, vždycky jsa živ k orodování za ne.
\par 26 Takovéhot zajisté nám slušelo míti nejvyššího kneze, svatého, nevinného, nepoškvrneného, oddeleného od hríšníku, a jenž by vyšší nad nebesa ucinen byl,
\par 27 Kterýž by nepotreboval na každý den, jako onino kneží, nejprv za své vlastní hríchy obeti obetovati, potom za za hríchy lidu. Nebot jest to ucinil jednou, samého sebe obetovav.
\par 28 Zákon zajisté lidi mající nedostatky ustavoval za nejvyšší kneží, ale slovo prísežné, kteréž se stalo po Zákonu, ustanovilo Syna Božího dokonalého na veky.

\chapter{8}

\par 1 Ale summa toho mluvení tato jest, že takového máme nejvyššího kneze, kterýž se posadil na pravici trunu velebnosti v nebesích.
\par 2 Služebník jsa svatyne, a pravého toho stánku, kterýž Pán vzdelal, a ne clovek.
\par 3 Všeliký zajisté nejvyšší knez k obetování daru a obetí bývá ustanoven, a protož potrebí bylo, aby i tento mel, co by obetoval.
\par 4 Nebo kdyby byl na zemi, aniž by knezem byl, když by zustávali ti kneží, kteríž obetují dary podle Zákona,
\par 5 Sloužíce podobenství a stínu nebeských vecí, jakož Mojžíšovi od Boha receno bylo, když dokonávati mel stánek: Hlediž, prý, abys udelal všecko ku podobenství, kteréžt jest ukázáno na hore.
\par 6 Nyní pak tento náš nejvyšší knez tím dustojnejšího došel úradu, címž i lepší smlouvy prostredníkem jest, kterážto lepšími zaslíbeními jest utvrzena.
\par 7 Nebo kdyby ona první byla bez úhony, nebylo by hledáno místa druhé.
\par 8 Nebo obvinuje Židy, dí: Aj, dnové jdou, praví Pán, v nichž vejdu s domem Izraelským a s domem Judským v smlouvu novou.
\par 9 Ne podle smlouvy té, kterouž jsem ucinil s otci jejich v ten den, když jsem je vzal za ruku jejich, abych je vyvedl z zeme Egyptské. Nebo oni nezustali v smlouve mé, a já opustil jsem je, praví Pán.
\par 10 Protož tatot jest smlouva, v kterouž vejdu s domem Izraelským po tech dnech, praví Pán: Dám zákony své v mysl jejich, a na srdcích jejich napíši je, a budu jejich Bohem, a oni budou mým lidem.
\par 11 A nebudout uciti jeden každý bližního svého, a jeden každý bratra svého, ríkajíce: Poznej Pána, protože mne všickni znáti budou, od nejmenšího z nich až do nejvetšího z nich.
\par 12 Nebo milostiv budu nepravostem jejich, a na hríchy jejich, ani na nepravosti jejich nikoli nezpomenu více.
\par 13 A kdyžt praví o nové, tedy pokládá první za vetchou; což pak vetší a schází, blízké jest zahynutí.

\chapter{9}

\par 1 Melat pak i první ona smlouva ustanovení z strany služeb a svatyni svetskou.
\par 2 Nebo udelán byl stánek první, v kterémž byl svícen, a stul, a posvátní chlebové, a ten sloul svatyne.
\par 3 Za druhou pak oponou byl stánek, kterýž sloul svatyne svatých,
\par 4 Zlatou maje kadidlnici, a truhlu smlouvy, všudy obloženou zlatem, kdežto bylo vederce zlaté, mající v sobe mannu, a hul Aronova, kteráž byla zkvetla, a dsky zákona,
\par 5 Nad truhlou pak byli dva cherubínové slávy, zastenujíci slitovnici. O kterýchž vecech není potrebí nyní vypravovati o jedné každé obzvláštne.
\par 6 To vše když tak jest zrízeno, do prvního stánku vždycky vcházejí kneží, služby vykonávajíce,
\par 7 Do druhého pak jedinou v rok sám nejvyšší knez, ne bez krve, kterouž obetuje sám za sebe, i za lidské nevedomosti.
\par 8 Címž Duch svatý ukazuje to, že ješte nebyla zjevena cesta k svatyni, dokudž první stánek trval.
\par 9 Kterýž byl podobenstvím na ten prítomný cas, v nemžto darové a obeti se obetují, kteréž nemohou dokonalého v svedomí uciniti toho, kdož obetuje,
\par 10 Toliko v pokrmích a v nápojích, a v rozlicných umýváních a ospravedlnováních telesných, až do casu napravení, záležející.
\par 11 Ale Kristus prišed, nejvyšší knez budoucího dobrého, skrze vetší a dokonalejší stánek, ne rukou udelaný, to jest ne tohoto stavení,
\par 12 Ani skrze krev kozlu a telat, ale skrze svou vlastní krev, všel jednou do svatyne, vecné vykoupení nalezl.
\par 13 Nebo jestližet krev býku a kozlu, a popel jalovice, pokropující poskvrnených, posvecuje jich k ocištení tela,
\par 14 Cím více krev Kristova, kterýžto skrze Ducha vecného samého sebe obetoval nepoškvrneného Bohu, ocistí svedomí vaše od skutku mrtvých k sloužení Bohu živému?
\par 15 A pro tu prícinu nové smlouvy prostredník jest, aby, když by smrt mezi to vkrocila k vyplacení prestoupení tech, kteráž byla za první smlouvy, zaslíbení vecného dedictví prijali ti, jenž jsou povoláni.
\par 16 Nebo kdež se deje kšaft, potrebí jest, aby k tomu smrt prikrocila toho, kdož ciní kšaft.
\par 17 Kšaft zajisté tech, kteríž zemreli, pevný jest, ponevadž ješte nemá moci, dokudž živ jest ten, jenž kšaft cinil.
\par 18 Protož ani první onen kšaft bez krve nebyl posvecován.
\par 19 Nebo když Mojžíš všecka prikázaní podle Zákona všemu lidu predložil, vzav krev telat a kozlu, s vodou a s vlnou cervenou a s yzopem, tak spolu i knihy i všeho lidu pokropil,
\par 20 Rka: Tatot jest krev Zákona, kterýž vám Buh vydal.
\par 21 Ano i stánku i všech nádob k službe náležitých rovne též krví pokropil.
\par 22 A témer všecko podle Zákona krví ocištováno bývá, a bez krve vylití nebývá odpuštení vin.
\par 23 Protož potrebí bylo, aby vecí nebeských príkladové temi vecmi ocištováni byli, nebeské pak veci lepšími obetmi, nežli jsou ty.
\par 24 Nebot nevšel Kristus do svatyne rukou udelané, kteráž by byla príklad s pravou svatyní se srovnávající, ale práve v nebe všel, aby nyní prítomný byl tvári Boží za nás.
\par 25 Ne aby castokrát obetoval sebe samého, jako nejvyšší knez vcházel do svatyne každý rok se krví cizí,
\par 26 (Sic jinak byl by musil castokrát trpeti od pocátku sveta,) ale nyní jednou pri skonání veku, na shlazení hrícha skrze obetování sebe samého, zjeven jest.
\par 27 A jakož uloženo lidem jednou umríti, a potom bude soud,
\par 28 Tak i Kristus jednou jest obetován, k shlazení mnohých lidi hríchu; podruhé pak bez hríchu ukáže se tem, kteríž ho cekají k spasení.

\chapter{10}

\par 1 Zákon zajisté, maje stín budoucího dobrého, a ne sám obraz pravý tech vecí, jednostejnými, kteréž po všecka léta obetují, obetmi nikdy nemuž pristupujících dokonalých uciniti.
\par 2 Sic jinak zdaliž by již neprestaly obetovány býti, protože by již nemeli žádného svedomí z hríchu ti, jenž obetují, jsouce jednou ocišteni?
\par 3 Ale pri tech obetech pripomínání hríchu deje se každého roku.
\par 4 Nebot možné není, aby krev býku a kozlu shladila hríchy.
\par 5 Protož vcházeje na svet, dí: Obetí a daru nechtel jsi, ale telo jsi mi zpusobil.
\par 6 Zápalných obetí, ani obetí za hrích jsi neoblíbil.
\par 7 Tehdy rekl jsem: Aj, jdut, (jakož v knihách psáno jest o mne), abych cinil, ó Bože, vuli tvou.
\par 8 Povedev napred: Že obetí a daru, a zápalu, i obetí za hrích, (kteréž se podle Zákona obetují), nechtel jsi, aniž jsi jich oblíbil,
\par 9 Tehdy rekl: Aj, jdut, (jakož v knihách psáno jest o mne) abych cinil, ó Bože, vuli tvou. Ruší první, aby druhé ustanovil.
\par 10 V kteréžto vuli posveceni jsme skrze obetování tela Ježíše Krista jednou.
\par 11 A všeliký zajisté knez prístojí, na každý den službu konaje, a jednostejné casto obetuje obeti, kteréž nikdy nemohou odjíti hríchu.
\par 12 Ale tento, jednu obet obetovav za hríchy, vždycky sedí na pravici Boží,
\par 13 Již dále ocekávaje, až by položeni byli neprátelé jeho za podnož noh jeho.
\par 14 Nebo jednou obetí dokonalé ucinil na veky ty , kteríž posveceni bývají.
\par 15 Svedcít pak nám to i sám Duch svatý. Nebo prve povedev:
\par 16 Tatot jest smlouva, kterouž uciním s nimi po tech dnech, praví Pán: Dám zákony své v srdce jejich, a na myslech jejich napíši je,
\par 17 Za tím rekl: A na hríchy jejich, i na nepravosti jejich nikoli nezpomenu více.
\par 18 Kdežt pak jest odpuštení jich, nenít potrebí více obeti za hrích.
\par 19 Majíce tedy, bratrí, plnou svobodu k vjíti do svatyne skrze krev Ježíšovu,
\par 20 Tou cestou novou a živou, kterouž nám zpusobil skrze oponu, to jest telo své,
\par 21 A majíce kneze velikého nad domem Božím,
\par 22 Pristupmež s pravým srdcem, v plné jistote víry, ocištená majíce srdce od svedomí zlého,
\par 23 A umyté telo vodou cistou, držmež nepochybné vyznání nadeje; (nebo vernýt jest ten, kterýž zaslíbil.)
\par 24 A šetrme jedni druhých, k roznecování se v lásce a dobrých skutcích,
\par 25 Neopouštejíce spolecného shromáždení našeho, jako nekterí obycej mají, ale napomínajíce se, a to tím více, címž více vidíte, že se ten den približuje.
\par 26 Nebo jestliže bychom dobrovolne hrešili po prijetí známosti pravdy, nezustávalo by již více obeti za hríchy,
\par 27 Ale hrozné nejaké ocekáváni soudu, a ohne prudká pálivost, kterýž žráti má protivníky.
\par 28 Kdož by koli pohrdal Zákonem Mojžíšovým, bez lítosti pode dvema neb trmi svedky umírá.
\par 29 Což se vám zdá, jak prísnejšího trestání hoden jest ten, kdož by Syna Božího pošlapával a krev smlouvy, kterouž byl posvecen, za nehodnou drahého vážení by mel, a duchu milosti potupu ucinil?
\par 30 Známet zajisté toho, jenž rekl: Mne pomsta, já odplatím, praví Pán. A opet: Pán souditi bude lid svuj.
\par 31 Hroznét jest upadnouti v ruce Boha živého.
\par 32 Rozpomentež se pak na predešlé dny, v nichžto osvíceni byvše, mnohý boj rozlicných utrpení snášeli jste,
\par 33 Budto když jste byli i pohaneními i ssouženími jako divadlo ucineni, budto úcastníci ucineni byvše tech, kteríž tak zmítáni byli.
\par 34 Nebo i vezení mého citelni jste byli, a rozchvátání statku svých s radosti jste strpeli, vedouce, že v sobe máte lepší zboží nebeské a trvanlivé.
\par 35 Protož neodmítejtež od sebe smelé doufanlivosti vaší, kterážto velikou má odplatu.
\par 36 Než potrebít jest vám trpelivosti, abyste vuli Boží ciníce, dosáhli zaslíbení.
\par 37 Nebo ješte velmi, velmi malicko, a aj, ten, kterýž prijíti má, prijde, a nebudet meškati.
\par 38 Spravedlivý pak z víry živ bude. Pakli by se kdo jinam obrátil, nezalibuje sobe duše má v nem.
\par 39 Ale myt nejsme pobehlci k zahynutí, ale verící k získání duše.

\chapter{11}

\par 1 Víra pak jest nadejných vecí podstata, a duvod neviditelných.
\par 2 Pro ni zajisté svedectví došli predkové.
\par 3 Verou rozumíme, že ucineni jsou vekové slovem Božím, takže z niceho jest to, což vidíme, ucineno.
\par 4 Verou lepší obet Bohu obetoval Abel, nežli Kain, skrze kteroužto svedectví obdržel, že jest spravedlivý, jakž sám Buh darum jeho svedectví vydal. A skrze tu víru, již umrev, ješte mluví.
\par 5 Verou Enoch prenesen jest, aby nevidel smrti, a není nalezen, proto že jej Buh prenesl. Prvé zajisté, než jest prenesen, svedectví mel, že se líbil Bohu.
\par 6 Bez víry pak nemožné jest líbiti se Bohu; nebo pristupující k Bohu veriti musí, že jest Buh, a tem, kteríž ho hledají, že odplatu dává.
\par 7 Verou napomenut jsa od Boha Noé o tom, cehož ješte nebylo videti, boje se, pripravoval koráb k zachování domu svého; skrze kterýžto koráb odsoudil svet, a spravedlnosti té, kteráž jest z víry, ucinen jest dedicem.
\par 8 Verou, povolán jsa Abraham, uposlechl Boha, aby odšel na to místo, kteréž mel vzíti za dedictví; i šel, neveda, kam príjde.
\par 9 Verou bydlil v zemi zaslíbené jako v cizí v staních prebývaje s Izákem a s Jákobem, spoludedici téhož zaslíbení.
\par 10 Nebo ocekával mesta základy majícího, jehožto remeslník a stavitel jest Buh.
\par 11 Verou také i Sára moc ku pocetí semene prijala, a mimo cas veku porodila, když verila, že jest verný ten, jenž zaslíbil.
\par 12 A protož i z jednoho, a to již témer umrtveného, rozplozeno jest potomku množství veliké, jako jest množství hvezd nebeských, a jako písek nescíslný, kterýž jest na brehu morském.
\par 13 Podle víry zemreli ti všickni, nevzavše zaslíbení, ale zdaleka je videvše, jim i verili, i je vítali, a vyznávali, že jsou hosté a príchozí na zemi.
\par 14 Nebo ti, kteríž tak mluví, zjevne to prokazují, že vlasti hledají.
\par 15 A jiste, kdyby se byli na onu rozpomínali, z kteréž vyšli, meli dosti casu zase se navrátiti.
\par 16 Ale oni lepší vlasti žádají, to jest nebeské. Protož i sám Buh nestydí se slouti jejich Bohem; nebo jim pripravil mesto.
\par 17 Verou obetoval Abraham Izáka, byv pokoušín, a to jednorozeného obetoval ten, kterýž byl zaslíbení prijal,
\par 18 K nemuž bylo receno: V Izákovi nazváno bude tobe síme,
\par 19 Tak o tom smýšleje, že jest mocen Buh i z mrtvých vzkrísiti; odkudžto jej jako z mrtvých vzkríšeného prijal.
\par 20 Verou o budoucích vecech požehnání dal Izák Jákobovi a Ezau.
\par 21 Verou Jákob, umíraje, každému z synu Jozefových požehnání dával, a poklonil se podeprev se na vrch hulky své.
\par 22 Verou Jozef, dokonávaje, o vyjití synu Izraelských zmínku ucinil, a o kostech svých porucil.
\par 23 Verou Mojžíš, narodiv se, ukryt byl za tri mesíce od rodicu svých, protože videli krásné pacholátko, a nebáli se rozkazu královského.
\par 24 Verou Mojžíš, dorostlý již jsa, odeprel slouti synem dcery faraonovy,
\par 25 Vyvoliv sobe radeji protivenství trpeti s lidem Božím, nežli casné a hríšné pohodlí míti,
\par 26 Vetší sobe pokládaje zboží nad Egyptské poklady pohanení Kristovo; nebo prohlédal k hojné odplate.
\par 27 Verou opustil Egypt, neboje se hnevu královského; nebo jako by videl Neviditelného, tak se utvrdil.
\par 28 Verou slavil hod beránka a vylití krve, aby ten, jenž hubil prvorozené, nedotekl se jich.
\par 29 Verou prešli more cervené jako po suchu; o cež pokusivše se Egyptští, ztonuli.
\par 30 Verou zdi Jericha padly, když je obcházeli za sedm dní.
\par 31 Verou Raab nevestka nezahynula s neposlušnými, pokojne prijavši špehére.
\par 32 A cot mám více praviti? Nepostacít mi zajisté cas k vypravování o Gedeonovi, a Barákovi, a Samsonovi a Jefte, a Davidovi, a Samuelovi, a prorocích.
\par 33 Kteríž skrze víru vybojovávali království, cinili spravedlnost, docházeli zaslíbení, zacpávali ústa lvum,
\par 34 Uhašovali moc ohne, utekli ostrosti mece, zmocneni bývali v mdlobách, silní ucineni v boji, vojska zaháneli cizozemcu.
\par 35 Ženy prijímaly mrtvé své vzkríšené. Jiní pak roztahováni jsou, neoblíbivše sobe vysvobození, aby lepšího dosáhli vzkríšení.
\par 36 Jiní pak posmechy a mrskáním trápeni, ano i vezeními a žalári.
\par 37 Kamenováni jsou, sekáni, pokoušíni, mecem zmordováni; chodili v kožích ovcích a kozelcích, opušteni, souženi, a zle s nimi nakládáno,
\par 38 Jichžto nebyl svet hoden, po pustinách bloudíce, i po horách, a jeskyních, i v doupatech zeme.
\par 39 A ti všickni svedectví dosáhše skrze víru, neobdrželi zaslíbení,
\par 40 Protože Buh neco lepšího nám obmýšlel, aby oni bez nás neprišli k dokonalosti.

\chapter{12}

\par 1 Protož i my, takový oblak svedku vukol majíce, odvrhouce všeliké bríme, i snadne obklicující nás hrích, skrze trpelivost konejme beh uloženého nám boje,
\par 2 Patríce na vudce a dokonavatele víry Ježíše, kterýžto místo predložené sobe radosti strpel kríž, opováživ se hanby, i posadil se na pravici trunu Božího.
\par 3 A považte, kteraký jest ten, jenž snášel od hríšníku taková proti sobe odmlouvání, abyste neustávali, v myslech vašich hynouce.
\par 4 Ješte jste se až do krve nezprotivili, proti hríchu bojujíce.
\par 5 A což jste zapomenuli na napomenutí, kteréž k vám jako k synum mluví:Synu muj, nepohrdej kázní Páne, aniž sobe stýskej, když od neho trestán býváš?
\par 6 Nebo kohož miluje Pán, tohot tresce, a švihá každého, kteréhož za syna prijímá.
\par 7 Jestliže kázen snášíte, Buh se vám podává jakožto synum. Nebo který jest syn, jehož by netrestal otec?
\par 8 Pakli jste bez kázne, kteréžto všickni synové úcastni jsou, tedy jste cizoložnata, a ne synové.
\par 9 Ano telesné otce naše meli jsme, kteríž nás trestali, a meli jsme je u vážnosti; i zdaliž nemáme mnohem více poddáni býti Otci duchu, abychom živi byli?
\par 10 A onino zajisté po nemnohé dny, jakž se jim videlo, trestali, ale tento v vecech preužitecných, totiž k tomu, abychom došli úcastnosti svatosti jeho.
\par 11 Každé pak trestání, když prítomné jest, nezdá se býti potešené, ale smutné, než potomt rozkošné ovoce spravedlnosti prináší tem, kteríž by v nem pocviceni byli.
\par 12 Protož opuštených rukou a zemdlených kolen posilnte,
\par 13 A prímé kroky cinte nohama svýma, aby, což zkulhavelo, do konce se nevyvinulo, ale radeji uzdraveno bylo.
\par 14 Pokoje následujte se všechnemi a svatosti, bez níž žádný neuzrí Pána,
\par 15 Prohlédajíce k tomu bedlive, aby nekdo neodpadl od milosti Boží, a aby nejaký koren horkosti nepodrostl, a neucinil prekážky, skrze nejž by poškvrneni byli mnozí;
\par 16 Aby nekdo nebyl smilník, aneb ohyzdný, jako Ezau, kterýžto za jednu krmi prodal prvorozenství své.
\par 17 Víte zajisté, že potom, chteje dedicne dosáhnouti požehnání, pohrdnut jest. Nebo nenalezl místa ku pokání, ac ho koli s plácem hledal.
\par 18 Nebo nepristoupili jste k hmotné hore a k horícímu ohni, a k vichru, a k mrákote, a k bouri,
\par 19 A zvuku trouby a k hlasu slov, kterýžto hlas kdož slyšeli, prosili, aby k nim nebylo více mluveno.
\par 20 (Nebo nemohli snésti toho, což bylo praveno: A kdyby se i hovado dotklo hory, budet ukamenováno, aneb šípem postreleno.
\par 21 A tak hrozné bylo to, což videli, že i Mojžíš rekl: Lekl jsem se, až se tresu.)
\par 22 Ale pristoupili jste k hore Sionu, a k mestu Boha živého, Jeruzalému nebeskému, a k nescíslnému zástupu andelu,
\par 23 K verejnému shromáždení a k církvi prvorozených, kteríž zapsáni jsou v nebesích, a k Bohu soudci všech, a k duchum spravedlivých dokonalých,
\par 24 A k prostredníku Nového Zákona Ježíšovi, a ku pokropení krví, lépe mluvící nežli Abelova.
\par 25 Viztež, abyste neodpírali mluvícím. Nebo ponevadž onino neušli pomsty, kteríž odpírali tomu, jenž na zemi na míste Božím mluvil, cím více my, jestliže tím, kterýž s nebe mluví k nám, pohrdneme?
\par 26 Jehožto hlas tehdáž byl zemí pohnul, nyní pak propovedel, rka: Ještet já jednou pohnu netoliko zemí, ale i nebem.
\par 27 A to, že dí: Ješte jednou, svetle ukazuje pohnutelných vecí prenesení, jakožto rukama ucinených, aby zustávaly ty, jenž jsou nepohnutelné.
\par 28 Protož království prijímajíce nepohnutelné, mejmež milost, skrze kteroužto služme libe Bohu, s vážností a uctivostí.
\par 29 Nebot Buh náš jest ohen spalující.

\chapter{13}

\par 1 Láska bratrská zustávejž mezi vámi.
\par 2 Na prívetivost k hostem nezapomínejte; skrze ni zajisté nekterí, nevedevše, andely za hoste prijímali.
\par 3 Pomnete na vezne, jako byste spolu veznové byli, na soužené, jakožto ti, jenž také v tele jste.
\par 4 Poctivét jest u všech lidí manželství a lože nepoškvrnené, smilníky pak a cizoložníky souditi bude Buh.
\par 5 Obcování vaše budiž bez lakomství, dosti majíce na tom, což máte. Ont zajisté rekl: Nikoli nenechám tebe tak, aniž te opustím;
\par 6 Tak abychom doufanlive ríkali: Pán spomocník muj, aniž se budu báti, by mi co uciniti mohl clovek.
\par 7 Zpomínejte na vudce vaše, kteríž vám mluvili slovo Boží, jejichžto jaký byl cíl obcování, spatrujíce, následujtež jejich víry.
\par 8 Ježíš Kristus jest vcera i dnes, tentýž i na veky.
\par 9 Ucením rozlicným a cizím nedejte se tociti. Výbornét jest zajisté, aby milostí upevneno bylo srdce, a ne pokrmy, kteríž neprospeli tem, jenž se jimi vázali.
\par 10 Mámet oltár, z nehož nemají moci jísti ti, jenž stánku slouží.
\par 11 Nebo kterýchž hovad krev vnášína bývá do svatyne skrze nejvyššího kneze za hrích, tech pálena bývají tela vne za stany.
\par 12 Protož i Ježíš, aby posvetil lidu skrze svou vlastní krev, vne za branou trpel.
\par 13 Vyjdemež tedy k nemu ven z stanu, pohanení jeho nesouce.
\par 14 Nebot nemáme zde mesta zustávajícího, ale onoho budoucího hledáme.
\par 15 Protož skrze neho obetujme Bohu obet chvály vždycky, to jest ovoce rtu, oslavujících jméno jeho.
\par 16 Na úcinnost pak a na sdílnost nezapomínejte; nebo v takových obetech zvláštní svou libost má Buh.
\par 17 Povolni budte správcum vašim a poslušni budte jich; onit zajisté bdejí nad dušemi vašimi, jako ti, kteríž pocet mají vydati; at by to s radostí cinili, a ne s stýskáním. Nebot by vám to nebylo užitecné.
\par 18 Modltež se za nás; doufámet zajisté, že dobré svedomí máme, jakožto ti, kteríž se chceme ve všem chvalitebne chovati.
\par 19 Procež prosímt vás, abyste to hojneji cinili, abych tím dríve navrácen byl vám.
\par 20 Buh pak pokoje, kterýž toho velikého pro krev smlouvy vecné Pastýre ovcí vzkrísil z mrtvých, Pána našeho Ježíše,
\par 21 Uciniž vás zpusobné ve všelikém skutku dobrém, k cinení vule jeho, pusobe v vás to, což jest libé pred oblicejem jeho, skrze Jezukrista, jemuž sláva na veky veku. Amen.
\par 22 Prosímt pak vás, bratrí, snestež trpelive slovo napomenutí tohoto; nebot jsem krátce psal vám.
\par 23 Veztež o bratru Timoteovi, že jest propušten, s kterýmžto, (prišel-li by brzo), navštívím vás.
\par 24 Pozdravtež všech správcu svých, i všech svatých. Pozdravují vás bratri z Vlach.
\par 25 Milost Boží se všemi vámi. Amen. K Židum psán jest z Vlach po Timoteovi.


\end{document}