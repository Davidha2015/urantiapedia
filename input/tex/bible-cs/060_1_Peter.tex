\begin{document}

\title{1 Peter}

\chapter{1}

\par 1 Petr, apoštol Ježíše Krista, príchozím rozptýleným v Pontu, Galacii, Kappadocii, v Azii a v Bitynii,
\par 2 Vyvoleným podle predzvedení Boha Otce, v posvecení Ducha svatého, ku poslušenství a skropení krví Ježíše Krista: Milost vám a pokoj rozmnožen bud.
\par 3 Požehnaný Buh a Otec Pána našeho Ježíše Krista, kterýžto podle mnohého milosrdenství svého znovu zplodil nás v nadeji živou skrze vzkríšení Ježíše Krista z mrtvých,
\par 4 K dedictví neporušitelnému a neposkvrnenému a neuvadlému, kteréž se chová v nebesích, nám,
\par 5 Kterížto mocí Boží ostríháni býváme skrze víru k spasení, kteréž hotovo jest, aby zjeveno bylo v casu posledním.
\par 6 V cemžto veselíte se, malicko nyní, (jestliže kdy potrebí jest,) zkormouceni jsouce v rozlicných pokušeních,
\par 7 Aby zkušení víry vaší, kteráž jest mnohem dražší nežli zlato, jenž hyne, avšak se v ohni zkušuje, nalezeno bylo vám k chvále, a ke cti i k sláve pri zjevení Ježíše Krista.
\par 8 Kteréhožto nevidevše, milujete; kteréhožto nyní nevidouce, avšak v neho veríce, veselíte se radostí nevýmluvnou a oslavenou,
\par 9 Docházejíce konce víry vaší, spasení duší vašich,
\par 10 O kterémžto spasení bedlive premyšlovali a je vystihati usilovali proroci, kteríž o té milosti, jenž se vám státi mela, prorokovali,
\par 11 Snažujíce se tomu vyrozumeti, na který aneb jaký cas mínil by ten, kterýž v nich byl, Duch Kristuv, predpovídající o utrpeních Kristových a o veliké sláve za tím jdoucí.
\par 12 Kterýmž zjeveno jest, že ne sobe, ale nám tím prisluhovali, což jest vám nyní zvestováno skrze ty, kteríž vám kázali evangelium, v Duchu svatém seslaném s nebe, na kteréžto veci žádostivi jsou andelé patriti.
\par 13 Protož prepášíce bedra mysli vaší, a strízlivi jsouce, dokonale doufejte v té milosti, kteráž vám dána bude pri zjevení Ježíše Krista,
\par 14 Jakožto synové poslušní, neprirovnávajíce se prvním neznámosti vaší žádostem.
\par 15 Ale jakž ten, kterýž vás povolal, Svatý jest, i vy svatí ve všem obcování vašem, budte;
\par 16 Jakož napsáno jest: Svatí budte, nebo já Svatý jsem.
\par 17 A ponevadž Otcem nazýváte toho, kterýž bez prijímání osob soudí vedle skutku jednoho každého, viztež, abyste v bázni Páne cas vašeho zde putování konali,
\par 18 Vedouce, že ne temi porušitelnými vecmi, stríbrem nebo zlatem, vykoupeni jste z marného vašeho obcování podle ustanovení otcu,
\par 19 Ale drahou krví jakožto Beránka nevinného a neposkvrneného, Krista,
\par 20 Predzvedeného zajisté pred ustanovením sveta, zjeveného pak v casích posledních pro vás,
\par 21 Kteríž skrze neho veríte v Boha, jenž jej vzkrísil z mrtvých, a dal jemu slávu, tak aby víra vaše a nadeje byla v Bohu.
\par 22 Ponevadž duše své ocistili jste poslušenstvím pravdy skrze Ducha svatého, k milování neošemetnému bratrstva, z cistého tedy srdce jedni druhé milujte snažne,
\par 23 Znovu zrozeni jsouce ne z porušitelného semene, ale z neporušitelného, totiž skrze živé slovo Boží a zustávající na veky.
\par 24 Nebo všeliké telo jest jako tráva, a všeliká sláva cloveka jako kvet trávy. Usvadla tráva, a kvet její spadl,
\par 25 Ale slovo Páne zustává na veky. Totot pak jest to slovo, kteréž zvestováno jest vám.

\chapter{2}

\par 1 Protož složíce zlost, a všelikou lest, a pokrytství, a závist, i všecka utrhání,
\par 2 Jakožto nyní zrozená nemluvnátka, mléka beze lsti, to jest Božího slova žádostivi budte, abyste jím rostli,
\par 3 Jestliže však okusili jste, kterak dobrotivý jest Pán.
\par 4 K kterémužto pristupujíce, jakožto k kameni živému, od lidí zajisté zavrženému, ale od Boha vyvolenému a drahému,
\par 5 I vy, jakožto kamení živé, vzdelávejte se v dum duchovní, knežstvo svaté, k obetování duchovních obetí, vzácných Bohu skrze Jezukrista.
\par 6 A protož praví Písmo: Aj, zakládámt na Sionu kámen úhelný, vybraný a drahý, a v kterýž kdokoli verí, nikoli nebude zahanben.
\par 7 Vám tedy verícím jest drahý, ale nepovolným kámen, kterýž zavrhli ti, jenž staveli, tent jest ucinen v hlavu úhelní, a kámen úrazu, a skála pohoršení,
\par 8 Totiž tem, jenž se urážejí na slovu, nepovolní jsouce, k cemuž i odloženi jsou.
\par 9 Ale vy jste rod vyvolený, královské knežstvo, národ svatý, lid dobytý, abyste zvestovali ctnosti toho, kterýž vás povolal ze tmy v predivné svetlo své.
\par 10 Kteríž jste nekdy ani lidem nebyli, nyní pak jste lid Boží; kteríž jste nekdy byli nedošli milosrdenství, již nyní jste došli milosrdenství.
\par 11 Nejmilejší, prosímt vás, abyste jakožto príchozí a pohostinní zdržovali se od telesných žádostí, kteréž ryterují proti duši,
\par 12 Obcování vaše mezi pohany majíce dobré, aby místo toho, kdež utrhají vám jako zlocincum, dobré skutky vaše spatrujíce, velebili Boha v den navštívení.
\par 13 Poddáni tedy budte všelikému lidskému zrízení pro Pána, budto králi, jako nejvyššímu,
\par 14 Budto vladarum, jako od neho poslaným, ku pomste zle cinících a k chvále dobre cinících.
\par 15 Nebo tak jest vule Boží, abyste dobre ciníce, zacpali ústa nemoudrých lidí z neznámosti vám utrhajících.
\par 16 Budtež jako svobodní, však ne jako zastrení majíce své zlosti svobodu, ale jakožto služebníci Boží.
\par 17 Všecky ctete, bratrstvo milujte, Boha se bojte, krále v uctivosti mejte.
\par 18 Služebníci poddáni budte ve vší bázni pánum, netoliko dobrým a mírným, ale i zlým.
\par 19 Neb tot jest milé, jestliže kdo pro svedomí Boží snáší zámutky, trpe bez viny.
\par 20 Nebo jaká jest chvála, byste pak i snášeli pohlavkování, hrešíce? Ale jestliže dobre ciníce, a bez viny trpíce, snášíte, tot jest milé pred Bohem.
\par 21 Nebo i k tomu povoláni jste, jako i Kristus trpel za nás, nám pozustaviv príklad, abychom následovali šlépejí jeho.
\par 22 Kterýž hríchu neucinil, aniž jest lest nalezena v ústech jeho.
\par 23 Kterýžto, když mu zlorecili, nezlorecil zase; trpev, nehrozil, ale poroucel krivdy tomu, jenž spravedlive soudí.
\par 24 Kterýžto hríchy naše na svém tele sám vnesl na drevo, abychom hríchum zemrouce, spravedlnosti živi byli, jehož zsinalostí uzdraveni jste.
\par 25 Nebo jste byli jako ovce bloudící, ale již nyní obráceni jste ku pastýri a biskupu duší vašich.

\chapter{3}

\par 1 Též podobne i ženy budte poddané mužum svým, aby, byt pak kterí i neverili slovu, skrze zbožné obcování žen bez slova získáni byli,
\par 2 Spatrujíce v bázni svaté vaše obcování.
\par 3 Kterýchžto ozdoba budiž ne ta zevnitrní, v spletání vlasu, a províjení jich zlatem, anebo v odívání pláštu,
\par 4 Ale ten skrytý srdce clovek, záležející v neporušitelnosti krotkého a pokojného ducha, kterýžto pred oblicejem Božím velmi drahý jest.
\par 5 Tak jsou zajisté nekdy i ony svaté ženy, kteréž nadeji mely v Bohu, ozdobovaly se, poddány byvše manželum svým,
\par 6 Jako Sára poslušna byla Abrahama, pánem jej nazývajici; jejížto vy jste dcerky, dobre ciníce a nebojíce se žádného prestrašení.
\par 7 Též podobne i muži spolu s nimi bydlíce podle umení, jakožto mdlejšímu osudí ženskému udelujíce cti, jakožto i spoludedickám života milosti, aby modlitby vaše nemely prekážky.
\par 8 A tak sumou, všickni budte jednomyslní, jedni druhých bíd citelní, bratrstva milovníci, milosrdní, dobrotiví,
\par 9 Neodplacujíce zlého za zlé, ani zlorecenství za zlorecenství, ale radeji dobrorecíce, vedouce, že jste k tomu povoláni, abyste požehnání dedicne obdrželi.
\par 10 Nebo kdož chce milovati život, a videti dny dobré, zkrocujž jazyk svuj od zlého, a ústa jeho at nemluví lsti.
\par 11 Uchyl se od zlého, a cin dobré; hledej pokoje, a stihej jej.
\par 12 Nebo oci Páne obrácené jsou na spravedlivé a uši jeho k prosbám jejich, ale zurivý oblicej Páne na ty, kteríž ciní zlé veci.
\par 13 A kdo jest, ješto by vám zle ucinil, jestliže budete následovníci dobrého?
\par 14 Ale kdybyste pak i trpeli pro spravedlnost, blahoslavení jste. Strachu pak jejich nebojte se, ani se kormutte,
\par 15 Ale Pána Boha posvecujte v srdcích vašich. Hotovi pak budte vždycky k vydání poctu všelikému, kdož by od vás požádal zprávy z nadeje té, kteráž jest v vás, a to s tichostí a s bázní,
\par 16 Majíce dobré svedomí, aby za to, že utrhají vám jako zlocincum, zahanbeni byli ti, kteríž hanejí vaše ctné v Kristu obcování.
\par 17 Lépet jest zajisté, abyste dobre ciníce, líbilo-li by se tak vuli Boží, trpeli, nežli zle ciníce.
\par 18 Nebo i Kristus jedinou za hríchy trpel, spravedlivý za nespravedlivé, aby nás privedl k Bohu, umrtven jsa z strany tela, ale obživen z strany Ducha.
\par 19 Skrze nehož i tem, kteríž jsou již v žalári, duchum pricházeje kázával,
\par 20 Nekdy nepovolným, když ono jednou ocekávala Boží snášelivost za dnu Noé, když delán byl koráb, v kterémžto málo, to jest osm duší, zachováno jest u vode.
\par 21 K cemužto pripodobnen jsa nyní krest, i nás spaseny ciní, ne to telesné špíny smytí, ale dobrého svedomí u Boha dotázání, skrze vzkríšení Ježíše Krista.
\par 22 Kterýž všed v nebe, jest na pravici Boží, podmaniv sobe andely, i mocnosti, i moci.

\chapter{4}

\par 1 Ponevadž tedy Kristus trpel za nás na tele, i vy také týmž myšlením odíni budte, totiž že ten, kdož na tele trpel, prestal od hríchu,
\par 2 K tomu, aby již nebyl více telesným žádostem, ale vuli Boží v tele živ, po všecken cas, což ho ješte zustává.
\par 3 Dostit jest zajisté nám na tom prebehlém casu života našeho, v nemž jsme vuli tela podle obyceje pohanu páchali, chodivše v nestydatých chlipnostech, v žádostech, v zbytecném pití vína, v hodování, v opilství a v ohyzdném modlosloužení.
\par 4 A protož když se k nim nepripojujete v takovém jejich vydávání se v rozpustilosti, zdá se jim to cosi nového býti, a rouhají se tomu.
\par 5 Tit vydadí pocet tomu, kterýž hotov jest souditi živé i mrtvé.
\par 6 Proto jest zajisté i mrtvým kázáno evangelium, aby souzeni byli podle lidí, to jest z strany tela, ale živi byli podle Boha duchem.
\par 7 Všemut se pak približuje konec.
\par 8 A protož budte stredmí a bedliví k modlitbám. Prede vším pak lásku jedni k druhým opravdovou mejte; nebo láska prikryje množství hríchu.
\par 9 Budte vespolek prívetiví k hostem, bez reptání.
\par 10 Jeden každý jakž vzal od Boha dar, tak vespolek tím sobe prisluhujte, jako dobrí šafári rozlicné milosti Boží.
\par 11 Mluví-li kdo, mluviž jako reci Boží; jestliže kdo prisluhuje, ciniž to jakožto z moci, kteréž jemu udeluje Buh, aby ve všem slaven byl Buh skrze Jezukrista, kterémuž jest sláva a císarství na veky veku. Amen.
\par 12 Nejmilejší, nebudiž vám divný ten prišlý na vás ohen, pro zkušení vás, jako by se vám neco nového prihodilo.
\par 13 Ale z toho, že jste úcastni utrpení Kristových, radujte se, abyste i pri zjevení slávy jeho radovali se s veselím.
\par 14 Trpíte-li pohanení pro jméno Kristovo, blahoslavení jste. Nebo Duch ten slávy a Boží na vás odpocívá, kterémužto z strany jich zajisté rouhání se deje, ale z strany vaší oslavován bývá.
\par 15 Žádný pak z vás netrp jako vražedlník, aneb zlodej, neb zlocinec, anebo všetecný.
\par 16 Jestliže pak kdo trpí jako krestan, nestyd se za to, ale oslavujž Boha v té cástce.
\par 17 Nebot jest cas, aby se zacal soud od domu Božího. A ponevadž nejprv zacíná se od nás, jakýž bude konec tech, kteríž nejsou povolni evangelium Božímu?
\par 18 A ponevadž spravedlivý sotva k spasení prichází, bezbožný a hríšník kde se ukáže?
\par 19 A protož i ti, kteríž trpí podle vule Boží, jakožto vernému Stvoriteli at poroucejí duše své, dobre ciníce.

\chapter{5}

\par 1 Starších, kteríž mezi vámi jsou, prosím já spolustarší, a svedek Kristových utrpení, a budoucí, kteráž zjevena bude, slávy úcastník:
\par 2 Paste stádo Boží, kteréž pri vás jest, opatrujíce je, ne bezdeky, ale dobrovolne, ne pro mrzký zisk, ale ochotne,
\par 3 Ani jako panujíce nad dedictvím Páne, ale jako príkladem jsouce stádu.
\par 4 A když se ukáže kníže pastýru, vezmete tu neuvadlou korunu slávy.
\par 5 Podobne i mládenci, budte poddáni starším. A všickni poddanost jedni druhým ukazujte, pokorou vnitr se ozdobte. Buh zajisté pyšným se protiví, ale pokorným dává milost.
\par 6 Pokortež se tedy pod mocnou ruku Boží, aby vás povýšil casem svým,
\par 7 Všelikou péci vaši uvrhouce na nej. Nebo ont má péci o vás.
\par 8 Strízliví budte, bdete; nebo protivník váš dábel jako lev rvoucí obchází, hledaje, koho by sežral.
\par 9 Jemužto odpírejte, silní jsouce u víre, vedouce, že tatáž utrpení bratrstvo vaše, kteréž na svete jest, obklicují.
\par 10 Buh pak všeliké milosti, kterýž povolal nás k vecné sláve své v Kristu Ježíši, když malicko potrpíte, on dokonalé vás ucin, utvrd, zmocni i upevni.
\par 11 Jemuž sláva a císarství na veky veku. Amen.
\par 12 Po Silvánovi, vám verném bratru, tak za to mám, že jsem psal vám krátce, napomínaje a osvedcuje, že tato jest pravá milost Boží, v kteréž stojíte.
\par 13 Pozdravuje vás ta církev, kteráž jest v Babylone, úcastnice vyvolení vašeho, a Marek syn muj.
\par 14 Pozdravtež jedni druhých v políbení laskavém. Pokoj vám všechnem, kteríž jste v Kristu Ježíši. Amen.


\end{document}