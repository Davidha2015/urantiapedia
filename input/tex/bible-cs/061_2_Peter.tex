\begin{document}

\title{2 list Petrův}

\chapter{1}

\par 1 Šimon Petr, služebník a apoštol Ježíše Krista, tem, kteríž spolu s námi zároven drahé dosáhli víry, pro spravedlnost Boha našeho a Spasitele Jezukrista:
\par 2 Milost vám a pokoj rozmnožen bud skrze známost Boha a Ježíše Pána našeho.
\par 3 Jakož nám od jeho Božské moci všecko, což potrebí bylo k životu a ku zbožnosti, darováno jest, skrze známost toho, kterýž povolal nás k sláve a k ctnosti.
\par 4 Odkudžto veliká nám a drahá zaslíbení dána jsou, tak abyste skrze ne Božského prirození úcastníci ucineni byli, utekše porušení toho, kteréž jest na svete v žádostech zlých.
\par 5 A na to pak vy všecku snažnost svou vynaložíce, prokazujte u víre své ctnost, a v ctnosti umení,
\par 6 V umení pak zdrželivost, a v zdrželivosti trpelivost, v trpelivosti pak zbožnost,
\par 7 V zbožnosti pak bratrstva milování, a v milování bratrstva lásku.
\par 8 Ty zajisté veci když budou pri vás a to rozhojnené, ne prázdné, ani neužitecné postaví vás v známosti Pána našeho Jezukrista.
\par 9 Nebo pri komž není techto vecí, slepýt jest, a toho, což vzdáleno jest, nevida, zapomenuv na ocištení svých starých hríchu.
\par 10 Protož, bratrí, radeji snažte se pevné povolání své i vyvolení uciniti; nebo to ciníce, nepadnete nikdy.
\par 11 Takt zajisté hojné zpusobeno vám bude vjití k vecnému království Pána našeho a Spasitele Jezukrista.
\par 12 Protož nezanedbámt vždycky vám pripomínati tech vecí, ackoli umelí i utvrzení jste v prítomné pravde.
\par 13 Nebot to mám za spravedlivé, dokudž jsem v tomto stánku, abych vás probuzoval napomínáním,
\par 14 Veda, že brzké jest složení stánku mého, jakož mi i Pán náš Ježíš Kristus oznámil.
\par 15 Pricinímt se i o to, abyste vy po odchodu mém casto se na ty veci rozpomínati mohli.
\par 16 Nebo ne nejakých vtipne složených básní následujíce, známu ucinili jsme vám Pána našeho Jezukrista moc a príchod, ale jakožto ti, kteríž jsme ocima svýma videli jeho velebnost.
\par 17 Prijalt jest zajisté od Boha Otce cest a slávu, když se stal k nemu hlas takový od velebné slávy: Tentot jest ten muj milý Syn, v nemž mi se zalíbilo.
\par 18 A ten hlas my jsme slyšeli s nebe pošlý, s ním byvše na oné hore svaté.
\par 19 A mámet prepevnou rec prorockou, kteréžto že šetríte jako svíce, jenž svítí v temném míste, dobre ciníte, až by se den rozednil a dennice vzešla v srdcích vašich,
\par 20 Toto nejprve znajíce, že žádného proroctví Písma svatého výklad nezáleží na rozumu lidském.
\par 21 Nebo nikdy z lidské vule nepošlo proroctví, ale Duchem svatým puzeni jsouce, mluvili svatí Boží lidé.

\chapter{2}

\par 1 Bývali pak i falešní proroci v lidu, jakož i mezi vámi budou falešní ucitelé, kteríž chytre uvedou sekty zatracení, i toho Pána, kterýž je vykoupil, zapírajíce, uvodíce na sebe rychlé zahynutí.
\par 2 A mnozí následovati budou jejich zahynutí, skrze než cesta pravdy bude v porouhání dávána.
\par 3 A lakome skrze vymyšlené reci vámi kupciti budou; kterýchžto odsouzení již dávno nemešká, a zahynutí jejich nespí.
\par 4 Nebo ponevadžt Buh andelum, kteríž zhrešili, neodpustil, ale strhna je do žaláre, retezum mrákoty oddal, aby k odsouzení chováni byli,
\par 5 I onomu prvnímu svetu neodpustil, ale sama osmého Noé, kazatele spravedlnosti, zachoval, když potopu na svet bezbožníku uvedl.
\par 6 A mesta Sodomských a Gomorských v popel obrátiv, podvrácením odsoudil, príklad budoucím bezbožníkum na nich ukázav,
\par 7 A spravedlivého Lota, ztrápeného tech nešlechetníku chlipným obcováním, vytrhl.
\par 8 Ten zajisté spravedlivý, bydliv mezi nimi, den ode dne hledením i slyšením spravedlivou duši nešlechetnými jejich skutky trápil.
\par 9 Umít Pán zbožné z pokušení vytrhnouti, nepravých pak ke dni soudu potrestaných dochovati,
\par 10 A zvlášte tech, jenž po tele v žádosti necisté chodí, a vrchností pohrdají, jsou i smelí, sobe se zalibující, neostýchají se dustojnostem porouhati.
\par 11 Ješto andelé, jsouce vetší v síle a v moci, neciní proti nim prede Pánem potupného soudu.
\par 12 Tito pak jako nerozumná hovada, kteráž za prirozením jdou, zplozená k zjímání a k zahynutí, tomu, cemuž nerozumejí, rouhajíce se, v tom svém porušení zahynou,
\par 13 A tak odplatu nepravosti své ponesou, jakožto ti, kterížto sobe za rozkoš položili, aby se na každý den v libostech svých kochali, nejsouce než poskvrny a mrzkosti, ti, kteríž s vámi hodujíce, v svých lstech se kochají,
\par 14 Oci majíce plné cizoložstva, a bez prestání hrešící, preluzujíce duše neustavicné, srdce majíce vycvicené v lakomství, synové zlorecenství.
\par 15 Kteríž opustivše cestu prímou, zbloudili, následujíce cesty Balámovy, syna Bozorova, kterýž mzdu nepravosti zamiloval.
\par 16 Ale mel, od koho by pokárán byl pro svuj výstupek. Nebo jhu poddaná oslice nemá, clovecím hlasem promluvivši, zbránila nemoudrosti proroka.
\par 17 Tit jsou studnice bez vody, a mlhy vichrem zbourené, jimžto mrákota tmy chová se na vecnost.
\par 18 Nebo prepyšne marné veci vypravujíce, žádostmi tela a chlipnostmi loudí ty, kteríž byli vpravde utekli od tech, jenž bludu obcují,
\par 19 Slibujíce jim svobodu, ješto sami jsou služebníci porušení, ponevadž od kohož kdo jest premožen, tomu jest i v službu podroben.
\par 20 Jestliže pak ti, jenž ušli poskvrn sveta, skrze známost Pána a Spasitele Jezukrista, opet zase v to zapleteni jsouce, premoženi byli, ucinen jest poslední zpusob jejich horší nežli první.
\par 21 Lépe by zajisté jim bylo nepoznávati cesty spravedlnosti, nežli po nabytí známosti odvrátiti se od vydaného jim svatého prikázání.
\par 22 Ale prihodilo se jim to, což se v prísloví pravém ríkává: Pes navrátil se k vývratku svému, a svine umytá do kalište bláta.

\chapter{3}

\par 1 Nejmilejší, již toto druhý list vám píši, v kterýchžto listech vzbuzuji skrze napomínání vaši uprímnou mysl,
\par 2 Abyste pamatovali na slova predpovedená od svatých proroku, a na prikázání vydané vám od nás apoštolu Pána a Spasitele,
\par 3 Toto nejprve vedouce, žet prijdou v posledních dnech posmevaci, podle svých vlastních žádostí chodící,
\par 4 A ríkající: Kdež jest to zaslibování príchodu jeho? Nebo jakž otcové naši zesnuli, všecko tak trvá od pocátku stvorení.
\par 5 Tohot zajisté z úmysla vedeti nechtí, že nebesa již dávno slovem Božím byla ucinena, i zeme z vody a na vode upevnena.
\par 6 Procež onen první svet vodou jsa zatopen, zahynul.
\par 7 Ta pak nebesa, kteráž nyní jsou, i zeme, týmž slovem odložena jsou a zachována k ohni, ke dni soudu a zatracení bezbožných lidí.
\par 8 Ale tato jedna vec nebudiž pred vámi skryta, nejmilejší, že jeden den u Pána jest jako tisíc let, a tisíc let jako jeden den.
\par 9 Nemeškát Pán s naplnením slibu, (jakož nekterí za to mají, že obmeškává,) ale shovívá nám, nechte, aby kterí zahynuli, ale všickni ku pokání se obrátili.
\par 10 Prijdet zajisté den Páne, jako zlodej v noci, v kterémžto nebesa jako v prudkosti vichru pominou, a živlové pálivostí ohne rozplynou se, zeme pak i ty veci, kteréž jsou na ní, vypáleny budou.
\par 11 Ponevadž tedy to všecko má se rozplynouti, jací pak vy býti máte v svatých obcováních a v zbožnosti,
\par 12 Ocekávajíce a chvátajíce ku príští dne Božího, v nemžto nebesa, horíce, rozpustí se, a živlové pálivostí ohne rozplynou se?
\par 13 Nového pak nebe a nové zeme podle zaslíbení jeho cekáme, v kterýchžto spravedlnost prebývá.
\par 14 Protož, nejmilejší, takových vecí cekajíce, snažtež se, abyste bez poskvrny a bez úhony pred ním nalezeni byli v pokoji;
\par 15 A Pána našeho dlouhocekání za spasení mejte, jakož i milý bratr náš Pavel, podle sobe dané moudrosti, psal vám,
\par 16 Jako i ve všech epištolách svých, mluve v nich o tech vecech. Mezi nimiž nekteré jsou nesnadné k vyrozumení, kterýchžto neucení a neutvrzení natahují, jako i jiných Písem, k svému vlastnímu zatracení.
\par 17 Vy tedy, nejmilejší, to prve vedouce, streztež se, abyste bludem tech nešlechetných lidí nebyli pojati a nevypadli od své pevnosti.
\par 18 Ale rozmáhejtež se v milosti a v známosti Pána našeho a Spasitele Jezukrista, jemuž sláva i nyní i na casy vecné. Amen.


\end{document}