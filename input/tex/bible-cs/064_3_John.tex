\begin{document}

\title{3 Janův}

\chapter{1}

\par 1 Starší Gáiovi milému, kteréhož já miluji v pravde.
\par 2 Nejmilejší, žádámt obzvláštne, abys se dobre mel a zdráv byl, tak jako duše tvá dobre se má.
\par 3 Zradovalt jsem se zajisté velice, když prišli bratrí a svedectví vydávali o tvé uprímnosti, vypravujíce, kterak ty v uprímnosti chodíš.
\par 4 Nemámt vetší radosti, nežli abych slyšel, že synové moji chodí v uprímnosti.
\par 5 Nejmilejší, verne deláš, cožkoli ciníš bratrím a hostem,
\par 6 Kterížto svedectví vydali o lásce tvé pred církví. Kteréžto vyprovodíš-li, tak jakž sluší na Boha, dobre uciníš.
\par 7 Nebot jsou pro jméno jeho vyšli, a nic nevzali od pohanu.
\par 8 Mámet tedy my takové prijímati, abychom byli pomocníci pravdy.
\par 9 Psal jsem sboru vašemu, ale Diotrefes, kterýž stojí o prvotnost mezi nimi, neprijímá nás.
\par 10 Protož prijdu-lit tam, pripomenut skutky jeho, kteréž ciní, mluve proti nám zlé reci. A nemaje dosti na tom, i sám bratrí neprijímá, i tem, kteríž by prijímati chteli, nedopouští, a ze sboru je vylucuje.
\par 11 Nejmilejší, nenásledujž zlého, ale dobrého. Kdož dobre ciní, z Boha jest; ale kdož zle ciní, nevidí Boha.
\par 12 Demetriovi svedectví vydáno jest ode všech, i od samé pravdy; ano i my svedectví o nem vydáváme, a víte, že svedectví naše pravé jest.
\par 13 Mnohot jsem mel psáti, ale nechci psáti cernidlem a pérem.
\par 14 Nebo mám nadeji, že te tudíž uzrím, a ústy k ústum mluviti budeme.
\par 15 Pokoj budiž tobe. Pozdravují te prátelé. Pozdraviž i ty dobrých prátel ze jména.


\end{document}