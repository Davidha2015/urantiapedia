\begin{document}

\title{Første Mosebog}



\chapter{1}

\par 1 I Begyndelsen skabte Gud Himmelen og Jorden.
\par 2 Og Jorden var øde og tom, og der var Mørke over Verdensdybet. Men Guds Ånd svævede over Vandene.
\par 3 Og Gud sagde: "Der blive Lys!" Og der blev Lys.
\par 4 Og Gud så, at Lyset var godt, og Gud satte Skel mellem Lyset og Mørket,
\par 5 og Gud kaldte Lyset Dag, og Mørket kaldte han Nat. Og det blev Aften, og det blev Morgen, første Dag.
\par 6 Derpå sagde Gud: "Der blive en Hvælving midt i Vandene til at skille Vandene ad!"
\par 7 Og således skete det: Gud gjorde Hvælvingen og skilte Vandet under Hvælvingen fra Vandet over Hvælvingen;
\par 8 og Gud kaldte Hvælvingen Himmel. Og det blev Aften, og det blev Morgen, anden Dag.
\par 9 Derpå sagde Gud: "Vandet under Himmelen samle sig på eet Sted, så det faste Land kommer til Syne!" Og således skete det;
\par 10 og Gud kaldte det faste Land Jord, og Stedet, hvor Vandet samlede sig, kaldte han Hav. Og Gud så, at det var godt.
\par 11 Derpå sagde Gud: "Jorden lade fremspire grønne Urter, der bærer Frø, og Frugttræer, der bærer Frugt med Kærne, på Jorden!" Og således skete det:
\par 12 Jorden frembragte grønne Urter, der bar Frø, efter deres Arter, og Træer, der bar Frugt med Kærne, efter deres Arter. Og Gud så, at det var godt.
\par 13 Og det blev Aften, og det blev Morgen, tredje Dag.
\par 14 Derpå sagde Gud: "Der komme Lys på Himmelhvælvingen til at skille Dag fra Nat, og de skal være til Tegn og til Fastsættelse af Højtider, Dage og År
\par 15 og tjene som Lys på Himmelhvælvingen til at lyse på Jorden! Og således sket det:
\par 16 Gud gjorde de to store Lys, det største til at herske om Dagen, det mindste til at herske om Natten, og Stjernerne;
\par 17 og Gud satte dem på Himmelhvælvingen til at lyse på Jorden
\par 18 og til at herske over Dagen og Natten og til at skille Lyset fra Mørket. Og Gud så, at det var godt.
\par 19 Og det blev Aften, og det blev Morgen, fjerde Dag.
\par 20 Derpå sagde Gud: "Vandet vrimle med en Vrimmel af levende Væsener, og Fugle flyve over Jorden oppe under Himmelhvælvingen!" Og således skete det:
\par 21 Gud skabte de store Havdyr og den hele Vrimmel af levende Væsener, som Vandet vrimler med, efter deres Arter, og alle vingede Væsener efter deres Arter. Og Gud så, at det var godt.
\par 22 Og Gud velsignede dem og sagde: "Bliv frugtbare og mangfoldige og opfyld Vandet i Havene, og Fuglene blive mangfoldige på Jorden!"
\par 23 Og det blev Aften, og det blev Morgen, femte Dag.
\par 24 Derpå sagde Gud: "Jorden frembringe levende Væsener efter deres Arter: Kvæg, Kryb og vildtlevende Dyr efter deres Arter! Og således skete det:
\par 25 Gud gjorde de vildtlevende Dyr efter deres Arter, Kvæget efter dets Arter og alt Jordens Kryb efter dets Arter. Og Gud så, at det var godt.
\par 26 Derpå sagde Gud: "Lad os gøre Mennesker i vort Billede, så de ligner os, til at herske over Havets Fisk og Himmelens Fugle, Kvæget og alle vildtlevende Dyr på Jorden og alt Kryb,der kryber på Jorden!"
\par 27 Og Gud skabte Mennesket i sit Billede; i Guds Billede skabte han det, som Mand og Kvinde skabte han dem;
\par 28 og Gud velsignede dem, og Gud sagde til dem: "Bliv frugtbare og mangfoldige og opfyld Jorden, gør eder til Herre over den og hersk over Havets Fisk og Himmelens Fugle, Kvæget og alle vildtlevende Dyr, der rører sig på Jorden!"
\par 29 Gud sagde fremdeles: "Jeg giver eder alle Urter på hele Jorden, som bærer Frø, og alle Træer, som bærer Frugt med Kærne; de skal være eder til Føde;
\par 30 men alle Jordens dyr og alle Himmelens Fugle og alt, hvad der kryber på Jorden, og som har Livsånde, giver jeg alle grønne Urter til Føde." Og således skete det.
\par 31 Og Gud så alt, hvad han havde gjort, og se, det var såre godt. Og det blev Aften, og det blev Morgen, sjette Dag.

\chapter{2}

\par 1 Således fuldendtes Himmelen og Jorden med al deres Hær.
\par 2 På den syvende Dag fuldendte Gud det Værk, han havde udført, og han hvilede på den syvende Dag efter det Værk, han havde udført;
\par 3 og Gud velsignede den syvende Dag og helligede den, thi på den hvilede han efter hele sit Værk, det, Gud havde skabt og udført.
\par 4 Det er Himmelens og Jordens Skabelseshistorie. Da Gud HERREN gjorde Jord og Himmel
\par 5 dengang fandtes endnu ingen af Markens Buske på Jorden, og endnu var ingen af Markens Urter spiret frem, thi Gud HERREN havde ikke ladet det regne på Jorden, og der var ingen Mennesker til at dyrke Agerjorden,
\par 6 men en Tåge vældede op at Jorden og vandede hele Agerjordens Flade
\par 7 da dannede Gud HERREN Mennesket af Agerjordens Muld og blæste Livsånde i hans Næsebor, så at Mennesket blev et levende Væsen.
\par 8 Derpå plantede Gud HERREN en Have i Eden ude mod Øst, og dem satte han Mennesket, som han havde dannet;
\par 9 Og Gud HERREN lod af Agerjorden fremvokse alle Slags Træer, en Fryd at skue og gode til Føde, desuden Livets Træ, der stod midt i Haven. og Træet til Kundskab om godt og ondt.
\par 10 Der udsprang en Flod i Eden til at vande Haven, og udenfor delte den sig i fire Hovedstrømme.
\par 11 Den ene hedder Pisjon; den løber omkring Landet Havila, hvor der findes Guld
\par 12 og Guldet i det Land er godt, Bdellium og Sjohamsten.
\par 13 Den anden Flod hedder Gihon; den løber omkring Landet Kusj.
\par 14 Den tredje Flod hedder Hiddekel; den løber østen om Assyrien.
\par 15 Derpå tog Gud HERREN Adam og satte ham i Edens Have til at dyrke og vogte den.
\par 16 Men Gud HERREN bød Adam: "Af alle Træer i Haven har du Lov at spise,
\par 17 kun af Træet til Kundskab om godt og ondt må du ikke spise; den Dag du spiser deraf, skal du visselig dø!"
\par 18 Derpå sagde Gud HERREN: "Det er ikke godt for Mennesket at være ene; jeg vil gøre ham en Medhjælp, som passer til ham!"
\par 19 Og Gud HERREN dannede af Agerjorden alle Markens Dyr og Himmelens Fugle og førte dem hen til Adam for at se, hvad han vilde kalde dem; thi hvad Adam kaldte de forskellige levende Væsener, det skulde være deres Navn.
\par 20 Adam gav da alt Kvæget, alle Himmelens Fugle og alle Markens Dyr Navne - men til sig selv fandt Adam ingen Medhjælp, der passede til ham.
\par 21 Så lod Gud HERREN Dvale falde over Adam, og da han var sovet ind, tog han et af hans Ribben og lukkede med Kød i dets Sted;
\par 22 og af Ribbenet, som Gud HERREN havde taget af Adam, byggede han en Kvinde og førte hende hen til Adam.
\par 23 Da sagde Adam: "Denne Gang er det Ben af mine Ben og Kød af mit Kød; hun skal kaldes Kvinde, thi af Manden er hun taget!"
\par 24 Derfor forlader en Mand sin Fader og Moder og holder sig til sin Hustru, og de to bliver eet Kød.
\par 25 Og de var begge nøgne, både Adam og hans Hustru, men de bluedes ikke.

\chapter{3}

\par 1 Men Slangen var træskere end alle Markens andre Dyr, som Gud HERREN havde gjort og den sagde til Kvinden: "Mon Gud virkelig ham sagt: I må ikke spise af noget Træ i Haven?"
\par 2 Kvinden svarede: "Vi har Lov at spise af Frugten på Havens Træer;
\par 3 kun af Frugten fra Træet midt i Haven, sagde Gud, må I ikke spise, ja, I må ikke røre derved, thi så skal I dø!"
\par 4 Da sagde Slangen til Kvinden: "I skal ingenlunde dø;
\par 5 men Gud ved, at når I spiser deraf, åbnes eders Øjne, så I blive som Gud til at kende godt og ondt!"
\par 6 Kvinden blev nu var, at Træet var godt at spise af, en Lyst for Øjnene og godt at få Forstand af; og hun tog af dets Frugt og spiste og gav også sin Mand, der stod hos hende, og han spiste.
\par 7 Da åbnedes begges Øjne, og de kendte, at de var nøgne. Derfor syede de Figenblade sammen og bandt dem om sig.
\par 8 Da Dagen blev sval, hørte de Gud HERREN vandre i Haven, og Adam og hans Hustru skjulte sig for ham inde mellem Havens Træer.
\par 9 Da kaldte Gud HERREN på Adam og råbte: "Hvor er du?"
\par 10 Han svarede: "Jeg hørte dig i Haven og blev angst, fordi jeg var nøgen, og så skjulte jeg mig!"
\par 11 Da sagde han: "Hvem fortalte dig, at du var nøgen. Mon du har spist af det Træ, jeg sagde, du ikke måtte spise af?"
\par 12 Adam svarede: "Kvinden, som du satte ved min Side, gav mig af Træet, og så spiste jeg."
\par 13 Da sagde Gud HERREN til Kvinde: "Hvad har du gjort!" Kvinden svarede: "Slangen forførte mig, og så spiste jeg."
\par 14 Da sagde Gud HERREN til Slangen: "Fordi du har gjort dette, være du forbandet blandt al Kvæget og blandt alle Markens Dyr! På din Bug skal du krybe, og Støv skal du æde alle dit Livs Dage!
\par 15 Jeg sætter Fjendskab mellem dig og Kvinden, mellem din Sæd og hendes Sæd; den skal knuse dit Hoved, og du skal hugge den i Hælen!"
\par 16 Til Kvinden sagde han: "Jeg vil meget mangfoldiggøre dit Svangerskabs Møje; med Smerte skal du føde Børn; men til din Mand skal din Attrå være, og han skal herske over dig!"
\par 17 Og til Adam sagde han: "Fordi du lyttede til din Hustrus Tale og spiste af Træet, som jeg sagde, du ikke måtte spise af, skal Jorden være forbandet for din Skyld; med Møje skal du skaffe dig Føde af den alle dit Livs Dage;
\par 18 Torn og Tidsel skal den bære dig, og Markens Urter skal være din Føde;
\par 19 i dit Ansigts Sved skal du spise dit Brød, indtil du vender tilbage til Jorden; thi af den er du taget; ja, Støv er du, og til Støv skal du vende tilbage!"
\par 20 Men Adam kaldte sin Hustru Eva, thi hun blev Moder til alt levende.
\par 21 Derpå gjorde Gud HERREN Skindkjortlet til Adam og hans Hustru og klædte dem dermed.
\par 22 Men Gud HERREN sagde: "Se, Mennesket er blevet som en af os til at kende godt og ondt. Nu skal han ikke række Hånden ud og tage også af Livets Træ og spise og leve evindelig!"
\par 23 Så forviste Gud HERREN ham fra Edens Have, for at han skulde dyrke Jorden, som han var taget af;
\par 24 og han drev Mennesket ud, og østen for Edens Have satte han Keruberne med det glimtende Flammesværd til at vogte Vejen til Livets Træ.

\chapter{4}

\par 1 Adam kendte sin Hustru Eva,og hun blev frugtsommelig og fødte Kain; og hun sagde: "Jeg har fået en Søn med HERRENS Hjælp!"
\par 2 Fremdeles fødte hun hans Broder Abel. Abel blev Fårehyrde, Kain Agerdyrker.
\par 3 Nogen Tid efter bragte Kain HERREN en Offergave af Jordens Frugt,
\par 4 medens Abel bragte en Gave af sin Hjords førstefødte og deres Fedme. Og HERREN så til Abel og hans Offergave,
\par 5 men til Kain og hans Offergave så han ikke. Kain blev da såre vred og gik med sænket Hoved.
\par 6 Da sagde HERREN til Kain: "Hvorfor er du vred, og hvorfor går du med sænket Hoved?
\par 7 Du ved, at når du handler vel, kan du løfte Hovedet frit; men handler du ikke vel, så lurer Synden ved Døren; dens Attrå står til dig, men du skal herske over den!"
\par 8 Men Kain yppede Kiv med sin Broder Abel; og engang de var ude på Marken, sprang Kain ind på ham og slog ham ihjel.
\par 9 Da sagde HERREN til Kain: "Hvor er din Broder Abel?" Han svarede: "Det ved jeg ikke; skal jeg vogte min Broder?"
\par 10 Men han sagde: "Hvad har du gjort! Din Broders Blod råber til mig fra Jorden!
\par 11 Derfor skal du nu være bandlyst fra Agerjorden, som åbnede sig og tog din Broders Blod af din Hånd!
\par 12 Når du dyrker Agerjorden, skal den ikke mere skænke dig sin Kraft du skal flakke hjemløs om på Jorden!"
\par 13 Men Kain sagde til HERREN: "Min Straf er ikke til at bære;
\par 14 når du nu jager mig bort fra Agerjorden, og jeg må skjule mig for dit Åsyn og flakke hjemløs om på Jorden, så kan jo enhver, der møder mig, slå mig ihjel!"
\par 15 Da svarede HERREN: "Hvis Kain bliver slået ihjel, skal han hævnes;syvfold!" Og HERREN satte et Tegn på Kain, for at ingen, der mødte ham, skulde slå ham ihjel.
\par 16 Så drog Kain bort fra HERRENs Åsyn og slog sig ned i Landet Nod østen for Eden.
\par 17 Kain kendte sin Hustru, og hun blev frugtsommelig og fødte Hanok. Han grundede en By og gav den sin;Søn Hanoks Navn.
\par 18 Hanok fik en Søn Irad; Irad avlede Mehujael; Mehujael avlede Mehujael; og Metusjael avlede Lemek
\par 19 Lemek tog sig to Hustruer; den ene hed Ada, den anden Zilla.
\par 20 Ada fødte Jabal; han blev Stamfader til dem, der bor i Telte og holder Kvæg;
\par 21 hans Broder hed Jubal; han blev Stamfader til alle dem, der spiller på Harpe og Fløjte.
\par 22 Også Zilla fik en Søn, Tubal-Kajin; han blev Stamfader til alle dem, der smeder Kobber og Jern. Tubal-Kajins Søster var Na'ama.
\par 23 Og Lemelk sagde til sine Hustruer: "Ada og Zilla, hør min Røst, Lemeks Hustruer, lyt til mit Ord: En Mand har jeg dræbt for et Sår, en Dreng for en Skramme!
\par 24 Blev Kain hævnet syvfold, så hævnes Lemek syv og halvfjerdsindstyve Gange!"
\par 25 Adam kendte på ny sin Hustru, og hun fødte en Søn, som hun gav Navnet Set; "thi," sagde hun, "Gud har sat mig andet Afkom i Abels Sted, fordi Kain slog ham ihjel!"
\par 26 Set fik også en Søn, som han kaldte Enosj; på den Tid begyndte man at påkalde HERRENs Navn.

\chapter{5}

\par 1 Dette er Adams Slægtebog. Dengang Gud skabte Mennesket, gjorde han det i Guds Billede;
\par 2 som Mand og Kvinde skabte han dem, og han velsignede dem og gav dem Navnet "Menneske", da de blev skabt.
\par 3 Da Adam havde levet i I30 År, avlede han en Søn, som var ham lig og i hans Billede, og han kaldte ham Set;
\par 4 og efter at Adam havde avlet Set, levede han 800 År og avlede Sønner og Døtre;
\par 5 således blev hans fulde Levetid 930 År, og derpå døde han.
\par 6 Da Set havde levet 105 År, avlede han Enosj;
\par 7 og efter at Set havde avlet Enosj, levede han 807 År og avlede Sønner og Døtre;
\par 8 således blev Sets fulde Levetid 912 År, og derpå døde han.
\par 9 Da Enosj havde levet 90 År, avlede han Henan;
\par 10 og efter at Enosj havde avlet Kenan, levede han 815 År og avlede Sønner og Døtre;
\par 11 således blev Enosjs fulde Levetid 905 År, og derpå døde han.
\par 12 Da Kenan havde levet 70 År, avlede han Mahalal'el;
\par 13 og efter at Kenan havde avlet Mahalal'el, levede han 840 År og avlede Sønner og Døtre;
\par 14 således blev Kenans fulde Levetid 910 År, og derpå døde han.
\par 15 Da Mahalal'el havde levet 65 År, avlede han Jered;
\par 16 og efter at Mahalal'el havde avlet Jered, levede han 830 År og avlede Sønner. og Døtre;
\par 17 således blev Mahalal'els fulde Levetid 895 År, og derpå døde han.
\par 18 Da Jered havde levet 162 År, avlede han Enok;
\par 19 og efter at Jered havde avlet Enok, levede han 800 År og avlede Sønner og Døtre;
\par 20 således blev Jereds fulde Levetid 962 År, og derpå døde han.
\par 21 Da Enok havde levet 65 År, avlede han Metusalem,
\par 22 og Enok vandrede med Gud; og efter at han havde avlet Metusalem, levede han 300 År og avlede Sønner og Døtre;
\par 23 således blev Enoks fulde Levetid 365 År;
\par 24 og Enok vandrede med Gud, og han var ikke mere, thi Gud tog ham.
\par 25 Da Metusalem havde levet l87 År, avlede han Lemek;
\par 26 og efter at Metusalem havde avlet Lemek, levede han 782 År og avlede Sønner og Døtre;
\par 27 således blev Metusalems fulde Levetid 969 År, og derpå døde han.
\par 28 Da Lemek havde levet 182 År, avlede han en Søn,
\par 29 som han gav Navnet Noa, idet, han sagde: "Han skal skaffe os.
\par 30 Og efter at Lemek havde avlet Noa, levede han 595 År og avlede Sønner og Døtre;
\par 31 således blev Lemeks fulde Levetid 777 År, og derpå døde han.
\par 32 Da Noa var 500 År gammel, avlede han Sem, Kam og Jafet.

\chapter{6}

\par 1 Da nu Menneskene begyndte at blive talrige på Jorden og der fødtes dem Døtre,
\par 2 fik Gudssønnerne Øje på Menneskedøtrenes Skønhed, og de tog: så mange af dem, som de lystede, til Hustruer.
\par 3 Da sagde HERREN: "Min Ånd: skal ikke for evigt blive i Menneskene, eftersom de jo dog er Kød; deres Dage skal være 120 År."
\par 4 I hine Dage, da Gudssønnerne gik ind til Menneskedøtrene og disse fødte dem Børn men også senere hen i Tiden - levede Kæmperne på Jorden. Det er Heltene, hvis Ry når tilbage til Fortids Dage.
\par 5 Men HERREN så, at Menneskenes, Ondskab tog til på Jorden, og at deres Hjerters Higen og Tragten kun var ond Dagen lang.
\par 6 Da angrede HERREN, at han havde gjort Menneskene på Jorden, og det. skar ham i Hjertet.
\par 7 Og HERREN sagde: "Jeg vil udslette Menneskene, som jeg har skabt, af Jordens Flade, både Mennesker, Kvæg, Kryb og Himmelens, Fugle, thi jeg angrer, at jeg gjorde dem!"
\par 8 Men Noa fandt Nåde for HERRENs Øjne
\par 9 Dette er Noas Slægtebog. Noa var en retfærdig, ustraffelig Mand blandt sine samtidige; Noa vandrede med Gud.
\par 10 Noa avlede tre Sønner: Sem, Kam og Jafet.
\par 11 Men Jorden fordærvedes for Guds Øjne, og Jorden blev fuld af Uret;
\par 12 og Gud så til Jorden, og se, den var fordærvet, thi alt Kød havde fordærvet sin Vej på Jorden.
\par 13 Da sagde Gud til Noa: "Jeg har besluttet at gøre Ende på alt Kød, fordi Jorden ved deres Skyld et fuld af Uret; derfor vil jeg nu udrydde dem af Jorden.
\par 14 Men du skal gøre dig en Ark af Gofertræ og indrette den med Rum ved Rum og overstryge den med Beg både indvendig og udvendig;
\par 15 og således skal du bygge Arken: Den skal være 300 Alen lang, 50 Alen bred og 30 Alen høj;
\par 16 du skal anbringe Taget på Arken, og det skal ikke rage længer ud end een Alen fra oven; på Arkens Side skal du sætte Døren, og du skal indrette den med et nederste, et mellemste og et øverste Stokværk.
\par 17 Se, jeg bringer nu Vandfloden over Jorden for at udrydde alt Kød under Himmelen, som har Livsånde; alt, hvad der er på Jorden, skal forgå.
\par 18 Men jeg vil oprette min Pagt med dig. Du skal gå ind i Arken med dine Sønner, din Hustru og dine Sønnekoner:
\par 19 og af alle Dyr, af alt Kød skal du bringe et Par af hver Slags ind i Arken for at holde dem i Live hos dig. Han og Hundyr skal det være,
\par 20 af Fuglene efter deres Arter, af Kvæget efter dets Arter og af alt Jordens Kryb efter dets Arter; Par for Par skal de gå ind til dig for at holdes i Live.
\par 21 Og du skal indsamle et Forråd af alle Slags Levnedsmidler, for at det kan tjene dig og dem til Føde."
\par 22 Og Noa gjorde ganske som Gud havde pålagt ham; således gjorde han.

\chapter{7}

\par 1 Derpå sagde HERREN til Noa: Gå ind i Arken med hele dit Hus, thi dig har jeg fundet retfærdig for mine Øjne i denne Slægt.
\par 2 Af alle rene Dyr skal du tage syv Par, Han og Hun, og af alle urene Dyr eet Par, Han og Hun,
\par 3 ligeledes af Himmelens Fugle syv Par, Han og Hun, for at de kan forplante sig på hele Jorden.
\par 4 Thi om syv Dage vil jeg lade det regne på Jorden i fyrretyve Dage og fyrretyve Nætter og udslette alle Væsener, som jeg har gjort, fra Jordens Flade."
\par 5 Og Noa gjorde ganske som HERREN havde pålagt ham.
\par 6 Noa var 600 År gammel, da Vandfloden kom over Jorden.
\par 7 Noa gik med sine Sønner, sin Hustru og sine Sønnekoner ind i Arken for at undslippe Flodens Vande.
\par 8 De rene og de urene Dyr, Fuglene og alt, hvad der kryber på Jorden,
\par 9 gik Par for Par ind i Arken til Noa, Han og Hundyr, som Gud havde pålagt Noa.
\par 10 Da nu syv Dage var omme, kom Flodens Vande over Jorden;
\par 11 i Noas 600de Leveår på den syttende Dag i den anden Måned, den Dag brast det store Verdensdybs Kilder, og Himmelens Sluser åbnedes,
\par 12 og Regnen faldt over Jorden i fyrretyve Dage og fyrretyve Nætter.
\par 13 Selvsamme Dag gik Noa ind i Arken og med ham hans Sønner Sem, Kam og Jafet, hans Hustru og hans tre Sønnekoner
\par 14 og desuden alle vildtlevende Dyr efter deres Arter, alt Kvæg efter dets Arter, alt Kryb på Jorden efter dets Arter og alle Fugle efter deres Arter, alle Fugle, alt, hvad der har Vinger;
\par 15 af alt Kød, som har Livsånde, gik Par for Par ind i Arken til Noa;
\par 16 Han og Hundyr af alt Kød gik ind, som Gud havde påbudt, og HERREN lukkede efter ham.
\par 17 Da kom Vandfloden over Jorden i fyrretyve Dage, og Vandet steg og løftede Arken, så den hævedes over Jorden.
\par 18 Og Vandet steg og stod højt over Jorden, og Arken flød på Vandet;
\par 19 og Vandet steg og steg over Jorden, så de højeste Bjerge under Himmelen stod under Vand;
\par 20 femten Alen stod Vandet over dem, så Bjergene stod helt under Vand.
\par 21 Da omkom alt Kød, som rørte sig på Jorden, Fugle, Kvæg, vildtlevende Dyr og alt Kryb på Jorden og alle Mennesker;
\par 22 alt, i hvis Næse det var Livets Ånde, alt, hvad der var på det faste Land, døde.
\par 23 Således udslettedes alle Væsener, der var på Jordens Flade, Mennesker, Kvæg, Kryb og Himmelens Fugle; de udslettedes af Jorden, og tilbage blev kun Noa og de, der var hos ham i Arken.
\par 24 Vandet steg over Jorden i 150 Dage.

\chapter{8}

\par 1 Da ihukom Gud Noa og alle de vilde Dyr og Kvæget, som var hos ham i Arken; og Gud lod en Storm fare hen over Jorden, så at Vandet begyndte at falde;
\par 2 Verdensdybets Kilder og Himmelens Sluser lukkedes, Regnen fra Himmelen standsede,
\par 3 og Vandet veg lidt efter lidt bort fra Jorden, og Vandet tog af efter de 150 Dages Forløb.
\par 4 På den syttende Dag i den syvende Måned sad Arken fast på Ararats Bjerge,
\par 5 og Vandet vedblev at synke indtil den tiende Måned, og på den første Dag i denne Måned dukkede Bjergenes Toppe frem.
\par 6 Da der var gået fyrretyve Dage: åbnede Noa den Luge, han havde lavet på Arken,
\par 7 og sendte en Ravn ud; den fløj frem og tilbage, indtil Vandet var tørret bort fra Jorden.
\par 8 Da sendte han en Due ud for at se, om Vandet var sunket fra Jordens Overflade;
\par 9 men Duen fandt intet Sted at sætte sin Fod og vendte tilbage til ham i Arken, fordi der endnu var Vand over hele Jorden; og han rakte Hånden ud og tog den ind i Arken til sig.
\par 10 Derpå biede han yderligere syv Dage og sendte så atter Duen ud fra Arken;
\par 11 ved Aftenstid kom Duen tilbage til ham, og se, den havde et friskt Olieblad i Næbbet. Da skønnede Noa, at Vandet var svundet bort fra Jorden.
\par 12 Derpå biede han syv Dage til, og da han så sendte Duen ud, kom den ikke mere tilbage til ham.
\par 13 I Noas 601ste Leveår på den første Dag i den første Måned var Vandet tørret bort fra Jorden. Da tog Noa Dækket af Arken, og da han så sig om, se, da var Jordens Overflade tør.
\par 14 På den syv og tyvende Dag i den anden Måned var Jorden tør.
\par 15 Da sagde Gud til Noa:
\par 16 "Gå ud af Arken med din Hustru, dine Sønner og dine Sønnekoner
\par 17 og for alle Dyr, der er hos dig, alt Kød, Fugle, Kvæg og alt Kryb, der kryber på Jorden, ud med dig, at de kan vrimle på Jorden og blive frugtbare og mangfoldige på Jorden!"
\par 18 Da gik Noa ud med sine Sønner, sin Hustru og sine Sønnekoner;
\par 19 og alle de vildtlevende Dyr, alt Kvæget, alle Fuglene og alt Krybet, der kryber på Jorden, efter deres Slægter, gik ud af Arken.
\par 20 Derpå byggede Noa HERREN et Alter og tog nogle af alle de rene Dyr og Fugle og ofrede Brændofre på Alteret.
\par 21 Og da HERREN indåndede den liflige Duft, sagde han til sig selv: "Jeg vil aldrig mere forbande Jorden for Menneskenes Skyld, thi Menneskehjertets Higen er ond fra Ungdommen af, og jeg vil aldrig mere tilintetgøre alt, hvad der lever, således som jeg nu har gjort!
\par 22 Herefter skal, så længe Jorden står, Sæd og Høst, Kulde og Hede, Sommer og Vinter, Dag og Nat ikke ophøre!"

\chapter{9}

\par 1 Derpå velsignede Gud Noa og hans Sønner og sagde til dem: Bliv frugtbare og mangfoldige og opfyld Jorden!
\par 2 Frygt for eder og Rædsel for eder skal være over alle Jordens vildtlevende Dyr og alle Himmelens Fugle og i alt, hvad Jorden vrimler med, og i alle Havets Fisk; i eders Hånd er de givet!
\par 3 Alt, hvad der rører sig og lever, skal tjene eder til Føde; ligesom de grønne Urter giver jeg eder det alt sammen.
\par 4 Dog Kød med Sjælen, det er Blodet, må I ikke spise!
\par 5 Men for eders eget Blod kræver jeg Hævn; af ethvert Dyr kræver jeg Hævn for det, og af Menneskene indbyrdes kræver jeg Hævn for Menneskenes Liv.
\par 6 Om nogen udøser Menneskers Blod, ved Mennesker skal hans Blod udøses, thi i sit Billede gjorde Gud Menneskene.
\par 7 Men I skal blive frugtbare og mangfoldige! Opfyld Jorden og gør eder til Herre over den!"
\par 8 Derpå sagde Gud til Noa og hans Sønner:
\par 9 "Se, jeg opretter min Pagt med eder og eders Efterkommere efter eder
\par 10 og med hvert levende Væsen, som er hos eder, Fuglene, Kvæget og alle Jordens vildtlevende Dyr, alt, hvad der gik ud af Arken, alle Jordens Dyr;
\par 11 jeg opretter min Pagt med eder og lover, at aldrig mere skal alt Kød udryddes af Flodens Vande, og aldrig mere skal der komme en Vandflod for at ødelægge Jorden!"
\par 12 Fremdeles sagde Gud: "Dette er Tegnet på den Pagt, jeg til evige Tider opretter mellem mig og eder og hvert levende Væsen, som er hos eder:
\par 13 Min Bue sætter jeg i Skyen, og den skal være Pagtstegn mellem mig og Jorden!
\par 14 Når jeg trækker Skyer sammen over Jorden, og Buen da viser sig i Skyerne,
\par 15 vil jeg komme den Pagt i Hu, som består mellem mig og eder og hvert levende Væsen, det er alt Kød, og Vandet skal ikke mere blive til en Vandflod, som ødelægger alt Kød.
\par 16 Når Buen da står i Skyerne, vil jeg se hen til den og ihukomme den evige Pagt mellem Gud og hvert levende Væsen, det er alt Kød på Jorden."
\par 17 Og Gud sagde til Noa: "Det er Tegnet på den Pagt, jeg opretter imellem mig og alt Kød på Jorden!"
\par 18 Noas Sønner, der gik ud af Arken, var Sem, Kam og Jafet; Kam var Fader til Kana'an;
\par 19 det var Noas tre Sønner, og fra dem stammer hele Jordens Befolkning.
\par 20 Noa var Agerdyrker og den første, der plantede en Vingård.
\par 21 Da han nu drak af Vinen, blev han beruset og blottede sig inde i, sit Telt.
\par 22 Da så Kana'ans Fader Kam sin Faders Blusel og gik ud og fortalte sine Brødre det;
\par 23 men Sem og Jafet tog Kappen, lagde den på deres Skuldre og gik baglæns ind og tildækkede deres Faders Blusel med bortvendte Ansigter, så de ikke så deres Faders Blusel.
\par 24 Da Noa vågnede af sin Rus og fik at vide, hvad hans yngste Søn havde gjort ved ham,
\par 25 sagde han: "Forbandet være Kana'an, Trælles Træl blive han for sine Brødre!"
\par 26 Fremdeles sagde han: "Lovet være HERREN, Sems Gud, og Kana'an blive hans Træl!
\par 27 Gud skaffe Jafet Plads, at han må bo i Sems Telte; og Kana'an blive hans Træl!"
\par 28 Noa levede 350 År efter Vandfloden;
\par 29 således blev Noas fulde Levetid 950 År, og derpå døde han.

\chapter{10}

\par 1 Dette er Noas Sønner, Sem, Kam og Jafets Slægtebog. Efter Vandfloden fødtes der dem Sønner.
\par 2 Jafets Sønner: Gomer, Magog. Madaj, Javan, Tubal, Mesjek og Tiras.
\par 3 Gomers Sønner: Asjkenaz, Rifaf og Togarma.
\par 4 Javans Sønner: Elisja, Tarsis. Kittæerne og Rodosboerne;
\par 5 fra dem nedstammer de fjerne Strandes Folk. Det var Jafets Sønner i deres Lande, hver med sit Tungemål, efter deres Slægter og i deres Folkeslag.
\par 6 Kams Sønner: Kusj, Mlizrajim, Put og Hana'an.
\par 7 Kusj's Sønner: Seba, Havila, Sabta, Ra'ma og Sabteka. Ra'mas Sønner: Saba og Dedan.
\par 8 Og Kusj avlede Nimrod, som var den første Storhersker på Jorden.
\par 9 Han var en vældig Jæger for HERRENs Øjne; derfor siger man: "En vældig Jæget for HERRENs Øjne som Nimrod."
\par 10 Fra først af omfattede hans Rige Babel, Erelk, Akkad og Kalne i Sinear;
\par 11 fra dette Land drog han til Assyrien og byggede Nineve, Rehobot- Ir, Kela
\par 12 og Resen mellem Nineve og Kela, det er den store By.
\par 13 Mizrajim avlede Luderne,Anamerne, Lehaberne, Naftuherne,
\par 14 Patruserne, Kasluherne, fra hvem Filisterne udgik, og Kaftorerne.
\par 15 Kana'an avlede Zidon, hans førstefødte, Het,
\par 16 Jebusiterne, Amoriterne, Girgasjiterne,
\par 17 Hivviterne, Arkiterne, Siniterne,
\par 18 Arvaditerne, Zemariterne og Hamatiterne; men senere bredte Kana'anæernes Slægter sig,
\par 19 så at Kana'anæernes Område strakte sig fra Zidon i Retning af Gerar indtil Gaza, i Retning af Sodoma, Gomorra, Adma,og Zebojim indtil Lasja.
\par 20 Det var Kams Sønner efter deres Slægter og Tungemål i deres Lande og Folk.
\par 21 Men også Sem, alle Ebersønnernes Fader, Jafets ældste Broder, fødtes der Sønner.
\par 22 Sems Sønner: Elam, Assur, Arpaksjad, Lud og Aram.
\par 23 Arams Sønner: Uz, Hul, Geter og Masj.
\par 24 Arpaksjad avlede Sjela; Sjela avlede Eber;
\par 25 Eber fødtes der to Sønner; den ene hed Peleg, thi på hans Tid adsplittedes Jordens Befolkning, og hans Broder hed Joktan.
\par 26 Joktan avlede Almodad, Sjelef, Hazarmavet, Jera,
\par 27 Hadoram, Uzal, Dikla,
\par 28 Obal, Abimael, Saba,
\par 29 0fir, Havila og Jobab. Alle disse var Joktans Sønner,
\par 30 og deres Bosteder strækker sig fra Mesja i Retning af Sefar, Østens Bjerge.
\par 31 Det var Sems Sønner efter deres Slægter og Tungemål i deres Lande og Folk.
\par 32 Det var Noas Sønners Slægter efter deres Nedstamning, i deres Folk; fra dem nedstammer Folkene, som efter Vandfloden bredte sig på Jorden.

\chapter{11}

\par 1 Hele Menneskeheden havde eet Tungemål og samme Sprog.
\par 2 Da de nu drog østerpå, traf de på en Dal i Sinear, og der slog de sig ned.
\par 3 Da sagde de til hverandre: "Kom, lad os stryge Teglsten og brænde dem godt!" De brugte nemlig Tegl som Sten og Jordbeg som Kalk.
\par 4 Derpå sagde de: "Kom, lad os bygge os en By og et Tårn, hvis Top når til Himmelen, og skabe os et Navn, for at vi ikke skal spredes ud over hele Jorden!"
\par 5 Men HERREN steg ned for at se Byen og Tårnet, som Menneskebørnene byggede,
\par 6 og han sagde: "Se, de er eet Folk og har alle eet Tungemål; og når de nu først er begyndt således, er intet, som de sætter sig for, umuligt for dem;
\par 7 lad os derfor stige ned og forvirre deres Tungemål der, så de ikke forstår hverandres Tungemål!"
\par 8 Da spredte HERREN dem fra det Sted ud over hele Jorden, og de opgav at bygge Byen.
\par 9 Derfor kaldte man den Babel, thi der forvirrede HERREN al Jordens Tungemål, og derfra spredte HERREN dem ud over hele Jorden.
\par 10 Dette er Sems Slægtebog. Da Sem var 100 År gammel, avlede han Arpaksjad, to År efter Vandfloden;
\par 11 og efter at Sem havde avlet Arpaksjad, levede han 500 År og avlede Sønner og Døtre.
\par 12 Da Atpaksjad havde levet 35 År, avlede han Sjela;
\par 13 og efter at Arpaksjad havde avlet Sjela, levede han 403 År og avlede Sønner og Døtre.
\par 14 Da Sjela havde levet 30 År, avlede han Eber;
\par 15 og efter at Sjela havde avlet Eber, levede han 403 År og avlede Sønner og Døtre.
\par 16 Da Eber havde levet 34 År, avlede han Peleg;
\par 17 og efter at Eber havde avlet Peleg, levede han 430 År og avlede Sønner og Døtre.
\par 18 Da Peleg havde levet 30 År, avlede han Re'u;
\par 19 og efter at Peleg havde avlet Re'u, levede han 209 År og avlede Sønner og Døtre.
\par 20 Da Re'u havde levet 32 År, avlede han Serug;
\par 21 og efter at Re'u havde avlet Serug, levede han 207 År og avlede Sønner og Døtre.
\par 22 Da Serug havde levet 30 År, avlede han Nakor;
\par 23 og efter at Serug havde avlet Nakor, levede han 200 År og avlede Sønner og Døtre.
\par 24 Da Nakor havde levet 29 År, avlede han Tara;
\par 25 og efter at Nakor havde avlet Tara, levede han 119 År og avlede Sønner og Døtre.
\par 26 Da Tara havde levet 70 År, avlede han Abram, Nakor og Haran.
\par 27 Dette er Taras Slægtebog. Tara avlede Abram, Nako og Haran.
\par 28 Haran døde i sin Fader Taras Levetid i sin Hjemstavn i Ur Kasdim.
\par 29 Abram og Nakor tog sig Hustruer; Abrams Hustru hed Saraj, Nakors Milka, en Datter af Haran, Milkas og Jiskas Fader.
\par 30 Men Saraj var ufrugtbar og havde ingen Børn.
\par 31 Tara tog sin Søn Abram, sin Sønnesøn Lot, Harans Søn, og sin Sønnekone Saraj, hans Søn Abrams Hustru, og førte dem fra Ur Kasdim for at begive sig til Kana'ans Land; men da de kom til Karan, slog de sig ned der.
\par 32 Taras Levetid var 205 År; og Tara døde i Karan.

\chapter{12}

\par 1 HERREN sagde til Abram: "Drag ud fra dit Land, fra din Slægt og din Faders Hus til det Land, jeg vil vise dig;
\par 2 så vil jeg gøre dig til et stort Folk, og jeg vil velsigne dig og gøre dit Navn stort. og vær en Velsignelse!
\par 3 Jeg vil velsigne dem, der velsigner dig, og forbande dem, der forbander dig; i dig skal alle Jordens Slægter velsignes!"
\par 4 Og Abram gik,som HERREN sagde til ham, og Lot gik med ham.
\par 5 og Abram tog sin Hustru Saraj og sin Brodersøn Lot og al den Ejendom, de havde samlet sig, og de Folk, de havde vundet sig i Karan, og de gav sig på Vej til Kana'ans Land og nåede derhen.
\par 6 Derpå drog Abram gennem Landet til Sikems hellige Sted, til Sandsigerens Træ. Det var dengang Kana'anæerne boede i Landet.
\par 7 Men HERREN åbenbarede sig for Abram og sagde til ham: "Dit Afkom giver jeg dette Land!" Da byggede han der et Alter for HERREN. som havde åbenbaret sig for ham.
\par 8 Derpå brød han op derfra og drog til Bjergene østen for Betel, og han slog Lejr med Betel mod Vest og Aj mod Øst; og han byggede HERREN et Alter der og påkaldte HERRENs Navn.
\par 9 Derpå drog Abram fra Plads til Plads og nåede Sydlandet.
\par 10 Der opstod Hungersnød i Landet; og da Hungersnøden i Landet blev trykkende, drog Abram ned til Ægypten for at bo der som fremmed.
\par 11 Da han nu nærmede sig Ægypten, sagde han til sin Hustru Saraj: Jeg ved jo, at du er en smuk Kvinde;
\par 12 når nu Ægypterne ser dig, og de mener, at du er min Hustru, slår de mig ihjel og lader dig leve;
\par 13 sig derfor, at du er min Søster, for at det må gå mig godt, og jeg ikke skal miste Livet for din Skyld!"
\par 14 Da han så drog ind i Ægypten, så Ægypterne, at hun var en såre smuk Kvinde;
\par 15 og Faraos Stormænd, der så hende, priste hende for Farao, og så blev Kvinden ført til Faraos Hus.
\par 16 Men Abram behandlede han godt for hendes Skyld, og han fik Småkvæg, Hornkvæg og Æsler, Trælle og Trælkvinder, Aseninder og Kameler.
\par 17 Men HERREN ramte Farao og hans Hus med svære Plager for Abrams Hustru Sarajs Skyld.
\par 18 Da lod Farao Abram kalde og sagde: "Hvad har du gjort imod mig! Hvorfor lod du mig ikke vide, at hun er din Hustru?
\par 19 Hvorfor sagde du, at hun var din Søster, så at jeg tog hende til Hustru? Se, her har du nu din Hustru, tag hende og gå bort!"
\par 20 Og Farao bød sine Mænd følge ham og hans Hustru og al deres Ejendom på Vej;

\chapter{13}

\par 1 og Abram drog atter med sin Hustru og al sin Ejendom fra Ægypten op til Sydlandet, og Lot drog med ham.
\par 2 Abram var meget rig på kvæghjorde, Sølv og Guld;
\par 3 og han vandrede fra Lejrplads til Lejrplads og nåede fra Sydlandet til Betel, til det Sted, hvor hans Teltlejr havde stået første Gang, mellem Betel og Aj,
\par 4 til det Sted, hvor han forrige Gang havde rejst et Alter; og Abram påkaldte der HERRENs Navn.
\par 5 Og Lot, der drog med Abram, ejede ligeledes Småkvæg, Hornkvæg og Telte.
\par 6 Men Landet formåede ikke at rumme dem, så de kunde bo sammen; thi deres Hjorde var for store til, at de kunde bo sammen.
\par 7 Da opstod der Strid mellem Abrams og Lots Hyrder; det var dengang Kana'anæerne og Perizziterne boede i Landet.
\par 8 Abram sagde derfor til Lot: "Der må ikke være Strid mellem os to eller mellem mine og dine Hyrder, vi er jo Frænder!
\par 9 Ligger ikke hele Landet dig åbent? Skil dig hellere fra mig; vil du til venstre, så går jeg til højre, og vil du til højre, så går jeg til venstre!"
\par 10 Da så Lot sig omkring, og da han så, at hele Jordanegnen (det var før HERREN ødelagde Sodoma og Gomorra) var vandrig som HERRENs Have, som Ægyptens Land, hen ad Zoar til,
\par 11 valgte han sig hele Jordanegnen. Så brød Lot op og drog østerpå, og de skiltes,
\par 12 idet Abram slog sig ned i Kana'ans Land, medens Lot slog sig ned i Jordanegnens Byer og drog med sine Telte fra Sted til Sted helt hen til Sodoma.
\par 13 Men Mændene i Sodoma var ugudelige og store Syndere mod HERREN.
\par 14 Efter at Lot havde skilt sig fra Abram, sagde HERREN til denne: "Løft dit Blik og se dig om der, hvor du står, mod Nord, mod Syd, mod Øst og mod Vest;
\par 15 thi hele det Land, du ser, vil jeg give dig og dit Afkom til evig Tid,
\par 16 og jeg vil gøre dit Afkom som Jordens Støv, så at det lige så lidt skal kunne tælles, som nogen kan tælle Jordens Støv.
\par 17 Drag nu gennem Landet på Kryds og tværs, thi dig giver jeg det!"
\par 18 Så drog Abram fra Sted til Sted med sine Telte og kom til Mamres Lund i Hebron, hvor han slog sig ned og byggede HERREN et Alter.

\chapter{14}

\par 1 Dengang Amrafel var Konge i Sinear, Arjok i Ellasar, Kedorlaomer i Elam og Tidal i Gojim.
\par 2 lå de i Krig med Kong Bera af Sodoma, Kong Birsja af Gomorra, Kong Sjin'ab af Adma, Kong Sjem'eher af Zebojim og Kongen i Bela, det et Zoar.
\par 3 Alle disse havde slået sig sammen og var rykket frem til Siddims Dal, det er Salthavet.
\par 4 I tolv År havde de stået under Kedorlaomer, men i det trettende faldt de fra;
\par 5 og i det fjortende År kom Kedorlaomer og de Konger, som fulgte ham. Først slog de Refaiterne i Asjtarot Karnajim, Zuziterne i Ham, Emiterne i Sjave Kirjatajim
\par 6 og Horiterne i Seirs Bjerge hen ad El-Paran til ved Ørkenens Rand;
\par 7 så vendte de om og drog til Misjpatkilden, det er Hadesj, og slog Amalekiterne i hele deres Område og ligeså de Amoriter, der boede i Hazazon Tamar.
\par 8 Da drog Sodomas, Gomorras, Admas, Zebojims og Belas, det er Zoats, Konger ud og indlod sig i Siddims Dal i Kamp
\par 9 med Kong Kedorlaomer af Elam, Kong Tid'al af Gojim, Kong Amrafel af Sinear og Kong Arjok af Ellasar, fire Konger mod fem.
\par 10 Men Siddims Dal var fuld af Jordbeggruber; og da Sodomas og Gomorras Konger blev slået på Flugt, styrtede de i dem, medens de, der blev tilbage, flyede op i Bjergene.
\par 11 Så tog Fjenden alt Godset i Sodoma og Gomorra og alle Levnedsmidlerne og drog bort;
\par 12 ligeledes tog de, da de drog bort, Abrams Brodersøn Lot og alt hans Gods med sig; thi han boede i Sodoma.
\par 13 Men en Flygtning kom og meldte det til Hebræeren Abram, der boede ved den Lund, som tilhørte Amoriten Mamre, en Broder til Esjkol og Aner, der ligesom han var Abrams Pagtsfæller.
\par 14 Da nu Abram hørte, at hans Frænde var taget til Fange, mønstrede han sine Husfolk, de hjemmefødte Trælle, 3l8 Mand, og satte efter Fjenden til Dan;
\par 15 der faldt han og hans Trælle over dem om Natten, slog dem på Flugt og forfulgte dem op til Hoba norden for Damaskus.
\par 16 Derefter bragte han alt Godset tilbage; også sin Frænde Lot og hans Gods førte han tilbage og ligeledes Kvinderne og Folket.
\par 17 Da han nu kom tilbage fra Sejren over Kedorlaomer og de Konger, der fulgte ham, gik Sodomas Konge ham i Møde i Sjavedalen, det er Kongedalen.
\par 18 Men Salems Konge Melkizedek, Gud den Allerhøjestes Præst, bragte Brød og Vin
\par 19 og velsignede ham med de Ord: "Priset være Abram for Gud den Allerhøjeste, Himmelens og Jordens Skaber,
\par 20 og priset være Gud den Allerhøjeste, der gav dine Fjender i din Hånd!" Og Abram gav ham Tiende af alt.
\par 21 Sodomas Konge sagde derpå til Abram: "Giv mig Menneskene og behold selv Godset!"
\par 22 Men Abram svarede Sodomas Konge: "Til HERREN, Gud den Allerhøjeste, Himmelens og Jordens Skaber, løfter jeg min Hånd på,
\par 23 at jeg ikke vil tage så meget som en Tråd eller en Sandalrem eller overhovedet noget som helst af din Ejendom; du skal ikke sige, at du har gjort Abram rig!
\par 24 Jeg vil intet have, kun hvad de unge Mænd har fortæret, og mine Ledsagere, Aner, Esjkol og Mamres Del, lad dem få deres Del!"

\chapter{15}

\par 1 Nogen Tid efter kom HERRENS Ord til Abram i et Syn således: "Frygt ikke, Abram, jeg er dit Skjold; din Løn skal blive såre stor!"
\par 2 Men Abram svarede: "Herre", HERRE, hvad kan du give mig, når jeg dog går barnløs bort, og en Mand fra Damaskus, Eliezer, skal arve mit Hus."
\par 3 Og Abram sagde: "Du har jo intet Afkom givet mig, og se, min Hustræl kommer til at arve mig!"
\par 4 Og se, HERRENs Ord kom til ham således: "Han kommer ikke til at arve dig, men den, der udgår af dit Liv, han skal arve dig."
\par 5 Derpå førte han ham ud i det fri og sagde: "Se op mod Himmelen og prøv, om du kan tælle Stjernerne!" Og han sagde til ham: "Således skal dit Afkom blive!"
\par 6 Da troede Abram HERREN, og han regnede ham det til Retfærdighed.
\par 7 Derpå sagde han til ham: "Jeg er HERREN, som førte dig bort fra Ur Kasdim for at give dig dette Land i Eje!"
\par 8 Men han svarede: "Herre, HERRE, hvorpå kan jeg kende, at jeg skal få det i Eje?"
\par 9 Da sagde han til ham: "Tag mig en treårs Kvie, en treårs Ged og en treårs Væder, en Turteldue og en Småfugl!"
\par 10 Så tog han alle disse Dyr skar dem midt over og lagde Halvdelene over for hinanden; dog skar han ikke Fuglene over.
\par 11 Da slog der Rovfugle ned på de døde Kroppe, men Abram skræmmede dem bort.
\par 12 Da Solen så var ved at gå ned, faldt der Dvale over Abram, og se, Rædsel faldt over ham, et stort Mørke.
\par 13 Og han sagde til Abram: "Vide skal du, at dit Afkom skal bo som fremmede i et Land, der ikke er deres eget; de skal trælle for dem og mishandles af dem i 400 År.
\par 14 Dog vil jeg også dømme det Folk, de kommer til at trælle for, og siden skal de vandre ud med meget Gods.
\par 15 Men du skal fare til dine Fædre i Fred og blive jordet i en god Alderdom.
\par 16 I fjerde Slægtled skal de vende tilbage hertil; thi endnu er Amoriternes Syndeskyld ikke fuldmoden."
\par 17 Da Solen var gået ned og Mørket faldet på, viste der sig en rygende Ovn med en flammende Ildslue, der skred frem mellem de sønderskårne Kroppe.
\par 18 På den Dag sluttede HERREN Pagt med Abram, idet han sagde: "Dit Afkom giver jeg dette Land fra Ægyptens Bæk til den store Flod, Eufratfloden,
\par 19 det er Keniterne, Henizziterne, Kadmoniterne,
\par 20 Hetiterne, Perizziterne, Refaiterne,
\par 21 Amoriterne, Kana'anæerne, Girgasjiterne, Hivviterne og Jebusiterne."

\chapter{16}

\par 1 Abrams Hustru Saraj fødte ham intet Barn. Men Saraj havde en Ægyptisk Trælkvinde ved Navn Hagar;
\par 2 og Saraj sagde til Abram: "HERREN har jo nægtet mig Børn; gå derfor ind til min Trælkvinde, måske kan jeg få en Søn ved hende!" Og Abram adlød Saraj.
\par 3 Så tog Abrams Hustru Saraj sin Trælkvinde, Ægypterinden Hagar, efter at Abram havde boet ti År i Kana'ans Land, og gav sin Mand Abram hende til Hustru;
\par 4 og han gik ind til Hagar, og hun blev frugtsommelig. Men da hun så, at hun var frugtsommelig, lod hun hånt om sin Herskerinde.
\par 5 Da sagde Saraj til Abram: "Min Krænkelse komme over dig! Jeg lagde selv min Trælkvinde i din Favn, og nu hun ser, at hun skal føde, lader hun hånt om mig; HERREN være Dommer mellem mig og dig!"
\par 6 Abram svarede Saraj: "Din Trældkvinde er i din Hånd, gør med hende, hvad du finder for godt!" Da plagede Saraj hende, så hun flygtede for hende.
\par 7 Men HERRENs Engel fandt hende ved Vandkilden i Ørkenen, ved Kilden på Vejen til Sjur;
\par 8 og han sagde: "Hvorfra kommer du, Hagar, Sarajs Trælkvinde, og hvor går du hen?" Hun svarede: "Jeg flygter for min Herskerinde Saraj!"
\par 9 Da sagde HERRENs Engel til hende: "Vend tilbage til din Herskerinde og find dig i hendes Mishandling!"
\par 10 Og HERRENs Engel sagde til hende: "Jeg vil gøre dit Afkom så talrigt, at det ikke kan tælles."
\par 11 Og HERRENs Engel sagde til hende: "Se, du er frugtsommelig, og du skal føde en Søn, som du skal kalde Ismael, thi HERREN har hørt, hvad du har lidt;
\par 12 og han skal blive et Menneske Vildæsel, hvis Hånd er mod alle, og alles Hånd mod ham, og han skal ligge i Strid med alle sine Frænder!"
\par 13 Så gav hun HERREN, der havde talet til hende, Navnet: Du er en Gud, som ser; thi hun sagde: "Har jeg virkelig her set et Glimt af ham, som ser mig?"
\par 14 Derfor kaldte man Brønden Be'erlahajro'i; den ligger mellem Kadesj og Bered.
\par 15 Og Hagar fødte Abram en Søn, og Abram kaldte Sønnen, Hagar fødte ham, Ismael.
\par 16 Abram var seks og firsindstyve År gammel, da Hagar fødte ham Ismael.

\chapter{17}

\par 1 Da Abram var ni og halvfemsindstyve År gammel, åbenbarede HERREN sig for ham og sagde til ham: "Jeg er Gud den Almægtige; vandre for mit Åsyn og vær ustraffelig,
\par 2 så vil jeg oprette min Pagt mellem mig og dig og give dig et overvættes stort Afkom!"
\par 3 Da faldt Abram på sit Ansigt, og Gud sagde til ham:
\par 4 "Fra min Side er min Pagt med dig, at du skal blive Fader til en Mængde Folk;
\par 5 derfor skal dit Navn ikke mere være Abram, men du skal hedde Abraham, thi jeg gør dig til Fader til en Mængde Folk.
\par 6 Jeg vil gøre dig overvættes frugtbar og lade dig blive til Folk, og Konger skal nedstamme fra dig.
\par 7 Jeg opretter min Pagt mellem mig og dig og dit Afkom efter dig fra Slægt til Slægt, og det skal være en evig Pagt, at jeg vil være din Gud og efter dig dit Afkoms Gud;
\par 8 og jeg giver dig og dit Afkom efter dig din Udlændigheds Land, hele Kana'ans Land, til evigt Eje, og jeg vil være deres Gud!"
\par 9 Derpå sagde Gud til Abraham: "Men du på din Side skal holde min Pagt, du og dit Afkom efter dig fra Slægt til Slægt;
\par 10 og dette er min Pagt, som I skal holde, Pagten mellem mig og eder, at alt af Mandkøn hos eder skal omskæres.
\par 11 I skal omskæres på eders Forhud, det skal være et Pagtstegn mellem mig og eder;
\par 12 otte Dage gamle skal alle af Mandkøn omskæres hos eder i alle kommende Slægter, både de hjemmefødte Trælle og de, som er købt, alle fremmede, som ikke hører til dit Afkom;
\par 13 omskæres skal både dine hjemmefødte og dine købte. Min Pagt på eders Legeme skal være en evig Pagt!
\par 14 Men de uomskårne, det af Mandkøn, der ikke Ottendedagen omskæres på Forhuden, de skal udryddes af deres Folk; de har brudt min Pagt!"
\par 15 Endvidere sagde Gud til Abraham: "Din Hustru Saraj skal du ikke mere kalde Saraj, hendes Navn skal være Sara;
\par 16 jeg vil velsigne hende og give dig en Søn også ved hende; jeg vil velsigne hende, og hun skal blive til Folk, og Folkeslags Konger skal nedstamme fra hende!"
\par 17 Da faldt Abraham på sit Ansigt og lo, idet han tænkte: "Kan en hundredårig få Børn, og kan Sara med sine halvfemsindstyve År føde en Søn?"
\par 18 Abraham sagde derfor til Gud: "Måtte dog Ismael leve for dit Åsyn!"
\par 19 Men Gud sagde: "Nej, din Ægtehustru Sara skal føde dig en Søn, som du skal kalde Isak; med ham vil jeg oprette min Pagt, og det skal være en evig Pagt, der skal gælde hans Afkom efter ham!
\par 20 Men hvad Ismael angår, har jeg bønhørt dig: jeg vil velsigne ham og gøre ham frugtbar og give ham et overvættes talrigt Afkom; tolv Stammehøvdinger skal han avle, og jeg vil gøre ham til et stort Folk.
\par 21 Men min Pagt opretter jeg med Isak, som Sara skal føde dig om et År ved denne Tid."
\par 22 Så hørte han op at tale med ham; og Gud steg op fra Abraham.
\par 23 Da tog Abraham sin Søn Ismael og alle sine hjemmefødte og de købte, alt af Mandkøn i Abrahams Hus, og omskar selvsamme Dag deres Forhud, således som Gud havde pålagt ham.
\par 24 Abraham var ni og halvfemsindstyve År, da han blev omskåret på sin Forhud;
\par 25 og hans Søn Ismael var tretten År, da han blev omskåret på sin Forhud.
\par 26 Selvsamme Dag blev Abraham og hans Søn Ismael omskåret;
\par 27 og alle Mænd i hans Hus, både de hjemmefødte og de, der var købt, de fremmede, blev omskåret tillige med ham.

\chapter{18}

\par 1 Siden åbenbarede HERREN sig for ham ved Mamres Lund, engang han sad i Teltdøren på den hedeste Tid af Dagen.
\par 2 Da han så op, fik han Øje på tre Mænd, der stod foran ham. Så snart han fik Øje på dem, løb han dem i Møde fra Teltdøren, bøjede sig til Jorden
\par 3 og sagde: "Herre, hvis jeg har fundet Nåde for dine Øjne, så gå ikke din Træl forbi!
\par 4 Lad der blive hentet lidt Vand, så I kan tvætte eders Fødder og hvile ud under Træet.
\par 5 Så vil jeg bringe et Stykke Brød, for at I kan styrke eder; siden kan I drage videre - da eders Vej nu engang har ført eder forbi eders Træl!" De svarede: "Gør, som du siger!"
\par 6 Da skyndte Abraham sig ind i Teltet til Sara og sagde: "Tag hurtigt tre Mål fint Mel, ælt det og bag Kager deraf!"
\par 7 Så ilede han ud til Kvæget, tog en fin og lækker Kalv og gav den til Svenden, og han tilberedte den i Hast.
\par 8 Derpå tog han Surmælk og Sødmælk og den tilberedte Kalv, satte det for dem og gik dem til Hånde under Træet, og de spiste.
\par 9 Da sagde de til ham: "Hvor er din Hustru Sara?" Han svarede: "Inde i Teltet!"
\par 10 Så sagde han: "Næste År ved denne Tid kommer jeg til dig igen, og så har din Hustru Sara en Søn!" Men Sara lyttede i Teltdøren bag ved dem;
\par 11 og da Abraham og Sara var gamle og højt oppe i Årene, og det ikke mere gik Sara på Kvinders Vis,
\par 12 lo hun ved sig selv og tænkte: "Skulde jeg virkelig føle Attrå. nu jeg er affældig, og min Herre er gammel?"
\par 13 Da sagde HERREN til Abraham: "Hvorfor ler Sara og tænker: Skulde jeg virkelig føde en Søn. nu jeg er gammel?
\par 14 Skulde noget være umuligt for Herren? Næste År ved denne Tid kommer jeg til dig igen, og så har Sara en Søn!"
\par 15 Men Sara nægtede og sagde: "Jeg lo ikke!" Thi hun frygtede.
\par 16 Så brød Mændene op derfra hen ad Sodoma til, og Abraham gik med for at følge dem på Vej.
\par 17 Men HERREN sagde ved sig selv: "Skulde jeg vel dølge for Abraham, hvad jeg har i Sinde at gøre.
\par 18 da Abraham dog skal blive til et stort og mægtigt Folk, og alle Jordens Folk skal velsignes i ham?
\par 19 Jeg har jo udvalgt ham, for at han skal pålægge sine Børn og sine Efterkommere at vogte på HERRENs Vej ved at øve Retfærdighed og Ret, for at HERREN kan give Abraham alt, hvad han har forjættet ham."
\par 20 Da sagde HERREN: "Sandelig. Skriget over Sodoma og Gomorra er stort, og deres Synd er såre svar.
\par 21 Derfor vil jeg stige ned og se. om de virkelig har handlet så galt. som det lyder til efter Skriget over dem, der har nået mig - derom vil jeg have Vished!"
\par 22 Da vendte Mændene sig bort derfra og drog ad Sodoma til; men HERREN blev stående foran Abraham.
\par 23 Og Abraham trådte nærmere og sagde: "Vil du virkelig udrydde retfærdige sammen med gudløse?
\par 24 Måske findes der halvtredsindstyve retfærdige i Byen; vil du da virkelig udrydde dem og ikke tilgive Stedet for de halvtredsindstyve retfærdiges Skyld, som findes derinde.
\par 25 Det være langt fra dig at handle således: at ihjelslå retfærdige sammen med gudløse, så de retfærdige får samme Skæbne som de gudløse - det være langt.fra dig! Skulde den, der dømmer hele Jorden, ikke selv øve Ret?"
\par 26 Da sagde HERREN: "Dersom jeg finder halvtredsindstyve retfærdige i Sodoma, i selve Byen, vil jeg for deres Skyld tilgive hele Stedet!"
\par 27 Men Abraham tog igen til Orde: "Se, jeg har dristet mig til at tale til min Herre, skønt jeg kun er Støv og Aske!
\par 28 Måske mangler der fem i de halvtredsindstyve retfærdige - vil du da ødelægge hele Byen for fems Skyld?" Han svarede: "Jeg vil ikke ødelægge Byen, hvis jeg finder fem og fyrretyve i den."
\par 29 Men han blev ved at tale til ham: "Måske findes der fyrretyve i den!" Han. svarede: "For de fyrretyves Skyld vil jeg lade det være."
\par 30 Men han sagde: "Min Herre må ikke blive vred, men lad mig tale: Måske findes der tredive i den!" Han svarede: "Jeg skal ikke gøre det, hvis jeg finder tredive i den."
\par 31 Men han sagde: "Se, jeg har dristet mig til at tale til min Herr: Måske findes de tyve i den!" Han svarede: "For de tyves Skyld vil jeg lade være at ødelægge den."
\par 32 Men han sagde: "Min Herre må ikke blive vred, men lad mig kun tale denne ene Gang endnu; måske findes der ti i den!" Han svarede: "For de tis Skyld vil jeg lade være at ødelægge den."
\par 33 Da nu HERREN havde talt ud med Abraham, gik han bort; og Abraham vendte tilbage til sin Bolig.

\chapter{19}

\par 1 De to Engle kom nu til Sodoma ved Aftenstid. Lot sad i Sodomas Port, og da han fik Øje på dem, stod han op og gik dem i Møde, bøjede sig til Jorden
\par 2 og sagde: "Kære Herrer, tag dog ind og overnat i eders Træls Hus og tvæt eders Fødder; i Morgen tidlig kan I drage videre!" Men de sagde: "Nej, vi vil overnatte på Gaden."
\par 3 Da nødte han dem stærkt, og de tog ind i hans Hus; derpå tilberedte han dem et Måltid og bagte usyrede Kager, og de spiste.
\par 4 Men endnu før de havde lagt sig, stimlede Byens Folk, Indbyggerne i Sodoma, sammen omkring Huset, både gamle og unge, alle uden Undtagelse,
\par 5 og de råbte til Lot: "Hvor er de Mænd, der kom til dig i Nat Kom herud med dem, for at vi kan stille vor Lyst på dem!"
\par 6 Da gik Lot ud til dem i Porten, men Døren lukkede han efter sig.
\par 7 Og han sagde: "Gør dog ikke noget ondt, mine Brødre!
\par 8 Se, jeg har to Døtre, der ikke har kendt Mand; dem vil jeg bringe ud til eder, og med dem kan I gøre, hvad I lyster; men disse Mænd må I ikke gøre noget, siden de nu engang er kommet ind under mit Tags Skygge!"
\par 9 Men de sagde: "Bort med dig! Her er den ene Mand kommet og bor som fremmed, og nu vil han spille Dommer! Kom, lad os handle værre med ham end med dem!" Og de trængte ind på Manden, på Lot, og nærmede sig Døren for at sprænge den.
\par 10 Da rakte Mændene Hånden ud og trak Lot ind til sig og lukkede Døren;
\par 11 men Mændene uden for Porten til Huset slog de med Blindhed, både store og små, så de forgæves søgte at finde Porten.
\par 12 Derpå sagde Mændene til Lot: "Hvem der ellers hører dig til her, dine Svigersønner, dine Sønner og Døtre, alle, som hører dig til i Byen, må du føre bort fra dette Sted;
\par 13 thi vi står i Begreb med at ødelægge Stedet her, fordi Skriget over dem er blevet så stort for HERREN, at HERREN har sendt os for at ødelægge dem."
\par 14 Da gik Lot ud og sagde til sine Svigersønner, der skulde ægte hans Døtre: "Stå op, gå bort herfra, thi HERREN vil ødelægge Byen!" Men hans Svigersønner troede, at han drev Spøg med dem.
\par 15 Da Morgenen så gryede, skyndede Englene på Lot og sagde: "Tag din Hustru og dine to Døtre, som bor hos dig, og drag bort, for at du ikke skal rives bort ved Byens Syndeslkyld!"
\par 16 Og da han tøvede, greb Mændene ham,.hans Hustru og hans to Døtre ved Hånden, thi HERREN vilde skåne ham, og de førte ham bort og bragte ham i Sikkerhed uden for Byen.
\par 17 Og idet de førte dem uden for Byen, sagde de: "Det gælder dit Liv! Se dig ikke tilbage og stands ikke nogensteds i Jordanegnen, men red dig op i Bjergene, for at du ikke skal omkomme!"
\par 18 Men Lot sagde til dem: "Ak nej, Herre!
\par 19 Din Træl har jo fundet Nåde for dine Øjne, og du har vist mig stor Godhed og frelst mit Liv; men jeg kan ikke nå op i Bjergene og undfly Ulykken; den indhenter mig så jeg mister Livet.
\par 20 Se, den By der er nær nok til at jeg kan flygte derhen; den betyder jo ikke stort, lad mig redde mig derhen, den betyder jo ikke stort, og mit Liv er frelst!"
\par 21 Da svarede han: "Også i det Stykke har jeg bønhørt dig; jeg vil ikke ødelægge den By, du nævner;
\par 22 men red dig hurtigt derhen, thi jeg kan intet gøre, før du når derhen!" Derfor kaldte man Byen Zoar.
\par 23 Da Solen steg op over Landet og Lot var nået til Zoar,
\par 24 lod HERREN Svovl og Ild regne over Sodoma og Gomorra fra HERREN, fra Himmelen;
\par 25 og han ødelagde disse Byer og hele Jordanegnen og alle Byernes Indbyggere og Landets Afgrøde.
\par 26 Men hans Hustru, som gik efter ham, så sig tilbage og blev til en Saltstøtte.
\par 27 Næste Morgen, da Abraham gik hen til det Sted, hvor han havde stået hos HERREN,
\par 28 og vendte sit Blik mod Sodoma og Gomorra og hele Jordanegnen.
\par 29 Da Gud tilintetgjorde Jordanegnens Byer, kom han Abraham i Hu og førte Lot ud af Ødelæggelsen, som han lod komme over de Byer, Lot boede i.
\par 30 Men Lot drog op fra Zoar og slog sig ned i Bjergene med sine Døtre, thi han turde ikke blive i Zoar; og han boede i en Hule med sine to Døtre.
\par 31 Da sagde den ældste til den yngste: "Vor Fader er gammel, og der findes ingen Mænd her i Landet, som kunde komme til os på vanlig Vis.
\par 32 Kom, lad os give vor Fader Vin at drikke og ligge hos ham for at få Afkom ved vor Fader!"
\par 33 De gav ham da Vin at drikke samme Nat; og den ældste lagde sig hos sin Fader, og han sansede hverken, at hun lagde sig, eller at hun stod op.
\par 34 Næste Dag sagde den ældste til den yngste: "Jeg lå i Går Nat hos min Fader; nu vil vi også give ham Vin at drikke i Nat, og gå du så ind og læg dig hos ham, for at vi kan få Afkom ved vor Fader!"
\par 35 Så gav de atter den Nat deres Fader Vin at drikke, og den yngste lagde sig hos ham, og han sansede hverken, at hun lagde sig, eller at hun stod op.
\par 36 Således blev begge Lots Døtre frugtsommelige ved deres Fader;
\par 37 og den ældste fødte en Søn, som hun kaldte Moab; han er Moabs Stamfader den Dag i Dag.
\par 38 Ligeså fødte den yngste en Søn, som hun kaldte Ben Ammi; han er Ammoniternes Stamfader den Dag i Dag.

\chapter{20}

\par 1 Der På brød Abraham op derfra til Sydlandet og slog sig ned mellem Kadesj og Sjur og boede som fremmed i Gerar.
\par 2 Da nu Abraham sagde om sin Hustru Sara, at hun var hans Søster, sendte Kong Abimelek af Gerar Bud og lod Sara hente til sig.
\par 3 Men Gud kom til Abimelek i en Drøm om Natten og sagde til ham: "Se, du skal dø for den Kvindes Skyld, som du har taget, thi hun er en anden Mands Hustru!"
\par 4 Abimelek var imidlertid ikke kommet hende nær; og han sagde: "Herre, vil du virkelig.slå retfærdige Folk ihjel?
\par 5 Har han ikke sagt mig, at hun er hans Søster? Og hun selv har også sagt, at han er hendes Broder; i mit Hjertes Troskyldighed og med rene Hænder har jeg gjort dette."
\par 6 Da sagde Gud til ham i Drømmen: "Jeg ved, at du har gjort det i dit Hjertes Troskyldighed, og jeg har også hindret dig i at synde imod mig; derfor tilstedte jeg dig ikke at røre hende.
\par 7 Men send nu Mandens Hustru tilbage, thi han er en Profet, så han kan gå i Forbøn for dig, og du kan blive i Live; men sender du hende ikke tilbage, så vid, at du og alle dine er dødsens!"
\par 8 Tidligt næste Morgen lod Abimelek alle sine Tjenere kalde og fortalte dem det hele, og Mændene blev såre forfærdede.
\par 9 Men Abimelek lod Abraham kalde og sagde til ham: "Hvad har du dog gjort imod os? Og hvad har jeg forbrudt imod dig, at du bragte denne store Synd over mig og mit Rige? Du har gjort imod mig, hvad man ikke bør gøre!"
\par 10 Og Abimelek sagde til Abraham: "Hvad bragte dig til at handle således?"
\par 11 Abraham svarede: "Jo, jeg tænkte: Her er sikkert ingen Gudsfrygt på dette Sted, så de vil slå mig ihjel for min Hustrus Skyld.
\par 12 Desuden er hun virkelig min Søster, min Faders Datter, kun ikke min Moders; men hun blev min Hustru.
\par 13 Og da nu Gud lod mig flakke om fjernt fra min Faders Hus, sagde jeg til hende: Den Godhed må du vise mig, at du overalt, hvor vi kommer hen, siger, at jeg er din Broder."
\par 14 Derpå tog Abimelelk Småkvæg og Hornkvæg, Trælle og Trælkvinder og gav Abraham dem og sendte hans Hustru Sara tilbage til ham;
\par 15 og Abimelek sagde til ham: "Se, mit Land ligger åbent for dig; slå dig ned, hvor du lyster!"
\par 16 Men til Sara sagde han: "Jeg har givet din Broder 1000 Sekel Sølv, det skal være dig Godtgørelse for alt, hvad der er tilstødt dig. Hermed har du fået fuld Oprejsning."
\par 17 Men Abraham gik i Forbøn hos Gud, og Gud helbredte Abimelek og hans Hustru og Medhustruer, så at de atter fik Børn.
\par 18 HERREN havde nemlig lukket for ethvert Moderliv i Abimeleks Hus for Abrahams Hustru Saras Skyld.

\chapter{21}

\par 1 HERREN så til Sara, som han havde lovet, og HERREN gjorde ved Sara, som han havde sagt,
\par 2 og hun undfangede og fødte Abraham en Søn i hans Alderdom, til den Tid Gud havde sagt ham.
\par 3 Abraham kaldte den Søn, han fik med Sara, Isak;
\par 4 og Abraham omskar sin Søn Isak, da han var otte Dage gammel, således som Gud havde pålagt ham.
\par 5 Abraham var 100 År gammel, da hans Søn Isak fødtes ham.
\par 6 Da sagde Sara: "Gud ham beredt mig Latter; enhver, der hører det, vil le ad mig."
\par 7 Og hun sagde: "Hvem skulde have sagt Abraham, at Sara ammer Børn! Sandelig, jeg har født ham en Søn i hans Alderdom!"
\par 8 Drengen voksede til og blev vænnet fra, og Abraham gjorde et stort Gæstebud, den dag Isak blev vænnet fra.
\par 9 Men da Sara så Ægypterinden Hagars Søn, som hun havde født Abraham, lege med hendes Søn Isak,
\par 10 sagde hun til Abraham: "Jag den Trælkvinde og hendes Søn bort, thi ikke skal denne Trælkvindes Søn arve sammen med min Søn, med Isak!"
\par 11 Derover blev Abraham såre ilde til Mode for sin Søns Skyld;
\par 12 men Gud sagde til Abraham: "Vær ikke ilde til Mode over Drengen og din Trælkvinde, men adlyd Sara i alt, hvad hun siger dig, thi efter Isak skal dit Afkom nævnes;
\par 13 men også Trælkvindens Søn vil jeg gøre til et stort Folk; han er jo dit Afkom!"
\par 14 Tidligt næste Morgen tog da Abraham Brød og en Sæk Vand og gav Hagar det, og Drengen satte han på hendes Skulder, hvorpå han sendte hende bort. Som hun nu vandrede af Sted, for hun vild i Be'ersjebas Ørken,
\par 15 og Vandet slap op i hendes Sæk; da lagde hun Drengen hen under en af Buskene
\par 16 og gik hen og satte sig i omtrent et Pileskuds Afstand derfra, idet hun sagde ved sig selv: "Jeg kan ikke udholde at se Drengen dø!" Og således sad hun, medens Drengen græd højt.
\par 17 Da hørte Gud Drengens Gråd, og Guds Engel råbte til Hagar fra Himmelen og sagde til hende: "Hvad fattes dig, Hagar? Frygt ikke, thi Gud har hørt Drengens Røst der, hvor,han ligger;
\par 18 rejs dig, hjælp Drengen op og tag ham ved Hånden, thi jeg vil gøre ham til et stort Folk!"
\par 19 Da åbnede Gud hendes Øjne, så hun fik Øje på en Brønd med Vand; og hun gik hen og fyldte Sækken med Vand og gav Drengen at drikke.
\par 20 Og Gud var med Drengen, og han voksede til; og han bosatte sig i Ørkenen og blev Bueskytte.
\par 21 Han boede i Parans Ørken, og hans Moder tog ham en Hustru fra Ægypten.
\par 22 Ved den Tid sagde Abimelek og hans Hærfører Pikol til Abraham: "Gud er med dig i alt, hvad du tager dig for;
\par 23 tilsværg mig derfor her ved Gud, at du aldrig vil være troløs mod mig eller mine Efterkommere, men at du vil handle lige så venligt mod mig og det Land, du gæster, som jeg har handlet mod dig!"
\par 24 Da svarede Abraham: "Jeg vil sværge!"
\par 25 Men Abraham krævede Abimelek til Regnskab for en Brønd, som Abimeleks Folk havde tilranet sig.
\par 26 Da svarede Abimelek: "Jeg ved intet om, hvem der har gjort det; hverken har du underrettet mig derom, ej heller har jeg hørt det før i Dag!"
\par 27 Da tog Abraham Småkvæg og Hornkvæg og gav Abimelek det, og derpå sluttede de Pagt med hinanden.
\par 28 Men Abraham satte syv Lam til Side,
\par 29 og da Abimelek spurgte ham: "Hvad betyder de syv Lam, du der har sat til Side?"
\par 30 svarede han: "Jo, de syv Lam skal du modtage af min Hånd til Vidnesbyrd om, at jeg har gravet denne Brønd."
\par 31 Derfor kaldte man dette Sted Be'ersjeba, thi der svor de hinanden Eder;
\par 32 og de sluttede Pagt ved Be'ersjeba. Så brød Abimelek og hans Hærfører Pikol op og vendte tilbage til Filisternes Land.
\par 33 Men Abraham plantede en Tamarisk i Be'ersjeba og påkaldte der HERREN den evige Guds Navn.
\par 34 Og Abraham boede en Tid lang; som fremmed i Filisternes Land.

\chapter{22}

\par 1 Efter disse Begivenheder satte Gud Abraham på Prøve og sagde til ham: "Abraham!" Han svarede: "Se, her er jeg!"
\par 2 Da sagde han: "Tag din Søn Isak, din eneste, ham, du elsker, og drag hen til Morija Land.og bring ham der som Brændoffer på et af Bjergene, som jeg vil vise dig!"
\par 3 Da sadlede Abraham tidligt næste Morgen sit Æsel, tog to af sine Drenge og sin Søn Isak med sig, og efter at have kløvet Offerbrænde gav han sig på Vandring; til det Sted, Gud havde sagt ham.
\par 4 Da Abraham den tredje Dag så. op, fi1k han Øje på Stedet langt borte.
\par 5 Så sagde Abraham til sine Drenge: "Bliv her med Æselet, medens jeg og Drengen vandrer der. hen for at tilbede; så kommer vi tilbage til eder."
\par 6 Abraham tog da Brændet til Brændofferet og lagde,det på sin Søn Isak; selv tog han Ilden og Offerkniven, og så gik de to sammen.
\par 7 Da sagde Isak til sin Fader Abraham: "Fader!" Han svarede: "Ja. min Søn!" Da sagde han: "Her er Ilden og Brændet, men hvor er Dyret til Brændofferet?"
\par 8 Abraham svarede: "Gud vil selv udse sig Dyret til Brændofferet, min Søn!" Og så gik de to sammen.
\par 9 Da de nåede det Sted, Gud havde sagt ham, byggede Abraham der et Alter og lagde Brændet til Rette; så bandt han sin Søn Isak og lagde ham på Alteret oven på Brændet.
\par 10 Og Abraham greb Kniven og rakte Hånden ud for at slagte sin Søn.
\par 11 Da råbte HERRENs Engel til ham fra Himmelen: "Abraham, Abraham!" Han svarede: "Se, her er jeg!"
\par 12 Da sagde Engelen: "Ræk ikke din Hånd ud mod Drengen og gør ham ikke noget; thi nu ved jeg, at du frygter Gud og end ikke sparer din Søn, din eneste, for mig!"
\par 13 Og da Abraham nu så op, fik han bag ved sig Øje på en Væder, hvis Horn havde viklet sig ind i de tætte Grene; og Abraham gik hen og tog Væderen og ofrede den som Brændoffer i sin Søns Sted.
\par 14 Derfor kaldte Abraham dette Sted: HERREN udser sig, eller, som man nu til dags siger: Bjerget, hvor HERREN viser sig.
\par 15 Men HERRENs Engel råbte atter til Abraham fra Himmelen:
\par 16 "Jeg sværger ved mig selv, lyder det fra HERREN: Fordi du har gjort dette og ikke sparet din Søn, din eneste, for mig,
\par 17 så vil jeg velsigne dig og gøre dit Afkom talrigt som Himmelens Stjerner og Sandet ved Havets Bred; og dit Afkom skal tage sine Fjenders Porte i Besiddelse;
\par 18 og i din Sæd skal alle Jordens Folk velsignes, fordi du adlød mig!"
\par 19 Derpå vendte Abraham tilbage til sine Drenge, og de brød op og tog sammen til Be'ersjeba. Og Abraham blev i Be'ersjeba.
\par 20 Efter disse Begivenheder meldte man Abraham: "Også Milka har født din Broder Nakor Sønner:
\par 21 Uz, hans førstefødte, dennes Broder Buz, Kemuel, Arams Fader,
\par 22 Kesed, Hazo, Pildasj, Jidlaf og Betuel;
\par 23 Betuel avlede Rebekka; disse otte har Milka født Abrahams Broder Nakor,
\par 24 og desuden har hans Medhustru Re'uma født Teba, Gaham, Tahasj og Ma'aka."

\chapter{23}

\par 1 Sara levede l27 År, så mange var Saras Leveår.
\par 2 Sara døde i Kirjat Arba, det er Hebron, i Kana'ans Land. Så gik Abraham hen og holdt Klage over Sara og begræd hende.
\par 3 Og da han havde rejst sig fra sin døde, talte han således til Hetiterne:
\par 4 "Jeg er Gæst og fremmed hos eder; men giv mig et Gravsted hos eder, så jeg kan jorde min døde og bringe hende bort fra mit Ansigt!"
\par 5 Da svarede Hetiterne Abraham:
\par 6 "Hør os, Herre! En Guds Fyrste er du jo iblandt os; jord du din døde i en af vore bedste Grave! Ikke en af os vil nægte dig sin Grav og hindre dig i at jorde din døde."
\par 7 Men Abraham stod op og bøjede sig for Hetiterne, Folkene der på Stedet,
\par 8 og sagde til dem: "Hvis I samtykker i, at jeg jorder min døde og bringer hende bort fra mit Ansigt, så føj mig i at lægge et godt Ord ind for mig hos Efron, Zohars Søn,
\par 9 så han giver mig sin Klippehule i Makpela ved Udkanten af sin Mark; for fuld Betaling skal han i eders Nærværelse give mig den til Gravsted!"
\par 10 Men Efron sad blandt Hetiterne; og Hetiten Efron svarede Abraham i Hetiternes Påhør, så mange som gik ind gennem hans Bys Port:
\par 11 "Gid min Herre vilde høre mig! Marken giver jeg dig, og Hulen derpå giver jeg dig; i mit Folks Nærværelse giver jeg dig den; jord du kun din døde!"
\par 12 Da bøjede Abraham sig for Folkene der på Stedet
\par 13 og sagde til Efron i deres Påhør: "Om du blot - gid du dog vilde høre mig! Jeg giver dig, hvad Marken er værd; modtag det dog af mig, så jeg kan jorde min døde der."
\par 14 Da sagde Efron til Abraham:
\par 15 "Gid min Herre vilde høre mig! Et Stykke Land til 400 Sekel Sølv, hvad har det at sige mellem mig og dig? Jord du kun din døde!"
\par 16 Og Abraham forstod Efron og tilvejede ham den Sum, han havde nævnet i Hetiternes Påhør, 400 Sekel Sølv i gangbar Mønt.
\par 17 Således gik Efrons Mark i Makpela over for Mamre i hele sin Udstrækning tillige med Klippehulen og alle Træerne på Marken
\par 18 over i Abrahams Eje i Hetiternes Næværelse, så mange som gik ind gennem hans Bys Port.
\par 19 Derefter jordede Abraham sin Hustru Sara i Klippehulen på Makpelas Mark over for Mamre, det er Hebron, i Kana'ans Land.
\par 20 Og Marken med Klippehulen derpå gik fra Hetiterne over til Abraham som Gravsted.

\chapter{24}

\par 1 Abraham var blevet gammel og til Års, og HERREN havde velsignet ham i alle Måder.
\par 2 Da sagde Abraham til sin Træl, sit Hus's ældste, som stod for hele hans Ejendom: "Læg din Hånd under min Lænd,
\par 3 så jeg kan tage dig i Ed ved HERREN, Himmelens og Jordens Gud, at du ikke vil tage min Søn en Hustru af Kana'anæernes Døtre.
\par 4 men drage til mit Land og min Hjemstavn og tage min Søn Isak en Hustru derfra!"
\par 5 Da sagde Trællen: "Men hvis nu Pigen ikke vil følge mig her til Landet, skal jeg så bringe din Søn tilbage til det Land, du vandrede ud fra?"
\par 6 Abraham svarede: "Vogt dig vel for at bringe min Søn tilbage dertil!
\par 7 HERREN, Himmelens Gud, som tog mig bort fra min Faders Hus og min Hjemstavns Land, som talede til mig og tilsvor mig, at han vil give mit Afkom dette Land, han vil sende sin Engel foran dig, så du kan tage min Søn en Hustru derfra;
\par 8 men hvis Pigen ikke vil følge dig, så er du løst fra Eden til mig; men i intet Tilfælde må du bringe min Søn tilbage dertil!"
\par 9 Da lagde Trællen sin Hånd under sin Herre Abrahams Lænd og svor ham Eden.
\par 10 Derpå tog Trællen ti af sin Herres Kameler og alle Hånde kostbare Gaver fra sin Herre og gav sig på Vej til Nakors By i Aram Naharajim.
\par 11 Uden for Byen lod han Kamelerne knæle ved Brønden ved Aftenstid, ved den Tid Kvinderne går ud for at hente Vand;
\par 12 og han bad således: "HERRE. du min Herre Abrahams Gud, lad det lykkes for mig i dag og vis Miskundhed mod min Herre Abraham!
\par 13 Se, jeg stiller mig her ved Vandkilden, nu Bymændenes Døtre går ud for at hente Vand;
\par 14 og siger jeg nu til en Pige: Hæld din Krukke og giv mig at drikke! og siger så hun: Drik kun, og jeg vil også give dine Kameler at drikke! lad det da være hende, du har udset til din Tjener Isak; således vil jeg kunne kende, at du har vist Miskundhed mod min Herre!"
\par 15 Knap var han færdig med at bede, se, da kom Rebekka, en Datter af Betuel, der var en Søn af Abrahams Broder Nakors Hustru Milka, gående med Krukken på Skulderen,
\par 16 en såre smuk Kvinde, Jomfru, endnu ikke kendt af nogen Mand.
\par 17 Da ilede Trællen hen til hende og sagde: "Giv mig lidt Vand at drikke af din Krukke!"
\par 18 Hun svarede: "Drik, Herre!" og løftede straks Krukken ned på sin Hånd og lod ham drikke;
\par 19 og da hun havde slukket hans Tørst, sagde hun: "Jeg vil også øse Vand til dine Kameler, til de har slukket deres Tørst."
\par 20 Så skyndte hun sig hen og tømte Krukken i Truget og løb tilbage til Brønden for at øse, og således øste hun til alle hans Kameler.
\par 21 Imidlertid stod Manden og så tavs på hende for at få at vide, om HERREN havde ladet hans Rejse lykkes eller ej;
\par 22 og da Kamelernes Tørst var slukket, tog han en gylden Næsering, der vejede en halv Sekel, og to Armbånd, der vejede ti Guldsekel, og satte dem på hendes Arme;
\par 23 og han sagde til hende: "Sig mig, hvis Datter du er! Er der Plads til os i din Faders Hus for Natten?"
\par 24 Hun svarede: "Jeg er Datter af Betuel, som Milka fødte Nakor;"
\par 25 og videre sagde hun: "Der er rigeligt både af Strå og Foder hos os og Plads til at overnatte "
\par 26 Da bøjede Manden sig og tilbad HERREN,
\par 27 idet han sagde: "Lovet være HERREN, min Herre Abrahams Gud.
\par 28 Pigen løb imidlertid hjem og fortalte alt dette i sin Moders Hus.
\par 29 Men Rebekka havde en Broder ved Navn Laban; han løb ud til Manden ved Kilden;
\par 30 og da han så Næseringen og Armbåndene på sin Søsters Arme og hørte sin Søster Rebekka fortælle, hvad Manden havde sagt til hende, gik han ud til Manden, som stod med sine Kameler ved Kilden;
\par 31 og han sagde: "Kom, du HERRENs velsignede, hvorfor står du herude? Jeg har ryddet op i Huset og gjort Plads til Kamelerne."
\par 32 Så kom Manden hen til Huset og tog Seletøjet af Kamelerne, og Laban bragte Strå og Foder til dem og Vand til Fodtvæt for Manden og hans Ledsagere.
\par 33 Men da der blev sat Mad for ham, sagde han: "Jeg vil intet nyde, før jeg har røgtet mit Ærinde!" De svarede: "Sig frem!"
\par 34 Så sagde han: "Jeg er Abrahams Træl.
\par 35 HERREN har velsignet min Herre i rigt Mål, så han er blevet en velstående Mand, og givet ham Småkvæg og Hornkvæg, Sølv og Guld, Trælle og Trælkvinder, Kameler og Æsler
\par 36 og Sara, min Herres Hustru, har født ham en Søn i hans Alderdom, og ham har han givet alt, hvad han ejer.
\par 37 Og nu har min Herre taget mig i Ed og sagt: Du må ikke tage min Søn en Hustru blandt Kana'anæernes Døtre, i hvis Land jeg bor;
\par 38 Men du skal drage til min Faders Hus og min Slægt og tage min Søn en Hustru derfra!
\par 39 Og da jeg sagde til min Herre: Men hvis nu Pigen ikke vil følge med mig?
\par 40 svarede han: HERREN, for hvis Åsyn jeg har vandret, vil sende sin Engel med dig og lade din Rejse lykkes, så du kan tage min Søn en Hustru af min Slægt og min Faders Hus;
\par 41 i modsat Fald er du løst fra Eden til mig; kommer du til min Slægt, og de ikke vil give dig hende, er du løst fra Eden til mig!
\par 42 Da jeg nu i Dag kom til Kilden, bad jeg således: HERRE, du min Herre Abrahams Gud! Vilde du dog lade den Rejse lykkes, som jeg nu har for!
\par 43 Se, jeg stiller mig her ved Kilden, og siger jeg nu til den Pige, der kommer for at øse Vand: Giv mig lidt Vand at drikke af din Krukke!
\par 44 og svarer så hun: Drik selv, og jeg vil også øse Vand til dine Kameler! lad hende da være den Kvinde, HERREN har udset til min Herres Søn!
\par 45 Og knap var jeg færdig med at tale således ved mig selv, se, da kom Rebekka med sin Krukke på Skulderen og steg ned til Kilden og øste Vand, og da jeg sagde til hende: Giv mig noget at drikke!
\par 46 løftede hun straks sin Krukke ned og sagde: Drik kun, og jeg vil også give dine Kameler at drikke! Så drak jeg, og hun gav også Kamelerne at drikke.
\par 47 Da spurgte jeg hende: Hvis Datter er du? Og hun sagde: Jeg er Datter af Betuel, Nakors og Milkas Søn! Så satte jeg Ringen i hendes Næse og Armbåndene på hendes Arme;
\par 48 og jeg bøjede mig og tilbad HERREN, og jeg lovede HERREN, min Herre Abrahams Gud, som havde ført mig den rigtige Vej, så jeg: kunde tage min Herres Broderdatter til hans Søn!
\par 49 Hvis I nu vil vise min Herre Godhed og Troskab, så sig mig det, og hvis ikke, så sig mig det. for at jeg kan have noget at rette mig efter!"
\par 50 Da sagde Laban og Betuel:"Denne Sag kommer fra HERREN,. vi kan hverken gøre fra eller til!
\par 51 Der står Rebekka foran dig, tag hende og drag bort, at hun kan: blive din Herres Søns Hustru, således som HERREN har sagt!"
\par 52 Da Abrahams Træl hørte deres Ord, kastede han sig til Jorden for HERREN.
\par 53 Derpå fremtog Trællen Sølv og Guldsmykker og Klæder og gav Rebekka dem, og til hendes Broder og Moder uddelte han Gaver.
\par 54 Så spiste og drak han og hans Ledsagere og overnattede der.
\par 55 Men Rebekkas Broder og Moder svarede: "Lad dog Pigen blive hos os i nogen Tid, en halv Snes Dage eller så, siden kan du drage bort"
\par 56 Da sagde han til dem: "Ophold mig ikke, nu HERREN har ladet min Rejse lykkes; lad mig fare! Jeg vil drage til min Herre!"
\par 57 De svarede; "Lad os kalde på. Pigen og spørge hende selv!"
\par 58 Og de kaldte på Rebekka og; spurgte hende: "Vil du drage med denne Mand?" Hun svarede: "Ja. jeg vil!"
\par 59 Da tog de Afsked med deres, Søster Rebekka og hendes Amme og med Abrahams Træl og hans Ledsagere;
\par 60 og de velsignede Rebekka og sagde: "Måtte du, vor Søster, blive til ti Tusind Tusinder, og måtte dit Afkom indtage dine Fjenders Porte!"
\par 61 Så brød Rebekka og hendes Piger op, og de satte sig på Kamelerne og fulgte med Manden; og Trællen tog Rebekka og drog bort.
\par 62 Isak var imidlertid draget til Ørkenen ved Be'erlahajro'i, og han boede i Sydlandet.
\par 63 Da han engang ved Aftenstid var gået ud på Marken for at bede, så han op og fik Øje på nogle Kameler, der nærmede sig.
\par 64 Men da Rebekka så op og fik Øje på Isak, lod hun sig glide ned af Kamelen
\par 65 og spurgte Trællen: "Hvem er den Mand der, som kommer os i Møde på Marken?" Trællen svarede: "Det er min Herre!" Da tog hun sit Slør og tilhyllede sig.
\par 66 Men Trællen fortalte Isak alt, hvad han havde udrettet.
\par 67 Da førte Isak Rebekka ind i sin Moder Saras Telt og tog hende til Hustru; og han fik hende kær. Således blev Isak trøstet efter sin Moder.

\chapter{25}

\par 1 Abraham tog sig en Hustru, som hed Ketura;
\par 2 og hun fødte ham Zimran, Joksjan, Medan, Midjan, Jisjbak og Sjua.
\par 3 Joksjan avlede Saba og Dedan. Dedans Sønner var Assjuriterne, Letusjiterne og Le'ummiterne.
\par 4 Midjans Sønner var Efa, Efer, Hanok, Abida og Elda'a. Alle disse var Keturas Sønner.
\par 5 Abraham gav Isak alt, hvad han ejede;
\par 6 men de Sønner, Abraham havde med sine Medhustruer, skænkede han Gaver og sendte dem, medens han endnu levede, bort fra sin Søn Isak, østpå til Østlandet.
\par 7 De År, Abraham levede, udgjorde 175;
\par 8 så udåndede han. Og Abraham døde i en god Alderdom, gammel og mæt af Dage, og samledes til sin Slægt.
\par 9 Og hans Sønner Isak og Ismael jordede ham i Makpelas Klippehule på Hetiten Efrons Zohars Søns, Mark over for Mamre,
\par 10 den Mark, Abraham havde købt af Hetiterne; der jordedes Abraham og hans Hustru Sara.
\par 11 Og da Abraham var død, velsignede Gud hans Søn Isak. Isak boede ved Be'erlahajro'i.
\par 12 Dette er Abrahams Søn Ismaels Slægtebog, hvem Saras Trælkvinde, Ægypterinden Hagar, fødte ham.
\par 13 Følgende er Navnene på Ismaels Sønner efter deres Navne og Slægter: Nebajot, Ismaels førstefødte, Kedar, Adbe'el, Mibsam,
\par 14 Misjma, Duma, Massa,
\par 15 Hadad, Tema, Jetur, Nafisj og Hedma.
\par 16 Det var Ismaels Sønner, og det var deres Navne i deres Indhegninger og Teltlejre, tolv Høvdinger med deres Stammer.
\par 17 Ismaels Leveår udgjorde 137; så udåndede han; han døde og samledes til sin Slægt.
\par 18 De havde deres Boliger fra Havila til Sjur over for Ægypten hen ad Assjur til. Lige for Øjnene af alle sine Brødre slog han sig ned.
\par 19 Dette er Abrahams Søn Isaks Slægtebog. Abraham avlede Isak.
\par 20 Isak var fyrretyve År gammel, da han tog Rebekka, en Datter af Aramæeren Betuel fra Paddan Aram og Søster til Aramæeren Laban, til Hustru.
\par 21 Men Isak bad til HERREN for sin Hustru, thi hun var ufrugtbar; og HERREN bønhørte ham, og Rebekka, hans Hustru, blev frugtsommelig.
\par 22 Men da Sønnerne brødes i hendes Liv, sagde hun: "Står det således til, hvorfor lever jeg da?" Og hun gik hen for at adspørge HERREN.
\par 23 Da svarede HERREN hende: "To Folkeslag er i dit Liv, to Folk skal gå ud af dit Skød! Det ene skal kue det andet, den ældste tjene den yngste!"
\par 24 Da nu Tiden kom, at hun skulde føde, var der Tvillinger i hendes Liv.
\par 25 Den første kom frem rødlig og lodden som en Skindkappe over hele Kroppen; og de kaldte ham Esau.
\par 26 Derefter kom hans Broder frem med Hånden om Esaus Hæl; derfor kaldte de ham Jakob. Isak var tresindstyve År gammel, da de fødtes.
\par 27 Drengene voksede til, og Esau blev en dygtig Jæger, der færdedes i Ødemarken, men Jakob en fredsommelig Mand, en Mand, som boede i Telt.
\par 28 Isak holdt mest af Esau, thi han spiste gerne Vildt; men Rebekka holdt mest af Jakob.
\par 29 Jakob havde engang kogt en Ret Mad, da Esau udmattet kom hjem fra Marken.
\par 30 Da sagde Esau til Jakob: "Lad mig få noget af det røde, det røde der, thi jeg er ved at dø af Sult!" Derfor kaldte de ham Edom.
\par 31 Men Jakob sagde: "Du må først sælge mig din Førstefødselsret!"
\par 32 Esau svarede: "Jeg er jo lige ved at omkomme; hvad bryder jeg mig om min Førstefødselsret!"
\par 33 Men Jakob sagde: "Du må først sværge mig det til!" Da svor Esau på det og solgte sin Førstefødselsret til Jakob.
\par 34 Så gav Jakob Esau Brød og kogte Linser, og da han havde spist og drukket, stod han op og gik sin Vej. Således lod Esau hånt om sin Førstefødselsret.

\chapter{26}

\par 1 Da der opstod Hungersnød i Landet - en anden end den forrige på Abrahams Tid - begav Isak sig til Filisterkongen Abimelek i Gerar.
\par 2 Og HERREN åbenbarede sig for ham og sagde: "Drag ikke ned til Ægypten, men bliv i det Land, jeg siger dig;
\par 3 bo som fremmed i det Land, så vil jeg være med dig og velsigne dig; thi dig og dit Afkom vil jeg give alle disse Lande og stadfæste den Ed, jeg tilsvor din Fader Abraham;
\par 4 og jeg vil gøre dit Afkom talrigt som Himmelens Stjerner og give dit Afkom alle disse Lande, og i din Sæd skal alle Jordens Folk velsignes,
\par 5 fordi Abraham adlød mine Ord og holdt sig mine Forskrifter efterrettelig, mine Bud, Anordninger og Love."
\par 6 Så blev Isak boende i Gerar.
\par 7 Da nu Mændene der på Stedet forhørte sig om hans Hustru, sagde han: "Det er min Søster!" Thi han turde ikke sige, at hun var hans Hustru, af Frygt for at Mændene der på Stedet skulde slå ham ihjel for Rebekkas Skyld; thi hun var meget smuk.
\par 8 Men da han havde boet der en Tid lang, hændte det, at Filisterkongen Abimelek lænede sig ud af Vinduet og så Isak kærtegne sin Hustru Rebekka.
\par 9 Så lod Abimelek Isak kalde og sagde: "Hun er jo din Hustru; hvor kunde du da sige, at hun er din Søster" Isak svarede: "Jo, jeg tænkte: Jeg vil ikke udsætte mig for at miste Livet for hendes Skyld."
\par 10 Men Abimelek sagde: "Hvad er det dog, du har gjort imod os! Hvor let kunde det ikke være sket, at en af Folket havde ligget hos din Hustru, og så havde du bragt Skyld over os!"
\par 11 Så bød Abimelek alt Folket: "Hver den, der rører denne Mand eller hans Hustru, skal lide Døden."
\par 12 Isak såede der i Landet og fik samme År 100 Fold; og HERREN velsignede ham,
\par 13 så han blev en mægtig Mand og stadig gik frem, indtil han blev såre mægtig,
\par 14 og han havde Småkvæg og Hornkvæg og Trælle i Mængde. Derover blev Filisterne skinsyge på ham.
\par 15 Alle de Brønde, hans Faders Trælle havde gravet i hans Fader Abrahams Dage, kastede Filisterne til.og fyldte dem med Jord;
\par 16 og Abimelek sagde til Isak: "Drag bort fra os, thi du er blevet os for stærk!"
\par 17 Så drog Isak bort og slog Lejr i Gerars Dal og bosatte sig der.
\par 18 Men Isak lod atter de Brønde udgrave, som hans Fader Abrahams Trælle havde gravet, og som Filisterne havde tilkastet efter Abrahams Død, og gav dem de samme Navne, som hans Fader havde givet dem.
\par 19 Da nu Isaks Trælle gravede i Dalen, stødte de på en Brønd med rindende Vand;
\par 20 men Gerars Hyrder yppede Kiv med Isaks og sagde: "Dette Vand tilhører os!" Derfor kaldte han Brønden Esek, thi der stredes de med ham.
\par 21 Så flyttede han derfra og lod grave en ny Brønd; og da de også yppede Kiv om den, kaldte han den Sitna.
\par 22 Så flyttede han derfra og lod grave en ny Brønd; og da de ikke yppede Kiv om den, kaldte han den Rehobot, idet han sagde: "Nu har HERREN skaffet os Plads, så vi kan blive talrige i Landet"
\par 23 Så drog han derfra til Be'ersjeba.
\par 24 Samme Nat åbenbarede HERREN sig for ham og sagde: "Jeg er din Fader Abrahams Gud; frygt ikke, thi jeg er med dig, og jeg vil velsigne dig og gøre dit Afkom talrigt for min Tjener Abrahams, Skyld!"
\par 25 Da byggede Isak et Alter der og påkaldte HERRENs Navn; og der opslog han sit Telt, og hans Trælle gravede der en Brønd.
\par 26 Imidlertid kom Abimelek til ham fra Gerar med sin Ven Ahuzzat og sin Hærfører Pikol.
\par 27 Isak sagde til dem: "Hvorfor kommer I til mig, når I dog hader mig og har jaget mig bort fra eder?"
\par 28 Men de svarede: "Vi ser tydeligt, at HERREN er med dig, derfor har vi tænkt: Lad der blive et Edsforbund mellem os og dig, og lad os slutte en Pagt med dig,
\par 29 at du ikke vil gøre os noget ondt, ligesom vi ikke har voldet dig Men, men kun handlet vel imod dig og ladet dig fare i Fred; du er og bliver jo HERRENs velsignede!"
\par 30 Så gjorde han et Gæstebud for dem, og de spiste og drak.
\par 31 Næste Morgen svor de hinanden Eder, og derefter tog Isak Afsked med dem, og de drog bort i Fred.
\par 32 Samme Dag kom Isaks Trælle og bragte ham Melding om den Brønd, de havde gravet, og sagde: "Vi har fundet Vand!"
\par 33 Så kaldte han den Sjib'a; og derfor hedder Byen den Dag i Dag Be'ersjeba.
\par 34 Da Esau var fyrretyve År gammel, tog han Judit, en Datter af Hetiten Be'eri, og Basemat, en Datter af Hetiten Elon, til Ægte.
\par 35 Det var Isak og Rebekka en Hjertesorg.

\chapter{27}

\par 1 Da Isak var blevet gammel og hans Syn sløvet, så han ikke kunde se, kaldte han sin ældste Søn Esau til sig og sagde til ham: "Min Søn!" Han svarede: "Her er jeg!"
\par 2 Da sagde han: "Se, jeg er nu gammel og ved ikke, hvad Dag Døden kommer
\par 3 tag derfor dine Jagtredskaber, dit Pilekogger og din Bue og gå ud på Marken og skyd mig et Stykke Vildt;
\par 4 lav mig en lækker Ret Mad efter min Smag og bring mig den, at jeg kan spise, før at min Sjæl kan velsigne dig, før jeg dør!"
\par 5 Men Rebekka havde lyttet, medens Isak talte til sin Søn Esau, og da, Esau var gået ud på Marken for at skyde et Stykke Vildt til sin Fader,
\par 6 sagde hun til sin yngste Søn Jakob; "Se, jeg hørte din Fader sige til din Broder Esau:
\par 7 Hent mig et Stykke Vildt og lav mig en lækker Ret Mad, at jeg kan spise, før at jeg kan velsigne dig for HERRENs Åsyn før min Død.
\par 8 Adlyd mig nu, min Søn, og gør, hvad jeg pålægger dig:
\par 9 Gå ud til Hjorden og hent mig to gode Gedekid; så laver jeg af dem en lækker Ret Mad til din Fader efter hans Smag;
\par 10 bring så den ind til din Fader, at han kan spise, for at han kan velsigne dig før sin Død!"
\par 11 Men Jakob sagde til sin Moder Rebekka: "Se, min Broder Esau er håret, jeg derimod glat;
\par 12 sæt nu, at min Fader føler på mig, så står jeg for ham som en Bedrager og henter mig en Forbandelse og ingen Velsignelse!"
\par 13 Men hans Moder svarede: "Den Forbandelse tager jeg på mig, min Søn, adlyd mig blot og gå hen og hent mig dem!"
\par 14 Så gik han hen og hentede dem og bragte sin Moder dem, og hun tillavede en lækker Ret Mad efter hans Faders Smag.
\par 15 Derpå tog Rebekka sin ældste Søn Esaus Festklæder, som hun havde hos sig i Huset, og gav sin yngste Søn Jakob dem på;
\par 16 Skindene af Gedekiddene lagde hun om hans Hænder og om det glatte på hans Hals,
\par 17 og så gav hun sin Søn Jakob Maden og Brødet, som hun havde tillavet.
\par 18 Så bragte han det ind til sin Fader og sagde: "Fader!" Han svarede: "Ja! Hvem er du, min Søn?"
\par 19 Da svarede Jakob sin Fader: "Jeg er Esau, din førstefødte; jeg har gjort, som du bød mig; sæt dig nu op og spis af mit Vildt, for at din Sjæl kan velsigne mig!"
\par 20 Men Isak sagde til sin Søn: "Hvor har du så hurtigt kunnet finde noget, min Søn?" Han svarede: "Jo, HERREN din Gud sendte mig det i Møde!"
\par 21 Men Isak sagde til Jakob: "Kom hen til mig, min Søn, så jeg kan føle på dig, om du er min Søn Esau eller ej!"
\par 22 Da trådte Jakob hen til sin Fader, og efter at have følt på ham sagde Isak: "Røsten er Jakobs, men Hænderne Esaus!"
\par 23 Og han kendte ham ikke, fordi hans Hænder var hårede som hans Broder Esaus. Så velsignede han ham.
\par 24 Og han sagde: "Du er altså virkelig min Søn Esau?" Han svarede: "Ja, jeg er!"
\par 25 Da sagde han: "Bring mig det, at jeg kan spise af min Søns Vildt, for at min Sjæl kan velsigne dig!" Så bragte han ham det, og han spiste, og han bragte ham Vin, og han drak.
\par 26 Derpå sagde hans Fader Isak til ham: "Kom hen til mig og kys mig, min Søn!"
\par 27 Og da, han kom hen til ham og kyssede ham, mærkede han Duften af hans Klæder. Så velsignede han ham og sagde: "Se, Duften af min Søn er som Duften af en Mark, HERREN har velsignet!
\par 28 Gud give dig af Himmelens Væde og Jordens Fedme, Korn og Most i Overflod!
\par 29 Måtte Folkeslag tjene dig og Folkefærd bøje sig til Jorden for dig! Bliv Hersker over dine Brødre, og din Moders Sønner bøje sig til Jorden for dig! Forbandet, hvo dig forbander; velsignet, hvo dig velsigner!"
\par 30 Da Isak var færdig med at velsigne Jakob, og lige som Jakob var gået fra sin Fader Isak, vendte hans Broder Esau hjem fra Jagten;
\par 31 også han lavede en lækker Ret Mad, bragte den til sin Fader og sagde: "Vil min Fader sætte sig op og spise af sin Søns Vildt, for at din Sjæl kan velsigne mig!"
\par 32 Så sagde hans Fader Isak: "Hvem er du?" Og han svarede: "Jeg er Esau, din førstefødte!"
\par 33 Da blev Isak højlig forfærdet og sagde: "Men hvem var da han.
\par 34 Da Esau hørte sin Faders Ord: udstødte han et højt og hjerteskærende Skrig og sagde: "Velsign dog også mig, Fader!"
\par 35 Men han sagde: "Din Broder kom med Svig og tog din Velsignelse!"
\par 36 Da sagde han: "Har man kaldt ham Jakob, fordi han skulde overliste mig? Nu har han gjort det to Gange: Han tog min Førstefødselsret, og nu har han også taget min Velsignelse!" Og han sagde: "Har du ingen Velsignelse tilbage til mig?"
\par 37 Men Isak svarede: "Se, jeg har sat ham til Hersker over dig, og alle hans Brødre har jeg gjort til hans Trælle, med Horn og Most. har jeg betænkt ham hvad kan jeg da gøre for dig, min Søn?"
\par 38 Da sagde Esau til sin Fader: "Har du kun den ene Velsignelse.
\par 39 Så tog hans Fader Isak til Orde og sagde til ham: "Se, fjern fra Jordens Fedme skal din Bolig være og fjern fra Himmelens Væde ovenfra;
\par 40 af dit Sværd skal du leve, og din Broder skal du tjene; men når du samler din Kraft, skal du sprænge hans Åg af din Hals!"
\par 41 Men Esau pønsede på ondt mod Jakob for den Velsignelse, hans Fader havde givet ham, og Esau sagde ved sig selv: "Der er ikke længe til, at vi skal holde Sorg over min Fader, så vil jeg slå min Broder Jakob ihjel!"
\par 42 Da nu Rebekka fik Nys om sin ældste Søn Esaus Ord, sendte hun Bud efter sin yngste Søn Jakob og sagde til ham: "Din Broder Esau vil hævne sig på dig og slå dig ihjel;
\par 43 adlyd nu mig min Søn: Flygt til min Broder Laban i Karan
\par 44 og bliv så hos ham en Tid, til din Broders Harme lægger sig,
\par 45 til din Broders Vrede vender sig fra dig, og han glemmer, hvad du har gjort ham; så skal jeg sende Bud og hente dig hjem. Hvorfor skal jeg miste eder begge på een Dag!"
\par 46 Men Rebekka sagde til Isak: "Jeg er led ved Livet for Hets Døtres Skyld; hvis Jakob tager sig sådan en hetitisk Kvinde, en af Landets Døtre, til Hustru, hvad skal jeg da med Livet!"

\chapter{28}

\par 1 Da kaldte Isak Jakob til sig og velsignede ham, idet han bød ham: "Du må ikke tage dig en Hustru blandt Kana'ans Døtre.
\par 2 Drag til Paddan-Aram, til din Morfader Betuels Hus, og tag dig der en af din Morbroder Labans Døtre til Hustru!
\par 3 Gud den Almægtige velsigne dig og gøre dig frugtbar og give dig et talrigt Afkom, så du bliver til Stammer i Hobetal.
\par 4 Han give dig og dit Afkom med dig Abrahams Velsignelse, så du får din Udlændigheds Land i Eje, det,Gud skænkede Abraham!"
\par 5 Så lod Isak Jakob fare, og han drog til Paddan-Aram, til Aramæeren Laban, Betuels Søn, som var Broder til Rebekka, Jakobs og Esaus Moder.
\par 6 Men Esau fik at vide, at Isak havde velsignet Jakob og sendt ham til Paddan-Aram for at tage sig en Hustru der, og at han, da han velsignede ham, havde pålagt ham ikke at tage sig en Hustru blandt Kana'ans Døtre,
\par 7 og at Jakob havde adlydt sin Fader og Moder og var draget til Paddan-Aram.
\par 8 Da skønnede Esau, at Kana'aos Døtre vakte hans Fader Isaks Mishag,
\par 9 og han gik til Ismael og tog Mahalat, en Datter af Abrahams Søn Ismael og Søster til Nebajot, til Hustru ved Siden af sine andre Hustruer.
\par 10 Så drog Jakob bort fra Be'ersjeba og vandrede ad Karan til.
\par 11 På sin Vandring kom han til det hellige Sted og overnattede der, da Solen var gået ned; og han tog en af Stenene på Stedet og brugte den som Hovedgærde og lagde sig til, Hvile der.
\par 12 Da drømte han, og se, på Jorden stod en Stige, hvis Top nåede til Himmelen, og se, Guds Engle steg op og ned ad den;
\par 13 og HERREN stod foran ham og sagde: "Jeg er HERREN, din Fader Abrahams og Isaks Gud! Det Land, du hviler på, giver jeg dig og dit Afkom;
\par 14 dit Afkom skal blive som Jordens Støv, og du skal brede dig mod Vest og Øst, mod Nord og Syd; og i dig og i din Sæd skal alle Jordens Slægter velsignes;
\par 15 se, jeg vil være med dig og vogte dig, hvorhen du end går og føre dig tilbage til dette Land; thi jeg vil ikke forlade dig, men opfylde alt, hvad jeg har lovet dig!"
\par 16 Da Jakob vågnede af sin Søvn, sagde han: "Sandelig, HERREN er på dette Sted, og jeg vidste det ikke!"
\par 17 Og han blev angst og sagde: "Hvor forfærdeligt er dog dette Sted! Visselig, her er Guds Hus, her er Himmelens Port!"
\par 18 Tidligt næste Morgen tog Jakob den Sten, han havde brugt som Hovedgærde rejste den som en Stenstøtte og gød Olie over den.
\par 19 Og han kaldte dette Sted Betel; før hed Byen Luz.
\par 20 Derpå gjorde Jakob følgende Løfte: "Hvis Gud er med mig og vogter mig på den Vej, jeg skal vandre, og giver mig Brød at spise og Klæder at iføre mig,
\par 21 og hvis jeg kommer uskadt tilbage til min Faders Hus, så skal HERREN være min Gud,
\par 22 og denne Sten, som jeg har rejst som en Støtte, skal være Guds Hus, og af alt, hvad du giver mig, vil jeg give dig Tiende!"

\chapter{29}

\par 1 Derpå fortsatte Jakob sin Vandring og drog til Østens Børns Land.
\par 2 Da fik han Øje på en Brønd på Marken og tre Hjorde af Småkvæg, der var lejrede ved den. Ved den Brønd vandede man Hjordene; og over Hullet lå der en stor Sten,
\par 3 som man først væltede bort, når alle Hjordene var samlede, for siden, når Dyrene var vandet, at vælte den på Plads igen.
\par 4 Jakob spurgte dem: "Hvor er I fra, Brødre?" De svarede: "Fra Karan!"
\par 5 Da spurgte han dem: "Kender I Laban, Nakors Søn?" De svarede: "Ja, ham kender vi godt."
\par 6 Han spurgte da: "Går det ham vel? De svarede: "Ja, det går ham vel; se, hans Datter Rakel kommer netop med Hjorden derhenne!"
\par 7 Da sagde han: "Det er jo endnu højlys Dag og for tidligt at drive Kvæget sammen; vand Dyrene og før dem ud på Græsgangene!"
\par 8 Men de svarede: "Det kan vi ikke, før alle Hyrderne er samlede; først når de vælter Stenen fra Brøndhullet, kan vi vande Dyrene."
\par 9 Medens han således stod og talte med dem, var Rakel kommet derhen med sin Faders Hjord, som hun vogtede;
\par 10 og så snart Jakob så sin Morbroder Labans Datter Rakel og hans Hjord, gik han hen og væltede Stenen fra Brøndhullet og vandede sin Morbroder Labans Hjord.
\par 11 Så kyssede han Rakel og brast i Gråd;
\par 12 og han fortalte hende; at han var hendes Faders Frænde, en Søn af Rebekka: Da skyndte hun sig hjem til sin Fader og fortalte ham det"
\par 13 og så snart Laban hørte om sin Søstersøn Jakob, løb han ham i Møde, omfavnede og kyssede ham og førte ham hjem til sit Hus.
\par 14 og Laban sagde: "Ja, du er mit Kød og Blod!" Han blev nu hos ham en Månedstid.
\par 15 Så sagde Laban til Jakob: "Skulde du tjene mig for intet fordi du er min Frænde? Sig mig. hvad du vil have i Løn!"
\par 16 Nu havde Laban to Døtre; den ældste hed Lea, den yngste Rakel;
\par 17 Leas Øjne var matte, men Rakel havde en dejlig Skikkelse og så dejlig ud,
\par 18 og Jakob elskede Rakel; derfor sagde han: "Jeg vil tjene dig syv År for din yngste Datter Rakel."
\par 19 Laban svarede: "Jeg giver hende hellere til dig end til en fremmed; bliv kun hos mig!"
\par 20 Så tjente Jakob syv År for Rakel; og de syntes ham kun nogle få Dage, fordi han elskede hende.
\par 21 Derefter sagde Jakob til Laban: "Giv mig min Hustru, nu min Tjenestetid er ude, at jeg kan gå ind til hende!"
\par 22 Så indbød Laban alle Mændene på Stedet til Gæstebud.
\par 23 Men da Aftenen kom, tog han sin, Datter Lea og bragte hende til ham, og han gik ind til hende.
\par 24 Og Laban gav sin Datter Lea sin Trælkvinde Zilpa til Trælkvinde.
\par 25 Da det nu om Morgenen viste sig at være Lea, sagde Jakob til Laban: "Hvad er det, du har gjort imod mig? Er det ikke for Rakel, jeg,har tjent hos dig? Hvorfor har, du bedraget mig?"
\par 26 Laban svarede: "Det er ikke Skik og Brug her til Lands at give den yngste bort før den ældste;
\par 27 men lad nu Bryllupsugen gå til Ende, så vil, jeg også give dig hende, imod at du bliver i min Tjeneste syv År til."
\par 28 Det gik Jakob ind på, og da Bryllupsugen var til Ende, gav Laban ham sin Datter Rakel til Hustru.
\par 29 Og Laban gav sin Datter Rakel sin Trælkvinde Bilha til Trælkvinde.
\par 30 Så gik Jakob også ind til Rakel, og han elskede Rakel højere end Lea. Derpå blev han i Labans Tjeneste syv År til.
\par 31 Da HERREN så, at Lea blev tilsidesat, åbnede han hendes Moderliv, medens Rakel var ufrugtbar.
\par 32 Så blev Lea frugtsommelig og fødte en Søn, som hun gav Navnet Ruben; thi hun sagde: "HERREN har set til min Ulykke; nu vil min Mand elske mig!"
\par 33 Siden blev hun frugtsommelig igen og fødte en Søn;og hun sagde: "HERREN hørte, at jeg var tilsidesat, så gav han mig også ham!" Derfor gav hun ham Navnet Simeon.
\par 34 Siden blev hun frugtsommelig igen og fødte en Søn; og hun sagde: "Nu må da endelig min Mand bolde sig til mig, da jeg har født ham tre Sønner." Derfor gav hun ham Navnet Levi.
\par 35 Siden blev hun frugtsommelig igen og fødte en Søn; og hun sagde:"Nu vil jeg prise HERREN!" Derfor gav hun ham Navnet Juda.

\chapter{30}

\par 1 Da Rakel så, at hun ikke fødte Jakob noget Barn, blev hun skinsyg på sin Søster og sagde til Jakob: "Skaf mig Børn, ellers dør jeg!"
\par 2 Men Jakob blev vred på Rakel og sagde: "Er jeg i Guds Sted? Det er jo ham, der har nægtet dig Livsfrugt!"
\par 3 Så sagde hun: "Der er min Trælkvinde Bilha; gå ind til hende, så hun kan føde på mine Knæ og jeg få Sønner ved hende!"
\par 4 Og hun gav ham sin Trælkvinde Bilha til Hustru, og Jakob gik ind til hende.
\par 5 Så blev Bilha frugtsommelig og fødte Jakob en Søn,
\par 6 og Rakel sagde: "Gud har hjulpet mig til min Ret, han har hørt min Røst og givet mig en Søn." Derfor gav hun ham Navnet Dan.
\par 7 Siden blev Rakels Trælkvinde Bilha frugtsommelig igen og fødte Jakob en anden Søn;
\par 8 og Rakel sagde: "Gudskampe har jeg kæmpet med min Søster og sejret." Derfor gav hun ham Navnet Naftali.
\par 9 Men da Lea så, at hun ikke fik flere Børn, tog hun sin Trælkvinde Zilpa og gav Jakob hende til Hustru;
\par 10 og da Leas Trælkvinde Zilpa fødte Jakob en Søn,
\par 11 sagde Lea: "Hvilken Lykke!" Derfor gav hun ham Navnet Gad.
\par 12 Siden fødte Leas Trælkvinde Zilpa Jakob en anden Søn;
\par 13 og Lea sagde: "Held mig! Kvinderne vil prise mit Held!" Derfor gav hun ham Navnet Aser.
\par 14 Men da Ruben engang i Hvedehøstens Tid gik på Marken, fandt han nogle Kærlighedsæbler og bragte dem til sin Moder Lea. Da sagde Rakel til Lea: "Giv mig nogle af din Søns Kærlighedsæbler!"
\par 15 Lea svarede: "Er det ikke nok, at du har taget min Mand fra mig? Vil du nu også tage min Søns Kærlighedsæbler?" Men Rakel sagde: "Til Gengæld for din Søns Kærlighedsæbler må han ligge hos dig i Nat!"
\par 16 Da så Jakob kom fra Marken om Aftenen, gik Lea ham i Møde og sagde: "Kom ind til mig i Nat, thi jeg har købt dig for min Søns Kærlighedsæbler!" Og han lå hos hende den Nat.
\par 17 Så bønhørte Gud Lea, og hun blev frugtsommelig og fødte Jakob en femte Søn;
\par 18 og Lea sagde: "Gud har lønnet mig, fordi jeg gav min Mand min Trælkvinde." Derfor gav hun ham Navnet Issakar.
\par 19 Siden blev Lea frugtsommelig igen og fødte Jakob en sjette Søn;
\par 20 og Lea sagde: "Gud har givet mig en god Gave, nu vil min Mand blive hos mig, fordi jeg har født ham seks Sønner." Derfor gav hun ham Navnet Zebulon.
\par 21 Siden fødte hun en Datter, som hun gav Navnet Dina.
\par 22 Så kom Gud Rakel i Hu, og Gud bønhørte hende og åbnede hendes Moderliv,
\par 23 så hun blev frugtsommelig og fødte en Søn; og hun sagde: "Gud har borttaget min Skændsel."
\par 24 Derfor gav hun ham Navnet Josef; thi hun sagde: "HERREN give mig endnu en Søn!"
\par 25 Da Rakel havde født Josef. sagde Jakob til Laban: "Lad mig fare, at jeg kan drage til min Hjemstavn og mit Land;
\par 26 giv mig mine Hustruer og mine Børn som jeg har tjent dig for, og lad mig drage bort; du ved jo selv, hvorledes jeg har tjent dig!"
\par 27 Men Laban svarede: "Måtte jeg have fundet Nåde for dine Øjne! Jeg har udfundet, at HERREN bar velsignet mig for din Skyld."
\par 28 Og han sagde: "Bestem, hvad du vil have i Løn af mig, så vil jeg give dig den!"
\par 29 Så sagde Jakob: "Du ved jo selv, hvorledes jeg har tjent dig, og hvad din Ejendom er blevet til under mine Hænder;
\par 30 thi før jeg kom, ejede du kun lidet, men nu har du Overflod; HERREN har velsignet dig, hvor som helst jeg satte min Fod. Men når kan jeg komme til at gøre noget for mit eget Hus?"
\par 31 Laban svarede: "Hvad skal jeg da give dig?" Da sagde Jakob: "Du skal ikke give mig noget; men hvis du går ind på, hvad jeg nu foreslår dig, vedbliver jeg at være Hyrde for dine Hjorde og vogte dem.
\par 32 Jeg vil i Dag gå hele din Hjord igennem og udskille alle spættede og blakkede Dyr alle de sorte Får og de blakkede eller spættede Geder skal være min Løn;
\par 33 i Morgen den Dag skal min Retfærdighed vidne for mig: Når du kommer og syner den Hjord, der skal være min Løn, da er alle de" Geder, som ikke er spættede eller blakkede, og de Får, som ikke er sorte, stjålet af mig."
\par 34 Laban svarede: "Vel, lad det blive, som du siger!"
\par 35 Så udskilte han samme Dag de stribede og blakkede Bukke og de spættede og blakkede Geder, alle dem der havde hvide Pletter, og alle de sorte Får og overgav dem til sine Sønner,
\par 36 og han lod der være tre Dagsrejser mellem dem og Jakob; og Jakob vogtede Resten af Labans Hjord.
\par 37 Men Jakob tog friske Grene af Hvidpopler, Mandeltræer og Plataner og afskrællede Barken således, at der kom hvide Striber på Grenene;
\par 38 og de afskrællede Grene stillede han op i Trugene foran Dyrene, i Vandrenderne, hvor Dyrene kom hen og drak; og de parrede sig, når de kom for at drikke;
\par 39 Dyrene parrede sig foran Grenene og fødte så stribet, spættet og blakket Afkom.
\par 40 Og Lammene udskilte Jakob. Og han lod Dyrene vende Hovedet mod de stribede og alle de sorte dyr i Labans Hjord. På den Måde fik han sine egne Hjorde, som han ikke bragte sammen med Labans.
\par 41 Og hver Gang de kraftige Dyr parrede sig, stillede Jakob Grenene op foran dem i Vandrenderne, for at de skulde parre sig foran Grenene;
\par 42 men når det var de svage Dyr, stillede han dem ikke op; således kom de svage til at tilhøre Laban, de kraftige Jakob.
\par 43 På den Måde blev Manden overmåde rig og fik Småkvæg i Mængde, Trælkvinder og Trælle, Kameler og Æsler.

\chapter{31}

\par 1 Men Jakob hørte Labans Sønner sige: "Jakob har taget al vor Faders Ejendom, og deraf har han skabt sig al den Velstand."
\par 2 Og Jakob læste i Labans Ansigt, at han ikke var sindet mod ham som tidligere.
\par 3 Da sagde HERREN til Jakob: "Vend tilbage til dine Fædres Land og din Hjemstavn, så vil jeg være med dig!"
\par 4 Så sendte Jakob Bud og lod Rakel og Lea kalde ud på Marken til sin Hjord;
\par 5 og han sagde til dem: "Jeg læser i eders Faders Ansigt, at han ikke er sindet mod mig som tidligere, nu da min Faders Gud har været med mig;
\par 6 og I ved jo selv, at jeg har tjent eders Fader af al min Kraft,
\par 7 medens eders Fader har bedraget mig og forandret min Løn ti Gange; men Gud tilstedte ham ikke at gøre mig Skade;
\par 8 sagde han, at de spættede Dyr skulde være min Løn, så fødte hele Hjorden spættet Afkom, og sagde han, at de stribede skulde være min Løn, så fødte hele Hjorden stribet Afkom.
\par 9 Således tog Gud Hjordene fra eders Fader og gav mig dem.
\par 10 Og ved den Tid Dyrene parrede sig, så jeg i Drømme, at Bukkene, der sprang, var stribede, spættede og brogede
\par 11 og Guds Engel sagde til mig i Drømme: Jakob! Jeg svarede: Se, her er jeg!
\par 12 Da sagde han: Løft dit Blik og se, hvorledes alle Bukkene, der springer, er stribede, spættede og brogede, thi jeg har set alt, hvad Laban har gjort imod dig.
\par 13 Jeg er den Gud, som åbenbarede sig for dig i, Betel, der, hvor du salvede en Stenstøtte og aflagde mig et Løfte; bryd op og forlad dette Land og vend tilbage til din Hjemstavn!"
\par 14 Så svarede Rakel og Lea ham: "Har vi vel mere Lod og Del i vor Faders Hus?
\par 15 Har han ikke regnet os for fremmede Kvinder, siden han solgte os og selv brugte de Penge, han fik for os?
\par 16 Al den Rigdom, Gud har taget fra vor Fader, tilhører os og vore Børn gør du kun alt, hvad Gud sagde til dig!"
\par 17 Så satte Jakob sine Børn og sine Hustruer på Kamelerne
\par 18 og tog alt sit Kvæg med sig, og al den Ejendom, han havde samlet sig, det Kvæg, han ejede og havde samlet sig i Paddan-Aram, for at drage til sin Fader Isak i Kana'ans, Land.
\par 19 Medens Laban var borte og klippede sine Får, stjal Rakel sin Faders Husgud.
\par 20 Og Jakob narrede Aramæeren Laban, idet han ikke lod ham mærke, at han vilde flygte;
\par 21 og han flygtede med alt, hvad han ejede; han brød op og satte over Floden og vandrede ad Gileads Bjerge til.
\par 22 Tredjedagen fik Laban Melding om, at Jakob var flygtet;
\par 23 han tog da sine Frænder med sig, satte efter ham så langt som syv Dagsrejser og indhentede ham: i Gileads Bjerge
\par 24 Men Gud kom til Aramæeren Laban i en Drøm om Natten og sagde til ham: "Vogt dig vel for at sige så meget som et ondt Ord til Jakob!"
\par 25 Da Laban traf Jakob havde han opslået sit Telt på Bjerget.
\par 26 sagde Laban til Jakob: "Hvad har du gjort! Mig har du narret, og mine Døtre har du ført bort. som var de Krigsfanger!
\par 27 Hvorfor har du holdt din Flugt hemmelig og narret mig og ikke meddelt mig det; så jeg kunde tage Afsked med dig med Lystighed og Sang, med Håndpauker og Harper?
\par 28 Du lod mig ikke kysse mine Sønner og Døtre - sandelig, det var dårligt gjort af dig!
\par 29 Det stod nu i min Magt at handle ilde med dig; men din Faders Gud sagde til mig i Nat: Vogt dig vel for at sige så meget som et ondt Ord til Jakob!
\par 30 Nu vel, så drog du altså bort fordi du længtes så meget efter din Faders Hus men hvorfor stjal du min Gud?"
\par 31 Da svarede Jakob Laban: "Jeg var bange; thi jeg tænkte, du vilde rive dine Døtre fra mig!
\par 32 Men den, hos hvem du finder din Gud, skal lade sit Liv! Gennemsøg i vore Frænders Påsyn, hvad: jeg har, og tag, hvad dit er!" Jakob vidste nemlig ikke, at Rakel havde; stjålet den.
\par 33 Laban gik nu ind og ledte i Jakobs, i Leas og i de to Tjenestekvinders Telte men fandt intet; og fra Leas gik han, til Rakels, Telt.
\par 34 Men Rakel havde taget Husguden og lagt den i Kamelsadlen og sat sig på den. Da Laban nu havde gennemsøgt hele Teltet og intet fundet,
\par 35 sagde hun til sin Fader: "Min Herre tage mig ikke ilde op, at jeg ikke kan rejse mig for dig, da det går mig på Kvinders Vis!" Således ledte han efter Husguden uden at finde den.
\par 36 Da blussede Vreden op i Jakob, og han gik i Rette med Laban; og Jakob sagde til Laban: "Hvad er min Brøde, og hvad er min Synd, at du satte efter mig!
\par 37 Du har jo nu gennemsøgt alle mine Ting! Hvad har du fundet af alle dine Sager? Læg det frem for mine Frænder og dine Frænder, at de kan dømme os to imellem!
\par 38 I de tyve År, jeg har været hos dig, fødte dine Får og Geder ikke i Utide, din Hjords Vædre fortærede jeg ikke,
\par 39 det sønderrevne bragte jeg dig ikke, men erstattede det selv; af min Hånd krævede du, hvad der blev stjålet både om Dagen og om Natten;
\par 40 om Dagen fortærede Heden mig, om Natten Kulden, og mine Øjne kendte ikke til Søvn.
\par 41 I tyve År har jeg tjent dig i dit Hus, fjorten År for dine to Døtre og seks År for dit Småkvæg, og ti Gange har du forandret min Løn.
\par 42 Havde ikke min Faders Gud, Abrahams Gud og Isaks Rædsel, stået mig bi, så havde du ladet mig gå med tomme Hænder; men Gud så min Elendighed og mine Hænders Møje, og i Nat afsagde han sin Kendelse!"
\par 43 Da sagde Laban til Jakob: "Døtrene er mine Døtre, Sønnerne er mine Sønner, Hjordene er mine Hjorde, og alt, hvad du ser, er mit men hvad skulde jeg i Dag kunne gøre imod mine Døtre eller de Sønner, de har født?
\par 44 Lad os to slutte et Forlig, og det skal tjene til Vidne mellem os."
\par 45 Så tog Jakob en Sten og rejste den som en Støtte;
\par 46 og Jakob sagde til sine Frænder: "Sank Sten sammen!" Og de tog Sten og byggede en Dysse og holdt Måltid derpå.
\par 47 Laban kaldte den Jegar-Sahaduta, og Jakob kaldte den Galed.
\par 48 Da sagde Laban: "Denne Dysse skal i Dag være Vidne mellem os to!" Derfor kaldte han den Galed
\par 49 og Mizpa; thi han sagde: "HERREN skal stå Vagt mellem mig og dig, når vi skilles.
\par 50 Hvis du handler ilde med mine Døtre eller tager andre Hustruer ved Siden af dem, da vid, at selv om intet Menneske er til Stede, er dog Gud Vidne mellem mig og dig!"
\par 51 Og Laban sagde til Jakob: "Se denne Stendysse og se denne Stenstøtte, som jeg har rejst mellem mig og dig!
\par 52 Vidne er denne Dysse, og Vidne er denne Støtte på, at jeg ikke i fjendtlig Hensigt vil gå forbi denne Dysse ind til dig, og at du heller ikke vil gå forbi den ind til mig;
\par 53 Abrahams Gud og Nakors Gud, deres Faders Gud, være Dommer imellem os!" Så svor Jakob ved sin Fader Isaks Rædsel,
\par 54 og derpå holdt Jakob Offerslagtning på Bjerget og indbød sine Frænder til Måltid; og de holdt Måltid og blev på Bjerget Natten over.
\par 55 Tidligt næste Morgen kyssede Laban sine Sønner og Døtre, velsignede dem og drog bort; og Laban vendte tilbage til sin Hjemstavn,

\chapter{32}

\par 1 men Jakob fortsatte sin Rejse. Og Guds Engle mødte ham;
\par 2 og da Jakob så dem, sagde han: "Her er Guds Lejr!" derfor kaldte han Stedet Mahanajim.
\par 3 Derpå sendte Jakob Sendebud i Forvejen til sin Broder Esau i Se'irs Land på Edoms Højslette,
\par 4 og han bød dem: "Sig til min Herre Esau: Din Træl Jakob lader dig vide, at jeg har levet som Gæst hos Laban og boet der indtil nu;
\par 5 jeg har samlet mig Okser,Æsler og Småkvæg, Trælle og Trælkvinder; og nu sender jeg Bud til min Herre med Efterretning herom i Håb om at finde Nåde for dine Øjne!"
\par 6 Men Sendebudene kom tilbage til Jakob og meldte: "Vi kom til din Broder Esau, og nu drager han dig i Møde med 400 Mand!"
\par 7 Da blev Jakob såre forfærdet, og i sin Angst delte han sine Folk, Småkvæget, Hornkvæget og Kamelerne i to Lejre,
\par 8 idet han tænkte: "Hvis Esau møder den ene Lejr og slår den, kan dog den anden slippe bort."
\par 9 Derpå bad Jakob: "Min Fader Abrahams og min Fader Isaks Gud, HERRE, du, som sagde til mig: Vend tilbage til dit Land og din Hjemstavn, så vil jeg gøre vel imod dig!
\par 10 Jeg er for ringe til al den Miskundhed og Trofasthed, du har udvist mod din Tjener; thi med min Stav gik jeg over Jordan der, og nu er jeg blevet til to Lejre;
\par 11 frels mig fra min Broder Esaus Hånd, thi jeg frygter for, at han skal komme og slå mig, både Moder og Børn!
\par 12 Du har jo selv sagt, at du vil gøre vel imod mig og gøre mit Afkom som Havets Sand, der ikke kan tælles for Mængde!"
\par 13 Og han blev der om Natten. Af hvad han havde, udtog han så en Gave til sin Broder Esau,
\par 14 200 Geder og 20 Bukke, 200 Får og 20 Vædre,
\par 15 34 diegivende Kamelhopper med deres Føl, 40 Køer og 10 Tyre, 20 Aseninder og 10 Æselhingste;
\par 16 han delte dem i flere Hjorde og overlod sine Trælle dem, idet han sagde til dem: "Gå i Forvejen og lad en Plads åben mellem Hjordene!"
\par 17 Og han bød den første: "Når min Broder Esau møder dig og spørger, hvem du tilhører, hvor du skal hen, og hvem din Drift tilhører,
\par 18 skal du svare: Den tilhører din Træl Jakob; det er en Gave.
\par 19 Og han bød den anden og den tredje og alle de andre, der fulgte med Hjordene, at sige det samme til Esau, når de traf ham:
\par 20 "Din Træl Jakob kommer selv bagefter!" Thi han tænkte: "Jeg vil søge at forsone ham ved den Gave. der drager foran, og først bagefter vil jeg træde frem for ham; måske han da tager venligt imod mig!"
\par 21 Så drog Gaven i Forvejen, medens han selv blev i Lejren om Natten.
\par 22 Samme Nat tog han sine to Hustruer, sine to Trælkvinder og sine elleve Børn og gik over Jakobs Vadested;
\par 23 han tog dem og bragte dem over Bækken; ligeledes bragte han alt. hvad han ejede, over.
\par 24 Men selv blev Jakob alene tilbage. Da var der en, som brødes, med ham til Morgengry;
\par 25 og da han så, at han ikke kunde få Bugt med ham, gav han ham et Slag på Hofteskålen; og Jakobs Hofteskål gik af Led, da han brødes med ham.
\par 26 Da sagde han: "Slip mig, thi Morgenen gryr!" Men han svarede: "Jeg slipper dig ikke, uden du velsigner mig!"
\par 27 Så spurgte han: "Hvad er dit Navn?" Han svarede: "Jakob!"
\par 28 Men han sagde: "Dit Navn skal ikke mere være Jakob, men Israel; thi du har kæmpet med Gud og Mennesker og sejret!"
\par 29 Da sagde Jakob:"Sig mig dit Navn!" Men han svarede: "Hvorfor spørger du om mit Navn?" Og han velsignede ham der.
\par 30 Og Jakob kaldte Stedet Peniel, idet han sagde: "Jeg har skuet Gud Ansigt til Ansigt og har mit Liv frelst."
\par 31 Og Solen stod op, da han drog forbi Penuel, og da haltede han på Hoften.
\par 32 Derfor undlader Israeliterne endnu den Dag i Dag at spise Hoftenerven, der ligger over Hofteskålen, thi han gav Jakob et Slag på Hofteskålen, på Hoftenerven.

\chapter{33}

\par 1 Da Jakob så op, fik han Øje på Esau, der kom fulgt af 400 Mand. Så delte han Børnene mellem Lea, Rakel og de to Trælkvinder,
\par 2 idet han stillede Trælkvinderne med deres Børn forrest, Lea med hendes Børn længere tilbage og bagest Rakel med Josef;
\par 3 selv gik han frem foran dem og kastede sig syv Gange til Jorden, før han nærmede sig sin Broder.
\par 4 Men Esau løb ham i Møde og omfavnede ham, faldt ham om Halsen og kyssede ham, og de græd;
\par 5 og da han så op og fik Øje på Kvinderne og Børnene, sagde han: "Hvem er det, du har der?" Han svarede: "Det er de Børn, Gud nådig har givet din Træl."
\par 6 Så nærmede Trælkvinderne sig med deres Børn og kastede sig til Jorden,
\par 7 derefter nærmede Lea sig med sine Børn og kastede sig til Jorden, og til sidst nærmede Josef og Rakel sig og kastede sig til Jorden.
\par 8 Nu spurgte han: "Hvad vilde du med hele den Lejr, jeg traf på?" Han svarede: "Finde Nåde for min Herres Øjne!"
\par 9 Men Esau sagde: "Jeg har nok, Broder; behold du, hvad dit er!"
\par 10 Da svarede Jakob: "Nej, hvis jeg har fundet Nåde for dine Øjne, så tag imod min Gave! Da jeg så dit Åsyn, var det jo som Guds Åsyn, og du har taget venligt imod mig!
\par 11 Tag dog den Velsignelse, som er dig bragt, thi Gud har været mig nådig, og jeg har fuldt op!" Således nødte han ham, til han tog det.
\par 12 Derpå sagde Esau: "Lad os nu bryde op og drage af Sted, og jeg vil drage foran dig!"
\par 13 Men Jakob svarede: "Min Herre ved jo, at jeg må tage Hensyn til de spæde Børn og de Får og Køer, som giver Die; overanstrenger jeg dem blot en eneste Dag, dør alt Småkvæget.
\par 14 Vil min Herre drage forud for sin Træl, kommer jeg efter i Ro og Mag, som det passer sig for Kvæget, jeg har med, og for Børnene, til jeg kommer til min Herre i Seir."
\par 15 Da sagde Esau: "Så vil jeg i alt Fald lade nogle af mine Folk ledsage dig!" Men han svarede: "Hvorfor dog det måtte jeg blot finde Nåde for min Herres Øjne!"
\par 16 Så drog Esau samme Dag tilbage til Seir.
\par 17 Men Jakob brød op og drog til Sukkot, hvor han byggede sig et Hus og indrettede Hytter til sit Kvæg; derfor gav han Stedet Navnet Sukkot.
\par 18 Og Jakob kom på sin Vandring fra Paddan Aram uskadt til Sikems By i Kana'ans Land og slog Lejr uden for Byen;
\par 19 og han købte det Stykke Jord, hvor han havde rejst sit Telt, af Sikems Pader Hamors Sønner for 100 Kesita
\par 20 og byggede der et Alter, som han kaldte: Gud, Israels Gud.

\chapter{34}

\par 1 Da Dina, den Datter, Jakob havde med Lea, engang gik ud for at besøge Landets Døtre,
\par 2 så Sikem, en Søn af Egnens Høvding, Hivviten Hamor, hende og greb hende og lå hos hende; og han krænkede hende;
\par 3 men hans Hjerte hang ved Jakobs Datter Dina, og han elskede Pigen og talte godt for hende;
\par 4 og Sikem sagde til sin Fader Hamor: "Skaf mig den Pige til Hustru!"
\par 5 Jakob hørte, at han havde skændet hans Datter Dina; men da hans Sønner dengang var med hans Kvæg på Marken, tav han, til de kom hjem."
\par 6 Sikems Fader Hamor gik nu til Jakob for at tale med ham.
\par 7 Men da Jakobs Sønner hørte det, kom de hjem fra Marken; og Mændene græmmede sig og var såre opbragte, fordi han havde øvet Skændselsdåd i Israel ved at ligge hos Jakobs Datter; thi sligt bør ikke ske.
\par 8 Og Hamor talte med dem og sagde: "Min Søn Sikems Hjerte hænger ved eders Datter; giv ham hende til Hustru
\par 9 og indgå Svogerskab med os; giv os eders Døtre og gift eder med vore Døtre;
\par 10 tag Ophold hos os, og Landet skal stå eder åbent; slå eder ned og drag frit omkring og saml eder Ejendom der!"
\par 11 Og Sikem sagde til hendes Fader og Brødre: "Måtte jeg finde Nåde for eders Øjne! Alt, hvad I kræver, vil jeg give
\par 12 forlang så høj en Brudesum og Gave, I vil; jeg giver, hvad I kræver, når I blot vil give mig Pigen til Hustru!"
\par 13 Da gav Jakobs Sønner Sikem og hans Fader Hamor et listigt Svar, fordi. han havde skændet deres Søster Dina,
\par 14 og sagde til dem: "Vi er ikke i Stand til at give vor Søster til en uomskåren Mand, thi det holder vi for en Skændsel.
\par 15 Kun på det Vilkår vil vi føje eder, at I bliver som vi og lader alle af Mandkøn iblandt eder omskære;
\par 16 i så Fald vil vi give eder vore Døtre og ægte eders Døtre og bosætte os iblandt eder, så vi bliver eet Folk;
\par 17 men hvis I ikke vil høre os og lade eder omskære, så tager vi vor Datter og drager bort"
\par 18 Deres Tale tyktes Hamor og Sikem, Hamors Søn, god;
\par 19 og den unge Mand tøvede ikke med at gøre således, thi han var indtaget i Jakobs Datter, og han var den, der havde mest at sige i sin Faders Hus
\par 20 og Harnor og hans Søn Siken gik til deres Bys Port og sagde til, Mændene i deres By:
\par 21 "Disse Mænd er os velsindede; lad dem bosætte sig og drage frit om her i Landet, der er jo Plads nok til dem i Landet; deres Døtre vil vi tage til Hustruer og give dem vore Døtre til Hustruer!
\par 22 Men kun på det Vilkår vil Mændene føje os og bosætte sig hos os, så vi kan blive eet Folk, at alle af Mandkøn hos os lader sig omskære, således som de er omskårne.
\par 23 Deres Hjorde og deres Gods og alt deres Kvæg bliver jo dog vort; lad os derfor føje dem, så de kan blive boende hos os!"
\par 24 Så adlød de Hamor og hans Søn Sikem, så mange som færdedes i hans Bys Port, og alle af Mandkøn, alle, som færdedes i hans Bys Port, lod sig omskære.
\par 25 Men Tredjedagen, da de havde Sårfeber, tog Jakobs to Sønner Simeon og Levi, Dinas Brødre, hver sit Sværd, trængte ind i Byen, uden at nogen anede Uråd, 0g slog alle Mændene ihjel
\par 26 og dræbte Hamor og hans Søn Sikem med Sværdet, tog Dina ud af Sikems Hus og drog bort.
\par 27 Så kastede Jakobs Sønner sig over de faldne og plyndrede Byen, fordi de havde skændet deres Søster;
\par 28 deres Småkvæg, Hornkvæg og Æsler, både hvad der var i Byen og på Markerne, tog de med sig,
\par 29 og al deres Ejendom og alle deres Børn og Kvinder førte de bort som Bytte, og de udplyndrede Byen for alt, hvad der var der.
\par 30 Men Jakob sagde til Simeon og Levi: "I styrter mig i Ulykke ved at lægge mig for Had hos Landets Indbyggere, Kana'anæerne og Perizziterne; thi jeg råder kun over få Folk; samler de sig mod mig og slår mig, så er det ude med mig og mit Hus!"
\par 31 Men de svarede: "Skal han behandle vor Søster som en Skøge!"

\chapter{35}

\par 1 Derpå sagde Gud til Jakob: "Drag op til Betel og bliv der og byg der et Alter for Gud, som åbenbarede sig for dig, da du flygtede for din Broder Esau!"
\par 2 Jakob sagde da til sit Hus og alle sine Folk: "Skaf de fremmede Guder, der findes hos eder, bort, rens eder og skift Klæder,
\par 3 og lad os drage op til Betel, for at jeg der kan bygge et Alter for Gud, der bønhørte mig i min Trængselstid og var med mig på den Vej, jeg vandrede!"
\par 4 De gav så Jakob alle de fremmede Guder, de førte med sig, og alle de Ringe, de havde i Ørene, og han gravede dem ned under Egen ved Sikem.
\par 5 Derpå brød de op; og en Guds Rædsel kom over alle Byerne rundt om, så de ikke forfulgte Jakobs Sønner.
\par 6 Og Jakob kom med alle sine Folk til Luz i Kana'ans Land, det er Betel;
\par 7 og han byggede et Alter der og kaldte Stedet: Betels Gud, thi der havde Gud åbenbaret sig for ham, da han flygtede for sin Broder.
\par 8 Så døde Rebekkas Amme Debora, og hun blev jordet neden for Betel under Egen; derfor kaldte han den Grædeegen.
\par 9 Gud åbenbarede sig atter for Jakob efter hans Hjemkomst fra Paddan Aram og velsignede ham;
\par 10 og Gud sagde til ham: "Dit Navn er Jakob; men herefter skal du ikke mere hedde Jakob; Israel skal være dit Navn!" Og han gav ham Navnet Israel.
\par 11 Derpå sagde Gud til ham: "Jeg er Gud den Almægtige! Bliv frugtbar og mangfoldig! Et Folk,ja Folk i Hobetal skal nedstamme fra dig, og.Konger skal udgå af din Lænd;
\par 12 det Land, jeg gav Abraham og Isak, giver jeg dig, og dit Afkom efter dig giver jeg Landet!"
\par 13 Derpå for Gud op fra ham på det Sted, hvor han havde talet med ham;
\par 14 og Jakob rejste en Støtte på det Sted, hvor han havde talet med ham, en Stenstøtte, og hældte et Drikofer over den og udgød Olie på den.
\par 15 Og Jakob kaldte det Sted, hvor Gud havde talet med ham, Betel.
\par 16 Derpå brød de op fra Betel, Da de endnu var et Stykke Vej fra Efrat, skulde Rakel føde, og hendes Fødselsveer var hårde.
\par 17 Midt under hendes hårde Fødselsveer sagde Jordemoderen til hende: "Frygt ikke, thi også denne Gang får du en Søn!"
\par 18 Men da hun droges med Døden thi det kostede hende Livet gav hun ham Navnet Ben'oni; men Faderen kaldte ham Benjamin".
\par 19 Så døde Rakel og blev jordet på Vejen til Efrat, det er Betlehem;
\par 20 og Jakob rejste en Stenstøtte på hendes Grav; det er Rakels Gravstøtte, som står endnu den Dag i Dag.
\par 21 Derpå brød Israel op og opslog sit Telt hinsides Migdal Eder.
\par 22 Men medens Israel boede i den Egn, gik Ruben hen og lå hos sin Faders Medhustru Bilha; og det kom Israel for Øre. Jakobs Sønner var tolv i Tal;
\par 23 Leas Sønner: Ruben, Jakobs førstefødte, Simeon, Levi, Juda, Issakar og Zebulon;
\par 24 Rakels Sønner: Josef og Benjamin;
\par 25 Rakels Trælkvinde Bilhas Sønner: Dan og Naftali;
\par 26 Leas Trælkvinde Zilpas Sønner: Gad og Aser. Det var Jakobs Sønner, der fødtes ham i Paddan Aram.
\par 27 Og Jakob kom til sin Fader Isak i Mamre i Kirjat Arba, det er Hebron, hvor Abraham og Isak havde levet som fremmede.
\par 28 Isaks Leveår var 180;
\par 29 så gik Isak bort; han døde og samledes til sin Slægt, gammel og mæt af Dage. Og hans Sønner Esau og Jakob jordede ham,

\chapter{36}

\par 1 Dette er Esaus, det er Edoms, Slægtsbog.
\par 2 Esau tog sine Hustruer af Kana'ans Døtre: Ada, en Datter af Hetiten Elon, Oholibama, en Datter af Ana, Hivviten Zibons Søn,
\par 3 og Ismaels Datter Basemat, Søster til Nebajot.
\par 4 Ada fødte Esau Elifaz, Basemat fødte Reuel,
\par 5 og Obolibama fødte Jeusj, Jalam og Kora. Det var Esaus Sønner, der fødtes ham i Kana'ans Land.
\par 6 Derpå tog Esau sine Hustruer sine Sønner og Døtre, hele sin Husstand, sit Kvæg og al den Ejendom, han havde samlet sig i Kana'ans Land, og drog til Landet lige over for sin Broder Jakob;
\par 7 deres Gods var for meget til, at de kunde bo sammen, og deres. Udlændigheds Land kunde ikke rumme dem, så store var deres. Hjorde;
\par 8 og Esau bosatte sig i Seirs. Bjerge; Esau, det er Edom.
\par 9 Dette er Esaus Slægtebog, han, som var Stamfader til Edomiterne i Seirs Bjerge.
\par 10 Følgende var Esaus Sønners Navne: Elifaz, en Søn af Esaus Hustru Ada, og Reuel, en Søn af Esaus Hustru Basemat.
\par 11 Elifazs Sønner var Teman, Omar, Zefo, Gatam og Kenaz.
\par 12 Timna, som var Esaus Søn Elifazs Medhustru, fødte ham Amalek. Det var Esaus Hustru Adas Sønner.
\par 13 Følgende var Reuels Sønner: Nahat, Zera, Sjamma og Mizza.
\par 14 Følgende var Sønner af Esaus Hustru Oholibama, Datter af Zibons Søn Ana; hun fødte for Esau: Jeusj, Jalam og Kora.
\par 15 Følgende var Esaus Sønners Stammehøvdinger: Elifaz's, Esaus førstefødtes, Sønner: Høvdingeroe Teman, Omar, Zefo, Henaz,
\par 16 Kora, Gatam og Amalek. Det var de fra Elifaz stammende Høvdinger i Edoms Land; det var Adas Sønner.
\par 17 Følgende var Esaus Søn Reuels Sønner: Høvdingerne Nahat, Zera, Sjamma og Mizza. Det var de fra Reuel stammende Høvdinger i Edoms Land; det var Esaus Hustru Basemats Sønner.
\par 18 Følgende var Esaus Hustru Oholibamas Sønner: Høvdingerne Jeusj, Jalam og Kora. Det var de Høvdinger, der stammede fra Oholibama, Esaus Hustru, Anas Datter.
\par 19 Det var Esaus Sønner, og det var deres Stammehøvdinger; det var Edom.
\par 20 Følgende var Horiten Seirs Sønner, Landets oprindelige Befolkning: Lotan, Sjobal, Zibon, Ana,
\par 21 Disjon, Ezer og Risjon. Det var Horiternes Stammebøvdinger, Seirs Sønner i Edoms Land.
\par 22 Lotans Sønner var Hori og Hemam, og Lotans Søster var Timna.
\par 23 Følgende var Sjobals Sønner: Alvan, Manahat, Ebal, Sjefo og Onam.
\par 24 Følgende var Zibons Sønner: Aja og Ana. Det var denne Ana, som fandt de varne Kilder i Ørkenen, da han vogtede sin Fader Zibons Æsler.
\par 25 Følgende var Anas Børn: Disjon og Oholibama, Anas Datter.
\par 26 Følgende var Disjons Sønner: Hemdan, Esjban, Jitran og Keran.
\par 27 Følgende var Ezers Sønner: Bilhan, Zåvan og Akan.
\par 28 Følgende var Risjons Sønner: Uz og Aran.
\par 29 Følgende var Horiternes Stammehøvdinger: Høvdingerne Lotan, Sjobal, Zibon, Ana,
\par 30 Disjon, Ezer og Risjon. Det var Horiternes Stammehøvdinger efter deres Stammer i Seirs Land.
\par 31 Følgende var de Konger, der herskede i Edoms Land, før Israeliterne fik Konger:
\par 32 Bela, Beors Søn, herskede i Edom; hans By hed Dinbaba.
\par 33 Da Bela døde, blev Jobab, Zeras Søn fra Bozra, Konge i hans Sted.
\par 34 Da Jobab døde, blev Husjam fra Temaniternes Land Konge i hans Sted.
\par 35 Da Husjam døde, blev Hadad, Bedads Søn, Kongei hans Sted; det var ham, der slog Midjaniterne på Moabs Slette; hans By hed Avit.
\par 36 Da Hadad døde, blev Samla fra Masreka Konge i hans Sted.
\par 37 Da Samla døde, blev Sjaul fra Rehobot ved Floden Konge i hans Sted.
\par 38 Da Sjaul døde, blev Bål Hanan, Akbors Søn, Konge i hans Sted.
\par 39 Da Bål Hanan, Akbors Søn, døde, blev Hadar Konge i hans Sted; hans By hed Pau, og hans Hustru hed Mehetabel, en Datter af Matred, en Datter af Mezahab.
\par 40 Følgende var Navnene på Esaus Stammehøvdinger efter deres Slægter, Bosteder og Navne: Høvdingerne Timna, Alva, Jetet,
\par 41 Oholibama, Ela, Pinon,
\par 42 Kenaz, Teman, Mibzar
\par 43 Magdiel og Iram. Det var Edoms Stammehøvdinger efter deres Boliger i det Land, de fik i Eje. Det var Esau, Edoms Fader.

\chapter{37}

\par 1 Men Jakob blev boende i sin Faders Udlændigheds Land, i Kana'ans Land
\par 2 Dette er Jakobs Slægtebog. Da Josef var sytten År gammel, vogtede han Småkvæget sammen med sine Brødre; som Dreng var han hos sin Faders Hustruer Bilhas og Zilpas Sønner, og han bragte ondt Rygte om dem til deres Fader.
\par 3 Israel elskede Josef fremfor alle sine andre Sønner, fordi han var hans Alderdoms Søn, og han lod gøre en fodsid Kjortel med Ærmer til ham.
\par 4 Men da hans Brødre så, at deres Fader foretrak ham for alle sine andre Sønner, fattede de Nag til ham og kunde ikke tale venligt til ham.
\par 5 Men Josef havde en Drøm, som han fortalte sine Brødre, og som yderligere øgede deres Had til ham.
\par 6 Han sagde til dem "Hør dog, hvad jeg har drømt!
\par 7 Se, vi bandt Neg ude på Marken, og se, mit Neg rejste sig op og blev stående, medens eders Neg stod rundt omkring og bøjede sig for det!"
\par 8 Da sagde hans Brødre til ham: "Vil du måske være vor Konge eller herske over os?" Og de hadede ham endnu mere for hans Drømme og hans Ord.
\par 9 Men han havde igen en Drøm, som han fortalte sine Brødre; han sagde: "Jeg har haft en ny Drøm, og se, Sol og Måne og elleve Stjerner bøjede sig for mig!"
\par 10 Da han fortalte sin Fader og sine Brødre det, skændte hans Fader på ham og sagde: "Hvad er det for en Drøm, du der har haft Skal virkelig jeg, din Moder og dine Brødre komme og bøje os til Jorden for dig?"
\par 11 Og hans Brødre fattede Avind til ham, men hans Fader gemte det i sit Minde.
\par 12 Da hans Brødre engang var gået hen for at vogte deres Faders Småkvæg ved Sikem,
\par 13 sagde Israel til Josef: "Dine Brødre vogter jo Kvæg ved Sikem; kom, jeg vil sende dig til dem!" Han svarede: "Her er jeg!"
\par 14 Så sagde Israel til ham: "Gå hen og se, hvorledes det står til med dine Brødre og Kvæget, og bring mig Bud tilbage!" Israel sendte ham så af Sted fra Hebrons Dal, og han kom til Sikem.
\par 15 Som han nu flakkede om på Marken, var der en Mand, som traf ham og spurgte: "Hvad søger du efter?"
\par 16 Han svarede: "Efter mine Brødre; sig mig, hvor de vogter deres Kvæg!"
\par 17 Da sagde Manden: "De er draget bort herfra, thi jeg hørte dem sige: Lad os gå til Dotan!" Så gik Josef efter sine Brødre og fandt dem i Dotan.
\par 18 Men da de så ham langt borte, før han endnu var kommet hen til dem, lagde de Råd op om at dræbe ham
\par 19 og sagde til hverandre: "Se, der kommer den Drømmemester!
\par 20 Kom, lad os slå ham ihjel og kaste ham i en Cisterne og sige, at et vildt Dyr har ædt ham; så skal vi se, hvad der kommer ud af hans Drømme!"
\par 21 Men da Ruben hørte det, vilde han redde ham af deres Hånd og sagde: "Lad os ikke tage hans Liv!"
\par 22 Og Ruben sagde til dem: "Udgyd dog ikke Blod! Kast ham i Cisternen her på Marken, men læg ikke Hånd på ham!" Han vilde nemlig redde ham af deres Hånd og bringe ham tilbage til Faderen.
\par 23 Da Josef nu kom hen til sine Brødre, rev de hans Kjortel af ham, Ærmekjortelen, han havde på,
\par 24 tog ham og kastede ham i Citernen; men Cisternen var tom, der var intet Vand i den.
\par 25 Derpå satte de sig til at holde Måltid. Og da de så op, fik de Øje på en Karavane af Ismaeliter, der kom fra Gilead, og deres Kameler var belæssede med Tragakantgummi, Mastiksbalsam og Cistusharpiks, som de var på Vej til Ægypten med.
\par 26 Så sagde Juda til sine Brødre: "Hvad vinder vi ved at slå vor Broder ihjel og skjule Mordet?
\par 27 Lad os hellere sælge ham til Ismaeliterne og ikke lægge Hånd på ham; han er jo dog vor Broder, vort Kød og Blod!" Og hans Brødre gik ind på Forslaget.
\par 28 Da nu midjanitiske Købmænd kom der forbi, trak de Josef op af Cisternen. Og de solgte Josef til Ismaeliterne for tyve Sekel Sølv, og disse bragte ham så til Ægypten.
\par 29 Da Ruben nu kom tilbage til Cisternen, se, da var Josef der ikke. Så sønderrev han sine Klæder
\par 30 og gik tilbage til sine Brødre og sagde: "Drengen er borte! Hvad skal jeg dog gøre!"
\par 31 Så tog de Josefs Kjortel og dyppede den i Blodet af en Gedebuk, som de slagtede;
\par 32 og de sendte Ærmekjortelen hjem til deres Fader med det Bud: "Den har vi fundet se efter, om det ikke er din Søns Kjortel!"
\par 33 Da så han efter og udbrød: "Det er min Søns Kjortel! Et vildt Dyr har ædt ham! Josef er visselig revet ihjel!"
\par 34 Så sønderrev Jakob sine Klæder og bandt Sæk om sine Lænder, og han sørgede over sin Søn i mange Dage.
\par 35 Og skønt alle hans Sønner og Døtre kom til ham for at trøste ham, vilde han ikke lade sig trøste, men sagde: "Nej, i min Sørgedragt vil jeg stige ned til min Søn i Dødsriget!" Og hans Fader begræd ham.
\par 36 Men Midjaniterne solgte ham I Ægypten til Faraos Hofmand Potifar, Livvagtens Øverste.

\chapter{38}

\par 1 Ved den Tid forlod Juda sine Brødre og sluttede sig til en Mand fra Adullam ved Navn Hira.
\par 2 Der så Juda en Datter af Kana'anæeren Sjua, og han tog hende til Ægte og gik ind til hende.
\par 3 Hun blev frugtsommelig og fødte en Søn, som hun gav Navnet Er;
\par 4 siden blev hun frugtsommelig igen og fødte en Søn, som hun gav Navnet Onan;
\par 5 og hun fødte endnu en Søn, som hun gav Navnet Sjela; da hun fødte ham, var hun i, Kezib.
\par 6 Juda tog Er, sin førstefødte, en Hustru, der hed Tamar.
\par 7 Men Er, Judas førstefødte, vakte HERRENs Mishag, derfor lod HERREN ham dø.
\par 8 Da sagde Juda til Onan: "Gå ind til din Svigerinde og indgå Svogerægteskab med hende for at skaffe din Broder Afkom!"
\par 9 Men Onan, som vidste, at Afkommet ikke vilde blive hans, lod, hver Gang han gik ind til sin Svigerinde, sin Sæd spildes på Jorden for ikke at skaffe sin Broder Afkom.
\par 10 Denne hans Adfærd vakte HERRENs Mishag, derfor lod han også ham dø.
\par 11 Da sagde Juda til sin Sønnekone Tamar: "Bliv som Enke i din Faders Hus, til min Søn Sjela bliver voksen!" Thi han var bange for, at han også skulde dø ligesom sine Brødre. Så gik Tamar hen og blev i sin Faders Hus.
\par 12 Lang Tid efter døde Judas Hustru, Sjuas Datter; og da Juda var hørt op at sørge over hende, rejste han med sin Ven, Hira fra Adullam, up til dem, der klippede hans Får i Timna.
\par 13 Og da Tamar fik at vide, at hendes Svigerfader var på Vej op til Fåreklipningen i Timna,
\par 14 aflagde hun sine Enkeklæder, hyllede sig i et Slør, så det skjulte hende, og satte sig ved indgangen til Enajim ved Vejen til Timna; thi hun så, at hun ikke blev givet Sjela til Ægte, skønt han nu var voksen.
\par 15 Da nu Juda så hende, troede han, det var en Skøge; hun havde jo tilhyllet sit Ansigt;
\par 16 og han bøjede af fra Vejen og kom hen til hende og sagde: "Lad mig gå ind til dig!" Thi han vidste ikke, at det var hans Sønnekone. Men hun sagde: "Hvad giver du mig derfor!"
\par 17 Han svarede: "Jeg vil sende dig et Gedekid fra Hjorden!" Da sagde hun: "Ja, men du skal give mig et Pant, indtil du sender det!"
\par 18 Han spurgte: "Hvad skal jeg give dig i Pant?" Hun svarede: "Din Seglring, din Snor og din Stav, som du har i Hånden!" Så gav han hende de tre Ting og gik ind til hende, og hun blev frugtsommelig ved ham.
\par 19 Derpå gik hun bort, tog Sløret af og iførte sig sine Enkeklæder.
\par 20 Imidlertid sendte Juda sin Ven fra Adullam med Gedekiddet for at få Pantet tilbage fra Kvinden; men han fandt hende ikke.
\par 21 Han spurgte da Folkene på Stedet: "Hvor er den Skøge, som sad på Vejen ved Enajim?" Og de svarede: "Her har ikke været nogen Skøge!"
\par 22 Så vendte han tilbage til Juda og sagde: "Jeg fandt hende ikke, og Folkene på Stedet siger, at der har ikke været nogen Skøge."
\par 23 Da sagde Juda: "Så lad hende beholde det, hellere end at vi skal blive til Spot; jeg har nu sendt det Kid, men du fandt hende ikke."
\par 24 En tre Måneders Tid efter meldte man Juda: "Din Sønnekone Tamar har øvet Utugt og er blevet frugtsommelig!" Da sagde Juda: "Før hende ud, for at hun kan blive brændt!"
\par 25 Men da hun førtes ud, sendte hun Bud til sin Svigerfader og lod sige: "Jeg er blevet frugtsommelig ved den Mand, som ejer disse Ting." Og hun lod sige: "Se dog efter, hvem der ejer denne Ring, denne Snor og denne Stav!"
\par 26 Da Juda havde set efter, sagde han: "Retten er på hendes Side og ikke på min, fordi jeg ikke gav hende til min Søn Sjela!" Men siden havde han ikke Omgang med hende.
\par 27 Da Tiden kom, at hun skulde føde, se, da var der Tvillinger i hendes Liv.
\par 28 Under Fødselen stak der en Hånd frem, og Jordemoderen tog og bandt en rød Snor om den, idet hun sagde: "Det var ham, der først kom frem."
\par 29 Men han trak Hånden tilbage. og Broderen kom frem; så sagde hun: "Hvorfor bryder du frem? For din Skyld er der sket et Brud.
\par 30 Derefter kom Broderen med den røde snor om Hånden frem, og ham kaldte man Zera.

\chapter{39}

\par 1 Da Josef var bragt ned til Ægypten, blev han af Ismaeliterne, der havde bragt ham derned, solgt til Faraos Hofmand Potifar, Livvagtens Øverste, en Ægypter.
\par 2 Men HERREN var med Josef, så Lykken fulgte ham. Han var i sin Herre Ægypterens Hus;
\par 3 og hans Herre så, at HERREN var med ham, og at HERREN lod alt, hvad han foretog sig, lykkes for ham.
\par 4 Således fandt Josef Nåde for hans Øjne og kom til at gå ham til Hånde; og han satte ham over sit Hus og gav alt, hvad han ejede, i hans Hånd;
\par 5 og fra det Øjeblik han satte ham over sit Hus og alt, hvad han ejede, velsignede HERREN Ægypterens Hus for Josefs Skyld, og HERRENs Velsignelse hvilede over alt, hvad han ejede, både inde og ude;
\par 6 og han betroede alt, hvad han ejede, til Josef, og selv bekymrede han sig ikke om andet end den Mad, han spiste. Men Josef havde en smuk Skikkelse og så godt ud.
\par 7 Nu hændte det nogen Tid derefter, at hans Herres Hustru kastede sine Øjne på Josef og sagde: "Kom og lig hos mig!"
\par 8 Men han vægrede sig og sagde til sin Herres Hustru: "Se, min Herre bekymrer sig ikke om noget i Huset, men alt, hvad han ejer, har han givet i min Hånd;
\par 9 han har ikke større Magt i Huset end jeg, og han har ikke unddraget mig noget som helst undtagen dig, fordi du er hans Hustru hvor skulde jeg da kunne øve denne store Misgerning og synde mod Gud!"
\par 10 Og skønt hun Dag efter Dag talte Josef til, vilde han dog ikke føje hende i at ligge hos hende og have med hende at gøre.
\par 11 Men en Dag han kom ind i Huset for at gøre sin Gerning, og ingen af Husfolkene var til Stede i Huset,
\par 12 greb hun fat i hans Kappe og sagde: "Kom og lig hos mig!" Men han lod Kappen blive i hendes Hånd og flygtede ud af Huset.
\par 13 Da hun nu så, at han havde ladet hende beholde Kappen og var flygtet ud af Huset,
\par 14 kaldte hun på sine Husfolk og sagde til dem: "Her kan I se! Han har bragt os en Hebræer til at drive Spot med os! Han kom ind til mig og vilde ligge hos mig, men jeg råbte af alle Kræfter,
\par 15 og da han hørte, at jeg gav mig til at råbe, lod han sin Kappe blive hos mig og flygtede ud af Huset!"
\par 16 Så lod hun Kappen blive liggende hos sig, indtil hans Herre kom hjem,
\par 17 og sagde så det samme til ham: "Den hebraiske Træl, du bragte os til at drive Spot med os, kom ind til mig;
\par 18 men da jeg gav mig til at råbe, lod han sin Kappe blive hos mig og flygtede ud af Huset."
\par 19 Da hans Herre hørte sin Hustrus Ord: "Således har din Træl behandlet mig!" blussede Vreden op i ham;
\par 20 og Josefs Herre tog ham og kastede ham i Fængsel der, hvor Kongens Fanger sad fængslet. Således kom Josef i Fængsel.
\par 21 Men HERREN var med Josef og skaffede ham Yndest og lod ham finde Nåde hos Fængselets Overopsynsmand,
\par 22 så at han gav ham Opsyn over alle Fangerne i Fængselet, og han sørgede for alt, hvad der skulde gøres der.
\par 23 Fængselets Overopsynsmand førte ikke Tilsyn med noget som helst af, hvad der var lagt i Josefs Hånd, eftersom HERREN var med ham og lod alt, hvad han foretog sig, lykkes.

\chapter{40}

\par 1 Nogen Tid efter hændte det, at Ægypterkongens Mundskænk og Bager forbrød sig mod deres Herre Ægypterkongen,
\par 2 og Farao vrededes på sine to Hofmænd, Overmundskænken og Overbageren,
\par 3 og lod dem sætte i Forvaring i Livvagtens Øverstes Hus, i samme Fængsel, hvor Josef sad fængslet;
\par 4 og Livvagtens Øverste gav dem Josef til Opvartning, og han gik dem til Hånde. Da de nu havde været i Forvaring en Tid lang,
\par 5 drømte Ægypterkongens Mundskænk og Bager, som sad i Fængselet, samme Nat hver sin Drøm med sin særlige Betydning.
\par 6 Da Josef om Morgenen kom ind til Faraos Hofmænd, der sammen med ham var i Forvaring i hans Herres Hus, og så, at de var nedslåede,
\par 7 spurgte han dem: "Hvorfor ser I så ulykkelige ud i Dag?"
\par 8 Besvarede: "Vi har haft en Drøm, og her er ingen, som kan tyde den." Da sagde Josef til dem: "Er det ikke Guds Sag at tyde Drømme? Fortæl mig det da!"
\par 9 Så fortalte Overmundskænken Josef sin Drøm og sagde: "Jeg så i Drømme en Vinstok for mig;
\par 10 på Vinstokken var der tre Ranker, og næppe havde den sat Skud. før Blomsterne sprang ud, og Klaserne bar modne Druer;
\par 11 og jeg havde Faraos Bæger i Hånden og tog Druerne og pressede dem i Faraos Bæger og rakte Farao det."
\par 12 Da sagde Josef: "Det skal udtydes således: De tre Ranker betyder tre Dage;
\par 13 om tre Dage skal Farao løfte dit Hoved og genindsætte dig i dit Embede, så du atter rækker Farao Bægeret som før, da du var hans Mundskænk.
\par 14 Vilde du nu blot tænke på mig, når det går dig vel, og vise mig Godhed og omtale mig for Farao og således hjælpe mig ud af dette Hus;
\par 15 thi jeg er stjålet fra Hebræernes Land og har heller ikke her gjort noget, de kunde sætte mig i Fængsel for."
\par 16 Da nu Overbageren så, at Josef gav Mundskænken en gunstig Tydning, sagde han til ham: "Jeg havde en lignende Drøm: Se, jeg bar tre Kurve Hvedebrød på mit Hoved.
\par 17 I den øverste Kurv var der alle Hånde Bagværk til Faraos Bord, men Fuglene åd det af Kurven på mit Hoved!"
\par 18 Da sagde Josef: "Det skal udtydes således: De tre Kurve betyder tre Dage;
\par 19 om tre Dage skal Farao løfte dit Hoved og hænge dig op på en Pæl, og Fuglene skal æde Kødet af din Krop!"
\par 20 Tre Dage efter, da det var Faraos Fødselsdag, gjorde han et Gæstebud for alle sine Tjenere, og da løftede han Overmundskænkens og Overbagerens, Hoveder iblandt sine Tjenere.
\par 21 Overmundskænken genindsatte han i hans Embede, så han atter rakte Farao Bægeret,
\par 22 og Overbageren lod han hænge, som Josef havde tydet det for dem.
\par 23 Men Overmundskænken tænkte ikke på Josef; han glemte ham.

\chapter{41}

\par 1 To År senere hændte det, at Farao havde en drøm. Han drømte, at han stod ved Nilen;
\par 2 og se, op af Floden steg der syv smukke og fede Køer, som gav sig til at græsse i Engen;
\par 3 efter dem steg der syv andre Køer op af Nilen, usle at se til og magre, og de stillede sig ved Siden af de første Køer på Nilens Bred;
\par 4 og de usle og magre Køer åd de syv smukke og fede Køer. Så vågnede Farao.
\par 5 Men han sov ind og havde en Drøm og så syv tykke og gode Aks skyde frem på et og samme Strå;
\par 6 men efter dem volksede der syv golde og vindsvedne Aks frem;
\par 7 og de golde Aks slugte de syv tykke og fulde Aks. Så vågnede Farao, og se, det var en Drøm.
\par 8 Men om Morgenen var hans Sind uroligt; og han sendte Bud efter alle Ægyptens Tegnsudlæggere og Vismænd og fortalte dem sin Drøm, men ingen kunde tyde den for Farao.
\par 9 Da sagde Overmundskænken til Farao: "Jeg må i Dag minde om mine Synder.
\par 10 Den Gang Farao vrededes på sine Tjenere og lod dem sætte i Forvaring i Livvagtens Øverstes Hus, mig og Overbageren,
\par 11 da drømte vi engang samme Nat hver en Drøm med sin særlige Betydning.
\par 12 Sammen med os var der en Hebraisk Yngling, som var Træl hos Livvagtens Øverste, og da vi fortalte ham vore Drømme, tydede han dem for os, hver på sin Måde;
\par 13 og som han tydede dem for os, således gik det: Jeg blev indsat i mit Embede, og Bageren blev hængt."
\par 14 Da sendte Farao Bud efter Josef, og man fik ham hurtigt ud af Fangehullet; og efter at have ladet sig rage og skiftet Klæder fremstillede han sig for Farao.
\par 15 Så sagde Farao til Josef: "Jeg har haft en Drøm, som ingen kan tyde; og nu har jeg hørt om dig, at du kun behøver at høre en Drøm, så kan du tyde den."
\par 16 Josef svarede Farao: "Ikke jeg men Gud vil give Farao et gunstigt Svar!"
\par 17 Da sagde Farao til Josef: "Jeg drømte, at jeg stod på Nilens Bred;
\par 18 og se, op af Floden steg der syv fede og smukke Køer, som gav sig til at græsse i Engen;
\par 19 efter dem steg der syv andre Køer op, ringe, såre usle og magre, så usle Dyr har jeg ikke set nogensteds i Ægypten;
\par 20 og de magre og usle Køer åd de syv første, fede Køer;
\par 21 men da de havde slugt dem, var det ikke til at kende på dem; de så lige så usle ud som før. Så vågnede jeg.
\par 22 Men jeg sov atter ind og så i Drømme syv fulde og gode Aks skyde frem på et og samme Strå;
\par 23 men efter dem voksede der syv udtørrede, golde og vindsvedne Aks frem,
\par 24 og de golde Aks slugte de syv gode Aks. Det fortalte jeg mine Tegnsudlæggere, men ingen kunde forklare mig det."
\par 25 Da sagde Josef til Farao: "Faraos Drømme betyder begge det samme, og Gud har kundgjort Farao, hvad han vil gøre.
\par 26 De syv gode Køer betyder syv År; de syv gode Aks betyder ligeledes syv År; det er en og samme Drøm.
\par 27 Og de syv magre og usle Køer, der steg op efter dem, betyder syv År, og de syv golde og vindsvedne Aks betyder syv Hungersnødsår.
\par 28 Det var det, jeg mente, når jeg sagde til Farao: Hvad Gud vil gøre, har han ladet Farao skue!
\par 29 Se, der kommer syv År med stor Overflod i hele Ægypten;
\par 30 men efter dem kommer der syv Hungersnødsår, og man skal gemme al Overfloden i Ægypten; og Hungersnøden skal hærge Jorden,
\par 31 så man intet mærker til Overfloden på Jorden på Grund af den påfølgende Hungersnød; thi den bliver såre hård.
\par 32 Men at Drømmen gentog sig to Gange for Farao, betyder, at Sagen er fast besluttet af Gud, og at han snart vil lade det ske.
\par 33 Men nu skulde Farao udse sig en indsigtsfuld og klog Mand og sætte ham over Ægypten,
\par 34 og Farao skulde tage og indsætte Tilsynsmænd over Landet og opkræve Femtedelen af Ægyptens Afgrøde i Overflodens syv År;
\par 35 og de skal samle al Afgrøden fra de gode År, der kommer, og oplagre Høsten som Faraos Eje og bringe Afgrøden under Lås og Lukke i Byerne,
\par 36 for at Afgrøden kan tjene til Forråd for Landet i Hungersnødens syv År, som skal komme over Ægypten, at ikke Landet skal gå til Grunde ved Hungersnøden."
\par 37 Både Farao og alle hans Tjenere syntes godt om den Tale,
\par 38 og Farao sagde til sine Tjenere: "Hvor finder vi en Mand, i hvem Guds Ånd er som i ham?"
\par 39 Og Farao sagde til Josef: "Efter at Gud har åbenbaret dig alt dette. kan ingen måle sig med dig i Indsigt og Kløgt;
\par 40 du skal forestå mit Hus, og efter dit Ord skal alt mit Folk rette sig; kun Tronen vil jeg have forud for dig."
\par 41 Og Farao sagde til Josef: "Så sætter jeg dig nu over hele Ægypten!"
\par 42 Og Farao tog Seglringen af sin Hånd og satte den på Josefs, klædte ham i fine Linnedklæder og hængte Guldkæden om hans Hals:
\par 43 han lod ham køre i sin næstbedste Vogn, og de råbte Abrek for ham. Således satte han ham over hele Ægypten.
\par 44 Og Farao sagde til Josef: "Jeg er Farao, men uden dit Minde skal ingen røre Hånd eller Fod nogensteds i Ægypten!"
\par 45 Derpå gav Farao Josef Navnet Zafenat Panea, og han lod ham ægte Asenat, en Datter af Præsten Potifera i On; og Josef drog omkring i Ægypten.
\par 46 Josef var tredive År gammel. da han stededes for Farao, Ægyptens Konge. Så forlod Josef Farao og drog omkring i hele Ægypten.
\par 47 Og Landet bar i bugnende Fylde i Overflodens syv År;
\par 48 og Josef samlede al Afgrøden i de syv År, i hvilke der var Overflod i Ægypten, og bragte den til Byerne; i hver By samlede han Afgrøden fra Markerne der omkring.
\par 49 Således ophobede Josef Korn i vældig Mængde, som Havets Sand, indtil man opgav at måle det, da det ikke var til at måle.
\par 50 Før Hungersnødens År kom. fik Josef to Sønner med Asenat, Præsten i On Potiferas Datter;
\par 51 og Josef gav den førstefødte Navnet Manasse, thi han sagde: "Gud har ladet mig glemme al min Møje og hele min Faders Hus."
\par 52 Og den anden gav han Navnet Efraim, thi han sagde: "Gud har givet mig Livsfrugt i min Elendigheds Land."
\par 53 Da Overflodens syv År, som kom over Ægypten, var omme,
\par 54 begyndte Hungersnødens syv År, som Josef havde sagt; og der opstod Hungersnød i alle Lande, men i hele Ægypten var der Brød.
\par 55 Så hungrede hele Ægypten; og Folket råbte til Farao om Brød; men Farao sagde til alle Ægypterne: "Gå til Josef og gør, hvad han siger eder!"
\par 56 Og der var Hungersnød over hele Jorden. Da åbnede Josef for alle Kornlagrene og solgte Korn til Ægypterne; men Hungersnøden tog til i Ægypten;
\par 57 og Alverden kom til Ægypten for at købe Korn hos Josef; thi Hungersnøden tog til over hele Jorden.

\chapter{42}

\par 1 Da Jakob hørte, at der var Korn at få i Ægypten, sagde han til sine Sønner: "Hvad venter I efter?"
\par 2 Og han sagde: "Jeg hører, at der er Korn at få i Ægypten; drag ned og køb os noget, at vi kan blive i Live og undgå Døden!"
\par 3 Så drog de ti af Josefs Brødre ned for at købe Korn i Ægypten;
\par 4 men Jakob sendte ikke Josefs Broder Benjamin med hans Brødre, thi han tænkte, der kunde tilstøde ham en Ulykke.
\par 5 Blandt dem, der kom for at købe Korn, var også Israels Sønner; thi der var Hungersnød i Kana'ans Land.
\par 6 Og da Josef var Hersker i Landet, og han var den, der solgte Korn til alt Folket i Landet, så kom Josefs Brødre og kastede sig til Jorden for ham.
\par 7 Da Josef så sine Brødre, kendte han dem; men han lod fremmed over for dem, talte dem hårdt til og sagde til dem: "Hvorfra kommer I?" De svarede: "Fra Kana'ans Land for at købe Føde!"
\par 8 Josef kendte sine Brødre, men de kendte ikke ham.
\par 9 Da kom Josef de Drømme i Hu. han havde drømt om dem; og han sagde til dem; "I er Spejdere, I kommer for at se, hvor Landet er åbent!"
\par 10 De svarede: "Nej, Herre, dine Trælle kommer for at købe Føde!
\par 11 Vi er alle Sønner af en og samme Mand; vi er ærlige Folk.
\par 12 Men han sagde: "Jo vist så! I kommer for at se, hvor Landet er åbent!"
\par 13 De svarede: "Vi, dine Trælle. var tolv Brødre, Sønner af en og samme Mand i Kana'ans Land; den yngste er for Tiden hjemme hos vor Fader, og een er ikke mere!"
\par 14 Men Josef sagde til dem: "Det er, som jeg siger eder: I er Spejdere!
\par 15 Men nu skal I sættes på Prøve: Så sandt Farao lever, slipper I ikke herfra, uden at eders yngste Broder kommer hid!
\par 16 Lad en af eder rejse hjem for at hente eders Broder, og imedens holdes I andre fangne; så vil det. vise sig, om eders Ord er Sandhed; og hvis ikke, så er I, så sandt, Farao lever, Spejdere!"
\par 17 Derpå holdt han dem i Forvaring tre Dage.
\par 18 Men Tredjedagen sagde Josef til dem: "Vil I beholde Livet, så skal I gøre således, thi jeg er en Mand, der frygter Gud:
\par 19 Er I virkelig ærlige Folk, lad så en af eder blive tilbage som Fange i det Fængsel, som I sad i, medens I andre drager hjem med Korn til at stille Hungeren i eders Huse;
\par 20 og bring så eders yngste Broder til mig, så skal eders Ord stå til Troende, og I skal slippe for at dø!" Og således gjorde de.
\par 21 Da sagde de til hverandre: "Sandelig, nu må vi bøde for, hvad vi forbrød mod vor Broder, da vi så hans Sjælevånde, medens han bønfaldt os, og dog ikke hørte ham; derfor stedes vi nu i denne Vånde!"
\par 22 Men Ruben tog til Orde og sagde til dem: "Sagde jeg eder ikke dengang: Forsynd eder ikke mod Drengen! Men I vilde ikke høre; se, nu kræves hans Blod!"
\par 23 Således talte de. Men de vidste ikke, at Josef forstod det, thi han forhandlede med dem ved Tolk;
\par 24 og han vendte sig bort fra dem og græd. Siden vendte han sig til dem og talte med dem; og han tog Simeon fra dem og lod ham fængsle for deres Øjne.
\par 25 Så gav Josef Befaling til at fylde deres Sække med Korn og lægge Pengene tilbage i hver enkelts Sæk og give dem Rejsekost; og således skete det.
\par 26 Så læssede de deres Korn på Æslerne og drog bort.
\par 27 Men da en af dem i Natteherberget åbnede sin Sæk for at give sit Æsel Foder, fik han Øje på sine Penge, der lå oven i Sækken;
\par 28 og han sagde til sine Brødre: "Mine Penge er kommet igen; se, de er i min Sæk!" Da sank Hjertet i Livet på dem, og de så forfærdede på hverandre og sagde: "Hvad har Gud dog gjort imod os!"
\par 29 Og da de kom hjem til deres Fader Jakob i Kana'ans Land, fortalte de ham alt, hvad der var hændet dem, og sagde:
\par 30 "Manden, der er Landets Herre, talte os hårdt til og holdt os i Forvaring, som var vi Folk, der vilde udspejde Landet;
\par 31 men vi sagde til ham: Vi er ærlige Folk og ikke Spejdere;
\par 32 vi var tolv Brødre, Sønner af en og samme Fader; een er ikke mere, og den yngste er for Tiden hjemme hos vor Fader i Kana'ans Land.
\par 33 Så sagde Manden, der er Landets Herre: Derpå vil jeg kende, at I er ærlige Folk: Lad en af eder blive hos mig, og tag I andre Korn til at stille Hungeren i eders Huse og drag hjem!
\par 34 Siden skal I bringe eders yngste Broder til mig, for at jeg kan kende, at I ikke er Spejdere, men ærlige Folk; så vil jeg udlevere eders Broder til eder, og I kan frit rejse i Landet."
\par 35 Da de nu tømte deres Sække, fandt hver sin Pengepose i sin Sæk; og da de og deres Fader så Pengeposerne, blev de forfærdede.
\par 36 Og deres Fader Jakob sagde til dem: "I gør mig barnløs; Josef er ikke mere, og Simeon er ikke mere, og nu vil I tage Benjamin; det går alt sammen ud over mig!"
\par 37 Så sagde Ruben til sin Fader: "Du må tage mine to Sønners Liv, hvis jeg ikke bringer ham til dig; betro ham til mig, og jeg skal bringe ham tilbage til dig!"
\par 38 Men han sagde: "Min Søn skal ikke rejse derned med eder, thi hans Broder er død, og han alene er tilbage; tilstøder der ham en Ulykke på den Rejse, I har for, så bringer I mine grå Hår ned i Dødsriget med Sorg!"

\chapter{43}

\par 1 Men Hungersnøden var hård i Landet;
\par 2 og da de havde fortæret det Korn, de havde hentet i Ægypten, sagde deres Fader til dem: "Køb os igen lidt Føde!"
\par 3 Men Juda svarede ham: "Manden sagde os ganske afgjort: I bliver ikke stedt for mit Åsyn, medmindre eders Broder er med!
\par 4 Hvis du derfor vil sende vor Broder med os, vil vi rejse ned og købe dig Føde;
\par 5 men sender du ham ikke med, så rejser vi ikke derned, thi Manden sagde til os: I bliver ikke stedt for mit Åsyn, medmindre eders Broder er med!"
\par 6 Så sagde Israel: "Hvorfor handlede I ilde imod mig og fortalte Manden, at I havde en Broder til?"
\par 7 De svarede: "Manden spurgte os nøje ud om os og vor Slægt og sagde: Lever eders Fader endnu? Har I en Broder til? Og vi svarede ham på hans Spørgsmål; kunde vi vide, at han vilde sige: Bring eders Broder herned!"
\par 8 Men Juda sagde til sin Fader Israel: "Send dog Drengen med mig, så vi kan komme af Sted og blive i Live og undgå Døden, både vi og du og vore Børn!
\par 9 Jeg svarer for ham, af min Hånd må du kræve ham: bringer jeg ham ikke til dig og stiller ham for dit Åsyn, vil jeg være din Skyldner for bestandig;
\par 10 havde vi nu ikke spildt Tiden, kunde vi have været tilbage to Gange!"
\par 11 Så sagde deres Fader Israel til dem: "Kan det ikke være anderledes, gør da i alt Fald således: Tag noget af det bedste, Landet frembringer, med i eders Sække og bring Manden en Gave, lidt Mastiksbalsam, lidt Honning, Tragakantgummi, Cistusharpiks, Pistacienødder og Mandler;
\par 12 og tag dobbelt så mange Penge med, så I bringer de Penge tilbage, som var lagt oven i eders Sække; måske var det en Fejltagelse;
\par 13 og tag så eders Broder og drag atter til Manden!
\par 14 Gud den Almægtige lade eder finde Barmhjertighed hos Manden, så han lader eders anden Broder og Benjamin fare - men skal jeg være barnløs, så lad mig da blive det!"
\par 15 Så tog Mændene deres Gave og dobbelt så mange Penge med; også Benjamin tog de med, brød op og drog ned til Ægypten, hvor de fremstillede sig for Josef.
\par 16 Da Josef så Benjamin iblandt dem, sagde han til sin Hushovmester: "Bring de Mænd ind i mit Hus, lad slagte og lave til, thi de skal spise til Middag hos mig."
\par 17 Manden gjorde, som Josef bød,. og førte Mændene ind i Josefs Hus.
\par 18 Men Mændene blev bange, da de førtes ind i Josefs Hus, og sagde: "Det er for de Penges Skyld, der forrige Gang kom tilbage i vore Sække, at vi føres herind, for at de kan vælte sig ind på os og kaste sig over os, gøre os til Trælle og tage vore Æsler."
\par 19 Derfor trådte de hen til Josefs Hushovmester ved Døren til Huset
\par 20 og sagde: "Hør os, Herre! Vi drog en Gang før herned for at købe Føde,
\par 21 og da vi kom til vort Natteherberge og åbnede vore Sække, se. da lå vore Penge oven i hver enkelts Sæk, vore Penge til sidste Hvid. Men nu har vi bragt dem med tilbage
\par 22 og desuden andre Penge for at købe Føde. Vi ved ikke, hvem der har lagt Pengene i vore Sække!"
\par 23 Men han svarede: "Vær ved godt Mod, frygt ikke! Eders Gud og eders Faders Gud har lagt en Skat i eders Sække - eders Penge har jeg modtaget!" Og han førte Simeon ud til dem.
\par 24 Så førte Manden dem ind i Josefs Hus og gav dem Vand til at tvætte deres Fødder og Foder til Æslerne.
\par 25 Og de fremtog deres Gave, før Josef kom hjem ved Middagstid, thi de hørte, at de skulde spise der.
\par 26 Da Josef trådte ind i Huset, bragte de ham den Gave, de havde med, og kastede sig til Jorden for ham.
\par 27 Han hilste på dem og spurgte: "Går det eders gamle Fader vel, ham, I talte om? Lever han endnu?"
\par 28 De svarede: "Det går din Træl, vor Fader, vel; han lever endnu!" Og de bøjede sig og kastede sig til Jorden.
\par 29 Da han så fik Øje på sin kødelige Broder Benjamin, sagde han: "Er det så eders yngste Broder, som I talte til mig om?" Og han sagde: "Gud være dig nådig, min Søn!"
\par 30 Men Josef brød hurtigt af, thi Kærligheden til Broderen blussede op i ham, og han kæmpede med Gråden; derfor gik han ind i sit Kammer og græd der.
\par 31 Men da han havde badet sit Ansigt, kom han ud, og han beherskede sig og sagde: "Sæt Maden frem!"
\par 32 Så blev Maden sat frem særskilt for ham og for dem og for de Ægyptere, der spiste hos ham; thi Ægypterne kan ikke spise sammen med Hebræere, det er dem en Vederstyggelighed.
\par 33 De blev bænket foran ham efter Alder, den førstefødte øverst og den yngste nederst, og Mændene undrede sig og så på hverandre;
\par 34 og han lod dem bringe Mad fra sit eget Bord, og Benjamin fik fem Gange så meget som hver af de andre. Og de drak og blev lystige sammen med ham.

\chapter{44}

\par 1 Derefter befalede han sin Hushovmester: "Fyld Mændenes Sække med Korn, så meget de kan have med sig, og læg hvers Pengesum oven i hans Sæk
\par 2 og læg mit eget Sølvbæger oven i den yngstes Sæk sammen med Pengene for hans Korn!" Og han gjorde, som Josef bød.
\par 3 Da Morgenen gryede, fik Mændene Lov at drage bort med deres Æsler.
\par 4 Men før de var kommet ret langt fra Byen, bød Josef sin Hushovmester: "Sæt efter Mændene, og når du indhenter dem, sig så til dem: Hvorfor har I gengældt godt med ondt?
\par 5 Hvorfor har I stjålet mit Sølvbæger? Det er jo min Herres Mundbæger, som han bruger til at tage Varsler af! Ilde har I handlet ved at gøre således!"
\par 6 Og da han havde indhentet dem, sagde han det til dem.
\par 7 Men de svarede: "Hvor kan min Herre tale således? Det være langt fra dine Trælle at gøre sligt!
\par 8 Se, de Penge, vi fandt oven i vore Sække, bragte vi tilbage til dig fra Kana'ans Land - hvorfor skulde vi da stjæle Guld eller Sølv fra din Herres Hus!
\par 9 Den af dine Trælle, det findes hos, skal dø, og desuden vil vi andre være din Herres Trælle!"
\par 10 Han svarede: "Vel, lad det blive, som I siger: Den, Bægeret findes hos, skal være min Træl, men I andre skal være sagesløse!"
\par 11 Så skyndte de sig at løfte hver sin Sæk ned på Jorden og åbne den,
\par 12 og han undersøgte dem fra den ældstes til den yngstes, og Bægeret blev fundet i Benjamins Sæk.
\par 13 Da sønderrev de deres Klæder, og efter at have læsset Sækkene hver på sit Æsel vendte de tilbage til Byen.
\par 14 Da Juda og hans Brødre kom ind i Josefs Hus, hvor han endnu var, kastede de sig til Jorden for ham;
\par 15 men Josef sagde til dem: "Hvad har I gjort! Ved I ikke, at en Mand som jeg forstår sig på hemmelige Kunster?"
\par 16 Da sagde Juda: "Hvad skal vi svare min Herre, hvad skal vi sige, og hvorledes skal vi retfærdiggøre os? Gud har fundet dine Trælles Brøde! Se, vi er min Herres Trælle, både vi andre og han, Bægeret fandtes hos!"
\par 17 Men han svarede: "Det være langt fra mig at handle således; den, Bægeret fandtes hos, skal være min Træl, men I andre kan i Fred drage hjem til eders Fader."
\par 18 Da trådte Juda hen til ham og sagde: "Hør mig, min Herre, lad din Træl tale et Ord for min Herres Ører og lad ikke Vreden blusse op i dig mod din Træl, thi du er jo som Farao!
\par 19 Min Herre spurgte sine Trælle Har I Fader eller Broder?
\par 20 Og vi svarede min Herre: Ja, vi har en gammel Fader, og der er en Dreng, som blev født i hans Alderdom; en Broder til ham er død, og selv er han den eneste, hans Moder efterlod sig, og hans Fader elsker ham.
\par 21 Så sagde du til dine Trælle: Bring ham med herned til mig, at jeg kan se ham med egne Øjne!
\par 22 Men vi svarede min Herre: Drengen kan ikke forlade sin Fader, thi hans Fader dør, hvis han forlader ham!
\par 23 Så sagde du til dine Trælle: Kommer eders yngste Broder ikke med herned, så bliver I ikke mere stedt for mit Åsyn!
\par 24 Vi rejste så op til din Træl. min Fader, og fortalte ham, hvad min Herre havde sagt.
\par 25 Da vor Fader siden sagde: Rejs atter hen og køb os lidt Føde!
\par 26 svarede vi: Vi kan ikke rejse derned, hvis ikke vor yngste Broder følger med, thi vi bliver ikke stedt for Mandens Åsyn, medmindre vor yngste Broder er med!
\par 27 Så sagde din Træl, min Fader. til os: I ved jo, at min Hustru fødte mig to Sønner;
\par 28 den ene gik bort fra mig, og jeg sagde: Han er sikkerlig revet ihjel! Og jeg har ikke set ham siden;
\par 29 hvis I nu også tager denne fra mig, og der tilstøder ham en Ulykke, bringer I mine grå Hår i Dødsriget med Smerte!
\par 30 Kommer jeg derfor hjem til din Træl, min Fader, uden at Drengen. ved hvem han hænger med hele sin Sjæl, er med,
\par 31 så bliver det hans Død, når han ser, at Drengen ikke er med.
\par 32 Men din Træl skal svare sin Fader for Drengen, og jeg har forpligtet mig til at være hans Skyldner for bestandig, hvis jeg ikke bringer ham til ham;
\par 33 lad derfor din Træl blive tilbage i Drengens Sted som min Herres Træl, men lad Drengen drage hjem med sine Brødre!
\par 34 Thi hvorledes skulde jeg kunne drage hjem til min Fader, når jeg ikke har Drengen med? Jeg vil ikke kunne være Vidne til den Ulykke, der rammer min Fader!"

\chapter{45}

\par 1 Da kunde Josef ikke længer beherske sig over for alle dem der stod hos ham, og han råbte "Lad alle gå ud!" Således var der ingen til Stede, da Josef gav sig til Kende for sine Brødre.
\par 2 Så brast han i lydelig Gråd, så Ægypterne hørte det, og det spurgtes i Faraos Hus;
\par 3 og Josef sagde til sine Brødre: "Jeg er Josef! Lever min Fader endnu?" Men hans Brødre kunde ikke svare ham, så forfærdede var de for ham.
\par 4 Så sagde Josef til sine Brødre: "Kom hen til mig!" Og da de kom derhen, sagde Josef: "Jeg er eders Broder Josef, som I solgte til Ægypten;
\par 5 men I skal ikke græmme eder eller være forknytte, fordi I solgte mig herhen, thi Gud har sendt mig forud for eder for at opholde Liv;
\par 6 i to År har der nu været Hungersnød i Landet, og fem År endnu skal der hverken pløjes eller høstes;
\par 7 derfor sendte Gud mig forud for eder, for at I kan få Efterkommere på Jorden, og for at mange hos eder kan reddes og holdes i Live.
\par 8 Og nu, ikke I, men Gud har sendt mig hid, og han har gjort mig til Fader hos Farao og til Herre over hele hans Hus og til Hersker over hele Ægypten.
\par 9 Skynd jer nu hjem til min Fader og sig til ham: Din Søn Josef lader sige: Gud har sat mig til Hersker over hele Ægypten; kom uden Tøven ned til mig
\par 10 og tag Bolig i Gosens Land og bo i min Nærhed med dine Sønner og Sønnesønner, dit Småkvæg og Hornkvæg og alt, hvad du ejer og har;
\par 11 der vil jeg sørge for dit Underhold - thi Hungersnøden vil vare fem År endnu - for at ikke du, dit Hus eller nogen, der hører dig til, skal gå til Grunde!
\par 12 Nu ser I, også min Broder Benjamin, med egne Øjne, at det er mig, der taler til eder;
\par 13 og I skal fortælle min Fader om al min Herlighed i Ægypten og om alt, hvad I har set, og så skal I skynde eder at bringe min Fader herned."
\par 14 Så faldt han grædende sin Broder Benjamin om Halsen, og Benjamin græd i hans Arme.
\par 15 Og han kyssede alle sine Brødre og græd ved deres Bryst; og nu kunde hans Brødre tale med ham.
\par 16 Men det spurgtes i Faraos Hus, at Josefs Brødre var kommet, og Farao og hans Tjenere glædede sig derover:
\par 17 og Farao sagde til Josef: "Sig til dine Brødre: Således skal I gøre: Læs eders Dyr og drag til Kana'ans Land,
\par 18 hent eders Fader og eders Familier og kom hid til mig, så vil jeg give eder det bedste, der er i Ægypten, og I skal nyde Landets Fedme.
\par 19 Byd dem at gøre således: Tag eder Vogne i Ægypten til eders Børn og Kvinder, sæt eders Fader op og kom hid;
\par 20 bryd eder ikke om eders Ejendele, thi det bedste, der er i hele Ægypten, skal være eders!"
\par 21 Det gjorde Israels Sønner så. Og efter Faraos Bud gav Josef dem Vogne og Rejsekost med;
\par 22 hver især gav han dem et Sæt Festklæder, men Benjamin gav han 300 Sekel Sølv og fem Sæt Festklæder;
\par 23 og sin Fader sendte han ti Æsler med det bedste, der var i Ægypten og ti Aseninder med Korn, Brød og Rejsetæring til Faderen.
\par 24 Så tog han Afsked med sine Brødre, og da de drog bort, sagde han til dem: "Kives ikke på Vejen!"
\par 25 Således drog de hjem fra Ægypten og kom til deres Fader Jakob i Kana'ans Land;
\par 26 og de fortalte ham det og sagde: "Josef lever endnu, og han er Hersker over hele Ægypten." Men hans Hjerte blev koldt, thi han troede dem ikke.
\par 27 Så fortalte de ham alt, hvad Josef havde sagt til dem; og da han så Vognene, som Josef havde sendt for at hente ham, oplivedes deres Fader Jakobs Ånd atter;
\par 28 og Israel sagde: "Det er stort, min Søn Josef lever endnu; jeg vil drage hen og se ham, inden jeg dør!"

\chapter{46}

\par 1 Da brød Israel op med alt, hvad han havde, og drog til Be'ersjeba, og han slagtede Ofre for sin Fader Isaks Gud.
\par 2 Men Gud sagde i et Nattesyn til Israel: "Jakob, Jakob!" Og han svarede: "Se, her er jeg!"
\par 3 Da sagde han: "Jeg er Gud, din Faders Gud, vær ikke bange for at drage ned til Ægypten, thi jeg vil gøre dig til et stort Folk der;
\par 4 jeg vil selv drage med dig til Ægypten, og jeg vil også føre dig tilbage, og Josef skal lukke dine Øjne!"
\par 5 Da brød Jakob op fra Be'ersjeba; og Israels Sønner satte deres Fader Jakob og deres Børn og Kvinder på de Vogne, Farao havde sendt til at hente ham på.
\par 6 Og de tog deres Kvæg og al deres Ejendom, som de havde erhvervet sig i Kana'ans Land, og drog til Ægypten, Jakob og alt hans Afkom med ham;
\par 7 således bragte han sine Sønner og Sønnesønner, sine Døtre og Sønnedøtre og alt sit Afkom med sig til Ægypten.
\par 8 Følgende er Navnene på Israels Sønner, der kom til Ægypten, Jakob og hans Sønner: Ruben, Jakobs førstefødte;
\par 9 Rubens Sønner Hanok, Pallu. Hezron og Karmi;
\par 10 Simeons Sønner Jemuel, Jamin, Ohad, Jakin, Zohar og Kana'anæerkvindens Søn Sjaul;
\par 11 Leois Sønner Gerson, Kehat og Merari;
\par 12 Judos Sønner Er, Onan, Sjela, Perez og Zera; men Er og Onan døde i Kana'ans Land. Perezs Sønner var Hezron og Hamul.
\par 13 Issakars Sønner Tola, Pua, Jasjub og Sjimron;
\par 14 Zedulons Sønner Sered, Eloo og Jalel;
\par 15 det var Leos Sønner, som hun fødte Jakob i Paddan-Aram; desuden fødte hun ham Datteren Dina; det samlede Tal på hans Sønner og Døtre var tre og tredive.
\par 16 Gods Sønner Zifjon, Haggi. Sjuni, Ezbon, Eri, Arodi og Areli.
\par 17 Asers Sønner Jimna, Jisjva. Jisjvi og Beria, og deres Søster Sera; og Berias Sønner Heber og Mallkiel;
\par 18 det var Sønnerne af Zilpo, som Laban gav sin Datter Lea, og som fødte Jakob dem, seksten i alt;
\par 19 Rakels, Jakobs Hustrus, Sønner Josef og Benjamin;
\par 20 og Josef fik Børn i Ægypten med Asenat, en Datter af Potifera, Præsten i On: Manasse og Efraim;
\par 21 Benjomins Sønner Bela, Beker, Asjbel, Gera, Nåman, Ebi, Rosj, Muppim, Huppim og Ard;
\par 22 det var Rakels Sønner, som hun fødte Jakob, fjorten i alt;
\par 23 Dons Søn Husjim;
\par 24 Noffolis Sønner Jazeel, Guni, Jezer og Sjillem;
\par 25 det var Sønnerne af Bilha, som Laban gav Rakel, og som fødte Jakob dem, syv i alt.
\par 26 Hele Jakobs Familie, der kom til Ægypten med ham, fraregnet Jakobs Sønnekoner, udgjorde tilsammen seks og tresindstyve;
\par 27 og Josefs Sønner, der fødtes ham i Ægypten, var to; alle de af Jakobs Hus, der kom til Ægypten, udgjorde halvfjerdsindstyve.
\par 28 Men Juda sendte han i Forvejen til Josef, for at man skulde vise ham Vej til Gosen, og de kom til Gosens Land.
\par 29 Da lod Josef spænde for sin Vogn og rejste sin Fader i Møde til Gosen; og da han traf ham, omfavnede han ham og græd længe i hans Arme;
\par 30 og Israel sagde til Josef: "Lad mig nu kun dø, da jeg har set dit Ansigt, at du endnu lever!"
\par 31 Men Josef sagde til sine Brødre og sin Faders Hus: "Jeg vil drage hen og melde det til Farao og sige til ham: Mine Brødre og min Faders Hus i Kana'an er kommet til mig;
\par 32 disse Mænd er Hyrder, thi de driver Kvægavl, og de har bragt deres Småkvæg og Hornkvæg og alt, hvad de ejer, med.
\par 33 Når så Farao lader eder kalde og spørger eder, hvad I er,
\par 34 skal I sige: Dine Trælle har drevet Kvægavl fra Barnsben af ligesom vore Fædre! for at I kan komme til at bo i Gosens Land.

\chapter{47}

\par 1 Så drog Josef hen og meldte det til Farao og sagde: "Min Fader og mine Brødre er kommet fra Kana'ans Land med deres Småkvæg og Hornkvæg og alt, hvad de ejer, og befinder sig i Gosen."
\par 2 Og han tog fem af sine Brødre med sig og forestillede dem for Farao.
\par 3 Da nu Farao spurgte dem, hvad de var, svarede de: "Dine Trælle er Hyrder ligesom vore Fædre!"
\par 4 Og de sagde til Farao: "Vi er kommet for at bo som Gæster i Landet, thi der er ingen Græsning for dine Trælles Småkvæg, da Hungersnøden er trykkende i Kana'an, og nu vilde dine Trælle gerne bosætte sig i Gosen."
\par 5 Da sagde Farao til Josef: "Din Fader og dine Brødre er kommet til dig;
\par 6 Ægypten står til din Rådighed, lad din Fader og dine Brødre bosætte sig i den bedste Del af Landet; de kan tage Ophold i Gosens Land; og hvis du ved, at der er dygtige Folk iblandt dem, kan du sætte dem til Opsynsmænd over mine Hjorde!"
\par 7 Da hentede Josef sin Fader Jakob og forestillede ham for Farao, og Jakob velsignede Farao.
\par 8 Farao spurgte Jakob: "Hvor mange er dine Leveår?"
\par 9 Jakob svarede ham: "Min Udlændigheds År er 130 År; få og onde var mine Leveår, og ikke når de op til mine Fædres År i deres Udlændigheds Tid."
\par 10 Derpå velsignede Jakob Farao og gik bort fra ham.
\par 11 Men Josef lod sin Fader og sine Brødre bosætte sig og gav dem Jordegods i Ægypten, i den bedste Del af Landet, i Landet Rameses, som Farao havde befalet.
\par 12 Og Josef forsørgede sin Fader og sine Brødre og hele sin Faders Hus med Brød efter Børnenes Tal.
\par 13 Der fandtes ikke mere brød Korn i Landet, thi Hungersnøden var overvættes stor, og Ægypten og Kana'an vansmægtede af Sult.
\par 14 Da lod Josef alle de Penge samle, som var indkommet i Ægypten og Kana'an for det Korn, der købtes, og lod dem bringe til Faraos Hus.
\par 15 Men da Pengene slap op i Ægypten og Kana'an, kom hele Ægypten til Josef og sagde: "Giv os Brødkorn, at vi ikke skal dø for dine Øjne, thi Pengene er sluppet op!"
\par 16 Josef svarede: "Kom med eders Hjorde, så vil jeg give eder Brødkorn for dem, siden Pengene er sluppet op."
\par 17 Da bragte de deres Hjorde til Josef, og han gav dem Brødkorn for Hestene, for deres Hjorde af Småkvæg og Hornkvæg og for Æslerne; og han forsørgede dem i det År med Brødkorn for alle deres Hjorde.
\par 18 Men da Året var omme, kom de til ham det følgende År og sagde: "Vi vil ikke dølge det for min Herre; men Pengene er sluppet op, og Kvæget har vi bragt til min Herre; nu er der ikke andet tilbage for min Herre end vore Kroppe og vor Jord;
\par 19 lad os dog ikke dø for dine Øjne, vi sammen med vor Jord, men køb os og vor Jord for Brødkorn, og lad os med vor Jord blive livegne for Farao, og giv os derfor Såsæd, så vi kan blive i Live og slippe for Døden, og vor Jord undgå at lægges øde!"
\par 20 Da købte Josef al Jord i Ægypten til Farao, idet enhver Ægypter solgte sin Mark, fordi Hungersnøden tvang dem, og således kom Landet i Faraos Besiddelse;
\par 21 og Befolkningen gjorde han til hans Trælle i hele Ægypten fra Ende til anden.
\par 22 Kun Præsternes Jord købte han ikke, thi de havde faste Indtægter fra Farao, og de levede af de Indtægter, Farao havde skænket dem; derfor behøvede de ikke at sælge deres Jord.
\par 23 Derpå sagde Josef til Folket: "Nu har jeg købt eder og eders Jord til Farao; nu har I Såsæd til Jorden;
\par 24 men af Afgrøden skal I afgive en Femtedel til Farao, medens de fire Femtedele skal tjene eder til Såsæd på Marken og til Føde for eder og eders Husstand og til Føde for eders Børn."
\par 25 De svarede: "Du har reddet vort Liv; måtte vi eje min Herres Gunst, så vil vi være Faraos Trælle!"
\par 26 Således gjorde Josef det til en Vedtægt, der endnu den Dag i Dag gælder i Ægypten, at afgive en Femtedel til Farao; kun Præsternes Jord kom ikke i Faraos Besiddelse.
\par 27 Israel bosatte sig nu i Ægypten, i Gosens Land; og de tog fast Ophold der, blev frugtbare og såre talrige.
\par 28 Jakob levede i Ægypten i sytten År, så at Jakobs Levetid blev 147 År.
\par 29 Da nu Tiden nærmede sig, at Israel skulde dø, kaldte han sin Søn Josef til sig og sagde til ham: "Hvis jeg har fundet Nåde for dine Øjne, så læg din Hånd under min Lænd og lov mig at vise mig den Kærlighed og Trofasthed ikke at jorde mig i Ægypten.
\par 30 Når jeg har lagt mig til Hvile hos mine Fædre, skal du føre mig fra Ægypten og jorde mig i deres Grav!" Han svarede: "Jeg skal gøre, som du siger."
\par 31 Da sagde han: "Tilsværg mig det!" Og han tilsvor ham det. Da bøjede Israel sig tilbedende over Lejets Hovedgærde.

\chapter{48}

\par 1 Efter disse Begivenheder fik Josef Melding om, at hans Fader var syg. Da tog han sine Sønner, Manasse og Efraim, med sig
\par 2 Da det nu meldtes Jakob, at hans Søn Josef var kommet, tog Israel sig sammen og satte sig oprejst på Lejet
\par 3 Jakob sagde til Josef: "Gud den Almægtige åbenbarede sig for mig i Luz i Kana'ans Land og velsignede mig;
\par 4 og han sagde til mig: Jeg vil gøre dig frugtbar og give dig et talrigt Afkom og gøre dig til en Mængde Stammer, og jeg vil give dit Afkom efter dig Land til evigt Eje!
\par 5 Nu skal dine to Sønner, der er født dig i Ægypten før mit komme til dig her i Ægypten, være mine, Efraim og Manasse skal være mine så godt som Ruben og Simeon;
\par 6 derimod skal de Børn, du har fået efter dem, være dine; men de skal nævnes efter deres Brødres Navne i deres Arvelod
\par 7 Da jeg kom fra Paddan, døde Rakel for mig, medens jeg var undervejs i Kana'an, da vi endnu var et stykke Vej fra Efrat, og jeg jordede hende der på vejen til Efrat, det er Betlehem".
\par 8 Da Israel så Josefs Sønner, sagde han: "Hvem bringer du der?"
\par 9 Josef svarede sin Fader: "Det er mine Sønner, som Gud har skænket mig her." Da sagde han:"Bring dem hen til mig, at jeg kan velsignedem!"
\par 10 Men Israels Øjne var svækkede af Alderdom, så at han ikke kunde se. Da førte han dem hen til ham. og han kyssede og omfavnede dem.
\par 11 Og Israel sagde til Josef: "Jeg: havde ikke turdet håbe at få dit Ansigt at se, og nu har Gud endog: ladet mig se dit Afkom!"
\par 12 Derpå tog Josef dem bort fra hans Knæ og kastede sig til Jorden. på sit Ansigt.
\par 13 Josef tog så dem begge, Efraim i sin højre Hånd til venstre for Israel og Manasse i sin venstre Hånd til højre for Israel, og førte dem hen til ham;
\par 14 men Israel udrakte sin højre Hånd og lagde den på Efraims Hoved, uagtet han var den yngste.. og sin venstre Hånd lagde han på Manasses Hoved, så at han lagde Hænderne over Kors; thi Manasse var den førstefødte.
\par 15 Derpå velsignede han Josef og sagde: "Den Gud, for hvis Åsyn mine Fædre Abraham og Isak vandrede, den Gud, der har vogtet mig: fra min første Færd og til nu,
\par 16 den Engel, der har udløst mig fra alt ondt, velsigne Drengene, så at mit Navn og mine Fædre Abrahams og Isaks Navn må blive nævnet ved dem, og de må vokse i Mængde i Landet!"
\par 17 Men da Josef så, at hans Fader lagde sin højre Hånd på Efraims Hoved, var det ham imod,. og han greb sin Faders Hånd for at tage den bort fra Efraims Hoved og lægge den på Manasses;
\par 18 og Josef sagde til sin Fader: "Nej, ikke således, Fader, thi denne er den førstefødte; læg din højre Hånd på hans Hoved!"
\par 19 Men hans Fader vægrede sig og sagde: "Jeg ved det, min Søn, jeg ved det! Også han skal blive til et Folk, også han skal blive stor; men hans yngre Broder skal blive større end han, og hans Afkom skal blive en Mangfoldighed af Folkeslag!"
\par 20 Således velsignede han dem på den Dag og sagde: "Med eder skal Israel velsigne og sige: Gud gøre dig som Efraim og Manasse!" Og han stillede Efraim foran Manasse.
\par 21 Da sagde Israel til Josef: "Jeg skal snart dø, men Gud skal være med eder og føre eder tilbage til eders Fædres Land.
\par 22 Dig giver jeg ud over dine Brødre en Højderyg, som jeg har fravristet Amonterne med mit Sværd og min Bue!"

\chapter{49}

\par 1 Derpå kaldte Jakob sine Sønner til sig og sagde: "Saml eder, så vil jeg forkynde eder, hvad der skal hændes eder i de sidste Dage:
\par 2 Kom hid og hør, Jakobs Sønner, lyt til eders Fader Israel!
\par 3 Ruben, du er min førstefødte, min Styrke og min Mandskrafts første, ypperst i Højhed, ypperst i Kraft!
\par 4 Du skummer over som Vandet, men du mister din Forret; thi du besteg din Faders Leje.Skændigt handled du da han besteg mit Leje!
\par 5 Simeon og Levi, det Broder Par, Voldsredskaber er deres Våben.
\par 6 I deres Råd giver min Sjæl ej Møde, i deres Forsamling tager min Ære ej Del; thi i Vrede dræbte de Mænd, egenrådigt lamslog de Okser.
\par 7 Forbandet være deres Vrede, så vild den er, deres Hidsighed, så voldsom den er! Jeg spreder dem i Jakob, splitter dem ad i Israel!
\par 8 Juda, dig skal dine Brødre prise, din Hånd skal gribe dine Fjender i Nakken, din Faders Sønner skal bøje sig for dig.
\par 9 En Løveunge er Juda. Fra Rov stiger du op, min Søn! Han ligger og strækker sig som en Løve, ja, som en Løvinde, hvo tør vække ham!
\par 10 Ikke viger Kongespir fra Juda, ej Herskerstav fra hans Fødder, til han, hvem den tilhører; kommer, ham skal Folkene lyde.
\par 11 Han binder sit Æsel ved Vinstokken, ved Ranken Asenindens Fole, tvætter i Vin sin Kjortel, sin Kappe i Drueblod,
\par 12 med Øjnene dunkle af Vin og Tænderne hvide af Mælk!
\par 13 Zebulon har hjemme ved Havets Byst, han bor ved Skibenes Kyst, hans Side er vendt mod Zidon.
\par 14 Issakar, det knoglede Æsel, der strækker sig mellem Foldene,
\par 15 fandt Hvilen sød og Landet lifligt; derfor bøjed han Ryg under Byrden og blev en ufri Træl.
\par 16 Dan dømmer sit Folk så godt som nogen Israels Stamme.
\par 17 Dan blive en Slange ved Vejen, en Giftsnog ved Stien, som bider Hesten i Hælen,så Rytteren styrter bagover!
\par 18 På din Frelse bier jeg, HERRE!
\par 19 Gad, på ham gør Krigerskarer Indhug, men han gør Indhug i Hælene på dem.
\par 20 Aser, hans Føde er fed, Lækkerier for Konger har han at give.
\par 21 Naftali er en løssluppen Hind, han fremfører yndig Tale.
\par 22 Et yppigt Vintræ er Josef, et yppigt Vintræ ved Kilden, Ranker slynger sig over Muren.
\par 23 Bueskytter fejder imod ham, strides med ham, gør Angreb på ham,
\par 24 men hans Bue er stærk, hans Hænders Arme rappe; det kommer fra Jakobs Vældige, fra Hyrden, Israels Klippe,
\par 25 fra din Faders Gud han hjælpe dig!Og Gud den Almægtige, han velsigne dig med Himmelens Velsignelser oventil og Dybets Velsignelser nedentil, med Brysters og Moderlivs Velsignelser!
\par 26 Din Faders Velsignelser overgår de ældgamle Bjerges Velsignelser, de evige Højes Herlighed. Måtte de komme over Josefs Hoved, over Issen på Fyrsten blandt Brødre!
\par 27 Benjamin, den rovlystne Ulv, om Morgenen æder han Rov, om Aftenen deler han Bytte!"
\par 28 Alle disse er Israels Stammer, tolv i Tal, og det var, hvad deres Fader talte til dem, og han velsignede dem, hver især af dem gav han sin særlige Velsignelse.
\par 29 Og han sagde til dem som sin sidste Vilje: "Nu samles jeg til mit Folk; jord mig da hos mine Fædre i Hulen på Hetiten Efrons Mark.
\par 30 i Hulen på Makpelas Mark over for Mamre i Kana'ans Land. den Mark, som Abraham købte af Hetiten Efron til Gravsted,
\par 31 hvor de jordede Abraham og hans Hustru Sara, hvor de jordede Isak og hans Hustru Rebekka, og hvor jeg jordede Lea.
\par 32 Marken og Hulen derpå blev købt af Hetiterne."
\par 33 Dermed havde Jakob givet sine Sønner sin Vilje til Kende, og han strakte sine Fødder ud på Lejet. udåndede og samledes til sin Slægt.

\chapter{50}

\par 1 Da kastede Josef sig over sin Faders Ansigt, græd og kyssede ham;
\par 2 og Josef befalede de lægekyndige blandt sine Tjenere at balsamere hans Fader, og Lægerne balsamerede Israel.
\par 3 Dermed gik fyrretyve Dage, thi så lang Tid tager Balsameringen: og Ægypterne begræd ham i halvfjerdsindstyve Dage.
\par 4 Da Grædetiden var omme, sagde Josef til Faraos Husfolk: "Hvis I har Godhed for mig, så sig på mine Vegne til Farao:
\par 5 Min Fader tog mig i Ed, idet han sagde: Når jeg er død, så jord mig i den Grav, jeg lod mig grave i Kana'ans Land! Lad mig derfor drage op og jorde min Fader og så vende tilbage hertil!"
\par 6 Da sagde Farao: "Drag kun op og jord din Fader, som han har ladet dig sværge."
\par 7 Så drog Josef op for at jorde sin Fader, og med ham drog alle Faraos Tjenere, de ypperste i hans Hus og de ypperste i Ægypten,
\par 8 hele Josefs Hus og hans Brødre og hans Faders Hus, kun deres Kvinder og Børn, Småkvæg og Hornkvæg lod de blive tilbage i Gosen;
\par 9 og med ham fulgte både Stridsvogne og Ryttere, så det blev en overmåde stor Karavane.
\par 10 Da de kom til Gorenhåtad hinsides Jordan, holdt de der en overmåde stor og højtidelig Dødeklage, og han fejrede Sørgefest for sin Fader i syv Dage.
\par 11 Men da Landets Indbyggere, Kana'anæerne, så denne Sørgefest i Gorenhåtad, sagde de: "Ægypterne holder en højtidelig Sørgefest." Derfor gav man det Navnet Abel Mizrajim; det ligger hinsides Jordan.
\par 12 Og hans Sønner gjorde, som han havde pålagt dem;
\par 13 hans Sønner førte ham til Kana'ans Land og jordede ham i Hulen på Makpelas Mark, den Mark, som Abraham havde købt til Gravsted af Hetiten Efron over for Mamre.
\par 14 Efter at have jordet sin Fader vendte Josef tilbage til Ægypten med sine Brødre og alle dem, der var draget op med ham til hans Faders Jordefærd.
\par 15 Da Josefs Brødre så, at deres Fader var død, sagde de: "Blot nu ikke Josef vil vise sig fjendsk mod os og gengælde os alt det onde, vi har gjort ham!"
\par 16 Derfor sendte de Bud til Josef og sagde: "Din Fader pålagde os før sin Død
\par 17 at sige til Josef: Tilgiv dog dine Brødres Brøde og Synd, thi de har gjort ondt imod dig! Så tilgiv nu din Faders Guds Tjenere deres Brøde!" Da græd Josef over deres Ord til ham.
\par 18 Siden kom hans Brødre selv og faldt ham til Fode og sagde: "Se, vi vil være dine Trælle!"
\par 19 Da sagde Josef til dem: "Frygt ikke, er jeg vel i Guds Sted?
\par 20 I tænkte ondt mod mig, men Gud tænkte at vende det til det gode for at gøre, hvad nu er sket, og holde mange Folk i Live;
\par 21 frygt ikke, jeg vil sørge for eder og eders Kvinder og Børn!" Således trøstede han dem og satte Mod i dem.
\par 22 Josef blev nu i Ægypten, både han og hans Faders Hus, og Josef blev 110 År gammel.
\par 23 Josef så Børn i tredje Led af Efraim; også Børn af Manasses Søn Makir fødtes på Josefs Knæ.
\par 24 Derpå sagde Josef til sine Brødre: "Jeg dør snart, men Gud vil se til eder og føre eder fra Landet her til det Land, han tilsvor Abraham, Isak og Jakob."
\par 25 Og Josef tog Israels Sønner i Ed og sagde: "Når Gud ser til eder, skal I føre mine Ben bort herfra!"
\par 26 Josef døde 110 År gammel, og man balsamerede ham og lagde ham i Kiste i Ægypten.



\end{document}