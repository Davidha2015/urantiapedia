\begin{document}

\title{Exodus}


\chapter{1}

\par 1 Dette er Navnene på Israels Sønner, der sammen med Jakob kom til Ægypten med deres Familier:
\par 2 Ruben, Simeon, Levi og Juda,
\par 3 Issakar, Zebulon og Benjamin,
\par 4 Dan og Naftali, Gad og Aser.
\par 5 Jakobs Efterkommere udgjorde i alt halvfjerdsindstyve, men Josef var i Ægypten.
\par 6 Imidlertid døde Josef og alle hans Brødre og hele dette Slægtled.
\par 7 Men Israeliterne var frugtbare og formerede sig, og de blev mange og overmåde talrige, så at Landet blev fuldt af dem.
\par 8 Da kom der en ny Konge over Ægypten, som ikke vidste noget om Josef;
\par 9 og han sagde til sit Folk: "Se, Israels Folk bliver talrigere og stærkere end vi.
\par 10 Velan, lad os gå klogt til Værks imod dem, for at de ikke skal blive for mange; ellers kan det hænde, når vi kommer i Krig, at de går over til vore Modstandere og kæmper mod os og til sidst forlader Landet!"
\par 11 Så satte man Fogeder over dem til at plage dem med Trællearbejde, og de måtte bygge Forrådsbyer for Farao: Pitom og Ra'amses.
\par 12 Men jo mere man plagede dem, des flere blev de, og des mere bredte de sig, så Ægypterne fik Rædsel for Israeliterne.
\par 13 Og Ægypterne tvang Israeliterne til Trællearbejde
\par 14 og gjorde dem Livet bittert ved hårdt Arbejde med Ler og Tegl og alle Hånde Markarbejde, ved alt det Arbejde, de tvang dem til at udføre for sig.
\par 15 Ægypterkongen sagde da til Hebræerkvindernes Jordemødre, af hvilke den ene hed Sjifra, den anden Pua:
\par 16 "Når I forløser Hebræerkvinderne, skal I se godt efter ved Fødselen, og er det et Drengebarn, tag så Livet af det, men er det et Pigebarn, lad det så leve!"
\par 17 Men Jordemødrene frygtede Gud og gjorde ikke, som Ægypterkongen havde befalet dem, men lod Drengebørnene leve.
\par 18 Da lod Ægypterkongen Jordemødrene kalde og sagde til dem: "Hvorfor har I båret eder således ad og ladet Drengebørnene leve?"
\par 19 Men Jordemødrene svarede Farao: "Hebræerkvinderne er ikke som de Ægyptiske Kvinder, de har let ved at føde; inden Jordemoderen kommer til dem, har de allerede født!"
\par 20 Og Gud gjorde vel imod Jordemødrene, og Folket blev stort og såre talrigt;
\par 21 og Gud gav Jordemødrene Afkom, fordi de frygtede ham.
\par 22 Da udstedte Farao den Befaling til hele sit Folk: "Alle Drengebørn, der fødes, skal I kaste i Nilen, men Pigebørnene skal I lade leve!"

\chapter{2}

\par 1 Og en Mand af Levis Hus gik hen og tog en Levi Datter til Ægte,
\par 2 og Kvinden blev frugtsommelig og fødte en Søn. Da hun så, at det var en dejlig Dreng, skjulte hun ham i tre Måneder;
\par 3 og da hun ikke længer kunde holde ham skjult, tog hun en Kiste af Papyrusrør, tættede den med Jordbeg og Tjære, lagde drengen i den og satte den hen mellem Sivene ved Nilens Bred.
\par 4 Og hans Søster stillede sig noget derfra for at se, hvad der vilde ske med ham.
\par 5 Da kom Faraos Datter ned til Nilen for at bade, og imedens gik hendes Jomfruer ved Flodens Bred. Så fik hun Øje på Kisten mellem Sivene og sendte sin Pige hen for at hente den.
\par 6 Og da hun åbnede den, så hun Barnet, og se, det var et Drengebarn, der græd. Da ynkedes hun over det og sagde: "Det må være et af Hebræernes Drengebørn!"
\par 7 Hans Søster sagde nu til Faraos Datter: "Skal jeg gå hen og hente dig en Amme blandt Hebræerkvinderne til at amme Barnet for dig?"
\par 8 Faraos Datter svarede hende: "Ja, gør det!" Så gik Pigen hen og hentede Barnets Moder.
\par 9 Og Faraos Datter sagde til hende: "Tag dette Barn med dig og am ham for mig, jeg skal nok give dig din Løn derfor!" Og Kvinden tog Barnet og ammede ham.
\par 10 Men da Drengen var blevet stor, bragte hun ham til Faraos Datter, og denne antog ham som sin Søn og gav ham Navnet Moses; "thi," sagde hun, "jeg har trukket ham op af Vandet."
\par 11 På den Tid gik Moses, som imidlertid var blevet voksen, ud til sine Landsmænd og så på deres Trællearbejde. Og han så en Ægypter slå en Hebræer, en af hans Landsmænd, ihjel.
\par 12 Da så han sig om til alle Sider, og efter at have forvisset sig om, at der ingen var i Nærheden, slog han Ægypteren ihjel og gravede ham ned i Sandet.
\par 13 Da han den næste Dag igen gik derud, så han to Hebræere i Slagsmål med hinanden. Da sagde han til ham, der havde Uret: "Hvorfor slår du din Landsmand?"
\par 14 Han svarede: "Hvem har sat dig til Herre og Dommer over os? Vil du måske slå mig ihjel, ligesom du slog Ægypteren ihjel?" Og Moses blev bange og tænkte: "Så er det dog blevet bekendt!"
\par 15 Da Farao fik Nys derom, søgte han at komme Moses til Livs, men Moses flygtede for Farao og tyede til Midjans Land, og der satte han sig ved en Brønd.
\par 16 Præsten i Midjan havde syv Døtre; de kom nu hen og øste Vand og fyldte Trugene for at vande deres Faders Småkvæg.
\par 17 Da kom Hyrderne og vilde jage dem bort, men Moses stod op og hjalp dem og vandede deres Småkvæg.
\par 18 Da de nu kom hjem til deres Fader Reuel, sagde han: "Hvorfor kommer I så tidligt hjem i Dag?"
\par 19 De svarede: "Der var en Ægypter, som hjalp os over for Hyrderne, ja han øste også Vand for os og vandede Småkvæget."
\par 20 Da sagde han til sine Døtre: "Hvor er han da? Hvorfor har I ladet Manden blive derude? Byd ham ind, at han kan få noget at spise!"
\par 21 Så bestemte Moses sig til at tage Ophold hos Manden, og han gav Moses sin Datter Zippora til Ægte,
\par 22 og hun fødte en Søn, som han kaldte Gersom; "thi," sagde han, "jeg er blevet Gæst i et fremmed Land."
\par 23 Således gik der lang Tid hen, og imidlertid døde Ægypterkongen. Men Israeliterne stønnede og klagede under deres Trældom, og deres Skrig over Trældommen nåede op til Gud.
\par 24 Da hørte Gud deres Jamren, og Gud ihukom sin Pagt med Abraham, Isak og Jakob,
\par 25 og Gud så til Israeliterne, og Gud kendtes ved dem.

\chapter{3}

\par 1 Moses vogtede nu Småkvæget for sin Svigerfader Jetro, Præsten i Midjan,og drev engang Småkvæget hen hinsides Ørkenen og kom til Guds Bjerg Horeb.
\par 2 Da åbenbarede HERRENs Engel sig for ham i en Ildslue, der slog ud af en Tornebusk, og da han så nærmere til, se, da stod Tornebusken i lys Lue, uden at den blev fortæret.
\par 3 Da sagde Moses: "Lad mig gå hen og se på dette underfulde Skue, hvorfor Tornebusken ikke brænder op."
\par 4 Men da HERREN så, at han gik hen for at se derpå, råbte Gud til ham fra Tornebusken: "Moses, Moses!" Og han svarede: "Se, her er jeg!"
\par 5 Da sagde han: "Kom ikke nærmere! Drag dine Sko af dine Fødder, thi det Sted, du står på, er hellig Jord!"
\par 6 Og han sagde: "Jeg er din Faders Gud, Abrahams Gud, Isaks Gud og Jakobs Gud." Da skjulte Moses sit Ansigt, thi han frygtede for at skue Gud.
\par 7 Derpå sagde HERREN: "Jeg har set mit Folks Elendighed i Ægypten, og jeg har hørt deres Klageskrig over deres Undertrykkere, ja, jeg kender deres Lidelser;
\par 8 og jeg er steget ned for at udfri dem af Ægyptens Hånd og føre dem bort fra dette Land til et godt og vidtstrakt Land, til et Land, der flyder med Mælk og Honning, til Kana'anæernes, Hetiternes, Amoriternes, Perizzitemes, Hivviternes og Jebusiternes Egn.
\par 9 Se, nu er Israeliternes Klageskrig nået til mig, og jeg har også set den Trængsel, Ægypterne har bragt over dem.
\par 10 Derfor vil jeg nu sende dig til Farao, og du skal føre mit Folk, Israeliterne, ud af Ægypten!"
\par 11 Men Moses sagde til Gud: "Hvem er jeg, at jeg skulde kunne gå til Farao og føre Israeliterne ud af Ægypten?"
\par 12 Han svarede: "Jo, jeg vil være med dig! Og dette skal være dig Tegnet på, at det er mig, der har sendt dig: Når du har ført Folket ud af Ægypten, skal I dyrke Gud på dette Bjerg!"
\par 13 Men Moses sagde til Gud: "Når jeg kommer til Israeliterne og siger dem, at deres Fædres Gud har sendt mig til dem, hvad skal jeg så svare dem, hvis de spørger om hans Navn7"
\par 14 Gud svarede Moses: "Jeg er den, jeg er!" Og han sagde: "Således skal du sige til Israeliterne: JEG ER har sendt mig til eder!"
\par 15 Og Gud sagde fremdeles til Moses: "Således skal du sige til Israeliterne: HERREN, eders Fædres Gud, Abrahams Gud, Isaks Gud og Jakobs Gud, har sendt mig til eder; dette er mit Navn til evig Tid, og således skal jeg kaldes fra Slægt til Slægt.
\par 16 Gå nu hen og kald Israels Ældste sammen og sig til dem: HERREN, eders Fædres Gud, Abrahams, Isaks og Jakobs Gud, har åbenbaret sig for mig og sagt: Jeg har givet Agt på eder og på, hvad man har gjort imod eder i Ægypten,
\par 17 og jeg har sat mig for at føre eder ud af Ægyptens Elendighed til Kana'anæernes, Hetiternes, Amoriternes, Perizziternes, Hivviternes og Jebusiternes Land, til et Land, der flyder med Mælk og Honning!
\par 18 De vil høre på dig, og du skal sammen med Israels Ældste gå til Ægypterkongen, og I skal sige til ham: HERREN, Hebræernes Gud, har mødt os, tillad os derfor at drage tre Dagsrejser ud i Ørkenen og ofre til HERREN vor Gud!
\par 19 Jeg ved vel, at Ægypterkongen ikke vil tillade eder at drage bort uden med Magt;
\par 20 men jeg skal udrække min Hånd og ramme Ægypten med alle mine Undergerninger, som jeg vil gøre der; så skal han give eder Lov til at drage af Sted.
\par 21 Og jeg vil stemme Ægypterne gunstigt mod dette Folk, så at I, når I drager bort, ikke skal drage bort med tomme Hænder.
\par 22 Enhver Kvinde skal bede sin Naboerske og de Kvinder, som er til Huse hos hende, om Sølv og Guldsmykker og Klæder, og I skal give eders Sønner og Døtre det på. Således skal I tage Bytte fra Ægypterne."

\chapter{4}

\par 1 Moses svarede; "Hvis de nu ikke tror mig og ikke hører mig, men siger, at HERREN ikke har åbenbaret sig for mig?"
\par 2 Da sagde HERREN til ham: "Hvad har du der i din Hånd?" Han svarede: "En Stav!"
\par 3 Og han sagde: "Kast den til Jorden!" Da kastede han den til Jorden, og den blev til en Slange, og Moses flyede for den.
\par 4 Og HERREN sagde til Moses: "Ræk Hånden ud og grib den i Halen!" Da rakte han sin Hånd ud, og den blev til en Stav i hans Hånd.
\par 5 "For at de nemlig kan tro, at HERREN, deres Fædres Gud, Abrahams Gud, Isaks Gud og Jakobs Gud, har åbenbaret sig for dig."
\par 6 Og HERREN sagde fremdeles til ham: "Stik din Hånd ind på Brystet!" Da stak han sin Hånd ind på Brystet, og da han trak den ud, se, da var den hvid som Sne af Spedalskhed.
\par 7 Derpå sagde han: "Stik atter Hånden ind på Brystet!" Så stak han atter Hånden ind på Brystet, og da han trak den ud, se, da var den igen som hans øvrige Legeme.
\par 8 "Hvis de nu ikke tror dig og lader sig overbevise af det første Tegn, så vil de tro det sidste;
\par 9 men hvis de end ikke tror på disse to Tegn og hører på dig, tag da Vand fra Nilen og hæld det ud på Jorden, så skal Vandet, som du tager fra Nilen, blive til Blod på Jorden."
\par 10 Men Moses sagde til HERREN: "Ak, Herre, jeg er ingen veltalende Mand, jeg var det ikke før og er det heller ikke nu, efter at du har talet til din Tjener, thi jeg har svært ved at udtrykke mig og tale for mig."
\par 11 Da svarede HERREN ham: "Hvem har givet Mennesket Mund, og hvem gør stum eller døv, seende eller blind? Mon ikke jeg, HERREN?
\par 12 Gå derfor kun, jeg vil være med din Mund og lære dig, hvad du skal sige!"
\par 13 Men han sagde: "Ak, Herre, send dog enhver anden end mig!"
\par 14 Da blussede HERRENs Vrede op imod Moses, og han sagde: "Har du ikke din Broder Aron, Leviten? Han, ved jeg, kan tale for sig.
\par 15 du skal tale til ham og lægge ham Ordene i Munden, så vil jeg være med din og hans Mund og lære eder, hvad I skal gøre.
\par 16 Han skal tale på dine Vegne til Folket; han skal være din Mund, og du skal være som Gud for ham.
\par 17 Tag nu i din Hånd denne Stav, som du skal gøre Tegnene med!"
\par 18 Derefter vendte Moses tilbage til sin Svigerfader Jetro og sagde til ham: "Lad mig vende tilbage til mine Landsmænd i Ægypten og se, om de endnu er i Live!" Og Jetro svarede Mose's: "Drag bort i Fred!"
\par 19 Da sagde HERREN til Moses i Midjan: "Vend tilbage til Ægypten, thi alle de Mænd, der stod dig efter Livet, er døde."
\par 20 Så tog Moses sin Hustru og sin Søn og satte dem på sit Æsel og vendte tilbage til Ægypten; og Moses tog Guds Stav i Hånden.
\par 21 Men HERREN sagde til Moses: "Når du vender tilbage til Ægypten, så mærk dig dette: Alle de Undergerninger, jeg giver dig Magt til at udføre, skal du gøre for Farao; men jeg vil forhærde hans Hjerte, så han ikke lader Folket rejse.
\par 22 Og da skal du sige til Farao: Så siger HERREN: Israel er min førstefødte Søn;
\par 23 men da jeg sagde til dig: Lad min Søn rejse, for at han kan dyrke mig! da nægtede du at lade ham rejse. Se, jeg dræber din førstefødte Søn!"
\par 24 Men undervejs, i Natteherberget, kom HERREN imod ham og vilde dræbe ham.
\par 25 Da greb Zippora en skarp Sten og afskar sin Søns Forhud og berørte hans Blusel dermed, idet hun sagde: "Du er mig en Blodbrudgom!"
\par 26 Så lod han ham i Fred. Ved den Lejlighed brugte hun Ordet "Blodbrudgom" med Hentydning til Omskærelsen.
\par 27 Derpå sagde HERREN til Aron: "Gå Moses i Møde i Ørkenen!" Og han gik ud og traf ham ved Guds Bjerg og kyssede ham.
\par 28 Og Moses fortalte Aron om alt, hvad HERREN havde pålagt ham, og om alle de Tegn, han havde befalet ham at gøre.
\par 29 Derefter gik Moses og Aron den og kaldte alle Israeliternes Ældste sammen;
\par 30 og Aron fortalte alt, hvad HERREN havde sagt til Moses, og denne gjorde Tegnene i Folkets Påsyn.
\par 31 Da troede Folket, og da de hørte, at HERREN havde givet Agt på Israeliterne og set til deres Elendighed, bøjede de sig og tilbad.

\chapter{5}

\par 1 Derefter gik Moses og Aron hen og sagde til Farao: "Så siger HERREN, Israels Gud: Lad mit Folk rejse, for at de kan holde Højtid for mig i Ørkenen!"
\par 2 Men Farao sagde: "Hvem er HERREN, at jeg skulde adlyde ham og lade Israeliterne rejse? Jeg kender ikke noget til HERREN, og jeg vil heller ikke lade Israeliterne rejse!"
\par 3 De svarede: "Hebræernes Gud har mødt os; tillad os nu at drage tre Dagsrejser ud i Ørkenen og ofre til HERREN. vor Gud, for at han ikke skal slå os med Pest eller Sværd!"
\par 4 Men Ægypterkongen sagde til dem: "Hvorfor vil I, Moses og Aron, forstyrre Folket i dets Arbejde? Gå til eders Trællearbejde!"
\par 5 Og Farao sagde: "Folket er så vist dovent nok; og nu vil I have dem fri fra deres Trællearbejde!"
\par 6 Samme Dag udstedte Farao følgende Befaling til Fogederne over Folket og dets Opsynsmænd:
\par 7 "I skal ikke mere som hidtil give Folket Halm til Teglarbejdet; de skal selv gå ud og sanke Halm;
\par 8 men alligevel skal I pålægge dem at lave lige så mange Teglsten som hidtil, I må ikke eftergive dem noget; thi de er dovne, og derfor er det, de råber op og siger: Lad os drage ud og ofre til vor Gud!
\par 9 Strengt Arbejde skal de Mennesker have, for at de kan være optaget deraf og ikke af Løgnetale."
\par 10 Da gik Folkets Fogeder og Opsynsmænd ud og sagde til Folket: "Således siger Farao: Jeg vil ikke mere give eder Halm;
\par 11 gå selv ud og sank eder Halm, hvor I kan finde det, men i eders Arbejde bliver der intet eftergivet!"
\par 12 Da spredte Folket sig over hele Ægypten for at samle Halmstrå.
\par 13 Men Fogederne trængte på og sagde: "I skal Dag for Dag yde fuldt Arbejde, ligesom dengang I fik Halm!"
\par 14 Og Israeliternes Opsynsmænd, som Faraos Fogeder havde sat over dem, fik Prygl, og der blev sagt til dem: "Hvorfor stryger I ikke mere det fastsatte Antal Teglsten ligesom før?"
\par 15 Da gik Israeliternes Opsynsmænd hen og råbte til Farao: "Hvorfor handler du således med din Fræne?
\par 16 Dine Trælle får ingen Halm, og dog siger de til os: Lav Teglsten! Og dine Trælle får Prygl; du forsynder dig mod dit Folk."
\par 17 Men han svarede: "I er dovne, det er det, I er! Derfor siger I: Lad os rejse ud og ofre til HERREN!
\par 18 Gå nu hen og tag fat på eders Arbejde; I får ingen Halm, men I skal levere det samme Antal Teglsten!"
\par 19 Israeliternes Opsynsmænd følte sig ilde stedt ved at skulle sige: "Der må intet eftergives i, hvad I daglig skal skaffe af Teglsten!"
\par 20 Og da de ved deres Bortgang fra Farao traf Moses og Aron, som stod og ventede på dem,
\par 21 sagde de til dem: "HERREN se til eder og dømme eder, fordi I har vakt Afsky mod os hos Farao og hans Tjenere og lagt dem et Sværd i Hånden til at dræbe os med!"
\par 22 Da vendte Moses sig igen til HERREN og sagde: "Herre, hvorfor har du handlet ilde med dette Folk? Hvorfor har du udsendt mig?
\par 23 Siden jeg har været hos Farao og talt i dit Navn, har han handlet ilde med dette Folk, og frelst dit Folk har du ikke!"

\chapter{6}

\par 1 Men HERREN svarede Moses: "Nu skal du få at se, hvad jeg vil gøre ved Farao! Med Magt skal han blive tvunget til at lade dem rejse, og med Magt skal han blive tvunget til at drive dem ud af sit Land!"
\par 2 Gud talede til Moses og sagde til ham: "Jeg er HERREN!
\par 3 For Abraham, Isak og Jakob åbenbarede jeg mig som Gud den Almægtige, men under mit Navn HERREN gav jeg mig ikke til Kende for dem.
\par 4 Eftersom jeg har oprettet min Pagt med dem om at skænke dem Kana'ans Land, deres Udlændigheds Land, hvor de levede som fremmede,
\par 5 har jeg nu hørt Israeliternes Klageråb over, at Ægypterne holder dem i Trældon, og jeg er kommet min Pagt i Hu.
\par 6 Derfor skal du sige til Israeliterne: Jeg er HERREN, og jeg vil udfri eder fra det Trællearbejde, Ægypterne har pålagt eder, og frelse eder fra deres Trældom og udløse eder med udrakt Arm og med vældige Straffedomme;
\par 7 og så vil jeg antage eder som mit Folk og være eders Gud, og I skal kende, at jeg er HERREN eders Gud, som udfrier eder fra Ægypternes Trællearbejde;
\par 8 og jeg vil føre eder til det Land, jeg har svoret at ville skænke Abraham, Isak og Jakob, og give eder det i Eje. Jeg er HERREN!"
\par 9 Moses kundgjorde nu dette for Israeliterne; men de hørte ikke på Moses, dertil var deres Modløshed for stor og deres Trællearbejde for hårdt
\par 10 Da talede HERREN til Moses og sagde:
\par 11 "Gå hen og sig til Farao, Ægyptens Konge, at han skal lade Israeliterne drage ud af, sit Land!"
\par 12 Men Moses sagde for HERRENs Åsyn: "Israeliterne har ikke hørt på mig, hvor skulde da Farao gøre det, tilmed da jeg er uomskåren på Læberne?"
\par 13 Da talede HERREN til Moses og Aron og sendte dem til Farao, Ægyptens Konge, for at føre Israeliterne ud af Ægypten.
\par 14 Følgende var Overhovederne for deres Fædrenehuse: Rubens, Israels førstefødtes, Sønner var: Hanok, Pallu, Hezron og Harmi; det er Rubens Slægter.
\par 15 Simeons. Sønner var Jemuel, Jamin, Ohad, Jakin, Zohar og Kana'anæerkvindens Søn Sjaul; det er Simeons Slægter.
\par 16 Følgende er Navnene på Levis Sønner efter deres Nedstamning: Gerson, Kehat og Merari. Levis Levetid var 137 År.
\par 17 Gersons Sønner var Libni og Sjimi efter deres Slægter.
\par 18 Hehats Sønner var Amram, Jizhar, Hebon og Uzziel. Hehats Levetid var 133 År.
\par 19 Meraris Sønner var Mali og Musji. Det er Levis Slægter efter deres Nedstamning.
\par 20 Amram tog sin Faster Jokebed til Ægte; og hun fødte ham Aron og Moses. Amrams Levetid var 137 År.
\par 21 Jizhars Sønner var Hora, Nefeg og Zikri.
\par 22 Uzziels Sønner var Misjael, Elzafan og Sitri.
\par 23 Aron tog Amminadabs Datter, Nahasjons Søster Elisjeba til Ægte; og hun fødte ham Nadab, Abibu, Eleazar og Itamar.
\par 24 Koras Sønner var Assir, Elkana og Abiasaf; det er Koraiternes Slægter.
\par 25 Arons Søn Eleazar tog en at Putiels Døtre til Ægte; og hun fødte ham Pinehas. Det er Overhovederne for Leviternes Fædrenehuse efter deres Slægter.
\par 26 Det var Aron og Moses, som HERREN sagde til: "Før Israeliterne ud af Ægypten, Hærafdeling for Hærafdeling!"
\par 27 Det var dem, der talte til Farao, Ægyptens Konge, om at føre Israeliterne ud af Ægypten, Moses og Aron.
\par 28 Dengang HERREN talede til Moses i Ægypten,
\par 29 talede HERREN til Moses således: "Jeg er HERREN! Forkynd Farao, Ægyptens Konge, alt, hvad jeg siger dig!"
\par 30 Men Moses sagde for HERRENs Åsyn: "Se, jeg er uomskåren på Læberne, hvorledes skulde da Farao ville høre på mig?"

\chapter{7}

\par 1 Da sagde HERREN til Moses: "Se, jeg gør dig til Gud for Farao, men din Broder Aron skal være din Profet.
\par 2 Du skal sige til ham alt, hvad jeg pålægger dig, men din Broder Aron skal sige det til Farao, for at han skal lade Israeliterne rejse ud af sit Land.
\par 3 Men jeg vil forhærde Faraos Hjerte og derefter gøre mange Tegn og Undere i Ægypten.
\par 4 Farao skal ikke høre på eder, men jeg vil lægge min Hånd på Ægypten og føre mine Hærskarer, mit Folk Israeliterne, ud af Ægypten med vældige Straffedomme;
\par 5 og når jeg udrækker min Hånd mod Ægypten og fører Israeliterne ud derfra, skal Ægypterne kende, at jeg er HERREN."
\par 6 Da gjorde Moses og Aron, som HERREN pålagde dem.
\par 7 Moses var firsindstyve og Aron tre og firsindstyve År gammel, da de talte til Farao.
\par 8 Og HERREN talede til Moses og Aron og sagde:
\par 9 "Når Farao kræver et Under af eder, sig så til Aron: Tag din Stav og kast den ned foran Farao, så skal den blive til en Slange!"
\par 10 Da gik Moses og Aron til Farao og gjorde, som HERREN bød; og da Aron kastede sin Stav foran Farao og hans Tjenere, blev den til en Slange.
\par 11 Men Farao lod som Modtræk Vismændene og Besværgerne kalde, og de, Ægyptens Koglere, gjorde også det samme ved Hjælp af deres hemmelige Kunster;
\par 12 de kastede hver sin Stav, og Stavene blev til Slanger, men Arons Stav opslugte deres Stave.
\par 13 Men Faraos Hjerte blev forhærdet, og han hørte ikke på dem, således som HERREN havde sagt.
\par 14 HERREN sagde nu til Moses: "Faraos Hjerte er forstokket, han vægrer sig ved at lade Folket rejse.
\par 15 Gå derfor i Morgen tidlig til Farao, når han begiver sig ned til Vandet, og træd frem for ham ved Nilens Bred og tag Staven, der forvandledes til en Slange, i Hånden
\par 16 og sig til ham: HERREN, Hebræernes Gud, sendte mig til dig med det Bud: Lad mit Folk rejse, at det kan dyrke mig i Ørkenen! Men hidtil har du ikke adlydt.
\par 17 Så siger HERREN: Deraf skal du kende, at jeg er HERREN: Se, jeg slår Vandet i Nilen med Staven, som jeg holder i min Hånd, og det skal forvandles til Blod,
\par 18 Fiskene i Nilen skal dø, og Nilen skal stinke, og Ægypterne skal væmmes ved at drikke Vand fra Nilen."
\par 19 Og HERREN sagde til Moses: "Sig til Aron: Tag din Stav og ræk din Hånd ud over Ægypternes Vande, deres Floder, Kanaler, Damme og alle Vandsamlinger, så skal Vandet blive til Blod, og der skal være Blod i hele Ægypten, både i Trækar og Stenkar."
\par 20 Og Moses og Aron gjorde, som HERREN bød; Moses løftede Staven og slog Vandet i Nilen for Øjnene af Farao og hans Tjenere, og alt Vandet i Nilen forvandledes til Blod;
\par 21 Fiskene i Nilen døde, og Nilen stank, så Ægypterne ikke kunde drikke Vand fra Nilen, og der var Blod i hele Ægypten.
\par 22 Men de Ægyptiske Koglere gjorde det samme ved Hjælp af deres hemmelige Kunster, og Faraos Hjerte blev forhærdet, så han ikke hørte på dem, således som HERREN havde sagt.
\par 23 Da vendte Farao sig bort og gik hjem, og heller ikke dette lagde han sig på Sinde.
\par 24 Men alle Ægypterne gravede i Omegnen af Nilen efter Drikkevand, thi de kunde ikke drikke Nilvandet.
\par 25 Og således gik der syv Dage, efter at HERREN havde slået Nilen.

\chapter{8}

\par 1 Derpå sagde HERREN til Moses: "Gå til Farao og sig til ham: Så siger HERREN: Lad mit Folk rejse, for at de kan dyrke mig!
\par 2 Men hvis du vægrer dig ved at lade dem rejse, se, da vil jeg plage hele dit Land med Frøer;
\par 3 Nilen skal vrimle af Frøer, og de skal kravle op og trænge ind i dit Hus og dit Sovekammer og på dit Leje og i dine Tjeneres og dit Folks Huse, i dine Bagerovne og dine Dejgtruge;
\par 4 ja på dig selv og dit Folk og alle dine Tjenere skal Frøerne kravle op."
\par 5 Da sagde HERREN til Moses: "Sig til Aron: Ræk din Hånd med Staven ud over Floderne, Kanalerne og Dammene og få Frøerne til at kravle op over Ægypten!"
\par 6 Og Aron rakte sin Hånd ud over Ægyptens Vande. Da kravlede Frøerne op og fyldte Ægypten.
\par 7 Men Koglerne gjorde det samme ved Hjælp af deres hemmelige Kunster og fik Frøerne til at kravle op over Ægypten.
\par 8 Da lod Farao Moses og Aron kalde og sagde: "Gå i Forbøn hos HERREN, at han skiller mig og mit Folk af med Frøerne, så vil jeg lade Folket rejse, at de kan ofre til HERREN."
\par 9 Moses svarede Farao: "Du behøver kun at befale over mig! Til hvilken Tid skal jeg gå i Forbøn for dig, dine Tjenere og dit Folk om at få Frøerne bort fra dig og dine Huse, så de kun bliver tilbage i Nilen?"
\par 10 Han svarede: "I Morgen!" Da sagde han: "Det skal ske, som du siger, for at du kan kende, at der ingen er som HERREN vor Gud;
\par 11 Frøerne skal vige bort fra dig, dine Huse, dine Tjenere og dit Folk; kun i Nilen skal de blive tilbage."
\par 12 Moses og Aron gik nu bort fra Farao, og Moses råbte til HERREN om at bortrydde Frøerne, som han havde sendt over Farao;
\par 13 og HERREN gjorde, som Moses bad: Frøerne døde i Husene, i Gårdene og på Markerne,
\par 14 og man samlede dem sammen i Dynger, så Landet kom til at stinke deraf.
\par 15 Men da Farao så, at han havde fået Luft, forhærdede han sit Hjerte og hørte ikke på dem, således som HERREN havde sagt.
\par 16 Derpå sagde HERREN til Moses: "Sig til Aron: Ræk din Stav ud og slå Støvet på Jorden med den, så skal det blive til Myg i hele Ægypten!"
\par 17 Og de gjorde således; Aron udrakte sin Hånd med Staven og slog Støvet på Jorden dermed. Da kom der Myg over Mennesker og Dyr; alt Støvet på Jorden blev til Myg i hele Ægypten.
\par 18 Koglerne søgte nu også ved Hjælp af deres hemmelige Kunster at fremkalde Myg, men de magtede det ikke. Og Myggene kom over Mennesker og Dyr.
\par 19 Da sagde Koglerne til Farao: "Det er Guds Finger!" Men Faraos Hjerte blev forhærdet, så han ikke hørte på dem, således som HERREN havde sagt.
\par 20 Derpå sagde HERREN til Moses: "Træd i Morgen tidlig frem for Farao, når han begiver sig ned til Vandet, og sig til ham: Så siger HERREN: Lad mit Folk rejse, for at de kan dyrke mig!
\par 21 Men hvis du ikke lader mit Folk rejse, se, da sender jeg Bremser over dig, dine Tjenere, dit Folk og dine Huse, og Ægypternes Huse skal blive fulde af Bremser, ja endog Jorden, de bor på;
\par 22 men med Gosens Land, hvor mit Folk bor, vil jeg til den Tid gøre en Undtagelse, så der ingen Bremser kommer, for at du kan kende, at jeg HERREN er i Landet;
\par 23 og jeg vil sætte Skel mellem mit Folk og dit Folk; i Morgen skal dette Tegn ske!"
\par 24 Og HERREN gjorde således: Vældige Bremsesværme trængte ind i Faraos og hans Tjeneres Huse og i hele Ægypten, og Landet hærgedes af Bremserne.
\par 25 Da lod Farao Moses og Aron kalde og sagde: "Gå hen og bring eders Gud et Offer her i Landet!"
\par 26 Men Moses sagde: "Det går ikke an at gøre således, thi til HERREN vor Gud ofrer vi, hvad der er Ægypterne en Vederstyggelighed; og når vi for Øjnene af Ægypterne ofrer, hvad der er dem en Vederstyggelighed, mon de da ikke stener os?
\par 27 Lad os drage tre Dagsrejser ud i Ørkenen og ofre til HERREN vor Gud, således som han har pålagt os!"
\par 28 Farao sagde: "Jeg vil lade eder rejse hen og ofre til HERREN eders Gud i Ørkenen; kun må I ikke rejse for langt bort; men gå i Forbøn for mig!"
\par 29 Moses svarede: "Se, så snart jeg kommer ud herfra, skal jeg gå i Forbøn hos HERREN, og Bremserne skal vige bort fra Farao, hans Tjenere og hans Folk i Morgen. Blot Farao så ikke igen narrer os og nægter at lade Folket rejse hen og ofre til HERREN!"
\par 30 Derpå gik Moses bort fra Farao og gik i Forbøn hos HERREN.
\par 31 Og HERREN gjorde, som Moses bad, og Bremserne veg bort fra Farao, hans Tjenere og hans Folk; der blev ikke en eneste tilbage.
\par 32 Men Farao forhærdede også denne Gang sit Hjerte og lod ikke Folket rejse.

\chapter{9}

\par 1 Derpå sagde HERREN til Moses: "Gå til Farao og sig til ham: Så siger HERREN, Hebræernes Gud: Lad mit Folk rejse, for at de kan dyrke mig!
\par 2 Men hvis du vægrer dig ved at lade dem rejse og bliver ved med at holde dem fast,
\par 3 se, da skal HERRENs Hånd komme over dit Kvæg på Marken, over Hestene, Æslerne og Kamelerne, Hornkvæget og Småkvæget med en såre forfærdelig Pest.
\par 4 Og HERREN skal sætte Skel mellem Israels Kvæget og Ægypterens Kvæg, så der ikke skal dø noget af, hvad der tilhører Israeliterne."
\par 5 Og HERREN satte en Tidsfrist, idet han sagde: "I Morgen skal HERREN lade dette ske i Landet."
\par 6 Den følgende Dag lod HERREN det så ske, og alt Ægypternes Kvæg døde, men af Israeliternes Kvæg døde ikke et eneste Dyr.
\par 7 Farao sendte da Bud, og se, ikke et eneste Dyr af Israeliternes Kvæg var dødt. Men Faraos Hjerte blev forhærdet, og han lod ikke Folket rejse.
\par 8 Derpå sagde HERREN til Moses og Aron: "Tag begge eders Hænder fulde af Sod fra Smelteovnen, og Moses skal kaste det i Vejret i Faraos Påsyn!
\par 9 Så skal det blive til en Støvsky over hele Ægypten og til Betændelse, der bryder ud i Bylder på Mennesker og Kvæg i hele Ægypten!"
\par 10 Da tog de Sod fra Smelteovnen og trådte frem for Farao, og Moses kastede det i Vejret; og det blev til Betændelse, der brød ud i Bylder på Mennesker og Kvæg.
\par 11 Og Koglerne kunde ikke holde Stand over for Moses på Grund af Betændelsen, thi Betændelsen angreb Koglerne såvel som alle de andre Ægyptere.
\par 12 Men HERREN forhærdede Faraos Hjerte, så han ikke hørte på dem, således som HERREN havde sagt til Moses.
\par 13 Derpå sagde HERREN til Moses: "Træd i Morgen tidlig frem for Farao og sig til ham: Så siger HERREN, Hebræernes Gud: Lad mit Folk rejse, for at de kan dyrke mig!
\par 14 Thi denne Gang vil jeg sende alle mine Plager mod dig selv og mod dine Tjenere og dit Folk, for at du kan kende, at der er ingen som jeg på hele Jorden.
\par 15 Thi ellers havde jeg nu udrakt min Hånd for at ramme dig og dit Folk med Pest, så du blev udryddet fra Jordens Overflade;
\par 16 dog derfor har jeg ladet dig blive i Live for at vise dig min Magt, og for at mit Navn kan blive forkyndt på hele Jorden.
\par 17 Endnu stiller du dig i Vejen for mit Folk og vil ikke lade det rejse.
\par 18 Se, jeg lader i Morgen ved denne Tid et frygteligt Haglvejr bryde løs, hvis Lige ikke har været i Ægypten, fra den Dag det blev til og indtil nu.
\par 19 Derfor må du sørge for at bringe dit Kvæg og alt, hvad du har på Marken, i Sikkerhed! Thi alle Mennesker og Dyr, der befinder sig på Marken og ikke er kommet under Tag, skal rammes af Haglen og omkomme."
\par 20 De blandt Faraos Tjenere, der frygtede HERRENs Ord, bragte nu deres Trælle og Kvæg under Tag;
\par 21 men de, der ikke lagde sig HERRENs Ord på Hjerte, lod deres Trælle og Kvæg blive ude på Marken.
\par 22 Da sagde HERREN til Moses: "Ræk din Hånd op mod Himmelen, så skal der falde Hagl i hele Ægypten på Mennesker og Dyr og på alle Markens Urter i Ægypten!"
\par 23 Da rakte Moses sin Stav op mod Himmelen, og HERREN sendte Torden og Hagl; Ild for ned mod Jorden, og HERREN lod Hagl falde over Ægypten;
\par 24 og der kom et Haglvejr, med Ildsluer flammende mellem Haglen, så voldsomt, at dets Lige aldrig havde været nogetsteds i Ægypten, siden det blev befolket;
\par 25 og i hele Ægypten slog Haglen alt ned, hvad der var på Marken, både Mennesker og Kvæg, og alle Markens Urter slog Haglen ned, og alle Markens Træer knækkede den;
\par 26 kun i Gosen, hvor Israeliterne boede, faldt der ikke Hagl.
\par 27 Da sendte Farao Bud efter Moses og Aron og sagde til dem: "Denne Gang har jeg syndet; HERREN har Ret, og jeg og mit Folk har Uret;
\par 28 gå i Forbøn hos HERREN, at det nu må være nok med Guds Torden og Haglvejret, så vil jeg lade eder rejse, og I skal ikke blive længer!"
\par 29 Moses svarede ham: "Så snart jeg kommer ud af Byen, skal jeg udbrede mine Hænder mod HERREN; så skal Tordenen høre op, og Haglen skal ikke falde mere, for at du kan kende, at Jorden tilhører HERREN."
\par 30 Dog, jeg ved, at du og dine Tjenere endnu ikke frygter for Gud HERREN."
\par 31 Hørren og Byggen blev slået ned, thi Byggen stod i Aks, og Hørren i Blomst;
\par 32 derimod blev Hveden og Spelten ikke slået ned, thi de modnes senere.
\par 33 Da Moses var gået bort fra Farao og var kommet ud af Byen, udbredte han sine Hænder mod HERREN, og da hørte Tordenen og Haglen op, og Regnen strømmede ikke mere ned.
\par 34 Men da Farao så, at Regnen, Haglen og Tordenen var hørt op, fremturede han i sin Synd, og han og hans Tjenere forhærdede deres Hjerte.
\par 35 Faraos Hjerte blev forhærdet, så at han ikke lod Israeliterne rejse, således som HERREN havde sagt ved Moses.

\chapter{10}

\par 1 Derpå sagde HERREN til Moses: "Gå til Farao! Thi jeg har forhærdet hans og hans Tjeneres Hjerte, at jeg kan komme til at gøre disse mine Tegn iblandt dem,
\par 2 for at du må kunne fortælle din Søn og din Sønnesøn, hvorledes jeg handlede med Ægypterne, og om de Tegn, jeg gjorde iblandt dem; så skal I kende, at jeg er HERREN."
\par 3 Da gik Moses og Aron til Farao og sagde til ham: "Så siger HERREN, Hebræernes Gud: Hvor længe vil du vægre dig ved at ydmyge dig for mig? Lad mit Folk rejse, at de kan dyrke mig!
\par 4 Men hvis du vægrer dig ved at lade mit Folk rejse, se, da vil jeg i Morgen sende Græshopper over dine Landemærker,
\par 5 og de skal skjule Landets Overflade, så man ikke kan se Jorden, og opæde Resten af det, som er blevet tilovers for eder efter Haglen, og opæde alle eders Træer, som gror på Marken;
\par 6 og de skal fylde dine Huse og alle dine Tjeneres og alle Ægypternes Huse således, at hverken dine Fædre eller dine Fædres Fædre nogen Sinde har oplevet Mage dertil, fra den Dag de kom til Verden og indtil denne Dag!" Dermed vendte han sig bort og forlod Farao.
\par 7 Men Faraos Tjenere sagde til ham: "Hvor længe skal denne Mand styrte os i Ulykke? Lad dog disse Mennesker rejse og lad dem dyrke HERREN deres Gud! Har du endnu ikke indset, at Ægypten går til Grunde?"
\par 8 Moses og Aron blev nu hentet tilbage til Farao, og han sagde til dem: "Drag af Sted og dyrk HERREN eders Gud! Men hvem er det nu, der vil af Sted?"
\par 9 Moses svarede: "Med vore Børn og vore gamle vil vi drage af Sted, med vore Sønner og vore Døtre, vort Småkvæg og vort Hornkvæg vil vi drage af Sted, thi vi skal fejre HERRENs Højtid."
\par 10 Da sagde han til dem: "HERREN være med eder, om jeg lader eder rejse sammen med eders Kvinder og Børn! Der ser man, at I har ondt i Sinde!
\par 11 Nej men I Mænd kan drage bort og dyrke HERREN; det var jo det, I ønskede!" Derpå jog man dem bort fra Farao.
\par 12 Da sagde HERREN til Moses: "Ræk din Hånd ud over Ægypten og få Græshopperne til at komme; de skal komme over Ægypten og opæde alt, hvad der vokser i Landet, alt, hvad Haglen har levnet!"
\par 13 Moses rakte da sin Stav ud over Ægypten, og HERREN lod en Østenstorm blæse over Landet hele den Dag og den påfølgende Nat; og da det blev Morgen, førte Østenstormen Græshopperne med sig.
\par 14 Da kom Græshopperne over hele Ægypten, og de slog sig ned i hele Ægyptens Område i uhyre Mængder; aldrig før havde der været så mange Græshopper, og ingen Sinde mere skal der komme så mange.
\par 15 Og de skjulte hele Jordens Overflade, så Jorden blev sort af dem, og de opåd alt, hvad der voksede i Landet, og alle Træfrugter, alt, hvad Haglen havde levnet, og der blev intet grønt tilbage på Træerne eller på Markens Urter i hele Ægypten.
\par 16 Da lod Farao skyndsomt Moses og Aron kalde til sig og sagde: "Jeg har syndet mod HERREN eders Gud og mod eder!
\par 17 Men tilgiv mig nu min Synd denne ene Gang og gå i Forbøn hos eders Gud, at han dog blot vil tage denne dødbringende Plage fra mig!"
\par 18 Da gik Moses bort fra Farao og bad til HERREN.
\par 19 Og HERREN lod Vinden slå om til en voldsom Vestenvind, som tog Græshopperne og drev dem ud i det røde Hav, så der ikke blev en eneste Græshoppe tilbage i hele Ægyptens Område.
\par 20 Men HERREN forhærdede Faraos Hjerte, så han ikke lod Israeliterne rejse.
\par 21 Derpå sagde HERREN til Moses: "Ræk din Hånd op mod Himmelen, så skal der komme et Mørke over Ægypten, som man kan tage og føle på!"
\par 22 Da rakte Moses sin Hånd op mod Himmelen, og der kom et tykt Mørke i hele Ægypten i tre Dage;
\par 23 den ene kunde ikke se den anden, og ingen flyttede sig af Stedet i tre Dage; men overalt, hvor Israeliterne boede, var det lyst.
\par 24 Da lod Farao Moses kalde og sagde: "Drag hen og dyrk HERREN.
\par 25 Men Moses sagde: "Du må også overlade os Slagtofre og Brændofre, som vi kan bringe HERREN vor Gud;
\par 26 også vore Hjorde må vi have med, ikke en Klov må blive tilbage, thi dem har vi Brug for, når vi skal dyrke HERREN vor Gud, og vi ved jo ikke,hvor meget vi behøver dertil, før vi kommer til Stedet."
\par 27 Da forhærdede HERREN Faraos Hjerte, så han nægtede at lade dem rejse.
\par 28 Og Farao sagde til ham: "Gå bort fra mig og vogt dig for at komme mig for Øje mere; thi den Dag du kommer mig for Øje, er du dødsens!"
\par 29 Da sagde Moses: "Du har sagt det, jeg skal ikke mere komme dig for Øje!"

\chapter{11}

\par 1 Derpå sagde HERREN til Moses: "Een Plage endnu vil jeg lade komme over Farao og Ægypterne, og efter den skal han lade eder rejse herfra; ja, når han lader eder rejse med alt, hvad I har, skal han endog drive eder herfra!
\par 2 Sig nu til Folket, at hver Mand skal bede sin Nabo, og hver Kvinde sin Naboerske om Sølv og Guld smykker!"
\par 3 Og HERREN stemte Ægypterne gunstigt imod Folket, og desuden var den Mand Moses højt anset i Ægypten både hos Faraos Tjenere og hos Folket.
\par 4 Moses sagde: "Så siger HERREN: Ved Midnatstid vil jeg drage igennem Ægypten,
\par 5 og så skal alle førstefødte i Ægypten dø, lige fra den førstefødte hos Farao, der skal arve hans Trone, til den førstefødte hos Trælkvinden, der arbejder ved Håndkværnen, og alt det førstefødte af Kvæget.
\par 6 Da skal der i hele Ægypten lyde et Klageskrig så stort, at dets Lige aldrig har været hørt og aldrig mere skal høres.
\par 7 Men end ikke en Hund skal bjæffe ad nogen af Israeliterne, hverken ad Folk eller Fæ for at du kan kende, at HERREN gør Skel mellem Ægypterne og Israel.
\par 8 Da skal alle dine Tjenere der komme ned til mig og kaste sig til Jorden for mig og sige: Drag dog bort med alt det Folk, der følger dig! Og så vil jeg drage bort !" Og han gik ud fra Farao med fnysende Vrede.
\par 9 Men HERREN sagde til Moses: "Farao skal ikke høre på eder, for at mine Undergerninger kan blive talrige i Ægypten."
\par 10 Og Moses og Aron gjorde alle disse Undergerninger i Faraos Påsyn, men HERREN forhærdede Faraos Hjerte, så han ikke lod Israeliterne drage ud af sit Land.

\chapter{12}

\par 1 Derpå talede HERREN til Moses og Aron i Ægypten og sagde:
\par 2 "Denne Måned skal hos eder være Begyndelsesmåneden, den skal hos eder være den første af Årets Måneder.
\par 3 Tal til hele Israels Menighed og sig: På den tiende Dag i denne Måned skal hver Familiefader tage et Lam, et Lam for hver Familie.
\par 4 Og dersom en Familie er for lille til et Lam, skal han sammen med sin nærmeste Nabo tage et Lam, svarende til Personernes Antal; hvor mange der skal være om et Lam, skal I beregne efter.
\par 5 Det skal være et lydefrit, årgammelt Handyr, og I kan tage det enten blandt Fårene eller Gederne.
\par 6 I skal have det gående til den fjortende Dag i denne Måned, og hele Israels Menigheds Forsamling skal slagte det ved Aftenstid.
\par 7 Og de skal tage noget af Blodet og stryge det på de to Dørstolper og Overliggeren i de Huse, hvor I spiser det.
\par 8 I skal spise Kødet samme Nat, stegt over Ilden, og I skal spise usyret Brød og bitre Urter dertil.
\par 9 I må ikke spise noget deraf råt eller kogt i Vand, men kun stegt over Ilden, og Hoved, Ben og Indmad må ikke være skilt fra.
\par 10 I må intet levne deraf til næste Morgen, men hvad der bliver tilovers deraf til næste Morgen, skal I brænde.
\par 11 Og når I spiser det, skal I have Bælte om Lænden ,Sko på Fødderne og Stav i Hånden, og I skal spise det i største Hast.
\par 12 I denne Nat vil jeg drage igennem Ægypten og ihjelslå alt det førstefødte i Ægypten både blandt Folk og Fæ; og over alle Ægyptens Guder vil jeg holde Dom. Jeg er HERREN!
\par 13 Men for eder skal Blodet på Husene, hvor I er, tjene som Tegn, og jeg vil se Blodet og gå eder forbi; intet ødelæggende Slag skal ramme eder, når jeg slår Ægypten.
\par 14 Denne Dag skal være eder en Mindedag, og I skal fejre den som en Højtid for HERREN, Slægt efter Slægt; som en evig gyldig Ordning skal I fejre den.
\par 15 I syv Dage skal I spise usyret Brød. Straks den første Dag skal I skaffe al Surdejg bort af eders Huse; thi hver den, som spiser noget syret fra den første til den syvende Dag, det Menneske skal udryddes af Israel.
\par 16 På den første dag skal I holde Højtidsstævne og ligeledes på den syvende Dag. Intet Arbejde må udføres på dem; kun det, enhver behøver til Føde, det og intet andet må I tilberede.
\par 17 I skal holde det usyrede Brøds Højtid, thi på denne selv samme Dag førte jeg eders Hærskarer ud af Ægypten, derfor skal I højtideligholde denne Dag i alle kommende Slægtled som en evig gyldig Ordning.
\par 18 På den fjortende Dag i den første Måned om Aftenen skal I spise usyret Brød og vedblive dermed indtil Månedens en og tyvende Dag om Aftenen.
\par 19 I syv Dage må der ikke findes Surdejg i eders Huse, thi hver den, som spiser noget syret, det Menneske skal udryddes af Israels Menighed, de fremmede så vel som de indfødte i Landet.
\par 20 I må ikke nyde noget som helst syret; hvor I end bor, skal I spise usyret Brød."
\par 21 Da kaldte Moses alle Israels Ældste sammen og sagde til dem: "Gå ud og hent eder Småkvæg til eders Familier og slagt Påskeofferet;
\par 22 og tag eder Ysopkoste, dyp dem i Blodet i Fadet og stryg noget deraf på Overliggeren og de to Dørstolper; og ingen af eder må gå ud af sin Husdør før i Morgen.
\par 23 Thi HERREN vil gå omkring og slå Ægypterne, og når han da ser Blodet på Overliggeren og de to Dørstolper, vil han gå Døren forbi og ikke give Ødelæggeren Adgang til eders Huse for at slå eder.
\par 24 Dette skal I varetage som en Anordning, der gælder for dig og dine Børn til evig Tid.
\par 25 Og når I kommer til det Land, HERREN vil give eder, således som han har forjættet, så skal I overholde denne Skik.
\par 26 Når da eders Børn spørger eder: Hvad betyder den Skik, I der har?
\par 27 så skal I svare: Det er Påskeoffer for HERREN, fordi han gik Israeliternes Huse forbi i Ægypten, dengang han slog Ægypterne, men lod vore Huse urørte!" Da bøjede Folket sig og tilbad.
\par 28 Og Israeliterne gik hen og gjorde, som HERREN havde pålagt Moses og Aron.
\par 29 Men ved Midnatstid ihjelslog HERREN alle de førstefødte i Ægypten lige fra Faraos førstefødte, der skulde arve hans Trone, til den førstefødte hos Fangen, der sad i Fangehullet, og alt det førstefødte af Kvæget.
\par 30 Da stod Farao op om Natten tillige med alle sine Tjenere og alle Ægypterne, og der lød et højt Klageskrig i Ægypten, thi der var intet Hus, hvor der ikke fandtes en død.
\par 31 Og han lod Moses og Aron kalde om Natten og sagde: "Bryd op og drag bort fra mit Folk, I selv og alle Israeliterne, og drag ud og dyrk HERREN, som I har forlangt.
\par 32 Tag også eders Småkvæg og Hornkvæg med, som I har forlangt, og drag bort; og bed også om Velsignelse for mig!"
\par 33 Og Ægypterne trængte på Folket for at påskynde deres Afrejse fra Landet, thi de sagde: "Vi mister alle Livet!"
\par 34 Og , Folket tog deres Dejg med sig, før den var syret, og de bar Dejtrugene på Skulderen, indsvøbte i deres Kapper.
\par 35 Men Israeliterne havde gjort, som Moses havde sagt, og bedt Ægypterne om Sølv og Guldsmykker og om Klæder;
\par 36 og HERREN havde stemt Ægypterne gunstigt mod Folket, så de havde givet dem, hvad de bad om. Således tog de Bytte fra Ægypterne.
\par 37 Så brød Israeliterne op fra Rameses til Sukkot, henved 600.000 Mand til Fods foruden Kvinder og Børn.
\par 38 Desuden fulgte en stor Hob sammenløbet Folk med og dertil Småkvæg og Hornkvæg, en vældig Mængde Kvæg.
\par 39 Og af den Dejg, de havde bragt med fra Ægypten, bagte de usyret Brød; den var nemlig ikke syret, de var jo drevet ud af Ægypten uden at få Tid til noget; de havde ikke engang tilberedt sig Tæring til Rejsen.
\par 40 Den Tid, Israeliterne havde boet i Ægypten, udgjorde 430 År.
\par 41 Netop på den Dag da de 430 År var til Ende, drog alle HERRENs Hærskarer ud af Ægypten.
\par 42 En Vågenat var det for HERREN, i hvilken han vilde føre dem ud af Ægypten. Den Nat er viet HERREN, en Vågenat for alle Israeliterne, Slægt efter Slægt.
\par 43 HERREN sagde til Moses og Aron: "Dette er Ordningen angående Påskelammet: Ingen fremmed må spise deraf.
\par 44 Men enhver Træl, der er købt for Penge, må spise deraf, såfremt du har fået ham omskåret.
\par 45 Ingen indvandret eller Daglejer må spise deraf.
\par 46 Det skal spises i et og samme Hus, og intet af Kødet må bringes ud af Huset; I må ikke sønderbryde dets Ben.
\par 47 Hele Israels Menighed skal fejre Påsken.
\par 48 Dersom en fremmed bor som Gæst hos dig og vil fejre Påske for HERREN, da skal alle af Mandkøn hos ham omskæres; så må han være med til at fejre den, og han skal være ligestillet med den indfødte i Landet; men ingen uomskåren må spise deraf.
\par 49 En og samme Lov skal gælde for den indfødte i Landet og for den fremmede, der bor som Gæst hos eder."
\par 50 Og Israeliterne gjorde, som HERREN havde pålagt Moses og Aron.
\par 51 På denne selv samme Dag førte HERREN Israeliterne ud af Ægypten, Hærskare for Hærskare.

\chapter{13}

\par 1 Og HERREN talede til Moses og sagde:
\par 2 "Du skal hellige mig alt det førstefødte, alt, hvad der åbner Moders Liv hos Israeliterne både af Mennesker og Kvæg; det skal tilhøre mig!"
\par 3 Og Moses sagde til Folket: "Kom denne Dag i Hu, på hvilken I vandrer ud af Ægypten, af Trællehuset, thi med stærk Hånd førte HERREN eder ud derfra! Og der må ikke spises syret Brød.
\par 4 I Dag vandrer I ud, i Abib Måned.
\par 5 Når nu HERREN fører dig til Kana'anæernes, Hetiternes, Amoriternes, Hivviternes og Jebusiternes Land, som han tilsvor dine Fædre at ville give dig, et Land, der flyder med Mælk og Honning, så skal du overholde denne Skik i denne Måned.
\par 6 I syv Dage skal du spise usyret Brød, og på den syvende Dag skal der være Højtid for HERREN.
\par 7 I disse syv Dage skal der spises usyret Brød, og der må hverken findes syret Brød eller Surdejg hos dig nogetsteds inden dine Landemærker.
\par 8 Og du skal på den Dag fortælle din Søn, at dette sker i Anledning af, hvad HERREN gjorde for dig, da du vandrede ud af Ægypten!
\par 9 Det skal være dig som et Tegn på din Hånd og et Erindringsmærke på din Pande, for at HERRENs Lov må være i din Mund, thi med stærk Hånd førte HERREN dig ud af Ægypten.
\par 10 Og du skal holde dig denne Anordning efterrettelig til den fastsatte Tid, År efter År.
\par 11 Og når HERREN fører dig til Kana'anæernes Land, således som han tilsvor dig og dine Fædre, og giver dig det,
\par 12 da skal du overlade HERREN alt, hvad der åbner Moders Liv; alt det førstefødte, som falder efter dit Kvæg, for så vidt det er et Handyr, skal tilhøre HERREN.
\par 13 Men alt det førstefødte af Æslerne skal du udløse med et Stykke Småkvæg, og hvis du ikke udløser det, skal du sønderbryde dets Hals; og alt det førstefødte af Mennesker blandt dine Sønner skal du udløse.
\par 14 Og når din Søn i Fremtiden spørger dig: Hvad betyder dette? skal du svare ham: Med stærk Hånd førte HERREN os ud af Ægypten, af Trællehuset;
\par 15 og fordi Farao gjorde sig hård og ikke vilde lade os drage bort, ihjelslog HERREN alt det førstefødte i Ægypten både af Folk og Fæ; derfor ofrer vi HERREN alt, hvad der åbner Moders Liv, for så vidt det er et Handyr, og alt det førstefødte blandt vore Sønner udløser vi!
\par 16 Og det skal være dig som et Tegn på din Hånd og et Erindringsmærke på din Pande, thi med stærk Hånd førte HERREN os ud af Ægypten."
\par 17 Da Farao lod Folket drage bort, førte Gud dem ikke ad Vejen til Filisterlandet, som havde været den nærmeste, thi Gud sagde: "Folket kunde komme til at fortryde det, når de ser, der bliver Krig, og vende tilbage til Ægypten."
\par 18 Men Gud lod Folket gøre en Omvej til Ørkenen i Retning af det røde Hav. Israeliterne drog nu væbnede ud af Ægypten.
\par 19 Og Moses tog Josefs Ben med sig, thi denne havde taget Israels Sønner i Ed og sagt: "Når Gud ser til eder, skal I føre mine Ben med eder herfra!"
\par 20 De brød op fra Sukkot og lejrede sig i Etam ved Randen af Ørkenen.
\par 21 Men HERREN vandrede foran dem, om dagen i en Skystøtte for at vise dem Vej og om Natten i en Ildstøtte for at lyse for dem; så kunde de rejse både Dag og Nat.
\par 22 Og Skystøtten veg ikke fra Folket om Dagen, ej heller Ildstøtten om Natten.

\chapter{14}

\par 1 Og HERREN talede til Moses og sagde:
\par 2 "Sig til Israeliterne, at de skal vende om og lejre sig ved Pi Hakirot mellem Migdol og Havet; lige over for Bål Zefon skal I lejre eder ved Havet.
\par 3 Farao vil da tænke om Israeliterne, at de er faret vild i Landet, og at Ørkenen har sluttet dem inde;
\par 4 og jeg vil forhærde Faraos Hjerte, så han sætter efter dem, og jeg vil forherlige mig på Farao og hele hans Krigsmagt, og Ægypterne skal kende, at jeg er HERREN!" Og de gjorde således.
\par 5 Da det nu neldtes Ægypterkongen, at Folket var flygtet, skiftede Farao og hans Tjenere Sind over for Folket og sagde: "Hvor kunde vi dog slippe Israeliterne af vor Tjeneste!"
\par 6 Da lod han spænde for sin Vogn og tog sine Krigsfolk med sig;
\par 7 han tog 600 udsøgte Stridsvogne og alle Ægyptens Krigsvogne, alle bemandede med Vognkæmpere.
\par 8 Og HERREN forhærdede Ægypterkongen Faraos Hjerte, så han satte efter Israeliterne; men Israeliterne var draget ud under en stærk Hånds Værn.
\par 9 Og Ægypterne, alle Faraos Heste og Vogne og hans Ryttere og øvrige Krigsfolk, satte efter dem og indhentede dem, da de havde slået Lejr ved Havet, ved Pi Hakirot over for Ba'al Zefon.
\par 10 Da nu Farao nærmede sig, så Israeliterne op og fik Øje på Ægypterne, der drog efter dem, og de grebes af stor Angst; da råbte Israeliterne til HERREN;
\par 11 og de sagde til Moses: "Er det, fordi der ingen Grave var i Ægypten, at du har fået os ud for at dø i Ørkenen? Hvad er det dog, du har gjort os, at du førte os ud af Ægypten?
\par 12 Var det ikke det, vi sagde til dig i Ægypten: Lad os i Fred, og lad os blive ved at trælle for Ægypterne! Thi det er bedre for os at trælle for Ægypterne end at dø i Ørkenen."
\par 13 Men Moses svarede Folket: "Frygt ikke! Hold blot Stand, så skal I se HERRENs Frelse, som han i Dag vil hjælpe eder til, thi som I ser Ægypterne i Dag, skal I aldrig i Evighed se dem mere.
\par 14 HERREN skal stride for eder, men I skal tie!"
\par 15 Da sagde HERREN til Moses: "Hvorfor råber du til mig? Sig til Israeliterne, at de skal bryde op!
\par 16 Løft din Stav og ræk din Hånd ud over Havet og skil det ad i to Dele, så Israeliterne kan vandre gennem Havet på tør Bund.
\par 17 Se, jeg vil forhærde Ægypternes Hjerte, så de følger efter dem, og jeg vil forherlige mig på Farao og hele hans Krigsmagt, på hans Vogne og Ryttere,
\par 18 og Ægypterne skal kende, at jeg er HERREN, når jeg forherliger mig på Farao, hans Vogne og Ryttere."
\par 19 Guds Engel, der drog foran Israels Hær, flyttede sig nu og gik bag ved dem; og Skystøtten flyttede sig fra Pladsen foran dem og stillede sig bag ved dem
\par 20 og kom til at stå imellem Ægypternes og Israels Hære; og da det blev mørkt; blev Skystøtten til en Ildstøtte og oplyste Natten. Således kom de ikke hinanden nær hele Natten.
\par 21 Moses rakte da sin Hånd ud over Havet, og HERREN drev Havet bort med en stærk Østenstorm, der blæste hele Natten, og han gjorde Havet til tørt Land. Og Vandet delte sig.
\par 22 Da gik Israeliterne midt igennem Havet på tør Bund, medens Vandet stod som en Mur på begge Sider af dem.
\par 23 Og Ægypterne, alle Faraos Heste,Vogne og Ryttere, satte efter dem og forfulgte dem midt ud i Havet.
\par 24 Men ved Morgenvagtens Tid skuede HERREN fra Ild og Skystøtten hen imod Ægypternes Hær og bragte den i Uorden;
\par 25 og han stoppede Vognenes Hjul, så de havde ondt ved at få dem frem. Da sagde Ægypterne: "Lad os flygte for Israel, thi HERREN kæmper for dem imod Ægypten!"
\par 26 Men HERREN sagde til Moses: "Ræk din Hånd ud over Havet, så skal Vandet vende tilbage over Ægypterne, deres Vogne og Ryttere!"
\par 27 Da rakte Moses sin Hånd ud over Havet; og Havet vendte tilbage til sit sædvanlige Leje ved Morgenens Frembrud, medens de flygtende Ægyptere kom lige imod det, og HERREN drev Ægypterne midt ud i Havet.
\par 28 Da vendte Vandet tilbage og overskyllede Vognene og Rytterne i hele Faraos Krigsmagt, som havde forfulgt dem ud i Havet; ikke en eneste af dem blev tilbage.
\par 29 Men Israeliterne var gået gennem Havet på tør Bund, medens Vandet stod som en Mur på begge Sider af dem.
\par 30 Og HERREN frelste på den dag Israel af Ægypternes Hånd, og Israel så Ægypterne ligge døde ved Havets Bred.
\par 31 Da så Israel den Stordåd, HERREN havde udført mod Ægypterne; og Folket frygtede HERREN, og de troede på HERREN og på hans Tjener Moses.

\chapter{15}

\par 1 Ved den Lejlighed sang Moses og Israeliterne denne Sang for HERREN: Jeg vil synge for HERREN, thi han er højt ophøjet, Hest og Rytter styrted han i Havet!
\par 2 HERREN er min Styrke og min Lovsang, og han blev mig til Frelse. Han er min Gud, og jeg vil prise ham, min Faders Gud, og jeg vil ophøje ham.
\par 3 HERREN er en Krigshelt, HERREN er hans Navn!
\par 4 Faraos Vogne og Krigsmagt styrted han i Havet, hans ypperste Vognkæmpere drukned i det røde Hav,
\par 5 Strømmene dækked dem, de sank som Sten i Dybet.
\par 6 Din højre, HERRE, er herlig i Kraft, din højre, HERRE, knuser Fjenden.
\par 7 I din Højheds Vælde fælder du dine Modstandere, du slipper din Harme løs, den fortærer dem som Strå.
\par 8 Ved din Næses Pust dyngedes Vandene op, Vandene stod som en Vold, Strømmene stivnede midt i Havet.
\par 9 Fjenden tænkte: "Jeg sætter efter dem, indhenter dem, uddeler Bytte, stiller mit Begær på dem; jeg drager mit Sværd, min Hånd skal udrydde dem."
\par 10 Du blæste med din Ånde, Havet skjulte dem; de sank som Bly i de vældige Vande.
\par 11 Hvo er som du blandt Guder, HERRE, hvo er som du, herlig i Hellighed, frygtelig i Stordåd, underfuld i dine Gerninger!
\par 12 Du udrakte din højre, og Jorden slugte dem.
\par 13 Du leded i din Miskundhed det Folk, du genløste, du førte det i din Vælde til din hellige Bolig.
\par 14 Folkene hørte det og bæved, Skælven greb Filisterlandets Folk.
\par 15 Da forfærdedes Edoms Høvdinger, Moabs Fyrster grebes af Rædsel, Kana'ans Beboere tabte alle Modet.
\par 16 Skræk og Angst faldt over dem, ved din Arms Vælde blev de målløse som Sten, til dit Folk var nået frem, o HERRE, til Folket, du vandt dig, var nået frem.
\par 17 du førte dem frem og planted dem i din Arvelods Bjerge, på det Sted du beredte dig til Bolig, HERRE, i den Helligdom, Herre, som dine Hænder grundfæsted.
\par 18 HERREN er Konge i al Evighed!
\par 19 Thi da Faraos Heste med hans Vogne og Ryttere drog ud i Havet, lod HERREN Havets Vande strømme tilbage over dem, medens Israeliterne gik gennem Havet på tør Bund.
\par 20 Da greb Profetinden Mirjam, Arons Søster, Pauken, og alle Kvinderne fulgte hende med Pauker og Danse,
\par 21 og Mirjam sang for: Syng for HERREN, thi han er højt ophøjet, Hest og Rytter styrted han i Havet!
\par 22 Derpå lod Moses Israel bryde op fra det røde Hav, og de drog ud i Sjurs Ørken, og de vandrede tre Dage i Ørkenen uden at finde Vand.
\par 23 Så nåede de Mara, men de kunde ikke drikke Vandet for dets bitre Smag, thi det var bittert; derfor kaldte man Stedet Mara.
\par 24 Da knurrede Folket mod Moses og sagde: "Hvad skal vi drikke?"
\par 25 Men han råbte til HERREN, og da viste HERREN ham en bestemt Slags Træ; og da han kastede det i Vandet, blev Vandet drikkeligt. Der gav han dem Bestemmelser om Lov og Ret, og der satte han dem på Prøve.
\par 26 Og han sagde: "Hvis du vil høre på HERREN din Guds Røst og gøre, hvad der er ret i hans Øjne, og lytte til hans Bud og holde dig alle hans Bestemmelser efterrettelig, så vil jeg ikke bringe nogen af de Sygdomme over dig, som jeg bragte over Ægypterne, men jeg HERREN er din Læge!"
\par 27 Derpå kom de til Elim, hvor der var tolv Vandkilder og halvfjerdsindstyve Palmetræer, og de lejrede sig ved Vandet der.

\chapter{16}

\par 1 Så brød de op fra Elim, og hele Israeliternes Menighed kom til Sins Ørken, der ligger mellem Elim og Sinaj, på den femtende Dag i den anden Måned efter deres Udvandring af Ægypten.
\par 2 Men hele Israeliternes Menighed knurrede mod Moses og Aron i Ørkenen,
\par 3 og Israeliterne sagde til dem: "Var vi dog blot døde for HERRENs Hånd i Ægypten, hvor vi sad ved Kødgryderne og kunde spise os mætte i Brød! Thi I har ført os ud i denne Ørken for at lade hele denne Forsamling dø af Sult."
\par 4 Da sagde HERREN til Moses: "Se, jeg vil lade Brød regne ned fra Himmelen til eder, og Folket skal gå ud og hver Dag samle så meget, som de daglig behøver, for at jeg kan prøve dem, om de vil følge min Lov eller ej.
\par 5 Og når de på den sjette Ugedag tilbereder, hvad de har bragt hjem, så skal det være dobbelt så meget, som de samler de andre Dage."
\par 6 Og Moses og Aron sagde til alle Israeliterne: "I Aften skal I kende, at det er HERREN, som har ført eder ud af Ægypten,
\par 7 og i Morgen skal I skue HERRENs Herlighed, thi han har hørt eders Knurren mod HERREN; thi hvad er vel vi, at I knurrer mod os!"
\par 8 Og Moses tilføjede: "Det skal ske, når HERREN i Aften giver eder Kød at spise og i Morgen Brød at mætte eder med; thi HERREN har hørt, hvorledes I knurrer mod ham; thi hvad er vi? Thi det er ikke os, I knurrer imod, men HERREN."
\par 9 Derpå sagde Moses til Aron: "Sig til hele Israeliternes Menighed: Træd frem for HERRENs Åsyn, thi han har hørt eders Knurren!"
\par 10 Og da Aron sagde det til hele Israeliternes Menighed, vendte de sig mod Ørkenen, og se, HERRENs Herlighed viste sig i Skyen.
\par 11 Da talede HERREN til Moses og sagde:
\par 12 "Jeg har hørt Israeliternes Knurren; sig til dem: Ved Aftenstid skal I få Kød at spise, og i Morgen tidlig skal I få Brød at mætte eder med, og I skal kende, at jeg er HERREN eders Gud."
\par 13 Da det nu blev Aften, kom en Sværm Vagtler flyvende og faldt i et tykt Lag over Lejren; og næste Morgen lå Duggen tæt rundt om Lejren,
\par 14 og da Duggen svandt, var Ørkenen dækket med noget fint, skælagtigt noget, noget fint der lignede Rim på Jorden.
\par 15 Da Israeliterne så det, spurgte de hverandre: "Hvad er det?" Thi de vidste ikke, hvad det var; men Moses sagde til dem: "Det er det Brød, HERREN har givet eder til Føde!
\par 16 Og således har HERREN påbudt: I skal samle deraf, enhver så meget som han har behov, en Omer for hvert Hoved; I skal tage deraf i Forhold til Antallet af eders Husfolk, enhver skal tage deraf til dem, der er i hans Telt!"
\par 17 Israeliterne gjorde nu således, og de samlede, nogle mere og andre mindre;
\par 18 og da de målte det med Omeren, havde den, der havde meget, ikke for meget, og den, der havde lidt, ikke for lidt, enhver havde samlet, hvad han behøvede til Føde.
\par 19 Derpå sagde Moses til dem: "Ingen må gemme noget deraf til næste Morgen!"
\par 20 Dog adlød de ikke Moses, og nogle af dem gemte noget deraf til næste Morgen; men da var det fuldt af Orme og lugtede. Da blev Moses vred på dem.
\par 21 Således samlede de nu hver Morgen, enhver så meget som han havde behov. Men når Solen begyndte at brænde, smeltede det.
\par 22 På den sjette Ugedag havde de samlet dobbelt så meget Brød, to Omer for hver Person. Og alle Menighedens Øverster kom og sagde det til Moses;
\par 23 men han sagde til dem: "Det er netop, som HERREN har sagt. I Morgen er det Hviledag, en hellig Sabbat for HERREN. Bag, hvad I vil bage, og kog, hvad I vil koge, men læg det tiloversblevne til Side for at gemme det til i Morgen!"
\par 24 De lagde det da til Side til næste Dag, som Moses havde befalet, og det kom ikke til at lugte, og der gik ikke Orm deri.
\par 25 Derpå sagde Moses: "Det skal I spise i Dag, thi i Dag er det Sabbat for HERREN; i Dag finder I intet ude på Marken.
\par 26 I seks bage skal I samle det, men på den syvende Dag, på Sabbaten, er der intet at finde."
\par 27 Alligevel var der nogle blandt Folket, der gik ud på den syvende Dag for at samle; men de fandt intet.
\par 28 Da sagde HERREN til Moses: "Hvor længe vil I vægre eder ved at holde mine Bud og Love?
\par 29 Betænk dog, at HERREN har givet eder Sabbaten! Derfor giver han eder på den sjette Dag Brød til to Dage. Enhver af eder skal blive, hvor han er, og ingen må forlade sin Bolig på den syvende Dag!"
\par 30 Da hvilede Folket på den syvende Dag.
\par 31 Men Israeliterne kaldte det Manna; det lignede hvide Korianderfrø og smagte som Honningkager.
\par 32 Moses sagde fremdeles: "Således har HERREN påbudt: En Omer fuld deraf skal gemmes til eders Efterkommere, for at de kan se det Brød, jeg gav eder at spise i Ørkenen, da jeg førte eder ud af Ægypten!"
\par 33 Og Moses sagde til Aron: "Tag en Krukke, kom en Omer Manna deri og stil den foran HERRENs Åsyn for at gemmes til eders Efterkommere!"
\par 34 Og Aron gjorde, som HERREN havde pålagt Moses, og han stillede den foran Vidnesbyrdet for at gemmes.
\par 35 Og Israeliterne spiste Manna i fyrretyve År, indtil de kom til beboede Egne; de spiste Manna, indtil de kom til Grænsen af Kana'ans Land.
\par 36 En Omer er Tiendedelen af en Efa.

\chapter{17}

\par 1 Så brød hele Israels Menighed op fra Sins Ørken og drog fra Lejrplads til Lejrplads efter HERRENs Bud. Men da de lejrede sig i Refidim, havde Folket intet Vand at drikke.
\par 2 Da kivedes Folket med Moses og sagde: "Skaf os Vand at drikke!" Men Moses svarede dem: "Hvorfor kives I med mig, hvorfor frister I HERREN?"
\par 3 Og Folket tørstede der efter Vand og knurrede mod Moses og sagde: "Hvorfor har du ført os op fra Ægypten? Mon for at lade os og vore Børn og vore Hjorde dø af Tørst?"
\par 4 Da råbte Moses til HERREN: "Hvad skal jeg gøre med dette Folk? Det er ikke langt fra, at de vil stene mig."
\par 5 Men HERREN sagde til Moses: "Træd frem for Folket med nogle af Israels Ældste og tag den Stav, du slog Nilen med, i din Hånd og kom så!
\par 6 Se, jeg vil stå foran dig der på Klippen ved Horeb, og når du slår på Klippen, skal der strømme Vand ud af den, så Folket kan få noget at drikke." Det gjorde Moses så i Påsyn af Israels Ældste.
\par 7 Og han kaldte dette Sted Massa og Meriba, fordi Israeliterne der havde kivedes og fristet HERREN ved at sige: "Er HERREN iblandt os eller ej?"
\par 8 Derefter kom Amalekiterne og angreb Israel i Refdim.
\par 9 Da sagde Moses til Josua: "Udvælg dig Mænd og ryk i. Morgen ud til Kamp mod Amalekiterne; jeg vil stille mig på Toppen af Højen med Guds Stav i Hånden!"
\par 10 Josua gjorde, som Moses bød, og rykkede ud til Kamp mod Amalekiterne. Men Moses, Aron og Hur gik op på Toppen af Højen.
\par 11 Når nu Moses løftede Hænderne i Vejret, fik Israeliterne Overtaget, men når han lod Hænderne synke, fik Amalekiterne Overtaget.
\par 12 Og da Moses's Hænder blev trætte, tog de en Sten og lagde under ham; så satte han sig på den, og Aron og Hur støttede hans Hænder, hver på sin Side; således var hans Hænder stadig løftede til Solen gik ned,
\par 13 og Josua huggede Amalekiterne og deres Krigsfolk ned med Sværdet.
\par 14 Da sagde HERREN til Moses: "Optegn dette i en Bog, for at det kan mindes, og indskærp Josua, at jeg fuldstændig vil udslette Amalekiternes Minde under Himmelen!"
\par 15 Derpå byggede Moses et Alter og kaldte det: "HERREN er mit Banner!"
\par 16 Og han sagde: "Der er en udrakt Hånd på HERRENs Trone! HERREN har Krig med Amalek fra Slægt til Slægt!"

\chapter{18}

\par 1 Da Jetro, Præsten i Midjan, Moses's Svigerfader, hørte om alt, hvad Gud havde gjort for Moses og hans Folk Israel, hvorledes HERREN havde ført Israel ud af Ægypten,
\par 2 tog Jetro, Moses's Svigerfader, Zippora, Moses's Hustru, som han havde sendt hjem,
\par 3 tillige med hendes to Sønner. Af dem hed den ene Gersom; "thi", havde han sagt, "jeg er blevet Gæst i et fremmed Land";
\par 4 og den anden hed Eliezer; "thi", havde han sagt, "min Faders Gud har været min Hjælp og frelst mig fra Faraos Sværd!"
\par 5 Og Jetro, Moses's Svigerfader, kom med hans Sønner og Hustru til Moses i Ørkenen, hvor han havde slået Lejr ved Guds Bjerg,
\par 6 og han lod Moses melde: "Jetro, din Svigerfader, kommer til dig med din Hustru og hendes to Sønner!"
\par 7 Da gik Moses sin Svigerfader i Møde, bøjede sig for ham og kyssede ham; og da de havde hilst på hinanden, gik de ind i Teltet.
\par 8 Moses fortalte sin Svigerfader om alt, hvad HERREN havde gjort ved Farao og Ægypten for Israels Skyld, og om alle de Besværligheder, der havde mødt dem undervejs, og hvorledes HERREN havde frelst dem.
\par 9 Da glædede Jetro sig over alt det gode, HERREN havde gjort mod Israel, idet han havde frelst dem af Ægypternes Hånd.
\par 10 Og Jetro sagde: "Lovet være HERREN, som har frelst eder af Ægypternes og Faraos. Hånd!"
\par 11 Nu ved jeg, at HERREN er større end alle Guder, thi netop ved det, de i deres Overmod foretog sig imod dem, frelste han Folket af Ægypternes Hånd.
\par 12 Derpå udtog Jetro, Moses's Svigerfader, Brændofre og Slagtofre til Gud; og Aron og alle Israels Ældste kom for at holde Måltid for Guds Åsyn med Moses's Svigerfader.
\par 13 Næste Morgen tog Moses Sæde for at holde Ret for Folket, og Folket stod omkring Moses fra Morgen til Aften.
\par 14 Men da Moses's Svigerfader så alt det Arbejde, han havde med Folket, sagde han: "Hvad er dog det for et Arbejde, du har med Folket? Hvorfor sidder du alene til Doms, medens alt Folket står omkring dig fra Morgen til Aften?"
\par 15 Moses svarede sin Svigerfader: "Jo, Folket kommer til mig for at rådspørge Gud;
\par 16 når de har en Retssag, kommer de til mig, og jeg dømmer Parterne imellem og kundgør dem Guds Anordninger og Love."
\par 17 Da sagde Moses's Svigerfader til ham: "Det er ikke klogt, som du bærer dig ad med det.
\par 18 På den Måde bliver jo både du selv og Folket der omkring dig ganske udmattet, thi det Arbejde er dig for anstrengende, du kan ikke overkomme det alene.
\par 19 Læg dig nu på Sinde, hvad jeg siger; jeg vil give dig et Råd, og Gud skal være med dig: Du skal. selv træde frem for Gud på Folkets Vegne og forelægge Gud de forefaldende Sager;
\par 20 og du skal indskærpe dem Anordningerne og Lovene og lære dem den Vej, de skal vandre, og hvad de har at gøre.
\par 21 Men du skal af hele Folket udvælge dig dygtige Mænd, som frygter Gud, Mænd, som er til at lide på og hader uretfærdig Vinding, og dem skal du sætte over dem som Forstandere, nogle over tusinde, andre over hundrede, andre over halvtredsindstyve, andre over ti;
\par 22 lad dem til Stadighed holde Ret for Folket. Alle vigtigere Sager skal de forebringe dig, men alle mindre Sager skal de selv afgøre. Let dig således Arbejdet og lad dem komme til at bære Byrden med dig.
\par 23 Dersom du handler således og Gud vil det så, kan du holde ud, og alt Folket der kan gå tilfreds hjem."
\par 24 Moses fulgte sin Svigerfaders Råd og gjorde alt, hvad han foreslog.
\par 25 Og Mose's udvalgte dygtige Mænd af hele Israel og gjorde dem til Øverster over Folket, til Forstandere, nogle over tusinde, andre over hundrede, andre over halvtredsindstyve, andre over ti.
\par 26 De holdt derpå til Stadighed Ret for Folket; de vanskelige Sager forebragte de Moses, men alle mindre Sager afgjorde de selv.
\par 27 Derpå tog Moses Afsked med sin Svigerfader, og denne begav sig til sit Land.

\chapter{19}

\par 1 I den tredje Måned efter Israeliternes Udvandring af Ægypten, på denne Dag nåede de Sinaj Ørken.
\par 2 De brød op fra Refdim og kom til Sinaj Ørken og slog Lejr i Ørkenen. Der slog . Israel Lejr lige over for Bjerget,
\par 3 men Moses steg op til Gud. Da råbte HERREN til ham fra Bjerget: "Dette skal du sige til Jakobs Hus og kundgøre for Israels Børn:
\par 4 I har set, hvad jeg gjorde ved Ægypterne, og hvorledes jeg bar eder på Ørnevinger og bragte eder hid til mig.
\par 5 Hvis I nu vil lyde min Røst og holde min Pagt, så skal I være min Ejendom blandt alle Folkene, thi mig hører hele Jorden til,
\par 6 og I skal blive mig et Kongerige af Præster og et helligt Folk! Det er de Ord, du skal tale til Israels Børn !"
\par 7 Da gik Moses hen og kaldte Folkets Ældste sammen og forelagde dem alle disse Ord, som HERREN havde pålagt ham.
\par 8 Og hele Folket svarede, alle som een: "Alt, hvad HERREN har sagt, vil vi gøre!" Da bragte Moses HERREN Folkets Svar.
\par 9 Derpå sagde HERREN til Moses: "Se, jeg vil komme til dig i en tæt Sky, for at Folket kan høre, at jeg taler med dig, og for stedse tro også på dig!" Og Moses kundgjorde HERREN Folkets Svar.
\par 10 Da sagde HERREN til Moses: "Gå til Folket og lad dem hellige sig i Dag og i Morgen og tvætte deres Klæder
\par 11 og holde sig rede til i Overmorgen, thi i Overmorgen vil HERREN stige ned for alt Folkets Øjne på Sinaj Bjerg.
\par 12 Og du skal rundt om spærre af for Folket og sige til dem: Vog eder for at gå op på Bjerget, ja for blot at røre ved Yderkanten deraf; enhver, der rører ved Bjerget, er dødsens!
\par 13 Ingen Hånd må røre ved ham, han skal stenes eller skydes ned; hvad enten det er et Dyr eller et Menneske, skal det miste Livet.
\par 14 Så steg Moses ned fra Bjerget til Folket og lod Folket hellige sig, og de tvættede deres Klæder;
\par 15 og han sagde til Folket: Hold eder rede til i Overmorgen, ingen må komme en Kvinde nær!"
\par 16 Da Morgenen gryede den tredje Dag, begyndte det at tordne og lyne, og en tung Sky lagde sig over Bjerget, og der hørtes vældige Stød i Horn. Da skælvede alt Folket i Lejren.
\par 17 Så førte Moses Folket fra Lejren hen for Gud, og de stillede sig neden for Bjerget.
\par 18 Men hele Sinaj Bjerg hylledes i Røg, fordi HERREN steg ned derpå i Ild, og Røgen stod i Vejret som Røg fra en Smelteovn; og hele Folket skælvede så1e.
\par 19 Og Stødene i Hornene blev stærkere og stærkere; Moses talte, og Gud svarede ham med høj Røst.
\par 20 Og da HERREN var steget ned på Sinaj Bjerg, på Toppen af Bjerget, kaldte han Moses op på Toppen af Bjerget, og Moses steg derop.
\par 21 Da sagde HERREN til Moses: "Stig ned og indskærp Folket, at de ikke må trænge sig frem til HERREN for at se ham, at der ikke skal ske et stort Mandefald iblandt dem.
\par 22 Selv Præsterne, som ellers træder frem for HERREN, skal hellige sig, for at ikke HERREN skal tynde ud i deres Rækker."
\par 23 Da sagde Moses til HERREN: "Folket kan jo ikke stige op på Sinaj Bjerg, thi du har selv indskærpet os at afspærre Bjerget og hellige det."
\par 24 Men HERREN sagde til ham: "Stig nu ned og kom atter herop sammen med Aron; men Præsterne og Folket må ikke trænge sig frem for at komme op til HERREN, at han ikke skal tynde ud i deres Rækker."
\par 25 Da steg Moses ned til Folket og sagde det til dem.

\chapter{20}

\par 1 Gud talede alle disse Ord og sagde:
\par 2 Jeg er HERREN din Gud, som førte dig ud af Ægypten, af Trællehuset.
\par 3 Du må ikke have andre Guder end mig.
\par 4 Du må ikke gøre dig noget udskåret Billede eller noget Afbillede af det, som er oppe i Himmelen eller nede på Jorden eller i Vandet under Jorden;
\par 5 du må ikke tilbede eller dyrke det, thi jeg HERREN din Gud er en nidkær Gud, der indtil tredje og fjerde Led straffer Fædres Brøde på Børn af dem, som hader mig,
\par 6 men i tusind Led viser Miskundhed mod dem, der elsker mig og holder mine Bud!
\par 7 Du må ikke misbruge HERREN din Guds Navn, thi HERREN lader ikke den ustraffet, der misbruger hans Navn!
\par 8 Kom Hviledagen i Hu, så du holder den hellig!
\par 9 I seks Dage skal du arbejde og gøre al din Gerning,
\par 10 men den syvende Dag skal være Hviledag for HERREN din Gud; da må du intet Arbejde udføre, hverken du selv, din Søn eller Datter, din Træl eller Trælkvinde, dit Kvæg eller den fremmede inden dine Porte.
\par 11 Thi i seks Dage gjorde HERREN Himmelen, Jorden og Havet med alt, hvad der er i dem, og på den syvende Dag hvilede han; derfor har HERREN velsignet Hviledagen og helliget den.
\par 12 Ær din Fader og din Moder, for at du kan få et langt Liv i det Land, HERREN din Gud vil give dig!
\par 13 Du må ikke slå ihjel!
\par 14 Du må ikke bedrive Hor!
\par 15 Du må ikke stjæle!
\par 16 Du må ikke sige falsk Vidnesbyrd imod din Næste!
\par 17 Du må ikke begære din Næstes Hus! Du må ikke begære din Næstes Hustru, hans Træl eller Trælkvinde, hans Okse eller Æsel eller noget, der hører din Næste til!
\par 18 Men da hele Folket fornam Tordenen, Lynene og Stødene i Hornene og så det rygende Bjerg forfærdedes Folket og holdt sig skælvende i Frastand;
\par 19 og de sagde til Moses: "Tal du med os, så vil vi lytte til; men lad ikke Gud tale med os, at vi ikke skal dø!"
\par 20 Men Moses svarede Folket: "Frygt ikke, thi Gud er kommet for at prøve eder, og for at I kan lære at frygte for ham, så I ikke synder."
\par 21 Da holdt Folket sig i Frastand medens Moses nærmede sig Mulmet, hvori Gud var.
\par 22 HERREN sagde da til Moses: Således skal du sige til Israeliterne: I har selv set, at jeg har talet med eder fra Himmelen!
\par 23 I må ikke gøre eder Guder ved Siden af mig; Guder af Sølv eller Guld må I ikke gøre eder!
\par 24 Du skal bygge mig et Alter af Jord, og på det skal du ofre dine Brændofre og Takofre, dit Småkvæg og dit Hornkvæg; på ethvert Sted, hvor jeg lader mit Navn ihukomme, vil jeg komme til dig og velsigne dig.
\par 25 Men hvis du opfører mig Altre af Sten, må du ikke bygge dem af tilhugne Sten, thi når du svinger dit Værktøj derover, vanhelliger du dem.
\par 26 Du må ikke stige op til mit, Alter ad Trin, for at ikke din Blusel skal blottes over det.

\chapter{21}

\par 1 De Lovbud, du skal forelægge dem, er følgende:
\par 2 Når du køber dig en hebræisk Træl, skal han trælle i seks År, men i det syvende skal han frigives uden Vederlag.
\par 3 Er han ugift, når han kommer til dig, skal han frigives alene; er han gift, skal hans Hustru frigives sammen med ham.
\par 4 Hvis hans Herre giver ham en Hustru og hun føder ham Sønner eller Døtre, da skal Hustruen og hendes Børn tilhøre hendes Herre, og Trællen frigives alene.
\par 5 Hvis han imidlertid erklærer: "Jeg har fået Kærlighed til min Herre, min Hustru og mine Børn, jeg vil ikke have min Frihed!"
\par 6 da skal hans Herre føre ham hen til Gud og stille ham op ad Døren eller Dørstolpen, og hans Herre skal gennembore hans Øre med en Syl, og så skal han være hans Træl for Livstid.
\par 7 Når en Mand sælger sin Datter som Trælkvinde, skal hun ikke frigives som Trællene.
\par 8 Dersom hun pådrager sig sin Herres Mishag, efter at han har haft Omgang med hende, skal han tillade, at hun købes fri; han har ikke Lov at sælge hende til fremmede Folk, når han har gjort Uret imod hende;
\par 9 hvis han derimod bestemmer, at hun skal være hans Søns Hustru, skal han behandle hende, som det tilkommer Døtre.
\par 10 Hvis han tager sig en anden, har han ikke Lov at forholde den første den Kødspise, Klædning og ægteskabelige Ret, der tilkommer hende.
\par 11 Forholder han hende nogen af disse tre Ting, skal hun frigives uden Vederlag og Betaling.
\par 12 Den, der slår en Mand ihjel, skal lide Døden.
\par 13 Gør han det imidlertid ikke med Forsæt, men styres hans Hånd af Gud, vil jeg anvise dig et Sted, hvor han kan ty hen.
\par 14 Når derimod en handler med Overlæg, så han med List slår sin Næste ihjel, da skal du rive ham bort fra mit Alter, for at han kan lide Døden.
\par 15 Den, der slår sin Fader eller Moder, skal lide Døden.
\par 16 Den, der stjæler et Menneske, skal lide Døden, hvad enten han har solgt det, eller det endnu findes hos ham.
\par 17 Den, der forbander sin Fader eller Moder, skal lide Døden.
\par 18 Når der opstår Strid mellem Mænd, og den ene slår den anden med en Sten eller med Næven, så at han vel ikke dør deraf, men dog må holde Sengen,
\par 19 så skal Gerningsmanden være sagesløs, hvis han kan stå op og gå ud støttet til sin Stok; kun skal han godtgøre ham hans Tidsspilde og sørge for hans Helbredelse.
\par 20 Når en Mand slår sin Træl eller Trælkvinde med sin Stok, så de dør på Stedet, skal han straffes derfor;
\par 21 men hvis de bliver i Live en Dag eller to, skal han ikke straffes; det er jo hans egne Penge.
\par 22 Når Mænd kommer i Slagsmål og støder til en frugtsommelig Kvinde, så hun nedkommer i Utide, men der ellers ingen Ulykke sker, da skal han bøde, hvad Kvindens Mand pålægger ham, og give Erstatning for det dødfødte Barn.
\par 23 Men hvis der sker en Ulykke, skal du bøde Liv for Liv,
\par 24 Øje for Øje, Tand for Tand, Hånd for Hånd, Fod for Fod,
\par 25 Brandsår for Brandsår, Sår for Sår, Skramme for Skramme.
\par 26 Når en Mand slår sin Træl eller sin Trælkvinde i Øjet og ødelægger det, skal han give dem fri til Erstatning for Øjet;
\par 27 og hvis han slår en Tand ud på sin Træl eller Trælkvinde, skal han give dem fri til Erstatning for Tanden.
\par 28 Når en Okse stanger en Mand eller Kvinde ihjel, skal Oksen stenes, og dens Kød må ikke spises, men Ejeren er sagesløs;
\par 29 men hvis Oksen allerede tidligere har villet stange, og dens Ejer er advaret, men alligevel ikke passer på den, og den så dræber en Mand eller Kvinde, da skal Oksen stenes, og dens Ejer skal også lide Døden;
\par 30 men hvis der pålægges ham Sonepenge, skal han betale så stor en Løsesum for sit Liv, som der kræves af ham.
\par 31 Også hvis den stanger en Dreng eller en Pige, skal han behandles efter samme Lovbud.
\par 32 Men hvis Oksen stanger en Træl eller Trælkvinde, skal han betale deres Herre tredive Sekel Sølv, og Oksen skal stenes.
\par 33 Når en Mand tager Dækket af en Cisterne eller graver en Cisterne uden at dække den til, og en Okse eller et Æsel så falder deri,
\par 34 da skal Brøndens Ejer erstatte det; han skal give Dyrets Ejer Erstatning i Penge, men det døde Dyr skal tilfalde ham,
\par 35 Når en Mands Okse stanger en andens Okse ihjel, skal de sælge den levende Okse og dele Pengene, og ligeledes skal de dele det døde dyr.
\par 36 Men hvis det er vitterligt, at Oksen tidligere har villet stange, og dens Ejer ikke har passet på den, da skal han erstatte Okse med Okse, men det døde Dyr skal tilfalde ham.

\chapter{22}

\par 1 Når en Mand stjæler en Okse eller et Får og slagter eller sælger dem, skal han give fem Okser i Erstatning for Oksen og fire Får for Fåret.
\par 2 Hvis en Tyv gribes på fersk Gerning ved et natligt Indbrud og bliver slået ihjel, da bliver der ikke Tale om Blodskyld;
\par 3 men hvis Solen er stået op. pådrager man sig Blodskyld.
\par 4 hvis derimod det stjålne findes levende i hans Besiddelse, da skal han give dobbelt Erstatning, hvad enten det er en Okse, et Æsel, eller et Får.
\par 5 Når en Mand afsvider en Mark eller en Vingård og lader Ilden brede sig, så den antænder en andens Mark, da skal han give det bedste af sin Mark eller Vingård i Erstatning;
\par 6 men breder Ilden sig ved at tage fat i Tjørnekrat, og Kornneg eller Sæd brænder, eller en Mark svides af, så skal den, der antændte Ilden, give simpel Erstatning.
\par 7 Når en Mand giver en anden Penge eller Sager i Varetægt, og de stjæles fra hans Hus, da skal Tyven, hvis han findes, give dobbelt Erstatning;
\par 8 men hvis Tyven ikke findes, skal Husets Ejer træde frem for Gud og sværge på, at han ikke har forgrebet sig på den andens Gods.
\par 9 I alle Tilfælde hvor det drejer sig om Uredelighed med en Okse, et Æsel, et Får, en Klædning eller en hvilken som helst bortkommen Ting, hvorom der rejses Krav, skal de to Parters Sag bringes frem for Gud, og den, som Gud dømmer skyldig, skal give den anden dobbelt Erstatning.
\par 10 Når en Mand giver en anden et Æsel, en Okse, et Får eller et andet Stykke Kvæg i Varetægt, og Dyret dør, kommer til Skade eller røves, uden at nogen ser det,
\par 11 da skal han sværge ved HERREN på, at han ikke har forgrebet sig på den andens Ejendom, og det skal være afgørende imellem dem; Dyrets Ejer skal tage Eden god, og den anden behøver ikke at give Erstatning.
\par 12 Stjæles det derimod fra ham, skal han give Ejeren Erstatning.
\par 13 Hvis det sønderrives, skal han bringe det sønderrevne Dyr med som Bevis; det sønderrevne skal han ikke erstatte.
\par 14 Når en låner et Dyr af en anden, og det kommer til Skade eller dør, uden at Ejeren er til Stede, skal han give Erstatning;
\par 15 er Ejeren derimod til Stede, skal han ikke give Erstatning; var det lejet, er Lejesummen Erstatning.
\par 16 Når en Mand forfører en Jomfru, der ikke er trolovet, og ligger hos hende, skal han udrede Brudekøbesummen for hende og tage hende til Hustru;
\par 17 og hvis hendes Fader vægrer sig ved at give ham hende, skal han tilveje ham den sædvanlige Brudekøbesum for en Jomfru.
\par 18 En Troldkvinde må du ikke lade leve.
\par 19 Enhver, der har Omgang med Kvæg, skal lide Døden.
\par 20 Den, der ofrer til andre Guder end HERREN alene, skal der lægges Band på.
\par 21 Den fremmede må du ikke undertrykke eller forulempe, thi I var selv fremmede i Ægypten.
\par 22 Enken eller den faderløse må I aldrig mishandle;
\par 23 hvis I mishandler dem, og de råber om Hjælp til mig, vil jeg visselig høre på deres Klageråb,
\par 24 og da vil min Vrede blusse op, og jeg vil slå eder ihjel med Sværdet, så eders egne Hustruer bliver Enker og eders Børn faderløse.
\par 25 Når du låner Penge til en fattig Mand af mit Folk i dit Nabolag, må du ikke optræde som en Ågerkarl over for ham. I må ikke tage Renter af ham.
\par 26 Hvis du tager din Næstes Kappe i Pant, skal du give ham den tilbage inden Solnedgang;
\par 27 thi den er det eneste, han har at dække sig med, det er den, han hyller sit Legeme i; hvad skulde han,ellers ligge med? Og når han råber til mig, vil jeg høre ham, thi jeg er barmhjertig.
\par 28 Gud må du ikke spotte, og dit Folks Øvrighed må du ikke forbande.
\par 29 Din Lades Overflod og din Vinperses Saft må du ikke holde tilbage. Den førstefødte af dine Sønner skal du give mig.
\par 30 Ligeså skal du gøre med dit Hornkvæg og dit Småkvæg; i syv Dage skal det blive hos Moderen, men på den ottende Dag skal I give mig det.
\par 31 I skal være mig hellige Mænd; Kød af sønderrevne Dyr må I ikke spise, I skal kaste det for Hundene.

\chapter{23}

\par 1 Du må ikke udsprede falske Rygter. Gør ikke fælles Sag med den, der har Uret, ved at optræde som uretfærdigt Vidne.
\par 2 Du må ikke følge Mængden i, hvad der er ondt, eller i dit Vidnesbyrd for Retten tage Hensyn til Mængden, så du bøjer Retten.
\par 3 Du må ikke tage Parti for den ringe i hans Retssag.
\par 4 Når du træffer din Fjendes Okse eller Æsel løbende løse, skal du bringe dem tilbage til ham.
\par 5 Når du ser din Uvens Æsel segne under sin Byrde, må du ikke lade ham i Stikken, men du skal hjælpe ham med at læsse Byrden af.
\par 6 Du må ikke bøje din fattige Landsmands Ret i hans Retssag.
\par 7 Hold dig fra en uretfærdig Sag; og den, som er uskyldig og har Ret. må du ikke berøve Livet; nej, du må ikke skaffe den Ret, som har Uret.
\par 8 Tag ikke mod Bestikkelse, thi Bestikkelse gør den seende blind og fordrejer Sagen for dem, der har Ret.
\par 9 Undertryk ikke den fremmede; I ved jo, hvorledes den fremmede er til Mode, thi I,var selv fremmede i Ægypten.
\par 10 Seks År igennem skal du tilså dit Land og indsamle dets Afgrøde;
\par 11 men i det syvende skal du lade det hvile og ligge urørt, så at de fattige i dit Folk kan gøre sig til gode dermed, og Markens vilde Dyr kan æde, hvad de levner; ligeså skal du gøre med din Vingård og dine Oliventræer.
\par 12 I seks Dage skal du gøre dit Arbejde, men på den syvende skal du hvile, for at dine Okser og Æsler kan få Hvile og din Trælkvindes Søn og den fremmede hvile ud.
\par 13 Hold eder alt, hvad jeg siger eder, efterrettelig; du må ikke nævne andre Guders Navn, det må ikke høres i din Mund.
\par 14 Tre Gange om Året skal du holde Højtid for mig.
\par 15 Du skal fejre de usyrede Brøds Højtid; i syv Dage skal du spise usyret Brød, som jeg har pålagt dig, på den fastsatte Tid i Abib Måned, thi i den Måned vandrede du ud af Ægypten, Man må ikke stedes for mit Åsyn med tomme Hænder.
\par 16 Fremdeles skal du fejre Højtiden for Høsten, Førstegrøden af dit Arbejde, af hvad du sår i din Mark, og Højtiden for Frugthøsten ved Årets Udgang, når du har bjærget Udbyttet af dit Arbejde hjem fra Marken.
\par 17 Tre Gange om Året skal alle dine Mænd stedes for den Herre HERRENs Åsyn.
\par 18 Du må ikke ofre Blodet af mit Slagtoffer sammen med syret Brød. Fedtet fra min Højtid må ikke gemmes til næste Morgen.
\par 19 Du skal bringe det første, Førstegrøden af din Jord, til HERREN din Guds Hus. Du må ikke koge et Kid i dets Moders Mælk.
\par 20 Se, jeg sender en Engel foran dig for at vogte dig undervejs og føre dig til det Sted, jeg har beredt.
\par 21 Tag dig vel i Vare for ham og adlyd ham; vær ikke genstridig imod ham, thi han skal ikke tilgive eders Overtrædelser, efterdi mit Navn er i ham.
\par 22 Når du adlyder ham og gør alt, hvad jeg siger, vil jeg være dine Fjenders Fjende og dine Modstanderes Modstander,
\par 23 Ja, min Engel skal drage foran dig og føre dig til Amoriterne, Hetiterne, Perizziterne, Kana'anæerne, Hivviterne og Jebusiterne, og jeg vil udrydde dem.
\par 24 Du må ikke tilbede eller dyrke deres Guder eller følge deres Skikke; men du skal nedbryde dem og sønderslå deres Stenstøtter.
\par 25 I skal dyrke HERREN eders Gud, så vil jeg velsigne dit Brød og dit Vand og holde Sygdomme borte fra dig.
\par 26 Utidige Fødsler eller Ufrugtbarhed skal ikke forekomme i dit Land, og dine Dages Mål vil jeg gøre fuldt.
\par 27 Jeg vil sende min Rædsel foran dig og bringe Bestyrtelse over alle de Folk, du kommer til, og jeg vil drive alle dine Fjender på Flugt for dig.
\par 28 Jeg vil sende Gedehamse foran dig, og de skal drive Hivviterne, Kana'anæerne og Hetiterne bort foran dig.
\par 29 Men jeg vil ikke drive dem bort foran dig i et og samme År, for at Landet ikke skal lægges øde, og for at Markens vilde Dyr ikke skal tage Overhånd for dig;
\par 30 lidt efter lidt vil jeg drive dem bort foran dig, indtil du bliver så talrig, at du kan tage Landet i Besiddelse.
\par 31 Jeg vil lade dine Landemærker nå fra det røde Hav til Filisternes Hav, fra Ørkenen til Floden, thi jeg giver Landets Indbyggere i eders Hånd, så du kan drive dem bort foran dig.
\par 32 Du må ikke slutte Pagt med dem eller deres Guder.
\par 33 De må ikke blive boende i dit Land, for at de ikke skal forlede dig til Synd imod mig, til at dyrke deres Guder, så det bliver dig til en Snare!

\chapter{24}

\par 1 Og han sagde til Moses: "Stig op til HERREN, du og Aron, Nadab og Abih og halvfjerdsindstyve af Israels Ældste, og tilbed i Frastand;
\par 2 Moses alene skal træde hen til HERREN, de andre ikke, og det øvrige Folk må ikke følge med ham derop."
\par 3 Derpå kom Moses og kundgjorde hele Folket alle HERRENs Ord og alle Lovbudene, og hele Folket svarede enstemmigt: "Alle de Ord, HERREN har talet, vil vi overholde."
\par 4 Da skrev Moses alle HERRENs Ord op; og tidligt næste Morgen rejste han ved Foden af Bjerget et Alter og tolv Stenstøtter svarende til Israels tolv Stammer.
\par 5 Derefter sendte han de unge Mænd blandt Israeliterne hen for at bringe Brændofre og slagte unge Tyre som Takofre til HERREN.
\par 6 Og Moses tog den ene Halvdel af Blodet og gød det i Offerskålene, men den anden Halvdel sprængte han på Alteret.
\par 7 Så tog han Pagtsbogen og læste den op i Folkets Påhør, og de sagde: "Vi vil gøre alt, hvad HERREN har talet, og lyde ham!"
\par 8 Derpå tog Moses Blodet og sprængte det på Folket, idet han sagde: "Se, dette er Pagtens Blod, den Pagt, HERREN har sluttet med eder på Grundlag af alle disse Ord."
\par 9 Og Moses, Aron, Nadab og Abihu og halvfjerdsindstyve af Israels Ældste steg op
\par 10 og skuede Israels Gud; under hans Fødder var der ligesom Safirfliser, som selve Himmelen i Stråleglans.
\par 11 Men han lagde ikke Hånd på Israeliternes ypperste Mænd. De skuede Gud, og de spiste og drak.
\par 12 Og HERREN sagde til Moses: "Stig op til mig på Bjerget og bliv der, så vil jeg give dig Stentavlerne, Loven og Budet, som jeg har opskrevet til Vejledning for dem."
\par 13 Da bød Moses og Josua, hans Medhjælper op, og Moses steg op på Guds Bjerg;
\par 14 men til de Ældste sagde han: "Vent på os her, til vi kommer tilbage til eder. Se, Aron og Hur er hos eder; er der nogen, der har en Retstrætte, kan han henvende sig til dem!"
\par 15 Derpå steg Moses op på Bjerget. Da indhyllede Skyen Bjerget,
\par 16 og HERRENs Herlighed nedlod sig på Sinaj Bjerg. Og Skyen indhyllede Sinaj Bjerg i seks Dage, men den syvende Dag råbte HERREN ud fra Skyen til Moses;
\par 17 og medens HERRENs Herlighed viste sig for Israeliternes Øjne som en fortærende Ild på Bjergets Top,
\par 18 gik Moses ind i Skyen og steg op på Bjerget. Og Moses blev på Bjerget i fyrretyve Dage og fyrretyve Nætter.

\chapter{25}

\par 1 HERREN talede til Moses og sagde:
\par 2 Sig til Israeliterne, at de skal bringe mig en Offerydelse; af enhver, som i sit Hjerte føler sig tilskyndet dertil, skal I tage min Offerydelse.
\par 3 Og Offerydelsen, som I skal tage af dem, skal bestå af Guld, Sølv, Kobber,
\par 4 violet og rødt Purpurgarn, karmoisinrødt Garn, Byssus, Gedehår,
\par 5 rødfarvede Væderskind, Tahasjskind, Akacietræ,
\par 6 Olie til Lysestagen, vellugtende Stoffer til Salveolien og Aøgelsen,
\par 7 Sjohamsten og Ædelsten til Indfatning på Efoden og Brystskjoldet.
\par 8 Og du skal indrette mig en Helligdom, for at jeg kan bo midt iblandt dem.
\par 9 Du skal indrette Boligen og alt dens Tilbehør nøje efter det Forbillede, jeg vil vise dig.
\par 10 Du skal lave en Ark af Akacietræ, halvtredje Alen lang, halvanden Alen bred og halvanden Alen høj,
\par 11 og overtrække den med pu1t Guld; indvendig og udvendig skal du overtrække den og sætte en gylden Krans rundt om den;
\par 12 og du skal støbe fire Guldringe til den og sætte dem på dens fire Fødder, to Ringe på hver Side at den.
\par 13 Så skal du lave Bærestænger af Akacietræ og overtrække dem med Guld,
\par 14 og du skal stikke Stængerne gennem Ringene på Arkens Sider, for at den kan bæres med dem;
\par 15 Stængerne skal blive i Ringene, de må ikke tages ud.
\par 16 Og i Arken skal du nedlægge Vidnesbyrdet, som jeg vil give dig.
\par 17 Så skal du lave et Sonedække af purt Guld, halvtredje Alen langt og halvanden Alen bredt;
\par 18 og du skal lave to Keruber af Guld, i drevet Arbejde skal du lave dem, ved begge Ender af Sonedækket.
\par 19 Den ene Kerub skal du anbringe ved den ene Ende, den anden Kerub ved den anden; du skal lave Keruberne således, at de er i eet med Sonedækket ved begge Ender.
\par 20 Og Keruberne skal brede deres Vinger i Vejret, således at de dækker over Sonedækket med deres Vinger, og de skal vende Ansigtet mod hinanden; nedad mod Sonedækket skal Kerubernes Ansigter vende.
\par 21 Og Sonedækket skal du lægge over Arken, men i Arken skal du lægge Vidnesbyrdet, som jeg vil give dig.
\par 22 Der vil jeg mødes med dig, og fra Sonedækket, fra Pladsen mellem de to Keruber på Vidnesbyrdets Ark, vil jeg meddele dig alle de Bud, jeg har at give dig til Israeliterne.
\par 23 Fremdeles skal du lave et Bord at Alkacietræ, to Alen langt, en Alen bredt og halvanden Alen højt,
\par 24 og overtrække det med purt Guld og sætte en gylden Krans rundt om det.
\par 25 Og du skal sætte en Liste af en Hånds Bredde rundt om det og en gylden Krans rundt om Listen.
\par 26 Så skal du lave fire Guldringe og sætte dem på de fire Hjørner ved dets fire Ben;
\par 27 lige ved Listen skal Ringene sidde til at stikke Bærestængerne i, så at man kan bære Bordet.
\par 28 Og du skal lave Bærestængerne af Akacietræ og overtrække dem med Guld, og med dem skal Bordet bæres.
\par 29 Og du skal lave de dertil hørende Fade og Kander, Krukker og Skåle til at udgyde Drikoffer med; af purt Guld skal du lave dem.
\par 30 På Bordet skal du altid have Skuebrød liggende for mit Åsyn.
\par 31 Fremdeles skal du lave en Lysestage af purt Guld, i drevet Arbejde skal Lysestagen, dens Fod og selve Stagen, laves, således af dens Blomster med Bægere og Kroner er i eet med den.
\par 32 Seks Arme skal udgå fra Lysestagens Side, tre fra den ene og tre fra den anden Side.
\par 33 På hver af Armene, der udgår fra Lysestagen, skal der være tre mandelblomstlignende Blomster med Bægere og Kroner,
\par 34 men på selve Stagen skal der være fire mandelblomstlignende Blomster med Bægere og Kroner,
\par 35 et Bæger under hvert af de tre Par Arme, der udgår fra Lysestagen.
\par 36 Bægrene og Armene skal være i eet med den, så at det hele udgør eet drevet Arbejde af purt Guld.
\par 37 Og du skal lave syv Lamper til den og sætte disse Lamper på den, for at de kan lyse Pladsen foran op.
\par 38 Dens Lampesakse og Bakker skal være af purt Guld.
\par 39 Der skal bruges en Talent purt Guld til den og til alt dette Tilbehør.
\par 40 Se til, at du udfører det efter det Forbillede, som vises dig på Bjerget.

\chapter{26}

\par 1 Boligen skal du lave af ti Tæpper af tvundet Byssus, violet og rødt Purpurgarn og karmoisinrødt Garn med Keruber på i Kunstvævning.
\par 2 Hvert Tæppe skal være otte og tyve Alen langt og fire Alen bredt; alle Tæpperne skal have samme Mål.
\par 3 Tæpperne skal sys sammen, fem og fem.
\par 4 I Kanten af det ene Tæppe, det yderste i det ene sammensyede Stykke, skal du sætte Løkker af violet Purpurgarn, og ligeledes skal du sætte Løkker i Kanten af det yderste Tæppe i det andet sammensyede Stykke;
\par 5 du skal sætte halvtredsindstyve Løkker på det ene Tæppe og halvtredsindstyve Løkker i Kanten af det tilsvarende Tæppe i det andet sammensyede Stykke, Løkke lige over for Løkke.
\par 6 Og du skal lave halvtredsindstyve Guldkroge til at forbinde Tæpperne med hinanden, så at Boligen udgør et Hele.
\par 7 Fremdeles skal du lave Tæpper af Gedehår til et Teltdække uden om Boligen, og her skal du lave elleve Tæpper,
\par 8 hvert Tæppe skal skal være tredive Alen langt og fire Alen bredt; alle Tæpperne skal have samme Mål.
\par 9 Og du skal sy de fem af Tæpperne sammen for sig og de seks for sig; det sjette Tæppe, det, der kommer til at ligge over Teltets Forside, skal du lægge dobbelt.
\par 10 Og du skal sætte halvtredsindstyve Løkker i Kanten af det yderste Tæppe i det ene sammensyede Stykke og halvtredsindstyve Løkker i Kanten af det tilsvarende Tæppe i det andet sammensyede Stykke.
\par 11 Og du skal lave halvtredsindstyve Kobberkroge og stikke dem i Løkkerne og sammenføje Teltdækket, så de udgør et Hele.
\par 12 Men hvad angår det overskydende af Teltdækkets Tæpper, skal Halvdelen deraf hænge ned over Boligens Bagside,
\par 13 og den overskydende Alen på begge Sider af Telttæppernes Længder skal hænge ned over begge Boligens Sider for at dække den.
\par 14 Fremdeles skal du lave et Dække over Teltdækket af rødfarvede Væderskind og derover endnu et Dække af Tahasjskind.
\par 15 Fremdeles skal du lave Brædderne til Boligen af Akacietræ til at stå op,
\par 16 hvert Bræt ti Alen højt og halvanden Alen bredt.
\par 17 På hvert Bræt skal der være to indbyrdes forbundne Tapper; således skal du indrette det ved alle Boligens Brædder.
\par 18 Af Brædderne, som du skal lave til Boligen, skal tyve være til Sydsiden,
\par 19 og til de tyve Brædder skal du lave fyrretyve Fodstykker af Sølv, to Fodstykker til de to Tapper på hvert Bræt.
\par 20 Andre tyve Brædder skal laves til Boligens anden Side, som vender mod Nord,
\par 21 med fyrretyve Fodstykker af Sølv, to Fodstykker til hvert Bræt.
\par 22 Og til Bagsiden, der vendet mod Vest, skal du lave seks Brædder.
\par 23 Til Boligens Baghjørner skal du lave to Brædder,
\par 24 som skal bestå af to Stykker forneden og ligeledes af to Stykker foroven, indtil den første Ring; således skal de begge indrettes for at danne de to Hjørner.
\par 25 Altså bliver der til Bagsiden otte Brædder med tilhørende seksten Fodstykker af Sølv, to til hvert Bræt.
\par 26 Og du skal lave Tværstænger af Alkacietræ, fem til de Brædder, der danner Boligens ene Side,
\par 27 fem til de Brædder, der danner Boligens anden Side, og fem til de Brædder, der danner Boligens Bagside mod Vest;
\par 28 den mellemste Tværstang midt på Brædderne skal nå fra den ene Ende af Væggen til den anden.
\par 29 Du skal overtrække Brædderne med Guld, og deres Ringe, som Tværstængerne skal stikkes i, skal du lave af Guld, og Tværstængerne skal du overtrække med Guld.
\par 30 Og du skal rejse Boligen på den Måde, som vises dig på Bjerget.
\par 31 fremdeles skal du lave et Forhæng af violet og rødt Purpurgarn, karmoisinrødt Garn og tvundet Byssus; det skal laves i Kunstvævning med Keruber på.
\par 32 Du skal hænge det på fire Piller af Akacietræ, overtrukne med Guld og med Knager af Guld, på fire Fodstykker af Sølv;
\par 33 og du skal hænge Forhænget under Krogene og bringe Vidnesbyrdets Ark ind i Rummet bag ved Forhænget, og Forhænget skal danne eder en Skillevæg mellem det Hellige og det Allerhelligste.
\par 34 Og Sonedækket skal du lægge over Vidnesbyrdets Ark i det Allerhelligste.
\par 35 Men Bordet skal du stille uden for Forhænget, og Lysestagen over for Bordet ved Boligens søndre Væg; Bordet skal du altså stille ved den nordre Væg.
\par 36 Fremdeles skal du lave et Forhæng til Teltets indgang af violet og rødt Purpurgarn, karmoisinrødt Garn og tvundet Byssus i broget Vævning;
\par 37 og til Forhænget skal du lave fem Piller af Akacietræ, som du skal overtrække med Guld, med Knager af Guld, og du skal støbe fem Fodstykker dertil af Kobber.

\chapter{27}

\par 1 Fremdeles skal du lave Alteret af Akacietræ, fem Alen langt og fem Alen bredt, firkantet skal Alteret være, og tre Alen højt.
\par 2 Du skal lave Horn til dets fire Hjørner, således at de er i eet dermed, overtrække det med Kobber
\par 3 og lave Kar dertil, for at Asken kan fjernes, ligeledes de dertil hørende Skovle, Skåle, Gafler og Pander; alt dets Tilbehør skal du lave af Kobber.
\par 4 Du skal omgive det med et flettet Kobbergitter, og du skal sætte fire Kobberringe på Fletværket. på dets fire Hjørner.
\par 5 Og du skal sætte Gitteret neden under Alterets Liste, således at Fletværket når op til Alterets halve Højde.
\par 6 Fremdeles skal du lave Bærestænger til Alteret, Stænger af Akacietræ, og overtrække dem med Kobber.
\par 7 Og Stængerne skal stikkes gennem Ringene, så at de sidder langs Alterets to Sider, når det bæres.
\par 8 Du skal lave det hult af Brædder; som det vises dig på Bjerget, skal du lave det.
\par 9 Boligens Forgård skal du indrette således: På Sydsiden skal der være et Forgårdsomhæng af tvundet Byssus, hundrede Alen langt på denne ene Side,
\par 10 med tyve Piller og tyve Fodstykker af Kobber og med Knager og Bånd af Sølv til Pillerne.
\par 11 Og på samme Måde skal der på den nordre Langside være et Omhæng, hundrede Alen langt, med tyve Piller og tyve Fodstykker af Kobber og med Knager og Bånd af Sølv til Pillerne.
\par 12 På Forgårdens Bredside mod Vest skal der være et Omhæng, halvtredsindstyve Alen langt, med ti Piller og ti Fodstykker,
\par 13 og Forgårdens Bredside mod Øst skal være halvtredsindstyve Alen lang.
\par 14 På den ene Side deraf skal der være femten Alen Omhæng med tre Piller og tre Fodstykker,
\par 15 på den anden Side ligeledes femten Alen Omhæng med tre Piller og tre Fodstykker.
\par 16 Forgårdens Indgang skal have et Forhæng på tyve Alen af violet og rødt Purpurgarn, karmoisinrødt Garn og tvundet Byssus i broget Vævning med fire Piller og fire Fodstykker.
\par 17 Alle Forgårdens Piller rundt omkring skal have Bånd af Sølv, Knager af Sølv og Fodstykker af Kobber.
\par 18 Omhænget om Forgården skal være hundrede Alen på hver Langside, halvtredsindstyve Alen på hver Bredside og fem Alen højt; det skal være af tvundet Byssus, og Fodstykkerne skal være af Kobber.
\par 19 Alle Redskaber, der bruges ved Arbejdet på Boligen, alle dens Pæle og alle Forgårdens Pæle skal være af Kobber.
\par 20 Fremdeles skal du pålægge Israeliterne at skaffe dig ren Olivenoli af knuste Frugter til Lysestagen, og der skal stadig sættes Lamper på.
\par 21 I Åbenbaringsteltet uden for Forhænget, der hænger foran Vidnesbyrdet, skal Aron og hans Sønner gøre den i Stand, at den kan brænde fra Aften til Morgen for HERRENs Åsyn. Det skal være en evig gyldig Bestemmelse, der skal påhvile Israeliterne fra Slægt til Slægt.

\chapter{28}

\par 1 Du skal lade din Broder Aron og hans Sønner tillige med ham træde frem af Israeliternes Midte og komme hen til dig, for at de kan gøre Præstetjeneste for mig, Aron og Arons Sønner, Nadab, Abihu, Eleazar og Itamar.
\par 2 Og du skal tilvirke din Broder Aron hellige Klæder til Ære og Pryd,
\par 3 og du skal byde alle kunstforstandige Mænd, hvem jeg har fyldt med Kunstfærdigheds Ånd, at tilvirke Aron Klæder, for at han kan helliges til at gøre Præstetjeneste for mig.
\par 4 Klæderne, som de skal tilvirke, er følgende: Brystskjold, Efod, Kåbe, Kjortel af mønstret Stof, Hovedklæde og Bælte. De skal tilvirke din Broder Aron og hans Sønner hellige Klæder, for at de kan gøre Præstetjeneste for mig,
\par 5 og dertil skal de bruge Guldtråd, violet og rødt Purpurgarn, Karmoisinrødt Garn og Byssus.
\par 6 Efoden skal du tilvirke af Guldtråd, violet og rødt Purpurgarn, karmoisinrødt Garn og tvundet Byssus i Kunstvævning.
\par 7 Den skal have to Skulderstykker, der skal være til at hæfte på; den skal hæftes sammen ved begge Hjørner.
\par 8 Og dens Bælte, som skal bruges, når den tages på, skal være af samme Arbejde og i eet med den; det skal være af Guldtråd, violet og rødt Purpurgarn, karmoisinrødt Garn og tvundet Byssus.
\par 9 Så skal du tage de to Sjohamsten og gravere Israels Sønners Navne i dem,
\par 10 seks af Navnene på den ene Sten og de andre seks på den anden efter Aldersfølge;
\par 11 med Stenskærerarbejde, som ved Gravering af Signeter, skal du indgravere Israels Sønners Navne i de 14 Sten, og du skal indfatte dem i Guldfletværk.
\par 12 Disse to Sten skal du fæste på Efodens Skulderstykker, for at Stenene kan bringe Israels Sønner i Minde, og Aron skal bære deres Navne for HERRENs Åsyn på sine Skuldre for at bringe dem i Minde.
\par 13 Og du skal tilvirke Fletværk af Guld
\par 14 og to Kæder af purt Guld; du skal lave dem i snoet Arbejde, som når man snor Reb, og sætte disse snoede Kæder på Fletværket.
\par 15 Fremdeles skal du tilvirke Retskendelsens Brystskjold i Kunstvævning på samme Måde som Efoden; af Guldtråd, violet og rødt Purpurgarn, karmoisinrødt Garn og tvundet Byssus skal du lave det;
\par 16 det skal være firkantet og lægges dobbelt, et Spand langt og et Spand bredt,
\par 17 Og du skal udstyre det med en Besætning af Sten, fire Rækker Sten: Karneol, Topas og Smaragd i den første Række,
\par 18 Rubin, Safir og Jaspis i den anden,
\par 19 Hyacint, Agat og Ametyst i den tredje,
\par 20 Krysolit, Sjoham og Onyks i den fjerde; og de skal omgives med Guldfletværk i deres Indfatninger.
\par 21 Der skal være tolv Sten, svarende til Israels Sønners Navne, en for hvert Navn; det skal være graveret Arbejde som Signeter, således at hver Sten bærer Navnet på en af de tolv Stammer.
\par 22 Til Brystskjoldet skal du lave snoede Kæder af purt Guld, snoet Arbejde, som når man snor Reb.
\par 23 Til Brystskjoldet skal du lave to Guldringe og sætte disse to Ringe på Brystskjoldets øverste Hjørner,
\par 24 og de to Guldsnore skal du knytte i de to Ringe på Brystskjoldets Hjørner;
\par 25 Snorenes anden Ende skal du anbringe i det Fletværker og fæste dem på Forsiden af Efodens Skulderstykke.
\par 26 Og du skal lave to andre Guldringe og sætte dem på Brystskjoldets to andre Hjørner på den indre, mod Efoden vendende Rand.
\par 27 Og du skal lave endnu to Guldringe og fæste dem på Efodens to Skulderstykker forneden på Forsiden, hvor den er hæftet sammen med Skulderstykkerne, oven over Efodens Bælte;
\par 28 og man skal med Ringene binde Brystskjoldet fast til Efodens Ringe ved Hjælp af en violet Purpursnor, så at det kommer til at sidde oven over Efodens Bælte og ikke løsner sig fra Efoden.
\par 29 Aron skal således stedse bære Israels Sønners Navne på Retskendelsens Brystskjold på sit Hjerte, når han går ind i Helligdommen, for at bringe dem i Minde for HERRENs Åsyn.
\par 30 Og i Retskendelsens Brystskjold skal du lægge Urim og Tummim, så at Aron.bærer dem på sit Hjerte, når han går ind for HERRENs Åsyn, og Aron skal stedse bære Israeliternes Retskendelse på sit Hjerte for HERRENs Åsyn.
\par 31 Fremdeles skal du tilvirke Kåben, som hører til Efoden, helt og holdent af violet Purpur.
\par 32 Midt på skal den have en Halsåbning ligesom Halsåbningen på en Panserskjorte, omgivet af en Linning i vævet Arbejde, for at den ikke skal rives itu;
\par 33 og langs dens nedeste Kant skal du sy Granatæbler af violet og rødt Purpurgarn og karmoisinrødt Garn og mellem dem Guldbjælder hele Vejen rundt,
\par 34 så at Guldbjælder og Granatæbler skifter hele Vejen rundt langs Kåbens nederste Kant.
\par 35 Aron skal bære den, når han gør Tjeneste, så at det kan høres, når han går ind i Helligdommen for HERRENs Åsyn, og når han går ud derfra, at han ikke skal dø,
\par 36 Fremdeles skal du lave en Pandeplade af purt Guld, og i den skal du gravere, som når man graverer Signeter: "Helliget HERREN."
\par 37 Den skal du fastgøre med en violet Purpursnor, og den skal sidde på Hovedklædet, foran på Hovedklædet skal den sidde.
\par 38 Aron skal bære den på sin Pande, for at han kan tage de Synder på sig, som klæder ved de hellige Gaver, Israeliterne frembærer, ved alle de hellige Gaver, de bringer; og Aron skal stedse have den på sin Pande for at vinde dem HERRENs Velbehag.
\par 39 Kjortelen skal du væve i mønstret Vævning af Byssus. Og du skal tilvirke et Hovedklæde af Byssus og et Bælte i broget Vævning.
\par 40 Også til Arons Sønner skal du tilvirke Kjortler, og du skal tilvirke Bælter til dem og Huer til Ære og Pryd.
\par 41 Og du skal iføre din Broder Aron og hans Sønner dem, og du skal salve dem, indsætte dem og hellige dem til at gøre Præstetjeneste for mig.
\par 42 Tillige skal du tilvirke Linnedbenklæder til dem til at skjule deres Blusel, og de skal nå fra Hoften ned på Lårene.
\par 43 Dem skal Aron og hans Sønner bære, når de går ind i Åbenbaringsteltet eller træder frem til Alteret for at gøre Tjeneste i Helligdommen, at de ikke skal pådrage sig Skyld og lide Døden.

\chapter{29}

\par 1 Således skal du bære dig ad med dem, når du helliger dem til at gøre Præstetjeneste for mig: Tag en ung Tyr, to lydefri Vædre,
\par 2 usyrede Brød, usyrede Kager, rørte i Olie, og usyrede Fladbrød, smurte med Olie; af fint Hvedemel skal du bage dem.
\par 3 Læg dem så i een Kurv og bær dem frem i Kurven sammen med Tyren og de to Vædre.
\par 4 Lad derpå Aron og hans Sønner træde hen til Åbenbaringsteltets Indgang og tvæt dem med Vand.
\par 5 Tag så Klæderne og ifør Aron Kjortelen, Efodkåben, Efoden og Brystskjoldet og bind Efoden fast på ham med Bæltet.
\par 6 Læg Hovedklædet om hans Hoved og fæst det hellige Diadem på Hovedklædet.
\par 7 Tag så Salveolien og udgyd den på hans Hoved og salv ham.
\par 8 Lad dernæst hans Sønner træde frem og ifør dem Kjortler,
\par 9 omgjord dem med Bælter og bind Huerne på dem. Og Præsteværdigheden skal tilhøre dem med evig Ret. Så skal du indsætte Aron og hans Sønner.
\par 10 Før Tyren frem foran Åbenbaringsteltet, og Aron og hans Sønner skal lægge deres Hænder på Tyrens Hoved.
\par 11 Slagt så Tyren for HERRENs Åsyn ved Indgangen til Åbenbaringsteltet
\par 12 og tag noget af Tyrens Blod og stryg det på Alterets Horn med din Finger og udgyd Resten af Blodet ved Alterets Fod.
\par 13 Tag så alt Fedtet på Indvoldene, Leverlappen og begge Nyrerne med Fedtet på dem og bring det som Røgoffer på Alteret;
\par 14 men Tyrens Kød, dens Hud og dens Skarn skal du brænde uden for Lejren. Det er et Syndoffer.
\par 15 Derpå skal du tage den ene Væder, og Aron og hans Sønner skal lægge deres Hænder på dens Hoved.
\par 16 Slagt så Væderen, tag dens Blod og spræng det rundt om på Alteret.
\par 17 Skær så Væderen i Stykker, tvæt dens Indvolde og Skinneben, læg dem på Stykkerne og Hovedet
\par 18 og bring så hele Væderen som Røgoffer på Alteret. Det er et Brændoffer for HERREN; en liflig Duft, et Ildoffer for HERREN er det.
\par 19 Derpå skal du tage den anden Væder, og Aron og hans Sønner skal lægge deres Hænder på dens Hoved.
\par 20 Slagt så Væderen, tag noget af dens Blod og stryg det på Arons og hans Sønners højre Øreflip og på deres højre Tommelfinger og højre Tommeltå og spræng Resten af Blodet rundt om på Alteret.
\par 21 Tag så noget af Blodet på Alteret og af Salveolien og stænk det på Aron og hans Klæder, ligeledes på hans Sønner og deres Klæder. så bliver han hellig, han selv og hans Klæder og ligeledes hans Sønner og deres klæder.
\par 22 Derpå skal du tage Fedtet af Væderen, Fedthalen, Fedtet på Indvoldene, Leverlappen, begge Nyrerne med Fedtet på dem, dertil den højre Kølle, thi det er en Indsættelsesvæder,
\par 23 og en Skive Brød, en Oliebrødkage og et Fladbrød af Kurven med de usyrede Brød, som står for HERRENs Åsyn,
\par 24 og lægge det alt sammen på Arons og hans Sønners Hænder og lade dem udføre Svingningen dermed for HERRENs Åsyn.
\par 25 Tag det så igen fra dem og bring det som Røgoffer på Alteret oven på Brændofferet til en liflig Duft for HERRENs Åsyn, et Ildoffer er det for HERREN.
\par 26 Tag derpå Brystet af Arons Indsættelsesvæder og udfør Svingningen dermed for HERRENs Åsyn: det skal være din Del.
\par 27 Således skal du hellige Svingningsbrystet og Offerydelseskøllen. det, hvormed Svingningen udføres. og det, som ydes af Arons og hans Sønners Indsættelsesvæder.
\par 28 Og det skal tilfalde Aron og hans Sønner som en Rettighed, de har Krav på fra Israeliternes Side til evig Tid; thi det er en Offerydelse, og som Offerydelse skal Israeliterne give det af deres Takofre, som deres Offerydelse til HERREN.
\par 29 Arons hellige Klæder skal tilfalde hans Sønner efter ham, for at de kan salves og indsættes i dem.
\par 30 I syv Dage skal de bæres af den af hans Sønner, som bliver Præst i hans Sted, den, som skal gå ind i Åbenbaringsteltet for at gøre Tjeneste i Helligdommen.
\par 31 Så skal du tage Indsættelsesvæderen og koge dens Kød på et helligt Sted;
\par 32 og Aron og hans Sønner skal spise Væderens Kød og Brødet i Kurven ved Indgangen til Åbenbaringsteltet;
\par 33 de skal spise de Stykker, hvorved der skaffes Soning ved deres Indsættelse og Indvielse, og ingen Lægmand må spise deraf, thi det er helligt.
\par 34 Og dersom der bliver noget af Indsættelseskødet eller Brødet tilovers til næste Morgen, da skal du opbrænde det tiloversblevne; spises må det ikke, thi det er helligt.
\par 35 Således skal du forholde dig over for Aron og hans Sønner, ganske som jeg har pålagt dig. Syv Dage skal du foretage Indsættelsen;
\par 36 daglig skal du ofre en Syndoffertyr til Soning og rense Alteret for Synd ved at fuldbyrde Soningen på det, og du skal salve det for at hellige det.
\par 37 Syv dage skal du fuldbyrde Soningen på Alteret og hellige det; således bliver Alteret højhelligt; enhver, der kommer i Berøring med Alteret, bliver hellig".
\par 38 Hvad du skal ofre på Alteret, er følgende: Hver Dag to årgamle Lam som stadigt Offer.
\par 39 Det ene Lam skal du ofre om Morgenen og det andet ved Aftenstid.
\par 40 Sammen med det første Lam skal du bringe en Tiendedel Efa fint Hvedemel, rørt i en Fjerdedel Hin Olie af knuste Oliven, og et Drikoffer af en Fjerdedel Hin Vin.
\par 41 Og det andet Lam skal du ofre ved Aftenstid; sammen med det skal du ofre et Afgrødeoffer og et Drikoffer som om Morgenen til en liflig Duft, et Ildoffer for HERREN.
\par 42 Det skal være et stadigt Brændoffer, som I skal bringe, Slægt efter Slægt, ved Indgangen til Åbenbaringsteltet for HERRENs Åsyn, hvor jeg vil åbenbare mig for dig for at tale til dig,
\par 43 og hvor jeg vil åbenbare mig for Israels Børn, og det skal helliges ved min Herlighed.
\par 44 Jeg vil hellige Åbenbaringsteltet og Alteret, og Aron og hans Sønner vil jeg hellige til at gøre Præstetjeneste for mig.
\par 45 Og jeg vil bo midt iblandt Israels Børn og være deres Gud;
\par 46 og de skal kende, at jeg HERREN er deres Gud, som førte dem ud af Ægypten for at bo midt iblandt dem, jeg HERREN deres Gud!

\chapter{30}

\par 1 Fremdeles skal du lave et Alter til at brænde Røgelse på: af Akacietræ skal du lave det,
\par 2 en Alen langt og en Alen bredt, firkantet skal det være, og to Alen højt, og dets Horn skal være i eet med det.
\par 3 Du skal overtrække det med purt Guld, både Pladen og Siderne hele Vejen rundt og Hornene, og sætte en Guldkrans rundt om;
\par 4 og du skal sætte to Guldringe under Kransen på begge Sider, på begge Sidestykkerne skal du sætte dem til at stikke Bærestænger i, for at det kan bæres med dem;
\par 5 og Bærestængerne skal du lave af Akacietræ og overtrække med Guld.
\par 6 Derpå skal du opstille det foran Forhænget, der hænger foran Vidnesbyrdets Ark, foran Sonedækket oven over Vidnesbyrdet der, hvor jeg vil åbenbare mig for dig.
\par 7 På det skal Aron brænde vellugtende Røgelse; hver Morgen, når han gør Lamperne i Stand, skal han antænde den.
\par 8 Og når Aron sætter Lamperne på Lysestagen ved Aftenstid, skal han ligeledes antænde den; det skal være et stadigt Røgelseoffer for HERRENs Åsyn fra Slægt til Slægt.
\par 9 I må ikke ofre et lovstridigt Røgelseoffer derpå, ej heller Brændofre eller Afgrødeofre, lige så lidt som I må udgyde Drikofre derpå.
\par 10 Men een Gang om Året skal Aron skaffe Soning på dels Horn; med noget af Forsoningssyndofferets Blod skal han een Gang om Året skaffe Soning på det, Slægt efter Slægt. Det er højhelligt for HERREN.
\par 11 HERREN talede fremdeles til Moses og sagde:
\par 12 Når du holder Mandtal over Israeliterne, skal enhver, som mønstres, ved Mønstringen give HERREN Sonepenge for sit Liv, at ingen Ulykke skal ramme dem i Anledning af Mønstringen.
\par 13 Enhver, der må underkaste sig Mønstringen, skal udrede en halv Sekel i hellig Mønt, tyve Gera på en Sekel, en halv Sekel som Offerydelse til HERREN.
\par 14 Enhver, der må underkaste sig Mønstringen, fra Tyveårsalderen og opefter, skal udrede HERRENs Offerydelse.
\par 15 hen rige må ikke give mere, den fattige ikke mindre end en halv Sekel, når de bringer HERRENs Offerydelse til Soning for deres Sjæle.
\par 16 Og du skal tage Sonepengene af Israeliterne og bruge dem til Tjenesten ved Åbenbaringsteltet. og de skal tjene til at bringe Israeliterne i Minde for HERRENs Åsyn, til Soning for eders Sjæle.
\par 17 HERREN talede fremdeles til Moses og sagde:
\par 18 Du skal lave en Vandkumme med Fodstykke af Kobber til at tvætte sig i op opstille den mellem Åbenbaringsteltet og Alteret og hælde Vand i den,
\par 19 for at Aron og hans Sønner kan tvætte deres Hænder og Fødder deri.
\par 20 Når de går ind i Åbenbaringsteltet, skal de tvætte sig med Vand for ikke at dø; ligeledes når de træder hen til Alteret for at gøre Tjeneste og brænde Ildofre for HERREN.
\par 21 De skal tvætte deres Hænder og Fødder for ikke at dø. Det skal være en evig Anordning for ham og hans Afkom fra Slægt til Slægt.
\par 22 HERREN talede fremdeles til Moses og sagde:
\par 23 Du skal tage dig vellugtende Stoffer af den bedste Slags, 500 Sekel ædel Myrra, halvt så meget. 25O Sekel, vellugtende Kanelbark, 25O Sekel vellugtende Kalmus
\par 24 og 500 Sekel Kassia, efter hellig Vægt, og en Hin Olivenolie.
\par 25 Deraf skal du tilberede en hellig Salveolie, en krydret Blanding, som Salveblanderne laver den; en hellig Salveolie skal det være.
\par 26 Med den skal du salve Åbenbaringsteltet, Vidnesbyrdets Ark,
\par 27 Bordet med alt dets Tilbehør, Lysestagen med dens Tilbehør, Røgelsealteret,
\par 28 Brændofferalteret med alt dets Tilbehør og Vandkummen med dens Fodstykke,
\par 29 Således skal du hellige dem, så de bliver højhellige. Enhver, der kommer i Berøring med dem, bliver hellig" .
\par 30 Ligeledes skal du salve Aron og hans Sønner og hellige dem til at gøre Præstetjeneste for mig.
\par 31 Men til Israeliterne skal du sige således: Dette skal være mig en hellig Salveolie fra Slægt til Slægt.
\par 32 Den må ikke udgydes på noget Menneskes Legeme, og i denne Blanding må I ikke tilberede lignende Salve til eget Brug, hellig er den, og hellig skal den være eder.
\par 33 Den, der tilbereder lignende Salve eller anvender den på en Lægmand, skal udryddes af sin Slægt.
\par 34 HERREN talede fremdeles til Moses og sagde: Tag dig Røgelseoffer, Stakte, Onyksmusling, Galbanum og ren Virak, lige meget af hvert,
\par 35 og tilbered deraf en krydret Røgelse, som Salveblanderne laver den, saltet, ren, til hellig Brug.
\par 36 Deraf skal du støde en Del til Pulver, og noget deraf skal du lægge foran Vidnesbyrdet i Åbenbaringsteltet, hvor jeg vil åbenbare mig for dig. Det skal være eder højhelligt.
\par 37 Den Røgelse, du tilbereder i denne Blanding, må I ikke tilberede til eget Brug. Hellig skal den være dig for HERREN.
\par 38 Den, der tilbereder lignende Røgelse for at nyde dens Duft, skal udryddes af sin Slægt.

\chapter{31}

\par 1 HERREN talede fremdeles til Moses og sagde:
\par 2 Se, jeg har kaldet Bezalel, en Søn af Hurs Søn Uri, af Judas Stamme
\par 3 og fyldt ham med Guds Ånd, med Kunstsnilde, Kløgt og Indsigt i alskens Arbejde
\par 4 til at udtænke Kunstværker og til at arbejde i Guld, Sølv og Kobber
\par 5 og med Udskæring af Sten til Indfatning og med Træskærerarbejde, kort sagt til at udføre alskens Arbejde.
\par 6 Og se, jeg har givet ham Oholiab, Ahisamaks Søn, af Dans Stamme til Medhjælper, og alle kunstforstandige Mænds Hjerte har jeg udrustet med Kunstsnilde, for at de kan udføre alt, hvad jeg har pålagt dig,
\par 7 Åbenbaringsteltet, Vidnesbyrdets Ark, Sonedækket derpå og alt Teltets Tilbehør,
\par 8 Bordet med dets Tilbehør, Lysestagen af purt Guld med alt dens Tilbehør, Røgelsealteret,
\par 9 Brændofferalteret med alt dets Tilbehør og Vandkummen med dens Fodstykke,
\par 10 Pragtklæderne, de hellige Klæder til Præsten Aron og hans Sønners Klædet til Brug ved Præstetjenesten,
\par 11 Salveolien og den vellugtende Røgelse til Helligdommen.
\par 12 HERREN talede fremdeles til Moses og sagde:
\par 13 Du skal tale til Israeliterne og sige: Fremfor alt skal I holde mine Sabbater, thi Sabbaten er et Tegn mellem mig og eder fra Slægt til Slægt, for at I skal kende, at jeg HERREN er den, der helliger eder.
\par 14 I skal holde Sabbaten, thi den skal være eder hellig; den, som vanhelliger den, skal lide Døden, ja enhver, som udfører noget Arbejde på den, det Menneske skal udryddes af sin Slægt.
\par 15 I seks Dage må der arbejdes, men på den syvende Dag skal I holde en fuldkommen Hviledag, helliget HERREN; enhver, som udfører Arbejde på Sabbatsdagen, skal lide Døden.
\par 16 Israeliterne skal holde Sabbaten, så at de fejrer Sabbaten fra Slægt til Slægt som en evig gyldig Pagt:
\par 17 Den skal være et Tegn til alle Tider mellem mig og Israeliterne. Thi i seks Dage gjorde HERREN Himmelen og Jorden, men på den syvende hvilede han og vederkvægede sig.
\par 18 Da han nu var færdig med at tale til Moses på Sinaj Bjerg, overgav han ham Vidnesbyrdets to Tavler, Stentavler, der var beskrevet med Guds Finger.

\chapter{32}

\par 1 Men da Folket så, at Moses tøvede med at komme ned fra Bjerget, samlede det sig om Aron, og de sagde til ham: "Kom og lav os en Gud, som kan drage foran os, thi vi ved ikke, hvad der er blevet af denne Moses, der førte os ud af Ægypten!"
\par 2 Da sagde Aron til dem: "Riv de Guldringe af, som eders Hustruer, Sønner og Døtre har i Ørene, og bring mig dem!"
\par 3 Så rev hele Folket deres Guldørenringe af og bragte dem til Aron.
\par 4 Og han modtog dem af deres Hånd, formede Guldet med en Mejsel og lavede en støbt Tyrekalv deraf. Da sagde de: "Her, Israel, er din Gud, som førte dig ud af Ægypten!"
\par 5 Og da Aron så det, byggede han et Alter for den, og Aron lod kundgøre: "I Morgen er det Højtid for HERREN!"
\par 6 Tidligt næste Morgen ofrede de så Brændofre og bragte Takofre og Folket satte sig til at spise og drikke, og derpå stod de op for at lege.
\par 7 Da sagde HERREN til Moses: "Skynd dig og stig ned, thi dit Folk, som du førte ud af Ægypten, har handlet ilde;
\par 8 hastigt veg de bort fra den Vej jeg bød dem at vandre; de har lavet sig en støbt Tyrekalv og tilbedt den og ofret til den med de Ord: Her, Israel, er din Gud, som førte dig ud af Ægypten!"
\par 9 Og HERREN sagde til Moses: "Jeg har iagttaget dette Folk og set, at det er et halsstarrigt Folk.
\par 10 Lad mig nu råde, at min Vrede kan blusse op imod dem så vil jeg tilintetgøre dem; men dig vil jeg gøre til et stort Folk!"
\par 11 Men Moses bønfaldt HERREN sin Gud og sagde: "Hvorfor HERRE skal din Vrede blusse op mod dit Folk, som du førte ud af Ægypten med vældig Kraft og stærk Hånd?
\par 12 Hvorfor skal Ægypterne kunne sige: I ond Hensigt førte han dem ud, for at slå dem ihjel ude mellem Bjergene og udrydde dem af Jorden? Lad din Vredes Glød høre op, og anger den Ulykke, du vilde gøre dit Folk!
\par 13 Kom Abraham, Isak og Israel i Hu, dine Tjenere, hvem du tilsvor ved dig selv: Jeg vil gøre eders Afkom talrigt som Himmelens Stjerner, og jeg vil give eders Afkom hele det Land, hvorom jeg har talet, og de skal eje det evindelig!"
\par 14 Da angrede HERREN den Ulykke han havde truet med at gøre sit Folk.
\par 15 Derpå vendte Moses tilbage og steg ned fra Bjerget med Vidnesbyrdets to Tavler i Hånden, Tavler, der var beskrevet på begge Sider, både på Forsiden og Bagsiden var de beskrevet.
\par 16 Og Tavlerne var Guds Værk, og Skriften var Guds Skrift, ridset ind i Tavlerne.
\par 17 Da hørte Josua Støjen af det larmende Folk, og han sagde til Moses: "Der høres Krigslarm i Lejren!"
\par 18 Men han svarede: "Det er ikke sejrendes eller slagnes Skrig, det er Sang, jeg hører!"
\par 19 Og da Moses nærmede sig Lejren og så Tyrekalven og Dansen, blussede hans Vrede op, og han kastede Tavlerne fra sig og sønderslog dem ved Bjergets Fod.
\par 20 Derpå tog han Tyrekalven, som de havde lavet, brændte den i Ilden og knuste den til Støv, strøde det på Vandet og lod Israeliterne drikke det.
\par 21 Og Moses sagde til Aron: "Hvad har dette Folk gjort dig, siden du har bragt så stor en Synd over det?"
\par 22 Aron svarede: "Vredes ikke, Herre! Du ved selv, at Folket ligger i det onde,
\par 23 og de sagde til mig: Lav os en Gud, som kan drage foran os, thi vi ved ikke, hvad der er blevet af denne Moses, der førte os ud af Ægypten!
\par 24 Da sagde jeg til dem: De, der har Guldsmykker, skal rive dem af! De bragte mig da Guldet, og jeg kastede det i Ilden, og så kom denne Tyrekalv ud deraf!"
\par 25 Da Moses nu så, at Folket var tøjlesløst til Skadefryd for deres Fjender, fordi Aron havde givet det fri Tøjler,
\par 26 stillede han sig ved Indgangen til Lejren og sagde: "Hvem der er for HERREN, han komme hid til mig!" Da samlede alle Leviterne sig om ham,
\par 27 og han sagde til dem: "Så siger HERREN, Israels Gud: Bind alle Sværd om Lænd og gå frem og tilbage fra den ene Indgang i Lejren til den anden og slå ned både Broder, Ven og Frænde!"
\par 28 Og Leviterne gjorde, som Moses havde sagt, og på den Dag faldt der af Folket henved 3000 Mand.
\par 29 Og Moses sagde: "Fra i Dag af skal I være Præster for HERREN, thi ingen skånede Søn eller Broder, derfor skal Velsignelse komme over eder i Dag."
\par 30 Næste Dag sagde Moses til Folket: "I har begået en stor Synd ; men nu vil jeg stige op til HERREN, måske kan jeg skaffe Soning for eders Synd!"
\par 31 Derpå gik Moses atter til HERREN og sagde: "Ak, dette Folk har begået en stor Synd, de har lavet sig en Gud af Guld.
\par 32 Om du dog vilde tilgive dem deres Synd! Hvis ikke, så udslet mig af den Bog, du fører!"
\par 33 HERREN svarede Moses: "Den, som har syndet imod mig, ham vil jeg udslette af min Bog!
\par 34 Men gå nu og før Folket hen, hvor jeg har befalet dig at føre det hen; se, min Engel skal drage foran dig! Men til sin Tid vil jeg straffe dem for deres Synd!"
\par 35 Og HERREN slog Folket, fordi de havde lavet Tyrekalven, den, Aron lavede.

\chapter{33}

\par 1 HERREN sagde til Moses: "Drag nu bort herfra med Folket, som du førte ud af Ægypten, til det Land, jeg tilsvor Abraham, Isak og Jakob med de Ord: Dit Afkom vil jeg give det!
\par 2 Jeg sender en Engel foran dig, og han skal drive Kana'anæerne, Amoriterne, Hetiterne, Perizziterne, Hivviterne og Jebusiterne bort
\par 3 til et Land, der flyder med Mælk og Honning. Men selv vil jeg ikke drage med i din Midte, thi du er et halsstarrigt Folk; drog jeg med, kunde jeg tilintetgøre dig undervejs!"
\par 4 Da Folket hørte denne onde Tidende, sørgede de, og ingen tog sine Smykker på.
\par 5 Da sagde HERREN til Moses: "Sig til Israeliterne; I er et halsstarrigt Folk! Vandrede jeg kun et eneste Øjeblik i din Midte, måtte jeg tilintetgøre dig. Tag du dine Smykker af, så skal jeg tænke over, hvad jeg vil gøre for dig!"
\par 6 Da aflagde Israeliterne deres Smykker fra Horebs Bjerg af.
\par 7 Moses plejede at tage Teltet og slå det op udenfor Lejren i nogen Afstand derfra; han gav det Navnet "Åbenbaringsteltet".
\par 8 Men hver Gang Moses gik ud til teltet, rejste alt Folket sig op og stillede sig alle ved Indgangen til deres Telte og så efter Moses, indtil han kom ind i Teltet.
\par 9 Og når Moses kom ind Teltet, sænkede Skystøtten sig og stillede sig ved Indgangen til Teltet; da talede HERREN med Moses.
\par 10 Men når alt Folket så Skystøtten stå ved Indgangen til Teltet, rejste de sig alle op og tilbad ved Indgangen til deres Telte.
\par 11 Så talede HERREN med Moses Ansigt til Ansigt, som når den ene Mand taler med den anden, og derpå vendte Moses tilbage til Lejren; men hans Medhjælper Josua, Nuns Søn, en ung Mand, veg ikke fra Teltet.
\par 12 Moses sagde til HERREN: "Se, du siger til mig: Før dette Folk frem! Men du har ikke ladet mig vide, hvem du vil sende med mig; og dog har du sagt: Jeg kender dig ved Navn, og du har fundet Nåde for mine Øjne!
\par 13 Hvis jeg nu virkelig har fundet Nåde for dine Øjne, så lær mig dine Veje at kende, at jeg kan kende dig og finde Nåde for dine Øjne; tænk dog på, at dette Folk er dit Folk!"
\par 14 Han svarede: "Skal mit Åsyn da vandre med, og skal jeg således føre dig til Målet?"
\par 15 Da sagde Moses til ham: "Hvis dit Åsyn ikke vandrer med, så lad os ikke drage herfra!
\par 16 Hvorpå skal det dog kendes. at jeg har fundet Nåde for dine Øjne, jeg og dit Folk? Mon ikke på, at du vandrer med os, og der således vises os, mig og dit Folk, en Udmærkelse fremfor alle and1e Folkeslag på Jorden?"
\par 17 HERREN svarede Moses: "Også hvad du der siger, vil jeg gøre, thi du har fundet Nåde for mine Øjne, og jeg kalder dig ved Navn."
\par 18 Da sagde Moses: "Lad mig dog skue din Herlighed!"
\par 19 Han svarede: "Jeg vil lade al min Rigdom drage forbi dig og udråbe HERRENs Navn foran dig, thi jeg viser Nåde, mod hvem jeg vil, og Barmhjertighed, mod hvem jeg vil!"
\par 20 Og han sagde: "Du kan ikke skue mit Åsyn, thi intet Menneske kan se mig og leve."
\par 21 Og HERREN sagde: "Se, her er et Sted i min Nærhed, stil dig på Klippen der!
\par 22 Når da min Herlighed drager forbi, vil jeg lade dig stå i Klippehulen, og jeg vil dække dig 1ned min Hånd, indtil jeg er kommet forbi.
\par 23 Så tager jeg min Hånd bort, og da kan du se mig bagfra; men mit Åsyn kan ingen skue!"

\chapter{34}

\par 1 Derpå sagde HERREN til Moses: "Tilhug dig to Stentavler ligesom de forrige, så vil jeg på Tavlerne skrive de samme Ord, som stod på de forrige Tavler, du slog i Stykker.
\par 2 Gør dig så rede til i Morgen, stig om Morgenen op på Sinaj Bjerg og stil dig hen og vent på mig der på Bjergets Top.
\par 3 Ingen må følge med dig derop, og ingen må vise sig noget Sted på Bjerget, end ikke Småkvæg eller Hornkvæg må græsse i Nærheden af dette Bjerg."
\par 4 Da tilhuggede han to Stentavler ligesom de forrige, og tidligt næste Morgen steg Moses op på Sinaj Bjerg, som Gud havde pålagt ham, og tog de to Stentavler med sig.
\par 5 Da steg HERREN ned i Skyen; og Moses stillede sig hos ham der og påkaldte HERRENs Navn.
\par 6 Og HERREN gik forbi ham og råbte: "HERREN, HERREN, Gud, som er barmhjertig og nådig, langmodig og rig på Miskundhed og Trofasthed,
\par 7 som bevarer Miskundhed mod Tusinder, som tilgiver Brøde, Overtrædelse og Synd, men ikke lader den skyldige ustraffet, som straffer Fædres Brøde på Børn og Børnebørn, på dem i tredje og fjerde Led!"
\par 8 Da bøjede Moses sig hastelig til Jorden, tilbad
\par 9 og sagde: "Har jeg fundet Nåde for dine Øjne, Herre, så drage min Herre med i vor Midte; thi det er et halsstarrigt Folk. Men tilgiv vor Brøde og vor Synd og lad os være din Ejendom!"
\par 10 Han sagde: "Se, jeg vil slutte en Pagt; i hele dit Folks Påsyn vil jeg gøre Undere, som aldrig før er sket nogensteds på Jorden og blandt noget Folkeslag, og hele det Folk, i hvis Midte du lever, skal se HERRENs Værk; thi det, jeg vil udføre ved dig, er forfærdeligt.
\par 11 Hold dig det efterrettelig, som jeg i Dag byder dig! Se, jeg vil drive Amoriterne, Kana'anæerne, Hetiterne, Perizziterne, Hivviterne og Jebusiterne bort foran dig!
\par 12 Vogt dig vel for at slutte nogen Pagt med Indbyggerne i det Land, du kommer til, for at de ikke skal blive en Snare for dig, når de lever i din Midte.
\par 13 Men I skal nedbryde deres Altre, sønderslå deres Stenstøtter og omhugge deres Asjerastøtter!
\par 14 Thi du må ikke tilbede nogen anden Gud, thi "Nidkær" er HERRENs Navn, nidkær Gud er han.
\par 15 Du må ikke slutte Pagt med Landets Indbyggere, og når de boler med deres Guder og ofrer til dem og man indbyder dig til at være med, må du ikke spise af deres Ofre;
\par 16 og du må ikke af deres Døtre tage Hustruer til dine Sønner, så deres Døtre, når de boler med deres Guder, får dine Sønner til også at bole med dem.
\par 17 Du må ikke gøre dig noget støbt Gudebillede.
\par 18 Du skal lejre de usyrede Brøds Højtid; i syv Dage skal du spise usyret Brød, som jeg har pålagt dig, lå den fastsatte Tid i Abib Måned, thi i Abib Måned drog du ud af Ægypten.
\par 19 Alt førstefødt tilhører mig; af dine Hjorde skal du ofre mig det førstefødte af Handyrene, både af Okset og småt Kvæg;
\par 20 men de førstefødte Æsler skal du udløse med et Stykke småt Kvæg, og hvis du ikke udløser det, skal du sønderbryde Halsen derpå; alle dine førstefødte Sønner skal du udløse. Du må ikke stedes for mit Åsyn med tomme Hænder
\par 21 I seks Dage må du arbejde, men på den syvende skal du hvile; i Pløje og Høsttiden skal du holde Hviledag.
\par 22 Du skal fejre Ugehøjtid med Førstegrøden af Hvedehøsten og Frugthøsthøjtid ved Jævndøgnstide.
\par 23 Tre Gange om Året skal alle at Mandkøn hos dig stedes for den Herre HERREN Israels Guds Åsyn.
\par 24 Thi jeg vil drive Folkeslag bort foran dig og gøre dine Landemærker vide, og ingen skal attrå dit Land, medens du drager hen for at stedes for HERREN din Guds Åsyn tre Gange 01n Året.
\par 25 Du må ikke ofre Blodet af mit offer sammen med syret Brød.
\par 26 Det bedste af din Jords Førstegrøde skal du bringe til HERREN din Guds Hus. Du må ikke koge et Hid i dets Moders Mælk!"
\par 27 Og HERREN sagde til Moses: "Skriv disse Ord op, thi på Grundlag af disse Ord slutter jeg Pagt med dig og Israel."
\par 28 Og han blev der hos HERREN fyrretyve Dage og fyrretyve Nætter uden at spise eller drikke; og han skrev Pagtsordene, de ti Ord, på Tavlerne.
\par 29 Da Moses steg ned fra Sinaj Bjerg med Vidnesbyrdets to Tavler i Hånden, vidste han ikke, at hans Ansigts Hud var kommet til at stråle, ved at han talede med ham.
\par 30 Men Aron og alle Israeliterne så Moses, og se, hans Ansigts Hud strålede, og de turde ikke komme ham nær.
\par 31 Men Moses kaldte på dem, og da vendte Aron og alle Menighedens Øverster tilbage til ham, og Moses talte til dem.
\par 32 Derpå kom alle Israeliterne hen til ham, og han pålagde dem alt, hvad HERREN havde talet til ham på Sinaj Bjerg.
\par 33 Men da Moses var færdig med at tale til dem, lagde han et Dække over sit Ansigt.
\par 34 Hver Gang han derefter trådte frem for HERRENs Åsyn for at tale med ham, tog han Sløret af, indtil han kom ud igen; og når han kom ud, meddelte han Israeliterne, hvad der var blevet ham påbudt.
\par 35 Da så Israeliterne, at Moses's Ansigts Hud strålede; og Moses lagde da Dækket over sit Ansigt, indtil han atter gik ind for at tale med ham.

\chapter{35}

\par 1 Moses kaldte hele Israeliternes Menighed sammen og sagde til dem: Dette er, hvad HERREN har pålagt eder at gøre:
\par 2 I seks Dage må der arbejdes, men på den syvende Dag skal I holde Helligdag, en fuldkommen Hviledag for HERREN. Enhver, der den Dag udfører noget Arbejde, skal lide Døden.
\par 3 På Sabbatsdagen må I ikke gøre Ild i nogen af eders Boliger.
\par 4 Derpå sagde Moses til hele Israeliternes Menighed: Dette er, hvad HERREN har påbudt:
\par 5 I skal tage en Offerydelse til HERREN af, hvad I ejer. Enhver, som i sit Hjerte føler sig tilskyndet dertil, skal komme med det, HERRENs Offerydelse, Guld, Sølv, Kobber,
\par 6 violet og rødt Purpurgatn, karmoisinrødt Garn, Byssus, Gedehår,
\par 7 rødfarvede Væderskind, Tahasjskind, Akacietræ,
\par 8 Olie til Lysestagen, vellugtende Stofer til Salveolien og Røgelsen,
\par 9 Sjohamsten og Ædelsten til Indfatning på Efoden og Brystskjoldet.
\par 10 Og alle kunstforstandige Mænd iblandt eder skal komme og lave alt, hvad HERREN har påbudt:
\par 11 Boligen med dens Teltdække og Dække, dens Kroge, Brædder, Tværstænger, Piller og Fodstykker,
\par 12 Arken med Bærestængerne, Sonedækket og det indre Forhæng,
\par 13 Bordet med dets Bærestænger og alt dets Tilbehør og Skuebrødene,
\par 14 Lysestagen med dens Tilbehør, dens Lamper og Olien til Lysestagen,
\par 15 Røgelsealteret med dets Bærestænger, Salveolien og Røgelsen.
\par 16 Brændofferalteret med Kobbergitteret, Bærestængerne og alt dets Tilbehør, Vandkummen med dens Fodstykke,
\par 17 Forgårdens Omhæng, dens Piller og Fodstykker og Forhænget til Forgårdens Indgang,
\par 18 Boligens og Forgårdens Pæle med Reb,
\par 19 Pragtklæderne til Tjenesten i Helligdommen, de hellige Klæder til Præsten Aron og hans Sønners Klæder til Brug ved Præstetjenesten.
\par 20 Da forlod hele Israeliternes Menighed Moses.
\par 21 Og enhver, som i sit Hjerte følte sig drevet dertil, og hvis Ånd tilskyndede ham, kom med HERRENs Offerydelse til Opførelsen af Åbenbaringsteltet og til alt Arbejdet derved og til de hellige Klæder.
\par 22 De kom dermed, både Mænd og Kvinder; enhver, som i sit Hjerte følte sig tilskyndet dertil, kom med Spænder, Ørenringe, Fingerringe og Halssmykker, alle Hånde Guldsmykker. Og enhver, der vilde vie HERREN en Gave af Guld, kom dermed.
\par 23 Og enhver, i hvis Eje der fandtes violet og rødt Purpurgarn, karmoisinrødt Garn, Byssus, Gedehår, rødfarvede Væderskind eller Tahasjskind, kom dermed.
\par 24 Og enhver, der vilde give en Offerydelse af Sølv eller Kobber, kom med HERRENs Offerydelse. Og enhver, der ejede Akacietræ til alt Byggearbejdet, kom dermed.
\par 25 Og alle kunstforstandige Kvinder spandt med egne Hænder og kom med deres Spind, violet og rødt Purpur, Karmoisin og Byssus.
\par 26 Og alle Kvinder, som i Kraft af deres Kunstsnilde følte sig tilskyndede dertil i deres Hjerte, spandt Gedehårene.
\par 27 Og Øversterne kom med Sjohamstenene og Ædelstenene til Indfatningen på Efoden og Brystskjoldet
\par 28 og de vellugtende Stoffer og Olien til Lysestagen og til Salveolien og Røgelsen.
\par 29 Enhver Mand og Kvinde af Israeliterne, som i sit Hjerte følte sig tilskyndet til at bringe, hvad der krævedes til Udførelsen af alt det Arbejde, HERREN gennem Moses havde påbudt, bragte det som en frivillig Gave til HERREN.
\par 30 Derpå sagde Moses til Israeliterne: Se, HERREN har kaldet Bezal'el, en Søn af Hurs Søn Uri, af Judas Stamme
\par 31 og fyldt ham med Guds Ånd, med Kunstsnilde, Kløgt og Indsigt i alskens Arbejde
\par 32 til at udtænke Kunstværker og til at arbejde i Guld, Sølv og Kobber
\par 33 og med Udskæring af Sten til Indfatning og med Træskærerarbejde, kort sagt til at udføre alskens Kunstarbejde.
\par 34 Og tillige har han givet både ham og Oholiab, Ahisamaks Søn, af Dans Stamme Gaver til at lære fra sig.
\par 35 Han har fyldt dem med Kunstsnilde til at udføre alskens Udskæringsarbejde, Kunstvævning, broget Vævning af violet og rødt Purpurgarn, karmoisinrødt Garn og Byssus og almindelig Vævning, så de kan udføre alt Slags Arbejde og udtænke Kunstværker.

\chapter{36}

\par 1 Derfor skal Bezal'el og Oholiab og alle andre kunstforstandige Mænd, hvem HERREN har givet Kunstsnilde og Kløgt, så de forstår sig på Arbejdet, udføre alt Arbejdet ved Helligdommens Opførelse i Overensstemmelse med alt, hvad HERREN har påbudt.
\par 2 Derpå tilkaldte Moses Bezal'el og Oholiab og alle de kunstforstandige Mænd, hvem HERREN havde givet Kunstsnilde, alle dem, som i deres Hjerte følte sig tilskyndet til at give sig i Lag med Udførelsen af Arbejdet.
\par 3 Og de modtog af Moses hele den Offerydelse, Israeliterne var kommet med til Arbejdet med Helligdommens Opførelse, for at det kunde blive udført. Men de blev ved at komme med frivillige Gaver til ham, Morgen efter Morgen.
\par 4 Da kom alle de kunstforstandige Mænd, der udførte alt Arbejdet ved Helligdommen, hver fra den Del af Arbejdet, han var beskæftiget med,
\par 5 og sagde til Moses: "Folket kommer med mere, end der kræves til Udførelsen af det Arbejde, HERREN har påbudt!"
\par 6 Da bød Moses, at følgende Kundgørelse skulde udråbes i Lejren: "Hverken Mænd eller Kvinder skal yde mere som Offergave til Helligdommen!" Så hørte Folket op med at komme med Gaver.
\par 7 Og det, der var ydet, var dem nok til at udføre hele Arbejdet, ja mer end nok.
\par 8 Så lavede alle de kunstforstandige Mænd blandt dem, der deltog i Arbejdet, Boligen, ti Tæpper af tvundet Byssus, violet og rødt Purpurgarn og karmoisinrødt Garn; han lavede dem med Keruber på i Kunstvævning,
\par 9 hvert Tæppe otte og tyve Alen langt og fire Alen bredt; alle Tæpperne havde samme Mål.
\par 10 Han syede Tæpperne sammen, fem og fem.
\par 11 I Kanten af det ene Tæppe, det yderste i det ene sammensyede Stykke, satte han Løkker af violet Purpurgarn, og ligeledes satte han Løkker i Kanten af det yderste Tæppe i det andet sammensyede Stykke;
\par 12 han satte halvtredsindstyve Løkker på det ene Tæppe og halvtredsindstyve Løkker i Kanten af det tilsvarende Tæppe i det andet sammensyede Stykke, Løkke lige over for Løkke.
\par 13 Derpå lavede han halvtredsindstyve Guldkroge til at forbinde Tæpperne med hinanden, så at Boligen udgjorde et Hele.
\par 14 Fremdeles lavede han Tæpper af Gedehår til et Teltdække uden om Boligen, og her lavede han elleve Tæpper,
\par 15 hvert Tæppe tredive Alen langt og fire Alen bredt; alle Tæpperne havde samme Mål.
\par 16 De fem af Tæpperne syede han sammen for sig og de seks for sig,
\par 17 og han satte halvtredsindstyve Løkker i Kanten af det yderste Tæppe i det ene sammensyede Stykke og halvtredsindstyve Løkker i Kanten af det tilsvarende Tæppe i det andet sammensyede Stykke.
\par 18 Og han lavede halvtredsindstyve Kobberkroge til at sammenføje Teltdækket med, så det udgjorde et Hele.
\par 19 Fremdeles lavede han over Teltdækket et Dække af rødfarvede Væderskind og derover endnu et Dække af Tahasjskind.
\par 20 Derpå lavede han Brædderne til Boligen af Akacietræ til at stå op,
\par 21 hvert Bræt ti Alen højt og halvanden Alen bredt,
\par 22 og på hvert Bræt to indbyrdes forbundne Tapper; således indrettede han det ved alle Boligens Brædder.
\par 23 Af Brædderne, som han lavede til Boligen, var tyve til Sydsiden,
\par 24 og til de tyve Brædder lavede han fyrretyve Fodstykker af Sølv, to Fodstykker til de to Tapper på hvert Bræt.
\par 25 Andre tyve Brædder lavede han til Boligens anden Side, som vendte mod Nord,
\par 26 med fyrretyve Fodstykker af Sølv, to Fodstykker til hvert Bræt.
\par 27 Og til Bagsiden, som vendte mod Vest, lavede han seks Brædder.
\par 28 Til Boligens Baghjørner lavede han to Brædder,
\par 29 der bestod af to Stykker forneden og ligeledes af to Stykker foroven, indtil den første Ring; således indrettede han dem begge for at danne de to Hjørner.
\par 30 Altså blev der til Bagsiden otte Brædder med tilhørende seksten Fodstykker af Sølv, to til hvert Bræt.
\par 31 Derpå lavede han Tværstænger af Akacietræ, fem til de Brædder, der dannede Boligens ene Side,
\par 32 fem til de Brædder, der dannede Boligens anden Side, og fem til de Brædder, der dannede Boligens Bagside mod Vest;
\par 33 den mellemste Tværstang lavede han således, at den midt på Brædderne nåede fra den ene Ende af Væggen til den anden.
\par 34 Brædderne overtrak han med Guld, og deres Ringe, som Tværstængerne skulde stikkes i, lavede han af Guld, og Tværstængerne overtrak han med Guld.
\par 35 Derpå lavede han Forhænget af violet og rødt Purpurgarn, karmoisinrødt Garn og tvundet Byssus, han lavede det i Kunstvævning med Keruber på,
\par 36 og han lavede dertil fire Piller af Akacietræ, som han overtrak med Guld, og Knagerne derpå lavede han af Guld, og han støbte fire Fodstykker af Sølv til dem.
\par 37 Derpå lavede han et Forhæng til Teltets Indgang af violet og rødt Purpurgarn, karmoisinrødt Garn og tvundet Byssus i broget Vævning
\par 38 og dertil fem Piller med Knager, hvis Hoveder og Bånd han overtrak med Guld, og fem Fodstykker af Kobber.

\chapter{37}

\par 1 Derpå lavede Bezal'el Arken af Akacietræ, halvtredje Alen lang, halvanden Alen bred og halvanden Alen høj,
\par 2 og overtrak den indvendig og udvendig med purt Guld og satte en gylden Krans rundt om den.
\par 3 Derefter støbte han fire Guldringe til den og satte dem på dens fire Fødder, to Ringe på hver Side af den.
\par 4 Og han lavede Bærestænger af Akacietræ og overtrak dem med Guld;
\par 5 så stak han Stængerne gennem Ringene på Arkens Sider, for at den kunde bæres med dem.
\par 6 Derpå lavede han Sonedækket af purt Guld, halvtredje Alen langt og halvanden Alen bredt,
\par 7 og han lavede to Keruber af Guld, i drevet Arbejde lavede han dem, ved begge Ender af Sonedækket,
\par 8 den ene Kerub ved den ene Ende, den anden Kerub ved den anden; han lavede Keruberne således, at de var i eet med Sonedækket ved begge Ender.
\par 9 Og Keruberne bredte deres Vinger i Vejret, således at de dækkede over Sonedækket med deres Vinger; de vendte Ansigtet mod hinanden; nedad mod Sone,dækket vendte Kerubernes Ansigter.
\par 10 Derpå lavede han Bordet af Akacietræ, to Alen langt, en Alen bredt og halvanden Alen højt,
\par 11 og overtrak det med purt Guld og satte en gylden Krans rundt om det.
\par 12 Og han satte en Liste af en Hånds Bredde rundt om det og en gylden Krans rundt om Listen.
\par 13 Og han støbte fire Guldringe og satte dem på de fire Hjørner ved dets fire Ben.
\par 14 Lige ved Listen sad Ringene til at stikke Bærestængerne i, så at man kunde bære Bordet.
\par 15 Og han lavede Bærestængerne at Akacietræ og overtrak dem med Guld, og med dem skulde Bordet bæres.
\par 16 Og han lavede af purt Guld de Ting, som hørte til Bordet, Fadene og Kanderne, Skålene og Krukkerne til at udgyde Drikoffer med.
\par 17 Derpå lavede han Lysestagen af purt Guld, i drevet Arbejde lavede han Lysestagen, dens Fod og selve Stagen, således at dens Blomster med Bægere og Kroner var i eet med den;
\par 18 seks Arme udgik fra Lysestagens Sider, tre fra den ene og tre fra den anden Side.
\par 19 På hver af Armene, der udgik fra Lysestagen, var der tre mandelblomstlignende Blomster med Bægere og Kroner,
\par 20 men på selve Stagen var der fire mandelblomstlignende Blomster med Bægere og Kroner,
\par 21 et Bæger under hvert af de tre Par Arme, der udgik fra den.
\par 22 Bægrene og Armene var i eet med den, så at det hele udgjorde eet drevet Arbejde af purt Guld.
\par 23 Derpå lavede han de syv Lamper til den, Lampesaksene og Bakkerne af purt Guld.
\par 24 En Talent purt Guld brugte han til den og til alt dens Tilbehør.
\par 25 Derpå lavede han Røgelsealteret af Akacietræ, en Alen langt og en Alen bredt, i Firkant, og to Alen højt, og dets Horn var i eet med det.
\par 26 Og han overtrak det med purt Guld, både Pladen og Siderne hele Vejen rundt og Hornene, og satte en Guldkrans rundt om;
\par 27 og han satte to Guldringe under Kransen på begge Sider til at stikke Bærestængerne i, for at det kunde bæres med dem;
\par 28 Bærestængerne lavede han af Akacietræ og overtrak dem med Guld.
\par 29 Han tilberedte også den hellige Salveolie og den rene, vellugtende Røgelse, som Salveblanderne laver den.

\chapter{38}

\par 1 Derpå lavede han Brændofferalteret af Akacietræ, fem Alen langt og fem Alen bredt, firkantet, og tre Alen højt.
\par 2 Han lavede Horn til dets fire Hjørner, således at de var i eet dermed, overtrak det med Kobber
\par 3 og lavede alt Alterets Tilbehør, Karrene, Skovlene, Skålene, Gaflerne og Panderne; alt dets Tilbehør lavede han af Kobber.
\par 4 Derpå omgav han Alteret med et flettet Kobbergitter neden under dets Liste, således at det nåede op til Alterets halve Højde.
\par 5 Derefter støbte han fire Ringe til Kobbergitterets fire Hjørner til at stikke Bærestængerne i.
\par 6 Bærestængerne lavede han at Akacietræ og overtrak dem med Kobber.
\par 7 Og Stængerne stak han gennem Ringene på Alterets Sider, for at det kunde bæres med dem. Han lavede det hult af Brædder.
\par 8 Derpå lavede han Vandkummen med Fodstykke af Kobber og brugte dertil Spejle, som tilhørte Kvinderne, der gjorde Tjeneste ved Indgangen til Åbenbaringsteltet.
\par 9 Derpå indrettede han Forgården: Til Sydsiden det hundrede Alen lange Forgårdsomhæng af tvundet Byssus
\par 10 med tyve Piller og tyve Fodstykker af Kobber og med Knager og Bånd af Sølv til Pillerne.
\par 11 Til Nordsiden hundrede Alen med tyve Piller og tyve Fodstykker af Kobber og med Knager og Bånd af Sølv til Pillerne.
\par 12 Til Vestsiden det halvtredsindstyve Alen lange Omhæng med ti Piller og ti Fodstykker og med Knager og Bånd af Sølv til Pillerne.
\par 13 Og til Forsiden mod Øst, der var halvtredsindstyve Alen bred,
\par 14 det femten Alen lange Omhæng med fire Piller og tre Fodstykker til den ene Side af Indgangen,
\par 15 og det femten Alen lange Omhæng med tre Piller og tre Fodstykker til den anden Side af Indgangen.
\par 16 Alle Omhæng rundt om Forgården var af tvundet Byssus,
\par 17 Fodstykkerne til Pillerne af Kobber,men deres Knager og Bånd af Sølv; alle Pillernes Hoveder var overtrukket med Sølv; og de havde Bånd af Sølv.
\par 18 Forhænget til Forgårdens Indgang var af violet og rødt Purpurgarn i broget Vævning, karmoisinrødt Garn og tvundet Byssus, tyve Alen langt og fem Alen højt, svarende til Bredden på Forgårdens Omhæng.
\par 19 Dertil hørte fire Piller med fire Fodstykker af Kobber; Knagerne var af Sølv og Overtrækket på Hovederne og Båndene ligeledes af Sølv.
\par 20 Alle Pælene til Boligen og Forgården rundt om var af Kobber.
\par 21 Her følger Regnskabet over Boligen, Vidnesbyrdets Bolig, som på Moses's Bud blev opgjort af Leviterne under Ledelse af Itamar, en Søn af Præsten Aron;
\par 22 Bezal'el, en Søn af Hurs Søn Uri, af Judas Stamme havde udført alt, hvad HERREN havde pålagt Moses,
\par 23 sammen med Oholiab, Ahisamaks Søn, af Dans Stamme, som udførte Udskæringsarbejdet, Kunstvævningen og den brogede Vævning af violet og rødt Purpur, Karmoisin og Byssus.
\par 24 Hvad angår Guldet, der anvendtes til Arbejdet, under hele Arbejdet på Helligdommen, så løb det som Gave viede Guld op til 29 talenter og 730 Sekel efter hellig Vægt.
\par 25 Det ved Menighedens Mønstring indkomne Sølv løb op til 100 Talenter og 1775 Sekel efter hellig Vægt
\par 26 en, Beka, det halve af en Sekel efter hellig Vægt, af enhver, der måtte lade sig mønstre, altså fra Tyveårsalderen og opefter, i alt 603 550 Mand:
\par 27 De 100 Talenter Sølv medgik til Støbningen af Helligdommens og Forhængets Fodstykker, 100 Talenter til 100 Fodstykker, en Talent til hvert Fodstykke.
\par 28 Men de 1775 Sekel anvendte han til Knager til Pillerne, til at overtrække deres Hoveder med og til Bånd på dem.
\par 29 Det som Gave viede Kobber udgjorde 70 Talenter og 2400 Sekel.
\par 30 Deraf lavede han Fodstykkerne til Åbenbaringsteltets Indgang, Kobberalteret med dets Kobbergitter og alt Alterets Tilbehør,
\par 31 Fodstykkerne til Forgården rundt om og til Forgårdens Indgang og alle Teltpælene til Boligen og alle Teltpælene til Forgården hele Vejen rundt.

\chapter{39}

\par 1 Af det violette og røde Purpurgarn og det karmoisinrøde Garn tilvirkede de Pragtklæderne til Tjenesten i Helligdommen; og de tilvirkede Arons hellige Klæder, således som HERREN havde pålagt Moses.
\par 2 De tilvirkede Efoden af Guldtråd, violet og rødt Purpurgarn, karmoisinrødt Garn og tvundet Byssus,
\par 3 idet de udhamrede Guldet i Plader og skar Pladerne ud i Tråde til at væve ind i det violette og røde Purpurgarn, det karmoisinrøde Garn og det tvundne Byssus ved Kunstvævning.
\par 4 Derpå forsynede de den med Skulderstykker til at hæfte på; den blev hæftet sammen ved begge Hjørner.
\par 5 Og dens Bælte, som brugtes, når den skulde tages på, var i eet med den og af samme Arbejde, af Guldtråd, violet og rødt Purpurgarn, karmoisinrødt Garn og tvundet Byssus, således som HERREN havde pålagt Moses.
\par 6 Derpå tilvirkede de Sjohamstenene, indfattede i Guldfletværk og graverede som Signeter med Israels Sønners Navne;
\par 7 og de fæstede dem på Efodens Skulderstykker, for at Stenene kunde bringe Israels Børn i Minde, således som HERREN havde pålagt Moses.
\par 8 Derpå tilvirkede de Brystskjoldet i Kunstvævning på samme Måde som Efoden, af Guldtråd, violet og rødt Purpurgarn, karmoisinrødt Garn og tvundet Byssus;
\par 9 det var firkantet, og de lagde Brystskjoldet dobbelt; det var et Spand langt og et Spand bredt, lagt dobbelt.
\par 10 De udstyrede det med fire Rækker Sten: Karneol, Topas og Smaragd i den første Række,
\par 11 Rubin, Safir og Jaspis i den anden,
\par 12 Hyacint, Agat og Ametyst i den tredje,
\par 13 Krysolit, Sjoham og Onyks i den fjerde, omgivne med Guldfletværk i deres Indfatninger.
\par 14 Der var tolv Sten, svarende til Israels Sønners Navne, en for hvert Navn; det var graveret Arbejde som Signeter, således at hver Sten bar Navnet på en af de tolv Stammer.
\par 15 Til Brystskjoldet lavede de snoede Kæder af purt Guld, snoet Arbejde, som når man snor Reb.
\par 16 Derpå lavede de to Guldfletværker og to Guldringe og satte disse to Ringe på Brystskjoldets øverste Hjørner,
\par 17 og de to Guldsnore knyttede de i de to Ringe på Brystskjoldets Hjørner;
\par 18 Snorenes anden Ende anbragte de i de to Fletværker og fæstede dem på Forsiden af Efodens Skulderstykker.
\par 19 Og d lavede to andre Guldringe og satte dem på Brystskjoldets to andre Hjørner på den indre Rand, der vendte mod Efoden.
\par 20 Og de lavede endnu to Guldringe og fæstede dem på Efodens to Skulderstykker forneden på Forsiden, hvor den var hæftet sammen med Skulderstykkerne, oven over Efodens Bælte;
\par 21 og de bandt med Ringene Brystskjoldet fast til Efodens Ringe ved Hjælp af en violet Purpursnor, så at det kom til at sidde oven over Efodens Bælte og ikke kunde løsne sig fra Efoden, som HERREN havde pålagt Moses.
\par 22 Derpå tilvirkede de Kåben, som hører til Efoden, i vævet Arbejde, helt og holdent af violet Purpur.
\par 23 Midt på havde Kåben en Halsåbning ligesom Halsåbningen på en Panserskjorte, omgivet af en Linning, for at den ikke skulde rives itu,
\par 24 og langs Kåbens nederste Kant syede de Granatæbler af violet og rødt Purpurgarn, karmoisinrødt Garn og tvundet Byssus,
\par 25 og de lavede Bjælder af purt Guld, som de anbragte mellem Granatæblerne langs Kåbens nederste Kant hele Vejen rundt, mellem Granatæblerne,
\par 26 så at Bjælder og Granatæbler skiftede hele Vejen rundt langs Kåbens nederste Kant, til at bære ved Tjenesten, som HERREN havde pålagt Moses.
\par 27 Derpå tilvirkede de Kjortlerne til Aron og hans Sønner af Byssus i vævet Arbejde,
\par 28 Hovedklædet af Byssus, Embedshuerne af Byssus, Linnedbenklæderne af tvundet Byssus,
\par 29 og Bæltet af tvundet Byssus, violet og rødt Purpurgarn og karmoisinrødt Garn i broget Vævning, som HERREN havde pålagt Moses.
\par 30 Derpå lavede de Pandepladen, det hellige Diadem, af purt Guld og forsynede den med en Indskrift i graveret Arbejde som ved Signeter: "Helliget HERREN."
\par 31 Og de fæstede en violet Purpursnor på den til at binde den fast med oven på Hovedklædet, som HERREN havde pålagt Moses.
\par 32 Således fuldførtes alt Arbejdet ved Åbenbaringsteltets Bolig; og Israeliterne gjorde ganske som HERREN havde pålagt Moses; således gjorde de.
\par 33 Derpå bragte de Boligen til Moses, Teltdækket med alt dets Tilbehør, Knagerne, Brædderne, Tværstængerne, Pillerne og Fodstykkerne,
\par 34 Dækket af rødfarvede Væderskind og Dækket af Tahasjskind, det indre Forhæng,
\par 35 Vidnesbyrdets Ark med Bærestængerne, Sonedækket,
\par 36 Bordet med alt dets Tilbehør, Skuebrødene,
\par 37 Lysestagen af purt Guld med Lamperne, der skulde sættes på den, og alt dens Tilbehør, Olien til Lysestagen,
\par 38 Guldalteret, Salveolien, den vellugtende Røgelse, Forhænget til Teltets Indgang,
\par 39 Kobberalteret med Kobbergitteret, Bærestængerne og alt dets Tilbehør, Vandkummen og Fodstykket,
\par 40 Omhængene til Forgården, Pillerne og Fodstykkerne, Forhænget til Forgårdens Indgang, Rebene og Teltpælene, alt Tilbehør til Tjenesten i Åbenbaringsteltets Bolig,
\par 41 Pragtklæderne til Tjenesten i Helligdommen, de hellige Klæder til Præsten Aron og hans Sønners Klæder til Præstetjenesten.
\par 42 Nøjagtigt som HERREN havde pålagt Moses, udførte Israeliterne hele Arbejdet.
\par 43 Da så Moses hele Arbejdet efter, og se, de havde udført det, som HERREN havde sagt; således havde de utdført det. Og Moses velsignede dem.

\chapter{40}

\par 1 Og HERREN talede til Moses og sagde:
\par 2 På den første Dag i den første Måned skal du rejse Åbenbaringsteltets Bolig.
\par 3 Sæt så Vidnesbyrdets Ark derind og hæng Forhænget op for Arken.
\par 4 Og du skal bringe Bordet ind og ordne, hvad dertil hører, og bringe Lysestagen ind og sætte Lamperne på.
\par 5 Stil Guldalteret til Røgelsen op foran Vidnesbyrdets Ark og hæng Forhænget op foran Boligens Indgang.
\par 6 Stil Brændofferalteret op foran Indgangen til Åbenbaringsteltets Bolig
\par 7 og stil Vandkummen op mellem Åbenbaringsteltet og Alteret og hæld Vand deri.
\par 8 Rejs Forgården rundt om og hæng Forhænget op foran Forgårdens Indgang.
\par 9 Tag Salveolien og salv Boligen og alle Ting de1i, og du skal hellige den med alt dens Tilbehør, så den bliver hellig.
\par 10 Du skal salve Brændofferalteret og alt dets Tilbehør og hellige Alteret, så det bliver højhelligt.
\par 11 Og du skal salve Vandkummen og Fodstykket og hellige den.
\par 12 Lad så Aron og hans Sønner træde hen til Åbenbaringsteltets Indgang, tvæt dem med Vand
\par 13 og ifør Aron de hellige Klæder. salv og hellig ham til at gøre Præstetjeneste for mig.
\par 14 Lad så hans Sønner træde frem, ifør dem Kjortler
\par 15 og salv dem, som du salver deres Fader, til at gøre Præstetjeneste for mig. Således skal det ske, for at et evigt Præstedømme kan blive dem til Del fra Slægt til Slægt i Kraft af denne Salvning, som du foretager på dem.
\par 16 Og Moses gjorde ganske som HERREN havde pålagt ham; således gjorde han.
\par 17 På den første Dag i den første Måned i det andet År blev Boligen rejst.
\par 18 Moses rejste Boligen, idet han anbragte Fodstykkerne, rejste Brædderne, stak Tværstængerne ind, rejste Pillerne,
\par 19 spændte Teltdækket ud over Boligen og lagde Teltdækkets Dække ovenover, som HERREN havde pålagt Moses.
\par 20 Derpå tog han Vidnesbyrdet og lagde det i Arken, stak Bærestængerne i Arken og lagde Sonedækket oven på den;
\par 21 så bragte han Arken ind i Boligen og hængte det indre Forhæng op og tilhyllede således Vidnesbyrdets Ark, som HERREN havde pålagt Moses.
\par 22 Derpå opstillede han Bordet i Åbenbaringsteltet ved Boligens nordre Væg uden for Forhænget,
\par 23 og han lagde Brødene i Række derpå for HERRENs Åsyn, som HERREN havde pålagt Moses.
\par 24 Derpå satte han Lysestagen ind i Åbenbaringsteltet lige over for Bordet, ved Boligens søndre Væg;
\par 25 og han satte Lamperne derpå for HERRENs Åsyn, som HERREN havde pålagt Moses.
\par 26 Derpå stillede han Guldalteret op i Åbenbaringsteltet foran Forhænget,
\par 27 og han tændte vellugtende Røgelse derpå, som HERREN havde pålagt Moses.
\par 28 Derpå hængte han Forhænget op for Boligens Indgang.
\par 29 Brændofferalteret opstillede han foran Indgangen til Åbenbaringsteltets Bolig og ofrede Brændofferet og Afgrødeofferet derpå, som HERREN havde pålagt Moses.
\par 30 Derpå opstillede han Vandkummen mellem Åbenbaringsteltet og Alteret og hældte Vand deri til Tvætning.
\par 31 Og Moses og Aron og hans Sønner tvættede deres Hænder og Fødder deri;
\par 32 når de gik ind i Åbenbaringsteltet og trådte hen til Alteret.
\par 33 Så rejste han Forgården rundt om Boligen og Alteret og hængte Forhænget op for Forgårdens Indgang. Hermed var Moses færdig med Arbejdet.
\par 34 Da dækkede Skyen Åbenbaringsteltet, og HERRENs Herlighed fyldte Boligen;
\par 35 og Moses kunde ikke gå ind i Åbenbaringsteltet, fordi Skyen havde lagt sig derover, og HERRENs Herlighed fyldte Boligen.
\par 36 Men under hele deres Vandring brød Israeliterne op, når Skyen løftede sig fra Boligen;
\par 37 og når Skyen ikke løftede sig. brød de ikke op, men ventede, til den atter løftede sig.
\par 38 Thi HERRENs Sky lå over Boligen om Dagen, og om Natten lyste Ild i Skyen for alle Israeliternes Øjne under hele deres Vandring.


\end{document}