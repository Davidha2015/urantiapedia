\begin{document}

\title{Leviticus}


\chapter{1}

\par 1 HERREN kaldte på Moses og talede til ham fra Åbenbaringsteltet og sagde:
\par 2 Tal til Israeliterne og sig til dem: Når nogen af eder vil bringe HERREN en Offergave af Kvæget, da skal Offergaven, I vil bringe, tages at Hornkvæget eller Småkvæget.
\par 3 Skal hans Offergave af Hornkvæget være et Brændoffer, så skal det være et lydefrit Handyr, han bringer; til Åbenbaringsteltets Indgang skal han bringe det for at vinde HERRENs Velbehag.
\par 4 Så skal han lægge sin Hånd på Brændofferdyrets Hoved, for at det kan vinde ham HERRENs Velbehag, idet det skaffer ham Soning.
\par 5 Derpå skal han slagte den unge Okse for HERRENs Åsyn, og Arons Sønner, Præsterne, skal frembære Blodet, og de skal sprænge Blodet rundt om på Alteret, som står ved Indgangen til Åbenbaringsteltet.
\par 6 Så skal han flå Huden af Brændofferdyret og skære det i Stykker,
\par 7 og Arons Sønner, Præsterne, skal gøre Ild på Alteret og lægge Brænde på Ilden;
\par 8 og Arons Sønner, Præsterne, skal lægge Stykkerne tillige med Hovedet og Fedtet til Rette på Brændet over Ilden på Alteret.
\par 9 Men Indvoldene og Skinnebenene skal han tvætte med Vand, og Præsten skal bringe det hele som Røgoffer på Alteret; det er et Brændoffer, et Ildoffer til en liflig Duft for HERREN.
\par 10 Hvis hans Offergave, der skal bruges som Brændoffer, tages at Småkvæget, af Fårene eller Gederne, så skal det være et lydefrit Handyr, han bringer.
\par 11 Han skal slagte det for HERRENs Åsyn ved Alterets Nordside, og Arons Sønner, Præsterne skal sprænge Blodet deraf rundt om på Alteret.
\par 12 Så skal han skære det i Stykker, og Præsten skal lægge Stykkerne tillige med Hovedet og Fedtet til Rette på Brændet over Ilden på Alteret.
\par 13 Men Indvoldene og Skinnebenene skal han tvætte med Vand. Så skal Præsten frembære det hele og bringe det som Røgoffer på Alteret; det er et Brændoffer, et Ildoffer til en liflig Duft for HERREN.
\par 14 Er hans Offergave til HERREN et Brændoffer af Fuglene, da skal Offergaven, han vil bringe, tages af Turtelduerne eller Dueungerne;
\par 15 Præsten skal frembære den til Alteret og knække Halsen på den og bringe Hovedet som Røgoffer på Alteret, og dens Blod skal udpresses mod Alterets Side.
\par 16 Så skal han udtage Kroen med dens Indhold og kaste den på Askedyngen ved Alterets Østside.
\par 17 Derpå skal han rive Vingerne løs på den uden dog at skille dem fra Kroppen. Så skal Præsten bringe den som Røgoffer på Alteret på Brændet over Ilden; det er et Brændoffer, et Ildoffer til en liflig Duft for HERREN.

\chapter{2}

\par 1 Når nogen vil frembære et Afgrødeoffer som Offergave for HERREN, skal hans Offergave bestå af fint Hvedemel, og han skal hælde Olie derover og komme Røgelse derpå.
\par 2 Og han skal bringe det til Arons Sønner, Præsterne; og Præsten skal tage en Håndfuld af Melet og Olien og al Røgelsen, det, som skal ofres af Afgrødeofferet, og bringe det som Røgoffer på Alteret, et Ildoffer til en liflig Duft for HERREN;
\par 3 men Resten af Afgrødeofferet skal tilfalde Aron og hans Sønner som en højhellig Del af HERRENs Ildofre.
\par 4 Men når du som Offergave vil bringe et Afgrødeoffer af Bagværk fra Bagerovnen, skal det bestå af fint Hvedemel, usyrede Kager, rørte i Olie, og usyrede Fladbrød, smurte med Olie.
\par 5 Er derimod din Offergave et Afgrødeoffer, bagt på Plade, så skal det bestå af usyret fint Hvedemel, rørt i Olie;
\par 6 du skal bryde det i Stykker og hælde Olie derover. Det er et Afgrødeoffer.
\par 7 Men er din Offergave et Afgrødeoffer, bagt i Pande, skal det tilberedes af fint Hvedemel med Olie.
\par 8 Det Afgrødeoffer, der tilberedes af disse Ting, skal du bringe HERREN; man skal bringe det til Præsten, og han skal bære det hen til Alteret;
\par 9 og Præsten skal af Afgrødeofferet udtage det, som skal ofres deraf, og bringe det som Røgoffer på Alteret, et Ildoffer til en liflig Duft for HERREN.
\par 10 Men Resten af Afgrødeofferet skal tilfalde Aron og hans Sønner som en højhellig Del af HERRENs Ofre.
\par 11 Intet Afgrødeoffer, som I bringer HERREN, må tilberedes syret; thi Surdejg eller Honning må I aldrig bringe som Røgoffer, som Ildoffer for HERREN.
\par 12 Kun som Offergave af Førstegrøde må I frembære disse Ting for HERREN, men de må ikke komme på Alteret til en liflig Duft.
\par 13 Og du skal komme Salt i enhver Afgrødeoffergave, du frembærer, du må ikke undlade at komme din Guds Pagts Salt i dit Afgrødeoffer, men du skal frembære Salt med enhver af dine Offergaver.
\par 14 Dersom du vil frembære HERREN et Afgrødeoffer af Førstegrøden, skal det, du frembærer som Afgrødeoffer af din Førstegrøde, være friske Aks, ristede over Ilden, knuste, af nyhøstet Korn;
\par 15 og du skal hælde Olie derover og komme Røgelse derpå. Det er et Afgrødeoffer,
\par 16 Al Røgelsen og det, som skal ofres af de knuste Aks og Olien, skal Præsten bringe som Røgoffer, et Ildoffer for HERREN.

\chapter{3}

\par 1 Men er hans Offergave et Takoffer, så skal det, hvis han bringer det af Hornkvæget, være et lydefrit Han eller Hundyr, han bringer HERREN.
\par 2 Han skal lægge sin Hånd på sin Offergaves Hoved og slagte Dyret ved Indgangen til Åbenbaringsteltet; og Arons Sønner, Præsterne, skal sprænge Blodet rundt om på Alteret.
\par 3 Så skal han af Takofferet som Ildoffer for HERREN frembære Fedtet, der dækker Indvoldene, og alt Fedtet på Indvoldene,
\par 4 begge Nyrerne med det Fedt. som sidder på dem ved Lændemusklerne, og Leverlappen, som han skal skille fra ved Nyrerne.
\par 5 Og Arons Sønner skal bringe det som Røgoffer på Alteret oven på Brændofferet på Brændet, der ligger på Ilden, et Ildoffer til en liflig Duft for HERREN.
\par 6 Men hvis hans Offergave, der bringes som Takoffer til HERREN.
\par 7 Er den Offergave, han vil bringe. et Lam, skal han bringe det hen for HERRENs Åsyn
\par 8 og lægge sin Hånd på sin Offergaves Hoved og slagte Dyret foran Åbenbaringsteltet, og Arons Sønner skal sprænge Blodet deraf rundt om på Alteret.
\par 9 Så skal han af Takofferet som Ildoffer for HERREN frembære Fedtet, hele Fedthalen, skilt fra Rygraden, Fedtet, som dækker Indvoldene, og alt Fedtet på Indvoldene,
\par 10 begge Nyrerne med det Fedt, som sidder på dem ved Lændemusklerne, og Leverlappen, som han skal skille fra ved Nyrerne.
\par 11 Og Præsten skal bringe det som Røgoffer på Alteret, Ildofferspise for HERREN.
\par 12 Men hvis hans Offergave er en Ged, skal han bringe den hen for HERRENs Åsyn
\par 13 og lægge sin Hånd på dens Hoved og slagte den foran Åbenbaringsteltet, og Arons Sønner skal sprænge Blodet rundt om på Alteret.
\par 14 Så skal han deraf frembære som sin Offergave, som et Ildoffer for HERREN, Fedtet, det dækker Indvoldene, og alt Fedtet på Indvoldene,
\par 15 begge Nyrerne med det Fedt, som sidder på dem ved Lændemusklerne, og Leverlappen, som han skal skille fra ved Nyrerne.
\par 16 Og Præsten skal bringe det som Røgoffer på Alteret, Ildofferspise til en liflig Duft. Alt Fedt skal være HERRENs.
\par 17 En evig Anordning skal det være for eder fra Slægt til Slægt, hvor I end bor: Intet Fedt og intet Blod må I nyde!

\chapter{4}

\par 1 Og HERREN talede til Moses og sagde:
\par 2 Tal til Israeliterne og sig: Når nogen af Vanvare forsynder sig mod noget af HERRENs Forbud og overtræder et af dem, da skal følgende iagttages:
\par 3 Er det den salvede Præst, der forsynder sig, så der pådrages Folket Skyld, skal han for den Synd, han har begået, bringe HERREN en lydefri ung Tyr som Syndoffer.
\par 4 Han skal føre Tyren hen til Åbenbaringsteltets Indgang for HERRENs Åsyn og lægge sin Hånd på dens Hoved og slagte den for HERRENs Åsyn,
\par 5 og den salvede Præst skal tage noget af Tyrens Blod og bringe det ind i Åbenbaringsteltet,
\par 6 og Præsten skal dyppe sin Finger i Blodet og stænke det syv Gange for HERRENs Åsyn foran Helligdommens Forhæng;
\par 7 og Præsten skal stryge noget af Blodet på Røgelsealterets Horn for HERRENs Åsyn, det, som står i Åbenbaringsteltet; Resten af Tyrens Blod skal han udgyde ved Foden af Brændofferalteret, som står ved Indgangen til Åbenbaringsteltet.
\par 8 Men alt Syndoffertyrens Fedt skal han tage ud Fedtet, som dækker Indvoldene, og alt Fedtet på Indvoldene,
\par 9 begge Nyrerne med det Fedt, som sidder på dem ved Lændemusklerne, og Leverlappen, som han skal skille fra ved Nyrerne,
\par 10 på samme Måde som det udtages af Takofferoksen. Og Præsten skal bringe det som Røgoffer på Brændofferalteret.
\par 11 Men Tyrens Hud og alt dens Kød tillige med dens Hoved, Skinneben, Indvolde og Skarn,
\par 12 hele Tyren skal han bringe uden for Lejren til et urent Sted, til Askedyngen, og brænde den på et Bål af Brænde; oven på Aske; dyngen skal den brændes.
\par 13 Men hvis det et hele Israels Menighed, der forser sig, uden at Forsamlingen ved af det, og de har overtrådt et af HERRENs Forbud og derved pådraget sig Skyld,
\par 14 da skal Forsamlingen, når den Synd, de har begået mod Forbudet, bliver kendt, bringe en ung, lydefri Tyr som Syndoffer; de skal føre den hen foran Åbenbaringsteltet,
\par 15 og Menighedens Ældste skal lægge deres Hænder på Tyrens Hoved for HERRENs Åsyn, og man skal slagte den for HERRENs Åsyn.
\par 16 Derpå skal den salvede Præst bringe noget af Tyrens Blod ind i Åbenbaringsteltet,
\par 17 og Præsten skal dyppe sin Finger i Blodet og stænke det syv Gange for HERRENs Åsyn foran Forhænget;
\par 18 og han skal stryge noget af Blodet på Hornene af Alteret, som står for HERRENs Åsyn i Åbenbaringsteltet; Resten af Blodet skal han udgyde ved Foden af Brændofferalteret, som står ved Indgangen til Åbenbaringsteltet,
\par 19 Men alt, Fedtet skal han tage ud og bringe det som Røgoffer på Alteret.
\par 20 Derpå skal han gøre med Tyren på samme Måde som med den før nævnte Syndoffertyr. Da skal Præsten skaffe dem Soning, så de finder Tilgivelse.
\par 21 Så skal Tyren bringes uden for Lejren og brændes på samme Måde som den før nævnte Tyr. Det er Menighedens Syndoffer.
\par 22 Men hvis det er en Øverste. der forsynder sig og af Vanvare overtræder et af HERRENs Forbud og derved på drager sig Skyld,
\par 23 og den Synd, han har begået, bliver ham vitterlig, så skal den Offergave, han bringer, være en lydefri Gedebuk.
\par 24 Han skal lægge sin Hånd på Bukkens Hoved og slagte den der, hvor Brændofferet slagtes for HERRENs Åsyn. Det er et Syndoffer.
\par 25 Præsten skal tage noget af Syndofferets Blod på sin Finger og stryge det på Brændofferalterets Horn, og Resten af Blodet skal han udgyde ved Brændofferalterets Fod.
\par 26 Og alt dets Fedt skal han bringe som Røgoffer på Alteret ligesom Fedtet fra Takofferet. Da skal Præsten skaffe ham Soning for hans Synd, så han finder Tilgivelse.
\par 27 Men hvis det er en af Almuen, der af Vanvare forsynder sig ved at overtræde et af HERRENs Forbud og derved pådrager sig Skyld,
\par 28 og den Synd, han har begået, bliver ham vitterlig, så skal Offergaven, han bringer for den Synd, han har begået, være en lydefri Ged.
\par 29 Han skal lægge sin Hånd på Syndofferets Hoved og slagte Syndofferet der, hvor Brændofferet slagtes.
\par 30 Præsten skal tage noget af Gedens Blod på sin Finger og stryge det på Brændofferalterets Horn, og Resten af Blodet skal han udgyde ved Alterets Fod.
\par 31 Og alt Fedtet skal han tage ud, på samme Måde som Fedtet tages ud af Takofferet, og Præsten skal bringe det som Røgoffer på Alteret til en liflig Duft for HERREN. Da skal Præsten skaffe ham Soning, så han finder Tilgivelse.
\par 32 Men hvis den Offergave, han vil bringe som Syndoffer, er et Lam, da skal det være et lydefrit Hundyr, han bringer.
\par 33 Han skal lægge sin Hånd på Syndofferets Hoved og slagte det som Syndoffer der, hvor Brændofferet slagtes.
\par 34 Og Præsten skal tage noget af Syndofferets Blod på sin Finger og stryge det på Brændofferalterets Horn, men Resten af Blodet skal han udgyde ved Alterets fod,
\par 35 og alt Fedtet skal han tage ud, på samme Måde som Takofferlammets Fedt tages ud, og Præsten skal bringe det som Røgoffer på Alteret oven på HERRENs Ildofre. Da skal Præsten skaffe ham Soning for den Synd, han har begået, så han finder Tilgivelse.

\chapter{5}

\par 1 Hvis nogen, når han hører en Forbandelse udtale, synder ved at undlade at vidne, skønt han var Øjenvidne eller på anden Måde kender Sagen, og således pådrager sig Skyld,
\par 2 eller hvis nogen, uden at det er ham vitterligt, rører ved noget urent, hvad enten det nu er et Ådsel af et urent vildt Dyr eller af urent Kvæg eller urent Kryb, og han opdager det og bliver sig sin Skyld bevidst,
\par 3 eller når han, uden at det er ham vitterligt, rører ved Urenhed hos et Menneske, Urenhed af en hvilken som helst Art, hvorved man bliver uren, og han opdager det og bliver sig sin Skyld bevidst,
\par 4 eller når nogen, uden at det er ham vitterligt, med sine Læber aflægger en uoverlagt Ed på at ville gøre noget, godt eller ondt, hvad nu et Menneske kan aflægge en uoverlagt Ed på, og han opdager det og bliver sig sin Skyld bevidst i et af disse Tilfælde,
\par 5 så skal han, når han bliver sig sin Skyld bevidst i et af disse Tilfælde, bekende det, han har forsyndet sig med,
\par 6 og til Bod for den Synd, han har begået, bringe HERREN et Hundyr af Småkvæget, et Får eller en Ged, som Syndoffer; da skal Præsten skaffe ham Soning for hans Synd.
\par 7 Men hvis han ikke evner at give et Stykke Småkvæg, skal han til Bod for sin Synd bringe HERREN to Turtelduer eller Dueunger, en som Syndoffer og en som Brændoffer.
\par 8 Han skal bringe dem til Præsten, og Præsten skal først frembære den, der skal bruges til Syndoffer; han skal knække Halsen på den ved Nakken uden at rive Hovedet helt af
\par 9 og stænke noget af Syndofferets Blod på Alterets Side, medens Resten af Blodet skal udpresses ved Alterets Fod. Det er et Syndoffer.
\par 10 Men den anden skal han ofre som Brændoffer på den foreskrevne Måde; da skal Præsten skaffe ham Soning for den Synd, han har begået, så han finder Tilgivelse.
\par 11 Men hvis han ej heller evner at give to Turtelduer eller Dueunger, skal han som Offergave for sin Synd bringe en Tiendedel Efa fint Hvedemel til Syndoffer, men han må ikke hælde Olie derover eller komme Røgelse derpå, thi det er et Syndoffer.
\par 12 Han skal bringe det til Præsten, og Præsten skal tage en Håndfuld deraf, det, som skal ofres deraf, og bringe det som Røgoffer på Alteret oven på HERRENs Ildofre. Det er et Syndoffer.
\par 13 Da skal Præsten skaffe ham Soning for den Synd, han har begået på en af de nævnte Måder, så han finder Tilgivelse. Resten skal tilfalde Præsten på samme Måde som Afgrødeofferet.
\par 14 HERREN talede fremdeles til Moses og sagde:
\par 15 Når nogen gør sig skyldig i Svig og af Vanvare forsynder sig mod HERRENs Helliggaver, skal han til Bod derfor som Skyldoffer bringe HERREN en lydefri Væder af Småkvæget, der er vurderet til mindst to Sølvsekel efter hellig Vægt;
\par 16 og han skal give Erstatning for, hvad han har forsyndet sig med over for det hellige, med Tillæg af en Femtedel af Værdien.
\par 17 Når nogen, uden at det er ham vitterligt, synder ved at overtræde et af HERRENs Forbud, så han bliver skyldig og pådrager sig Skyld,
\par 18 da skal han af Småkvæget bringe en lydefri Væder, der er taget god, som Skyldoffer til Præsten, og Præsten skal skaffe ham Soning for den uforsætlige Synd, han har begået, uden at den var ham vitterlig, så han finder Tilgivelse.
\par 19 Det er et Skyldoffer; han har pådraget sig Skyld over for HERREN.
\par 20 HERREN talede fremdeles til Moses og sagde:
\par 21 Når nogen forsynder sig og gør sig skyldig i Svig mod HERREN, idet han frakender sin Næste Retten til noget, der var ham betroet, et Håndpant eller noget, han har røvet, eller han aftvinger sin Næste noget,
\par 22 eller han finder noget, som er tabt, og nægter det, eller han aflægger falsk Ed angående en af alle de Ting, som Mennesket forsynder sig ved at gøre,
\par 23 så skal han, når han har forsyndet sig og føler sig skyldig, tilbagegive det, han har røvet, eller det, han har aftvunget, eller det, som var ham betroet, eller det tabte, som han har fundet,
\par 24 eller alt det, hvorom han har aflagt falsk Ed; han skal erstatte det med dets fulde Værdi med Tillæg af en Femtedel. Han skal give den retmæssige Ejer det, den Dag han gør Bod.
\par 25 Og til Bod skal han af Småkvæget bringe HERREN en lydefri Væder, der er taget god; som Skyldoffer skal han bringe den til Præsten.
\par 26 Da skal Præsten skaffe ham Soning for HERRENs Åsyn, så han finder Tilgivelse for enhver Ting, hvorved man pådrager sig Skyld.

\chapter{6}

\par 1 HERREN taled fremdeles til Moses og sagde:
\par 2 Giv Aron og hans Sønner dette Bud: Dette er Loven om Brændofferet. Brændofferet skal blive liggende på sit Bål på Alteret Natten over til næste Morgen, og Alterilden skal holdes ved lige dermed.
\par 3 Så skal Præsten iføre sig sin Linnedklædning, og Linnedbenklæder skal han iføre sig over sin Blusel, og han skal borttage Asken, som bliver tilbage, når Ilden fortærer Brændofferet på Alteret, og lægge den ved Siden af Alteret.
\par 4 Derefter skal han afføre sig sine Klæder og tage andre Klæder på og bringe Asken uden for Lejren til et urent Sted.
\par 5 Ilden på Alteret skal holdes ved lige dermed, den må ikke gå ud: og Præsten skal hver Morgen tænde ny Brændestykker på Alteret og lægge Brændofferet til Rette derpå og så bringe Takofrenes Fedtdele som Røgoffer derpå.
\par 6 En stadig Ild skal holdes ved lige på Alteret, den må ikke gå ud.
\par 7 Dette er Loven om Afgrødeofferet: Arons Sønner skal frembære det for HERRENs Åsyn, hen til Alteret.
\par 8 Så skal han tage en Håndfuld af Afgrødeofferets Mel og Olie og al Røgelsen, der følger med Afgrødeofferet, det, der skal ofres deraf, og bringe det som Røgoffer på Alteret til en liflig Duft for HERREN.
\par 9 Men Resten deraf skal Aron og hans Sønner spise; usyret skal det spises på et helligt Sted; i Åbenbaringsteltets Forgård skal de spise det.
\par 10 Det må ikke bages syret. Jeg har givet dem det som deres Del af mine Ildofre; det er højhelligt ligesom Syndofferet og Skyldofferet.
\par 11 Alle af Mandkøn blandt Arons Sønner må spise det; denne Del af HERRENs Ildofre skal være en evig gyldig Rettighed, de har Krav på fra Slægt til Slægt. Enhver, som rører derved, bliver hellig.
\par 12 HERREN talede fremdeles til Moses og sagde:
\par 13 Dette er den Offergave, Aron og hans Sønner skal frembære for HERREN: En Tiendedel Efa fint Hvedemel, et dagligt Afgrødeoffer, Halvdelen om Morgenen og Halvdelen om Aftenen.
\par 14 Det skal tilberedes på Plade med Olie, og du skal frembære det godt æltet, og du skal bryde det i Stykker; et Afgrødeoffer, som er brudt i Stykker, skal du frembære til en liflig Duft for HERREN.
\par 15 Den Præst iblandt hans Sønner, der salves i hans Sted, skal ofre det; det skal være en evig gyldig Rettighed for HERREN, og som Heloffer skal det ofres.
\par 16 Ethvert Afgrødeoffer fra en Præst skal være et Heloffer; det må ikke spises.
\par 17 HERREN talede fremdeles til Moses og sagde:
\par 18 Tal til Aron og hans Sønner og sig: Dette er Loven om Syndofferet. Der, hvor Brændofferet slagtes. skal Syndofferet slagtes for HERRENs Åsyn; det er højhelligt.
\par 19 Den Præst, der frembærer Syndofferet, skal spise det; det skal spises på et helligt Sted, i Åbenbaringsteltets Forgård.
\par 20 Enhver, som rører ved Kødet deraf, bliver hellig. Hvis noget af dets Blod stænkes på en Klædning, skal det Stykke, Blodet er stænket på, tvættes på et helligt Sted.
\par 21 Det Lerkar, det koges i, skal slås i Stykker; og hvis det er kogt i et Kobberkar, skal dette skures og skylles med Vand.
\par 22 Alle af Mandkøn blandt Præsterne må spise det; det er højhelligt.
\par 23 Men intet Syndoffer må spises, når noget af dets Blod bringes ind i Åbenbaringsteltet for at skaffe Soning i Helligdommen; det skal opbrændes.

\chapter{7}

\par 1 Dette er Loven om Skyldofferet. Det er højhelligt
\par 2 Der, hvor Brændofferet slagtes, skal Skyldofferet slagtes.
\par 3 og alt dets Fedt skal frembæres. Fedthalen, Fedtet, der dækker Indvoldene, og alt Fedtet på Indvoldene,
\par 4 begge Nyrerne med det Fedt, som sidder på dem ved Lændemusklerne, og Leverlappen, som skal skilles fra ved Nyrerne.
\par 5 Og Præsten skal bringe det som Røgoffer på Alteret, et Ildoffer for HERREN. Det er et Skyldoffer.
\par 6 Alle af Mandkøn blandt Præsterne må spise det; på et helligt Sted skal det spises; det er højhelligt.
\par 7 Det er med Skyldofferet som med Syndofferet, en og samme Lov gælder for dem: Det tilfalder den Præst, der skaffer Soning ved det.
\par 8 Den Præst, som frembærer nogens Brændoffer, ham skal Huden af det Brændoffer, han frembærer, tilfalde.
\par 9 Ethvert Afgrødeoffer, der bages i Ovnen, eller som er tilberedt i Pande eller på Plade, tilfalder den Præst, der frembærer det;
\par 10 men ethvert Afgrødeoffer, der er rørt i Olie eller tørt, tilfalder alle Arons Sønner, den ene lige så vel som den anden.
\par 11 Dette er Loven om Takofferet, som bringes HERREN.
\par 12 Hvis det bringes som Lovprisningsoffer, skal han sammen med Slagtofferet, der hører til hans Lovprisningsoffer, frembære usyrede Kager, rørte i Olie, usyrede Fladbrød, smurte med Olie, og fint Hvedemel, æltet til Kager, rørte i Olie;
\par 13 sammen med syrede Brødkager skal han frembære sin Offergave som sit Lovprisningstakoffer.
\par 14 Han skal deraf frembære een Kage af hver Offergave som en Offerydelse til HERREN; den tilfalder den Præst, der sprænger Blodet af Takofferet på Alteret.
\par 15 Kødet af hans Lovprisningstakoffer skal spises på selve Offerdagen, intet de1af må gemmes til næste Morgen.
\par 16 Er hans Offergaver derimod et Løfteoffer eller et Frivilligoffer, skal det vel spises på selve Offerdagen, men hvad der levnes, må spises Dagen efter;
\par 17 men hvad der så er tilbage af Offerkødet, skal opbrændes på den tredje Dag;
\par 18 og hvis der spises noget af hans Takoffers Kød på den tredje Dag, så vil den, som bringer Offeret, ikke kunne finde Guds Velbehag, det skal ikke tilregnes ham, men regnes for urent Kød, og den, der spiser deraf, skal undgælde for sin Brøde.
\par 19 Det Kød, der kommer i Berøring med noget som helst urent, må ikke spises, det skal opbrændes. I øvrigt må enhver, der er ren, spise Kødet;
\par 20 men enhver, som i uren Tilstand spiser Kød af HERRENs Takoffer, skal udryddes af sin Slægt;
\par 21 og når nogen rører ved noget urent, enten menneskelig Urenhed eller urent Kvæg eller nogen Slags urent Kryb, og så spiser Kød af HERRENs Takoffer, skal han udryddes af sin Slægt.
\par 22 HERREN talede fremdeles til Moses og sagde:
\par 23 Tal til Israeliterne og sig: I må ikke spise noget som helst Fedt af Okser, Får eller Geder.
\par 24 Fedt af selvdøde og sønderrevne Dyr må bruges til alt, men I må under ingen Omstændigheder spise det.
\par 25 Thi enhver, der spiser Fedtet af det Kvæg, hvoraf der bringes HERREN Ildofre, den, der spiser noget deraf, skal udryddes af sit Folk.
\par 26 Og I må heller ikke nyde noget som helst Blod af Fugle eller Kvæg, hvor I end opholder eder;
\par 27 enhver, der nyder noget som helst Blod, skal udryddes af sin Slægt.
\par 28 HERREN talede fremdeles til Moses og sagde:
\par 29 Tal til Israeliterne og sig: Den, der bringer HERREN sit Takoffer, skal af sit Takoffer frembære for HERREN den ham tilkommende Offergave;
\par 30 med egne Hænder skal han frembære HERRENs Ildofre. Han skal frembære Fedtet tillige med Brystet; Brystet, for at Svingningen kan udføres dermed for HERRENs Åsyn;
\par 31 og Præsten skal bringe Fedtet som Røgoffer på Alteret, men Brystet skal tilfalde Aron og hans Sønner.
\par 32 Desuden skal I give Præsten højre Kølle som Offerydelse af eders Takofre.
\par 33 Den af Arons Sønner, der frembærer Takofferets Blod og Fedtet, ham tilfalder højre Kølle som hans Del.
\par 34 Thi Svingningsbrystet og Offerydelseskøllen tager jeg fra Israeliterne af deres Takofre og giver dem til Præsten Aron og hans Sønner, en evig gyldig Rettighed, som de har Krav på hos Israeliterne.
\par 35 Det er Arons og hans Sønners Del af HERRENs Ildofre, den, som blev givet dem, den dag han lod dem træde frem for at gøre Præstetjeneste for HERREN,
\par 36 den, som HERREN, den Dag han salvede dem, bød Israeliterne at give dem, en evig gyldig Rettighed, som de har Krav på fra Slægt til Slægt.
\par 37 Det er Loven om Brændofferet, Afgrødeofferet, Syndofferet, Skyldofferet, Indsættelsesofferet og Takofferet,
\par 38 som HERREN pålagde Moses på Sinaj Bjerg, den Dag han bød Israeliterne at bringe HERREN deres Offergaver i Sinaj Ørken.

\chapter{8}

\par 1 HERREN talede fremdeles til Moses og sagde:
\par 2 "Tag Aron og hans Sønner sammen med ham, Klæderne, Salveolien, Syndoffertyren, de to Vædre og Kurven med de usyrede Brød
\par 3 og kald hele Menigheden sammen ved indgangen til Åbenbaringsteltet!"
\par 4 Moses gjorde som HERREN bød ham, og Menigheden forsamlede sig ved Indgangen til Åbenbaringsteltet.
\par 5 Og Moses sagde til Menigheden: "Dette har HERREN påbudt at gøre."
\par 6 Da lod Moses Aron og hans Sønner træde frem og tvættede dem med Vand.
\par 7 Og han gav ham Kjortelen på, omgjordede ham med Bæltet, iførte ham Kåben, gav ham Efoden på, omgjordede ham med Efodens Bælte og bandt dermed Efoden fast på ham;
\par 8 så anbragte han Brystskjoldet på ham, lagde Urim og Tummim i Brystskjoldet,
\par 9 lagde Hovedklædet om hans Hoved og fæstede Guldpladen, det hellige Diadem, på Forsiden af Hovedklædet, som HERREN havde pålagt Moses.
\par 10 Derpå tog Moses Salveolien og salvede Boligen og alle Tingene deri og helligede dem;
\par 11 så bestænkede han Alteret syv Gange dermed og salvede Alteret og alt dets Tilbehør og Vandkummen med dens Fodstykke for at hellige dem;
\par 12 derpå udgød han. noget af Salveolien over Arons Hoved og salvede ham for at hellige ham.
\par 13 Derefter lod Moses Arons Sønner træde frem, iførte dem Kjortler, omgjordede dem med Bælter og bandt Huer på deres Hoveder, som HERREN havde pålagt Moses.
\par 14 Så førte han Syndoffertyren frem, og Aron og hans Sønner lagde deres Hænder lå Syndoffertyrens Hoved.
\par 15 Derefter slagtede Moses den, tog Blodet og strøg med sin Finger noget deraf rundt om på Alterets Horn og rensede Alteret for Synd; men Resten af Blodet udgød han ved Alterets Fod; således helligede han det ved at skaffe Soning for det.
\par 16 Så tog Moses alt Fedtet på Indvoldene, Leverlappen og begge Nyrerne med Fedtet på dem og bragte det som Røgoffer på Alteret.
\par 17 Men Tyren, dens Hud, Kød og Skarn, brændte han uden for Lejren, som HERREN havde pålagt Moses.
\par 18 Derpå førte han Brændoffervæderen frem, og Aron og hans Sønner lagde deres Hænder på Væderens Hoved.
\par 19 Så slagtede Moses den og sprængte Blodet rundt om på Alteret;
\par 20 og Moses skar Væderen i Stykker og bragte Hovedet, Stykkerne og Fedtet som Røgoffer,
\par 21 men Indvoldene og Skinnebenene tvættede han med Vand; og så bragte Moses hele Væderen som Røgoffer på Alteret. Det var et Brændoffer til en liflig Duft, et Ildoffer for HERREN, som HERREN havde pålagt Moses.
\par 22 Derpå førte han den anden Væder, Indsættelsesvæderen, frem, og Aron og hans Sønner lagde deres Hænder på Væderens Hoved.
\par 23 Så slagtede Moses den, tog noget af dens Blod og strøg det på Arons højre Øreflip og på hans højre Tommelfinger og højre Tommeltå.
\par 24 Derpå lod Moses Arons Sønner træde frem og strøg noget af Blodet på deres højre Øreflip og på deres højre Tommelfinger og højre Tommeltå, Men Resten af Blodet sprængte han rundt om på Alteret.
\par 25 Derpå tog han Fedtet, Fedthalen, alt Fedtet på Indvoldene, Leverlappen og begge Nyrerne med Fedtet på dem og den højre Kølle:
\par 26 Og af Kurven med de usyrede Brød, som stod for HERRENs Åsyn, tog han en usyret Kage, en Oliebrødkage og et Fladbrød og lagde dem oven på Fedtstykkerne og den højre Kølle,
\par 27 lagde det så alt sammen på Arons og hans Sønners Hænder og lod dem udføre Svingningen dermed for HERRENs Åsyn.
\par 28 Derpå tog Moses det igen fra dem og bragte det som Røgoffer på Alteret oven på Brændofferet. Det var et Indsættelsesoffer til en liflig Duft, et Ildoffer for HERREN.
\par 29 Så tog Moses Brystet og udførte Svingningen dermed for HERRENs Åsyn; det tilfaldt Moses som hans Del af Indsættelsesvæderen, som HERREN havde pålagt Moses.
\par 30 Derefter tog Moses noget af Salveolien og af Blodet på Alteret og stænkede det på Aron og hans Klæder, ligeledes på hans Sønner og deres Klæder, og helligede således Aron og hans Klæder og ligeledes hans Sønner og deres Klæder.
\par 31 Og Moses sagde til Aron og hans Sønner: "Kog Kødet ved indgangen til Åbenbaringsteltet og spis det der tillige med Brødet, som er i Indsættelseskurven, således som Budet lød til mig: Aron og hans Sønner skal spise det!
\par 32 Men hvad der levnes af Kødet og Brødet skal I opbrænde.
\par 33 I syv bage må I ikke vige fra Indgangen til Åbenbaringsteltet, indtil eders Indsættelsesdage er omme; thi syv Dage varer eders Indsættelse.
\par 34 Ligesom i Dag har HERREN påbudt eder at gøre også de følgende Dage for at skaffe eder Soning.
\par 35 Ved Indgangen til Åbenbaringsteltet skal I opholde eder Dag og Nat i syv Dage og holde eder HERRENs Forskrift efterrettelig, for at I ikke skal dø; thi således lød hans Bud til mig!"
\par 36 Og Aron og hans Sønner gjorde alt, hvad HERREN havde påbudt ved Moses.

\chapter{9}

\par 1 Den ottende Dag kaldte Moses Aron og hans Sønner og Israels Ældste til sig
\par 2 og sagde til Aron: "Tag dig en Kalv til Syndoffer og en Væder til Brændoffer, begge uden Lyde, og før dem frem for HERRENs Åsyn.
\par 3 Og tal således til Israeliterne: Tag eder en Gedebuk til Syndoffer, en Kalv og et Lam, begge årgamle og uden Lyde, til Brændoffer
\par 4 og en Okse og en Væder til Takoffer for at ofre dem for HERRENs Åsyn og desuden et Afgrødeoffer, rørt i Olie; thi i Dag vil HERREN vise sig for eder!"
\par 5 Da tog de, hvad Moses havde pålagt dem, og bragte det hen foran Åbenbaringsteltet; og hele Menigheden trådte frem og stillede sig for HERRENs Åsyn.
\par 6 Og Moses sagde: "Det er dette, HERREN har pålagt eder at gøre, for at HERRENs Herlighed kan vise sig for eder."
\par 7 Så sagde Moses til Aron: "Træd hen til Alteret og bring dit Syndoffer og dit Brændoffer for at skaffe dig og dit Hus Soning og bring så Folkets Offergave for at skaffe det Soning, således som HERREN har påbudt!"
\par 8 Da trådte Aron hen til Alteret og slagtede sin Syndofferkalv;
\par 9 og Arons Sønner bragte ham Blodet, og han dyppede sin Finger i Blodet og strøg det på Alterets Horn; men Resten af Blodet hældte han ud ved Alterets Fod.
\par 10 Derpå bragte han Syndofferets Fedt, Nyrer og Leverlap som Røgoffer på Alteret, således som HERREN havde pålagt Moses;
\par 11 men Kødet og Huden opbrændte han uden for Lejren.
\par 12 Derefter slagtede han Brændofferet, og Arons Sønner rakte ham Blodet, og han sprængte det rundt om på Alteret.
\par 13 Så rakte de ham Brændofferet, Stykke for Stykke, tillige med Hovedet, og han bragte det som Røgoffer på Alteret.
\par 14 Men Indvoldene og Skinnebenene tvættede han med Vand og bragte det derpå som Røgoffer oven på Brændofferet på Alteret.
\par 15 Derefter frembar han Folkets Offergave. Først tog han Folkets Syndofferbuk, slagtede den og ofrede den som Syndoffer på samme Måde som den før nævnte;
\par 16 så frembar han Brændofferet og ofrede det på den foreskrevne Måde;
\par 17 derefter frembar han Afgrødeofferet, tog en Håndfuld deraf og bragte det som Røgoffer på Alteret foruden det daglige Morgenbrændoffer;
\par 18 så slagtede han Oksen og Væderen som Takoffer fra Folket. Og Arons Sønner rakte ham Blodet, og han sprængte det rundt om på Alteret.
\par 19 Men Fedtstykkerne af Oksen og Væderen, Fedthalen, Fedtet på Indvoldene, Nyrerne og Leverlappen,
\par 20 disse Fedtdele lagde de oven på Bryststykkerne, og Fedtstykkerne bragte han som Røgoffer på Alteret,
\par 21 men med Bryststykkerne og højre Kølle udførte Aron Svingningen for HERRENs Åsyn, som Moses havde påbudt.
\par 22 Derefter løftede Aron sine Hænder over Folket og velsignede dem og steg så ned efter at have bragt Syndofferet, Brændofferet og Takofferet.
\par 23 Moses og Aron gik derpå ind i Åbenbaringsteltet, og da de kom ud derfra, velsignede de Folket. Da viste HERRENs Herlighed sig for alt Folket;
\par 24 og Ild for ud fra HERRENs Åsyn og fortærede Brændofferet og Fedtstykkerne på Alteret. Og alt Folket så det, og de jublede og faldt ned på deres Ansigt.

\chapter{10}

\par 1 Men Arons Sønner Nadab og Abihu tog hver sin Pande, kom Ild i dem og lagde Røgelse derpå og frembar for HERRENs Åsyn fremmed Ild, som han ikke havde pålagt dem.
\par 2 Da for Ild ud fra HERRENs Åsyn og fortærede dem, så de døde for HERRENs Åsyn.
\par 3 Moses sagde da til Aron: "Det er det, HERREN talede om, da han sagde: Jeg viser min Hellighed på dem, der står mig nær, og min Herlighed for alt Folkets Øjne!" Og Aron tav.
\par 4 Da kaldte Moses Misjael og Elzafan, Arons Farbroder, Uzziels Sønner, til sig og sagde til dem: "Kom og bær eders Frænder bort fra Helligdommen uden for Lejren!"
\par 5 Og de kom og bar dem uden for Lejren i deres Kjortler, som Moses havde sagt.
\par 6 Men Moses sagde til Aron og hans Sønner Eleazar og Itamar: "I må hverken lade eders Hår vokse frit eller sønderrive eders Klæder, ellers skal I dø og Vrede komme over hele Menigheden; lad eders Brødre, hele Israels Hus, begræde den Brand, HERREN har antændt;
\par 7 og vig ikke fra Åbenbaringsteltets indgang, ellers skal I dø, thi HERRENs Salveolie er på eder!" Og de gjorde som Moses sagde.
\par 8 Og HERREN talede til Aron og sagde:
\par 9 Vin og stærk Drik må hverken du eller dine Sønner drikke, når I gået ind i Åbenbaringsteltet, for at I ikke skal dø. Det skal være eder en evig gyldig Anordning fra Slægt til Slægt,
\par 10 for at I kan gøre Skel mellem det hellige og det, der ikke er helligt, og mellem det urene og det rene,
\par 11 og for at I kan vejlede Israeliterne i alle de Love, HERREN har kundgjort dem ved Moses.
\par 12 Og Moses sagde til Aron og hans tilbageblevne Sønner Eleazar og Itamar: "Tag Afgrødeofferet, der er levnet fra HERRENs Ildoffer, og spis det usyret ved Siden af Alteret, thi det er højhelligt;
\par 13 I skal spise det på et helligt Sted; det er jo din og dine Sønners retmæssige Del af HERRENs Ildofre; thi således er det mig påbudt.
\par 14 Og Svingningsbrystet og Offerydelseskøllen skal I spise på et rent Sted, du, dine Sønner og Døtre, thi de er givet dig tillige med dine Sønner som en retmæssig Del af Israeliternes Takofre;
\par 15 Offerydelseskøllen og Svingningsbrystet skal man frembære sammen med de til Ildofre bestemte Fedtdele, for at Svingningen kan udføres for HERRENs Åsyn, og de skal tilfalde dig og dine Sønner tillige med dig som en evig gyldig Rettighed, således som HERREN har påbudt!"
\par 16 Og Moses spurgte efter Syndofferbukken, men se, den var opbrændt. Da blev han fortørnet på Eleazar og Itamar, Arons tilbageblevne Sønner, og sagde:
\par 17 "Hvorfor har I ikke spist Syndofferet på det hellige Sted? Det er jo dog højhelligt, og han har givet eder det. for at I skal borttage Menighedens Skyld og således skaffe dem Soning for HERRENs Åsyn.
\par 18 Se, Blodet deraf er ikke blevet bragt ind i Helligdommens Indre, derfor havde det været eders Pligt at spise det på det hellige Sted, således som jeg har påbudt!"
\par 19 Men Aron svarede Moses: "Se, de har i Dag frembåret deres Syndoffer og Brændoffer for HERRENs Åsyn, og en sådan Tilskikkelse har ramt mig! Om jeg i Dag havde spist Syndofferkød, vilde HERREN da have billiget det?"
\par 20 Da Mose's hørte dette, billigede han det.

\chapter{11}

\par 1 HERREN talede til Moses og Aron og sagde til dem:
\par 2 Tal til Israeliterne og sig: Af alle Dyr på Jorden må du spise følgende:
\par 3 Alle Dyr, der har Klove, helt spaltede Klove, og tygger Drøv, må I spise.
\par 4 Men følgende må I ikke spise af dem der tygger Drøv, og af dem, der har Klove: Kamelen, thi den tygger vel Drøv, men har ikke Klove; den skal være eder uren;
\par 5 Klippegrævlingen, thi den tygger vel Drøv, men har ikke Klove: den skal være eder uren;
\par 6 Haren, thi den tygger vel Drøv, men har ikke Klove; den skal være eder uren;
\par 7 Svinet, thi det har vel Klove, helt spaltede Klove, men tygger ikke Drøv; det skal være eder urent.
\par 8 Deres Kød må I ikke spise, og ved Åslerne af dem må I ikke røre; de skal være eder urene.
\par 9 Af alt, hvad der lever i Vandet, må I spise følgende: Alt i Vandet, både i Havet og i Floderne, som har Finner og Skæl, må I spise;
\par 10 men af alt, hvad der rører sig i Vandet, af alle levende Væsener i Vandet, skal det, som ikke har Finner eller Skæl, hvad enten det er i Havet eller Floderne, være eder en Vederstyggelighed.
\par 11 De skal være eder en Vederstyggelighed, deres Kød må I ikke spise, og Ådslerne af dem skal I regne for en Vederstyggelighed.
\par 12 Alt i Vandet, som ikke har Finner eller Skæl, skal være eder en Vederstyggelighed.
\par 13 Følgende er de Fugle, I skal regne for en Vederstyggelighed, de må ikke spises, de er en Vederstyggelighed: Ørnen, Lammegribben, Havørnen,
\par 14 Glenten, de forskellige Arter af Falke,
\par 15 alle de forskellige Arter af Ravne,
\par 16 Strudsen, Takmasfuglen, Mågen, de forskellige Arter af Høge,
\par 17 Uglen, Fiskepelikanen, Hornuglen,
\par 18 Tinsjemetfuglen,Pelikanen, Ådselgribben,
\par 19 Storken, de forskellige Arter af Hejrer, Hærfuglen og Flagermusen.
\par 20 Alt vinget Kryb der går på fire, skal være eder en Vederstyggelighed.
\par 21 Af alt det vingede Kryb, som går på fire, må I dog spise følgende: Dem, der har Skinneben oven over Fødderne til at hoppe med på Jorden.
\par 22 Af dem må I spise følgende: De forskellige Arter af Græshopper, Solamgræshopper, Hargolgræshopper og Hagabgræshopper.
\par 23 Men alt andet vinget Kryb, der går på fire, skal være eder en Vederstyggelighed.
\par 24 Ved følgende Dyr bliver I urene, enhver, der rører ved Ådslerne af dem, bliver uren til Aften,
\par 25 og enhver, der bærer et Ådsel af dem, skal tvætte sine Klæder og være uren til Aften:
\par 26 Alle Dyr, der har Klove, men ikke helt spaltede, og som ikke tygger Drøv, skal være urene; enhver, der rører ved dem, bliver uren.
\par 27 Alle firføddede Dyr, der går på Poter, skal være eder urene; enhver der rører ved et Ådsel af dem, bliver uren til Aften,
\par 28 og den, som bærer et Ådsel af dem, skal tvætte sine Klæder og være uren til Aften; de skal være eder urene.
\par 29 Af Krybet, der kryber på Jorden, skal følgende være eder urene: Væselen, Musen, de forskellige Arter af Firben,
\par 30 Anakaen, Koadyret, Letåen, Homedyret og Tinsjemetdyret.
\par 31 Det er dem, som skal være eder urene af alt Krybet. Enhver, der rører ved dem, når de er døde, skal være uren til Aften;
\par 32 alt, hvad et dødt Dyr af den Slags falder ned på, bliver urent, alle Træting, Klæder, Skind, Sække, overhovedet alt, hvad der bruges ved et eller andet Arbejde; det skal lægges i Vand og være urent til Aften; derpå skal det være rent.
\par 33 Og falder et af den Slags ned i et Lerkar, så bliver alt, hvad der er deri, urent, og ethvert sådant Kar skal I slå i Stykker;
\par 34 alt spiseligt, som man kommer Vand på, og alt flydende, der tjener til Drikke, bliver urent i et sådant Kar.
\par 35 Alt, hvad et dødt Dyr af den Slags falder ned på, bliver urent; Ovne og Ildsteder skal nedbrydes, de er urene, og urene skal de være eder.
\par 36 Dog bliver Kilder og Cisterner, Steder, hvor der samler sig Vand, ved at være rene; men den, der rører ved Ådslerne deri, bliver uren.
\par 37 Falder et dødt Dyr af den Slags ned på nogen som helst Slags Sæd, der bruges til Udsæd, bliver denne ved at være ren;
\par 38 men kommes der Vand på Sæden, og der så falder et dødt Dyr af den Slags ned på den, skal den være eder uren.
\par 39 Når noget af det Kvæg, der tjener eder til Føde, dør, skal den, der rører ved den døde Krop, være uren til Aften.
\par 40 Den, der spiser noget af den døde Krop, skal tvætte sine Klædet og være uren til Aften, og den, der bærer den døde Krop, skal tvætte sine Klæder og være uren til Aften.
\par 41 Alt Kryb, der kryber på Jorden, er en Vederstyggelighed, det må ikke spises;
\par 42 intet af det, der kryber på Bugen, og intet af det, der går på fire, så lidt som det, der har mange Ben, intet af Krybet, der kryber på Jorden, må I spise thi det er en Vederstyggelighed.
\par 43 Gør ikke eder selv til en Vederstyggelighed ved noget som helst krybende Kryb, gør eder ikke urene derved, så I bliver urene deraf:
\par 44 thi jeg er HERREN eders Gud, og I skal hellige eder og være hellige, thi jeg er hellig. Gør eder ikke urene ved noget som helst Kryb, der rører sig på Jorden;
\par 45 thi jeg er HERREN, der førte eder op fra Ægypten for at være eders Gud; I skal være hellige, thi jeg er hellig!
\par 46 Det er Loven om Dyr og Fugle og alle levende Væsener, der bevæger sig i Vandet, og alle levende Væsener, der kryber på Jorden,
\par 47 til Skel mellem det urene og det rene, mellem de Dyr, der må spises, og dem, der ikke må spises.

\chapter{12}

\par 1 Og HERREN talede til Moses og sagde:
\par 2 Tal til Israeliterne og sig: Når en Kvinde bliver frugtsommelig og føder en Dreng, skal hun være uren i syv Dage; ligesom i den Tid hun har sin månedlige Urenhed, skal hun være uren.
\par 3 På den ottende Dag skal Drengen omskæres på sin Forhud.
\par 4 Derefter skal hun holde sig hjemme i tre og tredive Dage, medens hun har sit Renselsesblod; hun må ikke røre ved noget helligt eller komme til Helligdommen, før hendes Renselsestid er omme.
\par 5 Føder hun derimod et Pigebarn, skal hun være uren i to Uger ligesom under sin månedlige Urenhed; og derpå skal hun holde sig hjemme i seks og tresindstyve Dage, medens hun har sit Renselsesblod.
\par 6 Når hendes Renselsestid er omme, skal hun, både efter et Drengebarn og et Pigebarn, bringe et årgammelt Lam som Brændoffer og en Dueunge eller Turteldue som Syndoffer til Præsten ved Åbenbaringsteltets Indgang;
\par 7 og han skal frembære det for HERRENs Åsyn og skaffe hende Soning, så hun bliver ren efter sit Blodtab. Det er Loven om en Kvinde. der føder, hvad enten det er en Dreng eller en Pige.
\par 8 Men hvis hun ikke evner at give et Lam, skal hun tage to Turtelduer eller Dueunger, en til Brændoffer og en til Syndoffer, og Præsten skal skaffe hende Soning, så hun bliver ren.

\chapter{13}

\par 1 Og HERREN talede til Moses og Aron og sagde:
\par 2 Når der på et Menneskes Hud viser sig en Hævelse eller Udslæt eller en lys Plet, som kan blive til Spedalskhed på hans Hud, skal han føres hen til Præsten Aron eller en af hans Sønner, Præsterne.
\par 3 Præsten skal da syne det syge Sted på Huden, og når Hårene på det syge Sted er blevet hvide og Stedet ser ud til at ligge dybere end Huden udenom, så er det Spedalskhed, og så skal Præsten efter at have synet ham erklære ham for uren.
\par 4 Men hvis det er en hvid Plet på Huden og den ikke ser ud til at ligge dybere end Huden udenom og Hårene ikke er blevet hvide.
\par 5 og på den syvende Dag skal Præsten syne ham. Viser det sig da, at Ondet ikke har skiftet Udseende eller bredt sig på Huden, skal Præsten igen lukke ham inde i syv Dage;
\par 6 og på den syvende Dag skal Præsten atter syne ham. Hvis det da viser sig, at Ondet er ved at svinde, og at det ikke har bredt sig på Huden, skal Præsten erklære ham for ren; da er det almindeligt Udslæt på Huden; han skal da tvætte sine Klæder og være ren.
\par 7 Men hvis Udslættet breder sig på Huden, efter at han har ladet Præsten syne sig for at blive erklæret for ren, og hvis Præsten, når han anden Gang lader sig syne af ham,
\par 8 ser, at Udslættet har bredt sig på Huden, så skal Præsten erklære ham for uren; det er Spedalskhed.
\par 9 Når et Menneske angribes af Spedalskhed, skal han føres hen til Præsten,
\par 10 og Præsten skal syne ham; og når der da viser sig at være en hvid Hævelse på Huden og Hårene derpå er blevet hvide og der vokser vildt Kød i Hævelsen,
\par 11 så er det gammel Spedalskhed på hans Hud, og da skal Præsten erklære ham for uren; han behøver ikke at lukke ham inde, thi han er uren.
\par 12 Men hvis Spedalskheden bryder ud på Huden og Spedalskheden bedækker hele den angrebnes Hud fra Top til Tå, så vidt Præsten kan se,
\par 13 og Præsten ser, at Spedalskheden bedækker hele hans Legeme, så skal han erklære den angrebne for ren; han er blevet helt hvid, han er ren.
\par 14 Men så snart der viser sig vildt Kød på ham, er han uren;
\par 15 og når Præsten ser det vilde Kød, skal han erklære ham for uren; det vilde Kød er urent, det er Spedalskhed.
\par 16 Hvis derimod det vilde Kød forsvinder og han bliver hvid, skal han gå til Præsten;
\par 17 og hvis det, når Præsten syner ham, viser sig, at den angrebne er blevet hvid, skal Præsten erklære den angrebne for ren; han er ren.
\par 18 Når nogen på sin Hud har haft en Betændelse, som er lægt,
\par 19 og der så på det Sted, som var betændt, kommer en hvid Hævelse eller en rødlighvid Plet, skal han lad sig syne af Præsten;
\par 20 og hvis Præsten finder, at Stedet ser ud til at ligge dybere end Huden udenom og Hårene derpå er blevet hvide, skal Præsten erklære ham for uren; det er Spedalskhed, der er brudt frem efter Betændelsen.
\par 21 Men hvis der, når Præsten syner det, ikke viser sig at være hvide Hår derpå og det ikke ligger dybere end Huden udenom, men er ved at svinde, da skal Præsten lukke ham inde i syv Dage;
\par 22 og når det da breder sig på Huden, skal Præsten erklære ham for uren; det er Spedalskhed.
\par 23 Men hvis den hvide Plet bliver, som den er, uden at brede sig, da er det et Ar efter Betændelsen, og Præsten skal erklære ham for ren.
\par 24 Eller når nogen får et Brandsår på Huden, og det Kød, der vokser i Brandsåret, frembyder en rødlighvid eller hvid Plet,
\par 25 så skal Præsten syne ham, og hvis det da viser sig, at Hårene på Pletten er blevet hvide og den ser ud til at ligge dybere end Huden udenom, så er det Spedalskhed, der er brudt frem i Brandsåret; og da skal Præsten erklære ham for uren; det er Spedalskhed.
\par 26 Men hvis det, når Præsten synet ham, viser sig, at der ingen hvide Hår er på den lyse Plet, og at en ikke ligger dybere end Huden udenom, men at den er ved at svinde, så skal Præsten lukke ham inde i syv Dage;
\par 27 og på den syvende Dag skal Præsten syne ham, og når den da har bredt sig på Huden, skal Præsten erklære ham for uren; det er Spedalskhed.
\par 28 Men hvis den lyse Plet bliver, som den er, uden at brede sig på Huden, og er ved at svinde, så er det en Hævelse efter Brandsåret, og da skal Præsten erklære ham for ren; thi det er et Ar efter Brandsåret.
\par 29 Når en Mand eller Kvinde angribes i Hoved eller Skæg,
\par 30 skal Præsten syne det syge Sted, og hvis det da ser ud til at ligge dybere end Huden udenom og der er guldgule, tynde Hår derpå, så skal Præsten erklære ham for uren; det er Skurv, Spedalskhed i Hoved eller Skæg.
\par 31 Men hvis det skurvede Sted, når Præsten syner det, ikke ser ud til at ligge dybere end Huden udenom, uden at dog Hårene derpå er sorte, da skal Præsten lukke den skurvede inde i syv dage;
\par 32 og på den syvende Dag skal Præsten syne ham, og hvis da Skurven ikke har bredt sig og der ikke er kommet guldgule Hår derpå og Skurven ikke ser ud til at ligge dybere end Huden udenom,
\par 33 da skal den angrebne lade sig rage uden dog at lade det skurvede Sted rage; så skal Præsten igen lukke den skurvede inde i syv Dage.
\par 34 På den syvende Dag skal Præsten syne Skurven, og hvis det da viser sig, at Skurven ikke har bredt sig på Huden, og at den ikke ser ud til at ligge dybere end Huden udenom, så skal Præsten erklære ham for ren; da skal han tvætte sine Klæder og være ren.
\par 35 Men hvis Skurven breder sig på Huden, efter at han er erklæret for ren,
\par 36 da skal Præsten syne ham; og hvis det så viser sig, at Skurven har bredt sig, behøver Præsten ikke at undersøge, om der er guldgule Hår; han er uren.
\par 37 Men hvis Skurven ikke har skiftet Udseende og der er vokset sorte Hår frem derpå, da er Skurven lægt; han er ren, og Præsten skal erklære ham for ren.
\par 38 Når en Mand eller Kvinde får lyse Pletter, hvide Pletter på Huden,
\par 39 skal Præsten syne dem; og hvis der da på deres Hud viser sig hvide Pletter, det er ved at svinde, er det Blegner, der er brudt ud på Huden; han er ren.
\par 40 Når nogen bliver skaldet på Baghovedet, så er han kun isseskaldet; han er ren.
\par 41 Og hvis han bliver skaldet ved Panden og Tindingerne, så er han kun pandeskaldet; han er ren.
\par 42 Men kommer der på hans skaldede isse eller Pande et rødlighvidt Sted, er det Spedalskhed. der bryder frem på hans skaldede Isse eller Pande.
\par 43 Så skal Præsten syne ham, og viser det sig da, at Hævelsen på det syge Sted på hans skaldede Isse eller Pande er rødlighvid, af samme Udseende som Spedalskhed på Huden,
\par 44 så er han spedalsk; han er uren, og Præsten skal erklære ham for uren; på sit Hoved er han angrebet.
\par 45 Den, der er spedalsk, den, som lider af Sygdommen, skal gå med sønder1evne Klæder, hans Hår skal vokse frit, han skal tilhylle sit Skæg, og: "uren, uren!" skal han råbe.
\par 46 Så længe han er angrebet, skal han være uren; uren er han, for sig selv skal han bo, uden for Lejren skal hans Opholdsted være.
\par 47 Når der kommer Spedalskhed på en Klædning enten af Uld eller Lærred
\par 48 eller på vævet eller knyttet Stof af Lærred eller Uld eller på Læder eller Læderting af enhver Art
\par 49 og det angrebne Sted på Klædningen, Læderet, det vævede eller knyttede Stof eller Lædertingene viser sig at være grønligt eller rødligt, så er det Spedalskhed og skal synes af Præsten.
\par 50 Og når Præsten har synet Skaden, skal han lukke den angrebne Ting inde i syv Dage.
\par 51 På den syvende Dag skal han syne den angrebne Ting, og dersom Skaden har bredt sig på Klædningen, det vævede eller knyttede Stof eller Læderet, de forskellige Læderting, så er Skaden ondartet Spedalskhed, det er urent.
\par 52 Da skal han brænde Klædningen eller det af Uld eller Lærred vævede eller knyttede Stof eller alle de Læderting, som er angrebet; thi det er ondartet Spedalskhed, det skal opbrændes.
\par 53 Men hvis Præsten finder, at Skaden ikke har bredt sig på Klædningen eller det vævede eller knyttede Stof eller på de forskellige Slags Læderting,
\par 54 så skal Præsten påbyde, at den angrebne Ting skal tvættes, og derpå igen lukke den inde i syv Dage.
\par 55 Præsten skal da syne den angrebne Ting, efter at den er tvættet, og viser det sig da, at Skaden ikke har skiftet Udseende, så er den uren, selv om Skaden ikke har bredt sig; du skal opbrænde den; det er ædende Udslæt på Retten eller Vrangen.
\par 56 Men hvis det, når Præsten syner det, viser sig, at Skaden er ved at svinde efter Tvætningen, så skal han rive det angrebne Sted af Klædningen eller Læderet eller det vævede eller knyttede Stof.
\par 57 Viser det sig da igen på Klædningen eller det vævede eller knyttede Stof eller de forskellige Læderling, da er det Spedalskhed, der er ved at bryde ud; du skal opbrænde de angrebne Ting.
\par 58 Men den Klædning eller det vævede eller knyttede Stof eller de forskellige Læderting, hvis Skade svinder efter Tvætningen, skal tvættes på ny; så er det rent.
\par 59 Det er Loven om Spedalskhed på Klæder af Uld eller Lærred eller på vævet eller knyttet Stof eller på alskens Læderting; efter den skal de erklæres for rene eller urene.

\chapter{14}

\par 1 HERREN talede fremdeles til Moses og sagde:
\par 2 Dette er Loven om, hvorledes man skal forholde sig med Renselsen af en spedalsk: Han skal fremstilles for Præsten,
\par 3 og Præsten skal gå uden for Lejren og syne ham, og viser det sig da, at Spedalskheden er helbredt hos den spedalske,
\par 4 skal Præsten give Ordre til at tage to levende, rene Fugle, Cedertræ, karmoisinrødt Garn og en Ysopkvist til den, der skal renses.
\par 5 Og Præsten skal give Ordre til at slagte den ene Fugl over et Lerkar med rindende Vand.
\par 6 Så skal han tage den levende Fugl, Cedertræet, det kamoisinrøde Garn og Ysopkvisten og dyppe dem tillige med den levende Fugl i Blodet af den Fugl, der er slagtet over det rindende Vand,
\par 7 og syv Gange foretage Bestænkning på den, der skal renses for Spedalskhed, og således rense ham; derpå skal han lade den levende Fugl flyve ud over Marken.
\par 8 Men den, der skal renses, skal tvætte sine Klæder, afrage alt sit Hår og bade sig i Vand; så er han ren. Derefter må han gå ind i Lejren, men han skal syv Dage opholde sig uden for sit Telt.
\par 9 På den syvende Dag skal han afrage alt sit Hår, sit Hovedhår, sit Skæg, sine Øjenbryn, alt sit Hår skal han afrage, og han skal tvætte sine Klæder og bade sit Legeme i Vand; så er han ren.
\par 10 Men den ottende dag skal han tage to lydefri Væderlam og et lydefrit, årgammelt Hunlam, desuden tre Tiendedele Efa fint Hvedemel, rørt i Olie, til Afgrødeoffer og en Log Olie.
\par 11 Så skal den Præst, der foretager Renselsen, stille den, der skal renses, tillige med disse Offergaver frem for HERRENs Åsyn ved Åbenbaringsteltets Indgang.
\par 12 Og Præsten skal tage det ene Væderlam og frembære det som Skyldoffer tillige med den Log Olie, som hører dertil, og udføre Svingningen dermed for HERRENs Åsyn.
\par 13 Og han skal slagte Lammet der, hvor Syndofferet og Brændofferet slagtes, på det hellige Sted, thi ligesom Syndofferet tilfalder Skyldofferet Præsten; det er højhelligt.
\par 14 Derpå skal Præsten tage noget af Skyldofferets Blod, og Præsten skal stryge det på højre Øreflip af den, der skal renses, og på hans højre Tomme1finger og højre Tommeltå;
\par 15 og Præsten skal tage noget af den Log Olie, som hører dertil, og hælde det i sin venstre Hånd,
\par 16 og Præsten skal dyppe sin højre Pegefinger i den Olie, han har i sin venstre Hånd, og med sin Finger stænke Olien syv Gange foran HERRENs Åsyn.
\par 17 Og af den Olie, han har tilbage i sin Hånd, skal Præsten stryge noget på højre Øreflip af den, der skal renses, og på hans højre Tommelfinger og højre Tommeltå oven på Skyldofferets Blod.
\par 18 Og Præsten skal hælde det, der er tilbage af Olien i hans Hånd, på Hovedet af den, der skal renses, og således skal Præsten skaffe ham Soning for HERRENs Åsyn.
\par 19 Derpå skal Præsten ofre Syndofferet og skaffe den, der skal renses, Soning for hans Urenhed.
\par 20 Og Præsten skal ofre Brændofferet og Afgrødeofferet på Alteret og således skaffe ham Soning; så er han ren.
\par 21 Men hvis han er fattig og ikke evner at give så meget, skal han tage et enkelt Lam til Skyldoffer. til at udføre Svingningen med, for at der kan skaffes ham Soning. desuden en Tiendedel Efa fint Hvedemel, rørt i Olie, til Afgrødeoffer og en Log Olie
\par 22 og to Turtelduer eller Dueunger, hvad han nu evner at give, den ene skal være til Syndoffer, den anden til Brændoffer.
\par 23 Dem skal han den ottende Dag efter sin Renselse bringe til Præsten ved Åbenbaringsteltets Indgang for HERRENs Åsyn.
\par 24 Så skal Præsten tage Skyldofferlammet med den Log Olie, som hører dertil, og Præsten skal udføre Svingningen dermed for HERRENs Åsyn.
\par 25 Og han skal slagte Skyldofferlammet, og Præsten skal tage noget af Skyldofferets Blod og stryge det på højre Øreflip af den, der skal renses, og på hans højre Tommelfinger og højre Tommeltå.
\par 26 Og af Olien skal Præsten hælde noget i sin venstre Hånd,
\par 27 og Præsten skal med sin højre Pegefinger syv Gangestænke noget af Olien, som er i hans venstre Hånd, for HERRENs Åsyn.
\par 28 Og af Olien, som er i hans Hånd, skal Præsten stryge noget på højre Øreflip af den, der skal renses, og på hans højre Tommelfinger og højre Tommeltå oven på Skyldofferets Blod.
\par 29 Og Resten af Olien, som er i hans Hånd, skal Præsten hælde på Hovedet af den, der skal renses, for at skaffe ham Soning for HERRENs Åsyn.
\par 30 Derpå skal han ofre den ene Turteldue eller Dueunge, hvad han nu har evnet at give,
\par 31 den ene som Syndoffer, den anden som Brændoffet, sammen med Afgrødeofferet; og Præsten skal skaffe den, der skal renses, Soning for HERRENs Åsyn.
\par 32 Det er Loven om den, der har været angrebet af Spedalskhed og ikke evner at bringe det almindelige Offer ved sin Renselse.
\par 33 Og HERREN talede til Moses og Aron og sagde:
\par 34 Når I kommer til Kanåns Land, som jeg vil give eder i Eje, og jeg lader Spedalskhed komme frem på et Hus i eders Ejendomsland,
\par 35 så skal Husets Ejer gå hen og melde det til Præsten og sige: "Der har i mit Hus vist sig noget, der ligner Spedalskhed!"
\par 36 Da skal Præsten give Ordre til at flytte alt ud af Huset, inden han kommer for at syne Pletten, for at ikke noget af, hvad der er i Huset, skal blive urent; derpå skal Præsten komme for at syne Huset.
\par 37 Viser det sig da, når han syner Pletten, at Pletten på Husets Vægge frembyder grønlige eller rødlige Fordybninger, der ser ud til at ligge dybere end Væggen udenom,
\par 38 skal Præsten gå ud af Huset til Husets Dør og holde Huset lukket i syv Dage.
\par 39 På den syvende Dag skal Præsten komme igen og syne det, og hvis det da viser sig, at Pletten har bredt sig på Husets Vægge,
\par 40 skal Præsten give Ordre til at udtage de angrebne Sten og kaste dem hen på et urent Sted uden for Byen
\par 41 og til at skrabe Lerpudset af Husets indvendige Vægge og hælde det fjernede Puds ud på et urent Sted uden for Byen.
\par 42 Derefter skal man tage andre Sten og sætte dem ind i Stedet for de gamle og ligeledes tage nyt Puds og pudse Huset dermed.
\par 43 Hvis Pletten så atter bryder frem i Huset, efter at Stenene er taget ud, Pudset skrabet af og Huset pudset på ny,
\par 44 skal Præsten komme og syne Huset, og viser det sig da, at Pletten har bredt sig på Huset, så er det ondartet Spedalskhed, der er på Huset; det er urent.
\par 45 Da skal man rive Huset ned, Sten, Træværk og alt Pudset på Huset, og bringe det til et urent Sted uden for Byen.
\par 46 Den, som går ind i Huset, så længe det er lukket, skal være uren til Aften;
\par 47 den, der sover deri skal tvætte sine Klæder, og den der spiser deri, skal tvætte sine Klæder.
\par 48 Men hvis det, når Præsten kommer og syner Huset, viser sig, at Pletten ikke har bredt sig på det, efter at det er pudset på ny, skal Præsten erklære Huset for rent, thi Pletten er helbredt.
\par 49 Da skal han for at rense Huset for Synd tage to Fugle, Cedertræ, karmoisinrødt Garn og en Ysopkvist.
\par 50 Den ene Fugl skal han slagte over et Lerkar med rindende Vand,
\par 51 og han skal tage Cedertræet, Ysopkvisten, det karmoisinrøde Garn og den levende Fugl og dyppe dem i Blodet af den slagtede Fugl og det rindende Vand og syv Gange foretage Bestænkning på Huset
\par 52 og således rense det for Synd med Fuglens Blod, det rindende Vand, den levende Fugl, Cedertræet, Ysopkvisten og det karmoisinrøde Garn.
\par 53 Og den levende Fugl skal han lade flyve ud af Byen hen over Marken og således skaffe Huset Soning; så er det rent.
\par 54 Det er Loven om al Slags Spedalskhed og Skurv,
\par 55 om Spedalskhed på Klæder og Huse
\par 56 og om Hævelser, Udslæt og lyse Pletter,
\par 57 til Belæring om, når noget er urent, og når det er rent. Det er Loven om Spedalskhed.

\chapter{15}

\par 1 HERREN talede fremdeles til Moses og Aron og sagde:
\par 2 Tal til Israeliterne og sig til dem: Når en Mand får Flåd fra sin Blusel, da er dette hans Flåd urent.
\par 3 Og således skal man forholde sig med den ved hans Flåd opståede Urenhed: Hvad enten hans Blusel flyder eller holder sit Flåd tilbage, er der Urenhed hos ham.
\par 4 Ethvert Leje, som den, der lider af Flåd, ligger på, skal være urent, og ethvert Sæde, han sidder på, skal være urent.
\par 5 Den der rører ved hans Leje, skal tvætte sine Klæder og bade sig i Vand og være uren til Aften;
\par 6 den, der sidder på et Sæde, som den, der lider af Flåd, har siddet på, skal tvætte sine Klæder og bade sig i Vand og være uren til Aften;
\par 7 den, der rører ved en. der lider af Flåd, skal tvætte sine Klæder og bade sig i Vand og være uren til Aften.
\par 8 Hvis den, der lider af Flåd, spytter på en, som er uren, skal denne tvætte sine Klæder og bade sig i Vand og være uren til Aften.
\par 9 Ethvert Befordringsmiddel, som bruges af den, der lider af Flåd, skal være urent.
\par 10 Den, der rører ved noget, han har ligget eller siddet på, skal være uren til Aften; den, der bærer noget sådant, skal tvætte sine Klæder og bade sig i Vand og være uren til Aften.
\par 11 Enhver, som den, der lider af Flåd, rører ved uden at have skyllet sine Hænder i Vand, skal tvætte sine Klæder og bade sig i Vand og være uren til Aften.
\par 12 Lerkar, som den, der lider af Flåd, rører ved, skal slås itu, og alle Trækar skal skylles i Vand.
\par 13 Men når den, der lider af Flåd, bliver ren for sit Flåd, skal han tælle syv bage frem, fra den Dag han bliver ren, og så tvætte sine Klæder og bade sit Legeme i rindende Vand; så er han ren.
\par 14 Og den ottende dag skal han tage sig to Turtelduer eller Dueunger og komme hen for HERRENs Åsyn ved Åbenbaringsteltets indgang og give Præsten dem.
\par 15 Så skal Præsten bringe dem som Offer, den ene som Syndoffer, den anden som Brændoffer, og således skal Præsten skaffe ham Soning for HERRENs Åsyn for hans Flåd.
\par 16 Når der går Sæd fra en Mand, skal han bade hele sit Legeme i Vand og være uren til Aften.
\par 17 Enhver Klædning og alt Læder, der er kommet Sæd på, skal tvættes i Vand og være urent til Aften.
\par 18 Og når en Mand har Samleje med en Kvinde, skal de bade sig i Vand og være urene til Aften.
\par 19 Når en Kvinde har Flåd, idet der flyder Blod fra hendes Blusel, da skal hendes Urenhed vare syv Dage. Enhver, der rører ved hende, skal være uren til Aften.
\par 20 Alt, hvad hun ligger på under sin månedlige Urenhed, skal være urent, og alt, hvad hun sidder på, skal være urent.
\par 21 Enhver, der rører ved hendes Leje, skal tvætte sine Klæder og bade sig i Vand og være uren til Aften;
\par 22 og enhver, der rører ved et Sæde, hun har siddet på, skal tvætte sine Klæder og bade sig i Vand og være uren til Aften.
\par 23 Og hvis nogen rører ved noget, der har ligget på Lejet eller Sædet, hun har siddet på, skal han være uren til Aften.
\par 24 Dersom en Mand ligger ved Siden af hende og hendes Urenhed kommer på ham, skal han være uren i syv Dage, og ethvert Leje, han ligger på, skal være urent.
\par 25 Men når en Kvinde har Blodflåd i længere Tid, uden for den Tid hun har sin månedlige Urenhed, eller når Flåddet varer længere end sædvanligt ved hendes månedlige Urenhed, så skal hun, i al den Tid hendes urene Flåd varer, være stillet, som i de Dage hun har sin månedlige Urenhed; hun skal være uren;
\par 26 ethvert Leje, hun ligger på, så længe hendes Flåd varer, skal være som hendes Leje under hendes månedlige Urenhed, og et hvert Sæde, hun sidder på, skal være urent som under hendes månedlige Urenhed;
\par 27 enhver, der rører derved, bliver uren og skal tvætte sine Klæder og bade sig i Vand og være uren til Aften.
\par 28 Men når hun bliver ren for sit Flåd, skal hun tælle syv Dage frem og så være ren.
\par 29 På den ottende Dag skal hun tage sig to Turtelduer eller Dueunger og bringe dem til Præsten ved Åbenbaringsteltets Indgang.
\par 30 Og Præsten skal ofre den ene som Syndoffer, den anden som Brændoffer, og således skal Præsten skaffe hende Soning for HERRENs Åsyn for hendes Urenheds Flåd.
\par 31 I skal advare Israeliterne for deres Urenhed, for at de ikke skal dø i deres Urenhed, når de gør min Bolig, som er i deres Midte, uren.
\par 32 Det er Loven om Mænd, der lider af Flåd, og fra hvem der går Sæd, så de bliver urene derved,
\par 33 og om Kvinder, der lider af deres månedlige Urenhed, om Mænd og Kvinder, der har Flåd, og om Mænd, der ligger ved Siden af urene Kvinder.

\chapter{16}

\par 1 HERREN talede til Moses, efter at Døden havde ramt Arons to Sønner, da de trådte frem for HERRENs Åsyn og døde,
\par 2 og HERREN sagde til Moses: Sig til din Broder Aron, at han ikke til enhver Tid må gå ind i Helligdommen inden for Forhænget foran Sonedækket på Arken, ellers skal han dø, thi jeg kommer til Syne i Skyen over Sonedækket.
\par 3 Kun således må Aron komme ind i Helligdommen: Med en ung Tyr til Syndoffer og en Væder til Brændoffer;
\par 4 han skal iføre sig en hellig Linnedkjortel, bære Linnedbenklæder over sin Blusel, omgjorde sig med et Linnedbælte og binde et Linned Hovedklæde om sit Hoved; det er hellige Klæder; og han skal bade sit Legeme i Vand, før han ifører sig dem.
\par 5 Af Israeliternes Menighed skal han tage to Gedebukke til Syndoffer og en Væder til Brændoffer.
\par 6 Så skal Aron ofre sin egen Syndoffertyr og skaffe sig og sit Hus Soning.
\par 7 Derefter skal han tage de to Bukke og stille dem frem for HERRENs Åsyn ved Indgangen til Åbenbaringsteltet.
\par 8 Og Aron skal kaste Lod om de to Bukke, et Lod for HERREN og et for Azazel;
\par 9 og den Buk, der ved Loddet tilfalder HERREN, skal Aron føre frem og ofre som Syndoffer;
\par 10 men den Buk, der ved Loddet tilfalder Azazel, skal fremstilles levende for HERRENs Åsyn, for at man kan fuldbyrde Soningen over den og sende den ud i Ørkenen til Azazel.
\par 11 Aron skal da føre sin egen Syndoffertyr frem og skaffe sig og sit Hus Soning og slagte sin egen Syndoffertyr.
\par 12 Derpå skal han tage en Pandefuld Gløder fra Alteret for HERRENs Åsyn og to Håndfulde stødt, vellugtende Røgelse og bære det inden for Forhænget.
\par 13 Og han skal komme Røgelse på Ilden for HERRENs Åsyn, så at Røgelsesskyen skjuler Sonedækket oven over Vidnesbyrdet, for at han ikke skal dø.
\par 14 Så skal han tage noget af Tyrens Blod og stænke det med sin Finger fortil på Sonedækket, og foran Sonedækket skal han syv Gange stænke noget af Blodet med sin Finger.
\par 15 Derefter skal han slagte Folkets Syndofferbuk, bære dens Blod inden for Forhænget og gøre med det som med Tyrens Blod, stænke det på Sonedækket og foran Sonedækket.
\par 16 Således skal han skaffe Helligdommen Soning for Israeliternes Urenhed og deres Overtrædelser, alle deres Synder, og på samme Måde skal han gøre med Åbenbaringsteltet, der har sin Plads hos dem midt i deres,Urenhed.
\par 17 Intet Menneske må komme i Åbenbaringsteltet, når han går ind for at skaffe Soning i Helligdommen, før han går ud igen. Således skal han skaffe sig selv, sit Hus og hele Israels Forsamling Soning.
\par 18 Så skal han gå ud til Alteret, som står for HERRENs Åsyn, og skaffe det Soning; han skal tage noget af Tyrens og Bukkens Blod og stryge det rundt om på Alterets Horn,
\par 19 og han skal syv Gange stænke noget af Blodet derpå med sin Finger og således rense det og hellige det for Israeliternes Urenheder.
\par 20 Når han så er færdig med at skaffe Helligdommen, Åbenbaringsteltet og Alteret Soning, skal han føre den levende Buk frem.
\par 21 Aron skal lægge begge sine Hænder på Hovedet af den levende Buk og over den bekende alle Israeliternes Misgerninger og alle deres Overtrædelser, alle deres Synder, og lægge dem på Bukkens Hoved og så sende den ud i Ørkenen ved en Mand, der holdes rede dertil.
\par 22 Bukken skal da bære alle deres Misgerninger til et øde Land, og så skal han slippe Bukken løs i Ørkenen.
\par 23 Derpå skal Aron gå ind i Åbenbaringsteltet, afføre sig Linnedklæderne, som han tog på, da han gik ind i Helligdommen, og lægge dem der;
\par 24 så skal han bade sit Legeme i Vand på et helligt Sted, iføre sig sine sædvanlige Klæder og gå ud og ofre sit eget Brændoffer og Folkets Brændoffer og således skaffe sig og Folket Soning.
\par 25 Og Syndofferets Fedt skal han bringe som Røgoffer på Alteret.
\par 26 Men den, som fører Bukken ud til Azazel, skal tvætte sine Klæder og bade sit Legeme i Vand; derefter må han komme ind i Lejren.
\par 27 Men Syndoffertyren og Syndofferbukken, hvis Blod blev båret ind for at skaffe Soning i Helligdommen, skal man bringe uden for Lejren, og man skal brænde deres Hud og deres Kød og Skarn.
\par 28 Og den, der brænder dem, skal tvætte sine Klæder og bade sit Legeme; derefter må han komme ind i Lejren.
\par 29 Det skal være eder en evig gyldig Anordning. Den tiende Dag i den syvende Måned skal I faste og afholde eder fra alt Arbejde, både den indfødte og den fremmede, der bor iblandt eder.
\par 30 Thi den Dag skaffes der eder Soning til eders Renselse; fra alle eders Synder renses I for HERRENs Åsyn.
\par 31 Det skal være eder en fuldkommen Hviledag, og I skal faste: det skal være en evig gyldig Anordning.
\par 32 Præsten, som salves og indsættes til at gøre Præstetjeneste i. Stedet for sin Fader, skal skaffe Soning, han skal iføre sig Linnedklæderne, de hellige Klæder,
\par 33 og han skal skaffe det Allerhelligste Soning; og Åbenbaringsteltet og Alteret skal han skaffe Soning; og Præsterne og alt Folkets Forsamling skal han skaffe Soning.
\par 34 Det skal være eder en evig gyldig Anordning, for at der kan skaffes Israeliterne Soning for alle deres Synder een Gang om Året. Og Aron gjorde som HERREN bød Moses.

\chapter{17}

\par 1 HERREN talede til Moses og sagde:
\par 2 Tal til Aron og hans Sønner og alle Israeliterne og sig til dem: Dette har HERREN påbudt:
\par 3 Om nogen af Israels Hus slagter et Stykke Hornkvæg, et Får eller en Ged i Lejren, eller han slagter dem uden for Lejren,
\par 4 uden at bringe dem hen til Åbenbaringsteltets Indgang for at bringe HERREN en Offergave foran HERRENs Bolig, da skal dette tillegnes den Mand som Blodskyld; han har udgydt Blod, og den Mand skal udryddes af sit Folk.
\par 5 Dette er anordnet, for at Israeliterne skal bringe deres Slagtofre, som de slagter ude på Marken, til HERREN, til Åbenbaringsteltets Indgang, til Præsten, og ofre dem som Takofre til HERREN.
\par 6 Præsten skal da sprænge Blodet på HERRENs Alter ved Indgangen til Åbenbaringsteltet og bringe Fedtet som Røgoffer, en liflig Duft for HERREN.
\par 7 Og de må ikke mere ofre deres Slagtofre til Bukketroldene, som de boler med. Det skal være en evig gyldig Anordning for dem fra Slægt til Slægt!
\par 8 Og du skal sige til dem: Om nogen af Israels Hus eller de fremmede, der bor iblandt eder, ofrer et Brændoffer eller Slagtoffer
\par 9 og ikke bringer det hen til Åbenbaringsteltets Indgang for at ofre det til HERREN, da skal den Mand udryddes af sin Slægt.
\par 10 Om nogen af Israels Hus eller af de fremmede, der bor iblandt dem, nyder noget Blod, så vender jeg mit Åsyn mod den, der nyder Blodet, og udrydde1 ham af hans Folk.
\par 11 Thi Kødets Sjæl er i Blodet, og jeg har givet eder det til Brug på Alteret til at skaffe eders Sjæle Soning; thi det er Blodet, som skaffer Soning, fordi det er Sjælen.
\par 12 Derfor har jeg sagt til Israeliterne: Ingen af eder må nyde Blod; heller ikke den fremmede, der bor iblandt eder, må nyde Blod.
\par 13 Om nogen af Israeliterne eller af de fremmede, der bor iblandt dem, nedlægger et Stykke Vildt eller en Fugl af den Slags, der må spises, da skal han lade Blodet løbe ud og dække det med Jord.
\par 14 Thi om alt Køds Sjæl gælder det, at dets Blod er dets Sjæl; derfor har jeg sagt til Israeliterne: I må ikke nyde Blodet af noget som helst Kød, thi alt Køds Sjæl er dets Blod; enhver, der nyder det, skal udryddes.
\par 15 Enhver, der spiser seldøde eller sønderrevne Dyr, det være sig en indfødt eller en fremmed, skal tvætte sine Klæder og bade sig i Vand og være uren til Aften; så er han ren.
\par 16 Men hvis han ikke tvætter sine Klæder og bader sig, skal han undgælde for sin Brøde.

\chapter{18}

\par 1 HERREN talede fremdeles til Moses og sagde:
\par 2 Tal til Israeliterne og sig til dem: Jeg er HERREN eders Gud!
\par 3 Som de handler i Ægypten, hvor I opholdt eder, må I ikke handle, og som de handler i Kana'ans Land, hvor jeg fører eder hen, må I ikke handle; I må ikke vandre efter deres Anordninger.
\par 4 Efter mine Lovbud skal I handle. og mine Anordninger skal I holde, så I vandrer efter dem; jeg er HERREN eders Gud!
\par 5 I skal holde mine Anordninger og Lovbud; det Menneske, der handler efter dem, skal leve ved dem. Jeg er HERREN!
\par 6 Ingen af eder må komme sine kødelige Slægtninge nær, så han blotter deres Blusel. Jeg er HERREN!
\par 7 Din Faders og din Moders Blusel må du ikke blotte; hun er din Moder, du må ikke blotte hendes Blusel!
\par 8 Din Faders Hustrus Blusel må du ikke blotte, det er din Faders Blusel.
\par 9 Din Søsters Blusel, hvad enten hun er din Faders eller din Moders Datter, hvad enten hun er født i eller uden for Hjemmet, hendes Blusel må du ikke blotte.
\par 10 Din Sønnedatters eller Datterdatters Blusel må du ikke blotte, det er din egen Blusel.
\par 11 En Datter, din Faders Hustru har med din Fader hun er din Søster hendes Blusel må du ikke blotte.
\par 12 Din Fasters Blusel må du ikke blotte, hun er din Faders kødelige Slægtning.
\par 13 Din Mosters Blusel må du ikke blotte, hun er din Moders kødelige Slægtning.
\par 14 Din Farbroders Blusel må du ikke blotte, du må ikke komme hans Hustru nær, hun er din Faster.
\par 15 Din Sønnekones Blusel må du ikke blotte, hun er din Søns Hustru, du må ikke blotte hendes Blusel.
\par 16 Din Broders Hustrus Blusel må du ikke blotte, det er din Broders Blusel.
\par 17 En Kvindes og hendes Datters Blusel må du ikke blotte, heller ikke må du ægte hendes Sønnedatter eller Datterdatter, så at du blotter hendes Blusel; de er hendes kødelige Slægtninge; det er grov Utugt.
\par 18 Søster må du ikke tage til Søsters Medhustru, så længe Søsteren lever, så du blotter både den enes og den andens Blusel.
\par 19 Du må ikke komme en Kvinde nær under hendes månedlige Urenhed, så du blotter hendes Blusel.
\par 20 Med din Næstes Hustru må du ikke have Samleje, så du bliver uren ved hende.
\par 21 Dit Afkom må du ikke give hen til at ofres til Molok; du må ikke vanhellige din Guds Navn. Jeg er HERREN!
\par 22 Hos en Mand må du ikke ligge, som man ligger hos en Kvinde; det er en Vederstyggelighed.
\par 23 Med intet som helst Dyr må du have Omgang, så du bliver uren derved; en Kvinde må ikke stille sig hen for et dyr til kønslig Omgang; det er en Skændsel.
\par 24 Gør eder ikke urene med noget sådant, thi med alt sådant har de Folkeslag, jeg driver bort foran eder, gjort sig urene.
\par 25 Derved blev Landet urent, og jeg straffede det for dets Brøde, og Landet udspyede sine Indbyggere.
\par 26 Hold derfor mine Anordninger og Lovbud og øv ikke nogen af disse Vederstyggeligheder, det gælder både den indfødte og den fremmede, der bor iblandt eder
\par 27 thi alle disse Vederstyggeligheder øvede Indbyggerne, som var der før eder, og Landet blev urent
\par 28 for at ikke Landet skal udspy eder, når I gør det urent, ligesom det udspyede det Folk, som var der før eder.
\par 29 Thi enhver, som øver nogen af alle disse Vederstyggeligheder, de, der øver dem, skal udryddes af deres Folk.
\par 30 Så hold mine Forskrifter, så I ikke øver nogen af de vederstyggelige Skikke, som øvedes før eders Tid, at I ikke skal gøre eder urene ved dem. Jeg er HERREN eders Gud!

\chapter{19}

\par 1 HERREN talede fremdeles til Moses og sagde:
\par 2 Tal til hele Israeliternes Menighed og sig til dem: I skal være hellige, thi jeg HERREN eders Gud er hellig!
\par 3 I skal frygte hver sin Moder og sin Fader, og mine Sabbater skal I holde. Jeg er HERREN eders Gud!
\par 4 Vend eder ikke til Afguderne og gør eder ikke støbte Gudebilleder! Jeg er HERREN eders Gud !
\par 5 Når I ofrer Takoffer til HERREN, skal I ofre det således, at I kan vinde Guds Velbehag.
\par 6 Den Dag, I ofrer det, og Dagen efter må det spises, men hvad der levnes til den tredje Dag, skal opbrændes;
\par 7 spises det den tredje Dag, er det at regne for råddent Kød og vinder ikke Guds Velbehag;
\par 8 den, der spiser deraf, skal undgælde for sin Brøde, thi han vanhelliger det, som var helliget HERREN, og det Menneske skal udryddes af sin Slægt.
\par 9 Når I høster eders Lands Høst, må du ikke høste helt hen til Kanten af din Mark, ej heller må du sanke Efterslætten efter din Høst.
\par 10 Heller ikke må du bolde Efterhøst eller sanke de nedfaldne Bær i din Vingård; til den fattige og den fremmede skal du lade det blive tilbage. Jeg er HERREN eders Gud!
\par 11 I må ikke stjæle, I må ikke lyve, I må ikke bedrage hverandre.
\par 12 I må ikke sværge falsk ved mit Navn, så du vanhelliger din Guds Navn. Jeg er HERREN!
\par 13 Du må intet aftvinge din Næste, du må intet røve; Daglejerens Løn må ikke blive hos dig Natten over.
\par 14 Du må ikke forbande den døve eller lægge Stød for den blindes Fod, du skal frygte din Gud. Jeg er HERREN!
\par 15 I må ikke øve Uret, når I holder Rettergang; du må ikke begunstige den ringe, ej heller tage Parti for den store, men du skal dømme din Næste med Retfærdighed.
\par 16 Du må ikke gå rundt og bagvaske din Landsmand eller stå din Næste efter Livet. Jeg er HERREN!
\par 17 Du må ikke bære Nag til din Broder i dit Hjerte, men du skal tale din Næste til Rette, at du ikke skal pådrage dig Synd for hans Skyld.
\par 18 Du må ikke hævne dig eller gemme på Vrede mod dit Folks Børn, du skal elske din Næste som dig selv. Jeg er HERREN!
\par 19 Hold mine Anordninger! Du må ikke lade to Slags Kvæg parre sig med hinanden; du må ikke så to Slags Sæd i din Mark; og du må ikke bære Klæder, der er vævede af to Slags Garn.
\par 20 Når en Mand har Samleje med en Kvinde, og det er en Trælkvinde, en anden Mands Medhustru, som ikke er løskøbt eller frigivet, så skal Afstraffelse finde Sted; dog skal de ikke lide Døden, thi hun var ikke frigivet.
\par 21 Og han skal bringe sit Skyldoffer for HERREN til Åbenbaringsteltets Indgang, en Skyldoffervæder,
\par 22 og Præsten skal med Skyldoffervæderen skaffe ham Soning for HERRENs Åsyn for den Synd, han har begået, så han finder Tilgivelse for den Synd, han har begået.
\par 23 Når I kommer ind i Landet og planter alskens Frugttræer, skal I lade deres Forhud, den første Frugt, urørt; i tre År skal de være eder uomskårne og må ikke spises;
\par 24 det fjerde År skal al deres Frugt under Høstjubel helliges HERREN;
\par 25 først det femte År må I spise deres Frugt, for at I kan få så meget større Udbytte deraf. Jeg er HERREN eders Gud!
\par 26 I må ikke spise noget med Blodet i. I må ikke give eder af med at tage Varsler og øve Trolddom.
\par 27 I må ikke runde Håret på Tindingerne; og du må ikke studse dit Skæg;
\par 28 I må ikke gøre Indsnit i eders Legeme for de dødes Skyld eller indridse Tegn på eder. Jeg er HERREN!
\par 29 Du må ikke vanhellige din Datter ved at lade hende bedrive Hor, for at ikke Landet skal forfalde til Horeri og fyldes med Utugt.
\par 30 Mine Sabbater skal I bolde, og min Helligdom skal I frygte.
\par 31 Henvend eder ikke til Genfærd og Sandsigerånder; søg dem ikke, så I gør eder urene ved dem. Jeg er HERREN eders Gud!
\par 32 Du skal rejse dig for de grå Hår og ære Oldingen, og du skal frygte din Gud. Jeg er HERREN!
\par 33 Når en fremmed bor hos dig i eders Land, må I ikke lade ham lide Overlast;
\par 34 som en af eders egne skal I regne den fremmede, der bor hos eder, og du1 skal elske ham som dig selv, thi I var selv fremmede i Ægypten. Jeg et HERREN eders Guld!
\par 35 Når I holder Rettergang, må I ikke øve Uret ved Længdemål, Vægt eller Rummål;
\par 36 Vægtskåle, der vejer rigtigt, Lodder, der holder Vægt, Efa og Hin, der holder Mål, skal I have. Jeg er HERREN eders Gud, som førte eder ud af Ægypten!
\par 37 Hold alle mine Anordninger og Lovbud og gør efter dem. Jeg er HERREN!

\chapter{20}

\par 1 HERREN talede fremdeles til Moses og sagde:
\par 2 Sig til Israeliterne: Om nogen af Israeliterne eller de fremmede, der bor hos Israel, giver sit Afkom hen til Molok, da skal han lide Døden; Landets Indbyggere skal stene ham,
\par 3 og jeg vender selv mit Åsyn imod den Mand og udrydder ham af hans Folk, fordi han gav sit Afkom hen til Molok for at gøre min Helligdom uren og vanhellige mit hellige Navn;
\par 4 og ser end Landets Indbyggere igennem Fingre med den Mand, når han giver sit Afkom hen til Molok, og undlader at dræbe ham,
\par 5 så vender jeg dog selv mit Åsyn imod den Mand og hans Slægt og udrydder ham og alle dem, der følger i hans Spor og boler med Molok, af deres Folk.
\par 6 Det Menneske, som henvender sig til Genfærd eller Sandsigerånder og boler med dem, mod det Menneske vil jeg vende mit Åsyn og udrydde ham af hans Folk.
\par 7 Helliger eder og vær hellige; thi jeg er HERREN eders Gud!
\par 8 Hold mine Anordninger og gør efter dem. Jeg er HERREN, som helliger eder!
\par 9 Thi enhver, som forbander sin Fader og sin Moder, skal lide Døden; han har forbandet sin Fader og sin Moder, derfor hviler der Blodskyld på ham.
\par 10 Om nogen bedriver Hor med en anden Mands Hustru, om nogen bedriver Hor med sin Næstes Hustru, da skal de lide Døden, Horkarlen såvel som Horkvinden.
\par 11 Om nogen har Samleje med sin Faders Hustru, har han blottet sin Faders Blusel; de skal begge lide Døden, der hviler Blodskyld på dem.
\par 12 Om nogen har Samleje med sin Sønnekone, skal de begge lide Døden; de har øvet Skændselsdåd, der hviler Blodskyld på dem.
\par 13 Om nogen ligger hos en Mand, på samme Måde som man ligger hos en Kvinde, da har de begge øvet en Vederstyggelighed; de skal lide Døden, der hviler Blodskyld på dem.
\par 14 Om nogen ægter en Kvinde og hendes Moder, er det grov Utugt: man skal brænde både ham og begge Kvinderne; der må ikke findes grov Utugt iblandt eder.
\par 15 Om nogen har Omgang med et Dyr, skal han lide Døden, og Dyret skal I slå ihjel.
\par 16 Om en Kvinde kommer noget Dyr nær for at have kønslig Omgang med det, da skal du ihjelslå både Kvinden og Dyret; de skal lide Døden, der hviler Blodskyld på dem.
\par 17 Om en Mand tager sin Søster, sin Faders eller sin Moders Datter, til Ægte, så han ser hendes Blusel, og hun hans, da er det en skammelig Gerning; de skal udryddes i deres Landsmænds Påsyn; han har blottet sin Søsters Blusel, han skal undgælde for sin Brøde.
\par 18 Om en Mand har Samleje med en Kvinde under hendes månedlige Svaghed og blotter hendes Blusel, idet han afdækker hendes Kilde, og hun afdækker sit Blods Kilde, da skal de begge udryddes af deres Folk.
\par 19 Du må ikke blotte din Mosters og din Fasters Blusel, thi den, der gør det, afdækker sin kødelige Slægtnings Blusel; de skal undgælde for deres Brøde.
\par 20 0m nogen har Samleje med sin Farbroders Hustru, da har han blottet sin Farbroders Blusel, de skal undgælde for deres Synd og dø barnløse.
\par 21 Om nogen tager sin Broders Hustru til Ægte, da er det en uren Gerning; han har blottet sin Broders Blusel; de skal blive barnløse.
\par 22 Hold alle mine Anordninger og Lovbud og gør efter dem, at ikke Landet, jeg føler eder ind at bo i, skal udspy eder.
\par 23 Følg ikke det Folks Skikke, som jeg driver bort foran eder, thi de har øvet alt dette; derfor væmmedes jeg ved dem
\par 24 og sagde til eder: I skal få deres Land i Eje; jeg giver eder det i Eje, et Land, der flyder med Mælk og Honning. Jeg er HERREN eders Gud, som har udskilt eder fra alle andre Folkeslag.
\par 25 I skal skelne mellem rene og urene Dyr og mellem urene og rene Fugle for ikke at gøre eder selv til en Vederstyggelighed ved de Dyr og de Fugle og alt, hvad der rører sig på Jorden, alt, hvad jeg har udskilt for eder og erklæret for urent.
\par 26 Og I skal være mig hellige, thi jeg HERREN er hellig, og jeg har udskilt eder fra alle andre Folkeslag til at høre mig til.
\par 27 Når der i en Mand eller Kvinde er en Genfærdsånd eller en Sandsigerånd, skal de lide Døden; de skal stenes, der hviler Blodskyld på dem.

\chapter{21}

\par 1 Og HERREN sagde til Moses: Tal til Præsterne, Arons Sønner, og sig til dem: Præsten må ikke gøre sig uren ved Lig blandt sin Slægt,
\par 2 medmindre det er hans nærmeste kødelige Slægtninge, hans Moder eller Fader, hans Søn eller Datter, hans Broder
\par 3 eller Søster, for så vidt hun var Jomfru og endnu hørte til hans Familie og ikke var gift; i så Fald må han gøre sig uren ved hende;
\par 4 men han må ikke gøre sig uren ved hende, når hun var gift med en Mand af hans Slægt, og således pådrage sig Vanhelligelse.
\par 5 De må ikke klippe sig en skaldet Plet på deres Hoved, ikke studse deres Skæg eller gøre Indsnit i deres Legeme.
\par 6 Hellige skal de være for deres Gud og må ikke vanhellige deres Guds Navn, thi de frembærer HERRENs Ildofre, deres Guds Spise; derfor skal de være hellige.
\par 7 En Horkvinde og en skændet Kvinde må de ikke ægte; heller ikke en Kvinde, der er forstødt af sin Mand, må de ægte; thi han er helliget sin Gud.
\par 8 Du skal regne ham for hellig, thi han frembærer din Guds Spise; han skal være dig hellig, thi hellig er jeg HERREN, som helliger dem.
\par 9 Når en Præstedatter vanhelliger sig ved at bedrive Hor, da vanhelliger hun sin Fader; hun skal brændes på Bål,
\par 10 Den Præst, der er den ypperste blandt sine Brødre, han, over hvis Hoved Salveolien udgydes, og som indsættes ved at iføres Klæderne, må hverken lade sit Hovedhår vokse frit eller sønderrive sine Klæder.
\par 11 Han må ikke gå hen til noget som helst Lig, end ikke ved sin Fader eller Moder må han gøre sig uren.
\par 12 Han må ikke forlade Helligdommen for ikke at vanhellige sin Guds Helligdom, thi hans Guds Salveolies Vielse er på ham; jeg er HERREN!
\par 13 Han skal ægte en Kvinde, der er Jomfru;
\par 14 en Enke, en forstødt, en skændet, en Horkvinde må han ikke ægte, kun en Jomfru af sin Slægt må han tage til Hustru,
\par 15 for at han ikke skal vanhellige sit Afkom blandt sin Slægt; thi jeg er HERREN, som helliger ham.
\par 16 HERREN talede fremdeles til Moses og sagde:
\par 17 Tal til Aron og sig: Ingen af dit Afkom i de kommende Slægter, som har en Legemsfejl, må træde frem for at frembære sin Guds Spise,
\par 18 det må ingen, som har en Legemsfejl, hverken en blind eller en halt eller en med vansiret Ansigt eller for lang en Legemsdel
\par 19 eller med Brud på Ben eller Arm
\par 20 eller en pukkelrygget eller en med Tæring eller en, der har Pletter i Øjnene eller lider af Skab eller Ringorm eller har svulne Testikler.
\par 21 Af Præsten Arons Efterkommere må ingen, som har en Legemsfejl, nærme sig for at frembære HERRENs Ildofre; han har en Legemsfejl, han må ikke nærme sig for at frembære sin Guds Spise.
\par 22 Han må vel spise sin Guds Spise, både det, som er højhelligt, og det, som er helligt,
\par 23 men til Forhænget må han ikke komme, og Alteret må han ikke nærme sig, thi han har en Legemsfejl og må ikke vanhellige mine hellige Ting; thi jeg er HERREN, som helliger dem.
\par 24 Og Moses talte således til Aron og hans Sønner og alle Israeliterne.

\chapter{22}

\par 1 HERREN talede fremdeles til Moses og sagde:
\par 2 Sig til Aron og hans Sønner, at de skal behandle Israeliternes Helliggaver, som de helliger mig, med Erefrygt, for at de ikke skal vanhellige mit hellige Navn. Jeg er HERREN!
\par 3 Sig til dem: Enhver af alle eders Efterkommere, som i de kommende Slægter i uren Tilstand kommer de Helliggaver nær, Israeliterne helliger HERREN, det Menneske skal udryddes fra mit Åsyn. Jeg er HERREN!
\par 4 Ingen af Arons Efterkommere, der er spedalsk eller lider af Flåd, må spise noget af Helliggaverne, før han bliver ren; den, der rører ved en, som er uren ved Lig, eller den, fra hvem der går Sæd,
\par 5 eller den, der rører ved noget Slags Kryb, ved hvilket man bliver uren, eller ved et Menneske, ved hvem man bliver uren, af hvad Art hans Urenhed være kan,
\par 6 enhver, der rører ved noget sådant, skal være uren til Aften og må ikke spise af Helliggaverne, før han har badet sit Legeme i Vand.
\par 7 Når Solen går ned, er han ren, og derefter må han spise af Helliggaverne, thi de er hans Mad.
\par 8 Selvdøde og sønderrevne Dyr må han ikke spise for ikke at gøre sig uren derved. Jeg er HERREN!
\par 9 De skal overholde mine Forskrifter, at de ikke skal pådrage sig Synd og dø derfor, fordi de vanhelliger det. Jeg er HERREN, som helliger dem.
\par 10 Ingen Lægmand må spise af det hellige; hverken den indvandrede hos Præsten eller hans Daglejer må spise af det hellige.
\par 11 Men når en Præst for sine Penge køber sig en Træl, da må denne spise deraf, og ligeledes må hans hjemmefødte Trælle spise af hans Mad.
\par 12 Når en Præstedatter ægter en Lægmand, må hun ikke spise af de ydede Helliggaver;
\par 13 men når en Præstedatter bliver Enke eller forstødes uden at have Børn og vender tilbage til sin Faders Hus og er der som i sine unge År, da må hun spise af sin Faders Mad. Men ingen Lægmand må spise deraf.
\par 14 Når nogen af Vanvare kommer til at spise af det hellige, skal han erstatte Præsten det hellige med Tillæg af en Femtedel.
\par 15 Præsterne må ikke vanhellige de Helliggaver, Israeliterne yder HERREN,
\par 16 og således bringe Brøde og Skyld over dem, når de spiser deres Helliggaver; thi jeg er HERREN, som helliger dem.
\par 17 HERREN talede fremdeles til Moses og sagde:
\par 18 Tal til Aron og hans Sønner og alle Israeliterne og sig til dem: Om nogen af Israels Hus eller af de fremmede i Israel bringer sin Offergave, hvad enten det er deres Løfteoffer eller Frivilligoffer, de bringer HERREN som Brændoffer,
\par 19 så skal I bringe dem således, at I kan vinde Guds Velbehag, et lydefrit Handyr af Hornkvæget, Fårene eller Gederne;
\par 20 I må ikke ofre noget Dyr, der har en Legemsfejl, thi derved vinder I ikke eders Guds Velbehag.
\par 21 Når nogen bringer HERREN et Takoffer af Hornkvæget eller Småkvæget enten for at indfri et Løfte eller som Frivilligoffer, da skal det være et lydefrit Dyr, for at det kan vinde Guds Velbehag; det må ingen som helst Legemsfejl have;
\par 22 et blindt Dyr eller et Dyr med Brud på Lemmerne eller et såret Dyr eller et Dyr, der lider af Bylder, Skab eller Ringorm, sådanne Dyr må I ikke bringe HERREN, og I må ikke lægge noget Ildoffer af den Slags på Alteret for HERREN.
\par 23 Et Stykke Hornkvæg eller Småkvæg med en for lang eller forkrøblet Legemsdel kan du bruge som Frivillig offer, men som Løfteoffer vinder det ikke Guds Velbehag.
\par 24 Dyr med udklemte, knuste, afrevne eller bortskårne Testikler må I ikke bringe HERREN; således må I ikke bære eder ad i eders Land.
\par 25 Heller ikke må I af en Udlænding købe den Slags Dyr og ofre dem som eders Guds Spise, thi de har en Lyde, de har en Legemsfejl; ved dem vinder I ikke Guds Velbehag.
\par 26 HERREN talede fremdeles til Moses og sagde:
\par 27 Når der fødes et Stykke Hornkvæg, et Får eller en Ged, skal de blive syv Dage hos Moderen; men fra den ottende Dag er de skikkede til at vinde HERRENs Velbehag som Ildoffergave til HERREN.
\par 28 I må ikke slagte et Stykke Hornkvæg eller Småkvæg samme Dag som dets Afkom.
\par 29 Når I ofrer et Lovprisningsoffer til HERREN, skal I ofre det således, at det kan vinde eder Guds Velbehag.
\par 30 Det skal spises samme Dag, I må intet levne deraf til næste Morgen. Jeg er HERREN!
\par 31 I skal holde mine Bud og handle efter dem. Jeg er HERREN!
\par 32 I må ikke vanhellige mit hellige Navn, for at jeg må blive helliget blandt Israeliterne. Jeg er HERREN, som helliger eder,
\par 33 som førte eder ud af Ægypten for at være eders Gud. Jeg er HERREN!

\chapter{23}

\par 1 HERREN talede fremdeles til Moses og sagde:
\par 2 Tal til Israeliterne og sig til dem: Hvad angår HERRENs Festtider, hvilke I skal udråbe som Højtidsstævner, da er mine Festtider følgende:
\par 3 I seks Dage skal der arbejdes, men den syvende Dag skal være en fuldkommen Hviledag med Højtidsstævne; I må intet Arbejde gøre, det er Sabbat for HERREN, overalt hvor I bor.
\par 4 Følgende er HERRENs Festtider med Højtidsstævner, som I skal udråbe, hver til sin Tid:
\par 5 På den fjortende Dag i den første Måned ved Aftenstid er det Påske for HERREN.
\par 6 På den femtende Dag i samme Måned er det de usyrede Brøds Højtid for HERREN; i syv Dage skal I spise usyret Brød.
\par 7 På den første Dag skal I bolde Højtidsstævne, I må intet Arbejde gøre.
\par 8 I skal bringe HERREN Ildoffer i syv Dage. På den syvende Dag skal der holdes Højtidsstævne, I må intet Arbejde gøre.
\par 9 HERREN talede fremdeles til Moses og sagde:
\par 10 Tal til Israeliterne og sig til dem: Når I kommer til det Land, jeg vil give eder, og høster dets Høst, skal I bringe Præsten Førstegrødeneget af eders Høst.
\par 11 Han skal udføre Svingningen med Neget for HERRENs Åsyn for at vinde eder Guds Velbehag; Dagen efter Sabbaten skal Præsten udføre Svingningen dermed.
\par 12 Og på den Dag I udfører Svingningen med Neget, skal I ofre et lydefrit, årgammelt Lam som Brændoffer til HERREN,
\par 13 og der skal høre to Tiendedele Eta fint Hvedemel, rørt i Olie, dertil som Afgrødeoffer, et Ildoffer for HERREN til en liflig Duft, og ligeledes en Fjerdedel Hin Vin som Drikoffer.
\par 14 Brød, ristede Aks eller nyhøstet Horn må I ikke spise før denne Dag, før I har frembåret eders Guds Offergave. Det skal være eder en evig gyldig Anordning fra Slægt til Slægt, overalt hvor I bor.
\par 15 Så skal I fra Dagen efter Sabbaten, fra den dag I bringer Svingningsneget, tælle syv Uger frem - det skal være hele Uger -
\par 16 til Dagen efter den syvende Sabbat, I skal tælle halvtredsinds tyve bage frem; da skal I frembære et nyt Afgrødeoffer for HERREN.
\par 17 Fra eders Boliger skal I bringe Svingningsbrød, to Brød, som skal laves af to Tiendedele Efa fint Hvedemel og bages syrede, en Førstegrødegave til HERREN.
\par 18 Og foruden Brødet skal I bringe syv lydefri, årgamle Lam, en ung Tyr og to Vædre, de skal være til Brændoffer for HERREN med tilhørende Afgrødeoffer og Drikoffer, et Ildoffer for HERREN til en liflig Duft.
\par 19 Og I skal ofre en Gedebuk som Syndoffer og to årgamle Lam som Takoffer.
\par 20 Og Præsten skal udføre Svingningen med dem, med de to Lam, for HERRENs Åsyn sammen med Førstegrødebrødet, de skal være HERREN helligede og tilfalde Præsten.
\par 21 På denne Dag skal I udråbe og holde et Højtidsstævne; I må intet Arbejde gøre. Det skal være eder en evig gyldig Anordning, overalt hvor I bor, fra Slægt til Slægt.
\par 22 Når I høster eders Lands Høst, må du ikke høste helt hen til Kanten af din Mark, ej heller må du sanke Efterslætten efter din Høst; til den fattige og den fremmede skal du lade det blive tilbage. Jeg er HERREN eders Gud!
\par 23 HERREN talede fremdeles til Moses og sagde:
\par 24 Tal til Israeliterne og sig: Den første dag i den syvende Måned skal I holde Hviledag med Hornblæsning til Ihukommelse og med Højtidsstævne;
\par 25 I må intet Arbejde gøre, og I skal bringe HERREN Ildofre.
\par 26 HERREN talede fremdeles til Moses og sagde:
\par 27 På den tiende Dag i samme syvende Måned falder Forsoningsdagen; da skal I holde Højtidsstævne, faste og bringe HERREN Ildofre;
\par 28 I må intet Arbejde gøre på denne Dag, thi det er Forsoningsdagen, den skal skaffe eder Soning for HERREN eders Guds Åsyn.
\par 29 Thi enhver, som ikke faster på denne Dag, skal udryddes af sin Slægt;
\par 30 og enhver, der gør noget som helst Arbejde på denne Dag, det Menneske vil jeg udslette af hans Folk.
\par 31 I må intet Arbejde gøre. Det skal være eder en evig Anordning fra Slægt til Slægt, overalt hvor I bor.
\par 32 Den skal være eder en fuldkommen Hviledag, og I skal faste; på den niende Dag i Måneden om Aftenen, fra denne Aften til næste Aften skal I holde eders Hviledag.
\par 33 HERREN talede fremdeles til Moses og sagde:
\par 34 Tal til Israeliterne og sig: Den femtende Dag i samme syvende Måned skal Løvhyttefesten fejres, den skal fejres i syv Dage for HERREN.
\par 35 På den første Dag skal der holdes Højtidsstævne, I må intet Arbejde gøre.
\par 36 Syv dage skal I bringe HERREN Ildofre; og på den ottende Dag skal I holde Højtidsstævne og bringe HERREN Ildofre; det er festlig Samling, I må intet Arbejde gøre.
\par 37 Det er HERRENs Festtider, hvilke I skal udråbe som Højtids stævner, ved hvilke der skal bringes HERREN Ildofre, Brændofre og Afgrødeofre, Slagtofre og drikofre, hver Dag de for den bestemte Ofre,
\par 38 foruden HERRENs Sabbater og foruden eders Gaver og alle eders Løfteofre og alle eders Frivilligofre, som I giver HERREN.
\par 39 Men den femtende Dag i den syvende Måned, når I har indsamlet Landets Afgrøde, skal I fejre HERRENs Højtid, og den skal fejres i syv Dage. På den første Dag skal der holdes Hviledag, og på den ottende dag skal der holdes Hviledag.
\par 40 Den første Dag skal I tage eder smukke Træfrugter, Palmegrene og Kviste af Løvtræer og Vidjer fra Bækkene og i syv Dage være glade for HERREN eders Guds Åsyn.
\par 41 I skal fejre den som en Højtid for HERREN syv Dage om Året; det skal være eder en evig gyldig Anordning fra Slægt til Slægt; i den syvende Måned skal I fejre den.
\par 42 I skal bo i Løvhytter i syv Dage, alle indfødte i Israel skal bo i Løvhytter,
\par 43 for at eders Efterkommere kan vide, at jeg lod Israeliterne bo i Løvhytter, da jeg førte dem ud af Ægypten. Jeg er HERREN eders Gud!
\par 44 Og Moses kundgjorde Israeliterne HERRENs Festtider.

\chapter{24}

\par 1 HERREN talede fremdeles til Moses og sagde:
\par 2 Byd Israeliterne at skaffe dig ren Olivenolie af knuste Frugter til Lysestagen, så Lamperne daglig kan sættes på.
\par 3 I Åbenbaringsteltet uden for Forhænget foran Vidnesbyrdet skal Aron gøre den i Stand, så den bestandig kan brænde fra Aften til Morgen for HERRENs Åsyn. Det skal være eder en evig gyldig Anordning fra Slægt til Slægt;
\par 4 på Guldlysestagen skal han holde Lamperne i Orden for HERRENs Åsyn, så de kan brænde bestandig.
\par 5 Du skal tage fint Hvedemel og bage tolv Kager, to Tiendedele Efa til hver Kage,
\par 6 og lægge dem i to Rækker, seks i hver Række, på Guldbordet for HERRENs Åsyn;
\par 7 på hver Række skal du lægge ren Røgelse, og den skal være Brødenes Offerdel, et Ildoffer for HERREN.
\par 8 Han skal bestandig hver Sabbatsdag lægge dem frem for HERRENs Åsyn; det skal være Israeliterne en evig Pagtspligt.
\par 9 De skal tilfalde Aron og hans Sønner, som skal spise dem på et helligt Sted, thi de er højhellige; ham tilfalder de som en evig, retmæssig Del af HERRENs Ildofre.
\par 10 En israelitisk Kvindes Søn, hvis Fader var Ægypter, gik ud blandt Israeliterne. Da opstod der Strid i Lejren mellem den israelitiske Kvindes Søn og en Israelit,
\par 11 og den israelitiske Kvindes Søn forbandede Navnet og bespottede det. Da førte man ham til Moses. Hans Moder hed Sjelomit, en Datter af Dibri af Dans Stamme.
\par 12 Og de satte ham i Varetægt for at få en Kendelse af HERRENs Mund.
\par 13 Og HERREN talede til Moses og sagde:
\par 14 Før Spotteren uden for Lejren. og alle de, der hørte det, skal lægge deres Hænder på hans Hoved, og derefter skal hele Menigheden stene ham.
\par 15 Og du skal tale til Israeliterne og sige: Når nogen bespotter sin Gud, skal han undgælde for sin Synd;
\par 16 og den, der forbander HERRENs Navn, skal lide Døden; hele Menigheden skal stene ham; en fremmed såvel som en indfødt skal lide Døden når han forbander Navnet.
\par 17 Når nogen slår et Menneske ihjel, skal han lide Døden.
\par 18 Den, der slår et Stykke Kvæg ihjel, skal erstatte det et levende Dyr for et levende Dyr.
\par 19 Når nogen tilføjer sin Næste Legemsskade, skal der handles med ham, som han har handlet,
\par 20 Brud for Brud, Øje for Øje, Tand for Tand; samme Skade, han tilføjer en anden, skal tilføjes ham selv.
\par 21 Den, der slår et Stykke Kvæg ihjel, skal erstatte det; men den, der slår et Menneske ihjel, skal lide Døden.
\par 22 En og samme Ret skal gælde for eder; for den fremmede såvel som for den indfødte; thi jeg er HERREN eders Gud!
\par 23 Og Moses sagde det til Israeliterne, og de førte Spotteren uden for Lejren og stenede ham; Israeliterne gjorde som HERREN havde pålagt Moses.

\chapter{25}

\par 1 Og HERREN talede til Moses på Sinaj Bjerg og sagde:
\par 2 Tal til Israeliterne og sig til dem: Når I kommer til det Land, jeg vil give eder, skal Landet holde Sabbatshvile for.
\par 3 Seks År skal du beså din Mark, og seks År skal du beskære din Vingård og indsamle Landets Afgrøde;
\par 4 men i det syvende År skal Landet have en fuldkommen Sabbatshvile, en Sabbat for HERREN; din Mark må du ikke beså, og din Vingård må du ikke beskære.
\par 5 Det selvgroede efter din Høst må du ikke høste, og Druerne på de ubeskårne Vinstokke må du ikke plukke; det skal være et Sabbatsår for Landet.
\par 6 Hvad der gror af sig selv, medens Landet holder Sabbat, skal tjene eder til Føde, dig selv, din Træl og din Trælkvinde, din Daglejer og den indvandrede hos dig, dem, der bor som fremmede hos dig;
\par 7 dit Kvæg og de vilde Dyr i dit Land skal hele Afgrøden tjene til Føde.
\par 8 Og du skal tælle dig syv Årsabbater frem, syv Gange syv År, så de syv Årsabbater udgør et Tidsrum af ni og fyrretyve År.
\par 9 Så skal du lade Alarmhornet lyde rundt om på den tiende Dag i den syvende Måned; på Forsoningsdagen skal I lade Hornet lyde rundt om i hele eders Land.
\par 10 Og I skal hellige det halvtredsindstyvende År og udråbe Frigivelse i Landet for alle Indbyggerne; et Jubelår skal det være eder; enhver af eder skal vende tilbage til sin Ejendom, og enhver af eder skal vende tilbage til sin Slægt;
\par 11 et Jubelår skal dette År, det halvtredsindstyvende, være eder; I må ikke så og ikke høste, hvad der gror af sig selv i det, eller plukke Druer af de ubeskårne Vinstokke;
\par 12 thi det er et Jubelår, helligt skal det være eder; I skal spise, hvad Marken bærer af sig selv.
\par 13 I Jubelåret skal enhver af eder vende tilbage til sin Ejendom.
\par 14 Når du sælger din Næste noget eller køber noget af ham, må I ikke forurette hinanden.
\par 15 Når du køber noget af din Næste, skal der regnes med Årene siden sidste Jubelår; når han sælger dig noget, skal der regnes med Afgrøderne indtil næste Jubelår.
\par 16 Jo flere År der er tilbage, des højere må du sætte Prisen, jo færre, des lavere, thi hvad han sælger dig, er et vist Antal Afgrøder.
\par 17 Ingen af eder må forurette sin Næste; du skal frygte din Gud, thi jeg er HERREN din Gud!
\par 18 I skal gøre efter mine Anordninger og holde mine Lovbud, så I gør efter dem; så skal I bo trygt i Landet,
\par 19 og Landet skal give sin Frugt, så I kan spise eder mætte og bo trygt deri.
\par 20 Men når I siger: "Hvad skal vi da spise i det syvende År, når vi hverken sår eller indsamler vor Afgrøde?"
\par 21 så vil jeg opbyde min Velsignelse til Bedste for eder i det sjette År, så det bærer Afgrøde for de tre År;
\par 22 i det ottende År skal I så, men I skal leve af gammelt Korn af den sidste Afgrøde indtil det niende År; indtil dets Afgrøde kommer, skal I leve af gammelt Korn.
\par 23 Jorden i Landet må I ikke sælge uigenkaldeligt; thi mig tilhører Landet, I er kun fremmede og indvandrede hos mig;
\par 24 overalt i det Land, I får i Eje, skal I sørge for, at Jorden kan indløses.
\par 25 Når din Broder kommer i Trang, så han må sælge noget af sin Ejendom, skal hans nærmeste Slægtning melde sig som Løser for ham og indløse, hvad hans Broder har solgt.
\par 26 Hvis en ingen Løser har, men selv bliver i Stand til at skaffe den fornødne Løsesum,
\par 27 skal han udregne, hvor mange År der er gået siden Salget, og kun tilbagebetale Manden der købte det, for den Tid, der er tilbage, og derpå igen overtage sin Ejendom.
\par 28 Er han derimod ikke i Stand til at skaffe den fornødne Sum til Tilbagebetalingen, så skal det, han har solgt, blive i Køberens Eje til Jubelåret; men i Jubelåret bliver det frit, så han atter kan overtage sin Ejendom.
\par 29 Når en Mand sælger et Beboelseshus i en By med Mure om, gælder hans Indløsningsret kun et fuldt År efter Salget; hans Indløsningsret gælder et År.
\par 30 Hvis derfor Indløsning ikke har fundet Sted, før et fuldt År er omme, går Huset i Byen med Mure om uigenkaldeligt over i Køberens og hans Efterkommeres Eje; det bliver ikke frit i Jubelåret.
\par 31 Derimod henregnes Huse i Landsbyer, der ikke er omgivne af Mure, til Marklandet;for dem gælder Indløsningsretten, og i Jubelåret bliver de fri.
\par 32 Men med Hensyn til Leviternes Byer, Husene i de Byer, der tilhører dem, da gælder der en ubegrænset Indløsningsret for Leviterne;
\par 33 og når en af Leviterne ikke gør sin Indløsningsret gældende, bliver Huset, han solgte, i de Byer, der tilhører dem, frit i Jubelåret, thi Husene i Leviternes Byer er deres Ejendom blandt Israeliterne.
\par 34 Heller ikke må Græsmarkerne, der hører til deres Byer, sælges, thi de tilhører dem som evigt Eje.
\par 35 Når din Broder i dit Nabolag kommer i Trang og ikke kan bjærge Livet, skal du holde ham oppe; som fremmed og indvandret skal han leve hos dig.
\par 36 Du må ikke tage Rente eller Opgæld af ham, men du skal frygte din Gud og lade din Broder leve hos dig;
\par 37 du må ikke låne ham Penge mod Renter eller give ham af din Føde mod Opgæld.
\par 38 Jeg er HERREN eders Gud, som førte eder ud af Ægypten for at give eder Kana'ans Land, for at være eders Gud.
\par 39 Når din Broder i dit Nabolag kommer i Trang og han må sælge sig selv til dig, må du ikke lade ham arbejde som Træl,
\par 40 men han skal være hos dig som Daglejer eller indvandret; han skal arbejde hos dig til Jubelåret.
\par 41 Da skal han gives fri sammen med sine Børn og vende tilbage til sin Slægt og sine Fædres Ejendom,
\par 42 thi mine Trælle er de, som jeg førte ud af Ægypten; de må ikke sælges, som man sælger Trælle.
\par 43 Du må ikke bruge din Magt over ham med Hårdhed; du skal flygte din Gud.
\par 44 Men har du Brug for Trælle og Trælkvinder, skal du købe dem af de Folkeslag, der bor rundt om eder;
\par 45 også af Børnene efter de indvandrede, der bor som fremmede hos eder, må I købe og af deres Familier, som er hos eder, og som de har avlet i eders Land; de må blive eders Ejendom,
\par 46 og dem må I lade gå i Arv og Eje til eders Børn efter eder; dem må I bruge som Trælle på Livstid; men over Israeliterne, eders Brødre, må du ikke bruge din Magt med Hårdhed, Broder over Broder.
\par 47 Når en fremmed eller en indvandret hos dig kommer til Velstand, og en af dine Brødre i hans Nabolag kommer i Trang, og han må sælge sig til den fremmede eller den indvandrede hos dig eller til en Efterkommer af en fremmeds Slægt,
\par 48 så gælder Indløsningsretten for ham efter Salget; en af hans Brødre må indløse ham,
\par 49 eller også må hans Farbroder eller Fætter eller en anden kødelig Slægtning af hans Familie indløse ham; han må også indløse sig selv, hvis han får Evne dertil.
\par 50 Da skal han sammen med den, der købte ham, udregne Tiden fra det År, han solgte sig til ham, til Jubelåret, og Købesummen skal svare til det Åremål; hans Arbejdstid hos ham skal regnes som en Daglejers.
\par 51 Er der endnu mange År tilbage, skal han i Løsesum udrede den tilsvarende Del af Købesummen.
\par 52 Og er der kun få År tilbage til Jubelåret, skal han regne dermed og udrede sin Løsesum i Forhold til de År, han har tilbage.
\par 53 Som en År for År lejet Daglejer skal han være hos ham; du må ikke roligt se på, at Køberen bruger sin Magt over ham med Hårdhed.
\par 54 Men indløses han ikke på en af disse Måder, skal han frigives i Jubelåret, både han selv og hans Børn.
\par 55 Thi mig tilhører Israeliterne som Trælle; mine Trælle er de, thi jeg førte dem ud af Ægypten. Jeg er HERREN eders Gud!

\chapter{26}

\par 1 I må ikke gøre eder Afguder; udskårne Billeder og Stenstøtter må I ikke rejse eder, ej heller må I opstille nogen Sten med Billedværk i eders Land for at tilbede den; thi jeg er HERREN eders Gud!
\par 2 Mine Sabbater skal I holde, og min Helligdom skal I frygte.
\par 3 Hvis I følger mine Anordninger og holder mine Bud og handler efter dem,
\par 4 vil jeg give eder den Regn, I behøver, til sin Tid, Landet skal give sin Afgrøde, og Markens Træer skal give deres Frugt.
\par 5 Tærskning skal hos eder vare til Vinhøst, og Vinhøst skal vare til Såtid. I skal spise eder mætte i eders Brød og bo trygt i eders Land.
\par 6 Jeg vil give Fred i Landet, så I kan lægge eder til Hvile, uden at nogen skræmmer eder op; jeg vil udrydde de vilde Dyr af Landet, og intet Sværd skal hærge eders Land.
\par 7 I skal forfølge eders Fjender, og de skal falde for Sværdet foran eder.
\par 8 Fem af eder skal forfølge hundrede, og hundrede af eder skal forfølge ti Tusinde, og eders, Fjender skal falde for Sværdet foran eder.
\par 9 Jeg vil vende mig til eder, jeg vil gøre eder frugtbare og mangfoldige, og jeg vil stadfæste min Pagt med eder.
\par 10 I skal spise gammelt Korn, til I for det nye Korns Skyld må tømme Laderne for det gamle.
\par 11 Jeg vil opslå min Bolig midt iblandt eder, og min Sjæl skal ikke væmmes ved eder.
\par 12 Jeg vil vandre iblandt eder og være eders Gud, og I skal være mit Folk.
\par 13 Jeg er HERREN eders Gud, som førte eder ud af Ægypten, for at I ikke mere skulde være deres Trælle; jeg sønderbrød eders Ågstænger og lod eder vandre med rank Nakke.
\par 14 Men hvis I ikke adlyder mig og handler efter alle disse Bud,
\par 15 hvis I lader hånt om mine Anordninger og væmmes ved mine Lovbud, så I ikke handler efter alle mine Bud, men bryder min Pagt,
\par 16 så vil også jeg gøre lige for lige imod eder og hjemsøge eder med skrækkelige Ulykker: Svindsot og Feberglød, så Øjnene sløves og Sjælen vansmægter. Til ingen Nytte sår I eders Sæd, thi eders Fjender skal fortære den.
\par 17 Jeg vender mit Åsyn imod eder, så I bliver slået på Flugt for eders Fjender; eders Avindsmænd skal underkue eder, og I skal flygte, selv om ingen forfølger eder.
\par 18 Og hvis I alligevel ikke adlyder mig, så vil jeg tugte eder endnu mere, ja syvfold, for eders Synder.
\par 19 Jeg vil bryde eders hovmodige Trods, jeg vil gøre eders Himmel som Jern og eders Jord som Kobber.
\par 20 Til ingen Nytte skal I slide eders Kræfter op, thi eders Jord skal ikke give sin Afgrøde, og Landets Træer skal ikke give deres Frugt.
\par 21 Og hvis I alligevel handler genstridigt imod mig og ikke adlyder mig, så vil jeg slå eder endnu mere, ja syvfold, for eders Synder.
\par 22 Jeg vil sende Markens vilde Dyr imod eder, for at de skal røve eders Børn fra eder, udrydde eders Kvæg og mindske eders Tal, så eders Veje bliver øde.
\par 23 Og hvis I alligevel ikke tager mod min Tugt, men handler genstridigt imod mig,
\par 24 så vil også jeg handle genstridigt imod eder og slå eder syvfold for eders Synder.
\par 25 Jeg vil bringe et Hævnens Sværd over eder til Hævn for den brudte Pagt; og søger I Tilflugt i eders Byer, vil jeg sende Pest iblandt eder, så I må overgive eder i Fjendens Hånd.
\par 26 Når jeg bryder Brødets Støttestav for eder, skal ti Kvinder bage eders Brød i een Bagerovn og give eder Brødet tilbage efter Vægt, så I ikke han spise eder mætte.
\par 27 Og hvis I alligevel ikke adlyder mig, men handler genstridigt mod mig,
\par 28 så vil også jeg i Vrede handle genstridigt mod eder og tugte eder syvfold for eders Synder.
\par 29 I skal fortære eders Sønners Kød, og eders Døtres Kød skal I fortære.
\par 30 Jeg vil lægge eders Offerhøje øde og tilintetgøre eders Solsøjler; jeg vil dynge eders Lig oven på Ligene af eders Afgudsbilleder, og min Sjæl skal væmmes ved eder.
\par 31 Jeg vil lægge eders Byer i Ruiner og ødelægge eders Helligdomme og ikke indånde eders liflige Offerduft.
\par 32 Jeg vil lægge eders Land øde, så eders Fjender, der bor deri, skal blive målløse derover;
\par 33 og eder selv vil jeg sprede blandt Folkeslagene, og jeg vil gå bag efter eder med draget Sværd. Eders Land skal blive en Ørken, og eders Byer skal lægges i Ruiner.
\par 34 Da skal Landet, medens det ligger øde, og I er i eders Fjenders Land, få sine Sabbater godtgjort, da skal Landet hvile og få sine Sabbater godtgjort;
\par 35 medens det ligger øde, skal det få den Hvile, det ikke fik på eders Sabbater, dengang I boede deri.
\par 36 Men dem, der bliver tilbage af eder, over deres Hjerter bringer jeg Modløshed i deres Fjenders Lande, så at Lyden af et raslende Blad kan drive dem på Flugt, så de flygter, som man flygter for Sværdet, og falder, skønt ingen forfølger dem;
\par 37 de skal falde over hverandre, som om Sværdet var efter dem, skønt ingen forfølger dem; og I skal ikke holde Stand over for eders Fjender.
\par 38 I skal gå til Grunde blandt Folkeslagene, eders Fjenders Land skal fortære eder.
\par 39 De, der bliver tilbage af eder, skal sygne hen for deres Misgernings Skyld i eders Fjenders Lande, også for deres Fædres Misgerninger skal de sygne hen ligesom de.
\par 40 Da skal de bekende deres Misgerning og deres Fædres Misgerning, den Troløshed, de begik imod mig. Også skal de bekende, at fordi de handlede genstridigt mod mig,
\par 41 måtte også jeg handle genstridigt mod dem og føre dem bort til deres Fjenders Land; ja, da skal deres uomskårne Hjerter ydmyges, og de skal undgælde for deres Skyld.
\par 42 Da vil jeg komme min Pagt med Jakob i Hu, også min Pagt med Isak, også min Pagt med Abraham vil jeg komme i Hu, og Landet vil jeg komme i Hu.
\par 43 Men først må Landet forlades af dem og have sine Sabbater godtgjort, medens det ligger øde og forladt af dem, og de skal undgælde for deres Skyld, fordi, ja, fordi de lod hånt om mine Lovbud og væmmedes ved mine Anordninger.
\par 44 Men selv da, når de er i deres Fjenders Land, vil jeg ikke lade hånt om dem og ikke væmmes ved dem til deres fuldkomne Undergang, så jeg skulde bryde min Pagt med dem; thi jeg er HERREN deres Gud!
\par 45 Jeg vil til deres Bedste ihukomme Pagten med Fædrene, som jeg førte ud af Ægypten for Folkeslagenes Øjne for at være deres Gud.
\par 46 Det er de Anordninger, Lovbud og Love, HERREN fastsatte mellem sig og Israeliterne på Sinaj Bjerg ved Moses.

\chapter{27}

\par 1 HERREN talede fremdeles til Moses og sagde:
\par 2 Tal til Israeliterne og sig til dem: Når nogen vil indfri et Løfte til HERREN med et Pengebeløb, et Løfte, der gælder Mennesker,
\par 3 så skal Vurderingssummen for Mænd fra det tyvende til det tresindstyvende År være halvtredsindstyve Sekel Sølv efter hellig Vægt;
\par 4 men for en Kvinde skal Vurderingssummen være tredive Sekel.
\par 5 Fra det femte til det tyvende År skal Vurderingssummen for Mandspersoner være tyve Sekel, for Kvinder ti.
\par 6 Fra den første Måned til det femte År skal Vurderingssummen for et Drengebarn være fem Sekel Sølv, for et Pigebarn tre.
\par 7 Fra det tresindstyvende År og opefter skal Vurderingssummen være femten Sekel, hvis det er en Mand, men ti, hvis det er en Kvinde.
\par 8 Men hvis Vedkommende er for fattig til at udrede Vurderingssummen, skal man stille ham frem for Præsten, og Præsten skal foretage en Vurdering af ham; Præsten skal foretage Vurderingen således, at han tager Hensyn til, hvad den, der har aflagt Løftet evner.
\par 9 Hvis det drejer sig om Kvæg. hvoraf man kan bringe HERREN Offergave, så skal alt, hvad man giver HERREN, være helligt;
\par 10 man må ikke erstatte eller ombytte det, hverken et bedre med et ringere eller et ringere med et bedre; men hvis man dog ombytter et Dyr med et andet, da skal ikke blot det, men også det, som det ombyttes med, være helligt.
\par 11 Men er det et urent Dyr, af den Slags man ikke kan bringe HERREN som Offergave, skal man fremstille Dyret for Præsten,
\par 12 og Præsten skal vurdere det, alt efter som det er godt eller dårligt; den Vurderingssum, Præsten fastsætter, skal gælde.
\par 13 Men vil man selv indløse det, skal man foruden Vurderingssummen yderligere udrede en Femtedel.
\par 14 Når nogen helliger HERREN sit Hus som Helliggave, skal Præsten vurdere det, alt efter som det er godt eller dårligt, og det skal da stå til den Værdi, Præsten fastsætter.
\par 15 Men vil den, der har helliget Huset, selv indløse det, skal han betale Vurderingssummen med Tillæg af en Femtedel; så skal det være hans.
\par 16 Hvis nogen helliger HERREN noget af sin Arvejord, skal Vurderingssummen rette sig efter Udsæden: en Udsæd på en Homer Byg skal regnes til halvtredsindstyve Sølvsekel.
\par 17 Helliger han sin Jord fra Jubelåret af, skal den stå til den fulde Vurderingssum;
\par 18 helliger han den derimod i Tiden efter Jubelåret, skal Præsten beregne ham Summen i Forhold til de År, der er tilbage til næste Jubelår, så der sker Fradrag i Vurderingssummen.
\par 19 Vil da han, der har helliget Jorden, indløse den, skal han udrede Vurderingssummen med Tillæg af en Femtedel; så går den over i hans Eje.
\par 20 Men hvis han ikke indløser Jorden og alligevel sælger den til en anden, kan den ikke mereindløses;
\par 21 da skal den, når den i Jubelåret bliver fri, være HERREN helliget på samme Måde som en Mark, der er lagt Band på, og tilfalde Præsten som Ejendom.
\par 22 Hvis nogen helliger HERREN Jord, han har købt, og som ikke hører til hans Arvejord,
\par 23 skal Præsten udregne ham Vurderingssummen til næste Jubelår, og han skal da straks erlægge Vurderingssummen som Helliggave til HERREN;
\par 24 i Jubelåret går Jorden så tilbage til den Mand, han købte den af, hvis Arvejord den var.
\par 25 Enhver Vurdering skal ske efter hellig Vægt, tyve Gera på en Sekel.
\par 26 Ingen må hellige HERREN noget af det førstefødte af Kvæget, da det som førstefødt allerede tilhører ham. Hvad enten det er et Stykke Hornkvæg eller Småkvæg, tilhører det HERREN.
\par 27 Hører det derimod til de urene Dyr, kan man løskøbe det efter Vurderingssummen med Tillæg af en Femtedel; indløses det ikke, skal det sælges for Vurderingssummen.
\par 28 Intet, der er lagt Band på, intet af, hvad nogen af sin Ejendom helliger HERREN ved at lægge Band derpå, være sig Mennesker, Kvæg eller Arvejord, må sælges eller indløses; alt, hvad der er lagt Band på, er højhelligt, det tilhører HERREN.
\par 29 Intet Menneske, der er lagt Band på, må løskøbes, det skal lide Døden.
\par 30 Al Tiende af Landet, både af Landets Sæd og Træernes Frugt, tilhører HERREN, det er helliget HERREN.
\par 31 Hvis nogen vil indløse noget af sin Tiende, skal han yderligere udrede en Femtedel.
\par 32 Hvad angår al Tiende af Hornkvæg og Småkvæg, alt hvad der går under Staven, da skal hvert tiende dyr være helliget HERREN.
\par 33 Der må ikke skelnes imellem gode og dårlige Dyr, og ingen Ombytning må finde Sted; hvis nogen ombytter et Dyr, skal ikke blot det, men også det, som det ombyttes med, være helligt; det må ikke indløses.
\par 34 Det er de Bud, HERREN gav Moses til Israeliterne på Sinaj Bjerg.


\end{document}