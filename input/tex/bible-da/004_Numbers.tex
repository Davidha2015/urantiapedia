\begin{document}

\title{Numbers}


\chapter{1}

\par 1 HERREN talede således til Moses i Sinaj Ørken i Åbenbaringsteltet på den første Dag i den anden Måned af det andet år efter deres Udvandring fra Ægypten:
\par 2 Optag det samlede Tal på hele Israelitternes Menighed efter deres Slægter, efter deres Fædrenehuse, ved at tælle Navnene på alle af Mandkøn, Hoved for Hoved;
\par 3 fra Tyveårsalderen og opefter skal du og Aron mønstre alle våbenføre Mænd i Israel, Hærafdeling for Hærafdeling;
\par 4 en Mand af hver Stamme skal hjælpe eder dermed, Overhovedet for Stammens Fædrenehuse.
\par 5 Navnene på de Mænd, der skal stå eder bi, er følgende: Af Ruben Elizur, Sjedeurs Søn;
\par 6 af Simeon Sjelumiel, Zurisjaddajs Søn;
\par 7 af Juda Nahasjon, Amminadabs Søn;
\par 8 af Issakar Netanel, Zuars Søn;
\par 9 af Zebulon Eliab, Helons Søn;
\par 10 af Josefs Sønner: Af Efraim Elisjama, Ammihuds Søn, af Manasse Gamliel, Pedazurs Søn;
\par 11 af Benjamin Abidan, Gidonis Søn;
\par 12 af Dan Ahiezer, Ammisjaddajs Søn;
\par 13 af Aser Pagiel, Okrans Søn;
\par 14 af Gad Eljasaf, Deuels Søn,
\par 15 og af Naftali Ahira, Enans Søn.
\par 16 Det var de Mænd, der udtoges af Menigheden, Øversterne for deres Fædrenestammer, Overhovederne for Israels Stammer.
\par 17 Da tog Moses, og Aron disse Mænd, hvis Navne var nævnet,
\par 18 og de kaldte hele Menigheden sammen på den første Dag i den anden Måned. Så lod de sig indføre i Familielisterne efter deres Slægter, efter deres Fædrenehuse, ved Optælling af Navnene fra Tyveårsalderen og opefter, Hoved for Hoved,
\par 19 som HERREN havde pålagt Moses. Således mønstrede han dem i Sinaj Ørken.
\par 20 Rubens, Israels førstefødtes, Sønner, deres Efterkommere efter deres Slægter, efter deres Fædrenehuse, ved Optælling af Navnene Hoved for Hoved, alle af Mandkøn fra Tyveårsalderen og opefter, alle våbenføre Mænd,
\par 21 de, som mønstredes af Rubens Stamme, udgjorde 46500.
\par 22 Simeons Sønner, deres Efterkommere efter deres Slægter, efter deres Fædrenehuse, så mange af dem, som mønstredes ved Optælling af Navnene Hoved for Hoved, alle af Mandkøn, fra Tyveårsalderen og opefter, alle våbenføre Mænd,
\par 23 de, som mønstredes af Simeons Stamme, udgjorde 59 300.
\par 24 Gads Sønner, deres Efterkommere efter deres Slægter, efter deres Fædrenehuse, ved Optælling af Navnene fra Tyveårsalderen og opefter, alle våbenføre Mænd,
\par 25 de, som mønstredes af Gads Stamme, udgjorde 45650.
\par 26 Judas Sønner, deres Efterkommere efter deres Slægter, efter deres Fædrenehuse, ved Optælling af Navnene fra Tyveårsalderen og opefter, alle våbenføre Mænd,
\par 27 de, som mønstredes af Judas Stamme, udgjorde 74600.
\par 28 Issakars Sønner, deres Efterkommere efter deres Slægter, efter deres Fædrenehuse, ved Optælling af Navnene fra Tyveårsalderen og opefter, alle våbenføre Mænd,
\par 29 de, som mønstredes af Issakars Stamme, udgjorde 54 400.
\par 30 Zebulons Sønner, deres Efterkommere efter deres Slægter, efter deres Fædrenehuse, ved Optælling af Navnene fra Tyveårsalderen og opefter, alle våbenføre Mænd,
\par 31 de, som mønstredes af Zebulons Stamme, udgjorde 57 4OO.
\par 32 Josefs Sønner: Efraims Sønner, deres Efterkommere efter deres Slægter, efter deres Fædrenehuse, ved Optælling af Navnene fra Tyveårsalderen og opefter, alle våbenføre Mænd,
\par 33 de, som mønstredes af Efraims Stamme, udgjorde 40500;
\par 34 Manasses Sønner, deres Efterkommere efter deres Slægter, efter deres Fædrenehuse, ved Optælling af Navnene fra Tyveårsalderen og opefter, alle våbenføre Mænd,
\par 35 de, som mønstredes af Manasses Stamme, udgjorde 32200.
\par 36 Benjamins Sønner, deres Efterkommere efter deres Slægter, efter deres Fædrenehuse, ved Optælling af Navnene fra Tyveårsalderen og opefter, alle våbenføre Mænd.
\par 37 de, som mønstredes at Benjamins Stamme, udgjorde 35400.
\par 38 Dans Sønner, deres Efterkommere efter deres Slægter, efter deres Fædrenehuse, ved Optælling af Navnene fra Tyveårsalderen og opefter, alle våbenføre Mænd,
\par 39 de, som mønstredes af Dans Stamme, udgjorde 62 700.
\par 40 Asers Sønner, deres Efterkommere efter deres Slægter, efter deres Fædrenehuse, ved Optælling af Navnene fra Tyveårsalderen og opefter, alle våbenføre Mænd,
\par 41 de, som mønstredes af Asers Stamme, udgjorde 41 500.
\par 42 Naftalis Sønner, deres Efterkommere efter deres Slægter, efter deres Fædrenehuse, ved Optælling af Navnene fra Tyveårsalderen og opefter, alle våbenføre Mænd,
\par 43 de, som mønstredes af Naftalis Stamme, udgjorde 53 400.
\par 44 Det var dem, som mønstredes, dem, Moses og Aron og Israels tolv Øverster, en for hvert Fædrenehus, mønstrede.
\par 45 Og alle, som mønstredes af Israeliterne efter deres Fædrenehuse fra Tyveårsalderen og opefter, alle våbenføre Mænd i Israel,
\par 46 alle, som mønstredes, udgjorde 603 550.
\par 47 Men Leviterne efter deres Fædrenestamme mønstredes ikke sammen med dem.
\par 48 HERREN talede til Moses og sagde:
\par 49 Kun Levis Stamme må du ikke mønstre, og dens samlede Tal må du ikke optage sammen med de andre Israeliters.
\par 50 Du skal overdrage Leviterne Tilsynet med Vidnesbyrdets Bolig, alle dens Redskaber og alt dens Tilbehør; de skal bære Boligen og alle dens Redskaber, de skal betjene den, og rundt om Boligen skal de have deres Lejr.
\par 51 Når Boligen skal bryde op, skal Leviterne tage den ned, og når Boligen skal gå i Lejr, skal Leviterne rejse den. Enhver Lægmand, der kommer den nær, skal lide Døden.
\par 52 Israeliterne skal lejre sig hver i sin Lejrafdeling og under sit Felttegn, Hærafdeling for Hærafdeling,
\par 53 men Leviterne skal lejre sig rundt om Vidnesbyrdets Bolig, for at der ikke skal komme Vrede over Israelitternes Menighed; og Leviterne skal tage Vare på, hvad der er at varetage ved Vidnesbyrdets Bolig.
\par 54 Og Israeliterne gjorde ganske, hvad HERREN havde pålagt Moses.

\chapter{2}

\par 1 HERREN talede til Moses og, Aron og sagde:
\par 2 Israeslitene skal lejre sig hver under sit Felttegn, under sit Fædrene hus's Mærke; i en Kreds om Åbenbaringsteltet skal de lejre sig.
\par 3 På Forsiden mod Øst skal Juda lejre sig under sin Lejrs Felttegn, Hærafdeling for Hærafdeling, med Nahasjon, Amminadabs Søn, som Øverste over Judæerne;
\par 4 de mønstrede, som udgør hans Hærafdeling, løber op til 74600 Mand.
\par 5 Ved Siden af ham skal Issakars Stamme lejre sig med Netanel, Zuars Søn, som Øverste over Issakariterne;
\par 6 de mønstrede, som udgør hans Hærafdeling, løber op til 54 40O Mand.
\par 7 Dernæst Zebulons Stamme med Eliab, Helons Søn, som Øverste over Zebuloniterne;
\par 8 de mønstrede, som udgør hans Hærafdeling, løber op til 57 400 Mand.
\par 9 De mønstrede i Judas Lejr udgør i alt 186 400 Mand, Hærafdeling for Hærafdeling. De skal bryde op først.
\par 10 Ruben skal lejre sig under sin Lejrs Felttegn mod Syd, Hærafdeling for Hærafdeling, med Elizur, Sjedeurs Søn, som Øverste over Rubeniterne;
\par 11 de mønstrede, som udgør hans Hærafdeling, løber op til 46 500 Mand.
\par 12 Ved Siden af ham skal Simeons Stamme lejre sig med Sjelumiel, Zurisjaddajs Søn, som Øverste over Simeoniterne;
\par 13 de mønstrede, som udgør hans Hærafdeling, løber op til 59300 Mand.
\par 14 Dernæst Gads Stamme med Eljasaf, Reuels Søn, som Øverste over Gadiferne;
\par 15 de mønstrede, som udgør hans Hærafdeling, løber op til 45 650 Mand.
\par 16 De mønstrede i Rubens Lejr udgør i alt 151 450 Mand, Hærafdeling for Hærafdeling. De skal bryde op i anden Række.
\par 17 Derpå skal Åbenbaringsteltet, Leviternes Lejr, bryde op midt imellem de andre Lejre; i den Rækkefølge, de lejrer sig, skal de bryde op, hver på sin Plads, Felttegn for Felttegn.
\par 18 Efraim skal lejre sig under sin Lejrs Felttegn mod Vest med Elisjama, Ammihuds Søn, som Øverste over Efraimiterne;
\par 19 de mønstrede, som udgør hans Hærafdeling, løber op til 4O 5OO Mand.
\par 20 Ved Siden af ham skal Manasses Stamme lejre sig med Gamliel, Pedazurs Søn, som Øverste over Manassiterne;
\par 21 de mønstrede, som udgør hans Hærafdeling, løber op til 32 200 Mand.
\par 22 Dernæst Benjamins Stamme med Abidan, Gidonis Søn, som Øverste over Benjaminiterne;
\par 23 de mønstrede, som udgør hans Hærafdeling, løber op til 35 400 Mand.
\par 24 De mønstrede i Efraims Lejr udgør i alt 1O8 100 Mand, Hærafdeling for Hærafdeling. De skal bryde op i tredje Række.
\par 25 Dan skal lejre sig under sin Lejrs Felttegn mod Nord, Hærafdeling for Hærafdeling, med Ahiezer, Ammisjaddajs Søn, som Øverste over Daniterne;
\par 26 de mønstrede, som udgør hans Hærafdeling, løber op til 62 700 Mand.
\par 27 Ved Siden af ham skal Asers Stamme lejre sig med Pagiel, Okrans Søn, som Øverste over Aseriterne;
\par 28 de mønstrede, som udgør hans Hærafdeling, løber op til 4l 5OO Mand.
\par 29 Dernæst Naftalis Stamme med Ahira, Enans Søn, som Øverste over Naftaliterne;
\par 30 de mønstrede, som udgør hans Hærafdeling, løber op fil 53 4OO Mand.
\par 31 De mønstrede i Dans Lejr udgør i alt 157600 Mand. De skal bryde op sidst, Felttegn for Felttegn.
\par 32 Det var de mønstrede af Israeliterne efter deres Fædrenehuse, alle de mønstrede i Lejrene, Hærafdeling for Hærafdeling, 603 550 Mand.
\par 33 Men Leviterne mønstredes ikke sammen med de andre Israelitter, således som HERREN havde pålagt Moses.
\par 34 Og ganske som HERREN havde pålagt Moses, slog Israeliterne Lejr, Felttegn for Felttegn, og i den Rækkefølge brød de op, enhver med sine Slægter, med sit Fædrenehus.

\chapter{3}

\par 1 Følgende var Arons og Moses's Efterkommere, på den Tid HERREN talede på Sinaj Bjerg.
\par 2 Navnene på Arons Sønner var følgende: Nadab, den førstefødte, Abihu, Eleazar og Itamar;
\par 3 det var Navnene på Arons Sønner, de salvede Præster, som indsattes til Præstetjeneste.
\par 4 Men Nadab og Abihu døde for HERRENs Åsyn, da de frembar fremmed Ild for HERRENs Åsyn i Sinaj Ørken, og de havde ingen Sønner. Således kom Eleazar og Itamar til at gøre Præstetjenest: for deres Fader Arons Åsyn.
\par 5 HERREN talede til Moses og sagde:
\par 6 Lad Levis Stamme træde frem og stil dem frem for Præsten Aron, for at de kan gå ham til Hånde.
\par 7 De skal tage Vare på, hvad han og hele Menigheden har at varetage foran Åbenbaringsteltet, og således udføre Arbejdet ved Boligen,
\par 8 og de skal tage Vare på alle Åbenbaringsteltets Redskaber og på, hvad Israeliterne har at varetage, og således udføre Arbejdet ved Boligen.
\par 9 Altså skal du overgive Aron og hans Sønner Leviterne; de er ham overgivet som Gave fra Israeliterne.
\par 10 Men Aron og hans Sønner skal du sætte til at tage Vare på deres Præstetjeneste; enhver Lægmand, som trænger sig ind deri, skal lide Døden.
\par 11 HERREN talede til Moses og sagde:
\par 12 Se, jeg har selv udtaget Leviterne af Israelitternes Midte i Stedet for alt det førstefødte, der åbner Moders Liv hos Israeliterne, og Leviterne er blevet min Ejendom;
\par 13 thi mig tilhører alt det førstefødte. Dengang jeg dræbte alt det førstefødte i Ægypten, helligede jeg mig alt det førstefødte i Israel, både af Mennesker og Dyr; mig HERREN skal de tilhøre.
\par 14 HERREN talede til Moses i Sinaj Ørken og sagde:
\par 15 Du skal mønstre Levis Sønner efter deres Fædrenehuse, efter deres Slægter; alle af Mandkøn fra en Måned og opefter skal du mønstre.
\par 16 Da mønstrede Moses dem på HERRENs Bud, som der var ham pålagt.
\par 17 Følgende var Levis Sønner efter deres Navne: Gerson, Kehat og Merari.
\par 18 Følgende var Navnene på Gersons Sønner efter deres Slægter: Libni og Sjimi;
\par 19 Hehats Sønner efter deres Slægter var: Amram, Jizhar, Hebron og Uzziel;
\par 20 Meraris Sønner efter deres Slægter var: Mali og Musji. Det var Leviternes Slægter efter deres Fædrenehuse.
\par 21 Fra Gerson nedstammede Libniternes og Sjimiternes Slægter; det var Gersoniternes Slægter.
\par 22 De, som mønstredes af dem, da alle af Mandkøn fra en Måned og opefter blev optalt, de, som mønstredes af dem, udgjorde 7500.
\par 23 Gersoniternes Slægter havde deres Lejrplads bag ved Boligen mod Vest.
\par 24 Øverste for Gersoniternes Fædrenehus var Eljasar, Laels Søn.
\par 25 Gersoniterne havde ved Åbenbaringsteltet at tage Vare på selve Boligen og Teltdækket, dets Dække, Forhænget for Åbenbaringsteltets Indgang,
\par 26 Forgårdens Omhæng, Forhænget for Indgangen til Forgården, der omgav Boligen og Alteret, og dens Teltreb, alt Arbejdet dermed.
\par 27 Fra Kehat nedstammede Amramiternes, Jizhariternes, Hebroniternes og Uzzieliternes Slægter; det var Kehatiternes Slægter.
\par 28 De, der mønstredes af dem, da alle af Mandkøn fra en Måned og opefter blev optalt, udgjorde 8600, som tog Vare på, hvad der var at varetage ved Helligdommen.
\par 29 Kehatiternes Slægter havde deres Lejrplads ved Boligens Sydside.
\par 30 Øverste for Kehatiternes Slægters Fædrenehus var Elizafan, Uzziels Søn.
\par 31 De havde at tage Vare på Arken, Bordet, Lysestagen, Altrene, Helligdommens Redskaber, som brugtes ved Tjenesten, og Forhænget med dertil hørende Arbejde.
\par 32 Øverste over Leviternes Øverster var Eleazar, Præsten Arons Søn, som havde Tilsyn med dem, der tog Vare på, hvad der var at varetage ved Helligdommen.
\par 33 Fra Merari nedstammede Maliternes og Musjiternes Slægter; det var Meraris Slægter.
\par 34 De, som mønstredes af dem, da alle af Mandkøn fra en Måned og opefter blev optalt, udgjorde 6200.
\par 35 Øverste for Meraris Slægters Fædrenehus var Zuriel, Abibajils Søn. De havde deres Lejrplads ved Boligens Nordside.
\par 36 Merariterne var sat til at tage Vare på Boligens Brædder, Tværstænger, Piller og Fodstykker, alle dens Redskaber og alt det dertil hørende Arbejde,
\par 37 Pillerne til Forgården, som var rundt om den, med Fodstykker, Pæle og Reb.
\par 38 Foran Boligen, på Åbenbaringsteltets Forside mod Øst, havde Moses, Aron og hans Sønner deres Lejrpladser, og de tog Vare på alt det, Israeliterne havde at varetage ved Boligen. Enhver Lægmand, der trængte sig ind i det, måtte lide Døden.
\par 39 De mønstrede af Leviterne, de, som Moses og Aron mønstrede på HERRENs Bud efter deres Slægter, alle af Mandkøn fra en Måned og opefter, udgjorde i alt 22 OOO.
\par 40 HERREN sagde til Moses: Du skal mønstre alle førstefødte af Mandkøn blandt Israeliterne fra en Måned og opefter og optage Tallet på deres Navne.
\par 41 Så skal du udtage Leviterne til mig HERREN i Stedet for alle Israelitternes førstefødte og ligeledes Leviternes Kvæg i Stedet for alt det førstefødte af Israelitternes Kvæg.
\par 42 Og Moses mønstrede, som HERREN havde pålagt ham, alle Israelitternes førstefødte;
\par 43 og da Navnene på dem fra en Måned og opefter optaltes, udgjorde de førstefødte af Mandkøn, alle de, som mønstredes i alt 22 273.
\par 44 Derpå talede HERREN til Moses og sagde:
\par 45 Tag Leviterne i Stedet for alle Israelitternes førstefødte og ligeledes Leviternes Kvæg i Stedet for deres Kvæg, så at Leviterne kommer til at tilhøre mig HERREN.
\par 46 Men til Udløsning af de 273, hvormed Antallet af Israelitternes førstefødte overstiger Leviternes Antal,
\par 47 skal du tage fem Sekel for hvert Hoved, efter hellig Vægt skal du tage dem, tyve Gera på en Sekel;
\par 48 og Pengene skal du give Aron og hans Sønner som Udløsning for de overskydende.
\par 49 Da tog Moses Løsepengene af de overskydende, dem, der ikke var udløst ved Leviterne;
\par 50 af Israelitternes førstefødte tog han Pengene, 1365 Sekel efter hellig Vægt.
\par 51 Og Moses gav Aron og hans Sønner Løsepengene efter HERRENs Bud, som HERREN havde pålagt Moses.

\chapter{4}

\par 1 HERREN talede til Moses og Aron og sagde:
\par 2 Optag blandt Leviterne Tallet på Kehatiterne efter deres Slægter, efter deres Fædrenehuse,
\par 3 fra Trediveårsalderen og opefter til Halvtredsårsalderen, alle, der skal gøre Tjeneste med' at udføre Arbejde ved Åbenbaringsteltet.
\par 4 Kehatiternes Arbejde ved Åbenbaringsteltet skal være med de højhellige Ting.
\par 5 Når Lejren bryder op, skal Aron og hans Sønner gå ind og tage det indre Forhæng ned og tildække Vidnesbyrdets Ark dermed;
\par 6 ovenover skal de lægge et Dække af Tahasjskind og derover igen brede et ensfarvet violet Purpurklæde; derpå skal de stikke Bærestængerne ind.
\par 7 Og over Skuebrødsbordet skal de brede et violet Purpurklæde og stille Fadene, Kanderne, Skålene og Krukkerne til Drikofferet derpå, og Brødet, som stadig skal ligge fremme, skal ligge derpå;
\par 8 ovenover skal de brede et karmoisinrødt Klæde og dække dette til med et Dække af Tahasjskind; derpå skal de stikke Bærestængerne ind.
\par 9 Så skal de tage et violet Purpurklæde og dermed tildække Lysestagen, dens Lamper, Sakse, Bakker og alle Oliekrukkerne, de Ting, som bruges ved Betjeningen deraf,
\par 10 og de skal lægge den med alt dens Tilbehør i et Dække af Tahasjskind og så lægge det på Bærebøren.
\par 11 Over Guldalteret skal de ligeledes brede et violet Purpurklæde og dække dette til med et Dække af Tahasjskind; derpå skal de stikke Bærestængerne ind.
\par 12 Og de skal tage alle Redskaber, som bruges ved Tjenesten i Helligdommen, og lægge dem i et violet Purpurklæde og dække dem til med et Dække af Tahasjskind og lægge dem på Bærebøren.
\par 13 Fremdeles skal de rense Alteret for Aske og brede et rødt Purpurklæde derover
\par 14 og på det lægge alle Redskaberne, som bruges til Tjenesten derved, Panderne, Gaflerne, Skovlene og Skålene, alle Alterets Redskaber, og derover skal de brede et Dække af Tahasjskind; derpå skal de stikke Bærestængerne ind.
\par 15 Når så ved Lejrens Opbrud Aron og hans Sønner er færdige med at tilhylle de hellige Ting og alle de hellige Redskaber, skal Kehatiterne træde til og bære dem; men de må ikke røre ved de hellige Ting; thi gør de det, skal de dø. Det er, hvad Kehatiterne skal bære af Åbenbaringsteltet.
\par 16 Med Eleazar, Præsten Arons Søn, påhviler Tilsynet med Olien til Lysestagen, den vellugtende Røgelse, det daglige Afgrødeoffer og Salveolien og desuden Tilsynet med hele Boligen og alt, hvad der er deri af hellige Ting og deres Tilbehør.
\par 17 HERREN talede til Moses og Aron og sagde:
\par 18 Sørg for, at Kehatiternes Slægters Stamme ikke udryddes af Leviternes Midte!
\par 19 Således skal I forholde eder med dem, for at de kan blive i Live og undgå Døden, når de nærmer sig de højhellige Ting: Aron og hans Sønner skal træde til og anvise hver enkelt af dem, hvad han skal gøre, og hvad han skal bære,
\par 20 for at de ikke et eneste Øjeblik skal komme til at se de hellige Ting; thi gør de det, skal de dø.
\par 21 HERREN talede til Moses og sagde:
\par 22 Optag også Tallet på Gersoniterne efter deres Fædrenehuse, efter deres Slægter;
\par 23 fra Trediveårsalderen og opefter til Halvtredsårsalderen skal du mønstre dem, alle, der skal gøre Tjeneste med at udføre Arbejdet ved Åbenbaringsteltet.
\par 24 Dette er Gersoniternes Arbejde, hvad de skal gøre, og hvad de skal bære
\par 25 De skal bære Boligens Tæpper, Åbenbaringsteltet med dets Dække og Dækket af Tahasjskind ovenover, Forhænget til Åbenbaringsteltets Indgang,
\par 26 Forgårdens Omhæng og Forhænget for Indgangen til Forgården, der er rundt om Boligen og Alteret, dens Teltreb og alle Redskaber, som hører til Arbejdet derved; og alt, hvad der skal gøres derved, skal de udføre.
\par 27 Efter Arons og hans Sønners Bud skal Gersoniterne udføre deres Arbejde både med det, de skal bære, og med det, de skal gøre; og I skal anvise dem alt, hvad de skal bære, Stykke for Stykke.
\par 28 Det er det Arbejde, Gersoniternes Sønners Slægter skal have ved Åbenbaringsteltet, og de skal varetage det under Itamars, Præsten Arons Søns, Tilsyn.
\par 29 Merariterne skal du mønstre efter deres Slægter, efter deres Fædrenehuse;
\par 30 fra Trediveårsalderen og opefter til Halvtredsårsalderen skal du mønstre dem, alle, som skal gøre Tjeneste med at udføre Arbejde ved Åbenbaringsteltet.
\par 31 Dette er, hvad der påhviler dem at bære, alt, hvad der hører til deres Arbejde ved Åbenbaringsteltet: Boligens Brædder, dens Tværstænger, Piller og Fodstykker,
\par 32 Pillerne til Forgården, som er rundt om den, med Fodstykker, Pæle og Reb, alle tilhørende Redskaber og alt, hvad der hører til Arbejdet derved; Stykke for Stykke skal I anvise dem alle de Ting, det påhviler dem at bære.
\par 33 Det er det Arbejde, der påhviler Merariternes Slægter, alt, hvad der hører til deres Arbejde ved Åbenbaringsteltet, under Itamars, Præsten Arons Søns, Tilsyn.
\par 34 Så mønstrede Moses og Aron og Menighedens Øverster Kehatiternes Sønner efter deres Slægter, efter deres Fædrenehuse,
\par 35 fra Trediveårsalderen og opefter til Halvtredsårsalderen, alle, som skulde gøre Tjeneste med at udføre Arbejde ved Åbenbaringsteltet,
\par 36 og de, der mønstredes af dem efter deres Slægter, udgjorde 2750.
\par 37 Det var dem, som mønstredes af Kehatiternes Slægter, alle dem, der skulde udføre Arbejde ved Åbenbaringsteltet, som Moses og Aron mønstrede efter HERRENs Bud ved Moses.
\par 38 De, der mønstredes af Gersoniterne efter deres Slægter, efter deres Fædrenehuse,
\par 39 fra Trediveårsalderen og opefter til Halvtredsårsalderen, alle, som skulde gøre Tjeneste med at udføre Arbejde ved Åbenbaringsteltet,
\par 40 de, der mønstredes af dem efter deres Slægter, efter deres Fædrenehuse, udgjorde 2630.
\par 41 Det var dem, som mønstredes af Gersoniternes Slægter, alle dem, der skulde udføre Arbejde ved Åbenbaringsteltet, som Moses og Aron mønstrede efter HERRENs Bud.
\par 42 De, der mønstredes af Merariternes Slægter efter deres Slægter, efter deres Fædrenehuse,
\par 43 fra Trediveårsalderen og opefter til Halvtredsårsalderen, alle, som skulde gøre Tjeneste med at udføre Arbejde ved Åbenbaringsteltet,
\par 44 de, der mønstredes af dem efter deres Slægter, udgjorde 3200.
\par 45 Det var dem, som mønstredes af Merariternes Slægter, som Moses og Aron mønstrede efter HERRENs Bud ved Moses.
\par 46 Alle, som mønstredes, som Moses og Aron og Israels Øverster mønstrede af Leviterne efter deres Slægter, efter deres Fædrenehuse,
\par 47 fra Trediveårsalderen og opefter til Halvtreds års alderen, alle, som skulde udføre Arbejde ved Åbenbaringsteltet både med hvad der skulde gøres, og hvad der skulde bæres,
\par 48 de, der mønstredes af dem, udgjorde 8580.
\par 49 Efter HERRENs Bud ved Moses anviste man hver enkelt af dem, hvad han skulde gøre eller bære; det blev dem anvist, som HERREN havde pålagt Moses.

\chapter{5}

\par 1 HERREN talede fremdeles til Moses og sagde:
\par 2 Byd Israeliterne at fjerne alle spedalske fra Lejren, alle, der lider af Flåd, og alle, der er blevet urene ved Lig;
\par 3 både Mænd og Kvinder skal I fjerne og føre uden for Lejren, for at de ikke skal gøre deres Lejr uren, hvor jeg bor midt iblandt dem.
\par 4 Det gjorde Israeliterne så; de førte dem uden for Lejren, således som HERREN havde pålagt Moses.
\par 5 HERREN talede fremdeles til Moses og sagde:
\par 6 Sig til Israeliterne: Når en Mand eller Kvinde begår nogen af alle de Synder, som Mennesker begår, således at han gør sig skyldig i Svig mod HERREN, og det Menneske derved pådrager sig Skyld,
\par 7 så skal de bekende Synden, de har begået, og Gerningsmanden skal erstatte det, han har forbrudt sig med, efter dets fulde Værdi med Tillæg af en Femtedel og give det til den, han har forbrudt sig imod.
\par 8 Og hvis denne ikke har efterladt sig nogen Løser, hvem han kan yde Erstatningen, så skal Erstatningen, som ydes, tilfalde HERREN, det vil sige Præsten, foruden den Soningsvæder, ved hvilken der skaffes ham Soning.
\par 9 Al Offerydelse, alle Helliggaver, som Israeliterne frembærer til Præsten, skal tilfalde ham.
\par 10 Alle Helliggaver skal tilfalde ham; hvad nogen giver Præsten, skal tilfalde ham.
\par 11 HERREN talede fremdeles til Moses og sagde:
\par 12 Tal til Israeliterne og sig til dem: Når en Hustru forser sig imod sin Mand og er ham utro,
\par 13 idet en anden Mand har Samleje med hende, uden at det er kommet til hendes Mands Kundskab, og uden at det er blevet opdaget, skønt hun har besmittet sig, og uden at der er noget Vidne imod hende, da hun ikke er grebet på fersk Gerning,
\par 14 og han gribes af Skinsygens Ånd, så han bliver skinsyg på sin Hustru, som også i Virkeligheden har besmittet sig, eller han gribes af Skinsygens Ånd, så han bliver skinsyg på sin Hustru, skønt hun ikke har besmittet sig,
\par 15 så skal Manden bringe sin Hustru til Præsten og medbringe som Offergave for hende en Tiendedel Efa Bygmel; han må hverken hælde Olie over eller komme Røgelse på, thi det er et Skinsyge Afgrødeoffer, et Minde Afgrødeoffer, der skal minde om Brøde.
\par 16 Så skal Præsten føre hende frem og stille hende for HERRENs Åsyn.
\par 17 Og Præsten skal tage helligt Vand i et Lerkar, og af Støvet på Boligens Gulv skal Præsten tage noget og komme i Vandet.
\par 18 Så skal Præsten stille Kvinden frem for HERRENs Åsyn, løse hendes Hår og lægge Minde Afgrødeofferet i hendes Hænder; det er et Skinsyge Afgrødeoffer; og Præsten skal have den bitre Vandes Forbandelsesvand i Hånden.
\par 19 Derpå skal Præsten besværge Kvinden og sige til hende: "Hvis ingen har haft Samleje med dig, hvis du ikke har forset dig imod din Mand og besmittet dig, så skal dette den bitre Vandes Forbandelsesvand ikke skade dig.
\par 20 Men har du forset dig imod din Mand og besmittet dig, og har en anden end din Mand haft Samleje med dig"
\par 21 Præsten besværger nu kvinden med Forbandelsens Ed og siger til hende "så gøre HERREN dig til en Forbandelse og Besværgelse i dit Folk, idet han lader din Lænd visne og din Bug svulme op;
\par 22 Forbandelsesvandet her komme ind i dine Indvolde, så din Bug svulmer op og din Lænd visner!" Og kvinden skal sige: "Amen, Amen!"
\par 23 Derpå skal Præsten skrive disse Forbandelser op på et Blad og vaske dem ud i den bitre Vandes Vand
\par 24 og give Kvinden den bitre Vandes Forbandelsesvand at drikke, for at Forbandelsesvandet kan komme ind i hende til bitter Vånde.
\par 25 Derefter skal Præsten tage Skinsyge Afgrødeofferet af Kvindens Hånd, udføre Svingningen dermed for HERRENs Åsyn og bære det hen til Alteret.
\par 26 Og Præsten skal tage en Håndfuld af Afgrødeofferet, det, som skal ofres deraf, og bringe det som Røgoffer på Alteret og derpå give Kvinden Vandet at drikke.
\par 27 Når han har givet hende Vandet at drikke, vil Forbandelsesvandet, dersom hun har besmittet sig og været sin Mand utro, blive til bitter Vånde, når det kommer ind i hende, hendes Bug vil svulme op og hendes Lænd visne, og Kvinden bliver en Forbandelse i sit Folk.
\par 28 Men dersom Kvinden ikke har besmittet sig, dersom hun er ren, bliver hun uskadt og kan få Børn.
\par 29 Det er Loven om Skinsyge; når en Hustru forser sig imod sin Mand og besmittes,
\par 30 eller når en Mand gribes af Skinsygens Ånd og bliver skinsyg på sin Hustru, så skal han fremstille Hustruen for HERRENs Åsyn, og Præsten skal handle med hende efter alt i denne Lov;
\par 31 Manden skal være sagesløs, men sådan en Hustru skal undgælde for sin Brøde.

\chapter{6}

\par 1 HERREN talede fremdeles til Moses og sagde:
\par 2 Tal til Israeliterne og sig til dem: Når en Mand eller Kvinde vil aflægge et Nasiræerløfte for således at indvie sig til HERREN,
\par 3 skal han afholde sig fra Vin og stærk Drik; Vineddike og stærk Drik må han ikke drikke, ej heller nogen som helst drik af Druer; han må hverken spise friske eller tørrede Druer;
\par 4 så længe hans Indvielse varer, må han intet som helst nyde, der kommer af Vinstokken, hverken umodne Druer eller friske Skud.
\par 5 Så længe hans Indvielsesløfte gælder, må ingen Ragekniv komme på hans Hoved; indtil Udløbet af den Tid han indvier sig til HERREN, skal han være hellig og lade sit Hovedhår vokse frit.
\par 6 Hele den Tid han har indviet sig til HERREN, må han ikke komme Lig nær;
\par 7 selv når hans Fader eller Moder, hans Broder eller Søster dør, må han ikke pådrage sig Urenhed ved dem, thi han bærer sin Guds indvielse på sit Hoved.
\par 8 Så længe hans Indvielse varer, er han helliget HERREN.
\par 9 Men når nogen uventet og pludselig dør i hans Nærhed, og han således bringer Urenhed over sit indviede Hoved, skal han rage sit Hoved, den Dag han atter bliver ren; den syvende Dag skal han rage det;
\par 10 og den ottende Dag skal han bringe to Turtelduer eller Dueunger til Præsten ved Åbenbaringsteltets Indgang.
\par 11 Og Præsten skal ofre den ene som Syndoffer og den anden som Brændoffer og skaffe ham Soning, fordi han har syndet ved at røre ved Lig. Derpå skal han samme Dag atter hellige sit Hoved
\par 12 og atter indvie sig til HERREN for lige så lang Tid, som han før havde indviet sig, og bringe et årgammelt Lam som Skyldoffer; den forløbne Tid regnes ikke med, da han har bragt Urenhed over sit indviede Hoved.
\par 13 Dette er Loven om Nasiræeren: Når hans indvielsestid er til Ende, skal han begive sig til Åbenbaringsteltets Indgang
\par 14 og som Offergave bringe HERREN et årgammelt, lydefrit Væderlam til Brændoffer, et årgammelt, lydefrit Hunlam til Syndoffer og en lydefri Væder til Takoffer,
\par 15 en Kurv med usyret Bagværk, Kager af fint Hvedemel, rørte i Olie, og usyrede Fladbrød, smurte med Olie, desuden det tilhørende Afgrødeoffer og de tilhørende Drikofre.
\par 16 Så skal Præsten bringe det for HERRENs Åsyn og ofre hans Syndoffer og Brændoffer,
\par 17 og Væderen skal han ofre som Takoffer til HERREN tillige med de usyrede Brød i Kurven; derpå skal Præsten ofre hans Afgrødeoffer og Drikofer.
\par 18 Så skal Nasiræeren ved Indgangen til Åbenbaringsteltet rage sit indviede Hoved og tage sit indviede Hovedhår og kaste det i Ilden under Takofferet.
\par 19 Og Præsten skal tage den kogte Bov af Væderen og een usyret Kage og eet usyret Fladbrød af Kurven og lægge dem på Nasiræerens Hænder, efter at han har afraget sit indviede Hovedhår.
\par 20 Og Præsten skal udføre Svingningen dermed for HERRENs Åsyn; det tilfalder Præsten som Helliggave foruden Svingningsbrystet og Offerydelseskøllen. Derefter må Nasiræeren atter drikke Vin.
\par 21 Det er Loven om Nasiræeren, der aflægger Løfte, om hans Offergave til HERREN i Anledning af Indvielsen, foruden hvad han ellers evner at give; overensstemmende med Løftet, han aflægger, skal han forholde sig efter den for hans Indvielse gældende Lov.
\par 22 HERREN talede fremdeles til Moses og sagde:
\par 23 Tal til Aron og hans Sønner og sig: Når I velsigner Israeliterne, skal I sige til dem:
\par 24 HERREN velsigne dig og bevare dig,
\par 25 HERREN lade sit Ansigt lyse over dig og være dig nådig,
\par 26 HERREN løfte sit Åsyn på dig og give dig Fred!
\par 27 Således skal de lægge mit Navn på Israeliterne, og jeg vil velsigne dem.

\chapter{7}

\par 1 Da Moses var færdig med at rejese Boligen og havde salvet og helliget den med alt dens Tilbehør og ligeledes salvet og helliget Alteret med alt dets Tilbehør,
\par 2 trådte Israels Øverster, Overhovederne for deres Fædrenehuse, Stammernes Øverster, der havde forestået Mønstringen, frem
\par 3 og førte deres Offergave frem for HERRENs Åsyn, seks lukkede Vogne og tolv Stykker Hornkvæg, en Vogn for hver to Øverster og et Stykke Hornkvæg for hver een, og de bragte dem hen foran Boligen.
\par 4 Da sagde HERREN til Moses:
\par 5 Modtag dette af dem, for at det kan bruges til Arbejdet ved Åbenbaringsteltet, og giv Leviterne det med Henblik på hver enkeltes særlige Arbejde!
\par 6 Så modtog Moses Vognene og Hornkvæget og gav Leviterne dem.
\par 7 To Vogne og fire Stykker Hornkvæg gav han Gersoniterne med Henblik på deres særlige Arbejde,
\par 8 og fire Vogne og otte Stykker Hornkvæg gav han Merariterne med Henblik på deres særlige Arbejde under Itamars, Præsten Arons Søns, Ledelse.
\par 9 Derimod gav han ikke Kebatiterne noget, thi dem var Arbejdet med de hellige Ting overdraget, og de skulde bære dem på Skuldrene.
\par 10 Fremdeles bragte Øversterne Offergaver til Alterets indvielse, dengang det blev salvet, og Øversterne bragte deres Offergaver hen foran Alteret.
\par 11 Da sagde HERREN til Moses: Lad hver af Øversterne få sin Dag til at bringe sin Offergave til Alterets Indvielse.
\par 12 Den, som første Dag bragte sin Offergave, var Nahasjon, Amminadabs Søn af Judas Stamme.
\par 13 Og hans Offergave var et Sølvfad, der vejede 130 Sekel, og en Sølvskål på 70 Sekel efter hellig Vægt, begge fyldte med fint Hvedemel, rørt i Olie, til Afgrødeoffer,
\par 14 en Kande på 10 Guldsekel, fyldt med Røgelse,
\par 15 en ung Tyr, en Væder, et årgammelt Lam til Brændoffer,
\par 16 en Gedebuk til Syndoffer
\par 17 og til Takoffer to Stykker Hornkvæg, fem Vædre, fem Bukke og fem årgamle Lam. Det var Nahasjons, Amminadabs Søns, Offergave.
\par 18 Anden Dag bragte Netanel, Zuars Søn, Issakars Øverste, sin Offergave;
\par 19 han bragte som Offergave et Sølvfad, der vejede 130 Sekel, og en Sølvskål på 70 Sekel efter hellig Vægt, begge fyldte med fint Hvedemel, rørt i Olie, til Afgrødeoffer,
\par 20 en Kande på 10 Guldsekel, fyldt med Røgelse,
\par 21 en ung Tyr, en Væder, et årgammelt Lam til Brændoffer,
\par 22 en Gedebuk til Syndoffer
\par 23 og til Takoffer to Stykker Hornkvæg, fem Vædre, fem Bukke og fem årgamle Lam. Det var Neanels, Zuars Søns, Offergave.
\par 24 Tredje Dag kom Zebuloniternes Øverste, Eliab, Helons Søn;
\par 25 hans Offergave var et Sølvfad, der vejede 130 Sekel, og en Sølvskål på 70 Sekel efter hellig Vægt, begge fyldte med fint Hvedemel, rørt i Olie, til Afgrødeoffer,
\par 26 en Kande på 10 Guldsekel, fyldt med Røgelse,
\par 27 en ung Tyr, en Væder, et årgammelt Lam til Brændoffer,
\par 28 en Gedebuk til Syndoffer
\par 29 og til Takoffer to Stykker Hornkvæg, fem Vædre, fem Bukke og fem årgamle Lam. Det var Eliabs, Helons Søns, Offergave.
\par 30 Fjerde Dag kom Rubeniternes Øverste, Elizur, Sjedeurs Søn;
\par 31 hans Offergave var et Sølvfad, der vejede 130 Sekel, og en Sølvskål på 70 Sekel efter hellig Vægt, begge fyldte med fint Hvedemel, rørt i Olie, til Afgrødeoffer,
\par 32 en Kande på 10 Guldsekel, fyldt med Røgelse,
\par 33 en ung Tyr, en Væder,
\par 34 et årgammelt Lam til Brændoffer, en Gedebuk til Syndoffer
\par 35 og til Takoffer to Stykker Hornkvæg, fem Vædre, fem Bukke og fem årgamle Lam. Det var Elizurs, Sjødeurs Søns, Offergave.
\par 36 Femte Dag kom Simenoiternes Øverste, Sjelumiel, Zurisjaddajs Søn;
\par 37 hans Offergave var et Sølvfad, der vejede 130 Sekel, og en Sølvskål på 70 Sekel efter hellig Vægt, begge fyldte med fint Hvedemel, rørt i Olie, til Afgrødeoffer,
\par 38 en Kande på 10 Guldsekel, fyldt med Røgelse,
\par 39 en ung Tyr, en Væder, et årgammelt Lam til Brændoffer,
\par 40 en Gedebuk til Syndoffer
\par 41 og til Takoffer to Stykker Hornkvæg, fem Vædre, fem Bukke. og fem årgamle Lam. Det var Sjelumiels, Zurisjaddajs Søns,Offergave.
\par 42 Sjette Dag kom Gaditernes Øverste, Eljasaf, Deuels Søn;
\par 43 hans Offergaver et Sølvfad, der vejede l3O Sekel, og en Sølvskål på 70 Sekel efter hellig Vægt, begge fyldte med fint Hvedemel, rørt i Olie, til Afgrødeoffer,
\par 44 en Kande på lO Guldsekel, fyldt med Røgelse,
\par 45 en ung Tyr, en Væder, et årgammelt Lam til Brændoffer,
\par 46 en Gedebuk til Syndoffer
\par 47 og til Takoffer to Stykker Hornkvæg, fem Vædre, fem Bukke og fem årgamle Lam. Det var Eljasafs, Deuels Søns, Offergave.
\par 48 Syvende Dag kom Efraimiternes Øverste, Elisjama, Ammihuds Søn:
\par 49 hans offergave var et Sølvfad, der vejede 130 Sekel, og en Sølvskål på 70 Sekel efter hellig Vægt, begge fyldte med fint Hvedemel, rørt i Olie, til Afgrødeoffer,
\par 50 en Kande på lO Guldsekel, fyldt med Røgelse,
\par 51 en ung Tyr, en Væder, et årgammelt Lam til Brændoffer,
\par 52 en Gedebuk til Syndoffer
\par 53 og til Takoffer to Stykker Hornkvæg, fem Vædre, fem Bukke og fem årgamle Lam. Det var Elisjamas, Ammihuds Søns, Offergave.
\par 54 Ottende Dag kom Mannassiternes Øverste, Gamliel, Pedazurs Søn;
\par 55 hans Offergave var et Sølvfad, der vejede 130 Sekel, og en Sølvskål på 7O Sekel efter hellig Vægt, begge fyldte med fint Hvedemel, rørt i Olie, til Afgrødeoffer,
\par 56 en Kande på lO Guldsekel, fyldt med Røgelse,
\par 57 en ung Tyr, en Væder, et årgammelt Lam til Brændoffer,
\par 58 en Gedebuk til Syndoffer
\par 59 og til Takoffer to Stykker Hornkvæg, fem Vædre, fem Bukke og fem årgamle Lam. Det var Gamliels, Pedazurs Søns, Offergave.
\par 60 Niende dag kom Benjaminiternes Øverste, Abidan, Gidonis Søn;
\par 61 hans Offergave var et Sølvfad, der vejede 130 Sekel, og en Sølvskål på 70 Sekel efter hellig Vægt, begge fyldte med fint Hvedemel, rørt i Olie, til Afgrødeoffer,
\par 62 en Kande på IO Guldsekel, fyldt med Røgelse,
\par 63 en ung Tyr, en Væder, et årgammelt Lam til Brændoffer,
\par 64 en Gedebuk til Syndoffer
\par 65 og til Takoffer to Stykker Hornkvæg, fem, Vædre, fem Bukke og fem årgamle Lam. Det var Abidans, Gidonis Søns, Offergave.
\par 66 Tiende Dag kom Daniternes Øverste, Ahiezer, Ammisjaddajs Søn;
\par 67 hans Offergave var et Sølvfad, der vejede 130 Sekel, og en Sølvskål på 70 Sekel efter hellig Vægt, begge fyldte med fint Hvedemel, rørt i Olie, til Afgrødeoffer,
\par 68 en Kande på lO Guldsekel, fyldt med Røgelse,
\par 69 en ung Tyr, en Væder, et årgammelt Lam til Brændoffer,
\par 70 en Gedebuk til Syndoffer
\par 71 og til Takoffer to Stykker Hornkvæg, fem Vædre, fem Bukke og fem årgamle Lam. Det var Ahiezers, Ammisjaddajs Søns, Offergave.
\par 72 Elevte Dag kom Aseriternes Øverste, Pagiel, Okrans Søn;
\par 73 hans Offergave var et Sølvfad, der vejede 130 Sekel, og en Sølvskål på 7O Sekel efter hellig Vægt, begge fyldte med fint Hvedemel, rørt i Olie, til Afgrødeoffer,
\par 74 en Kande på 1O Guldsekel, fyldt med Røgelse,
\par 75 en ung Tyr, en Væder, et årgammelt Lam til Brændoffer,
\par 76 en Gedebuk til Syndoffer
\par 77 og til Takoffer to Stykker Hornkvæg, fem Vædre, fem Bukke og fem årgamle Lam. Det var Pagiels, Okrans Søns, Offergave.
\par 78 Tolvte Dag kom Naftaliternes Øverste, Ahira, Enans Søn;
\par 79 hans Offergave var et Sølvfad, der vejede 130 Sekel, og en Sølvskål på 70 Sekel efter hellig Vægt, begge fyldte med fint Hvedemel, rørt i Olie, til Afgrødeoffer,
\par 80 en kande på lO Guldsekel, fyldt med Røgelse,
\par 81 en ung Tyr, en Væder, et årgammelt Lam til Brændoffer,
\par 82 en Gedebuk til Syndoffer
\par 83 og til Takoffer to Stykker Hornkvæg, fem Vædre, fem Bukke og fem årgamle Lam. Det var Ahiras, Enans Søns, Offergave.
\par 84 Det var Gaverne fra Israelitternes Øverster til Alterets Indvielse, dengang det blev salvet: 12 Sølvfade, 12 Sølvskåle, 12 Guldkander,
\par 85 hvert Sølvfad på 130 Sekel og hver Sølvskål på 70 Sekel, alle Sølvkar tilsammen 2400 Sekel efter hellig Vægt;
\par 86 12 Guldkander, fyldte med Røgelse, hver på 1O Sekel efter hellig Vægt, alle Guldkander tilsammen 12O Sekel.
\par 87 Kvæget til Brændofferet var i alt 12 unge Tyre, 12 Vædre, 12 årgamle Lam med tilhørende Afgrødeofre, 12 Gedebukke til Syndoffer;
\par 88 Kvæget til Takofferet var i alt 24 unge Tyre, 60 Vædre, 60 Bukke og 60 årgamle lam. Det var Gaverne til Alterets indvielse, efter at det var salvet.
\par 89 Da Moses gik ind i Åbenbaringsteltet for at tale med HERREN, hørte han Røsten tale til sig fra Sonedækket oven over Vidnesbyrdets Ark, fra Pladsen mellem de to Keruber. Og han talede til ham.

\chapter{8}

\par 1 Og HERREN talede til Moses og sagde:
\par 2 Tal til Aron og sig til ham: Når du sætter Lamperne på, skal du sætte dem således, at de syv Lamper kaster Lyset ud over Pladsen foran Lysestagen!
\par 3 Det gjorde Aron så; han satte Lamperne på således, at de vendte ud mod Pladsen foran Lysestagen, som HERREN havde pålagt Moses.
\par 4 Men Lysestagen var lavet af Guld i drevet Arbejde, fra Foden til Kronerne var den drevet Arbejde; efter det Forbillede, HERREN havde vist ham, havde Moses lavet Lysestagen.
\par 5 HERREN talede fremdeles til Moses og sagde:
\par 6 Udtag Leviterne af Israelitternes Midte og rens dem!
\par 7 Således skal du gå frem, når du renser dem: Stænk Renselsesvand på dem og lad dem gå hele deres Legeme over med Ragekniv, tvætte deres klæder og rense sig.
\par 8 Derpå skal de tage en ung Tyr til Brændoffer med tilhørende Afgrødeoffer af fint Hvedemel, rørt i Olie, og du skal tage en anden ung Tyr til Syndoffer.
\par 9 Lad så Leviterne træde hen foran Åbenbaringsteltet og kald hele Israelitternes Menighed sammen.
\par 10 Når du så har ladet Leviterne træde frem for HERRENs Åsyn, skal Israeliterne lægge deres Hænder på Leviterne.
\par 11 Derpå skal Aron udføre Svingningen med Leviterne for HERRENs Åsyn som et Offer fra Israeliterne, for at de kan udføre HERRENs Arbejde.
\par 12 Så skal Leviterne lægge deres Hænder på Tyrenes Hoved, og derefter skal du ofre den ene som Syndoffer og den anden som Brændoffer til HERREN for at skaffe Leviterne Soning.
\par 13 Derpå skal du stille Leviterne frem for Aron og hans Sønner og udføre Svingningen med dem for HERREN.
\par 14 Således skal du udskille Leviterne fra Israeliterne, så at Leviterne kommer til at tilhøre mig.
\par 15 Derefter skal Leviterne komme og gøre Arbejde ved Åbenbaringsteltet; du skal foretage Renselsen og udføre Svingningen med dem,
\par 16 thi de er skænket mig som Gave af Israelitternes Midte; i Stedet for alt hvad der åbner Moders Liv, alle førstefødte hos Israeliterne, har jeg taget mig dem til Ejendom.
\par 17 Thi mig tilhører alt det førstefødte hos Israeliterne, både af Mennesker og Kvæg. Dengang jeg slog alle de førstefødte i Ægypten, helligede jeg dem til at være min Ejendom.
\par 18 Jeg tog Leviterne i Stedet for alt det førstefødte hos Israeliterne
\par 19 og skænkede Leviterne som Gave til Aron og hans Sønner af Israelitternes Midte til at udføre Israelitternes Arbejde ved Åbenbaringsteltet og skaffe Israeliterne Soning, for at ingen Plage skal ramme Israeliterne, om de selv nærmer sig Helligdommen.
\par 20 Og Moses, Aron og hele Israelitternes Menighed gjorde således med Leviterne; ganske som HERREN havde pålagt Moses med Hensyn til Leviterne, gjorde Israeliterne med dem.
\par 21 Leviterne lod sig rense for Synd og tvættede deres Klæder, og Aron udførte Svingningen med dem for HERRENs Åsyn, og Aron skaffede dem Soning, så de blev rene.
\par 22 Derpå kom Leviterne for at udføre deres Arbejde ved Åbenbaringsteltet under Arons og hans Sønners Tilsyn; som HERREN havde pålagt Moses med Hensyn til Leviterne, således gjorde de med dem.
\par 23 HERREN talede fremdeles til Moses og sagde:
\par 24 Dette er, hvad der gælder Leviterne: Fra Femogtyveårsalderen og opefter skal de komme og gøre Tjeneste med Arbejdet ved Åbenbaringsteltet.
\par 25 Men fra Halvtredsårsalderen skal de holde op med at gøre Tjeneste og ikke arbejde mere;
\par 26 de kan gå deres Brødre til Hånde i Åbenbaringsteltet med at tage Vare på, hvad der skal varetages; men Arbejde skal de ikke mere gøre. Således skal du gøre med Leviterne i alt, hvad de har at varetage.

\chapter{9}

\par 1 HERREN talede således til Moses i Sinaj Ørken i den første Måned af det andet År efter deres Udvandring fra Ægypten:
\par 2 Israeliterne skal fejre Påsken til den fastsatte Tid,
\par 3 på den fjortende Dag i denne Måned ved Aftenstid skal I fejre den til den fastsatte Tid; overensstemmende med alle Anordningerne og Lovbudene om den skal I fejre den!
\par 4 Da sagde Moses til Israeliterne, at de skulde fejre Påsken;
\par 5 og de fejrede Påsken i Sinaj Ørken den fjortende Dag i den første Måned ved Aftenstid; nøjagtigt som HERREN havde pålagt Moses, således gjorde Israeliterne.
\par 6 Men der var nogle Mænd, som var blevet urene ved et Lig og derfor ikke kunde fejre Påske den Dag. Disse Mænd trådte nu den Dag frem for Moses og Aron
\par 7 og sagde til ham: "Vi er blevet urene ved et Lig; hvorfor skal det da være os forment at frembære HERRENs Offergave til den fastsatte Tid sammen med de andre Israelitter?"
\par 8 Moses svarede dem: "Vent, til jeg får at høre, hvad HERREN påbyder angående eder!"
\par 9 Da talede HERREN til Moses og sagde:
\par 10 Tal til Israeliterne og sig: Når nogen af eder eller eders Efterkommere er blevet uren ved et Lig eller er ude på en lang Rejse, skal han alligevel holde Påske for HERREN;
\par 11 den fjortende Dag i den anden Måned ved Aftenstid skal de holde den; med usyret Brød og bitre Urter skal de spise Påskelammet.
\par 12 De må intet levne deraf til næste Morgen, og de må ikke sønderbryde noget af dets Ben. De skal fejre Påsken i Overensstemmelse med alle de Anordninger, som gælder for den.
\par 13 Men den, som undlader at fejre Påsken, skønt han er ren og ikke på Rejse, det Menneske skal udryddes af sin Slægt, fordi han ikke har frembåret HERRENs Offergave til den fastsatte Tid. Den Mand skal undgælde for sin Synd.
\par 14 Og når en fremmed, der bor hos eder, vil bolde Påske for HERREN, skal han holde den efter de Anordninger og Lovbud, som gælder for Påsken. En og samme Lov skal gælde for eder, for den fremmede og for den indfødte.
\par 15 Den Dag Boligen blev rejst, dækkede Skyen Boligen, Vidnesbyrdets Telt; men om Aftenen var der som et Ildskær over Boligen, og det holdt sig til om Morgenen.
\par 16 Således var det til Stadighed: Om dagen dækkede Skyen den, om Natten Ildskæret.
\par 17 Hver Gang Skyen løftede sig fra Teltet, brød Israeliterne op, og der, hvor Skyen stod stille, slog Israeliterne Lejr.
\par 18 På HERRENs Bud brød Israeliterne op, og på HERRENs Bud gik de i Lejr, og så længe Skyen hvilede over Boligen, blev de liggende i Lejr;
\par 19 når Skyen blev over Boligen i længere Tid, rettede Israeliterne sig efter, hvad HERREN havde foreskrevet dem, og brød ikke op.
\par 20 Det hændte, at Skyen kun blev nogle få Dage over Boligen; da gik de i Lejr på HERRENs Bud og brød op på HERRENs Bud.
\par 21 Og det hændte, at Skyen kun blev der fra Aften til Morgen; når Skyen da løftede sig om Morgenen, brød de op. Eller den blev der en Dag og en Nat; når Skyen da løftede sig, brød de op.
\par 22 Eller den blev der et Par Dage eller en Måned eller længere endnu, idet Skyen i længere Tid hvilede over Boligen; så blev Israeliterne liggende i Lejr og brød ikke op, men når den løftede sig, brød de op.
\par 23 På HERRENs Bud gik de i Lejr, og på HERRENs Bud brød de op; de rettede sig efter, hvad HERREN havde foreskrevet dem, efter HERRENs Bud ved Moses.

\chapter{10}

\par 1 HERREN talede fremdeles til Moses og sagde:
\par 2 Du skal lave dig to Sølvtrompeter; i drevet Arbejde skal du lave dem. Dem skal du bruge, når Menigheden skal kaldes sammen, og når Lejrene skal bryde op.
\par 3 Når der blæses i dem begge to, skal hele Menigheden samle sig hos dig ved Indgangen til Åbenbaringsteltet.
\par 4 Blæses der kun i den ene, skal Øversterne, Overhovederne for Israels Stammer, samle sig hos dig.
\par 5 Når I blæser Alarm med dem, skal Lejrene på Østsiden bryde op;
\par 6 blæser Alarm anden Gang, skal Lejrene på Sydsiden bryde op: tredje Gang skal Lejrene mod Vest bryde op, fjerde Gang Lejrene mod Nord; der skal blæses Alarm, når de skal bryde op.
\par 7 Men når Forsamlingen skal sammenkaldes skal I blæse på almindelig Vis, ikke Alarm.
\par 8 Arons Sønner, Præsterne, skal blæse i Trompeterne; det skal være eder en evig gyldig Anordning fra Slægt til Slægt.
\par 9 Når I drager i Krig i eders Land mod en Fjende, der angriber eder, og blæser Alarm med Trompeterne, skal I ihukommes for HERREN eders Guds Åsyn og frelses fra eders Fjender.
\par 10 Og på eders Glædesdage, eders Højtider og Nymånedage, skal I blæse i Trompeterne ved eders Brændofre og Takofre; så skal de tjene eder til Ihukommelse for eders Guds Åsyn. Jeg er HERREN eders Gud!
\par 11 Den tyvende Dag i den anden Måned af det andet År løftede Skyen sig fra Vidnesbyrdets Bolig.
\par 12 Da brød Israeliterne op fra Sinaj Ørken, i den Orden de skulde bryde op i, og Skyen stod stille i Parans Ørken.
\par 13 Da de nu første Gang brød op efter Guds Bud ved Moses,
\par 14 var Judæernes Lejr den første, der brød op under sit Felttegn, Hærafdeling for Hærafdeling; deres Hær førtes af Nahasjon, Amminadabs Søn.
\par 15 Issakariternes Stammes Hær førtes af Netanel, Zuars Søn,
\par 16 og Zebuloniternes Stammes Hær af Eliab, Helons Søn.
\par 17 Da derpå Boligen var taget ned, brød Gersoniterne og Merariterne op og bar Boligen.
\par 18 Så brød Rubens Lejr op under sit Felttegn, Hærafdeling for Hærafdeling; deres Hær førtes af Elizur, Sjedeurs Søn.
\par 19 Simeoniternes Stammes Hær førtes af Sjelumiel, Zurisjaddajs Søn,
\par 20 og Gaditernes Stammes Hær af Eljasaf, Deuels Søn.
\par 21 Så brød Kehatiterne, der bar de hellige Ting, op; og før deres Komme havde man rejst Boligen.
\par 22 Så brød Efraimiternes Lejr op under sit Felttegn, Hærafdeling for Hærafdeling; deres Hær førtes af Elisjema, Ammihuds Søn.
\par 23 Manassiternes Stammes Hær førtes af Gamliel, Pedazurs Søn,
\par 24 og Benjaminiternes Stammes Hær af Abidan, Gidonis Søn.
\par 25 Så brød Daniternes Lejr op under sit Felttegn som Bagtrop i hele Lejrtoget, Hærafdeling for Hærafdeling; deres Hær førtes af Ahiezer, Ammisjaddajs Søn.
\par 26 Aseriternes Stammes Hær førtes af Pagiel, Okrans Søn,
\par 27 og Naftalitemes Stammes Hær af Ahira, Enans Søn.
\par 28 Således foregik Israelitternes Opbrud, Hærafdeling for Hærafdeling. Så brød de da op.
\par 29 Men Moses sagde til Midjaniten Hobab, Reuels Søn, Moses's Svigerfader: "Vi bryder nu op for at drage til det Sted, HERREN har lovet at give os; drag med os! Vi skal lønne dig godt, thi HERREN har stillet Israel gode Ting i Udsigt."
\par 30 Men han svarede ham: "Jeg vil ikke drage med; nej, jeg drager til mit Land og min Slægt."
\par 31 Da sagde han: "Forlad os ikke! Du kender jo de Steder, hvor vi kan slå Lejr i Ørkenen; vær Øje for os!
\par 32 Drager du med os, skal vi give dig Del i alt det gode, HERREN vil give os."
\par 33 Derpå brød de op fra HERRENs Bjerg og vandrede tre Dagsrejser frem, idet HERRENs Pagts Ark drog i Forvejen for at søge dem et Sted, hvor de kunde holde Hvil.
\par 34 Og HERRENs Sky svævede over dem om Dagen, når de brød op fra Lejren.
\par 35 Og hver Gang Arken brød op, sagde Moses: "Stå op, HERRE, at dine Fjender må splittes og dine Avindsmænd fly for dit Åsyn!"
\par 36 Og hver Gang den standsede, sagde han: "Vend tilbage, HERRE, til Israels Stammers Titusinder!"

\chapter{11}

\par 1 Men Folket knurrede højlydt for HERREN over deres usle Kår; og da HERREN hørte det, blussede hans Vrede op, og HERRENs Ild brød løs iblandt dem og åd om sig i den yderste Del af Lejren.
\par 2 Da råbte Folket til Moses, og Moses gik i Forbøn hos HERREN.
\par 3 Derfor kaldte man dette Sted Tabera, fordi HERRENs Ild brød løs iblandt dem.
\par 4 Men den sammenløbne Hob, som fandtes iblandt dem, blev lysten.
\par 5 Vi mindes Fiskene, vi fik at spise for intet i Ægypten, og Agurkerne, Vandmelonerne, Porrerne, Hvidløgene og Skalotterne,
\par 6 og nu vansmægter vi; her er hverken det ene eller det andet, vi ser aldrig andet end Manna."
\par 7 Mannaen lignede Horianderfrø og så ud som Bellium.
\par 8 Folket gik rundt og sankede den op; derpå malede de den i Håndkværne eller stødte den i Mortere; så kogte de den i Gryder og lavede Kager deraf; den smagte da som Bagværk tillavet i Olie.
\par 9 Når Duggen om Natten faldt over Lejren, faldt også Mannaen ned over den.
\par 10 Og Moses hørte, hvorledes alle Folkets Slægter græd, enhver ved Indgangen til sit Telt; da blussede HERRENs Vrede voldsomt op, og det vakte også Moses's Mishag.
\par 11 Da sagde Moses til HERREN: "Hvorfor har du handlet så ilde med din Tjener, og hvorfor har jeg ikke fundet Nåde for dine Øjne, siden du har lagt hele dette Folk som en Byrde på mig?
\par 12 Mon det er mig, der har undfanget hele dette Folk, mon det er mig, der har født det, siden du forlanger, at jeg i min Favn skal bære det hen til det Land, du tilsvor dets Fædre, som en Fosterfader bærer det diende Barn?
\par 13 Hvor skal jeg gå hen og skaffe hele dette Folk Kød? Thi de græder rundt om mig og siger: Skaf os Kød at spise?
\par 14 Jeg kan ikke ene bære hele dette Folk, det er mig for tungt.
\par 15 Hvis du vil handle således med mig, så dræb mig hellere, om jeg har fundet Nåde for dine Øjne, så at jeg ikke skal være nødt til at opleve sådan Elendighed!"
\par 16 HERREN svarede Moses: "Kald mig halvfjerdsindstyve af Israels Ældste sammen, Mænd, som du ved hører til Folkets Ældste og Tilsynsmænd, før dem hen til Åbenbaringsteltet og lad dem stille sig op hos dig der.
\par 17 Så vil jeg stige ned og tale med dig der, og jeg vil tage noget af den Ånd, der er over dig, og lade den komme over dem, for at de kan hjælpe dig med at bære Folkets Byrde, så du ikke ene skal bære den.
\par 18 Men til Folket skal du sige: Helliger eder til i Morgen, så skal I få Kød at spise! I har jo grædt højlydt for HERREN og sagt: Kunde vi dog få Kød at spise! Vi havde det jo bedre i Ægypten! Derfor vil HERREN give eder kød at spise;
\par 19 og ikke blot een eller to eller fem eller ti eller tyve Dage skal I spise det,
\par 20 men en hel Måned igennem, indtil det står eder ud af Næsen, og I væmmes derved, fordi I har ringeagtet HERREN, der er i eders Midte, og grædt for hans Åsyn og sagt: Hvorfor drog vi dog ud af Ægypten!"
\par 21 Moses svarede: "600000 Fodfolk tæller det Folk, jeg bar om mig, og du siger: Jeg vil skaffe dem Kød, så de har nok at spise en hel Måned!
\par 22 Kan der slagtes så meget Småkvæg og Hornkvæg til dem, at det kan slå til, eller kan alle Fisk i Havet samles sammen til dem, så det kan slå til?"
\par 23 HERREN svarede Moses: "Er HERRENs Arm for kort? Nu skal du få at se, om mit Ord går i Opfyldelse for dig eller ej."
\par 24 Da gik Moses ud og kundgjorde Folket HERRENs Ord. Og han samlede halvfjerdsindstyve af Folkets Ældste og lod dem stille sig rundt om Teltet.
\par 25 Så steg HERREN ned i Skyen og talede til ham; og han tog noget af den Ånd, der var over ham, og lod den komme over de halvfjerdsindstyve Ældste, og da Ånden hvilede over dem, kom de i profetisk Henrykkelse noget, som ikke siden hændtes dem.
\par 26 Imidlertid var to Mænd blevet tilbage i Lejren, den ene hed Eldad, den anden Medad. Også over dem kom Ånden, thi de hørte til dem, der var optegnede, men de var ikke gået ud til Teltet, og nu kom de i profetisk Henrykkelse inde i Lejren.
\par 27 Da løb en ung Mand ud og fortalte Moses det og sagde: "Eldad og Medad er kommet i profetisk Henrykkelse inde i Lejren."
\par 28 Josua, Nuns Søn, der fra sin Ungdom af havde gået Moses til Hånde, sagde da: "Min Herre Moses, stands dem i det!"
\par 29 Men Moses sagde til ham: "Er du skinsyg på mine Vegne? Gid alt HERRENs Folk var Profeter, gid HERREN vilde lade sin Ånd komme over dem!"
\par 30 Derpå trak Moses sig tilbage til Lejren med Israels Ældste.
\par 31 Da rejste der sig på HERRENs Bud en Vind, som førte Vagtler med sig fra Havet og drev dem hen over Lejren så langt som en Dagsrejse på begge Sider af Lejren i en Højde af et Par Alen over Jorden.
\par 32 Så gav Folket sig hele den Dag, hele Natten og hele den næste Dag til at samle Vagtlerne op; det mindste, nogen samlede, var ti Homer. Og de bredte dem ud til Tørring rundt om Lejren.
\par 33 Medens Kødet endnu var imellem Tænderne på dem, før det endnu var spist, blussede HERRENs Vrede op imod Folket, og HERREN lod en meget hård Straf ramme Folket.
\par 34 Og man kaldte Stedet Kibrot Hattåva, thi der blev de lystne Folk jordet.
\par 35 Fra Kibrot-Hattåva drog Folket til Hazerot, og de gjorde Holdt i Hazerot.

\chapter{12}

\par 1 Mirjam og Aron tog til Orde mod Moses i Anledning af den kusjitiske Kvinde, han havde ægtet - han havde nemlig ægtet en kusjitisk Kvinde -
\par 2 og sagde: "Har HERREN kun talet til Moses? Mon han ikke også har talet til os?" Og HERREN hørte det.
\par 3 Men den Mand Moses var såre sagtmodig, sagtmodigere end noget andet Menneske på Jorden.
\par 4 Da sagde HERREN i det samme til Moses, Aron og Mirjam: "Gå alle tre ud til Åbenbaringsteltet!" Og de gik alle tre derud.
\par 5 Men HERREN steg ned i Skystøtten og stillede sig ved Indgangen til Teltet og kaldte på Aron og Mirjam, og de gik begge derud.
\par 6 Da sagde han: "Hør, hvad jeg siger: Når der ellers er en Profet iblandt eder, giver jeg mig til Kende for ham i Syner eller taler med ham i Drømme.
\par 7 Anderledes er det med min Tjener Moses: han er tro i hele mit Hus;
\par 8 med ham taler jeg Ansigt til Ansigt, ikke i Syner eller Gåder, han skuer HERRENs Skikkelse; hvor tør I da tage til Orde mod min Tjener Moses?"
\par 9 Og HERRENs Vrede blussede op imod dem, og han gik bort.
\par 10 Da så Skyen trak sig bort fra Teltet, se, da var Mirjam hvid som Sne af Spedalskhed, og da Aron vendte sig mod Mirjam, se, da var hun spedalsk.
\par 11 Da sagde Aron til Moses: "Ak, Herre, lad os dog ikke undgælde for den Synd, vi i Dårskab begik!
\par 12 Lad hende dog ikke blive som et dødfødt Barn, hvis Kød er halvt fortæret, når det kommer ud af Moders Liv!"
\par 13 Moses råbte da til HERREN og sagde: "Ak, gør hende dog rask igen!"
\par 14 Og HERREN svarede Moses: "Hvis hendes Fader havde spyttet hende i Ansigtet, måtte hun da ikke have båret sin Skam i syv Dage? Derfor skal hun i syv Dage være udelukket fra Lejren; så kan hun atter optages."
\par 15 Da blev Mirjam udelukket fra Lejren i syv Dage, og Folket brød ikke op, før Mirjam atter var optaget.
\par 16 Så brød Folket op fra Hazerot og slog Lejr i Parans Ørken.

\chapter{13}

\par 1 HERREN talede fremdeles til Moses og sagde:
\par 2 "Send nogle Mænd af Sted for at undersøge Kana'ans Land, som jeg vil give Israeliterne; een Mand af hver Fædrenestamme skal I sende, og kun Mænd, der er Øverster iblandt dem!"
\par 3 Da udsendte Moses dem fra Parans Ørken på HERRENs Bud, Mænd, der allevar Overhovederfor Israeliterne.
\par 4 Deres Navne var følgende: Af Rubens Stamme Sjammua, Zakkurs Søn,
\par 5 af Simeons Stamme Sjafat, Horis Søn,
\par 6 af Judas Stamme Kaleb, Jefunnes Søn,
\par 7 af Issakars Stamme Jigal, Josefs Søn,
\par 8 af Efraims Stamme Hosea, Nuns Søn,
\par 9 af Benjamins Stamme Palti, Rafus Søn,
\par 10 af Zebulons Stamme Gaddiel, Sodis Søn,
\par 11 af Josefs Slamme, det er Manasses Stamme, Gaddi, Susis Søn,
\par 12 af Dans Stamme Ammiel, Gmallis Søn,
\par 13 af Asers Stamme Setur, Mikaels Søn,
\par 14 af Naftalis Stamme Nabi, Vofsis Søn,
\par 15 af Gads Stamme Geuel, Mtakis Søn.
\par 16 Det var Navnene på de Mænd, Moses udsendte for at undersøge Landet. Men Moses gav Hosea, Nuns Søn, Navnet Josua.
\par 17 Og Moses sendte dem af Sted for at undersøge Kana'ans Land.
\par 18 og se, hvordan Landet er, og om Folket, som bor der, er stærkt eller svagt, fåtalligt eller talrigt,
\par 19 om Landet, de bor i, er godt eller dårligt, og om Byerne, de bor i, er Teltlejre eller Fæstninger,
\par 20 og om Landet er fedt eller magert, om der findes Træer deri eller ej. Vær ved godt Mod og tag noget af Landets Frugt med tilbage!" Det var netop ved den Tid, de første Druer var modne.
\par 21 Så drog de af Sted og undersøgte Landet lige fra Zins Ørken til Rehob, Egnen hen imod Hamat.
\par 22 Så begav de sig op i Sydlandet og kom til Hebron; der boede Ahiman, Sjesjaj og Talmaj, Anaks Efterkommere. Men Hebron var grundlagt syv År før Zoan i Ægypten.
\par 23 Da de nåede Esjkoldalen, afskar de en Ranke med en Drueklase, som der måtte to Mand til at bære på en Bærestang, og plukkede nogle Granatæbler og Figener.
\par 24 Man kaldte dette Sted Esjkoldalen med Hentydning til den Drueklase, Israeliterne der skar af.
\par 25 Efter fyrretyve Dages Forløb vendte de tilbage efter at have undersøgt Landet;
\par 26 og de kom til Moses og Aron og hele Israeliternes Menighed i Parans Ørken i Kadesj og aflagde Beretning for dem og hele Menigheden og viste dem Landets Frugt.
\par 27 De fortalte ham: "Vi kom til landet, du sendte os til; det er virkelig et Land, der flyder med Mælk og Honning, og her er Flugt derfra;
\par 28 men stærkt er Folket, som bor i Landet, og Byerne er befæstede og meget store; ja, vi så også Efterkommere af Anak der.
\par 29 Amalek bor i Sydlandet, Hetiterte, Jebusiteme og Amoriterne i Bjerglandet og Kana'anæerne ved Havet og langs Jordan."
\par 30 Da søgte Kaleb at bringe Folket til Tavshed over for Moses og sagde: "Lad os kun drage op og underlægge os det, thi vi kan sikkert tage det!"
\par 31 Men de Mænd, der havde været med ham deroppe, sagde: "Vi kan ikke drage op imod dette Folk, thi det er stærkere end vi!"
\par 32 Og de talte nedsættende til Israeliterne om Landet, som de havde undersøgt, og sagde: "Landet, som vi har rejst igennem og undersøgt, er et Land, der fortærer sine Indbyggere, og alle de Folk, vi så der, var kæmpestore.
\par 33 Vi så der Kæmperne Anaks Sønner, der er af Kæmpeslægten og vi forekom både os selv og dem som Græshopper!"

\chapter{14}

\par 1 Da opløftede hele Menigheden sin Røst og brød ud i Klageråb, og Folket græd Natten igennem.
\par 2 Og alle Israelitterne knurrede imod Moses og Aron, og hele Menigheden sagde til dem: "Gid vi var døde i Ægypten eller her i Ørkenen!
\par 3 Hvorfor fører HERREN os til dette Land, når vi skal falde for Sværdet og vore Kvinder og Børn blive til Bytte? Var det dog ikke bedre for os at vende tilbage til Ægypten?"
\par 4 Og de sagde til hverandre: "Lad os vælge os en Fører og vende tilbage til Ægypten!"
\par 5 Da faldt Moses og Aron på deres Ansigt foran hele den israelitiske Menigheds Forsamling.
\par 6 Men Josua, Nuns Søn, og Kaleb, Jefunnes Søn, der havde været med til at undersøge Landet, sønderrev deres Klæder
\par 7 og sagde til hele Israelitternes Menighed: "Landet, vi har rejst igennem og undersøgt, er et såre, såre godt Land.
\par 8 Hvis HERREN har Behag i os, vil han føre os til det Land og give os det, et Land, der flyder med Mælk og Honning.
\par 9 Gør kun ikke Oprør imod HERREN og frygt ikke for Landets Befolkning, thi dem tager vi som en Bid Brød; deres Skygge er veget fra dem, men med os er HERREN; frygt ikke for dem!"
\par 10 Hele Menigheden tænkte allerede på at stene dem; men da kom HERRENs Herlighed til Syne for alle Israelitterne ved Åbenbaringsteltet.
\par 11 Og HERREN sagde til Moses: "Hvor længe skal dette Folk håne mig, og hvor længe vil det vægre sig ved at tro på mig, til Trods for alle de Tegn jeg har gjort i det?
\par 12 Jeg vil slå det med Pest og udrydde det, men dig vil jeg gøre til et Folk, større og stærkere end det!"
\par 13 Men Moses sagde til HERREN: "Ægypterne har hørt, at du i din Vælde har ført dette Folk bort fra dem;
\par 14 og ligeledes har alle dette Lands Indbyggere hørt, at du, HERRE, er midt iblandt dette Folk, thi du, HERRE, åbenbarer dig synligt, idet din Sky står over dem, og du vandrer foran dem om Dagen i en Skystøtte og om Natten i en Ildstøtte.
\par 15 Hvis du nu dræber dette Folk alle som een, vil de Folk, der har hørt dit Ry, sige:
\par 16 Fordi HERREN ikke evnede at føre dette Folk til det Land, han havde tilsvoret dem, lod han dem omkomme i Ørkenen.
\par 17 Derfor, HERRE, lad din Vælde nu vise sig i sin Storhed, således som du forjættede, da du sagde:
\par 18 HERREN er langmodig og rig på Miskundhed, han forlader Misgerning og Overtrædelse, men han lader ingen ustraffet, og han hjemsøger Fædrenes Brøde på Børnene i tredje og fjerde Led.
\par 19 Så tilgiv nu dette Folk dets Misgerning efter din store Miskundhed, således som du har tilgivet dette Folk hele Vejen fra Ægypten og hertil!"
\par 20 Da sagde HERREN: "Jeg tilgiver dem på din Bøn.
\par 21 Men så sandt jeg lever så sandt hele Jorden skal opfyldes af HERRENs Herlighed:
\par 22 Ingen af de Mænd, der har set min Herlighed og de Tegn, jeg har gjort i Ægypten og i Ørkenen, og dog nu for tiende Gang har fristet mig og ikke villet lyde min Røst,
\par 23 ingen af dem skal se det Land, jeg tilsvor deres Fædre! Ingen af dem, der har hånet mig, skal få det at se;
\par 24 kun min Tjener Kaleb lader jeg komme til det Land, han har været i, og hans Efterkommere skal få det i Eje, fordi han havde en anden Ånd og viste mig fuld Lydighed.
\par 25 Men Amalekiterne og Kana'anæerne bor i Lavlandet. Vend derfor om i Morgen, bryd op og drag ud i Ørkenen ad det røde Hav til!"
\par 26 Fremdeles talede HERREN til Moses og Aron og sagde:
\par 27 "Hvor længe skal jeg tåle denne onde Menighed, dem, som bestandig knurrer imod mig? Jeg har hørt Israelitternes Knurren, hørt, hvorledes de knurrer imod mig.
\par 28 Sig til dem: Så sandt jeg lever lyder det fra HERREN, som I har råbt mig i Øret, således vil jeg handle med eder!
\par 29 I Ørkenen her skal eders Kroppe falde, alle I, der blev mønstret, så mange I er fra Tyveårsalderen og opefter, I, som har knurret imod mig.
\par 30 Sandelig, ingen af eder skal komme til det Land, jeg med løftet Hånd svor at ville giver eder at bo i, med Undtagelse af Kaleb, Jefunnes Søn, og Josua, Nuns Søn.
\par 31 Eders små Børn, som I sagde vilde blive til Bytte, dem vil jeg lade komme derhen, og de skal tage det Land i Besiddelse, som I har vraget,
\par 32 men eders egne Kroppe skal falde i Ørkenen her,
\par 33 og eders Sønner skal flakke om i Ørkenen i fyrretyve År og undgælde for eders Bolen, indtil eders Kroppe er gået til Grunde i Ørkenen.
\par 34 Som I brugte fyrretyve Dage til at undersøge Landet, således skal I undgælde for eders Misgerninger i fyrretyve År, eet År for hver Dag, og således få mit Mishag at føle.
\par 35 Jeg HERREN har sagt det: Sandelig, således vil jeg handle med hele denne onde Menighed, der har rottet sig sammen imod mig; i Ørkenen her skal de gå til Grunde, i den skal de dø!"
\par 36 Og de Mænd, Moses havde udsendt for at undersøge Landet, og som efter deres Tilbagekomst havde fået hele Menigheden til at knurre imod ham ved at tale nedsættende om Landet,
\par 37 de Mænd, der havde talt nedsættende om Landet, fik en brat Død for HERRENs Åsyn:
\par 38 kun Josua, Nuns Søn, og Kaleb, Jefunnes Søn, blev i Live af de Mænd, der var draget hen for at undersøge Landet.
\par 39 Men da Moses forebragte alle Israelitterne disse Ord, grebes Folket af stor Sorg;
\par 40 og tidligt næste Morgen drog de op mod det øverste af Bjerglandet og sagde: "Se, vi er rede til at drage op til det Sted, HERREN har talet om, thi vi, har syndet."
\par 41 Da sagde Moses: "Hvorfor vil I overtræde HERRENs Bud? Det vil ikke gå godt!
\par 42 Drag ikke derop, thi HERREN er ikke iblandt eder; gør I det, bliver I slået af eders Fjender!
\par 43 Thi Amalekiterne og Kana'anæerne vil møde eder der, og I skal falde for Sværdet; I har jo vendt eder fra HERREN, og HERREN er ikke med eder!"
\par 44 Alligevel formastede de sig til at drage op mod det øverste af Bjerglandet; men HERRENs Pagts Ark og Moses forlod ikke Lejren.
\par 45 Da steg Amalekiterne og Kana'anæerne, der boede der i Bjerglandet, ned og slog dem og adsplittede dem lige til Horma.

\chapter{15}

\par 1 HERREN talede fremdeles til Moses og sagde:
\par 2 Tal til Israelitterne og sig til dem: Når I kommer til det Land, jeg vil give eder at bo i,
\par 3 og I vil ofre HERREN et Ildoffer, Brændoffer eller Slagtoffer, af Hornkvæg eller Småkvæg for at indfri et Løfte eller af fri Drift eller i Anledning af eders Højtider for at berede HERREN en liflig Duft,
\par 4 så skal den, der bringer HERREN sin Offergave, som Afgrødeoffer bringe en Tiendedel Efa fint Hvedemel, rørt i en Fjerdedel Hin Olie
\par 5 desuden skal du som Drikoffer til hvert Lam ofre en Fjerdedel Hin Vin, hvad enten det er Brændoffer eller Slagtoffer.
\par 6 Men til en Væder skal du som Afgrødeoffer ofre to Tiendedele Efa fint Hvedemel, rørt i en Tredjedel Hin Olie;
\par 7 desuden skal du som Drikoffer frembære en Tredjedel Hin Vin til en liflig Duft for HERREN.
\par 8 Og når du ofrer en ung Tyr som Brændoffer eller Slagtoffer for at indfri et Løfte eller som Takoffer til HERREN,
\par 9 skal du foruden Tyren frembære som Afgrødeoffer tre Tiendedele Efa fint Hvedemel, rørt i en halv Hin Olie;
\par 10 desuden skal du som Drikoffer frembære en halv Hin Vin, et Ildoffer til en liflig Duft for HERREN.
\par 11 Således skal der gøres for hver enkelt Tyr, hver enkelt Væder eller hvert Lam eller Ged;
\par 12 således skal I gøre for hvert enkelt Dyr, så mange I nu ofrer.
\par 13 Enhver indfødt skal gøre disse Ting på denne Måde, når han vil bringe et Ildoffer til en liflig Duft for HERREN.
\par 14 Og når en fremmed bor hos eder, eller nogen i de kommende Tider bor iblandt eder, og han vil bringe et Ildoffer til en liflig Duft for HERREN, skal han gøre på samme Måde som I selv.
\par 15 Inden for Forsamlingen skal en og samme Anordning gælde for eder og den fremmede, der bor hos eder; det skal være eder en evig gyldig Anordning fra Slægt til Slægt: hvad der gælder for eder, skal også gælde for den fremmede for HERRENs Åsyn;
\par 16 samme Lov og Ret gælder for eder og den fremmede, der bor hos eder.
\par 17 HERREN talede fremdeles til Moses og sagde:
\par 18 Tal til Israelitterne og sig til dem: Når I kommer til det Land, jeg fører eder til,
\par 19 og spiser af Landets Brød, skal I yde HERREN en Offerydelse.
\par 20 Som Førstegrøde af eders Grovmel skal I yde en Kage som Offerydelse; på samme Måde som Offerydelsen af Tærskepladsen skal I yde den.
\par 21 Af Førstegrøden af eders Grovmel skal I give HERREN en Offerydelse, Slægt efter Slægt.
\par 22 Dersom I synder af Vanvare og undlader at udføre noget af alle de Bud, HERREN har kundgjort Moses,
\par 23 noget af alt det, HERREN har pålagt eder gennem Moses, fra den Dag HERREN udstedte sit Bud og frem i Tiden fra Slægt til Slægt,
\par 24 så skal hele Menigheden, hvis det sker af Vanvare uden Menighedens Vidende, ofre en ung Tyr som Brændoffer til en liflig duft for HERREN med det efter Lovbudene dertil hørende Afgrødeoffer og Drikoffer og desuden en Gedebuk som Syndoffer.
\par 25 Og Præsten skal skaffe hele Israelitternes Menighed Soning, og dermed opnår de Tilgivelse; thi det skete af Vanvare, og de bar bragt deres Offergave som et Ildoffer til HERREN og desuden deres Syndoffer for HERRENs Åsyn, for hvad de gjorde af Vanvare.
\par 26 Således får både hele Israelitternes Menighed og den fremmede, der bor hos dem, Tilgivelse; thi alt Folket har Del i den Synd, der bliver begået af Vanvare.
\par 27 Men hvis et enkelt Menneske synder af Vanvare, skal han bringe en årgammel Ged som Syndoffer.
\par 28 Og Præsten skal skaffe den, der synder af Vanvare, Soning for HERRENs Åsyn ved at udføre Soningen for ham, og således opnår han Tilgivelse.
\par 29 For den indfødte hos Israelitterne og den fremmede, der bor iblandt dem, for eder alle gælder en og samme Lov; når nogen synder af Vanvare.
\par 30 Men den, der handler med Forsæt, hvad enten han er indfødt eller fremmed, han håner Gud, og det Menneske skal udryddes af sit Folk.
\par 31 Thi han har ringeagtet HERRENs Ord og brudt hans Bud; det Menneske skal udryddes, hans Misgerning kommer over ham.
\par 32 Medens Israelitterne opholdt sig i Ørkenen, traf de en Mand, som sankede Brænde på en Sabbat.
\par 33 De, der traf ham i Færd med at sanke Brænde, bragte ham til Moses, Aron og hele Menigheden,
\par 34 og de satte ham i Varetægt, da der ikke forelå nogen bestemt Kendelse for, hvad der skulde gøres ved ham.
\par 35 Da sagde HERREN til Moses: Den Mand skal lide Døden; hele Menigheden skal stene ham uden for Lejren!
\par 36 Hele Menigheden førte ham da uden for Lejren og stenede ham til Døde, som HERREN havde pålagt Moses.
\par 37 HERREN talede fremdeles til Moses og sagde:
\par 38 Tal til Israelitterne og sig til dem, at de Slægt efter Slægt skal sætte Kvaster på Fligene af deres Klæder, og at de på hver enkelt Kvast skal sætte en violet Purpursnor.
\par 39 Det skal tjene eder til Tegn, så at I, hver Gang I ser dem, skal komme alle HERRENs Bud i Hu og handle efter dem og ikke lade eder vildlede af eders Hjerter eller Øjne, af hvilke I lader eder forlede til Bolen
\par 40 for at I kan komme alle mine Bud i Hu og handle efter dem og blive hellige for eders Gud.
\par 41 Jeg er HERREN eders Gud, som førte eder ud af Ægypten for at være eders Gud. Jeg er HERREN eders Gud!

\chapter{16}

\par 1 Men Kora, en Søn af Jizhar, en Søn af Levis Søn Kehat, og Datan og Abiram, Sønner af Ejiab, en Søn af Rubens Søn Pallu, gjorde Oprør.
\par 2 De gjorde Oprør mod Moses sammen med 250 israelitiske Mænd, Øverster for Menigheden, udvalgte i Folkeforsamlingen, ansete Mænd.
\par 3 Og de samlede sig og trådte op imod Moses og Aron og sagde til dem: "Lad det nu være nok, thi hele Menigheden er hellig, hver og en, og HERREN er i dens Midte; hvorfor vil I da ophøje eder over HERRENs Forsamling?"
\par 4 Da Moses hørte det, faldt han på sit Ansigt.
\par 5 Derpå talte han til Kora og alle hans Tilhængere og sagde: "Vent til i Morgen, så vil HERREN give til kende, hvem der tilhører ham, og hvem der er hellig, så at han vil give ham Adgang til sig; den, han udvælger, vil han give Adgang til sig.
\par 6 Således skal I gøre: Skaf eder Pander, du Kora og alle dine Tilhængere,
\par 7 og læg så i Morgen Gløder på og kom Røgelse på for HERRENs Åsyn, så skal den, HERREN udvælger, være den, som er hellig; lad det nu være nok, I Levisønner!"
\par 8 Fremdeles sagde Moses til Kora: "Hør nu, I Levisønner!
\par 9 Er det eder ikke nok, at Israels Gud har udskilt eder af Israels Menighed og givet eder Adgang til sig for at udføre Arbejdet ved HERRENs Bolig og stå til Tjeneste for Menigheden?
\par 10 Han har givet dig og med dig alle dine Brødre, Levis Sønner, Adgang til sig og nu attrår I også Præsteværdigheden!
\par 11 Derfor, du og alle dine Tilhængere, som har rottet eder sammen mod HERREN, hvad er Aron, at I vil knurre mod ham!"
\par 12 Da sendte Moses Bud efter Datan og Abiram, Eliabs Sønner, men de sagde: "Vi kommer ikke!
\par 13 Er det ikke nok, at du har ført os bort fra et Land, der flyder med Mælk og Honning, for at lade os dø i Ørkenen, siden du oven i Købet vil opkaste dig til Herre over os!
\par 14 Du har sandelig ikke ført os til et Land, der flyder med Mælk og Honning, eller givet os Marker og Vinbjerge! Tror du, du kan stikke disse Mænd Blår i Øjnene? Vi kommer ikke!"
\par 15 Da harmedes Moses højlig og sagde til HERREN: "Vend dig ikke til deres Offergave! Ikke så meget som et Æsel har jeg frataget dem, ej heller har jeg gjort en eneste af dem noget ondt!"
\par 16 Og Moses sagde til Kora: "I Morgen skal du og alle dine Tilhængere komme frem for HERRENs Åsyn sammen med Aron;
\par 17 og enhver af eder skal tage sin Pande, lægge Gløder på og komme Røgelse på og frembære sin Pande for HERRENs Åsyn, 25O Pander, du selv og Aron skal også tage hver sin Pande!"
\par 18 Da tog hver sin Pande, lagde Gløder på og kom Røgelse på, og så stillede de sig ved indgangen til Åbenbaringsteltet sammen med Moses og Aron;
\par 19 og Kora kaldte hele Menigheden sammen imod dem ved Indgangen til Åbenbaringsteltet. Da kom HERRENs Herlighed til Syne for hele Menigheden,
\par 20 og HERREN talede til Moses og Aron og sagde:
\par 21 "Skil eder ud fra denne Menighed, så vil jeg i et Nu tilintetgøre den!"
\par 22 Men de faldt på deres Ansigt og sagde: "O Gud, du Gud over alt Køds Ånder, vil du vredes på hele Menigheden, fordi en enkelt Mand synder?"
\par 23 Da talede HERREN til Moses og sagde:
\par 24 "Tal til Menigheden og sig: Fjern eder fra Pladsen omkring Koras, Datans og Abirams Bolig!"
\par 25 Moses gik nu hen til Datan og Abiram, fulgt af Israels Ældste,
\par 26 og han talte til Menigheden og sagde: "Træk eder tilbage fra disse ugudelige Mænds Telte og rør ikke ved noget af, hvad der tilhører dem, for at I ikke skal rives bort for alle deres Synders Skyld!"
\par 27 Da fjernede de sig fra Pladsen om Koras, Datans og Abirams Bolig, og Datan og Abiram kom ud og stillede sig ved indgangen til deres Telte med deres Hustruer og Børn, store og små.
\par 28 Og Moses sagde: "Derpå skal I kende, at HERREN har sendt mig for at gøre alle disse Gerninger, og at jeg ikke handler i Egenrådighed:
\par 29 Dersom disse dør på vanlig menneskelig Vis, og der ikke rammer dem andet, end hvad der rammer alle andre, så har HERREN ikke sendt mig;
\par 30 men hvis HERREN lader noget uhørt ske, så Jorden spiler sit Gab op og sluger dem med alt, hvad der tilhører dem, så de farer levende ned i Dødsriget, da skal I derpå kende, at disse Mænd har hånet HERREN!"
\par 31 Og straks, da han havde talt alle disse Ord, åbnede Jorden sig under dem,
\par 32 og Jorden lukkede sit Gab op og slugte dem og deres Boliger og alle Mennesker, der tilhørte Kora, og alt, hvad de ejede;
\par 33 og de for levende ned i Dødsriget med alt, hvad der tilhørte dem, og Jorden lukkede sig over dem, og de blev udryddet af Forsamlingen.
\par 34 Men hele Israel, der stod omkring dem, flygtede ved deres Skrig, thi de sagde: "Blot ikke Jorden skal opsluge os!"
\par 35 Og Ild for ud fra HERREN og fortærede de 250 Mænd, der frembar Røgelse.
\par 36 Da talede HERREN til Moses og sagde:
\par 37 "Sig til Eleazar, Præsten Arons Søn, at han skal tage Panderne ud af Branden og strø Gløderne ud noget derfra; thi hellige
\par 38 er de Pander, der tilhørte disse Mænd, som begik en Synd, der kostede dem Livet. De skal udhamre dem til Plader til Overtræk på Alteret, thi de frembar dem for HERRENs Åsyn, og derfor er de hellige; de skal nu tjene Israelitterne til Tegn."
\par 39 Da tog Præsten Eleazar Kobberpaladerne, som de opbrændte Mænd havde frembåret, og hamrede dem ud til. Overtræk på Alteret
\par 40 som et Mindetegn for Israelitterne om, at ingen Lægmand, ingen, som ikke hører til Arons Efterkommere, må træde frem for at ofre Røgelse for HERRENs Åsyn, for at det ikke skal gå ham som Kora og hans Tilhængere, således som HERREN havde sagt ham ved Moses.
\par 41 Men Dagen efter knurrede hele Israels Menighed mod Moses og Aron og sagde: "Det er eder, der har, dræbt HERRENs Folk!"
\par 42 Men da Menigheden samlede sig mod Moses og Aron, vendte de sig mod Åbenbaringsteltet, og se, Skyen dækkede det, og HERRENs Herlighed kom til Syne.
\par 43 Da trådte Moses og Aron hen foran Åbenbaringsteltet,
\par 44 og HERREN talede til Moses og Aron og sagde:
\par 45 "Fjern eder fra denne Menighed, så vil jeg i et Nu tilintetgøre dem!" Da faldt de på deres Ansigt,
\par 46 og Moses sagde til Aron: "Tag din Pande, læg Gløder fra Alteret på og kom Røgelse på og skynd dig så hen til Menigheden og skaf den Soning, thi Vreden er brudt frem fra HERREN, Plagen har allerede taget fat!"
\par 47 Da tog Aron det, således som Moses havde sagt, og løb midt ind i Forsamlingen. Og se, Plagen havde allerede taget fat blandt Folket, men han kom Røgelsen på og skaffede Folket Soning.
\par 48 Og som han stod der midt imellem døde og levende, hørte Plagen op.
\par 49 Men de, der omkom ved Plagen, udgjorde 14700 Mennesker foruden dem, der omkom for Koras Skyld.
\par 50 Så vendte Aron tilbage til Moses ved Indgangen til Åbenbaringsteltet, efter at Plagen var ophørt.

\chapter{17}

\par 1 Og HERREN talede til Moses og sagde:
\par 2 "Sig til Israelitterne, at Øversterne for Fædrenehusene skal give dig en Stav for hvert Fædrenehus, tolv Stave i alt, og skriv så hver enkelts Navn på hans Stav
\par 3 og skriv Arons Navn på Levis Stav, thi hvert Overhoved for Fædrenehusene skal have een Stav.
\par 4 Læg dem så ind i Åbenbariogsteltet foran Vidnesbyrdet, der, hvor jeg åbenbarer mig for dig.
\par 5 Den Mand, jeg udvælger, hans Stav skal da grønnes; således vil jeg bringe Israelitternes Knurren imod eder til Tavshed, så jeg kan blive fri for den."
\par 6 Moses sagde nu dette til Israelitterne, og alle deres Øverster gav ham en Stav, så der blev en Stav for hver Øverste efter deres Fædrenehuse, tolv Stave i alt, og Arons Stav var imellem Stavene.
\par 7 Derpå lagde Moses Stavene hen foran HERRENs Åsyn i Vidnesbyrdets Telt.
\par 8 Da Moses næste Dag kom ind i Vidnesbyrdets Telt, se, da var Arons Stav, Staven for Levis Hus, grønnedes; den havde sat Skud, var kommet i Blomst og bar modne Mandler.
\par 9 Da tog Moses Stavene bort fra HERRENs Åsyn og bar dem ud til Israelitterne, og de så på dem og tog hver sin Stav.
\par 10 Men HERREN sagde til Moses: "Læg Arons Stav tilbage foran Vidnesbyrdet, for at den kan opbevares til Tegn for de genstridige, og gør Ende på deres Knurren, så jeg kan blive fri for den, at de ikke skal dø!"
\par 11 Og Moses gjorde som HERREN havde pålagt ham; således gjorde han.
\par 12 Men Israelitterne sagde til Moses: "Se, vi omkommer, det er ude med os, det er ude med os alle sammen!
\par 13 Enhver, der kommer HERRENs Bolig nær, dør. Skal vi da virkelig omkomme alle sammen?"

\chapter{18}

\par 1 HERREN sagde til Aron: Du tillige med dine Sønner og dit Fædrenehus skal have Ansvaret for Helligdommen, og du tillige med dine Sønner skal have Ansvaret for eders Præstetjeneste.
\par 2 Men også dine Brødre, Levis Stamme, din Fædrenestamme, skal du lade træde frem sammen med dig, og de skal holde sig til dig og gå dig til Hånde, når du tillige med dine Sønner gør Tjeneste foran Vidnesbyrdets Telt;
\par 3 og de skal tage Vare på, hvad du har at varetage, og på alt, hvad der er at varetage ved Teltet, men de må ikke komme de hellige Ting eller Alteret nær, at ikke både de og I skal dø..
\par 4 De skal holde sig til dig og tage Vare på, hvad der er at varetage ved Åbenbaringsteltet, alt Arbejdet derved, men ingen Lægmand må der komme eder nær.
\par 5 Men I skal tage Vare på, hvad der er at varetage ved Helligdommen og Alteret, for at der ikke atter skal komme Vrede over Israelitterne.
\par 6 Se, jeg har selv udtaget eders Brødre Leviterne af Israelitternes Midte som en Gave til eder, de er givet HERREN til at udføre Arbejdet ved Åbenbaringsteltet.
\par 7 Men du tillige med dine Sønner skal tage Vare på eders Præstegerning i alt, hvad der angår Alteret og det, der er inden for Forhænget, og udføre Arbejdet derved. Som en Gave skænker jeg eder Præstedømmet; men enhver Lægmand, der trænger sig ind deri, skal lide Døden.
\par 8 HERREN talede fremdeles til Aron: Se, jeg giver dig, hvad der skal lægges til Side af mine Offerydelser; alle Israelitternes Helliggaver giver jeg dig og dine Sønner som eders Del, en evig gyldig Rettighed.
\par 9 Følgende skal tilfalde dig af det højhellige, fraregnet hvad der opbrændes: Alle deres Offergaver, der falder ind under Afgrødeofre, Syndofre og Skyldofre, som de bringer mig til Erstatning; som højhelligt skal dette tilfalde dig og dine Sønner.
\par 10 På et højhelligt Sted skal du spise det, og alle af Mandkøn må spise deraf; det skal være dig helligt.
\par 11 Fremdeles skal følgende tilfalde dig som Offerydelse af deres Gaver: Alle Gaver fra Israelitterne, hvormed der udføres Svingning, giver jeg dig tillige med dine Sønner og Døtre som en evig gyldig Rettighed; enhver, som er ren i dit Hus, må spise deraf.
\par 12 Alt det bedste af Olien, Mosten og Kornet, Førstegrøden deraf, som de giver HERREN, giver jeg dig.
\par 13 De første Frugter af alt, hvad der gror i deres Land, som de bringer HERREN, skal tilfalde dig; enhver, som er ren i dit Hus, må spise deraf.
\par 14 Alt, hvad der lægges Band på i Israel, skal tilfalde dig.
\par 15 Af alt Kød, som de bringer til HERREN, såvel af Mennesker som af Dyr, skal alt, hvad der åbner Moders Liv, tilfalde dig; dog skal du lade de førstefødte Mennesker udløse, og ligeledes skal du lade de førstefødte urene Dyr udløse.
\par 16 Hine skal du lade udløse, når de er en Måned gamle eller derover, med en Vurderingssum af fem Sekel efter hellig Vægt, tyve Gera på en Sekel.
\par 17 Men de førstefødte Stykker Hornkvæg, Lam eller Geder må du ikke lade udløse; de er hellige, deres Blod skal du sprænge på Alteret, og Fedtet skal du bringe som Røgoffer, et Ildoffer til en liflig Duft for HERREN;
\par 18 men Kødet tilfalder dig; ligesom Svingningsbrystet og højre Kølle tilfalder det dig.
\par 19 Al Offerydelse af Helliggaver, som Israelitterne yder HERREN, giver jeg dig tillige med dine Sønner og Døtre som en evig gyldig Rettighed; det skal være en evig gyldig Saltpagt for HERRENs Åsyn for dig tillige med dine Efterkommere.
\par 20 HERREN sagde fremdeles til Aron: Du skal ingen Arvelod have i deres Land, og der skal ikke tilfalde dig nogen Lod iblandt dem; jeg selv er din Arvelod og Del blandt Israelitterne.
\par 21 Men se, Levisønnerne giver jeg al Tiende i Israel som Arvelod til Løn for det Arbejde, de udfører ved Åbenbaringsteltet.
\par 22 Israelitterne må herefter ikke komme Åbenbaringsteltet nær, for at de ikke skal pådrage sig Synd og dø;
\par 23 kun Leviterne må udføre Arbejdet ved Åbenbaringsteltet, og de skal have Ansvaret derfor. Det skal være eder en evig gyldig Anordning fra Slægt til Slægt. Men nogen Arvelod skal de ikke have blandt Israelitterne;
\par 24 thi Tienden, Israelitterne yder HERREN som Offerydelse, giver jeg Leviterne til Arvelod. Derfor sagde jeg dem, at de ikke skal have nogen Arvelod blandt Israelitterne.
\par 25 HERREN talede fremdeles til Moses og sagde:
\par 26 Tal til Leviterne og sig til dem: Når I af Israelitterne modtager Tienden, som jeg har givet eder som den Arvelod, I skal have af dem, så skal I yde HERREN en Offerydelse deraf, Tiende af Tienden;
\par 27 og denne eders Offerydelse skal ligestilles med Offerydelsen af Kornet fra Tærskepladsen og Overfloden fra Vinpersen.
\par 28 Således skal også I yde HERREN en Offerydelse af al den Tiende, I modtager af Israelitterne, og denne HERRENs Offerydelse skal I give Præsten Aron.
\par 29 Af alle de Gaver, I modtager, skal I yde HERRENs Offerydelse, af alt det bedste deraf, som hans Helliggave.
\par 30 Og sig til dem: Når I yder det bedste deraf, skal det ligestilles med Offerydelsen af, hvad der kommer fra Tærskepladsen og Vinpersen.
\par 31 I må spise det hvor som helst sammen med eders Familie, thi det er eders Løn for eders Arbejde ved Åbenbaringsteltet.
\par 32 Når I blot yder det bedste deraf, skal I ikke for den Sags Skyld pådrage eder Synd og ikke vanhellige Israelitternes Helliggaver og udsætte eder for at dø.

\chapter{19}

\par 1 HERREN talede fremdeles til Moses og Aron og sagde:
\par 2 Dette er det Lovbud, HERREN har kundgjort: Sig til Israelitterne, at de skal skaffe dig en rød, lydefri Kvie, der er uden Fejl og ikke har båret Åg.
\par 3 Den skal I overgive til Præsten Eleazar, og man skal føre den uden for Lejren og slagte den for hans Åsyn.
\par 4 Så skal Præsten Eleazar tage noget af dens Blod på sin Finger og stænke det syv Gange i Retning af Åbenbaringsteltets Forside.
\par 5 Derpå skal Kvien brændes i hans Påsyn; dens Hud, Kød og Blod tillige med Skarnet skal opbrændes.
\par 6 Derefter skal Præsten tage Cedertræ, Ysop og karmoisinrødt Uld og kaste det på Bålet, hvor Kvien brænder.
\par 7 Så skal Præsten tvætte sine Klæder og bade sit Legeme i Vand og derefter vende tilbage til Lejren. Men Præsten skal være uren til Aften.
\par 8 Ligeledes skal den Mand, der brænder Kvien, tvætte sine Klæder med Vand og bade sit Legeme i Vand og være uren til Aften.
\par 9 Og en Mand, der er ren, skal opsamle Kviens Aske og lægge den hen på et rent Sted uden for Lejren, hvor den skal opbevares for Israelitternes Menighed for at bruges til Renselsesvand. Det er et Syndoffer.
\par 10 Og den, der opsamler Kviens Aske, skal tvætte sine Klæder og være uren til Aften. For Israelitterne og den fremmede, der bor iblandt dem, skal dette være en evig gyldig Anordning:
\par 11 Den, der rører ved en død, ved noget som helst Lig, skal være uren i syv Dage.
\par 12 Han skal lade sig rense for Synd med Asken på den tredje og syvende Dag, så bliver han ren; men renser han sig ikke på den tredje og syvende Dag, bliver han ikke ren.
\par 13 Enhver, der rører ved en død, et Lig, og ikke lader sig rense for Synd, besmitter HERRENs Bolig, og det Menneske skal udryddes af Israel, fordi der ikke er stænket Renselsesvand på ham; han er uren, hans Urenhed klæber endnu ved ham.
\par 14 Således er Loven: Når et Menneske dør i et Telt,, bliver enhver, der træder ind i Teltet, og enhver, der er i Teltet, uren i syv Dage;
\par 15 og ethvert åbent Kar, et, der ikke er bundet noget over, bliver urent.
\par 16 Ligeledes bliver enhver, der på åben Mark rører ved en, der er dræbt med Sværd, eller ved en, der er død, eller ved Menneskeknogler eller en Grav, uren i syv Dage.
\par 17 For sådanne urene skal man da tage noget af Asken af det brændte Syndoffer og hælde rindende Vand derover i en Skål.
\par 18 Derpå skal en Mand, der er ren, tage en Ysopstængel, dyppe den i Vandet og stænke det på Teltet og på alle de Ting og Mennesker, der har været deri, og på den, der har rørt ved Menneskeknoglerne, den ihjelslagne, den døde eller Graven.
\par 19 Således skal den rene bestænke den urene på den tredje og syvende Dag og borttage hans Synd på den syvende Dag. Derefter skal han tvætte sine Klæder og bade sig i Vand, så er han ren, når det bliver aften.
\par 20 Men den, som bliver uren og ikke lader sig rense for Synd, han skal udryddes af Forsamlingen; thi han har besmittet HERRENs Helligdom, der er ikke stænket Renselsesvand på ham, han er uren.
\par 21 Det skal være eder en evig gyldig Anordning. Den, der stænker Renselsesvandet, skal tvætte sine Klæder, og den, der rører ved Renselsesvandet, skal være uren til Aften.
\par 22 Alt, hvad den urene rører ved skal være urent, og enhver, der rører ved ham, skal være uren til Aften.

\chapter{20}

\par 1 Derpå nåede Israelitterne, hele Menigheden, til Zins Ørken i den første Måned, og Folket slog sig ned i Kadesj. Der døde Mirjam, og der blev hun jordet.
\par 2 Men der var ikke Vand til Menigheden; derfor samlede de sig mod Moses og Aron,
\par 3 og Folket kivedes med Moses og sagde: "Gid vi var omkommet, dengang vore Brødre omkom for HERRENs Åsyn!
\par 4 Hvorfor førte I HERRENs Forsamling ind i denne Ørken, når vi skal dø her, både vi og vort Kvæg?
\par 5 Og hvorfor førte I os ud af Ægypten, når I vilde have os hen til dette skrækkelige Sted, hvor der hverken er Korn eller Figener, Vintræer eller Granatæbler, ej heller Vand at drikke?"
\par 6 Men Moses og Aron begav sig fra Forsamlingen hen til Åbenbaringsteltets indgang og faldt på deres Ansigt. Da viste HERRENs Herlighed sig for dem,
\par 7 og HERREN talede til Moses og sagde:
\par 8 "Tag Staven og kald så tillige med din Broder Aron Menigheden sammen og tal til Klippen i deres Påsyn, så giver den Vand; lad Vand strømme frem af Klippen til dem og skaf Menigheden og dens Kvæg noget at drikke!"
\par 9 Da tog Moses Staven fra dens Plads foran HERRENs Åsyn, som han havde pålagt ham;
\par 10 og Moses og Aron kaldte Forsamlingen sammen foran Klippen, og han sagde til dem: "Hør nu, I genstridige! Mon vi formår at få Vand til at strømme frem til eder af denne Klippe?"
\par 11 Og Moses løftede sin Hånd og slog to Gange på Klippen med sin Stav, og der strømmede Vand frem i Mængde, så at Menigheden og dens Kvæg kunde drikke.
\par 12 Men HERREN sagde til Moses og Aron: "Fordi I ikke troede på mig og helligede mig for Israelitternes Øjne, skal I ikke komme til at føre denne Forsamling ind i det Land, jeg vil give dem!"
\par 13 Dette er Meribas Vand, hvor Israelitterne kivedes med HERREN, og hvor han åbenbarede sin Hellighed på dem.
\par 14 Fra Kadesj sendte Moses Sendebud til kongen af Edom med det Bud: "Din Broder Israel lader sige: Du kender jo alle de Besværligheder, som er vederfaret os,
\par 15 hvorledes vore Fædre drog ned til Ægypten, hvorledes vi boede der i lange Tider, og hvorledes Ægypterne mishandlede os og vore Fædre;
\par 16 da råbte vi til HERREN, og han hørte vor Røst og sendte en Engel og førte os ud af Ægypten. Se, nu er vi i Byen Kadesj ved Grænsen af dine Landemærker.
\par 17 Lad os få Lov at vandre igennem dit Land. Vi vil hverken drage hen over Marker eller Vinhaver eller drikke Vandet i Brøndene; vi vil gå ad Kongevejen, vi vil hverken bøje af til højre eller venstre, før vi er nået igennem dit Land!"
\par 18 Men Edom svarede ham: "Du må ikke vandre igennem mit Land, ellers drager jeg imod dig med Sværd i Hånd!"
\par 19 Da sagde Israelitterne til ham: "Vi vil følge den slagne Landevej, og der som jeg eller mit Kvæg drikker af dine Vandingssteder, vil jeg betale derfor det er da ikke noget at være bange for, jeg vil kun vandre derigennem til Fods!"
\par 20 Men han svarede: "Du må ikke drage her igennem!" Og Edom rykkede imod ham med mange Krigere og stærkt rustet.
\par 21 Da Edom således formente Israel at drage igennem sine Landemærker, bøjede Israel af og drog udenom.
\par 22 Derpå brød Israelitterne, hele Menigheden, op fra Kadesj og kom til Bjerget Hor.
\par 23 Og HERREN talede til Moses og Aron ved Bjerget Hor ved Grænsen til Edoms Land og sagde:
\par 24 "Aron skal nu samles til sin Slægt, thi han skal ikke komme ind i det Land, jeg vil give Israelitterne, fordi I var genstridige mod mit Bud ved Meribas Vand.
\par 25 Tag Aron og hans Søn Eleazar og før dem op på Bjerget Hor;
\par 26 affør så Aron hans Klæder og ifør hans Søn Eleazar dem, thi der skal Aron tages bort og dø!"
\par 27 Da gjorde Moses som HERREN bød, og de gik op på Bjerget Hor i hele Menighedens Påsyn;
\par 28 og efter at Moses havde afført Aron hans Klæder og iført hans Søn Eleazar dem, døde Aron deroppe på Bjergets Top. Men Moses og Eleazar steg ned fra Bjerget,
\par 29 og da hele Menigheden skønnede, at Aron var død, græd de over Aron i tredive Dage, hele Israels Hus.

\chapter{21}

\par 1 Men da Kana'anæeren, Kongen af Arad, der boede i Sydlandet, hørte, at Israel rykkede frem ad Atarimvejen, angreb han Israel og tog nogle af dem til Fange.
\par 2 Da aflagde Israel et Løfte til Herren og sagde: "Hvis du giver dette Folk i min Hånd, vil jeg lægge Band på deres Byer!"
\par 3 Og HERREN hørte Israels Røst og gav Kana'anæerne i deres Hånd, hvorefter de lagde Band på dem og deres Byer. Derfor gav man Stedet Navnet Horma.
\par 4 Så brød de op fra Bjerget Hor i Retning af det røde Hav for at komme uden om Edoms Land. Undervejs blev Folket utålmodigt
\par 5 og talte mod Gud og Moses og sagde: "Hvorfor førte I os ud af Ægypten, når vi skal dø i Ørkenen? Her er jo hverken Brød eller Vand, og vi er lede ved denne usle Føde."
\par 6 Da sendte HERREN Giftslanger blandt Folket, og de bed Folket så en Mængde af Israel døde.
\par 7 Så kom Folket til Moses og sagde: "Vi har syndet ved at tale imod HERREN og dig; gå i Forbøn hos HERREN, at han tager Slangerne fra os!" Og Moses gik i Forbøn for Folket.
\par 8 Da sagde HERREN til Moses: "Lav dig en Slange og sæt den på en Stang, så skal enhver slangebidt, der ser hen på den, leve!"
\par 9 Da lavede Moses en Kobberslange og satte den på en Stang; og enhver, der så hen på Kobberslangen, når en Slange havde bidt ham, beholdt Livet.
\par 10 Så brød Israelitterne op og slog Lejr i Obot;
\par 11 og de brød op fra Obot og slog Lejr i tjje Håbarim i Ørkenen østen for Moab.
\par 12 Derfra brød de op og slog Lejr i Zereddalen.
\par 13 Derfra brød de op og slog Lejr i Ørkenen hinsides Arnon, som udspringer på Amoriternes Område, thi Arnon var Moabs Grænse mod Amoriterne.
\par 14 Derfor hedder det i Bogen om HERRENs Krige: Vaheb i Sufa og Dalene, Arnon og Dalenes Skrænt,
\par 15 som strækker sig til Ars Sæde og læner sig til Moabs Grænse.
\par 16 Derfra brød de op til Beer; det er det Be'er, HERREN havde for Øje, da han sagde til Moses: "Kald Folket sammen, så vil jeg give dem Vand!"
\par 17 Da sang Israelitterne denne Sang: Brønd, væld frem! Syng til dens Pris!
\par 18 Brønd, som Høvdinger grov, som Folkets ædle bored med Herskerspir og Stave! Fra Ørkenen brød de op til Mattana;
\par 19 fra Mattana til Nahaliel; fra Nahaliel til Bamot;
\par 20 fra Bamot til halen på Moabs Højslette, til Pisgas Top, som rager op over Ørkenen.
\par 21 Israel sendte nu Sendebud til Amoriterkongen Sihon og lod sige:
\par 22 "Lad mig få Lov at drage igennem dit Land! Vi vil ikke dreje ind på Marker eller i Vinhaver, vi vil ikke drikke Vand af Brøndene, vi vil følge Kongevejen, indtil vi er nået igennem dit Land!
\par 23 Men Sihon tillod ikke Israel at drage igennem sit Land; derimod samlede han alt sit krigsfolk og rykkede ud i Ørkenen imod Israel, og da han nåede Jaza, angreb han Israel.
\par 24 Men Israel slog ham med Sværdet og underlagde sig hans Land fra Arnon til Jabbok, til Ammoniternes Land; thi Jazer ligger ved Ammoniternes Grænse;
\par 25 og Israel indtog alle disse Byer, og Israel bosatte sig i alle Amoriternes Byer, i Hesjbon og alle tilhørende Småbyer.
\par 26 Thi Hesjbon var Amoriterkongen Sihons By; han havde angrebet Moabs forrige konge og frataget ham hele hans Land indtil Arnon.
\par 27 Derfor synger Skjaldene: Kom hid til Hesjbon, lad den blive bygget og grundfæstet, Sihons By!
\par 28 Thi Ild for ud fra Hesjbon, Ildslue fra Sihons Stad, den fortærede Moabs Byer, opåd Arnons Højder.
\par 29 Ve dig, Moab! Det er ude med dig, Kemosjs Folk! Han gjorde sine Sønner til Flygtninge og sine Døtre til Krigsfanger for Sihon, Amoriternes Konge.
\par 30 Og vi skød dem ned, Hesjbon er tabt indtil Dibon; vi lagde dem øde til Nofa, som ligger ved Medeba.
\par 31 Så bosatte Israel sig i Amoriternes Land.
\par 32 Derpå sendte Moses Spejdere til Jazer; og de erobrede det tillige med tilhørende Småbyer, og han drev de der boende Amoriter bort.
\par 33 Derpå vendte de om og drog ad Basan til. Da rykkede Og, Kongen af Basan, ud imod dem med alle sine Krigere og angreb dem ved Edrei.
\par 34 Men HERREN sagde til Moses: "Frygt ikke for ham! Thi jeg giver ham og alle hans krigere og hans Land i din Hånd, så at du kan handle med ham, som du handlede med Amoriterkongen Sihon, der boede i Hesjbon."
\par 35 Da slog de ham og hans Sønner og alle hans Krigere, så at ikke en eneste af dem undslap, og derpå underlagde de sig hans Land.

\chapter{22}

\par 1 Derefter brød Israelitterne op derfra og slog Lejer på Moabs Sletter hinsides Jordan over for Jeriko.
\par 2 Da Balak, Zippors Søn, så alt, hvad Israel havde gjort ved Amoriterne,
\par 3 grebes Moab af Rædsel for Folket, fordi det var så talrigt, og Moab gruede for Israelitterne.
\par 4 Da sagde Moab til Midjaniternes Ældste: "Nu vil denne Menneskemasse opæde alt, hvad der er rundt omkring os, som Okserne opæder Græsset på Marken!" På den Tid var Balak, Zippors Søn, Konge over Moab.
\par 5 Han sendte nu Sendebud til Bileam, Beors Søn, i Petor, der ligger ved Floden, til Ammoniternes Land, og bad ham komme til sig, idet han lod sige: "Se, et Folk er udvandret fra Ægypten; se, det har oversvømmet Landet og slået sig ned lige over for mig.
\par 6 Kom nu og forband mig det Folk, thi det er mig for mægtigt: måske jeg da kan slå det og jage det ud af Landet. Thi jeg ved, at den, du velsigner, er velsignet, og den, du forbander, forbandet!"
\par 7 Da gav Moabs og Midjans Ældste sig på Vej, forsynede med Spåmandsløn, og da de kom til Bileam, overbragte de ham Balaks Ord.
\par 8 Han sagde til dem: "Bliv her Natten over, så skal jeg give eder Svar, efter som HERREN vil tale til mig!" Moabs Høvdinger blev da hos Bileam.
\par 9 Men Gud kom til Bileam og spurgte: "Hvem er de Mænd, som er hos dig?"
\par 10 Men Bileam svarede Gud: "Zippors Søn, Kong Balak af Moab, har sendt mig det Bud:
\par 11 Se, et Folk er udvandret fra Ægypten og har oversvømmet Landet! Kom nu og forband mig det, måske jeg da kan overvinde det og jage det bort!"
\par 12 Men Gud sagde til Bileam: "Du må ikke gå med dem, du må ikke forbande det Folk, thi det er velsignet!"
\par 13 Næste Morgen stod Bileam op og sagde til Balaks Høvdinger: "Vend tilbage til eders Land, thi HERREN vægrer sig ved at give mig Tilladelse til at følge med eder!"
\par 14 Da brød Moabs Høvdinger op, og de kom til Balak og meldte: "Bileam vægrede sig ved at følge med os!"
\par 15 Men Balak sendte på ny Høvdinger af Sted, flere og mere ansete end de forrige;
\par 16 og de kom til Bileam og sagde til ham: "Således siger Balak, Zippors Søn: Undslå dig ikke for at komme til mig!
\par 17 Jeg vil lønne dig rigeligt og gøre alt, hvad du kræver af mig. Kom nu og forband mig det Folk!"
\par 18 Men Bileam svarede Balaks Folk: "Om Balak så giver mig alt det Sølv og Guld, han har i sit Hus, formår jeg dog hverken at gøre lidt eller meget imod HERREN min Guds Befaling;
\par 19 bliv derfor også I her Natten over, for at jeg kan få at vide, hvad HERREN yderligere vil tale til mig!"
\par 20 Da kom Gud om Natten til Bileam og sagde til ham: "Er disse Mænd kommet til dig for at hente dig, så følg med dem; men du må ikke gøre andet, end hvad jeg siger dig!"
\par 21 Så stod Bileam op næste Morgen og sadlede sit Æsel og fulgte med Moabs Høvdinger.
\par 22 Men Guds Vrede blussede op, fordi han fulgte med, og HERRENs Engel stillede sig på Vejen for at stå ham imod, da han kom ridende på sit Æsel fulgt af sine to Tjenere.
\par 23 Da nu Æselet så HERRENs Engel stå på Vejen med draget Sværd i Hånden, veg det ud fra Vejen og gik ind på Marken; men Bileam slog Æselet for at tvinge det tilbage på Vejen.
\par 24 Da stillede HERRENs Engel sig i Hulvejen mellem Vingårdene, hvor der var Mure på begge Sider;
\par 25 og da Æselet så HERRENs Engel, trykkede det sig op til Muren, så det trykkede Bileams Fod op mod Muren, og han gav sig atter til at slå det.
\par 26 HERRENs Engel gik nu længere frem og stillede sig i en Snævring, hvor det ikke var muligt at komme til Siden, hverken til højre eller venstre.
\par 27 Da Æselet så HERRENs Engel, lagde det sig ned med Bileam. Da blussede Bileams Vrede op, og han gav sig til at slå Æselet med Stokken.
\par 28 Men HERREN åbnede Æselets Mund, og det sagde til Bileam: "Hvad har jeg gjort dig, siden du nu har slået mig tre Gange?"
\par 29 Bileam svarede Æselet: "Du har drillet mig; havde jeg haft et Sværd i Hånden, havde jeg slået dig ihjel!"
\par 30 Men Æselet sagde til Bileam: "Er jeg ikke dit eget Æsel, som du har redet al din Tid indtil i Dag? Har jeg ellers haft for Vane at bære mig således ad over for dig?" Han svarede: "Nej!"
\par 31 Da åbnede HERREN Bileams Øjne, og han så HERRENs Engel stå på Vejen med draget Sværd i Hånden; og han bøjede sig og kastede sig ned på sit Ansigt.
\par 32 Men HERRENs Engel sagde til ham: "Hvorfor slog du dit Æsel de tre Gange? Se, jeg er gået ud for at stå dig imod, thi du handlede overilet ved at rejse imod min Vilje.
\par 33 Æselet så mig og veg tre Gange til Side for mig; og var det ikke veget til Side for mig, havde jeg slået dig ihjel, men skånet dets Liv!"
\par 34 Da sagde Bileam til HERRENs Engel: "Jeg har syndet, jeg vidste jo ikke, at det var dig, der trådte i Vejen for mig. Men hvis det er dig imod, vil jeg atter vende tilbage."
\par 35 HERRENs Engel sagde til Bileam: "Følg blot med disse Mænd, men du må kun sige de Ord, jeg siger dig!" Så fulgte Bileam med Balaks Høvdinger.
\par 36 Da Balak nu hørte, at Bileam var undervejs, gik han ham i Møde til Ar Moab ved den Grænse, Arnon danner, den yderste Grænse.
\par 37 Og Balak sagde til Bileam: "Sendte jeg dig ikke Bud og bad dig komme? Hvorfor kom du da ikke til mig? Skulde jeg virkelig være ude af Stand til at lønne dig?"
\par 38 Bileam sagde til Balak: "Se, nu er jeg kommet til dig; men mon det står i min Magt at sige noget? Det Ord, Gud lægger mig i Munden, må jeg tale!"
\par 39 Da fulgte Bileam med Balak, og de kom til Hirjat Huzot.
\par 40 Balak ofrede her Hornkvæg og Småkvæg og sendte noget til Bileam og Høvdingerne, der var hos ham.
\par 41 Næste Morgen tog Balak Bileam med sig og førte ham op til Bamot Bål, hvorfra han kunde øjne den yderste Del af Folket.

\chapter{23}

\par 1 Og Bileam sagde til Balak: "Byg mig syv Altre her og skaf mig syv unge Tyre og syv Vædre herhen!"
\par 2 Balak gjorde, som Bileam sagde, og Balak og Bileam ofrede en Tyr og en Væder på hvert Alter.
\par 3 Derpå sagde Bileam til Balak: "Bliv stående her ved dit Brændoffer, så vil jeg gå hen og se, om HERREN mulig kommer mig i Møde. Hvad han da lader mig skue, skal jeg lade dig vide." Så gik han op på en nøgen Klippetop.
\par 4 Da kom Gud Bileam i Møde. Og han sagde til ham: "Jeg har gjort de syv Altre i Stand og ofret en Tyr og en Væder på hvert."
\par 5 Og Gud lagde Bileam Ord i Munden og sagde: "Vend tilbage til Balak og tal således til ham!"
\par 6 Da vendte han tilbage til ham, og se, han stod ved sit Brændoffer sammen med alle Moabs Høvdinger.
\par 7 Og han fremsatte sit Sprog: Fra Aram lod Balak mig hente, Moabs Konge fra Østens Bjerge: "Kom og forband mig Jakob, kom og kald Vrede ned over Isrrael!"
\par 8 Hvor kan jeg forbande, når Gud ej forbander, nedkalde Vrede, når HERREN ej vreds!
\par 9 Jeg ser det fra Klippernes Top, fra Højderne skuer jeg det, et Folk, der bor for sig selv og ikke regner sig til Hedningefolkene.
\par 10 Hvem kan måle Jakobs Sandskorn, hvem kan tælle Israels Støvgran? Min Sjæl dø de oprigtiges Død, og mit Endeligt vorde som deres!
\par 11 Da sagde Balak til Bileam: "Hvad har du gjort imod mig! Jeg lod dig hente, for at du skulde forbande mine Fjender, og se, du har velsignet dem!"
\par 12 Men han svarede: "Skal jeg ikke omhyggeligt sige, hvad HERREN Iægger mig i Munden?"
\par 13 Da sagde Balak til ham: "Følg med mig til et andet Sted, hvorfra du kan se Folket, dog kun den yderste Del deraf og ikke det hele, og forband mig det så fra det Sted!"
\par 14 Og han tog ham med til Udkigsmarken på Toppen af Pisga og rejste der syv Altre og ofrede en Tyr og en Væder på hvert.
\par 15 Derpå sagde Bileam til Balak: "Bliv stående her ved dit Brændoffer, medens jeg ser efter, om der møder mig noget!"
\par 16 Da kom Gud Bileam i Møde og lagde ham Ord i Munden og sagde: "Vend tilbage til Balak og tal således!"
\par 17 Så kom han hen til ham, og se, han stod ved sit Brændoffer sammen med Moabs Høvdinger; og Balak spurgte ham: "Hvad har HERREN sagt?"
\par 18 Da fremsatte han sit Sprog: Rejs dig, Balak, og hør, lyt til mig, Zippors Søn!
\par 19 Gud er ikke et Menneske, at han skulde lyve, et Menneskebarn, at han skulde angre; mon han siger noget uden at gøre det, mon han taler uden at fuldbyrde det?
\par 20 Se, at velsigne er mig givet, så velsigner jeg og tager intet tilbage!
\par 21 Man skuer ej Nød i Jakob, ser ej Trængsel i Israel; HERREN dets Gud er med det, og Kongejubel lyder hos det.
\par 22 Gud førte det ud af Ægypten, det har en Vildokses Horn.
\par 23 Thi mod Jakob hjælper ej Galder, Trolddom ikke mod Israel. Nu siger man om Jakob og om Israel: "Hvad har Gud gjort?"
\par 24 Se, et Folk, der står op som en Løvinde, rejser sig som en Løve! Det lægger sig først, når det har ædt Rov og drukket de dræbtes Blod.
\par 25 Da sagde Balak til Bileam: "Vil du ikke forbande det, så velsign det i alt Fald ikke!"
\par 26 Men Bileam svarede og sagde til Balak: "Har jeg ikke sagt dig, at alt, hvad HERREN siger, det gør jeg!"
\par 27 Da sagde Balak til Bileam: "Kom, jeg vil tage dig med til et andet Sted, måske det vil behage Gud, at du forbander mig det fra det Sted."
\par 28 Og Balak førte Bileam op på Toppen af Peor, der rager op over Ørkenen.
\par 29 Så sagde Bileam til Balak: "Byg mig syv Altre her og skaf mig syv unge Tyre og syv Vædre herhen!"
\par 30 Balak gjorde, som Bileam sagde, og ofrede en Tyr og en Væder på hvert Alter.

\chapter{24}

\par 1 Men da Bileam så, at HERRENS Hu stod til at velsigne Israel, gik han ikke som de forrige Gange hen for at søge Varsler, men vendte sig mod Ørkenen
\par 2 og Bileam så op og fik Øje på Israel; som lå lejret Stamme for Stamme. Da kom Guds Ånd over ham,
\par 3 og han fremsatte sit Sprog: Så siger Bileam, Beors Søn, så siger Manden, hvis Øje er lukket,
\par 4 så siger han, der hører Guds Ord og kender den Højestes Viden, som skuer den Almægtiges Syner, hensunken, med opladt Øje:
\par 5 Hvor herlige er dine Telte, JIakob, og dine Boliger, Israel!
\par 6 Som Dale, der strækker sig vidt, som Haver langs med en Flod, som Aloetræer, HERREN har plantet, som Cedre ved Vandets Bred.
\par 7 Dets Spande flyder over med Vand, dets Korn fåt rigelig Væde.
\par 8 Gud førte det ud af Ægypten, det har en Vildokses Horn; det opæder de Folkeslag der står det imod, søndrer deres Ben og knuser deres Lænder
\par 9 Det lægger sig, hviler som en Løve, ja, som en Løvinde, hvo tør vække det! Velsignet, hvo dig velsigner, forbandet, hvo dig forbander!
\par 10 Da blussede Balaks Vrede op mod Bileam, og han slog Hænderne sammen; og Balak sagde til Bileam: "For at forbande mine Fjender bad jeg dig komme, og se, nu har du velsignet dem tre Gange!
\par 11 Skynd dig derhen, hvor du kom fra! Jeg lovede dig rigelig Løn, men mon har HERREN unddraget dig den!"
\par 12 Men Bileam sagde til Balak: "Sagde jeg ikke allerede til Sendebudene, du sendte mig:
\par 13 Om Balak så giver mig alt det Sølv og Guld, han har i sit Hus, kan jeg ikke være ulydig mod HERREN og gøre noget som helst af egen Vilje; hvad HERREN siger, vil jeg sige!
\par 14 Vel, jeg drager til mit Folk, men kom, jeg vil lade dig vide, hvad dette Folk skal gøre ved dit Folk i de sidste Dage."
\par 15 Derpå fremsatte han sit Sprog: Så siger Bileam, Beors Søn, så siger Manden, hvis Øje er lukket,
\par 16 så siger han, der hører Guds Ord og kender den Højestes Viden, som skuer den Almægtiges Syner, hensunken, med opladt Øje:
\par 17 Jeg ser ham, dog ikke nu, jeg skuer ham, dog ikke nær! En Sterne opgår af Jakob, et Herskerspir løfter sig fra Israel! Han knuser Moabs Tindinger og alle Setsønnernes Isse.
\par 18 Edom bliver et Lydland, og Seirs undslupne går til Grunde, Israel udfolder sin Magt,
\par 19 og Jakob kuer sine Fjender.
\par 20 Men da han så Amalekiterne, fremsatte han sit Sprog: Det første af Folkene er Amalek, men til sidst vies det til Undergang!
\par 21 Og da han så Keniterne, fremsatte han sit Sprog: Urokkelig er din Bolig, din Rede bygget på Klippen.
\par 22 Kain er dog hjemfalden til Undergang! Hvor længe? Assur skal føre dig bort!
\par 23 Derpå fremsatte han sit Sprog: Ve! Hvo bliver i Live, når Gud lader det ske!
\par 24 Der kommer Skibe fra Kittæernes Kyst; de kuer Assur, de kuer Eber men også han er viet til Undergang!
\par 25 Så drog Bileam tilbage til sin Hjemstavn; og Balak gik også bort.

\chapter{25}

\par 1 Israeliterne slog sig derpå ned i Sjitim. Men Folket begyndte at bedrive Hor med de moabitiske Kvinder;
\par 2 og da de indbød Folket til deres Guders Slagtofre, spiste Folket deraf og tilbad deres Guder.
\par 3 Og Israel holdt til med Bål Peor; derover blussede HERRENs Vrede op mod Israel,
\par 4 og HERREN sagde til Moses: "Kald alle Folkets Overhoveder sammen og hæng dem op for HERREN under åben Himmel, for at HERRENs Vrede må vige fra Israel!"
\par 5 Og Moses sagde til Israeliternes Dommere: "Enhver af eder skal slå dem af sine Mænd ihjel, der har holdt til med Ba'al Peor!"
\par 6 Og se, en af Israeliterne kom og førte en midjanitisk Kvinde hen til sine Landsmænd lige for Øjnene af Moses og hele Israeliternes Menighed, medens de stod og græd ved Indgangen til Åbenbaringsteltet.
\par 7 Da nu Pinehas, Præsten Arons Søn Eleazars Søn, så det, trådte han frem af Menighedens Midte, greb et Spyd,
\par 8 fulgte den israelitiske Mand ind i Sengekammeret og gennemborede dem begge, både den israelitiske Mand og Kvinden, hende gennem Underlivet. Da standsede Plagen blandt Israeliterne.
\par 9 Men Tallet på dem, Plagen havde kostet Livet, løb op til 24000.
\par 10 Da talede HERREN således til Moses:
\par 11 Pinehas, Præsten Arons Søn Eleazars Søn, har vendt min Vrede fra Israeliterne, idet han var nidkær iblandt dem med min Nidkærhed, så at jeg ikke tilintetgjorde Israeliterne i min Nidkærhed.
\par 12 Derfor skal du sige: Se, jeg giver ham min Fredspagt!
\par 13 Et evigt Præstedømmes Pagt skal blive hans og efter ham hans Efterkommeres Lod, til Løn for at han var nidkær for sin Gud og skaffede Israeliterne Soning."
\par 14 Den dræbte Israelit, han, der dræbtes sammenmed den midjanitiske kvinde, hed Zimri, Salus Søn, og var Øverste for et Fædrenehus blandt Simeoniterne;
\par 15 og den dræbte midjanitiske Kvinde hed Hozbi og var en Datter af Zur, der var Stammehøvding for et Fædrenehus blandt Midjaniterne,
\par 16 HERREN talede fremdeles til Moses og sagde:
\par 17 "Fald over Midjaniterne og slå dem;
\par 18 thi de faldt over eder med de Rænker, de spandt imod eder i den Sag med Peor og med Hozbi, den midjanitiske Høvdings Datter, deres Landsmandinde, der dræbtes, den Dag Plagen brød løs for Peors Skyld."

\chapter{26}

\par 1 Efter denne Plage talede HERREN til Moses og Eleazar, Præsten Arons Søn, og sagde:
\par 2 "Hold Mandtal over hele Israeliternes Menighed fra Tyveårsalderen og opefter, Fædrenehus for Fædrenehus, alle våbenføre Mænd i Israel!"
\par 3 Da mønstrede Moses og Præsten Eleazar dem på Moabs Sletter ved Jordan over for Jeriko
\par 4 fra Tyveårsalderen og opefter, som HERREN havde pålagt Moses.
\par 5 Ruben, Israels førstefødte; Rubens Sønner: Fra Hanok stammer Hanokiternes Slægt, fra Pallu Palluiternes Slægt,
\par 6 fra Hezron Hezroniternes Slægt og fra Karmi Karmiternes Slægt.
\par 7 Det var Rubeniternes Slægter, og de af dem, som mønstredes, udgjorde 43730.
\par 8 Pallus Søn var Eliab;
\par 9 Eliabs Sønner: Nemuel, Datan og Abiram; det var den Datan og den Abiram, Menighedens udvalgte, som satte sig op imod Moses og Aron sammen med Koras Tilhængere, da de satte sig op imod HERREN;
\par 10 og Jorden åbnede sit Gab og slugte dem sammen med Kora, da hans Tilhængere omkom, idet Ilden fortærede de 25O Mænd, og de blev et Tegn til Advarsel.
\par 11 Men Koras Sønner omkom ikke.
\par 12 Simeons Sønners Slægter var følgende: Fra Nemuel stammer Nemueliternes Slægt, fra Jamin Jaminiternes Slægt, fra Jakin Jakiniternes Slægt,
\par 13 fra Zera Zeraiternes Slægt og fra Sjaul Sjauliternes Slægt.
\par 14 Det var Simeoniternes Slægter, 22200.
\par 15 Gads Sønners Slægter var følgende: Fra Zefon stammer Zefoniternes Slægt, fra Haggi Haggiternes Slægt, fra Sjuni Sjuniternes Slægt,
\par 16 fra Ozni Ozniternes Slægt, fra Eti Eriternes Slægt,
\par 17 fra Arod Aroditernes Slægt og fra Ar'eli Ar'eliternes Slægt.
\par 18 Det var Gads Sønners Slægter, de af dem, som mønstredes, 4O5OO.
\par 19 Judas Sønner var Er og Onan. men Er og Onan døde i Kana'ans Land.
\par 20 Judas Sønners Slægter var følgende: Fra Sjela sammer Sjelaniternes Slægt, fra Perez Pereziternes Slægt og fra Zera Zeraiternes Slægt;
\par 21 Perezs Sønner: Fra Hezron stammer Heztoniternes Slægt og fra Hamul Hamuliternes Slægt.
\par 22 Det var Judas Slægter, de af dem, som mønstredes, 76500.
\par 23 Issakars Sønners Slægter var følgende: Fra Tola stammer Tolaiternes Slægt, fra Pua Puniternes Slægt;
\par 24 fra Jasjub Jasjubiternes Slægt og fra Sjimron Sjimroniternes Slægt.
\par 25 Det var Issakars Slægter, de af dem, som mønstredes, 64 300.
\par 26 Zebulons Sønners Slægter var følgende: Fra Sered stammer Serediternes Slægt, fra Elon Eloniternes Slægt og fra Jale'el Jale'eliternes Slægt.
\par 27 Det var Zebuloniternes Slægter, de af dem, som mønstredes, 60 508
\par 28 Josefs Sønners Slægter var følgende: Manasse og Efraim;
\par 29 Manasses Sønner: Fra Makir stammer Makiriternes Slægt; Makir avlede Gilead, fra Gilead stammer Gijeaditernes Slægt;
\par 30 Gileads Sønner var følgende: Fra Abiezer stammer Abiezriternes Slægt, fra Helek Helekiternes Slægt,
\par 31 fra Asriel Asrieliternes Slægt, fra Sjekem Sjekemiternes Slægt,
\par 32 fra Sjemida Sjemidaiternes Slægt og fra Hefer Heferiternes Slægt;
\par 33 men Zelofhad, Hefers Søn, havde ingen Sønner, kun Døtre; Zelofhads Døtre hed Mala, Noa, Hogla, Milka og Tirza.
\par 34 Det var Manasses Slægter, og de af dem, som mønstredes, udgjorde 52700.
\par 35 Efraims Sønners Slægter var følgende: Fra Sjutela stammer Sjutelaiternes Slægt, fra Beker Bekeriternes Slægt og fra Tahan Tahaniternes Slægt;
\par 36 Sjultelas Sønner var følgende: Fra Eran stammer Eraniternes Slægt.
\par 37 Detvar Efraimiternes Slægter, de af dem, som mønstedes, 32500. Det var Josefs Sønners Slægter.
\par 38 Benjamins Sønners Slægtervar følgende: Fra Bela stammer Belaiternes Slægt, fra Asjbel Asjbeliternes Slægt, fra Ahiram Ahiramiternes Slægt,
\par 39 fra Sjufam Sjufamiternes Slægt og fra Hufam Hufamiternes Slægt.
\par 40 Belas Sønner: Ard og Na'aman; fra Ard stammer Arditernes Slægt, fra Na'aman Na'amiteroes Slægt.
\par 41 Det var Benjamins Sønners Slægter, og de af dem, som mønstredes, udgjorde 45600.
\par 42 Dans Sønners Slægter var følgende: Fra Sjuham stammer.
\par 43 Alle Sjuhamiternes Slægter, de af dem, som mønstredes, udgjorde 64 400.
\par 44 Asers Sønners Slægter var følgende: Fra Jimna stammer Jimniternes Slægt, fra Jisjvi Jisjviternes Slægt og fra Beri'a Beri'aiternes Slægt;
\par 45 fra Beri'as Sønner: Fra Heber stammer Hebriternes Slægt og fra Malkiel Malkieliternes Slægt.
\par 46 Asers Datter hed Sera.
\par 47 Det var Asers Sønners Slægter, og de af dem, som mønsttedes, udgjorde 53400.
\par 48 Naftalis Sønners Slægter var følgende: Fra Jaze'el stammer Jaze'eliternes Slægt, fra Guni Guniternes Slægt,
\par 49 fra Jezer Jezeriternes Slægt og fra Sjillem Sjillemiternes Slægt,
\par 50 Det var Naftalis Slægter, Slægt for Slægt, og de af dem, som mønstredes, udgjorde 45400.
\par 51 Det var dem af Israeliterne, som mønstredes, 601730.
\par 52 HERREN talede fremdeles til Moses og sagde:
\par 53 Til dem skal Landet udskiftes til Arv og Eje efter de optalte Navne.
\par 54 En stor Stamme skal du give en stor Arvelod, en lille Stamme en lille; enhver af dem skal der gives en Arvelod efter Tallet på de mønstrede i den.
\par 55 Dog skal Landet udskiftes ved Lodkastning; de skal have deres Arvelodder efter Navnene på deres fædrene Stammer;
\par 56 ved Lodkastning skal enhver Stamme, stor eller lille, have sin Arvelod tildelt.
\par 57 Følgende er de af Leviterne, der mønstredes, Slægt for Slægt: Fra Gerson stammer Gersoniternes Slægt, fra Kehat Kehatiternes Slægt og fra Merari Merariternes Slægt.
\par 58 Følgende er Levis Slægter: Libniternes Slægt, Hebroniternes Slægt, Maliternes Slægt, Musjiternes Slægt og Koraiternes Slægt.
\par 59 Amrams Hustru hed Jokebed, Levis Datter, som fødtes Levi i Ægypten; hun fødte for Amram Aron, Moses og deres Søster Mirjam.
\par 60 For Aron fødtes Nadab, Abihu, Eleazar og Itamar.
\par 61 Men Nadab og Abihu omkom, da de frembar fremmed Ild for HERRENs Åsyn.
\par 62 De af dem, der mønstredes, udgjorde 23000, alle af Mandkøn fra en Måned og opefter. De mønstredes nemlig ikke sammen med de andre Israeliter, da der ikke var givet dem nogen Arvelod blandt Israeliterne.
\par 63 Det var dem, der mønstredes af Moses og Præsten Eleazar, da disse mønstrede Israeliterne på Moabs Sletter ved Jordan over for Jeriko.
\par 64 Blandt dem var der ingen, som var mønstret af Moses og Præsten Aron, da de mønstrede Israeliterne i Sinaj Ørken;
\par 65 thi HERREN havde sagt til dem, at de skulde dø i Ørkenen.

\chapter{27}

\par 1 Men Zelofhads Døtre, hvis Fader var en Søn af Hefer, en Søn af Gilead, en Søn af Makir, en Søn af Manasse - de hørte altså til Josefs Søn Manasses Slægter, og deres Navne var Mala, Noa, Hogla, Milka og Tirza - trådte hen
\par 2 og stillede sig frem for Moses, Præsten Eleazar, Øversterne og hele Menigheden ved Indgangen til Åbenbaringsteltet og sagde:
\par 3 "Vor Fader døde i Ørkenen han hørte ikke med til Koras Tilhængere, dem, der rottede sig sammen mod HERREN, men døde for sin egen Synds Skyld og han havde ingen Sønner.
\par 4 Hvorfor skal nu vor Faders Navn udslettes af hans Slægt, fordi han ingen Søn havde? Giv os Ejendom blandt vor Faders Brødre!"
\par 5 Og Moses lagde deres Sag frem for HERRENs Åsyn.
\par 6 Da talede HERREN således til Moses:
\par 7 "Zelofbads Døtre har Ret i, hvad de siger; giv dem Ejendom til Arvelod mellem deres Faders Brødre og lad deres Faders Arvelod tilfalde dem
\par 8 Og til Israeliterne skal du tale og sige således: Når en Mand dør uden at efterlade sig nogen Søn, da skal I lade hans Arvelod gå i Arv til hans Datter;
\par 9 har han heller ingen Datter, skal I give hans Arvelod til hans Brødre;
\par 10 har han heller ingen Brødre, skal I give hans Arvelod til hans Farbrødre;
\par 11 og har hans Fader ingen Brødre, skal I give hans Arvelod til hans nærmeste kødelige Slægtning, som så skal arve den. Det skal være Israeliterne en retsgyldig Anordning, som HERREN har pålagt Moses."
\par 12 Og HERREN sagde til Moses: "Stig op på Åbarimbjerget her og se ud over Landet, som jeg vil give Israeliterne.
\par 13 Og når du har set ud over det, skal også du samles til din Slægt ligesom din Broder Aron;
\par 14 I var jo genstridige mod mit Ord i Zins Ørken, dengang Menigheden yppede Kiv, så at I ikke helligede mig i deres Påsyn ved at skaffe Vand." Det er Meriba Kadesjs Vand i Zins Ørken.
\par 15 Og Moses talte således til HERREN:
\par 16 "Måtte HERREN, Gud over alt Køds Ånder, indsætte en Mand over Menigheden,
\par 17 som kan drage ud og hjem i Spidsen for dem og føre dem ud og hjem, for at ikke HERRENs Menighed skal blive som en Hjord uden Hyrde!"
\par 18 Da sagde HERREN til Moses: "Tag Josua, Nuns Søn, en Mand, i hvem der er Ånd, læg din Hånd på ham,
\par 19 fremstil ham for Præsten Eleazar og hele Menigheden og indsæt ham således i deres Påsyn;
\par 20 og overdrag ham noget af din Værdighed, for at hele Israeliternes Menighed kan adlyde ham.
\par 21 Men han skal træde frem for Præsten Eleazar, for at han kan skaffe ham Urims Kendelse for HERRENs Åsyn; på hans Bud skal han drage ud, og på hans Bud skal han vende hjem, han og alle Israeliterne, hele Menigheden."
\par 22 Moses gjorde som HERREN havde pålagt ham; han tog Josua og fremstillede ham for Præsten Eleazar og hele Menigheden;
\par 23 og han lagde sine Hænder på ham og indsatte ham, således som HERREN havde påbudt ved Moses.

\chapter{28}

\par 1 Og HERREN talede til Moses og sagde:
\par 2 Byd Israeliterne og sig til dem: I skal omhyggeligt bringe mig mine Oergaver, min Ildofferspise til en liflig Duft, til de fastsatte Tider!
\par 3 Og sig til dem: Dette er det Ildoffer, I skal bringe HERREN: Hver Dag to årgamle, lydefri Lam som dagligt Brændoffer;
\par 4 det ene Lam skal du ofre om Morgenen, det andet ved Aftenstid;
\par 5 og som Afgrødeoffer dertil en Tiendedel Efa fint Hvedemel, rørt i en Fjerdedel Hin Olie af knuste Frugter
\par 6 det er det daglige Brændoffer, som ofredes ved Sinaj Bjerg til en liflig Duft, et Ildoffer for HERREN
\par 7 fremdeles som Drikoffer dertil en Fjerdedel Hin Vin for hvert Lam; i Helligdommen skal derudgydes Drikoffer af stærk Drik for HERREN.
\par 8 Det andet Lam skal du ofre ved Aftenstid; med samme Afgrødeoffer og Drikoffer som om Morgenen skal du ofre det, et Ildoffer til en liflig Duft for HERREN.
\par 9 På Sabbatsdagen skal det være to årgamle, lydefri Lam og som Afgrødeoffer to Tiendedele Efa fint Hvedemel, rørt i Olie, med tilhørende Drikoffer,
\par 10 et Sabbatsbrændoffer på hver Sabbat foruden det daglige Brændoffer med tilhørende Drikofer.
\par 11 På den første Dag i hver Måned skal I som Brændoffer bringe HERREN to unge Tyre, en Væder og syv årgamle Lam, lydefri Dyr,
\par 12 og som Afgrødeoffer dertil for hver Tyr tre Tiendedele Efa fint Hvedemel, rørt i Olie, som Afgrødeoffer for hver Væder to Tiendedele Efa fint Hvedemel, rørt i Olie,
\par 13 og som Afgrødeoffer for hvert Lam en Tiendedel Efa fint Hvedemel, rørt i Olie, et Brændoffer til en liflig Duft, et Ildoffer for HERREN;
\par 14 desuden de tilhørende Drikofre, en halv Hin Vin for hver Tyr, en Tredjedel Hin for hver Væder og en Fjerdedel Hin for hvert Lam. Det er det månedlige Brændoffer for hver af Årets Måneder.
\par 15 Tillige skal der foruden det daglige Brændoffer ofres HERREN en Gedebuk som Syndoffer med tilhørende Drikoffer.
\par 16 På den fjortende Dag i den første Måned skal dervære Påske for HERREN.
\par 17 Den femtende Dag i den Måned er det Højtid; i syv Dage skal der spises usyrede Brød.
\par 18 På den første Dag skal der være Højtidsstævne, intet som belst Arbejde må I udføre.
\par 19 Da skal I bringe som Ildoffer, som Brændoffer for HERREN, to unge Tyre, en Væder og syv årgamle Lam, lydefri Dyr skal I tage
\par 20 og som Afgrødeoffer dertil fint Hvedemel rørt i Olie; tre Tiendedele Efa skal I ofre for hver Tyr, to Tiendedele for hver Væder
\par 21 og en Tiendedel for hvert af de syv Lam;
\par 22 desuden en Buk som Syndoffer for at skaffe eder Soning.
\par 23 Foruden Morgen brændofferet, det daglige Brændoffer, skal I ofre det.
\par 24 Sådanne Ofre skal I bringe hver af de syv Dage som Ildofferspise til en liflig Duft for HERREN; de skal ofres med tilhørende Drikoffer foruden det daglige Brændoffer.
\par 25 På den syvende Dag skal I holde Højtidsstævne; intet som helst Arbejde må I udføre.
\par 26 På Førstegrødens Dag, når I bringer HERREN Afgrødeoffer af den ny Afgrøde på eders Ugefest, skal I holde Højtidsstævne; intet som helst Arbejde må I udføre.
\par 27 Da skal I som Brændoffer til en liflig Duft for HERREN ofre to unge Tyre, en Væder og syv årgamle Lam
\par 28 og som Afgrødeoffer dertil fint Hvedemel, rørt i Olie, tre Tiendedele Efa for hver Tyr, to Tiendedele for hver Væder
\par 29 og en Tiendedel for hvert af de syv Lam;
\par 30 desuden en Gedebuk for at skaffe eder Soning.
\par 31 Foruden det daglige Brændoffer med tilhørende Afgrødeoffer skal I ofre det, lydefri Dyr skal I tage, med tilhørende Drikofre.

\chapter{29}

\par 1 På den første Dag i den syvende Måned skal I holde Højtidsstævne, intet som helst Arbejde må I udføre; I skal fejre den som en Hornblæsningsdag.
\par 2 Da skal I som Brændoffer til en liflig Duft for HERREN ofre en ung Tyr, en Væder og syv årgamle Lam, lydefri Dyr;
\par 3 som Afgrødeoffer dertil fint Hvedemel, rørt i Olie, tre Tiendedele Efa for Tyren, to Tiendedele for Væderen
\par 4 og en Tiendedel for hvert af de syv Lam;
\par 5 desuden en Gedebuk som Syndoffer for at skaffe eder Soning;
\par 6 alt foruden Nymånebrændofferet med tilhørende Afgrødeoffer og det daglige Brændoffer med tilhørende Afgrødeoffer og Drikofre efter de derom gældende forskrifter, til en liflig Duft, et Ildoffer for HERREN.
\par 7 På den tiende dag i samme syvende Måned skal I holde Højtidsstævne, I skal faste og må intet som helst Arbejde udføre.
\par 8 Da skal I som Brændoffer til en liflig duft for HERREN ofre en ung Tyr, en Væder og syv årgamle Lam, lydefri Dyr skal I tage,
\par 9 og som Afgrødeoffer dertil fint Hvedemel, rørt i Olie, tre Tiendedele Efa for Tyren, to Tiendedele for Væderen
\par 10 og en Tiendedel for hvert af de syv Lam;
\par 11 desuden en Gedebuk som Syndoffer; alt foruden Soningssyndoffe, det og det daglige Brændoffer med tilhørende Afgrødeoffer og Drikofre.
\par 12 På den femtende Dag i den syvende Måned skal I holde Højtidsstævne, I må intet som helst Atbejde udføre, og I skal holde Højtid for HERREN i syv Dage.
\par 13 Da skal I som Brændoffer, som Ildoffer til en liflig Duft for HERREN ofre tretten unge Tyre, to Vædre og fjorten årgamle Lam, lydefri Dyr skal det være,
\par 14 og som Afgrødeoffer dertil fint Hvedemel, rørt i Olie, tre Tiendedele Efa for hver af de tretten Tyre, to Tiendedele for hver af de to Vædre
\par 15 og en Tiendedel for hvert af de fjorten Lam;
\par 16 desuden en Gedebuk som Syndoffer; alt foruden det daglige Brændoffer med tilhørende Afgrødeoffer og Drikoffer.
\par 17 På den anden Dag skal I ofre tolv unge Tyre, to Vædre og fjorten årgamle Lam, lydefri Dyr,
\par 18 med tilhørende Afgrødeoffer og Drikofre for Tyrene, Vædrene og Lammene efter deres Tal på den foreskrevne Måde;
\par 19 desuden en Gedebuk som Syndoffer; alt foruden det daglige Brændoffer med tilhørende Afgrødeoffer og Drikoffer.
\par 20 På den tredje Dag skal I ofre elleve unge Tyre, to Vædre og fjorten årgamle Lam, lydefri Dyr,
\par 21 med tilhørende Afgrødeoffer og Drikofre for Tyrene, Vædrene og Lammene efter deres Tal på den foreskrevne Måde;
\par 22 desuden en Gedebuk som Syndoffer; alt foruden det daglige Brændoffer med tilhørende Afgrødeoffer og Drikoffer.
\par 23 På den fjerde Dag skal I ofre ti Tyre, to Vædre og fjorten årgamle Lam, lydefri Dyr,
\par 24 med tilhørende Afgrødeoffer og Drikofre for Tyrene, Vædrene og Lammene efter deres Tal på den foreskrevne Måde;
\par 25 desuden en Gedebuk som Syndoffer; alt foruden det daglige Brændoffer med tilhørende Afgrødeoffer og Drikoffer.
\par 26 På den femte Dag skal I ofre ni Tyre, to Vædre og fjorten årgamle Lam, lydefri Dyr,
\par 27 med tilhørende Afgrødeoffer og Drikofre for Tyrene,Vædrene og Lammene efter deres Tal på den foreskrevne Måde;
\par 28 desuden en Gedebuk som Syndoffer; alt foruden det daglige Brændoffer med tilhørende Afgrødeoffer og Drikoffer.
\par 29 På den sjette Dag skal I ofre otte Tyre, to Vædre og fjorten årgamle Lam, lydefri Dyr,
\par 30 med tilhørende Afgrødeoffer og Drikofre for Tyrene, Vædrene og Lammene efter deres Tal på den foreskrevne Måde;
\par 31 desuden en Gedebuk til Syndoffer; alt foruden det daglige Brændoffer med tilhørende Afgrødeoffer og Drikoffer.
\par 32 På den syvende dag skal I ofre syv Tyre, to Vædre og fjorten årgamle Lam, lydefri Dyr,
\par 33 med tilhørende Afgrødeoffer og Drikofre for Tyrene, Vædrene og Lammene efter deres Tal på den foreskrevne Måde;
\par 34 desuden en Gedebuk som Syndoffer; alt foruden det daglige Brændoffer med tilhørende Afgrødeoffer og Drikoffer.
\par 35 På den ottende Dag skal I holde festlig Samling, I må intet som helst Arbejde udføre.
\par 36 Da skal I som Brændoffer, som Ildoffer til en liflig Duft for HERREN ofre en Tyr, en Væder og syv årgamle Lam, lydefri Dyr,
\par 37 med tilhørende Afgrødeoffer og Drikofre for Tyren, Væderen og Lammene efter deres Tal på den foreskrevne Måde;
\par 38 desuden en Gedebuk som Syndoffer; alt foruden det daglige Brændoffer med tilhørende Afgrødeoffer og Drikoffer.
\par 39 Disse Ofre skal I bringe HERREN på eders Højtider, bortset fra eders Løfte og Frivilligofre, hvad enten det nu er Brændofre, Afgrødeofre, Drikofre eller Takofre.

\chapter{30}

\par 1 Og Moses talte til Israeliterne, ganske som HERREN havde pålagt Moses.
\par 2 Moses talte fremdeles til Overhovederne for Israelitternes Stammer og sagde: Dette er, hvad HERREN har påbudt:
\par 3 Når en Mand aflægger et Løfte til HERREN eller ved Ed forpligter sig til Afholdenhed i en eller anden Retning, må han ikke bryde sit Ord, men skal holde hvert Ord, der er udgået af hans Mund.
\par 4 Men når en Kvinde aflægger et Løfte til HERREN og forpligter sig til Afholdenhed i en eller anden Retning, medens hun endnu i sine unge År opholder sig i sin Faders Hus.
\par 5 og hendes Fader, når han hører om hendes Løfte og den Forpligtelse til Afholdenhed, hun har påtaget sig, ikke siger noget til hende, så skal alle hendes Løfter stå ved Magt, og enhver Forpligtelse til Afholdenhed, hun har påtaget sig, skal stå ved Magt.
\par 6 Hvis hendes Fader derimod formener hende det, samme Dag han får det at høre, skal ingen af hendes Løfter eller af de Forpligtelser til Afholdenhed, hun har påtaget sig, stå ved Magt, og HERREN skal tilgive hende, fordi hendes Fader har forment hende det.
\par 7 Hvis hun bliver gift medens der påhviler hende Løfter eller en Forpligtelse, hun har påtaget sig ved et uoverlagt Ord,
\par 8 og hendes Mand ikke siger noget til hende, samme Dag han får det at høre, skal hendes Løfter stå ved Magt, og den Forpligtelse til Afholdenhed, hun har påtaget sig, skal stå ved Magt.
\par 9 Hvis hendes Mand derimod formener hende det, samme Dag han får det at høre, gør han dermed det Løfte, der påhviler hende, og den Forpligtelse til Afholdenhed, hun har påtaget sig ved et uoverlagt Ord, ugyldig, og HERREN skal tilgive hende.
\par 10 En Enkes og en forstødt Hustrus Løfte, enhver Forpligtelse til Afholdenhed, hun har påtaget sig, er bindende for hende.
\par 11 Hvis en Kvinde i sin Mands Hus aflægger et Løfte eller ved Ed forpligter sig til Afholdenhed i en eller anden Retning,
\par 12 og hendes Mand, når han får det at høre, ikke siger noget til hende og ikke formener hende det, skal alle hendes Løfter stå ved Magt, og enhver Forpligtelse til Afholdenhed, hun har påtaget sig, skal stå ved Magt.
\par 13 Hvis hendes Mand derimod, samme Dag han får det at høre, gør det ugyldigt, så står intet af det, hun har udtalt, ved Magt, hverken hendes Løfter eller den påtagne Forpligtelse til Afholdenhed; hendes Mand har gjort dem ugyldige, og HERREN skal tilgive hende.
\par 14 Ethvert Løfte og enhver ved Ed påtagen Forpligtelse til Faste kan hendes Mand stadfæste eller gøre ugyldig.
\par 15 Men hvis hendes Mand tier stille over for hende til næste Dag, stadfæster han alle hendes Løfter og alle de Forpligtelser til Afholdenhed, hun har påtaget sig; han har stadfæstet dem, thi han sagde ikke noget til hende, samme Dag han fik det at høre;
\par 16 og hvis han vil gøre dem ugyldige, en Tid efter at han fik det at høre, skal han undgælde for hendes Brøde.
\par 17 Det er de Anordninger, HERREN gav Moses om Forholdet mellem Mand og Hustru og mellem Fader og Datter, medens hun endnu i sine unge År opholder sig i hans Hus.

\chapter{31}

\par 1 Og HERREN talede til Moses og sagde:
\par 2 "Skaf Israelitterne Hævn over Midjaniterne; så skal du samles til din Slægt!"
\par 3 Da talte Moses til Folket og sagde: "Udrust Mænd af eders Midte til Kamp, for at de kan falde over Midjan og fuldbyrde HERRENs Hævn på Midjan;
\par 4 1OOO Mand af hver af Israels Stammer skal I sende i Kamp!"
\par 5 Af Israels Tusinder udtoges da 1000 af hver Stamme, i alt 12.000 Mand, rustede til Kamp.
\par 6 Og Moses sendte dem i kamp, 1OOO af hver Stamme, og sammen med dem Pinehas, Præsten Eleazars Søn, der medbragte de hellige Redskaber og Alarmtrompeterne.
\par 7 De drog så ud i kamp mod Midjaniterne, som HERREN havde pålagt Moses, og dræbte alle af Mandkøn;
\par 8 og foruden de andre, der blev slået ihjel, dræbte de også Midjans Konger, Evi, Rekem, Zur, Hur og Reba, Midjans fem Konger; også Bileam, Beors Søn, dræbte de med Sværdet.
\par 9 Og Israelitterne bortførte Midjaniternes kvinder og Børn som Krigsfanger, og alt deres Kvæg alle deres Hjorde og alt deres Gods tog de med som Bytte;
\par 10 og alle deres Byer på de beboede Steder og alle deres Teltlejre stak de Ild på.
\par 11 Og alt det røvede og hele Byttet, både Mennesker og Dyr, tog de med sig,
\par 12 og de bragte Fangerne, Byttet og det røvede til Moses og Præsten Eleazar og Israelitternes Menighed i Lejren på Moabs Sletter ved Jordan over for Jeriko.
\par 13 Men Moses, Præsten Eleazar og alle Menighedens Øverste gik dem i Møde uden for Lejren,
\par 14 og Moses blev vred på Hærens Førere, Tusindførerne og Hundredførerne, som kom tilbage fra Krigstoget.
\par 15 Og Moses sagde til dem: "Har I ladet alle Kvinder i Live?
\par 16 Det var jo dem, der efter Bileams Råd blev Årsag til, at Israelitterne var troløse mod HERREN i den Sag med Peor, så at Plagen ramte HERRENs Menighed.
\par 17 Dræb derfor alle Drengebørn og alle kvinder, der har kendt Mand og haft Samleje med Mænd;
\par 18 men alle Piger, der ikke har haft Samleje med Mænd, skal I lade i Live og beholde,
\par 19 Men selv skal I lejre eder uden for Lejren i syv Dage, enhver, som har dræbt nogen, og enhver, som har rørt ved en dræbt, og rense eder på den tredje og den syvende Dag, både I selv og eders Krigsfanger.
\par 20 Og enhver Klædning, enhver Læder ting, alt, hvad der er lavet af Gedehår,og alle Træredskaber skal I rense!"
\par 21 Og Præsten Eleazar sagde til Krigerne, der havde været med i Kampen: "Dette er det Lovbud, HERREN har givet Moses:
\par 22 Kun Guld, Sølv, kobber, Jern, Tin og Bly,
\par 23 alt, hvad der kan tåle Ild, skal I lade gå gennem Ild, så bliver det rent; dog må det renses med Renselsesvand. Men alt, hvad der ikke kan tåle Ild, skal I lade gå gennem Vand.
\par 24 Og eders klæder skal I tvætte på den syvende Dag, så bliver I rene og kan gå ind i Lejren."
\par 25 Og HERREN talede til Moses og sagde:
\par 26 "Sammen med Præsten Eleazar og Overhovederne for Menighedens Fædrenehuse skal du opgøre det samlede Bytte, der er taget, både Mennesker og Dyr.
\par 27 Derefter skal du dele Byttet i to lige store Dele mellem dem, der har taget Del i Krigen og været i Kamp, og hele den øvrige Menighed.
\par 28 Derpå skal du udtage en Afgift til HERREN fra krigerne, der har været i Kamp, et Stykke af hver fem Hundrede, både af Mennesker, Hornkvæg, Æsler og Småkvæg;
\par 29 det skal du tage af den Halvdel, som tilfalder dem, og give Præsten Eleazar det som Offerydelse til HERREN.
\par 30 Men af den Halvdel, der tilfalder de andre Israelitter, skal du tage et Stykke af hver halvtredsindstyve, både af Mennesker, Hornkvæg, Æsler og Småkvæg, alt Kvæget, og give det til Leviterne, som tager Vare på, hvad der er at varetage ved HERRENs Bolig!"
\par 31 Da gjorde Moses og Præsten Eleazar, som HERREN havde pålagt Moses.
\par 32 Og det, de havde taget, det tiloversblevne af Byttet, som Krigsfolket havde gjort, udgjorde 675000 Stykker Småkvæg,
\par 33 72 000 Stykker Hornkvæg,
\par 34 61 000 Æsler
\par 35 og 32008 Mennesker, Kvinder, der ikke havde haft Samleje med Mænd.
\par 36 Den Halvdel, der tilfaldt dem, der havde været i Kamp, udgjorde altså et Tal af 337500 Stykker Småkvæg,
\par 37 hvoraf Afgiften til HERREN udgjorde 675 Stykker Småkvæg,
\par 38 36000 Stykker Hornkvæg, hvoraf 72 i Afgift til HERREN,
\par 39 3O5OO Æsler, hvoraf til i Afgift til HERREN,
\par 40 og 16000 Mennesker, hvoraf 32 i Afgift til HERREN.
\par 41 Og Moses overgav Afgiften, HERRENs Offerydelse, til Præsten Eleazar, som HERREN havde pålagt Moses.
\par 42 Og af den Halvdel, som tilfaldt de andre Israelitter, og som Moses havde taget som deres del fra de Mænd, der havde været i Kamp
\par 43 denne Menighedens Halvdel udgjorde 337 500 Stykker Småkvæg,
\par 44 36 OOO Stykker Hornkvæg,
\par 45 30 500 Æsler
\par 46 og 16000 Mennesker
\par 47 af den Halvdel, som tilfaldt de andre Israelitter, udtog Moses et Stykke for hver halvtredsindstyve, både af Mennesker og Kvæg, og gav dem til Leviterne, som tog Vare på, hvad der var at varetage ved HERRENs Bolig, som HERREN havde pålagt Moses.
\par 48 Da trådte Førerne for Hærens Afdelinger, Tusindførerne og Hundredførerne, hen til Moses
\par 49 og sagde til ham: "Dine Trælle har holdt Mandtal over de krigere, der stod under os; og der manglede ikke en eneste af os;
\par 50 derfor frembærer nu enhver af os som Offergave til HERREN, hvad han har taget af Guldsmykker, Armbånd, Spange, Fingerringe, Ørenringe og Halssmykker, for at skaffe os Soning for HERRENs Åsyn."
\par 51 Moses og Præsten Eleazar modtog Guldet af dem, alskens med Kunst virkede Smykker;
\par 52 og alt Offerydelsesguldet, som de ydede HERREN, udgjorde 16750 Sekel, hvilket Tusindførerne og Hundredførerne bragte som Gave.
\par 53 Krigerne havde taget Bytte hver for sig.
\par 54 Så modtog Moses og Præsten Eleazar Guldet af Tusindførerne og Hundredførerne og bragte det ind i Åbenbaringsteltet, for at det skulde bringe Israelitterne i Minde for HERRENs Åsyn.

\chapter{32}

\par 1 Rubeniterne og Gaditerne havde meget Kvæg i store mængder. Da de nu så Jazers Land og Gileads Land, fandt de, at Stedet egnede sig til Kvægavl.
\par 2 Derfor kom Gaditerne og Rubeniterne og sagde til Moses og Præsten Eleazar og Menighedens Øverste:
\par 3 "Atarot, Dibon, Ja'zer, Nimra, Hesjbon, Elale, Sebam, Nebo og Beon,
\par 4 det Land, HERREN har erobret for Israels Menighed, er et Land, der egner sig til Kvægavl, og dine Trælle ejer Hjorde."
\par 5 Og de sagde: "Dersom vi har fundet Nåde for dine Øjne, så lad dine Trælle få dette Land i Eje; før os ikke over Jordan!"
\par 6 Men Moses sagde til Gaditerne og Rubeniterne: "Skal eders Brødre drage i Krig, medens I bliver boende her?
\par 7 Og hvorfor vil I betage Israelitterne Modet til at drage over til det Land, HERREN har givet dem?
\par 8 Det gjorde eders Fædre, da jeg fra Kadesj Barnea sendte dem hen for at se på Landet;
\par 9 da de var draget op til Esjkoldalen og havde set på Landet, det og de Israelitterne Modet til at drage ind i det Land, HERREN havde givet dem.
\par 10 Men HERRENs Vrede blussede den Gang op, og han svor:
\par 11 De Mænd, der er draget op fra Ægypten, fra Tyveårsalderen og opefter, skal ikke få det Land at se, jeg tilsvor Abraham, Isak og Jakob, fordi de ikke viste mig fuld lydighed,
\par 12 med Undtagelse af Kenizziten Kaleb, Jetunnes Søn, og Josua, Nuns Søn, thi de viste HERREN fuld Lydighed!
\par 13 Og HERRENs Vrede blussede op mod Israel, og han lod dem vanke om i Ørkenen i fyrretyve År, indtil hele den Slægt var gået til Grunde, der gjorde, hvad der var ondt i HERRENs Øjne.
\par 14 Og se, I træder nu i eders Fædres Fodspor, en Yngel af Syndere, for yderligere at øge HERRENs Vrede mod Israel!
\par 15 Når I viger bort fra ham, vil han lade det blive endnu længer i Ørkenen, og I bringer Fordærvelse over hele dette Folk."
\par 16 Da trådte de frem for ham og sagde: "Vi vil kun bygge Kvægfolde til vore Hjorde her og Byer til vore Familier;
\par 17 men selv vil vi ruste os til Kamp og drage i Spidsen for Israelitterne, til vi har bragt dem hen til deres Sted; imens skal vore Familier blive i de befæstede Byer i Ly for Landets indbyggere.
\par 18 Vi vil ikke vende tilbage til vore Huse, før enhver af Israelitterne har fået sin Arvelod;
\par 19 thi vi vil ikke have Arvelod sammen med dem på den anden Side af Jordan og længere borte, eftersom vi har fået vor Arvelod her på denne Side af Jordan på Østsiden."
\par 20 Da sagde Moses til dem: "Hvis I gør det, hvis I ruster eder til Kamp for HERRENs Åsyn,
\par 21 hvis alle eders kamprustede Mænd overskrider Jordan for HERRENs Åsyn og bliver der, indtil han har jaget sine Fjender bort fra sit Åsyn,
\par 22 og hvis I først vender tilbage, når Landet er undertvunget for HERRENs Åsyn, skal I være sagesløse over for HERREN og Israel, og så skal Landet her blive eders Ejendom for HERRENs Åsyn.
\par 23 Men hvis I ikke gør det, se, da synder I mod HERREN, og da skal I få eders Synd at mærke, den skal nok finde eder.
\par 24 Byg eder Byer til eders Familier og Folde til eders Småkvæg og gør, som I har sagt!"
\par 25 Da sagde Gaditerne og Rubeniterne til Moses: "Dine Trælle vil gøre, som min Herre byder;
\par 26 vore Børn, Kvinder, Hjorde og alt vort Kvæg skal blive der i Gileads Byer,
\par 27 men dine Trælle vil drage over og tage Del i Krigen, så mange som er rustet til Kamp for HERRENs Åsyn, således som min Herre har sagt."
\par 28 Så gav Moses Præsten Eleazar og Josua, Nuns Søn, og Overhovederne for de israelitiske Stammers Fædrenehuse Befaling om dem,
\par 29 og Moses sagde til dem: "Hvis Gaditerne og Rubeniterne, så mange som er rustet til Kamp for HERRENs Åsyn, går over Jordan sammen med eder og Landet bliver eder underlagt, skal I give dem Gilead i Eje;
\par 30 men hvis de ikke går over sammen med eder, rustede til Kamp, skal de have Bopæl anvist blandt eder i Kana'ans Land."
\par 31 Da svarede Gaditerne og Rubeniterne: "Hvad HERREN har talt til dine Trælle, vil vi gøre;
\par 32 vi vil, rustede til Kamp for HERRENs Øjne, drage over til Kana'ans Land, men vor Arvelod på den anden Side af Jordan bliver i vort Eje."
\par 33 Da gav Moses Gaditerne, Rubeniterne og Josefs Søn Manasses halve Stamme Amoriterkongen Sihons Kongerige og kong Og af Basans kongerige, Landet med Byerne og deres Område, Landets Byer rundt om.
\par 34 Så byggede Gaditerne Dibon, Atarot, Aroer.
\par 35 Atarot Sjofan, Ja'zer, Jogbeba,
\par 36 Bet Nimra, Bet Haran, befæstede Byer, og Kvægfolde;
\par 37 og Rubeniterne byggede Hesjbon, Elale og Kirjatajim,
\par 38 Nebo og Ba'al Meon, hvis Navne ændredes, og Sibma; og de opkaldte Byerne, som de byggede, efter deres Navne.
\par 39 Og Manasses Søn Makirs Sønner drog til Gilead og erobrede det og drev de der boende Amoriter bort;
\par 40 og Moses overdrog Gilead til Manasses Søn Makir, og han bosatte sig der;
\par 41 men Manasses Søn Jair drog hen og erobrede deres Teltbyer og kaldte dem Jairs Teltbyer.
\par 42 Og Noba drog hen og erobrede Kenat med tilhørende Småbyer og kaldte det Noba efter sit eget Navn.

\chapter{33}

\par 1 Følgende er de enkelte Strækninger på Israelitternes Vandring, de tilbagelagde på Vejen fra Ægypten, Hærafdeling for Hærafdeling, under Anførsel af Moses og Aron.
\par 2 Moses optegnede på HERRENs Bud de Steder, de brød op fra, Strækning for Strækning; og følgende er de enkelte Strækninger efter de Steder, de brød op fra:
\par 3 De brød op fra Rameses på den femtende Dag i den første Måned; Dagen efter Påske drog Israelitterne ud, værnede af en stærk Hånd, for Øjnene af alle Ægypterne,
\par 4 medens Ægypterne jordede alle de førstefødte, som HERREN havde slået iblandt dem; thi HERREN havde holdt Dom over deres Guder.
\par 5 Israelitterne brød altså op fra Ra'meses og slog Lejr i Sukkot.
\par 6 Så brød de op fra Sukkot og slog Lejr i Etam, der ligger ved Ørkenens Rand.
\par 7 Så brød de op fra Etam og vendte om mod Pi Hakirot over for Ba'al Zefon og slog Lejr over for Migdol.
\par 8 Så brød de op fra Pi Hakirot og drog tværs igennem Havet til Ørkenen; og de vandrede tre Dagsrejser i Etams Ørken og slog Lejr i Mara.
\par 9 Så brød de op fra Mara og kom til Elim; i Elim var der tolv Vandkilder og halvfjerdsindstyve Palmetræer, og der slog de Lejr.
\par 10 Så brød de op fra Elim og slog Lejr ved det røde Hav.
\par 11 Så brød de op fra det røde Hav og slog Lejr i Sins Ørken.
\par 12 Så brød de op fra Sins Ørken og slog Lejr i Dofka.
\par 13 Så brød de op fra Dofka og slog Lejr i Alusj.
\par 14 Så brød de op fra Alusj og slog Lejr i Refdim, hvor Folket ikke havde Vand at drikke.
\par 15 Så brød de op fra Refdim og slog Lejr i Sinaj Ørken.
\par 16 Så brød de op fra Sinaj Ørken og slog Lejr i Kibrot Hatta'ava,
\par 17 Så brød de op fra Hibrot Hatta'ava og slog Lejr i Hazerot.
\par 18 Så brød de op fra Hazerot og slog Lejr i Ritma.
\par 19 Så brød de op fra Ritma og slog Lejr i Rimmon Perez.
\par 20 Så brød de op fra Rimmon Perez og slog Lejr i Libna.
\par 21 Så brød de op fra Libna og slog Lejr i Aissa.
\par 22 Så brød de op fra Aissa og slog Lejr i Kehelata.
\par 23 Så brød de op fra Kebelata og slog Lejr ved Sjefers Bjerg.
\par 24 Så brød de op fra Sjefers Bjerg og slog Lejr i Harada.
\par 25 Så brød de op fra Harada og slog Lejr i Makhelot.
\par 26 Så brød de op fra Makhelot og slog Lejr i Tahat.
\par 27 Så brød de op fra Tahat og slog Lejr i Tara.
\par 28 Så brød de op fra Tara og slog Lejr i Mitka.
\par 29 Så brød de op fra Mitka og slog Lejr i Hasjmona.
\par 30 Så brød de op fra Hasjmona og slog Lejr i Moserot.
\par 31 Så brød de op fra Moserot og slog Lejr i Bene Ja'akan.
\par 32 Så brød de op fra Bene Ja'akan og slog Lejr i Hor Haggidgad.
\par 33 Så brød de op fra Hor Haggidgad og slog Lejr i Jotbata.
\par 34 Så brød de op fra Jofbata og slog Lejr i Abrona.
\par 35 Så brød de op fra Abrona og slog Lejr i Ezjongeber.
\par 36 Så brød de op fra Ezjongeber og slog Lejr i Zins Ørken, det er Kadesj.
\par 37 Så brød de op fra Kadesj og slog Lejr ved Bjerget Hor ved Randen af Edoms Land.
\par 38 Og Præsten Aron steg på HERRENs Bud op på Bjerget Hor og døde der i det fyrretyvende År efter Israelitternes Udvandring af Ægypten, på den første Dag i den femte Måned;
\par 39 og Aron var l23 År gammel, da han døde på Bjerget Hor.
\par 40 Men Kana'anæeren, Kongen af Arad, der boede, i Sydlandet i Kana'ans Land, hørte, at Israelitterne var under Fremrykning.
\par 41 Så brød de op fra Bjerget Hor og slog Lejr i Zalmona.
\par 42 Så brød de op fra Zalmona og slog Lejr i Punon.
\par 43 Så brød de op fra Punon og slog Lejr i Obot.
\par 44 Så brød de op fra Obot og slog Lejr i Ijje Ha'abarim ved Moabs Grænse.
\par 45 Så brød de op fra Ijje Ha'abarim og slog Lejr i det gaditiske Dibon.
\par 46 Så brød de op fra det gaditiske Dibon og slog Lejr i Almon Diblatajim.
\par 47 Så brød de op fra Almon Diblatajim og slog Lejr på Abarimbjergene over for Nebo.
\par 48 Så brød de op fra Abarimbjergene og slog Lejr på Moabs Sletter ved Jordan over for Jeriko;
\par 49 og de slog Lejr ved Jordan fra Bet Jesjjimot og til Abel Sjittim på Moabs Sletter.
\par 50 Og HERREN talede til Moses på Moabs Sletter ved Jordan over for Jeriko og sagde:
\par 51 "Tal til Israelitterne og sig til dem: Når I kommer over Jordan til Kana'ans Land,
\par 52 skal I drive Landets Beboere bort foran eder og tilintetgøre alle deres Billedværker, alle deres støbte Billeder skal I tilintetgøre, og alle deres Offerhøje skal I ødelægge;
\par 53 I skal underlægge eder Landet og bosætte eder der, thi eder har jeg givet Landet i Eje;
\par 54 og I skal udskifte Landet mellem eder ved Lodkastning efter eders Slægter, således at I giver en stor Slægt en stor Arvelod og en lille Slægt en lille. Der, hvor Loddet falder for dem, skal deres Del være; efter eders Fædrenestammer skal I udskifte Landet mellem eder.
\par 55 Men hvis I ikke driver Landets Beboere bort foran eder, så skal de, som I levner af dem, blive Torne i eders Øjne og Brodde i eders Sider, og de skal bringe eder Trængsel i det Land, I bor i,
\par 56 og hvad jeg havde tænkt at gøre ved dem, gør jeg da ved eder."

\chapter{34}

\par 1 HERREN talede fremdeles til Moses og sagde:
\par 2 Byd Israelitterne og sig til dem: Når i kommer til Kana'ans Land det er det Land, der skal tilfalde eder som Arvelod, Kana'ans Land i hele dets Udstrækning
\par 3 så skal eders Sydside strække sig fra Zins Ørken langs med Edom; eders Sydgrænse skal mod Øst begynde ved Enden af Salthavet;
\par 4 så skal eders Grænse dreje sønden om Akrabbimpasset, nå til Zin og ende sønden for Kadesj Barnea; så skal den løbe hen til Hazar Addar øg nå til Azmon;
\par 5 fra Azmon skal Grænsen dreje hen til Ægyptens Bæk og ende ved Havet.
\par 6 Hvad Vestgrænsen angår, skal det store Hav være eders Grænse; det skal være eders Vestgrænse.
\par 7 Eders Nordgrænse skal være følgende: Fra det store Hav skal I udstikke eder en Linie til Bjerget Hor;
\par 8 fra Bjerget Hor skal I udstikke en Linie til Egnen hen imod Hamat, så at Grænsen ender ved Zedad;
\par 9 derpå skal Grænsen gå til Zifron og ende ved Hazar Enan. Det skal være eders Nordgrænse.
\par 10 Men til Østgrænse skal I afmærke eder en Linie fra Hazar Enan til Sjefam;
\par 11 og fra Sjefam skal Grænsen gå ned til Ribla østen for Ajin, og Grænsen skal løbe videre ned, til den støder til Bjergskrænten østen for Kinneretsøen;
\par 12 derpå skal Grænsen løbe ned langs Jordan og ende ved Salthavet. Det skal være eders Land i hele dets Udstrækning til alle Sider.
\par 13 Og Moses bød Israelitterne og sagde: Det er det Land, I skal udskifte mellem eder ved Lodkastning, og som efter HERRENs Bud skal gives de ni Stammer og den halve Stamme.
\par 14 Thi Rubeniternes Stamme efter deres Fædrenehuse og Gaditernes Stamme efter deres Fædrenehuse og Manasses halve Stamme har allerede fået deres Arvelod.
\par 15 De to Stammer og den halve Stamme har fået deres Arvelod hinsides Jordan over for Jeriko, mod Øst, mod Solens Opgang.
\par 16 Og HERREN talede til Moses og sagde:
\par 17 Navnene på de Mænd, der skal udskifte Landet mellem eder, er følgende: Præsten Eleazar og Josua, Nuns Søn;
\par 18 desuden skal I udtage een Øverste af hver Stamme til at udskifte Landet.
\par 19 Navnene på disse Mænd er følgende: Af Judas Stamme Kaleb, Jefunnes Søn,
\par 20 af Simeoniternes Stamme Sjemuel, Ammihuds Søn,
\par 21 af Benjamins Stamme Elidad, Kislons Søn,
\par 22 af Daniternes Stamme een Øverste, Bukki, Joglis Søn,
\par 23 af Josefs Sønner: af Manassiternes Stamme een Øverste, Hanniel, Efods Søn,
\par 24 og af Efraimiternes Stamme een Øverste, Kemuel, Sjiftans Søn,
\par 25 af Zebuloniternes Stamme een Øverste, Elizafan, Parnaks Søn,
\par 26 af Issakariternes Stamme een Øverste, Paltiel, Azzans Søn,
\par 27 af Aseriternes Stamme een Øverste, Ahihud, Sjelomis Søn,
\par 28 og af Naftaliternes Stamme een Øverste, Pedael, Ammibuds Søn.
\par 29 Det var dem, HERREN pålagde at udskifte Kana'ans Land mellem Israelitterne.

\chapter{35}

\par 1 HERREN talede fremdeles til Moses på Moabs Sletter ved Jordan over for Jeriko og sagde:
\par 2 byd Israelitterne, at de af de Besiddelser, de får i Arv, skal give Leviterne nogle Byer at bo i; I skal også give Leviterne Græsmarker rundt om disse Byer,
\par 3 Disse Byer skal de have at bo i, og deres Græsmarker skal de have til deres Kvæg, deres Hjorde og andre Dyr.
\par 4 Græsmarkerne om Byerne, som I skal give Leviterne, skal strække sig 1000 Alen fra Bymuren ud til alle Sider;
\par 5 og uden for Byen skal I til Østside opmåle 2OOO Alen, til Sydside 2000, til Vestside 2OOO og til Nordside 2OOO, med Byen i Midten. Det skal tilfalde dem som Græsgange til Byerne.
\par 6 Hvad de Byer angår, som I skal give Leviterne, så skal I give dem de seks Tilflugtsbyer, som Manddrabere kan ty ind i, og desuden to og fyrretyve Byer.
\par 7 De Byer, I skal give Leviterne, bliver således i alt otte og fyrretyve Byer med tilhørende Græsmarker.
\par 8 Og af de Byer, I skal give dem af Israelitternes Besiddelser, skal I lade de større Stammer give flere, de mindre færre; hver Stamme skal give Leviterne så mange af sine Byer, som svarer til den Arvelod, der tildeles den.
\par 9 HERREN talede fremdeles til Moses og sagde:
\par 10 Tal til Israelitterne og sig til dem: Når I kommer over Jordan til Kana'ans Land,
\par 11 skal I udse eder nogle Byer, I kan have som Tilflugtsbyer, så at en Manddraber, der begår et Drab af Vanvare, kan ty derhen.
\par 12 I disse Byer skal I have Ret til at søge Tilflugt for Blodhævneren, for at ikke Manddraberen skal dø, før han er blevet stillet for Menighedens Domstol.
\par 13 Det skal være seks Byer, I skal afstå til Tilflugtsbyer;
\par 14 de tre skal I afstå hinsides Jordan og de tre andre i Kana'ans Land; de skal være Tilflugtsbyer.
\par 15 Israelitterne, de fremmede og de indvandrede iblandt dem skal have Ret til at søge Tilflugt i de seks Byer, så at enhver, der begår et Drab af Vanvare, kan ty derhen.
\par 16 Men slår han ham ihjel med et Jernredskab, så er han en Manddraber, og Manddraberen skal lide Døden;
\par 17 og slår han ham ihjel med en Sten, som han har i Hånden, og som kan slå en Mand ihjel, så er han en Manddraber, og Manddraberen skal lide Døden;
\par 18 og slår han ham ihjel med et Træredskab, som han har i Hånden, og som kan slå en Mand ihjel, så er han en Manddraber, og Måddraberen skal lide Døden.
\par 19 Blodhævneren skal dræbe Manddraberen; når han træffer ham, skal han dræbe ham.
\par 20 Og støder han til ham af Had eller kaster noget på ham i ond Hensigt, så han dør deraf,
\par 21 eller slår han ham med Hånden i Fjendskab, så han dør deraf, skal drabsmanden lide Døden, thi han er en Manddraber; Blodhævneren skal dræbe Manddraberen, når han træffer ham.
\par 22 Støder han derimod til ham af Vanvare, ikke i Fjendskab, eller kaster han et Redskab på ham, uden at det er i ond Hensigt,
\par 23 eller rammer han ham uden at se ham med en Sten, som kan slå en Mand ihjel, så han dør deraf, og han ikke er hans Uven eller har pønset på ondt imod ham,
\par 24 så skal Menigheden dømme Drabsmanden og Blodhævneren imellem på Grundlag af disse Lovbud;
\par 25 og Menigheden skal værne Manddraberen mod Blodhævneren, og Menigheden skal føre ham tilbage til hans Tilflugtsby, hvorhen han var tyet, og der skal han blive boende, indtil den med hellig Olie salvede Ypperstepræst dør.
\par 26 Men hvis Manddraberen for lader sin Tilflugtsbys Område, hvorhen han er tyet,
\par 27 og Blodhævneren træffer ham uden for hans Tilflugtsbys Område, så kan Blodhævneren dræbe Manddraberen uden at pådrage sig Blodskyld;
\par 28 thi han skal blive i sin Tilflugtsby indtil Ypperstepræstens Død; først efter Ypperstepræstens Død kan Manddraberen vende tilbage til den Jord, han ejer.
\par 29 Det skal være eder en retsgyldig Anordning fra Slægt til Slægt, hvor I end bor.
\par 30 Når nogen slår et Menneske ihjel, må man kun dræbe Manddraberen efter flere Vidners Udsagn. Et enkelt Vidnes Udsagn er ikke nok til en Dødsdom.
\par 31 I må ikke tage mod Sonebøde for en Manddraber, når han har forbrudt sit Liv; han skal lide Døden,
\par 32 Heller ikke må I tage mod Sonebøde, således at den, der er tyet til sin Tilflugtsby, før Ypperstepræstens Død kan vende tilbage og bosætte sig i Landet.
\par 33 Vanhelliger ikke det Land, I er i, thi Blodet vanhelliger Landet, og Landet får kun Soning for det Blod, der er udgydt deri, ved dens Blod, der har udgydt det.
\par 34 Gør ikke det Land urent, I er bosat i, og i hvis Midte jeg bor; thi jeg HERREN bor midt iblandt Israels Børn.

\chapter{36}

\par 1 Overhovederne for Fædrenehsene i Gileaditernes Slægt Gilead var en Søn af Manasses Søn Makir af Josefs Sønners Slægter trådte frem og talte for Moses og Øversterne, Overhovederne for Israelitternes Fædrenehuse,
\par 2 og sagde: "HERREN har pålagt min Herre at udskifte Landet mellem Israelitterne ved Lodkastning, og min Herre har i HERRENs Navn påbudt at give vor Frænde Zelofhads Arvelod til hans Døtre.
\par 3 Men hvis de nu indgår Ægteskab med Mænd, der hører til en anden af Israels Stammer, så unddrages deres Arvelod jo vore Fædres Arvelod, og således øges den Stammes Arvelod, som de kommer til at tilhøre, medens den Arvelod, der er tilfaldet os ved Lodkastning, formindskes;
\par 4 og når Israelitterne får Jubelår, lægges deres Arvelod til den Stammes Arvelod, som de kommer til at tilhøre, og således unddrages deres Arvelod vor fædrene Stammes Arvelod."
\par 5 Da udstedte Moses efter HERRENs Ord følgende Bud til Israelitterne: "Josefs Sønners Stamme har talt ret!
\par 6 Således er HERRENs Bud angående Zelofhads Døtre: De må indgå Ægteskab med hvem de ønsker, men det må kun være Mænd af deres Faders Stammes Slægt, de indgår Ægteskab med.
\par 7 Thi ingen Arvelod, der tilhører Israel, må gå over fra en Stamme til en anden, men Israelitterne skal holde fast hver ved sin fædrene Stammes Arvelod.
\par 8 Enhver Datter, som får en Arvelod i en af Israelitternes Stammer, skal indgå Ægteskab med en Mand af sin fædrene Stammes Slægt, for at Israelitterne kan beholde hver sine Fædres Arvelod som Ejendom;
\par 9 og ingen Arvelod må gå over fra den ene Stamme til den anden, men Israelitternes Stammer skal holde fast hver ved sin Arvelod!"
\par 10 Zelofhads Døtre gjorde da, som HERREN pålagde Moses,
\par 11 idet Mala, Tirza, Hogla, Milka og Noa, Zelofhads Døtre, indgik Ægteskab med deres Farbrødres Sønner;
\par 12 de indgik Ægteskab med Mænd, som hørte til Josefs Søn Manasses Sønners Slægter, så at deres Arvelod vedblev at tilhøre deres fædrene Slægts Stamme.
\par 13 Det er de Bud og Lovbud, HERREN gav Israelitterne ved Moses på Moabs Sletter ved Jordan over for Jeriko.


\end{document}