\begin{document}

\title{Josvabogen}


\chapter{1}

\par 1 Efter at HERRENs Tjener Moses var død, sagde HERREN til Moses's Medhjælper Josua, Nuns Søn:
\par 2 "Min Tjener Moses er død; bryd nu op tillige med hele dette Folk og gå over Jordan derhenne til det Land, jeg vil give dem, Israeliterne.
\par 3 Ethvert Sted, eders Fod betræder, giver jeg eder, som jeg lovede Moses.
\par 4 Fra Ørkenen og Libanon til den store Flod, Eufratfloden, hele Hetiternes Land, og til det store Hav i Vest skal eders Landemærker nå.
\par 5 Så længe du lever, skal det ikke være muligt for nogen at holde Stand imod dig; som jeg var med Moses, vil jeg være med dig; jeg vil ikke slippe dig og ikke forlade dig.
\par 6 Vær frimodig og stærk, thi du skal skaffe dette Folk det Land i Eje, som jeg tilsvor deres Fædre at ville give dem.
\par 7 Vær kun helt frimodig og stærk. så du omhyggeligt handler efter hele den Lov, min Tjener Moses pålagde dig, vig ikke derfra til højre eller venstre, for at du må have Lykken med dig i alt, hvad du tager dig for.
\par 8 Denne Lovbog skal ikke vige fra din Mund, og du skal grunde over den Dag og Nat, for at du omhyggeligt kan handle efter alt, hvad der står skrevet i den; thi da vil det gå dig vel i al din Færd, og Lykken vil følge dig.
\par 9 Har jeg ikke budt dig: Vær frimodig og stærk; vær ikke bange og bliv ikke forfærdet, thi HERREN din Gud er med dig i alt, hvad du tager dig for!"
\par 10 Josua bød derpå Folkets Tilsynsmænd:
\par 11 "Gå omkring i Lejren og byd Folket: Sørg for Rejsetæring, thi om tre Dage skal I gå over Jordan derhenne for at drage ind og tage det Land i Besiddelse, som HERREN eders Gud vil give eder i Eje!"
\par 12 Men til Rubeniterne, Gaditerne og Manasses halve Stamme sagde Josua:
\par 13 "Husk på, hvad HERRENs Tjener Moses bød eder, da han sagde: HERREN eders Gud bringer eder nu til Hvile og giver eder Landet her!
\par 14 Eders Kvinder og Børn og Kvæg skal blive i det Land, Moses gav eder hinsides Jordan; men I selv, alle våbenføre, skal væbnet drage over i Spidsen for eders Brødre og hjælpe dem,
\par 15 indtil HERREN har bragt eders Brødre til Hvile ligesom eder, når også de har taget det Land i Besiddelse, som HERREN eders Gud vil give dem. Så kan I vende tilbage til eders eget Land og tage det i Besiddelse, det, som HERRENs Tjener Moses gav eder østpå hinsides Jordan!"
\par 16 Da svarede de Josua: "Alt, hvad du har pålagt os, vil vi gøre, og vi vil gå overalt, hvor du sender os;
\par 17 som vi har adlydt Moses i alt, vil vi adlyde dig. Måtte kun HERREN din Gud være med dig, som han var med Moses!
\par 18 Enhver, som sætter sig op imod dine Befalinger og ikke adlyder dine Ord i alt, hvad du pålægger ham, skal dø. Vær kun modig og uforsagt!"

\chapter{2}

\par 1 Josua, Nuns Søn, udsendte hemmeligt fra Sjittim to Mænd som Spejdere med den Befaling: "Gå hen og undersøg Landet, særlig Jeriko!" De gav sig da på Vej og kom ind i et Hus til en Skøge ved Navn Rahab, og der lagde de sig til Hvile.
\par 2 Da blev der meldt Kongen af Jeriko: "Se, der er kommet nogle israelitiske Mænd hertil i Nat for at udspejde Landet!"
\par 3 Og Kongen af Jeriko sendte Bud til Rahab og lod sige: "Udlever de Mænd, som er kommet til dig; thi de er kommet for at udspejde hele Landet!"
\par 4 Men Kvinden tog og skjulte de to Mænd og sagde: "Ja, Mændene kom ganske vist til mig, men jeg vidste ikke, hvor de var fra;
\par 5 og da Porten skulde lukkes i Mørkningen, gik Mændene bort. Jeg ved ikke, hvor de gik hen! Sæt hurtigt efter dem, så kan I indhente dem!"
\par 6 Men hun havde ført dem op på Taget og skjult dem under Hørstænglerne, som hun havde bredt ud på Taget.
\par 7 Mændene satte så efter dem ad Vejen til Jordan hen til Vadestederne, og Porten blev lukket, så snart Forfølgerne var udenfor.
\par 8 Før de to Mænd endnu havde lagt sig, kom hun op på Taget til dem;
\par 9 og hun sagde til dem: "Jeg ved, at HERREN har givet eder Landet, thi vi er grebet af Rædsel for eder, og alle Landets Indbyggere er skrækslagne over eder.
\par 10 Thi vi har hørt, hvorledes HERREN lod Vandet i det røde Hav tørre bort foran eder, da I drog ud af Ægypten, og hvad I gjorde ved de to Amoriterkonger hinsides Jordan, Sihon og Og, på hvem I lagde Band.
\par 11 Da vi hørte det, blev vi slaget af Rædsel og tabte alle Modet over for eder; thi HERREN eders Gud er Gud oppe i Himmelen og nede på Jorden.
\par 12 Men tilsværg mig nu ved HERREN, at I vil vise Godhed mod mit Fædrene hus, ligesom jeg har vist Godhed mod eder, og giv mig et sikkert Tegn på,
\par 13 at I vil lade min Fader og Moder, mine Brødre og Søstre og alt, hvad deres er, blive i Live og redde os fra Døden!"
\par 14 Da sagde Mændene til hende: "Vi indestår med vort Liv for eder, hvis du ikke røber vort Forehavende; og når HERREN giver os Landet, vil vi vise dig Godhed og Trofasthed!"
\par 15 Derpå hejsede hun dem med et Reb ned gennem Vinduet, thi hendes Hus lå ved Bymuren, og hun boede ved Muren;
\par 16 og hun sagde til dem: "Gå ud i Bjergene, for at eders Forfølgere ikke skal træffe eder, og hold eder skjult der i tre Dage, indtil eders Forfølgere er vendt tilbage; efter den Tid kan I gå eders Vej!"
\par 17 Mændene sagde til hende: "På denne Måde vil vi bære os ad med at holde den Ed, du har ladet os sværge
\par 18 Når vi kommer ind i Landet, skal du binde denne røde Snor fast i det Vindue, du har hejst os ned igennem, og så skal du samle din Fader og din Moder, dine Brødre og hele dit Fædrenehus hos dig i Huset.
\par 19 Enhver, som så går uden for din Husdør, må selv tage Ansvaret for sit Liv, uden at der falder Skyld på os; men hvis der lægges Hånd på nogen af dem, som bliver i Huset hos dig, hviler Ansvaret på os.
\par 20 Hvis du derimod røber vort Forehavende, er vi løst fra den Ed, du lod os sværge!"
\par 21 Da sagde hun: "Lad det være, som I siger!" Så lod hun dem drage bort; og hun bandt den røde Snor fast i Vinduet.
\par 22 Men de begav sig ud i Bjergene og blev der i tre Dage, indtil deres Forfølgere var vendt tilbage. Og Forfølgerne ledte overalt på Vejen uden at finde dem.
\par 23 Derpå begav de to Mænd sig på Tilbagevejen, og efter at være steget ned fra Bjergene gik de over Floden; og de kom til Josua, Nuns Søn, og fortalte ham alt, hvad der var hændet dem.
\par 24 Og de sagde til Josua: "HERREN har givet hele Landet i vor Magt, og alle Landets Indbyggere er skrækslagne over os!"

\chapter{3}

\par 1 Tidligt næste Morgen gjorde Josua sig rede, og hele Israel brød op fra Sjittim sammen med ham og kom til Jordan, og de tilbragte Natten der, før de drog over.
\par 2 Efter tre Dages Forløb gik Tilsynsmændene omkring i Lejren
\par 3 og bød Folket: "Når I ser HERREN eders Guds Pagts Ark og Levitpræsterne komme bærende med den, så skal I bryde op fra eders Plads og følge efter
\par 4 dog skal der være en Afstand af 2000 Alen mellem eder og den; I må ikke komme den for nær for at I kan vide, hvilken Vej I skal gå; thi I er ikke kommet den Vej før!"
\par 5 Og Josua sagde til Folket: "Helliger eder; thi i Morgen vil HERREN gøre Undere iblandt eder!"
\par 6 Og Josua sagde til Præsterne: "Løft Pagtens Ark op og drag over foran Folket!" Så løftede de Pagtens Ark op og gik foran Folket.
\par 7 Men HERREN sagde til Josua: "I Dag begynder jeg at gøre dig stor i hele Israels Øjne, for at de kan vide, at jeg vil være med dig, som jeg var med Moses.
\par 8 Du skal byde Præsterne, som bærer Pagtens Ark: Når I kommer til Kanten af Jordans Vand, skal I standse der ved Jordan!"
\par 9 Da sagde Josua til Israeliterne: "Kom hid og hør HERREN eders Guds Ord!"
\par 10 Og Josua sagde: "Derpå skal I kende, at der er en levende Gud iblandt eder, og at han vil drive Kana'anæerne, Hetiterne, Hivviterne, Perizziterne, Girgasjiterne, Amoriterne og Jebusiterne bort foran eder:
\par 11 Se, HERRENs, al Jordens Herres, Ark skal gå foran eder gennem Jordan.
\par 12 Vælg eder nu tolv Mænd af Israels Stammer, een Mand af hver Stamme.
\par 13 Og så snart Præsterne, som bærer HERRENs, al Jordens Herres, Ark, sætter Foden i Jordans Vand, skal Jordans Vand standse, det Vand, som kommer ovenfra, og stå som en Vold."
\par 14 Da Folket så brød op fra deres Telte for at gå over Jordan med Præsterne, som bar Arken, i Spidsen,
\par 15 og da de, som bar Arken, kom til Jordan, og Præsterne, som bar Arken, rørte ved Vandkanten med deres Fødder Jordan gik overalt over sine Bredder i hele Høsttiden
\par 16 standsede Vandet, som kom ovenfra, og stod som en Vold langt borte, oppe ved Byen Adam, som ligger ved Zaretan, medens det Vand, som flød ned mod Arabaeller Salthavet, løb helt bort; således gik Folket over lige over for Jeriko.
\par 17 Men Præsterne, som bar HERRENs Pagts Ark, blev stående på tør Bund midt i Jordan, medens hele Israel gik over på tør Bund, indtil hele Folket havde tilendebragt Overgangen over Jordan.

\chapter{4}

\par 1 Da nu hele Folket havde tilendebragt Overgangen over Jordan, sagde HERREN til Josua:
\par 2 "Vælg eder tolv Mænd af Folket, een Mand af hver Stamme,
\par 3 og byd dem: Tag eder tolv Sten her, midt i Jordan, hvor Præsterne stod stille, bring dem med over og stil dem på den Plads, hvor I holder Rast i Nat!"
\par 4 Så lod Josua de tolv Mænd kalde, som han havde til Rede af Israeliterne, een Mand af hver Stamme;
\par 5 og Josua sagde til dem: "Gå foran HERREN eders Guds Ark midt ud i Jordan, og tag så hver en Sten på Skulderen, svarende til Tallet på Israeliternes Stammer,
\par 6 for at det kan tjene til Tegn iblandt eder. Når eders Børn i Fremtiden spørger: Hvad Betydning har disse Sten for eder?
\par 7 så skal I sige til dem: De betyder, at Jordans Vand standsede foran HERRENs Pagts Ark; da den drog over Jordan, standsede Jordans Vand. Og disse Sten skal være Israeliterne et Mindetegn til evig Tid!"
\par 8 Da gjorde Israeliterne, som Josua bød, og tog tolv Sten midt i Jordan, som HERREN havde sagt til Josua, svarende til Tallet på Israeliternes Stammer, og de bragte dem med over til det Sted, hvor de holdt Rast, og stillede dem der.
\par 9 Og tolv Sten rejste Josua midt i Jordan på det Sted, hvor Præsterne, som bar Pagtens Ark, stod stille, og der står de den Dag i Dag.
\par 10 Men Præsterne, som bar Arken, blev stående midt i Jordan, indtil alt, hvad HERREN havde pålagt Josua at sige til Folket, var udført, i Overensstemmelse med alt, hvad Moses havde pålagt Josua; og Folket gik skyndsomt over.
\par 11 Da hele Folket så havde tilendebragt Overgangen, gik HERRENs Ark og Præsterne over og stillede sig foran Folket.
\par 12 Og Rubeniterne, Gaditerne og Manasses halve Stamme drog væbnet over i Spidsen for Israeliterne, som Moses havde sagt til dem;
\par 13 henved 40000 Mand i Tal, rustede til Strid, drog de foran HERREN over til Jerikos Sletter til Hamp.
\par 14 På den Dag gjorde HERREN Josua stor i hele Israels Øjne, og de frygtede ham alle hans Livs Dage, som de havde frygtet Moses.
\par 15 Da sagde HERREN til Josua:
\par 16 "Byd Præsterne, som bærer Vidnesbyrdets Ark, at stige op fra Jordan!"
\par 17 Og Josua bød Præsterne: "Stig op fra Jordan!"
\par 18 Så steg Præsterne, som bar HERRENs Pagts Ark, op fra Jordan, og næppe havde deres Fødder betrådt det tørre Land, før Jordans Vand vendte tilbage til sit Leje og overalt gik over sine Bredder som før.
\par 19 Og Folket steg op fra Jordan den tiende Dag i den første Måned og slog Lejr i Gilgal ved Østenden af Jerikolandet.
\par 20 Men de tolv Sten, som de havde taget op fra Jordan, rejste Josua i Gilgal,
\par 21 og han sagde til Israeliterne: "Når eders Børn i Fremtiden spørger deres Fædre: Hvad betyder disse Sten?
\par 22 så skal I fortælle eders Børn det og sige: På tør Bund gik Israel over Jordan derhenne;
\par 23 thi HERREN eders Gud lod Jordans Vand tørre bort foran eder, indtil I var kommet over, ligesom HERREN eders Gud gjorde med det røde Hav, som han lod tørre bort foran os, indtil vi var kommet over,
\par 24 for at alle Jordens Folk skal kende, at HERRENs Arm er stærk, at de må frygte HERREN eders Gud alle Dage."

\chapter{5}

\par 1 Men da alle Amoriterkongerne vesten for Jordan og alle Kana'anæerkongerne ved Havet hørte, at HERREN havde ladet Jordans Vand tørre bort foran Israeliterne, indtil de var gået over, blev de slaget af Rædsel og tabte Modet over for Israeliterne.
\par 2 På den Tid sagde HERREN til Josua: "Lav dig Stenknive og omskær på ny Israeliterne!"
\par 3 Da lavede Josua sig Stenknive og omskar Israeliterne ved Forhudshøjen.
\par 4 Og dette var Grunden til, at Josua omskar dem: Alt Mandkøn af Folket, som drog ud af Ægypten, alle våbenføre Mænd, var døde undervejs i Ørkenen, efter at de var draget ud af Ægypten.
\par 5 Hele Folket, som drog ud, havde nok været omskåret, men af det Folk, som var født i Ørkenen under Vandringen, efter at de var draget ud af Ægypten, var ingen blevet omskåret;
\par 6 thi i fyrretyve År havde Israeliterne vandret i Ørkenen, indtil hele Folket, de våbenføre Mænd, som var draget ud af Ægypten, var døde, fordi de ikke havde adlydt HERRENs Røst, hvorfor HERREN havde svoret, at han ikke vilde lade dem se det Land, HERREN havde tilsvoret deres Fædre at ville give os, et Land, der flyder med Mælk og Honning.
\par 7 Men deres Børn, som han havde ladet træde i deres Sted, dem omskar Josua; thi de var uomskårne, eftersom de ikke var blevet omskåret under Vandringen.
\par 8 Da så hele Folket var blevet omskåret, holdt de sig i Ro, hvor de var i Lejren, indtil de kom sig.
\par 9 Men HERREN sagde til Josua: "I Dag har jeg bortvæltet Ægypternes Forsmædelse fra eder." Og han kaldte dette Sted Gilgal, som det hedder den Dag i Dag.
\par 10 Medens Israeliterne lå i Lejr i Gilgal, fejrede de Påsken om Aftenen den fjortende Dag i Måneden på Jerikos Sletter;
\par 11 og Dagen efter Påsken spiste de af Landets Afgrøde, usyrede Brød og ristet Korn;
\par 12 selv samme Dag hørte Mannaen op, da de nu spiste af Landets Afgrøde; Israeliterne fik ikke Manna mer, men spiste det År af Høsten i Kana'ans Land.
\par 13 Og det skete, medens Josua opholdt sig ved Jeriko, at han så op og fik Øje på en Mand, som stod foran ham med draget Sværd i Hånden. Josua gik da hen til ham og sagde: "Er du en af vore eller en af vore Fjender?"
\par 14 Han svarede: "Ingen af Delene, jeg er Fyrsten over HERRENs Hær; lige nu er jeg kommet!" Da faldt Josua til Jorden på sit Ansigt og tilbad og sagde til ham: "Hvad har min Herre at sige sin Tjener?"
\par 15 Og Fyrsten over HERRENs Hær svarede Josua: "Drag dine Sko af Fødderne, thi det Sted, du står på, er helligt!" Det gjorde Josua.

\chapter{6}

\par 1 Men Jeriko var lukket og stængt for Israelitterne, ingen gik ud eller ind.
\par 2 Da sagde HERREN til Josua: "Se, jeg giver Jeriko og dets Konge og Krigere i din Hånd.
\par 3 Alle eders våbenføre Mænd skal gå rundt om Byen, een Gang rundt; det skal I gøre seks Dage;
\par 4 og syv Præster skal bære syv Væderhorn foran Arken. Men den syvende Dag skal I gå rundt om Byen syv Gange, og Præsterne skal støde i Hornene.
\par 5 Når der så blæses i Væderhornet, og I hører Hornets Lyd, skal alt Folket opløfte et vældigt Krigsskrig; så skal Byens Mur styrte sammen, og Folket kan gå lige ind, hver fra sin Plads."
\par 6 Josua, Nuns Søn, lod da Præsterne kalde og sagde til dem: "I skal bære Pagtens Ark, og syv Præster skal bære syv Væderhorn foran HERRENs Ark!"
\par 7 Og han sagde til Folket: "Drag ud og gå rundt om Byen, således at de, som bærer Våben, går foran HERRENs Ark!"
\par 8 Da nu Josua havde talt til Folket, gik de syv Præster, som bar de syv Væderhorn foran HERREN, frem og stødte i Hornene, medens HERRENs Pagts Ark fulgte efter;
\par 9 og de, som bar Våben, gik foran Præsterne, som stødte i Hornene, og de, som sluttede Toget, fulgte efter Arken, medens der blæstes i Hornene.
\par 10 Men Josua bød Folket: "I må ikke opløfte Krigsskrig eller lade eders Røst høre, og intet Ord må udgå af eders Mund, før den Dag jeg siger til eder: Råb! Men så skal I råbe!"
\par 11 Så lod han HERRENs Ark bære rundt om Byen, een Gang rundt, og derpå begav de sig tilbage til Lejren og overnattede der.
\par 12 Tidligt næste Morgen gjorde Josua sig rede, og Præsterne bar HERRENs Ark,
\par 13 og de syv Præster, som bar de syv Væderhorn foran HERRENs Ark, gik og stødte i Hornene; de, som bar Våben, gik foran dem, og de, som sluttede Toget, fulgte efter HERRENs Ark, medens der blæstes i Hornene.
\par 14 Anden Dag gik de een Gang rundt om Byen, hvorefter de vendte tilbage til Lejren; således gjorde de seks Dage.
\par 15 Men den syvende Dag brød de op tidligt, ved Morgenrødens frembrud og gik på samme Måde syv Gange rundt om Byen; kun på denne Dag drog de syv Gange rundt om Byen;
\par 16 og syvende Gang stødte Præsterne i Hornene, og Josua sagde til Folket: "Opløft Krigsskriget! Thi HERREN har givet eder Byen.
\par 17 Og Byen skal lyses i Band for HERREN med alt, hvad der er i den; kun Skøgen Rahab skal blive i Live tillige med alle dem, som er i hendes Hus, fordi hun skjulte Sendebudene, som vi udsendte.
\par 18 Men I skal tage eder vel i Vare for det bandlyste, så I ikke attrår og tager noget af det bandlyste og derved bringer Bandet over Israels lejr og styrter den i Ulykke.
\par 19 Men alt Sølv og Guld og alle Kobber og Jernsager skal helliges HERREN; det skal bringes ind i HERRENs Skatkammer!"
\par 20 Så opløftede Folket Krigsskrig, og de stødte i Hornene og da Folket hørte Hornene, opløftede det et vældigt Krigsskrig; da styrtede Muren sammen, og Folket gik lige ind i Byen; således indtog de Byen.
\par 21 Derpå lagde de med Sværdet Band på alt, hvad der var i Byen, Mænd og Kvinder, unge og gamle, Hornkvæg, Småkvæg og Æsler.
\par 22 Men til de to Mænd, som havde udspejdet Landet, sagde Josua: "Gå ind i Skøgens Hus og før Kvinden og alt, hvad hendes er, ud derfra, som I tilsvor hende!"
\par 23 De unge Mænd, som havde været Spejdere, gik da ind og førte Rabab ud tillige med hendes Fader og Moder og hendes Brødre, hele hendes Slægt og alt, hvad hendes var; de førte dem ud og lod dem stå uden for Israels Lejr.
\par 24 Men på Byen og alt, hvad der var i den, stak de Ild; kun Sølvet og Guldet og Kobber og Jernsagerne bragte de ind i HERRENs Hus's Skatkammer.
\par 25 Men Skøgen Rahab og hendes Fædrenehus og alt, hvad hendes var, lod Josua blive i Live, og hun kom til at bo blandt Israeliterne og gør det den Dag i Dag, fordi hun havde skjult Sendebudene, som Josua havde sendt ud for at udspejde Jeriko.
\par 26 På den Tid tog Josua Folket i Ed, idet han sagde: "Forbandet være den Mand for HERRENs Åsyn, som indlader sig på at opbygge denne By, Jeriko. Det skal koste ham hans førstefødte at lægge Grunden og hans yngste at sætte dens Portfløje ind."
\par 27 Således var HERREN med Josua, og hans Ry udbredte sig over hele Landet.

\chapter{7}

\par 1 Men Israeliterne forgreb sig på det bandlyste, idet Akan, en Søn af Karmi en Søn af Zabdi, en Søn af Zera, af Judas Stamme, tilvendte sig noget af det bandlyste. Da blussede HERRENs Vrede op mod Israeliterne.
\par 2 Derpå sendte Josua nogle Mænd fra Jeriko til Aj, som ligger ved Bet Aven, østen for Betel, og sagde til dem: "Drag op og udspejd Egnen!" Og Mændene drog op og udspejdede Aj.
\par 3 Da de kom tilbage til Josua, sagde de til ham: "Lad ikke hele Folket drage derop; lad en to-tre Tusind Mand drage op og indtage Aj; du behøver ikke at umage hele Folket med at drage derop, thi de er få!"
\par 4 Så drog henved 3000 Mand af Folket derop; men de blev slået på Flugt af Ajjiterne,
\par 5 og Ajjiterne dræbte seks og tredive Mand eller så af dem; de forfulgte dem uden for Porten indtil Stenbruddene og huggede dem ned på Skråningen. Da sank Folkets Mod og blev til Vand.
\par 6 Men Josua og Israels Ældste sønderrev deres Klæder og faldt på deres Ansigt på Jorden foran HERRENs Ark og blev liggende til Aften og kastede Støv på deres Hoveder.
\par 7 Og Josua sagde: "Ak, Herre, HERRE! Hvorfor lod du dette Folk gå over Jordan, når du vilde give os i Amoriternes Hånd og lade os gå til Grunde? Havde vi dog blot besluttet os til at blive hinsides Jordan!
\par 8 Ak, Herre! Hvad skal jeg sige, nu Israel har måttet tage Flugten for sine Fjender?
\par 9 Når Kana'anæerne og alle Landets Indbyggere hører det, falder de over os fra alle Sider og udsletter vort Navn af Jorden; hvad vil du da gøre for dit store Navns Skyld?"
\par 10 Da sagde HERREN til Josua: "Stå op! Hvorfor ligger du på dit Ansigt?
\par 11 Israel har syndet, thi de har forbrudt sig imod min Pagt, som jeg pålagde dem; de har tilvendt sig noget af det bandlyste, de har stjålet; de har skjult det, de har gemt det i deres Oppakning.
\par 12 Derfor kan Israeliterne ikke holde Stand over for deres Fjender, men må flygte for dem; thi de er hjemfaldne til Band! Jeg vil ikke mere være med eder, hvis I ikke bortrydder Bandet af eders Midte.
\par 13 Stå derfor op, lad Folket hellige sig og sig: Helliger eder til i Morgen, thi så siger HERREN, Israels Gud: Der er Band i din Midte, Israel; og du kan ikke holde Stand over for dine Fjender, før I skaffer Bandet bort fra eder!
\par 14 I Morgen skal I træde frem Stamme for Stamme, og den Stamme, HERREN rammer, skal træde frem Slægt for Slægt, og den Slægt, HERREN rammer, skal træde frem Familie for Familie, og den Familie, HERREN rammer, skal træde frem Mand for Mand.
\par 15 Den, som da rammes, fordi han har det bandlyste Gods, skal brændes tillige med alt, hvad der tilhører ham; thi han har brudt HERRENs Pagt og begået en Skændselsdåd i Israel!"
\par 16 Tidligt næste Morgen lod Josua Israel træde frem Stamme for Stamme, og da blev Judas Stamme ramt.
\par 17 Derpå lod han Judas Slægter træde frem, og Zeraiternes Slægt blev ramt. Derpå lod han Zeraiternes Slægt træde frem Familie for Familie, og Zabdi blev ramt.
\par 18 Derpå lod han dennes Familie træde frem Mand for Mand, og da blev Akan ramt, en Søn af Harmi, en Søn af Zabdi, en Søn af Zera, af Judas Stamme.
\par 19 Da sagde Josua til Akan: "Min Søn, giv HERREN, Israels Gud, Ære og Pris, og fortæl mig, hvad du har gjort, skjul ikke noget for mig!"
\par 20 Akan svarede Josua: "Ja, det er mig, som har syndet mod HERREN, Israels Gud. Således gjorde jeg:
\par 21 Jeg så imellem Byttet en prægtig babylonisk Kappe, 200 Sekel Sølv og en Guldtunge på halvtredsindstyve Sekel; det fik jeg Lyst til, og jeg tog det; se, det ligger nedgravet i Jorden midt i mit Telt, Sølvet nederst."
\par 22 Da sendte Josua nogle Folk hen, og de skyndte sig til Teltet, og se, det var gemt i hans Telt, Sølvet nederst;
\par 23 og de tog det ud af Teltet og bragte det til Josua og alle Israeliterne og lagde det hen foran HERREN.
\par 24 Men Josua tog sammen med hele Israel Akan, Zeras Søn, og Sølvet, Kappen og Guldtungen og hans Sønner og Døtre, hans Hornkvæg, Æsler og Småkvæg, hans Telt og alt, hvad der tilhørte ham, og førte dem op i Akors dal.
\par 25 Og Josua sagde: "Hvorfor har du styrtet os i Ulykke? HERREN skal styrte dig i Ulykke på denne Dag!" Derpå stenede hele Israel ham, og de brændte eller stenede dem.
\par 26 Og de opkastede en stor Stendysse over ham, som står der den Dag i Dag. Da lagde HERRENs heftige Vrede sig. Derfor fik Stedet Navnet Akors Dal, som det hedder den Dag i Dag.

\chapter{8}

\par 1 Derefter sagde HERREN til Josua: "Frygt ikke og vær ikke bange! Tag alt Krigsfolket med dig, bryd op og drag mod Aj! Se, jeg giver Kongen af Aj og hans Folk, hans By og hans Land i din Hånd!
\par 2 Og du skal gøre det samme ved Aj og dets Konge, som du gjorde ved Jeriko og dets Konge; men Godset, I røver der, må I selv beholde som Bytte, ligeledes Kvæget der. Læg Baghold vesten for Byen!"
\par 3 Derpå brød Josua op og drog med alt Krigsfolket op mod Aj; og Josua udvalgte sig 30000 tapre Mænd og sendte dem bort om Natten,
\par 4 idet han bød dem: "Se, I skal lægge eder i Baghold vesten for Byen, ikke for langt fra den, og hold eder alle rede.
\par 5 Jeg og alle de Folk, som er med mig, vil nærme os Byen, og når de gør Udfald imod os ligesom forrige Gang, flygter vi for dem.
\par 6 Når de så følger efter os så langt, at vi får dem lokket bort fra Byen, idet de tænker, at vi flygter for dem ligesom forrige Gang,
\par 7 skal I bryde frem fra Bagholdet og tage Byen, thi HERREN eders Gud vil give den i eders Hånd.
\par 8 Og når I har indtaget byen, skal I stikke den i Brand. Således skal I gøre, det er mit Bud til eder!"
\par 9 Derpå sendte Josua dem bort, og de gik hen og lagde sig i Baghold mellem Betel og Aj, vesten for Aj; men Josua blev den Nat iblandt Krigsfolket.
\par 10 Tidligt næste Morgen mønstrede Josua Folket, og sammen med Israels Ældste drog han op til Aj i Spidsen for Folket.
\par 11 Hele den Styrke, som var med ham, rykkede nærmere, til de havde Byen foran sig; og de lejrede sig norden for Aj med Dalen mellem sig og Aj.
\par 12 Derpå tog han henved 5000 Mand og lagde dem i Baghold mellem Betel og Aj, vesten for Byen;
\par 13 og Krigerne blev bragt i Stilling, hele Hæren, som var norden for Byen, og den Del, som lå i Baghold vesten for Byen; men Josua begav sig om Natten ned i Dalen.
\par 14 Da nu Kongen af Aj så det, skyndte Byens Mænd sig og rykkede tidligt om Morgenen ud til Kamp mod Israel, Kongen med hele sin Styrke, til Skråningen, hvor Lavningen begynder, uden at vide af, at der var lagt Baghold imod ham vesten for Byen;
\par 15 og da Josua og hele Israel lod sig slå på Flugt af dem og flygtede ad Ørkenen til,
\par 16 blev alt Folket i Byen stævnet sammen til at forfølge dem, og de forfulgte Josua og lod sig lokke bort fra Byen;
\par 17 ikke een Mand blev tilbage i Aj, men alle drog de ud for at forfølge Israel, og de lod Byen stå åben, medens de forfulgte Israel.
\par 18 Da sagde HERREN til Josua: "Ræk Spydet i din Hånd ud mod Aj; thi jeg giver det i din Hånd!" Og Josua rakte Spydet i sin Hånd ud mod Byen.
\par 19 Og Bagholdet brød hurtigt op fra sin Plads og løb hen til Byen, da han rakte Hånden ud, indtog den og skyndte sig at stikke den i Brand.
\par 20 Da Mændene fra Aj vendte sig om og så Røgen fra Byen stige op mod Himmelen, var de ude af Stand til at flygte til nogen af Siderne, idet Folket, som var flygtet ad Ørkenen til, vendte sig om mod deres Forfølgere.
\par 21 Og da Josua og hele Israel så, at Bagholdet havde indtaget Byen, og at Røgen fra byen steg til Vejrs, vendte de om og slog Mændene fra Aj;
\par 22 og da hine rykkede ud fra Byen imod dem, kom de midt ind imellem Israeliternes to Afdelinger, som huggede dem ned uden at lade en eneste af dem undkomme eller slippe bort.
\par 23 Men Kongen af Aj fangede de levende og bragte ham til Josua.
\par 24 Da Israel nu havde hugget alle Ajs Indbyggere ned på åben Mark, på Skråningen, hvor de forfulgte dem, og de alle til sidste Mand var faldet for Sværdet, vendte hele Israel tilbage til Aj og slog det med Sværdet.
\par 25 Alle de, der faldt på den Dag, Mænd og Kvinder, udgjorde 12000, alle Indbyggerne i Aj.
\par 26 Og Josua trak ikke sin Hånd med det udrakte Spyd tilbage, før han havde lagt Band på alle Ajs Indbyggere.
\par 27 Kun Kvæget og Godset, de røvede i denne By, beholdt Israeliterne som Bytte efter den Befaling, HERREN havde givet Josua.
\par 28 Og Josua stak Aj i Brand og gjorde den til en Ruinhob for bestandig, til den Ødemark, den er den Dag i Dag.
\par 29 Men Kongen af Aj hængte han på en Pæl og lod ham hænge til Aften; og da Solen gik ned, tog man på Josuas Befaling hans Lig ned af Pælen og kastede det hen ved Indgangen til Byens Port. Og man opkastede over ham en stor Stendysse, som står der den Dag i Dag.
\par 30 Da byggede Josua HERREN, Israels Gud, et Alter på Ebals Bjerg,
\par 31 som HERRENs Tjener Moses havde pålagt Israeliterne, i Overensstemmelse med, hvad der står skrevet i Moses's Lovbog, et Alter af utilhugne Sten, hvor over der ikke var svunget Jern; og de bragte HERREN Brændofre og ofrede Takofre der.
\par 32 Og han skrev der på Stenene i Israeliternes Påsyn en Afskrift af Mose Lov, som denne havde skrevet,
\par 33 idet hele Israel og dets Ældste, Tilsynsmænd og Dommere stod på begge Sider af Arken lige over for Levitpræsterne, som bar HERRENs Pagts Ark, de fremmede såvel som de indfødte, den ene Halvdel hen imod Garizims Bjerg, den anden Halvdel hen imod Ebals Bjerg, således som HERRENs Tjener Moses forhen havde påbudt at velsigne Israels Folk.
\par 34 Så oplæste han alle Lovens Ord, Velsignelsen og Forbandelsen, alt som det var skrevet i Lovbogen;
\par 35 ikke et Ord af alt, hvad Moses havde påbudt, undlod Josua at op læse for hele Israels Menighed, Mændene, Kvinderne og Børnene og de fremmede, som var draget med iblandt dem.

\chapter{9}

\par 1 Da alle Kongerne på den anden Side af Jordan, i Bjergene og i Lavlandet og langs hele det store Havs Kyst hen imod Libanon, Hetiterne, Amoriterne, Kana'anæerne, Perizziterne, Hivviterne og Jebusiterne, hørte, hvad der var sket,
\par 2 samlede de sig for i Fællesskab at kæmpe mod Josua og Israel.
\par 3 Men da Indbyggerne i Gibeon hørte, hvad Josua havde gjort ved Jeriko og Aj,
\par 4 greb også de til en List; de gik hen og forsynede sig med Rejsetæring, læssede nogle slidte Sække og nogle slidte, sprukne, stoppede Vinsække på deres Æsler
\par 5 og tog slidte, lappede Sko på Fødderne og slidte Klæder på Kroppen, og alt deres Rejsebrød var tørt og mullent.
\par 6 Så gik de til Josua i Lejren ved Gilgal og sagde til ham og Israels Mænd: "Vi kommer fra et fjernt Land; slut derfor Pagt med os!"
\par 7 Israels Mænd svarede Hivviterne: "Det kunde være, at I bor her midt iblandt os, hvorledes kan vi da slutte Pagt med eder?"
\par 8 De sagde til Josua: "Vi er dine Trælle!" Josua spurgte dem så: "Hvem er I, og hvorfra kommer I?"
\par 9 Og de svarede ham: "Fra et såre fjernt Land er dine Trælle kommet for HERREN din Guds Navns Skyld; thi vi har hørt hans Ry og alt, hvad han gjorde i Ægypten,
\par 10 og alt, hvad han gjorde mod de to Amoriterkonger hinsides Jordan, Kong Sihon af Hesjbon og Kong Og af Basan, som boede i Asjtarot.
\par 11 og vore Ældste og alle Indbyggerne i vort Land sagde til os: Tag Rejse tæring med eder, drag dem i Møde og sig til dem: Vi er eders Trælle; slut derfor nu Pagt med os!
\par 12 Vort Brød her var endnu varmt, da vi tog det med hjemmefra, dengang vi begav os af Sted for at drage til eder; men se, nu er det tørt og mullent;
\par 13 og vore Vinsække her var nye, da vi fyldte dem; se, nu er de sprukne; og vore Klæder og Sko her er slidte, fordi Vejen var så lang!"
\par 14 Så tog Mændene af deres Rejsetæring; men HERREN rådspurgte de ikke.
\par 15 Og Josua tilsagde dem Fred og sluttede Overenskomst med dem og lovede at lade dem leve, og Menighedens Øverster tilsvor dem det.
\par 16 Men tre Dage efter at de havde sluttet Pagt med dem, hørte de, at de var fra den nærmeste Omegn og boede midt iblandt dem.
\par 17 Og Israeliterne brød op og kom den tredje Dag til deres Byer, det var Gibeon, Kefira, Beerot og Kirjat Jearim.
\par 18 Men Israeliterne dræbte dem ikke, fordi Menighedens Øverster havde, tilsvoret dem Fred ved HERREN, Israels Gud. Da knurrede hele Menigheden mod Øversterne;
\par 19 men alle Øversterne sagde til hele Menigheden: "Vi har tilsvoret dem Fred ved HERREN, Israels Gud, derfor kan vi ikke gøre dem noget ondt.
\par 20 Men dette vil vi gøre med dem, når vi skåner deres Liv, at der ikke skal komme Vrede over os for den Ed, vi svor dem:
\par 21 De skal blive i Live, men være Brændehuggere og Vandbærere for hele Menigheden." Og hele Menigheden gjorde, som Øverstene havde sagt.
\par 22 Og Josua lod dem kalde og talte således til dem: "Hvorfor førte I os bag Lyset og sagde, at I havde hjemme langt borte fra os, skønt I bor her midt iblandt os?
\par 23 Derfor skal I nu være forbandede, og ingen af eder skal nogen Sinde ophøre at være Træl; Brændehuggere og Vandbærere skal I være ved min Gus Hus!"
\par 24 De svarede Josua og sagde:"Det var blevet dine Trælle sagt, at HERREN din Gud pålagde sin Tjener Moses, at når han gav eder hele Landet, skulde I udrydde alle Landets Indbyggere foran eder. Da påkom der os stor Frygt for, at I skulde tage vort Liv; derfor handlede vi således.
\par 25 Men se, nu er vi i din Hånd; gør med os, som det tykkes dig godt og ret!"
\par 26 Da handlede han således med dem; han friede dem fra Israeliternes Hånd, så de ikke dræbte dem;
\par 27 men Josua gjorde dem den Dag til Brændehuggere og Vandbærere for Menigheden og for HERRENs Alter på det Sted, han vilde udvælge. Og det er de den Dag i Dag.

\chapter{10}

\par 1 Da Kong Adonizedek af Jerusalem hørte, at Josua havde indfaget Aj og lagt Band på det som han havde gjort ved Jeriko og Kongen der, gjorde han ved Aj og Kongen der og at Gibeons Indbyggere havde sluttet Overenskomst med Israel og var optaget imellem dem,
\par 2 så påkom der ham stor Frygt; thi Gibeon var en stor By, som en af Kongsbyerne, større end Aj, og alle Mændene der var tapre Krigere.
\par 3 Derfor sendte Kong Adonizedek af Jerusalem Bud til Kong Hobam af Hebron, Kong Piram af Jarmut, Kong Jafia af Lakisj og Kong Debir af Eglon og lod sige:
\par 4 "Kom op til mig og hjælp mig med at slå Gibeon, thi det har sluttet Overenskomst med Josua og Israeliterne!"
\par 5 Da samledes fem Amoriterkonger, Kongerne i Jerusalem, Hebron, Jarmut, Lakisj og Eglon, og de drog op med hele deres Hær og slog Lejr uden for Gibeon og angreb det.
\par 6 Men Mændene i Gibeon sendte Bud til Josua i Lejren i Gilgal og lod sige: "Lad ikke dine Trælle i Stikken, men kom hurtigt op til os, hjælp os og stå os bi; thi alle Amoriterkongerne, som bor i Bjergene, har samlet sig imod os!"
\par 7 Da drog Josua op fra Gilgal med alle Krigerne, alle de kampdygtige Mænd.
\par 8 Og HERREN sagde til Josua: "Frygt ikke for dem, thi jeg giver dem i din Hånd; ikke een af dem skal kunne holde Stand imod dig!"
\par 9 Og Josua faldt pludselig over dem, efter at han i Nattens Løb var draget derop fra Gilgal,
\par 10 og HERREN bragte dem i Uorden foran Israel og tilføjede dem et stort Nederlag ved Gibeon; og de forfulgte dem hen imod Opgangen ved Bet Horon og slog dem lige til Azeka og Makkeda.
\par 11 Og da de flygtede for Israeliterne og netop var på Skråningen ved Bet Horon, lod HERREN store Sten falde ned på dem fra Himmelen helt hen til Azeka, så de døde; og de, som dræbtes af Haglstenene, var flere end dem, Israeliterne dræbte med Sværdet.
\par 12 Ved den Lejlighed, den Dag HERREN gav Amoriterne i Israeliternes Magt, talte Josua til HERREN og sagde i Israels Nærværelse: "Sol, stat stille i Gibeon, og Måne i Ajjalons Dal!"
\par 13 Og Solen stod stille, og Månen standsed, til Folket fik Hævn over Fjenden. Således står der jo skrevet i de Oprigtiges Bog. Og Solen blev stående midt på Himmelen og tøvede næsten en hel Dag med at gå ned.
\par 14 Og hverken før eller siden har der nogen Sinde været en Dag som denne, en Dag, da HERREN adlød et Menneskes Røst; thi HERREN kæmpede for Israel.
\par 15 Derpå vendte Josua med hele Israel tilbage til Lejren i Gilgal.
\par 16 Men de fem Konger flygtede og skjulte sig i Hulen ved Makkeda.
\par 17 Og der blev bragt Josua den Melding: "De fem Konger er fundet skjulte i Hulen ved Makkeda."
\par 18 Da sagde Josua: "Vælt store Sten for Hulens Indgang og sæt nogle Mænd udenfor til at vogte den.
\par 19 Men I andre må ikke standse, forfølg eders Fjender, hug Efternølerne ned og lad dem ikke komme ind i deres Byer; thi HERREN eders Gud har givet dem i eders Hånd!"
\par 20 Da Josua og Israeliterne så havde tilføjet dem et meget stort Nederlag og gjort det helt af med dem kun enkelte undslap og reddede sig ind i de befæstede Byer
\par 21 vendte hele Folket uskadt tilbage til Josua i Lejren ved Makkeda, uden at nogen havde vovet så meget som at knurre imod Israeliterne.
\par 22 Da sagde Josua: "Luk op for Hulens Indgang og før de fem Konger ud af Hulen fil mig!"
\par 23 Det gjorde de så og førte de fem Konger ud af Hulen til ham, Kongerne af Jerusalem, Hebron, Jarmut, Lakisj og Eglon.
\par 24 Da de nu havde ført disse fem Konger ud til Josua, kaldte Josua alle Israels Mænd sammen og sagde til Krigsøversterne, som var draget med ham: "Kom hid og sæt Foden på disse Kongers Nakke!" Og de kom og satte foden på deres Nakke.
\par 25 Da sagde Josua til dem: "Frygt ikke og vær ikke bange, vær frimodige og stærke! Thi således vil HERREN handle med alle eders Fjender, som I kommer til at kæmpe med!"
\par 26 Derefter lod Josua dem nedhugge og dræbe og ophænge på fem Pæle, og de blev hængende på Pælene til Aften.
\par 27 Men ved Solnedgang lod Josua dem tage ned af Pælene og kaste ind i den Hule, de havde skjult sig i, og for Indgangen til Hulen væltede man store Sten, som ligger der den Dag i Dag.
\par 28 Makkeda indtog Josua samme Dag, og han slog Byen og dens Konge ned med Sværdet; han lagde Band på den og på hver levende Sjæl i den uden at lade nogen undkomme; og han gjorde det samme ved Makkedas Konge, som han havde gjort ved Jerikos Konge.
\par 29 Derpå drog Josua med hele Israel fra Makkeda til Libna, og han angreb Libna;
\par 30 og HERREN gav også denne By og dens Konge i Israels Hånd, og den og hver levende Sjæl i den slog han ned med Sværdet uden at lade nogen undkomme; og han gjorde det samme ved dens Konge, som de havde gjort ved Jerikos Konge.
\par 31 Derpå drog Josua med hele Israel fra Libna til Lakisj, og, han slog Lejr udenfor og angreb Byen;
\par 32 og HERREN gav Lakisj i Israels Hånd, og den følgende Dag indtog han Byen, og den og hver levende Sjæl i den slog han ned med Sværdet, ganske som han havde gjort ved Libna.
\par 33 Da rykkede Kong Horam af Gezer Lakisj til Hjælp; men Josua slog ham og hans Folk ned uden at lade nogen undkomme.
\par 34 Så drog Josua med hele Israel fra Lakisj til Eglon, og de slog Lejr udenfor og angreb Byen;
\par 35 og de indtog den samme Dag og slog den ned med Sværdet; og på hver levende Sjæl i den lagde han den Dag Band, ganske som han havde gjort ved Lakisj.
\par 36 Derpå drog Josua med hele Israel op fra Eglon til Hebron, og de angreb Byen
\par 37 og indtog den og slog den ned med Sværdet, både Kongen der og alle de Byer, der hørte under den, og hver levende Sjæl i den uden at lade nogen undkomme, ganske som han havde gjort ved Eglon, og han lagde Band på Byen og hver levende Sjæl i den.
\par 38 Derpå vendte Josua sig med hele Israel imod Debir og angreb Byen;
\par 39 og han undertvang den tillige med dens Konge og alle de Byer der hørte under den; og de slog dem ned med Sværdet og lagde Band på hver levende Sjæl i dem uden at lade nogen undkomme; det samme, han havde gjort ved Hebron og ved Libna og Kongen der, gjorde han også ved Debir og Kongen der.
\par 40 Således slog Josua hele Landet, Bjerglandet, Sydlandet, Lavlandet og Bjergskråningerne og alle Kongerne der uden at lade nogen undkomme, og på hver levende Sjæl lagde han Band, således som HERREN, Israels Gud, havde påbudt;
\par 41 Josua slog dem fra Kadesj Barnea indtil Gaza, og hele Landskabet Gosjen indtil Gibeon.
\par 42 Og alle hine Konger og deres Lande undertvang Josua med et Slag; thi HERREN, Israels Gud, kæmpede for Israel.
\par 43 Derpå vendte Josua med hele Israel tilbage til Lejren i Gilgal.

\chapter{11}

\par 1 Da Kong Jabin af Hazor hørte herom sendte han Bud til Kong Jobab af Madon og Kongerne af Sjimron og Aksjaf.
\par 2 og til Kongerne nordpå i Bjergene, i Arabalavningen sønden for Kinnerot, i Lavlandet og på Højdedraget vestpå ved Dor,
\par 3 til Kana'anæerne i Øst og Vest, Amoriterne, Hivviterne, Perizziterne og Jebusiterne i Bjergene og Hetiterne ved Foden af Hermon i Mizpas Land;
\par 4 og de drog ud med alle deres Hære, Krigsfolk talrige som Sandet ved Havets Bred, og med en stor Mængde Heste og Stridsvogne.
\par 5 Alle disse Konger slog sig sammen og kom og lejrede sig i Forening ved Meroms Vand for at angribe Israel.
\par 6 Men HERREN sagde til Josua: "Frygt ikke for dem! Thi i Morgen ved denne Tid vil jeg lade dem ligge faldne foran Israel; deres Heste skal du lamme, og deres Vogne skal du brænde!"
\par 7 Da kom Josua med hele Hæren uventet over dem ved Meroms Vand og kastede sig over dem,
\par 8 og HERREN gav dem i Israels Hånd, så de slog dem og forfulgte dem til den store Stad Zidon, til, Misrefot Majim og Mizpes Lavning i Øst, og huggede dem ned, så ikke en eneste af dem blev tilbage.
\par 9 Josua gjorde derpå med dem, som HERREN havde sagt ham; deres Heste lammede han, og deres Vogne brændte han.
\par 10 Ved den Tid vendte Josua om og indtog Hazor, og Kongen huggede han ned med Sværdet; Hazor var nemlig fordum alle disse Kongerigers Hovedstad;
\par 11 og de huggede hver levende Sjæl i den ned med Sværdet og lagde Band på dem, så ikke en levende Sjæl blev tilbage; og Hazor stak han i Brand.
\par 12 Alle hine Kongsbyer med deres Konger undertvang Josua, og han huggede dem ned med Sværdet og lagde Band på dem, som HERRENs Tjener Moses havde påbudt.
\par 13 Men ingen af de Byer, som lå på deres Høje, stak Israel i Brand, alene med Undtagelse af Hazor; den stak Josua i Brand.
\par 14 Kvæget og alt det andet, der røvedes fra disse Byer, beholdt Israeliterne som Bytte; men alle Menneskene huggede de ned med Sværdet til sidste Mand uden at lade en eneste levende Sjæl blive tilbage.
\par 15 Hvad HERREN havde pålagt sin Tjener Moses, havde Moses pålagt Josua, og det gjorde Josua; han undlod intet som helst af, hvad HERREN havde pålagt Moses.
\par 16 Således indtog Josua hele dette Land, Bjerglandet, hele Sydlandet, hele Landskabet Gosjen, Lavlandet, Arabalavningen, Israels Bjergland og Lavland,
\par 17 fra det nøgne Bjergdrag, som højner sig hen imod Seir, indtil Ba'al Gad i Libanons Dal ved Hermonbjergets Fod; og alle deres Konger tog han til Fange, huggede dem ned og dræbte dem.
\par 18 I lang Tid førte Josua Krig med disse Konger
\par 19 Der var ingen By, som sluttede Overenskomst med Israeliterne, undtagen Hivviterne, som boede i Gibeon. Alt tog de i Kamp;
\par 20 thi HERREN voldte, at de forhærdede deres Hjerter, så de drog i Kamp mod Israel, for at de skulde lægge Band på dem uden Skånsel og udrydde dem, som HERREN havde pålagt Moses.
\par 21 Ved den Tid drog Josua hen og udryddede Anakiterne af Bjerglandet, af Hebron, Debir og Anab, og af hele Judas og hele Israels Bjergland; på dem og deres Byer lagde Josua Band.
\par 22 Der blev ingen Anakiter tilbage i Israeliternes Land, kun i Gaza, Gat og Asdod blev der Levninger tilbage.
\par 23 Således indtog Josua hele Landet, ganske som HERREN havde sagt til Moses, og Josua gav Israel det i Eje efter deres Afdelinger, Stamme for Stamme. Og Landet fik Ro efter Krigen.

\chapter{12}

\par 1 Følgende to Konger i Landet blev overvundet af Israeliterne og deres Land taget i Besiddelse af dem, Landet østen for Jordan fra Arnonfloden til Hermonbjerget og hele Arabalavningens østre Del:
\par 2 Amoriterkongen Sibon, som boede i Hesjbon og herskede fra Aroer ved Arnonflodens Bred og fra Midten af Floddalen over Halvdelen af Gilead indtil Jabbokfloden, der er Ammoniternes Grænse,
\par 3 og over Arabalavningen indtil Kionerotsøens Østside og Arabahavets, Salthavets, Østside hen imod Bet-Jesjimot og længere Syd på hen imod Egnen ved Foden af Pisgas Skrænter;
\par 4 og Kong Og af Basan. som hørte til dem, der var tilbage af Refaiterne, og boede i Asjtarot og Edrei
\par 5 og herskede over Hermonbjerget, Salka og hele Basan indtil Gesjuriternes og Måkatiternes Landemærke og over Halvdelen af Gilead indtil Kong Sihon af Hesjbons Landemærke.
\par 6 HERRENs Tjener Moses og Israeliterne havde overvundet dem, og HERRENs Tjener Moses havde givet Rubeniterne, Gaditerne og Manasses halve Stamme Landet i Eje.
\par 7 Følgende Konger i Landet blev overvundet af Josua og Israeliterne hinsides Jordan, på Vestsiden, fra Ba'al Gad i Dalen ved Libanon til det nøgne Bjergdrag, som højner sig mod Seir, og deres Land givet Israels Stammer i Eje af Josua efter deres Afdelinger,
\par 8 i Bjerglandet, i Lavlandet, i Arabalavningen, på Skråningerne, i Ørkenen og i Sydlandet, Hetiterne, Amoriterne, Kana'anæerne, Perizziterne, Hivviterne og Jebusiterne:
\par 9 Kongen i Jeriko een; Kongen i Aj ved Betel een;
\par 10 Kongen i Jerusalem een; Kongen i Hebron een;
\par 11 Kongen i Jarmut een; Kongen i Lakisj een;
\par 12 Kongen i Eglon een; Kongen i Gezer een;
\par 13 Kongen i Debir een; Kongen i Geder een;
\par 14 Kongen i Horma een; Kongen i Arad een;
\par 15 Kongen i Libna een; Kongen i Adullam een;
\par 16 Kongen i Makkeda een; Kongen i Betel een;
\par 17 Kongen i Tappua een; Kongen i Hefer een;
\par 18 Kongen i Afek een; Kongen i Lassjaron een;
\par 19 Kongen i Madon een; Kongen i Hazor een;
\par 20 Kongen i Sjimron Meron een; Kongen i Aksjaf een;
\par 21 Kongen i Ta'anak een; Kongen i Megiddo een;
\par 22 Kongen i Kedesj een; Kongen i Jokneam ved Karmel een;
\par 23 Kongen i Dor ved Højdedraget Dor een; Kongen over Folkene i Galilæa een;
\par 24 Kongen i Tirza een; tilsammen en og tredive Konger.

\chapter{13}

\par 1 Da Josua var blevet gammel og til Års, sagde HERREN til ham: "Du er blevet gammel og til Års, og der er endnu såre meget tilbage af Landet at indtage.
\par 2 Dette er det Land, som er tilbage: Hele Filisternes Landområde og alle Gesjuriterne,
\par 3 Landet fra Sjihor østen for Ægypten indtil Ekrons Landemærke i Nord det regnes til Kana'anæerne de fem Filisterfyrster i Gaza, Asdod, Askalon, Gat og Ekron, desuden Avviterne
\par 4 mod Syd, hele Kana'anæerlandet fra Meara, som tilhører Zidonierne, indtil Afek og til Amoriternes Landemærke,
\par 5 og det Land, som mod Øst grænser til Libanon fra Ba'al Gad ved Hermonbjergets Fod til Egnen hen imod Hamat.
\par 6 Alle Indbyggerne i Bjerglandet fra Libanon til Misrefot Majim, alle Zidonierne, vil jeg drive bort foran Israeliterne. Tildel kun Israel det som Ejendom, således som jeg har pålagt dig.
\par 7 Udskift derfor dette Land som Ejendom til de halvtiende Stammer." Manasses halve Stamme
\par 8 såvel som Rubeniterne og Gaditerne havde nemlig fået deres Arvelod, som Moses gav dem hinsides Jordan, på Østsiden, således som HERRENs Tjener Moses gav dem,
\par 9 fra Aroer ved Arnonflodens Bred og Byen midt nede i Dalen, hele Højsletten fra Medeba til Dibon,
\par 10 alle de Byer, som havde tilhørt Amoriterkongen Sibon, der herskede i Hesjbon, indtil Ammoniternes Landemærke,
\par 11 fremdeles Gilead og Gesjuriternes og Ma'akatiternes Landemærke, hele Hermonbjerget og hele Basan indtil Salka,
\par 12 hele det Rige, der havde tilhørt Og af Basan, som herskede i Asjtarot og Edrei, den sidste, der var tilbage af Refaiterne; Moses havde overvundet dem alle og drevet dem bort.
\par 13 Men Israeliterne drev ikke Gesjuriterne og Ma'akatiterne bort, så at Gesjur og Ma'akat bor i blandt Israel den Dag i bag.
\par 14 Kun Levis Stamme gav han ingen Arvelod; HERREN, Israels Gud, er hans Arvelod, således som han tilsagde ham.
\par 15 Moses gav Rubeniternes Stamme Land, Slægt for Slægt,
\par 16 og de fik deres Område fra Aroer ved Arnonflodens Bred og Byen midt nede i Dalen, hele Højsletten indtil
\par 17 Hesjbon og alle de Byer, som ligger på Højsletten, Dibon, Bamot Ba'al, Bet Ba'al Menn,
\par 18 Jaza, Kedemot, Mefa'at,
\par 19 Kirjatajim, Sibma, Zeret Sjahar på Dalbjerget,
\par 20 Bet Peor ved Pisgas Skrænter. Bet Jesjimot
\par 21 og alle de andre Byer på Højsletten og hele det Rige, der havde tilhørt Amoriterkongen Sibon, som herskede i Hesjbon, hvem Moses havde overvundet tillige med Midjans Fyrster Evi, Rekem, Zur, Hur og Reba, der var Sihons Lydkonger og boede i Landet;
\par 22 også Sandsigeren Bileam, Beors Søn, havde Israeliterne dræbt med Sværdet sammen med de andre af dem, der blev slået ihjel.
\par 23 Rubeniternes Grænse blev Jordan; den var Grænseskel. Det var Rubeniternes Arvelod efter deres Slægter, de nævnte Byer med Landsbyer.
\par 24 Og Moses gav Gads Stamme, Gaditerne, Land, Slægt for Slægt,
\par 25 og de fik følgende Landområde: Jazer og alle Byerne i Gilead og Halvdelen af Ammoniternes Land indtil Aroer, som ligger østen for Rabba,
\par 26 og Landet fra Hesjbon til Ramat Mizpe og Betonim, og fra Mahanajim til Lodebars Landemærke;
\par 27 og i Lavningen Bet Haram, Bet Nimra, Sukkot og Zafon, Resten af Kong Sihon af Hesjbons Rige, med Jordan til Grænse, indtil Enden af Kinnerets Sø, hinsides Jordan, på Østsiden.
\par 28 Det var Gaditernes Arvelod efter deres Slægter: de nævnte Byer med Landsbyer.
\par 29 Og Moses gav Manasses halve Stamme Land, Slægt for Slægt;
\par 30 og deres Landområde strakte sig fra Mahanajim over hele Basan, hele Kong Og af Basans Rige og alle Jairs Teltbyer, som ligger i Basan, tresindstyve Byer,
\par 31 Halvdelen af Gilead og Asjtatot og Edrei, Ogs Kongsbyer i Basan; det gav han Manasses Søn Makirs Sønner, Halvdelen af Makirs Sønner, Slægt for Slægt.
\par 32 Det er alt, hvad Moses udskiftede på Moabs Sletter hinsides Jordan over for Jeriko, på Østsiden.
\par 33 Men Levis Stamme gav Moses ingen Arvelod; HERREN, Israels Gud, er deres Arvelod, således som han tilsagde dem.

\chapter{14}

\par 1 Følgende er de Landstrækninger, Israeliterne fik til Arvelod i Kana'ans Land, som Præsten Eleazar og Josua, Nuns Søn, og Overhovederne for de israelitiske Stammers Fædrenehuse tildelte dem
\par 2 ved Lodkastning som deres Ejendom i Overensstemmelse med det Påbud, HERREN havde givet Moses om de halvtiende Stammer.
\par 3 Thi Moses havde givet de halvtredje Stammer Arvelod hinsides Jordan; men Leviterne gav han ikke Arvelod iblandt dem.
\par 4 Josefs Efterkommere udgjorde nemlig to Stammer, Manasse og Efraim; og Leviterne fik ikke Del i Landet, men kun Byer at bo i tillige med de tilhørende Græsmarker til deres Hjorde og Kvæg.
\par 5 Hvad HERREN havde pålagt Moses, gjorde Israeliterne, og de udskiftede Landet.
\par 6 Da trådte Judæerne frem for Josua i Gilgal, og Kenizziten Kaleb, Jefunnes Søn, sagde til ham: "Du ved, hvad det var, HERREN talede til den Guds Mand Moses i Kadesj Barnea om mig og dig.
\par 7 Fyrretyve År gammel var jeg, dengang HERRENs Tjener Moses udsendte mig fra Kadesj Barnea for at udspejde Landet; og jeg aflagde ham Beretning efter bedste Overbevisning.
\par 8 Men mine Brødre, som var draget med mig, gjorde Folket modløst, medens jeg viste HERREN min Gud fuld Lydighed.
\par 9 Og Moses svor den Dag: Sandelig, det Land, din Fod har betrådt, skal være din og dine Efterkommeres Arvelod til evig Tid, fordi du har vist HERREN min Gud fuld Lydighed!
\par 10 Og se, nu har HERREN opfyldt sit Ord og holdt mig i Live fem og fyrretyve År, siden dengang HERREN talede dette Ord til Moses, al den Tid Israel vandrede i Ørkenen, og se, jeg er nu fem og firsindstyve År.
\par 11 Endnu den Dag i Dag er jeg rask og rørig som på hin Dag, da Moses udsendte mig; nu som da er min Kraft den samme til Kamp og til at færdes omkring.
\par 12 Så giv mig da dette Bjergland, som HERREN dengang talede om; du hørte det jo selv. Thi der bor Anakiter der, og der er store, befæstede Byer; måske vil HERREN være med mig, så jeg kan drive dem bort, som HERREN har sagt!"
\par 13 Da velsignede Josua ham, og han gav Kaleb, Jefunnes Søn, Hebron til Arvelod.
\par 14 Derfor tilfaldt Hebron Kenizziten Kaleb, Jefunnes Søn,, som Arvelod, og den hører ham til den Dag i Dag, fordi han viste HERREN, Israels Gud, fuld Lydighed.
\par 15 Men Hebron hed forhen Arbas By; han var den største Mand blandt Anakiterne. Og Landet fik Ro efter krigen.

\chapter{15}

\par 1 Loddet faldt for Judæernes Stamme efter deres Slægter således, at deres Landområd strækker sig hen,imod Edoms Område, Zins Ørken mod Syd, yderst mod Syd.
\par 2 Deres Sydgrænse begynder ved Enden af Salthavet, ved den sydlige Bugt,
\par 3 og løber sønden om Akrabbimpasset, går videre til Zin, strækker sig opad sønden om Hadesj Barnea og går derpå videre til Hezron og op til Addar; så drejer den om mod Karka'a,
\par 4 går videre til Azmon og fortsætter til Ægyptens Bæk; så ender Grænsen ved Havet. Det er deres Sydgrænse.
\par 5 Østgrænsen er Salthavet indtil Jordans Udløb. Nordgrænsen begynder ved Havets Bugt ved Jordans Udløb;
\par 6 derpå strækker Grænsen sig opad til Bet Hogla og går videre norden om Bet Araba; så strækker Grænsen sig opad til Rubens Søn Bohans Sten;
\par 7 derpå strækker Grænsen sig fra Akors Dal op til Debir og drejer nordpå til Gilgal, som ligger lige over for Adummimpasset sønden for Dalen; derefter går Grænsen videre over til Vandet ved Sjemesjkilden og ender ved Rogelkilden;
\par 8 derpå strækker Grænsen sig op i Hinnoms Søns Dal til Sydsiden af Jebusiternes Bjergryg, det er Jerusalem; derpå strækker Grænsen sig op til Toppen af Bjerget lige vesten for Hinnoms Dal ved Refaimdalens Nordende;
\par 9 derpå bøjer Grænsen fra Toppen af dette Bjerg ben til Neftoas Vandkilde og løber videre til Byerne på Efronbjerget; så bøjer Grænsen om til Ba'ala, det er Kirjat-Jearim;
\par 10 derpå drejer Grænsen om fra Ba'ala mod Vest til Seirbjerget, går videre til Jearimbjergets nordre Udløber, det er Kesalon: så strækker den sig ned til Bet Sjemesj og går videre til Timna;
\par 11 derpå løber Grænsen i nordlig Retning til Bjergryggen ved Ekron; så bøjer Grænsen om til Sjikkaron, går videre til Ba'alabjerget, løber til Jabne'el og ender ved Havet.
\par 12 Vestgrænsen er det store Hav. Det er Grænsen rundt om Judæernes Område efter deres Slægter.
\par 13 Men Kaleb, Jefunnes Søn, gav han et Stykke Land imellem Judæerne efter HERRENs Befaling til Josua: Anaks Stamfader Arbas By, det er Hebron;
\par 14 og Kaleb drev de tre Anakiter bort derfra, Sjesjaj, Abiman og Talmaj, der nedstammede fra Anak.
\par 15 Derfra drog han op mod Debirs Indbyggere; Debir hed fordum Kirjat Sefer.
\par 16 Da sagde Kaleb: "Den, som slår Kirjat Sefer og indtager det, giver jeg min datter Aksa til Hustru!"
\par 17 Og da Benizziten Otniel, Kalebs Broder, indtog det, gav han ham sin Datter Aksa til Hustru.
\par 18 Men da hun kom til ham, æggede han hende til at bede sin Fader om Agerland. Hun sprang da ned af Æselet, og Kaleb spurgte hende: "Hvad vil du?"
\par 19 Hun svarede: "Giv mig en Velsignelse!" Siden du har bortgiftet mig i det tørre Sydland, må du give mig Vandkilder!" Da gav han hende de øvre og de nedre Vandkilder.
\par 20 Judæernes Stammes Arvelod efter deres Slægter er:
\par 21 Byerne i Udkanten af Judæernes Stamme ved Edoms Grænse i Sydlandet er følgende: Kabzeel, Eder, Jagur,
\par 22 Kina, Dimona, Arara,
\par 23 Kedesj Hazot, Jitnan,
\par 24 Zif, Telam, Bealot,
\par 25 Hazor Hadatta, Kerijjot Hezron, det er Hazor,
\par 26 Amam, Sjema, Molada,
\par 27 Hazar Gadda, Hesjmon, Bet Pelet,
\par 28 Hazar Sjual, Be'ersjeba med Småbyer,
\par 29 Ba'ala, Ijjim, Ezem,
\par 30 Eltolad, Betul, Horma,
\par 31 Ziklag, Madmanna, Sansanna,
\par 32 Lebaot, Sjilhim og En Rimmox; tilsammen ni og tyve Byer med Landsbyer.
\par 33 I Lavlandet: Esjtaol, Zora, Asjna,
\par 34 Zanoa, En Gannim, Tappua, Enam,
\par 35 Jarmut, Adullam, Soko, Azeka,
\par 36 Sja'arajim, Aditajim, Gedera og Gederotajim; tilsammen fjorten Byer med Landsbyer.
\par 37 Zenan, Hadasja, Migdal Gad,
\par 38 Dilan, Mizpe, Jokte'el,
\par 39 Lakisj, Bozkaf, Eglon,
\par 40 Kabbon, Lamas, Hitlisj,
\par 41 Gederot, Bet Dagon, Na'ama og Makkeda; tilsammen seksten Byer med Landsbyer.
\par 42 Libna, Eter, Asjan,
\par 43 Jifta, Asjna, Nezib,
\par 44 Keila, Akzib og Maresja; tilsammen ni Byer med Landsbyer.
\par 45 Ekron med Småbyer og Landsbyer;
\par 46 fra Ekron til Havet alt, hvad der ligger på Asdodsiden, med tilhørende Landsbyer;
\par 47 Asdod med Småbyer og Landsbyer; Gaza med Småbyer og Landsbyer indtil Ægyptens Bæk med det store Hav som Grænse.
\par 48 I Bjerglandet: Sjamir, Jattir, Soko,
\par 49 Danna, Kirjat Sefer, det er Debir,
\par 50 Anab, Esjtemo, Anim,
\par 51 Gosjen, Holon og Gilo; tilsammen elleve Byer med Landsbyer.
\par 52 Arab, Duma, Esjan,
\par 53 Janum, Bet Tappua, Afeka,
\par 54 Humta, Kirjat Arba, det er Hebron, og Zior; tilsammen ni Byer med Landsbyer.
\par 55 Maon, Karmel, Zif, Jutta,
\par 56 Jizre'el, Jokdeam, Zanoa,
\par 57 Hain, Gibea og Timna; tilsammen ti Byer med Landsbyer.
\par 58 Halhul, Bet Zur, Gedor,
\par 59 Ma'arat, Bet Anon og Eltekon; tilsammen seks Byer med Landsbyer. Tekoa, Efrata, det er Betlehem, Peor, Etam, Kulon, Tatam, Sores, Kerem, Gallim, Beter og Menoho; tilsammen elleve Byer med Landsbyer.
\par 60 Hirjat Ba'al, det er Hirjat Jearim, og Rabba; tilsammen to Byer med Landsbyer.
\par 61 I Ørkenen: Bet Araba, Middin, Sekaka,
\par 62 Nibsjan, Ir Mela og En Gedi; tilsammen seks Byer med Landsbyer.
\par 63 Men Jebusiterne, som boede i Jerusalem, kunde Judæerne ikke drive bort; og Jebusiterne bor i Jerusalem sammen med Judæerne den Dag i Dag.

\chapter{16}

\par 1 For Josefs Sønner faldt Loddet således: Mod Øst går Grænsen fra Jordan ved Jeriko, ved Jerikos Vande, op gennem Ørkenen, som fra Jeriko strækker sig op i Bjergland,et til Betel;
\par 2 fra Betel fortsætter den videre til Arkiternes Landemærke, til Atarot,
\par 3 og strækker sig nedad mod Vest til Ja Detiternes Landemærke, til Nedre Bet Horons Landemærke og til Gezer og ender ved Havet.
\par 4 Og Josefs Sønner, Manasse og Efraim, fik Arvelodder.
\par 5 Efraimiternes Landemærke efter deres Slægter var følgende: Grænsen for deres Arvelod er mod Øst Atarot Addar og går til Øvre Bet Horon;
\par 6 derpå går Grænsen ud til Havet. Mod Nord er Grænsen Mikmetat; Grænsen går så mod Øst til Ta'anat Sjilo, løber videre østen om Janoa,
\par 7 strækker sig så fra Janoa ned til Atarot og Na'ara, støder op til Jeriko og ender ved Jordan.
\par 8 Fra Tappua går Grænsen mod Vest til Kanabækken og ender ved Havet. Det er Efraimiternes Stammes Arvelod efter deres Slægter.
\par 9 Dertil kommer de Byer, som udskiltes til Efraimiterne inden for Manassiternes Arvelod, alle Byerne med Landsbyer.
\par 10 Men de fordrev ikke Kana'anæerne, som boede i Gezer, og således er Kana'anæerne blevet boende midt i Efraim indtil den Dag i Dag, idet de siden blev Hoveriarbejdere.

\chapter{17}

\par 1 Og Loddet faldt for Manasses Stamme: thi han var Josefs førstefødte. Makir, Manasses førstefødte, Gileads Fader - han var nemlig Kriger - fik Gilead og Basan.
\par 2 Og de øvrige Manassiter fik Land efter deres Slægter, Abiezers, Heleks, Asriels, Sjekems, Hefers og Sjemidas Sønner; det er Josef's Søn Manasses mandlige Efterkommere efter deres Slægter.
\par 3 Men Zelofhad, en Søn af Hefer, en Søn af Gilead, en Søn af Makir, en Søn af Manasse, havde ingen Sønner, kun Døtre; og hans Døtre hed Mala, Noa, Hogla, Milka og Tirza.
\par 4 De trådte frem for Præsten Eleazar og Josua, Nuns Søn, og Øversterne og sagde: "HERREN bød Moses give os Arvelod iblandt vore Brødre!" Da gav han dem efter HERRENs Bud en Arvelod iblandt deres Faders Brødre.
\par 5 Således faldt ti Parter på Manasse foruden Landet Gilead og Basan hinsides Jordan.
\par 6 Thi Manasses døtre fik Arvelod blandt hans Sønner. Men Landet Gilead tilfaldt Manasses øvrige Efterkommere.
\par 7 Og Manasses Grænse går fra Aser til Mikmetat, som ligger østen for Sikem; derpå går Grænsen mod Syd til Befolkningen i En-Tappua.
\par 8 Manasse fik Landskabet Tappua; men Byen Tappua ved Manasses Grænse tilfaldt Efraimiterne.
\par 9 Derpå strækker Grænsen sig ned til Kanabækken, sønden om Bækken; Byerne der tilfaldt Efraim, midt iblandt Manasses Byer; Manasses Landområde ligger norden for Bækken. Grænsen ender derpå ved Havet.
\par 10 Sydsiden tilhører Efraim og Nordsiden Manasse. Havet danner Grænse; mod Nord støder de op til Aser, mod Øst til Issakar.
\par 11 I Issakar og Aser tilfaldt følgende Byer Manasse: Bet-Sjean med Småbyer, Jibleam med Småbyer, Befolkningen i Dor med Småbyer, Befolkningen i En-Dor med Småbyer, Befolkningen i Ta'anak med Småbyer og Befolkningen i Megiddo med Småbyer, de tre Højdedrag.
\par 12 Men Manassiterne kunde ikke drive disse Byers Indbyggere bort, det lykkedes Kana'anæerne at holde sig i disse Egne.
\par 13 Da Israeliterne blev de stærkeste, gjorde de Kana'anæerne til Hoveriarbejdere, men drev dem ikke bort.
\par 14 Da talte Josefs Sønner til Josua og sagde: "Hvorfor har du kun givet mig een Lod og een Part til Arvelod, skønt jeg er et talrigt Folk, eftersom HERREN hidtil har velsignet mig?"
\par 15 Josua svarede dem: "Når du er et talrigt Folk, så drag op i Skovlandet og ryd dig Jord der i Perizziternes og Refaiternes Land, siden Efraims Bjergland er dig for trangt!"
\par 16 "Da sagde Josefs Sønner: "Bjerglandet er os ikke nok, og alle Kana'anærerne, som bor på Slettelandet, både de i Bet-Sjean med Småbyer og de på Jizre'elsletten, har jernbeslagne Vogne!"
\par 17 Da sagde Josua til Josefs Slægt, til Efraim og Manasse: "Du er et talrigt Folk og har stor Kraft; du skal ikke komme til at nøjes med een Lod,
\par 18 men et Bjergland skal tilfalde dig, thi det er skovbevokset.

\chapter{18}

\par 1 Hele Israeliternes Menighed kom sammen i Silo, og de rejste Åbenbaringsteltet der, da de nu havde underlagt sig Landet.
\par 2 Men der var endnu syv Stammer tilbage af Israeliterne, som ikke havde fået deres Arvelod tildelt.
\par 3 Josua sagde derfor til Israeliterne: "Hvor længe vil I endnu nøle med at drage hen og tage det Land i Besiddelse, som HERREN, eders Fædres Gud, har givet eder?
\par 4 Udse eder tre Mænd af hver Stamme, som jeg kan udsende; de skal gøre sig rede og drage Landet rundt og affatte en Beskrivelse derover til Brug ved Fastsættelsen af deres Arvelod og så komme tilbage til mig.
\par 5 De skal dele det i syv Dele; Juda skal beholde sit Område mod Syd og Josefs Slægt sit mod Nord.
\par 6 Og I skal så affatte en Beskrivelse over Landet, delt i syv Dele, og bringe mig den; så vil jeg kaste Lod for eder her for HERREN vor Guds Åsyn.
\par 7 Thi Leviterne får ingen Del iblandt eder, da HERRENs Præstedømme er deres Arvelod; og Gad, Ruben og Manasses halve Stamme har på Jordans Østside fået deres Arvelod, som HERRENs Tjener Moses gav dem!"
\par 8 Da begav Mændene sig på Vej; og Josua bød de bortdragende affatte en Beskrivelse over Landet, idet han sagde: "Drag Landet rundt, affat en Beskrivelse over det og kom så tilbage til mig; så vil jeg kaste Lod for eder her for HERRENs Åsyn i Silo!"
\par 9 Så begav, Mændene sig på Vej og drog igennem Landet og affattede en Beskrivelse derover, By for By, i syv Dele; og derpå kom de tilbage til Josua i Lejren ved Silo.
\par 10 Men Josua kastede Lod for dem i Silo for HERRENs Åsyn og udskiftede der Landet til Israeliterne, Afdeling, for Afdeling.
\par 11 Da faldt Loddet for Benjaminiternes Stamme efter deres Slægter; og det Område, der blev deres Lod, kom til at ligge mellem Judas og Josefs Sønner.
\par 12 Deres Nordgrænse begynder ved Jordan, og Grænsen strækker sig op til Bjergryggen norden for Jeriko og mod Vest op i Bjerglandet, så den ender i Bet-Avens Ørken;
\par 13 derfra går Grænsen videre til Luz, til Bjergryggen sønden for Luz, det er Betel, og strækker sig ned til Atarot-Addar over Bjerget sønden for Nedre-Bet-Horon.
\par 14 Derpå bøjer Grænsen om og løber som Vestgrænse sydpå fra Bjerget lige sønden for Bet-Horon og ender ved Kirjat-Ba'al, det er den judæiske By Kirjat-Jearim; det er Vestgrænsen.
\par 15 Sydgrænsen begynder ved Udkanten af Kirjat-Ba'al, og Grænsen går til Neftoas Vandkilde;
\par 16 derpå strækker den sig ned til Randen af Bjerget lige over for Hinnoms Søns dal norden for Refaimdalen og videre til Hinnoms Dal sønden om Jebusiternes Bjergryg og til Rogelkilden;
\par 17 så drejer den nordpå og fortsætter til Sjemesjkilden og videre til Gelilot lige over for Adummimpasset, derpå ned til Rubens Søn Bohans Sten
\par 18 og går så videre til Bjergryggen norden for Bet-Araba og ned i Arabalavningen;
\par 19 så går den videre til bjergryggen norden for Bet-Hogla og ender norden for Salthavets Bugt ved Jordans Udløb; det er Sydgrænsen.
\par 20 Mod Øst danner Jordan Grænse. Det er Benjaminiternes Arvelod med dens Grænser efter deres Slægter.
\par 21 Og Benjaminiternes Stammes Byer efter deres Slægter er følgende: Jeriko, Bet-Hogla, Emek-Keziz,
\par 22 Bet-Araba, Zemarajim, Betel,
\par 23 Avvim, Para, ofra,
\par 24 Kefar-Ammoni, Ofni og Geba; tilsammen tolv Byer med Landsbyer.
\par 25 Gibeon, Rama, Be'erot,
\par 26 Mizpe, Kefira, Moza,
\par 27 Rekem, Jirpe'el, Tar'ala,
\par 28 Zela, Elef, Jebus, det er Jerusalem, Gibeat og Kirjat-Jearim; tilsammen fjorten Byer med Landsbyer. Det er Benjaminiternes Arvelod efter deres Slægter.

\chapter{19}

\par 1 Det andet Lod faldt for Simeon, for Simeoniternes Stamme efter deres Slægter; og deres Arvelod kom til at ligge inde i Judæernes Arvelod.
\par 2 Til deres Arvelod hørte: Be'er-sjeba, Sjema, Molada,
\par 3 Hazar-Sjual, Bala, Ezem,
\par 4 Eltolad, Betul, Horma,
\par 5 Ziklag, Bet-Markabot, Hazar-Susa,
\par 6 Bet-Lebaot og Sjaruhen; tilsammen tretten Byer med Landsbyer.
\par 7 En-Rimmon, Token, Eter og Asjan; tilsammen fire Byer med Landsbyer.
\par 8 Desuden alle Landsbyerne rundt om disse Byer indtil Ba'alat-Be'er, Rama i Sydlandet. Det er Simeoniternes Stammes Arvelod efter deres Slægter.
\par 9 Fra Judæernes Del blev Simeoniternes Arvelod taget; thi Judæernes Del var for stor til dem; der for fik Simeoniterne Arvelod inde i deres Arvelod.
\par 10 Det tredje Lod faldt for Zebuloniterne efter deres Slægter.
\par 11 og deres Grænse strækker sig vestpå op til Mar'ala, berører Dabbesjet og støder til Bækken, som løber østen om Jokneam;
\par 12 fra Sarid drejer den østpå, mod Solens Opgang, hen imod Kis-Idt-Tabors Område og fortsætter til Daberat og op til Jafia;
\par 13 mod Øst, mod Solens Opgang, løber den derpå over til Gat-Hefer, til Et-Kazin og videre til Rimmona og bøjer om til Nea;
\par 14 derfra drejer Grænsen i nordlig Retning til Hannaton og ender i Dalen ved Jifta-El.
\par 15 Og den omfatter Kattat, Nabalal, Sjimron, Jid'ala og Betlehem; tilsammen tolv Byer med Lands-byer.
\par 16 Det er Zebulonilernes Arvelod efter deres Slægter, nævnte Byer med Landsbyer.
\par 17 For Issakar faldt det fjerde Lod, for Issakariterne efter deres Slægter;
\par 18 og deres Landemærke var: Jizre'el, Kesullot, Sjunem,
\par 19 Hafarajim, Sji'on, Anabarat,
\par 20 Rabbit, Kisjjon, Ebez,
\par 21 Remet, En-Gannim, En-Hadda og Bet-Pazzez.
\par 22 Og Grænsen berører Tabor, Sja-hazim og Bet-Sjemesj og ender ved Jordan; tilsammen seksten Byer med Landsbyer.
\par 23 Det er Issakariternes Stammes Område efter deres Slægter, nævnte Byer med Landsbyer.
\par 24 Det femte Lod faldt for Aseriternes Stamme efter deres Slægter.
\par 25 Deres Landemærke var: Helkat, Hali, Beten, Aksjaf,
\par 26 Alammelek, Am'ad og Misj'al; derpå berører Grænsen Harmel mod Vest og Sjihor-Libnat,
\par 27 drejer så østpå til Betbagon og berører Zebulon og Dalen ved Jifta-El mod Nord; derpå går den til Bet-Emek og Ne'iel og fortsætter nordpå til Kabul,
\par 28 Abdon, Rebob, Hammon og Hana indtil den store Stad Zidon;
\par 29 så drejer Grænsen mod Rama og til den befæstede By Tyrus; derpå drejer Grænsen mod Hosa og ender ved Havet; desuden Mahalab, Akzib,
\par 30 Akko, Afek, Rebob; tilsammen to og tyve Byer med Landsbyer.
\par 31 Det er Aseriternes Stammes Område efter deres Slægter, nævnte Byer med Landsbyer.
\par 32 For Naftaliterne faldt det sjette Lod, for Naftaliterne efter deres Slægter.
\par 33 Deres Landemærke går fra Helet, fra Allon-Beza'anannim og Adami-Nekeb og Jabne'el indtil Lakkum og ender ved Jordan;
\par 34 så drejer Grænsen vestpå til Aznot-Tabor, fortsætter derfra til Hukkok, berører Zebulon mod Syd, Aser mod Vest og Jordan mod Øst.
\par 35 Befæstede Byer er: Ziddim, Zer, Hammat, Rakkat, Kinneret,
\par 36 Adama, Rama, Hazor,
\par 37 Hedesj, Edre'i, En-Hazor,
\par 38 Jir'on, Migdal-El, Horem, Bet-Anat og Bet-Sjemesj; tilsammen nitten Byer med Landsbyer.
\par 39 Det er Naftaliternes Stammes Arvelod efter deres Slægter, nævnte Byer med Landsbyer.
\par 40 For Daniternes Stamme efter deres Slægter faldt det syvende Lod.
\par 41 Deres Arvelods Landemærke var: Zor'a, Esjtaol, Ir-Sjemesj,
\par 42 Sja'alabbin, Ajjalon, Jitla,
\par 43 Elon, Timna, Ekron,
\par 44 Elteke, Gibbeton, Ba'alat,
\par 45 Jehud, Bene-Berak, Gat-Rim-mon,
\par 46 Me-Ja1'kon og Rakkon og Egnen hen imod Jafo.
\par 47 Men Daniternes Område blev dem for trangt; derfor drog Daniterne op og angreb Lesjem, indtog det og slog det med Sværdet; derpå tog de det i Besiddelse og bosatte sig der og gav Lesjem Navnet Dan efter deres Stamfader Dan.
\par 48 Dette er Daniternes Stammes Arvelod efter deres Slægter, nævnte Byer med Landsbyer.
\par 49 Da Israeliterne var færdige med Udskiftningen af Landet, Stykke for Stykke, gav de Josua, Nuns Søn, en Arvelod imellem sig.
\par 50 Efter HERRENs Påbud gav de ham den By, han udbad sig, Timnat-Sera i Efraims Bjerge; og han befæstede Byen og bosatte sig der.
\par 51 Det er de Arvelodder, som Præsten Eleazar og Josua, Nuns Søn, og Overhovederne for de israelitiske Stammers Fædrenehuse udskiftede ved Lodkastning i Silo for HERRENs Åsyn ved Åbenbaringsteltets Indgang. Således blev de færdige med Udskiftningen af Landet.

\chapter{20}

\par 1 Og HERREN talede til Josua og sagde
\par 2 "Tal til Israeliterne og sig: Afgiv de Tilflugtsbyer, jeg talede til eder om ved Moses,
\par 3 for at en Manddraber, der uforsætligt og af Vanvare slår en ihjel, kan ty til dem, så at de kan være eder Tilflugtssteder mod Blodhævneren.
\par 4 Når han tyr hen til en af disse Byer og stiller sig i Byportens Indgang og forebringer sin Sag for Byens Ældste, skal de optage ham i Byen hos sig og anvise ham et Sted, hvor han kan bo bos dem;
\par 5 og når Blodhævneren forfølger ham, må de ikke udlevere Manddraberen til ham, thi han har slået sin Næste ihjel af Vanvare uden i Forvejen at have båret Nag til ham;
\par 6 han skal blive boende i denne By, indtil han har været stillet for Menighedens Domstol, eller den Mand, som på den Tid er Ypperstepræst, dør; derefter kan Manddraberen vende tilbage til sin By og sit Hjem, den By, han er flygtet fra."
\par 7 Da helligede de Kedesj i Galilæa i Naftalis Bjerge, Sikem i Efrims Bjerge og Kirjat-Arba, det er Hebron, i Judas Bjerge.
\par 8 Og østen for Jordan afgav de Bezer i Ørkenen, på Højsletten, af Rubens Stamme, Ramot i Gilead at Gads Stamme og Golan i Basan af Manasses Stamme.
\par 9 Det var de Byer, som fastsattes for alle Israeliterne og de fremmede, som bor iblandt dem, i det Øjemed at enhver, der uforsætligt slår en ihjel, kan ty derhen og undgå Døden for Blodhævnerens Hånd, før han har været stillet for Menighedens Domstol.

\chapter{21}

\par 1 Derpå trådte Overhovederne for Leviternes Fædrenehuse frem for Præsten Eleazar og Josua, Nuns Søn, og Overhovederne for de israelitiske Stammers Fædrenehuse
\par 2 og talte således til dem i Silo i Kana'ans Land: "HERREN bød ved Moses, at der skulde gives os nogle Byer at bo i med tilhørende Græsmarker til vort Kvæg."
\par 3 Da afgav Israeliterne i Følge HERRENs Bud af deres Arvelod følgende Byer med Græsmarker til Leviterne.
\par 4 Loddet faldt først for Kehatiternes Slægter, således at Præsten Arons Sønner blandt Leviterne ved Lodkastningen fik tretten Byer af Judas, Simeoniternes og Benjamins Stammer,
\par 5 mens de andre Hehatiter ved Lodkastningen efter deres Slægter fik ti Byer af Efraims og Dans Stammer og Manasses halve Stamme.
\par 6 Gersoniterne fik ved Lodkastningen efter deres Slægter tretten Byer af Issakars, Asers og Naftalis Stammer og Manasses halve Stamme i Basan.
\par 7 Merariterne fik efter deres Slægter tolv Byer af Rubens, Gads og Zebulons Stammer.
\par 8 Og Israeliterne afgav ved Lodkastning følgende Byer med Græsmarker til Leviterne, således som HERREN havde påbudt ved Moses.
\par 9 Af Judæernes og Simeoniternes Stammer afgav de følgende ved Navn, nævnte Byer.
\par 10 Arons Sønner, der hørte til Kehatiternes Slægter blandt Levis Sønner - thi for dem faldt Loddet først - fik følgende:
\par 11 Man gav dem Anaks Stamfader Arbas By, det er Hebron, i Judas Bjerge, med omliggende Græsmarker;
\par 12 men Byens Mark og Landsbyer gav man Kaleb, Jefunnes Søn, i Eje.
\par 13 Præsten Arons Sønner gav man Hebron, en af Tilflugtsbyerne for Manddrabere, med omliggende Græsmarker, Libna med omliggende Græsmarker,
\par 14 Jattir med omliggende Græsmarker, Esjtemoa med omliggende Græsmarker,
\par 15 Holon med omliggende Græsmarker, Debir med omliggende Græsmarker,
\par 16 Asjan med omliggende Græsmarker, Jutta med omliggende Græsmarker og Bet-Sjemesj med omliggende Græsmarker; tilsammen ni Byer af de to Stammer;
\par 17 og af Benjamins Stamme Gibeon med omliggende Græsmarker, Geba med omliggende Græsmarker,
\par 18 Anatot med omliggende Græsmarker og Alemet med omliggende Græsmarker; tilsammen fire Byer.
\par 19 Præsternes, Arons Sønners, Byer udgjorde i alt tretten Byer med omliggende Græsmarker.
\par 20 Kehatiternes Slægter af Leviterne, de øvrige Kehatiter, fik de Byer af Efraims Stamme, som tildeltes dem ved Lodkastning.
\par 21 Man gav dem Sikem, en af Tilflugtsbyerne for Manddrabere, med omliggende Græsmarker i Efraims Bjerge, Gezer med omliggende Græsmarker,
\par 22 Kibzajim med omliggende Græsmarker og Bet-Horon med omliggende Græsmarker; tilsammen fire Byer;
\par 23 og af Dans Stamme Elteke med omliggende Græsmarker, Gibbeton med omliggende Græsmarker,
\par 24 Ajjalon med omliggende Græsmarker og Gat-Rimmon med omliggende Græsmarker; tilsammen fire Byer;
\par 25 og af Manasses halve Stamme Tånak med omliggende Græsmarker og Jibleam med omliggende Græsmarker; tilsammen to Byer;
\par 26 i alt ti Byer med omliggende Græsmarker tilfaldt de øvrige Kehatiters Slægter.
\par 27 Blandt Leviternes Slægter fik fremdeles Gersoniterne af Manasses halve Stamme Golan i Basan, en af Tilflugtsbyerne for Manddrabere, med omliggende Græsmarker og Asjtarot med omliggende Græsmarker; tilsammen to Byer;
\par 28 og af Issakars Stamme Kisjjon med omliggende Græsmarker, Daberat med omliggende Græsmarker,
\par 29 Jarmut med omliggende Græsmarker og En-Gannim med omliggende Græsmarker; tilsammen fire Byer;
\par 30 og af Asers Stamme Misj'al med omliggende Græsmarker, Abdon med omliggende Græsmarker,
\par 31 Helkat med omliggende Græsmarker og Rehob med omliggende Græsmarker; tilsammen fire Byer;
\par 32 og af Naftalis Stamme Bedesj i Galilæa, en af Tilflugtsbyerne for Manddrabere, med omliggende Græsmarker, Hammot-Dor med omliggende Græsmarker og Hartan med omliggende Græsmarker; tilsammen tre Byer;
\par 33 Gersoniternes Byer efter deres Slægter udgjorde i alt tretten med omliggende Græsmarker.
\par 34 De øvrige Leviter, Merariternes Slægter, fik af Zebulons Stamme Jokneam med omliggende Græsmarker, Karta med omliggende Græsmarker,
\par 35 Rimmona med omliggende Græsmarker og Nahalal med omliggende Græsmarker; tilsammen fire Byer;
\par 36 og hinsides Jordan over for Jeriko af Rubens Stamme Bezer i Ørkenen på Højsletten, en af Tilflugtsbyerne for Manddrabere, med omliggende Græsmarker, Jaza med omliggende Græsmarker.
\par 37 Kedemot med omliggende Græsmarker og Mefa'at med omliggende Græsmarker; tilsammen fire Byer;
\par 38 og af Gads Stamme Ramot i Gilead, en af Tilflugtsbyerne for Manddrabere, med omliggende Græsmarker, Mahanajim med omliggende Græsmarker,
\par 39 Hesjbon med omliggende Græsmarker og Ja'zer med omliggende Græsmarker; tilsammen fire Byer;
\par 40 Byerne, der ved Lodkastningen tilfaldt de øvrige Levitslægter, Merariterne efter deres Slægter, udgjorde i alt tolv.
\par 41 Levitbyerne inden for Israeliternes Ejendom udgjorde i alt otte og fyrretyve med omliggende Græsmarker.
\par 42 Disse Byer skulde hver for sig have de omliggende Græsmarker med; det gjaldt for alle disse Byer.
\par 43 Således gav HERREN Israel hele det Land, han havde tilsvoret deres Fædre at ville give dem, og de tog det i Besiddelse og bosatte sig der.
\par 44 Og HERREN gav dem Ro rundt om, ganske som han havde tilsvoret deres Fædre, og ingen iblandt deres Fjender kunde holde Stand over for dem; alle deres Fjender gav HERREN i deres Hånd.
\par 45 Ikke eet af alle de gode Ord, HERREN havde talet til Israels Hus, faldt til Jorden; alle sammen gik de i Opfyldelse.

\chapter{22}

\par 1 Derpå lod Josua Rubeniterne, Gaditerne og Manasses halve Stamme kalde til sig
\par 2 og sagde til dem: "I har holdt alt, hvad HERRENs Tjener Moses bød eder, og adlydt mig i alt, hvad jeg har påbudt eder.
\par 3 I har ikke svigtet eders Brødre i denne lange Tid; indtil denne Dag har I holdt HERREN eders Guds Bud.
\par 4 Men nu har HERREN eders Gud skaffet eders Brødre Ro, som han lovede dem; vend derfor nu tilbage til eders Telte i det Land, hvor eders Ejendom ligger, som HERRENs Tjener Moses gav eder hinsides Jordan.
\par 5 Kun må I omhyggeligt agte på at holde det Bud og den Lov, HERRENs Tjener Moses pålagde eder, at elske HERREN eders Gud, vandre på alle hans Veje, holde hans Bud, holde fast ved ham og tjene ham af hele eders Hjerte og hele eders Sjæl!"
\par 6 Og Josua velsignede dem og lod dem drage bort, og de begav sig til deres Telte.
\par 7 Den ene Halvdel af Manasses Stamme havde Moses givet Land i Basan, den anden Halvdel derimod havde Josua givet Land sammen med deres Brødre i Landet vesten for Jordan. Og da Josua lod dem drage hver til sit efter at have velsignet dem,
\par 8 vendte de tilbage til deres Telte med store Rigdomme, med Kvæg i Mængde, med Sølv og Guld, Kobber og Jern og Klæder i stor Mængde; og det Bytte, de havde taget fra deres Fjender, delte de med deres Brødre.
\par 9 Så forlod Rubeniterne, Gaditerne og Manasses halve Stamme Israeliterne i Silo i Kana'ans Land og vendte tilbage til Gilead, det Land, de havde fået i Eje, hvor de havde nedsat sig i Følge HERRENs Bud ved Moses;'
\par 10 og da Rubeniterne, Gaditerne og Manasses halve Stamme kom til Gelilot ved Jordan i Kana'ans Land, byggede de et Alter der ved Jordan, et stort Alter. der sås viden om.
\par 11 Men det kom Israeliterne for Øre, at Rubeniterne, Gaditerne og Manasses halve Stamme havde bygget et Alter på Grænsen af Kana'ans Land, ved Gelilot ved Jordan, på Israeliternes Side.
\par 12 Og da Israeliterne hørte det, samledes hele Israeliternes Menighed i Silo for at drage i Kamp imod dem.
\par 13 Da sendte Israeliterne Pinehas, Præsten Eleazars Søn, til Rubeniterne, Gaditerne og Manasses halve Stamme i Gilead
\par 14 tillige med ti Øverster, een Øverste for hver af alle Israels Stammer; hver af dem var Overhoved for sin Stamme iblandt Israels Tusinder;
\par 15 og da de kom til Rubeniterne, Gaditerne og Manasses halve Stamme i Gilead, talte de således til dem:
\par 16 "Således siger hele HERRENs Menighed: Hvad er det for en Troløshed, I har begået mod Israels Gud, at I i Dag har vendt eder fra HERREN ved at bygge eder et Alter og vise Genstridighed mod HERREN?
\par 17 Har vi ikke nok i Brøden med Peor, som vi endnu den Dag i Dag ikke har fået os renset for, og for hvis Skyld der kom Plage over Israels Menighed?
\par 18 Og dog vender I eder i Dag fra HERREN! Når I i Dag er genstridige mod HERREN, vil hans Vrede i Morgen bryde løs over hele Israels Menighed.
\par 19 Hvis det Land, I har fået i Eje, er urent, så gå over til det Land, der er HERRENs Ejendom, der, hvor HERRENs Bolig står, og nedsæt eder iblandt os; men vær ikke genstridige mod HERREN, ej heller mod os ved at bygge eder et Alter til foruden HERREN vor Guds Alter!
\par 20 Dengang Akan, Zeras Søn, øvede Svig med det bandlyste, kom der da ikke Vrede over hele Israels Menighed, skønt han kun var en enkelt Mand? Måtte han ikke dø for sin Brøde?"
\par 21 Da svarede Rubeniterne, Gaditerne og Manasses halve Stamme overhovederne for Israels Tusinder således:
\par 22 "Gud, Gud HERREN, Gud, Gud HERREN ved det, og Israel skal vide det: Hvis det er i Genstridighed eller Troløshed mod HERREN, i den Hensigt at vende os fra HERREN,
\par 23 at vi har bygget os et Alter, gid han så må unddrage os sin Hjælp i Dag! Hvis det er for at bringe Brændofre og Afgrødeofre derpå eller for at bringe Takofre derpå, så straffe HERREN det!
\par 24 Nej, vi har gjort det af Frygt for det Tilfælde, at eders Børn engang i Fremtiden skulde sige til vore: Hvad har I med HERREN, Israels Gud, at gøre?
\par 25 HERREN har jo sat Jordan som Grænse imellem os, og eder, Rubeniter og Gaditer; I har ingen Del i HERREN! Og således kunde eders Børn få vore til at høre op med at frygte HERREN.
\par 26 Derfor tænkte vi: Lad os bygge dette Alter, ikke til Brændoffer eller Slagtoffer,
\par 27 men for at det kan være Vidne mellem os og eder og mellem vore Efterkommere efter os om, at vi vil forrette HERRENs Tjeneste' for hans Åsyn med vore Brændofre, Slagtofre og Takofre, for at eders Børn ikke engang i Fremtiden skal sige til vore: I har ingen Del i HERREN!
\par 28 Og vi tænkte: Hvis de i Fremtiden siger således til os og vore Efterkommere, så siger vi: Læg dog Mærke til, hvorledes det HERRENs Alter er bygget, som vore Forfædre rejste, ikke til Brændofre eller Slagtofre, men for at det kunne være Vidne mellem os og eder.
\par 29 Det være langt fra os at være genstridige mod HERREN eller vende os fra HERREN i Dag ved at bygge et Alter til Brændoffer, Afgrødeoffer og Slagtoffer foruden HERREN vor Guds Alter, som står foran hans Bolig!"
\par 30 Da Præsten Pinehas og Menighedens Øverster og Overhovederne for Israels Tusinder, som ledsagede ham, hørte de Ord, som Rubeniterne, Gaditerne og Manassiterne talte, var de tilfredse,
\par 31 og Pinehas, Præsten Eleazars Søn, sagde til Rubeniterne, Gaditerne og Manassiterne: "I Dag erkender vi, at HERREN er iblandt os, siden I ikke har øvet denne Svig imod HERREN; derved har I frelst Israeliterne fra HERRENs Hånd!"
\par 32 Derpå vendte Pinehas, Præsten Eleazars Søn, og Øversteme tilbage fra Rubeniterne, Gaditerne og Manasses halve Stamme i Gilead til Israeliterne i Kana'ans Land og aflagde dem Beretning,
\par 33 og Israeliterne var tilfredse ved Meddelelsen, og Israeliterne priste Gud og tænkte ikke mere på at drage i Kamp mod dem for at ødelægge det Land, Rubeniterne, Gaditerne og Manasses halve Stamme boede i.
\par 34 Og Rubeniterne, Gaditerne og Manasses halve Stamme kaldte Alteret: Vidne; "thi," sagde de, "det skal være Vidne mellem os om, at HERREN er Gud!"

\chapter{23}

\par 1 Efter længere Tids Forløb, da HERREN havde skaffet Israel Ro for alle dets Fjender rundt om, og Josua var blevet gammel og til Års,
\par 2 lod Josua hele Israel, de Ældste, Overhovederne, Dommerne og Tilsynsmændene kalde til sig og sagde til dem: "Jeg er blevet gammel og til Års.
\par 3 I har selv set alt, hvad HERREN eders Gud har gjort ved alle disse Folkeslag foran eder; thi det var HERREN eders Gud, som kæmpede for eder.
\par 4 Se, jeg har tildelt eders Stammer som Arvelod disse Folk, som er tilbage af alle de Folkeslag, jeg udryddede fra Jordan til det store Hav vestpå;
\par 5 og HERREN eders Gud vil trænge dem tilbage foran eder og drive dem bort foran eder, og I skal tage deres Land i Besiddelse, som HERREN eders Gud lovede eder.
\par 6 Vær nu stærke og faste, så I giver Agt på og handler efter alt, hvad der står skrevet i Moses's Lovbog, og ikke viger derfra til højre eller venstre
\par 7 og ikke indlader eder med disse Folk, som er tilbage iblandt eder; I må ikke påkalde deres Guders Navne eller sværge ved dem, ikke dyrke eller tilbede dem,
\par 8 men I skal holde fast ved HERREN eders Gud som hidtil.
\par 9 Derfor drev jo HERREN store og mægtige Folkeslag bort foran eder. Ingen har kunnet holde Stand over for eder til denne Dag;
\par 10 een Mand iblandt eder jog tusinde på Flugt; thi det var HERREN eders Gud, som kæmpede for eder, som han havde lovet eder.
\par 11 Våg da for eders Livs Skyld omhyggeligt over, at I elsker HERREN eders Gud!
\par 12 Thi dersom I falder fra og slutter eder til Levningerne af disse Folk, som er tilbage iblandt eder, og, besvogrer eder med dem eller, indlader eder i Forbindelse med dem,
\par 13 så skal I vide for vist, at HERREN eders Gud ikke mere vil drive disse Folkeslag bort fra eder, men de skal blive eder en Snare og en Fælde, en Svøbe i eders Sider og Torne i eders Øjne, indtil I selv bliver udryddet fra dette herlige Land, som HERREN eders Gud gav eder.
\par 14 Se, jeg går nu al Støvets Gang; så betænk da med hele eders Hjerte og hele eders Sjæl, at ikke eet af alle de gode Ord, HERREN eders Gud talede til eder, faldt til Jorden; alle sammen er de gået i Opfyldelse for eder; ikke eet Ord deraf faldt til Jorden.
\par 15 Men ligesom alle de gode Ord, HERREN eders Gud talede til eder, gik i Opfyldelse på eder, således vil HERREN også lade alle sine Trusler gå i Opfyldelse på eder, indtil han har udryddet eder fra dette herlige Land, som HERREN eders Gud gav eder.
\par 16 Når I overtræder HERREN eders Guds Pagt, som han pålagde eder, og går hen og Dyrker andre Guder og tilbeder dem, så vil HERRENs Vrede blusse op imod eder, og I vil hastelig blive udryddet fra det herlige Land, han gav eder!"

\chapter{24}

\par 1 Derpå kaldte Josua alle Israels Stammer sammen i Sikem og lod Israels Ældste og Overhoved, Dommere og Tilsynsmænd kalde til sig; og de stillede sig op for Guds Åsyn.
\par 2 Da sagde Josua til hele Folket: "Så siger HERREN, Israels Gud: Hinsides Floden boede eders Forfædre i gamle Dage, Tara, Abrahams og Nakors Fader, og de dyrkede andre Guder.
\par 3 Da førte jeg eders Stamfader Abraham bort fra Landet hinsides Floden og lod ham vandre omkring i hele Kana'ans Land, gav ham en talrig Æt og skænkede ham Isak.
\par 4 Og Isak skænkede jeg Jakob og Esau, og Esau gav jeg Se'irs Bjerge i Eje, medens Jakob og hans Sønner drog ned til Ægypten.
\par 5 Derpå sendte jeg Moses og Aron, og jeg plagede Ægypterne med de Gerninger, jeg øvede iblandt dem, og derefter førte jeg eder ud;
\par 6 og da jeg førte eders Fædre ud af Ægypten, og I var kommet til Havet, satte Ægypterne efter eders Fædre med Stridsvogne og Ryttere til det røde Hav.
\par 7 Da råbte de til HERREN, og han satte Mørke mellem eder og Ægypterne og bragte Havet over dem, så det dækkede dem; og I så med egne Øjne, hvad jeg gjorde ved Ægypterne. Og da I havde opholdt eder en Tid lang i Ørkenen,
\par 8 førte jeg eder ind i Amoriternes Land hinsides Jordan, og da de angreb eder, gav jeg dem i eders Hånd, så I tog deres Land i Besiddelse, og jeg tilintetgjorde dem foran eder.
\par 9 Da rejste Zippors Søn, Kong Balak af Moab, sig og angreb Israel; og han sendte Bud og lod Bileam, Beors Søn, hente, for at han skulde forbande eder;
\par 10 men jeg vilde ikke bønhøre Bileam, og han måtte velsigne eder; således friede jeg eder af hans Hånd.
\par 11 Derpå gik I over Jordan og kom til Jeriko; og Indbyggerne i Jeriko, Amoriterne, Perizziterne, Kana'anæerne, Hetiterne, Girgasjiterne, Hivviterne og Jebusiterne angreb eder, men jeg gav dem i eders Hånd.
\par 12 Jeg sendte Gedehamse foran eder, og de drev de tolv Amoriterkonger bort foran eder; det skete ikke ved dit Sværd eller din Bue.
\par 13 Og jeg gav eder et Land, I ikke havde haft Arbejde med, og Byer, I ikke havde bygget, og I tog Bolig i dem; af Vinhaver og Oliventræer, I ikke plantede, nyder I nu Frugten.
\par 14 Så frygt nu HERREN og tjen ham i Oprigtighed og Trofasthed, skaf de Guder bort, som eders Forfædre dyrkede hinsides Floden og i Ægypten, og tjen HERREN!
\par 15 Men hvis I ikke synes om at tjene HERREN, så vælg i Dag, hvem I vil tjene, de Guder, eders Forfædre dyrkede hinsides Floden, eller Amoriternes Guder, i hvis Land I nu bor. Men jeg og mit Hus, vi vil tjene HERREN!"
\par 16 Da svarede Folket: "Det være langt fra os at forlade HERREN for at dyrke andre Guder;
\par 17 nej, HERREN er vor Gud, han, som førte os og vore Fædre op fra Ægypten, fra Trællehuset, og gjorde hine store Tegn for vore Øjne og bevarede os under hele vor Vandring og blandt alle de Folk, hvis Lande vi drog igennem;
\par 18 og HERREN drev alle disse Folk og Amoriterne, som boede her i Landet, bort foran os. Derfor vil vi også tjene HERREN, thi han er vor Gud!"
\par 19 Da sagde Josua til Folket: "I vil ikke kunne tjene HERREN, thi han er en hellig Gud; han er en nidkær Gud, som ikke vil tilgive eders Overtrædelser og Synder.
\par 20 Når I forlader HERREN og dyrker fremmede Guder, vil han vende sig bort og bringe Ulykke over eder og tilintetgøre eder, skønt han tidligere gjorde vel imod eder."
\par 21 Da sagde Folket til Josua: "Nej HERREN vil vi tjene!"
\par 22 Josua sagde da til Folket: "I er Vidner imod eder selv på, at I har valgt at tjene HERREN.
\par 23 Så skaf da de fremmede Guder bort, som I har hos eder, og bøj eders Hjerte til HERREN, Israels Gud!"
\par 24 Da sagde Folket til Josua: "HERREN vor Gud vil vi tjene, og hans Røst vil vi lyde!"
\par 25 Derpå lod Josua samme bag Folket indgå en Pagt, og han fastsatte det Lov og Ret i Sikem.
\par 26 Og Josua opskrev disse Ord i Guds Lovbog; og han tog en stor Sten og rejste den der under den Eg, som står i HERRENs Helligdom;
\par 27 og Josua sagde til hele Folket: "Se, Stenen her skal være Vidne imod os; thi den har hørt alle HERRENs Ord, som han talede til os; den skal være Vidne imod eder; at I ikke skal fornægte eders Gud!"
\par 28 Derpå lod Josua Folket drage bort hver til sin Arvelod.
\par 29 Efter disse Begivenheder døde HERRENs Tjener Josua, Nuns Søn 110 År gammel.
\par 30 Og de jordede ham på hans Arvelod i Timnat-Sera i Efraims Bjerge norden for Bjerget Ga'asj.
\par 31 Og Israel dyrkede HERREN, så længe Josua levede, og så længe de Ældste var i Live, som overlevede Josua, og som havde kendt hele det Værk, HERREN havde øvet for Israel.
\par 32 Men Josefs Ben, som Israeliterne havde bragt op fra Ægypten jordede de i Sikem på den Mark Jakob havde købt af Hamors, Sikems Faders, Sønner for hundrede Kesita, og som han havde givet Josef i Eje.
\par 33 Da Arons Søn Eleazar døde jordede de ham i hans Søn Pinehass By Gibea, som var givet ham i Efraims Bjerge.



\end{document}