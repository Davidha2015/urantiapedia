\begin{document}

\title{Judges}


\chapter{1}

\par 1 Efter Josuas Død adspurgte Israelitterne HERREN og sagde: "Hvem af os skal først drage op til Kamp mod Kana'anæerne?"
\par 2 HERREN svarede: "Det skal Juda; se, jeg giver Landet i hans Hånd!"
\par 3 Juda sagde da til sin Broder Simeon: "Drag op med mig i min Lod og lad os sammen kæmpe med Kana'anæerne, så skal jeg også drage med dig ind i din Lod!" Så gik da Simeon med ham.
\par 4 Juda drog nu op, og HERREN gav Kana'anæerne og Perizziterne i deres Hånd, så de slog dem i Bezek, 10000 Mand.
\par 5 Og da de stødte på Adonibezek i Bezek, angreb de ham og slog Kana'anæerne og Perizziterne.
\par 6 Adonibezek flygtede, men de satte efter ham, og da de havde grebet ham, huggede de Tommelfingrene og Tommeltæerne af ham.
\par 7 Da sagde Adonibezek: "Halvfjerdsindstyve Konger med afhugne Tommelfingre og Tommeltæer havde jeg stadig til at sanke Smuler under mit Bord; hvad jeg har gjort, gengælder Gud mig!" Derpå førte man ham til Jerusalem, og der døde han.
\par 8 Og Judæerne angreb og indtog Jerusalem, huggede Indbyggerne ned og stak Ild på Byen.
\par 9 Senere drog Judæerne ned til Kamp mod Kana'anæerne i Bjergene, i Sydlandet og i Lavlandet.
\par 10 Og Juda drog mod Kana'anæerne i Hebron - Hebron hed fordum Kirjat Arba og slog Sjesjaj,Ahiman og Talmaj.
\par 11 Derfra drog han op mod Indbyggerne i Debir, der fordum hed Kirjaf-Sefer.
\par 12 Da sagde Kaleb: "Den, som slår Kirjat Sefer og indtager det, ham giver jeg min Datter Aksa til Hustru!"
\par 13 Og da Kenizziten Otniel,Kalebs yngre Broder, indtog det, gav han ham sin Datter Aksa til Hustru.
\par 14 Men da hun kom til ham, æggede han hende til at bede sin Fader om Agerland. Hun sprang da ned af Æselet, og Kaleb spurgte hende: "Hvad vil du?"
\par 15 Hun svarede: "Giv mig en Velsignelse! Siden du har bortgiftet mig i det tørre Sydland, må du give mig Vandkilder!" Da gav Kaleb hende de øvre og de nedre Vandkilder.
\par 16 Moses's Svigerfaders, Keniten Kobabs, Sønner, drog sammen med Judæerne op fra Palmestaden til Arads Ørken og bosatte sig hos Amalekiterne.
\par 17 Juda drog derpå ud med sin Broder Simeon, og de slog Kana'anæerne, som boede i Zefat, og lagde Band på Byen; derfor blev den kaldt Horma.
\par 18 Og Juda indtog Gaza med dets Område, Askalon med dets Område og Ekron med dets Område.
\par 19 Og HERREN var med Juda, så han tog Bjerglandet i Besiddelse; Lavlandets Indbyggere kunde han nemlig ikke drive bort, fordi de havde Jernvogne.
\par 20 Haleb gav de Hebron, som Moses havde sagt. Og han drev de tre Anaksønner bort derfra.
\par 21 Men Jebusiterne, som boede i Jerusalem, fik Judæerne ikke drevet bort, og Jebusiterne bor den Dag i Dag i Jerusalem sammen med Judæerne.
\par 22 Men også Josefs Hus drog op og gik mod Betel; og HERREN var med dem.
\par 23 Da Josefs Hus udspejdede Betel Byen hed fordum Luz
\par 24 fik Spejderne Øje på en Mand, der gik ud af Byen; og de sagde til ham: "Vis os, hvor vi kan komme ind i Byen, så vil vi skåne dig!"
\par 25 Da viste han dem, hvor de kunde komme ind i Byen. Derpå huggede de dens Indbyggere ned, men Manden og hele hans. Slægt lod de drage bort,
\par 26 og Manden begav sig til Hetiternes Land og byggede en By, som han kaldte Luz; og det hedder den endnu den Dag i Dag.
\par 27 Men Indbyggerne i Bet-Sjean og dets Småbyer og i Ta'anak og dets Småbyer, Indbyggerne i Dor og dets Småbyer, i Jibleam og dets Småbyer og i Megiddo og dets Småbyer fik Manasse ikke drevet bort; det lykkedes Kana'anæerne at blive boende i disse Egne.
\par 28 Da Israelitterne blev de stærkeste, gjorde de Kana'anæerne til Hoveriarbejdere, men drev dem ikke bort.
\par 29 Efraim fik ikke Kana'anæerne, som boede i Gezer, drevet bort; men Kana'anæerne blev boende midt iblandt dem i Gezer.
\par 30 Zebulon fik ikke Indbyggerne i Kitron og Nahalol drevet bort; men Kana'anæerne blev boende midt iblandt dem og blev Hoveriarbejdere.
\par 31 Aser fik ikke Indbyggerne i Akko drevet bort, ej heller Indbyggerne i Zidon, Mahalab, Akzib, Afik og Rehob.
\par 32 Men Aseriterne bosatte sig midt iblandt Kana'anæerne, der boede i Landet, thi de magtede ikke at drive dem bort.
\par 33 Naftali fik ikke Indbyggerne i Bet Sjemesj og Bet-Anat drevet bort, men bosatte sig midt iblandt Kana'anæerne, der boede i Landet: Men Indbyggerne i Bet Sjemesj og Bet Anat blev deres Hoveriarbejdere.
\par 34 Amoriterne trængte Daniterne op i Bjergene og lod dem ikke komme ned i Lavlandet;
\par 35 og det lykkedes Amoriterne at blive boende i Har-Heres, Ajjalon og Sja'albim. Men senere, da Josefs Hus fik Overtaget, blev de Hoveriarbejdere.
\par 36 Edomiternes Landemærke strakte sig fra Akrabbimpasset til Sela og højere op.

\chapter{2}

\par 1 HERRENs Engel drog fra Gilgal op til Betel. Og han sagde: "Jeg førte eder op fra Ægypten og bragte eder ind i det Land, jeg tilsvor eders Fædre. Og jeg sagde: Jeg vil i Evighed ikke bryde min Pagt med eder!
\par 2 Men I må ikke slutte Pagt med dette Lands Indbyggere; deres Altre skal I bryde ned! Men t adlød ikke min Røst! Hvad har I dog gjort!
\par 3 Derfor siger jeg nu: Jeg vil ikke drive dem bort foran eder, men de skal blive Brodde i eders Sider, og deres Guder skal blive eder en Snare!"
\par 4 Da HERRENs Engel talede disse Ord til alle Israelitterne, brast Folket i Gråd.
\par 5 Derfor kaldte man Stedet Bokim. Og de ofrede til HERREN der.
\par 6 Da Josua havde ladet Folket fare, drog Israelitterne hver til sin Arvelod for at tage Landet i Besiddelse.
\par 7 Og Folket dyrkede HERREN, så længe Josua levede, og så længe de Ældste var i Live, som overlevede Josua og havde set hele det Storværk, HERREN havde øvet for Israel.
\par 8 Og Josua, Nuns Søn, HERRENs Tjener, døde, 110 År gammel;
\par 9 og de jordede ham på hans Arvelod i Timnat-Heres i Efraims Bjerge norden for Bjerget Ga'asj.
\par 10 Men også hele hin Slægt samledes til sine Fædre, og efter dem kom en anden Slægt, som hverken kendte HERREN eller det Værk, han havde øvet for Israel.
\par 11 Da gjorde Israelitterne, hvad der var ondt i HERRENs Øjne, og dyrkede Ba'alerne;
\par 12 de forlod HERREN, deres Fædres Gud, som havde ført dem ud af Ægypten, og holdt sig til andre Guder, de omboende Folks Guder, og tilbad dem og krænkede HERREN.
\par 13 De forlod HERREN og dyrkede Ba'al og Astarte.
\par 14 Da blussede HERRENs Vrede op imod Israel, og han gav dem i Røveres Hånd, så de udplyndrede dem. Han gav dem til Pris for de omboende Fjender, så de ikke længer kunde holde Stand mod deres Fjender.
\par 15 Hvor som helst de rykkede frem, var HERRENs Hånd imod dem og voldte dem Ulykke, som HERREN havde sagt og tilsvoret dem.
\par 16 lod han Dommere fremstå, og de frelste dem fra deres Hånd, som udplyndrede dem.
\par 17 Dog heller ikke deres Dommere adlød de, men bolede med andre Guder og tilbad dem. Hurtig veg de bort fra den Vej, deres Fædre havde vandret på i Lydighed mod HERRENs Bud; de slægtede dem ikke på.
\par 18 Men hver Gang HERREN lod Dommere fremstå iblandt dem, var HERREN med Dommeren og frelste dem fra deres Fjenders Hånd, så længe Dommeren levede; thi HERREN ynkedes, når de jamrede sig over dem, som trængte og undertrykte dem.
\par 19 Men så snart Dommeren var død, handlede de atter ilde, ja endnu værre end deres Fædre, idet de holdt sig til andre Guder og dyrkede og tilbad dem. De holdt ikke op med deres onde Gerninger og genstridige Færd.
\par 20 Da blussede HERRENs Vrede op mod Israel, og han sagde: "Efterdi dette Folk har overtrådt min Pagt, som jeg pålagde deres Fædre, og ikke adlydt min Røst,
\par 21 vil jeg heller ikke mere bortdrive foran dem et eneste af de Folk, som Josua lod tilbage ved sin Død,
\par 22 for at jeg ved dem kan sætte Israel på Prøve, om de omhyggeligt vil følge HERRENs Veje, som deres Fædre gjorde, eller ej."
\par 23 HERREN lod da disse Folkeslag blive og hastede ikke med at drive dem bort; han gav dem ikke i Josuas Hånd.

\chapter{3}

\par 1 Dette var de Folkeslag, HERREN lod blive tilbage for ved dem at sætte Israel Prøve, alle de Israelitter, som ikke havde kendt til Kampene om Kana'an,
\par 2 alene for Israelitternes Slægters Skyld, for at øve dem i Krig, alene for deres Skyld, som ikke havde kendt noget til de tidligere Krige:
\par 3 De fem Filisterfyrster, alle Kana'anæerne, Zidonierne og Hetiterne, som boede i Libanons Bjerge fra Ba'al-Hermon til hen imod Hamat.
\par 4 De blev tilbage, for at Israel ved dem kunde blive sat på Prøve, så det kunde komme for Dagen, om de vilde lyde de Bud, HERREN havde givet deres Fædre ved Moses.
\par 5 Og Israelitterne boede blandt Kana'anæerne, Hetiterne, Amoriterne, Perizziterne, Hivviterne og Jebusiterne.
\par 6 De indgik Ægteskab med deres Døtre og gav deres Sønner deres Døtre til Ægte, og de dyrkede deres Guder.
\par 7 Israelitterne gjorde, hvad der var ondt i HERRENs Øjne; de glemte HERREN deres Gud og dyrkede Ba'alerne og Asjererne.
\par 8 Da blussede HERRENs Vrede op imod Israel, og han gav dem til Pris for Kong Kusjan-Risjatajim af Aram Naharajim, og Israelitterne stod under Kusjan Risjatajim i otte År.
\par 9 Da råbte Israelitterne til HERREN, og HERREN lod en Befrier fremstå iblandt Israelitterne, og han frelste dem, nemlig Kenizziten Otniel, Kalebs yngre Broder.
\par 10 HERRENs Ånd kom over ham, og han blev Dommer i Israel; han drog ud til Kamp, og HERREN gav Kong Kusjan Risjatajim af Aram i hans Hånd, så han fik Overtaget over ham.
\par 11 Og Landet havde Ro i fyrretyve År, og Kenizziten Otniel døde.
\par 12 Men Israelitterne blev ved at gøre, hvad der var ondt i HERRENs Øjne. Da gav HERREN Kong Eglon af Moab Magt over Israel, fordi de gjorde, hvad der var ondt i HERRENs Øjne.
\par 13 Han fik Ammoniterne og Amalekiterne med sig, drog ud og slog Israel og tog Palmestaden.
\par 14 Og Israelitterne stod under Kong Eglon af Moab i atten År.
\par 15 Men da Israelitterne råbte til HERREN, lod HERREN en Befrier fremstå iblandt dem, Benjaminiten Ehud, Geras Søn, som var kejthåndet. Da Israelitterne engang sendte Ehud til Kong Eglon af Moab med Skat,
\par 16 lavede han sig et tveægget Sværd, en Gomed langt, og bandt det til sin højre Hofte under Kappen.
\par 17 Derpå afleverede han Skatten til Kong Eglon af Moab Eglon var en meget fed Mand
\par 18 og da han var færdig dermed, ledsagede han Folkene, der havde båret Skatten, på Vej,
\par 19 men selv vendte han om ved Pesilim ved Gilgal og sagde: "Konge, jeg har noget at tale med dig om i Hemmelighed!" Men han bød ham tie, til alle de, der stod om ham, var gået ud.
\par 20 Da Ehud kom ind til ham, sad han i sin svale Stue på Taget, hvor han var alene; og Ehud sagde: "Jeg har et Gudsord til dig!" Da rejste han sig fra sit Sæde,
\par 21 men idet han stod op, rakte Ehud sin venstre Hånd ud og greb Sværdet ved sin højre Side og stak det i Underlivet på ham,
\par 22 så endog Grebet gik i med Klingen; og Fedtet sluttede om Klingen, thi han drog ikke Sværdet ud af Livet på ham; og Skarnet gik fra ham.
\par 23 Så gik Ehud bort gennem Søjlegangen, lukkede Døren til Stuen for ham og låsede den.
\par 24 Efter at han var gået bort, kom Kongens Folk, og da de fandt Døren til Stuen låset, tænkte de, at han tildækkede sine Fødder i det svale Hammer;
\par 25 men da de havde biet, indtil de skammede sig, uden at han åbnede Døren til Stuen, hentede de Nøglen og lukkede op; og se, der lå deres Herre død på Gulvet.
\par 26 Men Ehud slap bort, medens de blev opholdt, og han satte over ved Pesilim og undslap til Seira.
\par 27 Så snart han derpå nåede Efraims Bjerge, lod han støde i Hornet; og Israelitterne fulgte ham ned fra Bjergene, idet han gik i Spidsen for dem.
\par 28 Og han sagde til dem: "Følg mig ned, thi HERREN har givet eders Fjender Moabiterne i eders Hånd!" Og de fulgte ham ned og fratog Moabiterne Vadestederne over Jordan og lod ikke en eneste komme over.
\par 29 Ved den Lejlighed nedhuggede de omtrent 10000 Moabiter, lutter stærke og dygtige Mænd, ikke een undslap.
\par 30 Således blev Moab den Gang underkuet af Israel; og Landet havde Ro i firsindstyve År.
\par 31 Efter ham kom Sjamgar, Anats Søn. Han nedhuggede Filisterne, 600 Mand, med en Oksedriverstav; også han frelste Israel.

\chapter{4}

\par 1 Men da Ehud var død, blev Israelitterne ved at gøre, hvad der var ondt i HERRENs Øjne.
\par 2 Derfor gav han dem til Pris, for Kana'anæerkongen Jabin, som herskede i Hazor; hans Hærfører var Sisera, som boede i Harosjet-Haggojim.
\par 3 Da råbte Israelitterne til HERREN. Thi Jabin havde 900 Jernvogne, og han trængte Israelitterne hårdt i tyve År.
\par 4 Profetinden Debora, Lappidots Hustru, var på den Tid Dommer i Israel;
\par 5 hun sad under Deborapalmen imellem Rama og Betel i Efraims Bjerge, og Israelitterne drog op til hende med deres Retstrætter.
\par 6 Hun sendte nu Bud efter Barak, Abinoams Søn fra Kedesj i Naftali, og sagde til ham: "Har ikke HERREN, Israels Gud, budt: Bryd op, drag hen på Tabors Bjerg og tag 10000 Mand af Naftali og Zebulon med dig,;
\par 7 så skal jeg drage Jabins Hærfører Sisera med hans Vogne og Hærstyrke hen til dig ved Kisjonbækken og give ham i din Hånd!"
\par 8 Barak svarede hende: "Hvis du vil gå med, vil jeg gå; men hvis du ikke går med, går jeg ikke!"
\par 9 Da sagde hun: "Vel, jeg går med, men så får du ikke Æren af den Færd, du begiver dig ud på, thi HERREN vil overgive Sisera i en Kvindes Hånd!" Så brød Debora op og drog af Sted til Kedesj med Barak.
\par 10 Barak stævnede nu Zebulon og Naftali sammen i Kedesj, og 10000 Mand fulgte med ham derop; også Debora gik med.
\par 11 Men Keniten Heber havde skilt sig fra Keniterne, Moses's Svigerfader Hobabs Sønner, og slået Telt i Egnen hen imod Elon-Beza'anannim ved Kedesj.
\par 12 Da Sisera fik Melding om, at Barak, Abinoams Søn, var draget op på Tabors Bjerg,
\par 13 stævnede han alle sine Stridsvogne, 900 jernbeslagne Vogne, og hele sin Krigsstyrke fra Harosjet Haggojim til Kisjonbækken.
\par 14 Da sagde Debora til Barak: "Bryd nu op! Thi det er i Dag.
\par 15 Og foran Barak bragte HERREN Uorden iblandt alle Siseras Stridsvogne og i hele hans Hær. Sisera sprang af sin Vogn og flygtede til Fods;
\par 16 men Barak satte efter Vognene og Hæren lige til Harosjet Haggojim, og hele Siseras Hær faldt for Sværdet, ikke een blev tilbage.
\par 17 Sisera var imidlertid flygtet til Fods til Keniten Hebers Hustru Jaels Telt, thi der var Fred imellem Kong Jabin af Hazor og Keniten Hebers Slægt.
\par 18 Da gik Jael Sisera i Møde og sagde til ham: "Tag dog ind hos mig, Herre, du bar intet at frygte!" Han tog da ind i Teltet hos hende, og hun dækkede ham til med et Tæppe.
\par 19 Og han sagde til hende: "Giv mig lidt Vand at drikke, thi jeg er tørstig!" Da åbnede hun Mælkesækken og gav ham at drikke og dækkede ham atter til.
\par 20 Så sagde han til hende: "Stil dig hen i Teltdøren, og hvis der kommer en og spørger, om der er nogen herinde, så sig nej!"
\par 21 Men Jael, Hebers Hustru, greb en Teltpæl og tog en Hammer i Hånden, listede sig ind til ham og slog Pælen igennem hans Tinding, så den trængte ned i Jorden; thi han var faldet i dyb Søvn, træt som han var; således døde han.
\par 22 Og se, Barak, søm forfulgte Sisera, kom forbi. Da gik Jael ham i Møde og sagde til ham: "Kom, jeg skal vise dig den Mand, du søger efter!" Så kom han ind til hende. Og se, der lå Sisera død med Teltpælen gennem Tindingen.
\par 23 Således lod Gud på den Dag Kana'anæerkongen Jabin bukke under for Israelitterne;
\par 24 og Israelitternes Hånd faldt hårdere og hårdere på Kana'anæerkongen Jabin, til de fik ham tilintetgjort.

\chapter{5}

\par 1 Da sang Debora og Barnk, Abinoams Søn, denne Sang:
\par 2 Frem stod Høvdinger i Israel, Folket gav villigt Møde, lover HERREN!
\par 3 Hør, I Konger, lyt, I Fyrster: Synge vil jeg, synge for HERREN, lovsynge HERREN, Israels Gud!
\par 4 HERRE, da du brød op fra Seir, skred frem fra Edoms Mark, da rystede Jorden, Himmelen drypped, Skyerne drypped af Vand;
\par 5 Bjergene bæved for HERRENs Åsyn, for HERREN Israels Guds Åsyn!
\par 6 I Sjamgars, Anats Søns, Dage, i Jaels Dage lå Vejene øde, vejfarende sneg sig ad afsides Stier;
\par 7 der var ingen Fører i Israel mer, til jeg Debora stod frem, stod frem, en Moder i Israel.
\par 8 Ofre til Gud hørte op, med Bygbrødet fik det en Ende. Så man vel Skjold eller Spyd hos Israels fyrretyve Tusind?
\par 9 For Israels Førere slår mit Hjerte, for de villige af Folket! Lover HERREN!
\par 10 I, som rider på rødgrå Æsler, I, som sidder på Tæpper, I, som færdes på Vejene, syng!
\par 11 Hør, hvor de spiller mellem Vandtrugene! Der lovsynger de HERRENs Frelsesværk, hans Værk som Israels Fører. Da drog HERRENs Folk ned til Portene.
\par 12 Op, op, Debora, op, op, istem din Sang! Barak, stå op! Fang dig Fanger, du Abinoams Søn!
\par 13 Da drog Israel ned som Helte, som vældige Krigere drog HERRENs Folk frem.
\par 14 Fra Efraim steg de ned i Dalen, din broder Benjamin var blandt dine Skarer. Fra Makir drog Høvedsmænd ned, fra Zebulon de, der bar Herskerstav;
\par 15 Issakars Førere fulgte Debora, Naftali Baraks Spor, de fulgte ham ned i Dalen. Ved Rubens Bække var Betænkelighederne store.
\par 16 Hvorfor blev du mellem Foldene for at lytte til Hyrdernes Fløjter? Ved Rubens Bække var Betænkelighederne store!
\par 17 Gilead blev på hin Side Jordan, og Dan, hvi søgte han fremmed Hyre? Aser sad stille ved Havets Strand; han blev ved sine Vige.
\par 18 Zebulon var et Folk, der vovede Livet, Naftali med på Markens Høje.
\par 19 Kongerne kom, de kæmped; da kæmped Kana'ans Konger ved Ta'anak, ved Megiddos Vande de fanged ej Sølv som Bytte!
\par 20 Fra Himmelen kæmped Stjernerne, fra deres Baner stred de mod Sisera!
\par 21 Kisjon Bæk rev dem bort, Kisjons Bæk, den ældgamle Bæk. Træd frem, min Sjæl, med Styrke!
\par 22 Da stampede Hestenes Hove under Heltenes jagende Fart.
\par 23 "Forband", sagde HERRENs Engel "forband Meroz og dem, der bor deri! fordi de ikke kom HERREN til Hjælp kom HERREN til Hjælp som Helte!"
\par 24 Velsignet blandt Kvinder være Jael, Keniten Hebers Hustru, velsignet blandt Kvinder i Telte!
\par 25 Han bad om Vand, hun gav ham Mælk, frembar Surmælk i kostbar Skål.
\par 26 Med Hånden griber hun Pælen, med sin højre Arbejdshammeren, fælder Sisera, kløver hans Hoved, knuser, gennemborer hans Tinding.
\par 27 For hendes Fødder han segned og faldt; der, hvor han segned, der lå han fældet!
\par 28 Gennem Vinduet spejded Siseras Moder, gennem Gitteret stirred hun ud: "Hvi tøver hans Vogn med at komme? Hvi nøler hans Forspands Hovslag?"
\par 29 Da svarer den klogeste af hendes Fruer, og selv hun giver sig samme Svar: "Sikkert de deler det vundne Bytte, en Pige eller to til Mands,
\par 30 Bytte af spraglede Tøj er til Sisera, et broget Klæde eller to til hans Hals!"
\par 31 Således skal alle dine Fjender forgå, HERRE, men de, der elsker dig, skal være, som når Sol går op i sin Vælde! Derpå havde Landet Ro i fyrretyve År.

\chapter{6}

\par 1 Men da Israelitterne gjorde, hvad der var ondt i HERRENS Øjne, gav han dem syv År i Midjans Hånd.
\par 2 Og Midjan fik Overtaget over Israel. For at værge sig mod Midjan indrettede Israelitterne sig de Smuthuller, som findes i Bjergene, Hulerne og Klippeborgene.
\par 3 Hver Gang Israelitterne havde sået, kom Midjaniterne, Amalekiterne og Østens Stammer og drog op imod dem
\par 4 og lejrede sig imod dem, ødelagde Jordens Afgrøde lige til Egnen om Gaza og levnede intet at leve af i Israel, ej heller Småkvæg, Hornkvæg eller Æsler;
\par 5 thi de drog op med deres Hjorde og Telte og kom talrige som Græshopper, så hverken de selv eller deres Kameler kunde tælles, og de trængte ind i Landet for at hærge det.
\par 6 Således blev Israel rent forarmet ved Midjaniternes indfald, og Israelitterne råbte til HERREN.
\par 7 Men da Israelitterne råbte til HERREN over Midjaniterne,
\par 8 sendte han en Profet til dem, og denne sagde til dem: "Så siger HERREN, Israels Gud: Jeg førte eder op fra Ægypten, jeg førte eder ud af Trællehuset,
\par 9 jeg friede eder af Ægyptens Hånd og af deres Hånd, der trængte eder, og jeg drev dem bort foran eder og gav eder deres Land.
\par 10 Og jeg sagde til eder: Jeg er HERREN eders Gud; frygt ikke Amoriternes Guder, i hvis Land I bor! Men I adlød ikke min Røst!"
\par 11 Da kom HERRENs Engel og satte sig under Egen i Ofra, som tilhørte Abiezriten Joasj, medens hans Søn Gideon var ved at tærske Hvede i Vinpersen for at have den i Sikkerhed for Midjaniterne.
\par 12 HERRENs Engel viste sig for ham og sagde til ham: "HERREN er med dig, stærke Kriger!"
\par 13 Men Gideon svarede ham: "Ak, Herre! Hvis HERREN er med os, hvorledes er da alt dette kommet over os? Og hvad er der blevet af alle hans Undergerninger, som vore Fædre fortalte os om, idet de sagde: Førte HERREN os ikke ud af Ægypten? Nu har HERREN forstødt os og givet os i Midjans Hånd!"
\par 14 Da vendte HERREN sig til ham og sagde: "Drag hen i denne din Kraft, så skal du frelse Israel af Midjans Hånd; sandelig, jeg sender dig!"
\par 15 Men han svarede ham: "Ak, Herre! Hvorledes skal jeg kunne frelse Israel? Se, min Slægt er den ringeste i Manasse og jeg den yngste i mit Fædrenehus!"
\par 16 Han svarede ham: "HERREN vil være med dig, og du skal hugge Midjaniterne ned alle som een!"
\par 17 Da sagde han til ham: "Hvis jeg har fundet Nåde for dine Øjne, så lad mig få et Tegn på, at det er dig, som taler med mig;
\par 18 gå ikke herfra, før jeg kommer tilbage til dig og bringer dig min Gave og stiller den frem for dig!" Han svarede: "Jeg skal blive, til du kommer tilbage!"
\par 19 Gideon gik da ind og tillavede et Gedekid og usyrede Brød af en Efa Mel; Kødet lagde han i en Kurv, og Suppen hældte han i en Krukke og bar det ud til ham under Egen. Da han kom hen til ham med det,
\par 20 sagde Guds Engel til ham: "Tag Kødet og det usyrede Brød. Læg det på Klippen der og hæld Suppen ud derover!" Og han gjorde det.
\par 21 Da udrakte HERRENs Engel Spidsen af den Stav, han havde i Hånden, og rørte ved Kødet og Brødet. Og Ild slog op af Klippen og fortærede Kødet og Brødet; og HERRENs Engel forsvandt for hans Blik.
\par 22 Gideon skønnede nu, at det havde været HERRENs Engel; og han sagde: "Ve, Herre, HERRE, jeg har jo set HERRENs Engel Ansigt til Ansigt!"
\par 23 Men HERREN sagde til ham: "Fred være med dig! Frygt ikke, du skal ikke dø!"
\par 24 Da byggede Gideon HERREN et Alter der og kaldte det "HERREN er Fred"; det står endnu den Dag i Dag i Abiezriternes Ofra.
\par 25 Samme Nat sagde HERREN til ham: "Tag ti af dine Trælle og en syvårs Tyr; nedbryd din Faders Ba'alsalter og hug Asjerastøtten om, som står derved;
\par 26 byg så af Stensætningen HERREN din Gud et Alter på Toppen af Klippen her og tag Tyren og brænd den som Brændoffer med Træet af den omhuggede Asjerastøtte!"
\par 27 Gideon tog da ti af sine Trælle og gjorde, som HERREN bød ham; men han gjorde det om Natten, thi af Frygt for sin Familie og sine Bysbørn turde han ikke gøre det om Dagen.
\par 28 Da Byens Folk næste Morgen tidlig så Ba'alsalteret nedbrudt, Asjerastøtten ved Siden af hugget om og Tyren ofret på det nybyggede Alter,
\par 29 sagde de til hverandre: "Hvem mon der har gjort det?" Og da de spurgte sig for og foretog en Undersøgelse, blev der sagt, at det var Gideon, Joasjs Søn.
\par 30 Og Byens Folk sagde til Joasj: "Udlever din Søn, for at han kan lide Døden, thi han har nedbrudt Ba'alsalteret og omhugget Asjerastøtten ved Siden af!"
\par 31 Men Joasj svarede alle dem, der stod omkring ham: "Vil I virkelig stride for Ba'al eller hjælpe ham? Den, der strider for ham, skal dø inden i Morgen! Er han Gud, så lad ham stride for sig selv, siden hans Alter er nedbrudt!"
\par 32 Ved den Lejlighed fik Gideon Navnet Jerubba'al, idet man sagde: "Lad Ba'al stride med ham, siden han har nedbrudt hans Alter!"
\par 33 Alle Midjaniterne, Amalekiterne og Østens Stammer sluttede sig sammen, satte over Jordan og slog Lejr på Jizre'elsletten.
\par 34 Da iklædte HERRENs Ånd sig Gideon, og han stødte i Hornet; og Abiezriterne fylkede sig om ham.
\par 35 Og da han sendte Bud ud i hele Manasse, fylkede de sig også om ham; og han sendte Bud ud i Aser, Zebulon og Naftali, og de drog op før at møde Fjenderne.
\par 36 Da sagde Gideon til Gud: "Hvis du vil frelse Israel ved min Hånd, som du har lovet,
\par 37 så lægger jeg nu dette Fåreskind på Tærskepladsen, og falder der så Dug alene på Skindet, medens Jorden ellers bliver ved at være tør, da ved jeg, at du vil frelse Israel ved min Hånd, som du har lovet."
\par 38 Og det skete således. Da han næste Morgen vred Skindet, pressede han Dug af det, en hel Skålfuld Vand.
\par 39 Men Gideon sagde til Gud: "Vredes ikke på mig, når jeg endnu denne ene Gang taler til dig, lad mig blot denne Gang endnu prøve med Skindet: Lad Skindet alene være tørt, medens der falder Dug på Jorden rundt om!"
\par 40 Da gjorde Gud således om Natten: Skindet alene var ført, men der faldt Dug på Jorden rundt om.

\chapter{7}

\par 1 Næste Morgen tidlig brød Jerubba'al, det er Gideon, op med alle sine Folk og lejrede sig ved Harodkilden, medens Midjaniternes Lejr var nedenfor på Sletten, norden for Morehøjen.
\par 2 Da sagde HERREN til Gideon: "Du har for mange Folk hos dig, til at jeg kan give Midjaniterne i deres Hånd; gjorde jeg det, vilde Israel gøre sig stor over for mig og sige: Min egen Kraft skaffede mig Sejr!
\par 3 Lad derfor udråbe for Folket: Enhver, som er ræd og angst, skal vende hjem!" Gideon sigtede dem da, og 22000 Mand af Folket vendte hjem, medens 10000 blev tilbage.
\par 4 Men HERREN sagde til Gideon: "Endnu er der for mange Folk; før dem ned til Vandet, der vil jeg sigte dem for dig. Den, jeg siger skal gå med dig, han skal gå med; men enhver, om hvem jeg siger til dig: denne skal ikke gå med dig, han skal ikke gå!"
\par 5 Da han havde ført Folket ned til Vandet, sagde HERREN til Gideon: "Alle dem, der laber Vandet med Tungen som Hunde, skal du stille for sig; ligeså alle dem, der lægger sig på Knæ for at drikke,!"
\par 6 Og Tallet på dem, der labede, var 300; derimod lagde Resten af Folket sig på Knæ for at drikke af Vandet, idet de førte det til Munden med Hånden.
\par 7 Da sagde HERREN til Gideon: "Med de 300 Mand, som labede, vil jeg frelse eder og give Midjaniterne i din Hånd; men Resten af Folket skal drage hver til sit!"
\par 8 Derpå tog han Folkets Krukker og deres Horn fra dem; og alle Israelitterne lod han drage hjem, hver til sit, men de 300 Mand beholdt han hos sig. Og Midjaniternes Lejr var nede på Sletten.
\par 9 Samme Nat sagde HERREN til ham: "Stå op og drag ned imod Lejren, thi jeg har givet den i din Hånd!
\par 10 Men er du bange for at drage derned, så begiv dig med din Tjener Pura ned til Lejren
\par 11 og hør, hvad de siger der; så vil du få Mod til at drage ned imod Lejren!" Han gik da med sin Tjener Pura ned til Lejrens Forposter.
\par 12 Midjaniterne, Amalekiterne og alle Østens Stammer havde lejret sig på Sletten mangfoldige som Græshopper, og deres Kameler var utallige, mangfoldige som Sandet ved Havets Bred.
\par 13 Just som Gideon kom, var en Mand i Færd med at fortælle en anden noget, han havde drømt, idet han sagde: "Jeg har haft en Drøm! Se, et Bygbrød kom rullende ned mod Midjaniternes Lejr, og da det kom til Teltet, stødte det til det og væltede det over Ende, så at Teltet faldt."
\par 14 Da sagde den anden: "Det kan ikke betyde andet end Israelitten Gideons, Joasjs Søns, Sværd; Gud har givet Midjan og hele Lejren i hans Hånd!"
\par 15 Da Gideon hørte Beretningen om Drømmen og dens Tydning, tilbad han og vendte derefter tilbage til Israels Lejr, hvor han sagde: "Stå op, thi HERREN har givet Midjaniternes Lejr i eders Hånd!"
\par 16 Derpå delte han de 300 Mand i tre Afdelinger og gav dem alle Horn og tomme Krukker med Fakler i;
\par 17 og han sagde til dem: "Giv Agt på mig og gør som jeg! Når jeg kommer til Udkanten af Lejren. skal I gøre som jeg;
\par 18 når jeg og alle de, der er hos mig, støder i Hornet, skal I også støde i Hornene rundt om hele Lejren og råbe: For HERREN og for Gideon!"
\par 19 Da nu Gideon og de 100 Mand. der var hos ham, kom hen til Udkanten af Lejren ved Begyndelsen af den midterste Nattevagt, lige som man havde stillet Vagtposterne ud, stødte de i Hornene og slog deres Krukker itu.
\par 20 Så blæste de tre Afdelinger i Hornene og slog Krukkerne itu; men de holdt Faklerne i deres venstre Hånd og i deres højre Hånd Hornene for at blæse i dem, og de råbte: "HERRENs Sværd og Gideons!"
\par 21 Og de blev stående, hvor de stod, rundt om Lejren, hver på sin Plads. Da vågnede hele Lejren. og de skreg op og flygtede.
\par 22 Og da de blæste i de 300 Horn, satte HERREN den enes Sværd mod den andens i hele Lejren, og de flygtede til Bet Sjitta hen imod Zerera, til Bredden ved Abel Mehola over for Tabbat.
\par 23 Derpå stævnedes Israelitterne sammen fra Naftali, Aser og hele Manasse, og de satte efter Midjaniterne;
\par 24 og Gideon sendte Bud ud i hele Efraims Bjergland og lod kundgøre: "Drag Midjaniterne i Møde og afskær dem Adgangen til Vandet lige til Bet-Bara ved Jordan!" Da stævnedes alle Efraimiterne sammen, og de afskar dem Adgangen til Vandet lige til Bet-Bara ved Jordan
\par 25 og tog Midjaniternes to Høvdinger Oreb og Ze'eb til Fange; Oreb dræbte de på Orebs Klippe, Ze'eb i Ze'ebs Perse; og de forfulgte Midjaniterne og bragte Orebs og Ze'ebs Hoveder til Gideon hinsides Jordan.

\chapter{8}

\par 1 Men Efraimiterne sagde til ham: "Hvorfor handler du således over for os, at du ikke kaldte os til, da du drog i Kamp mod Midjaniterne?" Og de gik stærkt i Rette med ham.
\par 2 Men han svarede dem: "Hvad er min Dåd i Sammenligning med eders? Er ikke Efraims Efterhøst bedre end Abiezers Vinhøst?
\par 3 I eders Hånd gav Gud Midjaniternes Høvdinger Oreb og Ze'eb, hvad har jeg formået i Sammenligning med eder?" Og da han talte således, lagde deres Vrede sig mod ham.
\par 4 Da Gideon nåede Jordan, gik han og de 300 Mand, der var med ham, over, udmattede og sultne.
\par 5 Han sagde da til Folkene i Sukkot: "Giv Folkene, der følger mig, nogle Brød, thi de er udmattede, og jeg er i Færd med at forfølge Midjaniterkongerne Zeba og Zalmunna!"
\par 6 Men Høvedsmændene i Sukkot svarede: "Har du allerede fået fat i Zeba og Zalmunna, siden vi skal give dine Folk Brød?"
\par 7 Da sagde Gideon: "Til Gengæld skal jeg tærske eders Kroppe med Ørkentorne og Tidsler, når HERREN bar givet Zeba og Zalmunna i min Hånd!"
\par 8 Så drog han derfra til Penuel og sagde det samme til dem; og da Folkene i Penuel gav ham samme Svar som Folkene i Sukkot,
\par 9 tiltalte han dem på lignende Måde og sagde: "Når jeg kommer uskadt tilbage, vil jeg nedbryde Borgen her!"
\par 10 Zeba og Zalmunna befandt sig imidlertid med deres Hær i Karkor, henved 15000 Mand; det var alle dem, der var tilbage af Østens Stammers, Hær; de faldne udgjorde 120000 våbenføre Mænd.
\par 11 Og Gideon drog op ad Teltboernes Vej østen for Noba og Jogbeha og slog Hæren, der ikke anede Uråd.
\par 12 Zeba og Zalmunna flygtede, men han satte efter dem og fangede de to Midjaniterkonger Zeba og Zalmunna og slog hele Hæren med Rædsel.
\par 13 Da Gideon, Joasjs Søn, vendte tilbage fra Kampen, fra Herespasset,
\par 14 fik han fat i en ung Mand fra Sukkot og spurgte ham ud, og den unge Mand skrev Navnene på Høvedsmændene og de Ældste i Sukkot op til ham, syv og halvfjerdsindstyve Mænd.
\par 15 Så drog han til Folkene i Sukkot og sagde: "Se, her er Zeba og Zalmunna, som I hånede mig med, da I sagde: Har du allerede fået fat i Zeba og Zalmunna, siden vi skal give dine udmattede Mænd Brød?"
\par 16 Derpå greb han Byens Ældste, tog Ørkentorne og Tidsler og tærskede Folkene i Sukkot med dem;
\par 17 og Borgen i Penuel brød han ned og dræbte Folkene i Byen.
\par 18 Men til Zeba og Zalmunna sagde han: "Hvorledes var de Mænd, I ihjelslog ved Tabor?" De svarede: "De lignede dig, de så begge ud som Kongesønner!"
\par 19 Da sagde han: "Mine Brødre, min Moders Sønner var det! Så sandt HERREN lever; havde I skånet deres Liv, havde jeg ikke slået eder ihjel!"
\par 20 Derpå sagde han til sin førstefødte Søn Jeter: "Stå op og dræb dem!" Men den unge Mand drog ikke sit Sværd; han havde ikke Mod dertil, fordi han endnu var ung.
\par 21 Da sagde Zeba og Zalmunna: "Stå selv op og giv os Dødsstødet, thi som Manden således hans Styrke!" Så stod Gideon op og dræbte Zeba og Zalmunna. Og han tog de Halvmåner, deres Kameler havde om Halsen.
\par 22 Derefter sagde Israelitterne til Gideon: "Vær vor Hersker, du selv og siden din Søn og din Sønnesøn, da du nu har frelst os af Midjaniternes Hånd!"
\par 23 Men Gideon svarede dem: "Hverken jeg eller min Søn vil herske over eder; HERREN skal herske over eder!"
\par 24 Derpå sagde Gideon til dem: "Jeg har noget at bede eder om: Enhver af eder skal give mig de Ringe, der findes mellem hans Bytte!" Hine havde nemlig Guldringe, thi de var Ismaeliter.
\par 25 De svarede: "Ja, vi vil gerne give dig dem!" Så bredte han sin Kappe ud, og enhver af dem lagde de Ringe, der fandtes mellem hans Bytte, derpå.
\par 26 Vægten af Guldringene, som han havde bedt om, udgjorde 1700 Sekel Guld bortset fra Halvmånerne, Ørenringene og Purpurklæderne, som Midjaniterkongerne bar, og Kæderne på Kamelernes Halse.
\par 27 Gideon lavede deraf en Efod, som han gav Plads i sin Fødeby Ofra; og alle Israelitterne bolede med den, og den blev Gideon og hans Hus en Snare.
\par 28 Således bukkede Midjaniterne under for Israelitterne, og de rejste sig ikke mere; og Landet havde Ro i fyrretyve År, så længe Gideon levede.
\par 29 Og Jerubba'al, Joasjs Søn, begav sig til sit Hjem og opholdt sig der.
\par 30 Gideon havde halvfjerdsindstyve Sønner, der var avlede af ham, thi han havde mange Hustruer.
\par 31 Han havde en Medhustru i Sikem; hun fødte ham en Søn, som han gav Navnet Abimelek.
\par 32 Gideon, Joasjs Søn, døde i en høj Alder og blev jordet i sin Fader Abiezriten Joasjs Grav i Ofra.
\par 33 Men da Gideon var død, gav Israelitterne sig atter til at bole med Ba'alerne og gjorde Ba'al-Berit til deres Gud;
\par 34 og Israelitterne kom ikke HERREN deres Gud i Hu, ham, som frelste dem fra alle deres Fjender, der omgav dem på alle Kanter.
\par 35 Og de handlede ikke vel mod Jerubba'als, Gideons, Hus, trods alt det gode, han havde gjort Israel.

\chapter{9}

\par 1 Jerubba'als Søn Abrimelek begav sig til sine Morbrødere i Sikem og talte til dem og til hele sin Moders Fædrenehus's Slægt og sagde:
\par 2 "Sig til alle Sikems Borgere: Hvad båder eder vel bedst, at halvfjerdsindstyve Mænd, alle Jerubba'als Sønner, eller at en enkelt Mand hersker over eder? Kom i Hu, at jeg er eders Bød og Blod!"
\par 3 Hans Morbrødre talte da alle disse Ord til alle Sikems Borgere til Gunst til ham; og deres Hu vendte sig til Abimelek, idet de sagde: "Han er vor Broder."
\par 4 De gav ham derpå halvfjerdsindstyve Sekel Sølv fra Ba'al Berits Hus, og for dem lejede Abimelek nogle dårlige og frække Folk, som sluttede sig til ham.
\par 5 Derpå drog han til sin Faders Hus i Ofra og slog sine Brødre, Jerubba'als halvfjerdsindstyve Sønner, ihjel på een Sten. Kun Jotam, Jerubba'als yngste Søn, blev tilbage, thi han havde skjult sig.
\par 6 Derefter samledes alle Sikems Borgere og hele Millos Hus og gik hen og gjorde Abimelek til Konge ved Egen med Stenstøtten i Sikem.
\par 7 Da Jotam fik Efterretning herom, gik han hen og stillede sig på Toppen af Garizims Bjerg og råbte med høj Røst til dem: "Hør mig, Sikems Borgere, så skal Gud høre eder!
\par 8 Engang tog Træerne sig for at salve sig en Konge. De sagde da til Olietræet: Vær du vor Konge!
\par 9 Men Olietræet svarede dem: Skulde jeg give Afkald på min Fedme, for hvilken Guder og Mennesker priser mig, for at give mig til at svæve over Træerne?
\par 10 Så sagde Træerne til Figentræet: Kom du og vær vor Konge!
\par 11 Men Figentræet svarede dem: Skulde jeg give Afkald på min Sødme og min liflige Frugt for at give mig til at svæve over Træerne?
\par 12 Så sagde Træerne til Vinstokken: Kom du og vær vor Konge!
\par 13 Men Vinstokken svarede dem: Skulde jeg give Afkald på min Most, som glæder Guder og Mennesker, for at give mig til at svæve over Træerne?
\par 14 Da sagde alle Træerne til Tornebusken: Kom du og vær vor Konge!
\par 15 Og Tornebusken svarede Træerne: Hvis I mener det ærligt med at salve mig til eders Konge, kom så og søg ind under min Skygge; men hvis ikke, så vil Flammer slå op af Tornebusken og fortære Libanons Cedre!
\par 16 Hvis I nu er gået ærligt og redeligt til Værks, da I gjorde Abimelek til Konge, og hvis I har handlet vel mod Jerubba'al og hans Hus og gengældt ham, hvad han gjorde -
\par 17 min Fader kæmpede jo for eder og vovede sit Liv for at frelse eder af Midjaniternes Hånd,
\par 18 men I har i bag rejst eder mod min Faders Hus, dræbt hans Sønner, halvfjerdsindstyve Mænd, på een Sten og sat hans Trælkvindes Søn Abimelek til Konge over Sikems Borgere, fordi han er eders Broder
\par 19 ja, hvis I i Dag er gået ærligt og redeligt til Værks mod Jerubba'al og hans Hus, så gid I må få Glæde af Abimelek, og gid han må få Glæde af eder;
\par 20 men hvis ikke, så slå Flammer op fra Abimelek og fortære Sikems Borgere og Millos Hus, og Flammer slå op fra Sikems Borgere og Millos Hus og fortære Abimelek!"
\par 21 Derpå tog Jotam Flugten og flygtede til Be'er; og der tog han Ophold for at være i Sikkerhed for sin Broder Abimelek.
\par 22 Da Abimelek havde haft Magten over Israel i tre År,
\par 23 sendte Gud en ond Ånd mellem Abimelek og Sikems Borgere. og Sikems Borgere faldt fra Abimelek,
\par 24 for at Voldsgerningen mod Jerubba'als halvfjerdsindstyve Sønner kunde blive hævnet og deres Blod komme over deres Broder Abimelek, som havde dræbt dem, og over Sikems Borgere, som havde sat ham i Stand til at dræbe sine Brødre.
\par 25 Sikems Borgere lagde da Baghold på Bjergtoppene, og de udplyndrede alle vejfarende, der kom forbi dem. Dette meldtes Abimelek.
\par 26 Nu kom Ga'al, Ebeds Søn, med sine Brødre og flyttede ind i Sikem; og Sikems Borgere fattede Tillid til ham.
\par 27 De begav sig ud i Marken, plukkede Druer og pressede dem og fejrede deres Vinhøstfest. Og de gik ind i deres Guds Hus, hvor de spiste og drak og udstødte Forbandelser over Abimelek.
\par 28 Da sagde Ga'al, Ebeds Søn: "Hvem er Abimelek, og hvad er Sikem, at vi skal være hans Trælle! Var ikke Jerubba'als Søn og hans Foged Zebul Trælle for Hamors, Sikems Faders, Mænd hvorfor skal vi da være hans Trælle?
\par 29 Havde jeg blot Magten over Folket her, skulde jeg nok skaffe Abimelek af Vejen!"
\par 30 Da Zebul, Byens Høvedsmand, hørte Ga'als, Ebeds Søns, Urd, blussede hans Vrede op,
\par 31 og han sendte Bud til Abimelek i Aruma og lod sige: "Se, Ga'al, Ebeds Søn, og hans Brødre er kommet til Sikem, og se, de ophidser Byen imod dig; forstærk derfor din Hær og ryk ud!
\par 32 Og nu, bryd op ved Nattetide med dine Folk og læg dig i Baghold på Marken;
\par 33 og kast dig så over Byen tidligt om Morgenen, når Solen står op! Når da han og hans Folk rykker ud imod dig, kan du gøre med dem, hvad der falder for!"
\par 34 Abimelek brød da op ved Nattetide med alle sine Folk, og de lagde sig i Baghold imod Sikem i fire Afdelinger.
\par 35 Da gik Ga'al, Ebeds Søn, ned og stillede sig op ved Byporten.
\par 36 Da Ga'al fik Øje på Folkene, sagde han til Zebul: "Se, der stiger Folk ned fra Bjergtoppene!" Men Zebul sagde til ham: "Det er Bjergenes.Skygge, du tager for Mænd!"
\par 37 Men Ga'al sagde atter: "Der stiger Folk ned fra Landets Navle, og en anden Skare kommer i Retning af Sandsigernes Træ!"
\par 38 Da sagde Zebul til ham: "Hvor er nu dine store Ord fra før: Hvem er Abimelek, at vi skal være hans Trælle? Der er de Folk, du lod hånt om - ryk nu ud og kæmp med dem!"
\par 39 Ga'al rykkede da ud i Spidsen for Sikems Borgere for at kæmpe mod Abimelek.
\par 40 Men Abimelek slog ham på Flugt, og mange faldt og lå dræbte helt hen til Byporten.
\par 41 Abimelek tog så Ophold i Aruma; og Zebul jog Ga'al og hans Brødre bort fra Sikem.
\par 42 Næste Dag begav Folket sig ud på Marken, og det meldtes Abimelek.
\par 43 Han tog da sine Folk og delte dem i tre dele og lagde Baghold på Marken; og da han så Folket drage ud, overfaldt han dem og slog dem.
\par 44 Og Abimelek og den Afdeling, han havde hos sig, brød frem og tog Stilling ved Indgangen til Byen, medens de to andre Afdelinger kastede sig over alle dem, der var ude på Marken; og huggede dem ned;
\par 45 og efter at Abimelek hele Dagen igennem havde angrebet Byen, indtog han den, dræbte Folkene deri, nedbrød Byen og strøede Salt på den.
\par 46 Da hele Besætningen i Sikems Tårn hørte det, begav de sig ben til Kælderrummet i El Berits Hus.
\par 47 Og da Abimelek fik Melding om, at hele Besætningen i Sikems Tårn var samlet,
\par 48 gik han med alle sine Folk op på Zalmonbjerget. Her greb han en Økse, afhuggede et Knippe Grene, løftede det op og tog det på Skulderen; og han sagde til sine Folk: "Skynd eder at gøre det samme, som I så, jeg gjorde!"
\par 49 Alle Folkene afhuggede da også hver sit Knippe og fulgte efter Abimelek og lagde det oven på Kælderrummet og stak Ild på Kælderrummet oven over dem. Således omkom også hele Besætningen i Sikems Tårn, henved 1000 Mænd og Kvinder.
\par 50 Derefter drog Abimelek mod Tebez, og han belejrede Byen og indtog den.
\par 51 Inde i Byen var der et stærkt befæstet Tårn; derhen flygtede alle Mænd og Kvinder, alle Byens Indbyggere, idet de stængede efter sig og tyede op på Tårnets Tag;
\par 52 Abimelek rykkede da frem til Tårnet og angreb det; men da han nærmede sig Tårnets Indgang for at stikke Ild derpå,
\par 53 kastede en Kvinde en Møllesten ned på Abimeleks Hoved og knuste hans Hjerneskal.
\par 54 Da råbte han i Hast til sin Våbendrager: "Drag dit Sværd og dræb mig, for at det ikke skal siges, at en Kvinde har slået mig ihjel!" Og Våbendrageren gennemborede ham, så han døde.
\par 55 Men da Israelitterne så, at Abimelek var død, begav de sig hver til sit.
\par 56 Således gengældte Gud det onde, Abimelek havde øvet mod sin Fader ved at dræbe sine halvfjerdsindstyve Brødre;
\par 57 og al Sikemiternes Ondskab lod Gud komme over deres egne Hoveder. På den Måde kom Jotams, Jerubba'als Søns, Forbandelse over dem.

\chapter{10}

\par 1 Efter Abimelek fremstod som Israels Befrier Tola, en Søn af Dodos Søn Pua, en Mand af Issakar, som boede i Sjamir i Efraims Bjerge.
\par 2 Han var Dommer i Israel i tre og tyve År. Da han døde, blev han jordet i Sjamir.
\par 3 Efter ham fremstod Gileaditen Jair; han var Dommer i Israel i to og tyve År.
\par 4 Han havde tredive Sønner, som red på tredive Æsler, og de havde tredive Byer, som endnu den Dag i Dag kaldes Jairs Teltbyer; de ligger i Gilead.
\par 5 Da Jair døde, blev han jordet i Kamon.
\par 6 Men Israelitterne blev ved at gøre, hvad der var ondt i HERRENs Øjne, idet de dyrkede Ba'alerne og Astarterne og Aramæernes, Zidons, Moabs, Ammoniternes og Filisternes Guder og faldt fra HERREN og undlod at dyrke ham.
\par 7 Da blussede HERRENs Vrede op mod Israel, og han gav dem til Pris for Filisterne og Ammoniterne,
\par 8 som kuede og mishandlede Israelitterne i det År; i atten År kuede de alle Israelitterne hinsides Jordan i Amoritemes Land i Gilead.
\par 9 Og Ammoniterne satte over Jordan for også at angribe Juda, Benjamin og Efraims Slægt, så at Israelitterne kom i stor Nød.
\par 10 Da råbte Israelitterne til HERREN og sagde: "Vi har syndet imod dig, thi vi har forladt HERREN vor Gud og dyrket Ba'alerne!"
\par 11 Men HERREN svarede Israelitterne: "Har ikke Ægypterne, Amoriterne, Ammoniterne, Filisterne,
\par 12 Zidonierne, Amalekiterne og Midjaniterne mishandlet eder? Og da I råbte til mig, frelste jeg eder af deres Hånd.
\par 13 Men I forlod mig og dyrkede andre Guder! Derfor vil jeg ikke mere frelse eder!
\par 14 Gå nu hen og råb til de Guder, I udvalgte eder, og lad dem frelse eder i eders Nød:"
\par 15 Da sagde Israelitterne til HERREN: "Vi har syndet! Gør med os, hvad dig tykkes godt, men frels os blot nu!"
\par 16 Og de skilte sig af med de fremmede Guder og dyrkede HERREN; da kunde han ikke længer holde ud at se Israels Nød.
\par 17 Ammoniterne stævnedes sammen, og de slog Lejr i Gilead; også Israelitterne samlede sig, og de slog Lejr i Mizpa.
\par 18 Da sagde Folket, Gileads Høvdinger, til hverandre: "Hvis der findes en Mand, som vil tage Kampen op med Ammoniterne, skal han være Høvding over alle Gileads Indbyggere!"

\chapter{11}

\par 1 Gileaditen Jefta var en dygtig Kriger. Han var Søn af en Skøge.
\par 2 Men Gileads Hustru fødte ham Sønner, og da de voksede op, jog de Jefta bort med de Ord: "Du skal ikke have Arv og Lod i vor Faders Hus, thi du er en fremmed Kvindes Søn!"
\par 3 Jefta flygtede da for sine Brødre og bosatte sig i Landet Tob, hvor nogle dårlige Folk samlede sig om ham og deltog i hans Strejftog.
\par 4 Efter nogen Tids Forløb angreb Ammoniterne Israel.
\par 5 Og da Ammoniterne angreb Israel, drog Gileads Ældste hen for at hente Jefta hjem fra Landet Tob.
\par 6 De sagde til Jefta: "Kom og vær vor Fører, at vi kan tage Kampen op med Ammoniterne!"
\par 7 Jefta svarede Gileads Ældste: "Har I ikke hadet mig og jaget mig bort fra min Faders Hus? Hvorfor kommer I da til mig, nu I er i Nød?"
\par 8 Men Gileads Ældste sagde til Jefta: "Derfor kommer vi jo nu tilbage til dig! Vil du drage med os og kæmpe med Ammoniterne, skal du være Høvding over os, over alle Gileads Indbyggere!"
\par 9 Jefta svarede Gileads Ældste: "Dersom I fører mig tilbage, for at jeg skal kæmpe med Ammoniterne, og HERREN giver dem i min Magt, så vil jeg være eders Høvding!"
\par 10 Da sagde Gileads Ældste til Jefta: "HERREN hører Overenskomsten mellem os; visselig vil vi gøre, som du siger!"
\par 11 Da drog Jefta med Gileads Ældste; og Folket gjorde ham til deres Høvding og Fører. Og alle sine Ord udtalte Jefta for HERRENs Åsyn i Mizpa.
\par 12 Derpå sendte Jefta Sendebud til Ammoniternes Konge og lod sige: "Hvad er der dig og mig imellem, siden du er draget imod mig for at angribe mit Land?"
\par 13 Ammoniternes Konge svarede Jeftas Sendebud: "Jo, Israel tog mit Land, da de drog op fra Ægypten, lige fra Arnon til Jabbok og Jordan; giv det derfor tilbage med det gode!"
\par 14 Men Jefta sendte atter Sendebud til Ammoniternes Konge
\par 15 og lod sige: "Således siger Jefta: Israel har ikke taget Moabs eller Ammoniternes Land!
\par 16 Men da de drog op fra Ægypten, vandrede Israel igennem Ørkenen til det røde Hav og kom derpå til Kadesj.
\par 17 Da sendte Israel Sendebud til Edomiternes Konge og lod sige: Lad mig drage igennem dit Land! Men Edomiternes Konge ænsede det ikke. Ligeledes sendte de Bud til Moabiternes Konge, men han var heller ikke villig dertil. Israel blev da boende i Kadesj.
\par 18 Derpå drog de igennem Ørkenen og gik uden om Edomiternes og Moabiternes, Land, og da de nåede Egnen østen for Moab, slog de Lejr hinsides Arnon; men de betrådte ikke Moabs Enemærker, thi Arnon er Moabs Grænse.
\par 19 Israel sendte derpå Sendebud til Kongen af Hesjbon, Amoriterkongen Sihon, og lod sige: Lad os drage igennem dit Land for at nå hen, hvor vi skal!
\par 20 Men Sihon nægtede Israelitterne Tilladelse til at drage gennem hans Land; og Sihon samlede hele sin Hær, og de slog Lejr i Jaza og angreb Israel.
\par 21 Da gav HERREN, Israels Gud, Sihon og hele hans Hær i Israels Hånd, så at de slog dem. Og Israel underlagde sig hele det Land, Amoriterne boede i;
\par 22 de underlagde sig hele Amoriternes Område fra Arnon til Jabbok og fra Ørkenen til Jordan.
\par 23 Således drev HERREN, Israels Gud, Amoriterne bort foran sit Folk Israel; og nu vil du underlægge dig deres Land!
\par 24 Ikke sandt, når din Gud Kemosj driver nogen bort, så tager du hans Land? Og hver Gang HERREN vor Gud driver nogen bort foran os, tager vi hans Land.
\par 25 Er du vel bedre end Zippors Søn, Kong Balak af Moab? Stredes han med Israel, eller indlod han sig i Kamp med dem,
\par 26 da Israelitterne bosatte sig i Hesjbon med Småbyer, i Aroer med Småbyer og i alle Byerne langs Arnon nu har de boet der i 300 År? Hvorfor tilrev I eder dem ikke dengang?
\par 27 Det er ikke mig, der har forbrudt mig mod dig, men dig, der handler ilde mod mig ved at angribe mig. HERREN, Dommeren, vil i Dag dømme Israelitterne og Ammoniterne imellem!"
\par 28 Men Ammoniternes Konge ænsede ikke Jeftas Ord, som hans Sendebud overbragte.
\par 29 Da kom HERRENs Ånd over Jefta; og han drog igennem Gilead og Manasse; derpå drog han til Mizpe i Gilead, og fra Mizpe i Gilead drog han mod Ammoniterne.
\par 30 Og Jefta aflagde HERREN et Løfte og sagde: "Dersom du giver Ammoniterne i min Hånd,
\par 31 så skal den, som først kommer mig i Møde fra min Husdør når jeg vender uskadt, tilbage fra Ammoniterne, tilfalde HERREN, og jeg vil ofre ham som Brændoffer!"
\par 32 Så drog Jefta i Kamp mod Ammoniterne, og HERREN gav dem i hans Hånd,
\par 33 så at han tilføjede dem et stort Nederlag fra Aroer til Egnen ved Minnit, tyve Byer, og til Abel Keramim. Således bukkede Ammoniterne under for Israelitterne.
\par 34 Men da Jefta kom til sit Hjem i Mizpa, se, da kom hans Datter ham i Møde med Håndpauker og Dans. Hun var hans eneste Barn foruden hende havde han hverken Søn eller Datter.
\par 35 Da han fik Øje på hende, sønderrev han sine Klæder og råbte: "Ak, min Datter, du har bøjet mig dybt, og det er dig, der styrter mig i Ulykke! Thi jeg har åbnet min Mund for HERREN og kan ikke tage mit Ord tilbage!"
\par 36 Da svarede hun ham: "Fader, har du åbnet din Mund for HERREN, så gør med mig, som dit Ord lød, nu da HERREN har skaffet dig Hævn over dine Fjender, Ammoniterne!"
\par 37 Men hun sagde til sin Fader: "En Ting må du unde mig: Giv mig to Måneders Frist, så jeg kan gå omkring i Bjergene for at begræde min Jomfrustand sammen med mine Veninder!"
\par 38 Han sagde: "Gå!" og lod hende drage bort i to Måneder; og hun gik bort med sine Veninder for at begræde sin Jomfrustand i Bjergene.
\par 39 Da de to Måneder var omme, vendte hun tilbage til sin Fader, og han fuldbyrdede det Løfte, han havde aflagt, på hende; og hun havde ikke kendt Mand. Og det blev Skik i Israel,
\par 40 at Israels Døtre hvert År går hen for at klage over Gileaditen Jeftas Datter fire Dage om Året.

\chapter{12}

\par 1 Derpå stævnedes Efraimierne sammen, og de drog nordpå; og de sagde til Jefta: "Hvorfor drog du i Kamp mod Ammoniterne uden at opfordre os til at gå med? Nu brænder vi Huset af over Hovedet på dig!"
\par 2 Men Jeffa svarede dem: "Jeg og mit Folk var i Krig, og Ammoniterne trængte os hårdt; da sendte jeg Bud efter eder, men I hjalp mig ikke imod dem;
\par 3 og da jeg så, at ingen kom mig til Hjælp, vovede jeg Livet og drog mod Ammoniterne, og HERREN gav dem i min Hånd. Hvorfor drager I da nu i Kamp imod mig?"
\par 4 Derpå samlede Jefta alle Gileads Mænd og angreb Efraimiterne; og Gileads Mænd slog Efraimiterne.
\par 5 Og Gileaditerne afskar Efraimiterne fra Jordans Vadesteder.
\par 6 sagde de til ham: "Sig Sjibbolet!" Og når han da sagde Sibbolet, fordi han ikke kunde udtale Ordet rigtigt, greb de ham og huggede ham ned ved Jordans Vadesteder. Ved den Lejlighed faldt 42000 Efraimiter.
\par 7 Jefta var Dommer i Israel i seks År. Så døde Gileaditen Jefta og blev jordet i sin By i Gilead.
\par 8 Efter ham var Ibzan fra Betlehem Dommer i Israel.
\par 9 Han havde tredive Sønner; tredive Døtre giftede han bort, og tredive Svigerdøtre hjemførte han til sine Sønner.
\par 10 Han var Dommer i Israel i syv År. Så døde Ibzan og blev jordet i Betlehem.
\par 11 Efter ham var Zebuloniten Elon Dommer i Israel. Han var Dommer i Israel i ti År.
\par 12 Så døde Zebuloniten Elon og blev jordet i Ajjalon i Zebulons Land.
\par 13 Efter ham var Abdon, Hillels Søn, fra Piraton Dommer i Israel.
\par 14 Han havde fyrretyve Sønner og tredive Sønnesønner, som red på halvfjerdsindstyve Æsler. Han var Dommer i Israel i otte År.
\par 15 Så døde Abdon, Hillels Søn, fra Piraton og blev jordet i Piraton i Efraims Land på Amalekiterbjerget.

\chapter{13}

\par 1 Men Israelitterne blev ved at gøre, hvad der var ondt i HERRENS Øjne, og HERREN gav dem i Filisternes Hånd i fyrretyve År.
\par 2 Der levede i Zora en Mand af Daniternes Slægt ved Navn Manoa; hans Hustru var ufrugtbar og havde ingen Børn født.
\par 3 Nu viste HERRENs Engel sig for Kvinden og sagde til hende: "Se, du er ufrugtbar og har ingen Børn født; men du skal blive frugtsommelig og føde en Søn.
\par 4 Vogt dig vel for at drikke Vin eller stærk Drik og for at spise noget som helst urent!
\par 5 Thi se, du skal blive frugtsommelig og føde en Søn. Der må ikke komme Ragekniv på hans Hoved, thi Drengen skal være en Guds Nasiræer fra Moders Liv af; og han skal gøre de første Skridt til at frelse Israel af Filisternes Hånd!"
\par 6 Kvinden gik nu hen og sagde til sin Mand: "Der kom en Guds Mand til mig, og han så ud som en Guds Engel; såre frygtindgydende; jeg spurgte ham ikke, hvor han var fra, og sit Navn gav han mig ikke til Kende.
\par 7 Han sagde til mig: Se, du skal blive frugtsommelig og føde en Søn; drik nu ikke Vin eller stærk Drik og spis intet som helst urent, thi Drengen skal være en Guds Nasiræer fra Moders Liv af til sin Dødedag!"
\par 8 Da bad Manoa til HERREN og sagde: "Ak, HERRE, lad den Guds Mand, du sendte, atter komme til os for at lære os, hvorledes vi skal bære os ad med den Dreng, der skal fødes!"
\par 9 Og Gud bønhørte Manoa; og Guds Engel kom atter til Kvinden, medens hun sad ude på Marken, men Manoa, hendes Mand, var ikke hos hende.
\par 10 Da skyndte Kvinden sig hen til sin Mand, fortalte ham det og sagde til ham: "Se, den Mand, som kom til mig forleden, har vist sig for mig!"
\par 11 Manoa stod da op og gik med sin Hustru, og da han kom hen til Manden, sagde han til ham: "Er du den Mand, som talte til Kvinden?" Og han sagde: "Ja!"
\par 12 Så sagde Manoa: "Når nu dit Ord går i Opfyldelse, hvorledes skal vi da forholde os og bære os ad med Drengen?"
\par 13 HERRENs Engel svarede Manoa: "Alt det, jeg talte om til Kvinden, skal hun vogte sig for;
\par 14 intet af, hvad der vokser på Vinstokken, må hun spise; Vin og stærk Drik må hun ikke drikke, og intet urent må hun spise; alt, hvad jeg bød hende, skal hun overholde!"
\par 15 Da sagde Manoa til HERRENs Engel: "Vi vilde gerne holde dig tilbage og tillave dig et Gedekid!"
\par 16 Men HERRENs Engel svarede Manoa: "Selv om du holder mig tilbage, spiser jeg ikke af din Mad; men vil du ofre et Brændoffer, så bring HERREN det!" Thi Manoa vidste ikke, at det var HERRENs Engel.
\par 17 Og Manoa sagde til HERRENs Engel: "Hvad er dit Navn? Når dit Ord går i Opfyldelse, vil vi ære dig!"
\par 18 Men HERRENs Engel svarede: "Hvorfor spørger du om mit Navn? Du skal vide, det er underfuldt."
\par 19 Da tog Manoa Gedekiddet og Afgrødeofferet og ofrede det på Klippen til HERREN, ham, som handler underfuldt, og Manoa og hans Hustru så til.
\par 20 Og da Flammen slog op imod Himmelen fra Alteret, steg HERRENs Engel op i Alterflammen, medens Manoa og hans Hustru så til; og de faldt til Jorden på deres Ansigt.
\par 21 Og HERRENs Engel viste sig ikke mere for Manoa og hans Hustru.
\par 22 Og Manoa sagde til sin Hustru: "Vi er dødsens, thi vi har set Gud!"
\par 23 Men hans Hustru sagde til ham: "Havde HERREN i Sinde at dræbe os, havde han ikke modtaget Brændoffer og Afgrødeoffer af vor Hånd; heller ikke havde han ladet os se alt det og nu kundgjort os sådanne Ting!"
\par 24 Og Kvinden fødte en Søn, som hun gav Navnet Samson. Drengen voksede op, og HERREN velsignede ham;
\par 25 og HERRENs Ånd begyndte at drive på ham i Dans Lejr mellem Zora og Esjtaol.

\chapter{14}

\par 1 Engang Samson kom ned til Timna, så han en af Filisternes Døtre der.
\par 2 Og da han kom tilbage derfra, fortalte han sin Fader og Moder det og sagde: "Jeg har set en Kvinde i Timna, en af Filisternes Døtre; nu må I hjælpe mig at få hende til Hustru!"
\par 3 Hans Fader og Moder svarede ham: "Findes der da ingen Kvinde blandt dine Landsmænds Døtre eller i hele dit Folk, siden du vil gå hen og tage dig en Hustru hos de uomskårne Filistere?" Men Samson svarede sin Fader: "Nej, hende må du hjælpe mig til, thi det er hende, jeg synes om!"
\par 4 Hans Fader og Moder forstod ikke, at det kom fra HERREN, som søgte en Anledning til Strid over for Filisterne. På den Tid havde Filisterne nemlig Magten over Israel.
\par 5 Samson tog nu med sin Fader og Moder ned til Timna. Da de nåede Vingårdene uden for Timna, se, da kom en ung Løve brølende imod ham.
\par 6 Så kom HERRENs Ånd over ham, og han sønderrev den med sine bare Næver, som var det et Gedekid; men sin Fader og Moder fortalte han ikke, hvad han havde gjort.
\par 7 Derpå drog han ned og bejlede til Kvinden; thi Samson syntes om hende.
\par 8 Da han efter nogen Tids Forløb vendte tilbage for at ægte hende, gik han hen for at se til Løvens Ådsel, og se, da var der en Bisværm og Honning i Løvens Krop.
\par 9 Han tog da Honningen i sine Hænder og spiste deraf, medens han gik videre; og da han kom til sin Fader og Moder, gav han dem noget deraf, og de spiste; men han sagde dem ikke, at han havde taget Honningen fra Løvens Krop.
\par 10 Så drog Samson ned til Kvinden; og de holdt Gilde, som de unge havde for Skik.
\par 11 Da de så ham, udvalgte de tredive Brudesvende til at ledsage ham.
\par 12 Og Samson sagde til dem: "Jeg vil give eder en Gåde at gætte; hvis I i Løbet af de syv Gildedage kan sige mig Løsningen, vil jeg give eder tredive Linnedkjortler og tredive Sæt klæder;
\par 13 men kan I ikke sige mig den, skal I give mig tredive Linnedkjortler og tredive Sæt klæder!" De svarede: "Sig din Gåde frem og lad os høre den!"
\par 14 Da sagde han til dem: "Fra Æderen kom Æde, fra den stærke Sødme!" Men da de tre Dage var omme, havde de ikke kunnet gætte Gåden,
\par 15 og den fjerde Dag sagde de til Samsons Hustru: "Lok din Mand til at sige os Løsningen, ellers brænder vi dig og din Faders Hus inde! Har I budt os herhen for at tage alting fra os?"
\par 16 Da hang Samsons Hustru over ham med Gråd og sagde: "Du hader mig jo og elsker mig ikke! Du har givet mine Landsmænd en Gåde at gætte, og mig har du ikke sagt Løsningen!" Han svarede hende: "Jeg har ikke sagt min Fader eller Moder den, og så skulde jeg sige dig den!"
\par 17 Men hun hang over ham med Gråd, de syv Dage Gildet varede; og den syvende Dag sagde han hende Løsningen, fordi hun plagede ham.
\par 18 og den syvende Dag, før han gik ind i Kammeret, sagde Byens Mænd til ham: "Hvad er sødere end Honning, og hvad er stærkere end en Løve?" Men han svarede dem: "Havde I ikke pløjet med min Kalv, havde I ikke gættet min Gåde!"
\par 19 Da kom HERRENs Ånd over ham, og han drog ned til Askalon, slog tredive Mænd ihjel der, trak deres Tøj af dem og gav Klæderne til dem, der havde sagt Løsningen på Gåden. Og hans Vrede blussede op, og han drog tilbage til sin Faders Hus.
\par 20 Men Samsons Hustru blev givet til den Brudesvend, som havde været hans Brudefører.

\chapter{15}

\par 1 Efter nogen Tids Forløb, i Hvedehøstens Tid, besøgte Samson sin Hustru og havde et Gedekid med, og han sagde: "Lad mig gå ind i Kammeret til min Hustru!" Men hendes Fader tillod ham det ikke,
\par 2 men sagde: "Jeg tænkte for vist, at du havde fået Uvilje mod hende, derfor gav jeg hende til ham, der var din Brudesvend; men hendes yngre Søster er smukkere end hun, lad hende blive din Hustru i Søsterens Sted!"
\par 3 Da sagde Samson til dem: "Denne Gang er jeg sagesløs over for Filisterne, når jeg gør dem Fortræd!"
\par 4 Så gik Samson hen og fangede 300 Ræve; derpå tog han Fakler, bandt Halerne sammen to og to og fastgjorde en Fakkel midt imellem;
\par 5 så tændte han Faklerne, slap Rævene løs i Filisternes Korn og stak Ild både på Negene og Kornet på Roden, også på Vingårde og Oliventræer.
\par 6 Da Filisterne spurgte, hvem der havde gjort det, blev der sagt: "Det bar Samson, Timnitens Svigersøn, fordi han tog hans Hustru og gav hende til hans Brudesvend." Da gik Filisterne op og brændte hende og hendes Faders Hus inde.
\par 7 Men Samson sagde til dem: "Når I bærer eder således ad, under jeg mig ikke Ro, før jeg får hævnet mig på eder!"
\par 8 Så slog han dem sønder og sammen med vældige Slag, og derpå steg han ned i Fjeldkløften ved Etam og tog Ophold der.
\par 9 Filisterne drog nu op og slog Lejr i Juda og spredte sig ved Lehi.
\par 10 Da sagde Judas Mænd: "Hvorfor er I draget op imod os?" Og de svarede: "Vi er draget herop for at binde Samson og handle mod ham,som han har handlet mod os!"
\par 11 Så steg 3000 Mand fra Juda ned til Fjeldkløften ved Etam og sagde til Samson: "Ved du ikke, at Filisterne har Magten over os? Hvad er det dog, du har voldt os?." Han svarede dem: "Som de har handlet mod mig, har jeg handlet mod dem!"
\par 12 Men de sagde til ham: "Vi er kommet ned for at binde dig og overgive dig til Filisterne!" Da sagde Samson til dem: "Sværg mig til, at I ikke vil slå mig ihjel!"
\par 13 De svarede ham: "Nej, vi vil Kun binde dig og overgive dig til dem; slå dig ihjel vil vi ikke!" Så bandt de ham med to nye Reb og førte ham op af Fjeldkløften.
\par 14 Men da han kom til Lehi, og Filisterne hilste hans Komme med Jubelråb, kom HERRENs Ånd over ham, og Rebene om hans Arme blev som svedne Sytråde, og hans Bånd faldt skørnet af hans Hænder.
\par 15 Og da han fik Øje på en frisk Æselkæbe, rakte han sin Hånd ud, greb den og huggede 1000 Mand ned med den.
\par 16 Da sagde Samson: "Med en Æselkæbe har jeg flået dem sønder og sammen, med en Æselkæbe har jeg fældet 1000 Mand!"
\par 17 Med disse Ord kastede han Kæbebenet fra sig, og derfor kaldte man Stedet Ramat Lehi.
\par 18 Og da han var meget tørstig, råbte han til HERREN og sagde: "Ved din Tjeners Hånd har du skaffet os denne vældige Sejr, skal jeg da nu dø af Tørst og falde i de uomskårnes Hånd?"
\par 19 Da åbnede Gud Lavningen i Lehi, og der vældede Vand frem deraf; og da han havde drukket, fik han sin Livskraft igen.
\par 20 Han var Dommer i Israel i Filistertiden i tyve År.

\chapter{16}

\par 1 Samson drog så til Gaza. Der så han en Skøge og gik ind til hende.
\par 2 Da det spurgtes blandt Folkene i Gaza, at Samson var kommet derhen, gik de hen og lagde sig på Lur efter ham ved Byporten; men de holdt sig rolige Natten over, idet de sagde: "Vi vil vente, til det bliver lyst; så slår vi ham ihjel!"
\par 3 Samson blev liggende den halve Nat, men ved Midnatstide stod han op, greb fat i Byportens to Fløje og begge Portstolper, rykkede dem op tillige med Portslåen, tog dem på Skuldrene og bar dem op på Toppen af Bjerget over for Hebron.
\par 4 Siden fik han Kærlighed til en Kvinde ved Navn Dalila i Sorekdalen.
\par 5 Da kom Filisternes Fyrster til hende og sagde til hende: "Se at lokke ud af ham, hvad det er, der giver ham hans vældige Kræfter, og hvorledes vi kan få Bugt med ham, så vi kan binde og kue ham, så vil vi hver give dig 1100 Sekel Sølv!"
\par 6 Dalila sagde da til Samson: "Sig mig dog, hvad det er, der giver dig dine vældige Kræfter, og hvorledes man kan binde og kue dig!"
\par 7 Samson svarede hende: "Hvis man binder mig med syv friske Strenge, som ikke er blevet tørre, bliver jeg svag som ethvert andet Menneske."
\par 8 Filisternes Fyrster bragte hende da syv friske Strenge, der ikke var blevet tørre, og med dem bandt hun ham;
\par 9 samtidig havde hun Folk liggende på Lur i Kammeret. Da sagde hun til ham: "Filisterne er over dig, Samson!" Men han rev Strengene over, og de brast som Blårgarn, der kommer Ild for nær; og hvorledes det hang sammen med hans Kræfter, kom ikke for Dagen.
\par 10 Da sagde Dalila til Samson: "Se, du har narret mig og løjet for mig; sig mig dog, hvorledes man kan binde dig!"
\par 11 Han svarede hende: "Hvis man binder mig med nye Reb, som aldrig har været brug til noget, bliver jeg svag som ethvert andet Menneske!"
\par 12 Da tog Dalila nye Reb og bandt ham. Så sagde hun til ham: "Filisterne er over dig, Samson!" Samtidig lå der Folk på Lur i Kammeret. Men han flåede Rebene af sine Arme, som var det Tråde.
\par 13 Da sagde Dalila til Samson: "Hidtil har du narret mig og løjet for mig; sig mig dog, hvorledes man kan binde dig!" Han svarede hende: "Hvis du væver mine syv Hovedlokker ind i Rendegarnet og slår dem fast med Slagelen, bliver jeg svag som ethvert andet Menneske."
\par 14 Så dyssede hun ham i Søvn og vævede hans syv Hovedlokker ind i Rendegarnet og slog dem fast med Slagelen. Og hun sagde til ham: "Filisterne er over dig, Samson!" Så vågnede han og rykkede Væven op sammen med Rendegarnet.
\par 15 Da sagde hun til ham: "Hvor kan du sige, du elsker mig, når du ingen Fortrolighed har til mig? Tre Gange har du nu narret mig og ikke sagt mig, hvad det er, der giver dig dine vældige Kræfter!"
\par 16 Da hun således stadig pinte og plagede ham med sine Ord, blev hans Sjæl træt til Døden,
\par 17 og han talte rent ud og sagde til hende: "Ingen Ragekniv er kommet på mit Hoved, thi jeg har fra Moders Liv af været en Guds Nasiræer; hvis mit Hår rages af, mister jeg mine Kræfter og bliver svag som alle andre Mennesker."
\par 18 Da nu Dalila skønnede, at han havde talt rent ud til hende, sendte hun Bud efter Filisternes Fyrster og lod sige: "Kom nu herop, thi han har talt rent ud til mig!" Og Filisternes Fyrster kom op til hende og bragte Pengene med.
\par 19 Så dyssede hun ham i Søvn imellem sine Knæ og kaldte på en Mand, som ragede hans syv Hovedlokker af. Da blev han svagere og svagere, og Kræfterne veg fra ham.
\par 20 Derpå sagde hun: Filisterne er over dig, Samson!" Da vågnede han og tænkte: "Jeg skal nok slippe fra det ligesom de andre Gange og ryste det af mig!" Men han vidste ikke, at HERREN var veget fra ham.
\par 21 Da greb Filisterne ham og stak Øjnene ud på ham; derpå bragte de ham ned til Gaza og lagde ham i Kobberlænker; og han måtte dreje Kværnen i Fangehuset.
\par 22 Men hans Hovedhår begyndte at vokse igen, efter at det var raget af.
\par 23 Filisternes Fyrster samledes nu for at holde en stor Offerfest til Ære for deres Gud Dagon og for at fornøje sig; og de sagde: "Vor Gud gav os i Hænde Samson, vor Fjende!"
\par 24 Og da Folket så ham, priste de deres Gud og råbte: "Vor Gud gav os i Hænde Samson, vor Fjende, ham, som vort Land monne skænde, på manges Liv gjorde Ende!"
\par 25 Da de nu var kommet i godt Lune, sagde de: "Hent Samson, at vi kan more os over ham!" De lod da Samson hente fra Fangehuset, og de morede sig over ham. De stillede ham op ved Søjlerne;
\par 26 Da sagde Samson til den unge Mand, som holdt ham i Hånden: "Slip mig og lad mig røre ved Søjlerne, som bærer Hallen, så jeg kan læne mig til dem!".
\par 27 Hallen var fuld af Mænd og Kvinder; der var alle Filisternes Fyrster, og på Taget var der henved 3000 Mænd og Kvinder, som så til, medens de morede sig over Samson.
\par 28 Da råbte Samson til HERREN og sagde: "Herre, HERRE, kom mig i Hu og giv mig Kraft, o Gud, kun denne ene Gang, så jeg kan hævne mig på Filisterne for begge mine Øjne på een Gang!"
\par 29 Så greb Samson om de to Midtersøjler, som bar Hallen, og stemmede sig imod den ene med højre og imod den anden med venstre Hånd.
\par 30 Og Samson sagde: "Lad mig dø sammen med Filisterne!" Derpå bøjede han sig med sådan Kraft, at Hallen styrtede sammen over Fyrsterne og alle Folkene derinde. Således dræbte han ved sin Død flere, end han havde dræbt i levende Live.
\par 31 Men hans Brødre og hele hans Faders Hus drog ned og tog ham, bragte ham op og lagde ham i hans Fader Manoas Grav mellem Zora og Esjtaol. Han var Dommer i Israel i tyve År.

\chapter{17}

\par 1 I Efraims Bjerge levede en Mand, som hed Mika.
\par 2 Han sagde til sin Moder: "De 1100 Sekel Sølv, du har mistet, og for hvis Skyld du udtalte en Forbandelse, som jeg selv hørte, se, de Penge er hos mig; jeg har taget dem, men nu vil jeg give dig dem tilbage." Da sagde hans Moder: "HERREN velsigne dig, min Søn!"
\par 3 Så gav han sin Moder de 1100 Sekel Sølv tilbage; og Moderen sagde: "Disse Penge helliger jeg HERREN og giver min Søn, for at han kan lave et udskåret og støbt Billede."
\par 4 Så gav han sin Moder Pengene tilbage; og Moderen tog 200 Sekel Sølv deraf og gav dem til Guldsmeden, som lavede et udskåret og støbt Billede deraf, og det fik sin Plads i Mikas Hus.
\par 5 Manden Mika havde et Gudshus, og han lavede sig en Efod og en Husgud og indsatte en af sine Sønner til sin Præst.
\par 6 I de Dage var der ingen Konge i Israel; enhver gjorde, hvad han fandt for godt.
\par 7 Nu var der i Betlehem i Juda en ung Mand af Judas Slægt; han var Levit og boede der som fremmed.
\par 8 Denne Mand forlod sin By Betlehem i Juda for at slå sig ned som fremmed, hvor det kunde træffe sig, og på sin Vandring kom han til Mikas Hus i Efraims Bjerge.
\par 9 Da spurgte Mika ham: "Hvorfra kommer du?" Han svarede: "Jeg er Levit og har hjemme i Betlehem i Juda, og jeg er på Vandring for at slå mig ned som fremmed, hvor det kan træffe sig."
\par 10 Da sagde Mika til ham: "Tag Ophold hos mig og bliv min Fader og Præst; jeg vil give dig ti Sekel Sølv om Året og holde dig med Klæder og give dig Kosten!"
\par 11 Så gik Leviten ind på at tage Ophold hos Manden, og den unge Mand var ham som en af hans egne Sønner.
\par 12 Mika indsatte så Leviten, og den unge Mand blev Præst hos ham og tog Ophold i Mikas Hus.
\par 13 Da sagde Mika: "Nu ved jeg, at HERREN vil gøre vel imod mig, siden jeg har fået en Levit til Præst!"

\chapter{18}

\par 1 I de Dage var der ingen Konge i Israel, og i de Dage var Daniternes Stamme i Færd med at søge sig en Arvelod, hvor de kunde bo, thi hidindtil var der ikke tilfaldet dem nogen Arvelod blandt Israels Stammer.
\par 2 Daniterne udtog da af deres Slægt fem stærke Mænd fra Zora og Esjtaol og udsendte dem for at udspejde og undersøge Landet, og de sagde til dem: "Drag hen og undersøg Landet!" De kom da til Mikas Hus i Efraims Bjerge og overnattede der.
\par 3 Da de kom i Nærheden af Mikas Hus og kendte den unge Levits Stemme, tog de derind og spurgte ham: "Hvem har ført dig herhen, hvad tager du dig for på dette Sted, og hvorfor er du her?"
\par 4 Han svarede dem: "Det og det har Mika gjort for mig; han har lejet mig til Præst."
\par 5 Da sagde de til ham: "Adspørg da Gud, at vi kan få at vide, om vor Færd skal lykkes!"
\par 6 Præsten sagde da til dem: "Far med Fred, HERREN våger over eders Færd!"
\par 7 Så drog de fem Mænd videre og kom til Lajisj; og de så, at Byen levede trygt på Zidoniernes Vis, at Folket der levede sorgløst og trygt og ikke manglede nogen Verdens Ting, men var rigt, og at de boede langt fra Zidonierne og intet havde med Aramæerne at gøre.
\par 8 Da de kom tilbage til deres Brødre i Zora og Esjtaol, spurgte disse dem: "Hvad har I at melde?"
\par 9 De svarede: "Kom, lad os drage op til Lajisj, thi vi har set Landet, og se, det er såre godt! Hvorfor holder I eder uvirksomme? Nøl ikke med at drage hen og underlægge eder Landet!
\par 10 Thi Gud bar givet det i eders Hånd et Sted, hvor der ikke er Mangel på nogen Verdens Ting! Når l kommer derhen, kommer I til et Folk, der lever i Tryghed, og det er et vidtstrakt Land!"
\par 11 Så brød 600 væbnede Mænd af Daniternes Slægt op fra Zora og Esjtaol,
\par 12 og de drog op og slog Lejr i Kirjat Jearim i Juda; derfor kalder man endnu den Dag i Dag dette Sted hans Lejr; det ligger vesten for Kirjat Jearim.
\par 13 Derfra drog de over til Efraims Bjerge; og da de kom til Mikas Hus,
\par 14 tog de fem Mænd, der havde været henne at udspejde Landet, til Orde og sagde til deres Brødre: "Ved I, at der i Husene her findes en Efod, en Husgud og et udskåret og støbt Billede? Så indser I vel, hvad I har at gøre!"
\par 15 De begav sig derhen og kom til den unge Levits Hus, Mikas Hus, og hilste på ham,
\par 16 medens de 600 væbnede danitiske Mænd stod ved Porten.
\par 17 Og de fem Mænd, der havde været henne at udspejde Landet, gik op og tog det udskårne og støbte Billede, Efoden og Husguden, medens Præsten og de 600 væbnede Mænd stod ved Porten.
\par 18 Hine gik ind i Mikas Hus og tog det udskårne og støbte Billede, Efoden og Husguden. Præsten sagde til dem: "Hvad er det, I gør?"
\par 19 Og de svarede ham: "Stille, læg Fingeren på Munden og følg med os og bliv vor Fader og Præst! Hvad båder dig vel bedst, at være Præst for een Mands Hus eller for en Stamme og Slægt i Israel?"
\par 20 Da blev Præsten glad, tog Efoden, Husguden og Gudebilledet og sluttede sig til Krigsfolkene.
\par 21 Derpå vendte de om og drog bort, idet de stillede Kvinderne og Børnene, Kvæget og Trosset forrest i Toget.
\par 22 Da de var kommet et Stykke fra Mikas Hus, stævnedes Mændene i de Huse, der lå ved Mikas Hus, sammen, og de indhentede Daniterne.
\par 23 Da de råbte efter Daniterne, vendte disse sig om og sagde til Mika: "Hvad er der i Vejen, siden du har kaldt Folk til Hjælp?"
\par 24 Han svarede: "I har taget min Gud, som jeg havde lavet mig, tillige med Præsten og er rejst eders Vej! Hvad har jeg nu tilbage? Hvor kan I spørge mig, hvad der er i Vejen?"
\par 25 Men Daniterne svarede ham: "Lad os ikke høre et Ord mere fra dig, ellers kunde det hænde, at nogle Mænd, som er bitre i Hu, faldt over eder, og at du satte både dit eget og dine Husfolks Liv på Spil!"
\par 26 Dermed drog Daniterne deres Vej, og da Mika så, at de var ham for stærke, vendte han om og begav sig tilbage til sit Hus.
\par 27 De tog så Guden, som Mika havde lavet, tillige med hans Præst og drog mod Lajisj, mod et Folk, der levede sorgløst og trygt, huggede dem ned med Sværdet og stak Ild på Byen,
\par 28 uden at nogen kunde komme den til Hjælp, thi den lå langt fra Zidon, og de havde intet med Aramæerne at gøre. Den ligger i Bet Rehobs Dal. Så byggede de Byen op igen og bosatte sig der;
\par 29 og de gav den Navnet Dan efter deres Stamfader Dan, Israels Søn; men før var Byens Navn Lajisj.
\par 30 Derpå stillede Daniterne Gudebilledet op hos sig; og Jonatan, en Søn af Moses's Søn Gersom, og hans Efterkommere var Præster for Daniternes Stamme, indtil Landets Indbyggere førtes i Landflygtighed.
\par 31 Og det Gudebillede, Mika havde lavet sig, stillede de op hos sig, og det stod der, al den Tid Guds Hus var i Silo.

\chapter{19}

\par 1 I de Dage, da der ingen Konge var i Israel, var der en Mand, en Levit, der boede som fremmed i Udkanten af Efraims Bjerge. Han tog sig en Kvinde fra Betlehem i Juda til Medhustu.
\par 2 Men Medhustruen blev vred på ham, forlod ham og tog til sin Faders Hus i Betlehem i Juda. Da hun havde været der en fire Måneders Tid,
\par 3 drog hendes Mand af Sted efter hende for at overtale hende til at vende tilbage. Han havde sin Tjener og et Par Æsler med. Da han kom til hendes Faders Hus, fik den unge Kvindes Fader Øje på ham og kom ham glad i Møde;
\par 4 og hans Svigerfader, den unge Kvindes Fader, holdt på ham, så at han blev hos ham i tre Dage; og de spiste og drak og overnattede der.
\par 5 Tidligt om Morgenen den fjerde Dag gjorde han sig rede til at drage bort; men den unge Kvindes Fader sagde til sin Svigersøn: "Styrk dig først med en Bid Brød, så kan I siden drage bort!"
\par 6 De blev da og spiste og drak begge to sammen, og den unge Kvindes Fader sagde til Manden: "Bestem dig til at blive Natten over og gør dig til gode!"
\par 7 Men Manden gjorde sig rede til at drage bort. Da nødte hans Svigerfader ham, så han atter overnattede der.
\par 8 Tidligt om Morgenen den femte Dag vilde han drage bort, men den unge Kvindes Fader sagde til ham: "Styrk dig først!" De ventede da, til Dagen hældede, og spiste og drak begge to.
\par 9 Så gjorde Manden sig rede til at drage bort med sin Medhustru og sin Tjener. Men hans Svigerfader, den unge Kvindes Fader, sagde til ham: "Se, det lider mod Aften, overnat dog her; se, Dagen hælder, bliv dog her i Nat og gør dig til gode, så kan I i Morgen tidlig begive eder på Vej og du drage til dit Hjem!"
\par 10 Men Manden vilde ikke blive Natten over; han begav sig på Vej og kom ud for Jebus, det er Jerusalem, fulgt af sine to belæssede Æsler, sin Medhustru og sin Tjener.
\par 11 Da de var i Nærheden af Jebus og Dagen hældede stærkt, sagde Tjeneren til sin Herre: "Kom, lad os tage ind her i Jebusiternes By og blive der Natten over!"
\par 12 Men hans Herre svarede: "Vi vil ikke tage ind i en By, der ejes af fremmede, som ikke hører til Israelitterne, men vi vil drage videre til Gibea!"
\par 13 Og han sagde til sin Tjener: "Kom, lad os tage hen til et af de andre Steder og overnatte i Gibea eller Rama!"
\par 14 De drog så videre, og Solen gik ned, som de var ved Gibea i Benjamin.
\par 15 Så bøjede de af i den Retning for at nå til Gibea og overnatte der. Da han var kommet derind, Blev han på Byens Torv; men der var ingen, som bød dem ind i sit Hus for Natten.
\par 16 Så kom der om Aftenen en gammel Mand fra sit Arbejde på Marken, og Manden var fra Efraims Bjerge og boede som fremmed i Gibea, medens Stedets Indbyggere var Benjaminiter;
\par 17 og da den gamle Mand så op og fik Øje på den vejfarende Mand på Byens Torv, spurgte han: "Hvorhen gælder Rejsen, og hvorfra kommer du?"
\par 18 Han svarede ham: "Vi er på Rejse fra Betlehem i Juda til Udkanten af Efraims Bjerge, hvor jeg har hjemme; jeg har været i Betlehem i Juda og er nu på Vejen hjem; men der er ingen, som byder mig ind i sit Hus,
\par 19 skønt jeg har både Strå og Foder til vore Æsler og Brød og Vin til mig selv, din Trælkvinde og din Træls Tjener; vi mangler intet!"
\par 20 Da sagde den gamle Mand: "Vær velkommen! Lad kun alt, hvad du trænger til, være min Sag; men på Torvet må du ikke overnatte!"
\par 21 Så førte han ham ind i sit Hus og sørgede for Foder til Æslerne; og da de havde tvættet deres Fødder, spiste de og drak.
\par 22 Men medens de gjorde sig til gode, se, da omringede Mændene i Byen, Niddinger som de var, Huset og hamrede på Døren og råbte til den gamle Mand, Husets Ejer: "Før Manden, som er taget ind i dit Hus, herud, så at vi kan stille vor Lyst på ham!"
\par 23 Men Manden, der ejede Huset, gik ud til dem og sagde til dem: "Nej, mine Brødre, gør dog ikke noget ondt! Når denne Mand er taget ind i mit Hus, må I ikke øve sådan en Skændselsdåd!
\par 24 Se, her er min Datter, som er Jomfru; hende fører jeg herud, så kan I skænde hende og handle med hende, som I finder for godt! Men mod denne Mand må I ikke øve sådan en Skændselsdåd!"
\par 25 Men Mændene vilde ikke høre ham. Så greb Manden sin Medhustru og førte hende ud på Gaden til dem, og de stillede deres Lyst på hende og mishandlede hende Natten igennem til om Morgenen; først da Morgenen gryede, slap de hende.
\par 26 Ved Morgenens Frembrud kom Kvinden og faldt sammen ved Indgangen til den Mands Hus, hvor hendes Herre var, og lå der, til det blev lyst.
\par 27 Og da hendes Herre om Morgenen lukkede Husets Dør op og gjorde sig rede til at drage videre, se, da lå Kvinden, hans Medhustru, ved Indgangen til Huset med Hænderne på Tærskelen.
\par 28 Han sagde da til hende: "Rejs dig og lad os komme af Sted!" Men der kom intet Svar. Så løftede han hende op på Æselet og rejste til sit Hjem.
\par 29 Men da han kom hjem, greb han en Kniv, tog sin Medhustru, skar hende i tolv Stykker, Ledemod for Ledemod, og sendte Stykkerne rundt i hele Israels Land;
\par 30 og han gav de Mænd, han udsendte, den Befaling: "Således skal I sige til alle Israels Mænd: Er sligt hidtil hændet, siden Israelitterne drog op fra Ægypten? Overvej Sagen, hold Råd og sig eders Mening!" Og enhver, som så det, sagde: "Sligt er ikke hidtil hændet eller set, siden Israelitterne drog op fra Ægypten!"

\chapter{20}

\par 1 Da rykkende alle Israelitterne ud, og Menigheden samledes som een Mand fra Dan til Be'ersjeba for HERREN i Mizpa; også fra Gileads Land kom de.
\par 2 Alle Folkets Støtter, alle Israels Stammer indfandt sig i Guds Folks Forsamling, 400000 Mand Fodfolk, væbnet med Sværd.
\par 3 Benjaminiterne hørte, at Israelitterne var draget op til Mizpa.
\par 4 Da tog Manden, Leviten, den myrdede Kvindes Mand, til Orde og sagde: "Jeg og min Medhustru kom til Gibea i Benjamin for at overnatte der.
\par 5 Så rejste Gibeas Borgere sig imod mig og omringede mig om Natten i Huset; mig vilde de dræbe, og min Medhustru skændede de, så at hun døde.
\par 6 Da tog jeg min Medhustru, skar hende i Stykker og sendte Stykkerne rundt i hele Israels Arvelods Område, fordi de havde begået grov Utugt og Skændselsdåd i Israel!
\par 7 Nu er I her, alle Israelitter, sig nu eders Mening og kom med eders Råd!"
\par 8 Da rejste hele Folket sig som een Mand og sagde: "Ingen af os vil vende hjem, ingen af os vil begive sig til sit Hus!
\par 9 Men således vil vi handle med Gibea: Vi vil drage op imod det efter Lodkastning,
\par 10 og vi vil udtage ti Mænd af hundrede af alle Israels Stammer, hundrede af tusind og tusind af titusind til at hente Fødemidler til Folket, til dem, som er kommet for fuldt ud at gengælde Gibea i Benjamin den Skændselsdåd, de har øvet i Israel!"
\par 11 Derpå samlede alle Israels Mænd sig mod Byen, alle som een.
\par 12 Og Israels Stammer sendte Mænd ud i hele Benjamins Stamme og lod sige: "Hvad er det for en Misgerning, der er sket hos eder?
\par 13 Udlever nu Mændene i Gibea, de Niddinger, for at vi kan dræbe dem og skaffe Misgerningen bort fra Israel!" Men Benjaminiterne vilde ikke høre deres Brødre Israelitternes Ord.
\par 14 Og Benjaminiterne stævnede sammen fra deres Byer til Gibea for at drage i Kamp mod Israelitterne.
\par 15 Da Benjaminiterne fra Byerne mønstredes den bag, udgjorde de 25000 våbenføre Mænd, foruden dem af Gibeas Indbyggere, der mønstredes, 700 udsøgte Krigere;
\par 16 af alle disse Krigsfolk var 700 udvalgte Krigere kejthåndede; de kunde alle slynge med Sten, så de ramte på et Hår uden at fejle.
\par 17 Da Israels Mænd mønstredes, fraregnet Benjamin, udgjorde de 400000 våbenføre Mænd, der alle var Krigere.
\par 18 De brød så op og drog til Betel og rådspurgte Gud; og Israelitterne sagde: "Hvem af os skal først drage i Kamp mod Benjaminiterne?" HERREN svarede: "Det skal Juda!"
\par 19 Så brød Israelitterne op om Morgenen og slog Lejr uden for Gibea.
\par 20 Og Israels Mænd rykkede ud til Kamp imod Benjamin, og Israels Mænd stillede sig op til Kamp imod dem for at angribe Gibea.
\par 21 Men Benjaminiterne gjorde Udfald fra Gibea og fældede den Dag 22000 Mand af Israel.
\par 22 Folket, Israels Mænd, tog sig da sammen og stillede sig atter op til Kamp på samme Sted som den første Dag;
\par 23 Da drog Israelitterne op til Betel og græd lige til Aften for HERRENs Åsyn; og de adspurgte HERREN: "Skal jeg atter tage Kampen op med min Broder Benjamins Sønner?" Og HERREN svarede: "Drag op imod ham!"
\par 24 og Israelitterne rykkede Benjaminiterne på nært Hold den anden Dag.
\par 25 Men Benjaminiterne gjorde Udfald fra Gibea for at møde dem den anden Dag, og de fældede yderligere 18000 Mand af Israelitterne, alle sammen våbenføre Mænd.
\par 26 Så drog alle Israelitterne, hele Folket, op til Betel; og de græd og sad der for HERRENs Åsyn og fastede den Dag lige til Aften, og de ofrede Brændofre og Takofre for HERRENs Åsyn.
\par 27 Derpå rådspurgte Israelitterne HERREN i de Dage var Guds Pagts Ark der,
\par 28 og Pinehas, en Søn af Arons Søn Eleazar, gjorde i de Dage Tjeneste ved den - og de sagde: "Skal jeg atter drage i Kamp mod min Broder Benjamins Sønner eller lade være?" HERREN svarede: "Drag i Kamp, thi i Morgen giver jeg ham i din Hånd!"
\par 29 Israelitterne lagde nu Baghold rundt om Gibea.
\par 30 Og Israelitterne drog op mod Benjaminiterne på den tredje Dag og stillede sig op til Angreb på Gibea ligesom de tidligere Gange.
\par 31 Da nu Benjaminiterne gjorde Udfald mod Hæren, blev de afskåret fra Byen og lokket ud på Vejene til Betel og Gibeon; til at begynde med huggede de nogle al Folkene ned på åben Mark ligesom de tidligere Gange, omtrent tredive Mand af Israel,
\par 32 og Benjaminiterne tænkte nu: "Vi har slået dem ligesom før!" Men Israelitterne sagde: Lad os flygte og således afskære dem fra Byen og lokke dem ud på Vejene!"
\par 33 Så brød alle Israels Mænd op fra deres Plads og stillede sig op til Kamp i Ba'al-Tamar, medens Bagholdet brød op fra sin Plads vesten for Geba.
\par 34 Nu rykkede 10000 Mand, udvalgte Folk af hele Israel, frem for Gibea, og Kampen blev hård; men de vidste ikke, at Ulykken var ved at ramme dem.
\par 35 Så slog HERREN Benjamin foran Israel, og Israelitterne fældede den Dag 25100 Mand af Benjamin, alle våbenføre Mænd;
\par 36 da indså Benjaminiterne, at de var slagne. Israelitterne trak sig tilbage for Benjamin, idet de stolede på Bagholdet, de havde lagt mod Gibea;
\par 37 og Bagholdet kastede sig i en Fart over Gibea og drog frem og huggede hele Byens Befolkning ned med Sværdet.
\par 38 Der var truffet den Aftale mellem Israels Mænd og Bagholdet, at de skulde lade en Røgsøjle stige op fra Byen.
\par 39 Da Israels Mænd vendte om i Kampen, huggede Benjamin til at begynde med henved tredive af Israels Mænd ned, thi de tænkte: "Visselig, vi har slået dem ligesom i den forrige Kamp."
\par 40 Da nu Søjlen, Røgstøtten, begyndte at stige op fra Byen, vendte Benjamin sig om, og se, Røgen slog op mod Himmelen fra hele Byen,
\par 41 og samtidig vendte Israels Mænd om fra Flugten. Da blev Benjamins Mænd forfærdede, thi de indså, at Ulykken havde ramt dem;
\par 42 og de gjorde omkring for Israels Mænd og flygtede ad Ørkenen til.Men Kampen fortsattes i Hælene på dem. Og de fra Byerne huggede ned for Fode iblandt dem;
\par 43 de omringede Benjaminiterne og forfulgte dem, til de havde Geba foran sig mod Øst.
\par 44 Af Benjamin faldt 18000 Mand, lutter dygtige Krigere.
\par 45 De gjorde omkring og flygtede og deres Forfølgere gjorde på Vejene en Efterhøst på 5000 Mand: de forfulgte dem skarpt, til de fik dem tilintetgjort, og huggede 2000 Mand ned af dem.
\par 46 De, der faldt af Benjamin den Dag, var således i alt 25000 våbenføre Mænd, alle tapre Folk.
\par 47 De gjorde omkring og flygtede ud i Ørkenen til Rimmons Klippe, 600 Mand stærke, og blev der i fire Måneder.
\par 48 Men Israels Mænd vendte tilbage til Benjaminiterne og huggede dem ned med Sværdet, både Mennesker og Kvæg, overhovedet alt.

\chapter{21}

\par 1 Men Israels Mænd havde i Mizpa aflagt den Ed: "Ingen af os vil give en Benjaminit sin Datter til Ægte!"
\par 2 Da nu Folket var kommet til Betel, sad de der lige til Aften for Guds Åsyn og opløftede deres Røst, græd heftigt
\par 3 og sagde: "Hvorfor, HERRE, Israels Gud, er dog dette hændet i Israel, så at vi i Dag må savne en Stamme af Israel?"
\par 4 Tidligt næste Morgen byggede Folket et Alter der og ofrede Brændofre og Takofre.
\par 5 Derpå sagde Israelitterne: "Hvem blandt alle Israels Stammer undlod at drage op med Forsamlingen til HERREN?" Der var nemlig svoret en dyr Ed på, at enhver, der undlod at drage op til HERREN i Mizpa, skulde dø.
\par 6 Men nu gjorde det Israelitterne ondt for deres Broder Benjamin, og de sagde: "I Dag er en Stamme hugget af Israel!
\par 7 Hvad skal vi gøre for dem, der er tilbage, for at skaffe dem Hustruer, eftersom vi har svoret ved HERREN, at vi ikke vil give dem nogen af vore Døtre fil Ægte?"
\par 8 Så spurgte de: "Er der måske en af Israels Stammer, der undlod at drage op til HERREN i Mizpa?" Og se, der var ingen kommet til Lejren, til Forsamlingen, fra Jabesj i Gilead.
\par 9 Så blev Folket mønstret, og se, der var ingen af Indbyggerne fra Jabesj i Gilead.
\par 10 Da sendte Menigheden 12000 Mand af de tapreste Folk derhen med den Befaling: "Drag hen og hug Indbyggerne i Jabesj i Gilead ned med Sværdet tillige med deres Kvinder og Børn.
\par 11 Således skal I bære eder ad: Alle af Mandkøn og alle Kvinder, der har haft Omgang med Mænd, skal I lægge Band på!"
\par 12 De fandt så hos Indbyggerne i Jabesj i Gilead 400 unge Piger, der var Jomfruer og ikke havde haft Omgang med nogen Mand, og dem førte de til Lejren i Silo i Kana'ans Land.
\par 13 Derpå sendte hele Menigheden Sendebud hen for af underhandle med Benjaminiterne, der befandt sig på Rimmons Klippe, og tilbyde dem Fred.
\par 14 På det Tidspunkt vendte Benjaminiterne så tilbage, og de gav dem de Kvinder fra Jabesj i Gilead, som man havde ladet i Live.
\par 15 Da gjorde det Folket ondt for Benjamin, fordi HERREN havde gjort et Skår i Israels Stammer.
\par 16 Og Menighedens Ældste sagde: "Hvad skal vi gøre for dem, der er tilbage, for at skaffe dem Hustruer, eftersom alle Kvinder i Benjamin er udryddet?"
\par 17 Og de sagde: "Hvorledes kan der reddes en Rest af Benjamin, så at ikke en Stamme i Israel går til Grunde?
\par 18 Vi kan jo ikke give dem nogen af vore Døtre til Ægte!" Israelitterne havde nemlig svoret og sagt: "Forbandet være den, som giver Benjaminiterne en Hustru!"
\par 19 Da sagde de: "Se, HERRENs Højtid fejres jo hvert År i Silo!" Det ligger norden for Betel, østen for Vejen, der fører op fra Betel til Sikem, og sønden for Lebona.
\par 20 Og de bød Benjaminiterne: "Gå hen og læg eder på Lur i Vingårdene!
\par 21 Se så nøje til, og når de unge Kvinder fra Silo kommer ud for at opføre deres Danse, skal I komme frem af Vingårdene og røve hver sin Hustru af de unge Kvinder fra Silo og så drage hjem til Benjamins Land!
\par 22 Når så deres Fædre eller Brødre kommer for at gå i Rette med eder, skal I sige til dem: Skån os, thi vi fik os ikke alle en Hustru i Krigen! Det er jo ikke eder, der har givet os dem; i så Fald vilde I have forbrudt eder!"
\par 23 Det gjorde Benjaminiterne da, og de tog sig Hustruer af de dansende Kvinder, som de røvede, een til hver; derpå vendte de tilbage til deres Arvelod, opbyggede deres Byer og boede i dem.
\par 24 Og samtidig drog Israelitterne derfra, hver til sin Stamme og Slægt, og de gik derfra hver til sin Arvelod,
\par 25 I de Dage var der ingen Konge i Israel; enhver gjorde, hvad han fandt for godt.

\end{document}