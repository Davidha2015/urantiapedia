\begin{document}

\title{Første Samuelsbog}


\chapter{1}

\par 1 Der var en Mand fra Ramatajim, en Zufit fra Efraims Bjerge ved Navn Elkana, en Søn af Jerobam, en Søn af Elihu, en Søn af Tohu, en Søn af Zuf, en Efraimit.
\par 2 Han havde to Hustruer; den ene hed Hanna, den anden Peninna; Peninna havde Børn, men Hanna ikke.
\par 3 Denne Mand drog hvert År op fra sin By for at tilbede Hærskarers HERRE og ofre til ham i Silo, hvor Elis to Sønner Hofni og Pinehas Var Præster for HERREN.
\par 4 En Dag ofrede nu Elkana han plejede at give sin Hustru Peninna og alle hendes Sønner og Døtre flere Dele,
\par 5 men skønt han elskede Hanna, gav han hende kun een Del, fordi HERREN havde tillukket hendes Moderliv;
\par 6 hendes Medbejlerske tilføjede hende også grove Krænkelser for den Skam, at HERREN havde tillukket hendes Moderliv.
\par 7 Således gik det År efter År: hver Gang de drog op til HERRENs Hus, krænkede hun hende således så skete det, at hun græd og ikke vilde spise.
\par 8 Da sagde hendes Mand Elkana til hende: "Hanna, hvorfor græder du, og hvorfor spiser du ikke? Hvorfor er du mismodig? Er jeg dig ikke mere værd end ti Sønner?"
\par 9 Men da de havde holdt Måltid i Silo, stod Hanna op og trådte hen for HERRENs Åsyn, medens Præsten Eli sad på sin Stol ved en af Dørstolperne i HERRENs Hus;
\par 10 og i sin Vånde bad hun under heftig Gråd til HERREN
\par 11 og aflagde det Løfte: "Hærskarers HERRE! Hvis du vil se til din Tjenerindes Nød og komme mig i Hu og ikke glemme din Tjenerinde, men give din Tjenerinde en Søn, så vil jeg give ham til HERREN alle hans Levedage, og ingen Ragekniv skal komme på hans Hoved!"
\par 12 Således bad hun længe for HERRENs Åsyn, og Eli iagttog hendes Mund;
\par 13 men da Hanna talte ved sig selv, så kun hendes Læber bevægede sig, og hendes Stemme ikke kunde høres, troede Eli, at hun var beruset,
\par 14 og sagde til hende: "Hvor længe vil du gå og være drukken? Se at komme af med din Rus!"
\par 15 Men Hanna svarede: "Nej, Herre! Jeg er en hårdt prøvet Kvinde; Vin og stærk Drik har jeg ikke drukket; jeg udøste kun min Sjæl for HERRENs Åsyn.
\par 16 Regn ikke din Trælkvinde for en dårlig Kvinde! Nej, hele Tiden har jeg talt ud af min dybe Kummer og Kvide!"
\par 17 Eli svarede: "Gå bort i Fred! Israels Gud vil give dig, hvad du har bedt ham om!"
\par 18 Da sagde hun: "Måtte din Trælkvinde finde Nåde for dine Øjne!" Så gik Kvinden sin Vej , og hun spiste og så ikke længer forgræmmet ud.
\par 19 Næste Morgen stod de tidligt op og kastede sig ned for HERRENs Åsyn; og så vendte de tilbage og kom hjem til deres Hus i Rama.
\par 20 og hun blev frugtsommelig og fødte en Søn Året efter og gav ham Navnet Samuel; "thi," sagde hun, "jeg har bedt mig ham til hos HERREN!"
\par 21 Da Elkana nu med hele sit Hus drog op for at bringe HERREN det årlige Offer og sit Løfteoffer,
\par 22 drog Hanna ikke med; thi hun sagde til sin Mand: "Jeg vil vente, til Drengen er vænnet fra, så vil jeg bringe ham derhen, for at han kan stedes for HERRENs Åsyn og blive der for stedse!"
\par 23 Da sagde hendes Mand Elkana til hende : "Gør, som du synes! Bliv her, indtil du har vænnet ham fra! Måtte HERREN kun gøre dit Ord til Virkelighed!" Så blev Kvinden hjemme og ammede sin Søn, indtil hun vænnede ham fra.
\par 24 Men da hun havde vænnet ham fra, tog hun ham med, desuden en treårs Tyr, en Efa Mel og en Dunk Vin, og hun kom til HERRENs Hus i Silo og havde Drengen med.
\par 25 Da nu Tyren var slagtet, kom Drengens Moder til Eli
\par 26 og sagde: "Hør mig, Herre! Så sandt du lever, Herre, jeg er den Kvinde, som stod her ved din Side og bad til HERREN.
\par 27 Om denne Dreng bad jeg, og HERREN gav mig, hvad jeg bad ham om.
\par 28 Derfor vil jeg også overlade ham til HERREN; hele sit Liv skal han være overladt til HERREN!" Og hun lod ham blive der for HERRENs Åsyn.

\chapter{2}

\par 1 Da bad Hanna og sagde: Mit Hjerte jubler over HERREN, mit Horn er løftet ved min Gud, min Mund vidt opladt mod mine Fjender, jeg glæder mig over din Frelse.
\par 2 Der er ingen Hellig som HERREN, nej, der er ingen uden dig, der er ingen Klippe som vor Gud
\par 3 Vær varsomme med eders store Ord, Frækhed undslippe ej eders Mund! Thi en vidende Gud er HERREN, og Gerninger vejes af ham.
\par 4 Heltes Bue er brudt, men segnende omgjorder sig med Kraft;
\par 5 mætte lader sig leje for Brød, men sultnes Slid hører op; den ufrugtbare føder syv, men den med de mange vansmægter.
\par 6 HERREN døder, gør levende, fører ned i Dødsriget og fører op;
\par 7 HERREN gør fattig, gør rig, han nedbøjer
\par 8 han rejser ringe af Støvet, af Skarnet løfter han fattige for at bænke og give dem Ærespladsen. Thi HERRENs er Jordens Søjler, Jorderig bygged han på dem
\par 9 Han vogter sine frommes Skridt, men gudløse omkommer i Mørket; thi ingen vinder Sejr ved egen kraft.
\par 10 HERREN - hans Fjender forfærdes, den Højeste tordner i Himmelen, HERREN dømmer den vide Jord! Han skænker sin Konge Kraft, løfter sin Salvedes Horn!
\par 11 Så drog hun til Rama, men Drengen gjorde Tjeneste for HERREN under Præsten Elis Tilsyn.
\par 12 Men Elis Sønner var Niddinger; de ænsede hverken HERREN
\par 13 eller Præstens Ret over for Folket. Hver Gang en Mand bragte et Slagtoffer, kom Præstens Tjener, medens Kødet kogte, med en tregrenet Gaffel i Hånden
\par 14 og stak den ned i Karret, Krukken, Kedelen eller Gryden, og alt, hvad Gaffelen fik med op, tog Præsten for sin Del. Således bar de sig ad over for alle de Israelitter, som kom til Silo for at ofre der.
\par 15 Eller også kom Præstens Tjener, før de bragte Fedtet som Røgoffer, og sagde til Manden, som ofrede: "Giv Præsten Kød til at stege; han tager ikke mod kogt Kød af dig, kun råt!"
\par 16 Sagde Manden nu til ham: "Først må Fedtet bringes som Røgoffer, bagefter kan du tage så meget, du lyster!" svarede han: "Nej, giv mig det nu, ellers tager jeg det med Magt!"
\par 17 Og de unge Mænds Synd var såre stor for HERRENs Åsyn, idet de viste Ringeagt for HERRENs Offergaver.
\par 18 Imidlertid gjorde Samuel Tjeneste for HERRENs Åsyn; og Drengen var iført en linned Efod.
\par 19 Hans Moder lavede hvert År en lille Kappe til ham og bragte ham den, når hun drog op med sin Mand for at ofre det årlige Offer.
\par 20 Og Eli velsignede Elkana og hans Hustru og sagde: "HERREN give dig Afkom af denne Kvinde til Gengæld for ham, hun overlod HERREN!" Så gik de hjem igen.
\par 21 Og HERREN så til Hanna, og hun blev frugtsommelig og fødte tre Sønner og to Døtre. Men Drengen Samuel voksede op hos HERREN.
\par 22 Eli var meget gammel, og da han hørte, hvorledes hans Sønner behandlede hele Israel, og at de lå hos Kvinderne, som gjorde Tjeneste ved Indgangen til Åbenbaringsteltet,
\par 23 sagde han til dem: "Hvorfor gør I sådanne Ting, som jeg hører alt Folket tale om?
\par 24 Bær eder dog ikke således ad, mine Sønner! Thi det er ikke noget godt Rygte, jeg hører gå fra Mund til Mund i HERRENs Folk.
\par 25 Når en Mand synder mod en anden, dømmer Gud dem imellem; men synder en Mand mod HERREN, hvem kan da optræde som Dommer til Gunst for ham?" Men de brød sig ikke om deres Faders Advarsel, thi HERREN vilde deres Død.
\par 26 Men Drengen Samuel voksede til og gik stadig frem i Yndest både hos HERREN og Mennesker.
\par 27 Da kom en Guds Mand til Eli og sagde: Så siger HERREN: "Se, jeg åbenbarede mig for dit Fædrenehus, dengang de var Trælle for Faraos Hus i Ægypten,
\par 28 og jeg udvalgte det af alle Israels Stammer til at gøre Præstetjeneste for mig, til at træde op på mit Alter for at tænde Offerild og til at bære Efod for mit Åsyn; og jeg tildelte dit Fædrenehus alle Israeliternes Ildofre.
\par 29 Hvor kan du da se ondt til mit Slagtoffer og Afgrødeoffer, som jeg har påbudt, og ære dine Sønner fremfor mig, idet l gør eder tilgode med det bedste at alle mit Folk Israels Offergaver!
\par 30 Derfor lyder det fra HERREN, Israels Gud: Vel har jeg sagt, at dit Hus og dit Fædrenehus for stedse skulde færdes for mit Åsyn; men nu, lyder det fra HERREN, være det langt fra mig! Nej, dem, som ærer mig, vil jeg ære, og de, som ringeagter mig, skal beskæmmes.
\par 31 Se, den Tid skal komme, da jeg, afhugger din og dit Fædrenehus's Arm, så ingen i dit Hus skal blive gammel;
\par 32 og du skal se ondt til alt det gode, HERREN gør mod Israel, og ingen Sinde skal nogen i din Slægt blive gammel.
\par 33 Kun en eneste af din Slægt vil jeg undlade at bortrydde fra mit Alter for at lade hans Øjne hentæres og hans Sjæl vansmægte: men alle de andre i din Slægt skal dø for Menneskers Sværd.
\par 34 Og det Tegn, du får derpå, skal være det, der overgår dine to Sønner Hofni og Pinehas: På een Dag skal de begge dø.
\par 35 Men jeg vil udvælge mig en trofast Præst; han skal handle efter mit Hjerte og mit Sind, og ham vil jeg bygge et varigt Hus, så han altid skal færdes for min Salvedes Åsyn.
\par 36 Da skal enhver, som er tilbage af din Slægt, komme og kaste sig til Jorden for ham for at få en Skilling eller en Skive Brød, og han skal sige: Und mig dog Plads ved et af dine Præsteskaber, for at jeg kan have en Bid Brød at spise!"

\chapter{3}

\par 1 Den unge Samuel gjorde så Tjeneste for HERREN under Elis Tilsyn. HERRENs Ord var sparsomt i de Dage, et Syn var sjældent.
\par 2 Ved den Tid - engang da Eli, hvis Øjne var begyndt at blive svage, så han ikke kunde se, lå på sin vante Plads,
\par 3 og Guds Lampe endnu ikke var gået ud, og Samuel lå og sov i HERRENs Helligdom, hvor Guds Ark stod -
\par 4 kaldte HERREN: "Samuel, Samuel!" Han svarede: "Her er jeg!"
\par 5 Og han løb hen til Eli og sagde: "Her er jeg, du kaldte på mig!" Men han sagde: "Jeg kaldte ikke; læg dig kun hen igen!" Og han gik hen og lagde sig.
\par 6 Da kaldte HERREN atter: "Samuel, Samuel!" Og han gik hen til Eli og sagde: "Her er jeg, du kaldte på mig!" Men han sagde: "Jeg kaldte ikke, min Søn; læg dig kun hen igen!"
\par 7 Samuel havde nemlig endnu ikke lært HERREN at kende, og HERRENs Ord var endnu ikke åbenbaret ham.
\par 8 Da kaldte HERREN atter for tredje Gang på Samuel, og han stod op, gik hen til Eli og sagde: "Her er jeg, du kaldte på mig!" Så skønnede Eli, at det var HERREN, der kaldte på Drengen.
\par 9 Og Eli sagde til Samuel: "Læg dig hen igen, og hvis han kalder på dig, så sig: Tal, HERRE, din Tjener hører!" Så gik Samuel hen og lagde sig på sin Plads.
\par 10 Da kom HERREN og trådte hen til ham og kaldte ligesom de forrige Gange: "Samuel, Samuel!" Og Samuel svarede: "Tal, din Tjener hører!"
\par 11 Så sagde HERREN til Samuel: "Se, jeg vil lade noget ske i Israel, som skal få det til at ringe for begge Ører på enhver, som hører derom.
\par 12 På den Dag vil jeg lade alt, hvad jeg har talt om Elis Slægt, opfyldes på ham, alt fra først til sidst.
\par 13 Du skal kundgøre ham, at jeg har dømt hans Slægt for evigt, fordi han vidste, at hans Sønner ringeagtede Gud, og dog ikke talte dem alvorligt til.
\par 14 Derfor har jeg svoret over Elis Hus: Visselig, Elis Hus's Brøde skal aldrig i Evighed sones ved Slagtofre eller Afgrødeofre!"
\par 15 Samuel blev nu liggende til Morgen, og tidligt næste Morgen åbnede han Døren til HERRENs Hus; men Samuel turde ikke omtale Synet for Eli.
\par 16 Da kaldte Eli på Samuel og sagde: "Min Søn Samuel!" Han svarede: "Her er jeg!"
\par 17 Da sagde han: "Hvad var det,han sagde til dig? Dølg det ikke for mig! Gud ramme dig både med det ene og det andet, hvis du dølger noget af, hvad han sagde!"
\par 18 Så fortalte Samuel ham det hele uden at dølge noget. Da sagde han: "Han er HERREN; han gøre,hvad ham tykkes bedst!"
\par 19 Samuel voksede nu til og HERREN var med ham og lod ikke et eneste af sine Ord falde til Jorden.
\par 20 Og hele Israel fra Dan til Be'ersjeba forstod, at Samuel virkelig var kaldet til HERRENs Profet.
\par 21 Og HERREN vedblev at lade sig til Syne i Silo; thi HERREN åbenbarede sig for Samuel.

\chapter{4}

\par 1 Og Israel rykkede ud til Kamp imod Filisterne og lejrede sig ved Eben-Ezer, medens Filisterne lejrede sig ved Afek.
\par 2 Filisterne stillede sig op til Kamp mod Israel, og Kampen blev hed; Israel blev slået af Filisterne, og de dræbte i Slaget på åben Mark omtrent 4OOO Mand.
\par 3 Da Folket kom tilbage til Lejren, sagde Israels Ældste: "Hvorfor lod HERREN os i Dag bukke under for Filisterne? Lad os hente vor Guds Ark i Silo, for at han kan være i vor Midte og fri os af vore Fjenders Hånd!"
\par 4 Så sendte Folket Bud til Silo og hentede Hærskarers HERREs Pagts Ark, han, som troner over Keruberne; og Elis to Sønner Hofni og Pinehas fulgte med Guds Pagts Ark.
\par 5 Da nu HERRENs Pagts Ark kom til Lejren, brød hele Israel ud i et vældigt Jubelråb, så Jorden rystede derved.
\par 6 Og da Filisterne hørte Jubelråbet, sagde de: "Hvad er det for et vældigt Jubelråb i Hebræernes Lejr?" Og de fik at vide, at HERRENs Ark var kommet til Lejren.
\par 7 Da blev Filisterne bange, thi de tænkte: "Gud er kommet i Lejren!" Og de sagde: "Ve os! Sligt er ikke hændet før!
\par 8 Ve os! Hvem skal fri os af denne vældige Guds Hånd? Det er den Gud, som slog Ægypterne med alle Hånde Plager og Pest.
\par 9 Tag eder nu sammen og vær Mænd, Filistere, for at I ikke skal komme til at trælle for Hebræerne, som de har trællet for eder; vær Mænd og kæmp!"
\par 10 Så begyndte Filisterne Kampen, og Israel blev slået, og de flygtede hver til sit; Nederlaget blev meget stort; der faldt 30000 Mand af det israelitiske Fodfolk,
\par 11 Guds Ark blev gjort til Bytte, og Elis to Sønner Hofni og Pinehas faldt.
\par 12 En Benjaminit løb bort fra Slaget og nåede samme Dag til Silo med sønderrevne Klæder og Jord på sit Hoved.
\par 13 Da han kom derhen, se, da sad Eli på sin Stol ved Porten og spejdede hen ad Vejen; thi hans Hjerte var uroligt for Guds Ark.
\par 14 og Eli hørte Skriget og sagde: "Hvad er det for en Larm?" Og Manden skyndte sig hen og fortalte Eli det.
\par 15 Men Eli var otte og halvfemsindstyve År gammel, og hans Øjne var blevet sløve, så han ikke kunde se.
\par 16 Og Manden sagde til Eli: "Det er mig, som kommer fra Slaget; jeg flygtede fra Slaget i Dag!" Da spurgte han: "Hvorledes er det gået, min Søn?"
\par 17 Og Budbringeren svarede: "Israel flygtede for Filisterne, og Folket led et stort Nederlag; også begge dine Sønner Hofni og Pinehas er faldet, og Guds Ark er taget!"
\par 18 Og da han nævnede Guds Ark, faldt Eli baglæns ned af Stolen ved Siden af Porten og brækkede Halsen og døde; thi Manden var gammel og tung. Han havde været Dommer for Israel i fyrretyve År.
\par 19 Men da hans Sønnekone, Pinehas's Hustru, der var højt frugtsommelig, hørte Efterretningen om, at Guds Ark var taget, og at hendes Svigerfader og hendes Mand var døde, sank hun om og fødte, thi Veerne kom over hende.
\par 20 Da hun lå på sit yderste, sagde de omstående Kvinder: "Frygt ikke, du har født en Søn!" Men hun svarede ikke og ænsede det ikke.
\par 21 Og hun kaldte drengen Ikabod, idet hun sagde: "Borte er Israels Herlighed!" Dermed hentydede hun til, at Guds Ark var taget, og til sin Svigerfader og sin Mand.
\par 22 Hun sagde: "Borte er Israels Herlighed, thi Guds Ark er taget!"

\chapter{5}

\par 1 Filisterne tog da Guds Ark og bragte den fra Eben Ezer til Asdod.
\par 2 Og Filisterne tog Guds Ark og bragte den ind i Dagons Hus og stillede den ved Siden af Dagon.
\par 3 Men tidligt næste Morgen, da Asdoditerne gik ind i Dagons Hus, se, da var Dagon faldet næsegrus til Jorden foran HERRENs Ark. De tog da Dagon og stillede ham på Plads igen.
\par 4 Men da de kom tidligt næste Morgen, se, da var Dagon faldet næsegrus til Jorden foran HERRENs Ark; Hovedet og begge Hænder var slået af og lå på Tærskelen; kun Kroppen var tilbage af ham.
\par 5 Derfor undgår Dagons Præster og alle, som går ind i dagons Hus, endnu den Dag i Dag at træde på Dagons Tærskel i Asdod.
\par 6 Og HERRENs Hånd lå tungt på Asdoditerne; han bragte Fordærvelse over dem, og med Pestbylder slog han Asdod og Egnen der omkring.
\par 7 Da Asdoditerne skønnede, hvorledes det hang sammen, sagde de: "Israels Guds Ark må ikke blive hos os, thi hans Hånd tager hårdt på os og på Dagon, vor Gud!"
\par 8 De sendte da Bud og kaldte alle Filisterfyrsterne sammen hos sig og sagde: "Hvad skal vi gøre med Israels Guds Ark?" De svarede: "Israels Guds Ark skal flyttes til Gat!" Så flyttede de Israels Guds Ark;
\par 9 men efter at de havde flyttet den derhen, ramte HERRENs Hånd Byen, så de grebes af stor Rædsel; og han slog Indbyggerne i Byen, små og store, så der brød Pestbylder ud på dem.
\par 10 De sendte da Guds Ark til Ekron; men da Guds Ark kom til Ekron, råbte Ekroniterne: "De har flyttet Israels Guds Ark over til mig for at bringe Død over mig og mit Folk!"
\par 11 Og de sendte Bud og kaldte alle Filisterfyrsterne sammen og sagde: "Send Israels Guds Ark bort og lad den komme hen igen, hvor den har hjemme, for at den ikke skal bringe Død over mig og mit folk!"Thi der var kommet Dødsangst over hele Byen, Guds Hånd lå såre tungt på den.
\par 12 De Mænd, som ikke døde, blev slået med Pestbylder, så at Klageråbet fra Byen nåede op til Himmelen.

\chapter{6}

\par 1 Efter at HERREN Ark havde været i Filisternes Land syv Måneder,
\par 2 lod Filisterne Præsterne og Sandsigerne kalde og sagde: "Hvad skal vi gøre med HERRENs Ark? Lad os få at vide, hvorledes vi skal bære os ad, når vi sender den hen, hvor den har hjemme!"
\par 3 De svarede: "Når I sender Israels Guds Ark tilbage, må I ikke sende den bort uden Gave; men I skal give den en Sonegave med tilbage; så bliver I raske og skal få at vide, hvorfor hans Hånd ikke vil vige fra eder!"
\par 4 De spurgte da: "Hvilken Sonegave skal vi give den med tilbage?" Og de sagde: "Fem Guldbylder og fem Guldmus svarende til Tallet på Filisterfyrsterne; thi det er en og samme Plage, der har ramt eder og eders Fyrster;
\par 5 I skal eftergøre eders Bylder og Musene, som hærger eders Land, og således give Israels Gud Æren; måske vil han da tage sin Hånd bort fra eder, eders Gud og eders Land.
\par 6 Hvorfor vil I forhærde eders Hjerte, som Ægypterne og Farao forhærdede deres Hjerte? Da han viste dem sin Magt, måtte de da ikke lade dem rejse, så de kunde drage af Sted?
\par 7 Tag derfor og lav en ny Vogn og tag to diegivende Køer, som ikke har båret Åg, spænd Køerne for Vognen og tag Kalvene fra dem og driv dem hjem;
\par 8 tag så HERRENs Ark og sæt den på Vognen, læg de Guldting, I giver den med i Sonegave, i et Skrin ved Siden af og send den så af Sted.
\par 9 Læg siden Mærke til, om den tager Vejen hjem ad Bet-Sjemesj til, thi så er det ham, som har voldt os denne store Ulykke; i modsat Fald ved vi, at det ikke var hans Hånd, som ramte os, men at det var en Hændelse!"
\par 10 Mændene gjorde da således; de tog to diegivende Køer og spændte dem for Vognen, men Kalvene lukkede de inde i Stalden.
\par 11 Derpå satte de HERRENs Ark på Vognen tillige med Skrinet med Guldmusene og Bylderne.
\par 12 Men Køerne gik den slagne Vej ad Bet-Sjemesj til; under ustandselig Brølen fulgte de stadig den samme Vej uden at bøje af til højre eller venstre, og Filisterfyrsterne fulgte med dem til Bet-Sjemesj's Landemærke.
\par 13 Da Bet-Sjemesjiterne, der var ved at høste Hvede i Dalen, så op og fik Øje på Arken, løb de den glade i Møde;
\par 14 og da Vognen var kommet til Bet-Sjemesjiten Jehosjuas Mark, standsede den. Der lå en stor Sten; de huggede da Træet af Vognen i Stykker og ofrede Køerne som Brændoffer til HERREN.
\par 15 Og Leviterne løftede HERRENs Ark ned tillige med Skrinet med Guldtingene, der stod ved Siden af, og satte den på den store Sten, og Mændene i Bet-Sjemesj bragte den Dag Brændofre og ofrede Slagtofre til HERREN.
\par 16 Da de fem Filisterfyrster havde set det, vendte de ufortøvet tilbage til Ekron.
\par 17 Dette er de Guldbylder, Filisterne lod følge med i Sonegave til HERREN: For Asdod een, for Gaza een, for Askalon een, for Gat een og for Ekron een.
\par 18 Guldmusene svarede til Tallet på alle Filisterbyerne, der tilhørte de fem Fyrster, både de befæstede Byer og Landsbyerne.
\par 19 Men Jekonjas Efterkommere havde ikke taget Del i Bet-Sjemesjiternes Glæde over at se HERRENs Ark; derfor ihjelslog han halvfjerdsindstyve Mænd iblandt dem. Da sørgede Folket, fordi HERREN havde slået så mange af dem ihjel;
\par 20 og Bet-Sjemesjiterne sagde:"Hvem kan bestå for HERRENs, denne hellige Guds, Åsyn? Og hvor vil han drage hen fra os?"
\par 21 Og de sendte Bud til Indbyggerne i Kirjat-Jearim og lod sige: "Filisterne har sendt HERRENs Ark tilbage. Kom herned og hent den op til eder!"

\chapter{7}

\par 1 Da kom Mændene fra Kirjat-Jearim og hentede HERRENs Ark op til sig og bragte den til Abinadabs Hus på Højen; og hans Søn El'azar helligede de til at vogte HERRENs Ark.
\par 2 Fra den Dag Arken fik sin Plads i Kirjat Jerim gik der lang Tid; der gik tyve År, og hele Israels Hus sukkede efter HERREN.
\par 3 Da sagde Samuel til hele Israels Hus: "Hvis I vil omvende eder til HERREN af hele eders Hjerte, skil eder så af med de fremmede Guder og Astarterne; vend eders Hu til HERREN og dyrk ham alene, så vil han fri eder af Filisternes Hånd!"
\par 4 Derpå skilte Israeliterne sig af, med Ba'alerne og Astarterne og dyrkede HERREN alene.
\par 5 Da sagde Samuel: "Kald hele Israel sammen i Mizpa, så vil jeg bede til HERREN for eder!"
\par 6 Så samlede de sig i Mizpa og øste Vand og udgød det for HERRENs Åsyn, og de fastede den Dag og sagde der: "Vi har syndet mod HERREN!" Derpå dømte Samuel Israeliterne i Mizpa.
\par 7 Da Filisterne hørte, at Israeliterne havde samlet sig i Mizpa, drog Filisterfyrsterne op imod Israel; og da Israeliterne hørte det, blev de bange for Filisterne.
\par 8 Og Israeliterne sagde til Samuel: "Hold ikke op med at råbe til HERREN vor Gud, at han må frelse os af Filisternes Hånd!"
\par 9 Da tog Samuel et diende Lam og bragte HERREN det som Brændoffer, som Heloffer; og Samuel råbte til HERREN for Israel, og HERREN bønhørte ham.
\par 10 Medens Samuel var i Færd med at bringe Brændofferet, rykkede Filisterne frem til Kamp mod Israel,men HERREN sendte den Dag et vældigt Tordenvejr over Filisterne og bragte dem i Uorden, så de blev slået af Israel;
\par 11 og Israels Mænd rykkede ud fra Mizpa, satte efter Filisterne og huggede dem ned lige til neden for Bet-Kar.
\par 12 Derpå tog Samuel en Sten og stillede den op mellem Mizpa og Jesjana; og han kaldte den Eben-Ezer, idet han sagde: "Hidtil har HERREN hjulpet os!"
\par 13 Således bukkede filisterne under, og de faldt ikke mere ind i Israels Land, men HERRENs Hånd lå tungt på Filisterne, så længe Samuel levede.
\par 14 De Byer, Filisterne havde taget, fik Israel tilbage, fra Ekron til Gat; også Landet der omkring frarev Israeliterne Filisterne; og der var Fred mellem Israel og Amoriterne.
\par 15 Samuel var Dommer i Israel, så længe han levede;
\par 16 han plejede årlig at drage rundt til Betel, Gilgal og Mizpa og dømme Israeliterne på alle disse Steder;
\par 17 derefter kom han hjem til Rama; thi der havde han sit Hjem, og der dømte han Israel. Og han byggede HERREN et Alter der.

\chapter{8}

\par 1 Da Samuel var blevet gammel, satte han sine Sønner til Dommere over Israel;
\par 2 hans førstefødte Søn hed Joel og hans anden Søn Abija; de dømte i Be'ersjeba.
\par 3 Men hans Sønner vandrede ikke i hans Spor; de lod sig lede af egen Fordel, tog imod Bestikkelse og bøjede Retten.
\par 4 Da kom alle Israels Ældste sammen og begav sig til Samuel i Rama
\par 5 og sagde til ham: "Se, du er blevet gammel, og dine Sønner vandrer ikke i dit Spor. Sæt derfor en Konge over os til at dømme os, ligesom alle de andre Folk har det!"
\par 6 Men det vakte Samuels Mishag, at de sagde: "Giv os en Konge, som kan dømme os!" Og Samuel bad til HERREN.
\par 7 Da sagde HERREN til Samuel: "Ret dig i et og alt efter, hvad Folket siger, thi det er ikke dig, de vrager, men det er mig, de vrager som deres Konge.
\par 8 Ganske som de har handlet imod mig, lige siden jeg førte dem ud af Ægypten og indtil denne Dag, idet de forlod mig og dyrkede andre Guder, således handler de også imod dig.
\par 9 Men ret dig nu efter dem; dog skal du indtrængende advare dem og lade dem vide, hvad Ret den Konge skal have, som skal herske over dem!"
\par 10 Så forebragte Samuel Folket, som krævede en Konge af ham, alle HERRENs Ord
\par 11 og sagde: "Denne Ret skal den Konge have, som skal herske over eder: Eders Sønner skal han tage og sætte ved sin Vogn og sine Heste, så de må løbe foran hans Vogn,
\par 12 og sætte dem til Tusindførere og Halvhundredførere og til at pløje og høste for ham og lave hans Krigsredskaber og Vogntøj.
\par 13 Eders Døtre skal han tage til at blande Salver, koge og bage.
\par 14 De bedste af eders Marker, Vingårde og Oliventræer skal han tage og give sine Folk.
\par 15 Af eders Sæd og Vinhøst skal han tage Tiende og give sine Hofmænd og Tjenere.
\par 16 De bedste af eders Trælle og Trælkvinder, det bedste af eders Hornkvæg og Æsler skal han tage og bruge til sit eget Arbejde.
\par 17 Af eders Småkvæg skal han tage Tiende; og I selv skal blive hans Trælle.
\par 18 Og når l da til den Tid klager over eders Konge, som I har valgt eder, så vil HERREN ikke bønhøre eder!"
\par 19 Folket vilde dog ikke rette sig efter Samuel, men sagde: "Nej, en Konge vil vi have over os,
\par 20 vi vil have det som alle de andre Folk; vor Konge skal dømme os og drage ud i Spidsen for os og føre vore Krige!"
\par 21 Da Samuel havde hørt alle Folkets Ord. forebragte han HERREN dem;
\par 22 og HERREN sagde til Samuel: "Ret dig efter dem og sæt en Konge over dem!" Da sagde Samuel til Israels Mænd: "Gå hjem, hver til sin By!"

\chapter{9}

\par 1 I Benjamin var der en Mand ved Navn Kisj, en Søn af Abiel, en Søn af Zeror, en Søn af Bekorat, en Søn af Afia, en Benjaminit, en formuende Mand.
\par 2 Han havde en Søn ved Navn Saul, statelig og smuk, ingen blandt Israeliterne var smukkere end han; han var et Hoved højere end alt Folket.
\par 3 Engang var nogle af Sauls Fader Kisjs Æsler blevet borte, og Kisj sagde da til sin Søn Saul: "Tag en af Karlene med og gå ud og søg efter Æslerne!"
\par 4 De gik så først gennem Efraims Bjerge og Sjalisja egnen, men fandt dem ikke; derefter gik de gennem Sja'alimegnen, men der var de heller ikke; derpå gik de gennem Benjamins Land, men fandt dem ikke.
\par 5 Da de kom til Zufegnen, sagde Saul til Karlen, som var med ham: "Kom, lad os vende hjem, for at min Fader ikke skal holde op med at tænke på Æslerne og i Stedet blive urolig for os!"
\par 6 Men han svarede ham: "Se, i Byen der bor en Guds Mand, en anset Mand; hvad han siger, sker altid. Lad os nu gå derhen, måske kan han give os Besked angående det, vi går om."
\par 7 Da sagde Saul til Karlen: "Ja, lad os gå derhen! Men hvad skal vi give Manden? Thi vi har ikke mere Brød i vore Tasker, og nogen Gave har vi ikke at give den Guds Mand. Hvad har vi?"
\par 8 Karlen svarede atter Saul: "Se, jeg har en kvart Sekel Sølv, den kan du give den Guds Mand; så siger han os nok Besked om det, vi går om."
\par 9 Fordum sagde man i Israel, når man gik hen for at rådspørge Gud: "Kom, lad os gå til Seeren!" Thi hvad man nu til Dags kalder en Profet, kaldte man fordum en Seer.
\par 10 Da sagde Saul til Karlen: "Du har Ret! Kom, lad os gå derhen!" Så gik de hen til Byen, hvor den Guds Mand boede.
\par 11 Som de nu gik ad Vejen op til Byen, traf de nogle unge Piger, der gik ud for at øse Vand, og de spurgte dem: "Er Seeren her?"
\par 12 De svarede dem: "Ja, han er foran; han kom til Byen lige nu.
\par 13 Når I blot går ind i Byen, kan I træffe ham, før han går op på Offerhøjen til Måltidet; thi Folket spiser ikke, før han kommer, da han skal velsigne Slagtofferet; så først spiser de indbudne.
\par 14 De gik så op til Byen; og som de gik ind igennem Porten, kom Samuel gående imod dem på Vej op til Offerhøjen.
\par 15 HERREN havde Dagen før Sauls Komme åbnet Samuels Øre og sagt:
\par 16 "I Morgen ved denne Tid sender jeg en Mand til dig fra Benjamins Land; ham skal du salve til Fyrste over mit Folk Israel; han skal frelse mit Folk fra Filisternes Hånd; thi jeg har givet Agt på mit Folks Nød, dets Klageråb har nået mig!"
\par 17 Og straks da Samuel fik Øje på Saul, sagde HERREN til ham: "Se, der er den Mand, om hvem jeg sagde til dig: Han skal herske over mit Folk!"
\par 18 Da trådte Saul hen til Samuel midt i Porten og sagde: "Vær så god at sige mig, hvor Seerens Hus er!"
\par 19 Samuel svarede: "Seeren det er mig; gå i Forvejen op på Offerhøjen; du skal spise sammen med mig i Dag; i Morgen skal jeg følge dig på Vej og kundgøre dig alt, hvad der er i dit Hjerte;
\par 20 for Æslerne, som for tre Dage siden blev borte for dig, skal du ikke ængste dig; de et fundet. Men til hvem står alt Israels Begær uden til dig og hele dit Fædrenehus?"
\par 21 Da svarede Saul: "Er jeg ikke fra Benjamin, Israels mindste Stamme? Og min Slægt er den ringeste af alle Benjamins Stammes Slægter. Hvor kan du da tale således til mig?"
\par 22 Men Samuel tog Saul og hans Karl, førte dem til Gildesalen og gav dem Plads øverst blandt de indbudne der var omtrent tredive Mænd
\par 23 og Samuel sagde til Kokken: "Ræk mig det Stykke, jeg gav dig og sagde, du skulde lægge til Side!"
\par 24 Da tog Kokken Køllen og satte den for Saul. Og Samuel sagde: "Se, Kødet står for dig, spis! Thi til den fastsatte Tid har man ventet dig, for at du kunde spise sammen med de indbudne." Så spiste Saul sammen med Samuel den Dag.
\par 25 Derpå steg de ned fra Offerhøjen fil Byen, og der blev redt til Saul på Taget.
\par 26 Så lagde han sig til Hvile. Tidligt om Morgenen, da Morgenrøden brød frem, råbte Samuel til Saul oppe på Taget: "Stå op, jeg vil følge dig på Vej!" Da stod Saul op, og han og Samuel gik ud sammen,
\par 27 Men da de på Nedvejen var kommet til Udkanten af Byen, sagde Samuel til Saul: "Sig til Karlen, at han skal gå i Forvejen! Men bliv du stående et Øjeblik, så vil jeg kundgøre dig Guds Ord!"

\chapter{10}

\par 1 Da tog Samuel Olieflasken og udgød Olien over hans Hoved, kyssede ham og sagde: "Har HERREN ikke salvet dig til Fyrste over sit Folk Israel? Du skal herske over HERRENs Folk og frelse det fra dets Fjender. Og dette skal være dig Tegnet på, at HERREN har salvet dig til Fyrste over sin Arv:
\par 2 Når du i Dag går fra mig, skal du træffe to Mænd ved Rakels Grav ved Benjamins Grænse i Zelza, og de skal sige til dig: Æslerne, du gik ud at lede efter, er fundet; dem har din Fader slået af Tanke, men nu er han urolig for eder og siger: Hvad skal jeg gøre for min Søn?
\par 3 Og når du er gået et Stykke længere frem og kommer til Taboregen, skal du træffe tre Mænd, som er på Vej op til Gud i Betel; den ene bærer tre Kid, den anden tre Brødkager og den tredje en Dunk Vin;
\par 4 de skal hilse på dig og give dig to Brødkager, som du skal tage imod.
\par 5 Derefter kommer du til Guds Gibea, hvor Filisternes Foged bor; og når du kommer hen til Byen, vil du støde på en Flok Profeter, som kommer ned fra Offerhøjen i profetisk Henrykkelse til Harpers, Paukers, Fløjters og Citres Klang;
\par 6 så vil HERRENs Ånd overvælde dig, så du falder i profetisk Henrykkelse sammen med dem, og du skal blive til et andet Menneske.
\par 7 Når disse Tegn indtræffer for dig, kan du trygt gøre, hvad der falder for; thi Gud er med dig.
\par 8 Og du skal gå i Forvejen ned til Gilgal; så kommer jeg ned til dig for at bringe Brændofre og ofre Takofre. Syv Dage skal du vente, til jeg kommer og kundgør dig, hvad du skal gøre!"
\par 9 Da han derpå vendte sig for at gå bort fra Samuel, gav Gud ham et helt andet Hjerte, og alle disse Tegn indtraf samme Dag.
\par 10 Da han kom hen til Gibea, se, da kom en Flok Profeter ham i Møde, og Guds Ånd overvældede ham, og han faldt i profetisk Henrykkelse midt iblandt dem.
\par 11 Og da alle, som kendte ham fra tidligere Tid, så ham i profetisk Henrykkelse sammen med Profeterne, sagde de til hverandre: "Hvad går der af Kisjs Søn? Er også Saul iblandt Profeterne?"
\par 12 Så sagde en der fra Stedet: "Hvem er vel deres Fader?" Derfor blev det et Mundheld: "Er også Saul iblandt Profeterne?"
\par 13 Da hans profetiske Henrykkelse var ovre, gik han til Gibea.
\par 14 Sauls Farbroder spurgte da ham og Karlen: "Hvor har I været henne?" Han svarede: "Ude at lede efter Æslerne; og da vi ikke fandt dem, gik vi hen til Samuel."
\par 15 Da sagde Sauls Farbroder: "Fortæl mig, hvad Samuel sagde til eder!"
\par 16 Saul svarede: "Han fortalte os, at Æslerne var fundet!" Men hvad Samuel havde sagt om Kongedømmet, fortalte han ham ikke.
\par 17 Derpå stævnede Samuel Folket sammen hos HERREN i Mizpa;
\par 18 og han sagde til Israeliterne: "Så siger HERREN, Israels Gud: Jeg førte Israel op fra Ægypten og frelste eder af Ægypternes Hånd og fra alle de Riger, som plagede eder.
\par 19 Men nu vrager I eders Gud, som var eders Frelser i alle eders Ulykker og Trængsler, og siger: Nej, en Konge skal du sætte over os! Så træd nu frem for HERRENs Åsyn Stamme for Stamme og Slægt for Slægt!"
\par 20 Derpå lod Samuel alle Israels Stammer træde frem, og Loddet ramte Benjamins Stamme.
\par 21 Så lod han Benjamins Stamme træde frem Slægt for Slægt, og Matris Slægt ramtes. Så lod han Matris Slægt træde frem Mand for Mand, og Saul, Kisjs Søn, ramtes. Men da man ledte efter ham, var han ikke til at finde.
\par 22 Da adspurgte de på ny HERREN: "Er Manden her?" Og HERREN svarede: "Se, han holder sig skjult ved Trosset."
\par 23 Så løb de hen og hentede ham der; og da han trådte ind imellem Folket, var han et Hoved højere end alt Folket.
\par 24 Da sagde Samuel til hele Folket: "Ser I ham, HERREN har udvalgt? Hans Lige findes ikke i alt Folket!" Og hele Folket brød ud i Jubelskrig og råbte: "Kongen leve!"
\par 25 Derpå fremsagde Samuel Kongedømmets Ret for Folket og optegnede den i en Bog, som han lagde hen for HERRENs Åsyn. Så lod Samuel hele Folket gå hver til sit:
\par 26 også Saul gik til sit Hjem i Gibea, og de tapre Mænd, hvis Hjerte Gud rørte, gik med ham.
\par 27 Men nogle Niddinger sagde: "Hvor skulde denne kunne hjælpe os?" Og de ringeagtede ham og bragte ham ingen Hyldingsgave. Men han var, som han var døv.

\chapter{11}

\par 1 Siden efter drog Ammoniten Nahasj op og belejrede Jabesj i Gilead. Da sagde alle Mændene i Jabesj til Nahasj: "Slut Pagt med os, så vil vi underkaste os!"
\par 2 Men Ammoniten Nahasj svarede: "Ja, på det Vilkår vil jeg slutte Pagt med eder, at jeg må stikke det højre Øje ud på enhver af eder til Forsmædelse for hele Israel!"
\par 3 Da sagde de Ældste i Jabesj til ham: "Giv os syv Dages Frist, så vi kan sende Bud rundt i hele Israels Land; hvis så ingen kommer os til Hjælp, vil vi overgive os til dig!"
\par 4 Da Sendebudene kom til Sauls Gibea og forebragte Folket Sagen, brast hele Folket i Gråd.
\par 5 Og se, Saul kom netop hjem med sine Okser fra Marken, og han spurgte: "Hvad er der i Vejen med Folket, siden det græder?" De fortalte ham da, hvad Mændene fra Jabesj havde sagt;
\par 6 og da Saul hørte det, overvældede Guds Ånd ham, og hans Vrede blussede heftigt op.
\par 7 Så tog han et Spand Okser og sønderhuggede dem, sendte Folk ud med Stykkerne i hele Israels Land og lod sige: "Hvis nogen ikke følger Saul og Samuel, skal der handles således med hans Okser!" Da faldt en HERRENs Rædsel over Folket, så de alle som een drog ud.
\par 8 Og han mønstrede dem i Bezek, og der var 300000 Israeliter og 30000 Judæere.
\par 9 Derpå sagde han til Sendebudene, som var kommet: "Således skal I sige til Mændene i Jabesj i Gilead: I Morgen, når Solen begynder at brænde, skal I få Hjælp!" Da Sendebudene kom og meddelte Mændene i Jabesj det, blev de glade.
\par 10 Og Mændene i Jabesj sagde: "I Morgen vil vi overgive os til eder, så kan I gøre med os, hvad I finder for godt!"
\par 11 Dagen efter delte Saul Hæren i tre Afdelinger, og de trængte ind i Lejren ved Morgenvagten og huggede ned blandt Ammoniterne, til det blev hedt; og de, som undslap, splittedes til alle Sider, så ikke to og to blev sammen.
\par 12 Da sagde Folket til Samuel: "Hvem var det, som sagde: Skal Saul være Konge over os? Bring os de Mænd, at vi kan slå dem ihjel!"
\par 13 Men Saul sagde: "I Dag skal ingen slås ihjel; thi i Dag har HERREN givet Israel Sejr!"
\par 14 Da sagde Samuel til Folket: "Kom, lad os gå til Gilgal og gentage Kongevalget der!"
\par 15 Så gik hele Folket til Gilgal og gjorde Saul til Konge for HERRENs Åsyn der i Gilgal, og de bragte Takofre der for Herrens Åsyn. Og Saul og alle Israels Mænd var højlig glade.

\chapter{12}

\par 1 Da sagde Samuel til hele Israel: "Se, jeg har føjet eder i alt, hvad I har bedt mig om, og sat en Konge over eder.
\par 2 Se, nu færdes Kongen for eders Ansigt; jeg er gammel og grå, og mine Sønner er nu iblandt eder; men jeg har færdedes for eders Ansigt fra min Ungdom indtil i Dag.
\par 3 Se, her står jeg; viden imod mig i HERRENs og hans, Salvedes Påhør! Hvis Okse har jeg taget? Hvis Æsel har jeg taget? Hvem har jeg, undertrykt? Hvem har jeg gjort Uret? Af hvem har jeg taget Gave og derfor lukket Øjnene? I så Fald vil jeg give eder Erstatning!"
\par 4 Da sagde de: "Du har ikke undertrykt os eller gjort os Uret eller taget noget fra nogen."
\par 5 Derpå sagde han til dem: "Så er HERREN i Dag Vidne over for eder, også hans Salvede er Vidne, at I ikke har fundet noget hos mig." De sagde: "Ja!"
\par 6 Da sagde Samuel til Folket: "HERREN er Vidne, han, som udrustede Moses og Aron og førte eders Fædre op fra Ægypten.
\par 7 Så træd nu frem, at jeg kan gå i Rette med eder for HERRENs Åsyn og kundgøre eder alle de Gerninger, HERREN i sin Retfærdigher har øvet mod eder og eders Fædre.
\par 8 Da Jakob og hans Sønner var kommet til Ægypten, og Ægypterne plagede dem, råbte eders Fædre til HERREN, og HERREN sendte Moses og Aron, som førte eders Fædre ud af Ægypten, og han lod dem bosætte sig her.
\par 9 Men de glemte HERREN deres Gud; derfor prisgav han dem til Kong Jabin af Hazors Hærfører Sisera, til Filisterne og til Moabs Konge, så de angreb dem.
\par 10 Da råbte de til HERREN og sagde: Vi har syndet, thi vi forlod HERREN og dyrkede Ba'alerne og Astarterne; men fri os nu af vore Fjenders Hånd, så vil vi dyrke dig!
\par 11 Så sendte HERREN Jerubba'al, Barak, Jefta og Samuel; og han friede eder af eders Fjenders Hånd rundt om, så I kunde bo i Tryghed.
\par 12 Men da I så Ammoniterkongen Nahasj rykke frem imod eder, sagde I til mig: Nej, en Konge skal herske over os uagtet HERREN eders Gud var eders Konge!
\par 13 Og nu, her står Kongen, som I har valgt og krævet; se, HERREN har sat en Konge over eder!
\par 14 Hvis I frygter HERREN og tjener ham, adlyder hans Røst og ikke er genstridige mod HERRENs Bud, men følger HERREN eders Gud, både I og Kongen, som har fået Herredømmet over eder, da skal det gå eder vel.
\par 15 Adlyder I derimod ikke HERRENs Røst, men er genstridige mod HERRENs Bud, da skal HERRENs Hånd ramme eder og eders Konge og ødelægge eder.
\par 16 Træd nu frem og se den vældige Gerning, HERREN vil øve for eders Øjne!
\par 17 Har vi ikke Hvedehøst nu? Men jeg vil råbe til HERREN, at han skal sende Torden og Regn, for at I kan kende og se, at det i HERRENs Øjne var en stor Brøde I begik, da I krævede en Konge!"
\par 18 Derpå råbte Samuel til HERREN, og HERREN sendte samme Dag Torden og Regn. Da frygtede hele folket såre for HERREN og Samuel,
\par 19 og hele Folket sagde til Samuel: "Bed for dine Trælle til HERREN din Gud, at vi ikke skal dø, fordi vi til vore andre Synder har føjet den Brøde at kræve en Konge!"
\par 20 Da sagde Samuel til Folket: "Frygt ikke! Vel har I øvet al den Synd; men vend eder nu ikke fra HERREN, tjen ham af hele eders Hjerte
\par 21 og vend eder ikke til dem, som er Tomhed og hverken kan hjælpe eller frelse, fordi de er Tomhed.
\par 22 Thi for sit store Navns Skyld vil HERREN ikke forstøde sit Folk, da det jo har behaget HERREN at gøre eder til sit Folk.
\par 23 Det være også langt fra mig at synde mod HERREN og høre op med at bede for eder; jeg vil også vise eder den gode og rette Vej;
\par 24 men frygt HERREN og tjen ham oprigtigt af hele eders Hjerte; thi se, hvor store Ting han gjorde imod eder!
\par 25 Men hvis I handler ilde, skal både I og eders Konge gå til Grunde!"

\chapter{13}

\par 1 Saul var....År ved sin Tronbestigelse, og han herskede i....År over Israel.
\par 2 Saul udvalgte sig 3000 Mand af Israel; af dem var 2000 hos Saul i Mikmas og i Bjergene ved Betel, 1000 hos Jonatan i Gibea i Benjamin; Resten af Krigerne lod han gå hver til sit.
\par 3 Da fældede Jonatan Filisternes Foged i Geba. Det kom nu Filisterne for Øre, at Hebræerne havde revet sig løs. Men Saul havde ladet støde i Hornet hele Landet over,
\par 4 og hele Israel hørte, at Saul havde fældet Filisternes Foged, og at Israel havde vakt Filisternes Vrede. Og Folket stævnedes sammen i Gilgal til at følge Saul,
\par 5 men Filisterne havde samlet sig til Kamp mod Israel, 3000 Stridsvogne, 6000 Ryttere og Fodfolk så talrigt som Sandet ved Havets Bred, og de drog op og lejrede sig i Mikmas lige over for Bet-Aven.
\par 6 Da Israels Mænd skønnede, hvilken Fare de var i thi Folket blev trængt, skjulte Folket sig i Huler, Jordhuller, Klipperevner, Gruber og Cisterner
\par 7 eller gik over Jordans Vadesteder til Gads og Gileads Land. Men Saul var endnu i Gilgal, og hele Folket fulgte ham med Frygt i Sind.
\par 8 Han ventede syv Dage til den Tid, Samuel havde fastsat; men Samuel kom ikke til Gilgal. Da Folket så spredte sig og forlod Saul,
\par 9 sagde han: "Bring Brændofferet og Takofrene hen til mig!" Så ofrede han Brændofferet.
\par 10 Men lige som han var færdig med at ofre Brændofferet, se, da kom Samuel, og Saul gik ham i Møde for at hilse på ham.
\par 11 Da sagde Samuel: "Hvad har du gjort!" Saul svarede: "Jeg så, at Folket spredte sig og forlod mig, men du kom ikke til den fastsatte Tid, og Filisterne samlede sig ved Mikmas;
\par 12 så tænkte jeg: Nu drager Filisterne ned til Gilgal imod mig, og jeg har endnu ikke vundet HERRENs Gunst; da tog jeg Mod til mig og bragte Brændofferet!"
\par 13 Samuel sagde til Saul: "Tåbeligt har du handlet. Hvis du havde holdt den Befaling, HERREN din Gud gav dig, vilde HERREN nu have grundfæstet dit Kongedømme over Israel til evig Tid;
\par 14 men nu skal dit Kongedømme ikke bestå. HERREN har udsøgt sig en Mand efter sit Hjerte, og ham har HERREN kaldet til Fyrste over sit Folk, fordi du ikke holdt, hvad HERREN bød dig!"
\par 15 Derpå brød Samuel op og gik bort fra Gilgal; men den tilbageblevne Del af Folket drog op i Følge med Saul for at støde til Krigerne, og de kom fra Gilgal til Gibea i Benjamin. Da mønstrede Saul de Folk, han havde hos sig, omtrent 600 Mand;
\par 16 og Saul og hans Søn Jonatan og de Folk, de havde hos sig, lå i Geba i Benjamin, medens Filisterne lå lejret i Mikmas.
\par 17 Fra Filisternes Lejr drog så en Skare ud i tre Afdelinger for at plyndre; den ene Afdeling drog i Retning af Ofra til Sjualegnen,
\par 18 den anden i Retning af Bet Horon og den tredje i Retning af den Høj, som rager op over Zeboimdalen, ad Ørkenen til.
\par 19 Men der fandtes ingen Smede i hele Israels Land; thi Filisterne havde tænkt, at Hebræerne ellers kunde lave sig Sværd og Spyd;
\par 20 derfor måtte hele Israel drage ned til Filisterne for at få hvæsset deres Plovjern, Hakker, Økser eller Pigkæppe;
\par 21 det kostede en Pim at få slebet Plovjem og Hakker og en Tredjedel Sekel for Økser og for at indsætte Pig.
\par 22 Således fandtes der, den Dag Slaget stod ved Mikmas, hverken Sværd eller Spyd hos nogen af Krigerne, som var hos Saul og Jonatan; kun Saul og hans Søn Jonatan havde Våben.
\par 23 Filisternes Forpost rykkede frem til Mikmaspasset.

\chapter{14}

\par 1 Da hændte det en Dag, at Sauls Søn Jonatan sagde til sin Våbendrager: "Kom, lad os gå over til Filisternes Forpost her lige overfor!" Men til sin Fader sagde han intet derom.
\par 2 Saul sad just ved Udkanten af Geba under Granatæbletræet ved Tærskepladsen, og folkene, som var hos ham, var omtrent 600 Mand.
\par 3 Og Ahija, en Søn af Ahitub, der var Broder til Ikabod, en Søn af Pinehas, en Søn af Eli, HERRENs Præst i Silo, bar Efoden. Men Folkene vidste intet om, at Jonatan var gået.
\par 4 I Passet, som Jonatan søgte at komme over for at angribe Filisternes Forpost, springer en Klippespids frem på hver Side; den ene hedder Bozez, den anden Sene.
\par 5 Den ene Spids rager i Vejret på Nordsiden ud for Mikmas, den anden på Sydsiden ud for Geba.
\par 6 Jonatan sagde da til Våbendrageren: "Kom, lad os gå over til disse uomskårnes Forpost; måske vil HERREN stå os bi, thi intet hindrer HERREN i at give Sejr, enten der er mange eller få!"
\par 7 Våbendrageren svarede: "Gør, hvad du har i Sinde! Jeg går med; som du vil, vil også jeg!"
\par 8 Da sagde Jonatan: "Vi søger nu at komme over til de Mænd og sørger for, at de får os at se.
\par 9 Hvis de så siger til os: Stå stille, vi kommer hen til eder! så bliver vi stående, hvor vi står, og går ikke op til dem.
\par 10 Men siger de: Kom op til os! går vi derop; thi så har HERREN givet dem i vor Hånd; det skal være vort Tegn!"
\par 11 Da nu Filisternes Forpost fik Øje på dem, sagde Filisterne: "Se, der kommer nogle Hebræere krybende ud af de Jordhuller, de har skjult sig i!"
\par 12 Og Mændene, der stod på Forpost, råbte til Jonatan og hans Våbendrager: "Kom op til os, så skal vi lære jer!" Da sagde Jonatan til Våbendrageren: "Følg med derop, thi HERREN har givet dem i Israels Hånd!"
\par 13 Så klatrede Jonatan op på Hænder og Fødder, og Våbendrageren bagefter. Da flygtede de for Jonatan; og han huggede dem ned, og Våbendrageren fulgte efter og gav dem Dødsstødet;
\par 14 og i første Omgang fældede Jonatan og hans Våbendrager henved tyve Mand på en Strækning af omtrent en halv Dags Pløjeland.
\par 15 Da opstod der Rædsel både i og uden for Lejren. og alle Krigerne, både Forposten og Strejfskaren, sloges med Rædsel; tilmed kom der et Jordskælv, og det fremkaldte en Guds Rædsel.
\par 16 Men da Sauls Udkigsmænd i Geba i Benjamin så derhen, opdagede de, at det bølgede hid og did i Lejren.
\par 17 Da sagde Saul til sine Folk: "Hold Mønstring og se efter, hvem af vore der er gået bort!" Og ved Mønstringen viste det sig, at Jonatan og hans Våbendrager manglede.
\par 18 Da sagde Saul til Ahija: "Bring Efoden hid!" Han bar nemlig dengang Efoden foran Israel.
\par 19 Men medens Saul talte med Præsten, blev Forvirringen i Filisternes Lejr større og større. Saul sagde da til Præsten: "Lad det kun være!"
\par 20 Og alle Sauls Krigere samlede sig om ham, og da de kom til Kamppladsen, se, da var den enes Sværd løftet mod den andens, og alt var i stor Forvirring.
\par 21 Og de Hebræere, som tidligere havde stået under Filisterne og havde gjort dem Hærfølge, faldt fra og sluttede sig til Israel, som fulgte Saul og Jonatan.
\par 22 Og da de israelitiske Mænd, som havde skjult sig i Efraims Bjerge, hørte, at Filisterne var på Flugt, satte også de efter dem for at bekæmpe dem.
\par 23 Således gav HERREN Israel Sejr den Dag. Da Kampen havde strakt sig hen forbi Bet Horon -
\par 24 alle Krigerne var med Saul, omtrent 10000 Mand, og Kampen bredte sig over Efraims Bjerge begik Saul den Dag en stor Dårskab, idet han tog Folket i Ed og sagde: "Forbandet være hver den, som nyder noget før Aften, før jeg får taget Hævn over mine Fjender!" Og alt Folket afholdt sig fra at spise.
\par 25 Der fandtes nogle Bikager på Marken,
\par 26 og da Folket kom til Bikagerne, var Bierne borte; men ingen førte Hånden til Munden; thi Folket frygtede Eden.
\par 27 Jonatan havde dog ikke hørt, at hans Fader tog Folket i Ed, og han rakte Spidsen af den Stav, han havde i Hånden, ud, dyppede den i en Bikage og førte Hånden til Munden; derved fik hans Øjne atter Glans.
\par 28 Da tog en af Krigerne til Orde og sagde: "Din Fader tog Folket i Ed og sagde: Forbandet være hver den, som nyder noget i Dag! Og dog var Folket udmattet."
\par 29 Men Jonatan sagde: "Min Fader styrter Landet i Ulykke! Se, hvor mine Øjne fik Glans, fordi jeg nød den Smule Honning!
\par 30 Nej, havde Folket blot i bag spist dygtigt af Byttet, det tog fra Fjenden! Thi nu blev Filisternes Nederlag ikke stort."
\par 31 De slog da den Dag Filisterne fra Mikmas til Ajjalon, og Folket var meget udmattet.
\par 32 Derfor kastede Folket sig over Byttet, tog Småkvæg, Hornkvæg og Kalve og slagtede dem på Jorden og spiste Kødet med Blodet i.
\par 33 Da meldte man det til Saul og sagde: "Se, Folket synder mod HERREN ved at spise Kødet med Blodet i!" Og han sagde: "I forbryder eder! Vælt mig en stor Sten herhen!"
\par 34 Derpå sagde Saul: "Gå rundt iblandt Folket og sig til dem: Enhver skal bringe sin Okse eller sit Får hen til mig og slagte det her! Så kan I spise; men synd ikke mod HERREN ved at spise Kødet med Blodet i!" Da bragte hver og en af Folket, hvad han havde, og slagtede det der.
\par 35 Og Saul byggede HERREN et Alter; det var det første Alter, han byggede HERREN.
\par 36 Derpå sagde Saul: "Lad os drage ned efter Filisterne i Nat og udplyndre dem, før Dagen gryr, og ikke lade nogen af dem blive tilbage!" De svarede: "Gør, hvad du under for godt!" Men Præsten sagde: "Lad os her træde frem for Gud!"
\par 37 Så rådspurgte Saul Gud:"Skal jeg drage ned efter Filisterne? Vil du give dem i Israels Hånd?" Men han svarede ham ikke den Dag.
\par 38 Da sagde Saul: "Kom hid, alle Folkets Øverster, og se efter, hvad det er for en Synd, der er begået i Dag;
\par 39 thi så sandt HERREN lever, han, som har givet Israel Sejr: Om det så er min Søn Jonatan, der har begået den, skal han dø!" Men ingen af Folket svarede.
\par 40 Da sagde han til hele Israel: "I skal være den ene Part, jeg og min Søn Jonatan den anden!" Folket svarede Saul: "Gør, hvad du finder for godt!"
\par 41 Derpå sagde Saul til HERREN: "Israels Gud! Hvorfor svarer du ikke din Tjener i Dag? Hvis Skylden ligger hos mig eller min Søn Jonatan, HERRE, Israels Gud, så lad Urim komme frem; men ligger den hos dit Folk Israel, så lad Tummim komme frem!" Da ramtes Jonatan og Saul af Loddet, men Folket gik fri.
\par 42 Saul sagde da: "Kast Lod mellem mig og min Søn Jonatan!" Så ramtes Jonatan.
\par 43 Da sagde Saul til Jonatan: "Sig mig, hvad du har gjort!" Jonatan svarede: "Jeg nød lidt Honning på Spidsen af Staven, jeg havde i Hånden. Se, jeg er rede til at dø!"
\par 44 Da sagde Saul: "Gud ramme mig både med det ene og det andet! Du skal visselig dø, Jonatan!"
\par 45 Men Folket sagde til Saul: "Skal Jonatan dø, han, som har vundet Israel denne store Sejr? Det være langt fra! Så sandt HERREN lever, ikke et Hår skal krummes på hans Hoved; thi med Guds Hjælp vandt han Sejr i Dag!" Da udløste Folket Jonatan, og han blev friet fra Døden.
\par 46 Men Saul holdt op med at forfølge Filisterne og drog hjem, medens Filisterne trak sig tilbage til deres Land.
\par 47 Da Saul havde vundet Kongedømmet over Israel, førte han Krig med alle sine Fjender rundt om, Moab, Ammoniterne, Edom, Kongen af Zoba og Filisterne, og Sejren fulgte ham overalt, hvor han vendte sig hen.
\par 48 Han udførte Heltegerninger, slog Amalek og befriede Israel fra dem, som hærgede det.
\par 49 Sauls Sønner var Jonatan, Jisjvi og Malkisjua; af hans to Døtre hed den førstefødte Merab og den yngste Mikal.
\par 50 Sauls Hustru hed Ahinoam, en Datter af Ahimåz; hans Hærfører hed Abiner, en Søn af Sauls Farbroder Ner;
\par 51 Sauls Fader Kisj og Abners Fader Ner var Sønner af Abiel.
\par 52 Men Krigen med Filisterne var hård, lige så længe Saul levede; og hver Gang Saul traf en heltemodig og tapper Mand, knyttede han ham til sig.

\chapter{15}

\par 1 Samuel sagde til Saul: "Det var mig, HERREN sendte for at salve dig til konge over hans Folk Israel; lyd nu HERRENs Røst.
\par 2 Så siger Hærskarers HERRE: Jeg vil straffe Amalek for, hvad de gjorde mod Israel, da de stillede sig i Vejen for det på Vandringen op fra Ægypten.
\par 3 Drag derfor hen og slå Amalek og læg Band på dem og på alt, hvad der tilhører dem; skån dem ikke, men dræb både Mænd og Kvinder, Børn og diende, Okser og Får, Kameler og Æsler!"
\par 4 Så stævnede Saul Folket sammen og mønstrede dem i Telaim, 200000 Mand Fodfolk og 10000 Mand af Juda.
\par 5 Derpå drog Saul mod Amaleks By og lagde Baghold i Dalen.
\par 6 Men Saul sagde til Keniterne: "Skil eder fra Amalekiterne og gå eders Vej, for at jeg ikke skal udrydde eder sammen med dem; I viste jo Venlighed mod alle Israeliterne, dengang de drog op fra Ægypten!" Så trak Keniterne sig tilbage fra Amalek.
\par 7 Og Saul slog Amalek fra Havila til Sjur, som ligger østen for Ægypten,
\par 8 og tog Kong Agag af Amalek levende til Fange. På alt Folket lagde han Band og huggede dem ned med Sværdet;
\par 9 men Saul og Folket skånede Agag og det bedste af Småkvæget og Hornkvæget, de fede og velnærede Dyr, alt det bedste; de vilde ikke lægge Band på dem,men på alt det dårlige og værdiløse Kvæg lagde de Band.
\par 10 Da kom HERRENs Ord til Samuel således:
\par 11 "Jeg angrer, at jeg gjorde Saul til Konge; thi han har vendt sig fra mig og ikke holdt mine Befalinger!" Da vrededes Samuel og råbte til HERREN hele Natten.
\par 12 Næste Morgen tidlig, da Samuel vilde gå Saul i Møde, blev der meldt ham: "Saul kom til Karmel og rejste sig et Mindesmærke der; så vendte han om og drog videre ned til Gilgal!"
\par 13 Samuel begav sig da til Saul. Saul sagde til ham: "HERREN velsigne dig! Jeg har holdt HERRENs Befaling!"
\par 14 Men Samuel sagde: "Hvad er det for en Brægen af Småkvæg, som når mit Øre, og Brølen af Hornkvæg, jeg hører?"
\par 15 Saul svarede: "De tog dem med fra Amalekiterne; thi Folket skånede det bedste af Småkvæget og Hornkvæget for at ofre det til HERREN din Gud; på det andet derimod lagde vi Band!"
\par 16 Da sagde Samuel til Saul: "bet er nok! Jeg vil kundgøre dig, hvad HERREN i Nat har sagt mig!" Han svarede: "Tal!"
\par 17 Da sagde Samuel: "Om du end ikke regner dig selv for noget, er du så ikke Høvding for Israels Stammer, og salvede HERREN dig ikke til Konge over Israel?
\par 18 Og HERREN sendte dig af Sted med den Befaling: Gå hen og læg Band på Amalekiterne, de Syndere, og før Krig imod dem, indtil du har udryddet dem!
\par 19 Hvorfor adlød du da ikke HERRENs Røst, men styrtede dig over Byttet og gjorde, hvad der er ondt i HERRENs Øjne?"
\par 20 Saul svarede Samuel: "Jeg adlød HERRENs Røst og gik, hvor HERREN sendte mig hen; jeg har bragt Kong Agag af Amalek med og lagt Band på Amalek;
\par 21 men Folket tog Småkvæg og Hornkvæg af Byttet, det bedste af det bandlyste, for at ofre det til HERREN din Gud i Gilgal."
\par 22 Men Samuel sagde: "Mon HERREN har lige så meget Behag i Brændofre og Slagtofre som i Lydighed mod HERRENs Høst? Nej, at adlyde er mere værd end Slagtoffer, og at være lydhør er mere værd end Væderfedt;
\par 23 thi Genstridighed er Trolddomssynd, og Egenrådighed er Afgudsbrøde. Fordi du har forkastet HERRENs Ord, har han forkastet dig, så du ikke mere skal være Konge!"
\par 24 Da sagde Saul til Samuel: "Jeg har syndet, thi jeg har overtrådt HERRENs Befaling og dine Ord, men jeg frygtede Folket og føjede dem:
\par 25 tilgiv mig dog nu min Synd og vend tilbage med mig, for at jeg kan tilbede HERREN!"
\par 26 Men Samuel sagde til Saul: "Jeg vender ikke tilbage med dig; fordi du har forkastet HERRENs Ord, har HERREN forkastet dig, så du ikke mer skal være Konge over Israel!"
\par 27 Derpå vendte Samuel sig for at gå, men Saul greb fat i hans Kappeflig, så den reves af.
\par 28 Da sagde Samuel til ham: "HERREN har i Dag revet Kongedømmet over Israel fra dig og givet det til en anden, som er bedre end du!
\par 29 Visselig, han, som er Israels Herlighed, lyver ikke, ej heller angrer han; thi han er ikke et Menneske, at han skulde angre!"
\par 30 Saul sagde: "Jeg har syndet; men vis mig dog Ære for mit Folks Ældste og Israel og vend tilbage med mig, for at jeg kan tilbede HERREN din Gud!"
\par 31 Da vendte Samuel tilbage med Saul, og Saul tilbad HERREN.
\par 32 Derpå sagde Samuel: "Bring Kong Agag af Amalek hid til mig!" Og Agag gik frejdigt hen til ham og sagde: "Visselig, nu er Dødens Bitterhed svundet!"
\par 33 Da sagde Samuel: "Som dit Sværd har gjort Kvinder barnløse, skal din Moder blive barnløs fremfor andre Kvinder!" Derpå sønderhuggede Samuel Agag for HERRENs Åsyn i Gilgal.
\par 34 Samuel begav sig så til Rama, mens Saul drog op til sit Hjem i Sauls Gibea.
\par 35 Og Samuel så ikke mere Saul indtil sin Dødedag; thi Samuel sørgede over Saul. 16 HERREN angrede, at han havde gjort Saul til Konge over Israel;

\chapter{16}

\par 1 og HERREN sagde til Samuel: "Hvor længe vil du gå og sørge over Saul? Jeg har jo dog forkastet ham, så han ikke mere skal være Konge over Israel. Fyld dit Horn med Olie og drag af Sted! Jeg sender dig til Betlehemiten Isaj, thi jeg har udset mig en Konge blandt hans Sønner."
\par 2 Samuel svarede: "Hvorledes kan jeg det? Får Saul det at høre, dræber han mig!" Men HERREN sagde: "Tag en Kvie med og sig: Jeg kommer for at ofre HERREN et Offer!
\par 3 Og indbyd Isaj til Ofringen; så vil jeg lade dig vide, hvad du skal gøre; du skal salve mig den, jeg siger dig!"
\par 4 Samuel gjorde da, som HERREN sagde. Da han kom til Betlehem, gik Byens Ældste ham forfærdede i Møde og sagde: "Kommer du for det gode?"
\par 5 Han svarede: "Ja! Jeg kommer for at ofre til HERREN. Helliger eder og kom med til Ofringen!" Og han lod Isaj og hans Sønner hellige sig og indbød dem til Ofringen:
\par 6 Da de kom, og han så Eliab, tænkte han: "Visselig står nu HERRENs Salvede for ham!"
\par 7 Men HERREN sagde til Samuel: "Se ikke på hans Ydre eller høje Vækst; thi jeg har vraget ham; Gud ser jo ikke, som Mennesker ser, thi Mennesker ser på det, som er for Øjnene, men HERREN ser på Hjertet."
\par 8 Da kaldte Isaj på Abinadab og førte ham hen for Samuel; men han sagde: "Heller ikke ham har HERREN udvalgt!"
\par 9 Isaj førte da Sjamma frem; men han sagde: "Heller ikke ham har HERREN udvalgt!"
\par 10 Så førte Isaj de andre af sine syv Sønner frem for Samuel; men Samuel sagde til Isaj: "HERREN har ikke udvalgt nogen af dem!"
\par 11 Samuel spurgte da Isaj: "Er det alle de unge Mænd?" Han svarede: "Endnu er den yngste tilbage; men han vogter Småkvæget!" Da sagde Samuel til Isaj: "Send Bud efter ham! thi vi sætter os ikke til Bords, før han kommer!"
\par 12 Så sendte han Bud efter ham. Han var rødmosset, en Yngling med smukke Øjne og skøn at se til. Da sagde HERREN: "Stå op, salv ham, thi ham er det!"
\par 13 Samuel tog da Oliehornet og salvede ham, medens hans Brødre stod rundt om: Og HERRENs Ånd kom over David fra den Dag af.
\par 14 Efter at HERRENs Ånd var veget fra Saul, plagedes han af en ond Ånd fra HERREN.
\par 15 Sauls Folk sagde da til ham: "Se, en ond Ånd fra Gud plager dig;
\par 16 sig kun et Ord, Herre, dine Trælle står rede til at søge efter en Mand, der kan lege på Strenge; når en ond Ånd fra Gud kommer over dig, skal han røre Strengene; så får du det godt!"
\par 17 Da sagde Saul til sine Folk: "Find mig en Mand, der er dygtig til Strengeleg, og bring ham til mig!"
\par 18 En af Tjenerne tog til Orde og sagde: "Jeg har set en Søn af Betlehemiten Isaj, han kan lege på Strenge og er en dygtig Kriger, en øvet Krigsmand; han ved at føje sine Ord og er en smuk Mand, og HERREN er med ham!"
\par 19 Saul sendte da Bud til Isaj og lod sige: "Send mig din Søn David, som er ved Fårene!"
\par 20 Da tog Isaj ti Brød, en Lædersæk Vin og et Gedekid og sendte sin Søn David til Saul dermed.
\par 21 Således kom David til Saul og trådte i hans Tjeneste; Saul fik ham såre kær, og han blev hans Våbendrager.
\par 22 Og Saul sendte Bud til Isaj og lod sige: "Lad David blive i min Tjeneste, thi jeg har fattet Godhed for ham!"
\par 23 Når nu Ånden fra Gud kom over Saul, tog David sin Citer og rørte Strengene; så følte Saul Lindring og fik det godt, og den onde Ånd veg fra ham.

\chapter{17}

\par 1 Filisterne samlede deres Hær til Kamp. De samlede sig ved Soko i Juda og slog Lejr mellem Soko og Azeka i Efes-Dammim.
\par 2 Ligeledes samlede Saul og Israels Mænd sig og slog Lejr i Terebintedalen og gjorde sig rede til at angribe Filisterne.
\par 3 Filisterne stod ved Bjerget på den ene Side, Israeliterne ved Bjerget på den anden, med Dalen imellem sig.
\par 4 Da trådte en Tvekæmper ved Navn Goliat fra Gat ud af Filisternes Rækker, seks Alen og et Spand høj.
\par 5 Han havde en Kobberhjelm på Hovedet, var iført en Skælbrynje, hvis Kobber vejede 5000 Sekel,
\par 6 og havde Kobberskinner på Benene og et Kobberspyd over Skulderen.
\par 7 Hans Spydstage var som en Væverbom, og hans Spydsod var af Jern og vejede 600 Sekel; hans Skjolddrager gik foran ham.
\par 8 Han stod frem og råbte over til Israels Slagrækker: "Hvorfor drager I ud til Angreb? Er jeg ikke en Filister og I Sauls Trælle? Vælg jer en Mand og lad ham komme herned til mig!
\par 9 Hvis han kan tage Kampen op med mig og dræber mig, vil vi være eders Trælle, men får jeg Bugt med ham og dræber ham, skal I være vore Trælle og trælle for os!"
\par 10 Yderligere sagde Filisteren: "I Dag har jeg hånet Israels Slagrækker; kom med en Mand, så vi kan kæmpe sammen!"
\par 11 Da Saul og hele Israel hørte disse Filisterens Ord, blev de forfærdede og grebes af Rædsel.
\par 12 David var Søn af en Efratit i Betlehem i Juda ved Navn Isaj, som havde otte Sønner. Denne Mand var på Sauls Tid gammel og til Års.
\par 13 Isajs tre ældste Sønner havde fulgt Saul i Krigen, og Navnene på hans tre ældste Sønner, som var draget i Krigen, var Eliab, den førstefødte, Abinadab, den næstældste, og Sjamma, den tredje;
\par 14 David var den yngste. De tre ældste havde fulgt Saul;
\par 15 og David gik af og til hjem fra Saul for at vogte sin Faders Småkvæg i Betlehem.
\par 16 Men Filisteren trådte frem og tilbød Kamp hver Morgen og Aften i fyrretyve Dage.
\par 17 Nu sagde Isaj engang til sin Søn David: "Tag en Efa af det ristede Korn her og disse ti Brød til dine Brødre og løb hen til dem i Lejren med det
\par 18 og bring disse ti Skiver Flødeost til Tusindføreren; og se så, hvorledes det går dine Brødre, og få et Pant af dem;
\par 19 Saul ligger med dem og alle Israels Mænd i Terebintedalen og kæmper med Filisterne!"
\par 20 Næste Morgen tidlig overlod David Småkvæget til en Vogter, tog Sagerne og gav sig på Vej,som Isaj havde pålagt ham; og han kom til Vognborgen, netop som Hæren rykkede ud til Slag og opløftede Kampråbet.
\par 21 Både Israel og Filisterne stod rede til Kamp, Slagorden mod Slagorden.
\par 22 David lagde sine Sager fra sig og overlod dem til Vagten ved Trosset, løb ind mellem Slagrækkerne og gik hen og hilste på sine Brødre.
\par 23 Medens han talte med dem, se, da kom Tvekæmperen - Filisteren Goliat hed han og var fra Gat - frem fra Filisternes Slagrækker og talte, som han plejede, medens David hørte på det.
\par 24 Da Israels Mænd så Manden, flygtede de alle rædselsslagne for ham.
\par 25 Og Israels Mænd sagde: "Ser I den Mand, som kommer der? Det er for at håne Israel, han kommer; den, som dræber ham, vil Kongen give stor Rigdom; sin Datter vil han give ham, og hans Fædrenehus vil han fritage for Skat i Israel!"
\par 26 David spurgte da de Mænd,som stod om ham: "Hvilken Løn får den, som dræber denne Filister og tager Skammen fra Israel? Thi hvem er vel denne uomskårne Filister, at han vover at håne den levende Guds Slagrækker?"
\par 27 Og Folkene gentog for ham: "Det og det får den, som dræber ham!"
\par 28 Men da hans ældste Broder Eliab hørte ham tale med Mændene, blev han vred på David og sagde: "Hvad vil du her? Og hvem har du overladt de stakkels Får i Ørkenen? Jeg kender dit Overmod og dit Hjertes Ondskab; du kom jo herned for at se på Kampen!"
\par 29 Da sagde David: "Hvad har jeg nu gjort? Det var jo da kun et Spørgsmål!"
\par 30 Og han vendte sig fra ham til en anden og sagde det samme, og Folkene svarede ham som før.
\par 31 Imidlertid rygtedes det, hvad David havde sagt; det kom også Saul for Øre, og han lod ham hente.
\par 32 Da sagde David til Saul: "Min Herre må ikke tabe Modet! Din Træl vil gå hen og kæmpe med den Filister!"
\par 33 Saul svarede David: "Du kan ikke gå hen og kæmpe med den Filister; thi du er en ung Mand, og han har været Kriger fra sin Ungdom!"
\par 34 Men David sagde til Saul: "Din Træl har vogtet sin Faders Små kvæg; og kom der en Løve eller en Bjørn og slæbte et Dyr bort fra Hjorden,
\par 35 løb jeg efter den og slog den og rev det ud af Gabet på den; kastede den sig så over mig, greb jeg den i Skægget og slog den ihjel.
\par 36 Både Løve og Bjørn har din Træl dræbt, og det skal gå denne uomskårne Filister som en at dem; thi han har hånet den levende Guds Slagrækker!"
\par 37 Fremdeles sagde David: "HERREN, som har reddet mig fra Løvers og Bjørnes Vold, vil også redde mig fra denne Filisters Hånd!"Da sagde Saul til David: "Gå! HERREN være med dig!"
\par 38 Saul iførte nu David sin Våbenkjortel, satte en Kobberhjelm på hans Hoved, iførte ham en Brynje
\par 39 og spændte sit Sværd om ham over Våbenkjortelen; men det var forgæves, han søgte at gå dermed, thi han havde aldrig prøvet det før. Da sagde David til Saul: "Jeg kan ikke gå dermed, thi jeg har aldrig prøvet det før!" Og David tog det af.
\par 40 Derpå tog han sin Stav i Hånden og udsøgte sig fem af de glatteste Sten i Flodlejet, lagde dem i sin Hyrdetaske, der tjente ham som Slyngestenstaske, tog sin Slynge i Hånden og gik mod Filisteren.
\par 41 Imidlertid kom Filisteren David nærmere og nærmere med Skjolddrageren foran sig;
\par 42 og da Filisteren så til og fik Øje på David, ringeagtede han ham, fordi han var en ung Mand, rødmosset og smuk at se til.
\par 43 Og Filisteren sagde til David:"Er jeg en Hund, siden du kommer imod mig med en Stav?" Og Filisteren forbandede David ved sin Gud.
\par 44 Derpå sagde Filisteren til David: "Kom herhen, så skal jeg give Himmelens Fugle og Markens vilde Dyr dit Kød!"
\par 45 David svarede Filisteren: "Du kommer imod mig med Sværd og Spyd og kastevåben, men jeg kommer imod dig i Hærskarers HERREs, Israels Slagrækkers Guds, Navn, ham, du har hånet.
\par 46 I Dag giver HERREN dig i min Hånd; jeg skal slå dig ned og hugge Hovedet af dig og i Dag give Himmelens Fugle og Jordens vilde Dyr din og Filisterhærens døde Kroppe, for at hele Jorden kan kende, at der er en Gud i Israel,
\par 47 og for at hele denne Forsamling kan kende, at HERREN ikke giver Sejr ved Sværd eller Spyd; thi HERREN råder for Kampen, og han vil give eder i vor Hånd!"
\par 48 Da Filisteren nu satte sig i Bevægelse og gik nærmere hen imod David, løb David hurtigt hen imod Slagrækken for at møde Filisteren.
\par 49 Og David greb ned i Tasken, tog en Sten af den, slyngede den ud og ramte Filisteren i Panden, så Stenen trængte ind i hans Pande, og han styrtede næsegrus til Jorden.
\par 50 Således fik David Bugt med Filisteren med Slynge og Sten, og han slog Filisteren ihjel, skønt han ikke havde Sværd i Hånden.
\par 51 Så løb David hen ved Siden af Filisteren, greb hans Sværd, drog det af Skeden og gav ham Dødsstødet og huggede Hovedet af ham dermed. Da Filisterne så, at deres Helt var død, flygtede de;
\par 52 men Israels og Judas Mænd satte sig i Bevægelse, opløftede Kampråbet og forfulgte Filisterne lige til Gat og Ekrons Porte, og de faldne Filistere lå på Vejen fra Sja'arajim lige til Gat og Ekron.
\par 53 Derpå vendte Israeliterne tilbage fra Forfølgelsen af Filisterne og plyndrede deres Lejr.
\par 54 Og David tog Filisterens Hoved og bragte det til Jerusalem, men hans Våben lagde han i sit Telt.
\par 55 Da Saul så David gå imod Filisteren, sagde han til Hærføreren Abner: "Hvis Søn er denne unge Mand, Abner?" Abner svarede: "Så sandt du lever, Konge, jeg ved det ikke!"
\par 56 Da sagde Kongen: "Forhør dig om, hvis Søn denne Yngling er!"
\par 57 Da så David vendte tilbage efter at have dræbt Filisteren, tog Abner ham og førte ham frem for Saul, og han havde Filisterens Hoved i Hånden.
\par 58 Saul sagde til ham: "Hvis Søn er du, unge Mand?" David svarede: "Jeg er Søn af din Træl, Betlehemiten Isaj!"

\chapter{18}

\par 1 Efter Davids Samtale med Saul blev Jonatans Sjæl bundet til Davids Sjæl, og han elskede ham som sin egen Sjæl;
\par 2 og Saul tog ham samme dag til sig og tillod ham ikke at vende tilbage til sin Faders Hus.
\par 3 Og Jonatan sluttede Pagt med David, fordi han elskede ham som sin egen Sjæl.
\par 4 Og Jonatan afførte sig sin Kappe og gav David den tillige med sin Våbenkjortel, ja endog sit Sværd, sin Bue og sit Bælte.
\par 5 Og David drog ud; hvor som helst Saul sendte ham hen, havde han Lykken med sig; derfor satte Saul ham over Krigerne, og han vandt Yndest hos alt Folket, endog hos Sauls Folk.
\par 6 Men da de kom hjem, da David vendte tilbage efter at have fældet Filisteren, gik Kvinderne fra alle Israels Byer Saul i Møde med Sang og Dans, med Håndpauker, Jubel og Cymbler,
\par 7 og de dansende Kvinder sang: "Saul slog sine Tusinder, men David sine Titusinder!"
\par 8 Da blev Saul meget vred; disse Ord mishagede ham, og han sagde: "David giver de Titusinder, og mig giver de Tusinder; nu mangler han kun Kongemagten!"
\par 9 Og fra den Dag af så Saul skævt til David.
\par 10 Næste Dag overvældede en ond Ånd fra Gud Saul, så han rasede i Huset, medens David som sædvanligt legede på Strenge; Saul havde sit Spyd i Hånden
\par 11 og kastede det i den Tanke: "Jeg vil spidde David til Væggen!" Men David undveg ham to Gange.
\par 12 Da kom Saul til at frygte David, fordi HERREN var med ham, medens han var veget fra Saul.
\par 13 Derfor fjernede Saul ham fra sig og gjorde ham til Tusindfører; og han drog ud til Kamp og hjem igen i Spidsen for Krigerne;
\par 14 og Lykken fulgte David i alt, hvad han foretog sig; thi HERREN var med ham.
\par 15 Da Saul så, i hvor høj Grad Lykken fulgte ham, gruede han for ham;
\par 16 men hele Israel og Juda elskede David, fordi han drog ud til Kamp og hjem i Spidsen for dem.
\par 17 Da sagde Saul til David: "Se, her er min ældste Datter Merab; hende vil jeg give dig til Hustru, dersom du viser dig som en tapper Mand i min Tjeneste og fører HERRENs Krige!" Saul tænkte nemlig: "Han skal ikke falde for min, men for Filisternes Hånd!"
\par 18 David sagde til Saul: "Hvem er jeg, og hvad er min Familie, min Faders Slægt i Israel, at jeg skulde blive Kongens Svigersøn?"
\par 19 Men da Tiden kom, at Sauls Datter Merab skulde gives David til Ægte, blev hun givet til Adriel fra Mehol
\par 20 Sauls Datter Mikal fattede Kærlighed til David. Det kom Saul for Øre, og han syntes godt derom;
\par 21 Saul tænkte nemlig: "Jeg vil give hende til ham, for at hun kan blive ham en Snare, så han falder for Filisternes Hånd!" Da sagde Saul til David: "I Dag skal du for anden Gang blive min Svigersøn!"
\par 22 Og Saul gav sine Folk Befaling til underhånden at sige til David: "Kongen synes godt om dig, og alle hans Folk elsker dig; så bliv nu Kongens Svigersøn!"
\par 23 Men da Sauls Folk sagde det til David, svarede han: "Synes det eder en ringe Ting at blive Kongens Svigersøn? Jeg er jo en fattig og ringe Mand!"
\par 24 Og Sauls Folk meddelte ham det og sagde: "Det og det sagde David."
\par 25 Da sagde Saul: "Således skal I sige til David: Kongen ønsker ikke andet i Brudekøb end 100 Filisterforhuder, så at han kan få Hævn over sine Fjender!" Saul gjorde nemlig Regning på at få David fældet ved Filisternes Hånd.
\par 26 Da hans Folk fortalte David dette, samtykkede han i at blive Kongens Svigersøn.
\par 27 Derpå brød David op og drog ud med sine Mænd og dræbte 2OO Filistere, og David kom med deres Forhuder og leverede Kongen dem fuldtallige for at blive hans Svigersøn. Så gav Saul ham sin Datter Mikal til Ægte.
\par 28 Men da Saul så og skønnede, at HERREN var med David, og at hele Israel elskede ham,
\par 29 frygtede han David endnu mere, og Saul blev for stedse David fjendsk.
\par 30 Filisternes Høvdinger rykkede i Marken; og hver Gang de rykkede ud, havde David mere Held med sig end alle Sauls Folk, og han vandt stort Ry.

\chapter{19}

\par 1 Saul talte nu med sin Søn Jonatan og alle sine Folk om at slå David ihjel. Men Sauls Søn Jonatan holdt meget af David.
\par 2 Derfor fortalte Jonatan David det og sagde: "Min Fader Saul står dig efter Livet; tag dig derfor i Vare i Morgen, gem dig og hold dig skjult!
\par 3 Men jeg vil gå ud og stille mig hen hos min Fader på Marken der, hvor du er; så vil jeg tale til ham om dig, og hvis jeg mærker noget, vil jeg lade dig det vide."
\par 4 Så talte Jonatan Davids Sag hos sin Fader Saul og sagde til ham: "Kongen forsynde sig ikke mod sin Træl David; thi han har ikke forsyndet sig mod dig, og hvad han har udrettet, har gavnet dig meget;
\par 5 han vovede Livet for at dræbe Filisteren, og HERREN gav hele Israel en stor Sejr. Du så det selv og glædede dig derover; hvorfor vil du da forsynde dig ved uskyldigt Blod og dræbe David uden Grund?"
\par 6 Og Saul lyttede til Jonatans Ord, og Saul svor: "Så sandt HERREN lever, han skal ikke blive dræbt!"
\par 7 Derpå lod Jonatan David hente og fortalte ham det hele; og Jonatan førte David til Saul, og han var om ham som før.
\par 8 Men Krigen fortsattes, og David drog i Kamp mod Filisterne og tilføjede dem et stort Nederlag, så de flygtede for ham.
\par 9 Da kom der en ond Ånd fra HERREN over Saul, og engang han sad i sit Hus med sit Spyd i Hånden, medens David legede på Strengene,
\par 10 søgte Saul at spidde David til Væggen med Spydet; men han veg til Side for Saul, så han jog Spydet i Væggen, medens David flygtede og undslap.
\par 11 Om Natten sendte Saul Folk til Davids Hus for at passe på ham og dræbe ham om Morgenen. Men Davids Hustru Mikal røbede ham det og sagde: "Hvis du ikke redder dit Liv i Nat, er du dødsens i Morgen!"
\par 12 Så hejste Mikal David ned igennem Vinduet, og han flygtede bort og undslap.
\par 13 Derpå tog Mikal Husguden, lagde den i Sengen, bredte et Gedehårsnet over Hovedet på den og dækkede den til med et Tæppe.
\par 14 Da nu Saul sendte Folk hen for at hente David, sagde hun: "Han er syg."
\par 15 Men Saul sendte Sendebudene hen for at se David, idet han sagde: "Bring ham på Sengen op til mig, for at jeg kan dræbe ham!"
\par 16 Da Sendebudene kom derhen, opdagede de, at det var Husguden, der lå i Sengen med Gedehårsnettet over Hovedet.
\par 17 Da sagde Saul til Mikal: "Hvorfor førte du mig således bag Lyset og hjalp min fjende bort, så han undslap?" Mikal svarede Saul:"Han sagde til mig: Hjælp mig bort, ellers slår jeg dig ihjel!"
\par 18 Men David var flygtet og havde bragt sig i Sikkerhed. Derpå gik han til Samuel i Rama og fortalte ham alt, hvad Saul havde gjort imod ham; og han og Samuel gik hen og tog Ophold i Najot.
\par 19 Da nu Saul fik at vide, at David var i Najot i Rama,
\par 20 sendte han Folk ud for at hente David; men da de så Profetskaren i profetisk Henrykkelse og Samuel stående hos dem, kom Guds Ånd over Sauls Sendebud, så at også de faldt i profetisk Henrykkelse.
\par 21 Da Saul hørte det, sendte han andre Folk af Sted; men også de faldt i Henrykkelse. Så sendte Saul på ny, tredje Gang, Folk af Sted; men også de faldt i Henrykkelse.
\par 22 Da begav han sig selv til Rama, og da han kom til Cisternen på Tærskepladsen, som ligger på den nøgne Høj, spurgte han: "Hvor er Samuel og David?" Man svarede: "I Najot i Rama!"
\par 23 Men da han gik derfra til Najot i Rama, kom Guds Ånd også over ham, og han gik i Henrykkelse hele Vejen, lige til han nåede Najot i Rama.
\par 24 Da rev også han sine Klæder af sig, og han var i Henrykkelse foran Samuel og faldt nøgen om og blev liggende således hele den Dag og den følgende Nat. Derfor hedder det: "Er også Saul iblandt Profeterne?"

\chapter{20}

\par 1 Men David flygtede fra Najot i Rama og kom til Jonatan og sagde: "Hvad har jeg gjort? Hvad er min Brøde? Og hvad er min Synd mod din Fader, siden han står mig efter Livet?"
\par 2 Han svarede: "Det være langt fra! Du skal ikke dø! Min Fader foretager sig jo intet, hverken stort eller småt, uden at lade mig det vide; hvorfor skulde min Fader så dølge dette for mig? Der er intet om det!"
\par 3 Men David svarede: "Din Fader ved sikkert, at du har fattet Godhed for mig, og tænker så: Det må Jonatan ikke få at vide, at det ikke skal gøre ham ondt; nej, så sandt HERREN lever, og så sandt du lever, der er kun et Skridt imellem mig og Døden!"
\par 4 Da sagde Jonatan til David: "Alt, hvad du ønsker, vil jeg gøre for dig!"
\par 5 David sagde til Jonatan: "I Morgen er det jo Nymånedag, og jeg skulde sidde til Bords med Kongen; men lad mig gå bort og skjule mig på Marken indtil Aften.
\par 6 Hvis din Fader savner mig, så sig: David har bedt mig om Lov til at skynde sig til Betlehem, sin Fødeby, da hele hans Slægt har sit årlige Slagtoffer der.
\par 7 Hvis han så siger: Godt! er der ingen Fare for din Træl; men bliver han vred, så vid, at han vil min Ulykke.
\par 8 Vis din Træl den Godhed, siden du er gået i Pagt med din Træl for HERRENs Åsyn. Men har jeg forbrudt mig, så slå du mig ihjel; thi hvorfor skulde du bringe mig til din Fader?"
\par 9 Jonatan svarede: "Det være langt fra! Hvis jeg virkelig kommer under Vejr med, at min Fader vil din Ulykke, skulde jeg så ikke lade dig det vide?"
\par 10 Da sagde David til Jonatan: "Men hvem skal lade mig det vide, om din Fader giver dig et hårdt Svar?"
\par 11 Jonatan svarede David: "Kom, lad os gå ud på Marken!" Og de gik begge ud på Marken.
\par 12 Da sagde Jonatan til David: "HERREN, Israels Gud, er Vidne: Jeg vil i Morgen ved denne Tid udforske min Faders Sindelag, og hvis der ingen Fare er for David, skulde jeg da ikke sende dig Bud og lade dig det vide?
\par 13 HERREN ramme Jonatan både med det ene og det andet: Hvis det er min Faders bestemte Vilje at bringe Ulykke over dig, vil jeg lade dig det vide og hjælpe dig bort, så du kan fare i Fred.
\par 14 Og måtte du så, hvis jeg endnu er i Live, måtte du så vise HERRENs Godhed imod mig. Men skulde jeg være død,
\par 15 så unddrag ingen Sinde min Slægt din Godhed. Og når HERREN udrydder hver eneste af Davids Fjender af Jorden,
\par 16 måtte da Jonatans Navn ikke blive udryddet, men bestå sammen med Davids Hus, og måtte HERREN kræve det af Davids Fjenders Hånd!"
\par 17 Da svor Jonatan på ny David en Ed, fordi han elskede ham; thi han elskede ham af hele sin Sjæl.
\par 18 Da sagde Jonatan til ham: "I Morgen er det Nymånedag; da vil du blive savnet, når din Plads står tom;
\par 19 men i Overmorgen vil du blive savnet endnu mere; gå så hen til det Sted, hvor du holdt dig skjult, den Dag Skændselsdåden skulde have fundet Sted, og sæt dig ved Jorddyngen der;
\par 20 i Overmorgen vil jeg så skyde med Pile der, som om jeg skød til Måls.
\par 21 Jeg sender så Drengen hen for at lede efter Pilen, og hvis jeg da siger til ham: Pilen ligger her på denne Side af dig, hent den! så kan du komme; thi da står alt vel til for dig, og der er ingen Fare, så sandt HERREN lever.
\par 22 Men siger jeg til den unge Mand: Pilen ligger på den anden Side af dig, bedre frem! så fly,thi da vil HERREN have dig bort.
\par 23 Men om det, vi to har aftalt sammen, gælder, at HERREN står mellem mig og dig for evigt!"
\par 24 David skjulte sig så ude på Marken. Da Nymånedagen kom,satte Kongen sig til Bords for at spise;
\par 25 Kongen sad på sin vante Plads, på Pladsen ved Væggen, medens Jonatan sad lige overfor og Abner ved Siden af Saul, men Davids Plads stod tom.
\par 26 Saul sagde intet den Dag, thi han tænkte: "Der er vel hændet ham noget, så han ikke er ren, fordi han endnu ikke har renset sig."
\par 27 Men da Davids Plads også stod tom næste Dag, Dagen efter Nymånedagen, sagde Saul til sin Søn Jonatan: "Hvorfor kom Isajs Søn hverken til Måltidet i Går eller i Dag?"
\par 28 Jonatan svarede Saul: "David bad mig om Lov til at gå til Betlehem;
\par 29 han sagde: Lad mig gå, thi vor Slægt har Offerfest der i Byen, og mine Brødre har pålagt mig at komme; hvis du har Godhed for mig, lad mig så få fri, for at jeg kan besøge mine Slægtninge! Det er Grunden til, at han ikke er kommet til Kongens Bord!"
\par 30 Da blussede Sauls Vrede op imod Jonatan, og han sagde til ham; "Du Søn af en vanartet Kvinde! Ved jeg ikke, at du er Ven med Isajs Søn til Skam for dig selv og for din Moders Blusel?
\par 31 Thi så længe Isajs Søn er i Live på Jorden, er hverken du eller dit Kongedømme i Sikkerhed. Send derfor Bud og hent ham til mig, thi han er dødsens!"
\par 32 Jonatan svarede sin Fader Saul: "Hvorfor skal han dræbes? Hvad har han gjort?"
\par 33 Da kastede Saul Spydet efter ham for at ramme ham. Så skønnede Jonatan, at det var hans Faders bestemte Vilje at dræbe David.
\par 34 Og Jonatan rejste sig fra Bordet i heftig Vrede og spiste intet den anden Nymånedag, thi det gjorde ham ondt for David, at hans Fader havde smædet ham.
\par 35 Næste Morgen gik Jonatan fulgt af en dreng ud i Marken, til den Tid han havde aftalt med David.
\par 36 Derpå sagde han til Drengen, han havde med: "Løb hen og led efter den Pil, jeg skyder af!" Medens Drengen løb, skød han Pilen af over hans Hoved,
\par 37 og da Drengen nåede Stedet, hvor Pilen, som Jonatan havde afskudt, lå, råbte Jonatan til ham: "Pilen ligger jo på den anden Side af dig, bedre frem!"
\par 38 Derpå råbte Jonatan til Drengen: "Skynd dig alt, hvad du kan, og bliv ikke stående!" Så tog Jonatans dreng Pilen og bragte sin Herre den.
\par 39 Og Drengen vidste ikke noget, thi kun Jonatan og David kendte Sammenhængen.
\par 40 Jonatan gav derpå sin Dreng Våbnene og sagde til ham: "Tag dem med til Byen!"
\par 41 Da Drengen var gået, rejste David sig fra sit Skjul ved Jorddyngen og faldt til Jorden på sit Ansigt og bøjede sig ned tre Gange. Og de kyssede hinanden og græd bitterlig sammen.
\par 42 Derpå sagde Jonatan til David: "Far i Fred! Om det, vi to har tilsvoret hinanden i HERRENs Navn, gælder, at HERREN står mellem mig og dig, mellem mine og dine Efterkommere for evigt!"

\chapter{21}

\par 1 Så brød David op og drog bort, medens Jonatan gik ind i Byen. David kom derpå til Præsten Ahimelek i Nob. Ahimelek kom ængstelig David i Møde og sagde til ham: "Hvorfor er du alene og har ingen med dig?"
\par 2 David svarede Præsten Ahimelek: "Kongen overdrog mig et Ærinde og sagde til mig: Ingen må vide noget om det Ærinde, jeg sender dig ud i og overdrager dig! Derfor har jeg sat Folkene Stævne på et aftalt Sted.
\par 3 Men hvis du har fem Brød ved Hånden, så giv mig dem, eller hvad du har!"
\par 4 Præsten svarede David: "Jeg har intet almindeligt Brød ved Hånden, kun helligt Brød; Folkene har da vel holdt sig fra Kvinder?"
\par 5 David svarede Præsten: "Ja visselig, vi har været afskåret fra Omgang med Kvinder i flere Dage. Da jeg drog ud, var Folkenes Legemer rene, skønt det var en dagligdags Rejse; hvor meget mere må de da i Dag være rene på Legemet!"
\par 6 Præsten gav ham da det hellige Brød; thi der var ikke andet Brød der end Skuebrødene, som tages bort fra deres Plads for HERRENs Åsyn, samtidig med at der lægges frisk Brød i Stedet.
\par 7 Men den Dag var en Mand af Sauls Folk lukket inde der for HERRENs Åsyn, en Edomit ved Navn Doeg, den øverste af Sauls Hyrder.
\par 8 David spurgte derpå Ahimelek: "Har du ikke et Spyd eller et Sværd ved Hånden her? Thi hverken mit Sværd eller mine andre Våben fik jeg med, da Kongens Ærinde havde Hast."
\par 9 Præsten svarede: "Det Sværd, som tilhørte Filisteren Goliat, ham, som du dræbte i Terebintedalen, er her, hyllet i en Kappe bag Efoden. Vil du have det, så tag det! Thi her er intet andet!" Da sagde David: "Dets Lige findes ikke; giv mig det!"
\par 10 Derpå brød David op og flygtede samme Dag for Saul, og han kom til Kong Akisj af Gat.
\par 11 Men Akisj's Folk sagde til ham: "Er det ikke David, Landets Konge, er det ikke ham, om hvem man sang under Dans: Saul slog sine Tusinder, men David sine Titusinder!"
\par 12 Disse Ord gav David Agt på,og han grebes af stor Frygt for Kong Akisj af Gat;
\par 13 derfor lod han afsindig overfor dem og rasede imellem Hænderne på dem, idet han trommede på Portfløjene og lod sit Spyt flyde ned i Skægget.
\par 14 Da sagde Akisj til sine Folk: "I kan da se, at Manden er gal; hvorfor bringer I ham til mig?
\par 15 Har jeg ikke gale Mennesker nok, siden I bringer mig ham til at plage mig med sin Galskab? Skal han komme i mit Hus?"

\chapter{22}

\par 1 Derpå drog David bort derfra og Redde sig ind i Adullams Hule.
\par 2 Og alle Slags Mennesker, som var i Nød, flokkede sig om ham, forgældede Mennesker og Folk, som var bitre i Hu, og han blev deres Høvding. Henved 4OO Mand sluttede sig til ham.
\par 3 Derfra drog David til Mizpe i Moab og sagde til Moabiternes Konge: "Lad min Fader og min Moder bo hos eder, indtil jeg får at vide, hvad Gud har for med mig!"
\par 4 Han lod dem da tage Ophold hos Moabiternes Konge, og de boede hos ham, al den Tid David var i Klippeborgen.
\par 5 Men Profeten Gad sagde til David: "Du skal ikke blive i Klippeborgen; bryd op og drag til Judas Land!" Så drog David til Ja'ar-Heret.
\par 6 Nu kom dette Saul for Øre, thi David og de Mænd, han havde hos sig, havde vakt Opmærksomhed. Saul sad engang i Gibea under Tamarisken på Højen med sit Spyd i Hånden, omgivet af alle sine Folk.
\par 7 Da sagde Saul til sine folk, som stod hos ham: "Hør dog, I Benjaminiter! Vil Isajs Søn give eder alle sammen Marker og Vingårde eller gøre eder alle til Tusind- og Hundredførere,
\par 8 siden I alle har sammensvoret eder imod mig, og ingen lod mig det vide, da min Søn sluttede Pagt med Isajs Søn? Ingen af eder havde Medfølelse med mig og lod mig vide, at min Søn havde fået min Træl til at optræde som min Fjende, som han nu gør."
\par 9 Da tog Edomiten Doeg, der stod blandt Sauls Folk, Ordet og sagde: "Jeg så, at Isajs Søn kom til Ahimelek, Ahitubs Søn, i Nob,
\par 10 og han rådspurgte HERREN for ham og gav ham Rejsetæring og Filisteren Goliats Sværd!"
\par 11 Da sendte Kongen Bud og lod Præsten Ahimelek, Ahitubs Søn, hente tillige med hele hans Fædrenehus, Præsterne i Nob; og da de alle var kommet til Kongen,
\par 12 sagde Saul: "Hør nu, Ahitubs Søn!" Han svarede: "Ja, Herre!"
\par 13 Da sagde Saul til ham: "Hvorfor sammensvor du og Isajs Søn eder imod mig? Du gav ham jo Brød og Sværd og rådspurgte Gud for ham, så at han kunde optræde som min Fjende, som han nu gør!"
\par 14 Ahimelek svarede Kongen:"Hvem blandt alle dine Folk er så betroet som David? Han er jo Kongens Svigersøn, Øverste for din Livvagt og højt æret i dit Hus?
\par 15 Er det første Gang, jeg har rådspurgt Gud for ham? Det være langt fra! Kongen må ikke lægge sin Træl eller hele mit Fædrenehus noget til Last, thi din Træl kendte ikke det mindste til noget af dette!"
\par 16 Men Kongen sagde: "Du skal dø, Ahimelek, du og hele dit Fædrenehus!"
\par 17 Og Kongen sagde til Vagten, som stod hos ham: "Træd frem og dræb HERRENs Præster; thi de står også i Ledtog med David, og skønt de vidste, at han var på Flugt, gav de mig ikke Underretning derom!" Men Kongens Folk vilde ikke lægge Hånd på HERRENs Præster.
\par 18 Da sagde Kongen til Doeg: "Træd du så frem og stød Præsterne ned!" Da trådte Edomiten Doeg frem og stødte Præsterne ned; han dræbte den Dag femogfirsindstyve Mænd, som bar Efod.
\par 19 Og Nob, Præsternes By, lod Kongen hugge ned med Sværdet. Mænd og Kvinder, Børn og diende, Hornkvæg, Æsler og Får.
\par 20 Kun een af Ahimeleks, Ahitubs Søns, Sønner ved Navn Ebjatar undslap og flygtede til David.
\par 21 Og Ebjatar fortalte David, at Saul havde dræbt HERRENs Præster.
\par 22 Da sagde David til Ebjatar: "Jeg vidste dengang, at når Edomiten Doeg var der, vilde han give Saul Underretning derom! Jeg bærer Skylden for hele dit Fædrenehus's Død.
\par 23 Bliv hos mig, frygt ikke! Den, som står dig efter Livet, står mig efter Livet, thi du står under min Varetægt."

\chapter{23}

\par 1 Da fik David at vide, at Filisterne belejrede Ke'ila og plyndrede Tærskepladserne.
\par 2 Og David rådspurgte HERREN: "Skal jeg drage hen og slå Filisterne der?" HERREN svarede David; "Drag hen og slå Filisterne og befri Ke'ila!"
\par 3 Men Davids Mænd sagde til ham: "Se, vi lever i stadig Frygt her i Juda; kan der så være Tale om, at vi skal drage til Ke'ila mod Filisternes Slagrækker?"
\par 4 Da rådspurgte David på ny HERREN, og HERREN svarede ham: "Drag ned til Ke'ila, thi jeg giver Filisterne i din Hånd!"
\par 5 David og hans Mænd drog da til Ke'ila, angreb Filisterne, bortførte deres Kvæg og tilføjede dem et stort Nederlag. Således befriede David Ke'ilas Indbyggere.
\par 6 Dengang Ebjatar, Ahimeleks Søn, flygtede til David - han drog med David ned til Ke'ila - havde han Efoden med.
\par 7 Da Saul fik at vide, at David var kommet til Ke'ila, sagde han: "Gud har givet ham i min Hånd! Thi han lukkede sig selv inde, da han gik ind i en By med Porte og Slåer."
\par 8 Derfor stævnede Saul hele Folket sammen for at drage ned til Ke'ila og omringe David og hans Mænd.
\par 9 Da David hørte, at Saul pønsede på ondt imod ham, sagde han til Præsten Ebjatar: "Bring Efoden hid!"
\par 10 Derpå sagde David: "HERRE, Israels Gud! Din Tjener har hørt, at Saul har i Sinde at gå mod Ke'ila og ødelægge Byen for min Skyld.
\par 11 Vil Folkene i Ke'ila overgive mig i Sauls Hånd? Vil Saul drage herned, som din Tjener har hørt? HERRE, Israels Gud, kundgør din Tjener det!" HERREN svarede: "Ja, han vil!"
\par 12 Så spurgte David: "Vil Folkene i Ke'ila overgive mig og mine Mænd til Saul?" HERREN svarede: "Ja, de vil!"
\par 13 Da brød David op med sine Mænd, henved 600 i Tal, og de drog bort fra Ke'ila og flakkede om fra Sted til Sted. Men da Saul fik at vide, at David var sluppet bort fra Ke'ila, opgav han sit Togt.
\par 14 Nu opholdt David sig i Ørkenen på Klippehøjderne og i Bjergene i Zifs Ørken. Og Saul efterstræbte ham hele tiden, men Gud gav ham ikke i hans Hånd.
\par 15 Og David så, at Saul var draget ud for at stå ham efter Livet.
\par 16 begav Sauls Søn Jonatan sig til David i Horesj og styrkede hans Kraft i Gud,
\par 17 idet han sagde til ham: "Frygt ikke! Min Fader Sauls Arm skal ikke nå dig. Du bliver Konge over Israel og jeg den næste efter dig; det ved min Fader Saul også!"
\par 18 Derpå indgik de to en Pagt for HERRENs Åsyn, og David blev i Horesj, medens Jonatan drog hjem.
\par 19 Men nogle Zifiter gik op til Saul i Gibea og sagde: "David holder sig skjult hos os på Klippehøjderne ved Horesj i Gibeat-Hakila sønden for Jesjimon.
\par 20 Så kom nu herned, Konge, som du længe har ønsket; det skal da være vor Sag at overgive ham til Kongen!"
\par 21 Saul svarede: "HERREN velsigne eder, fordi l har Medfølelse med mig!
\par 22 Gå nu hen og pas fremdeles på og opspor, hvor han kommer hen på sin ilsomme Færd; thi man har sagt mig, at han er meget snu.
\par 23 Opspor alle de Skjulesteder, hvor han gemmer sig, og vend tilbage til mig med pålidelig Underretning; så vil jeg følge med eder, og hvis han er i Landet, skal jeg opsøge ham iblandt alle Judas Tusinder!"
\par 24 Da brød de op og drog forud for Saul til Zif. Men David var dengang med sine Mænd i Maons Ørken i Lavningen sønden for Jesjimon.
\par 25 Så drog Saul og hans Mænd ud for at opsøge ham, og da David kom under Vejr dermed, drog han ned til den Klippe, som ligger i Maons Ørken; men da det kom Saul for Øre, fulgte han efter David i Maons Ørken.
\par 26 Saul gik med sine Mænd på den ene Side af Bjerget, medens David med sine Mænd var på den anden, og David fik travlt med at slippe bort fra Saul. Men som Saul og hans Mænd var ved at omringe og gribe David og hans Mænd,
\par 27 kom der et Sendebud og sagde til Saul: "Skynd dig og kom! Filisterne har gjort Indfald i Landet!"
\par 28 Saul opgav da at forfølge David og drog mod Filisterne. Derfor kalder man det Sted Malekots Klippe.

\chapter{24}

\par 1 Derpå drog David op til Klippehøjderne ved En-Gedi og opholdt sig der.
\par 2 Da Saul kom tilbage fra Forfølgelsen af Filisterne, blev det meldt ham, at David var i En-Gedis Ørken.
\par 3 Så tog Saul 3000 Krigere, udsøgte af hele Israel, og drog ud for at søge efter David og hans Mænd østen for Stenbukke klipperne.
\par 4 Og han kom til Fårefoldene ved Vejen. Der var en Hule, og Saul gik derind for at tildække sine Fødder. Men David og hans Mænd lå inderst i Hulen.
\par 5 Da sagde Davids Mænd til ham: "Se, nu er den Dag kommet, HERREN havde for Øje, da han sagde til dig: Se, jeg giver din Fjende i din Hånd, så du kan gøre med ham, hvad du finder for godt!"
\par 6 Men han svarede sine Mænd: "HERREN lade det være langt fra mig! Slig en Gerning gør jeg ikke mod min Herre, jeg lægger ikke Hånd på HERRENs Salvede; thi HERRENs Salvede er han!" Og David satte sine Mænd strengt i Rette og tillod dem ikke at overfalde Saul.
\par 7 Da stod David op og skar ubemærket Fligen af Sauls Kappe. Men bagefter slog Samvittigheden David, fordi han havde skåret Sauls kappeflig af.
\par 8 Da nu Saul rejste sig og forlod Hulen for at drage videre,
\par 9 stod David op bagefter, gik ud af Hulen og råbte efter Saul: "Herre Konge!" Og da Saul så sig tilbage, kastede David sig ned med Ansigtet mod Jorden og bøjede sig for ham.
\par 10 Og David sagde til Saul: "Hvorfor lytter du til, hvad folk siger: Se, David har ondt i Sinde imod dig?
\par 11 I Dag har du dog med egne Øjne set, at HERREN gav dig i min Hånd inde i Hulen; og dog vilde jeg ikke dræbe dig, men skånede dig og sagde: Jeg vil ikke lægge Hånd på min Herre, thi han er HERRENs Salvede!
\par 12 Og se, Fader, se, her har jeg Fligen af din kappe i min Hånd! Når jeg skar din Kappeflig af og ikke dræbte dig, så indse dog, at jeg ikke har haft noget ondt eller nogen Forbrydelse i Sinde eller har forsyndet mig imod dig, skønt du lurer på mig for at tage mit Liv.
\par 13 HERREN skal dømme mig og dig imellem, og HERREN skal give mig Hævn over dig; men min Hånd skal ikke være imod dig!
\par 14 Som det gamle Ord siger: Fra de gudløse kommer Gudløshed! Men min Hånd skal ikke være imod dig.
\par 15 Hvem er det, Israels Konge er draget ud efter, hvem er det, du forfølger? En død Hund, en Loppe!
\par 16 Men HERREN skal være Dommer og dømme mig og dig imellem; han skal se til og føre min Sag og skaffe mig Ret over for dig!"
\par 17 Da David havde talt disse Ord til Saul, sagde Saul: "Er det din Røst, min Søn David?" Og Saul brast i Gråd
\par 18 og sagde til David: "Du er retfærdigere end jeg; thi du har gjort mig godt, medens jeg har gjort dig ondt,
\par 19 og du har i Dag vist mig stor Godhed, siden du ikke dræbte mig, da HERREN gav mig i din Hånd.
\par 20 Hvem træffer vel sin Fjende og lader ham gå i Fred? HERREN gengælde dig det gode, du har øvet imod mig i Dag!
\par 21 Se, jeg ved, at du bliver Konge, og at Kongedømmet over Israel skal blive i din Hånd;
\par 22 så tilsværg mig nu ved HERREN, at du ikke vil udrydde mine Efterkommere efter mig eller udslette mit Navn af mit Fædrenehus!"
\par 23 Det tilsvor David Saul, hvorefter Saul drog hjem, medens David og hans Mænd gik op i Klippeborgen.

\chapter{25}

\par 1 Da Samuel var død, samledes hele Israel og holdt Klage over ham; og man jordede ham i hans Hjem i Rama. Derpå brød David op og drog ned til Maons Ørken.
\par 2 I Maon boede en Mand, som havde sin Bedrift i Karmel; denne Mand var hovedrig, han havde 3000 får og 1000 Geder; og han var netop i Karmel til Fåreklipning.
\par 3 Manden hed Nabal, hans Hustru Abigajil; hun var en klog og smuk Kvinde, medens Manden var hård og ond. Han var Kalebit.
\par 4 Da David ude i Ørkenen hørte, at Nabal havde Fåreklipning,
\par 5 sendte han ti af sine Folk af Sted og sagde til dem: "Gå op til Karmel, og når l kommer til Nabal, så hils ham fra mig
\par 6 og sig til min Broder: Fred være med dig, Fred være med dit Hus, og Fred være med alt, hvad dit er!
\par 7 Jeg har hørt, at du har Fåreklipning. Nu har dine Hyrder opholdt sig hos os; vi har ikke fornærmet dem, og de har intet mistet, i al den Tid de har været i Karmel;
\par 8 spørg kun dine Folk, så skal de fortælle dig det. Fat Godhed for Folkene! Vi kommer jo til en Festdag; giv dine Trælle og din Søn David, hvad du vil unde os!"
\par 9 Davids Folk kom hen og sagde alt dette til Nabal fra David og biede så på Svar.
\par 10 Men Nabal svarede Davids Folk: "Hvem er David, hvem er Isajs Søn? Nu til Dags er der så mange Trælle, der løber fra deres Herre.
\par 11 Skulde jeg tage mit Brød, min Vin og mit Slagtekvæg, som jeg har slagtet til mine Fåreklippere, og give det til Mænd, jeg ikke ved, hvor er fra?"
\par 12 Så begav Davids Folk sig på Hjemvejen, og da de kom tilbage, fortalte de ham det hele:
\par 13 Da sagde David til sine Folk:"Spænd alle eders Sværd ved Lænd!" Da spændte de alle deres Sværd om, også David. Og henved 400 Mand fulgte David, medens 200 blev ved Trosset.
\par 14 Men en af Folkene fortalte Nabals Hustru Abigajil det og sagde: "David sendte Bud fra Ørkenen for at hilse på vor Herre; men han overfusede dem,
\par 15 skønt de Mænd har været meget gode mod os og ikke fornærmet os, og vi har intet mistet, i al den Tid vi færdedes sammen med dem, da vi var ude i Marken.
\par 16 De var en Mur om os både Nat og Dag, i al den Tid vi vogtede Småkvæget i Nærheden af dem.
\par 17 Se nu til, hvad du vil gøre, thi Ulykken hænger over Hovedet på vor Herre og hele hans Hus; han selv er jo en Usling, man ikke kan tale med!"
\par 18 Så gik Abigajil straks hen og tog 200 Brød, to Dunke Vin, fem tillavede Får, fem Sea ristet Korn, 100 Rosinkager og 200 Figenkager, lagde det på Æslerne
\par 19 og sagde til sine Karle: "Gå i Forvejen, jeg kommer bagefter!" Men sin Mand Nabal sagde hun intet derom.
\par 20 Som hun nu på sit Æsel red ned ad Vejen i Skjul af Bjerget, kom David og hans Mænd ned imod hende, så hun mødte dem.
\par 21 Men David havde sagt: "Det er slet ingen Nytte til, at jeg i Ørkenen har værnet om alt, hvad den Mand ejede, så intet deraf gik tabt; han har gengældt mig godt med ondt.
\par 22 Gud ramme David både med det ene og det andet, om jeg levner noget mandligt Væsen af alt, hvad hans er, til Morgenens Frembrud!"
\par 23 Da Abigajil fik Øje på David, sprang hun straks af Æselet og kastede sig ned for David på sit Ansigt, bøjede sig til Jorden,
\par 24 faldt ned for hans Fødder og sagde: "Skylden er min, Herre! Lad din Trælkvinde tale til dig og hør din Trælkvindes Ord!
\par 25 Min Herre må ikke regne med den Usling fil Nabal! Han svarer til sit Navn; Nabal hedder han, og fuld af Dårskab er han; men jeg, din Trælkvinde, så ikke min Herres Folk, som du sendte hid.
\par 26 Men nu, min Herre, så sandt HERREN lever, og så sandt du lever, du, hvem HERREN har holdt fra at pådrage dig Blodskyld og tage dig selv til Rette: Måtte det gå dine Fjender og dem, som pønser på ondt mod min Herre, som Nabal!
\par 27 Lad nu Folkene, som følger min Herre, få denne Gave, som din Trælkvinde bringer min Herre.
\par 28 Tilgiv dog din Trælkvinde hendes Brøde; thi HERREN vil visselig bygge min Herre et Hus, som skal stå, eftersom min Herre fører HERRENs Krige, og der ikke har været noget ondt at finde hos dig, så længe du har levet.
\par 29 Og skulde nogen rejse sig for at forfølge dig og stå dig efter Livet, måtte da min Herres Liv være bundet i de levendes Knippe hos HERREN din Gud; men dine Fjenders Liv slynge han bort med Slyngeskålen!
\par 30 Når HERREN så for min Herre opfylder alt det gode, han lovede dig, og sætter dig til Fyrste over Israel,
\par 31 da får du ikke dette at bebrejde dig selv, og min Herre får ikke Samvittighedsnag af, at han uden Grund udgød Blod, og at min Herre tog sig selv til Rette. Og når HERREN gør vel imod min Herre, kom da din Trælkvinde i Hu!"
\par 32 Da sagde David til Abigajil: "Priset være HERREN, Israels Gud, som i Dag sendte mig dig i Møde,
\par 33 priset være din Klogskab, og priset være du selv, som i bag holdt mig fra at pådrage mig Blodskyld og tage mig selv til Rette!
\par 34 Men så sandt HERREN, Israels Gud, lever, som holdt mig fra at gøre dig Men: Hvis du ikke var ilet mig i Møde, var ikke et mandligt Væsen levnet Nabal til i Morgen!"
\par 35 Og David modtog af hende, hvad hun havde bragt ham, og sagde til hende: "Gå op til dit Hus i Fred! Jeg har lånt dig Øre og opfyldt dit Ønske."
\par 36 Da Abigajil kom hjem til Nabal, holdt han netop i sit Hus et Gæstebud som en Konges; og da han var glad og stærkt beruset, sagde hun ham ikke det mindste, før det dagedes.
\par 37 Men om Morgenen, da Nabals Rus var ovre, fortalte hans Hustru ham Sagen. Da lammedes Hjertet i hans Bryst, og han blev som Sten;
\par 38 og en halv Snes Dage efter slog HERREN Nabal, så han døde.
\par 39 Da David fik at vide, at Nabal var død, sagde han: "Lovet være HERREN, som har hævnet den Krænkelse, Nabal tilføjede mig, og holdt sin Tjener fra at gøre ondt; HERREN har ladet Nabals Ondskab falde tilbage på hans eget Hoved!" Derpå sendte David Bud og bejlede til Abigajil.
\par 40 Og da Davids Trælle kom til Abigajil i Karmel, talte de således til hende: "David har sendt os til dig for at bejle til dig!"
\par 41 Da rejste hun sig, bøjede sig med Ansigtet mod Jorden og sagde: "Din Tjenerinde er rede til at blive min Herres Trælkvinde og tvætte hans Trælles Fødder!"
\par 42 Så stod Abigajil hastigt op og satte sig på sit Æsel, og hendes fem Piger ledsagede hende; og hun fulgte med Davids Sendebud og blev hans Hustru.
\par 43 Desuden havde David ægtet Ahinoam fra Jizre'el. Således blev de begge to hans Hustruer.
\par 44 Men Saul gav sin Datter Mikal, Davids Hustru, til Palti, Lajisj's Søn, fra Gallim.

\chapter{26}

\par 1 Zifiterne kom til Saul i Gibea og sagde: "Mon ikke David holder sig skjult i Gibeat-Hakila over for Jesjimon!"
\par 2 Da brød Saul op og drog ned til Zifs Ørken med 3000 udsøgte Mænd af Israel for at søge efter David i Zifs Ørken;
\par 3 og han slog Lejr i Gibeat-Hakila østen for Jesjimon ved Vejen, medens David opholdt sig i Ørkenen. Da David erfarede, at Saul var draget ind i Ørkenen for at forfølge ham,
\par 4 udsendte han Spejdere og fik at vide, at Saul var kommet til Nakon.
\par 5 Da stod David op og begav sig til det Sted, hvor Saul havde lejret sig, og David fik Øje på det Sted, hvor Saul og hans Hærfører Abner, Ners Søn, lå; det var i Vognborgen, Saul lå, og hans Folk var lejret rundt om ham.
\par 6 Og David tog til Orde og sagde til Hetiten Ahimelek og til Joabs Broder Abisjaj, Zerujas Søn: "Hvem vil følge mig ned til Saul i Lejren?" Abisjaj svarede: "Det vil jeg!"
\par 7 Så kom David og Abisjaj om Natten til Hæren, og se, Saul lå og sov i Vognborgen med sit Spyd stukket i Jorden ved sit Hovedgærde, medens Abner og Krigerne lå rundt om ham.
\par 8 Da sagde Abisjaj til David: "Gud har i Dag givet din Fjende i din Hånd! Lad mig nagle ham til Jorden med hans Spyd, så jeg ikke skal behøve at gøre det om!"
\par 9 Men David svarede Abisjaj: "Gør ham ikke noget ondt! Thi hvem lægger ustraffet Hånd på HERRENs Salvede?"
\par 10 Og David sagde endvidere: "Nej, så sandt HERREN lever, HERREN selv vil ramme ham; hans Time kommer, eller han vil blive revet bort, når han drager i Krigen.
\par 11 HERREN lade det være langt fra mig at lægge Hånd på HERRENs Salvede! Men tag nu Spydet ved hans Hovedgærde og Vandkrukken, og lad os så gå vor Vej!"
\par 12 Så tog David Spydet og Vandkrukken fra Sauls Hovedgærde, og de gik deres Vej, uden at nogen så eller mærkede det eller vågnede; thi de sov alle, eftersom en tung Søvn fra HERREN var faldet over dem.
\par 13 Derpå gik David over på den anden Side og stillede sig langt borte på Toppen af Bjerget, så at der var langt imellem dem.
\par 14 Så råbte han til Krigerne og Abner, Ners Søn: "Svarer du ikke, Abner?" Abner svarede: "Hvem er det, som kalder på Kongen?"
\par 15 David sagde til Abner: "Er du ikke en Mand? Og hvem er din Lige i Israel? Hvorfor vogtede du da ikke din Herre Kongen? Thi en af Krigerne kom for at gøre din Herre Kongen Men.
\par 16 Der har du ikke båret dig vel ad! Så sandt HERREN lever: I er dødsens, I, som ikke vogtede eders Herre, HERRENs Salvede! Se nu efter: Hvor er Kongens Spyd og Vandkrukken, som stod ved hans Hovedgærde?"
\par 17 Da kendte Saul Davids Røst, og han sagde: "Er det din Røst, min Søn David?" David svarede: "Ja, Herre Konge!"
\par 18 Og han føjede til: "Hvorfor forfølger min Herre dog sin Træl? Hvad har jeg gjort, og hvad ondt har jeg øvet?
\par 19 Måtte min Herre Kongen nu høre sin Træls Ord! Hvis det er HERREN, der har ægget dig imod mig, så lad ham få Duften af en Offergave. Men er det Mennesker,da være de forbandet for HERRENs Åsyn, fordi de nu har drevet mig bort, så at jeg er udelokket fra HERRENs Arvelod, og fordi de har sagt til mig: Gå bort og dyrk fremmede Guder!
\par 20 Nu beder jeg: Lad ikke mit Blod væde Jorden fjernt fra HERRENs Åsyn! Israels Konge er jo draget ud for at stå mig efter Livet, som når Ørnen jager en Agerhøne på Bjergene!"
\par 21 Da sagde Saul: "Jeg har syndet; kom tilbage, min Søn David! Thi jeg vil ikke mer gøre dig noget ondt, eftersom mit Liv i Dag var dyrebart i dine Øjne. Se, jeg har handlet som en Dåre og gjort mig skyldig i en såre stor Vildfarelse!
\par 22 David svarede: "Se, her er Kongens Spyd; lad en af Folkene komme herover og hente det.
\par 23 Men HERREN vil gengælde enhver hans Retfærdighed og Troskab; HERREN gav dig i Dag i min Hånd, men jeg vilde ikke lægge Hånd på HERRENs Salvede!
\par 24 Men som dit Liv i Dag var agtet højt i mine Øjne, måtte således mit Liv være agtet højt i HERRENs Øjne, så at han frier mig fra al Nød!"
\par 25 Da sagde Saul til David: "Velsignet være du, min Søn David! For dig lykkes alt, hvad du tager dig for!" Derpå gik David sin Vej, og Saul vendte tilbage til sit Hjem.

\chapter{27}

\par 1 Men David sagde til sig selv: "Jeg falder dog en skønne Dag for Sauls Hånd. Jeg har ingen anden udvej end at søge Tilflugt i Filisternes Land; så opgiver Saul at søge efter mig nogetsteds i Israels Land, og jeg er uden for hans Rækkevidde!"
\par 2 David brød da op og drog med sine 600 Mænd over til Maoks Søn, Kong Akisj af Gat;
\par 3 og David boede hos Akisj i Gat tillige med sine Mænd, som havde deres Familier med, ligesom David havde sine to Hustruer med, Ahinoam fra Jizre'el og Abigajil, Karmeliten Nabals Hustru.
\par 4 Da Saul fik at vide, at David var flygtet til Gaf, holdt han op at søge efter ham.
\par 5 Men David sagde til Akisj: "Hvis jeg har fundet Nåde for dine Øjne, lad mig så få et Sted at bo i en af Byerne ude i Landet, thi hvorfor skal din Træl bo hos dig i Hovedstaden?"
\par 6 Akisj lod ham da med det samme få Ziklag; og derfor tilhører Ziklag endnu den Dag i Dag Judas Konger.
\par 7 Den Tid, David boede i Filisternes Land, udgjorde et År og fire Måneder.
\par 8 Og David og hans Mænd drog op og plyndrede hos Gesjuriterne, Gizriterne og Amalekiterne; thi de boede i Landet fra Telam hen imod Sjur og hen til Ægypten;
\par 9 og når David plyndrede Landet, lod han hverken Mænd eller Kvinder blive i Live, men Småkvæg, Hornkvæg, Æsler, Kameler og Klæder tog han med; når han så vendte tilbage og kom til Akisj,
\par 10 og han spurgte: "Hvor hærgede I denne Gang?" svarede David: "Idet judæiske Sydland!" eller: "I det jerame'elitiske Sydland!" eller: "I det kenitiske Sydland!"
\par 11 David lod ingen Mand eller Kvinde blive i Live for ikke at måtte tage dem med til Gat; thi han tænkte: "De kunde røbe os og sige: Det og det har David gjort!" Således bar han sig ad, al den Tid han opholdt sig i Filisternes Land.
\par 12 Derfor fik Akisj Tillid til David, idet han tænkte: "Han har gjort sig grundig forhadt hos sit Folk Israel; han vil tjene mig for stedse!"

\chapter{28}

\par 1 På den Tid samlede Filisterne deres Hær til Kamp for at angribe Israel. Da sagde Akisj til David: "Det må du vide, at du og dine Mænd skal følge med mig i Hæren!"
\par 2 David svarede Akisj: "Godt, så skal du også få at se, hvad din Træl kan udrette!" Da sagde Akisj til David: "Godt, så sætter jeg dig til hele Tiden at vogte mit Liv!"
\par 3 Samuel var død, og hele Israel havde holdt Klage over ham, og han var blevet jordet i Rama, hans By. Og Saul havde udryddet Dødemanerne og Besværgerne af Landet.
\par 4 Imidlertid samlede Filisterne sig og kom og slog Lejr i Sjunem, og Saul samlede hele Israel, og de slog Lejr på Gilboa.
\par 5 Da Saul så Filisternes Hær, grebes han af Frygt og blev såre forfærdet.
\par 6 Da rådspurgte Saul HERREN; men HERREN svarede ham ikke, hverken ved Drømme eller ved Urim eller ved Profeterne.
\par 7 Saul sagde derfor til sine Folk: "Opsøg mig en Kvinde, som kan mane; så vil jeg gå til hende , og rådspørge hende!" Hans Folk svarede ham: "I En-Dor er der en Kvinde, som kan mane!"
\par 8 Da gjorde Saul sig ukendelig og tog andre Klæder på og gav sig på Vej, fulgt af to Mænd. Da de om Natten kom til Kvinden, sagde han: "Spå mig ved en Genfærdsånd og man mig den, jeg siger dig, frem!"
\par 9 Kvinden svarede ham: "Du ved jo, hvad Saul har gjort, hvorledes han har udryddet Dødemanerne og Besværgerne af Landet. Hvorfor vil du da lægge Snare for mit Liv og volde min Død?"
\par 10 Da tilsvor Saul hende ved HERREN: "Så sandt HERREN lever, skal intet lægges dig til Last i denne Sag!"
\par 11 Så sagde Kvinden: "Hvem skal jeg da mane dig frem?" Han svarede: "Man mig Samuel frem!"
\par 12 Kvinden så da Samuel og udstødte et højt Skrig; og Kvinden sagde til Saul: "Hvorfor har du ført mig bag Lyset? Du er jo Saul!"
\par 13 Da sagde Kongen til hende: "Frygt ikke! Men hvad ser du?" Kvinden svarede Saul: "Jeg ser en Ånd stige op af Jorden!"
\par 14 Han sagde atter til hende: "Hvorledes ser han ud?" Hun svarede: "En gammel Mand stiger op, hyllet i en Kappe!" Da skønnede Saul, at det var Samuel, og han kastede sig med Ansigtet til Jorden og bøjede sig.
\par 15 Men Samuel sagde til Saul: "Hvorfor har du forstyrret min Ro og kaldt mig frem?" Saul svarede: "Jeg er i stor Vånde; Filisterne angriber mig, og Gud har forladt mig og svarer mig ikke mere, hverken ved Profeterne eller ved Drømme. Derfor lod jeg dig kalde, for at du skal sige mig, hvad jeg skal gribe til."
\par 16 Da sagde Samuel: "Hvorfor spørger du mig, når HERREN har forladt dig og er blevet din Fjende?
\par 17 HERREN har handlet imod dig, som han kundgjorde ved mig, og HERREN har revet Kongedømmet ud af din Hånd og givet din Medbejler David det.
\par 18 Eftersom du ikke adlød HERREN og ikke lod hans glødende Vrede ramme Amalek, så har HERREN nu voldet dig dette;
\par 19 HERREN vil også give Israel i Filisternes Hånd sammen med dig! I Morgen skal både du og dine Sønner falde; også Israels Hær vil HERREN give i Filisternes Hånd!"
\par 20 Da faldt Saul bestyrtet til Jorden, så lang han var, rædselsslagen over Samuels Ord; han var også ganske afkræftet, da han Døgnet igennem intet havde spist.
\par 21 Kvinden kom nu hen til Saul, og da hun så, at han var ude af sig selv af Skræk, sagde hun til ham: "Se, din Trælkvinde adlød dig; jeg satte mit Liv på Spil og adlød de Ord, du talte til mig.
\par 22 Så adlyd du nu også din Trælkvinde; lad mig sætte et Stykke Brød frem for dig; spis det, for at du kan være ved Kræfter, når du går bort!"
\par 23 Men han værgede sig og sagde: "Jeg kan ikke spise!" Men da både hans Mænd og Kvinden nødte ham, gav han efter for dem, rejste sig fra Jorden og satte sig på Lejet.
\par 24 Kvinden havde en Fedekalv i Huset, den skyndte hun sig at slagte; derpå tog hun Mel, æltede det og bagte usyret Brød deraf.
\par 25 Så satte hun det frem for Saul og hans Mænd; og da de havde spist, stod de op og gik bort samme Nat.

\chapter{29}

\par 1 Filisterne samlede hele deres Hær i Afek, medens Israel havde slået lejr om Kilden ved Jizre'el.
\par 2 Og Filisternes Fyrster rykkede frem med deres Hundreder og Tusinder, og sidst kom David og hans Mænd sammen med Akisj.
\par 3 Da sagde Filisternes Høvdinger: "Hvad skal de Hebræere her?" Akisj svarede: "Det er jo David, Kong Saul af Israels Tjener, som nu allerede har været hos mig et Par År, og jeg har ikke opdaget noget mistænkeligt hos ham, siden han gik over til mig."
\par 4 Men Filisternes Høvdinger blev vrede på ham og sagde: "Send den Mand tilbage til det Sted, du har anvist ham. Han må ikke drage i Kamp med os, for at han ikke skal vende sig imod os under Slaget; thi hvorledes kan denne Mand bedre vinde sin Herres Gunst end med disse Mænds Hoveder?
\par 5 Det var jo David, om hvem man sang under Dans: Saul slog sine Tusinder, men David sine Titusinder!"
\par 6 Da lod Akisj David kalde og sagde til ham: "Så sandt HERREN lever: Du er redelig, og jeg er vel tilfreds med, at du går ud og ind hos mig i Lejren, thi jeg har ikke opdaget noget mistænkeligt hos dig, siden du kom til mig; men Fyrsterne er ikke glade for dig.
\par 7 Vend nu derfor tilbage og gå bort i Fred, for at du ikke skal gøre noget, som mishager Filisternes Fyrster!"
\par 8 Da sagde David til Akisj: "Hvad har jeg gjort, og hvad har du opdaget hos din Træl, fra den Dag jeg trådte i din Tjeneste, siden jeg ikke må drage hen og kæmpe mod min Herre Kongens Fjender?"
\par 9 Akisj svarede David: "Du ved, at du er mig kær som en Guds Engel, men Filisternes Høvdinger siger: Han må ikke drage med os i Kampen!
\par 10 Gør dig derfor rede i Morgen tidlig tillige med din Herres Folk, som har fulgt dig, og gå til det Sted, jeg har anvist eder; tænk ikke ilde om mig, thi du er mig kær; gør eder rede i Morgen tidlig og drag af Sted, så snart det bliver lyst!"
\par 11 David og hans Mænd begav sig da tidligt næste Morgen på Hjemvejen til Filisternes Land, medens Filisterne drog op til Jizre'el.

\chapter{30}

\par 1 Da David og hans Mænd Tredjedagen efter kom til Ziklag, var Amalekiterne faldet ind i Sydlandet og Ziklag, og de havde indtaget Ziklag og stukket det i Brand;
\par 2 Kvinderne og alle, som var der, små og store, havde de taget til Fange; de havde ingen dræbt, men ført dem med sig, da de drog bort.
\par 3 Da David og hans Mænd kom til Byen, se, da var den nedbrændt og deres Hustruer, Sønner og Døtre taget til Fange.
\par 4 Da brast David og hans Krigere i lydelig Gråd, og de græd, til de ikke kunde mere.
\par 5 Også Davids to Hustruer Ahinoam fra Jizre'el og Abigajil, Karmeliten Nabals Hustru, var taget til Fange.
\par 6 Og David kom i stor Vånde, thi Folkene tænkte på at stene ham, da de alle græmmede sig over deres Sønner eller Døtre. Men David søgte Styrke hos HERREN sin Gud;
\par 7 og David sagde til Præsten Ebjatar, Ahimeleks Søn: "Bring mig Efoden hid!" Og Ebjatar bragte David den.
\par 8 Da rådspurgte David HERREN: "Skal jeg sætte efter denne Røverskare? Kan jeg indhente den?" Han svarede: "Sæt efter den; thi du skal indhente den og bringe Redning!"
\par 9 Så drog David af Sted med de 600 Mand, som var hos ham, og de kom til Besorbækken, hvor de, som skulde lades tilbage, blev stående;
\par 10 men David begyndte Forfølgelsen med 400 Mand, medens 200 Mand, som var for udmattede til at gå over Besorbækken, blev tilbage.
\par 11 Og de fandt en Ægypter liggende på Marken; ham tog de med til David og gav ham Brød at spise og Vand at drikke;
\par 12 desuden gav de ham en Figenkage og to Rosinkager. Da han havde spist, kom han til Kræfter; - thi han havde hverken spist eller drukket i hele tre Døgn.
\par 13 David spurgte ham da: "Hvem tilhører du, og hvor er du fra?" Han svarede: "Jeg er en ung Ægypter, Træl hos en Amalekit; min Herre efterlod mig her, da jeg for tre Dage siden blev syg.
\par 14 Vi gjorde Indfald i det kretiske Sydland, i Judas Område og i Kalebs Sydland, og Ziklag stak vi i Brand."
\par 15 Da sagde David til ham: "Vil du vise mig Vej til denne Røverskare?" Han svarede: "Tilsværg mig ved Gud, at du hverken vil dræbe mig eller udlevere mig til min Herre, så vil jeg vise dig Vej til den!"
\par 16 Så viste han dem Vej, og de traf dem spredte rundt om i hele Egnen i Færd med at spise og drikke og holde Fest på hele det store Bytte, de havde taget fra Filisterlandet og Judas Land.
\par 17 David huggede dem da ned fra Dæmring til Aften; og ingen af dem undslap undtagen 400 unge Mænd, som svang sig på Kamelerne og flygtede.
\par 18 Og David reddede alt, hvad Amalekiterne havde røvet, også sine to Hustruer.
\par 19 Og der savnedes intet, hverken småt eller stort, hverken Byttet eller Sønnerne og Døtrene eller noget af, hvad de havde taget med; det hele bragte David tilbage.
\par 20 Da tog de alt Småkvæget og Hornkvæget og drev det hen for David og sagde: "Her er Davids Bytte!"
\par 21 Da David kom til de 200 Mand, som havde været for udmattede til at følge ham, og som han havde ladet blive ved Besorbækken, gik de David og hans Folk i Møde, og David gik hen til Folkene og hilste på dem.
\par 22 Men alle ildesindede Niddinger blandt dem, som havde fulgt David, tog til Orde og sagde: "De fulgte os ikke, derfor vil vi intet give dem af Byttet, vi har reddet; kun deres Hustruer og Børn må de tage med hjem!"
\par 23 David sagde: "Således må I ikke gøre, nu da HERREN har været gavmild imod os og skærmet os og givet Røverskaren, som overfaldt os, i vor Hånd.
\par 24 Hvem er der vel, som vil følge eder i det? Nej, den, der drog i Kampen, og den, der blev ved Trosset, skal have lige Del, de skal dele med hinanden!"
\par 25 Og derved blev det både den Dag og siden; han gjorde det til Lov og Ret i Israel, som det er den Dag i bag.
\par 26 Da David kom til Ziklag, sendte han noget af Byttet til de Ældste i Juda, som var hans Venner, med det Bud: "Her er en Gave til eder af Byttet, der er taget fra HERRENs Fjender!"
\par 27 Det var til dem i Betel, i Ramot i Sydlandet, i Jattir,
\par 28 dem i Ar'ara, i Sifmot, i Esjtemoa,
\par 29 i Karmel, i Jerame'eliternes Byer, i Keniternes Byer,
\par 30 i Horma, i Bor-Asjan, i Atak,
\par 31 i Hebron, og ligeledes til alle de andre Steder, hvor David havde færdedes med sine Mænd.

\chapter{31}

\par 1 Imidlertid angreb Filisterne Israel; og Israels Mænd flygtede for Filisterne, og de faldne lå rundt om på Gilboas Bjerg.
\par 2 Og Filisterne forfulgte Saul og hans Sønner og dræbte Sauls Sønner, Jonatan, Abinadab og Malkisjua.
\par 3 Kampen rasede om Saul, og han blev opdaget af Bueskytterne og grebes af stor Angst for dem.
\par 4 Da sagde Saul til sin Våbendrager: "Drag dit Sværd og gennembor mig, for at ikke disse uomskårne skal komme og gennembore mig og mishandle mig!" Men Våbendrageren vilde ikke, thi han gøs tilbage derfor. Da tog Saul Sværdet og styrtede sig i det;
\par 5 og da Våbendrageren så, at Saul var død, styrtede også han sig i sit Sværd og fulgte ham i Døden.
\par 6 Således fulgtes denne Dag Saul, hans tre Sønner, hans Våbendrager og alle hans Mænd i Døden.
\par 7 Men da Israels Mænd i Byerne i Dalen og ved Jordan så, at Israels Mænd var flygtet, og at Saul og hans Sønner var faldet, forlod de Byerne og flygtede, hvorpå Filisterne kom og besatte dem.
\par 8 Da Filisterne Dagen efter kom for at plyndre de faldne, fandt de Saul og hans tre Sønner liggende på Gilboas Bjerg;
\par 9 de huggede da Hovedet at ham, afførte ham hans Våben og sendte Bud rundt i Filisternes Land for at bringe deres Afguder og Folket Glædesbudet.
\par 10 Våbnene lagde de i Astartes Tempel, men Kroppen hængte de op på Bet-Sjans Mur.
\par 11 Men da Indbyggerne i Jabesj i Gilead hørte, hvad Filisterne havde gjort ved Saul,
\par 12 brød alle våbenføre Mænd op, og efter at have gået hele Natten igennem tog de Sauls og hans Sønners Kroppe ned fra Bet-Sjans Mur, bragte dem med til Jabesj og brændte dem der.
\par 13 Så tog de deres Ben og jordede dem under Tamarisken i Jabesj og fastede syv Dage.


\end{document}