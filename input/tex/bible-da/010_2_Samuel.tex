\begin{document}

\title{Anden Samuelsbog}


\chapter{1}

\par 1 Da David efter Sauls Død var vendt tilbage fra Sejeren over Amalek og havde opholdt sig to Dage i Ziklag,
\par 2 kom der Tredjedagen en Mand fra Hæren, fra Saul, med sønderrevne Klæder og Jord på Hovedet, og da han kom hen til David, kastede han sig til Jorden og bøjede sig.
\par 3 David spurgte ham: "Hvor kommer du fra?" Han svarede: "Jeg slap bort fra Israels Hær!"
\par 4 David sagde da til ham: "Hvorledes gik det? Fortæl mig det!" Han svarede: "Folket flygtede fra Kampen, og mange af Folket faldt og døde; også Saul og hans Søn Jonatan er døde."
\par 5 Da sagde David til den unge Mand, som bragte ham Budet: "Hvoraf ved du, at Saul og hans Søn Jonatan er døde?"
\par 6 Den unge Mand, der bragte ham Budet, svarede: "Det traf sig, at jeg var på Gilboas Bjerg, og se, Saul stod lænet til sit Spyd, medens Vognene og Rytterne trængte ham;
\par 7 og da han vendte sig om, fik han Øje på mig og kaldte på mig; og jeg sagde: Her er jeg!
\par 8 Da spurgte han mig: Hvem er du? Og jeg svarede: Jeg er en Amalekit!
\par 9 Så sagde han til mig: Kom herhen og giv mig Dødsstødet! Thi Krampen har grebet mig, men jeg lever endnu!
\par 10 Og jeg trådte hen til ham og gav ham Dødsstødet, thi jeg så, at han ikke kunde leve, når han faldt om. Så tog jeg Diademet, han havde på Hovedet, og et Armbånd, han bar på Armen, og dem har jeg med hid til min Herre."
\par 11 Da tog David fat i sine Klæder og sønderrev dem, og ligeså gjorde alle hans Mænd;
\par 12 og de holdt Klage, græd og fastede til Aften over Saul og hans Søn Jonatan og HERRENs Folk og Israels Hus, fordi de var faldet for Sværdet.
\par 13 Derpå sagde David til den unge Mand, som havde bragt ham Budet: "Hvor er du fra?" Han svarede: "Jeg er Søn af en Amalekit, der bor her som fremmed."
\par 14 Da sagde David: "Frygtede du dog ikke for at lægge Hånd på HERRENs Salvede og dræbe ham!"
\par 15 David kaldte så på en af sine Folk og sagde: "Kom herhen og stød ham ned!" Og han slog ham ihjel.
\par 16 Men David sagde til ham: "Dit Blod komme over dit eget Hoved! Thi din egen Mund vidnede imod dig, da du sagde: Jeg gav HERRENs Salvede Dødsstødet!"
\par 17 Da sang David denne Klagesang over Saul og hans Søn Jonatan.
\par 18 Den skal læres af Judas Sønner; den står optegnet i de Oprigtiges Bog.
\par 19 Din Pryd, Israel, ligger dræbt på dine Høje. Ak, at dog Heltene faldt!
\par 20 Forkynd det ikke i Gat, ej lyde der Glædesbud på Askalons Gader, at ikke Filisternes Døtre skal fryde sig, de uomskårnes Døtre juble!
\par 21 Gilboas Bjerge! Ej falde Dug og Regn på eder, I Dødens Vange! Thi Heltenes Skjolde vanæredes der; Sauls Skjold er ej salvet med Olie.
\par 22 Uden faldnes Blod, uden Heltes Fedt kom Jonatans Bue ikke tilbage, Sauls Sværd ikke sejrløst hjem.
\par 23 Saul og Jonatan, de elskelige, hulde, skiltes ikke i Liv eller Død; hurtigere var de end Ørne, stærkere var de end Løver!
\par 24 O, Israels Døtre, græd over Saul, som klædte eder yndigt i Purpur, satte Guldsmykker på eders Klæder!
\par 25 Ak, at dog Heltene faldt i Slagets Tummel - dræbt ligger Jonatan på dine Høje!
\par 26 Jeg sørger over dig, Jonatan, Broder, du var mig såre kær; underfuld var mig din Kærlighed, mere end Kvinders Kærlighed.
\par 27 Ak, at dog Heltene faldt, Stridsvåbnene lagdes øde!

\chapter{2}

\par 1 Derefter rådspurgte David HERREN: "Skal jeg drage op til en af Judas Byer?" HERREN svarede:"Gør det!" Og David spurgte: "Hvor skal jeg drage hen?" Da svarede han: "Til Hebron!"
\par 2 Så drog David derop tillige med sine to Hustruer Ahinoam fra Jizre'el og Abigajil, Karmeliten Nabals Hustru;
\par 3 og han tog sine Mænd med derop tillige med deres Familier, og de bosatte sig i Byerne omkring Hebron.
\par 4 Da kom Judas Mænd derhen og salvede David til Konge over Judas Hus. Da David fik at vide, at Mændene i Jabesj i Gilead havde jordet Saul,
\par 5 sendte han Sendebud til Mændene i Jabesj i Gilead og lod sige: "HERREN velsigne eder, fordi I således viste Godhed mod eders Herre Saul og jordede ham.
\par 6 Måtte nu HERREN vise eder Godhed og Trofasthed! Men også jeg vil gøre godt imod eder, fordi I gjorde dette.
\par 7 Tag eder derfor sammen og vis eder som stærke Mænd; thi eders Herre Saul er død, og Judas Hus har allerede salvet mig til Konge!"
\par 8 Men Abner, Ners Søn, Sauls Hærfører, tog Sauls Søn Isjbosjet og bragte ham over til Mahanajim
\par 9 og udråbte ham til Konge over Gilead, Aseriterne, Jizre'el, Efraim og Benjamin, over hele Israel.
\par 10 Isjbosjet, Sauls Søn, var fyrretyve År, da han blev Konge over Israel, og han herskede to År. Kun Judas Hus sluttede sig til David.
\par 11 Den Tid David herskede i Hebron over Judas Hus, var syv År og seks Måneder.
\par 12 Abner, Ne'rs Søn, drog med Isjbosjets, Sauls Søns, Folk fra Mahanajim til Gibeon;
\par 13 ligeledes drog Joab, Zerujas Søn ud med Davids folk, og de stødte sammen med dem ved Dammen i Gibeon; og de slog sig ned hver på sin Side af Dammen.
\par 14 Da sagde Abner til Joab: "Lad de unge Mænd stå op og udføre Våben lege for os!" Og Joab sagde: "Ja, lad dem stå op!"
\par 15 Så stod de op og gik frem lige mange fra hver Side, tolv Benjaminiter for Isjbosjet, Sauls Søn, og tolv af Davids Folk;
\par 16 men de greb hverandre om Hovedet og stødte Sværdet i Siden på hverandre, så de faldt alle til Hobe. Derfor kalder man dette Sted Helkat-Hazzurim; det ligger ved Gibeon.
\par 17 Samme Dag kom det til en meget hård Kamp, i hvilken Abner og Israels Mænd blev drevet på Flugt af Davids Folk.
\par 18 Ved den Lejlighed var Zerujastre Sønner med, Joab, Abisjaj og Asa'el; og Asa'el, der var rapfodet som Markens Gazeller,
\par 19 forfulgte Abner uden at bøje af til højre eller venstre.
\par 20 Abner vendte sig om og spurgte: "Er det dig, Asa'el?" Han svarede: "Ja, det er!"
\par 21 Da sagde Abner: "Bøj af til en af Siderne, grib en af de unge Mænd og tag dig hans Rustning!" Men Asa'el vilde ikke opgive at forfølge ham.
\par 22 Da sagde Abner videre til Asa'el: "Stands med at forfølge mig! Hvorfor skal jeg slå dig til Jorden? Og hvorledes skal jeg så kunne se din Broder Joab i Øjnene?"
\par 23 Men han vægrede sig ved at standse, og Abner stødte da baglæns Spydet gennem Underlivet på ham, så det kom ud af Ryggen, og han faldt død om på Stedet. Og alle, som kom til Stedet, hvor Asa'el lå og var død, stod stille.
\par 24 Men Joab og Abisjaj forfulgte Abner, og da Solen gik ned, havde de nået Gibeat-Amma, som ligger østen for Gia ved Vejen til Gibeons Ørken.
\par 25 Da samlede Benjaminiterne sig om Abne'r og stillede sig i Klynge på Toppen af Gibeat-Amma.
\par 26 Men Abner råbte til Joab: "Skal da Sværdet altid blive ved at fortære? Ved du ikke, at Eftersmagen er besk? Hvor længe skal det vare, inden du byder Folket standse med at forfølge deres Brødre?"
\par 27 Joab svarede: "Så sandt HERREN lever: Havde du ikke talt, vilde Folkene først i Morgen have standset med at forfølge deres Brødre!"
\par 28 Derpå stødte Joab i Hornet, og hele Folket standsede; de forfulgte ikke mere Israel og fortsatte ikke Kampen.
\par 29 Abner og hans Mænd vandrede så i Løbet af Natten igennem Arabalavningen, satte over Jordan, gik hele Kløften igennem og kom til Mahanajim.
\par 30 Da Joab vendte tilbage fra Forfølgelsen af Abner og samlede alle Krigerne, savnedes foruden Asa'el nitten Mand af Davids Folk,
\par 31 medens Davids Folk havde slået 360 Mand ihjel af Benjaminiterne, Abners Folk.
\par 32 Derpå bar de Asa'el bort og jordede ham i hans Faders Grav i Betlehem, og Joab og hans Mænd vandrede hele Natten igennem; da Solen stod op, nåede de Hebron.

\chapter{3}

\par 1 Krigen imellem Sauls og Davids Huse trak i Langdrag; men David blev stærkere og stærkere, Sauls Hus svagere og svagere.
\par 2 I Hebron fødtes der David Sønner; hans førstefødte var Amnon, Søn af Ahinoam fra Jizre'el,
\par 3 den næstældste Kil'ab, Søn af Abigajil, Harmeliten Nabals Hustru, den tredje Absalon, en Søn af Kong Talmaj af Gesjurs Datter Ma'aka,
\par 4 den fjerde Adonija, en Søn af Haggit, den femte Sjefatja, en Søn af Abital,
\par 5 og den sjette Jitream, Søn af Davids Hustru Egla. Disse fødtes David i Hebron.
\par 6 Under Krigen mellem Sauls og Davids Huse ydede Abner Sauls Hus kraftig Støtte.
\par 7 Nu havde Saul haft en Medhustru ved Navn Rizpa, Ajjas Datter.
\par 8 Abner blev opbragt over Isjbosjets Ord og sagde: "Er jeg nu blevet et Hundehoved fra Juda? Nu har jeg Gang på Gang vist Godhed mod din Fader Sauls Hus, hans Brødre og Venner og ikke ladet dig falde i Davids Hånd, og så går du nu i Rette med mig for en Kvindes Skyld!
\par 9 Gud ramme Abner både med det ene og det andet: Hvad HERREN tilsvor David, skal jeg nu sørge for bliver opfyldt på ham;
\par 10 jeg skal sørge for, at Sauls Hus mister Kongedømmet og Davids Trone bliver rejst over Israel og Juda fra Dan til Be'ersjeba!"
\par 11 Og af Frygt for Abner kunde Isjbosjet ikke svare et Ord.
\par 12 Så sendte Abner Sendebud til David i Hebron og lod sige: "Slut Pagt med mig! Se, jeg vil hjælpe dig og bringe hele Israel over på din Side!"
\par 13 Han svarede: "Vel, jeg vil slutte Pagt med dig; men een Ting kræver jeg af dig: Du bliver ikke stedet for mit Åsyn, med mindre du har Sauls Datter Mikal med, når du kommer!"
\par 14 David sendte derpå Bud til Isjbosjet, Sauls Søn, og lod sige:"Giv mig min Hustru Mikal, som jeg blev trolovet med for 100 Filisterforhuder!"
\par 15 Da sendte Isjbosjet Bud og lod hende hente hos hendes Mand Paltiel, Lajisj's Søn.
\par 16 Hendes Mand fulgte hende grædende lige til Bahurim; her sagde Abner til ham: "Gå nu hjem!" Så vendte han hjem.
\par 17 Men Abner havde forhandlet med Israels Ældste og sagt: "Allerede tidligere ønskede I David til Konge;
\par 18 så gør nu Alvor af det! Thi HERREN har sagt om David: Ved min Tjener Davids Hånd vil jeg frelse mit Folk Israel fra Filisternes og alle dets fjenders Hånd!"
\par 19 Abner talte også til Benjamin derom. Endelig gik Abner også til Hebron for at meddele David alt, hvad Israel og hele Benjamins Hus havde vedtaget.
\par 20 Da Abner, fulgt af tyve Mænd, kom til David i Hebron, gjorde David Gæstebud for Abner og Mændene, som var med ham.
\par 21 Derpå sagde Abner til David: "Lad mig bryde op og drage hen og samle hele Israel om min Herre Kongen, for at de kan slutte Pagt med dig, at du kan blive Konge over alt, hvad din Hu står til!" Da lod David Abner rejse, og han drog bort i Fred.
\par 22 Just da kom Davids Folk og Joab hjem fra et Strejftog og medførte et rigt Bytte; Abner var da ikke mere hos David i Hebron, thi David havde ladet ham rejse, og han var draget bort i Fred.
\par 23 Da nu Joab var vendt hjem med hele sin Hær, fik han at vide at Abner, Ners Søn, havde været hos Kongen, og at denne havde ladet ham rejse, så han var draget bort i Fred.
\par 24 Da gik Joab til Kongen og sagde: "Hvad har du gjort? Abner har jo været hos dig! Hvorfor lod du ham rejse, så han frit kunde drage bort?
\par 25 Indser du ikke, at Abner, Ners Søn, kun kom for at bedrage dig. udspejde din Færd og få at vide, hvad du har for!"
\par 26 Så gik Joab bort fra David og sendte uden Davids Vidende Sendebud efter Abner og de hentede ham tilbage fra Bor-Sira.
\par 27 Og da Abner kom tilbage til Hebron, tog Joab ham til Side midt i Porten for at tale uhindret med ham; og der dræbte han ham ved et Stik i Underlivet for at hævne sin Broder Asa'els Blod.
\par 28 Da det siden kom David for Øre, sagde han: "Jeg og mit Rige er til evig Tid uden Skyld for HERREN i Abners, Ners Søns, Blod!
\par 29 Det komme over Joabs Hoved og over hele hans Fædrenehus; og Joabs Hus være aldrig frit for Folk, som lider af Flåd eller Spedalsk hed, går med Krykke eller falder for Sværdet eller mangler Brød!"
\par 30 Men Joab og hans Broder Abisjaj havde slået Abner ihjel, fordi han havde fældet deres Broder Asa'el i Kampen ved Gibeon.
\par 31 Derpå sagde David til Joab og alle Folkene, som fulgte ham: "Sønderriv eders Klæder, tag Sæk om eder og hold Klage over Abner!" Og Kong David gik selv bag efter Båren.
\par 32 Og da man jordede Abner i Hebron, græd Kongen højt ved Abners Grav, og alt Folket græd med,
\par 33 Da sang Kongen denne Klagesang over Abner: Skulde Abner dø en Dåres Død?
\par 34 Dine Hænder var ikke bundne, dine Fødder ikke lagt i Lænker, du faldt, som man falder for Misdædere! Da græd hele Folket end mere over ham,
\par 35 og da hele Folket kom for at få David til af spise, medens det endnu var Dag, svor David: "Gud ramme mig både med det ene og det andet, om jeg smager Brød eller noget andet, før Sol går ned!"
\par 36 Hele Folket lagde Mærke dertil, og det gjorde et godt Indtryk på dem; alt, hvad Kongen foretog sig, gjorde et godt Indtryk på alt Folket;
\par 37 og hele Folket og hele Israel skønnede den Dag, at Kongen ikke var Ophavsmand til Drabet på Abner, Ners Søn.
\par 38 Og Kongen sagde til sine Folk: "Ved I ikke, af der i Dag er faldet en Øverste og Stormand i Israel?
\par 39 Men jeg er endnu for svag, skønt jeg er salvet til Konge, og disse Mænd, Zerujasønnerne, er mig for stærke. HERREN gengælde Udådsmanden hans Skændsels- dåd!"

\chapter{4}

\par 1 Da Isjbosjet, Sauls Søn, hørte, at Abner var død i Hebron, tabte han Modet, og hele Israel grebes af Skræk.
\par 2 Nu havde Isjbosjet, Sauls Søn, to Mænd, der var Førere for Strejfskarer, den ene hed Ba'ana, den anden Rekab, Sønner af Benjaminiten Rimmon fra Be'erot; thi også Be'erot regnes til Benjamin;
\par 3 dog var Be'erotiterne flygtet til Gittajim, hvor de bor som fremmede den Dag i Dag.
\par 4 Sauls Søn Jonatan havde en Søn, der var lam i Fødderne; han var fem År gammel, da Efterretningen om Saul og Jonatan kom fra Jizre'el, og hans Fostermoder tog ham og flygtede; men under hendes skyndsomme Flugt faldt han fra hende og blev lam; hans Navn var Mefibosjet.
\par 5 Be'erotiten Rimmons Sønner Rekab og Ba'ana gav sig på Vej og kom ved Middagstide til Isjbosjets Hus, medens han sov til Middag;
\par 6 og da Dørvogtersken, som var ved at rense Hvede, var faldet i Søvn, slap Rekab og hans Broder Ba'ana forbi
\par 7 og trængte ind i Huset, hvor Isjbosjet lå på sit Leje i Soveværelset; og de slog ham ihjel og huggede Hovedet af ham; derpå tog de Hovedet og vandrede i Løbet at Natten gennem Arabalavningen
\par 8 og bragte Isjbosjets Hoved til David i Hebron, idet de sagde til Kongen: "Her er Hovedet af Isjbosjet, din Fjende Sauls Søn, han, som stod dig efter Livet; i Dag har HERREN givet min Herre Kongen Hævn over Saul og hans Afkom!"
\par 9 Da svarede David Be'erotiten Rimmons Sønner Rekab og hans Broder Ba'ana: "Så sandt HERREN lever, som har udfriet mig af al Trængsel:
\par 10 Den, som bragte mig Efterretning om Sauls Død, i den Tro at han bragte et Glædesbud, ham greb jeg og lod dræbe i Ziklag for at give ham Løn for hans Glædesbud;
\par 11 hvor meget mere skulde jeg da ikke nu, når gudløse Mænd har myrdet en retfærdig Mand på hans Leje i hans eget Hus, kræve hans Blod af eder og udrydde eder af Jorden!"
\par 12 Derpå bød David sine Folk dræbe dem, og de huggede Hænder og Fødder af dem og hængte dem op ved Dammen i Hebron; men Isjbosjets Hoved tog de og jordede i Abners Grav i Hebron.

\chapter{5}

\par 1 Derpå kom alle Israels Stammer til David i Hebron og sagde: "Vi er jo dit Kød og Blod!
\par 2 Allerede før i Tiden, da Saul var Konge over os, var det dig,som førte Israel ud i Kamp og hjem igen; og HERREN sagde til dig: Du skal vogte mit Folk Israel og være Hersker over Israel!"
\par 3 Og alle Israels Ældste kom til Kongen i Hebron, og Kong David sluttede i Hebron Pagt med dem for HERRENs Åsyn, og de salvede David til Konge over Israel.
\par 4 David var tredive År, da han blev Konge, og han herskede fyrretyve År.
\par 5 I Hebron herskede han over Juda syv År og seks Måneder, og i Jerusalem herskede han tre og tredive År over hele Israel og Juda.
\par 6 Derpå drog Kongen med sine Mænd til Jerusalem mod Jebusiterne, som boede deri Landet. Man sagde til Kongen: "Her kan du ikke trænge ind, thi blinde og lamme vil slå dig tilbage!" Dermed vilde de sige: "David kommer ikke herind!"
\par 7 Men David indtog Klippeborgen Zion, det er Davidsbyen.
\par 8 På den Dag sagde David: "Enhver, som trænger frem til Vandledningen og slår en Jebusit, de halte og blinde, som Davids Sjæl hader, skal være Øverste og Hærfører". Derfor siger man: "En blind og en lam kommer ikke ind i Huset!"
\par 9 Så tog David Bolig i Klippeborgen og kaldte den Davidsbyen; og han befæstede Byen rundt om fra Millo og indefter.
\par 10 Og David blev mægtigere og mægtigere; HERREN, Hærskarers Gud, var med ham.
\par 11 Kong Hiram af Tyrus sendte Sendebud til David med Cedertræer og tillige Tømmermænd og Stenhuggere, som byggede ham et Hus.
\par 12 Da skønnede David, at HERREN havde sikret hans Kongemagt over Israel og højnet hans Kongedømme for sit Folk Israels Skyld.
\par 13 David tog i Jerusalem endnu flere Medhustruer og Hustruer, efter at han var kommet dertil fra Hebron, og der fødtes ham flere Sønner og Døtre.
\par 14 Navnene på dem, som fødtes ham i Jerusalem, er følgende: Sjammua, Sjobab, Natan, Salomo,
\par 15 Jibhar, Elisjua, Nefeg, Jafia,
\par 16 Elisjama, Ba'aljada og Elifelet.
\par 17 Men da Filisterne hørte, at David var salvet til Konge over Israel, rykkede de alle ud for at søge efter ham. Ved Efterretningen herom drog David ned til Klippeborgen,
\par 18 medens Filisterne kom og bredte sig i Refaimdalen.
\par 19 David rådspurgte da HERREN: "Skal jeg drage op mod Filisterne? Vil du give dem i min Hånd?" Og HERREN svarede David: "Drag op, thi jeg vil give Filisterne i din Hånd!"
\par 20 Så drog David til Ba'al-Perazim, og der slog han dem. Da sagde han: "HERREN har brudt igennem mine Fjender foran mig, som Vand bryder igennem!" Derfor kalder man Stedet Ba'al-Perazim.
\par 21 Og de lod deres Guder i Stikken der, og David og hans Mænd tog dem.
\par 22 Men Filisterne bredte sig på ny i Refaimdalen.
\par 23 Da David rådspurgte HERREN, svarede han: "Drag ikke imod dem, men omgå dem og fald dem i Ryggen ud for Bakabuskene.
\par 24 Når du da hører Lyden af Skridt i Bakabuskenes Toppe, skal du skynde dig, thi så er HERREN draget ud foran dig for at slå Filisternes Hær!"
\par 25 David gjorde, som HERREN bød,og slog Filisterne fra Gibeon til hen imod Gezer.

\chapter{6}

\par 1 David samlede alt udsøgt Mandskab i Israel, 30000 Mand.
\par 2 Derpå brød David op med alle sine Krigere og drog til Ba'al i Juda for der at hente Guds Ark, over hvilken Hærskarers HERREs Navn er nævnet, han, som troner over Keruberne.
\par 3 De satte da Guds Ark på en ny Vogn og førte den bort fra Abinadabs Hus på Højen, og Abinadabs Sønner Uzza og Ajo kørte Vognen,
\par 4 således at Uzza gik ved Siden af og Ajo foran Guds Ark.
\par 5 David og hele Israel legede af alle Kræfter for HERRENs Åsyn til Sang og til Citre, Harper, Pauker, Bjælder og Cymbler.
\par 6 Men da de kom til Nakons Tærskeplads, rakte Uzza Hånden ud og greb fat i Guds Ark, fordi Okserne snublede.
\par 7 Da blussede HERRENS Vrede op mod Uzza, og Gud slog ham der, fordi han rakte Hånden ud mod Arken, og han døde på Stedet ved Siden af Guds Ark.
\par 8 Men David græmmede sig over, at HERREN havde tilføjet Uzza et Brud. Derfor kaldte man Stedet Perez-Uzza, som det hedder den Dag i Dag.
\par 9 Og David grebes den Dag af Frygt for HERREN og sagde: "Hvor kan da HERRENs Ark komme hen hos mig!"
\par 10 Og David vilde ikke flytte HERRENs Ark hen hos sig i Davidsbyen, men lod den sætte ind i Gatiten Obed-Edoms Hus.
\par 11 HERRENs Ark blev så i Gatiten Obed-Edoms Hus tre Måneder, og HERREN velsignede Obed-Edom og hele hans Hus.
\par 12 Da nu Kong David fik Underretning om, at HERREN for Guds Arks Skyld havde velsignet Obed-Edoms Hus og alt, hvad hans var, gik han hen og lod under Festglæde Guds Ark bringe op fra Obed-Edoms Hus til Davidsbyen.
\par 13 Og da de, som bar HERRENs Ark, havde gået seks Skridt, ofrede han en Okse og en Fedekalv.
\par 14 Og David dansede af alle Kræfter for HERRENs Åsyn, iført en linned Efod.
\par 15 Således bragte David og hele Israel HERRENs Ark op under Festjubel og Hornblæsning.
\par 16 Men da HERRENs Ark kom til Davidsbyen, så Sauls Datter Mikal ud af Vinduet; og da hun så Kong David springe og danse for HERRENs Åsyn, ringeagtede hun ham i sit Hjerte.
\par 17 De førte så HERRENs Ark ind og stillede den på Plads midt i det Telt, David havde rejst den, og David ofrede Brændofre og Takofre for HERRENs Åsyn.
\par 18 Og da David var færdig med Brændofrene og Takofrene, velsignede han Folket i Hærskarers HERREs Navn
\par 19 og uddelte til alt Folket, til hver enkelt af hele Israels Mængde, både Mand og Kvinde, et Brød, et Stykke Kød og en Rosinkage; derpå gik alt Folket hver til sit.
\par 20 Men da David vendte hjem for at velsigne sit Hus, gik Sauls Datter Mikal ham i Møde og sagde: "Hvor ærbart Israels Konge opførte sig i Dag, da han blottede sig for sine Undersåtters Trælkvinders Øjne, som letfærdige Mennesker plejer at gøre!"
\par 21 David svarede Mikal: "For HERRENs Åsyn vil jeg lege, så sandt HERREN lever, som udvalgte mig fremfor din Fader og hele hans Hus, så han satte mig til Fyrste over HERRENs Folk Israel; jeg vil lege for HERRENs Åsyn,
\par 22 selv om jeg derved nedværdiges og synker endnu dybere i dine Øjne; men hos Trælkvinderne, du talte om, skal jeg vinde Ære!"
\par 23 Og Sauls Datter Mikal fik til sin Dødedag intet Barn.

\chapter{7}

\par 1 Engang Kongen sad i sit Hus, efter at HERREN havde skaffet ham Ro for alle hans Fjender rundt om,
\par 2 sagde han til Profeten Natan: "Se, jeg har et Cedertræshus at bo i, men Guds Ark har Plads i et Telt!"
\par 3 Natan svarede Kongen: "Gør alt, hvad din Hu står til, thi HERREN er med dig!"
\par 4 Men samme Nat kom HERRENs Ord til Natan således:
\par 5 "Gå hen og sig til min Tjener David: Så siger HERREN: Skulde du bygge mig et Hus at bo i?
\par 6 Jeg har jo ikke haft noget Hus at bo i, siden den Dag jeg førte Israeliterne op fra Ægypten, men vandrede med, boende i et Telt.
\par 7 Har jeg, i al den Tid jeg vandrede om blandt alle Israeliterne, sagt til nogen af Israels Dommere, som jeg satte til at vogte mit Folk Israel: Hvorfor bygger I mig ikke et Cedertræshus?
\par 8 Sig derfor til min Tjener David: Så siger Hærskarers HERRE: Jeg tog dig fra Græsgangen, fra din Plads bag Småkvæget til at være Fyrste over mit folk Israel,
\par 9 og jeg var med dig, overalt hvor du færdedes, og udryddede alle dine Fjender foran dig; jeg vil skabe dig et Navn som de størstes på Jorden
\par 10 og skaffe mit Folk Israel en Hjemstavn og plante det, så det kan blive boende på sit Sted uden mere at skulle forstyrres i sin Ro, og uden at Voldsmænd mere skal plage det som tidligere,
\par 11 dengang jeg satte Dommere over mit Folk Israel; og jeg vil give det Ro for alle dets Fjender. Så kundgør HERREN dig nu: Et Hus vil HERREN bygge dig!
\par 12 Når dine Dage er omme, og du hviler hos dine Fædre, vil jeg efter dig oprejse din Sæd, som udgår af dit Liv, og grundfæste hans Kongedømme.
\par 13 Han skal bygge mit Navn et Hus, og jeg vil grundfæste hans Kongetrone evindelig.
\par 14 Jeg vil være din Sæd en Fader, og den skal være mig en Søn! Når den synder, vil jeg tugte den med Menneskestok og Menneskers Slag,
\par 15 men min Miskundhed vil jeg ikke tage fra den, som jeg tog den fra din Forgænger.
\par 16 Dit Hus og dit Kongedømme skal stå fast for mit Åsyn til evig Tid, din Trone skal stå til evig Tid!"
\par 17 Alle disse Ord og hele denne Åbenbaring meddelte Natan David.
\par 18 Da gik Kong David ind og dvælede for HERRENs Åsyn og sagde: "Hvem er jeg, Herre, HERRE, og hvad er mit Hus, at du har bragt mig så vidt?
\par 19 Men end ikke det var dig nok Herre, HERRE, du gav også din Tjeners Hus Forjættelser for fjerne Tider og lod mig skue kommende Slægter, Herre, HERRE!
\par 20 Hvad mere har David at sige dig Du kender jo dog din Tjener, Herre, HERRE.
\par 21 For din Tjeners Skyld, og fordi din Hu stod dertil, gjorde du dette og kundgjorde din Tjener alt dette store,
\par 22 Herre, HERRE; thi ingen er som du, og der er ingen Gud uden dig efter alt hvad vi har hørt med vore Ører.
\par 23 Og hvor på Jorden findes et Folk som dit Folk Israel, et Folk.
\par 24 Du har grundfæstet dit Folk Israel som dit Folk til evig Tid, og du, HERRE, er blevet deres Gud.
\par 25 Så opfyld da, HERRE, Gud, til evig Tid den Forjættelse, du udtalte om din Tjener og hans Hus og gør, som du sagde!
\par 26 Da skal dit Navn blive stort til evig Tid, så man siger: Hærskarers HERRE, Gud over Israel! Og din Tjener Davids Hus skal stå fast for dit Åsyn.
\par 27 Thi du, Hærskarers HERRE, Israels Gud, har åbenbaret for din Tjener: Jeg vil bygge dig et Hus! Derfor har din Tjener fundet sit Hjerte til af bede denne Bøn til dig.
\par 28 Derfor, Herre, HERRE, du er Gud, og dine Ord er Sandhed! Du har givet din Tjener denne Forjættelse,
\par 29 så lad det behage dig at velsigne din Tjeners Hus, at det til evig Tid må stå fast for dit Åsyn. Thi du, Herre, HERRE, har talt, og med din Velsignelse skal din Tjeners Hus velsignes evindelig!"

\chapter{8}

\par 1 Nogen Tid efter slog David Filisterne og undertvang dem, og David fratog Filisterne Meteg-Ha'amma.
\par 2 Fremdeles slog han Moabiterne, og han målte dem med en Snor, idet han lod dem lægge sig ned på Jorden og afmålte to Snorlængder, der skulde dræbes, og een, der skulde blive i Live.
\par 3 Ligeledes slog David Rehobs Søn, Kong Hadad'ezer af Zoba, da han var draget ud for at genoprette sit Herredømme ved Floden.
\par 4 David fratog ham 1700 Ryttere og 20000 Mand Fodfolk og lod alle Stridshestene lamme på hundrede nær, som han skånede.
\par 5 Og da Aramæerne fra Damaskus kom Hadad'ezer af Zoba til Hjælp, slog David 22000 Mand af Aramæerne.
\par 6 Derpå indsatte David Fogeder i det damaskenske Aram, og Aramæerne blev Davids skatskyldige Undersåtter. Således, gav HERREN David Sejr, overalt hvor han drog frem.
\par 7 Og David tog de Guldskjolde, Hadad'ezers Folk havde båret, og bragte dem til Jerusalem;
\par 8 og fra Hadad'ezers Byer Teba og Berotaj bortførte Kong David Kobber i store Mængder.
\par 9 Men da Kong To'i af Hamat hørte, at David havde slået hele Hadad'ezers Stridsmagt,
\par 10 sendte han sin Søn Hadoram til Kong David for at hilse på ham og lykønske ham til, at han havde kæmpet med Hadad'ezer og slået ham - Hadad'ezer havde nemlig ligget i Krig med To'i - og han medbragte Sølv-, Guld- og Kobber-sager.
\par 11 Også dem helligede Kong David HERREN tillige med det Sølv og Guld, han havde helliget af Byttet fra alle de undertvungne Folk,
\par 12 Edom, Moab Ammoniterne, Filisterne, Amalek, og af det Bytte, han havde taget fra Rehobs Søn, Kong Hadad'ezer af Zoba.
\par 13 Og David vandt sig et Navn. Da han vendte tilbage fra Sejren over Aram, slog han Edom i Saltdalen, 18000 Mand;
\par 14 derpå indsatte han Fogeder i Edom; i hele Edom indsatte han Fogeder, og alle Edomiterne blev Davids Undersåtter, Således gav HERREN David Sejr, overalt hvor han drog frem.
\par 15 Og David var Konge over hele Israel, og han øvede Ret og Retfærdighed mod hele sit Folk.
\par 16 Joab, Zerujas Søn, var sat over Hæren; Josjafat, Ahiluds Søn, var Kansler;
\par 17 Zadok, Abitubs Søn, og Ebjatar, Ahimeleks Søn, var Præster; Seraja var Statsskriver;
\par 18 Benaja, Jojadas Søn, var sat over Kreterne og Pleterne, og Davids Sønner var Præster.

\chapter{9}

\par 1 David sagde: "Er der endnu nogen tilbage af Sauls Hus? Så vil jeg vise Godhed imod ham for Jonatans Skyld!"
\par 2 Nu var der i Sauls Hus en Træl ved Navn Ziba; han blev kaldt op til David, og Kongen sagde til ham:"Er du Ziba?" Han svarede: "Ja, det er din Træl!"
\par 3 Da sagde Kongen: "Er der ingen tilbage af Sauls Hus? Så vil jeg vise Guds Godhed imod ham." Ziba svarede Kongen: "Der lever endnu en Søn af Jonatan; han er lam i Fødderne."
\par 4 Da spurgte Kongen: "Hvor er han?" Og Ziba svarede Kongen: "Han er i Makirs, Ammiels Søns, Hus i Lodebar."
\par 5 Så lod Kong David ham hente i Makirs, Ammiels Søns, Hus i Lodebar.
\par 6 Da Mefibosjet, Sauls Søn Jonatans Søn, kom ind til David, faldt han på sit Ansigt og bøjede sig. David sagde: "Mefibosjet!" Han svarede: "Ja, her er din Træl!"
\par 7 David sagde til ham: "Frygt ikke! Jeg vil vise dig Godhed for din Fader Jonatans Skyld og give dig hele din Fader Sauls Jordegods tilbage; og du skal altid spise ved mit Bord."
\par 8 Da bøjede han sig og sagde: "Hvad er din Træl, siden du tager Hensyn til en død Hund som mig?"
\par 9 Derpå lod Kongen Sauls Tjener Ziba kalde og sagde til ham: "Alt, hvad der tilhørte Saul og hele hans Hus, har jeg givet din Herres Søn;
\par 10 men du tillige med dine Sønner og Trælle skal dyrke Jorden og indhøste Afgrøden, for at din Herres Hus kan have sit Underhold deraf; men din Herres Søn Mefibosjet skal altid spise ved mit Bord." Ziba havde femten Sønner og tyve Trælle.
\par 11 Da sagde Ziba til Kongen: "Din Træl vil gøre, ganske som min Herre Kongen byder!" Mefibosjet spiste så ved Davids Bord, som var han en af Kongens Sønner.
\par 12 Mefibosjet havde en lille Søn ved Navn Mika. Hele Zibas Husstand var Mefibosjets Trælle.
\par 13 Og Mefibosjet boede i Jerusalem, thi han spiste altid ved Kongens Bord. Og han var lam i begge Fødder.

\chapter{10}

\par 1 Nogen Tid efter døde Ammomiternes Konge, og hans Søn Hanun blev Konge i hans Sted.
\par 2 Da tænkte David: "Jeg vil vise Hanun, Nahasj's Søn, Venlighed, ligesom hans Fader viste mig Venlighed." Og David sendte Folk for at vise ham Deltagelse i Anledning af hans Faders Død. Men da Davids Mænd kom til Ammoniternes Land.
\par 3 sagde Ammoniternes Høvdinger til deres Herre Hanun: "Tror du virkelig, det er for at hædre din Fader, at David sender Bud og viser dig Deltagelse? Mon ikke det er for at udforske og udspejde Byen og ødelægge den, at David sender sine Folk til dig?"
\par 4 Da tog Hanun Davids Folk og lod det halve af deres Skæg afrage og Halvdelen af deres Klæder skære af til Sædet, og derpå lod han dem gå.
\par 5 Da David fik Efterretning herom, sendte han dem et Bud i Møde, thi Mændene var blevet grovelig forhånet; og Kongen lod sige: "Bliv i Jeriko, til eders Skæg er vokset ud!"
\par 6 Men da Ammoniterne så, at de havde lagt sig for Had hos David, sendte de Bud og lejede Aramæerne fra Bef-Rehob og Zoba, 20000 Mand Fodfolk, Kongen af Ma'aka med 1000 Mand og Folkene fra Tob, 12000 Mand.
\par 7 Da David hørte det, sendte han Joab af Sted med hele Hæren og Kærnetropperne.
\par 8 Ammoniterne rykkede så ud og stillede sig op til Kamp lige uden for Porten, medens Aramæerne fra Zoba og Rehob og Mændene fra Tob og Ma'aka stod for sig selv på åben Mark.
\par 9 Da Joa så, at Angreb truede ham både forfra og bagfra, gjorde han et Udvalg blandt alt Israels udsøgte Mandskab og tog Stilling over for Aramæerne,
\par 10 medens han overlod Resten af Mandskabet til sin Broder Abisjaj, som tog Stilling over for Ammoniterne.
\par 11 Og han sagde: "Hvis Aramæerne bliver mig for stærke, skal du ile mig til Hjælp; men bliver Ammoniterne dig for stærke, skal jeg komme og hjælpe dig.
\par 12 Tag Mod til dig og lad os tappert værge vort Folk og vor Guds Byer - så får HERREN gøre, hvad ham tykkes godt!"
\par 13 Derpå rykkede Joab frem med sine Folk til Kamp mod Aramæerne, og de flygtede for ham.
\par 14 Og da Ammoniterne så, at Aramæerne tog Flugten, flygtede de for Abisjaj og trak sig ind i Byen. Derpå vendte Joab tilbage fra Kampen med Ammoniterne og kom til Jerusalem.
\par 15 Men da Aramæerne så, at de var slået af Israel, samlede de sig,
\par 16 og Hadad'ezer sendte Bud og lod Aramæerne hinsides Floden rykke ud, og de kom til Helam med Sjobak, Hadad'ezers Hærfører, i Spidsen.
\par 17 Da David fik Efterretning herom, samlede han hele Israel, satte over Jordan og kom til Helam, hvor Aramæerne stillede sig op til Kamp mod David og angreb ham.
\par 18 Men Aramæerne flygtede for Israel, og David nedhuggede 700 Stridsheste og 40000 Mand af Aram; også deres Hærfører Sjobak slog han ihjel der.
\par 19 Da alle Hadad'ezers Lydkonger så, at de var slået af Israel, sluttede de Fred med Israel og underkastede sig. Og Aramæerne vovede ikke at hjælpe Ammoniterne mere.

\chapter{11}

\par 1 Næste År, ved den Tid Kongerne drager i Krig, sendte David Joab ud med sine folk og hele Israel, og de hærgede Ammoniternes Land og belejrede Rabba. David blev derimod selv i Jerusalem.
\par 2 Så skete det en Aftenstund, da David havde rejst sig fra sit Leje og vandrede på Kongepaladsets Tag, at han fik Øje på en Kvinde, der var i Færd med at bade sig; og Kvinden, var meget smuk.
\par 3 David sendte da Bud for at forhøre sig om hende, og der blev sagt: "Det er vist Batseba, Eliams Datter, Hetiten Urias's Hustru!"
\par 4 Så lod David hende hente, og da hun kom til ham, lå han hos hende; hun havde lige renset sig efter sin Urenhed. Derefter vendte hun hjem igen.
\par 5 Men da Kvinden blev frugtsommelig, sendte hun Bud til David og lod sige: "Jeg er frugtsommelig!"
\par 6 Da sendte David det Bud til Joab: "Send Hetiten Urias til mig!" Og Joab sendte Urias til David.
\par 7 Da Urias kom, spurgte David, hvorledes det stod til med Joab og Hæren, og hvorledes det gik med Krigen.
\par 8 Derpå sagde David til Urias: "Gå nu ned til dit Hus og tvæt dine Fødder!" Urias gik da ud af Kongens Palads, og en Gave fra Kongen blev sendt efter ham;
\par 9 men Urias lagde sig ved Indgangen til Kongens Palads hos sin Herres Folk og, gik ikke ned til sit Hus.
\par 10 Da David fik at vide, at Urias ikke var gået ned til sit Hus, sagde han til ham: "Kommer du ikke lige fra Rejsen? Hvorfor går du så ikke ned til dit Hus?"
\par 11 Urias svarede David: "Arken og Israel og Juda bor i Hytter, og min Herre Joab og min Herres Trælle ligger lejret på åben Mark; skulde jeg da gå til mit Hus for at spise og drikke og søge min Hustrus Leje? Så sandt HERREN lever, og så sandt du lever, jeg gør det ikke!"
\par 12 Da sagde David til Urias: "Så bliv her i Dag; i Morgen vil jeg lade dig rejse!" Urias blev da i Jerusalem den Dag.
\par 13 Næste Dag indbød David ham til at spise og drikke hos sig og fik ham beruset. Men om Aftenen gik han ud og lagde sig på sit Leje hos sin Herres Folk; til sit Hus gik han ikke ned.
\par 14 Næste Morgen skrev David et Brev til Joab og sendte det med Urias.
\par 15 I Brevet skrev han: "Sæt Urias der, hvor Kampen er hårdest, og lad ham i Stikken, så han kan blive dræbt!"
\par 16 Joab, der var ved at belejre Byen, satte da Urias på en Plads, hvor han vidste, der stod tapre Mænd over for ham:
\par 17 og da Mændene i Byen gjorde Udfald og angreb Joab, faldt nogle af Folket, af Davids Mænd; også Hetiten Urias faldt.
\par 18 Da sendte Joab David Melding om hele Slagets Gang,
\par 19 og han gav Sendebudet den Befaling: "Når du har givet Kongen Beretning om hele Slagets Gang,
\par 20 kan det være, at Kongen bruser op i Vrede og siger til dig: Hvorfor kom I Byen så nær i Slaget? I måtte jo vide, at der vilde blive skudt oppe fra Muren!.
\par 21 Hvem var det, der dræbte Abimelek, Jerubba'als Søn? Var det ikke en Kvinde, som kastede en Møllesten ned på ham fra Muren, så han fandt sin Død i Tebez? Hvorfor kom I Muren så nær? Så skal du sige: Også din Træl Hetiten Urias faldt!"
\par 22 Så drog Budet af Sted og kom og meldte David alt, hvad Joab havde pålagt ham, hele Slagets Gang. Da blussede Davids Vrede op mod Joab, og han sagde til Budet:"Hvorfor kom I Byen så nær i Slaget? I måtte jo vide, at der vilde blive skudt oppe fra Muren! Hvem var det, der dræbte Abimelek, Jerubba'als Søn? Var det ikke en Kvinde, som kastede en Møllesten ned på ham fra Muren, så han fandt sin Død i Tebez? Hvorfor kom I Muren så nær?"
\par 23 Budet sagde til David: "Mændene var os overlegne og rykkede ud imod os på åben Mark, men vi trængte dem tilbage til Portens Indgang;
\par 24 så skød Bueskytterne oppe fra Muren på dine Trælle, og nogle af Kongens Trælle faldt; også din Træl Hetiten Urias faldt!"
\par 25 Da sagde David til Budet: "Sig til Joab: Du skal ikke græmme dig over den Ting; thi Sværdet fortærer snart den ene, snart den anden; fortsæt med Kraft Kampen mod Byen og riv den ned! Med de Ord skal du sætte Mod i ham!"
\par 26 Da Urias's Hustru hørte, at hendes Mand var faldet, holdt hun Dødeklage over sin Ægtefælle.
\par 27 Men da Sørgetiden var omme, lod David hende hente til sit Hus, og hun blev hans Hustru og fødte ham en Søn. Men det, David havde gjort, var ondt i HERRENs Øjne.

\chapter{12}

\par 1 Og HERREN sendte Natan til David. Da han kom ind til ham, sagde han: "Der boede to Mænd i samme By, en rig og en fattig.
\par 2 Den rige havde Småkvæg og Hornkvæg i Mængde,
\par 3 medens den fattige ikke ejede andet end et eneste lille Lam, som han havde købt og opdrættet, og som var vokset op hos ham sammen med hans Børn; det åd af hans Brød, drak af hans Bæger og lå i hans Skød og var ham som en Datter.
\par 4 Men da den rige Mand engang fik Besøg, ømmede han sig ved at tage noget af sit eget Småkvæg eller Hornkvæg og tillave det til den vejfarende Mand, som var kommet til ham, men tog den fattige Mands Lam og tillavede det til sin Gæst!"
\par 5 Da blussede Davids Vrede heftigt op mod den Mand, og han sagde til Natan: "Så sandt HERREN lever: Den Mand, som gjorde det, er dødsens,
\par 6 og Lammet skal han erstatte trefold, fordi han handlede så hjerteløst!"
\par 7 Men Natan sagde til David: "Du er Manden! Så siger HERREN, Israels Gud: Jeg salvede dig til Konge over Israel, og jeg friede dig af Sauls Hånd;
\par 8 jeg gav dig din Herres Hus, lagde din Herres Hustruer i din Favn og gav dig Israels og Judas Hus; og var det for lidet, vilde jeg have givet dig endnu mere, både det ene og det andet.
\par 9 Hvorfor har du da ringeagtet HERRENs Ord og gjort, hvad der er ondt i hans Øjne? Hetiten Urias har du dræbt med Sværdet; hans Hustru har du taget til Ægte, og ham har du slået ihjel med Ammoniternes Sværd.
\par 10 Så skal nu Sværdet aldrig vige fra dit Hus, fordi du ringeagtede mig og tog Hetiten Urias's Hustru til Ægte.
\par 11 Så siger HERREN: Se, jeg lader Ulykke komme over dig fra dit eget Hus, og jeg tager dine Hustruer bort for Øjnene af dig og giver dem til en anden, som skal ligge hos dine Hustruer ved højlys Dag.
\par 12 Thi du handlede i det skjulte, men jeg vil opfylde dette Ord i hele Israels Påsyn og ved højlys Dag!"
\par 13 Da sagde David til Natan: "Jeg har syndet mod HERREN!" Og Natan sagde til David: "Så har HERREN også tilgivet dig din Synd; du skal ikke dø.
\par 14 Men fordi du ved denne Gerning har vist Foragt for HERREN, skal Sønnen, som er født dig, visselig dø!"
\par 15 Derpå gik Natan til sit Hus. Og HERREN ramte det Barn, Urias's Hustru havde født David, med Sygdom.
\par 16 Da søgte David Gud for Barnet, holdt Faste og gik hen og lagde sig om Natten på Jorden i Sæk.
\par 17 De ældste i hans Hus kom til ham for at få ham til at rejse sig, men han vilde ikke, og han holdt ikke Måltid sammen med dem.
\par 18 Syvendedagen døde Barnet; men Davids Folk turde ikke lade ham vide, at Barnet var død, thi de tænkte: "Da Barnet levede, vilde han ikke låne os Øre, når vi talte til ham; hvor kan vi da nu sige til ham, at Barnet er død? Han kunde gøre en Ulykke!"
\par 19 Men da David så, at hans Folk hviskede sammen, skønnede han, at Barnet var død. Så spurgte David sine Folk: "Er Barnet død?" Og de svarede: "Ja, han er død!"
\par 20 Da rejste David sig fra Jorden, tvættede og salvede sig, tog andre Klæder på og gik ind i HERRENs Hus og bad; så gik han hjem, forlangte at få Mad sat frem og spiste.
\par 21 Da sagde hans Folk til ham: "Hvorledes er det dog, du bærer dig ad? Medens Barnet endnu levede, fastede du og græd; og nu da Barnet er død, rejser du dig og spiser!"
\par 22 Han svarede: "Så længe Barnet levede, fastede jeg og græd; thi jeg tænkte: Måske er HERREN mig nådig, så Barnet bliver i Live.
\par 23 Men hvorfor skulde jeg faste, nu han er død? Kan jeg bringe ham tilbage igen? Jeg går til ham, men han kommer ikke tilbage til mig!"
\par 24 Derpå trøstede David sin Hustru Batseba, og efter at han var gået ind til hende og havde ligget hos hende, fødte hun en Søn, som han kaldte Salomo, og ham elskede HERREN.
\par 25 Han overgav ham til Profeten Natan, og på HERRENs Ord kaldte denne ham Jedidja.
\par 26 Joab angreb imidlertid Rabba i Ammon og indtog Vandbyen.
\par 27 Derpå sendte Joab Bud til David og lod sige: "Jeg har angrebet Rabba og indtaget Vandbyen;
\par 28 kald nu Resten af Hæren sammen, så du kan belejre Byen og indtage den, for at det ikke skal blive mig, der indtager den og får mit Navn udråbt over den!"
\par 29 Da samlede David hele Hæren og drog til Rabba, angreb og indtog det.
\par 30 Og han tog Kronen af Milkoms Hoved; den var af Guld og vejede en Talent; der var en Ædelsten på den, og den blev sat på Davids Hoved. Et vældigt Bytte fra Byen førte han med sig,
\par 31 og Indbyggerne slæbte han bort, satte dem til Savene, Jernhakkerne - og Jernøkserne og lod dem trælle ved Teglovnene.

\chapter{13}

\par 1 Nogen Tid efter tildrog følgende sig. Davids Søn Absalon havde en smuk Søster, som hed Tamar, og Davids Søn Amnon fattede Kærlighed til hende.
\par 2 Amnon blev syg af Attrå efter sin Søster Tamar; thi hun var Jomfru, og Amnon øjnede ingen Mulighed for at få sin Vilje med hende.
\par 3 Men Amnon havde en Ven ved Navn Jonadab, en Søn af Davids Broder Sjim'a, og denne Jonadab var en såre klog Mand;
\par 4 han sagde til ham: "Hvorfor er du så elendig hver Morgen, Kongesøn? Vil du ikke sige mig det?" Amnon svarede: "Jeg elsker min Broder Absalons Søster Tamar!"
\par 5 Da sagde Jonadab til ham: "Læg dig til Sengs og lad, som du er syg! Når så din Fader kommer for at se til dig, skal du sige: Lad min Søster Tamar komme og give mig noget at spise! Når hun laver Maden i mit Påsyn, så at jeg kan se det, og hun selv giver mig den, kan jeg spise."
\par 6 Så gik Amnon til Sengs og lod. som han var syg; og da Kongen kom for at se til ham, sagde Amnon til Kongen: "Lad min Søster Tamar komme og lave et Par Kager i mit Påsyn og selv give mig dem: så kan jeg spise."
\par 7 David sendte da Bud ind i Huset til Tamar og lod sige: "Gå over til din Broder Amnons Hus og lav Mad til ham!"
\par 8 Og Tamar gik over til sin Broder Amnons Hus, hvor han lå til Sengs, tog Dejen, æltede den og lavede Kagerne i hans Påsyn og bagte dem;
\par 9 derpå tog hun Panden og hældte dem ud i hans Påsyn; Amnon vilde dog ikke spise, men sagde: "Lad alle gå udenfor!" Og da de alle var gået udenfor,
\par 10 sagde Amnon til Tamar: "Bær Maden ind i Inderværelset og lad mig få den af din egen Hånd!" Da tog Tamar Kagerne, som hun havde lavet, og bar dem ind i Inderværelset til sin Broder Amnon.
\par 11 Men da hun bar dem hen til ham, for at han skulde spise, greb han fat i hende og sagde: "Kom og lig hos mig, Søster!"
\par 12 Men hun sagde: "Nej, Broder! Krænk mig ikke! Således gør man ikke i Israel! Øv dog ikke denne Skændselsdåd!
\par 13 Hvor skulde jeg gå hen med min Skam? Og du vilde blive regnet blandt Dårer i Israel! Tal hellere med Kongen; han nægter dig ikke at få mig!"
\par 14 Han, vilde dog ikke høre hende, men tog hende med Vold, krænkede hende og lå hos hende.
\par 15 Men bagefter hadede Amnon hende med et såre stort Had; ja det Had, han følte mod hende, var større end den Kærlighed, han havde båret til hende. Og Amnon sagde til hende: "Stå op og gå din Vej!"
\par 16 Da sagde hun til ham: "Nej, Broder! Den Udåd, at du nu jager mig bort, er endnu større end den anden, du øvede imod mig!" Han vilde dog ikke høre hende,
\par 17 men kaldte på den unge Mand, der var hans Tjener, og sagde: "Få mig hende der ud af Huset og stæng Døren efter hende!"
\par 18 Hun bar en fodsid Kjortel med Ærmer; thi således klædte Jomfruerne blandt Kongedøtrene sig fordum. Tjeneren førte hende da ud af Huset og stængede Døren efter hende.
\par 19 Men Tamar strøede Aske på sit Hoved og sønderrev den fodside Kjortel, hun havde på, og tog sig til Hovedet og skreg ustandseligt, medens hun gik bort.
\par 20 Da sagde hendes Broder Absalon til hende: "Har din Broder Amnon været hos dig? Ti nu stille, Søster! Han er jo din Broder; tag dig ikke den Sag nær!" Tamar sad da ensom hen i sin Broder Absalons Hus.
\par 21 Da Kong David hørte alt dette, blev han meget vred; men han bebrejdede ikke sin Søn Amnon noget, thi han elskede ham, fordi han var hans førstefødte.
\par 22 Og Absalon talte ikke til Amnon, hverken ondt eller godt; thi Absalon hadede Amnon, fordi han havde krænket hans Søster Tamar.
\par 23 Men et Par År efter holdt Absalon Fåreklipning i Ba'al-Hazor, som ligger ved Efraim, og dertil indbød Absalon alle Kongesønnerne.
\par 24 Absalon kom til Kongen og sagde: "Se, din Træl holder Fåreklipning; vil ikke Kongen og hans Folk tage med din Træl derhen?"
\par 25 Men Kongen sagde til Absalon: "Nej, min Søn! Vi vil ikke alle gå med, for at vi ikke skal falde dig til Byrde!" Og skønt han nødte ham, vilde han ikke gå med, men tog Afsked med ham.
\par 26 Da sagde Absalon: "Så lad i alt Fald min Broder Amnon gå med!" Men Kongen sagde til ham: "Hvorfor skal han med?"
\par 27 Da Absalon nødte ham, lod han dog Amnon og de andre Kongesønner gå med. Og Absalon gjorde et kongeligt Gæstebud.
\par 28 Men Absalon gav sine Folk den Befaling: "Pas på, når Vinen er gået Amnon til Hovedet; når jeg så siger til eder: Hug Amnon ned! dræb ham så! Frygt ikke; det er mig, som befaler jer det. Tag Mod til jer og vis jer som kække Mænd!"
\par 29 Absalons Folk gjorde ved Amnon, som Absalon havde befalet. Da brød alle Kongesønnerne op, besteg deres Muldyr og flyede.
\par 30 Medens de endnu var undervejs, nåede det Rygte David: "Absalon har hugget alle Kongesønnerne ned, ikke en eneste er tilbage af dem!"
\par 31 Da stod Kongen op, sønderrev sine Klæder og lagde sig på Jorden; også alle hans Folk, som stod hos, sønderrev deres Klæder.
\par 32 Men Jonadab Davids Broder Sjim'as Søn, tog til Orde og sagde: "Min Herre må ikke tro, at de har dræbt alle de unge Kongesønner; kun Amnon er død, thi der har været noget ved Absalons Mund, som ikke varslede godt, lige siden den Dag Amnon krænkede hans Søster Tamar.
\par 33 Derfor må min Herre Kongen ikke, tage sig det nær og tro, at alle Kongesønnerne er døde. Kun Amnon er død!"
\par 34 Da den unge Mand, som holdt Udkig, så ud, fik han Øje på en Mængde Mennesker, som kom ned ad Skråningen på Vejen til Horonajim, og han gik ind og meldte Kongen: "Jeg kan se, der kommer Mennesker ned ad Bjergsiden på Vejen til Horonajim."
\par 35 Da sagde Jonadab til Kongen: "Der kommer Kongesønnerne; det er, som din Træl sagde!"
\par 36 Og som han havde sagt det, kom Kongesønnerne, og de brast i Gråd; også Kongen og alle hans folk brast i heftig Gråd.
\par 37 Men Absalon flygtede og begav sig til Kong Talmaj, Ammihuds Søn, i Gesjur. Og Kongen sørgede over sin Søn i al den Tid.
\par 38 Da Absalon flygtede, begav han sig til Gesjur, og der blev han tre År.
\par 39 Men Kongen begyndte at længes inderligt efter Absalon, thi han havde trøstet sig over Amnons Død.

\chapter{14}

\par 1 Da nu Joab, Zerujas Søn, mærkede, at Kongens Hjerte hang ved Absalon,
\par 2 sendte han Bud til Tekoa efter en klog Kvinde og sagde til hende: "Lad, som om du har Sorg, ifør dig Sørgeklæder, lad være at salve dig med Olie, men bær dig ad som en Kvinde, der alt i lang Tid har sørget over en afdød;
\par 3 gå så til Kongen og sig således til ham" - og Joab lagde hende Ordene i Munden.
\par 4 Kvinden fra Tekoa gik så til Kongen, faldt til Jorden på sit Ansigt, bøjede sig og sagde: "Hjælp Konge!"
\par 5 Kongen spurgte hende: "Hvad fattes dig?" Og hun sagde: "Jo, jeg er Enke, min Mand er død.
\par 6 Din Trælkvinde havde to Sønner; de kom i Klammeri ude på Marken, og der var ingen til at bilægge deres Tvist; så slog den ene den anden ihjel.
\par 7 Men nu træder hele Slægten op imod din Trælkvinde og siger: Udlever Brodermorderen, for at vi kan slå ham ihjel og hævne Broderen, som han dræbte! Således siger de for at få Arvingen ryddet af Vejen og slukke den sidste Glød, jeg har tilbage, så at min Mand ikke får Eftermæle eller Efterkommere på Jorden!"
\par 8 Da sagde Kongen til Kvinden: "Gå kun hjem, jeg skal jævne Sagen for dig!"
\par 9 Men kvinden fra Tekoa sagde til Kongen: "Lad Skylden komme over mig og mit Fædrenehus, Herre Konge, men Kongen og hans Trone skal være skyldfri!"
\par 10 Kongen sagde da: "Enhver, som vil dig noget, skal du bringe til mig, så skal han ikke mere volde dig Men!"
\par 11 Men Kvinden sagde: "Vilde dog Kongen nævne HERREN din Guds Navn, for at Blodhævneren ikke skal volde endnu mere Ulykke og min Søn blive ryddet af Vejen!" Da sagde han: "Så sandt HERREN lever, der skal ikke krummes et Hår på din Søns Hoved!"
\par 12 Da sagde Kvinden: "Må din Trælkvinde sige min Herre Kongen et Ord?" Han svarede: "Tal!"
\par 13 Og Kvinden sagde: "Hvor kan du da tænke på at gøre det samme ved Guds Folk - når Kongen taler således, dømmer han jo sig selv - og det gør Kongen, når han ikke vil lade sin forstødte Søn vende tilbage.
\par 14 Thi vi skal visselig alle dø, vi er som Vandet, der ikke kan samles op igen, når det hældes ud på Jorden; men Gud vil ikke tage det Menneskes Liv, der omgås med den Tanke, at en forstødt ikke skal være forstødt for stedse.
\par 15 Når jeg nu er kommet for at tale om denne Sag til min Herre Kongen, så er det, fordi de Folk har gjort mig bange; din Trælkvinde tænkte: Jeg vil tale til Kongen, måske opfylder Kongen sin Trælkvindes Bøn;
\par 16 thi Kongen vil bønhøre mig og fri sin Trælkvinde af den Mands Hånd, som tragter efter at udrydde mig tillige med min Søn af Guds Arvelod.
\par 17 Og din Trælkvinde tænkte: Min Herre Kongens Ord vil være mig en Lindring; thi min Herre Kongen er som en Guds Engel til at skønne over godt og ondt! HER- REN din Gud være med dig!"
\par 18 Da sagde Kongen til Kvinden: "Svar mig på det Spørgsmål, jeg nu stiller dig, dølg ikke noget!"Kvinden sagde: "Min Herre Kongen tale!"
\par 19 Da sagde Kongen: "Har Joab en Finger med i alt dette?" Og Kvinden svarede: "Så sandt du lever, Herre Konge, det er umuligt at slippe uden om, hvad min Herre Kongen siger. Ja, det var din Træl Joab, som pålagde mig dette og lagde din Trælkvinde alle disse Ord i Munden.
\par 20 For at give Sagen et andet Udseende har din Træl Joab gjort således. Men min Herre er viis som en Guds Engel, så han ved alle Ting på Jorden!"
\par 21 Derpå sagde Kongen til Joab: "Vel, jeg vil gøre det! Gå hen og bring den unge Mand, Absalon, tilbage!"
\par 22 Da faldt Joab på sit Ansigt til Jorden og bøjede sig og velsignede Kongen og sagde: "Nu ved din Træl, at jeg har fundet Nåde for dine Øjne, Herre Konge, siden Kongen har opfyldt sin Træls Bøn!"
\par 23 Derpå begav Joab sig til Gesjur og hentede Absalon tilbage til Jerusalem.
\par 24 Men Kongen sagde: "Lad ham gå hjem til sit Hus; for mit Åsyn bliver han ikke stedet!" Da gik Absalon hjem til sit Hus, og for Kongens Åsyn blev han ikke stedet.
\par 25 Men ingen Mand i hele Israel blev beundret så højt for sin Skønhed som Absalon; fra Fodsål til Isse var der ikke en Lyde ved ham.
\par 26 Og når han lod sit Hår klippe - han lod det klippe, hver Gang der var gået et År, fordi det blev ham for tungt, derfor måtte han lade det klippe - vejede det 2OO Sekel efter kongelig Vægt.
\par 27 Der fødtes Absalon tre Sønner og een Datter ved Navn Tamar; hun var en smuk Kvinde.
\par 28 Absalon boede nu to År i Jerusalem uden at blive stedet for Kongen.
\par 29 Da sendte Absalon Bud efter Joab for at få ham til at gå til Kongen; men han vilde ikke komme. Han sendte Bud een Gang til, men han vilde ikke komme.
\par 30 Da sagde han til sine Folk "Se den Bygmark, Joab har der ved Siden af min! Gå hen og stik Ild på den!" Og Absalons Folk stak Ild på Marken.
\par 31 Da begav Joab sig ind til Absalon og sagde til ham: "Hvorfor har dine Folk stukket Ild på min Mark?"
\par 32 Absalon svarede Joab: "Se, jeg sendte Bud efter dig og bad dig komme herhen, for at jeg kunde sende dig til Kongen og sige: Hvorfor kom jeg tilbage fra Gesjur? Det havde været bedre for mig, om jeg var blevet der! Men nu vil jeg stedes for Kongen; er der Skyld hos mig så lad ham dræbe mig!"
\par 33 Joab gik da til Kongen og overbragte ham disse Ord. Så lod han Absalon kalde, og han kom ind til Kongen; og han bøjede sig for ham og faldt på sit Ansigt til Jorden for Kongen. Så kyssede Kongen Absalon.

\chapter{15}

\par 1 Men nogen tid efter skaffede Absalon sige Vogn og Heste og halvtredsindstyve Forløbere;
\par 2 og om Morgenen stillede han sig ved Portvejen, og når nogen gik til Kongen for af få en Retssag afgjort, kaldte Absalon ham til sig og spurgte ham: "Hvilken By er du fra?" Når han da svarede: "Din Træl er fra den eller den af Israels Stammer!"
\par 3 så sagde Absalon til ham: "Ja, din Sag er god og retfærdig; men hos Kongen finder du ikke Øre!"
\par 4 Og Absalon tilføjede: "Vilde man blot sætte mig til Dommer i Landet! Da måtte enhver, der har en Retssag eller Retstrætte, komme til mig, og jeg vilde hjælpe ham til hans Ret."
\par 5 Og når nogen nærmede sig for at kaste sig ned for ham, rakte han Hånden ud og holdt ham fast og kyssede ham.
\par 6 Således gjorde Absalon over for alle Israeliterne, som kom til Kongen for at få deres Sager afgjort, og Absalon stjal Israels Mænds Hjerte.
\par 7 Da der var gået fire År, sagde Absalon til Kongen: "Lad mig få Lov at gå til Hebron og indfri et Løfte, jeg har aflagt HERREN;
\par 8 thi medens din Træl boede i Gesjur i Aram, aflagde jeg det Løfte: Hvis HERREN lader mig komme tilbage til Jerusalem, vil jeg ære HERREN i Hebron!"
\par 9 Kongen svarede ham: "Gå med Fred!" Og han begav sig til Hebron.
\par 10 Men Absalon havde i al Hemmelighed sendt Bud ud i alle Israels Stammer og ladet sige: "Når I hører, der stødes i Horn, så skal I råbe: Absalon er blevet Konge i Hebron!"
\par 11 Og med Absalon fulgte fra Jerusalem 200 Mænd, som han havde indbudt, og som drog med i god Tro uden at vide af noget.
\par 12 Og da Absalon ofrede, Slagtofre, lod han Giloniten Akitofel, Davids Rådgiver, hente i hans By Gilo. Og Sammensværgelsen vandt i Styrke, idet flere og flere af Folket gik over til Absalon.
\par 13 Da kom en og meldte David det og sagde: "Israels Hu har vendt sig til Absalon!"
\par 14 Og David sagde til alle sine folk, som var hos ham i Jerusalem: "Kom, lad os flygte; ellers kan vi ikke undslippe Absalon; skynd jer af Sted, at han ikke skal skynde sig og nå os, bringe ulykke over os og nedhugge Byens Indbyggere med Sværdet!"
\par 15 Kongens Folk svarede: "Dine Trælle er rede til at gøre alt, hvad du finder rigtigt, Herre Konge!"
\par 16 Så drog Kongen ud, fulgt af hele sit Hus; dog lod Kongen ti Medhustruer blive tilbage for at se efter Huset.
\par 17 Så drog Kongen ud, fulgt af alle sine Folk. Ved det sidste Hus gjorde de Holdt,
\par 18 og alle Krigerne gik forbi ham, ligeledes alle Kreterne og Pleterne; også alle Gatiten Ittajs Mænd, 600 Mand, som havde fulgt ham fra Gat, gik forbi Kongen.
\par 19 Da sagde Kongen til Gatiten Ittaj: "Hvorfor går også du med? Vend om og bliv hos Kongen; thi du er Udlænding og er vandret ud fra din Hjemstavn;
\par 20 i Går kom du, og i Dag skulde jeg tage dig med på vor Omflakken, jeg, som går uden at vide hvorhen! Vend tilbage og tag dine Landsmænd med; HERREN vise dig Miskundhed og Trofasthed!"
\par 21 Men Ittaj svarede Kongen: "Så sandt HERREN lever, og så sandt du, Herre Konge, lever: Hvor du, Herre Konge, er, der vil din Træl være, hvad enten det bliver Liv eller Død!"
\par 22 Da sagde David til Ittaj: "Vel, så drag forbi!" Så drog Gatiten Ittaj forbi med alle sine Mænd og hele sit Følge af Kvinder og Børn.
\par 23 Hele Landet græd højt, medens alle Krigerne gik forbi; og Kongen stod i Kedrons Dal, medens alle Krigerne gik forbi ham ad Vejen til Oliventræet i Ørkenen.
\par 24 Også Zadok og Ebjatar, som bar Guds Pagts Ark, kom til Stede; de satte Guds Ark ned og lod den stå, indtil alle Krigerne fra Byen var gået forbi.
\par 25 Da sagde Kongen til Zadok: "Bring Guds Ark tilbage til Byen! Hvis jeg finder Nåde for HERRENs Øjne, fører han mig tilbage og lader mig stedes for ham og hans Bolig;
\par 26 siger han derimod: Jeg har ikke Behag i dig! se, da er jeg rede; han gøre med mig, hvad ham tykkes godt!"
\par 27 Og Kongen sagde til Præsten Zadok: "Se, du og Ebjatar skal med Fred vende tilbage til Byen tillige med eders to Sønner, din Søn Ahima'az og Ebjatars Søn Jonatan!
\par 28 Se, jeg bier ved Vadestederne på Jordansletten, indtil jeg får Bud fra eder med Efterretning."
\par 29 Zadok og Ebjatar bragte da Guds Ark tilbage til Jerusalem, og de blev der.
\par 30 Men David gik grædende op ad Oliebjerget med tilhyllet Hoved og bare Fødder, og alle Krigerne, som fulgte ham, havde tilhyllet deres Hoveder og gik grædende opefter.
\par 31 Da David fik at vide, at Akitofel var iblandt de sammensvorne, som holdt med Absalon, sagde han: "Gør Akitofels Råd til Skamme, HERRE!"
\par 32 Da David var kommet til Bjergets Top, hvor man plejede at tilbede Gud, kom Arkiten Husjaj, Davids Ven, ham i Møde med sønderrevet Kjortel og Jord på Hovedet.
\par 33 Da sagde David til ham: "Hvis du drager med, bliver du mig til Byrde;
\par 34 men vender du tilbage til Byen og siger til Absalon: Jeg vil være din Træl, Konge; din Faders Træl var jeg fordum, men nu vil jeg være din Træl! så kan du gøre mig Akitofels Råd til Skamme.
\par 35 Der har du jo Præsterne Zadokog Ebjatar; alt, hvad du hører fra Kongens Palads, må du give Præsterne Zadok og Ebjatar Nys om.
\par 36 Se, de har der deres to Sønner hos sig, Zadoks Søn Ahima'az og Ebjatars Søn Jonatan; send mig gennem dem Bud om alt, hvad I hører."
\par 37 Så kom Husjaj, Davids Ven, til Byen, og samtidig kom Absalon til Jerusalem.

\chapter{16}

\par 1 Da David var kommet lidt på den anden Side af Bjergets Top, Kom Mefibosjets Tjener Ziba ham i Møde med et Par opsadlede Æsler, som bar 200 Brød, 100 Rosinkager, 100 Frugter og en Dunk Vin.
\par 2 Da sagde Kongen til Ziba: "Hvad vil du med det?" Og Ziba svarede: "Æslerne er bestemt til Ridedyr for Kongens Hus, Brødene og Frugterne til Spise for Folkene og Vinen til Drikke for dem, der bliver trætte i Ørkenen!"
\par 3 Så sagde Kongen: '"Hvor er din Herres Søn?" Ziba svarede Kongen: "Han blev i Jerusalem; thi han tænkte: Nu vil Israels Hus give mig min Faders Kongedømme tilbage!"
\par 4 Da sagde Kongen til Ziba: "Dig skal hele Mefibosjets Ejendom tilhøre!" Og Ziba sagde: "Jeg bøjer mig dybt! Måtte jeg finde Nåde for min Herre Kongens Øjne!"
\par 5 Men da Kong David kom til Bahurim, se, da kom en Mand ved Navn Simei, Geras Søn, af samme Slægt som Sauls Hus, gående ud af Byen, alt imedens han udstødte Forbandelser,
\par 6 og han kastede Sten efter David og alle Kong Davids Folk, skønt alle Krigerne og alle Kærnetropperne gik på begge Sider af ham.
\par 7 Og Simei forbandede ham med de Ord: "Bort, bort med dig, din Blodhund, din Usling!
\par 8 HERREN har nu bragt alt Sauls Hus's Blod over dig, han, i hvis Sted du blev Konge, og HERREN har nu givet din Søn Absalon Kongedømmet; nu har Ulykken ramt dig. for,i du er en Blodhund!"
\par 9 Da sagde Abisjaj, Zerujas Søn, til Kongen: "Hvorfor skal den døde Hund have Lov at forbande min Herre Kongen? Lad mig gå hen og hugge Hovedet af ham!"
\par 10 Men Kongen svarede: "Hvad har jeg med eder at gøre, Zerujasønner! Når han forbander, og når HERREN har budt ham at forbande David, hvem tør da sige: Hvorfor gør du det?"
\par 11 Og David sagde til Abisjaj og alle sine Folk: "Når min egen Søn, som er udgået af min Lænd, står mig efter Livet, hvad kan man da ikke vente af denne Benjaminit! Lad ham kun forbande, når HERREN har budt ham det!
\par 12 Måske vil HERREN se til mig i min Nød og gøre mig godt til Gengæld for hans Forbandelse i Dag!"
\par 13 Derpå gik David med sine Mænd hen ad Vejen, medens Simei fulgte ham oppe på Bjergskråningen og stadig udstødte Forbandelser, slog med Sten og kastede Støv efter ham.
\par 14 Således kom Kongen og alle Krigerne, som fulgte ham, udmattede til Jordan og hvilede ud der.
\par 15 Imidlertid var Absalon draget ind i Jerusalem med alle Israels Mænd, og Akitofel var hos ham.
\par 16 Da nu Arkiten Husjaj, Davids Ven, kom til Absalon, sagde han til ham: "Kongen leve, Kongen leve!"
\par 17 Absalon sagde til Husjaj: "Er det sådan, du viser din Ven Godhed? Hvorfor fulgte du ikke din Ven?"
\par 18 Husjaj svarede Absalon: "Nej, den, som HERREN og dette Folk og alle Israels Mænd har valgt, i hans Tjeneste vil jeg træde, og hos ham vil jeg blive!
\par 19 Og desuden: Hvem er det, jeg tjener? Mon ikke hans Søn? Som jeg har tjent din Fader, vil jeg tjene dig!"
\par 20 Absalon sagde så til Akitofel: "Kom med eders Råd! Hvad skal vi gøre?"
\par 21 Akitofel svarede Absalon: "Gå ind til din Faders Medhustruer, som han har ladet blive tilbage for at se efter Huset; så kan hele Israel skønne, at du har lagt dig for Had hos din Fader, og alle de, der har sluttet sig til dig, vil få nyt Mod!"
\par 22 Absalons Telt blev så rejst på Taget, og Absalon gik ind til sin Faders Medhustruer i hele Israels Påsyn.
\par 23 Det Råd, Akitofel gav i de Tider, gjaldt nemlig lige så meget, som når man adspurgte Gud; så meget gjaldt ethvert Råd af Akitofel både hos David og Absalon.

\chapter{17}

\par 1 Derpå sagde Akitofel til Absalon; " Lad mig udvælge 12000 Mand og bryde op i Nat og sætte efter David.
\par 2 Når jeg overfalder ham, medens han er udmattet og modfalden, kan jeg indjage ham Skræk, og alle hans Folk vil flygte, så at jeg kan fælde Kongen uden at røre nogen anden;
\par 3 så bringer jeg hele Folket tilbage til dig, som en Brud vender tilbage til sin Mand. Du attrår jo dog kun en enkelt Mands Liv, og hele Folket vil da være uskadt!"
\par 4 Det Forslag tiltalte Absalon og alle Israels Ældste.
\par 5 Men Absalon sagde: "Kald dog også Arkiten Husjaj hid, for at vi også kan høre, hvad han råder til!"
\par 6 Da Husjaj kom ind, sagde Absalon til ham: "Det og det har Akitofel sagt; skal vi følge hans Råd? Hvis ikke, så sig du din Mening!"
\par 7 Husjaj svarede Absalon: "Denne Gang er Akitofels Råd ikke godt!"
\par 8 Og Husjaj sagde fremdeles: "Du ved, at din Fader og hans Mænd er Helte, og bitre i Hu er de som en Bjørn på Marken, hvem Ungerne er taget fra; desuden er din Fader en rigtig Kriger, som ikke lægger sig til Hvile om Natten med Folkene.
\par 9 For Øjeblikket holder han sig sikkert skjult i en Kløft eller et andet Sted; falder der nu straks i Begyndelsen nogle af Folkene, vil det rygtes, og man vil sige, at Absalons Tilhængere har lidt Nederlag;
\par 10 og da bliver selv den tapre, hvis Mod er som Løvens, forsagt; thi hele Israel ved, at din Fader er en Helt og hans Ledsagere tapre Mænd.
\par 11 Mit Råd er derfor: Lad hele Israel fra Dan til Be'ersjeba samles om dig, talrigt som Sandet ved Havet, og drag selv med i deres Midte.
\par 12 Støder vi så på ham et eller andet Sted, hvor han nu befinder sig, kan vi falde over ham som Dug over Jorden, og der skal ikke blive en eneste tilbage, hverken han eller nogen af alle hans Mænd;
\par 13 men kaster han sig ind i en By, skal hele Israel lægge Reb om den, og vi vil slæbe den ned i Dalen, så der ikke bliver Sten på Sten tilbage af den!" Da sagde Absalon og alle Israels Mænd: "Arkiten Husjajs Råd er bedre end Akitofels!"
\par 14 HERREN havde nemlig sat sig for at gøre Akitofels gode Råd til Skamme, for at HERREN kunde bringe Ulykke over Absalon.
\par 15 Derpå sagde Husjaj til Præsterne Zadok og Ebjatar: "Det og det Råd har Akitofel givet Absalon og Israels Ældste, og det og det Råd har jeg givet.
\par 16 Skynd eder nu at sende Bud til David og bring ham det Bud: Bliv ikke Natten over ved vadestederne på Jordansletten, men søg over på den anden Side, for at ikke Kongen og alle hans Folk skal gå til Grunde!"
\par 17 Jonatan og Ahima'az stod ved Rogelkilden, hvorhen en Tjenestepige til Stadighed kom og bragte dem Melding, hvorefter de gik hen og bragte Kong David Melding; thi de turde ikke vise sig i Byen.
\par 18 Men en ung Mand opdagede dem og meldte det til Absalon; så skyndte de sig begge bort og kom ind hos en Mand i Bahurim. Han havde en Brønd i Gården, og i den steg de ned;
\par 19 Konen tog et Tæppe, bredte det ud over Brønden og hældte Korn derpå, så at man intet kunde opdage.
\par 20 Nu kom Absalons Folk imod i Huset til Konen og spurgte: "Hvor er Ahima'az og Jonatan?" Konen svarede: "De gik over Vandbækken!" Så søgte de efter dem, men da de ikke fandt dem, vendte de tilbage til Jerusalem.
\par 21 Så snart de var gået bort, steg de to op af Brønden og gik hen og bragte Kong David Melding; og de sagde til David: "Bryd op og skynd eder over på den anden Side af Vandet; thi det og det Råd kom Akitofel med angående eder!"
\par 22 Da brød David op med alle sine Folk og satte over Jordan; og ved Daggry manglede ikke en eneste, alle var de kommet over Jordan.
\par 23 Men da Akitofel så, at hans Råd ikke blev fulgt, sadlede han sit Æsel og drog hjem til sin By; og efter at have beskikket sit Hus hængte han sig og døde. Han blev jordet i sin Faders Grav.
\par 24 David havde nået Mahanajim, da Absalon tillige med alle Israels Mænd gik over Jordan.
\par 25 Absalon havde i Joabs Sted sat Amasa over Hæren; Amasa var Søn af en Ismaelit ved Navn Jitra, som var gået ind til Abigajil, en Datter af Isaj og Søster til Joabs Moder Zeruja.
\par 26 Og Israel og Absalon slog Lejr i Gilead.
\par 27 Men da David kom til Mahaoajim, bragte Sjobi, Nahasj's Søn fra Rabba i Ammon, Makir, Ammiels Søn fra Lodebar, og Gileaditen Barzillaj fra Rogelim
\par 28 Senge, Tæpper, Skåle og Lerkar; og Hvede, Byg, Mel, ristet Korn, Bønner, Linser,
\par 29 Honning, Surmælk, Småkvæg og Komælksost bragte de David og hans Folk til Føde; thi de tænkte: "Folkene er sultne, udmattede og tørstige i Ørkenen."

\chapter{18}

\par 1 Derpå holdt David Mønstring over sit Mandskab og satte Tusindførere og Hundredførere over dem;
\par 2 og David delte Mandskabet i tre Dele; den ene Tredjedel stillede han under Joab, den anden under Abisjaj, Zerujas Søn og Joabs Broder, og den sidste Tredjedel under Gatiten Ittaj. Kongen sagde så til Folkene: "Jeg vil selv drage med i Kampen!"
\par 3 Men de svarede: "Du må ikke drage med; thi om vi flygter, ænser man ikke os; ja, selv om Halvdelen af os falder, ænser man ikke os, men du gælder lige så meget som ti Tusinde af os; derfor er det bedst, at du holder dig rede til at ile os til Hjælp fra Byen."
\par 4 Kongen svarede: "Jeg gør, hvad I finder bedst!" Derpå stillede Kongen sig ved Porten, medens hele Mandskabet drog ud, hundredvis og tusindvis.
\par 5 Men Kongen gav Joab, Abisjajog Ittaj den Befaling: "Far nu lempeligt med den unge Absalon!" Og hele Mandskabet hørte, hvorledes Kongen gav alle Øversterne Befaling om Absalon.
\par 6 Derpå rykkede Krigerne i Marken mod Israel, og Slaget kom til at stå i Efraims Skov.
\par 7 Der blev Israels Hær slået af Davids Folk, og der fandt et stort Mandefald Sted den Dag; der faldt 200000 Mand.
\par 8 Kampen bredte sig over hele Egnen, og Skoven fortærede den Dag flere Folk end Sværdet.
\par 9 Absalon selv stødte på nogle af Davids Folk; Absalon red på sit Muldyr, og da Muldyret kom ind under en stor Terebintes tætte Grene, blev hans Hoved hængende i Terebinten, så han hang mellem Himmel og Jord, medens Muldyret, han sad på, løb bort.
\par 10 En Mand, der så det meldte det til Joab og sagde: "Jeg så Absalon hænge i en Terebinte."
\par 11 Da sagde Joab til Manden, der havde meldt ham det: "Når du så det, hvorfor slog du ham da ikke til Jorden med det samme? Så havde jeg givet dig ti Sekel Sølv og et Bælte!"
\par 12 Men Manden svarede Joab: "Om jeg så havde fået tilvejet tusind Sekel, havde jeg ikke lagt Hånd på Kongens Søn; thi vi hørte jo selv, hvorledes Kongen bød dig, Abisjaj og Ittaj: Vogt vel på den unge Absalon!
\par 13 Dersom jeg havde handlet svigefuldt imod ham -intet bliver jo skjult for Kongen - vilde du have ladet mig i Stikken!"
\par 14 Da sagde Joab: "Så gør jeg det for dig!" Dermed greb han tre Spyd og stødte dem i Brystet på Absalon, som endnu levede og hang mellem Terebintens Grene.
\par 15 Derpå trådte ti unge Mænd, der var Joabs Våbendragere, til og gav Absalon Dødsstødet.
\par 16 Joab lod nu støde i Hornet, og Hæren opgav at forfølge Israel, thi Joab bød Folkene standse.
\par 17 Derpå tog de og kastede Absalon i en stor Grube i Skoven og ophobede en mægtig Stendynge over ham. Og hele Israel flygtede hver til sit.
\par 18 Men medens Absalon endnu levede, havde han taget og rejst sig den Stenstøtte, som står i Kongedalen; thi han sagde: "Jeg har ingen Søn til at bevare Mindet om mit Navn." Han havde opkaldt Stenstøtten efter sig selv, og endnu den Dag i Dag kaldes den "Absalons Minde".
\par 19 Da sagde Ahima'az, Zadoks Søn: "Lad mig løbe hen og bringe Kongen den gode Tidende, at HERREN har skaffet ham Ret over for hans Fjender!"
\par 20 Men Joab svarede ham: "Du skal ikke være den, der bringer Bud i Dag; en anden Gang kan du bringe Bud, men i bag skal du ikke gøre det, da Kongens Søn er død!"
\par 21 Og Joab sagde til Etiopieren: "Gå du hen og meld Kongen, hvad du har set!" Da kastede Etiopieren sig til Jorden for Joab og ilede af Sted.
\par 22 Men Ahima'az, Zadoks Søn, sagde atter til Joab: "Ske, hvad der vil! Lad også mig løbe bag efter Etiopieren!" Da sagde Joab: "Hvorfor vil du det, min Søn? For dig er der ingen Budløn at hente!"
\par 23 Men han blev ved: "Ske, hvad der vil! Jeg løber!" Så sagde han: "Løb da!" Og Ahima'az løb ad Vejen gennem Jordanegnen og nåede frem før Etiopieren.
\par 24 David sad just mellem de to Porte, og Vægteren steg op på Porttaget ved Muren; da han så ud, se, da kom en Mand løbende alene.
\par 25 Vægteren råbte og meldte det til Kongen, og Kongen sagde: "Er han alene, har han Bud at bringe!" Men medens Manden fortsatte sit Løb og kom nærmere,
\par 26 så Vægteren en anden Mand komme løbende og råbte ned i Porten: "Der kommer een Mand til løbende alene!" Kongen sagde: "Også han har Bud at bringe!"
\par 27 Da råbte Vægteren: "Den forreste løber således, at det ser ud til at være Ahima'az, Zadoks Søn!" Kongen sagde: "Det er en god Mand, han kommer med godt Bud!"
\par 28 Imidlertid var Ahima'az kommet nærmere og råbte til Kongen: "Hil dig!" Så kastede han sig ned på Jorden for Kongen og sagde: "Lovet være HERREN din Gud, som gav dem, der løftede Hånd mod min Herre Kongen, i din Hånd!"
\par 29 Kongen spurgte: "Er den unge Absalon uskadt?" Ahima'az svarede: "Jeg så, at der var stor Tummel, da Kongens Træl Joab sendte din Træl af Sted, men jeg ved ikke, hvad det var."
\par 30 Kongen sagde da: "Træd til Side og stil dig der!" Og han trådte til Side og blev stående.
\par 31 I det samme kom Etiopieren; og Etiopieren sagde: "Der er Bud til min Herre Kongen: HERREN har i Dag skaffet dig Ret over for alle dine Modstandere!"
\par 32 Kongen spurgte Etiopieren: "Er den unge Absalon uskadt?" Og han svarede: "Det gå min Herre Kongens Fjender og alle dine Modstandere som den unge Mand!"
\par 33 Da blev Kongen dybt rystet, og han gik op i Stuen på Taget over Porten og græd; og han gik frem og tilbage og klagede: Min Søn Absalon, min Søn, min Søn Absalon! Var jeg blot død i dit Sted! Absalon, min Søn, min Søn!"

\chapter{19}

\par 1 Joab fik nu Efterretning om, at Kongen græd og sørgede over Absalon,
\par 2 og Sejren blev den Dag til Sorg for alt Folket, fordi det hørte, at Kongen sørgede dybt over sin Søn.
\par 3 Og Folket stjal sig den Dag ind i Byen, som man stjæler sig bort af Skam, når man har taget Flugten i Kampen.
\par 4 Kongen havde tilhyllet sit Ansigt og klagede højt: "Min Søn Absalon, min Søn, Absalon, min Søn!"
\par 5 Da gik Joab ind til Kongen og sagde: "Du beskæmmer i Dag alle dine Folk, der dog i bag har reddet dit Liv og dine Sønners og Døtres, Hustruers og Medhustruers Liv,
\par 6 siden du elsker dem, som hader dig, og hader dem, som elsker dig; thi i dag viser du, at Øverster og Folk er intet for dig.
\par 7 Stå nu op og gå ud og tal godt for dine Folk; thi jeg sværger ved HERREN, at hvis du ikke gør det, bliver ikke en eneste Mand hos dig Natten over, og dette vil volde dig større Ulykke end alt, hvad der har ramt dig fra din Ungdom af og til nu!"
\par 8 Så stod Kongen op og satte sig i Porten; og da man fik at vide, at Kongen sad i Porten, kom alt Folket hen og stillede sig foran Kongen. Men efter at Israeliterne var flygtet hver til sit,
\par 9 begyndte alt Folket i alle Israels Stammer at gå i Rette med hverandre, idet de sagde: "Kongen frelste os fra vore Fjenders Hånd; det var ham, som reddede os af Filisternes Hånd; og nu har han måttet rømme Landet for Absalon.
\par 10 Men Absalon, som vi havde salvet til Konge over os, er faldet i Kampen. Hvorfor tøver l da med at føre Kongen tilbage?"
\par 11 Men da alle Israeliternes Ord kom Kong David for Øre, sendte han Bud til Præsterne Zadok og Ebjatar og lod sige: "Tal til Judas Ældste og sig: Hvorfor vil I være de sidste til at føre Kongen tilbage til hans Hus?
\par 12 I er jo mine Brødre, I er mit Kød og Blod. Hvorfor vil I være de sidste til at føre Kongen tilbage?
\par 13 Og til Amasa skal l sige: Er du ikke mit Kød og Blod? Gud ramme mig både med det ene og det andet, om du ikke for stedse skal være min Hærfører i Joabs Sted!"
\par 14 Så vendte alle Judas Mænds Hjerter sig til ham, alle som een, og de sendte Bud til Kongen: "Vend tilbage med alle dine Folk!"
\par 15 Og da Kongen på Hjemvejen kom til Jordan, var Judæerne kommet til Gilgal for at gå Kongen i Møde og føre ham over Jordan.
\par 16 Da skyndte Benjaminiten Simei, Geras Søn, fra Bahurim sig sammen med Judas Mænd ned for at gå Kong David i Møde,
\par 17 fulgt af tusind Mænd fra Benjamin. Også Ziba, som var Tjener i Sauls Hus, var med sine femten Sønner og tyve Trælle ilet til Jordan forud for Kongen,
\par 18 og de var sat over Vadestedet for at sætte Kongens Hus over og være ham til Tjeneste. Men da Kongen skulde til at gå over Floden kastede Simei, Geras Søn, sig ned for ham
\par 19 og sagde: "Min Herre tilregne mig ikke min Brøde og tænke ikke mere på, hvad din Træl forbrød, den Dag min Herre Kongen drog bort fra Jerusalem ' Kongen agte ikke derpå;
\par 20 thi din Træl ved, at han har syndet, men se, jeg er i Dag den første af hele Josefs Hus, der er kommet herned for at gå min Herre Kongen i Møde!"
\par 21 Da tog Abisjaj, Zerujas Søn, Ordet og sagde: "Skal Simei ikke lide Døden til Straf for, at han forbandede HERRENs Salvede?"
\par 22 Men David svarede: "Hvad er der mig og eder imellem, I Zerujasønner, at I vil være mine Modstandere i Dag? Skulde nogen i Israel lide Døden i Dag? Ved jeg da ikke, at jeg nu er Konge over Israel?"
\par 23 Derpå sagde Kongen til Simei: "Du skal ikke dø!" Og Kongen tilsvor ham det.
\par 24 Også Mefibosjet, Sauls Sønnesøn, var draget ned for at gå Kongen i Møde. Han havde ikke plejet sine Fødder eller sit Skæg eller tvættet sine Klæder, fra den Dag Kongen gik bort, til den Dag han kom uskadt tilbage.
\par 25 Da han nu kom fra Jerusalem for at gå Kongen i Møde, sagde Kongen til ham: "Hvorfor fulgte du mig ikke, Mefibosjet?"
\par 26 Han svarede: "Herre Konge, min Træl bedrog mig; thi din Træl bød ham sadle mit Æsel, for at jeg kunde sidde op og følge Kongen; din Træl er jo lam;
\par 27 men i Stedet bagtalte han din Træl hos min Herre Kongen. Dog, min Herre Kongen er jo som en Guds Engel. Gør, hvad du finder for godt!
\par 28 Thi skønt hele mit Fædrenehus kun havde Døden at vente af min Herre Kongen, gav du din Træl Plads imellem dine Bordfæller; hvad Ret har jeg da endnu til at kræve noget eller anråbe Kongen?"
\par 29 Da sagde Kongen til ham: "Hvorfor bliver du ved med at tale? Her er mit Ord: Du og Ziba skal dele Jordegodset!"
\par 30 Mefibosjet svarede Kongen: "Han må gerne få det hele, nu min Herre Kongen er kommet uskadt hjem!"
\par 31 Også Gileaditen Barzillaj drog ned fra Rogelim og fulgte med Kongen for at ledsage ham til Jordan.
\par 32 Barzillaj var en Olding på firsindstyve År; det var ham, som havde sørget for Kongens Underhold, medens han var i Mahanajim, thi han var en meget velstående Mand.
\par 33 Kongen sagde nu til Barzillaj: "Følg med mig, jeg vil sørge for, at du i din Alderdom får dit Underhold hos mig i Jerusalem!"
\par 34 Men Barzillaj svarede Kongen: "Hvor lang Tid har jeg endnu tilbage, at jeg skulde følge med Kongen op til Jerusalem?
\par 35 Jeg er nu firsindstyve År gammel; mon jeg kan skelne mellem godt og ondt, eller mon din Træl har nogen Smag for, hvad jeg spiser eller drikker, mon jeg endnu har Øre for Sangeres og Sangerinders Røst? Hvorfor skulde din Træl da i Fremtiden falde min Herre Kongen til Byrde?
\par 36 Kun det lille Stykke Vej til Jordan vilde din Træl ledsage Kongen; hvorfor vil Kongen give mig så meget til Gengæld?
\par 37 Lad din Træl vende tilbage, at jeg kan dø i min egen By ved mine Forældres Grav! Men her er din Træl Kimham; lad ham følge med min Herre Kongen, og gør med ham, hvad dig tykkes bedst!"
\par 38 Da sagde Kongen: "Kimham skal følge med mig, og jeg vil gøre med ham, hvad dig tykkes bedst; alt, hvad du ønsker, vil jeg gøre for dig!"
\par 39 Derpå gik alle Krigerne over Jordan, medens Kongen blev stående; og Kongen kyssede Barzillaj og velsignede ham, hvorefter Barzillaj vendte tilbage til sit Hjem.
\par 40 Så drog Kongen over til Gilgal, og Kimham drog med ham. Hele Judas Folk fulgte med Kongen og desuden Halvdelen af Israels Folk.
\par 41 Men nu kom alle Israeliterne til Kongen og sagde: "Hvorfor har vore Brødre, Judas Mænd, bortført dig og bragt Kongen og hans Hus over Jordan tillige med alle Davids Mænd?"
\par 42 Da svarede alle Judas Mænd Israels Mænd: "Kongen står jo os nærmest; hvorfor er I vrede over det? Har vi levet af Kongen eller taget noget fra ham?"
\par 43 Israels Mænd svarede Judas Mænd: "Vi har ti Gange Part i kongen, og tilmed har vi Førstefødselsretten fremfor eder; hvorfor har I da tilsidesat os? Og var det ikke os, der først talte om at føre vor Konge tilbage?" Men Judas Mænds Svar var endnu hårdere end Israels Mænds.

\chapter{20}

\par 1 Nu var der tilfældigvis en slet Person ved navn Sjeba, Bikris Søn, en Benjaminit. Han stødte i Hornet og sagde: "Vi har ingen Del i David, ingen Lod i Isajs Søn! Hver Mand til sine Telte, Israel!"
\par 2 Da faldt alle Israels Mænd fra David og gik over til Sjeba, Bikris Søn, medens Judas Mænd trofast fulgle deres Konge fra Jordan til Jerusalem.
\par 3 Da David kom til sit Hus i Jerusalem, tog Kongen sine ti Medhustruer, som han havde ladet tilbage for at se efter Huset, og lod dem bringe til et bevogtet Hus, hvor han sørgede for deres Underhold; men han gik ikke mere ind til dem, og således levede de indespærret til deres Dødedag som Kvinder, der er Enker, skønt deres Mænd endnu lever.
\par 4 Derpå sagde Kongen til Amasa: "Stævn Judas Mænd sammen i Løbet af tre Dage og indfind dig da her!"
\par 5 Amasa gik så bort for at stævne Judas Mænd sammen. Men da han tøvede ud over den fastsatte Frist,
\par 6 sagde David til Abisjaj: "Nu bliver Sjeba, Bikris Søn, os farligere end Absalon! Tag derfor din Herres Folk og sæt efter ham, for at han ikke skal kaste sig ind i befæstede Byer og slippe fra os!"
\par 7 Med Abisjaj drog Joab, Kreterne og Pleterne og alle Kærnetropperne ud fra Jerusalem for at sætte efter Sjeba, Bikris Søn.
\par 8 Men da de var ved den store Sten i Gibeon, kom Amasa dem i Møde. Joab var iført sin Våbenkjortel, og over den havde han spændt et Sværd, hvis Skede var bundet til hans Lænd; og det gled ud og faldt til Jorden.
\par 9 Joab sagde da til Amasa: "Går det dig vel, Broder?" Og Joab greb med højre Hånd om Amasas Skæg for at kysse ham.
\par 10 Men Amasa tog sig ikke i Vare for det Sværd, Joab havde i sin venstre Hånd; Joab stødte det i Underlivet på ham, så hans Indvolde væltede ud på Jorden, og han døde ved det ene Stød.
\par 11 medens en af Joabs Folk blev stående ved Amasa og råbte: "Enhver, der bryder sig om Joab og holder med David, følge efter Joab!"
\par 12 Men Amasa lå midt på Vejen, svømmende i sit Blod, og da Manden så, at alt Folket stod stille der, væltede han Amasa fra Vejen ind på Marken og kastede en Kappe over ham; thi han så, at alle, der kom forbi, stod stille.
\par 13 Så snart han var fjernet fra Vejen, fulgte alle efter Joab for at sætte efter Sjeba, Bikris Søn.
\par 14 Denne drog imidlertid gennem alle Israels Stammer indtil Abel-Bet-Ma'aka, og alle Bikriterne samlede sig og fulgte ham.
\par 15 Da kom de og belejrede ham i Abel-Bet-Ma'aka, idet de opkastede en Vold om Byen. Men medens alt Krigsfolket, der var med Joab, arbejdede på at bringe Muren til Fald,
\par 16 trådte en klog Kvinde fra Byen hen på Formuren og råbte: "Hør, hør! Sig til Joab, at han skal komme herhen; jeg vil tale med ham!"
\par 17 Da han kom hen til hende, spurgte Kvinden: "Er du Joab?" Og da han svarede ja, sagde hun: "Hør din Trælkvindes Ord!" Han svarede: "Jeg hører!"
\par 18 Så sagde hun: "I gamle Dage sagde man: Man spørge dog i Abel og Dan, om det er gået af Brug, hvad gode Folk i Israel vedtog!
\par 19 Og nu søger du at bringe Død over en By, som er en Moder i Israel! Hvorfor vil du ødelægge HERRENs Arvelod?"
\par 20 Joab svarede: "Det være langt fra mig, det være langt fra mig at ødelægge eller volde Fordærv!
\par 21 Således er det ingenlunde ment! Men en Mand fra Efraims Bjerge ved Navn Sjeba, Bikris Søn, har løftet sin Hånd mod Kong David; hvis I blot vil udlevere ham, bryder jeg op fra Byen!" Da sagde Kvinden til Joab: "Hans Hoved skal blive kastet ned til dig gennem Muren!"
\par 22 Kvinden fik så ved sin Klogskab hele Byen overtalt til at hugge Hovedet af Sjeba, Bikris Søn, og de kastede det ned til Joab. Da stødte han; i Hornet, og de brød op fra Byen og spredte sig hver til sit, medens Joab selv vendte tilbage til Kongen i Jerusalem.
\par 23 Joab stod over hele Israels Hær; Benaja, Jojadas Søn, over Kreterne og Pleterne;
\par 24 Adoniram havde Tilsynet med Hoveriarbejdet; Josjafat, Ahiluds Søn, var Kansler;
\par 25 Sjeja var Statsskriver, Zadok og Ebjatar Præster;
\par 26 også Ja'iriten Ira var Præst hos David.

\chapter{21}

\par 1 Under Davids Regering blev der Hungersnød tre År i Træk. Da søgte David HERRENs Åsyn; og HERREN sagde: "Der hviler Blodskyld på Saul og hans Hus, fordi han dræbte Gibeoniterne!"
\par 2 Kongen lod derfor Gibeoniterne kalde og sagde til dem - Gibeoniterne hørte ikke til Israeliterne, men til Levningerne af Amoriterne; og skønt Israeliterne havde givet dem endeligt Tilsagn, havde Saul i sin Iver for Israeliterne og Juda søgt at udrydde dem -
\par 3 David sagde til Gibeoniterne: "Hvad kan jeg gøre for eder, og hvorledes skal jeg skaffe Soning, så at I kan velsigne HERRENs Arvelod?"
\par 4 Gibeoniterne svarede: "Det er ikke Sølv eller Guld, der er os og Saul og hans Hus imellem, og vi har ikke Lov at dræbe nogen Mand i Israel!" Han sagde da: "Hvad I forlanger, vil jeg gøre for eder!"
\par 5 Så sagde de til Kongen: "Den Mand, som bragte Ødelæggelse over os og tænkte på at udrydde os, så vi ikke skulde kunne være nogetsteds inden for Israels Landemærke.
\par 6 lad syv Mænd af hans Efterkommere blive udleveret os, for at vi kan hænge dem op for HERREN i Gibeon på HERRENs Bjerg!" Kongen sagde: "Jeg vil udlevere dem!"
\par 7 Men Kongen skånede Mefibosjet, en Søn af Sauls Søn Jonatan, af Hensyn til den Ed ved HERREN, som var imellem David og Sauls Søn Jonatan.
\par 8 Derimod tog Kongen de to Sønnner, som Rizpa, Ajjas Datter, havde født Saul, Armoni og Mefibosjet, og de fem Sønner, som Merab, Sauls Datter, havde født Adriel, en Søn af Barzillaj fra Mehola,
\par 9 og udleverede dem til Gibeoniterne, som hængte dem op på Bjerget for HERRENs Åsyn. Således omkom alle syv på een Gang, og de blev dræbt først på Høsten, i Byghøstens Begyndelse.
\par 10 Men Rizpa, Ajjas Datter, tog sit Sørgeklæde, bredte det ud på Klippen og sad der fra Høstens Begyndelse, indtil der atter strømmede Vand fra Himmelen ned over dem; og hun tillod ikke Himmelens Fugle at kaste sig over dem om Da- gen eller Markens Dyr om Natten.
\par 11 Da David fik at vide, hvad Rizpa, Ajjas Datter, Sauls Medhustru, havde gjort,
\par 12 drog han hen og hentede Sauls og hans Søn Jonatans Ben hos Borgerne i Jabesj i Gilead, som havde stjålet dem på Torvet i Bet-Sjan, hvor Filisterne havde hængt dem op, dengang de slog Saul på Gilboa.
\par 13 Og da han havde hentet Sauls og hans Søn Jonatans Ben der, samlede man Benene af de hængte
\par 14 og jordede dem sammen med Sauls og hans Søn Jonatans Ben i Zela i Benjamins Land i hans Fader Kisj's Grav. Alt, hvad Kongen havde påbudt, blev gjort; derefter forbarmede Gud sig over Landet.
\par 15 Da det atter kom til Kamp mellem Filisterne og Israel, drog David med sine Folk ned og kastede sig ind i Gob og kæmpede med Filisterne.
\par 16 Da fremstod Dod, som var af Rafaslægten, og hvis Spyd vejede 300 Sekel Kobber; han var iført en ny Rustning, og han havde i Sinde at slå David ihjel.
\par 17 Men Abisjaj, Zerujas Søn, kom ham til Hjælp og huggede Filisteren ned. Da besvor Davids Mænd ham og sagde: "Du må ikke mere drage i Kamp med os, for at du ikke skal slukke Israels Lampe!"
\par 18 Siden hen kom det atter til Kamp med Filisterne i Gob.
\par 19 Atter kom det til Kamp med Filisterne i Gob, Beflehemiten Elhanan, Ja'irs Søn, nedhuggede da Gatiten Goliat, hvis Spydstage var som en Væverbom.
\par 20 Atter kom det til Kamp i Gat. Da var der en kæmpestor Mand med seks Fingre på hver Hånd og seks Tæer på hver Fod, i alt fire og tyve; han var også af Rafaslægten.
\par 21 Han hånede Israel, og derfor huggede Jonatan, en Søn af Davids Broder Sjim'a, ham ned.
\par 22 Disse fire var af Rafaslægten i Gat; de faldt for Davids og hans Mænds Hånd.

\chapter{22}

\par 1 David sang HERREN denne Sang, dengang HERREN havde frelst ham af alle hans Fjenders og af Sauls Hånd.
\par 2 Han sang: "HERRE, min Klippe, min Borg, min Befrier,
\par 3 min Gud, mit Bjerg, hvortil jeg tyr, mit Skjold, mit Frelseshorn, mit Værn, min Tilflugt, min Frelser, som frelser mig fra Vold!
\par 4 Jeg påkalder HERREN, den Højlovede, og frelses fra mine Fjender.
\par 5 Dødens Brændinger omsluttede mig, Ødelæggelsens Strømme forfærdede mig,
\par 6 Dødsrigets Reb omspændte mig, Dødens Snarer faldt over mig;
\par 7 i min Vånde påkaldte jeg HERREN og råbte til min Gud. Han hørte min Røst fra sin Helligdom, mit Råb fandt ind til hans Ører!
\par 8 Da rystede Jorden og skjalv, Himlens Grundvolde bæved og rysted, thi hans Vrede blussede op.
\par 9 Røg for ud af hans Næse, fortærende Ild af hans Mund, Gløder gnistrede fra ham.
\par 10 Han sænkede Himlen, steg ned med Skymulm under sine Fødder;
\par 11 båret af Keruber fløj han, svæved på Vindens Vinger;
\par 12 han omgav sig med Mulm som en Bolig, mørke Vandmasser, vandfyldte Skyer.
\par 13 Fra Glansen foran ham for der Hagl og Ildgløder ud.
\par 14 HERREN tordned fra Himlen, den Højeste lod høre sin Røst;
\par 15 han udslynged Pile, adsplittede dem, lod Lynene funkle og skræmmede dem.
\par 16 Havets Bund kom til Syne, Jordens Grundvolde blottedes ved HERRENs Trusel, for hans Vredes Pust.
\par 17 Han udrakte Hånden fra det høje og greb mig, drog mig op af de vældige Vande,
\par 18 frelste mig fra mine mægtige Fjender, fra mine Avindsmænd; de var mig for stærke.
\par 19 På min Ulykkes Dag faldt de over mig, men HERREN blev mig et Værn.
\par 20 Han førte mig ud i åbent Land, han frelste mig, thi han havde Behag i mig.
\par 21 HERREN gengældte mig efter min Retfærd, lønned mig efter mine Hænders Uskyld;
\par 22 thi jeg holdt mig til HERRENs Veje, svigted i Gudløshed ikke min Gud;
\par 23 hans Bud stod mig alle for Øje, jeg veg ikke fra hans Love.
\par 24 Ustraffelig var jeg for ham og vogtede mig for Brøde.
\par 25 HERREN lønned mig efter min Retfærd, mine Hænders Uskyld, som var ham for Øje!
\par 26 Du viser dig from mod den fromme, retsindig mod den retsindige,
\par 27 du viser dig ren mod den rene og vrang mod den svigefulde.
\par 28 De arme giver du Frelse, hovmodiges Øjne Skam!
\par 29 Ja, du er min Lampe, HERRE! HERREN opklarer mit Mørke.
\par 30 Thi ved din Hjælp søndrer jeg Mure, ved min Guds Hjælp springer jeg over Volde.
\par 31 Fuldkommen er Guds Vej, lutret er HERRENs Ord. Han er et Skjold for alle, der sætter deres Lid til ham.
\par 32 Ja, hvem er Gud uden HERREN, hvem er en Klippe uden vor Gud,
\par 33 den Gud, der omgjorded mig med Kraft, jævnede Vejen for mig,
\par 34 gjorde mine Fødder som Hindens og gav mig Fodfæste på Højne,
\par 35 oplærte min Hånd til Krig, så mine Arme spændte Kobberbuen?
\par 36 Du gav mig din Frelses Skjold, din Nedladelse gjorde mig stor;
\par 37 du skaffede Plads for mine Skridt, mine Ankler vaklede ikke.
\par 38 Jeg jog mine Fjender, indhentede dem, vendte først om, da de var gjort til intet,
\par 39 slog dem ned, så de ej kunde rejse sig, men lå faldne under min Fod.
\par 40 Du omgjorded mig med Kraft til Kampen, mine Modstandere tvang du i Knæ for mig;
\par 41 du slog mine Fjender på Flugt mine Avindsmænd ryddede jeg af Vejen.
\par 42 De råbte, men ingen hjalp, til HERREN, han svared dem ikke.
\par 43 Jeg knuste dem som Jordens Støv, som Gadeskarn tramped jeg på dem.
\par 44 Udlandets Sønner vansmægter, kommer skælvende frem af deres Skjul.
\par 45 Udlandets Sønner kryber for mig; blot de hører om mig, lyder de mig:
\par 46 Du friede mig af Folkekampe, du satte mig til Folkeslags Høvding; nu tjener mig ukendte Folk;
\par 47 HERREN lever, højlovet min Klippe, ophøjet være min Frelses Gud,
\par 48 den Gud, som giver mig Hævn, lægger Folkeslag under min Fod
\par 49 og frier mig fra mine Fjender! Du ophøjer mig over mine Modstandere, fra Voldsmænd frelser du mig.
\par 50 HERRE, derfor priser jeg dig blandt Folkene og lovsynger dit Navn,
\par 51 du, som kraftig hjælper din Konge og viser din Salvede Miskundhed. David og hans Æt evindelig.

\chapter{23}

\par 1 Dette er Davids sidste Ord; Så siger David, Isajs Søn, så siger Manden, Højt ophøjet, Jakobs Guds salvede, Helten i Israels Sange:
\par 2 Ved mig talede HERRENs Ånd, hans Ord var på min Tunge.
\par 3 Jakobs Gud talede til mig, Israels Klippe sagde: "En retfærdig Hersker blandt Mennesker, en, der hersker i Gudsfrygt,
\par 4 han stråler som Morgenrøden, som den skyfri Morgensol, der fremlokker Urter af Jorden efter Regn."
\par 5 Således har jo mit Hus det med Gud. Han gav mig en evig Pagt, fuldgod og vel forvaret. Ja, al min Frelse og al min Lyst, skulde han ikke lade den spire frem?
\par 6 Men Niddinger er alle som Torne i Ørk, der tages ikke på dem med Hænder;
\par 7 ingen rører ved dem uden med Jern og Spydstage, i Ilden brændes de op.
\par 8 Navnene på Davids Helte var følgende: Hakmoniten Isjbosjet, Anføreren for de tre; det var ham, der engang svang sit Spyd over 800 faldne på een Gang.
\par 9 Blandt de tre Helte kom efter ham Ahohiten El'azar, Dodis Søn; han var med David ved Pas-Dammim, dengang Filisterne samlede sig der til Kamp. Da Israels Mænd trak sig tilbage,
\par 10 holdt han Stand og huggede ned blandt Filisterne, til hans Hånd blev træt og klæbede fast ved Sværdet; HERREN gav dem den bag en stor Sejr. Så vendte Folket tilbage og fulgte ham, kun for at plyndre.
\par 11 Efter ham kom Harariten Sjamma, Ages Søn. Engang havde Filisterne samlet sig i Lehi, hvor der var en Mark fuld af Linser, og Folket flygtede for Filisterne;
\par 12 men han stillede sig op midt på Marken og holdt den og huggede Filisterne ned; og HERREN gav dem en stor Sejr.
\par 13 Engang drog tre af de tredive ned og kom til David på Klippens Top, til Adullams Hule, medens Filisterskaren var lejret i Refaimdalen.
\par 14 David var dengang i Klippeborgen, medens Filisternes Besætning lå i Betlehem.
\par 15 Så vågnede Lysten hos David, og han sagde: "Hvem skaffer mig en Drik Vand fra Cisternen ved Betlehems Port?"
\par 16 Da banede de tre Helte sig Vej gennem Filisternes Lejr, øste Vand af Cisternen ved Betlehems Port og bragte David det. Han vilde dog ikke drikke det, men udgød det for HERREN
\par 17 med de Ord: "HERREN vogte mig for at gøre det! Skulde jeg drikke de Mænds Blod, som har vovet deres Liv!" Og han vilde ikke drikke det. Den Dåd udførte de tre Helte.
\par 18 Abisjaj, Joabs Broder, Zerujas Søn, var Anfører for de tredive. Han svang sit Spyd over 300 faldne, og han var navnkundig iblandt de tredive;
\par 19 iblandt de tredive var han højtæret, og han var deres Anfører; men de tre nåede han ikke.
\par 20 Benaja, Jojadas Søn, var en tapper Mand fra Kabze'el, der havde udført store Heltegerninger; han fældede de to Arielsønner fra Moah; og han steg ned og fældede en Løve i en Cisterne, en Dag der var faldet Sne.
\par 21 Ligeledes fældede han Ægypteren, en kæmpestor Mand. Ægypteren havde et Spyd i Hånden, men han gik ned imod ham meden Stok, vristede Spydet ud af Hånden på ham og dræbte ham med hans eget Spyd.
\par 22 Disse Heltegerninger udførte Benaja, Jojadas Søn, og han var navnkundig iblandt de tredive Helte;
\par 23 iblandt de tredive var han højt æret, men de tre nåede han ikke. David satte ham over sin Livvagt.
\par 24 Til de tredive hørte Asa'el' Joabs Broder; Elbanan, Dodos Søn fra Betlehem;
\par 25 Haroditen Sjamma; Haroditen Elika;
\par 26 Paltiten Helez; Ira, Ikkesj's Søn fra Tekoa;
\par 27 Abiezer fra Anatot; Husjatiten Sibbekaj;
\par 28 Ahohiten Zalmon; Maharaj fra Netofa;
\par 29 Heled, Ba'anas Søn, fra Nefofa; Ittaj, Ribajs Søn, fra det benjaminitiske Gibea;
\par 30 Benaja fra Pir'aton; Hiddaj fra Nahale-Ga'asj;
\par 31 Abiba'al fra Araba; Azmavet fra Bahurim;
\par 32 Sja'alboniten Eljaba; Guniten Jasjen;
\par 33 Harariten Jonatan, Sjammas Søn; Harariten Ahi'am, Sjarars Søn;
\par 34 Elifelet, Ahazbajs Søn, fra Bet-Ma'aka; Eliam, Akitofels Søn, fra Gilo;
\par 35 Hezro fra Karmel; Pa'araj fra Arab;
\par 36 Jig'al, Natans Søn, fra Zoba; Gaditen Bani;
\par 37 Ammoniten Zelek; Naharaj fra Be'erot, der var Joabs, Zerujas Søns, Våbendrager;
\par 38 Ira fra Jattir; Gareb fra Jattir;
\par 39 Hetiten Urias. I alt syv og tredive.

\chapter{24}

\par 1 Men HERRENS vrede blussede atter op mod Israel, så at han æggede David mod dem og sagde: "Gå hen og hold Mandtal over Israel og Juda!"
\par 2 Kongen sagde da til Joab og Hærførerne, der var hos ham: "Drag rundt i alle Israels Stammer fra Dan til Be'ersjeba og hold Mønstring over Folket, for at jeg kan få Tallet på det at vide!"
\par 3 Men Joab svarede Kongen: "Måtte HERREN din Gud forøge Folket hundredfold, og måtte min Herre Kongen selv opleve det - men hvorfor har min Herre Kongen sat sig sligt for?"
\par 4 Men Joab og Hærførerne måtte bøje sig for Kongens Ord. Joab og Hærførerne forlod derfor Kongen for at holde Mønstring over Israels Folk.
\par 5 De gik over Jordan og begyndte ved Aroer og Byen, der ligger midt i Dalen; drog ad Gad til og i Retning af Ja'zer;
\par 6 så kom de til Gilead og til Hetiternes Land hen imod Kadesj, derpå til Dan, og fra Dan vendte de sig hen mod Zidon;
\par 7 så kom de til Fæstningen Tyrus og alle Hivviternes og Kana'anæernes Byer, hvorfra de gik til Be'ersjeba i det judæiske Sydland.
\par 8 Efter at de var draget hele Landet rundt i ni Måneder og tyve Dage, kom de tilbage til Jerusalem.
\par 9 Joab opgav derpå Kongen Tallet, der var fundet ved Folktællingen, og Israel talte 800000 kraftige, våbenføre Mænd, Juda 500000 Mænd.
\par 10 Men efter at David havde holdt Mandtal over Folket. slog Samvittigheden ham, og han sagde til HERREN: "Jeg har syndet svarlig i, hvad jeg har gjort! Men tilgiv nu, HERRE, din Tjeners Brøde, thi jeg har handlet som en stor Dåre!"
\par 11 Da David stod op om Morgenen kom HERRENs Ord til Profeten Gad, Davids Seer, således:
\par 12 "Gå hen og sig til David: Så siger HERREN; Jeg forelægger dig tre Ting; vælg selv, hvilken jeg skal lade times dig!"
\par 13 Gad kom da til David og kundgjorde ham det og sagde: "Skal der komme tre Hungersnødsår over dig i dit Land, eller vil du i tre Måneder jages på Flugt, forfulgt af din Fjende, eller skal der komme tre Dages Pest i dit Land? Overvej nu, hvad jeg skal svare ham, der har sendt mig!"
\par 14 David svarede Gad: "Jeg er i såre stor Vånde - lad os så falde i HERRENs Hånd, thi hans Barmhjertighed er stor; i Menneskehånd vil jeg ikke falde!"
\par 15 Så valgte David da Pesten. Ved Hvedehøstens Tid begyndte Soten at ramme Folket; og der døde 70000 Mand af Folket fra Dan til Be'ersjeba.
\par 16 Da Engelen udrakte sin Hånd mod Jerusalem for at ødelægge det, angrede HERREN det onde, og han sagde til Engelen, som ødelagde Folket "Nu er det nok, drag din Hånd tilbage!" HERRENs Engel var da ved Jebusiten Aravnas Tærskeplads.
\par 17 Men da David så Engelen, som slog Folket, sagde han til HERREN: "Det er mig, der har syndet, mig, der har begået Brøden; men Fårene der, hvad har de gjort? Lad din Hånd dog ramme mig og mit Fædrenehus!"
\par 18 Samme Dag kom Gad til David og sagde: "Gå op og rejs HERREN et Alter på Jebusiten Aravnas Tærskeplads!"
\par 19 Og David gik derop efter Gads Ord, således som HERREN havde påbudt.
\par 20 Da Aravna, der var ved at tærske Hvede, så ned og fik Øje på Kongen og hans Folk, som kom hen imod ham, gik han ud og kastede sig på sit Ansigt til Jorden for ham;
\par 21 og Aravna sagde: "Hvorfor kommer min Herre Kongen til sin Træl?" David svarede: "For at købe Tærskepladsen af dig og bygge HERREN et Alter, at Folket må blive friet fra Plagen!"
\par 22 Da sagde Aravna til David: "Min Herre Kongen tage den og ofre, hvad ham tykkes ret! Her er Okserne til Brændoffer og Tærskeslæderne og Oksernes Stavtøj til Brændsel!
\par 23 Min Herre Kongens Træl giver Kongen det hele!" Og Aravna sagde til Kongen: "Måtte HERREN din Gud have Behag i dig!"
\par 24 Men Kongen svarede Aravna: "Nej, jeg vil købe det af dig for dets fulde Værdi; jeg vil ikke bringe HERREN min Gud Brændofre, som intet koster mig!" Så købte David Tærskepladsen og Okserne for halvtredsindstyve Sølvsekel;
\par 25 og David byggede HERREN et Alter der og ofrede Brændofre og Takofre. Da forbarmede HERREN sig over Landet, og Israel blev friet fra Plagen.



\end{document}