\begin{document}

\title{Første Kongebog}


\chapter{1}

\par 1 Da Kong David var gammel og til års, kunne han ikke blive varm, skønt man dækkede ham til med Tæpper.
\par 2 Da sagde hans Folk til ham: "Det er bedst, man søger efter en ung Jomfru til min Herre Kongen, for at hun kan være om Kongen og pleje ham; når hun ligger i din Favn, bliver min Herre Kongen varm!"
\par 3 Så søgte de efter en smuk ung Pige i hele Israels Land og fandt Abisjag fra Sjunem og bragte hende til Kongen.
\par 4 Hun var en såre smuk Pige: og hun plejede kongen og gik ham til Hånde; men Kongen havde ikke Omgang med hende.
\par 5 Adonija, Haggits Søn, dristede sig til at sige: "Jeg vil være Konge!" Og han skaffede sig Vogne og Heste og halvtredsindstyve Mænd til at løbe foran sig.
\par 6 Hans Fader havde ingen Sinde irettesat ham og sagt: "Hvorfor bærer du dig således ad?" Han havde et såre smukt Ydre og var den ældste efter Absalom.
\par 7 Han underhandlede med Joab, Zerujas Søn, og Præsten Ebjatar; de tog Adonijas Parti og støttede ham,
\par 8 mens Præsten Zadok, Jojadas Søn Benaja, Profeten Natan, Sjim i og Re'i og Davids Kærnetropper ikke sluttede sig til Adonija.
\par 9 Adonija lod nu slagte Små kvæg, Hornkvæg og Fedekvæg ved Slangestenen, der står ved Rogelkilden, og indbød alle sine Brødre, Kongesønnerne, og alle de judæiske Mænd, der stod i Kongens Tjeneste;
\par 10 men Profeten Natan, Benaja, Kærnetropperne og sin Broder Salomo indbød han ikke.
\par 11 Da sagde Natan til Batseba, Salomos Moder: "Du har vel hørt, at Adonija, Haggits Søn, har opkastet sig til Konge uden vor Herre Davids Vidende?
\par 12 Lad mig nu give dig et Råd, for at du kan redde dit eget og din søn Salomos Liv:
\par 13 Du skal gå ind til Kong David og sige til ham: Herre konge, du har jo tilsvoret din Trælkvinde: Din Søn Salomo skal være Konge efter mig og sidde på min Trone! Hvorfor har da Adonija opkastet sig til Konge?
\par 14 Og medens du endnu står og taler med Kongen, kommer jeg til og bekræfter dine Ord!"
\par 15 Da gik Batseba ind til kongen i hans Værelse - Kongen var meget gammel, og Abisjag fra Sjunem gik ham til Hånde -
\par 16 og Batseba bøjede sig og kastede sig til Jorden for Kongen. Da sagde Kongen: "Hvad ønsker du?"
\par 17 Hun svarede: "Herre, du har jo tilsvoret din Trælkvinde ved HERREN din Gud: Din Søn Salomo skal være konge efter mig og sidde på min Trone!
\par 18 Men se, nu har Adonija opkastet sig til Konge uden dit Vidende, Herre Konge!
\par 19 Han har ladet slagte Hornkvæg, Fedekvæg og Småkvæg i Mængde og indbudt alle Kongesønnerne, Præsten Ebjatar og Hærføreren Joab, men din Træl Salomo har han ikke indbudt.
\par 20 På dig, Herre Konge, er hele Israels Øjne rettet, for at du skal give dem til Kende, hvem der skal være din Efterfølger og sidde på min Herre Kongens Trone.
\par 21 Ellers gælder det mit og min Søn Salomos Liv, når min Herre Kongen har lagt sig til Hvile hos sine Fædre!"
\par 22 Medens hun endnu talte med Kongen, kom Profeten Natan,
\par 23 og det blev meldt Kongen: "Profeten Natan er her!" Så trådte han frem for Kongen og kastede sig på sit Ansigt til Jorden for ham.
\par 24 Derpå sagde Natan: "Herre konge, du har vel sagt, at Adonija skal være Konge efter dig og sidde på din Trone?
\par 25 Thi han er i Dag draget ned og har ladet slagte Hornkvæg, Fedekvæg og Småkvæg i Mængde og indbudt alle Kongesønnerne, Hærførerne og Præsten Ebjatar, og nu spiser og drikker de sammen med ham og råber: Leve Kong Adonija!
\par 26 Men han har hverken indbudt mig, din Træl, eller Præsten Zadok eller Benaja, Jojadas Søn, eller din Træl Salomo!
\par 27 Er det virkelig sket efter min Herre Kongens Befaling, uden at du har ladet dine Trælle vide, hvem der skal sidde på min Herre Kongens Trone efter dig?"
\par 28 Da svarede Kong David: "Kald mig Batseba hid!" Og hun trådte frem for Kongen og stillede sig foran ham.
\par 29 Da svor Kongen og sagde: "Så sandt HERREN, der udløste mig af al min Nød, lever:
\par 30 Som jeg tilsvor dig ved HERREN, Israels Gud, at din Søn Salomo skulde være Konge efter mig og sidde på min Trone i mit Sted, således vil jeg handle i Dag!"
\par 31 Da bøjede Batseba sig med sit Ansigt til Jorden og faldt ned for Kongen og sagde: "Måtte min Herre, Kong David, leve evindelig!"
\par 32 Derpå sagde Kong David: "Kald mig Præsten Zadok, Profeten Natan og Benaja, Jojadas Søn, hid!" Og de trådte frem for Kongen.
\par 33 Da sagde Kongen til dem: "Tag eders Herres Folk med eder, sæt min Søn Salomo på mit eget Muldyr og før ham ned til Gihon.
\par 34 Der skal Præsten Zadok og Profeten Natan salve ham til Konge over Israel, og I skal støde i Hornet og råbe: Leve Kong Salomo!
\par 35 Så skal I følge ham herop, og han skal gå hen og sætte sig på min Trone og være Konge i mit Sted; thi det er ham, jeg har udset til Fyrste over Israel og Juda!"
\par 36 Da svarede Benaja, Jojadas Søn, Kongen: "Det ske! Måtte HERREN, min Herre Kongens Gud, gøre således!
\par 37 Måtte HERREN være med Salomo, som han har været med min Herre Kongen, og gøre hans Trone endnu mægtigere end min Herre kong Davids!"
\par 38 Derpå drog Præsten Zadok, Profeten Natan og Benaja, Jojadas Søn, og Kreterne og Pleterne ned og satte Salomo på Kong Davids Muldyr og førte ham til Gihon;
\par 39 og Præsten Zadok tog Oliehornet fra Teltet og salvede Salomo; de stødte i Hornet, og hele Folket råbte: "Leve Kong Salomo!"
\par 40 Så fulgte hele Folket ham op og Folket spillede på Fløjter og jublede højt, så at Jorden var ved at revne af deres Råb.
\par 41 Det hørte Adonija og alle hans Gæster, netop som de var færdige med Måltidet, og da Joab hørte Hornets klang, sagde han: "Hvorfor er der så stort Røre i Byen?"
\par 42 Endnu medens han talte, kom Jonatan, Præsten Ebjatars Søn. Adonija sagde: "Kom herhen, thi du er en brav Mand og bringer godt nyt!"
\par 43 Men Jonatan svarede og sagde til Adonija: "Tværtimod; vor Herre Kong David har gjort Salomo til Konge!
\par 44 Kongen sendte Præsten Zadok, Profeten Natan, Benaja, Jojadas Søn, og Kreterne og Pleterne med ham, og de satte ham på Kongens Muldyr;
\par 45 og Præsten Zadok og Profeten Natan salvede ham til Konge ved Gihon; derefter drog de under Jubel op derfra, og der blev Røre i Byen; det var den Larm, I hørte.
\par 46 Salomo satte sig også på Kongetronen;
\par 47 tilmed kom Kongens Folk og lykønskede vor Herre Kong David med de Ord: Måtte din Gud gøre Salomos Navn endnu herligere end dit og hans Trone mægtigere end din! Og Kongen tilbad på sit Leje;
\par 48 ydermere sagde han: Lovet være HERREN, Israels Gud, som i Dag har ladet en Mand sætte sig på min Trone, endnu medens jeg selv kan se det!"
\par 49 Da blev alle Adonijas Gæster skrækslagne og brød op og gik hver sin Vej;
\par 50 men Adonija frygtede for Salomo og ilede hen og greb fat om Alterets Horn.
\par 51 Og man meldte Salomo: "Se, Adonija frygter for Kong Salomo, og se, han har grebet fat om Alterets Horn og siger: Kong Salomo skal først sværge mig til, at han ikke vil lade sin Træl slå ihjel med Sværd!"
\par 52 Da sagde Salomo: "Dersom han opfører sig som en brav Mand, skal der ikke krummes et Hår på hans Hoved; men gribes han i noget ondt, skal han dø!"
\par 53 Derpå sendte Kong Salomo Bud og lod ham hente ned fra Alteret; og han kom og kastede sig ned for Kong Salomo. Da sagde Salomo til ham: "Gå til dit Hjem!"

\chapter{2}

\par 1 Da det nu lakkede ad Enden med Davids Liv, gav han sin Søn Salomo disse Befalinger:
\par 2 "Jeg går nu al Kødets Gang; så vær nu frimodig og vis dig som en Mand!
\par 3 Og hold HERREN din Guds Forskrifter, så du vandrer på hans Veje og, holder hans Anordninger, Bud, Bestemmelser og Vidnesbyrd, således som skrevet står i Mose Lov, for at du må have Lykken med dig i alt, hvad du gør, og i alt, hvad du tager dig for,
\par 4 for at HERREN kan opfylde den Forjættelse, han gav mig, da han sagde: Hvis dine Sønner vogter på deres Vej, så de vandrer i Trofasthed for mit Åsyn af hele deres Hjerte og hele deres Sjæl, skal der aldrig fattes dig en Efterfølger på Israels Trone!
\par 5 Du ved jo også, hvad Joab, Zerujas Søn, har voldet mig, hvorledes han handlede mod Israels to Hærførere, Abner, Ners Søn, og Amasa, Jeters Søn, hvorledes han slog dem ihjel og således i Fredstid hævnede Blod, der var udgydt i Krig, og besudlede Bæltet om min Lænd og Skoene på mine Fødder med uskyldigt Blod;
\par 6 gør derfor, som din Klogskab tilsiger dig, og lad ikke hans grå Hår stige ned i Dødsriget med Fred.
\par 7 Men mod Gileaditen Barzillajs Sønner skal du vise Godhed, og de skal have Plads mellem dem, der spiser ved dit Bord, thi på den Måde kom de mig i Møde, da jeg måtte flygte for din Broder Absalom.
\par 8 Og se, så har du hos dig Benjaminiten Simeon, Geras Søn, fra Bahurim, ham, som udslyngede en grufuld Forbandelse imod mig, dengang jeg drog til Mahanajim. Da han senere kom mig i Møde ved Jordan, tilsvor jeg ham ved HERREN: Jeg vil ikke slå dig ihjel med Sværd!
\par 9 Men du skal ikke lade ham ustraffet, thi du er en klog Mand og vil vide, hvorledes du skal handle med ham, og bringe hans grå Hår blodige ned i Dødsriget."
\par 10 Så lagde David sig til Hvile hos sine Fædre og blev jordet i Davidsbyen.
\par 11 Tiden, han havde været Konge over Israel, udgjorde fyrretyve År; i Hebron herskede han syv År, i Jerusalem tre og tredive År.
\par 12 Derpå satte Salomo sig på sin Fader Davids Trone, og hans Herredømme blev såre stærkt.
\par 13 Men Adonija, Haggits Søn, kom til Batseba, Salomos Moder. Hun spurgte da: "Kommer du for det gode?" Han svarede: "Ja, jeg gør!"
\par 14 Og han fortsatte: "Jeg har en Sag at tale med dig om." Hun svarede: "Så tal!"
\par 15 Da sagde han: "Du ved at Kongeværdigheden tilkom mig, og at hele Israel havde Blikket rettet på mig som den, der skulde være Konge; dog gik Kongeværdigbeden over til min Broder, thi HERREN lod det tilfalde ham.
\par 16 Men nu har jeg een eneste Bøn til dig; du må ikke afvise mig!" Hun svarede: "Så tal!"
\par 17 Da sagde han: "Sig til Kong Salomo - dig vil han jo ikke afvise - at han skal give mig Abisj fra Sjunem til Ægte!"
\par 18 Og Batseba svarede: "Vel, jeg skal tale din Sag hos Kongen!"
\par 19 Derpå begav Batseba sig til Kong Salomo for at tale Adonijas Sag; og Kongen rejste sig, gik hende i Møde og bøjede sig for hende; derpå satte han sig på sin Trone og lod også en Trone sætte frem til Kongemoderen, og hun satte sig ved hans højre Side.
\par 20 Så sagde hun: "Jeg har en eneste ringe Bøn til dig; du må ikke afvise mig!" Kongen svarede: "Kom med din Bøn, Moder, jeg vil ikke afvise dig!"
\par 21 Da sagde hun: "Lad din Broder Adonija få Abisjag fra Sjunem til Hustru!"
\par 22 Men Kong Salomo svarede sin Moder: "Hvorfor beder du om Abisjag fra Sjunem til Adonija? Du skulde hellere bede om Kongeværdigheden til ham; han er jo min ældre Broder, og Præsten Ebjatar og Joab, Zerujas Søn, står på hans Side!"
\par 23 Og Kong Salomo svor ved HERREN: "Gud ramme mig både med det ene og det andt, om ikke det Ord skal koste Adonija Livet!
\par 24 Så sandt HERREN lever, som indsatte mig og gav mig Plads på min Fader Davids Trone og byggede mig et Hus, som han lovede: Endnu i Dag skal Adonija miste Livet!"
\par 25 Derpå gav Kong Salomo Benaja, Jojadas Søn, Ordre til at hugge ham ned; således døde han.
\par 26 Men til Præsten Ebjatar sagde Kongen: "Begiv dig til din Landejendom i Anatot, thi du har forbrudt dit Liv; og når jeg ikke dræber dig i Dag, er det, fordi du bar den Herre HERRENs Ark foran min Fader David og delte alle min Faders Lidelser!"
\par 27 Derpå afsatte Salomo Ebjatar fra hans Stilling som HERRENs Præst for at opfylde det Ord, HERREN havde talet mod Elis Hus i Silo.
\par 28 Da Rygtet, herom nåede Joab - Joab havde jo sluttet sig til Adonijas Parti, medens han ikke havde sluttet sig til Absaloms - søgte han Tilflugt i HERRENs Telt og greb fat om Alterets Horn,
\par 29 og det meldtes Kong Salomo, at Joab havde søgt Tilflugt i HERRENs Telt og stod ved Alteret. Da sendte Salomo Benaja, Jojadas Søn, derhen og sagde: "Gå hen og hug ham ned!"
\par 30 Og da Benaja kom til HERRENs Telt, sagde han til ham: "Således siger Kongen: Kom herud!" Men han svarede: "Nej, her vil jeg dø!" Benaja meldte da tilbage til Kongen, hvad Joab havde sagt og svaret ham.
\par 31 Men Kongen sagde til ham: "Så gør, som han siger, hug ham ned og jord ham og fri mig og min Faders Hus for det uskyldige Blod, Joab har udgydt;
\par 32 HERREN vil lade hans Blodskyld komme over hans eget Hoved, at han huggede to Mænd ned, der var retfærdigere og bedre end han selv, og slog dem ihjel med Sværdet uden min Fader Davids Vidende, Abner, Ners Søn, Israels Hærfører, og Amasa, Jeters Søn, Judas Hærfører;
\par 33 så kommer deres Blod over Joabs og hans, Slægts Hoved evindelig, medens HERREN giver David og hans Slægt, hans Hus og hans Trone Fred til evig Tid!"
\par 34 Da gik Benaja, Jojadas Søn, hen og huggede ham ned og dræbte ham; og han blev jordet i sit Hus i Ørkenen.
\par 35 Og Kongen satte Benaja Jojadas Søn, over Hæren i hans Sted, medens han gav Præsten Zadok Ebjatars Stilling.
\par 36 Derpå lod Kongen Simei kalde og sagde til ham: "Byg dig et Hus i Jerusalem, bliv der og drag ikke bort, hvorhen det end er;
\par 37 thi den Dag du drager bort og overskrider Kedrons Dal, må du vide, du er dødsens; da kommer dit Blod over dit Hoved!"
\par 38 Og Simei svarede Kongen: "Godt! Som min Herre Kongen siger, således vil din Træl gøre!" Simei blev nu en Tid lang i Jerusalem.
\par 39 Men efter tre Års Forløb flygtede to af Simeis Trælle til Ma'akas Søn, Kong Akisj af Gat, og da Simei fik at vide, at hans Trælle var i Gat,
\par 40 brød han op, sadlede sit Æsel og drog til Akisj i Gat for at hente sine Trælle; Simei drog altså af Sted og fik sine Trælle med hjem fra Gat.
\par 41 Men da Salomo fik af vide, at Simei var rejst fra Jerusalem til Gat og kommet tilbage igen,
\par 42 lod Kongen ham kalde og sagde til ham: "Tog jeg dig ikke i Ed ved HERREN, og advarede jeg dig ikke: Den Dag du drager bort og begiver dig andetsteds hen, hvor det end er, må du vide, du er dødsens! Og svarede du mig ikke: Godt! Jeg har hørt det?
\par 43 Hvorfor holdt du da ikke den Ed, du svor ved HERREN, og den Befaling, jeg gav dig?"
\par 44 Endvidere sagde Kongen til Simei: "Du ved selv, og dit Hjerte er sig det bevidst, alt det onde, du gjorde min Fader David; nu lader HERREN din Ondskab komme over dit eget Hoved;
\par 45 men Kong Salomo skal være velsignet, og Davids Trone skal stå urokkelig fast for HERRENs Åsyn til evig Tid!"
\par 46 Derpå gav Kongen Ordre til Benaja, Jojadas Søn, og han gik hen og huggede ham ned; således døde han.

\chapter{3}

\par 1 Da nu Salomo havde fået Kongedømmet sikkert i hænde besvogrede han sig med Farao, Ægypterkongen, idet han ægtede Faraos Datter; og han førte hende ind i Davidsbyen, til han fik sit eget Hus, HERRENs Hus og Muren om Jerusalem bygget færdig.
\par 2 Kun ofrede Folket på Offerhøjene, thi hidindtil var der ikke bygget HERRENs Navn et Hus.
\par 3 Salomo elskede HERREN, så at han vandrede efter sin Fader Davids Anordninger; kun ofrede han på Højene og tændte Offerild der.
\par 4 Og Kongen begav sig til Gibeon for at ofre der; thi det var den store Offerhøj; tusind Brændofre ofrede Salomo på Alteret der.
\par 5 I Gibeon lod HERREN sig til Syne for Salomo i en Drøm om Natten. Og Gud sagde: "Sig, hvad du ønsker, jeg skal give dig!"
\par 6 Da sagde Salomo: "Du viste stor Miskundhed mod din Tjener, min Fader David, der jo også vandrede for dit Åsyn i Troskab, Retfærdighed og Hjertets Oprigtighed; og du lod denne store Miskundhed blive over ham og gav ham en Søn, der nu sidder på hans Trone.
\par 7 Ja, nu har du, HERRE min Gud, gjort din Tjener til Konge i min Fader Davids Sted. Men jeg er ganske ung og ved ikke, hvorledes jeg skal færdes ret;
\par 8 og din Tjener står midt i det Folk, du udvalgte, et stort Folk, som ikke kan tælles eller udregnes, så mange er de.
\par 9 Giv derfor din Tjener et lydhørt Hjerte, så han kan dømme dit Folk og skelne mellem godt og ondt; thi hvem kan dømme dette dit store Folk!"
\par 10 Det vakte HERRENs Velbehag, at Salomo bad derom;
\par 11 og Gud sagde til ham: "Fordi du bad om dette og ikke om et langt Liv, ej heller om Rigdom eller om dine Fjenders Liv, men om Forstand til at skønne, hvad ret er,
\par 12 se, derfor vil jeg gøre, som du beder: Se, jeg giver dig et viist og forstandigt Hjerte, så at din Lige aldrig før har været, ej heller siden skal fremstå;
\par 13 men jeg giver dig også, hvad du ikke bad om, både Rigdom og Ære, så at du ikke skal have din Lige blandt Konger, så længe du lever.
\par 14 Og hvis du vandrer på mine Veje, så du holder mine Anordninger og Bud, således som din Fader David gjorde, vil jeg give dig et langt Liv."
\par 15 Da vågnede Salomo, og se, det var en Drøm. Derpå begav han sig til Jerusalem og stillede sig foran HERRENs Pagts Ark og ofrede Brændofre, bragte Takofre og gjorde et Gæstebud for alle sine Folk.
\par 16 På den Tid kom to Skøger til Kongen og trådte frem for ham.
\par 17 Den ene Kvinde sagde: "Hør mig, Herre! Jeg og den Kvinde der bor i Hus sammen. Hjemme i vort Hus fødte jeg i hendes Nærværelse et Barn,
\par 18 og tre Dage efter min Nedkomst fødte også hun et Barn. Vi var sammen; der var ingen andre hos os i Huset, vi to var ene i Huset.
\par 19 Så døde hendes Dreng om Natten, fordi hun kom til at ligge på ham;
\par 20 men midt om Natten, medens din Trælkvinde sov, stod hun op og tog min Dreng fra min Side og lagde ham ved sit Bryst; men sin døde Dreng lagde hun ved mit Bryst.
\par 21 Da jeg så om Morgenen rejste mig for at give min Dreng at die, se, da var han død; men da jeg så nøje på ham om Morgenen, se, da var det ikke min Dreng, ham, som jeg havde født!"
\par 22 Men den anden Kvinde sagde: "Det er ikke sandt; den levende er min Dreng, og den døde er din!" Og den første sagde: "Nej, den døde er din Dreng, og den levende er min!" Således mundhuggedes de foran Kongen.
\par 23 Da sagde Kongen: "Den ene siger: Han her, den levende, er min Dreng, den døde er din! Og den anden siger: Nej, den døde er din Dreng, den levende min!"
\par 24 Derpå sagde Kongen: "Hent mig et Sværd!" Og de bragte Kongen et.
\par 25 Da sagde Kongen: "Hug det levende Barn over og giv hver af dem det halve!"
\par 26 Da rørte Kærligheden til Barnet sig heftigt i den Kvinde, som var Moder til det levende Barn, og hun sagde til Kongen: "Hør mig, Herre! Giv hende det levende Barn; dræb det endelig ikke!" Men den anden sagde: "Det skal hverken tilhøre mig eller dig, hug det kun over!"
\par 27 Da tog Kongen til Orde og sagde: "Giv hende der det levende Barn og dræb det ikke; thi hun er Moderen!"
\par 28 Og da Israel hørte om den Dom, Kongen havde fældet, fyldtes de alle af Ærefrygt for Kongen; thi de så, at han sad inde med Guds Visdom til at skifte Ret.

\chapter{4}

\par 1 Kong Salomo var Konge over hele Israel
\par 2 Hans øverste Embedsmænd var følgende: Azarja, Zadoks Søn, var Ypperstepræst;
\par 3 Elihoref og Ahija, Sjisjas Sønner, var Statsskrivere; Josjafat, Ahiluds Søn, var Kansler;
\par 4 Benaja, Jojadas Søn, stod i Spidsen for Hæren; Zadok og Ebjatar var Præster;
\par 5 Azarja, Natans Søn, var Overfoged; Præsten Zabud, Natans Søn, var Kongens Ven;
\par 6 Ahisjar var Slotshøvedsmand; Adoniram, Abdas Søn, havde Tilsyn med Hoveriarbejdet.
\par 7 Fremdeles havde Salomno tolv Fogeder over hele Israel, som skulde sørge for Kongens og Hoffets Underhold, hver af dem en Måned om Året.
\par 8 Deres Navne var: Hurs Søn i Efraims Bjerge;
\par 9 Dekers Søn i Makaz-Sja'albim, Bet-Sjemesj og Elon indtil Bet-Hanan;
\par 10 Heseds Søn i Arubbot; han havde Soko og hele Hefers Land;
\par 11 Abinadabs Søn havde hele Dors Højland; han var gift med Salomos Datter Tafat;
\par 12 Ba'ana, Ahiluds Søn, havde Ta'anak, Megiddo og hele Bet- Sjean op til Zaretan, neden for Jizre'el fra Bet-Sjean til Abel-Mehola ud over Jokmeam;
\par 13 Gebers Søn i Ramot i Gilead; han havde Manasses Søn Ja'irs Teltbyer i Gilead, og han havde Argoblandet i Basan, tresindstyve store Byer med Mure og Kobberportstænger;
\par 14 Ahinadab, Iddos Søn, havde Mahanajim;
\par 15 Ahima'az i Naftali; han var ligeledes gift med en Datter af Salomo, Basemat;
\par 16 Ba'ana, Husjajs Søn, i Aser og Bealot;
\par 17 Josjafat, Paruas Søn, i Issakar;
\par 18 Sjim'i, Elas Søn, i Benjamin;
\par 19 Geber, Uris Søn, i Gads Land, det Land, der havde tilhørt Amoriiterkongen Sihon og Kong Og af Basan; der var kun een Foged i det Land.
\par 20 Juda og Israel var talrige, så talrige som Sandet ved Havet, og de spiste og drak og var glade.
\par 21 Og Salomo herskede over alle Rigerne fra Floden til Filisternes Land og Ægyptens Grænse og de bragte Gaver og tjente Salomo, så længe han levede.
\par 22 Salomos daglige Bebov af Levnedsmidler var tredive Kor fint Hvedemel og tresindstyve Kor almindeligt Mel,
\par 23 ti Stykker Fedekvæg, tyve Stykker Græskvæg og hundrede Stykker Småkvæg foruden Hjorte, Gazeller, Antiloper og fede Gæs.
\par 24 Thi han herskede over hele Landet hinsides Floden, fra Tifsa til Gaza, over alle Kongerne hinsides Floden, og han håvde Fred rundt om til alle Sider;
\par 25 og Juda og Israel boede trygt, så længe Salomo levede, hver Mand under sin Vinstok og sit Figentræ, fra ban til Be'ersjeba.
\par 26 Og Salomo havde 40 000 Spand Heste til sit Vognhold og 12 000 Ryttere.
\par 27 Og de nævnte Fogeder sørgede for Underhold til Kong Salomo og alle, der havde Adgang til hans Bord, hver i sin Måned, og de lod det ikke skorte på noget;
\par 28 og Byggen og Strået til Hestene og Forspandene bragte de til det Sted, hvor han var, efter som Turen kom til hver enkelt.
\par 29 Gud skænkede Salomo Visdom og Kløgt i såre rigt Mål og en omfattende Forstand, som Sandet ved Havets Bred,
\par 30 så at Salomos Visdom var større end alle Østerlændingenes og alle Ægypternes Visdom.
\par 31 Han var visere end alle andre, ja visere end Ezraiten Etan og visere end Heman, Kalkol og Darda, Mabols Sønner, og hans Ry nåede ud til alle Folkeslag rundt om.
\par 32 Han fremsagde 3000 Tanke sprog, og Tallet på hans Sange var 1 005"
\par 33 Han talte om Træerne lige fra Cederen på Libanon til Ysopen, der vokser frem af Muren; og han talte om Dyrene, Fuglene, Krybdyrene og Fiskene.
\par 34 Fra alle Folkeslag kom man for at lytte til Salomos Visdom, fra alle Jordens Konger, der hørte om hans Visdom.

\chapter{5}

\par 1 Da Kong Hiram af Tyrus hørte, at Salomo var salvet til Konge i stedet for sin Fader, sendte han nogle af sine Folk til ham; thi Hiram havde altid været Davids Ven.
\par 2 Og Salomo sendte Hiram følgende Bud:
\par 3 "Du ved, at min Fader David ikke kunde bygge HERREN sin Guds Navn et Hus for de Kriges Skyld, man fra alle Sider påførte ham, indtil HERREN lagde hans Fjender under hans Fødder.
\par 4 Men nu har HERREN min Gud skaffet mig Ro til alle Sider; der findes ingen Modstandere, og der er ingen Fare på Færde.
\par 5 Se, derfor har jeg i Sinde at bygge HERREN min Guds Navn et Hus efter HERRENs Ord til min Fader David: Din Søn, som jeg sætter på din Trone i dit Sted, han skal bygge Huset for mit Navn.
\par 6 Giv derfor Ordre til, at der fældes Cedre til mig på Libanon. Mine Folk skal arbejde sammen med dine, og jeg vil give dineFolk den Løn, du kræver; thi du ved jo, at der ikke findes nogen hos os, der kan fælde Træer som Zidonierne!"
\par 7 Da Hiram modtog dette Bud fra Salomo, glædede det ham meget, og han sagde: "Lovet være HERREN i dag, fordi han har givet David en viis Søn til at herske over dette store Folk!"
\par 8 Og Hiram sendte Salomo følgende Svar: "Jeg har modtaget det Bud, du sendte mig, og jeg skal opfylde alt, hvad du ønsker med Hensyn til Ceder- og Cyprestræer;
\par 9 mine Folk skal bringe dem fra Libanon ned til Havet, og så skal jeg lade dem samle til Tømmerflåder på Havet og sende dem til det Sted, du anviser mig; der skal jeg lade dem skille ad, så at du kan lade dem hente. Men du vil da også opfylde mit Ønske og sende Fødevarer til mit Hof!"
\par 10 Så sendte Hiram Salomo alt, hvad han ønskede af Ceder- og Cyprestræer;
\par 11 og Salomo sendte Hiram 20 000 Kor Hvede til Underhold for hans Hof og 20 Kor Olie af knuste Oliven. Så meget sendte Salomo Hiram År efter År.
\par 12 Og HERREN gav Salomo Visdom, som han havde lovet ham; og der var Fred mellem Hiram og Salomo, og de sluttede Pagt med hinanden.
\par 13 Kong Salomo udskrev nu Hoveriarbejdere overalt i Israel, og Hoveriarbejderne udgjorde 30 000 Mand.
\par 14 Dem sendte han så til Libanon, et Hold på 10 000 om Måneden, således at de var en Måned på Libanon og to Måneder hjemme. Adoniram havd Tilsyn med Hoveriarbejderne.
\par 15 Salomo havde 70 000 Lastdragere og 80 000 Stenhuggere i Bjergene
\par 16 foruden Overfogeder, der ledede Arbejdet, 3 300 Mand, som havde Opsyn med Folkene, der arbejdede.
\par 17 På Kongens Bud brød de store Stenblokke, kostbare Sten til Templets Grundvold, Kvadersten;
\par 18 og Salomos og Hiroms Bygmestre og Folkene fra Gebal huggede dem til, og de gjorde Træstammerne og Stenene i Stand til Templets Opførelse.

\chapter{6}

\par 1 480 År efter at Israeliterne var vandret ud af Ægypten, i Ziv Måned, det er den anden Måned, i det fjerde År Salomo herskede i Israel, begyndte han at bygge HERREN Templet.
\par 2 Templet, som Kong Salomo byggede HERREN, var tresindstyve Alen langt, tyve Alen bredt og tredive Alen højt.
\par 3 Forhallen foran Templets Hellige var tyve Alen lang, svarende til Templets Bredde, og ti Alen bred.
\par 4 Han forsynede Templet med Gittervinduer i Bjælkerammer,
\par 5 og op til Tempelmuren byggede han en Tilbygning rundt om Templets Mure, rundt om det Hellige og inderhallen, og indrettede Siderum rundt om.
\par 6 Det nederste Rum var fem Alen bredt, det mellemste seks og det tredje syv, thi han byggede Fremspring i Templets Ydermur rundt om, for at man ikke skulde være nødt til at lade Bjælkerne gribe ind i Templets Mure.
\par 7 Ved Templets Opførelsebyggede man med Sten, der var gjort færdige i Stenbrudet, derfor hørtes hverken Lyd af Hamre, Mejsler eller andet Jernværktøj, medens Templet byggedes.
\par 8 Indgangen til det nederste Rum var på Templets Sydside, og derfra førte Vindeltrapper op til det mellemste og derfra igen op til det tredje Rum.
\par 9 Således byggede han Templet færdigt, og han lagde Taget med Bjælker og Planker af Cedertræ.
\par 10 Han byggede Tilbygningen rundt om hele Templet, hvert Stokværk fem Alen højt, og den blev forbundet med Templet med Cederbjælker.
\par 11 Da kom HERRENs Ord til Salomo således:
\par 12 "Dette Hus, som du er ved at bygge - dersom du vandrer efter mine Anordninger og gør efter mine Lovbud og omhyggeligt vandrer efter alle mine Bud, vil jeg på dig stadfæste det Ord, jeg talede til din Fader David, -
\par 13 og tage Bolig blandt Israeliterne og ikke forlade mit Folk Israel."
\par 14 Således byggede Salomo Templet færdigt.
\par 15 Templets Vægge dækkede han indvendig med Cederbrædder; fra Bygningens Gulv til Loftsbjælkerne dækkede han dem indvendig med Træ; og over Templets Gulv lagde han Cypresbrædder.
\par 16 Han dækkede de tyve bageste Alen af Templet med Cederbrædder fra Gulv til Bjælker og indrettede sig Rummet derinde til en Inderhal, det Allerhelligste.
\par 17 Fyrretyve Alen målte det Hellige foran Inderhallen.
\par 18 Templet var indvendig dækket med Cedertræ, udskåret Arbejde i Form af Agurker og Blomsterkranse; alt var af Cedertræ, ikke en Sten var at se.
\par 19 Han indrettede inderhallen inde i Templet for der at opstille HERRENs Pagts Ark.
\par 20 Inderhallen var tyve Alen lang, tyve Alen bred og tyve Alen høj, og han overtrak den med fint Guld. Fremdeles lavede han et Alter af Cedertræ
\par 21 foran Inderhallen og overtrak det med Guld. Og Salomo overtrak Templet indvendig med fint Guld og trak for med Guldkæder.
\par 22 Hele Templet overtrak han med Guld, hele Templet fra den ene Ende til den anden; også hele Alteret foran inderhallen overtrak han med Guld.
\par 23 I Inderhallen satte han to Keruber af vildt Oliventræ, ti Alen høje;
\par 24 den ene Kerubs Vinger var hver fem Alen, der var ti Alen fra den ene Vingespids til den anden;
\par 25 ti Alen målte også den anden Kerub; begge keruber havde samme Mål og Skikkelse;
\par 26 begge Keruber var ti Alen høje.
\par 27 Og han opstillede Keruberne midt i den inderste Del af Templet, og de udbredte deres Vinger således, at den enes ene Vinge rørte den ene Væg og den andens ene Vinge den modsatte Væg, medens de to andre Vinger rørte hinanden midt i Templet.
\par 28 Keruberne overtrak han med Guld.
\par 29 Rundt på alle Vægge i Templet anbragte han udskåret Arbejde, Keruber, Palmer og Blomsterkranse, både i den inderste og den yderste Hal,
\par 30 og Templets Gulv overtrak han med Guld, både i den inderste og den yderste Hal.
\par 31 Til Inderhallens Indgang lod han lave to Dørfløje af vildt Oliventræ; Overliggeren og Dørposterne dannede en Femkant.
\par 32 Og på de to Oliventræsfløje lod han udskære Keruber, Palmer og Blomsterkranse og overtrak dem med Guld, idet han lod Guldet trykke ned i de udskårne Keruber og Palmer.
\par 33 Ligeledes lod han til indgangen til det Hellige lave Dørstolper af vildt Oliventræ, firkantede Dørstolper,
\par 34 og to Dørfløje af Cyprestræ, således at hver af de to Dørfløje bestod af to bevægelige Dørflader;
\par 35 og han lod udskære Keruber, Palmer og Blomsterkranse i dem og overtrak dem med Guld, der lå i et tyndt Lag over de udskårne Figurer.
\par 36 Ligeledes indrettede han den indre Forgård ved at bygge tre Lag Kvadersten og et Lag Cederbjælker.
\par 37 I det fjerde År lagdes Grunden til HERRENs Hus i Ziv Måned;
\par 38 og i det ellevte År i Bul Måned, det er den ottende Måned, fuldførtes Templet i alle dets Dele og Stykker; han byggede på det i syv År.

\chapter{7}

\par 1 På sit Palads byggede Salomo i tretten År; så fik han hele sit Palads færdigt.
\par 2 Han byggede Libanonskovhuset, hundrede Alen langt, halvtredsindstyve Alen bredt og tredive Alen højt, hvilende på tre Rækker Cedersøjler med Skråstøtter af Cedertræ.
\par 3 Det var oven over Rummene tækket med Cederbjælker, der hvilede på fem og fyrretyve Søjler, femten i hver Række.
\par 4 Der var tre Lag Bjælker, og Lysåbning sad over for Lysåbning tre Gange.
\par 5 Alle Døre og Lysåbninger havde firkantede Bjælkerammer, og Lysåbning sad over for Lysåbning tre Gange.
\par 6 Fremdeles opførte han Søjlehallen, halvtredsindstyve Alen lang og tredive Alen bred, med en Hal, Søjler og Trappe foran.
\par 7 Fremdeles opførte han Tronhallen, hvor han holdt Rettergang, Domhallen; den var dækket med Cedertræ fra Gulv til Loft,
\par 8 Hans eget Hus, det, han boede i, i den anden Forgård inden for Hallen, var bygget på samme Måde. Og til Faraos Datter, som Salomo havde ægtet, opførte han et Hus i Lighed med denne Hal.
\par 9 Det hele var af kostbare Sten, tilhugget efter Mål, tilsavet både indvendig og udvendig, lige fra Grunden til Murkanten, hvilket også gjaldt den store Forgård uden om Templets Forgård.
\par 10 Grunden blev lagt med kostbare, store Sten, nogle på ti, andre på otte Alen.
\par 11 Ovenpå lagdes kostbare Sten, tilhugget efter Mål, og Cederbjælker.
\par 12 Den store Forgård var hele Vejen rundt omgivet af tre Lag tilhugne Sten og et Lag Cederbjælker, ligeledes HERRENs Huss Forgård, den indre, og Forgården om Paladsets Forhal.
\par 13 Kong Salomo sendte Bud til Tyrus efter Hiram.
\par 14 Han var Søn af en Enke fra Naftalis Stamme, men hans Fader var en Kobbersmed fra Tyrus. Han sad inde med Visdom, Forstand og Indsigt i at udføre alskens Kobberarbejde; og han kom til Kong Salomo og udførte alt det Arbejde, han skulde have udført.
\par 15 Han støbte de to kobbersøjler foran Forhallen. Den ene var atten Alen høj; den målte tolv Alen i Omkreds; den var hul, og Kobberet var fire Fingerbredder tykt. Ligeså den anden Søjle.
\par 16 Og han lavede to Søjlehoveder til at sidde oven på Søjlerne, støbt af Kobber, hvert Søjlehoved fem Alen højt.
\par 17 Og han lavede to Fletværker, flettet Arbejde, Snore, kædeformet Arbejde, til at dække Søjlehovederne oven på Søjlerne, et Fletværk fil hvert Søjleboved;
\par 18 og han lavede Granatæblerne, to Rækker rundt om det ene Fletværk; der var 200 Granatæbler i Rækker rundt om det ene Søjlehoved; på samme Måde gjorde han også ved det andet.
\par 19 Søjlehovedeme på de to Søjler var liljeformet Arbejde.
\par 20 Søjlehovederne sad på de to Søjler.
\par 21 Derpå opstillede han Søjlerne ved Templets Forhal; den Søjle, han opstillede til højre, kaldte han Jakin, og den, han opstillede til venstre, kaldte han Boaz.
\par 22 Øverst på Søjlerne var der liljeformet Arbejde. Således blev Arbejdet med Søjlerne færdigt.
\par 23 Fremdeles lavede han Havet i støbt Arbejde, ti Alen fra Rand til Rand, helt rundt, fem Alen højt; det målte tredive Alen i Omkreds.
\par 24 Under Randen var det hele Vejen rundt omgivet af agurklignende Prydelser, der nåede helt omkring Havet, tredive Alen; i to Rækker sad de agurklignende Prydelser, støbt i eet dermed.
\par 25 Det stod på tolv Okser, således at tre vendte mod Nord, tre mod Vest, tre mod Syd og tre mod Øst; Havet stod oven på dem; de vendte alle Bagkroppen indad.
\par 26 Det var en Håndsbred tykt, og Randen var formet som Randen på et Bæger, som en udsprungen Lilje. Det tog 2000 Bat.
\par 27 Fremdeles lavede han de ti Vognstel af Kobber; hvert Stel var fire Alen langt, fire Alen bredt og tre Alen højt.
\par 28 Og Stellene var indrettet så ledes: De havde Mellemstykker, og Mellemstykkerne sad mellem Rammestykkerne.
\par 29 På Mellemstykkerne mellem Rammestykkerne var der Løver, Okser og Keruber, ligeledes på Rammestykkerne. Over og under Løverne og Okserne var der Kranse, lavet således, at de hang ned.
\par 30 Hvert Stel havde fire Kobber hjul og Kobberaksler. De fire Hjørner havde Bærearme; under Bækkenet var Bærearmene faststøbt, og midt for hver af dem var der Kranse.
\par 31 Dets Rand var inden for Bærearmene, een Alen høj, og den var rund: også på Randen var der udskåret Arbejde. Mellemstykkerne var firkantede, ikke runde.
\par 32 De fire Hjul sad under Mellemstykkerne, og Hjulenes Akselholdere sad på Stellet; hvert Hjul var halvanden Alen højt.
\par 33 Hjulene var indrettet som Vognhjul, og deres Akselholdere, Fælge, Eger og Nav var alle støbt.
\par 34 Der var en Bæream på hvert Stels fire Hjørner, og Bærearmene var i eet med Stellet;
\par 35 og oven på Stellet var der en Slags Fatning, en halv Alen høj og helt rund; og Akselholdere og Mellemstykker sad fast på Stellet.
\par 36 På Fladerne indgraverede han Keruber, Løver og Palmer, efter som der var Plads til, omgivet af Kranse.
\par 37 Således lavede han de ti Stel; de var alle støbt på samme Måde, med samme Mål og af samme Form.
\par 38 Tillige lavede han ti Kobberbækkener; fyrretyve Bat tog hvert Bækken, og hvert Bækken målte fire Alen, et Bækken til hvert af de ti Stel.
\par 39 Og han satte fem af Stellene ved Templets Sydside, fem ved Nordsiden; og Havet opstillede han ved Templets Sydside, ved det sydøstre Hjørne.
\par 40 Fremdeles lavede Hirom Karrene, Skovlene og Skålene. Der med var Hiram færdig med alt sit Arbejde for Kong Salomo til HERRENs Hus:
\par 41 De to Søjler, og de to kuglefornede Søjlehoveder ovenpå, de to Fletværker til at dække de to kugleformede Søjlehoveder på Søjlerne,
\par 42 de 400 Granatæbler til de to Fletværker, to Rækker Granatæbler til hvert Fletværk til at dække de to kugleformede Søjlehoveder på de to Søjler,
\par 43 de ti Stel med de ti Bækkener på,
\par 44 Havet med de tolv Okser under,
\par 45 Karrene, Skovlene og Skålene. Alle disse Ting, som Hiram lavede for Kong Salomo til HERRENs Hus, var af blankt Kobber.
\par 46 I Jordanegnen lod Kongen dem støbe, ved Adamas Vadested mellem Sukkot og Zaretan.
\par 47 Salomo lod alle Tingene uvejet på Grund af deres såre store Mængde, Kobberet blev ikke vejet.
\par 48 Og Salomo lod alle Tingene, som hørte til HERRENs Hus, lave: Guldalteret, Guldbordet, som Skuebrødene lå på,
\par 49 Lysestagerne, fem til højre og fem til venstre, foran Inderhallen, af purt Guld, med Blomsterbægrene, Lamperne og Lysesaksene af Guld,
\par 50 Fadene, Knivene, Skålene, Kanderne og Panderne af fint Guld, Hængslerne til Dørene for den inderste Hal, det Allerhelligste, og til Dørene for den yderste Hal, det Hellige, af Guld.
\par 51 Da hele Arbejdet, som Salomo lod udføre ved HERRENs Hus, var færdigt, bragte Salomo sin Fader Davids Helliggaver, Sølvet og Guldet, derind og lagde alle Tingene i Skatkamrene i HERRENs Hus.

\chapter{8}

\par 1 Derpå kaldte Salomo Israels Ældste og alle Stammernes Overhoveder, Israeliternes Fædrenehuses Øverster, sammen hos sig i Jerusalem for at føre HERRENs Pagts Ark op fra Davidsbyen, det er Zion.
\par 2 Så samledes alle Israels Mænd hos Kong Salomo på Højtiden i Etanim Måned, det er den syvende Måned.
\par 3 Og alle Israels Ældste kom, og Præsterne bar Arken.
\par 4 Og de bragte HERRENs Ark op tillige med Åbenbaringsteltet og alle de hellige Ting, der var i Teltet; Præsterne og Leviterne bragte dem op.
\par 5 Og Kong Salomo tillige med hele Israels Menighed, som havde givet Møde hos ham foran Arken, ofrede Småkvæg og Hornkvæg, så meget, at det ikke var til at tælle eller overse.
\par 6 Så førte Præsterne HERRENs Pagts Ark ind på dens Plads i Templets Inderhal, det Allerhelligste, og stillede den under Kerubernes Vinger;
\par 7 thi Keruberne udbredte deres Vinger over Pladsen, hvor Arken stod, og således dannede Keruberne et Dække over Arken og dens Bærestænger.
\par 8 Stængerne var så lange, at Enderne af dem kunde ses fra det Hellige foran Inderhallen, men de kunde ikke ses længere ude; og de er der den Dag i Dag.
\par 9 Der var ikke andet i Arken end de to Stentavler, Moses havde lagt ned i den på Horeb, Tavlerne med den Pagt, HERREN havde sluttet med Israeliterne, da de drog bort fra Ægypten.
\par 10 Da Præsterne derpå gik ud af Helligdommen, fyldte Skyen HERRENs Hus,
\par 11 så at Præsterne af Skyen hindredes i at stå og udføre deres Tjeneste; thi HERRENs Herlighed fyldte HERRENs Hus.
\par 12 Ved den Lejlighed sang Salomo: HERREN satte Solen på Himlen, men selv, har han sagt, vil han bo i Mulmet.
\par 13 Nu har jeg bygget dig et Hus til Bolig, et Sted, du for evigt kan dvæle.
\par 14 Derpå vendte Kongen sig om og velsignede hele Israels Forsamling, der imens stod op;
\par 15 og han sagde: "Lovet være HERREN, Israels Gud, hvis Hånd har fuldført, hvad hans Mund talede til min Fader David, dengang han sagde:
\par 16 Fra den Dag jeg førte mit Folk Israel ud af Ægypten, har jeg ikke udvalgt nogen By i nogen af Israels Stammer for der at bygge et Hus til Bolig for mit Navn; men Jerusalem udvalgte jeg til Bolig for mit Navn, og David udvalgte jeg til at herske over mit Folk Israel.
\par 17 Og min Fader David fik i Sinde at bygge HERRENs, Israels Guds, Navn et Hus;
\par 18 men HERREN sagde til min Fader David: At du har i Sinde at bygge mit Navn et Hus, er ret af dig;
\par 19 dog skal du ikke bygge det Hus, men din Søn, der udgår af din Lænd; skal bygge mit Navn det Hus.
\par 20 Nu har HERREN opfyldt det Ord, han talede, og jeg er trådt i min Fader Davids Sted og sidder på Israels Trone, som HERREN sagde, og jeg har bygget HERRENs, Israels Guds, Navn Huset;
\par 21 og jeg har der beredt en Plads til Arken med den Pagt, HERREN sluttede med vore Fædre, da han førte dem bort fra Ægypten."
\par 22 Derpå trådte Salomo frem foran HERRENs Alter lige over for hele Israels Forsamling, udbredte sine Hænder mod Himmelen
\par 23 og sagde: "HERRE Israels Gud, der er ingen Gud som du i Himmelen oventil og på Jorden nedentil, du, som holder fast ved din Pagt og din Miskundhed mod dine Tjenere, når de af hele deres Hjerte vandrer for dit Åsyn,
\par 24 du, som har holdt, hvad du lovede din Tjener, min Fader David, og i Dag opfyldt med din Hånd, hvad du talede med din Mund.
\par 25 Så hold da nu, HERRE, Israels Gud, hvad du lovede din Tjener, min Fader David, da du sagde: En Efterfølger skal aldrig fattes dig til at sidde på Israels Trone for mit Åsyn, når kun dine Sønner vil tage Vare på deres Vej og vandre for mit Åsyn, som du har gjort.
\par 26 Så lad nu, HERRE, Israels Gud, det Ord opfyldes, som du tilsagde din Tjener, min Fader David!
\par 27 Men kan Gud da virkelig bo på Jorden? Nej visselig, Himlene, ja Himlenes Himle kan ikke rumme dig, langt mindre dette Hus, som jeg har bygget!
\par 28 Men vend dig til din Tjeners Bøn og Begæring, HERRE min Gud, så du hører det Råb og den Bøn, din Tjener i Dag opsender for dit Åsyn;
\par 29 lad dine Øjne være åbne over dette Hus både Nat og Dag, over det Sted, hvor du har sagt, dit Navn skal bo, så du hører den Bøn, din Tjener opsender, vendt mod dette Sted!
\par 30 Og hør den Bøn, din Tjener og dit Folk Israel opsender, vendt mod dette Sted; du høre den der, hvor du bor, i Himmelen, du høre og tilgive!
\par 31 Når nogen synder imod sin Næste, og man afkræver ham Ed og lader ham sværge, og han kommer og aflægger Ed foran dit Alter i dette Hus,
\par 32 så høre du det i Himmelen og gøre det og dømme dine Tjenere imellem, så du kender den skyldige skyldig og lader hans Gerning komme over hans Hoved og frikender den uskyldige og gør med ham efter hans Uskyld!
\par 33 Når dit Folk Israel tvinges til at fly for en Fjende, fordi de synder imod dig, og de så omvender sig til dig og bekender dit Navn og opsender Bønner og Begæringer til dig i dette Hus,
\par 34 så høre du det i Himmelen og tilgive dit Folk Israels Synd og føre dem tilbage til det Land, du gav deres Fædre!
\par 35 Når Himmelen lukkes, så Regnen udebliver, fordi de synder imod dig, og de så beder, vendt mod dette Sted, og bekender dit Navn og omvender sig fra deres Synd, fordi du revser dem,
\par 36 så høre du det i Himmelen og tilgive din Tjeners og dit Folk Israels Synd, ja du vise dem den gode Vej, de skal vandre, og lade det regne i dit Land, som du gav dit Folk i Eje!
\par 37 Når der kommer Hungersnød i Landet, når der kommer Pest, når der kommer Kornbrand og Rust, Græshopper og Ædere, når Fjenden belejrer Folket i en af dets Byer, når alskens Plage og Sot indtræffer -
\par 38 enhver Bøn, enhver Begæring, hvem den end kommer fra i hele dit Folk Israel, når de føler sig truffet i deres Samvittighed og udbreder Hænderne mod dette Hus,
\par 39 den høre du i Himmelen, der, hvor du bor, og tilgive og gøre det, idet du gengælder enhver hans Færd, fordi du kender hans Hjerte, thi du alene kender alle Menneskebørnenes Hjerter,
\par 40 for at de må frygte dig, al den Tid de lever på den Jord, du gav vore Fædre.
\par 41 Selv den fremmede, der ikke hører til dit Folk Israel, men kommer fra et fjernt Land for dit Navns Skyld, -
\par 42 thi man vil høre om dit store Navn, din stærke Hånd og din udstrakte Arm - når han kommer og beder, vendt mod dette Hus,
\par 43 da høre du det i Himmelen, der, hvor du bor, og da gøre du efter alt, hvad den fremmede råber til dig om, for at alle Jordens Folkeslag må lære dit Navn at kende og frygte dig ligesom dit Folk Israel og erkende, at dit Navn er nævnet over dette Hus, som jeg har bygget.
\par 44 Når dit Folk drager i Krig mod sin Fjende, hvor du end sender dem hen, og de beder til HERREN, vendt mod den By, du har udvalgt, og det Hus, jeg har bygget dit Navn,
\par 45 så høre du i Himmelen deres Bøn og Begæring og skaffe dem deres Ret!
\par 46 Når de synder imod dig thi der er intet Menneske, som ikke synder - og du vredes på dem og giver dem i Fjendens Magt, og Sejrherrerne fører dem fangne til Fjendens Land, det være sig fjernt eller nær,
\par 47 og de så går i sig selv i det Land, de er bortført til, og omvender sig og råber til dig i Sejrherrernes Land og siger: Vi har syndet, handlet ilde og været ugudelige!
\par 48 når de omvender sig til dig af hele deres Hjerte og af hele deres Sjæl i deres Fjenders Land, som de bortførtes til, og de beder til dig, vendt mod deres Land, som du gav deres fædre, mod den By, du har udvalgt, og det Hus, jeg har bygget dit Navn
\par 49 så høre du i Himmelen, der, hvor du bor, deres Bøn og Begæring og skaffe dem deres Ret,
\par 50 og du tilgive dit Folk, hvad de syndede imod dig, alle de Overtrædelser, hvori de gjorde sig skyldige imod dig, og lade dem finde Barmhjertighed hos Sejrherrerne, så de forbarmer sig over dem;
\par 51 de er jo dit Folk og din Ejendom, som du førte ud af Ægypten, af Smelteovnen.
\par 52 Lad dine Øjne være åbne for din Tjeners og dit Folk Israels Begæring, så du hører dem, hver Gang de råber til dig.
\par 53 Thi du har udskilt dem fra alle Jordens Folkeslag til at være din Ejendom, som du lovede ved din Tjener Moses, da du førte vore Fædre bort fra Ægypten, Herre, HERRE!"
\par 54 Da Salomo var færdig med hele denne Bøn og Begæring til HERREN rejste han sig fra Pladsen foran HERRENs Alter, hvor han havde ligget på Knæ med Hænderne udbredt mod Himmelen.
\par 55 Derpå trådte han frem og velsignede med høj Røst hele Israels Forsamling, idet han sagde:
\par 56 "Lovet være HERREN, der har givet sit Folk Israel Hvile, ganske som han talede, uden at et eneste Ord er faldet til Jorden af alle de herlige Forjættelser, han udtalte ved sin Tjener Moses.
\par 57 HERREN vor Gud være med os, som han var med vore Fædre, han forlade og forstøde os ikke,
\par 58 at vort Hjerte må drages til ham, så vi vandrer på alle hans Veje og holder hans Bud, Anordninger og Lovbud, som han pålagde vøre Fædre!
\par 59 Måtte disse Bønner, som jeg har opsendt for HERRENs Åsyn, være nærværende for HERREN vor Gud både Nat og Dag, så han skaffer sin Tjener og sit Folk Israel Ret efter hver Dags Behov,
\par 60 for at alle Jordens Folk må kende, at HERREN og ingen anden er Gud.
\par 61 Og måtte eders Hjerte være helt med HERREN vor Gud, så I følger hans Anordninger og holder hans Bud som i Dag!"
\par 62 Kongen ofrede nu sammen med hele Israel Slagtofre for HERRENs Åsyn.
\par 63 Til de Takofre, Salomo ofrede til HERREN, tog han 22 000 Stykker Hornkvæg og 12 000 Stykker Småkvæg. Således indviede Kongen og alle Israeliterne HERRNs Hus.
\par 64 Samme Dag helligede Kongen den mellemste Del af Forgården foran HERRENs Hus, thi der måtte han ofre Brændofrene, Afgrødeofrene og Fedtstykkerne af Takofrene, da Kobberalteret foran HERRENs Åsyn var for lille til at rumme Ofrene.
\par 65 Samtidig fejrede Salomo i syv Dage Højtiden for HERREN vor Guds Åsyn sammen med hele Israel, en vældig Forsamling (lige fra Egnen ved Hamat og til Ægyptens Bæk).
\par 66 Ottendedagen lod han Folket gå, og de velsignede Kongen og drog hver til sit, glade og vel til Mode over al den Godhed, HERREN havde vist sin Tjener David og sit Folk Israel.

\chapter{9}

\par 1 Men da Salomo var færdig med at opføre HERRENs Hus og Kongens Palads og alt, hvad han havde fået Lyst til og sat sig for at udføre,
\par 2 lod HERREN sig anden Gang til Syne for ham, som han havde ladet sig til Syne for ham i Gibeon;
\par 3 og HERREN sagde til ham: "Jeg har hørt den Bøn og Begæring, du opsendte for mit Åsyn. Jeg har helliget dette Hus, som du har bygget, for der at stedfæste mit Navn til evig Tid, og mine Øjne og mit Hjerte skal være der alle Dage.
\par 4 Hvis du nu vandrer for mit Åsyn som din Fader David i Hjertets Uskyld og i Upriglighed, så du gør alt, hvad jeg har pålagt dig, og holder mine Anordninger og Lovbud,
\par 5 så vil jeg opretholde din Kongetrone i Israel evindelig, som jeg lovede din Fader David, da jeg sagde: En Efterfølger skal aldrig fattes dig på Israels Trone.
\par 6 Men hvis I eller eders Børn vender eder bort fra mig og ikke holder mine Bud, mine Annordninger, som jeg har forelagt eder, men går hen og dyrker andre Guder og tilbeder dem
\par 7 så vil jeg udrydde Israel fra det Land, jeg gav dem; og det Hus, jeg har helliget for mit Navn, vil jeg forkaste fra mit Åsyn, og Israel skal blive til Spot og Spe blandt alle Folk,
\par 8 og dette Hus skal blive en Ruindynge, og enhver, som går der forbi, skal blive slået af Rædsel og give sig til at hånfløjte. Og når man siger: Hvorfor har HERREN handlet således mod dette Land og dette Hus?
\par 9 skal der svares: Fordi de forlod HERREN deres Gud, som førte deres Fædre ud af Ægypten, og holdt sig til andre Guder, tilbad og dyrkede dem; derfor har HERREN bragt al denne Elendighed over dem!"
\par 10 Da de tyve År var omme, i hvilke Salomo havde bygget på de to Bygninger, HERRENs Hus og Kongens Palads -
\par 11 Kong Hiram af Tyrus havde sendt Salomo Cedertræ, Cyprestræ og Guld, så meget han ønskede da gav Kong Salomo Hiram tyve Byer i Landskabet Galilæa.
\par 12 Men da Hiram kom fra Tyrus for at se de Byer, Salomo havde givet ham, syntes han ikke om dem;
\par 13 og han sagde: "Hvad er det for Byer, du har givet mig, Broder?" Derfor kaldte man den Habullandet, som det hedder den Dag i Dag
\par 14 Men Hiram sendte Kongen l20 Guldtalenter.
\par 15 På følgende Måde hang det sammen med de Hoveriarbejdere, Kong Salomo udskrev til at opføre HERRENs Hus, hans eget Palads, Millo Jerusalems Mur, Hazor, Megiddo og Gezer
\par 16 Farao, Ægypterkongen, var draget op, havde indtaget Gezer og stukket det i Brand; alle Kanånæere, der boede i Byen, havde han ladet dræbe og derpå givet sin Datter, Salomos Hustru den i Medgift.
\par 17 Nu genopbyggede Salomo Gezer, Nedre-Bet-Horon,
\par 18 Bålat, Tamar i Ørkenen i Juda Land,
\par 19 alle Salomos Forrådsbyer, Vognbyerne og Rytterbyerne, og alt andet, som Salomo fik Lyst til at bygge i Jerusalem, i Libanon og i hele sit Rige:
\par 20 Alt, hvad der var tilbage af Amoriterne, Hetiterne, Perizziterne, Hivviterne og Jebusiterne, og som ikke hørte til Israeliterne,
\par 21 deres Efterkommere, som var tilbage efter dem i Landet, og som Israeliterne ikke havde været i Stand til at lægge Band på, dem udskrev Salomo til Hoveriarbejde, som det er den Dag i Dag.
\par 22 Af Israeliterne derimod satte Salomo ingen til Arbejde,men de var Krigsfolk og Hoffolk, Hærførere og Høvedsmænd hos, ham og Førere for hans Stridsvogne og Rytteri. -
\par 23 Tallet på Overfogederne, der ledede Arbejdet for Salomo, var 550; de havde Tilsyn med Folkene, der arbejdede.
\par 24 Faraos Datter var lige flyttet fra Davidsbyen ind i det Hus, han havde bygget til hende, da tog han fat på at opføre Millo.
\par 25 Tre Gange om Året ofrede Salomo Brændofre og Takofre på det Alter, han bavde bygget HERREN, og tændte Offerild for HERRENs Åsyn; og han fuldførte Templet.
\par 26 Kong Salomo byggede også Skibe i Ezjongeber, der ligger ved Elat ved det røde Havs Kyst i Edom;
\par 27 og Hiram sendte sine Folk, befarne Søfolk, om Bord på Skibene sammen med Salomos Folk.
\par 28 De sejlede til Ofir, hvor de hen tede 420 Talenter Guld, som de bragte Kong Salomo.

\chapter{10}

\par 1 Da Dronningen af Saba hørte Salomos Ry, kom hun for at prøve ham med Gåder.
\par 2 Hun kom til Jerusalem med et såre stort Følge og med Kameler, der bar Røgelse, Guld i store Mængder og Ædelsten. Og da hun var kommet til Salomo, talte hun til ham om alt, hvad der lå hende på Hjerte.
\par 3 Men Salomo svarede på alle hendes Spørgsmål, og intet som helst var skjult for Kongen, han gav hende Svar på alt.
\par 4 Og da Dronningen af Saba så al Salomos Visdom, Huset han havde bygget,
\par 5 Maden på hans Bord, hans Folks Boliger, han træden og deres Klæder, hans Mundskænke og Brændofrene, han ofrede i HERRENs Hus, var hun ude af sig selv;
\par 6 og hun sagde til Kongen: "Sandt var, hvad jeg i mit Land hørte sige om dig og din Visdom!
\par 7 Jeg troede ikke, hvad der sagdes, før jeg kom og så det med egne Øjne; og se, ikke engang det halve er mig fortalt, thi din Visdom og Herlighed overgår, hvad rygte sagde.
\par 8 Lykkelige dine Hustruer, lykkelige dine Folk, som altid er om dig og hører din Visdom!
\par 9 Lovet være HERREN din Gud, som fandt behag i dig og satte dig på Israels Trone! Fordi HERREN elsker Israel evindelig, satte han dig til Konge, til at øve ret og Retfærdighed."
\par 10 Derpå gav hun Kongen 120 Guldtalenter, Røgelse i store Mængder og Ædelsten; og aldrig er der siden kommet så megen Røgelse til Landet som den, Dronningen af Saba gav Kong Salomo.
\par 11 Desuden bragte Hirams Skibe, som hentede Guld i Ofir, Almug gumtræ i store Mængder og Ædel sten fra Ofir,
\par 12 og af Almuggimtræet lod Kongen lave Rækværk til HERRENs Hus og Kongens Palads, desuden Citre og Harper til Sangerne. Så meget Almuggimtræ er hidtil ikke set eller kommet til Landet.
\par 13 Og Kong Salomo gav Dronningen af Saba alt, hvad hun ønskede og bad om, foruden hvad han af sig selv kongeligen skænkede hende. Derpå begav hun sig med sit Følge hjem til sit Land.
\par 14 Vægten af det Guld, som i et År indførtes af Salomo, udgjorde 666 Guldtalenter,
\par 15 de ikke medregnet, hvad der indkom i Afgift fra de undertvungne Folk og ved Købmændenes Handel og fra alle Arabiens Konger og Landets Statholdere.
\par 16 Kong Salomo lod hamre 200 Guldskjolde, hvert på 600 Sekel Guld,
\par 17 og 300 mindre Guldskjolde, hvert på tre Miner Guld; dem lod Kongen henlægge i Libanonskovhuset.
\par 18 Fremdeles lod Kongen lave en stor Elfenbenstrone, overtrukket med lutret Guld.
\par 19 Tronen havde seks Trin, og på dens Ryg var der Tyrehoveder; på begge Sider af Sædet var der Arme, og ved Armene stod der to Løver;
\par 20 tillige stod der tolv Løver påde seks Trin, seks på hver Side. Der er ikke lavet Mage til Trone i noget andet Rige.
\par 21 Alle Kong Salomos Drikkekar var af Guld og alle Redskaber i Libanonskovhuset af fint Guld; Sølv regnedes ikke for noget i Kong Salomos Dage.
\par 22 Kongen havde nemlig Tarsisskibe i Søen sammen med Hirams Skibe; og en Gang hvert tredje År kom Tarsisskibene, ladet med Guld, Sølv, Elfenben, Aber og Påfugle.
\par 23 Kong Salomo overgik alle Jordens Konger i Rigdom og Visdom.
\par 24 Fra alle Jordens Egne søgte man hen til Salomo for at høre den Visdom, Gud havde lagt i hans Hjerte;
\par 25 og alle bragte de Gaver med: Sølv og Guldsager, Klæder, Våben, Røgelse, Heste og Muldyr; således gik det År efter År.
\par 26 Salomo anskaffede sig Stridsvogne og Ryttere, og han havde 1.400 Vogne og l2.000 Ryttere; dem lagde han dels i Vognbyerne, dels hos sig i Jerusalem.
\par 27 Kongen bragte det dertil, at Sølv i Jerusalem var lige så almindeligt som Sten, og Cedertræ lige så almindeligt som Morbærfigentræ i Lavlandet. -
\par 28 Hestene, Salomo indførte, kom fra Mizrajim og Kove; Kongens Handelsfolk købte dem i Kove.
\par 29 En Vogn udførtes fra Mizrajim for 600 Sekel Sølv, en Hest for 150.

\chapter{11}

\par 1 Kong Salomo elskede foruden Faraos Datter mange fremmede Kvinder, moabitiske, ammonitiske, edomitiske, zidoniske og hetitiske Kvinder,
\par 2 Kvinder fra de Folkeslag, HERREN havde sagt om til Israeliterne: "I må ikke have med dem at gøre og de ikke med eder, ellers drager de eders Hjerte til deres Guder!" Ved dem hang Salomo i Kærlighed.
\par 3 Han havde 700 fyrsfelige Hustruer og 3OO Medhustruer, og hans Hustruer drog hans Hjerte bort fra Herren.
\par 4 Da Salomo blev gammel, drog hans Hustruer hans Hjerte til fremmede Guder, og hans Hjerte var ikke mere helt med HERREN hans Gud som hans fader Davids.
\par 5 Salomo holdt sig da til Astarte, Zidoniernes Gudinde, og til Milkom, Ammoniternes væmmelige Gud.
\par 6 Således gjorde Salomo, hvad der var ondt i HERRENs Øjne, og viste ikke HERREN fuld Lydighed som hans Fader David.
\par 7 Ved den Tid byggede Salomo en Offerhøj for Kemosj, Moabs væmmelige Gud, på Bjerget østen for Jerusalem, og for Milkom, Ammoniternes væmmelige Gud;
\par 8 og samme Hensyn viste han alle sine fremmede Hustruer, som tændte Offerild for deres Guder og ofrede til dem.
\par 9 Da vrededes HERREN på Salomo, fordi han vendte sit Hjerte bort fra HERREN, Israels Gud, der dog to Gange havde ladet sig til Syne for ham
\par 10 og udtrykkelig havde påbudt ham ikke at holde sig til fremmede Guder; men han holdt ikke, hvad HERREN havde påbudt ham.
\par 11 Derfor sagde HERREN til Salomo: "Fordi det står således til med dig, og fordi du ikke har holdt min Pagt og mine Anordninger, som jeg pålagde dig, vil jeg visselig rive Riget fra dig og give din Træl det.
\par 12 Dog vil jeg ikke gøre det i din Levetid for din Fader Davids Skyld men jeg vil rive det ud af din Søns Hånd.
\par 13 Kun vil jeg ikke rive hele Riget fra ham, men give din Søn en Stamme deraf for min Tjener Davids Skyld og for Jerusalems Skyld, den By, jeg udvalgte.
\par 14 HERREN gav Salomo en Modstander i Edomiten Hadad af Kongeslægten i Edom.
\par 15 Thi dengang David lod Edomiterne hugge ned, da Hærføreren Joab drog op for at jorde de faldne og hugge alle af Mandkøn ned i Edom -
\par 16 Joab og hele Israel blev der i seks Måneder, til han havde udryddet alle af Mandkøn i Edom -
\par 17 da var Adad med nogle edomitiske Mænd af hans Faders Folk flygtet ad Ægypten til. Dengang var Hadad endnu en lille Dreng.
\par 18 De brød op fra Midjan og nåede Paran; og efter at have taget nogle Mænd fra Paran med sig drog de til Ægypten, hvor Farao, Ægypterkongen, overlod ham et Hus, tilsagde ham daglig Føde og gav ham Land.
\par 19 Og da Farao fattede særlig Godhed for Hadad, gav han ham sin Svigerinde, en Søster til Dronning Takpenes, til Ægte.
\par 20 Takpeness Søster fødte ham Sønnen Genubat; og da Takpenes havde vænnet Barnet fra i Faraos Hus, blev Genubat i Faraos Hus blandt Faraos egne Børn.
\par 21 Da nu Hadad i Ægypten hørte, at David havde lagt sig til Hvile hos sine Fædre, og at Hærføreren Joab var død, sagde han til Farao: "Lad mig drage til mit Land!"
\par 22 Farao sagde til ham: "Hvad savner du her hos mig, siden du ønsker at drage til dit Land?" Men han svarede: "Å jo, lad mig nu rejse!" Så vendte Hadad tilbage til sit Land. Det var den ulykke, Hadad voldte: Han bragte Trængsel over Israel og blev Konge over Edom. -
\par 23 Fremdeles gav Gud ham en Modstander i Rezon, Eljadas Søn, der var flygtet fra sin Herre, Kong Hadadezer af Zoba.
\par 24 Han samlede en Del Mænd om sig og blev Høvding for en Friskare. Han indtog Damaskus, satte sig fast der og blev Konge i Damaskus.
\par 25 Han var Israels Modstander, så længe Salomo levede.
\par 26 Endvidere var der Efraimiten Jeroboam, Nebats Søn, fra Zereda, som stod i Salomos Tjeneste, og hvis Moder hed Zerua og var Enke; han løftede Hånd mod Kongen.
\par 27 Hermed gik det således til Salomo byggede på Millo; han lukkede Hullet i sin Fader Davids By.
\par 28 Nu var Jeroboam et dygtigt Menneske, og da Salomo så, hvorledes den unge Mand udførte Arbejdet, gav han ham Opsyn med hele Arbejdsstyrken af Josefs Hus.
\par 29 På den Tid hændte det sig, engang Jeroboam var rejst fra Jerusalem, at Profeten Ahija fra Silo traf ham på Vejen. Ahija var iført en ny Kappe, og de to var ene på Marken.
\par 30 Da greb Ahija fat i den ny Kappe, han havde på, rev den i tolv Stykker
\par 31 og sagde til Jeroboam: "Tag dig de ti Stykker, thi så siger HERREN, Israels Gud: Se, jeg river Riget ud af Salomos Hånd og giver dig de ti Stammer.
\par 32 Den ene Stamme skal han beholde for min Tjener Davids Skyld og for Jerusalems Skyld, den By, jeg udvalgte af alle Israels Stammer;
\par 33 det vil jeg gøre, fordi han har forladt mig og tilbedt Astarte, Zidoniernes Gudinde, Kemosj, Moabs Gud, og Milkom, Ammoniternes Gud, og ikke vandret på mine Veje og gjort, hvad der er ret i mine Øjne, eller holdt mine Anordninger og Lovbud som hans Fader David.
\par 34 Fra ham vil jeg dog ikke tage Riget, men lade ham være Fyrste, så længe han lever, for min Tjener Davids Skyld, som jeg udvalgte, og som holdt mine Bud og Anordninger.
\par 35 Men jeg vil tage Riget fra hans Søn og give dig det, de ti Stammer;
\par 36 og hans Søn vil jeg give en Stamme, for at min Tjener David altid kan have en Lampe for mit Åsyn i Jerusalem, den By, jeg udvalgte for der at stedfæste mit Navn.
\par 37 Men dig vil jeg tage og sætte til Hersker over alt, hvad du attrår, og du skal være Konge over Israel.
\par 38 Dersom du da er lydig i alt hvad jeg byder dig, vandrer på mine Veje og gør, hvad der er ret i mine Øjne, så du holder mine Anordnioger og Bud, som min Tjener David gjorde, vil jeg være med dig og bygge dig et varigt Hus, som jeg gjorde det for David. Dig giver jeg Israel;
\par 39 men jeg ydmyger Davids Slægt for den Sags Skyld, dog ikke for stedse!"
\par 40 Da nu Salomo stod Jeroboam efter Livet, flygtede han til Ægypten, til Ægypterkoogen Sjisjak; og han blev i Ægypten, til Salomo døde.
\par 41 Hvad der ellers er at fortælle om Salomo, alt, hvad han gjorde, og hans Visdom, står jo optegnet i Salomos Krønike.
\par 42 Den Tid, Salomo herskede i Jerusalem over hele Israel, udgjorde fyrretyve År.
\par 43 Så lagde Salomo sig til Hvile hos sine Fædre og blev jordet i sin Fader Davids By. Og hans Søn Rehabeam blev Konge i hans Sted.

\chapter{12}

\par 1 Men Rehabeam begav sig til Sikem, thi derhen var hele Israel stævnet for at hylde ham som Konge.
\par 2 Da Jeroboam, Nebats Søn, der endnu opholdt sig i Ægypten, hvorhen han var flygtet for Kong Salomo, fik Nys om, at Salomo var død, vendte han hjem fra Ægypten.
\par 3 Og de sagde til Rehabeam:
\par 4 "Din Fader lagde et hårdt Åg på os, men let du nu det hårde Arbejde, din Fader krævede, og det tunge Åg han lagde på os, så vil vi tjene dig!"
\par 5 Han svarede dem: "Gå bort, bi tre Dage og kom så til mig igen!" Så gik Folket.
\par 6 Derpå rådførte Kong Rehabeam sig med de gamle, der havde stået i hans Fader Salomos Tjeneste, dengang han levede, og spurgte dem: "Hvad råder I mig til at svare dette Folk?"
\par 7 De svarede: "Hvis du i Dag vil være dette Folk til Tjeneste, være dem til Behag, svare dem vel og give dem gode Ord, så vil de blive dine Tjenere for bestandig!"
\par 8 Men han fulgte ikke det Råd, de gamle gav ham; derimod rådførte han sig med de unge, der var vokset op sammen med ham og stod i hans Tjeneste,
\par 9 og spurgte dem: "Hvad råder I os til at svare dette Folk, som kræver af mig, at jeg skal lette dem det Åg, min Fader lagde på dem?"
\par 10 De unge, der var vokset op sammen med ham, sagde da til ham: "Således skal du svare dette Folk, som sagde til dig: Din Fader lagde et tungt Åg på os, let du det for os! Således skal du svare dem: Min Lillefinger er tykkere end min Faders Hofter!
\par 11 Har derfor min Fader lagt et tungt Åg på eder, vil jeg gøre Åget tungere; har min Fader tugtet eder med Svøber, vil jeg tugte eder med Skorpioner!"
\par 12 Da alt Folket Tredjedagen kom til Rehabeam, som Kongen havde sagt,
\par 13 gav han dem et hårdt Svar, og uden at tage Hensyn til de gamles Råd
\par 14 sagde han efter de unges Råd til dem: "Har min Fader lagt et tungt Åg på eder, vil jeg gøre Åget tungere; har min Fader tugtet eder med Svøber, vil jeg tugte eder med Skorpioner!"
\par 15 Kongen hørte ikke på Folket, thi HERREN føjede det således for at opfylde det Ord, HERREN havde talet ved Ahija fra Silo til Jeroboam, Nebats Søn.
\par 16 Men da hele Israel mærkede, at Kongen ikke hørte på dem, gav Folket Kongen det Svar: "Hvad Del har vi i David? Vi har ingen Lod i Isajs Søn! Til dine Telte, Israel! Sørg nu, David, for dit eget Hus!" Derpå vendte Israel tilbage til sine Telte.
\par 17 Men over de Israeliter, der boede i Judas Byer, blev Rehabeam Konge.
\par 18 Nu sendte Kong Rehabeam Adoniram, der havde Opsyn med Hoveriarbejdet, ud til dem, men hele Israel stenede ham til Døde. Da steg Kong Rebabeam i største Hast op på sin Stridsvogn og flygtede til Jerusalem.
\par 19 Således brød Israel med Davids Hus, og det er Stillingen den Dag i Dag.
\par 20 Men da hele Israel hørte, at Jeroboam var kommet tilbage, lod de ham hente til Forsamlingen og hyldede ham som Konge over hele Israel. Der var ingen, som holdt fast ved Davids Hus undtagen Judas Stamme.
\par 21 Da Rehabeam var kommet til Jerusalem, samlede han hele Judas Hus og Benjamins Stamme, 180 000 udsøgte Folk, øvede Krigere, til at føre Krig med Israels Hus og vinde Riget tilbage til Rehabeam, Salomos Søn.
\par 22 Men da kom Guds Ord til den Guds Mand Sjemaja således:
\par 23 "Sig til Judas Konge Rehabeam, Salomos Søn, og til hele Judas og Benjamins Hus og det øvrige Folk:
\par 24 Så siger HERREN: I må ikke drage op og kæmpe med eders Brødre Israeliterne; vend hjem hver til sit, thi hvad her er sket, har jeg tilskikket!" Da adlød de HERRENs Ord og vendte tilbage.
\par 25 Jeroboam befæstede Sikem i Efraims Bjerge og tog Bolig der; senere drog han derfra og befæstede Penuel.
\par 26 Men Jeroboam tænkte ved sig selv: "Som det nu går, vil Riget atter tilfalde Davids Hus;
\par 27 når Folket her drager op for at ofre i HERRENs Hus i Jerusalem, vil dets Hu atter vende sig til dets Herre, Kong Rehabeam af Juda; så slår de mig ihjel og vender tilbage til Kong Rehabeam af Juda!"
\par 28 Og da Kongen havde overvejet Sagen, lod han lave to Guldkalve og sagde til Folket "Det er for meget for eder med de Rejser til Jerusalem! Se, Israel, her er dine Guder, som førte dig ud af Ægypten!"
\par 29 Den ene lod han opstille i Betel, den anden i Dan.
\par 30 Det blev Israel til Synd. Og Folket ledsagede i Optog den ene til Dan.
\par 31 Tillige indrettede han Offerhuse på Højene og indsatte til Præster alle Slags Folk, der ikke hørte til Leviterne.
\par 32 Og Jeroboam lod fejre en Fest på den femtende Dagi den ottende Måned i Lighed med den Fest, man havde i Juda; og han steg op på Alteret, han havde ladet lave i Betel, for at ofre til de Tyrekalve han havde ladet lave; og han lod de Præster, han havde indsat på Højene, gøre Tjeneste i Betel.
\par 33 Han steg op på Alteret, han havde ladet lave i Betel, på den femtende Dag i den ottende Måned, den Måned, han egenmægtig hade udtænkt, og lod Israeliteme fejre en Fest; han steg op på Alteret for at tænde Offerild.

\chapter{13}

\par 1 Og se, på HERRENs Bud kom en Guds Mand fra Juda til Betel, netop som Jeroboam stod på Alteret for at tænde Offerild.
\par 2 Og han råbte med HERRENs Ord imod Alteret: "Alter, Alter! Så siger HERREN: Der skal fødes Davids Hus en Søn ved Navn Josias, og på dig skal han ofre Højenes Præster, som tænder Offerild på dig, og han skal brænde Menneskeknogler på dig!"
\par 3 Og samtidig kundgjorde han et Tegn, idet han sagde: "Dette er Tegnet på, at HERREN har talet: Se, Alteret skal revne, så Asken derpå vælter ud!"
\par 4 Da nu Kongen hørte de Ord, den Guds Mand råbte mod Alteret i Betel, rakte Jeroboam sin Hånd ud fra Alteret og sagde: "Grib ham!" Men Hånden, han rakte ud imod ham, visnede, og han kunde ikke tage den til sig igen;
\par 5 og Alteret revnede, så Asken væltede ud fra Alteret - det Tegn, den Guds Mand havde kundgjort med HERRENs Ord.
\par 6 Da tog Kongen til Orde og sagde til den Guds Mand: "Bed dog HERREN din Gud om Nåde og gå i Forbøn for mig, at jeg kan tage Hånden til mig igen!" Og den Guds Mand bad HERREN om Nåde, og Kongen kunde tage Hånden til sig igen, og den var som før.
\par 7 Derpå sagde Kongen til den Guds Mand: "Følg med mig hjem og vederkvæg dig, så vil jeg give dig en Gave!"
\par 8 Men den Guds Mand svarede Kongen: "Om du så giver mig Halvdelen af dit Hus, vil jeg ikke følge med dig, og jeg vil hverken spise eller drikke på dette Sted;
\par 9 thi det Bud har jeg fået med HERRENs Ord: Du må hverken spise eller drikke, og du må ikke vende hjem ad den Vej, du kom!"
\par 10 Derpå drog han bort ad en anden Vej og vendte ikke hjem ad den Vej, han var kommet til Betel.
\par 11 Nu boede der i Betel en gammel Profet; hans Sønner kom og fortalte ham om alt, hvad den Guds Mand den Dag havde gjort i Betel, og om de Ord, han havde talt til Kongen. Men da de havde fortalt deres Fader det,
\par 12 spurgte han dem: "Hvilken Vej gik han?" Og hans Sønner viste ham, hvilken Vej den Guds Mand, der var kommet fra Juda, var gået.
\par 13 Da sagde han til sine Sønner: "Læg Sadelen på mit Æsel!" Og da de havde sadlet Æselet, satte han sig op,
\par 14 red efter den Guds Mand og traf ham siddende under Egetræet. Han spurgte ham da: "Er du den Guds Mand, der kom fra Juda?" Han svarede: "Ja!"
\par 15 Så sagde han til ham: "Kom med mig hjem og få noget at spise!"
\par 16 Men han svarede: "Jeg kan ikke vende om og følge med dig, og jeg kan hverken spise eller drikke sammen med dig på dette Sted,
\par 17 thi der er sagt mig med HERRENs Ord: Du må hverken spise eller drikke der, og du må ikke vende tilbage ad den Vej, du kom!"
\par 18 Da sagde han til ham: "Også jeg er Profet som du, og en Engel har med HERRENs Ord sagt til mig: Tag ham med dig hjem, for at han kan få noget at spise og drikke!" Men han løj for ham.
\par 19 Så vendte han tilbage med ham og spiste og drak i hans Hus.
\par 20 Men medens de sad til Bords, kom HERRENs Ord til Profeten, der havde fået ham tilbage,
\par 21 og han råbte til den Guds Mand, der var kommet fra Juda: "Så siger HERREN: Fordi du har været genstridig mod HERRENs Ord og ikke holdt det Bud, HERREN din Gud pålagde dig,
\par 22 men vendte tilbage og spiste og drak på det Sted, hvor han sagde, du ikke måtte spise og drikke, derfor skal dit Lig ikke komme i dine Fædres Grav!"
\par 23 Efter at han havde spist og drukket, sadlede han Æselet til ham, og han gav sig på Hjemvjen.
\par 24 Men en Løve kom imod ham på Vejen og dræbte ham. Og hans Lig lå henslængt på Vejen, og Æselet stod ved Siden af; også Løven stod ved Siden af Liget.
\par 25 Og se, nogle Mænd kom der forbi og så Liget ligge henslængt på Vejen, og Løven stå ved Siden af, og de kom og fortalte det i Byen, hvor den gamle Profet boede;
\par 26 og da Profeten, der havde fået ham til at vende om, hørte det, sagde han: "Det er den Guds Mand, som var genstridig mod HERRENs Ord; derfor har HERREN givet ham i Løvens Vold, og den har sønderrevet ham og dræbt ham efter det Ord, HERREN talede til ham!"
\par 27 Derpå sagde han til sine Sønner: "Læg Sadelen på mit Æsel!" Og da de havde gjort det,
\par 28 red han hen og fandt hans Lig liggende henslængt på vejen og Æselet og Løven stående ved Siden af, uden at Løven havde ædt Liget eller sønderrevet Æselet.
\par 29 Da løftede Profeten den Guds Mands Lig op, lagde ham på Æselet og førte ham tilbage til Byen for at holde Dødeklage og jorde ham;
\par 30 og da han havde lagt Liget i sin egen Grav, holdt de Dødeklage over ham og sagde: "Ak ve min Broder!"
\par 31 Og efter at have jordet ham sagde han til sine Sønner: "Når jeg dør, skal I lægge mig i samme Grav, som den Guds Mand ligger i; ved Siden af hans Ben skal I lægge mig, for at mine Ben kan blive skånet sammen md hans;
\par 32 thi det Ord skal gå i Opfyldelse, som han med HERRENs Ord råbte mod Alteret i Betel og alle Offerhusene på Højene i Samarias Byer!"
\par 33 Heller ikke efter denne Begivenhed opgav Jeroboam sin onde Færd, men gjorde på ny alle Slags Folk til Præster på Højene, idet han indsatte enhver, der havde Lyst, til Præst på Højene.
\par 34 Og det blev Jeroboams Hus til Synd og førte til, at det blev tilintetgjort og udryddet af Jorden.

\chapter{14}

\par 1 Ved den Tid blev Jeroboams Søn Abija syg.
\par 2 Da sagde Jeroboam til sin Hustru: "Tag og forklæd dig, så man ikke kan kende, at du er Jeroboams Hustru, og begiv dig til Silo, thi der bor Profeten Ahija, som kundgjorde mig, at jeg skulde blive Konge over dette Folk;
\par 3 tag ti Brød, noget Bagværk og en Krukke Honning med og henvend dig til ham, så vil han sige dig, hvorledes det skal gå Drengen!"
\par 4 Jeroboams Hustru gjorde nu således; hun begav sig til Silo og gik ind i Ahijas Hus. Ahija kunde ikke se, da hans Øjne var sløve af Alderdom;
\par 5 men HERREN havde sagt til Ahija: "Se, Jeroboams Hustru kommer til dig for at høre sig for hos dig angående sin Søn, da han er syg; det og det skal du svare hende; men når hun kommer, er hun forklædt."
\par 6 Da nu Ahija hørte Lyden af hendes Trin, som hun gik ind ad Døren, sagde han: "Kom kun ind, Jeroboams Hustru! Hvorfor er du forklædt? Mig er det pålagt at bringe dig en tung Tidende.
\par 7 Gå hen og sig til Jeroboam: Så siger HERREN, Israels Gud: Jeg ophøjede dig af Folkets Midte og gjorde dig til Fyrste over mit Folk Israel
\par 8 og rev Riget fra Davids Hus og gav dig det; dog har du ikke været som min Tjener David, der holdt mine Bud og fulgte mig af hele sit Hjerte og kun gjorde, hvad der er ret i mine Øjne,
\par 9 men du har handlet værre end alle dine Forgængere; du gik hen og krænkede mig og gjorde dig andre Guder og støbte Billeder, men mig kastede du bag din Ryg;
\par 10 se, derfor vil jeg bringe Ulykke over Jeroboams Hus og udrydde hvert mandligt Væsen, hver og en af Jeroboams Slægt i Israel, og jeg vil feje Jeroboams Hus bort, som man fejer Skarn bort, til der ikke er Spor tilbage!
\par 11 Den af Jeroboams Slægt, som dør i Byen, skal Hundene æde, og den, som dør på Marken, skal Himmelens Fugle æde, thi det er HERREN, der har talet!
\par 12 Men gå nu hjem! Når din Fod betræder Byen, skal Barnet dø;
\par 13 og hele Israel skal holde Dødeklage over ham og jorde ham, thi han er den eneste af Jeroboams Slægt, der skal komme i en Grav; thi hos ham fandtes dog noget, der vandt HERREN Israels Guds Behag inden for Jeroboams Slægt.
\par 14 Men HERREN vil oprejse sig en Konge over Israel, der skal udrydde Jeroboams Hus på den Dag.
\par 15 Men også siden vil HERREN slå Israel, så at de svajer hid og did som Sivet i Vandet, og rykke Israel op fra dette herlige Land, som han gav deres Fædre, og sprede dem hinsides Floden, fordi de har lavet sig Asjerastøtter og krænket HERREN;
\par 16 og han vil give Israel til Pris for de Synders Skyld, Jeroboam har begået og forledt Israel til."
\par 17 Da gav Jeroboams Hustru sig på Vej og kom til Tirza; og da hun betrådte Husets Tærskel, døde Drengen;
\par 18 og man jordede ham, og hele Israel holdt Dødeklage over ham efter det Ord, HERREN havde talet ved sin Tjener, Profeten Ahija.
\par 19 Hvad der ellers er at fortælle om Jeroboam, hvorledes han førte Krig, og hvorledes han herskede står jo optegnet i Israels kongers Krønike.
\par 20 Jeroboams Regeringstid udgjorde to og tyve År. Så lagde han sig til Hvile hos sine Fædre, og hans Søn Nadab blev Konge i hans Sted.
\par 21 Rehabeam, Salomos Søn, blev Kongei Juda. Rehabeam var een og fyrretyve År gammel, da han blev Konge, og han herskede sytten År i Jerusalem, den By, HERREN havde udvalgt af alle Israels Stammer for der at stedfæste sit Navn.
\par 22 Og Juda gjorde, hvad der var ondt i HERRENs Øjne, og med de Synder, de begik, vakte de hans Nidkærhed, mere end deres Fædre havde gjort.
\par 23 Også de byggede sig Offerhøje, Stenstøtter og Asjerastøtter på alle høje Steder og under alle grønne Træer;
\par 24 ja, der var endog Mandsskøger i Landet. De øvede alle de Vederstyggeligheder, som var begået af de Folk, HERREN havde drevet bort foran Israeliterne.
\par 25 Men i Kong Rebabeams femte Regeringsår drog Ægypterkongen Sjisjak op imod Jerusalem
\par 26 og tog Skattene i HERRENs Hus og i Kongens Palads; alt tog han, også de Guldskjolde, Salomo havde ladet lave.
\par 27 Kong Rehabeam lod da i Stedet lave Kobberskjolde og gav dem i Forvaring hos Høvedsmændene for Livvagten, der holdt Vagt ved Indgangen til Kongens Palads;
\par 28 og hver Gang Kongen begav sig til HERRENs Hus, bentede Livvagten dem, og bagefter bragte de dem tilbage til Vagtstuen.
\par 29 Hvad der ellers er at fortælle om Rehabeam, alt, hvad han gjorde, står jo optegnet i Judas Kongers Krønike.
\par 30 Rehabeam og Jeroboam lå i Krig med hinanden hele Tiden.
\par 31 Så lagde Rehabeam sig til Hvile hos sine Fædre og blev jordet hos sine Fædre i Davidsbyen. Hans Moder var en ammonitisk Kvinde ved Navn Na'ama. Og hans Søn Abija blev Konge i hans Sted.

\chapter{15}

\par 1 I Kong Jeroboams, Nebats Søns, attende Regeringsår blev Abija Konge over Juda.
\par 2 Tre År herskede han i Jerusalem. Hans Moder hed Ma'aka og var en datter af Absalom.
\par 3 Han vandrede i alle de Synder, hans Fader havde begået før ham, og hans Hjerte var ikke helt med HERREN hans Gud som hans Fader Davids.
\par 4 Men for Davids Skyld lod HERREN hans Gud ham få en Lampe i Jerusalem, idet han ophøjede hans Sønner efter ham og lod Jerusalem bestå,
\par 5 fordi David havde gjort, hvad der var ret i HERRENs Øjne, og ikke, så længe han levede, var veget fra noget af, hvad han havde pålagt ham, undtagen over for Hetiten Urias.
\par 6 (Rehabeam lå i Krig med Jeroboam, så længe han levede).
\par 7 Hvad der ellers er at fortælle om Abija, alt, hvad han gjorde, står jo optegnet i Judas Kongers Krønike. Abija og Jeroboam lå i Krig med hinanden.
\par 8 Så lagde Abija sig til Hvile hos sine Fædre, og man jordede ham i Davidsbyen; og hans Søn Asa blev Konge i hans Sted.
\par 9 I Kong Jeroboam af Israels tyvende Regeringsår blev Asa Konge over Juda,
\par 10 og han herskede een og fyrretyve År i Jerusalem. Hans Moder hed Ma'aka og var en Datter af Absalom.
\par 11 Asa gjorde, hvad der var ret i HERRENs Øjne, ligesom hans Fader David;
\par 12 han jog Mandsskøgerne ud af Landet og fjernede alle Afgudsbillederne, som hans Fædre havde ladet lave.
\par 13 Han fratog endog sin Moder Ma'aka Værdigheden som Herskerinde, fordi hun havde ladet lave et Skændselsbillede til Ære for Asjera; Asa lod hendes Skændselsbillede nedbryde og brænde i Kedrons Dal.
\par 14 Vel forsvandt Offerhøjene ikke, men alligevel var Asas Hjerte helt med HERREN, så længe han levede.
\par 15 Og han bragte sin Faders og sine egne Helliggaver ind i HERRENs Hus, Sølv, Guld og forskellige Kar.
\par 16 Asa og Kong Basja af Israel lå i Krig med hinanden, så længe de levede.
\par 17 Kong Basja af Israel drog op imod Juda og befæstede Rama for at hindre, at nogen af Kong Asa af Judas Folk drog ud og ind.
\par 18 Da tog Asa alt det Sølv og Guld, der var tilbage i Skatkamrene i HERRENs Hus og i Kongens Palads, overgav det til sine Folk og sendte dem til Kong Benhadad af Aram, en Søn af Hezjons Søn Tabrimmon, som boede i Damaskus, idet han lod sige:
\par 19 "Der består en Pagt mellem mig og dig, mellem min Fader og din Fader; her sender jeg dig en Gave af Sølv og Guld; bryd derfor din Pagt med Kong Basja af Israel, så at han nødes til at drage bort fra mig!"
\par 20 Benhadad gik ind på Kong Asas Forslag og sendte sine Hærførere mod Israels Byer og indtog Ijjon, Dan, Abel-Bet-Ma'aka og hele Kinnerot tillige med hele Naftalis Land.
\par 21 Da Basja hørte det, opgav han at befæste Rama og vendte tilbage til Tirza.
\par 22 Men Kong Asa stævnede hver eneste Mand i hele Juda sammen, og de førte Stenene og Træværket bort, som Basja havde brugt ved Befæstningen af Rama; dermed befæstede Kong Asa så Geba i Benjamin og Mizpa.
\par 23 Hvad der ellers er at fortælle om Asa, alle hans Heltegerninger, alt, hvad han gjorde, og de Byer, han befæstede, står jo optegnet i Judas Kongers Krønike. I øvrig led han i sin Alderdom af en Sygdom i Fødderne.
\par 24 Så lagde han sig til Hvile hos sine Fædre og blev jordet hos sine Fædre i sin Fader Davids By; og hans Søn Josafat blev Konge i hans Sted.
\par 25 Nadab, Jeroboams Søn, blev Konge over Israel i Kong Asa af Judas andet Regeringsår, og han herskede to År over Israel.
\par 26 Han gjorde, hvad der var ondt i HERRENs Øjne, og vandrede i sin Faders Spor og i de Synder, han havde forledt Israel til.
\par 27 Da stiftede Basja, Ahijas Søn af Issakars Hus, en Sammensværgelse imod ham, og Basja huggede ham ned ved Gibbeton, der tilhørte Filisterne, medens Nadab og hele Israel belejrede Byen.
\par 28 Basja dræbte ham i Kong Asa af Judas tredje Regeringsår og blev Konge i hans Sted;
\par 29 nu da han var blevet Konge, lod han hele Jeroboams Hus nedhugge, idet han ikke skånede en eneste Sjæl af Jeroboams Slægt, men udryddede dem efter det Ord, HERREN havde talet ved sin Tjener Ahija fra Silo,
\par 30 for de Synders Skyld, Jeroboam havde begået og forledt Israel til, for den Krænkelse, han havde tilføjet HERREN, Israels Gud.
\par 31 Hvad der ellers er at fortælle om Nadab, alt, hvad han gjorde, står jo optegnet i Israels Kongers Krønike
\par 32 ( Asa og kong Ba'sja af Israel lå i Krig med hinanden, så længe de levede.)
\par 33 I Kong Asa af Judas tredje Regeringsår blev Basja, Ahijas Søn, Konge over hele Israel, og han herskede tre og tyve År i Tirza.
\par 34 Han gjorde, hvad der var ondt i HERRENs Øjne, og vandrede i Jeroboams Spor og de Synder, han havde forledt Israel til.

\chapter{16}

\par 1 Men til Jehu, Hananis Søn, kom HERRENs ord mod Ba'sja således:
\par 2 "Jeg ophøjede dig af Støvet og gjorde dig til Fyrste over mit Folk Israel, dog har du vandret i Jeroboams Spor og forledt mit Folk Israel til Synd, så de krænker mig ved deres Synder;
\par 3 se, derfor vil jeg nu feje Basja og hans Hus bort og gøre det samme ved dit Hus, som jeg gjorde ved Jeroboams, Nebats Søns, Hus;
\par 4 den af Basjas Slægt, som dør i Byen, skal Hundene æde, og den, som dør på Marken, skal Himmelens Fugle æde!"
\par 5 Hvad der ellers er at fortælle om Ba'sja, hvad han gjorde, og hans Heltegerninger, står jo optegnet i Israels Kongers Krønike.
\par 6 Så lagde Ba'sja sig til Hvile hos sine Fædre og blev jordet i Tirza; og hans Søn Ela blev Konge i hans Sted.
\par 7 Desuden kom HERRENs Ord ved Profeten Jehu, Hananis Søn, mod Ba'sja og hans Hus både på Grund af alt det, han havde gjort, som var ondt i HERRENs Øjne, idet han krænkede ham ved sine Hænders Værk og efterlignede Jeroboams Hus, og tillige fordi han lod dette nedhugge.
\par 8 I Kong Asa af Judas seks og tyvende Regeringsår blev Ela, Ba'sjas Søn, Konge over Israel, og han herskede to År i Tirza.
\par 9 Så stiftede en af hans Mænd, Zimri, der var Fører for den ene Halvdel af Stridsvognene, en Sammensværgelse imod ham; og engang, da han i Tirza var beruset ved et Drikkelag i sin Paladsøverste Arzas Hus,
\par 10 trængte Zimri ind og slog ham ihjel - i Kong Asa af Judas syv og tyvende Regeringsår - og blev Konge i hans Sted.
\par 11 Da han var blevet Konge og havde besteget Tronen, lod han hele Basjas Hus dræbe uden at levne et mandligt Væsen og tillige hans nærmeste Slægtninge og Venner;
\par 12 således udryddede Zimri hele Ba'sjas Hus efter det Ord, HERREN havde talet til Basja ved Profeten Jehu,
\par 13 for alle de Synders Skyld, som Ba'sja og hans Søn Ela havde begået og forledt Israel til, så at de krænkede HERREN, Israels Gud, ved deres Afguder.
\par 14 Hvad der ellers er af fortælle om Ela, alt, hvad han gjorde, står jo optegnet i Israels Kongers Krønike.
\par 15 I Kong Asa af Judas syv og tyvende Regeringsår blev Zimri Konge, og han herskede syv Dage i Tirza. Hæren var på det Tidspunkt ved at belejre Gibbeton, som tilhørte Filisterne;
\par 16 og da nu Hæren under Belejringen hørte, at Zimri havde sfiftet en Sammensværgelse mod Kongen og endda dræbt ham, udråbte hele Israel samme Dag i Lejren Omri, Israels Hærfører, til Konge.
\par 17 Derpå hrød Omri op med hele Israel fra Gibbeton og begyndte at belejre Tirza;
\par 18 og da Zimri så at Byen var taget, begav han sig ind i Kongens Palads og stak det i Brand over sig; således døde han
\par 19 for de Synders Skyld, han havde begået, idet han gjorde, hvad der var ondt i HERRENs Øjne, og vandrede i Jeroboams Spor og i de Synder, han havde begået, da han forledte Israel til at synde.
\par 20 Hvad der ellers er at fortælle om Zimri og den Sammensværgelse, han stiftede, står jo optegnet i Israels Kongers Krønike.
\par 21 Ved den Tid delte Israels Folk sig, idet den ene Halvdel slutfede sig fil Tibni, Ginats Søn, og udråbte ham til Konge, medens den anden sluttede sig til Omri.
\par 22 Men den Del af Folket, der sluttede sig til Omri, fik Overtaget over dem, der sluttede sig til Tibni, Ginats Søn, og da Tibni døde ved den Tid, blev Omri Konge.
\par 23 I Kong Asa af Judas een og tredivte Regeringsår blev Omri Konge over Israel, og han herskede tolv År. Først herskede han seks År i Tirza;
\par 24 men siden købte han Samarias Bjerg af Semer for to Talenter Sølv og byggede på Bjerget en By, som han efter Semer, Bjergets Ejer, kaldte Samaria.
\par 25 Omri gjorde, hvad der var ondt i HERRENs Øjne, og handlede endnu værre end alle hans Forgængere;
\par 26 han vandrede helt i Jeroboams, Nebats Søns, Spor og i de Synder, han havde forledt Israel til, så at de krænkede HERREN, Israels Gud, ved deres Afguder.
\par 27 Hvad der ellers er at fortælle om Omri, alt, hvad han gjorde, og de Heltegerninger, han udførte, står jo optegnet i Israels Kongers Krønike.
\par 28 Så lagde Omri sig til Hvile hos sine Fædre og blev jordet i Samaria; og hans Søn Akab blev Konge i hans Sted.
\par 29 Akab, Omris Søn, blev Konge over Israel i Kong Asa af Judas otte og tredivte Regeringsår, og Akab, Omris Søn, herskede to og tyve År over Israel i Samaria.
\par 30 Akab, Omris Søn, gjorde, hvad der var ondt i HERRENs Øjne, i højere Grad end alle hans Forgængere.
\par 31 Og som om det ikke var nok med, at han vandrede i Jeroboams, Nebats Søns, Synder, ægtede han oven i Købet Jesabel, en Datter af Zidoniemes Konge Etba'al, og gik hen og dyrkede Ba'al og tilbad ham.
\par 32 Han rejste Ba'al et Alter i Ba'alstemplet, som han lod bygge i Samaria.
\par 33 Og Akab lavede Asjerastøtten og gjorde endnu flere Ting, hvorved han krænkede HERREN, Israels Gud, værre end de Konger, der havde hersket før ham i Israel. -
\par 34 I hans Dage genopbyggede Beteliten Hiel Jeriko; efter det Ord, HERREN havde taleted Josua, Nuns Søn, kostede det ham hans førstefødte Abiram at lægge Grunden og hans yngste Søn Seguh at sætte dens Portfløje ind.

\chapter{17}

\par 1 Tisjbiten Elias fra Tisjbe i Gilead sagde til Akab: "Så sandt HERREN, Israels Gud, lever, han, for hvis Åsyn jeg står, i de kommende År skal der ikke falde dug eller Regn uden på mit udtrykkelige Bud!"
\par 2 Derpå kom HERRENs Ord til ham således:
\par 3 "Gå bort herfra og begiv dig østerpå og hold dig skjult ved Bækken Krit østen for Jordan;
\par 4 du skal drikke af Bækken, og Ravnene har jeg pålagt at sørge for Føde til dig der."
\par 5 Da gik han og gjorde efter HERRENs Ord, han gik hen og tog Bolig ved Bækken Krit østen for Jordan;
\par 6 og Ravnene bragte ham Brød om Morgenen og kød om Aftenen, og han drak af Bækken.
\par 7 Men nogen Tid efter tørrede Bækken ud, eftersom der ingen Regn faldt i Landet.
\par 8 Da kom HERRENs Ord til ham således:
\par 9 "Begiv dig til Zarepta, som hører til Zidon, og tag Bolig der; se, jeg har pålagt en Enke der at sørge for Føde til dig."
\par 10 Så begav han sig til Zarepta, og da han kom til Byens Port, fik han Øje på en Enke, som var ved at sanke Brænde, og råbte til hende: "Hent mig lidt Vand i et Kar, for at jeg kan drikke!"
\par 11 Og da hun gik bort for at hente det, råbte han efter hende: "Tag også et Stykke Brød med til mig!"
\par 12 Men hun svarede: "Så sandt HERREN din Gud lever, jeg ejer ikke Brød, men kun en Håndfuld Mel i Krukken og lidt Olie i Dunken; jeg var nettop ved at sanke et Par Stykker Brænde for at gå hjem og tillave det til mig og min Søn; og når vi har spist det, må vi dø!"
\par 13 Da sagde Elias til hende: "Frygt ikke! Gå hjem og gør, som du siger; men lav først et lille Brød deraf til mig og bring mig det; siden kan du lave noget til dig selv og din Søn!
\par 14 Thi så siger HERREN, Israels Gud: Melkrukken skal ikke blive tom, og Olien i Dunken skal ikke slippe op, før den Dag HERREN sender Regn over Jorden!"
\par 15 Da gik hun og gjorde, som Elias sagde; og både hun og han og hendes Søn havde noget at spise en Tid lang.
\par 16 Melkrukken blev ikke tom, og olien i Dunken slap ikke op, efter det Ord HERREN havde talet ved Elias.
\par 17 Men nogen Tid efter blev Kvindens, Husets Ejerindes, Søn syg, og hans Sygdom tog heftigt til, så der til sidst ikke mere var Liv i ham.
\par 18 Da sagde hun til Elias: "Hvad har jeg med dig at gøre, du Guds Mand! Er du kommet for at bringe min Synd i Erindring og volde min Søns Død?"
\par 19 Men han svarede hende: "Lad mig få din Søn!" Og han tog ham fra hendes Skød og bar ham op i Stuen på Taget, hvor han boede, og lagde ham på sin Seng.
\par 20 Så råbte han til HERREN: "HERRE min Gud, vil du virkelig handle så ilde mod den Enke; i hvis Hus jeg er Gæst, at du lader hendes Søn dø?"
\par 21 Derpå strakfe han sig tre Gange hen over Drengen og råbte til HERREN: "HERRE min Gud, lad dog Drengens Sjæl vende tilbage!"
\par 22 Og HERREN hørte Eliass Røst; Drengens Sjæl vendte tilbage, så han blev levende.
\par 23 Så tog Elias Drengen og bragte ham fra Stuen på Taget ned i Huset og gav hans Moder ham, idet han sagde: "Se, din Søn lever!"
\par 24 Da sagde Kvinden til Elias: "Nu ved jeg vist, at du er en Guds Mand, og at HERRENs Ord i din Mund er Sandhed."

\chapter{18}

\par 1 Lang Tid efter, i det tredie År, kom HERRENs ord således: "Gå hen og træd frem for Akab, så vil jeg sende Regn over Jorden!"
\par 2 Da gav Elias sig på Vej for at træde frem for Akab. Da Hungersnøden blev trykkende i Samaria,
\par 3 kaldte Akab Paladsøversten Obadja til sig. Obadja var en Mand, der alvorligt frygtede HERREN,
\par 4 og dengang Jesabel lod HERRENs Profeter udrydde, tog han og skjulte hundrede Profeter, halvtredsindstyve i een Hule og halvtredsindstyve i en anden, og sørgede for Brød og Vand til dem.
\par 5 Akab sagde nu til Obadja: "Kom, lad os drage rundt i Landet til alle Vandkilder og Bække, om vi mulig kan finde så meget Græs, at vi kan holde Liv i Hestene og Muldyrene og slippe for at dræbe noget af Dyrene!"
\par 6 Så delte de Landet, som de skulde gennemvandre, mellem sig, således at Akab og Obadja drog hver sin Vej.
\par 7 Medens nu Obadja var undervejs, se, da trådte Elias ham i Møde; Obadja genkendte ham og faldt på sit Ansigt og sagde: "Er det dig, min Herre Elias?"
\par 8 Han svarede: "Ja, det er mig! Gå hen og sig til din Herre, at Elias er her!"
\par 9 Men han sagde: "Hvormed har jeg dog syndet, siden du vil give din Træl i Akabs Hånd, for at han kan slå mig ihjel?
\par 10 Så sandt HERREN din Gud lever, der er ikke et Folk eller Rige, min Herre ikke har sendt Bud til for at lede efter dig; og blev der sagt, at du ikke var der, tog han Riget og Folket i Ed på, at de ikke havde fundet dig.
\par 11 Og nu siger du, at jeg skal gå hen og sige til min Herre, at Elias er her!
\par 12 Hvis nu HERRENs Ånd, når jeg har forladt dig, fører dig bort til et Sted, jeg ikke kender, og jeg kommer og melder det til Akab, og han ikke finder dig, lader han mig dræbe. Og din Træl har dog frygtet HERREN fra Ungdommen af!
\par 13 Er det ikke kommet min Herre for Øre, hvad jeg gjorde, da Jesabel lod HERRENs Profeter dræbe, hvorledes jeg skjulte hundrede af HERRENs Profeter, halvtredsindstyve i een Hule og halvtredsindstyve i en anden, og sørgede for Brød og Vand til dem?
\par 14 Og nu siger du, at jeg skal gå hen og sige til din Herre, at Elias er her - han lader mig dræbe!"
\par 15 Da sagde Elias: "Så sandt Hærskarers HERRE lever, han, for hvis Åsyn jeg står, i Dag vil jeg træde frem for ham."
\par 16 Obadja gik da Akab i Møde og meldte ham det, og Akab gik Elias i Møde.
\par 17 Da Akab fik Øje på Elias, sagde han til ham: "Er det dig, du, som bringer Ulykke over Israel!"
\par 18 Men han svarede: "Det er ikke mig, der har bragt Ulykke over Israel, men dig og din Faders Hus, fordi I har forladt HERREN og holder eder til Ba'alerne!
\par 19 Men send nu Bud og kald hele Israel sammen til mig på Karmels Bjerg og tillige de 450 Ba'alsprofeter og de 400 Asjeraprofeter, som spiser ved Jesabels Bord!"
\par 20 Da sendte Akab Bud rundt til alle Israeliterne og samlede Profeterne på Karmels Bjerg.
\par 21 Elias trådte så frem for alt Folket og sagde: "Hvor længe vil I blive ved at halte til begge Sider? Er HERREN Gud, så hold eder til ham, og er Ba'al Gud, så hold eder til ham!" Men Folket svarede ham ikke et Ord.
\par 22 Da sagde Elias til Folket: "Jeg er den eneste af HERRENs Profeter, der er tilbage, og Ba'als Profeter er 450 Mand;
\par 23 lad os nu få to unge Tyre; så skal de vælge den ene Tyr og hugge den i Stykker og lægge den på Brændet, men Ild må de ikke lægge til; den anden vil jeg lave til og lægge på Brændet, men uden at tænde Ild.
\par 24 Så skal I påkalde eders Guds Navn, og jeg vil påkalde HERRENs Navn; den Gud, der svarer med Ild, han er Gud!" Alt Folket sagde: "Det Forslag er godt!"
\par 25 Derpå sagde Elias til Ba'als Profefer: "Vælg eder den ene Tyr og lav den til først, thi I er de mange, og påkald så eders Guds Navn, men I må ikke tænde Ild!"
\par 26 Så tog de Tyren og lavede den til og påkaldte Ba'als Navn fra Morgen til Middag, idet de råbte: "Hør os, Ba'al!" Men ikke en Lyd hørtes, der var ingen, som svarede; og de dansede haltende omkring det Alter, de havde opført.
\par 27 Men da det var blevet Middag, hånede Elias dem og sagde: "I må råbe højt, thi han er jo en Gud! Han er vel faldet i Tanker eller gået afsides eller rejst bort, eller han er faldet i Søvn og må først vågne!"
\par 28 Da råbte de højt, og som de havde for Skik, sårede de deres Legemer med Sværd og Spyd, til Blodet flød ned ad dem.
\par 29 Og da det var over Middag, begyndte de at rase, og det varede lige til hen imod Afgrødeofferets Tid, men ikke en Lyd hørtes, ingen svarede, og ingen agtede derpå.
\par 30 Da sagde Elias til alt Folket: "Kom hen til mig!" Og da alt Folket var kommet hen til ham, satte han HERRENs nedbrudte Alter i Stand.
\par 31 Elias tog tolv Sten, svarende til Tallet på Jakobs Sønners Stammer, han, til hvem HERRENs Ord lød: "Israel skal dit Navn!"
\par 32 Og af disse Sten byggede han et Alter i HERRENs Navn og gravede rundt om Alteret en Rende på omtrent to Sea Land.
\par 33 Derpå lagde han Brændet tilrette, huggede Tyren i Stykker og lagde den på Brændet.
\par 34 Så sagde han: "Fyld fire Krukker med Vand og hæld det ud over Brændofferet og Brændet!" Og da de havde gjort det, sagde han: "Een Gang til!" Og da de havde gjort det anden Gang, sagde han: "Een Gang til!" Og de gjorde det endnu en Gang.
\par 35 Det drev af Vand rundt om Alteret, også Renden fik han fyldt med Vand.
\par 36 Men ved Afgrødeofferets Tid trådte Profeten Elias frem og sagde: "HERRE, Abrahams, Isaks og Israels Gud! Lad det kendes i Dag, at dut er Gud i Israel og jeg din Tjener, og at jeg har gjort alt dette på dit Ord!
\par 37 Hør mig, HERRE, hør mig, for at dette Folk må kende, at du HERRE er Gud, og at du atter drager deres Hjerte til dig"
\par 38 Da for HERRENs Ild ned og fortærede Brændofferet og Brændet og Stenene og Jorden; endog Vandet i Renden slikkede den bort.
\par 39 Da alt Folket så det, faldt de på deres Ansigt og råbte: "HERREN er Gud, HERREN er Gud!"
\par 40 Men Elias sagde til dem: "Grib Ba'als Profeter, lad ingen af dem slippe bort!" Og de greb dem, og Elias førte dem ned til Kisjonbækken og dræbte dem der.
\par 41 Derpå sagde Elias til Akab: "Gå op og spis og drik, thi der høres Susen af Regn."
\par 42 Da gik Akab op for at spise og drikke; men Elias gik op på Karmels Top og bøjede sig til Jorden med Ansigtet mellem Knæene.
\par 43 Så sagde han til sin Tjener: "Gå op og se ud over Havet!" Og han gik op og så ud, men sagde: "Der er intet!" Syv Gange sagde han til ham: "Gå derop igen!" Og syv Gange vendte Tjeneren tilbage.
\par 44 Men syvende Gang sagde han: "Nu stiger der en lille Sky op af Havet, så stor som en Mands Hånd!" Da sagde Elias: "Gå hen og sig til Akab: Spænd for og kør hjem, at du ikke skal blive opholdt af Regnen!"
\par 45 Et Øjeblik efter var Himmelen sort af Stormskyer, og der faldt en voldsom Regn. Akab steg til Vogns og kørte til Jizre'el;
\par 46 men HERRENs Hånd kom over Elias, så han omgjordede sine Lænder og løb foran Akab lige til Jizre'el.

\chapter{19}

\par 1 Akab fortalte nu Jesabel alt, hvad Elias havde gjort, og hvorledes han havde ihjelslået alle Profeterne med Sværd,
\par 2 og Jesabel sendte et Sendebud til Elias og lod sige: "Guderne ramme mig både med det ene og det andet, om jeg ikke i Morgen ved denne Tid handler med dit Liv, som der er handlet med deres!"
\par 3 Da blev han bange, stod op og drog bort for at redde sit Liv. Han kom da til Be'ersjeba i Juda. Der lod han sin Tjener blive
\par 4 og vandrede selv en Dagsrejse ud i Ørkenen og satte sig under en Gyvelbusk og ønskede sig Døden, idet han sagde: "Nu er det nok, HERRE; tag mit Liv, thi jeg er ikke bedre end mine Fædre!"
\par 5 Så lagde han sig til at sove under en Gyvelbusk. Og se, en Engel rørte ved ham og sagde: "Stå op og spis!"
\par 6 Og da han så sig om, se, da lå der, hvor hans Hoved havde hvilet, et ristet Brød, og der stod en Krukke Vand; og han spiste og drak og lagde sig igen.
\par 7 Men HERRENs Engel kom atter og rørte ved ham og sagde: "Stå op og spis, ellers bliver Vejen dig for lang!"
\par 8 Da stod han op og spiste og drak; og styrket af dette Måltid vandrede han i fyrretyve Dage og fyrretyve Nætter lige til Guds Bjerg Horeb.
\par 9 Der gik han ind i en Hule og overnattede. Da lød HERRENs Ord til ham: "Hvad er du her efter, Elias?"
\par 10 Han svarede: "Jeg har været fuld af Nidkærhed for HERREN, Hærskarers Gud, fordi Israeliterne har forladt din Pagt; dine Altre har de nedbrudt, og dine Profeter har de ihjelslået med Sværd! Jeg alene er tilbage, og nu står de mig efter Livet!"
\par 11 Da sagde han: "Gå ud og stil dig på Bjerget for HERRENs Åsyn!" Og se, HERREN gik forbi, og et stort og stærkt Vejr, der sønderrev Bjerge og sprængte Klipper, gik foran HERREN, men HERREN var ikke i Vejret. Efter Vejret kom der et Jordskælv, men HERREN var ikke i Jordskælvet.
\par 12 Efter Jordskælvet kom der Ild, men HERREN var ikke i Ilden. Men efter Ilden kom der en stille, sagte Susen,
\par 13 og da Elias hørte den, hyllede han sit Hoved i sin Kappe og gik ud og stillede sig ved Indgangen til Hulen; og se, en Røst lød til ham: "Hvad er du her efter Elias?"
\par 14 Han svarede: "Jeg har været fuld af Nidkærhed for HERREN, Hærskarers Gud, fordi Israeliterne har forladt din Pagt; dine Altre har de nedbrudt, og dine Profeter har de ihjelslået med Sværd! Jeg alene er tilbage, og nu står de mig efter Livet!"
\par 15 Da sagde HERREN til ham: "Vend tilbage ad den Vej, du kom, og gå til Ørkenen ved Damaskus; gå så hen og salv Hazael til Konge over Aram,
\par 16 salv Jehu, Nimsjis Søn, til Konge over Israel og salv Elisa, Sjafats Søn, fra Abel Mehola til Profet i dit Sted!
\par 17 Den, der undslipper Hazaels Sværd, skal Jehu dræbe, og den, der undslipper Jehus Sværd, skal Elisa dræbe.
\par 18 Jeg vil lade syv Tusinde blive tilbage i Israel, hvert Knæ, der ikke har bøjet sig for Ba'al, og hver Mund, der ikke har kysset ham."
\par 19 Så gik han derfra; og han traf Elisa, Sjafats Søn, i Færd med at pløje; tolv Spand Okser havde han foran sig, og selv var han ved det tolvte. Da nu Elias gik forbi ham, kastede han sin Kappe over ham.
\par 20 Så forlod han Okserne og løb efter Elias og sagde: "Lad mig først kysse min Fader og min Moder, så vil jeg følge dig!" Han svarede: "Gå kun tilbage, thi hvad er det ikke, jeg har gjort ved dig!"
\par 21 Da forlod han ham og vendte tilbage; så tog han og slagtede Oksespandet, kogte Okserne ved Stavtøjet og gav Folkene dem at spise; derpå brød han op og fulgte Elias og gik ham til Hånde.

\chapter{20}

\par 1 Kong Benhadad af Aram samlede hele sin Hær, og to og tredive Konger fulgte ham med Heste og Stridsvogne; og han drog op og indesluttede Samaria og belejrede det.
\par 2 Han sendte nu Sendebud ind i Byen til Kong Akab af Israel
\par 3 og lod sige til ham: "Således siger Benhadad: Dit Sølv og Guld er mit, men dine Hustruer og børn kan du beholde!"
\par 4 Israels Konge lod svare: "Som du byder, Herre Konge! Jeg og alt, hvad mit er, tilhører dig."
\par 5 Men Sendebudene vendte tilbage og sagde: "Således siger Benha dad: Jeg sendte Bud til dig og lod sige: Dit Sølv og Guld og dine Hustruer og Børn skal du give mig!
\par 6 Så sender jeg da i Morgen ved denne Tid mine Folk til dig, og de skal gennemsøge dit Hus og dine Folks Huse og tilvende sig og tage alt, hvad de lyster!"
\par 7 Da lod Israels Konge alle Landets Ældste kalde og sagde: "Der ser I, at Manden har ondt i Sinde, thi nu sender han Bud til mig om mine Hustruer og Børn, og mit Sølv og Guld havde jeg ikke nægtet ham!"
\par 8 Alle de Ældste og alt Folket svarede ham: "Hør ham ikke; du må ikke give efter!"
\par 9 Da sagde han til Benhadads Sendebud: "Sig til min Herre Kongen: Alt, hvad du første Gang krævede af din Træl, vil jeg gøre, men dette Krav kan jeg ikke opfylde!" Med det Svar vendte Sendebudene tilbage.
\par 10 Da sendte Benhadad Bud til ham og lod sige: "Guderne ramme mig både med det ene og det andet, om Støvet i Samaria forslår til at fylde Hænderne på alle de Krigere, der følger mig!"
\par 11 Men Israels Konge lod svare: "Sig således: Den, der spænder Bæltet, skal ikke rose sig som den, der løser det!"
\par 12 Benhadad modtog Svaret, just som han sad og drak sammen med Kongerne i Løvhytterne; da sagde han til sine Folk: "Til Storm!" Og de gjorde sig rede til at storme Byen.
\par 13 Men en Profet trådte hen til Kong Akab af Israel og sagde: "Så siger HERREN: Ser du hele den vældige Menneskemængde der? Se, jeg giver den i Dag i din Hånd, og du skal kende, at jeg er HERREN!"
\par 14 Akab spurgte: "Ved hvem?" Han svarede: "Så siger HERREN: Ved Fogedernes Folk!" Derpå spurgte han: "Hvem skal åbne Kampen?" Han svarede: "Du!"
\par 15 Så mønstrede han Fogedernes Folk, og de var 232; derefter mønstrede han hele Hæren, alle Israeliterne, 7000 Mand.
\par 16 Og ved Middagstid gjorde de et Udfald, just som Benhadad og de to og tredive Konger, der fulgte ham, sad og drak i Løvhytterne.
\par 17 Først rykkede Fogedernes Folk ud. Man sendte da Bud til Benhadad og meldte: "Der rykker Mænd ud fra Samaria!"
\par 18 Da sagde han: "Hvad enten de rykker ud for at få Fred eller for at kæmpe, så grib dem levende!"
\par 19 Da Fogedernes Folk og Hæren, som fulgte efter, var rykket ud fra Byen,
\par 20 huggede de ned for Fode, så at Aramæerne tog Flugten; og Israeliterne satte efter dem. Men Kong Benhadad af Aram undslap til Hest sammen med nogle Ryttere.
\par 21 Da rykkede Israels Konge ud og gjorde Hestene og Vognene til Bytte, og han tilføjede Aramæerne et stort Nederlag.
\par 22 Men Profeten trådte hen til Israels Konge og sagde til ham: "Tag dig sammen og se vel til, hvad du vil gøre, thi næste År drager Arams Konge op imod dig igen!"
\par 23 Men Aramæerkongens Folk sagde til ham: "Deres Gud er en Bjerggud, derfor blev de os for stærke; men lad os se, om vi ikke kan blive de stærkeste, når vi angriber dem på Slettelandet!
\par 24 Således skal du gøre: Afsæt alle Kongerne, sæt Statholdere i deres Sted
\par 25 og stil lige så stor en Hær på Benene som den, du mistede, og lige så mange Heste og Vogne som før! Når vi så kæmper med dem på Slettelandet, sandelig, om vi ikke bliver de stærkeste!" Og han fulgte deres Råd og handlede derefter.
\par 26 Næste År mønstrede Benha dad Aramæeme og drog op til Atek for at kæmpe med Israel.
\par 27 Også Israeliterne blev mønstret og forsynede sig med Levnedsmidler, hvorefter de rykkede dem i Møde og lejrede sig lige over for dem som to små Gedehjorde, medens Aramæerne oversvømmede Landet.
\par 28 Da trådte en Guds Mand hen til Israels Konge og sagde: "Så siger HERREN: Fordi Aramæerne siger: HERREN er en Bjerggud og ikke en Dalgud! vil jeg give hele den vældige Menneskemængde der i din Hånd, og I skal kende, at jeg er HERREN!"
\par 29 De lå nu lejret over for hinanden i syv Dage, men Syvendedagen kom det til Kamp, og Israeliterne huggede Aramæerne ned, 100.000 Mand Fodfolk på een Dag.
\par 30 De, der blev tilovers, flygtede til Byen Afek, men Muren styrtede ned over dem, der var tilbage, 27000 Mand. Benhadad flygtede ind i Byen, hvor han løb fra Kammer til Kammer.
\par 31 Da sagde hans Folk til ham: Vi har hørt, at Kongerne over Israels Hus er nådige Konger; lad os binde Sæk om Lænderne og Reb om Hovederne og gå ud til Israels Konge, måske han da vil skåne dit Liv!"
\par 32 Så bandt de Sæk om Lændeme og Reb om Hovederne og kom til Israels Konge og sagde: "Din Træl Benhadad siger: Lad mig leve!" Han svarede: "Er han endnu i Live? Han er min Broder!"
\par 33 Det tog Mændene for et godt Varsel. og de tog ham straks på Ordet, idet de sagde: "Benhadad er din Broder!" Da sagde han: "Gå hen og hent ham!" Så gik Benhadad ud til ham, og han tog ham op i Vognen til sig.
\par 34 Benhadad sagde nu til ham: "De Byer, min Fader fratog din Fader, vil jeg give tilbage, og du må bygge dig Gader i Damaskus, lige som min Fader gjorde i Samaria! På disse Vilkår give du mig fri!" Og han sluttede Pagt med ham og lod ham gå.
\par 35 Men en af Profetsønnerne sagde med HERRENs Ord til sin Fælle: "Slå mig!" Men han vægrede sig derved.
\par 36 Da sagde han til ham: "Fordi du ikke har adlydt HERRENs Ord. skal en Løve dræbe dig, når du går bort fra mig!" Og da han gik bort fra ham, traf en Løve på ham og dræbte ham.
\par 37 Så traf Profetsønnen en anden og sagde til ham: "Slå mig!" Og den anden slog ham og sårede ham.
\par 38 Så gik Profeten hen og stillede sig på den Vej, Kongen kom, og gjorde sig ukendelig med et Bind for Øjnene.
\par 39 Da Kongen kom forbi, råbte han til ham: "Din Træl var draget med i Kampen; da kom en hen til mig med en Mand og sagde: Vogt den Mand vel! Slipper han bort, skal du svare for hans Liv med dit eget Liv eller bøde en Talent Sølv!
\par 40 Men din Træl var optaget snart her, snart der, og borte var han." Da sagde Israels Konge til ham: "Det er din Dom, du har selv fældet den!"
\par 41 Så tog han hurtig Bindet fra Øjnene, og Israels Konge genkendte ham som en af Ptofeterne.
\par 42 Og han sagde til ham: "Så siger HERREN: Fordi du gav Slip på den Mand, der var hjemfaldet til mit Band, skal du svare for hans Liv med dit eget Liv og for hans Folk med dit eget Folk!"
\par 43 Da drog Israels Konge hjem, misfornøjet og ilde til Mode, og han kom til Samaria.

\chapter{21}

\par 1 Derefter hændte følgende. Jizrae'eliten Nabot havde en Vingård i Jizre'el lige ved Kong Akab af Samarias Palads.
\par 2 Akab sagde til Nabot: "Overlad mig din Vingård, for at jeg kan få den til Køkkenhave; den ligger jo lige ved mit Palads; jeg vil give dig en bedre Vingård i Bytte eller betale dig, hvad den er værd, i rede Penge, om du foretrækker det."
\par 3 Men Nabot svarede Akab: "HERREN bevare mig fra at overlade dig mine Fædres Arvelod!"
\par 4 Så gik Akab hjem, misfornøjet og ilde til Mode over det Svar, Jizre'eliten Nabot havde givet ham: "Jeg vil ikke overlade dig mine Fædres Arvelod!" Og han lagde sig til Sengs, vendte sit Ansigt bort og spiste ikke.
\par 5 Da kom hans Hustru Jesabel ind og sagde til ham: "Hvorfor er du så misfornøjet, og hvorfor spiser du ikke?"
\par 6 Han svarede hende: "Jo, jeg sagde til Jizreeliten Nabot: Overlad mig din Vingård for rede Penge, eller mod at jeg giver dig en anden Vingård i Bytte, om du hellere vil det! Men han svarede: Jeg vil ikke overlade dig min Vingård!"
\par 7 Da sagde hans Hustru Jesabel til ham: "Er det dig, der for Tiden er Konge i Israel? Stå op, spis og vær ved godt Mod, jeg skal skaffe dig Jizre'eliten Nabots Vingård!"
\par 8 Derpå skrev hun et Brev i Akabs Navn, satte hans Segl under og sendte det til de Ældste og de fornemme i Nabots By, dem, han boede imellem.
\par 9 I Brevet havde hun skrevet: "Udråb en Fastedag og sæt Nabot øverst blandt Folket
\par 10 og lige over for ham to Niddinger, som kan vidne imod ham og sige: Du har forbandet Gud og Kongen! Og før ham så ud og sten ham til Døde!"
\par 11 Hans Bysbørn, de Ældste og de fornemme, som boede i hans By, gjorde nu, som Jesabel havde sendt Bud til dem om, således som der stod i Brevet, hun havde sendt dem;
\par 12 de udråbte en Fastedag og satte Nabot øverst blandt Folket;
\par 13 og de to Niddinger kom og satte sig lige over for ham og vidnede imod ham i Folkets Påhør og sagde: "Nabot har forbandet Gud og Kongen!" Og derpå førte de ham uden for Byen og stenede ham til Døde.
\par 14 Så sendte de Jesabel det Bud: "Nabot er stenet til Døde!"
\par 15 Og da Jesabel hørte, at Nabot var stenet til Døde, sagde hun til Akab: "Stå op og tag Jizre'eliten Nabots Vingård, som han vægrede sig ved at sælge dig, i Besiddelse, thi Nabot lever ikke mere, han er død!"
\par 16 Så snart Akab hørte, at Nabot var død, rejste han sig og gik ned til Jizre'eliten Nabots Vingård for at tage den i Besiddelse.
\par 17 Men HERRENs Ord kom til Tisjbiten Elias således:
\par 18 "Stå op, gå Akab, Israels Konge i Samaria, i Møde; han er just i Nabots Vingård, som han er gået ned at tage i Besiddelse.
\par 19 Og tal således til ham: Så siger HERREN: Har du myrdet og allerede tiltrådt Arven? Sig fremdeles til ham: Så siger HERREN: På samme Sted, Hundene slikkede Nabots Blod, skal de også slikke dit!"
\par 20 Da sagde Akab til Elias: "Har du fundet mig, min Fjende?" Og han svarede: "Ja, jeg har fundet dig! Fordi du har solgt dig selv til at gøre, hvad der er ondt i HERRENs Øjne,
\par 21 se, derfor vil jeg bringe Ulykke over dig og feje dig bort og udrydde hvert mandligt Væsen, store og små, af Akabs Slægt, i Israel;
\par 22 jeg vil handle med dit Hus som med Jeroboams, Nebats Søns, Hus og Basjas, Abijas Søns, Hus for den Krænkelse, du har øvet, og fordi du har forledt Israel til Synd.
\par 23 Men også om Jesabel har HERREN talet og sagt: Hundene skal æde Jesabel på Jizre'els Mark!
\par 24 Den af Akabs Slægt, der dør i Byen, skal Hundene æde, og den, der dør på Marken, skal Himmelens Fugle æde!"
\par 25 Aldrig har der været nogen der som Akab solgte sig selv til at gøre, hvad der er ondt i HERRENs Øjne, fordi hans Hustru Jesabel forledte ham dertil;
\par 26 han handlede såre vederstyggeligt, idet han boldt sig til Afgudsbillederne ganske som Amoriterne, dem, HERREN drev bort foran Israeliterne.
\par 27 Da Akab hørte de Ord, sønderrev han sine Klæder og bandt Sæk om sin bare Krop og fastede, og han sov i Sæk og gik sagtelig om.
\par 28 Da kom HERRENs Ord til Tisjbiten Elias således:.
\par 29 "Har du set, hvorledes Akab ydmyger sig for mig? Fordi han ydmyger sig for mig, vil jeg ikke lade Ulykken komme i hans Dage; i hans Søns Dage vil jeg lade Ulykken komme over hans Hus!"

\chapter{22}

\par 1 De holdt sig nu rolige i tre År, der var ikke Krig mellem Aram og Israel.
\par 2 Og i det tredje År drog Kong Josafat af Juda ned til Israels Konge.
\par 3 Da sagde Israels Konge til sine Folk: "I ved jo, at Ramot i Gilead hører os til, og dog rører vi os ikke for at tage det fra Arams Konge!"
\par 4 Og han sagde til Josafat: "Vil du drage med i Krig mod Ramot i Gilead?" Josafat svarede Israels Konge: "Jeg som du, mit Folk som dit, mine Heste som dine!"
\par 5 Josafat sagde fremdeles til Israels Konge: "Spørg dog først om, hvad HERREN siger!"
\par 6 Da lod Israels Konge Profeterne kalde sammen, henved 400 Mand, og spurgte dem: "Skal jeg drage i Krig mod Ramot i Gilead, eller skal jeg lade være?" De svarede: "Drag derop, så skal HERREN give det i Kongens Hånd!"
\par 7 Men Josafat spurgte: "Er her ikke endnu een af HERRENs Profe ter, vi kan spørge?
\par 8 Israels Konge svarede: "Her er endnu en Mand, ved hvem vi kan rådspørge HERREN; men jeg hader ham, fordi han aldrig spår mig godt, kun ondt; det er Mika, Jimlas Søn." Men Josafat sagde: Således må Kongen ikke tale!"
\par 9 Da kaldte Israels Konge på en Hofmand og sagde: "Hent hurtig Mika, Jimlas Søn!"
\par 10 Imidlertid sad Israels Konge og Kong Josafat af Juda, iført deres Skrud, hver på sin Trone i Samarias Portåbning, og alle Profeterne spåede foran dem.
\par 11 Da lavede Zidkija, Kena'anas Søn, sig Horn af Jern og sagde: "Så siger HERREN: Med sådanne skal du støde Aramæerne ned, til de er tilintetgjort!
\par 12 Og alle Profeterne spåede det samme og sagde: "Drag op mod Ramot i Gilead, så skal Lykken følge dig, og HERREN vil give det i Kongens Hånd!"
\par 13 Men Budet, der var gået efter Mika, sagde til ham: Se, Profeterne har alle som een givet Kongen gunstigt Svar. Tal du nu som de og giv gunstigt Svar!
\par 14 Men Mika svarede: "Så sandt HERREN lever: Hvad HERREN siger mig, det vil jeg tale!
\par 15 Da han kom til Kongen, spurgte denne ham: "Mika, skal vi drage i Krig mod Ramot i Gilead, eller skal vi lade være?" Da svarede han: "Drag derop, så skal Lykken følge dig, og HERREN vil give det i Kongens Hånd!"
\par 16 Men Kongen sagde til ham: "Hvor mange Gange skal jeg besvære dig, at du ikke siger mig andet end Sandheden i HERRENs Navn?"
\par 17 Da sagde han: Jeg så hele Israel spredt på Bjergene som en Hjord uden Hyrd: og HERREN sagde: De Folk har ingen Herre, lad dem vende tilbage i Fred, hver til sit!
\par 18 Israels Konge sagde da til Josafat: "Sagde jeg dig ikke, at han aldrig spår mig godt, kun ondt!
\par 19 Da sagde Mika: "Så hør da HERRENs Ord! Jeg så HERREN sidde på sin Trone og hele Himmelens Hær stå til højre og venstre for ham;
\par 20 og HERREN sagde: Hvem vil dåre Akab, så han drager op og falder ved Ramot i Gilead? En sagde nu eet, en anden et andet;
\par 21 men så trådte en Ånd frem og stillede sig foran HERREN og sagde: Jeg vil dåre ham! HERREN spurgte ham: Hvorledes?
\par 22 Han svarede: Jeg vil gå hen og blive en Løgnens Ånd i alle hans Profefers Mund! Da sagde HERREN: Ja, du kan dåre ham; gå hen og gør det!
\par 23 Se, således har HERREN lagt en Løgnens Ånd i alle disse dine Profeters Mund, thi HERREN har ondt i Sinde imod dig!"
\par 24 Da trådte Zidkija, Kena'as Søn, frem og slog Mika på kinden og, sagde: "Ad hvilken Vej skulde HERRENs Ånd have forladt mig for at tale til dig?"
\par 25 Men Mika sagde: "Det skal du få at se, den Dag du flygter fra Kammer til Kammer for at skjule dig!"
\par 26 Så sagde Israels Konge: "Tag Mika og bring ham tilbage til Amon, Byens Øverste, og Kongesønnen Joasj
\par 27 og sig: Såleds siger Kongen: Kast denne Mand i Fængsel og sæt ham på Trængselsbrød og Trængselsvand, indtil jeg kommer uskadt tilbage!"
\par 28 Men Mika sagde: "Kommer du uskadt tilbage, så har HERREN ikke talet ved mig!" Og han sagde: "Hør, alle I Folkeslag!"
\par 29 Så drog Israels Konge og Kong Josafat af Juda op mod Ramot i Gilead.
\par 30 Og Israels Konge sagde til Josafat: "Jeg vil forklæde mig, før jeg drager i Kampen; men tag du dine egne Klæder på!" Og Israels Konge forklædte sig og drog så i Kampen.
\par 31 Men Arams Konge havde givet sine to og tredive Vognstyrere den Befaling: "I må ikke angribe nogen, være sig høj eller lav, uden Israels Konge alene!"
\par 32 Da nu Vognstyrerne fik Øje på Josafat, tænkte de: "Det er sikkert Israels Konge!" Og de rettede deres Angreb mod ham. Da gav Josafat sig til at råbe;
\par 33 og da Vognstyrene opdagede, at det ikke var Israels konge, trak de sig bort fra ham.
\par 34 Men en Mand, der skød en Pil af på Lykke og Fromme, ramte Israels Konge mellem Remmene og Brynjen. Da sagde han til sin Vognstyrer: "Vend og før mig ud af Slaget, thi jeg er såret!"
\par 35 Men Kampen blev hårdere og hårdere den Dag, og Kongen holdt sig oprejst i sin Vogn over for Aramæerne til Aften, skønt Blodet fra Såret flød ned i Bunden at Vognen; men om Aftenen døde han.
\par 36 Da Solen gik ned, gik det Råb gennem Lejren: "Enhver drage hjem til sin By og sit Land,
\par 37 thi Kongen er død!" Så kom de til Samaria, og de jordede Kongen der.
\par 38 Og da man skyllede Vognen ved Samarias Dam, slikkede Hundene hans Blod, og Skøgerne badede sig deri efter det Ord, HERREN havde talet.
\par 39 Hvad der ellers er at fortælle om Akab, alt, hvad han gjorde, Elfenbenshuset, han lod opføre, og alle de Byer, han befæstede, står jo optegnet i Israels Kongers Krønike.
\par 40 Så lagde Akab sig til Hvile hos sine Fædre; og hans Søn Ahazja blev Konge i hans Sted.
\par 41 Josafat, Asas Søn, blev Konge over Juda i Kong Akab af Israels fjerde Regeringsår.
\par 42 Josafat var fem og tredive År gammel, da han blev Konge, og han herskede fem og tyve År i Jerusalem. Hans Moder hed Azuba og var en Datter af Sjilhi.
\par 43 Han vandrede nøje i sin Fader Asas Spor og veg ikke derfra, idet han gjorde, hvad der var ret i HERRENs Øjne.
\par 44 Kun blev Offerhøjene ikke fjernet, og Folket blev ved at ofre og tænde Offerild på Højene.
\par 45 Og Josafat havde Fred med Israels Konge.
\par 46 Hvad der ellers er at fortælle om Josafat, de Heltegerninger, han udførte, og de Krige, han førte, står jo optegnet i Judas Kongers Krønike.
\par 47 De sidste af Mandsskøgerne, som var tilbage fra hans Fader Asas Tid, udryddede han af Landet.
\par 48 På den Tid var der ingen Konge i Edom.
\par 49 Kong Josafats Statholder byggede et Tarsisskib til Fart på Ofir efter Guld; men der kom intet ud af det, da Skibet gik under ved Ezjongeber.
\par 50 Da foreslog Ahazja, Akabs Søn, Josafat at lade sine Folk sejle med hans; men Josafat afslog det.
\par 51 Så lagde Josafat sig til Hvile hos sine Fædre og blev jordet hos sine Fædre i sin Fader Davids By og hans Søn Joram blev Konge i hans Sted.
\par 52 Ahazja, Akabs Søn, blev Konge over Israel i Samaria i Kong Josafat af Judas syttende Regeringsår, og han herskede to År over Israel.
\par 53 Han gjorde, hvad der var ondt i HERRENs Øjne, og vandrede i sin Faders og sin Moders Spor og i Jeroboams, Nebats Søns, Spor, han, som forledte Israel til at synde.
\par 54 Han dyrkede Ba'al og tilbad ham og krænkede HERREN, Israels Gud, nøjagtigt som hans Fader havde gjort.


\end{document}