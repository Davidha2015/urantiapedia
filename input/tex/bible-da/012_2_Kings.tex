\begin{document}

\title{Anden Kongebog}


\chapter{1}

\par 1 Efter Alkahs Død faldt Moab fra Israel.
\par 2 Og Ahazja faldt ud gennem Vinduesgitteret for sin Stue på Taget i Samaria og blev syg; da sendte han Sendebud af Sted og sagde til dem: "Gå hen og spørg Ekrons Gud Ba'al-Zebub, om jeg kommer mig af min Sygdom!"
\par 3 Men HERRENs Engel sagde til Tisjbiten Elias: "Gå Sendebudene fra Samarias Konge i Møde og sig til dem: Mon det er, fordi der ingen Gud er i Israel, at I drager hen for at rådspørge Ekrons Gud Ba'al-Zebub?
\par 4 Derfor, så siger HERREN: Det Leje, du steg op på, kommer du ikke ned fra, thi du skal dø!" Dermed gik Elias bort.
\par 5 Da Sendebudene kom tilbage til ham, spurgte han dem: "Hvorfor kommer I tilbage?"
\par 6 De svarede: "En Mand kom os i Møde og sagde til os: Vend tilbage til Kongen, som har sendt eder, og sig: Så siger HERREN: Mon det er, fordi der ingen Gud er i Israel, at du sender Bud for at rådspørge Ekrons Gud Ba'al-Zebub? Derfor: Det Leje, du steg op på, kommer du ikke ned fra, thi du skal dø!"
\par 7 Han spurgte dem da: "Hvorledes så den Mand ud, som kom eder i Møde og sagde disse Ord til eder?"
\par 8 De svarede: "Det var en Mand i en lådden Kappe med et Læderbælte om Lænderne." Da sagde han: "Det er Tisjbiten Elias."
\par 9 Derpå sendte han en Halvhundredfører med hans halvtredsindstyve Mand ud efter ham; og da han kom op til ham på Bjergets Top, hvor han sad, sagde han til ham: "Du Guds Mand! Kongen byder: Kom ned!"
\par 10 Men Elias svarede Halvhundredføreren: "Er jeg en Guds Mand, så fare Ild ned fra Himmelen og fortære dig og dine halvtredsindstyve Mand!" Da for Ild ned fra Himmelen og fortærede ham og hans halvtredsindstyve Mand.
\par 11 Atter sendte Kongen en Halvhundredfører med hans halvtredsindstyve Mand ud efter ham; og da han kom derop, sagde han til ham: "Du Guds Mand! Således siger Kongen: Kom straks ned!"
\par 12 Men Elias svarede ham: "Er jeg en Guds Mand, så fare Ild ned fra Himmelen og fortære dig og dine halvtredsindstyve Mand!" Da for Guds Ild ned fra Himmelen og fortærede ham og hans halvtredsindstyve Mand.
\par 13 Atter sendte Kongen en Halvhundredfører ud efter ham: men da den tredje Halvhundredfører kom derop, kastede han sig på Knæ for Elias, bønfaldt ham og sagde: "du Guds Mand! Lad dog mit og disse dine halvtredsindstyve Trælles Liv være dyrebarl i dine Øjne!
\par 14 Se, Ild for ned fra Himmelen og fortærede de to første Halvhundredførere og deres halvtredsindstyve Mand, men lad nu mit Liv være dyrebart i dine Øjne!"
\par 15 Da sagde HERRENs Engel til Elias: "Gå ned med ham, frygt ikke for ham!" Så gik han ned med ham og fulgte ham til Kongen.
\par 16 Og han sagde til Kongen: "Så siger HERREN: Fordi du har sendt Sendebud hen at rådspørge Ekrons Gud Ba'al-Zebub - men det er, fordi der ingen Gud er i Israel, du kunde rådspørge? - derfor: Det Leje, du steg op på, kommer du ikke ned fra, thi du skal dø!"
\par 17 Og han døde efter det HERRENs Ord, som Elias havde talt. Og hans Broder Joram blev Konge i hans Sted i Josafats Søns, Kong Joram af Judas, andet Regeringsår; thi han havde ingen Søn.
\par 18 Hvad der ellers er at fortælle om Ahazja, hvad han udførte, står jo optegnet i Israels Kongers Krønike.

\chapter{2}

\par 1 Dengang HERREN ville lade Elias fare op til Himmelen i et Stormvejr, gik Elias fra Gilgal.
\par 2 Og Elias sagde til Elisa: "Bliv her, thi HERREN vil have mig til Betel!" Men Elisa svarede: "Så sandt HERREN lever, og så sandt du lever, jeg går ikke fra dig!" De gik da ned til Betel.
\par 3 Så kom Profetsønnerne i Betel ud til Elisa og sagde til ham: "Ved du, at HERREN i dag vil tage din Herre bort fra dig?" Han svarede: "Ja, jeg ved det, ti kun stille!"
\par 4 Derpå sagde Elias til ham: "Bliv her, Elisa, thi HERREN vil have mig til Jeriko!" Men han svarede: "Så sandt HERREN lever, og så sandt du lever, jeg går ikke fra dig!" De kom da til Jeriko.
\par 5 Men Profetsønnerne i Jeriko trådte hen til Elisa og sagde til ham: "Ved du, at HERREN i Dag vil tage din Herre bort fra dig?" Han svarede: "Ja, jeg ved det, ti kun stille!"
\par 6 Derpå sagde Elias til ham: "Bliv her, thi HERREN vil have mig til Jordan!" Men han svarede: "Så sandt HERREN lever, og så sandt du lever, jeg går ikke fra dig!" Så fulgtes de ad.
\par 7 Men halvtredsindstyve af Profetsønnerne gik hen og stillede sig et godt Stykke derfra, medens de to stod ved Jordan.
\par 8 Elias tog nu sin Kappe, rullede den sammen og slog Vandet med den; da skiltes det ad, og de gik begge over på tør Bund.
\par 9 Og da de var kommet over, sagde Elias til Elisa: "Sig, hvad du ønsker, jeg skal gøre for dig, før jeg tages bort fra dig!" Elisa svarede: "Måtte to Dele af din Ånd komme over mig!"
\par 10 Da sagde han: "Det er et stort Forlangende, du kommer med! Dersom du ser mig, når jeg tages bort fra dig, skal det blive dig til Del, ellers ikke!"
\par 11 Medens de nu gik og talte sammen, se, da kom en lldvogn og Ildheste og skilte dem ad, og Elias for op til Himmelen i Stormvejret.
\par 12 Og Elisa så det og råbte: "Min Fader, min Fader, du Israels Vogne og Ryttere!" Og han så ham ikke mere. Så greb han sine Klæder og sønderrev dem.
\par 13 Derpå tog han Elias's Kappe, som var faldet af ham, op og gik tilbage og stillede sig ved Jordans Bred,
\par 14 og han tog Elias's Kappe, som var faldet af ham, slog Vandet med den og sagde: "Hvor er nu HERREN, Elias's Gud?" Og da han havde slået Vandet, skiltes det ad, og Elisa gik over.
\par 15 Da Profetsønnerne fra Jeriko så det derovre, sagde de: "Elias's Ånd hviler på Elisa!" Og de gik ham i Møde og kastede sig til Jorden for ham.
\par 16 Derpå sagde de til ham: "Se, her hos dine Trælle er der halvtredsindstyve raske Mænd, lad dem gå ud og lede efter din Herre; måske HERRENs Ånd har taget ham og kastet ham hen på et af Bjergene eller i en af Dalene!" Men han svarede: "I skal ikke sende dem af Sted!"
\par 17 Men da de blev ved at trænge ind på ham, sagde han: "Så send dem da af Sted!" Så sendte de halvtredsindstyve Mænd ud, og de ledte efter ham i tre Dage, men fandt ham ikke.
\par 18 Og da de kom tilbage, medens han endnu var i Jeriko, sagde han til dem: "Sagde jeg ikke til eder, at I ikke skulde gå?"
\par 19 Mændene i Byen sagde til Elisa: "Byen ligger godt nok, som min Herre ser, men Vandet er dårligt og volder utidige Fødsler i Egnen."
\par 20 Da sagde han: "Hent mig en ny Skål og kom Salt deri!" Og de hentede den til ham.
\par 21 Så gik han ned til Kildevældet og kastede Salt deri, idet han sagde: "Så siger HERREN: Jeg gør dette Vand sundt, så at der ikke mer skal komme Død eller utidige Fødsler deraf!"
\par 22 Så blev Vandet sundt, efter det Ord Elisa talte; og det er det den Dag i dag.
\par 23 Derfra begav han sig op til Betel. Som han var på Vej derop, kom nogle Smådrenge ud af Byen og spottede ham og råbte: "Kom herop, Skaldepande, kom herop, Skaldepande!"
\par 24 Han vendte sig om, og da han fik Øje på dem, forbandede han dem i HERRENs Navn. Så kom to Bjørne ud af Krattet og sønderrev to og fyrretyve af Drengene.
\par 25 Derfra begav han sig til Karmels Bjerg, og derfra vendte han tilbage til Samaria.

\chapter{3}

\par 1 Joram, Akabs Søn, blev Konge over Israel i Samaria i Kong Josafat af Judas attende Regeringsår, og han herskede i tolv År.
\par 2 Han gjorde, hvad der var ondt i HERRENs Øjne, dogikke som bans Fader og Moder, og han fjernede Ba'als Stenstøtter, som hans Fader havde ladet lave.
\par 3 Men han holdt fast ved de Synder, som Jeroboam, Nebats Søn, forledte Israel til; dem veg han ikke fra.
\par 4 Kong Mesja af Moab drev Kvægavl og svarede Israels Konge en Afgift på 100.000 Lam og Ulden af 100.000 Vædre.
\par 5 Men efter Akabs Død faldt Moabs Konge fra Israels Konge.
\par 6 Da drog Koog Joram straks ud fra Samaria og mønstrede hele Israel;
\par 7 desuden sendte han Bud til Kong Josafat af Juda og lod sige: "Moabs Konge er faldet fra mig; vil du drage med i Krig mod Moab?" Han svarede: "Ja, jeg vil; jeg som du, mit Folk som dit, mine Heste som dine!"
\par 8 Og han spurgte: "Hvilken Vej skal vi drage?" Han svarede: "Gennem Edoms Ørken!"
\par 9 Så drog Israels, Judas og Edoms Konger af Sted. Men da de havde tilbagelagt en Strækning af syv Dagsrejser, var der ikke Vand til Hæren og Dyrene, som de havde med.
\par 10 Da sagde Israels Konge: "Ak, at HERREN har kaldt disse tre Konger sammen for at overgive dem i Moabs Hånd!"
\par 11 Men Josafat sagde: "Er her ingen af HERRENs Profeter, ved hvem vi kan rådspørge HERREN?" Da svarede en af Israels Konges Folk: "Jo, her er Elisa, Sjafats Søn, som øste Vand på Elias's Hænder."
\par 12 Josafat sagde: "Hos ham er HERRENs Ord!" Og Israels Konge og Josafat og Edoms Konge begav sig ned til ham.
\par 13 Men Elisa sagde til Israels Konge: "Hvad har jeg med dig at gøre? Gå du til din Faders og Moders Profeter!" Israels Konge svarede: "Ak nej, thi HERREN har kaldt disse tre Konger sammen for at give dem i Moabs Hånd."
\par 14 Da sagde Elisa: "Så sandt Hærskarers HERRE lever, for hvis Åsyn jeg står: Var det ikke for Kong Josafat af Judas Skyld, vilde jeg ikke se til dig eller værdige dig et Blik!
\par 15 Men hent mig nu en Strengespiller!" Thi når Strengespilleren spillede, kom HERRENs Hånd over ham.
\par 16 Derpå sagde han: "Så siger HERREN: Grav Grøft ved Grøft i Dalen her!
\par 17 Thi så siger HERREN: I skal hverken mærke til Blæst eller Regn, men alligevel skal Dalen her fyldes med Vand, så at I, eders Hær og eders Dyr kan drikke!
\par 18 Dog tykkes dette HERREN for lidet, han vil også give Moab i eders Hånd;
\par 19 I skal indtage alle befæstede Byer og alle betydelige Byer, alle Frugttræer skal t fælde, alle Kilder skal I tilstoppe, og al frugtbar Agerjord skal I ødelægge med Sten!"
\par 20 Og næste Morgen ved Afgrødeofferets Tid kom der Vand fra den Kant, hvor Edom ligger, så hele Egnen blev fuld af Vand.
\par 21 Da alle Moabiterne hørte, at Kongerne var draget op for at føre Krig med dem, blev enhver, der overhovedet kunde bære Våben, opbudt, og de tog Stilling ved Grænsen.
\par 22 Men tidligt om Morgenen, da Solen stod op over Vandet, så Moabiterne Vandet foran sig rødt som Blod.
\par 23 Da råbte de: "Det er Blod! Kongerne har fuldstændig tilintetgjort hverandre, de har hugget hverandre ned; nu til Byttet, Moab!"
\par 24 Men da de nåede Israels Lejr, brød Israeliterne op og slog Moabiterne på Flugt. Derpå rykkede de frem og huggede Moabiterne ned;
\par 25 Byerne nedbrød de; på al frugtbar Agerjord kastede de hver sin Sten, så den blev fuld af Sten; alle Kildevæld tilstoppede de, og alle Frugttræer fældede de. Til sidst var kun Kir-Hareset tilbage, og denne By omringede Slyngekasterne og skød på den.
\par 26 Da Moabs Konge så, at han ikke kunde modstå Angrebet, samlede han 700 sværdvæbnede Mænd for at bryde igennem hen til Kongen af Edom. men det lykkedes ikke.
\par 27 Så tog han sin førstefødte Søn, der skulde følge ham på Tronen, og ofrede ham som Brændoffer på Muren. Da kom heftig Vrede over Israel, og de brød op og vendte hjem til deres Land.

\chapter{4}

\par 1 En Kvinde, som var gift med en af Profetsønnerne råbte til Elisa: "Din Træl, min Mand, er død; og du ved, at din Træl frygtede HERREN. Og nu kommer en, der har Krav på ham, for at tage mine fo Drenge til Trælle!"
\par 2 Da sagde Elisa til hende: "Hvad kan jeg gøre for dig? Sig mig, hvad du har i Huset?" Hun svarede: "Din Trælkvinde har ikke andet i Huset end et Krus Olie."
\par 3 Da sagde han: "Gå ud og bed alle dine Naboer om tomme Dunke, ikke for få!
\par 4 Så lukker du dig inde med dine Sønner og fylder på alle disse Dunke, og når de er fulde, sætter du dem til Side!"
\par 5 Så gik hun fra ham og lukkede sig inde med sine Sønner; og de rakte hende Dunkene, medens hun fyldte på.
\par 6 Og da Dunkene var fulde, sagde hun til Sønnen: "Ræk mig een Dunk til!" Men han svarede: "Der er ikke flere Dunke!" Da holdt Olien op at flyde.
\par 7 Det kom hun og fortalte den Guds Mand; og han sagde: "Gå hen og sælg Olien og betal din Gæld; og lev så med dine Sønner af Resten!"
\par 8 Det skete en Dag, at Elisa på sin Vej kom til Sjunem. Der boede en velhavende Kvinde, som nødte ham til at spise hos sig; og hver Gang han senere kom forbi, tog han derind og spiste.
\par 9 Hun sagde nu til sin Mand: "Jeg ved, af det er en hellig Guds Mand., der stadig kommer her forbi;
\par 10 lad os mure en lille Stue på Taget og sætfe Seng, Bord, Stol og Lampe ind til ham, for at han kan gå derind, når han kommer til os!"
\par 11 Så kom han en Dag derhen og gik op i Stuen og lagde sig.
\par 12 Og han sagde til sin Tjener Gehazi: "Kald på Sjunemkvinden!" Og han kaldte på hende, og hun trådte frem for ham.
\par 13 Da sagde han til Gehazi: "Sig til hende: Se, du har Itaft al den Ulejlighed for vor Skyld; hvad kan jeg gøre for dig? Ønsker du, at jeg skal tale din Sag hos Kongen eller Hærføreren?" Men hun svarede: "Jeg bor midt iblandt mit Folk!"
\par 14 Da sagde han: "Hvad kan jeg da gøre for hende?" Gehazi sagde: "Jo, hun har ingen Søn, og hendes Mand er gammel."
\par 15 Da sagde han: "Kald på hende!" Og han kaldte på hende, og bun tog Plads ved Døren.
\par 16 Da sagde han: "Om et År ved denne Tid har du en Dreng ved Brystet!" Men hun sagde: "Nej dog, Herre! Den Guds Mand må ikke narre sin Trælkvinde!"
\par 17 Men Kvinden blev frugtsommelig og fødte en Søn Året efter ved samnme Tid, således som Elisa havde sagt hende.
\par 18 Da Drengen var blevet stor, gik han en Dag ud til sin Fader hos Høstfolkene.
\par 19 Da sagde han til sin Fader: "Mit Hoved, mit Hoved!" Og hans Fader sagde til en Karl " "Bær ham hjem til hans Moder!"
\par 20 Han tog, ham og har ham hjem til hans Moder, og han sad på hendes Skød til Middag; så døde han.
\par 21 Men hun gik op og lagde ham på den Guds Mands Seng, og derefter lukkede hun Døren og gik.
\par 22 Så kaldte hun på sin Mand og sagde: "Send mig en af Karlene med et Æsel, for at jeg hurfig kan komme hen til den Guds Mand og hjem igen!"
\par 23 Han spurgte: "Hvad vil du hos ham i Dag? Det er jo hverken Nymånedag eller Sabbat!" Men hun sagde: "Lad mig om det!"
\par 24 Derpå sadlede hun Æselet og sagde til Karlen: "Driv nu godt på! Stands mig ikke i Farten, før jeg siger til!"
\par 25 Så drog hun af Sted og kom til den Guds Mand på Kamels Bjerg. Da den Guds Mand fik Øje på hende i Frastand, sagde han til sin Tjener Gehazi: "Se, der er Sjunemkvinden!
\par 26 Løb hende straks i Møde og spørg hende: Har du det godt? Har din Mand det godt? Har Drengen det godt?" Hun svarede: "Ja, vi har det godt!"
\par 27 Men da hun kom hen til den Guds Mand på Bjerget, klamrede hun sig til hans Fødder. Gehazi trådte til for at støde hende bort. men den Guds Mand sagde: "Lad hende være, hun er i Vånde, og HERREN har dulgt det for mig og ikke åbenbaret mig det!"
\par 28 Da sagde hun: "Har jeg vel bedt min Herre om en Søn? Sagde jeg ikke, at du ikke måtte narre mig?"
\par 29 Så sagde han til Gehazi: "Omgjord din Lænd, tag min Stav i Hånden og drag af Sted! Møder du nogen, så hils ikke på ham. og hilser nogen på dig, så gengæld ikke hans Hilsen; og læg min Stav på Drengens Ansigt!"
\par 30 Men Drengens Moder sagde: "Så sandt HERREN lever, og så sandt du lever, jeg går ikke fra dig!" Da stod han op og gik med hende.
\par 31 Imidlertid var Gehazi gået i Forvejen og havde lagt Staven på Drengens Ansigt; men ikke en Lyd hørtes, og der var intet Livstegn. Da vendte han tilbage og gik Elisa i Møde, meldte ham det og sagde: "Drengen vågnede ikke!"
\par 32 Og da Elisa var kommet ind i Huset, så han Drengen ligge død på Sengen.
\par 33 Han gik da hen og lukkede sig inde med ham og bad til HERREN.
\par 34 Derpå steg han op og lagde sig oven på Drengen med sin Mund på hans Mund, sine Øjne på hans Øjne og sine Hænder på hans Hænder, og medens han således bøjede sig over ham, blev Drengens Legeme varmt.
\par 35 Så steg han ned og gik een Gang frem og tilbage i Huset, og da han atter steg op og bøjede sig over Drengen, nyste denne syv Gange og slog Øjnene op.
\par 36 Derpå kaldte han på Gehazi og sagde: "Kald på Sjunemkvinden!" Han kaldte så på hende, og hun kom til ham. Da sagde han: "Tag din Dreng!"
\par 37 Og hun trådte hen og faldt ned for hans Fødder og kastede sig til Jorden, tog så sin Dreng og gik ud.
\par 38 Dengang Hungersnøden var i Landet, vendte Elisa tilbage til Gilgal. Som nu Profetsønnerne sad hos ham, sagde han til sin Tjener: "Sæt den store Gtyde over og kog en Ret Mad til Profetsønnerne!"
\par 39 Så gik en ud på Marken for at plukke Urter, og da han fandt en Slyngplante med vilde Agurker, plukkede han så mange, han kunde bære i sin Kappe; da han kom tilbage, skar han dem itu og kom dem i Gryden, thi han kendte dem ikke.
\par 40 Derpå øste man op for Mændene, for at de kunde spise, men så snart de smagte Maden, skreg de op og råbte: "Døden er i Gryden, du Guds Mand!" Og de kunde ikke spise Maden.
\par 41 Men han sagde: "Hent noget Mel!" Og da han havde hældt det i Gryden, sagde han: "Øs nu op for Folkene, så de kan spise!" Så var der ingen Ulykke i Gryden mere.
\par 42 Engang kom en Mand fra Ba'al-Sjalisja og bragte den Guds Mand Brød af nyt Korn, tyve Bygbrød, og nyhøstet Korn i sin Ransel. Da sagde han: "Giv Folkene det at spise!"
\par 43 Men hans Tjener sagde: "Hvorledes skal jeg kunne sætte dette frem for hundrede Mennesker?" Men han sagde: "Giv Folkene det at spise! Thi så siger HERREN: De skal spise og levne!"
\par 44 Da satte han det frem for dem, og de spiste og levnede efter HERRENs Ord.

\chapter{5}

\par 1 Na'aman, Kongen af Arams Hærfører, havde meget at sige hos sin Herre og var højt agtet; thi ved ham havde HERREN givet Aramæerne Sejr; men Manden var spedalsk.
\par 2 Nu havde Aramæerne engang på et Strejtog røvet en lille Pige i Israels Land; hun var kommet i Tjeneste hos Na'amans Hustru,
\par 3 og hun sagde til sin Frue: "Gid min Herre var hos Profeten i Samaria; han vilde sikkert skille ham af med hans Spedalskhed!"
\par 4 Så kom Na'aman og fortalte sin Herre, hvad Pigen fra Israels Land havde sagt.
\par 5 Da sagde Arams Konge: "Rejs derhen! Jeg skal sende etBrevmed til Israels Konge!" Så rejste han og tog ti Talenter Sølv, 6000 Sekel Guld og ti Sæt Festklæder med.
\par 6 Og han overbragte IsraelsKonge Brevet. Deri stod der: "Når dette Brev kommer dig i Hænde, skal du vide, at jeg sender min Tjener Na'aman til dig, for at du skal skille ham af med hans Spedalskhed!"
\par 7 Da Israels Konge havde læst Brevet, sønderrev han sine klæder og sagde: "Er jeg Gud, så jeg råder over Liv og Død, siden han skriver til mig, at jeg skal skille en Mand afmed hans Spedalskhed Nej, I kan da se, at han søger Lejlighed til Strid med mig!"
\par 8 Men da den Guds Mand Elisa hørte, at Israels Konge havde sønderrevet sine klæder, sendte han det Bud til Kongen: "Hvorfor sønderriver du dine Klæder? Lad ham komme til mig, så skal han kende, at der er en Profet i Israel!"
\par 9 Da kom Na'aman med Heste og Vogne og holdt uden for Døren til Elisas Hus.
\par 10 Elisa sendte et Bud ud til ham og lod sige: "Gå hen og bad dig syv Gange i Jordan, så bliver dit Legeme atter friskt, og du bliver ren!"
\par 11 Men Na'aman blev vred og drog bort med de Ord: "Se, jeg havde tænkt, at han vilde komme ud til mig, stå og påkalde HERREN sin Guds Navn og svinge sin Hånd i Retning af Helligdommen og således gøre Ende på Spedalskheden!
\par 12 Er ikke Damaskus's Floder Abana og Parpar fuldt så gode som alle Israels Vande? Kunde jeg ikke blive ren ved at bade mig i dem?" Og han vendte sig og drog bort i Vrede.
\par 13 Men hans Trælle kom og sagde til ham: "Dersom Profeten havde pålagt dig noget, som var vanskeligt vilde du så ikke have gjort det? Hvor meget mere da nu, da han sagde til dig: Bad dig, så bliver du ren!"
\par 14 Så drog han ned og dykkede sig syv Gange i Jordan efter den Guds Mands Ord; og hans Legeme blev atter friskt som et Barns, og han blev ren.
\par 15 Så vendte han med hele sit Følge tilbage til den Guds Mand, og da han var kommet derhen, trådtehan frem forham og sagde: "Nu ved jeg, at der ingensteds på Jorden er nogen Gud uden i Israel! Så modtag nu en Takkegave af din Træl!"
\par 16 Men han svarede: "Så sandt HERREN lever, for hvis Åsyn jeg står, jeg modtager ikke noget!" Og skønt han nødte ham, vægrede han sig ved at modtage noget.
\par 17 Da sagde Na'aman: "Så lad da være! Men lad din Træl få så megel Jord, som et Par Muldyr kan bære, thi din Træl vil aldrig mere ofre Brændoffer eller Slagtoffer til nogen anden Gud end HERREN!
\par 18 Men i een Ting vil HERREN nok bære over med din Træl: Når min Herre går ind i Rimmons Hus for at tilbede og støtfer sig til min Arm og jeg så sammen med ham kaster mig til Jorden i Rimmons Hus, da vil HERREN nok bære over ed din Træl i den Ting!"
\par 19 Han svarede: "Far i Fred!" Men da han var kommet et Stykke hen ad Vejen,
\par 20 sagde Gehazi, den Guds Mand Elisas Tjener, ved sig selv: "Der har min Herre ladet denne Aramæer Na'aman slippe og ikke modtaget af ham, hvad han havde med; så sandt HERREN lever, jeg vil løbe efter ham for at få noget af ham!"
\par 21 Så satte Gehazi efter Na'aman. og da Na'aman så ham komme løbende efter sig, sprang han af Vognen, gik ham i Møde og spurgte: "Står det godt til?"
\par 22 Han svarede: "Ja, det står godt til! Min Herre sender mig med det Bud: Der kom lige nu to unge Mænd, som hører til Profetsønnerne, til mig fra Efraims Bjerge: giv dem en Talent Sølv og to Sæt Festklæder!"
\par 23 Da sagde Na'aman: "Tag dog mod to Talenter Sølv!" Og han nødte ham. Så bandt han to Talenter ind i to Punge og tog to Sæt Festklæder og gav to af sine Trælle dem, for at de skulde bære dem foran ham.
\par 24 Men da de kom til Højen, tog han Pengene fra dem, gemte dem i Huset og lod Mændene gå.
\par 25 Så gik han ind til sin Herre og trådte hen til ham. Da spurgte Elisa: "Hvor har du været, Gehazi?" Han svarede: "Din Træl har ingen Steder været!"
\par 26 Så sagde han til ham: "Gik jeg ikke i Ånden hos dig, da en stod af sin Vogn og gik tilbage for at møde dig? Nu har du fået Penge, og du kan få Klæder, Olivenlunde og Vingårde, Småkvæg og Hornkvæg, Trælle og Trælkvinder,
\par 27 men Na'amans Spedalskhed skal hænge ved dig og dit Afkom til evig Tid!" Og Gehazi gik fra ham, hvid som Sne af Spedalskhed.

\chapter{6}

\par 1 Profetsønnerne sagde engang til Elisa: "Se, der er for lidt Plads til os her, hvor vi sidder hos dig.
\par 2 Vi vil gå til Jordan og hver fage en Bjælke og der indrette os et Rum, vi kan sidde i!" Han sagde: "Ja, gør det!"
\par 3 Men en af dem sagde: "Vil du ikke nok følge med dine Trælle!" Og han svarede: "Jo, det vil jeg!"
\par 4 Han gik så med, og da de kom til Jordan, gav de sig til at fælde Træer.
\par 5 Medens nu en af dem var ved at fælde en Bjælke, faldt hans Øksejern i Vandet. Da gav han sig til at råbe: "Ak, min Herre! Og det var endda lånt!"
\par 6 Men den Guds Mand sagde: "Hvor faldt det?" Og da han havde vist ham Stedet, skar han en Gren af og kastede den derhen. Da kom Øksen op på Overfladen;
\par 7 og han sagde: "Tag den op!" Så rakte han Hånden ud og tog den.
\par 8 Engang Arams Konge lå i Krig med Israel, aftalte han med sine Folk, at de skulde lægge sig iBaghold på det og det Sted.
\par 9 Men den Guds, Mand sendte Bud til Israels Konge og lod sige: "Vogt dig for at drage forbi det Sted, thi der ligger Aramæerne i Baghold!"
\par 10 Israels Konge sendte da Folk til det Sted, den Guds Mand havde sagt ham.
\par 11 Derover blev Arams Konge urolig i sit Sind, og han lod sine Folk kalde og spurgte dem: "Han I ikke sige mig, hvem det er, der forråder os til Israels Konge?"
\par 12 Da sagde en af hans Hærførere: "Det er ingen af os, Herre Konge; det er Profeten Elisa i Israel, der lader Israels Konge vide, hvad du taler i dit Sovekammer."
\par 13 Da sagde han: "Gå hen og se, hvor han er, for at jeg kan sende Folk ud og lade ham gribe!" Da det meldtes ham, at han var i Dotan,
\par 14 sendte han Heste og Vogne og en stor Hærstyrke derhen; og de kom ved Nattetide og omringede Byen.
\par 15 Næste Morgen tidlig, da den Guds Mand gik ud, se, da var Byen omringet af en Hær og Heste og Vogne, Da sagde hans Tjener til ham: "Ak, Herre, hvad skal vi dog gribe til?"
\par 16 Men han svarede: "Frygt ikke, thi de, der er med os, er flere end de, der er med dem!"
\par 17 Og Elisa bad og sagde: "HERRE, luk hans Øjne op, så han kan se!" Da lukkede HERREN Tjenerens Øjne op, og han så, at Bjerget var fuldt af Ildheste og Ildvogne rundt om Elisa.
\par 18 Da nu Fjenderne rykkede ned imod ham, bad Elisa til HERREN og sagde: "Slå de Folk med Blindhed!" Og han slog dem med Blindhed efter Elisas Ord.
\par 19 Da sagde Elisa til dem: "Det er ikke den rigtige Vej eller den rigtige By; følg med mig, så skal jeg føre eder til den Mand, I søger!" Han førte dem så til Samaria,
\par 20 og da de var kommet ind i Samaria, bad Elisa: "Herre, luk nu deres Øjne op, så at de kan se!" Da lukkede HERREN deres Øjne op, og de så, at de var midt i Samaria.
\par 21 Da Israels Konge så dem, spurgte han Elisa: "Skal jeg hugge dem ned, min Fader?"
\par 22 Men han svarede: "Nej, du må ikke hugge dem ned! Bruger du at hugge Folk ned, som du ikke har taget til Fange med Sværd eller Bue? Sæt Brød og Vand for dem, at de kan spise og drikke, og lad dem så vende tilbage til deres Herre!"
\par 23 Så gav han dem et godt Måltid, og da de havde spist og drukket, lod han dem gå, og de drog tilbage til deres Herre. Men fra den Tid af kom der ikke flere aramaiske Strejfskarer i Israels Land.
\par 24 Siden hændte det, at Kong Benhadad af Aram samlede hele sin Hær og drog op og belejrede Samaria;
\par 25 og under Belejringen blev der stor Hungersnød i Byen, så at et Æselhoved til sidst kostede tresindstyve Sekel Sølv og en Fjerdedel Kab Duegødning fem.
\par 26 Da Israels Konge en Dag gik oppe på Bymuren, råbte en Kvinde til ham: "Hjælp, Herre Konge!"
\par 27 Han svarede: "Hjælper HERREN dig ikke, hvor skal så jeg skaffe dig Hjælp fra? Fra Tærskepladsen eller Vinpersen?"
\par 28 Og Kongen spurgte hende vi dere: "Hvad fattes dig?" Da sagde hun: "Den Kvinde der sagde til mig: Hom med din Dreng, så fortærer vi ham i Dag; i Morgen vil vi så fortære min Dreng!
\par 29 Så kogte vi min breng og fortærede ham. Næste Dag sagde jeg til hende: Kom nu med din Dreng, at vi kan fortære ham! Men hun holdt Drengen skjult."
\par 30 Da Kongen hørte Kvindens Ord, sønderrev han sine Klæder, som han stod der på Muren; og Folket så da, at han indenunder har Sæk på den bare Krop.
\par 31 Og han sagde: "Gud ramme mig både med det ene og det andet, om Elisas, Sjafats Søns, Hoved skal blive siddende mellem Skuldrene på ham Dagen til Ende!"
\par 32 Elisa sad imidlertid i sit Hus sammen med de Ældste; da sendte Kongen en Mand i Forvejen. Men før Sendebudet kom til ham, sagde han til de Ældste: "Ved I, at denne Mordersjæl har sendt en Mand herhen for at tage mit Hoved? Se, når Budet kommer, skal I lukke Døren og stemme jer imod den! Allerede hører jeg hans Herres trin bag ham."
\par 33 Og medens han endnu talte med dem, kom Kongen ned til ham og sagde: "Se, hvilken Ulykke HERREN har bragt over os! Hvorfor skal jeg da bie længer på HERREN?"

\chapter{7}

\par 1 Men Elisa sagde: "Hør HERRENs Ord! Så siger HERREN: I Morgen ved denne Tid skal en Sea fint Hvedemel koste en Sekel og to Sea Byg ligeledes en Sekel i Samarias Port!"
\par 2 Men Høvedsmanden, til hvis Arm Kongen støttede sig, svarede den Guds Mand og sagde: "Om så HERREN satte Vinduer på Himmelen, mon da sligt kunde ske?" Han sagde: "Med egne Øjne skal du få det at se, men ikke komme til at spise deraf!"
\par 3 Imidlertid varder fire spedalske Mænd ved indgangen til Porten; de sagde til hverandre: "Hvorfor skal vi blive her, til vi dør?
\par 4 Dersom vi bestemmer os til at gå ind i Byen, dør vi der - der er jo Hungersnød i Byen - og bliver vi her, dør vi også! Kom derfor og lad os løbe over til Aramæernes Lejr! Lader de os leve, så bliver vi i Live, og slår de os ihjel, så dør vi!"
\par 5 Så begav de sig i Mørkningen på Vej til Aramæernes Lejr; men da de kom til Udkanten af Aramæernes Lejr, var der ikke et Menneske at se.
\par 6 HERREN havde nemlig ladet Aramæernes Lejr høre Larm at Vogne og Heste, Larm af en stor Hær, og de havde sagt til hverandre: "Se, Israels Konge har købt Hetiternes og Mizrajims Konger til at falde over os!"
\par 7 Derfor havde de taget Flugten i Mørkningen og efterladt Lejren, som den stod, deres Telte, Heste og Æsler, og var flygtet for at redde Livet.
\par 8 Da nu de spedalske havde nået Udkanten af Lejren, gik de ind i et af Teltene, spiste og drak og tog Sølv og Guld og klæder, som de gik hen og gemte; derpå kom de tilbage og gik ind i et andet Telt; også der tog de noget, som de gik hen og gemte.
\par 9 Så sagde de til hverandre: "Det er ikke rigtigt, som vi bærer os ad! Denne Dag er et godt Budskabs Dag; tier vi stille og venter til Daggry, pådrager vi os Skyld; lad os derfor gå hen og melde det i Kongens Palads!"
\par 10 Så gik de hen og råbte til Byens Portvægtere og bragte dem den Melding: "Vi kom til Aramæernes Lejr, og der var ikke et Menneske at se eller høre, men vi fandt Hestene og Æslerne bundet og Teltene urørt!"
\par 11 Portvægterne råbte det ud, og man meldte det inde i Kongens Palads.
\par 12 Men Kongen stod op om Natten og sagde til sine Folk: "Jeg skal sige eder, hvad Aramæerne har for med os; de ved, at vi er udsultet, derfor har de forladt Lejren og skjult sig på Marken, i den Tanke at vi skal gå ud af Byen, så de kan fange os levende og trænge ind i Byen!"
\par 13 Men en af Folkene svarede: - "Vi kan jo tage en fem Stykker af de Heste, der er tilbage her - det vil jo dog gå dem som alle de mange, der allerede var omkommet - og sende Folk derhen, så får vi se!"
\par 14 De tog da to Spand Heste, og Kongen sendte Folk efter Aramæernes Hær og sagde: "Rid hen og se efter!"
\par 15 Så drog de efter dem lige til Jordan og fandt hele Vejen fuld af Klæder og Våben, som Aramæerne havde kastet fra sig på deres hovedkulds Flugt. Derpå vendte Sendebudene tilbage og meldte det til Kongen.
\par 16 Så drog Folket ud og plyndrede Aramæernes Lejr; og således kom en Sea fint Hvedemel til at koste en Sekel og to Sea Byg ligeledes en Sekel, som HERREN havde sagt.
\par 17 Kongen havde overdraget Høvedsmanden, til hvis Arm han støttede sig, Tilsynet med Porten, men Folket trådte ham ned i Porten, så han døde, således som den Guds Mand havde sagt, dengang Kongen kom ned til ham.
\par 18 Da den Guds Mand sagde til Kongen: "To Sea Byg skal koste en Sekel og en Sea fitnt Hvedemel ligeledes en Sekel i Morgen ved denne Tid i Samarias Port!"
\par 19 da havde Høvedsmanden svaret ham: "Om så HERREN satte Vinduer på Himmelen, mon da sligt kunde ske?" Og den Guds Mand havde sagt: "Med egne Øjne skal du få det at se, men ikke komme til at spise deraf!"
\par 20 Således gik det ham; Folket trådte ham ned i Porten, så han døde.

\chapter{8}

\par 1 Elisa talte til den Kvinde, hvis Søn han havde kaldt til Live, og sagde: "Drag bort med dit Hus og slå dig ned som fremmed et eller andet Sted, thi HERREN har kaldt Hungersnøden hid; og den vil komme over Landet og vare syv År!"
\par 2 Da brød Kvinden op og gjorde, som den Guds Mand havde sagt, og drog med sit Hus hen og boede syv År som fremmed i Filisternes Land.
\par 3 Men da der var gået syv År, vendte Kvinden tilbage fra Filisternes Land; Og hun gik hen og påkaldte Kongens Hjælp til at få sit Hus og sin Jord tilbage.
\par 4 Kongen talte just med den Guds Mands Tjener Gehazi og sagde: "Fortæl mig om alle de storeGerninger, Elisa harudført!"
\par 5 Og netop som han fortalte Kongen, hvorledes han havde kaldt den døde til Live, kom Kvinden, hvis Søn han havde kaldt til Live, og påkaldt Kongens Hjælp til at få sit Hus og sin Jord tilbage. Da sagde Gehazi: "Herre Konge, der er den Kvinde, og der er hendes Søn. som Elisa kaldte til Live!"
\par 6 Kongen spurgte så Kvinden ud, og hun forfalte. Derpå gav Kongen hende en Hofmand med og sagde: "Sørg for, at hun får al sin Ejendom tilbage og alt, hvad hendes Jord har båret, siden den Dag hun forlod Landet!"
\par 7 Siden begav Elisa sig til Damaskus, hvor Kong Benbadad af Aram lå syg. Da Kongen fik at vide, at den Guds Mand var på Vej derhen,
\par 8 sagde han til Hazael: "Tag en Gave med, gå den Guds Mand i Møde og rådspørg HERREN gennem ham, om jeg kommer mig af min Sygdom!"
\par 9 Da gik Hazael ham i Møde; han tog en Gave med af alskens Kostbarheder, som fandfes i Damaskus, fyrretyve Kamelladninger, og trådte frem for ham og sagde: "Din Søn Benhadad, Arams Konge sender mig til dig og lader spørge: Kommer jeg mig af min Sygdom?"
\par 10 Elisa svarede: "Gå hen og sig ham: Du kommer dig!" Men HERREN har ladet mig skue, at han skal dø!"
\par 11 Og han stirrede stift frem for sig og var ude af sig selv af Rædsel. Så brast den Guds Mand i Gråd,
\par 12 og Hazael sagde: "Hvorfor græder min Herre?" Han svarede: "Fordi jeg ved, hvilke Ulykkkr du skal bringe over Israeliterne! Deres Fæstninger skal du stikke i Brand, deres unge Mænd skal du hugge ned med Sværdet, deres spæde Børn skal du knuse, og på deres frugtsommelige Kvinder skal du rive Livet op!"
\par 13 Da sagde Hazael: "Hvad er din Træl, den Hund, at han skal kunne gøre slige store Ting!" Elisa svarede: "HERREN har ladet mig skue dig som Konge over Aram!"
\par 14 Derpå forlod han Elisa og kom til sin Herre; og han spurgte ham: "Hvad sagde Elisa til dig?" Han svarede: "Han sagde: Du kommer dig!"
\par 15 Men næste Dag tog han et Klæde, dyppede det i Vand og bredte det over Ansigtet på Kongen, og det blev hans Død. Og Hazael blev Konge i hans Sted.
\par 16 I Akabs Søns, Kong Joram af Israels, femte Regeringsår blev Joram, Josafats Søn, Konge over Juda.
\par 17 Han var to og tredive År gammel, da han blev Konge, og han herskede otte År i Jerusalem.
\par 18 Han vandrede i Israels Kongers Spor ligesom Akabs Hus, thi han havde en Datter af Akab til Hustru, og han gjorde, hvad der var ondt i HERRENs Øjne.
\par 19 Dog vilde HERREN ikke tilintetgøre Juda for sin Tjener Davids Skyld efter det Løfte, han havde givet ham, at han altid skulde have en Lampe for hans Åsyn.
\par 20 I hans Dage rev Edomiterne sig løs fra Judas Overherredømme og valgte sig en Konge.
\par 21 Da drog Joram over til Za'ir med alle sine Stridsvogne. Og han stod op om Natten, og sammen med Vognstyrerne slog han sig igennem Edoms Rækker, der havde omringet ham, hvorpå Folket flygtede tilbage hver til sit.
\par 22 Således rev Edom sig løs fra Judas Overherredømme, og således er det den Dag i Dag. På samme Tid rev også Libna sig løs.
\par 23 Hvad der ellers er at fortælle om Joram, alt, hvad han udførte, står optegnet i Judas Kongers Krønike.
\par 24 Så lagde Joram sig til Hvile hos sine Fædre og blev jordet hos sine Fædre i Davidsbyen; og hans Søn Ahazja blev Konge i hans Sted.
\par 25 I Akabs Søns, Kong Joram af Israels, tolvte Regeringsår blev Ahazja, Jorams Søn, Konge over Juda.
\par 26 Ahazja var to og tyve År gammel, da han blev Konge, og han herskede eet År i Jerusalem. Hans Moder hed Atalja og var Datter af Kong Omri af Israel.
\par 27 Han vandrede i Akabs Hus's Spor og gjorde, hvad der var ondt i HERRENs Øjne, ligesom Akabs Hus, thi han var besvogret med Akabs Hus.
\par 28 Sammen med Joram, Akabs Søn, drog han i Krig mod Kong Hazael af Aram ved Ramot i Gilead. Men Aramæerne sårede Joram.
\par 29 Så vendte Kong Joram tilbage for i Jizre'el at søge Helbredelse for de Sår, Aramæerne havde tilføjet ham ved Ramot, da han kæmpede med Kong Hazael af Aram; og Jorams Søn, Kong Ahazja af Juda, drog ned for at se til Joram, Akabs Søn, i Jizre'el, fordi han lå syg.

\chapter{9}

\par 1 Profeten Elisa kaldte en af Profetsønnerne til sig og sagde til ham: "Omgjord dine Lænder, tag denne Flaske Olie med og drag til Ramot i Gilead.
\par 2 Når du kommer derhen, opsøg så Jehu, Nimsjis Søn Josjafats Søn; gå hen og få ham til at stå op fra sine Fæller og før ham ind i det inderste Hammer;
\par 3 tag så Olieflasken og gyd Olien ud over hans Hoved med de Ord: Så siger HERREN: Jeg salver dig til Konge over Israel! Derefter skal du lukke Døren op og flygte ufortøvet!"
\par 4 Den unge Mand, Profetens Tjener, drog så til Ramot i Gilead;
\par 5 og da han kom derhen, traf han Hærførerne siddende sammen. Han sagde da: "Jeg har et Ærinde til dig, Hærfører!" Jehu spurgte: "Til hvem af os?" Han svarede: "Til dig, Hærfører!"
\par 6 Så rejste han sig og gik ind i Huset; der gød han Olien ud over hans Hoved og sagde til ham: "Så siger HERREN, Israels Gud: Jeg salver dig til Konge over HERRENs Folk, over Israel!
\par 7 Du skal hugge din Herre Akabs Hus ned, så jeg får Hævn over Jesabel for mine Tjenere Profeternes og alle HERRENsjeneres Blod.
\par 8 Hele Akabs Hus skal omkomme, jeg vil udrydde hvert mandligt Væsen, hver og en af Akabs Slægt i Israel;
\par 9 jeg vil handle med Akabs Hus som med Jeroboams, Nebats Søns, og Ba'sjas, Ahijas Søns, Hus.
\par 10 Og Jesabel skal Hundene æde på Jizre'els Mark, og ingen skal jorde hende!" Derpå lukkede han Døren op og flygtede.
\par 11 Da Jebu kom ud til sin Herres Folk, spurgte de ham: "Hvorledes står det til? Hvad vilde den gale Mand hos dig?" Han svarede: "I kender jo den Mand og hans Snak!"
\par 12 Men de sagde: "Udflugter! Sig os det nu!" Da sagde han: "Således sagde han til mig: Så siger HERREN: Jeg salver dig til Konge over Israel!"
\par 13 Øjeblikkelig tog de da hver sin Kappe og lagde under ham på selve Trappen, og de stødte i Hornet og udråbte Jehu til Konge.
\par 14 Således stiftede Jehu, Nimsjis Søn Josjafats Søn, en Sammensværgelse mod Joram. Joram havde med hele Israel forsvaret Ramot i Gilead mod Kong Hazael af Aram;
\par 15 men Kong Joram var vendt tilbage for i Jizre'el at søge Helbredelse for de Sår, Aramæerne havde tilføjet ham, da han kæmpede med Kong Hazael af Aram. Da sagde Jehu: "Vil I som jeg, så lad ikke en eneste slippe ud af Byen og bringe Bud til Jizre'el."
\par 16 Derpå steg Jehu til Vogns og kørte til Jizre'el; thi der lå Joram syg, og Kong Ahazja af Juda var rejst ned for at se til ham.
\par 17 Da Vægteren, som stod på Tårnet i Jizre'el, så Støvskyen efter Jehu, sagde han: "Jeg ser en Støvsky!" Da sagde Joram: "Tag en Rytter og send ham ud imod dem, for at han kan spørge, om de kommer med Fred!"
\par 18 Så red Rytteren ham i Møde og sagde: "Således siger Kongen: Kommer du med Fred?" Jehu svarede: "Hvad vedkommer det dig, om det er med Fred? Omkring, følg mig!" Vægteren meldte: "Sendebudet har nået dem, men han kommer ikke tilbage!"
\par 19 Så sendte han en anden Rytter ud; og da han var kommet hen til dem, sagde han: "Således siger Kongen: Kommer du med Fred?" Jehu svarede: "Hvad vedkommer det dig, om jeg kommer med Fred? Omkring, følg mig!"
\par 20 Vægteren tneldte: "Han har nået dem, men han kommet ikke tilbage. Og de har en Fart på, som var det Jebu, Nimsjis Søn, thi han farer af Sted som rasende."
\par 21 Da sagde Joram: "Spænd for!" Og da der var spændt for, kørte Kong Joram af Israel og Kong Ahazja af Juda ud hver i sin Vogn. De kørte Jehu i Møde og traf ham ved Jizre'eliten Nabots Mark.
\par 22 Da Joram fik Øje på Jehu, spurgte han: "Kommer du med Fred, Jehu?" Men han svarede: "Hvad! Skulde jeg komme med Fred, så længe det ikke har Ende med din Moder Jesabels Bolen og hendes mange Trolddomskunster!"
\par 23 Da drejede Joram omkring og flygtede, idet han roabte tiIAhazja: "Svig, Ahazja!"
\par 24 Men Jehu greb sin Bue og skød Joram i Ryggen, så at Pilen gik igennem Hjertet, og han sank sammen i Vognen;
\par 25 og Jehu sagde til sin Høvedsmand Bidkar: "Tag og kast ham hen på Jizre'eliten Nabots Mark, thi det rinder mig i Hu, hvorledes jeg og du kørte sammen bag efter hans Fader Akab, dengang HERREN fremsatte dette Udsagn imod ham:
\par 26 Sandelig, Nabots og hans Sønners Blod så jeg i Går, lyder det fra HERREN, og jeg bringer Gengældelse over dig her på denne Mark, lyder det fra HERREN! Tag derfor og kast ham hen på Marken efter HERRENs Ord!"
\par 27 Da Kong Ahazja at Juda så det, flygtede han ad Vejen til BetHagan; men Jehu satte efter ham og råbte: "Også ham!" Og i Gurpasset, i Nærheden afJibleam, skød de ham ned i Vognen. Han undslap til Megiddo, men der døde han.
\par 28 Hans Folk førte ham til Jerusalem og jordede ham i hans Grav hos hans Fædre i Davidsbyen.
\par 29 I Akabs Søn Jorams ellevte Regerinsgår blev Ahazja Konge over Juda.
\par 30 Jehu kom nu til Jizre'el. Så snart Jesabel hørte det, sminkede hun sine Øjne og smykkede sit Hoved og bøjede sig ud af Vinduet;
\par 31 og da Jehu kørte ind igennem Porten, råbte hun: "Kommer du med Fred, Zimri Kongemorder?"
\par 32 Men han så op til Vinduet og sagde: "Hvem holder med mig? Hvem?" Så var der et Par Hofmænd, som så ud efter ham,
\par 33 og han råbte: "Styrt hende ned!" Så styrtede de hende ned. og Blodet sprøjtede op på Muren og på Hestene, og de trådte hende ned.
\par 34 Derpå gik han ind og spiste og drak. Så sagde han: "Tag jer af hende, den forbandede, og jord hende, hun var jo dog en Kongedatter!"
\par 35 Men da de gik ud for at jorde hende, fandt de ikke andet af hende end Hjerneskallen, Fødderne og Hænderne.
\par 36 Og de kom tilbage og meldte ham det; da sagde han: "Det er det Ord, HERREN talede ved sin Tjener Tisjbiten Elias: På Jizre'els Mark skal Hundene æde Jesabels Legeme!
\par 37 og Jesabels Lig skal blive som Gødning på Ageren på Jizre'els Mark, så ingen kan sige: Dette er Jesabel!"

\chapter{10}

\par 1 Der var i Samaria halvfjerdsindstyve Sønner af Akab. Jehu skrev nu Breve og sendte dem til Samaria til Byens Øverster, deÆldste og Akabs Sønners Fosterfædre. Deri stod:
\par 2 "I har jo eders Herres Sønner hos eder og råder over Stridsvognene og Hestene, Fæstningerne og Våbenforrådene. Når nu dette Brev kommer eder i Hænde,
\par 3 udvælg så den bedste og dygtigste af eders Herres Sønner, sæt ham på hans Faders Trone og kæmp for eders Herres Hus!"
\par 4 Men de grebes af stor Forfærdelse og sagde: "Se, de to Konger kunde ikke stå sig imod ham, hvor skal vi så kunne det?"
\par 5 Derfor sendte Paladsets og Byens øverste Befalingsmænd, de Ældste og Fosterfædrene det Bud til Jehu: "Vi er dine Trælle, og alt, hvad du kræver af os, vil vi gøre. Vi vil ikke gøre nogen til Konge; gør, hvad du finder for godt!"
\par 6 Da skrev han et nyt Brev til dem, og der stod: "Dersom I holder med mig og vil høre mig, tag så eders Herres Sønners Hoveder og kom i Morgen ved denne Tid til mig i Jizre'el!" Kongesønnerne, halvfjerdsindstyve Mænd, var nemlig hos Byens Stormænd, som var deres Fosterfædre.
\par 7 Da Brevet kom til dem, tog de Kongesønnerne og dræbte dem, halvfjerdsindstyve Mænd, lagde deres Hoveder i Kurve og sendte dem til Jehu i Jizre'el.
\par 8 Da Budet kom og meldte: "Kongesønnernes Hoveder er bragt hid!" sagde han: "Læg dem i to Bunker foran Porten til i Morgen!"
\par 9 Næste Morgen gik han ud, trådte frem og sagde til alt Folket: "I er uden Skyld; det er mig, der har stiftet en Sammensværgelse mod min Herre og dræbt ham - men hvem har dræbt alle disse?
\par 10 Kend nu, at intet af det Ord, HERREN talede mod Akabs Hus, var faldet til Jorden, men HERREN har gjort, hvad han talede ved sin Tjener Elias!"
\par 11 Derpå lod Jehu alle dem, der var tilbage af Akabs Hus i Jizre'el, dræbe, alle hans Stormænd, Venner og Præster, så at ikke en eneste blev tilbage og slap bort.
\par 12 Så brød han op og drog ad Samaria til. Da han kom til Bet Eked-Haro'im ved Vejen,
\par 13 mødte han Kong Ahazja af Judas Brødre. Han spurgte dem: "Hvem er I?" De svarede: "Vi er Ahazjas Brødre, og vi drager herned for at hilse på Kongens og Kongemoderens Sønner."
\par 14 Da sagde han: "Grib dem levende!" Så greb de dem levende, og han lod dem dræbe ved Bet Ekeds Brønd, to og fyrretyve Mænd; ikke een lod han blive tilbage.
\par 15 Da han drog videre, traf han Jonadab, Rekabs Søn, der kom ham i Møde, og han hilste på ham og spurgte: "Er du af Hjertet oprigtig mod mig som jeg mod dig?" Jonadab svarede: "Ja, jeg er!" Da sagde Jehu: "Så giv mig din Hånd!" Han gav ham da Hånden, og Jehu tog ham op til sig i Vognen
\par 16 og sagde: "Følg mig og se min Nidkærhed for HERREN!" Og han tog ham med i Vognen,
\par 17 Så drog han til Samaria og lod alle, der var tilbage af Akabs Slægt i Samaria, dræbe, så at den blev fuldstændig udryddet efter det Ord, HERREN havde talet til Elias.
\par 18 Derefter kaldte Jebu hele Folket sammen og sagde til dem: "Akab dyrkede Ba'al lidt, Jehu vil dyrke ham mere!
\par 19 Kald derfor alle Ba'als Profeter, alle, der dyrker ham, og alle hans Præster hid til mig, ikke een må udeblive, thi jeg har et stort Slagtoffer for til Ære for Ba'al; enhver, der udebliver, skal bøde med Livet!" Men det var en Fælde, Jehu stillede, for at udrydde Ba'alsdyrkerne.
\par 20 Derpå sagde Jehu: "Helliger en festlig Samling til Ære for Ba'al!" Og de udråbte en festlig Samling.
\par 21 Og Jehu sendte Bud rundt i hele Israel, og alle Ba'alsdyrkerne uden Undtagelse indfandt sig; de begav sig til Ba'als Hus, og det blev fuldt fra Ende til anden.
\par 22 Så sagde han til Opsynsman den over Klædekammeret: "Tag en Klædning frem til hver af Ba'als dyrkerne!" Og han tog Klædningerne frem til dem.
\par 23 Så gik Jehu og Jonadab, Rekabs Søn, ind i Ba'als Hus; og han sagde til Ba'alsdyrkeme: "Se nu godt efter; at der ikke her iblandt eder findes nogen, som dyrker HERREN, men kun Ba'alsdyrkere!"
\par 24 Derpå gik han ind for at ofre Slagtofre og Brændofre. Men Jehu havde opstillet firsindstyve Mand udenfor og sagt: "Den, der lader nogen af de Mænd undslippe, som jeg overgiver i eders Hænder, skal bøde Liv for Liv!"
\par 25 Da han så var færdig med at ofre Brændofferet, sagde han til Livvagten og Høvedsmændene: "Gå nu ind og hug dem ned! Ikke een må slippe bort!" Og de huggede dem ned med Sværdet, og Livvagten og Høvedsmændene slængte dem bort; så gik de ind i Ba'alshusets inderste Rum,
\par 26 bragte Ba'alshusets Asjerastøtte ud og opbrændte den;
\par 27 og de nedbrød Ba'als Stenstøtte, rev Ba'als Hus ned og gjorde det til Nødtørftssteder, og, de er der den Dag i Dag.
\par 28 Således udryddede Jehu Ba'al af Israel.
\par 29 Men fra de Synder, Jeroboam, Nebats Søn, havde forledt Israel til, Guldkalvene i Betel og Dan, veg Jehu ikke.
\par 30 Og HERREN sagde til Jehu: "Fordi du har handlet vel og gjort, hvad der er ret i mine Øjne, og handlet med Akabs Hus ganske efter mit Sind, skal dine Sønner sidde på IsraelsTrone indtil fjerde Led!"
\par 31 Men Jehu tog ikke Vare på at følge HERRENs, Israels Guds, Lov af bele sit Hjerte; han veg ikke fra de Synder, Jeroboam havde forledt Israel til.
\par 32 På den Tid begyndte HERREN at rive Stykker fra Israel, og Hazael slog Israel i alle dets Grænseegne,
\par 33 Øst for Jordan, hele Gilead, Gaditernes, Rubeniternes og Manassiternes Land fra Aroer ved Amonflodens Bred, både Gilead og Basan.
\par 34 Hvad der ellers er at fortælle om Jehu, alt, hvad han gjorde, og alle hans Heltegerninger, står jo optegnet i Israels Kongers Krønike.
\par 35 Så lagde Jehu sig til Hvile hos sine Fædre, og man jordede ham i Samaria; og hans Søn Joahaz blev Konge i hans Sted.
\par 36 Den Tid, Jehu herskede over Israel, udgjorde otte og tyve År.

\chapter{11}

\par 1 Da Atalja, Ahazjas Moder, fik at vide, at hendes Søn var død, tog hun sig for at udrydde hele den kongelige Slægt.
\par 2 Men Kong Jorams Datter Josjeba, Ahazjas Søster, tog Ahazjas Søn Joas og fik ham hemmeligt af Vejen, så han ikke var imellem Kongesønnerne, der blev dræbt, og hun gemte ham og hans, Amme i Sengekammeret og holdt ham skjult for Atalja, så han ikke blev dræbt;
\par 3 og han var i seks År skjult hos Josjebai HERRENs Hus, medens Atalja herskede i Landet.
\par 4 Men i det syvende År lod Jojada Hundredførerne for Harernel og Livvagten hente ind til sig i HERRENs Hus; og efter at have sluttet Pagt med dem og taget dem i Ed i HERRENs Hus fremstillede han Kongesønnen for dem.
\par 5 Derpå bød han dem og sagde: "Således skal I gøre: Den Tredje del af eder, der om Sabbaten rykker ind for at overtage Vagten i Kongens Palads,
\par 6 og de to Afdelinger af eder, som har Vagten i Kongens Palads den ene Tredjedel ved Surporten, den anden ved Porten bag Livvagten
\par 7 og som begge om Sabbaten rykker ud for at overtage Vagten i HERRENs Hus,
\par 8 I skal alle med Våben i Hånd slutte Kreds om Kongen, og enhver, der nærmer sig Rækkerne, skal dræbes. Således skal I være om Kongen, når han går ud, og når han går ind!"
\par 9 Hundredførerne gjorde alt hvad Præsten Jojada havde påbudt, idet de tog hver sine Folk, både dem, der rykkede ud, og dem, der ryk kede ind om Sabbaten, og kom til Præsten Jojada.
\par 10 Og Præsten gav Hundredfø rerne Spydene og Skjoldene, som havde tilhørt Kong David og var i HERRENs Hus.
\par 11 Livvagten stillede sig, alle med Våben i Hånd, fra Templets Syd side til Nordsiden, hen til Alteret og derfra igen hen til Templet, rundt om Kongen.
\par 12 Så førte han Kongesønnen ud og satte Kronen og Armspangene på ham; derefter udråbte de ham til Konge og salvede ham; og de klappede i Hænderne og råbte: "Kongen leve!"
\par 13 Da Atalja hørte Larmen af Folket, gik hun hen til Folket i HERRENs Hus,
\par 14 og der så hun Kongen stå ved Søjlen, som Skik var, og Øversterne og Trompetblæserne ved Siden af, medens alt Folket fra Landet jublede og blæste i Trompeterne. Da sønderrev Atalja sine Klæder og råbte: "Forræderi, Forræderi!"
\par 15 Men præsten Jojada bød Hundredførerne, Hærens Befalingsmænd: "Før hende uden for Forgårdene og hug enhver ned, der følger hende!" Præsten sagde nemlig: "Hun skal ikke dræbes i HERRENs Hus!"
\par 16 Så greb de hende, og da hun ad Hesteindgangen var kommet til Kongens Palads, blev hun dræbt der.
\par 17 Men Jojada sluttede Pagt mellem HERREN og Folket og Kongen om, at de skulde være HERRENs Folk, ligeledes mellem Kongen og Folket.
\par 18 Og alt Folket fra Landet begav sig til Ba'als Hus og nedbrød det; Altrene og Billederne ødelagde de i Bund og Grund, og Ba'als Præst Mattan dræbte de foran Altrene. Derpå satte Præsten Vagtposter ved HERRENs Hus;
\par 19 og han tog Hundredførerne, Karerne og Livvagten, desuden alt Folket fra Landet med sig, og de førte Kongen ned fra HERRENs Hus. gik igennem Livvagtens Port til Kongens Palads, og han satte sig på Kongetronen.
\par 20 Da glædede alt Folket fra Landet sig, ogByen holdt sig rolig. Men Atalja huggede de ned i Kongens Palads.
\par 21 Joas var syv År gammel, da han blev Konge.

\chapter{12}

\par 1 I Jehus syvende Regeringsår blev Joas Konge, og han herskede fyrretyveÅr i Jerusalem. Hans Moder hed Zibja og var fra Be'ersjeba.
\par 2 Joas gjorde hele sit Liv, hvad der var ret i HERRENs Øjne, idet Præsten Jojada vejledede ham.
\par 3 Hun forsvandt Offerhøjene ikke, men Folket blev ved med at ofre og tænde Offerild på Højene.
\par 4 Joas sagde til Præsterne: "Alle Penge, der indkommer i HERRENs Hus som Helliggaver, de Penge, det pålægges at udrede efter Vurdering - de Penge, Personer vurderes til - og alle Penge, man efter Hjertets Tilskyndelse bringer til HERRENs Hus,
\par 5 skal Præsterne tage imod, hver af sine Kendinge, og for dem skal de istandsætte de brøstfældige Steder på Templet, alle brøstfældige Steder, som findes."
\par 6 Men i Kong Joas's tre og tyvende Regeringsår havde Præsterne endnu ikke istandsat de brøstfældige Steder på Templet.
\par 7 Da lod Kong Joas Præsten Jojada og de andre Præster kalde og sagde til dem: "Hvorfor istandsætter I ikke de brøstfældige Steder på Templet? Nu må I ikke mere tage mod Penge af eders Kendinge, men I skal afgive Pengene til Istandsættelse af de brøstfældige Steder på Templet!"
\par 8 Og Præsterne gik ind på ikke mere at modtage Penge af Folket, mod at de blev fri for at istandsætte de brøstfældige Steder på Templet.
\par 9 Præsten Jojada tog så en Kiste, borede Hul i Låget og satte den ved Stenstøtten til højre for Indgangen til HERRENs Hus, og der lagde Præsterne, der holdt Vagt ved Dørtærskelen, alle de Penge, der indkom i HERRENs Hus.
\par 10 Og når de så, at der var mange Penge i Kisten, kom Kongens Skriver og Ypperstepræsten op og bandt Pengene, som fandtes i HERRENs Hus, sammen og talte dem.
\par 11 Derpå gav man de afvejede Penge til dem, der stod for Arbejdet, dem, der havde Tilsyn med HERRENs Hus, og de udbetalte dem til Tømrerne og Bygningsmændene, der arbejdede på HERRENs Hus,
\par 12 til Murerne og Stenhuggerne, eller brugte dem til Indkøb af Træ og tilhugne Sten til Istandsættelse af de brøstfældige Steder på HERRENs Hus og til at dække alle Udgifter ved Templets Istandsættelse.
\par 13 Derimod blev der for de Penge, der indkom i HERRENs Hus, hverken lavet Sølvfade eller Knive, Skåle, Trompeter eller nogen som helst anden Ting af Sølv eller Guld til HERRENs Hus;
\par 14 men Pengene blev givet til dem, der stod for Arbejdet, og de brugte dem til Istandsættelsen af HERRENs Hus.
\par 15 Og man holdt ikke Regnskab med de Mænd, hvem Pengene overlodes til Udbetaling til Arbejderne, men de handlede på Tro og Love.
\par 16 Men Skyldoffer- og Syndofferpengene blev ikke bragt til HERRENs Hus; de tilfaldt Præsterne.
\par 17 På den Tid drog Kong Hazael af Aram op og belejrede og indtog Gat. Da han truede med at drage mod Jerusalem,
\par 18 tog Kong Joas af Juda alle de Helliggaver, som hans Fædre, Judas Konger Josafat, Joram og Ahazja havde helliget, sine egne Helliggaver og alt det Guld, der fandtes i Skatkamrene i HERRENs Hus og Kongens Palads, og sendte det til Kong Hazael af Aram. Så opgav han Angrebet på Jerusalem og drog bort.
\par 19 Hvad der ellers er at fortælle om Joas, alt, hvad han udførte, står optegnet i Judas Kongers Krønike.
\par 20 Men Joas's Hoffolk rejste sig og stiftede en Sammensværgelse og dræbte ham, engang han gik ned til Millos Hus.
\par 21 Det var hans Hoffolk Jozakar, Sjimats Søn, og Jozabad, Sjomers Søn, der slog ham ibjel. Og man jordede ham hos hans Fædre i Davidsbyen; og hans Søn Amazjablev Konge i hans Sted.

\chapter{13}

\par 1 I Ahazjas Søns, Kong Joas af Judas, tre og tyvende Regeringsår blev Joahaz, Jehus Søn, Konge over Israel, og han herskede sytten År i Samaria.
\par 2 Han gjorde, hvad der var ondt i HERRNs Øjne, og vandrede i de Synder, Jeroboam, Nebats Søn, havde forledt Israel til; fra dem veg han ikke.
\par 3 Da blussede HERRENs Vrede op mod Israel, og han gav dem til Stadighed i Kong Hazael af Arams og hans Søn Benhadads Hånd.
\par 4 Men Joahaz bad HERREN om Nåde, og HERREN bønhørte ham, fordi han så Israels Trængsel; thi Arams Konge bragte Trængsel over dem;
\par 5 og HERREN gav Israel en Befrier, som friede dem af Arams Hånd; så boede Israeliterne i deres Telte som før.
\par 6 Dog veg de ikke fra de Synder, Jeroboams Hus havde forledt Israel til, men vandrede i dem; også Asjerastøtten blev stående i Samaria.
\par 7 Thi Arams Konge levnede,ikke Joahaz flere Krigsfolk end halvtredsindstyve Ryttere, ti Stridsvogne og ti Tusinde Mand Fodfolk, men han tilintetgjorde dem og knuste dem til Støv.
\par 8 Hvad der ellers er at fortælle om Joahaz, alt, hvad han udførte, og hans Heltegerninger står jo optegnet i Israels Kongers Krønike.
\par 9 Så lagde Joahaz sig til Hvile hos sine Fædre, og man jordede ham i Samaria; og hans Søn Joas blev Konge i hans Sted.
\par 10 I Kong Joas af Judas syv og tredivte Regeringsår blev Joas. Joahaz's Søn, konge over Israel, og han herskede seksten År i Samaria.
\par 11 Han gjorde, hvad der var ondt i HERRENs Øjne, og veg ikke fra nogen af de Synder, Jeroboam, Nebats Søn, havde forledt Israel til, men vandrede i dem.
\par 12 Hvad der ellers er at fortælle om Joas, alt, hvad han udførte, og hans Heltegerninger, hvorledes han førte Krig med Kong Amazja af Juda, står jo optegnet i Israels Kongers Krønike.
\par 13 Så lagde Joas sig til Hvile hos sine Fædre; og Jeroboam satte sig på hans Trone. Joas blev jordet i Samaria hos Israels Konger.
\par 14 Da Elisa blev ramt af den Sygdom, han døde af, kom Kong Joas af Israel ned til ham, bøjede sig grædende over ham og sagde: "Min Fader, min Fader, du Israels Vogne. og Ryttere!"
\par 15 Men Elisa sagde til ham: "bring Bue og Pile!" Og han bragte ham Bue og Pile.
\par 16 Da sagde han til Israels Konge: "Læg din Hånd på Buen!" Og da han gjorde det, lagde Elisa sine Hænder på Kongens
\par 17 og sagde: "Luk Vinduet op mod Øst!" Og da han havde gjort det, sagde Elisa: "Skyd!" Og han skød. Da sagde Elisa: "En Sejrspil fra HERREN, en Sejrspil mod Aram! Du skal tilføje Aram et afgørende Nederlag ved Afek!"
\par 18 Derpå sagde han: "Tag Pilene!" Og han tog dem. Da sagde han til Israels Konge: "Slå på Jorden!" Og han slog tre Gange, men holdt så op.
\par 19 Da vrededes den Guds Mand på ham og sagde: "Du burde have slået fem-seks Gange, så skulde du have tilføjet Aram et afgørende Nederlag, men nu skal du kun slå Aram tre Gange!"
\par 20 Så døde Elisa, og de jordede ham. År efter År trængte moabitiske Strejfskarer ind i Landet;
\par 21 og da nogle Israeliter engang fik Øje på en sådan Skare, netop som de var ved at jorde en Mand, kastede de Manden i Elisas Grav og løb deres Vej. Men da Manden kom i Berøring med Elisas Ben, blev han levende og rejste sig op.
\par 22 Kong Hazael af Aram trængle Israel hårdt, så længe Joabaz levede.
\par 23 Men HERREN forbarmede sig og ynkedes over dem og vendte sig til dem på Grund af sin Pagt med Abraham, Isak og Jakob; han vilde ikke tilintetgøre dem og havde endnu ikke stødt dem bort fra sit Åsyn.
\par 24 Men da Kong Hazael af Aram døde, og hans Søn Benhadad blev Konge i hans Sted,
\par 25 tog Joas, Joahaz's Søn, de Byer tilbage fra Benhadad, Hazaels Søn, som han i Krigen havde frataget hans Fader Joahaz. Tre Gange slog Joas ham og tog de israelitiske Byer tilbage.

\chapter{14}

\par 1 I Joahaz' Søns, Kong Joas af Israel, andet Regeringsår blev Amazja, Joas's Søn, Konge over Juda.
\par 2 Han var fem og tyve År gammel, da han blev Konge, og han herskede ni og tyve År i Jerusalem. Hans Moder hed Jehoaddan og var fra Jerusalem.
\par 3 Han gjorde, hvad der var ret i HERRENs Øjne, om end ikke som hans Fader David; han handlede ganske som sin Fader Joas.
\par 4 Hun forsvandt Offerhøjene ikke. men Folket blev ved med at ofre og tænde Offerild på Højene.
\par 5 Da han havde sikret sig Magten, lod han dem af sine Folk dræbe, der havde dræbt hans Fader Kongen;
\par 6 men Mordernes Børn lod han ikke ihjelslå, i Henhold til hvad der står skrevet i Moses's Lovbog, hvor HERREN byder: "Fædre skal ikke lide Døden for Børns Skyld, og Børn skal ikke lide Døden for Fædres Skyld. Men enhver skal lide Døden for sin egen Synd."
\par 7 Det var ham, der slog Edom i Saltdalen, 10.000 Mand, og indtog Sela, og han kaldte det Jokte'el, som det hedder den Dag i Dag.
\par 8 Ved den Tid sendte Amazja Sendebud til Jebus Søn Joahaz's Søn, Kong Joas af Israel, og lod sige: "Kom, lad os se hinanden under Øjne!"
\par 9 Men Kong Joas af Israel sendte Kong Amazja af Juda det Svar: "Tidselen på Libanon sendte engang det Bud til Cederen på Libanon: Giv min Søn din Datter til Ægte! Men Libanons vilde byr løb hen over Tidselen og trampede den ned.
\par 10 Du har slået Edom, og det har gjort dig overmodig; lad dig nu nøje med den Ære og bliv, hvor du er! Hvorfor vil du udfordre Ulykken og udsætte både dig selv og Juda for Fald?"
\par 11 Men Amazja vilde intet høre. Så drog Kong Joas af Israel ud, og han og Kong Amazja af Juda så hinanden under Øjne ved Bet Sjemesj i Juda;
\par 12 Juda blev slået af Israel, og de flygtede hver til sit.
\par 13 Men Kong Joas af Israel tog Ahazjas Søn Joas's Søn. Kong Amazja af Juda, til Fange ved BetSjemesj og førte ham til Jerusalem. Derpå nedrev han Jerusalems Mur på en Strækning af 400 Alen, fra Efraimsporten til Hjørneporten;
\par 14 og han tog alt det Guld og Sølv og alle de Kar, der fandtes i HERRENs Hus og i Skatkammeret i Kongens Palads; desuden tog han Gidsler og vendte så tilbage til Samaria.
\par 15 Hvad der ellers er at fortælle om Joas, alt, hvad han udførte, og alle hans Heltegerninger, og hvorledes han førte Krig med Kong Amazja af Juda, står jo optegoet i Israels Kongers Krønike.
\par 16 Så lagde Joas sig til Hvile hos sine Fædre og blev jordet i Samaria hos Israels Konger; og hans Søn Jeroboam blev Konge i hans Sted.
\par 17 Joas's Søn, Kong Amazja af Juda, levede endnu femten År, efter at Joahaz's Søn, Kong Joas af Israel, var død.
\par 18 Hvad der ellers er at fortælle om Amazja, står jo optegnet i Judas Kongers Krønike.
\par 19 Da der stiftedes en Sammensværgelse mod ham i Jerusalem, flygtede han til Lakisj; men der blev sendt Folk efter ham til Lakisj, og de dræbte ham der.
\par 20 Så løftede man ham op på Heste, og han blev jordet i Jerusalem hos sine Fædre i Davidsbyen.
\par 21 Hele Folket i Juda tog så Azarja, der dengang var seksten År gammel, og gjorde ham til Konge i hans Fader Amazjas Sted.
\par 22 Det var ham, der befæstede Elat og atter forenede det med Juda, efter at Kongen havde lagt sig til Hvile hos sine Fædre.
\par 23 I Joas's Søns, Kong Amazja af Judas, femtende Regeringsår ble Jeroboam, Joas's Søn, Konge over Israel, og han herskede een og fyrretyve År i Samaria.
\par 24 Han gjorde, hvad der var ondt i HERRENs Øjne, og veg ikke fra nogen af de Synder, Jeroboam, Nebats Søn, havde forledt Israel til.
\par 25 Han tog Israels Landområde tilbage fra Egnen hen imod Hamat og til Arabasøen, efter det Ord, HERREN, Israels Gud, havde talet ved sin Tjener, Profeten Jonas, Amittajs Søn, fra Gat-Hefer.
\par 26 Thi HERREN havde set Israels bitre Kvide, hvorledes de reves bort alle som een, fordi Israel ikke havde nogen Hjælper;
\par 27 og HERREN havde ikke talet om, at han vilde udslette Israels Navn under Himmelen, derfor frelste han dem ved Jeroboam, Joas's Søn.
\par 28 Hvad der ellers er at fortælle om Jeroboam, alt, hvad han udførte, og hans Heltegerninger, hvorledes han førte Krig, og hvorledes han tog Damaskus og Hamat til bage til Israel, står jo optegnet i Israels Kongers Krønike.
\par 29 Så lagde Jeroboam sig til Hvile hos sine Fædre og blev jordet i Samaria hos Israels Konger; og hans Søn Zekarja blev Konge i hans Sted.

\chapter{15}

\par 1 I Kong Jeroboam af Israels syvogtyvende Regeringsår blev Azarja, Amazjas Søn, Konge over Juda.
\par 2 Han var seksten År gammel, da han blev Konge, og han herskede to og halvtredsindstyve År i Jerusalem. Hans Moder hed Jekolja og var fra Jerusalem.
\par 3 Han gjorde, hvad der var ret i HERRENs Øjne, ganske som hans Fader Amazja.
\par 4 Kun forsvandt Offerhøjeneikke, men Folket blev ved med at ofre og tænde Offerild på Højene,
\par 5 Men HERREN ramte Kongen, så han blev spedalsk til sin Dødedag; og han fik Lov at blive boende i sit Hus, medens Kongens Søn Jotam rådede i Paladset og dømte Folket i Landet.
\par 6 Hvad der ellers er at fortælle om Azarja, alt, hvad han udførte, står jo optegnet i Judas Kongers Krønike.
\par 7 Så jagde han sig til Hvile hos sine Fædre, og man jordede ham hos hans Fædre i Davidsbyen; og hans Søn Jotam blev Konge i hans Sted.
\par 8 I Kong Azarja af Judas otte og tredivte Regeringsår blev Zekarja, Jeroboams Søn, Konge over Israel, og han herskede seks Måneder i Samaria.
\par 9 Han gjorde, hvad der var ondt i HERRENs Øjne, ligesom hans Fædre, og han veg ikke fra de Synder, Jeroboam Nebats Søn, havde forledt Israel til.
\par 10 Men Sjallum, Jabesjs Søn, stiftede en Sammensværgelse modham, huggede ham ned og dræbte ham i Jibleam og blev Konge i hans Sted.
\par 11 Hvad der ellers er at fortælle om Zekarja, står optegnet i Israels Kongers Krønike.
\par 12 Således opfyldtes det Ord HERREN havde talet til Jehu, da han sagde: "Dine Sønner skal sidde på Israels Trone indtil fjerde Led." Således gik det.
\par 13 I Kong Uzzija af Judas ni og tredivte Regeringsår blev Sjallum, Jabesjs Søn, Konge, og han herskede en Måneds Tid i Samaria.
\par 14 Da drog Menahem, Gadis Søn, op fra Tirza til Samaria, og der huggede han Sjallum, Jabesjs Søn, ned og dræbte ham og blev Konge i hans Sted.
\par 15 Hvad der ellers er at fortælle om Sjallum og den Sammensværgelse, han stiftede, står optegnet i Israels Kongers Krønike.
\par 16 Fra Tirza hærgede Menahem ved den Tid Tappua og alt, hvad der var deri, og hele dets Område, fordi de ikke havde åbnet Portene for ham; derfor hærgede han det og lod Livet rive op på alle frugtsommelige Kvinder der.
\par 17 I Kong Azarja af Judas ni og tredivte Regeringsår blev Menahem, Gadis Søn, Konge over Israel, og han herskede ti År i Samaria.
\par 18 Han gjorde, hvad der var ondt i HERRENs Øjne, og veg ikke fra nogen af de Synder, Jeroboam, Nebats Søn, havde forledt Israel til. I hans Dage
\par 19 faldt Kong Pul af Assyrien ind i Landet. Men Menahem gav Pul 1000 Talenter Sølv, for at han skulde støtfe ham og sikre ham Magten;
\par 20 Menahem inddrev disse Penge hos Israel, hos alle de velhavende, halvtredsindstyve Sekel Sølv hos hver, for at udbetale dem til Assyrerkongen. Så vendte Assyrerkongen hjem og blev ikke længer der i Landet.
\par 21 Hvad der ellers er at fortælle om Menahem, alt, hvad han udførte, står jo optegnet i Israels Kongers Krønike.
\par 22 Så lagde Menahem sig til Hvile hos sine Fædre, og hans Søn Pekaja blev Konge i hans Sted.
\par 23 I Kong Azarja af Judas halvtredsindstyvende Regeringsår blev Pekaja, Menahems Søn, Konge over Israel, og han herskede to År i Samaria.
\par 24 Han gjorde, hvad der var ondt i HERRENs Øjne, og veg ikke fra de Synder, Jeroboam, Nebats Søn, havde forledt Israel til.
\par 25 Men hans Høvedsmand Peka, Remaljas Søn, stiftede en Sammensværgelse mod ham, og fulgt af halvtredsindstyve gileaditiske Mænd huggede han ham ned i Samada i Kongeborgen..., og efter at have dræbt ham blev han Konge i hans Sted.
\par 26 Hvad der ellers er at fortælle om Pekaja, alt, hvad han udførte, står optegnet i Israels Kongers Krønike.
\par 27 I Kong Azarja af Judas to og halvtredsindstyvende Regeringsår blev Peka, Remaljas Søn, Konge over Israel, og han herskede tyve År i Samaria.
\par 28 Han gjorde, hvad der var ondt i HERRENs Øjne, og veg ikke fra de Synder, Jeroboam, Nebats Søn, havde forledt Israel til.
\par 29 I Kong Peka af Israels Dage kom Assyrerkongen Tiglat-Pileser og indtog Ijjon, Abel-Bet-Ma'aka, Ianoa, Kedesj, Hazor, Gilead og Galilæa, hele Naftalis Land, og førte Indbyggerne bort til Assyrien.
\par 30 Men Hosea, Elas' Søn, stiftede en Sammensværgelse mod Peka, Remaljas Søn, huggede ham ned og dræbteham; og han blev Konge i hans Sted i Jotams, Uzzijas Søns, tyvende Regeringsår.
\par 31 Hvad der ellers er at fortælle om Peka, alt, hvad han udførte, står optegnet i Israels Kongers Krønike.
\par 32 I Remaljas Søns, Kong Peka af Israels, andet Regeringsår blev Jotam, Azarjas Søn, Konge over Juda.
\par 33 Han var fem og tyve År gammel, da han blev Konge, og han herskede seksten År i Jerusalem. Hans Moder hed Jerusja og var en Datter af Zadok.
\par 34 Han gjorde, hvad der var ret i HERRENs Øjne, ganske som hans Fader Uzzija.
\par 35 Kun forsvandt Offerhøjene ikke, men Folket blev ved med at ofre og tænde Offerild på Højene. Det var ham, der lod Øvreporten i HERRENs Hus opføre.
\par 36 Hvad der ellers er at fortælle om Jotam, alt, hvad han udførte, står jo optegnet i Judas Kongers Krønike.
\par 37 På den Tid begyndte HERREN at lade Kong Rezin af Aram og Peka, Remaljas Søn, angribe Juda.
\par 38 Så lagde Jotam sig til Hvile hos sine Fædre, og han blev jordet hos sine Fædre i sin Fader Davids By; og hans Søn Akaz blev Konge i hans Sted.

\chapter{16}

\par 1 I Pekas, Remaljas Søns, syttende Regeringsår blev Akaz, Jotams Søn, Konge over Juda.
\par 2 Akaz var tyve År gammel, da han blev Konge, og han herskede seksten År i Jerusalem. Han gjorde ikke, hvad der var ret i HERREN hans Guds Øjne, som hans Fader David,
\par 3 men vandrede i Israels Kongers Spor. Ja, han lod endog sin Søn gå igennem Ilden efter de Folks vederstyggelige Skik, som HERREN havde drevet bort foran Israeliterne.
\par 4 Han ofrede og tændte Offerild på Offerhøjene og de høje Steder og under alle grønne Træer.
\par 5 På den Tid drog Kong Aezin af Aram og Remaljas Søn, Kong Peka af Israel, op for at angribe Jerusalem; og de indesluttede Akaz, men var ikke stærke nok til at angribe.
\par 6 Ved den Lejlighed tog Edoms Konge; Elat tilbage til Edom; og efter at han havde jaget Judæerne ud af Elat, kom Edomiterne og bosatte sig der, og de bor der den Dag i Dag.
\par 7 Men Akaz sendte Sendebud til Assyrerkongen Tiglat-Pileser og lod sige: "Jeg er din Træl og din Søn! Kom op og frels mig fra Arams og Israels Konger, som angriber mig!"
\par 8 Tillige tog Akaz det Sølv og Guld, der fandtes i HERRENs Hus og Skat Kammeret i Kongens Palads, og sendte det som Gave til Assyrerkongen.
\par 9 Assyrerkongen opfyldte hans Ønske og drog op mod Damaskus og indtog det; Indbyggerne førte han bort til Kir, og Rezin lod han dræbe.
\par 10 Da Kong Akaz var draget op til Damaskus for at mødes med Assyrerkongen Tiglat-Pileser, så han Alteret i Damaskus, og Kong Akaz sendte Alterets Mål og en Tegning af det i alle Enkeltheder til Præsten Urija.
\par 11 Og Præsten Urija byggede Alteret; i nøje Overensstemmelse med den Vejledning, Kong Akaz havde sendt fra Damaskus, udførte Præsten Urija det, før Kong Akaz kom hjem fra Damaskus.
\par 12 Da Kongen kom hjem fra Damaskus og så Alteret, trådte han hen og steg op derpå;
\par 13 og han ofrede sit Brændoffer og Afgrødeoffer, udgød sit Drikoffer og sprængte Blodet af sine Takofre på Alteret.
\par 14 Men Kobberalteret, der stod for HERRENs Åsyn, fjernede han fra dets Plads foran Templet mellem Alteret og HERRENS Hus og flyttede det hen til Nordsiden af Alteret.
\par 15 Derpå bød Kong Akaz Præsten Urija: "På det store Alter skal du ofre Morgenbrændofrene og Aftenafgrødeofrene, Kongens Brændofre og Afgrødeofre og Brændofrene fra alt Landets Folk såvel som deres Afgrødeofre og Drikofre og sprænge alt Blodet fra Brændofrene og Slagtofrene derpå. Hvad der skal gøres ved Kobberalteret, vil jeg tænke over."
\par 16 Og Præsten Urija gjorde ganske som Kong Akaz bød.
\par 17 Fremdeles lod Kong Aka Mellemstykkerne bryde af Stellene og Bækkenerne tage ned af dem; ligeledes lod han Havet løfte ned fra Kobberokserne, som har det, og opstille på Stenfliser;
\par 18 den overdækkede Sabbatsgang, man havde bygget i Templet, og den kongelige Indgang udenfor lod han fjerne fra HERRENs Hus for Assyrerkongens Skyld.
\par 19 Hvad der ellers er at fortælle om Akaz, alt, hvad han udførte. står jo optegnet i Judas Kongers Krønike.
\par 20 Så lagde Akaz sig til Hvile hos sine Fædre og blev jordet hos sine Fædre i Davidsbyen; og hans Søn Ezekias blev Konge i hans Sted.

\chapter{17}

\par 1 I Kong Akaz af Judas tolvte Regeringsår blev Hosea, Elas Søn, Konge i Samaria over Israel, og han herskede ni År.
\par 2 Han gjorde, hvad der var ondt i HERRENs Øjne, dog ikke som de Konger i Israel, der var før ham.
\par 3 Mod ham drog Assyrerkongen Salmanassar op, og Hosea underkastede sig og svarede ham Skat.
\par 4 Men siden opdagede Assyrerkongen, at Hosea var ved at stifte en Sammensværgelse, idet han sendte Sendebud til Kong So af Ægypten og ikke mere svarede Assyrerkongen den årlige Skat. Så berøvede Assyrerkongen ham Friheden og lod ham kaste i Fængsel.
\par 5 Assyrerkongen drog op og besatte hele Landet; han rykkede frem mod Samaria og belejrede det i tre År;
\par 6 og i Hoseas niende Regeringsår indtog Assyrerkongen Samaria, bortførte Israel til Assyrien og lod dem bosætte sig i Hala, ved Habor, Gozans Flod, og i Mediens Byer.
\par 7 Således gik det, fordi Israeliterne syndede mod HERREN deres Gud, der havde ført dem op fra Ægypten og udfriet dem af Ægypterkongen Faraos Hånd, og fordi de frygtede andre Guder;
\par 8 de fulgte de Folkeslags Skikke, som HERREN havde drevet bort foran Israeliterne, og de Kongers Skik, som Israel havde indsat;
\par 9 og Israeliterne udtænkte utilbørlige Ting mod HERREN deres Gud og byggede sig Offerhøje i alle deres Byer, lige fra Vagttårnene til de befæstede Byer;
\par 10 de rejste sig Stenstøtter og Asjerastøtter på alle høje Steder og under alle grønne Træer
\par 11 og tændte Offerild der på alle Høje ligesom de Folkeslag, HERREN havde ført bort foran dem, og øvede onde Ting, så at de krænkede HERREN;
\par 12 de dyrkede Afgudsbillederne, skønt HERREN havde sagt: "Det må I ikke gøre!"
\par 13 Og HERREN advarede Israel og Juda ved alle sine Profeter, alle Seerne, og sagde: "Vend om fra eders onde Færd og hold mine Bud og Anordninger i nøje Overensstemmelse med den Lov, jeg pålagde eders Fædre og kundgjorde eder ved mine Tjenere Profeterne!"
\par 14 Men de vilde ikke høre; de gjorde sig halsstarrige som deres Fædre, der ikke stolede på HERREN deres Gud;
\par 15 de lod hånt om hans Anordninger og den Pagt, han havde sluttet med deres Fædre, og om de Vidnesbyrd, han havde givet dem, og de holdt sig til Tomhed, så de blev til Tomhed, og efterlignede Folkeslagene rundt om dem, skønt HERREN havde pålagt dem ikke at gøre som de;
\par 16 de sagde sig løs fra HERREN deres Guds Bud og lavede sig støbte Billeder, to Tyrekalve; de lavede sig også Asjerastøtter, tilbad hele Himmelens Hær og dyrkede Ba'al:
\par 17 de lod deres Sønner og Døtre gå igennem Ilden, drev Spådomskunster og Sandsigeri og solgte sig til at gøre, hvad der er ondt i HERRENs Øjne, så de krænkede ham.
\par 18 Derfor blev HERREN såre fortørnet på Israel og drev dem bort fra sit Åsyn, så der ikke blev andet end Judas Stamme tilbage.
\par 19 Men heller ikke Juda holdt HERREN deres Guds Bud, men fulgte de Skikke, Israel havde indført.
\par 20 Derfor forkastede HERREN bele Israels Slægt, ydmygede dem, ga dem til Pris for Røvere og stødte dem til sidst bort fra sit Åsyn.
\par 21 Thi da Israel havde revet sig løs fra Davids Hus og gjort Jeroboam, Nebats Søn, til Konge, drog denne Israel bort fra HERREN og forledte dem til en stor Synd;
\par 22 og Israeliterne vandrede i alle de Synder, Jeroboam havde begået, og veg ikke derfra,
\par 23 så at HERREN til sidst drev Israel bort fra sit Åsyn, som han havde sagt ved alle sine Tjenere Profeterne; og Israel måtte vandre bort fra sit Land til Assyrien, hvor det er den Dag i Dag.
\par 24 Derefter lod Assyrerkongen Folk fra Babel, Kuta, Avva, Hamat og Sefarvajim komme og bosætte sig i Samarias Byer i Stedet for Israeliterne; og de tog Samaria i Besiddelse og bosatte sig i Byerne.
\par 25 Men den første Tid de boede der, frygtede de ikke HERREN; derfor sendte HERREN Løver iblandt dem, som dræbte dem.
\par 26 Da sendte de Assyrerkongen det Bud: "De Folk, du førte bort fra deres Hjem og lod bosætte sig i Samarias Byer, ved ikke, hvorledes Landets Gud skal dyrkes; derfor har han sendt Løver imod dem, og de dræber dem, fordi de ikke ved, hvorledes Landets Gud skal dyrkes!"
\par 27 Og Assyrerkongen bød: "Lad en af de Præster, jeg førte bort derfra, drage derhen, lad ham drage hen og bosætte sig der og lære dem, hvorledes Landets Gud skal dyrkes!"
\par 28 Så kom en af de Præster, de havde ført bort fra Samaria, og bosatte sig i Betel, og han lærte dem, hvorledes de skulde frygte HERREN.
\par 29 Men hvert Folk gav sig til at lave sig sin egen Gud og stillede ham op i Offerhusene på Højene, som Samaritanerne havde opførf, hvert Folk i sin By, hvor de havde bosat sig;
\par 30 Folkene fra Babel lavede Sukkot-Benot, Folkene fra Kuta Nergal, Folkene fra Hamat Asjima,
\par 31 Avvijiterne Nibhaz og Tartak, og Sefarviterne brændte deres Børn til Ære for Adrammelek og Anammelek, Sefarvajims Guder.
\par 32 Men de frygtede også HERREN og indsatte Folk at deres egen Midte til Præstr ved Offerhøjene, og disse ofrede for dem i Offerhusene på Højene.
\par 33 De frygte de HERREN, men dyrkede også deres egne Guder på de Folkeslags Vis, de var ført bort fra.
\par 34 Endnu den Dag i Dag følger de deres gamle Skikke. De frygtede ikke HERREN og handlede ikke efter de Anordninger og Lovbud, de havde fået, eller efter den Lov og det Bud, HERREN havde givet Jakobs Sønner, han, hvem han gav Navnet Israel.
\par 35 Og HERREN havde sluttet en Pagt med dem og givet dem det Bud: "I må ikke frygte andre Guder eller tilbede dem, ikke dyrke dem eller ofre til dem;
\par 36 men HERREN, som førte eder ud af Ægypten med vældig Kraft og udstrakt Arm, ham skal I frygte, ham skal I tilbede, og til ham skal I ofre!
\par 37 De Anordninger og Lovbud, den Lov og det Bud, han har opskrevet for eder, skal I omhyggeligt holde til alle Tider, og I må ikke frygte andre Guder!
\par 38 Den Pagt, han har sluttet med eder, må I ikke glemme, og I må ikke frygte andre Guder;
\par 39 men HERREN eders Gud skal I frygte, så vil han fri eder af alle eders Fjenders Hånd!"
\par 40 Dog vilde de ikke høre, men blev ved at handle som før.
\par 41 Således frygtede disse Folkeslag HERREN, men dyrkede tillige deres udskårne Billeder; og deres Børn og Børnebørn gør endnu den Dag i Dag som deres Fædre.

\chapter{18}

\par 1 I Elas Søns, Kong Hosea af Israels, tredje Regeringsår blev Ezekias, Akaz's Søn, Konge over Juda.
\par 2 Han var fem og tyve År gammel, da han blev Konge, og han herskede ni og tyve År i Jerusalem. Hans Moder hed Abi og var en Datter af Zekarja.
\par 3 Han gjorde, hvad der var ret i HERRENs Øjne, ganske som hans Fader David.
\par 4 Han skaffede Offerhøjene bort, sønderbrød Stenstøtterne, omhuggede Asjerastøtten og knuste Kobberslangen, som Moses havde lavet; thi indtil den Tid havde Israeliterne tændt Offerild for den, og man kaldte den Nehusjtan.
\par 5 Til HERREN, Israels Gud, satte han sin Lid, og hverken før eller siden fandtes hans Lige blandt alle Judas Konger.
\par 6 Han holdt fast ved HERREN og veg ikke fra ham, og han overholdt de Bud, HERREN havde givet Moses.
\par 7 Og HERREN var med ham; i alt, hvad han tog sig for, havde han Lykken med sig. Han gjorde Oprør mod Assyrerkongen og vilde ikke stå under ham.
\par 8 Han slog Filisterne lige til Gaza og dets Omegn, både Vagttårnene og de befæstede Byer.
\par 9 I Kong Ezekias's fjerde, Elas Søns, Kong Hosea af Israels, syvende Regeringsår, drog Assyrer kongen Salmanassar op mod Sa maria, belejrede
\par 10 og indtog det. Efter tre Års Forløb, i Ezekias's sjette, Kong Hosea af Israels niende Regeringsår, blev Samaria indtaget.
\par 11 Og Assyrerkongen førte Israel i Landflygtighed til Assyrien og lod dem bosætte sig i Hala, ved Habor, Gozans Flod, og i Mediens Byer,
\par 12 til Straf for at de ikke havde adlydt HERREN deres Guds Røst, men overtrådt hans Pagt, alt hvad HERRENs Tjener Moses havde påbudt; de hørte ikke derpå og gjorde ikke derefter.
\par 13 I Kong Ezekias's fjortende Regeringsår drog Assyrerkongen Sankerib op mod alle Judas befæstede Byer og indtog dem.
\par 14 Da sendte Kong Ezekias af Juda Bud til Assyrerkongen i Lakisj og lod sige: "Jeg har forbrudt mig; drag bort fra mig igen! Hvad du pålægger mig, vil jeg tage på mig!" Da pålagde Assyrerkongen Kong Ezekias af Juda at udrede 300 Talenter Sølv og 300 talenter Guld;
\par 15 og Ezekias udleverede alt det Sølv, der var i HERRENs Hus og i Skatkamrene i Kongens Palads.
\par 16 Ved den Lejlighed plyndrede Ezekias Dørene i HERRENs Helligdom og Pillerne for det Guld, han selv havde overtrukket dem med, og udleverede det til Assyrerkongen.
\par 17 Assyrerkongen sendte så Tartan, Rabsaris og Rabsjake med en anselig Styrke fra Lakisj til Kong Ezekias i Jerusalem, og de drog op og kom til Jerusalem og gjorde Holdt ved Øvredammens Vandledning, ved Vejen til Blegepladsen.
\par 18 Da de krævede at få Kongen i Tale, gik Paladsøversten Eljakim, Hilkijas Søn, Statsskriveren Sjebna og Kansleren Joa, Asafs Søn, ud til dem.
\par 19 Rabsjake sagde til dem: "Sig til Ezekias: Således siger Storkongen, Assyrerkongen: Hvad er det for en Fortrøstning, du hengiver dig til?
\par 20 Du mener vel, at et blot og bart Ord er det samme som Plan og Styrke i Krig? Og til hvem sætter du egentlig din Lid, siden du gør Oprør imod mig?
\par 21 Se nu, du sætter din Lid til Ægypten, denne brudte Rørkæp, som river Sår i Hånden på den, der støtter sig til den! Thi således går det alle dem, der sætter deres Lid til Farao, Ægyptens Konge.
\par 22 Men vil I sige til mig: Det er HERREN vor Gud, vi sætter vor Lid til! er det så ikke ham, hvis Offerhøje og Altre Ezekias skaffede bort, da han sagde til Juda og Jerusalem: Foran dette Alter, i Jerusalem skal I tilbede!
\par 23 Og nu, indgå et Væddemål med min Herre, Assyrerkongen: Jeg giver dig to Gange tusind Heste, hvis du kan stille Ryttere til dem!
\par 24 Hvorledes vil du afslå et Angreb af en eneste Statholder, en af min Herres ringeste Tjenere? Og du sætter din Lid til Ægypten, til Vogne og Heste?
\par 25 Mon det desuden er uden HERRENs Vilje, at jeg er draget op mod dette Sted for at ødelægge det? Det var HERREN selv, der sagde til mig: Drag op mod dette Land og ødelæg det!"
\par 26 Men Eljakim, Hilkijas Søn, Sjebna og Joa sagde til Rabsjake: "Tal dog Aramaisk til dine Trælle, det forstår vi godt; tal ikke Judæisk til os, medens Folkene på Muren hører på det!"
\par 27 Men Habsjake svarede dem: "Er det til din Herre og dig, min Herre har sendt mig med disse Ord? Er det ikke til de Mænd, der sidder på Muren hos eder og æder deres eget Skarn og drikker deres eget Vand!"
\par 28 Og Rabsjake trådte hen og råbte med høj Røst på Judæisk: "Hør Storkongens, Assyrerkongens, Ord!
\par 29 Således siger Kongen: Lad ikke Ezekias vildlede eder, thi han er ikke i Stand til at frelse eder af min Hånd!
\par 30 Og lad ikke Ezekias forlede eder til at sætte eders Lid til HERREN, når han siger: HERREN skal sikkert frelse os, og denne By skal ikke overgives i Assyrerkongens Hånd!
\par 31 Hør ikke på Ezekias; thi således - siger Assyrerkongen: Vil I slutte Fred med mig og overgive eder til mig, så skal enhver af eder spise af sin Vinstok og sit Figentræ og drikke af sin Brønd,
\par 32 indtil jeg kommer og tager eder med til et Land, der ligner eders, et Land med Korn og Most, et Land med Brød og Vingårde, et Land med Oliventræer og Honning; så skal I leve og ikke dø. Hør derfor ikke på Ezekias, når han vil forføre eder og siger: HERREN vil frelse os!
\par 33 Mon nogen af Folkeslagenes Guder har kunnet frelse sit Land af Assyrerkongens Hånd?
\par 34 Hvor er Hamats og Arpads Guder, hvor er Sefarvajims, Henas og Ivvas Guder? Hvor er Landet Samarias Guder? Mon de frelste Samaria af min Hånd?
\par 35 Hvor er der blandt alle Landes Guder nogen, der har frelst sit Land af min Hånd? Mon da HERREN skulde kunne frelse Jerusalem?"
\par 36 Men de tav og svarede ham ikke et Ord, thi Kongens Bud lød på, at de ikke måtte svare ham.
\par 37 Derpå gik Paladsøversten Eljakim, Hilkijas Søn, Statsskriveren Sjebna og Kansleren Joa, Asafs Søn, med sønderrevne Klæder til Ezekias og meddelte ham, hvad Rabsjake havde sagt.

\chapter{19}

\par 1 Da Kong Ezekias hørte det, sønderrev han sine Klæder, hyllede sig i Sæk og gik ind i HERRENs Hus.
\par 2 Og han sendte Paladsøversten Eljakim og Statsskriveren Sjebna og Præsternes Ældste, hyllet i Sæk, til Profeten Esajas, Amoz's Søn,
\par 3 for at sig til ham: "Ezekias lader sige: En Nødens, Tugtelsens og Forsmædelsens Dag er denne dag, thi Barnet er ved at fødes, men der er ikke Kraft til at bringe det til Verden!
\par 4 Dog vil HERREN din Gud måske høre alt, hvad Rabsjake har sagt, han, som er sendt af sin Herre, Assyrerkongen, for at håne den levende Gud, og måske vil han straffe ham for de Ord, som HERREN din Gud har hørt - gå derfor i Forbøn for den Rest, der endnu er tilbage!"
\par 5 Da Kong Ezekias's Folk kom til Esajas,
\par 6 sagde han til dem: "Således skal I svare eders Herre: Så siger HERREN: Frygt ikke for de Ord, du har hørt, som Assyrerkongens Trælle har hånet mig med!
\par 7 Se, jeg vil indgive ham en Ånd, og han skal få en Tidende at høre, så han vender tilbage til sit Land, og i hans eget Land vil jeg fælde ham med Sværdet!"
\par 8 Rabsjake vendte så tilbage og traf Assyrerkongen i Færd med at belejre Libna; thi han havde hørt, at Kongen var brudt op fra Lakisj.
\par 9 Så fik han Underretning om, at Kong Tirhaka af Ætiopien var rykket ud for at angribe ham, og han sendte atter Sendebud til Ezekias og sagde:
\par 10 "Således skal I sige til Kong Ezekias af Juda: Lad ikke din Gud, som du slår din Lid til, vildlede dig med at sige, at Jerusalem ikke skal gives i Assyrerkongens Hånd!
\par 11 Du har jo dog hørt, hvad Assyrerkongerne har gjort ved alle Lande, hvorledes de har lagt Band på dem - og du skulde kunne undslippe!
\par 12 De Folk, mine Fædre tilintetgjorde, Gozan, Haran, Rezef og Folkene fra Eden i Telassar, har deres Guder kunnet frelse dem?
\par 13 Hvor er Kongen af Hamat, Kongen af Arpad eller Kongen af La'ir, Sefarvajim, Hena og Ivva?"
\par 14 Da Ezekias havde modtaget Brevet af Sendebudenes Hånd og læst det, gik han op i HERRENs Hus og bredte det ud for HERRENs Åsyn.
\par 15 Derpå bad Ezekias den Bøn for HERRENs Åsyn: "HERRE, Israels Gud, du, som troner over heruberne, du alene er Gud over alle Jordens Riger; du har gjort Himmelen og Jorden!
\par 16 Bøj nu dit Øre, HERRE, og lyt, åbn dine Øjne, HERRE, og se! Læg Mærke til de Ord, Sankerib har sendt hid for at spotte den levende Gud!
\par 17 Det er sandt, HERRE, at Assyrerkongerne har tilintetgjort de Folk og deres Lande
\par 18 og kastet deres Guder i Ilden; men de er ikke Guder, kun Menneskehænders Værk af Træ eller Sten, derfor kunde de ødelægge dem.
\par 19 Men frels os nu, HERRE vor Gud, af hans Hånd, så alle Jordens Riger kan kende, at du, HERRE, alene er Gud!"
\par 20 Så sendte Esajas, Amoz's Søn, Bud til Ezekias og lod sige: "Så siger HERREN, Israels Gud: Din Bøn angående Assyrerkongen Sankerib har jeg hørt!"
\par 21 Således lyder det Ord, HERREN talede imod ham: Hun håner, hun spotter dig, Jomfruen, Zions Datter, Jerusalems Datter ryster på Hovedet ad dig!
\par 22 Hvem har du hånet og smædet, mod hvem har du løftet din Røst? Mod Israels Hellige løfted i Hovmod du Blikket!
\par 23 Ved dine Sendebud håned du HERREN og sagde: "Med mine talløse Vogne besteg jeg Bjergenes Højder, Libanons afsides Egne; jeg fælded dets Cedres Højskov, dets ædle Cypresser, trængte frem til dets øverste Raststed, dets Havers Skove.
\par 24 Fremmed Vand grov jeg ud, og jeg drak det, tørskoet skred jeg over Ægyptens Strømme!"
\par 25 Har du ej hørt det? For længst kom det op i min Tanke, jeg lagde det fordum til Rette, nu lod jeg det ske, og du gjorde murstærke Byer til øde Stenhobe,
\par 26 mens Folkene grebes i Afmagt af Skræk og Skam, blev som Græsset på Marken, det spirende Grønne, som Græs på Tage, som Mark for Østenvinden. Jeg ser, når du rejser
\par 27 og sætter dig, ved, når du går og kommer.
\par 28 Fordi du raser imod mig, din Trods er mig kommet for Øre, lægger jeg Ring i din Næse og Bidsel i Munden og fører dig bort ad Vejen, du kom!
\par 29 Og dette skal være dig Tegnet: I År skal man spise, hvad der såed sig selv, og Året derpå, hvad der skyder af Rode, tredje År skal man så oghøste, plante Vin og nyde dens Frugt.
\par 30 Den bjærgede Rest af Judas Hus slår atter Rødder forneden og bærer sin Frugt foroven; thi fra Jerusalem udgår en Rest, en Levning fra Zions Bjerg.
\par 31 HERRENs Nidkærhed virker dette.
\par 32 Derfor, så siger HERREN om Assyrerkongen: I Byen her skal han ej komme ind, ej sende en Pil herind, ej nærme sig den med Skjolde eller opkaste Vold imod dem;
\par 33 ad Vejen, han kom, skal han gå igen, i Byen her skal han ej komme ind så lyder det fra HERREN.
\par 34 Jeg værner og frelser denne By for min og min Tjener Davids Skyld!
\par 35 Samme Nat gik HERRENs Engel ud og ihjelslog i Assyrernes Lejr 185000 Mand; og se, næste Morgen tidlig lå de alle døde.
\par 36 Da brød Assyrerkongen Sankerib op, vendte hjem og blev siden i Nineve.
\par 37 Men da han engang tilbad i sin Gud Nisroks Hus, slog Adrammelek og Sarezer ham ihjel med deres Sværd, hvorefter de flygtede til Ararats Land; og hans Søn Asarhaddon blev Konge i hans Sted.

\chapter{20}

\par 1 Ved den Tid blev Ezekias dødssyg. Da kom Profeten Esajas, Amoz's Søn, til ham og sagde: "Så siger HERREN: Beskik dit Hus, thi du skal dø og ikke leve!"
\par 2 Da vendte han Ansigtet om mod Væggen og bad således til HERREN:
\par 3 "Ak, HERRE, kom dog i Hu, hvorledes jeg har vandret for dit Åsyn i Oprigtighed og med helt Hjerte og gjort, hvad der er godt i dine Øjne!" Og Ezekias græd højt.
\par 4 Men Esajas var endnu ikke ude af den mellemste Forgård, før HERRENs Ord kom til ham således:
\par 5 "Vend tilbage og sig til Ezekias, mit Folks Fyrste: Så siger HERREN, din Fader Davids Gud: Jeg har hørt din Bøn, jeg har set dine Tårer! Se, jeg vil helbrede dig; i Overmorgen kan du gå op i HERRENs Hus;
\par 6 og jeg vil lægge femten År til dit Liv og udfri dig og denne By af Assyrerkongens Hånd og værne om denne By for min og min Tjener Davids Skyld!"
\par 7 Derpå sagde Esajas: "Kom med et Figenplaster!" Og da de kom med Plasteret og lagde det på det syge Sted, blev han rask.
\par 8 Men Ezekias sagde til Esajas: "Hvad er Tegnet på, at HERREN vil helbrede mig, så jeg i Overmorgen kan gå op i HERRENs Hus?"
\par 9 Da svarede Esajas: "Dette skal være dig Tegnet fra HERREN på, at HERREN vil udføre, hvad han har sagt: Skal Skyggen gå ti Streger frem eller ti Streger tilbage?"
\par 10 Ezekias sagde: "Skyggen kan let strække sig ti Streger frem nej, lad den gå ti Streger tilbage!"
\par 11 Da råbte Profeten Esajas til HERREN; og han lod Skyggen på Akaz's Solur gå ti Streger tilbage.
\par 12 Ved den Tid sendte Bal'adans Søn, Kong Merodak-Bal'adan af Babel, Brev og Gave til Ezekias, da han hørte, at Ezekias havde været syg.
\par 13 Og Ezekias glædede sig over deres Komme og viste dem hele Huset, hvor han havde sine Skatte, Sølvet og Guldet, Røgelsestofferne, den fine Olie, hele sit Våbenoplag og alt, hvad der var i hans Skatkamre; der var ikke den Ting i hans Hus og hele hans Rige, som Ezekias ikke viste dem.
\par 14 Da kom Profeten Esajas til Kong Ezekias og sagde til ham: "Hvad sagde disse Mænd, og hvorfra kom de til dig?" Ezekias svarede: "De kom fra et fjernt Land, fra Babel!"
\par 15 Da spurgte han: "Hvad fik de at se i dit Hus?" Ezekias svarede: "Alf, hvad der er i mit Hus, så de; der er ikke den Ting i mine Skatkamre, jeg ikke viste dem!"
\par 16 Da sagde Esajas til Ezekias: "Hør HERRENs Ord!
\par 17 Se, Dage skal komme, da alt, hvad der er i dit Hus, og hvad dine Fædre har samlet indtil denne Dag, skal bringes til Babel og intet lades tilbage, siger HERREN!
\par 18 Og af dine Sønner, der nedstammer fra dig, og som du avler, skal nogle tages og gøres til Hofmænd i Babels Konges Palads!"
\par 19 Men Ezekias sagde til Esajas: "Det Ord fra HERREN, du har kundgjort, er godt!" Thi han tænkte: "Så bliver der da Fred og Tryghed, så længe jeg lever!"
\par 20 Hvad der ellers er at fortælle om Ezekias og hans Heltegerninger, og hvorledes han anlagde Vanddammen og Vandledningen og ledede Vandet ind i Byen, står jo optegnet i Judas Kongers Krønike.
\par 21 Så lagde Ezekias sig til Hvile hos sine Fædre; og hans Søn Manasse blev Konge i hans Sted.

\chapter{21}

\par 1 Manasse var tolv År gammel, da han blev Konge, og han herskede fem og halvtredsindstyve År i Jerusalem. Hans Moder hed Hefziba.
\par 2 Han gjorde, hvad der var ondt i HERRENs Øjne, og efterlignede de Folkeslags Vederstyggeligheder, som HERREN havde drevet bort foran Israeliterne.
\par 3 Han byggede atter de Offerhøje, som hans Fader Ezekias havde ødelagt, rejste Altre for Ba'al, lavede en Asjerastøtte, ligesom Kong Akab af Israel havde gjort, og tilbad og dyrkede hele Himmelens Hær.
\par 4 Og han byggede Altre i HERRENs Hus, om hvilket HERREN havde sagt: "I Jerusalem vil jeg stedfæste mit Navn."
\par 5 Og han byggede Altre for hele Himmelens Hær i begge HERRENs Hus's Forgårde.
\par 6 Han lod sin Søn gå igennem Ilden, drev Trolddom og tog Varsler og ansatte Dødemanere og Sandsigere; han gjorde meget, som var ondt i HERRENs Øjne, og krænkede ham.
\par 7 Den Asjerastøtte, han lod lave, opstillede han i det Hus, om bvilket HERREN havde sagt til David og hans Søn Salomo: "I dette Hus og i Jerusalem, som jeg har udvalgt af alle Israels Stammer, vil jeg stedfæste mit Navn til evig Tid;
\par 8 og jeg vil ikke mere lade Israel vandre bort fra det Land, jeg gav deres Fædre, dog kun på det Vilkår, at de omhyggeligt overholder alt, hvad jeg har pålagt dem, hele den Lov, min Tjener Moses gav dem."
\par 9 Men de vilde ikke høre, og Manasse forførte dem til at handle værre end de Folkeslag, HERREN havde udryddet foran Israeliterne.
\par 10 Da talede HERREN ved sine Tjenere Profeterne således:
\par 11 "Efterdi Kong Manasse af Juda har øvet disse Vederstyggeligheder, ja øvet Ting, som er værre end alt, hvad Amoriterne, der var før ham. øvede, og tillige ved sine Afgudsbilleder har forledt Juda til Synd,
\par 12 derfor, så siger HERREN, Israels Gud: Se, jeg bringer Ulykke over Jerusalem og Juda, så det skal ringe for Ørene på enhver, der hører derom!
\par 13 Jeg vil trække samme Målesnor over Jerusalem som over Samaria og bruge samme Lod som ved Akabs Hus; jeg vil afviske Jerusalem, som man afvisker et Fad og vender det om;
\par 14 jeg forstøder, hvad der er tilbage af min Ejendom, oggiver dem i deres Fjenders Hånd, og de skal blive til Rov og Bytte for alle deres Fjender,
\par 15 fordi de gjorde, hvad der var ondt i mine Øjne, og krænkede mig, lige fra den Dag deres Fædre drog ud af Ægypten og indtil i Dag!"
\par 16 Desuden udgød Manasse uskyldigt Blod i store Måder, så han fyldte Jerusalem dermed til Randen, for ikke at tale om den Synd; han begik ved at forlede Juda til Synd, så de gjorde, hvad der var ondt i HERRENs Øjne.
\par 17 Hvad der ellers er at fortælle om Manasse, alt, hvad han udførte, og den Synd, han begik, står jo optegnet i Judas Kongers Krønike.
\par 18 Så lagde Manasse sig til Hvile hos sine Fædre og blev jordet i Haven ved sit Hus, i Uzzas Have; og hans Søn Amon blev Konge i hans Sted.
\par 19 Amon var to og tyve År gammel, da han blev Konge, og han herskede to År i Jerusalem. Hans Moder hed Mesjullemet og var en Datter af Haruz fra Jotba.
\par 20 Han gjorde, hvad der var ondt i HERRENs Øjne,,ligesom hans Fader Manasse;
\par 21 han vandrede nøje i sin Faders Spor og dyrkede Afgudsbillederne, som hans Fader havde dyrket, og tilbad dem;
\par 22 han forlod HERREN, sine Fædres Gud, og vandrede ikke på HERRENs Vej.
\par 23 Amons Folk sammensvor sig imod ham og dræbte Kongen i hans
\par 24 men Folket fra Landet dræbte alle dem, der havde sammensvoret sig imod Kong Amon, og gjorde hans Søn Josias til Konge i hans Sted.
\par 25 Hvad der ellers er at fortælle om Amon, og hvad han udførte, står jo optegnet i Judas Kongers Krønike.
\par 26 Man jordede ham i hans Faders Grav i Uzzas Have; og hans Søn Josias blev Konge i hans Sted.

\chapter{22}

\par 1 Josias var otte År gammel, da han blev Konge, og han herskede een og tredive År i Jerusalem. Hans Moder hed Jedida og var en Datter af Adaja fra Bozkat.
\par 2 Han gjorde, hvad der var ret i HERRENs Øjne, og vandrede nøje i sin Fader Davids Spor uden at vige til højre eller venstre.
\par 3 I Kong Josias's attende Regeringsår sendte Kongen Statsskriveren Sjafan, en Søn af Mesjullams Søn Azalja, til HERRENs Hus og sagde:
\par 4 "Gå op til Ypperstepræsten Hilkija og lad ham tage de Penge frem, der er indkommet i HERRENs Hus, og som Dørvogterne har samlet ind hos Folket,
\par 5 for af man kan give Pengene til dem, der står for Arbejdet, dem, der har Tilsyn med HERRENs Hus; de skal så give dem til Arbejderne i HERRENs Hus til Istandsættelse af de brøstfældige Steder på Templet,
\par 6 til Tømrerne, Bygningsmændene og Murerne, og til Indkøb af Træ og tilhugne Sten til Templets Istandsættelse.
\par 7 Bog skal der ikke holdes Regnskab med dem over de Penge, der overlades dem, men de skal handle på Tro og Love."
\par 8 Da sagde Ypperstepræsten Hilkija til Statsskriveren Sjafan: "Jeg har fundet Lovbogen i HERRENS Hus!" Og Hilkija gav Sjafan Bogen, og han læste den.
\par 9 Derpå begav Statsskriveren Sjafan sig til Kongen og aflagde Beretning for ham og sagde: "Dine Trælle har taget de Penge frem, der fandtes i Templet, og givet dem til dem, der står for Arbejdet, dem, der har Tilsyn med HERRENs Hus."
\par 10 Derpå gav Statsskriveren Sjafan Kongen den Meddelelse: "Præsten Hilkija gav mig en Bog!" Og Sjafan læste den for Kongen.
\par 11 Men da Kongen hørte, hvad der stod i Lovbogen, sønderrev han sine Klæder;
\par 12 og Kongen bød Præsten Hilkija, Ahikam, Sjafans Søn, Akbor, Mikas Søn, Statsskriveren Sjafan og Kongens Tjener Asaja:
\par 13 "Gå hen og rådspørg på mine og Folkets og hele Judas Vegne HERREN om Indholdet af denne Bog, der er fundet; thi stor er Vreden, der er blusset op hos HERREN imod os, fordi vore Fædre ikke har adlydt Ordene i denne Bog og ikke har handlet nøje efter, hvad der står skrevet deri."
\par 14 Præsten Hilkija, Ahikam, Akbor, Sjafan og Asaja gik da hen og talte med Profetinden Hulda, som var gift med Sjallum, Opsynsmanden over Tøjet, en Søn af Harhas's Søn Tikva, og som boede i Jerusalem i den nye Bydel.
\par 15 Hun sagde til dem: "Så siger HERREN, Israels Gud: Sig til den Mand, der sendte eder til mig:
\par 16 Så siger HERREN: Se, jeg vil bringe Ulykke over dette Sted og dets Indbyggere, alt, hvad der står i den Bog, Judas Konge har læst,
\par 17 til Straf for at de har forladt mig og tændt Offerild for andre Guder, så at de krænkede mig med alt deres Hænders Værk, og min Vrede vil blusse op mod dette Sted uden at slukkes!
\par 18 Men til Judas Konge, der sendte eder for at rådspørge HERREN, skal I sige således: Så siger HERREN, Israels Gud: De Ord, du har hørt, står fast;
\par 19 men efterdi dit Hjerte bøjede sig og du ydmygede dig for HERREN, da du hørte, hvad jeg har talet mod dette Sted og dets Indbyggere, at de skal blive til Rædsel og Forbandelse, og efterdi du sønderrev dine Klæder og græd for mit Åsyn, så har også jeg hørt dig, lyder det fra HERREN!
\par 20 Og jeg vil lade dig samles til dine Fædre, og du skal samles til dem i Fred i din Grav, uden at dine Øjne får al den Ulykke at se, som jeg vil bringe over defte Sted." Det Svar bragte de Kongen.

\chapter{23}

\par 1 Da sendte Kongen Bud og lod alle Judas og Jerusalems Ældste kalde sammen hos sig.
\par 2 Derpå gik Kongen op i HERRENs Hus, fulgt af alle Judas Mænd og alle Jerusalems Indbyggere, Præsterne, Profeterne og hele Folket, små og store, og han forelæste dem alt, hvad der stod i Pagtsbogen, som var fundet i HERRENs Hus.
\par 3 Så tog Kongen Plads ved Søjlen og sluttede Pagt for HERRENs Åsyn om, at de skulde holde sig til HERREN og holde hans Bud, Vidnesbyrd og Anordninger af hele deres Hjerte og hele deres Sjæl, for at han kunde opfylde denne Pagts Ord, således som de var skrevet i denne Bog. Og alt Folket indgik Pagten.
\par 4 Derpå bød Kongen Ypperstepræsten Hilkija og Andenpræsten og Dørvogterne at bringe alle de Ting, der var lavet til Ba'al, Asjera og hele Himmelens Hær, ud af HERRENs Helligdom, og han lod dem opbrænde uden for Jerusalem på Markerne ved Hedron, og Asken lod han bringe til Betel.
\par 5 Og han afsatte Afgudspræsterne, som Judas Konger havde indsat, og som havde tændt Offerild på Højene i Judas Byer og Jerusalems Omegn, ligeledes dem, som havde tændt Offerild for Ba'al og for Solen, Månen, Stjernebillederne og hele Himmelens Hær.
\par 6 Og han lod Asjerastøtten bringe fra HERRENs Hus uden for Jerusalem til Kedrons Dal og opbrænde der, og han lod den knuse til Støv og Støvet kaste hen, hvor Småfolk havde deres Grave.
\par 7 Han lod Mandsskøgernes Kamre i HERRENs Hus rive ned, dem i hvilke Kvinderne vævede Kjortler til Asjera.
\par 8 Han lod alle Præsterne hente fra Judas Byer og vanhelligede Offer: højene, hvor Præsterne havde tændt offerild, fra Geba til Be'er-sjeba. Og han lod Portofferhøjen rive ned, som var ved Indgangen til Byøversten Jehosjuas Port til venstre, når man går ind ad Byporten.
\par 9 Dog fik Præsterne ved Højene ikke Adgang til HERRENs Alter i Jerusalem, men de måtte spise usyret Brød blandt deres Brødre.
\par 10 Han vanhelligede Ildstedet i Hinnoms Søns Dal, så at ingen mere kunde lade sin Søn eller Datter gå igennem Ilden for Molok.
\par 11 Han fjernede de Heste, Judas Konger havde opstillet for Solen ved Indgangen til HERRENs Hus hen imod Hofmanden Netan Me'leks Kammer i Parvarim, og Solens Vogne lod han opbrænde.
\par 12 Altrene, som Judas Konger havde rejst på Taget (Akaz's Tagbygning), og Altrene, som Manasse havde rejst i begge Forgårdene til HERRENs Hus, lod Kongen nebdryde og knuse på Stedet, og Støvet lod han kaste, hen i Kedrons Dal.
\par 13 Og Offerhøjene østen for Jerusalem på Sydsiden af Fordærvelsens Bjerg, som Kong Salomo af Israel havde bygget for Astarte; Zidoniernes væmmelige Gud, Kemosj, Moabiternes væmmelige Gud, og Milkom, Ammoniternes vederstyggelige Gud, vanhelligede Kongen.
\par 14 Han lod Stenstøtterne nedbryde og Asjerastøtterne omhugge og Pladsen fylde med Menneskeknogler.
\par 15 Også Alteret i Betel, den Offerhøj, som Jeroboam, Nebats Søn havde rejst han der forledte Israel til Synd, også det Alter tillige med Offerhøjen lod han nedbryde; Stenene lod han nedrive og knuse til Støv, og Asjerastøtten lod han opbrænde.
\par 16 Da Josias vendte sig om og fik Øje på Gravene på Bjerget, sendte han Folk hen og lod dem tage Benene ud af Gravene og opbrænde dem på Alteret for, at vanhellige det efter HERRENs Ord, som den Guds Mand udråbte, da han kundgjorde disse Ting,
\par 17 Så sagde han: "Hvad er det for et Gravmæle, jeg ser der?" Byens Folk svarede ham: "Det er den Guds Mands Grav, der kom fra Juda og kundgjorde det, du nu har gjort med Alteret i Betel."
\par 18 Da sagde han: "Lad ham ligge i Ro, ingen må flytte hans Ben!" Så lod de både hans og Profeten fra Samarias Ben fred.
\par 19 Også alle Offerhusene på Højene i Samarias Byer, som Israels Konger havde opført for at krænke HERREN, lod Josias fjerne, og han handlede med dem ganske som han havde gjort i Betel;
\par 20 og han lod alle Højenes Præster, som var der ved Altrene, dræbe, og han opbrændte Menneskeknogler på Altrene. Så vendte han tilbage til Jerusalem.
\par 21 Derefter gav Kongen alt Folket, den Befaling: "Hold Påske for HERREN eders Gud, som der er foreskrevet her i Pagtsbogen!"
\par 22 Thi slig en Påske var ikke nogen Sinde blevet holdt i Israels og Judas Kongers Tid, ikke siden Dommerne dømte Israel;
\par 23 men først i Kong Josias's attende Regeringsår blev en sådan Påske fejret for HERREN i Jerusalem.
\par 24 Også Dødemanerne og Sandsigerne, Husguderne, Afgudsbillederne og alle de væmmelige Guder, der var at se i Judas Land og Jerusalem, udryddede Josias for at opfylde Lovens Ord, der stod skrevet i den Bog, Præsten Hilkija havde fundet i HERRENs Hus.
\par 25 Der havde ikke før været nogen Konge, der som han af hele sit Hjerte og hele sin Sjæl og al sin Kraft omvendte sig til HERREN og fulgte hele Mose Lov; og heller ikke efter ham opstod hans Lige.
\par 26 Alligevel lagde HERRENs mægtige Vredesglød sig ikke, thi hans Vrede var blusset op mod Juda over alle de Krænkelser, Manasse havde tilføjet ham.
\par 27 Og HERREN sagde: "Også Juda vil jeg fjerne fra mit Åsyn, ligesom jeg fjernede Israel, og jeg vil forkaste Jerusalem, denne min udvalgte By, og det Hus, om hvilket jeg har sagt: Mit Navn skal være der!"
\par 28 Hvad der ellers er at fortælle om Josias, alt, hvad han udførte, står jo optegnet i Judas Kongers Krønike.
\par 29 I hans Dage drog Ægypterkongen Parao Neko op til Eufratfloden imod Assyrerkongen. Kong Josias rykkede imod ham, men Neko fældede ham ved Megiddo, straks han så ham.
\par 30 Og hans Folk førte ham død bort fra Megiddo, bragte ham til Jerusalem og jordede ham i hans Grav. Men Folket fra Landet tog Joahaz, Josias's Søn, og salvede og hyldede ham til Bonge i hans Faders Sted.
\par 31 Joahaz var tre og tyve År gammel, da han blev Konge, og han herskede tre Måneder i Jerusalem. Hans Moder hed Hamutal og var en Datter af Jirmeja fra, Libna.
\par 32 Han gjorde, hvad der var ondt i HERRENs Øjne, ganske som hans Fædre.
\par 33 Men Farao Neko lod ham fængsle i Ribla i Hamats Land og gjorde dermed Ende på hans Herredømme i Jerusalem og lagde en Skat af hundrede Talenter Sølv og ti Talenter Guld på Landet.
\par 34 Derpå gjorde Farao Neko Eljakim, Josias's Søn, til Konge i hans Fader Josias's Sted, og han ændrede hans Navn til Jojakim; Joahaz derimod tog han med til Ægypten, og der døde han.
\par 35 Sølvet og Guldet udredede Jojakim til Farao; men for.at kunne udrede Pengeneefter Faraos Befaling satte han Landet i Skat; efter som hver især var sat i Skat, inddrev han Sølvet og Guldet for at give Farao Neko det.
\par 36 Jojakim var fem og tyve År gammel, da han blev Konge, og han herskede elleve År i Jerusalem. Hans Moder hed Zebida og var en Datter af Pedaja fra Ruma.
\par 37 Han gjorde, hvad der var ondt i HERRENs Øjne, ganske som hans Fædre.

\chapter{24}

\par 1 I hans Dage drog Kong Nebukadnezar af Balel op, og Jojakim underkastede sig ham; men efter tre Års Forløb faldt han fra ham igen.
\par 2 Da sendte HERREN kaldæiske, aramaiske, moabitiske og ammonitiske Strejfskarer imod ham; dem sendte han ind i Juda for at ødelægge det efter det Ord, HERREN havde talet ved sine Tjenere Profeterne.
\par 3 Det skyldtes ene og alene HERRENs Vrede, at Juda blev drevet bort fra HERRENs Åsyn; det var for Manasses Synders Skyld, for alt det, han havde gjort,
\par 4 også for det uskyldige Blods Skyld, som han havde udgydt så meget af, at han havde fyldt Jerusalem dermed; det vilde HERREN ikke tilgive.
\par 5 Hvad der ellers er at fortælle om Jojakim, alt, hvad han udførte, står jo optegnet i Judas Kongers Krønike.
\par 6 Så lagde Jojakim sig til Hvile hos sine Fædre; og hans Søn Jojakin blev Kooge i hans Sted.
\par 7 Men Ægypterkongen drog ikke mere ud af sit Land, thi Babels Konge havde taget alt, hvad der tilhørte Ægypterkongen fra Ægyptens Bæk til Eufratfloden.
\par 8 Jojakin var atten År gammel, da han blev Konge, og han herskede tre Måneder i Jerusålem. Hans Moder hed Nehusjta og var en Datter af Elnatan fra Jerusalem.
\par 9 Han gjorde, hvad der var ondt i HERRENs Øjne, ganske som hans Fader.
\par 10 På den Tid drog Kong Nebukadnezar af Babels Folk op mod Jerusalem, og Byen blev belejret.
\par 11 Kong Nebukadnezar af Babel kom til Jerusalem, medens hans Folk belejrede det.
\par 12 Da overgav Kong Jojakin af Juda sig med sin Modet, sine Tjenere, Øverster og Hoffolk til Babels Konge, og han tog imod ham; det var i Kongen af Babels ottende Regeringsår.
\par 13 Og som HERREN havde forudsagt, førte han alle Skattene i HERRENs Hus og Kongens Palads bort og brød Guldet af alle de Ting, Kong Salomo af Israel havde ladet lave i HERRENs Helligdom.
\par 14 Og han førte hele Jerusalem, alle Øversterne og de velhavende, 10.000 i Tal, bort som Fanger, ligeledes Grovsmede og Låsesmede, så der ikke blev andre tilbage end de fattigste af Folket fra Landet.
\par 15 Så førte han Jojakin som Fange til Babel, også Kongens Moder og Hustruer, hans Hoffolk og alle de højtstående i Landet førte han som Fanger fra Jerusalem til Babel;
\par 16 fremdeles alle de velbavende, 7000 i Tal, Grovsmedene og Låsesmedene, 1800 i Tal, alle øvede Krigere - dem førte Babels Konge som Fanger til Babel.
\par 17 Men i Jojakins Sted gjorde Babels Konge hans Farbroder Mattanja til Konge, og han ændrede hans Navn til Zedekias.
\par 18 Zedekias var een og tyve År gammel, da han blev Konge, og han herskede elleve År i Jerusalem. Hans Moder hed Hamital og var en Datter af Jirmeja fra Libna.
\par 19 Han gjorde, hvad der var ondt i HERRENs Øjne, ganske som Jojakim.
\par 20 Thi for HERRENs Vredes Skyld timedes dette Jerusalem og Juda, og til sidst stødte han dem bort fra sit Åsyn. Og Zedekias faldt fra Babels Konge.

\chapter{25}

\par 1 I hans niende Regeringsår på den tiende Dag i den tiende Måned drog Kong Nebukadnezar af Babel da med hele sin Hær mod Jerusalem og belejrede det, og de byggede Belejringstårne imod det rundt omkring;
\par 2 og Belejringen varede til Kong Zedekias's ellevte Regeringsår.
\par 3 På den niende Dag i den fjerde Måned blev Hungersnøden hård i Byen, og Folket fra Landet havde ikke Brød.
\par 4 Da blev Byens Mur gennembrudt. Kongen og alle Krigsfolkene flygtede om Natten gennem Porten mellem de to Mure ved Kongens Have, medens Kaldæerne holdt Byen omringet, og han tog Vejen ad Arabalavningen til.
\par 5 Men Kaldæernes Hær satte efter Kongen og indhentede ham på Jerikosletten, efter at hele hans Hær var blevet splittet til alle Sider.
\par 6 Så greb de Kongen og bragte ham op til Ribla til Babels Konge, der fældede Dommen over ham.
\par 7 Hans Sønner lod han henrette i hans Påsyn, og på Zedekias selv lod han Øjnene stikke ud; derpå lod han ham lægge i Kobberlænker, og således førte de ham til Babel.
\par 8 På den syvende Dag i den femte Måned, det var Babels Konge Nebukadnezars nittende Regeringsår, kom Nebuzaradan, Øversten for Livvagten, Babels Konges Tjener, til Jerusalem.
\par 9 Han satte Ild på HERRENs Hus og Kongens Palads og alle Husene i Jerusalem; på alle Stormændenes Huse satte han Ild;
\par 10 og Murene om Jerusalem nedbrød alle Kaldæernes Folk, som Øversten for Livvagten havde med sig.
\par 11 De sidste Folk, som var tilbage i Byen, og Overløberne, der var gået over til Babels Konge, og,de sidste Håndværkere førte Nebuzaradan, Øversten for Livvagten, bort.
\par 12 Men nogle af de fattigste at Folket fra Landet lod Øversten for Livvagten blive tilbage som Vingårdsmænd og Agerdyrkere.
\par 13 Kobbersøjlerne i HERRENs Hus, Stellene og Kobberhavet i HERRENs Hus slog Kaldæerne i Stykker og førte Kobberet til Babel.
\par 14 Karrene, Skovlene, Hnivene, Kanderne og alle Kobbersagerne, som brugtes ved Tjenesten, røvede de;
\par 15 også Panderne og Skålene, alt, hvad der helt var af Guld eller Sølv, røvede Øversten for Livvagten.
\par 16 De to Søjler, Havet og Stellene, som Salomo havde ladet lave til HERRENs Hus - Kobberet i alle disse Ting var ikke til at veje.
\par 17 Atten Alen høj var den ene Søjle, og der var et Søjlehoved af Kobber oven på den, tre Alen højt, og rundt om Søjlehovederne var der Fletværk og Granatæbler, alt af Kobber; og på samme Måde var det med den anden Søjle.
\par 18 Øversten for Livvagten tog Ypperstepræsten Seraja, Andenpræsten Zefanja og de tre Dørvogtere;
\par 19 og fra Byen tog han en Hofmand, der havde Opsyn med Krigs folket, og fem Mænd, der hørte til Kongens nærmeste Omgivelser, og som endnu fandtes i Byen, desuden Hærførerens Skriver, der udskrev Folket fra Landet til Krigstjeneste, og dertil tresindstyve Mænd af Folket fra Landet, der fandtes i Byen,
\par 20 dem tog Øversten for Livvagten Nebuzaradan og førte til Babels Konge i Ribla;
\par 21 og Babels Konge lod dem dræbe i Ribla i Hamats Land. Så førtes Juda i Landflygtighed fra sit Land.
\par 22 Over de Folk, der blev tilbage i Judas Land, dem, Babels Konge Nebukadnezar lod blive tilbage, satte han Gedalja, Ahikams Søn, Sjafans Sønnesøn.
\par 23 Da nu alle Hærførerne med deres Folk hørte, at Babels Konge havde indsat Gedalja, kom de til ham i Mizpa, Jisjmael, Netanjas Søn, Johanan,Bareas Søn, Seraja, Tanhumets Søn fra Netofa, og Ja'azanja, Ma'akatitens Søn, tillige med deres Folk.
\par 24 Og Gedalja besvor dem og deres Folk og sagde: "Frygt ikke for Kaldæerne! Bliv i Landet og underkast eder Babels Konge, så vil det gå eder vel!"
\par 25 Men i den syvende Måned kom Jisjmael, Netanjas Søn, Elisjamas Sønnesøo, en Mand af kongelig Byrd, med ti Mænd og slog Gedalja ihjel tillige med de Judæere og Kaldæere, der var hos ham i Mizpa.
\par 26 Da brød hele Folket, store og små, og Hærførerne op og drog til Ægypten; thi de frygtede for Kaldæerne.
\par 27 I det syv og tredivte År efter Kong Jojakin af Judas Bortførelse på den syv og tyvende Dag i den tolvte Måned tog Babels Konge Evil-Merodak, der i det År kom på Tronen, Kong Jojakin af Juda til Nåde og førte ham ud af Fængselet.
\par 28 Han talte ham venligt til og gav ham Sæde oven for de Konger, der var hos ham i Babel.
\par 29 Jojakin aflagde sin Fangedragt og spiste daglig hos ham, så længe han levede.
\par 30 Han fik sit daglige Underhold af Kongen, hver Dag hvad han behøvede for den Dag, så længe han levede.



\end{document}