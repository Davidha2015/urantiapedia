\begin{document}

\title{Første Krønikebog}


\chapter{1}

\par 1 Adam, Set Enosj,
\par 2 Kenan, Mahalal'el, Jered,
\par 3 Enok, Metusalem, Lemek,
\par 4 Noa, Sem, Kam og Jafet.
\par 5 Jafets Sønner: Gomer, Magog, Madaj, Javan, Tubal, Mesjek og Tiras.
\par 6 Gomers Sønner: Asjkenaz, Rifat og Togarma.
\par 7 Javans Sønner: Elisja, Tarsis, Kitæerne og Rodosboerne.
\par 8 Kams Sønner: Kusj, Mizrajim, Put og Kana'an.
\par 9 Kusj' Sønner: Seba, Havila, Sabta, Ra'ma og Sabteka. Ra'mas Sønner: Saba og Dedan.
\par 10 Og Kusj avlede Nimrod, som var den første Storhersker på Jorden.
\par 11 Mizrajim avlede Luderne, Anamerne, Lehaberne, Naftuherne,
\par 12 Patruserne, Kasluherne, fra hvem Filisterne udgik, og Kaftorerne.
\par 13 Kana'an avlede Zidon, hans førstefødte, og Het,
\par 14 Jebusiterne, Amoriterne, Girgasjiterne,
\par 15 Hivviterne, Arkiterne, Siniterne,
\par 16 Arvaditerne, Zemariterne og Hamatiterne.
\par 17 Sems Sønner: Elam, Assur, Arpaksjad, Lud og Aram. Arams Sønner: Uz, Hul, Geter og Masj.
\par 18 Arpaksjad avlede Sjela; Sjela avlede Eber;
\par 19 Eber fødtes der to Sønner; den ene hed Peleg, thi på hans Tid adsplittedes Jordens Befolkning, og hans Broder hed Joktan.
\par 20 Joktan avlede Almodad, Sjelef, Hazarmavet, Jera,
\par 21 Hadoram, Uzal, Dikla,
\par 22 Ebal, Abimael, Saba,
\par 23 Ofir, Havila og Jobab. Alle disse var Joktans Sønner.
\par 24 Sems Sønner: Arpaksjad, Sjela,
\par 25 Eber, Peleg, Re'u,
\par 26 Serug, Nakor, Tara
\par 27 og Abram, det er Abraham.
\par 28 Abrahams Sønner: Isak og Ismael.
\par 29 Dette er deres Slægtebog: Ismaels førstefødte Nebajot, dernæst Kedar, Adbe'el, Mibsam,
\par 30 Misjma, Duma, Massa, Hadad, Tema,
\par 31 Jetur, Nafesj og Kedma. Det var Ismaels Sønner.
\par 32 De Sønner, som Abrahams Medhustru Hetura fødte: Zimran, Joksjan, Medan, Midjan, Jisjbak og Sjua. Joksjans Sønner: Saba og Dedan.
\par 33 Midjans Sønner: Efa, Efer, Hallok, Ahida og Elda'a. Alle disse var Keturas Sønner.
\par 34 Abraham avlede Isak. Isaks Sønner: Jakob og Esau.
\par 35 Esaus Sønner: Elifaz, Re'uel, Je'usj, Jalam og Kora.
\par 36 Elifaz's Sønner: Teman, Omar, Zef, Gatam, Kenaz, Timna og Amalek.
\par 37 Re'uels Sønner: Nahat, Zera, Sjamma og Mizza.
\par 38 Se'irs Sønner: Lotan, Sjobal, Zib'on, Ana, Disjon, Ezer og Disjan.
\par 39 Iotans Sønner: Hori og Hemam; og Lotans Søster var Timna.
\par 40 Sjobals Sønner: Alvan, Manahat, Ebal, Sjeft og Onam. Zib'ons Sønner: Ajja og Ana.
\par 41 Anas Sønner: Disjon. Disjons Sønner: Hemdan. Esjban, Jitran og Keran.
\par 42 Ezers Sønner: Bilhan, Za'avan og Akan. Disjans Sønner: Uz og Aran.
\par 43 Følgende var de Konger, der herskede i Edoms, Land, før Israeliterne fik Konger: Bela, Beors Søn; hans By hed Dinhaba.
\par 44 Da Bela døde, blev Jobab, Zeras Søn fra Bozra, Konge i hans Sted.
\par 45 Da Jobab døde, blev Husjam fra Temaniternes Land Konge i hans Sted.
\par 46 Da Husjam døde, blev Hadad, Bedads Søn, Konge i hans Sted; det var ham, der slog Midjaniterne på Moabs Slette; hans By hed Avit.
\par 47 Da Hadad døde, blev Samla fra Masreka Konge i hans Sted.
\par 48 Da Samla døde, blev Sja'ul fra Rehobot ved Floden Konge i hans Sted.
\par 49 Da Sja'ul døde, blev Ba'al-Hanan, Akbors Søn, Konge i hans Sted.
\par 50 Da Ba'al-Hanan døde, blev Hadad Konge i hans Sted; hans By hed Pa'i, og hans Hustru hed Mehetab'el, en Datter af Matred, en Datter af Mezahab.
\par 51 DaHadad døde, fremtrådte Edoms Stammehøvdinger: Høvdingerne Timna, Alja, Jetet,
\par 52 Oholibama, Ela, Pioon,
\par 53 Kenaz, Teman, Mibzar,
\par 54 Magdiel og Iram. Det var Edoms Stammehøvdinger.

\chapter{2}

\par 1 Israels Sønner var følgende: Ruben, Simeon, Levi, Juda, Issakar, Zabulon,
\par 2 Dan, Josef, Benjamin, Naftali, Gad og Aser.
\par 3 Judas Sønner: Er, Onan og Sjela; disse tre fødtes ham af Kana'anæer kvinden Batsjua. Men Er, Judas førstefødte, var HERREN imod, og han lod ham dø.
\par 4 Derpå fødte Judas Sønnekone Tamar ham Perez og Zera, så at Judas Sønner i alt var fem.
\par 5 Perez's Sønner: Hezron og Hamul.
\par 6 Zeras Sønner: Zimri, Etan, Heman, Kalkol og Darda, i alt fem.
\par 7 Karmis Sønner: Akar, der styrtede Israel i Ulykke, idet han forgreb sig på det Gods, der var lagt Band på.
\par 8 Etans Sønner: Azarja.
\par 9 Hezrons Sønner, som fødfes ham: Jerame'el, Ram og Kelubaj.
\par 10 Ram avlede Amminadab; Amminadab avlede Nahasjon, Judæernes Øverste;
\par 11 Nahasjon avlede Salma; Salma avlede Boaz;
\par 12 Boaz avlede Obed; Obed avlede Isaj;
\par 13 Isaj avlede sin førstefødte Eliab, sin anden Søn Abinadab, sin tredje Søn, Sjim'a,
\par 14 sin fjerde Søn Netan'el, sin femte Søn Raddaj,
\par 15 sin sjette Søn Ozem og sin syvende Søn David;
\par 16 deres Søstre var Zeruja og Abigajil. Zerujas Sønner: Absjaj, Joab og Asa'el, tre.
\par 17 Abigajil fødte Amasa, hvis Fader var Ismaeliten Jeter.
\par 18 Hezrons Søn Kaleb avlede med sin Hustru Azuba Jeriot; og hendes Sønner var følgende: Jesjer, Sjobab og Ardon.
\par 19 Da Azuba døde, ægtede Kaleb frat, som fødte ham Hur.
\par 20 Hur avlede Uri, og Uri avlede Bezal'el.
\par 21 Derefter gik Hezron ind til Gileads Fader Makirs Datter, som han ægtede, da han var tredsindstyve År gammel, og hun fødte ham Segub.
\par 22 Segub avlede Ja'ir, som ejede tre og tyve Byer i Gileads Land.
\par 23 Men Gesjur og Aram fratog dem Ja'irs Teltbyer, Kenat med Småbyer, tresindstyve Byer. Alle disse var Gileads Fader Makirs Sønner.
\par 24 Efter Hezrons Død gik Kaleb ind til sin Fader Hezrons Hustru Efrata, og hun fødte ham Asjhur, der blev Fader til Tekoa.
\par 25 Jerame'els, Hezrons førstefødtes, Sønner: Ram, den førstefødte, dernæst Buna, Oren og Ozem, hans Brødre.
\par 26 Og Jerame'el havde en anden Hustru ved Navn Atara, som var Moder til Onam.
\par 27 Rams, Jerame'els førstefødtes, Sønner: Ma'az, Jamin og Eker.
\par 28 Onams Sønner: Sjammaj og Jada. Sjammajs Sønner: Nadab og Abisjur.
\par 29 Abisjurs Hustru hed Abihajil; hun fødte ham Aban og Molid.
\par 30 Nadabs Sønner: Seled og Appajim; Seled døde barnløs.
\par 31 Appajims Sønner: Jisj'i. Jisj'is Sønner: Sjesjan. Sjesjans Sønner: Alaj.
\par 32 Sjammajs Broder Jadas Sønner: Jeter og Jonatan. Jeter døde barnløs.
\par 33 Jonatans Sønner: Pelet og Zaza. Det var Jerame'els Efterkommere.
\par 34 Sjesjan havde kun Døtre, ingen Sønner. Men Sjesjan havde en ægyptisk Træl ved Navn Jarha,
\par 35 og Sjesjan gav sin Træl Jarha sin Datter til Ægte, og hun fødte ham Attaj.
\par 36 Attaj avlede Natan; Natan avlede Zabad;
\par 37 Zabad avlede Eflal; Eflal avlede Obed;
\par 38 Obed avlede Jehu; Jehu avlede Azarja;
\par 39 Azarja avlede Helez; Helez avlede El'asa;
\par 40 El'asa avlede Sismaj; Sismaj avlede Sjallum;
\par 41 Sjallum avlede Jekamja; Jekamja avlede Elisjama.
\par 42 Jerame'els Broder Kalebs Sønner: Maresja, hans førstefødte, som var Fader til Zif. Maresjas Sønner: Hebron.
\par 43 Hebrons Sønner: Kora, Tappua, Rekem og Sjema.
\par 44 Sjema avlede Raham, der var Fader til Jorke'am. Rekem avlede Sjammaj.
\par 45 Sjammajs Søn var Maon, som var Fader til Bet-Zur.
\par 46 Kalebs Medhustru Efa fødte Karan, Moza og Gazez; Karan avlede Gazez.
\par 47 Jadajs Sønner: Regem, Jotam, Gersjan, Pelet, Efa og Sja'af.
\par 48 Halebs Medhustru Ma'aka fødfe Sjeber og Tirhana.
\par 49 Sja'af, Madmannasader, avlede Sjeva, Makbenas Fader og Gibeas Fader.
\par 50 Det var Halebs Efterkommere. Hurs, Efratas førstefødtes, Sønner: Sjobal, Kirjat-Jearims Fader,
\par 51 Salma, Betlehems Fader, og Haref, Bef-Gaders Fader.
\par 52 Sjobal, Birjat-Jearims Fader, havde følgende Sønner: Reaja, Halvdelen af Manahatiterne.
\par 53 Kirjat-Jearims Slægter: Jitriterne, Putiterne, Sjumatiterne og Misjraiterne; fra dem udgik Zor'atiterne og Esjtaoliterne.
\par 54 Salmas Sønner: Betlehem, Netofatiterne, Atarot-Bet-Joab, Halvdelen af Manahatiterne og Zor'iterne.
\par 55 De i Jabez bosatte skriftlærdes Slægter: Tir'atiterne, Sjim'atiterne og Sukatiterne, det er Kiniterne, som nedstammede fra Hammat, Rekabs Slægts Fader.

\chapter{3}

\par 1 Davids Sønner, som fødtes ham i Hebron, var følgende: Ammon, den førstefødte, som han havde med Ahinoam fra Jizre'el, den anden Daniel, med Abigajil fra Karmel,
\par 2 den tredje Absalom, en Søn af Ma'aka, Kong Talmaj af Gesjurs Datter, den fjerde Adonija, Haggits Søn,
\par 3 den femte Sjefafja, som han havde med Abital, den sjette Jitream, som han havde med sin Hustru Egla.
\par 4 Seks fødtes ham i Hebron, hvor han herskede syv År og seks Måneder. Tre og tredive År herskede han i Jerusalem.
\par 5 Følgende fødtes ham i Jerusalem: Sjim'a, Sjobab, Natan og Salomo, hvilke fire han havde med Ammiels Datter Batsjua;
\par 6 fremdeles Jibhar, Elisjama, Elifelet,
\par 7 Noga, Nefeg, Jafia,
\par 8 Elisjama. Be'eljada og Elifelet, i alt ni.
\par 9 Det var alle Davids Sønner foruden Medhustruernes Sønner; og Tamar var deres Søster.
\par 10 Salomos Søn Rehabeam, hans Søn Abija, hans Søn Asa, hans Søn Josafat,
\par 11 hans Søn Joram, hans Søn Ahazja, hans Søn Joas,
\par 12 hans Søn Amazja, hans Søn Azarja, hans Søn Jotam,
\par 13 hans Søn Akaz, hans Søn Ezekias, hans Søn Manasse,
\par 14 hans Søn Amon, hans Søn Josias.
\par 15 Josias's Sønner: Johanan, den førstefødte, den anden Jojakim, den tredje Zedekias, den fjerde Sjallum.
\par 16 Jojakims Sønner: Hans Søn Jekonja, hans Søn Zedekias.
\par 17 Den fængslede Jekonjas Sønner: Hans Søn Sjealtiel,
\par 18 Malkiram, Pedaja, Sjen'azzar, Jekamja, Hosjama og Nedabja.
\par 19 Pedajas Sønner: Zerubbabel og Sjim'i. Zerubbabels Sønner: Mesjullam og Hananja og deres Søster Sjelomit.
\par 20 Mesjullams Sønner: Hasjuba, Ohel, Berekja, Hasadja og Jusjab Hesed, fem.
\par 21 Hananjas Sønner: Pelatja, Jesja'ja, Refaja, Arnan, Obadja og Sjekanja.
\par 22 Sje'kanjas Sønner: Sjemaja, Hattusj, Jig'al, Baria, Nearja og Sjafat, seks.
\par 23 Nearjas Sønner: Eljoenaj, Hizkija og Azrikam, tre.
\par 24 Eljoenajs Sønner: Hodavja, Eljasjib, Pelaja, Akkub, Johanan, Delaja og Anani, syv.

\chapter{4}

\par 1 Judas Sønner: Perez, Hezron, Karmi, Hur og Sjobal.
\par 2 Sjobals Søn Reaja avlede Jahat; Jahat avlede Ahumaj og Lahad. Det var Zor'atitemes Slægter.
\par 3 Etams Fader Hurs Sønner var følgende: Jizre'el, Jisjma og Jidbasj; deres Søster hed Hazlelponi;
\par 4 og Penuel, Gedors Fader, og Ezer, Husjas Fader; det var Efratas' førstefødte Hurs, Betlehems Faders, Sønner.
\par 5 Asjhur, Tekoas Fader, havde to Hustruer: Hel'a og Na'ara.
\par 6 Na'ara fødte ham Ahuzzam, Hefer, Teme'ni og Ahasjtariteme; det var Na'aras Sønner.
\par 7 Hel'as Sønner: Zeret, Zohar, Etnan og Koz.
\par 8 Koz avlede Anub, Hazzobeba og Aharhels "Harums Søns, Slægter.
\par 9 Jabez var mere anset end sine Brødre. Hans Moder havde givet ham Navnet Jabez, idet hun sagde: "Jeg har født ham med Smerte!
\par 10 Jabez påkaldte Israels Gud således: "Gid " du vilde velsigne mig rigeligt og gøre mit Område stort, lade din Hånd være med mig og fri mig fra Ulykke, så der ikke voldes mig Smerte! Og Gud gav ham alt, hvad han bad om.
\par 11 Kelub, Sjuhas Broder, avlede Mehir, det er Esjtons Fader.
\par 12 Esjton avlede Bet-Rafa, Pasea og Tehinna, Fader til Nahasjs By; det er Mændene fra Reka.
\par 13 Kenaz's Sønner: Otniel og Seraja. Otniels Sønner: Hatat og Meonotaj.
\par 14 Meonotaj avlede Ofra. Seraja avlede Joab, Fader. til Ge-Harasjim; de var nemlig Håndværkere.
\par 15 Jefunnes Søn kalebs Sønner: Ir, Ela og Na'am, Elas Sønner. og Kenaz.
\par 16 Perez's Sønner: Jehallel'el og Ezra. Je'hallel'els Sønner: Zif, Zifa, Tireja og Asar'el.
\par 17 Ezras Sønner: Jeter, Mered og Efer. Jeter avlede Mirjam, Sjammaj og Jisjba, Esjtemoas Fader.
\par 18 Hans judæiske Hustru fødte Jered, Gedors Fader, Heber, Sokos Fader, og Jekutiel, Zanoas Fader.
\par 19 Sønnerne af Faraos Datter Bitja, som Mered ægtede, var følgende: Nahams Søsters, Sønner var følgende: Garmiten og Ma'akafiten Esjtemoa.
\par 20 Sjimons Sønner: Amnon og Rinna, Benhanan og Tilon. Jisj'is Sønner: Zohet.
\par 21 Judas Søn Sjelas Sønner: Er, Lekas Fader, Lada, Maresjas Fader Linnedvæveriets Slægter af Asjbeas Hus,
\par 22 Jokim, Kozebas Mænd og Joasj og Saraf, som herskede over Moab og vendte tilbage til Betlehem. Det er jo gamle Begivenheder.
\par 23 Dette er Pottemagerne og Beboerne i Netaim og Gedera; de boede der i Kongens Nærhed og stod i hans Tjeneste.
\par 24 Simeons Sønner: Nemuel, Ja'min, Jarib, Zera og Sja'ul
\par 25 hans Søn Sjallum, hans Søn Mibsam, hans Søn Misjma.
\par 26 Misjmas Sønner: Hans Søh Hammuel, hans Søn Zakkur, hans Søn Sjim'i.
\par 27 Sjim'i havde seksten Sønner og seks Døtre; men hans Brødre havde ikke mange Sønner, og deres hele Slægt blev ikke så talrig som Judæerne.
\par 28 De boede i Be'ersjeba Molada, Hazar-Sjual,
\par 29 Bilha, Ezem, Tolad,
\par 30 Betuel, Horma, Ziklag,
\par 31 Bet-Markabot, Hazar-Susim, Bet-Bir'i og Sja'arajim - det var indtil Davids Regering deres Byer
\par 32 med Landsbyer - fremdeles Etam, Ajin, Rimmon, Token og Asjan, fem Byer;
\par 33 desuden alle deres Landsbyer, som lå rundt om disse Byer indtil Ba'al. Det var deres Bosteder; og de havde deres egen Slægtebog.
\par 34 Fremdeles: Mesjobab, Jamlek, Amazjas Søn Josja,
\par 35 Joel, Jehu, en Søn af Josjibja, en Søn af Seraja, en Søn af Asiel,
\par 36 Eljoenaj, Ja'akoba, Jesjohaja, Asaja, Adiel, Jesimiel, Benaja
\par 37 og Ziza, en Søn af Sjif'i, en Søn af Allon, en Søn af Jedaja, en Søn af Sjimri, en Søn af Sjemaja;
\par 38 de her ved Navn nævnte var Øverster i deres Slægter, efter at deres Fædrenehuse havde bredt sig stærkt.
\par 39 Da de engang drog i Retning af Gerar østen for Dalen for at søge Græsning til deres Småkvæg,
\par 40 fandt de fed og god Græsning, og Landet var udstrakt, og der var Fred og Ro, da de tidligere Beboere nedstammede fra Kam.
\par 41 I Kong Ezekias af Judas Dage drog de her ved Navn nævnte hen og overfaldt deres Telte og slog Me'uniterne, som de traf der, og de lagde Band på dem, så de nu ikke mere er til; derefter bosate de sig i deres Land, da der var Græsning til deres Småkvæg.
\par 42 Af dem, af Simeoniterne, drog 500 Mand til Se'irs Bjerge under Ledelse af Pelatja, Nearj'a, Refaja og Uzziel, Jisj'is Sønner,
\par 43 og de nedhuggede de sidste Amalekiter, der var tilbage; og de bosatte sig der og bor der den Dag i Dag.

\chapter{5}

\par 1 Rubens, Israels førstefødtes, Sønner - han var nemlig den førstefødte, men da han vanærede sin Faders Leje, gaves hans Førstefødselsret til Israels Søn Josefs Sønner, dog ikke således, at de i Slægtebogen opføres som førstefødte;
\par 2 thi Juda herskede over sine Brødre, og af hans Midte skulde Fyrsten tages, men Førstefødselsretten blev Josefs
\par 3 Rubens, Israels førstefødtes, Sønner: Hanok, Pallu, Hezron og Karmi.
\par 4 Joels Sønner: Hans Søn Sjemaja, hans Søn Gog, hans Søn Sjim'i
\par 5 hans Søn Mika, hans Søn Reaja, hans Søn Ba'al
\par 6 hans Søn Be'era, hvem Assyrerkongen Tillegat-Pilneser førte i Landflygtighed, da han var Rubeniternes Øverste.
\par 7 Og hans Brødre efter deres Slægter, somde var optegnede i Slægtebogen efter deres Nedstamning Først Je'iel, dernæst Zekarja
\par 8 og Bela, en Søn af Azaz, en Søn af Sjema, en Søn af Joel, som boede i Aroer og hen til Nebo og Ba'al-Meon;
\par 9 og mod Øst nåede det Område, hvor han boede, hen imod Ørkenegnene, der strækker sig over mod Eufratfloden; thi de havde talrige Hjorde i Gileads Land.
\par 10 I Sauls Dage førte de Krig med Hagriterne, og disse faldt i deres Hånd; så bosatte de sig i deres Telte på hele Gileads Østside.
\par 11 Gads Sønner, som boede lige over for dem i Basans Land indtil Salka:
\par 12 Først Joel, dernæst Sjafam, Ja'naj og Sjafat i Basan;
\par 13 og deres Brødre efter deres Fædrenehuse: Mikael, Mesjullam, Sjeba, Joraj, Jakan, Zia og Eber, syv.
\par 14 De var Sønner af Abihajil, en Søn af Huri, en Søn af Jaroa, en Søn af Gilead, en Søn af Mikael, en Søn af Jesjisjaj, en Søn af Jado, en Søn af Buz.
\par 15 Ahi, en Søn af Abdiel, en Søn af Guni, var Overhoved for deres Fædrenehuse.
\par 16 De boede i Gilead, i Basan og Småbyerne det og i alle Sirjons Græsgange, så langt de strækker sig.
\par 17 De indførtes alle i Slægtebog i Kong Jotam af Judas og Kong Jeroboam af Israels Dage.
\par 18 Rubeniterne, Gaditerne og Manasses halve Stamme, alle de krigsdygtige Mænd, der har Skjold og Sværd, spændte Bue og var øvet i Kamp, 44760 Mand, der kunde drage i Kamp,
\par 19 førte Krig med Hagriterne, Jetur, Nafisj og Nodab;
\par 20 og de fik Hjælp imod dem, så at Hagriterne og alle deres Forbundsfæller overgaves i deres Hånd, thi de råbte til Gud i Kampen, og han bønhørte dem, fordi de slog deres Lid til ham.
\par 21 Så tog de deres Hjorde, 50.000 Kameler, 250.000 Stykker Småkvæg, 2.000 Æsler og 100.000 Mennesker som Bytte.
\par 22 Der skete nemlig et stort Mandefald, thi Gud havde villet Krigen; og de boede nu i Landet i deres Sted lige til Landflygtigheden.
\par 23 Manasses halve Stammes Sønner boede i Landet fra Basan til Ba'al-Hermon, Senir og Hermonbjerget; de var talrige.
\par 24 Overhovederne for deres Fædrenehuse var følgende: Efer, Jisj'i, Eliel, Azriel, Jirmeja, Hodavja og Jadiel, dygtige Krigere og navnkundige Mænd, Overhoveder for deres Fædrenehuse.
\par 25 Men de var deres Fædres Gud utro og bolede med de Guder, der dyrkedes af Landets Folkeslag, som Gud havde udryddet foran dem.
\par 26 Da æggede Israels Gud Assyrerkongerne Puls og Tillegat-Pilnesers Sind, så han slæbte dem bort, Rubeniterne, Gaditerne og Manasses halve Stamme, og bragte dem til Hala, Habor, Hara og Gozan-floden, hvor de er den Dag i Dag.

\chapter{6}

\par 1 Levis Sønner: Gerson, Kehat og Merari.
\par 2 Kehats Sønner: Amram, Jizhar, Hebron og Uzziel.
\par 3 Amrams Børn: Aron, Moses og Mirjam. Arons Sønner: Nadab, Abihu, Eleazar og Itamar.
\par 4 Eleazar avlede Pinehas; Pinehas avlede Abisjua;
\par 5 Abisjua avlede Bukki; Bukki avlede Uzzi;
\par 6 Uzzi avlede Zeraja; Zeraja avlede Merajot;
\par 7 Merajot avlede Amarja; Amarja avlede Ahitub;
\par 8 Ahitub avlede Zadok; Zadok avlede Ahima'az;
\par 9 Ahima'az avlede Azarja; Azarja avlede Johanan;
\par 10 Johanan avlede Azarja, der var Præst i det Tempel, Salomo byggede i Jerusalem;
\par 11 Azarja avlede Amarja; Amarja avlede Ahitub;
\par 12 Ahitub avlede Zadok; Zadokavlede Sjallum;
\par 13 Sjallum avlede Hilkija; Hilkija avlede Azarja;
\par 14 Azarja avlede Seraja; Seraja avlede Jozadak,
\par 15 Men Jozadak drog med, da HERREN lod Juda og Jerusalem føre i Landflygtighed af Nebukadnezar.
\par 16 Levis ønner: Gerson, Kehat og Me'rari.
\par 17 Navnene på Gersons Sønner var følgende: Libni og Sjim'i.
\par 18 Kehats Sønner: Amram, Jizhar, Hebron og Uzziel.
\par 19 Meraris Sønner: Mali og Nusji. Det er Leviternes Slægter efter deres Fædrenehuse.
\par 20 Fra Gerson nedstammede: Hans Søn Libni, hans Søn Jahat, hans Søn Zimma,
\par 21 hans Søn Joa, hans Søn Iddo, hans Søn Zera og hans Søn Jeateraj.
\par 22 Kehats Sønner: Hans Søn Amminadab, hans Søn Kora, hans Søn Assir,
\par 23 hans Søn Elkana, hans Søn Ebjasaf, hans Søn Assir,
\par 24 hans Søn Tahat, hans Søn Uriel, hans Søn Uzzija og hans Søn Sja'ul.
\par 25 Elkanas Sønner: Amasaj og Ahimot,
\par 26 hans Søn Elkana, hans Søn Zofaj, hans Søn Tohu,
\par 27 hans Søn Eliab, hans Søn Jeroham,hans SønElkana oghans Søn Samuel.
\par 28 Samoels Sønner: Joel, den førstefødte, og den anden Abija.
\par 29 Meraris Sønner: Mali, hans Søn Libni, hans Søn Sjim'i, hans Søn Uzza,
\par 30 hans Søn Sjim'a, hans Søn Haggija og hans Søn Asaja.
\par 31 Følgende er de, hem David overdrog Sangen i HERRENs Hus, efter at Arken havde fået et Hvilested,
\par 32 ogsom gjorde Tjeneste foran Åbenbaringsteltets Bolig som Sangere, indtil Salomo byggede HERRENs Hus i Jerusalem; de udførte deres Tjeneste efter de Forskrifter, der var dem givet.
\par 33 De, som udførfe denne Tjeneste, og deres Sønner var følgende: Af Hehatiterne Sangeren Heman, en Søn af Joel, en Søn af Samuel,
\par 34 en Søn af Elkana, en Søn af Jeroham, en Søn af Eliel, en Søn af Toa,
\par 35 en Søn af Zuf, en Søn af Elkana, en Søn af Mahat, en Søn af Amasaj,
\par 36 en Søn af Elkaoa, en Søn af Joel, en Søn af Azarja, en Søn af Zefanja,
\par 37 en Søn af Tahat, en Søn af Assir, en Søn af Ebjasaf, en Søn af Hora,
\par 38 en Søn af Jizhar, en Søn af Kehat, en Søn af Levi, en Søn af Israel.
\par 39 Hans Broder Asaf, der havde Plads til højre for ham: Asaf, en Søn af Berekja, en Søn af Sjim'a,
\par 40 en Søn af Mikael, en Søn af Ba'aseja, en Søn af Malkija,
\par 41 en Søn af Etni, en Søn af Zera, en Søn af Adaja,
\par 42 en Søn af Etan, en Søn af Zimma, en Søn af Sjim'i,
\par 43 en Søn af Jahat, en Søn af Gerson, en Søn af Levi.
\par 44 Deres Brødre, Meraris Sønner, der havde Plads til venstre: Etan, en, Søn af Kisji en Søn af Abdi, en af Malluk,
\par 45 en Søn af Hasjabja, en Søn af Amazja, en Søn af Hilkija,
\par 46 en Søn af Amzi, en Søn af Bani, en Søn af Sjemer,
\par 47 en Søn af Mali en Søn af Musji, en Søn af Merari, en Søn af Levi.
\par 48 Deres Brød Leviterne var pligtige at gøre alt Arbejdet ved Guds Hus's Bolig;
\par 49 men Aron og hans Sønner ofrede Røgofre på Brændofferalteret og Røgofferalteret, de udførte alt Arbejde i det Allerhelligste og skaffede Israel Soning, ganske som Guds Tjener Moses havde påbudt.
\par 50 Arons Sønner var følgende: Hans Søn Eleazar, hans Søn Pinehas, hans Søn Abisjua,
\par 51 hans Søn Bukki, hans Søn Uzzi, hans Søn Zeraja,
\par 52 hans Søn Merajot, hans Søn Amarja, hans Søn Ahitub,
\par 53 hans Søn Zadok og hans Søn Ahima'az.
\par 54 Deres Boliger, deres Teltlejre i deres Område var følgende: Arons Sønner af Kehatiternes Slægt - thi, for dem faldt Loddet først
\par 55 gav man Hebron i Judas Land med tilhørende Græsmarker;
\par 56 men Byens Landområde og Landsbyer gav, man Haleb, Jefunnes Søn.
\par 57 Arons Sønner gav, man Tilflugtsbyen Hebron, Libna.. med Græsmarker.
\par 58 Hilen med Græsmarker, Debir med Græsmarker,
\par 59 Asjan med Græsmarker, Jutta med Græsmarker og Bet-Sjemesj med Græsmarker:
\par 60 Af Benjamins Stamme: Gibeon med Græsmarker, Geba med Græsmarker, Alemet med Græsmarker og Anatot med Græsmarker. I alt tretten Byer med Græsmarker.
\par 61 Kehats øvrige Sønner tilfaldt efter deres Slægter ved Lodkastning ti Byer af Efraims og Dans Stammer og Manasses halve Stamme.
\par 62 Gersons Sønner tilfaldt efter deres Slægter tretten Byer af Issakars, Asers og Naftalis Stammer og Manasses halve Stamme i Basan.
\par 63 Meraris Sønner tilfaldt efter deres Slægter ved Lodkastning tolv Byer af Rubens, Gads og Zebulons Stammer.
\par 64 Så gav Israeliterne Leviterne Byerne med Græsmarker.
\par 65 De gav dem ved Lodkastning af Judæernes, Simeoniternes og Benjaminiternes Stammer de ovenfor nævnte Byer.
\par 66 Kehatiternes Slægter fik de dem ved Lodkastning tildelte Byer af Efraims Stamme;
\par 67 man gav dem Tilflugtsbyen Sikem med Græsmarker i Efraims Bjerge, Gezer med Græsmarker,
\par 68 Jokmeam med Græsmarker. Bet-Horon med Græsmarker,
\par 69 Ajjalon med Græsmarker og Gat: Rimmon med Græsmarker;
\par 70 af Manasses halve Stamme Aner med Græsmarker og Jibleam med Græsmarker; det tilfaldt de øvrige Hehatiters Slægter.
\par 71 Gersoniterne efter deres Slægter tilfaldt af den anden Halvdel af Manasses Stamme Golan i Basan med Græsmarker og Asjtarot med Græsmarker;
\par 72 af Issakars Stamme Kedesj med Græsmarker, Dobrat med Græsmarker,
\par 73 Jarmut med Græsmarker og En-Gannim med Græsmarker;
\par 74 af Asers Stamme Masjal med Græsmarker, Abdon med Græsmarker,
\par 75 Hukok med Græsmarker og Rehob med Græsmarker;
\par 76 af Naftalis Stamme Bedesj i Galilæa med Græsmarker, Hammot med Græsmarker og Kirjatajim med Græsmarker.
\par 77 De øvrige Leviter, Merariterne, tilfaldt af Zebulons Stamme Rimmon med Græsmarker og Tabor med Græsmarker;
\par 78 og hinsides Jordan over for Jeriko, østen for Jordan, af Rubens Stamme Bezer i Ørkenen med Græsmarker, Jaza med Græsmarker,
\par 79 Kedemot med Græsmarker og Mefa'at med Græsmarker;
\par 80 af Gads Stamme Ramot i Gilead med Græsmarker, Mahanajim med Græsmarker,
\par 81 Hesjbon med Græsmarker og Ja'zer med Græsmarker.

\chapter{7}

\par 1 Issakas Sønner var: Tola, Pua, Jasjub og Sjimron, fire.
\par 2 Tolas Sønner: Uzzi, Refaja, Jeriel, Jamaj, Jibsam og Sjemuel, Overhoveder for Tolas Fædrenehuse, dygtige Krigere. Efter deres Slægtebøger udgjorde deres Tal på Davids Tid 22600.
\par 3 Uzzis Sønner: Jizraja, Jizrajas Sønner Mikael, Obadja, Joel og Jissjija, fem, alle sammen Overhoveder.
\par 4 Til dem hørte efter deres Slægtebøger, efter deres Fædrenehuse, krigsrustede Skarer, 36000; de havde nemlig mange kvinder og Børn.
\par 5 Deres Brødre, alle Issakars Slægter, var dygtige Krigere; de, som var indført i deres Slægtebog, udgjorde i alt 87 000.
\par 6 Benjamins Sønner: Bela, Beker og Jediael, tre.
\par 7 Belas Sønner: Ezbon, Uzzi, Uzziel Jerimot og Iri, fem, Overhoveder for Fædrenehuse, dygtige Krigere; de, som var indført i deres Slægtebog, udgjorde 22034.
\par 8 Bekers Sønner: Zemira, Joasj, Eliezer, Eljoenaj, Omri, Jeremot, Abija, Anatot og Alemet; alle disse var Bekers Sønner;
\par 9 de, som var indført i deres Slægtebog efter deres Slægter, Overhovederne for Fædrenehusene, dygtige Krigere, udgjorde 20 200.
\par 10 Jediaels Sønner: Bilhan; Bilhans Sønner: Je'usj, Binjamin, Ehud, Kena'ana, Zetan, Tarsjisj og Ahisjahar;
\par 11 alle disse var Jediaels Sønner, Overhoveder for deres Fædrenehuse, dygtige Krigere, 17200 øvede Krigere.
\par 12 Og Sjuppim og Huppim var Irs Sønner, og Husjim var Ahers Sønner
\par 13 Naftalis Sønner var: Jabaziel, Guni, Jezer og Sjallum. Bilhas Sønner:.
\par 14 Manasses Sønner, som hans aramaiske Medhustru fødte: Hun fødte Makir, Gileads Fader.
\par 15 Gilead ægtede en Kvinde ved Navn Ma'aka; hans Søster hed Hammoleket, og hans Broder hed Zelobad; Zelofbad havde kun Døtre.
\par 16 Gileads Hustru Ma'aka fødte en Søn, som hun kaldte Peresj, medens hans Broder hed Sjeresj; hans Sønnervar Ulam og Rekem.
\par 17 Ulams Sønner: Bedan. Det var Sønner af Gilead, en Søn af Makir, en Søn af Manasse.
\par 18 Hans Søster Hammoleket fødte Isjhod, Abiezer og Mala.
\par 19 Sjemidas Sønner: Ajan, Sjekm, Likhi og Ani'am.
\par 20 Efraims Sønner: Sjutela hans Søn Bered, hans Søn Tahat, hans Søn El'ada, hans Sønahat,
\par 21 hans Søn Zabad, hans Søn Sjutela - og Ezer og El'ad. Dem dræbte Mændene fra Gat, de indfødfe i Landet, fordi de var draget ned for at tage deres Hjorde.
\par 22 Deres Fader Efraim sørgede i lang Tid over dem, og hans Brødre kom for at trøste ham.
\par 23 Så gik han ind til sin Hustru, og hun blev frugtsommelig og fødte en Søn, som han kaldte Beri'a, fordi hans Hus var i Ulykke, da det skete.
\par 24 Hans Datter var Sje'era, som byggede Nedre- og Øvre-Bet-Horon og Uzzen-Sje'era.
\par 25 Hans Søn var Refa, hans Søn Resjef, hans Søn Tela, hans Søn Tahan,
\par 26 hans Søn Ladan, hans Søn Ammihud, hans Søn Elisjama,
\par 27 hans Søn Nun, hans Søn Josua.
\par 28 Deres Besiddelser og Boliger var Betel med Småbyer, mod Øst Na'aran, mod Vest Gezer med Småbyer, fremdeles Sikem med Småbyer indtil Ajja med Småbyer;
\par 29 langs Manassiternes Grænse Bet-Sjean med Småbyer, Ta'anak med Småbyer, Megiddo med Småbyer og Dor med Småbyer. I dem boede Israels Søn Josefs Sønner.
\par 30 A Sønner: Jimna, Jisjva, Jisjvi og Beri'a og deres Søster Sera.
\par 31 Beri'as Sønner: Heber og Malkiel, som var Birzajits Fader.
\par 32 Heber avlede Jaflet, Sjomer, Hotam og deres Søster Sjua.
\par 33 Jaflets Sønner: Pasak, Bimhal og Asjvat; det var Jaflets Sønner.
\par 34 Sjemers Sønner: Ahi, Roga, Hubba og Aram.
\par 35 Hans Broder Hotams Sønner: Zofa, Jimna, Sjelesj og Amal.
\par 36 Zofas Sønner: Sua, Harnefer, Sjual, Beri, Jimra,
\par 37 Bezer, Hod, Sjamma, Sjilsja, Jitran og Be'era.
\par 38 Jeters Sønner: Jefunne, Pispa og Ara.
\par 39 Ullas Sønner: Ara, Hanniel og Rizja.
\par 40 Alle disse var Asers Sønner. Overhoveder for Fædrenehusene, udsøgte dygtige Krigere, Overhoveder for Øversterne. De, der var indført i Slægtebogen som brugelige til Krigstjeneste, talte 26000.

\chapter{8}

\par 1 Benjamin avlede Bela, den førstfødte, Asjbel den anden, Ahiram den tredje,
\par 2 Noha den fjerde og Rafa den femte.
\par 3 Bela havde Sønner: Ard, Gera, Ehuds Fader,
\par 4 Abisjua, Na'aman, Ahoa,
\par 5 Gera, Sjefufan og Hufam.
\par 6 Ehuds Sønner var følgende de var Overhoveder for Fædrenehusene blandt Gebas Indbyggere, men førtes bort til Manahat,
\par 7 da Na'aman, Ahija og Gerå førte dem bort -: Han avlede Uzza og Ahihud.
\par 8 Sjaharajim avlede på Moabs Slette - efter at han havde sendt sine Hustruer Husjim og Ba'ara bort
\par 9 han avlede med sin Hustru Hodesj: Jobab, Zibja, Mesja, Malkam,
\par 10 Je'uz, Sakeja og Mirma; det var hans Sønner, Overhoveder for Fædrenehuse;
\par 11 og med Husjim avlede han Abitub og Elpa'al.
\par 12 Elpa'als Sønner: Eber, Misj'am og Sjemer, som byggede Ono og Lod med Småbyer.
\par 13 Beri'a og Sjema var Overhoveder for Fædrenehusene blandt Indbyggerne i Ajjalon; det var dem, der slog Indbyggerne i Gat på Flugt.
\par 14 Deres Brødre var Elpa'al Sjasjak og Jeremot.
\par 15 Og Zebadja, Arad, Eder,
\par 16 Mikael, Jisjpa og Joha var Beri'as Sønner.
\par 17 Zebadja, Mesjullam' Hizki, Heber,
\par 18 Jisjmeraj, Jizli'a og Jobab var Elpa'als Sønner.
\par 19 Jakim, Zikri, Zabdi,
\par 20 Eljoenaj, Zilletaj, Eliel,
\par 21 Adaja, Beraja og Sjimrat var Sjim'is Sønner.
\par 22 Jisjpan, Eber, Eliel,
\par 23 Abdon, Zikri, Hanan,
\par 24 Hananja, Elam, Antotija,
\par 25 Jifdeja og Penuel var Sjasjaks Sønner.
\par 26 Sjamsjeraj, Sjeharja, Atalja,
\par 27 Ja'aresjja, Elija og Zikri var Jerohatns Sønner.
\par 28 Disse var Overhoveder for Fædrenehuse, Overhoveder efter deres Slægter; de boede i Jerusalem.
\par 29 I Gibeon boede Je'uel, Gibeons Fader, hvis Hustru hed Ma'aka;
\par 30 hans førstefødte Søn var Abdon, dernæst Zur, Kisj, Ba'al, Ner, Nadab,
\par 31 Gedor, Ajo, Zeker og Miklot.
\par 32 Miklot avlede Sjim'a. Også disse boede over for deres Brødre sammen med deres Brødre i Jerusalem.
\par 33 Ner avlede Kisj. Kisj avlede Saul. Saul avlede Jonatan, Malkisjua, Abinadab og Esjba'al.
\par 34 Jonatans Søn var Meribba'al. Meribba'al avlede Mika.
\par 35 Mikas Sønner: Piton, Melek, Ta'rea og Ahaz.
\par 36 Ahaz avlede Jehoadda. Jehoadda avlede Alemet, Azmavet og Zimri. Zimri avlede Moza.
\par 37 Moza avlede Bin'a, hans Søn var Rafa hans Søn El'asa, hans Søn Azel.
\par 38 Azel havde seks Sønner, hvis Navne var Azrikam, Bokeru, Jisjmael, Sjearja, Obadja og Hanan; alle disse var Azels Sønner.
\par 39 Hans Broder Esjeks Sønner: Ulam, den førstefødte, Je'usj den anden og Elifelet den tredje.
\par 40 Ulams Sønner var dygtige Krigere, der spændte Bue og havde mange Sønner og Sønnesønner. Alle disse var Benjamins Sønner.

\chapter{9}

\par 1 Alle Isreaeliter blev indført i Slægtebøger og findes optegnede i Israels Kongers Bog. Men Juda blev ført i Landflygtighed til Babel for sin Utroskabs Skyld.
\par 2 De tidligere Indbyggere, som levede på deres Ejendom i deres Byer: Israel, Præsterne, Leviterne og Tempeltrællene.
\par 3 I Jerusalem boede af Judæere, Benjaminiter, Efraimiter og Manassiter følgende.
\par 4 Af Judæerne: Utaj, en Søn af Ammihud, en Søn af Omri, en Søn af Imri, en Søn af Bani, en af Judas Søn Perez's Sønner.
\par 5 Af Sjelaniterne: Den førstefødte Asaja og hans Sønner.
\par 6 Af Zeraiterne: Je'uel og deres Brødre, 690.
\par 7 Af Benjaminiterne: Sallu, en Søn af Mesjullam, en Søn af Hodavja, en Søn af Hassenua;
\par 8 Jibneja, en Søn af Jeroham; Ela, en Søn af Mikris Søn Uzzi; Mesjullam, en Søn af Sjefatja, en Søn af Re'uel, en Søn af Jibneja;
\par 9 desuden deres Brødre efter deres Slægter, 956. Alle disse Mænd var Overhoveder for deres Fædrenehuse.
\par 10 Af Præsterne: Jedaja, Jojarib, Jakin,
\par 11 Azarja, en Søn af Hilkija, en Søn af Mesjullam, en Søn af Zadok, en Søn af Merajot, en Søn af Ahitub, Øversten over Guds Hus;
\par 12 Adaja, en Søn af Jeroham, en Søn af Pasjhur, en Søn af Malkija; Masaj, en Søn af Adiel, en Søn at Jazera, en Søn af Mesjullam, en Søn af Mesjillemit, en Søn af Immer;
\par 13 desuden deres Brødre, Overhovederne for deres Fædrenehuse, 1760, dygtige Mænd til Tjenesten i Guds Hus.
\par 14 Af Leviterne: Sjemaja, en Søn af Hassjub, en Søn at Azrikam, en Søn af Hasjabja af Merariterne,
\par 15 Bakbakkar, Heresj, Galal, Mattanja en Søn af Mika, en Søn af Zikri, en Søn af Asaf,
\par 16 Obadja, en Søn af Sjemaja, en Søn af Galal, en Søn af Jedutun, og Berekja, en Søn af Asa, en Søn af Elkana, som boede i Netofatiternes Landsbyer.
\par 17 Dørvogterne: Sjallum, Akkub, Talmon og Ahiman og deres Brødre; Sjallum var Overhovedet
\par 18 og har endnu sin Plads ved Kongeporten mod Øst. Det var Dørvogterne i Leviternes Lejre.
\par 19 Sjallum, en Søn af Kore, en Søn af Ebjasaf, en Søn af Hora, og hans Brødre af hans Fædrenehus, Koraiterne, havde Vagttjenesten ved Teltets Tærskler; deres Fædre havde nemlig holdt Vagt ved Indgangen til HERRENs Lejr;
\par 20 Pinehas, Eleazars Søn - HERREN være med ham! - var fordum deres Øverste;
\par 21 Mesjelemjas Søn Zekarja var Dørvogter ved Indgangen til Åbenbaringsteltet.
\par 22 Tilsammen udgjorde de, der var udvalgt til at holde Vagt ved Tærsklerne, 212. De indførtes i de res Slægtebøger i deres Landsbyer. David og Seeren Samuel indsatte dem i deres Embede;
\par 23 de og deres Sønner holdt Vagt ved Portene til HERRENs Hus, Teltboligen.
\par 24 Dørvogterne stod mod alle fire Verdenshjørner, mod Øst, Vest, Nord og Syd,
\par 25 og deres Brødre i deres Landsbyer skulde fra Tid til anden, syv Dage ad Gangen, møde for at stå dem bi;
\par 26 thi hine fire Øverster for Dørvogterne gjorde stadig Tjeneste. Det var Leviferne. Fremdeles havde de Tilsyn med Kamrene og Forrådsrummene i Guds Hus,
\par 27 og de overnattede rundt omkring Guds Hus, thi de havde det Hverv at holde Vagt, og de skulde lukke op hver Morgen.
\par 28 Nogle af dem havde Tilsyn med de til Tjenesten hørende Ting og talte dem, både når de gemte dem hen, og når de tog dem frem.
\par 29 Andre var sat til at føre Tilsyn med Tingene, alle de hellige Ting, og med Hvedemelet, Vinen, Olien, Røgelsen og de vellugtende Stoffer.
\par 30 Nogle af Præsternes Sønner lavede Salven af de vellugtende Stoffer.
\par 31 En af Leviterne, Mattitja, Koraiten Sjallums førstefødte Søn. havde det Hverv at tillave Bagværket.
\par 32 Nogle af deres Brødre Kehatiterne havde Tilsyn med Skuebrødene og skulde lægge dem til Rette hver Sabbat.
\par 33 Det var Sangerne, Overhovederne for Leviternes Fædrenehuse. De opholdt sig i Kamrene, fri for anden Gerning, da de havde Tjeneste Dag og Nat.
\par 34 Det var Overhovederne for Leviternes Fædrenehuse, Overhoveder efter deres Slægter. De boede i Jerusalem.
\par 35 I Gibeon boede Je'uel, Gibeons Fader, hvis Hustru hed Ma'aka;
\par 36 hans førstefødfe Søn var Abdon, dernæst Zur, Kisj, Ba'al, Ner, Nadab,
\par 37 Gedor, Ajo, Zekarja og Miklot.
\par 38 Miklot avlede Sjim'am. Også de boede over for deres Brødre sammen med deres Brødre i Jerusalem.
\par 39 Ner avlede Kisj. Kisj avlede Saul. Saul avlede Jonatan, Malkisjua, Abinadab og Esjba'al.
\par 40 Jonatans Søn var Meribba'al. Meribba'al avlede Mika.
\par 41 Mikas Sønner: Piton, Melek, Ta'rea og Ahaz.
\par 42 Ahaz avlede Jehoadda. Jehoadda avlede Alemet, Azmavet og Zimri. Zimri avlede Moza.
\par 43 Moza avlede Bin'a, hans Søn var Refaja, hans Søn El'asa, hans Søn Azel.
\par 44 Azel havde seks Sønner, hvis Navne var Azrikam, Bokeru, Jisjmael, Sjearja, Obadja og Hanan; det var Azels Sønner.

\chapter{10}

\par 1 Filisterne angreb Israel; og Israeals Mænd flygtede for Filisterne, og de faldne lå rundt om på Gilboas Bjerg.
\par 2 Og Filisterne forfulgte Saul og hans Sønner og dræbte Sauls Sønner, Jonatan, Abinadab og Malkisjua.
\par 3 Kampen rasede om Saul, og han blev opdaget af Bueskytterne og grebes af stor Angst for dem.
\par 4 Da sagde Saul til sin Våbendrager: "Drag dit Sværd og gennembor mig, for at ikke disse uomskårne skal komme og mishandle mig!" Men Våbendrageren vilde ikke, thi han gøs tilbage derfor. Da tog Saul Sværdet og styrtede sig i det,
\par 5 og da Våbendrageren så, at Saul var død, styrtede også han sig i sit Sværd og døde.
\par 6 Således fulgtes Saul, hans tre Sønner og hele hans Slægt i Døden.
\par 7 Men da alle Israeliteme i Dalen så, at Israels Mænd var flygtet, og at Saul og hans Sønner var faldet, forlod de deres Byer og flygtede, hvorpå Filisterne kom og besatte dem.
\par 8 Da Filisterne Dagen efter kom for at plyndre de faldne, fandt de Saul og hans Sønner liggende på Gilboas Bjerg.
\par 9 De plyndrede ham da og tog hans hoved og Våben med sig og sendte Bud rundt iilisternes Land for at bringe deres Afguder og Folket Glædesbudet.
\par 10 Våbnene lagde dei deres Guds Hus, men Hovedskallen hængte de op i Dagons Hus.
\par 11 Men da alle de, som boede i Gilead, hørte alt, hvad Filisterne havde gjort ved Saul,
\par 12 brød alle våbenføre Mænd op, og de tog Sauls og hans Sønners Lig ned, bragte dem med til Jabesj og jordede deres Ben under Terebinten i Jabesj og fastede syv Dage.
\par 13 Således døde Saul for den Utroskabs Skyld, han havde vist HERREN, fordi han ikke havde givet Agt på HERRENs Ord, også fordi han havde adspurgt en Ånd for at få et Råd
\par 14 og ikke søgt; Råd hos HERREN. Derfor lod han ham dø, og Kongemagten lod han gå over til David, Isajs Søn.

\chapter{11}

\par 1 Derpå samlede hele Israel sig hos David i Hebron og sagde: "Vi er jo dit Kød og Blod!"
\par 2 Allerede før i Tiden, medens Saul var Konge, var det dig, som førte Israel ud i Kamp og hjem igen; og HERREN din Gud sagde til dig: Du skal vogte mit Folk Israel og være Hersker over mit Folk Israel!"
\par 3 Og alle Israels Ældste kom til Kongen i Hebron, og David sluttede i Hebron Pagt med dem for HERRRNs Åsyn, og de salvede David til Konge over Israel efter HERRENs Ord ved Samuel.
\par 4 Derpå drog David med hele Israel til Jerusalem, det er Jebus; der boede Jebusiterne, Landets oprindelige Indbyggere;
\par 5 og Indbyggerne i Jebus sagde til David: "Her kan du ikke trænge ind!" Men David indtog Klippeborgen Zion, det er Davidsbyen.
\par 6 Og David sagde: "Den, der først slår en Jebusit, skal være Øverste og Hærfører!" Og da Joab, Zentjas Søn, var den første, der steg derop, blev han Øverste.
\par 7 Så tog David Bolig i Klippeborgen, hvorfor man kaldte den Davidsbyen;
\par 8 og han befæstede Byen hele Vejen rundt fra Millo af; Resten af Byen genopførte Joab.
\par 9 Og David blev mægtigere og mægtigere; Hærskarers HERRE var med ham.
\par 10 Følgende var de ypperste at Davids Helte, som sammen med hele Israel kraftig stod ham bi med at nå Kongedømmet, så de fik ham valgt til Konge efter HERRENs Ord til Israel.
\par 11 Navnene på Davids Helte var følgende: Isjba'al, Hakmonis Søn, Anføreren for de tre; det var ham, som engang svang sit Spyd over 300 faldne på een Gang.
\par 12 Blandt de tre Helte kom efter ham Ahohiten El'azar, Dodos Søn;
\par 13 han var med David ved PasDammim, dengang Filisterne samlede sig der til Kamp. Marken var fuld af Byg, og Folkene flygtedefor Filisterne;
\par 14 men han stillede sig op midt på Marken og holdt den og huggede Filisterne ned; og HERREN gav dem en stor Sejr.
\par 15 Engang drog tre af de tredive ned til David på Klippen, til Adullams Hule, medens Filisternes Hær var lejret i Refaimdalen.
\par 16 David var dengang i Hlippeborgen, medens Filisternes Foged var i Betlehem.
\par 17 Så vågnede Lysten hos David, og han sagde: "Hvem skaffer mig en Drik Vand fra Cisternen ved Betlehems Port?"
\par 18 Da banede de tre Helte sig Vej gennem Filisternes Lejr, øste Vand af Cisternen ved Betlehems Port og bragte David det. Han vilde dog ikke drikke det, men udgød det for HERREN
\par 19 med de Ord: "Gud vogte mig for at gøre det! Skulde jeg drikke de Mænds blod, som har vovet deres Liv? Thi med Fare for deres Liv har de hentet det!" Og han vilde ikke drikke det. Den Dåd udførte de tre Helte.
\par 20 Abisjaj, Joabs Broder, var Anfører for de tredive. Han svang sit Spyd over de faldne, og han var navnkundig blandt de tredive;
\par 21 iblandt de tredive var han højt æret, og han var deres Anfører; men de tre nåede han ikke.
\par 22 Benaja, Jojadas Søn, var en tapper Mand fra Kabze'el, der havde udført store Heltegeminger; han fældede de to Arielsønner fra Moab; og han steg ned og fældede en Løve i en Cisterne, en Dag den var faldet Sne.
\par 23 Ligeledes fældede han Ægypteren, en kæmpestor Mand, fem Alen høj. Ægypteren havde et Spyd som en Væverbom i Hånden, men han gik ned imod ham med en Stok, vristede Spydet ud af Hånden på ham og dræbte ham med hans eget Spyd.
\par 24 Disse Heltegerninger udførte Benaja, Jojadas Søn, og han var navnkundig iblandt de tredive Helte;
\par 25 iblandt de tredive var han højt æret, men de tre nåede han ikke. David satte ham over sin Livvagt.
\par 26 De tapre Helte var: Asa'el, Joabs Broder; Elhanan, Dodos Søn, fra Betlehem;
\par 27 Haroriten Sjammot; Peloniten Helez;
\par 28 Ira, Ikkesj's Søn, fra Tekoa; Abiezer fra Anatot;
\par 29 Husjatiten Sibbekaj; Ahohiten Ilaj;
\par 30 Maharaj fra Netofa; Heled, Ba'anas Søn, fra Netofa;
\par 31 Itaj, Ribajs Søn, fra det benjaminitiske Gibea; Benaja fra Pir'aton;
\par 32 Huraj fra Nahale-Ga'asj; Abiel fra Araba;
\par 33 Azmavet fra Bahurim.; Sja'alboniten Eljaba;
\par 34 Guniten Jasjen; Harariten Jonatan, Sjammas, Søn;
\par 35 Harariten Ahi'am, Sakars Søn; Elifal, Urs Søn;
\par 36 Mekeratiten Hefer; Peloniten Ahija;
\par 37 Hezro fra Karmel; Na'araj, Ezbajs Søn;
\par 38 Joel, Natans Broder; Mibhar, Hagritens Søn;
\par 39 Ammoniten Zelek; Naharaj fra Berot, der var Joabs, Zerujas Søns, Våbendrager;
\par 40 Ira fra Jattir; Gareb fra Jattir;
\par 41 Hetiten Urias; Zabad, Alajs Søn;
\par 42 Rubeniten Adina, Sjizas Søn, et af Rubeniternes Overhoveder over tredive;
\par 43 Hanan, Ma'akas Søn; Mitniten Josjafat;
\par 44 Uzzija fra Asjtarot; Sjama og Je'uel, Aroeriten Hotams Sønner;
\par 45 Jediael, Sjimris Søn, og hans Broder Tiziten Joha;
\par 46 Mahaviten Eliel; Jeribaj og Josjavja, Elna'ams Sønner; Moabiten Jitma;
\par 47 Eliel, Obed og Ja'asiel fra Zoba.

\chapter{12}

\par 1 Følgende er de, der kom til David i Ziklag, medens han måtte holde sig skjult for Saul, Kisj's Søn. De hørte til Heltene, som hjalp til i Kampen;
\par 2 de var væbnet med Bue og øvet i Stenkast og Pileskydning både med højre og venstre Hånd; de hørte til Sauls Brødre, Benjaminiterne.
\par 3 Deres Overhoved var Ahiezer; dernæst Joasj, Sjema'as Søn, fra Gibea, Jeziel og Pelet, Azmavets Sønner, Beraka, Jehu fra Anatot,
\par 4 Jisjmaja fra Gibeon, en Helt blandt de tredive og Høvedsmand over de tredive, Jirme'ja, Jahaziel, Johanan, Jozabad fra Gedera,
\par 5 El'uzaj, Jerimot, Bealja, Sjemarja, Sjefatja fra Harif,
\par 6 Elkana, Jissjija, Azar'el, Jo'ezer og Jasjobam, Koraiterne,
\par 7 Joela og Zebadja, Sønner af Jeroham fra Gedor.
\par 8 Af Gaditerne slutfede nogle sig dygtige Krigere, øvede Krigsmænd, væbnet med Skjold og Spyd; de var som Løver at se på og rappe som Gazellerne på Bjergene.
\par 9 Deres Overhoved var Ezer, Obadja den anden, Eliab den tredje,
\par 10 Masjmanna den fjerde, Jirmeja den femte,
\par 11 Attaj den sjette, Eliel den syvende,
\par 12 Johanan den ottende, Elzabad den niende,
\par 13 Jirmeja den tiende, Makbannaj den ellevte.
\par 14 De var Anførere blandt Gaditerne; den mindste af dem tog det op med hundrede, den største med tusind.
\par 15 Det var dem, der gik over Jordan i den første Måned, engang den overalt var gået over sine Bredder, og slog alle Dalboerne på Flugt både mod Øst og Vest.
\par 16 Engang kom nogle Benjaminiter og Judæere til David i Klippeborgen;
\par 17 David gik ud til dem, tog til Orde og sagde: "Hvis I kommer til mig i fredelig Hensigt, for at hjælpe mig, er jeg af Hjertet rede til at gøre fælles Sag med eder; men er det for at forråde mig til mine Fjender, skønt der ikke er Uret i mine Hænder, da se vore Fædres Gud til og straffe det!"
\par 18 Så iførte Ånden sig Amasaj, de tredives Anfører, og han sagde: "For dig, David, og med dig, Isajs Søn! Fred, Fred være med dig, og Fred med dine Hjælpere, thi dig hjælper din Gud!" Da tog David imod dem og satte dem i Spidsen for Krigerskaren.
\par 19 Af Manasse gik nogle over til David. Det var, dengang han sammen med Filisterne drog i Kamp mod Saul, uden at han dog blev dem til Hjælp, fordi Filisternes Fyrster efter at have holdt Rådslagning sendte ham bort, idet de sagde: "Det koster vore Hoveder, hvis han går over til sin Herre Saul!"
\par 20 Da han så drog til Ziklag, gik følgende Manassiter over til ham: Adna, Jozabad, Jediael, Mikael, Jozabad, Elibu og Zilletaj, der var Overhoveder for Slægter i Manasse;
\par 21 de hjalp siden David imod Strejfskarerne, thi de var alle dygtige Krigere og blev Førere i Hæren.
\par 22 Der kom nemlig daglig Folk til David for at hjælpe ham, så det til sidst blev en stor Hær, stor som Guds Hær.
\par 23 Følgende er Tallene på Førerne for de væbnede Krigere, der kom til David i Hebron for at gøre ham til Konge i Sauls Sted efter HERRENs Bud:
\par 24 Af Judæere, der har Skjold og Spyd, 6800 væbnede Krigere;
\par 25 af Simeoniterne 7100 dygfige Krigshelte;
\par 26 af Leviterne 4600,
\par 27 dertil Øverstenover Arons Slægt, Jojada, fulgt af 3700,
\par 28 og Zadok, en ung, dygtig Kriger, med sit Fædrenehus, 22 Førere;
\par 29 af Benjaminiterne, Sauls Brødre, 3000, men de fleste af dem holdt endnu fast ved Sauls Hus;
\par 30 af Efraimiterne 20800 dygtige Krigere, navnkundige Mænd i deres Fædrenehuse;
\par 31 af Manasses halve Stamme 18.000 navngivne Mænd, der skulde gå hen og gøre David til Konge;
\par 32 af Issakariterne, der forstod sig på Tiderne, så de skønnede, hvad Israel havde at gøre, 200 Førere og alle deres Brødre, der stod under dem;
\par 33 af Zebulon 50.000, øvede Krigere, udrustet med alle Hånde Våben, med een Vilje rede til Kamp;
\par 34 af Naftali 1.000 Førere, fulgt af 37.000 Mænd med Skjold og Spyd;
\par 35 af Daniterne 28.600, udrustede Mænd;
\par 36 af Aser 40.000, øvede Krigere, rustet til Kamp;
\par 37 fra den anden Side af Jordan, fra Rubeniterne, Gaditerne og Manasses halve Stamme, 120.000 Mænd med alle Hånde Krigsvåben.
\par 38 Med helt Hjerte og rede til Kamp kom alle disse Krigere til Hebron for at gøre David til Konge over hele Israel; men også det øvrige Israel var endrægtigt med til at gøre David til Konge.
\par 39 De blev der hos David i tre Dage og spiste og drak, thi deres Brødre havde forsynet dem;
\par 40 også de, der boede i deres Nærhed lige til Issakar, Zebulon og Naftali, bragte dem Levnedsmidler på Æsler, Kameler, Muldyr og Okser, Fødevarer af Mel, Figenkar, Rosinkager, Vin, Olie, Hornkvæg og Småkvæg i Mængde; thi der var Glæde i Israel.

\chapter{13}

\par 1 Efter at have rådført sig med Tusindførerne og Hundredeførerne, alle Øversterne,
\par 2 sagde David til hele Israels Forsamling: "Hvis det tykkes eder godt, og det er HERREN vor Guds Vilje, lad os så sende Bud til vore Brødre, der er tilbage i alle Israels Landsdele, og ligeledes til Præsterne og Leviterne i Byerne, hvor de har deres Græsmarker, at de skal samles hos os,
\par 3 for at vi kan bringe vor Guds Ark tilbage til os, thi i Sauls Dage spurgte vi ikke om den."
\par 4 Og hele Forsamlingen svarede, at det skulde man gøre, thi alt Folket fandt Forslaget rigtigt.
\par 5 Så samlede David bele Israel lige fra Sjihor i Ægypten til Egnen ved Hamat for at hente Guds Ark i Kirjat-Jearim.
\par 6 Derpå drog David og hele tsrael op til Ba'ala, til Kirjat-Jearim i Juda for der at hente Gud HERRENs Ark, over hvilken hans Navn er nævnet, han, som troner over Keruberne.
\par 7 De førte da Guds Ark bort fra Abinadabs Hus på en ny Vogn, og Uzza og Ajo kørte Vognen.
\par 8 David og hele Israel legede af alle Kræfter for Guds Åsyn til Sang og til Citre, Harper, Pauker, Cymbler og Trompeter.
\par 9 Men da de kom til Kidons Tærskeplads, rakte Uzza Hånden ud for at gribe fat i Arken, fordi Okserne snublede.
\par 10 Da blussede HERRENs Vrede op mod Uzza, og han slog ham, fordi han rakte Hånden ud mod Arken, og han døde på Stedet for Guds Åsyn.
\par 11 Men David græmmede sig over, at HERREN havde tilføjet Uzza et Brud. Derfor kaldte man Stedet Perez-Uzza, som det hedder den Dag i Dag.
\par 12 Og David grebes den Dag af Frygt for Gud og sagde: "Hvor kan jeg da lade Guds Ark komme hen hos mig!"
\par 13 Og David flyttede ikke Arken hen hos sig i Davidsbyen, men lod den sætte ind i Gatiten Obed-Edoms Hus.
\par 14 Guds Ark blev så i Obed-Edoms Hus tre Måneder, og HERREN velsignede Obed-Edoms Hus og alt, hvad hans var.

\chapter{14}

\par 1 Kong Hiram af Tyrus sendte Sendebud til David med Cedertræer og tillige Murere og Tømmermænd for at bygge ham et Hus.
\par 2 Da skønnede David, at HERREN havde sikret hans Kongemagt over Israel og højnet hans Kongedømme for sit Folk Israels Skyld.
\par 3 David tog i Jerusalem endnu flere Hustruer og avlede flere Sønner og Døtre.
\par 4 Navnene på de Børn, han fik i Jerusalem, er følgende: Sjammua, Sjobab, Natan, Salomo,
\par 5 Jibhar, Elisjua, Elpelet,
\par 6 Noga, Nefeg, Jaf1a,
\par 7 Elisjama, Be'eljada og Elifelet.
\par 8 Men da Fiilisterne hørte, at David var salvet til Konge over hele Israel, rykkede de alle ud for at søge efter ham. Ved Efterretningen herom drog David ud for at møde dem,
\par 9 medens Filisterne kom og bredte sig i Refaimdalen.
\par 10 David rådspurgte da Gud: "Skal jeg drage op mod Filisterne? Vil du give dem i min Hånd?" Og HERREN svarede ham: "Drag op, thi jeg vil give dem i din Hånd!"
\par 11 Så drog de op til Ba'al-Perazim, og der slog han dem. Da sagde David: "Gud har brudt igennem mine Fjender ved min Hånd, som Vand bryder igennem!" Derfor kalder man Stedet Ba'al-Perazim.
\par 12 Og de lod deres Guder i Stikken der, og David bød, at de skulde opbrændes.
\par 13 Men Filisterne bredte sig på ny i Dalen.
\par 14 Da David atter rådspurgte Gud, svarede han: "Drag ikke efter dem, men omgå dem og fald dem i Ryggen ud for Bakabuskene.
\par 15 Når du da hører Lyden af Skridt i Bakabuskenes Toppe, skal du drage i Kamp, thi så er Gud draget ud foran dig for at slå Filisternes Hær."
\par 16 David gjorde, som Gud bød, og de slog Filisternes Hær fra Gibeon til hen imod Gezer.
\par 17 Og Davids Ry bredte sig i alle Lande, idet HERREN Iod Frygt for ham komme over alle Hedningefolkene.

\chapter{15}

\par 1 Siden byggede han sig Huse i Davidsbyen og beredte Guds Ark et Sted, idet han rejste den et Telt
\par 2 Ved den Lejlighed sagde David: "Ingen andre end Leviterne må bære Guds Ark, thi dem har HERREN udvalgt til at bære Guds Ark og til at gøre Tjeneste for ham til evig Tid."
\par 3 Og David samlede hele Israel i Jerusalem for at føre HERRENs Ark op til det Sted, han havde beredt den.
\par 4 Og David samlede Arons Sønner og Leviterne:
\par 5 Af Kehatiterne Øversten Uriel og hans Brødre, 120;
\par 6 af Merariterne Øversten Asaja og hans Brødre, 220;
\par 7 af Gersoniterne Øversten Joel og hans Brødre, 130;
\par 8 af Elizafans Sønner Øversten Sjemaja og hans Brødre, 200;
\par 9 af Hebrons Sønner Øversten Eliel og hans Brødre, 80;
\par 10 af Uzziels Sønner Øversten Amminadab og hans Brødre, 112.
\par 11 Så lod David Præsterne Zadok og Ebjatar og Leviterne Uriel, Asaja, Joel, Sjemaja, Eliel og Amminadab kalde
\par 12 og sagde til dem: "I er Overhoveder for Leviternes Fædrenehuse; helliger eder tillige med eders Brødre og før HERRENs, Israels Guds, Ark op til det Sted, jeg har beredt den;
\par 13 det var jo, fordi I ikke var med første Gang, at HERREN vor Gud tilføjede os et Brud; thi vi søgte ham ikke på rette Måde."
\par 14 Så helligede Præsferne og Leviterne sig for at føre HERRENs, Israels Guds, Ark op;
\par 15 og Levis Sønner løftede med Bærestængerne Guds Ark op på Skuldrene, som Moses havde påbudt efter HERRENS Ord.
\par 16 Fremdeles bød David Leviternes Øverster at lade deres Brødre Sangerne stille sig op med Musikinstrumenter, Harper, Citre og Cymbler og lade høje Jubeltoner klinge.
\par 17 Så lod Leviterne Heman, Joels Søn, og af hans Brødre Asaf, Berekjas Søn, og af deres Brødre Merariterne Etan, Husjajas Søn, stille sig op
\par 18 og ved Siden af dem deres Brødre af anden Rang Zekarja, Uzziel, Sjemiramot, Jehiel, Unni, Eliab, Benaja, Ma'aseja, Mattitja, Elipelehu og Miknejahu og Dørvogterne Obed-Edom og Je'iel;
\par 19 Sangerne Heman, Asaf og Etan skulde spille på Kobbercymbler,
\par 20 Zekarja, Uzziel, Sjemiramot, Jehiel, Unni, Eliab, Ma'aseja og Benaja på Harper al-alamot;
\par 21 Mattitja, Elipelehu, Miknejahu, Obed-Edom og Je'iel skulde lede Sangen med Citre al-hassjeminit;
\par 22 Konanja, Leviternes Øverste over dem, der bar, skulde lede disse, da han forstod sig derpå;
\par 23 Berekja og Elkana skulde være Dørvogtere ved Arken;
\par 24 Præsterne Sjebanja, Josjafat, Netan'el, Amasaj, Zekarja, Benaja og Eliezer skulde blæse i Trompeterne foran Guds Ark; og Obed-Edom og Jehija skulde være Dørvogtere ved Arken.
\par 25 Derpå drog David, Israels Ældste og Tusindførerne hen for under Festglæde at lade HERRENs Pagts Ark bringe op fra Obed-Edoms Hus,
\par 26 og da Gud hjalp Leviterne, der har HERRENs Pagts Ark, ofrede man syv Tyre og syv Vædre.
\par 27 David har en fin linned Kappe, ligeledes alle Leviterne, der har Arken, og Sangerne og Konanja, som ledede dem, der bar. Og David var iført en linned Efod.
\par 28 Og hele Israel bragte HERRENs Pagts Ark op under Festjubel og til Hornets, Trompeternes og Cymblernes Klang, under Harpe- og Citerspil.
\par 29 Men da HERRENs Pagts Ark kom til Davidsbyen, så Sauls Datter Mikal ud af Vinduet, og da hun så Kong David springe og danse, ringeagtede hun ham i sit Hjerte.

\chapter{16}

\par 1 De førte så Guds Pagts Ark ind og stillede den midt i det Telt, David havde rejst den; og de ofrede Brændofre og Takofre for Guds Åsyn.
\par 2 Og da David var færdig med Brændofrene og Takofrene, velsignede han Folket i HERRENs Navn
\par 3 og uddelte til hver enkelt Israelit, både Mand og Kvinde, en Brødskive, et Stykke Kød og en Rosinkage.
\par 4 Foran HERRENs Ark stillede han nogle af Leviterne til at gøre Tjeneste og til at takke, love og prise HERREN, Israels Gud;
\par 5 Asaf var Leder, og næst efter ham kom Zekarja, så Uzziel, Sjemiramot, Jehiel, Mattitja, Eliab, Benaja, Obed-Edom og Je'iel med Harper og Citre, medens Asaf lod Cymblerne klinge,
\par 6 og Præsterne Benaja og Jahaziel stadig blæste i Trompeterne foran Guds Pagts Ark.
\par 7 Den Dag, ved den Lejlighed, overdrog David for første Gang Asaf og hans Brødre at lovsynge HERREN.
\par 8 Pris HERREN, påkald hans Navn, gør hans Gerninger kendte blandt Folkeslag!
\par 9 Syng og spil til hans Pris, tal om alle hans Undere,
\par 10 ros jer af hans hellige Navn, eders Hjerte glæde sig, I, som søger HERREN,
\par 11 spørg efter HERREN og hans Magt, søg bestandig hans Åsyn;
\par 12 kom i Hu de Undere, han øved, hans Tegn og hans Munds Domme,
\par 13 I, hans Tjener, Israels Sæd. hans udvalgte, Jakobs Sønner!
\par 14 Han, HERREN, er vor Gud, hans Domme når ud over Jorden;
\par 15 han ihukommer for evigt sin Pagt, i tusind Slægter sit Tilsagn,
\par 16 Pagten. han slutted med Abraham, Eden, han tilsvor Isak:
\par 17 han holdt dem i Hævd som Ret for Jakob, en evig Pagt for Israel,
\par 18 idet han sagde: "Dig giver jeg Kana'ans Land som eders Arvelod."
\par 19 Da de kun var en liden Hob, kun få og fremmede der,
\par 20 og vandred fra Folk til Folk, fra et Rige til et andet,
\par 21 tillod han ingen at volde dem Men, men tugted for deres Skyld Konger:
\par 22 "Rør ikke mine Salvede, gør ikke mine Profeter ondt!"
\par 23 Syng for HERREN, al Jorden, fortæl om hans Frelse Dag efter dag;
\par 24 kundgør hans Ære blandt Folkene, hans Undere blandt alle Folkeslag!
\par 25 Thi stor og højlovet er HERREN, forfærdelig over alle Guder;
\par 26 thi alle Folkeslagenes Guder er Afguder, HERREN er Himlens Skaber.
\par 27 For hans Åsyn er Højhed og Hæder, Pris og Fryd i hans Helligdom.
\par 28 Giv HERREN, I Folkeslags Slægter, giv HERREN Ære og Pris,
\par 29 giv HERREN hans Navns Ære, bring Gaver og kom for hans Åsyn, tilbed HERREN i helligt Skrud,
\par 30 bæv for hans Åsyn, al Jorden! Han grundfæsted Jorden, den rokkes ikke.
\par 31 Himlen glæde sig Jorden juble, det lyde blandt Folkene: "HERREN har vist, han er Konge!"
\par 32 Havet med dets Fylde bruse, Marken juble og alt, hvad den bærer.
\par 33 Da fryder sig Skovens Træer for HERRENs Åsyn, thi han kommer, han kommer at dømme Jorden.
\par 34 Lov HERREN, thi han er god, og hans Miskundhed varer evindelig!
\par 35 Og sig: "Frels os, vor Frelses Gud, saml os og fri os fra Folkene, at vi må love dit hellige Navn; med Stolthed synge din Pris!"
\par 36 Lovet være HERREN, Israels Gud, fra Evighed og til Evighed! Da sagde hele Folket: "Amen!" og: "Lov HERREN!"
\par 37 Så lod han Asaf og hans Brødre blive der foran HERRENs Pagts Ark for altid at gøre Tjeneste foran Arken efter hver Dags Behov;
\par 38 og Obed-Edom, Jedituns Søn, og Hosa med deres Brødre, i alt otte og tresindstyve, lod han blive som Dørvogtere.
\par 39 Men Præsten Zadok og hans Brødre Præsterne lod han blive foran HERRENs Bolig på Oerhøjen i Gibeon
\par 40 for daglig, både Aften og Morgen, at ofre HERREN Brændofre på Brændofferalteret ganske som det er foreskrevet i den Lov, HERREN havde pålagt Israel;
\par 41 og sammen med dem Heman og Jedutun og de øvrige før nævnte udvalgte Mænd til at love HERREN med Ordene "thi hans Miskundhed varer evindelig!"
\par 42 Og de havde hos sig Trompeter og Cymbler til dem, der spillede, og instrumenter til Guds Sange; men Jedutuns Sønner var Dørvogtere.
\par 43 Derpå gik alt Folket hver til sit, og David vendte hjem for at velsigne sit Hus.

\chapter{17}

\par 1 Engang David sad i sit Hus, sagde han til Profeten Natan: "Se, jeg har et Cedertræshus at bo i, men HERRENs Pagts Ark har Plads i et Telt!"
\par 2 Natan svarede David: "Gør alt, hvad din Hu står til, thi Gud er med dig!"
\par 3 Men samme Nat kom Guds Ord til Natan således:
\par 4 "Gå hen og sig til min Tjener David: Så siger HERREN: Ikke du skal bygge mig det Hus, jeg skal bo i!
\par 5 Jeg har jo ikke haft noget Hus at bo i, siden den Dag jeg førte Israeliterne op, men vandrede med, boende i et Telt.
\par 6 Har jeg, i al den Tid jeg vandrede om blandt alle Israeliterne, sagt til nogen af Israels Dommere, som jeg satte til at vogte mit Folk: Hvorfor bygger I mig ikke et Cedertræshus?
\par 7 Sig derfor til min Tjener David: Så siger Hærskarers HERRE: Jeg tog dig fra Græsgangen, fra din Plads bag Småkvæget til at være Fyrste over mit Folk Israel,
\par 8 og jeg var med dig, overalt hvor du færdedes, og udryddede alle dine Fjender foran dig; jeg vil skabe dig et Navn som de størstes på Jorden
\par 9 og skaffe mit Folk Israel en Hjemstavn og plante det, så det kan blive boende på sit Sted uden mere at skulle forstyrres i sin Ro, og uden at Voldsmænd mere skal ødelægge det som tidligere,
\par 10 dengang jeg satte Dommere over mit Folk Israel; og jeg vil underkue alle dine Fjender. Så kundgør jeg dig nu: Et Hus vil HERREN bygge dig!
\par 11 Når dine Dage er omme og du vandrer til dine Fædre, vil jeg efter dig oprejse din Sæd, en af dine Sønner, og grundfæste hans Kongedømme.
\par 12 Han skal bygge mig et Hus, og jeg vil grundfæste hans Trone evindelig.
\par 13 Jeg vil være ham en Fader, og han skal være mig en Søn; og min Miskundhed vil jeg ikke tage fra ham, som jeg tog den fra din Forgænger;
\par 14 jeg vil indsætte ham i mit Hus og mit Kongedømme til evig Tid, og hans Trone skal stå fast til evig Tid!"
\par 15 Alle disse Ord og hele denne Åbenbaring meddelte Natan David.
\par 16 Da gik Kong David ind og dvælede for HERRENs Åsyn og sagde: "Hvem er jeg, Gud HERRE, og hvad er mit Hus, at du har bragt mig så vidt?
\par 17 Men det var dig ikke nok o Gud, du gav også din Tjeners Hus Forjættelser for fjerne Tider og lod mig skue kommende Slægter, Gud HERRE!
\par 18 Hvad mere har David at sige dig? Du kender jo dog din Tjener,
\par 19 HERRE! For din Tjeners Skyld, og fordi din Hu stod dertil, gjorde du alt dette store og kundgjorde alle disse store Ting,
\par 20 HERRE! Ingen er som du, og der er ingen Gud uden dig, efter alt hvad vi har hørt med vore Ører.
\par 21 Og hvor på Jorden findes et Folk som dit Folk Israel, et Folk, som Gud kom og udfriede og gjorde til sit Folk for at vinde sig et Navn og udføre store og frygtelige Gerninger ved at drive andre Folkeslag bort foran sit Folk, det, du udfriede fra Ægypten?
\par 22 Du har grundfæstet dit Folk Israel som dit, Folk til evig Tid, og du, HERRE, er blevet deres Gud.
\par 23 Så lad da, HERRE, den Forjættelse, du udtalte om din Tjener og hans Hus, gælde til evig Tid og gør, som du sagde!
\par 24 Da skal dit Navn stå fast og blive stort til evig Tid, så man siger: Hærskarers HERRE, Israels Gud, Gud over Israel! Og din Tjener Davids Hus skal stå fast for dit Åsyn.
\par 25 Thi du, min Gud, har åbenbaret for din Tjener: Jeg vil bygge dig et Hus! Derfor har din Tjener dristet sig til at bede for dit Åsyn.
\par 26 Derfor, HERRE, du er Gud, du har givet din Tjener denne Forjættelse,
\par 27 så lad det behage dig at velsigne din Tjeners Hus, at det til evig Tid må stå fast for dit Åsyn. Thi du, HERRE, har velsignet det, og det bliver velsignet evindelig!"

\chapter{18}

\par 1 Nogen Tid efter slog David Filisterne og undertvang dem, og han fratog Filisterne Gat med Småbyer.
\par 2 Fremdeles slog han Moabiterne; og Moabiterne blev Davids skatskyldige Undersåtter.
\par 3 Ligeledes slog David Kong Hadar'ezer af Zoba i Nærheden af Hamat, da han var draget ud for at underlægge sig Egnene ved Eufratfloden.
\par 4 David fratog ham 1.000 Vogne, 7.008 Ryttere og 20.000 Mand Fodfolk og lod alle Hestene lamme på hundrede nær, som han skånede.
\par 5 Og da Aramæerne fra Darmaskus kom Kong Hadar'ezer af Zoba til Hjælp, slog David 22.000 Mand af Aramæerne.
\par 6 Derpå indsatte David Fogeder i det darmaskenske Aram, og Aramæerne blev Davids skatskyldige Undersåtter. Således gav HERREN David Sejr, overalt hvor han drog frem.
\par 7 Og David tog de Guldskjolde, Hadar'ezers Folk havde båret, og bragte dem til Jerusalem;
\par 8 og fra Hadar'ezers Byer, Tibhat og Kun, bortførte David Kobber i store Mængder; deraf lod Salomo Kobberhavet, Søjlerne og Kobbersagerne lave.
\par 9 Men da Kong To'u af Hamat hørte, at David havde slået hele Kong Hadar'ezer af Zobas Stridsmagt,
\par 10 sendte han sin Søn Hadoram til Kong David for at hilse på ham og lykønske ham til, at han havde kæmpet med Hadar'ezer og slået ham - Hadar'ezer havde nemlig ligget i Krig med To'u - og han medbragte alle Hånde Sølv-, Guld og Kobbersager.
\par 11 Også dem helligede Kong David HERREN tillige med det Sølv og Guld, han havde taget fra alle Folkeslagene, Edom, Moab, Ammoniterne, Filisterne og Amalek.
\par 12 Og Absjaj, Zerujas Søn, slog Edom i Saltdalen, 18000 Mand;
\par 13 derpå indsatte han Fogeder i Edom, og alle Edomiterne blev Davids Undersåtter. Således gav HERREN David Sejr, overalt hvor han drog frem.
\par 14 Og David var Konge over hele Israel, og han øvede Ret og Retfærdighed mod hele sit Folk.
\par 15 Joab, Zerujas Søn, var sat over Hæren; Josjafat, Ahiluds Søn, var Kansler;
\par 16 Zadok, Ahitubs Søn, og Abimelek, Ebjatars Søn, var Præster; Sjavsja var Statsskriver;
\par 17 Benaja, Jojadas Søn, var sat over Kreterne og Pleterne, og Davids Sønner var de ypperste ved Kongens Side.

\chapter{19}

\par 1 Nogen Tid efter døde Ammoniternes Konge Nahasj, og hans Søn Hanun blev Konge i hans Sted.
\par 2 Da tænkte David: "Jeg vil vise Hanun, Nahasj's Søn, Venlighed, thi hans Fader viste mig Venlighed." Og David sendte Folk for at vise ham Deltagelse i Anledning at hans Faders Død. Men da Davids Mænd kom til Ammoniteroes Land til Hanun for at vise ham Deltagelse,
\par 3 sagde Ammoniternes Høvdinger til Hanun: "Tror du virkelig, det er for at hædre din Fader, at David sender Bud og viser dig Deltagelse? Mon ikke det er for at udforske og ødelægge Landet og udspejde det, at hans Folk kommer til dig?"
\par 4 Da tog Hanun Davids Folk og lod dem rage og Halvdelen af deres Klæder skære af til Skridtet, og derpå lod han dem gå.
\par 5 Da David fik Efterretning om Mændenes Behandling, sendte han dem et Bud i Møde; thi Mændene var blevet grovelig forhånet; og Kongen lod sige: "Bliv i Jeriko, til eders Skæg er vokset ud!"
\par 6 Men da Ammoniterne så, at de havde lagt sig for Had hos David, sendte Hanun og Ammoniterne 1000 Talenter Sølv for at leje Vogne og Ryttere i Aram-Naharajim, Aram Ma'aka og Zoba;
\par 7 og de lejede 32000 Vogne og Kongen af Ma'aka med hans Folk, og de kom og slog Lejr uden for Medeba; imidlertid havde Ammoniterne samlet sig fra deres Byer og rykkede ud til Kamp.
\par 8 Da David hørte det, sendte han Joab af Sted med hele Hæren og Kærnetropperne.
\par 9 Ammoniterne rykkede så ud og stillede sig op til Kamp ved Byens Port, medens Kongerne, der var kommet til, stod for sig selv på åben Mark.
\par 10 Da Joab så, at Angreb truede ham både forfra og bagfra, gjorde han et Udvalg blandt alt Israels udsøgte Mandskab og tog Stilling over for Aramæerne,
\par 11 medens han overlod Resten af Mandskabet til sin Broder Absjaj, og de tog Stilling over for Ammoniterne.
\par 12 Og han sagde: "Hvis Aramæerne bliver mig for stærke, skal du ile mig til Hjælp; men bliver Ammoniterne dig for stærke, skal jeg hjælpe dig.
\par 13 Tag Mod til dig og lad os tappert værge vort Folk og vor Guds Byer - så får HERREN gøre, hvad ham tykkes godt!"
\par 14 Derpå rykkede Joab frem med sine Folk til Kamp mod Aramæerne, og de flygtede for ham.
\par 15 Og da Ammoniterne så, at Aramæerne tog Flugten, flygtede også de forhans Broder Absjaj og trak sig ind i Byen. Derpå kom Joab til Jerusalem.
\par 16 Men da Aramæerne så, at de var slået af Israel, sendte de Bud og fik Aramæerne hinsides Floden til at rykke ud med Sjofak, Hadar'ezers Hærfører, i Spidsen.
\par 17 Da David fik Efterretning herom, samlede han hele Israel, satte over Jordan og kom til Helam, hvor David stillede sig op til Kamp mod Aramæerne, og de angreb ham.
\par 18 Men Aramæerne flygtede for Israel, og David nedhuggede 7.000 Stridsheste og 40.000 Mand Fodfolk af Aram; også Hærføreren Sjofak huggede han ned.
\par 19 Da alle Hadar'ezers Lydkonger så, at de var slået af Israel, sluttede de Fred med David og underkastede sig, og Aramæerne vilde ikke hjælpe Ammoniterne mere.

\chapter{20}

\par 1 Næste År, ved den Tid Kongerne drager i Krig, førte Joab Krigshæren ud og hærgede Ammoniternes Land; derpå drog han hen og belejrede Rabba. David blev derimod selv i Jerusalem. Og Joab indtog Rabba og ødelagde det.
\par 2 Da tog David Kronen af Milkoms Hoved, og han fandt, at den var af Guld og vejede en Talent; der var en Ædelsten på den, og den blev sat på Davids Hoved. Et vældigt Bytte fra Byen førte han med sig,
\par 3 og Indbyggerne slæbte han bort og satte dem til Savene, Jernhakkerne og Økserne. Således gjorde han ved alle Ammonifemes Byer. Derpå vendte David og hele Hæren tilbage til Jerusalem.
\par 4 Siden hen kom det atter til Kamp med Filisterne i Gezer. Husjatiten Sibbekaj nedhuggede da Sippaj, som var af Rafaslægten, og de blev underkuet.
\par 5 Atter kom det til Kamp med Filis,terne. Elhanan, Ja'irs Søn, nedhuggede da Lami, Gatiten Goliats Broder, hvis Spydstage var som en Væverbom.
\par 6 Atter kom det til Kamp i Gat. Da var der en kæmpe stor Mand med seks Fingre på hver Hånd og seks Tæer på hver Fod, i alt fire og tyve; han var også af Rafaslægten.
\par 7 Han hånede Israel, og derfor huggede Jonatan, en Søn af Davids Broder Sjim'a, ham ned.
\par 8 Disse var af Rafaslægten i Gat; de faldt for Davids og hans Mænds Hånd.

\chapter{21}

\par 1 Satan trådte op mod Israel og æggede David til at holde Mandtal over Israel.
\par 2 David sagde da til Joab og Hærførerne: "Drag ud og tæl Israel fra Be'ersjeba til Dan og bring mig Efterretning, for at jeg kan få Tallet på dem at vide!"
\par 3 Men Joab svarede: "Måtte HERREN forøge sit Folk hundredfold! Er de ikke, Herre Konge, min Herres Trælle alle sammen? Hvorfor vil min Herre det? Hvorfor skal det blive Israel til Skyld?"
\par 4 Men Joab måtte bøje sig for Kongens Ord, og Joab begav sig derfor bort, drog hele Israel rundt og kom tilbage til Jerusalem.
\par 5 Joab opgav derpå David Tallet, der var fundet ved Folketællingen, og hele Israel talte 1.100.000 kraftige Mænd, Juda 470.000.
\par 6 Men Levi og Benjamin havde han ikke talt med, thi Kongens Ord var Joab en Gru.
\par 7 Dette var ondt i Guds Øjne, og han slog Israel.
\par 8 Da sagde David til Gud: "Jeg har syndet svarlig i, hvad jeg har gjort. Men tilgiv nu din Tjeners Brøde, thi jeg har handlet som en stor Dåre!"
\par 9 Men HERREN talede således til Gad, Davids Seer:
\par 10 "Gå hen og sig således til David: Så siger HERREN: Jeg forelægger dig tre Ting; vælg selv, hvilken jeg skal lade times dig!"
\par 11 Gad kom da til David og sagde til ham: "Så siger HERREN: Vælg!
\par 12 Vælger du tre Hungersnødsår, eller vil du i tre Måneder flygte for dine Fjender og dine Avindsmænds Sværd, eller skal der komme tre Dage med HERRENs Sværd og Pest i Landet, i hvilke HERRENs Engel spreder Ødelæggelse i hele Israels Område? Se nu til, hvad jeg skal svare ham, der har sendt mig!"
\par 13 David svarede Gad: "Jeg er i såre stor Vånde - lad mig så falde i HERRENs Hånd, thi hans Barmhjertighed er såre stor; i Menneskehånd vil jeg ikke falde!"
\par 14 Så sendte HERREN Pest over Israel, og af Israel døde 70.000 Mennesker.
\par 15 Og Gud sendte en Engel til Jerusalem for at ødelægge det. Men lige som han skulde til at ødelægge Byen, så HERREN til og angrede det onde; og han sagde til Engelen, som var ved at ødelægge: "Nu er det nok, drag din Hånd tilbage!" HERRENs Engel stod da ved Jebusiten Ornans Tærskeplads.
\par 16 Da David løftede sit Blik og så HERRENs Engel stå mellem Himmel og Jord med draget Sværd i Hånd, rettet mod Jerusalem, faldt han og de Ældste, der var klædt i Sæk, på deres Ansigt;
\par 17 og David sagde til Gud: "Var det ikke mig, der sagde, at Folket skulde tælles? Det er mig, der har syndet, og såre ilde har jeg handlet; men Fårene der, hvad har de gjort? HERRE min Gud, lad din Hånd dog ramme mig og mit Fædrenehus, men lad Plagen ikke ramme dit Folk!"
\par 18 Da sagde HERRENs Engel til Gad, at han skulde byde David gå op og rejse HERREN et Alter på Jebusiten Ornans Tærskeplads.
\par 19 Og David gik derop, efter det Ord Gad havde talt i HERRENs Navn.
\par 20 Da Ornan vendte sig om, så han Kongen og hans fire Sønner, der var hos ham, komme gående. Ornan var ved at tærske Hvede.
\par 21 Og David nåede hen til Ornan, og da Ornan så op og fik Øje på David,forlodhanTærskepladsen og kastede sig på sit Ansigt til Jorden for ham.
\par 22 Da sagde David til Oroan: "Overlad mig din Tærskeplads, for at jeg der kan bygge HERREN et Alter! For fuld betaling skal du overlade mig den, at Folkett må blive friet fra Plagen!"
\par 23 Da sagde Ornan til David: "Min Herre Kongen tage den og gøre, hvad ham tykkes ret; se, jeg giver Okserne til Brændofre, Tærskeslæderne til Brændsel og Hveden til Afgrødeoffer; jeg giver det hele!"
\par 24 - Men Kong David svarede Ornan: "Nej, jeg vil købe det for fuld Betaling, thi til HERREN vil jeg ikke tage, hvad dit er, eller bringe et Brændoffer, som intet koster mig!"
\par 25 Så gav David Ornan Guld til en Vægt af 600 Sekel for Pladsen;
\par 26 og David byggede HERREN et Alter der og ofrede Brændofre og Takofre; og da han råbte til HERREN, svarede HERREN ham ved at lade Ild falde ned fra Himmelen på Brændofferalteret,
\par 27 Og på HERRENs Bud stak Engelen sit Sværd i Balgen igen.
\par 28 På den Tid ofrede David på Jebusiten Ornans Tærskeplads, fordi han havde set, at HERREN havde svaret ham der;
\par 29 men HERRENs Bolig, som Moses havde rejst i Ørkenen, og Brændofferalteret stod på den Tid på Offerhøjen i Gibeon.
\par 30 Men David kunde ikke gå hen og søge Gud foran Alteret, thi han var rædselsslagen over HERRENs Engels Sværd.

\chapter{22}

\par 1 Da sagde David: "Her skal Gud HERRENs Hus stå, her skal Israels Brændofferalter stå!"
\par 2 David bød så, at man skulde samle alle de fremmede, som boede i Israels Land, og han satte Stenhuggere til at tilhugge Kvadersten til Opførelsen af Guds Hus;
\par 3 fremdeles anskaffede David Jern i Mængde til Nagler på Portfløjene og til Kramper, en umådelig Mængde Kobber
\par 4 og talløse Cederbjælker, idet Zidonierne og Tyrierne bragte David Cederbjælker i Mængde.
\par 5 Thi David tænkte: "Min Søn Salomo er ung og uudviklet, men Huset, som skal bygges HERREN, skal være stort og mægtigt, for at det kan vinde Navnkundighed og Ry i alle Lande; jeg vil træffe Forberedelser for ham." Og David traf store Forberedelser, før han døde.
\par 6 Derpå lod han sin Søn Salomo kalde og bød ham bygge HERREN, Israels Gud, et Hus.
\par 7 Og David sagde til Salomo: "Min Søn! Jeg havde selv i Sinde at bygge HERREN min Guds Navn et Hus;
\par 8 men da kom HERRENs Ord til mig således: Du har udgydt meget Blod og ført store brige; du må ikke bygge mit Navn et Hus, thi du har udgydt meget Blod på Jorden for mit Åsyn.
\par 9 Se, en Søn skal fødes dig; han skal være en Fredens Mand, og jeg vil skaffe ham Fred for alle hans Fjender rundt om; thi hans Navn skal være Salomo, og jeg vil give Israel Fred og Ro i hans Dage.
\par 10 Han skal bygge mit Navn et Hus; og han skal være mig en Søn, og jeg vil være ham en Fader, og jeg vil grundfæste hans Kongedømmes Trone over Israel til evig Tid!
\par 11 Så være da HERREN med dig, min Søn, at du må få Lykke til at bygge HERREN din Guds Hus, således som han har forjættet om dig.
\par 12 Måtte HERREN kun give dig Forstand og Indsigt, så han kan sætte dig over Israel og du kan holde HERREN din Guds Lov!
\par 13 Så vil det gå dig vel når du nøje holder Anordningerne og Lovbudene, som HERREN bød Moses at pålægge Israel. Vær frimodig og stærk, frygt ikke og tab ikke Modet!
\par 14 Se, med stor Møje har jeg til HERRENs Hus tilvejebragt 100.000 Talenter Guld og en Million Talenter Sølv og desuden Kobber og Jern, som ikke er til at veje, så meget er der; dertil har jeg tilvejebragt Bjælker og Sten; og du skal sørge for flere.
\par 15 En Mængde Arbejdere står til din Rådighed, Stenhuggere, Murere, Tømrere og alle Slags Folk, der forstår sig på Arbejder af enhver Art.
\par 16 Af Guld, Sølv, Kobber og Jern er der umådelige Mængder til Stede - så tag da fat, og HERREN være med dig!"
\par 17 Og David bød alle Israels Øverster hjælpe hans Søn Salomo og sagde:
\par 18 "Er HERREN eders Gud ikke med eder, har han ikke skaffet eder Ro til alle Sider? Han har jo givet Landets Indbyggere i min Hånd, og Landet er underlagt HERREN og hans Folk;
\par 19 så giv da eders Hjerter og Sjæle hen til at søge HERREN eders Gud og tag fat på at bygge Gud HERRENs Helligdom, så at HERRENs Pagts Ark og Guds hellige Ting kan føres ind i Huset, der skal bygges HERRENS Navn."

\chapter{23}

\par 1 Da David var blevet gammel og mæt af Dage, gjorde han sin Søn Salomo til Konge over Israel.
\par 2 Han samlede alle Israels Øverster og Præsterne og Leviterne.
\par 3 Og Leviterne blev talt fra Trediveårsalderen og opefter, og deres Tal udgjorde, Hoved for Hoved, Mand for Mand, 38000.
\par 4 "Af dem," sagde han, "skal 24000 forestå Arbejdet ved HERRENs Hus, 6000 være Tilsynsmænd og Dommere,
\par 5 4000 være Dørvogtere og 4000 love HERREN med de instrumenter, jeg har ladet lave til Lovsangen."
\par 6 Og David inddelte dem i Skifter efter Levis Sønner Gerson, Kehat og Merari.
\par 7 Til Gersoniterne hørte: Ladan og Sjim'i;
\par 8 Ladans Sønner: Jehiel, som var Overhoved, Zetam og Joel, tre;
\par 9 Sjim'is Sønner: Sjelomit, Haziel og Haran, tre. De var Overhoveder for Ladans Fædrenehuse.
\par 10 Sjim'is Sønner: Jahat,Ziza, Je'usj og Beri'a. Disse fire var Sjim'is Sønoer.
\par 11 Jahat var Overhoved og Ziza den næste; Je'usj og Beri'a havde ikke mange Sønner og regnedes derfor for eet Fædrenehus, eet Embedsskifte.
\par 12 Kehatiterne: Amram, Jizhar, Hebron og Uzziel, fire;
\par 13 Amrams Sønner: Aron og Moses. Aron udskiltes sammen med sine Sønner til at helliges som højhellig til evig Tid, til at tænde Offerild for HERRENs Åsyn, til at tjene ham og velsigne i hans Navn til evig Tid.
\par 14 Den Guds Mand Moses's Sønner regnedes derimod til Levis Stamme.
\par 15 Moses's Sønner: Gersom og Eliezer;
\par 16 Gersoms Sønner; Sjubael, som var Overhoved;
\par 17 Eliezers Sønner: Rehabja, som var Overhoved; andre Sønner havde Eliezer ikke, men Rehabjas Sønner var overmåde talrige.
\par 18 Jizhars Sønner: Sjelomit, som var Overhoved.
\par 19 Hebrons Sønner: Jerija, som var Overhoved, Amarja den anden, Uzziel den tredje, Jekam'am den fjerde.
\par 20 Uzziels Sønner: Mika, som var Overhoved, og Jissjija den anden.
\par 21 Meraiterne var: Mali og Musji. Malis Sønner: El'azar og Kisj.
\par 22 El'azar efterlod sig ved sin Død ingen Sønner, men kun Døtre, som deres Brødre, Kisj's Sønner, ægtede.
\par 23 Musjis Sønner: Mali, Eder og Jeremot, tre.
\par 24 Det var Levis Sønner efter deres Fædrenebuse, Overhovederne for Fædrenehusene, de, som mønstredes ved Optælling af Navnene, Hoved for Hoved, de, som udførte Arbejdet ved Tjenesten i HERRENs Hus, fra Tyveårsalderen og opefter.
\par 25 Thi David tænkte: "HERREN, Israels Gud, har skaffet sit Folk Ro og taget Boligi Jerusalem for evigt;
\par 26 derfor behøver Leviterne heller ikke mere at bære Boligen og alle de Ting, som hører til dens Tjeneste."
\par 27 (Ifølge Davids sidste Forordninger regnes Tallet på Leviterne fra Tyveårsalderen og opefter).
\par 28 Men deres Plads er ved Arons Sønners Side, for at de kan udføre Tjenesten i HERRENs Hus; de skal tage sig af Forgårdene, Kamrene, Renholdelsen af alle de hellige Ting og Arbejdet, der skal udføres i Guds Hus;
\par 29 de skal sørge for Skuebrødene, Melet til Afgrødeofrene, de usyrede Fladbrød, Panden, Dejgen og alle Rum- og Længdemål;
\par 30 hver Morgen skal de stå og love og prise HERREN, ligeså om Aftenen,
\par 31 og hver Gang der ofres Brændofre til HERREN på Sabbaterne, Nymånedagene og Højtiderne; i det fastsatte Antal efter den for dem gældende Forskrift skal de altid stå for HERRENs Åsyn.
\par 32 Således skal de tage Vare på, hvad der er at varetage ved Åbenbaringsteltet og ved det hellige og hjælpe deres Brødre, Arons Sønner, med Tjenesten i HERRENs Hus.

\chapter{24}

\par 1 Arons Sønner, delte i Skifter, var: Arons Sønner Nadab, Abihu, Eleazar og Itamar;
\par 2 Nadab og Abihu døde før deres Fader uden at efterlade sig Sønoer, men Eleazar og Itamar fik Præsteværdigheden.
\par 3 David tillige med Zadok af Eleazars Sønner og Ahimelek af Itamars Sønner inddelte dem efter deres Embedsskifter ved deres Tjeneste.
\par 4 Og da det viste sig, at Eleazars Sønner havde flere Overhoveder end Itamars, delte de dem således, at Eleazars Sønner fik seksten Overhoveder over deres Fædrenebuse. Itamars Sønner otte.
\par 5 Og de delte, begge Hold ved Lodkastning, thi der fandtes hellige Øverster og Guds Øverster både iblandt Eleazars og Itamars Sønner.
\par 6 Skriveren Sjemaja, Netan'els Søn af Levis Slægt, optegnede dem i Påsyn af Kongen, Øversterne, Præsten Zadok, Ahimelek, Ebjatars Søn, og Overhovederne for Præsternes og Leviternes Fædrenebuse. Der udtoges eet Fædrenehus af Itamar for hvert to af Eleazar.
\par 7 Det første Lod traf Jojarib, det andet Jedaja,
\par 8 det tredje Harim, det fjerde Seorim,
\par 9 det femte Malkija, det sjette Mijjamin,
\par 10 det syvende Hakkoz, det ottende Abija,
\par 11 det niende Jesua, det tiende Sjekanja,
\par 12 det ellevte Eljasjib, det tolvte Jakim,
\par 13 det trettende Huppa, det fjortende Jisjba'al,
\par 14 det femtende Bilga, det sekstende Immer,
\par 15 det syttende Hezir, det attende Happizzez,
\par 16 det nittende Petaja, det tyvende Jehezkel,
\par 17 det een og tyvende Jakin, det to og tyvende Gamul,
\par 18 det tre og tyvende Delaja og det fire og tyvende Ma'azja.
\par 19 Det var deres Embedsskifter ved deres Tjeneste, når de gik ind i HERRENs Hus, efter den Forpligtelse deres Fader Aron pålagde dem, efter hvad HERREN, Israels Gud, havde pålagt ham.
\par 20 De andre Leviter var: Af Amrams Sønner Sjubael; af Sjubaels Sønner Jedeja.
\par 21 Af Rehabjas Sønner Jissjija, som var Overhoved.
\par 22 Af Jizhariterne Sjelomot; af Sjelomots Sønner Jahat.
\par 23 Hebrons Sønner: Jerija, som var Overhoved, Amarja den anden, Uzziel den tredje, Jekam'am den fjerde.
\par 24 Uzziels Sønner: Mika; af Mikas Sønner Sjamir.
\par 25 Mikas Broder Jissjija; af Jissjijas' Sønner Zekarja.
\par 26 Meraris Sønner: Mali og Musji og hans Søn Uzzijas Sønner.
\par 27 Meraris Søn Uzzijas Sønner: Sjoham, Zakkur og Ibri.
\par 28 Af Mali El'azar, der ingen Sønner havde, og Kisj;
\par 29 af Kisj Kisj's Sønner: Jerame'el.
\par 30 Musjis Sønner: Mali, Eder og Jerimot. Det var Leviternes Efterkommere efter deres Fædrenehuse.
\par 31 Også de kastede Lod ligesom deres Brødre, Arons Sønner, i Påsyn af Kong David, Zadok og Ahimelek og Overhovederne for Præsternes og Leviternes Fædrenehuse - Fædrenehusenes Overhoveder ligesom deres yngste Brødre.

\chapter{25}

\par 1 Derpå udskilte David og Hærførerne til Tjenesten Asafs, Hemans og Jedutuns Sønner, som i profetisk Henrykkelse spillede på Citre, Harper og Cymbler; og Tallet på de Mænd, der havde med denne Tjeneste at gøre, var:
\par 2 Af Asafs Sønner: Zakkur, Josef, Netanja og Asar'ela, Asafs Sønner under Ledelse af Asaf, der spillede i profetisk Henrykkelse under Kongens Ledelse.
\par 3 Af Jedutun: Jedutuns Sønner Gedalja, Jizri, Jesja'ja, Sjim'i, Hasjabja og Mattitja, seks, under Ledelse af deres Fader Jedutun, der i profetisk Henrykkelse spillede på Citer, når HERREN blev lovet og priset.
\par 4 Af Heman: Hemans Sønner Bukkija, Mattanja, Uzziel, Sjubael, Jerimot, Hananja, Hanani, Eliata, Giddalti, Romamti-Ezer, Josjbekasja, Malloti, Hotir og Mahaziot.
\par 5 Alle disse var Sønner af Heman, Kongens Seer i Guds Ord; for at løfte hans Horn gav Gud Heman fjorten Sønner og tre Døtre.
\par 6 Alle disse spillede under deres Faders Ledelse ved Sangen i HERRENs Hus på Cymbler, Harper og Citre for således at gøre Tjeneste i Guds Hus under Ledelse af Kongen, Asaf, Jedutun og Heman.
\par 7 Deres Tal, sammenregnet med deres Brødre, der var oplært til at synge HERRENs Sange, var 288, kyndige Folk til Hobe.
\par 8 De kastede Lod om Ordningen af Tjenesten med lige Kår både for små og for store, Mestre og Lærlinge.
\par 9 Det første Lod traf Josef, ham selv med hans Brødre og Sønner, tolv; det andet Gedalja, ham selv med hans Brødre og Sønner, tolv;
\par 10 det tredje Zakkur, hans Sønner og Brødre, tolv;
\par 11 det fjerde Jizri, hans Sønner og Brødre, tolv;
\par 12 det femte Netanja, hans Sønner og Brødre, tolv;
\par 13 det sjette Bukkija, hans Sønner og Brødre, tolv;
\par 14 det syvende Jesar'ela, hans Sønner og Brødre, tolv;
\par 15 det ottende Jesjaja, hans Sønner og Brødre, tolv;
\par 16 det niende Mattanja, hans Sønner og Brødre, tolv;
\par 17 det tiende Sjim'i, hans Sønner og Brødre, tolv;
\par 18 det ellevte Uzziel, hans Sønner og Brødre, tolv;
\par 19 det tolvte Hasjabja, hans Sønner og Brødre, tolv;
\par 20 det trettende Sjubael, hans Sønner og Brødre, tolv;
\par 21 det fjortende Mattitja, hans Sønner og Brødre, tolv;
\par 22 det femtende Jeremot, hans Sønner og Brødre, tolv;
\par 23 det sekstende Hananja, hans Sønner og Brødre, tolv;
\par 24 det syttende Josjbekasja, hans Sønner og Brødre, tolv;
\par 25 det attende Hanani, hans Sønner og Brødre, tolv;
\par 26 det nittende Malloti, hans Sønner og Brødre, tolv;
\par 27 det tyvende Eliata, hans Sønner og Brødre, tolv;
\par 28 det een og tyvende Hotir, hans Sønner og Brødre, tolv;
\par 29 det to og tyvende Giddalti, hans Sønner og Brødre, tolv;
\par 30 det tre og tyvende Mahaziot, hans Sønner og Brødre, tolv;
\par 31 det fire og tyvende Romamti-Ezer, hans Sønner og Brødre, tolv.

\chapter{26}

\par 1 Dørvogternes Skifeter var følgende: af Koraiterne Mesjelemja, en Søn af Kore af Abi'asafs Sønner.
\par 2 Mesjelemja havde Sønner: Zekarja den førstefødte, Jediael den anden, Zebadja den tredje, Jatniel den fjerde,
\par 3 Elam den femte, Johanan den sjette og Eljoenaj den syvende.
\par 4 Obed-Edom havde Sønner: Sjemaja den førstefødte, Jozabad den anden, Joa den tredje, Sakar den fjerde, Netan'el den femte,
\par 5 Ammiel den sjette, Issakar den syvende og Pe'ulletaj den ottende; thi Gud havde velsignet ham.
\par 6 Hans Søn Sjemaja fødtes Sønner, som var Herskere i deres Fædrenehus, da de var dygtige Folk.
\par 7 Sjemajas Sønner var: Otni, Retael, Obed, Elzabad og hans Brødre, dygtige Folk, Elihu og Semakja;
\par 8 alle disse hørte tillige med deres Sønner og Brødre til Obed-Edoms Sønner, dygtige Folk med Evner til Tjenesten, i alt to og tresindstyve Efterkommere af Obed-Edom.
\par 9 Mesjelemja havde Sønner og Brødre, dygtige Folk, atten.
\par 10 Hosa at Meraris Sønner havde Sønner: Sjimri, som var Overhoved - thi skønt han ikke var den førstefødte, gjorde hans Fader ham til Overhoved
\par 11 Hilkija den anden, Tebalja den tredje og Zekarja den fjerde. Hosas Sønner og Brødre var i alt tretten.
\par 12 Disse Dørvogternes Skifter, deres Overhoveder sammen med deres Brødre, blev Vagttjenesten overdraget, og således gjorde de Tjeneste i HERRENs Hus;
\par 13 og om hver enkelt Port kastede de Lod mellem små som store, efter deres Fædrenehuse.
\par 14 Loddet for Østporten ramte Sjelemja. Også for hans Søn Zekarja, en klog Rådgiver, kastede man Lod, og Loddet traf Nordporten.
\par 15 Por Obed-Edom traf det Sydporten og for hans Sønner Forrådskamrene.
\par 16 Og for Hosa traf Loddet Vestporten tillige med Sjalleketporten ved Vejen, der fører opad, den ene Vagtpost ved den anden.
\par 17 Mod Øst var der seks Leviter, mod Nord daglig fire, mod Syd daglig tre, ved hvert af Forrådskamrene to,
\par 18 ved Parbar mod Vest var der fire ved Vejen, to ved Parbar.
\par 19 Det var Dørvogternes Skifter af Koraiternes og Meraris Efterkommere.
\par 20 Deres Brødre Leviterne, som havde Tilsyn med Guds Hus's Skatkamre og Skatkamrene til Helliggaverne:
\par 21 Ladans Sønner, Gersoniternes Efterkommere gennem Ladan, Overhovederne for Gersoniten Ladans Fædrenehuse: Jehieliterne.
\par 22 Jehieliternes Sønner Zetam og hans Broder Joel havde Tilsynet med HERRENs Hus's Skafte.
\par 23 Af Amramiterne, Jizhariterne, Hebroniterne og Uzzieliterne
\par 24 var Sjubael, en Søn af Moses's Søn Gersom, Overopsynsmand over Skattene.
\par 25 Hans fra Eliezer nedstammende Brødre: Hans Søn Rehabja, hans Søn Jesjaja, hans Søn Joram, hans Søn Zikri, hans Søn Sjelomit.
\par 26 Denne Sjelomit og hans Brødre havde Tilsynet med de Skatte af Helliggaver, som Kong David, Fædrenehusenes Overhoveder, Tusind- og Hundredførerne og Hærførerne havde helliget -
\par 27 de havde helliget dem af Krigsbyttet til Hjælp ved Bygningen at HERRENs Hus
\par 28 og med alt, hvad Seeren Samuel, Saul, Kisj's Søn, Abner, Ners Søn, og Joab, Zerujas Søn, havde helliget; alt, hvad der var helliget, var betroet Sjelomit og hans Brødre.
\par 29 Af Jizhariteme udtoges Honanja og hans Sønner til Arbejdet ude i Israel som Fogeder og Dommere.
\par 30 Af Hebroniterne var Hasjabja og hans Brødre, 1700 dygtige Folk, sat til at varetage alt, hvad der vedrørte HERRENs Tjeneste og Kongens Tjeneste i Israel vesten for Jordan.
\par 31 Til Hebroniterne hørte Jerija, Overhovedet forHebroniterne, efter deres Nedstamning, efter Fædrenehusene - i Davids fyrretyvende Regeringsår blev der iværksat en Undersøgelse angående dem, og der fandtes dygtige Folk iblandt dem i Ja'zer i Gilead
\par 32 og hans Brødre, 2700 dygtige Folk, Overhoveder for Fædrenehusene; dem satte Kong David over Rubeniteme, Gaditerne og Manasses halve Stamme til at varetage alle Sager, som vedrørte Gud og Kongen.

\chapter{27}

\par 1 Israeliterne efter deres Tal: Fædrenehusenes Overhoveder, Tusind- og Hundredførerne og deres Fogeder, som tjente Kongen i alle Sager vedrørende Skifterne, de, der skiftevis trådte til og fra hver Måned hele Året rundt, hvert Skifte på 24000 Mand:
\par 2 Over det første Skifte, den første Måneds Skifte stod Isjba'al, Zabdiels Søn - til hans Skifte hørte 24000 Mand
\par 3 af Perez's Efterkommere, Overhoved for alle Hærførerne; det var den første Måned.
\par 4 Over den anden Måneds Skifte stod Ahohiten El'azar, Dodajs Søn; til hans Skifte hørte 24000 Mand.
\par 5 Den tredje Hærfører, ham i den tredje Måned, var Benaja, Ypperstepræsten Jojadas Søn; til hans Skifte hørte 24000 Mand.
\par 6 Denne Benaja var Helten blandt de tredive og stod i Spidsen for de tredive, og ved hans Skifte var hans Søn Ammizabad.
\par 7 Den fjerde, ham i den fjerde Måned, var Joabs Broder Asa'el og efter ham hans Søn Zebadja; til hans Skifte hørte 24000 Mand.
\par 8 Den femte, ham i den femte Måned, var Hærføreren Zeraiten Sjamhut; til hans Skifte hørte 24000 Mand.
\par 9 Den sjette, ham i den sjette Måned, var Ira, Ikkesjs Søn, fra Tekoa; til hans Skifte hørte 24000 Mand.
\par 10 Den syvende, ham i den syvende Måned, var Peloniten Helez af Efraimiterne; til hans Skifte hørte 24000 Mand.
\par 11 Den ottende, ham i den ottende Måned, var Husjatiten Sibbekaj af Zeras Slægt; til hans Skifte hørte 24000 Mand.
\par 12 Den niende, ham i den niende Måned, var Abiezer fra Anatot at Benjaminiterne; til hans Skifte hørte 24000 Mand.
\par 13 Den tiende, ham i den tiende Måned, var Maharaj fra Netofa af Zeras Slægt, til hans Skifte hørte 24000 Mand.
\par 14 Den ellevte, ham i den ellevte Måned, var Benaja fra Pir'aton af Efraimiterne; til hans Skifte hørte 24000 Mand.
\par 15 Den tolvte, ham i den tolvte Måned, var Heldaj fra Netofa af Otniels Slægt; til hans Skifte hørte 24000 Mand.
\par 16 I Spidsen for Israels Stammer stod: Som Fyrste for Rubeniterne Eliezer, Zikris Søn; for Simeoniterne Sjefatja, Ma'akas Søn;
\par 17 for Levi Hasjabja, Hemuels Søn, for Aron Zadok;
\par 18 for Juda Eliab, en af Davids Brødre; for Issakar Omri, Mikaels Søn;
\par 19 for Zebulon Jisjmaja, Obadjas Søn; for Naftali Jerimot, Azriels Søn;
\par 20 for Efraimiterne Hosea, Azazjas Søn; for Manasses halve Stamme Joel, Pedajas Søn;
\par 21 for Manasses anden Halvdel i Gilead Jiddo, Zekarjas Søn; for Benjamin Ja'asiel, Abners Søn;
\par 22 for Dan Azar'el, Jerobams Søn. Det var Israels Stammers Øverster.
\par 23 David tog ikke Tal på dem, dervar under tyve År, thi HERREN havde forjættet at ville gøre Israel talrigt som Himmelens Stjerner.
\par 24 Joab, Zerujas Søn, begyndte at tælle dem, men fuldførfe det ikke; thi for den Sags Skyld ramtes Israel af Vrede, og Tallet indførtes ikke i Kong Davids Krønike.
\par 25 Over Kongens Skatte havde Azmavet, Adiels Søn, Opsynet, og over Forrådene ude på Landet, i Byerne, Landsbyerne og Fæstningerne Jonatan, Uzzijas Søn;
\par 26 over Markarbejderne ved Jordens Dyrkning Ezri, Kelubs Søn;
\par 27 over Vingårdene Sjim'i fra Rama; over Vinforrådene i Vingårdene Sjifmiten Zabdi;
\par 28 over Oliventræerne og Morbærfigentræerne i Lavlandet Ba'al-Hanan fra Geder; over Olieforrådene Joasj;
\par 29 over Hornkvæget, der græssede på Saron, Saroniten Sjitraj; over Hornkvæget i Dalene Sjafat, Adlajs Søn;
\par 30 over Kamelerne Ismaelitem Obil; over Æslerne Jedeja fra Meronot;
\par 31 over Småkvæget Hagriten Jaziz. Alle disse var Overopsynsmænd over Kong Davids Ejendele.
\par 32 Davids Farbroder Jonatan, en indsigtsfuld og skriffkyndig Mand, var Rådgiver. Jehiel, Hakmonis Søn, opdrog Kongens Sønner.
\par 33 Akitofel var Kongens Rådgiver og Arkiten Husjaj Kongens Ven.
\par 34 Akitofels Eftermand var Jojada, Benajas Søn, og Ebjatar. Joab var Kongens Hærfører.

\chapter{28}

\par 1 Derpå samlede David i Jerusalem alle Israels Øverster, Stammeøversterne, Skifternes Øverster, som var i Kongens Tjeneste, Tusind- og Hundredførerne, Overopsynsmændene over alle Kongens og hans Sønners Ejendele og Kvæg, ligeledes Hofmændene, Kærnetropperne og alle dygtige Krigere.
\par 2 Kong David rejste sig op og sagde: "Hør mig, mine Brødre og mit Folk! Jeg havde i Sinde at bygge HERRENs Pagts Ark og vor Guds Fodskammel et Hus at hvile i og havde truffet forberedelser til at bygge.
\par 3 Men Gud sagde til mig: Du skal ikke bygge mit Navn et Hus, thi du er en Krigens Mand og har udgydt Blod!
\par 4 Af mit Fædrenehus udvalgte HERREN, Israels Gud, ene mig til Konge over Israel evindelig, thi han udvalgte Juda til Fyrste og af Juda mit Fædrenehus, og mellem min Faders Sønner fandt han Behag i mig, så han gjorde mig til Konge over hele Israel.
\par 5 Og af alle mine Sønner - HERREN har givet mig mange Sønner har han udvalgt min Søn Salomo til at sidde på HERRENs Kongetrone og herske over Israel.
\par 6 Og han sagde til mig: Din Søn Salomo er den, som skal bygge mit Hus og mine Forgårde, thi ham har jeg udvalgt til min Søn, og jeg vil være ham en Fader;
\par 7 jeg vil grundfæste hans Kongedømme til evig Tid, hvis han holder fast ved mine Bud og Lovbud og gør efter dem således som nu.
\par 8 Og nu, for hele Israels, HERRENs Forsamlings, Øjne og i vor Guds Påhør siger jeg: Stræb at holde alle HERREN eders Guds Bud, at I må eje dette herlige Land og lade det gå i Arv til eders Efterkommere til evig Tid!
\par 9 Og du, min Søn Salomo! Kend din Faders Gud og tjen ham med et helt Hjerte og en villig Sjæl, thi HERREN ransager alle Hjerter og kender alt, hvad der rører sig i deres Tanker. Hvis du søger ham, vil han lade sig fmde af dig, men forlader du ham, vil han forkaste dig for evigt.
\par 10 Så se da til, thi HERREN har udvalgt dig til at bygge et Hus til Helligdom! Gå til Værket med Frimodighed!"
\par 11 Derpå gav David sin Søn Salomo Planen til Forhallen, Templets Bygninger, Forrådskamrene, Rummene på Taget, de indre Kamre og Hallen til Sonedækket
\par 12 og Planen til alt, hvad der stod for hans Tanke med Hensyn til HERRENs Hus's Forgårde og alle Kamrene rundt om, Guds Hus's Skatkamre og Skatkamrene til Helliggaverne,
\par 13 fremdeles Anvisninger om Præsternes og Leviternes Skifter og alt Arbejdet ved Tjenesten i HERRENs Hus og alting, som hørte til Tjenesten i HERRENs Hus,
\par 14 om Guldet, den Vægt, der skulde til hver enkelt Ting, som hørte til Tjenesten, og om alle Sølvtingene, den Vægt, der skulde til hver enkelt Ting, som hørte til Tjenesten,
\par 15 fremdeles om Vægten på Guldlysestagerne og deres Guldlamper, Vægten på hver enkelt Lysestage og dens Lamper, og på Sølvlysestagerne, Vægten på hver enkelt Lysestage og dens Lamper, svarende til hver enkelt Lysestages Brug ved Tjenesten,
\par 16 fremdeles om Vægten på Guldet til Skuebrødsbordene, til hvert enkelt Skuebrødsbord, og på Sølvet til Sølvbordene
\par 17 og om Gaflerne, Skålene og Krukkerne af purt Guld og om Guldbægrene, Vægten på hvert enkelt Bæger, og Sølvbægrene, Vægten på hvert enkelt Bæger,
\par 18 øg om Vægten på Røgofferalteret af purt Guld og om Tegningen til Vognen, til Guldkeruberoe, som udbredte Vingerne skærmende over HERRENs Pagts Ark.
\par 19 "HERREN har sat mig ind i alt dette ved et Skrift, jeg har fra hans egen Hånd, i alle de Arbejder, Planen omfatter."
\par 20 Derpå sagde David til sin Søn Salomo: "Gå til Værket og vær frimodig og stærk, frygt ikke og tab ikke Modet, thi Gud HERREN, min Gud, vil være med dig! Han vil ikke slippe dig og ikke forlade dig, før alt Arbejdet med HERRENs Hus er fuldført.
\par 21 Se, Præsternes og Leviternes Skifter er rede til alt Arbejdet ved Guds Hus; og til alt, hvad der skal udføres, har du Folk hos dig, der alle er villige og forstår sig på Arbejder af enhver Art, og Øversterne og hele Folket er rede til alt, hvad du kræver."

\chapter{29}

\par 1 Fremdeles sagde Kong David til hele Forsamlingen: "Min Søn Salomo, som Gud har udvalgt, er ung og uudviklet, og Arbejdet er stort, thi Borgen er ikke bestemt for et Menneske, men for Gud HERREN.
\par 2 Jeg har derfor sat al min Kraft ind på til min Guds Hus at tilvejebringe Guldet, Sølvet, Kobberet, Jernet og Træet til det, der skal laves af Guld, Sølv, Kobber, Jern og Træ, desuden Sjohamsten og Indfatningssten, Rubiner, brogede Sten, alle Slags Ædelsten og Marmorsten i Mængde.
\par 3 I min Glæde over min Guds Hus giver jeg der hos til min Guds Hus, hvad jeg ejer af Guld og Sølv, ud over alt, hvad jeg har bragt til Veje til det hellige Hus:
\par 4 3.000 Talenter Guld, Offrguld, og 7.000 Talenter lutret Sølv til at overtrække Bygningernes Vægge med,
\par 5 Guldet til det, der skal forgyldes, og Sølvet til det, der skal forsølves, og til ethvert Arbejde, der skal udføres af Håndværkernes Hånd. Hvem er nu villig til i Dag at bringe HERREN Gaver?"
\par 6 Da kom Øversterne for Fædrenehusene, Øversterne for Israels Stammer, Tusind- og Hundredførerne og Øversterne i Kongens Tjeneste frivilligt
\par 7 og gav til Arbejdet på Guds Hus 5.000 Talenter og 10.000 Darejker Guld, 10.000 Talenter Sølv, 18.000 Talenter Kobber og 100.000 Talenter Jern;
\par 8 og de, som ejede Ædelsten, gav dem til HERRENs Hus's Skat, der stod under Gersoniten Jehiels Tilsyn.
\par 9 Og Folket glædede sig over deres Vilje til at give, thi af et helt Hjerte gav de HERREN frivillige Gaver; også Kong David følte stor Glæde.
\par 10 Og David priste HERREN i hele forsamlingens Nærværelse, og David sagde: "Lovet være du HERRE, vor Fader Israels Gud fra Evighed til Evighed!
\par 11 Din, HERRE, er Storheden, Magten, Æren, Glansen og Herligheden, thi alt i Himmelen og på Jorden er dit; dit, o HERRE, er Riget, og selv løfter du dig som Hoved over alle.
\par 12 Rigdom og Ære kommer fra dig, og du hersker over alt; i din Hånd er Kraft og Vælde, og i din Hånd står det at gøre, hvem det skal være, stor og stærk.
\par 13 Derfor priser vi dig nu, vor Gud, og lovsynger dit herlige Navn!
\par 14 Thi hvad er jeg, og hvad er mit Folk, at vi selv skulde evne at give sådanne frivillige Gaver? Fra dig kommer det alt sammen, og af din egen Hånd har vi givet dig det.
\par 15 Thi vi er fremmede for dit Åsyn og Gæster som alle vore Fædre; som em Skygge er vore Dage på Jorden, uden Håb!
\par 16 HERRE vor Gud, al denne Rigdom, som vi har bragt til Veje for at bygge dit hellige Navn et Hus, fra din Hånd kommer den, og dig tilhører det alt sammen.
\par 17 Jeg ved, min Gud. at du prøver Hjerter og har Behag i Oprigtighed; af oprigtigt Hjerte har jeg villigt givet alt dette, og nu har jeg set med Glæde, at dit Folk, der er her til Stede, villigt har givet dig Gaver.
\par 18 HERRE, vore Fædre Abrahams, Isaks og Israels Gud, bevar til evig Tid et sådant Sind og sådanne Tanker i dit Folks Hjerte og vend deres Hjerter til dig!
\par 19 Og giv min Søn Salomo et helt Hjerte til at holde dine Bud, Vidnesbyrd og Anordninger og udføre det alt sammen og bygge den Borg, jeg har truffet Forberedelser til at opføre!"
\par 20 Derpå sagde David til hele Forsamlingen: "Lov HERREN eders Gud!" Og hele Forsamlingen lovede HERREN, deres Fædres Gud, og kastede sig ned for HERREN og Kongen.
\par 21 Så ofrede de Slagtofre til HERREN, og Dagen efter bragte de ham som Brændoffer l000 Tyre, 1000 Vædre og 1000 Lam med tilhørende Drikofre og en Mængde Slagtofre for hele Israel;
\par 22 og de spiste og drak den Dag for HERRENs Åsyn med stor Glæde. Derefter indsatte de på ny Davids Søn til Konge, og de hyldede ham som HERRENs Fyrste og Zadok som Præst;
\par 23 og Salomo satte sig på HERRENs Trone som Konge i sin Fader Davids Sted; Lykken var med ham, og hele Israel var ham lydigt;
\par 24 og alle Øversterne og Kærnetropperne, ligeledes alle Kong Davids Sønner hyldede Kong Salomo.
\par 25 HERREN gjorde Salomo overmåde mægtig for hele Israels Øjne og gav ham en kongelig Herlighed, som ingen Konge før ham havde haft i Israel.
\par 26 David, Isajs Søn, havde hersket over hele Israel.
\par 27 Tiden, han var Konge over Israel, udgjorde fyrretyve År; i Hebron herskede han syv År, i Jerusalem tre og tredive År.
\par 28 Han døde i en god Alder, mæt af Dage, Rigdom og Ære; og hans Søn Salomo blev Konge i hans Sted.
\par 29 Kong Davids Historie fra først til sidst står optegnet i Seeren Samuels Krønike, Profeten Natans Krønike og Seeren Gads Krønike
\par 30 tillige med hele hans Regering og hans Heltegerninger og de Tildragelser, som hændtes ham og Israel og alle Lande og Riger.


\end{document}