\begin{document}

\title{2 Chronicles}


\chapter{1}

\par 1 Salomo, Davids Søn, fik sikret sig Magten, og HERREN hans Gud var med ham og gjorde ham overmåde mægtig.
\par 2 Da tilsagde Salomo hele Israel, Tusind- og Hundredførerne, Dommerne og alle Israels Øverster, Fædrenehusenes Overhoveder,
\par 3 og ledsaget af hele Forsamlingen drog han til Offerhøjen i Gibeon. Der stod Guds Åbenbaringstelt, som HERRENs Tjener Moses havde rejst i Ørkenen,
\par 4 men Guds Ark havde David bragt op fra Kijat-Jearim til det Sted, han havde beredt den, idet han havde ladet rejse et Telt til den i Jerusalem.
\par 5 Men Kobberalteret, som Bezal'el, en Søn af Hurs Søn Uri, havde lavet, stod der foran HERRENs Bolig, og der søgte Salomo og Forsamlingen ham.
\par 6 Og Salomo ofrede der på Åbenbaringsteltets Kobberalter, som stod foran HERRENs Åsyn, han ofrede 1000 Brændofre derpå.
\par 7 Samme Nat lod Gud sig til Syne for Salomo og sagde til ham: "Sig, hvad du ønsker, jeg skal give dig!"
\par 8 Da sagde Salomo til Gud: "Du viste stor Miskundhed mod min Fader David, og du har gjort mig til Konge i hans Sted.
\par 9 Så lad da, Gud HERRE, din Forjæftelse til min Fader David gå i Opfyldelse, thi du har gjort mig til Konge over et Folk, der er talrigt som Jordens Støv.
\par 10 Giv mig derfor Visdom og Indsigt, så jeg kan færdes ret over for dette Folk, thi hvem kan dømme dette dit store Folk?"
\par 11 Da sagde Gud til Salomo: "Fordi din Hu står til dette, og du ikke bad om Rigdom, Gods og Ære eller om dine Avindsmænds Liv, ej heller om et langt Liv, men om Visdom og Indsigt til at dømme mit Folk, som jeg gjorde dig til Konge over,
\par 12 så skal du få Visdom og Indsigt; men jeg giver dig også Rigdom, Gods og Ære, så at ingen Konge før dig har haft Mage dertil. og ingen efter dig skal have det."
\par 13 Derpå begav Salomo sig fra Offerhøjen i Gibeon, fra Pladsen foran Åbenbaringsteltet, til Jerusalem; og han herskede som Konge over Israel.
\par 14 Salomo anskaffede sig Stridsvogne og Ryttere, og han havde 1400 Vogne og l2000 Ryttere; dem lagde han dels i Vognbyerne, dels hos sig i Jerusalem.
\par 15 Kongen bragte det dertil, af Sølv og Guld i Jerusalem var lige så almindeligt som Sten, og Cedertræ lige så almindeligt som Morbærfigentræ i Lavlandet.
\par 16 Hestene, Salomo indførte, kom fra Mizrajim og Kove; Kongens Handelsfolk købte dem i Kove.
\par 17 De udførte en Vogn fra Mizrajim for 600 Sekel Sølv, en Hest for 150. Ligeledes udførtes de ved Handelsfolkenes Hjælp til alle Hetiternes og Arams Konger.

\chapter{2}

\par 1 Salomo fik i Sinde at bygge et Hus for HERRENs Navn og et kongeligt Palads.
\par 2 Salomo lod derfor udskrive 70.000 Mand til Lastdragere og 80.000 Mand til Stenhuggere i Bjergene og 3600 til at føre Tilsyn med dem.
\par 3 Og Salomo sendte Kong Huram af Tyrus følgende Bud: "Vær mod mig som mod min Fader David, hvem du sendte Cedertræ, for at han kunde bygge sig et Palads at bo i.
\par 4 Se, jeg er ved at bygge HERREN min Guds Navn et Hus; det skal helliges ham, for at man kan brænde vellugtende Røgelse for hans Åsyn, altid have Skuebrødene fremme og ofre Brændofre hver Morgen og Aften, på Sabbaterne, Nymånedagene og HERREN vor Guds Højtider, som det til evig Tid påhviler Israel.
\par 5 Huset, jeg vil bygge, skal være stort; thi vor Gud er større end alle Guder.
\par 6 Hvem magter at bygge ham et Hus, når dog Himmelen og Himlenes Himle ikke kan rumme ham? Og hvem erjeg, at jeg skulde bygge ham et Hus, når det ikke var for at tænde Offerild for hans Åsyn!
\par 7 Så send mig da en Mand, der har Forstand på at arbejde i Guld. Sølv, Kobber, Jern, rødt Purpur, Karmoisin og violet Purpur og forstår sig på Billedskærerarbejde, til at arbejde sammen med mine egne Mestre her i Juda og Jerusalem, dem, som min Fader antog;
\par 8 og send mig Ceder- Cypres- og Algummimtræ fra Libanon, thi jeg ved at dine Folk forstår at fælde Libanons Træer; og mine Folk skal hjælpe dine;
\par 9 men Træ må jeg have i Mængde, thi Huset, jeg vil bygge, skal være stort, det skal være et Underværk.
\par 10 For Tømrerne, som fælder Træerne, vil jeg til Underhold for dine Folk give 20.000 Kor Hvede, 20.000 Kor Byg, 20.000 Bat Vin og 20.000 Bat Olie."
\par 11 Kong Huram af Tyrus svarede i et Brev, som han sendte Salomo: "Fordi HERREN elsker sit Folk, har han gjort dig til Konge over dem!"
\par 12 Og Huram føjede til: "Lovet være HERREN, Israels Gud, som har skabt Himmelen og Jorden, at han har givet Kong David en viis Søn, der har Forstand og Indsigt til at bygge et Hus for HERREN og et kongeligt Palads.
\par 13 Her sender jeg dig en kyndig og indsigtsfuld Mand, Huram-Abi;
\par 14 han er Søn af en Kvinde fra Dan, men hans Fader er en Tyrier; han forstår at arbejde i Guld. Sølv, Kobber, Jern, Stem, Træ, rødt og violet Purpur, fint Linned og karmoisinfarvet Stof, forstår sig på al Slags Billedskærerarbejde og kan udtænke alle de Kunstværker, han sættes til; han skal arbejde sammen med dine og med min Herres, din Fader Davids, Mestre.
\par 15 Min Herre skal derfor sende sine Trælle Hveden, Byggen, Vinen og Olien, han talte om;
\par 16 så vil vi fælde så mange Træer på Libanon, som du har Brug for, og sende dig dem i Tømmerflåder på Havet til Jafo; men du må selv få dem op til Jerusalem."
\par 17 Salomo lod da alle fremmede Mænd, der boede i Israels Land, tælle - allerede tidligere havde hans fader David ladet dem tælle - og de fandtes at udgøre 153.600.
\par 18 Af dem gjorde han 70.000 til Lastdragere, 80.000 til Stenhuggere i Bjergene og 3.600 til Opsynsmænd til at lede Folkenes Arbejde.

\chapter{3}

\par 1 Derpå tog Salomo fat på at bygge HERRENs Hus i Jerusalem på Morija Bjerg, hvor HERREN havde ladet sig til Syne for hans Fader David, på det Sted, David havde beredt, på Jebusiten Ornans Tærskeplads.
\par 2 Han tog fat på Byggearbejdet i den anden Måned i sit fjerde Regeringsår.
\par 3 Målene på Grunden, som Salomo lagde ved Opførelsen af Guds Hus, var følgende: Længden var tresindstyve Alen efter gammelt Mål, Bredden tyve.
\par 4 Forhallen foran Templets Hellige var tyve Alen lang, svarende til Templets Bredde, og tyve Alen høj; og han overtrak den indvendig med purt Guld.
\par 5 Den store Hal dækkede han med Cyprestræ og overtrak den desuden med ægte Guld og prydede den med Palmer og Kranse.
\par 6 Han smykkede Hallen med Ædelsten; og Guldet var Parvajimguld;
\par 7 han overtrak Templet, Bjælkerne, Dørtærsklerne, Væggene og Dørfløjene med Guld og lod indgravere Keruber på Væggene.
\par 8 Han byggede fremdeles det Allerhelligste; dets Længde på tværs af Templet var tyve Alen, dets Bredde tyve; og han overtrak det med ægte Guld til en Vægt af 600 Talenter.
\par 9 Naglerne havde en Vægt af halvtredsindstyve Guldsekel; og Rummene på Taget overtrak han med Guld.
\par 10 I det Allerhelligste satte han to Keruber i Billedskærerarbejde, og han overtrak dem med Guld.
\par 11 Kerubernes Vinger målte tilsammen tyve Alen i Længden; den enes ene Vinge, fem Alen lang, rørte Hallens ene Væg, medens den anden, fem Alen lang, rørte den andens Vinge;
\par 12 og den anden Kerubs ene Vinge, fem Alen lang, rørte Hallens modsatte Væg, medens den anden, fem Alen lang, nåede til den førsfe Kerubs Vinge.
\par 13 Disse Kerubers Vinger mnålte i deres fulde Udstrækning tyve Alen og de stod oprejst med Ansigtet indad.
\par 14 Tillige lavede han Forhænget af violet og rødt Purpur, karmoisinfarvet Stof og fint Linned og prvdede det med Keruber.
\par 15 Foran Templet lavede han to Søjler. De var fem og tredive Alen høje, og Søjlehovedet oven på dem var fem Alen.
\par 16 Så lavede han Kranse som en Halskæde og anbragte dem øverst på Søjlerne, og fremdeles lavede han l00 Granatæbler og satte dem på Kransene.
\par 17 Disse Søjler rejste han foran Helligdommen, en til højre og en til venstre: den højre kaldte han Jakin, den venstre Boaz.

\chapter{4}

\par 1 Fremdeles lavede han et Kobberalter, tyve Alen bredt og ti Alen højt.
\par 2 Fremdeles lavede han Havet i støbt Arbejde, ti Alen fra Rand til Rand, helt rundt, fem Alen højt; det målte tredive Alen i Omkreds.
\par 3 Under Randen var det hele Vejen rundt omgivet af agurklignende Prydelser, der omsluttede Havet helt rundt, tredive Alen; i to Rækker sad de agurklignende Prydelser, støbt i eet dermed.
\par 4 Det stod på tolv Okser, således at tre vendte mod Nord, tre mod Vest, tre mod Syd og tre mod Øst; Havet stod oven på dem; de vendte alle Bagkroppen indad.
\par 5 Det var en Håndsbred tykt, og Randen var formet som Randen på et Bæger, som en udsprungen Lilje. Det tog 3000 Bat.
\par 6 Fremdeles lavede han ti Bækkener og satte fem til højre og fem til venstre, til Tvætning; i dem skyllede man, hvad der brugtes ved Brændofrene, medens Præsterne brugte Havet til at tvætle sig i.
\par 7 Fremdeles lavede han de ti Guldlysestager, som de skulde væte, og satte dem i Helligdommen, fem til højre og fem til venstre.
\par 8 Fremdeles lavede han ti Borde og satte dem i Helligdommen, fem til højre og fem til venstre; fillige lavede han 100 Skåle af Guld.
\par 9 Fremdeles indrettede han Præsternes Forgård og den store Gård og Porte til Gården; Portfløjene overtrak han med Kobber.
\par 10 Havet opstillede han ved Templets Sydside, ved det sydøstre Hjørne.
\par 11 Fremdeles lavede Huram Karrene, Skovlene og Skålene. Dermed var Huram færdig med sit Arbejde for Kong Salomo ved Guds Hus:
\par 12 De to Søjler og de to kugleformede Søjlehoveder ovenpå, de to Fletværker til at dække de to, kugleformede Søjlehoveder på Søjlerne,
\par 13 de 400 Granatæbler til de to Fletværker, to Rækker Granatæbler til hvert Fletværk til at dække de to kugleformede Søjlehoveder på de to Søjler,
\par 14 de ti Stel med de ti Bækkener på,
\par 15 Havet med de tolv Okser under.
\par 16 Karrene, Skovlene og Skålene og alle de Ting, som hørte til, lavede Huram-Abi af blankt Kobber for Kong Salomo til HERRENs Hus.
\par 17 I Jordandalen lod Kongen dem støbe, ved Adamas Vadested mellem Sukkot og Zereda.
\par 18 Salomo lod alle disse Ting lave i stor Mængde, thi Kobberet blev ikke vejet.
\par 19 Og Salomo lod alle Tingene, som hørte til Guds Hus, lave: Guldalteret, Bordene, som Skuebrødene lå på,
\par 20 Lysestagerne med Lamperne, der skulde tændes på den foreskrevne Måde, foran Inderhallen, af purt Guld,
\par 21 med Blomsterbægrene, Lamperne og Lysesaksene af Guld, ja af det allerbedste Guld,
\par 22 Knivene, Skålene, Kanderne og Panderne af fint Guld og Dørhængslerne til Templet, til Inderdørene for det Allerltelligste og til Dørene for det Hellige, af Guld.

\chapter{5}

\par 1 Da hele Arbejdet, som Salomo lod udføre ved HERRENs Hus, var færdigt, bragte Salomo sin Fader Davids Helliggaver, Sølvet og Guldet, derind og lagde alle Tingene i Skatkamrene i Guds Hus.
\par 2 Derpå kaldte Salomo Israels Ældste og alle Stammernes Overhoveder, Israeliternes Fædrenehuses Øverster, sammen i Jerusalem for at føre HERRENs Pagts Ark op fra Davidsbyen, det er Zion.
\par 3 Så samledes alle Israels Mænd hos Kongen på Højtiden i Etanim Måned, det er den syvende Måned.
\par 4 Og alle Israels Ældste kom, og Leviterne har Arken.
\par 5 Og de bragte Arken op tillige med Åbenbaringsteltett og alle de hellige Ting, der var i Teltet; Præsterne og Leviterne bragte dem op:
\par 6 Og Kong Salomo tillige med hele Israels Menighed, som havde givet Møde hos ham foran Arken, ofrede Småkvæg og Hornkvæg, så meget, at det ikke var til at tælle eller overse.
\par 7 Så førte Præsterne HERRENs Pagts Ark ind på dens Plads i Templets Inderhal, det Allerhelligste, og stillede den under Kerubernes Vinger;
\par 8 og Keruberne udbredte deres Vinger over Pladsen, hvor Arken stod, og således dannede Keruberne et Dække over Arken og dens Bærestænger.
\par 9 Stængerne var så lange, at Enderne af dem kunde ses fra det Hellige foran Inderhallen, men de kunde ikke ses længere ude; og de er der den Dag i Dag.
\par 10 Der var ikke andet i Arken end de to Tavler, Moses havde lagt ned i den på Horeb, Tavlerne med den Pagt, HERREN havde sluttet med Israeliterne, da de drog bort fra Ægypten.
\par 11 Da Præsteme derpå gik ud af Helligdommen - alle de Præster, der var til Stede, havde nemlig helliget sig uden Hensyn til Skifterne;
\par 12 og alle de levitiske Sangere, Asaf, Heman og Jedutun tillige med deres Sønner og Brødre stod østen for Alteret i Klæder af fint Linned med Cymbler, Harper og Citre, og sammen med dem stod 120 Præster, der blæste i Trompeter -
\par 13 i samme Øjeblik som Trompetblæserne og Sangerne på een Gang stemte i for at love og prise HERREN og lod Trompeterne, Cymblerne og Musikinstrumenterne klinge og lovede HERREN med Ordene "thi han er god, og hans Miskundhed varer evindelig!" - fyldte Skyen HERRENs Hus,
\par 14 så at Præsterne af Skyen hindredes i at stå og udføre deres Tjeneste, thi HERRENs Herlighed fyldte Guds Hus.

\chapter{6}

\par 1 Ved den Lejlighed sang Salomo: "HERREN har sagt, han vil bo i Mulmet!"
\par 2 Nu har jeg bygget dig et Hus til Bolig, et Sted, du for evigt kan dvæle.
\par 3 Derpå vendte Kongen sig om og velsignede hele Israels Forsamling, der imens stod op;
\par 4 og han sagde: "Lovet være HERREN, Israels Gud, hvis Hånd har fuldført, hvad hans Mund talede til min Fader David, dengang han sagde:
\par 5 Fra den Dag jeg førte mit Folk ud af Ægypten, har jeg ikke udvalgt nogen By i nogen af Israels Stammer, hvor man skulde bygge et Hus til Bolig for mit Navn, og jeg har ikke udvalgt nogen Mand til at være Hersker over mit Folk Israel;
\par 6 men Jerusalem udvalgte jeg til Bolig for mit Navn, og David udvalgte jeg til at herske over mit Folk Israel.
\par 7 Og min Fader David fik i Sinde at bygge HERRENs, Israels Guds, Navn et Hus;
\par 8 men HERREN sagde til min Fader David: At du har i Sinde at bygge mit Navn et Hus, er ret af dig;
\par 9 dog skal ikke du bygge det Hus, men din Søn, der udgår af din Lænd, skal bygge mit Navn det Hus.
\par 10 Nu har HERREN opfyldt det Ord, han talede, og jeg er trådt i min Fader Davids Sted og sidder på Israels Trone, som HERREN sagde, og jeg har bygget HERRENs, Israels Guds, Navn Huset;
\par 11 og jeg har der beredt en Plads til Arken med den Pagt, HERREN sluttede med Israeliterne."
\par 12 Derpå trådte Salomo frem foran HERRENs Alfer lige over for hele Israels Forsamling og udbredte Hænderne.
\par 13 Salomo havde nemlig ladet lave en fem Alen lang, fem Alen bred og tre Alen høj Talerstol af Hobber og stillet den op midt i Gården; på den trådte han op og kastede sig på Knæ foran hele Israels Forsamling, udbredte sine Hænder mod Himmelen
\par 14 og sagde: "HERRE, Israels Gud, der er ingen Gud som du i Himmelen og på Jorden, du, som holder fast ved din Pagt og din Miskundhed mod dine Tjenere, når de af hele deres Hjerte vandrer for dit Åsyn,
\par 15 du, som har holdt, hvad du lovede din Tjener, min Fader David, og i Dag opfyldt med din Hånd, hvad du talede med din Mund.
\par 16 Så hold da nu, HERRE, Israels Gud, hvad du lovede din Tjener, min Fader David, da du sagde: En Efterfølger skal aldrig fattes dig for mit Åsyn til at sidde på Israels Trone, når kun dine Sønner vil tage Vare på deres Vej og vandre i min Lov, som du har vandret for mit Åsyn!
\par 17 Så lad nu, HERRE, Israels Gud, det Ord opfyldes, som du tilsagde din Tjener David!
\par 18 Men kao Gud da virkelig bo blandt Menneskene på Jorden? Nej visselig, Himlene, ja Himlenes Himle kan ikke rumme dig, langt mindre dette Hus, som jeg har bygget!
\par 19 Men vend dig til din Tjeners Bøn og Begæring, HERRE min Gud, så du hører det Råb og den Bøn, din Tjener opsender for dit Åsyn;
\par 20 lad dine Øjne være åbne over dette Hus både Dag og Nat, over det Sted, hvor du har sagt, du vilde stedfæste dit Navn, så du hører den Bøn, din Tjener opsender, vendt mod dette Sted!
\par 21 Og hør den Bøn, dinjener og dit Folk Israel opsender, vendt mod dette Sted; du høre den der, hvor du bor, i Himmelen, du høre og tilgive!
\par 22 Når nogen synder imod sin Næste, og man afkræver ham Ed og lader ham sværge, og han kommer og aflægger Ed foran dit Alter i dette Hus,
\par 23 så høre du det i Himmelen og gøre det og dømme dine Tjenere imellem, så du gengælder den skyldige og lader hans Gerning komme overhans Hoved og frikender den uskyldige og gør med ham efter hans Uskyld.
\par 24 Når dit Folk Israel tvinges til at fly for en Fjende, fordi de synder imod dig, og de så omvender sig og bekender dit Navn og opsender Bønner og Begæringer for dit Åsyn i dette Hus,
\par 25 så høre du det i Himmelen og tilgive dit Folk Israels Synd og føre dem tilbage til det Land, du gav dem og deres Fædre.
\par 26 Når Himmelen lukkes, så Regnen udebliver, fordi de synder imod dig, og de så beder, vendt mod dette Sted, og bekender dit Navn og omvender sig fra deres Synd, fordi du revser dem,
\par 27 så høre du det i Himmelen og tilgive din Tjeners og dit Folk Israels Synd, ja du vise dem den gode Vej, de skal vandre, og lade det regne i dit Land, som du gav dit Folk i Eje.
\par 28 Når der kommer Hungersnød i Landet, når der kommer Pest, når der kommer Hornbrand og Rust, Græshopper og Ædere, når Fjenden belejrer Folket i en af dets Byer, når alskens Plage og Sot indtræffer
\par 29 enhver Bøn, enhver Begæring, hvem den end kommer fra i hele dit Folk Israel, når de føler deres Plage og Smerte og udbreder Hænderne mod dette Hus,
\par 30 den høre du i Himmelen, der, hvor du bor, og tilgive, idet du gengælder enhver hans Færd, fordi du kender hans Hjerte, thi du alene kender Menneskebørnenes Hjerter,
\par 31 for at de må frygte dig og følge dine Veje, al den Tid de lever på den Jord, du gav vore Fædre.
\par 32 Selv den fremmede, der ikke hører til dit Folk Israel, men kommer fra et fjernt Land for dit store Navns, din stærke Hånds og din udstrakte Arms Skyld, når de kommer og beder, vendt mod dette Hus,
\par 33 da børe du det i Himmelen, der. hvor du bor, og da gøre du efter alt, hvad den fremmede råber til dig om, for at alle Jordens Folkeslag må lære dit Navn at kende og frygte dig ligesom dit Folk Israel og erkende, at dit Navn er nævnet over dette Hus, som jeg har bygget.
\par 34 Når dit Folk drager i Krig mod sine Fjender, hvor du end sender dem hen, og de beder til dig, vendt mod den By, du har udvalgt, og det Hus, jeg har bygget dit Navn,
\par 35 så høre du i Himmelen deres Bøn og Begæring og skaffe dem deres Ret.
\par 36 Når de synder imod dig - thi der er intet Menneske, som ikke synder - og du vredes på dem og giver dem i Fjendens Magt, og Sejrherrerne fører dem fangne til et andet Land, det være sig fjernt eller nær,
\par 37 og de så går i sig selv i det Land, de er bortført til, og omvender sig og råber til dig i deres Landflygtigheds Land og siger: Vi har syndet, handlet ilde og været ugudelige!
\par 38 når de omvender sig til dig al hele deres Hjerte og af hele deres Sjæl i Sejrherrernes Land, som de bortførtes til, og de beder, vendt mod deres Land, som du gav deres Fædre, mod den By, du har udvalgt, og det Hus, jeg har bygget dit Navn -
\par 39 så høre du i Himmelen, der, hvor du bor, deres Bøn og Begæring og skaffe dem deres Ret, og du tilgive dit Folk, hvad de syndede imod dig!
\par 40 Så lad da, min Gud, dine Øjne være åbne og dine Ører lytte til Bønnen, der bedes på dette Sted,
\par 41 bryd op da, Gud HERRE, til dit Hvilested, du selv og din Vældes Ark! Dine Præster, Gud HERRE, være iklædt Frelse, dine fromme glæde sig ved dine Goder!
\par 42 Gud HERRE, afvis ikke din Salvede, kom Nåden mod din Tjener David i Hu!"

\chapter{7}

\par 1 Da Salomo havde endt sin Bøn, for Ild ned fra Himmelen og fortærede Brændofferet og Slagtofrene, og HERRENs Herlighed fyldte Templet,
\par 2 og Præsterne kunde ikke gå ind i HERRENs Hus, fordi HERRENs Herlighed fyldte det.
\par 3 Og da alle Israeliterne så Ilden og HERRENs Herlighed fare ned over Templet, kastede de sig på Knæ på Stenbroen med Ansigtet mod Jorden og filbad og lovede HERREN med Ordene "thi han er god, og hans Miskundhed varer evindelig!"
\par 4 Kongen ofrede nu sammen med alt Folket Slagtofre for HEERRENs Åsyn.
\par 5 Til Slagtofferet tog Kong Salomo 22.000 Stykker Hornkvæg og 120.000 Stykker Småkvæg. Således indviede Kongen og alt Folket Guds Hus.
\par 6 Og Præsterne stod på deres Pladser, og Leviterne stød med HERRENs Musikinstrumenter, som Kong David havde ladet lave, for at love HERREN med Davids Lovsangs Ord "thi hans Miskundhed varer evindelig!" og Præsterne stod lige over for dem og blæste i trompeter, og hele Israel stod op:
\par 7 Og Salomo helligede den mellemste Del af Forgården foran HERRENs Hus, thi der måtte han ofre Brændofrene og Fedtstykkerne af Takofrene, da Kobberalteret, som Salomo havde ladet lave, ikke kunde rumme Brændofferet, Afgrødeoffret og Fedtstykkerne.
\par 8 Samtidig fejrede Salomo i syv Dage Højtiden sammen med hele Israel, en vældig Forsamling lige fra Egnen ved Hamat og til Ægyptens Bæk.
\par 9 Ottendedagen holdt man festlig Samling, thi de fejrede Alterets Indvielse i syv Dage og Højtiden i syv.
\par 10 Og på den tre og tyvende Dag i den syvende Måned lod han Folket gå hver til sit, glade og vel til Mode over den Godhed, HERREN havde vist sin Tjenet David og Salomo og sit Folk Israel.
\par 11 Salomo var nu færdig med at opføre HERRENs Hus og Kongens Palads; og alt, hvad Salomo havde sat sig for at udføre ved HERRENs Hus og sit Palads, havde han lykkeligt ført igennem.
\par 12 Da lod HERREN sig til Syne for Salomo om Natten og sagde til ham: "Jeg har hørt din Bøn og udvalgt mig dette Sted til Offersted.
\par 13 Dersom jeg tillukker Himmelen, så Regnen udebliver, eller jeg opbyder Græshopperne til at æde Landet op, eller jeg sender Pest i mit Folk,
\par 14 og mit Folk, som mit Navn nævnes over, da ydmyger sig, beder og søger mit Åsyn og vender om fra deres onde Veje, så vil jeg høre det i Himmelen og tilgive deres Synd og læge deres Land,
\par 15 Nu skal mine Øjne være åbne og mine Ører lytte til Bønnen, der bedes på dette Sted.
\par 16 Og nu har jeg udvalgt og billiget dette Hus, for at mit Navn kan bo der til evig Tid, og mine Øjne og mit Hjerte skal være der alle Dage.
\par 17 Hvis du nu vandrer for mit Åsyn som din Fader David, så du gør alt, hvad.jeg har pålagt dig, og, holder mine Anordninger og Lovbud,
\par 18 så vil jeg opretholde din Kongetrone, som jeg tilsagde din Fader David, da jeg sagde: En Efterfølger skal aldrig fattes dig til at herske over Israel.
\par 19 Men hvis I vender eder bort og forlader mine Anordninger og Bud, som jeg har forelagt eder, og går hen og dyrker fremmede Guder og tilbeder dem,
\par 20 så vil jeg rykke eder op fra mit Land, som jeg gav eder; og dette Hus, som jeg har belliget for mit Navn, vil jeg forkaste fra mit Åsyn og gøre det til Spot og Spe blandt alle Folk,
\par 21 og dette Hus, som var så ophøjet, over det skal enhver, som kommer der forbi, blive slået af Rædsel. Og når man siger: Hvorfor har HERREN handlet således mod dette Land og dette Hus?
\par 22 skal der svares: Fordi de forlod HERREN, deres Fædres Gud, som førte dem ud af Ægypten, og holdt sig til andre Guder, tilbad og dyrkede dem; derfor har HERREN bragt al denne Elendighed over dem!"

\chapter{8}

\par 1 Da de tyve År var omme, i hvilke Salomo havde bygget på HERRENs Hus og sit Palads -
\par 2 også de Byer, Huram afstod til Salomo, befæstede Salomo og lod Israeliterne bosætte sig i dem -
\par 3 drog Salomo til Hamat-Zoba og indtog det.
\par 4 Han befæstede også Tadmor i Ørkenen og alle de Forrådsbyer, han byggede i Hamat;
\par 5 ligeledes genopbyggedehan Øvre og Nedre-Bet-Horon, så de blev Fæstninger med Mure, Porte og Portslåer,
\par 6 ligeledes Ba'alat og alle Salomos Forrådsbyer, Vognbyerne og Rytterbyerne, og alt andet, som Salomo fik Lyst til at bygge i Jerusalem, i Libanon og i hele sit Rige.
\par 7 Alt, hvad der var tilbage af Hetiterne, Amoriterne, Perizziterne, Hivviterne og Jebusiterne, og som ikke hørte til Israeliterne,
\par 8 deres Efterkommere, som var tilbage efter dem i Landet, og som Israeliterne ikke havde tilintetgjort. dem udskrev Salomo til Hoveriarbejde, som det er den Dag i Dag.
\par 9 Af Israeliterne derimod gjorde Salomo ingen til Arbejdstrælle for sig, men de var Krigsfolk, Hærførere og Vognkæmpere hos ham og Førere for hans Stridsvogne og Rytteri.
\par 10 Tallet på Kong Salomos Overfogeder var 250; de havde Tilsyn med Folkene.
\par 11 Faraos Datter flyttede Salomo fra Davidsbyen ind i det Hus, han havde bygget til hende; thi han tænkte: "Jeg vil ikke have en Kvinde boende i Kong David af Israels Palads, thi bellige er de Steder, hvor HERRENs Ark kommer."
\par 12 Nu ofrede Salomo Brændofre til HERREN på HERRENs Alter, som han Havde bygget foran Forhallen,
\par 13 idet han ofrede, som det efter Moses's Bud hørte sig til hver enkelt Dag, på Sabbaterne, Nymånedagene og Højtiderne tre Gange om Året, de usyrede Brøds Højtid, Ugernes Højtid og Løvhytternes Højtid.
\par 14 Og efter den Ordning, hans Fader David havde truffet, satte han Præsternes Skifter til deres Arbejde og Leviterne til deres Tjeneste, til at synge Lovsangen og gå Præsterne til Hånde efter hver Dags Behov, ligeledes Dørvogterne efter deres Skifter til at holde Vagt ved de enkelte Porte; thi således var den Guds Mand Davids Bud.
\par 15 Og man fraveg ikke i mindste Måde Kongens Bud vedrørende Præsterne og Leviterne og Skatkamrene.
\par 16 Således fuldendtes hele Salomos Værk, fra den Dag Grundvolden lagdes til HERRENs Hus, til Salomo var færdig med HERRENs Hus.
\par 17 Ved den Tid drog Salomo til Ezjongeber og Elot ved Edoms Kyst;
\par 18 og Huram sendte ham Folk med Skibe og befarne Søfolk, der sammen med Salomos Folk sejlede til Ofir, hvor de hentede 450 Talenter Guld, som de bragte Kong Salomo.

\chapter{9}

\par 1 Da Dronningen af Saba hørte Salomos Ry, kom hun, for at prøve ham med Gåder, til Jerusalem med et såre stort Følge og med Kameler, der har Røgelse, Guld i Mængde og Ædelsten. Og da hun var kommet til Salomo, talte hun til ham om alt, hvad der lå hende på Hjerte.
\par 2 Men Salomo svarede på alle hendes Spørgsmål, og intet som helst var skjult for Salomo, han gav hende Svar på alt.
\par 3 Og da Dronningen at Saba så Salomos Visdom, Huset, han havde bygget,
\par 4 Maden på hans Bord, hans Folks Boliger, hans Tjeneres Optræden og deres Klæder, hans Mundskænke og deres Klæder og Brændofrene, han ofrede i HERRENs Hus, var hun ude af sig selv;
\par 5 og hun sagde til Kongen: "Sandt var, hvad jeg i mit Land hørte sige om dig og din Visdom!
\par 6 Jeg troede ikke, hvad der sagdes, før jeg kom og så det med egne Øjne; og se, ikke engang det halve af din store Visdom er mig fortalt, thi du overgår, hvad Rygtet sagde mig om dig!
\par 7 Lykkelige dine Mænd, lykkelige dine Folk, som altid er om dig og hører din Visdom!
\par 8 Lovet være HERREN din Gud, som fandt Behag i dig og satte dig på sin Trone som Konge for HERREN din Gud. Fordi din Gud elsker Israel og for at opretholde det evindeligt, satte han dig til Konge over dem, til at øve Ret og Retfærdighed!"
\par 9 Derpå gav hun Kongen l20 Guldtalenter, Røgelse i store Mængder og Ædelsten; og aldrig har der siden været så megen Røgelse i Landet som den, Dronningen af Saba gav Kong Salomo.
\par 10 Desuden bragte Hurams og Salomos Folk, som hentede Guld i Ofir, Algummimtræ og Ædelsten;
\par 11 og af Algummimtræet lod Kongen lave Rækværk til HERRENs Hus og Kongens Palads, desuden Citre og Harper til Sangerne; og Mage dertil var ikke tidligere set i Judas Land.
\par 12 Og Kong Salomo gav Dronningen af Saba alt, hvad hun ønskede og bad om, langt ud over hvad hun bragte til Kongen. Derpå vendte hun med sit Følge tilbage til sit Land.
\par 13 Vægten af det Guld, som i eet År indførtes af Salomo, udgjorde 666 Guldtalenter,
\par 14 deri ikke medregnet hvad der indkom i Afgift fra de undertvungne Folk og ved Købmændenes Handel og fra alle Arabiens Konger og Landets Statholdere, som bragte Salomo Guld og Sølv.
\par 15 Kong Salomo lod hamre 200 Guldskjolde, hvert på 600 Sekel Guld,
\par 16 og 300 mindre Guldskjolde, hvert på 300 Sekel Guld; dem lod Kongen henlægge i Libanonskovhuset.
\par 17 Fremdeles lod Kongen lave en stor, Elfenbenstrone, overtrukket mned lutret Guld.
\par 18 Tronen havde seks Trin, og på dens Ryg var der et Latn af Guld; på begge Sider af Sædet var der Arme, og ved Armene stod der to Løver;
\par 19 tillige stod der tolv Løver på de seks Trin, seks på hver Side. Der er ikke lavet Mage til Trone i noget andet Rige.
\par 20 Alle Kong Salomos Drikkekar var af Guld og alle Redskaber i Libanonskovhuset af fint Guld; Sølv regnedes ikke for noget i Kong Salomos Dage.
\par 21 Kongen havde nemlig Skibe, det sejlede på Tarsis med Hurams Folk; og een Gang hverttredje År kom Tarsisskibene, ladet med Guld, Sølv, Elfenben, Aber og Påfugle.
\par 22 Kong Salomo overgik alle Jordens Konger i Rigdom og Visdom.
\par 23 Alle Jordens Konger søgte hen til Salomo for at høre den Visdom, Gud havde lagt i hans Hjerte;
\par 24 og alle bragte de Gaver med: Sølv- og Guldsager, Klæder, Våben, Røgelse, Heste og Muldyr; således gik det År efter År.
\par 25 Salomo havde 4000 Spand Heste og Vogne og l2000 Ryttere; dem lagde han dels i Vognbyerne, dels hos sig i Jerusalem.
\par 26 Han herskede over alle Konger fra Floden til Filisternes Land og Ægyptens Grænse.
\par 27 Kongen bragte det dertil, at Sølv i Jerusalem var lige så almindeligt som Sten, og Cedertræ lige så almindeligt som Morbærfigentræ i Lavlandet.
\par 28 Der indførtes Heste til Salomo fra Mizrajim og fra alle Lande.
\par 29 Salomos øvrige Historie fra først til sidst findes optegnet i Profeten Natans Krønike, Siloniten Ahijas Profeti og Seeren Jedos Syn om Jeroboam, Nebats Søn.
\par 30 Salomo herskede i Jerusalem over hele Israel i fyrretyve År.
\par 31 Derpå lagde Salomo sig til Hvile hos sine Fædre og blev jordet i sin Fader Davids By. Og hans Søn Rehabeam blev Konge i hans Sted.

\chapter{10}

\par 1 Rehabeam begav sig til Sikem, thi derhen var hele Israel stævnet for at hylde ham som Konge.
\par 2 Men da Jeroboam, Nebats Søn, der opholdt sig i Ægypten, hvorhen han var flygtet for Kong Salomo, fik Nys derom, vendte han hjem fra Ægypten.
\par 3 Man sendte da Bud og lod ham kalde. Og Jeroboam kom. Da sagde hele Israel til Rehabeam:
\par 4 "Din Fader lagde et hårdt Åg på os, men let du nu det hårde Arbejde for os, som din Fader krævede, og det tunge Åg, han lagde på os, så vil vi tjene dig!"
\par 5 Han svarede dem: "Gå bort, bi tre Dage og kom så til mig igen!" Så gik Folket.
\par 6 Derpå rådførte Kong Rehabeam sig med de gamle, der havde stået i hans Fader Salomos Tjeneste, dengang han levede, og spurgte dem: "Hvad råder I mig til at svare dette Folk?"
\par 7 De svarede: "Hvis du i Dag vil være venlig mod dette Folk og føje dem og give dem gode Ord, så vil de blive dine Tjenere for bestandig!"
\par 8 Men han fulgte ikke det Råd, de gamle gav ham; derimod rådførte han sig med de unge, der var vokset op sammen med ham og stod i hans Tjeneste,
\par 9 og spurgte dem: "Hvad råder I os til at svare dette Folk, som kræver af mig, at jeg skal lette dem det Åg, min Fader lagde på dem?"
\par 10 De unge, der var vokset op sammen med ham, sagde da til ham: "Således skal du svare dette Folk, som sagde til dig: Din Fader lagde et tungt Åg på os, let du det for os! Således skal du svare dem: Min Lillefinger er tykkere end min Faders Hofter!
\par 11 Har derfor min Fader lagt et tungt Åg på eder, vil jeg gøre Åget tungere; har min Fader tugtet eder med Svøber, vil jeg tugte eder med Skorpioner!"
\par 12 Da Jeroboam og alt Folket Tredjedagen kom til Rehabeam, som Kongen havde sagt,
\par 13 gav han dem et hårdt Svar, og uden at tage Hensyn til de gamles Råd
\par 14 sagde han efter de unges Råd til dem: "Har min Fader lagt et tungt Åg på eder, vil jeg gøre det tungere; har min Fader tugtet eder med Svøber, vil jeg tugte eder med Skorpioner!"
\par 15 Kongen hørte ikke på Folket, thi Gud føjede det således for at opfylde det Ord, HERREN havde talet ved Ahija fra Silo til Jerøboam, Nebats Søn.
\par 16 Men da hele Israel mærkede, at Kongen ikke hørte på dem, gav Folket Kongen det Svar: "Hvad Del har vi i David? Vi har ingen Lod i Isajs Søn! Til dine Telte, Israel! Sørg nu, David, for dit eget Hus!" Derpå vendte Israel tilbage til sine Telte.
\par 17 Men over de Israeliter, der boede i Judas Byer, blev Rehabeam Konge.
\par 18 Nu sendte Kong Rehabeam Adoniram, der havde Opsyn med Hoveriarbejdet, ud til dem, men Israeliterne stenede ham til Døde. Da steg Kong Rehabeam i største Hast op på sin Stridsvogn og flygtede til Jerusalem.
\par 19 Så brød Israel med Davids Hus, og således er det den Dag i Dag.

\chapter{11}

\par 1 Da Rehabeam var kommet til Jerusalem, samlede han hele Judas og Benjamins Hus, 180.000 udsøgte Folk, øvede Krigere, til at føre Krig med Israel og vinde Kongedømmet tilbage til Rehabeam.
\par 2 Men da kom HERRENs Ord til den Guds Mand Sjemaja således:
\par 3 "Sig til Judas Konge Rehabeam, Salomos Søn, og til hele Israel i Juda og Benjamin:
\par 4 Så siger HERREN: I må ikke drage op og kæmpe med eders Brødre; vend hjem hver til sit, thi hvad her er sket, har jeg tilskikket!" Da adlød de HERRENs Ord og vendte tilbage og drog ikke mod Jeroboam.
\par 5 Rehabeam boede så i Jerusalem, og han befæstede flere Byer i Juda.
\par 6 Således befæstede han Betlehem, Etam, Tekoa,
\par 7 Bet-Zur, Soko, Adullam,
\par 8 Gat, Maresj a, Zif,
\par 9 Adorajim, Lakisj, Azeka,
\par 10 Zor'a, Ajjalon og Hebron, alle i Juda og Benjamin;
\par 11 og han gjorde Fæstningerne stærke, indsatte Befalingsmænd i dem og forsynede dem med Forråd af Levnedsmidler, Olie og Vin
\par 12 og hver enkelt By med Skjolde og Spyd og gjorde dem således meget stærke.
\par 13 Præsterne og Leviterne i hele Israel kom alle Vegne fra, hvor de boede, og stillede sig til hans Tjeneste;
\par 14 thi Leviterne forlod deres Græsmarker og Ejendom og begav sig til Juda og Jerusalem, fordi Jeroboam og hans Sønner afsatte dem fra Stillingen som HERRENs Præster,
\par 15 idet han indsatte sig Præster for Offerhøjene og Bukketroldene og Tyrekalvene, som han, havde ladet lave.
\par 16 Og i Følge med Leviterne kom fra alle Israels Stammer de, hvis Hjerte var vendt til at søge HERREN, Israels Gud, til Jerusalem for at ofre til HERREN, deres Fædres Gud;
\par 17 og de styrkede Juda Rige og hævdede Rehabeams, Salomos Søns, Magt i et Tidsrum af tre År. Thi i tre År fulgte han Davids og Salomos Veje.
\par 18 Rehabeam ægtede Mahalat; en Datter af Davids Søn Jerimot og Abihajil, en Datter af Eliab, Isajs Søn.
\par 19 Hun fødte ham Sønnerne Je'usj, Sjemarja og Zaham.
\par 20 Senere ægtede han Absaloms Datter Ma'aka, som fødte ham Abija, Attaj, Ziza og Sjelomit.
\par 21 Rehabeam elskede Absaloms Datter Ma'aka højere end sine andre Hustruer og Medhustruer; han havde nemlig atten Hustruer og tresindstyve Medhustruer og avlede otte og tyve Sønner og tresindstyve Døtre.
\par 22 Og Rehabeam satte Ma'akas Søn Abija til Overhoved, til Fyrste blandt hans Brødre; thi han havde i Sinde at gøre, ham til Konge;
\par 23 og han fordelte klogelig alle sine Sønner rundt i alle Judas og Benjamins Landsdele og i alle de befæstede Byer og gav dem rigeligt Underhold og skaffede dem Hustruer.

\chapter{12}

\par 1 Men da Rehabeams Kongedømme var grundfæstet og hans Magt styrket, forlod han tillige med hele Israel HERRENs Lov.
\par 2 Da drog i Kong Rehabeams femte Regeringsår Ægypterkongen Sjisjak op imod Jerusalem, fordi de havde været troløse mod HERREN,
\par 3 med 1200 Stridsvogne og 60.000 Ryttere, og der var ikke Tal på Krigerne, der drog med ham fra Ægypten, Libyere, Sukkijiter og Ætiopere;
\par 4 og efter at have indtaget Fæstningerne i Juda drog han mod Jerusalem.
\par 5 Da kom Profeten Sjemaja til Rehabeam og Judas Øverster, som var tyet sammen i Jerusalem for Sjisjak, og sagde til dem: "Så siger HERREN: I har forladt mig, derfor har jeg også forladt eder og givet eder i Sjisjaks Hånd!"
\par 6 Da ydmygede Israels Øverster og Kongen sig og sagde: "HERREN er retfærdig!"
\par 7 Og da HERREN så, at de havde ydmyget sig, kom HERRENs Ord til Sjemaja således: "De har ydmyget sig; derfor vil jeg ikke tilintetgøre dem, men frelse dem om ikke længe, og min Vrede skal ikke udgydes over Jerusalem ved Sjisjak;
\par 8 men de skal komme til at stå under ham og lære at kende Forskellen mellem at tjene mig og at tjene Hedningemagterne!"
\par 9 Så drog Sjisjak op mod Jerusalem og tog Skattene i HERRENs Hus og i Kongens Palads; alt tog han, også de Guldskjolde, Salomo havde ladet lave.
\par 10 Kong Rehabeam lod da i Stedet lave Kobberskjolde og gav dem i Forvaring hos Høvedsmændene for Livvagten, der holdt Vagt ved Indgangen til Kongens Palads;
\par 11 og hver Gang Kongen begav sig til HERRENs Hus, kom Livvagten og hentede dem, og bagefter bragte de dem tilbage til Vagtstuen.
\par 12 Men da han havde ydmyget sig, vendfe HERRENs Vrede sig fra ham, så han ikke helt tilintetgjorde ham; også i Juda var Forholdene gode.
\par 13 Således styrkede Kong Rehabeam sin Magt i Jerusalem og blev ved at herske; thi Rehabeam var een og fyrretyve År gammel, da han blev Konge, og han herskede sytten År i Jerusalem, den By, HERREN havde udvalgt af alle Israels Stammer for der at stedfæste sit Navn. Hans Moder var en ammonitisk Kvinde ved Navn Na'ama.
\par 14 Han gjorde, hvad der var ondt, thi hans Hjerte var ikke vendt til at søge HERREN.
\par 15 Rehabeams Historie fra først til sidst står jo optegnet i Profeten Sjemajas og Seeren tddos Krønike. Rehabeam og Jeroboam lå i Krig med hinanden hele Tiden.
\par 16 Så lagde Rehabeam sig til Hvile hos sine Fædre og blev jordet i Davidsbyen. Og hans Søn Abija blev Konge i hans Sted.

\chapter{13}

\par 1 I Kong Jerobeams attende Regeringsår blev Abija Konge over Juda.
\par 2 Tre År herskede han i Jerusalem. Hans Moder hed Mikaja og var Datter af Uriel fra Gibea. Abija og Jeroboam lå i Krig med hinanden.
\par 3 Abija åbnede Krigen med en krigsdygtig Hær, 400.000 udsøgte Mænd, og Jeroboam mødte ham med 800.000 udsøgte Mænd, dygtige Krigere,
\par 4 Da stillede Abija sig på Bjerget Zemarajim, der hører til Efraims Bjerge, og sagde: "Hør mig, Jeroboam og hele Israel!
\par 5 Burde I ikke vide, at HERREN, Israels Gud, har givet David og hans Efterkommere Kongemagten over Israel til evig Tid ved en Saltpagt?
\par 6 Men Jeroboam, Nebats Søn, Davids Søn Salomos Træl, rejste sig og gjorde Oprør mod sin Herre,
\par 7 og dårlige Folk, Niddinger, samlede sig om ham og bød Rehabeam, Salomos Søn, Trods; og Rebabeam var ung og veg og kunde ikke hævde sig over for dem.
\par 8 Og nu mener I at kunne hævde eder over for HERRENs Kongedømme i Davids Efterkommeres Hånd, fordi I er en stor Hob og på eders Side har de Guldkalve, Jeroboam lod lave eder til Guder!
\par 9 Har I ikke drevet HERRENs Præster, Arons Sønner, og Leviterne bort og skaffet eder Præster på samme Måde som Hedningefolkene? Enhver, der kommer med en ung Tyr og syv Vædre for at indsættes, bliver Præst for Guder, der ikke er Guder!
\par 10 Men vor Gud er HERREN, og vi har ikke forladt ham; de Præster, der tjener HERREN, er Arons Sønner og Leviterne udfører den øvrige Tjeneste;
\par 11 de antænder hver Morgen og Aften Brændofre til HERREN og vellugtende Røgelse, lægger Skuebrødene til Rette på Guldbordet og tænder Guldlysestagen og dens Lamper Aften efter Aften, thi vi holder HERREN vor Guds Forskrifter, men l har forladt ham!
\par 12 Se, med os, i Spidsen for os er Gud og hans Præster og Alarmtrompeterne, med hvilke der skal blæses til Kamp imod eder! Israeliter, indlad eder ikke i Kamp med HERREN, eders Fædres Gud, thi I får ikke Lykken med eder!"
\par 13 Jeroboam lod imidlertid Bagholdet gøre en omgående Bevægelse for at komme i Ryggen på dem, og således havde Judæerne Hæren foran sig og Bagholdet i Ryggen.
\par 14 Da Judæerne vendte sig om og så, at Angreb truede dem både forfra og bagfra, råbte de til HERREN, medens Præsterne blæste i Trompeterne.
\par 15 Så udstødte Judæerne Krigsskriget, og da Judæerne udstødte Krigsskriget, slog Gud Jeroboam og hele Israel foran Abija og Juda.
\par 16 Israeliterne flygtede for Judæerne, og Gud gav dem i deres Hånd;
\par 17 og Abija og hans Folk tilføjede dem et stort Nederlag, så der af Israeliterne faldt 500.000 udsøgte Krigere.
\par 18 Således ydmygedes Israeliterne den Gang, men Judæerne styrkedes, fordi de støttede sig til HERREN, deres Fædres Gud.
\par 19 Og Abija forfulgte Jeroboam og fratog ham flere Byer, Betel med Småbyer, Jesjana med Småbyer og Efrajin med Småbyer.
\par 20 Jeroboam kom ikke til Kræfter mere, så længe Abija levede; og HERREN slog ham, så han døde.
\par 21 Men Abijas Magt voksede. Han ægtede fjorten Hustruer og avlede to og tyve Sønner og seksten Døtre.
\par 22 Hvad der ellers er at fortælle om Abija, hans Færd og Ord, står jo optegnet i Profeten Iddos Udlægning".

\chapter{14}

\par 1 Så lagde Abija sig til Hvile hos sine Fædre, og man jordede ham i Davidsbyen; og hans Søn Asa blev Konge i hans Sted. På hans Tid havde Landet Fred i ti År.
\par 2 Asa gjorde, hvad der var godt og ret i HERREN hans Guds Øjne.
\par 3 Han fjernede de fremmede Altre og Offerhøjene, sønderbrød Stenstøtterne og omhuggede Asjerastøtterne
\par 4 og bød Judæerne søge HERREN, deres Fædres Gud, og holde Loven og Budet,
\par 5 og han fjernede Offerhøjene og Solstøtterne fra alle Judas Byer, og Landet havde Fred, så længe han levede.
\par 6 Han byggede Fæstninger i Juda, thi Landet havde Fred, og han havde ingen Krig i de År, thi HERREN lod ham have Ro.
\par 7 Han sagde da til Judæerne: "Lad os befæste disse Byer og omgive dem med Mure og Tårne, Porte og Portslåer, medens vi endnu har Landet i vor Magt, thi vi har søgt HERREN vor Gud; vi har søgt ham, og han har ladet os have Ro til alle Sider!" Så byggede de, og Lykken stod dem bi.
\par 8 Asa havde en Hær, af Juda 300.000 væbnet med Skjold og Spyd, og af Benjamin 280.000, der har Småskjolde og spændte Buer, alle sammen dygtige Krigere.
\par 9 Men Kusjiten Zera drog ud imod dem med en Hær på 1.000.000 Mand og 300 Stridsvogne. Da han havde nået Maresja,
\par 10 rykkede Asa ud imod ham, og de stillede sig op til Kamp i Zefatadalen ved Maresja.
\par 11 Da råbte Asa til HERREN sin Gud: "HERRE, hos dig er der ingen Forskel på at hjælpe den, der har megen Kraft, og den, der ingen har; hjælp os, HERRE vor Gud, thi til dig støtter vi os, og i dit Navn er vi draget mod denne Menneskemængde, HERRE, du er vor Gud, mod dig kan intet Menneske holde Stand."
\par 12 Da slog HERREN Kusjiterne foran Asa og Judæerne, og Kusjiterne tog Flugten.
\par 13 Asa og hans Folk forfulgte dem til Gerar, og alle Kusjiterne faldt, ingen reddede Livet, thi de knustes foran HERREN og hans Hær. Judæerne gjorde et umådeligt Bytte
\par 14 og indtog alle Byerne i Omegnen af Gerar, thi en HERRENs Rædsel var kommet over dem, og de plyndrede alle Byerne, thi der var et stort Bytte i dem;
\par 15 også indtog de Teltene til Kvæget og slæbte en Mængde Småkvæg og Kameler med sig; så vendte de tilbage til Jerusalem.

\chapter{15}

\par 1 Guds Ånd kom over Azarja, Odeds Søn,
\par 2 og han trådte frem for Asa og sagde til ham: "Hør mig, Asa og hele Juda og Benjamin! HERREN er med eder, når I er med ham; og hvis I søger ham, lader han sig finde af eder, men forlader I ham, forlader han også eder!
\par 3 I lange Tider var Israel uden sand Gud, uden Præster til at vejlede og uden Lov,
\par 4 men i sin Trængsel omvendte det sig til HERREN, Israels Gud, og søgte ham, og han lod sig finde at dem.
\par 5 I de Tider kunde ingen gå ud og ind i Fred, thi der var vild Rædsel over alle Landes Indbyggere;
\par 6 Folk knustes mod Folk, By mod By, thi Gud bragte dem i vild Rædsel med alle mulige Trængsler.
\par 7 Men I, vær frimodige og lad ikke Hænderne synke, thi der er Løn for eders Gerning."
\par 8 Da Asa hørte de Ord og den Profeti, tog han Mod til sig og fjernede de væmmelige Guder fra hele Judas og Benjamins Land og fra de Byer, han havde indtaget i Efraims Bjerge; oghan byggede påny HERRENs Alter foran HERRENs Forhal.
\par 9 Så samlede han hele Juda og Benjamin og de Folk fra Efraim, Manasse og Simeon, der boede som fremmede hos dem; thi en Mængde Israeliter var gået over til ham, da de så, at HERREN hans Gud var med ham.
\par 10 De samledes i Jerusalem i den tredje Måned i Asas femtende Regeringsår
\par 11 og ofrede den Dag HERREN Ofre af Byttet, de havde medbragt, 700 Stykker Hornkvæg og 7000 Stykker Småkvæg.
\par 12 Derpå sluttede de en Pagt om at søge HERREN, deres Fædres Gud, af hele deres Hjerte og hele deres Sjæl;
\par 13 og enhver, der ikke søgte HERREN, Israels Gud, skulde lide Døden, være sig lille eller stor, Mand eller Kvinde.
\par 14 Det tilsvor de HERREN med høj Røst under Jubel og til Trompeters og Horns Klang,
\par 15 Og hele Juda glædede sig over den Ed, thi de svor af hele deres Hjerte og søgte ham af hele deres Vilje; og han lod sig finde af dem og lod dem få Ro til alle Sider.
\par 16 Kong Asa fratog endog sin Moder Ma'aka Værdigheden som Herskerinde, fordi hun havde ladet lave et Skændselsbillede til Ære for Asjera; Asa lod hendes Skændselsbillede nedbryde, sønderknuse og brænde i Kedrons Dal.
\par 17 Vel forsvandt Offerhøjene ikke af Israel, men alligevel var Asas Hjerte helt med HERREN, så længe han levede.
\par 18 Og han bragte sin Faders og sine egne Helliggaver ind i Guds Hus, Sølv, Guld og forskellige Kar.
\par 19 Der var ikke Krige før efter Asas fem og tredivte Regeringsår.

\chapter{16}

\par 1 Men i Asas seks og tredivte Regeringsår drog Kong Ba'sja af Israel op imod Juda og befæstede Rama for at hindre, at nogen af Kong Asa af Judas Folk drog ud og ind.
\par 2 Da tog Asa Sølv og Guld ud af Skatkamrene i HERRENs Hus og i Kongens Palads og sendte det til Kong Benhadad af Aram, som boede i Darmaskus, idet han lod sige:
\par 3 "Der består en Pagt mellem mig og dig og mellem min Fader og din Fader; her sender jeg dig en Gave af Sølv og Guld; bryd derfor din Pagt med Kong Ba'sja af Israel, så at han nødes til at drage bort fra mig!"
\par 4 Benhadad gik ind på Kong Asas Forslag og sendte sine Hærførere mod Israels Byer og indtog Ijjon, Dan, Abel-Majim og Forrådshusene i Naftalis Byer.
\par 5 Da Ba'sja hørte det, opgav han at befæste Rama og standsede Arbejdet.
\par 6 Men Kong Asa tog hele Juda til at føre Stenene og Træværket, som Ba'sja havde brugt ved Befæstningen af Rama, bort, og han befæstede dermed Geba og Mizpa.
\par 7 På den Tid kom Seeren Hanani til Kong Asa af Juda og sagde til ham: "Fordi du søgte Støtte hos Aramæerkongen og ikke hos HERREN din Gud, skal Aramæerkongens Hær slippe dig af Hænde!
\par 8 Var ikke Kusjiterne og Libyerne en vældig Hær med en vældig Mængde Vogne og Ryttere? Men da du søgte Støtte hos HERREN, gav han dem i din Hånd!
\par 9 Thi HERRENs Øjne skuer omkring på hele Jorden, og han viser sig stærk til at hjælpe dem, hvis Hjerte er helt med ham. I denne Sag har du handlet som en Dåre; thi fra nu af skal du altid ligge i Krig!"
\par 10 Men Asa blev fortørnet på Seeren og satte ham i Huset med Blokken, thi han var blevet vred på ham derfor. Asa for også hårdt frem mod nogle af Folket på den Tid.
\par 11 Asas Historie fra først til sidst står jo optegnet i Bogen om Judas og Israels Konger.
\par 12 I sit ni og tredivte Regeringsår fik Asa en Sygdom i Fødderne, og Sygdommen blev meget alvorlig. Heller ikke under sin Sygdom søgte han dog HERREN, men Lægerne.
\par 13 Så lagde Asa sig til Hvile bos sine Fædre og døde i sit een og fyrretyvende Regerings,år.
\par 14 Man jordede ham i en Grav, han havde ladet sig udhugge i Davidsbyen, og lagde ham på et Leje, som man havde fyldt med vellugtende Urter og Stoffer, tillavet som Salve, og tændte et vældigt Bål til hans Ære.

\chapter{17}

\par 1 Hans Søn Josafat blev Konge i hans Sted. Han styrkede sin Stilling over for Israel
\par 2 ved at lægge Besætning i alle Judas befæstede Byer og indsætte Fogeder i Judas Land og de efraimitiske Byer, hans Fader Asa havde indtaget.
\par 3 Og HERREN var med Josafat, thi han vandrede de Veje, hans Fader David til at begynde med havde vandret, og søgte ikke hen til Ba'alerne,
\par 4 men til sin Faders Gud og fulgte hans Bud og gjorde ikke som Israel.
\par 5 Derfor grundfæstede HERREN Kongedømmet i hans Hånd; og hele Juda bragte Josafat Gaver, så han vandt stor Rigdom og Ære.
\par 6 Da voksede hans Mod til at vandre på HERRENs Veje, så han også udryddede Offerhøjene og Asjerastøtterne i Juda.
\par 7 I sit tredje Regeringsår sendte han sine Øverster Benhajil, Obadja, Zekarja, Netan'el og Mika ud for at undevise i Judas Byer,
\par 8 ledsaget af Leviterne Sjemaja, Netanja, Zebadja, Asa'el, Sjemiramot, Jonatan, Adonija, Tobija og Tob-Adonija, Leviterne, og præsterne Elisjama og Joram.
\par 9 De havde HERRENs Lovbog med sig og undeviste i Juda, og de drog rundt i alle Judas Byer og underviste Folket.
\par 10 En HERRENs Rædsel kom over alle Lande og Riger rundt om Juda, så de ikke indlod sig i Krig med Josafat.
\par 11 Fra Filisterne kom der Folk, som bragte Josafat Gaver og svarede Sølv i Skat; også Araberne bragte ham Småkvæg, 7700 Vædre og 7700 Bukke.
\par 12 Således gik det stadig fremad for Josafat, så han til sidst fik meget stor Magt; og han byggede Borge og Forrådsbyer i Juda,
\par 13 og han havde store Forråd i Judas Byer og Krigsfolk, dygtige Krigere i Jerusalem.
\par 14 Her følger en Fortegnelse over dem efter deres Fædrenehuse. Til Juda hørte følgende Tusindførere: Øversten Adna med 300.000 dygtige Krigere;
\par 15 ved Siden af ham Øversten Johanan med 280.000 Mand;
\par 16 ved Siden af ham Amasja, Zikris Søn, der frivilligt gav sig i HERRENs Tjeneste, med 200.000 dygtige Krigere;
\par 17 fra Benjamin var Eljada, en dygtig Koger, med 200.000 Mand, væbnet med Bue og Skjold;
\par 18 ved Siden af ham Jozabad med 80.000 vel rustede Mand.
\par 19 Disse stod i Kongens Tjeneste, og dertil kom de, som Kongen havde lagt i de befæstede Byer hele Juda over.

\chapter{18}

\par 1 Da Josafat havde vundet stor Rigdom og Hæder, besvogrede han sig med Akab.
\par 2 Efter nogle Års Forløb drog han ned til Akab i Samaria, og Akab lod slagte en Mængde Småkvæg og Hornkvæg til ham og Folkene, han havde med sig; og så overtalte han ham til at drage op mod Ramot i Gilead.
\par 3 Kong Akab af Israel sagde nemlig til Kong Josafat af Juda: "Vil du drage med mig mod Ramot i Gilead?" Han svarede ham: "Jeg som du, mit Folk som dit, jeg går med i Krigen."
\par 4 Josafat sagde fremdeles til Israels Konge: "Spørg dog først om, hvad HERREN siger!"
\par 5 Da lod Israels Konge Profeterne kalde sammen, 400 Mand, og spurgte dem: "Skal jeg drage i Krig mod Ramot i Gilead, eller skal jeg lade være?" De svarede: "Drag derop, så vil Gud give det i Kongens Hånd!"
\par 6 Men Josafat spurgte: "Er her ikke endnu en af HERRENs Profeter, vi kan spørge?"
\par 7 Israels Konge svarede: "Her er endnu en Mand, ved hvem vi kan rådspørge HERREN; men jeg hader ham, fordi han aldrig spår mig godt, men altid ondt; det er Mika, Jimlas Søn." Men Josafat sagde: "Således må Kongen ikke tale!"
\par 8 Da kaldte Israels Konge på en Hofmand og sagde: "Hent hurtigt Mika, Jimlas Søn!"
\par 9 Imidlertid sad Israels Konge og Kong Josafat af Juda, iført deres Skrud, hver på sin Trone i Samarias Portåbning, og alle Profeterne spåede foran dem.
\par 10 Da lavede Zidkija, Kena'anas Søn, sig Horn af Jern og sagde: "Så siger HERREN: Med sådanne skal du støde Aramæerne ned, til de er tilintetgjot!"
\par 11 Og alle Profeterne spåede det samme og sagde: "Drag op mod Ramot i Gilead, så skal Lykken følge dig, og HERREN vil give det i Kongens Hånd!"
\par 12 Men Budet der var gået efter Mika, sagde til ham: "Se, Profeterne har alle som een givet Kongen gunstigt Svar. Tal du nu som de og giv gunstigt Svar!"
\par 13 Men Mika svarede: "Så sandt HERREN lever: Hvad min Gud siger, det vil jeg tale!"
\par 14 Da han kom til Kongen, spurgte denne ham: "Mika, skal vi drage i Krig mod Ramot i Gilead, eller skal vi lade være?" Da svarede han: "Drag derop, så skal Lykken følge eder, og de skal gives i eders Hånd!"
\par 15 Men Kongen sagde til ham: "Hvor mange Gange skal jeg besværge dig, at du ikke siger mig andet end Sandheden i HERRENs Navn?"
\par 16 Da sagde han: "Jeg så hele Israel spredt på Bjergene som en Hjord uden Hyrde; og HERREN sagde: De Folk har ingen Herre, lad dem vende tilbage i Fred, hver til sit!"
\par 17 Israels Konge sagde da til Josafat: "Sagde jeg dig ikke, at han aldrig spår mig godt, men kun ondt!"
\par 18 Da sagde Mika: "Så hør da HERRENs Ord! Jeg så HERREN sidde på sin Trone og hele Himmelens Hær stå til højre og venstre for ham;
\par 19 og HERREN sagde: Hvem vil dåre Kong Akab af Israel, så han drager op og falder ved Ramot i Gilead? En sagde nu et, en anden et andet;
\par 20 men så trådte en Ånd frem og stillede sig foran HERREN og Sagde: Jeg vil dåre ham! HERREN spurgte ham: Hvorledes?
\par 21 Han svarede: Jeg vil gå hen og blive en Løgnens Ånd i alle hans Profeters Mund! Da sagde HERREN: Ja, du kan dåre ham; gå hen og gør det!
\par 22 Se, således har HERREN lagt en Løgnens Ånd i disse dine Profeters Mund, thi HERREN har ondt i Sinde imod dig!"
\par 23 Da trådte Zidkija, Kena'anas Søn, frem og slog Mika på Kinden og sagde: "Ad hvilken Vej skulde HERRENs Ånd have forladt mig for at tale til dig?"
\par 24 Men Mika sagde: "Det skal du få at see, den Dag du flygter fra Kammer til Kammer for at skjule dig!"
\par 25 Så sagde Israels Konge: "Tag Mika og bring ham til Amon, Byens Øverste, og Kongesønnen Joasj
\par 26 og sig: Således siger Kongen: Kast denne Mand i Fængsel og sæt ham på Trængselsbrød og Trængselsvand, indtil jeg kommer uskadt tilbage!"
\par 27 Men Mika sagde: "Kommer du uskadt tilbage, så har HERREN ikke talet ved mig!" Og han sagde: "Hør, alle I Folkeslag"l!"
\par 28 Så drog Israels Konge og Kong Josafat af Juda op mod Ramot i Gilead.
\par 29 Og Israels Konge sagde til Josafat: "Jeg vil forklæde mig, før jeg drager i Kampen; men tag du dine egne Klæder på!" Og Israels Konge forklædte sig, og så drog de i Kampen.
\par 30 Men Arams Konge havde ffivet sine Vognstyrere den Befaling: "I må ikke angribe nogen, være sig høj eller lav, uden Israels Konge alene."
\par 31 Da nu Vognstyrerne fik Øje på Josafat, tænkte de: "Det er sikkert Israels Konge!" Og de rettede deres Angreb mod ham fra alle Sider. Da gav Josafat sig til at, råbe, og HERREN frelste ham, idet Gud lokkede dem bort fra ham;
\par 32 og da Vognstyrerne opdagede, at det ikke var Israels Konge, trak de sig bort fra ham.
\par 33 Men en Mand, der skød en Pil af på Lykke og Fromme, ramte Israels Konge mellem Remmene og Brynjen. Da sagde han til sin Vognstyrer: "Vend og før mig ud af Slaget, thi jeg er såret!"
\par 34 Men kampen blev hårdere og hårdere den Dag, og Israels Konge holdt sig oprejst i sin Vogn over for Aramæerne til Aften; men da Solen gik ned, døde han.

\chapter{19}

\par 1 Kong Josafat af Juda derimod vendte uskadt hjem igen til sit Palads i Jerusalem.
\par 2 Da trådte Seeren Jehu, Hananis Søn, frem for ham, og han sagde til Kong Josafat: "Skal man hjælpe de ugudelige? Elsker du dem, der hader HERREN? For den Sags Skyld er HERRENs Vrede over dig!
\par 3 Noget godt er der dog hos dig, thi du har udryddet Asjererne af Landet, og du har vendt dit Hjerte til at søge Gud."
\par 4 Josafat blev nu i Jerusalem, Så drog han atter ud blandt Folket fra Be'ersjeba til Efraims Bjerge og førte dem tilbage til HERREN, deres Fædres Gud.
\par 5 Og han indsatte Dommere i Landet i hver enkelt af de befæsfede Byer i Juda.
\par 6 Og han sagde til Dommerne: "Se til, hvad I gør, thi det er ikke for Mennesker, men for HERREN, I fælder Dom, og han er hos eder, når I afsiger Hendelser.
\par 7 Derfor være HERRENs Frygt over eder! Vær varsomme med, hvad I foretager eder, thi hos HERREN vor Gud er der hverken Uret eller Personsanseelse, ej heller tager han imod Bestikkelse!"
\par 8 Også i Jerusalem indsatte Josafat nogle af Leviterne, Præsterne og Overhovederne for Israels Fædrenehuse til at dømme i HERRENs Sager og i Stridigheder mellem Jerusalems Indbyggere.
\par 9 Og han bød dem: "Således skal I gøre i HERRENs Frygt, i Sanddruhed og med redeligt Hjerte:
\par 10 Hver Gang der forelægges eder en Retssag af eders Brødre, der bor i deres Byer, hvad enten det drejer sig om Blodskyld eller om Love, Anordninger og Lovbud, skal I hjælpe dem til Rette, at de ikke skal pådrage sig Skyld forHERREN, så der kommer Vrede over eder og eders Brødre; gør således for ikke at pådrage eder Skyld.
\par 11 I alle HERRENs Sager skal Ypperstepræsten Amarja være eders foresatte, i alle Kongens Sager Zebadja, Jisjmaels Søn, Fyrsten i Judas Hus; og Leviterne står eder til Tjeneste som Retsskrivere. Gå nu frimodigt til Værket, HERREN vil være med enhver, der gør sin Pligt."

\chapter{20}

\par 1 Siden hændte det sig, at Moabiterne og ammoniterne sammen med Folk fra Maon drog i Krig mod Josafat.
\par 2 Og man kom og bragte Josafat den Efterretning: "En vældig Menneskemængde rykker frem imod dig fra Egnene binsides Havet), fra Edom, og de står allerede i Hazazon-Tamar (det er En-Gedi)!"
\par 3 Da grebes Josafat af Frygt, og han vendte sig til HERREN og søgte ham og lod en Faste udråbe i bele Juda.
\par 4 Så samledes Judæerne for at søge Hjælp hos HERREN; også fra alle Judas Byer kom de for at søge HERREN.
\par 5 Men Josafat trådte frem i Judas og Jerusalems Forsamling i HERRENs Hus foran den nye Forgård
\par 6 og sagde: "HERRE, vore Fædres Gud! Er du ikke Gud i Himmelen, er det ikke dig, der hersker over alle Hedningerigerne? I din Hånd er Kraft og Styrke, og mod dig kan ingen holde Stand!
\par 7 Var det ikke dig, vor Gud, der drev dette Lands Indbyggere bort foran dit Folk Israel og gav din Ven Abrahams Efterkommere det for evigt?
\par 8 Og de bosatte sig der og byggede dig der en Helligdom for dit Navn, idet de sagde:
\par 9 Hvis Ulykke rammer os, Sværd, Straffedom, Pest eller Hungersnød, vil vi træde frem foran dette Hus og for dit Åsyn, thi dit Navn bor i dette Hus, og råbe til dig om Hjælp i vor Nød, og du vil høre det og frelse os!
\par 10 Se nu, hvorledes Ammoniterne og Moabiterne og de fra Se'irs Bjerge, hvem du ikke tillod Israeliterne at angribe, da de kom fra Ægypten, tværtimod holdt de sig tilbage fra dem og tilintetgjorde dem ikke,
\par 11 se nu, hvorledes de gengælder os det med at komme for at drive os bort fra din Ejendom, som du gav os i Eje!
\par 12 Vor Gud, vil du ikke holde Dom over dem? Thi vi er afmægtige over for denne vældige Menneskemængde, som kommer over os; vi ved ikke, hvad vi skal gøre, men vore Øjne er vendt til dig!"
\par 13 Medens nu alle Judæerne stod for HERRENs Åsyn med deres Familier, Kvinder og Børn,
\par 14 kom HERRENs Ånd midt i Forsamlingen over Leviten Jahaziel, en Søn af Zekarja, en Søn af Benaja, en Søn af Je'iel, en Søn af Mattanja, af Asafs Sønner,
\par 15 og han sagde: "Lyt til, alle I Judæere, Jerusalems Indbyggere og Kong Josafat! Så siger HERREN til eder: Frygt ikke og forfærdes ikke for denne vældige Menneskemængde, thi Kampen er ikke eders, men Guds!
\par 16 Drag i Morgen ned imod dem; se, de er ved at stige op ad Vejen ved Hazziz, og l vil træffe dem ved Enden af Dalen østen for Jeruels Ørken.
\par 17 Det er ikke eder, der skal kæmpe her; stil eder op og bliv stående, så skal I se, hvorledes HERREN frelser eder, I Judæere og Jerusalems Indbyggere! Frygt ikke og forfærdes ikke, men drag i Morgen imod dem, og HERREN vil være med eder!"
\par 18 Da bøjede Josafat sig med Ansigtet til Jorden, og alle Judæerne og Jerusalems Indbyggere faldt ned for HERREN og tilbad ham;
\par 19 men Leviterne af Keh'atiternes og Koraiternes Sønner stod op for at lovprise HERREN, Israels Gud, med vældig Røst.
\par 20 Tidligt næste Morgen drog de ud til Tekoas Ørken; og medens de drog ud, stod Josafat og sagde: "Hør mig, I Judæere og Jerusalems Indbyggere! Tro på HERREN eders Gud, og l skal blive boende, tro på hans Profeter, og Lykken skal følge eder!"
\par 21 Og efter at have rådført sig med Folket opstillede han Sangere til med Ordene "lov HERREN, thi hans Miskundhed varer evindelig!" at lovprise HERREN i helligt Skrud, medens de drog frem foran de væbnede.
\par 22 Og i samme Stund de begyndte med Jubelråb og Lovsang, lod HERREN et Baghold komme over Ammoniterne, Moabiterne og dem fra Se'irs Bjerge, der rykkede frem mod Juda, så de blev slået.
\par 23 Ammoniterne og Moabiterne angreb dem, der boede i Se'irs Bjerge, og lagde Band på dem og tilintetgjorde dem, og da de var færdige med dem fra Se'ir, gav de sig til at udrydde hverandre.
\par 24 Da så Judæerne kom op på Varden, hvorfra man ser ud over Ørkenen, og vendte Blikket mod Menneskemængden, se, da lå deres døde Kroppe på Jorden, ingen var undsluppet.
\par 25 Så kom Josafat og hans Folk hen for at udplyndre dem, og de fandt en Mængde Kvæg, Gods, Klæder og kostbare Ting. De røvede så meget, at de ikke kunde slæbe det bort, og brugte tre Dage til at plyndre; så meget var der.
\par 26 Den fjerde Dag samledes de i Berakadalen, thi der lovpriste de HERREN, og derfor kaldte man Stedet Berakadalen, som det hedder den Dag i Dag.
\par 27 Derpå vendte alle Folkene fra Juda og Jerusalem med Josafat i Spidsen om og drog tilbage til Jerusalem med Glæde, thi HERREN havde bragt dem Glæde over deres Fjender;
\par 28 og med Harper, Citre og Trompeter kom de lil Jerusalem, til HERRENs Hus.
\par 29 Men en Guds Rædsel kom over alle Lande og Riger, da de hørte, at HERREN havde kæmpet mod Israels Fjender.
\par 30 Således fik Josafats Rige Fred, og hans Gud skaffede ham Ro til alle Sider.
\par 31 Josafat var fem og tredive År gammel, da han blev Konge over Juda, og han herskede fem og tyve År i Jerusalem. Hans Moder hed Azuba og var Datter af Sjilhi.
\par 32 Han vandrede i sin Fader Asas Spor og veg ikke derfra, idet han gjorde, hvad der var ret i HERRENs Øjne.
\par 33 Kun blev Offerhøjene ikke fjernet, og Folket vendte endnu ikke Hjertet til deres Fædres Gud.
\par 34 Hvad der ellers er at fortælle om Josafaf fra først til sidst, står jo optegnet i Jehus, Hananis Søns, Krønike, som er optaget i Bogen om Israels Konger.
\par 35 Senere slog Kong Josafat af Juda sig sammen med Kong Ahazja af Israel, der var ugudelig i al sin Færd;
\par 36 han slog sig sammen med ham om at bygge Skibe, der skulde sejle til Tarsis. De byggede Skibe i Ezjongeber.
\par 37 Men Eliezer, Dodavahus Søn, fra Maresja profeterede mod Josafat og sagde: "Fordi du har slået dig sammen med Ahazja, vil HERREN gøre dit Værk til intet!" Og Skibene gik under og nåede ikke Tarsis.

\chapter{21}

\par 1 Så lagde Josafat sig nu til Hvile hos sine Fædre og blev jordet hos sine Fædre i Davidsbyen; og hans Søn Joram blev Konge i hans Sted.
\par 2 Han havde nogle Brødre, Sønner af Josafat: Azarja, Jehiel, Zekarja, Azarjahu, Mikael og Sjefatja, alle Sønner af Kong Josatat af Juda.
\par 3 Deres Fader havde givet dem store Gaver, Sølv, Guld og Kostbarheder tillige med betæstede Byer i Juda, men Joram havde han givet Kongedømmet, fordi han var den førstefødte.
\par 4 Men da Joram havde overtaget sin Faders Rige og styrket sin Magt, lod han alle sine Brødre dræbe med Sværd tillige med nogle af Israels Øverster.
\par 5 Joram var fo og tredive År gammel, da han blev Konge, og han herskede otte År i Jerusalem.
\par 6 Han vandrede i Israels Kongers Spor ligesom Akabs Hus, thi han havde en Datter af Akab til Hustru, og han gjorde, hvad der var ondt i HERRENs Øjne.
\par 7 Dog vilde HERREN ikke tilintetgøre Davids Slægt for den Pagts Skyld, han havde sluttet med David, og efter det Løfte, han havde givet, at han altid skulde have en Lampe for hans Åsyn.
\par 8 I hans Dage rev Edomiterne sig løs fra Judas Overherredømme og valgte sig en Konge.
\par 9 Da drog Joram over til Za'ir med alle sine Stridsvogne. Og han stod op om Natten, og sammen med Vognstyrerne slog han sig igennem Edoms Rækker, der havde omringet ham.
\par 10 Således rev Edom sig løs fra Judas Overherredømme, og således er det den Dag i Dag. På samme Tid rev også Libna sig løs fra hans Overherredømme, fordi han forlod HERREN, sine Fædres Gud.
\par 11 Også han rejste Offerhøje i Judas Byer og forledte Jerusalems Indbyggere til at bole og Juda til at falde fra.
\par 12 Da kom der et Brev fra Profeten Elias til ham, og deri stod: "Så siger HERREN, din Fader Davids Gud: Fordi du ikke har vandret i din Fader Josafats og Kong Asa af Judas Spor,
\par 13 men i Israels Kongers Spor og forledt Juda og Jerusalems Indbyggere til at bole, ligesom Akabs Hus gjorde, og tilmed dræbt dine Brødre, din Faders Hus, der var bedre end du selv,
\par 14 se, derfor vil HERREN nu ramme dit Folk, dine Sønner, dine Hustruer og alt, hvad du ejer, med et tungt Slag.
\par 15 Og selv skal du falde i en hård Sygdom og blive syg i dine Indvolde, så at Indvoldene nogen Tid efter skal træde ud som Følge af Sygdommen!"
\par 16 Så æggede HERREN Filisterne og Araberne, der var Naboer til Kusjiterne, imod Joram,
\par 17 og de drog opmod Juda trængte ind og røvede alle Kongens Ejendele, som fandtes i hans Palads, også hans Sønner og Hustruer, så der ikke levnedes ham nogen Søn undtagen Joahaz, den yngste af hans Sønner.
\par 18 Og efter alt dette slog HERREN ham med en uhelbredelig Sygdom i hans Indvolde.
\par 19 Nogen Tid efter, da to År var gået, trådte Indvoldene ud som Følge af Sygdommen, og han døde under hårde Lidelser. Hans Folk tændte intet Bål til Ære for ham som for hans Fædre.
\par 20 Han var to og tredive År gammel, da han blev Konge, og han herskede otte År i Jerusalem. Han gik bort uden at savnes. Man jordede ham i Davidsbyen, dog ikke i Kongegravene.

\chapter{22}

\par 1 Derpå gjorde Jerusalems Indbyggere hans yngste Søn Ahazja til Konge i hans Sted, thi de ældre var dræbt af Røverskaren, der brød ind i Lejren sammen med Araberne. Således blev Ahazja, Jorams Søn, Konge i Juda.
\par 2 Ahazja var to og fyrretyve År gammel, da han blev Konge, og han herskede eet År i Jerusalem. Hans Moder hed Atalja og var Datter af Omri.
\par 3 Også han vandrede i Akabs Hus's Spor, thi hans Moder forledte ham til Ugudelighed ved sine Råd.
\par 4 Han gjorde, hvad der var ondt i HERRENs Øjne, ligesom Akabs Hus, thi efter Faderens Død blev de hans Rådgivere, og det blev hans Fordærv.
\par 5 Det var også efter deres Råd, han sammen med Akabs Søn, Kong Joram af Israel, drog i Krig mod Kong Hazael af Aram ved Ramot i Gilead. Men Aramæerne sårede Joram.
\par 6 Så vendte Joram tilbage for i Jizre'el at søge Helbredelse for de Sår, man havde tilføjet ham ved Rama, da han kæmpede med Kong Hazael af Aram; og Jorams Søn, Kong Ahazja af Juda, drog ned for at se til Joram, Akabs Søn, i Jizre'el, fordi han lå syg.
\par 7 Men til Ahazjas Undergang var det tilskikket af Gud, at han skulde komme til Joram; thi da han var kommet derhen, gik han med Joram ud til Jehu, Nimsjis Søn, som HERREN havde salvet til at udrydde Akabs Hus.
\par 8 Og da Jehu fuldbyrdede Dommen over Akabs Hus, traf han på Judas Øverster og Ahazjas Brodersønner, der var i Ahazjas Tjeneste, og dræbte dem;
\par 9 derpå lod han Ahazja eftersøge, og man fangede ham, medens han holdt sig skjult i Samaria, ogbragte ham til Jehu, der lod ham dræbe. Så jordede de ham, thi de sagde: "Han var dog en Søn af Josafat, der søgte HERREN af hele sit Hjerte." Men af Ahazjas Hus var ingen stærk nok til at tage Magten.
\par 10 Da Atalja, Ahazjas Moder, fik at vide, at hendes Søn var død, tog hun sig for at udrydde hele den kongelige Slægt af Judas Hus.
\par 11 Men Kongens Datter Josjab'at tog Ahazjas Søn Joas og fik ham hemmeligt af Vejen, så han ikke var imellem Kongesønnerne, der blev dræbt, og hun gemte ham og hans Amme i Sengekammeret. Således holdt Josjab'at, Kong Jorams Datter, Præsten Jojadas Hustru, der jo var Søster til Ahazja, ham skjult for Atalja, så hun ikke fik ham dræbt;
\par 12 og han var i seks År skjult hos dem i HERRENs Hus, medens Atalja herskede i Landet.

\chapter{23}

\par 1 Men i det syvende År tog Jojada Mod til sig og indgik Pagt med Hundredførerne Azarja, Jerhams Søn, Jisjmael, Johanans Søn, Azarja, Obeds Søn, Ma'aseja, Adajas Søn, og Elisjafat, Zikris Søn.
\par 2 De drog Juda rundt og samlede Leviterne fra alle Judas Byer og Overhovederne for Israels Fædrenehuse, og de kom så til Jerusalem.
\par 3 Så sluttede hele Forsamlingen i Guds Hus en Pagt med Kongen. Og Jojada sagde til dem: "Se, Kongesønnen skal være Konge efter det Løfte, HERREN har givet om Davids Sønner!
\par 4 Og således skal I gøre: Den Tredjedel af eder Præster og Leviter, der rykker ind om Sabbaten, skal tjene som Dørvogtere;
\par 5 den anden Tredjedel skal besætte Kongens Palads og den tredje Jesodporten, medens alt Folket skal besætte Forgården etiI HERRENs Hus.
\par 6 Men ingen må betræde HERRENs Hus undtagen Præsterne og de Leviter, der gør Tjeneste; de må gå derind, thi de er hellige; men hele Folket skal holde sig HERRENs Forskrift efterrettelig.
\par 7 Så skal Leviterne, alle med Våben i Hånd, slutte Kreds om Kongen, og enhver, der nærmer sig Templet, skal dræbes. Således skal I være om Kongen, når han går ind, og når han går ud."
\par 8 Leviterne og alle Judæerne gjorde alt, hvad Præsten Jojada havde påbudt, idet de tog hver sine Folk, både dem, der rykkede ud, og dem, der rykkede ind om Sabbaten, thi Præsten Jojada gav ikke Skifterne Orlov.
\par 9 Og Præsten Jojada gav Hundredførerne Spydene og de små og store Skjolde, som havde tilhørt Kong David og var i Guds Hus.
\par 10 Derpå opstillede han alt Folket, alle med Spyd i Hånd, fra Templets Sydside til Nordsiden, hen til Alteret og derfra igen hen til Templet, rundt om Kongen.
\par 11 Så førte de Kongesønnen ud, satte Kronen og Vidnesbyrdet på ham; derefter udråbte de ham til Konge, og Jojada og hans Sønner salvede ham og råbte: "Kongen leve!"
\par 12 Da Atalja hørte Larmen af Folket, som løb og jublede for Kongen, gik hun hen til Folket i HERRENs Hus,
\par 13 og der så hun Kongen stå på sin Plads ved Indgangen og Øversterne og Trompetblæserne ved Siden af, medens alt Folket fra Landet jublede og blæste i Trompeterne, og Sangerne med deres Instrumenter ledede Lovsangen. Da sønderrev Atalja sine Klæder og råbte: "Forræderi, Forræderi!"
\par 14 Men Præsten Jojada bød Hundredførerne, Hærens Befalingsmænd: "Før hende uden for Forgårdene og hug enhver ned, der følger hende, thi - sagde Præsten - I må ikke dræbe hende i HERRENs Hus!"
\par 15 Så greb de hende, og da hun ad Hesteporten var kommet til Kongens Palads, dræbte de hende der.
\par 16 Men Jojada sluttede Pagt mellem sig og hele Folket og Kongen om, at de skulde være HERRENS folk.
\par 17 Og alt Folket begav sig til Ba'als Hus og nedbrød det; Altrene og Billederne huggede de i Stykker, og Ba'als Præst Mattan dræbte deforan Altrene.
\par 18 Derpå satte Jojada Vagtposter ved HERRENs Hus under Ledelse af Præsterne og Leviterne, som David havde tildelt HERRENs Hus, for at de skulde bringe HERREN Brændofre, som det er foreskrevet i Mose Lov, under Jubel og Sang efter Davids Anordning;
\par 19 og han opstillede Dørvogterne ved HERRENs Hus's Porte, for at ingen, der i nogen Måde var uren, skulde gå derind;
\par 20 og han tog Hundredførerne og Stormændene og Folkets overordnede og alt Folket fra Landet og førte Kongen ned fra HERRENs Hus; de gik igennem Øvreporten til Kongens Palads og satte Kongen på Kongetronen.
\par 21 Da glædede alt Folket fra Landet sig, og Byen holdt sig rolig. Men Atalja huggede de ned.

\chapter{24}

\par 1 Joas var syv År gammel da han blev konge og han herskede i fyrretyve År i Jerusalem. Hans Moder hed Zibja og var fra Be'ersjeba.
\par 2 Joas gjorde, hvad der var ret i HERRENs Øjne, så længe Præsten Jojada levede.
\par 3 Jojada tog ham to Hustruer, og han avlede Sønner og Døtre.
\par 4 Siden kom Joas i Tanker om at udbedre HERRENs Hus.
\par 5 Han samlede derfor Præsterne og Leviterne og sagde til dem: "Drag ud til Judas Byer og saml Penge ind i hele Israel til at istandsætte eders Guds Hus År efter År; men I må skynde eder!" Leviterne skyndte sig dog ikke.
\par 6 Så lod Kongen Ypperstepræsten Jojada kalde og sagde til ham: "Hvorfor har du ikke holdt Øje med, at Leviterne i Juda og Jerusalem indsamler den Afgift, HERRENs Tjener Moses pålagde Israels Forsamling til Vidnesbyrdets Telt?
\par 7 Thi den ugudelige Atalja og hendes Sønner har ødelagt Guds Hus og oven i Købet brugt Helliggaverne i HERRENs Hus til Ba'alerne."
\par 8 På Kongens Bud lavede man så en Kiste og satte den uden for Porten til HERRENs Hus;
\par 9 og det kundgjordes i Juda ogJerusalem, at den Afgift, Moses havde pålagt Israeliterne i Ørkenen, skulde udredes til HERREN.
\par 10 Alle Øversterne og hele Folket bragte da Afgiften med Glæde og lagde den i Kisten, til den var fuld;
\par 11 og hver Gang kisten af Leviterne blev bragt til de kongelige Tilsynsmænd, når de så, at der var mange Penge, kom Kongens Skriver og Ypperstepræstens Tilsynsmand og tømte Kisten; og så tog de og satte den på Plads igen. Således gjorde de Gang på Gang, og de samlede en Mængde Penge.
\par 12 Og Kongen og Jojada gav Pengene til dem, der stod for Arbejdet ved HERRENs Hus, og de lejede Stenhuggere og Tømmermænd til at udbedre HERRENs Hus og derhos Jern og Kobbersmede til at istandsætte HERRENs Hus.
\par 13 Og de, der stod for Arbejdet, tog fat, og Istandsættelsesarbejdet skred frem under deres Hænder, og de bygggede Guds Hus op efter de opgivne Mål og satte det i god Stand.
\par 14 Da de var færdige, bragte de Resten af Pengene til Kongen og Jojada, og for dem lod han lave Redskaber til HERRENs Hus, Redskaber til Tjenesten og Ofrene, Kander og Kar af Guld og Sølv. Og de ufrede stadig Brændofre i HERRENs Hus, så længe Jojada levede.
\par 15 Men Jojada blev gammel og mæt af Dage og døde; han var ved sin Død 130 År gammel.
\par 16 Man jordede ham i Davidsbyen hos Kongerne, fordi han havde gjort sig fortjent af Israel og over for Gud og hans Hus.
\par 17 Men efter Jojadas Død kom Judas Øverster og kastede sig ned for Kongen. Da hørte Kongen dem villig,
\par 18 Og de forlod HERREN, deres Fædres Guds Hus og dyrkede Asjererne og Gudebillederne. Og der kom Vrede over Juda og Jerusalem for denne deres Synds Skyld.
\par 19 Så sendte han Profeter iblandt dem for at omvende dem til HERREN, og de advarede dem, men de hørte ikke.
\par 20 Men Guds Ånd iførte sig Zekarja, Præsten Jojadas Søn, og han trådte frem for Folket og sagde til dem: "Så siger Gud: Hvorfor overtræder I HERRENs Bud så Lykken viger fra eder? Fordi I har forladt HERREN, har han forladt eder!"
\par 21 Men de sammensvor sig imod ham og stenede ham på Kongens Bud i Forgården til HERRENs Hus;
\par 22 Kong Joas kom ikke den Kærlighed i Hu, hans Fader Jojada havde vist ham, men lod Sønnen dræbe, Men da han døde, råbte han: "HERREN ser og straffer!"
\par 23 Ved Jævndøgnstide drog Aramæernes Hær imod Kongen, og de trængte ind i Juda og Jerusalem og udryddede alle Øverster af Folket og sendte alt Byttet, de tog fra dem. til Kongen af Darmaskus.
\par 24 Skønt Aramæernes Hær kom i et ringe Tal, gav HERREN en såre stor Hær i deres Hånd, fordi de havde forladt HERREN, deres Fædres Gud; således fuldbyrdede de Dommen over Joas.
\par 25 Og da de drog bortfra ham - de forlod ham nemlig i hårde Lidelser - stiftede hans Folk en Sammensærgelse imod ham til Straf for Mordet på Præsten Jojadas Søn og dræbte ham i hans Seng. Således døde han, og man jordede ham i Davidsbyen, dog ikke i Kongegravene.
\par 26 De sammensvorne var Zakar, en Søn af Ammoniterkvinden Sjim'at. og Jozahad, en Søn af Moabiterkvinden Sjimrit.
\par 27 Hans Sønners Navne, de mange profetiske Udsagn imod ham og hans grundige Udbedring af Guds, Hus står jo optegnet i Udlægningen til Kongernes Bog. Hans Søn Amazja blev Konge i hans Sted.

\chapter{25}

\par 1 Amazja var fem og tyve År gammel, da han blev Konge, og han herskede ni og tyve År i Jerusalem. Hans Moder hed Jehoaddan og var fra Jerusalem.
\par 2 Han gjorde, hvad der var ret i HERRENs Øjne, dog ikke med et helt Hjerte.
\par 3 Da han havde sikret sig Magten, lod han dem af sine Folk dræbe, der havde dræbt hans Fader Kongen,
\par 4 men deres Børn lod han ikke ihjelslå, i Henhold til hvad der står skrevet i Moses's Lovbog, hvor HERREN påbyder: "Fædre skal ikke lide Døden for Børns Skyld, og Børn skal ikke lide Døden for Fædres Skyld. Men enhver skal lide Døden for sin egen Synd."
\par 5 Amazja samlede Judæerne og opstillede dem efter Fædrenehuse under Tusind- og Hundredførerne, hele Juda og Benjamin, og da han mnønstrede dem fra Tyveårsalderen og opefter, fandt han, at de udgjorde 300.000 udvalgte Folk, øvede Krigere, der har Skjold og Spyd.
\par 6 Dertil lejede han i Israel 100.000 dygtige Krigere for 100 Sølvtalenter.
\par 7 Men en Guds Mand kom til ham og sagde: "Israels Hær må ikke følge dig, Konge, thi HERREN er ikke med Israel, ikke med nogen af Efraimiterne;
\par 8 og hvis du mener, at du kan vinde Styrke på den Måde, vil Gud bringe dig til Fald for Fjenden, thi hos Gud er der Kraft til at hjælpe og til at bringe til Fald!"
\par 9 Amazja spurgte den Guds Mand: "Hvad så med de 100 Sølvtalenter, jeg gav de israelitiske Krigsfolk?" Og den Guds Mand svarede: "HERREN kan give dig langt mere end det!"
\par 10 Da udskilte Amazja de Krigsfolk, der var kommet til ham fra Efraim, og lod dem drage hjem; men deres Vrede blussede heftigt op mod Juda, og de vendte hjem i fnysende Vrede.
\par 11 Amazja tog nu Mod til sig og førte sine Krigere til Saltdalen og nedhuggede 10.000 af Se'iriterne;
\par 12 desuden tog Judæerne 10.000 levende til Fange; dem førte de op på Klippens Top og styrtede dem ned derfra, så de alle knustes.
\par 13 Men de Krigsfolk, Amazja havde sendt hjem, så de ikke kom til at følge ham i Krigen, faldt ind i Judas Byer fra Samaria til Bet-Horon, huggede 3.000 af Indbyggerne ned og gjorde stort Bytte.
\par 14 Da Amazja kom hjem fra Sejren over Edomiterne, havde han Se'iriternes Guder med, og han opstillede dem som sine Guder, tilbad dem og tændte Offerild for dem.
\par 15 Da blussede HERRENs Vrede op mod Amazja, og han sendte en Profet til ham, og denne sagde til ham: "Hvorfor søger du dette Folks Guder, som ikke kunde frelse deres Folk af din Hånd?"
\par 16 Men da han talte således, sagde Kongen til ham: "Har man gjort dig til Kongens Rådgiver? Hold inde, ellers bliver du slået ihjel!" Da holdt Profeten inde og sagde: "Jeg indser, at Gud har i Sinde at ødelægge dig, siden du bærer dig således ad og ikke hører mit Råd!"
\par 17 Efter at have overtænkt Sagen sendte Kong Amazja af Juda Bud til Jehus Søn Joahaz's Søn, Kong Joas af Israel, og lod sige: "Kom, lad os se hinanden under Øjne!"
\par 18 Men Kong Joas af Israel sendte Kong Amazja, af Juda det Svar: "Tidselen på Libanon sendte engang det Bud til Cederen på Libanon: Giv min Søn din Datter til Ægte! Men Libanons vilde Dyr løb hen over Tidselen og trampede den ned!
\par 19 Du tænker: Se, jeg har slået Edom! Og det har gjort dig overmodig, så du vil vinde mere Ære. Bliv, hvor du er! Hvorfor vil du udfordre Ulykken og udsætte både dig selv og Juda for Fald?"
\par 20 Men Amazja vilde intet høre, thi Gud føjede det såledesfor at give dem til Pris, fordi de søgte Edoms Guder.
\par 21 Så drog Kong Joas af Israel ud, og han og Kong Amazja af Juda så hinanden under Øjne ved Bet-Sjemesj i Juda;
\par 22 Juda blev slået af Israel, og de flygtede hver til sit.
\par 23 Men Kong Joas af Israel tog Joahaz's Søn Joas's Søn, Kong Amazja af Juda, tifange ved Bet-Sjemesj og førte ham til Jerusalem. Derpå nedrev han Jerusalems Mur på en Strækning af 400 Alen, fra Efraimsporten til Hjørneporten;
\par 24 og han tog alt det Guld og Sølv og alle de Kar, der fandtes i Guds Hus hos Obed-Edom og i Skatkammeret i Kongens Palads; desuden tog han Gidsler og vendte så tilbage til Samaria.
\par 25 Joas's Søn, Kong Amazja af Juda, levede endnu femten År, efter at Joahaz's Søn, Kong Joas at Israel, var død.
\par 26 Hvad der ellers er at fortælle om Amazja fra først til sidst, står optegnet i Bogen om Judas og Israels Konger.
\par 27 Men ved den Tid Amazja faldt fra HERREN, stiftedes der en Sammensværgelse mod ham i Jerusalem, og han flygtede til Lakisj, men der blev sendt Folk efter ham til Lakisj, og de dræbte ham der.
\par 28 Så løftede man ham op på Heste og jordede ham hos hans Fædre i Davidsbyen.

\chapter{26}

\par 1 Alt Folket i Juda tog så Uzzija, der da var seksten År gammel, og gjorde ham til Konge i hans Fader Amazjas Sted.
\par 2 Det var ham, der befæstede Elot og atter forenede det med Juda efter at Kongen havde lagt sig til Hvile hos sine Fædre.
\par 3 Uzzija var seksten År gammel da han blev Konge, og han herskede to og halvtredsindstyve År i Jerusalem. Hans Moder hed Jekolja og var fra Jerusalem.
\par 4 Han gjorde, hvad der var ret i HERRENs Øjne, ganske som hans Fader Amazja;
\par 5 og han blev ved med at søge Gud, så længe Zekarja, der havde undervist ham i Gudsfrygt, levede. Og så længe han søgte HERREN, gav Gud ham Lykke.
\par 6 Han gjorde et Krigstog mod Filisterne og nedrev Bymurene i Gat, Jabne og Asdod og byggede Byer i Asdods Område og Filisterlandet.
\par 7 Gud hjalp ham mod Filisterne og Araberne, der boede i Gur-Ba'al og mod Me'uniterne.
\par 8 Atnmoniterne svarede Uzzija Skat, og hans Ry nåede til Ægypten, thi han blev overmåde mægtig.
\par 9 Og Uzzija byggede Tårne i Jerusalem ved Hjørneporten, Dalporten og Murhjørnet og befæstede
\par 10 Fremdeles byggede han Tårne i Ørkenen og lod mange Cisterner udhugge, thi han havde store Hjorde både i Lavlandet og på Højsletten, der hos Agerdyrkere og Vingårdsmænd på Bjergene og i Frugtlandet, thi han var ivrig Landmand.
\par 11 Og Uzzija havde en øvet Krigshær, der drog i Krig Deling for Deling i det Tal, der fremgik af Mønstringen, som Statsskriveren Je'uel og Retsskriveren Ma'aseja foretog under Tilsyn af Hananja, en at Kongens Øverster.
\par 12 Det fulde Tal på de dygtige Krigere, som var Overhoveder for Fædrenehusene, var 2600.
\par 13 Under deres Befaling stod en Hærstyrke på 307500 krigsvante Folk i deres fulde Kraft til at hjælpe Kongen mod Fjenden.
\par 14 Uzzija udrustede hele Hæren med Skjolde, Spyd, Hjelme, Brynjer, Buer og Slyngesten.
\par 15 Ligeledes lod han i Jerusalem lave snildt udtænkte Krigsmaskiner, der opstilledes på Tårnene og Murtinderne til at udslynge Pile og store Sten. Og hans Ry nåede viden om, thi på underfuld Måde blev han hjulpet til stor Magt.
\par 16 Men da han var blevet mægtig, blev hans Hjerte overmodigt, så han gjorde, hvad fordærveligt var; han handlede troløst mod HERREN sin Gud, idet han gik ind i HERRENs Helligdom for at brænde Røgelse på Røgelsealteret.
\par 17 Præsten Azarja fulgte efter ham, ledsaget af firsindstyve af HERRENs Præster, modige Mænd;
\par 18 og de trådte frem for Kong Uzzija og sagde til ham: "At ofre HERREN Røgelse tilkommer ikke dig, Uzzija, men Arons Sønner, Præsterne, som er helliget til at ofre Røgelse; gå ud af Helligdommen, thi du har handlet troløst, og det tjener dig ikke til Ære for HERREN din Gud!"
\par 19 Uzzija, der holdt Røgelsekarret i Hånden for at brænde Røgelse, blev rasende; men som han rasede mod Præsterne, slog Spedalskhed ud på hans Pande, medens han stod der over for Præsterne foran Røgelsealteret i HERRENs Hus;
\par 20 og da Ypperstepræsten Azarja og alle Præsterne vendte sig imod ham, se, da var han spedalsk i Panden. Så førte de ham hastigt bort derfra, og selv fik han travlt med at komme ud, fordi HERREN havde ramt ham.
\par 21 Så var Kong Uzzija spedalsk til sin Dødedag; og skønt spedalsk fik han Lov at blive boende i sit Hus, men var udelukket fra HERRENs Hus, medens hans Søn Jotam rådede i Kongens Palads og dømte Folket i Landet.
\par 22 Hvad der ellers er at fortælle om Uzzija fra først til sidst, har Profeten Esajas, Amoz's Søn, optegnet.
\par 23 Så lagde han sig til Hvile hos sine Fædre, og man jordede ham hos hans Fædre på den Mark, hor Kongegravene var, under Hensyn til at han havde været spedalsk; og hans Søn Jotam blev Konge i hans Sted.

\chapter{27}

\par 1 Jotam var fem og tyve År gammel, da han blev Konge, og han herskede seksten År i Jerusalem. Hans Moder hed Jerusja og var Datter af Zadok.
\par 2 Han gjorde, hvad der var ret i HERRENs Øjne, ganske som hans Fader Uzzija havde gjort, men han gik ikke ind i HERRENs Helligdom. Men Folket gjorde fremdeles, hvad fordærveligt var.
\par 3 Det var ham, der opførte Øvreporten i HERRENs Hus, og desuden byggede han meget på Ofels Mur.
\par 4 Han opførte Byer i Judas Bjerge og Borge og Tårne i Skovene.
\par 5 Han førte Krig med Ammoniternes Konge og overvandt dem, så Ammoniterne det År måtte svare ham 100 Talenter Sølv, 10.000 Kor Hvede og 10.000 Kor Byg i Skat; og lige så meget svarede Ammoniterne ham det andet og tredje År.
\par 6 Således blev Jotam stærk, fordi han vandrede troligt for HERREN sin Guds Åsyn.
\par 7 Hvad der ellers er at fortælle om Jotam, alle hans Krige og Foretagender, står jo optegnet i Bogen om Israels og Judas Konger.
\par 8 Han var fem og tyve År gammel, da han blev Konge, og han herskede seksten År i Jerusalem.
\par 9 Så lagde Jotam sig til Hvile hos sine Fædre, og man jordede ham i Davidsbyen; oghans Søn Akazblev Konge i hans Sted.

\chapter{28}

\par 1 Akaza var tyve År gammel da han blev Konge, og han herskede seksten År i Jerusalem. Han gjorde ikke, hvad der var ret i HERRENs Øjne, som hans Fader David,
\par 2 men vandrede i Israels Kongers Spor Ja, han lod lave støbte Billeder til Ba'alerne;
\par 3 han tændte selv Offerild i Hinnoms Søns Dal og lod sine Sønner gå igennem Ilden efter de Folks vederstyggelige Skik, som HERREN havde drevet bort foran Israeliterne.
\par 4 Han ofrede og tændte Offerild på Offerhøjene og de høje Steder og under alle grønne Træer.
\par 5 Derfor gav HERREN hans Gud ham i Arams Konges Hånd, og de slog ham og tog mange af hans Folk til Fange og førte dem til Darmaskus. Ligeledes blev han givet i Israels Konges Hånd, og denne tilføjede ham et stort Nederlag.
\par 6 Peka, Remaljas Søn, dræbte i Juda 120.000 Mennesker på een Dag, lutter dygtige Krigsfolk, fordi de havde forladt HERREN, deres Fædres Gud.
\par 7 Den efraimitiske Helt Zikri dræbte Kongesønnen Ma'aseja, Paladsøversten Azrikam og Elkana, den ypperste næst Kongen.
\par 8 Israeliterne bortførte fra deres Brødre 200.000 Hustruer, Sønner og Døtre som Fanger og fratog dem et stort Bytte, som de førte til Samaria.
\par 9 Her boede en HERRENs Profet ved Navn Oded; han gik Hæren i Møde, da den kom til Samaria, og sagde til dem: "Fordi HERREN, eders Fædres Gud, var vred på Juda, gav han dem i eders Hånd; men I har anrettet et Blodbad iblandt dem med et Raseri, der når til Himmelen!
\par 10 Og nu tænker I på at få Magten over Folkene fra Juda og Jerusalem og gøre dem til eders Trælle og Trælkvinder! Har l da ikke også selv nok på Samvittigheden over for HERREN eders Gud!
\par 11 Lyd derfor mig og send de Fanger tilbage, som I har gjort blandt eders Brødre, thi HERRENs glødende Vrede er over eder!"
\par 12 Da trådte nogle af Efraimiternes Overhoveder, Azarja, Johanans Søn, Berekja, Mesjillemots Søn, Hizkija, Sjallums Søn, og Amasa, Hadlajs Søn, frem for de hjemvendte Krigere
\par 13 og sagde til dem: "I må ikke bringe Fangerne herhen! Thi I har i Sinde at øge vote Synder og vor Skyld ud over den Skyld, vi har over for HERREN; thi stor er vor Skyld, og glødende Vrede er over Israel!"
\par 14 Da gav de væbnede Slip på Fangerne og Byttet i Øversternes og hele Forsamlingens Påsyn;
\par 15 og de ovenfor nævnte Mænd stod op og tog sig af Fangerne, gav alle de nøgne iblandt dem Klæder af Byttet, forsynede dem med Klæder og Sko, gav dem Mad og Drikke, salvede dem, lod dem, der ikke kunde gå, ride på Æsler og bragte dem til Jeriko, Palmestaden, hen i Nærheden af deres Brødre; derpå vendte de tilbage til Samaria.
\par 16 På den Tid sendte Kong Akaz Assyrerkongen Bud om Hjælp.
\par 17 Tilmed trængte Edomiterne ind og slog Judæerne og slæbte Krigsfanger bort.
\par 18 Og Filisterne faldt ind i Lavlandets Byer og den judæiske Del af Sydlandet og indtog Bet-Sjemesj, Ajjalon, Gederot, Soko med Småbyer, Timna med Småbyer og Gimzo med Småbyer og bosatte sig der.
\par 19 Thi HERREN ydmygede Juda for Kong Akaz af Judas Skyld, fordi han havde ladet Tøjlesløshed opkomme i Juda og været troløs mod HERREN.
\par 20 Men Assyrerkongen Tillegat-Pilneser drog imod ham og bragte ham i Nød i Stedet for at hjælpe ham;
\par 21 thi Akaz plyndrede HERRENs Hus, og Kongens Palads og Øversternes Palads og gav det til Assyrerkongen, men det hjalp ham intet.
\par 22 Og selv da Assyrerkongen bragte ham i Nød, var Kong Akaz på ny troløs mod HERREN;
\par 23 han ofrede til Darmaskus's Guder, som havde slået ham, idet han sagde: "Aramæerkongernes Guder har jo hjulpet dem; til dem vil jeg ofre, at de også må hjælpe mig!" Men de blev ham og hele Israel til Fald.
\par 24 Også samlede Akaz Karrene i Guds Hus og slog dem i Stykker; og han lukkede HERRENs Hus's Porte og lavede sig Altre ved hvert et Hjørne i Jerusalem;
\par 25 og i hver eneste By i Juda indrettede han Offerhøje for at tænde Offerild for fremmede Guder; således krænkede han HERREN, sine Fædres Gud.
\par 26 Hvad der ellers er at fortælle om ham og hele hans Færd fra først til sidst, står jo optegnet i Bogen om Judas og Israels Konger.
\par 27 Så lagde Akaz sig til Hvile bos sine Fædre, og man jordede ham i Jerusalem, inde i Byen, thi man vilde ikke jorde ham i Israels Kongegrave; og hans Søn Ezekias blev Konge i hans Sted.

\chapter{29}

\par 1 Ezekias var fem og tyve År gammel, da han blev Konge, og han herskede ni og tyve År i Jerusalem. Hans Moder hed Abija og var en Datter af Zekarja.
\par 2 Han gjorde, hvad der var ret i HERRENs Øjne, ganske som hans Fader David.
\par 3 I sit første Regeringsårs første Måned lod han HERRENs Hus's Porte åbne og sætte i Stand.
\par 4 Derpå lod han Præsterne og Leviterne komme, samlede dem på den åbne Plads mod Øst
\par 5 og sagde til dem: "Hør mig, Leviter! Helliger nu eder selv og helliger HERRENs, eders Fædres Guds, Hus og få det urene ud af Helligdommen.
\par 6 Thi vore Fædre var troløse og gjorde, hvad der var ondt i HERREN vor Guds Øjne, de forlod ham, idet de vendte Ansigtet bort fra HERRENs Bolig og vendte den Ryggen;
\par 7 de lukkede endog Forhallens Porte, slukkede Lamperne, brændte ikke Røgelse og bragte ikke Israels Gud Brændofre i Helligdommen.
\par 8 Derfor kom HERRENs Vrede over Juda og Jerusalem, og han gjorde dem til Rædsel, Forfærdelse og Skændsel, som I kan se med egne Øjne.
\par 9 Se, vore Fædre er faldet for Sværdet, vore Sønner, Døtre og Hustruer ført i Fangenskab for den Sags Skyld.
\par 10 Men nu har jeg i Sinde at slutte en Pagt med HERREN, Israels Gud, for at hans glødende Vrede må vende sig fra os.
\par 11 Så lad det nu, mine Sønner, ikke skorte på Iver, thi eder har HERREN udvalgt til at stå for hans Åsyn og tjene ham og til at være hans Tjenere og tænde Offerild for ham!"
\par 12 Da rejste følgende Leviter sig: Mahat, Amasajs Søn, og Joel, Azarjas Søn, af Kehatiternes Sønner; af Merariterne Kisj, Abdis Søn, og Azarja, Jehallel'els Søn; af Gersoniterne Joa, Zimmas Søn, og Eden, Joas Søn;
\par 13 af Elizafans Sønner Sjimri og Je'uel; af Asafs Sønner Zekarja og Mattanja;
\par 14 af Hemans Sønner Jehiel og Sjim'i; og af Jedutuns Sønner Sjemaja og Uzziel;
\par 15 og de samlede deres Brødre, og de helligede sig og skred så efter Kongens Befaling til at rense HERRENs Hus i Henhold til HERRENs Forskrifter.
\par 16 Og Præsterne gik ind i det indre af HERRENS Hus for at rense det, og alt det urene, de fandt i HERRENs Tempel, bragte de ud i HERRENs Hus's Forgård, hvor Leviterne tog imod det for at bringe det ud i Kedrons Dal.
\par 17 På den første Dag i den første Måned begyndte de at hellige, og på den ottende Dag i Måneden var de kommet til HERRENs Forhal; så helligede de HERRENs Hus i otte Dage, og på den sekstende Dag i den første Måned var de færdige.
\par 18 Derpå gik de ind til Kong Ezekias og sagde: "Vi har nu renset hele HERRENs Hus, Brændofferalteret med alt, hvad der hører dertil, og Skuebrødsbordet med alt, hvad der hører dertil;
\par 19 og alle de Kar, som Kong Akaz i sin Troløshed vanhelligede, da han var Konge, dem har vi bragt på Plads og helliget; se, de står nu foran HERRENs Alter!"
\par 20 Næste Morgen tidlig samlede Kong Ezekias Byens Øverster og gik op til HERRENs Hus.
\par 21 Derpå bragte man syv Tyre, syv Vædre, syv Lam og syv Gedebukke til Syndoffer for Riget, Helligdommen og Juda; og han bød Arons Sønner Præsterne ofre dem på HERRENs Alter.
\par 22 De slagtede så Tyrene, og Præsterne tog imod Blodet og sprængte det på Alteret; så slagtede de Vædrene og sprængte Blodet på Alteret; så slagtede de Lammene og sprængte Blodet på Alteret;
\par 23 endelig bragte de Syndofferbukkene frem for Kongen og Forsamlingen, og de lagde Hænderne på dem;
\par 24 så slagtede Præsterne dem og bragte Blodet på Alteret som Syndoffer for at skaffe hele Israel Soning; thi Kongen havde sagt, at Brændofferet og Syndofferet skulde være for hele Israel.
\par 25 Og han opstillede Leviterne ved HERRENs Hus med Cymbler, Harper og Citre efter Davids, Kongens Seer Gads og Profeten Natans Bud, thi Budet var givet af HERREN gennem hans Profeter.
\par 26 Og Leviterne stod med Davids Instrumenter og Præsterne med Trompeterne.
\par 27 Derpå bød Ezekias, at Brændofferet skulde ofres på Alteret, og samtidig med Ofringen begyndte også HERRENs Sang og Trompeterne, ledsaget af Kong David af Israels Instrumenter.
\par 28 Da kastede hele Forsamlingen sig til Jorden, medens Sangen lød og Trompeterne klang, og alt dette varede, til man var færdig med Brændofferet.
\par 29 Så snart man var færdig med Brændofferet, knælede Kongen og alle, der var hos ham, ned og tilbad.
\par 30 Derpå bød Kong Ezekias og Øversterne Leviterne at lovsynge HERREN med Davids og Seeren Asafs Ord; og de sang Lovsangen med Jubel og bøjede sig og tilbad.
\par 31 Ezekias tog da til Orde og sagde: "I har nu indviet eder til HERREN; så træd da frem og bring Slagtofre og Lovprisningsofre til HERRENs Hus!" Så bragte Forsamlingen Slagtofre og Lovprisningsofre, og enhver, hvis Hje1te tilskyndede ham dertil, bragte Brændofre.
\par 32 De Brændofre, Forsamlingen bragte, udgjorde 70 Stykker Hornkvæg, 100 Vædre og 200 Lam, alt som Brændofre til HERREN;
\par 33 og Helligofrene udgjorde 600 Stykker Hornkvæg og 3000 Stykker Småkvæg.
\par 34 Dog var Præsterne for få til at flå Huden af alle Brændofrene, derfor hjalp deres Brødre Leviterne dem, indtil Arbejdet var fuldført og Præsterne havde helliget sig; thi Leviterne viste redeligere Vilje til at hellige sig end Præsterne.
\par 35 Desuden var der en Mængde Brændofre, hvortil kom Fedtstykkerne af Takofrene og Drikofrene til Brændofrene. Således bragtes Tjenesten i HERRENs Hus i Orden.
\par 36 Og Ezekias og alt Folket glædede sig over, hvad Gud havde beredt Folkel, thi det hele var sket så hurtigt.

\chapter{30}

\par 1 Derpå sendte Ezekias Bud til hele Israel og Juda og skrev desuden Breve til Efraim og Manasse om at komme til HERRENs Hus i Jerusalem for at fejre Påsken for HERREN, Israels Gud.
\par 2 Og Kongen, hans Øverster og hele Forsamlingen i Jerusalem rådslog om at fejre Påsken i den anden Måned;
\par 3 thi de kunde ikke fejre den med det samme, da Præsterne ikke havde helliget sig i tilstrækkeligt Tal, og Folket ikke var samlet i Jerusalem.
\par 4 Kongen og hele Forsamlingen fandt det rigtigt;
\par 5 derfor vedtog de at lade et Opråb udgå i hele Israel fra Be'ersjeba til Dan om at komme og fejre Påsken i Jerusalem for HERREN, Israels Gud, thi man havde ikke fejret den så fuldtalligt som foreskrevet.
\par 6 Så gik Ilbudene ud i hele Israel og Juda med Breve fra Kongens og hans Øversters Hånd og sagde efter Kongens Befaling: "Israeliter! Vend tilbage til HERREN, Abrahams, Isaks og Israels Gud, at han må vende sig til den Levnin: af eder, der er undsluppet Assyrerkongernes Hånd.
\par 7 Vær ikke som eders Fædre og Brødre, der var troløse mod HERREN, deres Fædres Gud, hvorfor han gjorde dem til Rædsel, som I selv kan se.
\par 8 Vær nu ikke halsstarrige som eders Fædre, men ræk HERREN Hånden og kom til hans Helligdom, som han har helliget til evig Tid, og tjen HERREN eders Gud, at hans glødende Vrede må vende sig fra eder.
\par 9 Thi når I omvender eder til HERREN, skal eders Brødre og Sønner finde Barmhjertighed hos dem, der førte dem bort, og vende tilbage til dette Land.
\par 10 Og Ilbudene gik fra By til By i Efraims og Manasses Land og lige til Zebulon, men man lo dem ud og hånede dem.
\par 11 Dog var der nogle i Aser, Manasse og Zebulon, der ydmygede sig og kom til Jerusalem;
\par 12 også i Juda virkede Guds Hånd, så at han gav dem et endrægtigt Hjerte til at udføre, hvad Kongen og Øversterne havde påbudt i Kraft af HERRENs Ord.
\par 13 Så samlede der sig i Jerusalem en Mængde Mennesker for at fejre de usyrede Brøds Højtid i den anden Måned, en vældig Forsamling.
\par 14 De gav sig til at fjerne de Altre, der var i Jerusalem; ligeledes fjernede de alle Røgelsekarrene og kastede dem ned i Kedrons Dal.
\par 15 Derpå slagtede de Påskelammet på den fjortende Dag i den anden Måned.
\par 16 og de stillede sig på deres Plads, som deres Pligt var efter den Guds Mand Moses's Lov; Præsterne sprængte Blodet, som de modtog af Leviterne.
\par 17 Thi mange i Forsamlingen havde ikke helliget sig; derfor slagtede Leviterne Påskelammene for alle dem, der ikke var rene, for således at hellige HERREN dem.
\par 18 Thi de flesfe af Folket, især mange fra Efraim, Manasse, Issakar og Zebulon, havde ikke renset sig, men spiste Påskelammet anderledes end foreskrevet. Men Ezekias giki Forbøn for dem og sagde: "HERREN, den gode, tilgive
\par 19 enhver, som har vendt sit Hjerte til at søge Gud HERREN, hans Fædres Gud, selv om han ikke er ren, som Helligdommen kræver det!"
\par 20 Og HERREN bønhørte Ezekias og lod Folket uskadt.
\par 21 Så fejrede de Israeliter, der var til Stede i Jerusalem, de usyrede Brøds Højtid med stor Glæde i syv Dage; og Leviterne og Præsterne sang af alle Kræfter dagligt Lovsange for HERREN.
\par 22 Og Ezekias talte venlige Ord til alle de Leviter, der havde udvist særlig Dygtighed i HERRENs Tjeneste; og de fejrede Højtiden til Ende de syv Dage, idet de ofrede Takofre og lovpriste HERREN, deres Pædres Gud.
\par 23 Men derefter vedtog hele Forsamlingen at holde Højtid syv Dage til, og det gjorde de så med Glæde,
\par 24 thi Kong Ezekias af Juda gav Forsamlingen en Ydelse af 1.000 Tyre og 7.000 Stykker Småkvæg, og Øversterne gav Forsamlingen 1.000 Tyre og 10.000 Stykker Småkvæg; og en Mængde Præster helligede sig.
\par 25 Da frydede hele Judas Forsamling sig, ligeledes Præsterne og Leviterne og hele den Forsamling, der var kommet fra Israel, og de fremmede, der var kommet fra Israels Land eller boede i Juda;
\par 26 og der var stor Glæde i Jerusalem, thi siden Davids Søns, Kong Salomo af Israels, Dage var sligt ikke sket i Jerusalem;
\par 27 og Præsterne og Leviterne stod op og velsignede Folket, og deres Røst hørtes, og deres Bøn nåede Himmelen, hans hellige Bolig.

\chapter{31}

\par 1 Da alt det var til ende, drog alle de Israeliter, som var til stede, ud til Judas Byer, og de sønderbrød Stenstøtterne, omhuggede Asjerastøtterne og nedrev Offerhøjene og Altrene i hele Juda og Benjamin og i Efraim og Manasse, så der ikke blev Spor tilhage; så vendte alle Israeliterne hjem, hver til sin Ejendom i deres Byer.
\par 2 Så ordnede Ezekias Præsternes og Leviternes Skifter, Skifte for Skifte, så at hver enkelt Præst og Levit fik sin særlige Gerning med at ofre Brændofre og Takofre og med af gøre Tjeneste og love og prise i HERRENs Lejrs Porte.
\par 3 Hvad Kongen gav af sit Gods, var til Brændofrene, Morgen- og Aftenbrændofrene og Brændofrene på Sabbaterne, Nymånerne og Højtiderne, som det er foreskrevet i HERRENs Lov.
\par 4 Og han bød Folket, dem, der boede i Jerusalem, at afgive, hvad der tilkom Præsterne og Leviterne, for at de kunde holde fast ved HERRENs Lov.
\par 5 Så snart det Bud kom ud, bragte Israeliterne i rigelig Mængde Førstegrøde af Korn, Most, Olie og Honning og al Markens Afgrøde, og de gav Tiende af alt i rigeligt Mål;
\par 6 også de Israeliter, der boede i Judas Byer, gav Tiende af Hornkvæg og Småkvæg, og de bragte Helliggaverne, der var belliget HERREN deres Gud, og lagde dem Bunke for Bunke.
\par 7 I den tredje Måned begyndte de at ophobe Bunkerne, og i den syvende Måned var de færdige.
\par 8 Ezekias og Øversterne kom så og synede Bunkerne, og de priste HERREN og hans Folk Israel.
\par 9 Da Ezekias spurgfe sig for hos Præsterne og Leviterne om Bunkerne,
\par 10 svarede Ypperstepræsten Azarja af Zadoks Hus: "Siden man begyndte at bringe Offerydelsen til HERRENs Hus, har vi spist os mætte og fået rigelig tilovers, thi HERREN har velsignet sit Folk, så at vi har fået al den Rigdom her tilovers!"
\par 11 Ezekias gav da Befaling til at indrette Kamre i HERRENs Hus; og det gjorde man.
\par 12 Så bragte man samvittighedsfuldt Offerydelsen, Tienden og Helliggaverne derind. Den øverste Opsynsmand derover var Leviten Konanja, den næstøverste hans Broder Sjim'i;
\par 13 og Jehiel, Azarja, Nahat, Asa'el Jerimot, Jazabad, Eliel, Jismakjahu, Mahat og Benaja var Opsynsmænd underKonanja oghans Broder Sjim'i efter den Ordning, som var truffet af Ezekias og Azarja, Øversten i Guds Hus.
\par 14 Leviten Kore, Jimnas Søn, der var Dørvogter på Østsiden, havde Tilsyn med de frivillige Gaver til Gud og skulde uddele HERRENs Offerydelse og de højhellige Gaver;
\par 15 under ham sattes Eden, Minjamin, Jesua, Sjemaja, Amarja og Sjekanja i Præstebyerne til samvittighedsfuldt attorestå Uddelingen til deres Brødre i Skitterne, både store og små,
\par 16 dem af Mandkøn, der var indført i Fortegnelserne fra Treårsalderen og opefter. Undtaget var alle, der kom til HERRENs Hus for efter de enkelte Dages Krav at udføre deres Embedsgerning efter deres Skifter.
\par 17 Præsterne indførtes i Fortegnelserne efter deres Fædrenehuse, Leviterne fra Tyveårsalderen og opefter efter deres Embedspligter, efter deres Skifter,
\par 18 Og de indførtes i Fortegnelserne med hele deres Familie, deres Hustruer, Sønner og Døtre, hele Standen, thi de tog sig samvittighedsfuldt af det hellige.
\par 19 Arons Sønner, Præsterne, som boede på Græsmarkerne omkring deres Byer, havde i hver By nogle navngivne Mænd til at uddele, hvad der tilkom alle af Mandkøn blandt Præsterne og de i Fortegnelserne indførte Leviter.
\par 20 Således gik Ezekias frem i hele Juda, og han gjorde, hvad der var godt, ret og sandt for HERREN hans Guds Åsyn.
\par 21 Alt, hvad han tog fat på vedrørende Tjenesten i Guds Hus eller Loven eller Budet for således at søge sin Gud, det gjorde han af hele sit Hjerte, og det lykkedes for ham.

\chapter{32}

\par 1 Efter disse Tildragelser og disse Vidnesbyrd om Troskab kom Assyrerkongen Sankerib og rykkede ind i Juda og belejrede de befæstede Byer i det Øjemed at bemægtige sig dem.
\par 2 Da Ezekias så, at Sankerib kom, og at han havde i Sinde at angribe Jerusalem,
\par 3 rådførte han sig med sine Hærførere og tapre Folk om at stoppe for Vandet i Kilderne uden for Byen, og de tilsagde ham deres Hjælp.
\par 4 Så samledes en Mængde Folk, og de stoppede for alle Kilderne og for Bækken, som løber midt igennem Landet, idet man sagde: "Hvorfor skulde Assyrerkongerne finde rigeligt Vand, når de kommer?"
\par 5 Derpå tog han Mod til sig og byggede Muren op, overalt hvor den var brudt ned, og byggede Tårne på den, og han byggede den anden Mur udenfor og befæstede Millo i Davidsbyen og lod lave en Mængde Kastevåben og Skjolde.
\par 6 Tillige indsatte han Hærtørere over Krigsfolket, samlede dem om sig på den åbne Plads ved Byporten og talte opmuntrende Ord til dem og sagde:
\par 7 "Vær frimodige og slærke; frygt ikke og forfærdes ikke for Assyrerkongen og hele den Menneskemængde, han har med sig; thi en større er med os end med ham!
\par 8 Med ham er en Arm at Kød, men med os er HERREN vor Gud, der vil hjælpe os og føre vore Krige!" Og Folket satte sin Lid til Kong Ezekias af Judas Ord.
\par 9 Derefter sendte Assyrerkongen Sankerib, der selv med hele sin Krigsmagt lå foran Lakisj, sine. Tjenere til Kong Ezekias af Juda og alle Judæerne i Jerusalem og lod sige:
\par 10 "Således siger Assyrerkongen Sankerib: Hvad er det, I fortrøster eder til, nu I sidder indesluttet i Jerusalem?
\par 11 Mon ikke Ezekias lokker eder til at dø af Hunger og Tørst, når han siger: HERREN vor Gud vil frelse os af Assyrerkongens Hånd!
\par 12 Har ikke samme Ezekias skaffet hans Offerhøje og Altre bort og sagt til Juda og Jerusalem: Kun foran et eneste Alter må I tilbede, og på det skal I tænde Offerild!
\par 13 Ved I ikke, hvad jeg og mine Fædre har gjort ved alle Landenes Folkeslag? Mon Landenes Folks Guder kunde frelse deres Land af min Hånd?
\par 14 Hvilken af alle de Guder, som dyrkedes af disse Folk, på hvilke mine Fædre lagde Band, har kunnet frelse sit Folk af min Hånd? Og så skutlde eders Gud kunne frelse eder af min Hånd!
\par 15 Lad derfor ikke Ezekias vildlede eder og lokke eder på den Måde! Tro ham ikke, thi ikke et eneste Folks eller Riges Gud har kunnet frelse sit Folk af min Hånd eller af mine Fædres Hånd; hvor meget mindre kan da eders Gud frelse eder af min Hånd!"
\par 16 Og hans Tjenere talte endnu flere Ord mod Gud HERREN og mod hans Tjener Ezekias.
\par 17 Han skrev også et Brev for at smæde HERREN, Israels Gud, og tale imod ham; heri stod der: "Så lidt som Landenes Folks Guder har frelst deres Folk af min Hånd, skal Ezekias's Gud frelse sit Folk af min Hånd!"
\par 18 Og de råbte med høj Røst på Judæisk til Folkene i Jerusalem, som stod på Muren, for at indjage dem Angst og Skræk, i Håb om at kunne tage Byen.
\par 19 Og de talte om Jerusalems Gud på samme Måde som om Jordens Folkeslags Guder, der er Værker af Menneskehænder!
\par 20 Derfor bad Kong Ezekias og Profeten Esajas, Amoz's Søn, og råbte til Himmelen.
\par 21 Og HERREN sendte en Engel, der tilintetgjorde alle Krigere, Høvedsmænd og Hærførere i Assyrerkongens Lejr, så han med Spot og Spe måtte vende hjem til sit Land. Og da han gik ind i sin Guds Hus, fældede nogle af hans kødelige Frænder ham der med Sværdet.
\par 22 Således frelste HERREN Ezekias og Jerusalems Indbyggere af Assyrerkongen Sankeribs Hånd og at alle andres og skaffede dem Ro på alle Kanter.
\par 23 Og mange bragte Gaver til Jerusalem til HERREN og kostbare Ting til Kong Ezekias af Juda, så han siden blev højt anset blandt alle Hedningefolk.
\par 24 Ved den Tid blev Ezekias dødssyg. Da bad han til HERREN, og han svarede ham og gav ham et Tegn.
\par 25 Men Ezekias gengældte ikke den Velgerning, der var vist ham; hans Hjerte blev hovmodigt, og derfor kom der Vrede over ham og over Juda og Jerusalem.
\par 26 Men da Ezekias ydmygede sig og vendte om fra sit Hovmod sammen med Jerusalems Indbyggere, kom HERRENs Vrede ikke over dem i Ezekias's Dage.
\par 27 Ezekias var overmåde rig og æret. Han byggede sig Skatkamre til Sølv, Guld, Ædelsten, Høgelsestoffer, Skjolde og alle Hånde kostelige Ting
\par 28 og Forrådskamre til Afgrøden af Horn, Most og Olie, Stalde til alle Slags Kvæg og Folde til Hjordene;
\par 29 Byer byggede han sig også, og han havde Hjorde i Mængde af Hornkvæg og Småkvæg, thi Gud gav ham såre meget Gods.
\par 30 Samme Ezekias tilstoppede Gihons øvre Kilde og ledede Vandet mod Vest nedad til Davidsbyen. Alt, hvad Ezekias tog sig for, lykkedes for ham.
\par 31 Derfor var det også, at Gud gav ham til Pris for Sendebudene, der var sendt til ham fra Babels Fyrster for at høre om det Under, der var sket i Landet; det var for at sætte ham på Prøve og således få Kendskab til alt, hvad der var i hans Hjerte.
\par 32 Hvad der ellers er at fortælle om Ezekias og hans fromme Gerninger, står jo optegnet i Profeten Esajas's, Amoz's Søns, Åbenbaring og i Bogen om Judas og Israels Konger.
\par 33 Så lagde Ezekias sig til Hvile hos sine Fædre, og man jordede ham på Skråningen op til Davids Efterkommeres Grave; og hele Juda og Jerusalems Indbyggere viste ham stor Ære ved hans Død; og hans Søn Manasse blev Konge i hans Sted.

\chapter{33}

\par 1 Manasse var tolv År gammel, da han blev Konge, og han herskede fem og halvtredsindstyve År i Jerusalem.
\par 2 Han gjorde, hvad der var ondt i HERRENs Øjne, og efterlignede de Folkeslags Vederstyggeligheder, som HERREN havde drevet bort foran Israeliterne.
\par 3 Han byggede atter de Offerhøje, som hans Fader Ezekias havde nedrevet, rejste Altre for Ba'alerne, lavede Asjerastøtter og tilbad hele Himmelens Hær og dyrkede dem.
\par 4 Og han byggede Altre i HERRENs Hus, om hvilket HERREN havde sagt: "I Jerusalem skal mit Navn være til evig Tid."
\par 5 Og han byggede Altre for hele Himmelens Hær i begge HERRENs Hus's Forgårde.
\par 6 Han lod sine Sønner gå igennem Ilden i Hinnoms Søns Dal, drev Trolddom og tog Varsler, drev hemmelige Kunster og ansatte Dødemanere og Sandsigere; han gjorde meget, som var ondt i HERRENs Øjne, og krænkede ham.
\par 7 Det Gudebillede, han lod lave, opstillede han i Guds Hus, om hvilket Gud havde sagt til David og hans Søn Salomo: "I dette Hus og i Jerusalem, som jeg har udvalgt af alle Israels Stammer, vil jeg stedfæste mit Navn til evig Tid;
\par 8 og jeg vil ikke mere fjerne Israels Fod fra det Land, jeg gav deres Fædre, dog kun på det Vilkår, at de omhyggeligt overholder alt, hvad jeg har pålagt dem, hele Loven, Anordningerne og Lovbudene, som de fik ved Moses,"
\par 9 Men Manasse forførte Juda og Jerusalems Indbyggere til at handle værre end de Folkeslag, HERREN havde udryddet for Israeliterne.
\par 10 Da talede HERREN Manasse og hans Folk til, men de ænsede det ikke.
\par 11 Så førte HERREN Assyrerkongens Hærførere mod dem, og de fangede Manasse med Kroge, lagde ham i Kobberlænker og førte ham til Babel.
\par 12 Men da han var i Nød, bad han HERREN sin Gud om Nåde og ydmygede sig dybt for sine Fædres Gud.
\par 13 Og da han bad til ham, bønhørte han ham; han hørte hans Bøn og bragte ham tilbage til Jerusalem til hans Kongedømme; da indså Manasse, at HERREN er Gud.
\par 14 Senere byggede han en ydre Mur ved Davidsbyen vesten for Gihon i Dalen og hen imod Fiskeporten, så at den omsluttede Ofel; og han byggede den meget høj. I alle de befæstede Byer i Juda ansatte han Hærførere.
\par 15 Han fjernede de fremmede Guder og Gudebilledet fra HERRENs Hus og alle de Altre, han havde bygget på Tempelbjerget og i Jerusalem, og kastede dem uden for Byen.
\par 16 Og han istandsatte HERRENs Alter og ofrede Tak- og Lovprisningsofre derpå; og han bød Juda at dyrke HERREN, Israels Gud.
\par 17 Men Folket vedblev at ofre på Offerhøjene, dog kun til HERREN deres Gutd.
\par 18 Hvad der ellers er at fortælle om Manasse, hans Bøn til sin Gud og de Seeres Ord, som talte til ham i HERRENs, Israels Guds, Navn, står jo optegnet i Israels Kongers Krønike;
\par 19 hans Bøn og Bønhørelse, al hans Synd og Troløshed og de Steder, hvor han opførte Offerhøje og opstillede Asjerastøtter og Gudebilleder, før han ydmygede sig, står jo optegnet i Seernes Krønike.
\par 20 Så lagde Manasse sig til Hvile hos sine Fædre, og man jordede ham i Haven ved hans Hus; og hans Søn Amon blev Konge i hans Sted.
\par 21 Amon var to og tyve År gammel, da han blev Konge, og han hersltede to År i Jerusalem.
\par 22 Han gjorde, hvad der var ondt i HERRENs Øjne, ligesom hans Fader Manasse, og Amon ofrede til alle de Gudebilleder, hans Fader Manasse havde ladet lave, og dyrkede dem.
\par 23 Han ydmygede sig ikke for HERRENs Åsyn, som hans Fader Manasse havde gjort, men Amon dyngede Skyld på Skyld.
\par 24 Hans Tjenere sammensvor sig imod ham og dræbte ham i hans Hus;
\par 25 men Folket fra Landet dræbte alle dem, der havde sammensvoret sig imod Kong Amon, og gjorde hans Søn Josias til Konge i hans Sted.

\chapter{34}

\par 1 Josias var otte År gammel, da han blev Konge, og han herskede en og tredive År i Jerusalem.
\par 2 Han gjorde, hvad der var ret i HERRENs Øjne, og vandrede i sin lader Davids Spor uden al vige til højre eller venstre.
\par 3 I sit ottende Regeringsår, endnu ganske ung, begyndte han at søge sin Fader Davids Gud, og i det tolvte År begyndte han at rense Juda og Jerusalem for Offerhøjene, Asjerastøtterne og de udskårne og støbte Billeder.
\par 4 I hans Påsyn nedrev man Ba'alernes Altre; Solstøtterne, der stod oven på dem, huggede han om, og Asjerastøtterne og de udskårne og støbte Billeder lod han sønderhugge og knuse og strø ud på deres Grave, som havde ofret til dem;
\par 5 Benene af Præsterne lod han brænde på deres Altre. Således rensede han Juda og Jerusalem.
\par 6 Men også i Byerne i Manasse, Efraim og Simeon og lige til Naftali, rundt om i deres Ruinhobe,
\par 7 lod han Altrene nedbryde, Asjerastøtterne og Gudebillederne sønderslå og knuse og alle Solstøtterne omhugge i hele Israels Land; så vendte han tilbage til Jerusalem.
\par 8 I sit attende Regeringsår gav han sig til at rense Landet og Templet; han sendte Sjafan, Azaljas Søn, Byens Øverste Ma'aseja og Kansleren Joa, Joahaz's Søn, hen for at istandsætte HERREN hans Guds Hus.
\par 9 Da de kom til Ypperstepræsten Hilkija, afleverede de Pengene, der var kommet ind til Guds Hus, dem, som Leviterne, der holdt Vagt ved Tærskelen, havde samlet hos Manasse og Efraim og det øvrige Israel og hos hele Juda og Benjamin og Jerusalems Indbyggere;
\par 10 de overgav Pengene til dem, der stod for Arbejdet, dem, der havde Tilsyn med HERRENs Hus; og de, der stod for Arbejdet på HERRENs Hus, brugte dem til at udbedre og istandsætte Templet,
\par 11 idet de overgav dem til Tømrerne og Bygningsmændene til Indkøb af tilhugne Sten og Tømmer til Tværbjælker og til Bjælker i de Bygninger, Judas Konger havde ødelagt.
\par 12 Folkene udførte Arbejdet samvittighedsfuldt; og Tilsynet med dem var overdraget Leviterne Jahat og Obadja af Merariterne og Zekarja og Mesjullam af Kehatiternes Sønner, for at de skulde lede dem.
\par 13 Og Leviterne havde Tilsyn med Lastdragerne og ledede alle dem, der havde med de forskellige Arbejder at gøre. Og af Leviterne var nogle Skrivere, Fogeder og Dørvogtere.
\par 14 Men da de tog Pengene frem, der var kommet ind til HERRENs Hus, fandt Præsten Hilkija Bogen med HERRENs Lov, som var givet ved Moses;
\par 15 og Hilkija tog til Orde og sagde til Statsskriveren Sjafan: "Jeg har fundet Lovbogen i HERRENs Hus!" Og Hilkija gav Sjafan Bogen,
\par 16 og Sjafan bragte Bogen til Kongen og aflagde der hos Beretning for ham, idet han sagde: "Alt, hvad dine Trælle er sat til, udfører de;
\par 17 de har taget de Penge frem, der fandtes i HERRENs Hus, og givet dem til Tilsynsmændene og dem, der står for Arbejdet."
\par 18 Derpå gav Statsskriveren Sjafan Kongen den Meddelelse: "Præsten Hilkija gav mig en Bog." Og Sjafan læste op af den for Kongen.
\par 19 Men da Kongen hørte, hvad der stod i Loven, sønderrev han sine Klæder;
\par 20 og Kongen bød Hilkija, Ahikam. Sjafans Søn, Abdon, Mikas Søn, Statsskriveren Sjafan og Kongens Tjener Asaja:
\par 21 "Gå hen og rådspørg HERREN på mine Vegne og på deres, som er blevet tilovers i Israel og Juda, om Indholdet af denne Bog, der er fundet; thi stor er Vreden, der er blusset op hos HERREN imod os, fordi vore Fædre ikke adlød HERRENs Ord og handlede nøje efter. hvad der står skrevet i denne Bog!"
\par 22 Hilkija og de andre, Kongen sendte af Sted, gik da hen og talte derom med Profetinden Hulda, som var gift med Sjallum, Opsynsmanden over Tøjet, en Søn af Hasras Søn Tokhat, og som boede i Jerusalem i den nye Bydel.
\par 23 Hun sagde til dem: "Så siger HERREN, Israels Gud: Sig til den Mand, der sendte eder til mig:
\par 24 Så siger HERREN: Se, jeg vil bringe Ulykke over dette Sted og dets Indbyggere, alle de Forbandelser, der er optegnet i den Bog. som er læst op for Judas Konge,
\par 25 til Straf for at de har forladt mig og tændt Offerild for andre Guder, så de krænkede mig med alt deres Hænders Værk, og min Vrede vil blusse op mod dette Sted uden at slukkes!
\par 26 Men til Judas Konge, der sendte eder for at rådspørge HERREN,skal I sige således: Så siger HERREN. Israels Gud: De Ord, du har hørt, står fast;
\par 27 men efterdi dit Hjerte bøjede sig og du ydmygede dig for Gud. da du hørte hans Ord mod dette Sted og dets Indbyggere, og efterdi du ydmygede dig for mit Åsyn og sønderrev, dine Klæder og græd for mit Åsyn, så har også jeg hørt dig, lyder det fra HERREN!
\par 28 Så vil jeg da lade dig samles til dine Fædre, og du skal samles til dem i Fred i din Grav, uden at dine Øjne får al den Ulykke at se, som jeg vil bringe over dette Sted og dets Beboere!" Det Svar hragte de til Kongen.
\par 29 Da sendte Kongen Bud og lod alle Judas og Jerusalems Ældste kalde sammen.
\par 30 Derpå gik Kongen op i HERRENs Hus, fulgt af alle Judas Mænd og Jerusalems Indbyggere, Præsterne, Leviterne og alt Folket, store og små, og han forelæste dem alt, hvad der stod i Pagtsbogen, som var fundet i HERRENs Hus.
\par 31 Derpå stillede Kongen sig på sin Plads og sluttede Pagt for HERRENs Åsyn om, at de skulde holde sig til HERREN og holde hans Bud, Vidnesbyrd og Anordninger af hele deres Hjerte og hele deres Sjæl, for at han kunde opfylde Pagtens Ord, dem, der var skrevet i denne Bog.
\par 32 Og han lod alle dem, der var til Stede i Jerusalem, indgå Pagten; og Jerusalems Indbyggere haodlede efter Guds, deres Fædres Guds, Pagt.
\par 33 Derpå fjernede Josias alle Vederstyggelighederne fra alle de Landsdele, der tilhørte Iraeliterne, og sørgede for, at enhver i Israel dyrkede HERREN deres Gud. Så længe han levede, veg de ikke fra HERREN, deres Fædres Gud.

\chapter{35}

\par 1 Derpå fejrede Josias Påske for HERREN i Jerusalem, og de slagtede Påskelammet den fjortende Dag i den første Måned.
\par 2 Han satte Præsterne til det, de havde at varetage, og opmuntrede dem til Tjenesten i HERRENs Hus;
\par 3 og til Leviterne, som underviste hele Israel og var helliget HERREN, sagde han: "Sæt den hellige Ark i Templet, som Davids Søn, Kong Salomo af Israel, byggede; I skal ikke mere bære den på Skuldrene. Tjen nu HERREN eders Gud og hans Folk Israel!
\par 4 Gør eder rede Fædrenehus for Fædrenehus, Skifte for Skifte, efter Kong David af Israels Forskrift og hans Søn Salomos Anvisning,
\par 5 og stil eder op i Helligdommen, således at der bliver et Skifte af et levitisk Fædrenehus for hver Afdeling af eders Brødres, Almuens, Fædrenehuse,
\par 6 og slagt så Påskeofferdyrene, helliger eder og tillav dem til eders Brødre for at handle efter HERRENs Ord ved Moses."
\par 7 Josias gav frivilligt Almuen, alle dem, der var til Stede, en Ydelse af Småkvæg, Lam og Gedekid, alt sammen til Påskeofferdyr, 30.000 Stykker i Tal, og 3.000 Stykker Hornkvæg, alt af Kongens Ejendom;
\par 8 og hans Øverster gav frivilligt Folket, Præslerne og Leviterne en Ydelse; Hilkija, Zekarja og Jehiel, Guds Hus's Øverster, gav Præsterne til Påskeofferdyr 2.600 Stykker Småkvæg og 300 Stykker Hornkvæg.
\par 9 Leviternes Øverster Konanja og hans Brød1e Sjemaja og Netan'el, Hasjabja, Je'iel og Jozabad ydede Leviterne til Påskeofferdyr 5.000 Stykker Småkvæg og 500 Stykker Hornkvæg.
\par 10 Således ordnedes Tjenesten, og Præsterne stod på deres Plads, ligeledes Leviterne, Skifte for Skifte efter Kongens Bud.
\par 11 De slagtede Påskedyrene, og Præsterne sprængte Blodet, som de rakte dem, medens Leviterne flåede Huden af.
\par 12 Derpå gjorde de Brændofrene tede for at give dem til de enkelte Afdelinger af Almuens Fædrenehuse, så at de kunde frembæres for HERREN, som det er foreskrevet i Moses's Bog, og på samme Måde gjorde de med Hornkvæget.
\par 13 Påskedyrene stegte de over Ilden på den foreskrevne Måde, men de hellige Stykker kogte de i Gryder, Kedler og Skåle og bragte dem skyndsomt til Almuen.
\par 14 Derefter gjorde de Påskedyr rede til sig selv og Præsterne, thi Præsterne, Arons Sønner, var sysselsatte med at ofre 81ændofrene og Fedtstykkerne lige til Nattens Frembrud; derfor gjorde Leviterne Ofre rede både for sig selv og Præsterne, Arons Sønner.
\par 15 Sangerne, Asafs Sønner, var på deres Plads efterDavids, Asafs, Hemans og Kongens Seer Jedutuns Bud, og Dørvogterne ved de forskellige Porte; de måtte ikke forlade deres Plads, men deres Brødre Leviterne gjorde Påskedyr rede for dem.
\par 16 Således ordnedes hele HERRENs Tjeneste den Dag, idet man fejrede Påsken og bragte Brændofre på HERRENs Alter efter Kong Josias's Bud;
\par 17 og Israeliterne, som var til Stede, fejrede dengang Påsken og de usyrede Brøds Højtid i syv Dage.
\par 18 En Påske som den var ikke blevet fejret i Israel siden Profeten Samuels Dage, og ingen af Israels Konger havde fejret en Påske som den, Josias, Præsterne og Leviterne og alle de Judæere og Israeliter, som var til Stede, og Jerusalems Indbyggere fejrede.
\par 19 I Josias's attende Regeringsår blev denne Påsk fejret.
\par 20 Efter alt dette, da Josias havde sat Templet i Stand, drog Ægypterkongen Neko op til Kamp ved Karkemisj, der ligger ved Eufrat. Josias drog imod ham;
\par 21 men han sendte Sendebud til ham og lod sige: "Hvad er der mig og dig imellem, Judas Konge? Det er ikke dig, det nu gælder, men det Kongehus, jeg ligget i Krig med; og Gud har sagt, at jeg skulde haste. Gå ikke imod den Gud, der er med mig, at han ikke skal ødelægge dig!"
\par 22 Josias vendte dog ikke om, men vovede at indlade sig i Hamp med ham; han tog ikke Hensyn til Nekos Ord, der dog kom fra Guds Mund, men drog ud til Kamp på Megiddos Slette.
\par 23 Da ramte Bueskytterne Kong Josias; og Kongen sagde til sine Folk: "Før mig bort, thi jeg er hårdt såret!"
\par 24 Hans Folk bragte ham da bort fra Vognen og satte ham på hans anden Vogn og førte ham til Jerusalem, hvor han døde. De jordede ham i hans Fædres Grave, og hele Juda og Jerusalem sørgede over Josias.
\par 25 Jeremias sang en Klagesang over Josias, og alle Sangerne og Sangerinderne talte i deres Klagesange om ham, som de gør den Dag i Dag; man gjorde dette til en stående Skik i Israel, og Sangene står optegnet blandt Klagesangene.
\par 26 Hvad der ellers er at fortælle om Josias og hans fromme Gerninger, der sfemte med, hvad der er foreskrevet i HERRENs Lov,
\par 27 hans Historie fra først til sidst står jo optegnet i Bogen om Israels og Judas Konger.

\chapter{36}

\par 1 Folket fra Landet tog nu Josias's Søn Joahaz og hyldede ham til Konge i Jerusalem i hans Faders Sted.
\par 2 Joahaz var tre og tyve År gammel, da han blev Konge, og han herskede tre Måneder i Jerusalem.
\par 3 Men Ægypterkongen afsatte ham fra Regeringen i Jerusalem og lagde en Skat af hundrede Talenter Sølv og ti Talenter Guld på Landet.
\par 4 Derpå gjorde Ægypterkongen hans Broder Eljakim til Konge over Juda og Jerusalem, og han ændrede hans Navn til Jojakim; hans Broder Joahaz derimod tog Neko med til Ægypten.
\par 5 Jojakim var fem og tyve År gammel, da han blev Konge, og han herskede elleve År i Jerusalem. Han gjorde, hvad der var ondt i HERREN hans Guds Øjne.
\par 6 Kong Nebukadnezer af Babel drog op imod ham og lagde ham i Kobberlænker for at føre ham til Babel;
\par 7 og Nebukadnezar lod en Del af HERRENs Hus's Kar bringe til Babel og opstillede dem i sin Borg i Babel.
\par 8 Hvad der ellers er at fortælle om Jojakim og de Vederstyggeligheder, han øvede, hvad der er at sige om ham, står optegnet i Bogen om Israels og Judas Konger. Og hans Søn Jojakin blev Konge i hans Sted.
\par 9 Jojakin var atten År gammel, da han blev Konge, og han herskede tre Måneder og ti Dage i Jerusalem. Han gjorde, hvad der var ondt i HERRENs Øjne.
\par 10 Næste År sendfe Kong Nebukadnezar Folk og lod ham bringe til Babel tillige med HERRENs Hus's kostelige Kar; og han gjorde hans Broder Zedekias til Konge over Juda og Jerusalem.
\par 11 Zedekias var een og tyve År gammel, da han blev Konge, og han herskede i elleve År i Jerusalem.
\par 12 Han gjorde, hvad der var ondt i HERREN hans Guds Øjne. Han ydmygede sig ikke under de Ord, Profeten Jeremias talte fra HERRENs Mund.
\par 13 Desuden faldt han fra Kong Nebudkanezar, der havde taget ham i Ed ved Gud; og han var halsstarrig og forhærdede sit Hjerte, så han ikke omvendte sig til HERREN, Israels Gud.
\par 14 Ligeledes gjorde alle Judas Øverster og Præsterne og Folket sig skyldige i megen Troløshed ved at efterligne alle Hedningefolkenes Vederstyggeligheder, og de besmittede HERRENs Hus, som han havde helliget i Jerusalem.
\par 15 HERREN, deres Fædres Gud, sendte tidlig og silde manende Ord til dem ved sine Sendebud, fordi han ynkedes oer sit Folk og sin Bolig;
\par 16 men de spottede Guds Sendebud, lod hånt om hans Ord og gjorde sig lystige over hans Profeter, indtil HERRENs Vrede mod hans Folk tog til i den Grad, at der ikke mere var Lægedom.
\par 17 Han førte Kaldæernes Konge imod dem, og han dræbte deres unge Mandskab med Sværdet i deres hellige Tempel og ynkedes ikke over Yngling eller Jomfru, gammel eller Olding - alt overgav han i hans Hånd.
\par 18 Alle Karrene i Guds Hus, store og små, HERRENs Hus's Skatte og Kongens og hans Øverstes Skatte lod han alt sammen bringe til Babel.
\par 19 De stak Ild på Guds Hus, nedrev Jerusalems Mur, opbrændte alle dets Borge og ødelagde alle kostelige Ting deri.
\par 20 Dem, Sværdet levnede, førte han som Fanger til Babel, hvor de blev Trælle for ham og hans Sønner, indtil Perserriget fik Magten,
\par 21 for at HERRENs Ord gennem Jeremias's Mund kunde opfyldes, indtil Landet fik sine Sabbater godtgjort; så længe Ødelæggelsen varede, hvilede det, til der var gået halvfjerdsindstyve År.
\par 22 Men i Perserkongen Kyros's første Regeringsår vakte HERREN, for at hans Ord gennem Jeremias's Mund kunde opfyldes, Perserkongen Kyros's Ånd, så han lod følgende udråbe i hele sit Rige og desuden kundgøre ved en Skrivelse:
\par 23 "Perserkongen Kyros gør vitterligt: Alle Jordens Riger har HERREN, Himmelens Gud, givet mig; og han har pålagt mig at bygge ham et Hus i Jerusalem i Juda. Hvem iblandt eder, der hører til hans Folk, med ham være HERREN hans Gud, og han drage derop!"


\end{document}