\begin{document}

\title{Esther}


\chapter{1}

\par 1 I Ahasverus's Dage - den Ahasverus, der herskede over Landene fra Indien til Ætiopien, 127 Lande -
\par 2 i hine Dage, da Kong Ahasverus sad på sin Kongetrone i Borgen Susan, tildrog der sig følgende.
\par 3 I sit tredje Regeringsår gjorde han et Gæstebud for alle sine Fyrster og sine Folk; Persiens og Mediens ypperste Hærførere og Landsdelenes Fyrster var hans Gæster,
\par 4 og han udfoldede sin kongelige Herligheds Rigdom og sin Magts Glans og Pragt for dem i mange Dage, 180 Dage.
\par 5 Og da disse Dage var omme, gjorde Kongen for hele Folket i Borgen Susan, fra den højeste til den laveste, et syv Dages Gæstebud på den åbne Plads foran Parken ved Kongeborgen.
\par 6 Hvidt Linned og violet Purpur var med Snore af fint Linned og rødt Purpur hængt op på Sølvstænger og Marmorsøjler, og Guld- og Sølvdivaner stod på et Gulv, der var indlagt med broget og hvidt Marmor, Perlemor og sorte Sten.
\par 7 Drikkene skænkedes i Guldbægre, alle forskellige, og der var kongelig Vin i store Måder på ægte Fyrstevis;
\par 8 og ved Drikkelaget gjaldt den Regel, at man ikke nødte nogen; thi Kongen havde pålagt alle sine Hovmestre at lade enhver om, hvor meget han vilde have.
\par 9 Også Dronning Vasjti gjorde et Gæstebud for Kvinderne i Kong Ahasverus's Kongeborg.
\par 10 Da Kongen den syvende Dag var oprømt af Vinen, bød han Mehuman, Bizta, Harbona, Bigta, Abagta, Zetar og Karkas, de syv Hofmænd, som stod i Kong Ahasverus's Tjeneste,
\par 11 at føre Dronning Vasjti, prydet med det kongelige Diadem, frem for Kongen, for at han kunde vise Folkene og Fyrsterne hendes Dejlighed. Thi hun var meget smuk.
\par 12 Men bronning Vasjti vægrede sig ved at komme på Kongens Bud, som Hofmændene overbragte. Da blev Kongen harmfuld, og Vreden blussede op i ham.
\par 13 Og Kongen spurgte de vise, som kendte til Tidernes Tydning - thi Kongens Ord blev efter Skik og Brug forelagt alle de lov- og retskyndige,
\par 14 og de, der stod ham nærmest, var Karsjena, Sjetar, Admata, Tarsjisj, Meres, Marsena og Memukan, de syv persiske og mediske Fyrster, som så Kongens Åsyn og havde den øverste Magt i Riget:
\par 15 Hvad skal der efter Loven gøres ved Dronning Vasjti, fordi hun ikke adlød den Befaling, Kong Ahasverus gav hende ved Hofmændene?
\par 16 Da sagde Memukan i Kongens og Fyrsternes Påhør: "Dronning Vasjti har ikke alene forbrudt sig imod Kongen, men også imod alle Fyrster og alle Folk i alle Kong Ahasverus's Lande;
\par 17 thi Dronningens Opførsel vil rygtes blandt alle Kvinderne, og Følgen bliver, at de viser deres Mænd Ringeagt, når det hedder sig: Kong Ahasverus bød, at man skulde føre Dronning Vasjti til ham, men hun kom ikke!
\par 18 Og så snart de hører om Dronningens Adfærd, lader Persiens og Mediens Fyrstinder alle Kongens Fyrster det høre; deraf kan der kun komme Ringeagt og Vrede.
\par 19 Hvis Kongen synes, så lade han udgå et kongeligt Bud, som skal optegnes i Persiens og Mediens Love og være uigenkaldeligt, om at Vasjti aldrig mere må vise sig for Kong Ahasverus; og Kongen skal give hendes kongelige Værdighed til en anden, som er bedre end hun.
\par 20 Når så den Forordning, kongen lader udgå, bliver kendt i hele hans Rige - thi det er stort - da vil alle Kvinderne, både høje og lave, vise deres Mænd Agtelse.
\par 21 Det Forslag var godt i Kongens og Fyrsternes Øjne, og Kongen fulgte Memukans Forslag.
\par 22 Han sendte Skrivelser til alle Kongens Lande, til hver Landsdel med dens egen Skrift og til hvert Folk på dets eget Sprog, om at hver Mand skulde være Herre i sit eget Hus og tale sit Folks Sprog.

\chapter{2}

\par 1 Men da der var gået nogen Tid, og Kong Ahasverus's Vrede havde lagt sig, kom han til at tænke på Vasjti, og hvad hun havde gjort, og hvad der var besluttet om hende.
\par 2 Da sagde Kongens Folk, der gik ham til Hånde: Man bør søge efter unge, smukke Jomfruer til Kongen;
\par 3 og Kongen bør overdrage Folk i alle sit Riges Dele det Hverv at samle alle unge, smukke Jomfruer og sende dem til Fruerstuen i Borgen Susan og der lade dem stille under Opsyn af Kongens Hofmand Hegaj, som vogter Kvinderne; lad dem så gennemgå Skønhedsplejen,
\par 4 og den unge Pige, Kongen synes om, skal være Dronning i Vasjtis Sted. Det Forslag var godt i Hongens Øjne, og han gjorde derefter.
\par 5 Nu var der i Borgen Susao en jødisk Mand ved Navn Mordokaj, en Søn af Ja'ir, en Søn af Sjim'i, en Søn af Kisj, en Benjaminit,
\par 6 som var ført bort fra Jerusalem blandt de Fanger, Kong Nebukadnezar af Babel bortførte sammen med Kong Jekonja af Juda.
\par 7 Han var Plejefader for Hadassa - det er Ester - hans Farbroders Datter; thi hun havde hverken Fader eller Moder. Den unge Pige havde en smuk Skikkelse og så godt ud; og efter hendes Forældres Død havde Mordokaj taget hende til sig i Datters Sted.
\par 8 Da nu Kongens Befaling og Bud blev kendt, og mange unge Piger samledes i Borgen Susan, hvor de stilledes under Opsyn af Hegaj, blev også Ester bragt til Kongens Hus og stillet under Opsyn af Hegaj, som vogtede Kvinderne.
\par 9 Pigen tiltalte ham og vandt hans Yndest, og så hurtigt som muligt lod han Skønhedsplejen foretage på hende og gav hende den Kost, hun skulde have, og stillede tillige de syv dertil udsete Piger fra Kongens Hus til hendes Tjeneste; og han lod hende og Pigerne flytte til den bedste Del af Fruerstuen.
\par 10 Ester røbede imidlertid ikke sit Folk og sin Slægt, thi det havde Mordokaj forbudt hende.
\par 11 Og Mordokaj gik Dag efter Dag frem og tilbage foran Fruerstuens Gård for at få at vide, hvorledes Ester havde det, og hvorledes det gik hende.
\par 12 Nu var det således, at når en af de unge Pigers Tid til at gå ind til Kong Ahasverus kom, efter at hun i tolv Måneder var behandlet efter Forskriften for Kvinderne sålang Tid tog nemlig Skønhedsplejen; seks Måneder blev de behandlet med Myrraolie og andre seks Måneder med vellugtende Stoffer og de andre Skønhedsmidler, som bruges af Kvinder
\par 13 når så den unge Pige gik ind til Kongen, gav man hende alt, hvad hun bad om, med fra Fruerstuen til Kongens Hus.
\par 14 Hun gik da derind om Aftenen, og næste Morgen vendte hun tilbage og kom så ind i den anden Fruerstue og blev stillet under Opsyn af Sja'asjgaz, den kongelige Hofmand, som vogtede Medhustruerne; så kom hun ikke mere til Kongen, medmindre Kongen havde syntes særlig godt om hende og hun udtrykkelig blev kaldt til ham.
\par 15 Da nu Tiden kom til, at Ester, en Datter af Abihajil, der var Farbroder til Mordokaj, som havde taget hende til sig i Datters Sted, skulde gå ind til Kongen, krævede hun ikke andet, end hvad Hegaj, den kongelige Hofmand, som vogtede Kvinderne, rådede til. Og Ester vandt Yndest hos alle, som så hende.
\par 16 Så blev Ester hentet til Kong Ahasverus i hans kongelige Palads i den tiende Måned, det er Tebet Måned, i hans syvende Regeringsår.
\par 17 Og Kongen fik Ester kærere end alle de andre Kvinder, og hun vandt hans Yndest og Gunst mere end alle de andre Jomfruer. Og han satte et kongeligt Diadem på hendes Hoved og gjorde hende til Dronning i Vasjtis Sted.
\par 18 Derpå gjorde Kongen et stort Gæstebud for alle sine Fyrster og Folk til Ære for Ester, og han eftergav Straf i sine Lande og uddelte Gaver, som det sømmede sig en Konge.
\par 19 Da Mordokaj engang sad i Kongens Port -
\par 20 Ester havde, som Mordokaj havde pålagt hende, intet røbet om sin Slægt og sit Folk; thi Ester gjorde, hvad Mordokaj sagde, som hun havde gjort, da hun var i Pleje hos ham
\par 21 som Mordokaj ved den Tid engang sad i Kongens Port, blev Bigtan og Teresj, to kongelige Hofmænd, der hørte til Dørvogterne, vrede på Kong Ahasverus og søgte Lejlighed til at lægge Hånd på ham.
\par 22 Det fik Mordokaj at vide og meddelte Dronning Ester det; og Ester sagde det til Kongen fra Mordokaj.
\par 23 Sagen blev undersøgt, og da den havde sin Rigtighed, blev de begge hængt i en Galge. Det blev optegnet i Krøniken i Kongens Påsyn.

\chapter{3}

\par 1 Nogen Tid efter gav Kong Ahasverus Agagiten Haman, Hammedatas Søn, en høj Stilling og udmærkede ham og gav ham Forsædet blandt alle Fyrsterne, som var hos ham.
\par 2 Og alle Kongens Tjenere, som var i Kongens Port, faldt på Knæ og kastede sig til Jorden for Haman, thi den Ære havde Kongen påbudt at vise ham. Men Mordokaj faldt ikke på Knæ og kastede sig ikke til Jorden.
\par 3 Kongens Tjenere, som var i Kongens Port, sagde da til Mordokaj: "Hvorfor overtræder du Kongens Bud?"
\par 4 Og da de havde sagt det til ham flere Dage i Træk, uden at han ænsede det, meldte de Haman det for at se, om Mordokajs Ord vilde blive taget for gyldige; thi han havde gjort gældende over for dem, at han var Jøde.
\par 5 Da nu Haman så, at Mordokaj hverken faldt på Knæ eller kastede sig til Jorden for ham, blev han såre opbragt.
\par 6 Og da det blev fortalt ham, hvilket Folk Mordokaj tilhørte, var han ikke tilfreds med kun at lægge Hånd på Mordokaj, men satte sig til Mål at få alle Jøderne i hele Ahasverus's Rige udryddet, fordi det var Mordokajs Folk.
\par 7 I den første Måned, det er Nisan Måned, i Kong Ahasverus's tolvte Regeringsår kastede man i Hamans Påsyn Pur, det er Lod, om hver enkelt Dag og hver enkelt Måned, og Loddet traf den trettende Dag i den tolvte Måned, det er Adar Måned.
\par 8 Haman sagde derpå til Kong Ahasverus: Der findes et Folk, som bor spredt og lever for sig selv iblandt Folkene i alle dit Riges Dele; deres Love er anderledes end alle andre Folks, og Kongens Love holder de ikke. Derfor er det ikke Kongen værdigt at lade dem være i Fred.
\par 9 Hvis Kongen synes, lad der så udgå skriftlig Befaling til at udrydde dem; jeg vil da kunne tilveje Embedsmændene 10.000 Talenter Sølv til at lægge i Kongens Skatkamre.
\par 10 Da tog Kongen Seglringen at sin Hånd og gav den til Agagiten Haman, Hammedatas Søn, Jødernes Fjende,
\par 11 og Kongen sagde til Haman: "Sølvet skal tilhøre dig, og med Folket kan du gøre, hvad du finder for godt!"
\par 12 Kongens Skrivere blev så tilkaldt den trettende Dag i den første Måned; og ganske som Haman bød, affattedes Skrivelser til de kongelige Satraper og Statholdere over hver enkelt Laodsdel og til hvert enkelt Folks Fyrster, til hver Landsdel med dens egen Skrift og til hvert Folk på dets eget Sprog. I Kong Ahasverus's Navn blev de skrevet, og de forsegledes med Kongens Seglring.
\par 13 Skrivelser sendtes så ved Ilbud ud i alle Kongens Lande med Befaling til at udrydde, ihjelslå og tilintetgøre alle Jøder, unge og gamle, Børn og Kvinder, på een Dag, den trettende Dag i den tolvte Måned, det er Adar Måned, og at prisgive deres Ejendele.
\par 14 En Afskrift af Skrivelsen, der skulde udstedes som Forordning i alle Rigets Dele, blev kundgjort for alle Folkene, for at de kunde være rede til den Dag.
\par 15 Ilbudene skyndte sig af Sted på Kongens Bud, så snart Forordningen var udgået i Borgen Susan. Kongen og Haman satte sig så til at drikke; men Byen Susan var rædselsslagen.

\chapter{4}

\par 1 Da Mordokaj fik at vide alt, hvad der var sket, sønderrev han sine Klæder, klædte sig i Sæk og Aske og gik ud i Byen og udstødte høje Veråb;
\par 2 og han kom hen på Pladsen foran Kongens Port, men heller ikke længere, fordi det ikke var tilladt at gå ind i Kongens Port, når man var klædt i Sæk.
\par 3 Og i hver eneste Landsdel, overalt, hvor Kongens Bud og Forordning nåede hen, var der blandt Jøderne stor Sorg og Faste, Gråd og Klage, og mange af dem redte sig et Leje af Sæk og Aske.
\par 4 Da nu Esters Piger og Hofmænd kom og fortalte hende det, grebes Dronningen af heftig Smerte; og hun sendte Klæder ud til Mordokaj, for at man skulde give ham dem på og tage Sørgeklæderne af ham; men han tog ikke imod dem.
\par 5 Da lod Ester Hatak, en af Kongens Hofmænd, som han havde stillet til hendes Tjeneste, kalde, og sendte ham til Mordokaj for at få at vide, hvad det skulde betyde, og hvad Grunden var dertil.
\par 6 Da Hatak kom ud til Mordokaj på Byens Torv foran Kongens Port,
\par 7 fortalte Mordokaj ham alt, hvad der var hændt ham, og opgav ham nøje, hvor meget Sølv Haman havde lovet at tilveje Kongens Skatkamre for at få Lov til at tilintetgøre Jøderne.
\par 8 Desuden gav han ham en Afskrift af Skrivelsen med den den i Susan udgåede Forordning om at udrydde dem, for at han skulde vise Ester den og tilkendegive hende det og pålægge hende at gå ind til Kongen og bede ham om Nåde og gå i Forbøn hos ham for sit Folk.
\par 9 Halak gik så ind og lod Ester vide, hvad Mordokaj havde sagt.
\par 10 Men Ester sendte Halak til Mordokaj med følgende Svar:
\par 11 Alle Kongens Tjenere og Folkene i Kongeos Lande ved, at der for enhver, Mand eller Kvinde, som ukaldet går ind til Kongen i den inderste Gård, kun gælder een Lov, den, at han skal lide Døden, medmindre Kongen rækker sit gyldne Septer ud imod ham; i så Fald beholder han Livet. Men jeg har nu i tredive Dage ikke været kaldt til Kongen!
\par 12 Da han havde meddelt Mordokaj Esters Ord,
\par 13 bød Mordokaj ham svare Ester: "Tro ikke, at du alene af alle Jøder skal undslippe, fordi du er i Kongens Hus!
\par 14 Nej, dersom du virkelig tier ved denne Lejlighed, så kommer der andetsteds fra Hjælp og Redning til Jøderne; men du og din Slægt skal omkomme. Hvem ved, om det ikke netop er for sligt Tilfældes Skyld, at du er kommet til kongelig Værdighed!
\par 15 Da sendte Ester Mordokaj det Svar:
\par 16 Gå hen og kald alle Susans Jøder sammen og hold Faste for mig, således at I hverken spiser eller drikker Dag eller Nat i tre Døgn; på samme Måde vil også jeg og mine Terner faste; og derefter vil jeg gå ind til Kongen, skønt det er imod Loven; skal jeg omkomme, så lad mig da omkomme!
\par 17 Så gik Mordokaj hen og gjorde ganske som Ester havde pålagt ham.

\chapter{5}

\par 1 Den tredje Dag iførte Ester sig det kongelige Skrud og trådte ind i den indre Gård til Kongens Palads, foran Kongens Palads, medens Kongen sad på sin Kongetrone i det kongelige Palads ud imod Indgangen.
\par 2 Da Kongen så Dronning Ester stå i Gården, fandt hun Nåde for hans Øjne, og Kongen rakte det gyldne Scepter, som han havde i Hånden, ud imod Ester. Da trådte Ester hen og rørte ved Spidsen af Scepteret;
\par 3 og Kongen sagde til hende: "Hvad fattes dig, Dronning Ester, og hvad er dit Ønske? Om det så er Halvdelen af Riget, skal du få det!
\par 4 Ester svarede: "Hvis Kongen synes, vil jeg bede Kongen og Haman om i Dag at komme til et Gæstebud, jeg har gjort rede for ham."
\par 5 Da sagde Kongen: Send hurtigt Bud efter Haman, for at Esters Ønske kan blive opfyldt!" Så kom Haman og Kongen til det Gæstebud, Ester havde gjort rede,
\par 6 og da de sad ved Vinen, sagde Kongen til Ester: Hvad er din Bøn? Du skal få den opfyldt. Og hvad er dit Ønske? Om det så er Halvdelen af Riget, skal det tilstås dig!
\par 7 Ester svarede: "Min Bøn og mit Ønske er
\par 8 hvis jeg har fundet Nåde for Kongens Øjne, og hvis det synes Kongen ret at opfylde min Bøn og tilstå mig mit Ønske, så komme Kongen og Haman til et Gæstebud, jeg vil gøre rede for dem. I Morgen vil jeg da gøre, som Kongen siger!"
\par 9 Haman gik glad og vel til Mode derfra den Dag. Men da Haman så Mordokaj i Kongens Port, og han hverken rejste sig op eller rørte sig af Pletten for ham, opfyldtes han af Vrede mod Mordokaj.
\par 10 Dog tvang han sig; men da han var kommet hjem, sendte han Bud efter sine Venner og sin Hustru Zeresj;
\par 11 og Haman talte til dem om sin overvættes Rigdom og sine mange Sønner og om al den Ære, Kongen havde vist ham, og hvorledes han havde udmærket ham frem for Fyrsterne og Kongens Folk.
\par 12 Og Haman sagde: Dronning Ester lod heller ikke andre end mig komme med Kongen til det Gæstebud, hun havde gjoet rede; og jeg er også indbudt af hende til i Morgen sammen med Kongen.
\par 13 Men alt det er mig ikke nok, så længe jeg ser denne Jøde Mordokaj sidde i Kongens Port."
\par 14 Da sagde hans Hustru Zeresj og alle hans Venner til ham: "Lad en Galge rejse, halvtredsindstyve Alen høj, og bed i Morgen tidlig Kongen om, at Mordokaj må blive hængt i den; så kan du gå glad til Gæstebudet med Kongen.

\chapter{6}

\par 1 Samme Nat veg Søvnen fra Kongen. Da bød han, at man skulde hente Krøniken, i hvilken mindeværdige Tildragelser var optegnet, og man læste op for Kongen af den.
\par 2 Man fandt da optegnet, hvorledes Mordokaj havde meldt, at Bigtana og Teresj, to kongelige Hofmænd, der hørte til Dørvogterne, havde søgt Lejlighed til at lægge Hånd på Kong Ahasverus.
\par 3 Kongen spurgte da: "Hvilken Ære og Udmærkelse er der vist Mordokaj til Gengæld?" Kongens Folk, som gik ham til Hånde, svarede: "Der er ingen Ære vist ham."
\par 4 Så spurgte Kongen: Hvem er ude i Gården? Haman var netop kommet ind i den ydre Gård til Kongens Palads for at bede Kongen om, at Mordokaj måtte blive hængt i den Galge, han havde rejst til ham.
\par 5 Kongens Folk svarede ham: "Det er Haman, der står ude i Gården." Da sagde Kongen: "Lad ham komme ind!"
\par 6 Da Haman var kommet ind; sagde Kongen til ham: "Hvad gør man ved den Mand, Kongen ønsker at hædre?" Haman tænkte ved sig selv: "Hvem andre end mig skulde Kongen ønske at hædre?"
\par 7 Derfor svarede Haman Kongen: "Hvis Kongen ønsker at hædre en Mand,
\par 8 skal man lade hente en kongelig Klædning, som Kongen selv har båret, og en Hest, som Kongen selv har redet, og på hvis Hoved der er sat en kongelig Krone,
\par 9 og man skal overgive Klædningen og Hesten til en af Kongens ypperste Fyrster og give den Mand, Kongen ønsker at hædre, Klædningen på og føre ham på Hesten over Byens Torv og råbe foran ham: Således gør man ved den Mand, Kongen ønsker at hædre!
\par 10 Da sagde Kongen til Haman: "Skynd dig at hente Klædningen og Hesten, som du sagde, og gør således ved Jøden Mordokaj, som sidder i den kongelige Port! Undlad intet af, hvad du sagde!
\par 11 Så hentede Haman Klædningen og Hesten, gav Mordokaj Klædningen på'og førte ham på Hesten over Byens Torv og råbte foran ham: Således gør man ved den Mand, Kongen ønsker at hædre!
\par 12 Derefter gik Morkodaj tilbage til Kongens Port. Men Haman skyndte sig hjem, nedslået og med tilhyllet Hoved.
\par 13 Og Haman fortalte sin Hustru Zeresj og alle sine Venner alt, hvad der var hændet ham. Da sagde hans Venner og hans Hustru Zeresj til ham: Hvis Mordokaj, over for hvem du nu for første Gang er kommet til kort, er af jødisk Æt, så kan du intet udrette imod ham, men det bliver dit Fald til sidst!
\par 14 Medens de endnu talte med ham, indtraf de kongelige Hofmænd for hurtigt at hente Haman til det Gæstebud, Ester havde gjort rede.

\chapter{7}

\par 1 Da Kongen tillige med Haman var kommet til Gæstebudet hos Dronning Ester
\par 2 spurgte Kongen atter den Dag Ester, medens de sad ved Vinen: "Hvad er din Bøn, Dronning Ester? Du skal få den opfyldt. Og hvad er dit Ønske? Om det så er Halvdelen af Riget, skal det tilstås dig!"
\par 3 Dronning Ester svarede: "Hvis jeg har fundet Nåde for dine Øjne, Konge, og hvis Kongen synes, giv mig så mit Liv på min Bøn og giv mig mit Folk på mit Ønske;
\par 4 thi jeg og mit Folk er solgt til at udryddes, ihjelslås og tilintetgøres.
\par 5 Da svarede Kong Ahasverus Dronning Ester: "Hvem er han, og hvor er han, som har fået i Sinde at gøre dette?"
\par 6 Ester svarede: En fjendsk og ildesindet Mand, den onde Haman der!" Da blev Haman slaget af Rædsel for Kongen og Dronningen.
\par 7 Og Kongen rejste sig i Vrede fra Gæstebudet og gik ud i Paladsets Park, men Haman blev tilbage for al bønfalde Dronning Ester om sit Liv; thi han mærkede, at det var Kongens faste Vilje at styrte ham i Ulykke.
\par 8 Da Kongen kom tilbage fra Paladsets Park til Gæstebudsalen, havde Haman netop kastet sig ned over Divanen, som Ester lå på. Så sagde Kongen: "Vil han oven i Købet øve Vold imod Dronningen her i Huset i min Nærværelse!" Næppe var det Ord udgået af Kongens Mund, før man tilhyllede Hamans Ansigt;
\par 9 og Harbona, en af Hofmændene, der stod i Kongens Tjeneste, sagde: "Ved Hamans Hus står allerede den halvtredsindstyve Alen høje Galge, som Haman har ladet rejse til Mordokaj, hvis Ord dog var Kongen til Gavn!" Da sagde Kongen: "Hæng ham i den!"
\par 10 Og de hængte Haman i den Galge, han havde rejst til Mordokaj. Så lagde Kongens Vrede sig.

\chapter{8}

\par 1 Samme Dag gav Kong Ahasverus Dronning Ester Hamans, Jødernes Fjendes, Hus.
\par 2 Og Kongen tog sin Seglring, som han havde frataget Haman, og gav Mordokaj den. Og Ester satte Mordokaj over Hamans Hus.
\par 3 Men Ester henvendte sig atter til Kongen, og hun kastede sig ned for hans Fødder og græd og bønfaldt ham om at afværge de onde Råd, Agagiten Haman havde lagt op imod Jøderne.
\par 4 Kongen rakte det gyldne Scepter ud imod Ester, og Ester rejste sig op og trådte hen til Kongen
\par 5 og sagde: Hvis Kongen synes, og hvis jeg har fundet Nåde for hans Ansigt og Kongen holder det for ret, og han har Behag i mig, lad der så blive givet skriftlig Befaling til at tilbagekalde de Skrivelser, Agagiten Haman, Hammedatas Søn, udpønsede, og som han lod udgå for at udrydde Jøderne i alle Kongens Lande;
\par 6 thi hvor kan jeg udholde at se den Ulykke, som rammer mit Folk, og hvor kan jeg udholde at se min Slægts Undergang!"
\par 7 Da sagde Kong Ahasverus til Dronning Ester og Jøden Mordokaj: Hamans Hus har jeg givet Ester, og han, selv er blevet hængt i Galgen, fordi han stod Jødeme efter Livet.
\par 8 Nu kan I selv i Kongens Navn affatte en Skrivelse om Jøderne, som I finder det for godt, og sætte det kongelige Segl under; thi en Skrivelse, der een Gan: er udgået i Kongens Navn og forseglet med det kongelige Segl, kan ikke kaldes tilbage.
\par 9 Så blev Kongens Skrivere med det samme tilkaldt på den tre og tyvende Dag i den tredje Måned, det er Sivan Måned, og der blev skrevet, ganske som Mordokaj bød, til Jøderne og til Satraperne og Statholderne og Fyrsterne i Landene fra Indien til Ætiopien, 127 Landsdele, til hver Landsdel med dens egen Skrift og til hvert Folk på dets eget Sprog, også til Jøderne med deres egen Skrift og på deres eget Sprog.
\par 10 Han affattede Skrivelser i Kong Ahasverus's Navn og forseglede dem med Kongens Seglring; derefter sendte han dem ud ved ridende Ilbud, der red på Gangere fra de kongelige Stalde, med Kundgørelse om,
\par 11 at Kongen tilstedte Jøderne i hver enkelt By at slutte sig sammen og værge deres Liv og i hvert Folk og hvert Land at udrydde, ihjelslå og tilintetgøre alle væbnede Skarer, som angreb dem, tillige med Børn og Kvinder, og at plyndre deres Ejendele,
\par 12 alt på en og samme Dag i alle Kong Ahasverus's Lande, på den trettende Dag i den tolvte Måned, det er Adar Måned.
\par 13 En Afskrift af Skrivelsen, der skulde udgå som Forordning i alle Rigets Dele, blev kundgjort for alle Folkene, for at Jøderne den Dag kunde være rede til at tage Hævn over deres Fjender.
\par 14 Så snart Forordningen var givet i Borgen Susan, skyndte Ilbudene, som red på de kongelige Gangere, sig på Kongens Bud af Sted så hurtigt, de kunde.
\par 15 Men Mordokaj gik fra Kongen i en Kongelig Klædning af violet Purpur og hvidt Linned, med et stort Gulddiadem og en Kappe af fint Linned og rødt Purpur, medens Byen Susan jublede og glædede sig.
\par 16 Jøderne havde nu Lykke og Glæde, Fryd og Ære;
\par 17 og i hver eneste Landsdel og i hver eneste By, hvor bongens Bud og Forordning nåede hen, var der Fryd og Glæde blandt Jøderne med Gæstebud og Fest. Og mange af Hedningerne gik over til Jødedommen, thi Frygt for Jøderne var faldet på dem.

\chapter{9}

\par 1 På den trettende Dag i den tolvte Måned, det er Adar Måned, det er den Dag, da Kongens Befaling og Forordning skulde udføres, den Dag, da Jødernes Fjender havde håbet at kunne overvælde dem, medens det nu omvendt blev Jøderne, der på den Dag skulde overvælde deres Avindsmænd,
\par 2 sluttede Jøderne sig sammen i deres Byer i alle Kong Ahasverus's Lande for at lægge Hånd på dem, der vilde dem ondt; og ingen holdt Stand imod dem, thi Frygt for dem var faldet på alle Folkene.
\par 3 Og alle Landenes Fyrster og Satraperne og Statholderne og de kongelige Embedsmænd hjalp Jøderne, thi Frygt for Mordokaj var faldet på dem.
\par 4 Thi Mordokaj havde meget at sige ved Kongens Hof, og der gik Ry af ham i alle Lande; thi samme Mordokaj blev mægtigere og mægtigere.
\par 5 Således slog Jøderne løs på alle deres Fjender med Sværdhug, Drab og Ødelæggelse, og de handlede med deres Avindsmænd, som de havde Lyst.
\par 6 I Borgen Susan dræbte og tilintetgjorde Jøderne 500 Mand;
\par 7 og Parsjandata, Dalfon, Aspata,
\par 8 Porata, Adalja, Aridata,
\par 9 Parmasjta, Arisaj, Aridaj og Vajezata,
\par 10 de ti Sønner af Haman, Hammedatas Søn, Jødernes Fjende, dræbte de. Men efter Byttet rakte de ikke Hænderne ud.
\par 11 Samme dag kom Tallet på dem, der var dræbt i Borgen Susan, Kongen for Øre.
\par 12 Da sagde Kongen til Dronning Ester: I Borgen Susan har Jøderne dræbt og tilintetgjort 500 Mand, også Hamans ti Sønner; hvad må de da ikke have gjort i de andre kongelige Landsdele! Dog, hvad er din Bøn? Du skal få den opfyldt.
\par 13 Ester svarede: Hvis Kongen synes, lad det så også i Morgen tillades Jøderne i Susan at handle som i Dag og lad Hamans ti Sønner blive hængt op i Galger!
\par 14 Da bød Kongen, at det skulde ske; og der udgik en Forordning derom i Susan, og Hamans ti Sønner blev hængt op i Galger.
\par 15 Så sluttede Jøderne i Susan sig også sammen på den fjortende Dag i Adar Måned og dræbte 300 Mand i Susan. Men efter Byttet rakte de ikke Hænderne ud.
\par 16 Men også de andre Jøder i Kongens Lande sluttede sig sammen og værgede deres Liv og tog Hævn over deres Fjenderog dræbte blandt deres Avindsmænd 75000
\par 17 på den trettende Dag i Adar Måned; men efter Byttet rakte de ikke Hænderne ud; og de hvilede på den fjortende og gjorde den til en Gæstebuds- og Glædesdag.
\par 18 Men Jøderne i Susan sluttede sig sammen både den trettende og fjortende Dag i Måoeden og hvilede på den femtende, og den gjorde de til Gæstebuds- og Glædesdag.
\par 19 Derfor fejrer Jøderne på Landet, de, der bor i Landsbyerne, den fjortende Dag i Adar Måned som en Glædes-, Gæstebuds- og Festdag, på hvilken de sender hverandre Gaver.
\par 20 Og Mordokaj nedskrev disse Tildragelser og udsendte Skrivelser til alle Jøder i alle Kong Ahasverus's Lande nær og fjern
\par 21 for at gøre det til Pligt for dem hvert År at fejre den fjortende og femtende Adar
\par 22 de Dage, da Jøderne fik Ro for deres Fjender, og den Måned, da deres Trængsel vendtes til Glæde og deres Sorg til en Festdag - at fejre dem som Gæstebuds- og Glædesdage, på hvilke de skulde sende hverandre af deres Mad og de fattige Gaver.
\par 23 Og Jøderne vedtog, at det, som de nu for første Gang havde gjort, og som Mordokaj havde skrevet til dem om, skulde være en fast Skik.
\par 24 Fordi Agagiten Haman, Hammedatas Søn, alle Jøders Fjende, havde lagt Råd op imod Jøderne om at tilintetgøre dem og kastet Pur - det er Lod - for at ødelægge og tilintetgøre dem,
\par 25 men Kongen havde, da det kom ham for Øre, givet skriftlig Befaling til, at det onde Råd, Homan havde lagt op mod Jøderne, skulde falde tilbage på hans eget Hoved, og ladet ham og hans Sønner hænge i Galgen,
\par 26 derfor kaldte man de bage Purim efter Ordet Pur. Og derfor, på Grund af alt, hvad Brevet indeholdt, og hvad de selv havde oplevet i så Henseende, og hvad der var tilstødt dem,
\par 27 gjorde Jøderne det til en fast Skik og Brug for sig selv, deres Efterkommere og alle, som sluttede sig til dem, at de ubrødeligt År efter År skulde fejre de to Dage efter Forskrifterne om dem og til den fastsatte Tid,
\par 28 og at de Dage skulde ihukommes og fejres i alle Tidsaldre og Slægter, i hvert Laod og hver By, så at disse Purimsdage aldrig skulde gå af Brug hos Jøderne og deres Ihukommelse aldrig ophøre blandt deres Efterkommere.
\par 29 Derpå lod Dronning Ester, Abihajils Datter, og Jøden Mordokaj en eftertrykkelig Skrivelse udgå for at stadfæste dette Brev om Purim.
\par 30 Og han sendte Breve til alle Jøder i de 127 Lande i Ahasverus's Rige med Freds og Sandheds Ord
\par 31 om at holde disse Purimsdage i Hævd på den fastsatte Tid, således som Jøden Mordokaj og Dronning Ester havde gjort det til Pligt for dem, og således som de havde bundet sig selv og deres Efterkommere til de foreskrevne Faster og Klageråb.
\par 32 Således stadfæstedes disse Purimsforskrifter ved Esters Befaling; og det blev optegnet i en Bog.

\chapter{10}

\par 1 Kong Ahasverus lagde Skat på Fastlandet og Kystlandene.
\par 2 Og alt, hvad han gjorde i sin Magt og Vælde, og en nøjagtig Skildring af den høje Værdighed, Kongen ophøjede Mordokaj til, står optegnet i Mediens og Persiens Kongers Krønike.
\par 3 Thi Jøden Mordokaj havde den højeste Værdighed efter Kong Ahasverus, og han stod i høj Anseelse hos Jøderne og var elsket af sine mange Landsmænd, fordi han virkede for sit Folks vel og talte til Bedste for al sin Slægt.


\end{document}