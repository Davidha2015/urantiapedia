\begin{document}

\title{Jobs Bog}


\chapter{1}

\par 1 Der levede engang i Landet Uz en mand ved Navn Job. Det var en from og retsindig Mand, der frygtede Gud og veg fra det onde.
\par 2 Syv Sønner og tre Døtre fødtes ham;
\par 3 og hans Ejendom udgjorde 7000 Stykker Småkvæg, 3000 Kameler, 500 Spand Okser, 500 Aseninder og såre mange Trælle, så han var mægtigere end alle Østens Sønner.
\par 4 Hans Sønner havde for Skik at holde Gæstebud på Omgang hos hverandre, og de indbød deres tre Søstre til at spise og drikke sammen med sig.
\par 5 Når så Gæstebudsdagene havde nået Omgangen rundt, sendte Job Bud og lod Sønnerne hellige sig, og tidligt om Morgenen ofrede han Brændofre, et for hver af dem. Thi Job sagde: "Måske har mine Sønner syndet og forbandet Gud i deres Hjerte." Således gjorde Job hver Gang.
\par 6 Nu hændte det en Dag, at Guds Sønner kom og trådte frem for HERREN, og iblandt dem kom også Satan.
\par 7 HERREN spurgte Satan: "Hvor kommer du fra?" Satan svarede HERREN: "Jeg har gennemvanket Jorden på Kryds og tværs."
\par 8 HERREN spurgte da Satan: " Har du lagt Mærke til min Tjener Job? Der findes ingen som han på Jorden, så from og retsindig en Mand, som frygter Gud og viger fra det onde."
\par 9 Men Satan svarede HERREN: "Mon det er for intet, Job frygter Gud?
\par 10 Har du ikke omgærdet ham og hans Hus og alt, hvad han ejer, på alle Kanter? Hans Hænders Idræt har du velsignet, og hans Hjorde breder sig i Landet.
\par 11 Men ræk engang din Hånd ud og rør ved alt, hvad han ejer! Sandelig, han vil forbande dig lige op i dit Ansigt!"
\par 12 Da sagde HERREN til Satan: "Se, alt hvad han ejer, er i din Hånd; kun mod ham selv må du ikke udrække din Hånd!" Så gik Satan bort fra HERRENs Åsyn.
\par 13 Da nu en Dag hans Sønner og Døtre spiste og drak i den ældste Broders Hus,
\par 14 kom et Sendebud til Job og sagde: "Okserne gik for Ploven, og Aseninderne græssede i Nærheden;
\par 15 så faldt Sabæerne over dem og tog dem; Karlene huggede de ned med Sværdet; jeg alene undslap for at melde dig det."
\par 16 Medens han endnu talte, kom en anden og sagde: "Guds Ild faldt ned fra Himmelen og slog ned iblandt Småkvæget og Karlene og fortærede dem; jeg alene undslap for at melde dig det."
\par 17 Medens han endnu talte, kom en tredje og sagde: "Kaldæerne kom i tre Flokke og kastede sig over Kamelerne og tog dem; Karlene huggede de ned med Sværdet; jeg alene undslap for at melde dig det."
\par 18 Medens han endnu talte, kom en fjerde og sagde: "Dine Sønner og Døtre spiste og drak i deres ældste Broders Hus;
\par 19 og se, da for der et stærkt Vejr hen over Ørkenen, og det tog i Husets fire Hjørner, så det styrtede ned over de unge Mænd, og de omkom; jeg alene undslap for at melde dig det."
\par 20 Da stod Job op, sønderrev sin Kappe, skar sit Hovedhår af og kastede sig til Jorden, tilbad
\par 21 og sagde: Nøgen kom jeg af Moders Skød, og nøgen vender jeg did tilbage.
\par 22 I alt dette syndede Job ikke og tillagde ikke Gud noget vrangt.

\chapter{2}

\par 1 Nu hændte det en Dag, at Guds Sønner kom og trådte frem for HERREN, og iblandt dem kom også Satan og trådte frem for ham.
\par 2 HERREN spurgte Satan: "Hvor kommer du fra?" Satan svarede HERREN: "Jeg bar gennemvanket Jorden på Kryds og tværs."
\par 3 HERREN spurgte da Satan: "Har du lagt Mærke til min Tjener Job? Der findes ingen som han på Jorden, så from og retsindig en Mand, som frygter Gud og viger fra det onde. Endnu holder han fast ved sin Fromhed, og uden Grund har du ægget mig til at ødelægge ham!"
\par 4 Men Satan svarede HERREN: "Hud for Hud! En Mand giver alt, hvad han ejer, for sit Liv!
\par 5 Men ræk engang din Hånd ud og rør ved hans Ben og Kød! Sandelig, han vil forbande dig lige op i dit Ansigt!"
\par 6 Da sagde HERREN til Satan: "Se, han er i din Hånd; kun skal du skåne hans Liv!"
\par 7 Så gik Satan bort fra HERRENs Åsyn, og han slog Job med ondartet Bylder fra Fodsål til Isse,
\par 8 Og Job tog sig et Potteskår til at skrabe sig med, medens han sad i Askedyngen.
\par 9 Da sagde hans Hustru til ham: "Holder du endnu fast ved din Fromhed? Forband Gud og dø!"
\par 10 Men han svarede hende: "Du taler som en Dåre! Skulde vi tage imod det gode fra Gud, men ikke imod det onde?" I alt dette syndede Job ikke med sine Læber.
\par 11 Da Jobs tre Venner hørte om al den Ulykke, der havde ramt ham, kom de hver fra sin Hjemstavn. Temaniten Elifaz, Sjuhiten Bildad og Na'amatiten Zofar, og aftalte at gå hen og vise ham deres Medfølelse og trøste ham.
\par 12 Men da de i nogen Frastand så op og ikke kunde genkende ham, opløftede de deres Røst og græd, sønderrev alle tre deres Kapper og kastede Støv op over deres Hoveder.
\par 13 Så sad de på Jorden hos ham i syv Dage og syv Nætter, uden at nogen af dem mælede et Ord til ham; thi de så, at hans Lidelser var såre store.

\chapter{3}

\par 1 Derefter oplod Job sin Mund og forbandede sin Dag,
\par 2 og Job tog til Orde og sagde:
\par 3 Bort med den Dag, jeg fødtes, den Nat, der sagde: "Se, en Dreng!"
\par 4 Denne Dag vorde Mørke, Gud deroppe spørge ej om den, over den stråle ej Lyset frem!
\par 5 Mulm og Mørke løse den ind, Tåge lægge sig over den, Formørkelser skræmme den!
\par 6 Mørket tage den Nat, den høre ej hjemme blandt Årets Dage, den komme ikke i Måneders Tal!
\par 7 Ja, denne Nat vorde gold, der lyde ej Jubel i den!
\par 8 De, der besværger Dage, forbande den, de, der har lært at hidse Livjatan";
\par 9 dens Morgenstjerner formørkes, den bie forgæves på Lys, den skue ej Morgenrødens Øjenlåg,
\par 10 fordi den ej lukked mig Moderlivets Døre og skjulte Kvide for mit Blik!
\par 11 Hvi døde jeg ikke i Moders Liv eller udånded straks fra Moders Skød?
\par 12 Hvorfor var der Knæ til at tage imod mig, hvorfor var der Bryster at die?
\par 13 Så havde jeg nu ligget og hvilet, så havde jeg slumret i Fred
\par 14 blandt Konger og Jordens Styrere, der bygged sig Gravpaladser,
\par 15 blandt Fyrster, rige på Guld, som fyldte deres Huse med Sølv.
\par 16 Eller var jeg dog som et nedgravet Foster. som Børn, der ikke fik Lyset at se!
\par 17 Der larmer de gudløse ikke mer, der hviler de trætte ud,
\par 18 alle de fangne har Ro, de hører ej Fogedens Røst;
\par 19 små og store er lige der og Trællen fri for sin Herre.
\par 20 Hvi giver Gud de lidende Lys, de bittert sørgende Liv,
\par 21 dem, som bier forgæves på Døden, graver derefter som efter Skatte,
\par 22 som glæder sig til en Stenhøj, jubler, når de finder deres Grav
\par 23 en Mand, hvis Vej er skjult, hvem Gud har stænget inde?
\par 24 Thi Suk er blevet mit daglige Brød, mine Ve råb strømmer som Vand.
\par 25 Thi hvad jeg gruer for, rammer mig, hvad jeg bæver for, kommer over mig.
\par 26 Knap har jeg Fred, og knap har jeg Ro, knap har jeg Hvile, så kommer Uro!

\chapter{4}

\par 1 Så tog Temaniten Elifaz til Orde og sagde:
\par 2 Ærgrer det dig, om man taler til dig? Men hvem kan her være tavs?
\par 3 Du har selv talt mange til Rette og styrket de slappe Hænder,
\par 4 dine Ord holdt den segnende oppe, vaklende Knæ gav du Kraft.
\par 5 Men nu det gælder dig selv, så taber du Modet, nu det rammer dig selv, er du slaget af Skræk!
\par 6 Er ikke din Gudsfrygt din Tillid, din fromme Færd dit Håb?
\par 7 Tænk efter! Hvem gik uskyldig til Grunde, hvor gik retsindige under?
\par 8 Men det har jeg set: Hvo Uret pløjer og sår Fortræd, de høster det selv.
\par 9 For Guds Ånd går de til Grunde, for hans Vredes Pust går de til.
\par 10 Løvens Brøl og Vilddyrets Glam Ungløvernes Tænder slås ud;
\par 11 Løven omkommer af Mangel på Rov, og Løveungerne spredes.
\par 12 Der sneg sig til mig et Ord mit Øre opfanged dets Hvisken
\par 13 i Nattesynernes Tanker, da Dvale sank over Mennesker;
\par 14 Angst og Skælven kom over mig, alle mine Ledemod skjalv;
\par 15 et Pust strøg over mit Ansigt, Hårene rejste sig på min Krop.
\par 16 Så stod det stille! Jeg sansed ikke, hvordan det så ud; en Skikkelse stod for mit Øje, jeg hørte en hviskende Stemme:
\par 17 "Har et Menneske Ret for Gud, mon en Mand er ren for sin Skaber?
\par 18 End ikke sine Tjenere tror han, hos sine Engle finder han Fejl,
\par 19 endsige hos dem, der bor i en Hytte af Ler og har deres Grundvold i Støvet!
\par 20 De knuses ligesom Møl, imellem Morgen og Aften, de sønderslås uden at ænses, for evigt går de til Grunde.
\par 21 Rives ej deres Teltreb ud? De dør, men ikke i Visdom."

\chapter{5}

\par 1 Råb kun! Giver nogen dig Svar? Og til hvem af de Hellige vender du dig?
\par 2 Thi Dårens Harme koster ham Livet, Tåbens Vrede bliver hans Død.
\par 3 Selv har jeg set en Dåre rykkes op, hans Bolig rådne brat;
\par 4 hans Sønner var uden Hjælp, trådtes ned i Porten, ingen reddede dem;
\par 5 sultne åd deres Høst, de tog den, selv mellem Torne, og tørstige drak deres Mælk.
\par 6 Thi Vanheld vokser ej op af Støvet, Kvide spirer ej frem af Jorden,
\par 7 men Mennesket avler Kvide, og Gnisterne flyver til Vejrs.
\par 8 Nej, jeg vilde søge til Gud og lægge min Sag for ham,
\par 9 som øver ufattelig Vælde og Undere uden Tal,
\par 10 som giver Regn på Jorden og nedsender Vand over Marken
\par 11 for at løfte de bøjede højt, så de sørgende opnår Frelse,
\par 12 han, som krydser de kloges Tanker, så de ikke virker noget, der varer,
\par 13 som fanger de vise i deres Kløgt, så de listiges Råd er forhastet;
\par 14 i Mørke raver de, selv om Dagen, famler ved Middag, som var det Nat.
\par 15 Men han frelser den arme fra Sværdet og fattig af stærkes Hånd,
\par 16 så der bliver Håb for den ringe og Ondskaben lukker sin Mund.
\par 17 Held den Mand, som revses at Gud; ringeagt ej den Almægtiges Tugt!
\par 18 Thi han sårer, og han forbinder, han slår, og hans Hænder læger.
\par 19 Seks Gange redder han dig i Trængsel, syv går Ulykken uden om dig;
\par 20 han frier dig fra Døden i Hungersnød, i Krig fra Sværdets Vold;
\par 21 du er gemt for Tungens Svøbe, har intet at frygte, når Voldsdåd kommer;
\par 22 du ler ad Voldsdåd og Hungersnød og frygter ej Jordens vilde dyr;
\par 23 du har Pagt med Markens Sten, har Fred med Markens Vilddyr;
\par 24 du kender at have dit Telt i Fred, du mønstrer din Bolig, og intet fattes;
\par 25 du kender at have et talrigt Afkom, som Jordens Urter er dine Spirer;
\par 26 Graven når du i Ungdomskraft, som Neg føres op, når Tid er inde.
\par 27 Se, det har vi gransket, således er det; det har vi hørt, så vid også du det!

\chapter{6}

\par 1 Så tog Job til Orde og svarede:
\par 2 "Gid man vejed min Harme og vejed min Ulykke mod den!
\par 3 Thi tungere er den end Havets Sand, derfor talte jeg over mig!
\par 4 Thi i mig sidder den Almægtiges Pile, min Ånd inddrikker deres Gift; Rædsler fra Gud forvirrer mig.
\par 5 Skriger et Vildæsel midt i Græsset, brøler en Okse ved sit Foder?
\par 6 Spiser man ferskt uden Salt, smager mon Æggehvide godt?
\par 7 Min Sjæl vil ej røre derved, de Ting er som Lugt af en Løve.
\par 8 Ak, blev mit Ønske dog opfyldt, Gud give mig det, som jeg håber
\par 9 vilde d dog knuse mig, række Hånden ud og skære mig fra,
\par 10 så vilde det være min Trøst - jeg hopped af Glæde trods skånselsløs Kval at jeg ikke har nægtet den Helliges Ord.
\par 11 Hvad er min Kraft, at jeg skal holde ud, min Udgang, at jeg skal være tålmodig?
\par 12 Er da min Kraft som Stenens, er da mit Legeme Kobber?
\par 13 Ak, for mig er der ingen Hjælp, hver Udvej lukker sig for mig.
\par 14 Den, der nægter sin Næste Godhed, han bryder med den Almægtiges Frygt.
\par 15 Mine Brødre sveg mig som en Bæk, som Strømme, hvis Vand svandt bort,
\par 16 de, der var grumset af os, og som Sneen gemte sig i,
\par 17 men som svandt ved Solens Glød, tørredes sporløst ud i Hede;
\par 18 Karavaner bøjer af fra Vejen, drager op i Ørkenen og går til Grunde;
\par 19 Temas Karavaner spejder, Sabas Rejsetog håber på dem,
\par 20 men de beskæmmes i deres Tillid, de kommer derhen og skuffes!
\par 21 Ja, slige Strømme er I mig nu, Rædselen så I og grebes af Skræk!
\par 22 Har jeg mon sagt: "Giv mig Gaver, løs mig med eders Velstand,
\par 23 red mig af Fjendens Hånd, køb mig fri fra Voldsmænds Hånd!"
\par 24 Lær mig, så vil jeg tie, vis mig, hvor jeg har fejlet!
\par 25 Redelig Tale, se, den gør Indtryk; men eders Revselse, hvad er den værd?
\par 26 Er det jer Hensigt at revse Ord? Den fortvivledes Ord er dog Mundsvejr!
\par 27 Selv om en faderløs kasted I Lod og købslog om eders Ven.
\par 28 Men vilde I nu dog se på mig! Mon jeg lyver jer op i Ansigtet?
\par 29 Vend jer hid, lad der ikke ske Uret, vend jer, thi end har jeg Ret!
\par 30 Er der Uret på min Tunge, eller skelner min Gane ej, hvad der er ondt?

\chapter{7}

\par 1 Har Mennesket på Jord ej Krigerkår? Som en Daglejers er hans Dage.
\par 2 Som Trællen, der higer efter Skygge som Daglejeren, der venter på Løn,
\par 3 så fik jeg Skuffelses Måneder i Arv kvalfulde Nætter til Del.
\par 4 Når jeg lægger mig, siger jeg: "Hvornår er det Dag, af jeg kan stå op?" og når jeg står op: "Hvornår er det Kvæld?" Jeg mættes af Uro, til Dagen gryr.
\par 5 Mit Legeme er klædt med Orme og Skorpe, min Hud skrumper ind og væsker.
\par 6 Raskere end Skyttelen flyver mine Dage, de svinder bort uden Håb.
\par 7 Kom i Hu, at mit Liv er et Pust, ej mer får mit Øje Lykke at skue!
\par 8 Vennens Øje skal ikke se mig, dit Øje søger mig - jeg er ikke mere.
\par 9 Som Skyen svinder og trækker bort, bliver den, der synker i Døden, borte,
\par 10 han vender ej atter hjem til sit Hus, hans Sted får ham aldrig at se igen.
\par 11 Så vil jeg da ej lægge Bånd på min Mund, men tale i Åndens Kvide, sukke i bitter Sjælenød.
\par 12 Er jeg et Hav, eller er jeg en Drage, siden du sætter Vagt ved mig?
\par 13 Når jeg tænker, mit Leje skal lindre mig, Sengen lette mit Suk,
\par 14 da ængster du mig med Drømme, skræmmer mig op ved Syner,
\par 15 så min Sjæl vil hellere kvæles. hellere dø end lide.
\par 16 Nu nok! Jeg lever ej evigt, slip mig, mit Liv er et Pust!
\par 17 Hvad er et Menneske, at du regner ham og lægger Mærke til ham,
\par 18 hjemsøger ham hver Morgen, ransager ham hvert Øjeblik?
\par 19 Når vender du dog dit Øje fra mig, slipper mig, til jeg har sunket mit Spyt?
\par 20 Har jeg syndet, hvad skader det dig, du, som er Menneskets Vogter? Hvi gjorde du mig til Skive, hvorfor blev jeg dig til Byrde?
\par 21 Hvorfor tilgiver du ikke min Synd og lader min Brøde uænset? Snart ligger jeg jo under Mulde, du søger mig - og jeg er ikke mere!

\chapter{8}

\par 1 Så tog Sjuhiten Bildad til Orde og sagde:
\par 2 "Hvor længe taler du så, hvor længe skal Mundens Uvejr rase?
\par 3 Mon Gud vel bøjer Retten, bøjer den Almægtige Retfærd?
\par 4 Har dine Sønner syndet imod ham, og gav han dem deres Brøde i Vold,
\par 5 så søg du nu hen til Gud og bed hans Almagt om Nåde!
\par 6 Såfremt du er ren og oprigtig, ja, da vil han våge over dig, genrejse din Retfærds Bolig;
\par 7 din fordums Lykke vil synes ringe, såre stor skal din Fremtid blive.
\par 8 Thi spørg dog den befarne Slægt, læg Mærke til Fædrenes Granskning!
\par 9 Vi er fra i Går, og intet ved vi, en Skygge er vore Dage på Jord.
\par 10 Mon ej de kan lære dig, sige dig det og give dig Svar af Hjertet:
\par 11 Vokser der Siv, hvor der ikke er Sump, gror Nilgræs frem, hvor der ikke er Vand?
\par 12 Endnu i Grøde, uden at høstes, visner det før alt andet Græs.
\par 13 Så går det enhver, der glemmer Gud, en vanhelliges Håb slår fejl:
\par 14 som Sommerspind er hans Tilflugt, hans Tillid er Spindelvæv;
\par 15 han støtter sig til sit Hus, det falder, han klynger sig til det, ej står det fast.
\par 16 I Solskinnet vokser han frodigt, hans Ranker breder sig Haven over,
\par 17 i Stendynger fletter hans Rødder sig ind, han hager sig fast mellem Sten;
\par 18 men rives han bort fra sit Sted, fornægter det ham: "Jeg har ikke set dig!"
\par 19 Se, det er Glæden, han har af sin Vej, og af Jorden fremspirer en anden!
\par 20 Se, Gud agter ej den uskyldige ringe, han holder ej fast ved de ondes Hånd.
\par 21 End skal han fylde din Mund med Latter og dine Læber med Jubel;
\par 22 dine Avindsmænd skal klædes i Skam og gudløses Telt ej findes mer!

\chapter{9}

\par 1 Så tog Job til Orde og svarede:
\par 2 "Jeg ved forvist, at således er det, hvad Ret har en dødelig over for Gud?
\par 3 Vilde Gud gå i Rette med ham, kan han ikke svare på et af tusind!
\par 4 Viis af Hjerte og vældig i Kraft hvo trodsede ham og slap vel derfra?
\par 5 Han flytter Bjerge så let som intet, vælter dem om i sin Vrede,
\par 6 ryster Jorden ud af dens Fuger, så dens Grundstøtter bæver;
\par 7 han taler til solen, så skinner den ikke, for Stjernerne sætter han Segl,
\par 8 han udspænder Himlen ene, skrider hen over Havets Kamme,
\par 9 han skabte Bjørnen, Orion, Syvstjernen og Sydens Kamre,
\par 10 han øver ufattelig Vælde og Undere uden Tal!
\par 11 Går han forbi mig, ser jeg ham ikke, farer han hen, jeg mærker ham ikke;
\par 12 røver han, hvem mon der hindrer ham i det? Hvo siger til ham: "Hvad gør du?"
\par 13 Gud lægger ikke Bånd på sin Vrede, Rahabs Hjælpere bøjed sig under ham;
\par 14 hvor kan jeg da give ham Svar og rettelig føje min Tale for ham!
\par 15 Har jeg end Ret, jeg kan dog ej svare, må bede min Dommer om Nåde!
\par 16 Nævned jeg ham, han svared mig ikke, han hørte, tror jeg, ikke min Røst,
\par 17 han, som river mig bort i Stormen, giver mig - Sår på Sår uden Grund,
\par 18 ikke lader mig drage Ånde, men lader mig mættes med beskeing.
\par 19 Gælder det Kæmpekraft, melder han sig! Gælder det Ret, hvo stævner ham da!
\par 20 Har jeg end Ret, må min Mund dog fælde mig, er jeg end skyldfri, han gør mig dog vrang!
\par 21 Skyldfri er jeg, ser bort fra min Sjæl og agter mit Liv for intet!
\par 22 Lige meget; jeg påstår derfor: Skyldfri og skyldig gør han til intet!
\par 23 Når Svøben kommer med Død i et Nu, så spotter han skyldfries Hjertekval;
\par 24 Jorden gav han i gudløses Hånd, hylder dens Dommeres Øjne til, hvem ellers, om ikke han?
\par 25 Raskere end Løberen fløj mine Dage, de svandt og så ikke Lykke,
\par 26 gled hen som Både af Si, som en Ørn, der slår ned på Bytte.
\par 27 Dersom jeg siger: "Mit Suk vil jeg glemme, glatte mit Ansigt og være glad,"
\par 28 må jeg dog grue for al min Smerte, jeg ved, du kender mig ikke fri.
\par 29 Jeg skal nu engang være skyldig, hvorfor da slide til ingen Nytte?
\par 30 Toed jeg mig i Sne og tvætted i Lud mine Hænder,
\par 31 du dypped mig dog i Pølen, så Klæderne væmmedes ved mig.
\par 32 Thi du er ikke en Mand som jeg, så jeg kunde svare, så vi kunde gå for Retten sammen;
\par 33 vi savner en Voldgiftsmand til at lægge sin Hånd på os begge!
\par 34 Fried han mig for sin Stok, og skræmmed hans Rædsler mig ikke,
\par 35 da talte jeg uden at frygte ham,, thi min Dom om mig selv er en anden!

\chapter{10}

\par 1 Min Sjæl er led ved mit Liv, frit Løb vil jeg give min Klage over ham, i min bitre Sjælenød vil jeg tale,
\par 2 sige til Gud: Fordøm mig dog ikke, lad mig vide, hvorfor du tvister med mig!
\par 3 Gavner det dig at øve Vold, at forkaste det Værk, dine Hænder danned, men smile til gudløses Råd?
\par 4 Har du da kødets Øjne, ser du, som Mennesker ser,
\par 5 er dine Dage som Menneskets Dage, er dine År som Mandens Dage,
\par 6 siden du søger efter min Brøde, leder efter min Synd,
\par 7 endskønt du ved, jeg ikke er skyldig; men af din Hånd er der ingen Redning!
\par 8 Dine Hænder gjorde og danned mig først, så skifter du Sind og gør mig til intet!
\par 9 Kom i Hu, at du dannede mig som Ler, og til Støv vil du atter gøre mig!
\par 10 Mon du ikke hældte mig ud som Mælk og lod mig skørne som Ost,
\par 11 iklædte mig Hud og kød og fletted mig sammen med Ben og Sener?
\par 12 Du gav mig Liv og Livskraft, din Omhu vogted min Ånd
\par 13 og så gemte du dog i dit Hjerte på dette, jeg skønner, dit Øjemed var:
\par 14 Synded jeg, vogted du på mig og tilgav ikke min Brøde.
\par 15 Fald jeg forbrød mig, da ve mig! Var jeg retfærdig, jeg skulde dog ikke løfte mit Hoved, men mættes med Skændsel, kvæges med Nød.
\par 16 Knejsed jeg, jog du mig som en Løve, handlede atter ufatteligt med mig;
\par 17 nye Vidner førte du mod mig, øged din Uvilje mod mig, opbød atter en Hær imod mig!
\par 18 Hvi drog du mig da af Moders Liv? Jeg burde have udåndet, uset af alle;
\par 19 jeg burde have været som aldrig født, været ført til Graven fra Moders Skød.
\par 20 Er ej mine Livsdage få? Så slip mig, at jeg kan kvæges lidt,
\par 21 før jeg for evigt går bort til Mørkets og Mulmets Land,
\par 22 Landet med bælgmørkt Mulm, med Mørke og uden Orden, hvor Lyset selv er som Mørket."

\chapter{11}

\par 1 Så tog Na'amatiten Zofar til Orde og sagde:
\par 2 "Skal en Ordgyder ej have Svar, skal en Mundheld vel have Ret?
\par 3 Skal Mænd vel tie til din Skvalder, skal du spotte og ikke få Skam?
\par 4 Du siger: "Min Færd er lydeløs, og jeg er ren i hans Øjne!"
\par 5 Men vilde dog Gud kun tale, oplade sine Læber imod dig,
\par 6 kundgøre dig Visdommens Løndom, thi underfuld er den i Væsen; da vilde du vide, at Gud har glemt dig en Del af din Skyld!
\par 7 Har du loddet Bunden i Gud og nået den Almægtiges Grænse?
\par 8 Højere er den end Himlen hvad kan du? Dybere end Dødsriget - hvad ved du?
\par 9 Den overgår Jorden i Vidde, er mere vidtstrakt end Havet.
\par 10 Farer han frem og fængsler, stævner til Doms, hvem hindrer ham?
\par 11 Han kender jo Løgnens Mænd, Uret ser han og agter derpå,
\par 12 så tomhjernet Mand får Vid, og Vildæsel fødes til Menneske.
\par 13 Hvis du får Skik på dit Hjerte og breder dine Hænder imod ham,
\par 14 hvis Uret er fjern fra din Hånd, og Brøde ej bor i dit Telt,
\par 15 ja, da kan du lydefri løfte dit Åsyn og uden at frygte stå fast,
\par 16 ja, da skal du glemme din Kvide, mindes den kun som Vand, der flød bort;
\par 17 dit Liv skal overstråle Middagssolen, Mørket vorde som lyse Morgen.
\par 18 Tryg skal du være, fordi du har Håb; du ser dig om og går trygt til Hvile,
\par 19 du ligger uden at skræmmes op. Til din Yndest vil mange bejle.
\par 20 Men de gudløses Øjne vansmægter; ude er det med deres Tilflugt, deres Håb er blot at udånde Sjælen!

\chapter{12}

\par 1 Så tog Job til Orde og svarede:
\par 2 "Ja, sandelig, I er de rette, med eder dør Visdommen ud!
\par 3 Også jeg har som I Forstand, står ikke tilbage for eder, hvo kender vel ikke sligt?
\par 4 Til Latter for Venner er den, der råbte til Gud og fik Svar. den retfærdige er til Latter.
\par 5 I Ulykke falder de fromme, den sorgløse spotter Faren, hans Fod står fast, mens Fristen varer.
\par 6 I Fred er Voldsmænds Telte, og trygge er de, der vækker Guds Vrede, den, der fører Gud i sin Hånd.
\par 7 Spørg dog Kvæget, det skal lære dig, Himlens Fugle, de skal oplyse dig,
\par 8 se til Jorden, den skal lære dig lad Havets Fisk fortælle dig det!
\par 9 Hvem blandt dem alle ved vel ikke, at HERRENs Hånd har skabt det;
\par 10 han holder alt levendes Sjæl i sin Hånd, alt Menneskekødets Ånd!
\par 11 Prøver ej Øret Ord, og smager ej Ganen Maden?
\par 12 Er Alderdom eet med Visdom, Dagenes Række med Indsigt?
\par 13 Hos ham er der Visdom og Vælde, hos ham er der Råd og Indsigt.
\par 14 Hvad han river ned, det bygges ej op, den, han lukker inde, kommer ej ud;
\par 15 han dæmmer for Vandet, og Tørke kommer, han slipper det løs, og det omvælter Jorden.
\par 16 Hos ham er der Kraft og Fasthed; den, der farer og fører vild, er hans Værk.
\par 17 Rådsherrer fører han nøgne bort, og Dommere gør han til Tåber;
\par 18 han løser, hvad Konger bandt, og binder dem Reb om Lænd;
\par 19 Præster fører han nøgne bort og styrter ældgamle Slægter;
\par 20 han røver de dygtige Mælet og tager de gamles Sans;
\par 21 han udøser Hån over Fyrster og løser de stærkes Bælte;
\par 22 han drager det skjulte frem af Mørket og bringer Mulmet for Lyset,
\par 23 gør Folkene store og lægger dem øde, udvider Folkeslags Grænser og fører dem atter bort;
\par 24 han tager Jordens Høvdingers Vid og lader dem rave i vejløst Øde;
\par 25 de famler i Mørke uden Lys og raver omkring som drukne.

\chapter{13}

\par 1 Se, mit Øje har skuet alt dette, mit Øre har hørt og mærket sig det;
\par 2 hvad I ved, ved også jeg, jeg falder ikke igennem for jer.
\par 3 Men til den Almægtige vil jeg tale, med Gud er jeg sindet at gå i Rette,
\par 4 mens I smører på med Løgn; usle Læger er I til Hobe.
\par 5 Om I dog vilde tie stille, så kunde I regnes for vise!
\par 6 Hør dog mit Klagemål, mærk mine Læbers Anklage!
\par 7 Forsvarer I Gud med Uret, forsvarer I ham med Svig?
\par 8 Vil I tage Parti for ham, vil I træde i Skranken for Gud?
\par 9 Går det godt, når han ransager eder, kan I narre ham, som man narrer et Menneske?
\par 10 Revse jer vil han alvorligt, om I lader som intet og dog er partiske.
\par 11 Vil ikke hans Højhed skræmme jer og hans Rædsel falde på eder?
\par 12 Eders Tankesprog bliver til Askesprog, som Skjolde af Ler eders Skjolde.
\par 13 Ti stille, at jeg kan tale, så overgå mig, hvad der vil!
\par 14 Jeg vil bære mit Kød i Tænderne og tage mit Liv i min Hånd;
\par 15 se, han slår mig ihjel, jeg har intet Håb, dog lægger jeg for ham min Færd.
\par 16 Det er i sig selv en Sejr for mig, thi en vanhellig vover sig ikke til ham!
\par 17 Hør nu ret på mit Ord, lad mig tale for eders Ører!
\par 18 Se, til Rettergang er jeg rede, jeg ved, at Retten er min!
\par 19 Hvem kan vel trætte med mig? Da skulde jeg tie og opgive Ånden!
\par 20 Kun for to Ting skåne du mig, så kryber jeg ikke i Skjul for dig:
\par 21 Din Hånd må du tage fra mig, din Rædsel skræmme mig ikke!
\par 22 Så stævn mig, og jeg skal svare, eller jeg vil tale, og du skal svare!
\par 23 Hvor stor er min Skyld og Synd? Lad mig vide min Brøde og Synd!
\par 24 Hvi skjuler du dog dit Åsyn og regner mig for din Fjende?
\par 25 Vil du skræmme et henvejret Blad, forfølge et vissent Strå,
\par 26 at du skriver mig så bitter en Dom og lader mig arve min Ungdoms Skyld,
\par 27 lægger mine Fødder i Blokken, vogter på alle mine Veje. indkredser mine Fødders Trin!
\par 28 Og så er han dog som smuldrende Trøske, som Klæder, der ædes op af Møl,

\chapter{14}

\par 1 Mennesket, født af en Kvinde, hans Liv er stakket, han mættes af Uro;
\par 2 han spirer som Blomsten og visner, flyr som Skyggen, står ikke fast.
\par 3 Og på ham vil du rette dit Øje, ham vil du stævne for Retten!
\par 4 Ja, kunde der komme en ren af en uren! Nej, end ikke een!
\par 5 Når hans Dages Tal er fastsat, hans Måneder talt hos dig, og du har sat ham en uoverskridelig Grænse,
\par 6 tag så dit Øje fra ham, lad ham i Fred, at han kan nyde sin Dag som en Daglejer!
\par 7 Thi for et Træ er der Håb: Fældes det, skyder det atter, det fattes ej nye Skud;
\par 8 ældes end Roden i Jorden, dør end Stubben i Mulde:
\par 9 lugter det Vand, får det nye Skud, skyder Grene som nyplantet Træ;
\par 10 men dør en Mand, er det ude med ham, udånder Mennesket, hvor er han da?
\par 11 Som Vand løber ud af Søen og Floden svinder og tørres,
\par 12 så lægger Manden sig, rejser sig ikke, vågner ikke, før Himlen forgår, aldrig vækkes han af sin Søvn.
\par 13 Tag dog og gem mig i Dødens Rige, skjul mig, indtil din Vrede er ovre, sæt mig en Frist og kom mig i Hu!
\par 14 Om Manden dog døde for atter at leve! Da vented jeg rolig al Stridens Tid, indtil min Afløsning kom;
\par 15 du skulde kalde - og jeg skulde svare længes imod dine Hænders Værk!
\par 16 Derimod tæller du nu mine Skridt, du tilgiver ikke min Synd,
\par 17 forseglet ligger min Brøde i Posen, og over min Skyld har du lukket til.
\par 18 Nej, ligesom Bjerget skrider og falder, som Klippen rokkes fra Grunden,
\par 19 som Vandet udhuler Sten og Plaskregn bortskyller Jord, så har du udslukt Menneskets Håb.
\par 20 For evigt slår du ham ned, han går bort, skamskænder hans Ansigt og lader ham fare.
\par 21 Hans Sønner hædres, han ved det ikke, de synker i Ringhed, han mærker det ikke;
\par 22 ikkun hans eget Kød volder Smerte, ikkun hans egen Sjæl volder Sorg.

\chapter{15}

\par 1 Så tog Temaniten Elifaz til Orde og sagde:
\par 2 "Mon Vismand svarer med Mundsvejr og fylder sit Indre med Østenvind
\par 3 for at hævde sin Ret med gavnløs Tale, med Ord, som intet båder?
\par 4 Desuden nedbryder du Gudsfrygt og krænker den Stilhed, som tilkommer Gud.
\par 5 Din Skyld oplærer din Mund, du vælger de listiges Sprog.
\par 6 Din Mund domfælder dig, ikke jeg, dine Læber vidner imod dig!
\par 7 Var du den første, der fødtes, kom du til Verden, før Højene var?
\par 8 Mon du lytted til, da Gud holdt Råd, og mon du rev Visdommen til dig?
\par 9 Hvad ved du, som vi ikke ved, hvad forstår du, som vi ikke kender?
\par 10 Også vi har en gammel iblandt os, en Olding, hvis Dage er fler end din Faders!
\par 11 Er Guds Trøst dig for lidt, det Ord, han mildelig talede til dig?
\par 12 Hvi river dit Hjerte dig hen, hvi ruller dit Øje vildt?
\par 13 Thi du vender din Harme mod Gud og udstøder Ord af din Mund.
\par 14 Hvor kan et Menneske være rent, en kvindefødt have Ret?
\par 15 End ikke sine Hellige tror han, og Himlen er ikke ren i hans Øjne,
\par 16 hvad da den stygge, den onde, Manden, der drikker Uret som Vand!
\par 17 Jeg vil sige dig noget, hør mig, jeg fortæller, hvad jeg har set,
\par 18 hvad vise Mænd har forkyndt, deres Fædre ikke dulgt,
\par 19 dem alene var Landet givet, ingen fremmed færdedes blandt dem:
\par 20 Den gudløse ængstes hele sit Liv, de stakkede År, en Voldsmand lever;
\par 21 Rædselslyde fylder hans Ører, midt under Fred er Hærgeren over ham;
\par 22 han undkommer ikke fra Mørket, opsparet er han for Sværdet,
\par 23 udset til Føde for Gribbe, han ved, at han står for Fald;
\par 24 Mørkets Dag vil skræmme ham. Trængsel og Angst overvælde ham som en Konge, rustet til Strid.
\par 25 Thi Hånden rakte han ud mod Gud og bød den Almægtige Trods,
\par 26 stormed bårdnakket mod ham med sine tykke, buede Skjolde.
\par 27 Thi han dækked sit Ansigt med Fedt og samlede Huld på sin Lænd.
\par 28 tog Bolig i Byer, der øde lå hen. i Huse, man ikke må bo i, bestemt til at ligge i Grus.
\par 29 Han bliver ej rig, hans Velstand forgår, til Jorden bøjer sig ikke hans Aks;
\par 30 han undkommer ikke fra Mørket. Solglød udtørrer hans Spire, hans Blomst rives bort af Vinden.
\par 31 Han stole ikke på Tomhed han farer vild thi Tomhed skal være hans Løn!
\par 32 I Utide visner hans Stamme, hans Palmegren skal ikke grønnes;
\par 33 han ryster som Ranken sin brue af og kaster som Olietræet sin Blomst.
\par 34 Thi vanhelliges Samfund er goldt, og Ild fortærer Bestikkelsens Telte;
\par 35 svangre med Kvide, føder de Uret, og deres Moderskød fostrer Svig!

\chapter{16}

\par 1 Så tog Job til Orde og svarede:
\par 2 "Nok har jeg hørt af sligt, besværlige Trøstere er I til Hobe!
\par 3 Får Mundsvejret aldrig Ende? Hvad ægged dig dog til at svare?
\par 4 Også jeg kunde tale som I, hvis I kun var i mit Sted, føje mine Ord imod jer og ryste på Hovedet ad jer,
\par 5 styrke jer med min Mund, ej spare på ynksomme Ord!
\par 6 Taler jeg, mildnes min Smerte ikke og om jeg tier, hvad Lindring får jeg?
\par 7 Dog nu har han udtømt min Kraft, du bar ødelagt hele min Kreds;
\par 8 at du greb mig, gælder som Vidnesbyrd mod mig, min Magerhed vidner imod mig.
\par 9 Hans Vrede river og slider i mig, han skærer Tænder imod mig. Fjenderne hvæsser Blikket imod mig,
\par 10 de opspiler Gabet imod mig, slår mig med Hån på Kind og flokkes til Hobe omkring mig;
\par 11 Gud gav mig hen i Niddingers Vold, i gudløses Hænder kasted han mig.
\par 12 Jeg leved i Fred, så knuste han mig, han greb mig i Nakken og sønderslog mig; han stilled mig op som Skive,
\par 13 hans Pile flyver omkring mig, han borer i Nyrerne uden Skånsel, udgyder min Galde på Jorden;
\par 14 Revne på Revne slår han mig, stormer som Kriger imod mig.
\par 15 Over min Hud har jeg syet Sæk og boret mit Horn i Støvel;
\par 16 mit Ansigt er rødt af Gråd, mine Øjenlåg hyllet i Mørke,
\par 17 skønt der ikke er Vold i min Hånd, og skønt min Bøn er ren!
\par 18 Dølg ikke, Jord, mit Blod, mit Skrig komme ikke til Hvile!
\par 19 Alt nu er mit Vidne i Himlen, min Talsmand er i det høje;
\par 20 gid min Ven lod sig finde! Mit Øje vender sig med Tårer til Gud,
\par 21 at han skifter Ret mellem Manden og Gud, mellem Mennesket og hans Ven!
\par 22 Thi talte er de kommende År, jeg skal ud på en Færd, jeg ej vender hjem fra.

\chapter{17}

\par 1 Brudt er min Ånd, mine Dage slukt, og Gravene venter mig;
\par 2 visselig, Spot er min Del, og bittert er, hvad mit Øje må skue.
\par 3 Stil Sikkerhed for mig hos dig! Hvem anden giver mig Håndslag?
\par 4 Thi du lukked deres Hjerte for Indsigt, derfor vil du ikke ophøje dem;
\par 5 den, der forråder Venner til Plyndring, hans Sønners Øjne hentæres.
\par 6 Til Mundheld har du gjort mig for Folk, jeg er blevet et Jærtegn for dem;
\par 7 mit Øje er sløvet af Kvide, som Skygger er mine Lemmer til Hobe;
\par 8 retsindige stivner af Rædsel ved sligt, over vanhellig harmes den skyldfri,
\par 9 men den retfærdige holder sin Vej, en renhåndet vokser i Kraft.
\par 10 Men I, mød kun alle frem igen, en Vismand fnder jeg ikke iblandt jer!
\par 11 Mine Dage stunder mod Døden, brudt er mit Hjertes Ønsker;
\par 12 Natten gør jeg til Dag, Lyset for mig er Mørke;
\par 13 vil jeg håbe, får jeg dog Bolig i Døden, jeg reder i Mørket mit Leje,
\par 14 Graven kalder jeg Fader, Forrådnelsen Moder og Søster.
\par 15 Hvor er da vel mit Håb, og hvo kan øjne min Lykke?
\par 16 Mon de vil følge mig ned i Dødsriget, skal sammen vi synke i Støvet?

\chapter{18}

\par 1 Så tog Sjuhiten Bildad til Orde og sagde:
\par 2 "Så gør dog en Ende på dine Ord, kom til Fornuft og lad os tale!
\par 3 Hvi skal vi regnes for Kvæg og stå som umælende i dine Øjne?
\par 4 Du, som i Vrede sønderslider din Sjæl, skal for din Skyld Jorden blive øde og Klippen flyttes fra sit Sted?
\par 5 Nej, den gudløses Lys bliver slukt, hans Ildslue giver ej Lys;
\par 6 Lyset i hans Telt går ud, og hans Lampe slukkes for ham;
\par 7 hans kraftige Skridt bliver korte, han falder for eget Råd;
\par 8 thi hans Fod drives ind i Nettet, på Fletværk vandrer han frem,
\par 9 Fælden griber om Hælen, Garnet holder ham fast;
\par 10 Snaren er skjult i Jorden for ham og Saksen på hans Sti;
\par 11 Rædsler skræmmer ham alle Vegne og kyser ham Skridt for Skridt:
\par 12 Ulykken hungrer efter ham, Undergang lurer på hans Fald:
\par 13 Dødens førstefødte æder hans Lemmer, æder hans Legemes Lemmer;
\par 14 han rives bort fra sit Telt, sin Fortrøstning; den styrer hans Skridt til Rædslernes Konge;
\par 15 i hans Telt har Undergang hjemme, Svovl strøs ud på hans Bolig;
\par 16 nedentil tørrer hans Rødder, oventil visner hans Grene;
\par 17 hans Minde svinder fra Jord, på Gaden nævnes ikke hans Navn;
\par 18 man støder ham ud fra Lys i Mørket og driver ham bort fra Jorderig;
\par 19 i sit Folk har han ikke Afkom og Æt, i hans Hjem er der ingen tilbage;
\par 20 de i Vester stivner ved hans Skæbnedag, de i Øst bliver slagne af Rædsel.
\par 21 Ja, således går det den lovløses Bolig, dens Hjem, der ej kender Gud!

\chapter{19}

\par 1 Så tog Job til Orde og svarede:
\par 2 "Hvor længe vil I krænke min Sjæl og slå mig sønder med Ord?
\par 3 I håner mig nu for tiende Gang, mishandler mig uden Skam.
\par 4 Har jeg da virkelig fejlet, hænger der Fejl ved mig?
\par 5 Eller gør I jer store imod mig og revser mig ved at smæde?
\par 6 Så vid da, at Gud har bøjet min Ret, omspændt mig med sit Net.
\par 7 Se, jeg skriger: Vold! men får ikke Svar, råber om Hjælp, der er ingen Ret.
\par 8 Han spærred min Vej, jeg kom ikke frem, han hylled mine Stier i Mørke;
\par 9 han klædte mig af for min Ære, berøved mit Hoved Kronen,
\par 10 brød mig ned overalt, så jeg må bort, oprykked mit Håb som Træet;
\par 11 hans Vrede blussede mod mig, han regner mig for sin Fjende;
\par 12 samlede rykker hans Flokke frem og bryder sig Vej imod mig, de lejrer sig om mit Telt.
\par 13 Mine Brødre har fjernet sig fra mig, Venner er fremmede for mig,
\par 14 mine nærmeste og Hendinge holder sig fra mig, de, der er i mit Hus, har glemt mig;
\par 15 mine Piger regner mig for en fremmed, vildfremmed er jeg i deres Øjne;
\par 16 ej svarer min Træl, når jeg kalder, jeg må trygle ham med min Mund;
\par 17 ved min Ånde væmmes min Hustru, mine egne Brødre er jeg en Stank;
\par 18 selv Drenge agter mig ringe, når jeg reljser mig, taler de mod mig;
\par 19 Standsfælleræmmes til Hobe ved mig, de, jeg elskede, vender sig mod mig.
\par 20 Benene hænger fast ved min Hud, med Kødet i Tænderne slap jeg bort.
\par 21 Nåde, mine Venner, Nåde, thi Guds Hånd har rørt mig!
\par 22 Hvi forfølger og I mig som Gud og mættes ej af mit Kød?
\par 23 Ak, gid mine Ord blev skrevet op, blev tegnet op i en Bog,
\par 24 med Griffel af Jern, med Bly indristet i Hlippen for evigt!
\par 25 Men jeg ved, at min Løser lever, over Støvet vil en Forsvarer stå frem.
\par 26 Når min sønderslidte Hud er borte, skal jeg ud fra mit Kød skue Gud,
\par 27 hvem jeg skal se på min Side; ham skal mine Øjne se, ingen fremmed! Mine Nyrer forgår i mit Indre!
\par 28 Når I siger: "Hor vi skal forfølge ham, Sagens Rod vil vi udfinde hos ham!"
\par 29 så tag jer i Vare for Sværdet; thi Vrede rammer de lovløse, at I skal kende, der kommer en Dom!

\chapter{20}

\par 1 Så tog Na'amatiten Zofar til Orde og sagde
\par 2 "Derfor bruser Tankerne i mig, og derfor stormer det i mig;
\par 3 til min Skam må jeg høre på Tugt, får tankeløst Mundsvejr til Svar!
\par 4 Ved du da ikke fra Arilds Tid, fra Tiden, da Mennesket sattes på Jorden,
\par 5 at gudløses Jubel er kort og vanhelliges Glæde stakket?
\par 6 Steg end hans Hovmod til Himlen, raged hans Hoved i Sky,
\par 7 som sit Skarn forgår han for evigt, de, der så ham, siger: "Hvor er han?"
\par 8 Han flyr som en Drøm, man finder ham ikke, som et Nattesyn jages han bort;
\par 9 Øjet, der så ham, ser ham ej mer, hans Sted får ham aldrig at se igen.
\par 10 Hans Sønner bejler til ringes Yndest, hans Hænder må give hans Gods tilbage.
\par 11 Hans Ben var fulde af Ungdomskraft, men den lægger sig med ham i Støvet.
\par 12 Er det onde end sødt i hans Mund, når han gemmer det under sin Tunge,
\par 13 sparer på det og slipper det ikke, holder det fast til sin Gane,
\par 14 så bliver dog Maden i hans Indre til Slangegift inden i ham;
\par 15 Godset, han slugte, må han spy ud, Gud driver det ud af hans Bug,
\par 16 han indsuger Slangernes Gift, og Øgleungen slår ham ihjel;
\par 17 han skuer ej Strømme af Olie, Bække af Honning og Fløde;
\par 18 han må af med sin Vinding, svælger den ej, får ingen Glæde af tilbyttet Gods.
\par 19 Thi han knuste de ringe og lod dem ligge, ranede Huse, han ej havde bygget.
\par 20 Thi han har ingen Hjælp af sin Rigdom, trods sine Skatte reddes han ikke;
\par 21 ingen gik fri for hans Glubskhed, derfor varer hans Lykke ikke;
\par 22 midt i sin Overflod har han det trangt, al Slags Nød kommer over ham.
\par 23 For at fylde hans Bug sender Gud sin Vredes Glød imod ham, lader sin Harme regne på ham.
\par 24 Flyr han for Brynje af Jern, så gennemborer ham Kobberbuen;
\par 25 en Kni kommer ud af hans Ryg, et lynende Stål af hans Galde; over ham falder Rædsler,
\par 26 idel Mørke er opsparet til ham; Ild, der ej blæses op, fortærer ham, æder Levningen i hans Telt.
\par 27 Himlen bringer hans Brøde for Lyset, og Jorden rejser sig mod ham.
\par 28 Hans Huses Vinding må bort, rives bort på Guds Vredes Dag.
\par 29 Slig er den gudløses Lod fra Gud og Lønnen fra Gud for hans Brøde!

\chapter{21}

\par 1 Så tog Job til Orde og svarede:
\par 2 "Hør dog, hør mine Ord, lad det være Trøsten, I giver!
\par 3 Find jer nu i, at jeg taler, siden kan I jo håne!
\par 4 Gælder min Klage Mennesker? Hvi skulde jeg ej være utålmodig?
\par 5 Vend jer til mig og stivn af Rædsel, læg Hånd på Mund!
\par 6 Jeg gruer, når jeg tænker derpå, mit Legeme gribes af Skælven:
\par 7 De gudløse, hvorfor lever de, bliver gamle, ja vokser i Kraft?
\par 8 Deres Æt har de blivende hos sig, deres Afkom for deres Øjne;
\par 9 deres Huse er sikre mod Rædsler, Guds Svøbe rammer dem ikke;
\par 10 ej springer deres Tyr forgæves, Koen kælver, den kaster ikke;
\par 11 de slipper deres Drenge ud som Får, deres Børneflok boltrer sig ret;
\par 12 de synger til Pauke og Citer, er glade til Fløjtens Toner;
\par 13 de lever deres Dage i Lykke og synker med Fred i Dødsriget,
\par 14 skønt de siger til Gud: "Gå fra os, at kende dine Veje er ikke vor Lyst!
\par 15 Den Almægtige? Hvad han? Skal vi tjene ham? Hvad Gavn at banke på hos ham?"
\par 16 Er ej deres Lykke i deres Hånd og gudløses Råd ham fjernt?
\par 17 Når går de gudløses Lampe ud og når kommer Ulykken over dem? Når deler han Loddet ud i sin Vrede,
\par 18 så de bliver som Strå for Vinden, som Avner, Storm fører bort?
\par 19 Gemmer Gud hans Ulykkeslod til hans Børn? Ham selv gengælde han, så han mærker det,
\par 20 lad ham selv få sit Vanheld at se, den Almægtiges Vrede at drikke!
\par 21 Thi hvad bryder han sig siden om sit Hus, når hans Måneders Tal er udrundet?
\par 22 Kan man vel tage Gud i Skole, ham, som dømmer de højeste Væsner?
\par 23 En dør jo på Lykkens Tinde, helt tryg og så helt uden Sorger:
\par 24 hans Spande er fulde af Mælk, hans Knogler af saftig Marv;
\par 25 med bitter Sjæl dør en anden og har aldrig nydt nogen Lykke;
\par 26 de lægger sig begge i Jorden, og begge dækkes af Orme!
\par 27 Se, jeg kender så vel eders Tanker og de Rænker, I spinder imod mig,
\par 28 når I siger: "Hvor er Stormandens Hus og det Telt, hvor de gudløse bor?"
\par 29 Har I aldrig spurgt de berejste og godkendt deres Beviser:
\par 30 Den onde skånes på Ulykkens Dag og frelses på Vredens Dag.
\par 31 Hvem foreholder ham vel hans Færd, gengælder ham, hvad han gør?
\par 32 Til Graven bæres han hen, ved hans Gravhøj holdes der Vagt;
\par 33 i Dalbunden hviler han sødt, Alverden følger så efter, en Flok uden Tal gik forud for ham.
\par 34 Hvor tom er den Trøst, som I giver! Eders Svar - kun Svig er tilbage!

\chapter{22}

\par 1 Så tog Temaniten Elifaz til Orde og sagde:
\par 2 "Gavner et Menneske Gud? Nej, den kloge gavner sig selv.
\par 3 Har den Almægtige godt af din Retfærd, Vinding af, at din Vandel er ret?
\par 4 Revser han dig for din Gudsfrygt? Eller går han i Rette med dig derfor?
\par 5 Er ikke din Ondskab stor og din Brøde uden Ende?
\par 6 Thi du pantede Brødre uden Grund, trak Klæderne af de nøgne,
\par 7 gav ikke den trætte Vand at drikke og nægted den sultne Brød.
\par 8 Den mægtige - hans var Landet, den hædrede boede der.
\par 9 Du lod Enker gå tomhændet bort, knuste de faderløses Arme.
\par 10 Derfor var der Snaret omkring dig, og Rædsel ængsted dig brat.
\par 11 Dit Lys blev Mørke, du kan ej se, og Strømme af Vand går over dig!
\par 12 Er Gud ej i højen Himmel? Se Stjernernes Tinde, hvor højt de står!
\par 13 Dog siger du: "Hvad ved Gud, holder han Dom bag sorten Sky?
\par 14 Skyerne skjuler ham, så han ej ser, på Himlens Runding går han!"
\par 15 Vil du følge Fortidens Sti, som Urettens Mænd betrådte,
\par 16 de, som i Utide reves bort, hvis Grundvold flød bort som en Strøm,
\par 17 som sagde til Gud: "Gå fra os! Hvad kan den Almægtige gøre os?"
\par 18 Og han havde dog fyldt deres Huse med godt. Men de gudløses Råd er ham fjernt.
\par 19 De retfærdige så det og glædede sig, den uskyldige spottede dem:
\par 20 For vist, vore Fjender forgik, og Ild fortæred de sidste af dem.
\par 21 Bliv Ven med ham og hold Fred. derved vil der times dig Lykke;
\par 22 tag dog mod Lærdom af ham og læg dig hans Ord på Sinde!
\par 23 Vender du ydmygt om til den Almægtige, fjerner du Uretten fra dit Telt,
\par 24 kaster du Guldet på Jorden, Ofirguldet blandt Bækkenes Sten,
\par 25 så den Almægtige bliver dit Guld, hans Lov dit Sølv,
\par 26 ja, da skal du fryde dig over den Almægtige og løfte dit Åsyn til Gud.
\par 27 Beder du til ham, hører han dig, indfri kan du, hvad du har lovet;
\par 28 hvad du sætter dig for, det lykkes, det lysner på dine Veje;
\par 29 thi stolte, hovmodige ydmyger han, men hjælper den, der slår Øjnene ned;
\par 30 han frelser uskyldig Mand; det sker ved hans Hænders Renhed!

\chapter{23}

\par 1 Så tog Job til Orde og svarede:
\par 2 "Også i Dag er der Trods i min Klage, tungt ligger hans Hånd på mit Suk!
\par 3 Ak, vidste jeg Vej til at finde ham, kunde jeg nå hans Trone!
\par 4 Da vilde jeg udrede Sagen for ham og fylde min Mund med Beviser,
\par 5 vide, hvad Svar han gav mig, skønne, hvad han sagde til mig!
\par 6 Mon han da satte sin Almagt imod mig? Nej, visselig agted han på mig;
\par 7 da gik en oprigtig i Rette med ham, og jeg bjærged for evigt min Ret.
\par 8 Men går jeg mod Øst, da er han der ikke, mod Vest, jeg mærker ej til ham;
\par 9 jeg søger i Nord og ser ham ikke, drejer mod Syd og øjner ham ej.
\par 10 Thi han kender min Vej og min Vandel, som Guld går jeg frem af hans Prøve.
\par 11 Min Fod har holdt fast ved hans Spor, hans Vej har jeg fulgt, veg ikke derfra,
\par 12 fra hans Læbers Bud er jeg ikke veget, hans Ord har jeg gemt i mit Bryst.
\par 13 Men han gjorde sit Valg, hvem hindrer ham Han udfører, hvad hans Sjæl attrår.
\par 14 Thi han fuldbyrder, hvad han bestemte, og af sligt har han meget for.
\par 15 Derfor forfærdes jeg for ham og gruer ved Tanken om ham.
\par 16 Ja, Gud har nedbrudt mit Mod, forfærdet mig har den Almægtige;
\par 17 thi jeg går til i Mørket, mit Åsyn dækkes af Mulm.

\chapter{24}

\par 1 Hvorfor har ej den Almægtige opsparet Tider, hvi får de, som kender ham, ikke hans Dage at se?
\par 2 De onde flytter Markskel, ranede Hjorde har de på Græs.
\par 3 faderløses Æsel fører de bort, tager Enkens Okse som Borgen:
\par 4 de trænger de fattige af Vejen. Landets arme må alle skjule sig.
\par 5 Som vilde Æsler i Ørkenen går de ud til deres Gerning søgende efter Næring; Steppen er Brød for Børnene.
\par 6 De høster på Marken om Natten, i Rigmandens Vingård sanker de efter.
\par 7 Om Natten ligger de nøgne, uden Klæder, uden Tæppe i Hulden.
\par 8 De vædes af Bjergenes Regnskyl, klamrer sig af Mangel på Ly til Klippen.
\par 9 - Man river den faderløse fra Brystet, tager den armes Barn som Borgen.
\par 10 Nøgne vandrer de, uden Klæder, sultne bærer de Neg;
\par 11 mellem Murene presser de Olie. de træder Persen og tørster.
\par 12 De drives fra By og Hus, og Børnenes Hunger skriger. Men Gud, han ænser ej vrangt.
\par 13 Andre hører til Lysets Fjender, de kender ikke hans Veje og holder sig ej på hans Stier:
\par 14 Før det lysner, står Morderen op, han myrder arm og fattig; om Natten sniger Tyven sig om;
\par 15 Horkarlens Øje lurer på Skumring, han tænker: "Intet Øje kan se mig!" og skjuler sit Ansigt under en Maske.
\par 16 I Mørke bryder de ind i Huse, de lukker sig inde om Dagen, thi ingen af dem vil vide af Lys.
\par 17 For dem er Mørket Morgen, thide er kendt med Mørkets Rædsler.
\par 18 Over Vandfladen jages han hen, hans Arvelod i Landet forbandes, han færdes ikke på Vejen til Vingården.
\par 19 Som Tørke og Hede tager Snevand, så Dødsriget dem, der har syndet.
\par 20 Han er glemt på sin Hjemstavns Torv, hans Storhed kommes ej mer i Hu, Uretten knækkes som Træet.
\par 21 Han var ond mod den golde, der ikke fødte, mod Enken gjorde han ikke vel;
\par 22 dem, det gik skævt, rev han bort i sin Vælde. Han står op og er ikke tryg på sit Liv,
\par 23 han styrtes uden Håb og Støtte, og på hans Veje er idel Nød.
\par 24 Hans Storhed er stakket, så er han ej mer, han bøjes og skrumper ind som Melde og skæres af som Aksenes Top.
\par 25 Og hvis ikke - hvo gør mig til Løgner, hvo gør mine Ord til intet?

\chapter{25}

\par 1 Så tog Sjuhiten Bildad til Orde og sagde:
\par 2 "Hos ham er der Vælde og Rædsel, han skaber Fred i sin høje Bolig.
\par 3 Er der mon Tal på hans Skarer? Mod hvem står ikke hans Baghold op?
\par 4 Hvor kan en Mand have Ret imod Gud, hvor kan en kvindefødt være ren?
\par 5 Selv Månen er ikke klar i hans Øjne og Stjernerne ikke rene
\par 6 endsige en Mand, det Kryb, et Menneskebarn, den Orm!

\chapter{26}

\par 1 Så tog Job til Orde og svarede:
\par 2 "Hvor har du dog hjulpet ham, den afmægtige, støttet den kraftløse Arm!
\par 3 Hvor har du dog rådet ham, den uvise, kundgjort en Fylde af Visdom!
\par 4 Hvem hjalp dig med at få Ordene frem, hvis Ånd mon der talte af dig?
\par 5 Skyggerne skælver af Angst, de, som bor under Vandene;
\par 6 blottet er Dødsriget for ham, Afgrunden uden Dække.
\par 7 Han udspænder Norden over det tomme, ophænger Jorden på intet;
\par 8 Vandet binder han i sine Skyer, og Skylaget brister ikke derunder;
\par 9 han fæstner sin Trones Hjørner og breder sit Skylag derover;
\par 10 han drog en Kreds over Vandene, der, hvor Lys og Mørke skilles.
\par 11 Himlens Støtter vakler, de gribes af Angst ved hans Trusel;
\par 12 med Vælde bragte han Havet til Ro og knuste Rahab med Kløgt;
\par 13 ved hans Ånde klarede Himlen op hans Hånd gennembored den flygtende Slange.
\par 14 Se, det er kun Omridset af hans Vej, hvad hører vi andet end Hvisken? Hans Vældes Torden, hvo fatter vel den?

\chapter{27}

\par 1 Job vedblev at fremsætte sit Tankesprog:
\par 2 "Så sandt Gud lever, som satte min Ret til Side, den Almægtige, som gjorde mig mod i Hu:
\par 3 Så længe jeg drager Ånde og har Guds Ånde i Næsen,
\par 4 skal mine Læber ej tale Uret, min Tunge ej fare med Svig!
\par 5 Langt være det fra mig at give jer Ret; til jeg udånder, opgiver jeg ikke min Uskyld.
\par 6 Jeg hævder min Ret, jeg slipper den ikke, ingen af mine Dage piner mit Sind.
\par 7 Som den gudløse gå det min Fjende, min Modstander som den lovløse!
\par 8 Thi hvad er den vanhelliges Håb, når Gud bortskærer og kræver hans Sjæl?
\par 9 Hører mon Gud hans Skrig, når Angst kommer over ham?
\par 10 Mon han kan fryde sig over den Almægtige, føjer han ham, når han påkalder ham?
\par 11 Jeg vil lære jer om Guds Hånd, den Almægtiges Tanker dølger jeg ikke;
\par 12 se, selv har I alle set det, hvi har I så tomme Tanker?
\par 13 Det er den gudløses Lod fra Gud, Arven, som Voldsmænd får fra den Almægtige:
\par 14 Vokser hans Sønner, er det for Sværdet, hans Afkom mættes ikke med Brød;
\par 15 de øvrige bringer Pesten i Graven, deres Enker kan ej holde Klage over dem.
\par 16 Opdynger han Sølv som Støv og samler sig Klæder som Ler
\par 17 han samler, men den retfærdige klæder sig i dem, og Sølvet arver den skyldfri;
\par 18 han bygger sit Hus som en Edderkops, som Hytten, en Vogter gør sig;
\par 19 han lægger sig rig, men for sidste ang, han slår Øjnene op, og er det ej mer;
\par 20 Rædsler når ham som Vande, ved Nat river Stormen ham bort;
\par 21 løftet af Østenstorm farer han bort, den fejer ham væk fra hans Sted.
\par 22 Skånselsløst skyder han på ham, i Hast må han fly fra hans Hånd;
\par 23 man klapper i Hænderne mod ham og piber ham bort fra hans Sted!

\chapter{28}

\par 1 Sølvet har jo sit Leje, som renses, sit sted
\par 2 Jern hentes op af Jorden, og Sten smeltes om til Kobber.
\par 3 På Mørket gør man en Ende og ransager indtil de dybeste Kroge Mørkets og Mulmets Sten;
\par 4 man bryder en Skakt under Foden, og glemte, foruden Fodfæste, hænger de svævende fjernt fra Mennesker.
\par 5 Af Jorden fremvokser Brød, imedens dens Indre omvæltes som af Ild;
\par 6 i Stenen der sidder Safiren, og der er Guldstøv i den.
\par 7 Stien derhen er Rovfuglen ukendt, Falkens Øje udspejder den ikke;
\par 8 den trædes ikke af stolte Vilddyr, Løven skrider ej frem ad den.
\par 9 På Flinten lægger man Hånd og omvælter Bjerge fra Roden;
\par 10 i Klipperne hugger man Gange, alskens Klenodier skuer Øjet;
\par 11 man tilstopper Strømmenes Kilder og bringer det skjulte for Lyset.
\par 12 Men Visdommen - hvor mon den findes, og hvor er Indsigtens Sted?
\par 13 Mennesket kender ikke dens Vej, den findes ej i de levendes Land;
\par 14 Dybet siger: "I mig er den ikke!" Havet: "Ej heller hos mig!"
\par 15 Man får den ej for det fineste Guld, for Sølv kan den ikke købes,
\par 16 den opvejes ikke med Ofirguld, med kostelig Sjoham eller Safir;
\par 17 Guld og Glar kan ej måle sig med den, den fås ej i Bytte for gyldne Kar,
\par 18 Krystal og Koraller ikke at nævne. At eje Visdom er mere end Perler,
\par 19 Ætiopiens Topas kan ej måle sig med den, den opvejes ej med det rene Guld.
\par 20 Men Visdommen - hvor mon den kommer fra, og hvor er Indsigtens Sted?
\par 21 Den er dulgt for alt levendes Øje og skjult for Himmelens Fugle;
\par 22 Afgrund og Død må sige: "Vi hørte kun tale derom."
\par 23 Gud er kendt med dens Vej, han ved, hvor den har sit Sted;
\par 24 thi han skuer til Jordens Ender, alt under Himmelen ser han.
\par 25 Dengang han fastsatte Vindens Vægt og målte Vandet med Mål,
\par 26 da han satte en Lov for Regnen, afmærked Tordenskyen dens Vej,
\par 27 da skued og mønstred han den, han stilled den op og ransaged den.
\par 28 Men til Mennesket sagde han: "Se, HERRENs Frygt, det er Visdom, at sky det onde er Indsigt."

\chapter{29}

\par 1 Og Job vedblev at fremsætte sit Tankesprog:
\par 2 Ak, havde jeg det som tilforn, som dengang Gud tog sig af mig,
\par 3 da hans Lampe lyste over mit Hoved, og jeg ved hans Lys vandt frem i Mørke,
\par 4 som i mine modne År, da Guds Fortrolighed var over mit Telt,
\par 5 da den Almægtige end var hos mig og mine Drenge var om mig,
\par 6 da mine Fødder vaded i Fløde, og Olie strømmede, hvor jeg stod,
\par 7 da jeg gik ud til Byens Port og rejste mit Sæde på Torvet.
\par 8 Når Ungdommen så mig, gemte deo sig, Oldinge rejste sig op og stod,
\par 9 Høvdinger standsed i Talen og lagde Hånd på Mund,
\par 10 Stormænds Røst forstummed, deres Tunge klæbed til Ganen;
\par 11 Øret hørte og priste mig lykkelig, Øjet så og tilkendte mig Ære.
\par 12 Thi jeg redded den arme, der skreg om Hjælp, den faderløse, der savned en Hjælper;
\par 13 den, det gik skævt, velsignede mig, jeg frydede Enkens Hjerte;
\par 14 jeg klædte mig i Retfærd, og den i mig, i Ret som Kappe og Hovedbind.
\par 15 Jeg var den blindes Øje, jeg var den lammes Fod;
\par 16 jeg var de fattiges Fader, udreded den mig ukendtes Sag;
\par 17 den lovløses Tænder brød jeg, rev Byttet ud af hans Gab.
\par 18 Så tænkte jeg da: "Jeg skal dø i min Rede, leve så længe som Føniksfuglen;
\par 19 min Rod kan Vand komme til, Duggen har Nattely i mine Grene;
\par 20 min Ære er altid ny, min Bue er altid ung i min Hånd!"
\par 21 Mig hørte de på og bied, var tavse, mens jeg gav Råd;
\par 22 ingen tog Ordet, når jeg havde talt, mine Ord faldt kvægende på dem;
\par 23 de bied på mig som på Regn, spærred Munden op efter Vårregn.
\par 24 Mistrøstige smilte jeg til, mit Åsyns Lys fik de ej til at svinde.
\par 25 Vejen valgte jeg for dem og sad som Høvding, troned som Konge blandt Hærmænd, som den, der gav sørgende Trøst.

\chapter{30}

\par 1 Nu derimod ler de ad mig, Folk, der er yngre end jeg, hvis Fædre jeg fandt for ringe at sætte iblandt mine Hyrdehunde.
\par 2 Og hvad skulde jeg med deres Hænders Kraft? Deres Ungdomskraft har de mistet,
\par 3 tørrede hen af Trang og Sult. De afgnaver Ørk og Ødemark
\par 4 og plukker Melde ved Krattet, Gyvelrødder er deres Brød.
\par 5 Fra Samfundet drives de bort, som ad Tyve råbes der efter dem.
\par 6 De bor i Kløfter, fulde af Rædsler, i Jordens og Klippernes Huler.
\par 7 De brøler imellem Buske, i Tornekrat kommer de sammen,
\par 8 en dum og navnløs Æt, de joges med Hug af Lande.
\par 9 Men nu er jeg Hånsang for dem, jeg er dem et Samtaleemne;
\par 10 de afskyr mig, holder sig fra mig, nægter sig ikke af spytte ad mig.
\par 11 Thi han løste min Buestreng, ydmyged mig, og foran mig kasted de Tøjlerne af.
\par 12 Til højre rejser sig Ynglen, Fødderne slår de fra mig, bygger sig Ulykkesveje imod mig
\par 13 min Sti har de opbrudt, de hjælper med til mit Fald, og ingen hindrer dem i det;
\par 14 de kommer som gennem et gabende Murbrud, vælter sig frem under Ruiner,
\par 15 Rædsler har vendt sig imod mig; min Værdighed joges bort som af Storm, min Lykke svandt som en Sky.
\par 16 Min Sjæl opløser sig i mig; Elendigheds Dage har ramt mig:
\par 17 Natten borer i mine Knogler, aldrig blunder de nagende Smerter.
\par 18 Med vældig Kraft vanskabes mit Kød, det hænger om mig, som var det min Kjortel.
\par 19 Han kasted mig ud i Dynd, jeg er blevet som Støv og Aske.
\par 20 Jeg skriger til dig, du svarer mig ikke, du står der og ænser mig ikke;
\par 21 grum er du blevet imod mig, forfølger mig med din vældige Hånd.
\par 22 Du løfter og vejrer mig hen i Stormen, og dens Brusen gennemryster mig;
\par 23 thi jeg ved, du fører mig hjem til Døden, til det Hus, hvor alt levende samles.
\par 24 Dog, mon den druknende ej rækker Hånden ud og råber om Hjælp, når han går under?
\par 25 Mon ikke jeg græder over den, som havde det hårdt, sørgede ikke min Sjæl for den fattiges Skyld?
\par 26 Jeg biede på Lykke, men Ulykke kom, jeg håbed på Lys, men Mørke kom;
\par 27 ustandseligt koger det i mig, Elendigheds Dage traf mig;
\par 28 trøstesløs går jeg i Sorg, i Forsamlingen rejser jeg mig og råber;
\par 29 Sjakalernes Broder blev jeg, Strudsenes Fælle.
\par 30 Min Hud er sort, falder af, mine Knogler brænder af Hede;
\par 31 min Citer er blevet til Sorg, min Fløjte til hulkende Gråd!

\chapter{31}

\par 1 Jeg sluttede en Pagt med mit Øje om ikke at se på en Jomfru;
\par 2 hvad var ellers min Lod fra Gud hist oppe, den Arv, den Almægtige gav fra det høje?
\par 3 Har ikke den lovløse Vanheld i Vente, Udådsmændene Modgang?
\par 4 Ser han ej mine Veje og tæller alle mine Skridt?
\par 5 Har jeg holdt til med Løgn, og hasted min Fod til Svig
\par 6 på Rettens Vægtskål veje han mig, så Gud kan kende min Uskyld
\par 7 er mit Skridt bøjet af fra Vejen, og har mit Hjerte fulgt mine Øjne, hang noget ved mine Hænder,
\par 8 da gid jeg må så og en anden fortære, og hvad jeg planted, oprykkes med Rode!
\par 9 Blev jeg en Dåre på Grund at en Kvinde, og har jeg luret ved Næstens Dør,
\par 10 så dreje min Hustru Kværn for en anden, og andre bøje sig over hende!
\par 11 Thi sligt var Skændselsdåd, Brøde, der drages for Retten,
\par 12 ja, Ild, der æder til Afgrunden og sætter hele min Høst i Brand!
\par 13 Har jeg ringeagtet min Træls og min Trælkvindes Ret, når de trættede med mig,
\par 14 hvad skulde jeg da gøre, når Gud stod op, hvad skulde jeg svare, når han så efter?
\par 15 Har ikke min Skaber skabt ham i Moders Skød, har en og samme ej dannet os begge i Moders Liv?
\par 16 Har jeg afslået ringes Ønske, ladet Enkens Øjne vansmægte,
\par 17 var jeg ene om at spise mit Brød, har den faderløse ej spist deraf
\par 18 nej, fra Barnsben fostred jeg ham som en Fader, jeg ledede hende fra min Moders Skød.
\par 19 Har jeg set en Stakkel blottet for Klæder, en fattig savne et Tæppe
\par 20 visselig nej, hans Hofter velsigned mig, når han varmed sig i Uld af mine Lam.
\par 21 Har jeg løftet min Bånd mod en faderløs, fordi jeg var vis på Medhold i Retten,
\par 22 så falde min Skulder fra Nakken, så rykkes min Arm af Led!
\par 23 Thi Guds Rædsel var kommet over mig, og når han rejste sig, magted jeg intet!
\par 24 Har jeg slået min Lid til Guld, kaldt det rene Guld min Fortrøstning,
\par 25 var det min Glæde, at Rigdommen voksed, og at min Hånd fik sanket så meget,
\par 26 så jeg, hvorledes Sollyset stråled, eller den herligt skridende Måne,
\par 27 og lod mit Hjerte sig dåre i Løn, så jeg hylded dem med Kys på min Hånd
\par 28 også det var Brøde, der drages for Retten, thi da fornægted jeg Gud hist oppe.
\par 29 Var min Avindsmands Fald min Glæd jubled jeg, når han ramtes af Vanheld
\par 30 nej, jeg tillod ikke min Gane at synde, så jeg bandende kræved hans Sjæl.
\par 31 Har min Husfælle ej måttet sige: "Hvem mættedes ej af Kød fra hans Bord"
\par 32 nej, den fremmede lå ej ude om Natten, jeg åbned min Dør for Vandringsmænd.
\par 33 Har jeg skjult mine Synder, som Mennesker gør, så jeg dulgte min Brøde i Brystet
\par 34 af Frygt for den store Hob, af Angst for Stamfrænders Ringeagt, så jeg blev inden Døre i Stilhed!
\par 35 Ak, var der dog en, der hørte på mig! Her er mit Bomærke - lad den Almægtige svare! Havde jeg blot min Modparts Indlæg!
\par 36 Sandelig, tog jeg det på min Skulder, kransed mit Hoved dermed som en Krone,
\par 37 svared ham for hvert eneste Skridt og mødte ham som en Fyrste.
\par 38 Har min Mark måttet skrige over mig og alle Furerne græde,
\par 39 har jeg tæret dens Kraft uden Vederlag, udslukt dens Ejeres Liv,
\par 40 så gro der Tjørn for Hvede og Ukrudt i Stedet for Byg! Her ender Jobs Ord.

\chapter{32}

\par 1 Da nu hine tre Mænd ikke mere svarede Job, fordi han var retfærdig i sine egne Øjne,
\par 2 blussede Vreden op i Buziten Elihu, Barak'els Søn, af Rams Slægt. På Job vrededes han, fordi han gjorde sig retfærdigere end Gud,
\par 3 og på hans tre Venner, fordi de ikke fandt noget Svar og dog dømte Job skyldig.
\par 4 Elihu havde ventet, så længe de talte med Job, fordi de var ældre end han;
\par 5 men da han så, at de tre Mænd intet havde at svare, blussede hans Vrede op;
\par 6 og Buziten Elihu, Barak'els Søn, tog til Orde og sagde: Ung af Dage er jeg, og I er gamle Mænd, derfor holdt jeg mig tilbage, angst for at meddele eder min Viden;
\par 7 jeg tænkte: "Lad Alderen tale og Årenes Mængde kundgøre Visdom!"
\par 8 Dog Ånden, den er i Mennesket, og den Almægtiges Ånde giver dem Indsigt;
\par 9 de gamle er ikke altid de kloge, Oldinge ved ej altid, hvad Ret er;
\par 10 derfor siger jeg: Hør mig, lad også mig komme frem med min Viden!
\par 11 Jeg biede på, at I skulde tale, lyttede efter forstandige Ord, at I skulde finde de rette Ord;
\par 12 jeg agtede nøje på eder; men ingen af eder gendrev Job og gav Svar på hans Ord.
\par 13 Sig nu ikke: "Vi stødte på Visdom, Gud må fælde ham, ikke et Menneske!"
\par 14 Mod mig har han ikke rettet sin Tale, og med eders Ord vil jeg ikke svare ham.
\par 15 De blev bange, svarer ej mer, for dem slap Ordene op.
\par 16 Skal jeg tøve, fordi de tier og står der uden at svare et Ord?
\par 17 Også jeg vil svare min Del, også jeg vil frem med min Viden!
\par 18 Thi jeg er fuld af Ord, Ånden i mit Bryst trænger på;
\par 19 som tilbundet Vin er mit Bryst, som nyfyldte Vinsække nær ved at sprænges;
\par 20 tale vil jeg for at få Luft, åbne mine Læber og svare.
\par 21 Forskel gør jeg ikke og smigrer ikke for nogen;
\par 22 thi at smigre bruger jeg ikke, snart rev min Skaber mig ellers bort!

\chapter{33}

\par 1 Men hør nu Job, på min Tale og lyt til alle mine Ord!
\par 2 Se, jeg har åbnet min Mund, min Tunge taler i Ganen;
\par 3 mine Ord er talt af oprigtigt Hjerte, mine Læber fører lutret Tale.
\par 4 Guds Ånd har skabt mig, den Almægtiges Ånde har givet mig Liv.
\par 5 Svar mig, i Fald du kan, rust dig imod mig, mød frem!
\par 6 Se, jeg er din Lige for Gud, også jeg er taget af Ler;
\par 7 Rædsel for mig skal ikke skræmme dig, min Hånd skal ej ligge tyngende på dig.
\par 8 Dog, det har du sagt i mit Påhør, jeg hørte så lydende Ord:
\par 9 "Jeg er ren og uden Brøde, lydeløs, uden Skyld;
\par 10 men han søger Påskud imod mig, regner mig for sin Fjende;
\par 11 han lægger mine Fødder i Blokken, vogter på alle mine Veje."
\par 12 Se, der har du Uret, det er mit Svar, thi Gud er større end Mennesket.
\par 13 Hvorfor tvistes du med ham, fordi han ej svarer på dine Ord?
\par 14 Thi på een Måde taler Gud, ja på to, men man ænser det ikke:
\par 15 I Drømme, i natligt Syn, når Dvale falder på Mennesker, når de slumrende hviler på Lejet;
\par 16 da åbner han Menneskers Øre, gør dem angst med Skræmmebilleder
\par 17 for at få Mennessket bort fra Uret og udrydde Hovmod af Manden,
\par 18 holde hans Sjæl fra Graven, hans Liv fra Våbendød.
\par 19 Eller han revses med Smerter på Lejet, uafbrudt sfår der Hamp i hans Ben;
\par 20 Livet i ham væmmes ved Brød og hans Sjæl ved lækker Mad
\par 21 hans Kød svinder hen, så det ikke ses, hans Knogler, som før ikke sås, bliver blottet;
\par 22 hans Sjæl kommer Graven nær, hans Liv de dræbende Magter.
\par 23 Hvis da en Engel er på hans Side, een blandt de tusind Talsmænd, som varsler Mennesket Tugt,
\par 24 og den viser ham Nåde og siger: "Fri ham fra at synke i Graven, Løsepenge har jeg fået!"
\par 25 da svulmer hans Legem af Friskhed, han oplever atter sin Ungdom.
\par 26 Han beder til Gud, og han er ham nådig, han skuer med Jubel hans Åsyn, fortæller Mennesker om sin Frelse.
\par 27 Han synger det ud for Folk: "Jeg synded og krænkede Retten og fik dog ej Løn som forskyldt!
\par 28 Han har friet min Sjæl fra at fare i Grav, mit Liv ser Lyset med Lyst!"
\par 29 Se, alle disse Ting gør Gud to Gange, ja tre med Mennesket
\par 30 for at redde hans Sjæl fra Graven, så han skuer Livets Lys!
\par 31 Lyt til og hør mig, Job, ti stille, så jeg kan tale!
\par 32 Har du noget at sige, så svar mig, tal, thi gerne gav jeg dig Ret;
\par 33 hvis ikke, så høre du på mig, ti stille, at jeg kan lære dig Visdom!

\chapter{34}

\par 1 Og Elihu tog til Orde og sagde:
\par 2 "Hør mine Ord, I vise, I forstandige Mænd, lån mig Øre!
\par 3 Thi Øret prøver Ord, som Ganen smager på Mad;
\par 4 lad os udgranske, hvad der er Ret, med hinanden skønne, hvad der er godt!
\par 5 Job sagde jo: "Jeg er retfærdig, min Ret har Gud sat til Side;
\par 6 min Ret til Trods skal jeg være en Løgner? Skønt brødefri er jeg såret til Døden!"
\par 7 Er der mon Mage til Job? Han drikker Spot som Vand,
\par 8 søger Selskab med Udådsmænd og Omgang med gudløse Folk!
\par 9 Thi han sagde: "Det båder ikke en Mand, at han har Venskab med Gud!"
\par 10 Derfor, I kloge, hør mig: Det være langt fra Gud af synde, fra den Almægtige at gøre ondt;
\par 11 nej, han gengælder Menneskets Gerning, handler med Manden efter hans Færd;
\par 12 Gud forbryder sig visselig ej, den Almægtige bøjer ej Retten!
\par 13 Hvo gav ham Tilsyn med Jorden, hvo vogter, mon hele Verden?
\par 14 Drog han sin Ånd tilbage og tog sin Ånde til sig igen,
\par 15 da udånded Kødet til Hobe, og atter blev Mennesket Støv!
\par 16 Har du Forstand, så hør derpå, lån Øre til mine Ord!
\par 17 Mon en, der hadede Ret, kunde styre? Dømmer du ham, den Retfærdige, Vældige?
\par 18 Han, som kan sige til Kongen: "Din Usling!" og "Nidding, som du er!" til Stormænd,
\par 19 som ikke gør Forskel til Fordel for Fyrster ej heller foretrækker rig for ringe, thi de er alle hans Hænders Værk.
\par 20 Brat må de dø, endda midt om Natten; de store slår han til, og borte er de, de vældige fjernes uden Menneskehånd.
\par 21 Thi Menneskets Veje er ham for Øje, han skuer alle dets Skridt;
\par 22 der er intet Mørke og intet Mulm, som Udådsmænd kan gemme sig i.
\par 23 Thi Mennesket sættes der ingen Frist til at møde i Retten for Gud;
\par 24 han knuser de vældige uden Forhør og sætter andre i Stedet.
\par 25 Jeg hævder derfor: Han ved deres Gerninger, og ved Nattetide styrter han dem;
\par 26 for deres Gudløshed slås de sønder, for alles Øjne tugter han dem,
\par 27 fordi de veg borf fra ham og ikke regned hans Veje det mindste,
\par 28 så de voldte, at ringe råbte til ham, og han måtte høre de armes Skrig.
\par 29 Tier han stille, hvo vil dømme ham? Skjuler han sit Åsyn, hvo vil laste ham? Over Folk og Mennesker våger han dog,
\par 30 for at ikke en vanhellig skal herske, en af dem, der er Folkets Snarer.
\par 31 Siger da en til Gud: "Fejlet har jeg, men synder ej mer,
\par 32 jeg ser det, lær du mig; har jeg gjort Uret, jeg gør det ej mer!"
\par 33 skal han da gøre Gengæld, fordi du vil det, fordi du indvender noget? Ja du, ikke jeg, skal afgøre det, så sig da nu, hvad du ved!
\par 34 Kloge Folk vil sige til mig som og vise Mænd, der hører mig:
\par 35 "Job taler ikke med Indsigt, hans Ord er uoverlagte!
\par 36 Gid Job uden Ophør må prøves, fordi han svarer som slette Folk!
\par 37 Thi han dynger Synd på Synd, han optræder hovent iblandt os og fremfører mange Ord imod Gud!"

\chapter{35}

\par 1 Og Elihu tog til Orde og sagde:
\par 2 "Holder du det for Ret, og kalder du det din Ret for Gud,
\par 3 at du siger: "Hvad båder det mig, hvad hjælper det mig, at jeg ikke synder?"
\par 4 Jeg vil give dig Svar og tillige med dig dine Venner:
\par 5 Løft dit Blik imod Himlen og se, læg Mærke til Skyerne, hvor højt de, er over dig!
\par 6 Hvis du synder, hvad skader du ham? Er din Brøde svar, hvad gør det da ham?
\par 7 Er du retfærdig, hvad gavner du ham, hvad mon han får af din Hånd?
\par 8 Du Menneske, dig vedkommer din Gudløshed, dig, et Menneskebarn, din Retfærd!
\par 9 Man skriger over den megen Vold, råber om Hjælp mod de mægtiges Arm,
\par 10 men siger ej: "Hvor er Gud, vor Skaber, som giver Lovsang om Natten,
\par 11 lærer os mer end Jordens Dyr, gør os vise fremfor Himlens Fugle?"
\par 12 Der råber man, uden at han giver Svar, over de ondes Hovmod;
\par 13 til visse, Gud hører ej tomme Ord, den Almægtige ænser dem ikke,
\par 14 endsige din Påstand om ikke at se ham! Vær stille for hans Åsyn og bi på ham!
\par 15 Men nu, da hans Vrede ej bringer Straf og han ikke bekymrer sig stort om Synd,
\par 16 så oplader Job sin Mund med Tant, uden Indsigt taler han store Ord.

\chapter{36}

\par 1 Og videre sagde Elihu:
\par 2 Bi nu lidt, jeg har noget at sige dig, thi end har jeg Ord til Forsvar for Gud.
\par 3 Jeg vil hente min Viden langvejsfra og skaffe min Skaber Ret;
\par 4 thi for vist, mine Ord er ikke Opspind, en Mand med fuldkommen Indsigt har du for dig.
\par 5 Se, Gud forkaster det stive Sind,
\par 6 den gudløse holder han ikke i Live; de arme lader han få deres Ret,
\par 7 fra retfærdige vender han ikke sit Blik, men giver dem Plads for stedse hos Konger på Tronen i Højhed.
\par 8 Og hvis de bindes i Lænker, fanges i Nødens Bånd,
\par 9 så viser han dem deres Gerning, deres Synder, at de hovmodede sig,
\par 10 åbner deres Øre for Tugt og byder dem vende sig bort fra det onde.
\par 11 Hvis de så hører og bøjer sig, da ender de deres Dage i Lykke, i liflig Fryd deres År.
\par 12 Men hører de ikke, falder de for Sværd og opgiver Ånden i Uforstand.
\par 13 Men vanhellige Hjerter forbitres; når han binder dem, råber de ikke om Hjælp;
\par 14 i Ungdommen dør deres Sjæl, deres Liv får Mandsskøgers Lod.
\par 15 Den elendige frelser han ved hans Elende og åbner hans Øre ved Trængsel.
\par 16 Men dig har Medgangen lokket, du var i Fred for Ulykkens Gab; ingen Trængsel indjog dig Skræk, fuldt var dit Bord af fede Retter.
\par 17 Den gudløses som kom til fulde over dig, hans retfærdige Dom greb dig fat.
\par 18 Lad dig ikke lokke af Vrede til Spot eller Bødens Storhed lede dig vild!
\par 19 Kan vel dit Skrig gøre Ende på Nøden, eller det at du opbyder al din Kraft?
\par 20 Ej må du længes efter Natten, som. opskræmmer Folkeslag der, hvor de er;
\par 21 var dig og vend dig ikke til Uret, så du foretrækker ondt for at lide.
\par 22 Se, ophøjet er Gud i sin Vælde, hvo er en Lærer som han?
\par 23 Hvo foreskrev ham hans Vej, og hvo turde sige: "Du gjorde Uret!"
\par 24 Se til at ophøje hans Værk, som Mennesker priser i Sang!
\par 25 Alle Mennesker ser det med Fryd, skønt dødelige skuer det kun fra det fjerne.
\par 26 Se, Gud er ophøjet, kan ikke ransages, Tal på hans År kan ikke fides.
\par 27 Thi Dråber drager han ud af Havet, i hans Tåge siver de ned som Regn,
\par 28 og Skyerne lader den strømme og dryppe på mange Folk.
\par 29 Hvo fatter mon Skyernes Vidder eller hans Boligs Bulder?
\par 30 Se, han breder sin Tåge om sig og skjuler Havets Rødder;
\par 31 Thi dermed nærer han Folkene, giver dem Brød i Overflod;
\par 32 han hyller sine Hænder i Lys og sender det ud imod Målet;
\par 33 hans Torden melder hans Komme, selv Kvæget melder hans Optræk.

\chapter{37}

\par 1 Ja, derover skælver mit Hjerte, bævende skifter det Sted!
\par 2 Lyt dog til hans bragende Røst, til Drønet, der går fra hans Mund!
\par 3 Han slipper det løs under hele Himlen, sit Lys til Jordens Ender;
\par 4 efter det brøler hans Røst, med Højhed brager hans Torden; han sparer ikke på Lyn, imedens hans Stemme høres.
\par 5 Underfuldt lyder Guds Tordenrøst, han øver Vælde, vi fatter det ej.
\par 6 Thi han siger til Sneen: "Fald ned på Jorden!" til Byger og Regnskyl: "Bliv stærke!"
\par 7 For alle Mennesker sætter han Segl, at de dødelige alle må kende hans Gerning.
\par 8 De vilde Dyr søger Ly og holder sig i deres Huler:
\par 9 Fra Kammeret kommer der Storm, fra Nordens Stjerner Kulde.
\par 10 Ved Guds Ånde bliver der Is, Vandfladen lægges i Fængsel.
\par 11 Så fylder han Skyen med Væde, Skylaget spreder hans Lys;
\par 12 det farer hid og did og bugter sig efter hans Tanke og udfører alt, hvad han byder, på hele den vide Jord,
\par 13 hvad enten han slynger det ud som Svøbe, eller han sender det for at velsigne.
\par 14 Job du må lytte hertil, træd frem og mærk dig Guds Underværker!
\par 15 Fatter du, hvorledes Gud kan magte dem og lade Lys stråle frem fra sin Sky?
\par 16 Fatter du Skyernes Svæven, den Alvises Underværker?
\par 17 Du, hvis Klæder ophedes, når Jorden døser ved Søndenvind?
\par 18 Hvælver du Himlen sammen med ham, fast som det støbte Spejl?
\par 19 Lær mig, hvad vi skal sige ham! Intet kan vi få frem for Mørke.
\par 20 Meldes det ham, at jeg taler? Siger en Mand, at han er fra Samling?
\par 21 Og nu: Man ser ej Lyset, skygget af mørke Skyer, men et Vejr farer hen og renser Himlen,
\par 22 fra Norden kommer en Lysning. Over Gud er der frygtelig Højhed,
\par 23 og den Almægtige finder vi ikke. Almægtig og rig på Retfærd bøjer han ikke Retten;
\par 24 derfor frygter Mennesker ham, men af selv kloge ænser han ingen.

\chapter{38}

\par 1 Så svarede HERREN Job ud fra Stormvejret og sagde:
\par 2 "Hvem fordunkler mit Råd med Ord, som er uden Mening?
\par 3 Omgjord som en Mand dine Lænder, jeg vil spørge, og du skal lære mig!
\par 4 Hvor var du, da jeg grundede Jorden? Sig frem, om du har nogen Indsigt!
\par 5 Hvem bestemte dens Mål - du kender det jo - hvem spændte Målesnor ud derover?
\par 6 Hvorpå blev dens Støtter sænket, hvem lagde dens Hjørnesten,
\par 7 mens Morgenstjernerne jubled til Hobe, og alle Gudssønner råbte af Glæde?
\par 8 Hvem stængte for Havet med Porte, dengang det brusende udgik af Moders Skød,
\par 9 dengang jeg gav det Skyen til Klædning og Tågemulm til Svøb,
\par 10 dengang jeg brød det en Grænse og indsatte Portslå og Døre
\par 11 og sagde: "Hertil og ikke længer! Her standse dine stolte Vover!"
\par 12 Har du nogen Sinde kaldt Morgenen frem, ladet Morgenrøden vide sit Sted,
\par 13 så den greb om Jordens Flige og gudløse rystedes bort,
\par 14 så den dannedes til som Ler under Segl, fik Farve, som var den en Klædning?
\par 15 De gudløses Lys toges fra dem, den løftede Arm blev knust.
\par 16 Har du mon været ved Havets Kilder, har du mon vandret på Dybets Bund?
\par 17 Mon Dødens Porte har vist sig for dig, skued du Mulmets Porte?
\par 18 Så du ud over Jordens Vidder? Sig frem, om du ved, hvor stor den er!
\par 19 Hvor er Vejen til Lysets Bolig, og hvor har Mørket mon hjemme,
\par 20 så du kunde hente det til dets Rige og bringe det hen på Vej til dets Bolig?
\par 21 Du ved det, du blev jo født dengang, dine Dages Tal er jo stort!
\par 22 Har du været, hvor Sneen gemmes, og skuet, hvor Hagelen vogtes,
\par 23 den, jeg gemmer til Trængselens Tid, til Kampens og Krigens Dag?
\par 24 Hvor er Vejen did, hvor Lyset deler sig, hvor Østenvinden spreder sig ud over Jorden?
\par 25 Hvem åbnede Regnen en Rende og Tordenens Lyn en Vej
\par 26 for at væde folketomt Land, Ørkenen, hvor ingen bor,
\par 27 for at kvæge Øde og Ødemark og fremkalde Urter i Ørkenen?
\par 28 Har Regnen mon en Fader, hvem avlede Duggens Dråber?
\par 29 Af hvilket Skød kom Isen vel frem, hvem fødte mon Himlens Rim?
\par 30 Vandet størkner som Sten, Dybets Flade trækker sig sammen.
\par 31 Knytter du Syvstjernens Bånd, kan du løse Orions Lænker?
\par 32 Lader du Aftenstjemen gå op i Tide, leder du Bjørnen med Unger?
\par 33 Kender du Himmelens Love, fastsætter du dens Magt over Jorden?
\par 34 Kan du løfte Røsten til Sky, så Vandskyl adlyder dig?
\par 35 Sender du Lynene ud, så de går, og svarer de dig: "Her er vi!"
\par 36 Hvem lagde Visdom i sorte Skyer, hvem gav Luftsynet Kløgt?
\par 37 Hvem er så viis, at han tæller Skyerne, hvem hælder Himmelens Vandsække om,
\par 38 når Jorden ligger i Ælte, og Leret klumper sig sammen?
\par 39 Jager du Rov til Løvinden, stiller du Ungløvers hunger,
\par 40 når de dukker sig i deres Huler; ligger på Lur i Krat?
\par 41 Hvem skaffer Ravnen Æde, når Ungerne skriger til Gud og flakker om uden Føde?

\chapter{39}

\par 1 Kender du Tiden, da Stengeden føder, tager du Vare på Hindenes Veer,
\par 2 tæller du mon deres Drægtigheds Måneder, kender du Tiden, de føder?
\par 3 De lægger sig ned og føder og kaster Kuldet,
\par 4 Ungerne trives, gror til i det frie, løber bort og kommer ej til dem igen.
\par 5 Hvem slap Vildæslet løs, hvem løste mon Steppeæslets Reb,
\par 6 som jeg gav Ørkenen til Hjem, den salte Steppe til Bolig?
\par 7 Det ler ad Byens Larm og hører ej Driverens Skælden;
\par 8 det ransager Bjerge, der har det sin Græsgang, det leder hvert Græsstrå op.
\par 9 Er Vildoksen villig at trælle for dig, vil den stå ved din Krybbe om Natten?
\par 10 Binder du Reb om dens Hals, pløjer den Furerne efter dig?
\par 11 Stoler du på dens store Kræfter; overlader du den din Høst?
\par 12 Tror du, den kommer tilbage og samler din Sæd på Loen?
\par 13 Mon Strudsens Vinge er lam, eller mangler den Dækfjer og Dun,
\par 14 siden den betror sine Æg til Jorden og lader dem varmes i Sandet,
\par 15 tænker ej på, at en Fod kan knuse dem, Vildtet på Marken træde dem sønder?
\par 16 Hård ved Ungerne er den, som var de ej dens; spildt er dens Møje, det ængster den ikke.
\par 17 Thi Gud lod den glemme Visdom og gav den ej Del i Indsigt.
\par 18 Når Skytterne kommer, farer den bort, den ler ad Hest og Rytter.
\par 19 Giver du Hesten Styrke, klæder dens Hals med Manke
\par 20 og lærer den Græshoppens Spring? Dens stolte Prusten indgyder Rædsel.
\par 21 Den skraber muntert i Dalen, går Brynjen væligt i Møde;
\par 22 den ler ad Rædselen, frygter ikke og viger ikke for Sværdet;
\par 23 Koggeret klirrer over den, Spydet og Køllen blinker;
\par 24 den sluger Vejen med gungrende Vildskab, den tøjler sig ikke, når Hornet lyder;
\par 25 et Stød i Hornet, straks siger den: Huj! Den vejrer Kamp i det fjerne, Kampskrig og Førernes Råb.
\par 26 Skyldes det Indsigt hos dig, at Falken svinger sig op og breder sin Vinge mod Sønden?
\par 27 Skyldes det Bud fra dig, at Ørnen flyver højt og bygger sin højtsatte Rede?
\par 28 Den bygger og bor på Klipper, på Klippens Tinde og Borg;
\par 29 den spejder derfra efter Æde, viden om skuer dens Øjne.
\par 30 Ungerne svælger i Blod; hvor Valen findes, der er den!

\chapter{40}

\par 1 Og HERREN svarede Job og sagde:
\par 2 Vil den trættekære tvistes med den Almægtige? Han, som revser Gud, han svare herpå!
\par 3 Da svarede Job HERREN og sagde:
\par 4 Se, jeg er ringe, hvad skal jeg svare? Jeg lægger min Hånd på min Mund!
\par 5 Een Gang har jeg talt, gentager det ikke, to Gange, men gør det ej mer!
\par 6 Da svarede HERREN Job ud fra Stormvejret og sagde:
\par 7 "Omgjord som en Mand dine Lænder, jeg vil spørge, og du skal lære mig!
\par 8 Mon du vil gøre min Ret til intet, dømme mig, for af du selv kan få Ret?
\par 9 Har du en Arm som Gud, kan du tordne med Brag som han?
\par 10 Smyk dig med Højhed og Storhed, klæd dig i Glans og Herlighed!
\par 11 Udgyd din Vredes Strømme, slå de stolte ned med et Blik,
\par 12 bøj med et Blik de stolte og knus på Stedet de gudløse,
\par 13 skjul dem i Støvet til Hobe og lænk deres Åsyn i Skjulet!
\par 14 Så vil jeg også love dig for Sejren, din højre har vundet.
\par 15 Se Nilhesten! Den har jeg skabt såvel som dig. Som Oksen æder den Græs.
\par 16 Se, hvilken Kraft i Lænderne og hvilken Styrke i Bugens Muskler!
\par 17 Halen holder den stiv som en Ceder, Bovens Sener er flettet sammen;
\par 18 dens Knogler er Rør af, Kobber, Benene i den som Stænger af Jern.
\par 19 Den er Guds ypperste Skabning, skabt til at herske over de andre;
\par 20 thi Foder til den bærer Bjergene, hvor Markens Vildt har Legeplads.
\par 21 Den lægger sig hen under Lotusbuske, i Skjul af Siv og Rør;
\par 22 Lotusbuskene giver den Tag og Skygge, Bækkens Pile yder den Hegn.
\par 23 Den taber ej Modet, når Jordan stiger, er rolig, om Strømmen end svulmer mod dens Gab.
\par 24 Hvem kan gribe den i dens Tænder og trække Reb igennem dens Snude?
\par 25 Kan du trække Krokodillen op med Krog og binde dens Tunge med Snøre?
\par 26 Kan du mon stikke et Siv i dens Snude, bore en Krog igennem dens Kæber?
\par 27 Mon den vil trygle dig længe og give dig gode Ord?
\par 28 Mon den vil indgå en Pagt med dig, så du får den til Træl for evigt?
\par 29 Han du mon lege med den som en Fugl og tøjre den for dine Pigebørn?
\par 30 Falbyder Fiskerlauget den og stykker den ud mellem Sælgerne?
\par 31 Mon du kan spække dens Hud med Kroge og med Harpuner dens Hoved?
\par 32 Læg dog engang din Hånd på den! Du vil huske den Kamp og gør det ej mer.

\chapter{41}

\par 1 Det Håb vilde blive til Skamme, alene ved Synet lå du der.
\par 2 Ingen drister sig til at tirre den, hvem holder Stand imod den?
\par 3 Hvem møder den og slipper fra det hvem under hele Himlen?
\par 4 Jeg tier ej om dens Lemmer, hvor stærk den er, hvor smukt den er skabt.
\par 5 Hvem har trukket dens Klædning af, trængt ind i dens dobbelte Panser?
\par 6 Hvem har åbnet dens Ansigts Døre? Rundt om dens Tænder er Rædsel.
\par 7 Dens Ryg er Reder af Skjolde, dens Bryst er et Segl af Sten;
\par 8 de sidder tæt ved hverandre, Luft kommer ikke ind derimellem;
\par 9 de hænger fast ved hverandre, uadskilleligt griber de ind i hverandre.
\par 10 Dens Nysen fremkalder strålende Lys, som Morgenrødens Øjenlåg er dens Øjne.
\par 11 Ud af dens Gab farer Fakler, Ildgnister spruder der frem.
\par 12 Em står ud af dens Næsebor som af en ophedet, kogende Kedel.
\par 13 Dens Ånde tænder som glødende Kul, Luer står ud af dens Gab.
\par 14 Styrken bor på dens Hals, og Angsten hopper foran den.
\par 15 Tæt sidder Kødets Knuder, som støbt til Kroppen; de rokkes ikke;
\par 16 fast som Sten er dens Hjerte støbt, fast som den nederste Møllesten.
\par 17 Når den rejser sig, gyser Helte, fra Sans og Samling går de af Skræk.
\par 18 Angriberens Sværd holder ikke Stand, ej Kastevåben, Spyd eller Pil.
\par 19 Jern regner den kun for Halm og Kobber for trøsket Træ;
\par 20 Buens Søn slår den ikke på Flugt, Slyngens Sten bliver Strå for den,
\par 21 Stridskøllen regnes for Rør, den ler ad det svirrende Spyd.
\par 22 På Bugen er der skarpe Rande, dens Spor i Dyndet er som Tærskeslædens;
\par 23 Dybet får den i Kog som en Gryde, en Salvekedel gør den af Floden;
\par 24 bag den er der en lysende Sti, Dybet synes som Sølverhår.
\par 25 Dens Lige findes ikke på Jord, den er skabt til ikke at frygte.
\par 26 Alt, hvad højt er, ræddes for den, den er Konge over alle stolte Dyr.

\chapter{42}

\par 1 Så svarede Job HERREN og sagde:
\par 2 "Jeg ved, at du magter alt, for dig er intet umuligt!
\par 3 "Hvem fordunkler mit Råd med Ord, som er uden Indsigt?" Derfor: jeg talte uden Forstand om noget, som var mig for underfuldt, og som jeg ej kendte til.
\par 4 Hør dog, og jeg vil tale, jeg vil spørge, og du skal lære mig!
\par 5 Jeg havde kun hørt et Rygte om dig, men nu har mit Øje set dig;
\par 6 jeg tager det derfor tilbage og angrer i Støv og Aske!
\par 7 Men efter at HERREN havde talet disse Ord til Job, sagde han til Temaniten Elifaz: "Min Vrede er blusset op mod dig og dine to Venner, fordi I ikke talte rettelig om mig som min Tjener Job!
\par 8 Tag eder derfor syv Tyre og syv Vædre og gå til min Tjener Job og bring et Brændoffer for eder. Og min Tjener Job skal gå i Forbøn for eder, thi ham vil jeg bønhøre, så jeg ikke gør en Ulykke på eder, fordi I ikke talte rettelig om mig som min Tjener Job!"
\par 9 Så gik Temaniten Elifaz, Sjuhiten Bildad og Na'amatiten Zofar hen og gjorde, som HERREN havde sagt, og HERREN bønhørte Job.
\par 10 Og HERREN vendte Jobs Skæbne, da han gik i Forbøn for sine Venner; og HERREN gav Job alt, hvad han havde ejet, tvefold igen.
\par 11 Så kom alle hans Brødre og Søstre og alle, der kendte ham tilforn, og holdt Måltid med ham i hans Hus, og de viste ham deres Medfølelse og trøstede ham over al den Ulykke, HERREN havde bragt over ham, og de gav ham hver en Kesita og en Guldring.
\par 12 Og HERREN velsignede Jobs sidste Tid mere end hans første. Han fk 14000 Stykker Småkvæg, og 1000 Aseninder.
\par 13 Og han fik syv Sønner og tre Døtre,
\par 14 og han kaldte den ene Jemima, den anden Kezia og den tredje Keren-Happuk.
\par 15 Så smukke Kvinder som Jobs Døtre fandtes ingensteds på Jorden; og deres Fader gav dem Arv imellem deres Brødre:
\par 16 Siden levede Job 140 År og så sine Børn og Børnebørn, fire Slægtled.
\par 17 Så døde Job gammel og mæt af Dage.


\end{document}