\begin{document}

\title{Salmernes Bog}


\chapter{1}

\par 1 Salig den Mand, som ikke går efter gudløses Råd, står på Synderes Vej eller sidder i Spotteres Lag,
\par 2 men har Lyst til HERRENs Lov, og som grunder på hans Lov både Dag og Nat.
\par 3 Han er som et Træ, der, plantet ved Bække, bærer sin Frugt til rette Tid, og Bladene visner ikke: Alt, hvad han gør, får han Lykke til.
\par 4 De gudløse derimod er som Avner, Vinden bortvejrer.
\par 5 Derfor består de gudløse ikke i Dommen og Syndere ej i retfærdiges Menighed.
\par 6 Thi HERREN kender retfærdiges Vej, men gudløses Vej brydes af.

\chapter{2}

\par 1 Hvorfor fnyser Hedninger, hvi pønser Folkefærd på hvad fåfængt er?
\par 2 Jordens Konger rejser sig, Fyrster samles til Råd mod HERREN og mod hans Salvede:
\par 3 "Lad os sprænge deres Bånd og kaste Rebene af os!"
\par 4 Han, som troner i Himlen, ler, Herren, han spotter dem.
\par 5 Så taler han til dem i Vrede, forfærder dem i sin Harme:
\par 6 "Jeg har dog indsat min Konge på Zion, mit hellige Bjerg!"
\par 7 Jeg kundgør HERRENs Tilsagn. Han sagde til mig: "Du er min Søn, jeg har født dig i Dag!
\par 8 Bed mig, og jeg giver dig Hedningefolk til Arv og den vide Jord i Eje;
\par 9 med Jernspir skal du knuse dem og sønderslå dem som en Pottemagers Kar!"
\par 10 Og nu, I Konger, vær kloge, lad eder råde, I Jordens Dommere,
\par 11 tjener HERREN i Frygt, fryd jer med Bæven!
\par 12 Kysser Sønnen, at ikke han vredes og I forgår! Snart blusset hans Vrede op. Salig hver den, der lider på ham!

\chapter{3}

\par 1 En Salme af David, da han flygtede for sin Søn Absalom.
\par 2 HERRE, hvor er mine Fjender mange! Mange er de, som rejser sig mod mig,
\par 3 mange, som siger om min Sjæl: "Der er ingen Frelse for ham hos Gud!" - Sela.
\par 4 Men, HERRE, du er et Skjold for mig, min Ære og den, der løfter mit Hoved.
\par 5 Jeg råber højlydt til HERREN, han svarer mig fra sit hellige Bjerg. - Sela.
\par 6 Jeg lagde mig og sov ind, jeg vågned, thi HERREN holder mig oppe.
\par 7 Jeg frygter ikke Titusinder af Folk, som trindt om lejrer sig mod mig.
\par 8 Rejs dig, HERRE, frels mig, min Gud, thi alle mine Fjender slog du på Kind, du brød de gudløses Tænder!
\par 9 Hos HERREN er Frelsen; din Velsignelse over dit Folk! - Sela.

\chapter{4}

\par 1 Til Korherren. Til strengespil. Salme af David.
\par 2 Svar, når jeg råber, min Retfærds Gud! I Trængsel skaffede du mig Rum. Vær nådig og hør min Bøn!
\par 3 Hvor længe, I Mænd, skal min Ære skændes? Hvor længe vil I elske Tomhed, søge Løgn? - Sela.
\par 4 Vid dog, at HERREN er mig underfuldt god; når jeg påkalder HERREN, hører han mig.
\par 5 Vredes kun, men forsynd eder ikke, tænk efter på eders Leje og ti! - Sela.
\par 6 Bring rette Ofre og stol på HERREN!
\par 7 Mange siger: "Hvo bringer os Lykke?" Opløft på os dit Åsyns Lys!
\par 8 HERRE, du skænked mit Hjerte en Glæde, større end deres, da Korn og Most flød over.
\par 9 I Fred går jeg til Hvile og slumrer straks, thi, HERRE, du lader mig bo alene i Tryghed.

\chapter{5}

\par 1 Til Korherren. Til fløjtespil Salme af David.
\par 2 HERRE, lytt til mit Ord og agt på mit suk,
\par 3 lån Øre til mit Nødråb, min Konge og Gud, thi jeg beder til dig!
\par 4 Årle hører du, HERRE, min Røst, årle bringer jeg dig min Sag og spejder.
\par 5 Thi du er ikke en Gud, der ynder Ugudelighed, den onde kan ikke gæste dig,
\par 6 for dig skal Dårer ej træde frem, du hader hver Udådsmand,
\par 7 tilintetgør dem, der farer med Løgn; en blodstænkt, svigefuld Mand er HERREN en Afsky.
\par 8 Men jeg kan gå ind i dit Hus af din store Nåde og vendt mod dit hellige Tempel bøje mig i din Frygt.
\par 9 Så led mig for mine Fjenders Skyld i din Retfærd, HERRE, jævn din Vej for mit Ansigt!
\par 10 Thi blottet for Sandhed er deres Mund, deres Hjerte en Afgrund, Struben en åben Grav, deres Tunge er glat.
\par 11 Døm dem, o Gud, lad dem falde for egne Rænker, bortstød dem for deres Synders Mængde, de trodser jo dig.
\par 12 Lad alle glædes, som lider på dig, evindelig frydes, skærm dem, som elsker dit Navn, lad dem juble i dig!
\par 13 Thi du velsigner den retfærdige, HERRE, du dækker ham med Nåde som Skjold.

\chapter{6}

\par 1 Til Korherren. Til strengespil. Al-ha-sheminit. Salme af David.
\par 2 HERRE, revs mig ej i din Vrede, tugt mig ej i din Harme,
\par 3 vær mig nådig Herre, jeg sygner hen, mine Ledmod skælver, læg mig, Herre!
\par 4 Såre skælver min Sjæl; o HERRE, hvor længe endnu?
\par 5 Vend tilbage, HERRE, og frels min Sjæl, hjælp mig dog for din Miskundheds Skyld!
\par 6 Thi i Døden kommes du ikke i Hu, i Dødsriget hvo vil takke dig der?
\par 7 Jeg er så træt af at sukke; jeg væder hver Nat mit Leje, bader med Tårer min Seng;
\par 8 mit Øje hentæres af Sorg, sløves for alle mine Fjenders Skyld.
\par 9 Vig fra mig, alle I Udådsmænd, thi HERREN har hørt min Gråd,
\par 10 HERREN har hørt min Tryglen, min Bøn tager HERREN imod.
\par 11 Beskæmmes skal alle mine Fjender og såre forfærdes, brat skal de vige med Skam.

\chapter{7}

\par 1 Shiggajon af David. Han sang den til Herren på grund af benjaminitten Kush.
\par 2 HERRE min Gud, jeg lider på dig, frels mig og fri mig fra hver min Forfølger,
\par 3 at han ej som en Løve skal rive mig sønder, bortrive, uden at nogen befrier.
\par 4 HERRE min Gud, har jeg handlet så, er der Uret i mine Hænder,
\par 5 har jeg voldet dem ondt, der holdt Fred med mig, uden Årsag gjort mine Fjender Men,
\par 6 så forfølge og indhente Fjenden min Sjæl, han træde mit Liv til Jorden og kaste min Ære i Støvet. - Sela.
\par 7 HERRE, stå op i din Vrede, rejs dig imod mine Fjenders Fnysen, vågn op, min Gud, du sætte Retten!
\par 8 Lad Folkeflokken samles om dig, tag Sæde over den hist i det høje!
\par 9 HERREN dømmer Folkeslag. Mig dømme du, HERRE, efter min Retfærd og Uskyld!
\par 10 På gudløses Ondskab gøre du Ende, støt den retfærdige, du, som prøver Hjerter og Nyrer, retfærdige Gud.
\par 11 Mit Skjold er hos Gud, han frelser de oprigtige af Hjertet;
\par 12 retfærdig som Dommer er Gud, en Gud, der hver Dag vredes.
\par 13 Visselig hvæsser han atter sit Sværd, han spænder sin Bue og sigter;
\par 14 men mod sig selv har han rettet de dræbende Våben, gjort sine Pile til brændende Pile.
\par 15 Se, hanundfanger Tomhed, svanger med Ulykke føder han Blændværk;
\par 16 han grov en Grube, han huled den ud, men faldt i den Grav, han gjorde.
\par 17 Ulykken falder ned på hans Hoved, hans Uret rammer hans egen Isse.
\par 18 Jeg vil takke HERREN for hans Retfærd, lovsynge HERREN den Højestes Navn.

\chapter{8}

\par 1 Til Korherren. Al-ha-gittit. Salme af David.
\par 2 HERRE, vor Herre, hvor herligt er dit Navn på den vide Jord du, som bredte din Højhed ud over Himlen!
\par 3 Af spædes og diendes Mund har du rejst dig et Værn for dine Modstanderes Skyld, for at bringe til Tavshed Fjende og Hævner.
\par 4 Når jeg ser din Himmel, dine Fingres Værk, Månen og Stjernerne, som du skabte,
\par 5 hvad er da et Menneske, at du kommer ham i Hu, et Menneskebarn, at du tager dig af ham?
\par 6 Du gjorde ham lidet ringere end Gud. med Ære og Herlighed kroned du ham;
\par 7 du satte ham over dine Hænders Værk, alt lagde du under hans Fødder,
\par 8 Småkvæg og Okser til Hobe, ja, Markens vilde Dyr,
\par 9 Himlens Fugle og Havets Fisk, alt, hvad der farer ad Havenes Stier.
\par 10 HERRE, vor Herre, hvor herligt er dit Navn på den vide Jord!

\chapter{9}

\par 1 Til Korherren. Al-mut-labben. Salme af David.
\par 2 Jeg vil takke HERREN af hele mit Hjerte, kundgøre alle dine Undere,
\par 3 glæde og fryde mig i dig, lovsynge dit Navn, du Højeste,
\par 4 fordi mine Fjender veg, faldt og forgik for dit Åsyn.
\par 5 Thi du hævded min Ret og min Sag, du sad på Tronen som Retfærds Dommer.
\par 6 Du trued ad Folkene, rydded de gudløse ud, deres Navn har du slettet for evigt.
\par 7 Fjenden er borte, lagt øde for stedse, du omstyrted Byer, de mindes ej mer.
\par 8 Men HERREN troner evindelig, han rejste sin Trone til Dom,
\par 9 skal dømme Verden med Retfærd, fælde Dom over Folkefærd med Ret.
\par 10 HERREN blev de fortryktes Tilflugt, en Tilflugt i Trængselstider;
\par 11 og de stoler på dig, de, som kender dit Navn, thi du svigted ej dem, der søgte dig, HERRE.
\par 12 Lovsyng HERREN, der bor på Zion, kundgør blandt Folkene, hvad han har gjort!
\par 13 Thi han, der hævner Blodskyld, kom dem i Hu, han glemte ikke de armes Råb:
\par 14 "HERRE, vær nådig, se, hvad jeg lider af Avindsmænd, du, som løfter mig op fra Dødens Porte,
\par 15 at jeg kan kundgøre al din Pris, juble over din Frelse i Zions Datters Porte!"
\par 16 Folkene sank i Graven, de grov, deres Fod blev hildet i Garnet, de satte.
\par 17 HERREN blev åbenbar, holdt Dom, den gudløse hildedes i sine Hænders Gerning. - Higgajon Sela.
\par 18 Til Dødsriget skal de gudløse fare, alle Folk, der ej kommer Gud i Hu.
\par 19 Thi den fattige glemmes ikke for evigt, ej skuffes evindelig ydmyges Håb.
\par 20 Rejs dig, HERRE, lad ikke Mennesker få Magten, lad Folkene dømmes for dit Åsyn;
\par 21 HERRE, slå dem med Rædsel, lad Folkene kende, at de er Mennesker! - Sela.

\chapter{10}

\par 1 Hvorfor står du så fjernt, o Herre, hvi dølger du dig i Trængselstider?
\par 2 Den gudløse jager i Hovmod den arme, fanger ham i de Rænker, han spinder;
\par 3 thi den gudløse praler af sin Sjæls Atfrå, den gridske forbander, ringeagter HERREN.
\par 4 Den gudløse siger i Hovmod: "Han hjemsøger ej, der er ingen Gud"; det er alle hans Tanker.
\par 5 Dog altid lykkes hans Vej, højt over ham går dine Domme; han blæser ad alle sine Fjender.
\par 6 Han siger i Hjertet: "Jeg rokkes ej, kommer ikke i Nød fra Slægt til Slægt."
\par 7 Hans Mund er fuld af Banden og Svig og Vold, Fordærv og Uret er under hans Tunge;
\par 8 han lægger sig på Lur i Landsbyer, dræber i Løn den skyldfri, efter Staklen spejder hans Øjne;
\par 9 han lurer i Skjul som Løve i Krat, på at fange den arme lurer han, han fanger den arme ind i sit Garn;
\par 10 han dukker sig, sidder på Spring, og Staklerne falder i hans Kløer.
\par 11 Han siger i Hjertet: "Gud glemmer, han skjuler sit Åsyn; han ser det aldrig."
\par 12 Rejs dig, HERRE! Gud, løft din Hånd, de arme glemme du ikke!
\par 13 Hvorfor skal en gudløs spotte Gud, sige i Hjertet, du hjemsøger ikke?
\par 14 Du skuer dog Møje og Kvide, ser det og tager def i din Hånd; Staklen tyr hen til dig, du er den faderløses Hjælper.
\par 15 Knus den ondes, den gudløses Arm, hjemsøg hans Gudløshed, så den ej findes!
\par 16 HERREN er Konge evigt og altid, Hedningerne er ryddet bort af hans Land;
\par 17 du har hørt de ydmyges Ønske, HERRE, du styrker deres Hjerte, vender Øret til
\par 18 for at skaffe fortrykte og faderløse Ret. Ikke skal dødelige mer øve Vold.

\chapter{11}

\par 1 Jeg tager min Tilflugt til HERREN! Hvor kan I sige til min Sjæl: "Fly som en Fugl til Bjergene!
\par 2 Thi se, de gudløse spænder Buen, lægger Pilen til Rette på Strengen for i Mørke at ramme de oprigtige af Hjertet.
\par 3 Når selv Grundpillerne styrter, hvad gør den retfærdige da?"
\par 4 HERREN er i sin hellige Hal, i Himlen er HERRENs Trone; på Jorderig skuer hans Øjne ned, hans Blik ransager Menneskens Børn;
\par 5 retfærdige og gudløse ransager HERREN; dem, der elsker Uret, hader hans Sjæl;
\par 6 over gudløse sender han Regn af Gløder og Svovl, et Stormvejr er deres tilmålte Bæger.
\par 7 Thi retfærdig er HERREN, han elsker at øve Retfærd, de oprigtige skuer hans Åsyn!

\chapter{12}

\par 1 Til Korherren. Al-ha-sheminit. Salme af David.
\par 2 HERRE, hjælp, thi de fromme er borte, svundet er Troskab blandt Menneskens Børn;
\par 3 de taler Løgn, den ene til den anden, med svigefulde Læber og tvedelt Hjerte.
\par 4 Hver svigefuld Læbe udrydde HERREN, den Tunge, der taler store Ord,
\par 5 dem, som siger: "Vor Tunge gør os stærke, vore Læber er med os, hvo er vor Herre?"
\par 6 "For armes Nød og fattiges Suk vil jeg nu stå op", siger HERREN, "jeg frelser den, som man blæser ad."
\par 7 HERRENs Ord er rene Ord, det pure, syvfold lutrede Sølv.
\par 8 HERRE, du vogter os, værner os evigt mod denne Slægt.
\par 9 De gudløse færdes frit overalt, når Skarn ophøjes blandt Menneskens Børn.

\chapter{13}

\par 1 Til Korherren.Salme af David.
\par 2 Hvor længe vil du evigt glemme mig, Herre, hvor længe skjule dit Åsyn for mig?
\par 3 Hvor længe skal jeg huse Sorg i min Sjæl, Kvide i Hjertet Dag og Nat? Hvor længe skal Fjenden ophøje sig over mig?
\par 4 Se til og svar mig, HERRE min Gud, klar mine Øjne, så jeg ej sover ind i Døden
\par 5 og min Fjende skal sige: "Jeg overvandt ham!" mine Uvenner juble, fordi jeg vakler!
\par 6 Dog stoler jeg fast på din Miskundhed, lad mit Hjerte juble over din Frelse!
\par 6 Dog stoler jeg fast på din Miskundhed, lad mit Hjerte juble over din Frelse!

\chapter{14}

\par 1 Dårerne siger i Hjertet: "Der er ingen Gud!" Slet og afskyeligt handler de, ingen gør godt.
\par 2 HERREN skuer ned fra Himlen på Menneskens Børn for at se, om der findes en forstandig, nogen, der søger Gud.
\par 3 Afveget er alle, til Hobe fordærvet, ingen gør godt, end ikke een!
\par 4 Er alle de Udådsmænd da uden Forstand, der æder mit Folk, som åd de Brød, og ikke påkalder HERREN?
\par 5 Af Rædsel gribes de da, thi Gud er i de retfærdiges Slægt.
\par 6 Gør kun den armes Råd til Skamme, HERREN er dog hans Tilflugt.
\par 7 Ak, kom dog fra Zion Israels Frelse! Når HERREN vender Folkets Skæbne, skal Jakob juble, Israel glædes!

\chapter{15}

\par 1 HERRE, hvo kan gæste dit Telt, hvo kan bo på dit hellige Bjerg?
\par 2 Den, som vandrer fuldkomment og øver Ret, taler Sandhed af sit Hjerte;
\par 3 ikke bagtaler med sin Tunge, ikke volder sin Næste ondt og ej bringer Skam over Ven,
\par 4 som agter den forkastede ringe, men ærer dem, der frygter HERREN, ej bryder Ed, han svor til egen Skade,
\par 5 ej låner Penge ud mod Åger og ej tager Gave mod skyldfri. Hvo således gør, skal aldrig rokkes.

\chapter{16}

\par 1 Vogt mig, Gud, thi jeg lider på dig!
\par 2 Jeg siger til HERREN: "Du er min Herre; jeg har ikke andet Gode end dig.
\par 3 De hellige, som er i Landet, de er de herlige, hvem al min Hu står til."
\par 4 Mange Kvaler rammer dem, som vælger en anden Gud; deres Blodofre vil jeg ikke udgyde, ej tage deres Navn i min Mund.
\par 5 HERREN er min tilmålte Del og mit Bæger. Du holder min Arvelod i Hævd.
\par 6 Snorene faldt mig på liflige Steder, ja, en dejlig Arvelod tilfaldt mig.
\par 7 Jeg vil prise HERREN, der gav mig Råd, mine Nyrer maner mig, selv om Natten.
\par 8 Jeg har altid HERREN for Øje, han er ved min højre, jeg rokkes ikke.
\par 9 Derfor glædes mit Hjerte, min Ære jubler, endogså mit Kød skal bo i Tryghed.
\par 10 Thi Dødsriget giver du ikke min Sjæl, lader ikke din hellige skue Graven.
\par 11 Du lærer mig Livets Vej; man mættes af Glæde for dit Åsyn, Livsalighed er i din højre for evigt.

\chapter{17}

\par 1 HERRE, hør en retfærdig Sag, lyt til min Klage lån Øre til Bøn fra svigløse Læber!
\par 2 Fra dig skal min Ret udgå, thi hvad ret er, ser dine Øjne.
\par 3 Prøv mit Hjerte, se efter om Natten, ransag mig, du finder ej Svig hos mig.
\par 4 Ej synded min Mund, hvad end Mennesker gjorde; ved dine Læbers Ord vogted jeg mig for Voldsmænds Veje;
\par 5 mine Skridt har holdt dine Spor, jeg vaklede ej på min Gang.
\par 6 Jeg råber til dig, thi du svarer mig, Gud, bøj Øret til mig, hør på mit Ord!
\par 7 Vis, dig underfuldt nådig, du Frelser for dem, der tyr til din højre for Fjender!
\par 8 Vogt mig som Øjestenen, skjul mig i dine Vingers Skygge
\par 9 for gudløse, der øver Vold imod mig, glubske Fjender, som omringer mig;
\par 10 de har lukket deres Hjerte med Fedt, deres Mund fører Hovmodstale.
\par 11 De omringer os, overalt hvor vi går, de sigter på at slå os til Jorden.
\par 12 De er som den rovgridske Løve, den unge Løve, der ligger på Lur.
\par 13 Rejs dig, HERRE, træd ham i Møde, kast ham til Jorden, fri med dit Sværd min Sjæl fra den gudløses Vold,
\par 14 fra Mændene, HERRE, med din Hånd, fra dødelige Mænd - lad dem få deres Del i levende Live! Fyld deres Bug med dit Forråd af Vrede, lad Børnene mættes dermed og efterlade deres Børn, hvad de levner!
\par 15 Men jeg skal i Retfærd skue dit Åsyn, mættes ved din Skikkelse, når jeg vågner.

\chapter{18}

\par 1 Til Korherren. Af Herrens Tjener. Af David. Han sang denne Sang til Herren dengang Herren havde reddet ham fra alle has Fjender og fra Sauls Hånd
\par 2 HERRE, jeg har dig! hjerteligt kær, min Styrke!
\par 3 HERRE, min Klippe, min Borg. min Befrier, min Gud, mit Bjerg, hvortil jeg tyr, mit Skjold, mit Frelseshorn, mit Værn!
\par 4 Jeg påkalder HERREN, den Højlovede, og frelses fra mine Fjender.
\par 5 Dødens Reb omsluttede mig, Ødelæggelsens Strømme forfærdede mig,
\par 6 Dødsrigets Reb omspændte mig, Dødens Snarer faldt over mig;
\par 7 i min Vånde påkaldte jeg HERREN og råbte til min Gud. Han hørte min Røst fra sin Helligdom, mit Råb fandt ind til hans Ører!
\par 8 Da rystede Jorden og skjalv, Bjergenes Grundvolde bæved og rysted, thi hans Vrede blussede op.
\par 9 Røg for ud af hans Næse, fortærende Ild af hans Mund, Gløder gnistrede fra ham.
\par 10 Han sænkede Himlen, steg ned med Skymulm under sine Fødder;
\par 11 båret af Keruber fløj han, svæved på Vindens Vinger;
\par 12 han omgav sig med Mulm som en Bolig, mørke Vandmasser, vandfyldte Skyer.
\par 13 Fra Glansen foran ham for der Hagl og Ildgløder gennem hans Skyer.
\par 14 HERREN tordnede fra Himlen, den Højeste lod høre sin Røst, Hagl og Ildgløder.
\par 15 Han udslyngede Pile, adsplittede dem, Lyn i Mængde og skræmmede dem.
\par 16 Vandenes Bund kom til Syne, Jordens Grundvolde blottedes ved din Trusel, HERRE, for din Vredes Pust.
\par 17 Han udrakte Hånden fra det høje og greb mig, drog mig op af de vældige Vande,
\par 18 frelste mig fra mine mægtige Fjender, fra mine Avindsmænd; de var mig for stærke.
\par 19 På min Ulykkes Dag faldt de over mig, men HERREN blev mig til Værn.
\par 20 Han førte mig ud i åbent Land, han frelste mig, thi han havde Behag i mig.
\par 21 HERREN gengældte mig efter min Retfærd, lønned mig efter mine Hænders Uskyld;
\par 22 thi jeg holdt mig til HERRENs Veje, svigted i Gudløshed ikke min Gud
\par 23 hans Bud stod mig alle for Øje, hans Lov skød jeg ikke fra mig.
\par 24 Ustraffelig var jeg for ham og vogtede mig for Brøde.
\par 25 HERREN lønned mig efter min Retfærd, mine Hænders Uskyld, som stod ham for Øje!
\par 26 Du viser dig from mod den fromme, retsindig mod den retsindige,
\par 27 du viser dig ren mod den rene og vrang mod den svigefulde.
\par 28 De arme giver du Frelse, hovmodiges Øjne Skam!
\par 29 Ja, min Lampe lader du lyse, HERRE, min Gud opklarer mit Mørke.
\par 30 Thi ved din Hjælp søndrer jeg Mure, ved min Guds Hjælp springer jeg over Volde.
\par 31 Fuldkommen er Guds Vej, lutret er HERRENs Ord. Han er et Skjold for alle, der sætter deres Lid til ham.
\par 32 Ja, hvem er Gud uden HERREN, hvem er en Klippe uden vor Gud,
\par 33 den Gud, der omgjorded mig med Kraft, jævnede Vejen for mig,
\par 34 gjorde mine Fødder som Hindens og gav mig Fodfæste på Højene,
\par 35 oplærte min Hånd til Krig, så mine Arme spændte Kobberbuen!
\par 36 Du gav mig din Frelses Skjold, din højre støttede mig, din Nedladelse gjorde mig stor;
\par 37 du skaffede Plads for mine Skridt, mine Ankler vaklede ikke.
\par 38 Jeg jog mine Fjender, indhentede dem, vendte først om, da de var gjort til intet,
\par 39 slog dem ned, så de ej kunde rejse sig, men lå faldne under min Fod.
\par 40 Du omgjorded mig med Kraft til Kampen, mine Modstandere tvang du i Knæ for mig;
\par 41 du slog mine Fjender på Flugt, mine Avindsmænd rydded jeg af Vejen.
\par 42 De råbte, men ingen hjalp, til HERREN, han svared dem ikke.
\par 43 Jeg knuste dem som Støv for Vinden, fejed dem bort som Gadeskarn.
\par 44 Du friede mig af Folkekampe, du satte mig til Folkeslags Høvding; nu tjener mig ukendte Folk;
\par 45 hører de om mig, lyder de mig, Udlandets Sønner kryber for mig;
\par 46 Udlandets Sønner vansmægter, slæber sig frem af deres Skjul.
\par 47 HERREN lever, højlovet min Klippe, ophøjet være min Frelses Gud,
\par 48 den Gud, som giver mig Hævn, tvinger Folkeslag under min Fod
\par 49 og frier mig fra mine vrede Fjender! Du ophøjer mig over mine Modstandere, fra Voldsmænd frelser du mig.
\par 50 HERRE, derfor priser jeg dig blandt Folkene og lovsynger dit Navn,
\par 51 du, som kraftig hjælper din Konge og viser din Salvede Miskundhed, David og hans Æt evindelig.

\chapter{19}

\par 1 Til Korherren. Salme af David
\par 2 Himlen forkynder Guds Ære, Hvælvingen kundgør hans Hænders værk.
\par 3 Dag bærer Bud til Dag, Nat lader Nat det vide.
\par 4 Uden Ord og uden Tale, uden at Lyden høres,
\par 5 når Himlens Røst over Jorden vide, dens Tale til Jorderigs Ende. På Himlen rejste han Solen et Telt;
\par 6 som en Brudgom går den ud af sit Kammer, er glad som en Helt ved at løbe sin Bane,
\par 7 rinder op ved Himlens ene Rand, og dens Omløb når til den anden. Intet er skjult for dens Glød.
\par 8 HERRENs Lov er fuldkommen, kvæger Sjælen, HERRENs Vidnesbyrd holder, gør enfoldig viis,
\par 9 HERRENs Forskrifter er rette, glæder Hjertet, HERRENs Bud er purt, giver Øjet Glans,
\par 10 HERRENs Frygt er ren, varer evigt, HERRENs Lovbud er Sandhed, rette til Hobe,
\par 11 kostelige fremfor Guld, ja fint Guld i Mængde, søde fremfor Honning og Kubens Saft.
\par 12 Din Tjener tager og Vare på dem; at holde dem lønner sig rigt.
\par 13 Hvo mærker selv, at han fejler? Tilgiv mig lønlige Brøst!
\par 14 Værn også din Tjener mod frække, ej råde de over mig! Så bliver jeg uden Lyde og fri for svare Synder.
\par 15 Lad min Munds Ord være dig til Behag, lad mit Hjertes Tanker nå frem for dit Åsyn, HERRE, min Klippe og min Genløser!

\chapter{20}

\par 1 Til Korherren. Salme af David
\par 2 På trængselens dag bønhøre Herren dig, værne dig Jakobs Guds Navn!
\par 3 Han sende dig Hjælp fra Helligdommen, fra Zion styrke han dig;
\par 4 han komme alle dine Afgrødeofre i Hu og tage dit Brændoffer gyldigt! - Sela.
\par 5 Han give dig efter dit Hjertes Attrå, han fuldbyrde alt dit Råd,
\par 6 at vi må juble over din Frelse, løfte Banner i vor Guds Navn ! HERREN opfylde alle dine Bønner!
\par 7 Nu ved jeg, at HERREN frelser sin Salvede og svarer ham fra sin hellige Himmel med sin højres frelsende Vælde.
\par 8 Nogle stoler på Heste, andre på Vogne, vi sejrer ved HERREN vor Guds Navn.
\par 9 De synker i Knæ og falder, vi rejser os og kommer atter på Fode.
\par 10 HERRE, frels dog Kongen og svar os, den Dag vi kalder!

\chapter{21}

\par 1 Til Korherren. Salme af David
\par 2 HERRE, Kongen er glad ved din Vælde, hvor frydes han højlig over din Frelse!
\par 3 Hvad hans Hjerte ønskede, gav du ham, du afslog ikke hans Læbers Bøn. - Sela.
\par 4 Du kom ham i Møde med rig Velsignelse, satte en Krone af Guld på hans Hoved.
\par 5 Han bad dig om Liv, og du gav ham det, en Række af Dage uden Ende.
\par 6 Stor er hans Glans ved din Frelse, Højhed og Hæder lægger du på ham.
\par 7 Ja, evig Velsignelse gav du ham, med Fryd for dit Åsyn glæded du ham.
\par 8 Thi Kongen stoler på HERREN, ved den Højestes Nåde rokkes han ikke.
\par 9 Til alle dine Fjender når din Hånd, din højre når dine Avindsmænd.
\par 10 Du gør dem til et luende Bål, når du viser dig; HERREN sluger dem i sin Vrede. Ild fortærer dem.
\par 11 Du rydder bort deres Frugt af Jorden, deres Sæd blandt Menneskens Børn.
\par 12 Thi de søger at volde dig ondt, spinder Rænker, men evner intet;
\par 13 thi du slår dem på Flugt, med din Bue sigter du mod deres Ansigt.
\par 14 HERRE, stå op i din Vælde, med Sang og med Spil vil vi prise dit Storværk!

\chapter{22}

\par 1 Til Korherren. Al-ajjelet-ha-shahar. Salme af David
\par 2 Min Gud, min Gud, hvorfor har du forladt mig? Mit Skrig til Trods er Frelsen mig fjern.
\par 3 Min Gud, jeg råber om Dagen, du svarer ikke, om Natten, men finder ej Hvile.
\par 4 Og dog er du den hellige, som troner på Israels Lovsange.
\par 5 På dig forlod vore Fædre sig, forlod sig, og du friede dem;
\par 6 de råbte til dig og frelstes, forlod sig på dig og blev ikke til Skamme.
\par 7 Men jeg er en Orm og ikke en Mand, til Spot for Mennesker, Folk til Spe;
\par 8 alle, der ser mig, håner mig, vrænger Mund og ryster på Hovedet:
\par 9 "Han har væltet sin Sag på HERREN; han fri ham og frelse ham, han har jo Velbehag i ham."
\par 10 Ja, du drog mig af Moders Liv, lod mig hvile trygt ved min Moders Bryst;
\par 11 på dig blev jeg kastet fra Moders Skød, fra Moders Liv var du min Gud.
\par 12 Vær mig ikke fjern, thi Trængslen er nær, og ingen er der, som hjælper!
\par 13 Stærke Tyre står omkring mig, Basans vældige omringer mig,
\par 14 spiler Gabet op imod mig som rovgridske, brølende Løver.
\par 15 Jeg er som Vand, der er udgydt, alle mine Knogler skilles, mit Hjerte er blevet som Voks, det smelter i Livet på mig;
\par 16 min Gane er tør som et Potteskår til Gummerne klæber min Tunge, du lægger mig ned i Dødens Støv.
\par 17 Thi Hunde står omkring mig, onde i Flok omringer mig, de har gennemboret mine Hænder og Fødder,
\par 18 jeg kan tælle alle mine Ben; med Skadefryd ser de på mig.
\par 19 Mine Klæder deler de mellem sig, om Kjortelen kaster de Lod.
\par 20 Men du, o HERRE, vær ikke fjern, min Redning, il mig til Hjælp!
\par 21 Udfri min Sjæl fra Sværdet, min eneste af Hundes Vold!
\par 22 Frels mig fra Løvens Gab, fra Vildoksens Horn! Du har bønhørt mig.
\par 23 Dit Navn vil jeg kundgøre for mine Brødre, prise dig midt i Forsamlingen:
\par 24 "I, som frygter HERREN, pris ham, ær ham; al Jakobs Æt, bæv for ham, al Israels Æt!
\par 25 Thi han foragtede ikke, forsmåede ikke den armes Råb, skjulte ikke sit Åsyn for ham, men hørte, da han råbte til ham!"
\par 26 Jeg vil synge din Pris i en stor Forsamling, indfri mine Løfter iblandt de fromme;
\par 27 de ydmyge skal spise og mættes; hvo HERREN søger, skal prise ham; deres Hjerte leve for evigt!
\par 28 Den vide Jord skal mærke sig det og omvende sig til HERREN, og alle Folkenes Slægter skal tilbede for hans Åsyn;
\par 29 Thi HERRENs er Riget, han er Folkenes Hersker.
\par 30 De skal tilbede ham alene, alle Jordens mægtige; de skal bøje sig for hans Åsyn, alle, der nedsteg i Støvet og ikke holdt deres Sjæl i Live.
\par 31 Ham skal Efterkommeme tjene; om HERREN skal tales til Slægten, der kommer;
\par 32 de skal forkynde et Folk, der fødes, hans Retfærd. Thi han greb ind.

\chapter{23}

\par 1 HERREN er min Hyrde, mig skal intet fattes,
\par 2 han lader mig ligge på grønne Vange. Til Hvilens Vande leder han mig, han kvæger min Sjæl,
\par 3 han fører mig ad rette Veje for sit Navns Skyld.
\par 4 Skal jeg end vandre i Dødsskyggens Dal, jeg frygter ej ondt; thi du er med mig, din Kæp og din Stav er min Trøst.
\par 5 I mine Fjenders Påsyn dækker du Bord for mig, du salver mit Hoved med Olie, mit Bæger flyder over.
\par 6 Kun Godhed og Miskundhed følger mig alle mine Dage, og i HERRENs Hus skal jeg bo gennem lange Tider.

\chapter{24}

\par 1 HERRENs er Jorden og dens Fylde, Jorderig og de, som bor derpå;
\par 2 thi han har grundlagt den på Have, grundfæstet den på Strømme.
\par 3 Hvo kan gå op på HERRENs Bjerg, og hvo kan stå på hans hellige Sted?
\par 4 Den med skyldfri Hænder og Hjertet rent, som ikke sætter sin Hu til Løgn og ikke sværger falsk;
\par 5 han får Velsignelse fra HERREN, Retfærdighed fra sin Frelses Gud.
\par 6 Så er den Slægt, som spørger efter ham, som søger dit Åsyn, Jakobs Gud! - Sela.
\par 7 Løft eders Hoveder, I Porte, løft jer, I ældgamle Døre, at Ærens Konge kan drage ind!
\par 8 Hvo er den Ærens Konge? HERREN, stærk og vældig, HERREN, vældig i Krig!
\par 9 Løft eders Hoveder, I Porte, løft jer, I ældgamle Døre, at Ærens Konge kan drage ind!
\par 10 Hvo er han, den Ærens Konge? HERREN, Hærskarers Herre, han er Ærens Konge! - Sela.

\chapter{25}

\par 1 HERRE, jeg løfter min sjæl til dig
\par 2 min Gud jeg stoler på dig, lad mig ikke beskæmmes, lad ej mine Fjender fryde sig over mig.
\par 3 Nej, ingen som bier på dig, skal beskæmmes; beskæmmes skal de, som er troløse uden Grund.
\par 4 Lad mig kende dine Veje, HERRE lær mig dine Stier.
\par 5 Led mig på din Sandheds Vej og lær mig, thi du er min Frelses Gud; jeg bier bestandig på dig.
\par 6 HERRE, kom din Barmhjertighed i Hu og din Nåde, den er jo fra Evighed af.
\par 7 Mine Ungdomssynder og Overtrædelser komme du ikke i Hu, men efter din Miskundhed kom mig i Hu, for din Godheds Skyld, o HERRE!
\par 8 God og oprigtig er HERREN, derfor viser han Syndere Vejen.
\par 9 Han vejleder ydmyge i det, som er ret, og lærer de ydmyge sin Vej.
\par 10 Alle HERRENs Stier er Miskundhed og Trofasthed for dem, der holder hans Pagt og hans Vidnesbyrd.
\par 11 For dit Navns Skyld, HERRE, tilgive du min Brøde, thi den er stor.
\par 12 Om nogen frygter HERREN, ham viser han den Vej, han skal vælge;
\par 13 selv skal han leve i Lykke og hans Sæd få Landet i Eje.
\par 14 Fortroligt Samfund har HERREN med dem, der frygter ham, og han kundgør dem sin Pagt.
\par 15 Mit Øje er stadig vendt imod HERREN, thi han frier mine Fødder af Snaren.
\par 16 Vend dig til mig og vær mig nådig, thi jeg er ene og arm.
\par 17 Let mit Hjertes Trængsler og før mig ud af min Nød.
\par 18 Se hen til min Nød og min Kvide og tilgiv alle mine Synder.
\par 19 Se hen til mine Fjender, thi de er mange og hader mig med Had uden Grund.
\par 20 Vogt min Sjæl og frels mig, jeg lider på dig, lad mig ikke beskæmmes.
\par 21 Lad Uskyld og Retsind vogte mig, thi jeg bier på dig, HERRE.
\par 22 Forløs, o Gud, Israel af alle dets Trængsler!

\chapter{26}

\par 1 Skaf mig ret o Herre, thi jeg vandrer i Uskyld, stoler på HERREN uden at vakle.
\par 2 Ransag mig, HERRE, og prøv mig, gransk mine Nyrer og mit Hjerte;
\par 3 thi din Miskundhed står mig for Øje, jeg vandrer i din Sandhed.
\par 4 Jeg tager ej Sæde blandt Løgnere, blandt falske kommer jeg ikke.
\par 5 Jeg hader de ondes Forsamling, hos gudløse sidder jeg ej.
\par 6 Jeg tvætter mine Hænder i Renhed, at jeg kan vandre omkring dit Alter, HERRE,
\par 7 for at istemme Takkesang, fortælle om alle dine Undere.
\par 8 HERRE, jeg elsker dit Hus, det Sted, hvor din Herlighed bor.
\par 9 Bortriv ikke min Sjæl med Syndere, mit Liv med blodstænkte Mænd,
\par 10 i hvis Hænder er Skændselsdåd, hvis højre er fuld af Bestikkelse.
\par 11 Jeg har jo vandret i Uskyld, forløs mig og vær mig nådig!
\par 12 Min Fod står på den jævne Grund, i Forsamlinger vil jeg love HERREN.

\chapter{27}

\par 1 HERREN er mit Lys og min Frelse, hvem skal jeg frygte? HERREN er Værn for mit Liv, for hvem skal jeg ræddes?
\par 2 Når onde kommer imod mig for at æde mit Kød, så snubler og falder de, Uvenner og Fjender!
\par 3 Om en Hær end lejrer sig mod mig, er mit Hjerte uden Frygt; om Krig bryder løs imod mig, dog er jeg tryg.
\par 4 Om eet har jeg bedet HERREN, det attrår jeg: alle mine Dage at bo i HERRENs Hus for at skue HERRENs Livsalighed og grunde i hans Tempel.
\par 5 Thi han gemmer mig i sin Hytte på Ulykkens Dag, skjuler mig i sit Telt og løfter mig op på en Klippe.
\par 6 Derfor løfter mit Hoved sig over mine Fjender omkring mig. I hans Telt vil jeg ofre Jubelofre, med Sang og med Spil vil jeg prise HERREN.
\par 7 HERRE, hør mit Råb, vær nådig og svar mig!
\par 8 Jeg mindes, du sagde: "Søg mit Åsyn!" Dit Åsyn søger jeg, HERRE;
\par 9 skjul ikke dit Åsyn for mig! Bortstød ikke din Tjener i Vrede, du er min Hjælp, opgiv og slip mig ikke, min Frelses Gud!
\par 10 Thi Fader og Moder forlod mig, men HERREN tager mig til sig.
\par 11 Vis mig, HERRE, din Vej og led mig ad jævne Stier for Fjendernes Skyld;
\par 12 giv mig ikke i glubske Uvenners Magt! Thi falske Vidner, der udånder Vold, står frem imod mig.
\par 13 Havde jeg ikke troet, at jeg skulde skue HERRENs Godhed i de levendes Land -
\par 14 Bi på HERREN, fat Mod, dit Hjerte være stærkt, ja bi på HERREN!

\chapter{28}

\par 1 Jeg råber til dig, o Herre, min Klippe, vær ikke tavs imod mig, at jeg ej, når du tier, skal blive som de, der synker i Graven.
\par 2 Hør min tryglende Røst, når jeg råber til dig, løfter Hænderne op mod dit hellige Tempel.
\par 3 Riv mig ej bort med gudløse, Udådsmænd, som har ondt i Sinde mod Næsten trods venlige Ord.
\par 4 Løn dem for deres Idræt og onde Gerninger; løn dem for deres Hænders Værk, gengæld dem efter Fortjeneste!
\par 5 Thi HERRENs Gerning ænser de ikke, ej heller hans Hænders Værk. Han nedbryde dem og opbygge dem ej!
\par 6 Lovet være HERREN, thi han har hørt min tryglende Røst;
\par 7 min Styrke, mit Skjold er HERREN, mit Hjerte stoler på ham. Jeg fik Hjælp, mit Hjerte jubler, jeg takker ham med min Sang.
\par 8 HERREN er Værn for sit Folk, sin Salvedes Tilflugt og Frelse.
\par 9 Frels dit Folk og velsign din Arv, røgt dem og bær dem til evig Tid!

\chapter{29}

\par 1 Giver HERREN, I Guds Sønner, giver Herren Ære og Pris,
\par 2 giver HERREN hans Navns Ære; tilbed HERREN i helligt Skrud!
\par 3 HERRENs Røst er over Vandene, Ærens Gud lader Tordenen rulle, HERREN, over de vældige Vande!
\par 4 HERRENs Røst med Vælde, HERRENs Røst i Højhed,
\par 5 HERRENs Røst, den splintrer Cedre, HERREN splintrer Libanons Cedre,
\par 6 får Libanon til at springe som en Kalv og Sirjon som den vilde Okse!
\par 7 HERRENs Røst udslynger Luer.
\par 8 HERRENs Røst får Ørk til at skælve, HERREN får Kadesj's Ørk til at skælve!
\par 9 HERRENs Røst får Hind til at føde, og den gør lyst i Skoven.
\par 10 HERREN tog Sæde og sendte Vandfloden, HERREN tog Sæde som Konge for evigt.
\par 11 HERREN give Kraft til sit Folk, HERREN velsigne sit Folk med Fred!

\chapter{30}

\par 1 Salm af David. En sang ved Tempelindvielsen
\par 2 HERRE, jeg ophøjer dig, thi du bjærgede mig, lod ej mine Fjender glæde sig over mig;
\par 3 HERRE min Gud, jeg råbte til dig, og du helbredte mig.
\par 4 Fra Dødsriget, HERRE, drog du min Sjæl, kaldte mig til Live af Gravens Dyb.
\par 5 Lovsyng HERREN, I hans fromme, pris hans hellige Navn!
\par 6 Thi et Øjeblik varer hans Vrede, Livet igennem hans Nåde; om Aftenen gæster os Gråd, om Morgenen Frydesang.
\par 7 Jeg tænkte i min Tryghed: "Jeg rokkes aldrig i Evighed!"
\par 8 HERRE, i Nåde havde du fæstnet mit Bjerg; du skjulte dit Åsyn, jeg blev forfærdet.
\par 9 Jeg råbte, HERRE, til dig, og tryglende bad jeg til HERREN:
\par 10 "Hvad Vinding har du af mit Blod, af at jeg synker i Graven? Kan Støv mon takke dig, råbe din Trofasthed ud?
\par 11 HERRE, hør og vær nådig, HERRE, kom mig til Hjælp!"
\par 12 Du vendte min Sorg til Dans, løste min Sørgedragt, hylled mig i Glæde,
\par 13 at min Ære skal prise dig uden Ophør. HERRE min Gud, jeg vil takke dig evigt!

\chapter{31}

\par 1 Til Korherren. Salme af David
\par 2 HERRE, jeg lider på dig, lad mig aldrig i Evighed skuffes.
\par 3 du bøje dit Øre til mig; red mig i Hast og vær mig en Tilflugtsklippe, en Klippeborg til min Frelse;
\par 4 thi du er min Klippe og Borg. For dit Navns Skyld lede og føre du mig,
\par 5 fri mig fra Garnet, de satte for mig; thi du er min Tilflugt,
\par 6 i din Hånd befaler jeg min Ånd. Du forløser mig, HERRE, du tro faste Gud,
\par 7 du hader dem, der bolder på Løgneguder. Men jeg, jeg stoler på HERREN,
\par 8 jeg vil juble og glæde mig over din Miskundhed; thi du har set min Nød, agtet på min Sjælekvide.
\par 9 Du gav mig ikke i Fjendens Hånd, men skaffede Rum for min Fod.
\par 10 Vær mig nådig, HERRE, thi jeg er angst, af Kummer hentæres mit Øje, min Sjæl og mit Indre.
\par 11 Thi mit Liv svinder hen i Sorg, mine År i Suk, min Kraft er brudt for min Brødes Skyld, mine Ben hentæres.
\par 12 For alle mine Fjenders Skyld er jeg blevet til Spot, mine Naboers Gru, mine Keodinges Rædsel; de, der ser mig på Gaden, flygter for mig.
\par 13 Som en død er jeg gået dem at Minde, jeg er som et ødelagt Kar.
\par 14 Thi mange hører jeg hviske, trindt om er Rædsel, når de holder Råd imod mig, pønser på at tage mit Liv.
\par 15 Men, HERRE, jeg stoler på dig; jeg siger: Du er min Gud,
\par 16 mine Tider er i din Hånd. Red mig fra Fjenders Hånd, fra dem, der forfølger mig,
\par 17 lad dit Ansigt lyse over din Tjener, frels mig i din Miskundhed.
\par 18 HERRE, lad mig ej blive til Skamme, jeg råber jo til dig, lad de gudløse blive til Skamme og synke tavse i Døden.
\par 19 Lad de falske Læber forstumme, som taler frækt om den retfærdige med Hovmod og Foragt.
\par 20 Hvor stor er dog din Godhed, som du gemmer til dem, der frygter dig, over mod dem, der lider på dig, for Menneskebørnenes Øjne.
\par 21 Du skjuler dem i dit Åsyns Skjul for Menneskers Stimmel; du gemmer dem i en Hytte for Tungers Kiv.
\par 22 Lovet være HERREN, thi under fuld Miskundhed har han vist mig i en befæstet Stad.
\par 23 Og jeg, som sagde i min Angst: "Jeg er bortstødt fra dine Øjne!" Visselig, du hørte min tryglende Røst, da jeg råbte til dig.
\par 24 Elsk HERREN, alle hans fromme; de trofaste skærmer HERREN; men den, der handler i Hovmod, gengælder han mangefold.
\par 25 Fat Mod, eders Hjerte være stærkt, alle I, som bier på HERREN!

\chapter{32}

\par 1 Salig den, hvis Overdtrærdselser er forladt, hvis Synd er skjult:
\par 2 saligt det Menneske, HERREN ej tilregner Skyld, og i hvis Ånd der ikke er Svig.
\par 3 Mine Ben svandt hen, så længe jeg tav, under Jamren Dagen igennem,
\par 4 thi din Hånd lå tungt på mig både Dag og Nat, min Livskraft svandt som i Sommerens Tørke. - Sela.
\par 5 Min Synd bekendte jeg for dig, dulgte ikke min Skyld; jeg sagde: "Mine Overtrædelser vil jeg bekende for HERREN!" Da tilgav du mig min Syndeskyld. - Sela.
\par 6 Derfor bede hver from til dig, den Stund du findes. Kommer da store Vandskyl, ham skal de ikke nå.
\par 7 Du er mit Skjul, du frier mig af Trængsel, med Frelsesjubel omgiver du mig. - Sela.
\par 8 Jeg vil lære dig og vise dig, hvor du skal gå, jeg vil råde dig ved at fæste mit Øje på dig.
\par 9 Vær ikke uden Forstand som Hest eller Muldyr, der tvinges med Tømme og Bidsel, når de ikke vil komme til dig.
\par 10 Den gudløses Smerter er mange, men den, der stoler på HERREN, omgiver han med Nåde.
\par 11 Glæd jer i HERREN, I retfærdige, fryd jer, jubler, alle I oprigtige af Hjertet!

\chapter{33}

\par 1 Jubler i Herren, I retfærdige, for de oprigtige sømmer sig Lovsang;
\par 2 lov HERREN med Citer, tak ham til tistrenget Harpe;
\par 3 en ny Sang synge I ham, leg lifligt på Strenge til Jubelråb!
\par 4 Thi sandt er HERRENs Ord, og al hans Gerning er trofast;
\par 5 han elsker Retfærd og Ret, af HERRENs Miskundhed er Jorden fuld.
\par 6 Ved HERRENs Ord blev Himlen skabt og al dens Hær ved hans Munds Ånde.
\par 7 Som i Vandsæk samled han Havets Vand, lagde Dybets Vande i Forrådskamre.
\par 8 Al Jorden skal frygte for HERREN, Alverdens Beboere skælve for ham;
\par 9 thi han talede, så skete det, han bød, så stod det der.
\par 10 HERREN kuldkasted Folkenes Råd, gjorde Folkeslags Tanker til intet;
\par 11 HERRENs Råd står fast for evigt, hans Hjertes Tanker fra Slægt til Slægt.
\par 12 Saligt det Folk, der har HERREN til Gud, det Folkefærd, han valgte til Arvelod!
\par 13 HERREN skuer fra Himlen, ser på alle Menneskens Børn;
\par 14 fra sit Højsæde holder han Øje med alle, som bor på Jorden;
\par 15 han, som danned deres Hjerter til Hobe, gennemskuer alt deres Værk.
\par 16 Ej frelses en Konge ved sin store Stridsmagt, ej fries en Helt ved sin store Kraft;
\par 17 til Frelse slår Stridshesten ikke til, trods sin store Styrke redder den ikke.
\par 18 Men HERRENs Øje ser til gudfrygtige, til dem, der håber på Nåden,
\par 19 for at fri deres Sjæl fra Døden og holde dem i Live i Hungerens Tid.
\par 20 På HERREN bier vor Sjæl, han er vor Hjælp og vort Skjold;
\par 21 thi vort Hjerte glæder sig i ham, vi stoler på hans hellige Navn.
\par 22 Din Miskundhed være over os, HERRE, så som vi håber på dig.

\chapter{34}

\par 1 Af David. Dengang han spillede vanvittig overfor Abimelek, som jog ham væk så han drog bort.
\par 2 Jeg vil love HERREN til hver en Tid, hans Pris skal stadig fylde min Mund
\par 3 min Sjæl skal rose sig af HERREN, de ydmyge skal høre det og glæde sig.
\par 4 Hylder HERREN i Fællig med mig, lad os sammen ophøje hans Navn!
\par 5 Jeg søgte HERREN, og han svarede mig og friede mig fra alle mine Rædsler.
\par 6 Se hen til ham og strål af Glæde, eders Åsyn skal ikke beskæmmes.
\par 7 Her er en arm, der råbte, og HERREN hørte, af al hans Trængsel frelste han ham.
\par 8 HERRENs Engel slår Lejr om dem, der frygter ham, og frier dem.
\par 9 Smag og se, at HERREN er god, salig den Mand, der lider på ham!
\par 10 Frygter HERREN, I hans hellige, thi de, der frygter ham, mangler intet.
\par 11 Unge Løver lider Nød og sulter, men de, der søger HERREN, dem fattes intet godt.
\par 12 Kom hid, Børnlill, og hør på mig, jeg vil lære jer HERRENs Frygt.
\par 13 Om nogen attrår Liv og ønsker sig Dage for at skue Lykke,
\par 14 så var din Tunge for ondt, dine Læber fra at tale Svig;
\par 15 hold dig fra ondt og øv godt, søg Fred og j ag derefter.
\par 16 på retfærdige hviler hans Øjne, hans Ører hører deres Råb;
\par 17 Mod dem, der gør ondt, er HERRENs Åsyn for at slette deres Minde af Jorden; (vers 16 og 17 har byttet plads)
\par 18 når de skriger, hører HERREN og frier dem af al deres Trængsel.
\par 19 HERREN er nær hos dem, hvis Hjerte er knust, han frelser dem, hvis Ånd er brudt.
\par 20 Den retfærdiges Lidelser er mange, men HERREN frier ham af dem alle;
\par 21 han vogter alle hans Ledemod, ikke et eneste brydes.
\par 22 Ulykke bringer de gudløse Død, og bøde skal de, der hader retfærdige.
\par 23 HERREN forløser sine Tjeneres Sjæl, og ingen, der lider på ham, skal bøde.

\chapter{35}

\par 1 HERRE, træt med dem, der trætter med mig, strid imod dem, der strider mod mig,
\par 2 grib dit Skjold og dit Værge, rejs dig og hjælp mig,
\par 3 tag Spyd og Økse frem mod dem, der forfølger mig, sig til min Sjæl: "Jeg er din Frelse!"
\par 4 Lad dem beskæmmes og blues, som vil mig til Livs, og de, der ønsker mig ondt, lad dem rødmende vige,
\par 5 de blive som Avner for Vinden, og HERRENs Engel nedstøde dem,
\par 6 deres Vej blive mørk og glat, og HERRENs Engel forfølge dem!
\par 7 Thi uden Grund har de sat deres Garn for mig, gravet min Sjæl en Grav.
\par 8 Lad Undergang uventet ramme ham, lad Garnet, han satte, hilde ham selv, lad ham falde i Graven.
\par 9 Min Sjæl skal juble i HERREN, glædes ved hans Frelse,
\par 10 alle mine Ledemod sige: "HERRE, hvo er som du, du, som frelser den arme fra hans Overmand, den arme og fattige fra Røveren!"
\par 11 Falske Vidner står frem, de spørger mig om, hvad jeg ej kender til;
\par 12 de lønner mig godt med ondt, min Sjæl er forladt.
\par 13 Da de var syge, gik jeg i Sæk, med Faste spæged jeg mig, jeg bad med sænket Hoved,
\par 14 som var det en Ven eller Broder; jeg gik, som sørged jeg over min Moder, knuget af Sorg.
\par 15 Men nu jeg vakler, glæder de sig, de stimler sammen, Uslinger, fremmede for mig, stimler sammen imod mig, håner mig uden Ophør;
\par 16 for min Venlighed dænger de mig med Hån, de skærer Tænder imod mig.
\par 17 Herre, hvor længe vil du se til? Frels dog min Sjæl fra deres Brøl, min eneste fra Løver.
\par 18 Jeg vil takke dig i en stor Forsamling, love dig blandt mange Folk.
\par 19 Lad ej dem, som med Urette er mine Fjender, glæde sig over mig, lad ej dem, som hader mig uden Grund, sende spotske Blikke!
\par 20 Thi de taler ej Fred mod de stille i Landet udtænker de Svig;
\par 21 de spærrer Munden op imod mig og siger: "Ha, ha! Vi så det med egne Øjne!"
\par 22 Du så det, HERRE, vær ikke tavs, Herre, hold dig ej borte fra mig;
\par 23 rejs dig, vågn op for min Ret, for min Sag, min Gud og Herre,
\par 24 døm mig efter din Retfærd HERRE, min Gud, lad dem ikke glæde sig over mig
\par 25 Og sige i Hjertet: "Ha! som vi ønsked!" lad dem ikke sige: "Vi slugte ham!"
\par 26 Til Skam og Skændsel blive enhver, hvem min Ulykke glæder; lad dem, der hovmoder sig over mig, hyldes i Spot og Spe.
\par 27 Men de, der vil min Ret, lad dem juble og glæde sig, stadigen sige: "Lovet være HERREN, som under sin Tjener Fred!"
\par 28 Min Tunge skal forkynde din Retfærd, Dagen igennem din Pris.

\chapter{36}

\par 1 Til Korherren. Af Herrens Tjener. Af David
\par 2 Synden taler til den Gudløse inde i hans Hjerte; Gudsfrygt har han ikke for Øje;
\par 3 thi den smigrer ham frækt og siger, at ingen skal finde hans Brøde og hade ham.
\par 4 Hans Munds Ord er Uret og Svig, han har ophørt at handle klogt og godt;
\par 5 på sit Leje udtænker han Uret, han træder en Vej, som ikke er god; det onde afskyr han ikke.
\par 6 HERRE, din Miskundhed rækker til Himlen, din Trofasthed når til Skyerne,
\par 7 din Retfærd er som Guds Bjerge, dine Domme som det store Dyb; HERRE, du frelser Folk og Fæ,
\par 8 hvor dyrebar er dog din Miskundhed, Gud! Og Menneskebørnene skjuler sig i dine Vingers Skygge;
\par 9 de kvæges ved dit Huses Fedme, du læsker dem af din Lifligheds Strøm;
\par 10 thi hos dig er Livets Kilde, i dit Lys skuer vi Lys!
\par 11 Lad din Miskundhed blive over dem, der kender dig, din Retfærd over de oprigtige af Hjertet.
\par 12 Lad Hovmods Fod ej træde mig ned, gudløses Hånd ej jage mig bort.
\par 13 Se, Udådsmændene falder, slås ned, så de ikke kan rejse sig.

\chapter{37}

\par 1 Græm dig ikke over Ugerningsmænd, misund ikke dem, der gør Uret!
\par 2 Thi hastigt svides de af som Græsset, visner som det friske Grønne.
\par 3 Stol på HERREN og gør det gode, bo i Landet og læg Vind på Troskab,
\par 4 da skal du have din Fryd i HERREN, og han skal give dig, hvad dit Hjerte attrår.
\par 5 Vælt din Vej på HERREN, stol på ham, så griber han ind
\par 6 og fører din Retfærdighed frem som Lyset, din Ret som den klare Dag.
\par 7 Vær stille for HERREN og bi på ham, græm dig ej over den, der har Held, over den, der farer med Rænker.
\par 8 Tæm din Harme, lad Vreden fare, græm dig ikke, det volder kun Harm.
\par 9 Thi Ugerningsmænd skal ryddes ud, men de, der bier på HERREN, skal arve Landet.
\par 10 En liden Stund, og den gudløse er ikke mere; ser du hen til hans Sted, så er han der ikke.
\par 11 Men de sagtmodige skal arve Landet, de fryder sig ved megen Fred.
\par 12 Den gudløse vil den retfærdige ilde og skærer Tænder imod ham;
\par 13 men Herren, han ler ad ham, thi han ser hans Time komme.
\par 14 De gudløse drager Sværdet og spænder Buen for at fælde arm og fattig, for at nedslagte dem, der vandrer ret;
\par 15 men Sværdet rammer dem selv i Hjertet, og Buerne brydes sønder og sammen.
\par 16 Det lidt, en retfærdig har, er bedre end mange gudløses Rigdom;
\par 17 thi de gudløses Arme skal brydes, men HERREN støtter de retfærdige;
\par 18 HERREN kender de uskyldiges Dage, deres Arvelod bliver evindelig;
\par 19 de beskæmmes ikke i onde Tider, de mættes i Hungerens Dage.
\par 20 Thi de gudløse går til Grunde, som Engenes Pragt er HERRENs Fjender, de svinder, de svinder som Røg.
\par 21 Den gudløse låner og bliver i Gælden, den retfærdige ynkes og giver;
\par 22 de, han velsigner, skal arve Landet, de, han forbander, udryddes.
\par 23 Af HERREN stadfæstes Mandens Skridt, når han har Behag i hans Vej;
\par 24 om end han snubler, falder han ikke, thi HERREN støtter hans Hånd.
\par 25 Ung har jeg været, og nu er jeg gammel, men aldrig så jeg en retfærdig forladt eller hans Afkom tigge sit Brød;
\par 26 han ynkes altid og låner ud, og hans Afkom er til Velsignelse.
\par 27 Vig fra ondt og øv godt, så bliver du boende evindelig;
\par 28 thi HERREN elsker Ret og svigter ej sine fromme. De onde udslettes for evigt, de gudløses Afkom udryddes;
\par 29 de retfærdige arver Landet og skal bo der til evig Tid.
\par 30 Den retfærdiges Mund taler Visdom; hans Tunge siger, hvad ret er;
\par 31 sin Guds Lov har han i Hjertet, ikke vakler hans Skridt.
\par 32 Den gudløse lurer på den retfærdige og står ham efter Livet,
\par 33 men, HERREN giver ham ej i hans Hånd og lader ham ikke dømmes for Retten.
\par 34 Bi på HERREN og bliv på hans Vej, så skal han ophøje dig til at arve Landet; du skal skue de gudløses Undergang.
\par 35 Jeg har set en gudløs trodse, bryste sig som en Libanons Ceder
\par 36 men se, da jeg gik der forbi, var han borte; da jeg søgte ham, fandtes han ikke.
\par 37 Vogt på Uskyld, læg Vind på Oprigtighed, thi Fredens Mand har en Fremtid;
\par 38 men Overtræderne udryddes til Hobe, de gudløses Fremtid går tabt.
\par 39 De retfærdiges Frelse kommer fra HERREN, deres Tilflugt i Nødens Stund;
\par 40 HERREN hjælper og frier dem, fra de gudløse frier og frelser han dem; thi hos ham har de søgt deres Tilflugt.

\chapter{38}

\par 1 Salme af David. Le-hazkir
\par 2 HERRE, revs mig ej i din vrede, tugt mig ej i din Harme!
\par 3 Thi dine pile sidder i mig, din Hånd har lagt sig på mig.
\par 4 Intet er karskt på min Krop for din Vredes Skyld, intet uskadt i mine Ledemod for mine Synders Skyld;
\par 5 thi over mit Hoved skyller min Brøde som en tyngende Byrde, for tung for mig.
\par 6 Mine Sår både stinker og rådner, for min Dårskabs Skyld går jeg bøjet;
\par 7 jeg er såre nedtrykt, sorgfuld vandrer jeg Dagen lang.
\par 8 Thi Lænderne er fulde af Brand, intet er karskt på min Krop,
\par 9 jeg er lammet og fuldkommen knust, jeg skriger i Hjertets Vånde.
\par 10 HERRE, du kender al min Attrå, mit Suk er ej skjult for dig;
\par 11 mit Hjerte banker, min Kraft har svigtet, selv mit Øje har mistet sin Glans.
\par 12 For min Plages Skyld flyr mig Ven og Frænde, mine Nærmeste holder sig fjert;
\par 13 de, der vil mig til Livs, sætter Snarer, og de, der vil mig ondt, lægger Råd om Fordærv, de tænker Dagen igennem på Svig.
\par 14 Men jeg er som en døv, der intet hører, som en stum, der ej åbner sin Mund,
\par 15 som en Mand, der ikke kan høre, i hvis Mund der ikke er Svar.
\par 16 Thi til dig står mit Håb, o HERRE, du vil bønhøre, Herre min Gud,
\par 17 når jeg siger: "Lad dem ikke glæde sig over mig, hovmode sig over min vaklende Fod!"
\par 18 Thi jeg står allerede for Fald, mine Smerter minder mig stadig;
\par 19 thi jeg må bekende min Skyld må sørge over min Synd.
\par 20 Mange er de, der med Urette er mine Fjender, talrige de, der hader mig uden Grund,
\par 21 som lønner mig godt med ondt, som står mig imod, fordi jeg søger det gode.
\par 22 HERRE, forlad mig ikke, min Gud, hold dig ikke borte fra mig,
\par 23 il mig til Hjælp, o Herre, min Frelse!

\chapter{39}

\par 1 Til Korherren. For Jedutun. Salme af David
\par 2 Jeg sagde: "Mine Veje vil jeg vogte på, så jeg ikke synder med Tungen; min Mund vil jeg holde i Tømme, så længe den gudløse er mig nær!"
\par 3 Jeg var stum og tavs, jeg tav for at undgå tomme Ord, men min Smerte naged,
\par 4 mit Hjerte brændte i Brystet, Ild lued op, mens jeg grunded; da talte jeg med min Tunge.
\par 5 Lær mig, HERRE, at kende mit Endeligt, det Mål af Dage, jeg har, lad mig kende, hvor snart jeg skal bort!
\par 6 Se, i Håndsbredder målte du mine Dage ud, mit Liv er som intet for dig, som et Åndepust står hvert Menneske der. - Sela.
\par 7 Kun som en Skygge er Menneskets Vandring, kun Tomhed er deres Travlhed; de samler og ved ej, hvem der får det.
\par 8 Hvad bier jeg, Herre, da efter? Mit Håb står ene til dig.
\par 9 Fri mig for al min Synd, gør mig ikke til Spot for Dårer!
\par 10 Jeg tier og åbner ikke min Mund, du voldte det jo.
\par 11 Borttag din Plage fra mig, under din vældige Hånd går jeg til.
\par 12 Når du tugter en Mand med Straf for hans Brøde, smuldrer du hans Herlighed hen som Møl; kun et Åndepust er hvert Menneske. - Sela.
\par 13 Hør, o HERRE, min Bøn og lyt til mit Skrig, til mine Tårer tie du ej! Thi en fremmed er jeg hos dig, en Gæst som alle mine Fædre.
\par 14 Se bort fra mig, så jeg kvæges, før jeg går bort og ej mer er til!

\chapter{40}

\par 1 Til Korherren. Salme af David
\par 2 Jeg biede troligt på Herren, han bøjede sig til mig, og hørte mit Skrig.
\par 3 Han drog mig op af den brusende Grav, af det skidne Dynd, han satte min Fod på en Klippe, gav Skridtene Fasthed,
\par 4 en ny Sang lagde han i min Mund, en Lovsang til vor Gud. Mange skal se det og frygte og stole på HERREN.
\par 5 Salig den Mand, der sætter sin Lid til HERREN, ej vender sig til hovmodige eller dem, der hælder til Løgn.
\par 6 Mange Undere gjorde du, HERRE min Gud, og mange Tanker tænkte du for os; de kan ikke opregnes for dig; ellers forkyndte og fortalte jeg dem; til at tælles er de for mange.
\par 7 Til Slagt- og Afgrødeoffer har du ej Lyst, du gav mig åbne Ører, Brænd- og Syndoffer kræver du ikke.
\par 8 Da sagde jeg: "Se, jeg kommer, i Bogrullen er der givet mig Forskrift;
\par 9 at gøre din Vilje, min Gud, er min Lyst, og din Lov er i mit Indre."
\par 10 I en stor Forsamling forkyndte jeg Retfærd, se, mine Læber lukked jeg ikke; HERRE, du ved det.
\par 11 Din Retfærd dulgte jeg ej i mit Hjerte, din Trofasthed og Frelse talte jeg Om, din Nåde og Sandhed fornægted jeg ej i en stor Forsamling.
\par 12 Du, HERRE, vil ikke lukke dit Hjerte for mig, din Nåde og Sandhed skal altid være mit Værn.
\par 13 Thi Ulykker lejrer sig om mig i talløs Mængde, mine Synder har indhentet mig, så jeg ikke kan se, de er flere end Hovedets Hår, og Modet har svigtet.
\par 14 Du værdiges, HERRE, at fri mig, HERRE, il mig til Hjælp.
\par 15 Lad dem beskæmmes og rødme, som vil mig til Livs, og de, der ønsker mig ondt, lad dem vige med Skændsel;
\par 16 Lad dem stivne af Rædsel ved deres Skam, de, som siger: "Ha, ha!" til mig!
\par 17 Lad alle, som søger dig, frydes og glædes i dig; lad dem, som elsker din Frelse, bestandig sige: "HERREN er stor!"
\par 18 Er jeg end arm og fattig, vil Herren dog tænke på mig. Du er min Hjælp og min Frelser; tøv ej, min Gud!

\chapter{41}

\par 1 Til Korherren. Salme af David
\par 2 Salig den Mand, der tager sig af de svage, ham frelser HERREN på Ulykkens Dag;
\par 3 HERREN vogter ham, holder ham i Live, det går ham vel i Landet, han giver ham ikke i Fjendevold.
\par 4 På Sottesengen står HERREN ham bi, hans Smertensleje gør du ham let.
\par 5 Så siger jeg da: Vær mig nådig, HERRE, helbred min Sjæl, jeg har syndet mod dig!
\par 6 Mine Fjender ønsker mig ondt: "Hvornår mon han dør og hans Navn udslettes?"
\par 7 Kommer en i Besøg, så fører han hyklerisk Tale, hans Hjerte samler på ondt, og så går han bort og taler derom.
\par 8 Mine Avindsmænd hvisker sammen imod mig, alle regner de med, at det går mig ilde:
\par 9 "En dødelig Sot har grebet ham; han ligger der - kommer aldrig op!"
\par 10 Endog min Ven, som jeg stolede på, som spiste mit Brød, har løftet Hælen imod mig.
\par 11 Men du, o HERRE, vær mig nådig og rejs mig, så jeg kan øve Gengæld imod dem.
\par 12 Deraf kan jeg kende, at du har mig kær, at min Fjende ikke skal juble over mig.
\par 13 Du holder mig oppe i Kraft af min Uskyld, lader mig stå for dit Åsyn til evig Tid.
\par 14 Lovet være HERREN, Israels Gud, fra Evighed og til Evighed, Amen, Amen!

\chapter{42}

\par 1 Til Korherren. Maskil af Kora-Sønnerne
\par 2 Som Hjorten skriger efter rindende Vand, således skriger min Sjæl efter dig, o Gud.
\par 3 Min Sjæl tørster efter Gud, den levende Gud; når skal jeg komme og stedes for Guds Åsyn?
\par 4 Min Gråd er blevet mit Brød både Dag og Nat, fordi de stadig spørger mig: "Hvor er din Gud?"
\par 5 Min Sjæl er opløst, når jeg kommer i Hu, hvorledes jeg vandred med Skaren op til Guds Hus under Jubelråb og Lovsang i Højtidsskaren.
\par 6 Hvorfor er du nedbøjet, Sjæl, hvi bruser du i mig? Bi efter Gud, thi end skal jeg takke ham, mit Åsyns Frelse og min Gud!
\par 7 Nedbøjet er min Sjæl, derfor mindes jeg dig fra Jordans og Hermontindernes Land, fra Mizars Bjerg.
\par 8 Dyb råber til Dyb ved dine Vandfalds Brusen, alle dine Brændinger og Bølger skyller hen over mig.
\par 9 Sin Miskundhed sender HERREN om Dagen, hans Sang er hos mig om Natten, en Bøn til mit Livs Gud.
\par 10 Jeg siger til Gud, min Klippe: Hvorfor har du glemt mig, hvorfor skal jeg vandre sorgfuld, trængt af Fjender?
\par 11 Det er, som knustes mine Ben, når Fjenderne håner mig, når de stadig spørger mig : "Hvor er din Gud?"
\par 12 Hvorfor er du nedbøjet, Sjæl, hvi bruser du i mig? Bi efter Gud, thi end skal jeg takke ham, mit Åsyns Frelse og min Gud!

\chapter{43}

\par 1 Skaf mig Ret, o Gud, og strid for mig mod Folk, som ej kender til Mildhed, fri mig fra en falsk, uretfærdig Mand!
\par 2 Thi du er min Tilflugts Gud, hvi har du forstødt mig? Hvorfor skal jeg vandre sorgfuld, træn:t af Fjender?
\par 3 Send dit Lys og din Sandhed, de lede mig, bringe mig til dit hellige Bjerg og til dine Boliger,
\par 4 at jeg må komme til Guds Alter, til min Glædes Gud, juble og prise dig til Citer, Gud, min Gud!
\par 5 Hvorfor er du nedbøjet, Sjæl, hvi bruser du i mig? Bi efter Gud, thi end skal jeg takke ham, mit Åsyns Frelse og min Gud!

\chapter{44}

\par 1 Til Korherren. Maskil af Kora-Sønnerne
\par 2 Gud, vi har hørt det med egne ører, vore Fædre har fortalt os derom; du øved en Dåd i deres Dage, i Fortids Dage med din Hånd;
\par 3 Folk drev du bort, men plantede hine, Folkeslag knuste du, men dem lod du brede sig;
\par 4 thi de fik ej Landet i Eje med Sværdet, det var ej deres Arm, der gav dem Sejr, men det var din højre, din Arm og dit Ansigts Lys, thi du havde dem kær.
\par 5 Du, du er min Konge, min Gud, som sender Jakob Sejr.
\par 6 Ved dig nedstøder vi Fjenden, Modstanderne træder vi ned i dit Navn;
\par 7 thi ej på min Bue stoler jeg, mit Sværd kan ikke give mig Sejr;
\par 8 men du gav os Sejr over Fjenden, du lod vore Avindsmænd blive til Skamme.
\par 9 Vi roser os altid af Gud, dit Navn vil vi love for evigt. - Sela.
\par 10 Dog har du forstødt os, gjort os til Spot, du drager ej med vore Hære;
\par 11 du lader os vige for Fjenden, vore Avindsmænd tager sig Bytte;
\par 12 du har givet os hen som Slagtekvæg, og strøet os ud mellem Folkene,
\par 13 dit Folk har du solgt til Spotpris, vandt ikke Rigdom ved Salget.
\par 14 Til Hån for Naboer gør du os, til Spot og Spe for Grander,
\par 15 du gør os til Mundheld blandt Folkene, lader Folkeslagene ryste på Hovedet ad os.
\par 16 Min Skændsel er mig altid i Tanke, og Skam bedækker mit Åsyn
\par 17 for spottende, hånende Tale, for Fjendens og den hævngerriges Blikke.
\par 18 Alt det kom over os, skønt vi glemte dig ikke, sveg ikke heller din Pagt!
\par 19 Vort Hjerte veg ikke fra dig, vore Skridt forlod ej din Vej.
\par 20 Dog knuste du os, hvor Sjakalerne bor, og indhylled os i Mørke.
\par 21 Havde vi glemt vor Guds Navn, bredt Hænderne ud mod en fremmed Gud,
\par 22 vilde Gud ej opspore det? Han kender jo Hjerternes Løn dom
\par 23 nej, for din Skyld dræbes vi Dagen lang og regnes som Slagtekvæg!
\par 24 Vågn op, hvi sover du, Herre? Bliv vågen, forstød ej for stedse!
\par 25 Hvorfor vil du skjule dit Åsyn, glemme vor Nød og Trængsel?
\par 26 Thi vor Sjæl ligger bøjet i Støvet, vort Legeme klæber ved Jorden.
\par 27 Stå op og kom os til Hjælp, forløs os for din Miskundheds Skyld!

\chapter{45}

\par 1 Til Korherren. Al-shoshannim. Maskil af Kora-Sønnerne. En Bryllupssang
\par 2 Mit Hjerte svulmer af liflige Ord, jeg kvæder mit Kvad til Kongens Pris, som Hurtigskriverens Pen er min Tunge.
\par 3 Den skønneste er du af Menneskens Børn, Ynde er udgydt på dine Læber, derfor velsignede Gud dig for evigt.
\par 4 Omgjord din Lænd med Sværdet, o Helt,
\par 5 Lykken følge din Højhed og Hæder, far frem for Sandhedens Sag, for Ydmyghed og Retfærd, din høj re lære dig frygtelige Ting!
\par 6 Dine Pile er hvæssede, Folkeslag falder for din Fod, Kongens Fjender rammes i Hjertet.
\par 7 Din Trone, o Gud, står evigt fast, en Retfærds Stav er din Kongestav.
\par 8 Du elsker Ret og hader Uret; derfor salvede Gud, din Gud, dig med Glædens Olie fremfor dine Fæller,
\par 9 af Myrra, Aloe og Kassia dufter alle dine Klæder. Du glædes ved Strengeleg fra Elfenbenshaller,
\par 10 Kongedøtre står i kostbare Klæder, Dronningen i Ofrguldets Skrud ved din højre.
\par 11 Hør, min Datter, opmærksomt og bøj dit Øre : Glem dit Folk og din Faders Hus,
\par 12 at Kongen må attrå din Skønhed, thi han er din Herre.
\par 13 Tyrus's Datter skal hylde dig med Gaver, Folkets Rigmænd bejle til din Yndest.
\par 14 Idel Pragt er Kongedatteren, hendes Dragt er Perler, stukket i Guld;
\par 15 fulgt af Jomfruer føres hun frem i broget Pragt, Veninderne fører hende hen til Kongen.
\par 16 De føres frem under Glæde og Jubel, holder deres Indtog i Kongens Palads.
\par 17 Dine Sønner træde ind i dine Fædres Sted, sæt dem til Fyrster rundt i Landet!
\par 18 Jeg vil minde om dit Navn fra Slægt til Slægt; derfor skal Folkene love dig evigt og altid.

\chapter{46}

\par 1 Til Korherren. Af Kora-Sønnerne. Al-alamot. En Sang
\par 2 Gud er vor Tilflugt og Styrke, en Hjælp i Angster, prøvet til fulde
\par 3 Derfor frygter vi ikke, om Jorden end bølger og Bjergene styrter i Havenes Skød,
\par 4 om end deres Vande bruser og syder og Bjergene skælver ved deres Vælde. - Sela.
\par 5 En Flod og dens Bække glæder Guds Stad, den Højeste har helliget sin Bolig;
\par 6 i den er Gud, den rokkes ikke, Gud bringer den Hjælp, når Morgen gryr.
\par 7 Folkene larmed, Rigerne vakled, han løfted Røsten, så Jorden skjalv,
\par 8 Hærskarers HERRE er med os, Jakobs Gud er vor faste Borg. - Sela.
\par 9 Kom hid og se på HERRENs Værk, han har udført frygtelige Ting på Jord.
\par 10 Han gør Ende på Krig til Jordens Grænser, han splintrer Buen, sønderbryder Spydene, Skjoldene tænder han i Brand.
\par 11 Hold inde og kend, at jeg er Gud, ophøjet blandt Folkene, ophøjet på Jorden!
\par 12 Hærskarers HERRE er med os, Jakobs Gud er vor faste Borg. - Sela.

\chapter{47}

\par 1 Til Korherren. Salme af Kora-Sønnerne
\par 2 Alle Folkeslag, klap i Hænderne, bryd ud i jublende Lovsang for Gud!
\par 3 Thi HERREN, den Højeste, er frygtelig, en Konge stor over hele Jorden.
\par 4 Han bøjede Folkefærd under os og Folkeslag under vor Fod;
\par 5 han udvalgte os vor Arvelod, Jakob hans elskedes Stolthed. - Sela.
\par 6 Gud steg op under Jubel, HERREN under Homets Klang.
\par 7 Syng, ja syng for Gud, syng, ja syng for vor Konge;
\par 8 thi han er al Jordens Konge, syng en Sang for Gud.
\par 9 Gud har vist, han er Folkenes Konge, på sin hellige Trone har Gud taget Sæde.
\par 10 Folkenes Stormænd samles med Folket, der tilhører Abrahams Gud; thi Guds er Jordens Skjolde, højt ophøjet er han!

\chapter{48}

\par 1 En Sang. Salme af Kora-Sønnerne
\par 2 Stor og højlovet er vor Gud i sin Stad.
\par 3 Smukt løfter sig hans hellige Bjerg, al Jordens Fryd, Zions Bjerg i det højeste Nord, den store Konges By.
\par 4 Som Værn gjorde Gud sig kendt i dens Borge.
\par 5 Thi Kongerne samlede sig, rykked frem tilsammen;
\par 6 de så og tav på Stedet, flyed i Angst,
\par 7 af Rædsel grebes de brat, af Veer som en, der føder.
\par 8 Med Østenstormen knuser du Tarsisskibe.
\par 9 Som vi havde hørt det, så vi det selv i Hærskarers HERREs By, i vor Guds By; til evig Tid lader Gud den stå. - Sela.
\par 10 I din Helligdom tænker vi, Gud, på din Miskundhed;
\par 11 som dit Navn så lyder din Pris til Jordens Grænser. Din højre er fuld af Retfærd,
\par 12 Zions Bjerg fryder sig, Judas Døtre jubler over dine Domme.
\par 13 Drag rundt om Zion, gå rundt og tæl dets Tårne,
\par 14 læg Mærke til dets Ringmur, så gennem dets Borge, at I kan melde Slægten, der
\par 15 kommer: Sådan er Gud, vor Gud for evigt og altid, han skal lede os.

\chapter{49}

\par 1 Til Korherren. Salme af Kora-Sønnerne
\par 2 Hør det, alle Folkeslag, lyt til, al Verdens Folk,
\par 3 både høj og lav, både rig og fattig!
\par 4 Min Mund skal tale Visdom, mit Hjerte udgransker Indsigt;
\par 5 jeg bøjer mit Øre til Tankesprog, råder min Gåde til Strengeleg.
\par 6 Hvorfor skulle jeg frygte i de onde dage, når mine lumske Fjender omringer mig med Brøde,
\par 7 de, som stoler på deres gods og bryster sig af deres store rigdom?
\par 8 Visselig, ingen kan købe sin sjæl fri og give Gud en løsesum
\par 9 - Prisen for hans sjæl blev for høj, for evigt måtte han opgive det - så han kunde blive i Live
\par 10 og aldrig få Graven at se;
\par 11 nej, han skal se den; Vismænd dør, både Dåre og Tåbe går bort. Deres Gods må de afstå til andre,
\par 12 deres Grav er deres Hjem for evigt, deres Bolig Slægt efter Slægt, om Godser end fik deres Navn.
\par 13 Trods Herlighed bliver Mennesket ikke, han er som Dyrene, der forgår.
\par 14 Så går det dem, der tror sig trygge, så ender det for dem, deres Tale behager. - Sela.
\par 15 I Dødsriget drives de ned som Får, deres Hyrde skal Døden være; de oprigtige træder på dem ved Gry, deres Skikkelse går Opløsning i Møde, Dødsriget er deres Bolig.
\par 16 Men Gud udløser min Sjæl af Dødsrigets Hånd, thi han tager mig til sig. - Sela.
\par 17 Frygt ej, når en Mand bliver rig, når hans Huses Herlighed øges;
\par 18 thi intet tager han med i Døden, hans Herlighed følger ham ikke.
\par 19 Priser han end i Live sig selv: "De lover dig for din Lykke!"
\par 20 han vandrer til sine Fædres Slægt, der aldrig får Lyset at skue.
\par 21 Den, som lever i Herlighed, men uden Forstand, han er som Dyrene, der forgår.

\chapter{50}

\par 1 Salme af Ashaf
\par 2 Gud, Gud HERREN talede og stævnede Jorden hid fra Sol i Opgang til Sol i Bjærge;
\par 2 Gud, Gud HERREN talede og stævnede Jorden hid fra Sol i Opgang til Sol i Bjærge;
\par 3 vor Gud komme og tie ikke! - Foran ham gik fortærende Ild, omkring ham rasede Storm;
\par 4 han stævnede Himlen deroppe hid og Jorden for at dømme sit Folk:
\par 5 "Saml mig mine fromme, der sluttede Pagt med mig ved Ofre!"
\par 6 Og Himlen forkyndte hans Retfærd, at Gud er den, der dømmer.
\par 7 Hør, mit Folk, jeg vil tale, Israel, jeg vil vidne imod dig, Gud, din Gud er jeg!
\par 8 Jeg laster dig ikke for dine Slagtofre, dine Brændofre har jeg jo stadig for Øje;
\par 9 jeg tager ej Tyre fra dit Hus eller Bukke fra dine Stalde;
\par 10 thi mig tilhører alt Skovens Vildt, Dyrene på de tusinde Bjerge;
\par 11 jeg kender alle Bjergenes Fugle, har rede på Markens Vrimmel.
\par 12 Om jeg hungred, jeg sagde det ikke til dig, thi mit er Jorderig og dets Fylde!
\par 13 Mon jeg æder Tyres Kød eller drikker Bukkes Blod?
\par 14 Lovsang skal du ofre til Gud og holde den Højeste dine Løfter.
\par 15 Og kald på mig på Nødens Dag; jeg vil udfri dig, og du skal ære mig,
\par 16 Men til Den gudløse siger Gud: Hvi regner du op mine Bud og fører min Pagt i Munden,
\par 17 når du dog hader Tugt og kaster mine Ord bag din Ryg?
\par 18 Ser du en Tyv, slår du Følge med ham, med Horkarle bolder du til,
\par 19 slipper Munden løs med ondt, din Tunge bærer på Svig.
\par 20 Du sidder og skænder din Broder, bagtaler din Moders Søn;
\par 21 det gør du, og jeg skulde tie, og du skulde tænke, jeg er som du! Revse dig vil jeg og gøre dig det klart.
\par 22 Mærk jer det, I, som glemmer Gud, at jeg ikke skal rive jer redningsløst sønder.
\par 23 Den, der ofrer Taksigelse, ærer mig; den, der agter på Vejen, lader jeg se Guds Frelse.

\chapter{51}

\par 1 Til Korherren. Salme af David
\par 2 Dengang Profeten Nathan kom til ham, fordi David havde været sammen med Bathseba
\par 3 Gud, vær mig nådig efter din Miskundhed, udslet mine Overtrædelser efter din store Barmhjertighed,
\par 4 tvæt mig fuldkommen ren for min Skyld og rens mig for min Synd!
\par 5 Mine Overtrædelser kender jeg jo, min Synd står mig altid for Øje.
\par 6 Mod dig har jeg syndet, mod dig alene, og gjort, hvad i dine Øjne er ondt, at du må få Ret, når du taler, stå ren, når du dømmer.
\par 7 Se, jeg er født i Misgerning, min Moder undfanged mig i Synd.
\par 8 Du elsker jo Sandhed i Hjertets Løndom, så lær mig da Visdom i Hjertedybet.
\par 9 Rens mig for Synd med Ysop, tvæt mig hvidere end Sne;
\par 10 mæt mig med Fryd og Glæde, lad de Ben, du knuste, juble;
\par 11 skjul dit Åsyn for mine Synder, udslet alle mine Misgerninger;
\par 12 skab mig, o Gud, et rent Hjerte, giv en ny, en stadig Ånd i mit Indre;
\par 13 kast mig ikke bort fra dit Åsyn, tag ikke din hellige Ånd fra mig;
\par 14 glæd mig igen med din Frelse, giv mig til Støtte en villig Ånd!
\par 15 Da vil jeg lære Overtrædere dine Veje, og Syndere skal vende om til dig.
\par 16 Fri mig fra Blodskyld, Gud, min Frelses Gud, så skal min Tunge lovsynge din Retfærd;
\par 17 Herre, åben mine Læber, så skal min Mund forkynde din Pris.
\par 18 Thi i Slagtoffer har du ikke Behag, og gav jeg et Brændoffer, vandt det dig ikke.
\par 19 Offer for Gud er en sønderbrudt Ånd; et sønderbrudt, sønderknust Hjerte agter du ikke ringe, o Gud.
\par 20 Gør vel i din Nåde mod Zion, opbyg Jerusalems Mure!
\par 21 Da skal du have Behag i rette Ofre, Brænd- og Heloffer, da bringes Tyre op på dit Alter.

\chapter{52}

\par 1 Til Korherren. Maskil af David
\par 2 Dengang Edomitten Do'eg gik hen og fortalte Saul, at David var taget ind til Akimelek
\par 3 Du stærke, hvi bryster du dig af din Ondskab imod den fromme?
\par 4 Du pønser hele Dagen på ondt; din Tunge er hvas som en Kniv, du Rænkesmed,
\par 5 du foretrækker ondt for godt, Løgn for sanddru Tale. - Sela.
\par 6 Du elsker al ødelæggende Tale, du falske Tunge!
\par 7 Derfor styrte Gud dig for evigt, han gribe dig, rive dig ud af dit Telt, han rykke dig op af de levendes Land! - Sela.
\par 8 De retfærdige ser det, frygter og håner ham leende:
\par 9 "Se der den Mand, der ej gjorde Gud til sit Værn, men stoled på sin megen Rigdom, trodsede på sin Velstand!"
\par 10 Men jeg er som et frodigt Olietræ i Guds Hus, Guds Miskundhed stoler jeg evigt og altid på.
\par 11 Evindelig takker jeg dig, fordi du greb ind; jeg vidner iblandt dine fromme, at godt er dit Navn.

\chapter{53}

\par 1 Til Korherren. Al-mahalat. Maskil af David
\par 2 Dårerne siger i Hjertet: "Der er ingen Gud!" Slet og afskyeligt handler de, ingen gør godt.
\par 3 Gud skuer ned fra Himlen på Menneskenes Børn for at se, om der findes en forstandig, nogen, der søger Gud.
\par 4 Afveget er alle, til Hobe fordærvet, ingen gør godt, end ikke een.
\par 5 Er de Udådsmænd da uden Forstand de, der æder mit Folk, som åd de Brød, og ikke påkalder Gud?
\par 6 Af Rædsel gribes de da, hvor ingen Rædsel var; thi Gud adsplitter din Belejres Ben; de bliver til Skamme, thi Gud forkaster dem.
\par 7 Ak, kom dog fra Zion Israels Frelse! Når Gud vender sit Folks Skæbne, skal Jakob juble, Israel glædes.

\chapter{54}

\par 1 Til Korherren. Til Strengespil. Maskil af David
\par 2 Dengang Folkene fra Zim kom og fortalte Saul, at David skjulte sig hos dem
\par 3 Frels mig o Gud, ved dit navn og skaf mig min ret ved din Vælde,
\par 4 hør, o Gud, min Bøn, lyt til min Munds Ord!
\par 5 Thi frække står op imod mig, Voldsmænd vil tage mit Liv; Gud har de ikke for Øje. - Sela.
\par 6 Se, min Hjælper er Gud, Herren støtter min Sjæl!
\par 7 Det onde vende sig mod mine Fjender, udryd dem i din Trofasthed!
\par 8 Da vil jeg frivilligt ofre til dig, prise dit Navn, o HERRE, thi det er godt;
\par 9 thi det frier mig ud af al Nød; mit Øje skuer med Fryd mine Fjender!

\chapter{55}

\par 1 Til Korherren. Til Strengespil. Maskil af David
\par 2 Lyt, o Gud, til min Bøn, skjul dig ej for min tryglen,
\par 3 lå mig Øre og svar mig, jeg vånder mig i Klage,
\par 4 jeg stønner ved Fjendernes Råb og de gudløses Skrig; thi Ulykke vælter de over mig, forfølger mig grumt;
\par 5 Hjertet er angst i mit Bryst, Dødens Rædsler er faldet over mig.
\par 6 Frygt og Angst falder på mig, Gru er over mig.
\par 7 Jeg siger: Ak, havde jeg Vinger som Duen, da fløj jeg i Ly,
\par 8 ja, langt bort vilde jeg fly og blive i Ørkenen. - Sela.
\par 9 Da søgte jeg skyndsomt Tilflugt for rivende Storm og Uvejr.
\par 10 Herre, forvir og split deres Tungemål! Thi Vold og Ufred ser jeg i Byen;
\par 11 de går Rundgang Dag og Nat på dens Mure;
\par 12 Ulykke, Kvide og Vanheld råder derinde, Voldsfærd og Svig viger aldrig bort fra dens Torve.
\par 13 Det var ikke en Fjende, som hånede mig - det kunde bæres; min uven ydmygede mig ej - ham kunde jeg undgå;
\par 14 men du, en Mand af min Stand, en Ven og fortrolig,
\par 15 og det skønt vi delte Samværets Sødme, vandrede endrægtelig i Guds Hus.
\par 16 Over dem komme Død, lad dem levende synke i Dødsriget! Thi der er Ondskab i deres Bolig, i deres Indre!
\par 17 Jeg, jeg råber til Gud, og HERREN vil frelse mig.
\par 18 Jeg klager og stønner ved Kvæld, ved Gry og ved Middag; min Røst vil han høre
\par 19 og udfri min Sjæl i Fred, så de ikke kan komme mig nær; thi mange er de imod mig.
\par 20 Gud, som troner fra Fortids Dage, vil høre og ydmyge dem. - Sela. Thi der er ingen Forandring hos dem, og de frygter ikke for Gud.
\par 21 På Venner lagde han Hånd og brød sin Pagt.
\par 22 Glattere end Smør er hans Mund, men Hjertet vil Krig, blødere end Olie hans Ord, skønt dragne Sværd.
\par 23 Kast din Byrde på HERREN, så sørger han for dig, den retfærdige lader han ikke i Evighed rokkes.
\par 24 Og du, o Gud, nedstyrt dem i Gravens Dyb! Ej skal blodstænkte, svigefulde Mænd nå Hælvten af deres Dage. Men jeg, jeg stoler på dig!

\chapter{56}

\par 1 Til Korherren. Al-jonat-elem-rehokim. Miktam af David. Dengang Filistrene i Gat greb ham
\par 2 Vær mig nådig Gud, thi Mennesker vil mig til livs, jeg trænges stadig af Stridsmænd;
\par 3 mine Fjender vil mig stadig til Livs, thi mange strider bittert imod mig!
\par 4 Når jeg gribes af Frygt, vil jeg stole på dig,
\par 5 og med Guds Hjælp skal jeg prise hans Ord. Jeg stoler på Gud, jeg frygter ikke, hvad kan Kød vel gøre mig?
\par 6 De oplægger stadig Råd imod mig, alle deres Tanker går ud på ondt.
\par 7 De flokker sig sammen, ligger på Lur, jeg har dem lige i Hælene, de står mig jo efter Livet.
\par 8 Gengæld du dem det onde, stød Folkene ned i Vrede, o Gud!
\par 9 Selv har du talt mine Suk, i din Lædersæk har du gemt mine Tårer; de står jo i din Bog.
\par 10 Da skal Fjenderne vige, den Dag jeg kalder; så meget ved jeg, at Gud er med mig.
\par 11 Med Guds Hjælp skal jeg prise hans Ord, med HERRENs Hjælp skal jeg prise hans Ord.
\par 12 Jeg stoler på Gud, jeg frygter ikke, hvad kan et Menneske gøre mig?
\par 13 Jeg har Løfter til dig at indfri, o Gud, med Takofre vil jeg betale dig.
\par 14 Thi fra Døden frier du min Sjæl, ja min Fod fra Fald, at jeg kan vandre for Guds Åsyn i Livets Lys.

\chapter{57}

\par 1 Til Korherren. Al-tashket. Miktam af David. Dengang han flygtede ind i Hulen for Saul
\par 2 Vær mig nådig, Gud, vær mig nådig, thi hos dig har min Sjæl søgt Ly; i dine Vingers Skygge søger jeg Ly, til Ulykken er drevet over.
\par 3 Gud, den Højeste, påkalder jeg, den Gud, der gør vel imod mig;
\par 4 han sender mig Hjælp fra Himlen og frelser min Sjæl fra dem, som vil mig til Livs. Gud sender sin Nåde og Trofasthed.
\par 5 Jeg må ligge midt iblandt Løver, bo mellem Folk, der spyr Ild, hvis Tænder er Spyd og Pile, hvis Tunge er hvas som et Sværd.
\par 6 Løft dig, o Gud, over Himlen, din Herlighed være over al Jorden!
\par 7 Et Net har de udspændt for mine Skridt, deres egen Fod skal hildes deri; en Grav har de gravet foran mig, selv skal de falde deri. - Sela.
\par 8 Mit Hjerte er trøstigt, Gud, mit Hjerte er trøstigt; jeg vil synge og lovprise dig,
\par 9 vågn op, min Ære! Harpe og Citer vågn op, jeg vil vække Morgenrøden.
\par 10 Jeg vil takke dig, Herre, blandt Folkeslag, prise dig blandt Folkefærd;
\par 11 thi din Miskundhed når til Himlen, din Sandhed til Skyerne.
\par 12 Løft dig, o Gud, over Himlen, din Herlighed være over al Jorden!

\chapter{58}

\par 1 Til Korherren. Al-tashket. Miktam af David
\par 2 Er det virkelig Ret, I taler, I Guder, dømmer I Menneskenes Børn retfærdigt?
\par 3 Nej, alle øver I Uret på Jord, eders Hænder udvejer Vold.
\par 4 Fra Moders Liv vanslægted de gudløse, fra Moders Skød for Løgnerne vild.
\par 5 Gift har de i sig som Slangen, den stumme Øgle, der døver sit Øre
\par 6 og ikke vil høre på Tæmmerens Røst, på den kyndige Slangebesværger.
\par 7 Gud, bryd Tænderne i deres Mund, Ungløvernes Kindtænder knuse du, HERRE;
\par 8 lad dem svinde som Vand, der synker, visne som nedtrampet Græs.
\par 9 Lad dem blive som Sneglen, opløst i Slim som et ufuldbårent Foster, der aldrig så Sol.
\par 10 Før eders Gryder mærker til Tjørnen, ja, midt i deres Livskraft river han dem bort i sin Vrede
\par 11 Den retfærdige glæder sig, når han ser Hævn, hans Fødder skal vade i gudløses Blod;
\par 12 Og Folk skal sige: "Den retfærdige får dog sin Løn, der er dog Guder, som dømmer på Jord!"

\chapter{59}

\par 1 Til Korherren. Al-tashket. Miktam af David. Dengang Saul sendte Folk ud for at bevogte Huset og dræbe ham
\par 2 Fri mig fra mine Fjender, min Gud bjærg mig fra dem, der rejser sig mod mig;
\par 3 fri mig fra Udådsmænd, frels mig fra blodstænkte Mænd!
\par 4 Thi se, de lurer efter min Sjæl, stærke Mænd stimler sammen imod mig, uden at jeg har Skyld eller Brøde.
\par 5 Uden at jeg har forbrudt mig, HERRE, stormer de frem og stiller sig op. Vågn op og kom mig i Møde, se til!
\par 6 Du er jo HERREN, Hærskarers Gud, Israels Gud. Vågn op og hjemsøg alle Folkene, skån ej een af de troløse Niddinger! - Sela.
\par 7 Ved Aften kommer de tilbage, hyler som Hunde og stryger gennem Byen!
\par 8 Se, deres Mund løber over, på deres Læber er Sværd, thi: "Hvem skulde høre det?"
\par 9 Men du, o HERRE, du ler ad dem, du spotter alle Folk,
\par 10 dig vil jeg lovsynge, du, min Styrke, thi Gud er mit Værn;
\par 11 med Nåde kommer min Gud mig i Møde, Gud lader mig se mine Fjender med Fryd!
\par 12 Slå dem ikke ihjel, at ikke mit Folk skal glemme, gør dem hjemløse med din Vælde og styrt dem,
\par 13 giv dem hen, o Herre, i Mundens Synd, i Læbernes Ord, og lad dem hildes i deres Hovmod for de Eder og Løgne, de siger;
\par 14 udryd dem i Vrede, gør Ende på dem, så man kan kende til Jordens Ender, at Gud er Hersker i Jakob! - Sela.
\par 15 Ved Aften kommer de tilbage, hyler som Hunde og stryger gennem Byen,
\par 16 vanker rundt efter Føde og knurrer, når de ikke mættes.
\par 17 Men jeg, jeg vil synge om din Styrke, juble hver Morgen over din Nåde; thi du blev mig et Værn, en Tilflugt på Nødens Dag.
\par 18 Dig vil jeg lovsynge, du, min Styrke, thi Gud er mit Værn, min nådige Gud.

\chapter{60}

\par 1 Til Korherren. Al-shushan-edut. Miktam af David. Til Belæring.
\par 2 Dengang han førte Krig mod Aram-Naharajim og Aram-Soba, og Joab vendte tilbage og slog tolv tusind Edomitter i Saltdalen
\par 3 Gud, du har stødt os fra dig, nedbrudt os, du vredes - vend dig til os igen;
\par 4 du lod Landet skælve, slå Revner, læg nu dets Brist, thi det vakler!
\par 5 Du lod dit Folk friste ondt, iskænked os døvende Vin.
\par 6 Dem, der frygter dig, gav du et Banner, hvorhen de kan fly for Buen. - Sela.
\par 7 Til Frelse for dine elskede hjælp med din højre, bønhør os!
\par 8 Gud talede i sin Helligdom: "Jeg vil udskifte Sikem med jubel, udmåle Sukkots Dal;
\par 9 mit er Gilead, mit er Manasse, Efraim er mit Hoveds Værn, Juda min Herskerstav,
\par 10 Moab min Vaskeskål, på Edom kaster jeg min Sko, over Filisterland jubler jeg."
\par 11 Hvo bringer mig hen til den faste Stad, hvo leder mig hen til Edom?
\par 12 Har du ikke, Gud, stødt os fra dig? Du ledsager ej vore Hære.
\par 13 Giv os dog Hjælp mod Fjenden! Blændværk er Menneskers Støtte.
\par 14 Med Gud skal vi øve vældige Ting, vore Fjender træder han ned!

\chapter{61}

\par 1 Til Korherren. Til Strengespil. Af David
\par 2 Hør, o Gud, på mit råb og lyt til min bøn!
\par 3 Fra Jordens Ende råber jeg til dig. Når mit Hjerte vansmægter, løft mig da op på en Klippe,
\par 4 led mig, thi du er min Tilflugt, et mægtigt Tårn til Værn imod Fjenden.
\par 5 Lad mig evigt bo som Gæst i dit Telt, finde Tilflugt i dine Vingers Skjul! - Sela.
\par 6 Ja du, o Gud, har hørt mine Løfter, opfyldt deres Ønsker, der frygter dit Navn.
\par 7 Til Kongens Dage lægger du Dage, hans År skal være fra Slægt til Slægt.
\par 8 Han skal trone evigt for Guds Åsyn; send Nåde og Sandhed til at bevare ham!
\par 9 Da vil jeg evigt love dit Navn og således Dag efter Dag indfri mine Løfter.

\chapter{62}

\par 1 Til Korherren. Efter Jedutun. Salme af David
\par 2 Min Sjæl er Stille for Gud alene, min Frelse kommer fra ham;
\par 3 ja, han er min Klippe, min Frelse, mit Værn, jeg skal ikke rokkes meget.
\par 4 Hvor længe stormer I løs på en Mand, - alle slår I ham ned - som på en hældende Væg, en faldende Mur?
\par 5 Ja, de oplægger Råd om at styrte ham fra hans Højhed. De elsker Løgn, velsigner med Munden, men forbander i deres Indre. - Sela.
\par 6 Vær stille hos Gud alene, min Sjæl, thi fra ham kommer mit Håb;
\par 7 ja, han er min Klippe, min Frelse, mit Værn, jeg skal ikke rokkes.
\par 8 Hos Gud er min Hjælp og min Ære, min stærke Klippe, min Tilflugt har jeg i Gud;
\par 9 stol på ham, al Folkets Forsamling, udøs for ham eders Hjerte, Gud er vor Tilflugt. - Sela.
\par 10 Kun Tomhed er Mennesker, Mænd en Løgn, på Vægtskålen vipper de op, de er Tomhed til Hobe.
\par 11 Forlad eder ikke på vold, lad jer ikke blænde af Ran; om Rigdommen vokser, agt ikke derpå!
\par 12 Een Gang talede Gud, to Gange hørte jeg det: at Magten er Guds,
\par 13 Og Miskundhed er hos dig, o Herre. Thi enhver gengælder du efter hans Gerning.

\chapter{63}

\par 1 Salme af David. Dengang han var i Judas ørken
\par 2 Gud, du er min Gud, dig søger jeg, efter dig tørster min Sjæl, efter dig længes mit Kød i et tørt, vansmægtende,vandløst Land
\par 3 (således var det, jeg så dig i Helligdommen) for at skue din Vælde og Ære;
\par 4 thi din Nåde er bedre end Liv, mine Læber skal synge din Pris.
\par 5 Da vil jeg love dig hele mit Liv, opløfte Hænderne i dit Navn,
\par 6 Som med fede Retter mættes min Sjæl, med jublende Læber priser min Mund dig,
\par 7 når jeg kommer dig i Hu på mit Leje, i Nattevagterne tænker på dig;
\par 8 thi du er blevet min Hjælp, og jeg jubler i dine Vingers Skygge.
\par 9 Dig klynger min Sjæl sig til, din højre holder mig fast.
\par 10 Forgæves står de mig efter livet, i Jordens Dyb skal de synke,
\par 11 gives i Sværdets Vold og vorde Sjakalers Bytte.
\par 12 Men Kongen glædes i Gud; enhver, der sværger ved ham, skal juble, thi Løgnernes Mund skal lukkes.

\chapter{64}

\par 1 Til Korherren. Salme af David
\par 2 Hør, o Gud, min røst, når jeg klager, skærm mit Liv mod den rædsomme Fjende;
\par 3 skjul mig for Ugerningsmændenes Råd, for Udådsmændenes travle Hob.
\par 4 der hvæsser Tungen som Sværd, lægger giftige Ord på Buen
\par 5 for i Løn at ramme den skyldfri, ramme ham brat og uset.
\par 6 Ihærdigt lægger de onde Råd, skryder af, at de lægger Snarer siger: "Hvem skulde se os?"
\par 7 De udtænker onde Gerninger, fuldfører en gennemtænkt Tanke - og Menneskets Indre og Hjerte er dybt.
\par 8 Da rammer Gud dem med en Pil af Slaget rammes de brat;
\par 9 han styrter dem for deres Tunges Skyld. Enhver, som ser dem, ryster på Hovedet;
\par 10 alle Mennesker frygter, forkynder, hvad Gud har gjort, og fatter hans Hænders Geming;
\par 11 de retfærdige glædes i HERREN og lider på ham, de oprigtige af Hjertet jubler til Hobe!

\chapter{65}

\par 1 Til Korherren. Salme af David. En Sang
\par 2 Lovsang tilkommer dig på Zion,o Gud, dig indfrier man Løfter, du, som hører Bønner;
\par 3 alt Kød kommer til dig, når Brøden tynger.
\par 4 Vore Overtrædelser blev os for svare, du tilgiver dem.
\par 5 Salig den, du udvælger, lader bo i dine Forgårde! Vi mættes af dit Huses Rigdom, dit Tempels Hellighed.
\par 6 Du svarer os underfuldt i Retfærd, vor Frelses Gud, du Tilflugt for den vide Jord, for fjerne Strande,
\par 7 du, som grundfæster Bjerge med Vælde, omgjorde med Kraft,
\par 8 du, som dæmper Havenes Brusen, deres Bølgers Brusen og Folkefærds Larm,
\par 9 så Folk ved Verdens Ende gruer for dine Tegn; hvor Morgen og Aften oprinder, bringer du Jubel.
\par 10 Du så til Landet, vanded det, gjorde det såre rigt, Guds Bæk er fuld af Vand, du bereder dets Korn,
\par 11 du vander dets Furer, jævner knoldene, bløder det med Regn, velsigner dets Sæd.
\par 12 Med din Herlighed kroner du Året, dine Vognspor flyder af Fedme;
\par 13 de øde Græsgange flyder, med Jubel omgjordes Højene;
\par 14 Engene klædes med Får, Dalene hylles i Korn, i Jubel bryder de ud og synger!

\chapter{66}

\par 1 Bryd ud i Jubel for Gud, al Jorden,
\par 2 lovsyng hans Navns Ære, syng ham en herlig Lovsang,
\par 3 sig til Gud: "Hvor forfærdelige er dine Gerninger! For din vældige Styrkes Skyld logrer Fjenderne for dig,
\par 4 al Jorden tilbeder dig, de lovsynger dig, lovsynger dit Navn." - Sela.
\par 5 Kom hid og se, hvad Gud har gjort i sit Virke en Rædsel for Menneskenes Børn.
\par 6 Han forvandlede Hav til Land, de vandrede til Fods over Strømmen; lad os fryde os højlig i ham.
\par 7 Han hersker med Vælde for evigt, på Folkene vogter hans Øjne, ej kan genstridige gøre sig store. - Sela.
\par 8 I Folkeslag, lov vor Gud, lad lyde hans Lovsangs Toner,
\par 9 han, som har holdt vor Sjæl i Live og ej lod vor Fod glide ud!
\par 10 Thi du ransaged os, o Gud, rensede os, som man renser Sølv;
\par 11 i Fængsel bragte du os, lagde Tynge på vore Lænder,
\par 12 lod Mennesker skride hen over vort Hoved, vi kom gennem Ild og Vand; men du førte os ud og bragte os Lindring!
\par 13 Med Brændofre vil jeg gå ind i dit Hus og indfri dig mine Løfter,
\par 14 dem, mine Læber fremførte, min Mund udtalte i Nøden.
\par 15 Jeg bringer dig Ofre af Fedekvæg sammen med Vædres Offerduft, jeg ofrer Okser tillige med Bukke. - Sela.
\par 16 Kom og hør og lad mig fortælle jer alle, som frygter Gud, hvad han har gjort for min Sjæl!
\par 17 Jeg råbte til ham med min Mund og priste ham med min Tunge.
\par 18 Havde jeg tænkt på ondt i mit Hjerte, da havde Herren ej hørt;
\par 19 visselig, Gud har hørt, han lytted til min bedende Røst.
\par 20 Lovet være Gud, som ikke har afvist min Bøn eller taget sin Miskundhed fra mig!

\chapter{67}

\par 1 Til Korherren. Til Strengespil En Salme.  En Sang
\par 2 Gud være os nådig og velsigne os, han lade sit Ansigt lyse over os - Sela -
\par 3 for at din Vej må kendes på Jorden, din Frelse blandt alle Folk.
\par 4 Folkeslag skal takke dig, Gud, alle Folkeslag takke dig;
\par 5 Folkefærd skal glædes og juble, thi med Retfærd dømmer du Folkeslag, leder Folkefærd på Jorden, - Sela.
\par 6 Folkeslag skal takke dig Gud, alle Folkeslag takke dig!
\par 7 Landet har givet sin Grøde, Gud, vor Gud, velsigne os,
\par 8 Gud velsigne os, så den vide Jord må frygte ham!

\chapter{68}

\par 1 Til Korherren. Salme af David. En Sang
\par 2 Når Gud står op, da splittes hans fjender, hans Avindsmænd flyr for hans Åsyn,
\par 3 som Røg henvejres, så henvejres de; som Voks, der smelter for Ild, går gudløse til Grunde for Guds Åsyn.
\par 4 Men retfærdige frydes og jubler med Glæde og Fryd for Guds Åsyn.
\par 5 Syng for Gud, lovsyng hans Navn, hyld ham, der farer frem gennem Ørknerne! HERREN er hans Navn, jubler for hans Åsyn,
\par 6 faderløses Fader, Enkers Værge, Gud i hans hellige Bolig,
\par 7 Gud, som bringer ensomme hjem, fører Fanger ud til Lykke; men genstridige bor i tørre Egne.
\par 8 Da du drog ud, o Gud, i Spidsen for dit Folk, skred frem gennem Ørkenen - Sela - da rystede Jorden,
\par 9 ja, Himlen dryppede for Guds Åsyn, for Guds Åsyn, Israels Guds.
\par 10 Regn i Strømme lod du falde, o Gud, din vansmægtende Arvelod styrkede du;
\par 11 din Skare tog Bolig der, for de arme sørged du, Gud, i din Godhed,
\par 12 Ord lægger Herren de Kvinder i Munden, som bringer Glædesbud, en talrig Hær:
\par 13 "Hærenes Konger flyr, de flyr, Husets Frue uddeler Bytte.
\par 14 Vil l da blive imellem Foldene? Duens Vinger dækkes af Sølv, dens Fjedre af gulgrønt Guld.
\par 15 Da den Almægtige splittede Kongerne der, faldt der Sne på Zalmon."
\par 16 Et Gudsbjerg er Basans Bjerg, et Bjerg med spidse Tinder er Basans Bjerg;
\par 17 Hvi skæver I Bjerge med spidse Tinder til Bjerget, Gud ønskede til Bolig, hvor HERREN også vil bo for evigt?
\par 18 Titusinder er Guds Vogne, tusinde Gange tusinde, HERREN kom fra Sinaj til Helligdommen.
\par 19 Du steg op til det høje, du bortførte Fanger, Gaver tog du blandt Mennesker, også iblandt de genstridige, at du måtte bo der, HERRE, o Gud.
\par 20 Lovet være Herren! Fra Dag til Dag bærer han vore Byrder; Gud er vor Frelse. - Sela.
\par 21 En Gud til Frelse er Gud for os, hos den Herre HERREN er Udgange fra Døden.
\par 22 Men Fjendernes Hoveder knuser Gud, den gudløses Isse, der vandrer i sine Synder.
\par 23 Herren har sagt: "Jeg henter dem hjem fra Basan, henter dem hjem fra Havets Dyb,
\par 24 at din Fod må vade i Blod, dine Hundes Tunger få del i Fjenderne."
\par 25 Se på Guds Højtidstog, min Guds, min Konges Højtidstog ind i Helligdommen!
\par 26 Sangerne forrest, så de, der spiller, i Midten unge Piger med Pauker:
\par 27 "Lover Gud i Festforsamlinger, Herren, I af Israels Kilde!"
\par 28 Der er liden Benjamin forrest, Judas Fyrster i Flok, Zebulons Fyrster, Naftalis Fyrster.
\par 29 Opbyd, o Gud, din Styrke, styrk, hvad du gjorde for os, o Gud!
\par 30 For dit Tempels Skyld skal Konger bringe dig Gaver i Jerusalem.
\par 31 Tru ad Dyret i Sivet, Tyreflokken, Folkeslags Herrer, så de hylder dig med deres Sølvstykker. Adsplit Folkeslag, der elsker Strid!
\par 32 De kommer med Olie fra Ægypten, Ætiopeme iler til Gud med fulde Hænder.
\par 33 I Jordens Riger, syng for Gud, lovsyng HERREN;
\par 34 hyld ham der farer frem på Himlenes Himle, de gamle! Se, han løfter sin Røst, en vældig Røst.
\par 35 Giv Gud Ære! Over Israel er hans Højhed, Hans Vælde i Skyerne,
\par 36 frygtelig er Gud i sin Helligdom. Israels Gud; han giver Folket Styrke og Kraft. Lovet være Gud!

\chapter{69}

\par 1 Til Korherren. Al-shoshannim. Af David
\par 2 Frels mig Gud, thi Vandene når mig til Sjælen,
\par 3 jeg er sunket i bundløst Dynd, hvor der intet Fodfæste er, kommet i Vandenes Dyb, og Strømmen går over mig;
\par 4 træt har jeg skreget mig, Struben brænder, mit Øje er mat af at bie på min Gud;
\par 5 flere end mit Hoveds Hår er de, der hader mig uden Grund, mange er de, som vil mig til Livs, uden Skel er mig fjendske; hvad jeg ikke har ranet, skal jeg dog erstatte!
\par 6 Gud, du kender min Dårskab, min Skyld er ej skjult for dig.
\par 7 Lad mig ej bringe Skam over dem, som bier på dig, o Herre, Hærskarers HERRE, lad mig ej bringe Skændsel over dem der søger dig, Israels Gud!
\par 8 Thi for din Skyld bærer jeg Spot, mit Åsyn dækkes af Skændsel;
\par 9 fremmed er jeg for mine Brødre en Udlænding for min Moders Sønner.
\par 10 Thi Nidkærhed for dit Hus har fortæret mig, Spotten mod dig er faldet på mig:
\par 11 jeg spæged min Sjæl med Faste, og det blev mig til Spot;
\par 12 i Sæk har jeg klædt mig, jeg blev dem et Mundheld.
\par 13 De, der sidder i Porten, taler om mig, ved Drikkelagene synger de om mig.
\par 14 Men jeg beder, HERRE, til dig i Nådens Tid, o Gud, i din store Miskundhed svare du mig!
\par 15 Frels mig med din trofaste Hjælp fra Dyndet, at jeg ikke skal synke; red mig fra dem, der hader mig, fra Vandenes Dyb,
\par 16 lad Strømmen ikke gå over mig; lad Dybet ikke sluge mig eller Brønden lukke sig over mig.
\par 17 Svar mig, HERRE, thi god er din Nåde, vend dig til mig efter din store Barmhjertighed;
\par 18 dit Åsyn skjule du ej for din Tjener, thi jeg er i Våde, skynd dig og svar mig;
\par 19 kom til min Sjæl og løs den, fri mig for mine Fjenders Skyld!
\par 20 Du ved, hvorledes jeg smædes og bærer Skam og Skændsel; du har Rede på alle mine Fjender.
\par 21 Spot har ulægeligt knust mit Hjerte; jeg bied forgæves på Medynk, på Trøstere uden at finde;
\par 22 de gav mig Malurt at spise og slukked min Tørst med Eddike.
\par 23 Lad Bordet foran dem blive en Snare, deres Takofre blive en Fælde;
\par 24 lad Øjnene slukkes, så Synet svigter, lad Lænderne altid vakle!
\par 25 Din Vrede udøse du over dem din glødende Harme nå dem;
\par 26 deres Teltlejr blive et Øde, og ingen bo i deres Telte!
\par 27 Thi de forfølger den, du slog, og øger Smerten for dem, du såred.
\par 28 Tilregn dem hver eneste Brøde lad dem ikke få Del i din Retfærd;
\par 29 lad dem slettes af Livets Bog, ej optegnes blandt de retfærdige!
\par 30 Men mig, som er arm og lidende, bjærge din Frelse, o Gud!
\par 31 Jeg vil prise Guds Navn med Sang og ophøje ham med Tak;
\par 32 det er mer for HERREN end Okser end Tyre med Horn og Klove!
\par 33 Når de ydmyge ser det, glæder de sig; I, som søger Gud, eders Hjerte oplives!
\par 34 Thi HERREN låner de fattige Øre, han agter ej fangne Venner ringe.
\par 35 Himmel og Jord skal prise ham, Havet og alt, hvad der rører sig der;
\par 36 thi Gud vil frelse Zion og opbygge Judas Byer; der skal de bo og tage det i Eje;
\par 37 hans Tjeneres Afkom skal arve det, de, der elsker hans Navn, skal bo deri.

\chapter{70}

\par 1 Til Korherren. Af David. Le-hazkir
\par 2 Du værdiges, Gud, at fri mig, Herre, il mig til Hjælp!
\par 3 Lad dem beskæmmes og røme, som vil mig til Livs, og de, der ønsker mig ondt, lad dem vige med Skændsel;
\par 4 lad dem stivne af Rædsel ved deres Skam, de, som siger: "Ha, ha!"
\par 5 Lad alle, som søger dig, frydes og glædes i dig; lad dem, som elsker din Frelse, bestandig sige: "Gud er stor!"
\par 6 Arm og fattig er jeg, il mig til Hjælp, o Gud! Du er min Hjælp og min Frelser; tøv ej, HERRE!

\chapter{71}

\par 1 HERRE, jeg lider på dig, lad mig aldrig i evighed skuffes.
\par 2 Frels mig og udfri mig i din Retfærdighed, du bøjede dit Øre til mig;
\par 3 red mig og vær mig en Tilflugtsklippe, en Klippeborg til min Frelse; thi du er min Klippe og Borg!
\par 4 Min Gud, fri mig ud af gudløses Hånd, af Niddings og Voldsmands Kløer;
\par 5 thi du er mit Håb, o Herre! Fra min Ungdom var HERREN min Tillid;
\par 6 fra Moders Skød har jeg støttet mig til dig, min Forsørger var du fra Moders Liv, dig gælder altid min Lovsang.
\par 7 For mange står jeg som mærket af Gud, men du er min stærke Tilflugt;
\par 8 min Mund er fuld af din Lovsang, af din Ære Dagen lang.
\par 9 Forkast mig ikke i Alderdommens Tid og svigt mig ikke, nu Kraften svinder;
\par 10 thi mine Fjender taler om mig, de der lurer på min Sjæl, holder Råd:
\par 11 "Gud har svigtet ham! Efter ham! Grib ham, thi ingen frelser!"
\par 12 Gud, hold dig ikke borte fra mig, il mig til Hjælp, min Gud;
\par 13 lad dem blive til Skam og Skændsel, dem, der står mig imod, lad dem hylles i Spot og Spe, dem, der vil mig ondt!
\par 14 Men jeg, jeg vil altid håbe, blive ved at istemme din Pris;
\par 15 min Mund skal vidne om din Retfærd, om din Frelse Dagen lang; thi jeg kender ej Ende derpå.
\par 16 Jeg vil minde om den Herre HERRENs Vælde, lovsynge din Retfærd, kun den alene.
\par 17 Gud, du har vejledt mig fra min Ungdom af, dine Undere har jeg forkyndt til nu;
\par 18 indtil Alderdommens Tid og de grånende Hår svigte du mig ikke, o Gud. End skal jeg prise din Arm for alle kommende Slægter.
\par 19 Din Vælde og din Retfærdighed når til Himlen, o Gud; du, som øvede store Ting, hvo er din Lige, Gud?
\par 20 Du, som lod os skue mange fold Trængsel og Nød, du kalder os atter til Live og drager os atter af Jordens Dyb;
\par 21 du vil øge min Storhed og atter trøste mig.
\par 22 Til Gengæld vil jeg til Harpespil prise din Trofasthed, min Gud, lege på Citer for dig, du Israels Hellige;
\par 23 juble skal mine Læber - ja, jeg vil lovsynge dig og min Sjæl, som du udløste;
\par 24 også min Tunge skal Dagen igennem forkynde din Retfærd, thi Skam og Skændsel får de, som vil mig ilde.

\chapter{72}

\par 1 Gud, giv Kongen din ret, Kongesønnen din retfærd,
\par 2 så han dømmer dit Folk med Retfærdighed og dine arme med Ret!
\par 3 Da bærer Bjerge og Høje Fred for Folket i Retfærd.
\par 4 De arme blandt Folket skaffer han Ret, han bringer de fattige Frelse, og han slår Voldsmanden ned.
\par 5 Han skal leve, så længe Solen lyser og Månen skinner, fra Slægt til Slægt.
\par 6 Han kommer som Regn på slagne Enge, som Regnskyl, der væder Jorden;
\par 7 i hans dage blomstrer Retfærd, og dyb Fred råder, til Månen forgår.
\par 8 Fra Hav til Hav skal han herske, fra Floden til Jordens Ender;
\par 9 hans Avindsmænd bøjer knæ for ham, og hans Fjender slikker Støvet;
\par 10 Konger fra Tarsis og fjerne Strande frembærer Gaver, Sabas og Sebas Konger kommer med Skat;
\par 11 alle Konger skal bøje sig for ham, alle Folkene være hans Tjenere.
\par 12 Thi han skal redde den fattige, der skriger om Hjælp, den arme, der savner en Hjælper,
\par 13 ynkes over ringe og fattig og frelse fattiges Sjæle;
\par 14 han skal fri deres Sjæle fra Uret og vold, deres Blod er dyrt i hans Øjne.
\par 15 Måtte han leve og Guld fra Saba gives ham! De skal bede for ham bestandig, velsigne ham Dagen igennem.
\par 16 Korn skal der være i Overflod i Landet, på Bjergenes Top; som Libanon skal dets Afgrøde bølge og Folk spire frem af Byen som Jordens Urter.
\par 17 Velsignet være hans Navn evindelig, hans Navn skal leve, mens Solen skinner. Ved ham skal man velsigne sig, alle Folk skal prise ham lykkelig!
\par 18 Lovet være Gud HERREN, Israels Gud som ene gør Undergerninger,
\par 19 og lovet være hans herlige Navn evindelig; al Jorden skal fyldes af hans Herlighed. Amen, Amen!
\par 20 Her ender Davids, Isajs Søns, Bønner.

\chapter{73}

\par 1 Visselig, god er Gud mod Israel; mod dem, der er rene af Hjertet!
\par 2 Mine Fødder var nær ved at snuble, mine Skridt var lige ved at glide;
\par 3 thi over Dårerne græmmed jeg mig, jeg så, at det gik de gudløse vel;
\par 4 thi de kender ikke til Kvaler, deres Livskraft er frisk og sund;
\par 5 de kender ikke til menneskelig Nød, de plages ikke som andre.
\par 6 Derfor har de Hovmod til Halssmykke, Vold er Kappen, de svøber sig i.
\par 7 Deres Brøde udgår af deres Indre, Hjertets Tanker bryder igennem.
\par 8 I det dybe taler de ondt, i det høje fører de Urettens Tale,
\par 9 de løfter Munden mod Himlen, Tungen farer om på Jorden.
\par 10 Derfor vender mit Folk sig hid og drikker Vand i fulde Drag.
\par 11 De siger: "Hvor skulde Gud vel vide det, skulde den Højeste kende dertil?"
\par 12 Se, det er de gudløses kår, altid i Tryghed, voksende Velstand!
\par 13 Forgæves holdt jeg mit Hjerte rent og tvætted mine Hænder i Uskyld,
\par 14 jeg plagedes Dagen igennem, blev revset på ny hver Morgen!
\par 15 Men jeg tænkte: "Taler jeg så, se, da er jeg troløs imod dine Sønners Slægt."
\par 16 Så grundede jeg på at forstå det, møjsommeligt var det i mine Øjne,
\par 17 Til jeg kom ind i Guds Helligdomme, skønned, hvordan deres Endeligt bliver:
\par 18 Du sætter dem jo på glatte Steder, i Undergang styrter du dem.
\par 19 Hvor brat de dog lægges øde, går under, det ender med Rædsel!
\par 20 De er som en Drøm, når man vågner, man vågner og regner sit Syn for intet.
\par 21 Så længe mit Hjerte var bittert og det nagede i mine Nyrer,
\par 22 var jeg et Dyr og fattede intet, jeg var for dig som Kvæg.
\par 23 Dog bliver jeg altid hos dig, du holder mig fast om min højre;
\par 24 du leder mig med dit Råd og tager mig siden bort i Herlighed.
\par 25 Hvem har jeg i Himlen? Og har jeg blot dig, da attrår jeg intet på Jorden!
\par 26 Lad kun mit Kød og mit Hjerte vansmægte, Gud er mit Hjertes Klippe, min Del for evigt.
\par 27 Thi de, der fjerner sig fra dig, går under, - du udsletter hver, som er dig utro.
\par 28 Men at leve Gud nær er min Lykke, min Lid har jeg sat til den Herre HERREN, at jeg kan vidne om alle dine Gerninger.

\chapter{74}

\par 1 Hvorfor har du, Gud, stødt os bort for evig, hvi ryger din Vrede mod Hjorden, du røgter?
\par 2 Kom din Menighed i Hu, som du fordum vandt dig, - du udløste den til din Ejendoms Stamme - Zions Bjerg, hvor du har din Bolig.
\par 3 Løft dine Fjed til de evige Tomter: Fjenden lagde alt i Helligdommen øde.
\par 4 Dine Fjender brøled i dit Samlingshus, satte deres Tegn som Tegn deri.
\par 5 Det så ud, som når man løfter Økser i Skovens Tykning.
\par 6 Og alt det udskårne Træværk der! De hugged det sønder med Økse og Hammer.
\par 7 På din Helligdom satte de Ild, de skændede og nedrev dit Navns Bolig.
\par 8 De tænkte: "Til Hobe udrydder vi dem!" De brændte alle Guds Samlingshuse i Landet.
\par 9 Vore Tegn, dem ser vi ikke, Profeter findes ej mer; hvor længe, ved ingen af os.
\par 10 Hvor længe, o Gud, skal vor Modstander smæde, Fjenden blive ved at håne dit Navn?
\par 11 Hvorfor holder du din Hånd tilbage og skjuler din højre i Kappens Fold?
\par 12 Vor Konge fra fordums Tid er dog Gud, som udførte Frelsens Værk i Landet.
\par 13 Du kløvede Havet med Vælde, knuste på Vandet Dragernes Hoved;
\par 14 du søndrede Hovederne på Livjatan og gav dem som Æde til Ørkenens Dyr;
\par 15 Kilde og Bæk lod du vælde frem, du udtørred stedseflydende Strømme;
\par 16 din er Dagen, og din er Natten, du grundlagde Lys og Sol,
\par 17 du fastsatte alle Grænser på Jord, du frembragte Sommer og Vinter.
\par 18 Kom i Hu, o HERRE, at Fjenden har hånet, et Folk af Dårer har spottet dit Navn!
\par 19 Giv ikke Vilddyret din Turteldues Sjæl, glem ikke for evigt dine armes Liv;
\par 20 se hen til Pagten, thi fyldte er Landets mørke Steder med Voldsfærds Boliger.
\par 21 Lad ej den fortrykte gå bort med Skam, lad de arme og fattige prise dit Navn!
\par 22 Gud, gør dig rede, før din Sag, kom i Hu, hvor du stadig smædes af bårer,
\par 23 lad ej dine Avindsmænds Røst uænset! Ustandseligt lyder dine Fjenders Larm!

\chapter{75}

\par 1 Til Korherren. Al-tashket. Salme af Asaf. En Sang
\par 2 Vi takker dig, Gud, vi takker dig; de, der påkalder dit Navn, fortæller dine Undere.
\par 3 "Selv om jeg udsætter Sagen, dømmer jeg dog med Retfærd;
\par 4 vakler end Jorden og alle, som bor derpå, har jeg dog grundfæstet dens Støtter." - Sela.
\par 5 Til Dårerne siger jeg: "Vær ej Dårer!" og til de gudløse: "Løft ej Hornet,
\par 6 løft ikke eders Horn mod Himlen, tal ikke med knejsende Nakke!"
\par 7 Thi hverken fra Øst eller Vest kommer Hjælp, ej heller fra Ørk eller Bjerge.
\par 8 Nej, den, som dømmer, er Gud, han nedbøjer en, ophøjer en anden.
\par 9 Thi i HERRENs Hånd er et Bæger med skummende, krydet Vin. han skænker i for een efter een, selv Bærmen drikker de ud; alle Jordens gudløse drikker.
\par 10 Men jeg skal juble evindelig, lovsynge Jakobs Gud;
\par 11 alle de gudløses Horn stødes af, de retfærdiges Horn skal knejse!

\chapter{76}

\par 1 Til Korherren. Til Strengespil. Salme af Asaf. En Sang
\par 2 Gud er kendt i Juda, hans navn er stort i Israel,
\par 3 i Salem er hans Hytte, hans Bolig er på Zion.
\par 4 Der brød han Buens Lyn, skjold og Sværd og Krigsværn. - Sela.
\par 5 Frygtelig var du, herlig på de evige Bjerge.
\par 6 De tapre gjordes til Bytte, i Dvale sank de, og kraften svigted alle de stærke Kæmper.
\par 7 Jakobs Gud, da du truede, faldt Vogn og Hest i den dybe Søvn.
\par 8 Frygtelig er du! Hvo holder Stand mod dig i din Vredes Vælde?
\par 9 Fra Himlen fældte du Dom. Jorden grued og tav,
\par 10 da Gud stod op til Dom for at frelse hver ydmyg på Jord. - Sela.
\par 11 Thi Folkestammer skal takke dig, de sidste af Stammerne fejre dig.
\par 12 Aflæg Løfter og indfri dem for HERREN eders Gud, alle omkring ham skal bringe den Frygtindgydende Gaver.
\par 13 Han kuer Fyrsternes Mod, indgyder Jordens Konger Frygt.

\chapter{77}

\par 1 Til Korherren.Efter Jedutun. Salme af Asaf
\par 2 Jeg råber, højt til Gud, og han hører mig,
\par 3 jeg søger Herren på Nødens Dag, min Hånd er om Natten utrættet udrakt, min Sjæl vil ikke lade sig trøste;
\par 4 jeg ihukommer Gud og stønner, jeg sukker, min Ånd vansmægter. - Sela.
\par 5 Du holder mine Øjne vågne, jeg er urolig og målløs.
\par 6 Jeg tænker på fordums dage, ihukommer længst henrundne År;
\par 7 jeg gransker om Natten i Hjertet, grunder og ransager min Ånd.
\par 8 Vil Herren bortstøde for evigt og aldrig mer vise Nåde,
\par 9 er hans Miskundhed ude for stedse, hans Trofasthed omme for evigt og altid,
\par 10 har Gud da glemt at ynkes, lukket sit Hjerte i Vrede? - Sela.
\par 11 Jeg sagde: Det er min Smerte; at den Højestes højre er ikke som før.
\par 12 Jeg kommer HERRENs Gerninger i Hu, ja kommer dine fordums Undere i Hu.
\par 13 Jeg tænker på al din Gerning og grunder over dine Værker.
\par 14 Gud, din Vej var i Hellighed, hvo er en Gud så stor som Gud!
\par 15 Du er en Gud, som gør Undere, du gjorde din Vælde kendt blandt Folkene,
\par 16 udøste dit Folk med din Arm, Jakobs og Josefs Sønner. - Sela.
\par 17 Vandene så dig, Gud, Vandene så dig og vred sig i Angst, ja Dybet tog til at skælve;
\par 18 Skyerne udøste Vand, Skyhimlens Stemme gjaldede, dine Pile for hid og did;
\par 19 din bragende Torden rullede, Lynene oplyste Jorderig, Jorden bæved og skjalv;
\par 20 din Vej gik midt gennem Havet, din Sti gennem store Vande, dine Fodspor kendtes ikke.
\par 21 Du førte dit Folk som en Hjord ved Moses's og Arons Hånd.

\chapter{78}

\par 1 Lyt, mit folk til min lære, bøj eders øre til ord fra min Mund;
\par 2 jeg vil åbne min Mund med Billedtale, fremsætte Gåder fra fordums Tid,
\par 3 hvad vi har hørt og ved, hvad vore Fædre har sagt os;
\par 4 vi dølger det ikke for deres Børn, men melder en kommende Slægt om HERRENs Ære og Vælde og Underne, som han har gjort.
\par 5 Han satte et Vidnesbyrd i Jakob, i Israel gav han en Lov, idet han bød vore Fædre at lade deres Børn det vide,
\par 6 at en senere Slægt kunde vide det, og Børn, som fødtes siden, stå frem og fortælle deres Børn derom,
\par 7 så de slår deres Lid til Gud og ikke glemmer Guds Gerninger, men overholder hans Bud,
\par 8 ej slægter Fædrene på, en vanartet, stridig Slægt, hvis Hjerte ikke var fast, hvis Ånd var utro mod Gud
\par 9 - Efraims Børn var rustede Bueskytter, men svigted på Stridens Dag -
\par 10 Gudspagten holdt de ikke, de nægtede at følge hans Lov;
\par 11 hans Gerninger gik dem ad Glemme, de Undere, han lod dem skue.
\par 12 Han gjorde Undere for deres Fædre i Ægypten på Zoans Mark;
\par 13 han kløvede Havet og førte dem over, lod Vandet stå som en Vold;
\par 14 han ledede dem ved Skyen om Dagen, Natten igennem ved Ildens Skær;
\par 15 han kløvede Klipper i Ørkenen, lod dem rigeligt drikke som af Strømme,
\par 16 han lod Bække rinde af Klippen og Vand strømme ned som Floder.
\par 17 Men de blev ved at synde imod ham og vække den Højestes Vrede i Ørkenen;
\par 18 de fristede Gud i Hjertet og krævede Mad til at stille Sulten,
\par 19 de talte mod Gud og sagde: "Kan Gud dække Bord i en Ørken?
\par 20 Se, Klippen slog han, så Vand flød frem, og Bække vælded ud; mon han også kan give Brød og skaffe kød til sit Folk?"
\par 21 Det hørte HERREN, blev vred, der tændtes en Ild mod Jakob, ja Vrede kom op mod Israel,
\par 22 fordi de ikke troede Gud eller stolede på hans Frelse.
\par 23 Da bød han Skyerne oventil, lod Himlens Døre åbne
\par 24 og Manna regne på dem til Føde, han gav dem Himmelkorn;
\par 25 Mennesker spiste Englebrød, han sendte dem Mad at mætte sig med.
\par 26 Han rejste Østenvinden på Himlen, førte Søndenvinden frem ved sin Kraft;
\par 27 Kød lod han regne på dem som Støv og vingede Fugle som Havets Sand,
\par 28 lod dem falde midt i sin Lejr, rundt omkring sine Boliger;
\par 29 Og de spiste sig overmætte, hvad de ønskede, lod han dem få.
\par 30 Men før deres Attrå var stillet, mens Maden var i deres Mund,
\par 31 rejste Guds Vrede sig mod dem; han vog deres kraftige Mænd, fældede Israels Ynglinge.
\par 32 Og dog blev de ved at synde og troede ej på hans Undere.
\par 33 Da lod han deres Dage svinde i Tomhed og endte brat deres År.
\par 34 Når han vog dem, søgte de ham, vendte om og spurgte om Gud,
\par 35 kom i Hu, at Gud var deres Klippe, Gud den Allerhøjeste deres Genløser.
\par 36 De hyklede for ham med Munden, løj for ham med deres Tunge;
\par 37 deres Hjerter holdt ikke fast ved ham, hans Pagt var de ikke tro.
\par 38 Og dog er han barmhjertig, han tilgiver Misgerning, lægger ej øde, hans Vrede lagde sig Gang på Gang, han lod ikke sin Harme fuldt bryde frem;
\par 39 han kom i Hu, de var Kød, et Pust, der svinder og ej vender tilbage.
\par 40 Hvor tit stod de ham ikke imod i Ørkenen og voldte ham Sorg i det øde Land!
\par 41 De fristede alter Gud, de krænkede Israels Hellige;
\par 42 hans Hånd kom de ikke i Hu, de Dag han friede dem fra Fjenden,
\par 43 da han gjorde sine Tegn i Ægypten, sine Undere på Zoans Mark,
\par 44 forvandlede deres Floder til Blod, så de ej kunde drikke af Strømmene,
\par 45 sendte Myg imod dem, som åd dem, og Frøer, som lagde dem øde,
\par 46 gav Æderen, hvad de avlede, Græshoppen al deres Høst,
\par 47 slog deres Vinstokke ned med Hagl, deres Morbærtræer med Frost,
\par 48 prisgav Kvæget for Hagl og deres Hjorde for Lyn.
\par 49 Han sendte sin Vredesglød mod dem, Harme, Vrede og Trængsel, en Sendefærd af Ulykkesengle;
\par 50 frit Løb gav han sin Vrede, skånede dem ikke for Døden, gav deres Liv til Pris for Pest;
\par 51 alt førstefødt i Ægypten slog han, Mandskraftens Førstegrøde i Kamiternes Telte,
\par 52 lod sit Folk bryde op som en Hjord, ledede dem som Kvæg i Ørkenen,
\par 53 ledede dem trygt, uden Frygt, mens Havet lukked sig over deres Fjender;
\par 54 han bragte dem til sit hellige Land, de Bjerge, hans højre vandt,
\par 55 drev Folkeslag bort foran dem, udskiftede ved Lod deres Land og lod Israels Stammer bo i deres Telte.
\par 56 Dog fristed og trodsede de Gud den Allerhøjeste og overholdt ikke hans Vidnesbyrd;
\par 57 de faldt fra, var troløse som deres Fædre, svigtede som en slappet Bue,
\par 58 de krænkede ham med deres Offerhøje, æggede ham med deres Gudebilleder.
\par 59 Det hørte Gud og blev vred følte højlig Lede ved Israel;
\par 60 han opgav sin Bolig i Silo, det Telt, hvor han boede blandt Mennesker;
\par 61 han gav sin Stolthed i Fangenskab, sin Herlighed i Fjendehånd,
\par 62 prisgav sit Folk for Sværdet, blev vred på sin Arvelod;
\par 63 Ild fortærede dets unge Mænd, dets Jomfruer fik ej Bryllupssange,
\par 64 dets Præster faldt for Sværdet, dets Enker holdt ikke Klagefest.
\par 65 Da vågnede Herren som en, der har sovet, som en Helt, der er døvet af Vin;
\par 66 han slog sine Fjender på Ryggen, gjorde dem evigt til Skamme.
\par 67 Men han fik Lede ved Josefs Telt, Efraims Stamme udvalgte han ikke;
\par 68 han udvalgte Judas Stamme, Zions Bjerg, som han elsker;
\par 69 han byggede sit Tempel himmelhøjt, grundfæstede det evigt som Jorden.
\par 70 Han udvalgte David, sin Tjener, og tog ham fra Fårenes Folde,
\par 71 hentede ham fra de diende Dyr til at vogte Jakob, hans Folk, Israel, hans Arvelod;
\par 72 han vogtede dem med oprigtigt Hjerte,ledede dem med kyndig Hånd.

\chapter{79}

\par 1 Hedninger er trængt ind i din arvelod, Gud de har besmittet dit hellige Tempel og gjort Jerusalem til en Stenhob;
\par 2 de har givet Himlens Fugle dine Tjeneres Lig til Æde, Jordens vilde Dyr dine frommes Kød;
\par 3 deres Blod har de udøst som Vand omkring Jerusalem, ingen jorder dem;
\par 4 vore Naboer er vi til Hån, vore Grander til Spot og Spe.
\par 5 Hvor længe vredes du, HERRE - for evigt? hvor længe skal din Nidkærhed lue som Ild?
\par 6 Udøs din Vrede på Folk, der ikke kender dig, på Riger, som ikke påkalder dit Navn;
\par 7 thi de har opædt Jakob og lagt hans Bolig øde.
\par 8 Tilregn os ikke Fædrenes Brøde, lad din Barmhjertighed komme os snarlig i Møde, thi vi er såre ringe,
\par 9 Hjælp os, vor Frelses Gud, for dit Navns Æres Skyld, fri os, forlad vore Synder for dit Navns Skyld!
\par 10 Hvorfor skal Hedninger sige: "Hvor er deres Gud?" Lad dine Tjeneres udgydt Blod blive hævnet på Hedningerne for vore Øjne!
\par 11 Lad de fangnes Suk nå hen for dit Åsyn, udløs Dødens Børn efter din Arms Vælde,
\par 12 lad syvfold Gengæld ramme vore Naboer for Hånen, de viser dig, Herre!
\par 13 Men vi dit Folk og den Hjord, du røgter, vi vil evindelig takke dig, forkynde din Pris fra Slægt til Slægt!

\chapter{80}

\par 1 Til Korherren. El-shoshannim-edut. Salme af Asaf
\par 2 Lyt til, du Israels Hyrde, der leder Josef som en Hjord, træd frem i Glans, du, som troner på Keruber,
\par 3 for Efraims, Benjamins og Manasses Øjne; opbyd atter din Vælde og kom til vor Frelse!
\par 4 Hærskarers Gud, bring os atter på Fode, lad dit Ansigt lyse, at vi må frelses!
\par 5 HERRE, Hærskarers Gud, hvor længe vredes du trods din Tjeners Bøn?
\par 6 Du har givet os Tårebrød at spise, Tårer at drikke i bredfuldt Mål.
\par 7 Du har gjort os til Stridsemne for vore Naboer, vore Fjender håner os.
\par 8 Hærskarers Gud, bring os atter på Fode, lad dit Ansigt lyse, at vi må frelses!
\par 9 Du rykked en Vinstok op i Ægypten, drev Folkeslag bort og plantede den;
\par 10 du rydded og skaffed den Plads, den slog Rod og fyldte Landet;
\par 11 Bjergene skjultes af dens Skygge. Guds Cedre af dens Ranker;
\par 12 den bredte sine Skud til Havet og sine kviste til Floden.
\par 13 Hvorfor har du nedbrudt dens Hegn, så alle vejfarende plukker deraf?
\par 14 Skovens Vildsvin gnaver deri, Dyrene på Marken æder den op!
\par 15 Hærskarers Gud, vend tilbage, sku ned fra Himlen og se! Drag Omsorg for denne Vinstok,
\par 16 for Skuddet, din højre planted!
\par 17 Lad dem, der sved den og huggede den sønder, gå til for dit Åsyns Trussel!
\par 18 Lad din Hånd være over din højres Mand, det Menneskebarn, du opfostrede dig!
\par 19 Da viger vi ikke fra dig, hold os i Live, så påkalder vi dit Navn!
\par 20 HERRE, Hærskarers Gud, bring os atter på Fode, lad dit Ansigt lyse, at vi må frelses!

\chapter{81}

\par 1 Til Korherren. Al-ha-gittit. Af Asaf
\par 2 Jubler for Gud, vor Styrke, råb af fryd for Jakobs Gud,
\par 3 istem Lovsang, lad Pauken lyde, den liflige Citer og Harpen;
\par 4 stød i Hornet på Nymånedagen, ved Fuldmåneskin på vor Højtidsdag!
\par 5 Thi det er Lov i Israel, et Bud fra Jakobs Gud;
\par 6 han gjorde det til en Vedtægt i Josef, da han drog ud fra Ægypten, hvor han hørte et Sprog, han ikke kendte.
\par 7 "Jeg fried hans Skulder for Byrden, hans Hænder slap fri for Kurven.
\par 8 I Nøden råbte du, og jeg frelste dig, jeg svarede dig i Tordenens Skjul, jeg prøvede dig ved Meribas Vande. - Sela.
\par 9 Hør, mit Folk, jeg vil vidne for dig, Israel, ak, om du hørte mig!
\par 10 En fremmed Gud må ej findes hos dig, tilbed ikke andres Gud!
\par 11 Jeg, HERREN, jeg er din Gud! som førte dig op fra Ægypten; luk din Mund vidt op, og jeg vil fylde den!
\par 12 Men mit Folk vilde ikke høre min Røst, Israel lød mig ikke.
\par 13 Da lod jeg dem fare i deres Stivsind, de vandrede efter deres egne Råd.
\par 14 Ak, vilde mit Folk dog høre mig, Israel gå mine Veje!
\par 15 Da kued jeg snart deres Fjender, vendte min Hånd mod deres Uvenner!
\par 16 Deres Avindsmænd skulde falde og gå til Grunde for evigt;
\par 17 jeg nærede dig med Hvedens Fedme, mættede dig med Honning fra Klippen!"

\chapter{82}

\par 1 Gud står frem i Guders Forsamling midt iblandt Guder holder han Dom
\par 2 "Hvor længe vil I dømme uredeligt og holde med de gudløse? - Sela.
\par 3 Skaf de ringe og faderløse Ret, kend de arme og nødstedte fri;
\par 4 red de ringe og fattige, fri dem ud af de gudløses Hånd!
\par 5 Dog, de kender intet, sanser intet, i Mørke vandrer de om, alle Jordens Grundvolde vakler.
\par 6 Jeg har sagt, at I er Guder, I er alle den Højestes Sønner;
\par 7 dog skal I dø som Mennesker, styrte som en af Fyrsterne!"
\par 8 Rejs dig, o Gud, døm Jorden, thi alle Folkene får du til Arv!

\chapter{83}

\par 1 En Sang. Salme af Asaf
\par 2 Und dig, o Gud, ikke Ro, vær ej tavs, vær ej stille, o Gud!
\par 3 Thi se, dine Fjender larmer, dine Avindsmænd løfter Hovedet,
\par 4 oplægger lumske Råd mod dit Folk, holder Råd imod dem, du værner:
\par 5 "Kom, lad os slette dem ud af Folkenes Tal, ej mer skal man ihukomme Israels Navn!"
\par 6 Ja, de rådslår i Fællig og slutter Pagt imod dig,
\par 7 Edoms Telte og Ismaeliterne, Moab sammen med Hagriterne,
\par 8 Gebal, Ammon, Amalek, Filister land med Tyrus's Borgere;
\par 9 også Assur har sluttet sig til dem, Lots Sønner blev de en Arm. - Sela.
\par 10 Gør med dem som med Midjan, som med Sisera og Jabin ved Kisjons Bæk,
\par 11 der gik til Grunde ved En-Dor og blev til Gødning på Marken!
\par 12 Deres Høvdinger gå det som Oreb og Ze'eb, alle deres Fyrster som Zeba og Zalmunna,
\par 13 fordi de siger: "Guds Vange tager vi til os som Eje."
\par 14 Min Gud, lad dem blive som hvirvlende Løv som Strå, der flyver for Vinden.
\par 15 Ligesom Ild fortærer Krat og Luen afsvider Bjerge,
\par 16 så forfølge du dem med din Storm, forfærde du dem med din Hvirvelvind;
\par 17 fyld deres Åsyn med Skam, så de søger dit Navn, o HERRE;
\par 18 lad dem blues, forfærdes for stedse, beskæmmes og gå til Grunde
\par 19 Og kende, at du, hvis Navn er HERREN, er ene den Højeste over al Jorden!

\chapter{84}

\par 1 Til Korherren. Al-ha-gittit. Salme af Kora-Sønnerne
\par 2 Hvor elskelig er dine boliger, Hærskares Herre!
\par 3 Af Længsel efter HERRENs Forgårde vansmægtede min Sjæl, nu jubler mit Hjerte og Kød for den levende Gud!
\par 4 Ja, Spurven fandt sig et Hjem og Svalen en Rede, hvor den har sine Unger - dine Altre, Hærskarers HERRE, min Konge og Gud!
\par 5 Salige de, der bor i dit Hus, end skal de love dig. - Sela.
\par 6 Salig den, hvis Styrke er i dig, når hans Hu står til Højtidsrejser!
\par 7 Når de går gennem Bakadalen, gør de den til Kildevang, og Tidligregnen hyller den i Velsignelser.
\par 8 Fra Kraft til Kraft går de frem, de stedes for Gud på Zion.
\par 9 Hør min Bøn, o HERRE, Hærskarers Gud, Lyt til, du Jakobs Gud! - Sela.
\par 10 Gud, vort Skjold, se til og vend dit Blik til din Salvedes Åsyn!
\par 11 Thi bedre een Dag i din Forgård end tusinde ellers, hellere ligge ved min Guds Hus's Tærskel end dvæle i Gudløsheds Telte.
\par 12 Thi Gud HERREN er Sol og Skjold, HERREN giver Nåde og Ære; dem, der vandrer i Uskyld, nægter han intet godt.
\par 13 Hærskarers HERRE, salig er den, der stoler på dig!

\chapter{85}

\par 1 Til Korherren. Salme af Kora-Sønnerne
\par 2 Du var nådig, HERRE, imod dit land du vendte Jakobs Skæbne,
\par 3 tog Skylden bort fra dit Folk og skjulte al deres Synd. - Sela.
\par 4 Du lod al din Vrede fare, tvang din glødende Harme.
\par 5 Vend tilbage, vor Frelses Gud, hør op med din Uvilje mod os!
\par 6 Vil du vredes på os for evigt, holde fast ved din Harme fra Slægt til Slægt?
\par 7 Vil du ikke skænke os Liv På ny, så dit Folk kan glæde sig i dig!
\par 8 Lad os skue din Miskundhed, HERRE, din Frelse give du os!
\par 9 Jeg vil høre, hvad Gud HERREN taler! Visselig taler han Fred til sit Folk og til sine fromme og til dem, der vender deres Hjerte til ham;
\par 10 ja, nær er hans Frelse for dem, som frygter ham, snart skal Herlighed bo i vort Land;
\par 11 Miskundhed og Sandhed mødes, Retfærd og Fred skal kysse hinanden;
\par 12 af Jorden spirer Sandhed frem, fra Himlen skuer Retfærd ned.
\par 13 Derhos giver HERREN Lykke, sin Afgrøde giver vort Land;
\par 14 Retfærd vandrer foran ham og følger også hans Fjed.

\chapter{86}

\par 1 Bøj dit Øre, HERRE, og svar mig, thi jeg er arm og fattig!
\par 2 Vogt min Sjæl, thi jeg ærer dig; frels din Tjener, som stoler på dig!
\par 3 Vær mig nådig, Herre, du er min Gud; thi jeg råber til dig Dagen igennem.
\par 4 Glæd din Tjeners Sjæl, thi til dig, o Herre, løfter jeg min Sjæl;
\par 5 thi du, o Herre, er god og rund til at forlade, rig på Nåde mod alle, der påkalder dig.
\par 6 Lyt til min Bøn, o HERRE, lån Øre til min tryglende Røst!
\par 7 På Nødens Dag påkalder jeg dig, thi du svarer mig.
\par 8 Der er ingen som du blandt Guderne, Herre, og uden Lige er dine Gerninger.
\par 9 Alle Folk, som du har skabt, skal komme, Herre, og tilbede dig, og de skal ære dit Navn.
\par 10 Thi du er stor og gør vidunderlige Ting, du alene er Gud.
\par 11 Lær mig, HERRE, din Vej, at jeg kan vandre i din Sandhed; vend mit Hjerte til dette ene: at frygte dit Navn.
\par 12 Jeg vil takke dig, Herre min Gud, af hele mit Hjerte, evindelig ære dit Navn;
\par 13 thi stor er din Miskundhed mod mig, min Sjæl har du frelst fra Dødsrigets Dyb.
\par 14 Frække har rejst sig imod mig, Gud; Voldsmænd, i Flok vil tage mit Liv, og dig har de ikke for Øje.
\par 15 Men, Herre, du er en barmhjertig og nådig Gud, langmodig og rig på Nåde og Sandhed.
\par 16 Vend dig til mig og vær mig nådig, giv din Tjener din Styrke, frels din Tjenerindes Søn!
\par 17 Und mig et Tegn på din Godhed; at mine Fjender med Skamme må se, at du, o HERRE, hjælper og trøster mig!

\chapter{87}

\par 1 Sin Stad, grundfæstet på hellige Bjerge, har Herren kær,
\par 2 Zions Porte fremfor alle Jakobs Boliger.
\par 3 Der siges herlige Ting om dig, du Guds Stad. - Sela.
\par 4 Jeg nævner Rahab og Babel blandt dem, der kender HERREN, Filisterland, Tyrus og Kusj, en fødtes her, en anden der.
\par 5 Men Zion kalder man Moder, der fødtes enhver, den Højeste holder det selv ved Magt.
\par 6 HERREN tæller efter i Folkeslagenes Liste, en fødtes her, en anden der. - Sela.
\par 7 Syngende og dansende siger de: "Alle mine Kilder er i dig!"

\chapter{88}

\par 1 En Sang. Salme af Kora-Sønnerne Til Korherren. Al-mahalat-le-annot. Maskil af Ezraitten Heman
\par 2 HERRE min Gud, jeg råber om dagen, om Natten når mit Skrig til dig;
\par 3 lad min Bøn komme frem for dit Åsyn, til mit Klageråb låne du Øre!
\par 4 Thi min Sjæl er mæt af Lidelser, mit Liv er Dødsriget nær,
\par 5 jeg regnes blandt dem, der sank i Graven, er blevet som den, det er ude med,
\par 6 kastet hen imellem de døde, blandt faldne, der hviler i Graven, hvem du ej mindes mere, thi fra din Hånd er de revet.
\par 7 Du har lagt mig i den underste Grube, på det mørke, det dybe Sted;
\par 8 tungt hviler din Vrede på mig, alle dine Brændinger lod du gå over mig. - Sela.
\par 9 Du har fjernet mine Frænder fra mig, gjort mig vederstyggelig for dem; jeg er fængslet, kan ikke gå ud,
\par 10 mit Øje er sløvt af Vånde. Hver Dag, HERRE, råber jeg til dig og rækker mine Hænder imod dig.
\par 11 Gør du Undere for de døde, står Skyggerne op og takker dig? - Sela.
\par 12 Tales der om din Nåde i Graven, i Afgrunden om din Trofasthed?
\par 13 Er dit Under kendt i Mørket, din Retfærd i Glemselens Land?
\par 14 Men jeg, o HERRE, jeg råber til dig, om Morgenen kommer min Bøn dig i Møde.
\par 15 Hvorfor forstøder du, HERRE, min Sjæl og skjuler dit Åsyn for mig?
\par 16 Elendig er jeg og Døden nær, dine Rædsler har omgivet mig fra min Ungdom;
\par 17 din Vredes Luer går over mig, dine Rædsler har lagt mig øde,
\par 18 som Vand er de om mig Dagen lang, til Hobe slutter de Kreds om mig;
\par 19 Ven og Frænde fjerned du fra mig, holdt mine Kendinge borte.

\chapter{89}

\par 1 Maskil af Ezraitten Etan
\par 2 Om HERRENs, Nåde vil jeg evigt synge, fra Slægt til Slægt med min Mund forkynde din Trofasthed.
\par 3 Thi du har sagt: "En evig Bygning er Nåden!" I Himlen har du grundfæstet din Trofasthed.
\par 4 Jeg sluttede en Pagt med min udvalgte, tilsvor David, min Tjener:
\par 5 "Jeg lader din Sæd bestå for evigt, jeg bygger din Trone fra Slægt til Slægt!" - Sela.
\par 6 Og Himlen priser dit Under, HERRE, din Trofasthed i de Helliges Forsamling.
\par 7 Thi hvem i Sky er HERRENs Lige, hvo er som HERREN iblandt Guds Sønner?
\par 8 En forfærdelig Gud i de Helliges Kreds, stor og frygtelig over alle omkring ham.
\par 9 HERRE, Hærskarers Gud, hvo er som du? HERRE, din Nåde og Trofasthed omgiver dig.
\par 10 Du mestrer Havets Overmod; når Bølgerne bruser, stiller du dem.
\par 11 Du knuste Rahab som en fældet Kriger, splitted dine Fjender med vældig Arm.
\par 12 Din er Himlen, og din er Jorden, du grundede Jorderig med dets Fylde.
\par 13 Norden og Sønden skabte du, Tabor og Hermon jubler over dit Navn.
\par 14 Du har en Arm med Vælde, din Hånd er stærk, din højre løftet.
\par 15 Retfærd og Ret er din Trones Grundvold, Nåde og Sandhed står for dit Åsyn.
\par 16 Saligt det Folk, der kender til Frydesang, vandrer, HERRE, i dit Åsyns Lys!
\par 17 De lovsynger Dagen igennem dit Navn, ophøjes ved din Retfærdighed.
\par 18 Thi du er vor Styrkes Stolthed, du løfter vort Horn ved din Yndest;
\par 19 thi vort Skjold er hos HERREN, vor Konge er Israels Hellige!
\par 20 Du taled engang i et Syn til dine fromme : "Krone satte jeg på en Helt, ophøjed en Yngling af Folket;
\par 21 jeg har fundet David, min Tjener, salvet ham med min hellige Olie;
\par 22 thi min Hånd skal holde ham fast, og min Arm skal give ham Styrke.
\par 23 Ingen Fjende skal overvælde ham, ingen Nidding trykke ham ned;
\par 24 jeg knuser hans Fjender foran ham og nedstøder dem, der bader ham;
\par 25 med ham skal min Trofasthed og Miskundhed være, hans Horn skal løfte sig ved mit Navn;
\par 26 jeg lægger Havet under hans Hånd og Strømmene under hans højre;
\par 27 mig skal han kalde: min Fader, min Gud og min Frelses Klippe.
\par 28 Jeg gør ham til førstefødt, den største blandt Jordens Konger;
\par 29 jeg bevarer for evigt min Miskundhed mod ham, min Pagt skal holdes ham troligt;
\par 30 jeg lader hans Æt bestå for evigt, hans Trone, så længe Himlen er til.
\par 31 Hvis hans Sønner svigter min Lov og ikke følger mine Lovbud,
\par 32 hvis de bryder min Vedtægt og ikke holder mit Bud,
\par 33 da hjemsøger jeg deres Synd med Ris, deres Brøde med hårde Slag;
\par 34 men min Nåde tager jeg ikke fra ham, min Trofasthed svigter jeg ikke;
\par 35 jeg bryder ikke min Pagt og ændrer ej mine Læbers Udsagn.
\par 36 Ved min Hellighed svor jeg een Gang for alle - David sviger jeg ikke:
\par 37 Hans Æt skal blive for evigt, hans Trone for mig som Solen,
\par 38 stå fast som Månen for evigt, og Vidnet på Himlen er sanddru, - Sela.
\par 39 Men du har forstødt og forkastet din Salvede og handlet i Vrede imod ham;
\par 40 Pagten med din Tjener har du brudt, vanæret hans Krone og trådt den i Støvet;
\par 41 du har nedbrudt alle hans Mure, i Grus har du lagt hans Fæstninger;
\par 42 alle vejfarende plyndrer ham, sine Naboer blev han til Spot.
\par 43 Du har løftet hans Uvenners højre og glædet alle hans Fjender;
\par 44 hans Sværd lod du vige for Fjenden, du holdt ham ej oppe i Kampen;
\par 45 du vristed ham Staven af Hænde og styrted hans Trone til Jorden,
\par 46 afkorted hans Ungdoms Dage og hylled ham ind i Skam. - Sela.
\par 47 Hvor længe vil du skjule dig, HERRE, for evigt, hvor længe skal din Vrede lue som Ild?
\par 48 Herre, kom i Hu, hvad Livet er, til hvilken Tomhed du skabte hvert Menneskebarn!
\par 49 Hvo bliver i Live og skuer ej Død, hvo frelser sin sjæl fra Dødsrigets Hånd? - Sela.
\par 50 Hvor er din fordums Nåde, Herre, som du i Trofasthed tilsvor David?
\par 51 Kom, Herre, din Tjeners Skændsel i Hu, at jeg bærer Folkenes Spot i min Favn,
\par 52 hvorledes dine Fjender håner, HERRE, hvorledes de håner din Salvedes Fodspor.
\par 53 Lovet være HERREN i Evighed, Amen, Amen!

\chapter{90}

\par 1 Herre, du var vor Bolig slægt efter slægt.
\par 2 Førend Bjergene fødtes og Jord og Jorderig blev til, fra Evighed til Evighed er du, o Gud!
\par 3 Mennesket gør du til Støv igen, du siger: "Vend tilbage, I Menneskebørn!"
\par 4 Thi tusind År er i dine Øjne som Dagen i Går, der svandt, som en Nattevagt.
\par 5 Du skyller dem bort, de bliver som en Søvn. Ved Morgen er de som Græsset, der gror;
\par 6 ved Morgen gror det og blomstrer, ved Aften er det vissent og tørt.
\par 7 Thi ved din Vrede svinder vi hen, og ved din Harme forfærdes vi.
\par 8 Vor Skyld har du stillet dig for Øje, vor skjulte Brøst for dit Åsyns Lys.
\par 9 Thi alle vore Dage glider hen i din Vrede, vore År svinder hen som et Suk.
\par 10 Vore Livsdage er halvfjerdsindstyve År, og kommer det højt, da firsindstyve. Deres Herlighed er Møje og Slid, thi hastigt går det, vi flyver af Sted.
\par 11 Hvem fatter din Vredes Vælde, din Harme i Frygt for dig!
\par 12 At tælle vore Dage lære du os, så vi kan få Visdom i Hjertet!
\par 13 Vend tilbage, HERRE! Hvor længe! Hav Medynk med dine Tjenere;
\par 14 mæt os årle med din Miskundhed, så vi kan fryde og glæde os alle vore Dage.
\par 15 Glæd os det Dagetal, du ydmygede os, det Åremål, da vi led ondt!
\par 16 Lad dit Værk åbenbares for dine Tjenere og din Herlighed over deres Børn!
\par 17 HERREN vor Guds Livsalighed være over os! Og frem vore Hænders Værk for os, ja frem vore Hænders Værk!

\chapter{91}

\par 1 Den der sidder i den Højestes Skjul og dvæler i den Almægtiges Skygge,
\par 2 siger til HERREN: Min Tilflugt, min Klippeborg, min Gud, på hvem jeg stoler.
\par 3 Thi han frier dig fra Fuglefængerens Snare, fra ødelæggende Pest;
\par 4 han dækker dig med sine Fjedre, under hans Vinger finder du Ly, hans Trofasthed er Skjold og Værge.
\par 5 Du frygter ej Nattens Rædsler, ej Pilen der flyver om Dagen
\par 6 ej Pesten, der sniger i Mørke, ej Middagens hærgende Sot.
\par 7 Falder end tusinde ved din Side, ti Tusinde ved din højre Hånd, til dig når det ikke hen;
\par 8 du ser det kun med dit Øje, er kun Tilskuer ved de gudløses Straf;
\par 9 (thi du, HERRE, er min Tilflugt) den Højeste tog du til Bolig.
\par 10 Der times dig intet ondt, dit Telt kommer Plage ej nær;
\par 11 thi han byder sine Engle at vogte dig på alle dine Veje;
\par 12 de skal bære dig på deres Hænder, at du ikke skal støde din Fod på nogen Sten;
\par 13 du skal træde på Slanger og Øgler, trampe på Løver og Drager.
\par 14 "Da han klynger sig til mig, frier jeg ham ud, jeg bjærger ham, thi han kender mit Navn;
\par 15 kalder han på mig, svarer jeg ham, i Trængsel er jeg hos ham, jeg frier ham og giver ham Ære:
\par 16 med et langt Liv mætter jeg ham og lader ham skue min Frelse!"

\chapter{92}

\par 1 En Salme. En Sang på Sabbatsdagen
\par 2 Det er godt at takke HERREN, lovsynge dit navn, du højeste,
\par 3 ved Gry forkynde din Nåde, om Natten din Trofasthed
\par 4 til tistrenget Lyre, til Harpe, til Strengeleg på Citer!
\par 5 Thi ved dit Værk har du glædet mig, HERRE, jeg jubler over dine Hænders Gerning.
\par 6 Hvor store er dine Gerninger, HERRE, dine Tanker såre dybe!
\par 7 Tåben fatter det ikke, Dåren skønner ej sligt.
\par 8 Spirer de gudløse end som Græsset, blomstrer end alle Udådsmænd, er det kun for at lægges øde for stedse,
\par 9 men du er ophøjet for evigt, HERRE.
\par 10 Thi se, dine Fjender, HERRE, se, dine Fjender går under, alle Udådsmænd spredes!
\par 11 Du har løftet mit Horn som Vildoksens, kvæget mig med den friskeste Olie;
\par 12 det fryder mit Øje at se mine Fjender, mit Øre at høre mine Avindsmænd.
\par 13 De retfærdige grønnes som Palmen, vokser som Libanons Ceder;
\par 14 plantet i HERRENs Hus grønnes de i vor Guds Forgårde;
\par 15 selv grånende bærer de Frugt, er friske og fulde af Saft
\par 16 for at vidne, at HERREN er retvis, min Klippe, hos hvem ingen Uret findes.

\chapter{93}

\par 1 HERREN har vist, han er Konge, har iført sig Højhed, HERREN har omgjordet sig med Styrke. Han grundfæsted Jorden, den rokkes ikke.
\par 2 Din Trone står fast fra fordum, fra Evighed er du!
\par 3 Strømme lod runge, HERRE, Strømme lod runge deres Drøn, Strømme lod runge deres Brag.
\par 4 Fremfor vældige Vandes Drøn, fremfor Havets Brændinger er HERREN herlig i det høje!
\par 5 Dine Vidnesbyrd er fuldt at lide på, Hellighed tilkommer dit Hus, HERRE, så længe Dagene varer!

\chapter{94}

\par 1 HERRE du hævnens Gud, du Hævnens Gud, træd frem i Glans;
\par 2 stå op, du Jordens Dommer, øv Gengæld mod de hovmodige!
\par 3 Hvor længe skal gudløse, HERRE, hvor længe skal gudløse juble?
\par 4 De fører tøjlesløs Tale, hver Udådsmand ter sig som Herre;
\par 5 de underkuer, o HERRE, dit Folk og undertrykker din Arvelod;
\par 6 de myrder Enke og fremmed faderløse slår de ihjel;
\par 7 de siger: "HERREN kan ikke se,Jakobs Gud kan intet mærke!"
\par 8 Forstå dog, I Tåber blandt Folket! Når bliver I kloge, I Dårer?
\par 9 Skulde han, som plantede Øret, ej høre, han, som dannede Øjet, ej se?
\par 10 Skulde Folkenes Tugtemester ej revse, han som lærer Mennesket indsigt?
\par 11 HERREN kender Menneskets Tanker, thi de er kun Tomhed.
\par 12 Salig den Mand, du tugter, HERRE, og vejleder ved din Lov
\par 13 for at give ham Ro for onde Dage, indtil der graves en Grav til den gudløse;
\par 14 thi HERREN bortstøder ikke sit Folk og svigter ikke sin Arvelod.
\par 15 Den retfærdige kommer igen til sin Ret, en Fremtid har hver oprigtig af Hjertet.
\par 16 Hvo står mig bi mod Ugerningsmænd? hvo hjælper mig mod Udådsmænd?
\par 17 Var HERREN ikke min Hjælp, snart hviled min Sjæl i det stille.
\par 18 Når jeg tænkte: "Nu vakler min Fod", støtted din Nåde mig, HERRE;
\par 19 da mit Hjerte var fuldt af ængstede Tanker, husvaled din Trøst min Sjæl.
\par 20 står du i Pagt med Fordærvelsens Domstol, der skaber Uret i Lovens Navn?
\par 21 Jager de end den ret,færdiges Liv og dømmer uskyldigt Blod,
\par 22 HERREN er dog mit Bjærgested, min Gud er min Tilflugtsklippe;
\par 23 han vender deres Uret imod dem selv, udsletter dem for deres Ondskab; dem udsletter HERREN vor Gud.

\chapter{95}

\par 1 Kom, lad os Juble, for HERREN, råbe af fryd for vor Frelses Klippe,
\par 2 møde med Tak for hans Åsyn, juble i Sang til hans Pris!
\par 3 Thi HERREN er en vældig Gud, en Konge stor over alle Guder;
\par 4 i hans Hånd er Jordens dybder, Bjergenes Tinder er hans;
\par 5 Havet er hans, han har skabt det, det tørre Land har hans Hænder dannet.
\par 6 Kom, lad os bøje os, kaste os ned, knæle for HERREN, vor Skaber!
\par 7 Thi han er vor Gud, og vi er det Folk, han vogter, den Hjord, han leder. Ak, lytted I dog i Dag til hans Røst:
\par 8 "Forhærder ej eders Hjerte som ved Meriba, som dengang ved Massa i Ørkenen,
\par 9 da eders Fædre fristede mig, prøved mig, skønt de havde set mit Værk.
\par 10 Jeg væmmedes fyrretyve År ved denne Slægt, og jeg sagde: Det er et Folk med vildfarne Hjerter, de kender ej mine Veje.
\par 11 Så svor jeg da i min Vrede: De skal ikke gå ind til min Hvile!

\chapter{96}

\par 1 Syng HERREN en ny sang, syng for Herren, al jorden,
\par 2 syng for HERREN og lov hans Navn, fortæl om hans Frelse Dag efter Dag,
\par 3 kundgør hans Ære blandt Folkene, hans Undere blandt alle Folkeslag!
\par 4 Thi stor og højlovet er HERREN, forfærdelig over alle Guder;
\par 5 thi alle Folkeslagenes Guder er Afguder, HERREN er Himlens Skaber.
\par 6 For hans Åsyn er Højhed og Hæder, Lov og Pris i hans Helligdom.
\par 7 Giv HERREN, I Folkeslags Slægter, giv HERREN Ære og Pris,
\par 8 giv HERREN hans Navns Ære, bring Gaver og kom til hans Forgårde,
\par 9 tilbed HERREN i helligt Skrud, bæv for hans Åsyn, al Jorden!
\par 10 Sig blandt Folkeslag: "HERREN har vist, han er Konge, han grundfæsted Jorden, den rokkes ikke, med Retfærd dømmer han Folkene."
\par 11 Himlen glæde sig, Jorden juble, Havet med dets Fylde bruse,
\par 12 Marken juble og alt, hvad den bærer! Da fryder sig alle Skovens Træer
\par 13 for HERRENs Åsyn, thi han kommer, han kommer at dømme Jorden; han dømmer Jorden med Retfærd og Folkene i sin Trofasthed.

\chapter{97}

\par 1 HERREN har vist, han er Konge! Jorden juble, lad glædes de mange Strande!
\par 2 Skyer og Mulm er om ham, Retfærd og Ret er hans Trones Støtte;
\par 3 Ild farer frem foran ham, og luer iblandt hans Fjender.
\par 4 Hans Lyn lyste op på Jorderig, Jorden så det og skjalv;
\par 5 Bjergene smelted som Voks for HERREN, for hele Jordens Herre;
\par 6 Himlen forkyndte hans Retfærd, alle Folkeslag skued hans Herlighed.
\par 7 Til Skamme blev alle, som dyrkede Billeder, de, som var stolte af deres Afguder; alle Guder bøjed sig for ham.
\par 8 Zion hørte det og glædede sig, og Judas Døtre jublede over dine Domme, HERRE!
\par 9 Thi du, o HERRE, er den Højeste over al Jorden, højt ophøjet over alle Guder!
\par 10 I, som elsker HERREN, hade det onde! Han vogter sine frommes Sjæle og frier dem af de gudløses Hånd;
\par 11 over de retfærdige oprinder Lys og Glæde over de oprigtige af Hjertet.
\par 12 I retfærdige, glæd jer i HERREN, lovsyng hans hellige Navn!

\chapter{98}

\par 1 Syng HERREN en sang, thi vidunderlige ting har han gjort; Sejren vandt ham hans højre, hans hellige Arm.
\par 2 Sin Frelse har HERREN gjort kendt, åbenbaret sin Retfærd for Folkenes Øjne;
\par 3 han kom sin Nåde mod Jakob i Hu, sin Trofasthed mod Israels Hus. Den vide Jord har skuet vor Guds Frelse.
\par 4 Råb af Fryd for HERREN, al Jorden, bryd ud i Jubel og Lovsang;
\par 5 lovsyng HERREN til Citer, lad Lovsang tone til Citer,
\par 6 råb af Fryd for Kongen, HERREN, til Trompeter og Hornets Klang!
\par 7 Havet med dets Fylde skal bruse, Jorderig og de, som bor der,
\par 8 Strømmene klappe i Hænder, Bjergene juble til Hobe
\par 9 for HERRENs Åsyn, thi han kommer, han kommer at dømme Jorden; han dømmer Jorden med Retfærd og Folkeslag med Ret!

\chapter{99}

\par 1 HERREN har vist, han er Konge, Folkene bæver, han troner på Keruber, Jorden skælver!
\par 2 Stor er HERREN på Zion, ophøjet over alle Folkeslag;
\par 3 de priser dit Navn, det store og frygtelige; hellig er han!
\par 4 Du er en Konge, der elsker Retfærd, Retten har du grundfæstet, i Jakob, øved du Ret og Retfærd.
\par 5 Ophøj HERREN vor Gud, bøj eder for hans Fødders Skammel; hellig er han!
\par 6 Moses og Aron er blandt hans Præster og Samuel blandt dem, der påkalder hans Navn; de råber til HERREN, han svarer;
\par 7 i Skystøtten taler han til dem, de holder hans Vidnesbyrd, Loven, han gav dem;
\par 8 HERRE vor Gud, du svarer dem. Du var dem en Gud, som tilgav og frikendte dem, for hvad de gjorde.
\par 9 Ophøj HERREN vor Gud, bøj eder for hans hellige Bjerg, thi hellig er HERREN vor Gud!

\chapter{100}

\par 1 Råb af Fryd for HERREN, al jorden,
\par 2 tjener HERREN med Glæde, kom for hans Åsyn med Jubel!
\par 3 Kend, at HERREN er Gud! Han skabte os, vi er hans, hans Folk og den Hjord, han vogter.
\par 4 Gå ind i hans Porte med Takkesang, med Lovsange ind i hans Forgårde, tak ham og lov hans Navn!
\par 5 Thi god er HERREN, hans Miskundhed varer evindelig, fra Slægt til Slægt hans Trofasthed!

\chapter{101}

\par 1 Om Nåde og Ret vil jeg synge, dig vil jeg lovsynge, Herre.
\par 2 Jeg vil agte på uskyldiges Vej, når den viser sig for mig, vandre i Hjertets Uskyld bag Hjemmets Vægge,
\par 3 på Niddingsdåd lader jeg aldrig mit Øje hvile. Jeg hader den, der gør ondt, han er ej i mit Følge;
\par 4 det falske Hjerte må holde sig fra mig, den onde kender jeg ikke;
\par 5 den, der sværter sin Næste, udrydder jeg; den opblæste og den hovmodige tåler jeg ikke.
\par 6 Til Landets trofaste søger mit Øje, hos mig skal de bo; den, der vandrer uskyldiges Vej, skal være min Tjener;
\par 7 ingen, der øver Svig, skal bo i mit Hus, ingen, som farer med Løgn, bestå for mit Øje.
\par 8 Alle Landets gudløse gør jeg til intet hver Morgen for at udrydde alle Udådsmænd af HERRENs By.

\chapter{102}

\par 1 En Bøn. Til Brug for en Hjælpeløs, når han føler Afmagt og udøser sin Klage for Herren
\par 2 HERRE, lyt til min bøn, lad mit råb komme til dig,
\par 3 skjul dog ikke dit Åsyn for mig; den Dag jeg stedes i Nød, bøj da dit Øre til mig; når jeg kalder, så skynd dig og svar mig!
\par 4 Thi mine Dage svinder som Røg, mine Ledemod brænder som Ild;
\par 5 mit Hjerte er svedet og - visnet som Græs, thi jeg glemmer at spise mit Brød.
\par 6 Under min Stønnen klæber mine Ben til Huden;
\par 7 jeg ligner Ørkenens Pelikan, er blevet som Uglen på øde Steder;
\par 8 om Natten ligger jeg vågen og jamrer så ensom som Fugl på Taget;
\par 9 mine Fjender håner mig hele Dagen; de der spotter mig, sværger ved mig.
\par 10 Thi Støv er mit daglige Brød, jeg blander min Drik med Tårer
\par 11 over din Harme og Vrede, fordi du tog mig og slængte mig bort;
\par 12 mine Dage hælder som Skyggen, som Græsset visner jeg hen.
\par 13 Men du troner evindelig, HERRE, du ihukommes fra Slægt til Slægt;
\par 14 du vil rejse dig og forbarme dig over Zion, når Nådens Tid, når Timen er inde;
\par 15 thi dine Tjenere elsker dets Sten og ynkes over dets Grushobe.
\par 16 Og HERRENs Navn skal Folkene frygte, din Herlighed alle Jordens Konger;
\par 17 thi HERREN opbygger Zion, han lader sig se i sin Herlighed;
\par 18 han vender sig til de hjælpeløses Bøn, lader ej deres Bøn uænset.
\par 19 For Efterslægten skal det optegnes, af Folk, der skal fødes, skal prise HERREN;
\par 20 thi han ser ned fra sin hellige Højsal, HERREN skuer ned fra Himmel til Jord
\par 21 for at høre de fangnes Stønnen og give de dødsdømte Frihed,
\par 22 at HERRENs Navn kan forkyndes i Zion, hans - Pris i Jerusalem,
\par 23 når Folkeslag og Riger til Hobe samles for at tjene HERREN.
\par 24 Han lammed min Kraft på Vejen, forkorted mit Liv.
\par 25 Jeg siger: Min Gud, tag mig ikke bort i Dagenes Hælvt! Dine År er fra Slægt til Slægt.
\par 26 Du grundfæsted fordum Jorden, Himlene er dine Hænders Værk;
\par 27 de falder, men du består, alle slides de op som en Klædning;
\par 28 som Klæder skifter du dem; de skiftes, men du er den samme, og dine År får aldrig Ende!
\par 29 Dine Tjeneres Børn fæster Bo, deres Sæd skal bestå for dit Åsyn.

\chapter{103}

\par 1 Min Sjæl, lov Herren, og alt i mig love hans hellige navn!
\par 2 Min Sjæl, lov HERREN, og glem ikke alle hans Velgerninger!
\par 3 Han, som tilgiver alle dine Misgerninger og læger alle dine Sygdomme,
\par 4 han, som udløser dit Liv fra Graven og kroner dig med Miskundhed og Barmhjertighed,
\par 5 han, som mætter din Sjæl med godt, så du bliver ung igen som Ørnen!
\par 6 HERREN øver Retfærdighed og Ret mod alle fortrykte.
\par 7 Han lod Moses se sine Veje, Israels Børn sine Gerninger;
\par 8 barmhjertig og nådig er HERREN, langmodig og rig på Miskundhed;
\par 9 han går ikke bestandig i Rette, gemmer ej evigt på Vrede;
\par 10 han handled ej med os efter vore Synder, gengældte os ikke efter vor Brøde.
\par 11 Men så højt som Himlen er over Jorden, er hans Miskundhed stor over dem, der frygter ham.
\par 12 Så langt som Østen er fra Vesten, har han fjernet vore Synder fra os.
\par 13 Som en Fader forbarmer sig over sine Børn, forbarmer HERREN sig over dem, der frygter ham.
\par 14 Thi han kender vor Skabning, han kommer i Hu, vi er Støv;
\par 15 som Græs er Menneskets dage, han blomstrer som Markens Blomster;
\par 16 når et Vejr farer over ham, er han ej mere, hans Sted får ham aldrig at se igen.
\par 17 Men HERRENs Miskundhed varer fra Evighed og til Evighed over dem, der frygter ham, og hans Retfærd til Børnenes Børn
\par 18 for dem, der holder hans Pagt og kommer hans Bud i Hu, så de gør derefter.
\par 19 HERREN har rejst sin Trone i Himlen, alt er hans Kongedømme underlagt.
\par 20 Lov HERREN, I hans Engle, I vældige i Kraft, som gør, hvad han byder, så snart I hører hans Røst.
\par 21 Lov HERREN, alle hans Hærskarer, hans Tjenere, som fuldbyrder hans Vilje.
\par 22 Lov HERREN, alt, hvad han skabte, på hvert eneste Sted i hans Rige! Min Sjæl, lov HERREN!

\chapter{104}

\par 1 Min sjæl, lov Herren! Herren min Gud, du er såre stor! Du er klædt i Højhed og Herlighed,
\par 2 hyllet i Lys som en Kappe! Himlen spænder du ud som et Telt;
\par 3 du hvælver din Højsal i Vandene, gør Skyerne til din Vogn, farer frem på Vindens Vinger;
\par 4 Vindene gør du til Sendebud, Ildsluer til dine Tjenere!
\par 5 Du fæsted Jorden på dens Grundvolde, aldrig i Evighed rokkes den;
\par 6 Verdensdybet hylled den til som en Klædning, Vandene stod over Bjerge.
\par 7 For din Trusel flyede de, skræmtes bort ved din Tordenrøst,
\par 8 for op ad Bjerge og ned i Dale til det Sted, du havde beredt dem;
\par 9 du satte en Grænse, de ej kommer over, så de ikke igen skal tilhylle Jorden.
\par 10 Kilder lod du rinde i Dale, hen mellem Bjerge flød de;
\par 11 de læsker al Markens Vildt, Vildæsler slukker deres Tørst;
\par 12 over dem bygger Himlens Fugle, mellem Grenene lyder deres Kvidder.
\par 13 Fra din Højsal vander du Bjergene, Jorden mættes fra dine Skyer;
\par 14 du lader Græs gro frem til Kvæget og Urter til Menneskets Tjeneste, så du frembringer Brød af Jorden
\par 15 og Vin, der glæder Menneskets Hjerte, og lader Ansigtet glinse af Olie, og Brødet skal styrke Menneskets Hjerte.
\par 16 HERRENs Træer bliver mætte, Libanons Cedre, som han har plantet,
\par 17 hvor Fuglene bygger sig Rede; i Cypresser har Storken sin Bolig.
\par 18 Højfjeldet er for Stenbukken, Klipperne Grævlingens Tilflugt.
\par 19 Du skabte Månen for Festernes Skyld, Solen kender sin Nedgangs Tid;
\par 20 du sender Mørke, Natten kommer, da rører sig alle Skovens Dyr;
\par 21 de unge Løver brøler efter Rov, de kræver deres Føde af Gud.
\par 22 De sniger sig bort, når Sol står op, og lægger sig i deres Huler;
\par 23 Mennesket går til sit Dagværk, ud til sin Gerning, til Kvæld falder på.
\par 24 Hvor mange er dine Gerninger, HERRE, du gjorde dem alle med Visdom; Jorden er fuld af, hvad du har skabt!
\par 25 Der er Havet, stort og vidt, der vrimler det uden Tal af Dyr, både små og store;
\par 26 Skibene farer der, Livjatan, som du danned til Leg deri.
\par 27 De bier alle på dig, at du skal give dem Føde i Tide;
\par 28 du giver dem den, og de sanker, du åbner din Hånd, og de mættes med godt.
\par 29 Du skjuler dit Åsyn, og de forfærdes; du tager deres Ånd, og de dør og vender tilbage til Støvet;
\par 30 du sender din Ånd, og de skabes, Jordens Åsyn fornyer du.
\par 31 HERRENs Herlighed vare evindelig, HERREN glæde sig ved sine Værker!
\par 32 Et Blik fra ham, og Jorden skælver, et Stød fra ham, og Bjergene ryger
\par 33 Jeg vil synge for HERREN, så længe jeg lever, lovsynge min Gud, den Tid jeg er til.
\par 34 Min Sang være ham til Behag, jeg har min Glæde i HERREN.
\par 35 Måtte Syndere svinde fra Jorden og gudløse ikke mer være til! Min Sjæl, lov HERREN! Halleluja!

\chapter{105}

\par 1 Pris Herren, påkald hans navn, gør hans Gerninger kendte blandt Folkeslag!
\par 2 Syng og spil til hans Pris, tal om alle hans Undere;
\par 3 ros jer af hans hellige Navn, eders Hjerte glæde sig, I, som søger HERREN;
\par 4 spørg efter HERREN og hans magt, søg bestandig hans Åsyn;
\par 5 kom i Hu de Undere, han gjorde, hans Tegn og hans Munds Domme,
\par 6 I, hans Tjener Abrahams Sæd, hans udvalgte, Jakobs Sønner!
\par 7 Han, HERREN, er vor Gud, hans Domme når ud over Jorden;
\par 8 han ihukommer for evigt sin Pagt, i tusind Slægter sit Tilsagn,
\par 9 Pagten, han slutted med Abraham, Eden, han tilsvor Isak;
\par 10 han holdt den i Hævd som Ret for Jakob, en evig Pagt for Israel,
\par 11 idet han sagde: "Dig giver jeg Kana'ans Land som eders Arvelod."
\par 12 Da de kun var en liden Hob, kun få og fremmede der,
\par 13 og vandrede fra Folk til Folk, fra et Rige til et andet,
\par 14 tillod han ingen at volde dem Men, men tugted for deres Skyld Konger
\par 15 "Rør ikke mine Salvede, gør ikke mine Profeter ondt!"
\par 16 Hungersnød kaldte han frem over Landet, hver Brødets Støttestav brød han;
\par 17 han sendte forud for dem en Mand, Josef solgtes som Træl;
\par 18 de tvang hans Fødder med Lænker, han kom i Lænker af Jern,
\par 19 indtil hans Ord blev opfyldt; ved HERRENs Ord stod han Prøven igennem.
\par 20 På Kongens Bud blev han fri, Folkenes Hersker lod ham løs:
\par 21 han tog ham til Herre for sit Hus, til Hersker over alt sit Gods;
\par 22 han styred hans Øverster efter sin Vilje og viste hans Ældste til Rette.
\par 23 Og Israel kom til Ægypten, Jakob boede som Gæst i Kamiternes Land.
\par 24 Han lod sit Folk blive såre frugtbart og stærkere end dets Fjender;
\par 25 han vendte deres Sind til Had mod sit Folk og til Træskhed imod sine Tjenere.
\par 26 Da sendte han Moses, sin Tjener, og Aron, sin udvalgte Mand;
\par 27 han gjorde sine Tegn i Ægypten og Undere i Kamiternes Land;
\par 28 han sendte Mørke, så blev det mørkt, men de ænsede ikke hans Ord;
\par 29 han gjorde deres Vande til Blod og slog deres Fisk ihjel;
\par 30 af Frøer vrimlede Landet, selv i Kongens Sale var de;
\par 31 han talede, så kom der Bremser og Myg i alt deres Land;
\par 32 han sendte dem Hagl for Regn og luende Ild i Landet;
\par 33 han slog både Vinstok og Figen og splintrede Træerne i deres Land;
\par 34 han talede, så kom der Græshopper, Springere uden Tal,
\par 35 de åd alt Græs i Landet, de åd deres Jords Afgrøde;
\par 36 alt førstefødt i Landet slog han, Førstegrøden af al deres Kraft;
\par 37 han førte dem ud med Sølv og Guld, ikke een i hans Stammer snubled
\par 38 Ægypterne glæded sig, da de drog bort, thi de var grebet af Rædsel for dem.
\par 39 Han bredte en Sky som Skjul og Ild til at lyse i Natten;
\par 40 de krævede, han bragte Vagtler, med Himmelbrød mættede han dem;
\par 41 han åbnede Klippen, og Vand strømmede ud, det løb som en Flod i Ørkenen.
\par 42 Thi han kom sit hellige Ord i Hu til Abraham, sin Tjener;
\par 43 han lod sit Folk drage ud med Fryd, sine udvalgte under Jubel;
\par 44 han gav dem Folkenes Lande, de fik Folkeslags Gods i Eje.
\par 45 Derfor skulde de holde hans Bud og efterkomme hans Love.

\chapter{106}

\par 1 Halleluja! Lov Herren, thi han er god, thi hans miskundhed varer evindelig!
\par 2 Hvo kan opregne Herrens vældige gerninger, finde ord til at kundgøre al hans pris?
\par 3 Salige de, der holder på ret, som altid øver retfærdighed!
\par 4 Husk os, Herre, når dit folk finder nåde, lad os få godt af din frelse,
\par 5 at vi må se dine udvalgtes lykke, glæde os ved dit folks glæde og med din arvelod prise vor lykke!
\par 6 Vi syndede som vore Fædre, handlede ilde og gudløst.
\par 7 Vore Fædre i Ægypten ænsede ej dine Undere, kom ikke din store Miskundhed i Hu, stod den Højeste imod ved det røde Hav.
\par 8 Dog frelste han dem for sit Navns Skyld, for at gøre sin Vælde kendt;
\par 9 han trued det røde Hav, og det tørrede ud, han førte dem gennem Dybet som gennem en Ørk;
\par 10 han fried dem af deres Avindsmænds Hånd og udløste dem fra Fjendens Hånd;
\par 11 Vandet skjulte dem, som trængte dem, ikke een blev tilbage af dem;
\par 12 da troede de på hans Ord og kvad en Sang til hans Pris.
\par 13 Men de glemte snart hans Gerninger, biede ej på hans Råd;
\par 14 de grebes af Attrå i Ørkenen, i Ødemarken fristed de Gud;
\par 15 så gav han dem det, de kræved og sendte dem Lede i Sjælen.
\par 16 De bar Avind mod Moses i Lejren, mod Aron, HERRENs hellige;
\par 17 Jorden åbned sig, slugte Datan, lukked sig over Abirams Flok;
\par 18 Ilden rasede i deres Flok, Luen brændte de gudløse op.
\par 19 De lavede en Kalv ved Horeb og tilbad det støbte Billed;
\par 20 de byttede deres Herlighed bort for et Billed af en Okse, hvis Føde er Græs;
\par 21 de glemte Gud, deres Frelser, som øvede store Ting i Ægypten,
\par 22 Undere i Kamiternes Land, frygtelige Ting ved det røde Hav.
\par 23 Da tænkte han på at udrydde dem, men Moses, hans udvalgte Mand, stilled sig i Gabet for hans Åsyn for at hindre, at hans Vrede lagde øde.
\par 24 De vraged det yndige Land og troede ikke hans Ord,
\par 25 men knurrede i deres Telte og hørte ikke på HERREN;
\par 26 da løfted han Hånden og svor at lade dem falde i Ørkenen,
\par 27 splitte deres Sæd blandt Folkene, sprede dem rundt i Landene.
\par 28 De holdt til med Ba'al-Peor og åd af de dødes Ofre;
\par 29 de krænked ham med deres Gerninger, og Plage brød løs iblandt dem.
\par 30 Da stod Pinehas frem og holdt Dom, og Plagen blev bragt til at standse,
\par 31 og det regnedes ham til Retfærdighed fra Slægt til Slægt, evindelig.
\par 32 De vakte hans Vrede ved Meribas Vand, og for deres Skyld gik det Moses ilde;
\par 33 thi de stod hans Ånd imod, og han talte uoverlagte Ord.
\par 34 De udryddede ikke de Folk, som HERREN havde sagt, de skulde,
\par 35 med Hedninger blandede de sig og gjorde deres Gerninger efter;
\par 36 deres Gudebilleder dyrkede de, og disse blev dem en Snare;
\par 37 til Dæmonerne ofrede de, og det både Sønner og Døtre;
\par 38 de udgød uskyldigt Blod, deres Sønners og Døtres Blod, som de ofred til Kana'ans Guder, og Landet blev smittet ved Blod;
\par 39 de blev urene ved deres Gerninger, bolede ved deres idrætter.
\par 40 Da blev HERREN vred på sit Folk og væmmedes ved sin Arv;
\par 41 han gav dem i Folkenes Hånd, deres Avindsmænd blev deres Herrer;
\par 42 deres Fjendervoldte dem Trængsel, de kuedes under deres Hånd.
\par 43 Han frelste dem Gang på Gang, men de stod egensindigt imod og sygnede hen i Brøden;
\par 44 dog så han til dem i Trængslen, så snart han hørte dem klage;
\par 45 han kom sin Pagt i Hu og ynkedes efter sin store Miskundhed;
\par 46 han lod dem finde Barmhjertighed hos alle, der tog dem til Fange.
\par 47 Frels os, HERRE vor Gud, du samle os sammen fra Folkene, at vi må love dit hellige Navn, med Stolthed synge din Pris.
\par 48 Lovet være HERREN, Israels Gud, fra Evighed og til Evighed! Og alt Folket svare Amen!

\chapter{107}

\par 1 Halleluja! Lov Herren, thi han er god, thi hans Miskundhed varer evindelig!
\par 2 Så skal HERRENs genløste sige, de, han løste af Fjendens Hånd
\par 3 og samlede ind fra Landene, fra Øst og Vest, fra Nord og fra Havet.
\par 4 I den øde Ørk for de vild, fandt ikke Vej til beboet By,
\par 5 de led både Sult og Tørst, deres Sjæl var ved at vansmægte;
\par 6 men de råbte til HERREN i Nøden, han frelste dem at deres Trængsler
\par 7 og førte dem ad rette Vej, så de kom til beboet By.
\par 8 Lad dem takke HERREN for hans Miskundhed, for hans Underværker mod Menneskens Børn.
\par 9 Thi han mættede den vansmægtende Sjæl og fyldte den sultne med godt.
\par 10 De sad i Mulm og Mørke, bundne i pine og Jern,
\par 11 fordi de havde stået Guds Ord imod og ringeagtet den Højestes Råd.
\par 12 Deres Hjerte var knuget af Kummer, de faldt, der var ingen, som hjalp;
\par 13 men de råbte til HERREN i Nøden, han frelste dem af deres Trængsler,
\par 14 førte dem ud af Mørket og Mulmet og sønderrev deres Bånd.
\par 15 Lad dem takke HERREN for hans Miskundhed, for hans Underværker mod Menneskens Børn.
\par 16 Thi han sprængte Døre af Kobber og sønderslog Slåer af Jern.
\par 17 De sygnede hen for Synd og led for Brødes Skyld,
\par 18 de væmmedes ved al Slags Mad, de kom Dødens Porte nær
\par 19 men de råbte til Herren i Nøden, han frelste dem af deres Trængsler,
\par 20 sendte sit Ord og lægede dem og frelste deres Liv fra Graven.
\par 21 Lad dem takke HERREN for hans Miskundhed, for hans Underværker mod Menneskens Børn
\par 22 og ofre Lovprisningsofre og med Jubel forkynnde hans Gerninger.
\par 23 De for ud på Havet i Skibe, drev Handel på vældige Vande,
\par 24 blev Vidne til HERRENs Gerninger, hans Underværker i Dybet;
\par 25 han bød, og et Stormvejr rejste sig, Bølgerne tårnedes op;
\par 26 mod Himlen steg de, i Dybet sank de, i Ulykken svandt deres Mod;
\par 27 de tumled og raved som drukne, borte var al deres Visdom;
\par 28 men de råbte til HERREN i Nøden, han frelste dem af deres Trængsler,
\par 29 skiftede Stormen til Stille, så Havets Bølger tav;
\par 30 og glade blev de, fordi det stilned; han førte dem til Havnen, de søgte.
\par 31 Lad dem takke HERREN for hans Miskundhed, for hans Underværker mod Menneskens Børn,
\par 32 ophøje ham i Folkets Forsamling og prise ham i de Ældstes Kreds!
\par 33 Floder gør han til Ørken og Kilder til øde Land,
\par 34 til Saltsteppe frugtbart Land for Ondskabens Skyld hos dem, som - bor der.
\par 35 Ørken gør han til Vanddrag, det tørre Land til Kilder;
\par 36 der lader han sultne bo, så de grunder en By at bo i,
\par 37 tilsår Marker og planter Vin og høster Afgrødens Frugt.
\par 38 Han velsigner dem, de bliver mange, han lader det ikke skorte på Kvæg.
\par 39 De bliver få og segner under Modgangs og Kummers Tryk,
\par 40 han udøser Hån over Fyrster og lader dem rave i vejløst Øde.
\par 41 Men han løfter den fattige op af hans Nød og gør deres Slægter som Hjorde;
\par 42 de oprigtige ser det og glædes, men al Ondskab lukker sin Mund.
\par 43 Hvo som er viis, han mærke sig det og lægge sig HERRENs Nåde på Sinde!

\chapter{108}

\par 1 En Sang. Salme af David
\par 2 Mit Hjerte er trøstigt, Gud, mit hjerte er trøstigt; jeg vil synge og lovprise dig, vågn op, min Ære!
\par 3 Harpe og Citer, vågn op, jeg vil vække Morgenrøden.
\par 4 Jeg vil takke dig, HERRE, blandt Folkeslag, lovprise dig blandt Folkefærd;
\par 5 thi din Miskundhed når til Himlen, din Sandhed til Skyerne.
\par 6 Løft dig, o Gud, over Himlen, din Herlighed være over al Jorden!
\par 7 Til Frelse for dine elskede hjælp med din højre, bønhør os!
\par 8 Gud talede i sin Helligdom: "Jeg vil udskifte Sikem med Jubel, udmåle Sukkots Dal;
\par 9 mit er Gilead, mit er Manasse, Efraim er mit Hoveds Værn, Juda min Herskerstav,
\par 10 Moab min Vaskeskål, på Edom kaster jeg min Sko, over Filisterland jubler jeg."
\par 11 Hvo bringer mig til den befæstede By, hvo leder mig hen til Edom?
\par 12 Har du ikke, Gud, stødt os fra dig? Du ledsager ej vore Hære.
\par 13 Giv os dog Hjælp mod Fjenden! Blændværk er Menneskers Støtte.
\par 14 Med Gud skal vi øve vældige Ting, vore Fjender træder han ned!

\chapter{109}

\par 1 Du min Lovsangs Gud, vær ej tavs!
\par 2 Thi en gudløs, svigefuld Mund har de åbnet imod mig, taler mig til med Løgntunge,
\par 3 med hadske Ord omringer de mig og strider imod mig uden Grund;
\par 4 til Løn for min Kærlighed er de mig fjendske, skønt jeg er idel Bøn;
\par 5 de gør mig ondt for godt, gengælder min Kærlighed med Had.
\par 6 Straf ham for hans Gudløshed, lad en Anklager stå ved hans højre,
\par 7 lad ham gå dømt fra Retten, hans Bøn blive regnet for Synd;
\par 8 hans Livsdage blive kun få, hans Embede tage en anden;
\par 9 hans Børn blive faderløse, hans Hustru vorde Enke;
\par 10 hans Børn flakke om og tigge, drives bort fra et øde Hjem;
\par 11 Ågerkarlen rage efter alt, hvad han har, og fremmede rane hans Gods;
\par 12 ingen være langmodig imod ham, ingen ynke hans faderløse;
\par 13 hans Afkom gå til Grunde, hans Navn slettes ud i næste Slægt:
\par 14 lad hans Fædres Skyld ihukommes hos HERREN, lad ikke hans Moders Synd slettes ud,
\par 15 altid være de, HERREN for Øje; hans Minde vorde udryddet af Jorden,
\par 16 fordi det ej faldt ham ind at vise sig god, men han forfulgte den arme og fattige og den, hvis Hjerte var knust til Døde;
\par 17 han elsked Forbandelse, så lad den nå ham; Velsignelse yndede han ikke, den blive ham fjern!
\par 18 Han tage Forbandelse på som en Klædning, den komme som Vand i hans Bug, som Olie ind i hans Ben;
\par 19 den blive en Dragt, han tager på, et Bælte, han altid bærer!
\par 20 Det være mine Modstanderes Løn fra HERREN, dem, der taler ondt mod min Sjæl.
\par 21 Men du, o HERRE, min Herre, gør med mig efter din Godhed og Nåde, frels mig for dit Navns Skyld!
\par 22 Thi jeg er arm og fattig, mit Hjerte vånder sig i mig;
\par 23 som Skyggen, der hælder, svinder jeg bort, som Græshopper rystes jeg ud;
\par 24 af Faste vakler mine Knæ, mit Kød skrumper ind uden Salve;
\par 25 til Spot for dem er jeg blevet, de ryster på Hovedet, når de
\par 26 Hjælp mig, HERRE min Gud, frels mig efter din Miskundhed,
\par 27 så de sander, det var din Hånd, dig, HERRE, som gjorde det!
\par 28 Lad dem forbande, du vil velsigne, mine uvenner vorde til Skamme, din Tjener glæde sig;
\par 29 lad mine Fjender klædes i Skændsel, iføres Skam som en Kappe!
\par 30 Med min Mund vil jeg højlig takke HERREN, prise ham midt i Mængden;
\par 31 thi han står ved den fattiges højre at fri ham fra dem, der dømmer hans Sjæl.

\chapter{110}

\par 1 HERREN sagde til min Herre: "Sæt dig ved min højre hånd, til jeg lægger dine fjender som en skammel for dine fødder!"
\par 2 Fra Zion udrækker HERREN din Vældes Spir; hersk midt iblandt dine Fjender!
\par 3 Dit Folk møder villigt frem på din Vældes Dag; i hellig Prydelse kommer dit unge Mandskab til dig, som Dug af Morgenrødens Moderskød.
\par 4 HERREN har svoret og angrer det ej: "Du er Præst evindelig på Melkizedeks Vis."
\par 5 Herren ved din højre knuser Konger på sin Vredes Dag,
\par 6 blandt Folkene holder han Dom, fylder op med døde, knuser Hoveder viden om Lande.
\par 7 Han drikker af Bækken ved Vejen, derfor løfter han Hovedet højt.

\chapter{111}

\par 1 Halleluja! jeg takker Herren af hele mit hjerte i oprigtiges kreds og i menighed!
\par 2 Store er Herrens gerninger, gennemtænkte til bunds.
\par 3 Hans værk er højhed og herlighed, hans retfærd bliver til evig tid.
\par 4 Han har sørget for, at hans undere mindes, nådig og barmhjertig er Herren.
\par 5 Dem, der frygter ham, giver han føde, han kommer for evigt sin pagt i hu.
\par 6 Han viste sit folk sine vældige gerninger, da han gav dem folkenes eje.
\par 7 Hans hænders værk er sandhed og ret, man kan lide på alle hans bud;
\par 8 de står i al evighed fast, udført i sandhed og retsind.
\par 9 Han sendte sit folk udløsning, stifted sin pagt for evigt.
\par 10 Herrens frygt er visdoms begyndelse; forstandig er hver, som øver den. Evigt varer hans pris!

\chapter{112}

\par 1 Halleluja! Salig er den, der frygter Herren og ret har lyst til hans bud!
\par 2 Hans Æt bliver mægtig på Jord, den oprigtiges Slægt velsignes;
\par 3 Velstand og Rigdom er i hans Hus, hans Retfærdighed varer evindelig.
\par 4 For den oprigtige oprinder Lys i Mørke; han er mild, barmhjertig retfærdig.
\par 5 Salig den, der ynkes og låner ud og styrer sine Sager med Ret;
\par 6 thi han rokkes aldrig i Evighed, den retfærdige ihukommes for evigt;
\par 7 han frygter ikke for onde Tidender, hans Hjerte er trøstigt i Tillid, til HERREN;
\par 8 fast er hans Hjerte og uden Frygt, indtil han skuer sine Fjender med Fryd;
\par 9 til fattige deler han rundhåndet ud, hans Retfærdighed varer evindelig; med Ære løfter hans Horn sig.
\par 10 Den gudløse ser det og græmmer sig, skærer Tænder og går til Grunde; de gudløses Attrå bliver til intet.

\chapter{113}

\par 1 Halleluja! Pris, I Herrens tjenere, pris Herrens navn!
\par 2 Herrens navn være lovet fra nu og til evig tid;
\par 3 fra sol i opgang til sol i bjærge være Herrens navn lovpriset!
\par 4 Over alle folk er Herren ophøjet, hans herlighed højt over himlene.
\par 5 Hvo er som HERREN vor Gud, som rejste sin Trone i det høje
\par 6 og skuer ned i det dybe - i Himlene og på Jorden -
\par 7 som rejser den ringe af Støvet, løfter den fattige op af Skarnet
\par 8 og sætter ham mellem Fyrster, imellem sit Folks Fyrster,
\par 9 han, som lader barnløs Hustru sidde som lykkelig Barnemoder!

\chapter{114}

\par 1 Halleluja! Da Israel drog fra Ægypten, Jakobs Hus fra det stammende Folk,
\par 2 da blev Juda hans Helligdom, Israel blev hans Rige.
\par 3 Havet så det og flyede, Jordan trak sig tilbage,
\par 4 Bjergene sprang som Vædre, Højene hopped som Lam.
\par 5 Hvad fejler du, Hav, at du flyr, Jordan, hvi går du tilbage,
\par 6 hvi springer I Bjerge som Vædre, hvi hopper I Høje som Lam?
\par 7 Skælv, Jord, for HERRENs Åsyn, for Jakobs Guds Åsyn,
\par 8 han, som gør Klipper til Vanddrag, til Kildevæld hården Flint!

\chapter{115}

\par 1 Ikke os, o Herre, ikke os, men dit navn, det give du ære for din Miskundheds og Trofastheds Skyld!
\par 2 Hvi skal Folkene sige: "Hvor er dog deres Gud?"
\par 3 Vor Gud, han er i Himlen; alt, hvad han vil, det gør han!
\par 4 Deres Billeder er Sølv og Guld, Værk af Menneskehænder;
\par 5 de har Mund, men taler ikke, Øjne, men ser dog ej;
\par 6 de har Ører, men hører ikke, Næse men lugter dog ej;
\par 7 de har Hænder, men føler ikke, Fødder, men går dog ej, deres Strube frembringer ikke en Lyd.
\par 8 Som dem skal de, der lavede dem, blive, enhver, som stoler på dem!
\par 9 Israel stoler på HERREN, han er deres Hjælp og Skjold;
\par 10 Arons Hus stoler på HERREN, han er deres Hjælp og Skjold;
\par 11 de, som frygter HERREN, stoler på ham, han er deres Hjælp og Skjold.
\par 12 HERREN kommer os i Hu, velsigner, velsigner Israels Hus, velsigner Arons Hus,
\par 13 velsigner dem, der frygter HERREN, og det både små og store.
\par 14 HERREN lader eder vokse i Tal, eder og eders Børn;
\par 15 velsignet er I af HERREN, Himlens og Jordens Skaber.
\par 16 Himlen er HERRENs Himmel, men Jorden gav han til Menneskens Børn.
\par 17 De døde priser ej HERREN, ingen af dem, der steg ned i det tavse.
\par 18 Men vi, vi lover HERREN, fra nu og til evig Tid!

\chapter{116}

\par 1 Halleluja! Jeg elsker Herren, thi han hører min røst, min tryglende bøn,
\par 2 ja, han bøjed sit Øre til mig, jeg påkaldte HERRENs Navn.
\par 3 Dødens Bånd omspændte mig, Dødsrigets Angster greb mig, i Trængsel og Nød var jeg stedt.
\par 4 Jeg påkaldte HERRENs Navn: "Ak, HERRE, frels min Sjæl!"
\par 5 Nådig er HERREN og retfærdig, barmhjertig, det er vor Gud;
\par 6 HERREN vogter enfoldige, jeg var ringe, dog frelste han mig.
\par 7 Vend tilbage, min Sjæl, til din Ro, thi HERREN har gjort vel imod dig!
\par 8 Ja, han fried min Sjæl fra Døden, mit Øje fra Gråd, min Fod fra Fald.
\par 9 Jeg vandrer for HERRENs Åsyn udi de levendes Land;
\par 10 jeg troede, derfor talte jeg, såre elendig var jeg,
\par 11 sagde så i min Angst: "Alle Mennesker lyver!"
\par 12 Hvorledes skal jeg gengælde HERREN alle hans Velgerninger mod mig?
\par 13 Jeg vil løfte Frelsens Bæger og påkalde HERRENs Navn.
\par 14 Jeg vil indfri HERREN mine Løfter i Påsyn af alt hans Folk.
\par 15 Kostbar i HERRENs Øjne er hans frommes Død.
\par 16 Ak, HERRE, jeg er jo din Tjener, din Tjener, din Tjenerindes Søn, mine Lænker har du løst.
\par 17 Jeg vil ofre dig Lovprisningsoffer og påkalde HERRENs Navn;
\par 18 mine Løfter vil jeg indfri HERREN i Påsyn af alt hans Folk
\par 19 i HERRENs Hus's Forgårde og i din Midte, Jerusalem!

\chapter{117}

\par 1 Halleluja! Lovsyng HERREN, alle I Folk, pris ham, alle Stammer,
\par 2 thi stor er hans Miskundhed mod os, HERRENs Trofasthed varer evindelig!

\chapter{118}

\par 1 Halleluja! Tak Herren, thi han er god, thi hans miskundhed varer evindelig.
\par 2 Israel sige: "Thi hans miskundhed varer evindelig!"
\par 3 Arons Hus sige: "Thi hans Miskundhed varer evindelig!"
\par 4 De, som frygter HERREN, sige: "Thi hans Miskundhed varer evindelig!"
\par 5 Jeg påkaldte HERREN i Trængslen, HERREN svared og førte mig ud i åbent Land.
\par 6 HERREN, er med mig, jeg frygter ikke, hvad kan Mennesker gøre mig?
\par 7 HERREN, han er min Hjælper, jeg skal se med Fryd på dem, der hader mig.
\par 8 At ty til HERREN er godt fremfor at stole på Mennesker;
\par 9 at ty til HERREN er godt fremfor at stole på Fyrster.
\par 10 Alle Folkeslag flokkedes om mig, jeg slog dem ned i HERRENs Navn;
\par 11 de flokkedes om mig fra alle Sider, jeg slog dem ned i HERRENs Navn;
\par 12 de flokkedes om mig som Bier, blussed op, som Ild i Torne, jeg slog dem ned i HERRENs Navn.
\par 13 Hårdt blev jeg ramt, så jeg faldt, men HERREN hjalp mig.
\par 14 Min Styrke og Lovsang er HERREN, han blev mig til Frelse.
\par 15 Jubel og Sejrsråb lyder i de retfærdiges Telte: "HERRENs højre øver Vælde,
\par 16 HERRENs højre er løftet, HERRENs højre øver Vælde!"
\par 17 Jeg skal ikke dø, men leve og kundgøre HERRENs Gerninger.
\par 18 HERREN tugted mig hårdt, men gav mig ej hen i Døden.
\par 19 Oplad mig Retfærdigheds Porte, ad dem går jeg ind og lovsynger HERREN!
\par 20 Her er HERRENs Port, ad den går retfærdige ind.
\par 21 Jeg vil takke dig, thi du bønhørte mig, og du blev mig til Frelse.
\par 22 Den Sten; Bygmestrene forkastede, er blevet Hovedhjørnesten.
\par 23 Fra HERREN er dette kommet, det er underfuldt for vore Øjne.
\par 24 Denne er Dagen, som HERREN har gjort, lad os juble og glæde os på den!
\par 25 Ak, HERRE, frels dog, ak, HERRE; lad det dog lykkes!
\par 26 Velsignet den, der kommer, i HERRENs Navn; vi velsigner eder fra HERRENs Hus!
\par 27 HERREN er Gud, og han lod det lysne for os. Festtoget med Grenene slynge sig frem, til Alterets Horn er nået!
\par 28 Du er min Gud, jeg vil takke dig, min Gud, jeg vil ophøje dig!
\par 29 Tak HERREN, thi han er god, thi hans Miskundhed varer evindelig!

\chapter{119}

\par 1 Salige de, hvis Vandel er fulde, som vandrer i HERRENs Lov.
\par 2 Salige de, der agter på hans Vidnesbyrd, søger ham af hele deres Hjerte.
\par 3 de, som ikke gør Uret, men vandrer på hans Veje.
\par 4 Du har givet dine Befalinger, for at de nøje skal holdes.
\par 5 O, måtte jeg vandre med faste Skridt, så jeg holder dine Vedtægter!
\par 6 Da skulde jeg ikke blive til - Skamme, thi jeg så hen til alle dine Bud.
\par 7 Jeg vil takke dig af oprigtigt Hjerte, når jeg lærer din Retfærds Lovbud.
\par 8 Jeg vil holde dine Vedtægter, svigt mig dog ikke helt!
\par 9 Hvorledes holder en ung sin Vej ren? Ved at bolde sig efter dit Ord.
\par 10 Af hele mit Hjerte søger jeg dig, lad mig ikke fare vild fra dine Bud!
\par 11 Jeg gemmer dit Ord i mit Hjerte for ikke at synde imod dig.
\par 12 Lovet være du, HERRE, lær mig dine Vedtægter!
\par 13 Jeg kundgør med mine Læber alle din Munds Lovbud.
\par 14 Jeg glæder mig over dine Vidnesbyrds Vej, som var det al Verdens Rigdom.
\par 15 Jeg vil grunde på dine Befalinger og se til dine Stier.
\par 16 I dine Vedtægter har jeg min Lyst, jeg glemmer ikke dit Ord.
\par 17 Und din Tjener at leve, at jeg kan holde dit Ord.
\par 18 Oplad mine Øjne, at jeg må skue de underfulde Ting i din Lov.
\par 19 Fremmed er jeg på Jorden, skjul ikke dine Bud for mig!
\par 20 Altid hentæres min Sjæl af Længsel efter dine Lovbud.
\par 21 Du truer de frække; forbandede er de, der viger fra dine Bud.
\par 22 Vælt Hån og Ringeagt fra mig, thi jeg agter på dine Vidnesbyrd.
\par 23 Om Fyrster oplægger Råd imod mig, grunder din Tjener på dine Vedtægter.
\par 24 Ja, dine Vidnesbyrd er min Lyst, det er dem, der giver mig Råd.
\par 25 I Støvet ligger min Sjæl, hold mig i Live efter dit Ord!
\par 26 Mine Veje lagde jeg frem, og du bønhørte mig, dine Vedtægter lære du mig.
\par 27 Lad mig fatte dine Befalingers Vej og grunde på dine Undere.
\par 28 Af Kummer græder. min Sjæl, oprejs mig efter dit Ord!
\par 29 Lad Løgnens Vej være langt fra mig og skænk mig i Nåde din Lov!
\par 30 Troskabs Vej har jeg valgt, dine Lovbud attrår jeg.
\par 31 Jeg hænger ved dine Vidnesbyrd, lad mig ikke beskæmmes, HERRE!
\par 32 Jeg vil løbe dine Buds Vej, thi du giver mit Hjerte at ånde frit.
\par 33 Lær mig, HERRE, dine Vedtægters Vej, så jeg agter derpå til Enden.
\par 34 Giv mig Kløgt, så jeg agter på din Lov og holder den af hele mit Hjerte.
\par 35 Før mig ad dine Buds Sti, thi jeg har Lyst til dem.
\par 36 Bøj mit Hjerte til dine Vidnesbyrd og ej til uredelig Vinding.
\par 37 Vend mine Øjne bort fra Tant, hold mig i Live ved dit Ord!
\par 38 Stadfæst for din Tjener dit Ord, så jeg lærer at frygte dig.
\par 39 Hold borte fra mig den Skændsel, jeg frygter, thi dine Lovbud er gode.
\par 40 Se, dine Befalinger længes jeg efter, hold mig i Live ved din Retfærd!
\par 41 Lad din Miskundhed komme over mig, HERRE, din Frelse efter dit Ord,
\par 42 så jeg har Svar til dem, der spotter mig, thi jeg stoler på dit Ord.
\par 43 Tag ikke ganske Sandheds Ord fra min Mund, thi jeg bier på dine Lovbud.
\par 44 Jeg vil stadig holde din Lov, ja evigt og altid;
\par 45 jeg vil vandre i åbent Land, thi dine Befalinger ligger mig på Sinde.
\par 46 Jeg vil tale om dine Vidnesbyrd for Konger uden at blues;
\par 47 jeg vil fryde mig over dine Bud, som jeg højlige elsker;
\par 48 jeg vil udrække Hænderne mod dine Bud og grunde på dine Vedtægter.
\par 49 Kom Ordet til din Tjener i Hu, fordi du har ladet mig håbe.
\par 50 Det er min Trøst i Nød, at dit Ord har holdt mig i Live.
\par 51 De frække hånede mig såre, dog veg jeg ej fra din Lov.
\par 52 Dine Lovbud fra fordum, HERRE, kom jeg i Hu og fandt Trøst.
\par 53 Harme greb mig over de gudløse, dem, der slipper din Lov.
\par 54 Dine vedtægter blev mig til Sange i min Udlændigheds Hus.
\par 55 Om Natten kom jeg dit Navn i Hu, HERRE, jeg holdt din Lov.
\par 56 Det blev min lykkelige Lod: at agte på dine Befalinger.
\par 57 Min Del er HERREN, jeg satte mig for at holde dine Ord.
\par 58 Jeg bønfaldt dig af hele mit Hjerte, vær mig nådig efter dit Ord!
\par 59 Jeg overtænkte mine Veje og styred min Fod tilbage til dine Vidnesbyrd.
\par 60 Jeg hasted og tøved ikke med at holde dine Bud.
\par 61 De gudløses Snarer omgav mig, men jeg glemte ikke din Lov.
\par 62 Jeg, står op ved Midnat og takker dig for dine retfærdige Lovbud.
\par 63 Jeg er Fælle med alle, der frygter dig og holder dine Befalinger.
\par 64 Jorden er fuld af din Miskundhed, HERRE, lær mig dine Vedtægter!
\par 65 Du gjorde vel mod din Tjener, HERRE, efter dit Ord.
\par 66 Giv mig Forstand og indsigt, thi jeg tror på dine Bud.
\par 67 For jeg blev ydmyget, for jeg vild, nu holder jeg dit Ord.
\par 68 God er du og gør godt, lær mig dine Vedtægter!
\par 69 De frække tilsøler mig med Løgn, men på dine Bud tager jeg hjerteligt Vare.
\par 70 Deres Hjerte er dorskt som Fedt, jeg har min Lyst i din Lov.
\par 71 Det var godt, at jeg blev ydmyget, så jeg kunde lære dine Vedtægter.
\par 72 Din Munds Lov er mig mere værd end Guld og Sølv i Dynger.
\par 73 Dine Hænder skabte og dannede mig, giv mig Indsigt; så jeg kan lære dine Bud!
\par 74 De, der frygter dig, ser mig og glædes, thi jeg bier på dit Ord.
\par 75 HERRE, jeg ved, at dine Bud er retfærdige, i Trofasthed har du ydmyget mig.
\par 76 Lad din Miskundhed være min Trøst efter dit Ord til din Tjener!
\par 77 Din Barmhjertighed finde mig, at jeg må leve, thi din Lov er min Lyst.
\par 78 Lad de frække beskæmmes, thi de gør mig skammelig Uret, jeg grunder på dine Befalinger.
\par 79 Lad dem, der frygter dig, vende sig til mig, de, der kender dine Vidnesbyrd.
\par 80 Lad mit Hjerte være fuldkomment i dine Vedtægter, at jeg ikke skal blive til Skamme.
\par 81 Efter din Frelse længes min Sjæl, jeg bier på dit Ord.
\par 82 Mine Øjne længes efter dit Ord og siger: "Hvornår mon du trøster mig?"
\par 83 Thi jeg er som en Lædersæk i Røg, men dine Vedtægter glemte jeg ikke.
\par 84 Hvor langt er vel din Tjeners Liv? Når vil du dømme dem, der forfølger mig?
\par 85 De frække grov mig Grave, de, som ej følger din Lov.
\par 86 Alle dine Bud er trofaste, med Løgn forfølger man mig, o hjælp mig!
\par 87 De har næsten tilintetgjort mig på Jorden, men dine Befalinger slipper jeg ikke.
\par 88 Hold mig i Live efter din Miskundhed, at jeg kan holde din Munds Vidnesbyrd.
\par 89 HERRE, dit Ord er evigt, står fast i Himlen.
\par 90 Din Trofasthed varer fra Slægt til Slægt, du grundfæsted Jorden, og den står fast.
\par 91 Dine Lovbud står fast, de holder dine Tjenere oppe.
\par 92 Havde din Lov ej været min Lyst, da var jeg omkommet i min Elende.
\par 93 Aldrig i Evighed glemmer jeg dine Befalinger, thi ved dem holdt du mig i Live.
\par 94 Din er jeg, frels mig, thi dine Befalinger ligger mig på Sinde.
\par 95 De gudløse lurer på at lægge mig øde, dine Vidnesbyrd mærker jeg mig.
\par 96 For alting så jeg en Grænse, men såre vidt rækker dit Bud.
\par 97 Hvor elsker jeg dog din Lov! Hele Dagen grunder jeg på den.
\par 98 Dit Bud har gjort mig visere end mine Fjender, thi det er for stedse mit.
\par 99 Jeg er klogere end alle mine Lærere, thi jeg grunder på dine Vidnesbyrd.
\par 100 Jeg har mere Forstand end de gamle; jeg agter på dine Bud.
\par 101 Jeg holder min Fod fra hver Vej, som er ond, at jeg kan holde dit Ord.
\par 102 Fra dine Lovbud veg jeg ikke, thi du underviste mig.
\par 103 Hvor sødt er dit Ord for min Gane, sødere end Honning for min Mund.
\par 104 Ved dine Befalinger fik jeg Forstand, så jeg hader al Løgnens Vej.
\par 105 Dit Ord er en Lygte for min Fod, et Lys på min Sti.
\par 106 Jeg svor en Ed og holdt den: at følge dine retfærdige Lovbud.
\par 107 Jeg er såre ydmyget, HERRE, hold mig i Live efter dit Ord!
\par 108 Lad min Munds frivillige Ofre behage dig, HERRE, og lær mig dine Lovbud!
\par 109 Altid går jeg med Livet i Hænderne, men jeg glemte ikke din Lov.
\par 110 De gudløse lægger Snarer for mig, men fra dine Befalinger for jeg ej vild.
\par 111 Dine Vidnesbyrd fik jeg til evigt Eje, thi de er mit Hjertes Glæde.
\par 112 Jeg bøjed mit Hjerte til at holde dine Vedtægter for evigt til Enden.
\par 113 Jeg hader tvesindet Mand, men jeg elsker din Lov.
\par 114 Mit Skjul og mit Skjold er du, jeg bier på dit Ord.
\par 115 Vig fra mig, I, som gør ondt, jeg vil holde min Guds Bud.
\par 116 Støt mig efter dit Ord, at jeg må leve, lad mig ikke beskæmmes i mit Håb!
\par 117 Hold mig oppe, at jeg må frelses og altid have min Lyst i dine Vedtægter!
\par 118 Du forkaster alle, der farer vild fra dine Vedtægter, thi de higer efter Løgn.
\par 119 For Slagger regner du alle Jordens gudløse, derfor elsker jeg dine Vidnesbyrd.
\par 120 Af Rædsel for dig gyser mit Kød, og jeg frygter for dine Lovbud.
\par 121 Ret og Skel har jeg gjort, giv mig ikke hen til dem, der trænger mig!
\par 122 Gå i Borgen for din Tjener, lad ikke de frække trænge mig!
\par 123 Mine Øjne vansmægter efter din Frelse og efter dit Retfærds Ord.
\par 124 Gør med din Tjener efter din Miskundhed og lær mig dine Vedtægter!
\par 125 Jeg er din Tjener, giv mig Indsigt, at jeg må kende dine Vidnesbyrd!
\par 126 Det er Tid for HERREN at gribe ind, de har krænket din Lov.
\par 127 Derfor elsker jeg dine Bud fremfor Guld og Skatte.
\par 128 Derfor følger jeg oprigtigt alle dine Befalinger og hader hver Løgnens Sti.
\par 129 Underfulde er dine Vidnesbyrd, derfor agted min Sjæl på dem.
\par 130 Tydes dine Ord, så bringer de Lys, de giver enfoldige Indsigt.
\par 131 Jeg åbned begærligt min Mund, thi min Attrå stod til dine Bud.
\par 132 Vend dig til mig og vær mig nådig, som Ret er for dem, der elsker dit Navn!
\par 133 Lad ved dit Ord mine Skridt blive faste og ingen Uret få Magten over mig!
\par 134 Udløs mig fra Menneskers Vold, at jeg må holde dine Befalinger!
\par 135 Lad dit Ansigt lyse over din Tjener og lær mig dine Vedtægter!
\par 136 Vand i Strømme græder mine Øjne, fordi man ej holder din Lov.
\par 137 Du er retfærdig, HERRE, og retvise er dine Lovbud.
\par 138 Du slog dine Vidnesbyrd fast ved Retfærd og Troskab så såre.
\par 139 Min Nidkærhed har fortæret mig, thi mine Fjender har glemt dine Ord.
\par 140 Dit Ord er fuldkommen rent, din Tjener elsker det.
\par 141 Ringe og ussel er jeg, men dine Befalinger glemte jeg ikke.
\par 142 Din Retfærd er Ret for evigt, din Lov er Sandhed.
\par 143 Trængsel og Angst har ramt mig, men dine Bud er min Lyst.
\par 144 Dine Vidnesbyrd er Ret for evigt, giv mig indsigt, at jeg må leve!
\par 145 Jeg råber af hele mit Hjerte, svar mig, HERRE, jeg agter på dine Vedtægter.
\par 146 Jeg råber til dig, o frels mig, at jeg kan holde dine Vidnesbyrd!
\par 147 Årle råber jeg til dig om Hjælp, og bier på dine Ord.
\par 148 Før Nattevagtstimerne våger mine Øjne for at grunde på dit Ord.
\par 149 Hør mig efter din Miskundhed, HERRE, hold mig i Live efter dit Lovbud!
\par 150 De, der skændigt forfølger mig, er mig nær, men de er langt fra din Lov.
\par 151 Nær er du, o HERRE, og alle dine Bud er Sandhed.
\par 152 For længst vandt jeg Indsigt af dine Vidnesbyrd, thi du har grundfæstet dem for evigt.
\par 153 Se min Elende og fri mig, thi jeg glemte ikke din Lov.
\par 154 Før min Sag og udløs mig, hold mig i Live efter dit Ord!
\par 155 Frelsen er langt fra de gudløse, thi dine Vedtægter ligger dem ikke, på Sinde.
\par 156 Din Barmhjertighed er stor, o HERRE, hold mig i Live efter dine Lovbud!
\par 157 Mange forfølger mig og er mig fjendske, fra dine Vidnesbyrd veg jeg ikke.
\par 158 Jeg væmmes ved Synet af troløse, der ikke holder dit Ord.
\par 159 Se til mig, thi jeg elsker dine Befalinger, HERRE, hold mig i Live efter din Miskundhed!
\par 160 Summen af dit Ord er Sandhed, og alt dit retfærdige Lovbud varer evigt.
\par 161 Fyrster forfulgte mig uden Grund, men mit Hjerte frygted dine Ord.
\par 162 Jeg glæder mig over dit Ord som en, der har gjort et vældigt Bytte.
\par 163 Jeg hader og afskyr Løgn, din Lov har jeg derimod kær.
\par 164 Jeg priser dig syv Gange daglig for dine retfærdige Lovbud.
\par 165 Megen Fred har de, der elsker din Lov, og intet bliver til Anstød for dem.
\par 166 Jeg håber på din Frelse, HERRE, og jeg har holdt dine Bud.
\par 167 Min Sjæl har holdt dine Vidnesbyrd, jeg har dem såre kære.
\par 168 Jeg holder dine Befalinger og Vidnesbyrd, thi du kender alle mine Veje.
\par 169 Lad min Klage nå frem for dit Åsyn, HERRE, giv mig Indsigt efter dit Ord!
\par 170 Lad min Bøn komme frem for dit Åsyn, frels mig efter dit Ord!
\par 171 Mine Læber skal synge din Pris, thi du lærer mig dine Vedtægter.
\par 172 Min Tunge skal synge om dit Ord, thi alle dine Bud er Retfærd.
\par 173 Lad din Hånd være mig til Hjælp, thi jeg valgte dine Befalinger.
\par 174 Jeg længes efter din Frelse, HERRE, og din Lov er min Lyst.
\par 175 Gid min Sjæl må leve, at den kan prise dig, og lad dine Lovbud være min Hjælp!
\par 176 Farer jeg vild som det tabte Får, så opsøg din Tjener, thi jeg glemte ikke dine Bud.

\chapter{120}

\par 1 Jeg råbte til HERREN i Nød, og han svarede mig.
\par 2 HERRE, udfri min Sjæl fra Løgnelæber, fra den falske Tunge!
\par 3 Der ramme dig dette og hint, du falske Tunge!
\par 4 Den stærkes Pile er hvæsset ved glødende Gyvel.
\par 5 Ve mig, at jeg må leve som fremmed i Mesjek, bo iblandt Kedars Telte!
\par 6 Min Sjæl har længe nok boet blandt Folk, som hader Fred.
\par 7 Jeg vil Fred; men taler jeg, vil de Krig!

\chapter{121}

\par 1 Jeg løfter mine Øjne til Bjergene: Hvorfra kommer min Hjælp?
\par 2 Fra HERREN kommer min Hjælp, fra Himlens og Jordens Skaber.
\par 3 Din fod vil han ej lade vakle, ej blunder han, som bevarer dig;
\par 4 nej, han blunder og sover ikke, han, som bevarer Israel.
\par 5 HERREN er den, som bevarer dig, HERREN er din Skygge ved din højre;
\par 6 Solen stikker dig ikke om Dagen, og Månen ikke om Natten;
\par 7 HERREN bevarer dig mod alt ondt, han bevarer din Sjæl;
\par 8 HERREN bevarer din Udgang og Indgang fra nu og til evig Tid!

\chapter{122}

\par 1 Jeg frydede mig, da de sagde til mig: "Vi drager til HERRENs Hus!"
\par 2 Så står vore Fødder da i dine Porte, Jerusalem,
\par 3 Jerusalem bygget som Staden, hvor Folket samles;
\par 4 thi did op drager Stammerne, HERRENs Stammer en Vedtægt for Israel om at prise HERRENs Navn.
\par 5 Thi der står Dommersæder, Sæder for Davids Hus.
\par 6 Bed om Jerusalems Fred! Ro finde de, der elsker dig!
\par 7 Der råde Fred på din Mur, Tryghed i dine Borge!
\par 8 For Brødres og Frænders Skyld vil jeg ønske dig Fred,
\par 9 for Herren vor Guds hus's skyld vil jeg søge dit bedste.

\chapter{123}

\par 1 Jeg løfter mine Øjne til dig, som troner i Himlen!
\par 2 Som trælles øjne følger deres Herres Hånd, som en Trælkvindes Øjne følger hendes Frues Hånd, så følger vore Øjne HERREN vor Gud, til han er os nådig.
\par 3 Forbarm dig over os, HERRE, forbarm dig! Thi overmætte er vi af Spot,
\par 4 overmæt er vor Sjæl af de sorgløses Hån, de stoltes Spot!

\chapter{124}

\par 1 Havde HERREN ej været med os - så siger Israel -
\par 2 havde Herren ej været med os, da Mennesker rejste sig mod os;
\par 3 så havde de slugt os levende, da deres Vrede optændtes mod os;
\par 4 så havde Vandene overskyllet os, en Strøm var gået over vor Sjæl,
\par 5 over vor Sjæl var de gået, de vilde Vande.
\par 6 Lovet være HERREN, som ej gav os hen, deres Tænder til Rov!
\par 7 Vor Sjæl slap fri som en Fugl at Fuglefængernes Snare, Snaren reves sønder, og vi slap fri.
\par 8 Vor Hjælper HERRENs Navn, Himlens og Jordens Skaber.

\chapter{125}

\par 1 De der stoler På HERREN, er som Zions bjerg, der aldrig i evighed rokkes.
\par 2 Jerusalem ligger hegnet af Bjerge; og HERREN hegner sit Folk fra nu og til evig Tid :
\par 3 Han lader ej Gudløsheds Herskerskerstav tynge retfærdiges Lod, at retfærdige ikke skal udrække Hånden til Uret.
\par 4 HERRE, vær god mod de gode, de oprigtige af Hjertet;
\par 5 men dem, der slår ind på krogveje, dem bortdrive HERREN tillige med Udådsmænd. Fred over Israel!

\chapter{126}

\par 1 Da HERREN hjemførte Zions fanger, var vi som drømmende;
\par 2 da fyldtes vor Mund med Latter, vor Tunge med Frydesang; da hed det blandt Folkene: "HERREN har gjort store Ting imod dem!"
\par 3 HERREN har gjort store Ting imod os, og vi blev glade.
\par 4 Vend, o Herre, vort Fangenskab, som Sydlandets Strømme!
\par 5 De; som sår med Gråd, skal høste med Frydesang;
\par 6 de går deres Gang med Gråd, når de udstrør Sæden, med Frydesang kommer de hjem, bærende deres Neg.

\chapter{127}

\par 1 Dersom HERREN ikke bygger huset, er Bygmestrenes Møje forgæves, dersom HERREN ikke vogter Byen, våger Vægteren forgæves.
\par 2 Det er forgæves, I står årle op og går sent til Ro, ædende Sliddets Brød; alt sligt vil han give sin Ven i Søvne.
\par 3 Se, Sønner er HERRENs Gave, Livsens Frugt er en Løn.
\par 4 Som Pile i Krigerens Hånd er Sønner, man får i sin Ungdom.
\par 5 Salig den Mand, som fylder sit Kogger med dem; han beskæmmes ej, når han taler med Fjender i Porten.

\chapter{128}

\par 1 Salig enhver, som frygter Herren og går på hans veje!
\par 2 Dit Arbejdes Frugt skal du nyde, salig er du, det går dig vel!
\par 3 Som en frugtbar Vinranke er din Hustru inde i dit Hus, som Oliekviste er dine Sønner rundt om dit Bord.
\par 4 Se, så velsignes den Mand, der frygter HERREN.
\par 5 HERREN velsigne dig fra Zion, at du må se Jerusalems Lykke alle dit Livs Dage
\par 6 og se dine Sønners Sønner! Fred over Israel!

\chapter{129}

\par 1 De trængte mig hårdt fra min ungdom - så siger Israel -
\par 2 de trængte mig hårdt fra min Ungdom, men kued mig ikke.
\par 3 Plovmænd pløjed min Ryg, trak lange Furer;
\par 4 retfærdig er HERREN, han overskar de gudløses Reb.
\par 5 Alle, som hader Zion, skal vige med Skam,
\par 6 blive som Græs på Tage, der visner, førend det skyder Strå,
\par 7 og ikke fylder Høstkarlens Hånd og Opbinderens Favn;
\par 8 Folk, som går forbi, siger ikke: "HERRENs Velsignelse over eder! Vi velsigner eder i HERRENs Navn!"

\chapter{130}

\par 1 Fra det dybe råber jeg til
\par 2 o Herre hør min Røst! Lad dine Ører lytte til min tryglende Røst!
\par 3 Tog du Vare, HERRE, på Misgerninger, Herre, hvo kunde da bestå?
\par 4 Men hos dig er der Syndsforladelse, at du må frygtes.
\par 5 Jeg håber.på HERREN, min Sjæl håber på hans Ord,
\par 6 på Herren bier min Sjæl mer end Vægter på Morgen, Vægter på Morgen.
\par 7 Israel, bi på HERREN! Thi hos HERREN er Miskundhed, hos ham er Forløsning i Overflod.
\par 8 Og han vil forløse Israel fra alle dets Misgerninger,

\chapter{131}

\par 1 Herre, mit hjerte er ikke hovmodigt, mine øjne er ikke stolte, jeg sysler ikke med store Ting, med Ting, der er mig for høje.
\par 2 Nej, jeg har lullet og tysset min Sjæl; som afvant Barn hos sin Moder har min Sjæl det hos HERREN.
\par 3 Israel, bi på HERREN fra nu og til evig Tid!

\chapter{132}

\par 1 HERRE, kom David i Hu for al hans møje,
\par 2 hvorledes han tilsvor HERREN, gav Jakobs Vældige et Løfte :
\par 3 "Jeg træder ej ind i mit Huses Telt, jeg stiger ej op på mit Leje,
\par 4 under ikke mine Øjne Søvn, ikke mine Øjenlåg Hvile,
\par 5 før jeg har fundet HERREN et Sted, Jakobs Vældige en Bolig!"
\par 6 "Se, i Efrata hørte vi om den, fandt den på Ja'ars Mark;
\par 7 lad os gå hen til hans Bolig, tilbede ved hans Fødders Skammel!"
\par 8 HERRE, bryd op til dit Hvilested, du og din Vældes Ark!
\par 9 Dine Præster være klædte i Retfærd, dine fromme synge med Fryd!
\par 10 For din Tjener Davids Skyld afvise du ikke din Salvede!"
\par 11 HERREN tilsvor David et troværdigt, usvigeligt Løfte: "Af din Livsens Frugt vil jeg sætte Konger på din Trone.
\par 12 Såfremt dine Sønner holder min Pagt og mit Vidnesbyrd, som jeg lærer dem, skal også deres Sønner sidde evindelig på din Trone!
\par 13 Thi HERREN har udvalgt Zion, ønsket sig det til Bolig :
\par 14 Her er for evigt mit Hvilested, her vil jeg bo, thi det har jeg ønsket.
\par 15 Dets Føde velsigner jeg, dets fattige mætter jeg med Brød,
\par 16 dets Præster klæder jeg i Frelse, dets fromme skal synge med Fryd.
\par 17 Der lader jeg Horn vokse frem for David, sikrer min Salvede Lampe.
\par 18 Jeg klæder hans Fjender i Skam, men på ham skal Kronen stråle!"

\chapter{133}

\par 1 Se, hvor godt og lifligt er det, når brødre bor tilsammen:
\par 2 som kostelig Olie, der flyder fra Hovedet ned over Skægget, Arons Skæg, der bølger ned over Kjortelens Halslinning,
\par 3 som Hermons Dug, der falder på Zions Bjerge. Thi der skikker HERREN Velsignelse ned, Liv til evig Tid.

\chapter{134}

\par 1 Op og lov nu HERREN, alle HERRENs, som står i HERRENs Hus ved Nattetide!
\par 2 Løft eders Hænder mod Helligdommen og lov HERREN!
\par 3 HERREN velsigne dig fra Zion, han, som skabte Himmel og Jord.

\chapter{135}

\par 1 Halleluja! Pris Herrens navn, pris det, I HERRENs Tjenere,
\par 2 som står i HERRENs Hus, i vor Guds Huses Forgårde!
\par 3 Pris HERREN, thi god er HERREN, lovsyng hans Navn, thi lifligt er det.
\par 4 Thi HERREN udvalgte Jakob, Israel til sin Ejendom.
\par 5 Ja, jeg ved, at HERREN er stor, vor Herre er større end alle Guder.
\par 6 HERREN gør alt, hvad han vil, i Himlene og på Jorden, i Have og alle Verdensdyb.
\par 7 Han lader Skyer stige op fra Jordens Ende, får Lynene til at give Regn, sender Stormen ud fra sine Forrådskamre;
\par 8 han, som slog Ægyptens førstefødte, både Mennesker og Kvæg,
\par 9 og sendte Tegn og Undere i din Midte, Ægypten, mod Farao og alle hans Folk;
\par 10 han, som fældede store Folk og veg så mægtige Konger,
\par 11 Amoriternes konge Sion og Basans Konge Og, og alle Kana'ans Riger
\par 12 og gav deres Land i Eje, i Eje til Israel, hans Folk.
\par 13 HERRE, dit Navn er evigt, din Ihukommelse, HERRE, fra Slægt til Slægt,
\par 14 thi Ret skaffer HERREN sit Folk og ynkes over sine Tjenere.
\par 15 Folkenes Billeder er Sølv og Guld, Værk af Menneskehænder;
\par 16 de har Mund, men taler ikke, Øjne, men ser dog ej;
\par 17 de har Ører, men hører ikke, ej heller er der Ånde i deres Mund.
\par 18 Som dem skal de, der laved dem, blive enhver, som stoler på dem.
\par 19 Lov HERREN, Israels Hus, lov HERREN, Arons Hus,
\par 20 lov HERREN, Levis Hus, lov HERREN, I, som frygter HERREN!
\par 21 Fra Zion være HERREN lovet, han, som bor i Jerusalem!

\chapter{136}

\par 1 Halleluja! Tak HERREN, thi han er god; thi hans Miskundhed varer evindelig!
\par 2 Tak Gudernes Gud; thi hans miskundhed varer evindelig!
\par 3 Tak Herrens Herre; thi hans miskundhed varer evindelig!
\par 4 Han, der ene gør store undere; thi hans miskundhed varer evindelig!
\par 5 Som skabte Himlen med indsigt; thi hans miskundhed varer evindelig!
\par 6 Som bredte jorden på vandet; thi hans miskundhed varer evindelig!
\par 7 Som skabte de store lys; thi hans miskundhed varer evindelig!
\par 8 Sol til at råde om dagen; thi hans miskundhed varer evindelig!
\par 9 Måne og stjerner til at råde om natten; thi hans miskundhed varer evindelig!
\par 10 Som slog Ægyptens førstefødte; thi hans Miskundhed varer evindelig!
\par 11 Og førte Israel ud derfra; thi hans Miskundhed varer evindelig!
\par 12 Med stærk 'Hånd og udstrakt Arm; thi hans Miskundhed varer evindelig!
\par 13 Som kløved det røde Hav; thi hans Miskundhed varer evindelig!
\par 14 Og førte tsrael midt igennem det; thi hans Miskundhed varer evindelig!
\par 15 Som drev Farao og hans Hær i det røde Hav thi hans Miskundhed varer evindelig!
\par 16 Som førte sit Folk i Ørkenen; thi hans Miskundhed varer evindelig!
\par 17 Som fældede store Konger; thi hans Miskundhed varer evindelig!
\par 18 Og veg så vældige Konger; thi hans Miskundhed varer evindelig!
\par 19 Amoriternes Konge Sion, thi hans Miskundhed varer evindelig!
\par 20 Og Basans Konge Og thi hans Miskundhed varer evindelig!
\par 21 Og gav deres Land i Eje; thi hans Miskundhed varer evindelig!
\par 22 I Eje til hans Tjener Israel; thi hans Miskundhed varer evindelig!
\par 23 Som kom os i Hu i vor Ringhed; thi hans Miskundhed varer evindelig!
\par 24 Og friede os fra vore Fjender; thi hans Miskundhed varer evindelig!
\par 25 Som giver alt Kød Føde; thi hans Miskundhed varer evindelig!
\par 26 Tak Himlenes Gud; thi hans Miskundhed varer evindelig!

\chapter{137}

\par 1 Ved Babels Floder, der sad vi og græd, når Zion randt os i hu.
\par 2 Vi hængte vore Harper i Landets Pile.
\par 3 Thi de, der havde bortført os, bad os synge, vore Bødler bad os være glade: "Syng os af Zions Sange!"
\par 4 Hvor kan vi synge HERRENs Sange på fremmed Grund?
\par 5 Jerusalem, glemmer jeg dig, da visne min højre!
\par 6 Min Tunge hænge ved Ganen, om ikke jeg ihukommer dig, om ikke jeg sætter Jerusalem over min højeste Glæde!
\par 7 HERRE, ihukom Edoms Sønner for Jerusalems Dag, at de råbte: "Nedbryd, nedbryd lige til Grunden!"
\par 8 Du Babels Datter, du Ødelægger! Salig den, der gengælder dig, hvad du gjorde imod os!
\par 9 Salig den, der griber dine spæde og knuser dem mod Klippen!

\chapter{138}

\par 1 Jeg vil prise dig HERRE, at hele lovsynge dig for Guderne;
\par 2 jeg vil tilbede, vendt mod dit hellige Tempel, og mere end alt vil jeg prise dit Navn for din Miskundheds og Trofastheds Skyld; thi du har herliggjort dit Ord.
\par 3 Den Dag jeg råbte, svared du mig, du gav mig Mod, i min Sjæl kom Styrke.
\par 4 Alle Jordens Konger skal prise dig, HERRE, når de hører din Munds Ord,
\par 5 og synge om HERRENs Veje; thi stor er HERRENs Ære,
\par 6 thi HERREN er ophøjet, ser til den ringe, han kender den stolte i Frastand.
\par 7 Går jeg i Trængsel, du værger mig Livet, mod Fjendernes Vrede udrækker du Hånden, din højre bringer mig Frelse.
\par 8 HERREN vil føre det igennem for mig, din Miskundhed, HERRE, varer evindelig. Opgiv ej dine Hænders Værk!

\chapter{139}

\par 1 HERRE, du ransager mig og kender mig!
\par 2 Du ved, når jeg står op, du fatter min Tanke i Frastand,
\par 3 du har Rede på, hvor jeg går eller ligger, og alle mine Veje kender du grant.
\par 4 Thi før Ordet er til på min Tunge, se, da ved du det, HERRE, til fulde.
\par 5 Bagfra og forfra omslutter du mig, du lægger din Hånd på mig.
\par 6 At fatte det er mig for underfuldt, for højt, jeg evner det ikke!
\par 7 Hvorhen skal jeg gå for din Ånd, og hvor skal jeg fly for dit Åsyn?
\par 8 Farer jeg op til Himlen, da er du der, reder jeg Leje i Dødsriget, så er du der;
\par 9 tager jeg Morgenrødens Vinger, fæster jeg Bo, hvor Havet ender,
\par 10 da vil din Hånd også lede mig der, din højre holde mig fast!
\par 11 Og siger jeg: "Mørket skal skjule mig, Lyset blive Nat omkring mig!"
\par 12 så er Mørket ej mørkt for dig, og Natten er klar som Dagen, Mørket er som Lyset.
\par 13 Thi du har dannet mine Nyrer, vævet mig i Moders Liv.
\par 14 Jeg vil takke dig, fordi jeg er underfuldt skabt; underfulde er dine Gerninger, det kender min Sjæl til fulde.
\par 15 Mine Ben var ikke skjult for dig, da jeg blev skabt i Løndom, virket i Jordens Dyb;
\par 16 som Foster så dine Øjne mig, i din Bog var de alle skrevet, Dagene var bestemt, før en eneste af dem var kommet.
\par 17 Hvor kostelige er dine Tanker mig, Gud, hvor stor er dog deres Sum!
\par 18 Tæller jeg dem, er de flere end Sandet, jeg vågner - og end er jeg hos dig.
\par 19 Vilde du dog dræbe de gudløse, Gud, måtte Blodets Mænd vige fra mig,
\par 20 de, som taler om dig på Skrømt og sværger falsk ved dit Navn.
\par 21 Jeg hader jo dem, der hader dig, HERRE, og væmmes ved dem, der står dig imod;
\par 22 med fuldt Had bader jeg dem, de er også mine Fjender.
\par 23 Ransag mig, Gud, og kend mit Hjerte, prøv mig og kend mine Tanker!
\par 24 Se, om jeg er på Smertens Vej, og led mig på Evigheds Vej!

\chapter{140}

\par 1 Til Korherren. Salme af David
\par 2 Red mig, HERRE, fra onde Mennesker, vær mig et Værn mod Voldsmænd,
\par 3 der pønser på ondt i Hjertet og daglig ægger til Strid.
\par 4 De hvæsser Tungen som Slanger, har Øglegift under deres Læber.
\par 5 Vogt mig, HERRE, for gudløses Hånd, vær mig et Værn mod Voldsmænd, som pønser på at bringe mig til Fald.
\par 6 Hovmodige lægger Snarer og Strikker for mig, breder et Net for min Fod, lægger Fælder for mig ved Vejen. - Sela.
\par 7 Jeg siger til HERREN: Du er min Gud, HERRE, lyt til min tryglende Røst!
\par 8 HERRE, Herre, min Frelses Styrke, du skærmer mit Hoved på Stridens Dag.
\par 9 Opfyld ej, HERRE, den gudløses Ønsker, lad ikke hans Råd have Fremgang!
\par 10 Lad dem ikke løfte Hovedet mod mig, lad deres Trusler ramme dem selv!
\par 11 Det regne på dem med gloende Kul, styrt dem i Dybet, ej stå de op!
\par 12 Lad ikke Bagtaleren holde sig i Landet, ondt ramme Voldsmanden Slag i Slag!
\par 13 Jeg ved, at HERREN vil føre de armes Sag og skaffe de fattige Ret.
\par 14 For vist skal retfærdige prise dit Navn, oprigtige bo for dit Åsyn.

\chapter{141}

\par 1 HERRE, jeg råber til dig, il mig til hjælp, hør min Røst, når jeg råber til dig;
\par 2 som Røgoffer, gælde for dig min Bøn, mine løftede Hænder som Aftenoffer!
\par 3 HERRE, sæt Vagt ved min Mund, vogt mine Læbers Dør!
\par 4 Bøj ikke mit Hjerte til ondt, til at gøre gudløs Gerning sammen med Udådsmænd; deres lækre Mad vil jeg ikke smage.
\par 5 Slår en retfærdig mig, så er det Kærlighed; revser han mig, er det Olie for Hovedet, ej skal mit Hoved vise det fra sig, end sætter jeg min Bøn imod deres Ondskab.
\par 6 Ned ad Klippens Skrænter skal Dommerne hos dem styrtes, og de skal høre, at mine Ord er liflige.
\par 7 Som når man pløjer Jorden i Furer, spredes vore Ben ved Dødsrigets Gab.
\par 8 Dog, mine Øjne er rettet på dig, o HERRE, Herre, på dig forlader jeg mig, giv ikke mit Liv til Pris!
\par 9 Vogt mig for Fælden, de stiller for mig, og Udådsmændenes Snarer;
\par 10 lad de gudløse falde i egne Gram, medens jeg går uskadt videre.

\chapter{142}

\par 1 Maskil af David. En Bøn. Dengang han var i Hulen
\par 2 Jeg løfter min røst og råber til Herren, jeg løfter min Røst og trygler HERREN,
\par 3 udøser min Klage for ham, udtaler min Nød for ham.
\par 4 Når Ånden vansmægter i mig, kender du dog min Sti. På Vejen, ad hvilken jeg vandrer, lægger de Snarer for mig.
\par 5 Jeg skuer til højre og spejder, men ingen vil kendes ved mig, afskåret er mig hver Tilflugt, ingen bryder sig om min Sjæl.
\par 6 HERRE, jeg råber til dig og siger: Du er min Tilflugt, min Del i de levendes Land!
\par 7 Lyt til mit Klageråb, thi jeg er såre ringe, frels mig fra dem, der forfølger mig, de er for stærke for mig;
\par 8 udfri min Sjæl af dens Fængsel, at jeg kan prise dit Navn! De retfærdige venter i Spænding på at du tager dig af mig.

\chapter{143}

\par 1 HERRE, hør min Bøn og lyt til min tryglen, bønhør mig i din Trofasthed, i din Retfærd,
\par 2 gå ikke i Rette med din Tjener, thi for dig er ingen, som lever, retfærdig!
\par 3 Thi Fjender forfølger min Sjæl, de træder mit Liv i Støvet, lader mig bo i Mørke som de, der for længst er døde.
\par 4 Ånden hensygner i mig, mit Hjerte stivner i Brystet.
\par 5 Jeg kommer fordums Dage i Hu, tænker på alle dine Gerninger, grunder på dine Hænders Værk.
\par 6 Jeg udbreder Hænderne mod dig, som et tørstigt Land så længes min Sjæl efter dig. - Sela.
\par 7 Skynd dig at svare mig, HERRE, min Ånd svinder hen; skjul ikke dit Åsyn for mig, så jeg bliver som de, der synker i Graven.
\par 8 Lad mig årle høre din Miskundhed, thi jeg stoler på dig. Lær mig den Vej, jeg skal gå, thi jeg løfter min Sjæl til dig.
\par 9 Fri mig fra mine Fjender, HERRE, til dig flyr jeg hen;
\par 10 lær mig at gøre din Vilje, thi du er min Gud, mig føre din gode Ånd ad den jævne Vej!
\par 11 For dit Navns Skyld, HERRE, holde du mig i Live, udfri i din Retfærd min Sjæl af Trængsel,
\par 12 udslet i din Miskundhed mine Fjender, tilintetgør alle, som trænger min Sjæl! Thi jeg er din Tjener.

\chapter{144}

\par 1 Lovet være HERREN, min Klippe, som oplærer mine hænder til Strid mine Fingre til Krig,
\par 2 min Miskundhed og min Fæstning, min Klippeborg, min Frelser, mit Skjold og den, jeg lider på, som underlægger mig Folkeslag!
\par 3 HERRE, hvad er et Menneske, at du kendes ved det, et Menneskebarn, at du agter på ham?
\par 4 Mennesket er som et Åndepust, dets Dage som svindende Skygge.
\par 5 HERRE, sænk din Himmel, stig ned og rør ved Bjergene, så at de ryger;
\par 6 slyng Lynene ud og adsplit Fjenderne, send dine Pile og indjag dem Rædsel;
\par 7 udræk din Hånd fra det høje, fri og frels mig fra store Vande,
\par 8 fra fremmedes Hånd, de, hvis Mund taler Løgn, hvis højre er Løgnehånd.
\par 9 Gud, jeg vil synge dig en ny Sang, lege for dig på tistrenget Harpe,
\par 10 du, som giver Konger Sejr og udfrier David, din Tjener.
\par 11 Fri mig fra det onde Sværd, frels mig fra fremmedes Hånd, de, hvis Mund taler Løgn, hvis højre er Løgnehånd.
\par 12 I Ungdommen er vore Sønner som højvoksne Planter, vore Døtre er som Søjler, udhugget i Tempelstil;
\par 13 vore Forrådskamre er fulde, de yder Forråd på, Forråd, vore Hjorde føder Tusinder, Titusinder på vore Marker,
\par 14 fede er vore Okser; intet Murbrud, ingen Udvandring, ingen Skrigen på Torvene.
\par 15 Saligt det Folk, der er således stedt, saligt det Folk, hvis Gud er HERREN!

\chapter{145}

\par 1 Jeg vil ophøje dig, min Gud, min konge, evigt og alt love dit Navn.
\par 2 Jeg vil love dig Dag efter Dag, evigt og altid prise dit Navn.
\par 3 Stor og højlovet er HERREN, hans Storhed kan ikke ransages.
\par 4 Slægt efter Slægt lovpriser dine Værker, forkynder dine vældige Gerninger.
\par 5 De taler om din Højheds herlige Glans, jeg vil synge om dine Undere;
\par 6 de taler om dine ræddelige Gerningers Vælde, om din Storhed vil jeg vidne;
\par 7 de udbreder din rige Miskundheds Ry og synger med Fryd om din Retfærd.
\par 8 Nådig og barmhjertig er HERREN, langmodig og rig på Miskundhed.
\par 9 God er HERREN mod alle, hans Barmhjertighed er over alle hans Værker.
\par 10 Dine Værker takker dig alle, HERRE, og dine fromme lover dig.
\par 11 De forkynder dit Riges Ære og taler om din Vælde
\par 12 for at kundgøre Menneskenes Børn din Vælde, dit Riges strålende Herlighed.
\par 13 Dit Rige står i al Evighed, dit Herredømme varer fra Slægt til Slægt. (Trofast er HERREN i alle sine Ord og miskundelig i alle sine Gerninger).
\par 14 HERREN støtter alle, der falder, og rejser alle bøjede.
\par 15 Alles Øjne bier på dig, du giver dem Føden i rette Tid;
\par 16 du åbner din Hånd og mætter alt, hvad der lever, med hvad det ønsker.
\par 17 Retfærdig er HERREN på alle sine Veje, miskundelig i alle sine Gerninger.
\par 18 Nær er HERREN hos alle, som kalder, hos alle, som kalder på ham i Sandhed.
\par 19 Han gør, hvad de, der frygter ham, ønsker, hører deres Råb og frelser dem,
\par 20 HERREN vogter alle, der elsker ham, men alle de gudløse sletter han ud.
\par 21 Min Mund skal udsige HERRENs Pris, alt Kød skal love hans hellige Navn evigt og altid.

\chapter{146}

\par 1 Halleluja! Pris HERREN, min Sjæl!
\par 2 Jeg vil prise HERREN hele mit Liv, lovsynge min Gud, så længe jeg lever.
\par 3 Sæt ikke eders Lid til Fyrster, til et Menneskebarn, der ikke kan hjælpe!
\par 4 Hans Ånd går bort, han bliver til Jord igen, hans Råd er bristet samme Dag.
\par 5 Salig den, hvis Hjælp er Jakobs Gud, hvis Håb står til HERREN hans Gud,
\par 6 som skabte Himmel og Jord, Havet og alf, hvad de rummer, som evigt bevarer sin Trofasthed
\par 7 og skaffer de undertrykte Ret, som giver de sultne Brød! HERREN løser de fangne,
\par 8 HERREN åbner de blindes Øjne, HERREN rejser de bøjede, HERREN elsker de retfærdige,
\par 9 HERREN vogter de fremmede, opholder faderløse og Enker, men gudløses Vej gør han kroget.
\par 10 HERREN er Konge for evigt, din Gud, o Zion, fra Slægt til Slægt. Halleluja!

\chapter{147}

\par 1 Halleluja! Ja, det er godt at lovsynge vor Gud, ja, det er lifligt, lovsang sømmer sig.
\par 2 Herren bygger Jerusalem, han samler de spredte af Israel,
\par 3 han læger dem, hvis Hjerte er sønderknust, og forbinder deres Sår;
\par 4 han fastsætter Stjemernes Tal og giver dem alle Navn.
\par 5 Vor Herre er stor og vældig, hans Indsigt er uden Mål;
\par 6 HERREN holder de ydmyge oppe, til Jorden bøjer han gudløse.
\par 7 Syng for HERREN med Tak, leg for vor Gud på Citer!
\par 8 Han dækker Himlen med Skyer, sørger for Regn til Jorden, lader Græs spire frem på Bjergene og Urter til Menneskers Brug;
\par 9 Føde giver han Kvæget og Ravneunger, som skriger;
\par 10 hans Hu står ikke til stærke Heste, han har ikke Behag i rapfodet Mand;
\par 11 HERREN har Behag i dem, der frygter ham, dem, der bier på hans Miskundhed.
\par 12 Lovpris HERREN, Jerusalem, pris, o Zion, din Gud!
\par 13 Thi han gør dine Portstænger stærke, velsigner dine Børn i din Midte;
\par 14 dine Landemærker giver han Fred, mætter dig med Hvedens Fedme;
\par 15 han sender sit Bud til Jorden, hastigt løber hans Ord,
\par 16 han lader Sne falde ned som Uld, som Aske spreder han Rim,
\par 17 som Brødsmuler sender han Hagl, Vandene stivner af Kulde fra ham;
\par 18 han sender sit Ord og smelter dem, de strømmer, når han rejser sit Vejr.
\par 19 Han kundgør sit Ord for Jakob, sine Vedtægter og Lovbud for Israel.
\par 20 Så gjorde han ikke mod andre Folk, dem kundgør han ingen Lovbud. Halleluja!

\chapter{148}

\par 1 Halleluja! Pris Herren i himlen, pris ham i det høje!
\par 2 Pris ham, alle hans engle, pris ham, alle hans hærskarer,
\par 3 pris ham, sol og måne, pris ham, hver lysende stjerne,
\par 4 pris ham, himlenes himle og vandene over himlene!
\par 5 De skal prise Herrens navn, thi han bød, og de blev skabt;
\par 6 han gav dem deres plads for evigt, han gav en lov, som de ej overtræder!
\par 7 Lad pris stige op til Herren fra jorden, I havdyr og alle dyb,
\par 8 Ild og hagl, sne og røg, storm, som gør hvad han siger,
\par 9 I bjerge og alle høje, frugttræer og alle cedre,
\par 10 I vilde dyr og alt kvæg, krybdyr og vingede fugle,
\par 11 I jordens konger og alle folkeslag, fyrster og alle jordens dommere,
\par 12 ynglinge sammen med jomfruer, gamle sammen med unge!
\par 13 De skal prise Herrens navn, thi ophøjet er hans navn alene, hans højhed omspender jord og himmel.
\par 14 Han løfter et horn for sit folk, lovprist af alle sine fromme, af Israels børn, det folk, der står ham nær. Halleluja!

\chapter{149}

\par 1 Halleluja! syng Herren en ny sang, hans Pris i de frommes Forsamling!
\par 2 Israel glæde sig over sin Skaber, over deres Konge fryde sig Zions Børn,
\par 3 de skal prise hans Navn under Dans, lovsynge ham med Pauke og Citer;
\par 4 thi HERREN har Behag i sit Folk, han smykker de ydmyge med Frelse.
\par 5 De fromme skal juble med Ære, synge på deres Lejer med Fryd,
\par 6 med Lovsang til Gud i Mund og tveægget Sværd i Hånd
\par 7 for at tage Hævn over Folkene og revse Folkeslagene,
\par 8 for at binde deres Konger med Lænker, deres ædle med Kæder af Jern
\par 9 og fuldbyrde på dem den alt skrevne Dom til Ære for alle hans fromme! Halleluja! -

\chapter{150}

\par 1 Halleluja! Pris Gud i hans Helligdom, pris ham i hans stærke Hvælving,
\par 2 pris ham for hans vældige Gerninger, pris ham for hans mægtige Storhed;
\par 3 pris ham med Hornets Klang, pris ham med Harpe og Citer,
\par 4 pris ham med Pauke og Dans, pris ham med Strengeleg og Fløjte,
\par 5 pris ham med klingre Cymbler, pris ham med gjaldende Cymbler;
\par 6 alt hvad der ånder, pris HERREN! Halleluja!


\end{document}