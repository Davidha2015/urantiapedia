\begin{document}

\title{Højsangen}


\chapter{1}

\par 1 Salomos Højsang
\par 2 Kys mig, giv mig Kys af din mund thi din Kærlighed er bedre end Vin.
\par 3 Lifligt dufter dine Salver, dit Navn er en udgydt Salve, derfor har Kvinder dig kær.
\par 4 Drag mig efter dig, kom, lad os løbe; Kongen tog mig ind i sine Kamre. Vi vil juble og glæde os i dig, prise din Hærlighed fremfor Vin. Med Rette har de dig kær.
\par 5 Jeg er sort, dog yndig, Jerusalems Døtre, som Kedars Telte, som Salmas Forhæng.
\par 6 Se ej på mig, fordi jeg er sortladen, fordi jeg er brændt af Solen. Min Moders Sønner vrededes på mig, til Vingårdsvogterske satte de mig - min egen Vingård vogted jeg ikke.
\par 7 Sig mig, du, som min Sjæl har kær, hvor du vogter din Hjord, hvor du holder Hvil ved Middag. Thi hvi skal jeg gå som en Landstryger ved dine Fællers Hjorde?
\par 8 Såfremt du ikke ved det, du fagreste blandt Kvinder, følg da kun Hjordens Spor og vogt dine Geder ved Hyrdernes Boliger.
\par 9 Ved Faraos Forspand ligner jeg dig, min Veninde.
\par 10 Dine Kinder er yndige med Snorene din Hals med Kæderne.
\par 11 Vi vil gøre dig Snore af Guld med Stænk af Sølv.
\par 12 Min Nardus spreder sin Duft, mens Kongen er til Bords;
\par 13 min Ven er mig en Myrrapose, der ligger ved mit Bryst,
\par 14 min Ven er mig en Koferklase fra En-Gedis Vingårde.
\par 15 Hvor du er fager, min Veninde, hvor du er fager, dine Øjne er Duer!
\par 16 Hvor du er fager, min Ven, ja dejlig er du, vort Leje er grønt,
\par 17 vor Boligs Bjælker er Cedre, Panelet Cypresser!

\chapter{2}

\par 1 Jeg er Sarons Rose, Dalenes Lilje
\par 2 Som en Lilje midt iblandt Torne er min Veninde blandt Piger.
\par 3 Som et Æbletræ blandt Skovens Træer er min Ven blandt unge Mænd. I hans Skygge har jeg Lyst til at sidde, hans Frugt er sød for min Gane.
\par 4 Til en Vinhal bragte han mig, hvor Mærket over mig er Kærlighed.
\par 5 Styrk mig med Rosinkager, kvæg mig med Æbler, thi jeg er syg af Kærlighed.
\par 6 Hans venstre er under mit Hoved, hans højre tager mig i Favn.
\par 7 Jeg besværger eder, Jerusalems Døtre, ved Gazeller og Markens Hjorte: Gør ikke Kærligheden Uro, væk den ikke, før den ønsker det selv!
\par 8 Hør! Der er min Ven! Ja se, der kommer han i Løb over Bjergene, i Spring over Højene.
\par 9 Min Ven er som en Gazel, han er som den unge Hjort. Se, nu står han alt bag vor Mur. Han ser gennem Vinduet, kigger gennem Gitteret.
\par 10 Min Ven stemmer i og siger så til mig: Stå op, min Veninde, du fagre, kom!
\par 11 Thi nu er Vinteren omme, Regntiden svandt, for hen,
\par 12 Blomster ses i Landet, Sangens Tid er kommet, Turtelduens Kurren høres i vort Land;
\par 13 Figentræets Småfrugter svulmer, Vinstokken blomstrer, udspreder Duft. Stå op, min Veninde, du fagre, kom,
\par 14 min Due i Fjeldets Kløfter, i Bjergvæggens Skjul! Lad mig skue din Skikkelse, høre din Røst, thi sød er din Røst og din Skikkelse yndig.
\par 15 Fang os de Ræve, de Ræve små,som hærger Vinen, vor blomstrende Vin!
\par 16 Min Ven er min, og jeg er hans,som vogter blandt Liljer;
\par 17 til Dagen svales og Skyggerne længes, kom hid, min Ven, og vær som Gazellen, som den unge Hjort på duftende Bjerge!

\chapter{3}

\par 1 På mit Leje om Natten søgte Ham, som min Sjæl har kær, jeg søgte, men fandt ham ikke.
\par 2 "Så står jeg op og går om i Byen, om På dens Gader og Torve og søger ham, som min Sjæl har kær. "Jeg søgte, men fandt ham ikke.
\par 3 Vægterne, som færdes i Byen, traf mig: "Så I mon ham, som min Sjæl har kær?"
\par 4 Knap var jeg kommet forbi dem, så fandt jeg ham, som min Sjæl har kær; jeg greb ham og slap ham ikke, før jeg fik ham ind i min Moders Hus, i hendes Kammer, som fødte mig.
\par 5 Jeg besværger eder, Jerusalems Døtre, ved Gazeller og Markens Hjorte: Gør ikke Kærligheden Uro, væk den ikke, før den ønsker det selv!
\par 6 Hvad er det, som kommer fra Ørkenen i Støtter af Røg, omduftet af Myrra og Røgelse, alskens Vellugt?
\par 7 Se, det er Salomos Bærestol, omgivet af tresindstyve Helte, Israels Helte,
\par 8 alle med Sværd i Hånd, oplærte til Krig, hver med sit Sværd ved Lænd mod Nattens Rædsler.
\par 9 Kong Salomo laved sig en Bærekarm af Træ fra Libanon,
\par 10 gjorde dens Støtter af Sølv, dens Arme af Guld; i Midten er Ibenholt indlagt, Sædet er Purpur.
\par 11 Jerusalems Døtre, gå ud og se på Kong Salomo, på Kronen, hans Moder kroned ham med på hans Brudefærds Dag, hans Hjertens Glædes Dag!

\chapter{4}

\par 1 Hvor du er fager, min veninde, hvor er du fager! Dine Øjne er Duer bag sløret, dit Hår som en Gedeflok bølgende ned fra Gilead,
\par 2 dine Tænder som en nyklippet Fåreflok, der kommer fra Bad, som alle har Tvillinger, intet er uden Lam;
\par 3 som en Purpursnor er dine Læber, yndig din Mund, din Tinding som et bristet Granatæble bag ved dit Slør;
\par 4 din Hals er som Davids Tårn, der er bygget til Udkig, tusinde Skjolde hænger derpå, kun Helteskjolde;
\par 5 dit Bryst som to Hjortekalve, Gazelle tvillinger, der græsser blandt Liljer.
\par 6 Til Dagen svales og Skyggerne længes, vil jeg vandre til Myrrabjerget og Vellugtshøjen.
\par 7 Du er fuldendt fager, min Veninde og uden Lyde.
\par 8 Kom med mig fra Libanon, Brud, kom med mig fra Libanon, stig ned fra Amanas Tinde, fra Senirs og Hermons Tinde, fra Løvers Huler, fra Panteres Bjerge!
\par 9 Du har fanget mig, min Søster, min Brud, du har fanget mig med et af dine Blikke, med en af din Halses Kæder.
\par 10 Hvor herlig er din Kærlighed, min Søster, min Brud, hvor din Kærlighed er god fremfor Vin, dine Salvers Duft fremfor alskens Vellugt!
\par 11 Dine Læber drypper af Sødme, min Brud, under din Tunge er Honning og Mælk; dine Klæders Duft er som Libanons Duft.
\par 12 Min Søster, min Brud er en lukket Have, en lukket Kilde, et Væld under Segl.
\par 13 Dine Skud er en Lund af Granattræer med kostelige Frugter, Kofer,
\par 14 Nardus og Kalmus og Kanel og alle Slags Vellugtstræer, Myrra og Safran og Aloe og alskens ypperlig Balsam.
\par 15 Min Haves Væld er en Brønd med rindende Vand og Strømme fra Libanon.
\par 16 Nordenvind, vågn, Søndenvind kom, blæs gennem min Have, så dens Vellugt spredes! Min Ven komme ind i sin Have og nyde dens udsøgte Frugt!

\chapter{5}

\par 1 Jeg kommer i min Have, min Søster, min Brud, jeg plukker min Myrra og Balsam, jeg spiser min Honning og Saft, jeg drikker min Vin og Mælk. Venner, spis og drik og berus jer i Kærlighed!
\par 2 Jeg sov, men mit hjerte våged; tys, da banked min ven: "Luk op for mig, o Søster, min Veninde, min Due, min rene, thi mit Hoved er fuldt af Dug, mine Lokker af Nattens Dråber."
\par 3 Jeg har taget min Kjortel af, skal jeg atter tage den på? Jeg har tvættet mine Fødder, skal jeg atter snavse dem til?
\par 4 Gennem Gluggen rakte min Ven sin Hånd, det brusede stærkt i mit Indre.
\par 5 Jeg stod op og åbned for min Ven; mine Hænder drypped af Myrra, mine Fingre af flydende Myrra, da de rørte ved Låsens Håndtag.
\par 6 Så lukked jeg op for min Ven, men min Ven var gået sin Vej. Jeg var ude af mig selv ved hans Ord. Jeg søgte, men fandt ham ikke, kaldte, han svared mig ikke.
\par 7 Vægterne, som færdes i Byen, traf mig, de slog og såred mig; Murens Vægtere rev Kappen af mig.
\par 8 Jeg besværger eder, Jerusalems Døtre: Såfremt I finder min Ven, hvad skal I da sige til ham? At jeg er syg af Kærlighed!
\par 9 "Hvad Fortrin har da, din Ven, du fagreste, blandt Kvinder? Hvad Fortrin har da din Ven, at du besværger os så?"
\par 10 Min Ven er hvid og rød, herlig blandt Titusinder,
\par 11 hans Hoved er det fineste Guld, hans Lokker er Ranker, sorte som Ravne,
\par 12 hans Øjne som Duer ved rindende Bække, badet i Mælk og siddende ved Strømme,
\par 13 hans Kinder som Balsambede; Skabe med Vellugt, hans Læber er Liljer, de drypper, af flydende Myrra,
\par 14 hans Hænder er Stænger af Guld, fyldt med Rubiner, hans Liv en Elfenbensplade, besat med Safirer,
\par 15 hans Ben er Søjler af Marmor På Sokler af Guld, hans Skikkelse som Libanon, herlig som Cedre,
\par 16 hans Gane er Sødme, han er idel Ynde. Sådan er min elskede, sådan min Ven, Jerusalems Døtre.

\chapter{6}

\par 1 Hvor er din Ven gået hen, du fagreste blandt Kvinder? Hvor har din ven vendt sig hen? Vi vil søge ham med dig.
\par 2 Min Ven gik ned i sin Have, ti lBalsambedene, for at vogte sin Hjord i Haverne og sanke Liljer.
\par 3 Jeg er min Vens, og min Ven er min, han, som vogter blandt Liljer.
\par 4 Du er fager, min Veninde, som Tirza, yndig som Jerusalem, frygtelig som Hære under Banner.
\par 5 Vend dine Øjne fra mig, de forvirrer mig så! Dit Hår er som en Gedeflok, bølgende ned fra Gilead.
\par 6 dine Tænder som en Fåreflok, der kommer fra Bad, som alle har Tvillinger, intet er uden Lam;
\par 7 din Tinding er et bristet Granatæble bag ved dit Slør.
\par 8 Dronningernes Tal er tresindstyve, Medhustruernes firsindstyve, på Terner er der ej Tal.
\par 9 Men een er hun, min Due, min rene, hun, sin Moders eneste, hun, sin Moders Kælebarn. Blev hun set af Piger, fik hun Pris, af Dronninger og Medhustruer Hyldest.
\par 10 Hvo er hun, der titter frem som Morgenrøden, fager som Månen, skær som Solen, frygtelig som Hære under Banner?
\par 11 Jeg gik ned i Nøddehaven for at se, hvor det grønnes i Dale for at se, om Vintræet skød, om Granattræet nu stod i Blomst.
\par 12 Før jeg vidste af det, satte min Sjæl mig på mit ædle Folks Vogne.
\par 13 Vend dig, vend dig, Sulamit, vend dig, vend dig, så vi kan se dig!"Hvad vil I se på Sulamit, mens Sværddansen trædes?"

\chapter{7}

\par 1 Hvor skønne er dine Trin i Skoene, du ædelbårne! Dine Hofters Runding er som Halsbånd, Kunstnerhånds Værk,
\par 2 dit Skød som det runde Bæger, ej savne det Vin, dit Liv som en Hvededynge, hegnet af Liljer;
\par 3 dit Bryst som to Hjortekalve, Gazelletvillinger,
\par 4 din Hals som Elfenbenstårnet, dine Øjne som Hesjbons damme ved Bat-Rabbims Port, din Næse som Libanons Tårn, der ser mod Damaskus,
\par 5 Hovedet på dig som Karmel, dit Hoveds Lokker som Purpur; en Konge er fanget i Garnet.
\par 6 Hvor er du fager og yndig, du elskede, yndefulde!
\par 7 Som Palmen, så er din Vækst, dit Bryst som Klaser.
\par 8 Jeg tænker: Jeg vil op i Palmen, gribe fat i dens Stilke; dit Bryst skal være som Vinstokkens Klaser, din Næses Ånde som Æbleduft,
\par 9 din Gane som ædel Vin, der liflig flyder ind i min Mund, glider over mine Læber og Tænder.
\par 10 Jeg er min Vens, og til mig står hans Attrå.
\par 11 Kom min Ven, vi vil ud på Landet, blive i Landsbyer Natten over;
\par 12 Vingårde søger vi årle, vi ser, om Vinstokken skyder, om Knopperne åbnes, Granattræet blomstrer. Der giver jeg dig min Kærlighed.
\par 13 Kærlighedsæblerne dufter, for vor Dør er al Slags Frugt, ny og gammel tillige; til dig, min Ven, har jeg gemt dem.

\chapter{8}

\par 1 Oh, var du min broder, som died min moders bryst! Jeg kyssed dig derude, når vi mødtes, og blev ikke agtet ringe,
\par 2 tog dig ind i min Moders hus, i min moders kamre, gav dig krydret vin at drikke, Granatæblers Most.
\par 3 Hans venstre under mit hoved, hans højre tager mig i favn.
\par 4 Jeg besværger eder, Jerusalems Døtre: Gør ikke Kærligheden Uro, væk den ikke før den ønsker det selv!
\par 5 Hvem er hun, der kommer fra Ørkenen, støttet til sin ven?"Under Æbletræet vækked jeg dig; der nedkom din moder med dig, der nedkom hun, som dig fødte."
\par 6 Læg mig som en seglring om dit hjerte, som et Armbånd om din Arm! Thi Kærlighed er stærk som døden, Nidkærhed hård som Dødsriget; i dens gløder er Brændende Glød, dens lue er HERRENS Lue.
\par 7 Mange Vande kan ikke slukke den, Strømme skylle den bort. Gav nogen alt Gods i sit Hus for Kærlighed, Hvem vilde agte ham ringe?
\par 8 Vi har en lille Søster, som endnu ej har bryster; hvad gør vi med med vor Søster, den dag hun får en Bejler?
\par 9 Er hun en Mur, så bygger vi en krone af sølv derpå, men er hun en Dør, så spærrer vi den med Cederplanke.
\par 10 Jeg er en Mur, Mine Bryster Tårne. Da blev jeg i hans Øjne som en, der finder Fred.
\par 11 Salomo havde en Vingård i Ba'al-Hamon, til vogtere gav han den Vingård; hver kunne tjene tusind sekel Sølv på dens Frugt.
\par 12 Jeg har for mig selv min Vingård; de tusinde, Salomo, er dine, to hundrede deres, som vogter dens Frugt.
\par 13 Du, som bor i Haverne, Vennerne lytter, lad mig høre din røst!
\par 14 Fly, min Ven, og vær som en Gazel, som den unge Hjort på Balsambjerge!


\end{document}