\begin{document}

\title{Isaiah}


\chapter{1}

\par 1 Det Syn, Esajas, Amoz's Søn, skuede om Juda og Jerusalem, i de dage da Uzzija, Jotam, Akaz og Ezekias var Konger i Juda.
\par 2 Hør, I Himle, lyt, du Jord, thi HERREN taler: Børn har jeg opfødt og fostret, men de forbrød sig imod mig.
\par 3 En Okse kender sin Ejer, et Æsel sin Herres Krybbe; men Israel kender intet, mit Folk kan intet fatte.
\par 4 Ve det syndefulde Folk, en brødetynget Slægt, Ugerningsmænds Æt, vanartede Børn! De svigtede HERREN, lod hånt om Israels Hellige, vendte ham Ryg.
\par 5 Kan I tåle flere Hug, siden I stadig falder fra? Kun Sår er Hovedet, sygt hele Hjertet;
\par 6 fra Fodsål til isse er intet helt kun Flænger, Strimer og friske Sår; de er ej trykket ud, ej heller forbundet og ikke lindret med Olie.
\par 7 Eders Land er øde, eders Byer brændt, fremmede æder eders Jord for eders Øjne så øde som ved Sodomas Undergang.
\par 8 Zions Datter er levnet som en Hytte i en Vingård, et Vagtskur i en Græskarmark en omringet By.
\par 9 Havde ikke Hærskarers HERRE levnet os en Rest, da var vi som Sodoma, ligned Gomorra.
\par 10 Lån Øre til HERRENs Ord i Sodomadommere, lyt til vor Guds Åbenbaring, du Gomorrafolk!
\par 11 Hvad skal jeg med alle eders Slagtofre? siger HERREN;jeg er mæt af Væderbrændofre, af Fedekalves Fedt, har ej Lyst til Blod af Okser og Lam og Bukke.
\par 12 Når I kommer at stedes for mit Åsyn, hvo kræver da af jer, at min Forgård trampes ned?
\par 13 Bring ej flere tomme Afgrødeofre, vederstyggelig Offerrøg er de mig! Nymånefest, Sabbat og festligt Stævne jeg afskyr Uret og festlig Samling.
\par 14 Eders Nymånefester og Højtider hader min, Sjæl de er mig en Byrde, jeg er træt af at bære.
\par 15 Breder I Hænderne ud, skjuler jeg Øjnene for jer. Hvor meget I så end beder, jeg hører det ikke. Eders Hænder er fulde af Blod;
\par 16 tvæt jer, rens jer, bort med de onde Gerninger fra mine Øjne! Hør op med det onde,
\par 17 lær det gode, læg Vind på, hvad Ret er; hjælp fortrykte, skaf faderløse Ret, før Enkens Sag!
\par 18 Kom, lad os gå i Rette med hinanden, siger HERREN. Er eders Synder som Skarlagen, de skal blive hvide som Sne; er de end røde som Purpur, de skal dog blive som Uld.
\par 19 Lyder I villigt, skal I æde Landets Goder;
\par 20 står I genstridigt imod, skal I ædes af Sværd. Thi HERRENs Mund har talt.
\par 21 At den skulde ende som Skøge, den trofaste By, Zion, så fuld af Ret, Retfærdigheds Hjem, men nu er der Mordere.
\par 22 Dit Sølv er blevet til Slagger, din Vin er spædet med Vand.
\par 23 Tøjlesløse er dine Førere, Venner med Tyve; Gaver elsker de alle, jager efter Stikpenge, skaffer ej faderløse Ret og tager sig ikke af Enkens Sag.
\par 24 Derfor lyder det fra Herren, Hærskarers HERRE, Israels Vældige: "Min Gengælds Ve over Avindsmænd, min Hævn over Fjender!
\par 25 Jeg vender min Hånd imod dig, renser ud dine Slagger i Ovnen og udskiller alt dit Bly.
\par 26 Jeg giver dig Dommere som fordum, Rådsherrer som før; så kaldes du Retfærdigheds By, den trofaste Stad.
\par 27 Zion genløses ved Ret, de omvendte der ved Retfærd.
\par 28 Men Overtrædere og Syndere knuses til Hobe; hvo HERREN svigter, forgår.
\par 29 Thi Skam vil I få af de Ege, I elsker, Skuffelse af Lundene, I sætter så højt;
\par 30 thi I bliver som en Eg med visnende Løv, som en Lund, hvor der ikke er Vand.
\par 31 Den stærke bliver til Blår, hans Værk til en Gnist; begge brænder med hinanden, og ingen slukker.

\chapter{2}

\par 1 Dette er, hvad Esajas, Amoz's Søn, skuede om Jerusalem:
\par 2 Det skal ske i de sidste Dage, at HERRENs Huses Bjerg, grundfæstet på Bjergenes Top, skal løfte sig op over Højene. Did skal Folkene strømme
\par 3 og talrige Folkeslag vandre: "Kom, lad os drage til HERRENs Bjerg, til Jakobs Guds Hus; han skal lære os sine Veje, så vi kan gå på hans Stier; thi fra Zion udgår Åbenbaring, fra Jerusalem, HERRENs Ord:"
\par 4 Da dømmer han Folk imellem, skifter Ret mellem talrige Folkeslag; deres Sværd skal de smede til Plovjern, deres Spyd til Vingårdsknive; Folk skal ej løfte Sværd mod Folk, ej øve sig i Våbenfærd mer.
\par 5 Kom, Jakobs Hus, lad os vandre i HERRENs Lys!
\par 6 Thi du forskød dit Folk, Jakobs Hus; de er fulde af Østens Væsen og spår, som var de Filistre, giver Folk fra Udlandet Håndslag.
\par 7 Deres Land er fuldt af Sølv og Guld, talløse er deres Skatte; deres Land er fuldt af Heste, talløse er deres vogne;
\par 8 deres Land er fuldt af Afguder, de tilbeder Værk af deres Hænder, Ting, deres Fingre har lavet.
\par 9 Men bøjes skal Mennesket, Manden ydmyges, tilgiv dem ikke!
\par 10 Gå ind i Klippen, skjul dig, i Støvet for HERRENs Rædsel, hans Højheds Herlighed!
\par 11 Sine stolte Øjne skal Mennesket slå ned, Mændenes Hovmod skal bøjes, og HERREN alene være høj på hin Dag.
\par 12 Thi en Dag har Hærskarers HERRE mod alt det høje og knejsende, mod alt ophøjet og stolt,
\par 13 med alle Libanons Cedre, de knejsende høje, og alle Basans Ege,
\par 14 mod alle knejsende Bjerge og alle høje Fjelde,
\par 15 mod alle stolte Tårne og alle stejle Mure,
\par 16 mod alle Tarsisskibe og hver en kostelig Ladning.
\par 17 Da skal Menneskets Stolthed bøjes, Mændenes Hovmod ydmyges. og HERREN alene være høj på hin Dag.
\par 18 Afguderne skal helt forsvinde.
\par 19 Og man skal gå ind i Klippehuler og Jordhuller for HERRENs Rædsel, hans Højheds Herlighed når han står op for at forfærde Jorden.
\par 20 På hin Dag skal Mennesket slænge sine Guder af Sølv og Guld, som han lavede sig for at tilbede dem, hen til Muldvarpe og Flagermus
\par 21 for at gå ind i Klipperevner og Fjeldkløfter for HERRENs Rædsel, hans Højheds Herlighed, når han står op for at forfærde Jorden.
\par 22 Slå ikke mer eders Lid til Mennesker, i hvis Næse der kun er flygtig Ånde, thi hvad er de at regne for?

\chapter{3}

\par 1 Thi se, HERREN, Hærskarers HERRE, fratager Jerusalem og Juda støtte og Stav, hver Støtte af Brød og hver støtte af vand:
\par 2 Helt og Krigsmand, Dommer og Profet, Spåmand og Ældste,
\par 3 Halvhundredfører og Stormand, Rådsherre og, Håndværksmester og den, der er kyndig i Trolddom.
\par 4 Jeg giver dem Drenge til Øverster, Drengekådhed skal herske over dem.
\par 5 I Folket undertrykker den ene den anden, hver sin Næste; Dreng sætter sig op imod Olding, Usling mod Hædersmand.
\par 6 Når da en Mand tager fat på en anden i hans Fædrenehus og siger: "Du har en Kappe, du skal være vor Hersker, under dig skal dette faldefærdige Rige stå!"
\par 7 så svarer han på hin Dag: "Jeg vil ikke være Sårlæge! Jeg har hverken Brød eller Klæder i Huset, gør ikke mig til Folkets Overhoved!"
\par 8 Thi Jerusalem snubler, og Juda falder, fordi de med Tunge og Gerning er mod HERREN i Trods mod hans Herligheds Øjne.
\par 9 Deres Ansigtsudtryk vidner imod dem; de kundgør deres Synd som Sodoma, dølger intet. Ve deres Sjæl, de styrted sig selv i Våde.
\par 10 Salige de retfærdige, dem går det godt, deres Gerningers frugt skal de nyde;
\par 11 ve den gudløse, ham går det ilde; han får, som hans Hænder har gjort.
\par 12 Mit, Folk har en Dreng ved Styret, og over det hersker Kvinder. Dine Ledere, mit Folk, leder, vild, gør Vejen, du, vandrer, vildsom.
\par 13 Til Rettergang er HERREN trådt frem, han tår og vil dømme sit Folk.
\par 14 HERREN møder til Doms med de Ældste i sit Folk og dets Fyrster: "Det er jer, som gnaved Vingården af, I har Rov fra den arme til Huse.
\par 15 Hvor kan I træde på mit Folk og male de arme sønder?" så lyder det fra Herren, Hærskarers HERRE.
\par 16 HERREN siger: Eftersom Zions Døtre bryster sig og går med knejsende Nakke og kælne Blikke, går med trippende Gang og med raslende Ankelkæder -
\par 17 gør Herren issen skaldet på Zions Døtre og blotter deres Tindingers Lokker.
\par 18 På hin Dag afriver Herren al deres Pynt: Ankelringe, Pandebånd, Halvmåner,
\par 19 Perler, Armbånd, Flor,
\par 20 Hovedsmykker, Ankelkæder, Bælter, Lugtedåser, Trylleringe,
\par 21 Fingerringe,Næseringe,
\par 22 Festklæder, Underdragter, Sjaler, Tasker,
\par 23 Spejle, Lin, Hovedbånd og Slør.
\par 24 For Vellugt kommer der Stank, i Stedet for Bælte Reb, for Fletninger skaldet Isse, for Stadsklæder Sæk om Hofte, for Skønhedsmærke Brændemærke.
\par 25 Dine Mænd skal falde for Sværd, dit unge Mandskab i Strid.
\par 26 Hendes Porte skal sukke og klage og hun sidde ensom på Jorden.

\chapter{4}

\par 1 Syv Kvinder skal på hin Dag gribe fat i een Mand og sige: "Vi vil æde vort eget Brød og holde os selv med Klæder, blot vi må bære dit Navn. Tag Vanæren fra os!"
\par 2 På hin Dag Bliver HERRENs spire til Fryd og Ære og Landets Frugt til Stolthed og Smykke for de undslupne i Israel.
\par 3 Den, som er levnet i Zion og blevet tilovers i Jerusalem, skal kaldes hellig, enhver, der er indskrevet til Livet i Jerusalem,
\par 4 når Herren får aftvættet Zions Døtres Smuds og bortskyllet Jerusalems Blodskyld fra dets Midte med Doms og Udrensnings Ånd.
\par 5 Da skaber Herren over hvert et Sted på Zions Bjerg og over dets Festforsamlinger en Sky om Dagen og Røg med luende Ildskær om Natten; thi over alt, hvad herligt er, skal der være et Dække
\par 6 og Ly til Skygge mod Hede og til Skærm og Skjul mod Skybrud og Regn.

\chapter{5}

\par 1 Jeg vil synge en Sang om min Ven, en Kærlighedssang om hans Vingård: Min Ven, han havde en Vingård på en frugtbar Høj.
\par 2 Han grov den, rensed den for Sten og plantede ædle Ranker; han bygged et Vagttårn deri og huggede også en Perse. Men den bar vilde Druer, skønt han ventede Høst af ædle.
\par 3 Og nu, Jerusalems Borgere, Judas Mænd, skift Ret mellem mig og min Vingård!
\par 4 Hvad mer var at gøre ved Vingården, hvad lod jeg ugjort? Hvi bar den vilde Druer, skønt jeg ventede Høst af ædle?
\par 5 Så vil jeg da lade jer vide, hvad jeg vil gøre ved min Vingård: Nedrive dens Hegn, så den ædes op, nedbryde dens Mur, så den trampes ned!
\par 6 Jeg lægger den øde; den skal ikke beskæres og ikke graves, men gro sammen i Torn og Tidsel; og Skyerne giver jeg Påbud om ikke at sende den Regn.
\par 7 Thi Hærskarers HERREs Vingård er Israels Hus, og Judas Mænd er hans Yndlingsplantning. Han vented på Retfærd se, der kom Letfærd, han vented på Lov se, Skrig over Rov!
\par 8 Ve dem, der føjer Hus til Hus, dem, der lægger Mark til Mark, så der ikke er Plads tilbage, men kun I har Landet i Eje.
\par 9 Det lyder i mine Ører fra Hærskarers HERRE: "For vist skal de mange Huse blive øde, de store og smukke skal ingen bebo;
\par 10 thi på ti Tønder Vinland skal høstes en Bat, af en Homers Udsæd skal høstes en Efa.
\par 11 Ve dem, der årle jager efter Drik og ud på Natten blusser af Vin!
\par 12 Med Citre og Harper holder de Gilde, med Håndpauker, Fløjter og Vin, men ser ikke HERRENs Gerning, har ej Syn for hans Hænders Værk.
\par 13 Derfor skal mit Folk føres bort, før det ved det, dets Adel blive Hungerens Bytte, dets Hob vansmægte af Tørst.
\par 14 Derfor vokser Dødsrigets Gridskhed, det spiler sit Gab uden Grænse; dets Stormænd styrter derned, dets larmende, lystige Slæng.
\par 15 Mennesket bøjes, og Manden ydmyges, de stolte slår Øjnene ned;
\par 16 men Hærskarers HERRE ophøjes ved Dommen, den hellige Gud bliver helliget ved Retfærd.
\par 17 Og der går Får på Græs, Geder afgnaver omkomnes Tomter.
\par 18 Ve dem, der trækker Straffen hid med Brødens Skagler og Syndebod hid som med Vognreb,
\par 19 som siger: "Lad ham skynde sig, haste med sit Værk, så vi får det at se; lad Israels Helliges Råd dog komme snart, at vi kan kende det!"
\par 20 Ve dem, der kalder ondt for godt og godt for ondt, gør Mørke til Lys og Lys til Mørke, gør beskt til sødt og sødt til beskt!
\par 21 Ve dem, der tykkes sig vise og er kloge i egne Tanker!
\par 22 Ve dem, der er Helte til at drikke Vin og vældige til at blande stærke Drikke,
\par 23 som for Gave giver den skyldige Ret og røver den skyldfri Retten, han har.
\par 24 Derfor, som Ildens Tunge æder Strå og Hø synker sammen i Luen, så skal deres Rod blive rådden, deres Blomst henvejres som Støv; thi om Hærskarers HERRERs Lov lod de hånt og ringeagted Israels Helliges Ord.
\par 25 Så blusser da HERRENs Vrede mod hans Folk, og han udrækker Hånden imod det og slår det, så Bjergene skælver og Ligene ligger som Skarn på Gaden. Men trods alt har hans Vrede ej lagt sig, hans Hånd er fremdeles rakt ud.
\par 26 For et Folk i det fjerne løfter han Banner og fløjter det hid fra Jordens Ende; og se, det kommer hastigt og let.
\par 27 Ingen iblandt dem er træt eller snubler, ingen blunder, og ingen sover; Bæltet om Lænden løsnes ikke, Skoens Rem springer ikke op;
\par 28 hvæssede er dets Pile, alle dets Buer spændte; som Flint er Hestenes Hove, dets Vognhjul som Hvirvelvind.
\par 29 Det har et Brøl som en Løve, brøler som unge Løver, brummende griber det Byttet, bjærger det, ingen kan fri det.
\par 30 Men på hin Dag skal der bryde en Brummen løs imod det, som når Havet brummer; og skuer det ud over Jorden, se, da er der Trængselsmørke, Lyset slukkes af tykke Skyer.

\chapter{6}

\par 1 I Kong Uzzijas Dødsår så jeg Herren sidde på en såre høj Trone, og hans Slæb fyldte Helligdommen.
\par 2 Serafer stod hos ham, hver med seks Vinger; med de to skjulte de Ansigtet, med de to Fødderne, og med de to fløj de;
\par 3 og de råbte til hverandre: "Hellig, hellig, hellig er Hærskarers HERRE, al Jorden er fuld af hans Herlighed!"
\par 4 Og Dørstolpernes Hængsler rystede ved Råbet, medens Templet fyldtes af Røg.
\par 5 Da sagde jeg: "Ve mig, det er ude med mig, thi jeg er en Mand med urene Læber, og jeg bor i et Folk med urene Læber, og nu har mine Øjne set Kongen, Hærskarers HERRE!"
\par 6 Men en af Seraferne fløj hen til mig; og han havde i Hånden et glødende Kul, som han med en Tang havde taget fra Alteret;
\par 7 det lod han røre min Mund og sagde: "Se, det har rørt dine Læber; din Skyld er borte, din Synd er sonet!"
\par 8 Så hørte jeg Herren sige: "Hvem skal jeg sende, hvem vil gå Bud for os?" Og jeg sagde: "Her er jeg, send mig!"
\par 9 Da sagde han: "Gå hen og sig til det folk: Hør kun, dog skal I intet fatte, se kun, dog skal I intet indse!
\par 10 Gør Hjertet sløvt på dette Folk, gør dets Ører tunge, dets Øjne blinde, så det ikke kan se med Øjnene, ej heller høre med Ørene, ej heller fatte med Hjertet og omvende sig og læges."
\par 11 Jeg spurgte: "Hvor længe, Herre?" Og han svarede: "Til Byerne er øde, uden Beboere, og Husene uden et Menneske, og Ager jorden ligger som Ørk!"
\par 12 Og HERREN vil fjerne Menneskene, og Tomhed skal brede sig i Landet;
\par 13 og er der endnu en Tiendedel deri, skal også den udryddes som en Terebinte eller Eg, af hvilken en Stub bliver tilbage, når den fældes. Dens Stub er hellig Sæd.

\chapter{7}

\par 1 Og det skete i de Dage da Akaz, Jotams Søn, Uzzijas Sønnesøn, var Konge i Juda, at Kong Rezin af Syrien og Remaljas Søn, kong Peka af Israel, drog op for at angribe Jerusalem, hvad de dog ikke var stærke nok til.
\par 2 Da det meldtes Davids Hus, at Syrerne havde lejret sig i Efraim, skjalv hans og hans Folks Hjerte, som Skovens Træer skælver for Vinden.
\par 3 Så sagde HERREN til Esajas: "Med din Søn Sjearjasjub" skal du gå Akaz i Møde ved Enden af Øvredammens Vandledning ved Vejen til Blegepladsen
\par 4 og sige til ham: Tag dig i Vare og hold dig i Ro! Frygt ikke og lad ikke dit Hjerte ængste sig for disse to rygende Brandstumper, for Rezins og Syriens og Remaljas Søns fnysende Vrede!
\par 5 Fordi Syrien, Efraim og Remaljas Søn har lagt onde Råd op imod dig og siger:
\par 6 Lad os drage op mod Juda og indjage det Skræk, lad os tilrive os det og gøre Tabeals Søn til Konge der!
\par 7 derfor, så siger den Herre HERREN: Det skal ikke lykkes; det skal ikke ske!
\par 8 Thi Syriens Hoved er Damaskus, og Damaskus's Hoved er Rezin, og om fem og tresindstyve År er Efraim knust og ikke længer et Folk.
\par 9 Og Efraims Hoved er Samaria, og Samarias Hoved er Remaljas Søn.
\par 10 Fremdeles sagde HERREN til Akaz:
\par 11 "Kræv dig et Tegn af HERREN din Gud nede i Dødsriget eller oppe i Himmelen!"
\par 12 Men Akaz svarede: "Jeg kræver intet, jeg frister ikke HERREN."
\par 13 Da sagde Esajas: "Hør nu, Davids Hus! Er det eder ikke nok at trætte Mennesker, siden I også trætter min Gud?
\par 14 Derfor vil Herren selv give eder et Tegn: Se, Jomfruen bliver frugtsommelig og føder en Søn, og hun kalder ham Immanuel".
\par 15 Surmælk og Vildhonning skal være hans Føde, ved den Tid han ved at vrage det onde og vælge det gode;
\par 16 thi før Drengen ved at vrage det onde og vælge det gode, skal Landet, for hvis to Konger du gruer, være folketomt.
\par 17 Over dig, dit Folk og din Faders Hus vil Herren bringe Dage, hvis Lige ikke har været, siden Efraim rev sig løs fra Juda: Assyrerkongen!"
\par 18 På hin Dag skal HERREN fløjte ad Fluerne ved Udløbet af Ægyptens Strømme og ad Bierne i Assyrien;
\par 19 og de skal komme og alle til Hobe kaste sig over Dalkløfter og Klipperevner, over hvert Tjørnekrat og hvert Vandingssted.
\par 20 På hin Dag afrager Herren med en hinsides Floden lejet Ragekniv, Assyrerkongen, både Hovedhåret og Kroppens Hår; ja, selv Skægget skraber den af.
\par 21 På hin Dag kan en Mand holde sig en ung Ko og et Par Får;
\par 22 og på Grund af den megen Mælk, de giver, skal han spise Sur mælk; thi Surmælk og Vildhonning skal enhver, der er tilbage i Landet, spise.
\par 23 På hin Dag skal det ske, at hvert et Sted, hvor der nu er tusind Vinstokke, tusind Sekel Sølv værd, skal blive til Torn og Tidsel;
\par 24 med Pil og Bue kommer man der, thi hele Landet skal blive til Torn og Tidsel;
\par 25 og alle de Bjerge, der nu dyrkes med Hakke, skal man holde sig fra af Frygt for Torn og Tidsel. Det bliver Overdrev for Okser og trædes ned af Får.

\chapter{8}

\par 1 Og HERREN sagde til mig: "Tag dig en stor Tavle og skriv derpå med Menneskeskrift: Hurtigt-Bytte, Hastigt-Rov!,
\par 2 Og tag mig pålidelige Vidner, Præsten Urija og Zekarja, Jeberekjahus Søn!"
\par 3 Og jeg nærmede mig Profetinden, og hun blev frugtsommelig og fødte en Søn. Så sagde HERREN til mig: "Kald ham Hurtigt-Bytte, Hastigt-Rov!
\par 4 Thi før Drengen kan sige Fader og Moder, skal Rigdommene fra Damaskus og Byttet fra Samaria bringes til Assyrerkongen!"
\par 5 Fremdeles sagde HERREN til mig:
\par 6 Eftersom dette Folk lader hånt om Siloas sagte rindende Vande i Angst for Bezin og Remaljas Søn,
\par 7 se, så lader Herren Flodens Vande, de vældige, store, oversvømme dem, Assyrerkongen og al hans Herlighed; over alle sine Bredder skal den gå, trænge ud over alle sine Diger,
\par 8 styrte ind i Juda, skylle over, vælte frem og nå til Halsen; og dens udbredte Vinger skal fylde dit Land, så vidt det når Immanuel!
\par 9 I Folkeslag, mærk jer det med Rædsel, lyt til, alle fjerne Lande: Rust jer, I skal ræddes, rust jer, I skal ræddes.
\par 10 Læg Råd op, det skal dog briste, gør Aftale, det slår dog fejl, thi - Immanuel!
\par 11 Thi så sagde HERREN til mig, da hans Hånd greb mig med Vælde, og han advarede mig mod at vandre på dette Folks Vej:
\par 12 Kald ikke alt Sammensværgelse, hvad dette Folk kalder Sammensværgelse, frygt ikke, hvad det frygter, og ræddes ikke!
\par 13 Hærskarers HERRE, ham skal I holde hellig, han skal være eders Frygt, han skal være eders Rædsel.
\par 14 Han bliver en Helligdom, en Anstødssten og en Klippe til Fald for begge Israels Huse og en Snare og et Fangegarn for Jerusalems Indbyggere,
\par 15 og mange iblandt dem skal snuble, falde og kvæstes, fanges og bindes.
\par 16 Bind Vidnesbyrdet til og sæt Segl for Læren i mine disciples Sind!
\par 17 Jeg bier på HERREN, han, som dølger sit Åsyn for Jakobs Hus, til ham står mit Håb:
\par 18 Se, jeg og de Børn, HERREN gav mig, er Varsler og Tegn i Israel fra Hærskarers HERRE, som bor på Zions Bjerg.
\par 19 Og siger de til eder: "Søg Genfærdene og Ånderne, som hvisker og mumler!" skal et Folk ikke søge sin Gud, skal man søge de døde for de levende?
\par 20 Nej! Til Læren og Vidnesbyrdet! Således skal visselig de komme til at tale, som nu er uden Morgenrøde.
\par 21 Han skal vanke om i Landet, trykket og hungrig. Og når han hungrer, skal han blive rasende og bande sin Konge og sin Gud.
\par 22 eller skuer han ud over Jorden, se da er der Trængsel og Mørke, knugende Mulm; i Bælgmørke er han stødt ud.

\chapter{9}

\par 1 Men engang skal der ikke længer være Mørke i det Land, hvor der nu er Trængsel; i Fortiden bragte han Skændsel over Zebulons og Naftalis Land, Men i Fremtiden bringer han Ære over Vejen langs Søen, Landet hinsides Jordan, Hedningernes Kreds.
\par 2 Det Folk, som vandrer i Mørke, skal skue så stort et Lys; Lys stråler frem over dem, som bor i Mulmets Land.
\par 3 Du gør Fryden mangfoldig, Glæden stor, de glædes for dit Åsyn, som man glædes i Høst, ret som man jubler, når Bytte deles.
\par 4 Thi dets tunge Åg og Stokken til dets Ryg, dets Drivers Kæp, har du brudt som på Midjans Dag;
\par 5 ja, hver en Støvle, der tramper i Striden, og Kappen, der søles i Blod, skal brændes og ende som Luernes Rov.
\par 6 Thi et Barn er født os, en Søn er os givet, på hans Skulder skal Herredømmet hvile; og hans Navn skal være: Underfuld Rådgiver, Vældig Gud, Evigheds Fader, Fredsfyrste.
\par 7 Stort bliver Herredømmet, endeløs Freden over Davids Trone og over hans Rige, at det må grundes og fæstnes ved Ret og Retfærd fra nu og til evig Tid. Hærskarers HERREs Nidkærhed gør det.
\par 8 Et Ord sender Herren mod Jakob, i Israel slår det ned;
\par 9 alt Folket får det at kende, Efraim og Samarias Borgere. Thi de siger i Hovmod og Hjertets Stolthed:
\par 10 "Teglsten faldt, vi bygger med Kvader, Morbærtræer blev fældet, vi får Cedre i Stedet!"
\par 11 Da rejser HERREN dets Uvenner mod det og ægger dets Fjender op,
\par 12 Syrerne forfra, Filisterne bagfra, de æder Israel med opspilet Gab. Men trods alt har hans Vrede ej lagt sig, hans Hånd er fremdeles rakt ud.
\par 13 Men til ham, der slår det, vender Folket ej om, de, søger ej Hærskarers HERRE.
\par 14 Da hugger HERREN Hoved og Hale af Israel, Palme og Siv på en eneste Dag.
\par 15 Den ældste og agtede er Hoved, Løgnprofeten er Hale.
\par 16 De ledende i dette Folk leder vild, og de, der ledes, opsluges.
\par 17 Derfor glædes ej Herren ved dets unge Mænd, har ej Medynk med dets faderløse og Enker. Thi alle er Niddinger og Ugerningsmænd, og hver en Mund taler Dårskab. Men trods alt har hans Vrede ej lagt sig, hans Hånd er fremdeles rakt ud.
\par 18 Thi Gudløshed brænder som Ild, fortærer Torn og Tidsel, sætter Ild på det tætte Krat, så det hvirvler op i Røg.
\par 19 Ved Hærskarers HERREs Vrede står Landet i Brand, og Folket bliver som Føde for Ilden; de skåner ikke hverandre.
\par 20 Man snapper til højre og hungrer, æder om sig til venstre og mættes dog ej. Hver æder sin Næstes Kød,
\par 21 Manasse Efraim, Efraim Manasse, og de overfalder Juda sammen.

\chapter{10}

\par 1 Ve dem, der giver Ulykkeslove og ivrigt fører Uret til Bogs
\par 2 for at trænge de ringe fra Retten og røve de armes Ret i mit Folk, at Enker kan blive deres Bytte og faderløse kan plyndres.
\par 3 Hvad gør I på Straffens Dag, når Undergang kommer fra det fjerne? Til hvem vil I ty om Hjælp, hvor gemmer I da eders Rigdom?
\par 4 Enten må I knæle blandt Fanger, eller også falde på Valen! Men trods alt har hans Vrede ej lagt sig, hans Hånd er fremdeles rakt ud.
\par 5 Ve Assur, min Harmes Kæp, min Vrede er Stokken i hans Hånd.
\par 6 Jeg sender ham mod et vanhelligt Folk, opbyder ham mod min Vredes, Folk til at gøre Bytte og røve og trampe det ned som Skarn på Gaden.
\par 7 Men han, han mener det ej så, hans Hjerte tænker ej så. Nej, at ødelægge, det er hans Attrå, at udrydde Folk, ikke få.
\par 8 Thi han siger: "Er ej mine Høvedsmænd Konger til Hobe?
\par 9 Gik det ej Kalno som Karkemisj, mon ikke Hamat som Arpad, Samaria som Damaskus?
\par 10 Som min Hånd fandt hen til Afgudernes Riger, der dog havde flere Gudebilleder end Jerusalem og Samaria
\par 11 mon jeg da ikke skal handle med Jerusalem og dets Gudebilleder, som jeg handlede med Samaria og dets Afguder?"
\par 12 Men når Herren fuldbyrder alt sit Værk på Zions Bjerg og i Jerusalem, vil jeg hjemsøge Assyrerkongens Hjertes Hovmodsfrugt og hans Øjnes trodsige Pral,
\par 13 fordi han siger: "Med min, stærke Hånd greb jeg ind, med min Visdom, thi jeg er klog. Jeg flyttede Folkeslags Grænser og rev deres Skatte til mig, stødte Folk fra Tronen i Almagt;
\par 14 min Hånd fandt til Folkenes Rigdom som hen til en Fuglerede; som man sanker forladte Æg, har jeg sanket den vide Jord, og ingen rørte en Vinge, åbnede Næbbet og peb."
\par 15 Mon Øksen bryster sig mod den, som hugger, gør Saven sig til mod den, som saver? Som om Kæppen kan svinge den, der løfter den, Stokken løfte, hvad ikke er Træ!
\par 16 Derfor sender Herren Hærskarers HERRE, Svindsot i hans Fedme, og under hans Herlighed luer en Lue som luende Ild;
\par 17 Israels Lys bliver til Ild og dets Hellige til en Flamme, og den brænder og fortærer hans Tidsel og Torn på een Dag;
\par 18 hans Skovs og Frugthaves Herlighed skal den rydde Rub og Stub, og han bliver som en syg, der hentæres.
\par 19 De Træer, som levnes i hans Skov, bliver det let at tælle; et Barn kan skrive dem op.
\par 20 På hin Dag skal Israels Rest og det, som undslipper af Jakobs Hus, ikke mere støtte sig til den, der slår det, men til HERREN, Israels Hellige, i Sandhed.
\par 21 En Rest skal omvende sig, Jakobs Rest, til den vældige Gud.
\par 22 Thi var end dit Folk som Sandet ved Havet, Israel, kun en Rest deraf skal omvende sig. Ødelæggelse er fastslået, og med Retfærdighed vælter den frem;
\par 23 thi Ødelæggelse og fastslået Råd fuldbyrder Herren, Hærskarers HERRE, over al Jorden.
\par 24 Derfor, så siger Herren, Hærskarers HERRE: Frygt ikke, mit Folk, som bor i Zion, for Assyrien, når det slår dig med Kæppen og løfter sin Stok imod dig som fordum Ægypten!
\par 25 Thi end om en føje Stund er Vreden ovre, og min Harme vender sig til deres Fordærv.
\par 26 Så svinger Hærskarers HERRE Svøben imod det, som da Midjan blev slået ved Orebs Klippe; hans Stok er udrakt imod Havet, og han løfter den som fordum mod Ægypten.
\par 27 På hin Dag tager han Byrden af din Skulder og Åget af din Nakke, ja, Åget brister for Fedme.
\par 28 Han rykker mod Ajjat, drager uden om Migron, i Mikmas lader han Trosset blive;
\par 29 de går over Passet: "I Geba holder vi Natterast!" Rama ryster, Sauls Gibea flyr.
\par 30 Skrig højt, Gallims Datter! Lyt til, Lajsja! Stem i, Anatot!
\par 31 Madmena flyr, Gebims Folk bjærger deres Gods.
\par 32 Endnu i Dag står han i Nob; han svinger Hånden mod Zions Datters Bjerg, Jerusalems Høj -
\par 33 Se, Herren, Hærskarers HERRE, afhugger hans Grene med Gru; de knejsende Stammer fældes, de stolte Træer må segne.
\par 34 Med Jernet gør han lyst i Skovens Tykning, og Libanon falder for den Herlige.

\chapter{11}

\par 1 Men der skyder en Kvist af Isajs Stub, et Skud gror frem af hans Rod;
\par 2 og HERRENs Ånd skal hvile over ham, Visdoms og Forstands Ånd, Råds og Styrkes Ånd, HERRENs Kundskabs og Frygts Ånd.
\par 3 Hans Hu står til HERRENs Frygt; han dømmer ej efter, hvad Øjnene ser, skønner ej efter, hvad Ørene hører.
\par 4 Han dømmer de ringe med Retfærd, fælder redelig Dom over Landets arme. Voldsmanden slår han med Mundens Ris, gudløse dræber han med Læbernes Ånde.
\par 5 Og Retfærd er Bæltet, han har om sin Lænd, Trofasthed Hofternes Bælte.
\par 6 Og Ulven skal gå hos Lammet, Panteren hvile hos Kiddet, Kalven og Ungløven græsse sammen, dem driver en lille Dreng.
\par 7 Kvien og Bjørnen bliver Venner, deres Unger ligger Side om Side, og Løven æder Strå som Oksen;
\par 8 den spæde skal lege ved Øglens Hul, den afvante række sin Hånd til Giftslangens Rede.
\par 9 Der gøres ej ondt og voldes ej Men i hele mit hellige Bjergland; thi Landet er fuldt af HERRENs Kundskab, som Vandene dækker Havets Bund.
\par 10 På hin Dag skal Hedningerne søge til Isajs Rodskud, der står som et Banner for Folkeslagene, og hans Bolig skal være herlig.
\par 11 På hin Dag skal Herren atter udrække sin Hånd for at vinde, hvad der er til Rest af hans Folk, fra Assur og fra Ægypten, fra Patros, Ætiopien og Elam, fra Sinear, Hamat og Havets Strande.
\par 12 For Folkene rejser han Banner, samler Israels bortdrevne Mænd, sanker Judas spredte Kvinder fra Verdens fire Hjørner.
\par 13 Efraims Skinsyge viger, og Judas Avind svinder; Efraim er ikke skinsygt på Juda, og Juda bærer ej Avind mod Efraim.
\par 14 I Vest slår de ned på Filisternes Skulder, sammen plyndrer de Østens Sønner, mod Edom og Moab rækker de Hånden, Ammons Sønner lyder dem.
\par 15 HERREN udtørrer Ægypterhavets Vig og svinger Hånden mod Floden i sin Ånds Vælde; han kløver den i syv Bække, så man kan gå over med Sko;
\par 16 der bliver en banet Vej for dem af hans Folk, som levnes fra Assyrien, således som der var for Israel, da det drog op fra Ægypten.

\chapter{12}

\par 1 På hin Dag skal du sige: Jeg takker dig, HERRE, thi du vrededes på mig; men din Vrede svandt, og du trøstede mig.
\par 2 Se, Gud er min Frelse, jeg er trøstig og uden Frygt; thi HERREN er min Styrke og min Lovsang, og han blev mig til Frelse.
\par 3 I skal øse Vand med Glæde af Frelsens Kilder
\par 4 og sige på hin Dag: Tak HERREN, påkald hans Navn, gør hans Gerninger kendt blandt Folkene, kundgør, at hans Navn er højt!
\par 5 Lovsyng HERREN, thi stort har han øvet,lad det blive kendt på den vide Jord!
\par 6 Bryd ud i Fryderåb, Zions Beboere, thi stor i eders Midte er Israels Hellige!

\chapter{13}

\par 1 Et Udsagn om Babel, som Esajas, Amoz's Søn skuede:
\par 2 Rejs Banner på et nøgent Bjerg, råb også til dem, vink, så de drager igennem Fyrsternes Porte!
\par 3 Jeg har opbudt min viede Hær til at tjene min Vrede og kaldt mine Helte hid, de jublende, stolte.
\par 4 Hør i Bjergene Larm som af talrigt Krigsfolk, hør, hvor det buldrer af Riger, af samlede Folk! Hærskarers HERRE er ved at mønstre sin Krigshær.
\par 5 De kommer fra fjerne Egne, fra Himlens Grænse, HERREN og hans Vredes Værktøj for at hærge al Jorden.
\par 6 Jamrer, thi HERRENs Dag er nær, den kommer som Vold fra den Vældige.
\par 7 Derfor slappes hver Hånd, hvert Menneskehjerte smelter,
\par 8 de ræddes, gribes af Veer og Smerter, vånder sig som fødende Kvinde; de stirrer i Angst på hverandre med blussende røde Kinder.
\par 9 Se, HERRENs Dag kommer, grum, med Harme og brændende Vrede; Jorden gør den til Ørk og rydder dens Synder bort.
\par 10 Thi Himlens Stjerner og Billeder udsfråler ej deres Lys, mørk rinder Solen op, og Månen skinner ikke.
\par 11 Jeg hjemsøger Jorden for dens Ondskab, de gudløse for deres Brøde, gør Ende på frækkes Overmod, bøjer Voldsmænds Hovmod.
\par 12 En Mand gør jeg sjældnere end Guld og et Menneske end Ofirs Guld.
\par 13 Derfor bæver Himlen, og Jorden flytter sig skælvende ved Hærskarers HERREs Harme på hans brændende Vredes Dag.
\par 14 Og som en skræmt Gazel, som Får, der ej holdes i Flok, skal hver søge hjem til sit Folk, og hver skal fly til sit Land.
\par 15 Enhver, der indhentes, spiddes, enhver, der gribes, falder for Sværdet;
\par 16 deres spæde knuses for deres Øjne, Husene plyndres, Kvinderne skændes.
\par 17 Se, imod dem rejser jeg Mederne, som agter Sølv for intet og ej regner Guld for noget.
\par 18 Deres Buer fælder de unge, Livsfrugt skåner de ej, med Børn har deres Øjne ej Medynk.
\par 19 Det går med Babel, Rigernes Krone, Kaldæernes stolte Pryd, som dengang Gud omstyrtede Sodoma og Gomorra.
\par 20 Det skal aldrig i Evighed bebos, ej bebygges fra Slægt til Slægt; der telter Araberen ikke, der lejrer Hyrder sig ej;
\par 21 men Vildkatte lejrer sig der, og Husene fyldes med Ugler; der holder Strudsene til, og Bukketroldene springer;
\par 22 Sjakaler tuder i Borgene, Hyæner i de yppige Slotte. Dets Time stunder nu til, dets Dage bliver ej mange.

\chapter{14}

\par 1 Thi HERREN forbarmer sig over Jakob og udvælger atter Israel.
\par 2 Folkeslag skal tage dem og bringe dem hjem igen; og Israels Hus skal i HERRENs Land tage Folkeslagene i Eje som Trælle og Trælkvinder; de skal gøre dem til Fanger, hvis Fanger de var, og herske over deres Bødler.
\par 3 På den Dag HERREN giver dig Hvile for din Møje og Uro og for den hårde Trældom, der lagdes på dig,
\par 4 skal du istemme denne Spottevise om Babels Konge: Hvor er dog Bødlen stille, Tvangshuset tyst!
\par 5 HERREN har brudt de gudløses Stok, Herskernes Kæp,
\par 6 som slog i Vrede Folkeslag, Slag i Slag, og tvang i Harme Folk med skånselsløs Tvang.
\par 7 Al Jorden har Fred og Ro, bryder ud i Jubel;
\par 8 selv Cypresserne glæder sig over dig, Libanons Cedre: "Siden dit Fald kommer ingen op for at fælde os!"
\par 9 Dødsriget nedentil stormer dig heftigt i Møde, vækker for din Skyld Dødninger, al Jordens store, jager alle Folkenes Konger op fra Tronen;
\par 10 de tager alle til Orde og siger til dig: "Også du blev kraftløs som vi, du blev vor Lige!"
\par 11 Til Dødsriget sendtes din Højhed, dine Harpers Brus, dit Leje er redt med Råddenskab, dit Tæppe er Orme.
\par 12 Nej, at du faldt fra Himlen; du strålende Morgenstjerne, fældet og kastet til Jorden, du Folkebetvinger!
\par 13 Du, som sagde. i Hjertet: "Jeg stormer Himlen, rejser min, Trone deroppe over Guds Stjerner, tager Sæde på Stævnets Bjerg i yderste Nord,
\par 14 stiger op over Skyernes Højder, den Højeste lig" -
\par 15 ja, ned i Dødsriget styrtes du, nederst i Hulen!
\par 16 Ser man dig, stirrer man på dig med undrende Blikke: "Er det ham, som fik Jorden til at bæve, Riger til at skælve,
\par 17 ham, som gjorde Verden til Ørk og jævnede Byer, ikke gav Fangerne fri til at drage mod Hjemmet?"
\par 18 Folkenes Konger hviler med Ære hver i sit Hus,
\par 19 men du er slængt hen uden Grav som et usseligt Foster, dækket af faldne, slagne med Sværd og kastet i Stenbruddets Hul som et nedtrådt Ådsel.
\par 20 I Graven samles du ikke med dine Fædre, fordi du ødte dit Land og dræbte dit Folk. Ugerningsmændenes Afkom skal aldrig nævnes.
\par 21 Bered hans Sønner et Blodbad for Faderens Brøde! De skal ikke stå op og indtage Jorden og fylde Verden med Sfæder.
\par 22 Jeg står op imod dem, lyder det fra Hærskarers HERRE, og udrydder af Babel Navn og Rest, Skud og Spire, lyder det fra HERREN;
\par 23 jeg gør det til Rørdrummers Eje og til side Sumpe; jeg fejer det bort med Undergangens Kost, lyder det fra Hærskarers HERRE.
\par 24 Hærskarers HERRE har svoret således: Visselig, som jeg har tænkt det, så skal det ske, og som jeg satte mig for, så står det fast:
\par 25 Jeg knuser Assur i mit Land, ned tramper ham på mine Bjerge, hans Åg skal vige fra dem, hans Byrde skal vige fra dets Skuldre.
\par 26 Det er, hvad jeg satte mig for imod al Jorden, det er den Hånd, som er udrakt mod alle Folk.
\par 27 Thi Hærskarers HERREs Råd, hvo kuldkaster det? Hans udrakte Hånd, hvo tvinger vel den tilbage?
\par 28 I Kong Akaz's Dødsår kom dette Udsagn:
\par 29 Glæd dig ej, hele Filisterland, at kæppen, der slog dig, er brudt! Thi af Slangerod kommer en Øgle, dens frugt er en flyvende Drage.
\par 30 På min Vang skal de ringe græsse de fattige lejre sig trygt; men jeg dræber dit Afkom ved Sult; hvad der levnes, slår jeg ihjel.
\par 31 Hyl, Port, skrig, By, Angst gribe dig, hele Filisterland! Thi nordenfra kommer Røg, i Fjendeskaren nøler ingen.
\par 32 Og hvad skal der svares Folkets Sendebud? At HERREN har grundfæstet Zion, og de arme i hans Folk søger Tilflugt der.

\chapter{15}

\par 1 Et udsagn om Moab. Ak, Ar lægges øde ved Nat, det er ude med Moab, Kir lægges øde ved Nat, det er ude med Moab.
\par 2 Dibons Datter går op på Høje for at græde, oppe på Nebo og Medeba jamrer Moab; hvert et Hoved er skaldet, alt skæg skåret af,
\par 3 på Gader og oppe på Tage bærer de Sæk, på Torvene jamrer de alle, opløst i Gråd.
\par 4 Hesjbon og El'ale skriger, det høres til Jahaz. Derfor skælver Moabs Lænder, dets Sjæl er i Vånde.
\par 5 Hjertet skriger i Moab, man flygter til Zoar, til Eglat-Sjelisjija. Ak, grædende stiger de op ad Luthiths Skråning, undervejs til Horonajim opløfter de Jammerskrig;
\par 6 Nimrims Vande bliver Ødemarker, thi Græsset visner, Grønsværet svinder, Grønt er der ikke.
\par 7 Derfor slæber de Godset, de vandt, deres hengemte Ting over Vidjebækken.
\par 8 Ak, Nødråbet omspænder Moabs Land, dets Jamren når til Eglajim og Be'er-Elim;
\par 9 thi Dimons Vand er fuldt af Blod. Men jeg sender endnu mer over Dimon: en Løve over Moabs undslupne, Landets Rest.

\chapter{16}

\par 1 Landets Herrer sender en Gave fra Sela gennem Ørknen til Zions Datters Bjerg.
\par 2 Og ret som flagrende Fugle, som en opskræmt Rede er Moabs Døtre ved Arnons Vadesteder.
\par 3 "Kom med et Råd, gør Ende derpå, lad din Skygge blive som Natten ved højlys Dag, skjul de bortdrevne, røb ej de flyende!
\par 4 Giv Moabs bortdrevne Tilhold hos dig, vær dem et Skjul for den, som hærger! Er først Voldsmanden borte, Ødelæggelsen omme, Undertrykkeren ude af Landet.
\par 5 skal en Trone rejses med Mildhed, og på den skal sidde en Dommer med Trofasthed i Davids Telt, ivrig for Ret og øvet i Retfærd."
\par 6 "Vi har hørt om Moabs Hovmod, det såre store, dets Overmod, Hovmod og Frækhed, dets tomme Snak."
\par 7 Derfor jamrer Moab over Moab, alle jamrer; Kir-Haresets Rosinkager sukker de sønderknust over.
\par 8 Thi visne er Hesjbons Marker, Sibmas Vinstok, hvis Druer slog Folkenes Herrer til Jorden; den nåede Ja'zer, famled gennem Ørkenen, dens Ranker bredte sig, overskred Havet.
\par 9 Derfor græder jeg Ja'zers Gråd over Sibmas Vinstok, væder med min Tåre Hesjbon, og El'ale; thi et Vinperserråb slog ned på,din Frugt og din Høst,
\par 10 fra Frugthaver svandt både Glæde og Jubel; i Vingårde jubles der ikke, der lyder ej Råb, i Karrene trampes ej Vin, Vinperserråbet er tystnet.
\par 11 Derfor bæver mit Indre som Citren for Moab, mit Hjerte for Kir-Heres.
\par 12 Og når Moab viser sig på Offerhøjen, når det gør sig Møje og kommer til sin Helligdom for at bede, udretter det intet.
\par 13 Det er Ordet, HERREN fordum talede til Moab.
\par 14 Men nu siger HERREN: Om tre År, som Daglejeren regner Året, skal Moabs Herlighed vanæres med al den store larmende Hob.

\chapter{17}

\par 1 Et Udsagn om Damaskus. Se, Damaskus går ud af Byernes Tal og bliver til Sten og Grus;
\par 2 dets Stæder forlades for evigt og bliver Hjordes Eje; de lejrer sig uden at skræmmes.
\par 3 Det er ude med Efraims Værn, Damaskus's Kongedømme, Arams Rest; det går dem som Israels Sønners Herlighed, lyder det fra Hærskarers HERRE.
\par 4 Og det skal ske på hin Dag: Ringe bliver Jakobs Herlighed, Huldet på hans Krop svinder hen;
\par 5 det skal gå som når Høstkarlen griber om Korn og hans Arm skærer Aksene af, det skal gå, som når Aksene samles i Refaims Dal.
\par 6 En Efterslæt levnes deraf, som når Olietræets Frugt slås ned, to tre Bær øverst i Kronen, fire fem på Frugttræets Grene, så lyder det fra HERREN, Israels Gud.
\par 7 På hin Dag skal Menneskene se hen til deres Skaber, og deres Øjne skal skue hen til Israels Hellige;
\par 8 og de skal ikke se hen til Altrene, deres Hænders Værk, eller skue hen til, hvad deres Fingre har lavet, hverken til Asjerastøtterne eller Solstøtterne.
\par 9 På hin Dag ligger dine Byer forladt som de Tomter, Hivviter og Amoriter forlod for Israels Børn; og Landet skal blive en Ørk.
\par 10 Thi du glemte din Frelses Gud, slog din Tilflugtsklippe af Tanke. Derfor planter du yndige Plantninger og sætter fremmede Skud;
\par 11 hver Dag får du din Plantning i Vækst, hver Morgen dit Skud i Blomst - indtil Høsten på Sotens, den ulægelige Smertes Dag.
\par 12 Hør Bulder af mange Folkeslag! De buldrer som Havets Bulder.
\par 13 Folkefærdene drøner som Drønet af mange Vande. Men truer han ad dem, flygter de bort, vejres hen som Avner på Bjerge for Vinden, som hvirvlende Løv for Stormen.
\par 14 Ved Aftenstid kommer Rædsel; før Morgen gryr, er de borte. Det er vore Plyndreres Del, det er vore Ransmænds Lov

\chapter{18}

\par 1 Hør Land med de surrende Vinger hinsides Ætiopiens Strømme,
\par 2 du, som sender Bud over Havet i Både af Siv på Vandspejlet: Gå, I hastige Bud, til det ranke, glinsende Folk, til Folket, som frygtes så vide, Kraftens og Sejrens Folk, hvis Land gennemstrømmes af Floder.
\par 3 Alle I Jorderigs Folk, som bygger på Jord: Rejses Banner på Bjerge, så se, når der stødes i Horn, så hør!
\par 4 Thi så sagde HERREN til mig: "Rolig ser jeg til fra mit Sæde som glødende Luft i Solskin, som Dugsky i Høstens Tid."
\par 5 Thi før Høst, når Blomstring er endt, når Blomst sætter modnende Drue, afskærer han Rankerne med Knive, og Skuddene kapper han bort.
\par 6 De gives alle til Bjergenes Fugle og Jordens Dyr, om Somren Føde for Fugle, om Vintren for al Jordens Dyr.
\par 7 Til hin Tid skal der bringes Hærskarers HERRE Gave fra et rankt og glinsende Folk, et Folk, som frygtes så vide, Kraftens og Sejrens Folk, hvis Land gennemstrømmes af Floder, til Stedet, hvor Hærskarers HERREs Navn bor, til Zions Bjerg.

\chapter{19}

\par 1 Et Udsagn om Ægypten. Se, HERREN farer på letten Sky og kommer til Ægypten; Ægyptens Guder bæver for ham, Ægyptens Hjerte smelter i Brystet.
\par 2 Jeg hidser Ægypten mod Ægypten, så de kæmper Broder mod Broder, Ven mod Ven, By mod By, Rige mod Rige.
\par 3 Ægyptens Forstand står stille, dets Råd gør jeg til intet, så de søger Guder og Manere, Genfærd og Ånder.
\par 4 Jeg giver Ægypten hen i en hårdhjertet Herres Hånd, en Voldskonge bliver deres Hersker, så lyder det fra Herren, Hærskarers HERRE.
\par 5 Vandet i Floden svinder, Strømmen bliver sid og tør;
\par 6 Strømmene udspreder Stank, Ægyptens Floder svinder og tørres; Rør og Siv visner hen,
\par 7 alt Græsset ved Nilbredden dør, al Sæd ved Nilen hentørres, svinder og er ikke mere.
\par 8 Fiskerne sukker og sørger, alle, som meder i Nilen; de, som, sætter Garn i Vandet, gribes af Modløshed.
\par 9 Til Skamme er de, som væver Linned, Heglersker og de, som væver Byssus;
\par 10 Spinderne er sønderknust, hver Daglejer sørger bittert.
\par 11 Kun Dårer er Zoans Øverster, Faraos viseste Rådmænd så dumt et Råd. Hvor kan I sige til Farao: "Jeg er en Ætling af Vismænd, Ætling af Fortidens Konger?"
\par 12 Ja, hvor er nu dine Vismænd? Lad dem dog kundgøre dig og lade dig vide, hvad Hærskarers HERRE har for mod Ægypten!
\par 13 Hans Fyrster blev Dårer: Fyrster i Nof blev Tåber. Ægypten er bragt til at rave af Stammernes Hjørnesten.
\par 14 I dets Indre har HERREN udgydt Svimmelheds Ånd; Ægypten fik de til at rave i al dets Id, som den drukne raver i sit Spy.
\par 15 For Ægypten lykkes intet, hverken for Hoved eller Hale, Palme eller Siv.
\par 16 På hin Dag skal Ægypten blive som Kvinder; det skal ængstes og grue for Hærskarers HERREs svungne Hånd, som han svinger imod det.
\par 17 Judas Land bliver Ægypten en Rædsel; hver Gang nogen minder dem derom, gribes de af Angst for, hvad Hærskarers HERRE har for imod det.
\par 18 På hin Dag skal fem Byer i Ægypten tale Kana'ans Tungemål og sværge ved Hærskarers HERRE; en af dem skal kaldes Ir-Haheres.
\par 19 På hin Dag skal HERREN have et Alter midt i Ægypten og en Stenstøtte ved dets Grænse.
\par 20 Det skal være Tegn og Vidne for Hærskarers HERRE i Ægypten; når de råber til HERREN over dem, som mishandler dem, vil han sende dem en Frelser; han skal stride og udfri dem.
\par 21 Da skal HERREN give sig til Kende for Ægypten, Ægypterne skal lære HERREN at kende på hin Dag; de skal bringe Slagtoffer og Afgrødeoffer og gøre Løfter til HERREN og indfri dem.
\par 22 HERREN skal slå Ægypten, slå og læge; og når de omvender sig til HERREN, bønhører han dem og læger dem.
\par 23 På hin Dag skal der gå en banet Vej fra Ægypten til Assyrien, og Assyrien skal komme til Ægypten og Ægypten til Assyrien, og Ægypten skal tjene Herren sammen med Assyrien.
\par 24 På hin Dag skal Israel selvtredje, sammen med Ægypten og Assyrien, være en Velsignelse midt på Jorden,
\par 25 som Hærskarers HERRE velsigner med de Ord: "Velsignet være Ægypten, mit Folk, og Assyrien, mine Hænders Værk, og Israel, min Arvelod!"

\chapter{20}

\par 1 I det År Tartan kom til Asdod, dengang Assyrerkongen Sargon sendte ham og han angreb Asdod og indtog det,
\par 2 på den Tid talede HERREN ved Esajas, Amoz's Søn, således: "Gå hen og løs Sørgeklædet af dine Lænder og drag Skoene af dine Fødder!" Og han gjorde således og gik nøgen og barfodet.
\par 3 Så sagde HERREN: "Som min Tjener Esajas i trende År har vandret nøgen og barfodet som Tegn og Varsel mod Ægypten og Ætiopien,
\par 4 således skal Assyrerkongen slæbe fangne Ægyptere og bortførte Ætiopere med sig, unge og gamle, nøgne og barfodede, med blottet Bag til Skændsel før Ægypten."
\par 5 Da skal de forfærdes og blues over Ætiopien, som de så hen til, og over Ægypten, som var deres Stolthed
\par 6 Og de, som bor på denne Strand, skal på hin Dag sige: "Se, således gik det med den, vi så hen til, til hvem vi tyede om Hjælp for at frelses fra Assyrerkongen; hvor skal da vi kunne undslippe!"

\chapter{21}

\par 1 Et Udsagn om Havørknen. Som hvirvlende Storme, der jager i Sydlandet, kommer det fra Ørkenen, det grufulde Land
\par 2 Så svart et Syn blev mig meldt: "Ransmænd raner, Hærmænd hærger! Frem, Elamiter! Til Belejring, Meder! Alle Suk gør jeg Ende på!"
\par 3 Derfor fyldes mine Lænder af Skælven, jeg gribes af Veer som, fødende Kvinde, døv af Svimmelhed, blind af Skræk,
\par 4 mit Hjerte forvirres, Gru falder på mig; Skumringen, jeg elsker, bliver mig til Angst.
\par 5 Bordet dækkes, Hynder bredes, man spiser og drikker, "Op I Fyrster, salv eders Skjolde!"
\par 6 Thi så sagde Herren til mig: "Gå hen og stil Vægteren ud! Hvad han får at se, skal han melde.
\par 7 Og ser han Ryttere, et Par komme ridende, en Rytter på Æsel, en Rytter på Kamel, da skal han lytte, ja lytte spændt!"
\par 8 Og han råbte: "Se, o Herre, på Varden står jeg bestandig, Dagen lang, og på min Vagtpost står jeg trolig
\par 9 Nat efter Nat!" Men se, da kom der ridende Mænd, et Par kom ridende; de råbte: "Faldet, faldet er Babel, han knuste alle dets Guder i Støvet!"
\par 10 Mit knuste, mit tærskede Folk! Hvad jeg, har hørt fra Hærskarers HERRE, fra Israels Gud, det melder jeg eder.
\par 11 Et Udsagn om Duma. Der råbes til mig fra Se'ir: "Vægter, hvordan skrider Natten, Vægter, hvordan skrider Natten?"
\par 12 Vægteren svarer: "Morgen kommer, men også Nat! Vil I spørge, så spørg! Kom kun igen!"
\par 13 Et Udsagn: "I Ødemarken". Søg Nattely i Ødemarkens Krat, I Dedans Karavaner!
\par 14 Bring de tørstige Vand i Møde, I, som bor i Temas Land, mød de flyende med Brød!
\par 15 Thi de er på Flugt for Sværd, på Flugt for det dragne Sværd, på Flugt for den spændte Bue, på Flugt for Krigens Tynge.
\par 16 Thi så sagde Herren til mig:"Et År endnu, som Daglejeren regner Året, og det er ude med al Kedars Herlighed.
\par 17 Resten af Kedars Heltes Buer skal være ringe, så sandt HERREN, Israels Gud, har talet."

\chapter{22}

\par 1 Et Udsagn: "Synernes Dal". Hvad tænker du på, siden alle stiger op på Tagene,
\par 2 du larmende, støjende By, du jublende Stad? Dine slagne er vel ikke sværdslagne, døde i Krig!
\par 3 Alle dine Høvdinger flygted, flyed langt bort, alle dine Helte, væbnet med Buer, blev fanget.
\par 4 Derfor siger jeg: Gå fra mig, lad mig græde bittert, træng ej på for at trøste mig over, at mit Folk er lagt øde!
\par 5 Thi en Dag, da man, ræddes, trædes og trænges, har Herren, Hærskarers HERRE, til Rede! I Synernes Dal brødes Mure ned, mod Bjerget hørtes Skrig;
\par 6 Elam løftede Koggeret, Aram satte sig til Hest, Kir tog Skjoldene ud;
\par 7 og de bedste iblandt dine Dale fyldtes med Vogne og Heste, lige til Porten stod de.
\par 8 Han borttog Judas Værn. På den Dag så I hen til Skovhusets Rustkammer,
\par 9 og I så, hvor mange Revner der var i Davidsbyen. I samlede Nedredammens Vand,
\par 10 gik Jerusalems Huse igennem og rev Husene ned for at gøre Muren stærk.
\par 11 I gravede mellem de to Mure en Fordybning til den gamle Dams Vand. Men til ham, der virked det, skued I ikke, så ej hen til ham, som beredte det for længst.
\par 12 På hin Dag kaldte Herren, Hærskarers HERRE, til Gråd og Sorg, til Hovedragning og Sæk.
\par 13 Men se, der er Fryd og Glæde, man slår Okser ned, slagter Får, æder Kød og får Vin at drikke: "Lad os æde og drikke, thi i Morgen dør vi!"
\par 14 Men Hærskarers HERRE åbenbared for mit Øre: "Den Synd," siger Herren, Hærskarers HERRE, "får I ikke sonet, førend I dør!"
\par 15 Så siger Herren, Hærskarers HERRE: Gå hen og sig til denne Foged, Slotshøvedsmanden Sjebna:
\par 16 Hvad har du her, og hvem har du her, at du her udhugger din Grav, udhugger dig en Grav højt oppe, huler dig en Bolig i Klippen!
\par 17 Se, HERREN slynger dig bort og bøjer dig sammen, du stolte,
\par 18 han knytter dig sammen til et Knytte og kaster dig ud i et vidtstrakt Land! Der skal du dø, der din Æresvogn komme, du Skændsel for din Herres Hus!
\par 19 Jeg støder dig bort fra din Stilling og styrter dig fra din Post.
\par 20 Men på hin Dag kalder jeg min Tjener Eljakim, Hilkijas Søn,
\par 21 og iklæder ham din Kjortel, omgjorder ham med dit Bælte og lægger din Myndighed i hans Hånd. Han skal blive en Fader for Jerusalems Indbyggere og Judas Hus.
\par 22 Jeg lægger Nøglen til Davids Hus på hans Skulder; når han lukker op, skal ingen lukke i, og når han lukker i, skal ingen lukke op,
\par 23 Jeg fæster ham som en Nagle på et sikkert Sted, og han skal blive til Hæder for sit Fædrenehus.
\par 24 Men hænger hans Fædrenehus's hele Vægt sig på ham, Skud og Vildskud, alle Småkar, fra Fadene til alle Krukkerne,
\par 25 så skal det ske på den Dag, lyder det fra Hærskarers HERRE, at Naglen, der var fæstet på et sikkert Sted, giver efter, rives ud og falder ned, og hele Vægten, som hænger derpå, skal slås sønder. Thi HERREN har talet!

\chapter{23}

\par 1 Et Udsagn om Tyrus. Jamrer, I Tarsisskibe, eders Fæstning er i Grus! De får det at vide på Vejen fra Kyperns Land.
\par 2 Det er ude med Kystlandets Folk, med Zidons Købmænd, hvis Sendebud for over Havet,
\par 3 de mange Vande, hvis Indkomst var Sjihors Sæd, hvis Vinding Alverdens Varer.
\par 4 Skam få du, Zidon, thi Havet siger: "Jeg har ikke haft Veer, jeg fødte ikke, ej har jeg fostret Ynglinge, opfødt Jomfruer!"
\par 5 Når Rygtet når Ægypten, skælver de ved Rygtet om Tyrus.
\par 6 Drag over til Tarsis og jamrer, I Kystlandets Folk!
\par 7 Er det eders jublende By fra Urtids Dage, hvis Fødder førte den viden om som Gæst?
\par 8 Hvo satte sig dette for mod det kronede Tyrus, hvis Købmænd var Fyrster, hvis Kræmmere Jordens Adel?
\par 9 Det gjorde Hærskarers HERRE for at vanære Hovmod, skænde al Stolthed, al Jordens Adel.
\par 10 Græd, I Tarsisskibe, Havn er der ikke mer!
\par 11 Han udrakte Hånden mod Havet, rystede Riger, HERREN samled Folk for at jævne Kana'ans Fæstninger.
\par 12 Han sagde: "Aldrig mer skal du juble, du voldtagne Jomfru, Zidons Datter! Stå op, drag over til Kypern, selv der skal du ej finde Hvile!"
\par 13 Se til Kyprioternes Land! Søfarere grunded det Folk; de rejste dets Vagttårne, Byer og Borge. Han gjorde det til en Ruinhob.
\par 14 Jamrer, I Tarsisskibe, eders Fæstning er i Grus!
\par 15 På hin Dag skal Tyrus gå ad Glemme i halvfjerdsindstyve År, som i een Konges Dage. Men efter halvfjerdsindstyve Års Forløb skal det gå med Tyrus som med Skøgen i Visen:
\par 16 Tag din Citer, gå rundt i Byen, du glemte Skøge, leg smukt på Strenge, syng, hvad du kan, så du kommes i Hu!"
\par 17 Efter halvfjerdsindstyve Års Forløb vil HERREN se til Tyrus; det skal atter modtage Skøgeløn og bole med Alverdens Riger på den vide Jord.
\par 18 Men dets Vinding og Skøgeløn skal helliges HERREN; den skal ikke gemmes hen eller lægges op; dem, der bor for HERRENs Åsyn, skal dets Vinding tjene til Føde, Mættelse og prægtige Klæder.

\chapter{24}

\par 1 Se, HERREN gør Jorden tom og øde og vender op og ned på dens Overflade, han spreder dens Beboere;
\par 2 det går Lægfolk som Præst, Træl som Herre, Trælkvinde som Frue, Køber som Sælger, Långiver som Låntager, Ågerkarl som Skyldner.
\par 3 Jorden tømmes og plyndres i Bund og Grund, thi HERREN har talet dette Ord.
\par 4 Jorden blegner og segner, Jorderig sygner og segner, Jordens Højder sygner hen.
\par 5 Vanhellig blev Jorden under dem, som bor der, thi Lovene krænked de, overtrådte Budet, brød den evige Pagt.
\par 6 Derfor fortærer Forbandelse Jorden, og bøde må de, som bor der.
\par 7 Druesaften sørger, Vinranken sygner alle de hjertensglade sukker;
\par 8 Håndpaukens Klang er endt, de jublendes Larm hørt op, endt er Citrens Klang.
\par 9 De drikker ej Vin under Sang, besk smager den stærke Drik.
\par 10 Den øde Stad ligger nedbrudt, stængt er hver Boligs Indgang.
\par 11 Man jamrer over Vinen på Gaden, bort er al Glæde svundet; landflygtig er Landets Fryd.
\par 12 I Byen er Øde tilbage, og Porten er hugget i Splinter.
\par 13 Thi på Jorden midt iblandt Folkene går det, som når Olietræets Frugt slås ned, som ved Efterslæt, når Vinen er høstet:
\par 14 Disse opløfter Røsten, jubler over HERRENs Storhed, råber fra Vesten:
\par 15 "Derfor skal I ære HERREN i Østen, på Havets Strande HERRENs, Israels Guds, Navn!"
\par 16 Fra Jordens Grænse hører vi Lovsange: "Hil den retfærdige!" Men jeg siger: Jeg usle, jeg usle, ve mig, Ransmænd raner, Ransmænd raner Ran,
\par 17 Gru og Grav og Garn over dig, som bor på Jorden!
\par 18 Den, der flygter for Gru, skal falde i Grav, og den, der når op af Grav, skal fanges i Garn. Thi Sluserne oventil åbnes, og Jordens Grundvolde vakler.
\par 19 Jorden smuldrer og smuldrer, Jorden gynger og gynger, Jorden skælver og skælver;
\par 20 Jorden raver og raver som drukken og svajer som Vogterens Hytte; tungt ligger dens Brøde på den, den segner og rejser sig ikke.
\par 21 På hin Dag hjemsøger HERREN Himlens Hær i Himlen og Jordens Konger på Jorden.
\par 22 De slæbes i Fængsel som Fanger, holdes under Lås og Lukke og straffes lang Tid efter.
\par 23 Månen blues og Solen skæmmes, thi Hærskarers HERRE viser, han er Konge på Zions Bjerg, i Jerusalem; for hans Ældstes Øjne er Herlighed.

\chapter{25}

\par 1 HERRE, min Gud er du; jeg priser dig, lover dit Navn. Thi du har gjort et Under, Råd fra fordum var tro og sande.
\par 2 Thi du lagde Byen i Grus, den faste Stad i Ruiner; de fremmedes Borg er nedbrudt, aldrig mer skal den bygges.
\par 3 Derfor ærer dig et mægtigt Folk, frygter dig grumme Hedningers Stad.
\par 4 Thi du blev de ringes Værn, den fattiges Værn i Nøden, et Ly mod Skylregn, en Skygge mod Hede; thi som isnende Regn er Voldsmænds Ånde,
\par 5 som Hede i det tørre Land. Du kuer de fremmedes Larm; som Hede ved Skyens Skygge så dæmpes Voldsmænds Sang.
\par 6 Hærskarers HERRE gør på dette Bjerg et Gæstebud for alle Folkeslag med fede Retter og stærk Vin, med fede, marvfulde Retter og stærk og klaret Vin.
\par 7 Og han borttager på dette Bjerg Sløret, som tilslører alle Folkeslag, og Dækket, der dækker alle Folk.
\par 8 Han opsluger Døden for stedse. Og den Herre HERREN aftørrer Tåren af hver en Kind og gør Ende på sit Folks Skam på hele Jorden, så sandt HERREN har talet.
\par 9 På hin Dag skal man sige: Se, her er vor Gud, som vi biede på, og som frelste os; her er HERREN, som vi biede på. Lad os juble og glæde os over hans Frelse;
\par 10 thi HERRENs Hånd hviler over dette Bjerg. Men Moab trampes ned, hvor det står, som Strå i Møddingpølen;
\par 11 det breder sine Hænder ud deri som Svømmeren gør for at svømme, og han ydmyger dets Hovmod trods Hændernes Kunstgreb.
\par 12 Han nedbryder og nedstyrter de stejle Mures Værn; han jævner dem med Jorden, så de ligger i Støvet.

\chapter{26}

\par 1 På hin Dag skal denne Sang synges i Judas Land: "En stærk Stad har vi, til Frelse satte han Mur og Bolværk.
\par 2 Luk Portene op for et retfærdigt Folk, som gemmer på Troskab,
\par 3 hvis Sind er fast, som vogter på Fred, thi det stoler på dig.
\par 4 Stol for evigt på HERREN, thi HERREN er en evig Klippe.
\par 5 Thi han ydmyger dem, der bor i det høje, den knejsende By, styrter den til Jorden, lægger den i Støvet.
\par 6 De armes Fod, de ringes Trin skal træde den ned.
\par 7 Den retfærdiges Sti er jævn, du jævner den retfærdiges Vej.
\par 8 Ja, vi venter dig, HERRE, på dine Dommes Sti; til dit Navn og dit Ry står vor Sjæls Attrå.
\par 9 Min Sjæl attrår dig om Natten, min Ånd i mit indre søger dig.
\par 10 Vises der Nåde mod den gudløse, lærer han aldrig Retfærd; i Rettens Land gør han Uret og ser ikke HERRENs Højhed.
\par 11 HERRE, din Hånd er løftet, men de ser det ikke; lad dem med Skam se din Nidkærhed for Folket, lad dine Fjenders Ild fortære dem!
\par 12 HERRE, du skaffe os Fred, thi alt, hvad vi har udrettet, gjorde du for os.
\par 13 HERRE vor Gud, andre Herrer end du har hersket over os; men dit Navn alene priser vi.
\par 14 Døde bliver ikke levende, Dødninger står ikke op; derfor hjemsøgte og tilintetgjorde du dem og udslettede hvert et Minde om dem.
\par 15 Du har mangfoldiggjort Folket, HERRE, du har mangfoldiggjort Folket, du herliggjorde dig, du udvidede alle Landets Grænser.
\par 16 HERRE, i Nøden søgte de dig; de udgød stille Bønner, medens din Tugtelse var over dem.
\par 17 Som den frugtsommelige; der er ved at føde, vrider og vånder sig i Veer, således fik vi det, HERRE, fra dig.
\par 18 Vi er svangre og vrider os, som om vi fødte Vind; Landet frelser vi ikke og Jordboere fødes ikke til Verden.
\par 19 Dine døde skal blive levende, mine dødes Legemer opstå; de, som hviler i Støvet, skal vågne og juble. Thi en Lysets bug er din Dug, og Jorden giver Dødninger igen.
\par 20 Mit Folk, gå ind i dit Kammer og luk dine Døre bag dig; hold dig skjult en liden Stund, til Vreden er draget over.
\par 21 Thi HERREN går ud fra sin Bolig for at straffe Jordboernes Brøde; sit Blod bringer Jorden for Lyset og dølger ej mer sine dræbte.

\chapter{27}

\par 1 På hin Dag hjemsøger HERREN med sit hårde, vældige, stærke Sværd Livjatan, Den flugtsnare Slange, og ihjelslår Dragen i Havet.
\par 2 På hin Dag skal man sige,: Syng om en liflig Vingård!
\par 3 Jeg, HERREN, jeg er dens Vogter, jeg vander den atter og atter.
\par 4 Vrede nærer jeg ikke. Fandt jeg kun Torn og Tidsel, gik jeg løs derpå i Kamp og satte det alt i Brand
\par 5 med mindre man tyr til mit Værn, slutter Fred med mig, slutter Fred med mig.
\par 6 På hin Dag skal Jakob slå Rod, Israel skyde og blomstre og fylde Verden med Frugt.
\par 7 Har han vel slået det, som de, der slog det, blev slagne, eller blev det myrdet, som deres Mordere myrdedes?
\par 8 Ved at støde det bort og sende det bort trættede han med det; han jog det bort med sin voldsomme Ånde på Østenstormens Dag.
\par 9 Derfor sones Jakobs Brøde således, og dette er al Frugten af, at hans Synd tages bort: at han gør alle Altersten til sønderhuggede Kalksten, at Asjerastøtterne og Solstøtterne ikke mere rejser sig.
\par 10 Thi den faste Stad ligger ensom, et folketomt Sted, forladt som en Ørken. Der græsser Ungkvæget, der lejrer det sig og afgnaver Kvistene.
\par 11 Når Grenene er tørre, kommer Kvinderne og bryder dem af for at tænde Bål. Thi det er et Folk uden Indsigt; derfor kan dets Skaber ikke forbarme sig, dets Ophav ikke være det nådig.
\par 12 På hin Dag slår HERREN Frugten ned fra Flodens Strøm til Ægyptens Bæk, og I skal opsankes een for een, Israels Børn.
\par 13 På hin Dag skal der stødes i det store Horn, og de tabte i Assyrien og de bortdrevne i Ægypten skal komme og tilbede HERREN på det hellige Bjerg i Jerusalem.

\chapter{28}

\par 1 Ve Efraims berusedes stolte og dets herlige Smykkes visnende Blomster på Tindingen af de druknes fede Dal!
\par 2 Se, Herren har en vældig Kæmpe til Rede; som Skybrud af Hagl, som hærgende Storm, som Skybrud af mægtige skyllende Vande slår han til Jorden med Vælde.
\par 3 Med Fødderne trampes de ned, Efraims berusedes stolte Krans
\par 4 og dets herlige Smykkes visnende Blomster på Tindingen af den fede bal; det går den som en tidligmoden Figen før Frugthøst: Hvo der får Øje på den, plukker den, og knap er den i Hånden, før han har slugt den.
\par 5 På hin bag bliver Hærskarers HERRE en smuk Krans og en herlig Krone for sit Folks Rest
\par 6 og en Rettens Ånd for dem, som sidder til Doms, og Styrke for dem, der driver Krigen tilbage til Portene.
\par 7 Også disse raver af Vin, er svimle af Drik, Præst og Profet, de raver af Drik, fra Samling af Vin og svimle af Drik; de raver under Syner, vakler, når de dømmer.
\par 8 Thi alle Borde er fulde af Spy, Uhumskhed flyder på hver en Plet.
\par 9 "Hvem vil han belære, hvem tyder han Syner mon afvante Børn, nys tagne fra Brystet?
\par 10 Kun hakke og rakke, rakke og hakke, lidt i Vejen her og lidt i Vejen der!"
\par 11 Ja, med lallende Læber, med fremmed Mål vil han tale til dette Folk,
\par 12 han, som dog sagde til dem: "Her er der Hvile, lad den trætte hvile, her er der Ro!" men de vilde ej høre.
\par 13 Så bliver for dem da HERRENs Ord: "Hakke og rakke, rakke og bakke, lidt i Vejen her og lidt i Vejen der!" så de går hen og styrter bagover, sønderslås, fanges og hildes.
\par 14 Hør derfor HERRENs Ord, I spotske Mænd, I Nidvisens Mestre blandt dette Jerusalems Folk!
\par 15 Fordi I siger: "Vi slutted en Pagt med Døden, Dødsriget gjorde vi Aftale med; når den susende Svøbe går frem, da når den ej os, thi Løgn har vi gjort til vort Ly, vi har gemt os i Svig;"
\par 16 derfor, så siger den Herre HERREN: Se, jeg lægger i Zion en prøvet Sten, en urokkelig, kostelig Hjørnesten; tror man, baster man ikke.
\par 17 Og jeg gør Ret til Målesnor, Retfærd til Blylod; Hagl skal slå Løgnelyet ned, Vand skylle Gemmestedet bort.
\par 18 Eders Pagt med Døden skal brydes, Aftalen med Dødsriget glippe. Når den susende Svøbe går frem, skal den slå jer til Jorden,
\par 19 jer skal den ramme, hver Gang den går frem; thi Morgen efter Morgen går den frem, ved Dag og ved Nat, idel Angst skal det blive at få Syner tydet.
\par 20 Vil man strække sig, er Lejet for kort; vil man dække sig, er Tæppet for smalt.
\par 21 Thi som på Perazims Bjerg vil HERREN stå op, som i Gibeons Dal vil han vise sin Vrede for at gøre sin Gerning en underlig Gerning, og øve sit Værk et sælsomt Værk.
\par 22 Derfor hold inde med Spot, at ej eders Bånd skal snære; thi om hele Landets visse Undergang hørte jeg fra Herren, Hærskarers HERRE.
\par 23 Lyt til og hør min Røst, lån Øre og hør mit Ord!
\par 24 Bliver Plovmanden ved med at pløje til Sæd, med at bryde og harve sin Jord?
\par 25 Mon han ikke, når den er jævnet, sår Dild og udstrør Kommen, lægger Hvede, Hirse og Byg på det udsete Sted og Spelt i Kanten deraf?
\par 26 Hans Gud vejleder ham, lærer ham det rette.
\par 27 Thi med Tærskeslæde knuser man ikke Dild, lader ikke Vognhjul gå over Kommen; nej, Dilden tærskes med Stok og Kommen med Kæp.
\par 28 Mon Brødkorn knuses? Nej, det bliver ingen ved med at tærske; Vognhjul og Heste drives derover man knuser det ikke.
\par 29 Også dette kommer fra Hærskarers HERRE, underfuld i Råd og stor i Visdom.

\chapter{29}

\par 1 Ve dig, Ariel, Ariel, Byen, hvor David slog lejr! Lad år blive føjet til år, lad Højtid følge på Højtid,
\par 2 da bringer jeg Ariel Trængsel, da kommer Sorg og Kvide, da bliver du mig et Ariel,
\par 3 jeg lejrer mig mod dig som David; jeg opkaster Volde om dig, og Bolværker rejser jeg mod dig.
\par 4 Da taler du dybt fra Jorden, dine Ord er Mumlen fra Støvet; din Røst fra Jorden skal ligne et Genfærds, dine Ord er Hvisken fra Støvet.
\par 5 Dine Fjenders Hob skal være som Sandstøv, Voldsmændenes Hob som flyvende Avner. Brat, i et Nu skal det ske:
\par 6 hjemsøges skal du af Hærskarers HERRE under Torden og Brag og vældigt Drøn, Storm og Vindstød og ædende Lue.
\par 7 Som et natligt Drømmesyn bliver Hoben af alle de Folk, som angriber Ariel, af alle, der angriber det og dets Fæstning og trænger det;
\par 8 som når den sultne drømmer, at han spiser, men vågner og føler sig tom, som når den tørstige drømmer, at han drikker, men vågner mat og vansmægtende, således skal det gå Hoben af alle de Folk, der angriber Zions Bjerg.
\par 9 Undres og studs, stir jer kun blinde, vær drukne uden Vin og rav uden Drik!
\par 10 Thi HERREN har udgydt over jer en Dvalens Ånd, tilbundet eders Øjne (Profeterne), tilhyllet eders Hoveder (Seerne).
\par 11 Derfor er ethvert Syn blevet eder som Ordene i en forseglet Bog; giver man den til en, som kan læse, og siger: "Læs!" så svarer han: "Jeg kan ikke, den er jo forseglet;"
\par 12 og giver man den til en, som ikke kan læse, og siger: "Læs!" så svarer han: "Jeg kan ikke læse."
\par 13 Og Herren sagde: Eftersom dette Folk kun holder sig nær med sin Mund og ærer mig med sine Læber, mens Hjertet er fjernt fra mig, og fordi deres Frygt for mig blev tillærte Menneskebud,
\par 14 se, derfor handler jeg fremdeles sært og sælsomt med dette Folk; dets Vismænds Visdom forgår, de kloges Klogskab glipper.
\par 15 Ve dem, der dølger deres Råd i det dybe for HERREN, hvis Gerninger sker i Mørke, som siger: "Hvem ser os, og hvem lægger Mærke til os?"
\par 16 I Dårer, regnes Ler og Pottemager lige, så Værk kan sige om Mester: "Han skabte mig ikke!" eller Kunstværk om Kunstner: "Han fattes Forstand!"
\par 17 Se, end om en liden Stund skal Libanon blive til Frugthave, Frugthaven regnes for Skov.
\par 18 På hin Dag hører de døve Skriftord, og friet fra Mulm og Mørke kan blindes Øjne se.
\par 19 De ydmyge glædes end mere i HERREN, de fattige jubler i Israels Hellige.
\par 20 Thi Voldsmand er borte, Spotter forsvundet, bortryddet hver, som er vågen til ondt,
\par 21 som med Ord får et Menneske gjort skyldigt, lægger Fælde for Dommeren i Porten og kuer en retfærdig ved Opspind.
\par 22 Derfor, så siger HERREN, Jakobs Huses Gud, han, som udløste Abraham: Nu høster Jakob ej Skam, nu blegner hans Åsyn ikke;
\par 23 thi når han ser mine Hænders Værk i sin Midte, da skal han hellige mit Navn, holde Jakobs Hellige hellig og frygte Israels Gud;
\par 24 de, hvis Ånd for vild, vinder Indsigt, de knurrende tager mod Lære.

\chapter{30}

\par 1 Ve de genstridige Børn så lyder det fra HERREN som fuldbyrder Råd, der ej er fra mig, slutter Forbund, uden min Ånd er med, for at dynge Synd på Synd,
\par 2 de, som går ned til Ægypten uden at spørge min Mund for at værne sig ved Faraos Værn, søge Ly i Ægyptens Skygge!
\par 3 Faraos Værn skal blive jer til Skam og Lyet i Ægyptens Skygge til Skændsel.
\par 4 Thi er end hans Fyrster i Zoan, hans Sendebud nået til Hanes,
\par 5 enhver skal få Skam af et Folk, der ikke kan bringe dem Hjælp, ej være til Gavn eller Hjælp, men kun til Skam og Skændsel.
\par 6 Et Udsagn om Sydlandets Dyr: Gennem Angstens og Trængselens Land, hvor Løvinde og Løve har hjemme, Giftsnog og vinget Slange, fører de på Æslers Ryg deres Gods, på Kamelers Pukkel deres Skatte til et Folk, der ikke kan hjælpe.
\par 7 Ægyptens Hjælp er Vind og Luft. Derfor kalder jeg det "Rahab, der hytter sig."
\par 8 Gå nu hen og skriv det på en Tavle i deres Påsyn og optegn det i en Bog, at det i kommende Tider kan stå som Vidnesbyrd evindelig.
\par 9 Thi det er et stivsindet Folk, svigefulde Børn, Børn, der ikke vil høre HERRENs Lov,
\par 10 som siger til Seerne: "Se ingen Syner!" til fremsynte: "Skuer os ikke det rette! Tal Smiger til os, skuer os Blændværk,
\par 11 vig bort fra Vejen, bøj af fra Stien, lad os være i Fred for Israels Hellige!"
\par 12 Derfor, så siger Israels Hellige: Siden I ringeagter dette Ord og stoler på krumt og kroget og støtter jer til det,
\par 13 derfor skal denne Brøde blive for eder som en truende, voksende Revne i en knejsende Mur, hvis Fald vil indtræffe brat, lige i et Nu;
\par 14 den sønderbrydes som Lerkar; der skånselsløst knuses; blandt Stumperne finder man ikke et Skår, hvori man kan hente en Glød fra Bålet eller øse Vand af Brønden.
\par 15 Thi således sagde den Herre HERREN, Israels Hellige: Ved Omvendelse og Stilhed skal I frelses, i Ro og Tillid er eders Styrke.
\par 16 Men I vilde ikke; I sagde: "Nej, vi jager på Heste" I skal derfor jages!"Vi rider på Rapfod" I skal derfor forfølges af rappe.
\par 17 Tusind skal fly for een, som truer; Flugten skal I tage for fem, som truer, til I kun er en Rest som Stangen på Bjergets Tinde, som Banneret oppe på Højen.
\par 18 Derfor længes HERREN efter at vise eder Nåde, derfor står han op for at forbarme sig over eder. Thi Rettens Gud er HERREN; salige alle, der længes efter ham!
\par 19 Ja, du Folk i Zion, du, som bor i Jerusalem, lad ikke Gråden overmande dig! Nådig vil han vise dig Nåde, når du råber; så snart han hører dig, svarer han.
\par 20 Herren skal give eder Trængselsbrød og Fængselsdrik; men så skal din Vejleder ikke mere dølge sig, dine, Øjne skal skue din Vejleder;
\par 21 dine Ører skal høre det Ord bag ved dig: "Her er Vejen, I skal gå!" hver Gang I er ved at vige til højre eller venstre.
\par 22 Da skal du holde dine Sølvbilleders Overtræk og dine Guldbilleders Klædning for urene; du skal slænge dem bort som Skarn.
\par 23 Da giver han Regn til Sæden, du,sår i din Jord; og Brødet, som din Jord bærer, skal være kraftigt og nærende. På hin Dag græsser dit Kvæg på vide Vange;
\par 24 Okserne og Æslerne, der arbejder på Marken, skal æde saltet Blandfoder, renset med Kasteskovl og Fork.
\par 25 På hvert højt Bjerg og hver knejsende Banke skal Kilder vælde frem med rindende Vand på det store Blodbads Dag, når Tårne falder.
\par 26 Månens Lys skal blive som Solens, og Solens Lys skal blive syvfold stærkere, som syv Dages Lys, på hin Dag da HERREN forbinder sit Folks Brud og læger dets slagne Sår.
\par 27 Se, HERRENs Navn kommer langvejsfra i brændende Vrede, med tunge Skyer; hans Læber skummer af Vrede, fortærende Ild er hans Tunge,
\par 28 hans Ånde som en rivende Strøm, der når til Halsen. Folkene ryster han i Undergangens Sold, lægger Vildelsens Bidsel i Folkeslags Mund.
\par 29 Sang skal der være hos eder som i Natten, når Højtid går ind, en Hjertens Glæde som en Vandring til Fløjte mod HERRENs Bjerg, mod Israels Klippe.
\par 30 HERREN lader høre sin Højheds Røst og viser sin Arm, der slår ned med fnysende Vrede, ædende Lue, Skybrud, skyllende Regn og Hagl.
\par 31 For HERRENs Røst bliver Assur ræd, med Kæppen slår han;
\par 32 hvert et Slag af Tugtelsens Stok, som HERREN lader falde på Assur, er til Paukers og Citres Klang; med Svingnings Kampe kæmper han mod det.
\par 33 Thi for længst står et Alter rede mon det og er rejst for Molok? han gjorde dets Ildfang dybt og bredt, bragte Ild og Ved i Mængde; HERRENs Ånde sætter det i Brand som en Strøm af Svovl.

\chapter{31}

\par 1 Ve dem, som går ned til Ægypten om hjælp og slår Lid til heste, som stoler på Vognenes mængde, på Rytternes store Tal, men ikke ser hen til Israels Hellige, ej rådspørger HERREN.
\par 2 Men viis er og han, lader Ulykke komme og går ej fra sit Ord.
\par 3 Ægypterne er Mennesker, ikke Gud, deres Heste er Kød, ikke Ånd.
\par 4 Thi så sagde HERREN til mig: Som en Løve knurrer, en Ungløve over sit Rov, og ikke, når Hyrdernes Flok kaldes hid imod den, skræmmes af Skriget eller viger for Larmen, så stiger Hærskarers HERRE ned til Kamp på Zions Bjerg og Høj.
\par 5 Som svævende Fugle så skærmer Hærskarers HERRE Jerusalem, skærmer og frier, skåner og redder.
\par 6 Vend om til ham, hvem Israels Børn faldt fra så dybt!
\par 7 Thi på hin Dag vrager enhver sine Guder af Sølv sine Guder af Guld, eders Hænders syndige Værk.
\par 8 Assur falder for Sværd, men ikke en Mands, et Sværd fortærer det, ikke et Menneskes. Og han skal fly for Sværdet, til Hoveriarbejde tvinges hans Stridsmænd;
\par 9 hans Klippe viger bort af Rædsel, hans Fyrster skræmmes fra Fanen. Så lyder det fra HERREN, hvis Ild er i Zion, som har sin Ovn i Jerusalem.

\chapter{32}

\par 1 Se, en Konge skal herske med Retfærd, Fyrster styre med Ret,
\par 2 hver af dem som Læ imod Storm og Ly imod Regnskyl, som Bække i Ørk, som en vældig Klippes Skygge i tørstende Land.
\par 3 De seendes Øjne skal ej være blinde, de hørendes Ører skal lytte;
\par 4 letsindiges Hjerte skal nemme Kundskab, stammendes Tunge tale flydende, rent.
\par 5 Dåren skal ikke mer kaldes ædel, højsindet ikke Skalken.
\par 6 Thi Dåren taler kun Dårskab, hans Hjerte udtænker Uret for at øve Niddingsværk og prædike Frafald fra HERREN, lade den sultne være tom og den tørstige mangle Vand.
\par 7 Skalkens Midler er onde, han oplægger lumske Råd for at ødelægge arme med Løgn, skønt Fattigmand godtgør sin Ret.
\par 8 Men den ædle har ædelt for og står fast i, hvad ædelt er.
\par 9 Op, hør min Røst, I sorgløse Kvinder, I trygge Døtre, lyt til min Tale!
\par 10 Om År og Dag skal I trygge skælve, thi med Vinhøst er det ude, der kommer ej Frugthøst.
\par 11 Bæv, I sorgløse, skælv, I trygge, klæd jer af og blot jer, bind Sæk om Lænd;
\par 12 slå jer for Brystet og klag over yndige Marker, frugtbare Vinstokke,
\par 13 mit Folks med Tidseltorn dækkede Jord, ja, hvert Glædens Hus, den jublende By!
\par 14 Thi Paladset er øde, Bylarmen standset, Ofel med Tårnet en Grushob for evigt, Vildæslers Fryd, en Græsgang for Hjorde -
\par 15 til Ånd fra det høje udgydes over os. Da bliver Ørkenen til Frugthave, Frugthaven regnes for Skov.
\par 16 Ret fæster Bo i Ørkenen, i Frugthaven dvæler Retfærd;
\par 17 Retfærds Frugt bliver Fred og Rettens Vinding Tryghed for evigt.
\par 18 Da bor mit Folk i Fredens Hjem, i trygge Boliger, sorgfri Pauluner.
\par 19 Skoven styrter helt, Byen bøjes dybt.
\par 20 Salige I, som sår ved alle Vande, lader Okse og Æsel frit løbe om!

\chapter{33}

\par 1 Ve dig, du Hærværksmand, selv ikke hærget, du Ransmand, skånet for Ran! Når dit Hærværk er endt, skal du hærges, når din Ranen har Ende, skal der ranes fra dig!
\par 2 HERRE, vær os nådig, vi bier på dig, vær du vor Arm hver Morgen, vor Frelse i Nødens Stund!
\par 3 For Bulderet må Folkeslag fly; når du rejser dig, splittes Folkene.
\par 4 Som Græshopper bortriver, bortrives Bytte, man styrter derover som Græshoppesværme.
\par 5 Ophøjet er HERREN, thi han bor i det høje, han fylder Zion med Ret og Retfærd.
\par 6 Trygge Tider skal du have, en Frelsesrigdom er Visdom og Indsigt, HERRENs Frygt er din Skat.
\par 7 Se, deres Helte skriger derude, Fredens Sendebud græder bittert;
\par 8 Vejene er øde, vejfarende borte. Han brød sin Pagt, agted Byer ringe, Mennesker regned han ikke.
\par 9 Landet blegner og sygner, Libanon skæmmes og visner; Saron er som en Ørken, Basan og Karmel uden Løv.
\par 10 Nu står jeg op, siger HERREN, nu vil jeg rejse mig, nu træde frem!
\par 11 I undfanger Strå og føder Halm, eders Ånde er Ild, der fortærer jer selv;
\par 12 til Kalk skal Folkene brændes som afhugget Torn, der brænder i Ild.
\par 13 Hvad jeg gør, skal rygtes til fjerne Folk, nære skal kende min Vælde.
\par 14 På Zion skal Syndere bæve, Niddinger gribes af Skælven: "Hvem kan bo ved fortærende Ild, hvem kan bo ved evige Bål?"
\par 15 Den, der vandrer i Retfærd og taler oprigtigt, ringeagter Vinding, vundet ved Uret, vægrer sig ved at tage mod Gave, tilstopper Øret over for Blodråd og lukker Øjnene over for det onde -
\par 16 højt skal en sådan bo, hans Værn skal Klippeborge være; han får sit Brød, og Vand er ham sikret.
\par 17 Dine Øjne får Kongen at se i hans Skønhed, de skuer et vidtstrakt Land.
\par 18 Dit Hjerte skal tænke på Rædselen: "Hvor er nu han, der talte og vejede, han, der talte Tårnene?"
\par 19 Du ser ej det vilde Folk med dybt, uforståeligt Mål, med stammende, ufattelig Tunge.
\par 20 Se på Zion, vore Højtiders By! Dine Øjne skal skue Jerusalem, et sikkert Lejrsted, et Telt, der ej flytter, hvis Pæle aldrig rykkes op, hvis Snore ej rives over.
\par 21 Nej der træder HERRENs Bæk for os i Floders og brede Strømmes Sted; der kan ej Åreskib gå, ej vældigt Langskib sejle.
\par 22 Thi HERREN er vor Dommer, HERREN er vor Hersker, HERREN er vor Konge, han bringer os Frelse.
\par 23 Slapt hænger dit Tovværk, det holder ej Råen og spænder ej Sejlet. Da uddeles røvet Bytte i Overflod, halte tager Del i Rovet.
\par 24 Ingen Indbygger siger: "Jeg er syg!" Folket der har sin Synd forladt.

\chapter{34}

\par 1 Kom hid, I Folk, og hør, lån Øre, I Folkefærd! Jorden og dens Fylde høre, Jorderig og al dets Grøde!
\par 2 Thi HERREN er vred på alle Folkene, harmfuld på al deres Hær; han slår dem med Band og giver dem hen til at slagtes;
\par 3 henslængt ligger de dræbte, Stank stiger op fra Ligene, Bjergene flyder af Blodet;
\par 4 al Himlens Hær opløses; som en Bog rulles Himlen sammen, og al dens Hær visner hen som Vinstokkens visnende Blad, som Figentræets visnende Frugt.
\par 5 Thi på Himlen kredser HERRENs Sværd, og se, det slår ned på Edom, det Folk, han har bandlyst til Dom.
\par 6 HERRENs Sværd er fuldt af Blod, det drypper af Fedt, af Fårs og Bukkes Blod, af Fedt fra Vædres Nyrer. Thi HERREN slagter Offer i Bozra har vældig Slagtning i Edom;
\par 7 Urokser styrter med dem, Ungkvæg sammen med Tyre. Landet svælger i Blod, Jorden drypper af Fedt.
\par 8 Thi en Hævndag har HERREN til Rede, Zions Værge et Gengældsår.
\par 9 Dets Bække forvandles til Tjære, dets Jord til Svovl, og Landet bliver til Tjære, der brænder ved Nat og ved Dag,
\par 10 det slukkes aldrig; evigt stiger Røgen op, det er øde fra Slægt til Slægt, ingen skal færdes der.
\par 11 Pelikan og Rørdrum arver det, Ugle og Ravn skal bo der. HERREN spænder Tomheds Snor og Ødelæggelses Blylod derover.
\par 12 Der skal Bukketrolde bo, dets ypperste bliver til intet, til Kongevalg kaldes ej der, det er ude med alle dets Fyrster.
\par 13 Dets Paladser gror til i Tom, dets Borge i Tidsel og Nælde, et Tilholdssted for Hyæner og Enemærke for Strudse.
\par 14 Der mødes Sjakal med Vildkat, og Bukketrolde holder Stævne; kun der skal Natteheksen raste og lægge sig der til Ro;
\par 15 der bygger Pilslangen Rede, lægger Æg og samler dem og ruger.
\par 16 Se efter i HERRENs Bog og læs: Ej fattes en eneste af dem, ej savner den ene den anden. Thi HERRENs Mund, den bød, hans Ånd har samlet dem sammen;
\par 17 han kastede Loddet for dem, hans Hånd udskifted dem Land med Snoren; de tager det evigt i Eje, bor der fra Slægt til Slægt.

\chapter{35}

\par 1 Ørken og hede skal fryde sig, Ødemark juble og blomstre;
\par 2 blomstre frodigt som Rosen og juble, ja juble med Fryd.
\par 3 Styrk de slappe Hænder, lad de vaklende Knæ blive faste,
\par 4 sig til de ængstede Hjerter: Vær stærke, vær uden Frygt! Se eders Gud! Han kommer med Hævn, Gengæld kommer fra Gud; han kommer og frelser eder.
\par 5 Da åbnes de blindes Øjne, de døves Ører lukkes op;
\par 6 da springer den halte som Hjort, den stummes Tunge jubler; thi Vand vælder frem i Ørkenen, Bække i Ødemark;
\par 7 det glødende Sand bliver Vanddrag, til Kildevæld tørstigt Land.
\par 8 Der bliver en banet Vej, .den hellige Vej skal den kaldes; ingen uren færdes på den, den er Valfartsvej for hans Folk, selv enfoldige farer ej vild.
\par 9 På den er der ingen Løver, Rovdyr træder den ej, der skal de ikke findes. De genløste vandrer ad den,
\par 10 HERRENs forløste vender hjem, de drager til Zion med Jubel, med evig Glæde om Issen; Fryd og Glæde får de, Sorg og Suk skal fly.

\chapter{36}

\par 1 I Kong Ezekias's fjortende Regeringsår drog Assyrerkongen Sankerib op mod alle Judas befæstede Byer og indtog dem.
\par 2 Assyrerkoogen sendte så Rasjake med en anselig Styrke fra Lakisj til Kong Ezekias i Jerusalem, og han gjorde Holdt ved Øvredammens Vandledning, ved Vejen til Blegepladsen.
\par 3 Da gik Paladsøversten Eljakim, Hilkijas Søn, Statsskriveren Sjebna og Kansleren Joa, Asafs Søn, ud til ham.
\par 4 Rabsjake sagde fil dem: "Sig til Ezekias: Således siger Storkongen, Assyrerkongen: Hvad er det for en Fortrøsfning, du hengiver dig til?
\par 5 Du mener vel, at et blot og bart Ord er det samme som Plan og Styrke i Krig? Og til hvem sætter du egentlig din Lid, siden du gør Oprør imod mig?
\par 6 Se, du sætter din Lid til Ægypten, denne brudte Rørkæp, som river Sår i Hånden på den, der støtter sig til den! Thi således går det alle dem, der sætter deres Lid til Farao, Ægyptens Konge.
\par 7 Men vil du sige til mig: Det er HERREN vor Gud, vi sætter vor Lid til! er det så ikke ham, hvis Offerhøje og Altre Ezekias skaffede bort, da han sagde til Juda og Jerusalem: Foran dette Alter skal I tilbede!
\par 8 Og nu, indgå et Væddemål med min Herre, Assyrerkongen: Jeg giver dig to Gange tusinde Heste, hvis du kan stille Ryttere til dem!
\par 9 Hvorledes vil du afslå et Angreb af en eneste Statholder, en af min Herres ringeste Tjenere? Og du sætter din Lid til Ægypten, til Vogne og Heste?
\par 10 Mon det desuden er uden HERRENs Vilje, at jeg er draget op mod dette Land for at ødelægge det? Det var HERREN selv, der sagde til mig: Drag op mod dette Land og ødelæg det!"
\par 11 Men Eljakim, Sjebna og Joa sagde til Rabsjake: "Tal dog Aramæisk til dine Trælle, det forstår vi godt; tal ikke Judæisk til os, medens Folkene på Muren hører på det!"
\par 12 Men Rabsjake svarede dem: "Er det til din Herre og dig, min Herre har sendt mig med disse Ord? Er det ikke til de Mænd, der sidder på Muren hos eder og æder deres eget Skarn og drikker deres eget Vand?"
\par 13 Og Rabsjake trådte hen og råbte med høj Røst på Judæisk: "Hør Storkongens, Assyrerkongens, Ord!
\par 14 Således siger Kongen: Lad ikke Ezekias vildlede eder, thi han er ikke i Stand til at frelse eder!
\par 15 Og lad ikke Ezekias forlede eder til at sætte eders Lid til HERREN, når han siger: HERREN skal sikkert frelse os, og denne By skal ikke overgives i Assyrerkongens Hånd!
\par 16 Hør ikke på Ezekias, thi så; ledes siger Assyrerkongen: Vil I slutte Fred med mig og overgive eder til mig, så skal enhver af eder spise af sin Vin, stok og sit Figentræ og drikke af sin Brønd,
\par 17 indtil jeg kommer og tager eder med til et Land, der ligner eders, et Land med Korn og Most, et Land med Brød og Vingårde.
\par 18 Lad ikke Ezekias forføre eder med at sige: HERREN vil frelse os! Mon nogen af Folkenes Guder har kunnet frelse sit Land af Assyrerkongens Hånd?
\par 19 Hvor er Hamats og Arpads Guder, hvor er Sefarvajims Guder, hvor er Landet Samarias Guder? Mon de frelste Samaria af min Hånd?
\par 20 Hvor er der blandt alle disse Landes Guder nogen, der har frelst sit Land af min Hånd? Mon da HERREN skulde kunne frelse Jerusalem?"
\par 21 Men de tav og svarede ham ikke et Ord, thi Kongens Bud lød på, at de ikke måtte svare ham.
\par 22 Derpå gik Paladsøversten Eljakim, Hilkijas Søn, Statsskriveren Sjebna og Kansleren Joa, Asafs Søn, med sønderrevne Klæder til Ezekias og meddelte ham, hvad Rabsjake havde sagt.

\chapter{37}

\par 1 Da Kong Ezekias hørte det, sønderrev han sine Klæder, hyllede sig i Sæk og gik ind i HERRENs Hus.
\par 2 Og han sendte Paladsøversten Eljakim og Statsskriveren Sjebna og Præsternes Ældste, hyllet i Sæk, til Profeten Esajas, Amoz's Søn,
\par 3 for at sige til ham: "Ezekias lader sige: En Nødens, Tugtelsens og Forsmædelsens Dag er denne Dag, thi Barnet er ved at fødes, men der er ikke Kraft til at bringe det til Verden!
\par 4 Dog vil HERREN din Gud måske høre, hvad Rabsjake har sagt, han, som er sendt af sin Herre, Assyrerkongen, for at håne den levende Gud, og måske vil han straffe ham for de Ord, som HERREN din Gud har hørt gå derfor i Forbøn for den Rest, der endnu er tilbage!"
\par 5 Da Kong Ezekias's Folk kom til Esajas,
\par 6 sagde han til dem: "Således skal I svare eders Herre: Så siger HERREN: Frygt ikke for de Ord, du har hørt, som Assyrerkongens Trælle har hånet mig med!
\par 7 Se, jeg vil indgive ham en Ånd, og han skal få en Tidende at høre, så han vender tilbage til sit Land, og i hans eget Land vil jeg fælde ham med Sværdet!"
\par 8 Rabsjake vendte så tilbage og traf Assyrerkongen i Færd med at belejre Libna, thi han havde hørt, at Kongen var brudt op fra Lakisj.
\par 9 Så fik han Underretning: om, at Kong Tirhaka af Ætiopien var rykket ud for at angribe ham, og han sendte Sendebud til Ezekias og sagde:
\par 10 "Således skal I sige til Kong Ezekias af Juda: Lad ikke din Gud, som du slår din Lid til, vildlede dig med at sige, at Jerusalem ikke skal gives i Assyrerkongens Hånd!
\par 11 Du har jo dog hørt, hvad Assyrerkongerne har gjort ved alle Lande, hvorledes de har lagt Band på dem og du skulde kunne undslippe!
\par 12 De Folk, mine Fædre tilintetgjorde, Gozan, Karan, Rezef og Folkene fra Eden i Telassar, har deres Guder kunnet frelse dem?
\par 13 Hvor er Kongen af Hamat, Kongen af Arpad eller Kongen af La'ir, Sefarvajim, Hena og Ivva?"
\par 14 Da Ezekias havde modtaget Brevet af Sendebudenes Hånd og læst det, gik han op i HERRENs Hus og bredte det ud for HERRENs Åsyn.
\par 15 Derpå bad Ezekias den Bøn for HERRENs Åsyn:
\par 16 "Hærskarers HERRE, Israels Gud, du, som troner over Keruberne, du alene er Gud over alle Jordens Riger; du har gjort Himmelen og Jorden!
\par 17 Bøj nu dit Øre, HERRE, og lyt, åbn dine Øjne, HERRE, og se! Læg Mærke til alle de Ord, Sankerib har sendt hid for at spotte den levende Gud!
\par 18 Det er sandt, HERRE, at Assyrerkongerne har tilintetgjort alle de Folk og deres Lande
\par 19 og kastet deres Guder i Ilden; men de er ikke Guder, kun Menneskehænders Værk af Træ eller Sten; derfor kunde de ødelægge dem.
\par 20 Men frels os nu, HERRE vor Gud, af hans Hånd, så alle Jordens Riger kan kende, at du, HERRE, alene er Gud!"
\par 21 Så sendte Esajas, Amoz's Søn, Bud til Ezekias og lod sige: "Så siger HERREN, Israels Gud: Din Bøn angående Assyrerkongen Sankerib har jeg hørt!"
\par 22 Således lyder det Ord, HERREN talede imod ham: Hun håner, hun spotter dig, Jomfruen, Zions Datter, Jerusalems Datter ryster på Hovedet ad dig!
\par 23 Hvem har du hånet og smædet, mod hvem har du løftet din Røst? Til Israels Hellige løfted i Hovmod du Blikket!
\par 24 Ved dine Trælle håned du HERREN og sagde: "Med mine talløse Vogne besteg jeg Bjergenes Højder, Libanons afsides Egne; jeg fælded dets Cedres Højskov, dets ædle Cypresser, trængte frem til dets øverste Raststed, dets Havers Skove.
\par 25 Fremmed Vand grov jeg ud, og jeg drak det, tørskoet skred jeg over Ægyptens Strømme!"
\par 26 Har du ej hørt det? For længst kom det op i min Tanke, jeg lagde det fordum til Rette, nu lod jeg det ske, og du Gjorde murstærke Byer til øde Stenhobe
\par 27 mens Folkene grebes i Afmagt af Skræk og Skam, blev som Græsset på Marken, det spirende Grønne, som Græs på Tage, som Mark for Østenvinden.
\par 28 Jeg ser, når du rejser og sætter dig, ved, når du går og kommer.
\par 29 Fordi du raser imod mig, din Trods bar nået mit Øre, lægger jeg Ring i din Næse og Bidsel i Munden og fører dig bort ad Vejen, du kom!
\par 30 Og dette skal være dig Tegnet: I År skal man spise, hvad der såed sig selv, og Året derpå, hvad der skyder af Rode, tredje År skal man så og høste, plante Vin og nyde dens Frugt.
\par 31 Den, bjærgede Rest af Judas Hus slår atter Rødder forneden og bærer sin Frugt foroven;
\par 32 thi fra Jerusalem udgår en Rest, en Levning fra Zions Bjerg.
\par 33 Derfor, så siger HERREN om Assyrerkongen: I Byen her skal han ej komme ind, ej sende en Pil herind, ej nærme sig den med Skjolde eller opkaste Vold imod den;
\par 34 ad Vejen, han kom, skal han gå igen, i Byen her skal han ej komme ind så lyder det fra HERREN.
\par 35 Jeg værner og frelser denne By for min og min Tjener Davids Skyld!
\par 36 Så gik HERRENs Engel ud og ihjelslog i Assyrernes Lejr 185000 Mand; og se, næste Morgen tidlig lå de alle døde.
\par 37 Da brød Assyrerkongen Sankerib op, vendte hjem og blev siden i Nineve.
\par 38 Men da han engang tilbad i sin Gud Nisroks Hus, slog hans Sønner Adrammelek og Sar'ezer ham ihjel med deres Sværd, hvorefter de flygtede til Ararats Land: og hans Søn Asarhaddon blev Konge i hans Sted.

\chapter{38}

\par 1 Ved den Tid blev Ezekias dødssyg. Da kom Profeten Esajas, Amoz's Søn, til ham og sagde: "Så siger HERREN: Beskik dit Hus, thi du skal dø og ikke leve!"
\par 2 Da vendte han Ansigtet om mod Væggen og bad således til HERREN:
\par 3 "Ak, HERRE, kom dog i Hu, hvorledes jeg har vandret for dit Åsyn i Oprigtighed og med helt Hjerte og gjort, hvad der er godt i dine Øjne!" Og Ezekias græd højt.
\par 4 Da kom HERRENs Ord til Esajas således:
\par 5 "Gå hen og sig til Ezekias: Så siger HERREN, din Fader Davids Gud: Jeg har hørt din Bøn, jeg har set dine Tårer! Se, jeg vil lægge femten År til dit Liv
\par 6 og udfri dig og denne By af Assyrerkongens Hånd og værne om denne By!
\par 7 Og Tegnet fra HERREN på, at HERREN vil udføre, hvad han har sagt, skal være dig dette:
\par 8 Se, jeg vil lade Skyggen gå de Streger tilbage, som den har flyttet sig med Solen på Akaz's Solur, ti Streger!" Da gik Solen de ti Streger, som den havde flyttet sig, tilbage på Soluret.
\par 9 En Bøn af Kong Ezekias af Juda, da han var syg og kom sig af sin Sygdom:
\par 10 Jeg tænkte: Bort må jeg gå i min bedste Alder, hensættes i Dødsrigets Porte mine sidste År.
\par 11 Jeg tænkte: Ej skuer jeg HERREN i de levendes Land, ser ingen Mennesker mer blandt Skyggerigets Folk;
\par 12 min Bolig er nedbrudt, ført fra mig som Hyrdernes Telt, som en Væver sammenrulled du mit Liv og skar det fra Tråden. Du ofrer mig fra Dag til Nat,
\par 13 jeg skriger til daggry; som en Løve knuser han alle Benene i mig; du giver mig ben fra Dag til Nat.
\par 14 Jeg klynker som klagende Svale, sukker som Duen, jeg skuer med Tårer mod Himlen: HERRE, jeg trænges, vær mig Borgen!
\par 15 Hvad skal jeg,sige? Han talede til mig og selv greb han ind.
\par 16 Herre, man skal bære Bud derom til alle kommende Slægter.
\par 17 Se, Bitterhed, Bitterhed blev mig til Fred. Og du skåned min Sjæl for Undergangens Grav; thi alle mine Synder kasted du bag din Ryg.
\par 18 Thi Dødsriget takker dig ikke, dig lover ej Døden, på din Miskundbed håber ej de, der synker i Graven.
\par 19 Men den levende, den levende takker dig som jeg i Dag. Om din Trofasthed taler Fædre til deres Børn.
\par 20 HERRE, frels os! Så vil vi røre Strengene alle vore Levedage ved HERRENs Hus.
\par 21 Da bød Esajas, at man skulde tage en Figenkage og lægge den som Plaster på det syge Sted, for at han kunde blive rask.
\par 22 Og Ezekias sagde: "Hvad er Tegnet på, at jeg skal gå op til HERRENs Hus?"

\chapter{39}

\par 1 Ved den Tid sendte Bal'adans Søn, Kong Merodak-Bal'adan af Babel, Brev og Gave til Ezekias, da han hørte, at han havde været syg, men var blevet rask.
\par 2 Og Ezekias glædede sig over deres Komme og viste dem Huset, hvor han havde sine Skatte, Sølvet og Guldet, Røgelsestofferne, den fine Olie, hele sit Våbenoplag og alt, hvad der var i hans Skatkamre; der var ikke den Ting i hans Hus og hele hans Rige, som Ezekias ikke viste dem.
\par 3 Da kom Profeten Esajas til Kong Ezekias og sagde til ham: "Hvad sagde disse Mænd, og hvorfra kom de til dig?" Ezekias svarede: "De kom fra et fjernt Land, fra Babel."
\par 4 Da spurgte han: "Hvad fik de at se i dit Hus?" Ezekias svarede: "Alt, hvad der er i mit Hus, så de; der er ikke den Ting i mine Skatkamre, jeg ikke viste dem."
\par 5 Da sagde Esajas til Ezekias: "Hør Hærskarers HERREs Ord!
\par 6 Se, Dage skal komme, da alt, hvad der er i dit Hus, og hvad dine Fædre har samlet indtil denne Dag, skal bringes til Babel og intet lades tilbage, siger HERREN.
\par 7 Og af dine Sønner, der nedstammer fra dig, og som du avler, skal nogle tages og gøres til Hofmænd i Babels Konges Palads!"
\par 8 Men Ezekias sagde til Esajas: "det Ord fra HERREN, du har talt, er godt!" Thi han tænkte: "Så bliver der da Fred og Tryghed, så længe jeg lever!"

\chapter{40}

\par 1 Trøst, ja trøst mit Folk, så siger eders Gud,
\par 2 tal Jerusalem kærligt til og råb kun til det, at nu er dets Strid til Ende, dets Skyld betalt, tvefold Straf har det fået af HERRENs Hånd for alle sine Synder.
\par 3 I Ørkenen råber en Røst: "Ban HERRENs Vej, jævn i det øde Land en Højvej for vor Gud!
\par 4 Hver Dal skal højnes, hvert Bjerg, hver Høj, skal sænkes, bakket Land blive fladt og Fjeldvæg til Slette.
\par 5 Åbenbares skal HERRENs Herlighed, alt Kød til Hobe skal se den.
\par 6 Der lyder en Røst, som siger: "Råb!" Jeg svarer: "Hvad skal jeg råbe?" "Alt Kød til Hobe er Græs, al dets Ynde som Markens Blomst;
\par 7 Græsset tørres, Blomsten visner, når HERRENs Ånde blæser derpå; visselig, Folket er Græs,
\par 8 Græsset tørres, Blomsten visner, men vor Guds Ord bliver evindelig."
\par 9 Stig op på højen Bjerg, du Zions Glædesbud, løft din Røst med Kraft, du Jerusalems Glædesbud, løft den uden Frygt og sig til Judas Byer: "Se eders Gud!"
\par 10 Se, den Herre HERREN kommer med Vælde, han hersker med sin Arm. Se, hans Løn er med ham, hans Vinding foran ham,
\par 11 han vogter sin Hjord som en Hyrde, samler den med Armen, bærer Lammene i Favn og leder de diende Får.
\par 12 Hvo måler Vandet i sin Hånd, afmærker med Fingerspand Himlen, måler Jordens Støv i Skæppe og vejer Bjerge med Bismer eller i Vægtskål Høje?
\par 13 Hvo leder HERRENs Ånd, råder og lærer ham noget?
\par 14 Hos hvem får han Råd og Indsigt, hvem lærer ham Rettens Vej, hvem kan give ham Kundskab, hvem kundgør ham indsigts Vej?
\par 15 Se, som Dråbe på Spand er Folkene, at regne som Fnug på Vægt, som et Gran vejer fjerne Strande.
\par 16 Libanon giver ej Brændsel, dets Dyr ej Brændoffer nok.
\par 17 Alle Folk er som intet for ham, for Luft og Tomhed at regne.
\par 18 Med hvem vil I ligne Gud, hvad stiller I op som hans Lige?
\par 19 Et Billede det støber en Mester, en Guldsmed lægger Guld derpå, og Sølvkæder støber en anden.
\par 20 Den, som vil rejse en Afgud, vælger sig Træ, som ej rådner; han søger sig en kyndig Mester til at rejse et Billede, som står.
\par 21 Ved I, hører I det ikke, er det . ikke forkyndt jer for længst? Har I da ikke skønnet det, fra Jordens Grundvold blev lagt?
\par 22 Han troner over Jordens Kreds, som Græshopper er dens Beboere; han udbreder Himlen som en Dug og spænder den ud som et Teltbo.
\par 23 Fyrster gør han til intet, Jordens Dommere til Luft.
\par 24 Knap er de plantet, knap er de sået, knap har Stiklingen Rod i Jorden, så ånder han på dem, de visner; som Strå fejer Storm dem bort.
\par 25 Hvem vil I ligne mig med som min Ligemand? siger den Hellige.
\par 26 Løft eders Blik til Himlen og se: Hvo skabte disse? Han mønstrer deres Hær efter Tal, kalder hver enkelt ved Navn; så stor er hans Kraft og Vælde, at ikke en eneste mangler.
\par 27 Hvorfor siger du, Jakob, hvi taler du, Israel, så: "Min Vej er skjult for HERREN, min Ret gled min Gud af Hænde."
\par 28 Ved du, hørte du ikke, at HERREN er en evig Gud, den vide Jord har han skabt? Han trættes og mattes ikke, hans Indsigt udgrundes ikke;
\par 29 han giver den trætte Kraft, den svage Fylde af Styrke.
\par 30 Ynglinge trættes og mattes, Ungersvende snubler brat,
\par 31 ny Kraft får de, der bier på HERREN, de får nye Svingfjer som Ørnen; de løber uden at mattes, vandrer uden at trættes.

\chapter{41}

\par 1 Hør mig i Tavshed, I fjerne Strande, lad Folkene hente ny Kraft, komme hid og tage til Orde, lad os sammen gå frem for Retten!
\par 2 Hvo vakte i Østen ham, hvis Fod går fra Sejr til Sejr, hvo giver Folk i hans Vold og gør ham til Kongers Hersker? Han gør deres Sværd til Støv, deres Buer til flagrende Strå,
\par 3 forfølger dem, går uskadt frem ad en Vej, hans Fod ej har trådt.
\par 4 Hvo gjorde og virkede det? Han, som længst kaldte Slægterne frem, jeg, HERREN, som er den første og end hos de sidste den samme.
\par 5 Fjerne Strande så det med Gru, den vide Jord følte Rædsel, de nærmede sig og kom.
\par 6 Den ene hjælper den anden og siger: "Broder, fat Mod!"
\par 7 Mesteren opmuntrer Guldsmeden, Glatteren ham, der hamrer; Lodningen tager han god og sømmer det fast, så det står.
\par 8 Men Israel, du min Tjener, Jakob, hvem jeg har udvalgt, Ætling af Abraham, min Ven -
\par 9 hvem jeg tog fra Jordens Grænser og kaldte fra dens fjerneste Kroge, til hvem jeg sagde: "Du min Tjener, som jeg valgte og ikke vraged":
\par 10 Frygt ikke, thi jeg er med dig, vær ej rådvild, thi jeg er din Gud! Med min Retfærds højre styrker, ja hjælper, ja støtter jeg dig.
\par 11 Se, Skam og Skændsel får alle, som er dig fjendske, til intet bliver de, der trætter med dig, de forgår.
\par 12 Du søger, men finder ej dem, der kives med dig, til intet, til Luft bliver de, der strides med dig.
\par 13 Thi jeg, som er HERREN din Gud, jeg griber din Hånd, siger til dig: Frygt kun ikke, jeg er din Hjælper.
\par 14 Frygt ikke, Jakob, du Orm, Israel, du Kryb! Jeg hjælper dig, lyder det fra HERREN, din Genløser er Israels Hellige.
\par 15 Se, jeg gør dig til Tærskeslæde, en ny med mange Tænder; du skal tærske og knuste Bjerge, og Høje skal du gøre til Avner;
\par 16 du kaster dem, Vinden tager dem, Stormen hvirvler dem bort.
\par 17 Forgæves søger de arme og fattige Vand, deres Tunge brænder af Tørst; jeg, HERREN, vil bønhøre dem, dem svigter ej Israels Gud.
\par 18 Fra nøgne Høje sender jeg Floder og Kilder midt i Dale; Ørkenen gør jeg til Vanddrag, det tørre Land til Væld.
\par 19 I Ørkenen giver jeg Cedre, Akacier, Myrter, Oliven; i Ødemark sætter jeg Cypresser tillige med Elm og Gran,
\par 20 at de må se og kende, mærke sig det og indse, at HERRENs Hånd har gjort det, Israels Hellige skabt det.
\par 21 Fremlæg eders Sag, siger HERREN, kom med Bevis! siger Jakobs Konge.
\par 22 De træde nu frem og forkynde os, hvad der herefter skal ske.
\par 23 Forkynd, hvad der siden vil ske, at vi kan se, I er Guder! Gør noget, godt eller ondt, så måler vi os med hinanden!
\par 24 Se, I er intet, eders Gerning Luft, vederstyggelig, hvo eder vælger.
\par 25 Jeg vakte ham fra Norden, og han kom, jeg kaldte ham fra Solens Opgang. Han nedtramper Fyrster som Dynd, som en Pottemager ælter sit Ler.
\par 26 Hvo forkyndte det før, så vi vidste det, forud, så vi sagde: "Han fik Ret!" Nej, ingen har forkyndt eller sagt det, ingen har hørt eders Ord.
\par 27 Først jeg har forkyndt det for Zion, sendt Jerusalem Glædesbud.
\par 28 Jeg ser mig om der er ingen, ingen af dem ved Råd, så de svarer mig på mit Spørgsmål.
\par 29 Se, alle er de intet, deres Værker Luft, deres Billeder Vind og Tomhed.

\chapter{42}

\par 1 Se min Tjener, ved hvem jeg min udvalgte, hvem jeg har kær! På ham har jeg lagt min Ånd, han skal udbrede Ret til Folkene.
\par 2 Han råber og skriger ikke, løfter ej Røsten på Gaden,
\par 3 bryder ej knækket Rør og slukker ej rygende Tande. Han udbreder Ret med Troskab,
\par 4 vansmægter, udmattes ikke, før han får sat Ret på Jorden; og fjerne Strande bier på hans Lov.
\par 5 Så siger Gud HERREN, som skabte og udspændte Himlen, udbredte Jorden med dens Grøde, gav Folkene på den Åndedræt og dem, som vandrer der, Ånde.
\par 6 Jeg, HERREN, har kaldet dig i Retfærd og grebet dig fast om Hånd; jeg vogter dig, og jeg gør dig til Folkepagt, til Hedningelys
\par 7 for at åbne de blinde Øjne og føre de fangne fra Fængslet, fra Fangehullet Mørkets Gæster.
\par 8 Jeg er HERREN, så lyder mit Navn. Jeg giver ej andre min Ære, ej Gudebilleder min Pris.
\par 9 Hvad jeg forudsagde, se, det er sket, jeg forkynder nu nye Ting, kundgør dem, før de spirer frem.
\par 10 Syng HERREN en ny Sang, hans Pris over Jorden vide; Havet og dets Fylde skal juble, fjerne Strande og de, som bebor dem;
\par 11 Ørkenen og dens Byer stemmer i, de Lejre, hvor Kedar bor; Klippeboerne jubler, råber fra Bjergenes Tinder;
\par 12 HERREN giver de Ære, forkynder hans Pris på fjerne Strande.
\par 13 HERREN drager ud som en Helt, han vækker som en Stridsmand sin Kamplyst, han udstøder Krigsskrig, han brøler, æsker sine Fjender til Strid.
\par 14 En Evighed lang har jeg tiet, været tavs og lagt Bånd på mig selv; nu skriger jeg som Kvinde i Barnsnød, stønner og snapper efter Luft.
\par 15 Jeg gør Bjerge og Høje tørre, afsvider alt deres Grønt, gør Strømme til udtørret Land, og Sumpe lægger jeg tørre.
\par 16 Jeg fører blinde ad ukendt Vej, leder dem ad ukendte Stier, gør Mørket foran dem til Lys og Bakkelandet til Slette. Det er de Ting, jeg gør, og dem går jeg ikke fra.
\par 17 Vige og dybt beskæmmes skal de, som stoler på Billeder, som siger til støbte Billeder: "I er vore Guder!"
\par 18 I, som er døve, hør, løft Blikket, I blinde, og se!
\par 19 Hvo er blind, om ikke min Tjener, og døv som Budet, jeg sendte? Hvo er blind som min håndgangne Mand, blind som HERRENs Tjener?
\par 20 Meget så han, men ænsed det ikke, trods åbne Ører hørte han ej.
\par 21 For sin Retfærds Skyld vilde HERREN løfte Loven til Højhed og Ære.
\par 22 Men Folket er plyndret og hærget; de er alle bundet i Huler, skjult i Fangers Huse, til Ran blev de, ingen redder, til Plyndring, ingen siger: "Slip dem!"
\par 23 Hvem af jer vil lytte til dette, mærke sig og fremtidig høre det:
\par 24 Hvo hengav Jakob til Plyndring, gav Israel hen til Ransmænd? Mon ikke HERREN: mod hvem vi synded, hvis Veje de ej vilde vandre, hvis Lov de ikke hørte?
\par 25 Han udgød over det Harme, sin Vrede og Krigens Vælde; den luede om det, det ænsed det ej, den sved det, det tog sig det ikke til Hjerte.

\chapter{43}

\par 1 Men nu, så siger HERREN, som skabte dig, Jakob, danned dig, Israel: Frygt ikke, jeg genløser dig, jeg kalder dig ved Navn, du er min!
\par 2 Når du går gennem Vande, er jeg med dig, gennem Strømme, de river dig ikke bort; når du går gennem Ild, skal du ikke svides, Luen brænder dig ikke.
\par 3 Thi jeg er din Gud, jeg, HERREN, Israels Hellige din Frelser.
\par 4 fordi du er dyrebar for mig, har Værd, og jeg elsker dig; jeg giver Mennesker for dig og Folkefærd for din Sjæl.
\par 5 Frygt ikke, thi j eg er med dig! Jeg bringer dit Afkom fra Østen, sanker dig sammen fra Vesten,
\par 6 siger til Norden: "Giv hid!" til Sønden: "Hold ikke tilbage! Bring mine Sønner fra det fjerne, mine Døtre fra Jordens Ende,
\par 7 enhver, der er kaldt med mit Navn, hvem jeg skabte, danned og gjorde til min Ære!"
\par 8 Før det blinde Folk frem, der har Øjne, de døve, der dog har Ører!
\par 9 Lad alle Folkene samles, lad Folkefærdene flokkes! Hvo blandt dem kan forkynde sligt eller påvise Ting, de har forudsagt? Lad dem føre Vidner og få Ret, lad dem høre og sige: "Det er sandt!"
\par 10 Mine Vidner er I, så lyder det fra HERREN, min Tjener, hvem jeg har udvalgt, at I må kende det, fro mig og indse, at jeg er den eneste. Før mig blev en Gud ej dannet, og efter mig kommer der ingen;
\par 11 jeg, jeg alene er HERREN, uden mig er der ingen Frelser.
\par 12 Jeg har forkyndt det og frelser, kundgjort det, ej fremmede hos jer; I er mine Vidner, lyder det fra HERREN. Jeg er fra Evighed Gud,
\par 13 den eneste også i Fremtiden. Ingen frier af min Hånd, jeg handler - hvo gør det ugjort?
\par 14 Så siger HERREN, eders Genløser, Israels Hellige: For eder gør jeg Opbud mod Babel og fjerner deres Fængsels Portslåer, mens Kaldæerne bindes i Halsjern.
\par 15 Jeg, HERREN, jeg er eders Hellige, Israels Skaber eders Konge.
\par 16 Så siger HERREN, som lagde en Vej i Havet, en Sti i de stride Vande,
\par 17 førte Vogne og Heste derud, Hær og Kriger tillige; de segned og rejste sig ikke, sluktes, gik ud som en Væge:
\par 18 Kom ikke det svundne i Hu, tænk ikke på Fortidens Dage!
\par 19 Thi se, nu skaber jeg nyt, alt spirer det, ser I det ikke? Gennem Ørkenen lægger jeg Vej, Floder i, det øde Land;
\par 20 de vilde Dyr skal ære mig, Sjakaler tillige med Strudse. Thi Vand vil jeg give i Ørkenen, Floder i det øde Land, for at læske mit udvalgte Folk.
\par 21 Det Folk, jeg har dannet mig, skal synge min Pris.
\par 22 Jakob, du kaldte ej på mig eller trætted dig, Israel, med mig;
\par 23 du bragte mig ej Brændofferlam, du æred mig ikke med Slagtofre; jeg plaged dig ikke for Afgrødeoffer, trætted dig ikke for Røgelse;
\par 24 du købte mig ej Kalmus for Sølv eller kvæged mig med Slagtofres Fedt. Nej, du plaged mig med dine Synder, trætted mig med din Brøde.
\par 25 Din Misgerning sletter jeg ud, jeg, jeg, for min egen Skyld, kommer ej dine Synder i Hu.
\par 26 Mind mig, lad vor Sag gå til Doms, regn op, så du kan få Ret!
\par 27 Allerede din Stamfader synded, dine Talsmænd forbrød sig imod mig,
\par 28 så jeg vanæred hellige Fyrster, gav Jakob hen til Band og Israel hen til Spot.

\chapter{44}

\par 1 Men hør nu, Jakob, min Tjener, Israel, hvem jeg har udvalgt:
\par 2 Så siger HERREN, som skabte dig og fra Moders Liv danned dig, din Hjælper: Frygt ikke, min Tjener Jakob, Jesjurun, hvem jeg har udvalgt!
\par 3 Thi jeg udgyder Vand på det tørstende, Strømme på det tørre Land, udgyder min Ånd på din Æt, min Velsignelse over dit Afkom;
\par 4 de skal spire som Græs mellem Vande, som Pile ved Bækkenes Løb.
\par 5 En skal sige: "Jeg er HERRENs", en kalde sig med Jakobs Navn, en skrive i sin Hånd: "For HERREN!" og tage sig Israels Navn.
\par 6 Så siger HERREN, Israels Konge, dets Genløser, Hærskarers HERRE: Jeg er den første og den sidste, uden mig er der ingen Gud.
\par 7 Hvo der er min Lige, træde frem, forkynde og godtgøre for mig: Hvo kundgjorde fra Urtid det kommende? De forkynde os, hvad der skal ske!
\par 8 Ræddes og ængstes ikke! Har ej længst jeg kundgjort og sagt det? I er mine Vidner: Er der Gud uden mig, er der vel anden Klippe? Jeg ved ikke nogen.
\par 9 De, der laver Gudebilleder, er alle intet, og deres kære Guder gavner intet; deres Vidner ser intet og kender intet, at de må blive til Skamme.
\par 10 Når nogen laver en Gud og støber et Billede, er det ingen Gavn til;
\par 11 se, alle dets Tilbedere bliver til Skamme; Mestrene er jo kun Mennesker lad dem samles til Hobe og træde frem, de skal alle som een forfærdes og blive til Skamme.
\par 12 En smeder Jern til en Økse og arbejder ved Kulild, tildanner den med Hamre og gør den færdig med sin stærke Arm; får han ikke Mad, afkræftes han, og får han ikke Vand at drikke, bliver han træt.
\par 13 Så fælder ban Træer, udspænder Målesnoren, tegner Billedet med Gravstikken, skærer det ud med Kniven og sætter det af med Cirkelen; han laver det efter en Mands Skikkelse, efter menneskelig Skønhed, til at stå i et Hus.
\par 14 Han fældede sig Cedre, tog Elm og Eg og arbejdede af al sin Kraft på Skovens Træer, som Gud havde plantet og Regnen givet Vækst.
\par 15 Det tjener et Menneske til Brændsel, han tager det og varmer sig derved; han sætter Ild i det og bager Brød - og desuden laver han en Gud deraf og tilbeder den, han gør et Gudebillede deraf og knæler for det.
\par 16 Halvdelen brænder han i Ilden, og over Gløderne steger han Kød; han spiser Stegen og mættes; og han varmer sig derved og siger: "Ah, jeg bliver varm, jeg mærker Ilden" -
\par 17 og af Resten laver han en Gud, et Billede; han knæler for det, kaster sig ned og beder til det og siger: "Frels mig, thi du er min Gud!"
\par 18 De skønner ikke, de fatter ikke, thi deres Øjne er lukket, så de ikke ser, og deres Hjerter, så de ikke skønner.
\par 19 De tænker ikke over det, de har ikke Indsigt og Forstand til at sige sig selv: "Halvdelen brændte jeg i Bålet, over Gløderne bagte jeg Brød, stegte Kød og spiste; skulde jeg da af Resten gøre en Vederstyggelighed? Skulde jeg knæle for en Træklods?"
\par 20 Den, som tillægger Aske Værd, ham har et vildfarende Hjerte dåret; han redder ikke sin Sjæl, så han siger: "Er det ikke en Løgn, jeg har i min højre Hånd?"
\par 21 Jakob, kom dette i Hu, Israel, thi du er min Tjener! Jeg skabte dig, du er min Tjener, ej skal du glemmes, Israel;
\par 22 jeg sletted som Tåge din Misgerning og som en Sky dine Synder.
\par 23 Jubler, I Himle, thi HERREN greb ind, fryd jer, I Jordens Dybder, bryd ud i Jubel, I Bjerge, Skoven og alle dens Træer, thi HERREN genløser Jakob og herliggør sig ved Israel.
\par 24 Så siger HERREN, din Genløser, som danned dig fra Moders Liv: Jeg er HERREN, som skabte alt, som ene udspændte Himlen, udbredte Jorden, hvo hjalp mig?
\par 25 som tilintetgør Løgnernes Tegn og gør Spåmænd til Dårer, som tvinger de vise tilbage, beskæmmer deres Lærdom,
\par 26 stadfæster sine Tjeneres Ord, fuldbyrder sine Sendebuds Råd, Jeg siger om Jerusalem: "Det skal bebos!" om Judas Byer: "de skal bygges!" Ruinerne rejser jeg atter!
\par 27 Jeg siger til Dybet: "Bliv tørt, dine Floder gør jeg tørre!"
\par 28 Jeg siger om Kyros: "Min Hyrde, som fuldbyrder al min Vilje!" Jeg siger om Jerusalem: "Det skal bygges!" om Templet: "Det skal grundes "

\chapter{45}

\par 1 Så siger HERREN til sin Salvede, til Kyros, hvis højre jeg greb for at nedstyrte Folk for hans Ansigt og løsne Kongernes Gjord, for at åbne Dørene for ham, så Portene ikke var stængt:
\par 2 Selv går jeg frem foran dig, Hindringer jævner jeg ud; jeg sprænger Porte af Kobber og sønderhugger Slåer af Jern.
\par 3 Jeg giver dig Mulmets Skatte, Rigdomme gemt i Løn, så du kender, at den, der kaldte dig ved Navn, er mig, er HERREN, Israels Gud.
\par 4 For Jakobs, min Tjeners, Skyld, for min udvalgtes, Israels, Skyld kalder jeg dig ved dit Navn, ved et Æresnavn, skønt du ej kender mig.
\par 5 HERREN er jeg, ellers ingen, uden mig er der ingen Gud; jeg omgjorder dig, endskønt du ej kender mig,
\par 6 så de kender fra Solens Opgang til dens Nedgang: der er ingen uden mig. HERREN er jeg, ellers ingen,
\par 7 Lysets Ophav og Mørkets Skaber, Velfærds Kilde og Ulykkes Skaber: Jeg er HERREN, der virker alt.
\par 8 Lad regne, I Himle deroppe, nedsend Retfærd, I Skyer, Jorden åbne sit Skød, så Frelse må spire frem og Retfærd vokse tillige.
\par 9 Ve den, der trættes med sit Ophav, et Skår kun blandt Skår af Jord! Siger Ler til Pottemager: "Hvad kan du lave?" hans Værk: "Du har ikke Hænder!"
\par 10 Ve den, der siger til sin Fader: "Hvad kan du avle?" til sin Moder: "Hvad kan du føde?"
\par 11 Så siger HERREN, Israels Hellige, Fremtidens Ophav: I spørger mig om mine Børn, for mine Hænders Værk vil I råde!
\par 12 Det var mig, som dannede Jorden og skabte Mennesket på den; mine Hænder udspændte Himlen, jeg opbød al dens Hær;
\par 13 det var mig, som vakte ham i Retfærd, jeg jævner alle hans Veje; han skal bygge min By og give mine bortførte fri ikke for Løn eller Gave, siger Hærskarers HERRE.
\par 14 Så siger HERREN: Ægyptens Løn, Ætiopiens Vinding, Sebæernes granvoksne Mænd, de skal komme og tilhøre dig, og dig skal de følge; de skal komme i Lænker og kaste sig ned for dig og bønfalde dig: "Kun hos dig er Gud, der er ingen anden Gud."
\par 15 Sandelig, du er en Gud, som er skjult, Israels Gud er en Frelser!
\par 16 Skam og Skændsel bliver alle hans Fjender til Del, til Hobe går Gudemagerne om med Skændsel.
\par 17 Israel frelses ved HERREN, en evig Frelse, i Evighed bliver I ikke til Skam og Skændsel.
\par 18 Thi så siger HERREN, Himlens Skaber, han, som er Gud, som dannede Jorden, frembragte, grundfæsted den, ej skabte den øde, men danned den til at bebos: HERREN er jeg, ellers ingen.
\par 19 Jeg talede ikke i Løndom, i Mørkets Land, sagde ikke til Jakobs Æt: "Søg mig forgæves!" Jeg, HERREN, taler hvad ret er, forkynder, hvad sandt er.
\par 20 Kom samlede hid, træd frem til Hobe, I Folkenes undslupne! Uvidende er de, som bærer et Billede af Træ, de, som beder til en Gud, der ikke kan frelse.
\par 21 Forkynd det, kom frem dermed, lad dem rådslå sammen: Hvo kundgjorde dette tilforn, forkyndte det forud? Mon ikke jeg, som er HERREN? Uden mig er der ingen Gud, uden mig er der ingen retfærdig, frelsende Gud.
\par 22 Vend dig til mig og bliv frelst, du vide Jord, thi Gud er jeg, ellers ingen;
\par 23 jeg svor ved mig selv, fra min Mund kom Sandhed, mit Ord vender ikke tilbage: Hvert Knæ skal bøjes for mig, hver Tunge sværge mig til.
\par 24 "Kun hos HERREN," skal man sige, "er Retfærd og Styrke; til ham skal alle hans Avindsmænd komme med Skam."
\par 25 Ved HERREN når al Israels Æt til sin Ret og jubler.

\chapter{46}

\par 1 I Knæ er Bel, og Nebo er bøjet, deres billeder gives til Dyr og Fæ, de læsses som byrde på trætte Dyr
\par 2 De bøjes, i Knæ er de alle, de kan ikke frelse Byrden, og selv må de vandre i Fangenskab.
\par 3 Hør mig, du Jakobs Hus, al Resten af Israels Hus, løftet fra Moders Liv, båret fra Moders Skød.
\par 4 Til Alderdommen er jeg den samme, jeg bærer jer, til Hårene gråner; ret som jeg bar, vil jeg bære, jeg, jeg vil bære og redde.
\par 5 Med hvem vil I jævnstille ligne mig, hvem vil I gøre til min Lige?
\par 6 De øser Guld af Pung, Sølv får de vejet på Vægt, de lejer en Guldsmed, som gør det til en Gud, de bøjer sig, kaster sig ned;
\par 7 de løfter den på Skuldren og bærer den, sætter den på Plads, og den står, den rører sig ikke af Stedet råber de til den, svarer den ikke, den frelser dem ikke i Nød.
\par 8 Kom dette i Hu, lad jer råde, I frafaldne, læg jer det på Sinde!
\par 9 Kom i Hu, hvad er forudsagt før, thi Gud er jeg, ellers ingen, ja Gud, der er ingen som jeg,
\par 10 der forud forkyndte Enden, tilforn, hvad der ikke var sket, som sagde: "Mit Råd står fast, jeg fuldbyrder al min Vilje,"
\par 11 som fra Øst kalder Ørnen hid, fra det fjerne mit Råds Fuldbyrder. Jeg taled og lader det ske, udtænkte og fuldbyrder det.
\par 12 Hør på mig, I modløse, som tror, at Retten er fjern:
\par 13 Jeg bringer min Ret, den er ej fjern, min Frelse tøver ikke; jeg giver Frelse på Zion, min Herlighed giver jeg Israel.

\chapter{47}

\par 1 Stig ned, sid i Støvet, du Jomfru, Babels Datter, sid uden Trone på Jorden, Kaldæernes Datter! Thi ikke mer skal du kaldes den fine, forvænte!
\par 2 Tag fat på Kværnen, mal Mel, læg Sløret bort, løft Slæbet, blot dine Ben og vad over Strømmen!
\par 3 Din Blusel skal blottes, din Skam skal ses. Hævn tager jeg uden Skånsel, siger vor Genløser,
\par 4 hvis Navn er Hærskarers HERRE, Israels Hellige.
\par 5 Sid tavs og gå ind i Mørke, Kaldæernes Datter, thi ikke mer skal du kaldes Rigernes Dronning!
\par 6 Jeg vrededes på mit Folk, vanæred min Arv, gav dem hen i din Hånd; du viste dem ingen Medynk, du lagde dit tunge Åg på Oldingens Nakke.
\par 7 Du sagde: "Jeg bliver evindelig Evigheds Dronning." Du tog dig det ikke til Hjerte, brød dig ikke om Enden.
\par 8 Så hør nu, du yppige, du, som sidder i Tryghed, som siger i Hjertet: "Kun jeg, og ellers ingen! Aldrig skal jeg sidde Enke, ej kende til Barnløshed."
\par 9 Begge Dele skal ramme dig brat samme Dag, Barnløshed og Enkestand ramme dig i fuldeste Mål, dine mange Trylleord, din megen Trolddom til Trods,
\par 10 skønt du tryg i din Ondskab sagde: "ingen ser mig." Din Visdom og Viden var det, der ledte dig vild, så du sagde i Hjertet: "Kun jeg, og ellers ingen!"
\par 11 Dig rammer et Onde, du ikke kan købe bort, over dig falder et Vanheld, du ikke kan sone, Undergang rammer dig brat, når mindst du aner det.
\par 12 Kom med din Trolddom og med dine mange Trylleord, med hvilke du umaged dig fra din Ungdom, om du kan bøde derpå og skræmme det bort.
\par 13 Med Rådgiverhoben sled du dig træt, lad dem møde, lad Himmelgranskerne frelse dig, Stjernekigerne, som Måned for Måned kundgør, hvad dig skal ske!
\par 14 Se, de er blevet som Strå, de,fortæres af Ild, de frelser ikke deres Liv fra Luens Magt. "Ingen Glød til Varme, ej Bål at sidde ved!"
\par 15 Sligt får du af dem, du umaged dig med, dine Troldmænd fra Ungdommen af; de raver hver til sin Side, dig frelser ingen.

\chapter{48}

\par 1 Hør dette, du Jakobs Hus, I, som kaldes med Israels Navn og er rundet af Judas Kilde, som sværger ved HERRENs Navn og priser Israels Gud - dog ikke redeligt og sandt -
\par 2 fra den hellige By har de jo Navn, deres Støtte er Israels Gud, hvis Navn er Hærskarers HERRE:
\par 3 Jeg forudsagde det, som er sket, af min Mund gik det ud, så det hørtes, brat greb jeg ind, og det indtraf.
\par 4 Thi stivsindet er du, det ved jeg, din Nakke et Jernbånd, din Pande af Kobber.
\par 5 Jeg sagde det forud til dig, kundgjorde det, førend det indtraf, at du ikke skulde sige: "Det gjorde mit Billede, mit skårne og støbte bød det."
\par 6 Du hørte det, se det nu alt! Og vil I mon ikke stå ved det? Fra nu af kundgør jeg nyt, skjulte Ting, du ej kender;
\par 7 nu skabes det, ikke før,før i Dag har I ikke hørt det, at du ikke skulde sige: "Jeg vidste det."
\par 8 Hverken har du hørt eller vidst det, det kom dig ej før for Øre. Thi jeg ved, du er gennemtroløs, fra Moders Liv hed du "Frafalden";
\par 9 for mit Navns Skyld holder jeg Vreden hen, for min Ære vil jeg skåne, ej udrydde dig.
\par 10 Se, jeg smelted dig Sølv blev det ikke prøved dig i Lidelsens Ovn.
\par 11 For min egen Skyld griber jeg ind; thi hvor krænkes dog ikke mit Navn! Jeg giver ej andre min Ære.
\par 12 Hør mig dog nu, o Jakob, Israel, du, som jeg kaldte: Mig er det, jeg er den første, også jeg er den sidste.
\par 13 Min Hånd har grundlagt Jorden, min højre udspændt Himlen; så såre jeg kalder på dem, møder de alle frem.
\par 14 Samler jer alle og hør: Hvem af dem forkyndte mon dette? Min Ven fuldbyrder min Vilje på Babel og Kaldæernes Æt.
\par 15 Jeg, jeg har talet og kaldt ham, fik ham frem, hans Vej lod jeg lykkes.
\par 16 Kom hid til mig og hør: Jeg taled ej fra først i Løndom, jeg var der, så snart det skete. Og nu har den Herre HERREN sendt mig med sin Ånd.
\par 17 Så siger HERREN, din Genløser, Israels Hellige: Jeg er HERREN, din Gud, som lærer dig, hvad der båder, leder dig ad Vejen, du skal gå.
\par 18 Ak, lytted du til mine Bud! Da blev din Fred som Floden, din Retfærd som Havets Bølger,
\par 19 da blev dit Afkom som Sandet, din Livsfrugt talløs som Sandskorn; dit Navn skulde ej slettes ud og ej lægges øde for mit Åsyn.
\par 20 Gå ud af Babel, fly fra Kaldæa, kundgør, forkynd det med jublende Røst, udspred det lige til Jordens Ende, sig: "HERREN har genløst Jakob, sin Tjener,
\par 21 lod dem gå gennem Ørk, de tørstede ikke, lod Vand vælde frem af Klippen til dem, kløvede Klippen, så Vand strømmed ud."
\par 22 De gudløse har ingen Fred, siger HERREN.

\chapter{49}

\par 1 Hør mig, I fjerne Strande, lyt til, I Folk i det fjerne! HERREN har fra Moders Liv kaldt mig, fra Moders Skød nævnet mit Navn;
\par 2 til et skarpt Sværd gjorde han min Mund og skjulte mig i Skyggen af sin Hånd, til en sleben Pil har han gjort mig og gemt mig i sit Kogger,
\par 3 sagt til mig: "Du er min Tjener, Israel, ved hvem jeg vinder Ære."
\par 4 Jeg sagde: "Min Møje er spildt, på Tomhed og Vind sled jeg mig op dog er min Ret hos HERREN, min Løn er hos min Gud."
\par 5 Og nu siger HERREN, som danned mig fra Moders Liv til sin Tjener for at hjemføre Jakob til ham og samle Israel til ham og i HERRENs Øjne er jeg æret, min Gud er blevet min Styrke -
\par 6 han siger: "For lidt for dig som min Tjener at rejse jakobs Stammer og hjemføre Israels frelste! Jeg gør dig til Hedningers Lys, at min Frelse må nå til Jordens Ende."
\par 7 Så siger HERREN, Israels Genløser, dets Hellige, til den dybt foragtede, skyet af Folk, Herskernes Træl: Konger skal se det og rejse sig, Fyrster skal kaste sig ned for HERRENs Skyld, den trofaste, Israels Hellige, der udvælger dig.
\par 8 Så siger HERREN: Jeg hører dig i Nådens Stund, jeg hjælper dig på Frelsens Dag, vogter dig og gør dig til Folkepagt for at rejse Landet igen, udskifte øde Lodder
\par 9 og sige til de bundne: "Gå ud!" til dem i Mørket: "Kom frem!" Græs skal de finde langs Vejene, Græsgang på hver nøgen Høj;
\par 10 de hungrer og tørster ikke, dem stikker ej Hede og Sol. Thi deres Forbarmer fører dem, leder dem til Kildevæld;
\par 11 jeg gør alle Bjerge til Vej, og alle Stier skal højnes.
\par 12 Se, nogle kommer langvejsfra, nogle fra Nord og Vest, nogle fra Sinims Land.
\par 13 Jubler, I Himle, fryd dig, du Jord, I Bjerge, bryd ud i Jubel! Thi HERREN trøster sit Folk, forbarmer sig over sine arme.
\par 14 Dog siger Zion: "HERREN har svigtet mig, Herren har glemt mig!"
\par 15 Glemmer en Kvinde sit diende Barn, en Moder, hvad hun bar under Hjerte? Ja, selv om de kunde glemme, jeg glemmer ej dig.
\par 16 Se, i mine Hænder har jeg tegnet dig, dine Mure har jeg altid for Øje.
\par 17 Dine Børn kommer ilende; de, som nedbrød og lagde dig øde, går bort.
\par 18 Løft Øjnene, se dig om, de samles, kommer alle til dig. Så sandt jeg lever, lyder det fra HERREN: Du skal bære dem alle som Smykke, binde dem som Bruden sit Bælte.
\par 19 Thi dine Tomter og Grusdynger, dit hærgede Land ja, nu er du Beboerne for trang; de, som åd dig, er borte;
\par 20 end skal du høre dem sige, din Barnløsheds Børn: "Her er for trangt, så flyt dig, at jeg kan sidde!"
\par 21 Da tænker du i dit Hjerte: "Hvo fødte mig dem? Jeg var jo barnløs og gold, landflygtig og bortstødt, hvo fostrede dem? Ene sad jeg tilbage, hvor kommer de fra?"
\par 22 Så siger den Herre HERREN: Se, jeg løfter min Hånd for Folkene, rejser mit Banner for Folkeslag, og de bringer dine Sønner i Favnen, dine Døtre bæres på Skulder.
\par 23 Konger bliver Fosterfædre for dig, deres Dronninger skal være dine Ammer. De kaster sig på Ansigtet for dig, slikker dine Fødders Støv. Du skal kende, at jeg er HERREN; de, som bier på mig, bliver ikke til Skamme.
\par 24 Kan Bytte fratages en Helt, kan den stærkes Fanger slippe bort?
\par 25 Thi så siger HERREN: Om Fanger end fratages Helten, slipper Bytte end bort fra den stærke, jeg strider mod dem, der strider mod dig, og dine Børn vil jeg frelse.
\par 26 Dem, der trænger dig, lader jeg æde deres eget Kød, deres Blod skal de drikke som Most; REN, er din Frelser, din Genløser Jakobs Vældige.

\chapter{50}

\par 1 Så siger HERREN: Hvor er eders Moders Skilsmissebrev, med hvilket jeg sendte hende bort; eller hvem var jeg noget skyldig, så jeg solgte eder til ham? Nej, for eders Brøde solgtes I, for eders Synd blev eders Moder sendt bort.
\par 2 Hvi var der da ingen, da jeg kom, hvi svarede ingen, da jeg kaldte? Er min Hånd for kort til af udfri, har jeg ingen Kraft til at redde? Ved min Trussel udtørrer jeg Havet, Strømme gør jeg til Ørk, så Fiskene rådner af Mangel på Vand og dør af Tørst;
\par 3 jeg klæder Himlen i sort og hyller den ind i Sæk.
\par 4 Den Herre HERREN gav mig Lærlinges Tunge, at jeg skulde vide at styrke de trætte med Ord; han vækker hver Morgen mit Øre, han vækker det til at høre, som Lærlinge hører.
\par 5 Den Herre HERREN åbnede mit Øre, og jeg stred ikke imod, jeg unddrog mig ikke;
\par 6 min Ryg bød jeg frem til Hug, mit Skæg til at rives, mit Ansigt skjulte jeg ikke for Hån og Spyt.
\par 7 Mig hjælper den Herre HERREN, så jeg ikke beskæmmes; jeg gør derfor mit Ansigt som Sten og ved, jeg bliver ikke til Skamme.
\par 8 Min Retfærdiggører er nær; lad os mødes, om nogen vil Strid; om nogen vil trætte med mig, så træde han hid!
\par 9 Se, den Herre HERREN hjælper mig, hvo vil fordømme mig? Se, alle slides op som en Klædning, Møl fortærer dem.
\par 10 Frygter nogen af jer HERREN, han lytte til hans Tjener, enhver, som vandrer i Mørke og uden Lys; han stole på HERRENs Navn, søge Støtte hos sin Gud!
\par 11 Alle I, som optænder Ild og sætter Pile i Brand, gå ind i eders brændende Ild, i Pilene, I tændte! Fra min Hånd skal det ramme eder, i Kval skal I ligge.

\chapter{51}

\par 1 Hør mig, I, som jager efter Retfærd, som søger HERREN! Se til Klippen, I huggedes af, til Gruben, af hvilken I brødes,
\par 2 se til eders Fader Abraham, til Sara, der fødte eder: Da jeg kaldte ham, var han kun een, jeg velsigned ham, gjorde ham til mange.
\par 3 Thi HERREN trøster Zion, trøster alle dets Tomter, han gør dets Ørk som Eden, dets Ødemark som HERRENs Have; der skal findes Fryd og Glæde, Lovsang og Strengespil.
\par 4 I Folkeslag, lyt til mig, I Folkefærd, lån mig Øre! Thi Lov går ud fra mig, min Ret som Folkeslags Lys;
\par 5 min Retfærd nærmer sig hastigt, min Frelse oprinder, mine Arme bringer Folkeslag Ret; fjerne Strande bier på mig og længes efter min Arm.
\par 6 Løft eders Øjne mod Himlen og se på Jorden hernede! Thi Himlen skal svinde som Røg, Jorden som en opslidt Klædning, dens Beboere skal dø som Myg. Men min Frelse varer evigt, min Retfærd ophører aldrig.
\par 7 Hør mig, I, som kender Retfærd, du Folk med min Lov i dit Hjerte, frygt ej Menneskers Hån, vær ikke ræd for deres Spot!
\par 8 Som en Klædning skal Møl fortære dem,,Orm fortære dem som Uld, men min Retfærd varer evigt, min Frelse fra Slægt til Slægt.
\par 9 Vågn op, vågn op, HERRENs Arm, og ifør dig Styrke, vågn op som i henfarne Dage, i Urtidens Slægter! Mon du ej kløvede Rahab, gennembored Dragen,
\par 10 mon du ej udtørred Havet, Stordybets Vande, gjorde Havets dyb til en Vej, hvor de genløste gik?
\par 11 HERRENs forløste vender hjem, de drager til Zion med Jubel med evig Glæde om Issen; Fryd og Glæde får de, Sorg og Suk skal fly.
\par 12 Jeg, jeg er eders Trøster, hvem er da du, at du frygter dødelige, jordiske Mennesker, der bliver som Græs,
\par 13 at du glemmer HERREN, din Skaber, der udspændte Himlen og grundfæsted Jorden, at du altid Dagen lang frygter for Undertrykkerens Vrede. Så snart han vil til at lægge øde, hvor er da Undertrykkerens Vrede?
\par 14 Snart skal den krumsluttede løses og ikke dø og synke i Graven eller mangle Brød,
\par 15 så sandt jeg er HERREN din Gud, som rører Havet, så Bølgerne bruser, den, hvis Navn er Hærskarers HERRE.
\par 16 jeg lægger mine Ord i din Mund og gemmer dig under min Hånds Skygge for at udspænde Himmelen og grundfæste Jorden og sige til Zion: "Du er mit Folk."
\par 17 Vågn op, vågn op, stå op, Jerusalem, som af HERRENs Hånd fik rakt hans Vredes Bæger og tømte den berusende Kalk til sidste Dråbe.
\par 18 Af alle de Børn, hun fødte, ledte hende ingen, af alle de Børn, hun fostred, greb ingen hendes Hånd.
\par 19 To Ting timedes dig hvo ynker dig vel? Vold og Våde, Hunger og Sværd - hvo trøster dig?
\par 20 Ved alle Gadehjørner lå dine Sønner i Afmagt som i Garn Antiloper, fyldte med HERRENs Vrede, med Trusler fra din Gud.
\par 21 Hør derfor, du arme, drukken, men ikke af Vin:
\par 22 Så siger din Herre, HERREN, din Gud, der strider for sit Folk: Se, jeg tager den berusende Kalk fra din Hånd, aldrig mer skal du drikke min Vredes Bæger;
\par 23 og jeg rækker det til dine Plagere, dem, som bød dig: "Bøj dig, så vi kan gå over!" og du gjorde din Ryg til Gulv, til Gade for Vandringsmænd.

\chapter{52}

\par 1 Vågn op, vågn op, ifør dig tag dit Højtidsskrud på, Jerusalem, hellige By! Thi uomskårne, urene Folk skal ej mer komme ind.
\par 2 Ryst Støvet af dig, stå op, tag Sæde, Jerusalem, fri dig for Halslænken, Zions fangne Datter!
\par 3 Thi så siger HERREN: For intet solgtes I, og uden Sølv skal I løskøbes.
\par 4 Thi så siger den Herre HERREN: I Begyndelsen drog mit Folk ned til Ægypten for at bo der som fremmed, og siden undertrykte Assyrien det uden Vederlag.
\par 5 Og nu? Hvad har jeg at gøre her? lyder det fra HERREN; mit Folk er jo ranet for intet. De, der hersker over det, brovter, lyder det fra HERREN, og mit Navn vanæres ustandseligt Dagen lang.
\par 6 Derfor skal mit Folk kende mit Navn på hin Dag, at det er mig, som har talet, ja mig.
\par 7 Hvor liflige er på Bjergene Glædesbudets Fodtrin, han, som udråber Fred, bringer gode Tidender, udråber Frelse, som siger til Zion: "Din Gud har vist, han er Konge."
\par 8 Hør, dine Vægtere råber, de jubler til Hobe, thi de ser for deres Øjne HERREN vende hjem til Zion.
\par 9 Bryd ud til Hobe i Jubel, Jerusalems Tomter! Thi HERREN trøster sit Folk, genløser Jerusalem.
\par 10 Han blotter sin hellige Arm for al Folkenes Øjne, den vide Jord skal skue Frelsen fra vor Gud.
\par 11 Bort, bort, drag ud derfra, rør ej noget urent, bort, tvæt jer, I, som bærer HERRENs Kar!
\par 12 Thi i Hast skal I ej drage ud, I skal ikke flygte; nej, foran eder går HERREN, eders Tog slutter Israels Gud.
\par 13 Se, min Tjener får Fremgang han stiger, løftes og ophøjes såre.
\par 14 Som mange blev målløse over ham, så umenneskelig ussel så han ud, han ligned ej Menneskenes Børn
\par 15 skal Folk i Mængde undres, Konger blive stumme over ham; thi hvad ikke var sagt dem, ser de, de skuer, hvad de ikke havde hørt.

\chapter{53}

\par 1 Hvo troede det, vi hørte, for hvem åbenbaredes HERRENs Arm?
\par 2 Han skød op som en Kvist for hans Åsyn, som et Rodskud af udtørret Jord, uden Skønhed og Pragt til at drage vort Blik, uden Ydre, så vi syntes om ham,
\par 3 ringeagtet, skyet af Folk, en Smerternes Mand og kendt med Sygdom, en, man skjuler sit Ansigt for, agtet ringe, vi regned ham ikke.
\par 4 Og dog vore Sygdomme bar han, tog vore Smerter på sig; vi regnede ham for plaget, slagen, gjort elendig af Gud.
\par 5 Men han blev såret for vore Overtrædelser, knust for vor Brødres Skyld; os til Fred kom Straf over ham, vi fik Lægedom ved hans Sår.
\par 6 Vi for alle vild som Får, vi vendte os hver sin Vej, men HERREN lod falde på ham den Skyld, der lå på os alle.
\par 7 Han blev knust og bar det stille, han oplod ikke sin Mund som et Lam, der føres hen at slagtes, som et Får, der er stumt, når det klippes. han oplod ikke sin Mund.
\par 8 Fra Trængsel og Dom blev han taget, men hvem i hans Samtid tænkte, da han reves fra de levendes Land, at han ramtes for mit Folks Overtrædelse?
\par 9 Hos gudløse gav man ham Grav og Gravkammer hos den rige, endskønt han ej gjorde Uret, og der ikke var Svig i hans Mund.
\par 10 Det var HERRENs Vilje at slå ham med Sygdom; når hans Sjæl havde fuldbragt et Skyldoffer, skulde han se Afkom, leve længe og HERRENs Vilje lykkes ved hans Hånd.
\par 11 Fordi hans Sjæl har haft Møje, skal han se det, hvorved han skal mættes. Når han kendes, skal min retfærdige Tjener retfærdiggøre de mange, han, som bar deres Overtrædelser.
\par 12 Derfor arver han mange, med mægtige deler han Bytte, fordi han udtømte sin Sjæl til Døden og regnedes blandt Overtrædere; dog bar han manges Synd, og for Overtrædere bad han.

\chapter{54}

\par 1 Jubl, du golde, der ej fødte, jubl og fryd dig, du uden Veer! Thi den forladtes Børn er flere end Hustruens Børn, siger HERREN.
\par 2 Vid Rummet ud i dit Telt, spar ikke, men udspænd din Boligs Tæpper. Du må gøre dine Teltreb lange og slå dine Teltpæle fast.
\par 3 Thi du skal brede dig til højre og, venstre, dit Afkom tage Folk i Eje og bo i de øde Byer.
\par 4 Frygt ej, du skal ikke beskæmmes, vær ej ræd, du skal ikke skuffes! Thi din Ungdoms Skam skal du glemme, ej mindes din Enkestands Skændsel.
\par 5 Thi din Ægtemand er din Skaber, hans Navn er Hærskarers HERRE, din Genløser er Israels Hellige, han kaldes al Jordens Gud.
\par 6 Som en Hustru, der sidder forladt med Sorg i Sinde, har HERREN kaldt dig. En Ungdomsviv, kan hun forstødes? siger din Gud.
\par 7 Jeg forlod dig et lidet Øjeblik, men favner dig i stor Barmhjertighed;
\par 8 jeg skjulte i skummende Vrede et Øjeblik mit Åsyn for dig, men forbarmer mig med evig Kærlighed, siger din Genløser, HERREN.
\par 9 Mig er det som Noas Dage: Som jeg svor, at Noas Vande ej mer skulde oversvømme Jorden, så sværger jeg nu at jeg aldrig vil vredes og skænde på dig.
\par 10 Om også Bjergene viger, om også Højene rokkes, min Kærlighed viger ej fra dig, min Fredspagt rokkes ikke, så siger HERREN, din Forbarmer.
\par 11 Du arme, forblæste, utrøstede! Se, jeg bygger dig op med Smaragder, lægger din Grund med Safirer,
\par 12 af Rubiner sætter jeg Tinderne, og Portene gør jeg af Karfunkler, af Ædelsten hele din Ringmur.
\par 13 Alle dine Børn bliver oplært af HERREN, og stor bliver Børnenes Fred;
\par 14 i Retfærd skal du grundfæstes. Vær tryg for Vold, du har intet at frygte, for Rædsler, de kommer dig ikke nær!
\par 15 Angribes du, er det uden min Vilje, falde skal hver, som angriber dig;
\par 16 det er mig, der skaber Smeden, som blæser en Kulild op og tilvirker Våben ved sin Kunst, men Ødelæggeren skaber jeg også.
\par 17 Intet Våben, der smedes mod dig, skal du, hver Tunge, der trætter med dig, får du dømt. Dette er HERRENs Tjeneres Lod, den Retfærd, jeg giver dem, lyder det fra HERREN.

\chapter{55}

\par 1 Hid, alle, som tørster, her er Vand, kom, I, som ikke har Penge! Køb Korn og spis uden Penge, uden Vederlag Vin og Mælk.
\par 2 Hvi giver I Sølv for, hvad ikke er Brød, eders Dagløn for, hvad ej mætter? Hør mig, så får I, hvad godt er, at spise, eders Sjæl skal svælge i Fedt;
\par 3 bøj eders Øre, kom til mig, hør, og eders Sjæl skal leve! Så slutter jeg med jer en evig Pagt: de trofaste Nådeløfter til David.
\par 4 Se, jeg gjorde ham til Vidne for Folkeslag, til Folkefærds Fyrste og Hersker.
\par 5 Se, på Folk, du ej kender, skal, du kalde, til dig skal Folk, som ej kender dig, ile for HERREN din Guds Skyld, Israels Hellige, han gør dig herlig.
\par 6 Søg HERREN, medens han findes, kald på ham, den Stund han er nær!
\par 7 Den gudløse forlade sin Vej, Urettens Mand sine Tanker og vende sig til HERREN, at han må forbarme sig, til vor Gud, thi han er rund til at forlade.
\par 8 Thi mine Tanker er ej eders, og eders Veje ej mine, lyder det fra HERREN;
\par 9 nej, som Himlen er højere end Jorden, er mine Veje højere end eders og mine Tanker højere end eders.
\par 10 Thi som Regnen og Sneen falder fra Himlen og ikke vender tilbage, før den har kvæget Jorden, gjort den frugtbar og fyldt den med Spirer, givet Sæd til at så og Brød til at spise,
\par 11 så skal det gå med mit Ord, det, som går ud af min Mund: det skal ej vende tomt tilbage, men udføre, hvad mig behager, og fuldbyrde Hvervet, jeg gav det.
\par 12 Ja, med Glæde skal I drage ud; og i Fred skal I ledes frem; foran jer råber Bjerge og Høje med Fryd, alle Markens Træer skal klappe i Hånd;
\par 13 i Stedet for Tjørnekrat vokser Cypresser, i Stedet for Tidsler Myrter et Æresminde for HERREN, et evigt, uudsletteligt Tegn.

\chapter{56}

\par 1 Så siger HERREN: Tag vare på Ret og øv Retfærd! Thi min Frelses komme er nær, min Ret skal snart åbenbares.
\par 2 Salig er den; der gør så, det Menneske, som fastholder dette: holder Sabbatten hellig og varer sin Hånd fra at øve noget ondt.
\par 3 Ej sige den fremmede, som slutter sig til HERREN: "HERREN vil skille mig ud fra sit Folk!" Og Gildingen sige ikke: "Se, jeg er et udgået Træ!"
\par 4 Thi så siger HERREN: Gildinger, som holder mine Sabbatter, vælger, hvad jeg har Behag i, og holder fast ved min Pagt,
\par 5 dem vil jeg give i mit Hus, på mine Mure et Minde, et Navn, der er bedre end Sønner og Døtre; jeg giver dem et evigt Navn, et Navn, der ikke skal slettes.
\par 6 Og de fremmede, som slutter sig til HERREN for at tjene ham og elske hans Navn, for at være hans Tjenere, alle, som helligholder Sabbatten og holder fast ved min Pagt,
\par 7 vil jeg bringe til mit hellige Bjerg og glæde i mit Bedehus; deres Brændofre og deres Slagtofre bliver til Behag på mit Alter; thi mit Hus skal kaldes et Bedehus for alle Folk.
\par 8 Det lyder fra den Herre HERREN: Når jeg samler Israels bortstødte,,samler jeg andre dertil, til dets egen samlede Flok.
\par 9 Alle I Markens Dyr, kom hid og æd, alle I Dyr i Skoven!
\par 10 Blinde er alle dets Vogtere, intet ved de, alle er stumme Hunde, som ikke kan gø, de ligger og drømmer, de elsker Søvn;
\par 11 grådige er de Hunde, kender ikke til Mæthed. Og sådanne Folk er Hyrder! De skønner intet, de vender sig hver sin Vej, hver søger sin Fordel:
\par 12 Kom, så henter jeg Vin, vi drikker af Mosten; som i Dag skal det være i Morgen, ovenud herligt!"

\chapter{57}

\par 1 Den retfærdige omkom, og ingen tænkte derover, fromme Mænd reves bort, det ænsede ingen; thi bort blev den retfærdige revet for Ondskabens Skyld
\par 2 og gik ind til Fred; på Gravlejet hviler nu de, som vandrede ret.
\par 3 Men I, kom nu her, I Troldkvindens Børn, I, Horkarls og Skøges Yngel:
\par 4 Hvem er det, I spotter? Hvem vrænger I Mund, hvem rækker I Tunge ad? Er I ej Syndens Børn og Løgnens Yngel?
\par 5 I, som er i Brynde ved Ege, under hvert grønt Træ, I, som slagter Børn i Dale. i Klippernes Kløfter!
\par 6 Dalenes glatte Sten er din Del og din Lod, du udgyder Drikofre for dem, bringer dem Gaver. Skal jeg være tilfreds med sligt?
\par 7 På høje og knejsende Bjerge redte du Leje, også der steg du op for at slagte dit Offer;
\par 8 bag Døren og Dørens Stolpe satte du dit Tegn; thi du sveg mig og blotted dig, steg op, gjorde lejet bredt, købte Samlejets Elskov af dem, så deres Skam;
\par 9 du salved dig med Olie for Molok og ødsled med Røgelse, sendte dine Bud til det fjerne, steg ned til Dødsriget,
\par 10 du blev træt af den lange Vej, men opgav ej Ævred; du samlede atter din Kraft og gav ikke op.
\par 11 For hvem var du ræd og angst? Thi på Løgn var du inde, og mig kom du ikke i Hu, brød dig ikke om mig. Jeg er jo stum og blind, mig frygted du ikke.
\par 12 Ja, jeg vil forkynde din Retfærd og dine Gerninger;
\par 13 din Gudeflok hjælper og redder dig ej på dit Råb; en Storm bortfejer dem alle, el Vindstød tager dem. Men den, der lider på mig, skal arve Landet og eje mit hellige Bjerg.
\par 14 Og det lyder: "Byg Vej, byg Vej og ban den. ryd Hindringer bort for mit Folk!"
\par 15 Thi så siger den højt ophøjede, som troner evigt, hvis Navn er "Hellig": I Højhed og Hellighed bor jeg, hos den knuste, i Ånden bøjede for at kalde de bøjedes Ånd og de knustes Hjerte til Live.
\par 16 Thi evigt går jeg ikke i Rette, evindelig vredes jeg ej; så vansmægted Ånden for mit Ansigt, Sjæle, som jeg har skabt,
\par 17 For hans Gridskheds Skyld blev jeg vred, slog ham og skjulte mig i Harme; han fulgte i Frafald sit Hjertes Vej.
\par 18 Jeg så hans Vej, men nu vil jeg læge og lede ham og give ham Trøst til Bod;
\par 19 hos de sørgende skaber jeg Læbernes Frugt, Fred, Fred for fjern og nær, siger HERREN, og nu vil jeg læge ham.
\par 20 Men de gudløse er som det oprørte Hav, der ikke kan komme til Ro, hvis Bølger opskyller Mudder og Dynd.
\par 21 De gudløse har ingen Fred, siger min Gud.

\chapter{58}

\par 1 Råb højt spar ikke din Strube, løft din Røst som Basunen, forkynd mit Folk dets Brøde og Jakobs Hus deres Synder!
\par 2 Mig søger de Dag efter bag og ønsker at kende mine Veje, som var de et Folk, der øver Retfærd, ej svigter, hvad dets Gud fandt ret. De spørger mig om Lov og Ret, de ønsker, at Gud er dem nær:
\par 3 "Hvi faster vi, uden du ser os, spæger os, uden du ænser det?" Se, I driver Handel, når I faster, og pisker på Arbejdsflokken.
\par 4 Se, I faster til Strid og Kiv, til Hug med gudløse Næver; som I faster i Dag, er det ikke, for at eders Røst skal høres i det høje.
\par 5 Er det Faste efter mit Sind, en Dag, da et Menneske spæger sig? At hænge med sit Hoved som Siv, at ligge i Sæk og Aske, kalder du det for Faste, en Dag, der behager HERREN?
\par 6 Nej, Faste efter mit Sind er at løse Gudløsheds Lænker. at løsne Ågets Bånd, at slippe de kuede fri og sønderbryde hvert Åg,
\par 7 at bryde dit Brød til de sultne, bringe hjemløse Stakler i Hus, at du klæder den nøgne, du ser, ej nægter at hjælpe dine Landsmænd.
\par 8 Som Morgenrøden bryder dit Lys da frem, da læges hastigt dit Sår, foran dig vandrer din Retfærd, HERRENs Herlighed slutter Toget.
\par 9 Da svarer HERREN, når du kalder; på dit Råb er hans Svar: "Her er jeg!" Fjerner du Åget fra din Midte, holder op at tale ondt og pege Fingre,
\par 10 rækker du den sultne dit Brød og mætter en vansmægtende Sjæl, skal dit Lys stråle frem i Mørke, dit Mulm skal blive som Middag;
\par 11 HERREN skal altid lede dig, mætte din Sjæl, hvor der er goldt, og give dig nye Kræfter; du bliver som en vandrig Have, som rindende Væld, hvor Vandet aldrig svigter.
\par 12 Da bygges på ældgamle Tomter, du rejser længst faldne Mure; da kaldes du "Murbrudsbøder", "Genskaber af farbare Veje".
\par 13 Varer du din Fod på Sabbatten, så du ej driver Handel på min Helligdag, kalder du Sabbatten en Fryd, HERRENs Helligdag ærværdig, ærer den ved ikke at arbejde, holder dig fra Handel og unyttig Snak,
\par 14 da skal du frydes over HERREN; jeg lader dig færdes over Landets Høje og nyde din Fader Jakobs Eje. Thi HERRENs Mund har talet.

\chapter{59}

\par 1 Se, for kort til at frelse er ej HERRENs Arm, Hans Øre er ikke for sløvt til at høre.
\par 2 Eders Brøde er det, der skiller mellem eder og eders Gud, eders Synder skjuler hans Åsyn for jer, så han ikke hører.
\par 3 Eders Hænder er jo sølet at Blod, eders Fingre sølet af Brøde; Læberne farer med Løgn, Tungen taler, hvad ondt er.
\par 4 Med Ret stævner ingen til Doms eller fører ærligt sin Sag. Man stoler på tomt, taler falsk, man undfanger Kval, føder Uret.
\par 5 Slangeæg ruger de ud, og Spindelvæv er, hvad de væver. Man dør, hvis man spiser et Æg, en Øgle kommer frem, hvis det knuses.
\par 6 Deres Spind kan ej bruges til Klæder, ingen hyller sig i, hvad de laver; deres Værk er Ulykkesværk, og i deres Hænder er Vold;
\par 7 deres Fødder haster til ondt, til at udgyde skyldfrit Blod; deres Tanker er Ulykkestanker; hvor de færdes, er Vold og Våde;
\par 8 de kender ej Fredens Veje, der er ingen Ret i deres Spor; de gør sig krogede Stier; Fred kender ingen, som træder dem.
\par 9 Derfor er Ret os fjern, og Retfærd når os ikke; vi bier på Lys se, Mørke, på Dagning, men vandrer i Mulm;
\par 10 vi famler langs Væggen som blinde, famler, som savnede vi Øjne, vi snubler ved Middag som i Skumring, er som døde i vor kraftigste Alder;
\par 11 vi brummer alle som Bjørne, kurrer vemodigt som Duer; vi bier forgæves på Ret, på Frelse, den er os fjern.
\par 12 Thi du ser; vore Synder er mange, vor Brøde vidner imod os: ja, vi har vore Synder for Øje, vi kender såvel vor Skyld:
\par 13 Vi faldt fra og fornægtede HERREN, veg langt bort fra vor Gud, vor Tale var Vold og Frafald, og vi fremførte Løgne fra Hjertet.
\par 14 Retten trænges tilbage, Retfærd står i det fjerne, thi Sandhed snubler på Gaden, Ærlighed har ingen Gænge;
\par 15 Sandhedens Plads står tom, og skyr man det onde, flås man. Og HERREN så til med Harme, fordi der ikke var Ret;
\par 16 han så, at der ingen var, og det undrede ham, at ingen greb ind. Da kom hans Arm ham til Hjælp, hans Retfærd, den stod ham bi;
\par 17 han tog Retfærds Brynje på, satte Frelsens Hjelm på sit Hoved, tog Hævnens Kjortel på og hylled sig i Nidkærheds Kappe.
\par 18 Han gengælder efter Fortjeneste, Vrede mod Uvenner, Gengæld mod Fjender, mod fjerne Strande gør han Gengæld,
\par 19 så HERRENs Navn frygtes i Vest, hans Herlighed, hvor Sol står op. Thi han kommer som en indestængt Flom, der drives af HERRENs Ånde.
\par 20 En Genløser kommer fra Zion og fjerner Frafald i Jakob, lyder det fra HERREN.
\par 21 Dette er min Pagt med dem, siger HERREN: Min Ånd, som er over dig, og mine Ord, som jeg har lagt i din Mund, skal ikke vige fra din eller dit Afkoms eller dit Afkoms Afkoms Mund, siger HERREN, fra nu og til evig Tid.

\chapter{60}

\par 1 Gør dig rede, bliv Lys, thi dit Lys er kommet, HERRENs Herlighed er oprundet over dig.
\par 2 Thi se, Mørke skjuler Jorden og Dunkelhed Folkene, men over dig skal HERREN oprinde, over dig skal hans Herlighed ses.
\par 3 Til dit Lys skal Folkene vandre, og Konger til dit strålende Skær.
\par 4 Løft Øjnene, se dig om, de samles, kommer alle til dig. Dine Sønner kommer fra det fjerne, dine Døtre bæres på Hofte;
\par 5 da stråler dit Øje af Glæde, dit Hjerte banker og svulmer; thi Havets Skatte bliver dine, til dig kommer Folkenes Rigdom:
\par 6 Kamelernes Vrimmel skjuler dig, Midjans og Efas Foler, de kommer alle fra Saba; Guld og Røgelse bærer de og kundgør HERRENs Pris;
\par 7 alt Kedars Småkvæg samles til dig, dig tjener Nebajots Vædre, til mit Velbehag ofres de til mig, mit Bedehus herliggøres.
\par 8 Hvem flyver mon der som Skyer, som Duer til Dueslag?
\par 9 Det er Skibe, der kommer med Hast, i Spidsen er Tarsisskibe, for at bringe dine Sønner fra det fjerne; deres Sølv og Guld har de med til HERREN din Guds Navn, Israels Hellige, han gør dig herlig.
\par 10 Udlændinge skal bygge dine Mure, og tjene dig skal deres Konger; thi i Vrede slog jeg dig vel, men i Nåde forbarmer Jeg mig over dig.
\par 11 Dine Porte holdes altid åbne, de lukkes hverken Dag eller Nat, at Folkenes Rigdom kan bringes dig med deres Konger som Førere.
\par 12 Thi det Folk og, Rige, som ikke tjener dig, skal gå til Grunde, og Folkene skal lægges øde i Bund og Grund.
\par 13 Til dig skal Libanons Herlighed komme, både Cypresser og Elm og Gran, for at smykke min Helligdoms Sted, så jeg ærer mine Fødders Skammel.
\par 14 Dine Undertrykkeres Sønner kommer bøjet til dig, og alle, som håned dig, kaster sig ned for din Fod og kalder dig HERRENs By, Israels Helliges Zion.
\par 15 Medens du før var forladt og hadet, så ingen drog gennem dig, gør jeg dig til evig Højhed, til Glæde fra Slægt til Slægt.
\par 16 Du skal indsuge Folkenes Mælk og die Kongernes Bryst. Du skal kende, at jeg, HERREN, er din Frelser, din Genløser Jakobs Vældige.
\par 17 Guld sætter jeg i Stedet for Kobber og Sølv i Stedet for Jern, Kobber i Stedet for Træ og Jern i Stedet for Sten. Til din Øvrighed sætter jeg Fred, til Hersker over dig Retfærd.
\par 18 Der høres ej mer i dit Land om Uret, om Vold og Ufærd inden dine Grænser; du kalder Frelse dine Mure og Lovsang dine Porte.
\par 19 Ej mer skal Solen være dit Lys eller Månen Skinne for dig:.
\par 20 Din Sol skal ej mer gå ned, din Måne skal ej tage af; thi HERREN skal være dit Lys for evigt, dine Sørgedage har Ende.
\par 21 Enhver i dit Folk er retfærdig, evigt ejer de Landet, et Skud, som HERREN har plantet, hans Hænders Værk, til hans Ære.
\par 22 Den mindste bliver en Stamme, den ringeste et talrigt Folk.

\chapter{61}

\par 1 Den Herre HERRENs Ånd er over mig, fordi han salvede mig; han sendte mig med Glædesbud til ydmyge, med Lægedom for sønderbrudte Hjerter, for at udråbe Frihed for Fanger og Udgang for dem, som er bundet,
\par 2 udråbe et Nådeår fra HERREN, en Hævnens Dag fra vor Gud, for at trøste alle, som sørger,
\par 3 give dem, som sørger i Zion, Højtidspragt for Sørgedragt, for Sørgeklædning Glædens Olie; Lovsang for modløst Sind. Man kalder dem Retfærds Ege, HERRENs Plantning til hans Ære.
\par 4 De skal bygge på ældgamle Tomter, rejse Fortidsruiner, genopbygge nedbrudte Byer, der fra Slægt til Slægt lå i Grus.
\par 5 Fremmede skal stå og vogte eders Småkvæg, Udlændinge slide på Mark og i Vingård.
\par 6 Men I skal kaldes HERRENs Præster, vor Guds Tjenere være eders Navn. Af Folkenes Gods skal I leve, deres Herlighed får I til Eje.
\par 7 Fordi de fik tvefold Skændsel, og Spot og Spyt var deres Lod, får de tvefold Arv i deres Land, dem tilfalder evig Glæde;
\par 8 thi jeg elsker Ret, jeg, HERREN, jeg hader forbryderisk Rov.
\par 9 Deres Æt skal kendes blandt Folkene, deres Afkom ude blandt Folkeslag; alle, der ser dem, skal kende dem som Slægten, HERREN velsigner.
\par 10 Jeg vil glæde mig højlig i HERREN, min Sjæl skal juble i min Gud; thi han klædte mig i Frelsens Klæder, hylled mig i Retfærds Kappe, som en Brudgom, der binder sit Hovedbind, som Bruden, der fæster sine Smykker.
\par 11 Thi som Spiren gror af Jorden, som Sæd spirer frem i en Have, så lader den Herre HERREN Retfærd gro og Lovsang for al Folkenes Øjne.

\chapter{62}

\par 1 For Zions Skyld vil jeg ej tie, for Jerusalems skyld ej hvile før dets Ret rinder op som Lys, som en luende Fakkel dets Frelse.
\par 2 Din Ret skal Folkene skue og alle Konger din Ære. Et nyt Navn giver man dig, som HERRENs Mund skal nævne.
\par 3 Og du bliver en dejlig Krone i HERRENs Hånd, et kongeligt Hovedbind i Hånden på din Gud.
\par 4 Du kaldes ej mer "den forladte", dit Land "den ensomme"; nej, "Velbehag" kaldes du selv, og dit Land kaldes "Hustru". Thi HERREN har Velbehag i dig, dit Land skal ægtes.
\par 5 Som Ynglingen ægter en Jomfru, så din Bygmester dig, som Brudgom glædes ved Brud, så din Gud ved dig.
\par 6 Jeg sætter Vægtere på dine Mure, Jerusalem; ingen Sinde bag eller Nat skal de tie. I, som minder HERREN, und jer ej Ro
\par 7 og lad ham ikke i Ro, før han bygger Jerusalem, før han får gjort Jerusalem til Pris på Jorden.
\par 8 HERREN svor ved sin højre, sin vældige Arm: Jeg giver ej mer dine Fjender dit Korn til Føde; Mosten, du sled for, skal Udlandets Sønner ej drikke;
\par 9 nej, de, der høster, skal spise prisende HERREN; de, der sanker, skal drikke i min hellige Forgård.
\par 10 Drag ud gennem Portene, drag ud, ban Folket Vej, byg Vej, byg Vej, sank alle Stenene af, rejs Banner over Folkeslagene!
\par 11 Se, HERREN lader det høres til Jordens Ende: Sig til Zions Datter: "Se, din Frelse kommer, se, hans Løn er med ham, hans Vinding foran ham!"
\par 12 De skal kaldes: det hellige Folk, HERRENs genløste, og du: den søgte, en By, som ej er forladt.

\chapter{63}

\par 1 Hvem kommer der fra Edom, i Højrøde Klæder fra Bozra, han i det bølgende Klædebon, stolt i sin vældige Kraft?"Det er mig, som taler i Retfærd, vældig til at frelse!"
\par 2 Hvorfor er dit Klædebon rødt, dine Klæder som en Persetræders?
\par 3 "Jeg trådte Vinpersen ene, af Folkeslagene var ingen med mig; jeg trådte dem i min Vrede, tramped dem i min Harme; da sprøjted deres Blod på mine Klæder, jeg tilsøled hele min Klædning.
\par 4 Thi til Hævnens Dag stod min Hu, mit Genløsningsår var kommet.
\par 5 Jeg spejded, men ingen hjalp til, jeg studsed, men ingen stod mig bi. Da kom min Arm mig til Hjælp, og min Harme, den stod mig bi;
\par 6 jeg søndertrådte Folkeslag i Vrede, i Harme knuste jeg dem, deres Blod lod jeg strømme til Jorden."
\par 7 Jeg vil synge om HERRENs Nåde, kvæde hans Pris, efter alt, hvad HERREN har gjort os, huld imod Israels Hus, gjort os efter sin Miskundhed, sin Nådes Fylde.
\par 8 Han sagde: "De er jo mit Folk, de er Børn, som ej sviger." Og en Frelser blev han for dem
\par 9 i al deres Trængsel; intet Bud, ingen Engel, hans Åsyn frelste dem. I sin Kærlighed og Skånsel genløste han dem, han løfted og bar dem alle Fortidens Dage.
\par 10 Men de stred imod og bedrøved hans hellige Ånd; så blev han deres Fjende, han kæmped imod dem.
\par 11 Da tænkte hans Folk på gamle Dage, på Moses: "Hvor er han, som drog sit Småkvægs Hyrde op af Vandet? Hvor er han, som lagde sin hellige Ånd i hans Hjerte,
\par 12 lod vandre sin herlige Arm ved Moses's højre, kløvede Vandet for dem og vandt et evigt Navn,
\par 13 førte dem gennem Dybet som en Hest på Steppen?
\par 14 Som Kvæg, der går ned i dalen, snubled de ikke. Dem ledte HERRENs Ånd. Således ledte du dit Folk for at vinde dig et herligt Navn."
\par 15 Sku ned fra Himlen, se ud fra din hellige, herlige Bolig! Hvor er din Nidkærhed og Vælde, dit svulmende Hjerte, din Medynk? Hold dig ikke tilbage,
\par 16 du, som dog er vor Fader. Thi Abraham ved ej af os, Israel kendes ej ved os, men du er vor Fader, HERRE, "vor Genløser" hed du fra Evighed.
\par 17 Hvi leder du os vild fra dine Veje, HERRE, forhærder vort Hjerte mod din Frygt? Vend tilbage for dine Tjeneres, din Arvelods Stammers Skyld!
\par 18 Hvi har gudløse trådt i din Helligdom, vore Fjender nedtrampet dit Tempel?
\par 19 Vi er som dem, du aldrig har styret, over hvem dit Navn ej er nævnt.
\par 20 Gid du sønderrev Himlen og steg ned, så Bjergene vakled for dit Åsyn!

\chapter{64}

\par 1 Som Vokset smelter i Ild, så lad Ild fortære dine Fjender, at dit Navn må kendes iblandt dem, og Folkene bæve for dit Åsyn,
\par 2 når du gør Undere, vi ikke vented, du stiger ned, for dit Åsyn vakler Bjergene
\par 3 og som ingen Sinde er hørt. Intet Øre har hørt, intet Øje har set en Gud uden dig, som hjælper den, der håber på ham.
\par 4 Du ser til dem, der øver Retfærd og kommer dine Veje i Hu. Men se, du blev vred, og vi synded, og skyldige blev vi derved.
\par 5 Som urene blev vi til Hobe, som en tilsølet Klædning al vor Retfærd. Vi visnede alle som løvet, vort Brøde bortvejred os som Vinden.
\par 6 Ingen påkaldte dit Navn, tog sig sammen og holdt sig til dig; thi du skjulte dit Åsyn for os og gav os vor Brøde i Vold.
\par 7 Men du, o HERRE, er dog vor Fader, vi er Leret, og du har dannet os, Værk af din Hånd er vi alle.
\par 8 Vredes ej, HERRE, så såre, kom ej evigt Brøde i Hu, se dog til, vi er alle dit Folk!
\par 9 Dine hellige Byer er Ørk, Zion er blevet en Ørk, Jerusalem ligger i Grus;
\par 10 vort hellige, herlige Tempel, hvor Fædrene priste dig, er blevet Luernes Rov, en Grushob er alt, hvad vi elskede.
\par 11 Ser du roligt HERRE, på sligt, kan du tie og bøje os så dybt?

\chapter{65}

\par 1 Jeg havde Svar til dem, som ej spurgte, var at finde for dem, som ej søgte; jeg sagde: "Se her, her er jeg!" til et Folk, der ej påkaldte mig.
\par 2 Jeg udbredte Dagen lang Hænderne til et genstridigt Folk, der vandrer en Vej, som er ond, en Vej efter egne Tanker,
\par 3 et Folk, som uden Ophør krænker mig op i mit Åsyn, som slagter Ofre i Haver, lader Offerild lue på Teglsten,
\par 4 som tager Sæde i Grave og om Natten er på skjulte Steder, som spiser Svinekød og har væmmelige Ting i deres Skåle,
\par 5 som siger: "Bliv mig fra Livet, rør mig ej, jeg gør dig hellig!" De Folk er som Røg i min Næse, en altid luende Ild;
\par 6 se, det står skrevet for mit Åsyn, jeg tier ej, før jeg får betalt det, betalt dem i deres Brystfold
\par 7 deres egen og Fædrenes Brøde, begge med hinanden, siger HERREN, de, som tændte Offerild på Bjergene og viste mig Hån på Højene: deres Løn vil jeg tilmåle dem, betale dem i deres Brystfold.
\par 8 Så siger HERREN: Som man, når der findes Saft i Druen, siger: "Læg den ej øde, thi der er Velsignelse deri!" så gør jeg for mine Tjeneres Skyld for ikke at ødelægge alt.
\par 9 Sæd lader jeg gro af Jakob og af Juda mine Bjerges Arving; dem skal mine udvalgte arve, der skal mine Tjenere bo;
\par 10 Saron bliver Småkvægets Græsgang, i Akors Dal skal Hornkvæget ligge for mit Folk, som opsøger mig.
\par 11 Men I, som svigter HERREN og glemmer mit hellige Bjerg, dækker Bord for Lykkeguden, blander Drikke for Skæbneguden,
\par 12 jeg giver jer Sværdet i Vold, I skal alle knæle til Slagtning, fordi I ej svared, da jeg kaldte, ej hørte, endskønt jeg taled, men gjorde, hvad der vakte mit Mishag, valgte, hvad ej var min Vilje.
\par 13 Derfor, så siger den Herre HERREN: Se, mine Tjenere skal spise, men I skal sulte, se, mine Tjenere skal drikke, men I skal tørste, se, mine Tjenere skal glædes, men I skal beskæmmes,
\par 14 se, mine Tjenere skal juble af Hjertens Fryd, men I skal skrige af Hjerteve, jamre af sønderbrudt Ånd.
\par 15 Eders Navn skal I efterlade mine udvalgte som Forbandelsesord: "Den Herre HERREN give dig slig en Død!" Men mine Tjenere skal kaldes med et andet Navn.
\par 16 Den, som velsigner sig i Landet, velsigner sig ved den trofaste Gud, og den, der sværger i Landet, sværger ved den trofaste Gud; thi glemt er de fordums Trængsler, skjult for mit Blik.
\par 17 Thi se, jeg skaber nye Himle og en ny Jord, det gamle huskes ej mer, rinder ingen i Hu;
\par 18 men man frydes og jubler evigt over det, jeg skaber, thi se, jeg skaber Jerusalem til Jubel, dets Folk til Fryd;
\par 19 jeg skal juble over Jerusalem og frydes ved mit Folk; der skal ej mer høres Gråd, ej heller Skrig.
\par 20 Der skal ikke være Børn, der dør som spæde, eller Olding, som ikke når sine Dages Tal thi den yngste, som dør, er hundred År, og forbandet er den, som ej når de hundred.
\par 21 Da bygger de Huse og bor der selv, planter Vin og spiser dens Frugt;
\par 22 de bygger ej, for at andre kan bo, de planter ej, for at andre kan spise; thi mit Folk skal opnå Træets Alder, mine udvalgte bruge, hvad de virker med Hånd;
\par 23 de skal ikke have Møje forgæves, ej avle Børn til brat Død; thi de er HERRENs velsignede Æt og har deres Afkom hos sig.
\par 24 Førend de kalder, svarer jeg; endnu mens de taler, hører jeg.
\par 25 Ulv og Lam skal græsse sammen og Løven æde Strå som Oksen, men Slangen får Støv til Brød; der gøres ej ondt og voldes ej Men i hele mit hellige Bjergland, siger HERREN.

\chapter{66}

\par 1 Så siger HERREN: Himlen er og Jorden mine Fødders Skammel Hvad for et Hus vil I bygge mig, og hvad for et Sted er min Bolig?
\par 2 Alt dette skabte min Hånd, så det fremkom, lyder det fra HERREN. Jeg ser hen til den arme, til den, som har en sønderknust Ånd, og den, som bæver for mit Ord.
\par 3 Den, som slagter Okse, er en Manddraber, den, som ofrer Lam, er en Hundemorder, den, som ofrer Afgrøde, frembærer Svineblod, den, som brænder Røgelse, hylder en Afgud. Som de valgte deres egne Veje og ynder deres væmmelige Guder,
\par 4 så vælger og jeg deres Smerte, bringer over dem, hvad de frygter, fordi de ej svared, da jeg kaldte, ej hørte, endskønt jeg taled, men gjorde, hvad der vakte mit Mishag, valgte, hvad ej var min Vilje.
\par 5 Hør HERRENs Ord, I, som bæver for hans Ord: Således siger eders Brødre, der hader eder og støder eder bort for mit Navns Skyld: "Lad HERREN vise sig i sin Herlighed, så vi kan se eders Glæde!" Men de skal blive til Skamme!
\par 6 Hør, hvor det drøner fra Byen, drøner fra Templet, hør, hvor HERREN øver Gengæld imod sine Fjender!
\par 7 Før hun er i Barnsnød, føder hun, før end Veer kommer over hende, har hun en Dreng.
\par 8 Hvo hørte vel Mage dertil, hvo så vel sligt? Kommer et Land til Verden på en eneste Dag, fødes et Folk på et Øjeblik? Thi Zion kom i Barnsnød og fødte med det samme sine Børn.
\par 9 Åbner jeg et Moderliv og hindrer det i Fødsel? siger HERREN.
\par 10 Glæd dig, Jerusalem! Der juble enhver, som har det kær, tag Del i dets Glæde, alle, som sørger over det,
\par 11 for at I må die dets husvalende Barm og mættes, for at I må drikke af dets fulde Bryst og kvæges.
\par 12 Thi så siger HERREN: Se, jeg leder til hende Fred som en svulmende Flod og Folkenes Rigdom som en Strøm; hendes spæde skal bæres på Hofte, og Kærtegn får de på Skød;
\par 13 som en Moder trøster sin Søn, således trøster jeg eder, i Jerusalem finder I Trøst.
\par 14 I skal se det med Hjertens Glæde, eders Ledemod skal spire som Græs. Hos HERRENs Tjenere kendes hans Hånd, men hos hans Fjender Vrede.
\par 15 Thi se, som Ild kommer HERREN, og hans Vogne er som et Stormvejr, han vil vise sin Harme i Gløder, sin Trussel i flammende Luer;
\par 16 thi med Ild og med sit Sværd skal HERREN dømme alt Kød, og mange er HERRENs slagne.
\par 17 De, som helliger og vier sig for Lundene, følgende en i deres Midte, de, som æder Svinekød og Kød af Kryb og Mus, deres Gerninger og deres Tanker skal forgå til Hobe, lyder det fra HERREN.
\par 18 Jeg kommer for at samle alle Folk og Tungemål, og de skal komme og se min Herlighed.
\par 19 Jeg fuldbyrder et Under i blandt dem og sender undslupne af dem til Folkene, Tarsis, Pul, Lud, Mesjek, Rosj, Tujal, Javan, de fjerne Strande, som ikke har hørt mit Ry eller set min Herlighed; og de skal forkynde min Herlighed blandt Folkene.
\par 20 Og de skal bringe alle eders Brødre fra alle Folk som Gave til HERREN, til Hest, til Vogns, i Bærestol, på Muldyr og Kameler til mit hellige Bjerg Jerusalem, siger HERREN, som når Israelitterne bringer Offergaver i rene Kar til HERRENs Hus.
\par 21 Også af dem vil jeg udtage Levitpræster, siger HERREN.
\par 22 Thi ligesom de nye Himle og den ny Jord, som jeg skaber, skal bestå for mit Åsyn, lyder det fra HERREN, således skal eders Afkom og Navn bestå.
\par 23 Hver Måned på Nymånedagen og hver Uge på Sabbatten skal alt Kød komme og tilbede for mit Åsyn, siger HERREN,
\par 24 og man går ud for at se på Ligene af de Mænd, der faldt fra mig; thi deres Orm dør ikke, og deres Ild slukkes ikke; de er alt Kød en Gru.


\end{document}