\begin{document}

\title{Jeremias' Bog}


\chapter{1}

\par 1 Ord af Jeremias, Hilkijas søn, en af præsterne i Anatot i Benjamins Land,
\par 2 til hvem HERRENs Ord kom i Amons Søns, Kong Josias af Judas, Dage, i hans Herredømmes trettende År,
\par 3 og fremdeles i Josias's Søns, Kong Jojakim af Judas, Dage indtil Slutningen af Josias's Søns, Kong Zedekias af Judas, elevte Regeringsår, til Jerusalems Indbyggere førtes i Landflygtighed i den femte Måned.
\par 4 HERRENs Ord kom til mig således:
\par 5 Før jeg danned dig i Moderskød, kendte jeg dig; før du kom ud af Moderliv, helligede jeg dig, til Profet for Folkene satte jeg dig.
\par 6 Men jeg svarede: "Ak, Herre, HERRE, jeg kan jo ikke tale, thi jeg er ung."
\par 7 Så sagde HERREN til mig: "Sig ikke: Jeg er ung! men gå, hvorhen jeg end sender dig, og tal alt, hvad jeg byder dig;
\par 8 frygt ikke for dem, thi jeg er med dig for at frelse dig, lyder det fra HERREN."
\par 9 Og, HERREN udrakte sin Hånd, rørte ved min Mund og sagde til mig: "Nu lægger jeg mine Ord i din Mund.
\par 10 Se, jeg giver dig i Dag Myndighed over Folk og Riger til at oprykke og nedbryde, til at ødelægge og nedrive, til at opbygge og plante."
\par 11 Siden kom HERRENs Ord til mig således: "Hvad ser du, Jeremias?" Jeg svarede: "Jeg ser en Mandelgren."
\par 12 Da sagde HERREN til mig: "Du ser ret, thi jeg er årvågen over mit Ord for at fuldbyrde det."
\par 13 Og HERRENs Ord kom atter til mig således: "Hvad ser du?" Jeg svarede: "Jeg ser en sydende Kedel med Fyrsted mod Nord."
\par 14 Da sagde HERREN til mig: "Nordfra skal Ulykke syde ud over alle Landets Indbyggere;
\par 15 thi se, jeg hidkalder alle Nordens Riger, lyder det fra HERREN, og man skal komme og rejse hver sin Trone ved Indgangen til Jerusalems Porte mod alle dets Mure trindt omkring og mod alle Judas Byer;
\par 16 og jeg vil afsige Dom over dem for al deres Ondskab, at de forlod mig, tændte Offerild for fremmede Guder og tilbad deres Hænders Værk.
\par 17 Så omgjorde dine Lænder og stå frem og tal til dem, alt hvad jeg byder dig! Vær ikke ræd for dem, at jeg ikke skal gøre dig ræd for dem!
\par 18 Og jeg, se, jeg gør dig i Dag til en fast Stad, til en Jernstøtte og en kobbermur mod hele Landet, Judas Konger og Fyrster, Præsterne og Almuen;
\par 19 de skal kæmpe imod dig, men ikke kunne magte dig; thi jeg er med dig for at frelse dig, lyder det fra HERREN."

\chapter{2}

\par 1 HERRENs Ord kom til mig således:
\par 2 Gå hen og råb Jerusalem i Ørene: Så siger HERREN: Jeg mindes din Kærlighed som ung, din Elskov som Brud, at du fulgte mig i Ørkenen, et Land, hvor der ikke sås;
\par 3 Israel var helliget HERREN, hans Førstegrøde, alle, som åd det, måtte bøde, Ulykke ramte dem, lyder det fra HERREN.
\par 4 Hør HERRENs Ord, Jakobs Hus og alle Slægter i Israels Hus:
\par 5 Så siger HERREN: Hvad ondt fandt eders Fædre hos mig, siden de gik bort fra mig og holdt sig til Tomhed, til de selv blev tomme?
\par 6 De spurgte ikke: "Hvor er HERREN, som førte os op fra Ægypten og ledte os i Ørkenen, Ødemarkens og Kløfternes Land, Tørkens og Mulmets Land, Landet, hvor ingen færdes eller bor?"
\par 7 Jeg bragte eder til Frugthavens Land, at I kunde nyde dets Frugt og Goder; men da I kom derind, gjorde I mit Land urent og min Arvelod vederstyggelig.
\par 8 Præsterne spurgte ikke: "Hvor er HERREN?" De, der syslede med Loven, kendte mig ikke, Hyrderne faldt fra mig, og Profeterne profeterede ved Ba'al og holdt sig til Guder, som intet evner.
\par 9 Derfor må jeg fremdeles gå i Rette med eder, lyder det fra HERREN, og med eders Sønners Sønner må jeg gå i Rette.
\par 10 Drag engang over til Kittæernes Strande og se efter, send Bud til Kedar og spørg jer nøje for; se efter, om sligt er hændet før!
\par 11 Har et Hedningefolk nogen Sinde skiftet Guder? Og så er de endda ikke Guder. Men mit Folk har skiftet sin Ære bort for det, der intet gavner.
\par 12 Gys derover, I Himle, Skræk og Rædsel gribe eder, lyder det fra HERREN;
\par 13 thi to onde Ting har mit Folk gjort: Mig, en Kilde med levende Vand, har de forladt for at hugge sig Cisterner, sprukne Cisterner, der ikke kan holde Vand.
\par 14 Er Israel da en Træl, en hjemmefødt Træl? Hvorfor er han blevet til Bytte?
\par 15 Løver brøler imod ham med rungende Røst; hans Land har de gjort til en Ørk, hans Byer er brændt, så ingen bor der.
\par 16 Selv Nofs og Takpankes's Sønner afgnaver din Isse.
\par 17 Mon ikke det times dig, fordi du svigted mig? lyder det fra HERREN din Gud.
\par 18 Hvorfor skal du nu til Ægypten og drikke af Sjihor? Hvorfor skal du nu til Assur og drikke af Floden?
\par 19 Lad din Ulykke gøre dig klog og lær af dit Frafald, kend og se, hvor ondt og bittert det er, at du svigted HERREN din Gud; Frygt for mig findes ikke hos dig, så lyder det fra Herren, Hærskarers HERRE.
\par 20 Thi længst har du brudt dit Åg og sprængt dine Bånd. Du siger: "Ej vil jeg tjene!" Nej, Skøgeleje har du på hver en Høj, under alle de grønne Træer.
\par 21 Som en Ædelranke plantede jeg dig, en fuldgod Stikling; hvor kunde du da blive Vildskud, en uægte Ranke?
\par 22 Om du end tor dig med Lud og ødsler med Sæbe, jeg ser dog din Brødes Snavs, så lyder det fra HERREN.
\par 23 Hvor kan du sige: "Ej er jeg uren, til Ba'alerne holdt jeg mig ikke!" Se på din Færd i Dalen, kend, hvad du gjorde, en let Kamelhoppe, løbende hid og did,
\par 24 et Vildæsel, kendt med Steppen! Den snapper i Brynde efter Luft, hvo tæmmer dens Brunst? At søge den trætter ingen, den findes i sin Måned.
\par 25 Spar dog din Fod for Slid, din Strube for Tørst! Dog siger du: "Nej, lad mig være! Jeg elsker de fremmede, dem vil jeg holde mig til."
\par 26 Som Tyven får Skam, når han gribes, så Israels Hus, de, deres Konger og Fyrster, Præster og Profeter,
\par 27 som siger til Træ: "Min Fader!" til Sten: "Du har født mig." Thi Ryggen og ikke Ansigtet vender de til mig, men siger i Ulykkestid: "Stå op og frels os!"
\par 28 Hvor er de da, dine Guder, dem, du har gjort dig? Lad dem stå op! Kan de frelse dig i Ulykkestiden? Thi som dine Byers Tal er dine Guders, Juda.
\par 29 Hvorfor tvistes I med mig? I har alle forbrudt jer imod mig, lyder det fra HERREN.
\par 30 Forgæves slog jeg eders Børn, de tog ikke ved Lære, som hærgende Løve fortærede Sværdet Profeterne.
\par 31 Du onde Slægt, så mærk jer dog HERRENs Ord! Har jeg været en Ørk for Israel, et bælgmørkt Land? Hvorfor mon mit Folk da siger: "Vi går, hvor vi vil, og kommer ej mer til dig."
\par 32 Glemmer en Jomfru sit Smykke, en Brud sit Bælte? Og mig har mit Folk dog glemt i talløse Dage.
\par 33 Hvor snildt du dog går til Værks for at søge dig Elskov! Du vænned dig derfor også til ondt i din Færd.
\par 34 Endog findes Blod på dine Hænder af fattige, skyldfri Sjæle, Blod, jeg ej fandt hos en Tyv, men på alle disse.
\par 35 Og du siger: "Jeg er frikendt, hans Vrede har vendt sig fra mig." Se, med dig går jeg i Rette, da du siger: "Jeg har ikke syndet."
\par 36 Hvor let det dog falder for dig at skifte din Vej! Du skal også få Skam af Ægypten, som du fik det af Assur;
\par 37 også derfra skal du gå med Hænder på Hoved, thi HERREN har forkastet dine Støtter, de båder dig intet.

\chapter{3}

\par 1 Når en Mand frastøder sin hustru, og hun går fra ham og ægter en anden, kan hun så gå tilbage til ham? Er slig en Kvinde ej sunket til Bunds i Vanære? Og du, som boled med mange Elskere, vil tilbage til mig! så lyder det fra HERREN.
\par 2 Se til de nøgne Høje: Hvor mon du ej lod dig skænde? Ved Vejene vogted du på dem som Araber i Ørk. Du vanæred Landet både ved din Utugt og Ondskab,
\par 3 en Snare blev dine mange Elskere for dig. En Horkvindes Pande har du, trodser al Skam.
\par 4 Råbte du ikke nylig til mig: "Min Fader! Du er min Ungdoms Ven.
\par 5 Vil han evigt gemme på Vrede, bære Nag for stedse?" Se, således taler du, men øver det onde til Gavns.
\par 6 HERREN sagde til mig i Kong Josias's Dage: Så du, hvad den troløse Kvinde Israel gjorde? Hun gik op på ethvert højt Bjerg og hen under ethvert grønt Træ og bolede.
\par 7 Jeg tænkte, at hun efter at have gjort alt det vilde vende om til mig; men hun vendte ikke om. Det så hendes svigefulde Søster Juda;
\par 8 hun så, at jeg forstødte den troløse Kvinde Israel for al hendes Hors Skyld, og at jeg gav hende Skilsmissebrev; den svigefulde Søster Juda frygtede dog ikke, men gik også hen og bolede.
\par 9 Ved sin letsindige Bolen vanærede hun Landet og horede med Sten og Træ.
\par 10 Og alligevel vendte den svigefulde Søster Juda ikke om til mig af hele sit Hjerte, men kun på Skrømt, lyder det fra HERREN.
\par 11 Og HERREN sagde til mig: Det troløse Israels Sag står bedre end det svigefulde Judas.
\par 12 Gå hen og udråb disse Ord mod Nord: Omvend dig, troløse Israel, lyder det fra HERREN; jeg vil ikke vredes på eder, thi nådig er jeg, lyder det fra HERREN; jeg gemmer ej evigt på Vrede;
\par 13 men vedgå din Uret, at du forbrød dig mod HERREN din Gud; for de fremmedes Skyld løb du hid og did under hvert grønt Træ og hørte ikke min Røst, så lyder det fra HERREN.
\par 14 Vend om, I frafaldne Søoner, lyder det fra HERREN; thi jeg er eders Herre; jeg tager eder, een fra en By og en fra en Slægt og bringer eder til Zion,
\par 15 og jeg giver eder Hyrder efter mit Sind, og de skal vogte eder med Indsigt og Kløgt.
\par 16 Og når I bliver mangfoldige og frugtbare i Landet i hine Dage, lyder det fra HERREN, skal de ikke mere tale om HERRENs Pagts Ark, og Tanken om den skal ikke mere opkomme i noget Hjerte; de skal ikke mere komme den i Hu eller savne den, og en ny skal ikke laves.
\par 17 På hin Tid skal man kalde Jerusalem HERRENs Trone, og der, til HERRENs Navn i Jerusalem, skal alle Folk strømme sammen, og de skal ikke mere følge deres onde Hjertes Stivsind.
\par 18 I hine Dage skal Judas Hus vandre til Israels Hus, og samlet skal de drage fra Nordens Land til det Land, jeg gav deres Fædre i Eje.
\par 19 Og jeg, jeg sagde til dig: "Blandt Sønnerne sætter jeg dig, jeg giver dig et yndigt Land, Folkenes herligste Arvelod." Jeg sagde: "Kald mig din Fader, vend dig ej fra mig!"
\par 20 Men som en Kvinde sviger sin Ven, så sveg du mig, Israels Hus, så lyder det fra HERREN.
\par 21 Hør, der lyder Gråd på de nøgne Høje, Tryglen af Israels Børn, fordi de vandrede Krogveje, glemte HERREN deres Gud.
\par 22 Vend om, I frafaldne Sønner, jeg læger eders Frafald. Se, vi kommer til dig, thi du er HERREN vor Gud.
\par 23 Visselig, Blændværk var Højene, Bjergenes Larm; visselig, hos HERREN vor Gud er Israels Frelse.
\par 24 Skændselen åd fra vor Ungdom vore Fædres Gods, deres Småkvæg og Hornkvæg, Sønner og Døtre.
\par 25 Vi lægger os ned i vor Skændsel, vor Skam er vort Tæppe, thi mod HERREN vor Gud har vi syndet, vi og vore Fædre fra Ungdommen af til i Dag; vi høre ikke på Herren vor Guds røst.

\chapter{4}

\par 1 Omvender du dig, Israel, lyder det fra Herren, så vend dig til mig; hvis du fjerner dine væmmelige Guder, skal du ikke fly for mit Åsyn.
\par 2 Sværger du: "Så sandt HERREN lever," redeligt, ærligt og sandt, skal Folkeslag velsigne sig ved ham og rose sig af ham.
\par 3 Thi så siger HERREN til Judas Mænd og Jerusalems Borgere: Bryd eder Nyjord og så dog ikke blandt, Torne!
\par 4 Omskær jer for HERREN og fjern eders Hjertes Forhud, Judas Mænd og Jerusalems Borgere, at min Vrede ikke slår ud som Ild og hrænder uslukket for eders onde Gerningers Skyld.
\par 5 Forkynd i Juda og Jerusalem, kundgør og tal, lad Hornet gjalde i Landet, råb, hvad I kan, og sig: Flok jer sammen! Vi går ind i de faste Stæder!
\par 6 Rejs Banner hen imod Zion, fly uden Standsning! Thi Ulykke sender jeg fra Nord, et vældigt Sammenbrud.
\par 7 En Løve steg op fra sit Krat, en Folkehærger brød op, gik bort fra sin Hjemstavn for at gøre dit Land til en Ørk; dine Byer skal hærges, så ingen bor der.
\par 8 Derfor skal I klæde jer i Sæk og klage og jamre, thi ej vender HERRENs glødende Vrede sig fra os.
\par 9 På hin Dag, lyder det fra HERREN, skal Kongen og Fyrsterne tabe Modet, Præsterne stivne af Skræk og Profeterne slås af Rædsel;
\par 10 og de skal sige: "Ak, Herre, HERRE! Sandelig, du førte dette Folk og Jerusalem bag Lyset, da du sagde: I skal have Fred! Nu har Sværdet nået Sjælen."
\par 11 På hin Tid skal der siges til dette Folk og Jerusalem: Et glødende Vejr fra Ørkenens nøgne Høje trækker op mod mit Folks Datter, ej til Kastning og Rensning af Korn,
\par 12 et Vejer for vældigt dertil kommer mod mig. Derfor vil jeg også nu tale Domsord imod dem.
\par 13 Se, det kommer som Skyer, dets Vogne som Stormvejr, dets Heste er hurtigere end Ørne; ve, vi lægges øde!
\par 14 Rens dit Hjerte for ondt, Jerusalem, at du må frelses! Hvor længe skal dit Indre huse de syndige Tanker?
\par 15 Thi hør, en Råber fra Dan, et Ulykkesbud fra Efraims Bjerge:
\par 16 Kundgør Folkene: Se! Lad det høres i Jerusalem! Belejrere kommer fra et Land i det fjerne, de opløfter Røsten mod Byerne i Juda.
\par 17 Som Markens Vogtere stiller de sig rundt omkring det, thi genstridigt var det imod mig, lyder det fra HERREN.
\par 18 Det kan du takke din Færd, dine Gerninger for; det skyldes din Ondskab; hvor bittert! Det gælder Livet.
\par 19 Mit indre, mit Indre! Jeg skælver! Mit Hjertes Vægge! Mit Hjerte vånder sig i mig, ej kan jeg tie. Thi Hornets klang må jeg høre, Skrig fra Kampen;
\par 20 der meldes om Fald på Fald, thi alt Landet er hærget. Mine Telte hærges brat, i et Nu mine Forhæng.
\par 21 Hvor længe skal jeg skue Banneret, høre Hornet?
\par 22 Thi mit Folk er tåbeligt, kender ej mig, de er dumme Sønner og uden Indsigt; de er vise til at gøre det onde, men Tåber til det gode.
\par 23 Jeg så på Jorden, og se, den var øde og tom, på Himlen, dens Lys var borte;
\par 24 Bjergene så jeg, og se, de skjalv, og alle Højene bæved;
\par 25 jeg så, og se, der var mennesketomt, og alle Himlens Fugle var fløjet;
\par 26 jeg så, og se, Frugthaven var Ørken, alle dens Byer lagt øde for HERREN, for hans glødende Vrede.
\par 27 Thi så siger HERREN: Al Jorden bliver Ørk, men helt ødelægger jeg ikke.
\par 28 Derfor sørger Jorden, og Himlen deroppe er sort; thi jeg talede og angrer det ikke, tænkte og går ikke fra det.
\par 29 For Larmen af Ryttere og Bueskytter flyr alt Landet, de tyr ind i Krat, stiger op på Klipper; hver By er forladt, og ikke et Menneske bor der.
\par 30 Og du, hvad vil du mon gøre? Om end du klæder dig i Skarlagen, smykker dig med Guld og gør Øjnene store med Sminke det er spildt, du gør dig smuk.
\par 31 Thi jeg hører Råb som ved Barnsnød, Skrig som ved Førstefødsel. Hør, hvor Zions Datter stønner med udrakte Hænder: "Ve mig, min Sjæl bukker under for dem, som myrder."

\chapter{5}

\par 1 Løb i Jerusalems Gader, mærk jer, hvad I ser, og søg på dets Torve, om I kan finde nogen, om der er en, som øver Retfærd, lægger Vind på Sandhed, så jeg kan tilgive dem.
\par 2 Siger de: "Så sandt HERREN lever", sværger de falsk.
\par 3 HERRE, dine Øjne ser jo efter Sandhed. Du slog dem, de omvendte sig ikke; du lagde dem øde, de vilde ej tage ved Lære, gjorde Ansigtet hårdere end Flint, vilde ej vende om.
\par 4 Da tænkte jeg: "Det er kun Småfolk, Dårer er de, thi de kender ej HERRENs Vej, deres Guds Ret;
\par 5 jeg vil vende mig til de store og tale med dem, de kender da HERRENs Vej, deres Guds Ret!" Men alle havde sønderbrudt Åget, sprængt deres Bånd.
\par 6 Derfor skal en Løve fra Skoven slå dem, en Ulv fra Ødemarken hærge dem, en Panter lure ved Byerne; enhver, som går derfra, rives sønder; thi talrige er deres Synder, mange deres Frafald.
\par 7 Hvor kan jeg vel tilgive dig? Dine Sønner forlod mig og svor ved Guder, som ikke er Guder. Når jeg mætted dem, horede de, slog sig ned i Skøgens Hus;
\par 8 de blev fede, gejle Hingste, de vrinsker hver efter Næstens Hustru.
\par 9 Skal jeg ikke hjemsøge sligt? så lyder det fra HERREN, skal ikke min Sjæl tage Hævn over sligt et Folk?
\par 10 Stig op på dets Mure, læg øde, men ikke helt! Ryk Rankerne op, thi HERREN tilhører de ikke.
\par 11 Thi svigefulde er de imod mig, Israels Hus og Judas Hus, så lyder det fra HERREN.
\par 12 De fornægter HERREN og siger: "Det betyder intet! Ulykke kommer ej over os, vi skal ikke se Sværd og Hunger;
\par 13 Profeterne bliver til Vind, Guds Ord er ej i dem; gid Ordet må ramme dem selv!"
\par 14 Derfor, så siger HERREN, Hærskarers Gud: Fordi I siger dette Ord, se, derfor gør jeg mine Ord i din Mund til Ild og dette Folk til Brænde, som Ild skal fortære.
\par 15 Se, jeg bringer over eder et Folk fra det fjerne, Israels Hus, så lyder det fra HERREN, et Folk, som er stærkt, et Folk fra Fortids Dage, et Folk, hvis Mål du ej kender, hvis Tale du ikke fatter;
\par 16 som en åben Grav er dets kogger, de er alle Kæmper;
\par 17 det skal æde dit Korn og dit Brød, det skal æde dine Sønner og Døtte, det skal æde dit Småkvæg og Hornkvæg, det skal æde din Vinstok og dit Figentræ; med Sværd skal de lægge dine Fæstninger øde, dem, som du stoler på.
\par 18 Men selv i de Dage, lyder det fra HERREN, vil jeg ikke udslette eder.
\par 19 Og når de siger: "Hvorfor har HERREN vor Gud gjort os alt det?" sig så til dem: "Som I forlod mig og tjente fremmede Guder i eders Land, således skal I tjene som fremmede i et Land, der ikke er eders."
\par 20 Forkynd dette i Jakobs Hus og kundgør det i Juda:
\par 21 Hør dette, du tåbelige Folk, som er uden Forstand, som har Øjne, men ikke ser, og Ører, men ikke hører:
\par 22 Vil I ikke frygte mig, lyder det fra HERREN, eller bæve for mit Åsyn? Jeg, som gjorde. Sandet til Havets Grænse, et evigt Skel, som det ikke kan overskride; selv om det bruser, evner det intet; om end dets Bølger larmer, kan de ikke overskride det.
\par 23 Dette Folk har et trodsigt og genstridigt Hjerte, de faldt fra og gik bort.
\par 24 De siger ikke i deres Hjerte: "Lad os frygte HERREN vor Gud, som giver os Regn, Tidligregn og Sildigregn, til rette Tid og sikrer os Ugerne, da der skal høstes."
\par 25 Eders Misgerninger bragte dem i Ulave, eders Synder unddrog eder det gode.
\par 26 Thi der findes gudløse i mit Folk; de ligger på Lur, som Fuglefængere dukker de sig; de sætter Fælder, de fanger Mennesker.
\par 27 Som et Bur er fuldt af Fugle, således er deres Huse fulde af Svig; derfor blev de store og rige.
\par 28 De er tykke og fede; og så strømmer de over med onde Ord; de hævder ikke den faderløses Ret, at det måtte gå dem vel, og hjælper ikke de fattige til deres Ret.
\par 29 Skulde jeg ikke hjemsøge sligt? lyder det fra HERREN; skulde min Sjæl da ikke tage Hævn over sligt et Folk?
\par 30 Gyselige, grufulde Ting går i Svang i Landet;
\par 31 Profeterne profeterer Løgn, Præsterne skraber til sig, og mit Folk vil have det så. Men hvad vil I gøre, når Enden kommer?

\chapter{6}

\par 1 Fly, I Benjamins Sønner, bort fra Jerusalem og stød i hornet i Tekoa, hejs Mærket over Bet-Kerem! Thi Ulykke truer fra Nord, et vældigt Sammenbrud.
\par 2 Jeg tilintetgør Zions Datter, den yndige, forvænte
\par 3 til hende kommer der Hyrder med deres Hjorde; de opslår Telte i Ring om hende, afgræsser hver sit Stykke.
\par 4 Helliger Angrebet på hende! Op! Vi rykker frem ved Middag! Ve os, thi Dagen hælder, thi Aftenskyggerne længes.
\par 5 Op! Vi rykker frem ved Nat og lægger hendes Borge øde.
\par 6 Thi så siger Hærskarers HERRE: Fæld Træer og opkast en Vold imod Jerusalem ! Ve Løgnens By med lutter Voldsfærd i sin Midte!
\par 7 Som Brønden sit Vand holder Byen sin Ondskab frisk; der høres om Voldsfærd og Hærværk, Sår og Slag har jeg altid for Øje.
\par 8 Jerusalem, tag ved Lære, at min Sjæl ej vender sig fra dig, at jeg ikke skal gøre dig til Ørk, til folketomt, Land.
\par 9 Så siger Hærskarers HERRE: Hold Efterhøst på Israels Rest, som det sker på en Vinstok, ræk som en Vingårdsmand atter din Hånd til dens Ranker!
\par 10 "For hvem skal jeg tale og vidne, så de hører derpå? Se, de har uomskårne Ører, kan ej lytte til; se, HERRENs Ord er til Spot og huer dem ikke.
\par 11 Jeg er fuld af HERRENs Vrede og træt af at tæmme den." Gyd den ud over Barnet på Gaden, over hele de unges Flok; både Mand og Kvinde skal fanges, gammel og Olding tillige;
\par 12 deres Huse, Marker og Kvinder skal alle tilfalde andre; thi jeg udrækker Hånden mod Landets Folk, så lyder det fra HERREN.
\par 13 Thi fra små til store søger hver eneste Vinding, de farer alle med Løgn fra Profet til Præst.
\par 14 De læger mit Folks Brøst som den simpleste Sag, idet de siger: "Fred, Fred!" skønt der ikke er Fred.
\par 15 De skal få Skam, thi de har gjort vederstyggelige Ting, og dog blues de ikke, dog kender de ikke til Skam. Derfor skal de falde på Valen; på Hjemsøgelsens Dag skal de snuble, siger HERREN.
\par 16 Så siger HERREN: Stå ved Vejene og se efter, spørg efter de gamle Stier, hvor Vejen er til alt godt, og gå på den; så finder I Hvile for eders Sjæle.
\par 17 Og jeg satte Vægtere over dem: "Hør Hornets Klang!" Men de svarede: "Det vil vi ikke."
\par 18 Hør derfor, I Folk, og vidn imod dem!
\par 19 Hør, du Jord! Se, jeg sender Ulykke over dette Folk, Frugten at deres Frafald, thi de lyttede ikke til mine Ord og lod hånt om min Lov.
\par 20 Hvad skal jeg med Røgelsen, der kommer fra Saba, med den dejlige Kalmus fra det fjerne Land? Eders Brændofre er ej til Behag, eders Slagtofre huer mig ikke.
\par 21 Derfor, så siger HERREN: Se, jeg sætter Anstød for dette Folk, og de skal støde an derimod, både Fædre og Sønner; både Nabo og Genbo skal omkomme.
\par 22 Så siger HERREN: Se, et Folkeslag kommer fra Nordens Land, et vældigt Folk bryder op fra det yderste af Jorden.
\par 23 De fører Bue og Spyd, er skånselsløst grumme; deres Røst er som Havets Brusen, de rider på Heste, rustet som Stridsmand mod dig, du Zions Datter.
\par 24 Vi hørte Rygtet derom, vore Hænder blev slappe, Rædsel greb os, Skælven som fødende Kvinde.
\par 25 Gå ikke ud på Marken og følg ej Vejen, thi Fjenden bærer Sværd, trindt om er Rædsel.
\par 26 Klæd dig i Sæk, mit Folks Datter, vælt dig i Støvet, hold Sorg som over den enbårne, bitter Klage! Thi Hærværksmanden skal brat komme over os.
\par 27 Til Metalprøver, Guldprøver, gjorde jeg dig i mit Folk til at kende og prøve deres Færd.
\par 28 De faldt alle genstridige fra, de går og bagtaler, er kun Kobber og Jern, alle handler de slet.
\par 29 Bælgen blæser, af Ilden kommer kun Bly. Al Smelten er spildt, de onde udskilles ej.
\par 30 Giv dem Navn af vraget Sølv, thi dem har HERREN vraget.

\chapter{7}

\par 1 Det Ord som kom til Jeremias fra Herren
\par 2 Stå hen i Porten til HERRENs Hus og udråb dette Ord: Hør HERRENs Ord, hele Juda, I, som går ind gennem disse Porte for at tilbede HERREN!
\par 3 Så siger Hærskarers HERRE, Israels Gud: Bedrer eders Veje og eders Gerninger, så vil jeg lade eder bo på dette Sted.
\par 4 Stol ikke på den Løgnetale: Her er HERRENs Tempel, HERRENs Tempel, HERRENs Tempel!
\par 5 Men bedrer eders Veje og eders Gerninger! Dersom I virkelig øver Ret Mand og Mand imellem,
\par 6 ikke undertrykker den fremmede, den faderløse og Enken, ej heller udgyder uskyldigt Blod på dette Sted, ej heller til egen Skade holder eder til fremmede Guder,
\par 7 så vil jeg til evige Tider lade eder bo på dette Sted i det Land, jeg gav eders Fædre.
\par 8 Se, I stoler på Løgnetale, som intet båder.
\par 9 Stjæle, slå ihjel, hore, sværge falsk, tænde Offerild for Ba'al, holde eder til fremmede Guder, som I ikke kender til
\par 10 og så kommer I og står for mit Åsyn i dette Hus, som mit Navn nævnes over, og siger: "Vi er frelst!" for at gøre alle disse Vederstyggeligheder.
\par 11 Holder I dette Hus, som mit Navn nævnes over, for en Røverkule? Men se, også jeg har Øjne, lyder det fra HERREN.
\par 12 Gå dog hen til mit hellige Sted i Silo, hvor jeg først stadfæstede mit Navn, og se, hvad jeg gjorde ved det for mit Folk Israels Ondskabs Skyld.
\par 13 Og nu, fordi I øver alle disse Gerninger, lyder det fra HERREN, og fordi I ikke vilde høre, når jeg årle og silde talede til eder, eller svare, når jeg kaldte på eder,
\par 14 derfor vil jeg gøre med Huset, som mit Navn nævnes over, og som I stoler på, og med Stedet, jeg gav eder og eders Fædre, ligesom jeg gjorde med Silo;
\par 15 og jeg vil støde eder bort fra mit Åsyn, som jeg stødte alle eders Brødre bort, al Efraims Æt.
\par 16 Men du må ikke gå i Forbøn for dette Folk eller frembære klage og Bøn eller trænge ind på mig for dem, thi jeg hører dig ikke.
\par 17 Ser du ikke, hvad de har for i Judas Byer og på Jerusalems Gader?,,
\par 18 Børnene sanker Brænde, Fædrene tænder Ild, og Kvinderne ælter Dejg for at bage Offerkager til Himmelens Dronning og udgyde Drikofre for fremmede Guder og og krænke mig.
\par 19 Mon det er mig, de krænker, lyder det fra HERREN, mon ikke sig selv til deres Ansigters Skam?
\par 20 Derfor, så siger den Herre HERREN: Se, min Vrede og Harme udgyder sig over dette Sted, over Folk og Fæ, over Markens Træer og Jordens Frugt, og den skal brænde uden at slukkes.
\par 21 Så siger Hærskarers HERRE, Israels Gud: Læg eders Brændofre til eders Slagtofre og æd Kød!
\par 22 Thi dengang jeg førte eders Fædre ud af Ægypten, talede jeg ikke til dem eller bød dem noget om Brændoffer og Slagtoffer;
\par 23 men dette bød jeg dem: "I skal høre min Røst, så vil jeg være eders Gud, og I skal være mit Folk; og I skal gå på alle de Veje, jeg byder eder, at det må gå eder vel."
\par 24 Men de hørte ikke og lånte ikke Øre; de fulgte deres onde Hjertes Stivsind og gik tilbage, ikke fremad.
\par 25 Fra den Dag eders Fædre drog ud af Ægypten, og til i Dag har jeg Dag efter Dag, årle og silde sendt eder alle mine Tjenere Profeterne;
\par 26 men de hørte ikke og lånte ikke Øre; de gjorde Nakken stiv og øvede mere ondt end deres Fædre.
\par 27 Når du siger dem alle disse Ord, hører de dig ikke, og kalder du på dem, svarer de dig ikke.
\par 28 Sig så til dem: Det er det Folk, som ej hørte HERREN deres Guds Røst, det, som ej tog ved Lære; Sandhed er svundet, udryddet af deres Mund.
\par 29 Afklip dit Hår, kast det bort og klag på de nøgne Høje! Thi HERREN har forkastet og bortstødt den Slægt, han var vred på.
\par 30 Thi Judas Sønner har gjort, hvad der er ondt i mine Øjne, lyder det fra HERREN; de har opstillet deres væmmelige Guder i Huset, som mit Navn nævnes over, for at gøre det urent;
\par 31 de har bygget Tofets Offerhøje i Hinnoms Søns Dal for at brænde deres Sønner og Døtre i Ilden, hvad jeg ikke har påbudt, og hvad aldrig var i min Tanke.
\par 32 Se, derfor skal Dage komme, lyder det fra HERREN, da man ikke mere skal sige Tofet og Hinnoms Søns Dal, men Morddalen, og man skal jorde de døde i Tofet, fordi Pladsen ikke slår til.
\par 33 Og dette Folks Lig skal blive Himmelens Fugle og Jordens Dyr til Æde, og ingen skal skræmme dem bort.
\par 34 Og i Judas Byer og på Jerusalems Gader gør jeg Ende på Fryderåb og Glædesråb, Brudgoms Røst og Bruds Røst, thi Landet skal lægges øde.

\chapter{8}

\par 1 Til hin Tid, lyder det fra HERREN, skal man tage Judas kongers ben, dets Fyrsters Ben, Præsternes, Profeternes og Jerusalems Indbyggeres Ben ud af deres Grave
\par 2 og sprede dem for Sol og Måne og al Himmelens Hær, som de elskede og dyrkede, holdt sig til, rådspurgte og tilbad; de skal ikke samles og jordes, men blive til Gødning på Marken.
\par 3 Og Døden skal foretrækkes for Livet af hele den Rest, der er tilbage af denne onde Slægt på alle de Steder, jeg driver dem hen, lyder det fra Hærskarers HERRE.
\par 4 Og du skal sige til dem: Så siger HERREN: Mon man falder og ej står op, går tilbage og ej vender om?
\par 5 Hvi falder da dette Folk i Jerusalem fra i evigt Frafald? De fastholder Svig, vil ikke vende om igen.
\par 6 Jeg lytter, hører nøje efter: de taler ej Sandhed, ingen angrer sin Ondskab og siger: "Hvad gjorde jeg!" Hver styrter frem i sit Løb, som Hest stormer frem i Strid.
\par 7 Selv Storken oppe i Luften kender sin Tid, Turtelduen, Svalen og Tranen holder den Tid, de skal komme; men mit Folk, de kender ej HERRENs Ret.
\par 8 Hvor kan I sige: "Vi er vise, og hos os er HERRENs Lov!" Nej, de skriftkloges Løgnegriffel virked i Løgnens Tjeneste.
\par 9 De vise skal blive til Skamme, ræddes og fanges. Se, HERRENs Ord har de vraget, hvad Visdom har de?
\par 10 Derfor giver jeg andre deres Kvinder, nye Herrer deres Marker. Thi fra små til store søger hver eneste Vinding, de farer alle med Løgn fra Profet til Præst.
\par 11 De læger mit Folks Datters Brøst som den simpleste Sag, idet de siger: "Fred, Fred!" skønt der ikke er Fred.
\par 12 De skal få Skam, thi de har gjort vederstyggelige Ting, og dog blues de ikke, dog kender de ikke til Skam. Derfor skal de falde på Valen; på Hjemsøgelsens Dag skal de snuble, siger HERREN.
\par 13 Jeg vil bjærge deres Høst, så lyder det fra HERREN, men Vinstokken er uden Druer, Figentræet uden Figner, og Løvet er vissent.
\par 14 Hvorfor sidder vi stille? Flok jer dog sammen, lad os gå til de faste Stæder og tilintetgøres der! Thi HERREN vor Gud tilintetgør os, Gift er vor Drik, thi vi synded mod HERREN.
\par 15 Man håber på Fred, men det bliver ej godt, på Lægedoms Tid, men se, der er Rædsel.
\par 16 Hans Hestes Fnysen høres fra Dan, ved Lyden af hans Hingstes Vrinsken skælver alt Landet. De kommer og opæder Landet og dets Fylde, Byen og dens Borgere.
\par 17 Thi se, jeg sender imod jer Slanger; Basilisker, som ikke lader sig besværge, bide jer skal de, lyder det fra HERREN.
\par 18 Min Kvide er ikke til at læge, mit Hjerte er sygt.
\par 19 Hør mit Folks Datters Skrig viden om fra Landet! Er HERREN da ikke i Zion, har det ingen konge? Hvi krænked I mig med eders Billeder, fremmed Tomhed?
\par 20 Kornhøst er omme, Frugthøst endt, og vi er ej frelst!
\par 21 Ved mit Folks Datters Sammenbrud er jeg brudt sammen, jeg sørger, grebet af Rædsel.
\par 22 Er der ikke Balsam i Gilead, ingen Læge der? Hvorfor heles da ikke mit Folks Datters Sår?

\chapter{9}

\par 1 Ak, var mit Hoved Vand, mine Øjne en Tårekilde! Så græd jeg Dag og Nat over mit Folks Datters slagne.
\par 2 Ak, fandt jeg i Ørkenen et Herberg for vandringsmænd. Så drog jeg bort fra mit Folk og gik fra dem. Thi Horkarle er de alle, en svigefuld Bande;
\par 3 de spænder deres Tunges Bue. Løgn, ikke Sandhed råder i Landet; thi de går fra Ondskab til Ondskab og kender ej mig, så lyder det fra HERREN.
\par 4 Vogt eder hver for sin Næste, tro ingen Broder, thi hver Broder er fuld af List, hver sværter sin Næste.
\par 5 De fører hverandre bag Lyset, taler ikke Sandhed: de øver Tungen i Løgn, skejer ud, vil ej vende om,
\par 6 Voldsdåd Slag i Slag og Svig på Svig; de nægter at kendes ved mig, så lyder det fra HERREN.
\par 7 Derfor, så siger Hærskarers HERRE: Se, jeg smelter og prøver dem, ja, hvad må jeg dog gøre for mit Folks Datters Skyld!
\par 8 Deres Tunge er en morders Pil, deres Munds Ord Svig; med Næsten taler de Fred, men i Hjertet bærer de Svig.
\par 9 Skal jeg ikke hjemsøge dem for sligt, så lyder det fra HERREN, skal ikke min Sjæl tage Hævn over sligt et Folk?
\par 10 Over Bjergene bryder jeg ud i Gråd og Klage, over ØrkenensGræsgang i Klagesang. Thi de er afsvedet, mennesketomme, der høres ej Lyd af Kvæg; Himlens Fugle og Dyrene flygtede bort.
\par 11 Jerusalem gør jeg til Stenhob. Sjakalers Bolig, og Judas Byer til Ørk, hvor ingen bor.
\par 12 Hvem er viis nok til at fatte dette, og til hvem har HERRENs Mund talet, så han kan sige det: Hvorfor er Landet lagt øde, afsvedet som en Ørken, mennesketomt?
\par 13 Og HERREN sagde: Fordi de forlod min Lov, som jeg forelagde dem, og ikke hørte min Røst eller vandrede efter den,
\par 14 men fulgte deres Hjertes Stivsind og Baalerne, som deres Fædre lærte dem at kende,
\par 15 derfor, så siger Hærskarers HERRE Israels Gud: Se, jeg giver dette Folk Malurt at spise og Gift at drikke;
\par 16 jeg spreder dem blandt Folk, som hverken de eller deres Fædre før kendte til, og sender Sværdet efter dem, til jeg får gjort Ende på dem.
\par 17 Så siger Hærskarers HERRE, mærk jer det vel! Kald Klagekvinder hid, lad dem komme, hent kyndige Kvinder, lad dem komme,
\par 18 lad dem haste og istemme Klage over os! Vore Øjne skal rinde med Gråd, vore Øjenlåg strømme med Vand.
\par 19 Thi Klageråb høres fra Zion: "Hvor er vi dog hærgede, beskæmmede dybt, fordi vi må bort fra Landet, thi de brød vore Boliger ned."
\par 20 Ja, hør, I Kvinder, mit Ord, eders Øre fange Ord fra min Mund, og lær eders døtre Klage, hverandre Klagesang:
\par 21 "Døden steg op i vore Vinduer; kom i Paladserne, udrydded Barnet på Gaden, de unge på Torvene."
\par 22 Sig: så lyder det fra HERREN: De døde Legemer faldt som Gødning på Marken, som Skåret efter Høstmanden; ingen binder op.
\par 23 Så siger HERREN: Den vise rose sig ikke af sin Visdom, den stærke ikke af sin Styrke, den rige ikke af sin Rigdom;
\par 24 men den, som vil rose sig, skal rose sig af at han har Forstand til at kende mig, at jeg, HERREN, øver Miskundhed, Ret og Retfærdighed på Jorden; thi i sådanne har jeg Behag, lyder det fra HERREN.
\par 25 Se, Dage skal komme, lyder det fra HERREN, da jeg hjemsøger alle de omskårne, som har Forhud:
\par 26 Ægypten, Juda, Edom, Ammoniterne, Moab og alle Ørkenboere med rundklippet Hår; thi Hedningerne er alle uomskårne, men alt Israels Hus har uomskåret Hjerte.

\chapter{10}

\par 1 Hør det Ord HERREN taler til eder, Israels hus!
\par 2 Så siger HERREN: Væn eder ikke til Hedningernes Færd og frygt ikke Himmelens Tegn, fordi Hedningerne frygter dem.
\par 3 Thi Folkenes Rædsel er Tomhed; thi det er Træ, fældet i Skoven, et Værk, som Håndværkerhænder tilhugger med Økse:
\par 4 han smykker det med Sølv og Guld og fæster det med Søm og Hammer, så det ikke vakler.
\par 5 De er som et Fugleskræmsel i Agurkhaven og kan ikke tale; de må bæres, da de ikke kan gå. Frygt dem ikke, thi de gør intet ondt, så lidt som de evner at gøre noget godt.
\par 6 Din Lige findes ikke, HERRE; stor er du og stort dit Navn i Vælde.
\par 7 Hvo skulde ikke frygte dig, du Folkenes Konge! Thi sådant tilkommer dig; thi blandt alle Folks Vismænd og i alle deres Riger findes ikke din Lige.
\par 8 Alle sammen er de dumme og Tåber; Afgudernes Lærdom, den er Træ.
\par 9 Hamret Sølv, indført fra Tarsis, og Guld fra Ofir, et Værk af en Håndværker og Guldsmedens Hænder! De er klædt i violet og rødt Purpur; et Værk af kunstsnilde Folk er de alle.
\par 10 Men HERREN er Gud i Sandhed, han er en levende Gud og en evig Konge; for hans Vrede skælver Jorden, og Folkene udholder ikke hans Harme.
\par 11 Således skal I sige til dem: Guder, der ikke har skabt Himmel og Jord, skal forsvinde fra Jorden og under Himmelen.
\par 12 Han skabte Jorden i sin Vælde, grundfæstede Jorderig i sin Visdom, og i sin Indsigt udspændte han Himmelen.
\par 13 Når han løfter sin Røst, bruser Vandene i Himmelen, og han lader Skyer stige op fra Jordens Ende, får Lynene til at give Regn og sender Stormen ud af sine Forrådskamre.
\par 14 Dumt er hvert Menneske, uden Indsigt; hver Guldsmed får Skam af sit Billede; thi hvad han støber, er Løgn, og der er ikke Ånd i den;
\par 15 Tomhed er de, et dårende Værk; når deres Hjemsøgelses Tid kommer, er det ude med dem.
\par 16 Jakobs Arvelod er ikke som de; thi han har skabt alt, og Israel er hans Arvelods Stamme; Hærskarers HERRE er hans Navn.
\par 17 Tag din Bylt op fra Jorden, du, som sidder belejret!
\par 18 Thi så siger HERREN: Se, denne Gang slynger jeg Landets indbyggere bort og bringer dem i Trængsel, for at de kan bøde.
\par 19 "Ve mig for min Brøst, mit Sår er svart! Men jeg siger: "Det er min Smerte, den vil jeg bære."
\par 20 Mit Telt er hærget og alle mine Teltreb sprængt, mine Børn går fra mig, de er borte; mit Telt spænder ingen ud mer eller opsætter Tæpperne.
\par 21 Thi dumme er Hyrderne, HERREN søger de ikke, du er derfor til intet, og hele deres Hjord er spredt.
\par 22 Der lyder en Tidende, se, den kommer med vældigt Drøn fra Nordens Land og gør Judas Byer til Ørk, til Sjakalers Bo.
\par 23 Jeg ved, HERRE, at et Menneskes Vej ikke står til ham selv, og at det ikke står til en Mand at vandre og styre sine Fjed.
\par 24 Tugt os, HERRE, men med Måde, ikke i Vrede, for ikke at gøre os færre!
\par 25 Udøs din Vrede på Folk, som ikke kender dig, på Slægter, som ikke påkalder dit Navn; thi de har opædt Jakob, tilintetgjort det og lagt dets Bolig øde.

\chapter{11}

\par 1 Det Ord, som kom til Jeremias fra Herren:
\par 2 Hør denne Pagts Ord og tal til Judas Mænd og Jerusalems. Borgere
\par 3 og sig: Så siger HERREN, Israels Gud: Forbandet være den, der ikke hører denne Pagts Ord,
\par 4 som jeg bød eders Fædre holde, dengang jeg førte dem ud af Ægypten, af Jernovnen, idet jeg sagde: "Hør min Røst og gør alt, hvad jeg pålægger eder, så skal I være mit Folk, og jeg vil være eders Gud
\par 5 og holde den Ed, jeg tilsvor eders Fædre om at give dem et Land, der flyder med Mælk og Honning, som det nu er sket!" Og jeg svarede: "Amen, HERRE!"
\par 6 Og HERREN sagde til mig: Udråb alle disse Ord i Judas Byer og på Jerusalems Gader: Hør denne Pagts Ord og hold dem!
\par 7 Thi jeg besvor eders Fædre, dengang jeg førte dem ud af Ægypten, ja til den Dag i Dag, årle og silde: "Hør min Røst!"
\par 8 Men de hørte ikke og bøjede ikke deres Øre, men fulgte alle deres onde Hjertes Stivsind. Derfor bragte jeg over dem alle denne Pagts Ord, som jeg havde pålagt dem at holde, men som de ikke holdt.
\par 9 Og HERREN sagde til mig: Der er fundet en Sammensværgelse blandt Judas Mænd og Jerusalems Borgere;
\par 10 de er vendt tilbage til deres Forfædres Misgerninger, de, som vægrede sig ved at høre mine Ord og holdt sig til fremmede Guder og dyrkede dem; Israels Hus og Judas Hus har brudt den Pagt, jeg sluttede med deres Fædre.
\par 11 Derfor, så siger HERREN: Se, jeg sender en Ulykke over dem, som de ikke kan slippe fra; og når de da råber til mig, vil jeg ikke høre dem.
\par 12 Da skal Judas Byer og Jerusalems Borgere gå hen og råbe til de Guder, de tænder Offerild for; men de kan ikke frelse dem i Nødens Stund.
\par 13 Thi mange som dine Byer er dine Guder, Juda, og mange som Gaderne i Jerusalem er Altrene, I har rejst for Skændselen, Altrene til af tænde Offerild for Baal.
\par 14 Men du må ikke gå i Forbøn for dette Folk eller frembære Klage og Bøn for det; thi jeg hører ikke, når de råber til mig i Nødens Stund.
\par 15 Hvad vil min elskede i mit Hus, hun, som øved Svig? Kan Fedt og helligt kød borttage din Ondskab, eller kan du reddes ved sligt?
\par 16 Et grønt Oliventræ, skønt at skue, så kaldte HERREN dit Navn. Under voldsom Buldren og Bragen afsved Ilden dets Løv og brændte dets Grene.
\par 17 Hærskarers HERRE, som plantede dig, truer dig med Ulykke til Straf for det onde, Israels Hus og Judas Hus gjorde for at krænke mig, idet de tændte Offerild for Baal.
\par 18 HERREN lod mig det vide, derfor ved jeg det; da lod du mig se deres Gerninger.
\par 19 Og jeg var som et tålsomt Lam, der føres til Slagtning. Jeg vidste ej af, at de tænkte på Rænker imod mig: "Lad os ødelægge Træet i Blomst, udrydde ham af de levendes Land, så hans Navn ej ihukommes mer."
\par 20 Hærskarers HERRE, retfærdige Dommer, som prøver Nyrer og Hjerte, lad mig skue din Hævn på dem, thi på dig har jeg væltet min Sag.
\par 21 Derfor, så siger HERREN om Mændene i Anatot, som står mig efter Livet og siger: "Du må ikke profetere i HERRENs Navn; ellers skal du dø for vor Hånd"
\par 22 derfor, så siger Hærskarers HERRE: Se, jeg, vil hjemsøge dem; deres unge Mænd skal dø for Sværd, deres Sønner og Døtre af Hunger;
\par 23 der skal ikke levnes dem nogen Rest, thi jeg sender Ulykke over Mændene i Anatot, når Året, de skal hjemsøges kommer.

\chapter{12}

\par 1 Herre retten er din, når jeg trætter med dig om ret og dog må jeg tale med dig om Ret. Hvi følger Lykken de gudløses Vej, hvi er alle troløse trygge?
\par 2 Du planter dem, og de slår rod, de trives og bærer Frugt. De har dig i Munden, men ikke i Hjertet.
\par 3 Du, HERRE, du kender mig, ser mig og prøver mit Hjertelag mod dig. Riv dem bort som Får til Slagtning, vi dem til Blodbadets Dag!
\par 4 (Hvor længe skal Landet sørge, al Markens Urter visne? For indbyggernes Ondskabs Skyld omkommer Dyr og Fugle; thi man siger: "Han skuer ikke, hvorledes det vil gå os.")
\par 5 "Når Fodgængere løber dig træt, hvor kan du da kappes med Heste? Og er du ej tryg i et fredeligt Land, hvad vil du så gøre i Jordans Stolthed"?
\par 6 Thi selv dine Brødre og din Faders Hus er troløse imod dig, selv de skriger af fuld Hals efter dig; tro dem ikke, når de giver dig gode Ord!"
\par 7 Mit Hus har jeg opgivet, bortstødt min Arvelod, givet min elskede hen i hendes Fjenders Hånd.
\par 8 Min Arvelod blev for mig som en Løve i Skoven, den løftede Røsten imod mig, derfor må jeg hade den.
\par 9 Er min Arvelod blevet mig en spraglet Fugl, omgivet af Fugle? Lad alle de vilde Dyr samles, hent dem hid for af æde!
\par 10 Hyrder i Mængde ødelægger min Vingård, nedtramper min Arvelod, min yndige Arvelod gør de til øde Ørk;
\par 11 de lægger den øde, den sørger øde for mit Åsyn. Hele Landet er ødelagt, thi ingen brød sig om det.
\par 12 Over alle Ørkenens nøgne Høje kom Hærværksmænd. Thi HERREN har et Sværd; det fortærer alt fra den ene Ende af Landet til den anden; intet Kød har Fred.
\par 13 De såede Hvede og høstede Torne, sled til ingen Gavn og blev til Skamme med deres Afgrøde for HERRENs glødende Vredes Skyld.
\par 14 Så siger HERREN om alle mine onde Naboer, der rører den Arvelod, jeg gav mit Folk Israel i Eje: Se, jeg rykker dem op af deres Land, og Judas Hus rykker jeg op midt iblandt dem.
\par 15 Men siden, når jeg har rykket dem op, forbarmer jeg mig atter over dem og bringer dem hjem, hver til sin Arvelod og hver til sit Land.
\par 16 Hvis de da lærer mit Folks Veje, så de sværger ved mit Navn: "Så sandt HERREN lever!" ligesom de lærte mit Folk at svæærge ved Baal, skal de opbygges iblandt mit Folk.
\par 17 Men hører de ikke, rykker jeg et sådant Folk helt op og tilintetgør det, lyder det fra HERREN.

\chapter{13}

\par 1 Således saghde Herren til mig: "Gå hen og køb dig et linned bælte og bind det om din lænd, lad det ikke komme i vand!"
\par 2 Og jeg købte bæltet efter Herrens ord og bandt det om min lænd.
\par 3 Så kom HERRENs Ord atter til mig således:
\par 4 "Tag Bæltet, du købte og har om Lænden, og gå til Frat og gem det der i eu Klipperevne!"
\par 5 Og jeg gik hen og gemte det ved Frat, som HERREN bød.
\par 6 Men lang Tid efter sagde HERREN til mig: "Gå til Frat og hent Bæltet, jeg bød dig gemme der!"
\par 7 Og jeg gik til Frat og gravede Bæltet op, hvor jeg havde gemt det: og se, Bæltet var ødelagt og duede ikke til noget.
\par 8 Og HERRENs Ord kom til mig således:
\par 9 Så siger HERREN: Således vil jeg ødelægge Judas og Jerusalems store Herlighed.
\par 10 Dette onde Folk, som vægrer sig ved at høre mine Ord og vandrer i deres Hjertes Stivsind og holder sig til andre Guder og dyrker og tilbeder dem, skal blive som dette Bælte, der ikke duer til noget.
\par 11 Thi som Bæltet slutter sig tæt til en Mands Lænd, således har jeg sluttet hele Israels Hus og hele Judas Hus tæt til mig, lyder det fra HERREN, for at de skulde være mit Folk og blive mig til Navnkundigbed, Pris og Ære; men de hørte ikke.
\par 12 Og du skal sige til dette Folk: Så siger HERREN, Israels Gud: Enhver Vindunk fyldes med Vin! Og siger de til dig: "Skulde vi ikke vide, at enhver Vindunk fyldes med Vin?"
\par 13 så svar dem: Så siger HERREN: Se, jeg vil fylde alle dette Lands Indbyggere, Kongerne, der sidder på Davids Trone, Præsterne, Profeterne og alle Jerusalems Borgere, så de hliver drukne;
\par 14 og jeg knuser dem mod hinanden, både Fædre og Sønner, lyder det fra HERREN; uden Skånsel, Medynk og Barmhjertighed ødelægger jeg dem.
\par 15 Hør og lyt uden Hovmod, thi HERREN taler.
\par 16 Lad HERREN eders Gud få Ære, før det mørkner, før I støder eders Fødder på Skumringsbjerge, så I må bie på Lys, men han gør det til Mulm, han gør det til Mørke.
\par 17 Men dersom I ikke hører, da græder min Sjæl i Løn for Hovmodets Skyld, den fælder så bitre Tåmer; mit Øje rinder med Gråd, thi HERRENs Hjord føres bort.
\par 18 Sig til Kongen og til Herskerinden: "Tag lavere Sæde, thi af eders Hoved faldt den dejlige Krone."
\par 19 Sydlandets Byer er lukkede, ingen lukker op, hele Juda er bortført til sidste Mand.
\par 20 Løft dine Øjne og se dem komme fra Nord! Hvor er den Hjord, du fik, dine dejlige Får?
\par 21 Hvad vil du sige, når du får dem til Herrer, hvem du lærte at komme til dig som Venner? Vil ikke Veer da gribe dig som Kvinde i Barnsnød?
\par 22 Og siger du i dit Hjerte: "Hvi hændtes mig dette?" For din svare Skyld blev dit Slæb løftet op, dine Hæle skændet.
\par 23 Hvis en Neger kunde skifte sin Hud, en Panter sine Striber, så kunde og I gøre godt, I Mestre i ondt!
\par 24 Jeg spreder dem som Strå, der flyver for Ørkenens Vind:
\par 25 det er din Lod, din tilmålte Del fra mig, så lyder det fra HERREN, fordi du lod mig gå ad Glemme og stoled på Løgn.
\par 26 Ja, dit Slæb slår jeg over dit Ansigt, din Skam skal ses,
\par 27 dit Ægteskabsbrud og din Vrinsken, din skamløse Utugt; på Højene og ude på Marken så jeg dine væmmelige Guder. Ve dig, Jerusalem, du bliver ej ren hvor længe endnu?

\chapter{14}

\par 1 HERRENs Ord, som kom til Jeremias om tørken.
\par 2 Juda sørger; dets Porte vansmægter sørgeklædt i Støvet, Jerusalems Skrig stiger op,
\par 3 og dets Stormænd sender deres Drenge efter Vand, de kommer til Brønde, men finder ej Vand, vender hjem med tomme Spande, med Skam og Skændsel og tilhyllet Hoved.
\par 4 Over Jorden, som revner af Angst, da Regn ej falder i Landet, er Bønderne beskæmmede, tilhyller Hovedet.
\par 5 Selv Hinden på Marken forlader sin nyfødte Kalv, thi Græs er der ikke.
\par 6 På nøgne Høje står Vildæsler og snapper efter Luft som Sjakaler, deres Øjne vansmægter, thi Grønt er der ikke.
\par 7 Vidner vore Synder imod os, HERRE, grib så for dit Navns Skyld ind! Thi mange Gange faldt vi fra, mod dig har vi syndet.
\par 8 Du Israels Håb og Frelser i Nødens Stund! Hvorfor er du som fremmed i Landet, som en Vandringsmand, der kun søger Nattely?
\par 9 Hvorfor er du som en rådvild Mand, som en Helt, der ikke kan frelse? Du er dog i vor Midte, HERRE, dit Navn er nævnet over os, så lad os ej fare!
\par 10 Så siger HERREN til dette Folk: De elsker at flakke omkring og sparer ej Fødderne, men ejer ikke HERRENs Behag. Han ihukommer nu deres Brøde, hjemsøger deres Synder.
\par 11 Og HERREN sagde til mig: "Bed ikke om Lykke for dette Folk!
\par 12 Når de faster, hører jeg ikke deres Klage, og når de ofrer Brændoffer og Afgrødeoffer, har jeg ikke Behag i dem; nej, med Sværd, Hunger og Pest vil jeg gøre Ende på dem!"
\par 13 Da sagde jeg: "Ak, Herre, HERRE! Profeterne siger jo til dem: I skal ikke se Sværd, og Hungersnød skal ikke komme over eder, thi tryg Fred giver jeg eder på dette Sted."
\par 14 HERREN svarede: "Profeterne profeterer Løgn i mit Navn; jeg har ikke sendt dem eller givet dem noget Bud eller talet til dem. Løgnesyner og falsk Spådom og deres Hjertes Bedrag er det, de profeterer for eder!
\par 15 Derfor, så siger HERREN til Profeterne, der profeterer i mit Navn, skønt jeg ikke har sendt dem, og som siger, at der ikke skal komme Sværd eller Hunger i dette Land: Disse Profeter skal omkomme ved Sværd og Hunger;
\par 16 og folket, de profeterer for, skal slænges hen på Jerusalems Gader for Hunger og Sværd, og ingen skal jorde dem, hverken dem eller deres Hustruer, Sønner eller Døtre. Jeg udøser deres Ondskab over dem."
\par 17 Og du skal sige dette Ord til dem: Mine Øjne skal rinde med Gråd ved Nat og ved Dag og aldrig høre op; thi mit Folks jomfruelige Datter ligger lemlæstet hårdt, Såret er såre svart.
\par 18 Hvis jeg går ud på Marken, se sværdslagne Mænd, og kommer jeg ind i Byen, se Hungerens Kvaler! Thi både Profet og Præst drager bort til et Land, de ej kender.
\par 19 Har du ganske vraget Juda, væmmes din Sjæl ved Zion? Hvorfor har du slået os, så ingen kan læge? Man håber på Fred, men det bliver ej godt, på Lægedoms Tid, men se, der er Rædsel.
\par 20 Vi kender vor Gudløshed, HERRE, vore Fædres Brøde, thi vi synded mod dig.
\par 21 Bortstød os ikke for dit Navns Skyld, vanær ej din Herligheds Trone, kom i Hu og bryd ej din Pagt med os!
\par 22 Kan blandt Hedningeguderne nogen sende Regn, giver Himlen Nedbør af sig selv? Er det ikke dig, o HERRE vor Gud? Så bier vi på dig, thi du skabte alt dette.

\chapter{15}

\par 1 Da sagde HERREN til mig: Om så Moses og Samuel stod for mit Åsyn, vilde mit Hjerte ikke vende sig til dem. Jag dette Folk bort fra mit Åsyn!
\par 2 Og når de spørger dig: "Hvor skal vi gå hen?" så svar dem: Så siger HERREN: Hvo Dødens er, til Død, hvo Sværdets er, til Sværd, hvo Hungerens er, til Hunger, hvo Fangenskabets er, til Fangenskab!
\par 3 Jeg sætter fire Magter over dem, lyder det fra HERREN: Sværdet til at slå ihjel, Hundene til at slæbe bort, Himmelens Fugle og Jordens Dyr til at æde og ødelægge.
\par 4 Jeg gør dem til Rædsel for alle Jordens Riger for Ezekiass Søns, Kong Manasse af Judas, Skyld, for alt, hvad han gjorde i Jerusalem.
\par 5 Hvo føler, Jerusalem, for dig, hvo ynker dig vel, hvo bøjer af fra Vejen og spørger til dig?
\par 6 Du vragede mig, så lyder def fra HERREN; du veg bort. Jeg udrækker Hånden, udsletter dig, træt af at ynkes.
\par 7 Med Kasteskovl kaster jeg dem i Landets Porte, mit Folk gør jeg barnløst og til intet; de vendte ej om.
\par 8 Flere end Havets Sandskorn bliver deres Enker. Jeg sender over Ynglingens Moder ved Middag en Hærger, brat lader jeg Angst og Rædsel falde på hende.
\par 9 Syvsønnemoder vansmægter, opgiver Ånden, hendes Sol går alt ned ved Dag, hun beskæmmes og blues. De overblevne giver jeg til Sværdet for Fjendernes Øjne, lyder det fra HERREN.
\par 10 Ve mig, min Moder, at du fødte mig, en Tvistens og kivens Mand for Alerden ! Jeg gav eller modtog ej Lån, og de bander mig alle.
\par 11 HERREN sagde: Sandelig, jeg løser dig, at det må gå dig vel. Sandelig, jeg lader Fjenden bønfalde dig i Ulykkens og Trængselens Tid.
\par 12 Sønderbryder man Jern, Jern fra Norden, og kobber?
\par 13 Din Rigdom og dine Skatte giver jeg hen til Rov, ikke for Betaling, men til Straf for alle dine Synder i alle dine Landemærker;
\par 14 jeg lader dig trælle for dine Fjender i et Land, du ikke kender, thi Ild luer op i min Vrede; den brænder mod eder.
\par 15 Du kender det, HERRE, kom mig i Hu, tag dig af mig; hævn mig på dem, som forfølger mig, vær ikke langmodig, så jeg rives bort! Vid, at for din Skyld bærer jeg Hån
\par 16 fra dem, der lader hånt om dit Ord; ryd dem ud!" Men mig blev dit Ord til Fryd og til Hjertens Glæde; thi dit Navn er nævnet over mig, HERRE, Hærskarers Gud.
\par 17 Ikke sad jeg og jubled i glades Lag; grebet af din Hånd sad jeg ene, thi du fyldte mig med Harme.
\par 18 Hvorfor er min Smerte evig, ulægeligt mit Sår? Det vil ikke læges. Du blev mig som en skuffende Bæk, som Vand, der sviger.
\par 19 Derfor så siger Herren: Omvender du dig, vil jeg omvende dig, så du står for mit Åsyn; giver du det ædle, ej det uædle, Vækst, skal du være som min Mund. De skal vende om til dig, du ikke til dem.
\par 20 Jeg gør dig for dette Folk til en Kobbermur, ingen kan storme; de skal kæmpe mod dig, men ikke få Overhånd over dig, thi jeg er med dig for at frelse og redde dig, lyder det fra HERREN.
\par 21 Jeg redder dig af ondes Vold og frier dig af Voldsmænds Hånd.

\chapter{16}

\par 1 HERRENs Ord kom til mig således
\par 2 Du skal ikke tage dig en Hustru og ikke have Sønner eller Døtre på dette Sted.
\par 3 Thi så siger HERREN om de Sønner og Døtre, der fødes på dette Sted, og om Mødrene, som føder dem, og Fædrene, som avler dem i dette Land:
\par 4 En smertefuld Død skal de dø; der skal ikke holdes Dødeklage over dem, og de skal ikke jordes; til Gødning på Marken skal de blive. De skal omkomme ved Sværd og Hunger; deres Lig skal være Hinmmmmelens Fugle og Jordens Dyr til Æde.
\par 5 Thi så sigem HERREN: Kom ikke i Sorgens Hus, gå ikke til Klage, vis dem ikke Medynk, thi jeg lager min Fred fra dette Folk, lyder det fra HERREN, både Nåde og Barmhjerlighed;
\par 6 og store og små skal dø i dette Land og ikke jordes. De skal ikke holde Dødeklage eller ridse Huden eller klippe sig for deres Skyld,
\par 7 bryde Brød til en, der har Sorg, til Trøst for den døde, eller kvæge ham med Trøstebæger for Fader og Moder.
\par 8 Og kom ikke i et Gildehus for at sidde iblandt dem og spise og drikke;
\par 9 thi så siger Hærskarers HERRE, Israels Gud: Se, for eders Øjne og i eders Dage gør jeg på dette Sted Ende på Fryderåb og Glædesråb, Brudgoms Røst og Bruds Røst.
\par 10 Når du forkynder dette Folk alle disse Ord, og de siger til dig: "Hvorfor udtaler HERREN al den store Ulykke over os, og hvad er det for en Brøde og Synd, vi har gjort mod HERREN vor Gud?"
\par 11 svar dem så: Fordi eders Fædre forlod mig, lyder det fra HERREN, og holdt sig til andre Guder og dyrkede og tilbad dem; mig forlod de og holdt ikke min Lov;
\par 12 og I bærer eder værre ad end eders Fædre, thi se, I vandrer hver efter sit onde Hjertes Stivsind uden at høre mig;
\par 13 derfor slænger jeg eder bort fra dette Land til et Land, I ikke kender, så lidt som eders Fædre, og der skal I dyrke andre Guder både bag og Nat; thi jeg vil ikke give eder Nåde.
\par 14 Se, derfor skal Dage komme, lyder det fra HERREN, da det ikke mere hedder: "Så sandt HERREN lever, der førte Israeliterne op fra Ægypten!"
\par 15 men: "Så sandt HERREN lever, der førte Israeliterne op fra Nordens Land og alle de Lande, til hvilke han havde stødt dem bort!" Og jeg fører dem hjem til deres Land, som jeg gav deres Fædre.
\par 16 Se, jeg sender Bud efter Fiskere i Mængde, lyder det fra HERREN, og de skal fiske dem; og siden sender jeg Bud efter Jægere i Mængde, og de skal jage dem fra hvert Bjerg, hver Høj og Klippernes Kløfter.
\par 17 Thi mine Øjne er rettet på alle deres Veje; de er ikke skjult for mig, og deres Brøde er ikke dulgt for mine Øjne.
\par 18 Og først giver jeg dem tvefold Gengæld for deres Brøde og Synd, fordi de vanhelligede mit Land med deres væmmelige Guders Ådsler og fyldte min Arvelod med deres Vederstyggeligheder.
\par 19 Herre, min Styrke, mit Værn, min Tilflugt i Nødens Stund! Til dig skal Folkeslag komme fra den vide Jord og sige: "Vore Fædre arved kun Løgn, Afguder, ingen af dem hjælper.
\par 20 Kan et Menneske lave sig Guder? De er dog ikke Guder!"
\par 21 Se, derfor lader jeg dem mærke, denne Gang lader jeg dem mærke min Hånd og min Styrke; og de skal kende, at mit Navn er HERREN.

\chapter{17}

\par 1 Optegnet er Judas Synd med med griffel af jern, med Diamantspids ristet i deres Hjertes Tavle og på deres Altres Horn,
\par 2 når Sønnerne kommner deres Altre og Asjerer i Hu, på alle grønne Træer, på de høje Steder,
\par 3 på Bjergene på Marken. Din Rigdom, alle dine Skatte giver jeg hen til Rov til Løn for din Synd, så langt dine Grænser når.
\par 4 Din Hånd må slippe din Arvelod, den, jeg gav dig. Jeg lader dig, trælle for Fjender i et ukendt Land, thi Ild luer op i min Vrede, den brænder evigt.
\par 5 Så siger HERREN: Forbandet være den Mand, som stoler på Mennesker, og som holder Kød for sin Arm, hvis Hjerte viger fra HERREN.
\par 6 Han bliver som Ødemarkens Ene og får ej Lykke at se; han bor i glødende Ørk, i Saltland, hvor ingen fæster Bo.
\par 7 Velsignet være den Mand, som stoler på HERREN, og hvis Tillid HERREN er.
\par 8 Han bliver somn et Træ, der er plantet ved Vand, og strækker sine Rødder til Bækken, ej ængstes, når Heden komnmer, hvis Løv er frodig grønt, som ej ængstes i Tørkens År eller ophører med at bære Frugt.
\par 9 Hjertet er svigefuldt fremfor alt, det er sygt, hvo kender det?
\par 10 Jeg, HERREN, jeg ransager Hjerte og prøver Nyrer for at gengælde hver hans Færd, hans Gerningers Frugt.
\par 11 Som en Agerhømme på Æg, den ikke har lagt, er den, der vinder Rigdom med Uret; han må slippe den i Dagenes Hælvt og slår ved sin Død som en Dåre.
\par 12 En Herlighedstrone, en urgammel Høj er vor Helligdoms Sted.
\par 13 HERRE, du Israels Håb, enhver, der forlader dig, får Skam; de, der falder fra dig, skal udryddes af Landet, thi HERREN, er Kilden med levende Vand, forlod de.
\par 14 Læg mig, HERRE, så jeg læges, frels du mig, så jeg frelses, thi du er min Ros.
\par 15 Se, de andre siger til mig: "Hvor er HERRENs Ord? Lad det konmme!"
\par 16 Jeg vægred mig ej ved at være Hyrde i dit Spor", begæred ej heller Ulykkens Dag, du ved det; hvad der udgik fra mine Læber, er for dit Åsyn.
\par 17 Bliv ikke en Rædsel for mig, du min Tilflugt på Ullykkens Dag.
\par 18 Lad Forfølgerne beskæmmes, lad ej mig beskæmmes, lad dem forfærdes, lad ej mig forfærdes; send over dem Ulykkens Dag, knus dem og gentag Slaget!
\par 19 Således sagde HERREN til mig: Gå hen og stil dig i Folkets Sønners Port, ad hvilken Judas Konger går ind og ud, og i alle Jerusalemns Porte
\par 20 og sig til dem: Hør HERRENS Ord, I Judas Konger og hele Juda og alle Jerusalems Borgere, som går ind ad disse Porte!
\par 21 Så siger HERREN: Vogt eder for eders Sjæles Skyld, at I ikke bærer Byrder ind gennem Jerusalems Porte på Sabbatsdagen!
\par 22 Bring ingen Byrde ud af eders Huse på Sabbatsdagen og gør intet Arbejde, men hold Sabbalsdagen hellig, som jeg bød eders Fædre.
\par 23 De hørte ikke og bøjede ikke deres Øre, men gjorde Nakken stiv for ikke at høre eller tage ved Lære.
\par 24 Men hvis I hører mig, lyder det fra HERREN, så I ikke bringer nogen Byrde ind gennem denne Bys Porte på Sabbatsdagen, men holder den hellig og ikke gør noget Arbejde på den,
\par 25 så skal Konger og Fyrster, som sidder på Davids Trone, drage ind ad denne Bys Porte med Vogne og Heste, de og deres Fyrster, Judas Mænd og Jerusalems Borgere, og denne By skal stå til evig Tid.
\par 26 Og fra Judas Byer, fra Jerusalems Omegn, fra Benjamins Land, fra Lavlandet, Bjergene og Sydlandet skal man komme og bringe Brændoffer, Slagtoffer, Afgrødeoffer og Røgelse og Takoffer til HERRENs Hus.
\par 27 Men hvis I ikke hører mit Ord om at holde Sabbatsdagen hellig og om ikke at bære nogen Byrde ind gennem Jerusalems Porte på Sabbatsdagen, så sætter jeg Ild på dets Porle, og den skal fortære Jerusalems Paladser uden at slukkes.

\chapter{18}

\par 1 Det Ord, som kom til Jeremmas fra Herren
\par 2 "Gå ned til pottemagerens Hus! Der skal du få mine Ord at høre."
\par 3 Så gik jeg ned til Pottemagerens Hus, og se, han var i Arbejde ved Drejeskiven.
\par 4 Og når et Kar, han arbejdede på, mislykkedes, som det kan gå med Leret i Pottemagerens Hånd, begyndte han igen og lavede det om til et andet, som han nu vilde have det gjort.
\par 5 Da kom HERRENs Ord til mig:
\par 6 Skulde jeg ikke kunne gøre med eder, Israels Hus, som denne Pottemager? lyder det fra HERREN. Se, som Leret i Pottemagerens Hånd er I i min Hånd, Israels Hus.
\par 7 Snart truer jeg et Folk og et Rige med at rykke detop, nedbryde og ødelægge det;
\par 8 men når det Folk, jeg har truet, omvender sig fra sin Ondskab, angrer jeg det onde, jeg tænkte at gøre det.
\par 9 Og snart lover jeg et Folk og et Rige at opbygge og planle det;
\par 10 men gør det så, hvad der er ondt i mine Øjne, idet det ikke hører min Røst, angrer jeg det gode, jeg havde lovet at gøre det.
\par 11 Og sig nu til Judas Mænd og Jerusalems Borgere: Så siger HERREN: Se, jeg skaber eder en Ulykke og udtænker et Råd imod eder; vend derfor om, hver fra sin onde Vej, og bedrer eders Veje og eders Gerninger.
\par 12 Men de svarer: "Nej! Vi vil følge vore egne Tanker og gøre hver efter sit onde Hjertes Stivsind."
\par 13 Derfor, så siger HERREN: Spørg dog rundt blandt Folkene: Hvo hørte mon sligt? Grufulde Ting har hun øvet, Israels Jomfru.
\par 14 Forlader Libanons Sne den Almægtiges klippe, eller udtørres Bjergenes kølige, rislende Vande,
\par 15 siden mit Folk har glemt mig og ofrer til Løgn? De snubler på deres Veje, de ældgamle Spor, og vandrer ad Stier, en Vej, der ikke er højnet,
\par 16 for at gøre deres Land til Gru, til evig Spot; hver farende Mand skal grue og ryste på Hovedet.
\par 17 Som en Østenstorm splitter jeg dem for Fjendens Ansigt, jeg viser dem Ryg, ej Åsyn på Vanheldets Dag.
\par 18 De sagde: Kom, vi spinder Rænker imod Jeremias! Thi ej glipper Loven for Præsten, ej Rådet for den vise, ej Ordet for Profeten. Kom, lad os slå ham med Tungen og lure på alle hans Ord!"
\par 19 Lyt, o Herre, til mig og hør min Modparts Ord!
\par 20 Skal godt gengældes med ondt? De grov jo min Sjæl en Grav. Kom i Hu, at jeg stod for dit Åsyn for at tale til Bedste for dem og vende din Vrede fra dem!
\par 21 Giv derfor deres Sønner til Hunger, styrt dem i Sværdets Vold; Barnløshed og Enkestand ramme deres Kvinder, deres Mænd vorde slagne af Døden, deres ungdom sværdslagne i Krig;
\par 22 lad der høres et Skrig fra Husene, når du lader en Mordbande brat komme over dem. Thi de grov en Grav for at fange mig og lagde Snarer for min Fod.
\par 23 Ja du, o HERRE, du kender alle deres Dødsråd imod mig. Tilgiv ikke deres Brøde, slet ikke deres Synd for dit Åsyn, lad dem komme til Fald for dit Åsyn, få med dig at gøre i din Vredes Stund!

\chapter{19}

\par 1 Således sagde HERREN: Gå hen og køb dig et krus hos pottemageren, tag nogle af Folkets og Præsternes Ældste med
\par 2 og gå ud i Hinnoms Søns Dal ved Indgangen til Potteskårporten og udråb der de Ord, jeg taler til dig!
\par 3 Du skal sige: Hør HERRENs Ord, Judas Konger og Jerusalems Borgere: Så siger Hærskarers HERRE, Israels Gud: Se, jeg sender over dette Sted en Ulykke, så det skal ringe for Ørene på enhver, der hører derom,
\par 4 fordi de forlod mig og gjorde dette Sted fremmed og tændte Offerild der for andre Guder, som hverken de eller deres Fædre før kendte til, og Judas Konger fyldte dette Sted med skyldfries Blod,
\par 5 og de byggede Baalshøjene for at brænde deres Børn i Ild som Brændofre til Baal, hvad jeg ikke havde påbudt eller talt om, og hvad aldrig var opkommet i min Tanke.
\par 6 Se, derfor skal Dage komme, lyder det fra HERREN, da dette Sted ikke mere skal hedde Tofet og Hinnoms Søns Dal, men Morddalen.
\par 7 Jeg gør Juda og Jerusalem rådvilde på dette Sted og lader dem falde for Sværdet for deres Fjenders Øjne og for deres Hånd, som står dem efter Livet, og jeg giver Himmelens Fugle og Jordens Dyr deres Lig til Føde.
\par 8 Jeg gør denne By til Gru og Spot; alle, der komnmer forbi, skal grue og spotte over alle dens Sår.
\par 9 Jeg lader dem æde deres Sønners og Døtres Kød, den ene skal æde den andens Kød under Belejringen og den Trængsel, deres Fjender og de, der står dem efter Livet, volder dem.
\par 10 Knus så Kruset i de Mænds Påsyn, der følger med dig,
\par 11 og sig til dem: Så siger Hærskarers HERRE: Jeg vil knuse dette Folk og denne By, som man knuser et Lerkar, så det ikke kan heles igen. De døde skal jordes i Tofet, fordi Pladsen til at jorde på ikke slår til.
\par 12 Således vil jeg gøre med dette Sted og dets indbyggere, lyder det fra HERREN, idet jeg gør denne By til et Tofet:
\par 13 Jerusalems og Judas Kongers Huse skal blive urene som Tofets Sted, alle de Huse, på hvis Tage de tændte Offerild for al Himmelens Hær og udgød Drikofre for andre Guder.
\par 14 Derpå gik Jerenmias fra Tofet, hvorhen HERREN havde sendt ham for at profetere, og stod frenm i Forgården til HERRENs Hus og sagde til alt Folket:
\par 15 Så siger Hærskarers HERRE, Israels Gud: Se, over denne By og alle Byerne, der hører til den, sender jeg al den Ulykke, jeg har truet den med, fordi de gjorde Nakken stiv og ikke hørte mine Ord.

\chapter{20}

\par 1 Da præsten Pasjhur, Immers søn, der var overopsynsmand i HERRENs Hus, hørte Jeremias profetere således,
\par 2 slog han ham og lod ham lægge i Blokken i den øvre Benjaminsport i HERRENs Hus.
\par 3 Men da Pasjhur Dagen efter slap Jeremias ud af Blokken, sagde Jeremias til ham: HERREN kalder dig ikke Pasjhur, men: Trindt-om-er-Rædsel.
\par 4 Thi så siger HERREN: Se, jeg gør dig til Rædsel for dig selv og for alle dine Venner; de skal falde for deres Fjenders Sværd, og dine Øjne skal se det. Og hele Juda giver jeg i Babels Konges Hånd; han skal føre dem til Babel og hugge dem ned med Sværdet.
\par 5 Og jeg giver alt denne Bys Gods og al dens Velstand og alle dens koslelige Ting og alle Judas Kongers Skatte i deres fjenders Hånd; de skal rane dem og tage dem og føre dem til Babel.
\par 6 Og du Pasjhur og alle, der bor i dit Hus, skal gå i Fangenskab. Du skal komme til Babel; der skal du dø, og der skal du jordes sammen med alle dine Venner, for hvem du har profeteret Løgn.
\par 7 Du overtalte mig, HERRE, og jeg lod mig overtale, du tvang mig med Magt.
\par 8 Thi så tit jeg faler, må jeg skrige, råbe: "Vold og Overfald!" Thi HERRENs Ord er mig Dagen lang til Skændsel og Spot.
\par 9 Men tænkte jeg: "Ej vil jeg mindes ham, ej tale mer i hans Navn," da blev det som brændende Ild i mit indre, som brand i mine Ben; jeg er træt, jeg kan ikke mere, jeg evner det ej;
\par 10 thi jeg hører mange hviske, trindt om er Rædsel: "Angiv ham!" og: "Vi vil angive ham!" Alle mine Venner lurer på et Fejltrin af mig: "Måske går han i Fælden, så vi får ham i vor Magt, og da kan vi hævne os på ham!"
\par 11 Men HERREN er med mig som en vældig Helt; derfor skal de, som forfølger mig, snuble i Afmagt, højlig beskæmmes, thi Heldet svigter dem, få Skændsel, der aldrig glemmes.
\par 12 Du Hærskarers HERRE, som prøver den retfærdige, gennemskuer Nyrer og Hjerte, lad mig skue din Hævn på dem, thi på dig har jeg væltet min Sag.
\par 13 Syng for HERREN, lovpris HERREN! Thi han redder den fattiges Sjæl af de ondes Hånd.
\par 14 Forbandet være den Dag, på hvilken jeg fødtes; den Dag, min Moder fødte mig, skal ikke velsignes.
\par 15 Forbandet den Mand, som bragte min Fader det Bud: "Et Barn, en Dreng er født dig!" og glæded ham såre.
\par 16 Det gå den Mand som Byerne, HERREN omstyrted uden Medynk; han høre Skrig ved Gry, Kampråb ved Middagstide.
\par 17 At han ej lod mig dø i Moders Liv, så min Moder var blevet min Grav og hendes Moderliv evigt svangert!
\par 18 Hvi kom jeg af Moders Liv, når jeg kun skulde opleve Møje og Harm, mine Dage svinde i Skam!.

\chapter{21}

\par 1 Det Ord, som kom til Jeremias fra Herren, da kong Zedekias sendte Pasjhur, Malkias Søn, og Præsten Zefanja, Maasejas Søn, til ham og lod sige:
\par 2 "Rådspørg HERREN for os, thi kong Nebukadrezar af Babel angriber os; måske vil HERREN handle med os efter alle sine Undergerninger, så Nebukadrezar drager bort fra os."
\par 3 Jeremias svarede dem: "Sig til Zedekias:
\par 4 Så siger HERREN, Israels Gud: Se, Våbnene i eders Hånd, med hvilke I uden for Muren kæmper mod Babels Konge og Kaldæerne, der belejrer eder, dem driver jeg tilbage og samler dem midt i denne By;
\par 5 og jeg vil selv kæmpe mod eder med utdrakt Hånd og stærk Arm, i Vrede og Harme og stor Fortørnelse;
\par 6 og jeg slår denne Bys Indbyggere, ja både Folk og Fæ, med voldsom Pest, så de dør.
\par 7 Og siden, lyder det fra HERREN, giver jeg Kong Zedekias af Juda og hans Tjenere og Folket, der levnes i denne By af Pesten, Sværdet og Hungeren, i Kong Nebukadrezar af Babels og i deres Fjenders Hånd, og i deres Hånd, som står dem efter Livet; de skal hugge dem ned med Sværdet, og jeg vil ikke ynkes over dem eller vise Skånsel eller Barmhjertighed!"
\par 8 Og sig til dette Folk: "Så siger HERREN: Se, jeg forelægger eder Livets Vej og Dødens Vej.
\par 9 Den, som bliver i denne By, skal dø ved Sværd, Hunger og Pest: men den, som går ud og overgiver sig til Kaldæerne, der belejrer eder, skal leve og vinde sit Liv som Bytte,
\par 10 Thi jeg retter mit Åsyn mod denne By til Ulykke og ikke til Lykke, lyder det fra HERREN; i Babels Konges Hånd skal den gives, og han skal opbrænde den med Ild."
\par 11 Og sig til Judas Konges Hus: Hør HERRENs Ord,
\par 12 Davids Hus! Så siger HERREN: Hold årle retfærdig Dom, fri den, som er plyndret, af Voldsmandens Hånd, at ikke min Vrede slår ud som Ild og brænder, så ingen kan slukke, for eders onde Gerningers Skyld.
\par 13 Se, jeg kommer over dig, du By i Dalen, du Slettens Klippe, lyder det fra HERREN, I, som siger: "Hvo falder over os, hvo trænger ind i vore Boliger?"
\par 14 Efter eders Gerningers Frugt hjemsøger jeg jer, lyder det fra HERREN; jeg sætter Ild på dens Skov, den fortærer alt deromkring.

\chapter{22}

\par 1 Så siger Herren: Gå ned til Judas konges palads og tal dette Ord
\par 2 og sig: Hør HERRENs Ord, Judas Konge, som sidder på Davids Trone, du, dine Tjenere og dit Folk, som går ind ad disse Porte!
\par 3 Så siger HERREN: Øv Ret og Retfærd, fri den, som er plyndret, af Voldsmandens Hånd, undertryk ikke den fremmede, den faderløse og Enken, øv ikke Vold og udgyd ikke uskyldigt Blod på dette Sted.
\par 4 Thi dersom I efterkommer dette Krav, skal konger, der sidder på Davids Trone, drage ind ad Portene til dette Hus med Vogne og Heste, de, deres Tjenere og Folk:
\par 5 Men hører I ikke disse Ord, så sværger jeg ved mig selv, lyder det fra HERREN, at dette Hus skal lægges øde.
\par 6 Thi så siger HERREN om Judas konges Palads: Et Gilead var du for mig, en Libanons Tinde; visselig, jeg gør dig til Ørk, til folketomme Byer;
\par 7 Hærværksmænd helliger jeg mod dig, hver med sit Værktøj, de skal fælde dine udvalgte Cedre og kaste dem i Ilden.
\par 8 Mange Folkeslag skal drage forbi denne By og spørge hverandre: "Hvorfor handlede HERREN således med denne store By?"
\par 9 Og man skal svare: "Fordi de forlod HERREN deres Guds Pagt og tilbad og dyrkede andre Guder."
\par 10 Græd ej over den døde, beklag ham ikke! Græd over ham, der drog bort, thi han vender ej hjem, sit Fødeland genser han ikke.
\par 11 Thi så siger HERREN om Josiass Søn, Kong Sjallum af Juda, der blev Konge i sin Fader Josiass Sted: Han, som gik bort fra dette Sted, skal ikke vende hjem igen:
\par 12 men på det Sted, til hvilket de førte ham i Landflygtighed, skal han dø, og han skal ikke gense dette Land.
\par 13 Ve ham, der bygger Hus uden Retfærd, Sale uden Ret, lader Landsmand trælle for intet, ej giver ham Løn,
\par 14 som siger: "Jeg bygger mig et rummeligt Hus med luftige Sale," som hugger sig Vinduer ud, klæder Væg med Cedertræ og maler det rødt.
\par 15 Er du Konge, fordi du brammer med Cedertræ? Din Fader, mon ikke han spiste og drak og øvede Ret og Retfærd? Da gik det ham vel;
\par 16 han hjalp arm og fattig til sin Ret; da gik det ham vel. Er dette ikke at kende mig? lyder det fra HERREN.
\par 17 Men dit Øje og Hjerte higer kun efter Vinding, efter at udgyde skyldfries Blod, øve Undertrykkelse og Vold.
\par 18 Derfor, så siger HERREN om Josiass Søn, kong Jojakim af Juda: Over ham skal ej klages: "Ve min Broder, ve min Søster!" eller grædes: "Ve min Herre, ve hans Herlighed!"
\par 19 Et Æsels Jordefærd får han, slæbes ud, slænges hen uden for Jerusalems Porte.
\par 20 Stig op på Libanon og skrig, løft Røsten i Basan, skrig fra Abarim, thi knuste er alle dine kære.
\par 21 Jeg taled dig til i din Tryghed, du nægted at høre; at overhøre min Røst var din Skik fra din Ungdom.
\par 22 For alle dine Hyrder skal Storm være Hyrde, i Fangenskab går dine kære; da får du Skam og Skændsel for al din Ondskab.
\par 23 Du, som bor på Libanon og bygger i Cedrene, hvor stønner du, når Smerter kommer over dig, Veer som en fødendes!
\par 24 Så sandt jeg lever, lyder det fra HERREN: Om også Konja, Kong Jojakim af Judas Søn, var en Seglring på min højre Hånd, jeg rev ham bort.
\par 25 Jeg giver dig i deres Hånd, som står dig efter Livet, i deres Hånd, for hvem du ræddes, og i Kong Nebukadrezar af Babels og Kaldæernes Hånd.
\par 26 Jeg slynger dig og din Moder, som fødte dig, bort til et andet Land, hvor I ikke fødtes, og der skal I dø;
\par 27 men til det Land, deres Sjæle længes tilbage til, skal de ikke vende hjem.
\par 28 Er denne Konja da et usselt, sønderslået Kar, et Redskab, ingen bryder sig om? Hvorfor skal han og hans Afkom slynges og kastes til et Land, de ikke kender?
\par 29 Land, Land, Land, hør HERRENs Ord:
\par 30 Så siger HERREN: Optegn denne Mand som barnløs, som en Mand, der ingen Lykke har i sit Liv; thi det skal ikke lykkes nogen af hans Afkom at sætte sig på Davids Trone og atter herske over Juda.

\chapter{23}

\par 1 Ve Hyrderne, der ødelægger og adsplitter de får jeg græsser, lyder det fra HERREN.
\par 2 Derfor, så siger HERREN, Israels Gud, til de Hyrder, som vogter mit Folk: Da I har adsplittet og spredt mine Får og ikke taget eder af dem, vil jeg nu tage mig af eder for eders onde Gerningers Skyld, lyder det fra HERREN.
\par 3 Men dem, der er tilovers af mine Får, vil jeg sanke sammen fra alle de Lande, til hvilke jeg har bortstødt dem, og føre dem tilbage til deres Græsgange, og de skal blive frugtbare og mangfoldige.
\par 4 Da vil jeg sætte Hyrder over dem, og de skal vogte dem; og de skal ikke mere frygte eller ræddes og ingen skal savnes, lyder det fra HERREN.
\par 5 Se, Dage skal komme, lyder det fra HERREN, da jeg opvækker David en retfærdig Spire, og han skal herske som Konge og handle viselig og øve Ret og Retfærd i Landet.
\par 6 I hans Dage skal Juda frelses og Israel bo trygt. Og det Navn, man skal give ham er: HERREN vor Retfærdighed.
\par 7 Se, derfor skal Dage komme, lyder det fra HERREN, da det ikke mere hedder: "Så sandt HERREN lever, der førte Israeliterne op fra Ægypten!"
\par 8 men: "Så sandt HERREN lever, der førte og bragte Israels Huss Afkom op fra Nordens Land og fra alle de Lande, til hvilke han havde bortstødt dem!" Og de skal bo i deres Land.
\par 9 Om Profeterne. Mit Hjerte er knust i Brystet, hvert Ledemod er slapt, jeg er som en drukken, en Mand, overvældet af Vin, for HERRENs Skyld, for hans hellige Ords Skyld.
\par 10 Thi Landet er fuldt af Horkarle, og under Forhandelse, sørger Landet, Ørkenens Græsgange visner. Man haster til det, som er ondt, og er stærk i Uret.
\par 11 Thi både Profet og Præst er vanhellig, selv i mit Hus har jeg mødt deres Ondskab, lyder det fra HERREN.
\par 12 Derfor bliver deres Vej det, som slibrige Stier, i Mørke stødes de ud og snubler deri. Thi Ulykke sender jeg over dem, Hjemsøgelsens År, så lyder det fra HERREN.
\par 13 Hos Samarias Profeter så jeg slemme Ting; ved Baal profetered de og vildledte Israel, mit Folk.
\par 14 Hos Jerusalems Profeter så jeg grufulde Ting: de horer og vandrer i Løgn, de styrker de ondes Hænder, så de ikke vender om enhver fra sin Ondskab. Som Sodoma er de mig alle, dets Folk som Gomorra.
\par 15 Derfor, så siger Hærskarers HERRE om Profeterne: Se, jeg giver dem Malurt at spise og Giftvand at drikke; thi fra Jerusalems Profeter udgår Vanhelligelse over hele Landet.
\par 16 Så siger Hærskarers HERRE: Hør ikke Profeternes Ord, når de profeterer for eder; de dårer eder kun. Deres eget Hjertes Syn fremfører de, ikke Ord fra HERRENs Mund.
\par 17 De siger til dem, der ringeagter HERRENs Ord: "Det skal gå eder vel!" og til enhver, som vandrer i sit Hjertes Stivsind: "Der skal ikke ske eder noget ondt!"
\par 18 Thi hvem stod i HERRENs fortrolige Råd, så han så og hørte hans Ord, hvem lyttede til hans Ord og hørte det?
\par 19 Se, HERRENs Stormvejr, Vreden, er brudt frem, et hvirvlende Stormvejr; det hvirvler over de gudløses Hoved.
\par 20 HERRENS Vrede lægger sig ikke, før han har udført og fuldbyrdet sit Hjertes Tanker; i de sidste Dage skal I forstå det.
\par 21 Jeg har ej sendt Profeterne, alligevel løber de, jeg talede ikke til dem, og dog profeterer de.
\par 22 Hvis de står i mit fortrolige Råd og hører mine Ord, så lad dem vende mit Folk fra deres onde Vej og deres Gerningers Ondskab.
\par 23 Er jeg kun en Gud i det nære, så lyder det fra HERREN, og ikke en Gud i det fjerne?
\par 24 Kan nogen krybe i Skjul, så jeg ikke ser ham? lyder det fra HERREN. Er det ikke mig, der fylder Himmel og Jord? lyder det fra HERREN.
\par 25 Jeg har hørt, hvad Profeterne, der profeterer Løgn i mit Navn, siger: "Jeg har drømt, jeg har drømt!"
\par 26 Hvor længe skal det vare? Har Profeterne, som profeterer Løgn og deres Hjertes Svig, mon i Sinde
\par 27 og higer de efter at få mit Folk til at glemme mit Navn ved de Drømme, de meddeler hverandre, ligesom deres Fædre glemte mit Navn over Baal?
\par 28 Den Profet, som har en Drøm, meddele sin Drøm, men den, hos hvem mit Ord er, tale mit Ord i Sandhed! Hvad har Strå med Kærne at gøre? lyder det fra HERREN.
\par 29 Er ikke mit Ord som Ild, lyder det fra HERREN, og som en Hammer, der knuser Fjelde?
\par 30 Se, derfor kommer jeg over Profeterne, lyder det fra HERREN, de, som stjæler mine Ord fra hverandre.
\par 31 Se, jeg kommer over Profeterne, lyder det fra HERREN, de, som taler af sig selv og dog siger: "Så lyder def fra HERREN."
\par 32 Se, jeg kommer over Profeterne, som profeterer og udspreder Løgnedrømme, lyder det fra HERREN, og vildleder mit Folk med deres Løgne og Pralen, og jeg har ikke sendt dem eller givet dem nogen Befaling; de bringer ikke dette Folk nogen Hjælp, lyder det fra HERREN.
\par 33 Når dette Folk eller en Profet eller Præst spørger dig: "Hvad er HERRENs Byrde?" skal du svare: "Byrden er I, men jeg kaster eder af," lyder det fra HERREN.
\par 34 Og Profeten, Præsten og Folket, som siger "HERRENs Byrde", den Mand og hans Hus vil jeg hjemsøge.
\par 35 Således skal I sige til hverandre, Mand til Mand: "Hvad svarede HERREN?" og: "Hvad talede HERREN?"
\par 36 Men om HERRENs Byrde må I ikke mere tale, thi Byrden for enhver skal være hans eget Ord. Og I laver om på den levende Guds, Hærskarers HERREs, vor Guds, Ord.
\par 37 Således skal du sige til Profeten: "Hvad svarede HERREN?" og: "Hvad talede HERREN?"
\par 38 Og dersom I siger: "HERRENs Byrde" derfor, så siger HERREN: Fordi I siger dette Ord: "HERRENs Byrde", skønt jeg sendte eder det Bud: "I må ikke sige "HERRENs Byrde!"
\par 39 se, derfor vil jeg løfte eder op og kaste eder og den By, jeg gav eder og eders Fædre, bort fra mit Åsyn
\par 40 og pålægge eder evig Skændsel og Spot, som aldrig glemmes.

\chapter{24}

\par 1 HERREN lod mig skue et syn, og se, der var to kurve, som stod foran HERRENs Tempel: det var, efter at Kong Nebukadrezar af Babel havde bortført Jojakims Søn, Kong Jekonja af Juda, og Judas Fyrster, Kunsthåndværkerne og Smedene fra Jerusalem til Babel.
\par 2 Den ene kurv indeholdt såre gode Figener, så gode som tidligmodne, den anden såre slette Figener, så slette, at de ikke kunde spises.
\par 3 Og HERREN sagde til mig: "Hvad ser du, Jeremias?" Jeg svarede: "Figener! De gode er såre gode og de slette såre slette, så slette, at de ikke kan spises."
\par 4 Da kom HERRENs Ord til mig således:
\par 5 Så siger HERREN, Israels Gud: Som man ser på disse gode Figener, vil jeg se på de bortførte Judæere, som jeg drev bort fra dette Sted til Kaldæernes Land.
\par 6 Jeg vil fæste mine Øjne på dem med Velbehag og føre dem hjem til dette Land. Jeg vil opbygge og ikke nedbryde dem, plante og ikke oprykke dem.
\par 7 Jeg giver dem Hjerte til at kende mig, at jeg er HERREN; de skal være mit Folk, og jeg vil være deres Gud, når de omvender sig til mig af hele deres Hjerte.
\par 8 Men som man, gør med de slette Figener, for slette til at spises, vil jeg, så siger HERREN, gøre med Kong Zedekias af Juda og hans Fyrster og Resten af Jerusalem, dem, der er levnet i dette Land, og dem, der bor i Ægypten;
\par 9 jeg gør dem til Rædsel for alle Jordens Riger, til Spot og Mundheld, til Hån og til et Forbandelsens Tegn på alle de Steder, hvorhen jeg bortstøder dem;
\par 10 jeg sender Sværd, Hunger og Pest imod dem, indtil de er udryddet af det Land, jeg gav dem og deres Fædre.

\chapter{25}

\par 1 Det Ord, som kom til Jeremias om alt Judases folk i Joasias Søns, Kong Jojakim af Judas, fjerde År, det er Kong Nebukadrezar af Babels første År,
\par 2 og som Profeten Jeremias talte til alt Judas Folk og alle Jerusalems Borgere:
\par 3 Fra Amons Søns, Kong Josias af Judas, trettende År til den Dag i Dag, i fulde tre og tyve År er HERRENs Ord kommet til mig, og jeg, talte til eder årle og silde, men I hørte ikke;
\par 4 og HERREN sendte årle og silde alle sine Tjenere Profeterne til eder, men I hørte ikke; I bøjede ikke eders Øre til at høre,
\par 5 når han sagde: "Omvend eder, hver fra sin onde Vej og sine onde Gerninger, at I fra Evighed til Evighed må bo i det Land, jeg gav eder og eders Fædre;
\par 6 og hold eder ikke til andre Guder, så I dyrker og tilbeder dem, og krænk mig ikke med eders Hænders Værker til eders Ulykke."
\par 7 Nej, I hørte mig ikke, lyder det fra HERREN, og så krænkede I mig med eders Hænders Værker til eders Ulykke.
\par 8 Derfor, så siger Hærskarers HERRE: Fordi I ikke vilde høre mine Ord,
\par 9 vil jeg sende Bud efter alle Nordens Stammer, lyder det fra HERREN, og til kong Nebukadrezar af Babel, min Tjener, og lade dem komme over dette Land og dets Indbyggere og over alle Folkene heromkring, og jeg vil ødelægge dem og gøre dem til Rædsel, Latter og Spot for evigt.
\par 10 Jeg fjerner fra dem Fryderåb og Glædesråb, Brudgoms Røst og Bruds Røst, Kværnens Lyd og Lampens Skin,
\par 11 og hele dette Land skal blive til Ørk og Øde, og disse Folkeslag skal trælle for Babels konge i halvfjerdsindstyve År.
\par 12 Men når der er gået halvfjerdsindstyve År, hjemsøger jeg Babels Konge og Folket der for deres Misgerning, lyder det fra HERREN, også Kaldæernes Land hjemsøger jeg og gør det til evige Ørkener,
\par 13 og jeg opfylder på dette Land alle mine Ord, som jeg har talet imod det, alt, hvad der er skrevet i denne Bog, alt, hvad Jeremias har profeteret mod alle Folkene.
\par 14 Thi også dem skal mange Folk og vældige Konger gøre til Trælle, og jeg gengælder dem deres Gerning og deres Hænders Værk.
\par 15 Thi således sagde HERREN, Israels Gud, til mig: "Tag dette Bæger med min Vredes Vin af min Hånd og giv alle de Folk, jeg sender dig til, at drikke deraf;
\par 16 de skal drikke og rave og rase for Sværdet, jeg sender iblandt dem!"
\par 17 Og jeg tog Bægeret af HERRENs Hånd og gav alle de Folk, han sendte mig til, at drikke deraf:
\par 18 Jerusalem og Judas Byer og dets Konger og Fyrster, for at gøre dem til Ørk og Øde, til Spot og til et Forbandelsens Tegn, som det er på denne Dag;
\par 19 Farao, Ægypterkongen, med alle hans Tjenere og Fyrster og alt hans Folk,
\par 20 alt Blandingsfolket og alle konger i Uz og Filisterland, Askalon, Gaza og Ekron og Asdods Rest;
\par 21 Edom, Moab og Ammoniterne;
\par 22 alle Tyruss og Zidons Konger og den fjerne strands Konger hinsides Havet;
\par 23 Dedan, Tema og Buz og alle dem med rundklippet Hår;
\par 24 alle Arabernes konger og alle Blandingsfolkets Konger, som hor i Ørkenen;
\par 25 alle Zimris Konger, alle Elams Konger og alle Mediens Konger;
\par 26 alle Nordens Konger, nær og fjern, den ene efter den anden. alle Riger på Jordens Overflade; og Kongen af Sjesjak skal drikke efter dem.
\par 27 Og du skal sige til dem: Så siger Hærskarers HERRE, Israels Gud: Drik, bliv drukne og spy, fald og rejs eder ikke mere for Sværdet jeg sender iblandt eder!
\par 28 Og hvis de vægrer sig ved at tage Bægeret af din Hånd og drikke, skal du sige til dem: Så siger Hærskarers HERRE: Drikke skal I!
\par 29 Thi se, med den By, mit Navn er nævnet over, begynder jeg at handle ilde, og så skulde I gå fri! Nej, I går ikke fri; thi jeg kalder Sværdet hid mod alle dem, som bor på Jorden, lyder det fra Hærskarers HERRE.
\par 30 Og du skal profetere alle disse Ord for dem og sige: HERREN brøler fra det høje, løfter sin Røst fra sin hellige Bolig; han brøler over sin Græsgang, istemmer Vinperserråbet over alle, som bor på Jorden.
\par 31 Drønet når til Jordens Ende, thi HERREN går i Rette med Folkene; over alt Kød holder han Dom, de gudløse giver han til Sværdet, lyder det fra HERREN.
\par 32 Thi så siger Hærskarers HERRE: Se, Ulykken går fra det ene Folk til det andet, et vældigt Vejr bryder løs fra Jordens Rand.
\par 33 HERRENs slagne skal på den Dag ligge fra Jordens ene Ende til den anden; der skal ikke holdes Klage over dem, og de skal ikke sankes og jordes; de skal blive til Gødning på Marken.
\par 34 Jamrer, I Hyrder, og skrig, I Hjordens ypperste, vælt jer i Støvet! Thi Tiden, I skal slagtes, er kommet, som en kostelig Skål skal I splintres.
\par 35 Hyrderne finder ej Tilflugt, ej Hjordens ypperste Redning.
\par 36 Hør, hvor Hyrderne skriger, hvor Hjordens ypperste jamrer! Thi HERREN hærger deres Græsgange,
\par 37 og Fredens Vange lægges øde for HERRENs glødende Vrede;
\par 38 Løven går bort fra sin Tykning, thi deres Land er lagt øde for det hærgende Sværd, for HERRENs glødende Vrede.

\chapter{26}

\par 1 I Joasiasses søns, kong Jojakim af Judas, første regeringstid kom dette ord fra HERRN:
\par 2 Så siger HERREN: Stå frem i Forgården til HERRENs Hus og tal til hele Juda, som kommer for at tilbede i HERRENs Hus, alle de Ord, jeg har pålagt dig at tale til dem; udelad ikke et Ord!
\par 3 Måske hører de og omvender sig, hver fra sin onde Vej, så jeg kan angre det onde, jeg har i Sinde at gøre dem for deres onde Gerningers Skyld.
\par 4 Sig til dem: Så siger HERREN: Hvis I ikke hører mig og følger den Lov, jeg har forelagt eder,
\par 5 så I hører mine Tjenere Profeternes Ord, som jeg årle og silde sendte eder, skønt I ikke vilde høre,
\par 6 så gør jeg med dette Hus som med Silo og giver alle Jordens Folk denne By at forbande ved.
\par 7 Præsterne, Profeterne og alt Folket hørte nu Jeremias tale disse Ord i HERRENs Hus;
\par 8 og da Jeremias havde sagt alt, hvad HERREN havde pålagt ham at sige til alt Folket, greb Præsterne og Profeterne og alt Folket ham og sagde: "Du skal dø !
\par 9 Hvor tør du profetere i HERRENs Navn og sige: Det skal gå dette Hus som Silo, og denne By skal ødelægges, så ingen bor der!" Og alt Folket stimlede sammen om Jeremias i HERRENs Hus.
\par 10 Da Judas Fyrster hørte det, gik de fra Kongens Palads op til HERRENs Hus og tog Sæde ved Indgangen til HERRENs nye Port.
\par 11 Så sagde Præsterne og Profeterne til Fyrsterne og alt Folket: "Denne Mand har gjort halsløs Gerning, thi han har profeteret mod denne By, som I selv hørte."
\par 12 Men Jeremias sagde til Fyrsterne og alt Folket: "HERREN sendte mig for at profetere mod dette Hus og denne By alle de Ord, I hørte.
\par 13 Bedrer dog eders Veje og eders Gerninger og hør på HERREN eders Guds Røst, at HERREN må angre det onde, han har talet imod eder.
\par 14 Men se, jeg er i eders Hånd; gør med mig, hvad der er godt og billigt i eders Øjne!
\par 15 Dog skal I vide, at hvis I dræber mig, så bringer I uskyldigt Blod over eder og denne By og dens Indbyggere; thi sandelig sendte HERREN mig for at tale alle disse ord til eder."
\par 16 Da sagde Fyrsterne og alt Folket til Præsterne og Profeterne: "Denne Mand har ikke gjort halsløs Gerning, men talt til os i HERREN vor Guds Navn."
\par 17 Og nogle af Landets Ældste trådte frem og sagde til hele Folkets Forsamling:
\par 18 "Mika fra Moresjet profeterede på Kong Ezekias af Judas Tid og sagde til alt Judas Folk: Så siger Hærskarers HERRE: Zion skal pløjes som en Mark, Jerusalem blive til Grushobe, Tempelbjerget til Krathøj.
\par 19 Mon Kong Ezekias af Juda og hele Juda dræbte ham? Frygtede de ikke HERREN og bad ham om Nåde, så HERREN angrede det onde, han havde truet dem med? Vi er ved at bringe stor Ulykke over vore Sjæle."
\par 20 Der var også en anden Mand, som profeterede i HERRENs Navn, Urija, Sjemajas Søn, fra Kirjat Jearim; og han profeterede mod denne By og dette Land med de samme Ord som Jeremias.
\par 21 Da Kong Jojakim og alle hans Krigsfolk og alle Fyrsterne hørte hans Ord, stod han ham efter Livet; og da Urija hørte det, blev han bange og flygtede og kom til Ægypten.
\par 22 Men Kong Jojakim sendte Folk til Ægypten; han sendte Elnatan, Akbors Søn, og nogle andre til Ægypten,
\par 23 og de bragte Urija hjem fra Ægypten og førte ham til Kong Jojakim, som lod ham hugge ned med Sværdet og hans Lig kaste hen, hvor Småfolk havde deres Grave.
\par 24 Men Ahikam, Sjafans Søn, holdt Hånden over Jeremias, så han ikke blev overgivet i Folkets Hånd og dræbt.

\chapter{27}

\par 1 I Josiass Søns, Kong Jojakim af Judas, første regeringstid kom dette Ord til Jeremias fra HERREN:
\par 2 Således sagde HERREN til mig: Gør dig Reb og Ågstænger og læg dem på din Hals
\par 3 og send Edoms, Moabs, Ammoniternes, Tyruss og Zidons Konger Bud ved deres Sendemænd, som er kommet til Kong Zedekias af Juda i Jerusalem;
\par 4 byd dem at sige til deres Herrer: Så siger Hærskarers HERRE. Israels Gud: Sig til eders Herrer:
\par 5 Jeg skabte Jorden og Menneskene og Kvæget på Jorden ved min vældige Styrke og min udrakte Hånd, og jeg giver den, til hvem jeg finder for godt.
\par 6 Og nu giver jeg alle disse Lande i min Tjener Kong Nebukadnezar af Babels Hånd, selv Markens Vildt giver jeg hen til at trælle for ham.
\par 7 Alle Folk skal trælle for ham, hans Søn og Sønnesøn, indtil også hans Lands Time slår og mange Folkeslag og store Konger gør ham til deres Træl.
\par 8 Og det Folk og det Rige, som ikke vil trælle for ham, Kong Nebukadnezar af Babel, og bøje Hals under Babels Konges Åg, det vil jeg hjemsøge med Sværd, Hunger og Pest, lyder det fra HERREN, til det er tilintetgjort ved hans Hånd.
\par 9 I skal ikke høre på eders Profeter og Spåmænd, eders Drømmere, Sandsigere og Troldmænd, som siger til eder: "I skal ikke komme til at trælle for Babels Konge;
\par 10 thi det er Løgn, de profeterer for eder for at få eder bort fra eders Jord, idet jeg da driver eder bort og I går til Grunde.
\par 11 Men det Folk, der bøjer Hals under Babels Konges Åg og træller for ham, vil jeg lade blive på sin Jord, lyder det fra HERREN, så det kan dyrke den og bo der.
\par 12 Og til Kong Zedekias af Juda talte jeg i Overensstemmelse med alle disse Ord: Bøj Hals under Babels Konges Åg og træl for ham og hans Folk, så skal I leve.
\par 13 Hvorfor vil du og dit Folk dø ved Sværd, Hunger og Pest, således som HERREN truede det Folk, der ikke vil trælle for Babels Konge?
\par 14 Hør ikke på Profeternes Ord, når de siger til eder: "I skal ikke komme til at trælle for Babels Konge"; thi Løgn profeterer de eder.
\par 15 Jeg har ikke sendt dem, lyder det fra HERREN, og de profeterer Løgn i mit Navn, for at jeg skal bortstøde eder, så I går til Grunde sammen med Profeterne, der profeterer for eder.
\par 16 Og til Præsterne og alt dette Folk talte jeg således: Så siger HERREN: Hør ikke på eders Profeters Ord, når de profeterer for eder og siger: "Se, HERRENs Huss Kar skal nu snart føres hjem fra Babel." Thi Løgn profeterer de eder.
\par 17 Hør dem ikke, men træl for Babels Konge, så skal I leve. Hvorfor skal denne By lægges øde?
\par 18 Er de Profeter og har HERRENs Ord, så lad dem gå i forbøn hos Hærskarers HERRE, at de Kar, der er tilbage i HERRENs Hus og Judas konges Palads, ikke også skal komme til Babel.
\par 19 Thi så siger Hærskarers HERRE om Søjlerne, Havet og Stellene og om de sidste Kar, der er tilbage i denne By,
\par 20 dem, som Kong Nebukadnezar af Babel ikke tog med, da han bortførte Jojakims søn, Kong Jekonja af Juda, fra Jerusalem til Babel med alle de ypperste i Juda og Jerusalem,
\par 21 ja, så siger Hærskarers HERRE, Israels Gud, om de kar, der er tilbage i HERRENs Hus og Judas Konges Palads og i Jerusalem:
\par 22 De skal føres til Babel, og der skal de blive, til den Dag jeg tager mig af dem og fører dem op og bringer dem tilbage hertil, lyder det fra HERREN.

\chapter{28}

\par 1 Kong Zedekias af Judas fjerde regeringsår i den femte måned sagde Profeten Hananja, Azzurs Søn, fra Gibeon til mig i HERRENs Hus i Præsternes og alt Folkets Nærværelse:
\par 2 "Så siger Hærskarers HERRE, Israels Gud: Jeg har sønderbrudt Babels Konges Åg.
\par 3 Om to År fører jeg tilbage hertil alle HERRENs Huss Kar, som Kong Nebukadnezar af Babel tog herfra og førte til Babel;
\par 4 og Jojakims Søn, Kong Jekonja af Juda, og alle de landflygtige fra Juda, som kom til Babel, fører jeg tilbage hertil, lyder det fra HERREN; thi jeg sønderbryder Babels Konges Åg."
\par 5 Profeten Jeremias svarede Profeten Hananja i Nærværelse af Præsterne og alt Folket, som stod i HERRENs Hus,
\par 6 således: "Amen! Måtte HERREN gøre således og stadfæste, hvad du har profeteret, og føre HERRENs Huss Kar og alle de landflygtige fra Babel tilbage hertil!
\par 7 Men hør dog dette Ord, som jeg vil tale til dig og alt Folket:
\par 8 De Profeter, som levede før mig og dig fra Fortids Dage, profeterede mod mange Lande og mægtige Riger om krig, Hunger og Pest;
\par 9 men når en Profet profeterer om Fred, kendes den Profet, HERREN virkelig har sendt, på at hans Ord går i Opfyldelse."
\par 10 Så rev Profeten Hananja Ågstængerne af Profeten Jeremiass Hals og sønderbrød dem;
\par 11 og Hananja sagde i alt Folkets Nærværelse: "Så siger HERREN: Således sønderbryder jeg om to År Kong Nebukadnezar af Babels Åg og tager det fra alle Folkenes Hals." Men Profeten Jeremias gik sin Vej.
\par 12 Men efter at Profeten Hananja havde sønderbrudt Ågstængerne og revet dem af Profeten Jeremiass Hals, kom HERRENs Ord til Jeremias således:
\par 13 "Gå hen og sig til Hananja: Så siger HERREN: Du har sønderbrudt Ågstænger af Træ, men jeg vil lave Ågstænger af Jern i Stedet.
\par 14 Thi så siger Hærskarers HERRE, Israels Gud: Et Jernåg lægger jeg på alle disse Folks Hals, at de må trælle for Kong Nebukadnezar af Babel; de skal trælle for ham, selv Markens Vildt har jeg givet ham."
\par 15 Så sagde Profeten Jeremias til Profeten Hananja: "Hør, Hananja! HERREN har ikke sendt dig, og du har fået dette Folk til at slå Lid til Løgn.
\par 16 Derfor, så siger HERREN: Se, jeg slænger dig bort fra Jordens Flade; du skal dø i År, thi du har prædiket Frafald fra HERREN."
\par 17 Og Profeten Hananja døde samme År i den syvende Måned.

\chapter{29}

\par 1 Følgende er indholdet af det brev, profeten Jeramias sendte fra Jerusalem til de Ældste, som var tilbage blandt de bortførte, og til Præsterne og Profeterne og alt Folket, som Nebukadnezar havde ført fra Jerusalem til Babel,
\par 2 efter at Kong Jekonja, Herskerinden, Hofmændene, Judas og Jerusalems Fyrster, Kunsthåndværkerne og Smedene havde forladt Jerusalem,
\par 3 ved Elasa, Sjafans Søn, og Gemarja, Hilkijas Søn, som kong Zedekias af Juda sendte til Babel, til Kong Nebukadnezar af Babel.
\par 4 Så siger Hærskarers HERRE, Israels Gud, til alle de landflygtige, som jeg førte fra Jerusalem til Babel:
\par 5 Byg Huse og bo deri, plant Haver og spis deres Frugt,
\par 6 tag eder Hustruer og avl Sønner og Døtre, tag Hustruer til eders Sønner og bortgift eders Døtre, at de kan føde Sønner og Døtre, bliv mange der og ikke færre;
\par 7 og lad det Lands Vel, til hvilket jeg har ført eder, ligge eder på Sinde, og bed for det til HERREN; thi når det går det godt, går det også eder godt.
\par 8 Thi så siger Hærskarers HERRE, Israels Gud: Lad ikke de Profeter, som er iblandt eder, eller eders Spåmænd bilde eder noget ind, og lyt ikke til de Drømme, I drømmer;
\par 9 thi Løgn profeterer de eder i mit Navn; jeg har ikke sendt dem, lyder det fra HERREN.
\par 10 Thi så siger HERREN: Når halvfjerdsindstyve År er gået for Babel, vil jeg se til eder og på eder opfylde min Forjættelse om at føre eder tilbage hertil.
\par 11 Thi jeg ved, hvilke Tanker jeg tænker om eder, lyder det fra HERREN, Tanker om Fred og ikke om Ulykke, at jeg må give eder Fremtid og Håb.
\par 12 Kalder I på mig, vil jeg svare eder; beder I til mig, vil jeg høre eder;
\par 13 leder I efter mig, skal I finde mig; såfremt I søger mig af hele eders Hjerte,
\par 14 vil jeg lade mig finde af eder, lyder det fra HERREN, og vende eders Skæbne og sanke eder sammen fra alle de Folkeslag og alle de Steder, jeg har bortstødt eder til, lyder det fra HERREN, og føre eder tilbage til det Sted, fra hvilket jeg førte eder bort.
\par 15 Men når I siger: "HERREN har opvakt os Profeter i Babel
\par 16 Thi så siger HERREN om Kongen der sidder på Davids Trone, og om alt Folket, der bor i denne By, eders Brødre, som ikke drog. i Landflygtighed med eder,
\par 17 så siger Hærskarers HERRE: Se, jeg sender Sværd, Hunger og Pest over dem og gør dem som de usle Figener, der er for dårlige at spise;
\par 18 jeg forfølger dem med Sværd, Hunger og Pest og gør dem til Rædsel for alle Jordens Riger, til Forbandelsesord, til Gru, Spot og Spe blandt alle de Folk, jeg bortstøder dem til,
\par 19 til Straf fordi de ikke hørte mine Ord, lyder det fra HERREN, når jeg årle og silde sendte mine Tjenere Profeterne til dem, men de vilde ikke høre, lyder det fra HERREN.
\par 20 Men hør dog HERRENs Ord, alle I landflygtige, som jeg sendte fra Jerusalem til Babel!
\par 21 Så siger Hærskarers HERRE, Israels Gud, om Aab, Kolajas Søn, og Zidkija, Maasejas Søn, som profeterer eder Løgn i mit Navn: Se, jeg giver dem i Kong Nebukadrezar af Babels Hånd, og han skal lade dem hugge ned for eders Øjne,
\par 22 og de skal bruges af alle de landflygtige fra Juda i Babel til at forbande ved, idet man skal sige: "HERREN gøre med dig som med Zidkija og Aab, hvem Babels Konge lod stege i Ild!"
\par 23 Thi de øvede dårskab i Israel og bedrev Hor med deres Landsmænds Kvinder og talte i mit Navn løgnagtige Ord, som jeg ikke havde bedt dem at tale; jeg ved det og kan vidne det, lyder det fra HERREN.
\par 24 Til Nehelamiten Sjemaja skal du sige:
\par 25 Så siger Hærskarers HERRE, Israels Gud: Fordi du i dit eget Navn har sendt alt Folket i Jerusalem og Præsten Zefanja, Maasejas Søn, og alle Præsterne et så lydende Brev:
\par 26 "HERREN har gjort dig til Præst i Præsten Jojadas Sted til i HERRENs Hus at have Opsyn med alle gale og Folk i profetisk Henrykkelse, hvilke du skal lægge i Blok og Halsjern.
\par 27 Hvorfor skrider du da ikke ind mod Jeremias fra Anatot, der profeterer hos eder?
\par 28 Nu har han kunnet sende Bud til os i Babel og ladet sige: Det trækker i Langdrag! Byg Huse og bo deri, plant Haver og spis deres Frugt!"
\par 29 Dette Brev læste Præsten Zefanja for Profeten Jeremias.
\par 30 Da kom HERRENs Ord til Jeremias således:
\par 31 Send Bud til alle de landflygtige og sig: Så siger HERREN om Nehelamiten Sjemaja: Fordi Sjemaja har profeteret for eder, uden at jeg har sendt ham, og får eder til at slå Lid til Løgn,
\par 32 derfor, så siger HERREN: Se, jeg hjemsøger Nehelamiten Sjemaja og hans Efterkommere; han skal ingen have, der bor iblandt eder og oplever den Lykke, jeg giver eder, lyder det fra HERREN, fordi han har prædiket Frafald fra HERREN.

\chapter{30}

\par 1 Det ord, som kom til Jeramias fra Herren.
\par 2 Skriv alle de Ord, jeg har talet til dig, op i en Bog.
\par 3 Thi se Dage skal komme, lyder det fra HERREN, da jeg vender mit Folk Israels og Judas Skæbne, siger HERREN, og fører dem hjem til det Land, jeg gav deres Fædre, og de skal tage det i Eje.
\par 4 Dette er de Ord, HERREN talede til Israel og Juda.
\par 5 Så siger HERREN: Vi hørte et Udbrud af Skræk, af Rædsel og Ufred;
\par 6 spørg og se dog til, om en Mand kan føde! Hvi ser jeg da alle Mænd med Hånd på Hofte som Kvinde i Barnsnød og alle Åsyn blegne?
\par 7 Thi stor er denne Dag, den er uden Lige, en Trængselstid for Jakob, men fra den skal han frelses.
\par 8 På hin Dag, lyder det fra Hærskarers HERRE, vil jeg sønderbryde deres Åg og tage det af deres Hals og sprænge deres Bånd, og de skal ikke mere trælle for fremmede.
\par 9 De skal tjene HERREN deres Gud og David, deres Konge, som jeg vil oprejse dem.
\par 10 Frygt derfor ikke, min Tjener Jakob, lyder det fra HERREN, og vær ikke bange, Israel; thi se, jeg frelser dig fra det fjerne og dit Afkom fra deres Fangenskabs Land; og Jakob skal vende hjem og bo roligt og trygt, og ingen skal forfærde ham.
\par 11 Thi jeg er med dig, lyder det fra HERREN, for at frelse dig; thi jeg vil tilintetgøre alle de Folk, blandt hvilke jeg har spredt dig, men dig vil jeg ikke tilintetgøre; jeg vil tugte dig med Måde, ikke lade dig helt ustraffet.
\par 12 Thi så siger HERREN: Ulægeligt er dit Brud, dit Sår er svart.
\par 13 Ingen fører din Sag. For din Byld er ingen Lægedom, for dig ingen Helse.
\par 14 Alle dine Venner har glemt dig, søger dig ikke, thi med Fjendeslag slog jeg dig, med skånselløs Straf, fordi din Brøde var stor, dine Synder mange.
\par 15 Hvi skriger du over dit Brud, er dit Sår ulægeligt? Fordi din Brøde var stor, dine Synder mange, gjorde jeg dette imod dig.
\par 16 Derfor skal alle, som fortærer dig, fortæres, alle dine Fjender, alle skal de vandre i Fangenskab; de, der plyndrer dig, skal plyndres, til Ran gør jeg alle dine Ransmænd.
\par 17 Thi jeg heler dig, læger dine Sår, så lyder det fra HERREN; du kaldtes jo, Zion, "den bortstødte, som ingen søger."
\par 18 Så siger HERREN: Se, jeg vender Jakobs Skæbne, forbarmer mig over hans Boliger, Byen skal bygges på sin Høj, Paladset stå, bvor det stod.
\par 19 Fra dem skal Lovsang lyde og legendes Råb; de bliver ej færre, jeg gør dem mange; de bliver ej ringe, jeg giver dem Hæder.
\par 20 Hans Sønner skal blive som fordum, hans Menighed stå fast for mit Åsyn.
\par 21 Hans Fyrste stammer fra ham selv, hans Hersker går frem af hans Midte. Jeg lader ham komme mig nær, han skal nærme sig mig; thi hvem ellers sætter Livet i Vov ved at nærme sig mig? lyder det fra HERREN.
\par 22 I skal være mit Folk, og jeg vil være eders Gud.
\par 23 Se, HERRENs Stormvejr, Vreden er brudt løs, et hvirvlende Stormvejr; det hvirvler hen over de gudløses Hoved.
\par 24 HERRENs glødende Vrede lægger sig ikke, før han har udført og fuldbyrdet sit Hjertes Tanker; i de sidste dage skal I forstå det.

\chapter{31}

\par 1 Til hin tid, lyder det fra Herren, vil jeg være alle Israel slægters Gud, og de skal være mit Folk.
\par 2 Så siger HERREN: Folket, der undslap Sværdet, fandt Nåde i Ørkenen, Israel vandred til sin Hvile,
\par 3 i det fjerne åbenbarede HERREN sig for dem: Jeg elsked dig med evig Kærlighed, drog dig derfor i Nåde.
\par 4 Jeg bygger dig atter, du skal bygges, Israels Jomfru, igen skal du smykkes med Håndpauke, gå med i de legendes Dans.
\par 5 Vin skal du atter plante på Samarias Bjerge, plante skal du og høste.
\par 6 Thi en Dag skal Vogterne råbe på Efraims Bjerge: "Kom, lad os drage til Zion, til HERREN vor Gud!"
\par 7 Thi så siger HERREN: Fryd jer over Jakob med Glæde, jubl over det første blandt Folkene, kundgør med Lovsang og sig: "HERREN har frelst sit Folk, Israels Rest."
\par 8 Se, jeg bringer dem hid fra Nordens Land, samler dem fraJordens Afkroge; iblandt dem er blinde og lamme, frugtsommelige sammen med fødende, i en stor Forsamling vender de hjem.
\par 9 Se, de kommer med Gråd; mens de ydmygt beder, leder jeg dem; jeg fører dem hen til Vandløb ad en jævn Vej, hvor de ej snubler; thi jeg er Israel en Fader, min førstefødte er Efraim.
\par 10 Hør HERRENs Ord, I Folk, forkynd på fjerne Strande: Han, som spredte Israel, samler det, vogter det som Hyrden sin Hjord;
\par 11 thi HERREN har udfriet Jakob, genløst det af den stærkeres Hånd.
\par 12 De kommer til Zions bjerg og jubler over HERRENs Fylde, over Kom og Most og Olie og over Lam og Kalve. Deres Sjæl er som en vandrig Have, de skal aldrig vansmægte mer.
\par 13 Da fryder sig Jomfru i Dans, Yngling og Olding tilsammen. Jeg vender deres Kummer til Fryd, giver Trøst og Glæde efter Sorgen.
\par 14 Jeg kvæger Præsterne med Fedt, mit Folk skal mættes med min Fylde, lyder det fra HERREN.
\par 15 Så siger HERREN: En Klagerøst høres i Rama, bitter Gråd, Rakel begræder sine Børn, vil ikke trøstes over sine Børn, fordi de er borte.
\par 16 Så siger HERREN: Din Røst skal du holde fra Gråd, dine Øjne fra Tårer, thi derer Løn fordin Møje,lyder det fra HERREN; fra Fjendeland vender de hjem;
\par 17 og der er Håb for din Fremtid, lyder det fra HERREN, Børn vender hjem til deres Land.
\par 18 Jeg hører grant, hvor Efraim klager: "Du tugted mig, og jeg blev tugtet som en utæmmet Kalv; omvend mig, så bliver jeg omvendt, thi du er HERREN min Gud.
\par 19 Thi nu jeg er omvendt, angrer jeg; nu jeg har besindet mig, slår jeg mig på Hofte; jeg er skamfuld og beskæmmet, thi jeg bærer min Ungdoms Skændsel."
\par 20 Er Efraim min dyrebare Søn, mit Yndlingsbarn? thi så tit jeg taler om ham, må jeg mindes ham kærligt, derfor bruser mit indre, jeg ynkes over ham, lyder det fra HERREN.
\par 21 Rejs dig Vejvisersten, sæt Mærkesten op, ret din Tanke på Højvejen, Vejen, du gik, vend hjem, du Israels Jomfru, til disse dine Byer!
\par 22 Hvor længe vil du dog tøve, du frafaldne Datter? Thi HERREN skaber nyt i Landet: Kvinde værner om Mand.
\par 23 Så siger Hærskarers HERRE, Israels Gud: End skal de i Judas Land og Byer sige dette Ord, når jeg vender deres Skæbne: "HERREN velsigne dig, du Retfærds Bolig, du hellige Bjerg!"
\par 24 Og deri skal Juda bo og alle dets Byer til Hobe, Agerdyrkerne og de omvankende Hyrder.
\par 25 Thi jeg kvæger den trætte Sjæl og mætter hver vansmægtende Sjæl.
\par 26 (Herved vågnede jeg og så mig om, og Søvnen havde været mig sød.)
\par 27 Se, Dage skal komme, lyder det fra HERREN, da jeg tilsår Israels Hus og Judas Hus med Sæd at Mennesker og Kvæg.
\par 28 Og som jeg har været årvågen over dem for at oprykke, nedbryde, omstyrte, ødelægge og gøre ilde, således vil jeg være årvågen over dem for at bygge og plante, lyder det fra HERREN.
\par 29 I hine Dage skal man ikke mere sige: Fædre åd sure Druer, og Børnenes Tænder blev ømme.
\par 30 Nej, enhver skal dø for sin egen Brøde; enhver, der æder sure Druer, får selv ømme Tænder.
\par 31 Se, Dage skal komme, lyder det fra HERREN, da jeg slutter en ny Pagt med Israels Hus og Judas Hus,
\par 32 ikke som den Pagt jeg sluttede med deres Fædre, dengang jeg tog dem ved Hånden for at føre dem ud af Ægypten, hvilken Pagt de brød, så jeg væmmedes ved dem, lyder det fra HERREN;
\par 33 nej, dette er den Pagt, jeg efter hine Dage slutter med Israels Hus, lyder det fra HERREN: Jeg giver min Lov i deres Indre og skriver den på deres Hjerter, og jeg vil være deres Gud, og de skal være mit Folk.
\par 34 Ven skal ikke mere lære sin Ven eller Broder sin Broder og sige: "Kend HERREN!" Thi de skal alle kende mig fra den mindste til den største, lydet det fra HERREN; thi jeg tilgiver deres Brøde og kommer ikke mer deres Synd i Hu.
\par 35 Så siger HERREN, han, som satte Solen til at lyse om Dagen og Månen og Stjernerne til at lyse om Natten, han, som oprører Havet, så Bølgerne bruser, han, hvis Navn er Hærskarers HERRE:
\par 36 Når disse Ordninger viger fra mit Åsyn, lyder det fra HERREN, så skal også Israels Æt for alle Tider ophøre at være et Folk for mit Åsyn.
\par 37 Så siger HERREN: Når Himmelen oventil kan udmåles og Jordens Grundvolde nedentil udgranskes, så vil jeg også forkaste Israels Æt for alt, hvad de har gjort, lydet det fra HERREN.
\par 38 Se, Dage skal komme, lyder det fra HERREN, da Byen skal opbygges for HERREN fra Hananeltårnet til Hjørneporten;
\par 39 og videre skal Målesnoren gå lige ud til Garebs Høj og så svinge mod Goa;
\par 40 og hele Dalen, Ligene og Asken, og alle Markerne ned til Kedrons Bæk, til Hesteportens Hjørne mod Øst skal være HERREN helliget; det skal aldrig mere oprykkes eller nedlbrydes.

\chapter{32}

\par 1 Det Ord, som kom til Jeramias fra Herren i kong Zedekias af Judas tiende År, det er Nebukadrezars attende.
\par 2 Dengang belejrede Babels Konges Hær Jerusalem, og Profeten Jeremas sad fængslet i Vagtforgården i Judas Konges Palads,
\par 3 hvor Kong Zedekias af Juda havde ladet ham fængsle med de Ord: "Hvor tør du profetere og sige: Så siger HERREN: Se, jeg giver denne By i Babels Konges Hånd, og han skal indtage den;
\par 4 og Kong Zedekias af Juda skal ikke undslippe Kaldæernes Hånd, men overgives i Babels Konges Hånd, og han skal tale med ham Mund til Mund og se ham Øje i Øje;
\par 5 og han skal føre Zedekias til Babel, og der skal han blive, til jeg ser til ham, lyder det fra HERREN; når I kæmper med Kaldæerne, får I ikke Lykke!"
\par 6 Og Jeremias sagde: HERRENs Ord kom til mig således:
\par 7 Se, Hanamel, din Farbroder Sjallums Søn, kommer til dig og siger: "Køb min Mark i Anatot, thi du har Indløsningsret."
\par 8 Så kom Hanamel, min Farbroders Søn, til mig i Vagtforgården, som HERREN havde sagt, og sagde til mig: "Køb min Mark i Anatot i Benjamins Land, thi du har Arveretten, og indløsningsretten er din; køb dig den!" Da forstod jeg, at det var HERRENs Ord.
\par 9 Og jeg købte Marken i Anatot af Hanamel, min Farbroders Søn, og tilvejede ham Pengene, sytten Sekel Sølv;
\par 10 og jeg skrev Skøde og forseglede det, tilkaldte Vidner og afvejede Pengene på Vægtskål.
\par 11 Så tog jeg Skødet, både det forseglede og det åbne,
\par 12 og overgav Skødet til Baruk, Masejas Søn Nerijas Søn, i Nærværelse af Hanamel, min Farbroders Søn, og Vidnerne, som havde underskrevet Skødet, og alle de Judæee, som var til Stede i Vagtforgården;
\par 13 og i deres Nærværelse bød jeg Baruk:
\par 14 "Så siger Hærskarers HERRE, Israels Gud: Tag disse Skøder, både det forseglede og det åbne, og læg dem i en Lerkrukke, for at de kan holde sig i lange Tider.
\par 15 Thi så siger Hærskarers HERRE, Israels Gud: End skal der købes Huse, Marker og Vingårde i dette Land!"
\par 16 Efter at have overgivet Skødet til Baruk, Nerijas Søn, bad jeg således til HERREN:
\par 17 Ak, Herre, HERRE, du har jo skabt Himmelen og Jorden ved din vældige Styrke og din udstrakte Arm, intet er dig for underfuldt,
\par 18 du, som øver Miskundhed mod Tusinder og gengælder Fædres Misgerning på deres Sønner efter dem; du store, vældige Gud, hvis Navn er Hærskarers HERRE,
\par 19 rig på Råd og stor i Dåd, hvis Øjne er åbne over alle Menneskebørnenes Veje, for at du kan give enhver efter hans Vej og hans Gerningers Frugt;
\par 20 du, som gjorde Tegn og Undere i Ægypten og gør det den Dag i Dag både i Israel og blandt andre Mennesker og skabte dig det Navn, du har i Dag,
\par 21 du, som førte dit Folk Israel ud af Ægypten med Tegn og Undere, med stærk Hånd og udstrakt Arm og stor Rædsel
\par 22 og gav dem dette Land, som du havde svoret deres Fædre at ville give dem, et Land, der flyder med Mælk og Honning;
\par 23 og de kom og tog det i Eje; men de hørte ikke din Røst og adlød ikke din Lov; de gjorde intet af, hvad du havde pålagt dem; så lod du al denne Ulykke ramme dem.
\par 24 Se, Stormvoldene har nået Byen, så den er ved at blive indtaget, og med Sværd, Hunger og Pest er Byen givet i de angribende Kaldæeres Hånd; had du talede, er sket, og du ser det selv.
\par 25 Og skønt Byen er givet i Kaldæernes Hånd, siger du til mig, Herre, HERRE: "Køb dig Marken for Penge og tag Vidner derpå!"
\par 26 Da kom HERRENs Ord til Jeremias således:
\par 27 Se, jeg er HERREN, alt Køds Gud; skulde noget være mig for underfuldt?
\par 28 Derfor, så siger HERREN: Se, jeg giver denne By i kaldæernes og Kong Nebukadrezar af Babels Hånd, og han skal idtage den;
\par 29 og Kaldæerne, der angriber denne By, skal komme og sætte Ild på den og afbrænde Husene, på hvis Tage man tændte Offerild for Baal og udgød Drikofre for andre Guder for at krænke mig.
\par 30 Thi fra deres Ungdom af har Israeliterne og Judæerne kun gjort, hvad der var ondt i mine Øjne; thi Israeliterne gør ikke andet end krænke mig ved deres Hænders Værk, lyder det fra HERREN.
\par 31 Ja, en Kilde til Vrede og Harme har denne By været mig, lige fra den Dag de byggede den og til i Dag, så at jeg må fjerne den fra mit Åsyn
\par 32 for alt det ondes Skyld, som Israeliterne og Judæerne gjorde for at krænke mig, de deres Konger, Fyrster, Præster og Profefer, Judas Mænd og Jerusalems Borgere.
\par 33 De vendte Ryggen og ikke Ansigtet til mig, og skønt jeg advarede dem årle og silde, vilde de ikke høre eller tage ved Lære.
\par 34 De opstillede deres væmmelige Guder i det Hus, mit Navn nævnes over, for at gøre det urent;
\par 35 og de byggede Baals Offerhøje i Hinnoms Søns Dal for at ofre deres Sønner og Døtre til Molok, hvad jeg ikke havde budt dem, og hvad aldrig var i min Tanke, at man skulde gøre så vederstyggelig en Ting for derved at lokke Juda til Synd.
\par 36 Men nu, så siger HERREN, Israels Gud, om denne By, som I siger er givet i Babels Konges Hånd med Sværd, Hunger og Pest:
\par 37 Se, jeg vil samle dem fra alle de Lande, som jeg har bortstødt dem til i min Vrede og Harme og i stor Fortørnelse, og føre dem hjem til dette Sted og lade dem bo trygt.
\par 38 De skal være mit Folk, og jeg vil være deres Gud;
\par 39 og jeg vil give dem eet Hjerte og een Vej, så de frygter mig alle Dage, at det må gå dem og deres Sønner efter dem vel.
\par 40 Jeg slutter en evig Pagt med dem, at jeg ikke vil drage mig tilbage fra dem, men gøre vel itnod dem; og min Frygt lægger jeg i deres Hjerter, så de ikke viger fra mig.
\par 41 Jeg vil glæde mig over dem og gøre vel imod dem; og jeg planter dem i dette Land i Trofasthed af hele mit Hjerte og hele min Sjæl.
\par 42 Thi så siger HERREN: Som jeg bragte al denne store Ulykke over dette Folk, således vil jeg bringe over dem alt det gode, jeg taler til dem om.
\par 43 End skal der købes Marker i det Land, som I siger er en Ørken uden Mennesker og Kvæg og givet i Kaldæernes Hånd;
\par 44 man skal købe Marker for Penge og skrive Skøder og forsegle dem og tilkalde Vidner i Benjamins Land, i Jerusalems Omegn, i Judas Byer, i Bjerglandets, Lavlandets og Sydlandets Byer; thi jeg vender deres Skæbne, lyder det fra HERREN.

\chapter{33}

\par 1 Herrens ord kom anden gang til Jeremias, medens han endnu sad fængslet i Vagtforgården, således:
\par 2 Så siger HERREN, som skabte Jorden og dannede den, idet han grundfæstede den, han, hvis Navn er HERREN:
\par 3 Kald på mig, så vil jeg svare dig og kundgøre dig store og lønlige Ting, du ikke kender.
\par 4 Thi så siger HERREN, Israels Gud, om denne Bys Huse og om Judas Kongers Huse, som nedbrødes for at bruges til Volde og Mur,
\par 5 da man gav sig til at stride imod Kaldæerne, og som fyldtes med Ligene af de Mennesker, jeg slog i min Vrede og Harme, og for hvem jeg skjulte mit Åsyn for al deres Ondskabs Skyld:
\par 6 Se, jeg vil lade Byens Sår heles og læges, og jeg helbreder dem og oplader for dem en Rigdom af Fred og Sandhed.
\par 7 Jeg vender Judas og Israels Skæbne og opbygger dem som tilforn.
\par 8 Jeg renser dem for al deres Brøde, med hvilken de syndede imod mig, og tilgiver alle deres Misgerninger, med hvilke de syndede og forbrød sig imod mig.
\par 9 Byen skal blive til Glæde, til Pris og Ære blandt alle Jordens Folk; og når de hører om alt det gode, jeg gør den, skal de frygte og bæve over alt det gode og al den Lykke, jeg lader den times.
\par 10 Så siger HERREN: På dette Sted, som I siger er ødelagt, uden Mennesker og Kvæg, i Judas Byer og på Jerusalems Gader, der er lagt øde, uden Mennesker og kvæg,
\par 11 skal atter høres Fryderåb og Glædesråb, Brudgoms Røst og Bruds Røst, Råb af Folk, som siger: "Tak Hærskarers HERRE; thi HERREN er god, og hans Miskundhed varer evindelig!" og som bringer Takoffer til HERRENs Hus; thi jeg vender Landets Skæbne, så det bliver som tilforn, siger HERREN.
\par 12 Så siger Hærskarers HERRE: På dette ødelagte Sted, som er uden Mennesker og Kvæg, og i alle dets Byer skal der atter være Græsgange, hvor Hyrder lader deres Hjorde ligge;
\par 13 i Bjerglandets, Lavlandets og Sydlandets Byer, i Benjamins land, i Jerusalems Omegn og i Judas Byer skal Småkvæget atter gå forbi under Tællerens Hånd, siger HERREN.
\par 14 Se, Dage skal komme, lyder det fra HERREN, da jeg opfylder den Forjættelse, jeg udtalte om Israels og Judas Hus.
\par 15 I hine Dage og til hin Tid lader jeg en Retfærds Spire fremspire for David, og han skal øve Ret og Retfærd i Landet.
\par 16 I hine Dage skal Juda frelses og Jerusalem bo trygt, og man skal kalde det: HERREN vor Retfærdighed.
\par 17 Thi så siger HERREN: David skal ikke fattes en Mand til at sidde på Israels Huss Trone.
\par 18 Og Levitpræsterne skal aldrig fattes en Mand til at stå for mit Åsyn og frembære Brændoffer, brænde Afgrødeoffer og ofre Slagtoffer.
\par 19 Og HERRENs Ord kom til Jeremias således:
\par 20 Så siger HERREN Hvis min Pagt med Dagen og Natten brydes, så det ikke bliver Dag og Nat, når Tid er inde,
\par 21 da skal også min Pagt med min Tjener David brydes, så han ikke har nogen Søn til at sidde som Konge på sin Trone, og med Levitpræsterne, som tjener mig.
\par 22 Som Himmelens Hær ikke kan tælles og Havets Sand ikke måles, således vil jeg mangfoldiggøre min Tjener Davids Afkom og Leviterne, som tjener mig.
\par 23 Og HERRENs Ord kom til Jeremias således:
\par 24 Har du ikke lagt Mærke til, hvorledes dette Folk siger: "De to Slægter, HERREN udvalgte, har han forkastet!" Og de smæder mit Folk, fordi det i deres Øjne ikke 1ner er et Folk.
\par 25 Så siger HERREN: Hvis jeg ikke har fastsat min Pagt med Dag og Nat, givet Love for Himmel og Jord,
\par 26 så vil jeg også forkaste Jakobs Afkom og min Tjener David og ikke af hans Afkom tage Herskere oer Abrabams, Isaks og Jakobs Afkom; thi jeg vender deres Skæbne og forbarmer mig over dem.

\chapter{34}

\par 1 Det Ord, som kom til Jeremias fra Herren, da kong Nebukadnezar af Babel og hele hans Hær og alle Riger på Jorden, der stod under hans Herredømme, og alle Folkeslag angreb Jerusalem og alle dets Byer; det lød:
\par 2 Så siger HERREN, Israels Gud: Gå hen og sig til Kong Zedekias af Juda: Så siger HERREN: Se, jeg giver denne By i Babels Konges Hånd, og han skal afbrænde den.
\par 3 Og du skal ikke undslippe hans Hånd, men gribes og overgives i hans Hånd og se Babels Konge Øje til Øje, og han skal tale med dig Mund til Mund, og du skal komme til Babel.
\par 4 Hør dog HERRENs Ord, Kong Zedekias af Juda: Så siger HERREN om dig: Du skal ikke falde for Sværdet,
\par 5 men dø i Fred, og ligesom man brændte til Ære for dine Fædre, kongerne før dig, således skal man brænde til Ære for dig og klage over dig: "Ve, Herre!" så sandt jeg har talet, lyder det fra HERREN.
\par 6 Og Profeten Jeremias talte alle disse Ord til kong Zedekias af Juda i Jerusalem,
\par 7 medens Babels Konges Hær angreb Jerusalem og begge de Byer i Juda, der var tilbage, Lakisj og Azeka; thi disse faste Stæder var tilbage af Judas Byer.
\par 8 Det Ord, som kom til Jeremias fra HERREN, efter at kong Zedekias havde sluttet en Pagt med alt Folket i Jerusalem og udråbt Frigivelse,
\par 9 således at enhver skulde lade sin Træl og Trælkvinde gå bort i Frihed, såfremt de var Hebræere, og ikke mere lade en judæisk Broder trælle.
\par 10 Og alle Fyrsterne og alt Folket, som havde indgået Pagten om, at enhver skulde lade sin Træl og Trælkvinde gå bort i Frihed og ikke mere lade dem trælle, adlød; de adlød og lod dem gå.
\par 11 Men siden skiftede de Sind og tog Trællene og Trælkvinderne, som de havde ladet gå bort i Frihed, tilbage og tvang dem til at være Trælle og Trælkvinder.
\par 12 Da kom HERRENs Ord til Jeremias således:
\par 13 Så siger HERREN, Israels Gud: Jeg sluttede en Pagt med eders Fædre, dengang jeg førte dem ud af Ægypten, af Trællehuset, idet jeg sagde:
\par 14 "Når der er gået syv År, skal enhver af eder lade sin hebraiske Landsmand, som har solgt sig til dig og tjent dig i seks År, gå bort; du skal lade ham gå af din Tjeneste i Frihed!" Men eders Fædre hørte mig ikke og lånte mig ikke Øre.
\par 15 Nys omvendte I eder og gjorde, hvad der er ret i mine Øjne, idet I udråbte Frigivelse, hver for sin Broder, og I sluttede en Pagt for mit Åsyn i det Hus, mit Navn er nævnet over,
\par 16 men siden skiftede I Sind og vanhelligede mit Navn, idet enhver af eder tog sin Træl eller Trælkvinde tilbage, som I havde ladet gå bort i Frihed, om de ønskede det, og tvang dem til at være eders Trælle og Trælkvinder.
\par 17 Derfor, så siger HERREN: Da I ikke hørte mig og udråbte Frigivelse, hver for sin Broder og hver for sin Næste, vil jeg nu udråbe Frigivelse for eder, lyder det fra HERREN, så I hjemfalder til Sværd, Pest og Hunger, og jeg vil gøre eder til Rædsel for alle Jordens Riger.
\par 18 Og jeg giver de Mænd, som har overtrådt min Pagt og ikke holdt den Pagts Ord, som de sluttede for mit Åsyn, da de slagtede Kalven og skar den i to Stykker, mellem hvilke de gik,
\par 19 Judas og Jerusalems Fyrster, Hofmændene og Præsterne og hele Landets Befolkning, som gik mellem Stykkerne af Kalven
\par 20 dem giver jeg i deres Fjenders Hånd og i deres Hånd, som står dem efter Livet, og deres Lig skal blive Himmelens Fugle og Jordens Dyr til Æde.
\par 21 Og Kong Zedekias af Juda og hans Fyrster giver jeg i deres Fjenders Hånd og i deres Hånd, som står dem efter Livet. Og Babels Konges Hær, som drog bort fra eder,
\par 22 se, den bydet jeg, lyder det fra HERREN, at vende tilbage til denne By, og de skal angribe den og indfage og afbrænde den; og Judas Byer lægger jeg øde, så ingen bor der!

\chapter{35}

\par 1 Det Ord, som kom til Jeremias fra Herren i Joasiases søn kong Jojakim af Judas Dage:
\par 2 "Gå hen til Rekabiternes Hus og tal dem til, bring dem til et af Kamrene i HERRENs Hus og giv dem Vin at drikke!"
\par 3 Så hentede jeg Jaazanja, en Søn af Jirmeja, Habazzinjas Søn, og hans Brødre og alle hans Sønner og hele Rekabiternes Hus
\par 4 og bragte dem til HERRENs Hus, til den Guds Mand Hanans, Jigdaljahus Søns, Sønners Kammer ved Siden af Fyrsternes Kammer oven over Dørvogteren Maasejas, Sjallums Søns, Kammer.
\par 5 Og jeg satte krukker, som var fulde af Vin, og Bægre for dem og sagde: "Drik!"
\par 6 Men de svarede: Vi drikker ikke Vin, thi vor Fader Jonadab, Rekabs Søn, gav os det Bud: I og eders Børn må aldrig drikke Vin,
\par 7 ej heller bygge Huse eller så Korn eller plante eller eje Vingårde, men I skal bo i Telte hele eders Liv, for at I må leve længe i det Land, I bor i som frem1uede.
\par 8 Og vi har adlydt vor Fader Jonadab, Rekabs Søn, i alt, hvad han bød os, idet både vi, vore kvinder, Sønner og Døtre hele vort Liv afholder os fra at drikke Vin,
\par 9 bygge Huse at bo i og eje Vingårde, Marker eller Sæd,
\par 10 men bor i Telte; vi har adlydt og nøje gjort, som vor Fader Jonadab bød os.
\par 11 Men da Kong Nebukadrezar af Babel faldt ind i Landet, sagde vi: Kom, lad os ty til Jerusalem for Kaldæernes og Aramernes Hære! Og vi slog os ned i Jerusalem."
\par 12 Da kom HERRENs Ord til mig således
\par 13 Så siger Hærskarers HERRE, Israels Gud: Gå hen og sig til Judas Mænd og Jerusalems Borgere: Vil I ikke tage ved Lære og høre mine Ord? lyder det fra HERREN.
\par 14 Jonadabs, Rekabs Søns, Bud er blevet overholdt; thi han forbød sine Sønner at drikke Vin, og de har ikke drukket Vin til den Dag i Dag, men adlydt deres Faders Bud; men jeg har talet til eder årle og silde, uden at I vilde høre mig.
\par 15 Jeg sendte alle mine Tjenere Profeterne til eder årle og silde, for at de skulde sige: "Omvend eder hver fra sin onde Vej, gør gode Gerninger og hold eder ikke til andre Guder, så I dyrker dem; så skal I bo i det Land, jeg gav eder og eders Fædre." Men I bøjede ikke eders Øre og hørte mig ikke.
\par 16 Fordi Jonadabs, Rekabs Søns, Sønner overholdt deres Faders Bud, medens dette Folk ikke vilde høre mig,
\par 17 derfor, så siger HERREN, Hærskarers Gud, Israels Gud: Se, jeg bringer over Juda og Jerusalems Borgere al den Ulykke, jeg har truet dem med, fordi de ikke hørte, da jeg talede, og ikke svarede, da jeg kaldte ad dem.
\par 18 Men til Rekabiternes Hus sagde Jeremias: Så siger Hærskarers HERRE, Israels Gud: Fordi I har adlydt eders Fader Jonadabs Bud og overholdt alle hans Bud og gjort alt, hvad han bød eder,
\par 19 derfor, så siger Hærskarers HERRE, Israels Gud: Ingen Sinde skal Jonadab, Rekabs Søn, fattes en Mand til at stå for mit Åsyn.

\chapter{36}

\par 1 I Josiases søn kong Jojakim af Judas fjerde regeringsår kom dette Ord til Jeremias fra HERREN:
\par 2 "Tag dig en Bogrulle og skriv deri alle de Ord, jeg har talet til dig om Jerusalem og Juda og om alle Folkene, fra den Dag jeg først talede til dig, fra Josiass dage og til den Dag i Dag.
\par 3 Måske vil Judas Hus mærke sig al den Ulykke, jeg har i Sinde at gøre dem, for at de må omvende sig hver fra sin onde Vej, så jeg kan tilgive deres Brøde og Synd."
\par 4 Så tilkaldte Jeremias Baruk, Nerijas Søn, og Baruk optegnede i Bogrullen efter Jeremiass Mund alle de Ord, HERREN havde talet til ham.
\par 5 Derpå sagde Jeremias til Baruk: "Jeg er hindret i at gå ind i HERRENs Hus;
\par 6 men gå du ind og læs HERRENs Ord op af Bogrullen, som du skrev efter min Mund, for Folket i HERRENs Hus på en Fastedag; også for alle Judæere, der kommer ind fra deres Byer, skal du læse dem.
\par 7 Måske når deres klage HERRENs Åsyn, måske omvender de sig hver fra sin onde Vej; thi stor er Vreden og Harmen, som HERREN har udtalt mod dette Folk."
\par 8 Og Baruk, Nerijas Søn, gjorde ganske som Profeten Jeremias pålagde ham, og oplæste HERRENs Ord af Bogen i HERRENs Hus.
\par 9 I Josiass Søns, Kong Jojakim af Judas, femte Regeringsår i den niende Måned udråbte alt Folket i Jerusalem og alt Folket, der fra Judas Byer kom ind til Jerusalem, en Faste for HERREN.
\par 10 Da oplæste Baruk for alt Folket Jeremiass Ord af Bogen i HERRENs Hus, i Gemarjahus, Statsskriveren Sjafans Søns, Kammer i den øvre Forgård ved Indgangen til HERRENs Huss nye Port.
\par 11 Da nu Mika, en Søn af Sjatans Søn Gemarjahu, havde hørt HERRENs Ord oplæse af Bogen,
\par 12 gik han ned i Kongens Hus til Statsskriverens Kammer, hvor han traf alle Fyrsterne siddende, Statsskriveren Elisjama, Delaja Sjemajas Søn, Elnatan Akbors Søn, Gemarjahu Sjafans Søn, Zidkija Hananjas Søn og alle de andre Fyrster;
\par 13 og Mika meldte dem alt, had han havde hørt, da Baruk læste Bogen op for Folket.
\par 14 Da sendte alle Fyrsterne Jehudi, en Søn af Netanja, en Søn af Sjelemja, en Søn af Kusji, til Baruk og lod sige: "Tag Bogrullen, du læste op for Folket, og kom her ned!" Så tog Baruk, Nerijas Søn, Bogrullen og kom til dem.
\par 15 De sagde til ham: "Sæt dig og læs den for os!" Og Baruk læste for dem.
\par 16 Men da de havde hørt alle disse Ord, så de rædselslagne på hverandre og sagde: "Alt det må vi sige Kongen."
\par 17 Og de spurgte Baruk: "Sig os, hvorledes du kom til at optegne alle disse Ord!"
\par 18 Baruk svarede: "Jeremias foresagde mig alle Ordene, og jeg optegnede dem i Bogen med Blæk."
\par 19 Så sagde Fyrsterne til Baruk: "Gå hen og gem eder, du og Jeremias, og lad ingen vide, hvor I er!"
\par 20 Efter så at have lagt Bogrullen til Side i Statsskriveren Elisjamas Kammer kom de til Kongen i hans Stue og sagde ham alt.
\par 21 Så sendte Kongen Jehudi hen at hente Bogrullen i Statsskriveren Elisjamas Kammer; og Jehudi læste den op for Kongen og alle Fyrsterne, der stod om Kongen.
\par 22 Kongen sad i Vinterhuset med et brændende Kulbækken foran sig;
\par 23 og hver Gang Jehudi havde læst tre fire Spalter, skar Kongen dem af med Statsskriverens Pennekniv og kastede dem på Ilden i Bækkenet, indtil hele Bogrullen var fortæret af Ilden i Bækkenet.
\par 24 Og hverken Kongen eller nogen af hans Folk blev rædselslagen eller sønderrev deres Klæder, da de hørte alle disse Ord;
\par 25 men skønt Elnafan, Delaja og Gemarjahu bad Kongen ikke brænde Bogrullen, hørte han dem ikke.
\par 26 Derpå bød Kongen Kongesønnen Jerameel, Seraja Azriels Søn og Sjelemja Abdeels Søn at gribe Skriveren Baruk og Profeten Jeremias; men HERREN skjulte dem.
\par 27 Men da Kongen havde brændt Bogrullen med de Ord, Baruk havde optegnet efter Jeremiass Mund, kom HERRENs Ord til Jeremias således:
\par 28 "Tag dig en anden Bogrulle og optegn i den alle de Ord, som stod i den første Bogrulle, den, Kong Jojakim af Juda brændte.
\par 29 Og til Kong Jojakim af Juda skal du sige: Så siger HERREN: Du brændte denne Bogrulle og sagde: Hvorfor skrev du i den: Babels Konge skal komme og ødelægge dette Land og udrydde både Folk og Fæ?
\par 30 Derfor, så siger HERREN om Kong Jojakim af Juda: Han skal ikke have nogen Mand til at sidde på Davids Trone, og hans Lig skal slænges hen og gives Dagens Hede og Nattens Kulde i Vold;
\par 31 jeg vil hjemsøge ham, hans Afkom og hans Tjenere for deres Brøde og bringe over dem og Jerusalems Borgere og Judas Mænd al den Ulykke, jeg har udtalt over dem, uden at de vilde høre."
\par 32 Så tog Jeremias en anden Bogrulle og gav den til Skriveren Baruk, Nerijas Søn; og han optegnede i den efter Jeremiass Mund alle Ordene fra den Bog, Kong Jojakim af Juda havde brændt. Og flere lignende Ord lagdes til.

\chapter{37}

\par 1 Zedekias, Joasiases søn, blev konge Konjas, Jojakims søns sted, idet Kong Nebukadrezar af Babel satte ham til Konge i Judas Land.
\par 2 Men han og hans Mænd og Landets Befolkning hørte ikke på de Ord, HERREN talede ved Profeten Jeremias.
\par 3 Kong Zedekias sendte Jukal. Sjelemjas Søn, og Præsten Zefanja, Maasejas Søn, til Profeten Jeremias og lod sige: "Gå i Forbøn for os hos HERREN vor Gud!"
\par 4 Dengang gik Jeremias frit ud og ind blandt Folket, thi man havde endnu ikke kastet ham i Fængsel.
\par 5 Faraos Hær var rykket ud fra Ægypten; og da kaldæerne, som belejrede Jerusalem, fik Nys herom, var de brudt op fra Jerusalem.
\par 6 Da kom HERRENs Ord til Profeten Jeremias således:
\par 7 Så siger HERREN, Israels Gud: Således skal du sige til Judas Konge, som har sendt Bud til dig for at rådspørge mig: Se, Faraos Hær, som er rykket ud for at hjælpe eder, skal vende hjem til Ægypten;
\par 8 og Kaldæerne skal vende tilbage og angribe denne By, indtage og afbrænde den.
\par 9 Så siger HERREN: Når ikke eder selv ved at sige: "Kaldæerne drager bort fra os for Alvor!" Thi de drager ikke bort.
\par 10 Ja, om, I så slog hele Kaldæernes Hær, der angriber eder, så der kun blev nogle sårede tilbage, hver i sit Telt, så skulde de stå op og afbrænde denne By.
\par 11 Da Kaldæernes Hær var brudt op fra Jerusalem for Faraos Hær,
\par 12 gik Jeremias ud af Jerusalem for at drage til Benjamins Land og få en Arvelod iblandt Befolkningen.
\par 13 Men da han kom til Benjaminsporten, var der en Vagthavende ved Navn Jirija, en Søn af Hananjas Søn Sjelemja, og han greb Profeten Jeremias og sagde: "Du vil løbe over til Kaldæerne."
\par 14 Jeremias svarede: "Det er Løgn; jeg vil ikke løbe over til Kaldæerne." Jirija vilde dog ikke høre ham, men greb ham og bragte ham til Fyrsterne;
\par 15 og Fyrsterne vrededes på Jeremias, slog ham og lod ham bringe til Statsskriveren Jonatans Hus; thi det havde de gjort til Fængsel.
\par 16 Således kom Jeremias i Fangehuset i kælderen; og der sad han en Tid lang.
\par 17 Men Kong Zedekias sendte Bud og lod ham hente; og Kongen spurgte ham i al Hemmelighed i sit Palads: "Er der et Ord fra HERREN?" Jeremias svarede: "Ja, der er: Du skal overgives i Babels Konges Hånd."
\par 18 Derpå sagde Jeremias til Kong Zedekias: "Hvad Synd har jeg gjort imod dig, dine Mænd og dette Folk, siden I har kastet mig i Fængsel?
\par 19 Og hvor er nu eders Profeter, som profeterede for eder, at Babels Konge ikke skulde komme over eder og dette Land?
\par 20 Så hør da, Herre Konge! Lad min Bøn nå dig og lad mig ikke bringe tilbage til Statsskriveren Jonatans Hus, af jeg ikke skal dø der!"
\par 21 Da bød Kong Zedekias, at man skulde holde Jeremias i Varetægt i Vagtforgården; og der gaves ham daglig et Stykke Brød fra Bagerens Gade, indtil Brødet slap op i Byen. Således sad nu Jeremias i Vagtforgården.

\chapter{38}

\par 1 Men da Sjefatja Mattans søn, Gedalja Pasjhurs søn, Jukal Sjelemjas Søn og Pasjhur Malkijas Søn hørte Jeremias tale til alt Folket således:
\par 2 "Så siger HERREN: Den, der bliver i denne By, skal dø ved Sværd, Hunger og Pest, men den, som overgiver sig til Kaldæerne, skal leve og vinde sit Liv som Bytte;
\par 3 thi så siger HERREN: Denne By skal gives i Babels Konges Hærs Hånd, og han skal indtage den"
\par 4 da sagde Fyrsterne til Kongen: "Denne Mand må dø, thi han tager Modet fra Krigsmændene, som er tilbage i denne By, og fra alt Folket ved at tale således til dem; thi denne Mand tænker ikke på dette Folks Vel, men på dets Ulykke."
\par 5 Kong Zedekias svarede: "Se, han er i eders Hånd." Thi Kongen evnede intet over for dem.
\par 6 Så tog de Jeremias og kastede ham i Kongesønnen Malkijas Cisterne i Vagtforgården, idet de hejsede ham ned med Reb. Der var ikke Vand i Cisternen, men Dynd, og Jeremias sank i Dyndet.
\par 7 Imidlertid hørte Ætioperen Ebed Melek, en Hofmand i Kongens Palads, at Jeremias var kastet i Cisternen; og da Kongen var i Benjaminsporten,
\par 8 gik Ebed-Melek fra Paladset og talte således til Kongen:
\par 9 "Herre Konge, ilde har de gjort ved at lade denne Mand dø af Hunger, fordi der ikke er mere Brød i Byen!"
\par 10 Så bød Kongen Ætioperen Ebed-Melek: "Tag tredive Mænd med herfra og drag Profeten Jeremias op af Cisternen, før han dør!"
\par 11 Ebed-Melek tog Mændene med og gik til Kælderen under Skatkammeret i Kongens Palads, hvor han hentede nogle Klude af slidte og iturevne klæder; dem hejsede han med Reb ned til Jeremias i Cisternen,
\par 12 idet han sagde: "Læg Kludene om Rebet!" Det gjorde Jeremias,
\par 13 og de drog ham op af Cisternen med Rebet. Således kom Jeremias atter til at sidde i Vagtforgården.
\par 14 Kong Zedekias sendte Bud og lod Profeten Jeremias hente til sig i Livvagtens Indgang til HERRENs Hus. Og Kongen sagde til ham: "Jeg vil spørge dig om noget, dølg intet for mig!"
\par 15 Jeremias svarede Zedekias: "Hvis jeg siger dig det, vil du da ikke lade mig dræbe? Og selv om jeg råder dig, vil du dog ikke høre mig."
\par 16 Da tilsvor Kong Zedekias i al Hemmelighed Jeremias: "Så sandt HERREN lever, som har skabt vor Sjæl, jeg vil ikke lade dig dræbe eller give dig i disse Mænds Hånd, som står dig efter Livet."
\par 17 Så sagde Jeremias til Zedekias: "Så siger HERREN, Hærskarers Gud, Israels Gud: Hvis du overgiver dig til Babels Konges Fyrster, skal du redde dit Liv; denne By skal ikke afbrændes, og du og dit Hus skal blive i Live;
\par 18 men overgiver du dig ikke til dem, skal Byen gives i Kaldæernes Hånd, og de skal afbrænde den, og du skal ikke undslippe deres Hånd."
\par 19 Men kong Zedekias sagde til Jeremias: "Jeg er ræd for de Judæere, der er løbet over til Kaldæerne, at Kaldæerne skal overgive mig i deres Hånd, og at de skal drive Spot med mig."
\par 20 Så sagde Jeremias: "Det gør de ikke! Adlyd kun HERRENs Ord, som jeg taler til dig, så skal det gå dig vel, og du skal blive i Live.
\par 21 Men vægrer du dig ved at overgive dig, så hør nu, hvad HERREN har ladet mig skue:
\par 22 Se, alle Kvinder, der er tilbage i Judas Konges Palads, førtes ud til Babels Konges Fyrster, medens de sang: Dig forledte og tvang dine gode Venner, de ledte din Fod i en Sump og trak sig tilbage.
\par 23 Alle dine Hustruer og Børn skal føres ud til Kaldæerne, og du skal ikke undslippe deres Hånd, men gribes af Babels Konges Hånd, og denne By skal abrændes!"
\par 24 Så sagde Zedekias til Jeremias: "Ingen må vide noget om denne Samtale, ellers er du dødsens;
\par 25 og hvis Fyrsterne skulde få Nys om, at jeg har talt med dig, og komme til dig og sige: Sig os hvad du sagde til Kongen; dølg ikke noget for os, ellers dræber vi dig; sig os også, hvad Kongen sagde til dig!
\par 26 sig så til dem: Jeg fremførte en ydmyg Bøn for Kongen om ikke at lade mig føre tilbage til Jonatans Hus for at dø der."
\par 27 Og alle Fyrsterne kom til Jeremias og spurgte ham; og han svarede dem nøje, som Kongen havde påbudt. Så lod de ham i Fred, eftersom Sagen ikke var blevet kendt.
\par 28 Således sad Jeremias i Vagtforgården, lige til den Dag Jerusalem blev indtaget.

\chapter{39}

\par 1 Efter at Jerusalem var indtaget, i kong Zedekias af Judas niene regeringsår i den tiende Måned, kom Kong Nebudkadrezar af Babel med hele sin Hær til Jerusalem og belejrede det;
\par 2 i Zedekiass ellevte År på den niende Dag i den fjerde Måned blev Byen stormet
\par 3 da kom alle Babels konges Fyrster og satte sig i Midterporten: Overhofmanden Nebusjazban, Magernes Øverste Nergal-Sarezer og alle Babels Konges andre Fyrster.
\par 4 Da Kong Zedekias af Juda og alle hans Krigsmænd så dem, flygtede de om Natten fra Byen ad Vejen til Kongens Have gennem Porten mellem de to Mure og tog Vejen ad Araba til.
\par 5 Men Kaldæernes Hær satte efter dem og indhentede Zedekias på Jerikos Lavslette; og de tog ham med og bragte ham op til Kong Nebukadnezar af Babel i Ribla i Hamats Land; og han fældede hans Dom.
\par 6 Babels Konge lod i Ribla Zedekiass Sønner dræbe i hans Påsyn; også alle de ypperste i Juda lod Babels Konge dræbe;
\par 7 derpå lod han Øjnene stikke ud på Zedekias og lod ham lægge i Kobberlænker for at føre ham til Babel.
\par 8 kaldæerne satte Ild på Kongens Palads og Folkets Huse og nedbrød Jerusalems Mure.
\par 9 Resten af Folket, der var levnet i Byen, Overløberne, der var løbet over til ham, og Resten af Håndværkerne førte Livvagtens øverste Nebuzaradan som Fanger til Babel,
\par 10 og kun nogle af den fattigste Befolkning, der intet ejede, lod Livvagts øverste Nebuzaradan blive tilbage i Judas Land, idet han samtidig gav dem Vingårde og Agre.
\par 11 Men om Jeremias bød Kong Nebukadrezar af Babel Livvagts øverste Nebuzaradan:
\par 12 "Tag ham og hav Øje med ham og gør ham ingen Men; gør med ham, som han selv ønsker!"
\par 13 Så sendte Livvagts øverste Nebuzaradan, Overhofmanden Nebusjazban og Magernes Øverste Nergal-Sarezer og alle Babels Konges andre Stormænd
\par 14 Bud og, lod Jeremias hente i Vagtforgården og overgav ham til Gedalja, en Søn af Sjafans Søn Ahikam, for at han skulde føre ham til hans Hjem; og han boede iblandt Folket.
\par 15 Medens Jeremias sad fængslet i Vagtforgården, kom HERRENs Ord til ham således:
\par 16 Gå hen og sig til Ætioperen Ebed-Melek: Så siger Hærskarers HERRE, Israels Gud: Se, jeg lader mine Ord gå i Opfyldelse på denne By til Ulykke og ikke til Lykke, og du skal have dem i Tankerne på hin Dag.
\par 17 Men på hin Dag redder jeg dig, lyder det fra HERREN, og du skal ikke gives i de Mænds Hånd, for hvem du frygter;
\par 18 thi jeg vil frelse dig, så du ikke falder for Sværdet, og du skal vinde dit Liv som Bytte, fordi du stolede på mig, lyder det fra HERREN

\chapter{40}

\par 1 Det Ord, som kom fra HERREN til Jeremias, efter at livvagtens øverste Nebuzaradan havde løsladt ham i Rama; han lod ham hente, medens han var bundet med Lænker iblandt alle Fangerne fra Jerusalem, og Juda, der førtes til Babel.
\par 2 Livvagts øverste lod Jeremias hente og sagde til ham: "HERREN din Gud har udtalt denne Ulykke over dette Sted,
\par 3 og HERREN lod det ske og gjorde, hvad han havde sagt, fordi I syndede mod HERREN og ikke adlød hans, Røst; derfor timedes dette eder.
\par 4 Se, nu tager jeg i Dag Lænkerne af dine Hænder. Hvis det tykkes dig godt at drage med mig til Babel, så drag med, og jeg vil have Øje med dig; men tykkes det dig ilde, så lad være! Se, hele Landet står dig åbent; gå, hvor det tykkes dig godt og ret!"
\par 5 Og da han tøvede med at vende tilbage, tilføjede han: "Så vend tilbage til Gedalja Sjafans Søn Ahikams Søn, som Babels konge har sat over Judas Land, og bosæt dig hos, ham iblandt Folket, eller gå, hvor som helst det tykkes dig ret!" Og Livvagts øverste gav ham Rejsetæring og Gave og lod ham gå.
\par 6 Jeremias gik da til Gedalja, Ahikams Søn, i Mizpa og bosatte sig hos ham iblandt Folket, der var levnet i Landet.
\par 7 Da alle Hærførerne, som var ude i åbent Land, og deres Mænd hørte, at Babels konge havde sat Gedalja, Ahikams Søn, over Landet og over Mænd, kvinder og Børn og dem af den fattige Befolkning i Landet, som ikke var ført til Babel,
\par 8 kom de til Gedalja i Mizpa: Jisjmael Netanjas Søn Johanan Kareas Søn.
\par 9 Og Gedalja, Sjafans Søn Ahikams Søn, tilsvor dem og deres Mænd således: "Frygt ikke for at stå under Kaldæerne; bosæt eder i Landet og underkast eder Babels Konge, så skal det gå eder vel.
\par 10 Se, selv bliver jeg i Mizpa for at tage mod Kaldæerne, når de kommer til os; men I skal samle Vin, Frugt og Olie i eders Kar og bo i de Byer, I tager i Eje!"
\par 11 Og da også alle de Judæere, der var i Moab, hos Ammoniterne, i Edom og alle de andre Lande, hørte, at Babels Konge havde levnet Juda en Rest og sat Gedalja. Sjafans Søn Ahikams Søn, over dem,
\par 12 vendte de alle tilbage fra alle de Steder, som de var fordrevet til, og kom til Judas Land til Gedalja i Mizpa; og de indsamlede Vin og Frugt i store Måder.
\par 13 Men Johanan, Kareas Søn, og alle de andre Hærførere, som havde været ude i åbent Land, kom til Gedalja i Mizpa
\par 14 og sagde: "Mon du ved, at Baalis, Ammoniternes Konge, har sendt Jisjmael, Netanjas Søn, for at myrde dig?" Men Gedalja, Ahikams Søn, troede dem ikke.
\par 15 Da sagde Johanan, Kareas Søn, i al Hemmelighed til Gedalja I Mizpa: "Lad mig gå hen og myrde Jisjmael, Netanjas Søn; ingen skal få det at vide.
\par 16 Men Gedalja, Ahikas Søn, svarede Johanan, Kareas Søn: "Det må du ikke gøre, thi du lyver om Jisjmael!"

\chapter{41}

\par 1 Men i den syvende måned kom Jisjmael, Elisjamas søn Netanjas søn, en mand af kongelig Æt, der hørte til Kongens Stormænd, fulgt af ti Mænd til Gedalja, Ahikams Søn, i Mizpa; og de holdt Måltid sammen der i Mizpa.
\par 2 Jisjmael, Netanjas Søn, og de ti Mænd, der fulgte ham, stod da op og huggede Gedalja, Sjafans Søn Ahikams Søn, ned med Sværdet og dræbte således den Mand, Babels Konge havde sat over Landet;
\par 3 også alle de Judæere, som var hos ham i Mizpa, og alle de Kaldæere, som fandtes der, alle Krigerne huggede Jisjmael ned.
\par 4 Dagen efter Gedaljas Mord, endnu før nogen kendte dertil,
\par 5 kom firsindstyve Mænd fra Sikem, Silo og Samaria med afklippet Skæg, sønderrevne Klæder og Flænger i Huden; de havde Afgrødeoffer og Røgelse med til at ofre i HERRENs Hus"
\par 6 Jisjmael, Netanjas Søn, gik dem i Møde fra Mizpa og græd hele Vejen, og da han traf dem, sagde han: "Kom med til Gedalja, Ahikams Søn!"
\par 7 Men da de var kommet ind i Byen, huggede Jisjmael og hans Mænd dem ned og kastede dem i Cisternen.
\par 8 Men der var ti Mænd iblandt dem, som sagde til Jisjmael: "Dræb os ikke, thi vi har skjulte Forråd på Marken, Hvede, Byg, Olie og Honning." Så lod han dem være og dræbte dem ikke med de andre.
\par 9 Cisternen, hvori Jisjmael kastede Ligene af alle dem, han havde hugget ned, var den store Cisterne, Kong Asa havde bygget i Kampen mod Kong Basja af Israel; den fyldte Jisjmael, Netanjas Søn, med dræbte.
\par 10 Derpå bortførte Jisjmael som Fanger hele Resten af Folket i Mizpa, Kongedøtrene og hele Folket, der var ladt tilbage i Mizpa, og over hvem Livvagts øverste Nebuzaradan havde sat Gedalja, Ahikams Søn; dem bortførte Jisjmael, Netanjas Søn, som Fanger og gav sig på Vej til Ammoniterne.
\par 11 Men da Johanan, Kareas Søn, og alle de Hærførere, som var hos ham, hørte om al den Ulykke, Jisjmael, Netanjas Søn, havde gjort,
\par 12 tog de alle deres Mænd og drog imod ham, og de traf ham ved den store Dam i Gibeon;
\par 13 og da alt Folket, der var hos Jisjmael, så Johanan, Kareas Søn og alle Hærførerne, der var med ham, blev de glade;
\par 14 og alt Folket, som Jisjmael havde ført fanget fra Mizpa, vendte om og gik over til Johanan, Kareas Søn.
\par 15 Men Jisjmael Netanjas Søn, slap fra Johanan med otte Mand og drog til Ammoniterne.
\par 16 Johanan, Kareas Søn, og alle Hærførerne, der var med ham, tog derpå hele Resten af Folket, som Jisjmael, Netanjas Søn, efter at have myrdet Gedalja, Ahikams Søn, havde ført bort fra Mizpa, de Mænd, Krigere, Kvinder, Børn og Hofmænd, som han bragte tilbage fra Gibeon,
\par 17 og de drog hen og slog sig ned i Gidrot-Kimham i BetlehemsNabolag for at drage til Ægypten.
\par 18 af Frygt for Kaldæerne; thi de frygtede dem, fordi Jisjmael, Netanjas Søn, havde dræbt Gedalja, Ahikams Søn, som Babels Konge havde sat over Landet.

\chapter{42}

\par 1 Så kom alle hærførerene og Johanan, Kareas søn, og Azarja, Maasejas Søn, med alt Folket, store og små,
\par 2 og sagde til Profeten Jeremias: "Måtte vor Bøn nå dit Øre, så du beder til HERREN din Gud for hele denne Rest, thi som du ser os her, er vi kun få tilbage af mange.
\par 3 Måtte HERREN din Gud kundgøre os, hvilken Vej vi skal gå, og hvad vi skal gøre!"
\par 4 Profeten Jeremias svarede: "Godt! Jeg vil bede til HERREN eders Gud, som I ønsker; og alt hvad HERREN svarer, vil jeg kundgøre eder uden at forholde eder et Ord."
\par 5 De sagde da til Jeremias: "HERREN skal være et sandt og troværdigt Vidne imod os, hvis vi ikke retter os efter hvert Ord, HERREN din Gud sender os ved dig.
\par 6 Det være godt eller ondt, vi vil adlyde HERREN vor Guds Røst, til hvem vi sender dig, at det må gå os vel, når vi adlyder HERREN vor Guds Røst."
\par 7 Ti Dage efter kom HERRENs Ord til Jeremias.
\par 8 Så sammenkaldte han Johaoan, Kareas Søn, alle Hærføreme, der var med ham, og alt Folket, store og små,
\par 9 og sagde: Så siger HERREN, Israels Gud, til hvem I sendte mig, for at eders Bøn måtte nå ind for hans Åsyn:
\par 10 Hvis I bliver her i Landet, vil jeg bygge eder og ikke nedbryde eder, plante eder og ikke rykke eder op, thi jeg angrer det onde, jeg har gjort eder.
\par 11 Frygt ikke for Babels Konge, således som I gør, frygt ikke for ham, lyder det fra HERREN, thi jeg er med eder for at frelse og redde eder af hans Hånd.
\par 12 Jeg vil lade eder finde Barmhjertighed, og han skal forbarme sig over eder og lade eder bo i eders Land.
\par 13 Hvis I derimod ikke hører HERREN eders Guds Røst, idet I siger, at I ikke vil bo her i Landet,
\par 14 men drage til Ægypten og bo der for ikke mere at se Krig eller høre Hornets Klang eller hungre efter Brød,
\par 15 så hør nu HERRENs Ord, Judas Rest. Så siger Hærskarers HERRE, Israels Gud: Hvis I virkelig har i Sinde at drage til Ægypten og drager derned for at bo der som fremmede,
\par 16 så skal Sværdet, som I frygter, nå eder der i Ægypten, og Hungeren, som I ængstes for, skal følge efter eder til Ægypten, og I skal omkomme der;
\par 17 alle de Mænd, som har i Sinde at drage til Ægypten for at bo der som fremmede, skal dø ved Sværd, Hunger og Pest, og ingen af dem skal blive tilovers og undslippe fra den Ulykke, jeg sender over dem.
\par 18 Thi så siger Hærskarers HERRE, Israels Gud: Som min Vrede og Harme udgød sig over Jerusalems Indbyggere, således skal min Harme udgyde sig over eder, når I drager til Ægypten, og I skal blive et. Edens, Rædselens, Forbandelsens og Spottens Tegn og ikke mere få dette Sted at se.
\par 19 Dette er HERRENs Ord til eder. Judas Rest: Drag ikke til Ægypten! I skal vide, at jeg i Dag har advaret eder.
\par 20 Thi I nedkalder ondt over eder selv, når I sender mig til HERREN eders Gud og siger: "Bed for os til HERREN vor Gud! Hvad HERREN vor Gud siger, skal du nøje kundgøre os, så vil vi gøre det,"
\par 21 og I så alligevel ikke adlyder HERREN eders Guds Røst og gør alt, hvad han sendte eder Bud om.
\par 22 Så vid da nu, at I skal omkomme ved Sværd, Hunger og Pest på det Sted, hvor I agter at gå hen for at bo der somfremmede.

\chapter{43}

\par 1 Men da Jeramias var til ende med at forkynde alle de Ord, med hvilke HERREN deres Gud havde sendt ham til dem, alle de nævnte Ord,
\par 2 sagde Azarja, Maasejas Søn, og Johanan, Kareas Søn, og alle de andre overmodige Mænd til Jeremias: "Du lyver! HERREN vor Gud har ikke sendt dig for at sige, at vi ikke skal drage til Ægypten for at bo der som fremmede;
\par 3 nej, Baruk, Nerijas Søn, har ophidset dig imod os, for at vi skal gives i Kaldæernes Hånd, så de dræber os eller fører os bort til Babel."
\par 4 Og Johanan, Karens Søn, alle Hærførerne og alt Folket adlød ikke HERRENs Røst om at blive i Judas Land;
\par 5 men Johanan, Kareas Søn, og alle Hærførerne tog hele Judas Rest, som var vendt tilbage for at bo i Judas Land,
\par 6 Mænd, Kvinder og Børn, Kongedøtrene og enhver, som Livvagts øverste Nebuzaradan havde ladet blive hos Gedalja, Sjafans Søn Ahikams Søn, også Profeten Jeremias og Baruk, Nerijas Søn,
\par 7 og drog til Ægypten; thi de adlød ikke HERRENs Røst. Og de kom til Takpankes.
\par 8 Men HERRENs Ord kom til Jeremias i Takpankes således:
\par 9 Tag dig nogle store Sten og grav dem ned i Teglstensgulvets Underlag ved Indgangen til Faraos Hus i Takpankes i de judæiske Mænds Påsyn
\par 10 og sig til dem: Så siger Hærskarers HERRE, Israels Gud: Jeg lader min Tjener Kong Nebukadrezar af Babel hente, og han skal rejse sin Trone oven over de Sten, du gravede ned, og brede sit Trontæppe derover.
\par 11 Han skal komme og slå Ægypten; dem, der hører Døden til, skal han overgive til Død, dem der hører Fangenskabet til, til Fangenskab og dem, der hører Sværdet til, til Sværd.
\par 12 Han skal sætte Ild på Ægyptens Gudehuse og afbrænde dem og bortføre Guderne som Fanger, og han skal svøbe Ægypten om sig, som en Hyrde sin Kappe; derpå skal han drage bort derfra i Fred.
\par 13 Han skal nedbryde Stenstøtterne i Bet-Sjemesj og afbrænde Ægyptens Gudehuse.

\chapter{44}

\par 1 Det ord, som kom til Jeremias om alle de Judærere, der boede i Ægypten, i Migdol, Takpankes, Nof og Patros:
\par 2 Så siger Hærskarers HERRE, Israels Gud: I så selv al den Ulykke, jeg bragte over Jerusalem og alle Judas Byer; se, de ligger nu øde hen, og ingen bor i dem;
\par 3 det er Straf for det onde, de gjorde, idet det krænkede mig ved at gå hen og tænde Offerild for og dyrke andre Guder, som hverken de eller deres Fædre før kendte til.
\par 4 Jeg sendte årle og silde alle mine Tjenere Profeterne til dem, for at de skulde sige: "Gør dog ikke disse vederstyggelige Ting, som jeg hader!"
\par 5 Men de hørte ikke og bøjede ikke deres Øre dertil, så de omvendte sig fra deres Ondskab og hørte op med at tænde Offerild for andre Guder.
\par 6 Derfor udgød min Vrede og Harme sig og luede op i Judas Byer og Jerusalems Gader, så de blev til Ødemark og Ørk, som de er den Dag i Dag.
\par 7 Og nu, så siger HERREN, Hærskarers Gud, Israels Gud: Hvorfor nedkalder I stor Ulykke over eder selv og udrydder Mænd og Kvinder, Børn og diende af Juda, så I ikke levner eder nogen Rest,
\par 8 idet I krænker mig med eders Hænders Værker og tænder Offerild for andre Guder i Ægypten, hvor I kom hen for at bo som fremmede? Følgen bliver, at I udrydder eder selv og bliver et Forbandelsens og Spottens Tegn blandt alle Jordens Folk.
\par 9 Har I glemt de onde Gerninger, eders Fædre og Judas Konger og Fyrster og eders Kvinder gjorde i Judas Land og på Jerusalems Gader?
\par 10 Hidtil har de ikke ydmyget sig; de frygter ikke og vandrer ikke efter min Lov og mine Bud, som jeg forelagde eder og eders Fædre.
\par 11 Derfor, så siger Hærskarers HERRE, Israels Gud: Se, jeg har ondt i Sinde imod eder; jeg vil udrydde hele Juda.
\par 12 Og jeg tager Judas Rest, dem, som fik i Sinde at drage til Ægypten og bo der som fremmede; de skal alle omkomme i Ægypten; de skal falde for Sværd og Hunger og omkomme, store og små; for Sværd og Hunger skal de dø og blive et Edens, Rædselens, Forbandelsens og Spottens Tegn.
\par 13 Og jeg hjemsøger dem, der bor i Ægypten, som jeg bjemsøgte Jerusalem, med Sværd, Hunger og Pest.
\par 14 Og af Judas Rest, dem, der kom til Ægypten for at bo der som fremmede, skal ingen reddes eller undslippe, så han kan vende hjem til Judas Land, hvor de længes efter at bo igen; nej, ingen skal vende hjem undtagen enkelte, som reddes.
\par 15 Men alle Mændene, der vel vidste, at deres Kvinder tændte Offerild for andre Guder, og alle Kvinderne, som stod der i en stor Klynge, og alt Folket, som boede i Ægypten, i Patros, svarede Jeremias:
\par 16 "Det Ord, du har talt til os i HERRENs Navn, vil vi ikke høre;
\par 17 nej, vi vil opfylde hvert Løfte; som er udgået af vor Mund, og tænde Offerild for Himmelens Dronning og udgyde Drikofre for hende, som vi og vore Fædre, vore konger og Fyrster gjorde det i Judas Byer og på Jerusalems, Gader. Dengang havde vi Brød nok og var lykkelige og kendte ikke til Ulykke;
\par 18 men fra den Stund vi hørte op med at tænde Offerild for Himmelens Dronning og udgyde Drikofre for hende, led vi Mangel på alt og omkom ved Sværd og Hunger.
\par 19 Og når vi tænder Offerild for Himmelens Dronning og udgyder Drikofre for hende, mon det så er uden vore Mænds Vidende, at vi bager hende Offerkager, som afbilder hende, og udgyder Drikofre for hende?"
\par 20 Jeremias sagde til alt Folket, Mændene, Kvinderne og alt Folket, som havde svaret ham således:
\par 21 "Mon ikke den offerild, som I, eders Fædre, eders Konger og Fyrster og Landets Befolkning tændte i Judas Byer og på Jerusalems Gader, randt HERREN i Hu og kom ham i Tanke?
\par 22 HERREN kunde ikke mere holde det ud for eders onde Gerninger og de vederstyggelige Ting, I gjorde; derfor blev eders Land til Ørk, til et Rædselens og Forbandelsens Tegn, som det er den Dag i Dag.
\par 23 Fordi I tændte Offerild og syndede mod HERREN og ikke adlød HERRENs Røst eller fulgte hans Lov, Vedtægter og Vidnesbyrd, derfor ramtes I af denne Ulykke, som varer ved den Dag i Dag."
\par 24 Og Jeremias sagde til alt Folket og alle kvinderne: "Hør HERRENs Ord, hele Juda i Ægypten!
\par 25 Så siger Hærskarers HERRE, Israels Gud: I og eders Kvinder lover med eders Mund og opfylder det med eders Hænder! I siger: Vi vil opfylde de Løfter, vi har aflagt, og tænde Offerild for Himmelens Dronning og udgyde Drikofre for hende. Så hold da eders Løfter og indfri dem!
\par 26 Men hør da også HERRENs Ord, alle I Judæere, som bor i Ægypten: Se, jeg sværger ved mit store Navn, siger HERREN: Ikke skal mere nogensteds i Ægypten mit Navn nævnes i nogen judæisk Mands Mund, så han siger: Så sandt den Herre HERREN lever!
\par 27 Se, jeg er årvågen over dem til Ulykke og ikke til Lykke, og hver judæisk Mand i Ægypten skal omkomme ved Sværd og Hunger, indtil de er udryddet.
\par 28 Kun de, der undslipper Sværdet, skal vende hjem fra Ægypten til Judas Land, et ringe Tal; og hele Judas Rest, der er kommet til Ægypten for at bo der som fremmede, skal kende, hvis Ord der står fast, mit eller deres.
\par 29 Og dette, lyder det fra HERREN, skal være eder et Tegn på, at jeg hjemsøger eder på dette Sted, for at I skal kende, at mine Ord opfyldes på eder til eders Ulykke:
\par 30 Så siger HERREN: Se, jeg giver Ægypterkongen Farao Hofra i hans Fjenders Hånd og i deres Hånd, som står ham efter Livet, ligesom jeg gav Kong Zedekias af Juda i hans Fjende, Kong Nebukadrezar af Babels Hånd, som stod ham efter livet.

\chapter{45}

\par 1 Det ord som profeten Jeremias talte til Baruk, Nerijas søn, da han optegnede alle disse ord i en Bog efter Jeremiass Mund i Josiass Søns, Kong Jojakim af Judas, fjerde Regeringsår:
\par 2 Så siger HERREN, Israels Gud, om dig, Baruk:
\par 3 Fordi Baruk siger: Ve mig, thi Kummer har HERREN føjet til min Smerte, jeg er træt af at sukke og finder ej Hvile!
\par 4 skal du sige til ham: Så siger HERREN: Se, hvad jeg har bygget, nedbryder jeg; hvad jeg har plantet, rykker jeg op; det gælder al Jorden
\par 5 og du søger store Ting for dig selv! Gør det ikke! Thi se, jeg sender Ulykke over alt Kød, lyder det fra HERREN. Men dig giver jeg dit Liv som Bytte, alle Vegne hvor du kommer.

\chapter{46}

\par 1 HERRENs Ord, som kom til Profeten Jeremias om folkene.
\par 2 Til Ægypten, om Ægypterkongen Farao Nekos Hær, som stod ved Floden Eufrat i Karkemisj, og som Kong Nebukadrezar af Babel slog i Josiass Søns, Kong Jojakim af Judas, fjerde Regeringsår.
\par 3 Gør Skjold og Værge rede, kom hid til Strid!
\par 4 Spænd Hestene for, sid op på Gangeme, stil eder op med Hjelmene på, gør Spydene blanke, tag Brynjeme på!
\par 5 Hvorfor er de rædselsslagne, veget tilbage deres Helte knust, på vild Flugt uden at vende sig? Trindt om er Rædsel, lyder det fra HERREN:
\par 6 De rapfodede undflyr ikke, og Helten redder sig ikke. Mod Nord ved Eufrats Flod falder de og styrter.
\par 7 Hvem stiger der som Nilen, hvis Vande svulmer som Strømme?
\par 8 Det er Ægypten, der stiger som Nilen, og Vandene svulmer som Strømme. Det tænkte: "Jeg vil stige op og oversvømme Jorden, ødelægge dem, som bor derpå."
\par 9 Stejl, I Heste, tag vanvittig Fart, I Vogne, lad Heltene rykke frem, Kusj, Put, som bærer Skjold, og Luderne, som spænder Bue.
\par 10 Dette er Herrens, Hærskarers HERREs Dag, en Hævnens Dag til Hævn over hans Fjender. Sværdet æder sig mæt og svælger i deres Blod; thi Herren Hærskarers HERRE har Offerslagtning i Nordens Land ved Eufrats Flod.
\par 11 Drag op til Gilead og hent Balsam, du Jomfru, Ægyptens Datter! Forgæves bruger du Lægemidler i Mængde; der er ingen Lægedom for dig.
\par 12 Folkene hører dit Råb, dit Skrig opfylder Jorden; thi Helt snubler over Helt, sammen styrter de begge,
\par 13 Det Ord, HERREN talede til Profeten Jeremias, om at Kong Nebukadrezar af Babel skulle komme og slå Ægypten.
\par 14 Forkynd det i Ægypten, kundgør det i Migdol, kundgør det i Nof og Takpankes! Sig: Stil dig op og gør dig rede, thi Sværdet fortærer trindt om dig.
\par 15 Hvorfor flyede Apis, din Tyr? Den holdt ikke Stand, fordi HERREN jog den bort.
\par 16 Din brogede Folkesværm falder og styrter; de siger til hverandre: "Kom, lad os vende hjem til vort Folk og vort Fædreland for det hærgende Sværd!"
\par 17 Kald Farao, Ægyptens konge: Bulderet, som lader den belejlige Tid gå forbi.
\par 18 Så sandt jeg lever, siger Kongen, hvis Navn er Hærskarers HERRE: Som Tabor mellem Bjergene, som Karmel ved Havet kommer han.
\par 19 Skaf dig Rejsetøj, du, som bor der, Ægyptens Datter! Thi Nof skal ødelægges og afbrændes, så ingen bor der.
\par 20 En smuk kvie er Ægypten, men en Bremse fra Nord falder over det.
\par 21 Selv dets Lejesvende, der er som Fedekalve, vender sig alle til Flugt; de holder ikke Stand, thi deres Ulykkes Dag er kommet over dem, deres Hjemsøgelses Tid.
\par 22 Dets Røst er som den hvislende Slanges; thi med Hærmagt farer de frem, og med Økser kommer de over det som Brændehuggere.
\par 23 De fælder dets Skov, lyder det fra HERREN, fordi den ikke er til at trænge igennem. Thi de er talrigere end Græshopper, ikke til at tælle.
\par 24 Til Skamme bliver Ægyptens Datter; hun gives i Nordfolkets Hånd.
\par 25 Så siger Hærskarers HERRE, Israels Gud: Se, jeg hjemsøger Amon i No og Farao og Ægypten med dets Guder og Konger, Farao og dem, der stoler på ham;
\par 26 og jeg giver dem i deres Hånd, som står dem efter Livet, i kong Nebukadrezar af Babels og hans Tjeneres Hånd; men siden skal Landet bebos som i fordums Tid lyder det fra HERREN.
\par 27 Frygt derfor ikke, min Tjener Jakob, vær ikke bange, Israel; thi se, jeg frelser dig fra det fjerne og dit Afkom fra deres Fangenskabs Land; og Jakob skal vende hjem og bo roligt og trygt, og ingen skal skræmme ham.
\par 28 Frygt ikke, min Tjener Jakob, lyder det fra HERREN, thi jeg er med dig; thi jeg vil tilintetgøre alle de Folk, blandt hvilke jeg har adsplittet dig; kun dig vil jeg ikke tilintetgøre; jeg vil tugte dig med Måde, ikke lade dig helt ustraffet.

\chapter{47}

\par 1 HERRENs Ord, som kom til profeten Jeremias om filistrene, før Farao slog Gaza.
\par 2 Så siger HERREN: Se, Vande stiger fra Nord, de bliver en Strøm, der svømmer over, de oversvømmer Landet og dets Fylde, Byerne og dem, som bor der.
\par 3 For Lyden af hans Hingstes Hovslag, hans Vognes Drøn, hans raslende Hjul får Fædre ej set efter Bøm, thi Hænderne er slappe,
\par 4 nu Dagen er kommet at ødelægge alle Filistre, at udrydde hver Hjælper, som levnes Tyrus og Zidon; thi HERREN ødelægger Filisterne, Resten af Kaftors Ø.
\par 5 Skaldet er Gaza blevet, Askalon tilintetgjort. Du Rest af Anakiter, hvor længe vil du såre dig?
\par 6 Ve, HERRENs Sværd, hvornår vil du falde til Ro? Far i din Skede, hvil og vær stille!
\par 7 Hvorledes får det Ro, når HERREN opbød det mod Askalon og Havets Strand og stævned det did?

\chapter{48}

\par 1 Om Moab. Så siger Hærskares Herre, Israels Gud. Ve over Nebo, thi det er lagt øde, blevet til Skamme; indtaget er Kirjatajim, med Skam er Borgen brudt ned.
\par 2 Der er ingen Lægedom mer for Moab, intet Fryderåb i Hesjbon; de oplægger onde Råd imod det: "Kom, lad os udrydde det af Folkenes Tal!" Også du, Madmen, skal omkomme, Sværdet skal forfølge dig.
\par 3 Hør Skriget fra Horonajim, frygteligt Brag og Sammenbrud!
\par 4 Moab er brudt sammen; lad Skriget lyde til Zoar.
\par 5 Ak, grædende stiger de op ad Luhits Skråning; ak, på Vejen til Horonajim hører de Jammerskrig.
\par 6 Fly, red eders Liv, og I skal blive som en Enebærbusk i Ørkenen.
\par 7 Ja, fordi du stolede på dine Borge og Skatte, skal også du fanges. Kemosj skal vandre i Landflygtighed, hans, Præster og Fyrster til Hobe.
\par 8 Hærværksmænd skal komme over hver By, ingen By skal reddes; Dalen skal ødelægges og Højsletten hærges, som HERREN har sagt.
\par 9 Giv Moab Vinger, at det kan flyve bort; dets Byer skal blive en Ørken, så ingen bor der.
\par 10 Forbandet være den, der er lad til at gøre HERRENs Værk, forbandet den, som holder sit Sværd fra Blod.
\par 11 Moab var tryg fra sin Ungdom, lå roligt på sin Bærme; det hældtes ikke fra Fad til Fad og vandrede ikke i Landflygtighed; derfor holdt det sin Smag, og dets Duft tabte sig ikke.
\par 12 Se, derfor skal Dage komme, lyder det fra HERREN, da jeg sender Vintappere, som skal tappe det og tømme dets Fade og knuse dets Dunke.
\par 13 Da skal Moab få Skam af Kemosj, som Israels Hus havde Skam af Betel, som de stolede på.
\par 14 Hvor kan I sige: "Helte er vi og djærve Folk til Krig?"
\par 15 Moab skal hærges med sine Byer og dets ypperste Ynglinge stige ned til at slagtes, lyder det fra Kongen, hvis Navn er Hærskarers HERRE.
\par 16 Moabs Undergang er nær, dets Ulykke kommer såre hastigt.
\par 17 Ynk det, alle dets Naboer og alle, som kender dets Navn; sig: Hvor knækkedes dog den stærke Stav, det herlige Spir!
\par 18 Stig ned fra Æressædet, sæt dig i Skarnet, du, som bor der, Dibons Datter! Thi han, der hærger Moab, drager op imod dig, nedbryder dine Fæstninger.
\par 19 Stå hen på Vejen og se dig om, du, som bor i Aroer, spørg Flygtningene og de undslupne Kvinder, sig: "Hvad er der sket?"
\par 20 Moab er blevet til Skamme, ja knust. Jamrer og skrig, meld ved Arnon, at Moab er hærget,
\par 21 at Dommen er kommet over Højslettelandet, over Holon, Jaza, Mefaat,
\par 22 Dibon, Nebo, Bet-Diblatajim,
\par 23 Kirjatajim, Bet-Gamul, Bet Meon,
\par 24 Kerijot, Bozra og alle Byer i Moabs Land fjernt og nær.
\par 25 Afhugget er Moabs Horn, og dets Arm er brudt, lyder det fra HERREN.
\par 26 Gør det drukkent! Thi det hovmodede sig mod HERREN; og Moab skal falde omkuld i sit eget Spy, også det skal blive til Latter.
\par 27 Var ikke Israel til Latter for dig? Blev det måske grebet blandt Tyve, siden du bliver så ivrig, hver Gang du taler derom?
\par 28 Kom fra Byerne og fæst Bo på Klippen, Moabs Indbyggere, vær som Duen, der bygger Rede hist ved Afgrundens Rand.
\par 29 Vi har hørt om Moabs Hovmod, det såre store, dets Stolthed. Overmod og Hovmod, dets opblæste Hjerte.
\par 30 Jeg kender, lyder det fra HERREN, dets Frækhed, dets tomme Snak, dets tomme Gerninger.
\par 31 Derfor må jeg jamre over Moab, skrige over hele Moab, over Mændene i Kir-Heres må jeg sukke.
\par 32 Jazers Gråd græder jeg over dig, Sibmas Vinstok; dine Skud overskred Havet, nåede til Jazer; på din Frugt og din Høst slog Hærværksmanden ned.
\par 33 Glæde og Jubel er svundet fra Frugthaven og Moabs Land. Jeg lader Vinen svinde fra Persekarrene, ingen træder Vin.
\par 34 Hesjbon og Elale skriger, det høres til Jahaz; Horonajim og Eglat-Sjelisjija skriger; ak, Nimrims Vande bliver Ødemarker.
\par 35 Jeg udrydder af Moab den, der stiger op på Offerhøjen og tænder Offerild for dets Guder, lyder det fra HERREN.
\par 36 Derfor klager mit Hjerte som Fløjter over Moab, og mit Hjerte klager som Fløjter over Kir-Heress Mænd. Godset, de vandt, går derfor til Spilde.
\par 37 Thi hvert Hoved er skaldet, hvert Skæg revet af; i alle Hænder er der Rifter, over alle Lænder Sæk.
\par 38 Alt er Klage på alle Moabs Tage og Torve; thi jeg sønderbryder Moab som et usselt Kar, lyder det fra HERREN.
\par 39 Hvor er Moab forfærdet! Hvor vender det Ryg med Skam! Ja, Moab er blevet til Latter og Rædsel for alle sine Naboer.
\par 40 Thi så siger HERREN: Se, som en Ørn med udbredte Vinger svæver han over Moab.
\par 41 Kerijot er taget og Borgene faldet. Moabs Heltes Hjerte bliver på hin Dag som en nødstedt Kvindes Hjerte.
\par 42 Moab er ødelagt og ikke mer et Folk, fordi det hovmodede sig mod HERREN.
\par 43 Gru og Grav og Garn kommer over dig, du, som bor i Moab, lyder det fra HERREN;
\par 44 den, der flygter for Gru, falder i Grav, den, der når op af Grav, fanges i Garn. Thi jeg bringer over Moab deres Hjemsøgelses År, lyder det fra HERREN.
\par 45 I Ly af Hesjbon står Flygtninge uden Kraft. Thi Ild farer ud fra Hesjbon, Ildsluefra Sihons Stad; den fortærer Moabs Tinding og de larmende Mænds isse.
\par 46 Ve dig, Moab, det er ude med dig, Kemosjs Folk. Thi dine Sønner slæbes i Fangenskab, dine Døtre ligeså.
\par 47 Menjeg vender Moabs Skæbne i de sidste Dage, lyder det fra HERREN. Så vidt Moabs Dom.

\chapter{49}

\par 1 Om Ammonitterne. Så siger Herren, har Israel ingen sønner eller har det ingen arvinger? Hvorfor har Milkom taget Gad i Eje og hans Folk bosat sig i dets Byer?
\par 2 Se, derfor skal Dage komme, lyder det fra HERREN, da jeg lader Krigsskrig lyde mod Rabba i Ammon, og det skal blive en Grusdynge, og dets Døtre skal gå op i Luer. Da arver Israel sine Arvinger, siger HERREN.
\par 3 Klag, Hesjbon, thi Aj er ødelagt; skrig, I Rabbas Døtre klæd jer i Sæk og klag, gå rundt i Foldene! Thi Milkom vandrer i Landflygtighed, hans Præster og Fyrster til Hobe.
\par 4 Hvorfor gør du dig til af dine Dale, du frafaldne Datter, som stoler på dine Skatte og siger: "Hvem kan komme til mig,?"
\par 5 Se, jeg lader Rædsel komme over dig fra alle Kanter, lyder det fra Hærskarers HERRE. I skal drives bort i hver sin Retning, og ingen samler de flygtende.
\par 6 Men siden vender jeg Ammoniternes Skæbne, lyder det fra HERREN.
\par 7 Om Edom. Så siger Hærskarers HERRE: Er der ikke mer Visdom i Teman, svigter de kloges Råd, er deres Visdom rådden?
\par 8 Fly, søg Ly i det dybe, I, som bor i Dedan! Thi Esaus Ulykke sender jeg over ham, Straffens Tid.
\par 9 Gæstes du af Vinhøstmænd, levner de ej Efterslæt, af Tyve om Natten, ødelægger de, hvad de lyster.
\par 10 Thi selv blotter jeg Esau, hans Skjulesteder røber jeg; at gemme sig evner han ikke. Han er ødelagt ved Brødres og Naboers Arm, han er borte.
\par 11 Lad mig om dine faderløse, jeg holder dem i Live, dine Enker kan stole på mig.
\par 12 Thi så siger HERREN: Se, de, hvem det ikke tilkom at tømme Bægeret, må tømme det, og du skulde gå fri? Du går ikke fri, men kommer til at tømme det.
\par 13 Thi jeg sværger ved mig selv. lyder det fra HERREN: til Rædsel og Spot, til Ørk og til et Forbandelsens Tegn skal Bozra blive, og alle dets Byer skal blive til evige Tomter.
\par 14 Fra HERREN har jeg hørt en Tidende: Et Bud skal sendes ud blandt Folkene: Samler eder! Drag ud imod det og rejs jer til Strid!
\par 15 Se, ringe har jeg gjort dig iblandt Folkene, foragtet blandt Mennesker.
\par 16 Rædsel over dig! Dit Hjertes Overmod bedrog dig. Du, som bor i Klippekløft og klynger dig til Fjeldtop: Bygger du Rede højt som Ørnen, jeg styrter dig ned, så lyder det fra HERREN.
\par 17 Edom skal blive til Rædsel; alle, der kommer forbi, skal slås af Rædsel og spotte over alle dets Sår.
\par 18 Som det gik, da Sodoma og Gomorra og Nabobyerne omstyrtedes, siger HERREN, skal intet Menneske bo der, intet Menneskebarn dvæle der.
\par 19 Som en Løve, der fra Jordans Stolthed skrider op til den stedsegrønne Græsgang, således vil jeg i et Nu drive dem bort derfra. Thi hvem er den udvalgte, jeg vil sætte over dem? Thi hvem er min Lige, og hvem kræver mig til Regnskab? Hvem er den Hyrde, der står sig mod mig?
\par 20 Hør derfor det Råd, HERREN har for mod Edom, og de Tanker, han har mod Temans Indbyggere: Visselig skal Hjordens ringeste slæbes bort, visselig skal deres Græsgang forfærdes over dem.
\par 21 Ved Braget af deres Fald skal Jorden skælve; Skriget kan høres til det røde Hav.
\par 22 Se, som en Ørn med udbredte Vinger svæver han over Bozra; og Edoms Heltes Hjerte bliver på hin Dag som en nødstedt Kvindes Hjerte.
\par 23 Om Damaskus. Til Skamme er Hamat og Arpad, thi de hører ond Tidende; de er ude af sig selv, i Uro som Havet, der ikke kan falde til Ro.
\par 24 Damaskus er modfaldent, vender sig til Flugt, Angst falder over det, Vånde og Veer griber det som en fødende Kvinde.
\par 25 Ve det! Forladt er den lovpriste By, Glædens Stad.
\par 26 Derfor falder dets Ynglinge på dets Torve, alle Krigsfolkene omkommer på hin Dag, lyder det fra Hærskarers HERRE.
\par 27 Jeg sætter Ild på Damaskuss Mur, og den skal fortære Benhadads Borge.
\par 28 Om Kedar og Hazors Riger, som Kong Nebukadrezar af Babel slog. Så siger HERREN: Kom og drag op mod Kedar, ødelæg Østens Sønner!
\par 29 Man skal tage deres Telte og Hjorde, deres Telttæpper, alle deres Kar, bortføre Kamelerne fra dem og råbe til dem: "Trindt om er Rædsel!"
\par 30 Fly i Hast, søg Ly i det dybe, Hazors Borgere, lyder det fra HERREN. Thi kong Nebukadrezar af Babel har oplagt et Råd imod eder og undfanget en Tanke imod eder.
\par 31 Kom, drag op mod et roligt Folk, der bor i Tryghed, lyder det fra HERRREN, uden Porte og Slåer; de bor for sig selv.
\par 32 Deres Kameler gøres til Bytte, deres mange Hjorde til Rov. Jeg spreder dem, der har rundklippet Hår, for alle Vinde, og fra alle kanter bringer jeg Undergang over dem, lyder det fra HERREN.
\par 33 Hazor bliver Sjakalers Bo, en Ørken til evig Tid; der skal ej bo et Menneske, ej dvæle et Menneskebarn.
\par 34 HERRENs Ord, som kom til Profeten Jeremias om Elam i Kong Zedekias af Judas første Regeringstid:
\par 35 Så siger Hærskarers HERRE: Jeg knækker Elams Bue, det ypperste af deres Kraft;
\par 36 og jeg bringer over Elam de fire Vinde fra de fire Verdenshjørner og spreder dem for alle disse Vinde; der skal ikke være et Folk, som de bortdrevne Elamiter ikke kommer hen til.
\par 37 Jeg knuser dem foran deres Fjender og dem, der står dem efter Livet, og jeg sender Ulykke over dem, min glødende Vrede, lyder det fra HERREN. Jeg sender Sværdet efter dem, til jeg får dem udslettet.
\par 38 Jeg rejser min Trone i Elam og tilintetgør der både Konge og Fyrster, lyder det fra HERREN.
\par 39 Men i de sidste Dage vender jeg Elams Skæbne, lyder det fra HERREN.

\chapter{50}

\par 1 Det ord Herren talte mod Babel, mod kaldærenes land, ved profeten Jeremias.
\par 2 Forrkynd det blandt Folkene, kundgør det, rejs et Banner, kundgør det, dølg det ikke, sig: Babel er indtaget, Bel gjort til Skamme, Merodak knust, til Skamme er dets Afguder blevet, knust dets Afgudsbilleder.
\par 3 Thi et Folk fra Nord drager op imod det og gør dets Land til en Ørken, så ingen bor der; både Mennesker og Dyr er flygtet.
\par 4 I hine Dage og til hin Tid, lyder det fra HERREN, skal israeliterne, sammen med Judæerne, komme: de skal vandre under Gråd og søge HERREN deres Gud;
\par 5 de skal spørge om Vej til Zion, did er deres Ansigtervendt; de skal komme og klynge sig til HERREN i en evig Pagt, der aldrig glemmes.
\par 6 En Flok bortkomne Får var mit Folk, deres Hyrder havde ført dem vild, på Afveje i Bjergene; de flakkede fra Bjerg til Høj, glemte deres Hvilested.
\par 7 Enhver, som traf på dem, fortærede dem; deres Fjender sagde: "Vi er sagesløse!" Det skete, fordi de syndede mod HERREN, Retfærds græsgangen og deres fædres Håb, HERREN.
\par 8 Fly ud af Babel, drag bort fra kaldæernes Land, bliv som Bukke foran en Hjord!
\par 9 Thi se, jeg vækker fra Nordens Land en Sværm af vældige Folk og fører dem frem mod Babel, og de skal ruste sig imod det; fra den Kant skal det indtages; dens Pile er som den sejrsæle Helts, der ikke vender tomhændet hjem.
\par 10 Kaldæa gøres til Bytte; alle, som gør det til Bytte, mættes, lyder det fra HERREN.
\par 11 Glæd eder kun og jubl, I, som plyndrede min Arvelod, spring som Kalve i Engen, vrinsk som Hingste
\par 12 eders Moder skal dybt beskæmmes; hun, som bar eder, skal blive til Skamme.
\par 13 For HERRENs Vredes Skyld skal det ligge ubeboet hen og overalt være en Ørken; alle, som kommer forbi Babel, skal slås af Rædsel og spotte over alle dets Sår.
\par 14 Rust eder mod Babel på alle Kanter, alle, som spænder Bue; skyd på det, spar ikke på Pile, thi mod HERREN har det syndet.
\par 15 Jubl over det fra alle kanter: "Det har udrakt sin Hånd, dets Støttemure er faldet, dets Volde nedbrudt." Thi det er HERRENs Hævn. Hævn eder på det, gør med det, som det selv har gjort!
\par 16 Udryd af Babel den, der sår, og den, der svinger Le i Høstens Tid! For det hærgende Sværd vender enhver hjem til sit Folk, enhver flyr til sit Land.
\par 17 En adsplittet Hjord er Israel, Løver har spredt det. Først fortærede Assyrerkongen det, og nu sidst har Kong Nebukadrezar af Babel gnavet dets Knogler.
\par 18 Derfor, så siger Hærskarers HERRE, Israels Gud: Se, jeg hjemsøger Babels Konge og hans Land, som jeg hjemsøgte Assyrerkongen;
\par 19 og jeg fører Israel tilbage til dets Græsgang; det skal græsse på Karmel og Basan og mættes i Efraims Bjerge og Gilead.
\par 20 I hine Dage og til hin Tid, lyder det fra HERREN, skal man søge efter Israels Brøde, og den er der ikke, efter Judas Synder, og de findes ikke; thi jeg tilgiver dem, jeg lader blive til Rest,
\par 21 Drag op mod Meratajims Land, drag op imod det og mod dem, som bor i Pekod, læg øde, læg Band på dem, så lyder det fra HERREN, gør nøje, som jeg har budt dig!
\par 22 Krigslarm lyder i Landet, alt bryder sammen.
\par 23 Hvor er dog al Jordens Hammer knækket og brudt, hvor er dog Babel blevet til Rædsel blandt Folkene!
\par 24 Jeg lagde dig Snarer, du fangedes, Babel, og mærked det ej; du grebes, og fast blev du holdt, thi du kæmped mod HERREN.
\par 25 HERREN lukked op for sit Forråd og fremtog sin Vredes Værktøj. Thi et Værk har Herren, Hærskarers HERRE, for i Kaldæernes Land.
\par 26 Træng derind fra Ende til anden, luk op for dets Lader, dyng det op som Neg og læg Band derpå, lad intet levnes.
\par 27 ødelæg alle dets Okser, før dem ned til Slagtning! Ve dem, deres Dag er kommet, Hjemsøgelsens Tid.
\par 28 Hør, hvor de flyr og redder sig fra Babels Land for at melde i Zion om Hævnen fra HERREN vor Gud, Hævn for hans Helligdom.
\par 29 Kald Skytterne sammen mod Babel, enhver, som spænder Bue, slå Ring omkring det, lad ingen få Lov at slippe; gengæld det efter dets Gerning; efter alt, hvad det gjorde, skal I gøre imod det; thi Frækhed viste det mod HERREN, Israels Hellige.
\par 30 Derfor falder dets Ynglinge på dets Torve, alle Krigsfolkene omkommer på hin Dag, lyder det fra HERREN.
\par 31 Se, jeg kommer over dig, "Frækhed", lyder det fra Herren, Hærskarers HERRE, thi din Dag er kommet, Hjemsøgelsens Tid.
\par 32 Da falder "Frækhed" og styrter, og ingen rejser det. Jeg sætter Ild på dets Byer, og den fortærer alt deromkring.
\par 33 Så siger Hærskarers HERRE: Både med Israeliterne og Judæerne er der handlet ilde; alle de, der bortførte dem, holder fast på dem, vægrer sig ved at give dem fri.
\par 34 Deres Genløser er stærk, Hærskarers HERRE er hans Navn; han vil føre deres Strid og give Jorden Ro og Babels Indbyggere Uro.
\par 35 Sværd over Kaldæerne, lyder det fra HERREN, og over Babels Indbyggere, over dets Fyrster og Vismænd!
\par 36 Sværd over Sandsigerne, så de bliver Tåber! Sværd over dets Belte, så de taber Modet!
\par 37 Sværd over dets Heste og Vogne og over alt det blandede Slæng i dets Midte, så de bliver til Kvinder! Sværd over dets Skatte, så de plyndres!
\par 38 Tørke over dets Vande, så de tørres ud! Thi det er et Land for Gudebilleder, og de gør sig til af dem, de frygter.
\par 39 Derfor skal Ørkendyr bo der sammen med Sjakaler, også Strudse skal bo der; aldrig mer skal det bebos, men være ubeboet fra Slægt til Slægt.
\par 40 Som det gik, da Gud omstyrtede Sodoma og Gomorra og Nabobyerne, lyder det fra HERREN, skal intet Menneske bo der, intet Menneskebarn dvæle der.
\par 41 Se, der kommer et Folk fra Nord, et vældigt Folk og mange Konger bryder op fra det yderste af Jorden.
\par 42 De fører Bue og Spyd, er grumme uden Barmhjertighed, deres Røst er som Havets Brusen, de rider på Heste, rustet som en Mand til Strid mod dig, Babels Datter!
\par 43 Babels Konge hørte Rygtet derom, og hans Hænder blev slappe, Rædsel greb ham, Skælven som den fødende Kvindes.
\par 44 Som en Løve, der fra Jordans Stolthed skrider op til den stedsegrønne Græsgang, således vil jeg i et Nu drive dem bort derfra. Thi hvem er den udvalgte, jeg vil sætte over dem? Thi hvem er min Lige, og hvem kræver mig til Regnskab? Hvem er den Hyrde, der står sig mod mig?
\par 45 Hør derfor det Råd, HERREN har for mod Babel, og de Tanker, han har tænkt mod Kaldæernes Land: Visselig skal Hjordens ringeste slæbes bort, visselig skal deres Græsgang forfærdes over dem.
\par 46 Ved Råbet: "Babel er indtaget!" skal Jorden skælve, og deres Skrig skal høres blandt Folkene.

\chapter{51}

\par 1 Så siger HERREN: Jeg opvækker en ødelæggelsens ånd mod Babel og dem, som bor i "mine Modstanderes Hjerte".
\par 2 Og jeg sender til Babel Kastere, de skal kaste det og tømme dets Land, thi fra alle Hanter er de over det på Ulykkens Dag.
\par 3 Ingen skal spænde sin Bue eller rejse sig i Brynje. Spar ikke dets Ynglinge, læg Band på hele dets Hær!
\par 4 Dræbte Mænd skal falde i Kaldæernes Land og gennemborede i Gaderne;
\par 5 thi Israel og Juda er ikke forladt af deres Gud, Hærskarers HERRE, men deres Land var fuldt af Skyld mod Israels Hellige.
\par 6 Fly ud af Babel, enhver redde sit Liv, at I ikke skal omkomme for dets Brøde! Thi det er Hævnens Tid for HERREN, han øver Gengæld imod det.
\par 7 Et gyldent Bæger var Babel i HERRENs Hånd, det gjorde al Jorden drukken; Folkene drak af Vinen, derfor blev Folkene galne.
\par 8 Babel faldt i et Nu, det knustes; jamrer over det! Hent Balsam hid til dets Sår, om det muligt kan læges!
\par 9 Vi vilde læge Babel, men det lod sig ikke læge. Gå fra det og lad os drage hver til sit Land, thi dets Straffedom når til Himmelen, løfter sig til Skyerne.
\par 10 HERREN har bragt vor Ret for Lyset; kom, lad os kundgøre HERREN vor Guds Værk i Zion!
\par 11 Hvæs Pilene, gør Skjoldene blanke! HERREN har vakt Mederkongens Ånd, thi hans Hu står til at ødelægge Babel; thi det er HERRENs Hævn, Hævn for hans Tempel.
\par 12 Løft Banner mod Babels Mure, forstærk Vagten, sæt Vagtposter ud, læg Baghold! Thi HERREN har et Råd for og gør, hvad han har talet mod Babels Indbyggere.
\par 13 Du, som bor ved de mange Vande, rig på Skatte, din Ende er kommet, den Alen, hvor man skære dig af.
\par 14 Hærskarers Herre har svoret ved sig selv: Jeg vil fylde dig med Mennesker som Græshopper, og de skal istemme Vinperserråbet over dig.
\par 15 Han skabte Jorden i sin Vælde, grundfæstede Jorderig i sin Visdom, og i sin indsigt udspændte han Himmelen.
\par 16 Når han løfter sin Røst, bruser Vandene i Himmelen, og han lader Skyer stige op fra Jordens Ende; han får Lynene til at give Regn og sender Stormen ud af sine Forrådskamre.
\par 17 Dumt er hvert Menneske, uden Indsigt, hver Guldsmed får Skam af sit Billede; thi Løgn er hans Støbning, der er ingen Ånd i dem;
\par 18 Tomhed er de, et dårende Værk; når deres Hjemsøgelses Tid kommer, er det ude med dem.
\par 19 Jakobs Arvelod er ikke som de; thi han, der har skabt alt, er dets Arvelod; Hærskarers HERRE er hans Navn.
\par 20 Du var mig en Stridshammer, et Våben; med dig knuste jeg Folk, med dig ødelagde jeg Riger;
\par 21 med dig knuste jeg Hest og Rytter, med dig knuste jeg Vogn og Vognstyrer,
\par 22 med dig knuste jeg Mand og Kvinde med dig knuste jeg gammel og ung, med dig knuste jeg Yngling og Jomfru,
\par 23 med dig knuste jeg Hyrde og Hjord, med dig knuste jeg Agerdyrker og Oksespand, med dig knuste jeg Statholder og Landshøvding.
\par 24 Men jeg vil gengælde Babel og alle Kaldæas Indbyggere alt det onde, de gjorde mod Zion for eders Øjne, lyder det fra HERREN.
\par 25 Se, jeg kommer over dig, du ødelæggende Bjerg, lyder det fra HERREN, du, som ødelægger hele Jorden; jeg udrækker Hånden imod dig og vælter dig ned fra Klipperne og gør dig til et afsvedet Bjerg;
\par 26 man skal ikke fra dig hente Sten til Tinder eller Grundvolde, thi du skal blive en evig Ørken, lyder det fra HERREN.
\par 27 Løft Banner på Jorden, stød i Horn blandt Folkene, vi Folkene til Kamp imod det, opbyd Ararats, Minnis og Asjkenazs Riger imod det, indsæt en Tipsar, lad Hestene fare frem som lodne Græshopper;
\par 28 vi Folkene til Strid imod det, Mederkongen, hans Statholdere og alle hans Landshøvdinger og hele det Land, han råder over!
\par 29 Jorden skal skælve og vride sig, thi HERRENs Tanker mod Babel fuldbyrdes, at gøre Babels Land til en Ørken, hvor ingen bor.
\par 30 Babels Helte opgiver Kampen, de sidder stille i Borgene, deres kraft ebber ud, de er blevet til Kvinder; dets Boliger afbrændes, dets Portstænger knækkes.
\par 31 Løber iler Løber i Møde, og Bud iler Bud i Møde for at melde Babels Konge, at hans By er indtaget fra Ende til anden,
\par 32 Overgangsstederne taget, Borgene brændt og Krigsfolkene rædselslagne.
\par 33 Thi så siger Hærskarers HERRE, Israels Gud: Babels Datter er som en Tærskeplads, når den stampes endnu en liden Stund, så kommer Høstens Tid for den.
\par 34 Kong Nebukadrezar af Babel har fortæret mig, oprevet mig, sat mig til Side som et tomt kar. Som en Drage har han slugt mig, fyldt sin Vom med mine Lækkerbidskener og drevet mig bort.
\par 35 Den Vold, jeg led, og min Overlast komme over Babel, siger de, som bor i Zion, mit Blod over Kaldæas Indbyggere, siger Jerusalem.
\par 36 Derfor, så siger HERREN: Se, jeg fører din Sag og giver digHævn, jeg lægger dets Hav tørt og udtørrer dets Kilde.
\par 37 Babel skal blive en Grushob, Sjakalers Bolig, til Rædsel og Spot, så ingen bor der.
\par 38 De brøler alle som Løver, knurrer som Løveunger i deres Vildskab.
\par 39 Jeg holder et Drikkelag for dem og gør dem drukne, så de døves og falder i evig Søvn uden at vågne, lyder det fra HERREN.
\par 40 Jeg fører dem ned til af slagtes som Får, som Vædre sammen med Bukke.
\par 41 Hvor Sjesjak blev fanget og grebet, al Jordens Stolthed, hvor Babel dog blev til Rædsel imellem Folkene!
\par 42 Havet steg over Babel, af dets Bølgers Brus blev det skjult.
\par 43 Dets Byer er blevet en Ørken, øde Land og Ødemark; intet Menneske bor i dem, intet Menneskebarn færdes i dem.
\par 44 Jeg hjemsøger Bel i Babel, river ud af hans Mund, hvad han slugte, til ham skal ej Folkeslag strømme mer. Også Babels Mur er faldet.
\par 45 Drag ud deraf, mit Folk, enhver redde sit Liv for HERRENs glødende Harme.
\par 46 Lad ikke eders Hjerter blive modfaldne og frygt ikke ved de Tidender, der høres på Jorden, når der i det ene År kommer een Tidende og i det næste en anden, når der er Voldsfærd på Jorden og Hersker følger på Hersker.
\par 47 Se, derfor skal Dage komme, da jeg hjemsøger Babels Gudebilleder, og alt dets Land bliver til Skamme, og alle deri skal falde på Valen.
\par 48 Jubl over Babel, Himmel og Jord med alt, hvad i dem er, thi fra Nord kommer Hærværksmænd over det, lyder det fra HERREN.
\par 49 Også Babel skal falde for de dræbte af Israels Skyld, ligesom dræbte på hele Jorden faldt for Babel.
\par 50 I, som undslap Sværdet, drag bort, stands ikke, kom HERREN i Hu i det fjerne, lad Jerusalem komme frem i eders Tanker!
\par 51 Vi blev til Skamme, thi Smædeord måtte vi høre; Blusel lagde sig over vore Ansigter, thi fremmede overfaldt HERRENs Huss Helligdomme.
\par 52 Se, derfor skal Dage komme, lyder det fra HERREN, da jeg hjemsøger dets Gudebilleder, og sårede skal stønne i hele dets Land.
\par 53 Selv om Babel stiger op til Himmelen, og selv om det gør sin Borg utilgængelig i det høje, fra mig skal der komme Hærværksmænd over det, lyder det fra HERREN.
\par 54 Der lyder Skrig fra Babel, et vældigt Sammenbrud fra Kaldæernes Land.
\par 55 Thi HERREN hærger Babel og gør Ende på den vældige Larm der; deres Bølger bruser som mange Vande, deres Brag lyder højt.
\par 56 Thi en Hærværksmand kommer over Babel, og dets Helte skal fanges, deres Buer knækkes; thi en Gengældelsens Gud er HERREN, han giver fuld Løn.
\par 57 Jeg gør dets Fyrster og Vismænd, dets Statholdere, Landshøvdinger og Helte drukne, og de skal falde i evig Søvn uden at vågne, lyder det fra kongen, hvis Navn er Hærskarers HERRE.
\par 58 Så siger Hærskarers HERRE: Babels brede Mur skal nedbrydes til Grunden og dets høje Porte opbrændes. Folkeslagenes Møje er spildt, og Folkefærdene slider sig trætte for Ilden.
\par 59 Det Ord, som Profefen Jeremias sendte med Seraja, en Søn at Masejas Søn Nerija, da han rejste, til Babel med Kong Zedekias af Juda i hans fjerde Regeringsår; Seraja sørgede for Nattely til Kongen, når han var på Rejse.
\par 60 Jeremias optegnede al den Ulykke, som skulde komme over Babel, i en og samme Bog, alle disse Ord, der er skrevet mod Babel;
\par 61 og Jeremias sagde til Seraja: "Når du kommer til Babel og ser Lejlighed dertil, skal du oplæse alle disse Ord
\par 62 og sige: HERRE, du truede selv dette Sted med Udryddelse, så der ikke bliver nogen, som bor der, - hverken Folk eller Fæ, men det skal blive en evig Ørken.
\par 63 Og når du er til Ende med at oplæse denne Bog, skal du binde en Sten til den, kaste den i Eufrat
\par 64 og sige: Således skal Babel gå til Bunds og ikke mere komme op for al den Ulykke, jeg sender over det!" Til Ordene "slider sig trætte for Ilden går Jeremiass Ord.

\chapter{52}

\par 1 Zedekias var enogtyve år gammel da han blev konge, og han herskede elleve År i Jerusalem. Hans Moder hed Hamital og var en Datter af Jirmeja fra Libna.
\par 2 Han gjorde, hvad der var ondt i HERRENs Øjne, ganske som Jojakim.
\par 3 Thi for HERRENs Vredes Skyld kom dette over Jerusalem og Juda, og til sidst stødte han dem bort fra sit Åsyn. Og Zedekias faldt fra Babels konge.
\par 4 I hans niende Regeringsår på den tiende Dag i den tiende Måned drog Kong Nebukadrezar af Babel da med hele sin Hær mod Jerusalem, og de belejrede det og byggede Belejringstårne imod det rundt omkring;
\par 5 og Belejringen varede til Kong Zedekiass ellevte Regeringsår.
\par 6 På den niende Dag i den fjerde Måned blev Hungersnøden hård i Byen, og Folket fra Landet havde ikke Brød. Da blev Byens Mur gennembrudt.
\par 7 Alle krigsfolkene flygtede om Natten ud af Byen gennem Porten mellem de to Mure ved Kongens Have, medens Kaldæerne holdt Byen omringet, og de tog Vejen ad Arabalavningen til.
\par 8 Men Kaldæernes Hær satte efter Kongen og indhentede ham på Jerikosletten, efter at hele hans Hær var blevet splittet til alle Sider.
\par 9 Så greb de Kongen og bragte ham op til Ribla i Hamats Land til Babels Konge, der fældede Dommen over ham.
\par 10 Hans Sønner lod han henrette i hans Påsyn, ligeledes lod han alle Judas Øverster henrette i Ribla;
\par 11 og på Zedekias selv lod Babels Konge Øjnene stikke ud; derpå lod han ham lægge i Kobberlænker, og således førte han ham til Babel; og han lod ham kaste i Fængsel, hvor han blev til sin Dødedag.
\par 12 På den tiende Dag i den femte Måned, det var Babels Konge Nebukadrezars nittende Regeringsår, kom Nebuzaradan, Øversten for Livvagten, Babels konges Tjener, til Jerusalem.
\par 13 Han satte Ild på HERRENs Hus og Kongens Palads og alle Husene i Jerusalem; på alle Stormændenes Huse satte han Ild;
\par 14 og Murene om Jerusalem nedbrød hele kaldæernes Hær, som Øversten for Livvagten havde med sig.
\par 15 De sidste Folk, som var tilbage i Byen, og Overløberne, der var gået over til Babels Konge, og de sidste Håndværkere førte Nebuzaradan, Øversten for Livvagten, bort.
\par 16 Men nogle af de fattigste af Folket fra Landet lod Nebuzaradan, Øversten for Livvagten, blive tilbage som Vingårdsmænd og Agerdyrkere,
\par 17 Kobbersøjlerne i HERRENs Hus, Stellene og Kobberhavet i HERRENs Hus slog Kaldæerne i Stykker og førte Kobberet til Babel.
\par 18 Karrene, Skovlene, knivene og Kanderne og alle Kobbersagerne, som brugtes ved Tjenesten, røvede de;
\par 19 også Fadene, Panderne, Skålene, Karrene, Lysestagerne, Kanderne og Offerskålene, der helt var af Guld eller Sølv, røvede Øversten for Livvagten.
\par 20 De to Søjler, Havet med de tolv Kobberokser under og Stellene, som Salomo havde ladet lave til HERRENs Hus Kobberet i alle disse Ting var ikke til at veje.
\par 21 Atten Alen høj var hver Søjle, og en Snor på tolv Alen kunde nå om den, og den var hul, og Kobberet var fire Fingre tykt.
\par 22 Og der var et Søjlehoved af Kobber oven på den, fem Alen højt, og rundt om Søjlehovedet var der Fletværk og Granatæbler, alt af Kobber; og på samme Måde var det med den anden Søjle.
\par 23 Og der var seks og halvfemsindstyve Granatæbler, som hang frit; der var i alt hundrede Granatæbler rundt om Fletværket.
\par 24 Øversten for Livvagten tog Ypperstepræsten Seraja, Anden præsten Zefanja og de tre Dørvogtere;
\par 25 og fra Byen tog han en Hofmand, der havde Opsyn med krigsfolket, og syv Mænd, der hørte til Kongens nærmeste Omgivelser, og som endnu fandtes i Byen, desuden Hærførerens Skriver, der udskrev Folket fra Landet til Krigstjeneste, og dertil tresindstyve Mænd at Folket fra Landet, der fandtes i Byen
\par 26 dem tog Øversten for Livvagten Nebuzaradan og førte til Babels Konge i Ribla
\par 27 og Babels Konge lod dem dræbe i Ribla i Hamats Land. Så førtes Juda i Landflygtighed fra sit Land.
\par 28 Følgende er Tallet på de Folk, Nebukadrezar bortførte i Fangenskab: I hans syvende År 3023 Judæere,
\par 29 i Nebukadrezars attende År 832 fra Jerusalem;
\par 30 i Nebukadrezars tre og tyvende År bortførte Nebuzaradan, Øversten for Livvagten, 745 af Judæerne; tilsammen 4600.
\par 31 I det syv og tredivte År efter Kong Jojakin af Judas Bortførelse på den fem og tyvende Dag i den tolvte Måned tog Babels Konge Evil-Merodak, der i det År kom på Tronen, Kong Jojakin af Juda til Nåde og førte ham ud af Fængselet.
\par 32 Han talte ham venligt til og gav ham Sæde oven for de Konger, som var hos ham i Babel.
\par 33 Jojakin aflagde sin Fangedragt og spiste daglig hos ham, så længe han levede.
\par 34 Han fik sit daglige Underhold af Babels Konge, hver Dag hvad han behøvede for den Dag, indtil sin Dødedag, så længe han levede.


\end{document}