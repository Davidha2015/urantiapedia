\begin{document}

\title{Ezekiels Bog}


\chapter{1}

\par 1 I det tredivte År på den femte dag i den fjerde Måned da jeg var blandt de landflygtige ved Floden Kebar, skete det, at Himmelen åbnede sig, og jeg skuede Syner fra Gud.
\par 2 Den femte Dag i Måneden - det var det femte År efter at Kong Jojakin var bortført -
\par 3 kom HERRENs Ord til Præsten Ezekiel, Buzis Søn, i Kaldæernes Land ved Floden Kebar, og HERRENs Hånd kom over ham der.
\par 4 Jeg skuede og se, et Stormvejr kom fra Nord, og en vældig Sky fulgte med, omgivet af Stråleglans og hvirvlende Ild, i hvis Midte det glimtede som funklende Malm.
\par 5 Midt i Ilden var der noget ligesom fire levende Væsener, og de så således ud: De havde et Menneskes Skikkelse;
\par 6 men de havde hver fire Ansigter og fire Vinger;
\par 7 deres Ben var lige og deres Fodsåler som en Kalvs; de skinnede som funklende Kobber, og deres Vingeslag var hurtigt;
\par 8 der var Menneskehænder under Vingerne på alle fire Sider.
\par 9 De fire levende Væseners Ansigter vendte sig ikke, når de gik, men de gik alle lige ud".
\par 10 Ansigterne så således ud: De havde alle fire et Menneskeansigt fortil, et Løveansigt til højre, et Okseansigt til venstre og et Ørneansigt bagtil;
\par 11 Vingerne var på dem alle fire udbredt opad, således at to og to rørte hinanden, og to skjulte deres Legemer.
\par 12 De gik alle lige ud; hvor Ånden vilde have dem hen gik de; de vendte sig ikke, når de gik.
\par 13 Midt imellem de levende Væsener var der noget som glødende Kul at se til, som Blus, der for hid og did imellem dem, og Ilden udsendte Stråleglans, og Lyn for ud derfra.
\par 14 Og Væsenerne løb frem og tilbage, som Lynglimt at se til
\par 15 Videre skuede jeg, og se, der var et Hjul på Jorden ved Siden af hvert af de fire levende Væsener;
\par 16 og Hjulene var at se til som funklende Krysolit; de så alle fire ens ud, og de var lavet således, at der i hvert Hjul var et andet Hjul;
\par 17 de kunde derfor gå til alle fire Sider; de vendte sig ikke, når de gik.
\par 18 Videre skuede jeg, og se, der var Fælge på dem, og Fælgene var på dem alle fire rundt om fulde af Øjne.
\par 19 Og når de levende Væsener gik, gik også Hjulene ved Siden af, og når Væsenerne løftede sig fra Jorden, løftede også Hjulene sig;
\par 20 hvor Ånden vilde have dem hen, gik Hjulene, og de løftede sig samtidig, thi det levende Væsens Ånd var i Hjulene;
\par 21 når Væsenerne gik, gik også de; når de standsede, standsede også de, og når de løftede sig fra Jorden, løftede også Hjulene sig samtidig, thi det levende Væsens Ånd var i Hjulene.
\par 22 Oven over Væsenernes Hoveder var der noget ligesom en Himmelhvælving, funklende som Krystal, udspændt oven over deres Hoveder;
\par 23 og under Hvælvingen var deres Vinger udspændt, den ene mod den anden; hvert af dem havde desuden to, som skjulte deres Legemer.
\par 24 Og når de gik, lød Vingesuset for mig som mange Vandes Brus, som den Almægtiges Røst; det buldrede som en Hær.
\par 25 Det drønede oven over Hvælvingen over deres Hoveder; men når de stod, sænkede de Vingerne.
\par 26 Men oven over Hvælvingen over deres Hoveder var der noget som Safir at se til, noget ligesom en Trone, og på den, ovenover, var der noget ligesom et Menneske at se til.
\par 27 Og jeg skuede noget som funklende Malm fra det, der så ud som hans Hofter, og opefter; og fra det, der så ud som hans Hofter, og nedefter skuede jeg noget som Ild at se til; og Stråleglans omgav ham.
\par 28 Som Regnbuen, der viser sig i Skyen på en Regnvejrsdag, var Stråleglansen om ham at se til. Således så HERRENs Herlighedsåbenbarelse ud. Da jeg skuede det, faldt jeg på mit Ansigt. Og jeg hørte en Røst, som talede.

\chapter{2}

\par 1 Han sagde til mig: "Menneskesøn stå op på dine fødder så jeg kan tale med dig!"
\par 2 Og som han talede til mig, kom Ånden i mig og rejste mig på mine Fødder, og jeg hørte ham tale til mig.
\par 3 Han sagde: "Menneskesøn! Jeg sender dig til Israeliterne, de genstridige, der har sat sig op imod mig; de og deres Fædre har forbrudt sig imod mig til den Dag i Dag.
\par 4 Og Sønnerne har stive Ansigter og hårde Hjerter; jeg sender dig til dem, og du skal sige: Så siger den Herre HERREN!
\par 5 Hvad enten de hører eller ej - thi de er en genstridig Slægt - skal de kende, at en Profet er kommet iblandt dem.
\par 6 Men du, Menneskesøn frygt ikke for dem og vær ikke ræd for deres Ord, når du færdes mellem Nælder og Tidsler og bor blandt Skorpioner; frygt ikke for deres Ord og vær ikke ræd for deres Ansigter, thi de er en genstridig Slægt.
\par 7 Du skal tale mine Ord til dem, hvad enten de hører eller ej, thi de er en genstridig Slægt..
\par 8 Og du, Menneskesøn, hør, hvad jeg taler til dig! Vær ikke genstridig som den genstridige Slægt, men luk din Mund op og slug, hvad jeg her giver dig! "
\par 9 Og jeg skuede, og se, en Hånd var udrakt imod mig, og i den lå en Bogrulle;
\par 10 og han rullede den op for mig, - og der var skrevet på den både for og bag; og hvad der stod skrevet, var Klage, Suk og Ve.

\chapter{3}

\par 1 Så sagde han til mig: "Menneskesøn, slug hvad du her har for dig, slug denne Bogrulle og gå så hen og tal til Israels Hus!"
\par 2 Så åbnede jeg Munden, og han lod mig sluge Bogrullen
\par 3 og sagde til mig: "Menneskesøn! Lad din Bug fortære den Bogrulle, jeg her giver dig, og fyld dine Indvolde dermed!" Og jeg slugte den, og den var sød som Honning i min Mund.
\par 4 Så sagde han til mig: "Menneskesøn, gå til Israels Hus og tal mine Ord til dem!
\par 5 Thi du sendes til Israels Folk, ikke til et Folk med dybt Mål og tungt Mæle,
\par 6 ikke til mange Hånde Folkeslag med dybt Mål og tungt Mæle, hvis Tale du ikke fatter, hvis jeg sendte dig til dem, vilde de høre dig.
\par 7 Men Israels Hus vil ikke høre dig, thi de vil ikke høre mig; thi hele Israels Hus har hårde Pander og stive Hjerter.
\par 8 Se, jeg gør dit Ansigt hårdt som deres Ansigter og din Pande hård som deres Pander;
\par 9 som Diamant, hårdere end Flint gør jeg din Pande. Frygt ikke for dem og vær ikke ræd for deres Ansigter, thi de er en genstridig Slægt!"
\par 10 Videre sagde han til mig: "Menneskesøn, alle mine Ord, som jeg taler til dig, skal du optage i dit Hjerte og høre med dine Ører;
\par 11 og gå så hen til dine landflygtige Landsmænd og tal til dem og sig: Så siger den Herre HERREN! - hvad enten de så hører eller ej!"
\par 12 Så løftede Ånden mig, og jeg hørte bag mig Larmen af et vældigt Jordskælv, da HERRENs Herlighed hævede sig fra sit Sted,
\par 13 og Suset af de levende Væseners Vinger, der rørte hverandre, og samtidig Lyden af Hjulene og Larmen af Jordskælvet.
\par 14 Og Ånden løftede mig og førte mig bort, og jeg vandrede bitter og gram i Hu, idet HERRENs Hånd var over mig med Vælde.
\par 15 Så kom jeg til de landflygtige i Tel-Abib, de, som boede ved Floden Kebar, og der sad jeg syv Dage iblandt dem og stirrede hen for mig.
\par 16 Syv Dage senere kom HERRENs Ord til mig således:
\par 17 Menneskesøn! Jeg sætter dig til Vægter for Israels Hus; hører du et Ord af min Mund, skal du advare dem fra mig.
\par 18 Når jeg siger til den gudløse: "Du skal visselig dø!" og du ikke advarer ham eller for at bevare hans Liv taler til ham om at omvende sig fra sin gudløse Vej, så skal samme gudløse dø for sin Misgerning, men hans Blod vil jeg kræve af din Hånd.
\par 19 Advarer du derimod den gudløse, og han ikke omvender sig fra sin Gudløshed og sin Vej, så skal samme gudløse dø for sin Misgerning, men du har reddet din Sjæl.
\par 20 Og når en retfærdig vender sig fra sin Retfærdighed og gør Uret, og jeg lægger Anstød for ham, så han dør, og du ikke har advaret ham, så dør han for sin Synd, og den Retfærdighed, han har øvet, skal ikke tilregnes ham, men hans Blod vil jeg kræve af din Hånd.
\par 21 Har du derimod advaret den retfærdige mod at synde, og han ikke synder, så skal samme retfærdige leve, fordi han lod sig advare, og du har reddet din Sjæl.
\par 22 Siden kom HERRENs Hånd over mig der, og han sagde til mig: "Stå op og gå ud i dalen, der vil jeg tale med dig!"
\par 23 Så stod jeg op og gik ud i Dalen, og se, der stod HERRENs Herlighed, som jeg havde set denved Floden Kebar. Da faldt jeg på mit Ansigt.
\par 24 Men Ånden kom i mig og rejste mig på mine Fødder. Så taIede han til mig og sagde: Gå hjem og luk dig inde i dit Hus!
\par 25 Og du, Menneskesøn, se, man skal lægge Bånd på dig og binde dig, så du ikke kan gå ud iblandt dem;
\par 26 og din Tunge lader jeg hænge ved Ganen, så du bliver stum og ikke kan være dem en Revser; thi de er en genstridig Slægt.
\par 27 Men når jeg taler til dig, vil jeg åbne din Mund, og du skal sige til dem: Så siger den Herre HERREN! Så får den, der vil høre, høre, og den, der ikke vil, får lade være; thi de er en genstridig Slægt.

\chapter{4}

\par 1 Du Mennneskesøn tag dig en teglsten, læg den for dig og indrids i den et Billede af en By, Jerusalem;
\par 2 og kast en Vold op omkring den, byg Belejringstårne, opkast Stormvold, lad Hære lejre sig imod den og rejs Stormbukke mod den fra alle Sider;
\par 3 tag dig så en Jernpande og sæt den som en Jernvæg op mellem dig og Byen og ret dit Ansigt imod den. Således skal den være omringet, og du skal trænge den. Det skal være Israels Hus et Tegn.
\par 4 Og læg du dig på din venstre Side og tag, Israels Huss Misgerning på dig; alle de Dage du ligger således, skal du bære deres Misgerning.
\par 5 Deres Misgernings År gør jeg til lige så mange Dage for dig, 190 Dage; så længe skal du bære Israels Huss Misgerning.
\par 6 Og når de er til Ende, læg dig så på din højre Side og bær Judas Huss Misgerning 40 Dage; for hvert År giver jeg dig en Dag.
\par 7 Og du skal rette dit Ansigt og din blottede Arm mod det omringede Jerusalem og profetere imod det.
\par 8 Og se, jeg lægger Bånd på dig, så du ikke kan vende dig fra den ene Side til den anden, før din Belejrings Dage er til Ende.
\par 9 Og tag du dig Hvede, Byg, Bønner, Linser, Hirse og Spelt, kom det i et og samme Kar og lav dig Brød deraf; alle de Dage du ligger på Siden, 190 Dage, skal det være din Mad;
\par 10 og Maden, du får, skal være efter Vægt, tyve Sekel daglig; du skal spise den een Gang daglig.
\par 11 Og Vand skal du drikke efter Mål, en Sjettedel Hin; du skal drikke een Gang daglig.
\par 12 Og som Bygkager skal du spise det og bage det ved Menneskeskarn i deres Påsyn.
\par 13 Og du skal sige: "Så sige HERREN: Således skal Israeliterne have urent Brød til Føde blandt de Folk, jeg bortstøder dem til!"
\par 14 Men jeg sagde: "Ak, Herre, HERRE, jeg har endnu aldrig været uren; noget selvdødt eller sønderrevet har jeg fra Barnsben aldrig spist, og urent Kød kom aldrig i min Mund!"
\par 15 Da svarede han: "Vel, jeg tillader dig at tage Oksegødning i Stedet for Menneskeskarn og bage dit Brød derved."
\par 16 Videre sagde han til mig: Menneskesøn! Se, jeg bryder Brødets Støttestav i Jerusalem; Brød skal de spise efter Vægt og i Angst, og Vand skal de drikke efter Mål og i Rædsel,
\par 17 for at de må mangle Brød og Vand og alle som een være slagne af Rædsel og hensmægte i deres Misgerning.

\chapter{5}

\par 1 Og du Menneskesøn tag dig et skarbt sværd, brug det som Ragekniv og lad det gå over dit Hoved og dit Skæg; tag dig så en Vægtskål og del Håret.
\par 2 En Tredjedel skal du brænde i et Bål midt i Byen, når Belejringens Dage er omme; en Tredjedel skal du tage og slå den med Sværdet rundt om Byen; og en Tredjedel skal du sprede for Vinden; så drager jeg Sværdet bag dem.
\par 3 Derefter skal du tage lidt deraf og svøbe det ind i din kappeflig;
\par 4 og deraf skal du atter tage noget og kaste det midt ind i Ilden og brænde det. Og du skal sige til hele Israels Hus:
\par 5 Så siger den Herre HERREN: Dette er Jerusalem; jeg satte det midt iblandt Folkene, omgivet af lande.
\par 6 Men det var gudløst og genstridigt mod mine Lovbud mere end Folkene og mod mine Vedtægter mere end Landene rundt om; thi de lod hånt om mine Lovbud og vandrede ikke efter mine Vedtægter.
\par 7 Derfor, så siger den Herre HERREN: Fordi I var mere genstridige end Folkene rundt om og ikke vandrede efter mine Vedtægter eller holdt mine Lovbud, men gjorde efter de omboende Folks Lovbud,
\par 8 derfor, så siger den Herre HERREN: Se, jeg kommer over dig og holder Dom i din Midte for Folkenes Øjne.
\par 9 For alle dine Vederstyggeligheders Skyld vil jeg gøre med dig, hvad jeg aldrig har gjort og aldrig vil gøre Mage til.
\par 10 Derfor skal Fædre æde deres Børn i din Midte og Børn deres Fædre; jeg holder Dom over dig, og alle, som er til Rest i dig, spreder jeg for alle Vinde.
\par 11 Derfor, så sandt jeg lever, lyder det fra den Herre HERREN: Sandelig, fordi du gjorde min Helligdom uren med alle dine væmmelige Guder og Vederstyggeligheder, vil jeg også støde dig fra mig uden Medynk eller Skånsel.
\par 12 En Tredjedel af dig skal dø af Pest og omkomme af Hunger i din Midte, en Tredjedel skal falde for Sværdet rundt om dig, og en Tredjedel spreder jeg for alle Vinde og drager Sværdet bag dem.
\par 13 Min Vrede skal udtømme sig, og jeg vil køle min Harme på dem og tage Hævn, så de skal kende at jeg, HERREN, har talet i min Nidkærhed, når jeg udtømmer min Vrede over dem.
\par 14 Jeg gør dig til en Grushob og til Spot blandt Folkene rundt om, for Øjnene af enhver, som drager forbi.
\par 15 Du skal blive til Spot og Hån, til Advarsel og Rædsel for Folkene rundt om, når jeg holder Dom over dig i Vrede og Harme og harm fuld Revselse. Jeg, HERREN, har talet!
\par 16 Når jeg sender Hungerens onde Pile mod eder, og de, som jeg sender for af ødelægge eder, volder Ødelæggelse, og jeg lader Hungeren tage til, da sønderbryder jeg Brødets Støttestav for eder
\par 17 og sender Hunger over eder og Rovdyr, som skal affolke dig; Pest og Blodsudgydelse skal hjemsøge dig, og jeg bringer Sværd over dig. Jeg, HERREN, har talet.

\chapter{6}

\par 1 Og Herrens ord kom til mig således:
\par 2 Menneskesøn, vend dit Ansigt mod Israels Bjerge, profeter imod dem
\par 3 og sig: Israels Bjerge, hør den Herre HERRENs Ord! Så siger den Herre HERREN til Bjergene og Højene, til Kløfterne og Dalene: Se, jeg sender Sværd over eder og tilintetgør eders Offerhøje.
\par 4 Eders Altre skal ødelægges, eders Solstøtter sønderbrydes, og eders dræbte lader jeg segne foran eders Afgudsbilleder;
\par 5 jeg kaster Israeliternes Lig hen for deres Afgudsbilleder og strør eders Ben rundt om eders Altre.
\par 6 Overalt hvor I bor, skal Byerne lægges øde og Offerhøjene gå til Grunde, for at eders Altre kan lægges øde og gå til Grunde, eders Afudsbilleder sønderbrydes og udryddes, eders Solstøtter hugges om og eders Værker tilintetgøres.
\par 7 Mandefald skal ske iblandt eder, og I skal kende, at jeg er HERREN.
\par 8 Men en Rest lader jeg blive tilbage, idet nogle af eder undslipper fra Sværdet blandt Folkene, når I spredes i Landene,
\par 9 og de undslupne skal komme mig i Hu blandt Folkene, hvor de er Fanger; jeg sønderbryder deres bolerske Hjerter, som faldt fra mig, og deres bolerske Øjne, som hang ved deres Afgudsbilleder; og de skal væmmes ved sig selv over alt det onde, de har gjort, over alle deres Vederstyggeligheder.
\par 10 Og de skal kende, at jeg er HERREN; det var ikke tomme Ord, når jeg talede om at gøre den Ulykke på dem.
\par 11 Så siger den Herre HERREN: Slå Hænderne sammen, stamp med Foden og råb Ve over alle Israels Huses grimme Vederstyggeligheder! De skal falde for Sværd, Hunger og Pest.
\par 12 Den, som er langt borte, skal dø af Pest; den, som er nær, skal falde for Sværd; og den, som levnes og reddes, skal dø af Hunger; således udtømmer jeg min Vrede over dem.
\par 13 De skal kende, at jeg er HERREN, når deres dræbte ligger midt iblandt deres Afgudsbilleder rundt om deres Altre på hver høj Bakke, på alle Bjergenes Tinder, under hvert grønt Træ og hver løvrig Eg, der, hvor de opsendte liflig Duft til deres Afgudsbilleder.
\par 14 Jeg udrækker Hånden imod dem og gør Landet øde og tomt lige fra Ørkenen til Ribla, overalt hvor de bor; og de skal kende, at jeg er HERREN.

\chapter{7}

\par 1 Og HERRENs Ord kom til mig således
\par 2 Du, Menneskesøn, sig: Så siger den Herre HERREN til Israels Land: Enden kommer, Enden kommer over Landet vidt og bredt!
\par 3 Nu kommer Enden over dig, og jeg sender min Vrede imod dig og dømmer dig efter dine Veje og gengælder dig alle dine Vederstyggeigheder.
\par 4 Jeg viser dig ingen Medynk eller Skånsel, men gengælder dig dine Veje, og dine Vederstyggeligheder skal blive i din Midte; og du skal kende, at jeg er HERREN.
\par 5 Så siger den Herre HERREN: Ulykke følger på Ulykke; se, det kommer!
\par 6 Enden kommer, Enden kommer; den er vågnet og tager Sigte på dig; se, det kommer!
\par 7 Turen kommer til dig, som bor i Landet; Tiden er inde, Dagen er nær, en Dag med Rædsel og ikke med Frydeskrig på Bjergene.
\par 8 Nu udøser jeg snart min Harme over dig og udtømmer min Vrede på dig, dømmer dig efter dine Veje og gengælder dig alle dine Vederstyggeligheder.
\par 9 Jeg viser dig ingen Medynk eller Skånsel, men gengælder dig dine Veje, og dine Vederstyggeligheder skal blive i din Midte; og I skal kende, at jeg, HERREN, er den, som slår.
\par 10 Se, Dagen! Se, det kommer; Turen kommer til dig"! Riset blomstrer, Overmodet grønnes.
\par 11 Vold rejser sig til et Ris over Gudløshed; der bliver intet tilbage af dem, intet af deres larmende Hob, intet af deres Gods, og der er ingen Herlighed iblandt dem.
\par 12 Tiden er inde, Dagen er nær; Køberen skal ikke glæde sig og Sælgeren ikke sørge, thi Vrede kommer over al den larmende Hob derinde.
\par 13 Thi Sælgeren skal ikke vende tilbage til det solgte, om han end bliver i Live; thi Synet om al den larmende Hob derinde tages ikke tilbage, og ingen skal styrke sit Liv ved sin Misgerning.
\par 14 Man støder i Hornet og gør alt rede, men ingen drager i krig; thi min Vrede kommer over al den larmende Hob derinde.
\par 15 Sværd ude og Pest og Hunger inde! De, der er i Marken, omkommer for Sværd, og dem, der er i Byen, fortærer Hunger og Pest.
\par 16 Og selv om nogle af dem undslipper og når op i Bjergene som Kløfternes Duer, skal de alle dø, hver for sin Misgerning.
\par 17 Alle Hænder er slappe, alle Knæ flyder som Vand.
\par 18 De klæder sig i Sæk, og Rædsel omhyller dem; alle Ansigter er skamfulde, alle Hoveder skaldede.
\par 19 Deres Sølv kaster de ud på Gaden, deres Guld regnes for Snavs; deres Sølv og Guld kan ikke redde dem på HERRENs Vredes Dag; de kan ikke stille deres Hunger eller fylde deres Bug dermed, thi det var dem Årsag til Skyld.
\par 20 I dets strålende Pragt satte de deres Stolthed, og deres vederstyggelige Billeder, deres væmmelige Guder, lavede de deraf: derfor gør jeg det til Snavs for dem.
\par 21 Jeg giver det som Bytte i de fremmedes Hånd og som Rov til de mest gudløse på Jorden, og de skal vanhellige det.
\par 22 Jeg vender mit Åsyn fra dem og man skal vanhellige mit Kleodie, Ransmænd skal trænge ind og vanhellige det.
\par 23 Gør Lænkerne rede! Thi Landet er fuldt af Blodskyld og Byen af Vold.
\par 24 Jeg henter de værste af Folkene, og de skal tage Husene i Eje; jeg gør Ende på de mægtiges Stolthed, og deres Helligdomme skal vanhelliges.
\par 25 Der opstår Angst; man søger Redning, men finder den ikke.
\par 26 Uheld følger på Uheld, Rygte på Rygte; man skal tigge Profeten om et Syn, Præsten kommer til kort med Vejledning og de Ældste med Råd.
\par 27 Kongen sørger, Fyrsten hyller sig i Rædsel, og Landboernes Hænder lammes af Forfærdelse. Jeg gør med dem efter deres Færd og dømmer dem, som de fortjener; og de skal kende, at jeg er HERREN.

\chapter{8}

\par 1 I det sjette År på den femte dag i den sjette måned da jeg sad i mit Hus og Judas Ældste sad hos mig, faldt den Herre HERRENs Hånd på mig.
\par 2 Og jeg skuede, og se, der var noget ligesom en Mand; fra hans Hofter og nedefter var der Ild, og fra Hofterne og opefter så det ud som strålende Lys, som funklende Malm.
\par 3 Han rakte noget som en Hånd ud og greb mig ved en Lok af mit Hovedhår, og Ånden løftede mig op mellem Himmel og Jord og førte mig i Guds Syner til Jerusalem, til Indgangen til den indre Forgårds Nordport, hvor Nidkærhedsbilledet, som vakte Nidkærhed, stod.
\par 4 Og se, der var Israels Guds Herlighed; at se til var den, som jeg så den i Dalen.
\par 5 Og han sagde til mig: "Menneskesøn, løft dit Blik mod Nord!" Jeg løftede mit Blik mod Nord, og se, norden for Alterporten stod Nidkærhedsbilledet, ved Indgangen.
\par 6 Og han sagde til mig: "Menneskesøn, ser du, hvad de gør? Store er de Vederstyggeligheder, Israels Hus øver her, så jeg må vige langt bort fra min Helligdom. Men du skal få endnu større Vederstyggeligheder at se!"
\par 7 Så førte han mig hen til Indgangen til Forgården.
\par 8 Og han sagde til mig: "Menneskesøn, bryd igennem Væggen!" Og da jeg brød igennem Væggen, så jeg en Indgang.
\par 9 Og han sagde til mig: "Gå ind og se, hvilke grimme Vederstyggeligheder de øver der!"
\par 10 Og da jeg kom derind og skuede, se, da var alskens væmmelige Billeder af Kryb og kvæg og alle Israels Huses Afgudsbilleder indridset rundt om på Væggen.
\par 11 Og halvfjerdsindstyve af Israels Huses Ældste med Jaazanja, Sjafans Søn, i deres Midte stod foran dem, hver med sit Røgelsekar i Hånden, medens Røgelseskyens Duft steg op.
\par 12 Da sagde han til mig: "Ser du, Menneskesøn, hvad Israels Huses Ældste øver i Mørke hver i sine Billedkamre? Thi de siger: HERREN ser intet, HERREN har forladt Landet!"
\par 13 Og han sagde til mig: "Du skal få endnu større Vederstyggeligheder at se, som de øver!"
\par 14 Så førte han mig hen til Indgangen til HERRENs Huses Nordport, og se, der sad Kvinder og græd over Tammuz.
\par 15 Og han sagde til mig: "Ser du det, Menneskesøn? Men du skal få endnu større Vederstyggeligheder at se!"
\par 16 Så førte han mig hen til HERRENs Huss indre Forgård, og se, ved Indgangen til HERRENs Helligdom mellem Forhallen og Alteret var der omtrent fem og tyve Mænd; med Ryggen mod HERRENs Helligdom og Ansigtet mod Øst tilbad de Solen.
\par 17 Og han sagde til mig: "Ser du det, Menneskesøn? Har Judas Hus ikke nok i at øve de Vederstyggeligheder her, siden de fylder Landet med Vold og krænker mig endnu mere? Se, hvor de sender Stank op i Næsen på mig"!
\par 18 Men derfor vil også jeg handle med dem i Vrede; jeg viser dem ingen Medynk eller Skånsel, og selv om de højlydt råber mig ind i øret vil jeg ikke høre dem.

\chapter{9}

\par 1 Så hørte jeg ham råbe med vældig røst: "Byens hjemsøgelse nærmer sig, og hver har sit Mordvåben i Hånden!"
\par 2 Og se, seks Mænd kom fra den øvre nordport, hver med sin Stridshammer i Hånden, og een iblandt dem bar linned Klædebon og havde et Skrivetøj ved sin Lænd; og de kom og stillede sig ved Siden af Kobberalteret.
\par 3 Men Israels Guds Herlighed havde hævet sig fra Keruberne, som den hvilede på, og flyttet sig hen til Templets Tærskel; og han råbte til Manden i det linnede Klædebon og med Skrivetøjet ved Lænden,
\par 4 og HERREN sagde til ham: "Gå midt igennem Byen, igennem Jerusalem, og sæt et Mærke på de Mænds Pander, der sukker og jamrer over alle de Vederstyggeligbeder, som øves i dets Midte!"
\par 5 Og til de andre hørte jeg ham sige: "Gå efter ham ud gennem Byen og hug ned! Vis ingen Medynk eller Skånsel!
\par 6 Oldinge og Ynglinge, jomfruer, Børn og Kvinder skal I hugge ned og udrydde; men ingen af dem, der bærer Mærket, må I røre! Begynd ved min Helligdom!" Så begyndte de med de Ældste, som stod foran Templet,
\par 7 Og han sagde til dem: "Gør Templet urent, fyld Forgårdene med dræbte og drag så ud!" Og de drog ud og huggede ned i Byen.
\par 8 Men medens de huggede ned og jeg var ene tilbage, faldt jeg på mit Ansigt og råbte: "Ak, Herre, HERRE vil du da tilintetgøre alt, hvad der er levnet af Israel, ved at udøse din Vrede over Jerusalem?"
\par 9 Han svarede: "lsraels og Judas Huses Brøde er såre, såre stor, thi Landet er fuldt af Blodskyld og Byen af Retsbrud; thi de siger, at HERREN har forladt Byen, og at HERREN intet ser.
\par 10 Derfor viser jeg heller ingen Medynk eller Skånsel, men gengælder dem deres Færd."
\par 11 Og se, Manden i det linnede klædebon og med Skrivetøjet ved Lænden kom tilbage og meldte: "Jeg har gjort, som du bød."

\chapter{10}

\par 1 Og jeg skuede og se, over hvælvingen over kerubernes hoveder var der noget som Safir; noget ligesom en Trone viste sig over dem.
\par 2 Så sagde han til Manden i det linnede Klædebon: "Gå ind mellem Hjulene under keruberne og tag Hænderne fulde af glødende Kul fra Rummet mellem Keruberne og strø det ud over Byen!" Og jeg så ham gå derhen.
\par 3 Keruberne stod sønden for Templet, da Manden gik derhen, og Skyen fyldte den indre Forgård.
\par 4 Og HERRENs Herlighed hævede sig fra Keruberne og flyttede sig hen til Templets Tærskel; da fyldtes Templet af Skyen, og Forgården fyldtes af HERRENs Herligheds Glalans.
\par 5 Og Suset af Kerubernes Vinger hørtes helt ud i den ydre Forgård som Gud den Almægtiges Røst, når han taler.
\par 6 Så bød han Manden i det linnede Klædebon: "Tag Ild fra Rummet mellem Hjulene, inde mellem keruberne!" Og Manden stillede sig hen ved Siden af det ene Hjul
\par 7 og rakte Hånden ind i Ilden, som brændte mellem Keruberne, og kom ud med noget deraf.
\par 8 Under Kerubernes Vinger sås noget, der lignede en Menneskehånd;
\par 9 og jeg skuede, og se, der var fire Hjul ved Siden af Keruberne, eet ved hver Kerub, og Hjulene var som funklende Krysolit at se til.
\par 10 De så alle fire ens ud, og det var, som om der i hvert Hjul var et andet Hjul,
\par 11 De kunde gå til alle fire Sider de vendte sig ikke, når de gik. Thi de gik i den Retning, den forreste vendte, og de vendte sig ikke, når de gik.
\par 12 Hele deres Legeme, Ryg, Hænder og Vinger og ligeledes Hjulene var fulde af Øjne rundt om; således var det med alle fire Hjul.
\par 13 Og jeg hørte, at Hjulene kaldtes Galgal.
\par 14 Hver af dem havde fire Ansigter; det ene var et Kerubansigt, det andet et Menneskeansigt, det tredje et Løveansigt og det fjerde et Ørneansigt.
\par 15 Og Keruberne hævede sig i Vejret. Det var det samme levende Væsen, jeg så ved Floden Kebar.
\par 16 Når Keruberne gik, gik også Hjulene ved Siden af, og når Keruberne løftede Vingerne for at hæve sig fra Jorden, vendte Hjulene sig ikke fra dem;
\par 17 når de standsede, standsede også de; og når de hævede sig, hævede de sig med, thi Væsenets Ånd var i dem.
\par 18 Så forlod HERRENs Herlighed Templets Tærskel og stillede sig over Keruberne.
\par 19 Og jeg så, hvorledes Keruberne løftede Vingerne og hævede sig fra Jorden, da de gik, og Hjulene med dem; og de standsede ved Indgangen til HERRENs Huses Østport, og Israels Guds Herlighed var oven over dem.
\par 20 Det var det samme levende Væsen, jeg så under Israels Gud ved Floden Kebar; og jeg skønnede, at det var Keruber.
\par 21 Hver af dem havde fire Ansigter og fire Vinger og noget ligesom Menneskehænder under Vingerne.
\par 22 Og deres Ansigter var ligesom de Ansigter, jeg så ved Floden Kebar. De gik alle lige ud.

\chapter{11}

\par 1 Så løftede Ånden mig og bragte mig til Herrens huses østport, den der vender mod Øst. Og se, ved Indgangen til Porten var der fem og tyve Mænd, og jeg så iblandt dem Jaazanja, Azzurs Søn, og Pelatja, Benajas Søn, Folkets Fyrster.
\par 2 Og han sagde til mig: "Menneskesøn! Det er de Mænd, som pønser på Uret og lægger onde Råd op i denne By,
\par 3 idet de siger: "Er Husene ikke nys bygget? Byen er Gryden, vi Kødet!"
\par 4 Profeter derfor imod dem, profeter, Menneskesøn!
\par 5 Så faldt HERRENs Ånd på mig, og han sagde til mig: Sig: Så siger HERREN: Således taler I, Israels Hus; jeg kender godt, hvad der stiger op i eders Ånd.
\par 6 Mange har I dræbt i denne By; I har fyldt dens Gader med dræbte.
\par 7 Derfor, så siger den Herre HERREN: De, som I dræbte og henslængte i dens Midte, de er Kødet, og Byen er Gryden; men eder vil jeg føre ud af den.
\par 8 I frygter for Sværd, og Sværd vil jeg bringe over eder, lyder det fra den Herre HERREN.
\par 9 Jeg vil føre eder ud af den og give eder i fremmedes Hånd, og jeg vil holde Dom over eder.
\par 10 For Sværd skal I falde; ved Israels Grænse vil jeg dømme eder. Og I skal kende, at jeg er HERREN.
\par 11 Byen skal ikke være eder en Gryde, og I skal ikke være Kødet deri; ved Israels Grænse vil jeg dømme eder.
\par 12 Og I skal kende, at jeg er HERREN, hvis Vedtægter I ikke fulgte, og hvis Lovbud I ikke levede efter, hvorimod I levede efter eders Nabofolks Lovbud. -
\par 13 Men medens jeg profeterede således, døde Pelatja, Benajas Søn. Da faldt jeg på mit Ansigt og råbte med høj Røst: "Ak, Herre, HERRE, vil du da helt udrydde Israels Rest?"
\par 14 Så kom HERRENs Ord til mig således:
\par 15 Menneskesøn! Dine Brødre, dine Medfanger og alt Israels Hus, alle de, om hvem Jerusalems Indbyggere siger: "De er langt borte fra HERREN, os er Landet givet i Eje!" -
\par 16 derfor skal du sige: Så siger den Herre HERREN: Ja, jeg har ført dem langt bort blandt Folkene og spredt dem i Landene, og kun i ringe Måde var jeg dem en Helligdom i de Lande, hvor de kom hen.
\par 17 Men derfor skal du sige: Så siger den Herre HERREN: Jeg vil samle eder sammen fra Folkeslagene og sanke eder op i Landene, hvor I er spredt, og give eder Israels Jord.
\par 18 Derhen skal de komme og fjerne alle dets væmmelige Guder og alle dets Vederstyggeligheder;
\par 19 jeg giver dem et nyt Hjerte og indgiver dem en ny Ånd; jeg tager Stenhjertet ud af deres Legeme og giver dem et Kødhjerte,
\par 20 for at de må følge mine Vedtægter og holde mine Lovbud og gøre efter dem. Så skal de være mit Folk, og jeg vil være deres Gud.
\par 21 Men hines Hjerter holder sig til deres væmmelige Guder og deres Vederstyggeligheder; dem gengælder jeg deres Færd, lyder det fra den Herre HERREN.
\par 22 Så løftede keruberne Vingerne og samtidig Hjulene; og Israels Guds Herlighed var oven over dem.
\par 23 Og HERRENs Herlighed steg op fra Byen og stillede sig på Bjerget østen for.
\par 24 Derpå løftede Ånden mig og bragte mig ved Guds Ånd i Synet til de landflygtige i Kaldæa; og Synet, som jeg havde skuet, steg op og svandt bort.
\par 25 Så kundgjorde jeg de landflyggige alle de Ord, HERREN havde åbenbaret mig.

\chapter{12}

\par 1 HERRENs Ord kom til mig således:
\par 2 Menneskesøn! Du bor midt i den genstridige Slægt, som har Øjne at se med, men ikke ser, og Ører at høre med, men ikke hører, thi de er en genstridig Slægt.
\par 3 Men du, Menneskesøn, udrust dig ved højlys Dag i deres Påsyn som en, der drager i Landflygtighed, og drag så i deres Påsyn fra Stedet, hvor du bor, til et andet Sted! Måske de så får Øjnene op; thi de er en genstridig Slægt.
\par 4 Bær ved højlys dag i deres Påsyn dine Sager udenfor, som om du skal i Landflygtighed, men selv skal du drage bort om Aftenen i deres Påsyn som en, der drager i Landflygtighed.
\par 5 Slå i deres Påsyn Hul i Væggen og drag ud derigennem;
\par 6 tag Sagerne på Skulderen og drag ud i Bælgmørke med tilhyllet Ansigt uden at se Landet; thi jeg gør dig til et Tegn for Israels Hus!
\par 7 Og jeg gjorde, som der bødes mig: Jeg bar ved højlys Dag mine Sager udenfor, som om jeg skulde i Landflygtighed, og om Aftenen slog jeg med Hånden Hul i Væggen, og i Bælgmørke drog jeg ud; jeg tog det på Skulderen i deres Påsyn.
\par 8 Næste Morgen kom HERRENs Ord til mig således:
\par 9 Menneskesøn! Har Israels Hus, den genstridige Slægt, ikke spurgt dig: "Hvad gør du der?"
\par 10 Sig til dem: Så siger den Herre HERREN: Således skal det være med Fyrsten, denne Byrde i Jerusalem, og hele Israels Hus derinde.
\par 11 Sig: Jeg er eder et Tegn; som jeg har gjort, skal der gøres med dem: I Landflygtighed og Fangenskab skal de drage.
\par 12 Og Fyrsten i deres Midte skal tage sine Sager på Skulderen, og i Bælgmørke skal han drage ud, han skal slå Hul i Væggen for at drage ud derigennem, og han skal tilhylle sit Ansigt for ikke at se Landet.
\par 13 Men jeg breder mit Net over ham, og han skal fanges i mit Garn; og jeg bringer ham til Bael i kaldæernes Land, som han dog ikke skal se; og der skal han dø.
\par 14 Og alle hans Omgivelser, hans Hjælpere og alle hans Hærskarer vil jeg udstrø for alle Vinde og drage Sværdet bag dem.
\par 15 Da skal de kende, at jeg er HERREN, når jeg spreder dem: blandt Folkene og udstrør dem i Landene.
\par 16 Kun nogle få af dem levner jeg fra Sværd, Hunger og Pest, for at de kan fortælle om alle deres Vederstyggeligheder blandt de Folk, de kommer til; og de skal kende, at jeg er HERREN.
\par 17 HERRENs Ord kom til mig således:
\par 18 Menneskesøn, spis Brød i Angst og drik Vand i Frygt og Bæven;
\par 19 og sig til Landets Folk: Så siger den Herre HERREN om Jerusalems Indbyggere i Israels Land: Brød skal de spise med Bæven, og Vand skal de drikke med Rædsel, for at deres Land og alt deri må ligge øde til Straf for alle dets Indbyggeres Voldsfærd;,
\par 20 og Byerne, der nu er beboet, skal ligge øde, og Landet skal blive til Ørk; og I skal kende, at jeg er HERREN.
\par 21 HERRENs Ord kom til mig således:
\par 22 Menneskesøn! Hvad er det for et Mundheld, I har om Israels Land: "Det trækker i Langdrag, og alle Syner slår fejl!"
\par 23 Sig derfor til dem: Så siger den Herre HERREN: Jeg vil bringe dette Mundheld til at forstumme, så de ikke mere bruger det i Israel. Sig tværtimod til dem: "Tiden er nær, og alle Syner træffer ind!"
\par 24 Thi der skal ikke mere være noget Løgnesyn eller nogen falsk Spådom i Israels Hus,
\par 25 men jeg, HERREN taler, hvad jeg vil, og det skal ske. Det skal ikke længer trække i Langdrag; men i eders Dage, du genstridige Slægt, vil jeg tale et Ord og fuldbyrde det, lyder det fra den Herre HERREN.
\par 26 HERRENs Ord kom til mig således:
\par 27 Menneskesøn! Se, Israels Hus siger: "Synet, han skuer, gælder sene Dage, og han profeterer om fjerne Tider!"
\par 28 Sig derfor til dem: Så siger den Herre HERREN: Intet af mine Ord skal lade vente på sig mere; hvad jeg taler, skal ske, lyder det fra den Herre HERREN.

\chapter{13}

\par 1 HERRENs Ord kom til mig således:
\par 2 Menneskesøn, profeter mod Israels Profeter, profeter og sig til dem, som profeterer efter deres eget Hjertes Tilskyndelse: Hør HERRENs Ord!
\par 3 Så siger den Herre HERREN: Ve Profeterne, de Dårer, som følger deres egen Ånd uden at have skuet noget!
\par 4 Dine Profeter, Israel, er som Sjakaler i Ruiner.
\par 5 De stillede sig ikke i Murbruddet og byggede ikke en Mur om Israels Hus, så det kunde stå sig i Striden på HERRENs Dag.
\par 6 De skuede Tomhed og spåede Løgn, idet de sagde: "Så lyder det fra HERREN! " uden at HERREN havde sendt dem, og dog ventede de Ordet stadfæstet,
\par 7 Var det ikke tomme Syner, I skuede, og Løgnespådomme, I fremsatte? Og I siger: "Så lyder det fra HERREN!" uden at jeg har talet.
\par 8 Derfor, så siger den Herre HERREN: Fordi I forkynder Tomhed og skuer Løgn, se, derfor kommer jeg over eder, lyder det fra den Herre HERREN.
\par 9 Jeg udrækker min Hånd mod Profeterne, der skuer Tomhed og spår Løgn; i mit Folks Samfund skal de ikke være, de skal ikke optages i Israels Huses Mandtalsbog og ikke komme til Israels Land; og I skal kende, at jeg er den Herre HERREN.
\par 10 Fordi, ja fordi de vildleder mit Folk ved af forkynde Fred, hvor ingen Fred er, og når det bygger en Væg, stryger den over med Kalk,
\par 11 så sig til dem, der stryger over med kalk: Skylregn skal komme, Isstykker skal falde og Stormvejr bryde løs,
\par 12 og når så Væggen styrter sammen, vil man da ikke spørge eder: "Hvor er Kalken, I strøg på?"
\par 13 Derfor, så siger den Herre HERREN: Og jeg lader et Stormvejr bryde løs i min Harme, og Skylregn skal komme i min Vrede og Isstykker i min Harme til Undergang;
\par 14 og Væggen, som I strøg over med Kalk, river jeg ned og lader den styrte til Jorden, så dens Grundvold blottes, og ved dens Fald skal I gå til Grunde derinde; og I skal kende, at jeg er HERREN.
\par 15 Jeg vil udtømme min Vrede over Væggen og dem, der strøg den over med Kalk, så man skal sige til eder: "Hvor er Væggen, og hvor er de, som strøg den over,
\par 16 Israels Profeter, som profeterede om Jerusalem og skuede Fredssyner for det, hvor ingen Fred var - lyder det fra den Herre HERREN.
\par 17 Og du, Menneskesøn, vend dit Ansigt mod dit Folks Døtre, som profeterer efter deres eget Hjertes Tilskyndelse; profeter imod dem
\par 18 og sig: Så siger den Herre HERREN: Ve dem, der syr Bind til alle Håndled og laver Slør til alle Hoveder efter hver Legemshøjde for at fange Sjæle! Dræber I Sjæle, der hører til mit Folk, og holder Sjæle i Live af Egennytte?
\par 19 I, som vanhelliger mig for mit Folk for nogle Håndfulde Bygkorn og nogle Bidder Brød og således dræber Sjæle, der ikke skulde dø, og holder Sjæle i Live, som ikke skulde leve, idet I lyver for mit Folk, som gerne hører på Løgn!
\par 20 Derfor, så siger den Herre HERREN: Se, jeg kommer over eders Bind, med hvilke I fanger Sjæle, og river dem af eders Arme, og de Sjæle, I fanger, lader jeg slippe fri som Fugle;
\par 21 og jeg sønderriver eders Slør og frier mit Folk af eders Hånd, så de ikke mere er Bytte i eders Hånd; og I skal kende, at jeg er HERREN.
\par 22 Fordi I ved Svig volder den retfærdiges Hjerte Smerte, skønt jeg ikke vilde volde ham Smerte, og styrker den gudløses Hænder, så han ikke omvender sig fra sin onde Vej, at jeg kan holde ham i Live,
\par 23 derfor skal I ikke mere skue Tomhed eller drive eders Spådomskunst; jeg frier mit Folk af eders Hånd; og I skal kende, at jeg er HERREN.

\chapter{14}

\par 1 Nogle af Israels ældste kom til mig og satte sig lige over for mig.
\par 2 Så kom HERRENs Ord til mig således;
\par 3 Menneskesøn! Disse Mænd har lukket deres Afgudsbilleder ind i deres Hjerte og stillet det, der blev dem Årsag til Skyld, for deres Ansigt - skulde jeg lade mig rådspørge af dem?
\par 4 Tal derfor med dem og sig: Så siger den Herre HERREN: Hver den af Israels Hus, som lukker sine Afgudsbilleder ind i sit Hjerte og sættet det, der blev dem Årsag til Skyld, for sit Ansigt og så kommer til Profeten, ham vil jeg, HERREN, selv svare trods hans mange Afgudsbilleder
\par 5 for at gribe Israels Hus i Hjertet, fordi de faldt fra mig med alle deres Afgudsbilleder.
\par 6 Sig derfor til Israels Hus: Så siger den Herre HERREN: Vend om, vend eder bort fra eders Afgudsbilleder og vend eders Ansigt bort fra alle eders Vederstyggeligheder!
\par 7 Thi hver den af Israels Hus og af de fremmede, der bor i Israel, som skiller sig fra mig og lukker sine Afgudsbilleder ind i sit Hjerte og sætter det, der blev ham Årsag til Skyld, for sit Ansigt og så kommer til Profeten, for at denne skal rådspørge mig for ham, ham vil jeg, HERREN, selv svare;
\par 8 jeg retter mit Åsyn mod den mand og gør ham til Tegn og Mundheld og udrydder ham af mit Folk; og I skal kende, at jeg er HERREN.
\par 9 Men lader Profeten sig lokke til at sige et Ord, så er det mig, HERREN, der har lokket ham, og jeg udrækker min Hånd imodham og udrydder ham af mit Folk Israel.
\par 10 De skal begge bære deres Skyld, Spørgeren og Profeten skal være lige skyldige,
\par 11 for at Israels Hus ikke mere skal fare vild fra mig og blive urent ved alle sine Overtrædelser; da skal de være mit Folk, og jeg vil være deres Gud, lyder det fra den Herre HERREN.
\par 12 HERRENs Ord kom til mig således:
\par 13 Menneskesøn! Når et Land troløst synder imod mig, og jeg udrækker min Hånd imod det og bryder Brødets Støttestav for det og sender Hungersnød over det og udrydder Folk og Fæ,
\par 14 og disse tre Mænd var i dets Mtidte: Noa, Daniel og Job, så skulde kun de tre redde deres Liv ved deres Retfærdighed, lyder det fra den Herre HERREN.
\par 15 Eller lader jeg Rovdyr fare igennem Landet og gøre det folketomt, så det bliver øde og ingen tør gå derigennem for de vilde Dyr,
\par 16 og disse tre Mænd var i dets Midte - så sandt jeg lever, lyder det fra den Herre HERREN: De skulde ikke redde deres Sønner eller Døtre; de selv alene skulde reddes, men Landet måtte blive øde.
\par 17 Eller lader jeg Sværdet komme over dette Land og siger: "Sværdet skal fare igennem Landet!" og udrydder Folk og Fæ deraf,
\par 18 og disse tre Mænd var i dets Midte - så sandt jeg lever, lyder det fra den Herre HERREN: De skulde ikke redde deres Sønner eller Døtre; de selv alene skulde reddes.
\par 19 Eller sender jeg Pest over dette Land og udgyder min Harme over det med Blod og udrydder Folk og Fæ deraf,
\par 20 og Noa, Daniel og Job var i dets Midte - så sandt jeg lever, lyder det fra den Herre HERREN: De skulde ikke redde Søn eller Datter; de selv alene skulde redde deres Liv ved deres Retfærdighed.
\par 21 Men så siger den Herre HERREN: Og dog, når jeg sender mine fire grumme Straffedomme, Sværd, Hunger, Rovdyr og Pest, over Jerusalem for at udrydde Folk og Fæ deraf,
\par 22 se, da skal der levnes en Flok undslupne, som fører Sønner og Døtre ud derfra; se, de skal drage hid til eder, og I skal se deres Færd og Gerninger; da skal I trøste eder over den Ulykke, jeg har bragt over Jerusalem, alt det, jeg har bragt over det.
\par 23 De skal være eder en Trøst, når I ser deres Færd og Gerninger, og I skal skønne, at jeg ikke uden Grund gjorde alt, hvad jeg lod det times, lyder det fra den Herre HERREN.

\chapter{15}

\par 1 HERRENs Ord kom til mig således:
\par 2 Menneskesøn! Hvad har Vinstokken forud for alle andre Træer, Ranken, som står iblandt Skovens Træer?
\par 3 Tager man Gavntræ deraf? Eller tager man deraf en Knag til at hænge alskens Redskaber på?
\par 4 Når den så oven i købet har været givet Ilden til Føde, så at Ilden har fortæret begge dens Ender, og Midten er svedet, duer den så til noget?
\par 5 Se, da den endnu var uskadt, brugtes den ikke til noget, endsige at den skulde kunne bruges til noget nu, da Ilden har fortæret den og den er svedet.
\par 6 Derfor, så siger den Herre HERREN: Som det går Vinstokken blandt Skovens Træer, hvilke jeg giver Ilden til Føde, således giver jeg Jerusalems Indbyggere hen;
\par 7 jeg vender mit Åsyn imod dem; af Ilden slap de ud, men Ild skal dog fortære dem; og I skal kende, at jeg er HERREN, når jeg vender mit Åsyn imod dem.
\par 8 Og jeg gør Landet øde, fordi de var troløse, lyder det fra den Herre HERREN.

\chapter{16}

\par 1 HERRENs Ord kom til mig således:
\par 2 Menneskesøn, forehold Jerusalem dets Vederstyggeligheder
\par 3 og sig: Så siger den Herre HERREN til Jerusalem: Dit Udspring og din Oprindelse var i Kanaanæernes Land; din Fader var Amorit, din Moder Hetiterinde.
\par 4 Og ved din Fødsel gik det således til: Da du fødtes, blev din Navlestreng ikke skåret over, ej heller blev du tvættet ren med Vand eller gnedet med Salt eller lagt i Svøb.
\par 5 Ingen så på dig med så megen Medynk, at han af Medlidenhed gjorde nogen af disse Ting for dig, men du henslængtes på Marken, den Dag du fødtes; således væmmedes man ved din Sjæl.
\par 6 Men jeg kom forbi, og da jeg så dig sprælle i Blod, sagde jeg til dig, som du lå der i Blodet: "Du skal leve
\par 7 og vokse som en Urt på Marken!" Og du voksede, blev stor og trådte ind i din Skønheds Fylde; dine Bryster blev faste, og dit Hår voksede; men du var nøgen og bar.
\par 8 Så kom jeg forbi og så dig, og se, din Tid var inde, din Elskovstid; og jeg bredte min Kappeflig over dig og tilhyllede din Blusel; så tilsvor jeg dig Troskab og indgik Pagt med dig, lyder det fra den Herre HERREN, og du blev min.
\par 9 Så tvættede jeg dig med Vand, skyllede Blodet af dig og salvede dig med Olie;
\par 10 jeg klædte dig i broget vævede Klæder, gav dig Sko af Tahasjskind på, bandt Byssusklæde om dit Hoved og hyllede dig i Silke;
\par 11 jeg smykkede dig, lagde Spange om dine Arme og Kæde om din Hals,
\par 12 fæstede en Ring i din Næse, kugler i dine Ører og en herlig krone på dit Hoved;
\par 13 du smykkedes med Guld og - Sølv, din Klædning var Byssus, Silke og broget vævede Klæder; fint Hvedemel, Honning og Olie var din Mad, og du blev såre dejlig og drev det til at blive Dronning.
\par 14 Dit Ry kom ud blandt Folkene for din Dejligheds Skyld; thi den var fuldendt ved de Smykker, jeg udstyrede dig med, lyder det fra den Herre HERREN.
\par 15 Men du stolede på din Dejlighed og bolede i Kraft af dit Ry; du udøste din bolerske Attrå over enhver, som kom forbi; du blev hans.
\par 16 Af dine Klæder tog du og gjorde dig spraglede Offerhøje og bolede på dem.
\par 17 Du tog dine Smykker af mit Guld og Sølv, som jeg havde givet dig, og gjorde dig Mandsbilleder og bolede med dem.
\par 18 Du tog dine broget vævede Klæder og hyllede dem deri, og min Olie og Røgelse satte du for dem.
\par 19 Brødet, som jeg havde givet dig - fint Hedemel, Olie og Honning gav jeg dig at spise - satte du for dem til en liflig Duft, lyder det fra den Herre HERREN.
\par 20 Og du tog dine Sønner og Døtre, som du havde født mig, og slagtede dem til Føde for dem. Var det ikke nok med din Bolen,
\par 21 siden du slagtede mine Sønner og gav dem hen, idet du indviede dem til dem?
\par 22 Og under alle dine Vederstyggeligheder og din Bolen kom du ikke din Ungdoms dage i Hu, da du var nøgen og bar og lå og sprællede i Blod.
\par 23 Og efter al denne din Ondskab - ve dig, ve! lyder det fra den Herre HERREN -
\par 24 byggede du dig en Alterfod og gjorde dig en Offerhøj på alle Torve.
\par 25 Ved hvert Gadehjørne byggede du dig en Offerhøj og vanærede din Dejlighed; du spredte Benene for enhver, som kom forbi, og drev din Bolen vidt.
\par 26 Du bolede med Ægypterne, dine sværlemmede Naboer, og drev din Bolen vidt og krænkede mig.
\par 27 Men se, jeg udrakte min Hånd imod dig og unddrog dig, hvad der tilkom dig, og jeg gav dig dine Fjender Filisterindernes Gridskhed i Vold, de, som skammede sig over din utugtige Færd.
\par 28 Siden bolede du med Assyrerne, umættelig som du var; du bolede med dem, men blev endda ikke mæt.
\par 29 Så udstrakte du din Bolen til Kræmmerlandet, Kaldæernes Land, men blev endda ikke mæt.
\par 30 Hvor vansmægtede dog dit Hjerte, lyder det fra den Herre HERREN, da du gjorde alt dette, som kun en arg Skøge kan gøre,
\par 31 da du byggede dig en Alterfod ved hvert Gadehjørne og gjorde dig en Offerhøj på hvert Torv. Men du lignede ikke Skøgen i at samle Skøgeløn;
\par 32 hvilken Horkvinde, der tager fremmede i sin Mands Sted! -
\par 33 ellers giver man Skøgen en Gave, men du gav alle dine Elskere Gaver og købte dem til at komme til dig rundt om fra og bole med dig.
\par 34 Hos dig var det modsat af, hvad Tilfældet ellers er med Kvinder; ingen løb efter dig for at bole, men du gav Skøgeløn og fik selv ingen; det var det modsatte.
\par 35 Derfor, du Skøge, hør HERRENs Ord!
\par 36 Så siger den Herre HERREN: Fordi din Skam ødtes bort og din Blusel blottedes for dine Elskere ved din Boler, derfor og for alle dine vederstyggelige Afgudsbilleders Skyld og for dine Sønners Blods Skyld, som du gav dem,
\par 37 se, derfor vil jeg samle alle dine Elskere, hvem du var til Glæde, både alle dem, du elskede, og alle dem, du hadede; jeg vil samle dem imod dig trindt om fra og blotte din Blusel for dem, så de ser den helt.
\par 38 Jeg vil dømme dig efter Horkvinders og Morderskers Ret og lade Vrede og Nidkærhed ramme dig.
\par 39 Jeg giver dig i deres Hånd, og de skal nedbryde din Alterfod, ødelægge dine Offerhøje, rive Klæderne af dig, tage dine Smykker og lade dig stå nøgen og bar.
\par 40 De skal sammenkalde en Forsamling imod dig, stene dig og med deres Sværd hugge dig sønder og sammen;
\par 41 de skal sætte Ild på dine Huse og fuldbyrde Dommen over dig i mange Kvinders Påsyn. Jeg gør Ende på din Bolen, og du skal ikke mere komme til at give Skøgeløn.
\par 42 Jeg stiller min Vrede på dig, til min Nidkærhed viger fra dig, så jeg får Ro og ikke mere er krænket.
\par 43 Fordi du ikke kom dine Ungdoms Dage i Hu, men vakte min Vrede ved alt dette, se, derfor vil jeg gøre Gengæld og lade din Færd komme over dit Hoved, lyder det fra den Herre HERREN. Du skal ikke vedblive at føje Skændsel til alle dine Vederstyggeligheder.
\par 44 Se, enhver, som ynder Ordsprog, skal bruge det Ordsprog om dig: "Som Moder så datter!"
\par 45 Du er din Moders Datter, hun lededes ved sin Mand og sine Børn; og du er dine Søstres Søster, de lededes ved deres Mænd og Børn. Eders Moder var Hetiterinde, eders Fader Amorit.
\par 46 Din store søster var samaria og hendes Døtre norden for dig, og din lille Søster sønden for dig var Sodoma og hendes Døtre.
\par 47 På deres Veje vandrede du ikke, og Vederstyggeligheder som deres øvede du ikke; kun en liden Stund, så handlede du endnu værre end de på alle dine Veje.
\par 48 Så sandt jeg lever, lyder det fra den Herre HERRE, din Søster Sodoma og hendes Døtre handlede ikke som du og dine Døtre!
\par 49 Se, din Søster Sodomas Brøde var Overmod; Brød i Overflod og sorgløs Tryghed blev hende og hendes Døtre til Del, men de rakte ikke den arme og fattige en hjælpende Hånd;
\par 50 de blev hovmodige og øvede Vederstyggelighed for mine Øjne; derfor stødte jeg dem bort, så snart jeg så det.
\par 51 Heller ikke Samaria syndede halvt så meget som du! Du har øvet flere Vederstyggeligheder end de og retfærdiggjort dine Søstre ved alle de Vederstyggeligheder, du øvede.
\par 52 Så bær da også du din Skændsel, du, som har skaffet dine Søstre Oprejsning; da dine Synder er vederstyggeligere end deres, står de retfærdigere end du; så skam da også du dig og bær din Skændsel, fordi du har retfærdiggjort dine Søstre.
\par 53 Og jeg vil vende deres Skæbne, Sodomas og hendes Døtres og Samarias og hendes Døtres, og jeg vil vende din Skæbne midt iblandt dem,
\par 54 for at du kan bære din Skændsel og blues ved alt, hvad du har gjort, idet du derved skaffede demen Trøst.
\par 55 Dine Søstre Sodoma og hendes Døtre og Samaria og hendes Døtre skal blive, hvad de fordum var, og du og dine Døtre, hvad I fordum var.
\par 56 Din Søster Sodomas Navn tog du ikke i din Mund i dit Overmods Dage,
\par 57 da din Blusel endnu ikke var blottet som nu, du Spot for Edoms Kvinder, og alle kvinder deromkring og for Filisternes Kvinder, som hånede dig fra alle Sider!
\par 58 Du må bære din Skændsel og dine Vederstyggeligheder, lyder det fra HERREN.
\par 59 Ja, så siger den Herre HERREN: Jeg gør med dig, som du har gjort, du, som lod hånt om Eden og brød Pagten.
\par 60 Men jeg vil ihukomme min Pagt med dig i din Ungdoms Dage og oprette en evig Pagt med dig.
\par 61 Og du skal komme dine Veje i Hu og blues, når jeg tager dine Søstre, både dem, der er større, og dem, der er mindre end du, og giver dig dem til Døtre, men ikke fordi du var tro i Pagten.
\par 62 Jeg opretter min Pagt med dig, og du skal kende, at jeg er HERREN,
\par 63 for at du skal komme det i Hu med Skam og ikke mere kunne åbne din Mund, fordi du blues, når jeg tilgiver dig alt, hvad du har gjort, lyder det fra den Herre HERREN.

\chapter{17}

\par 1 HERRENs Ord kom til mig således:
\par 2 Menneskesøn, fremsæt en Gåde og tal i Lignelse til Israels Slægt;
\par 3 sig: Så siger den Herre HERREN: Den store Ørn med vældigt Vingefang, lange Vinger, tæt Fjederham og brogede Farver kom til Libanon og tog Cederens Top;
\par 4 Spidsen af dens Skud brød den af, bragte den til et Kræmmerland og satte den i en Handelsby.
\par 5 Så tog den en Plante der i Landet og plantede den i en Sædemark ved rigeligt Vand...",
\par 6 for at den skulde vokse og blive en yppig, lavstammet Vinstok, hvis Ranker skulde vende sig til den, og hvis Rødder skulde blive under den. Og den blev en Vinstok, som skød Grene og bredte sine Kviste.
\par 7 Men der var en anden stor Ørn med vældigt Vingefang og rig Fjederham; og se, Vinstokken bøjede sine Rødder imod den og strakte sine Ranker hen til den, for at den skulde give den mere Vand end Bedet, den stod i.
\par 8 På en frugtbar Mark ved rigeligt Vand var den plantet for at skyde Grene, bære Frugt og blive en herlig Vinstok.
\par 9 Sig derfor: Så siger den Herre HERREN: Mon det lykkes den? Mon den første Ørn ikke rykker dens Rødder op og afriver dens Frugt, så alle de friske Skud tørres hen? Der skal jo ingen kraftig Arm eller mange Folk til at rive den løs fra Roden.
\par 10 Se, den er plantet, men mon det lykkes den? Mon den ikke, så snart Østenvinden når den, hentørres i Bedet, den voksede i?
\par 11 Og HERRENs Ord kom til mig således:
\par 12 Sig til den genstridige Slægt: Ved I ikke, hvad dette betyder? Sig: Babels Konge kom til Jerusalem, tog Kongen og Fyrsterne og førte dem med hjem til Babel.
\par 13 Derpå tog han entling af kongehuset og sluttede Pagt med ham og lod ham aflægge Ed. Landets Stormænd tog han dog med,
\par 14 for at Riget skulde holdes nede og ikke hovmode sig, men holde hans Pagt, at den måtte stå fast.
\par 15 Men han faldt fra og sendte sine Bud til Ægypten, for at de skulde give ham Heste og Folk i Mængde. Mon det lykkes ham? Mon den, der bærer sig således ad, slipper godt derfra? Skal den, der bryder en Pagt, slippe fra det?
\par 16 Så sandt jeg lever, lyder det fra den Herre HERREN: Hvor den Konge bor, som gjorde ham til Konge, hvis Ed han lod hånt om, og hvis Pagt han brød, der hos ham i Babel skal han dø.
\par 17 Og Farao skal ikke hjælpe ham i Krigen med en stor Hær eller en talrig Skare, når der opkastes Stormvold og bygges Belejringstårne til Undergang for mange Mennesker.
\par 18 Thi han lod hånt om Eden og brød Pagten trods givet Håndslag; alt dette gjorde han; han skal ikke undslippe!
\par 19 Sig derfor: Så siger den Herre HERREN: Så sandt jeg lever: Min Ed, som han lod hånt om, og min Pagt, som han brød, vil jeg visselig lade komme over hans Hoved!
\par 20 Jeg breder mit Net over ham, så han fanges i mit Garn, og jeg bringer ham til Babel for der at gå i Rette med ham for den Troløshed, han viste mig.
\par 21 Alle hans udvalgte Folk i alle hans Hære skal falde for Sværd, og de, der er til Rest, spredes for alle Vinde; og I skal kende, at jeg, HERREN, har talet.
\par 22 Så siger den Herre HERREN: Så tager jeg selv en Gren af Cederens Top, af dens Skuds Spidser bryder jeg en tynd Kvist og planter den på et højt, knejsende Bjerg.
\par 23 På Israels høje Bjerg vil jeg plante den, og den skal skyde Grene og bære Løv og blive en herlig Ceder. Under den skal alle vingede Fugle bygge, i dens Grenes Skygge skal de bo.
\par 24 Og alle Markens Træer skal kende, at jeg, HERREN, nedbøjer det høje Træ og ophøjer det lave, udtørrer det friske Træ og lader det tørre blomstre.

\chapter{18}

\par 1 HERRENs Ord kom til mig således:
\par 2 Hvor tør I bruge det Mundheld i Israels Land: Fædre åd sure Druer, og Børnenes Tænder blev ømme.
\par 3 Så sandt jeg lever, lyder det fra den Herre HERREN: Ingen skal mere bruge dette Mundheld i Israel.
\par 4 Se, alle Sjæle er mine; både Faderens Sjæl og Sønnens Sjæl er mine; den sjæl der synder skal dø.
\par 5 Når en Mand er retfærdig og gør Ret og Skel,
\par 6 ikke spiser på Bjergene eller løfter sit Blik til Israels Huses Afgudsbilleder eller skænder sin Næstes Hustru eller nærmer sig en Kvinde, så længe hun er uren,
\par 7 eller volder noget Menneske Men, men giver sit Håndpant tilbage, ikke raner, men giver den sultne sit Brød og klæder den nøgne,
\par 8 ikke låner ud mod Åger eller tager Opgæld, men holder sin Hånd fra Uret, fælder redelig Dom Mand og Mand imellem,
\par 9 vandrer efter mine Anordninger og tager Vare på at udføre mine Lovbud, han er retfærdig, han skal visselig leve, lyder det fra den Herre HERREN.
\par 10 Men avler han en Voldsmand til Søn, som udøser Blod og gør en eneste af disse Ting
\par 11 medens han selv ikke gjorde nogen af disse Ting - spiser på Bjergene, skænder sin Næstes Hustru
\par 12 volder de arme og fattige Men raner, ikke giver Håndpant tilbage, men løfter sit Blik til Afgudsbillederne, gør, hvad vederstyggeligt er,
\par 13 låner ud mod Åger og tager Opgæld, så skal han ingenlunde leve; han har øvet alle disse Vederstyggeligheder, han skal visselig lide Døden, hans Blod skal komme over ham.
\par 14 Men sæt, at Sønnen avler en Søn, som ser alle de Synder, Faderen gjorde, og at han bliver angst og ikke bærer sig således ad,
\par 15 ikke spiser på Bjergene eller løfter sit Blik til Israels Huses Afgudsbilleder eller skænder sin Næstes Hustru
\par 16 eller volder noget Menneske Men eller tager Håndpant eller raner, men giver den sultne sit Brød og klæder den nøgne,
\par 17 holder sin Hånd fra Uret, ikke tager Åger eller Opgæld, men holder mine Lovbud og vandrer efter mine Anordninger, så skal han ikke dø for sin Faders Misgerning, men visselig leve.
\par 18 Hans Fader derimod døde for sin Misgerning, fordi han øvede Vold, ranede og gjorde i sit Folk hvad ikke var godt.
\par 19 Og I siger: "Hvorfor skulde Sønnen ikke bære Faderens Misgerning?" Nej, thi Sønnen gjorde Ret og Skel, holdt alle mine Lovbud og levede efter dem.
\par 20 Den Sjæl, der synder, den skal dø; Søn skal ikke bære Faders Misgeming, ej heller Fader Søns. Over den retfærdige skal hans Retfærdighed komme, over den gudløse hans Gudløshed.
\par 21 Men når den gudløse omvender sig fra alle de Synder, han har gjort, og holder alle mine Anordninger og gør Ret og Skel, da skal han visselig leve og ikke dø.
\par 22 Ingen af alle de Overtrædelser, han har øvet, skal tilregnes ham; i Kraft af den Retfærdighed, han øver, skal han leve.
\par 23 Mon jeg har Lyst til den gudløses Død, lyder det fra den Herre HERREN, mon ikke til, at han omvender sig fra sin Vej, så han må leve?
\par 24 Men når den retfærdige vender sig fra sin Retfærdighed og gør Uret, lignende Vederstyggeligheder, som den gudløse øver, så skal ingen af de retfærdige Gerninger, han har gjort tilregnes ham; for den Troløshed, han øvede, og den Synd, han gjorde, skal han dø.
\par 25 Og I siger: "HERRENs Vej er ikke ret!" Hør dog, Israels Hus! Er det min Vej, der ikke er ret? Er det ikke snarere eders Vej, der ikke er ret?
\par 26 Når den retfærdige vender sig fra sin Retfærdighed og gør Uret, skal han dø; for den Uret, han gør, skal han dø.
\par 27 Men når en gudløs vender sig fra den Gudløshed, han har øvet, og gør Ret og Skel, skal han holde sin Sjæl i Live.
\par 28 Han vendte sig fra alle de Overtrædelser, han havde øvet; han skal visselig leve og ikke dø.
\par 29 Og Israels Hus siger: "HRRENs Vej er ikke ret!" Er det min Vej, Israels Hus, der ikke er ret? Er det ikke snarere eders Vej, der ikke er ret?
\par 30 Derfor dømmer jeg enhver af eder efter hans Veje, Israels Hus, lyder det fra den Herre HERREN. Vend om og omvend eder fra alle eders Overtrædelser, at de ikke skal blive eder Årsag til Skyld.
\par 31 Gør eder fri for alle de Overtrædelser, I har øvet imod mig, og skab eder et nyt Hjerte og en ny Ånd; thi hvorfor vil I dø, Israels Hus?
\par 32 Thi jeg har ikke Lyst til nogens Død, lyder det fra den Herre HERREN.

\chapter{19}

\par 1 Du menneskesøn istem en klagesang over Israels fyrster og sig:
\par 2 Hvor var dog din Moder en Løvinde midt iblandt Løver! Hun hviled blandt unge Løver opfostred Unger.
\par 3 En Unge voksede til, en Ungløve blev den; den lærte at røve Rov, Mennesker åd den.
\par 4 Da opbød man Folkene mod den, i Grav blev den fanget, de slæbte den bort med Kroge til Ægyptens Land.
\par 5 Da hun så, at den var ført bort, at Håbet var bristet, tog hun en anden Unge og gjorde til Løve.
\par 6 Den gik imellem Løvinder, en Ungløve blev den, den lærte at røve Rov, Mennesker åd den.
\par 7 Den overfaldt Vædre på Græs, var Hjordenes Rædsel Landet og dets Fylde stivned af Angst for dens Brøl.
\par 8 Og Folkene lagde Snarer rundt omkring den, over den bredte de Nettet, i Grav blev den fanget.
\par 9 De trak den med Kroge i Bur og førte den til Babels Konge, hen til Borgen, at dens Røst ej mer skulde høres på Israels Bjerge.
\par 10 Din Moder var en Vinstok i Vingården, plantet ved Vand, frugtbar og rig på Grene ved rigelig Væde.
\par 11 En af dens Grene blev til et Herskerspir dens knejsende Vækst skød op imellem Løvet, let at se i sin Højde, med mange Ranker.
\par 12 Men, i Vrede blev Vinstokken oprykt, slænget til Jorden, Østenstorm tørred dens Frugt, den reves af, dens stolte Gren blev vissen, Ild åd den op.
\par 13 Nu er den plantet i Ørkenen, et tørt og tørstigt Land.
\par 14 Ild for ud af dens Gren, fortæred dens Ranker og Frugt: en stolt Gren findes ej på den til Herskerspir. Dette er en Klagesang, og en Klagesang blev det.

\chapter{20}

\par 1 I det syvende år på den tiende dag i den femte måned kom nogle af Israels Ældste for at rådspørge HERREN, og de satte sig lige over for mig.
\par 2 Så kom HERRENs Ord til mig således
\par 3 Menneskesøn, tal til Israels Ældste og sig: Så siger den Herre HERREN: Kommer I for at rådspørge mig? Så sandt jeg lever: Jeg lader mig ikke rådspørge af eder, lyder det fra den Herre HERREN.
\par 4 Vil du dømme dem, vil du dømme, Menneskesøn? Så forehold dem deres Fædres Vederstyggeligheder
\par 5 og sig til dem: Så siger den Herre HERREN: Dengang jeg udvalgte Israel, løftede jeg min Hånd til Ed for Jakobs Huses Afkom og gav mig til Kende for dem i Ægypten; jeg løftede min Hånd for dem og svor: Jeg er HERREN eders Gud.
\par 6 Dengang løftede jeg min Hånd og tilsvor dem, at jeg vilde føre dem ud af Ægypten til Landet, jeg havde givet dem, et Land, der flyder med Mælk og Honning, det dejligste af alle Lande.
\par 7 Og jeg sagde til dem: Enhver skal bortkaste sine væmmelige Guder, som hans Øjne hænger ved; og I må ikke gøre eder urene ved Ægyptens Afgudsbilleder.
\par 8 Men de var genstridige imod mig og vilde ikke høre mig; de bortkastede ikke deres væmmelige Guder, som deres Øjne hang ved, og lod ikke Ægyptens Afgudsbilleder fare. Så tænkte jeg på at udøse min Vrede over dem og køle min Harme på dem midt i Ægypten.
\par 9 For mit Navns Skyld greb jeg dog ind, at det ikke skulde vanæres for de Folks Øjne, blandt hvilke de levede, og i hvis Påsyn jeg havde åbenbaret mig for dem, idet jeg førte dem ud af Ægypten.
\par 10 Så førte jeg dem ud af Ægypten og bragte dem ud i Ørkenen;
\par 11 og jeg gav dem mine Anordninger og kundgjorde dem mine Lovbud; det Menneske, som gør efter dem, skal leve ved dem.
\par 12 Også mine Sabbater gav jeg dem, for at de skulde være et Tegn mellem mig og dem, at det skal kendes, at jeg, HERREN, er den, som helliger dem.
\par 13 Men Israels Hus var genstridigt imod mig i Ørkenen; de vandrede ikke efter mine Anordninger, men lod hånt om mine Lovbud - det Menneske, som gør efter dem, skal leve ved dem og mine Sabbater vanhelligede de grovelig. Så tænkte jeg på at udøse min Vrede over dem i Ørkenen og tilintetgøre dem.
\par 14 For mit Navns Skyld greb jeg dog ind, at det ikke skulde vanæres for de Folks Øjne, i hvis Påsyn jeg havde ført dem ud.
\par 15 Og jeg løftede min Hånd for dem i Ørkenen og svor, at jeg ikke vilde føre dem ind i det Land, jeg havde givet dem, et Land, der flydler med Mælk og Honning, det dejligste af alle Lande,
\par 16 fordi de lod hånt om mine Lovbud og ikke vandrede efter mine Anordninger, men vanhelligede mine Sabbater; thi deres Hjerte holdt sig til deres Afgudsbilleder.
\par 17 Jeg havde Medlidenhed med dem, så jeg ikke tilintetgjorde dem; jeg gjorde ikke Ende på dem i Ørkenen.
\par 18 Så sagde jeg til deres Sønner i Ørkenen: Følg ikke eders Fædres Anordninger, hold ikke deres Lovbud og gør eder ikke urene med deres Afgudsbilleder.
\par 19 Jeg, HERREN, er eders Gud! Følg mine Anordninger og tag Vare på at holde mine Lovbud;
\par 20 hold mine Sabbater hellige, så de bliver et Tegn mellem mig og eder, at det må kendes, at jeg, HERREN, er eders Gud.
\par 21 Men også Sønnerne var genstridige imod mig; de fulgte ikke mine Anordninger og tog ikke Vare på at holde mine Lovbud - det Menneske, som gør efter dem, skal leve ved dem - og vanhelligede mine Sabbater. Så tænkte jeg på at udøse min Harme over dem og køle min Vrede på dem i Ørkenen.
\par 22 Dog holdt jeg min Hånd tilbage, og jeg greb ind for mit Navns Skyld, at det ikke skulde vanæres for de Folks Øjne, i hvis Påsyn jeg havde ført dem ud.
\par 23 Jeg løftede min Hånd for dem i Ørkenen og svor, at jeg vilde sprede dem blandt Folkene og udstrø dem i Landene,
\par 24 fordi de ikke holdt mine Lovbud, men lod hånt om mine Anordninger og vanhelligede mine Sabbater, og deres Øjne hang ved deres Fædres Afgudsbilleder.
\par 25 Derfor gav jeg dem Anordninger, som ikke er gode, og Lovbud ved hvilke de ikke vandt Liv;
\par 26 jeg gjorde dem urene ved deres Gaver, idet de lod alt hvad der åbner Moders Liv, gå igennem Ilden; thi jeg vilde have dem til at stivne af Rædsel, at de måtte kende, at jeg er HERREN.
\par 27 Derfor, Menneskesøn, tal til Israels Hus og sig til dem: Så siger den Herre HERREN: Eders Fædre hånede mig ydermere ved at være troløse imod mig.
\par 28 Jeg bragte dem til det Land jeg med løftet Hånd havde svoret at give dem; men hver Gang de så en høj Bakke eller et løvrigt Træ ofrede de der deres Slagtofre og bragte deres krænkende Offergave; der beredte de deres liflige Duft og udgød deres Drikofre.
\par 29 Da sagde jeg til dem: "Hvad er det for en Offerhøj, I går hen til?" Og derfor bærer den endnu den Dag i Dag Navnet Offerhøj.
\par 30 Sig derfor til Israels Hus: Så siger den Herre HERREN: Gør I eder ikke urene på eders Fædres Vis og boler med deres væmmelige Guder?
\par 31 Ja, når I bringer eders Gaver, når I lader eders Sønner gå igennem Ilden, gør I eder den Dag i Dag urene til Ære for alle eders Afgudsbilleder - og så skulde jeg lade mig rådspørge af eder, Israels Hus? Så sandt jeg lever, lyder det fra den Herre HERREN: Jeg lader mig ikke rådspørge af eder!
\par 32 Hvad der er kommet op i eders Sind, skal visselig ikke ske; I siger: "Vi vil være som Folkene, som Slægterne i andre Lande og dyrke Træ og Sten!"
\par 33 Så sandt jeg lever, lyder det fra den Herre HERREN: Med stærk Hånd og udstrakt Arm og udøst Vrede vil jeg vise, at jeg er eders Konge.
\par 34 Med stærk Hånd og udstrakt Arm og udøst Vrede vil jeg føre eder bort fra Folkeslagene og samle eder fra de Lande, hvor I er spredt,
\par 35 og bringe eder til Folkeslagenes Ørken, og der vil jeg gå i Rette med eder Ansigt til Ansigt.
\par 36 Som jeg gik i Rette med eders Fædre i Ægyptens Ørken, vil jeg gå i Rette med eder, lyder det fra den Herre HERREN.
\par 37 Jeg vil lade eder gå under Staven og føre eder fuldtalligt frem.
\par 38 Jeg vil fraskille dem, der var genstridige og faldt fra mig; jeg fører dem ud af deres Udlændigheds Land, men til Israels Land skal de ikke komme; og I skal kende, at jeg er HERREN.
\par 39 Men I, Israels Hus! Så siger den Herre HERREN: Gå hen og dyrk hver sit Afgudsbillede, men siden skal I visselig høre min Røst og ikke mere vanhellige mit hellige Navn med eders Offergaver og Afgudsbilleder.
\par 40 Thi på mit hellige Bjerg, på Israels høje Bjerg, lyder det fra den Herre HERREN, der skal hele Israels Hus i Landet tjene mig; der vil jeg vise dem mit Velbehag, og der vil jeg spørge efter eders Offerydelser og Førstegrødegaver, alt, hvad I vil hellige.
\par 41 Ved den liflige Duft vil jeg vise eder mit Velbehag, når jeg fører eder ud fra Folkeslagene og samler eder fra alle de Lande, hvor I er spredt, og jeg vil på eder vise mig som den Hellige for Folkenes Øjne.
\par 42 Og I skal kende, at jeg er HERREN, når jeg fører eder til Israels Jord, det Land, jeg med løftet Hånd svor at give eders Fædre.
\par 43 Der skal I ihukomme eders Veje og alle de Gerninger, I gjorde eder urene med, så I ledes ved eder selv for alt det onde, I øvede.
\par 44 Og I skal kende, at jeg er HERREN, når jeg gør således med eder for mit Navns Skyld, ikke efter eders onde Veje og skændige Gerninger, Israels Hus, lyder det fra den Herre HERREN.
\par 45 HERRENs Ord kom til mig således:
\par 46 Menneskesøn, vend dit Ansigt mod Sønden og lad din Tale strømme sønderpå og profeter mod Sydlandets Skov!
\par 47 Sig til Sydlandets Skov: Hør HERRENs Ord! Så siger den Herre HERREN: Jeg sætter Ild på dig, og den skal fortære alle Træer i dig, både friske og tørre; den luende Ild skal ikke slukkes, og alle Ansigter skal svides fra Syd til Nord.
\par 48 Og alt Kød skal se, at jeg, HERREN, har tændt den; den skal ikke slukkes!
\par 49 Da sagde jeg: "Ak, Herre, HERRE, de siger om mig, at jeg altid taler i Lignelser!"

\chapter{21}

\par 1 Da kom HERRENs Ord til mig sååledes:
\par 2 Mennesskesøn, vend dit Ansigt mod Jerusalem, lad din Tale strømme mod Helligdommen og profeter mod Israels Land!
\par 3 Sig til Israels Land: Så siger HERREN: Se, jeg kommer over dig og drager mit Sværd af Skedn for at udrydde både retfærdige og gudløse af dig.
\par 4 Fordi jeg vil udrydde både retfærdige og gudløse af dig, derfor skal mit Sværd fare af Skeden mod alt Kød fra Syd til Nord.
\par 5 Og alt Kød skal kende, at jeg, HERREN, har draget mit Sværd at Skeden; det skal ikke vende tilbage!
\par 6 Men du, Menneskesøn, støn, støn for deres Øjne, som om dine Lænder skulde briste, i bitter Smerte!
\par 7 Og når de spørger: "Hvorfor stønner du?" så svar: "Over en Tidende; thi når den kommer, skal hvert Hjerte smelte, alle Hænder synke, hver Ånd sløves og alle Knæ flyde som Vand. Se, den kommer, den fuldbyrdes, lyder det fra den Herre HERREN."
\par 8 HERRENs Ord kom til mig således:
\par 9 Menneskesøn, profeter og sig: Så siger HERREN:
\par 10 Et Sværd, et Sværd er hvæsset og slebet blankt, hvæsset med Slagtning for Øje, blankt til at udsende Lyn..."
\par 11 Jeg gav en Slagter det, at han skal tage det fat; det er hvæsset og slebet for at gives en Drabsmand i Hænde,
\par 12 Råb og vånd dig, Menneskesøn! Thi det kommet over mit Folk, over alle Israels Fyrster; sammen med mit Folk er de givet til Sværdet. Derfor slå dig på Hoften!
\par 13 lyder det fra den Herre HERREN.
\par 14 0g du, Menneskesøn, profeter og slå Hænderne sammen, gør Sværdet som to, ja, gør det som tre! Det er et dræbende Sværd, den store Hednings Sværd; indjag dem Rædsel dermed,
\par 15 at deres Hjerter må ængstes og mange må falde ved alle Porte. Jeg sætter dig til at slagte, du Sværd, som er gjort til at lyne, hvæsset til Slagtning.
\par 16 Indjag Rædsel både til højre og venstre, hvor din Od rettes hen!
\par 17 Også jeg vil slå Hænderne sammen og køle min Vrede. Jeg, HERREN, har talet!
\par 18 HERRENs Ord kom til mig således:
\par 19 Du, Menneskesøn, afsæt dig to Veje, ad hvilke Babels konges Sværd skal komme, således at begge udgår fra et og samme Land; og opstil en Vejviser der, hvor de to Byveje skilles,
\par 20 så at Sværdet både kan komme til Rabba i Ammoniternes Land og til Juda og Jerusalem midt i Juda.
\par 21 Thi Babels Konge står på Vejskellet, hvor de to Veje skilles, for at tage Varsler; han ryster Pilene", rådspørger Husguderne, ransager Leveren.
\par 22 I sin højre holder han Loddet "Jerusalem", at han skal åbne Munden til Skrig og løfte Røsten til Krigsråb, rejse Stormbukke mod Portene, opkaste Stormvold og bygge Belejringstårne.
\par 23 Ederne, svorne ved Gud, regnede de lige med falsk Spådom, men han bringer deres Brøde i Minde, for at de skal fanges.
\par 24 Derfor, så siger den Herre HERREN: Fordi I bringer eders Brøde i Minde, idet eders Overtrædelser åbenbares, så eders Synder bliver synlige i alt, hvad I gør, fordi I bringer eder i Minde ved dem, skal I fanges.
\par 25 Og du, gudløse Hedning, Israels Fyrste, hvis Time slår, når din Misgerning er fuldmoden,
\par 26 så siger den Herre HERREN: Bort med Hovedbindet, ned med kronen! Som det var, er det ikke mere! Op med det lave, ned med det høje!
\par 27 Grushobe, Grushobe, Grushobe gør jeg det til. Ve det! Således skal det være, til han kommer, som har Retten til det; ham vil jeg give det.
\par 28 Du, Menneskesøn, profeter således: Så siger den Herre HERREN om Ammoniterne og deres Hån! Og sig: Et Sværd, et Sværd er draget til at slagte hvæsset til at udsende Lyn,
\par 29 medens man skuer dig Tomhed og spår dig Løgn, for at det skal lægges på de gudløse Hedningers Hals, hvis Time slår, når deres Misgerning er fuldmoden.
\par 30 Vend tilbage til din Borg! På det Sted, hvor du skabtes, i det Land, du stammer fra, vil jeg dømme dig.
\par 31 Jeg vil udøse min Vrede over dig, blæse min Harmes Ild op imod dig og give dig i grumme Menneskers Hånd, som er Mestre i at tilintetgøre.
\par 32 Du skal blive Ildens Føde, dit Blod skal flyde i dit Land; du skal ikke kommes i Hu, thi jeg, HERREN, har talet.

\chapter{22}

\par 1 Herrens ord kom til mig således:
\par 2 Du Menneskesøn! Vil du dømme Blodbyen? Så forehold den alle dens Vederstyggeligheder
\par 3 og sig: Så siger den Herre HERREN: Ve Byen, der udøser Blod i sin Midte, for at dens Time skal komme, og laver sig Afgudsbilleder for at gøre sig uren.
\par 4 Ved dine egne Folks Blod, som du har udgydt, har du pådraget dig Skyld, og ved de Afgudsbilleder du har lavet, er du blevet uren; du har bragt din Time nær og hidført dine Års Frist. Derfor gør jeg dig til Hån for Folkene og til Spot for alle Lande;
\par 5 fra nær og fjern skal man spotte dig, du, hvis Navn er skændet, og som er fuld af Larm.
\par 6 Se, Israels Fyrster optræder hver og een egenmægtigt i dig og udøser Blod.
\par 7 Fader og Moder ringeagtes, den fremmede undertrykkes i dig, den faderløse og Enken lider Uret.
\par 8 Mine hellige Ting agter du ringe og vanhelliger mine Sabbater.
\par 9 Du har Æreskændere i din Midte, som bagtaler, så der udøses Blod, og Folk, som spiser på Bjergene. Utugt går i Svang i din Midte;
\par 10 i dig blottes Faderens Skam, og Kvinder krænkes i deres Urenheds Tid.
\par 11 Man øver Vederstyggelighed mod sin Næstes Hustru, man gør sin Sønnekone uren ved Utugt, man skænder i dig sin kødelige Søster.
\par 12 I dig tager man Gave for at udøse Blod; du tager Åger og Opgæld, voldelig øver du Svig mod din Næste; og mig glemmer du, lyder det fra den Herre HERREN.
\par 13 Men se, jeg slår Hænderne sammen over den uredelige Vinding, du har skaffet dig, og over det Blod, der er udøst i dig.
\par 14 Kan dit Hjerte holde Stand, kan dine Hænder være faste i de Dage, da jeg tager fat på dig? Jeg, HERREN, har talet, og jeg fuldbyrder det.
\par 15 Jeg vil sprede dig blandt Folkene og udstrø dig i Landene og tage din Urenhed fra dig;
\par 16 jeg vil lade mig vanære ved dig for Folkenes Øjne; og du skal kende, at jeg er HERREN.
\par 17 Og HERRESs Ord kom til mig således:
\par 18 Menneskesøn! Israels Hus er blevet mig til Slagger; de er alle blevet Kobber, Tin, Jern og Bly i Smelteovnen; Sølvslagger er de blevet.
\par 19 Derfor, så siger den Herre HERREN: Fordi I alle er blevet til Slagger, derfor vil jeg samle eder i Jerusalem;
\par 20 som man samler Sølv, Kobber, Jern, Bly og Tin i en Smelteovn og blæser til Ilden for at smelte det, således vil jeg i min Vrede og Harme samle eder og kaste eder ind og smelte eder;
\par 21 jeg vil samle eder og blæse min Vredes Ild op imod eder, så I smeltes deri.
\par 22 Som Sølv smeltes i Smelteovnen, skal I smeltes deri; og I skal kende, atjeg, HERREN, har udøst min Vrede over eder.
\par 23 Og HERRENs Ord kom til mig således:
\par 24 Menneskesøn, sig til Landet: Du er et Land, der ikke får Regn og Byger på Vredens Dag,
\par 25 hvis Fyrster er som brølende Løver på Rov; de tilintetgør Sjæle, tilriver sig Gods og Guld, gør mange til Enker deri.
\par 26 Præsterne øver Vold mod min Lov, vanhelliger mine hellige Ting og gør ikke Skel mellem det, der er helligt, og det, der ikke er helligt, lærer ikke Forskel mellem rent og urent og lukker Øjnene for mine Sabbater, så jeg er blevet vanhelliget iblandt dem.
\par 27 Fyrsterne er som Ulve på Rov: de er ivrige efter at udøse Blod og tilintetgøre Menneskeliv for at skaffe sig Vinding.
\par 28 Profeterne stryger over med Kalk, idet de skuer Tomhed og varsler Løgn og siger: "Så siger den Herre HERREN!" uden at HERREN har talet.
\par 29 Det menige Folk øver Vold og går på Rov, den arme og fattige gør de Uret, og den fremmede undertrykker de imod Lov og Ret.
\par 30 Jeg søgte iblandt dem efter en, der vilde bygge en Mur og stille sig i Gabet for mit Åsyn til Værn for Landet, at jeg ikke skulde ødelægge det; men jeg fandt ingen.
\par 31 Derfor udøser jeg min Vrede over dem, med min Harmes Ild tilintetgør jeg dem; deres Færd lader jeg komme over deres Hoved, lyder det fra den Herre HERREN.

\chapter{23}

\par 1 HERRENs Ord kom til mig således:
\par 2 Meneskesøn! Der var to Kvinder, Døtre af en og samme Moder.
\par 3 De bolede i deres Ungdom i Ægypten: der krammedes deres Bryster, der krænkede man deres Jomfrubarm.
\par 4 Den ældste hed Ohola, hendes Søster Oholiba. Og de blev mine og fødte Sønner og Døtre. Ohola er Samaria, Oholiba Jerusalem.
\par 5 Men i Stedet for at holde sig til mig bolede Ohola og var i Brynde for sine Elskere, Assurs Sønner, der nærmede sig hende
\par 6 klædt i Purpur, Statholdere og Landshøvdinger, alle sammen smukke unge Mænd, Ryttere højt til Hest;
\par 7 og hun gav dem sin bolerske Elskov, alle Assurs ypperste Sønner, og alle Vegne, hvor hun kom i Brynde, gjorde hun sig uren ved deres Afgudsbilleder.
\par 8 Men sin Bolen med Ægypterne opgav hun ikke, thi de hade ligget hos hende i hendes Ungdom; de havde skændet hendes Jomfrubarm og udøst deres bolerske Attrå over hende.
\par 9 Derfor gav jeg hende i hendes Elskeres, Assurs Sønners, Hånd, for hvem hun var i Brynde;
\par 10 og de blottede hendes Blusel, tog hendes Sønner og Døtre og dræbte hende selv med Sværd, så hun fik Vanry blandt Kvinder; således fuldbyrdede de Dommen over hende.
\par 11 Det så hendes Søster Oholiba, og dog kom hun i endnu værre Brynde og bolede endnu værre end Søsteren.
\par 12 Hun kom i Brynde for Assurs Sønner, Statholdere og Landshøvdinger, der nærmede sig hende herligt klædt, Ryttere højt til Hest, alle sammen smukke unge Mænd.
\par 13 Og jeg så, at hun blev uren; begge fulgte samme Vej.
\par 14 Men hun drev sin Bolen videre endnu; thi da hun så Mænd afbildede på Muren, Billeder af Kaldæere, malet med rødt,
\par 15 med Bælte om Lænd og nedhængende Hovedbind, alle at se til som Høvedsmænd, en Afbildning af Babels Sønner, hvis Hjemstavn Kaldæa er,
\par 16 så kom hun i Brynde for dem, så snart hun fik Øje på dem, og hun sendte Bud til dem i Kaldæa:
\par 17 Og Babels Sønner gik ind til hende, lå hos hende i Elskov og gjorde hende uren ved deres Bolen; og hun blev uren ved dem, til hun følte Lede ved dem.
\par 18 Da hun havde åbenbaret sin bolerske Attrå og blottet sin Blusel, følte jeg Lede ved hende, som jeg var blevet led ved hendes søster.
\par 19 Men hun drev sin Bolen videre endnu, idet hun kom sin Ungdoms Dage i Hu, da hun havde bolet i Ægypten,
\par 20 og hun kom i Brynde ved dets Bolere, der havde Kød som Æsler og var gejle som Hingste;
\par 21 og hun optog sin Ungdoms Skændsel, dengang Ægypterne krænkede hendes Jomfrubarm og krammede hendes unge Bryster,
\par 22 Derfor, Oholiba, så siger den Herre HERREN: Jeg hidser dine Elskere på dig, dem, du følte Lede ved, og fører dem mod dig fra alle Kanter,
\par 23 Babels Sønner og alle Kaldæerne, Pekod, Sjoa og Koa og med dem alle Assurs Sønner, alle sammen smukke unge Mænd, Statholdere og Landshøvdinger, Høvedsmænd og navnkundige Mænd, alle højt til Hest;
\par 24 og de skal komme imod dig med en Vrimmel af Vogne og Hjul og en Hærskare af Folkeslag; store og små Skjolde og Hjelme skal de rette imod dig fra alle Kanter. Jeg overlader dem Dommen, og de skal dømme dig efter deres Ret.
\par 25 Jeg retter min Nidkærhed imod dig, og de skal handle med dig i Vrede; din Næse og dine Ører skal de skære af, dit Afkom skal falde for Sværdet; de skal tage dine Sønner og Døtre, og dit Afkom skal fortæres af Ild;
\par 26 de skal rive Klæderne af dig og tage alle dine Smykker;
\par 27 jeg gør Ende på din Skændsel og din Bolen, som du hengav dig til i Ægypten, og du skal ikke mere løfte dit Blik til dem eller komme Ægypten i Hu.
\par 28 Thi så siger den Herre HERREN: Se, jeg giver dig i deres Hånd, som du hader og ledes ved;
\par 29 de skal handle med dig i Had og tage alt dit Gods og efterlade dig nøgen og bar, så din bolerske blussel blottes.
\par 30 Det kan du takke din Skændsel og din Bolen for, thi du bolede med Folkene og gjorde dig uren med deres Afgudsbilleder.
\par 31 Du vandrede i din Søsters Spor; derfor giver jeg dig hendes Bæger i Hånden.
\par 32 Så siger den Herre HERREN: Du skal drikke af din Søsters Bæger, som er både dybt og bredt; du skal blive til Latter og Spot; Bægeret rummer meget;
\par 33 det er fuldt af Beruselse og Stønnen, et Bæger med Gru og Rædsel er din Søster Samarias Bæger.
\par 34 Du skal drikke det ud til sidste Dråbe, gnave dets Skår og sønderrive dine Bryster, så sandt jeg har talet, lyder det fra den Herre HERREN.
\par 35 Derfor, så siger den Herre HERREN: Fordi du glemte mig og kastede mig bag din Ryg, skal du bære Følgen af din Skændsel og Bolen.
\par 36 Og HERREN sagde til mig: Menneskesøn! Vil du dømme Ohola og Oholiba, så forehold dem deres Vederstyggeligheder,
\par 37 at de horede og har Blod på Hænderne; de horede med deres Afgudsbilleder, og til Føde for dem lod de Børnene, de fødte mig, gå igennem Ilden.
\par 38 Fremdeles har de gjort mig dette: De har gjort min Helligdom uren og vanhelliget mine Sabbater;
\par 39 endog samme Dag de havde slagtet deres Børn til Afgudsbillederne, kom de til min Helligdom og vanhelligede den. Sandelig, således gjorde de i mit Hus.
\par 40 Ja, for Mænd, som kom langvejs fra, straks der var sendt dem Bud, badede du dig, sminkede du dine Øjne og tog dine Smykker på;
\par 41 og du satte dig på et prægtigt Leje, og foran det dækkedes et Bord, på hvilket du satte min Røgelse og min Olie.
\par 42 Og det ligefrem larmede hos Søstrene; så mange Mænd kom der fra Ørkenen; og de lagde Spange om deres Arme og satte en herlig Krone på deres Hoved.
\par 43 Så sagde jeg: Således har de horet, på Skøgevis har de bolet.
\par 44 Man gik ind til dem som til en Skøge; således gik man ind til Ohola og Oholiba og øvede Skændsel.
\par 45 Uvildige Mænd skal dømme dem efter Ægteskabsbryderskers og Morderskers Ret; thi de horede og har Blod på Hænderne.
\par 46 Thi så siger den Herre HERREN: En Forsamling skal sammenkaldes imod dem, og de skal gives hen til at mishandles og udplyndres;
\par 47 Forsamlingen skal stene dem og hugge dem sønder og sammen med Sværd; deres Sønner og Døtre skal man dræbe, og deres Huse skal man brænde.
\par 48 Jeg gør Ende på Skændselen i Landet, og alle kvinder skal lade sig advare derved og ikke efterligne eders Skændsel.
\par 49 Man skal lade eders Skændsel komme over eder, og I skal bære Følgen af de Synder, I gjorde med eders Afgudsbilleder; og I skal kende, at jeg er den Herre HERREN.

\chapter{24}

\par 1 HERRENs Ord kom i det niende år på den tiende dag i den tiende Måned til mig således:
\par 2 Menneskesøn, opskriv dig Navnet på denne Dag, Dagen i Dag, thi netop i Dag har Babels Konge kastet sig over Jerusalem.
\par 3 Og tal i Lignelse til den genstridige Slægt og sig: Så siger den Herre HERREN: Sæt kedelen over, sæt den over; kom også Vand deri;
\par 4 læg Kødstykker i, alle Hånde gode Stykker, Kølle og Bov, fyld den med udsøgte Knogler;
\par 5 tag af Hjordens bedste Dyr og læg en Stabel Brænde under den; kog Stykkerne, så også Knoglerne koges ud!
\par 6 Derfor, så siger den Herre HERREN: Ve Blodbyen, den rustne Kedel, hvis Rust ikke er gået af. Stykke efter Stykke giver den fra sig; der kastes ikke Lod om dem;
\par 7 - thi Blodet er endnu midti Byen; den hældte det på den nøgne Klippe og udgød det ikke på Jorden for at dække det med Muld.
\par 8 For at fremkalde Vrede, for at tage Hævn hældte jeg Blodet på den nøgne klippe uden at dække det.
\par 9 Derfor, så siger den Herre HERREN: Ve Blodbyen! Nu vil også jeg gøre brændestabelen stor;
\par 10 hent brænde i Mængde, lad Ilden lue og Kødet blive mørt, hæld Suppen ud og lad Benene brændes
\par 11 og sæt kedelen tom på de glødende kul, for at den kan blive så hed, at Kobberet gløder og Urenheden smelter; Rusten skal svinde!
\par 12 Møje har den kostet, men den megen Rust gik ikke af. I Ilden med Rusten!
\par 13 Fordi du er uren af Utugt, fordi du ikke kom af med din Urenhed, skønt jeg rensede dig, skal du ikke blive ren igen, før jeg har kølet min Harme på dig.
\par 14 Jeg, HERREN, har talet, og det kommer! Jeg griber ind og opgiver det ikke, jeg skåner ikke og angre ej heller; efter dine Veje og dine Gerninger vil jeg dømme dig, lyder det fra den Herre HERREN.
\par 15 HERRENs Ord kom til mig således:
\par 16 Menneskesøn! Se, jeg tager dine Øjnes Lyst fra dig ved en brat Død; men du skal ikke klage eller græde; du skal ikke fælde Tårer;
\par 17 suk i Stilhed og hold ikke Døde klage, bind Huen på og tag Sko på Fødderne, tilhyl ikke dit Skæg og spis ikke Sørgebrød!
\par 18 Tal så om Morgenen til Folket! - Så døde min Hustru om Aftenen, og næste Morgen gjorde jeg, som mig var pålagt.
\par 19 Og da Folket sagde til mig: "Vil du ikke lade os vide, hvad det, du der gør, skal sige os?"
\par 20 svarede jeg: "HERRENs Ord kom til mig således:
\par 21 Sig til Israels Hus: Så siger den Herre HERREN: Se, jeg vil vanhellige min Helligdom, eders Hovmods Stolthed, eders Øjnes Lyst, eders Sjæles Længsel; og de Sønner og Døtre, I lod tilbage, skal falde for Sværdet!
\par 22 Og I skal gøre, som jeg nu gør: I skal ikke tilhylle eders Skæg eller spise Sørgebrød;
\par 23 eders Huer skal blive på Hovederne og eders Sko på Fødderne; I skal ikke klage eller græde, men svinde hen i eders Synder og sukke for hverandre.
\par 24 Ezekiel skal være eder et Tegn; når det sker, skal I gøre, som han gør; og I skal kende, at jeg er den Herre HERREN,"
\par 25 Og du, Menneskesøn! På den Dag jeg tager deres Værn, deres herlige Fryd, deres Øjnes Lyst og deres Sjæles Længsel, deres Sønner og Døtre fra dem,
\par 26 på den Dag skal en Flygtning komme til dig og melde det for dine Ører;
\par 27 på den Dag skal din Mund åbnes, når Flygtningen kommer, og du skal tale og ikke mere være stum; du skal være dem et Tegn; og de skal kende, at jeg er HERREN.

\chapter{25}

\par 1 HERRENs Ord kom til mig således:
\par 2 Mennesskesøn, vend dit Ansigt imod Ammoniterne, profeter imod dem
\par 3 og sig til dem: Hør den Herre HERRENs Ord: Så siger den Herre HERREN: Fordi du råbte "Ha, ha!" over min Helligdom, da den vanhelligedes, og over Israels Land, da det lagdes øde, og over Judas Hus, da de vandrede i Landflygtighed,
\par 4 se, derfor giver jeg dig i Eje til Østens Sønner; de skal opslå deres Teltlejre og indrette deres Boliger i dig; de skal spise din Frugt og drikke din Mælk.
\par 5 Jeg gør Rabba til Græsgang for Kameler og Ammons Byer til Lejrsted for Småkvæg; og I skal kende, at jeg er HERREN.
\par 6 Thi så siger den Herre HERREN: Fordi du klappede i Hænderne og stampede med Fødderne og med dyb Ringeagt godtede dig af Hjertet over Israels Land,
\par 7 se, derfor udrækker jeg Hånden imod dig og gør dig til Rov for Folkene; jeg udrydder dig af Folkeslagene, udsletter dig af Landene og tilintetgør dig; og du skal kende, at jeg er HERREN.
\par 8 Så siger den Herre HERREN: Fordi Moab siger: "Se, det er med Judas Hus som med alle de andre Folk!"
\par 9 se, derfor lægger jeg Moabs Skrænter åbne, så Byerne går tabt fra dets ene Ende til den anden, Landets Pryd, Bet Jesjimot, Baal-Meon og Hirjatajim.
\par 10 Østens Sønner giver jeg det i Eje, for at det ikke mere skal ihukommes blandt Folkene.
\par 11 Jeg holder Dom over Moab; og de skal kende, at jeg er HERREN.
\par 12 Så siger den Herre HERREN: Fordi Edom optrådte hævngerrigt mod Judas Hus og pådrog sig svar Skyld ved at hævne sig på dem,
\par 13 derfor, så siger den Herre HEEREN: Jeg udrækker Hånden mod Edom og udrydder Folk og Fæ deraf og gør det øde; fra Teman til Dedan skal de falde for Sværdet.
\par 14 Jeg fuldbyrder min Hævn på Edom ved mit Folk Israels Hånd, og de skal handle med Edom efter min Vrede og Harme, og Edom skal kende min Hævn, lyder det fra den Herre HERREN.
\par 15 Så siger den Herre HERREN: Fordi Filisterne optrådte hævngerrigt og med Foragt i Hjertet tog Hævn og hærgede i endeløst Had,
\par 16 derfor, så siger den Herre HERREN: Se, jeg udrækker Hånden mod Filisterne og udrydder kreterne, og jeg tilintetgør, hvad der er levnet ved Havets Strand.
\par 17 Jeg tager vældig Hævn oer dem og revser dem i Vrede; og de skal kende, at jeg er HERREN, når jeg fuldbyrder min Hævn på dem.

\chapter{26}

\par 1 I det ellevte År På den første dag i ... månede kom HERRENs Ord til mig således:
\par 2 Menneskesøn! Fordi Tyrus siger om Jerusalem: "Ha, ha! Folkeslagenes Port er knust; den står mig åben; den var rig, men er nu øde!"
\par 3 derfor, så siger den Herre HERREN: Se, jeg kommer over dig, Tyrus, og fører mange Folk imod dig, som når Havet rejser sine Bølger.
\par 4 De skal ødelægge Tyruss Mure og nedbryde Tårnene. Jeg fejer Muldet bort og gør Tyrus til nøgen Klippe;
\par 5 en Tørreplads for Net skal det være ude i Havet, så sandt jeg har talet, lyder det fra den Herre HERREN; og det skal blive et Bytte for Folkene.
\par 6 Dets Døtre på Land skal hugges ned med Sværd; og de skal kende, at jeg er HERREN.
\par 7 Thi så siger den Herre HERREN: Se, nordenfra fører jeg mod Tyrus Kong Nebukadrezar af Babel, Kongernes Konge, med Heste, Vogne og Ryttere og en vældig Hærskare.
\par 8 Dine Døtre på Land skal han hugge ned med Sværd; han skal bygge Belejringstårne, opkaste Stormvold og rejse Skjoldtag imod dig;
\par 9 sin Murbrækkers Stød skal han rette imod dine Mure og nedbryde dine Tårne med sine Sværd.
\par 10 Hans Heste skal myldre, så Støvet, de rejser, skjuler dig; Ryttere, Hjul og Vogne skal larme, så dine Mure ryster, når han drager igennem dine Porte, som når man trænger ind i en stormet By.
\par 11 Med sine Hestes Hove skal han nedtrampe alle dine Gader; dit Folk skal han hugge ned med Sværdet og styrte dine stolte Støtter til Jorden.
\par 12 De skal rane din Rigdom og gøre dine Handelsvarer til Bytte; de skal nedbryde dine Mure og nedrive dine herlige Huse: Sten, Tømmer og Ler skal de kaste i Vandet.
\par 13 Jeg gør Endepå dine brusende Sange, og dine Citres Klang skal ikke mere høres.
\par 14 Jeg gør dig til nøgen Klippe, en Tørreplads for Net skal du være. Aldrig mere skal du bygges, så sandt jeg, HERREN, har talet, lyder det fra den Herre HERREN.
\par 15 Så siger den Herre HERREN til Tyrus: For sandt, ved dit drønende Fald, ved de såredes Stønnen, når Sværd hugger løs i din Midte, skal Strandene skælve.
\par 16 Ned fra sin Trone stiger hver Fyrste ved Havet, Kapperne lægger de bort, aflægger de brogede Klæder og klæder sig i Sorg; de sidder på Jorden og skæler uafbrudt, slagne af Rædsel over dig.
\par 17 De synger en Klagesang om dig og siger til dig: Ak, du gik under, forsvandt fra Havet, du fejrede By, du, som var vældig på Havet, du og dine Borgere, du, der jog Rædsel i alle, som boede der!
\par 18 Nu gribes Strandene af Skælven, den Dag du falder, og Havets Øer forfærdes over din Udgang!
\par 19 Thi så siger den Herre HERREN: Når jeg gør dig til en øde By og lige med affolkede Byer, når jeg fører Verdensdybet over dig, og de vældige Vande skjuler dig,
\par 20 så støder jeg dig ned iblandt dem, der steg ned i Dybet, blandt Fortidens Folk, og lader dig ligge i Underverdenen som ældgamle Ruiner blandt dem, der steg ned i Dybet, for at du aldrig mere skal bebos eller rejse dig i de levendes Land;
\par 21 jeg giver dig hen til brat Undergang, og det skal være ude med dig; selv om der søges efter dig, skal du aldrig i Evighed findes, lyder det fra den Herre HERREN.

\chapter{27}

\par 1 HERRENs Ord kom til mig således:
\par 2 Du, Menneskesøn, istem en Klagesang over Tyrus
\par 3 og sig til Tyrus, som ligger ved Adgangen til Havet og driver Handel med Folkeslagene på de mange fjerne Strande: Så siger den Herre HERREN: Tyrus, du siger: "Fuldendt i Skønhed er jeg!"
\par 4 De bygged dig midt i Havet, fuldendte din Skønhed.
\par 5 De tømred af Senircypresser hver Planke i dig, fra Libanon hented de Cedre at lave din Mast,
\par 6 af Basans højeste Ege skar de dig Årer, de lagde dit Dæk af Fyr fra Kittæernes Strande;
\par 7 dit Sejl var ægyptisk Byssus i broget Væv, dit Tag var Purpur i blåt og rødt fra Elisjas Strande.
\par 8 Zidons og Arvads Folk var Rorkarle for dig, om Bord var de kyndigste i Zarepta, de var dine Styrmænd,
\par 9 de ældste og kyndigste i Gebal, de bøded din Læk.
\par 10 Folk fra Persien, Lydien og Put var i din Hær som Krigsfolk, de ophængte Skjolde og Hjelme i dig; de gav dig Glans.
\par 11 Arvaditerne og deres Hær stod rundt på dine Mure, Gammaditerne på dine Tårne; de ophængte deres Skjolde rundt på dine Mure, de fuldendte din Skønhed.
\par 12 Tarsis var din Handelsven, fordi du havde alskens Gods i Mængde; Sølv, Jern, Tin og Bly gav de dig for dine Varer.
\par 13 Javan, Tubal og Mesjek drev Handel med dig; Trælle og Kobberkar gav de dig i Bytte.
\par 14 Togarmas Hus gav dig Køreheste, Rideheste og Muldyr for dine Varer.
\par 15 Rodosboerne drev Handel med dig, mange fjerne Strande var dine Handelsvenner; Elfenben og Ibenholt bragte de dig som Vederlag.
\par 16 Edom var din Handelsven, fordi du havde Varer i Mængde; Karfunkler, Purpur, brogede Tøjer, fint Linned, Koraller og Rubiner gav de dig for dine Varer.
\par 17 Juda og Israels Land drev Handel med dig; Hvede fra Minnit, Bagværk, Honning, Olie og Mastiksbalsam gav de dig i Bytte.
\par 18 Damaskus var din Handelsven fordi du havde alskens Gods i Mængde; de kom med Vin fra Helbon og Uld fra Zahar.
\par 19 Vedan og Javan gav Sager fra Uzal for dine Varer; smeddet Jern, Hassia og Kalmus fik du i Bytte.
\par 20 Dedan drev Handel med dig med Sadeldækkener til Ridning.
\par 21 Araberne og alle Kedars Fyrster var dine Hardelsvenner; med Lam, Vædre og Bukke handlede de med dig.
\par 22 Sabas og Ramas Handelsfolk drev Handel med dig; den allerfineste Balsam, alle Slags Ædelsten og Guld gav de dig for dine Varer.
\par 23 Karan, Kanne og Eden, Assyrerne og hele Medien drev Handel med dig;
\par 24 de handlede med dig med smukke Klæder, Purpurkapper, brogede Tøjer, farvede Tæpper, tvundet og fastsnoet Reb på dine Markeder;
\par 25 Tarsisskibene tjente dig ved din Omsætning. Du fyldtes, blev såre tung midt ude i Havet.
\par 26 I rum Sø fik de dig ud, dine roende Mænd; da knuste en Østenstorm dig midt ude på Havet;
\par 27 dit Gods, dine Varer, din Vinding, dine Søfolk og Styrmænd, de, der bøded din Læk, dine Handelsfolk, alt dit krigsfolk, som var om Bord, alt Mandskab i din Midte styrter i Havets dyb, den Dag du falder.
\par 28 Markerne skælver ved dine Styrmænds Skrig.
\par 29 Alle, der sidder ved Årer, går da fra Borde, Søfolk og alle Styrmænd går da i Land;
\par 30 de løfter Røsten over dig, klager bittert; på Hovedet kaster de Jord og vælter sig i Støvet,
\par 31 klipper sig skaldet for dig, klæder sig i Sæk, begræder dig, bitre i Hu, med Sjælekvide,
\par 32 istemtner jamrende Klage over dig, klager: Ak, hvor Tyrus er øde midt i Havet!
\par 33 Når din Vinding kom ind fra Havet, mætted du mange Folkeslag; med dit meget Gods og dine Varer gjorde du Jordens konger rige.
\par 34 Nu led du Skibbrud på Havet, på Vandets Dyb, dine Varer og alt dit Mandskab gik under med dig.
\par 35 Over dig gyser alle, som bor på de fjerne Strande, deres Konger er slagne af Angst, deres Ansigt blegner.
\par 36 Deres Kræmmere hånfløjter ad dig, til Rædsel blev du, er borte for evigt.

\chapter{28}

\par 1 HERRENs Ord kom til mig således:
\par 2 Mennesskesøn, sig til Tyruss Fyrste: Så siger den Herre HERREN: Fordi dit Hjerte hovmoder sig og du siger: "Jeg er en Gud, på et Gudesæde sidder jeg midt ude i Havet!" skønt du er et Menneske og ingen Gud, og fordi du føler dig i Hjertet som en Gud;
\par 3 se, du er visere end Daniel, ingen Vismand måler sig med dig;
\par 4 ved din Visdom og indsigt vandt du dig Rigdom og samlede dig Guld og Sølv i dine Skatkamre;
\par 5 ved dit store Handelssnilde øgede du din Rigdom, så dit Hjerte hovmodede sig over den -
\par 6 derfor, så siger den Herre HERREN: Fordi du i dit Hjerte føler dig som en Gud,
\par 7 se, derfor bringer jeg fremmede over dig, de grummeste Folk, og de skal drage deres Sværd mod din skønne Visdom og vanhellige din Glans.
\par 8 De skal styrte dig i Graven, og du skal dø de ihjelslagnes Død i Havets Dyb.
\par 9 Mon du da Ansigt til Ansigt med dem, der dræber dig, vil sige: "Jeg er en Gud!" du, som i deres Hånd, der slår dig ihjel, er et Menneske og ikke en Gud.
\par 10 De uomskårnes Død skal du dø for fremmedes Hånd, så sandt jeg har talet, lyder det fra den Herre HERREN.
\par 11 Og HERRENs Ord kom til mig således:
\par 12 Menneskesøn, istem en Klagesang over kongen af Tyrus og sig til ham: Så siger den Herre HERREN: Du var Indsigtens Segl, fuld af Visdom og fuldkommen i Skønhed.
\par 13 I Eden, Guds Have, var du; alle Slags Ædelsten var din Klædning, Harneol, Topas, Jaspis, Krysolit, Sjoham, Onyks, Safir, Rubin, Smaragd og Guld var på dig i indfattet og indlagt Arbejde; det var til Rede, den Dag du skabtes.
\par 14 Du var en salvet, skærmende Kerub; jeg gjorde dig dertil; på det hellige Gudebjerg var du; du vandrede imellem Guds Sønner.
\par 15 Fuldkommen var du i din Færd, fra den Dag du skabtes, indtil der fandtes Brøde hos dig.
\par 16 Ved din megen Handel fyldte du dit Indre med Uret og forbrød dig; da vanhelligede jeg dig og viste dig bort fra Gudebjerget og tilintetgjorde dig, skærmende Kerub, så du ikke blev mellem Guds Sønner.
\par 17 Dit Hjerte hovmodede sig over din Skønhed, du satte din Visdom til på Grund af din Glans. Jeg, slængte dig til Jorden og overgav dig til Konger, at de skulde nyde Skuet af dig.
\par 18 Med dine mange Misgerninger, ved din uredelige Handel vanhelligede du dine Helligdomme. Da lod jeg Ild bryde løs i din Midte, og den fortærede dig; jeg gjorde dig til Støv på Jorden for alle, som så dig.
\par 19 Alle blandt Folkeslagene, der kendte dig, stivnede af Skræk over dig; du blev en Rædsel, og borte er du for evigt.
\par 20 HERRENs Ord kom til mig således:
\par 21 Menneskesøn, vend dit Ansigt mod Zidon og profeter imod det
\par 22 og sig: Så siger den Herre HERREN: Se, jeg kommer over dig, Zidon, og herliggør mig på dig; og du skal kende, at jeg er HERREN, når jeg holder Dom over dig og viser min Hellighed på dig.
\par 23 Jeg sender Pest over dig og Blod i dine Gader; ihjelslagneMænd skal segne i din Midte for Sværd, der er rettet imod dig fra alle Sider; og du skal kende, at jeg er HERREN.
\par 24 Fremtidig skal der ikke være nogen Tidsel til at såre eller Torn til at stikke Israels Hus blandt alle dets Naboer, som nu håner dem; og de skal kende, at jeg er den Herre HERREN.
\par 25 Så siger den Herre HERREN: Når jeg samler Israels Slægt fra de Folkeslag, de er spredt iblandt, vil jeg hellige mig på dem for Folkenes Øjne, og de skal bo i deres Land, som jeg gav min Tjener Jakob;
\par 26 de skal bo trygt deri, bygge Huse og plante Vingårde, ja bo trygt, medens jeg holder Dom over alle dem, der håner dem fra alle Sider; og de skal kende, at jeg er HERREN deres Gud.

\chapter{29}

\par 1 I det tiende år, på den tolvte dag i den tiende måned kom HERRENs Ord til mig således:
\par 2 Menneskesøn, vend dit Ansigt mod Farao, Ægyptens Konge, og profeter mod ham og hele Ægypten.
\par 3 Tal og sig: Så siger den Herre HERREN: Se, jeg kommer over dig, Farao, Ægyptens Konge, du store drage, som ligger midt i dine Strømme, som siger: "Nilen er min, jeg skabte den selv!"
\par 4 Jeg sætter Kroge i Kæberne på dig og lader dine Strømmes Fisk hænge ved dine Skæl og drager dig op af dine Strømme med alle deres Fisk, som hænger ved dine Skæl.
\par 5 Jeg slænger dig hen i Ørkenen med alle dine Strømes Fisk; på åben Mark skal du falde, ej samles op eller jordes; til Jordens Dyr og Himlens Fugle giver jeg dig som Føde.
\par 6 Og kende skal hver en Ægypter, at jeg er HERREN. Fordi du har været en Rørkæp for Israels Hus -
\par 7 du splintredes, når de greb om dig, og flænged dem hele Hånden; du brast, når de støtted sig til dig, fik hver en Lænd til at vakle -
\par 8 derfor, så siger den Herre HERREN: Se, jeg bringer Sværd over dig og udrydder Folk og Fæ af dig;
\par 9 Ægypten skal blive til Ørk og Øde; og de skal kende, at jeg er HERREN, fordi du sagde: "Nilen er min, jeg skabte den selv!"
\par 10 Se, derfor kommer jeg over dig og dine Strømme og gør Ægypten til Øde og Ørk fra Migdol til Syene og Ætiopiens Grænse.
\par 11 Hverken Mennesker eller Dyr skal sætte deres Fod der, ingen skal færdes der, og det skal ligge hen uden Indbyggere i fyrretyve År.
\par 12 Jeg gør Ægypten til en Ørk blandt øde Lande, og Byerne skal ligge øde hen blandt tilintetgjorte Byer i fyrretyve År; og jeg spreder Ægypterne blandt Folkene og udstrør dem i Landene.
\par 13 Thi så siger den Herre HERREN: Efter fyrretyvears Forløb vil jeg sanke Ægypterne sammen fra de Folkeslag, de er spredt iblandt,
\par 14 og vende Ægyptens Skæbne og føre dem tilbage til Patros, det Land, de stammer fra, og der skal de blive et lille Rige.
\par 15 Det skal blive mindre end de andre Riger og ikke mere svinge sig op over Folkene; jeg gør dem få i Tal, for at de ikke mere skal råde over Folkene.
\par 16 Og fremtidig skal de ikke være Israels Huss Tillid og således minde mig om dets Misgerning, når det slutter sig til dem; og de skal kende, at jeg er den Herre HERREN.
\par 17 I det syv og tyvende År på den første Dag i den første Måned kom HERRENs Ord til mig således:
\par 18 Menneskesøn! Kong Nebukadrezar af Babel har ladet sin Hær udføre et stort Arbejde mod Tyrus; hvert Hoved er skaldet, hver Skulder flået, og hverken han eller Hæren fik Løn af Tyrus for det Arbejde, han udførte imod det.
\par 19 Derfor, så siger den Herre HERREN: Se, jeg giver Kong Nebukadrezar af Babel Ægypten; han skal bortføre dets Rigdom, tage Bytte og røve Rov der; og det skal være hans Hærs Løn;
\par 20 som Vederlag for hans Arbejde giver jeg ham Ægypten, fordi de sled for mig, lyder det fra den Herre HERREN.
\par 21 På den Dag lader jeg et Horn vokse frem for Israels Hus, og dig giver jeg at åbne din Mund iblandt dem; og de skal kende, at jeg er HERREN.

\chapter{30}

\par 1 HERRENs Ord kom til mig således:
\par 2 Menneskesøn, profeter og sig: Så siger den Herre HERREN: Jamrer: Ak, hvilken Dag!
\par 3 Thi nær er Dagen, ja nær er HERRENs Dag; det bliver en Mulmets Dag, Hedningernes Tid.
\par 4 Et Sværd kommer over Ægypten, og Ætiopien gribes af Skælven, når de slagne segner i Ægypten, når dets Rigdom bortføres og dets Grundvolde nedbrydes.
\par 5 Ætioperne, Put og Lud og alt Blandingsfolket, Kub og min Pagts Sønner" skal falde for Sværdet med dem.
\par 6 Så siger HERREN: Alle, som støtter Ægypten, skal falde og dets stolte Herlighed synke sammen; fra Migdol til Syene skal de falde for Sværdet, lyder det fra den Herre HERREN.
\par 7 Det skal lægges øde blandt øde Lande, og Byerne skal ligge hen blandt tilintetgjorte Byer;
\par 8 og de skal kende, at jeg er HERREN, når jeg sætter Ild på Ægypten og alle dets Hjælpere knuses.
\par 9 På hin Dag skal der udgå Sendebud fra mig på Skibe for at indjage det sorgløse Ætiopien Rædsel, og de skal gribes af Skælven over Ægyptens Dag; thi se, den kommer.
\par 10 Så siger den Herre HERREN: Jeg gør Ende på Ægyptens Herlighed ved kong Nebukadrezar af Babel.
\par 11 Han og hans Folk med ham, de grummeste blandt Folkene, skal hentes for at ødelægge Landet; de skal drage deres Sværd mod Ægypten og fylde Landet med slagne.
\par 12 Jeg tørlægger Strømmene, sælger Landet til onde Folk, og ved fremmede ødelægger jeg det med alt, hvad der er deri. Jeg, HERREN, har talet.
\par 13 Så siger den Herre HERREN: Jeg tilintetgør Afgudsbillederne og udrydder Høvdingerne af Nof og Fyrsterne af Ægypten; de skal ikke findes mere; og jeg indjager Ægypten Rædsel.
\par 14 Jeg lægger Patros øde, sætter Ild på Zoan og holder Dom over No.
\par 15 Jeg udøser min Vrede over Sin, Ægyptens Bolværk, og udrydder Nos larmende Hob.
\par 16 Jeg sætter Ild på Ægypten, Syene skal skælve af Angst, der skal brydes Hul på No, og dets Mure skal nedrives.
\par 17 De unge Mænd i On og Pibeset skal falde for Sværdet og Kvinderne vandre i Fangenskab.
\par 18 I Takpankes sortner Dagen, når jeg der sønderbryder Ægyptens Herskerstav, og dets stolte Herlighed får Ende der. Selv skal det skjules af Skyer og dets Småbyer vandre i Fangenskab.
\par 19 Jeg holder Dom over Ægypten; og de skal kende, at jeg er HERREN.
\par 20 I det ellevte År på den syvende Dag i den første Måned kom HERRENs Ord til mig således:
\par 21 Menneskesøn! Ægypterkongen Faraos Arm har jeg brudt; og se den skal ikke forbindes, ikke læges ved at der lægges Bind om den, så den kunde få Kræfter til atter at gribe Sværdet.
\par 22 Derfor, så siger den Herre HERREN: Se, jeg kommer over Farao, Ægyptens Konge, og bryder hans Arme, både den hele og den brudte, og lader Sværdet falde af hans Hånd.
\par 23 Jeg spreder Ægypterne blandt Folkene og udstrør dem i Landene.
\par 24 Men jeg styrker Babels Konges Arme og lægger mit Sværd i hans Hånd; og jeg bryder Faraos Arme, og han skal stønne for ham på såredes Vis.
\par 25 Jeg styrker Babels Konges Arme, men Faraos skal synke; og de skal kende, at jeg er HERREN, når jeg lægger mit Sværd i Babels Konges Hånd og han svinger det imod Ægypten.
\par 26 Og jeg spreder Ægypterne blandt Folkene og udstrør dem i Landene; og de skal kende, at jeg er HERREN.

\chapter{31}

\par 1 I det ellevte År på den første dag i den tredje måned kom Herrens Ord til mig således:
\par 2 Menneskesøn, sig til Farao, Ægyptens Konge, og til hans larmende Hob: Ved hvem kan du lignes i Storhed?
\par 3 Se, du er en Libanonceder med smukke Grene og skyggefuld Krone, høj af Vækst, hvis Top rager op i Skyerne;
\par 4 Vand gav den Vækst, Verdensdybet Højde; sine Strømme lod det flyde rundt om dens Sted og sendte sine Vandløb til hele dens Mark.
\par 5 Derfor blev den større af Vækst end hvert Træ på Marken; mange blev dens Kviste og Grenene lange af megen Væde.
\par 6 Alle Himlens Fugle bygged i dens Grene, under dens Kviste fødte hvert Markens Dyr, i dens Skygge boede al Verdens folk.
\par 7 Den blev stor og dejlig med lange Grene, den stod jo med Roden ved rigeligt Vand.
\par 8 Guds Haves Cedre var ikke dens Lige, ingen Cypres havde Mage til Grene, ingen Platan havde Kviste som den; intet Træ i Guds Have målte sig med den i Skønhed.
\par 9 Jeg gjorde den skøn med dens mange Kviste, så alle Edens Træer i Guds Have misundte den.
\par 10 Derfor, så siger den Herre HERREN: Fordi den blev høj af Vækst og løftede sin Krone op i Skyerne og Hjertet hovmodede sig over dens Højde,
\par 11 derfor overgiver jeg den til en, som er vældig blandt Folkene; han skal gøre med den efter dens Gudløshed og tilintetgøre den.
\par 12 Fremmede, de grummeste blandt Folkene hugger den om og kaster den hen; på Bjerge og i alle Dale falder dens Kviste; dens Grene ligger knækket i alle Landets Kløfter, og alle Jordens Folkeslag går bort fra dens Skygge og lader den ligge.
\par 13 På den faldne Stamme slår alle Himmelens Fugle sig ned, og på Grenene lejrer alle Markens Dyr sig,
\par 14 for at ingen Træer ved Vande skal hovmode sig over deres Vækst og løfte deres Krone op i Skyerne og gøre sig til af deres Højde, ingen Træer, som smager Vand; thi alle er hjemfaldne til Døden og må til Underverdenen, midt iblandt Menneskens Børn, blandt dem, der steg ned i Graven.
\par 15 Så siger den Herre HERREN: Den Dag den farer ned i Dødsriget, lukker jeg Verdensdybet for den og holder dets Strømme tilbage, så de mange Vande standses; jeg lader Libanon sørge over den, og alle Markens Træer vansmægter over den.
\par 16 Ved Drønet af dens Fald bringer jeg Folkene til at bæve, når jeg styrter den ned i Dødsriget til dem, der steg ned i Graven; og nede i Underverdenen trøster alle Edens Træer sig, de ypperste og bedste på Libanon, alle, som smager Vand.
\par 17 Også de farer med den ned i Dødsriget til de sværdslagne, da de som dens Hjælpere har siddet i dens Skygge blandt Folkene.
\par 18 Hvem var din Lige i Herlighed og Størrelse blandt Edens Træer? Og dog styrtes du med Edens Træer ned i Underverdenen; midt iblandt uomskårne skal du ligge hos de sværdslagne. Dette er Farao og al hans larmende Hob, lyder det fra den Herre HERREN.

\chapter{32}

\par 1 I det tolvte År på den første dag i den tolvte måned kom HERRENs Ord til mig således:
\par 2 Menneskesøn, istem en Klagesang over Farao, Ægyptens Konge, og sig til ham: Du Folkenes Løve, det er ude med dig! Du var som en Drage i Havet med prustende Næse, med Fødderne plumred du Vandet, oproded dets Strømme.
\par 3 Så siger den Herre HERREN: Jeg breder mit Garn over dig ved en Sværm af mange Folk, de skal drage dig op i mit Net.
\par 4 Jeg kaster dig på Land og slænger dig hen på Marken, lader alle Himlens Fugle slå sig ned på dig og al Jordens Dyr blive mætte ved dig.
\par 5 Jeg lægger dit Kød på Bjergene, fylder Dalene op med dit Ådsel,
\par 6 vander med dit Udflåd Jorden lige til Bjergene, Kløfterne skal fyldes af dit Blod.
\par 7 Jeg skjuler Himlen, når du slukkes, klæder dens Stjerner i Sorg, jeg skjuler Solen i Skyer, og Månen skinner ej mer.
\par 8 Alle Himmellys klæder jeg i Sorg for dig, hyller dit Land i Mørke, lyder det fra den Herre HERREN.
\par 9 Jeg volder mange Folkeslags Hjerter Kvide, når jeg bringer dine Fanger til Folkene, til Lande du ikke kender;
\par 10 jeg lader mange Folkeslag stivne af Rædsel over dig, og deres Konger skal gyse over dig, når jeg svinger mit Sværd for deres Ansigter; de skal ængstes uafbrudt, hver for sit Liv, den Dag du falder.
\par 11 Thi så siger den Herre HERREN: Babels Konges Sværd skal komme over dig.
\par 12 Jeg styrter din Hob ved Heltes Sværd, de grummeste af alle Folkene; de hærger Ægyptens Pragt, og al dets Hob lægges øde.
\par 13 Jeg udrydder alt dets kvæg ved de mange Vande, Menneskefod skal ej plumre dem mer, ej Dyreklov rode dem op;
\par 14 så lader jeg Vandene klares og Strømmene flyde som Olie - lyder det fra den Herre HERREN -
\par 15 når jeg gør Ægypten til Ørk, så Landet og dets Fylde er øde, når jeg nedhugger alle, som bor der, så de kender, at jeg er HERREN.
\par 16 Dette er en Klagesang, som du skal kvæde; Folkenes Kvinder skal kvæde den; over Ægypten og al dets larmende Hob skal de kvæde den, lyder det fra den Herre HERREN.
\par 17 I det tolvte År på den femtende Dag i..." Måned kom HERRENs Ord til mig således:
\par 18 Menneskesøn, klag over Ægyptens larmende Hob; syng Klagesang over den, du og Folkenes Kvinder! Far ned i Underverdenen blandt dem, der steg ned i Dybet!
\par 19 Er du lifligere end nogen anden? Stig ned og lig blandt de uomskårne!
\par 20 Midt iblandt sværdslagne skal han segne, og al hans larmende Hob skal ligge hos ham.
\par 21 Heltenes Førere skal tale til ham midt i Dødsriget: "Er du mægtigere end nogen anden? Stig ned og lig blandt de uomskårne!"
\par 22 Der er Assur og hele hans Flok rundt om hans Grav; alle er de dræbt, faldet for Sværd;
\par 23 han fik sin Grav i en krog af Dybet, og hans Flok ligger rundt om hans Grav; alle er de dræbt, faldet for Sværd, de, som spredte Rædsel i de levendes Land.
\par 24 Der er Elam med al sin larmende Hob rundt om sin Grav; alle er de dræbt, faldet før Sværd, og de for uomskårne ned i Underverdenen, de, som spredte Rædsel i de levendes Land; nu bærer de deres Skændsel blandt dem, der steg ned i Dybet.
\par 25 Iblandt dræbte fik han et Leje med al sin larmende Hob rundt om sin Grav; alle er de uomskårne, sværdslagne; thi Rædsel for dem bredte sig i de levendes Land; nu bærer de deres Skændsel blandt dem, der steg ned i Dybet; de lagdes blandt dræbte.
\par 26 Der er Mesjek og, Tubal med al deres larmende Hob rundt om deres Grave; alle er de uomskårne, sværdslagne; thi de spredte Rædsel i de levendes Land;
\par 27 de kom ikke til at ligge hos Heltene, Fortidens Kæmper, som for til Dødsriget i deres Rustninger, hvis Sværd blev lagt under deres Hoveder, og hvis Skjolde dækkede deres knogler; thi Rædsel for Heltene rådede i de levendes Land.
\par 28 Også du skal ligge knust imellem de uomskårne, blandt de sværdslagne.
\par 29 Der er Edom med sine Konger og alle sine Fyrster, som fik deres Grave hos de sværdslagne; hos de uomskårne ligger de, hos dem, der steg ned i Dybet.
\par 30 Der er Nordens Herskere alle sammen og alle Zidoniere, som for ned til de dræbte, beskæmmede trods den Rædsel, de spredte ved deres Heltekraft; de ligger uomskårne blandt de sværdslagne, de bærer deres Skændsel blandt dem, der steg ned i Dybet.
\par 31 Dem ser Farao og trøster sig over al sin larmende Hob, lyder det fra HERREN.
\par 32 Thi han spredte Rædsel i de levendes Land, men nu ligger han imellem uomskårne, blandt de sværdslagne, Farao med al sin larmende Hob, lyder det fra den Herre HERREN.

\chapter{33}

\par 1 HERRENs Ord kom til mig således:
\par 2 Menneskesøn, tal til dine Landsmænd og sig: Når jeg fører Sværdet over et Land, og Folket i Landet tager en af sin Midte og gør ham til deres Vægter,
\par 3 og han ser Sværdet komme over Landet og støder i Hornet og advarer Folket,
\par 4 men den, der hører Hornets Klang, ikke lader sig advare, og Sværdet kommer og river ham bort, da kommer hans Blod over hans Hoved.
\par 5 Han hørte Hornets klang uden at lade sig advare, hans Blod kommer over hans Hoved; men den, som har advaret, har reddet sin Sjæl.
\par 6 Men når Vægteren ser Sværdet komme og ikke støder i Hornet, så at Folket ikke advares, og Sværdet kommer og over en af dem bort, så rives han vel bort for sin Misgerning, men hans Blod vil jeg kræve af Vægterens Hånd.
\par 7 Men dig, Menneskesøn, har jeg sat til Vægter for Israels Hus; hører du et Ord af min Mund, skal du advare dem fra mig.
\par 8 Når jeg siger til den gudløse: "Du skal visselig dø!" og du ikke taler for at advare ham mod hans Vej, så skal den gudløse vel dø for sin Misgerning, men hans Blod vil jeg kræve af din Hånd.
\par 9 Advarer du derimod den gudløse mod hans Vej, for at han skal omvende sig fra den, og han ikke omvender sig, så skal han dø for sin Misgerning, men du har reddet din Sjæl.
\par 10 Og du, Menneskesøn, sig til Israels Hus: I siger: "Vore Overtrædelser og Synder tynger os, og vi svinder hen i dem, hvor kan vi da leve?"
\par 11 Sig til dem: Så sandt jeg lever, lyder det fra den Herre HERREN: Jeg har ikke Lyst til den gudløses Død, men til at han omvender sig fra sin Vej, at han må leve! Vend om, vend om fra eders onde Veje Hvorfor vil I dø, Israels Hus?
\par 12 Men du, Menneskesøn, sig til dine Landsmænd: Den retfærdiges Retfærdighed skal ikke redde ham, den Dag han synder, og den gudløses Gudløshed skal ikke fælde ham, den Dag han omvender sig fra sin Gudløshed, og en retfærdig skal ikke blive i Live ved sin Retfærdighed, den Dag han gør Synd.
\par 13 Når jeg siger til den retfærdige: "Du skal visselig leve!" og han stoler på sin Retfærdighed og øver Uret, så skal intet af hans Retfærdighed tilregnes ham, men han skal dø for den Uret, han øver.
\par 14 Og når jeg siger til den gudløse: "Du skal visselig dø!" og han omvender sig fra sin Synd og gør Ret og Skel,
\par 15 idet han giver Pant tilbage, godtgør, hvad han har ranet, og følger Livets Bud uden at øve Uret, så skal han leve og ikke dø;
\par 16 ingen af de Synder, han har gjort, skal tilregnes ham; han har gjort Ret og Skel, visselig skal han leve.
\par 17 Og så siger dine Landsmænd: "Herrens Vej er ikke ret!" Men det er deres Vej, som ikke er ret.
\par 18 Når den retfærdige vender sig fra sin Retfærdighed og øver Uret, skal han dø;
\par 19 og når den gudløse omvender sig fra sin Gudløshed og gør Ret og Skel, skal han leve!
\par 20 Og dog siger I: "Herrens Vej er ikke ret!" Jeg vil dømme eder hver især efter eders Veje, Israels Hus.
\par 21 I vor Landflygtigheds ellevte År på den femte Dag i den tiende Måned kom en Flygtning fra Jerusalem til mig med det Bud: "Byen er indtaget!"
\par 22 Men HERRENs Hånd var kommet over mig, om Aftenen før Flygtningen kom, og han åbnede min Mund, før han kom til mig om Morgenen; så åbnedes min Mund, og jeg var ikke mere stum.
\par 23 HERRENs Ord kom til mig således:
\par 24 Menneskesøn! De, der bor i Ruinerne i Israels Land, siger: "Abraham var kun een og fik dog Landet i eje; vi er mange, og os er Landet givet i Eje!"
\par 25 Sig derfor til dem: Så siger den Herre HERREN: I spiser Kød med Blod i, løfter eders Blik til eders Afgudsbilleder og udgyder Blod, og så vil I have Landet i Eje!
\par 26 I støtter eder til eders Sværd, I øver Vederstyggelighed, I gør hverandres Hustruer urene, og så vil I have Landet i Eje!
\par 27 Således skal du sige til dem: Så siger den Herre HERREN: Så sandt jeg lever: De i Ruinerne skal falde for Sværdet; dem i åbent Land giver jeg de vilde Dyr til Æde, og, de i Klippeborgene og Hulerne skal dø af Pest.
\par 28 Jeg, gør Landet til Ørk og Ødemark, dets stolte Herlighed får Ende, og Israels Bjerge skal ligge øde, så ingen færdes der;
\par 29 og de, skal kende, at Jeg er HERREN, når jeg gør Landet til Ørk og Ødemark for alle de Vederstyggeligheder, de har øvet.
\par 30 Og du, Menneskesøn, se, dine Landsmænd taler om dig langs Murene og i Husdørene, den ene til den anden, hver til sin Broder, og siger: "Kom og hør, hvad det er for et ord, der udgår fra HERREN!"
\par 31 Og de kommer til dig, som var der Opløb, og sætter sig lige over for dig for at høre dine Ord. Men de gør ikke derefter; thi der er Løgn i deres Mund, og deres Hjerte higer efter Vinding.
\par 32 Og se, du er dem som en, der synger en Elskovssang med liflig Røst og er dygtig til at spille; de hører dine Ord, men gør ikke derefter.
\par 33 Men når det kommer - og se, det kommer - skal de kende, at en Profet har været iblandt dem.

\chapter{34}

\par 1 HERRENs Ord kom til mig således:
\par 2 Menneskesøn, profeter mod Israels Hyrder, profeter og sig til dem: Så siger den Herre HERREN: Ve Israels Hyrder, som røgtede sig selv! Skal Hyrderne ikke røgte Hjorden?
\par 3 I fortærede Mælken, med Ulden klædte I eder, de fede Dyr slagtede I, men Hjorden røgtede I ikke;
\par 4 de svage Dyr styrkede I ikke, de syge lægte I ikke, de sårede forbandt I ikke, de adsplittede bragte I ikke tilbage, de vildfarende opsøgte I ikke, men I styrede dem med Hårdhed og Grumhed.
\par 5 Derfor spredtes de, eftersom der ingen Hyrde var, og blev til Æde for alle Markens vilde Dyr; ja, de spredtes.
\par 6 Min Hjord flakkede om på alle Bjerge og på hver en høj Banke, og over hele Jorden spredtes min Hjord, og ingen spurgte eller ledte efter dem.
\par 7 Derfor, I Hyrder, hør HERRENs Ord!
\par 8 Så sandtjeg lever, lyder det fra den Herre HERREN: Fordi min Hjord blev til Rov, fordi min Hjord blev til Æde for alle Markens vilde Dyr, eftersom der ingen Hyrde var, og Hyrderne ikke spurgte efter min Hjord, og fordi Hyrderne røgtede sig selv og ikke min Hjord,
\par 9 derfor, I Hyrder, hør HERRENs Ord!
\par 10 Så siger den Herre HERREN: Se, jeg, kommer over Hyrderne og kræver min Hjord af deres Hånd, og jeg sætter dem fra at vogte min Hjord; Hyrderne skal ikke mere kunne røgte sig selv; jeg redder min Hjord af deres Gab, så den ikke skal tjene dem til Æde.
\par 11 Thi så siger den Herre HERREN: Se, jeg vil selv spørge efter min Hjord og tage mig af den.
\par 12 Som en Hyrde tager sig af sin Hjord på Stormvejrets Dag, således tager jeg mig af min Hjord og redder den fra de Steder, hvorhen de spredtes på Skyernes og Mulmets Dag;
\par 13 jeg fører dem bort fra Folkeslagene, samler dem fra Landene og bringer dem til deres Land, og jeg, røgter dem på Israels Bjerge, i Kløfterne og på alle Landets beboede Steder.
\par 14 På gode Græsgange vil jeg vogte dem, og på Israels Bjerghøjder skal deres Græsmarker være; der skal de lejre sig på gode Græsmarker, og i fede Græsgange skal de græsse på Israels Bjerge.
\par 15 Jeg vil selv røgte min Hjord og selv lade dem lejre sig, lyder det fra den Herre HERREN.
\par 16 De vildfarende Dyr vil jeg opsøge, de adsplittede vil jeg bringe tilbage, de sårede vil jeg forbinde de svage vil jeg styrke, og de fede og kraftige vil jeg vogte; jeg vil røgte dem, som det er ret.
\par 17 Og I, min Hjord! Så siger den Herre HERREN: Se, jeg vil skifte. Ret mellem Får og Får, mellem Vædre og Bukke.
\par 18 Er det eder ikke nok at græsse på den bedste Græsgang, siden I nedtramper, hvad der er levnet af eders Græsgange? Er det eder ikke nok at drikke det klare Vand, siden I med eders Fødder plumrer, hvad der er levnet?
\par 19 Min Hjord må græsse, hvad I har nedtrampet, og drikke, hvad I har plumret med eders Fødder!
\par 20 Derfor, så siger den Herre HERREN: Se, jeg kommer for at skifte Ret mellem de fede og de magre Får.
\par 21 Fordi I med Side og Skulder skubbede alle de svage Dyr bort og stangede dem med eders Horn, til I fik dem drevet ud,
\par 22 derfor vil jeg hjælpe min Hjord, så den ikke mere skal blive til Rov, og skifte Ret mellem Får og Får.
\par 23 Jeg sætter een Hyrde over dem, min Tjener David, og han skal vogte dem; han skal vogte dem, og han skal være deres Hyrde.
\par 24 Og jeg, HERREN, vil være deres Gud, og min Tjener David skal være Fyrste iblandt dem, så sandt jeg, HERREN, har talet.
\par 25 Jeg vil slutte en Fredspagt med dem og udrydde de vilde Dyr at Landet, så de trygt kan bo i Ørkenen og sove i Skovene.
\par 26 Og jeg gør dem og Landet rundt om min Høj til Velsignelse, og jeg sender Regn i rette Tid, mine Byger skal blive til Velsignelse.
\par 27 Markens Træer skal give deres Frugt og Landet sin Afgrøde; trygt skal de bo på deres Jord, og de skal kende, at jeg er HERREN, når jeg bryder Stængerne på deres Åg og frelser dem af deres Hånd, som gjorde dem til Trælle.
\par 28 Ikke mere skal de blive til Rov for Folkene, og Landets vilde Dyr skal ikke æde dem; trygt skal de bo, uden at nogen skræmmer dem.
\par 29 Jeg lader en Fredens Plantning vokse op for dem, og ingen skal rives bort af Hunger i Landet, og de skal ikke mere bære Folkenes Hån.
\par 30 De skal kende, at jeg, HERREN deres Gud, er med dem, og at de er mit Folk, Israels Hus, lyder det fra den Herre HERREN
\par 31 I er min Hjord, I er den Hjord, jeg røgter, og jeg er eders Gud, lyder det fra den Herre HERREN.

\chapter{35}

\par 1 HERRENs Ord kom til mig således:
\par 2 Menneskesøn, vend dit Ansigt mod Seirs Bjergland og profeter imod det
\par 3 og sig til det: Så siger den Herre HERREN: Se, jeg kommer over dig, Seirs Bjergland, og løfter min Hånd imod dig; jeg gør dig til Ørk og Ødemark,
\par 4 dine Byer lægger jeg i Grus. Du selv skal blive til Ørk og kende, at jeg er HERREN.
\par 5 Fordi du nærede evigt Had og i Nødens Stund overgav Israeliterne til Sværdet, da deres Misgerning var fuldmoden,
\par 6 derfor, så sandt jeg lever, lyder det fra den Herre HERREN: Jeg gør dig til Blod, og Blod skal forfølge dig; sandelig, du forbrød dig ved Blod, og Blod skal forfølge dig.
\par 7 Jeg gør Seirs Bjergland til Ørk og Ødemark og udrydder deraf enhver, som kommer og går;
\par 8 jeg fylder dets Bjerge med dræbte; på dine Høje og i dine Dale og Kløfter skal de sværdslagne falde.
\par 9 Jeg gør dig til Ørk for evigt, dine Byer skal ikke bebos; og du skal kende, at jeg er HERREN.
\par 10 Fordi du sagde: "De tvende Folk og de tvende Lande skal tilhøre mig, jeg vil tage dem i Eje!" skønt HERREN var der,
\par 11 derfor, så sandt jeg lever, lyder det fra den Herre HERREN: Jeg vil gøre med dig efter den Vrede og det Nid, du hadefuldt udviste imod dem, og jeg vil give mig til Hende for dig, når jeg dømmer dig;
\par 12 og du skal kende, at jeg er HERREN. Jeg har hørt al den Spot, du udslyngede mod Israels Bjerge: "De er ødelagt, os er de givet til Føde!"
\par 13 Du gjorde dig stor imod mig med din Mund og overfusede mig med Ord; jeg hørte det.
\par 14 Så siger den Herre HERREN: Som det var din Glæde, at mit Land blev Ørk, således vil jeg gøre med dig;
\par 15 som det var din Glæde, at Israels Huss Arvelod blev Ørk, således vil jeg gøre med dig. Ørk skal du blive, Seirs Bjergland og hele Edom med; og de skal kende, at jeg er HERREN.

\chapter{36}

\par 1 Men du menneskesøn, profeter om Israels bjerge og sig: Israels bjerge, hør Herrens ord.
\par 2 Så siger den Herre HERREN: Fordi Fjenden sagde om eder: "Ha! Ørkener for stedse! De er blevet vor Ejendom!"
\par 3 derfor skal du profetere og sige: Så siger den Herre HERREN: Fordi man har higet og snappet efter eder fra alle Sider, for at I skulde tilfalde Resten af Folkene som Ejendom, og fordi I er kommet i Folkemunde,
\par 4 derfor, Israels Bjerge, hør HERRENs Ord: Så siger den Herre HERREN til Bjergene, Højene, kløfterne og Dalene, til de øde Tomter og de forladte Byer, som er blevet til Rov og til Spot for Resten af Folkene rundt om,
\par 5 derfor, så siger den Herre HERREN: Sandelig, i brændende Nidkærhed vil jeg tale mod Resten af Folkene og mod hele Edom, som med al Hjertets Glæde og Sjælens Ringeagt udså sig mit Land til Ejendom for at drive indbyggerne bort og gøre det til Rov.
\par 6 Profeter derfor om Israels Land og sig til Bjergene, Højene, Kløfteme og Dalene: Så siger den Herre HERREN: Se, jeg faler i Nidkærhed og Harme, fordi I bærer Folkenes Hån.
\par 7 Derfor, så siger den Herre HERREN: Jeg løfter min Hånd og sværger: Sandelig, Folkene rundt om eder skal selv bære deres Hån.
\par 8 Og I, Israels Bjerge, skal skyde Grene og bære Frugt for mit Folk Israel, thi de skal snart komme hjem.
\par 9 Thi se, jeg kommer og vender mig til eder, og I skal dyrkes og tilsås;
\par 10 jeg gør Mennesker, alt Israels Hus, mangfoldige på eder, Byerne skal bebos og Ruinerne genopbygges;
\par 11 jeg gør Mennesker og Dyr mangfoldige på eder, ja de skal blive mangfoldige og frugtbare; jeg lader eder bebos som i fordums Tider og gør det bedre for eder end i Fortiden; og I skal kende, at jeg er HERREN.
\par 12 Jeg lader Mennesker, mit Folk Israel, færdes på eder; de skal tage dig i Eje, og du skal være deres Arvelod og ikke mere gøre dem barnløse.
\par 13 Så siger den Herre HERREN: Fordi de siger til dig: "Du er en Menneskeæder, som gør dit Folk barnløst!"
\par 14 derfor skal du ikke mere æde Mennesker eller gøre dit Folk barnløst, lyder det fra den Herre HERREN.
\par 15 Og jeg lader dig ikke mere høre Folkenes Hån, og du skal ikke mere bære Folkeslagenes Spot eller gøre dit Folk harnløst, lyder det fra den Herre HERREN.
\par 16 HERRENs Ord kom til mig således:
\par 17 Menneskesøn! Da Israels Slægt boedei deres Land, gjorde de det urent ved deres Færd og Gerninger; som en Kvindes renhed var deres Færd for mit Åsyn.
\par 18 Så udøste jeg min Vrede over dem for det Blods Skyld, de udgød i Landet, og fordi de gjorde det urent med deres Afgudsbilleder.
\par 19 Jeg spredte dem blandt Folkene, og de strøedes ud i Landene; efter deres Færd og Gerninger dømte jeg dem.
\par 20 Således kom de til Folkene: hvor de kom hen, vanærede de mit hellige Navn, idet man sagde om dem: "De der er HERRENs Folk, og dog måtte de ud af hans Land!"
\par 21 Da ynkedes jeg over mit hellige Navn, som Israels Hus vanærede blandt de Folk, de kom til.
\par 22 Sig derfor til Israels Hus: Så siger den Herre HERREN: Det er ikke for eders Skyld, jeg griber ind, Israels Hus, men for mit hellige Navns Skyld, som I har vanæret blandt de Folk, I kom til.
\par 23 Jeg vil hellige mit store Navn, som vanæres blandt Folkene, idet I har vanæret det iblandt dem; og Folkene skal kende, at jeg er HERREN, lyder det fra den Herre HERREN, når jeg helliger mig på eder for deres Øjne.
\par 24 Jeg vil hente eder fra Folkene, samle eder fra alle Lande og bringe eder til eders Land.
\par 25 Da stænker jeg rent Vand på eder, så I bliver rene; jeg renser eder for al eders Urenhed og alle eders Afgudsbilleder.
\par 26 Jeg giver eder et nyt Hjerte, og en ny Ånd giver jeg i eders Indre; Stenhjertet tager jeg ud af eders Kød og giver eder et Kødhjerte.
\par 27 Jeg giver min Ånd i eders Indre og virker, at I følger mine Vedtægter og tager Vare på at holde mine Lovbud,
\par 28 I skal bo i det Land, jeg gav eders Fædre, og I skal være mit Folk, og jeg vil være eders Gud.
\par 29 Jeg frelser eder fra al eders Urenbed; jeg kalder Kornet frem og mangfoldiggør det og sender ikke mere Hungersnød over eder.
\par 30 Jeg mangfoldiggør Træernes Frugter og Markens Afgrøde, for at I ikke mere skal tage en Hungersnøds Skændsel på eder blandt Folkene.
\par 31 Da skal I ihukomme eders onde Veje og eders Gerninger, som ikke var gode, og ledes ved eder selv over eders Misgerninger og Vederstyggeligheder.
\par 32 Ikke for eders Skyld griber jeg ind, lyder det fra den Herre HERREN, det være eder kundgjort! Skam jer og blues over eders Veje, Israels Hus!
\par 33 Så siger den Herre HERREN: Den Dag jeg renser eder for alle eders Misgerninger, lader jeg Byerne bebos, og Ruinerne skal genopbygges;
\par 34 det ødelagte Land skal dyrkes, i Stedet for at det har været en Ødemark for alles Øjne, som kom forbi.
\par 35 Da skal man sige: "Dette Land, som var ødelagt, er blevet som Edens Have, og Byerne, som var omstyrtet, ødelagt og nedrevet, er befæstet og beboet."
\par 36 Og Folkene, der er tilhage rundt om eder, skal kende, at jeg, HERREN, har opbygget de nedrevne Byer og tilplantet det ødelagte Land; jeg, HERREN, har talet, og jeg fuldbyrder det.
\par 37 Så siger den Herre HERREN: Også dette vil jeg gøre for Israels Hus på deres Bøn: Jeg vil gøre dem mangfoldige som en Hjord af Mennesker;
\par 38 som en Hjord af Offerdyr, som Jerusalems Fårehjorde på dets Højtider skal de omstyrtede Byer blive fulde af Menneskehjorde; og de skal kende, at jeg er HERREN.

\chapter{37}

\par 1 Herrens hånd kom over mig, og han førte mig i ånden ud og satte mig midt i dalen. Den var fuld af Ben;
\par 2 og han førte mig rundt omkring dem, og se, de lå i store Mængder ud over Dalen, og se, de var aldeles tørre.
\par 3 Derpå sagde han til mig: "Menneskesøn! kan disse Ben blive levende?" Jeg svarede: "Herre, HERRE, du ved det!"
\par 4 Så sagde han til mig: Profeter over disse Ben og sig til dem: I tørre Ben, hør HERRENs Ord!
\par 5 Så siger den Herre HERREN til disse Ben: Se, jeg bringer Ånd i eder, så I bliver levende.
\par 6 Jeg lægger Sener om eder, lader Kød vokse frem på eder, overtrækker eder med Hud og indgiver eder Ånd, så I bliver levende; og I skal kende, at jeg er HERREN.
\par 7 Så profeterede jeg, som mig var pålagt, og der hørtes en Lyd, da jeg profeterede, og se, der hørtes Raslen, og Benene nærmede sig hverandre.
\par 8 Og jeg skuede, og se, der kom Sener på dem, Kød voksede frem, og de blev overtrukket med Hud, men der var ingen Ånd i dem.
\par 9 Så sagde han til mig: Profeter og tal til Ånden, profeter, du Menneskesøn, og sig til dem: Så siger den Herre HERREN: Ånd, kom fra de fre Verdenshjørner og blæs på disse dræbte, at de må blive levende!
\par 10 Da profeterede jeg, som han bød mig, og Ånden kom i dem, og de blev levende og rejste sig på deres Fødder, en såre, såre stor Hær.
\par 11 Derpå sagde han til mig: Menneskesøn! Disse Ben er alt Israels Hus. Se, de siger: "Vore Ben er tørre, vort Håb er svundet, det er ude med os!"
\par 12 Profeter derfor og sig til dem: Så siger den Herre HERREN: Se, jeg åbner eders Grave og fører eder ud af dem, mit Folk, og bringer eder til Israels Land;
\par 13 og I skal kende, at jeg er HERREN, når jeg åbner eders Grave og fører eder ud af dem, mit Folk.
\par 14 Jeg indgiver eder min Ånd, så I bliver levende, og jeg bosætter eder i eders Land; og I skal kende, at jeg er HERREN; jeg har talet, og jeg fuldbyrder det, lyder det fra HERREN.
\par 15 HERRENs Ord kom til mig således:
\par 16 Du, Menneskesøn, tag dig et Stykke Træ og skriv derpå: Juda og hans Medbrødre blandt Israeliterne! Tag så et andet Stykke Træ og skriv derpå: Josef Efraims Træ og hans Medbrødre, alt Israels Hus!
\par 17 Føj dem så sammen til eet Stykke, så de bliver eet i din Hånd.
\par 18 Og når så dine Landsmænd siger til dig: "Vil du ikke sige os, hvad du mener dermed?"
\par 19 sig så til dem: Så siger den Herre HERREN: Se, jeg tager Josefs Træ", som var i Efraims Hånd, og Israels Stammer, hans Medbrødre, og føjer dem til Judas Træ og gør dem til eet Stykke og de skal blive eet i Judas Hånd.
\par 20 Og Træstykkerne, du skrev på, skal være i din Hånd, så de kan se dem.
\par 21 Tal så til dem: Så siger den Herre HERREN: Se, jeg henter Israeliterne fra Folkene, til hvilke de vandrede hen, og samler dem alle Vegne fra og bringer dem til deres Land.
\par 22 Jeg gør dem til eet Folk i Landet på Israels Bjerge; og de skal alle have en og samme Konge og ikke mere være to Folk eller delt i to Riger.
\par 23 De skal ikke mere gøre sig urene ved deres Afgudsbilleder og væmmelige Guder eller alle deres Overtrædelser, og jeg vil frelse dem fra alt deres Frafald, hvormed de forsyndede sig, og rense dem, og de skal være mit Folk, og jeg vil være deres Gud.
\par 24 Min Tjener David skal være Konge over dem, og alle skal de have en og samme Hyrde. De skal følge mine Lovbud og holde mine Vedtægter og gøre efter dem.
\par 25 De skal bo i det Land, jeg gav min Tjener Jakob, der hvor deres Fædre boede; de skal bo der til evig Tid, de, deres Børn og Børnebørn; og min Tjener David skal være deres Fyrste evindelig.
\par 26 Jeg slutter en Fredspagt med dem, en evig Pagt skal det være; og jeg gør dem mangfoldige og sætter min Helligdom i deres Midte evindelig;
\par 27 min Bolig skal være over dem; jeg vil være deres Gud, og de skal være mit Folk.
\par 28 Og Folkene skal kende, at jeg er HERREN, som helliger Israel, når min Helligdom bliver i deres Midte evindelig

\chapter{38}

\par 1 HERRENs Ord kom til mig således:
\par 2 Menneskesøn, vend dit Ansigt mod Gog i Magogs Land, Fyrsten over Rosj, Mesjek og Tubal, og profeter imod ham
\par 3 og sig: Så siger den Herre HERREN: Se, jeg kommer over dig, Gog, Fyrste over Rosj, Mesjek og Tubal.
\par 4 Jeg vender dig og sætter Kroge i dine Kæber og trækker dig frem med hele din Hær, Heste og Ryttere, alle i smukke Klæder, en vældig Skare med store og små Skjolde, alle med Sværd i Hånd.
\par 5 Persere, Ætiopere og Putæere er med dem, alle med Skjold og Hjelm,
\par 6 Gomer med alle dets Hobe, Togarmas Hus fra det yderste Nord med alle dets Hobe, mange Folkeslag er med.
\par 7 Rust dig og hold dig rede med hele din Skare, som er samlet om dig, og vær mig rede til Tjeneste.
\par 8 Lang Tid herefter skal der komme Bud efter dig; ved Årenes Fjende skal du overfalde et Land, som atter er unddraget Sværdet, et Folk, som fra mange Folkeslag er sanket sammen på Israels Bjerge, der stadig lå øde hen, et Folk, som er ført bort fra Folkeslagene og nu bor trygt til Hobe.
\par 9 Du skal trække op som et Uvejr og komme som en Sky og oversvømme Landet, du og alle dine Hobe og de mange Folkeslag, som følger dig.
\par 10 Så siger den Herre HERREN: På hin Dag skal en Tanke stige op i dit Hjerte, og du skal oplægge onde Råd
\par 11 og sige: "Jeg vil drage op imod et åbent Land og overfalde fredelige Folk, som bor trygt, som alle bor uden Mure og hverken har Portstænger eller Porte,
\par 12 for at gøre Bytte og røve Rov, lægge Hånd på genopbyggede ruiner og på et Folk, der er indsamlet fra Folkene og vinder sig Fæ og Gods, og som bor på Jordens Navle."
\par 13 Sabæerne og Dedaniterne, Tarsiss Købmænd og alle dets Handelsfolk skal sige til dig: "Kommer du for at gøre Bytte, har du samlet din Skare for at røve Rov, for at bortføre Sølv og Guld, rane Fæ og Gods og gøre et vældigt Bytte?"
\par 14 Profeter derfor, Menneskesøn, og sig til Gog: Så siger den Herre HERREN: Ja, på hin Dag skal du bryde op, medens mit Folk Israel bor trygt,
\par 15 og komme fra din Hjemstavn yderst i Nord, du og de mange Folkeslag, der følger dig, alle til Hest, en stor Skare, en vældig Hær;
\par 16 som en Sky skal du drage op mod mit Folk Israel og oversvømme Landet. I de sidste Dage skal det ske; jeg fører dig imod mit Land; og Folkene skal kende mig, når jeg for deres Øjne helliger mig på dig, Gog.
\par 17 Så siger den Herre HERREN: Er det dig, jeg talede om i gamle dage ved mine Tjenere, Israels Profeter, som profeterede i hine Tider, at jeg vilde bringe dig over dem?
\par 18 Men på hin dag, når Gog overfalder Israels Land, lyder det fra den Herre HERREN, vil jeg give min Vrede Luft.
\par 19 I Nidkærhed, i glødende Vrede udtaler jeg det: Sandelig, på hin Dag skal et vældigt Jordskælv komme over Israels Land;
\par 20 for mit Åsyn skal Havets Fisk, Himmelens Fugle, Markens vilde Dyr og alt kryb på Jorden og alle Mennesker på Jordens Flade skælve, Bjergene skal styrte, Klippevæggene falde og hver Mur synke til Jord.
\par 21 Jeg nedkalder alle Rædsler over ham, lyder det fra den Herre HERREN; den enes Sværd skal rettes mod den anden;
\par 22 jeg går i Rette med ham med Pest og Blod, med Regnskyl og Haglsten; Ild og Svovl lader jeg regne over ham, hans Hobe og de mange Folkeslag, som følger ham.
\par 23 Jeg viser mig stor og hellig og giver mig til Kende for de mange Folks Øjne; og de skal kende, at jeg er HERREN.

\chapter{39}

\par 1 Du, Menneskesøn, Profeter mod Gog og sig: Så siger den Herre HHERREN. Se, Jeg kommer over dig, Gog, Fyrste over Rosj, Mesjek og Tubal!
\par 2 Jeg vender dig, leder dig og fører dig op fra det yderste Nord og bringer dig til Israels Bjerge.
\par 3 Så slår jeg Buen ud af din venstre Hånd og lader Pilene falde ud af din højre.
\par 4 På Israels Bjerge skal du falde, du og alle dine Hobe og Folkeslagene, der følger dig; jeg giver dig til Føde for alle Hånde Rovfugle og Markens vilde Dyr.
\par 5 I åben Mark skal du falde, så sandt jeg har talet, lyder det fra den Herre HERREN.
\par 6 Og jeg sætter Ild på Magog og på de fjerne Strandes trygge Indbyggere; og de skal kende, at jeg er HERREN.
\par 7 Mit hellige Navn kundgør jeg midt i mit Folk Israel, og jeg vil ikke mere vanhellige mit hellige Navn; og Folkene skal kende, at jeg er HERREN, den Hellige i Israel.
\par 8 Se, det kommer, det skal ske, lyder det fra den Herre HERREN; det er Dagen, jeg har talet om.
\par 9 Så skal Indbyggerne i Israels Byer gå ud og gøre Ild på og tænde op med Rustninger, små og store Skjolde, Buer, Pile, Håndstave og Spyd; og de skal bruge det til at gøre Ild med i syv År.
\par 10 De skal ikke hente Træ i Marken eller hugge Brænde i Skovene, men gøre Ild på med Rustningerne. De skal plyndre dem, de plyndredes af, og hærge dem, de hærgedes af, lyder det fra den Herre HERREN.
\par 11 Og på hin Dag giver jeg Gog et Gravsted i Israel, Vandringsmændenes dal østen for Havet, og den skal spærre Vejen for Vandringsmænd; der skal de jorde Gog og hele hans larmende Hob og kalde Stedet: Gogs larmende Hobs dal.
\par 12 Israels Hus skal have hele syv Måneder til at jorde dem og således rense Landet.
\par 13 Alt folket i Landet skal jorde dem, og det skal tjene til deres Ros, på den Dag jeg herliggør mig, lyder det fra den Herre HERREN.
\par 14 Man skal udvælge fast Mandskab til at drage Landet rundt og søge efter dem, der er blevet tilbage ud over Landet, for at rense det; når syv Måneder er gået, skal de skride til at søge;
\par 15 og når de vandrer Landet rundt og en får Øje på Menneskeknogler, skal han sætte et Mærke derved, for at Graverne kan jorde dem i Gogs larmende Hobs Dal;
\par 16 også skal en By have Navnet Hamona". Således skal de rense Landet.
\par 17 Og du, Menneskesøn! Så siger den Herre HERREN: Sig til alle Fugle og alle Markens vilde Dyr: Saml eder og kom hid, kom sammen alle Vegne fra til mit Slagtoffer, som jeg slagter for eder, et vældigt Slagtoffer på Israels Bjerge; I skal æde Kød og drikke Blod!
\par 18 Kød af Helte skal I æde, Blod af Jordens Fyrster skal I drikke, Vædre, Får, Bukke og Tyre, alle fedet i Basan.
\par 19 I skal æde eder mætte i Fedt og drikke eder drukne i Blod af mit Slagtoffer, som jeg slagter for eder.
\par 20 I skal mætte eder ved mit Bord med Køreheste og Rytterheste, Helte og alle Hånde Krigsfolk, lyder det fra den Herre HERREN.
\par 21 Jeg åbenbarer min Herlighed blandt Folkene, og alle Folkene skal skue den Dom, jeg fuldbyrder, og min Hånd, som jeg lægger på dem.
\par 22 Israels Hus skal kende, at jeg, HERREN, er deres Gud fra hin Dag og fremdeles;
\par 23 og Folkene skal kende, at Israels Hus vandrede i Landflygtighed for deres Misgerningers Skyld, fordi de var troløse imod mig, så jeg skjulte mit Åsyn for dem og gav dem i deres Fjenders Hånd, hvorfor de alle faldt for Sværdet.
\par 24 Efter deres Urenhed og deres Overtrædelser handlede jeg med dem og skjulte mit Åsyn for dem.
\par 25 Derfor, så siger den Herre HERREN; Nu vil jeg vende Jakobs Skæbne, forbarme mig over alt Israels Hus og være nidkær for mit hellige Navn;
\par 26 og de skal glemme deres Skændsel og al den Troløshed, de viste mig, når de bor trygt i deres Land, uden at nogen skræmmer,
\par 27 når jeg fører dem tilbage fra Folkeslagene, samler dem fra deres Fjenders Lande og helliger mig på dem for mange Folks Øjne.
\par 28 Og de skal kende, at jeg er HERREN deres Gud, når jeg efter at have ført dem i Landflygtighed blandt Folkene samler dem i deres Land uden at lade nogen af dem blive tilbage derude
\par 29 og ikke mere skjuler mit Åsyn for dem, da jeg har udgydt min Ånd over Israels Hus, lyder det fra den Herre HERREN.

\chapter{40}

\par 1 I det fem og tyvende år efter at vi var ført i landflygtighed, ved nytårstide, på den tiende dag i Måneden i det fjortende År efter Byens indtagelse, netop på den Dag kom HERRENs Hånd over mig, og han førte mig
\par 2 i Guds Syner til Israels Land og satte mig på et såre højt Bjerg, og på det var der bygget noget som en By mod Syd;
\par 3 og da han havde ført mig derhen, se, da var der en Mand som Kobber at se til med en Hørgarnssnor og en Målestang i Hånden, og han stod ved Porten.
\par 4 Manden sagde til mig: "Menneskesøn, se med dine Øjne, hør med dine Ører og læg vel Mærke til alt, hvad jeg viser dig; thi du er ført hid, for at jeg skal vise dig det.
\par 5 Og se, der var en Mur uden om Templet til alle Sider.
\par 6 Så gik han op ad syv Trappetrin ind i den Port, hvis Forside vendte mod Øst; og han målte Portens Tærskel til eet Mål i Bredden,
\par 7 hvert af Portens Siderum ligeledes til eet Mål i Længden og eet i Bredden, Murpillerne mellem Siderummene til fem Alen og Tærskelen ved Portens Forhal på den Side,
\par 8 der vendte indad i Porten, til eet Mål.
\par 9 Og han målte Portens Forhal til otte Alen og dens Murpiller til to; Portens Forhal lå på indersiden.
\par 10 Portens Siderum, tre på hver Side, lå over for hverandre; de var lige store alle tre; også Murpillerne på begge Sider var lige store.
\par 11 Så målte han Portindgangens Bredde til ti Alen og Portgangens til tretten.
\par 12 Foran Siderummene var der på begge Sider afspærrede Pladser på een Alen, og selve Siderummene på begge Sider var seks Alen.
\par 13 Så målte han Porten fra Indervæggen i et Siderum til Indervæggen i Siderummet lige overfor til en Bredde af fem og tyve Alen, Dør over for Dør.
\par 14 Så målte han Forhallen til tyve Alen; og Forgården omgav Portens Forhal".
\par 15 Fra Portens Forside udad til Portforhallens Forside indad var der halvtredsindstyve Alen.
\par 16 Porten havde på begge Sider Gittervinduer, som udvidede sig indad i Siderummene og deres Murpiller; ligeledes havde Forhallen på alle Sider Vinduer, som udvidede sig indad. På Murpillerne til begge Sider var der Palmer.
\par 17 Derpå førte han mig ind i den ydre Forgård. Og se, der var Kamre, og Forgården rundt var der et stenlagt Stykke; der var tredive Kamre på Stenlægningen.
\par 18 Det stenlagte Stykke stødte op til Portenes Sidemure, lige så bredt som Portene var lange; det var den nedre Stenlægning.
\par 19 Han målte Forgårdens Bredde fra den nedre Ports indre Forside til den indre Ports ydre Forside til hundrede Alen. Og han førte mig mod Nord,
\par 20 og se, der var en Port, som vendte mod Nord, i den ydre Forgård, og han målte dens Længde og Bredde.
\par 21 Den havde tre Siderum til hver Side, og Murpillerne og Forhallen havde samme Mål som i den første Port; den var halvtredsindstyve Alen lang og fem og tyve Alen bred.
\par 22 Vinduer, Forhal og Palmer havde samme Mål som i den Port, hvis Forside vendte mod Øst; ad syv Trin steg man op dertil, og Forhallen lå inderst inde.
\par 23 En Port til den indre Forgård lå over for Nordporten, ligesom Forholdet var ved Østporten; og han målte fra Port til Port hundrede Alen.
\par 24 Så førte han mig mod Syd og se, der var også en Port mod Syd, og han målte dens Murpiller og Forhal; de havde samme Mål som de andre.
\par 25 Porten og dens Forhal havde Vinduer af samme Slags som de andre. Den var halvtredsindstyve Alen lang og fem og tyve Alen bred;
\par 26 syv Trin førte op til den; Forhallen lå inderst inde, og der var Palmer på Murpillerne til begge Sider.
\par 27 Endelig var der en Port til den indre Forgård over for Sydporten: han målte hundrede Alen fra Port til Port.
\par 28 Derpå førte han mig til den indre Forgård gennem Sydporten, og den målte han; den havde samme Størrelse som de andre,
\par 29 og dens Siderum, Murpiller og Forhal havde samme Størrelse som de andre; Porten og dens Forhal havde Vinduer rundt om. Den var halvtredsindstyve Alen lang og fem og tyve Alen bred.
\par 30 (Der var Forhaller rundt om, fem og tyve Alen lange og fem Alenbrede.
\par 31 Forhallen vendte ud mod den ydre Forgård med Palmer på Murpillerne, og otte Trin dannede dens Opgang.
\par 32 Så førte han mig til Østporten og målte denne Port; den havde samme Størrelse som de andre,
\par 33 og Siderum, Murpiller og Forgård havde samme Størrelse som de andre; Porten og dens Forhal havde Vinduer rundt om. Den var halvtredsindstyve Alen lang og fem og tyve Alen bred.
\par 34 Forhallen vendte ud mod den ydre Forgård med Palmer på Murpillerne til begge Sider, og otte Trin dannede dens Opgang.
\par 35 Så førte han mig til Nordporten og målte den; den havde samme Størrelse som de andre,
\par 36 ligeledes Siderum, Murpiller og Forhal; Porten og dens Forhal havde Vinduer rundt om. Den var halvtredsindstyve Alen lang og fem og tyve Alen bred.
\par 37 Forhallen vendte ud mod den ydre Forgård med Palmer på Murpillerne til begge Sider, og otte Trin dannede dens Opgang.
\par 38 Derpå vendte han sig til det Indre, idet han førte mig til Østporten; der skyllede man Brændofferet
\par 39 Og i Portens Forhal stod to Borde på den ene Side og to på den anden til at slagte Brændofferet, Syndofferet og Skyldofferet på.
\par 40 Også ved det ydre Hjørne, mod Nord når man steg op i Portindgangen, stod to Borde og ved Portforhallens andet Hjørne andre to,
\par 41 fire Borde på hver Side ved Portens Hjørner, i alt otte. På dem slagtede man Slagtofferet.
\par 42 Og, til Brændofferet stod der tre Kvaderstensborde, halvanden Alen lange, halvanden Alen brede og en Alen høje; på dem lagde man de Redskaber, med hvilke man slagtede Brændofferet og Slagtofferet.
\par 43 De havde hele Vejen rundt en Rand på en Håndsbred, der vendte indad; og oven over Bordene var der Tage til Værn mod Regn og Sol. Derpå førte han mig atter
\par 44 til den indre Forgård, og se, der var to Kamre, et ved Nordportens Hjørne med Forsiden mod Syd og et andet ved Sydportens Hjørne med Forsiden mod Nord.
\par 45 Og han sagde til mig: "Kammeret her, hvis Forside vender mod Syd, er for Præsterne, der tager Vare på, hvad der er at varetage i Templet,
\par 46 og Kammeret der, hvis Forside vender mod Nord, er for Præsterne, der tager Vare på, hvad der er at varetage ved Alteret. Det er Zadoks Sønoer, som alene af Levis Sønner må nærme sig HERREN for at tjene ham."
\par 47 Så målte han Forgården; den var hundrede Alen lang og hundrede Alen bred i Firkant; og Alteret stod foran Templet.
\par 48 Derpå førte han mig til Templets Forhal og målte Forhallens Piller" ; de var fem Alen brede på begge Sider; Porten var fjorten Alen bred og dens Sidevægge tre Alen på begge Sider,
\par 49 og Forhallen var tyve Alen lang og tolv Alen bred. Ad ti Trin steg man op til den; og der stod Søjler op ad Pillerne, een på hver Side.

\chapter{41}

\par 1 Derpå førte han mig til det hellige og målte pillerne, de var seks Al brede på begge Sider;
\par 2 Indgangen var ti Alen bred, dens Sidevægge fem Alen til begge Sider; og han målte dets Længde til fyrretyve Alen og Bredden til tyve.
\par 3 Derpå gik han ind i Inderhallen og målte indgangens Piller; de var to Alen, og Indgangen var seks Alen bred og Sidevæggene syv Alen brede til begge Sider.
\par 4 Og han målte dets Længde til tyve Alen og Bredden til tyve ud for Tempelrummet. Og han sagde til mig: "Dette er det Allerhelligste."
\par 5 Derpå målte han Templets Mur; den var seks Alen bred; og Tilbygningen var fire Alen bred Templet rundt.
\par 6 Tilbygningen lå Rum ved Rum, tre Aum oven på hverandre tredive Gange, og der var Fremspring, så Bjælkerne ikke greb ind i Templets Mur.
\par 7 Således var Tilbygningens Rum bredere og bredere opad, efter som Tempelmuren var trukket tilbage opad, Templet rundt. Fra det nederste Stokværk steg man op til det mellemste, og derfra op til det øverste.
\par 8 Og jeg så ved Templet en ophøjet brolagt Plads hele Vejen rundt. Tilbygningens Grundmure var et fuldt Mål høje, seks Alen til Kanten.
\par 9 Tilbygningens Ydermur var fem Alen bred. Der var en åben Plads langs Templets Tilbygning.
\par 10 En afspærret Plads, tyve Alen bred, omgav Templet på alle Sider.
\par 11 Tilbygningens Døre førte ud til den åbne Plads, en Dør mod Nord og en anden mod Syd; og den åbne Plads var fem Alen bred på alle Sider.
\par 12 Den Bygning, som lå ved den afspærrede Plads imod Vest, var halvfjerdsindstyve Alen bred, dens Mur var fem Alen tyk til alle Sider, og den var halvfemsindstyve Alen lang.
\par 13 Han målte Templet; det var hundrede Alen langt; den afspærrede Plads tillige med Bagbygningen og dens Mure var hundrede Alen lang,
\par 14 og Templets Forside tillige med den afspærrede Plads mod Øst var hundrede Alen bred.
\par 15 Og han målte Længden af Bagbygningen langs den afspærrede Plads, som lå bag den; den var hundrede Alen. Det Hellige, Inderhallen og den ydre Forhal
\par 16 var træklædt. Vinduer, som udvidede sig indad, gav Lys rundt om i alle tre Rum, og Væggene derinde var klædt med Træ rundt om fra Gulv til Vinduer,
\par 17 og fra Indgangens Sidevægge til det indre Rum var der Væggen rundt
\par 18 udskåret Arbejde, Keruber og Palmer, en Palme mellem to Keruber; Keruberne havde to Ansigter;
\par 19 Menneskeansigtet vendte mod Palmen på den ene Side og Løveansigtet mod Palmen på den anden Side; således var der gjort hele Templet rundt.
\par 20 Fra Gulv til Vinduer var der fremstillet Keruber og Palmer på det Helliges Væg.
\par 21 Ved Indgangen til det Hellige var der firkantede Dørstolper. Foran Helligdommen var der noget, der så ud som
\par 22 et Træalter, tre Alen højt, to Alen langt og to Alen bredt; det havde Hjørner, og dets Fodstykke og Vægge var af Træ. Og han sagde til mig: "Dette er Bordet, som står for HERRENs Åsyn."
\par 23 Det Hellige havde to Dørfløje;
\par 24 ligeledes havde Helligdommen to Dørfløje; hver Fløj var to bevægelige Dørflader, to på hver Fløj.
\par 25 Og på dem var der fretillet Keruber og Palmer ligesom på Væggene. Der var et Trætag uden for Forhallen.
\par 26 Der var gitrede Vinduer og Palmer på Forhallens Sidevægge til begge Sider..."

\chapter{42}

\par 1 Derpå førte han mig ud i den indre forgård i nordlig retning, og han førte mig til Kamrene, som lå ud imod den afspærrede Plads og Bagbygningen, nogle på den ene Side, andre på den anden.
\par 2 Længden var hundrede Alen og Bredden halvtredsindstyve.
\par 3 Over for Portene, som hørte til den indre Forgård, og over for Stenbroen, som hørte til den ydre Forgård, var der Gang over for Gang i tre Stokværk.
\par 4 Foran Kamrene var der en Forgang, ti Alen bred og hundrede Alen lang, og deres Døre vendte mod Nord.
\par 5 De øvre Kamre var de snævreste, thi Gangen tog noget af deres Plads, så at de var mindre end de nederste og mellemste.
\par 6 Thi de havde tre Stokværk og ingen Søjler svarende til den ydre Forgårds; derfor var de øverste snævrere end de nederste og mellemste.
\par 7 Der løb en Mur udenfor langs Kamrene i Retning af den ydre Forgård; foran Kamrene var dens Længde halvtredsindstyve Alen;
\par 8 thi Kamrene, som lå ved den ydre Forgård, havde en samlet Længde af halvtredsindstyve Alen, og de lå over for hine, i alt hundrede Alen.
\par 9 Neden for disse Kamre var indgangen på Østsiden, når man gik ind i dem fra den ydre Forgård,
\par 10 ved Begyndelsen af den ydre Mur. Derpå førte han mig mod Syd; og der var også Kamre ud imod den afspærrede Plads og Bagbygningen
\par 11 med en Gang foran; de så ud som Kamrene mod Nord; havde samme Længde og Bredde, og alle Udgange var her som hist, ligesom de var indrettet på samme Måde.
\par 12 Men deres Døre vendte mod Syd..."
\par 13 Og han sagde til mig: "Kamrene mod Nord og Syd, som ligger ud mod den afspærrede Plads, er de hellige Kamre, hvor Præsterne, der nærmer sig HERREN, skal spise det højhellige; der skal de gemme det højhellige, Afgrødeofrene, Syndofrene og Skyldofrene, thi Stedet er helligt;
\par 14 og når Præsterne træder ind - de må ikke fra Helligdommen træde ud i den ydre Forgård - skal de der nedlægge deres Klæder, som de gør Tjeneste i, da de er hellige, og iføre sig andre Klæder; da først må de nærme sig det, der hører Folket til."
\par 15 Da han var til Ende med at udmåle Templets Indre, førte han mig hen til den Port, hvis Forside vendte mod Øst, og målte til alle Sider;
\par 16 han målte med Målestangen Østsiden; den var efter Målestangen 500 Alen; så vendte han sig og
\par 17 målte med Målestangen Nordsiden til 500 Alen;
\par 18 så vendte han sig og målte med Målestangen Sydsiden til 500 Alen;
\par 19 og han vendte sig og målte med Målestangen Vestsiden til 500 Alen.
\par 20 Til alle fire Sider målte han Pladsen; og der var en Mur rundt om, 500 Alen lang og 500 Alen bred, til at sætte Skel mellem det, som er helligt, og det, som ikke er helligt.

\chapter{43}

\par 1 Derpå førte han mig hen til Østporten.
\par 2 Og se Israels Guds Herlighed kom østerfra, og det lød som mange Vandes Brus, og Jorden lyste af hans Herlighed.
\par 3 Synet var som det, jeg havde set, da han kom for at ødelægge Byen, og Vognen så ud som den, jeg havde set ved Floden Kebar. Da faldt jeg på mit Ansigt.
\par 4 Og HERRENs Herlighed drog ind i Templet gennem den Port, hvis Forside vendte mod Øst.
\par 5 Men Ånden løftede mig op og bragte mig ind i den indre Forgård, og se, HERRENs Herlighed fyldte Templet.
\par 6 Og jeg hørte en tale til mig ud fra Templet, medens Manden stod ved Siden af mig,
\par 7 og han sagde: Menneskesøn! Her er min Trones og mine Fodsålers Sted, hvor jeg vil bo midt iblandt Israeliterne til evig Tid. Israels Hus skal ikke mere vanhellige mit hellige Navn, hverken de eller deres konger, med deres Bolen eller deres kongers Lig,
\par 8 de, som satte deres Tærskel lige ved min og deres Dørstolper lige ved mine, kun med en Mur imellem mig og dem, og vanhelligede mit hellige Navn ved de Vederstyggeligheder, de øvede, så jeg måtte tilintetgøre dem i min Vrede.
\par 9 Nu skal de fri mig for deres Bolen og deres Kongers Lig, så jeg kan bo iblandt dem til evig Tid.
\par 10 Men du, Menneskesøn, giv Israels Hus en Beskrivelse af Templet, dets Udseende og Form, at de må skamme sig over deres Misgerninger.
\par 11 Og dersom de skammer sig over alt, hvad de har gjort, så kundgør dem Templets Omrids og Indretning, dets Udgange og Indgange, et helt Billede deraf; ligeledes alle Vedtægter og Love derom; og skriv det op for deres Øjne, at de må mærke sig Billedet i sin Helhed og alle Vedtægterne og holde dem.
\par 12 Dette er Loven om Templet: På Bjergets Tinde skal alt dets Område til alle Sider være højhelligt; se, det er Loven om Templet.
\par 13 Følgende er Alterets Mål i Alen, en Alen en Håndsbred længere end sædvanlig: Foden var en Alen høj og en Alen bred, Kantlisten Randen rundt et Spand høj. Om Alterets Højde gælder følgende:
\par 14 Fra Foden underneden op til det nederste Fremspring to Alen med en Alens Bredde; og fra det lille Fremspring til det store fire Alen med en Alens Bredde.
\par 15 Ildstedet var fire Alen højt, og fra Ildstedet ragede fire Horn i Vejret.
\par 16 Ildstedet var tolv Alen langt og tolv Alen bredt, så det dannede en ligesidet Firkant.
\par 17 Det store Fremspring var fjorten Alen langt og fjorten Alen bredt på alle fire Sider; det lille Fremspring seksten Alen langt og seksten Alen bredt på alle fire Sider; Kantlisten rundt om en halv Alen bred og Foden en Alen bred rundt om. Trappen var på Østsiden.
\par 18 Og han sagde til mig: Menneskesøn! Så siger den Herre HERREN: Følgende er Vedtægteme om Alteret, på den Dag det bygges til at ofre Brændofre og sprænge Blod på:
\par 19 Så lyder det fra den Herre HERREN: Levitpræsterne, som nedstammer fra Zadok og må nærme sig mig for at gøre Tjeneste for mig, skal du give en ung Tyr til Syndoffer;
\par 20 og du skal tage noget af dens Blod og stryge det på Alterets fire Horn, på Fremspringets fire Hjørner og på Kantlisten rundt om og således rense det for Synd og fuldbyrde Soningen for det.
\par 21 Og du skal tage Syndoffertyren og brænde den ved Tempelvagten uden for Helligdommen.
\par 22 Næste Dag skal du bringe en lydefri Gedebuk som Syndoffer, og de skal rense Alteret for Synd, ligesom de rensede det med Tyren.
\par 23 Og når du er til Ende med at rense det for Synd, skal du bringe en lydefri ung Tyr og en lydefri Væder af Småkvæget;
\par 24 du skal bringe dem for HERRENs Åsyn, og Præsterne skal strø Salt på dem og ofre dem som Brændoffer for HERREN.
\par 25 Syv Dage skal du daglig ofre en Syndofferbuk, og man skal ofre en ung Tyr og en Væder af Småkvæget, lydefri Dyr;
\par 26 i syv Dage skal man fuldbyrde Soningen for Alteret og rense det og indvie det.
\par 27 Således skal man bære sig ad i disse Dage. Og på den ottende Dag og siden hen skal Præsterne ofre eders Brændofre og Takofre på Alteret; og jeg vil have Behag i eder, lyder det fra den Herre HERREN.

\chapter{44}

\par 1 Derpå førte han mig tilbage ad helligdommens ydre østport til, og den var lukket."
\par 2 Og HERREN sagde til mig: Denne Port skal være lukket og må ikke åbnes! Ingen må gå ind derigennem, thi igennem den drog HERREN, Israels Gud, ind; derfor skal den være lukket.
\par 3 Kun Fyrsten må sidde i den og holde Måltid for HERRENs Åsyn; men han skal gå ind igennem Portforhallens Dør og samme Vej ud.
\par 4 Derpå førte han mig i Retning af Nordporten til Pladsen foran Templet, og jeg skuede, og se, HERRENs Herlighed fyldte HERRENs Hus, og jeg faldt på mit Ansigt.
\par 5 Da sagde HERREN til mig: Menneskesøn, mærk dig og se med dine Øjne og hør med dine Ører alt, hvad jeg taler til dig med Hensyn til alle Vedtægter og Love om HERRENs Hus, og læg vel Mærke til, hvad der gælder om Adgang til Templet gennem en hvilken som helst af Helligdommens Udgange.
\par 6 Og sig til Israels Hus, den genstridige Slægt: Så siger den Herre HERREN: Lad det nu være nok med alle eders Vederstyggeligheder, Israels Hus,
\par 7 at I lod fremmede med uomskårne Hjerter og uomskåret Kød komme ind i min Helligdommen, for at de skulde være der og vanhellige mit Hus, når I frembar min Mad, Fedt og Blod og således brød min Pagt ved alle eders Vederstyggeligheder.
\par 8 I tog ikke Vare på, hvad der var at varetage ved mine hellige Ting, men overlod de fremmede at tage Vare på, hvad der var at varetage i min Helligdom.
\par 9 Derfor, så siger den Herre HERREN: Ingen fremmed med uomskåret Hjerte og uomskåret Kød må komme i min Helligdom, ikke een af de fremmede, som lever blandt Israels Børn.
\par 10 Men de Leviter, som fjernede sig fra mig, da Israel for vild, idet de for vild fra mig og holdt sig til deres Afgudsbilleder, de skal bære deres Misgerning.
\par 11 De skal i min Helligdom gøre Vagttjeneste ved Tempelportene og udføre Arbejdet i Templet, idet de skal slagte Brændofrene og Slagtofrene for Folket og stå dem til Tjeneste og gå dem til Hånde.
\par 12 Fordi de gik dem til Hånde over for deres Afgudsbilleder og således blev Årsag til Skyld for Israels Hus, derfor løfter jeg min Hånd imod dem, lyder det fra den Herre HERREN, på at de skal bære deres Misgerning.
\par 13 De må ikke nærme sig mig for at gøre Præstetjeneste for mig, ej heller må de nærme sig nogen af mine hellige Ting, det højhellige, men de skal bære deres Skændsel og de Vederstyggeligheder, de øvede.
\par 14 Jeg sætter dem til at tage Vare på, hvad der er at varetage i Templet ved alt Arbejde der, ved alt, hvad der er at gøre derinde.
\par 15 Men Levitpræsterne, Zadoks Efterkommere, som tog Vare på, hvad der var at varetage i min Helligdom, dengang Israeliterne for vild fra mig, skal nærme sig mig for at gå mig til Hånde og være mig til Tjeneste og ofre mig Fedt og Blod, lyder det fra den Herre HERREN.
\par 16 De skal gå ind i min Helligdom og nærme sig mit Bord for at gå mig til Hånde og tage Vare på, hvad jeg vil have varetaget.
\par 17 Og når de går ind ad den indre Forgårds Port, skal de være iført Linnedklæder; de må ikke have uld på Kroppen, når de gør Tjeneste i den indre Forgårds Porte eller længere inde.
\par 18 De skal bære Linnedhuer på Hovedet og Linnedbenklæder om Lænderne; de må ikke omgjorde sig med noget, som fremkalder Sved.
\par 19 Og når de går ud i den ydre Forgård til Folket, skal de afføre sig de Klæder, i hvilke de gør Tjeneste, gemme dem i Helligdommens kamre og iføre sig andre Klæder, at de ikke skal gøre Folket helligt med deres Klæder.
\par 20 Hovedet må de ikke rage, dog heller ikke lade Håret vokse frit, men de skal klippe deres Hår.
\par 21 Vin må ingen Præst drikke, når han går ind i den indre Forgård.
\par 22 Enke eller fraskilt må de ikke tage til Ægte, men kun Jomfruer af Israels Hus; dog må de ægte Enken efter en Præst.
\par 23 De skal lære mit Folk at skelne mellem det, som er helligt, og det, som ikke er helligt, og undervise dem i Forskellen mellem rent og urent.
\par 24 Ved Retstrætter skal de optræde som Dommere; efter mine Lovbud skal de dømme. De skal overholde mine Love og Vedtægter på alle mine Højtidsdage og helligholde mine Sabbater.
\par 25 Et Lig må de ikke komme nær, at de ikke skal blive urene derved; kun ved Fader, Moder, Søn, Datter, Broder eller ugift Søster må de gøre sig urene;
\par 26 og efter at være blevet uren skal han tælle syv Dage frem, så er han atter ren;
\par 27 den Dag han atter går ind i Helligdommen, i den indre Forgård, for at gøre Tjeneste i Helligdommen, skal han frembære et Syndoffer, lyder det fra den Herre HERREN.
\par 28 De skal ingen Arvelod have; jeg er deres Arvelod..Og Ejendom i Israel må I ikke give dem; jeg er deres Ejendom.
\par 29 Afgrødeofferet, Syndofferet og Skyldofferet skal de spise, og alt, hvad der er lagt Band på i Israel, skal tilfalde dem.
\par 30 Det bedste af al Førstegrøde af enhver Art, alle Offerydelser af enhver Art, alt, hvad I måtte yde, skal tilfalde Præsterne, og Førstegrøden af eders Grovmel skal I give Præsten for at nedkalde Velsignelse over eders Huse.
\par 31 Intet Ådsel og intet, som er sønderrevet, være sig Fugl eller firføddet Dyr, må Præsterne spise.

\chapter{45}

\par 1 Når I udskifter landet ved lodkastning, skal i yde Herren en Offerydelse, en hellig Del af Landet, 25 000 Alen lang og 20 000 Alen bred; hellig skal den være i hele sin Udstrækning."
\par 2 Deraf skal til Helligdommen fratages et firkantet Stykke på 500 Alen til alle Sider, omgivet af en åben Plads på 50 Alen.
\par 3 Af denne Strækning skal du afmåle et Stykke på 25000 Alens Længde og 10000 Alens Bredde; der skal Helligdommen, den højhellige, ligge.
\par 4 Det er en hellig Gave af Landet og skal tilfalde Præsterne, som gør Tjeneste i Helligdommen, dem, som træder frem for at gøre Tjeneste for HERREN; og det skal give dem Plads til Boliger og Græsgang.
\par 5 Et Stykke på 25000 Alens Længde og 10000 Alens Bredde skal som Grundejendom tilfalde Leviteme, som gør Tjeneste i Templet, til Byer at bo i.
\par 6 Byens Grundejendom skal I give en Bredde af 5000 Alen og en Længde af 25000 Alen, samme Længde som den hellige Offerydelse; den skal tilhøre hele Israels Hus.
\par 7 Og Fyrsten skal på begge Sider af den hellige Offerydelse og Byens Grundejendom have et Område langs den hellige Offerydelse og Byens Grundejendom både på Vestsiden og Østsiden af samme Længde som en af Stammelodderne fra Landets Vestgrænse til Østgrænsen;
\par 8 det skal tilhøre ham som Grundejendom i Israel, for at mine Fyrster ikke fremtidig skal undertrykke mit Folk; men det øvrige Land skal gives Israels Hus, Stamme for Stamme.
\par 9 Så siger den Herre HERREN: Lad det nu være nok, I Israels Fyrster! Afskaf Vold og Undertrykkelse, gør Ret og Skel og hør op med eders Overgreb mod mit Folk, lyder det fra den Herre HERREN.
\par 10 Vægt, som vejer rigtigt, Efa og Bat, som holder Mål, skal I have.
\par 11 Efa og Bat skal have ens Mål, så at en Bat holder en Tiendedel Homer, og en Efa ligeledes en Tiendedel Homer; efter en Homer skal Målet fastslås.
\par 12 En Sekel skal holde tyve Gera; fem Sekel skal være fem, ti Sekel ti, og til halvtredsindstyve Sekel skal l regne en Mine.
\par 13 Dette er den offerydelse, I skal yde: En Sjetedel Efa af hver Homer Hvede og en Sjetedel Efa af hver Homer Byg.
\par 14 Den fastsatte Ydelse af Olien: En Tiendedel Bat af hver Kor, ti Bat udgør jo en Kor;
\par 15 et Lam fra Småkvæget af hver 200 som Offerydelse fra alle Israels Slægter til Afgrødeofre, Brændofre og Takofre for at skaffe eder Soning, lyder det fra den Herre HERREN
\par 16 Alt Folket i Landet skal give Fyrsten i Israel denne Offerydelse.
\par 17 Men Fyrsten skal det påhvile at udrede Brændofrene, Afgrødeofrene og Drikofrene på Højtiderne, Nymånefesterne og Sabbaterne, alle Israels Huses Fester; han skal sørge for Syndofrene, Afgrødeofrene, Brændofrene og Takofrene for at skaffe Israels Hus Soning.
\par 18 Så siger den Herre HERREN: På den første Dag i den første Måned skal I fage en lydefri ung Tyr og rense Helligdommen for Synd.
\par 19 Præsten skal tage noget af Syndofferets Blod og stryge det på Templets Dørstolper, Alterfremspringets fire Hjørner og Dørstolperne til den indre Forgårds Port.
\par 20 Det samme skal han gøre på den første Dag i den syvende Måned for deres Skyld, som har fejlet af Vanvare eller Uvidenhed, og således skaffe Templet Soning.
\par 21 På den fjortende Dag i den første Måned skal I fejre Påskefesten: syv Dage skal I spise usyret Brød;
\par 22 og Fyrsten skal på den Dag for sig selv og for alt Folket i Landet ofre en Tyr som Syndoffer;
\par 23 på de syv Festdage skal han som Brændoffer for HERREN ofre syv Tyre og syv Vædre, lydefri Dyr, på hver af de syv Dage, og ligeledes daglig som Syndoffer en Gedebuk;
\par 24 og som Afgrødeoffer skal han ofre en Efa med Tyren og ligeledes een med Væderen, desuden en Hin Olie med Efaen.
\par 25 På den femtende Dag i den syvende Måned skal han på Festen ofre lige så meget som Syndoffer, Brændoffer, Afgrødeoffer og lige så megen Olie; det skal han gøre syv Dage.

\chapter{46}

\par 1 Så siger den Herre HERREN: Den indre forgårds østport skal være lukket de seks hverdage, men på Sabbatsdagen skal den åbnes, ligeledes på Nymånedagen:
\par 2 og Fyrsten skal udefra gå ind gennem Portens Forhal og stille sig ved Portens Dørstolpe. Præsterne skal ofre hans Brændoffer og Takofre, og han skal tilbede på Portens Tærskel og så gå ud igen; og Porten skal stå åben til Affen.
\par 3 Men Folket i Landet skal på Sabbaterne og Nymånedagene tilbede for HERRENs Åsyn ved denne Ports Indgang.
\par 4 Det Brændoffer, Fyrsten bringer HERREN på Sabbatsdagen, skal udgøre seks lydefri Lam og en lydefri Væder,
\par 5 dertil et Afgrødeoffer på en Efa med Væderen og et Afgrødeoffer efter Behag med Lammene, desuden en Hin Olie med hver Efa.
\par 6 På Nymånedagen skal det udgøre en ung, lydefri Tyr, seks Lam og en Væder, lydefri Dyr;
\par 7 med Tyren skal han ofre et Afgrødeoffer på en Efa, med Væderen ligeledes en Efa og med Lammene efter Behag desuden en Hin Olie med hver Efa.
\par 8 Når Fyrsten går ind, skal han komme gennem Portens Forhal, og samme Vej skal han gå ud;
\par 9 men når Folket i Landet konmmer for HERRENs Åsyn på Festerne, skal den, der kommer ind gennem Nordporten for at tilbede, gå ud gennem Sydporten, og den, der kommer ind gennem Sydporten, gå ud genmmem Nordporten; han må ikke vende tilbage gennem den Port, han kom ind ad, men skal gå ud på den modsatte Side.
\par 10 Fyrsten skal være iblandt dem; når de går ind, skal han også gå ind, og når de går ud, skal han også gå ud.
\par 11 På Festerne og Højtiderne skal Afgrødeofferet være en Efa med hver Tyr og ligeledes en Efa med hvem Væder, men med Lammene efter Behag; desuden en Hin Olie med hver Efa.
\par 12 Når Fyrsten ofrer et frivilligt Offer, et Brændoffer eller Takofre som frivilligt ofer til HERREN, skal man åbne Østporten for ham, og han skal ofre sit Brændoffer eller sine Takofre, som han gør på Sabbatsdagen; og når han er gået ud, skal man lukke Porten efter ham.
\par 13 Et årgammelt, lydefrif Lam skal han daglig ofre som Brændoffer for HERREN; hver Morgen skal han ofre det;
\par 14 og dertil skal han hver Morgen sonm Afgrødeoffer ofre en Sjettedel Efa og til af fugte Melet desuden en Tredjedel Hin Olie; det er et Afgmødeoffer for HERREN, en evigt gældende Ordning.
\par 15 Således skal de hver Morgen ofre Lammet, Afgrødeofferet og Olien som dagligt Brændoffer.
\par 16 Så siger den Herre HERREN: Når Fyrsten giver en af sine Sønner en Gave af sin Arvelod, skal den tilhøre hans Sønner; den skal være deres arvelige grundejendom;
\par 17 men giver han en af sine Tjenere en Gave af sin Arvelod, skal den kun tilhøre ham til Frigivningsåret; så skal den vende tilbage til Fyrsten. Kun hans Sønner skal varigt eje en sådan Arvelod.
\par 18 Og Fyrsten må ikke tage noget af Folkets Arvelod, idet han med Vold trænger dem ud af deres Grundejendom; af sin egen Grundejendom skal han give sine Sønner Arvelod, at ingen i mit Folk skal jages bort fra sin Grundejendom.
\par 19 Derpå førte han mig ind gennem Indgangen ved Siden af Porten til de hellige Kamre, som var indrettet til Præsterne og vendte mod Nord, og se, der var et Rum i den inderste krog mod Vest.
\par 20 Han sagde til mig: "Her er det Rum, hvor Præsterne skal koge Syndofferet og Skyldoferet og bage Afgrødeofferet for ikke at tvinges til at bringe det ud i den ydre Forgård og således hellige Folket."
\par 21 Så bragte han mig ud i den ydre Forgård og førte mig rundt til Forgårdens fire Hjørner, og se, i hvert Hjørne var der et Gårdsrum;
\par 22 i Forgårdens fire Hjørner var der små Gårdsrum, fyrretyve Alen lange og tredive Alen brede, alle fire lige store;
\par 23 der var Mure rundt om dem alle fire, og der var indrettet Køkkener rundt om langs Murene.
\par 24 Og han sagde til mig: "Her er Køkkenerne, hvor de, der gør Tjemmesle i Templet, skal koge Folkets Slagtofre."

\chapter{47}

\par 1 Derpå førte han mig tilbage til tempelets indgang, og se, vand sprang ud under tempelets Tærskel i østlig Retning, thi Templets Forside vendte mod Øst; og Vandet løb ned under Templets Sydside sønden for Alteret.
\par 2 Så førte han mig ud gennenm Nordporten og rundt udenom til den ydre Østport, og se, Vand rislede frem fra Sydsiden.
\par 3 Derpå gik Manden ud mod Øst med en Målesnor i Hånden, og da han havde målt 1000 Alen, lod han mig gå gennem Vandet, Vand til Anklerne.
\par 4 Da han atter havde målt 1000 Alen, lod han mig på ny gå gennem Vandet, Vand til Knæene; og da han atter havde målt 1000 Alen, lod han mig på ny gå gennem Vandet, som der nåede til Hoften.
\par 5 Da han atter havde målt 1000 Alen, var det en Strøm, som jeg ikke kunde vade over, thi Vandet gik så højt, at man måtte svømme over; det var en Strøm, man ikke kunde vade over,
\par 6 Da sagde han til mig: "Har du set det, Menneskesøn?" Og han førte mig tilbage langs Strømmens Bred.
\par 7 Da jeg kom tilbage, se, da var der ved Strømmens Bred en stor Mængde Træer på begge Sider;
\par 8 og han sagde til mig: "Dette Vand løber ud i Østerkredsen og ned i Araba, og når det falder ud i Havet, Salthavet, bliver Vandet der sundt;
\par 9 alle de levende Væsener, hvoraf det vrimler, skal leve, overalt hvor Strømmen kommer hen, og der skal være en stor Mængde Fisk; thi når dette Vand kommer derhen, bliver Havvandet sundt, og alt skal leve, hvor Strømmen kommer hen.
\par 10 Fiskere skal stå ved det fra En-Gedi til En-Eglajim; et Sted til at udspænde Fiskegarn skal det være; dels Fisk skal være som det store Havs Fisk, såre mange.
\par 11 Men dets Sumpe og Vandhuller skal ikke blive sunde; af dem skal udvindes Salt.
\par 12 På begge Flodens Bredder skal der vokse alle Hånde Frugttræer, hvis Blade ikke falder af, og hvis Frugter aldrig får Ende; hver Måned bærer de nye Frugter; thi dens Vand udspringer i Helligdommen. Frugterne skal tjene til Føde og Bladene til Lægedom."
\par 13 Så siger den Herre HERREN: Dette er den Grænse, inden for hvilken I skal udskifte Landet imellem eder efter Israels tolv Stammer; Josef skal have to Lodder.
\par 14 I skal alle uden Undtagelse tage det Land i Eje, som jeg med løftet Hånd svor at give eders Fædre; det skal nu tilfalde eder som Arvelod.
\par 15 Således skal Landets Nordgrænse være: Fra det store Hav i Retning af Hetlon til det Sted, hvor Vejen går til Zedad,
\par 16 Hamat, Berota, Sibrajim, mellem Damaskuss og Hamats Områder, Hazar-Enon ved Haurans Grænse.
\par 17 Grænsen går fra det store Hav til Hazar-Enon, således at Damaskuss Område ligger norden for ved Siden af Hamat. Det er Nordgrænsen.
\par 18 Østgrænsen: Fra Hazar-Enon mellem Hauran og Damaskus danner Jordan Grænse mellem Gilead og Israels Land til Havet i Øst, hen til Tamar. Det er Øsfgrænsen.
\par 19 Sydgrænsen: Fra Tamar over Meribas Vand ved Kadesj til Bækken, ud til det store Hav. Det er Sydgrænsen.
\par 20 Vestgrænsen: Det store Hav danner Grænse til tværs over for det Sted, hvor Vejen går til Hamat. Det er Vestgrænsen.
\par 21 Dette Land skal I udskifte iblandt eder efter Israels Stammer;
\par 22 I skal ved Lodkastning udskifte det som Arvelod mellem eder og de fremmede, som bor iblandt eder og der har avlet Sønner og Døtre; de skal være eder som ind fødte Israeliter og kaste Lod med eder om Arvelod blandt Israels Stammer.
\par 23 I den Stamme, hvor den fremmede bor, skal I give ham hans Arvelod, lyder det fra den Herre HERREN.

\chapter{48}

\par 1 Følgende er navnene på stammerne: Yderst i nord fra havet i retning af Hetlon til det sted, hvor Vejen går til Hamat, og videre til Hazar-Enon, med Damaskuss Område mod Nord ved Siden af Hamat, fra Østsiden til Vestsiden: Dan, een Stammelod;
\par 2 langs Dans Område fra Østsiden til Vestsiden: Aser, een Stammelod;
\par 3 langs Asers Område fra Østsiden til Vestsiden: Naftali, een Stammelod;
\par 4 langs Naftalis Område fra Østsiden til Vestsiden: Manasse, een Stammelod;
\par 5 langs Manasses Område fra Østsiden til Vestsiden: Efraim, een Stammelod;
\par 6 langs Efrainms Område fma Østsiden til Vestsiden: Ruben, een Stammelod;
\par 7 langs Rubens Område fra Østsiden til Vestsiden: Juda, een Stammelod.
\par 8 Langs Judas Område fra Østsiden til Vestsiden skal Offerydelsen, som I yder, være 25000 Alen bred og lige så lang som hver Stammelod fra Østsiden til Vestsiden; og Helligdommen skal ligge i Midten.
\par 9 Offerydelsen, som I skal yde HERREN, skal være 25 000 Alen lang og 20800 Alen bred;
\par 10 og den hellige Offerydelse skal tilhøre følgende: Præsterne skal have et Stykke, som mod Nord er 25 000 Alen langt, mod Vest 10.000 Alen bredt, mod Øst 10000 Alen bredt og mod Syd 25.000 Alen langt; og HERRENs Helligdom skal ligge i Midten.
\par 11 De helligede Præster, Zadoks Efterkommere, som tog Vare på, hvad jeg vilde have varetaget, og ikke som Leviterne for vild, da Israeliterne gjorde det,
\par 12 skal det tilhøre som en Offerydelse af Landets Offerydelse, et højhelligt Område langs Leviternes.
\par 13 Og Leviterne skal have et lige så stort Område som Præsterne, 25000 Alen langt og 10000 Alen bredt; den samlede Længde bliver således 25000 Alen, Bredden 20000.
\par 14 De må ikke sælge eller bortbytte noget deraf eller overdrage denne førstegrøde af Landet til andre, thi den er helliget HERREN.
\par 15 Det Stykke på 5000 Alens Bredde, som er tilovers af Offerydelsens Bredde langs de 25000 Alen, skal være uindviet Land og tilfalde Byen til Boliger og Græsgang, og Byen skal ligge i Midten;
\par 16 dens Mål skal være følgende: Nordsiden 4500 Alen, Sydsiden 4500, Østsiden 4500 og Vestsiden 4500.
\par 17 Byens Græsgang skal være 250 Alen mod Nord, 250 mod Syd, 250 mod Øst og 250.
\par 18 Af det Stykke, som endnu er tilovers langs med den hellige Offerydelse, 10000 Alen mod Øst og 10000 mod Vest, skal Afgrøden tjene Byens indbyggere til Mad.
\par 19 Byens Befolkning skal sammensættes således, at Folk fra alle israels Stammer bor der:
\par 20 I alt skal I som Offerydelse yde en Firkant på 25000 Alen, den hellige Offerydelse foruden Byens Grundejendom.
\par 21 Men Resten skal tilfalde Fyrsten; hvad der ligger på begge Sider af den hellige Offerydelse og Byens Grundejendom, østen for de 25000 Alen hen til Østgrænsen og vesten for de 25000 Alen hen til Vestgrænsen, langs Stammelodderne, skal tilhøre Fyrsten; den hellige Offerydelse, Templets Helligdomn i Midten
\par 22 og Leviternes og Byens Grundejendom skal ligge midt imellem de Stykker, som tilfalder Fyrsten mellem Judas og Benjamins Område.
\par 23 Så følger de sidste Stammer: Fra Østsiden til Vestsiden Benjammmin, een Stammelod;
\par 24 langs Benjamins Område fra Østsiden til Vestsiden: Simeon, een Stammelod;
\par 25 langs Simeons Område fra Østsiden til Vestsiden: Issakar, een Stammelod;
\par 26 langs Issakars Område fra Østsiden til Vestsiden: Zebulon, een Stammelod;
\par 27 langs Zebulons Område fra Østsiden til Vestsiden: Gad, een Stammelod;
\par 28 og langs Gads Område på Sydsiden skal Grænsen gå fra Tamar over Meribas Vandved Kadesj til Bækken ud til det store Hav.
\par 29 Det er det Land, I ved Lodkastning skal udskifte som Arvelod til Israels Stammer, og det er deres Stammelodder, lyder det fra den Herre HERREN.
\par 30 Følgende er Byens udgange; Byens Porte skal opkaldes efter Israels Stammer:
\par 31 På Nordsiden, der måler 4500 Alen, er der tre Porte, den første Rubens, den anden Judas og den tredje Levis;
\par 32 på Østsiden, der måler 4500 Alen, er der tre Porte, den første Josefs, den anden Benjamins og den tredje Dans;
\par 33 på Sydsiden, der måler 4500 Alen, er der tre Porte, den første Simeons, den anden Issakars og den tredje Zebulons;
\par 34 på Vestsiden, der måler 4500 Alen, er der tre Porte, den første Gads, den anden Asers og den tredje Naftalis.
\par 35 Omkredsen er 18000 Alen. Og Byens Navn skal herefter være: HERREN er der.


\end{document}