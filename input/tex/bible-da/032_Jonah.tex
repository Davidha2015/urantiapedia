\begin{document}

\title{Jonah}


\chapter{1}

\par 1 HERRENs Ord kom til Jonas, Amittajs Søn, således:
\par 2 "Stå op og gå til Nineve, den store Stad, udråb over den, at deres Ondskab er kommet op for mit Åsyn."
\par 3 Men Jonas stod op for at fly fra HERRENs Åsyn til Tarsis. Han drog ned til Jafo, og da han fandt et Skib, som skulde til Tarsis betalte han, hvad Rejsen kostede, og gik om Bord for at sejle med til Tarsis bort fra HERRENs Åsyn.
\par 4 Men HERREN Iod et stærkt Vejr fare hen over Havet, og en stærk Storm rejste sig på Havet, så Skibet var ved at gå under.
\par 5 Sømændene blev rædde og råbte hver til sin Gud, og de kastede alle Sager i Skibet over Bord for at lette det. Men Jonas var gået ned i det underste Skibsrum og lå i dyb Søvn dernede.
\par 6 Skibsføreren gik da ned til ham og sagde: "Hvor kan du sove? Stå op og råb til din Gud! Måske vil Gud komme os i Hu, så vi ikke omkommer."
\par 7 Så sagde de til hverandre: "Kom, lad os kaste Lod for at få at vide, hvem der er Skyld i, at denne Ulykke er tilstødt os!" Og de kastede Lod, og Loddet ramte Jonas.
\par 8 Da sagde de til ham: "Sig os, hvem der er Skyld i, at denne Ulykke er tilstødt os! Hvad er du, og hvor kommer du fra? Hvilket Land er du fra, og hvilket Folk hører du til?"
\par 9 Han svarede: "Jeg er Hebræer, og jeg frygter HERREN, Himmelens Gud, som har skabt Havet og det tørre Land."
\par 10 Så grebes Mændene af stor Rædsel og sagde til ham: "Hvad har du gjort!" Thi de fik at vide, at han flyede fra HERRENs Åsyn; det sagde han dem.
\par 11 Og de spurgte ham: "Hvad skal vi gøre med dig, så vi får Havet til at lægge sig? Thi det rejser sig mere og mere."
\par 12 Han svarede: "Tag og kast mig i Havet! Så får I det til at lægge sig; thi jeg ved, at jeg er Skyld i, at dette stærke Vejr er over eder."
\par 13 Mændene søgte nu at ro tilbage til Land, men kunde ikke, da Ha- vet rejste sig mere og mere imod dem.
\par 14 Så råbte de til HERREN; "Ak, HERRE! Lad os ikke omkomme for denne Mands Sjæls Skyld og lad ikke uskyldigt Blod komme over os; thi det erjo dig, HERRE, derhar gjort, som du vilde."
\par 15 Derpå tog de Jonas og kastede ham i Havet, og straks lagde det sig.
\par 16 Og Mændene grebes af stor Rædsel for HERREN, bragte ham et Slagtoffer og aflagde Løfter.

\chapter{2}

\par 1 Men HERREN bød en stor Fisk slugeJonas; og Jonas var i Fiskens Bug tre Dage og tre Nætter.
\par 2 Da bad Jonas i Fiskens Bug til HERREN sin Gud
\par 3 og sagde: Jeg råbte i Nøden til HERREN, og han svarede mig; jeg skreg fra Dødsrigets Skød, og du hørte min Røst.
\par 4 Du kasted mig i Dybet midt i Havet, Strømmen omgav mig; alle dine Brændinger og Bølger skyllede over mig.
\par 5 Jeg tænkte: "Bort er jeg stødt fra dine Øjne, aldrig mer skal jeg skue dit hellige Tempel."
\par 6 Vandene trued min Sjæl, Dybet omgav mig, Tang var viklet om mit Hoved; til Bjergenes Rødder
\par 7 steg jeg ned, til Jordens Slåer, de evige Grundvolde; da drog du mit Liv op af Graven, HERRE min Gud.
\par 8 Da min Sjæl vansmægtede i mig, kom jeg HERREN i Hu, og min Bøn steg op til dig i dit hellige Tempel.
\par 9 De, der dyrker det tomme Gøgl, lader Gudsfrygt fare;
\par 10 men jeg vil bringe dig Ofre med Lovsangs Toner og indfri de Løfter, jeg gav. Hos HERREN er Frelse.
\par 11 Så talede HERREN til Fisken, og den spyede Jonas ud på det tørre Land.

\chapter{3}

\par 1 Jonas i Nineve Men HERRENs Ord kom for anden gang til Jonas således:
\par 2 "Stå op og gå til Nineve, den store Sfad, og udråb over den, hvad jeg tilsiger dig!"
\par 3 Så stod Jonas op og gik til Nineve efter HERRENs Ord. Men Nineve var selv for Gud en stor By, tte Dagsrejser stor.
\par 4 Da nu Jonas var gået den første bagsrejse ind i Byen, råbte han: "Om fyrretyve Dage skal Nineve styrtes i Grus!"
\par 5 Da troede Folkene i Nineve påGud, og de udråbte en Faste og klædte sig i Sæk, både store og små;
\par 6 og da Sagen kom Nineves Konge for Øre, stod han op fra sin Trone,tog Kappen af, klædte sig i Sæk og satte sig i Støvet,
\par 7 og han lod udråbe i Nineve:"Kongen og hans Stormænd gørvitterligt: Hverken Folk eller Fæ,Hornkvæg eller Småkvæg, må nyde noget, græsse eller drikke Vand;
\par 8 men Folk og Fæ skal klædes i Sæk og opløfte et vældigt Skrig til Gud og omvende sig, hver fra sin onde Vej og den Uret, som hænger ved deres Hænder.
\par 9 Måske vil Gud da angre og holde sin glødende Vrede tilbage, så vi ikke omkommer."
\par 10 Da Gud så, hvad de gjorde, hvorledes de omvendte sig fra deres onde Vej, angrede han den Ulykke, han havde truet med at føre over dem, og gjorde ikke Alvor deraf.

\chapter{4}

\par 1 Men det tog Jonas såre fortrydeligt op, og han blev vred.
\par 2 Så bad han til HERREN og sagde: "Ak, HERRE! Var det ikke det, jeg tænkte, da jeg endnu var hjemme i mit Land? Derfor vilde jeg også før fly til Tarsis; jeg vidste jo, at du er en nådig og barmhjertig Gud, langmodig og rig på Miskundhed, og at du angrer det onde.
\par 3 Så tag nu, HERRE, mit Liv; thi jeg vil hellere dø end leve."
\par 4 Men HERREN sagde: "Er det med Rette; du er vred?"
\par 5 Så gik Jonas ud og slog sig ned østen for Byen; der byggede han sig en Løvhytte og satte sig i Skygge under den for at se, hvorledes det gik Byen.
\par 6 Da bød Gud HERREN en Olieplante skyde op over Jonas og skygge over hans Hoved for at tage hans Mismod, og Jonas glædede sig højligen over den.
\par 7 Men ved Morgengry næste Dag bød Gud en Orm stikke Olieplanten, så den visnede;
\par 8 og da Solen stod op, rejste Gud en glødende Østenstorm, og Solen stak Jonas i Hovedet, så han vansmægtede og ønskede sig Døden, idet han tænkte: "Jeg vil hellere dø end leve."
\par 9 Men Gud sagde til Jonas: "Er det med Rette, du er vred for Olieplantens Skyld?" Han svarede: "Ja, med Rette er jeg så vred, at jeg kunde tage min Død derover,"
\par 10 Da sagde HERREN: "Du ynkes over Olieplanten, som du ingen Møje har haft med eller opelsket, som blev til på een Nat og gik ud på een Nat.
\par 11 Og jeg skulde ikke ynkes over Nineve, den store Stad med mer end tolv Gange 10000 Mennesker, som ikke kan skelne højre fra venstre, og meget Kvæg."


\end{document}