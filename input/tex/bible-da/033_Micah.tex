\begin{document}

\title{Mikas Bog}


\chapter{1}

\par 1 HERRENs Ord, som, i de Dage da Jotam, Akaz og Ezekias var Konger i Juda, kom til Mika fra. Moresjet, og som han skuede om Samaria og Jerusalem.
\par 2 Alle I Folkeslag, hør, lyt til, du Jord, med din Fylde, at den Herre HERREN kan stå som Vidne blandt eder, Herren fra sit hellige Tempel.
\par 3 Thi se, fra sit Sted går HERREN ud, stiger ned, skrider frem over Jordens Høje;
\par 4 under ham smelter Bjerge, og Dale slår dybe Revner, som Voks, der smelter i Ilden, som Vand, gydt ned ad en Skrænt -
\par 5 alt dette for Jakobs Brøde, for Israels Huses Synder. Hvem voldte Jakobs Brøde? Mon ikke Samaria? Hvem voldte Judas Synd? Mon ikke Jerusalem?
\par 6 Samaria gør jeg til Grushob, dets Mark til Vingårdsjord; jeg styrter dets Sten i Dalen, dets Grundvolde bringer jeg for Lyset.
\par 7 Dets Billeder sønderslås alle, dets Skøgeløn brændes i Ild; jeg tilintetgør alle dets Afguder; thi af Skøgeløn er de samlet, til Skøgeløn bliver de atter.
\par 8 Derfor vil jeg klage og jamre, gå nøgen med bare Fødder, istemme Klage som Sjakaler, jamrende Skrig som Strudse:
\par 9 Ulægeligt er HERRENs Slag, thi det når til Juda, til mit Folks Port rækker det hen, til Jerusalem.
\par 10 Forkynd det ikke i Gat, græd ikke i Bokim! Vælt jer i Støvet i Bet-Leafra!
\par 11 Der stødes i Horn for eder, Sjafirs Borgere; ej går Za'anans Borgere ud af deres By. Bet-Ezels Lod bliver Klage, Hug og Ve;
\par 12 og hvor kan Marots indbyggere håbe på Lykke? Thi Ulykke kom ned fra HERREN til Jerusalems Porte.
\par 13 Spænd Hestene for Vognen, I, som bor i Lakisj! Syndens Begyndelse var du for. Zions Datter; ja, Israels Overtrædelser fandtes i dig.
\par 14 Giv derfor Moresjet-Gat en Skilsmissegave! En svigtende Bæk er Akzibs Huse for Israels Konger.
\par 15 End sender jeg eder en Ransmand, Maresjas Borgere! Til Adullam skal Israels Herlig hed komme.
\par 16 Klip dig skaldet over dine elskede. Børn, bredskaldet som en Grib; thi de bortføres fra dig.

\chapter{2}

\par 1 Ve dem, der på Lejet udtænker Uret og Udåd, og sætter det i Værk, når det dages, da det står i deres Magt.
\par 2 De attrår Marker og raner dem, Huse og tager dem, undertrykker Mand og Hus, Ejendom og Ejer.
\par 3 Derfor, så siger HERREN: Se, jeg optænker Ulykke mod denne Slægt, fra hvilken l ikke skal kunne fri eders Hals eller gå med oprejst Hoved; thi en ond Tid er det.
\par 4 På denne Dag skal der bruges et Mundheld om jer og klages: "Sket som talt! Vi er helt lagt øde; mit Folk får sin Lod skiftet ud, ingen giver den tilbage; vor Mark skiftes ud til dem, som fører os bort."
\par 5 Derfor har du ingen til at udspænde Snoren over en Lod i HERRENs Forsamling.
\par 6 "Præk ikke!" så præker de, "man præker ikke om sligt; får hans Smæden ej Ende?" Hvad siger du, Jakobs Hus?
\par 7 "Er HERREN da hastig til Vrede, handler han så? Er hans Ord ej milde mod den, som vandrer ret?"
\par 8 Men I er fjendske imod, på Nakken af mit Folk; Kappen over Kjortelen river l af dem, som vandrer trygt og afskyr Strid.
\par 9 Mit Folks Kvinder driver l ud af det Hjem, de holdt af, I tager for evigt min Ære fra deres Børn:
\par 10 "Op, ryk ud! Thi her kan I ikke bo for den Urenheds Skyld, som volder svar Fordærv."
\par 11 I Fald der kom en Mand med Tomhed og Svig og Løgn: "Jeg vil præke for dig om Vin og Drik!" det var en Præker for dette Folk.
\par 12 Jeg vil samle dig, hele Jakob, opsanke Israels Rest, få dem sammen som Får i Fold, som en Hjord i Græsgangens Midte; af Mennesker bliver der en Summen.
\par 13 En Vejbryder går foran dem; de bryder gennem Porten og går ud. Foran dem skrider deres Konge og HERREN i Spidsen for dem.

\chapter{3}

\par 1 Da sagde jeg: Hør dog, I Jakobs Høvdinger, I Dommere af Israels Hus!. Kan Kendskab til Ret ej ventes af eder,
\par 2 I, som hader det gode og elsker. det onde, I, som flår Huden af Folk. og Kødet af deres Ben,
\par 3 æder mit Folks Kød og flænger dem Huden af Kroppen, sønderbryder deres Ben. og breder dem som Kød i en Gryde, som Suppekød i en Kedel?
\par 4 Engang skal de råbe til HERREN, han lader dem uden Svar;. da skjuler han sit Åsyn for. dem, fordi deres Gerninger er onde.
\par 5 Så siger HERREN om Profeterne, de, som vildleder mit Folk, de, som råber om Fred, når kun de får noget at tygge, men ypper Krig med den, der intet giver dem i Munden:
\par 6 Derfor skal I opleve Nat uden. Syner, Mørke, som ej bringer Spådom; Solen skal gå ned for Profeterne, Dagen skal sortne for dem.
\par 7 Til Skamme skal Seeren blive, blues skal de, som spår; alle skal tilhylle Skægget, thi Svar er der ikke fra Gud.
\par 8 Jeg derimod er ved HERRENs Ånd fuld af Styrke, af Ret og af Kraft til at forkynde Jakob dets Brøde, Israel, hvad det har syndet.
\par 9 Hør det, I Jakobs Huses Høvdinger, I Dommere af Israels Hus, I, som afskyr Ret og gør alt, som er lige, kroget,
\par 10 som bygger Zion med Blod. og Jerusalem med Uret.
\par 11 Dets Høvdinger dømmer for Gave, dets Præster kender Ret for Løn; dets Profeter spår for Sølv og støtter sig dog til HERREN: "Er HERREN ej i vor Midte? Over os kommer intet ondt."
\par 12 For eders, Skyld skal derfor Zion pløjes som en Mark, Jerusalem blive til Grushobe, Tempelbjerget til Krathøj.

\chapter{4}

\par 1 Og det skal ske i de sidste Dage, at HERRENS Huses Bjerg, grundfæstet på Bjergenes Top, skal løfte sig op over Højene.
\par 2 og talrige Folk komme vandrende: "Kom, lad os drage til HERRENs Bjerg, til Jakobs Guds Hus; os skal han lære sine Veje, så vi kan gå på hans Stier; thi fra Zion udgår Åbenbaring, fra Jerusalem HERRENs Ord."
\par 3 Da dømmer han mange Folkeslag imellem, skifter Ret mellem talrige, fjerne Folk; deres Sværd skal de smede til Plovjern, deres Spyd til Vingårdsknive; Folk skal ej løfte Sværd mod Folk, ej øve sig i Våbenfærd mer.
\par 4 Da sidder hver under sin Vinstok og sit Figenfræ, og ingen.
\par 5 Thi alle Folkeslag vandrer hvert i sin Guds Navn, men vi vil vandre i HERREN vor Guds Navn for evigt og altid.
\par 6 På hin Dag, lyder det fra HERREN, samler jeg det, der halter, sanker det spredte sammen og det, jeg har hjemsøgt med ondt.
\par 7 Det haltende gør jeg til en Rest, det svage til et kraftigt Folk; og HERREN er Konge over dem på Zions Bjerg fra nu og til evig Tid.
\par 8 Men du, o Hyrdetårn, Zions Datters Høj, til dig skal det komme, det forrige Herredømme tilfalde dig, Jerusalems Datters Rige.
\par 9 Hvi skriger du dog så højt?. Er Kongen ikke i dig?. Er da din Rådgiver borte, nu du grebes af Fødselsveer?
\par 10 Vrid dig og vånd dig som i Barnsnød, Zions Datter! Thi nu skal du ud af Byen og bo på Marken, og du skal komme til Babel; der skal du frelses, der vil HERREN genløse dig af dine Fjenders Hånd.
\par 11 Nu er de samlet imod dig, de mange Folk, som siger: "Vanæres skal det; vort Øje skal se med Skadefryd på Zion."
\par 12 Men de, de kender ikke det mindste til HERRENs Tanker, de fatter ikke hans Råd, at han samled dem som Neg på Lo.
\par 13 0p og tærsk, du Zions Datter! Thi jeg giver dig Horn af Jern, jeg giver dig Klove af Kobber. Du skal knuse de mange Folk, lægge Band på Byttet for HERREN, på Godset for al Jordens Herre.
\par 14 Riv nu Sår i din Hud! De bar opkastet en Vold imod os; med Stokken slår de Israels Hersker på Kinden.

\chapter{5}

\par 1 Og du du Betlehems-Efrata, liden til at være blandt Judas Tusinder! Af dig skal udgå mig een til at være Hersker i Israel. Hans Udspring er fra fordum, fra Evigheds Dage.
\par 2 Derfor giver han dem hen, så længe til hun, som skal føde, føder, og Resten af hans Brødre vender hjem til Israeliterne.
\par 3 Han skal stå og vogte i HERRENs Kraft, i HERREN sin Guds høje Navn. De skal bo trygt, thi nu skal hans Storhed nå Jordens Grænser.
\par 4 Og han skal være Fred. Når Assur trænger ind i vort Land, og når han træder ind i vore Borge, stiller vi syv Hyrder imod ham og otte fyrstelige Mænd,
\par 5 som skal vogte Assurs Land med Sværd og Nimrods Land med Klinge. Og han skal fri os fra Assur, når han trænger ind i vort Land, træder ind på vore Enemærker.
\par 6 Da bliver Jakobs Rest i de mange Folkeslags Midte som Dug, der kommer fra HERREN, som Regnens Dråber på Græs, der ikke venter på nogen eller bier på Menneskens Børn.
\par 7 Da bliver Jakobs Rest blandt Folkene i de mange Folkeslags Midte som en Løve blandt Skovens Dyr, en Ungløve blandt Fårehjorde, der nedtramper, når den går frem, og sønderriver redningsløst.
\par 8 Din Hånd skal være over dine Uvenner, alle dine Fjender ryddes bort.
\par 9 På hin Dag, lyder det fra HERREN, udrydder jeg Hestene af dig, dine Stridsvogne gør jeg til intet.
\par 10 rydder Byerne bort i dit Land, river alle dine Fæstninger ned,
\par 11 rydder Trolddommen bort af din Hånd, Tegntydere får du ej mer;
\par 12 jeg rydder dine Billeder bort, Stenstøtterne bort af din Midte og du skal ikke mer tilbede dine Hænders Værk.
\par 13 Jeg udrydder dine Asjerer og lægger dine Afguder øde;
\par 14 i Vrede og Harme tager jeg Hævn over Folk, som ikke vil høre.

\chapter{6}

\par 1 Hør, hvad HERREN taler: Kom fremfør din Trætte for Bjergene, lad Højene høre din Røst!
\par 2 I Bjerge, hør HERRENs Trætte, lyt til, I Jordens Grundvolde! Thi HERREN har Trætte med sit Folk, med Israel går han i Rette:
\par 3 Hvad har jeg gjort dig, mit Folk. med hvad har jeg plaget dig? Svar!
\par 4 Jeg førte dig jo op fra Ægypten og udløste dig af Trællehuset, og jeg sendte for dit Ansigt Moses, Aron og Mirjam.
\par 5 Mit Folk, kom i Hu, hvad Kong Balak af Moab havde i Sinde, og hvad Bileam, Beors Søn, svarede ham, fra Sjittim til Gilgal, for at du kan kende HERRENs Retfærdsgerninger.
\par 6 "Med hvad skal jeg møde HERREN, bøje mig for Højhedens Gud? Skal jeg møde ham med Brændofre, møde med årgamle Kalve?
\par 7 Har HERREN Behag i Tusinder af Vædre, Titusinder af Oliestrømme? Skal jeg give min førstefødte for min Synd, mit Livs Frugt som Bod for min Sjæl?"
\par 8 Det er sagt dig, o Menneske, hvad der er godt, og hvad HERREN kræver af dig: hvad andet end at øve Ret, gerne vise Kærlighed og vandre ydmygt med din Gud.
\par 9 Hør, HERREN råber til Byen (at frygte dit Navn er Visdom): Hør, Stamme og Byens Menighed!
\par 10 Skal jeg tåle Skattene i den gudløses Hus og den magre, forbandede Efa,
\par 11 tilgive Gudløsheds Vægt og Pungen med falske Lodder?
\par 12 Dens Rigmænd er fulde af Vold, dens Borgeres Tale er Løgn, og Tungen er falsk i deres Mund.
\par 13 Derfor tog jeg til at slå dig, ødelægge dig for dine Synder.
\par 14 Du skal spise, men ikke mættes, lige tomt skal dit Indre være; hvad du hengemmer, skal du ej bjærge, og hvad du bjærger, giver jeg Sværdet;
\par 15 du skal så, men ikke høste, perse Oliven, men ikke salve dig, perse Most, men ej drikke Vin.
\par 16 Du fulgte Omris Skikke, al Akabs Huses Færd; I vandrede efter deres Råd, så jeg må gøre dig til Ørk og Byens Borgere til Spot; Folkenes Hån skal I bære.

\chapter{7}

\par 1 Ve mig! Det går mig som ved ved Frugthøst, ved Vinhøstens Efterslæt: Ikke en Drue at spise, ej en Figen, min Sjæl har Lyst til!
\par 2 De fromme er svundet af Landet, ikke et Menneske er sanddru.
\par 3 Deres Hænder er flinke til ondt, Fyrsten kræver, Dommeren er villig for Betaling; Stormanden nævner, hvad han begærer; og derefter snor de det sammen.
\par 4 Den bedste er som en Tornebusk, den ærlige værre end en Tjørnehæk. Dine Vægteres Dag, din Hjemsøgelse kommer, af Rædsel rammes de nu.
\par 5 Tro ikke eders Næste, stol ikke. på en Ven, vogt Mundens Døre for hende. du favner!
\par 6 Thi Søn agter Fader ringe, Datter står Moder imod Svigerdatter Svigermoder, en Mand har sine Husfolk til Fjender.
\par 7 Men jeg vil spejde efter HERREN, jeg bier på min Frelses Gud; min Gud vil høre mig.
\par 8 Glæd dig ej over mig, min Fjende! Thi jeg faldt, men står op; om end jeg sidder i Mørke, er HERREN mit Lys.
\par 9 Jeg vil bære HERRENs Vrede - jeg synded jo mod ham - indtil han strider for mig og skaffer mig Ret; han fører mig ud i Lys, jeg skal skue hans Retfærd.
\par 10 Min Fjende skal se derpå og fyldes med Skam, han, som spørger mig: "Hvor er HERREN din Gud?". Mine Øjne skal med Skadefryd se ham, når han trampes ned som Skarn på Gaden.
\par 11 En Dag skal dine Mure bygges, en Dag skal Grænsen vides ud,
\par 12 en Dag skal man komme til dig lige fra Assur til Ægypten, lige fra Ægypten til Floden, fra Hav til Hav, fra Bjerg til Bjerg.
\par 13 Men Jorden og de, som bor derpå, lægges øde til Løn for deres Værk.
\par 14 Vogt med din Stav dit Folk, din Ejendoms Hjord, som bor for sig selv i Skoven, i Frugthavens Midte; lad dem græsse i Basan og Gilead som i gamle Dage!
\par 15 Giv os Undere at skue, som da du drog ud af Ægypten;
\par 16 lad Folkene se det og blues ved al deres Vælde, lægge Hånd på Mund, lad Ørene døves på dem!
\par 17 Lad dem slikke Støv som Slangen, som Jordens Kryb, rædde komme frem af deres Borge til HERREN vor Gud og ængstes og frygte for dig!
\par 18 Hvo er en Gud som du, der tilgiver Brøde, bærer over med Synd hos din Ejendoms Rest, ej evigt gemmer på Vrede, men gerne er nådig?
\par 19 Han vil atter forbarme sig over os, træde vor Brøde under Fod, du vil kaste alle vore Synder i Havets Dyb!
\par 20 Du vil vise Jakob Trofasthed, Abraham Nåde, som du svor vore Fædre til i fordums Dage.



\end{document}