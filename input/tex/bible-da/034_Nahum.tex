\begin{document}

\title{Nahums Bog}


\chapter{1}

\par 1 Et udsagn om Nineve. En bog om Elkosjiten Nahums Syn.
\par 2 En nidkær Gud, en Hævner er HERREN, en Hævner er HERREN og fuld af Vrede, en Hævner er HERREN mod Uvenner, han gemmer på Vrede mod Fjender.
\par 3 HERREN er langmodig, hans Kraft er stor, HERREN lader intet ustraffet. I Uvejr og Storm er hans Vej, Skyer er hans Fødders Støv.
\par 4 Han truer og udtørrer Havet, gør alle Strømme tørre; Basan og Karmel vansmægter, Libanons Skud visner hen.
\par 5 Bjergene skælver for ham, Højene står og svajer; Jorden krummer sig for ham, Jorderig og alle, som bor der.
\par 6 Hvem kan stå for hans Vrede, hvo holder Stand mod hans Harmglød? Hans Harme strømmer som Ild, og Fjeldene styrter for ham.
\par 7 HERREN er god, et Værn på Trængselens Dag; han kender dem, som lider på ham,
\par 8 og fører dem gennem Skybrud. Sine Avindsmænd gør han til intet, støder Fjenderne ud i Mørke.
\par 9 Hvad pønser I på mod HERREN? Han tilintetgør i Bund og Grund; ej kommer der to Gange Nød.
\par 10 Er de end som sammenflettet Tjørn og gennemdrukne af Vin, skal de dog fortæres som fuldført Strå.
\par 11 Fra dig er der en draget ud med ondt i Sinde mod Herren, med Niddingeråd.
\par 12 Så siger HERREN: Er de end fuldtallige og aldrig så mange, skal de dog omhugges og forsvinde. Har jeg end ydmyget dig, gør jeg det ikke mere.
\par 13 Nu sønderbryder jeg det Åg, han lagde på dig, og sprænger dine Bånd.
\par 14 Om dig lyder HERRENs Bud: Dit Navn skal ikke ihukommes mere.

\chapter{2}

\par 1 Se, Glædsesbudet, som kundgør Fred, skrider frem over bjergene Fejr dine Fester, Juda, indfri dine Løfter! Thi aldrig skal Niddingen mer drage gennem dig; han er udryddet helt og holdent.
\par 2 Hærgeren drager imod dig, hold Vagt med Omhu, hold Udkig, omgjord din Lænd, saml al din Kraft!
\par 3 Thi HERREN genrejser Jakobs og Israels Højhed; dem har jo Hærværksmænd hærget og ødt deres Ranker.
\par 4 Hans Heltes Skjolde er røde, hans Stridsmænd skarlagenklædt, hans Vogne funkler af Stål, den Dag han ruster og Spydene svinges.
\par 5 Igennem Gaderne raser Vognene frem, hen over Torvene farer de i susende Fart; de ser ud som Fakler, farer frem og tilbage som Lyn.
\par 6 Hans Helte kaldes frem, de snubler i Farten, de styrter frem imod Muren. Skjoldtaget er rejst.
\par 7 Flodportene bliver åbnet, Kongsgården vakler.
\par 8 Herskerinden føres bort i Landflygtigbed med sine Terner; de sukker som kurrende Duer, slår sig for Brystet.
\par 9 Nineve er som en Dam, hvis Vand flyder bort. "Stands dog, stands dog!" råbes der, men ingen vender om.
\par 10 Ran Sølv, ran Guld! Der er Liggendefæ uden Ende, alskens kostbare Ting i store Måder.
\par 11 Tomt og tømt og udtømt, ængstede Hjerter, rystende Knæ og Skælven i alle Lænder! Og alle Ansigter blegner.
\par 12 Hvor er nu Løvernes Bo, Ungløvernes Hule, hvor Løven frak sig tilbage, hvor Ungløven ej kunde skræmmes?
\par 13 Den røved til Ungernes Tarv og myrded til Løvinderne, fyldte sine Hier med Bytte. sit Bo med Rov.
\par 14 Se, jeg kommer over dig, lyder det fra Hærskarers HERRE, dit Lejrsted lader jeg gå op i Røg. Dine Ungløver skal Sværdet fortære; jeg rydder din Røverfærd bort fra Jorden. Dine Sendebuds Røst skal aldrig høres mer.

\chapter{3}

\par 1 Ve Byen, der drypper af Blod, hvor der kun tales Løgn, så fuld af Ran, med Rov uden Ende!
\par 2 Hør Smæld og raslende Vogne, jagende Heste,
\par 3 Stridsvognenes vilde Dans og stejlende Heste! Sværdblink og lynende Spyd, faldne i Mængde, Masser af døde, endeløse Dynger af Lig, man snubler over Lig!
\par 4 For Skøgens vidt drevne Utugt, den fagre, udlært i Trolddom, som besnærede Folk ved Utugt, Stammer ved Trolddom,
\par 5 kommer jeg over dig, lyder det fra Hærskarers HERRE; dit Slæb slår jeg op i Ansigtet på dig, lader Folkeslag se din Blusel, Riger din Skam,
\par 6 dænger dig til med Skarn og vanærer dig, ja sætter dig i Gabestok.
\par 7 Enhver, som får dig at se, skal fly fra dig og sige: "Nineve er ødelagt, hvem vil ynke det, hvor skal jeg hente en til at give det Trøst?"
\par 8 Mon du er bedre end No-Amon, der lå ved Strømme, omgivet af Vand som Bolværk, med Vand til Mur?
\par 9 Dets Styrke var Ætiopere og Ægyptere uden Tal; Put og Libyer kom det til Hjælp.
\par 10 Dog førtes det bort, i Fangenskab måtte det vandre, på alle Gadebjørner knustes også dets spæde; og om dets ædle kastedes Lod, alle dets Stormænd lagdes i Lænker.
\par 11 Også du skal drikke og synke i Afmagt, også du skal søge i Ly for Fjenden.
\par 12 Alle dine Fæstninger er Figener og tidligmoden Frugt; når de rystes, falder de den spisende i Munden.
\par 13 Se, Folket i dig er som Kvinder, vidåbne for Fjenden er Portene ind til dit Land, Ild fortæred dine Slåer.
\par 14 Øs Vand til Brug, når du omringes, styrk dine Fæstninger, træd Dynd, stamp Ler, tag fat på Teglstensformen.
\par 15 Ild skal fortære dig på Stedet. Sværd udrydde dig, fortære dig som Springere. Er du end talrig som Springere, talrig som Græshopper,
\par 16 er end dine Købmænd flere end Himlens Stjerner - Græshoppen kaster sin Vingeskal og flyver!
\par 17 Dine Fogeder er som Græshopper, dine Tipsarer som Græshoppesværme; de lejrer sig i Hegn, når Dagen er sval; men når Solen står op, er de borte, man ved ej hvor.
\par 18 Hvor sov dine Hyrder fast, du Assurs Konge! Dine Helte blunded; dit Folk er spredt på Bjergene, ingen samler dem.
\par 19 Ulægeligt er dit Brud, dit Sår er til Døden. Alle, som hører om dig, klapper i Hånd; thi hvem fik ikke din Ondskab stadig at føle?


\end{document}