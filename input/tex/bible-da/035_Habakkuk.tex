\begin{document}

\title{Habakkuks Bog}


\chapter{1}

\par 1 Det Udsagn, Profeten Habakkuk skuede.
\par 2 Hvor længe skal jeg klage, HERRE, uden du hører, skrige til dig over Vold, uden du frelser?
\par 3 Hvi lader du mig skue Uret, være Vidne til Kvide? Ødelæggelse og Vold har jeg for Øje, der opstod Kiv, og Strid kom op.
\par 4 Derfor ligger Loven lammet, og Ret kommer aldrig frem. Thi når gudløse trænger retfærdige, fremkommer krøget Ret.
\par 5 Se eder om blandt Folkene til Skræk og Rædsel for eder! Thi en Gerning gør han i eders Dage, som I ej vilde tro, om det fortaltes.
\par 6 Thi se, han vækker Kaldæerne, det grumme og raske Folk, som drager viden om Lande for at indtage andres Bo.
\par 7 Forfærdeligt, frygteligt er det, Ødelæggelse udgår derfra.
\par 8 Dets Heste er rappere end Pantere, mer viltre end Ulve ved Kvæld; dets Rytterheste kommer i Spring, flyvende langvejs fra.
\par 9 er de alle på Vej efter Vold. De higede stadig mod Øst og samlede Fanger som Sand.
\par 10 Med Kongerne drev det Spot, Fyrsterne lo det kun ad. Det lo ad hver en Fæstning, opdynged en Vold og tog den.
\par 11 Så suste det videre som Stormen og gjorde sin Kraft til sin Gud.
\par 12 Er du ikke fra fordum HERREN, min hellige Gud? - Vi skal ej dø - HERRE, har du sat ham til Dommer, givet ham Fuldmagt til Straf?
\par 13 Dit rene Blik afskyr ondt, du tåler ej Synet af Kvide; hvi ser du da tavs på Ransmænd, at gudløs sluger sin Overmand i Retfærd?
\par 14 Med Mennesker gør du som med Havets Fisk, som med Kryb, der er uden Hersker:
\par 15 Han fisker dem alle med Krog, slæber dem bort i sit Vod og samler dem i sit Garn; derfor er han jublende glad;
\par 16 han ofrer derfor til sit Vod, tænder Offerild for sit Garn.
\par 17 Skal han altid tømme sit Vod og slå Folk ihjel uden Skånsel?

\chapter{2}

\par 1 Op på min Varde vil jeg stige; stå hen på mit Vagtsted og spejde, og se, hvad han taler i mig, hvad Svar han har på min Klage.
\par 2 Og HERREN gav mig til Svar de Ord: "Skriv Synet op og rist det ind i Tavler, at det kan læses let;
\par 3 thi Synet står ved Magt, træffer ind til Tide, usvigeligt iler det mod Målet; tøver det, bi så på det, thi det kommer; det udebliver ikke."
\par 4 Se, opblæst, uredelig er Sjælen i ham, men den retfærdige skal leve ved sin Tro.
\par 5 Han er der hos den frækkeste Røver, en hoven, frastødende Mand, der som Dødsriget opspiler Gabet, som Døden uden at mættes, skraber alle Folkene til sig, sanker alle Folkeslag til sig.
\par 6 Visselig skal de alle istemme en Hånsang, en Smædevise fuld af Hentydninger til ham og sige: Ve ham, der dynger andres Gods op - hvor længe? - og læsser Pantegods på sig!
\par 7 Thi brat står dine Skyldherrer op; de, som vil rykke dig, vågner; da bliver du dem til Bytte.
\par 8 Fordi du har plyndret mange Folk, skal du plyndres af al Folkeslagenes Rest for Menneskeblods Skyld, for Vold mod Landet, mod Byen og alle, som bor der.
\par 9 Ve ham, som søger ublu Vinding til sit Hus for at bygge sin Rede højt og redde sig fra Ulykkens Hånd.
\par 10 Dit Hus får Skam af dit Råd.Du nedtrådte mange Folkeslag, men satte din Sjæl i Vove.
\par 11 Thi Stenen råber fra Væggen, fra Træværket svarer Bjælken.
\par 12 Ve ham, som bygger By med Blod og rejser en Stad med Uret,
\par 13 (er dette ikke, fra Hærskarers HERRE?) så Folkeslag slider for Ilden, og Folkefærds Møje er spildt.
\par 14 Thi Jorden skal fyldes af Kundskab om HERRENs Herlighed, som Vandene dækker Havets Bund.
\par 15 Ve ham, som lader Venner drikke en Rus af Fade og Skåle for at få deres Blusel at se.
\par 16 Du mætted dig med Skam for Ære. Drik selv, vis din Forhud frem! Nu kommer Bægeret fra HERRENs højre til dig og Skændsel til din Ære.
\par 17 Thi du tynges af Vold mod Libanon, knuses for Misbrug af Dyr, for Menneskeblods Skyld, for Vold mod Landet, mod Byen og alle, som bor der.
\par 18 Hvad gavner det skårne Billed, at en Billedskærer skærer det ud, det støbte Billed, hvis Spådom er falsk, at en Billedskærer stoler derpå, så han laver stumme Guder?
\par 19 Ve den, som siger til Træ: "Vågn op!" til Sten uden Mæle:" - Stå op!" Den skulde kunne spå! Se, den er klædt i Guld og Sølv, men af Ånd har den intet i sig.
\par 20 Men HERREN er i sin Helligdom; stille for ham, al Jorden!

\chapter{3}

\par 1 En Bøn af Profeten Hahakkuk. Al-sjigjonot.
\par 2 HERRE, jeg har hørt dit Ry, jeg har skuet din Gerning, HERRE.
\par 3 Gud drager frem fra Teman, den Hellige fra Parans Bjerge. - Sela.
\par 4 Under ham er Glans som Ild, fra hans Side udgår Stråler; der er hans Vælde i Skjul.
\par 5 Foran ham vandrer Pest, og efter ham følger Sot.
\par 6 Hans Fjed får Jorden til at skælve, hans Blik får Folk til at bæve. De ældgamle Bjerge brister, de evige Høje synker, ad evige Stier går han.
\par 7 Kusjans Telte bæver, Telttæpperne i Midjans Land.
\par 8 Er HERREN da vred på Strømmene, gælder din Vrede Strømmene, gælder din Harme Havet, siden du farer frem på dine Heste og dine Vogne drøner.
\par 9 Din Bue kom blottet til Syne, din Buestreng mætter du med Pile. - Sela.
\par 10 Bjergene ser dig og skælver. Skyerne nedsender Regnskyl, og Afgrunden løfter sin Røst.
\par 11 Solen glemmer at stå op, Månen bliver i sit Bo; de flygter for Skinnet af dine Pile, for Glansen af dit lynende Spyd.
\par 12 I Harme skrider du hen over Jorden, du nedtramper Folk i Vrede.
\par 13 Du drager ud til Frelse for dit Folk, ud for at frelse din Salvede. Du knuser den gudløses Hustag, blotter Grunden til Klippen. - Sela.
\par 14 Med dit Spyd gennemborer du hans Hoved, bans Høvdinger splittes.
\par 15 Du tramper hans Heste i Havet, i de mange Vandes Dynd.
\par 16 Jeg hørte det; da bæved min Krop, ved Braget skjalv mine Læber; Edder for i mine Ben, og under mig vakled mine Skridt.
\par 17 Thi Figentræet blomstrer ikke, Vinstokken giver intet, Olietræets Afgrøde svigter, Markerne giver ej Føde. Fårene svandt af Folden, i Staldene findes ej Okser.
\par 18 Men jeg vil frydes i HERREN, juble i min Frelses Gud.
\par 19 Den Herre HERREN er min Styrke, han gør mine Fødder som Hindens og lader mig gå på mine Høje.


\end{document}