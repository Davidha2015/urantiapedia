\begin{document}

\title{Haggajs Bog}


\chapter{1}

\par 1 I Kong Dariun's andet Regeringsår på den første Dag i den sjette Måned kom HERRENs Ord ved Profeten Haggaj således: Sig til Judas Statholder Zerubbabel, Sjealtiels Søn, og til Ypperstepræsten Josua, Jozadaks Søn:
\par 2 Så siger Hærskarers HERRE: Dette Folk siger: "Endnu er det ikke Tid at bygge HERRENs Hus."
\par 3 Og HERRENs Ord kom ved Profeten Haggaj således:
\par 4 Er det da Tid for eder at bo i Huse med træklædte Vægge, når dette Hus ligger øde?
\par 5 Derfor, så siger Hærskarers HERRE: Læg Mærke til, hvorledes det går eder!
\par 6 I sår meget, men bringer lidet i Hus; I spiser, men mættes ikke; I drikker, men får ikke Tørsten slukket; I klæder eder på, men bliver ikke varme; og Daglejerens Løn går i en hullet Pung.
\par 7 Så siger Hærskarers HERRE: Læg Mærke til, hvorledes det går eder!
\par 8 Gå op i Bjergene, hent Tømmer og byg Templet, så jeg kan have Glæde deraf og blive æret, siger HERREN.
\par 9 I venter rig Høst, men det bliver kun til lidt; og når I bringer det i Hus, blæser jeg derpå. Hvorfor? lyder det fra Hærskarers HERRE. Fordi mit Hus ligger øde, medens enhver af eder har travlt med sit eget Hus.
\par 10 Derfor holder Himmelen sin Dug og Jorden sin Afgrøde tilbage;
\par 11 og jeg har kaldt Tørke hid over Land og Bjerge, over Korn, Most og Olie, over alt, hvad Jorden frembringer, over Folk og Fæ, over alt, hvad Hænder virker.
\par 12 Og Zerubbabel, Sjaltiels Søn, og Ypperstepræsten Josua, Jozadaks Søn, og hele Resten af Folket adlød HERREN deres Guds Røst og Profeten Haggajs Ord, eftersom HERREN havde sendt ham til dem, og Folket frygtede HERREN.
\par 13 Da sagde Haggaj, HERRENs Sendebud, i HERRENs Ærinde til Folket: Jeg er med eder, lyder det fra HERREN.
\par 14 Og HERREN vakte Ånden i Judas Statholder Zerubbabel, Sjaltiels Søn, og i Ypperstepræsten Josua, Jozadaks Søn, og i hele Resten at Folkef, så de kom og tog fat på Arbejdet med deres Guds, Hærskarers HERREs, Hus
\par 15 på den fire og tyvende Dag i den sjette Måned i Kong Darius's andet Regeringsår.

\chapter{2}

\par 1 På den een og tyvende Dag i den syvende Måned kom HERRENs Ord ved Profeten Haggaj således:
\par 2 Sig til Judas Statholder Zerubbabel, Sjaltiels Søn, til Ypperstepræsten Josua, Jozadaks Søn, og til Resten af Folket således:
\par 3 Er der nogen tilbage iblandt eder af dem, der har set dette Hus i dets fordums Herlighed? Hvorledes tykkes det eder da nu? Er det ikke som intet i eders Øjne?
\par 4 Dog vær kun trøstig, Zerubbabel, lyder det fra HERREN, vær kun trøstig, du Ypperstepræst Josua, Jozadaks Søn, vær kun trøstigt, alt Folket i Landet, lyder det fra HERREN. Arbejd kun, thi jeg er med eder, lyder det fra Hærskarers HERRE,
\par 5 og min Ånd bliver iblandt eder med den Pagt, jeg sluttede med eder, da I drog bort fra Ægypten; frygt ikke!
\par 6 Thi så sigerHærskarers HERRE: Endnu en Gang om en liden Stund vil jeg ryste Himmel og Jord, Hav og tørt Land,
\par 7 og jeg vil ryste alle Folkene, og da skal alle Folkenes Skatte komme hid, og jeg fylder dette Hus med Herlighed, siger Hærskarers HERRE.
\par 8 Mit er Sølvet, og mit er Guldet, lyder det fra Hærskarers HERRE.
\par 9 Dette Hus's kommende Herlighed bliver større end den tidligere, siger Hærskarers HERRE, og på dette Sted vil jeg give Fred, lyder det fra Hærskarers HERRE.
\par 10 På den fire og tyvende Dag i den niende Måned i Darius's andet Regeringsår kom HERRENs Ord ved Profeten Haggaj således:
\par 11 Så siger Hærskarers HERRE: Bed Præsterne om Svar på følgende Spørgsmål:
\par 12 "Dersom en Mand bærer helligt Kød i sin kappeflig og med Fligen rører ved Brød eller noget, som er kogt, eller ved Vin eller Olie eller nogen Slags Mad, bliver disse Ting så hellige?" Præsterne svarede nej.
\par 13 Haggaj spurgte da: "Hvis en, som er blevet uren ved Lig, rører ved nogen af disse Ting, bliver den så uren?" Præsterne svarede ja.
\par 14 Så tog Haggaj til Orde og sagde: Således er det i mine Øjne med disse Mennesker, således med dette Folk, lyder det fra HERREN, og således med alt deres Hænders Værk og med, hvad de ofrer der: det er urent.
\par 15 Men læg nu Mærke til, hvorledes det går fra i Dag! Før Sten lagdes på Sten i HERRENs Hus,
\par 16 hvorledes gik det eder da? Når man kom til en Dynge Korn på tyve Mål, var der ti; og kom man til en Vinperse for at øse halvtresindstyve Mål af Kummen, var der tyve.
\par 17 Jeg slog eder med Kornbrand, Rust og Hagl ved alt eders Arbejde, men I omvendte eder ikke til mig, lyder det fra HERREN.
\par 18 Læg Mærke til, hvorledes det går fra i Dag, fra den fire og tyvende Dag i den niende Måned, fra den Dag Grunden lagdes til HERRENs Hus, læg Mærke dertil!
\par 19 Er Sæden endnu i Laderne, står Vinstokken, Figentræet, Granatæbletræet og Oliventræet endnu uden Frugt? Fra i Dag velsigner jeg.
\par 20 Og HERRENs Ord kom for anden Gang til Haggaj på den fire og tyvende bag i Måneden således:
\par 21 Sig til Judas Statholder Zerubbabel: Jeg ryster Himmelen og Jorden
\par 22 og omstyrter Kongernes Troner, tilintetgør Hedningerigernes Styrke og vælter Stridsvognene med deres Førere, og Heste og Ryttere skal styrte, og den ene skal falde for den andens Sværd.
\par 23 På hin Dag, lyder det fra Hærskarers HERRE, tager jeg dig, min Tjener Zerubbabel, Sjealtiels Søn. lyder det fra HERREN, og gør dig til en Seglring; thi dig har jeg udvalgt, lyder det fra Hærskarers HERRE.


\end{document}