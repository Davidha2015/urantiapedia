\begin{document}

\title{Zechariah}


\chapter{1}

\par 1 I den ottende Måned i Darius's andet Regeringsår kom HERRENs Ord til Profeten Zakarias, en Søn af Berekja, en Søn af Iddo, således:
\par 2 HERREN var fuld af Harme imod eders Fædre.
\par 3 Men sig til dem: Så siger Hærskarers HERRE: Vend om til mig, lyder det fra Hærskarers HERRE, så vil jeg vende om til eder, siger Hærskarers HERRE.
\par 4 Vær ikke som eders Fædre, til hvem de tidligere Profeter talte således: Så siger Hærskarers HERRE: Vend om fra eders onde Veje og onde Gerninger! Men de hørte ikke og ænsede mig ikke, lyder det fra HERREN.
\par 5 Eders Fædre, hvor er de? Og Profeterne, lever de evigt?
\par 6 Men mine Ord og Bud, som jeg overgav mine Tjenere Profeterne, nåede de ikke eders Fædre, så de måtte vende om og sige: "Som Hærskarers HERRE havde sat sig for at gøre imod os for vore Vejes og Gerningers Skyld, således gjorde han."
\par 7 På den fire og tyvende dag i den ellevte Måned, det er Sjebat Måned, i Darius's andet Regeringsår kom HERRENs Ord til Profeten Zakarias, en Søn af Berekja, en Søn af Iddo, således:
\par 8 Jeg skuede i Nat, og se, en Mand på en rød Hest holdt mellem Bjergene ved den dybe Kløft, og bag ham var der røde, mørke, hvide og brogede Heste.
\par 9 Jeg spurgte: "Hvad betyder de, Herre?" Og Engelen, som talte med mig, sagde: "Jeg vil vise dig, hvad de betyder."
\par 10 Så tog Manden, som holdt mellem Bjergene, til Orde og sagde: "Det er dem, HERREN udsender, for at de skal drage Jorden rundt."
\par 11 Og de tog til Orde og sagde til HERRENs Engel, som stod mellem Bjergene: "Vi drog Jorden rundt,og se, hele Jorden er rolig og stille.
\par 12 HERRENs Engel tog da til Orde og sagde: "Hærskarers HERRE! Hvor længe varer det, før du forbarmer dig over Jerusalem og Judas Byer, som du nu har været vred på i halvfjerdsindstyve År?"
\par 13 Og til Svar gav HERREN Engelen, som talte med mig, gode og trøstende Ord.
\par 14 Engelen, som talte med mig, sagde så til mig: Tal og sig: Så siger Hærskarers HERRE: Jeg er fuld af Nidkærhed for Jerusalem og Zion
\par 15 og af Harme mod de trygge Hedninger, fordi de hjalp til at gøre Ulykken stor, da min Vrede kun var lille.
\par 16 Derfor, så siger HERREN: Jeg vender mig til Jerusalem og forbarmer mig over det; mit Hus skal opbygges der, lyder det fra Hærskarers HERRE, og der skal udspændes Målesnor over Jerusalem.
\par 17 Tal videre: Så siger Hærskarers HERRE: Mine Byer skal atter strømme med Velsignelse, og HERREN vil atter trøste Zion og udvælge Jerusalem.

\chapter{2}

\par 1 Derpå løftede jeg mine Øjne og skuede, og se, der var fire Horn.
\par 2 Jeg spurgte Engelen, som talte med mig: "Hvad betyder de?" Han svarede: "Det er de Horn, som har spredt Juda, Israel og Jerusalem."
\par 3 Så lod HERREN mig se fire Smede.
\par 4 Jeg spurgte: "Hvad kommer de for?" Og han svarede: "Hine er de Horn, som spredte Juda, så det ikke kunde løfte sit Hoved; og nu kommer disse for at hvæsse Økser til at slå Hornene til Jorden på de Hedninger, som løftede deres Horn mod Judas Land for at sprede det."
\par 5 Derpå løftede jeg mine Øjne og skuede, og se, der var en Mand med en Målesnor i Hånden.
\par 6 Jeg spurgte: "Hvor skal du hen?" Han svarede: "Hen at måle Jerusalem og se, hvor bredt og langt det er."
\par 7 Og se, Engelen, som talte med mig, trådte frem, og en anden Engel trådte frem over for ham,
\par 8 og han sagde til ham: "Løb ben og sig til den unge Mand der: Som åbent Land skal Jerusalem ligge, så mange Mennesker og Dyr skal der være i det.
\par 9 Jeg vil selv, lyder det fra HERREN, være en Ildmur omkring det og herliggøre mig i det.
\par 10 Op, op, fly bort fra Nordlandet, lyder det fra HERREN, thi fra Himmelens fire Vinde samler jeg eder, lyder det fra HERREN.
\par 11 Op, red jer til Zion, I, som bor hos Babels Datter!
\par 12 Thi så siger Hærskarers HERRE, hvis Herlighed sendte mig til Folkene, der hærger eder: Den, som rører eder, rører min Øjesten.
\par 13 Thi se, jeg svinger min Hånd imod dem, og de skal blive til Bytte for dem, som nu er deres Trælle; og I skal kende, at Hærskarers HERRE har sendt mig.
\par 14 Jubl og glæd dig, Zions Datter! Thi se, jeg kommer og fæster Bo i din Midte, lyder det fra HERREN."
\par 15 Og mange Folk skal på hin Dag slutte sig til HERREN og være hans Folk og bo i din Midte, og du skal kende, at Hærskarers HERRE har sendt mig til dig.
\par 16 Og HERREN tager Juda til Arvelod på den hellige Jord og udvælger atter Jerusalem.
\par 17 Stille, alt Kød, for HERREN, thi han har rejst sig fra sin hellige Bolig.

\chapter{3}

\par 1 Derpå lod han mig se Ypperstepræsten Josua, og han stod foran HERRENs Engel, medens Satan stod ved hans højre Side for at føre Klage imod ham.
\par 2 Men HERREN,sagde til Satan: "HERREN true dig, Satan, HERREN true dig, han, som udvalgte Jerusalem. Er denne ikke en Brand, som er reddet ud af Ilden?"
\par 3 Josua havde snavsede Klæder på og stod foran Engelen;
\par 4 men denne tog til Orde og sagde til dem, som stod ham til Tjeneste: "Tag de snavsede Klæder af ham!" Og til ham sagde han: "Se, jeg har taget din Skyld fra dig, og du skal have Højtidsklæder på."
\par 5 Og han sagde: "Sæt et rent Hovedbind på hans Hoved!" Og de satte et rent Hovedbind på hans Hoved og gav ham rene Klæder på. Så trådte HERRENs Engel frem,
\par 6 og HERRENs Engel vidnede for Josua og sagde:
\par 7 Så siger Hærskarers HERRE: Hvis du vandrer på mine Veje og holder mine Forskrifter, skal du både Råde i mit Hus og vogte mine Forgårde, og jeg giver dig Gang og Sæde blandt dem, som står her"
\par 8 Hør, du Ypperstepræst Josua, du og dine Embedsbrødre, som sidder for dit Ansigt: de er Varselmænd! Thi se, jeg lader min Tjener Zemak komme.
\par 9 Thi se, den Sten, jeg lægger hen for Josua - på den ene Sten er syv Øjne - se, jeg rister selv dens Indskrift, lyder det fra Hærskarers HERRE, og på een Dag udsletter jeg dette Lands Skyld.
\par 10 På hin Dag, lyder det fra Hærskarers HERRE, skal I byde hverandre til Gæst under Vinstok og Figenfræ.

\chapter{4}

\par 1 Engelen, som talte med mig, vakte mig så atter, som man vækker et Menneske af hans Søvn,
\par 2 og spurgte mig: "Hvad skuer du?" Jeg svarede: "Jeg, skuer, og se, der er en Lysestage, helt og holdent af Guld, og et Oliekar ovenpå og syv Lamper og syv Rør til Lamperne,
\par 3 desuden to Olietræer ved Siden af den, et til højre, et andet tilvenstre for Oliekarret."
\par 4 Og jeg spurgte Engelen, som talte med mig:"Hvad betyder disse Ting, Herre?"
\par 5 Han svarede: "Ved du ikke, hvad de betyder?" Jeg sagde: "Nej.
\par 6 Da svarede han og sagde til mig: Dette er HERRENs Ord til Zerubbabel: Ikke ved Magt og ikke ved Styrke, men ved min Ånd, siger Hærskarers HERRE.
\par 7 Hvem er du, du store Bjerg? For Zerubbabel skal du blive Slette! Han skal hente Topstenen, medens der råbes: "Nåde, Nåde være med den!"
\par 8 Og HERRENs Ord kom til mig således:
\par 9 Zerubbabels Hænder har lagt Grunden til dette Hus, hans Hænder skal også fuldende det; og du skal kende, at Hærskarers HERRE har sendt mig til eder.
\par 10 Thi den, der lod hånt om de ringe Begyndelsers Dag, skal glæde sig, når han ser Blystenen i Zerubbabels Hånd. Hine syv er HERRENs Øjne, som søger ud over hele Jorden.
\par 11 Derpå spurgte jeg ham:"Hvad betyder de to Olietræer der til højre og venstre for Lysestagen?"
\par 12 Og videre spurgte jeg: "Hvad betyder de to Oliegrene ved Siden af de to Guldrør, som leder den gyldne Olie ned derfra?"
\par 13 Han svarede: "Ved du ikke, hvad de betyder?" Jeg sagde: "Nej, Herre!"
\par 14 Så sagde han: "Det er de to med Olie salvede, som står for al Jordens Herre."

\chapter{5}

\par 1 Derpå løftede jeg mine Øjne og skuede, og se, der var en flyvende Bogrulle.
\par 2 Og han spurgte mig: "Hvad ser du?" Jeg svarede: "Jeg ser en flyvende Bogrulle, som er tyve Alen lang og ti Alen bred."
\par 3 Da sagde han til mig: "Det er Forbandelsen, som udgår over hele Landet; thi alle, som stjæler, er nu længe nok gået fri, og alle, som sværger, er nu længe nok gået fri.
\par 4 Jeg lader den udgå, lyder det fra Hærskarers HERRE, for at den skal komme i Tyvens Hus og dens Hus, som sværger falsk ved mit Navn, og sætte sig fast i deres Huse og tilintetgøre dem med Tømmer og Sten."
\par 5 Derpå trådte Engelen, som talte med mig, frem og sagde til mig: "Løft dine Øjne og se, hvad det er, som kommer der!"
\par 6 Jeg spurgte: "Hvad er det?" Og han svarede: "Det er Efaen, som kommer." Og han vedblev: "Det er deres Brøde i hele Landet."
\par 7 Og se, et Blylåg løftedes, og se, i Efaen sad en Kvinde.
\par 8 Og han sagde: "Det er Gudløsheden!" Så stødte han hende ned i Efaen og slog Blylåget i over Åbningen.
\par 9 Og jeg løftede mine Øjne og skuede, og se, to Kvinder kom båret af Vinden, og deres Vinger var som Storkevinger; og de løftede Efaen op mellem Himmel og Jord.
\par 10 Så spurgte jeg Engelen, som talte med mig: "Hvor bærer de Efaen hen?"
\par 11 Han svarede: "Hen at bygge hende et Hus i Sinears Land, og når det er rejst, sætter de hende der, hvor hendes Sted er."

\chapter{6}

\par 1 Atter løftede jeg mine Øjne og skuede, og se, fire Vogne kom frem mellem de to Bjerge, og Bjergene var af Kobber.
\par 2 For den første Vogn var der røde Heste, for den anden sorte,
\par 3 for den tredje hvide og for den fjerde brogede.
\par 4 Jeg spurgte Engelen, som talte med mig: "Hvad betyder disse Herre?"
\par 5 Og Engelen svarede: "Det er Himmelens fire Vinde, som drager ud efter at have fremstillet sig for al Jordens Herre.
\par 6 Vognen med de sorte Heste for drager ud til Nordlandet, de hvide til Østlandet og de brogede til Sydlandet;
\par 7 de røde drager mod Vest. Og de var ivrige efter at komme af Sted for at drage ud over Jorden. Da sagde han: "Af Sted; drag ud over Jorden!" Og de drog ud over Jorden.
\par 8 Så kaldte han på mig og talte således til mig: "Se, de, som drager ud til Nordlandet, lader min Ånd dale ned over Nordens Land."
\par 9 HERRENs Ord kom til mig således:
\par 10 Tag imod Gaver fra de landflygtige, fra Heldaj, Tobija og Jedaja; endnu i Dag skal du gå indtil Josjija, Zefanjas Søn, som er kommet fra Babel,
\par 11 og modtage Sølv og Guld. Lad så lave en Krone og sæt den på Josuas Hoved
\par 12 med de Ord: "Så siger Hærskarers HERRE: Se, der kommer en Mand, hvis Navn er Zemak under ham skal det spire, og han skal bygge HERRENs Helligdom.
\par 13 Han skal bygge HERRENs Helligdom, og han skal vinde Højhed og sidde som Hersker på sin Trone: og han skal være Præst ved hans højre Side, og der skal være fuld Enighed mellem de to.
\par 14 Men Kronen skal blive i HERRENs Helligdom til Minde om Heldaj, Tobija, Jedaja og Hen, Zefanjas Søn.
\par 15 Langvejs fra skal man komme og bygge på HERRENs Helligdom; og I skal kende, at Hærskarers HERRE har sendt mig til eder. Og dersom I adlyder HERREN eders Gud - - -

\chapter{7}

\par 1 I Kong Darius's fjerde Regeringsår kom HERRENs Ord til Zakarias på den fjerde Dag i den niende Måned, Kislev.
\par 2 Da sendte Betel-Sar'ezer og Regem-Melek og hans Mænd Bud for at bede HERREN om Nåde
\par 3 og spørge Præsterne ved Hærskarers HERREs Hus og Profeterne: "Skal jeg græde og spæge mig i den femte Måned, som jeg nu har gjort i så mange År?"
\par 4 Da kom Hærskarers HERREs Ord til mig således:
\par 5 Sig til alt Folket i Landet og til Præsterne: Når l har fastet og klaget i den femte og syvende Måned i halvfjerdsindstyve År, var det da mig, I fastede for?
\par 6 Og når l spiser og drikker, er det da ikke eder, som spiser og drikker?
\par 7 Kender I ikke de Ord, HERREN forkyndte ved de tidligere Profeter, dengang Jerusalem og dets Byer trindt om var beboet og havde Fred, og Sydlandet og Lavlandetvar beboet?
\par 8 Og HERRENs Ord kom til Zaka rias således:
\par 9 Så siger Hærskarers HERRE: Fæld redelig Dom, vis Miskundhed og Barmhjertighed mod hverandre,
\par 10 undertryk ikke Enker og faderløse, fremmede og nødlidende og tænk ikke i eders Hjerter ondt mod hverandre!
\par 11 Men de vilde ikke høre; de var stivnakkede og gjorde deres Ører døve
\par 12 og deres Hjerter hårde som Diamant for ikke at høre Loven og de Ord, Hærskarers HERRE sendte gennem sin Ånd ved de tidligere Profeter. Derfor kom der stor Vrede fra Hærskarers HERRE.
\par 13 Ligesom de ikke hørte, når han kaldte, således vil jeg, sagde Hærskarers HERRE, ikke høre, når de kalder;
\par 14 og jeg blæste dem bort blandt alle de Folk, de ikke kendte, og Landet blev øde efter dem, så ingen drog ud eller hjem; og de gjorde det yndige Land til en Ørk.

\chapter{8}

\par 1 Hærskarers HERREs Ord kom således:
\par 2 Så siger Hærskarers HERRE: Jeg er fuld af Nidkærhed for Zion, ja i stor Vrede er jeg nidkær for det.
\par 3 Så siger HERREN: Jeg vender tilbage til Zion og fæster Bo i Jerusalem; Jerusalem skal kaldes den trofaste By, og Hærskarers HERREs Bjerg det hellige Bjerg.
\par 4 Så siger Hærskarers HERRE: Atter skal gamle Mænd og Kvinder sidde på Jerusalems Torve, alle med Stav i Hånd for deres Ældes Skyld,
\par 5 og Byens Torve skal vrimle af legende Drenge og Piger.
\par 6 Så siger Hærskarers HERRE: Fordi det i disse Dage synes det tiloversblevne af dette Folk umuligt, skulde det så også synes mig umuligt, lyder det fra Hærskarers HERRE.
\par 7 Så siger Hærskarers HERRE: Se, jeg frelser mit Folk fra Østerleden og Vesterleden
\par 8 og fører dem hjem, og de skal bo i Jerusalem og være mit Folk, og jeg vil være deres Gud i Trofasthed og Retfærd.
\par 9 Så siger Hærskarers HERRE: Fat Mod, I, som i denne Tid hører disse Ord af Profeternes Mund, fra den dag Grunden lagdes til Hærskarers HERREs Hus, Helligdommen, som skulde bygges.
\par 10 Thi før disse Dage gav hverken Menneskers eller Kvægs Arbejde Udbytte; de, som drog ud og ind, havde ikke Fred for Fjenden, og jeg slap alle Mennesker løs påhverandre.
\par 11 Men nu er jeg ikke mod det tiloversblevne af dette Folk som i fordums Dage, lyder det fra Hærskarers HERRE;
\par 12 jeg udsår Fred, Vinstokken skal give sin Frugt, Jorden sin Afgrøde og Himmelen sin Dug, og jeg giver det tiloversblevne af dette Folk det alt sammen i Eje.
\par 13 Og som I, både Judas og Israels Hus, har været et Forbandelsens Tegn blandt Folkene, således skal I, når jeg har frelst eder,blive et Velsignelsens. Frygt ikke, fat Mod!
\par 14 Thi så siger Hærskarers HERRE: Som jeg, da eders Fædre vakte min Vrede, satte mig for at handle ilde med eder og ikke angrede det, siger Hærskarers HERRE,
\par 15 således har jeg nu i disse Dage omvendt sat mig for at gøre vel mod Jerusalem og Judas Hus. Frygt ikke!
\par 16 Men hvad I skal gøre, er dette; Tal Sandhed hver med sin Næste, fæld i eders Porte Domme, der hvier på Sandhed og fører til Fred,
\par 17 tænk ikke i eders Hjerter ondt mod hverandre og elsk ikke falske Eder! Thi alt sligt hader jeg, lyder det fra HERREN.
\par 18 Og Hærskarers HERREs Ord kom til mig således:
\par 19 Så siger Hærskarers HERRE: Fasten i den fjerde, femte, syvende og tiende Måned skal blive Judas Hus til Fryd og Glæde og gode Højtidsdage. Elsk Sandhed og Fred!
\par 20 Så siger Hærskarers HERRE: Endnu skal det ske, at Folkeslag og mange Byers Indbyggere skal komme,
\par 21 og den ene Bys Indbyggere skal gå til den andens og sige: "Lad os vandre hen og bede HERREN om Nåde og søge Hærskarers HERRE; også jeg vil med."
\par 22 Og mange Folkeslag og talrige Folk skal komme og søge Hærskarers HERRE i Jerusalem for at bede HERREN om Nåde.
\par 23 Så siger Hærskarers HERRE: I hine Dage skal ti Mænd af alle Folks Tungemål gribe fat i en Jødes Kappeflig og sige: "Vi vil gå med eder; thi vi har hørt, at Gud er med eder."

\chapter{9}

\par 1 Et Udsagn: HERRENs Ord er over Hadraks Land, i Damaskus slår det sig ned - thi Aram forbrød sig mod HERREN - det er over alle, som hader Israel,
\par 2 også over Hamat, som grænser dertil, Tyrus og Zidon, thi det er såre viist.
\par 3 Tyrus bygged sig en Fæstning og ophobed Sølv som Støv og Guld som Gadeskarn.
\par 4 Se, Herren vil tage det i Eje og styrte dets Bolværk i Havet.
\par 5 Askalon ser det og frygter, Gaza og Ekron skælver voldsomt, thi Håbet brast. Gaza mister sin Konge, i Askalon skal ingen bo,
\par 6 i Asdod skal Udskud bo, Jeg gør Ende på Filisterens Hovmod,
\par 7 tager Blodet ud af hans Mund og Væmmelsen bort fra hans Tænder. Også han bliver reddet for vor Gud, han bliver som en Slægt i Juda, Ekron som, en, Jebusit.
\par 8 Som en Vagt lejrer jeg mig formit Hus mod dem, som kommer og går; aldrig mer skal en Voldsmand gå igennem deres Land, thi nu har jeg set det med mine egne Øjne.
\par 9 Fryd dig såre, Zions Datter, råb med Glæde, Jerusalems Datter! Se, din Konge kommer til dig. Retfærdig og sejrrig er han, ydmyg, ridende på et Æsel, på en Asenindes Føl.
\par 10 Han udrydder Vognene af Efraim, Hestene af Jerusalem, Stridsbuerne ryddes til Side. Hans Ord stifter Fred mellem Folkene, han hersker fra Hav til Hav, fra Floden til Jordens Ende.
\par 11 For Pagtblodets Skyld vil jeg også slippe dine Fanger ud, ja ud af den vandløse Brønd.
\par 12 Hjem til Borgen, I Fanger med Håb! Også i Dag forkyndes: Jeg giver dig tvefold Bod!
\par 13 Thi jeg spænder mig Juda som Bue, lægger Efraim på som Pil og vækker dine Sønner, Zion, imod dine Sønner, Javan. Jeg gør dig som Heltens Sværd.
\par 14 0ver dem viser sig Herren, hans Pil farer ud som et Lyn. Den Herre HERREN støder i Horn, skrider frem i Søndenstorm;
\par 15 dem værner Hærskarers HERRE. De opæder, nedtramper Slyngekasterne, drikker deres Blod som Vin og fyldes som Offerskålen, som Alterets Hjørner.
\par 16 HERREN deres Gud skal på denne Dag frelse dem som sit Folks Hjord; thi de er Kronesten, der funkler over hans Land.
\par 17 Hvor det er dejligt, hvor skønt! Thi Korn giver blomstrende Ungersvende, Most giver blomstrende Møer.

\chapter{10}

\par 1 HERREN skal I bede om Regn ved Tidlig- og Sildigregnstide; HERREN skaber Uvejr; Regnskyl giver han dem, hver Mand Urter på Marken.
\par 2 Men Husgudens Tale er Svig, Sandsigeres Syner er Blændværk: de kommer med tomme Drømme, hul er Trøsten, de giver; derfor vandrer de om som en Hjord, lider Nød, thi de har ingen Hyrde.
\par 3 Mod Hyrdeme blusser min Vrede, Bukkene vil jeg bjemsøge; thi Hærskarers HERRE ser til sin Hjord. Han ser til Judas Hus; han gør dem til en Ganger, sin stolte Ganger i Strid.
\par 4 Fra ham kommer Hjørne og Teltpæl, fra ham kommer Krigens Bue, fra ham kommer hver en Hersker.
\par 5 De bliver til Hobe som Helte, der i Striden tramper i Gadens Dynd; de kæmper, thi HERREN er med dem. Rytterne bliver til Skamme;
\par 6 jeg styrker Judas Hus og frelser Josefs Hus. Jeg ynkes og fører dem hjem, som havde jeg aldrig forstødt dem: thi jeg er HERREN deres Gud og bønhører dem.
\par 7 Efraim bliver som en Helt, deres Hjerte glædes som af Vin, deres Sønner glædes ved Synet. Deres Hjerte frydes i HERREN;
\par 8 jeg fløjter ad dem og samler dem; thi jeg udløser dem, og de bliver mange som fordum.
\par 9 Blandt Folkeslag strøede jeg dem ud, men de kommer mig i Hu i det fjerne og opfostrer Børn til Hjemfærd.
\par 10 Jeg fører dem hjem fra Ægypten, fra Assur samler jeg dem og bringer dem til Gilead og Libanon, som ikke skal være dem nok.
\par 11 De går gennem Trængselshavet, han slår dets Bølger ned. Alle Nilstrømme tørkner, Assurs Stolthed styrtes, Ægyptens Herskerspir viger.
\par 12 Jeg gør dem stærke i HERREN, de vandrer i hans Navn, så lyder det fra HERREN.

\chapter{11}

\par 1 Libanon, luk Dørerne op, så Ild kan fortære dine Cedre!
\par 2 Klag, Cypres, thi Cedren er faldet. de ædle Træer lagt øde! Klag, I Basans Ege, thi Fredskoven ligger fældet!
\par 3 Hør, hvor Hyrderne klager, thi deres Græsgang er hærget; hør, hvor Løverne brøler, thi Jordans Tykning er hærget.
\par 4 Således sagde HERREN min Gud: Røgt Slagtefårene,
\par 5 hvis Køber slagter dem uden at føle Skyld, og hvis Sælger siger: "HERREN være lovet, jeg blev rig." Og deres Hyrder sparer dem ikke.
\par 6 (Thi jeg vil ikke længer spare Landets Indbyggere, lyder det fra HERREN; men se, jeg lader hvert Menneske falde i sin Hyrdes og sin Konges Hånd; og de skal ødelægge Landet, og jeg vil ingen redde af deres Hånd).
\par 7 Så røgtede jeg Slagtefårene for Fåreprangerne og tog mig to Stave; den ene kaldte jeg "Liflighed", den anden "Bånd"; og jeg røgtede Fårene.
\par 8 (Og jeg ryddede de tre Hyrder af Vejen i een Måned). Så tabte jeg Tålmodigheden med dem, og de blev også kede af mig.
\par 9 Og jeg sagde: "Jeg vil ikke røgte eder; lad dø, hvad dø skal, lad bortkomme, hvad bortkomme skal, og lad de andre æde hverandres Kød!"
\par 10 Så tog jeg Staven, som hed "Liflighed". og sønderbrød den for at bryde den Overenskomst, jeg havde sluttet (med alle Folkeslag;
\par 11 og den blev brudt samme Dag, og Fåreprangerne, som holdt Øje med mig, kendte, at det var HERRENs Ord).
\par 12 Og jeg sagde til dem: "Om I synes, så giv mig min Løn; hvis ikke, så lad være!" Så afvejede de min Løn, tredive Sekel Sølv.
\par 13 Men HERREN sagde til mig: "Kast den til Pottemageren", den dejlige Pris, de har vurderet mig til!" Og jeg tog de tredive Sekel Sølv og kastede dem til Pottemageren i HERRENs Hus.
\par 14 Så sønderbrød jeg den anden Hyrdestav "Bånd" for at bryde Broderskabet imellem Juda og Jerusalem.
\par 15 Siden sagde HERREN til mig: Udstyr dig atter som en Hyrde, en Dåre af en Hyrde!
\par 16 (Thi se, jeg lader en Hyrde fremstå i Landet). Han tager sig ikke af det bortkomne, leder ikke efter det vildfarne, læger ikke det brudte og har ikke Omhu for det sunde, men spiser Kødef af de fede Dyr og river Klovene af dem.
\par 17 Ve, min Dåre af en Hyrde, som svigter Fårene! Et Sværd imod hans Arm og hans højre Øje! Hans Arm skal vorde vissen, hans højre Øje blindes.

\chapter{12}

\par 1 Et Udsagn; HERRENs Ord om Israel. Det lyder fra HERREN; som udspændte Himmelen, grundfæstede Jorden og dannede Menneskets Ånd i dets Indre:
\par 2 Se, jeg gør Jerusalem til et berusende Bæger for alle Folkeslag trindt om; også Juda skal være med til at belejre Jerusalem.
\par 3 På hin Dag gør jeg Jerusalem til Løftesten for alle Folkeslag - enhver, som løfter den, skal rive sig på den! Og alle Jordens Folk skal samle sig imod det.
\par 4 På hin Dag, lyder det fra HERREN, slår jeg alle Heste med Angst og Rytterne med Vanvid; Judas Hus åbner jeg Øjnene på, men alle Folkeslagene slår jeg med Blindhed.
\par 5 Og Judas Stammer skal tænke: "Jerusalems Indbyggere er stærke i Hærskarers HERRE, deres Gud."
\par 6 På hin Dag gør jeg Judas Stammer til en Ildgryde mellem Brændestykker, et brændende Blus mellem Neg, og de skal æde til højre og venstre alle Folkeslag trindt om, og Jerusalem bliver roligt på sit Sted, i Jerusalem.
\par 7 Så giver HERREN først Judas Telte Sejr, for at Davids Hus og Jerusalems Indbyggere ikke skal vinde større Ry end Juda.
\par 8 På hin Dag værner HERREN om Jerusalems Indbyggere, og den skrøbeligste iblandt dem skal på hin Dag blive som David, men Davids Hus som Gud, som HERRENs Engel foran dem.
\par 9 På hin Dag vil jeg søge at tilintetgøre alle de Folk, som kommer imod Jerusalem.
\par 10 Og så udgyder jeg over Davids Hus og Jerusalems Indbyg- gere Nådens og Bønnens Ånd, så de ser hen til ham, de har gennemstunget, og sørger over ham, som man sørger over en enbåren Søn, og holder Klage over ham, son1 man holder Klage over den førstefødte.
\par 11 På hin Dag skal Sorgen blive stor i Jerusalem som Sorgen over Hadadrimmon i Megiddos Dal.
\par 12 Landet skal sørge, hver Slægt for sig, Davids Hus's Slægt for sig og deres Kvinder for sig, Natans Hus's Slægt for sig og deres Kvinder for sig,
\par 13 Levis Hus's Slægt for sig og deres Kvinder for sig, Sjim'iternes Slægt for sig og deres Kvinder for sig,
\par 14 alle de tiloversblevne Slægter hver for sig og deres Kvinder for sig.

\chapter{13}

\par 1 På hin Dag skal en Kilde vælde frem for Davids Hus og Jerusalems Indbyggere mod Synd og Urenhed.
\par 2 Og på hin bag, lyder det fra Hærskarers HERRE, udrydder jeg Afgudernes Navne af Landet, så de ikke mer skal ihukommes; også Profeterne og Urenhedens Ånd driver jeg ud af Landet.
\par 3 Når nogen da atter profeterer, skal hans egne Forældre, hans Fader og Moder, sige til ham: Du har forbrudt dit Liv, thi du har talt Løgn i HERRENs Navn." Og hans egne Forældre, hans Fader og Moder, skal gennembore ham, når han profeterer.
\par 4 På hin Dag skal hver en Profet skamme sig over sine Syner, når han profeterer, og han skal ikke klæde sig i lådden Kappe for at føre Folk bag Lyset,
\par 5 men sige: "Jeg er ingen Profet; jeg er Bonde og har dyrket Jord fra min Ungdom."
\par 6 Og spørger man ham: "Hvad er det for Sår på dit Bryst?" skal han sige: "Dem fik jeg i mine Boleres Hus."
\par 7 Frem, Sværd, imod min Hyrde, mod Manden, som står mig nær, så lyder det fra Hærskarers HERRE. Hyrden vil jeg slå, så Fårehjorden spredes; mod Drengene løfter jeg Hånden.
\par 8 Og i hele Landet lyder det fra HERREN, skal to Tredjedele udryddes og udånde, men een Tredjedel skal levnes.
\par 9 Og denne Tredjedel fører jeg i Ild og renser den, som man renser Sølv, prøver den, som man prøver Guld. Den skal påkalde mit Navn,og jeg svarer; jeg siger: "Den er mit Folk." Og den skal sige: "HERREN er min Gud."

\chapter{14}

\par 1 Se, en Dag kommer, HERRENs Dag, da dit Bytte skal deles i dig.
\par 2 Da samler jeg alle Folkene til Angreb på Jerusalem; Byen indtages, Husene plyndres, Kvinderne skændes, og Halvdelen af Byens Indbyggere vandrer i Landflygtighed; men Resten af Folket skal ikke udryddes af Byen.
\par 3 Og HERREN drager ud og strider mod disse Folk, som han fordum stred på Kampens bag.
\par 4 På hin Dag står hans Fødder på Oliebjerget østen for Jerusalem, og Oliebjerget skal revne midt over fra Øst til Vest og danne en vældig Dal, idet Bjergets ene Halvdel viger mod Nord, den anden mod Syd.
\par 5 I skal flygte til mine Bjerges Dal, thi Bjergdalen når til Azal. I skal fly, som l flyede for Jordskælvet i Kong Uzzija af Judas Dage. Og HERREN min Gud kommer og alle de Hellige med ham.
\par 6 På hin Dag skal der ikke være Hede eller Kulde og Frost.
\par 7 Det skal være een eneste Dag - HERREN kender den - ikke Dag og Nat; det skal være lyst ved Aftentide.
\par 8 På hin Dag skal rindende Vand vælde frem fra Jerusalem; det halve løber ud i Havet mod Øst, det halve i Havet mod Vest, og det både Sommer og Vinter.
\par 9 Og HERREN skal være Konge over hele Jorden. På hin Dag skal HERREN være een og hans Navn eet.
\par 10 Og hele Landet bliver en Slette fra Geba til Rimmon i Sydlandet; men Jerusalem skal ligge højt på sit gamle Sted. Fra Benjaminsporten til den gamle Ports Sted, til Hjørneporten, og fra Hanan'eltårnet til de kongelige Vinperser skal det være beboet.
\par 11 Der skal ikke mere lægges Band derpå, og Jerusalem skal ligge trygt.
\par 12 Men dette skal være den Plage, HERREN lader ramme alle de Folkeslag, som drager i Leding mod Jerusalem: han lader Kødet rådne på dem i levende Live, Øjnene rådner i deres Øjenhuler og Tungen i deres Mund.
\par 13 På hin Dag skal en vældig HERRENs Rædsel opstå iblandt dem, så de griber fat i og løfter Hånd mod hverandre.
\par 14 Også Juda skal stride i Jerusalem. Og alle Folkenes Rigdom skal samles trindt om fra, Guld, Sølv og Klæder i såre store Måder.
\par 15 Og samme Plage skal ramme Heste, Muldyr, Kameler, Æsler og alt Kvæg i Lejrene der.
\par 16 Men alle de, der bliver tilbage af alle Folkene, som kommer imod Jerusalem, skal År efter År drage derop for at tilbede Kongen, Hærskarers HERRE, og fejre Løvhyttefest.
\par 17 Og dersom nogen af Jordens Slægter ikke drager op til Jerusalem for at tilbede Kongen, Hærskarers HERRE, skal der ikke falde Regn hos dem.
\par 18 Og dersom Ægyptens Slægt ikke drager derop og kommer derhen, så skal de rammes af den Plage, HERREN lader ramme Folkene.
\par 19 Det er Straffen over Ægypterne og alle de Folk, som ikke drager op for at fejre Løvhyttefest.
\par 20 På hin Dag skal der stå på Hestenes Bjælder "Helliget HERREN". Og Gryderne i HERRENs Hus skal være som Offerskålene for Alteret;
\par 21 hver Gryde i Jerusalem og Juda skal være helliget Hærskarers HERRE, så alle de ofrende kan komme og tage af dem og koge deri. Og på hin Dag skal der ikke mere være nogen Kana'anæer i Hærskarers HERREs Hus.


\end{document}