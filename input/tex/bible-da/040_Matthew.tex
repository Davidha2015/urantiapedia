\begin{document}

\title{Matthew}


\chapter{1}

\par 1 Jesu Kristi Davids Søns, Abrahams Søns, Slægtsbog.
\par 2 Abraham avlede Isak; og Isak avlede Jakob; og Jakob avlede Juda og hans Brødre;
\par 3 og Juda avlede Fares og Zara med Thamar; og Fares avlede Esrom; og Esrom avlede Aram;
\par 4 og Aram avlede Aminadab; og Aminadab avlede Nasson; og Nasson avlede Salmon;
\par 5 og Salmon avlede Boas med Rakab; og Boas avlede Obed med Ruth; og Obed avlede Isaj;
\par 6 og Isaj avlede Kong David; og David avlede Salomon med Urias's Hustru;
\par 7 og Salomon avlede Roboam; og Roboam avlede Abia; og Abia avlede Asa;
\par 8 og Asa avlede Josafat; og Josafat avlede Joram; og Joram avlede Ozias;
\par 9 og Ozias avlede Joatham; og Joatham avlede Akas; og Akas avlede Ezekias;
\par 10 og Ezekias avlede Manasse; og Manasse avlede Amon; og Amon avlede Josias;
\par 11 og Josias avlede Jekonias og hans Brødre på den Tid, da Bortførelsen til Babylon fandt Sted.
\par 12 Men efter Bortførelsen til Babylon avlede Jekonias Salathiel; og Salathiel avlede Zorobabel;
\par 13 og Zorobabel avlede Abiud; og Abiud avlede Eliakim: og Eliakim avlede Azor;
\par 14 og Azor avlede Sadok; og Sadok avlede Akim; og Akim avlede Eliud;
\par 15 og Eliud avlede Eleazar; og Eleazar avlede Matthan; og Matthan avlede Jakob;
\par 16 og Jakob avlede Josef, Marias Mand; af hende blev Jesus født, som kaldes Kristus.
\par 17 Altså ere alle Slægtledene fra Abraham indtil David fjorten Slægtled, og fra David indtil Bortførelsen til Babylon fjorten Slægtled, og fra Bortførelsen til Babylon indtil Kristus fjorten Slægtled.
\par 18 Men med Jesu Kristi Fødsel gik det således til. Da Maria, hans Moder, var trolovet med Josef, fandtes hun, førend de kom sammen, at være frugtsommelig af den Helligånd.
\par 19 Men da Josef, hendes Mand, var retfærdig og ikke vilde beskæmme hende offentligt, besluttede han hemmeligt at skille sig fra hende.
\par 20 Men idet han tænkte derpå, se, da viste en Herrens Engel sig for ham i en drøm og sagde: "Josef, Davids Søn! frygt ikke for at tage din Hustru Maria til dig; thi det, som er avlet i hende, er af den Helligånd.
\par 21 Og hun skal føde en Søn, og du skal kalde hans Navn Jesus; thi han skal frelse sit Folk fra deres Synder."
\par 22 Men dette er alt sammen sket, for at det skulde opfyldes, som er talt af Herren ved Profeten, som siger:
\par 23 "Se, Jomfruen skal blive frugtsommelig og føde en Søn, og man skal kalde hans Navn Immanuel", hvilket er udlagt: Gud med os.
\par 24 Men da Josef vågnede op at Søvnen, gjorde han, som Herrens Engel havde befalet ham, og han tog sin Hustru til sig.
\par 25 Og han kendte hende ikke, førend hun havde født sin Søn, den førstefødte, og han kaldte hans Navn Jesus.

\chapter{2}

\par 1 Men da Jesus var født i Bethlehem i Judæa, i Kong Herodes's Dage, se, da kom der vise fra Østerland til Jerusalem og sagde:
\par 2 "Hvor er den Jødernes Konge, som er født? thi vi have set hans Stjerne i Østen og ere komne for at tilbede ham."
\par 3 Men da Kong Herodes hørte det, blev han forfærdet, og hele Jerusalem med ham;
\par 4 og han forsamlede alle Folkets Ypperstepræster og skriftkloge og adspurgte dem, hvor Kristus skulde fødes.
\par 5 Og de sagde til ham: "I Bethlehem i Judæa; thi således er der skrevet ved Profeten:
\par 6 Og du, Bethlehem i Judas Land, er ingenlunde den mindste iblandt Judas Fyrster; thi af dig skal der udgå en Fyrste, som skal vogte mit Folk Israel."
\par 7 Da kaldte Herodes hemmeligt de vise og fik af dem nøje Besked om Tiden, da Stjernen havde ladet sig til Syne.
\par 8 Og han sendte dem til Bethlehem og sagde: "Går hen og forhører eder nøje om Barnet; men når I have fundet det, da forkynder mig det, for at også jeg kan komme og tilbede det."
\par 9 Men da de havde hørt Kongen, droge de bort; og se, Stjernen, som de havde set i Østen, gik foran dem, indtil den kom og stod oven over, hvor Barnet var.
\par 10 Men da de så Stjernen, bleve de såre meget glade.
\par 11 Og de gik ind i Huset og så Barnet med dets Moder Maria og faldt ned og tilbade det og oplode deres Gemmer og ofrede det Gaver, Guld og Røgelse og Myrra.
\par 12 Og da de vare blevne advarede af Gud i en Drøm, at de ikke skulde vende tilbage til Herodes, droge de ad en anden Vej tilbage til deres Land.
\par 13 Men da de vare dragne bort, se, da viser en Herrens Engel sig i en Drøm for Josef og siger: "Stå op, og tag Barnet og dets Moder med dig og fly til Ægypten og bliv der, indtil jeg siger dig til; thi Herodes vil søge efter Barnet for at dræbe det."
\par 14 Og han stod op og tog Barnet og dets Moder med sig om Natten og drog bort til Ægypten.
\par 15 Og han var der indtil Herodes's Død, for at det skulde opfyldes, som er talt af Herren ved Profeten, der siger: "Fra Ægypten kaldte jeg min Søn."
\par 16 Da Herodes nu så, at han var bleven skuffet af de vise, blev han såre vred og sendte Folk hen og lod alle Drengebørn ihjelslå, som vare i Bethlehem og i hele dens Omegn, fra to År og derunder, efter den Tid, som han havde fået Besked om af de vise.
\par 17 Da blev det opfyldt,som er talt ved Profeten Jeremias,som siger:
\par 18 "En Røst blev hørt i Rama, Gråd og megen Jamren; Rakel græd over sine Børn og vilde ikke lade sig trøste, thi de ere ikke mere."
\par 19 Men da Herodes var død, se, da viser en Herrens Engel sig i en Drøm for Josef i Ægypten og siger:
\par 20 "Stå op, og tag Barnet og dets Moder med dig, og drag til Israels Land; thi de ere døde, som efterstræbte Barnets Liv."
\par 21 Og han stod op og tog Barnet og dets Moder med sig og kom til Israels Land.
\par 22 Men da han hørte, at Arkelaus var Konge over Judæa i sin Fader Herodes's Sted, frygtede han for at komme derhen; og han blev advaret af Gud i en Drøm og drog bort til Galilæas Egne.
\par 23 Og han kom og tog Bolig i en By, som kaldes Nazareth, for at det skulde opfyldes, som er talt ved Profeterne, at han skulde kaldes Nazaræer.

\chapter{3}

\par 1 Men i de Dage fremstår Johannes Døberen og prædiker i Judæas Ørken og siger:
\par 2 "Omvender eder, thi Himmeriges Rige er kommet nær."
\par 3 Thi han er den, om hvem der er talt ved Profeten Esajas, som siger: "Der er en Røst af en, som råber i Ørkenen: Bereder Herrens Vej, gører hans Stier jævne!"
\par 4 Men han, Johannes, havde sit Klædebon af Kamelhår og et Læderbælte om sin Lænd; og hans Føde var Græshopper og vild Honning.
\par 5 Da drog Jerusalem ud til ham og hele Judæa og hele Omegnen om Jordan.
\par 6 Og de bleve døbte af ham i Floden Jordan, idet de bekendte deres Synder.
\par 7 Men da han så mange af Farisæerne og Saddukæerne komme til hans Dåb, sagde han til dem: "I Øgleunger! hvem har lært eder at fly fra den kommende Vrede?
\par 8 Bærer da Frugt, som er Omvendelsen værdig,
\par 9 og mener ikke at kunne sige ved eder selv: Vi have Abraham til Fader; thi jeg siger eder, at Gud kan opvække Abraham Børn af disse Sten.
\par 10 Men Øksen ligger allerede ved Roden af Træerne; så bliver da hvert Træ, som ikke bærer god Frugt, omhugget og kastet i Ilden.
\par 11 Jeg døber eder med Vand til Omvendelse, men den, som kommer efter mig, er stærkere end jeg, han, hvis Sko jeg ikke er værdig at bære; han skal døbe eder med den Helligånd og Ild.
\par 12 Hans Kasteskovl er i hans Hånd, og han skal gennemrense sin Lo og samle sin Hvede i Laden; men Avnerne skal han opbrænde med uslukkelig Ild."
\par 13 Da kommer Jesus fra Galilæa til Jordan til Johannes for at døbes af ham.
\par 14 Men Johannes vilde formene ham det og sagde: "Jeg trænger til at døbes af dig, og du kommer til mig!"
\par 15 Men Jesus svarede og sagde til ham: "Tilsted det nu; thi således sømmer det sig for os at fuldkomme al Retfærdighed." Da tilsteder han ham det.
\par 16 Men da Jesus var bleven døbt, steg han straks op af Vandet, og se, Himlene åbnedes for ham, og han så Guds Ånd dale ned som en Due og komme over ham.
\par 17 Og se, der kom en Røst fra Himlene, som sagde: "Denne er min Søn, den elskede, i hvem jeg har Velbehag."

\chapter{4}

\par 1 Da blev Jesus af Ånden ført op i Ørkenen for at fristes af Djævelen.
\par 2 Og da han havde fastet fyrretyve Dage og fyrretyve Nætter, blev han omsider hungrig.
\par 3 Og Fristeren gik til ham og sagde: "Er du Guds Søn, da sig, at disse Sten skulle blive Brød."
\par 4 Men han svarede og sagde: "Der er skrevet: Mennesket skal ikke leve af Brød alene, men af hvert Ord, som udgår igennem Guds Mund."
\par 5 Da. tager Djævelen ham med sig til den hellige Stad og stiller ham på Helligdommens Tinde og siger til ham:
\par 6 "Er du Guds Søn, da kast dig herned; thi der er skrevet: Han skal give sine Engle Befaling om dig, og de skulle bære dig på Hænder, for at du ikke skal støde din Fod på nogen Sten."
\par 7 Jesus sagde til ham: "Der er atter skrevet: Du må ikke friste Herren din Gud."
\par 8 Atter tager Djævelen ham med sig op på et såre højt Bjerg og viser ham alle Verdens Riger og deres Herlighed; og han sagde til ham:
\par 9 "Alt dette vil jeg give dig, dersom du vil falde ned og tilbede mig."
\par 10 Da siger Jesus til ham: "Vig bort, Satan! thi der er skrevet: Du skal tilbede Herren din Gud og tjene ham alene."
\par 11 Da forlader Djævelen ham, og se, Engle kom til ham og tjente ham.
\par 12 Men da Jesus hørte, at Johannes var kastet i Fængsel, drog han bort til Galilæa.
\par 13 Og han forlod Nazareth og kom og tog Bolig i Kapernaum, som ligger ved Søen, i Sebulons og Nafthalis Egne,
\par 14 for at det skulde opfyldes, som er talt ved Profeten Esajas, som siger:
\par 15 "Sebulons Land og Nafthalis Land langs Søen, Landet hinsides Jordan, Hedningernes Galilæa,
\par 16 det Folk, som sad i Mørke, har set et stort Lys, og for dem, som sad i Dødens Land og Skygge, for dem er der opgået et Lys."
\par 17 Fra den Tid begyndte Jesus at prædike og sige: "Omvender eder, thi Himmeriges Rige er kommet nær."
\par 18 Men da han vandrede ved Galilæas Sø, så han to Brødre, Simon, som kaldes Peter, og Andreas, hans Broder, i Færd med at kaste Garn i Søen; thi de vare Fiskere.
\par 19 Og han siger til dem: "Følger efter mig, så vil jeg gøre eder til Menneskefiskere."
\par 20 Og de forlode straks Garnene og fulgte ham.
\par 21 Og da han derfra gik videre, så han to andre Brødre, Jakob, Zebedæus's Søn, og Johannes, hans Broder, i Skibet med deres Fader Zebedæus, i Færd med at bøde deres Garn, og han kaldte på dem.
\par 22 Og de forlode straks Skibet og deres Fader og fulgte ham.
\par 23 Og Jesus gik omkring i hele Galilæa, idet han lærte i deres Synagoger og prædikede Rigets Evangelium og helbredte enhver Sygdom og enhver Skrøbelighed iblandt Folket.
\par 24 Og hans Ry kom ud over hele Syrien; og de bragte til ham alle dem, som lede af mange Hånde Sygdomme og vare plagede af Lidelser, både besatte og månesyge og værkbrudne; og han helbredte dem.
\par 25 Og store Skarer fulgte ham fra Galilæa og Dekapolis og Jerusalem og Judæa og fra Landet hinsides Jordan.

\chapter{5}

\par 1 Men da han så Skarerne, steg han op på Bjerget; og da han havde sat sig, gik hans Disciple hen til ham,
\par 2 og han oplod sin Mund, lærte dem og sagde:
\par 3 "Salige ere de fattige i Ånden, thi Himmeriges Rige er deres.
\par 4 Salige ere de, som sørge, thi de skulle husvales.
\par 5 Salige ere de sagtmodige, thi de skulle arve Jorden.
\par 6 Salige ere de, som hungre og tørste efter Retfærdigheden, thi de skulle mættes.
\par 7 Salige ere de barmhjertige, thi dem skal vises Barmhjertighed.
\par 8 Salige ere de rene af Hjertet, thi de skulle se Gud.
\par 9 Salige ere de, som stifte Fred, thi de skulle kaldes Guds Børn.
\par 10 Salige ere de, som ere forfulgte for Retfærdigheds Skyld, thi Himmeriges Rige er deres.
\par 11 Salige ere I, når man håner og forfølger eder og lyver eder alle Hånde ondt på for min Skyld.
\par 12 Glæder og fryder eder, thi eders Løn skal være stor i Himlene; thi således have de forfulgt Profeterne, som vare før eder.
\par 13 I ere Jordens Salt; men dersom Saltet mister sin Kraft, hvormed skal det da saltes? Det duer ikke til andet end at kastes ud og nedtrædes af Menneskene.
\par 14 I ere Verdens Lys; en Stad, som ligger på et Bjerg, kan ikke skjules.
\par 15 Man tænder heller ikke et Lys og sætter det under Skæppen, men på Lysestagen; så skinner det for alle dem, som ere i Huset.
\par 16 Lader således eders Lys skinne for Menneskene, at de må se eders gode Gerninger og ære eders Fader, som er i Himlene.
\par 17 Mener ikke, at jeg er kommen for at nedbryde Loven eller Profeterne;jeg er ikke kommen for at nedbryde, men for at fuldkomme.
\par 18 Thi sandelig, siger jeg eder, indtil Himmelen og Jorden forgår, skal end ikke det mindste Bogstav eller en Tøddel forgå af Loven, indtil det er sket alt sammen.
\par 19 Derfor, den, som bryder et at de mindste af disse Bud og lærer Menneskene således, han skal kaldes den mindste i Himmeriges Rige; men den, som gør dem og lærer dem, han skal kaldes stor i Himmeriges Rige.
\par 20 Thi jeg siger eder: Uden eders Retfærdighed overgår de skriftkloges og Farisæernes, komme I ingenlunde ind i Himmeriges Rige.
\par 21 I have hørt, at der er sagt til de gamle: Du må ikke slå ihjel, men den, som slår ihjel, skal være skyldig for Dommen.
\par 22 Men jeg siger eder, at hver den, som bliver vred på sin Broder uden Årsag, skal være skyldig for Dommen; og den, som siger til sin Broder: Raka! skal være skyldig for Rådet; og den, som siger: Du Dåre! skal være skyldig til Helvedes Ild.
\par 23 Derfor, når du ofrer din Gave på Alteret og der kommer i Hu, at din Broder har noget imod dig,
\par 24 så lad din Gave blive der foran Alteret, og gå hen, forlig dig først med din Broder, og kom da og offer din Gave!
\par 25 Vær velvillig mod din Modpart uden Tøven, medens du er med ham på Vejen, for at Modparten ikke skal overgive dig til Dommeren, og Dommeren til Tjeneren, og du skal kastes i Fængsel.
\par 26 Sandelig, siger jeg dig, du skal ingenlunde komme ud derfra, førend du får betalt den sidste Hvid.
\par 27 I have hørt, at der er sagt: Du må ikke bedrive Hor.
\par 28 Men jeg siger eder, at hver den, som ser på en Kvinde for at begære hende, har allerede bedrevet Hor med hende i sit Hjerte.
\par 29 Men dersom dit højre Øje forarger dig, så riv det ud, og kast det fra dig; thi det er bedre for dig, at eet af dine Lemmer fordærves, end at hele dit Legeme bliver kastet i Helvede.
\par 30 Og om din højre Hånd forarger dig, så hug den af og kast den fra dig; thi det er bedre for dig, at eet af dine Lemmer fordærves, end at hele dit Legeme kommer i Helvede.
\par 31 Og der er sagt: Den, som skiller sig fra sin Hustru, skal give hende et Skilsmissebrev.
\par 32 Men jeg siger eder, at enhver, som skiller sig fra sin Hustru, uden for Hors Skyld, gør, at hun bedriver Hor, og den, som tager en fraskilt Kvinde til Ægte, bedriver Hor.
\par 33 I have fremdeles hørt, at der er sagt til de gamle: Du må ikke gøre nogen falsk Ed, men du skal holde Herren dine Eder.
\par 34 Men jeg siger eder, at I må aldeles ikke sværge, hverken ved Himmelen, thi den er Guds Trone,
\par 35 ej heller ved Jorden, thi den er hans Fodskammel, ej heller ved Jerusalem, thi det er den store Konges Stad.
\par 36 Du må heller ikke sværge ved dit Hoved, thi du kan ikke gøre et eneste Hår hvidt eller sort.
\par 37 Men eders Tale skal være ja, ja, nej, nej; hvad der er ud over dette, er af det onde.
\par 38 I have hørt, at der er sagt: Øje for Øje, og Tand for Tand.
\par 39 Men jeg siger eder, at I må ikke sætte eder imod det onde; men dersom nogen giver dig et Slag på din højre Kind, da vend ham også den anden til!
\par 40 Og dersom nogen vil gå i Rette med dig og tage din Kjortel, lad ham da også få Kappen!
\par 41 Og dersom nogen tvinger dig til at gå een Mil,da gå to med ham!
\par 42 Giv den, som beder dig, og vend dig ikke fra den, som vil låne af dig.
\par 43 I have hørt, at der er sagt: Du skal elske din Næste og hade din Fjende.
\par 44 Men jeg siger eder: Elsker eders Fjender, velsigner dem, som forbande eder, gører dem godt, som hade eder, og beder for dem, som krænke eder og forfølge eder,
\par 45 for at I må vorde eders Faders Børn, han, som er i Himlene; thi han lader sin Sol opgå over onde og gode og lader det regne over retfærdige og uretfærdige.
\par 46 Thi dersom I elske dem, som elske eder, hvad Løn have I da? Gøre ikke også Tolderne det samme?
\par 47 Og dersom I hilse eders Brødre alene, hvad stort gøre I da? Gøre ikke også Hedningerne det samme?
\par 48 Værer da I fuldkomne, ligesom eders himmelske Fader er fuldkommen.

\chapter{6}

\par 1 Vogter eder at i ikke øve eders Retfærdighed for Menneskene for at beskues af dem; ellers have I ikke Løn hos eders Fader, som er Himlene.
\par 2 Derfor, når du giver Almisse, må du ikke lade blæse i Basun foran dig, som Hyklerne gøre i Synagogerne og på Gaderne, for at de kunne blive ærede af Menneskene; sandelig, siger jeg eder, de have allerede fået deres Løn.
\par 3 Men når du giver Almisse, da lad din venstre Hånd ikke vide, hvad din højre gør,
\par 4 for at din Almisse kan være i Løndom, og din Fader, som ser i Løndom, skal betale dig.
\par 5 Og når I bede, skulle I ikke være som Hyklerne; thi de stå gerne i Synagogerne og på Gadehjørnerne og bede, for at de kunne vise sig for Menneskene; sandelig, siger jeg eder, de have allerede fået deres Løn.
\par 6 Men du, når du beder, da gå ind i dit Kammer, og luk din Dør, og bed til din Fader, som er i Løndom, og din Fader, som ser i Løndom, skal betale dig.
\par 7 Men når I bede, må I ikke bruge overflødige Ord som Hedningerne; thi de mene, at de skulle blive bønhørte for deres mange Ord.
\par 8 Ligner derfor ikke dem; thi eders Fader ved, hvad I trænge til, førend I bede ham,
\par 9 Derfor skulle I bede således: Vor Fader, du, som er i Himlene! Helliget vorde dit Navn;
\par 10 komme dit Rige; ske din Villie, som i Himmelen således også på Jorden;
\par 11 giv os i dag vort daglige Brød:
\par 12 og forlad os vor Skyld, som også vi forlade vore Skyldnere;
\par 13 og led os ikke i Fristelse; men fri os fra det onde; (thi dit er Riget og Magten og Æren i Evighed! Amen.)
\par 14 Thi forlade I Menneskene deres Overtrædelser, vil eders himmelske Fader også forlade eder;
\par 15 men forlade I ikke Menneskene deres Overtrædelser, vil eders Fader ikke heller forlade eders Overtrædelser.
\par 16 Og når I faste, da ser ikke bedrøvede ud som Hyklerne; thi de gøre deres Ansigter ukendelige, for at de kunne vise sig for Menneskene som fastende; sandelig, siger jeg eder, de have allerede fået deres Løn.
\par 17 Men du, når du faster, da salv dit Hoved, og to dit Ansigt,
\par 18 for at du ikke skal vise dig for Menneskene som fastende, men for din Fader, som er i Løndom; og din Fader, som ser i Løndom, skal betale dig.
\par 19 Samler eder ikke Skatte på Jorden, hvor Møl og Rust fortære, og hvor Tyve bryde ind og stjæle;
\par 20 men samler eder Skatte i Himmelen, hvor hverken Møl eller Rust fortære, og hvor Tyve ikke bryde ind og stjæle.
\par 21 Thi hvor din Skat er, der vil også dit Hjerte være.
\par 22 Øjet er Legemets Lys; derfor, dersom dit Øje er sundt, bliver hele dit Legeme lyst;
\par 23 men dersom dit Øje er dårligt, bliver hele dit Legeme mørkt.
\par 24 Ingen kan tjene to Herrer; thi han må enten hade den ene og elske den anden eller holde sig til den ene og ringeagte den anden. I kunne ikke tjene Gud og Mammon.
\par 25 Derfor siger jeg eder: Bekymrer eder ikke for eders Liv, hvad I skulle spise, eller hvad I skulle drikke; ikke heller for eders Legeme, hvad I skulle iføre eder. Er ikke Livet mere end Maden, og Legemet mere end Klæderne?
\par 26 Ser på Himmelens Fugle; de så ikke og høste ikke og sanke ikke i Lader, og eders himmelske Fader føder dem; ere I ikke meget mere værd end de?
\par 27 Og hvem af eder kan ved at bekymre sig lægge een Alen til sin Vækst?
\par 28 Og hvorfor bekymre I eder for Klæder? Betragter Lillierne på Marken, hvorledes de vokse; de arbejde ikke og spinde ikke;
\par 29 men jeg siger eder, at end ikke Salomon i al sin Herlighed var klædt som en af dem.
\par 30 Klæder da Gud således det Græs på Marken, som står i dag og i Morgen kastes i Ovnen, skulde han da ikke meget mere klæde eder, I lidettroende?
\par 31 Derfor må I ikke bekymre eder og sige: Hvad skulle vi spise? eller: Hvad skulle vi drikke? eller: Hvormed skulle vi klæde os?
\par 32 - efter alt dette søge jo Hedningerne -. Thi eders himmelske Fader ved, at I have alle disse Ting nødig.
\par 33 Men søger først Guds Rige og hans Retfærdighed, så skulle alle disse Ting gives eder i Tilgift.
\par 34 Bekymrer eder derfor ikke for den Dag i Morgen; thi den Dag i Morgen skal bekymre sig for sig selv. Hver Dag har nok i sin Plage.

\chapter{7}

\par 1 Dømmer ikke,for at I ikke skulle dømmes; thi med hvad Dom I dømme, skulle I dømmes,
\par 2 og med hvad Mål I måle, skal der tilmåles eder.
\par 3 Men hvorfor ser du Skæven, som er i din Broders Øje, men Bjælken i dit eget Øje bliver du ikke var?
\par 4 Eller hvorledes kan du sige til din Broder: Lad mig drage Skæven ud af dit Øje; og se, Bjælken er i dit eget Øje.
\par 5 Du Hykler! drag først Bjælken ud af dit Øje, og da kan du se klart til at tage Skæven ud af din Broders Øje.
\par 6 Giver ikke Hunde det hellige, kaster ikke heller eders Perler for Svin, for at de ikke skulle nedtræde dem med deres Fødder og vende sig og sønderrive eder.
\par 7 Beder, så skal eder gives; søger, så skulle I finde; banker på, så skal der lukkes op for eder.
\par 8 Thi hver den, som beder, han får, og den, som søger, han finder, og den, som banker på, for ham skal der lukkes op.
\par 9 Eller hvilket Menneske er der iblandt eder, som, når hans Søn beder ham om Brød, vil give ham en Sten?
\par 10 Eller når han beder ham om en Fisk, mon han da vil give ham en Slange?
\par 11 Dersom da I, som ere onde, vide at give eders Børn gode Gaver, hvor meget mere skal eders Fader, som er i Himlene, give dem gode Gaver, som bede ham!
\par 12 Altså, alt hvad I ville, at Menneskene skulle gøre imod eder, det skulle også I gøre imod dem; thi dette er Loven og Profeterne.
\par 13 Går ind ad den snævre Port; thi den Port er vid, og den Vej er bred,som fører til Fortabelsen, og de ere mange, som gå ind ad den;
\par 14 thi den Port er snæver, og den Vej er trang, som fører til Livet og de er få, som finde den
\par 15 Men vogter eder for de falske Profeter, som komme til eder i Fåreklæder, men indvortes ere glubende Ulve.
\par 16 Af deres Frugter skulle I kende dem. Sanker man vel Vindruer af Torne eller Figener af Tidsler?
\par 17 Således bærer hvert godt Træ gode Frugter, men det rådne Træ bærer slette Frugter.
\par 18 Et godt Træ kan ikke bære slette Frugter, og et råddent Træ kan ikke bære gode Frugter.
\par 19 Hvert Træ, som ikke bærer god Frugt, omhugges og kastes i Ilden.
\par 20 Altså skulle I kende dem af deres Frugter.
\par 21 Ikke enhver, som siger til mig: Herre, Herre! skal komme ind i Himmeriges Rige, men den, der gør min Faders Villie, som er i Himlene.
\par 22 Mange skulle sige til mig på hin Dag: Herre, Herre! have vi ikke profeteret ved dit Navn, og have vi ikke uddrevet onde Ånder ved dit Navn, og have vi ikke gjort mange kraftige Gerninger ved dit Navn?
\par 23 Og da vil jeg bekende for dem Jeg kendte eder aldrig; viger bort fra mig, I, som øve Uret!
\par 24 Derfor, hver den, som hører disse mine Ord og gør efter dem, ham vil jeg ligne ved en forstandig Mand, som byggede sit Hus på Klippen,
\par 25 og Skylregnen faldt, og Floderne kom, og Vindene blæste og sloge imod dette Hus, og det faldt ikke; thi det var grundfæstet på Klippen.
\par 26 Og hver den, som hører disse mine Ord og ikke gør efter dem, skal lignes ved en Dåre, som byggede sit Hus på Sandet,
\par 27 og Skylregnen faldt, og Floderne kom, og Vindene blæste og stødte imod dette Hus, og det faldt, og dets Fald var stort."
\par 28 Og det skete, da Jesus havde fuldendt disse Ord, vare Skarerne slagne af Forundring over hans Lære;
\par 29 thi han lærte dem som en, der havde Myndighed, og ikke som deres skriftkloge.

\chapter{8}

\par 1 Men da han var gået ned ad Bjerget, fulgte store Skarer ham.
\par 2 Og se, en spedalsk kom, faldt ned for ham og sagde: "Herre! om du vil, så kan du rense mig."
\par 3 Og han udrakte Hånden, rørte ved ham og sagde: "Jeg vil; bliv ren!" Og straks blev han renset for sin Spedalskhed
\par 4 Og Jesus siger til ham: "Se til, at du ikke siger det til nogen; men gå hen, fremstil dig selv for Præsten, og offer den Gave, som Moses har befalet, til Vidnesbyrd for dem."
\par 5 Men da han gik ind i Kapernaum, trådte en Høvedsmand hen til ham, bad ham og sagde:
\par 6 "Herre! min Dreng ligger hjemme værkbruden og, pines svarlig."
\par 7 Jesus siger til ham: "Jeg vil komme og helbrede ham."
\par 8 Og Høvedsmanden svarede og sagde: "Herre! jeg er ikke værdig til, at du skal gå ind under mit Tag; men sig det blot med et Ord, så bliver min Dreng helbredt.
\par 9 Jeg er jo selv et Menneske, som står under Øvrighed og har Stridsmænd under mig; og siger jeg til den ene: Gå! så går han; og til den anden: Kom! så kommer han; og til min Tjener: Gør dette! så gør han det."
\par 10 Men da Jesus hørte det, forundrede han sig og sagde til dem, som fulgte ham: "Sandelig, siger jeg eder, end ikke i Israel har jeg fundet så stor en Tro.
\par 11 Men jeg siger eder, at mange skulle komme fra Øster og Vester og sidde til Bords med Abraham og Isak og Jakob i Himmeriges Rige.
\par 12 Men Rigets Børn skulle kastes ud i Mørket udenfor; der skal der være Gråd og Tænders Gnidsel."
\par 13 Og Jesus sagde til Høvedsmanden:"Gå bort,dig ske, som du troede!" Og Drengen blev helbredt i den samme Time.
\par 14 Og Jesus kom ind i Peters Hus og så, at hans Svigermoder lå og havde Feber.
\par 15 Og han rørte ved hendes Hånd, og Feberen forlod hende, og hun stod op og vartede ham op.
\par 16 Men da det var blevet Aften, førte de mange besatte til ham, og han uddrev Ånderne med et Ord og helbredte alle de syge;
\par 17 for at det skulde opfyldes, som er talt ved Profeten Esajas, der siger: "Han tog vore Skrøbeligheder og bar vore Sygdomme."
\par 18 Men da Jesus så store Skarer omkring sig, befalede han at fare over til hin Side.
\par 19 Og der kom een, en skriftklog, og sagde til ham: "Mester! jeg vil følge dig, hvor du end går hen."
\par 20 Og Jesus siger til ham: "Ræve have Huler, og Himmelens Fugle Reder; men Menneskesønnen har ikke det, hvortil han kan hælde sit Hoved."
\par 21 Men en anden af disciplene sagde til ham: "Herre! tilsted mig først at gå hen og begrave min Fader."
\par 22 Men Jesus siger til ham: "Følg mig, og lad de døde begrave deres døde!"
\par 23 Og da han gik om Bord i Skibet, fulgte hans Disciple ham.
\par 24 Og se, det blev en stærk Storm på Søen, så at Skibet skjultes af Bølgerne; men han sov.
\par 25 Og de gik hen til ham, vækkede ham og sagde: "Herre, frels os! vi forgå."
\par 26 Og han siger til dem: "Hvorfor ere I bange, I lidettroende?" Da stod han op og truede Vindene og Søen,og det blev ganske blikstille.
\par 27 Men Menneskene forundrede sig og sagde: "Hvem er dog denne, siden både Vindene og Søen ere ham lydige?"
\par 28 Og da han kom over til hin Side til Gadarenernes Land, mødte ham to besatte, som kom ud fra Gravene, og de vare såre vilde, så at ingen kunde komme forbi ad den Vej.
\par 29 Og se, de råbte og sagde: "Hvad have vi med dig at gøre, du Guds Søn? Er du kommen hid før Tiden for at pine os?"
\par 30 Men der var langt fra dem en stor Hjord Svin, som græssede.
\par 31 Og de onde Ånder bade ham og sagde: "Dersom du uddriver os, da send os i Svinehjorden!"
\par 32 Og han sagde til dem: "Går!" Men de fore ud og fore i Svinene; og se, hele Hjorden styrtede sig ned over Brinken ud i Søen og døde i Vandet.
\par 33 Men Hyrderne flyede og gik hen i Byen og fortalte det alt sammen, og hvorledes det var gået til med de besatte.
\par 34 Og se, hele Byen gik ud for at møde Jesus; og da de så ham, bade de ham om; at han vilde gå bort fra deres Egn.

\chapter{9}

\par 1 Og han gik om Bord i et Skib og for over og kom til sin egen By.
\par 2 Og se, de bare til ham en værkbruden, som lå på en Seng; og da Jesus så deres Tro, sagde han til den værkbrudne: "Søn! vær frimodig, dine Synder forlades dig."
\par 3 Og se, nogle af de skriftkloge sagde ved sig selv: "Denne taler bespotteligt."
\par 4 Og da Jesus så deres Tanker, sagde han: "Hvorfor tænke I ondt i eders Hjerter?
\par 5 Thi hvilket er lettest at sige: Dine Synder forlades dig, eller at sige: Stå op og gå?
\par 6 Men for at I skulle vide, at Menneskesønnen har Magt på Jorden til at forlade Synder," da siger han til den værkbrudne: "Stå op, og tag din Seng, og gå til dit Hus!"
\par 7 Og han stod op og gik bort til sit Hus.
\par 8 Men da Skarerne så det, frygtede de og priste Gud, som havde givet Menneskene en sådan Magt.
\par 9 Og da Jesus gik videre derfra, så han en Mand, som hed Matthæus, sidde ved Toldboden; og han siger til ham: "Følg mig!" Og han stod op og fulgte ham.
\par 10 Og det skete, da han sad til Bords i Huset, se, da kom der mange Toldere og Syndere og sade til Bords med Jesus og hans Disciple.
\par 11 Og da Farisæerne så det, sagde de til hans Disciple: "Hvorfor spiser eders Mester med Toldere og Syndere?"
\par 12 Men da Jesus hørte det, sagde han: "De raske trænge ikke til Læge, men de syge.
\par 13 Men går hen og lærer, hvad det vil sige: Jeg har Lyst til Barmhjertighed og ikke til Offer; thi jeg er ikke kommen for at kalde retfærdige, men Syndere,"
\par 14 Da komme Johannes's Disciple til ham og sige: "Hvorfor faste vi og Farisæerne meget, men dine Disciple faste ikke?"
\par 15 Og Jesus sagde til dem: "Kunne Brudesvendene sørge, så længe Brudgommen er hos dem? Men der skal komme Dage, da Brudgommen bliver tagen fra dem, og da skulle de faste.
\par 16 Men ingen sætter en Lap af uvalket Klæde på et gammelt Klædebon; thi Lappen river Klædebonnet itu, og der bliver et værre Hul.
\par 17 Man kommer heller ikke ung Vin på gamle Læderflasker, ellers sprænges Læderflaskerne, og Vinen spildes, og Læderflaskerne ødelægges; men man kommer ung Vin på nye Læderflasker, så blive begge Dele bevarede."
\par 18 Medens han talte dette til dem, se, da kom der en Forstander og faldt ned for ham og sagde: "Min Datter er lige nu død; men kom og læg din Hånd på hende, så bliver hun levende."
\par 19 Og Jesus stod op og fulgte ham med sine Disciple.
\par 20 Og se, en Kvinde, som havde haft Blodflod i tolv År, trådte hen bagfra og rørte ved Fligen af hans Klædebon;
\par 21 thi hun sagde ved sig selv: "Dersom jeg blot rører ved hans Klædebon, bliver jeg frelst."
\par 22 Men Jesus vendte sig om, og da han så hende, sagde han: "Datter! vær frimodig, din Tro har frelst dig." Og Kvinden blev frelst fra den samme Time.
\par 23 Og da Jesus kom til Forstanderens Hus og så Fløjtespillerne og Hoben, som larmede, sagde han:
\par 24 "Gå bort, thi Pigen er ikke død, men hun sover." Og de lo ad ham.
\par 25 Men da Hoben var dreven ud, gik han ind og tog hende ved Hånden; og Pigen stod op.
\par 26 Og Rygtet herom kom ud i hele den Egn.
\par 27 Og da Jesus gik bort derfra, fulgte der ham to blinde, som råbte og sagde: "Forbarm dig over os, du Davids Søn!"
\par 28 Men da han kom ind i Huset, gik de blinde til ham; og Jesus siger til dem: "Tro I, at jeg kan gøre dette?"De siger til ham:"Ja,Herre!"
\par 29 Da rørte han ved deres Øjne og sagde: "Det ske eder efter eders Tro!"
\par 30 Og deres Øjne bleve åbnede. Og Jesus bød dem strengt og sagde: "Ser til, lad ingen få det at vide."
\par 31 Men de gik ud og udbredte Rygtet om ham i hele den Egn.
\par 32 Men da disse gik ud, se, da førte de til ham et stumt Menneske, som var besat.
\par 33 Og da den onde Ånd var uddreven, talte den stumme. Og Skarerne forundrede sig og sagde: "Aldrig er sådant set i Israel."
\par 34 Men Farisæerne sagde: "Ved de onde Ånders Fyrste uddriver han de onde Ånder."
\par 35 Og Jesus gik omkring i alle Byerne og Landsbyerne, lærte i deres Synagoger og prædikede Rigets Evangelium og helbredte enhver Sygdom og enhver Skrøbelighed.
\par 36 Men da han så Skarerne, ynkedes han inderligt over dem; thi de vare vanrøgtede og forkomne som Får, der ikke have Hyrde.
\par 37 Da siger han til sine Disciple: "Høsten er stor, men Arbejderne ere få;
\par 38 beder derfor Høstens Herre om, at han vil sende Arbejdere ud til sin Høst."

\chapter{10}

\par 1 Og han kaldte sine tolv Disciple til sig og gav dem Magt over urene Ånder, til at uddrive dem og at helbrede enhver Sygdom og enhver Skrøbelighed.
\par 2 Og disse ere de tolv Apostles Navne: Først Simon, som kaldes Peter, og Andreas, hans Broder, og Jakob, Zebedæus's Søn, og Johannes, hans Broder,
\par 3 Filip og Bartholomæus, Thomas og Tolderen Matthæus, Jakob, Alfæus's Søn, og Lebbæus med Tilnavn Thaddæus,
\par 4 Simon Kananæeren og Judas Iskariot, han, som forrådte ham.
\par 5 Disse tolv udsendte Jesus, bød dem og sagde: "Går ikke hen på Hedningers Vej, og går ikke ind i Samaritaners By!
\par 6 Men går hellere hen til de fortabte Får af Israels Hus!
\par 7 Men på eders Vandring skulle I prædike og sige: Himmeriges Rige er kommet nær.
\par 8 Helbreder syge, opvækker døde, renser spedalske, uddriver onde Ånder! I have modtaget det for intet, giver det for intet!
\par 9 Skaffer eder ikke Guld, ej heller Sølv, ej heller Kobber i eders Bælter;
\par 10 ej Taske til at rejse med, ej heller to Kjortler, ej heller Sko, ej heller Stav; thi Arbejderen er sin Føde værd.
\par 11 Men hvor I komme ind i en By eller Landsby, der skulle I spørge, hvem i den der er det værd, og der skulle I blive, indtil I drage bort.
\par 12 Men når I gå ind i Huset, da hilser det;
\par 13 og dersom Huset er det værd, da komme eders Fred over det; men dersom det ikke er det værd, da vende eders Fred tilbage til eder!
\par 14 Og dersom nogen ikke modtager eder og ej hører eders Ord, da går ud af det Hus eller den By og ryster Støvet af eders Fødder!
\par 15 Sandelig, siger jeg eder, det skal gå Sodomas og Gomorras Land tåleligere på Dommens Dag end den By.
\par 16 Se, jeg sender eder som Får midt iblandt Ulve; vorder derfor snilde som Slanger og enfoldige som Duer!
\par 17 Vogter eder for Menneskene; thi de skulle overgive eder til Rådsforsamlinger og hudstryge eder i deres Synagoger.
\par 18 Og I skulle føres for Landshøvdinger og Konger for min Skyld, dem og Hedningerne til et Vidnesbyrd.
\par 19 Men når de overgive eder, da bekymrer eder ikke for, hvorledes eller hvad I skulle tale; thi det skal gives eder i den samme Time, hvad I skulle tale.
\par 20 Thi I ere ikke de, som tale; men det er eders Faders Ånd, som taler i eder.
\par 21 Men Broder skal overgive Broder til Døden, og Fader sit Barn, og Børn skulle sætte sig op imod Forældre og slå dem ihjel.
\par 22 Og I skulle hades af alle for mit Navns Skyld; men den, som holder ud indtil Enden, han skal blive frelst.
\par 23 Men når de forfølge eder i een By, da flyr til en anden; thi sandelig, siger jeg eder, I skulle ikke komme til Ende med Israels Byer, førend Menneskesønnen kommer.
\par 24 En Discipel er ikke over sin Mester, ej heller en Tjener over sin Herre.
\par 25 Det er Disciplen nok, at han bliver som sin Mester, og Tjeneren som sin Herre. Have de kaldt Husbonden Beelzebul, hvor meget mere da hans Husfolk?
\par 26 Frygter altså ikke for dem; thi intet er skjult, som jo skal åbenbares, og intet er lønligt, som jo skal blive kendt.
\par 27 Taler i Lyset, hvad jeg siger eder i Mørket; og prædiker på Tagene, hvad der siges eder i Øret!
\par 28 Og frygter ikke for dem, som slå Legemet ihjel, men ikke kunne slå Sjælen ihjel; men frygter hellere for ham, som kan fordærve både Sjæl og Legeme i Helvede.
\par 29 Sælges ikke to Spurve for en Penning? Og ikke een af dem falder til Jorden uden eders Faders Villie.
\par 30 Men på eder ere endog alle Hovedhår talte.
\par 31 Frygter derfor ikke; I ere mere værd end mange Spurve.
\par 32 Altså, enhver som vedkender sig mig for Menneskene, ham vil også jeg vedkende mig for min Fader, som er i Himlene.
\par 33 Men den, som fornægter mig for Menneskene, ham vil også jeg fornægte for min Fader, som er i Himlene.
\par 34 Mener ikke, at jeg er kommen for at bringe Fred på Jorden; jeg er ikke kommen for at bringe Fred, men Sværd.
\par 35 Thi jeg er kommen før at volde Splid imellem en Mand og hans Fader og imellem en Datter og hendes Moder og imellem en Svigerdatter og hendes Svigermoder,
\par 36 og en Mands Husfolk skulle være hans Fjender.
\par 37 Den, som elsker Fader eller Moder mere end mig, er mig ikke værd; og den, som elsker Søn eller Datter mere end mig, er mig ikke værd;
\par 38 og den, som ikke tager sit Kors og følger efter mig, er mig ikke værd.
\par 39 Den, som bjærger sit Liv, skal miste det; og den, som mister sit Liv for min Skyld, skal bjærge det.
\par 40 Den, som modtager eder, modtager mig; og den, som modtager mig, modtager ham, som udsendte mig.
\par 41 Den, som modtager en Profet, fordi han er en Profet, skal få en Profets Løn; og den, som modtager en retfærdig, fordi han er en retfærdig, skal få en retfærdigs Løn.
\par 42 Og den, som giver en af disse små ikkun et Bæger koldt Vand at drikke, fordi han er en Discipel, sandelig, siger jeg eder, han skal ingenlunde miste sin Løn."

\chapter{11}

\par 1 Og det skete, da Jesus var færdig med at give sine tolv Disciple Befaling, gik han videre derfra for at lære og prædike i deres Byer.
\par 2 Men da Johannes hørte i Fængselet om Kristi Gerninger, sendte han Bud med sine Disciple og lod ham sige:
\par 3 "Er du den, som kommer, eller skulle vi vente en anden?"
\par 4 Og Jesus svarede og sagde til dem: "Går hen, og forkynder Johannes de Ting, som I høre og se:
\par 5 blinde se, og lamme gå, spedalske renses, og døve høre, og døde stå op, og Evangeliet forkyndes for fattige;
\par 6 og salig er den, som ikke forarges på mig."
\par 7 Men da disse gik bort, begyndte Jesus at sige til Skarerne om Johannes: "Hvad gik I ud i Ørkenen at skue? Et Rør, som bevæges hid og did af Vinden?
\par 8 Eller hvad gik I ud at se? Et Menneske, iført bløde Klæder? Se, de, som bære bløde Klæder, ere i Kongernes Huse.
\par 9 Eller hvad gik I ud at se? En Profet? Ja, siger jeg eder, endog mere end en Profet.
\par 10 Thi han er den, om hvem der er skrevet: Se,jeg sender min Engel for dit Ansigt, han skal berede din Vej foran dig.
\par 11 Sandelig, siger jeg eder, iblandt dem, som ere fødte af Kvinder, er ingen større fremstået end Johannes Døberen; men den mindste i Himmeriges Rige er større end han.
\par 12 Men fra Johannes Døberens Dage indtil nu tages Himmeriges Rige med Vold, og Voldsmænd rive det til sig.
\par 13 Thi alle Profeterne og Loven have profeteret indtil Johannes.
\par 14 Og dersom I ville tage imod det: Han er Elias, som skal komme.
\par 15 Den, som har Øren at høre med, han høre!
\par 16 Men hvem skal jeg ligne denne Slægt ved? Den ligner Børn, som sidde på Torvene og råbe til de andre og sige:
\par 17 Vi blæste på Fløjte for eder, og I dansede ikke; vi sang Klagesange, og I jamrede ikke.
\par 18 Thi Johannes kom, som hverken spiste eller drak, og de sige: Han er besat.
\par 19 Menneskesønnen kom, som spiser og drikker, og de sige: Se, en Frådser og en Vindranker, Tolderes og Synderes Ven! Dog, Visdommen er retfærdiggjort ved sine Børn."
\par 20 Da begyndte han at skamme de Byer ud, i hvilke hans fleste kraftige Gerninger vare gjorte, fordi de ikke havde omvendt sig:
\par 21 "Ve dig, Korazin! ve dig, Bethsajda! thi dersom de kraftige Gerninger, som ere skete i eder, vare skete i Tyrus og Sidon, da havde de for længe siden omvendt sig i Sæk og Aske.
\par 22 Men jeg siger eder: Det skal gå Tyrus og Sidon tåleligere på Dommens Dag end eder.
\par 23 Og du, Kapernaum! som er bleven ophøjet indtil Himmelen, du skal nedstødes indtil Dødsriget; thi dersom de kraftige Gerninger, som ere skete i dig, vare skete i Sodoma, da var den bleven stående indtil denne Dag.
\par 24 Men jeg siger eder: Det skal gå Sodomas Land tåleligere på Dommens Dag end dig."
\par 25 På den Tid udbrød Jesus og sagde: "Jeg priser dig, Fader, Himmelens og Jordens Herre! fordi du har skjult dette for vise og forstandige og åbenbaret det for umyndige.
\par 26 Ja, Fader! thi således skete det, som var velbehageligt for dig.
\par 27 Alle Ting ere mig overgivne af min Fader; og ingen kender Sønnen uden Faderen, og ingen kender Faderen uden Sønnen, og den, for hvem Sønnen vil åbenbare ham.
\par 28 Kommer hid til mig alle, som lide Møje og ere besværede, og jeg vil give eder Hvile.
\par 29 Tager mit Åg på eder, og lærer af mig; thi jeg er sagtmodig og ydmyg af Hjertet; så skulle I finde Hvile for eders Sjæle.
\par 30 Thi mit Åg er gavnligt, og min Byrde er let."

\chapter{12}

\par 1 På den Tid vandrede Jesus på Sabbaten igennem en Sædemark; men hans Disciple bleve hungrige og begyndte at plukke Aks og at spise.
\par 2 Men da Farisæerne så det, sagde de til ham: "Se, dine Disciple gøre, hvad det ikke er tilladt at gøre på en Sabbat."
\par 3 Men han sagde til dem: "Have I ikke læst, hvad David gjorde, da han blev hungrig og de, som vare med ham?
\par 4 hvorledes han gik ind i Guds Hus og spiste Skuebrødene, som det ikke var ham tilladt at spise, ej heller dem, som vare med ham, men alene Præsterne?
\par 5 Eller have I ikke læst i Loven, at på Sabbaterne vanhellige Præsterne Sabbaten i Helligdommen og ere dog uden Skyld?
\par 6 Men jeg siger eder, at her er det, som er større end Helligdommen.
\par 7 Men dersom I havde vidst, hvad det Ord betyder: Jeg har Lyst til Barmhjertighed og ikke til Offer, da havde I ikke fordømt dem, som ere uden Skyld.
\par 8 Thi Menneskesønnen er Herre over Sabbaten."
\par 9 Og han gik videre derfra og kom ind i deres Synagoge.
\par 10 Og se, der var en Mand, som havde en vissen Hånd; og de spurgte ham ad og sagde: "Er det tilladt at helbrede på Sabbaten?" for at de kunde anklage ham.
\par 11 Men han sagde til dem: "Hvilket Menneske er der iblandt eder, som har kun eet Får, og ikke tager fat på det og drager det op, dersom det på Sabbaten falder i en Grav?
\par 12 Hvor meget er nu ikke et Menneske mere end et Får? Altså er det tilladt at gøre vel på Sabbaten."
\par 13 Da siger han til Manden: "Ræk din Hånd ud!" og han rakte den ud, og den blev igen sund som den anden.
\par 14 Men Farisæerne gik ud og lagde Råd op imod ham, hvorledes de kunde slå ham ihjel.
\par 15 Men da Jesus mærkede det, drog han bort derfra; og mange fulgte ham, og han helbredte dem alle.
\par 16 Og han bød dem strengt, at de ikke måtte gøre ham kendt;
\par 17 for at det skulde opfyldes, som er talt ved Profeten Esajas, som siger:
\par 18 "Se, min Tjener, som jeg har udvalgt, min elskede, i hvem min Sjæl har Velbehag; jeg vil give min Ånd over ham, og han skal forkynde Hedningerne Ret.
\par 19 Han skal ikke kives og ikke råbe, og ingen skal høre hans Røst på Gaderne.
\par 20 Han skal ikke sønderbryde det knækkede Rør og ikke udslukke den rygende Tande, indtil han får ført Retten frem til Sejr.
\par 21 Og på hans Navn skulle Hedninger håbe."
\par 22 Da blev en besat, som var blind og stum, ført til ham; og han helbredte ham, så at den stumme talte og så.
\par 23 Og alle Skarerne forfærdedes og sagde: "Mon denne skulde være Davids Søn?"
\par 24 Men da Farisæerne hørte det, sagde de: "Denne uddriver ikke de onde Ånder uden ved Beelzebul, de onde Ånders Fyrste."
\par 25 Men såsom han kendte deres Tanker, sagde han til dem: "Hvert Rige, som er kommet i Splid med sig selv, lægges øde; og hver By eller Hus, som er kommet i Splid med sig selv, kan ikke bestå.
\par 26 Og hvis Satan uddriver Satan, så er han kommen i Splid med sig selv; hvorledes skal da hans Rige bestå?
\par 27 Og dersom jeg uddriver de onde Ånder ved Beelzebul, ved hvem uddrive da eders Sønner dem? Derfor skulle de være eders Dommere.
\par 28 Men dersom jeg uddriver de onde Ånder ved Guds Ånd, da er jo Guds Rige kommet til eder.
\par 29 Eller hvorledes kan nogen gå ind i den stærkes Hus og røve hans Ejendele, uden han først binder den stærke? Da kan han plyndre hans Hus.
\par 30 Den, som ikke er med mig, er imod mig; og den, som ikke samler med mig, adspreder.
\par 31 Derfor siger jeg eder: Al Synd og Bespottelse skal forlades Menneskene, men Bespottelsen imod Ånden skal ikke forlades.
\par 32 Og den, som taler et Ord imod Menneskesønnen, ham skal det forlades; men den som taler imod den Helligånd, ham skal det ikke forlades, hverken i denne Verden eller i den kommende.
\par 33 Lader enten Træet være godt og dets Frugt god; eller lader Træet være råddent, og dets Frugt rådden; thi Træet kendes på Frugten.
\par 34 I Øgleunger! hvorledes kunne I tale godt, når I ere onde? Thi af Hjertets Overflødighed taler Munden.
\par 35 Et godt Menneske fremtager gode Ting af sit gode Forråd; og et ondt Menneske fremtager onde Ting af sit onde Forråd.
\par 36 Men jeg siger eder, at Menneskene skulle gøre Regnskab på Dommens Dag for hvert utilbørligt Ord, som de tale.
\par 37 Thi af dine Ord skal du retfærdiggøres, og af dine Ord skal du førdømmes."
\par 38 Da svarede nogle af de skriftkloge og Farisæerne ham og sagde: "Mester! vi ønske at se et Tegn at dig."
\par 39 Men han svarede og sagde til dem: "En ond og utro Slægt forlanger Tegn, men der skal intet Tegn gives den uden Profeten Jonas's Tegn.
\par 40 Thi ligesom Jonas var tre Dage og tre Nætter i Havdyrets Bug, således skal Menneskesønnen være tre Dage og tre Nætter i Jordens Skød.
\par 41 Mænd fra Ninive skulle opstå ved Dommen sammen med denne Slægt og fordømme den; thi de omvendte sig ved Jonas's Prædiken; og se, her er mere end Jonas.
\par 42 Sydens Dronning skal oprejses ved Dommen sammen med denne Slægt og fordømme den; thi hun kom fra Jordens Grænser for at høre Salomons Visdom; og se, her er mere end Salomon.
\par 43 Men når den urene Ånd er faren ud af Mennesket, vandrer den igennem vandløse Steder, søger Hvile og finder den ikke.
\par 44 Da siger den: Jeg vil vende om til mit Hus, som jeg gik ud af; og når den kommer, finder den det ledigt, fejet og prydet.
\par 45 Så går den hen og tager syv andre Ånder med sig, som ere værre end den selv, og når de ere komne derind, bo de der; og det sidste bliver værre med dette Menneske end det første. Således skal det også gå denne onde Slægt."
\par 46 Medens han endnu talte til Skarerne, se, da stode hans Moder og hans Brødre udenfor og begærede at tale med ham.
\par 47 Da sagde en til ham: "Se, din Moder og dine Brødre stå udenfor og begære at tale med dig."
\par 48 Men han svarede og sagde til den, som sagde ham det: "Hvem er min Moder? og hvem ere mine Brødre?"
\par 49 Og han rakte sin Hånd ud over sine Disciple og sagde: "Se, her er min Moder og mine Brødre!
\par 50 Thi enhver, der gør min Faders Villie, som er i Himlene, han er min Broder og Søster og Moder."

\chapter{13}

\par 1 På hin dag gik Jesus ud af Huset og satte sig ved Søen.
\par 2 Og store Skarer samlede sig om ham, så han gik om Bord i et Skib og satte sig; og hele Skaren stod på Strandbredden.
\par 3 Og han talte meget til dem i Lignelser og sagde: "Se, en Sædemand gik ud at så.
\par 4 Og idet han såede, faldt noget ved Vejen; og Fuglene kom og åde det op.
\par 5 Og noget faldt på Stengrund, hvor det ikke havde megen Jord; og det voksede straks op, fordi det ikke havde dyb Jord.
\par 6 Men da Solen kom op, blev det svedet af, og fordi det ikke havde Rod, visnede det.
\par 7 Og noget faldt iblandt Torne; og Tornene voksede op og kvalte det.
\par 8 Og noget faldt i god Jord og bar Frugt, noget hundrede, noget tresindstyve, noget tredive Fold.
\par 9 Den, som har Øren, han høre!"
\par 10 Og Disciplene gik hen og sagde til ham: "Hvorfor taler du til dem i Lignelser?"
\par 11 Men han svarede og sagde til dem: "Fordi det er eder givet at kende Himmeriges Riges Hemmeligheder; men dem er det ikke givet.
\par 12 Thi den, som har, ham skal der gives, og han skal få Overflod; men den, som ikke har, fra ham skal endog det tages, som han har.
\par 13 Derfor taler jeg til dem i Lignelser, fordi de skønt seende dog ikke se, og hørende dog ikke høre og forstå ikke heller.
\par 14 Og på dem opfyldes Esajas's Profeti, som siger: Med eders Øren skulle I høre og dog ikke forstå og se med eders Øjne og dog ikke se.
\par 15 Thi dette Folks Hjerte er blevet sløvet, og med Ørene høre de tungt, og deres Øjne have de tillukket, for at de ikke skulle se med Øjnene og høre med Ørene og forstå med Hjertet og omvende sig, på jeg kunde helbrede dem.
\par 16 Men salige ere eders Øjne, fordi de se, og eders Øren, fordi de høre.
\par 17 Thi sandelig, siger jeg eder, mange Profeter og retfærdige attråede at se, hvad I se, og så det ikke; og at høre, hvad I høre, og hørte det ikke.
\par 18 Så hører nu I Lignelsen om Sædemanden!
\par 19 Når nogen hører Rigets Ord og ikke forstår det, da kommer den Onde og river det bort, som er sået i hans Hjerte; denne er det, som blev sået ved Vejen.
\par 20 Men det, som blev sået på Stengrund, er den, som hører Ordet og straks modtager det med Glæde.
\par 21 Men han har ikke Rod i sig og holder kun ud til en Tid; men når der kommer Trængsel eller Forfølgelse for Ordets Skyld, forarges han straks.
\par 22 Men det, som blev sået iblandt Torne, er den, som hører Ordet, og Verdens Bekymring og Rigdommens Forførelse kvæler Ordet, og det bliver uden Frugt.
\par 23 Men det, som blev sået i god Jord, er den, som hører Ordet og forstår det, og som så bærer Frugt, en hundrede, en tresindstyve, en tredive Fold."
\par 24 En anden Lignelse fremsatte han for dem og sagde: "Himmeriges Rige lignes ved et Menneske, som såede god Sæd i sin Mark.
\par 25 Men medens Folkene sov, kom hans Fjende og såede Ugræs iblandt Hveden og gik bort.
\par 26 Men da Sæden spirede frem og bar Frugt, da kom også Ugræsset til Syne.
\par 27 Og Husbondens Tjenere kom til ham og sagde: Herre, såede du ikke god Sæd i din Mark? Hvor har den da fået Ugræsset fra?
\par 28 Men han sagde til dem: Det har et fjendsk Menneske gjort. Da sige Tjenerne til ham: Vil du da, at vi skulle gå hen og sanke det sammen?
\par 29 Men han siger: Nej, for at I ikke, når I sanke Ugræsset sammen, skulle rykke Hveden op tillige med det.
\par 30 Lader dem begge vokse tilsammen indtil Høsten; og i Høstens Tid vil jeg sige til Høstfolkene: Sanker først Ugræsset sammen og binder det i Knipper for at brænde det, men samler Hveden i min Lade!"
\par 31 En anden Lignelse fremsatte han for dem og sagde: "Himmeriges Rige ligner et Sennepskorn, som en Mand tog og såede i sin Mark.
\par 32 Dette er vel mindre end alt andet Frø; men når det er vokset op, er det støre end Urterne og bliver et Træ, så at Himmelens Fugle komme og bygge Rede i dets Grene."
\par 33 En anden Lignelse talte han til dem: "Himmeriges Rige ligner en Surdejg, som en Kvinde tog og lagde ned i tre Mål Mel, indtil det blev syret alt sammen."
\par 34 Alt dette talte Jesus til Skarerne i Lignelser, og uden Lignelse talte han intet til dem,
\par 35 for at det skulde opfyldes, som er talt ved Profeten, der siger: "Jeg vil oplade min Mund i Lignelser; jeg vil udsige det, som har været skjult fra Verdens Grundlæggelse."
\par 36 Da forlod han Skarerne og gik ind i Huset; og hans Disciple kom til ham og sagde: "Forklar os Lignelsen om Ugræsset på Marken!"
\par 37 Men han svarede og sagde: "Den, som sår den gode Sæd, er Menneskesønnen,
\par 38 og Marken er Verden, og den gode Sæd er Rigets Børn, men Ugræsset er den Ondes Børn,
\par 39 og Fjenden, som såede det, er Djævelen; og Høsten er Verdens Ende; og Høstfolkene ere Engle.
\par 40 Ligesom nu Ugræsset sankes sammen og opbrændes med Ild, således skal det ske ved Verdens Ende.
\par 41 Menneskesønnen skal udsende sine Engle, og de skulle sanke ud af hans Rige alle Forargelserne og dem, som gøre Uret;
\par 42 og de skulle kaste dem i Ildovnen; der skal være Gråd og Tænders Gnidsel.
\par 43 Da skulle de retfærdige skinne som Solen i deres Faders Rige. Den.
\par 44 Himmeriges Rige ligner en Skat. som er skjult i en Mark, og en Mand fandt og skjulte den, og af Glæde over den går han hen og sælger alt, hvad han har, og køber den Mark.
\par 45 Atter ligner Himmeriges Rige en Købmand, som søgte efter skønne Perler;
\par 46 og da han fandt een meget kostbar Perle, gik han hen og solgte alt, hvad han havde, og købte den.
\par 47 Atter ligner Himmeriges Rige et Vod, som blev kastet i Havet og samlede Fisk af alle Slags.
\par 48 Og da det var blevet fuldt, drog man det op på Strandbredden og satte sig og sankede de gode sammen i Kar, men kastede de rådne ud.
\par 49 Således skal det gå til ved Verdens Ende. Englene skulle gå ud og skille de onde fra de retfærdige
\par 50 og kaste dem i Ildovnen; der skal være Gråd og Tænders Gnidsel.
\par 51 Have I forstået alt dette?" De sige til ham: "Ja."
\par 52 Men han sagde til dem: "Derfor er hver skriftklog, som er oplært for Himmeriges Rige, ligesom en Husbond, der tager nyt og gammelt frem af sit Forråd."
\par 53 Og det skete, da Jesus havde fuldendt disse Lignelser, drog han bort derfra.
\par 54 Og han kom til sin Fædrene by og lærte dem i deres Synagoge, så at de bleve slagne af Forundring og sagde: "Hvorfra har han denne Visdom og de kraftige Gerninger?
\par 55 Er denne ikke Tømmermandens Søn? Hedder ikke hans Moder Maria og hans Brødre Jakob og Josef og Simon og Judas?
\par 56 Og hans Søstre, ere de ikke alle hos os? Hvorfra har han alt dette?"
\par 57 Og de forargedes på ham. Men Jesus sagde til dem: "En Profet er ikke foragtet uden i sit eget Fædreland og i sit Hus,"
\par 58 Og han gjorde ikke mange kraftige Gerninger der for deres Vantros Skyld.

\chapter{14}

\par 1 På den Tid hørte Fjerdingsfyrsten Herodes Rygtet om Jesus.
\par 2 Og han sagde til sine Tjenere: "Det er Johannes Døberen; han er oprejst fra de døde, derfor virke Kræfterne i ham."
\par 3 Thi Herodes havde grebet Johannes og bundet ham og sat ham i Fængsel for sin Broder Filips Hustru, Herodias's Skyld.
\par 4 Johannes sagde nemlig til ham: "Det er dig ikke tilladt at have hende."
\par 5 Og han vilde gerne slå ham ihjel, men frygtede for Mængden, thi de holdt ham for en Profet.
\par 6 Men da Herodes's Fødselsdag kom, dansede Herodias's Datter for dem; og hun behagede Herodes.
\par 7 Derfor lovede han med en Ed at give hende, hvad som helst hun begærede.
\par 8 Og tilskyndet af sin Moder siger hun: "Giv mig Johannes Døberens Hoved hid på et Fad!"
\par 9 Og Kongen blev bedrøvet; men for sine Eders og for Gæsternes Skyld befalede han, at det skulde gives hende.
\par 10 Og han sendte Bud og lod Johannes halshugge i Fængselet.
\par 11 Og hans Hoved blev bragt på et Fad og givet Pigen, og hun bragte det til sin Moder.
\par 12 Da kom hans Disciple og toge Liget og begravede ham, og de kom og forkyndte Jesus det.
\par 13 Og da Jesus hørte det, drog han bort derfra i et Skib til et øde Sted afsides; og da Skarerne hørte det, fulgte de ham til Fods fra Byerne.
\par 14 Og da han kom i Land, så han en stor Skare, og han ynkedes inderligt over dem og helbredte deres syge.
\par 15 Men da det blev Aften, kom Disciplene til ham og sagde: "Stedet er øde, og Tiden er allerede forløben; lad Skarerne gå bort, for at de kunne gå hen i Landsbyerne og købe sig Mad."
\par 16 Men Jesus sagde til dem: "De have ikke nødig at gå bort; giver I dem at spise!"
\par 17 Men de sige til ham: "Vi have ikke her uden fem Brød og to Fisk."
\par 18 Men han sagde: "Henter mig dem hid!"
\par 19 Og han bød Skarerne at sætte sig ned i Græsset og tog de fem Brød og de to Fisk, så op til Himmelen og velsignede; og han brød Brødene og gav Disciplene dem, og Disciplene gave dem til Skarerne.
\par 20 Og de spiste alle og bleve mætte; og de opsamlede det, som blev tilovers af Stykkerne, tolv Kurve fulde
\par 21 Men de, som spiste, vare omtrent fem Tusinde Mænd, foruden Kvinder og Børn.
\par 22 Og straks nødte han sine Disciple til at gå om Bord i Skibet og i Forvejen sætte over til hin Side, medens han lod Skarerne gå bort.
\par 23 Og da han havde ladet Skarerne gå bort, gik han op på Bjerget afsides for at bede. Og da det blev silde, var han der alene.
\par 24 Men Skibet var allerede midt på Søen og led Nød af Bølgerne; thi Vinden var imod.
\par 25 Men i den fjerde Nattevagt kom han til dem, vandrende på Søen.
\par 26 Og da Disciplene så ham vandre på Søen, bleve de forfærdede og sagde: "Det er et Spøgelse;" og de skrege af Frygt.
\par 27 Men straks talte Jesus til dem og sagde: "Værer frimodige; det er mig, frygter ikke!"
\par 28 Men Peter svarede ham og sagde: "Herre! dersom det er dig, da byd mig at komme til dig på Vandet!"
\par 29 Men han sagde: "Kom!" Og Peter trådte ned fra Skibet og vandrede på Vandet for at komme til Jesus.
\par 30 Men da han så det stærke Vejr, blev han bange; og da han begyndte at synke, råbte han og sagde: "Herre, frels mig!"
\par 31 Og straks udrakte Jesus Hånden og greb ham, og han siger til ham: "Du lidettroende, hvorfor tvivlede du?"
\par 32 Og da de stege op i Skibet, lagde Vinden sig.
\par 33 Men de, som vare i Skibet, faldt ned for ham og sagde: "Du er sandelig Guds Søn."
\par 34 Og da de vare farne over, landede de i Genezareth.
\par 35 Og da Folkene på det Sted kendte ham, sendte de Bud til hele Egnen der omkring og bragte alle de syge til ham.
\par 36 Og de bade ham, at de blot måtte røre ved Fligen af hans Klædebon; og alle de, som rørte derved, bleve helbredede.

\chapter{15}

\par 1 Da kommer der fra Jerusalem Farisæere og skriftkloge til Jesus og sige:
\par 2 "Hvorfor overtræde dine Disciple de gamles Overlevering? thi de to ikke deres Hænder, når de holde Måltid."
\par 3 Men han svarede og sagde til dem: "Hvorfor overtræde også I Guds Bud for eders Overleverings Skyld?
\par 4 Thi Gud har påbudt og sagt: "Ær din Fader og Moder;" og: "Den, som hader Fader eller Moder, skal visselig dø."
\par 5 Men I sige: "Den, som siger til sin Fader eller sin Moder: "Det, hvormed du skulde være hjulpet af mig, skal være en Tempelgave," han skal ingenlunde ære sin Fader eller sin Moder."
\par 6 Og I have ophævet Guds Lov for eders Overleverings Skyld.
\par 7 I Hyklere! Rettelig profeterede Esajas om eder, da han sagde:
\par 8 "Dette Folk ærer mig med Læberne; men deres Hjerte er langt borte fra mig.
\par 9 Men de dyrke mig forgæves,idet de lære Lærdomme, som ere Menneskers Bud."
\par 10 Og han kaldte Folkeskaren til sig og sagde til dem: "Hører og forstår!
\par 11 Ikke det, som går ind i Munden, gør Mennesket urent, men det, som går ud af Munden, dette gør Mennesket urent."
\par 12 Da kom hans Disciple hen og sagde til ham: "Ved du, at Farisæerne bleve forargede, da de hørte den Tale?"
\par 13 Men han svarede og sagde: "Enhver Plantning, som min himmelske Fader ikke har plantet, skal oprykkes med Rode.
\par 14 Lader dem fare, det er blinde Vejledere for blinde; men når en blind leder en blind, falde de begge i Graven."
\par 15 Men Peter svarede og sagde til ham: "Forklar os Lignelsen!"
\par 16 Og han sagde: "Ere også I endnu så uforstandige?
\par 17 Forstår I endnu ikke, at alt, hvad der går ind i Munden, går i Bugen og føres ud ad den naturlige Vej?
\par 18 Men det, som går ud af Munden, kommer ud fra Hjertet, og det gør Mennesket urent.
\par 19 Thi ud fra Hjertet kommer der onde Tanker, Mord, Hor, Utugt, Tyverier, falske Vidnesbyrd, Forhånelser.
\par 20 Det er disse Ting, som gøre Mennesket urent; men at spise med utoede Hænder gør ikke Mennesket urent."
\par 21 Og Jesus gik bort derfra og drog til Tyrus's og Sidons Egne.
\par 22 Og se, en kananæisk Kvinde kom fra disse Egne, råbte og sagde: "Herre, Davids Søn! forbarm dig over mig! min Datter plages ilde af en ond Ånd."
\par 23 Men han svarede hende ikke et Ord. Da trådte hans Disciple til, bade ham og sagde: "Skil dig af med hende, thi hun råber efter os."
\par 24 Men han svarede og sagde: "Jeg er ikke udsendt uden til de fortabte Får af Israels Hus."
\par 25 Men hun kom og kastede sig ned for ham og sagde: "Herre, hjælp mig!"
\par 26 Men han svarede og sagde: "Det er ikke smukt at tage Børnenes Brød og kaste det for de små Hunde."
\par 27 Men hun sagde: "Jo, Herre! de små Hunde æde jo dog også af de Smuler, som falde fra deres Herrers Bord."
\par 28 Da svarede Jesus og sagde til hende: "O Kvinde, din Tro er stor, dig ske, som du vil!" Og hendes Datter blev helbredt fra samme Time.
\par 29 Og Jesus gik bort derfra og kom hen til Galilæas Sø, og han gik op på Bjerget og satte sig der.
\par 30 Og store Skarer kom til ham og havde lamme, blinde, stumme, Krøblinge og mange andre med sig; og de lagde dem for hans Fødder, og han helbredte dem,
\par 31 så at Skaren undrede sig, da de så, at stumme talte, Krøblinge bleve raske, lamme gik, og blinde så; og de priste Israels Gud.
\par 32 Men Jesus kaldte sine Disciple til sig og sagde: "Jeg ynke s inderligt over Skaren; thi de have allerede tøvet hos mig tre Dage og have intet at spise; og lade dem gå fastende bort, vil jeg ikke, for at de ikke skulle vansmægte på Vejen."
\par 33 Og hans Disciple sige til ham: "Hvorfra skulle vi få så mange Brød i en Ørken, at vi kunne mætte så mange Mennesker?"
\par 34 Og Jesus siger til dem: "Hvor mange Brød have I?" Men de sagde: "Syv og nogle få Småfisk."
\par 35 Og han bød Skaren at sætte sig ned på Jorden
\par 36 og tog de syv Brød og Fiskene, takkede, brød dem og gav Disciplene dem, og Disciplene gave dem til Skarerne.
\par 37 Og de spiste alle og bleve mætte; og de opsamlede det, som blev tilovers af Stykkerne, syv Kurve fulde.
\par 38 Men de, som spiste, vare fire Tusinde Mænd, foruden Kvinder og Børn.
\par 39 Og da han havde ladet Skarerne gå bort, gik han om Bord i Skibet og kom til Magadans Egne.

\chapter{16}

\par 1 Og Farisæerne og Saddukæerne kom hen og fristede ham og begærede, at han vilde vise dem et Tegn fra Himmelen.
\par 2 Men han svarede og sagde til dem: "Om Aftenen sige I: Det bliver en skøn Dag, thi Himmelen er rød;
\par 3 og om Morgenen: Det bliver Storm i Dag, thi Himmelen er rød og mørk.
\par 4 En ond og utro Slægt forlanger Tegn; men der skal intet Tegn gives den uden Jonas's Tegn." Og han forlod dem og gik bort.
\par 5 Og da hans Disciple kom over til hin Side, havde de glemt at tage Brød med.
\par 6 Og Jesus sagde til dem: "Ser til, og tager eder i Vare for Farisæernes og Saddukæernes Surdejg!"
\par 7 Men de tænkte ved sig selv og sagde: "Det er, fordi vi ikke toge Brød med."
\par 8 Men da Jesus mærkede dette, sagde han: "I lidettroende! hvorfor tænke I ved eder selv på, at I ikke have taget Brød med?
\par 9 Forstå I ikke endnu? Komme I heller ikke i Hu de fem Brød til de fem Tusinde, og hvor mange Kurve I da toge op?
\par 10 Ikke heller de syv Brød til de fire Tusinde, og hvor mange Kurve I da toge op?
\par 11 Hvorledes forstå I da ikke, at det ej var om Brød, jeg sagde det til eder? Men tager eder i Vare for Farisæernes og Saddukæernes Surdejg."
\par 12 Da forstode de, at han havde ikke sagt, at de skulde tage sig i Vare for Surdejgen i Brød, men for Farisæernes og Saddukæernes Lære.
\par 13 Men da Jesus var kommen til Egnen ved Kæsarea Filippi, spurgte han sine Disciple og sagde: "Hvem sige Folk, at Menneskesønnen er?"
\par 14 Men de sagde: "Nogle sige Johannes Døberen; andre Elias; andre Jeremias eller en af Profeterne."
\par 15 Han siger til dem: "Men I, hvem sige I, at jeg er?"
\par 16 Da svarede Simon Peter og sagde: "Du er Kristus, den levende Guds Søn."
\par 17 Og Jesus svarede og sagde til ham: "Salig er du, Simon Jonas's Søn! thi Kød og Blod har ikke åbenbaret dig det, men min Fader, som er i Himlene.
\par 18 Så siger jeg også dig, at du er Petrus, og på denne Klippe vil jeg bygge min Menighed, og Dødsrigets Porte skulle ikke få Overhånd over den.
\par 19 Og jeg vil give dig Himmeriges Riges Nøgler, og hvad du binder på Jorden, det skal være bundet i Himlene, og hvad du løser på Jorden, det skal være løst i Himlene."
\par 20 Da bød han sine Disciple, at de måtte ikke sige til nogen at han var Kristus.
\par 21 Fra den Tid begyndte Jesus at give sine Disciple til Kende, at han skulde gå til Jerusalem og lide meget af de Ældste og Ypperstepræsterne og de skriftkloge og ihjelslås og oprejses på den tredje Dag.
\par 22 Og Peter tog ham til Side, begyndte at sætte ham i Rette og sagde: "Gud bevare dig, Herre; dette skal ingenlunde ske dig!"
\par 23 Men han vendte sig og sagde til Peter: "Vig bag mig, Satan! du er mig en Forargelse; thi du sanser ikke, hvad Guds er, men hvad Menneskers er."
\par 24 Da sagde Jesus til sine Disciple: "Vil nogen komme efter mig, han fornægte sig selv og tage sit Kors op og følge mig!
\par 25 Thi den, som vil frelse sit.Liv, skal miste det; men den, som mister sit Liv for min Skyld, skal bjærge det.
\par 26 Thi hvad gavner det et Menneske, om han vinder den hele Verden, men må bøde med sin Sjæl? Eller hvad kan et Menneske give til Vederlag for sin Sjæl?
\par 27 Thi Menneskesønnen skal komme i sin Faders Herlighed med sine Engle; og da skal han betale enhver efter hans Gerning.
\par 28 Sandelig siger jeg eder, der er nogle af dem, som stå her, der ingenlunde skulle smage Døden, førend de se Menneskesønnen komme i sit Rige."

\chapter{17}

\par 1 Og seks Dage derefter tager Jesus Peter og Jakob og hans Broder Johannes med sig og fører dem afsides op på et højt Bjerg.
\par 2 Og han blev forvandlet for deres Øjne, og hans Åsyn skinnede som Solen, men hans Klæder bleve hvide som Lyset.
\par 3 Og se, Moses og Elias viste sig for dem og samtalede med ham.
\par 4 Da tog Peter til Orde og sagde til Jesus: "Herre! det er godt, at vi ere her; vil du, da lader os gøre tre Hytter her, dig en og Moses en og Elias en."
\par 5 Medens han endnu talte, se, da overskyggede en lysende Sky dem; og se, der kom fra Skyen en Røst. som sagde: "Denne er min Søn. den elskede, i hvem jeg har Velbehag; hører ham!"
\par 6 Og da Disciplene hørte det, faldt de på deres Ansigt og frygtede såre.
\par 7 Og Jesus trådte hen og rørte ved dem og sagde: "Står op, og frygter ikke!"
\par 8 Men da de opløftede deres Øjne, så de ingen uden Jesus alene.
\par 9 Og da de gik ned fra Bjerget, bød Jesus dem og sagde: "Taler ikke til nogen om dette Syn, førend Menneskesønnen er oprejst fra de døde."
\par 10 Og hans Disciple spurgte ham og sagde: "Hvad er det da, de skriftkloge sige, at Elias bør først komme?"
\par 11 Og han svarede og sagde: "Vel kommer Elias og skal genoprette alting.
\par 12 Men jeg siger eder, at Elias er allerede kommen, og de erkendte ham ikke, men gjorde med ham alt, hvad de vilde; således skal også Menneskesønnen lide ondt af dem."
\par 13 Da forstode Disciplene, at han havde talt til dem om Johannes Døberen.
\par 14 Og da de kom til Folkeskaren, kom en Mand til ham og faldt på Knæ for ham og sagde:
\par 15 "Herre! forbarm dig over min Søn, thi han er månesyg og lidende; thi han falder ofte i Ild og ofte i Vand;
\par 16 og jeg bragte ham til dine Disciple, og de kunde ikke helbrede ham."
\par 17 Og Jesus svarede og sagde: "O du vantro og forvendte Slægt! hvor længe skal jeg være hos eder, hvor længe skal jeg tåle eder? Bringer mig ham hid!"
\par 18 Og Jesus talte ham hårdt til, og den onde Ånd for ud af ham, og Drengen blev helbredt fra samme Time.
\par 19 Da gik Disciplene til Jesus afsides og sagde: "Hvorfor kunde vi ikke uddrive den?"
\par 20 Og han siger til dem: "For eders Vantros Skyld; thi sandelig, siger jeg eder, dersom I have Tro som et Sennepskorn, da kunne I sige til dette Bjerg: Flyt dig herfra derhen, så skal det flytte sig, og intet skal være eder umuligt.
\par 21 Men denne Slags farer ikke ud uden ved Bøn og Faste."
\par 22 Og medens de vandrede sammen i Galilæa, sagde Jesus til dem: "Menneskesønnen skal overgives i Menneskers Hænder;
\par 23 og de skulle slå ham ihjel, og på den tredje Dag skal han oprejses." Og de bleve såre bedrøvede.
\par 24 Men da de kom til Kapernaum, kom de, som opkrævede Tempelskatten, til Peter og sagde: "Betaler eders Mester ikke Skatten?"
\par 25 Han sagde: "Jo." Og da han kom ind i Huset, kom Jesus ham i Forkøbet og sagde: "Hvad tykkes dig, Simon? Af hvem tage Jordens Konger Told eller Skat, af deres egne Sønner eller af de fremmede?"
\par 26 Og da han sagde: "Af de fremmede," sagde Jesus til ham: "Så ere jo Sønnerne fri.
\par 27 Men for at vi ikke skulle forarge dem, så gå hen til Søen, kast en Krog ud, og tag den første Fisk, som kommer op; og når du åbner dens Mund, skal du finde en Stater; tag denne, og giv dem den for mig og dig!"

\chapter{18}

\par 1 I den samme Stund kom Disciplene hen til Jesus og sagde: "Hvem er da den største i Himmeriges Rige?"
\par 2 Og han kaldte et lille Barn til sig og stillede det midt iblandt dem
\par 3 og sagde: "Sandelig, siger jeg eder, uden I omvende eder og blive som Børn, komme I ingenlunde ind i Himmeriges Rige.
\par 4 Derfor, den, som fornedrer sig selv som dette Barn, han er den største i Himmeriges Rige.
\par 5 Og den, som modtager et eneste sådant Barn for mit Navns Skyld, modtager mig.
\par 6 Men den, som forarger een af disse små, som tro på mig, ham var det bedre, at der var hængt en Møllesten om hans Hals, og han var sænket i Havets Dyb.
\par 7 Ve Verden for Forargelserne! Thi vel er det nødvendigt, at Forargelserne komme; dog ve det Menneske, ved hvem Forargelsen kommer!
\par 8 Men dersom din Hånd eller din Fod forarger dig, da hug den af, og kast den fra dig! Det er bedre for dig at gå lam eller som en Krøbling ind til Livet end at have to Hænder og to Fødder og blive kastet i den evige Ild.
\par 9 Og dersom dit Øje forarger dig, da riv det ud, og kast det fra dig! Det er bedre for dig at gå enøjet ind til Livet end at have to Øjne og blive kastet i Helvedes Ild.
\par 10 Ser til, at I ikke foragte en eneste af disse små; thi jeg siger eder: Deres Engle i Himlene se altid min Faders Ansigt, som er i Himlene.
\par 11 Thi Menneskesønnen er kommen for at frelse det fortabte.
\par 12 Hvad tykkes eder? Om et Menneske har hundrede Får, og eet af dem farer vild, forlader han da ikke de ni og halvfemsindstyve og går ud i Bjergene og leder efter det vildfarne?
\par 13 Og hænder det sig, at han finder det, sandelig, siger jeg eder, han glæder sig mere over det end over de ni og halvfemsindstyve, som ikke ere farne vild.
\par 14 Således er det ikke eders himmelske Faders Villie, at en eneste af disse små skal fortabes.
\par 15 Men om din Broder synder imod dig, da gå hen og revs ham mellem dig og ham alene; hører han dig, da har du vundet din Broder.
\par 16 Men hører han dig ikke, da tag endnu een eller to med dig, for at "hver Sag må stå fast efter to eller tre Vidners Mund."
\par 17 Men er han dem overhørig, da sig det til Menigheden; men er han også Menigheden overhørig, da skal han være for dig ligesom en Hedning og en Tolder.
\par 18 Sandelig, siger jeg eder, hvad som helst I binde på Jorden, skal være bundet i Himmelen; og hvad som helst I løse på Jorden, skal være løst i Himmelen.
\par 19 Atter siger jeg eder, at dersom to af eder blive enige på Jorden om hvilken som helst Sag, hvorom de ville bede, da skal det blive dem til Del fra min Fader, som er i Himlene.
\par 20 Thi hvor to eller tre ere forsamlede om mit Navn, der er jeg midt iblandt dem."
\par 21 Da trådte Peter frem og sagde til ham: "Herre! hvor ofte skal jeg tilgive min Broder, når han synder imod mig? mon indtil syv Gange?"
\par 22 Jesus siger til ham: "Jeg siger dig: ikke indtil syv Gange, men indtil halvfjerdsindstyve Gange syv Gange.
\par 23 Derfor lignes Himmeriges Rige ved en Konge, som vilde holde Regnskab med sine Tjenere.
\par 24 Men da han begyndte at holde Regnskab, blev en, som var ti Tusinde Talenter skyldig, ført frem for ham.
\par 25 Og da han intet havde at betale med, bød hans Herre, at han og hans Hustru og Børn og alt det, han havde, skulde sælges, og Gælden betales.
\par 26 Da faldt Tjeneren ned for ham, bønfaldt ham og sagde: Herre, vær langmodig med mig, så vil jeg betale dig det alt sammen.
\par 27 Da ynkedes samme Tjeners Herre inderligt over ham og lod ham løs og eftergav ham Gælden.
\par 28 Men den samme Tjener gik ud og traf en af sine Medtjenere, som var ham hundrede Denarer skyldig; og han greb fat på ham og var ved at kvæle ham og sagde: Betal, hvad du er skyldig!
\par 29 Da faldt hans Medtjener ned for ham og bad ham og sagde: Vær langmodig med mig, så vil jeg betale dig.
\par 30 Men han vilde ikke, men gik hen og kastede ham i Fængsel, indtil han betalte, hvad han var skyldig.
\par 31 Da nu hans Medtjenere så det, som skete, bleve de såre bedrøvede og kom og forklarede for deres Herre alt, hvad der var sket.
\par 32 Da kalder hans Herre ham for sig og siger til ham: Du onde Tjener! al den Gæld eftergav jeg dig, fordi du bad mig.
\par 33 Burde ikke også du forbarme dig over din Medtjener, ligesom jeg har forbarmet mig over dig.
\par 34 Og hans Herre blev vred og overgav ham til Bødlerne, indtil han kunde få betalt alt det, han var ham skyldig.
\par 35 Således skal også min himmelske Fader gøre mod eder, om I ikke af Hjertet tilgive, enhver sin Broder."

\chapter{19}

\par 1 Og det skete, da Jesus havde fuldendt disse Ord, drog han bort fra Galilæa og kom til Judæas Egne, hinsides Jordan.
\par 2 Og store Skarer fulgte ham, og han helbredte dem der.
\par 3 Og Farisæerne kom til ham, fristede ham og sagde: "Er det tilladt at skille sig fra sin Hustru af hvilken som helst Grund?"
\par 4 Men han svarede og sagde: "Have I ikke læst, at Skaberen fra Begyndelsen skabte dem som Mand og Kvinde
\par 5 og sagde: Derfor skal en Mand forlade sin Fader og sin Moder og holde sig til sin Hustru, og de to skulle blive til eet Kød?
\par 6 Således ere de ikke længer to, men eet Kød. Derfor, hvad Gud har sammenføjet, må et Menneske ikke adskille."
\par 7 De sige til ham: "Hvorfor bød da Moses at give et Skilsmissebrev og skille sig fra hende?"
\par 8 Han siger til dem: "Moses tilstedte eder at skille eder fra eders Hustruer for eders Hjerters Hårdheds Skyld; men fra Begyndelsen har det ikke været således.
\par 9 Men jeg siger eder, at den, som skiller sig fra sin Hustru, når det ikke er for Hors Skyld, og tager en anden til Ægte, han bedriver Hor; og den, som tager en fraskilt Hustru til Ægte, han bedriver Hor."
\par 10 Hans Disciple sige til ham: "Står Mandens Sag med Hustruen således, da er det ikke godt at gifte sig."
\par 11 Men han sagde til dem: "Ikke alle rummer dette Ord, men de, hvem det er givet:
\par 12 Thi der er Gildinger, som ere fødte således fra Moders Liv; og der er Gildinger, som ere gildede af Mennesker; og der er Gildinger, som have gildet sig selv for Himmeriges Riges Skyld. Den, som kan rumme det, han rumme det!"
\par 13 Da blev der båret små Børn til ham, for at han skulde lægge Hænderne på dem og bede; men Disciplene truede dem.
\par 14 Da sagde Jesus: "Lader de små Børn komme, og formener dem ikke at komme til mig; thi Himmeriges Rige hører sådanne til."
\par 15 Og han lagde Hænderne på dem, og han drog derfra.
\par 16 Og se, en kom til ham og sagde: "Mester! hvad godt skal jeg gøre, for at jeg kan få et evigt Liv?"
\par 17 Men han sagde til ham: "Hvorfor spørger du mig om det gode? Een er den gode. Men vil du indgå til Livet, da hold Budene!"
\par 18 Han siger til ham: "Hvilke?" Men Jesus sagde: "Dette: Du må ikke slå ihjel; du må ikke bedrive Hor; du må ikke stjæle; du må ikke sige falsk Vidnesbyrd;
\par 19 ær din Fader og din Moder, og: Du skal elske din Næste som dig selv."
\par 20 Den unge Mand siger til ham: "Det har jeg holdt alt sammen; hvad fattes mig endnu?"
\par 21 Jesus sagde til ham: "Vil du være fuldkommen, da gå bort, sælg, hvad du ejer, og giv det til fattige, så skal du have en Skat i Himmelen; og kom så og følg mig!"
\par 22 Men da den unge Mand hørte det Ord, gik han bedrøvet bort; thi han havde meget Gods.
\par 23 Men Jesus sagde til sine Disciple: "Sandelig, siger jeg eder: En rig Kommer vanskeligt ind i Himmeriges Rige.
\par 24 Atter siger jeg eder: Det er lettere for en Kamel at gå igennem et Nåleøje end for en rig at gå ind i Guds Rige."
\par 25 Men da Disciplene hørte dette, forfærdedes de såre og sagde: "Hvem kan da blive frelst?"
\par 26 Men Jesus så på dem og sagde: "For Mennesker er dette umuligt, men for Gud ere alle Ting mulige."
\par 27 Da svarede Peter og sagde til ham: "Se, vi have forladt alle Ting og fulgt dig; hvad skulle da vi have?"
\par 28 Men Jesus sagde til dem: "Sandelig, siger jeg eder, at i Igenfødelsen, når Menneskesønnen sidder på sin Herligheds Trone, skulle også I, som have fulgt mig, sidde på tolv Troner og dømme Israels tolv Stammer.
\par 29 Og hver, som har forladt Hus eller Brødre eller Søstre eller Fader eller Moder eller Hustru eller Børn eller Marker for mit Navns Skyld, skal få det mange Fold igen og arve et evigt Liv.
\par 30 Men mange af de første skulle blive de sidste, og af de sidste de første.

\chapter{20}

\par 1 Thi Himmeriges Rige ligner en Husbond, som gik ud tidligt om Morgenen for at leje Arbejdere til sin Vingård.
\par 2 Og da han var bleven enig med Arbejderne om en Denar om Dagen, sendte han dem til sin Vingård.
\par 3 Og han gik ud ved den tredje Time og så andre stå ledige på Torvet,
\par 4 og han sagde til dem: Går også I hen i Vingården, og jeg vil give eder, hvad som ret er. Og de gik derhen.
\par 5 Han gik atter ud ved den sjette og niende Time og gjorde ligeså.
\par 6 Og ved den ellevte Time gik han ud og fandt andre stående der, og han siger til dem: Hvorfor stå I her ledige hele Dagen?
\par 7 De sige til ham: Fordi ingen lejede os. Han siger til dem: Går også I hen i Vingården!
\par 8 Men da det var blevet Aften, siger Vingårdens Herre til sin Foged: Kald på Arbejderne, og betal dem deres Løn, idet du begynder med de sidste og ender med de første!
\par 9 Og de, som vare lejede ved den ellevte Time, kom og fik hver en Denar.
\par 10 Men da de første kom, mente de, at de skulde få mere; og også de fik hver en Denar.
\par 11 Men da de fik den, knurrede de imod Husbonden og sagde:
\par 12 Disse sidste have kun arbejdet een Time, og du har gjort dem lige med os, som have båret Dagens Byrde og Hede.
\par 13 Men han svarede og sagde til en af dem: Ven! jeg gør dig ikke Uret; er du ikke bleven enig med mig om en Denar?
\par 14 Tag dit og gå! Men jeg vil give denne sidste ligesom dig.
\par 15 Eller har jeg ikke Lov at gøre med mit, hvad jeg vil? Eller er dit Øje ondt, fordi jeg er god?
\par 16 Således skulle de sidste blive de første, og de første de sidste; thi mange ere kaldede, men få ere udvalgte."
\par 17 Og da Jesus drog op til Jerusalem, tog han de tolv Disciple til Side og sagde til dem på Vejen:
\par 18 "Se, vi drage op til Jerusalem, og Menneskesønnen skal overgives til Ypperstepræsterne og de skriftkloge; og de skulle dømme ham til Døden
\par 19 og overgive ham til Hedningerne til at spottes og hudstryges og korsfæstes; og på den tredje Dag skal han opstå."
\par 20 Da gik Zebedæus's Sønners Moder til ham med sine Sønner og faldt ned for ham og vilde bede ham om noget.
\par 21 Men han sagde til hende: "Hvad vil du?" Hun siger til ham: "Sig, at disse mine to Sønner skulle i dit Rige sidde den ene ved din højre, den anden ved din venstre Side."
\par 22 Men Jesus svarede og sagde: "I vide ikke, hvad I bede om. Kunne I drikke den Kalk, som jeg skal drikke?" De sige til ham: "Det kunne vi."
\par 23 Han siger til dem: "Min Kalk skulle I vel drikke; men det at sidde ved min højre og ved min venstre Side tilkommer det ikke mig at give; men det gives til dem, hvem det er beredt af min Fader."
\par 24 Og da de ti hørte dette, bleve de vrede på de to Brødre.
\par 25 Men Jesus kaldte dem til sig og sagde: "I vide, at Folkenes Fyrster herske over dem, og de store bruge Myndighed over dem.
\par 26 Således skal det ikke være iblandt eder; men den, som vil blive stor iblandt eder, han skal være eders Tjener;
\par 27 og den, som vil være den ypperste iblandt eder, han skal være eders Træl.
\par 28 Ligesom Menneskesønnen ikke er kommen for at lade sig tjene, men for at tjene og give sit Liv til en Genløsning for mange."
\par 29 Og da de gik ud af Jeriko, fulgte en stor Folkeskare ham.
\par 30 Og se, to blinde sade ved Vejen, og da de hørte, at Jesus gik forbi, råbte de og sagde: "Herre, forbarm dig over os, du Davids Søn!"
\par 31 Men Skaren truede dem, at de skulde tie; men de råbte endnu stærkere og sagde: "Herre, forbarm dig over os, du Davids Søn!"
\par 32 Og Jesus stod stille og kaldte på dem og sagde: "Hvad ville I, at jeg skal gøre for eder?"
\par 33 De sige til ham: "Herre! at vore Øjne måtte oplades."
\par 34 Og Jesus ynkedes inderligt og rørte ved deres Øjne. Og straks bleve de seende, og de fulgte ham.

\chapter{21}

\par 1 Og da de nærmede sig Jerusalem og kom til Bethfage ved Oliebjerget, da udsendte Jesus to Disciple og sagde til dem:
\par 2 "Går hen i den Landsby, som ligger lige for eder; og straks skulle I finde en Aseninde bunden og et Føl hos hende; løser dem og fører dem til mig!
\par 3 Og dersom nogen siger noget til eder, da siger, at Herren har Brug for dem, så skal han straks sende dem."
\par 4 Men dette er sket, for at det skulde opfyldes, der er talt ved Profeten, som siger:
\par 5 "Siger til Zions Datter: Se, din Konge kommer til dig, sagtmodig og ridende på et Asen og på et Trældyrs Føl."
\par 6 Men Disciplene gik hen og gjorde, som Jesus befalede dem;
\par 7 og de hentede Aseninden og Føllet og lagde deres Klæder på dem, og han satte sig derpå.
\par 8 Men de fleste af Folkeskaren bredte deres Klæder på Vejen, andre huggede Grene af Træerne og strøede dem på Vejen.
\par 9 Men Skarerne, som gik foran ham og fulgte efter, råbte og sagde: "Hosanna Davids Søn! velsignet være den, som kommer, i Herrens Navn! Hosanna i det højeste!"
\par 10 Og da han drog ind i Jerusalem, kom hele Staden i Bevægelse og sagde: "Hvem er denne?"
\par 11 Men Skarerne sagde: "Det er Profeten Jesus fra Nazareth i Galilæa."
\par 12 Og Jesus gik ind i Guds Helligdom og uddrev alle dem, som solgte og købte i Helligdommen, og han væltede Vekselerernes Borde og Duekræmmernes Stole.
\par 13 Og han siger til dem: "Der er skrevet: Mit Hus skal kaldes et Bedehus; men I gøre det til en Røverkule."
\par 14 Og der kom blinde og lamme til ham i Helligdommen, og han helbredte dem.
\par 15 Men da Ypperstepræsterne og de skriftkloge så de Undergerninger, som han gjorde, og Børnene, som råbte i Helligdommen og sagde: "Hosanna Davids Søn!" bleve de vrede og sagde til ham:
\par 16 "Hører du, hvad disse sige?" Men Jesus siger til dem: "Ja! have I aldrig læst: Af umyndiges og diendes Mund har du beredt dig Lovsang?"
\par 17 Og han forlod dem og gik uden for Staden til Bethania og overnattede der.
\par 18 Men da han om Morgenen igen gik ind til Staden, blev hen hungrig.
\par 19 Og han så et Figentræ ved Vejen og gik hen til det, og han fandt intet derpå uden Blade alene. Og han siger til det: "Aldrig i Evighed skal der vokse Frugt mere på dig!" Og Figentræet visnede straks.
\par 20 Og da Disciplene så det, forundrede de sig og sagde: "Hvorledes kunde Figentræet straks visne?"
\par 21 Men Jesus svarede og sagde til dem: "Sandelig, siger jeg eder, dersom I have Tro og ikke tvivle, da skulle I ikke alene kunne gøre det med Figentræet, men dersom I endog sige til dette Bjerg: Løft dig op og kast dig i Havet, da skal det ske.
\par 22 Og alt, hvad I begære i Bønnen troende, det skulle I få."
\par 23 Og da han kom ind i Helligdommen, kom Ypperstepræsterne og Folkets Ældste hen til ham, medens han lærte, og de sagde: "Af hvad Magt gør du disse Ting, og hvem har givet dig denne Magt?"
\par 24 Men Jesus svarede og sagde til dem: "Også jeg vil spørge eder om een Ting, og dersom I sige mig det, vil også jeg sige eder, af hvad Magt jeg gør disse Ting.
\par 25 Johannes's Dåb, hvorfra var den? Fra Himmelen eller fra Mennesker?" Men de tænkte ved sig selv og sagde: "Sige vi: Fra Himmelen, da vil han sige til os: Hvorfor troede I ham da ikke?
\par 26 Men sige vi: Fra Mennesker, frygte vi for Mængden; thi de holde alle Johannes for en Profet."
\par 27 Og de svarede Jesus og sagde: "Det vide vi ikke." Da sagde også han til dem: "Så siger ikke heller jeg eder, af hvad Magt jeg gør disse Ting.
\par 28 Men hvad tykkes eder? En Mand havde to Børn; og han gik til den første og sagde: Barn! gå hen, arbejd i Dag i min Vingård!
\par 29 Men han svarede og sagde: Nej, jeg vil ikke; men bagefter fortrød han det og gik derhen.
\par 30 Og han gik til den anden og sagde ligeså. Men han svarede og sagde: Ja, Herre! og gik ikke derhen.
\par 31 Hvem af de to gjorde Faderens Villie?" De sige: "Den første." Jesus siger til dem: "Sandelig, siger jeg eder, at Toldere og Skøger gå forud for eder ind i Guds Rige.
\par 32 Thi Johannes kom til eder på Retfærdigheds Vej, og I troede ham ikke, men Toldere og Skøger troede ham; men endskønt I så det, fortrøde I det alligevel ikke bagefter, så I troede ham.
\par 33 Hører en anden Lignelse: Der var en Husbond, som plantede en Vingård og satte et Gærde omkring den og gravede en Perse i den og byggede et Tårn; og han lejede den ud til Vingårdsmænd og drog udenlands.
\par 34 Men da Frugttiden nærmede sig, sendte han sine Tjenere til Vingårdsmændene for at få dens Frugter.
\par 35 Og Vingårdsmændene grebe hans Tjenere, og en sloge de, en dræbte de, og en stenede de.
\par 36 Atter sendte han andre Tjenere hen, flere end de første; og de gjorde ligeså med dem.
\par 37 Men til sidst sendte han sin Søn til dem og sagde: De ville undse sig for min Søn.
\par 38 Men da Vingårdsmændene så Sønnen, sagde de til hverandre: Det er Arvingen; kommer lader os slå ham ihjel og få hans Arv!
\par 39 Og de grebe ham og kastede ham ud af Vingården og sloge ham ihjel.
\par 40 Når da Vingårdens Herre kommer, hvad vil han så gøre med disse Vingårdsmænd?"
\par 41 De sige til ham: "Ilde vil han ødelægge de onde og leje sin Vingård ud til andre Vingårdsmænd, som ville give ham Frugterne i deres Tid."
\par 42 Jesus siger til dem: "Have I aldrig læst i Skrifterne: Den Sten, som Bygningsmændene forkastede, den er bleven til en Hovedhjørnesten; fra Herren er dette kommet, og det er underligt for vore Øjne.
\par 43 Derfor siger jeg eder, at Guds Rige skal tages fra eder og gives til et Folk, som bærer dets Frugter.
\par 44 Og den, som falder på denne Sten, skal slå sig sønder; men hvem den falder på, ham skal den knuse."
\par 45 Og da Ypperstepræsterne og Farisæerne hørte hans Lignelser, forstode de, at han talte om dem.
\par 46 Og de søgte at gribe ham, men frygtede for Skarerne; thi de holdt ham for en Profet.

\chapter{22}

\par 1 Og Jesus tog til Orde og talte atter i Lignelser til dem og sagde:
\par 2 "Himmeriges Rige lignes ved en Konge, som gjorde Bryllup for sin Søn.
\par 3 Og han udsendte sine Tjenere for at kalde de budne til Brylluppet; og de vilde ikke komme.
\par 4 Han udsendte atter andre Tjenere og sagde: Siger til de budne: Se, jeg har beredt mit Måltid, mine Okser og Fedekvæget er slagtet, og alting er rede; kommer til Brylluppet!
\par 5 Men de brøde sig ikke derom og gik hen, den ene på sin Mark, den anden til sit Købmandsskab;
\par 6 og de øvrige grebe hans Tjenere, forhånede og ihjelsloge dem.
\par 7 Men Kongen blev vred og sendte sine Hære ud og slog disse Manddrabere ihjel og satte Ild på deres Stad.
\par 8 Da siger han til sine Tjenere: Brylluppet er beredt, men de budne vare det ikke værd.
\par 9 Går derfor ud på Skillevejene og byder til Brylluppet så mange, som I finde!
\par 10 Og de Tjenere gik ud på Vejene og samlede alle dem, de fandt, både onde og gode; og Bryllupshuset blev fuldt af Gæster.
\par 11 Da nu Kongen gik ind for at se Gæsterne, så han der et Menneske, som ikke var iført Bryllupsklædning.
\par 12 Og han siger til ham: Ven! hvorledes er du kommen herind og har ingen Bryllupsklædning på? Men han tav.
\par 13 Da sagde Kongen til Tjenerne: Binder Fødder og Hænder på ham, og kaster ham ud i Mørket udenfor; der skal der være Gråd og Tænders Gnidsel.
\par 14 Thi mange ere kaldede, men få ere udvalgte."
\par 15 Da gik Farisæerne hen og holdt Råd om, hvorledes de kunde fange ham i Ord.
\par 16 Og de sende deres Disciple til ham tillige med Herodianerne og sige: "Mester! vi vide, at du er sanddru og lærer Guds Vej i Sandhed og ikke bryder dig om nogen; thi du ser ikke på Menneskers Person.
\par 17 Sig os derfor: Hvad tykkes dig? Er det tilladt at give Kejseren Skat eller ej?"
\par 18 Men da Jesus mærkede deres Ondskab, sagde han: "I Hyklere, hvorfor friste I mig?
\par 19 Viser mig Skattens Mønt!" Og de bragte ham en Denar".
\par 20 Og han siger til dem: "Hvis Billede og Overskrift er dette?"
\par 21 De sige til ham: "Kejserens." Da siger han til dem: "Så giver Kejseren, hvad Kejserens er, og Gud, hvad Guds er!"
\par 22 Og da de hørte det,undrede de sig, og de forlode ham og gik bort.
\par 23 Samme Dag kom der Saddukæere til ham, hvilke sige, at der ingen Opstandelse er, og de spurgte ham og, sagde:
\par 24 "Mester! Moses har sagt: Når nogen dør og ikke har Børn, skal hans Broder for Svogerskabets Skyld tage hans Hustru til Ægte og oprejse sin Broder Afkom.
\par 25 Men nu var der hos os syv Brødre; og den første giftede sig og døde; og efterdi han ikke havde Afkom, efterlod han sin Hustru til sin Broder.
\par 26 Ligeså også den anden og den tredje, indtil den syvende;
\par 27 men sidst af alle døde Hustruen.
\par 28 Hvem af disse syv skal nu have hende til Hustru i Opstandelsen? thi de have alle haft hende."
\par 29 Men Jesus svarede og sagde til dem: "I fare vild, idet I ikke kende Skrifterne, ej heller Guds Kraft.
\par 30 Thi i Opstandelsen tage de hverken til Ægte eller bortgiftes, men de ere ligesom Guds Engle i Himmelen.
\par 31 Men hvad de dødes Opstandelse angår, have I da ikke læst, hvad der er talt til eder af Gud, når han siger:
\par 32 Jeg er Abrahams Gud og Isaks Gud og Jakobs Gud. Han er ikke dødes, men levendes Gud."
\par 33 Og da Skarerne hørte dette, bleve de slagne af Forundring over hans Lære.
\par 34 Men da Farisæerne hørte, at han havde stoppet Munden på Saddukæerne, forsamlede de sig.
\par 35 Og en af dem, en lovkyndig, spurgte og fristede ham og sagde:
\par 36 "Mester, hvilket er det store Bud i Loven?"
\par 37 Men han sagde til ham: "Du skal elske Herren din Gud med hele dit Hjerte og med hele din Sjæl og med hele dit Sind.
\par 38 Dette er det store og første Bud.
\par 39 Men et andet er dette ligt: Du skal elske din Næste som dig selv.
\par 40 Af disse to Bud afhænger hele Loven og Profeterne."
\par 41 Men da Farisæerne vare forsamlede, spurgte Jesus dem og sagde:
\par 42 "Hvad tykkes eder om Kristus? Hvis Søn er han?" De sige til ham: "Davids."
\par 43 Han siger til dem: "Hvorledes kan da David i Ånden kalde ham Herre, idet han siger:
\par 44 Herren sagde til min Herre: Sæt dig ved min højre Hånd, indtil jeg får lagt dine Fjender under dine Fødder.
\par 45 Når nu David kalder ham Herre, hvorledes er han da hans Søn?"
\par 46 Og ingen kunde svare ham et Ord, og ingen vovede mere at rette Spørgsmål til ham efter den Dag.

\chapter{23}

\par 1 Da talte Jesus til Skarerne og til sine Disciple og sagde:
\par 2 På Mose Stol sidde de skriftkloge og Farisæerne.
\par 3 Gører og holder derfor alt, hvad de sige eder; men gører ikke efter deres Gerninger; thi de sige det vel, men gøre det ikke.
\par 4 Men de binde svare Byrder, vanskelige at bære, og lægge dem på Menneskenes Skuldre;men selv ville de ikke bevæge dem med en Finger.
\par 5 Men de gøre alle deres Gerninger for at beskues af Menneskene; thi de gøre deres Bederemme brede og Kvasterne på deres Klæder store.
\par 6 Og de ville gerne sidde øverst til Bords ved Måltiderne og på de fornemste Pladser i Synagogerne
\par 7 og lade sig hilse på Torvene og kaldes Rabbi af Menneskene.
\par 8 Men I skulle ikke lade eder kalde Rabbi; thi een er eders Mester, men I ere alle Brødre.
\par 9 Og I skulle ikke kalde nogen på Jorden eders Fader; thi een er eders Fader, han, som er i Himlene.
\par 10 Ej heller skulle I lade eder kalde Vejledere; thi een er eders Vejleder, Kristus.
\par 11 Men den største iblandt eder skal være eders Tjener.
\par 12 Men den, som ophøjer sig selv, skal fornedres, og den, som fornedrer sig selv, skal ophøjes.
\par 13 Men ve eder, I skriftkloge og Farisæere, I Hyklere! thi I tillukke Himmeriges Rige for Menneskene; thi I gå ikke derind, og dem, som ville gå ind, tillade I det ikke.
\par 14 Ve eder, I skriftkloge og Farisæere, I Hyklere! thi I opæde Enkers Huse og bede på Skrømt længe; derfor skulle I få des hårdere Dom.
\par 15 Ve eder, I skriftkloge og Farisæere, I Hyklere! thi I drage om til Vands og til Lands for at vinde en eneste Tilhænger; og når han er bleven det, gøre I ham til et Helvedes Barn, dobbelt så slemt, som I selv ere.
\par 16 Ve eder, I blinde Vejledere! I, som sige: Den, som sværger ved Templet, det er intet; men den, som sværger ved Guldet i Templet, han er forpligtet.
\par 17 I Dårer og blinde! hvilket er da størst? Guldet eller Templet, som helliger Guldet?
\par 18 Fremdeles: Den, som sværger ved Alteret, det er intet; men den, som sværger ved Gaven derpå, han er forpligtet.
\par 19 I Dårer og blinde! hvilket er da størst? Gaven eller Alteret, som helliger Gaven?
\par 20 Derfor, den, som sværger ved Alteret, sværger ved det og ved alt det, som er derpå.
\par 21 Og den, som sværger ved Templet, sværger ved det og ved ham, som bor deri.
\par 22 Og den, som sværger ved Himmelen, sværger ved Guds Trone og ved ham, som sidder på den.
\par 23 Ve eder, I skriftkloge og Farisæere, I Hyklere! thi I give Tiende af Mynte og Dild og Kommen og have forsømt de Ting i Loven, der have større Vægt, Retten og Barmhjertigheden og Troskaben. Disse Ting burde man gøre og ikke forsømme hine.
\par 24 I blinde Vejledere, I, som si Myggen af, men nedsluge Kamelen!
\par 25 Ve eder, I skriftkloge og Farisæere, I Hyklere! thi I rense det udvendige af Bægeret og Fadet; men indvendigt ere de fulde af Rov og Umættelighed.
\par 26 Du blinde Farisæer! rens først det indvendige af Bægeret og Fadet, for at også det udvendige af dem kan blive rent.
\par 27 Ve eder, I skriftkloge og Farisæere, I Hyklere! thi I ere ligesom kalkede Grave, der jo synes dejlige udvendigt,men indvendigt ere fulde af døde Ben og al Urenhed.
\par 28 Således synes også I vel udvortes retfærdige for Menneskene; men indvortes ere I fulde af Hykleri og Lovløshed.
\par 29 Ve eder, I skriftkloge og Farisæere, I Hyklere! thi I bygge Profeternes Grave og pryde de retfærdiges Gravsteder og sige:
\par 30 Havde vi været til i vore Fædres Dage, da havde vi ikke været delagtige med dem i Profeternes Blod.
\par 31 Altså give I eder selv det Vidnesbyrd, at I ere Sønner af dem, som have ihjelslået Profeterne.
\par 32 Så gører da også I eders Fædres Mål fuldt!
\par 33 I Slanger! I Øgleunger! hvorledes kunne I undfly Helvedes Dom?
\par 34 Derfor se, jeg sender til eder Profeter og vise og skriftkloge; nogle af dem skulle I slå ihjel og korsfæste, og nogle af dem skulle I hudstryge i, eders Synagoger og forfølge fra Stad til Stad,
\par 35 for at alt det retfærdige Blod skal komme over eder, som er udgydt på Jorden, fra den retfærdige Abels Blod indtil Sakarias's, Barakias's Søns,Blod, hvem I sloge ihjel imellem Templet og Alteret.
\par 36 Sandelig, siger jeg eder, alt dette skal komme over denne Slægt.
\par 37 Jerusalem! Jerusalem! som ihjelslår Profeterne og stener dem, som ere sendte til dig, hvor ofte vilde jeg samle dine Børn, ligesom en Høne samler sine Kyllinger under Vingerne! Og I vilde ikke.
\par 38 Se, eders Hus lades eder øde!
\par 39 Thi jeg siger eder: I skulle ingenlunde se mig fra nu af, indtil I sige: Velsignet være den, som kommer, i Herrens Navn!"

\chapter{24}

\par 1 Og Jesus gik ud, bort fra Helligdommen, og hans Disciple kom til ham for at vise ham Helligdommens Bygninger.
\par 2 Men han svarede og sagde til dem: "Se I ikke alt dette? Sandelig, siger jeg eder, her skal ikke lades Sten på Sten, som jo skal nedbrydes."
\par 3 Men da han sad på Oliebjerget, kom hans Disciple til ham afsides og sagde: "Sig os, når skal dette ske? Og hvad er Tegnet på din Tilkommelse og Verdens Ende?"
\par 4 Og Jesus svarede og sagde til dem: "Ser til, at ingen forfører eder!
\par 5 Thi mange skulle på mit Navn komme og sige: Jeg er Kristus; og de skulle forføre mange.
\par 6 Men I skulle få at høre om Krige og Krigsrygter. Ser til, lader eder ikke forskrække; thi det må ske; men Enden er ikke endda.
\par 7 Thi Folk skal rejse sig mod Folk, og Rige mod Rige, og der skal være Hungersnød og Jordskælv her og der.
\par 8 Men alt dette er Veernes Begyndelse.
\par 9 Da skulle de overgive eder til Trængsel og slå eder ihjel, og I skulle hades af alle Folkeslagene for mit Navns Skyld.
\par 10 Og da skulle mange forarges og forråde hverandre og hade hverandre.
\par 11 Og mange falske Profeter skulle fremstå og forføre mange.
\par 12 Og fordi Lovløsheden bliver mangfoldig, vil Kærligheden blive kold hos de fleste.
\par 13 Men den, som holder ud indtil Enden, han skal frelses.
\par 14 Og dette Rigets Evangelium skal prædikes i hele Verden til et Vidnesbyrd for alle Folkeslagene; og da skal Enden komme.
\par 15 Når I da se Ødelæggelsens Vederstyggelighed, hvorom der er talt ved Profeten Daniel, stå på hellig Grund, (den, som læser det, han give Agt!)
\par 16 da skulle de, som ere i Judæa, fly ud på Bjergene;
\par 17 den, som er på Taget, stige ikke ned for at hente, hvad der er i hans Hus;
\par 18 og den, som er på Marken, vende ikke tilbage før at hente sine Klæder!
\par 19 Men ve de frugtsommelige og dem, som give Die, i de Dage!
\par 20 Og beder om, at eders Flugt ikke skal ske om Vinteren, ej heller på en Sabbat;
\par 21 thi der skal da være en Trængsel så stor, som der ikke har været fra Verdens Begyndelse indtil nu og heller ikke skal komme.
\par 22 Og dersom disse Dage ikke bleve afkortede, da blev intet Kød frelst; men for de udvalgtes Skyld skulle disse Dage afkortes.
\par 23 Dersom nogen da siger til eder: Se, her er Kristus, eller der! da skulle I ikke tro det.
\par 24 Thi falske Krister og falske Profeter skulle fremstå og gøre store Tegn og Undergerninger, så at også de udvalgte skulde blive forførte, om det var muligt.
\par 25 Se, jeg har sagt eder det forud.
\par 26 Derfor, om de sige til eder: Se, han er i Ørkenen, da går ikke derud; se. han er i Kamrene, da tror det ikke!
\par 27 Thi ligesom Lynet udgår fra Østen og lyser indtil Vesten, således skal Menneskesønnens Tilkommelse være.
\par 28 Hvor Ådselet er, der ville Ørnene samle sig.
\par 29 Men straks efter de Dages Trængsel skal Solen formørkes og Månen ikke give sit Skin og Stjernerne falde ned fra Himmelen, og Himmelens Kræfter skulle rystes.
\par 30 Og da skal Menneskesønnens Tegn vise sig på Himmelen; og da skulle alle Jordens Stammer jamre sig, og de skulle se Menneskesønnen komme på Himmelens Skyer med Kraft og megen Herlighed.
\par 31 Og han skal udsende sine Engle med stærktlydende Basun, og de skulle samle hans udvalgte fra de fire Vinde, fra den ene Ende af Himmelen til den anden.
\par 32 Men lærer Lignelsen af Figentræet: Når dets Gren allerede er bleven blød,og Bladene skyde frem, da skønne I, at Sommeren er nær.
\par 33 Således skulle også I, når I se alt dette, skønne, at han er nær for Døren.
\par 34 Sandelig, siger jeg eder, denne Slægt skal ingenlunde forgå, førend alle disse Ting ere skete.
\par 35 Himmelen og Jorden skulle forgå, men mine Ord skulle ingenlunde forgå.
\par 36 Men om den Dag og Time ved ingen, end ikke Himmelens Engle, heller ikke Sønnen, men kun Faderen alene.
\par 37 Og ligesom Noas dage vare, således skal Menneskesønnens Tilkommelse være.
\par 38 Thi ligesom de i Dagene før Syndfloden åde og drak, toge til Ægte og bortgiftede, indtil den Dag, da Noa gik ind i Arken,
\par 39 og ikke agtede det, førend Syndfloden kom og tog dem alle bort, således skal også Menneskesønnens Tilkommelse være.
\par 40 Da skulle to Mænd være på Marken; den ene tages med, og den anden lades tilbage.
\par 41 To Kvinder skulle male på Kværnen; den ene tages med, og den anden lades tilbage.
\par 42 Våger derfor, thi I vide ikke, på hvilken Dag eders Herre kommer.
\par 43 Men dette skulle I vide, at dersom Husbonden vidste. i hvilken Nattevagt Tyven vilde komme, da vågede han og tillod ikke, at der skete Indbrud i hans Hus.
\par 44 Derfor vorder også I rede; thi Menneskesønnen kommer i den Time, som I ikke mene.
\par 45 Hvem er så den tro og forstandige Tjener, som hans Herre har sat over sit Tyende til at give dem deres Mad i rette Tid?
\par 46 Salig er den Tjener, hvem hans Herre, når han kommer, finder handlende således.
\par 47 Sandelig, siger jeg eder, han skal sætte ham over alt, hvad han ejer.
\par 48 Men dersom den onde Tjener siger i sit Hjerte: Min Herre tøver,
\par 49 og så begynder at slå sine Medtjenere og spiser og drikker med Drankerne,
\par 50 da skal den Tjeners Herre komme på den Dag, han ikke venter, og i den Time, han ikke ved,
\par 51 og hugge ham sønder og give ham hans Lod sammen med Hyklerne; der skal der være Gråd og Tænders Gnidsel.

\chapter{25}

\par 1 Da skal Himmeriges Rige lignes ved ti Jomfruer, som toge deres Lamper og gik Brudgommen i Møde.
\par 2 Men fem af dem vare Dårer, og fem kloge.
\par 3 Dårerne toge nemlig deres Lamper, men toge ikke Olie med sig.
\par 4 Men de kloge toge Olie i deres Kar tillige med deres Lamper.
\par 5 Og da Brudgommen tøvede, slumrede de alle ind og sov.
\par 6 Men ved Midnat lød der et Råb: Se, Brudgommen kommer, går ham i Møde!
\par 7 Da vågnede alle Jomfruerne og gjorde deres Lamper i Stand.
\par 8 Men Dårerne sagde til de kloge: Giver os af eders Olie; thi vore Lamper slukkes.
\par 9 Men de kloge svarede og sagde: Der vilde vist ikke blive nok til os og til eder; går hellere hen til Købmændene og køber til eder selv!
\par 10 Men medens de gik bort for at købe, kom Brudgommen, og de, som vare rede, gik ind med ham til Brylluppet; og Døren blev lukket.
\par 11 Men senere komme også de andre Jomfruer og sige: Herre, Herre, luk op for os!
\par 12 Men han svarede og sagde: Sandelig, siger jeg eder, jeg kender eder ikke.
\par 13 Våger derfor, thi I vide ikke Dagen, ej heller Timen.
\par 14 Thi det er ligesom en Mand, der drog udenlands og kaldte på sine Tjenere og overgav dem sin Ejendom;
\par 15 og en gav han fem Talenter, en anden to, og en tredje en, hver efter hans Evne; og straks derefter drog han udenlands.
\par 16 Men den, som havde fået de fem Talenter, gik hen og købslog med dem og vandt andre fem Talenter
\par 17 Ligeså vandt også den, som havde fået de to Talenter,andre to.
\par 18 Men den, som havde fået den ene, gik bort og gravede i Jorden og skjulte sin Herres Penge.
\par 19 Men lang Tid derefter kommer disse Tjeneres Herre og holder Regnskab med dem.
\par 20 Og den, som havde fået de fem Talenter, kom frem og bragte andre fem Talenter og sagde: Herre! du overgav mig fem Talenter; se, jeg har vundet fem andre Talenter.
\par 21 Hans Herre sagde til ham: Vel, du gode og tro Tjener! du var tro over lidet, jeg vil sætte dig over meget; gå ind til din Herres Glæde!
\par 22 Da kom også han frem, som havde fået de to Talenter, og sagde: Herre! du overgav mig to Talenter; se, jeg har vundet to andre Talenter.
\par 23 Hans Herre sagde til ham: Vel, du gode og tro Tjener! du var tro over lidet, jeg vil sætte dig over meget; gå ind til din Herres glæde!
\par 24 Men også han, som havde fået den ene Talent, kom frem og sagde: Herre! jeg kendte dig, at du er en hård Mand, som høster, hvor du ikke såede, og samler, hvor du ikke spredte;
\par 25 og jeg frygtede og gik hen og skjulte din Talent i Jorden; se, her har du, hvad dit er.
\par 26 Men hans Herre svarede og sagde til ham: Du onde og lade Tjener! du vidste, at jeg høster, hvor jeg ikke såede, og samler, hvor jeg ikke spredte;
\par 27 derfor burde du have overgivet Vekselererne mine Penge; og når jeg kom, da havde jeg fået mit igen med Rente.
\par 28 Tager derfor den Talent fra ham, og giver den til ham, som har de ti Talenter.
\par 29 Thi enhver, som har, ham skal der gives, og han skal få Overflod; men den, som ikke har, fra ham skal endog det tages, som han har.
\par 30 Og kaster den unyttige Tjener ud i Mørket udenfor; der skal der være Gråd og Tænders Gnidsel.
\par 31 Men når Menneskesønnen kommer i sin Herlighed og alle Englene med ham, da skal han sidde på sin Herligheds Trone.
\par 32 Og alle Folkeslagene skulle samles foran ham, og han skal skille dem fra hverandre, ligesom Hyrden skiller Fårene fra Bukkene.
\par 33 Og han skal stille Fårene ved sin højre Side og Bukkene ved den venstre.
\par 34 Da skal Kongen sige til dem ved sin højre Side: Kommer hid. I min Faders velsignede! arver det Rige, som har været eder beredt fra Verdens Grundlæggelse.
\par 35 Thi jeg var hungrig, og I gave mig at spise; jeg var tørstig, og I gave mig at drikke; jeg var fremmed, og I toge mig hjem til eder;
\par 36 jeg var nøgen, og I klædte mig; jeg var syg, og I besøgte mig; jeg var i Fængsel, og I kom til mig.
\par 37 Da skulle de retfærdige svare ham og sige: Herre! når så vi dig hungrig og gave dig Mad, eller tørstig og gave dig at drikke?
\par 38 Når så vi dig fremmed og toge dig hjem til os, eller nøgen og klædte dig?
\par 39 Når så vi dig syg eller i Fængsel og kom til dig?
\par 40 Og Kongen skal svare og sige til dem: Sandelig, siger jeg eder: Hvad I have gjort imod een af disse mine mindste Brødre, have I gjort imod mig.
\par 41 Da skal han også sige til dem ved den venstre Side: Går bort fra mig, I forbandede! til den evige Ild, som er beredt Djævelen og hans Engle.
\par 42 Thi jeg var hungrig, og I gave mig ikke at spise; jeg var tørstig, og I gave mig ikke at drikke;
\par 43 jeg var fremmed, og I toge mig ikke hjem til eder; jeg var nøgen, og I klædte mig ikke; jeg var syg og i Fængsel, og I besøgte mig ikke.
\par 44 Da skulle også de svare og sige: Herre! når så vi dig hungrig eller tørstig eller fremmed eller nøgen eller syg eller i Fængsel og tjente dig ikke?
\par 45 Da skal han svare dem og sige: Sandelig, siger jeg eder: Hvad I ikke have gjort imod een af disse mindste, have I heller ikke gjort imod mig.
\par 46 Og disse skulle gå bort til evig Straf, men de retfærdige til evigt Liv."

\chapter{26}

\par 1 Og det skete, da Jesus havde fuldendt alle disse Ord, sagde han til sine Disciple:
\par 2 "I vide, at om to Dage er det Påske; så forrådes Menneskesønnen til at korsfæstes."
\par 3 Da forsamledes Ypperstepræsterne og Folkets Ældste i Ypperstepræstens Gård; han hed Kajfas.
\par 4 Og de rådsloge om at gribe Jesus med List og ihjelslå ham.
\par 5 Men de sagde: "Ikke på Højtiden, for at der ikke skal blive Oprør iblandt Folket."
\par 6 Men da Jesus var kommen til Bethania, i Simon den spedalskes Hus,
\par 7 kom der en Kvinde til ham, som havde en Alabastkrukke med såre kostbar Salve, og hun udgød den på hans Hoved, medens han sad til Bords.
\par 8 Men da Disciplene så det, bleve de vrede og sagde: "Hvortil denne Spilde?
\par 9 Dette kunde jo være solgt til en høj Pris og være givet til fattige."
\par 10 Men da Jesus mærkede det, sagde han til dem: "Hvorfor volde I Kvinden Fortrædeligheder? Hun har jo gjort en god Gerning imod mig.
\par 11 Thi de fattige have I altid hos eder; men mig have I ikke altid.
\par 12 Thi da hun udgød denne Salve over mit Legeme, gjorde hun det for at berede mig til at begraves.
\par 13 Sandelig, siger jeg eder, hvor som helst i hele Verden dette Evangelium bliver prædiket, skal også det, som hun har gjort, omtales til hendes Ihukommelse."
\par 14 Da gik en af de tolv, han, som hed Judas Iskariot, hen til Ypperstepræsterne
\par 15 og sagde: "Hvad ville I give mig, så skal jeg forråde ham til eder?" Men de betalte ham tredive Sølvpenge".
\par 16 Og fra den Stund søgte han Lejlighed til at forråde ham.
\par 17 Men på den første Dag af de usyrede Brøds Højtid kom Disciplene til Jesus og sagde: "Hvor vil du, at vi skulle træffe Forberedelse for dig til at spise Påskelammet?"
\par 18 Men han sagde: "Går ind i Staden til den og den Mand, og siger til ham: Mesteren siger: Min Time er nær; hos dig holder jeg Påske med mine Disciple."
\par 19 Og Disciplene gjorde, som Jesus befalede dem, og beredte Påskelammet.
\par 20 Men da det var blevet Aften, sad han til Bords med de tolv.
\par 21 Og medens de spiste, sagde han: "Sandelig, siger jeg eder, en af eder vil forråde mig."
\par 22 Og de bleve såre bedrøvede og begyndte hver især at sige til ham: "Det er dog vel ikke mig, Herre?"
\par 23 Men han svarede og sagde: "Den, som dyppede Hånden tillige med mig i Fadet, han vil forråde mig.
\par 24 Menneskesønnen går vel bort, som der er skrevet om ham; men ve det Menneske, ved hvem Menneskesønnen bliver forrådt! Det var godt for det Menneske, om han ikke var født."
\par 25 Men Judas, som forrådte ham, svarede og sagde: "Det er dog vel ikke mig, Rabbi?" Han siger til ham: "Du har sagt det."
\par 26 Men medens de spiste, tog Jesus Brød, og han velsignede og brød det og gav Disciplene det og sagde: "Tager, æder; dette er mit Legeme."
\par 27 Og han tog en Kalk og takkede. gav dem den og sagde: "Drikker alle deraf;
\par 28 thi dette er mit Blod, Pagtens, hvilket udgydes for mange til Syndernes Forladelse.
\par 29 Men jeg siger eder, fra nu af skal jeg ingenlunde drikke af denne Vintræets Frugt indtil den Dag, da jeg skal drikke den ny med eder i min Faders Rige."
\par 30 Og da de havde sunget Lovsangen, gik de ud til Oliebjerget.
\par 31 Da siger Jesus til dem: "I skulle alle forarges på mig i denne Nat; thi der er skrevet: Jeg vil slå Hyrden, og Hjordens Får skulle adspredes.
\par 32 Men efter at jeg er bleven oprejst, vil jeg gå forud for eder til Galilæa."
\par 33 Men Peter svarede og sagde til ham: "Om end alle ville forarges på dig, så vil jeg dog aldrig forarges."
\par 34 Jesus sagde til ham: "Sandelig, siger jeg dig, i denne Nat, førend Hanen galer, skal du fornægte mig tre Gange."
\par 35 Peter siger til ham: "Om jeg end skulde dø med dig, vil jeg ingenlunde fornægte dig." Ligeså sagde også alle Disciplene.
\par 36 Da kommer Jesus med dem til en Gård, som kaldes Gethsemane, og han siger til Disciplene: "Sætter eder her, medens jeg går derhen og beder."
\par 37 Og han tog Peter og Zebedæus's to Sønner med sig, og han begyndte at bedrøves og svarlig at ængstes.
\par 38 Da siger han til dem: "Min Sjæl er dybt bedrøvet indtil Døden; bliver her og våger med mig!"
\par 39 Og han gik lidt frem, faldt på sit Ansigt, bad og sagde: "Min Fader! er det muligt, da gå denne Kalk mig forbi; dog ikke som jeg vil, men som du vil."
\par 40 Og han kommer til Disciplene og finder dem sovende, og han siger til Peter: "Så kunde I da ikke våge een Time med mig!
\par 41 Våger og beder, for at I ikke skulle falde i Fristelse! Ånden er vel redebon, men Kødet er skrøbeligt."
\par 42 Han gik atter anden Gang hen, bad og sagde: "Min Fader! hvis denne Kalk ikke kan gå mig forbi, uden jeg drikker den, da ske din Villie!"
\par 43 Og han kom og fandt dem atter sovende, thi deres Øjne vare betyngede.
\par 44 Og han forlod dem og gik atter hen og bad tredje Gang og sagde atter det samme Ord.
\par 45 Da kommer han til Disciplene og siger til dem: "Sove I fremdeles og Hvile eder? Se, Timen er nær, og Menneskesønnen forrådes i Synderes Hænder.
\par 46 Står op, lader os gå; se, han, som forråder mig, er nær."
\par 47 Og medens han endnu talte, se, da kom Judas, en af de tolv, og med ham en stor Skare, med Sværd og Knipler, fra Ypperstepræsterne og Folkets Ældste.
\par 48 Men han, som forrådte ham, havde givet dem et Tegn og sagt: "Den, som jeg kysser, ham er det; griber ham!"
\par 49 Og han trådte straks hen til Jesus og sagde: "Hil være dig, Rabbi!" og kyssede ham.
\par 50 Men Jesus sagde til ham: "Ven, hvorfor kommer du her?" Da trådte de til og lagde Hånd på Jesus og grebe ham.
\par 51 Og se, en af dem, som vare med Jesus, rakte Hånden ud og drog sit Sværd og slog Ypperstepræstens Tjener og huggede hans Øre af.
\par 52 Da siger Jesus til ham: "Stik dit Sværd igen på dets Sted; thi alle de, som tage Sværd, skulle omkomme ved Sværd.
\par 53 Eller mener du, at jeg ikke kan bede min Fader, så han nu tilskikker mig mere end tolv Legioner Engle?
\par 54 Hvorledes skulde da Skrifterne opfyldes, at det bør gå således til?"
\par 55 I den samme Time sagde Jesus til Skarerne: "I ere gåede ud ligesom imod en Røver med Sværd og Knipler for at fange mig. Daglig sad jeg i Helligdommen og lærte, og I grebe mig ikke.
\par 56 Men det er alt sammen sket, for at Profeternes Skrifter skulde opfyldes." Da forlode alle Disciplene ham og flyede.
\par 57 Men de, som havde grebet Jesus, førte ham hen til Ypperstepræsten Kajfas, hvor de skriftkloge og de Ældste vare forsamlede.
\par 58 Men Peter fulgte ham i Frastand indtil Ypperstepræstens Gård, og han gik indenfor og satte sig hos Svendene for at se, hvad Udgang det vilde få.
\par 59 Men Ypperstepræsterne og hele Rådet søgte falsk Vidnesbyrd imod Jesus, for at de kunde aflive ham.
\par 60 Og de fandt intet, endskønt der trådte mange falske Vidner frem.
\par 61 "Denne har sagt: Jeg kan nedbryde Guds Tempel og bygge det op i tre Dage."
\par 62 Og Ypperstepræsten stod op og sagde til ham: "Svarer du intet på, hvad disse vidne imod dig?"
\par 63 Men Jesus tav. Og Ypperstepræsten tog til Orde og sagde til ham: "Jeg besværger dig ved den levende Gud, at du siger os, om du er Kristus, Guds Søn."
\par 64 Jesus siger til ham: "Du har sagt det; dog jeg siger eder: Fra nu af skulle I se Menneskesønnen sidde ved Kraftens højre Hånd og komme på Himmelens Skyer."
\par 65 Da sønderrev Ypperstepræsten sine Klæder og sagde: "Han har talt bespotteligt; hvad have vi længere Vidner nødig? se, nu have I hørt Bespottelsen.
\par 66 Hvad tykkes eder?" Og de svarede og sagde: "Han er skyldig til Døden."
\par 67 Da spyttede de ham i Ansigtet og gave ham Næveslag; andre sloge ham på Kinden
\par 68 og sagde: "Profeter os, Kristus, hvem var det, der slog dig?"
\par 69 Men Peter sad udenfor i Gården; og en Pige kom hen til ham og sagde: "Også du var med Jesus Galilæeren."
\par 70 Men han nægtede det i alles Påhør og sagde: "Jeg forstår ikke, hvad du siger."
\par 71 Men da han gik ud i Portrummet, så en anden Pige ham; og hun siger til dem, som vare der: "Denne var med Jesus af Nazareth."
\par 72 Og han nægtede det atter med en Ed: "Jeg kender ikke det Menneske."
\par 73 Men lidt efter kom de, som stode der, hen og sagde til Peter: "Sandelig, også du er en af dem. dit Mål røber dig jo også."
\par 74 Da begyndte han at forbande sig og sværge: "Jeg kender ikke det Menneske." Og straks galede Hanen.
\par 75 Og Peter kom Jesu Ord i Hu at han havde sagt: "Førend Hanen galer, skal du fornægte mig tre Gange." Og han gik udenfor og græd bitterligt.

\chapter{27}

\par 1 Men da det var blevet Morgen, holdt alle Ypperstepræsterne og Folkets Ældste Råd imod Jesus for at aflive ham.
\par 2 Og de bandt ham og førte ham bort og overgave ham til Landshøvdingen Pontius Pilatus.
\par 3 Da nu Judas, som forrådte ham, så, at han var bleven domfældt, fortrød han det og bragte de tredive Sølvpenge tilbage til Ypperstepræsterne og de Ældste og sagde:
\par 4 "Jeg har syndet, idet jeg forrådte uskyldigt Blod." Men de sagde: "Hvad kommer det os ved? se du dertil."
\par 5 Og han kastede Sølvpengene ind i Templet, veg bort og gik hen og hængte sig.
\par 6 Men Ypperstepræsterne toge Sølvpengene og sagde: "Det er ikke tilladt at lægge dem til Tempelskatten; thi det er Blodpenge."
\par 7 Men efter at have holdt Råd købte de Pottemagermarken derfor til Gravsted for de fremmede.
\par 8 Derfor blev den Mark kaldt Blodmarken indtil den Dag i Dag.
\par 9 Da opfyldtes det, som er talt ved Profeten Jeremias, som siger: "Og de toge de tredive Sølvpenge, Prisen for den vurderede, hvem de vurderede for Israels Børn,
\par 10 og de gav dem for Pottemagermarken, som Herren befalede mig."
\par 11 Men Jesus blev stillet for Landshøvdingen, og Landshøvdingen spurgte ham og sagde: "Er du Jødernes Konge?" Men Jesus sagde til ham: "Du siger det."
\par 12 Og da han blev anklaget af Ypperstepræsterne og de Ældste, svarede han intet.
\par 13 Da siger Pilatus til ham: "Hører du ikke, hvor meget de vidne imod dig?"
\par 14 Og han svarede ham end ikke på et eneste Ord, så at Landshøvdingen undrede sig såre.
\par 15 Men på Højtiden plejede Landshøvdingen at løslade Mængden een Fange, hvilken de vilde.
\par 16 Og de havde dengang en berygtet Fange, som hed Barabbas.
\par 17 Da de vare forsamlede, sagde Pilatus derfor til dem: "Hvem ville I, at jeg skal løslade eder: Barabbas eller Jesus, som kaldes Kristus?"
\par 18 Thi han vidste, at det var af Avind, de havde overgivet ham.
\par 19 Men medens han sad på Dommersædet, sendte hans Hustru Bud til ham og sagde: "Befat dig ikke med denne retfærdige; thi jeg har lidt meget i Dag i en Drøm før hans Skyld."
\par 20 Men Ypperstepræsterne og de Ældste overtalte Skarerne til, at de skulde begære Barabbas, men ihjelslå Jesus.
\par 21 Og Landshøvdingen svarede og sagde til dem: "Hvilken af de to ville I, at jeg skal løslade eder?" Men de sagde: "Barabas."
\par 22 Pilatus siger til dem: "Hvad skal jeg da gøre med Jesus, som kaldes Kristus?" De sige alle: "Lad ham blive korsfæstet!"
\par 23 Men Landshøvdingen sagde: "Hvad ondt har han da gjort?" Men de råbte end mere og sagde: "Lad ham blive korsfæstet!"
\par 24 Men da Pilatus så, at han intet udrettede, men at der blev større Larm, tog han Vand og toede sine Hænder i Mængdens Påsyn og sagde: "Jeg er uskyldig i denne retfærdiges Blod; ser I dertil!"
\par 25 Og hele Folket svarede og sagde: "Hans Blod komme over os og over vore Børn!"
\par 26 Da løslod han dem Barabbas; men Jesus lod han hudstryge og gav ham hen til at korsfæstes.
\par 27 Da toge Landshøvdingens Stridsmænd Jesus med sig ind i Borgen og samlede hele Vagtafdelingen omkring ham.
\par 28 Og de afklædte ham og kastede en Skarlagens Kappe om ham.
\par 29 Og de flettede en Krone af Torne og satte den på hans Hoved og gave ham et Rør i hans højre Hånd; og de faldt på Knæ for ham og spottede ham og sagde: "Hil være dig, du Jødernes Konge!"
\par 30 Og de spyttede på ham og toge Røret og sloge ham på Hovedet.
\par 31 Og da de havde spottet ham, toge de Kappen af ham og iførte ham hans egne Klæder og førte ham hen for at korsfæste ham.
\par 32 Men medens de gik derud, traf de en Mand fra Kyrene, ved Navn Simon; ham tvang de til at bære hans Kors.
\par 33 Og da de kom til et Sted, som kaldes Golgatha, det er udlagt: "Hovedskalsted",
\par 34 gave de ham Eddike at drikke blandet med Galde og da han smagte det, vilde han ikke drikke.
\par 35 Men da de havde korsfæstet ham, delte de hans Klæder imellem sig ved Lodkastning, for at det skulde opfyldes, som er sagt af Profeten: "De delte mine Klæder imellem sig og kastede Lod om mit Klædebon."
\par 36 Og de sade der og holdt Vagt over ham.
\par 37 Og oven over hans Hoved satte de Beskyldningen imod ham skreven således: "Dette er Jesus, Jødernes Konge."
\par 38 Da bliver der korsfæstet to Røvere sammen med ham, en ved den højre og en ved den venstre Side,
\par 39 Og de, som gik forbi, spottede ham, idet de rystede på deres Hoveder og sagde:
\par 40 "Du, som nedbryder Templet og bygger det op i tre Dage, frels dig selv; er du Guds Søn, da stig ned af Korset!"
\par 41 "Ligeså spottede Ypperstepræsterne tillige med de skriftkloge og de Ældste og sagde:
\par 42 "Andre har han frelst, sig selv kan han ikke frelse; er han Israels Konge, så lad ham nu stige ned af Korset, så ville vi tro på ham.
\par 43 Han har sat sin Lid til Gud;han fri ham nu, om han har Behag i ham; thi han har sagt: Jeg er Guds Søn."
\par 44 Og på samme Måde hånede også Røverne ham, som vare korsfæstede med ham.
\par 45 Men fra den sjette Time blev der Mørke over hele Landet indtil den niende Time.
\par 46 Og ved den niende Time råbte Jesus med høj Røst og sagde: "Eli! Eli! Lama Sabaktani?" det er: "Min Gud! min Gud! hvorfor har du forladt mig?"
\par 47 Men nogle af dem, som stode der og hørte det, sagde: "Han kalder på Elias."
\par 48 Og straks løb en af dem hen og tog en Svamp og fyldte den med Eddike og stak den på et Rør og gav ham at drikke.
\par 49 Men de andre sagde: "Holdt! lader os se, om Elias kommer for at frelse ham."
\par 50 Men Jesus råbte atter med høj Røst og opgav Ånden.
\par 51 Og se, Forhænget i Templet splittedes i to Stykker, fra øverst til nederst; og Jorden skjalv, og Klipperne revnede,
\par 52 og Gravene åbnedes; og mange af de hensovede helliges legemer bleve oprejste,
\par 53 og de gik ud af Gravene efter hans Opstandelse og kom ind i den hellige Stad og viste sig for mange.
\par 54 Men da Høvedsmanden og de, som tillige med, ham holdt Vagt over Jesus, så Jordskælvet, og hvad der skete, frygtede de såre og sagde: "Sandelig, denne var Guds Søn."
\par 55 Men der var mange Kvinder der, som så til i Frastand, hvilke havde fulgt Jesus fra Galilæa og tjent ham.
\par 56 Iblandt dem vare Maria Magdalene og Maria, Jakobs og Josefs Moder, og Zebedæus's Sønners Moder.
\par 57 Men da det var blevet Aften, kom en rig Mand fra Arimathæa, ved Navn Josef, som også selv var bleven Jesu Discipel.
\par 58 Han gik til Pilatus og bad om Jesu Legeme. Da befalede Pilatus, at det skulde udleveres.
\par 59 Og Josef tog Legemet og svøbte det i et rent, fint Linklæde
\par 60 og lagde det i sin nye Grav, som han havde ladet hugge i Klippen, og væltede en stor Sten for Indgangen til Graven og gik bort.
\par 61 Men Maria Magdalene og den anden Maria vare der, og de sade lige over for Graven.
\par 62 Men den næste Dag, som var Dagen efter Beredelsesdagen, forsamlede Ypperstepræsterne og Farisæerne sig hos Pilatus
\par 63 og sagde: "Herre! vi ere komne i Hu, at denne Forfører sagde, medens han endnu levede: Tre Dage efter bliver jeg oprejst.
\par 64 Befal derfor, at Graven skal sikkert bevogtes indtil den tredje Dag, for at ikke hans Disciple skulle komme og stjæle ham og sige til Folket: "Han er oprejst fra de døde; og da vil den sidste Forførelse blive værre end den første,"
\par 65 Pilatus sagde til dem: "Der have I en Vagt; går hen og bevogter den sikkert, som I bedst vide!"
\par 66 Og de gik hen og bevogtede Graven sikkert med Vagten efter at have sat Segl for Stenen.

\chapter{28}

\par 1 Men efter Sabbaten, da det gryede ad den første Dag i Ugen, kom Maria Magdalene og den anden Maria for at se til Graven.
\par 2 Og se, der skete et stort Jordskælv; thi en Herrens Engel for ned fra Himmelen og trådte til og væltede Stenen bort og satte sig på den.
\par 3 Men hans Udseende var ligesom et Lyn og hans Klædebon hvidt som Sne.
\par 4 Men de, som holdt Vagt, skælvede af Frygt for ham og bleve som døde.
\par 5 Men Engelen tog til Orde og sagde til Kvinderne: "I skulle ikke frygte! thi jeg ved, at I lede efter Jesus den korsfæstede.
\par 6 Han er ikke her; thi han er opstanden, som han har sagt. Kommer hid, ser Stedet, hvor Herren lå!
\par 7 Og går hastigt hen og siger hans Disciple, at han er opstanden fra de døde; og se, han går forud for eder til Galilæa; der skulle I se ham. Se, jeg har sagt eder det."
\par 8 Og de gik hastig bort fra Graven med Frygt og stor Glæde og løb hen for at forkynde hans Disciple det.
\par 9 Men medens de gik for at forkynde hans Disciple det, se, da mødte Jesus dem og sagde: "Hil være eder!" Men de trådte til og omfavnede hans Fødder og tilbade ham.
\par 10 Da siger Jesus til dem: "Frygter ikke! går hen og forkynder mine Brødre, at de skulle gå bort til Galilæa, og der skulle de se mig."
\par 11 Men medens de gik derhen, se da kom nogle af Vagten ind i Staden og meldte Ypperstepræsterne alt det, som var sket.
\par 12 Og de samledes med de Ældste og holdt Råd og gave Stridsmændene rigelige Penge
\par 13 og sagde: "Siger: Hans Disciple kom om Natten og stjal ham, medens vi sov.
\par 14 Og dersom Landshøvdingen får det at høre, ville vi stille ham tilfreds og holde eder angerløse."
\par 15 Men de toge Pengene og gjorde, som det var lært dem. Og dette Ord blev udspredt iblandt Jøderne indtil den Dag i Dag.
\par 16 Men de elleve Disciple gik til Galilæa, til det Bjerg, hvor Jesus havde sat dem Stævne.
\par 17 Og da de så ham tilbade de ham; men nogle tvivlede.
\par 18 Og Jesus trådte frem, talte til dem og sagde: "Mig er given al Magt i Himmelen og på Jorden.
\par 19 Går derfor hen og gører alle Folkeslagene til mine Disciple, idet I døbe dem til Faderens og Sønnens og den Helligånds Navn,
\par 20 og idet I lære dem at holde alt det, som jeg har befalet eder. Og se, jeg er med eder alle Dage indtil Verdens Ende."



\end{document}