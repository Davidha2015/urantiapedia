\begin{document}

\title{Markusevangeliet}


\chapter{1}

\par 1 Jesu Kristi, Guds Søns, Evangeliums Begyndelse er således,
\par 2 som der er skrevet hos Profeten Esajas: "Se, jeg sender min Engel for dit Ansigt, han skal berede din Vej.
\par 3 Der er en Røst af en, som råber i Ørkenen: Bereder Herrens Vej, gører hans Stier jævne!"
\par 4 Johannes kom, han, som døbte i Ørkenen og prædikede Omvendelses-Dåb til Syndernes Forladelse.
\par 5 Og hele Judæas Land og alle i Jerusalem gik ud og bleve døbt, af ham i Floden Jordan, idet de bekendte deres Synder
\par 6 Og Johannes var klædt i Kamelhår og havde et Læderbælte om sin Lænd og spiste Græshopper og vild Honning.
\par 7 Og han prædikede og sagde: "Efter mig kommer den, som er stærkere end jeg, hvis Skotvinge jeg ikke er værdig at bøje mig ned og løse.
\par 8 Jeg har døbt eder med Vand, men han skal døbe eder med den Helligånd."
\par 9 Og det skete i de dage, at Jesus kom fra Nazareth i Galilæa og blev døbt af Johannes i Jordan.
\par 10 Og straks da han steg op af Vandet, så han Himlene skilles ad og Ånden ligesom en Due dale ned over ham;
\par 11 og der kom en Røst fra Himlene: "Du er min Søn, den elskede, i dig har jeg Velbehag."
\par 12 Og straks driver Ånden ham ud i Ørkenen.
\par 13 Og han var i Ørkenen fyrretyve Dage, medens han fristedes af Satan, og han var blandt Dyrene; og Englene tjente ham.
\par 14 Men efter at Johannes var kastet i Fængsel, kom Jesus til Galilæa og prædikede Guds Evangelium
\par 15 og sagde: "Tiden er fuldkommet, og Guds Rige er kommet nær; omvender eder og tror på Evangeliet!"
\par 16 Og medens han gik langs Galilæas Sø, så han Simon og Simons Broder Andreas i Færd med at kaste Garn i Søen; thi de vare Fiskere.
\par 17 Og Jesus sagde til dem: "Følger efter mig, så vil jeg gøre eder til Menneskefiskere."
\par 18 Og de forlode straks Garnene og fulgte ham.
\par 19 Og da han gik lidt videre frem, så han, Jakob, Zebedæus's Søn, og hans Broder Johannes, som også vare i Færd med at bøde deres Garn i Skibet;
\par 20 og han kaldte straks på dem, og de forlode deres Fader Zebedæus i Skibet med Lejesvendene og gik efter ham.
\par 21 Og de gå ind i Kapernaum. Og straks på Sabbaten gik han ind i Synagogen og lærte,
\par 22 og de bleve slagne af Forundring over hans Lære; thi han lærte dem som en, der havde Myndighed, og ikke som de skriftkloge.
\par 23 Og der var i deres Synagoge et Menneske med en uren Ånd, og han råbte højt
\par 24 og sagde: "Hvad have vi med dig at gøre, Jesus af Nazareth? Er du kommen for at ødelægge os; jeg kender dig, hvem du er, du Guds hellige."
\par 25 Og Jesus truede ham og sagde: "Ti, og far ud af ham!"
\par 26 Og den urene Ånd sled i ham og råbte med høj Røst og for ud af ham.
\par 27 Og de bleve alle forfærdede, så at de spurgte hverandre og sagde: "Hvad er dette? en ny Lære med Myndighed; også over de urene Ånder byder han, og de lyde ham."
\par 28 Og Rygtet om ham kom straks ud alle Vegne i hele det omliggende Land i Galilæa.
\par 29 Og straks, da de vare gåede ud af Synagogen, kom de ind i Simons og Andreas's Hus med Jakob og Johannes.
\par 30 Men Simons Svigermoder lå og havde Feber, og straks tale de til ham om hende;
\par 31 og han gik hen til hende, tog hende ved Hånden og rejste hende op, og Feberen forlod hende, og hun vartede dem op.
\par 32 Men da det var blevet Aften, og Solen var gået ned, førte de til ham alle de syge og besatte,
\par 33 og hele Byen var forsamlet foran Døren.
\par 34 Og han helbredte mange, som lede af mange Hånde Sygdomme, og han uddrev mange onde Ånder; og han tillod ikke de onde Ånder at tale, fordi de kendte ham.
\par 35 Og om Morgenen længe før Dag stod han op og gik ud og gik hen til et øde Sted, og der bad han:
\par 36 Og Simon og de, som vare med ham, skyndte sig efter ham.
\par 37 Og de fandt ham, og de sige til ham: "Alle lede efter dig."
\par 38 Og han siger til dem: "Lader os gå andetsteds hen til de nærmeste Småbyer, for at jeg kan prædike også der; thi dertil er jeg udgået."
\par 39 Og han kom og prædikede i deres Synagoger i hele Galilæa og uddrev de onde Ånder.
\par 40 Og en spedalsk kommer til ham, beder ham og falder på Knæ for ham og siger til ham: "Om du vil, så kan du rense mig."
\par 41 Og han ynkedes inderligt og udrakte Hånden og rørte ved ham og siger til ham: "Jeg vil; bliv ren!"
\par 42 Og straks forlod Spedalskheden ham, og han blev renset.
\par 43 Og han drev ham straks bort, idet han bød ham strengt
\par 44 og sagde til ham: "Se til, at du ikke siger noget til nogen herom; men gå hen, fremstil dig selv for Præsten, og offer for din Renselse det, som Moses har befalet, til Vidnesbyrd for dem!"
\par 45 Men da han kom ud, begyndte han at fortælle meget og udsprede Rygtet derom, så at han ikke mere kunde gå åbenlyst ind i en By; men han var udenfor på øde Steder, og de kom til ham alle Vegne fra.

\chapter{2}

\par 1 Og da han nogle Dage derefter atter gik ind i Kapernaum, spurgtes det, at han var hjemme.
\par 2 Og der samledes mange, så at der ikke mere var Plads, end ikke foran Døren; og han talte Ordet til dem.
\par 3 Og de komme og bringe til ham en værkbruden, der blev båren af fire.
\par 4 Og da de ikke kunde komme nær til ham for Folkeskaren, toge de Taget af, hvor han var; og da de havde brudt Hul, firede de Sengen ned, hvorpå den værkbrudne lå.
\par 5 Og da Jesus så deres Tro, siger han til den værkbrudne: "Søn! dine Synder ere forladte."
\par 6 Men nogle af de skriftkloge sade der og tænkte i deres Hjerter:
\par 7 "Hvorfor taler denne således.? Han taler bespotteligt. Hvem kan forlade Synder uden een, nemlig Gud?"
\par 8 Og Jesus kendte straks i sin Ånd, at de tænkte således ved sig selv, og sagde til dem: "Hvorfor tænke I dette i eders Hjerter?
\par 9 Hvilket er lettest, at sige til den værkbrudne: Dine Synder ere forladte, eller at sige: Stå op, og tag din Seng, og gå?
\par 10 Men for at I skulle vide, at Menneskesønnen har Magt på Jorden til at forlade Synder," siger han til den værkbrudne:
\par 11 "Jeg siger dig: Stå op, tag din Seng, og gå til dit Hus!"
\par 12 Og han stod op og tog straks Sengen og gik ud for alles Øjne, så de alle bleve forfærdede og priste Gud og sagde: "Aldrig have vi set noget sådant."
\par 13 Og han gik atter ud langs Søen, og hele Skaren kom til ham, og han lærte dem.
\par 14 Og da han gik forbi, så han Levi, Alfæus's Søn sidde ved Toldboden, og han siger til ham: "Følg mig!" Og han stod op og fulgte ham.
\par 15 Og det skete, at han sad til Bords i hans Hus, og mange Toldere og Syndere sade til Bords med Jesus, og hans Disciple; thi de vare mange. Og der fulgte også
\par 16 nogle skriftkloge af Farisæerne med ham, og da de så, at han spiste med Toldere og Syndere, sagde de til hans Disciple: "Han spiser og drikker med Toldere og Syndere!"
\par 17 Og da Jesus hørte det, siger han til dem: "De raske trænge ikke til Læge, men de syge. Jeg er ikke kommen for at kalde retfærdige, men Syndere."
\par 18 Og Johannes's Disciple og Farisæerne fastede, og de komme og sige til ham: "Hvorfor faste Johannes's Disciple og Farisæernes Disciple, men dine Disciple faste ikke?"
\par 19 Og Jesus sagde til dem: "Kunne Brudesvendene faste, medens Brudgommen er hos dem? Så længe de have Brudgommen hos sig kunne de ikke faste.
\par 20 Men der skal komme Dage, da Brudgommen bliver tagen fra dem, da skulle de faste på den Dag.
\par 21 Ingen syr en Lap af uvalket Klæde på et gammelt Klædebon; ellers river den nye Lap på det gamle Klædebon dette itu, og der bliver et værre Hul.
\par 22 Og ingen kommer ung Vin på gamle Læderflasker; ellers sprænger Vinen Læderflaskerne, og Vinen ødelægges såvel som Læderflaskerne; men kom ung Vin på nye Læderflasker!"
\par 23 Og det skete, at han vandrede på Sabbaten igennem en Sædemark, og hans Disciple begyndte, imedens de gik, at plukke Aks.
\par 24 Og Farisæerne sagde til ham: "Se, hvorfor gøre de på Sabbaten, hvad der ikke er tilladt?"
\par 25 Og han siger til dem "Have I aldrig læst, hvad David gjorde, da han kom i Nød og blev hungrig, han selv og de, som vare med ham?
\par 26 Hvorledes han gik ind i Guds Hus, da Abiathar var Ypperstepræst, og spiste Skuebrødene, som det ikke er nogen tilladt at spise uden Præsterne, og gav også dem, som vare med ham?"
\par 27 Og han sagde til dem: "Sabbaten blev til for Menneskets Skyld og ikke Mennesket for Sabbatens Skyld.
\par 28 Derfor er Menneskesønnen Herre også over Sabbaten."

\chapter{3}

\par 1 Og han, gik atter ind i en Synagoge, og der var der en Mand, som havde en vissen Hånd.
\par 2 Og de toge Vare på ham, om han vilde helbrede ham på Sabbaten, for at de kunde anklage ham.
\par 3 Og han siger til Manden, som havde den visne Hånd!"Træd frem her i Midten!"
\par 4 Og han siger til dem: "Er det tilladt at gøre godt på Sabbaten eller at gøre ondt, at frelse Liv eller at slå ihjel?" Men de tav.
\par 5 Og han så omkring på dem med Vrede, bedrøvet over deres Hjertes Forhærdelse, og siger til Manden: "Ræk din Hånd ud!" og han rakte den ud, og hans Hånd blev sund igen.
\par 6 Og Farisæerne gik straks ud og holdt Råd med Herodianerne imod ham, hvorledes de kunde slå ham ihjel.
\par 7 Og Jesus drog med sine Disciple bort til Søen, og en stor Mængde fulgte med fra Galilæa; og fra Judæa
\par 8 og fra Jerusalem og fra Idumæa og Landet hinsides Jordan og fra Egnen om Tyrus og Sidon kom de til ham i stor Mængde, da de hørte, hvor store Gerninger han gjorde.
\par 9 Og han sagde til sine Disciple, at en Båd skulde være til Rede til ham for Skarens Skyld, for at de ikke skulde trænge ham.
\par 10 Thi han helbredte mange, så at alle, som havde Plager, styrtede ind på ham for at røre ved ham.
\par 11 Og når de urene Ånder så ham, faldt de ned for ham og råbte og sagde: "Du er Guds Søn."
\par 12 Og han truede dem meget, at de ikke måtte gøre ham kendt.
\par 13 Og han stiger op på Bjerget og hidkalder, hvem han selv vilde; og de gik hen til ham.
\par 14 Og han beskikkede tolv, til at de skulde være hos ham, og til at han kunde udsende dem til at prædike
\par 15 og at have Magt til at uddrive de onde Ånder.
\par 16 Og han beskikkede de tolv, og han tillagde Simon Navnet Peter;
\par 17 fremdeles Jakob, Zebedæus's Søn, og Johannes, Jakobs Broder, og han tillagde dem Navnet Boanerges, det er Tordensønner;
\par 18 og Andreas og Filip og Bartholomæus og Matthæus og Thomas og Jakob, Alfæus's Søn, og Thaddæus og Simon Kananæeren
\par 19 og Judas Iskariot, han, som forrådte ham.
\par 20 Og han kommer hjem, og der samles atter en Skare, så at de end ikke kunne få Mad.
\par 21 Og da hans nærmeste hørte det, gik de ud for at drage ham til sig thi de sagde: "Han er ude af sig selv."
\par 22 Og de skriftkloge, som vare komne ned fra Jerusalem, sagde: "Han har Beelzebul, og ved de onde Ånders Fyrste uddriver han de onde Ånder:"
\par 23 Og han kaldte dem til sig og sagde til dem i Lignelser: "Hvorledes kan Satan uddrive Satan?
\par 24 Og dersom et Rige er kommet i Splid med sig selv, kan samme Rige ikke bestå.
\par 25 Og dersom et Hus er kommet i Splid med sig selv, vil samme Hus ikke kunne bestå.
\par 26 Og dersom Satan har sat sig op imod sig selv og er kommen i Splid med sig selv, kan han ikke bestå, men det er ude med ham.
\par 27 Men ingen kan gå ind i den stærkes Hus og røve hans Ejendele, uden han først binder den stærke, og da kan han plyndre hans Hus.
\par 28 Sandelig, siger jeg eder, alle Ting skulle forlades Menneskenes Børn, Synder og Bespottelser, hvor store Bespottelser de end tale;
\par 29 men den. som taler bespotteligt imod den Helligånd, har evindeligt ingen Forladelse, men skal være skyldig i en evig Synd."
\par 30 De sagde nemlig: "Han har en uren Ånd."
\par 31 Og hans Moder,og hans Brødre komme, og de stode udenfor og sendte Bud ind til ham og lode ham kalde.
\par 32 Og en Skare sad omkring ham; og de sige til ham: "Se; din Moder og dine; Brødre og dine Søstre ere udenfor og spørge efter dig."
\par 33 Og han svarer dem og siger: "Hvem er min Moder og mine Brødre?"
\par 34 Og han så omkring på dem, som sade rundt om ham, og sagde: "Se, her er min Moder og mine Brødre!
\par 35 Thi den, som gør Guds Villie, det er min Broder og Søster og Moder."

\chapter{4}

\par 1 Og han begyndte atter at lære ved søen. Og en meget stor Skare samles om ham, så at han måtte gå om Bord og sætte sig i et Skib på Søen; og hele Skaren var på Land ved Søen.
\par 2 Og han lærte dem meget i Lignelser og sagde til dem i sin Undervisning:
\par 3 "Hører til: Se, en Sædemand gik ud at så.
\par 4 Og det skete, idet han såede, at noget faldt ved Vejen, og Fuglene kom og åde det op.
\par 5 Og noget faldt på Stengrund, hvor det ikke havde megen Jord; og det voksede straks op, fordi det ikke havde dyb Jord.
\par 6 Og da Solen kom op, blev det svedet af, og fordi det ikke havde Rod, visnede det.
\par 7 Og noget faldt iblandt Torne, og Tornene voksede op og kvalte det, og det bar ikke Frugt.
\par 8 Og noget faldt i god Jord og bar Frugt, som skød frem og voksede, og det bar tredive og tresindstyve og hundrede Fold."
\par 9 Og han sagde: "Den som har Øren at høre med, han høre!"
\par 10 Og da han blev ene, spurgte de, som vare om ham, tillige med de tolv ham om Lignelserne.
\par 11 Og han sagde til dem: "Eder er Guds Riges Hemmelighed givet; men dem, som ere udenfor, meddeles alt ved Lignelser,
\par 12 for at de, skønt seende, skulle se og ikke indse og, skønt hørende, skulle høre og ikke forstå, for at de ikke skulle omvende sig og få Forladelse "
\par 13 Og han siger til dem: "Fatte I ikke denne Lignelse? Hvorledes ville I da forstå alle de andre Lignelser?
\par 14 Sædemanden sår Ordet.
\par 15 Men de ved Vejen, det er dem, hvor Ordet bliver sået, og når de høre det, kommer straks Satan og borttager Ordet,som er sået i dem.
\par 16 Og ligeledes de, som blive såede på Stengrunden, det er dem, som, når de høre Ordet, straks modtage det med Glæde;
\par 17 og de have ikke Rod i sig, men holde kun ud til en Tid; derefter, når der kommer Trængsel eller forfølgelse for Ordets Skyld, forarges de straks.
\par 18 Og andre ere de, som blive såede blandt Torne; det er dem, som have hørt Ordet
\par 19 og denne Verdens Bekymringer og Rigdommens Forførelse og Begæringerne efter de andre Ting komme ind og kvæle Ordet, så det bliver uden Frugt.
\par 20 Og de, der bleve såede i god Jord, det er dem, som høre Ordet og modtage det og bære Frugt,tredive og tresindstyve og hundrede Fold."
\par 21 Og han sagde til dem: "Mon Lyset kommer ind for at sættes under Skæppen eller under, Bænken? Mon ikke for at sættes på Lysestagen?
\par 22 Thi ikke er noget skjult uden for at åbenbares; ej heller er det blevet lønligt uden for at komme for Lyset.
\par 23 Dersom nogen har Øren at høre med, han høre!"
\par 24 Og han sagde til dem: "Agter på, hvad I høre! Med hvad Mål I måle, skal der tilmåles eder, og der skal gives eder end mere.
\par 25 Thi den, som har, ham skal der gives; og den, som ikke har, fra ham skal endog det tages, som han har."
\par 26 Og han sagde: "Med Guds Rige er det således, som når en Mand har lagt Sæden i Jorden
\par 27 og sover og står op Nat og Dag, og Sæden spirer og bliver høj, han ved ej selv hvorledes.
\par 28 Af sig selv bærer Jorden Frugt, først Strå, derefter Aks, derefter fuld Kærne i Akset;
\par 29 men når Frugten er tjenlig, sender han straks Seglen ud; thi Høsten er for Hånden."
\par 30 Og han sagde: "Hvormed skulle vi ligne Guds Rige, eller under hvilken Lignelse skulle vi fremstille det?
\par 31 Det er som et Sennepskorn, som, når det sås i Jorden, er mindre end alt andet Frø på Jorden,
\par 32 og når det er sået, vokser det op og bliver større end alle Urterne og skyder store Grene, så at Himmelens Fugle kunne bygge Rede i dets Skygge."
\par 33 Og i mange sådanne Lignelser talte han Ordet til dem, efter som de kunde fatte det.
\par 34 Men uden Lignelse talte han ikke til dem; men i Enerum udlagde han det alt sammen for sine Disciple.
\par 35 Og på den Dag, da det var blevet Aften, siger han til dem: "Lader os fare over til hin Side!"
\par 36 Og de forlade Folkeskaren og tage ham med, som ham sad i Skibet; men der var også andre Skibe med ham.
\par 37 Og der kommer en stærk Stormvind, og Bølgerne sloge ind i Skibet, så at Skibet allerede var ved at fyldes.
\par 38 Og han var i Bagstavnen og sov på en Hovedpude, og de vække ham og sige til ham: "Mester! bryder du dig ikke om, at vi forgå?"
\par 39 Og han stod op og truede Vinden og sagde til Søen: "Ti, vær stille!" og Vinden lagde sig, og det blev ganske blikstille.
\par 40 Og han sagde til dem: "Hvorfor ere I så bange? Hvorfor have I ikke Tro?"
\par 41 Og de frygtede såre og sagde til hverandre: "Hvem er dog denne siden både Vinden og Søen ere ham lydige?"

\chapter{5}

\par 1 Og de kom over til hin Side af Søen til Gerasenernes Land.
\par 2 Og da han trådte ud af Skibet, kom der ham straks i Møde ud fra Gravene en Mand med en uren Ånd.
\par 3 Han havde sin Bolig i Gravene, og ingen kunde længer binde ham, end ikke med Lænker.
\par 4 Thi han havde ofte været bunden med Bøjer og Lænker, og Lænkerne vare sprængte af ham og Bøjerne sønderslidte, og ingen kunde tæmme ham.
\par 5 Og han var altid Nat og Dag i Gravene og på Bjergene, skreg og slog sig selv med Sten.
\par 6 Men da han så Jesus. Langt borte, løb han hen og kastede sig ned for ham
\par 7 og råbte med høj Røst og sagde: "Hvad har jeg med dig at gøre, Jesus, den højeste Guds Søn? Jeg besvæger dig ved Gud, at du ikke piner mig."
\par 8 Thi han sagde til ham: "Far ud af Manden, du urene Ånd!"
\par 9 Og han spurgte ham: "Hvad er dit Navn?" Og han siger til ham: "Legion er mit Navn; thi vi ere mange."
\par 10 Og han bad ham meget om ikke at drive dem ud af Landet.
\par 11 Men der var der ved Bjerget en stor Hjord Svin, som græssede;
\par 12 og de bade ham og sagde: "Send os i Svinene, så vi må fare i dem."
\par 13 Og han tilstedte dem det. Og de urene Ånder fore ud og fore i Svinene; og Hjorden styrtede sig ned over Brinken ud i Søen, omtrent to Tusinde, og de druknede i Søen
\par 14 Og deres Hyrder flyede og forkyndte det i Byen og på Landet; og de kom for at se, hvad det var, som var sket.
\par 15 Og de komme til Jesus og se den besatte, ham, som havde haft Legionen, sidde påklædt og ved Samling, og de frygtede.
\par 16 Men de, som havde set det, fortalte dem, hvorledes det var gået den besatte, og om Svinene.
\par 17 Og de begyndte at bede ham om, at han vilde gå bort fra deres Egn.
\par 18 Og da han gik om Bord i Skibet, bad den, som havde været besat, ham om, at han måtte være hos ham.
\par 19 Og han tilstedte ham det ikke, men siger til ham: "Gå til dit Hus, til dine egne, og forkynd dem, hvor store Ting Herren har gjort imod dig, og at han har forbarmet sig over dig."
\par 20 Og han gik bort og begyndte at kundgøre i Bekapolis, hvor store Ting Jesus havde gjort imod ham; og alle undrede sig.
\par 21 Og da Jesus igen i Skibet var faren over til hin Side, samledes der en stor Skare om ham, og han var ved Søen.
\par 22 Og der kommer en af Synagogeforstanderne ved Navn Jairus, og da han ser ham, falder han ned for hans Fødder.
\par 23 Og han beder ham meget og siger: "Min lille Datter er på sit yderste; o! at du vilde komme og lægge Hænderne på hende, for at hun må frelses og leve!"
\par 24 Og han gik bort med ham, og en stor Skare fulgte ham, og de trængte ham.
\par 25 Og der var en Kvinde, som havde haft Blodflod i tolv År,
\par 26 og hun havde døjet meget af mange Læger og havde tilsat alt, hvad hun ejede, og hun var ikke bleven hjulpen, men tværtimod, det var blevet værre med hende.
\par 27 Da hun havde hørt om Jesus, kom hun bagfra i Skaren og rørte ved hans Klædebon.
\par 28 Thi hun sagde: "Dersom jeg rører blot ved hans Klæder, bliver jeg frelst."
\par 29 Og straks tørredes hendes Blods Kilde, og hun mærkede i sit Legeme, at hun var bleven helbredt fra sin Plage.
\par 30 Og straks da Jesus mærkede på sig selv, at den Kraft var udgået fra ham, vendte han sig om i Skaren og sagde: "Hvem rørte ved mine Klæder?"
\par 31 Og hans Disciple sagde til ham: "Du ser, at Skaren trænger dig, og du siger: Hvem rørte ved mig?"
\par 32 Og han så sig om for at se hende, som havde gjort dette.
\par 33 Men da Kvinden vidste, hvad der var sket hende, kom hun frygtende og bævende og faldt ned for ham og sagde ham hele Sandheden.
\par 34 Men han sagde til hende: "Datter! din Tro har frelst dig; gå bort med Fred, og vær helbredt fra din Plage!"
\par 35 Endnu medens han talte, komme nogle fra Synagogeforstanderens Hus og sige: "Din Datter er død, hvorfor umager du Mesteren længere?"
\par 36 Men Jesus hørte det Ord, som blev sagt, og han siger til Synagogeforstanderen: "Frygt ikke, tro blot!"
\par 37 Og han tilstedte ingen at følge med sig uden Peter og Jakob og Johannes, Jakobs Broder.
\par 38 Og de komme ind i Synagogeforstanderens Hus, og han ser en larmende Hob, der græd og hylede meget.
\par 39 Og han går ind og siger til dem: "Hvorfor larme og græde I? Barnet er ikke død, men det sover."
\par 40 Og de lo ad ham; men han drev dem alle ud, og han tager Barnets Fader og Moder og sine Ledsagere med sig og går ind, hvor Barnet var.
\par 41 Og han tager Barnet ved Hånden og siger til hende: "Talitha kumi!" hvilket er udlagt: "Pige, jeg siger dig, stå op!"
\par 42 Og straks stod Pigen op og gik omkring; thi hun var tolv År gammel. Og de bleve straks overmåde forfærdede
\par 43 Og han bød dem meget, at ingen måtte få dette at vide; og han sagde, at de skulde give hende noget at spise.

\chapter{6}

\par 1 Og han gik bort derfra Og han kommer til sin Fædreneby, og hans Disciple følge ham.
\par 2 Og da det blev Sabbat, begyndte han at lære i Synagogen, og de mange, som hørte ham, bleve slagne af Forundring og sagde: "Hvorfra har han dog dette, og hvad er det for en Visdom, som er given ham,og hvilke kraftige Gerninger der dog sker ved hans Hænder!
\par 3 Er denne ikke Tømmermanden, Marias Søn og Jakobs og Joses's og Judas's og Simons Broder? Og ere ikke hans Søstre her hos os?" Og de forargedes på ham.
\par 4 Og Jesus sagde til dem: "En Profet er ikke foragtet uden i sit eget Fædreland og iblandt sine Slægtninge og i sit Hus."
\par 5 Og han kunde ikke gøre nogen kraftig Gerning der; kun lagde han Hænderne på nogle få syge og helbredte dem
\par 6 Og han forundrede sig over deres Vantro. Og han gik om i Landsbyerne der omkring og lærte.
\par 7 Og han hidkalder de tolv, og han begyndte at udsende dem, to og to, og gav dem Magt over de urene Ånder.
\par 8 Og han bød dem, at de skulde intet tage med på Vejen uden en Stav alene, ikke Brød, ikke Taske, ikke Kobber i Bæltet,
\par 9 men have Sko på og: "Ifører eder ikke to Kjortler!"
\par 10 Og han sagde til dem: "Hvor I komme ind i et Hus, der skulle I blive, indtil I drage bort fra Stedet.
\par 11 Og hvor man ikke vil modtage eder og ikke vil høre eder, der skulle I gå bort fra og afryste Støvet under eders Fødder til Vidnesbyrd imod dem."
\par 12 Og de gik ud og prædikede, at man skulde omvende sig.
\par 13 Og de dreve onde Ånder ud og salvede mange syge med Olie og helbredte dem.
\par 14 Og Kong Herodes hørte det (thi hans Navn var blevet bekendt), og han sagde: "Johannes Døberen er oprejst fra de døde, og derfor virke Kræfterne i ham."
\par 15 Andre sagde: "Det er Elias; " men andre sagde: "Det er en Profet ligesom en af Profeterne."
\par 16 Men da Herodes hørte det, sagde han: "Johannes, som jeg har ladet halshugge, han er oprejst."
\par 17 Thi Herodes havde selv sendt Bud og ladet Johannes gribe og kaste i Fængsel for sin Broder Filips Hustru, Herodias's Skyld; thi han havde taget hende til Ægte.
\par 18 Johannes sagde nemlig til Herodes: "Det er dig ikke tilladt at have din Broders Hustru."
\par 19 Men Herodias bar Nag til ham og vilde gerne slå ham ihjel, og hun kunde det ikke.
\par 20 Thi Herodes frygtede for Johannes, fordi han vidste, at han var en retfærdig og hellig Mand, og han holdt sin Hånd over ham; og når han hørte ham, var han tvivlrådig om mange Ting, og han hørte ham gerne.
\par 21 Og da der kom en belejlig Dag, da Herodes på sin Fødselsdag gjorde et Gæstebud for sine Stormænd og Krigsøversterne og de ypperste i Galilæa,
\par 22 og da selve Herodias's Datter kom ind og dansede, behagede hun Herodes og Gæsterne. Og Kongen sagde til Pigen: "Bed mig, om hvad som helst du vil, så vil jeg give dig det."
\par 23 Og han svor hende til og sagde: "Hvad som helst du beder om, vil jeg give dig, indtil Halvdelen af mit Rige."
\par 24 Og hun gik ud og sagde til sin Moder: "Hvad skal jeg bede om?" Men hun sagde: "Om Johannes Døberens Hoved."
\par 25 Og hun gik straks skyndsomt ind til Kongen, bad og sagde: "Jeg vil, at du straks giver mig Johannes Døberens Hoved på et Fad."
\par 26 Om end Kongen blev meget bedrøvet, vilde han dog for Edernes og Gæsternes Skyld ikke afvise hende:
\par 27 Og Kongen sendte straks en at Vagten og befalede at bringe hans Hoved
\par 28 Og denne gik hen og halshuggede ham i Fængselet; og han bragte hans Hoved på et Fad og gav det til Pigen, og Pigen gav det til sin Moder.
\par 29 Og da hans Disciple hørte det, kom de og toge hans Lig og lagde det i en Grav.
\par 30 Og Apostlene samle sig om Jesus, og de forkyndte ham alt, hvad de havde gjort, og hvad de havde lært.
\par 31 Og han siger til dem: "Kommer nu I med afsides til et øde Sted og hviler eder lidt;" thi der var mange, som gik til og fra, og de havde ikke engang Ro til at spise.
\par 32 Og de droge bort i Skibet til et øde Sted afsides.
\par 33 Og man så dem drage bort, og mange kendte dem, og til Fods strømmede de sammen derhen fra alle Byerne og kom før end de.
\par 34 Og da han gik i Land, så han en stor Skare, og han ynkedes inderligt over dem; thi de vare som Får, der ikke have Hyrde; og han begyndte at lære dem meget.
\par 35 Og da Tiden allerede var fremrykket, kom hans Disciple til ham og sagde: "Stedet er øde, og Tiden er allerede fremrykket.
\par 36 Lad dem gå bort, for at de kunne gå hen i de omliggende Gårde og Landsbyer og købe sig noget at spise."
\par 37 Men han svarede og sagde til dem: "Giver I dem at spise!" Og de sige til ham: "Skulle vi gå hen og købe Brød for to Hundrede Denarer og give dem at spise?"
\par 38 Men han siger til dem: "Hvor mange Brød have I? Går hen og ser efter!" Og da de havde fået det at vide, sige de: "Fem, og to Fisk."
\par 39 Og han bød dem at lade dem alle sætte sig ned i små Flokke i det grønne Græs.
\par 40 Og de satte sig ned, Hob ved Hob, somme på hundrede og somme på halvtredsindstyve.
\par 41 Og han tog de fem Brød og de to Fisk, så op til Himmelen og velsignede; og han brød Brødene og gav sine Disciple dem at lægge for dem, og han delte de to Fisk til dem alle.
\par 42 Og de spiste alle og bleve mætte.
\par 43 Og de optoge tolv Kurve fulde af Stykker, også af Fiskene.
\par 44 Og de, som spiste Brødene, vare fem Tusinde Mænd.
\par 45 Og straks nødte han sine Disciple til at gå om Bord i Skibet og i Forvejen sætte over til hin Side, til Bethsajda, medens han selv lod Skaren gå bort.
\par 46 Og da han havde taget Afsked med dem, gik han op på Bjerget for at bede.
\par 47 Og da det var blevet silde, var Skibet midt på Søen og han alene på Landjorden.
\par 48 Og da han så, at de havde deres Nød med at ro (thi Vinden var dem imod); kommer han ved den fjerde Nattevagt til dem vandrende på Søen. Og han vilde gå dem forbi.
\par 49 Men da de så ham vandre på Søen, mente de, at det var et Spøgelse, og de skrege.
\par 50 Thi de så ham alle og bleve forfærdede. Men han talte straks med dem og sagde til dem: "Værer frimodige, det er mig, frygter ikke!"
\par 51 Og han steg op i Skibet til dem, og Vinden lagde sig, og de forfærdedes over al Måde ved sig selv.
\par 52 Thi de havde ikke fået Forstand af det, som var sket med Brødene; men deres Hjerte var forhærdet
\par 53 Og da de vare farne over til Landet, kom de til Genezareth og lagde til der.
\par 54 Og da de trådte ud af Skibet, kendte man ham straks.
\par 55 Og de løb om i hele den Egn og begyndte at bringe de syge på deres Senge omkring, hvor de hørte, at han var.
\par 56 Og hvor som helst han gik ind i Landsbyer eller Byer eller Gårde, lagde de de syge på Torvene og bade ham om, at de måtte røre blot ved Fligen af hans Klædebon; og alle de, som rørte ved ham, bleve helbredte.

\chapter{7}

\par 1 Og Farisæerne og nogle af de skriftkloge, som vare komne fra Jerusalem, samle sig om ham.
\par 2 Og da de så nogle af hans Disciple holde Måltid med vanhellige, det er utoede, Hænder
\par 3 thi Farisæerne og alle Jøderne spise ikke uden at to Hænderne omhyggeligt, idet de fastholde de gamles Overlevering;
\par 4 og når de komme fra Torvet, spise de ikke uden først at tvætte sig; og der er mange andre Ting, som de have vedtaget at holde, Tvætninger af Bægere og Krus og Kobberkar og Bænke,
\par 5 så spurgte Farisæerne og de skriftkloge ham ad: "Hvorfor vandre dine Disciple ikke efter de gamles Overlevering, men holde Måltid med vanhellige Hænder?"
\par 6 Men han sagde til dem: "Rettelig profeterede Esajas om eder, I Hyklere! som der er skrevet: "Dette Folk ærer mig med Læberne, men deres Hjerte er langt borte fra mig.
\par 7 Men de dyrke mig forgæves, idet de lære Lærdomme, som ere Menneskers Bud."
\par 8 I forlade Guds Bud og holde Menneskers Overlevering."
\par 9 Og han sagde til dem: "Smukt ophæve I Guds Bud, for at I kunne holde eders Overlevering.
\par 10 Thi Moses har sagt: "Ær din Fader og din Moder"; og:"Den, som hader Fader eller Moder, skal visselig dø".
\par 11 Men I sige: Når en Mand siger til sin Fader eller sin Moder: "Det, hvormed du skulde være hjulpen af mig, skal være Korban (det er: Tempelgave),"
\par 12 da tilstede I ham ikke mere at gøre noget for sin Fader eller Moder,
\par 13 idet I ophæve Guds Ord ved eders Overlevering, som I have overleveret; og mange lignende Ting gøre I."
\par 14 Og han kaldte atter Folkeskaren til sig og sagde til dem: "Hører mig alle, og forstår!
\par 15 Der er intet uden for Mennesket, som, når det går ind i ham, kan gøre ham uren; men hvad der går ud af Mennesket, det er det, som gør Mennesket urent.
\par 16 Dersom nogen har Øren at høre med, han høre!"
\par 17 Og da han var gået ind i Huset og var borte fra Skaren, spurgte hans Disciple ham om Lignelsen.
\par 18 Og han siger til dem: "Ere også I så uforstandige? Forstå I ikke, at intet, som udefra går ind i Mennesket, kan gøre ham uren?
\par 19 Thi det går ikke ind i hans Hjerte men i hans Bug og går ud ad den naturlige Vej, og således renses al Maden."
\par 20 Men han sagde: "Det, som går ud af Mennesket, dette gør Mennesket urent.
\par 21 Thi indvortes fra, fra Menneskenes Hjerte, udgå de onde Tanker, Utugt, Tyveri, Mord,
\par 22 Hor, Havesyge, Ondskab, Svig, Uterlighed, et ondt Øje, Forhånelse, Hovmod, Fremfusenhed;
\par 23 alle disse onde Ting udgå indvortes fra og gøre Mennesket urent."
\par 24 Og han stod op og gik bort derfra til Tyrus's og Sidons Egne. Og han gik ind i et Hus og vilde ikke, at nogen skulde vide det. Og han kunde dog ikke være skjult;
\par 25 men en Kvinde, hvis lille Datter havde en uren Ånd, havde hørt om ham og kom straks ind og faldt ned for hans Fødder;
\par 26 (men Kvinden var græsk, af Herkomst en Syrofønikerinde), og hun bad ham om, at han vilde uddrive den onde Ånd af hendes Datter.
\par 27 Og han sagde til hende: "Lad først Børnene mættes; thi det er ikke smukt at tage Børnenes Brød og kaste det for de små Hunde."
\par 28 Men hun svarede og siger til ham: "Jo, Herre! også de små Hunde æde under Bordet af Børnenes Smuler."
\par 29 Og han sagde til hende: "For dette Ords Skyld gå bort; den onde Ånd er udfaren af din Datter"
\par 30 Og hun gik bort til sit Hus og fandt Barnet liggende på Sengen og den onde Ånd udfaren.
\par 31 Og da han gik bort igen fra Tyrus's Egne, kom han over Sidon midt igennem Dekapolis's Egne til Galilæas Sø.
\par 32 Og de bringe ham en døv, som også vanskeligt kunde tale, og bede ham om, at han vilde lægge Hånden på ham.
\par 33 Og han tog ham afsides fra Skaren og lagde sine Fingre i hans Øren og spyttede og rørte ved hans Tunge
\par 34 og så op til Himmelen, sukkede og sagde til ham: "Effata!" det er: lad dig op!
\par 35 Og hans Øren åbnedes, og straks løstes hans Tunges Bånd, og han talte ret.
\par 36 Og han bød dem, at de ikke måtte sige det til nogen; men jo mere han bød dem, desto mere kundgjorde de det.
\par 37 Og de bleve over al Måde slagne af Forundring og sagde: "Han har gjort alle Ting vel; både gør han, at de døve høre, og at målløse tale."

\chapter{8}

\par 1 I de Dage da der atter var en stor Skare, og de intet havde at spise, kaldte han sine Disciple til sig og siger til dem:
\par 2 "Jeg ynkes inderligt over Skaren; thi de have allerede tøvet hos mig i tre Dage og have intet at spise.
\par 3 Og dersom jeg lader dem gå fastende hjem, ville de vansmægte på Vejen, og nogle af dem ere komne langvejsfra."
\par 4 Og hans Disciple svarede ham: "Hvorfra skal nogen kunne mætte disse med Brød her i en Ørken?"
\par 5 Og han spurgte dem: "Hvor mange Brød have I?" Og de sagde:"Syv."
\par 6 Og han byder Skaren af sætte sig ned på Jorden; og han tog de syv Brød, takkede, brød, dem og gav sine Disciple dem, at de skulde lægge dem for; og de lagde dem for Skaren.
\par 7 Og de havde nogle få Småfisk; og han velsignede dem og sagde, af også disse skulde lægges for.
\par 8 Og de spiste og bleve mætte; og de opsamlede af tiloversblevne Stykker syv Kurve.
\par 9 Men de vare omtrent fire Tusinde; og han lod dem gå bort.
\par 10 Og straks gik han om Bord i Skibet med sine Disciple og kom til Dalmanuthas Egne.
\par 11 Og Farisæerne gik ud og begyndte at strides med ham og forlangte af ham et Tegn fra Himmelen for at friste ham.
\par 12 Og han sukkede dybt i sin Ånd og siger: "Hvorfor forlanger denne Slægt et Tegn? Sandelig, siger jeg eder, der skal ikke gives denne Slægt noget Tegn!"
\par 13 Og han forlod dem og gik atter om Bord og for over til hin Side.
\par 14 Og de havde glemt at tage Brød med og havde kun eet Brød med sig i Skibet.
\par 15 Og han bød dem og sagde: "Ser til, tager eder i Vare for Farisæernes Surdejg og Herodes's Surdejg!"
\par 16 Og de tænkte med hverandre: "Det er, fordi vi ikke have Brød."
\par 17 Og da han mærkede dette, siger han til dem: "Hvorfor tænke I på, at I ikke have Brød? Skønne I ikke endnu, og forstå I ikke? Er eders Hjerte forhærdet?
\par 18 Have I Øjne og se ikke? Og have I Øren og høre ikke? Og komme I ikke i Hu?
\par 19 Da jeg brød de fem Brød til de fem Tusinde, hvor mange Kurve fulde af Stykker toge I da op?" De sige til ham: "Tolv."
\par 20 "Og da jeg brød de syv til de fire Tusinde, hvor mange Kurve fulde af Stykker toge I da op?" Og de sige til ham: "Syv."
\par 21 Og han sagde til dem "Hvorledes forstå I da ikke?"
\par 22 Og de komme til Bethsajda. Og man fører en blind til ham og beder ham om, at han vil røre ved ham.
\par 23 Og han tog den blinde ved Hånden og førte ham uden for Landsbyen og spyttede på hans Øjne og lagde Hænderne på ham og spurgte ham, om han så noget.
\par 24 Og han så op og sagde: "Jeg ser Menneskene; thi jeg ser noget ligesom Træer gå omkring."
\par 25 Derefter lagde han atter Hænderne på hans Øjne, og han blev klarsynet og var helbredt og kunde se alle Ting tydeligt.
\par 26 Og han sendte ham hjem og sagde: "Du må ikke gå ind i Landsbyen, ej heller sige det til nogen i Landsbyen."
\par 27 Og Jesus og hans Disciple gik ud til Landsbyerne ved Kæsarea Filippi; og på Vejen spurgte han sine Disciple og sagde til dem: "Hvem sige Menneskene, at jeg er?"
\par 28 Og de sagde til ham: "Johannes Døberen; og andre: Elias; men andre: en af Profeterne."
\par 29 Og han spurgte dem: "Men I, hvem sige I, at jeg er?" Peter svarede og siger til ham: "Du er Kristus."
\par 30 Og han bød dem strengt,at de ikke måtte sige nogen dette om ham.
\par 31 Og han begyndte at lære dem, at Menneskesønnen skulde lide meget og forkastes af de Ældste og Ypperstepræsterne og de skriftkloge og ihjelslås og opstå efter tre Dage.
\par 32 Og han talte dette frit ud. Og Peter tog ham til Side og begyndte at sætte ham i Rette.
\par 33 Men han vendte sig og så på sine Disciple og irettesatte Peter og siger: "Vig bag mig, Satan! thi du sanser ikke, hvad Guds er, men hvad Menneskers er."
\par 34 Og han kaldte Skaren tillige med sine Disciple til sig og sagde til dem: "Den, som vil følge efter mig, han fornægte sig selv og tage sit Kors op og følge mig!
\par 35 Thi den, som vil frelse sit Liv, skal miste det; men den, som mister sit Liv for min og Evangeliets Skyld, han skal frelse det.
\par 36 Thi hvad gavner det et Menneske at vinde den hele Verden og at bøde med sin Sjæl?
\par 37 Thi hvad kunde et Menneske give til Vederlag for sin Sjæl?
\par 38 Thi den, som skammer sig ved mig og mine Ord i denne utro og syndige Slægt, ved ham skal også Menneskesønnen skamme sig, når han kommer i sin Faders Herlighed med de hellige Engle."

\chapter{9}

\par 1 Og han sagde til dem: "Sandelig, siger jeg eder, der er nogle af dem, som stå her, der ingenlunde skulle smage Døden, førend de se Guds Rige være kommet med Kraft."
\par 2 Og seks Dage derefter tager Jesus Peter og Jakob og Johannes med sig og fører dem alene afsides op på et højt Bjerg, og han blev forvandlet for deres Øjne.
\par 3 Og hans Klæder bleve skinnende, meget hvide, så at ingen Blegemand på Jorden kan gøre Klæder så hvide.
\par 4 Og Elias tillige med Moses viste sig for dem, og de samtalede med Jesus.
\par 5 Og Peter tog til Orde og siger til Jesus: "Rabbi! det er godt, at vi ere her, og lader os gøre tre Hytter, dig en og Moses en og Elias en."
\par 6 Thi han vidste ikke, hvad han skulde sige; thi de vare blevne helt forfærdede.
\par 7 Og der kom en Sky, som overskyggede dem; og en Røst kom fra Skyen: "Denne er min Søn, den elskede, hører ham!"
\par 8 Og pludseligt, da de så sig om, så de ingen mere uden Jesus alene hos dem.
\par 9 Og da de gik ned fra Bjerget, bød han dem, at de ikke måtte fortælle nogen, hvad de havde set, førend Menneskesønnen var opstanden fra de døde.
\par 10 Og de fastholdt dette Ord hos sig selv og spurgte hverandre, hvad det er at opstå fra de døde.
\par 11 Og de spurgte ham og sagde: "De skriftkloge sige jo, at Elias bør først komme?"
\par 12 Men han sagde til dem: "Elias kommer først og genopretter alting; og hvorledes er der skrevet om Menneskesønnen? At han skal lide meget og foragtes.
\par 13 Men jeg siger eder, at både er Elias kommen, og de gjorde ved ham alt, hvad de vilde, efter som der er skrevet om ham."
\par 14 Og da de kom til Disciplene, så de en stor Skare omkring dem og skriftkloge, som tvistedes med dem.
\par 15 Og straks studsede hele Skaren, da de så ham, og de løb hen og hilsede ham.
\par 16 Og han spurgte dem: "Hvorom tvistes I med dem?"
\par 17 Og en af Skaren svarede ham: "Mester! jeg har bragt min Søn til dig; han har en målløs Ånd.
\par 18 Og hvor som helst den griber ham, slider den i ham, og han fråder og skærer Tænder, og han visner hen; og jeg har sagt til dine Disciple, at de skulde uddrive den, og de kunde ikke."
\par 19 Men han svarede dem og sagde: "O du vantro Slægt! hvor længe skal jeg være hos eder, hvor længe skal jeg tåle eder? Bringer ham til mig!"
\par 20 Og de ledte ham frem til ham; og da han så ham, sled Ånden straks i ham, og han faldt om på Jorden og væltede sig og fraadede.
\par 21 Og han spurgte hans Fader: "Hvor længe er det siden, at dette er kommet over ham?" Men han sagde: "Fra Barndommen af;
\par 22 og den har ofte kastet ham både i Ild og i Vand for at ødelægge ham; men om du formår noget, da forbarm dig over os, og hjælp os!"
\par 23 Men Jesus sagde til ham: "Om du formår! Alle Ting ere mulige for den, som tror."
\par 24 Straks råbte Barnets Fader og sagde med Tårer: "Jeg tror, hjælp min Vantro!"
\par 25 Men da Jesus så, at Skaren stimlede sammen, truede han den urene Ånd og sagde til den: "Du målløse og døve Ånd! jeg byder dig, far ud af ham, og far ikke mere ind i ham!"
\par 26 Da skreg og sled den meget i ham og for ud, og han blev ligesom død, så at de fleste sagde: "Han er død."
\par 27 Men Jesus tog ham ved Hånden og rejste ham op; og han stod op.
\par 28 Og da han var kommen ind i et Hus, spurgte hans Disciple ham i Enrum: "Hvorfor kunde vi ikke uddrive den?"
\par 29 Og han sagde til dem: "Denne Slags kan ikke fare ud ved noget, uden ved Bøn og Faste."
\par 30 Og da de gik ud derfra, vandrede de igennem Galilæa; og han vilde ikke, at nogen skulde vide det.
\par 31 Thi han lærte sine Disciple og sagde til dem: "Menneskesønnen overgives i Menneskers Hænder, og de skulle slå ham ihjel; og når han er ihjelslået, skal han opstå tre Dage efter."
\par 32 Men de forstode ikke det Ord og frygtede for at spørge ham.
\par 33 Og de kom til Kapernaum, og da han var kommen ind i Huset, spurgte han dem: "Hvad var det, I overvejede med hverandre på Vejen?"
\par 34 Men de tav; thi de havde talt med hverandre på Vejen om, hvem der var den største.
\par 35 Og han satte sig og kaldte på de tolv og siger til dem: "Dersom nogen vil være den første, han skal være den sidste af alle og alles Tjener."
\par 36 Og han tog et lille Barn og stillede det midt iblandt dem og tog det i Favn og sagde til dem:
\par 37 "Den, som modtager eet af disse små Børn for mit Navns Skyld, modtager mig; og den, som modtager mig, modtager ikke mig, men den, som udsendte mig."
\par 38 Johannes sagde til ham: "Mester! vi så en, som ikke følger os, uddrive onde Ånder i dit Navn; og vi forbød ham det, fordi han ikke følger os."
\par 39 Men Jesus sagde: "Forbyder ham det ikke; thi der er ingen, som gør en kraftig Gerning i mit Navn og snart efter kan tale ilde om mig.
\par 40 Thi den, som ikke er imod os, er for os.
\par 41 Thi den, som giver eder et Bæger Vand at drikke i mit Navn, fordi I høre Kristus til, sandelig, siger jeg eder, han skal ingenlunde miste sin Løn
\par 42 Og den, som forarger en af disse små, som tro, for ham var det bedre, at der lå en Møllesten om hans Hals, og han var kastet i Havet.
\par 43 Og dersom din Hånd forarger dig, så hug den af; det er bedre for dig at gå som en Krøbling ind til Livet end at have to Hænder og fare til Helvede til den uudslukkelige Ild,
\par 44 hvor deres Orm ikke dør, og Ilden ikke udslukkes.
\par 45 Og dersom din Fod forarger dig, så hug den af; det er bedre for dig at gå lam ind til Livet end at have to Fødder og blive kastet i Helvede,
\par 46 hvor deres Orm ikke dør, og Ilden ikke udslukkes.
\par 47 Og dersom dit Øje forarger dig, så riv det ud; det er bedre for dig at gå enøjet ind i Guds Rige end at have to Øjne og blive kastet i Helvede,
\par 48 hvor deres Orm ikke dør, og Ilden ikke udslukkes.
\par 49 Thi enhver skal saltes med Ild, og alt Offer skal saltes med Salt.)
\par 50 Saltet er godt; men dersom Saltet bliver saltløst, hvormed ville I da give det sin Kraft igen? Haver Salt i eder selv, og holder Fred med hverandre!"

\chapter{10}

\par 1 Og han bryder op derfra og kommer til Judæas Egne og Landet hinsides Jordan, og atter samler der sig Skarer om ham; og han lærte dem atter, som han plejede.
\par 2 Og Farisæerne kom hen og spurgte ham for at friste ham: "Er det en Mand tilladt at skille sig fra sin Hustru?"
\par 3 Men han svarede og sagde til dem: "Hvad har Moses budt eder?"
\par 4 Men de sagde: "Moses tilstedte at skrive et Skilsmissebrev og skille sig fra hende."
\par 5 Og Jesus sagde til dem: "For eders Hjerters Hårdheds Skyld skrev han eder dette Bud.
\par 6 Men fra Skabningens Begyndelse skabte Gud dem som Mand og Kvinde.
\par 7 Derfor skal en Mand forlade sin Fader og Moder, og holde fast ved sin Hustru;
\par 8 og de to skulle blive til eet Kød. Således ere de ikke længer to, men eet Kød.
\par 9 Derfor, hvad Gud har sammenføjet, må et Menneske ikke adskille.
\par 10 Og i Huset spurgte Disciplene ham atter om dette.
\par 11 Og han siger til dem: "Den, som skiller sig fra sin Hustru og tager en anden til Ægte, han bedriver Hor imod hende.
\par 12 Og dersom hun efter at have skilt sig fra sin Mand ægter en anden, bedriver hun Hor."
\par 13 Og de bare små Børn til ham, for at han skulde røre ved dem; men Disciplene truede dem, som bare dem frem.
\par 14 Men da Jesus så det, blev han vred og sagde til dem: "Lader de små Børn komme til mig; formener dem det ikke, thi Guds Rige hører sådanne til.
\par 15 Sandelig, siger jeg eder, den, som ikke modtager Guds Rige ligesom et lille Barn, han skal ingenlunde komme ind i det."
\par 16 Og han tog dem i Favn og lagde Hænderne på dem og velsignede dem.
\par 17 Og da han gik ud på Vejen, løb en hen og faldt på Knæ for ham og spurgte ham: "Gode Mester! hvad skal jeg gøre, for at jeg kan arve et evigt Liv?"
\par 18 Men Jesus sagde til ham: "Hvorfor kalder du mig god? Ingen er god, uden een, nemlig Gud.
\par 19 Du kender Budene: Du må ikke bedrive Hor; du må ikke slå ihjel; du må ikke stjæle; du må ikke sige falsk Vidnesbyrd; du må ikke besvige; ær din Fader og din Moder."
\par 20 Men han sagde til ham: "Mester! det har jeg holdt alt sammen fra min Ungdom af."
\par 21 Men Jesus så på ham og fattede Kærlighed til ham og sagde til ham: "Een Ting fattes dig; gå bort, sælg alt, hvad du har, og giv det til de fattige, så skal du have en Skat i Himmelen; og kom så og følg mig!"
\par 22 Men han blev ilde til Mode over den Tale og gik bedrøvet bort; thi han havde meget Gods.
\par 23 Og Jesus så sig omkring og siger til sine Disciple: "Hvor vanskeligt komme de, som have Rigdom, ind i Guds Rige!"
\par 24 Men Disciplene bleve forfærdede over hans Ord. Men Jesus tog atter, til Orde og siger til dem: "Børn, hvor vanskeligt er det, at de som forlade sig på Rigdom, kunne komme ind i Guds Rige!
\par 25 Det er lettere for en Kamel at gå igennem et Nåleøje end for en rig at gå ind i Guds Rige."
\par 26 Men de forfærdedes overmåde og sagde til hverandre: "Hvem kan da blive frelst?"
\par 27 Jesus så på dem og siger: "For Mennesker er det umuligt, men ikke for Gud; thi alle Ting ere mulige for Gud."
\par 28 Peter tog til Orde og sagde til ham: "Se, vi have forladt alle Ting og fulgt dig."
\par 29 Jesus sagde: "Sandelig, siger jeg eder, der er ingen, som har forladt Hus eller Brødre eller Søstre eller Moder eller Fader eller Børn eller Marker for min og for Evangeliets Skyld,
\par 30 uden at han jo skal få hundrede Fold igen, nu i denne Tid Huse og Brødre og Søstre og Mødre og Børn og Marker tillige med Forfølgelser, og i den kommende Verden et evigt Liv.
\par 31 Men mange af de første skulle blive de sidste, og af de sidste de første."
\par 32 Men de vare på Vejen op til Jerusalem; og Jesus gik foran dem, og de vare forfærdede, og de, som fulgte med, vare bange. Og han tog atter de tolv til sig og begyndte at sige dem, hvad der skulde times ham
\par 33 "Se, vi drage op til Jerusalem, og Menneskesønnen skal overgives til Ypperstepræsterne og de skriftkloge, og de skulle dømme ham til Døden og overgive ham til Hedningerne;
\par 34 og de skulle spotte ham og spytte på ham og hudstryge ham og ihjelslå ham, og tre Dage efter skal han opstå."
\par 35 Og Jakob og Johannes, Zebedæus's Sønner, gå hen til ham og sige: "Mester! vi ønske, at du vil gøre for os det, vi ville bede dig om."
\par 36 Og han sag,de til dem: "Hvad ønske I, at jeg skal gøre for eder?"
\par 37 Men de sagde til ham: "Giv os, at vi må sidde, den ene ved din højre Side og den anden ved din venstre Side i din Herlighed."
\par 38 Men Jesus sagde til dem: "I vide ikke, hvad I bede om. Kunne I drikke den Kalk, som jeg drikker, eller døbes med den Dåb, som jeg døbes med?"
\par 39 Men de sagde til ham: "Det kunne vi." Men Jesus sagde til dem: "Den Kalk, som jeg drikker, skulle I drikke, og den Dåb, som jeg døbes med, skulle I døbes med;
\par 40 men det at sidde ved min højre eller ved min venstre Side tilkommer det ikke mig at give; men det gives til dem, hvem det er beredt."
\par 41 Og da de ti hørte det, begyndte de at blive, vrede på Jakob og Johannes.
\par 42 Og Jesus kaldte dem til sig og siger til dem: "I vide, at de, der gælde for Folkenes Fyrster; herske over dem, og de store iblandt dem bruge Myndighed over dem.
\par 43 Men således er det ikke iblandt eder; men den, som vil blive stor iblandt eder, skal være eders Tjener;
\par 44 og den, som vil blive den første af eder, skal være alles Tjener;
\par 45 thi også Menneskesønnen er ikke kommen for at lade sig tjene, men for at tjene og give sit Liv til en Genløsning for mange."
\par 46 Og de komme til Jeriko; og da han gik ud af Jeriko tillige med sine, Disciple og en stor Skare, sad Timæus's Søn, Bartimæus, en blind Tigger, ved Vejen.
\par 47 Og da han hørte, at det var Jesus af Nazareth, begyndte han at råbe og sige: "Du Davids Søn, Jesus, forbarm dig over mig!"
\par 48 Og mange truede ham,for at han skulde tie; men han råbte meget stærkere: "Du Davids Søn, forbarm dig over mig!"
\par 49 Og Jesus stod stille og sagde: "Kalder på ham!" Og de kalde på den blinde og sige til ham: "Vær frimodig, stå op! han kalder på dig."
\par 50 Men han kastede sin Overkjortel af sig, sprang op og kom til Jesus.
\par 51 Og Jesus tog til Orde og sagde til ham: "Hvad vil du, at jeg skal gøre for dig?" Men den blinde sagde til ham: "Rabbuni, at jeg kan blive seende!"
\par 52 Og Jesus sagde til ham: "Gå bort, din Tro har frelst dig." Og straks blev han seende, og han fulgte ham på Vejen.

\chapter{11}

\par 1 Og da de nærme sig Jerusalem til Bethfage og Betania ved Oliebjerget, udsender han to af sine Disciple og siger til dem:
\par 2 "Går hen til den Landsby, som ligger lige for eder, og straks, når I komme ind i den, skulle I finde et Føl bundet, på hvilket der endnu aldrig har siddet noget Menneske; løser det og fører det hid!
\par 3 Og dersom nogen siger til eder: Hvorfor gøre I dette? da siger: Herren har Brug for det, og han sender det straks herhen igen."
\par 4 Og de gik hen og fandt Føllet bundet ved Døren udenfor ved Gyden, og de løse det.
\par 5 Og nogle af dem, som stode der, sagde til dem: "Hvad gøre I, at I løse Føllet?"
\par 6 Men de sagde til dem, ligesom Jesus havde sagt, og de tilstedte dem det.
\par 7 Og de føre Føllet til Jesus og lægge deres Klæder på det, og han satte sig på det.
\par 8 Og mange bredte deres Klæder på Vejen, andre Kviste, som de afskare på Markerne.
\par 9 Og de, som gik foran, og de, som fulgte efter, råbte: "Hosanna! velsignet være den, som kommer, i Herrens Navn!
\par 10 Velsignet være vor Fader Davids Rige, som kommer, Hosanna i det højeste!"
\par 11 Og han gik ind i Jerusalem, i Helligdommen, og da han havde beset alt, gik han, da det allerede var Aftenstid, ud til Bethania med de tolv.
\par 12 Og den følgende Dag; da de gik ud fra Bethania, blev han hungrig.
\par 13 Og da han så et Figentræ langt borte, som havde Blade, gik han derhen, om han måske kunde finde noget derpå, og da han kom til det, fandt han intet uden Blade; thi det var ikke Figentid.
\par 14 Og han tog til Orde og sagde til det: "Aldrig i Evighed skal nogen mere spise Frugt af dig!" Og hans Disciple hørte det.
\par 15 Og de komme til Jerusalem; og han gik ind i Helligdommen og begyndte at uddrive dem, som solgte og købte i Helligdommen, og han væltede Vekselerernes Borde og Duekræmmernes Stole.
\par 16 Og han tilstedte ikke, at nogen bar nogen Ting igennem Helligdommen.
\par 17 Og han lærte og sagde til dem: "Er der ikke skrevet, at mit Hus skal kaldes et Bedehus for alle Folkeslagene? Men I have gjort det til en Røverkule."
\par 18 Og Ypperstepræsterne og de skriftkloge hørte det, og de søgte, hvorledes de kunde slå ham ihjel; thi de frygtede for ham, eftersom hele Skaren blev slagen af, Forundring over hans Lære.
\par 19 Og da det blev Aften, gik han uden for Staden.
\par 20 Og da de om Morgenen gik forbi, så de, at Figentræet var visnet fra Roden af.
\par 21 Og Peter kom det i Hu og siger til ham; "Rabbi! se, Figentræet, som du forbandede, er visnet."
\par 22 Og Jesus svarede og siger til dem: "Haver Tro til Gud!
\par 23 Sandelig, siger jeg eder, den, som siger til dette Bjerg: Løft dig op og; kast dig i Havet, og ikke tvivler i sit Hjerte, men tror, at det sker, som han siger, ham skal det ske.
\par 24 Derfor siger jeg eder: Alt, hvad I bede om og begære, tror, at I have fået det, så skal det ske eder.
\par 25 Og når I stå og bede, da forlader, dersom I have noget imod nogen, for at også eders Fader, som er i Himlene, må forlade eder eders Overtrædelser.
\par 26 Men dersom I ikke forlade, skal eders Fader, som er i Himlene, ej heller forlade eders Overtrædelser"
\par 27 Og de komme atter til Jerusalem; og medens han gik omkring i Helligdommen, komme Ypperstepræsterne og de skriftkloge og de Ældste hen til ham.
\par 28 Og de sagde til ham: "Af hvad Magt gør du disse Ting? eller hvem har givet dig denne Magt til at gøre disse Ting?"
\par 29 Men Jesus sagde til dem: "Jeg vil spørge eder om een Ting, og svarer mig derpå, så vil jeg sige eder, af hvad Magt jeg gør disse Ting.
\par 30 Johannes's Dåb, var den fra Himmelen eller fra Mennesker? Svarer mig!"
\par 31 Og de tænkte ved sig selv og sagde: "Sige vi: Fra Himmelen, da vil han sige, hvorfor troede I ham da ikke?
\par 32 Men sige vi: Fra Mennesker" så frygtede de for Folket; thi alle holdt for, at Johannes virkelig var en Profet.
\par 33 Og de svare og sige til Jesus: "Vi vide det ikke." Og Jesus siger til dem: "Så siger jeg eder ikke heller, af hvad Magt jeg gør disse Ting."

\chapter{12}

\par 1 Og han begyndte at tale til dem i Lignelser: "En Mand plantede en Vingård og satte et Gærde derom og gravede en Perse og byggede et Tårn, og han lejede den ud til Vingårdsmænd og drog udenlands.
\par 2 Og da Tiden kom, sendte han en Tjener til Vingårdsmændene, for at han af Vingårdsmændene kunde få af Vingårdens Frugter.
\par 3 Og de grebe ham og sloge ham og sendte ham tomhændet bort.
\par 4 Og han sendte atter en anden Tjener til dem; og ham sloge de i Hovedet og vanærede.
\par 5 Og han sendte en anden; og ham sloge de ihjel; og mange andre; nogle sloge de, og andre dræbte de.
\par 6 Endnu een havde han, en elsket Søn; ham sendte han til sidst til dem, idet han sagde: "De ville undse sig for min Søn."
\par 7 Men hine Vingårdsmænd sagde til hverandre: "Der er Arvingen; kommer lader os slå ham ihjel, så bliver Arven vor."
\par 8 Og de grebe ham og sloge ham ihjel og kastede ham ud af Vingården.
\par 9 Hvad vil da Vingårdens Herre gøre? Han vil komme og ødelægge Vingårdsmændene og give Vingården til andre.
\par 10 Have I ikke også læst dette Skriftord: Den Sten, som Bygningsmændene forkastede, den er bleven til en Hovedhjørnesten?
\par 11 Fra Herren er dette kommet, og det er underligt for vore Øjne."
\par 12 Og de søgte at gribe ham, men de frygtede for Mængden; thi de forstode, at han sagde denne Lignelse imod dem; og de forlode ham og gik bort.
\par 13 Og de sendte nogle til ham af Farisæerne og af Herodianerne, for at de skulde fange ham i Ord.
\par 14 Og de kom og sagde til ham: "Mester! vi vide, at du er sanddru og ikke bryder dig om nogen; thi du ser ikke på Menneskers Person, men lærer Guds Vej i Sandhed. Er det tilladt at give Kejseren Skat eller ej? Skulle vi give eller ikke give?"
\par 15 Men da han så deres Hykleri, sagde han til dem: "Hvorfor friste I mig? Bringer mig en Denar", for at jeg kan se den."
\par 16 Men de bragte den. Og han siger til dem: "Hvis Billede og Overskrift er dette?" Men de sagde til ham: "Kejserens."
\par 17 Og Jesus sagde til dem: "Giver Kejseren, hvad Kejserens er, og Gud, hvad Guds er." Og de undrede sig over ham.
\par 18 Og der kommer Saddukæere til ham, hvilke jo sige, at der ingen Opstandelse er, og de spurgte ham og sagde:
\par 19 "Mester! Moses har foreskrevet os, at når nogens Broder dør og og efterlader, en Hustru og ikke efterlader noget Barn, da skal hans Broder tage hans Hustru og oprejse sin Broder Afkom.
\par 20 Der var syv Brødre; og den første tog en Hustru, og da han døde, efterlod han ikke Afkom.
\par 21 Og den anden tog hende og døde uden at efterlade Afkom, og den tredje ligeså.
\par 22 Og alle syv, de efterlode ikke Afkom. Sidst af dem alle døde og så Hustruen.
\par 23 I Opstandelsen, når de opstå, hvem af dem skal så have hende til Hustru? Thi de have alle syv haft hende til Hustru."
\par 24 Jesus sagde til dem: "Er det ikke derfor, I fare vild, fordi I ikke kende Skrifterne, ej heller Guds Kraft?
\par 25 Thi når de opstå fra de døde, da tage de hverken til Ægte eller bortgiftes, men de ere som Engle i Himlene.
\par 26 Men hvad de døde angår, at de oprejses, have I da ikke læst i Mose Bog i Stedet om Tornebusken, hvorledes Gud talede til ham og sagde: Jeg er Abrahams Gud og Isaks Gud og Jakobs Gud?
\par 27 Han er ikke dødes, men levendes Gud; I fare meget vild."
\par 28 Og en af de skriftkloge, som havde hørt deres Ordskifte og set, at han svarede dem godt, kom til ham og spurgte ham: "Hvilket Bud er det første af alle?"
\par 29 Jesus svarede: "Det første er: Hør Israel! Herren, vor Gud, Herren er een;
\par 30 og du skal elske Herren, din Gud af hele dit Hjerte og af hele din Sjæl og af hele dit Sind og af hele din Styrke.
\par 31 Et andet er dette: Du skal elske din Næste som dig selv. Større end disse er intet andet Bud."
\par 32 Og den skriftkloge sagde til ham: "Rigtigt, Mester, og med Sandhed har du sagt, at han er een, og der er ingen anden foruden ham.
\par 33 Og at elske ham af hele sit Hjerte og af hele sin Forstand og af hele sin Styrke og at elske sin Næste som sig selv, det er mere end alle Brændofrene og Slagtofrene."
\par 34 Og da Jesus så, at han svarede forstandigt, sagde han til ham: "Du er ikke langt fra Guds Rige." Og ingen vovede mere at rette Spørgsmål til ham.
\par 35 Og da Jesus lærte i Helligdommen, tog han til Orde og sagde: "Hvorledes sige de skriftkloge, at Kristus er Davids Søn?
\par 36 David selv sagde ved den Helligånd: Herren sagde til min Herre: Sæt dig ved min, højre Hånd, indtil jeg får lagt dine Fjender som en Skammel for dine Fødder.
\par 37 David selv kalder ham Herre; hvorledes er han da hans Søn?" Og den store Skare hørte ham gerne.
\par 38 Og han sagde i sin Undervisning: "Tager eder i Vare for de skriftkloge, som gerne ville gå i lange Klæder og lade sig hilse på Torvene
\par 39 og gerne ville have de fornemste Pladser i Synagogerne og sidde øverst til Bords ved Måltiderne;
\par 40 de, som opæde Enkers Huse og på Skrømt bede længe, disse skulle få des hårdere Dom."
\par 41 Og han satte sig lige over for Tempelblokken og så, hvorledes Mængden lagde Penge i Blokken, og mange rige lagde meget deri.
\par 42 Og der kom en fattig Enke og lagde to Skærve i, hvilket er en Hvid".
\par 43 Og han kaldte sine Disciple til sig og sagde til dem: "Sandelig, siger jeg eder, denne fattige Enke har lagt mere deri end alle de som lagde i Tempelblokken.
\par 44 Thi de lagde alle af deres Overflod; men hun lagde af sin Fattigdom alt det, hun havde, sin hele Ejendom."

\chapter{13}

\par 1 og da han gik ud af Helligdommen, siger en af hans Disciple til ham: "Mester, se, hvilke Sten og hvilke Bygninger!"
\par 2 Og Jesus sagde til ham: "Ser du disse store Bygninger? der skal ikke lades Sten på Sten, som jo skal nedbrydes."
\par 3 Og da han sad på Oliebjerget, lige over for Helligdommen, spurgte Peter og Jakob og Johannes og Andreas ham afsides:
\par 4 "Sig os, når skal dette ske, og hvilket er Tegnet, når alt dette skal til at fuldbyrdes?"
\par 5 Men Jesus begyndte at sige til dem: "Ser til, at ingen forfører eder!"
\par 6 Mange skulle på mit Navn komme og sige: Det er mig; og de skulle forføre mange.
\par 7 Men når I høre om Krige og Krigsrygter, da lader eder ikke forskrække, thi det må ske; men Enden er ikke endda.
\par 8 Thi Folk skal rejse sig mod Folk, og Rige mod Rige, og der skal være Jordskælv her og der, og der skal være Hungersnød og Oprør. Dette er Veernes Begyndelse.
\par 9 Men I, tager Vare på eder selv; de skulle overgive eder til Rådsforsamlinger og til Synagoger; I skulle piskes og stilles for Landshøvdinger og Konger for min Skyld, dem til et Vidnesbyrd.
\par 10 Og Evangeliet bør først prædikes for alle Folkeslagene.
\par 11 Og når de føre eder hen og overgive eder, da bekymrer eder ikke forud for, hvad I skulle tale; men hvad der bliver givet eder i den samme Time, det skulle I tale; thi I ere ikke de, som tale, men den Helligånd.
\par 12 Og Broder skal overgive Broder til Døden, og Fader sit Barn og Børn skulle stå op mod Forældre og slå dem ihjel.
\par 13 Og I skulle hades af alle for mit Navns Skyld; men den, som holder ud indtil Enden, han skal blive frelst.
\par 14 Men når I se Ødelæggelsens Vederstyggelighed stå, hvor den ikke bør, (den, som læser det, han give Agt! ) da skulle de, som ere i Judæa, fly til Bjergene;
\par 15 men den, som er på Taget, stige ikke ned eller gå ind for at hente noget fra sit Hus;
\par 16 og den, som er på Marken, vende ikke tilbage for at hente sine Klæder!
\par 17 Men ve de frugtsommelige og dem, som give Die, i de Dage!
\par 18 Men beder om, at det ikke skal ske om Vinteren;
\par 19 thi i de Dage skal der være en sådan Trængsel som der ikke har været fra Skabningens Begyndelse, da Gud skabte den, indtil nu, og som der heller ikke skal komme.
\par 20 Og dersom Herren ikke afkortede de dage, da blev intet Kød frelst; men for de udvalgtes Skyld, som han har udvalgt, har han afkortet de Dage
\par 21 Og dersom nogen da siger til eder: Se, her er Kristus, eller se der! da tror det ikke.
\par 22 Thi falske Krister og falske Profeter skulle fremstå og gøre Tegn og Undergerninger for at forføre, om det var muligt, de udvalgte.
\par 23 Men I, vogter eder; jeg har sagt eder alt forud.
\par 24 Men i de dage, efter den Trængsel, skal Solen formørkes, og Månen ikke give sit Skin,
\par 25 og Stjernerne skulle falde ned fra Himmelen, og de Kræfter, som ere i Himlene, skulle rystes.
\par 26 Og da skulle de se Menneskesønnen komme i Skyerne med megen Kraft og Herlighed.
\par 27 Og da skal han udsende sine Engle og samle sine udvalgte fra de fire Vinde, fra Jordens Ende indtil Himmelens Ende.
\par 28 Men lærer Lignelsen af Figentræet: Når dets Gren allerede er bleven blød, og Bladene skyde frem, da skønne I, at Sommeren er nær.
\par 29 Således skulle også I, når I se disse Ting, skønne, af han er nær for Døren.
\par 30 Sandelig, siger jeg eder, denne Slægt skal ingenlunde forgå, førend alle disse ting ere skete
\par 31 Himmelen og Jorden skulle forgå, men mine Ord skulle ingenlunde forgå:
\par 32 Men om den Dag og Time ved ingen, end ikke Englene i Himmelen, heller ikke Sønnen, men alene Faderen.
\par 33 Ser til, våger og beder; thi I vide ikke, når Tiden er der.
\par 34 Ligesom en Mand, der drog udenlands, forlod sit Hus og gav sine Tjenere Fuldmagt, hver sin Gerning, og bød Dørvogteren, at han skulde våge,
\par 35 våger derfor; thi I vide ikke, når Husets Herre kommer, enten om Aftenen eller ved Midnat eller ved Hanegal eller om Morgenen;
\par 36 for at han ikke, når han kommer pludseligt, skal finde eder sovende!
\par 37 Men hvad jeg siger eder, det siger jeg alle: Våger!"

\chapter{14}

\par 1 Men to Dage derefter var det Påske og de usyrede Brøds Højtid. Og Ypperstepræsterne og de skriftkloge søgte, hvorledes de med List kunde gribe og ihjelslå ham.
\par 2 Thi de sagde: "Ikke på Højtiden, for at der ikke skal blive Oprør iblandt Folket."
\par 3 Og da han var i Bethania, i Simon den spedalskes Hus, kom der, medens han sad til Bords, en Kvinde, som havde en Alabastkrukke med ægte, såre kostbar Nardussalve; og hun sønderbrød Alabastkrukken og udgød den på hans Hoved.
\par 4 Men der var nogle, som bleve vrede hos sig selv og sagde: "Hvortil er denne Spilde af Salven sket?
\par 5 Denne Salve kunde jo være solgt for mere end tre Hundrede Denarer og være given til de fattige." Og de overfusede hende.
\par 6 Men Jesus sagde: "Lader hende være, hvorfor volde I hende Fortrædeligheder? Hun har gjort en god Gerning imod mig.
\par 7 De fattige have I jo altid hos eder, og når I ville, kunne I gøre vel imod dem; men mig have I ikke altid.
\par 8 Hun gjorde, hvad hun kunde; hun salvede forud mit Legeme til Begravelsen.
\par 9 Sandelig, siger jeg eder, hvor som helst i hele Verden Evangeliet bliver prædiket, skal også det, som hun har gjort, omtales til hendes Ihukommelse."
\par 10 Og Judas Iskariot, en af de tolv, gik hen til Ypperstepræsterne for at forråde ham til dem.
\par 11 Men da de hørte det, bleve de glade, og de lovede at give ham Penge; og han søgte, hvorledes han kunde få Lejlighed til at forråde ham.
\par 12 Og på de usyrede Brøds første Dag, da man slagtede Påskelammet, sige hans Disciple til ham: "Hvor vil du, at vi skulle gå hen og træffe Forberedelse til, at du kan spise Påskelammet?"
\par 13 Og han sender to af sine Disciple og siger til dem: "Går ind i Staden, så skal der møde eder en Mand, som bærer en Vandkrukke; følger ham;
\par 14 og hvor han går ind, der skulle I sige til Husbonden: Mesteren siger: Hvor er mit Herberge, hvor jeg kan spise Påskelammet med mine Disciple?
\par 15 Og han skal vise eder en stor Sal, opdækket og rede; og der skulle I berede det for os."
\par 16 Og hans Disciple gik bort og kom ind i Staden og fandt det, således som han havde sagt dem; og de beredte Påskelammet.
\par 17 Og da det var blevet Aften, kommer han med de tolv.
\par 18 Og medens de sade til Bords og spiste, sagde Jesus: "Sandelig, siger jeg eder, en af eder, som spiser med mig, vil forråde mig."
\par 19 De begyndte at bedrøves og at sige til ham, en efter en: "Det er dog vel ikke mig?"
\par 20 Men han sagde til dem: "En af de tolv, den, som dypper med mig i Fadet
\par 21 Thi Menneskesønnen går vel bort, som der er skrevet om ham; men ve det, Menneske ved hvem Menneskesønnen bliver forrådt! Det var godt for det Menneske, om han ikke var født."
\par 22 Og medens de spiste, tog han Brød, velsignede og brød det og gav dem det og sagde: "Tager det; dette er mit Legeme."
\par 23 Og han tog en Kalk, takkede og gav dem den; og de drak alle deraf.
\par 24 Og han sagde til dem: "Dette er mit Blod, Pagtens, hvilket udgydes for mange.
\par 25 Sandelig, siger jeg eder, at jeg skal ingen Sinde mere drikke af Vintræets Frugt indtil den Dag, da jeg skal drikke den ny i Guds Rige."
\par 26 Og da de havde sunget Lovsangen, gik de ud til Oliebjerget
\par 27 Og Jesus siger til dem: "I skulle alle forarges; thi der er skrevet: Jeg vil slå Hyrden, og Fårene skulle adspredes.
\par 28 Men efter at jeg er bleven oprejst, vil jeg gå forud for eder til Galilæa."
\par 29 Men Peter sagde til ham: "Dersom de endog alle forarges, vil jeg dog ikke forarges."
\par 30 Og Jesus siger til ham: "Sandelig siger jeg dig, i Dag, i denne Nat, førend Hanen galer to Gange, skal du fornægte mig tre Gange."
\par 31 Men han sagde end yderligere: "Om jeg end skulde dø med dig, vil jeg ingenlunde fornægte dig." Men ligeså sagde de også alle.
\par 32 Og de komme til en Gård, hvis Navn var Gethsemane; og han siger til sine Disciple "Sætter eder her, imedens jeg beder."
\par 33 Og han tager Peter og Jakob og Johannes med sig, og han begyndte at forfærdes og svarlig at ængstes.
\par 34 Og han siger til dem: "Min Sjæl er dybt bedrøvet indtil Døden; bliver her og våger!"
\par 35 Og han gik lidt frem, kastede sig ned på Jorden og bad om, at den Time måtte gå ham forbi, om det var muligt.
\par 36 Og han sagde: "Abba Fader! alting er dig muligt; tag denne Kalk fra mig; dog ikke hvad jeg vil, men hvad du vil."
\par 37 Og han kommer og finder dem sovende og siger til Peter: "Simon, sover du? Kunde du ikke våge een Time?
\par 38 Våger, og beder, for at I ikke skulle falde i Fristelse; Ånden er vel redebon, men Kødet er skrøbeligt."
\par 39 Og han gik atter hen og bad og sagde det samme Ord,
\par 40 Og han vendte tilbage og fandt dem atter sovende; thi deres Øjne vare betyngede, og de vidste ikke, hvad de skulde svare ham.
\par 41 Og han kommer tredje Gang og siger til dem: "Sove I fremdeles og hvile eder? Det er nok; Timen er kommen; se, Menneskesønnen forrådes i Synderes Hænder.
\par 42 Står op, lader os gå; se, han, som forråder mig, er nær."
\par 43 Og straks, medens han endnu talte, kommer Judas, en af de tolv, og med ham en stor Skare med Sværd og Knipler fra Ypperstepræsterne og de skriftkloge og de Ældste.
\par 44 Men han, som forrådte ham, havde givet dem et aftalt Tegn og sagt: "Den, som jeg kysser, ham er det; griber ham, og fører ham sikkert bort!"
\par 45 Og da han kom, trådte han straks hen til ham og siger: "Rabbi! Rabbi!" og han kyssede ham.
\par 46 Men de lagde Hånd på ham og grebe ham.
\par 47 Men en af dem, som stode hos, drog Sværdet, slog Ypperstepræstens Tjener og afhuggede hans Øre.
\par 48 Og Jesus svarede og sagde til dem: "I ere gåede ud som imod en Røver, med Sværd og med Knipler for at fange mig.
\par 49 Daglig var jeg hos eder i Helligdommen og lærte, og I grebe mig ikke; men dette sker, for af Skrifterne skulle opfyldes."
\par 50 Og de forlode ham alle og flyede.
\par 51 Og en enkelt, et ungt Menneske, som havde et Linklæde over det blotte Legeme, fulgte med ham; og de gribe ham;
\par 52 men han slap Linklædet og flygtede nøgen.
\par 53 Og de førte Jesus hen til Ypperstepræsten; og alle Ypperstepræsterne og de Ældste og de skriftkloge komme sammen hos ham.
\par 54 Og Peter fulgte ham i Frastand til ind i Ypperstepræstens Gård, og han sad hos Svendene og varmede sig ved Ilden.
\par 55 Men Ypperstepræsterne og hele Rådet søgte Vidnesbyrd imod Jesus, for at de kunde aflive ham; og de fandt intet.
\par 56 Thi mange sagde falsk Vidnesbyrd imod ham, men Vidnesbyrdene stemte ikke overens.
\par 57 Og nogle stode op og vidnede falsk imod ham og sagde:
\par 58 "Vi have hørt ham sige: Jeg vil nedbryde dette Tempel, som er gjort med Hænder, og i tre Dage bygge et andet, som ikke er gjort med Hænder."
\par 59 Og end ikke således stemte deres Vidnesbyrd overens.
\par 60 Og Ypperstepræsten stod op midt iblandt dem og spurgte Jesus og sagde: "Svarer du slet intet på, hvad disse vidne imod dig?"
\par 61 Men han tav og svarede intet. Atter spurgte Ypperstepræsten ham og siger til ham: "Er du Kristus, den Højlovedes Søn?"
\par 62 Men Jesus sagde: "Jeg er det; og I skulle se Menneskesønnen sidde ved Kraftens højre Hånd og komme med Himmelens Skyer."
\par 63 Men Ypperstepræsten sønderrev sine Klæder og sagde: "Hvad have vi længere Vidner nødig?
\par 64 I have hørt Gudsbespottelsen; hvad tykkes eder?" Men de fældede alle den Dom over ham, at han var skyldig til Døden.
\par 65 Og nogle begyndte af spytte på ham og tilhylle hans Ansigt og give ham Næveslag og sige til ham: "Profeter!" og Svendene modtoge ham med Slag på Kinden.
\par 66 Og medens Peter var nedenfor i Gården, kommer en af Ypperstepræstens Piger,
\par 67 og da hun ser Peter varme sig, ser hun på ham og siger: "Også du var med Nazaræeren, med Jesus."
\par 68 Men han nægtede og sagde: "Jeg hverken ved eller forstår, hvad du siger;" og han gik ud i Forgården, og Hanen galede.
\par 69 Og Pigen så ham og begyndte atter at sige til dem, som stode hos: "Denne er en af dem."
\par 70 Men han nægtede det atter. Og lidt derefter sagde atter de, som stode hos, til Peter: "Sandelig, du er en af dem; du er jo også en Galilæer."
\par 71 Men han begyndte at forbande sig og sværge: "Jeg kender ikke dette Menneske, om hvem I tale."
\par 72 Og straks galede Hanen anden Gang. Og Peter kom det Ord i Hu, som Jesus sagde til ham: "Førend Hanen galer to Gange, skal du fornægte mig tre Gange." Og han brast i Gråd.

\chapter{15}

\par 1 Og straks om Morgenen, da Ypperstepræsterne havde holdt Råd med de Ældste og de skriftkloge, hele Rådet, bandt de Jesus og førte ham bort og overgave ham til Pilatus.
\par 2 Og Pilatus spurgte ham: "Er du Jødernes Konge?" Og han svarede og sagde til ham: "Du siger det."
\par 3 Og Ypperstepræsterne anklagede ham meget.
\par 4 Men Pilatus spurgte ham atter og sagde: "Svarer du slet intet? Se, hvor meget de anklage dig for!"
\par 5 Men Jesus svarede ikke mere noget, så at Pilatus undrede sig.
\par 6 Men på Højtiden plejede han at løslade dem een Fange, hvilken de forlangte.
\par 7 Men der var en, som hed Barabbas, der var fangen tillige med de Oprørere, som under Oprøret havde begået Mord
\par 8 Og Mængden gik op og begyndte at bede om, at han vilde gøre for dem, som han plejede.
\par 9 Men Pilatus svarede dem og sagde: "Ville I, at jeg skal løslade eder Jødernes Konge?"
\par 10 Thi han skønnede, at det var af Avind, at Ypperstepræsterne havde overgivet ham.
\par 11 Men Ypperstepræsterne ophidsede Mængden til at bede om, at han hellere skulde løslade dem Barabbas.
\par 12 Men Pilatus svarede atter og sagde til dem: "Hvad ville I da, jeg skal gøre med ham, som I kalde Jødernes Konge?"
\par 13 Men de råbte atter: "Korsfæst ham!"
\par 14 Men Pilatus sagde til dem: "Hvad ondt har han da gjort?" Men de råbte højlydt: "Korsfæst ham!"
\par 15 Og da Pilatus vilde gøre Mængden tilpas, løslod han dem Barabbas; og Jesus lod han hudstryge og gav ham hen til at korsfæstes.
\par 16 Men Stridsmændene førte ham ind i Gården, det vil sige Borgen, og de sammenkalde hele Vagtafdelingen.
\par 17 Og de iføre ham en Purpurkappe og flette en Tornekrone og sætte den på ham.
\par 18 Og de begyndte at hilse ham: "Hil være dig, du Jødernes Konge!"
\par 19 Og de sloge ham på Hovedet med et Rør og spyttede på ham og faldt på Knæ og tilbade ham.
\par 20 Og da de havde spottet ham, toge de Purpurkappen af ham og iførte ham hans egne Klæder. Og de føre ham ud for at korsfæste ham.
\par 21 Og de tvinge en, som gik forbi, Simon fra Kyrene, som kom fra Marken, Aleksanders og Rufus's Fader, til af bære hans Kors.
\par 22 Og de føre ham til det Sted Golgatha, det er udlagt: "Hovedskalsted"
\par 23 Og de gave ham Vin at drikke med Myrra i; men han tog det ikke.
\par 24 Og de korsfæste ham, og de dele hans Klæder ved at kaste Lod om dem, hvad enhver skulde tage.
\par 25 Men det var den tredje Time, da de korsfæstede ham.
\par 26 Og Overskriften med Beskyldningen imod ham var påskreven således: "Jødernes Konge".
\par 27 Og de korsfæste to Røvere sammen med ham, en ved hans højre og en ved hans venstre Side.
\par 28 Og Skriften blev opfyldt, som siger: "Og han blev regnet iblandt Overtrædere."
\par 29 Og de, som gik forbi, spottede ham, idet de rystede på deres Hoveder og sagde: "Tvi dig! du som nedbryder Templet og bygger det op i tre Dage;
\par 30 frels dig selv ved at stige ned af Korset!"
\par 31 Ligeså spottede også Ypperstepræsterne indbyrdes tillige med de skriftkloge og sagde: "Andre har han frelst, sig selv kan han ikke frelse.
\par 32 Kristus, Israels Konge - lad ham nu stige ned af Korset, for at vi kunne se det og, tro!" Også de, som vare korsfæstede med ham, hånede ham.
\par 33 Og da den sjette Time var kommen, blev der Mørke over hele Landet indtil den niende Time.
\par 34 Og ved den niende Time råbte Jesus med høj Røst og sagde: "Eloi! Eloi! Lama Sabaktani?" det er udlagt: "Min Gud! min Gud! hvorfor har du forladt mig?"
\par 35 Og nogle af dem, som stode hos, sagde, da de hørte det: "Se; han kalder på Elias."
\par 36 Men en løb hen og fyldte en Svamp med Eddike og stak den på et Rør og gav ham at drikke og sagde: "Holdt! lader os se, om Elias kommer for at tage ham ned."
\par 37 Men Jesus råbte med høj Røst og udåndede.
\par 38 Og Forhænget i Templet splittedes i to fra øverst til nederst.
\par 39 Men da Høvedsmanden, som stod hos, lige over for ham, så, af han udåndede på denne Vis, sagde han: "Sandelig, dette Menneske var Guds Søn."
\par 40 Men der var også Kvinder, som så til i Frastand, iblandt hvilke også vare Maria Magdalene og Maria, Jakob den Lilles og Joses's Moder, og Salome,
\par 41 hvilke også fulgte ham og tjente ham, da han var i Galilæa, og mange andre Kvinder, som vare gåede op til Jerusalem med ham.
\par 42 Og da det allerede var blevet Aften, (thi det var Beredelsesdag, det er Forsabbat,)
\par 43 kom Josef fra Arimathæa, en anset Rådsherre, som også selv forventede Guds Rige; han tog Mod til sig og gik ind til Pilatus og bad om Jesu Legeme.
\par 44 Men Pilatus forundrede sig over, at han allerede skulde være død,
\par 45 og han hidkaldte Høvedsmanden og spurgte ham, om han allerede nogen Tid havde været død; og da han fik det at vide af Høvedsmanden, skænkede han Josef Liget.
\par 46 Og denne købte et fint Linklæde, tog ham ned, svøbte ham i Linklædet og lagde ham i en Grav, som var udhugget i en Klippe, og han, væltede en Sten for Indgangen til Graven.
\par 47 Men Maria Magdalene og Maria, Joses's Moder, så, hvor ham blev lagt.

\chapter{16}

\par 1 Og da Sabbaten var forbi købte Maria Magdalene og Maria, Jakobs Moder, og Salome vellugtende Salver for at komme og Salve ham.
\par 2 Og meget årle på den første Dag i Ugen komme de til Graven, da Solen var stået op.
\par 3 Og de sagde til hverandre: "Hvem skal vælte os Stenen fra Indgangen til Graven?"
\par 4 Og da de så op, bleve de var, at Stenen var væltet fra; (thi den var meget stor)
\par 5 Og da de kom ind i Graven, så de en Yngling sidde ved den højre Side, iført et hvidt Klædebon, og de forfærdedes.
\par 6 Men han siger til dem: "Forfærdes ikke! I lede efter Jesus at Nazareth, den korsfæstede; han er opstanden, han er ikke her, se, der er Stedet, hvor de lagde ham.
\par 7 Men går bort, siger til hans Disciple og til Peter at han går forud for eder til Galilæa; der skulle I se ham, som han har sagt eder."
\par 8 Og de gik ud og flyede fra Graven; thi Skælven og Forfærdelse betog dem; og de sagde ikke noget til nogen; thi de frygtede.
\par 9 Men da han var opstanden årle den første Dag i Ugen, åbenbaredes han først for Maria Magdalene, af hvem han havde uddrevet syv onde Ånder.
\par 10 Hun gik hen og forkyndte det for dem, der havde været med ham, og som sørgede og græd.
\par 11 Og da disse hørte, at han levede og var set af hende, troede de det ikke.
\par 12 Men derefter åbenbaredes han for to af dem på Vejen i en anden Skikkelse, medens de gik ud på Landet.
\par 13 Og disse gik hen og forkyndte de andre det. Ikke heller dem troede de.
\par 14 Siden åbenbaredes han for de elleve selv, medens de sade til Bords, og han bebrejdede dem deres Vantro og Hjerters Hårdhed, fordi de ikke havde troet dem, som havde set ham opstanden.
\par 15 Og han sagde til dem: "Går ud i al Verden og prædiker Evangeliet for al Skabningen!
\par 16 Den, som tror og bliver døbt, skal blive frelst; men den, som ikke tror, skal blive fordømt.
\par 17 Men disse Tegn skulle følge dem, som tro: I mit Navn skulle de uddrive onde Ånder; de skulle tale med nye Tunger;
\par 18 de skulle tage på Slanger, og dersom de drikke nogen Gift, skal det ikke skade dem; på syge skulle de lægge Hænder, og de skulle helbredes."
\par 19 Så blev Herren efter at han havde talt med dem, optagen til Himmelen og satte sig ved Guds højre Hånd.
\par 20 Men de gik ud og prædikede alle Vegne, idet Herren arbejdede med og stadfæstede Ordet ved de medfølgende Tegn.



\end{document}