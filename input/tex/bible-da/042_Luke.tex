\begin{document}

\title{Lukasevangeliet}


\chapter{1}

\par 1 Efterdi mange have taget sig for at forfatte en Beretning om de Ting, som ere fuldbyrdede iblandt os,
\par 2 således som de, der fra Begyndelsen bleve Øjenvidner og Ordets Tjenere, have overleveret os:
\par 3 så har også jeg besluttet, efter nøje at have gennemgået alt forfra, at nedskrive det for dig i Orden, mægtigste Theofilus!
\par 4 for at du kan erkende Pålideligheden af de Ting, hvorom du er bleven mundtligt undervist.
\par 5 I de Dage, da Herodes var Konge i Judæa, var der en Præst af Abias Skifte, ved Navn Sakarias; og han havde en Hustru af Arons Døtre, og hendes Navn var Elisabeth.
\par 6 Men de vare begge retfærdige for Gud og vandrede udadlelige i alle Herrens Bud og Forskrifter.
\par 7 Og de havde intet Barn, efterdi Elisabeth var ufrugtbar, og de vare begge fremrykkede i Alder.
\par 8 Men det skete, medens han efter sit Skiftes Orden gjorde Præstetjeneste for Gud,
\par 9 tilfaldt det ham efter Præstetjenestens Sædvane at gå ind i Herrens Tempel og bringe Røgelseofferet.
\par 10 Og hele Folkets Mængde holdt Bøn udenfor i Røgelseofferets Time.
\par 11 Men en Herrens Engel viste sig for ham, stående ved den højre Side af Røgelsesalteret.
\par 12 Og da Sakarias så ham, forfærdedes han, og Frygt faldt over ham.
\par 13 Men Engelen sagde til ham: "Frygt ikke, Sakarias! thi din Bøn er hørt, og din Hustru Elisabeth skal føde dig en Søn, og du skal kalde hans Navn Johannes.
\par 14 Og han skal blive dig til Glæde og Fryd, og mange skulle glædes over hans Fødsel;
\par 15 thi han skal være stor for Herren. Og Vin og stærk Drik skal han ej drikke, og han skal fyldes med den Helligånd alt fra Moders Liv,
\par 16 og mange af Israels Børn skal han omvende til Herren deres Gud.
\par 17 Og han skal gå foran for ham i Elias's Ånd og Kraft for at vende Fædres Hjerter til Børn og genstridige til retfærdiges Sind for at berede Herren et velskikket folk."
\par 18 Og Sakarias sagde til Engelen: "Hvorpå skal jeg kende dette? thi jeg er gammel, og min Hustru er fremrykket i Alder."
\par 19 Og Engelen svarede og sagde til ham: "Jeg er Gabriel, som står for Guds Åsyn, og jeg er udsendt for at tale til dig og for at forkynde dig dette Glædesbudskab.
\par 20 Og se, du skal blive stum og ikke kunne tale indtil den Dag, da dette sker, fordi du ikke troede mine Ord, som dog skulle fuldbyrdes i deres Tid,"
\par 21 Og folket biede efter Sakarias, og de undrede sig over, at han tøvede i Templet.
\par 22 Og da han kom ud, kunde han ikke tale til dem, og de forstode, at han havde set et Syn i Templet; og han gjorde Tegn til dem og forblev stum.
\par 23 Og det skete, da hans Tjenestes Dage vare fuldendte, gik han hjem til sit Hus.
\par 24 Men efter disse Dage blev hans Hustru Elisabeth frugtsommelig, og hun skjulte sig fem Måneder og sagde:
\par 25 "Således har Herren gjort imod mig i de Dage, da han så til mig for at borttage min Skam iblandt Mennesker:"
\par 26 Men i den sjette Måned blev Engelen Gabriel sendt fra Gud til en By i Galilæa, som hedder Nazareth,
\par 27 til en Jomfru, som var trolovet med en Mand ved Navn Josef, af Davids Hus; og Jomfruens Navn var Maria.
\par 28 Og Engelen kom ind til hende og sagde: "Hil være dig, du benådede, Herren er med dig, du velsignede iblandt Kvinder!"
\par 29 Men hun blev forfærdet over den Tale, og hun tænkte, hvad dette skulde være for en Hilsen.
\par 30 Og Engelen sagde til hende: "Frygt ikke, Maria! thi du har fundet Nåde hos Gud.
\par 31 Og se, du skal undfange og føde en Søn,og du skal kalde hans Navn Jesus.
\par 32 Han skal være stor og kaldes den Højestes Søn; og Gud Herren skal give ham Davids, hans Faders Trone.
\par 33 Og han skal være Konge over Jakobs Hus evindelig, og der skal ikke være Ende på hans Kongedømme."
\par 34 Men Maria sagde til Engelen: "Hvorledes skal dette gå til, efterdi jeg ikke ved af nogen Mand?"
\par 35 Og Engelen svarede og sagde til hende: Den Helligånd skal komme over dig, og den Højestes Kraft skal overskygge dig; derfor skal også det hellige, som fødes,. kaldes Guds Søn.
\par 36 Og se, Elisabeth din Frænke, også hun har undfanget en Søn i sin Alderdom, og denne Måned er den sjette for hende, som kaldes ufrugtbar.
\par 37 Thi intet vil være umuligt for Gud."
\par 38 Men Maria sagde: "Se, jeg er Herrens Tjenerinde; mig ske efter dit Ord!" Og Engelen skiltes fra hende.
\par 39 Men Maria stod op i de samme Dage og drog skyndsomt til Bjergegnen til en By i Juda.
\par 40 Og hun kom ind i Sakarias's Hus og hilste Elisabeth.
\par 41 Og det skete, da Elisabeth hørte Marias Hilsen, sprang Fosteret i hendes Liv. Og Elisabeth blev fyldt med den Helligånd
\par 42 og råbte med høj Røst og sagde: "Velsignet er du iblandt Kvinder! og velsignet er dit Livs Frugt!
\par 43 Og hvorledes times dette mig, at min Herres Moder kommer til mig?
\par 44 Thi se, da din Hilsens Røst nåede mine Øren, sprang Fosteret i mit Liv med Fryd.
\par 45 Og salig er hun, som troede; thi det skal fuldkommes, hvad der er sagt hende af Herren,"
\par 46 Og Maria sagde: "Min Sjæl ophøjer Herren;
\par 47 og min Ånd fryder sig over Gud, min Frelser;
\par 48 thi han har set til sin Tjenerindes Ringhed. Thi se, nu herefter skulle alle Slægter prise mig salig,
\par 49 fordi den mægtige har gjort store Ting imod mig. Og hans Navn er helligt;
\par 50 og hans Barmhjertighed varer fra Slægt til Slægt over dem, som frygte ham.
\par 51 Han har øvet Vælde med sin Arm; han har adspredt dem, som ere hovmodige i deres Hjertes Tanke.
\par 52 Han har nedstødt mægtige fra Troner og ophøjet ringe.
\par 53 Hungrige har han mættet med gode Gaver, og rige har han sendt tomhændede bort.
\par 54 Han har taget sig af sin Tjener Israel for at ihukomme Barmhjertighed
\par 55 imod Abraham og hans Sæd til evig Tid, således som han talte til vore Fædre."
\par 56 Og Maria blev hos hende omtrent tre Måneder, og hun drog til sit Hjem igen.
\par 57 Men for Elisabeth fuldkommedes Tiden til, at hun skulde føde, og hun fødte en Søn.
\par 58 Og hendes Naboer og Slægtninge hørte, at Herren havde gjort sin Barmhjertighed stor imod hende, og de glædede sig med hende.
\par 59 Og det skete på den ottende Dag, da kom de for at omskære Barnet; og de vilde kalde det Sakarias efter Faderens Navn.
\par 60 Og hans Moder svarede og sagde: "Nej, han skal kaldes Johannes."
\par 61 Og de sagde til hende: "Der er ingen i din Slægt, som kaldes med dette Navn."
\par 62 Men de gjorde Tegn til hans Fader om, hvad han vilde, det skulde kaldes.
\par 63 Og han forlangte en Tavle og skrev disse Ord: "Johannes er hans Navn." Og de undrede sig alle.
\par 64 Men straks oplodes hans Mund og hans Tunge, og han talte og priste Gud.
\par 65 Og der kom en Frygt over alle, som boede omkring dem, og alt dette rygtedes over hele Judæas Bjergegn.
\par 66 Og alle, som hørte det, lagde sig det på Hjerte og sagde: "Hvad mon der skal blive af dette Barn?" Thi Herrens Hånd var med ham.
\par 67 Og Sakarias, hans Fader, blev fyldt med den Helligånd, og han profeterede og sagde:
\par 68 "Lovet være Herren, Israels Gud! thi han har besøgt og forløst sit Folk
\par 69 og har oprejst os et Frelsens Horn" i sin Tjener Davids Hus,
\par 70 således som han talte ved sine hellige Profeters Mund fra fordums Tid,
\par 71 en Frelse fra vore Fjender og fra alle deres Hånd, som hade os,
\par 72 for at gøre Barmhjertighed imod vore Fædre og ihukomme sin hellige Pagt,
\par 73 den Ed, som han svor vor Fader Abraham, at han vilde give os,
\par 74 at vi, friede fra vore Fjenders Hånd, skulde tjene ham uden Frygt,
\par 75 i Hellighed og Retfærdighed for hans Åsyn, alle vore Dage.
\par 76 Men også du, Barnlille! skal kaldes den Højestes Profet; thi du skal gå foran for Herrens Åsyn for at berede hans Veje,
\par 77 for at give hans Folk Erkendelse af Frelse ved deres Synders Forladelse,
\par 78 for vor Guds inderlige Barmhjertigheds Skyld, ved hvilken Lyset fra det høje har besøgt os
\par 79 for at skinne for dem, som sidde i Mørke og i Dødens Skygge, for at lede vore Fødder ind på Fredens Vej,"
\par 80 Men Barnet voksede og blev styrket i Ånden; og han var i Ørkenerne indtil den Dag, da han trådte frem for Israel.

\chapter{2}

\par 1 Men det skete i de dage, at en Befaling udgik fra Kejser Augustus, at al Verden skulde skrives i Mandtal.
\par 2 (Denne første Indskrivning skete, da Kvirinius var Landshøvding i Syrien,)
\par 3 Og alle gik for at lade sig indskrive, hver til sin By.
\par 4 Og også Josef gik op fra Galilæa, fra Byen Nazareth til Judæa til Davids By, som kaldes Bethlehem, fordi han var af Davids Hus og Slægt,
\par 5 for at lade sig indskrive tillige med Maria, sin trolovede, som var frugtsommelig.
\par 6 Men det skete, medens de vare der, blev Tiden fuldkommet til, at hun skulde føde.
\par 7 Og hun fødte sin Søn, den førstefødte, og svøbte ham og lagde ham i en Krybbe; thi der var ikke Rum for dem i Herberget.
\par 8 Og der var Hyrder i den samme Egn, som lå ude på Marken og holdt Nattevagt over deres Hjord.
\par 9 Og se, en Herrens Engel stod for dem, og Herrens Herlighed skinnede om dem, og de frygtede såre
\par 10 Og Engelen sagde til dem: "Frygter ikke; thi se, jeg forkynder eder en stor Glæde, som skal være for hele Folket.
\par 11 Thi eder er i dag en Frelser født, som er den Herre Kristus i Davids By.
\par 12 Og dette skulle I have til Tegn: I skulle finde et Barn svøbt, liggende i en Krybbe."
\par 13 Og straks var der med Engelen en himmelsk Hærskares Mangfoldighed, som lovede Gud og sagde:
\par 14 "Ære være Gud i det højeste! og Fred på Jorden! i Mennesker Velbehag!
\par 15 Og det skete, da Englene vare farne fra dem til Himmelen, sagde Hyrderne til hverandre: "Lader os dog gå til Bethlehem og se dette, som er sket, hvilket Herren har kundgjort os."
\par 16 Og de skyndte sig og kom og fandt både Maria og Josef, og Barnet liggende i Krybben.
\par 17 Men da de så det, kundgjorde de, hvad der var talt til dem om dette Barn.
\par 18 Og alle de, som hørte det, undrede sig over det, der blev talt til dem af Hyrderne.
\par 19 Men Maria gemte alle disse Ord og overvejede dem i sit Hjerte.
\par 20 Og Hyrderne vendte tilbage, idet de priste og lovede Gud for alt, hvad de havde hørt og set, således som der var talt til dem.
\par 21 Og da otte Dage vare fuldkommede, så han skulde omskæres, da blev hans Navn kaldt Jesus, som det var kaldt af Engelen, før han blev undfangen i Moders Liv.
\par 22 Og da deres Renselsesdage efter Mose Lov vare fuldkommede, bragte de ham op til Jerusalem for at fremstille ham for Herren,
\par 23 som der er skrevet i Herrens Lov, at alt Mandkøn, som åbner Moders Liv, skal kaldes helligt for Herren,
\par 24 og for at bringe Offer efter det, som er sagt i Herrens Lov, et Par Turtelduer eller to unge Duer.
\par 25 Og se, der var en Mand i Jerusalem ved Navn Simeon, og denne Mand var retfærdig og gudfrygtig og forventede Israels Trøst, og den Helligånd var over ham.
\par 26 Og det var varslet ham af den Helligånd, at han ikke skulde se Døden, førend han havde set Herrens Salvede.
\par 27 Og han kom af Åndens Drift til Helligdommen; og idet Forældrene bragte Barnet Jesus ind for at gøre med ham efter Lovens Skik,
\par 28 da tog han det på sine Arme og priste Gud og sagde:
\par 29 "Herre! nu lader du din Tjener fare i Fred, efter dit Ord.
\par 30 Thi mine Øjne have set din Frelse,
\par 31 som du beredte for alle Folkeslagenes Åsyn,
\par 32 et Lys til at oplyse Hedningerne og en Herlighed for dit Folk Israel."
\par 33 Og hans Fader og hans Moder undrede sig over de Ting, som bleve sagte om ham.
\par 34 Og Simeon velsignede dem og sagde til hans Moder Maria: "Se, denne er sat mange i Israel til Fald og Oprejsning og til et Tegn, som imodsiges,
\par 35 ja, også din egen Sjæl skal et Sværd gennemtrænge! for at mange Hjerters Tanker skulle åbenbares."
\par 36 Og der var en Profetinde Anna, Fanuels Datter, af Asers Stamme; hun var meget fremrykket i Alder, havde levet syv År med sin Mand efter sin Jomfrustand
\par 37 og var nu en Enke ved fire og firsindstyve År, og hun veg ikke fra Helligdommen, tjenende Gud med Faste og Bønner Nat og Dag.
\par 38 Og hun trådte til i den samme Stund og priste Gud og talte om ham til alle, som forventede Jerusalems Forløsning.
\par 39 Og da de havde fuldbyrdet alle Ting efter Herrens Lov, vendte de tilbage til Galilæa til deres egen By Nazareth.
\par 40 Men Barnet voksede og blev stærkt og blev fuldt af Visdom: og Guds Nåde var over det.
\par 41 Og hans Forældre droge hvert År op til Jerusalem på Påskehøjtiden.
\par 42 Og da han var bleven tolv År gammel, og de gik op efter Højtidens Sædvane
\par 43 og havde tilendebragt de Dage, blev Barnet Jesus i Jerusalem, medens de droge hjem, og hans Forældre mærkede det ikke.
\par 44 Men da de mente, at han var i Rejsefølget, kom de en Dags Rejse frem, og de ledte efter ham iblandt deres Slægtninge og Kyndinge.
\par 45 Og da de ikke fandt ham, vendte de tilbage til Jerusalem og ledte efter ham.
\par 46 Og det skete efter tre Dage, da fandt de ham i Helligdommen, hvor han sad midt iblandt Lærerne og både hørte på dem og adspurgte dem.
\par 47 Men alle, som hørte ham, undrede sig såre over hans Forstand og Svar.
\par 48 Og da de så ham, bleve de forfærdede; og hans Moder sagde til ham:"Barn! hvorfor gjorde du således imod os? Se, din Fader og jeg have ledt efter dig med Smerte."
\par 49 Og han sagde til dem: "Hvorfor ledte I efter mig? Vidste I ikke, at jeg bør være i min Faders Gerning?"
\par 50 Og de forstode ikke det Ord, som han talte til dem.
\par 51 Og han drog ned med dem og kom til Nazareth og var dem lydig og hans Moder gemte alle de Ord i sit Hjerte.
\par 52 Og Jesus forfremmedes i Visdom og Alder og yndest hos Gud og Mennesker.

\chapter{3}

\par 1 Men i Kejser Tiberius's femtende Regeringsår, da Pontius Pilatus var Landshøvding i Judæa, og Herodes var Fjerdingsfyrste i Galilæa, og hans Broder Filip var Fjerdingsfyrste i Ituræa og Trakonitis's Land og Lysanias Fjerdingsfyrste i Abilene,
\par 2 medens Annas og Kajfas vare Ypperstepræster, kom Guds Ord til Johannes, Sakarias's Søn, i Ørkenen.
\par 3 Og han gik ud i hele Omegnen om Jordan og prædikede Omvendelses-Dåb til Syndernes Forladelse,
\par 4 som der er skrevet i Profeten Esajas's Talers Bog: "Der er en Røst af en, som råber i Ørkenen: Bereder Herrens Vej, gører hans Stier jævne;
\par 5 hver Dal skal opfyldes, og hvert Bjerg og Høj skal nedtrykkes, og det krumme skal gøres lige, og de ujævne Veje skulle gøres jævne;
\par 6 og alt Kød skal se Guds Frelse."
\par 7 Han sagde altså til de Skarer, som gik ud for at døbes af ham: "I Øgleunger! hvem har lært eder at fly fra den kommende Vrede?
\par 8 Bærer da Frugter, som ere Omvendelsen værdige, og begynder ikke at sige ved eder selv: Vi have Abraham til Fader; thi jeg siger eder, at Gud kan opvække Abraham Børn af disse Sten.
\par 9 Men Øksen ligger også allerede ved Roden af Træerne; så bliver da hvert Træ, som ikke bærer god Frugt, omhugget og kastet i Ilden."
\par 10 Og Skarerne spurgte ham og sagde: "Hvad skulle vi da gøre?"
\par 11 Men han svarede og sagde til dem: "Den, som har to Kjortler, dele med den, som ingen har; og den, som har Mad, gøre ligeså!"
\par 12 Men også Toldere kom for at døbes, og de sagde til ham: "Mester! hvad skulle vi gøre?"
\par 13 Men han sagde til dem: "Kræver intet ud over, hvad eder er forordnet."
\par 14 Men også Krigsfolk spurgte ham og sagde: "Hvad skulle vi da gøre?" Og han sagde til dem: "Øver ikke Vold imod nogen, bruger ikke Underfundighed imod nogen, og lader eder nøje med eders Sold!"
\par 15 Men da Folket var i Forventning, og alle tænkte i deres Hjerter om Johannes, om ikke han skulde være Kristus,
\par 16 da svarede Johannes og sagde til alle: "Jeg døber eder med Vand; men den kommer, som er stærkere end jeg, og hvis Skotvinge jeg ikke er værdig at løse; han skal døbe eder med den Helligånd og Ild.
\par 17 Hans Kasteskovl er i hans Hånd, for at han skal gennemrense sin Lo og sanke Hveden i sin Lade, men Avnerne skal han opbrænde med uslukkelig Ild."
\par 18 Ligeså formanede han også Folket om mange andre Ting og forkyndte dem Evangeliet.
\par 19 Men da Fjerdingsfyrsten Herodes blev revset af ham for hans Broders Hustru, Herodias's Skyld og for alt det onde, som Herodes gjorde,
\par 20 så føjede han til alt det øvrige også dette, at han kastede Johannes i Fængsel.
\par 21 Men medens hele Folket blev døbt, skete det, da også Jesus var bleven døbt og bad, at Himmelen åbnedes,
\par 22 og at den Helligånd dalede ned over ham i legemlig Skikkelse som en Due, og at en Røst lød fra Himmelen: "Du er min Søn, den elskede, i dig har jeg Velbehag."
\par 23 Og Jesus selv var omtrent tredive År, da han begyndte, og han var, som man holdt for, en Søn af Josef Elis Søn,
\par 24 Matthats Søn, Levis Søn, Melkis Søn, Jannajs Søn, Josefs Søn,
\par 25 Mattathias's Søn, Amos's Søn, Naums Søn, Eslis Søn, Naggajs Søn,
\par 26 Måths Søn, Mattathias's Søn, Semeis Søn, Josefs Søn, Judas Søn,
\par 27 Joanans Søn, Resas Søn, Zorobabels Søn; Salathiels Søn, Neris Søn.
\par 28 Melkis Søn, Addis Søn, Kosams Søn, Elmadams Søn, Ers Søn,
\par 29 Jesu Søn, Eliezers Søn, Jorims Søn, Matthats Søn, Levis Søn,
\par 30 Simeons Søn, Judas Søn, Josets Søn, Jonams Søn, Eliakims Søn,
\par 31 Meleas Søn, Mennas Søn, Mattathas Søn, Nathans Søn, Davids Søn,
\par 32 Isajs Søn, Obeds Søn, Boos's Søn, Salmons Søn, Nassons Søn,
\par 33 Aminadabs Søn, Arams Søn, Esroms Søn, Fares's Søn, Judas Søn,
\par 34 Jakobs Søn, Isaks Søn, Abrahams Søn, Tharas Søn, Nakors Sn,
\par 35 Seruks Søn, Ragaus Søn, Faleks Søn, Ebers Søn, Salas Søn,
\par 36 Kajnans Søn, Arfaksads Søn, Sems Søn, Noas Søn, Lameks Søn,
\par 37 Methusalas Søn, Enoks Søn, Jareds Søn, Maleleels Søn, Kajnans Søn,
\par 38 Enos's Søn, Seths Søn, Adams Søn, Guds Søn.

\chapter{4}

\par 1 Men Jesus vendte tilbage fra Jorden fuld af den Helligånd og blev ført af Ånden i Ørkenen
\par 2 i fyrretyve Dage, medens han blev fristet af Djævelen. Og han spiste intet i de Dage; og da de havde Ende, blev han hungrig.
\par 3 Og Djævelen sagde til ham: "Dersom du er Guds Søn, da sig til denne Sten, at den skal blive Brød."
\par 4 Og Jesus svarede ham: "Der er skrevet: Mennesket skal ikke leve af Brød alene."
\par 5 Og han førte ham op og viste ham alle Verdens Riger i et øjeblik.
\par 6 Og Djævelen sagde til ham: "Dig vil jeg give hele denne Magt og deres Herlighed; thi den er mig overgiven, og jeg giver den, til hvem jeg vil.
\par 7 Dersom du altså vil tilbede mig, skal den helt tilhøre dig."
\par 8 Og Jesus svarede ham og sagde: "Der er skrevet: Du skal tilbede Herren din Gud og tjene ham alene."
\par 9 Og han førte ham til Jerusalem og stillede ham på Helligdommens Tinde og sagde til ham: "Dersom du er Guds Søn, da kast dig ned herfra;
\par 10 thi der er skrevet: Han skal give sine Engle Befaling om dig, at de skulle bevare dig,
\par 11 og at de skulle bære dig på Hænderne, for at du ikke skal støde din Fod på nogen Sten."
\par 12 Og Jesus svarede og sagde til ham: "Der er sagt: Du må ikke friste Herren din Gud."
\par 13 Og da Djævelen havde endt al Fristelse, veg han fra ham til en Tid.
\par 14 Og Jesus vendte i Åndens Kraft tilbage til Galilæa, og Rygtet om ham kom ud i hele det omliggende Land.
\par 15 Og selv lærte han i deres Synagoger og blev prist af alle.
\par 16 Og han kom til Nazareth, hvor han var opfødt, og gik efter sin Sædvane på Sabbatsdagen ind i Synagogen og stod op for at forelæse.
\par 17 Og man gav ham Profeten Esajas's Bog, og da han slog Bogen op; fandt han det Sted, hvor der stod skrevet:
\par 18 "Herrens Ånd er over mig, fordi han salvede mig til at forkynde Evangelium for fattige; han har sendt mig for at forkynde fangne, at de skulle lades løs, og blinde, at de skulle få deres Syn, for at sætte plagede i Frihed,
\par 19 for at forkynde et Herrens Nådeår."
\par 20 Og han lukkede Bogen sammen og gav Tjeneren den igen og satte sig; og alles Øjne i Synagogen stirrede på ham.
\par 21 Men han begyndte at sige til dem: "I Dag er dette Skriftord gået i Opfyldelse for eders Øren."
\par 22 Og de berømmede ham alle og undrede sig over de livsalige Ord, som udgik af hans Mund, og de sagde: "Er dette ikke Josefs Søn?"
\par 23 Og han sagde til dem: "I ville sikkerlig sige mig dette Ordsprog: Læge! læg dig selv; gør også her i din Fædreneby så. store Ting, som vi have hørt ere skete i Kapernaum."
\par 24 Men han sagde: "Sandelig, siger jeg eder, at ingen Profet er anerkendt i sit Fædreland.
\par 25 Men jeg siger eder i Sandhed: Der var mange Enker i Israel i Elias's Dage, da Himmelen var lukket i tre År og seks Måneder, den Gang der var en stor Hunger i hele Landet;
\par 26 og til ingen af dem blev Elias sendt uden til Sarepta ved Sidon til en Enke.
\par 27 Og der var mange spedalske i Israel på Profeten Elisas Tid, og ingen af dem blev renset, uden Syreren Naman."
\par 28 Og alle, som vare i Synagogen, bleve fulde af Harme, da de hørte dette.
\par 29 Og de stode op og stødte ham ud af Byen og førte ham hen til Skrænten af det Bjerg, på hvilket deres By var bygget, for at styrte ham ned.
\par 30 Men han gik igennem, midt imellem dem, og drog bort.
\par 31 Og han kom ned til Kapernaum, en By i Galilæa, og lærte dem på Sabbaterne.
\par 32 Og de bleve slagne af Forundring over hans Lære, thi hans Tale var med Myndighed.
\par 33 Og i Synagogen var der et Menneske, som havde en uren ond Ånd, og han råbte med høj Røst:
\par 34 "Ak! hvad have vi med dig at gøre, Jesus af Nazareth? Er du kommen for at ødelægge os? Jeg kender dig, hvem du er, du Guds hellige."
\par 35 Og Jesus truede ham og sagde: "Ti, og far ud af ham!" Og den onde Ånd kastede ham ind imellem dem og for ud af ham uden at have gjort ham nogen Skade.
\par 36 Og der kom en Rædsel over alle; og de talte med hverandre og sagde "Hvad er dog dette for et Ord; thi han byder over de urene Ånder med Myndighed og Kraft, og de fare ud?"
\par 37 Og Rygtet om ham udbredtes alle Vegne i det omliggende Land.
\par 38 Men han stod op og gik fra Synagogen ind i Simons Hus; og Simons Svigermoder plagedes at en stærk Feber; og de bade ham for hende.
\par 39 Og han stillede sig hen over hende og truede Feberen, og den forlod hende. Men hun stod straks op og vartede dem op.
\par 40 Men da Solen gik ned, førte alle de, som havde syge med alle Hånde Svagheder, dem til ham; og han lagde Hænderne på hver enkelt af dem og helbredte dem
\par 41 Også onde Ånder fore ud al mange, råbte og sagde: "Du er Guds Søn;" og han truede dem og tillod dem ikke at tale, fordi de vidste, at han var Kristus.
\par 42 Men da det var blevet Dag, gik han ud og drog til et øde Sted; og Skarerne ledte efter ham; og de kom hen til ham, og de holdt på ham, for at han ikke skulde gå fra dem.
\par 43 Men han sagde til dem: "Også for de andre Byer bør jeg forkynde Evangeliet om Guds Rige; thi dertil blev jeg udsendt."
\par 44 Og han prædikede i Galilæas Synagoger.

\chapter{5}

\par 1 Men det skete, da Folkeskaren trængte sig sammen om ham og hørte Guds Ord, og han stod ved Genezareths Sø,
\par 2 da så han to Skibe stå ved Søen; men Fiskerne vare gåede fra dem og toede Garnene.
\par 3 Og han gik om Bord i et af Skibene, som var Simons, og bad ham at lægge lidt fra Land; og han satte sig og lærte Skarerne fra Skibet.
\par 4 Men da han holdt op med at tale, sagde han til Simon: "Far ud på Dybet, og kaster eders Garn ud til en Dræt!"
\par 5 Og Simon svarede og sagde til ham: "Mester! vi have arbejdet hele Natten og fik intet; men på dit Ord vil jeg kaste Garnene ud."
\par 6 Og da de gjorde det, fangede de en stor Mængde Fisk, og deres Garn sønderreves.
\par 7 Og de vinkede ad deres Staldbrødre i det andet Skib, at de skulde komme og hjælpe dem; og de kom og de fyldte begge Skibene, så at de var nær ved at synke.
\par 8 Men da Simon Peter så det, faldt han ned for Jesu Knæ og sagde: "Gå bort fra mig, thi jeg er en syndig Mand, Herre!"
\par 9 Thi en Rædsel var påkommen ham og alle dem, som vare med ham, over den Fiskedræt, som de havde fået;
\par 10 ligeledes også Jakob og Johannes, Zebedæus's Sønner, som vare Simons Staldbrødre. Og Jesus sagde til Simon: "Frygt ikke, fra nu af skal du fange Mennesker."
\par 11 Og de lagde Skibene til Land og forlode alle Ting og fulgte ham.
\par 12 Og det skete, medens han var i en af Byerne, se, da var der en Mand fuld af Spedalskhed; og da han så Jesus, faldt han på sit Ansigt, bad ham og sagde: "Herre! om du vil, kan du rense mig."
\par 13 Og han udrakte Hånden og rørte ved ham og sagde: "Jeg vil; bliv ren!" Og straks forlod Spedalskheden ham.
\par 14 Og han bød ham, at han skulde ikke sige det til nogen, men "gå bort, og fremstil dig for Præsten, og offer for din Renselse, således som Moses har befalet, til Vidnesbyrd for dem!"
\par 15 Men Rygtet om ham udbredte sig end mere, og store Skarer kom sammen for at høre og for at helbredes for deres Sygdomme.
\par 16 Men han gik bort til Ørkenerne og bad.
\par 17 Og det skete en af de Dage, at han lærte, og der sad Farisæere og Lovlærere, som vare komne fra enhver Landsby i Galilæa og Judæa og fra Jerusalem; og Herrens Kraft var hos ham til at helbrede.
\par 18 Og se, nogle Mænd bare på en Seng en Mand, som var værkbruden, og de søgte at bære ham ind og lægge ham foran ham.
\par 19 Og da de ikke fandt nogen Vej til at bære ham ind for Skarens Skyld, stege de op oven på Taget og firede ham tillige med Sengen ned imellem Tagstenene midt iblandt dem foran Jesus.
\par 20 Og da han så deres Tro, sagde han: "Menneske! dine Synder ere dig forladte."
\par 21 Og de skriftkloge og Farisæerne begyndte at tænke således ved sig selv: "Hvem er denne, som taler Gudsbespottelser? Hvem kan forlade Synder, uden Gud alene?"
\par 22 Men da Jesus kendte deres Tanker, svarede han og sagde til dem: "Hvad tænke I på i eders Hjerter?
\par 23 Hvilket er lettest at sige: Dine Synder ere dig forladte? eller at sige: Stå op og gå?
\par 24 Men for at I skulle vide, at Menneskesønnen har Magt på Jorden til at forlade Synder," så sagde han til den værkbrudne: "Jeg siger dig, stå op, og tag din Seng, og gå til dit Hus!"
\par 25 Og han stod straks op for deres Øjne og tog det, som han lå på, og gik hen til sit Hus og priste Gud.
\par 26 Og Forfærdelse betog alle, og de priste Gud; og de bleve fulde af Frygt og sagde: "Vi have i Dag set utrolige Ting."
\par 27 Og derefter gik han ud og så en Tolder ved Navn Levi sidde ved Toldboden, og han sagde til ham: "Følg mig!"
\par 28 Og han forlod alle Ting og stod op og fulgte ham.
\par 29 Og Levi gjorde et stort Gæstebud for ham i sit Hus; og der var en stor Skare af Toldere og andre, som sade til Bords med dem.
\par 30 Og Farisæerne og deres Skriftkloge knurrede imod hans Disciple og sagde: "Hvorfor spise og drikke I med Toldere og Syndere?"
\par 31 Og Jesus svarede og sagde til dem: "De raske trænge ikke til Læge, men de syge.
\par 32 Jeg er ikke kommen for at kalde retfærdige, men Syndere til Omvendelse."
\par 33 Men de sagde til ham: "Johannes's Disciple faste ofte og holde Bønner og Farisæernes ligeså; men dine spise og drikke?"
\par 34 Men Jesus sagde til dem: "Kunne I vel få Brudesvendene til at faste, så længe Brudgommen er hos dem?
\par 35 Men der skal komme Dage, da Brudgommen bliver tagen fra dem; da skulle de faste i de Dage."
\par 36 Men han sagde også en Lignelse til dem: "Ingen river en Lap af et nyt Klædebon og sætter den på et gammelt Klædebon; ellers river han både det nye sønder, og Lappen fra det nye vil ikke passe til det gamle.
\par 37 Og ingen kommer ung Vin på gammle Læderflasker; ellers sprænger den unge Vin Læderflaskerne, og den spildes, og Læderflaskerne ødelægges.
\par 38 Men man skal komme ung Vin på nye Læderflasker, så blive de begge bevarede.
\par 39 Og ingen, som har drukket den gamle, vil have den unge; thi han siger: Den gamle er god."

\chapter{6}

\par 1 Men det skete på den næstførste Sabbat, at han vandrede igennem en Sædemark, og hans Disciple plukkede Aks og gned dem med Hænderne og spiste.
\par 2 Men nogle af Farisæerne sagde: "Hvorfor gøre I, hvad det ikke er tilladt at gøre på Sabbaten?"
\par 3 Og Jesus svarede og sagde til dem: "Have I da ikke læst, hvad David gjorde, da han blev hungrig, han og de, som vare med ham?
\par 4 hvorledes han gik ind i Guds Hus og tog Skuebrødene og spiste og gav også dem, som vare med ham, skønt det ikke er nogen tilladt at spise dem uden Præsterne alene."
\par 5 Og han sagde til dem: "Menneskesønnen er Herre også over Sabbaten."
\par 6 Men det skete på en anden Sabbat, at han kom ind i Synagogen og lærte. Og der var der en Mand, hvis højre Hånd var vissen.
\par 7 Men de skriftkloge og Farisæerne toge Vare på ham, om han vilde helbrede på Sabbaten, for at de kunde finde noget at anklage ham for.
\par 8 Men han vidste deres Tanker; og han sagde til Manden, som havde den visne Hånd: "Rejs dig og stå frem her i Midten!" Og han rejste sig og stod frem.
\par 9 Men Jesus sagde til dem: "Jeg spørger eder, om det er tilladt at gøre godt på Sabbaten eller at gøre ondt, at frelse Liv eller at ødelægge det?"
\par 10 Og han så omkring på dem alle og sagde til ham: "Ræk din Hånd ud!" Og han gjorde det; da blev hans Hånd sund igen som den anden.
\par 11 Men de bleve fulde af Raseri og talte med hverandre om, hvad de skulde gøre ved Jesus.
\par 12 Men det skete i disse Dage, at han gik ud på et Bjerg for at bede; og han tilbragte Natten i Bøn til Gud.
\par 13 Og da det blev Dag, hidkaldte han sine Disciple og udvalgte tolv af dem, hvilke han også kaldte Apostle:
\par 14 Simon, hvem han også kaldte Peter, og Andreas, hans Broder, og Jakob og Johannes og Filip og Bartholomæus
\par 15 og Matthæus og Thomas, Jakob, Alfæus's Søn, og Simon, som kaldes Zelotes,
\par 16 Judas, Jakobs Søn, og Judas Iskariot, som blev Forræder.
\par 17 Og han gik ned med dem og stod på et jævnt Sted, og der var en Skare af hans Disciple og en stor Mængde af Folket fra hele Judæa og Jerusalem og Kysten ved Tyrus og Sidon,
\par 18 som vare komne for at høre ham og helbredes for deres Sygdomme.
\par 19 og hele Skaren søgte at røre ved ham; thi en Kraft gik ud fra ham og helbredte alle.
\par 20 Og han opløftede sine Øjne på sine Disciple og sagde: "Salige ere I fattige, thi eders er Guds Rige.
\par 21 Salige ere I, som nu hungre, thi I skulle mættes. Salige ere I, som nu græde, thi I skulle le.
\par 22 Salige er I, når Menneskene hade eder, og når de udstøde eder og håne eder og forkaste eders Navn som ondt for Menneskesønnens Skyld.
\par 23 Glæder eder på den Dag og jubler; thi se, eders Løn er stor i Himmelen. Thi på samme Måde gjorde deres Fædre ved Profeterne.
\par 24 Men ve eder, I rige, thi I have allerede fået eders Trøst.
\par 25 Ve eder, I, som nu ere mætte, thi I skulle hungre. Ve eder, I, som nu le, thi I skulle sørge og græde.
\par 26 Ve eder, når alle Mennesker tale godt om eder; thi på samme Måde gjorde deres Fædre ved de falske Profeter.
\par 27 Men jeg siger eder, I, som høre: Elsker eders Fjender, gører dem godt, som hade eder;
\par 28 velsigner dem, som forbande eder, og beder for dem, som krænke eder.
\par 29 Den, som slår dig på den ene Kind, byd ham også den anden til; og den, som tager Kappen fra dig, formen ham heller ikke Kjortelen!
\par 30 Giv enhver, som beder dig; og af den, som tager, hvad dit er, kræve du det ikke igen!
\par 31 Og som I ville, at Menneskene skulle gøre imod eder, ligeså skulle også I gøre imod dem!
\par 32 Og dersom I elske dem, som elske eder, hvad Tak have I derfor? Thi også Syndere elske dem, som dem elske.
\par 33 Og dersom I gøre vel imod dem, der gøre vel imod eder, hvad Tak have I derfor? Thi også Syndere gøre det samme.
\par 34 Og dersom I låne dem, af hvem I håbe at få igen, hvad Tak have I derfor? Thi også Syndere låne Syndere for at få lige igen.
\par 35 Men elsker eders Fjender, og gører vel, og låner uden at vente noget derfor, så skal eders Løn være stor, og I skulle være den Højestes Børn; thi han er god imod de utaknemmelige og onde.
\par 36 Vorder barmhjertige, ligesom eders Fader er barmhjertig.
\par 37 Og dømmer ikke, så skulle I ikke dømmes; fordømmer ikke, så skulle I ikke fordømmes; forlader, så skal der forlades eder;
\par 38 giver, så skal der gives eder. Et godt, knuget, rystet, topfuldt Mål skulle de give i eders Skød; thi med hvad Mål I måle, skal der tilmåles eder igen."
\par 39 Men han sagde dem også en Lignelse: "Mon en blind kan lede en blind? Ville de ikke begge falde i Graven?
\par 40 En Discipel er ikke over sin Mester; men enhver, som er fuldt færdig, skal være som sin Mester.
\par 41 Men hvorfor ser du Skæven, som er i din Broders Øje; men Bjælken, som er i dit eget Øje, bliver du ikke var?
\par 42 Eller hvorledes kan du sige til din Broder: Broder! lad mig drage Skæven ud, som er i dit Øje, du, som ikke ser Bjælken i dit eget Øje? Du Hykler! drag først Bjælken ud af dit Øje, og da kan du se klart til at drage Skæven ud, som er i din Broders Øje.
\par 43 Thi der er intet godt Træ, som bærer rådden Frugt, og intet råddent Træ, som bærer god Frugt
\par 44 Thi hvert Træ kendes på sin egen Frugt; thi man sanker ikke Figener af Torne, ikke heller plukker man Vindruer af en Tornebusk.
\par 45 Et godt Menneske fremfører det gode af sit Hjertes gode Forråd, og et ondt Menneske fremfører det onde af sit onde Forråd; thi af Hjertets Overflødighed taler hans Mund.
\par 46 Men hvorfor kalde I mig Herre, Herre! og gøre ikke, hvad jeg siger?
\par 47 Hver den, som kommer til mig og hører mine Ord og gør efter dem, hvem han er lig, skal jeg vise eder.
\par 48 Han er lig et Menneske, der byggede et Hus og gravede i Dybden og lagde Grundvolden på Klippen; men da en Oversvømmelse kom, styrtede Floden imod det Hus, og den kunde ikke ryste det; thi det var bygget godt.
\par 49 Men den, som hører og ikke gør derefter, han er lig et Menneske, der byggede et Hus på Jorden, uden Grundvold; og Floden styrtede imod det, og det faldt straks sammen, og dette Hus's Fald blev stort."

\chapter{7}

\par 1 Men da han havde fuldendt alle sine Ord i Folkets Påhør, gik han ind i Kapernaum.
\par 2 Men en Høvedsmands Tjener, som denne holdt meget af, var syg og nær ved at dø.
\par 3 men da han hørte om Jesus, sendte han nogle af Jødernes Ældste til ham og bad ham om, at han vilde komme og helbrede hans Tjener.
\par 4 Men da de kom til Jesus, bade de ham indtrængende og sagde: "Han er vel værd, at du gør dette for ham;
\par 5 thi han elsker vort Folk, og han har bygget Synagogen for os."
\par 6 Og Jesus gik med dem. Men da han allerede ikke var langt fra Huset, sendte Høvedsmanden nogle Venner og lod ham sige: "Herre! umag dig ikke; thi jeg er ikke værdig til, at du skal gå ind under mit Tag.
\par 7 Derfor agtede jeg heller ikke mig selv værdig til at komme til dig; men sig det med et Ord, så bliver min Dreng helbredt.
\par 8 Jeg er jo selv et Menneske, som står under Øvrighed og har Stridsmænd under mig; og siger jeg til den ene: Gå! så går han; og til den anden: Kom! så kommer han; og til min Tjener: Gør dette! så gør han det."
\par 9 Men da Jesus hørte dette, forundrede han sig over ham; og han vendte sig om og sagde til Skaren, som fulgte ham: "Jeg siger eder, end ikke i Israel har jeg fundet så stor en Tro."
\par 10 Og da de, som vare udsendte, kom tilbage til Huset, fandt de den syge Tjener sund.
\par 11 Og det skete Dagen derefter, at han gik til en By, som hed Nain, og der gik mange af hans Disciple og en stor Skare med ham.
\par 12 Men da han nærmede sig Byens Port, se, da blev en død båren ud, som var sin Moders enbårne Søn, og hun var Enke; og en stor Skare fra Byen gik med hende.
\par 13 Og da Herren så hende, ynkedes han inderligt over hende og sagde til hende: "Græd ikke!"
\par 14 Og han trådte til og rørte ved Båren; men de, som bare, stode stille, og han sagde: "du unge Mand, jeg siger dig, stå op!"
\par 15 Og den døde rejste sig op og begyndte at tale; og han gav ham til hans Moder.
\par 16 Men Frygt betog alle, og de priste Gud og sagde: "Der er en stor Profet oprejst iblandt os, og Gud har besøgt sit Folk."
\par 17 Og denne Tale om ham kom ud i hele Judæa og i hele det omliggende Land.
\par 18 Og Johannes's Disciple fortalte ham om alt dette. Og Johannes kaldte to af sine Disciple til sig
\par 19 og sendte dem til Herren og lod sige: "Er du den, som kommer, eller skulle vi vente en anden?"
\par 20 Og da Mændene kom til ham, sagde de: "Johannes Døberen har sendt os til dig og lader sige: Er du den, som kommer, eller skulle vi vente en anden?"
\par 21 I den samme Time helbredte han mange for Sygdomme og Plager og onde Ånder og skænkede mange blinde Synet.
\par 22 Og han svarede og sagde til dem: "Går hen, og forkynder Johannes de Ting, som I have set og hørt: Blinde se, lamme gå, spedalske renses, døve høre, døde stå op, Evangeliet forkyndes for fattige;
\par 23 og salig er den, som ikke forarges på mig."
\par 24 Men da Johannes's Sendebud vare gåede bort, begyndte han at sige til Skarerne om Johannes: "Hvad gik I ud i Ørkenen at skue? Et Rør, som bevæges hid og did af Vinden?
\par 25 Eller hvad gik I ud at se? Et Menneske, iført bløde Klæder? Se, de, som leve i prægtige Klæder og i Vellevned, ere i Kongsgårdene.
\par 26 Eller hvad gik I ud at se? En Profet? Ja, siger jeg eder, endog mere end en Profet!
\par 27 Han er den, om hvem der er skrevet: Se, jeg sender min Engel for dit Ansigt, han skal berede din Vej foran dig.
\par 28 Jeg siger eder: Iblandt dem, som ere fødte af Kvinder, er ingen større Profet end Johannes; men den mindste i Guds Rige er større end han.
\par 29 Og hele Folket, som hørte ham, endog Tolderne, gav Gud Ret, idet de bleve døbte med Johannes's Dåb.
\par 30 Men Farisæerne og de lovkyndige have foragtet Guds Råd med dem selv, idet de ikke bleve døbte af ham.
\par 31 Ved hvem skal jeg da ligne denne Slægts Mennesker? og hvem ligne de?
\par 32 De ligne Børn, som sidde på Torvet og råbe til hverandre og sige: Vi blæste på Fløjte for eder, og I dansede ikke, vi sang Klagesange for eder, og I græd ikke.
\par 33 Thi Johannes Døberen kom, som hverken spiste Brød eller drak Vin, og I sige: Han er besat.
\par 34 Menneskesønnen kom, som spiser og drikker, og I sige: Se, en Frådser og en Vindranker, Tolderes og Synderes Ven!
\par 35 Dog Visdommen er retfærdiggjort ved alle sine Børn!"
\par 36 Men en af Farisæerne bad ham om, at han vilde spise med ham; og han gik ind i Farisæerens Hus og satte sig til Bords.
\par 37 Og se, der var en Kvinde, som var en Synderinde i Byen; da hun fik at vide, at han sad til Bords i Farisæerens Hus, kom hun med en Alabastkrukke med Salve;
\par 38 og hun stillede sig bag ved ham, ved hans Fødder og græd og begyndte at væde hans Fødder med sine Tårer og aftørrede dem med sit Hovedhår og kyssede hans Fødder og salvede dem med Salven.
\par 39 Men da Farisæeren, som havde indbudt ham, så det, sagde han ved sig selv: "Dersom denne var en Profet, vidste han, hvem og hvordan en Kvinde denne er, som rører ved ham, at hun er en Synderinde."
\par 40 Og Jesus tog til Orde og sagde til ham: "Simon! jeg har noget at sige dig." Men han siger: "Mester, sig frem!"
\par 41 "En Mand, som udlånte Penge, havde to Skyldnere; den ene var fem Hundrede Denarer skyldig. men den anden halvtredsindstyve.
\par 42 Da de ikke havde noget at betale med, eftergav han dem det begge.
\par 43 Simon svarede og sagde: "Jeg holder for, den, hvem han eftergav mest?" Men han sagde til ham: "Du dømte ret."
\par 44 Og han vendte sig imod Kvinden og sagde til Simon: "Ser du denne Kvinde? Jeg kom ind i dit Hus; du gav mig ikke Vand til mine Fødder; men hun vædede mine Fødder med sine Tårer og aftørrede dem med sit Hår.
\par 45 Du gav mig intet Kys; men hun ophørte ikke med at kysse mine Fødder, fra jeg kom herind.
\par 46 Du salvede ikke mit Hoved med Olie; men hun salvede mine Fødder med Salve.
\par 47 Derfor siger jeg dig: Hendes mange Synder ere hende forladte, eftersom hun elskede meget; men den, hvem lidet forlades, elsker lidet."
\par 48 Men han sagde til hende: "Dine Synder ere forladte!"
\par 49 Og de, som sade til Bords med ham, begyndte at sige ved sig selv: "Hvem er denne, som endog forlader Synder?"
\par 50 Men han sagde til Kvinden: "Din Tro har frelst dig, gå bort med Fred!"

\chapter{8}

\par 1 Og det skete i Tiden der efter, at han rejste igennem Byer og Landsbyer og prædikede og forkyndte Evangeliet om Guds Rige, og med ham de tolv
\par 2 og nogle Kvinder, som vare helbredte fra onde Ånder og Sygdomme, nemlig: Maria, der kaldes Magdalene, af hvem syv onde Ånder vare udfarne;
\par 3 og Johanna, Herodes's Husfoged Kuzas Hustru, og Susanna og mange andre, som tjente dem med, hvad de ejede.
\par 4 Men da en stor Skare kom sammen, og de droge til ham fra de forskellige Byer, sagde han ved en Lignelse:
\par 5 "En Sædemand gik ud at så sin Sæd; og idet han såede, faldt noget ved Vejen og blev nedtrådt, og Himmelens Fugle åde det op.
\par 6 Og noget faldt på Klippen; og da det voksede op, visnede det, fordi det ikke havde Væde.
\par 7 Og noget faldt midt iblandt Torne, og Tornene voksede op med og kvalte det.
\par 8 Og noget faldt i den gode Jord, og det voksede op og bar hundrede Fold Frugt." Da han sagde dette, råbte han: "Den, som har Øren at høre med, han høre!"
\par 9 Men hans Disciple spurgte ham, hvad denne Lignelse skulde betyde.
\par 10 Og han sagde: "Eder er det givet at kende Guds Riges Hemmeligheder, men de andre i Lignelser, for at de, skønt seende, ikke skulle se, og, skønt hørende, ikke skulle forstå.
\par 11 Men dette er Lignelsen: Sæden er Guds Ord.
\par 12 Men de ved Vejen ere de, som høre det; derefter kommer Djævelen og tager Ordet bort af deres Hjerte, for at de ikke skulle tro og blive frelste.
\par 13 Men de på Klippen ere de, som modtage Ordet med Glæde, når de høre det, og disse have ikke Rod; de tro til en Tid og falde fra i Fristelsens Tid.
\par 14 Men det, som faldt iblandt Torne, det er dem, som have hørt og så gå hen og kvæles under Livets Bekymringer og Rigdom og Nydelser og ikke bære moden Frugt.
\par 15 Men det i den gode Jord, det er dem, som, når de have hørt Ordet, beholde det i et smukt og godt Hjerte og bære Frugt i Udholdenhed.
\par 16 Men ingen, som tænder et Lys, skjuler det med et Kar eller sætter det under en Bænk; men han sætter det på en Lysestage, for at de, som komme ind, kunne se Lyset.
\par 17 Thi der er ikke noget skjult, som jo skal blive åbenbart; og ikke noget lønligt, som jo skal blive kendt og komme for Lyset.
\par 18 Ser derfor til, hvorledes I høre; thi den, som har, ham skal der gives; og den, som ikke har, fra ham skal endog det tages, han synes at have."
\par 19 Men hans Moder og Brødre kom til ham og kunde ikke nå frem til ham for Skaren.
\par 20 Og det blev ham meddelt: "Din Moder og dine Brødre stå udenfor og begære at se dig."
\par 21 Men han svarede og sagde til dem: "Min Moder og mine Brødre ere disse, som høre Guds Ord og gøre efter det."
\par 22 Og det skete en af de Dage, at han gik om Bord i et Skib tillige med sine Disciple, og han sagde til dem: "Lader os fare over til hin Side af Søen;" og de sejlede ud.
\par 23 Men medens de sejlede, faldt han i Søvn; og en Stormvind for ned over Søen, og Skibet blev fuldt af Vand, og de vare i Fare.
\par 24 Da trådte de hen og vækkede ham og sagde: "Mester, Mester! vi forgå." Men han stod op og truede Vinden og Vandets Bølger; og de lagde sig, og det blev blikstille.
\par 25 Og han sagde til dem: "Hvor, er eders Tro?" Men de frygtede og undrede sig, og sagde til hverandre: "Hvem er dog denne, siden han byder både over Vindene og Vandet, og de ere ham lydige?"
\par 26 Og de sejlede ind til Gadarenernes Land, som ligger lige over for Galilæa.
\par 27 Men da han trådte ud på Landjorden, mødte der ham en Mand fra Byen, som i lang Tid havde været besat af onde Ånder og ikke havde haft Klæder på og ikke opholdt sig i Hus, men i Gravene.
\par 28 Men da han så Jesus, råbte han og faldt ned for ham og sagde med høj Røst: "Hvad har jeg med dig at gøre, Jesus, den højeste Guds Søn? jeg beder dig om, at du ikke vil pine mig."
\par 29 Thi han bød den urene Ånd at fare ud af Manden; thi i lange Tider havde den revet ham med sig, og han blev bunden med Lænker og Bøjer og bevogtet, og han sønderrev, hvad man bandt ham med, og dreves af den onde Ånd ud i Ørkenerne.
\par 30 Men Jesus spurgte ham og sagde: "Hvad er dit Navn?" Men han sagde: "Leion"; thi mange onde Ånder vare farne i ham.
\par 31 Og de bade ham om at han ikke vilde byde dem at fare ned i Afgrunden;
\par 32 men der var sammesteds en stor Hjord Svin, som græssede på Bjerget; og de bade ham om, at han vilde tilstede dem at fare i dem; og han tilstedte dem det.
\par 33 Men de onde Ånder fore ud at Manden og fore i Svinene, og Hjorden styrtede sig ned over Brinken ud i Søen og druknede.
\par 34 Men da Hyrderne så det, som var sket, flyede de og forkyndte det i Byen og på Landet.
\par 35 Da gik de ud for af se det, som var sket, og de kom til Jesus og fandt Manden, af hvem de onde Ånder vare udfarne, siddende ved Jesu Fødder, påklædt og ved Samling; og de frygtede.
\par 36 Og de, som havde set det, fortalte dem, hvorledes den besatte var bleven frelst.
\par 37 Og hele Mængden fra Gadarenernes Omegn bad ham om, at han vilde gå bort fra dem; thi de vare betagne af stor Frygt. Men han gik om Bord i et Skib og vendte tilbage igen.
\par 38 Men Manden, af hvem de onde Ånder vare udfarne, bad ham om, at han måtte være hos ham; men han lod ham fare og sagde:
\par 39 "Vend tilbage til dit Hus, og fortæl, hvor store Ting Gud har gjort imod dig." Og han gik bort og kundgjorde over hele Byen, hvor store Ting Jesus havde gjort imod ham.
\par 40 Men det skete, da Jesus kom tilbage, tog Skaren imod ham; thi de ventede alle på ham.
\par 41 Og se, det kom en Mand, som hed Jairus, og han var Forstander for Synagogen; og han faldt ned for Jesu Fødder og bad ham komme ind i hans Hus;
\par 42 thi han havde en enbåren Datter, omtrent tolv År gammel, og hun droges med Døden. Men idet han gik, trængte Skarerne sig sammen om ham.
\par 43 Og en Kvinde, som havde haft Blodflod i tolv År og havde kostet al sin Formue på Læger og ikke kunde blive helbredt af nogen,
\par 44 hun gik til bagfra og rørte ved Fligen af hans Klædebon, og straks standsedes hendes Blodflod.
\par 45 Og Jesus sagde: "Hvem var det, som rørte ved mig?" Men da alle nægtede det, sagde Peter og de, som vare med ham: "Mester! Skarerne trykke og trænge dig, og du siger: Hvem var det, som rørte ved mig?"
\par 46 Men Jesus sagde: "Der rørte nogen ved mig; thi jeg mærkede, at der udgik en Kraft fra mig."
\par 47 Men da Kvinden så, at det ikke var skjult, kom hun bævende og faldt ned for ham og fortalte i alt Folkets Påhør, af hvad Årsag hun havde rørt ved ham, og hvorledes hun straks var bleven helbredt.
\par 48 Men han sagde til hende: "Datter! din Tro har frelst dig; gå bort med Fred!"
\par 49 Medens han endnu talte, kommer der en fra Synagogeforstanderens Hus og siger til ham: "Din Datter er død; umag ikke Mesteren!"
\par 50 Men da Jesus hørte det, svarede han ham: "Frygt ikke; tro blot; så skal hun blive frelst."
\par 51 Men da han kom til Huset, tillod han ingen at gå ind med sig uden Peter og Johannes og Jakob og Pigens Fader og Moder.
\par 52 Og de græd alle og holdt Veklage over hende; men han sagde: "Græder ikke; hun er ikke død, men sover."
\par 53 Og de lo ad ham; thi de vidste, at hun var død.
\par 54 Men han greb hendes Hånd og råbte og sagde: "Pige, stå op!"
\par 55 Og hendes Ånd vendte tilbage, og hun stod straks op; og han befalede, at de skulde give hende noget at spise.
\par 56 Og hendes Forældre bleve forfærdede; men han bød dem, at de ikke måtte sige nogen det, som var sket.

\chapter{9}

\par 1 Men han sammenkaldte de tolv og gav dem Magt og Myndighed over alle de onde Ånder og til at helbrede Sygdomme.
\par 2 Og han sendte dem ud for at prædike Guds Rige og helbrede de syge.
\par 3 Og han sagde til dem: "Tager intet med på Vejen, hverken Stav eller Taske eller Brød eller Penge, ej heller skal nogen have to Kjortler.
\par 4 Og hvor I komme ind i et Hus, der skulle I blive og derfra drage bort.
\par 5 Og hvor som helst de ikke modtage eder, fra den By skulle I gå ud og endog ryste Støvet af eders Fødder til Vidnesbyrd imod dem."
\par 6 Men de gik ud og droge fra Landsby til Landsby, idet de forkyndte Evangeliet og helbredte alle Vegne.
\par 7 Men Fjerdingsfyrsten Herodes hørte alt det, som skete; og han var tvivlrådig, fordi nogle sagde, at Johannes var oprejst fra de døde;
\par 8 men nogle, at Elias havde vist sig; men andre, at en af de gamle Profeter var opstanden.
\par 9 Men Herodes sagde: "Johannes har jeg ladet halshugge; men hvem er denne, om hvem jeg hører sådanne Ting?" Og han søgte at få ham at se.
\par 10 Og Apostlene kom tilbage og fortalte ham, hvor store Ting de havde gjort. Og han tog dem med sig og drog bort afsides til en By, som kaldes Bethsajda.
\par 11 Men da Skarerne fik det at vide, fulgte de efter ham; og han tog imod dem og talte til dem om Guds Rige og helbredte dem, som trængte til Lægedom.
\par 12 Men Dagen begyndte at hælde. Og de tolv kom hen og sagde til ham: "Lad Skaren gå bort, for at de kunne gå herfra til de omliggende Landsbyer og Gårde og få Herberge og finde Føde; thi her ere vi på et øde Sted."
\par 13 Men han sagde til dem: "Giver I dem at spise!"Men de sagde: "Vi have ikke mere end fem Brød og to Fisk, med mindre vi skulle gå bort og købe Mad til hele denne Mængde."
\par 14 De vare nemlig omtrent fem Tusinde Mænd. Men han sagde til sine Disciple: "Lader dem sætte sig ned i Hobe, halvtredsindstyve i hver."
\par 15 Og de gjorde så og lode dem alle sætte sig ned.
\par 16 Men han tog de fem Brød og de to Fisk, så op til Himmelen og velsignede dem, og han brød dem og gav sine Disciple dem at lægge dem for Skaren.
\par 17 Og de spiste og bleve alle mætte; og det, som de fik tilovers af Stykker, blev opsamlet, tolv Kurve.
\par 18 Og det skete, medens han bad, vare hans Disciple alene hos ham; og han spurgte dem og sagde: "Hvem sige Skarerne, at jeg er?"
\par 19 Men de svarede og sagde: "Johannes Døberen; men andre: Elias; men andre: En af de gamle Profeter er opstanden."
\par 20 Og han sagde til dem: "Men I hvem sige I, at jeg er?" Og Peter svarede og sagde: "Guds Kristus."
\par 21 Men han bød dem strengt ikke at sige dette til nogen,
\par 22 idet han sagde: "Menneskesønnen skal lide meget og forkastes af de Ældste og Ypperstepræsterne og de skriftkloge og ihjelslås og oprejses på den tredje Dag."
\par 23 Men han sagde til alle: "Vil nogen komme efter mig, han fornægte sig selv og tage sit Kors op daglig og følge mig;
\par 24 thi den, som vil frelse sit Liv. skal miste det; men den, som mister sit Liv for min Skyld, han skal frelse det.
\par 25 Thi hvad gavner det et Menneske, om han har vundet den hele Verden, men mistet sig selv eller bødet med sig selv?
\par 26 Thi den, som skammer sig ved mig og mine Ord, ved ham skal Menneskesønnen skamme sig, når han kommer i sin og Faderens og de hellige Engles Herlighed.
\par 27 Men sandelig, siger jeg eder: Der er nogle af dem, som stå her.
\par 28 Men det skete omtrent otte Dage efter denne Tale, at han tog Peter og Johannes og Jakob med sig og gik op på Bjerget for at bede.
\par 29 Og det skete, medens han bad, da blev hans Ansigts Udseende anderledes, og hans Klædebon blev hvidt og strålende.
\par 30 Og se, to Mænd talte med ham, og det var Moses og Elias,
\par 31 som bleve set i Herlighed og talte om hans Udgang, som han skulde fuldbyrde i Jerusalem.
\par 32 Men Peter og de, som vare med ham, vare betyngede af Søvn; men da de vågnede op, så de hans Herlighed og de to Mænd, som stode hos ham.
\par 33 Og det skete, da disse skiltes fra ham, sagde Peter til Jesus: "Mester! det er godt, at vi ere her; og lader os gøre tre Hytter, en til dig og en til Moses og en til Elias;" men han vidste ikke, hvad han sagde.
\par 34 Men idet han sagde dette, kom en Sky og overskyggede dem; men de frygtede, da de kom ind i Skyen.
\par 35 Og der kom fra Skyen en Røst, som sagde: "Denne er min Søn, den udvalgte, hører ham!"
\par 36 Og da Røsten kom, blev Jesus funden alene. Og de tav og forkyndte i de Dage ingen noget af det, de havde set.
\par 37 Men det skete Dagen derefter, da de kom ned fra Bjerget, at der mødte ham en stor Skare.
\par 38 Og se, en Mand af Skaren råbte og sagde: "Mester! jeg beder dig, se til min Søn: thi han er min enbårne.
\par 39 Og se, en Ånd griber ham, og pludseligt skriger han, og den slider i ham, så at han fråder, og med Nød viger den fra ham, idet den mishandler ham;
\par 40 og jeg bad dine Disciple om at uddrive den; og de kunde ikke."
\par 41 Men Jesus svarede og sagde: "O du vantro og forvendte Slægt! hvor længe skal jeg være hos eder og tåle eder? Bring din Søn hid!"
\par 42 Men endnu medens han gik derhen, rev og sled den onde Ånd i ham.
\par 43 Men de bleve alle slagne af Forundring over Guds Majestæt. Men da alle undrede sig over alt det, han gjorde, sagde han til sine Disciple:
\par 44 "Gemmer i eders Øren disse Ord: Menneskesønnen skal overgives i Menneskers Hænder."
\par 45 Men de forstode ikke dette Ord, og det var skjult for dem, så de ikke begreb det, og de frygtede for at spørge ham om dette Ord.
\par 46 Men der opstod den Tanke hos dem, hvem der vel var den største af dem.
\par 47 Men da Jesus så deres Hjertes Tanke, tog, han et Barn og stillede det hos sig.
\par 48 Og han sagde til dem: "Den, som modtager dette Barn for mit Navns Skyld, modtager mig; og den, som modtager mig, modtager den, som udsendte mig; thi den, som er den mindste iblandt eder alle, han er stor."
\par 49 Men Johannes tog til Orde og sagde: "Mester! vi så en uddrive onde Ånder i dit Navn; og vi forbød ham det, fordi han ikke følger med os."
\par 50 Men Jesus sagde til ham: "Forbyder ham det ikke; thi den, som ikke er imod eder, er for eder."
\par 51 Men det skete, da hans Optagelses Dage vare ved at fuldkommes, da fæstede han sit Ansigt på at drage til Jerusalem.
\par 52 Og han sendte Sendebud forud for sig; og de gik og kom ind i en Samaritanerlandsby for at berede ham Herberge.
\par 53 Og de modtoge ham ikke, fordi han var på Vejen til Jerusalem.
\par 54 Men da hans Disciple, Jakob og Johannes, så det, sagde de: "Herre! vil du, at vi skulle byde Ild fare ned fra Himmelen og fortære dem, ligesom også Elias gjorde?"
\par 55 Men han vendte sig og irettesatte dem.
\par 56 Og de gik til en anden Landsby.
\par 57 Og medens de vandrede på Vejen, sagde en til ham: "Jeg vil følge dig, hvor du end går hen."
\par 58 Og Jesus sagde til ham: "Ræve have Huler, og Himmelens Fugle Reder; men Menneskesønnen har ikke det, hvortil han kan hælde sit Hoved."
\par 59 Men han sagde til en anden: "Følg mig!" Men denne sagde: "Herre! tilsted mig først at gå hen at begrave min Fader."
\par 60 Men han sagde til ham: "Lad de døde begrave deres døde; men gå du hen og forkynd Guds Rige!"
\par 61 Men også en anden sagde: "Herre! jeg vil følge dig; men tilsted mig først at tage Afsked med dem, som ere i mit Hus."
\par 62 Men Jesus sagde til ham: "Ingen, som lægger sin Hånd på Ploven og ser tilbage, er vel skikket for Guds Rige."

\chapter{10}

\par 1 Men derefter udvalgte Herren også halvfjerdsindstyve andre og sendte dem ud to og to forud for sig, til hver By og hvert Sted, hvorhen han selv vilde komme.
\par 2 Og han sagde til dem: "Høsten er stor, men Arbejderne ere få; beder derfor Høstens Herre om, at han vil sende Arbejdere ud til sin Høst.
\par 3 Går ud! Se, jeg sender eder som Lam midt iblandt Ulve.
\par 4 Bærer ikke Pung, ikke Taske, ej heller Sko; og hilser ingen på Vejen!
\par 5 Men hvor I komme ind i et Hus, siger der først: Fred være med dette Hus!
\par 6 Og er der sammesteds et Fredens Barn, skal eders Fred Hvile på ham; men hvis ikke, da skal den vende tilbage til eder igen.
\par 7 Men bliver i det samme Hus, spiser og drikker, hvad de have; thi Arbejderen er sin Løn værd. I må ikke flytte fra Hus til Hus
\par 8 Og hvor I komme ind i en By. og de modtage eder, spiser der, hvad der sættes for eder;
\par 9 og Helbreder de syge, som ere der, og siger dem: Guds Rige er kommet nær til eder.
\par 10 Men hvor I komme ind i en By og de ikke modtage eder, der skulle I gå ud på dens Gader og sige:
\par 11 Endog det Støv, som hænger ved vore Fødder fra eders By, tørre vi af til eder; dog dette skulle I vide, at Guds Rige er kommet nær.
\par 12 Men jeg siger eder, det skal gå Sodoma tåleligere på hin Dag end den By.
\par 13 Ve dig, Korazin! ve dig, Bethsajda! thi dersom de kraftige Gerninger, som ere skete i eder, vare skete i Tyrus og Sidon, da havde de for længe siden omvendt sig, siddende i Sæk og Aske.
\par 14 Men det skal gå Tyrus og Sidon tåleligere ved Dommen end eder.
\par 15 Og du, Kapernaum, som er bleven ophøjet indtil Himmelen, du skal nedstødes indtil Dødsriget.
\par 16 Den, som hører eder, hører mig, og den, som foragter eder, foragter mig; men den, som foragter mig, foragter den, som udsendte mig."
\par 17 Men de halvfjerdsindsfyve vendte tilbage med Glæde og sagde: "Herre! også de onde Ånder ere os lydige i dit Navn."
\par 18 Men han sagde til dem: "Jeg så Satan falde ned fra Himmelen som et Lyn.
\par 19 Se, jeg har givet eder Myndighed til at træde på Slanger og Skorpioner og over hele Fjendens Magt, og slet intet skal skade eder.
\par 20 Dog, glæder eder ikke derover, at Ånderne ere eder lydige; men glæder eder over, at eders Navne ere indskrevne i Himlene."
\par 21 I den samme Stund frydede Jesus sig i den Helligånd og sagde: "Jeg priser dig, Fader, Himmelens og Jordens Herre! fordi du har skjult dette for vise og forstandige og åbenbaret det for umyndige. Ja, Fader! thi således skete det, som var velbehageligt for dig.
\par 22 Alle Ting ere mig overgivne af min Fader; og ingen kender, hvem Sønnen er, uden Faderen, og hvem Faderen er, uden Sønnen og den, for hvem Sønnen vil åbenbare ham."
\par 23 Og han vendte sig til Disciplene og sagde særligt til dem: "Salige ere de Øjne, som se det, I se.
\par 24 Thi jeg siger eder, at mange Profeter og Konger have ville se det, I se, og have ikke set det, og høre det, I høre, og have ikke hørt det."
\par 25 Og se, en lovkyndig stod op og fristede ham og sagde: "Mester! hvad skal jeg gøre, for at jeg kan arve et evigt Liv?"
\par 26 Men han sagde til ham: "Hvad er der skrevet i Loven, hvorledes læser du?".
\par 27 Men han svarede og sagde til ham: "Du skal elske Herren din Gud af hele dit Hjerte og med hele din Sjæl og med hele din Styrke og med hele dit Sind, og din Næste som dig selv."
\par 28 Men han sagde til ham: "Du svarede ret; gør dette, så skal du leve."
\par 29 Men han vilde gøre sig selv retfærdig og sagde til Jesus: "Hvem er da min Næste?"
\par 30 Men Jesus svarede og sagde: "Et Menneske gik ned fra Jerusalem til Jeriko, og han faldt iblandt Røvere, som både klædte ham af og sloge ham og gik bort og lod ham ligge halvdød.
\par 31 Men ved en Hændelse gik en Præst den samme Vej ned, og da han så ham, gik han forbi.
\par 32 Ligeså også en Levit; da han kom til Stedet, gik han hen og så ham og gik forbi.
\par 33 Men en Samaritan, som var på Rejse, kom til ham, og da han så ham, ynkedes han inderligt.
\par 34 Og han gik hen til ham, forbandt hans Sår og gød Olie og Vin deri, løftede ham op på sit eget Dyr og førte ham til et Herberge og plejede ham.
\par 35 Og den næste Dag tog han to Denarer frem og gav Værten dem og sagde: Plej ham! og hvad mere du lægger ud, vil jeg betale dig, når jeg kommer igen.
\par 36 Hvilken af disse tre tykkes dig nu at have været hans Næste, der var falden iblandt Røverne?"
\par 37 Men han sagde: "Han, som øvede.Barmhjertighed imod ham." Og Jesus sagde til ham: "Gå bort, og gør du ligeså!"
\par 38 Men det skete, medens de vare på Vandring, gik han ind i en Landsby; og en Kvinde ved Navn Martha modtog ham i sit Hus.
\par 39 Og hun havde en Søster, som hed Maria, og hun satte sig ved Herrens Fødder og hørte på hans Tale.
\par 40 Men Martha havde travlt med megen Opvartning; og hun kom hen og sagde: "Herre! bryder du dig ikke om, at min Søster har ladet mig opvarte ene? Sig hende dog, at hun skal hjælpe mig."
\par 41 Men Herren svarede og sagde til hende: "Martha! Martha! du gør dig Bekymring og Uro med mange Ting;
\par 42 men eet er fornødent. Maria har valgt den gode Del, som ikke skal tages fra hende."

\chapter{11}

\par 1 Og det skete, da han var på et Sted og bad, at en af hans Disciple sagde til ham, da han holdt op: "Herre! lær os at bede, som også Johannes lærte sine Disciple."
\par 2 Da sagde han til dem: "Når I bede, da siger: Fader, Helliget vorde dit Navn; komme dit Rige;
\par 3 giv os hver dag vort daglige Brød;
\par 4 og forlad os vore Synder, thi også vi forlade hver, som er os skyldig; og led os ikke i Fristelse!"
\par 5 Og han sagde til dem: "Om nogen af eder har en Ven og går til ham ved Midnat og siger til ham: Kære! lån mig tre Brød,
\par 6 efterdi en Ven af mig er kommen til mig fra Rejsen, og jeg har intet at sætte for ham;
\par 7 og hin så svarer derinde fra og siger: Vold mig ikke Besvær; Døren er allerede lukket, og mine Børn ere med mig i Seng; jeg kan ikke stå op og give dig det:
\par 8 da, siger jeg eder, om han end ikke står op og giver ham det, fordi han er hans Ven, så står han dog op for hans Påtrængenheds Skyld og giver ham alt, hvad han trænger til.
\par 9 Og jeg siger eder: Beder, så skal eder gives; søger, så skulle I finde; banker på, så skal der lukkes op for eder.
\par 10 Thi hver den, som beder, han får, og den, som søger, han finder, og den, som banker på, for ham skal der lukkes op.
\par 11 Men hvilken Fader iblandt eder vil give sin Søn en Sten, når han beder om Brød, eller når han beder om en Fisk, mon han da i Stedet for en Fisk vil give ham en Slange?
\par 12 Eller når han beder om et Æg, mon han da vil give ham en Skorpion?
\par 13 Dersom da I, som ere onde, vide at give eders Børn gode Gaver, hvor meget mere skal da Faderen fra Himmelen give den Helligånd til dem, som bede ham!"
\par 14 Og han uddrev en ond Ånd, og den var stum; men det skete, da den onde Ånd var udfaren, talte den stumme, og Skaren forundrede sig.
\par 15 Men nogle af dem sagde: "Ved Beelzebul, de onde Ånders Fyrste, uddriver han de onde Ånder."
\par 16 Men andre fristede ham og forlangte af ham et Tegn fra Himmelen..
\par 17 Men da han kendte deres Tanker, sagde han til dem: "Hvert Rige, som er kommet i Splid med sig selv, lægges øde, og Hus falder over Hus.
\par 18 Men hvis også Satan er kommen i Splid med sig selv, hvorledes skal hans Rige da bestå? Thi l sige, at jeg uddriver de onde Ånder ved Beelzebul.
\par 19 Men dersom jeg uddriver de onde Ånder ved Beelzebul, ved hvem uddrive da eders Sønner dem? Derfor skulle de være eders Dommere.
\par 20 Men dersom jeg uddriver de onde Ånder ved Guds Finger, da er jo Guds Rige kommet til eder.
\par 21 Når den stærke bevæbnet vogter sin Gård, bliver det, han ejer, i Fred.
\par 22 Men når en stærkere end han er kommen over ham og har overvundet ham, da tager han hans fulde Rustning, som han forlod sig på, og uddeler hans Bytte.
\par 23 Den, som ikke er med mig, er imod mig; og den, som ikke samler med mig, adspreder.
\par 24 Når den urene Ånd er faren ud af Mennesket, vandrer den igennem vandløse Steder og søger Hvile; og når den ikke finder den, siger den: Jeg vil vende tilbage til mit Hus, som jeg gik ud af.
\par 25 Og når den kommer, finder den det fejet og prydet.
\par 26 Da går den bort og tager syv andre Ånder med sig, som ere værre end den selv, og når de ere komne derind, bo de der; og det sidste bliver værre med dette Menneske end det første."
\par 27 Men det skete, medens han sagde disse Ting, da opløftede en Kvinde af Skaren sin Røst og sagde til ham: "Saligt er det Liv, som bar dig, og de Bryster, som du diede."
\par 28 Men han sagde: "Ja, salige ere de, som høre Guds Ord og bevare det."
\par 29 Men da Skarerne strømmede til, begyndte han at sige: "Denne Slægt er en ond Slægt; et Tegn forlanger den, og der skal intet Tegn gives den uden Jonas's Tegn.
\par 30 Thi ligesom Jonas blev et Tegn for Niniviterne, således skal også Menneskesønnen være det for denne Slægt.
\par 31 Sydens Dronning skal oprejses ved Dommen sammen med Mændene af denne Slægt og fordømme dem; thi hun kom fra Jordens Grænser for at høre Salomons Visdom; og se, her er mere end Salomon.
\par 32 Mænd fra Ninive skulle opstå ved Dommen sammen med denne Slægt og fordømme den; thi de omvendte sig ved Jonas's Prædiken; og se, her er mere end Jonas.
\par 33 Ingen tænder et Lys og sætter det i Skjul, ikke heller under Skæppen, men på Lysestagen, for at de, som komme ind, kunne se dets Skin.
\par 34 Dit Øje er Legemets Lys; når dit Øje er sundt, er også hele dit Legeme lyst, men dersom det er dårligt, er også dit Legeme mørkt.
\par 35 Se derfor til, at det Lys, der er i dig, ikke er Mørke.
\par 36 Dersom da hele dit Legeme et lyst, så at ingen Del deraf er mørk, vil det være helt lyst, som når Lyset bestråler dig med sin Glans."
\par 37 Men idet han talte, beder en Farisæer ham om, at han vilde spise Middagsmåltid hos ham, og han gik ind og satte sig til Bords.
\par 38 Men Farisæeren forundrede sig, da han så, at han ikke toede sig først før Måltidet.
\par 39 Men Herren sagde til ham: "I Farisæere rense nu det udvendige af Bægeret og Fadet; men eders Indre er fuldt af Rov og Ondskab.
\par 40 I Dårer! han, som gjorde det ydre, gjorde han ikke også det indre?
\par 41 Men giver det, som er indeni, til Almisse; se, så ere alle Ting eder rene.
\par 42 Men ve eder, I Farisæere! thi I give Tiende af Mynte og Rude og alle Hånde Urter og forbigå Retten og Kærligheden til Gud; disse Ting burde man gøre og ikke forsømme hine.
\par 43 Ve eder, I Farisæere! thi I elske den fornemste Plads i Synagogerne og Hilsenerne på Torvene.
\par 44 Ve eder, thi I ere som de ukendelige Grave, og Menneskene, som gå over dem, vide det ikke."
\par 45 Men en af de lovkyndige svarede og siger til ham: "Mester! idet du siger dette, forhåner du også os,"
\par 46 Men han sagde: "Ve også eder, I lovkyndige! thi I lægge Menneskene Byrder på, vanskelige at bære, og selv røre I ikke Byrderne med een af eders Fingre.
\par 47 Ve eder! thi I bygge Profeternes Grave, og eders Fædre sloge dem ihjel.
\par 48 Altså ere I Vidner og samtykke i eders Fædres Gerninger; thi de sloge dem ihjel, og I bygge.
\par 49 Derfor har også Guds Visdom sagt: Jeg vil sende Profeter og Apostle til dem, og nogle af dem skulle de slå ihjel og forfølge,
\par 50 for at alle Profeternes Blod, som er udøst fra Verdens Grundlæggelse, skal kræves af denne Slægt,
\par 51 fra Abels Blod indtil Sakarias's Blod, som blev dræbt imellem Alteret og Templet; ja, jeg siger eder: Det skal kræves af denne Slægt.
\par 52 Ve eder, I lovkyndige! thi I have taget Kundskabens Nøgle; selv ere I ikke gåede ind, og dem, som vilde gå ind, have I forhindret."
\par 53 Og da han var gået ud derfra, begyndte de skriftkloge og Farisæerne at trænge stærkt ind på ham og at lokke Ord af hans Mund om flere Ting;
\par 54 thi de lurede på ham for at opfange noget af hans Mund, for at de kunde anklage ham.

\chapter{12}

\par 1 Da imidlertid mange Tusinde Mennesker havde samlet sig, så at de trådte på hverandre, begyndte han at sige til sine Disciple: "Tager eder først og fremmest i Vare for Farisæernes Surdejg, som er Hykleri.
\par 2 Men intet er skjult, som jo skal åbenbares, og intet er lønligt, som jo skal blive kendt.
\par 3 Derfor, alt hvad I have sagt i Mørket, skal høres i Lyset; og hvad I have talt i Øret i Kamrene, skal blive prædiket på Tagene.
\par 4 Men jeg siger til eder, mine Venner! frygter ikke for dem, som slå Legemet ihjel og derefter ikke formå at gøre mere.
\par 5 Men jeg vil vise eder, for hvem I skulle frygte: Frygter for ham, som har Magt til, efter at have slået ihjel, at kaste i Helvede; ja, jeg siger eder: Frygter for ham!
\par 6 Sælges ikke fem Spurve for to Penninge? og ikke een af dem er glemt hos Gud.
\par 7 Ja, endog Hårene på eders Hoved ere alle talte; frygter ikke, I ere mere værd end mange Spurve.
\par 8 Men jeg siger eder: Enhver, som vedkender sig mig for Menneskene, ham vil også Menneskesønnen vedkende sig for Guds Engle.
\par 9 Og den, som har fornægtet mig for Menneskene, skal fornægtes for Guds Engle.
\par 10 Og enhver, som taler et Ord imod Menneskesønnen, ham skal det forlades; men den, som har talt bespotteligt imod den Helligånd, ham skal det ikke forlades.
\par 11 Men når de føre eder frem for Synagogerne og Øvrighederne og Myndighederne, da bekymrer eder ikke for, hvorledes eller hvormed I skulle forsvare eder, eller hvad I skulle sige.
\par 12 Thi den Helligånd skal lære eder i den samme Time, hvad I bør sige."
\par 13 Men en af Skaren sagde til ham: "Mester! sig til min Broder, at han skal dele Arven med mig."
\par 14 Men han sagde til ham: "Menneske! hvem har sat mig til Dommer eller Deler over eder?"
\par 15 Og han sagde til dem: "Ser til og vogter eder for al Havesyge; thi ingens Liv beror på, hvad han ejer, selv om han har Overflod."
\par 16 Og han sagde en Lignelse til dem: "Der var en rig Mand, hvis Mark havde båret godt.
\par 17 Og han tænkte ved sig selv og sagde: Hvad skal jeg gøre? thi jeg har ikke Rum, hvori jeg kan samle min Afgrøde.
\par 18 Og han sagde: Dette vil jeg gøre, jeg vil nedbryde mine Lader og bygge dem større, og jeg vil samle deri al min Afgrøde og mit Gods;
\par 19 og jeg vil sige til min Sjæl: Sjæl! du har mange gode Ting liggende for mange År; slå dig til Ro, spis, drik, vær lystig!
\par 20 Men Gud sagde til ham: Du Dåre! i denne Nat kræves din Sjæl af dig; men hvem skal det høre til. som du har beredt?
\par 21 Således er det med den, som samler sig Skatte og ikke er rig i Gud."
\par 22 Men han sagde til sine Disciple: "Derfor siger jeg eder: Bekymrer eder ikke for Livet, hvad I skulle spise; ikke heller for Legemet, hvad I skulle iføre eder.
\par 23 Livet er mere end Maden, og Legemet mere end Klæderne.
\par 24 Giver Agt på Ravnene, at de hverken så eller høste og de have ikke Forrådskammer eller Lade, og Gud føder dem; hvor langt mere værd end Fuglene ere dog I?
\par 25 Og hvem af eder kan ved at bekymre sig lægge en Alen til sin Vækst?
\par 26 Formå I altså ikke engang det mindste, hvorfor bekymre I eder da for det øvrige?
\par 27 Giver Agt på Lillierne, hvorledes de vokse; de arbejde ikke og spinde ikke; men jeg siger eder: End ikke Salomon i al sin Herlighed var klædt som en af dem.
\par 28 Klæder da Gud således det Græs på Marken, som i Dag står og i Morgen kastes i Ovnen, hvor meget mere eder, I lidettroende!
\par 29 Og I, spørger ikke efter, hvad I skulle spise, og hvad I skulle drikke; og værer ikke ængstelige!
\par 30 Thi efter alt dette søge Hedningerne i Verden; men eders Fader ved, at I have disse Ting nødig.
\par 31 Men søger hans Rige, så skulle disse Ting gives eder i Tilgift.
\par 32 Frygt ikke, du lille Hjord! thi det var eders Fader velbehageligt at give eder Riget.
\par 33 Sælger, hvad I eje, og giver Almisse! Gører eder Punge, som ikke ældes, en Skat i Himlene, som ikke slipper op, der hvor ingen Tyv kommer nær, og intet Møl ødelægger.
\par 34 Thi hvor eders Skat er, der vil også eders Hjerte være.
\par 35 Eders Lænder være omgjordede, og eders Lys brændende!
\par 36 Og værer I ligesom Mennesker, der vente på deres Herre, når han vil bryde op fra Brylluppet, for at de straks, når han kommer og banker på, kunne lukke op for ham.
\par 37 Salige ere de Tjenere, som Herren finder vågne, når han kommer.
\par 38 Og dersom han kommer i den anden Nattevagt og kommer i den tredje Nattevagt og finder det således, da ere disse Tjenere salige.
\par 39 Men dette skulle I vide, at dersom Husbonden vidste, i hvilken Time Tyven vilde komme, da vågede han og tillod ikke, at der skete Indbrud i hans Hus.
\par 40 Vorder også I rede; thi Menneskesønnen kommer i den Time, som I ikke mene."
\par 41 Men Peter sagde til ham: "Herre! siger du denne Lignelse til os eller også til alle?"
\par 42 Og Herren sagde: "Hvem er vel den tro og forstandige Husholder, som Herren vil sætte over sit Tyende til at give dem den bestemte Kost i rette Tid?
\par 43 Salig er den Tjener, hvem hans Herre, når han kommer, finder handlende således.
\par 44 Sandelig, siger jeg eder, han skal sætte ham over alt, hvad han ejer.
\par 45 Men dersom hin Tjener siger i sit Hjerte: "Min Herre tøver med at komme" og så begynder at slå Karlene og Pigerne og at spise og drikke og beruse sig,
\par 46 da skal den Tjeners Herre komme på den Dag, han ikke venter, og i den Time, han ikke ved, og hugge ham sønder og give ham hans Lod sammen med de utro:
\par 47 Men den Tjener, som har kendt sin Herres Villie og ikke har truffet Forberedelser eller handlet efter hans Villie, skal have mange Hug;
\par 48 men den, som ikke har kendt den og har gjort, hvad der er Hug værd, skal have få Hug. Enhver, hvem meget er givet, af ham skal man kræve meget; og hvem meget er betroet, af ham skal man forlange mere.
\par 49 Ild er jeg kommen at kaste på Jorden, og hvor vilde jeg, at den var optændt allerede!
\par 50 Men en Dåb har jeg at døbes med, og hvor ængstes jeg, indtil den er fuldbyrdet!
\par 51 Mene I, at jeg er kommen for at give Fred på Jorden? Nej, siger jeg eder, men Splid,
\par 52 Thi fra nu af skulle fem i eet Hus være i Splid indbyrdes, tre imod to, og to imod tre.
\par 53 De skulle være i Splid, Fader med Søn og Søn med Fader, Moder med Datter og Datter med Moder, Svigermoder med sin Svigerdatter og Svigerdatter med sin Svigermoder."
\par 54 Men han sagde også til Skarerne: "Når I se en Sky komme op i Vester, sige I straks: Der kommer Regn, og det sker således.
\par 55 Og når I se en Søndenvind blæse, sige I: Der kommer Hede: og det sker.
\par 56 I Hyklere! Jordens og Himmelens Udseende vide I at skønne om; men hvorfor have I da intet Skøn om den nærværende Tid?
\par 57 Og hvorfor dømme I ikke også fra eder selv, hvad der er det rette?
\par 58 Thi medens du går hen med din Modpart til Øvrigheden, da gør dig Flid på Vejen for at blive forligt med ham, for at han ikke skal trække dig for Dommeren, og Dommeren skal overgive dig til Slutteren, og Slutteren skal kaste dig i Fængsel.
\par 59 Jeg siger dig: Du skal ingenlunde komme ud derfra, førend du får betalt endog den sidste Skærv."

\chapter{13}

\par 1 Men på den samme Tid var der nogle til Stede, som fortalte ham om de Galilæere, hvis Blod Pilatus havde blandet med deres Ofre.
\par 2 Og han svarede og sagde til dem: "Mene I, at disse Galilæere vare Syndere frem for alle Galilæere, fordi de have lidt dette?
\par 3 Nej, siger jeg eder; men dersom I ikke omvende eder, skulle I alle omkomme ligeså.
\par 4 Eller hine atten, som Tårnet i Siloam faldt ned over og ihjelslog, mene I, at de vare skyldige fremfor alle Mennesker, som bo i Jerusalem?
\par 5 Nej, siger jeg eder; men dersom I ikke omvende eder, skulle I alle omkomme ligeså."
\par 6 Men han sagde denne Lignelse: "En havde et Figentræ, som var plantet i hans Vingård; og han kom og ledte efter Frugt derpå og fandt ingen.
\par 7 Men han sagde til Vingårdsmanden: Se, i tre År er jeg nu kommen og har ledt efter Frugt på dette Figentræ og ingen fundet; hug det om; hvorfor skal det tilmed gøre Jorden unyttig?
\par 8 Men han svarede og sagde til ham: Herre! lad det stå endnu dette År, indtil jeg får gravet om det og gødet det;
\par 9 måske vil det bære Frugt i Fremtiden; men hvis ikke, da hug det om!"
\par 10 Men han lærte i en af Synagogerne på Sabbaten.
\par 11 Og se, der var en Kvinde, som havde haft en Svagheds Ånd i atten År, og hun var sammenbøjet og kunde aldeles ikke rette sig op.
\par 12 Men da Jesus så hende, kaldte han på hende og sagde til hende: "Kvinde! du er løst fra din Svaghed."
\par 13 Og han lagde Hænderne på hende; og straks rettede hun sig op og priste Gud.
\par 14 Men Synagogeforstanderen, som var vred, fordi Jesus helbredte på Sabbaten, tog til Orde og sagde til Folkeskaren: "Der er seks Dage, på hvilke man bør arbejde; kommer derfor på dem og lader eder helbrede, og ikke på Sabbatsdagen!"
\par 15 Men Herren svarede ham og sagde: "I Hyklere! løser ikke enhver iblandt eder sin Okse eller sit Asen fra Krybben på Sabbaten og fører dem til Vands?
\par 16 Men denne, som er en Abrahams Datter, hvem Satan har bundet, se, i atten År, burde hun ikke løses fra dette Bånd på Sabbatsdagen?"
\par 17 Og da han sagde dette, bleve alle hans Modstandere beskæmmede; og hele Skaren glædede sig over alle de herlige Gerninger, som gjordes af ham.
\par 18 Han sagde da: "Hvad ligner Guds Rige, og hvormed skal jeg ligne det?
\par 19 Det ligner et Sennepskorn, som et Menneske tog og lagde i sin Have; og det voksede og blev til et Træ, og Himmelens Fugle byggede Rede i dets Grene."
\par 20 Og atter sagde han: "Hvormed skal jeg ligne Guds Rige?
\par 21 Det ligner en Surdejg, som en Kvinde tog og lagde ned i tre Mål Mel, indtil det blev syret alt sammen."
\par 22 Og han gik igennem Byer og Landsbyer og lærte og tog Vejen til Jerusalem.
\par 23 Men en sagde til ham: "Herre mon de ere få, som blive frelste?" Da sagde han til dem:
\par 24 "Kæmper for at komme ind igennem den snævre Port; thi mange, siger jeg eder, skulle søge at komme ind og ikke formå det.
\par 25 Fra den Stund Husbonden er stået op og har lukket Døren, og I begynde at stå udenfor og banke på Døren og sige: Herre, luk op for os! da vil han svare og sige til eder: Jeg kender eder ikke, hvorfra I ere;
\par 26 da skulle I begynde at sige: vi spiste og drak for dine Øjne, og du lærte på vore Gader,
\par 27 og han skal sige: Jeg siger eder, jeg kender eder ikke, hvorfra I ere; viger bort fra mig, alle I, som øve Uret!
\par 28 Der skal der være Gråd og Tænders Gnidsel, når I må se Abraham og Isak og Jakob og alle Profeterne i Guds Rige, men eder selv blive kastede udenfor.
\par 29 Og de skulle komme fra Øster og Vester og fra Norden og Sønden og sidde til Bords i Guds Rige.
\par 30 Og se, der er sidste, som skulle være iblandt de første, og der er første, som skulle være iblandt de sidste."
\par 31 I den samme Stund kom nogle Farisæere og sagde til ham: "Gå bort, og drag herfra; thi Herodes vil slå dig ihjel."
\par 32 Og han sagde til dem: "Går hen og siger til denne Ræv: Se, jeg uddriver onde Ånder og fuldfører Helbredelser i Dag og i Morgen, og på den tredje dag fuldendes jeg.
\par 33 Dog bør jeg vandre i Dag og i Morgen og den Dag derefter; thi det sømmer sig ikke, at en Profet dræbes uden for Jerusalem.
\par 34 Jerusalem! Jerusalem! som ihjelslår Profeterne og stener dem, som ere sendte til dig! hvor ofte vilde jeg samle dine Børn, ligesom en Høne samler sine Kyllinger under Vingerne! Og I vilde ikke.
\par 35 Se, eders Hus overlades til eder selv. Men jeg siger eder: I skulle ingenlunde se mig, førend den Tid kommer, da I sige: Velsigtnet være den, som kommer, i Herrens Navn!"

\chapter{14}

\par 1 Og det skete, da han kom ind i en af de øverste Farisæeres Hus på en Sabbat for at holde Måltid, at de toge Vare på ham.
\par 2 Og se, der stod en vattersottig Mand foran ham.
\par 3 Og Jesus tog til Orde og sagde til de lovkyndige og Farisæerne: "Er det tilladt at helbrede på Sabbaten eller ej?"
\par 4 Men de tav. Og han tog på ham og helbredte ham og lod ham fare.
\par 5 Og han tog til Orde og sagde til dem: "Hvem er der iblandt eder, som ikke straks, når hans Søn eller Okse falder i en Brønd, drager dem op på Sabbatsdagen?"
\par 6 Og de kunde ikke give Svar derpå.
\par 7 Men han sagde en Lignelse til de budne, da han gav Agt på, hvorledes de udvalgte sig de øverste Pladser ved Bordet, og sagde til dem:
\par 8 "Når du bliver buden af nogen til Bryllup, da sæt dig ikke øverst til Bords, for at ikke en fornemmere end du måtte være buden af ham,
\par 9 og han, som indbød dig og ham, måtte komme og sige til dig: Giv denne Plads, og du da med Skam komme til at sidde nederst.
\par 10 Men når du bliver buden, da gå hen og sæt dig nederst, for at, når han kommer, som har indbudt dig, han da må sige til dig: Ven! sæt dig højere op; da skal du have Ære for alle dem, som sidde til Bords med dig.
\par 11 Thi enhver, som Ophøjer sig selv, skal fornedres; og den, som fornedrer sig selv, skal ophøjes."
\par 12 Men han sagde også til ham, som havde indbudt ham: "Når du gør Middags- eller Aftensmåltid, da byd ikke dine Venner, ej heller dine Brødre, ej heller dine Frænder, ej heller rige Naboer, for at ikke også de skulle indbyde dig igen, og du få Vederlag.
\par 13 Men når du gør et Gæstebud, da indbyd fattige, vanføre, lamme, blinde!
\par 14 Så skal du være salig; thi de have intet at gengælde dig med; men det skal gengældes dig i de retfærdiges Opstandelse."
\par 15 Men da en af dem, som sade med til Bords, hørte dette, sagde han til ham: "Salig er den, som holder Måltid i Guds Rige."
\par 16 Men han sagde til ham: "der var en Mand, som gjorde en stor Nadver og indbød mange.
\par 17 Og han udsendte sin Tjener på Nadverens Time for at sige til de budne: Kommer! thi nu er det beredt.
\par 18 Og de begyndte alle som een at undskylde sig. Den første sagde til ham: Jeg har købt en Mark og har nødig at gå ud og se den; jeg beder dig, hav mig undskyldt!
\par 19 Og en anden sagde: Jeg har købt fem Par Okser og går hen at prøve dem; jeg beder dig, hav mig undskyldt!
\par 20 Og en anden sagde: Jeg har taget mig en Hustru til Ægte, og derfor kan jeg ikke komme.
\par 21 Og Tjeneren kom og meldte sin Herre dette; da blev Husbonden vred og sagde til sin Tjener: Gå hurtig ud på Byens Stræder og Gader, og før de fattige og vanføre og lamme og blinde herind!
\par 22 Og Tjeneren sagde: Herre! det er sket, som du befalede, og der er endnu Rum.
\par 23 Og Herre sagde til Tjeneren: Gå ud på Vejene og ved Gærderne og nød dem til at gå ind, for at mit Hus kan blive fuldt.
\par 24 Thi jeg siger eder, at ingen af hine Mænd, som vare budne, skal smage min Nadver."
\par 25 Men store Skarer gik med ham, og han vendte sig og sagde til dem:
\par 26 "Dersom nogen kommer til mig og ikke hader sin Fader og Moder og Hustru og Børn og Brødre og Søstre, ja endog sit eget Liv, kan han ikke være min Discipel.
\par 27 Den, som ikke bærer sit Kors og følger efter mig, kan ikke være min Discipel.
\par 28 Thi hvem iblandt eder, som vil bygge et Tårn, sætter sig ikke først hen og beregner Omkostningen, om han har nok til at fuldføre det,
\par 29 for at ikke, når han får lagt Grunden og ej kan fuldende det, alle, som se det, skulle begynde at spotte ham og sige:
\par 30 Dette Menneske begyndte at bygge og kunde ikke fuldende det.
\par 31 Eller hvilken Konge, som drager ud for at gå i Kamp imod en anden Konge, sætter sig ikke først hen og rådslår, om han er mægtig til med ti Tusinde at møde den, som kommer imod ham med tyve Tusinde?
\par 32 Men hvis ikke, sender han, medens den anden endnu er langt borte, Sendebud hen og underhandler om Fred.
\par 33 Således kan da ingen af eder, som ikke forsager alt det, han ejer, være min Discipel.
\par 34 Saltet er altså godt; men dersom også Saltet mister sin Kraft, hvorved skal det da få den igen?
\par 35 Det er ikke tjenligt hverken til Jord eller til Gødning; man kaster det ud. Den, som har Øren at høre med, han høre!"

\chapter{15}

\par 1 Men alle Toldere og Syndere holdt sig nær til ham for at høre ham.
\par 2 Og både Farisæerne og de skriftkloge knurrede og sagde: "Denne tager imod Syndere og spiser med dem."
\par 3 Men han talte denne Lignelse til dem og sagde:
\par 4 "Hvilket Menneske af eder, som har hundrede Får og har mistet eet af dem, forlader ikke de ni og halvfemsindstyve i Ørkenen og går ud efter det, han har mistet, indtil han finder det?
\par 5 Og når han har fundet det, lægger han det på sine Skuldre med Glæde.
\par 6 Og når han kommer hjem, sammenkalder han sine Venner og Naboer og siger til dem: Glæder eder med mig; thi jeg har fundet mit Får, som jeg havde mistet.
\par 7 Jeg siger eder: Således skal der være Glæde i Himmelen over een Synder, som omvender sig, mere end over ni og halvfemsindstyve retfærdige, som ikke trænge til Omvendelse.
\par 8 Eller hvilken Kvinde, som har ti Drakmer og taber een Drakme, tænder ikke Lys og fejer Huset og søger med Flid, indtil hun finder den?
\par 9 Og når hun har fundet den, sammenkalder hun sine Veninder og Naboersker og siger: Glæder eder med mig; thi jeg har fundet den Drakme, som jeg havde tabt.
\par 10 Således, siger jeg eder, bliver der Glæde hos Guds Engle over een Synder, som omvender sig."
\par 11 Men han sagde: "En Mand havde to Sønner.
\par 12 Og den yngste af dem sagde til Faderen: Fader! giv mig den Del af Formuen, som tilfalder mig. Og han skiftede Godset imellem dem.
\par 13 Og ikke mange Dage derefter samlede den yngste Søn alt sit og drog udenlands til et fjernt Land og ødte der sin Formue i et ryggesløst Levned.
\par 14 Men da han havde sat alt til, blev der en svær Hungersnød i det samme Land; og han begyndte at lide Mangel.
\par 15 Og han gik hen og holdt sig til en af Borgerne der i Landet, og denne sendte ham ud på sine Marker for at vogte Svin.
\par 16 Og han attråede at fylde sin Bug med de Bønner, som Svinene åde; og ingen gav ham noget.
\par 17 Men han gik i sig selv og sagde: Hvor mange Daglejere hos min Fader have ikke Brød i Overflødighed? men jeg omkommer her af Hunger.
\par 18 Jeg vil stå op og gå til min Fader og sige til ham: Fader! jeg har syndet imod Himmelen og over for dig,
\par 19 jeg er ikke længer værd at kaldes din Søn, gør mig som en af dine Daglejere!
\par 20 Og han stod op og kom til sin Fader. Men da han endnu var langt borte, så hans Fader ham og ynkedes inderligt, og han løb til og faldt ham om Halsen og kyssede ham.
\par 21 Men Sønnen sagde til ham: Fader! jeg har syndet imod Himmelen og over for dig, jeg er ikke længer værd at kaldes din Søn.
\par 22 Men Faderen sagde til sine Tjenere: Henter det bedste Klædebon frem, og ifører ham det, og giver ham en Ring på hans Hånd og Sko på Fødderne;
\par 23 og henter Fedekalven og slagter den, og lader os spise og være lystige!
\par 24 Thi denne min Søn var død og er bleven levende igen, han var fortabt og er funden. Og de begyndte at være lystige!
\par 25 Men hans ældste Søn var på Marken, og da han kom og nærmede sig Huset, hørte han Musik og Dans.
\par 26 Og han kaldte en af Karlene til sig og spurgte, hvad dette var?
\par 27 Men han sagde til ham: Din Broder er kommen, og din Fader har slagtet Fedekalven, fordi han har fået ham sund igen.
\par 28 Men han blev vred og vilde ikke gå ind. Men hans Fader gik ud og bad ham.
\par 29 Men han svarede og sagde til Faderen: Se, så mange År har jeg tjent dig, og aldrig har jeg overtrådt noget af dine Bud, og du har aldrig givet mig et Kid, for at jeg kunde være lystig med mine Venner.
\par 30 Men da denne din Søn kom, som har fortæret dit Gods med Skøger, slagtede du Fedekalven til ham.
\par 31 Men han sagde til ham: Barn! du er altid hos mig, og alt mit er dit.
\par 32 Men man burde være lystig og glæde sig, fordi denne din Broder var død og er bleven levende og var fortabt og er funden."

\chapter{16}

\par 1 Men han sagde også til Disciplene: "Der var en rig Mand, som havde en Husholder, og denne blev angiven for ham som en, der ødte hans Ejendom.
\par 2 Og han lod ham kalde og sagde til ham: Hvad er dette, jeg hører om dig? Aflæg Regnskabet for din Husholdning; thi du kan ikke længer være Husholder.
\par 3 Men Husholderen sagde ved sig selv: Hvad skal jeg gøre, efterdi min Herre tager Husholdningen fra mig? Jeg formår ikke at Grave, jeg skammer mig ved at tigge.
\par 4 Nu ved jeg, hvad jeg vil gøre, for af de skulle modtage mig i deres Huse, når jeg bliver sat fra Husholdningen.
\par 5 Og han kaldte hver enkelt af sin Herres Skyldnere til sig og sagde til den første: Hvor meget er du min Herre skyldig?
\par 6 Men han sagde: Hundrede Fade Olie. Og han sagde til ham: Tag dit Skyldbrev, og sæt dig hurtig ned og skriv halvtredsindstyve!
\par 7 Derefter sagde han til en anden: Men du, hvor meget er du skyldig? Men han sagde: Hundrede Mål Hvede. Han siger til ham: Tag dit Skyldbrev og skriv firsindstyve!
\par 8 Og Herren roste den uretfærdige Husholder, fordi han havde handlet klogelig; thi denne Verdens Børn ere klogere end Lysets Børn imod deres egen Slægt.
\par 9 Og jeg siger eder: Gører eder Venner ved Uretfærdighedens Mammon, for at de, når det er forbi med den, må modtage eder i de evige Boliger.
\par 10 Den, som er tro i det mindste, er også tro i meget, og den, som er uretfærdig i det mindste, er også uretfærdig i meget.
\par 11 Dersom I da ikke have været tro i den uretfærdige Mammon, hvem vil da betro eder den sande?
\par 12 Og dersom I ikke have været tro i det, som andre eje, hvem vil da give eder noget selv at eje?
\par 13 Ingen Tjener kan tjene to Herrer; thi han vil enten hade den ene og elske den anden, eller holde sig til den ene og ringeagte den anden; I kunne ikke tjene Gud og Mammon."
\par 14 Men alt dette hørte Farisæerne, som vare pengegerrige, og de spottede ham.
\par 15 Og han sagde til dem: "I ere de, som gøre eder selv retfærdige for Menneskene; men Gud kender eders Hjerter; thi det, som er højt iblandt Mennesker, er en Vederstyggelighed for Gud.
\par 16 Loven og Profeterne vare indtil Johannes; fra den Tid forkyndes Evangeliet om Guds Rige, og enhver trænger derind med Vold.
\par 17 Men det er lettere, at Himmelen og Jorden forgå, end at en Tøddel af Loven bortfalder.
\par 18 Hver, som skiller sig fra sin Hustru og tager en anden til Ægte, bedriver Hor; og hver, som tager til Ægte en Kvinde, der er skilt fra sin Mand, bedriver Hor.
\par 19 Men der var en rig Mand, og han klædte sig i Purpur og kostbart Linned og levede hver Dag i Fryd og Herlighed.
\par 20 Men en fattig ved Navn Lazarus var lagt ved hans Port, fuld af Sår.
\par 21 Og han attråede at mættes af det, som faldt fra den Riges Bord; men også Hundene kom og slikkede hans Sår.
\par 22 Men det skete, at den fattige døde, og at han blev henbåren af Englene i Abrahams Skød; men den rige døde også og blev begravet.
\par 23 Og da han slog sine Øjne op i Dødsriget, hvor han var i Pine, ser han Abraham langt borte og Lazarus i hans Skød.
\par 24 Og han råbte og sagde: Fader Abraham! forbarm dig over mig, og send Lazarus, for at han kan dyppe det yderste af sin Finger i Vand og læske min Tunge; thi jeg pines svarlig i denne Lue.
\par 25 Men Abraham sagde: Barn! kom i Hu, at du har fået dit gode i din Livstid, og Lazarus ligeså det onde; men nu trøstes han her, og du pines.
\par 26 Og foruden alt dette er der fæstet et stort Svælg imellem os og eder, for at de, som ville fare herfra over til eder, ikke skulle kunne det, og de ikke heller skulle fare derfra over til os.
\par 27 Men han sagde: Så beder jeg dig, Fader! at du vil sende ham til min Faders Hus
\par 28 thi jeg har fem Brødre for at han kan vidne for dem, for at ikke også de skulle komme i dette Pinested.
\par 29 Men Abraham siger til ham: De have Moses og Profeterne, lad dem høre dem!
\par 30 Men han sagde: Nej, Fader Abraham! men dersom nogen fra de døde kommer til dem, ville de omvende sig.
\par 31 Men han sagde til ham: Høre de ikke Moses og Profeterne, da lade de sig heller ikke overbevise, om nogen opstår fra de døde."

\chapter{17}

\par 1 Men han sagde til sine Disciple: "Det er umuligt, at Forargelser ikke skulde komme; men ve den, ved hvem de komme!
\par 2 Det er bedre for ham, om en Møllesten er lagt om hans Hals, og han er kastet i Havet end at han skulde forarge een af disse små.
\par 3 Vogter på eder selv! Dersom din Broder synder, da irettesæt ham; og dersom han angrer, da tilgiv ham!
\par 4 Og dersom han syv Gange om Dagen synder imod dig og syv Gange vender tilbage til dig og siger: Jeg angrer det, da skal du tilgive ham."
\par 5 Og Apostlene sagde til Herren: "Giv os mere Tro!"
\par 6 Men Herren sagde: "Dersom I havde Tro som et Sennepskorn, da kunde I sige til dette Morbærfigentræ: Ryk dig op med Rode, og plant dig i Havet, og det skulde adlyde eder.
\par 7 Men hvem af eder, som har en Tjener, der pløjer eller vogter, siger til ham, når han kommer hjem fra Marken: Gå straks hen og sæt dig til Bords?
\par 8 Vil han ikke tværtimod sige til ham: Tilbered, hvad jeg skal have til Nadver, og bind op om dig, og vart mig op, medens jeg spiser og drikker; og derefter må du spise og drikke?
\par 9 Mon han takker Tjeneren, fordi han gjorde det, som var befalet? Jeg mener det ikke.
\par 10 Således skulle også I, når I have gjort alle de Ting, som ere eder befalede, sige: Vi ere unyttige Tjenere; kun hvad vi vare skyldige at gøre, have vi gjort."
\par 11 Og det skete, medens han var på Vej til Jerusalem, at han drog midt imellem Samaria og Galilæa.
\par 12 Og da han gik ind i en Landsby, mødte der ham ti spedalske Mænd, som stode langt borte,
\par 13 og de opløftede Røsten og sagde: "Jesus, Mester, forbarm dig over os!"
\par 14 Og da han så dem, sagde han til dem: "Går hen og fremstiller eder for Præsterne!" Og det skete, medens de gik bort, bleve de rensede.
\par 15 Men en af dem vendte tilbage, da han så, at han var helbredt, og priste Gud med høj Røst.
\par 16 Og han faldt på sit Ansigt for hans Fødder og takkede ham; og denne var en Samaritan.
\par 17 Men Jesus svarede og sagde: "Bleve ikke de ti rensede? hvor ere de ni?
\par 18 Fandtes der ingen, som vendte tilbage for at give Gud Ære, uden denne fremmede?"
\par 19 Og han sagde til ham:"Stå op, gå bort; din Tro har frelst dig!"
\par 20 Men da han blev spurgt af Farisæerne om, når Guds Rige kommer, svarede han dem og sagde: "Guds Rige kommer ikke således, at man kan vise derpå.
\par 21 Ikke heller vil man sige: Se her, eller: Se der er det; thi se, Guds Rige er inden i eder."
\par 22 Men han sagde til Disciplene: "Der skal komme Dage, da I skulle attrå at se en af Menneskesønnens Dage, og I skulle ikke se den.
\par 23 Og siger man til eder: Se der, eller: Se her er han, så går ikke derhen, og løber ikke derefter!
\par 24 Thi ligesom Lynet, når det lyner fra den ene Side af Himmelen, skinner til den anden Side af Himmelen, således skal Menneskesønnen være på sin Dag.
\par 25 Men først bør han lide meget og forkastes af denne Slægt.
\par 26 Og som det skete i Noas Dage, således skal det også være i Menneskesønnens Dage:
\par 27 De spiste, drak, toge til Ægte, bleve bortgiftede indtil den Dag, da Noa gik ind i Arken, og Syndfloden kom og ødelagde alle.
\par 28 Ligeledes, som det skete i Loths Dage: De spiste, drak, købte, solgte, plantede, byggede;
\par 29 men på den Dag, da Loth gik ud af Sodoma, regnede Ild og Svovl ned fra Himmelen og ødelagde dem alle:
\par 30 på samme Måde skal det være på den Dag, da Menneskesønnen åbenbares.
\par 31 På den dag skal den, som er på Taget og har sine Ejendele i Huset, ikke stige ned for at hente dem; og ligeså skal den, som er på Marken, ikke vende tilbage igen!
\par 32 Kommer Loths Hustru i Hu!
\par 33 Den, som søger at bjærge sit Liv, skal miste det; og den, som mister det, skal beholde Livet.
\par 34 Jeg siger eder: I den Nat skulle to Mænd være på eet Leje; den ene skal tages med, og den anden skal lades tilbage.
\par 35 To Kvinder skulle male på samme Kværn; den ene skal tages med, og den anden skal lades tilbage.
\par 36 To Mænd skulle være på Marken; den ene skal tages med. og den anden skal lades tilbage."
\par 37 Og de svare og sige til ham: "Hvor, Herre?" Men han sagde til dem: "Hvor Ådselet er, der ville også Ørnene samle sig."

\chapter{18}

\par 1 Men han talte til dem en Lignelse om, at de burde altid bede og ikke blive trætte,
\par 2 og sagde: "Der var i en By en Dommer, som ikke frygtede Gud og ikke undså sig for noget Menneske.
\par 3 Og der var en Enke i den By, og hun kom til ham og sagde: Skaf mig Ret over min Modpart!
\par 4 Og længe vilde han ikke. Men derefter sagde han ved sig selv: Om jeg end ikke frygter Gud, ej heller undser mig for noget Menneske,
\par 5 så vil jeg dog, efterdi denne Enke volder mig Besvær, skaffe Hende Ret, for at hun ikke uophørligt skal komme og plage mig."
\par 6 Men Herren sagde: "Hører, hvad den uretfærdige Dommer siger!
\par 7 Skulde da Gud ikke skaffe sine udvalgte Ret, de, som råbe til ham Dag og Nat? og er han ikke langmodig, når det gælder dem?
\par 8 Jeg siger eder, han skal skaffe dem Ret i Hast. Men mon Menneskesønnen, når han kommer, vil finde Troen på Jorden?"
\par 9 Men han sagde også til nogle, som stolede på sig selv, at de vare retfærdige, og foragtede de andre, denne Lignelse:
\par 10 "Der gik to Mænd op til Helligdommen for at bede; den ene var en Farisæer, og den anden en Tolder.
\par 11 Farisæeren stod og bad ved sig selv således: Gud! Jeg takker dig, fordi jeg ikke er som de andre Mennesker, Røvere, uretfærdige, Horkarle, eller også som denne Tolder.
\par 12 Jeg faster to Gange om Ugen, jeg giver Tiende af al min indtægt.
\par 13 Men Tolderen stod langt borte og vilde end ikke opløfte Øjnene til Himmelen, men slog sig for sit Bryst og sagde: Gud, vær mig Synder nådig!
\par 14 Jeg siger eder: Denne gik retfærdiggjort hjem til sit Hus fremfor den anden; thi enhver, som Ophøjer sig selv, skal fornedres; men den, som fornedrer sig selv, skal ophøjes."
\par 15 Men de bare også de små Børn til ham, for at han skulde røre ved dem; men da Disciplene så det, truede de dem.
\par 16 Men Jesus kaldte dem til sig og sagde: "Lader de små Børn komme til mig, og formener dem det ikke; thi Guds Rige hører sådanne til.
\par 17 Sandelig, siger jeg eder, den, som ikke modtager Guds Rige ligesom et lille Barn, han skal ingenlunde komme ind i det."
\par 18 Og en af de Øverste spurgte ham og sagde: "Gode Mester! hvad skal jeg gøre, for at jeg kan arve et evigt Liv?"
\par 19 Men Jesus sagde til ham: "Hvorfor kalder du mig god? Ingen er god uden een, nemlig Gud.
\par 20 Du kender Budene: Du må ikke bedrive Hor; du må ikke slå ihjel; du må ikke stjæle; du må ikke sige falsk Vidnesbyrd; Ær din Fader og din Moder."
\par 21 Men han sagde: "Det har jeg holdt alt sammen fra min Ungdom af."
\par 22 Men da Jesus hørte det, sagde han til ham: "Endnu een Ting fattes dig: Sælg alt, hvad du har, og uddel det til fattige, så skal du have en Skat i Himmelen; og kom så og følg mig!"
\par 23 Men da han hørte dette, blev han dybt bedrøvet; thi han var såre rig.
\par 24 Men da Jesus så, at han blev dybt bedrøvet, sagde han: "Hvor vanskeligt komme de, som have Rigdom, ind i Guds Rige!
\par 25 thi det er lettere for en Kamel at gå igennem et Nåleøje end for en rig at gå ind i Guds Rige."
\par 26 Men de, som hørte det, sagde: "Hvem kan da blive frelst?"
\par 27 Men han sagde: "Hvad der er umuligt for Mennesker, det er muligt for Gud."
\par 28 Men Peter sagde: "Se, vi have forladt vort eget og fulgt dig."
\par 29 Men han sagde til dem: "Sandelig, siger jeg eder, der er ingen, som har forladt Hus eller Forældre eller Brødre eller Hustru eller Børn for Guds Riges Skyld,
\par 30 uden at han skal få det mange Fold igen i denne Tid og i den kommende Verden et evigt Liv."
\par 31 Men han tog de tolv til sig og sagde til dem: "Se, vi drage op til Jerusalem, og alle de Ting, som ere skrevne ved Profeterne, skulle fuldbyrdes på Menneskesønnen.
\par 32 Thi han skal overgives til Hedningerne og spottes, forhånes og bespyttes,
\par 33 og de skulle hudstryge og ihjelslå ham; og på den tredje Dag skal han opstå."
\par 34 Og de fattede intet deraf, og dette Ord var skjult for dem, og de forstode ikke det, som blev sagt.
\par 35 Men det skete, da han nærmede sig til Jeriko, sad der en blind ved Vejen og tiggede.
\par 36 Og da han hørte en Skare gå forbi, spurgte han, hvad dette var.
\par 37 Men de fortalte ham, at Jesus af Nazareth kom forbi.
\par 38 Og han råbte og sagde:"Jesus, du Davids Søn,forbarm dig over mig!"
\par 39 Og de, som gik foran, truede ham, for at han skulde tie; men han råbte meget stærkere: "Du Davids Søn, forbarm dig over mig!"
\par 40 Og Jesus stod stille og bød, at han skulde føres til ham; men da han kom nær til ham, spurgte han ham og sagde:
\par 41 "Hvad vil du, at jeg skal gøre for dig?" Men han sagde: "Herre! at jeg må blive seende."
\par 42 Og Jesus sagde til ham: "Bliv seende! din Tro har frelst dig."
\par 43 Og straks blev han seende, og han fulgte ham og priste Gud; og hele Folket lovpriste Gud, da de så det.

\chapter{19}

\par 1 Og han kom ind i Jeriko og drog derigennem.
\par 2 Og se, der var en Mand, som hed Zakæus; han var Overtolder, og han var rig.
\par 3 Og han søgte at få at se, hvem der var Jesus, og kunde ikke for Skaren, fordi han var lille af Vækst.
\par 4 Og han løb forud og steg op i et Morbær Figentræ, for at han kunde se ham; thi han skulde komme frem ad den Vej.
\par 5 Og da Jesus kom til Stedet, så han op og blev ham var og sagde til ham: "Zakæus! skynd dig og stig ned; thi jeg skal i Dag blive i dit Hus."
\par 6 Og han skyndte sig og steg ned og tog imod ham med Glæde.
\par 7 Og da de så det, knurrede de alle og sagde: "Han er gået ind for at tage Herberge hos en syndig Mand."
\par 8 Men Zakæus stod frem og sagde til Herren: "Se, Herre! Halvdelen af min Ejendom giver jeg de fattige; og dersom jeg har besveget nogen for noget, da giver jeg det fire Fold igen."
\par 9 Men Jesus sagde til ham: "I Dag er der blevet dette Hus Frelse til Del, efterdi også han er en Abrahams Søn;
\par 10 thi Menneskesønnen er kommen for at søge og frelse det fortabte."
\par 11 Men medens de hørte på dette, fortsatte han og sagde en Lignelse, fordi han var nær ved Jerusalem, og de mente, at Guds Rige skulde straks komme til Syne.
\par 12 Han sagde da: "En højbåren Mand drog til et fjernt Land for at få Kongemagt og vende tilbage igen.
\par 13 Men han kaldte ti af sine Tjenere og gav dem ti Pund og sagde til dem: Købslår dermed, indtil jeg kommer.
\par 14 Men hans Medborgere hadede ham og skikkede Sendebud efter ham og lod sige: Vi ville ikke, at denne skal være Konge over os.
\par 15 Og det skete, da han kom igen, efter at han havde fået Kongemagten, sagde han, at disse Tjenere, hvem han havde givet Pengene, skulde kaldes for ham, for at han kunde få at vide, hvad hver havde vundet.
\par 16 Og den første trådte frem og sagde: Herre! dit Pund har erhvervet ti Pund til.
\par 17 Og han sagde til ham: Vel, du gode Tjener! efterdi du har været tro i det mindste, skal du have Magt over ti Byer.
\par 18 Og den anden kom og sagde: Herre! dit Pund har indbragt fem Pund.
\par 19 Men han sagde også til denne: Og du skal være over fem Byer.
\par 20 Og en anden kom og sagde: Herre! se, her er dit Pund, som jeg har haft liggende i et Tørklæde.
\par 21 Thi jeg frygtede for dig, efterdi du er en streng Mand; du tager, hvad du ikke lagde, og høster, hvad du ikke såede.
\par 22 Han siger til ham: Efter din egen Mund dømmer jeg dig, du onde Tjener! Du vidste, at jeg er en streng Mand, som tager, hvad jeg ikke lagde, og høster, hvad jeg ikke såede;
\par 23 hvorfor gav du da ikke mine Penge til Vekselbordet, så jeg ved min Hjemkomst kunde have krævet dem med Rente?
\par 24 Og han sagde til dem, som stode hos: Tager Pundet fra ham, og giver det til ham, som har de ti Pund.
\par 25 Og de sagde til ham: Herre! han har ti Pund.
\par 26 Jeg siger eder, at enhver, som har, ham skal der gives; men den, som ikke har, fra ham skal endog det tages, som han har.
\par 27 Men fører disse mine Fjender hid, som ikke vilde, at jeg skulde være Konge over dem, og hugger dem ned for mine Øjne!"
\par 28 Og da han havde sagt dette, gik han foran og drog op til Jerusalem.
\par 29 Og det skete, da han nærmede sig til Bethfage og Bethania ved det Bjerg, som kaldes Oliebjerget, udsendte han to af sine Disciple og sagde:
\par 30 "Går hen til den Landsby, som ligger lige for eder. Når I komme derind, skulle I finde et Føl bundet, på hvilket der endnu aldrig har siddet noget Menneske; og løser det, og fører det hid!
\par 31 Og dersom nogen spørger eder: Hvorfor løse I det? da skulle I sige således: Herren har Brug for det."
\par 32 Men de udsendte gik hen og fandt det, ligesom han havde sagt dem.
\par 33 Men da de løste Føllet, sagde dets Herrer til dem: "Hvorfor løse I Føllet?"
\par 34 Og de sagde: "Herren har Brug for det."
\par 35 Og de førte det til Jesus, og de lagde deres Klæder på Føllet og lod Jesus sætte sig derpå.
\par 36 Og da han drog frem, bredte de deres Klæder under ham på Vejen.
\par 37 Men da han nu nærmede sig til Nedgangen fra Oliebjerget, begyndte hele Disciplenes Mængde med Glæde at prise Gud med høj Røst for alle de kraftige Gerninger, som de havde set, og de sagde:
\par 38 "Velsignet være Kongen, som kommer, i Herrens Navn! Fred i Himmelen, og Ære i det højeste!"
\par 39 Og nogle af Farisæerne i Skaren sagde til ham: "Mester! irettesæt dine Disciple!"
\par 40 Og han svarede og sagde til dem: "Jeg siger eder, at hvis disse tie, skulle Stenene råbe."
\par 41 Og da han kom nær til og så Staden, græd han over den og sagde:
\par 42 "Vidste dog også du, ja, selv på denne din Dag, hvad der tjener til din Fred! Men nu er det skjult for dine Øjne.
\par 43 Thi der skal komme Dage over dig, da dine Fjender skulle kaste en Vold op omkring dig og omringe dig og trænge dig alle Vegne fra;
\par 44 og de skulle lægge dig helt øde og dine Børn i dig og ikke lade Sten på Sten tilbage i dig, fordi du ikke kendte din Besøgelses Tid."
\par 45 Og han gik ind i Helligdommen og begyndte at uddrive dem, som solgte,
\par 46 og sagde til dem: "Der er skrevet: Og mit Hus er et Bedehus; men I have gjort det til en Røverkule."
\par 47 Og han lærte daglig i Helligdommen; men Ypperstepræsterne og de skriftkloge og de første i Folket søgte at slå ham ihjel.
\par 48 Og de fandt ikke, hvad de skulde gøre; thi hele Folket hang ved ham og hørte ham.

\chapter{20}

\par 1 Og det skete på en af de Dage, medens han lærte folket i Helligdommen og forkyndte Evangeliet, da trådte Ypperstepræsterne og de skriftkloge tillige med de Ældste hen til ham.
\par 2 Og de talte til ham og sagde: "Sig os, af hvad Magt gør du disse Ting, eller hvem er det, som har givet dig denne Magt?"
\par 3 Men han svarede og sagde til dem: "Også jeg vil spørge eder om en Ting, siger mig det:
\par 4 Johannes's Dåb, var den fra Himmelen eller fra Mennesker?"
\par 5 Men de overvejede med hverandre og sagde: "Sige vi: Fra Himmelen, da vil han sige: Hvorfor troede I ham ikke?
\par 6 Men sige vi: Fra Mennesker, da vil hele Folket stene os; thi det er overbevist om, at Johannes var en Profet."
\par 7 Og de svarede, at de vidste ikke hvorfra.
\par 8 Og Jesus sagde til dem: "Så siger ikke heller jeg eder, af hvad Magt jeg gør disse Ting."
\par 9 Men han begyndte at sige denne Lignelse til Folket: "En Mand plantede en Vingård og lejede den ud til Vingårdsmænd og drog udenlands for lange Tider.
\par 10 Og da Tiden kom, sendte han en Tjener til Vingårdsmændene, for at de skulde give ham af Vingårdens Frugt; men Vingårdsmændene sloge ham og sendte ham tomhændet bort.
\par 11 Og han sendte fremdeles en anden Tjener; men de sloge også ham og forhånede ham og sendte ham tomhændet bort.
\par 12 Og han sendte fremdeles en tredje; men også ham sårede de og kastede ham ud.
\par 13 Men Vingårdens Herre sagde: Hvad skal jeg gøre? Jeg vil sende min Søn, den elskede; de ville dog vel undse sig, for ham.
\par 14 Men da Vingårdsmændene så ham, rådsloge de indbyrdes og sagde: Det er Arvingen; lader os slå ham ihjel, for at Arven kan blive vor.
\par 15 Og de kastede ham ud af Vingården og sloge ham ihjel. Hvad vil nu Vingårdens Herre gøre ved dem?
\par 16 Han vil komme og ødelægge disse Vingårdsmænd og give Vingården til andre." Men da de hørte det, sagde de: "Det ske aldrig!"
\par 17 Men han så på dem og sagde: "Hvad er da dette, som er skrevet: Den Sten, som Bygningsmændene forkastede, den er bleven til en Hovedhjørnesten?
\par 18 Hver, som falder på denne Sten, skal slå sig sønder; men hvem den falder på, ham skal den knuse."
\par 19 Og Ypperstepræsterne og de skriftkloge søgte at lægge Hånd på ham i den samme Time, men de frygtede for Folket; thi de forstode, at han sagde denne Lignelse imod dem.
\par 20 Og de toge Vare på ham og udsendte Lurere, der anstillede sig, som om de vare retfærdige, for at fange ham i Ord, så de kunde overgive ham til Øvrigheden og Landshøvdingens Magt.
\par 21 Og de spurgte ham og sagde: "Mester! vi vide, at du taler og lærer Rettelig og ikke ser på Personer, men lærer Guds Vej i Sandhed.
\par 22 Er det os tilladt at give Kejseren Skat eller ej?"
\par 23 Men da han mærkede deres Træskhed, sagde han til dem: "Hvorfor friste I mig?
\par 24 Viser mig en Denar"; hvis Billede og Overskrift bærer den?" Men de svarede og sagde: "Kejserens."
\par 25 Men han sagde til dem: "Så giver da Kejseren, hvad Kejserens er, og Gud, hvad Guds er."
\par 26 Og de kunde ikke fange ham i Ord i Folkets Påhør, og de forundrede sig over hans Svar og tav.
\par 27 Men nogle af Saddukæerne, som nægte, at der er Opstandelse, kom til ham og spurgte ham og sagde:
\par 28 "Mester! Moses har foreskrevet os: Dersom en har en Broder, som er gift, og denne dør barnløs, da skal hans Broder tage Hustruen og oprejse sin Broder Afkom.
\par 29 Nu var der syv Brødre; og den første tog en Hustru og døde barnløs.
\par 30 Ligeså den anden.
\par 31 Og den tredje tog hende, og således også alle syv; de døde uden at efterlade Børn.
\par 32 Men til sidst døde også Hustruen.
\par 33 Hvem af dem får hende så til Hustru i Opstandelsen? thi de have alle syv haft hende til Hustru."
\par 34 Og Jesus sagde til dem: "Denne Verdens Børn tage til Ægte og bortgiftes;
\par 35 men de, som agtes værdige til at få Del i hin Verden og i Opstandelsen fra de døde, tage hverken til Ægte eller bortgiftes.
\par 36 Thi de kunne ikke mere dø; thi de ere Engle lige og ere Guds Børn, idet de ere Opstandelsens Børn.
\par 37 Men at de døde oprejses, har også Moses givet til Kende i Stedet om Tornebusken, når han kalder Herren: Abrahams Gud og Isaks Gud og Jakobs Gud.
\par 38 Men han er ikke dødes, men levendes Gud; thi for ham leve de alle."
\par 39 Men nogle af de skriftkloge svarede og sagde: "Mester! du talte vel."
\par 40 Og de turde ikke mere spørge ham om noget.
\par 41 Men han sagde til dem: "Hvorledes siger man, at Kristus er Davids Søn?
\par 42 David selv siger jo i Salmernes Bog: Herren sagde til min Herre: Sæt dig ved min højre Hånd,
\par 43 indtil jeg får lagt dine Fjender som en Skammel for dine Fødder.
\par 44 Altså kalder David ham en Herre, hvorledes er han da hans Søn?"
\par 45 Men i hele Folkets Påhør sagde han til Disciplene:
\par 46 "Vogter eder for de skriftkloge. som gerne ville gå i lange Klæder og holde af at lade sig hilse på Torvene og at have de fornemste Pladser i Synagogerne og at sidde øverst til Bords ved Måltiderne,
\par 47 de, som opæde Enkers Huse og på Skrømt bede længe; disse skulle få des hårdere Dom."

\chapter{21}

\par 1 Men idet han så op, fik han Øje på de rige, som lagde deres Gaver i Tempelblokken.
\par 2 Men han så en fattig Enke. som lagde to Skærve deri.
\par 3 Og han sagde: "Sandelig, siger jeg eder, at denne fattige Enke lagde mere i end de alle.
\par 4 Thi alle disse lagde af deres Overflod hen til Gaverne; men hun lagde af sin Fattigdom al sin Ejendom, som hun havde."
\par 5 Og da nogle sagde om Helligdommen, at den var prydet med smukke Sten og Tempelgaver. sagde han:
\par 6 "Disse Ting, som I se - der skal komme Dage, da der ikke lades Sten på Sten, som jo skal nedbrydes."
\par 7 Men de spurgte ham og sagde: "Mester! når skal dette da ske? og hvad er Tegnet på, når dette skal ske?"
\par 8 Men han sagde: "Ser til, at I ikke blive forførte; thi mange skulle på mit Navn komme og sige: Det er mig, og: Tiden er kommen nær. Går ikke efter dem!
\par 9 Men når I høre om Krige og Oprør, da forskrækkes ikke; thi dette må først ske, men Enden er der ikke straks."
\par 10 Da sagde han til dem: "Folk skal rejse sig imod Folk, og Rige imod Rige.
\par 11 Og store Jordskælv skal der være her og der og Hungersnød og Pest, og der skal ske frygtelige Ting og store Tegn fra Himmelen.
\par 12 Men forud for alt dette skulle de lægge Hånd på eder og forfølge eder og overgive eder til Synagoger og Fængsler, og I skulle føres frem for Konger og Landshøvdinger for mit Navns Skyld.
\par 13 Det skal falde ud for eder til Vidnesbyrd.
\par 14 Lægger det da på Hjerte, at I ikke forud skulle overtænke, hvorledes I skulde forsvare eder.
\par 15 Thi jeg, vil give eder Mund og Visdom, som alle eders Modstandere ikke skulle kunne modstå eller modsige.
\par 16 Men I skulle endog forrådes af Forældre og Brødre og Frænder og Venner, og de skulle slå nogle af eder ihjel.
\par 17 Og I skulle hades af alle for mit Navns Skyld.
\par 18 Og ikke et Hår på eders Hoved skal gå tabt.
\par 19 Ved eders Udholdenhed skulle I vinde eders Sjæle.
\par 20 Men når I se Jerusalem omringet af Krigshære, da forstår, at dens Ødelæggelse er kommen nær.
\par 21 Da skulle de, som ere i Judæa, fly til Bjergene; og de, som ere inde i Staden, skulle vige bort derfra; og de, som ere på Landet, skulle ikke gå ind i den.
\par 22 Thi disse ere Hævnens Dage, da alt, hvad skrevet er, skal opfyldes.
\par 23 Men ve de frugtsommelige og dem, som give Die, i de Dage; thi der skal være stor Nød på Jorden og Vrede over dette Folk.
\par 24 Og de skulle falde for Sværdets Od og føres fangne til alle Hedningerne; og Jerusalem skal nedtrædes af Hedningerne, indtil Hedningernes Tider fuldkommes.
\par 25 Og der skal ske Tegn i Sol og Måne og Stjerner, og på Jorden skulle Folkene ængstes i Fortvivlelse over Havets og Bølgernes Brusen,
\par 26 medens Mennesker forsmægte af Frygt og Forventning om de Ting, som komme over Jorderige; thi Himmelens Kræfter skulle rystes.
\par 27 Og da skulle de se Menneskesønnen komme i Sky med Kraft og megen Herlighed.
\par 28 Men når disse Ting begynde at ske, da ser op og opløfter eders Hoveder, efterdi eders Forløsning stunder til."
\par 29 Og han sagde dem en Lignelse: "Ser Figentræet og alle Træerne;
\par 30 når de alt springe ud, da se I og skønne af eder selv, at Sommeren nu er nær.
\par 31 Således skulle også I, når I se disse Ting ske, skønne, at Guds Rige er nær.
\par 32 Sandelig, siger jeg eder, at denne Slægt skal ingenlunde forgå, førend det er sket alt sammen.
\par 33 Himmelen og Jorden skulle forgå; men mine Ord skulle ingenlunde forgå.
\par 34 Men vogter eder, at eders Hjerter ikke, nogen Tid besværes af Svir og Drukkenskab og timelige Bekymringer, så hin dag kommer pludseligt over eder som en Snare.
\par 35 Thi komme skal den over alle dem, der bo på hele Jordens Flade.
\par 36 Og våger og beder til enhver Tid, for at I må blive i Stand til at undfly alle disse Ting, som skulle ske, og bestå for Menneskesønnen."
\par 37 Men han lærte om Dagene i Helligdommen, men om Nætterne gik han ud og overnattede på det Bjerg, som kaldes Oliebjerget.
\par 38 Og hele Folket kom årle til ham i Helligdommen for at høre ham.

\chapter{22}

\par 1 Men de usyrede Brøds Højtid, som kaldes Påske, nærmede sig.
\par 2 Og Ypperstepræsterne og de skriftkloge søgte, hvorledes de kunde slå ham ihjel; thi de frygtede for Folket.
\par 3 Men Satan gik ind i Judas, som kaldes Iskariot og var en af de tolv.
\par 4 Og han gik hen og talte med Ypperstepræsterne og Høvedsmændene om, hvorledes han vilde forråde ham til dem.
\par 5 Og de bleve glade og lovede at give ham Penge.
\par 6 Og han tilsagde det; og han søgte Lejlighed til at forråde ham til dem uden Opløb.
\par 7 Men de usyrede Brøds Dag kom, på hvilken man skulde slagte Påskelammet.
\par 8 Og han udsendte Peter og Johannes og sagde: "Går hen og bereder os Påskelammet, at vi kunne spise det."
\par 9 Men de sagde til ham: "Hvor vil du, at vi skulle berede det?"
\par 10 Men han sagde til dem: "Se, når I ere komne ind i Staden, skal der møde eder en Mand, som bærer en Vandkrukke; følger ham til Huset, hvor han går ind,
\par 11 og I skulle sige til Husbonden i Huset: Mesteren siger: Hvor er det Herberge, hvor jeg kan spise Påskelammet med mine Disciple?
\par 12 Og han skal vise eder en stor Sal opdækket; der skulle I berede det."
\par 13 Og de gik hen og fandt det således, som han havde sagt dem; og de beredte Påskelammet.
\par 14 Og da Timen kom, satte han sig til Bords, og Apostlene med ham.
\par 15 Og han sagde til dem: "Jeg har hjerteligt længtes efter at spise dette Påskelam med eder, førend jeg lider.
\par 16 Thi jeg siger eder, at jeg skal ingen Sinde mere spise det, førend det bliver fuldkommet i Guds Rige."
\par 17 Og han tog en Kalk, takkede og sagde: "Tager dette, og deler det imellem eder!
\par 18 Thi jeg siger eder, at fra nu af skal jeg ikke drikke af Vintræets Frugt, førend Guds Rige kommer."
\par 19 Og han tog Brød, takkede og brød det og gav dem det og sagde: "Dette er mit Legeme, det, som gives for eder; gører dette til min Ihukommelse!"
\par 20 Ligeså tog han også Kalken efter Aftensmåltidet og sagde: "Denne Kalk er den nye Pagt i mit Blod, det, som udgydes for eder.
\par 21 Men se, hans Hånd, som forråder mig, er her på Bordet hos mig.
\par 22 Thi Menneskesønnen går bort, som det er beskikket; dog ve det Menneske, ved hvem han bliver forrådt!"
\par 23 Og de begyndte at spørge hverandre indbyrdes om, hvem af dem det dog kunde være, som skulde gøre dette.
\par 24 Men der opstod også en Trætte iblandt dem om, hvem at dem der måtte synes at være den største.
\par 25 Men han sagde til dem: "Folkenes Konger herske over dem, og de, som bruge Myndighed over dem, kaldes deres Velgørere.
\par 26 I derimod ikke således; men den ældste iblandt eder blive som den yngste, og Føreren som den, der tjener.
\par 27 Thi hvem er størst: den, som sidder til Bords? eller den, som tjener? Mon ikke den, som sidder til Bords? Men jeg er iblandt eder som den, der tjener.
\par 28 Men I ere de, som have holdt ud med mig i mine Fristelser.
\par 29 Og ligesom min Fader har tildelt mig Kongedømme, tildeler jeg eder
\par 30 at skulle spise og drikke ved mit Bord i mit Rige og sidde på Troner og dømme Israels tolv Stammer."
\par 31 Men Herren sagde: "Simon, Simon! se, Satan begærede eder for at sigte eder som Hvede.
\par 32 Men jeg bad for dig, at din Tro ikke skal svigte; og når du engang omvender dig, da styrk dine Brødre!"
\par 33 Men han sagde til ham: "Herre! jeg er rede til at gå med dig både i Fængsel og i Døden."
\par 34 Men han sagde: "Peter! jeg siger dig: Hanen skal ikke gale i Dag, førend du tre Gange har nægtet, at du kender mig."
\par 35 Og han sagde til dem: "Da jeg udsendte eder uden Pung og Taske og Sko, manglede I da noget?" Og de sagde: "Intet."
\par 36 Men han sagde til dem: "Men nu, den, som har en Pung, tage den med, ligeså også en Taske; og den, som ikke har noget Sværd, sælge sin Kappe og købe et!
\par 37 Thi jeg siger eder: Det, som er skrevet, bør opfyldes på mig, dette: "Og han blev regnet iblandt Overtrædere;" thi også med mig har det en Ende."
\par 38 Men de sagde: "Herre! se, her er to Sværd." Men han sagde til dem: "Det er nok."
\par 39 Og han gik ud og gik efter sin Sædvane til Oliebjerget; men også Disciplene fulgte ham.
\par 40 Men da han kom til Stedet, sagde han til dem: "Beder om ikke at falde i Fristelse."
\par 41 Og han rev sig løs fra dem, så meget som et Stenkast, og faldt på Knæ, bad og sagde:
\par 42 "Fader, vilde du dog tage denne Kalk fra mig! dog ske ikke min Villie, men din!"
\par 43 Men en Engel fra Himmelen viste sig for ham og styrkede ham.
\par 44 Og da han var i Dødsangst, bad han heftigere; men hans Sved blev som Blodsdråber, der faldt ned på Jorden.
\par 45 Og da han stod op fra Bønnen og kom til Disciplene, fandt han dem sovende af Bedrøvelse.
\par 46 Og han sagde til dem: "Hvorfor sove I? Står op og beder, for at I ikke skulle falde i Fristelse."
\par 47 Medens han endnu talte, se, da kom der en Skare; og han, som hed Judas, en af de tolv, gik foran dem og nærmede sig til Jesus for at kysse ham.
\par 48 Men Jesus sagde til ham: "Judas! forråder du Menneskesønnen med et Kys?"
\par 49 Men da de,som vare omkring ham, så, hvad der vilde ske, sagde de: "Herre! skulle vi slå til med Sværd?"
\par 50 Og en af dem slog Ypperstepræstens Tjener og afhuggede hans højre Øre.
\par 51 Men Jesus tog til Orde og sagde: "Lad dem gøre også dette!" Og han rørte ved hans Øre og lægte ham.
\par 52 Men Jesus sagde til Ypperstepræsterne og Høvedsmændene for Helligdommen og de ældste, som vare komne til ham: "I ere gåede ud som imod en Røver med Sværd og Knipler.
\par 53 Da jeg var daglig hos eder i Helligdommen, udrakte I ikke Hænderne imod mig; men dette er eders Time og Mørkets Magt."
\par 54 Og de grebe ham og førte ham bort og bragte ham ind i Ypperstepræstens Hus; men Peter fulgte efter i Frastand.
\par 55 Og de tændte en Ild midt i Gården og satte sig sammen, og Peter sad midt iblandt dem.
\par 56 Men en Pige så ham sidde i Lysskæret og stirrede på ham og sagde: "Også denne var med ham."
\par 57 Men han fornægtede ham og sagde: "Jeg kender ham ikke. Kvinde!"
\par 58 Og lidt derefter så en anden ham og sagde: "Også du er en af dem." Men Peter sagde: "Menneske! det er jeg ikke."
\par 59 Og omtrent en Time derefter forsikrede en anden det og sagde: "I Sandhed, også denne var med ham; han er jo også en Galilæer."
\par 60 Men Peter sagde: "Menneske! jeg forstår ikke, hvad du siger." Og straks, medens han endnu talte. galede Hanen.
\par 61 Og Herren vendte sig og så på Peter; og Peter kom Herrens Ord i Hu, hvorledes han havde sagt til ham: "Førend Hanen galer i Dag, skal du fornægte mig tre Gange."
\par 62 Og han gik udenfor og græd bitterligt.
\par 63 Og de Mænd, som holdt Jesus, spottede ham og sloge ham;
\par 64 og de kastede et Klæde over ham og spurgte ham og sagde: "Profeter! hvem var det, som slog dig?"
\par 65 Og mange andre Ting sagde de spottende til ham.
\par 66 Og da det blev Dag, samlede Folkets Ældste sig og Ypperstepræsterne og de skriftkloge, og de førte ham hen for deres Råd
\par 67 og sagde: "Er du Kristus, da sig os det!" Men han sagde til dem: "Siger jeg eder det, tro I det ikke.
\par 68 Og om jeg spørger, svare I mig ikke, ej heller løslade I mig.
\par 69 Men fra nu af skal Menneskesønnen sidde ved Guds Krafts højre Hånd."
\par 70 Men de sagde alle: "Er du da Guds Søn?" Og han sagde til dem: "I sige det; jeg er det."
\par 71 Men de sagde: "Hvad have vi længere Vidnesbyrd nødig? vi have jo selv hørt det af hans Mund!"

\chapter{23}

\par 1 Og hele Mængden stod op og førte ham for Pilatus.
\par 2 Og de begyndte at anklage ham og sagde: "Vi have fundet, at denne vildleder vort Folk og forbyder at give Kejseren Skat og siger om sig selv at han er Kristus, en Konge."
\par 3 Men Pilatus spurgte ham og sagde: "Er du Jødernes Konge?" Og han svarede og sagde til ham: "Du siger det."
\par 4 Men Pilatus sagde til Ypperstepræsterne og til Skarerne: "Jeg finder ingen Skyld hos dette Menneske."
\par 5 Men de bleve ivrigere og sagde: "Han oprører Folket, idet han lærer over hele Judæa fra Galilæa af, hvor han begyndte, og lige hertil."
\par 6 Men da Pilatus hørte om Galilæa, spurgte han, om Manden var en Galilæer.
\par 7 Og da han fik at vide, at han var fra Herodes's Område, sendte han ham til Herodes, som også selv var i Jerusalem i disse Dage.
\par 8 Men da Herodes så Jesus, blev han meget glad; thi han havde i lang Tid gerne villet se ham, fordi han hørte om ham, og han håbede at se et Tegn blive gjort af ham.
\par 9 Og han gjorde ham mange Spørgsmål; men han svarede ham intet.
\par 10 Men Ypperstepræsterne og de skriftkloge stode og anklagede ham heftigt.
\par 11 Men da Herodes med sine Krigsfolk havde hånet og spottet ham, kastede han et prægtigt Klædebon om ham og sendte ham til Pilatus igen.
\par 12 På den Dag bleve Herodes og Pilatus Venner med hinanden; thi de vare før i Fjendskab med hinanden.
\par 13 Men Pilatus sammenkaldte Ypperstepræsterne og Rådsherrerne og Folket
\par 14 og sagde til dem: "I have ført dette Menneske til mig som en, der forfører Folket til Frafald; og se. jeg har forhørt ham i eders Påhør og har ingen Skyld fundet hos dette Menneske i det, som I anklage ham for,
\par 15 og Herodes ikke heller, thi han sendte ham tilbage fil os; og se, han har intet gjort som han er skyldig at dø for.
\par 16 Derfor vil jeg revse ham og lade ham løs."
\par 17 (Men han var nødt til at løslade dem een på Højtiden.)
\par 18 Men de råbte alle sammen og sagde: "Bort med ham, men løslad os Barabbas!"
\par 19 Denne var kastet i Fængsel for et Oprør, som var sket i Staden, og for Mord.
\par 20 Og atter talte Pilatus til dem, da han gerne vilde løslade Jesus.
\par 21 Men de råbte til ham og sagde: "Korsfæst, korsfæst ham!
\par 22 Men han sagde tredje Gang til dem: "Hvad ondt har da denne gjort Jeg har ingen Dødsskyld fundet hos ham; derfor vil jeg revse ham og lade ham løs."
\par 23 Men de trængte på med stærke Råb og forlangte, at han skulde korsfæstes; og deres Råb fik Overhånd.
\par 24 Og Pilatus dømte, at deres Forlangende skulde opfyldes;
\par 25 og han løslod den, de forlangte, som var kastet i Fængsel for Oprør og Mord; men Jesus overgav han til deres Villie.
\par 26 Og da de førte ham bort, toge de fat på en vis Simon fra Kyrene, som kom fra Marken, og lagde Korset på ham, for at han skulde bære det bag efter Jesus.
\par 27 Men der fulgte ham en stor Hob af Folket, og af Kvinder, som jamrede og græd over ham.
\par 28 Men Jesus vendte sig om til dem og sagde: "I Jerusalems Døtre! græder ikke over mig, men græder over eder selv og over eders Børn!
\par 29 Thi se, der kommer Dage, da man skal sige: Salige ere de ufrugtbare og de Liv, som ikke fødte, og de Bryster, som ikke gave Die.
\par 30 Da skulle de begynde at sige til Bjergene: Falder over os! og til Højene: Skjuler os!
\par 31 Thi gør man dette ved det grønne Træ, hvad vil da ske med det tørre?"
\par 32 Men der blev også to andre Misdædere førte ud for at henrettes med ham.
\par 33 Og da de vare komne til det Sted, som kaldes "Hovedskal", korsfæstede de ham der, og Misdæderne, den ene ved hans højre, og den anden ved hans venstre Side.
\par 34 Men Jesus sagde: "Fader! forlad dem; thi de vide ikke, hvad de gøre." Men de delte hans Klæder imellem sig ved Lodkastning.
\par 35 Og Folket stod og så til; men også Rådsherrerne spottede ham og sagde: "Andre har han frelst, lad ham frelse sig selv, dersom han er Guds Kristus, den udvalgte."
\par 36 Men også Stridsmændene spottede ham, idet de trådte til, rakte ham Eddike og sagde:
\par 37 "Dersom du er Jødernes Konge, da frels dig selv!"
\par 38 Men der var også sat en Overskrift over ham (skreven på Græsk og Latin og Hebraisk): "Denne er Jødernes Konge."
\par 39 Men en af de ophængte Misdædere spottede ham og sagde: "Er du ikke Kristus? Frels dig selv og os!"
\par 40 Men den anden svarede og irettesatte ham og sagde: "Frygter heller ikke du Gud, da du er under den samme Dom?
\par 41 Og vi ere det med Rette; thi vi få igen, hvad vore Gerninger have forskyldt; men denne gjorde intet uskikkeligt."
\par 42 Og han sagde: "Jesus! kom mig i Hu, når du kommer i dit Rige!"
\par 43 Og han sagde til ham: "Sandelig, siger jeg dig, i Dag skal du være med mig i Paradiset."
\par 44 Og det var nu ved den sjette Time, og der blev Mørke over hele Landet indtil den niende Time,
\par 45 idet Solen formørkedes; og Forhænget i Templet splittedes midt over.
\par 46 Og Jesus råbte med høj Røst og sagde: "Fader! i dine Hænder befaler jeg min Ånd;" og da han havde sagt det, udåndede han.
\par 47 Men da Høvedsmanden så det, som skete, gav han Gud Æren og sagde: "I Sandhed, dette Menneske var retfærdigt."
\par 48 Og alle Skarerne, som vare komne sammen til dette Skue, sloge sig for Brystet, da de så, hvad der skete, og vendte tilbage.
\par 49 Men alle hans Kyndinge stode langt borte, ligeså de Kvinder, som fulgte med ham fra Galilæa, og så dette.
\par 50 Og se, en Mand ved Navn Josef, som var Rådsherre, en god og retfærdig Mand,
\par 51 han havde ikke samtykket i deres Råd og Gerning, han var fra Arimathæa, en jødisk By, og han forventede Guds Rige;
\par 52 han gik til Pilatus og bad om Jesu Legeme.
\par 53 Og han tog det ned og svøbte det i et fint Linklæde, og han lagde ham i en Grav, som var hugget i en Klippe, hvor endnu ingen nogen Sinde var lagt.
\par 54 Og det var Beredelsesdag, og Sabbaten stundede til.
\par 55 Men Kvinderne, som vare komne med ham fra Galilæa, fulgte efter og så Graven, og hvorledes hans Legeme blev lagt.
\par 56 Og de vendte tilbage og beredte vellugtende Urter og Salver; og Sabbaten over holdt de sig stille efter Budet.

\chapter{24}

\par 1 Men på den første Dag i Ugen meget årle kom de til Graven og bragte de vellugtende Urter, som de havde beredt.
\par 2 Og de fandt Stenen bortvæltet fra Graven.
\par 3 Men da de gik derind, fandt de ikke den Herres Jesu Legeme.
\par 4 Og det skete, da de vare tvivlrådige om dette, se, da stode to Mænd for dem i strålende Klædebon.
\par 5 Men da de bleve forfærdede og bøjede deres Ansigter imod Jorden, sagde de til dem: "Hvorfor lede I efter den levende iblandt de døde?
\par 6 Han er ikke her, men han er opstanden; kommer i Hu, hvorledes han talte til eder, medens han endnu var i Galilæa, og sagde,
\par 7 at Menneskesønnen burde overgives i syndige Menneskers Hænder og korsfæstes og opstå på den tredje Dag."
\par 8 Og de kom hans Ord i Hu.
\par 9 Og de vendte tilbage fra Graven og kundgjorde alle disse Ting for de elleve og for alle de andre.
\par 10 Men det var Maria Magdalene og Johanna og Maria, Jakobs Moder, og de øvrige Kvinder med dem; de sagde Apostlene disse Ting.
\par 11 Og disse Ord kom dem for som løs Tale; og de troede dem ikke.
\par 12 Men Peter stod op og løb til Graven; og da han kiggede derind ser han Linklæderne alene liggende der, og han gik hjem i Undren over det, som var sket.
\par 13 Og se, to af dem vandrede på den samme Dag til en Landsby, som lå tresindstyve Stadier fra Jerusalem, dens Navn var Emmaus.
\par 14 Og de talte med hinanden om alle disse Ting, som vare skete.
\par 15 Og det skete, medens de samtalede og spurgte hinanden indbyrdes, da kom Jesus selv nær og vandrede med dem.
\par 16 Men deres Øjne holdtes til, så de ikke kendte ham.
\par 17 Men han sagde til dem: "Hvad er dette for Ord, som I skifte med hinanden på Vejen?" Og de standsede bedrøvede.
\par 18 Men en af dem, som hed Kleofas, svarede og sagde til ham: "Er du alene fremmed i Jerusalem og ved ikke, hvad der er sket der i disse dage?"
\par 19 Og han sagde til dem: "Hvilket?" Men de sagde til ham: "Det med Jesus af Nazareth, som var en Profet, mægtig i Gerning og Ord for Gud og alt Folket;
\par 20 og hvorledes Ypperstepræsterne og vore Rådsherrer have overgivet ham til Dødsdom og korsfæstet ham.
\par 21 Men vi håbede, at han var den, som skulde forløse Israel. Men med alt dette er det i Dag den tredje Dag, siden dette skete.
\par 22 Men også nogle af vore Kvinder have forfærdet os, idet de kom årle til Graven,
\par 23 og da de ikke fandt hans Legeme, kom de og sagde, at de havde også set et Syn af Engle, der sagde, at han lever.
\par 24 Og nogle af vore gik hen til Graven, og de fandt det således, som Kvinderne havde sagt; men ham så de ikke."
\par 25 Og han sagde til dem: "O I uforstandige og senhjertede til at tro på alt det, som Profeterne have talt!
\par 26 Burde ikke Kristus lide dette og indgå til sin Herlighed?"
\par 27 Og han begyndte fra Moses og fra alle Profeterne og udlagde dem i alle Skrifterne det, som handlede om ham.
\par 28 Og de nærmede sig til Landsbyen, som de gik til; og han lod, som han vilde gå videre.
\par 29 Og de nødte ham meget og sagde: "Bliv hos os; thi det er mod Aften, og Dagen hælder." Og han gik ind for at blive hos dem.
\par 30 Og det skete, da han havde sat sig med dem til Bords, tog han Brødet, velsignede og brød det og gav dem det.
\par 31 Da bleve deres Øjne åbnede, og de kendte ham; og han blev usynlig for dem.
\par 32 Og de sagde til hinanden: "Brændte ikke vort Hjerte i os, medens han talte til os på Vejen og oplod os Skrifterne?"
\par 33 Og de stode op i den samme Time og vendte tilbage til Jerusalem og fandt forsamlede de elleve og dem, som vare med dem, hvilke sagde:
\par 34 "Herren er virkelig opstanden og set af Simon."
\par 35 Og de fortalte, hvad der var sket på Vejen, og hvorledes han blev kendt af dem, idet han brød Brødet.
\par 36 Men medens de talte dette, stod han selv midt iblandt dem; og han siger til dem: "Fred være med eder!"
\par 37 Da forskrækkedes de og betoges af Frygt og mente, at de så en Ånd.
\par 38 Og han sagde til dem: "Hvorfor ere I forfærdede? og hvorfor opstiger der Tvivl i eders Hjerter?
\par 39 Ser mine Hænder og mine Fødder, at det er mig selv; føler på mig og ser; thi en Ånd har ikke Kød og Ben, som I se, at jeg har."
\par 40 Og da han havde sagt dette, viste han dem sine Hænder og sine Fødder.
\par 41 Men da de af Glæde herover endnu ikke kunde tro og undrede sig, sagde han til dem: "Have I her noget at spise?"
\par 42 Og de gave ham et Stykke af en stegt Fisk.
\par 43 Og han tog det og spiste det for deres Øjne.
\par 44 Men han sagde til dem: "Dette er mine Ord, som jeg talte til eder, medens jeg endnu var hos eder, at de Ting bør alle sammen opfyldes, som ere skrevne om mig i Mose Lov og Profeterne og Salmerne."
\par 45 Da oplod han deres Forstand til at forstå Skrifterne.
\par 46 Og han sagde til dem: "Således er der skrevet, at Kristus skulde lide og opstå fra de døde på den tredje Dag,
\par 47 og at der i hans Navn skal prædikes Omvendelse og Syndernes Forladelse for alle Folkeslagene og begyndes fra Jerusalem.
\par 48 I ere Vidner om disse Ting.
\par 49 Og se, jeg sender min Faders Forjættelse over eder; men I skulle blive i Staden, indtil I blive iførte Kraft fra det høje."
\par 50 Men han førte dem ud til hen imod Bethania, og han opløftede sine Hænder og velsignede dem.
\par 51 Og det skete, idet han velsignede dem, skiltes han fra dem og opløftedes til Himmelen.
\par 52 Og efter at have tilbedt ham vendte de tilbage til Jerusalem med stor Glæde.
\par 53 Og de vare stedse i Helligdommen og priste Gud.



\end{document}