\begin{document}

\title{Acts of the Apostles}


\chapter{1}

\par 1 Den første Bog skrev jeg, o Theofilus! om alt det, som Jesus begyndte både at gøre og lære,
\par 2 indtil den Dag, da han blev optagen, efter at han havde givet Apostlene, som han havde udvalgt, Befaling ved den Helligånd;
\par 3 for hvem han også, efter at han havde lidt, fremstillede sig levende ved mange Beviser, idet han viste sig for dem i fyrretyve Dage og talte om de Ting, der høre til Guds Rige.
\par 4 Og medens han var sammen med dem, bød han dem, at de ikke måtte vige fra Jerusalem, men skulde oppebie Faderens Forjættelse, "hvorom," sagde han, "I have hørt af mig.
\par 5 Thi Johannes døbte med Vand; men I skulle døbes med den Helligånd om ikke mange Dage."
\par 6 Som de nu vare forsamlede, spurgte de ham og sagde: "Herre! opretter du på denne Tid Riget igen for Israel?"
\par 7 Men han sagde til dem: "Det tilkommer ikke eder at kende Tider eller Timer, hvilke Faderen har fastsat i sin egen Magt.
\par 8 Men I skulle få Kraft, når den Helligånd kommer over eder; og I skulle være mine Vidner både i Jerusalem og i hele Judæa og Samaria og indtil Jordens Ende."
\par 9 Og da han havde sagt dette, blev han optagen, medens de så derpå, og en Sky tog ham bort fra deres Øjne.
\par 10 Og som de stirrede op imod Himmelen, medens han for bort, se, da stode to Mænd hos dem i hvide Klæder,
\par 11 og de sagde: "I galilæiske Mænd, hvorfor stå I og se op imod Himmelen? Denne Jesus, som er optagen fra eder til Himmelen, skal komme igen på samme Måde, som I have set ham fare til Himmelen."
\par 12 Da vendte de tilbage til Jerusalem fra det Bjerg, som kaldes Oliebjerget og er nær ved Jerusalem, en Sabbatsvej derfra.
\par 13 Og da de kom derind, gik de op på den Sal, hvor de plejede at opholde sig, Peter og Johannes og Jakob og Andreas, Filip og Thomas, Bartholomæus og Matthæus, Jakob, Alfæus's Søn og Simon Zelotes, og Judas, Jakobs Søn.
\par 14 Alle disse vare endrægtigt vedholdende i Bønnen tillige med nogle Kvinder og Maria, Jesu Moder, og med hans Brødre.
\par 15 Og i disse Dage stod Peter op midt iblandt Brødrene og sagde: (og der var en Skare samlet på omtrent hundrede og tyve Personer):
\par 16 "I Mænd, Brødre! det Skriftens Ord burde opfyldes, som den Helligånd forud havde talt ved Davids Mund om Judas, der blev Vejleder for dem, som grebe Jesus;
\par 17 thi han var regnet iblandt os og havde fået denne Tjenestes Lod.
\par 18 Han erhvervede sig nu en Ager for sin Uretfærdigheds Løn, og han styrtede ned og brast itu, og alle hans Indvolde væltede ud,
\par 19 hvilket også er blevet vitterligt for alle dem, som bo i Jerusalem, så at den Ager kaldes på deres eget Mål Hakeldama, det er Blodager.
\par 20 Thi der er skrevet i Salmernes Bog: "Hans Bolig blive øde, og der være ingen, som bor i den," og: "Lad en anden få hans Tilsynsgerning."
\par 21 Derfor bør en af de Mænd, som vare sammen med os i hele den Tid, da den Herre Jesus gik ind og gik ud hos os,
\par 22 lige fra Johannes's Dåb indtil den Dag, da han blev optagen fra os, blive Vidne sammen med os om hans Opstandelse."
\par 23 Og de fremstillede to, Josef, som kaldtes Barsabbas med Tilnavn Justus, og Matthias.
\par 24 Og de bade og sagde: "Du Herre! som kender alles Hjerter, vis os den ene, som du har udvalgt af disse to
\par 25 til at få denne Tjenestes og Apostelgernings Plads, som Judas forlod for at gå hen til sit eget Sted."
\par 26 Og de kastede Lod imellem dem, og Loddet faldt på Matthias. og han blev regnet sammen med de elleve Apostle.

\chapter{2}

\par 1 Og da Pinsefestens Dag kom, vare de alle endrægtigt forsamlede.
\par 2 Og der kom pludseligt fra Himmelen en Lyd som af et fremfarende vældigt Vejr og fyldte hele Huset, hvor de sade.
\par 3 Og der viste sig for dem Tunger som af Ild, der fordelte sig og satte sig på hver enkelt af dem.
\par 4 Og de bleve alle fyldte med den Helligånd, og de begyndte at tale i andre Tungemål, efter hvad Ånden gav dem at udsige.
\par 5 Men der var Jøder, bosiddende i Jerusalem, gudfrygtige Mænd af alle Folkeslag under Himmelen.
\par 6 Da denne Lyd kom, strømmede Mængden sammen og blev forvirret; thi hver enkelt hørte dem tale på hans eget Mål.
\par 7 Og de forbavsedes alle og undrede sig og sagde: "Se, ere ikke alle disse, som tale, Galilæere?
\par 8 Hvor kunne vi da høre dem tale, hver på vort eget Mål, hvor vi ere fødte,
\par 9 Parthere og Medere og Elamiter, og vi, som høre hjemme i Mesopotamien, Judæa og Happadokien. Pontus og Asien,
\par 10 i Frygien og Pamfylien, Ægypten og Libyens Egne ved Kyrene, og vi her boende Romere,
\par 11 Jøder og Proselyter, Kretere og Arabere, vi høre dem tale om Guds store Gerninger i vore Tungemål?"
\par 12 Og de forbavsedes alle og,vare tvivlrådige og sagde den ene til den anden: "Hvad kan dette være?"
\par 13 Men andre sagde spottende: "De ere fulde af sød Vin."
\par 14 Da stod Peter frem med de elleve og opløftede sin Røst og talte til dem: "I jødiske Mænd og alle I, som bo i Jerusalem! dette være eder vitterligt, og låner Øre til mine Ord!
\par 15 Thi disse ere ikke drukne, som I mene; det er jo den tredje Time på Dagen;
\par 16 men dette er, hvad der et sagt ved Profeten Joel:
\par 17 "Og det skal ske i de sidste Dage, siger Gud, da vil jeg udgyde af min Ånd over alt Kød; og eders Sønner og eders Døtre skulle profetere, og de unge iblandt eder skulle se Syner, og de gamle iblandt eder skulle have Drømme.
\par 18 Ja, endog over mine Trælle og over mine Trælkvinder vil jeg i de Dage udgyde af min Ånd, og de skulle profetere.
\par 19 Og jeg vil lade ske Undere på Himmelen oventil og Tegn på Jorden nedentil, Blod og Ild og rygende Damp.
\par 20 Solen skal forvandles til Mørke og Månen til Blod, førend Herrens store og herlige Dag kommer.
\par 21 Og det skal ske, enhver, som påkalder Herrens Navn, skal frelses."
\par 22 I israelitiske Mænd! hører disse Ord: Jesus af Nazareth, en Mand, som fra Gud var godtgjort for eder ved kraftige Gerninger og Undere og Tegn, hvilke Gud gjorde ved ham midt iblandt eder, som I jo selv vide,
\par 23 ham, som efter Guds bestemte Rådslutning og Forudviden var bleven forrådt, ham have I ved lovløses Hånd korsfæstet og ihjelslået.
\par 24 Men Gud oprejste ham, idet han gjorde Ende på Dødens Veer, eftersom det ikke var muligt, at han kunde fastholdes af den.
\par 25 Thi David siger med Henblik på ham: "Jeg havde altid Herren for mine Øjne; thi han er ved min højre Hånd, for at jeg ikke skal rokkes,
\par 26 Derfor glædede mit Hjerte sig, og min Tunge jublede, ja, også mit Kød skal bo i Håb;
\par 27 thi du skal ikke lade min Sjæl tilbage i Dødsriget, ikke heller tilstede din hellige at se Forrådnelse,
\par 28 Du har kundgjort mig Livets Veje; du skal fylde mig med Glæde for dit Åsyn."
\par 29 I Mænd Brødre! Jeg kan sige med Frimodighed til eder om Patriarken David, at han er både død og begraven, og hans Grav er hos os indtil denne Dag.
\par 30 Da han nu var en Profet og vidste, at Gud med Ed havde tilsvoret ham, at af hans Lænds Frugt skulde en sidde på hans Trone,
\par 31 talte han, forudseende, om Kristi Opstandelse, at hverken blev han ladt tilbage i Dødsriget, ej heller så hans Kød Forrådnelse.
\par 32 Denne Jesus oprejste Gud, hvorom vi alle ere Vidner.
\par 33 Efter at han nu ved Guds højre Hånd er ophøjet og af Faderen har fået den Helligånds Forjættelse, har han udgydt denne, hvilket I både se og høre.
\par 34 Thi David for ikke op til Himmelen; men han siger selv: "Herren sagde til min Herre: Sæt dig ved min højre Hånd,
\par 35 indtil jeg får lagt dine Fjender som en Skammel for dine Fødder."
\par 36 Derfor skal hele Israels Hus vide for vist, at denne Jesus, hvem I korsfæstede, har Gud gjort både til Herre og til Kristus,"
\par 37 Men da de hørte dette, stak det dem i Hjertet, og de sagde til Peter og de øvrige Apostle: "I Mænd, Brødre! hvad skulle vi gøre?"
\par 38 Men Peter sagde til dem: "Omvender eder, og hver af eder lade sig døbe på Jesu Kristi Navn til eders Synders Forladelse; og I skulle få den Helligånds Gave.
\par 39 Thi for eder er Forjættelsen og for eders Børn og for alle dem, som ere langt borte, så mange som Herren vor Gud vil tilkalde."
\par 40 Også med mange andre Ord vidnede han for dem og format dem, idet han sagde: "Lader eder frelse fra denne vanartede Slægt!"
\par 41 De, som nu toge imod hans Ord, bleve døbte; og der føjedes samme Dag omtrent tre Tusinde Sjæle til.
\par 42 Og de holdt fast ved Apostlenes Lære og Samfundet, Brødets Brydelse og Bønnerne.
\par 43 Men der kom Frygt over en hver Sjæl, og der skete mange Undere og Tegn ved Apostlene.
\par 44 Og alle de troende holdt sig sammen og havde alle Ting fælles.
\par 45 Og de solgte deres Ejendom og Gods og delte det ud iblandt alle, efter hvad enhver havde Trang til.
\par 46 Og idet de hver Dag vedholdende og endrægtigt kom i Helligdommen og brød Brødet hjemme, fik de deres Føde med Fryd og i Hjertets Enfold,
\par 47 idet de lovede Gud og havde Yndest hos hele Folket. Men Herren føjede daglig til dem nogle, som lode sig frelse.

\chapter{3}

\par 1 Men Peter og Johannes gik op i Helligdommen ved Bedetimen, den niende Time.
\par 2 Og en Mand, som var lam fra Moders Liv af, blev båren frem; ham satte de daglig ved den Dør til Helligdommen, som kaldtes den skønne, for at han kunde bede dem, som gik ind i Helligdommen, om Almisse.
\par 3 Da han så Peter og Johannes, idet de vilde gå ind i Helligdommen, bad han om at få en Almisse.
\par 4 Da så Peter tillige med Johannes fast på ham og sagde: "Se på os!"
\par 5 Og han gav Agt på dem, efterdi han ventede at få noget af dem.
\par 6 Men Peter sagde: "Sølv og Guld ejer jeg ikke, men hvad jeg har, det giver jeg dig: I Jesu Kristi Nazaræerens Navn stå op og gå!"
\par 7 Og han greb ham ved den højre Hånd og rejste ham op.
\par 8 Men straks bleve hans Ben og Ankler stærke, og han sprang op og stod og gik omkring og gik med dem ind i Helligdommen, hvor han gik omkring og sprang og lovede Gud.
\par 9 Og hele Folket så ham gå omkring og love Gud.
\par 10 Og de kendte ham som den, der havde siddet ved den skønne Port til Helligdommen for at få Almisse; og de bleve fulde af Rædsel og Forfærdelse over det, som var timedes ham.
\par 11 Medens han nu holdt fast ved Peter og Johannes, løb alt Folket rædselsslagent sammen om dem i den Søjlegang, som kaldes Salomons.
\par 12 Men da Peter så det, talte han til Folket: "I israelitiske Mænd! Hvorfor undre I eder over dette? eller hvorfor stirre I på os, som om vi af egen Magt eller Gudfrygtighed havde gjort, at han kan gå?
\par 13 Abrahams og Isaks og Jakobs Gud, vore Fædres Gud, har herliggjort sin Tjener" Jesus, hvem I prisgave og fornægtede for Pilatus, da han dømte, at han skulde løslades.
\par 14 Men I fornægtede den hellige og retfærdige og bade om, at en Morder måtte skænkes eder.
\par 15 Men Livets Fyrste sloge I ihjel, hvem Gud oprejste fra de døde, hvorom vi ere Vidner.
\par 16 Og i Troen på hans Navn har hans Navn styrket denne, hvem I se og kende, og Troen, som virkedes ved ham, har givet denne hans Førlighed i Påsyn af eder alle.
\par 17 Og nu, Brødre! jeg ved, at I handlede i Uvidenhed, ligesom også eders Rådsherrer.
\par 18 Men Gud har således fuldbyrdet, hvad han forud forkyndte ved alle Profeternes Mund, at hans Salvede skulde lide.
\par 19 Derfor fatter et andet Sind og vender om, for at eders Synder må blive udslettede, for at Vederkvægelsens Tider må komme fra Herrens Åsyn,
\par 20 og han må sende den for eder bestemte Kristus, Jesus,
\par 21 hvem Himmelen skal modtage indtil alle Tings Genoprettelses Tider, hvorom Gud har talt ved sine hellige Profeters Mund fra de ældste Dage.
\par 22 Moses sagde: "En Profet skal Herren eders Gud oprejse eder af eders Brødre ligesom mig; ham skulle I høre i alt, hvad han end vil tale til eder.
\par 23 Men det skal ske, hver Sjæl, som ikke hører den Profet, skal udryddes af Folket."
\par 24 Men også alle Profeterne, fra Samuel af og derefter, så mange som talte, have også forkyndt disse Dage.
\par 25 I ere Profeternes Sønner og Sønner af den Pagt, som Gud sluttede med vore Fædre, da han sagde til Abraham: "Og i din Sæd skulle alle Jordens Slægter velsignes."
\par 26 For eder først har Gud oprejst sin Tjener og sendt han for at velsigne eder, når enhver af eder vender om fra sin Ondskab."

\chapter{4}

\par 1 Men medens de talte til Folket, kom Præsterne og Høvedsmanden for Helligdommen og Saddukæerne over dem,
\par 2 da de harmedes over, at de lærte Folket og i Jesus forkyndte Opstandelsen fra de døde.
\par 3 Og de lagde Hånd på dem og satte dem i Forvaring til den følgende Dag; thi det var allerede Aften.
\par 4 Men mange af dem, som havde hørt Ordet, troede, og Tallet på Mændene blev omtrent fem Tusinde.
\par 5 Men det skete Dagen derefter, at deres Rådsherrer og Ældste og skriftkloge forsamlede sig i Jerusalem,
\par 6 ligeså Ypperstepræsten Annas og Kajfas og Johannes og Alexander og alle, som vare af ypperstepræstelig Slægt.
\par 7 Og de stillede dem midt iblandt sig og spurgte: "Af hvad Magt eller i hvilket Navn have I gjort dette?"
\par 8 Da sagde Peter, fyldt med den Helligånd, til dem: "I Folkets Rådsherrer og Ældste!
\par 9 Når vi i Dag forhøres angående denne Velgerning imod en vanfør Mand, om hvorved han er bleven helbredt;
\par 10 da skal det være eder alle og hele Israels Folk vitterligt, at ved Jesu Kristi Nazaræerens Navn, hvem I have korsfæstet, hvem Gud har oprejst fra de døde, ved dette Navn er det, at denne står rask her for eders Øjne,
\par 11 Han er den Sten, som blev agtet for intet af eder, I Bygningsmænd, men som er bleven til en Hovedhjørnesten.
\par 12 Og der er ikke Frelse i nogen anden; thi der er ikke noget andet Navn under Himmelen, givet iblandt Mennesker, ved hvilket vi skulle blive frelste."
\par 13 Men da de så Peters og Johannes Frimodighed og kunde mærke, at de vare ulærde Mænd og Lægfolk, forundrede de sig, og de kendte dem, at de havde været med Jesus.
\par 14 Og da de så Manden, som var helbredt, stå hos dem, havde de intet at sige derimod.
\par 15 Men de bøde dem at træde ud fra Rådet og rådførte sig med hverandre og sagde:
\par 16 "Hvad skulle vi gøre med disse Mennesker? thi at et vitterligt Tegn er sket ved dem, det er åbenbart for alle dem, som bo i Jerusalem, og vi kunne ikke nægte det.
\par 17 Men for at det ikke skal komme videre ud iblandt Folket, da lader os true dem til ikke mere at tale til noget Menneske i dette Navn."
\par 18 Og de kaldte dem ind og forbøde dem aldeles at tale eller lære i Jesu Navn.
\par 19 Men Peter og Johannes svarede og sagde til dem: "Dømmer selv. om det er ret for Gud at lyde eder mere end Gud.
\par 20 Thi vi kunne ikke lade være at tale om det, som vi have set og hørt."
\par 21 Men de truede dem end mere og løslode dem, da de ikke kunde udfinde, hvorledes de skulde straffe dem, for Folkets Skyld; thi alle priste Gud for det, som var sket.
\par 22 Thi den Mand, på hvem dette Helbredelsestegn var sket, var mere end fyrretyve År gammel.
\par 23 Da de nu vare løsladte, kom de til deres egne og fortalte dem alt, hvad Ypperstepræsterne og de Ældste havde sagt til dem.
\par 24 Men da de hørte dette, opløftede de endrægtigt Røsten til Gud og sagde: "Herre, du, som har gjort Himmelen og Jorden og Havet og alle Ting, som ere i dem,
\par 25 du, som har sagt ved din Tjener Davids Mund: "Hvorfor fnyste Hedninger, og Folkeslag oplagde forfængelige Råd?
\par 26 Jordens Konger rejste sig, og Fyrsterne samlede sig til Hobe imod Herren og imod hans Salvede."
\par 27 Ja, de have i Sandhed forsamlet sig i denne Stad imod din hellige Tjener Jesus, hvem du har salvet, både Herodes og Pontius Pilatus tillige med Hedningerne og Israels Folkestammer
\par 28 for at gøre det, som din Hånd og dit Råd forud havde bestemt skulde ske.
\par 29 Og nu, Herre! se til deres Trusler, og giv dine Tjenere at tale dit Ord med al Frimodighed,
\par 30 idet du udrækker din Hånd til Helbredelse, og der sker Tegn og Undere ved din hellige Tjeners Jesu Navn."
\par 31 Og da de havde bedt, rystedes Stedet, hvor de vare forsamlede; og de bleve alle fyldte med den Helligånd, og de talte Guds Ord med Frimodighed.
\par 32 Men de troendes Mængde havde eet Hjerte og een Sjæl; og end ikke een kaldte noget af det, han ejede, sit eget; men de havde alle Ting fælles.
\par 33 Og med stor Kraft aflagde Apostlene Vidnesbyrdet om den Herres Jesu Opstandelse, og der var stor Nåde over dem alle.
\par 34 Thi der var end ikke nogen trængende iblandt dem; thi alle de, som vare Ejere af Jordstykker eller Huse, solgte dem og bragte Salgssummerne
\par 35 og lagde dem for Apostlenes Fødder; men der blev uddelt til enhver, efter hvad han havde Trang til.
\par 36 Og Josef, som af Apostlene fik Tilnavnet Barnabas, (det er udlagt: Trøstens Søn), en Levit, født på Kypern
\par 37 som ejede en Jordlod, solgte den og bragte Pengene og lagde dem for Apostlenes Fødder.

\chapter{5}

\par 1 Men en Mand, ved Navn Ananias, tillige med Safira, hans Hustru, solgte en Ejendom
\par 2 og stak med sin Hustrus Vidende noget af Værdien til Side og bragte en Del deraf og lagde den for Apostlenes Fødder.
\par 3 Men Peter sagde: "Ananias! hvorfor har Satan fyldt dit Hjerte, så du har løjet imod den Helligånd og stukket noget til Side af Summen for Jordstykket?
\par 4 Var det ikke dit, så længe du ejede det, og stod ikke det, som det blev solgt for, til din Rådighed? Hvorfor har du dog sat dig denne Gerning for i dit Hjerte? Du har ikke løjet for Mennesker, men for Gud."
\par 5 Men da Ananias hørte disse Ord, faldt han om og udåndede. Og der kom stor Frygt over alle, som hørte det.
\par 6 Men de unge Mænd stode op og lagde ham til Rette og bare ham ud og begravede ham.
\par 7 Men det skete omtrent tre Timer derefter, da kom hans Hustru ind uden at vide, hvad der var sket.
\par 8 Da sagde Peter til hende: "Sig mig, om I solgte Jordstykket til den Pris?" Og hun sagde: "Ja, til den Pris."
\par 9 Men Peter sagde til hende: "Hvorfor ere I dog blevne enige om at friste Herrens Ånd? Se, deres Fødder, som have begravet din Mand, ere for Døren, og de skulle bære dig ud."
\par 10 Men hun faldt straks om for hans Fødder og udåndede. Men da de unge Mænd kom ind, fandt de hende død, og de bare hende ud og begravede hende hos hendes Mand.
\par 11 Og stor Frygt kom over hele Menigheden og over alle, som hørte dette.
\par 12 Men ved Apostlenes Hænder skete der mange Tegn og Undere iblandt Folket; og de vare alle endrægtigt sammen i Salomons Søjlegang.
\par 13 Men af de andre turde ingen holde sig til dem; dog priste Folket dem højt,
\par 14 og der føjedes stedse flere troende til Herren, Skarer både af Mænd og Kvinder,
\par 15 så at de endogså bare de syge ud på Gaderne og lagde dem på Senge og Løjbænke, for at når Peter kom, endog blot hans Skygge kunde overskygge nogen af dem.
\par 16 Ja, selv fra Byerne i Jerusalems Omegn strømmede Mængden sammen og bragte syge og sådanne, som vare plagede af urene Ånder, og de bleve alle helbredte.
\par 17 Men Ypperstepræsten stod op samt alle de, som holdt med ham, nemlig Saddukæernes Parti, og de bleve fulde af Nidkærhed.
\par 18 Og de lagde Hånd på Apostlene og satte dem i offentlig Forvaring.
\par 19 Men en Herrens Engel åbnede Fængselets Døre om Natten og førte dem ud og sagde:
\par 20 "Går hen og træder frem og taler i Helligdommen alle disse Livets Ord for Folket!"
\par 21 Men da de havde hørt dette, gik de ved Daggry ind i Helligdommen og lærte. Men Ypperstepræsten og de, som holdt med ham, kom og sammenkaldte Rådet og alle Israels Børns Ældste og sendte Bud til Fængselet, at de skulde føres frem.
\par 22 Men da Tjenerne kom derhen, fandt de dem ikke i Fængselet; og de kom tilbage og meldte det og sagde:
\par 23 "Fængselet fandt vi tillukket helt forsvarligt, og Vogterne stående ved Dørene; men da vi lukkede op, fandt vi ingen derinde."
\par 24 Men da Høvedsmanden for Helligdommen og Ypperstepræsterne hørte disse Ord, bleve de tvivlrådige om dem,hvad dette skulde blive til.
\par 25 Men der kom en og meldte dem: "Se, de Mænd, som I satte i Fængselet, stå i Helligdommen og lære Folket."
\par 26 Da gik Høvedsmanden hen med Tjenerne og hentede dem, dog ikke med Magt; thi de frygtede for Folket, at de skulde blive stenede.
\par 27 Men da de havde hentet dem, stillede de dem for Rådet; og Ypperstepræsten spurgte dem og sagde:
\par 28 "Vi bøde eder alvorligt, at I ikke måtte lære i dette Navn, og se, I have fyldt Jerusalem med eders Lære, og I ville bringe dette Menneskes Blod over os!"
\par 29 Men Peter og Apostlene svarede og sagde: "Man bør adlyde Gud mere end Mennesker.
\par 30 Vore Fædres Gud oprejste Jesus, hvem I hængte på et Træ og sloge ihjel.
\par 31 Ham har Gud ved sin højre Hånd ophøjet til en Fyrste og Frelser for at give Israel Omvendelse og Syndernes Forladelse.
\par 32 Og vi ere hans Vidner om disse Ting, ligesom også den Helligånd, som Gud har givet dem, der adlyde ham."
\par 33 Men da de hørte dette, skar det dem i Hjertet, og de rådsloge om at slå dem ihjel.
\par 34 Men der rejste sig i Rådet en Farisæer ved Navn Gamaliel, en Lovlærer, højt agtet af hele Folket, og han bød, at de skulde lade Mændene træde lidt udenfor.
\par 35 Og han sagde til dem: "I israelitiske Mænd! ser eder vel for, hvad I gøre med disse Mennesker.
\par 36 Thi for nogen Tid siden fremstod Theudas, som udgav sig selv for at være noget, og et Antal af omtrent fire Hundrede Mænd sluttede sig til ham; han blev slået ihjel, og alle de, som adløde ham, adsplittedes og bleve til intet.
\par 37 Efter ham fremstod Judas Galilæeren i Skatteindskrivningens Dage og fik en Flok Mennesker til at følge sig. Også han omkom, og alle de, som adløde ham, bleve adspredte.
\par 38 Og nu siger jeg eder: Holder eder fra disse Mennesker, og lader dem fare; thi dersom dette Råd eller dette Værk er af Mennesker, bliver det til intet;
\par 39 men er det af Gud, kunne I ikke gøre dem til intet. Lader eder dog ikke findes som de, der endog ville stride mod Gud!"
\par 40 Og de adløde ham; og de kaldte Apostlene frem og lode dem piske og forbøde dem at tale i Jesu Navn og løslode dem.
\par 41 Så gik de da glade bort fra Rådets Åsyn, fordi de vare blevne agtede værdige til at vanæres for hans Navns Skyld.
\par 42 Og de holdt ikke op med hver Dag at lære i Helligdommen og i Husene og at forkynde Evangeliet om Kristus Jesus.

\chapter{6}

\par 1 Men da i de Dage Disciplenes Antal forøgedes, begyndte Hellenisterne at knurre imod Hebræerne", fordi deres Enker bleve tilsidesatte ved den daglige Uddeling.
\par 2 Da sammenkaldte de tolv Disciplenes Skare og sagde: "Det huer os ikke at forlade Guds Ord for at tjene ved Bordene.
\par 3 Udser derfor, Brødre! iblandt eder syv Mænd, som have godt Vidnesbyrd og ere fulde af Ånd og Visdom; dem ville vi så indsætte til denne Gerning.
\par 4 Men vi ville holde trolig ved i Bønnen og Ordets Tjeneste."
\par 5 Og denne Tale behagede hele Mængden; og de udvalgte Stefanus. en Mand fuld af Tro og den Helligånd, og Filip og Prokorus og Nikaiior og Timon og Parmenas og Nikolaus, en Proselyt fra Antiokia;
\par 6 dem stillede de frem for Apostlene; og disse bade og lagde Hænderne på dem.
\par 7 Og Guds Ord havde Fremgang og Disciplenes Tal forøgedes meget i Jerusalem; og en stor Mængde af Præsterne adløde Troen.
\par 8 Men Stefanus, fuld af Nåde og Kraft, gjorde Undere og store Tegn iblandt Folket,
\par 9 Da stod der nogle frem af den Synagoge, som kaldes de frigivnes og Kyrenæernes og Aleksandrinernes, og nogle af dem fra Kilikien og Asien, og de tvistedes med Stefanus.
\par 10 Og de kunde ikke modstå den Visdom og den Ånd, som han talte af.
\par 11 Da fik de hemmeligt nogle Mænd til at sige: "Vi have hørt ham tale bespottelige Ord imod Moses og imod Gud."
\par 12 Og de ophidsede Folket og de Ældste og de skriftkloge, og de overfaldt ham og slæbte ham med sig og førte ham for Rådet;
\par 13 og de fremstillede falske Vidner, som sagde: "Dette Menneske holder ikke op med at tale Ord imod dette hellige Sted og imod Loven.
\par 14 Thi vi have hørt ham sige, at denne Jesus af Nazareth skal nedbryde dette Sted og forandre de Skikke, som Moses har overgivet os."
\par 15 Og alle de, som sade i Rådet, stirrede på ham, og de så hans Ansigt som en Engels Ansigt.

\chapter{7}

\par 1 Men Ypperstepræsten sagde: "Forholder dette sig således?"
\par 2 Men han sagde: "I Mænd, Brødre og Fædre, hører til! Herlighedens Gud viste sig for vor Fader Abraham, da han var i Mesopotamien, førend han tog Bolig i Karan.
\par 3 Og han sagde til ham: "Gå ud af dit Land og fra din Slægt, og kom til det Land, som jeg vil vise dig."
\par 4 Da gik han ud fra Kaldæernes Land og tog Bolig i Karan; og efter hans Faders Død lod Gud ham flytte derfra hen i dette Land, hvor I nu bo.
\par 5 Og han gav ham ikke Ejendom deri, end ikke en Fodsbred; dog forjættede han ham at give ham det til Eje og hans Sæd efter ham, endskønt han intet Barn havde.
\par 6 Men Gud talte således: "Hans Sæd skal være Udlændinge i et fremmed Land, og man skal gøre dem til Trælle og handle ilde med dem i fire Hundrede År.
\par 7 Og det Folk, for hvilket de skulle trælle, vil jeg dømme, sagde Gud; og derefter skulle de drage ud og tjene mig på dette Sted."
\par 8 Og han gav ham Omskærelsens Pagt. Og så avlede han Isak og omskar ham den ottende Dag, og Isak avlede Jakob, og Jakob de tolv Patriarker.
\par 9 Og Patriarkerne bare Avind imod Josef og solgte ham til Ægypten; og Gud var med ham,
\par 10 og han udfriede ham af alle hans Trængsler og gav ham Nåde og Visdom for Farao, Kongen i Ægypten, som satte ham til Øverste over Ægypten og over hele sit Hus.
\par 11 Men der kom Hungersnød over hele Ægypten og Kanån og en stor Trængsel, og vore Fædre fandt ikke Føde.
\par 12 Men da Jakob hørte, at der var Korn i Ægypten, sendte han vore Fædre ud første Gang.
\par 13 Og anden Gang blev Josef genkendt af sine Brødre, og Josefs Herkomst blev åbenbar for Farao.
\par 14 Men Josef sendte Bud og lod sin Fader Jakob og al sin Slægt kalde til sig, fem og halvfjerdsindstyve Sjæle.
\par 15 Og Jakob drog ned til Ægypten. Og han og vore Fædre døde,
\par 16 og de bleve flyttede til Sikem og lagte i den Grav, som Abraham havde købt for en Sum Penge af Hemors Sønner i Sikem.
\par 17 Som nu Tiden nærmede sig for den Forjættelse, Gud havde tilsagt Abraham, voksede Folket og formeredes i Ægypten,
\par 18 indtil der fremstod en anden Konge, som ikke kendte Josef.
\par 19 Han viste Træskhed imod vor Slægt og handlede ilde med vore Fædre, så de måtte sætte deres små Børn ud, for at de ikke skulde holdes i Live.
\par 20 På den Tid blev Moses født, og han var dejlig for Gud; han blev opfostret i tre Måneder i sin Faders Hus.
\par 21 Men da han var sat ud, tog Faraos Datter ham op og opfostrede ham til sin Søn.
\par 22 Og Moses blev oplært i al Ægypternes Visdom; og han var mægtig i sine Ord og Gerninger.
\par 23 Men da han blev fyrretyve År gammel, fik han i Sinde at besøge sine Brødre, Israels Børn.
\par 24 Og da han så en lide Uret, forsvarede han ham og hævnede den mishandlede, idet han slog Ægypteren ihjel.
\par 25 Men han mente, at hans Brødre forstode, at Gud gav dem Frelse ved hans Hånd; men de forstode det ikke.
\par 26 Og den næste Dag viste han sig iblandt dem under en Strid og vilde forlige dem til at holde Fred, sigende: "I Mænd! I ere Brødre, hvorfor gøre I hinanden Uret?"
\par 27 Men den, som gjorde sin Næste Uret, stødte ham fra sig og sagde: "Hvem har sat dig til Hersker og Dommer over os?
\par 28 Vil du slå mig ihjel, ligesom du i Går slog Ægypteren ihjel?"
\par 29 Da flygtede Moses for denne Tales Skyld og boede som fremmed i Midians Land, hvor han avlede to Sønner.
\par 30 Og efter fyrretyve Års Forløb viste en Engel sig for ham i Sinai Bjergs Ørken i en Tornebusk, der stod i lys Lue.
\par 31 Men da Moses så det, undrede han sig over Synet, og da han gik hen for at betragte det, lød Herrens Røst til ham:
\par 32 "Jeg er dine Fædres Gud, Abrahams og Isaks og Jakobs Gud." Da bævede Moses og turde ikke se derhen.
\par 33 Men Herren sagde til ham: "Løs Skoene af dine Fødder; thi det Sted, som du står på, er hellig Jord.
\par 34 Jeg har grant set mit Folks Mishandling i Ægypten og hørt deres Suk, og jeg er stegen ned for at udfri dem; og nu kom, lad mig sende dig til Ægypten!"
\par 35 Denne Moses, hvem de fornægtede, idet de sagde: "Hvem har sat dig til Hersker og Dommer," ham har Gud sendt til at være både Hersker og Befrier ved den Engels Hånd,som viste sig for ham i Tornebusken.
\par 36 Ham var det, som førte dem ud,idet han gjorde Undere og Tegn i Ægyptens Land og i det røde Hav og i Ørkenen i fyrretyve År.
\par 37 Han er den Moses, som sagde til Israels Børn: "En Profet skal Gud oprejse eder af eders Brødre ligesom mig."
\par 38 Han er den, som i Menigheden i Ørkenen færdedes med Engelen, der talte til ham på Sinai Bjerg, og med vore Fædre; den, som modtog levende Ord at give os;
\par 39 hvem vore Fædre ikke vilde adlyde, men de stødte ham fra sig og vendte sig med deres Hjerter til Ægypten, idet de sagde til Aron:
\par 40 "Gør os Guder, som kunne gå foran os; thi vi vide ikke, hvad der er sket med denne Moses, som førte os ud af Ægyptens Land."
\par 41 Og de gjorde en Kalv i de Dage og bragte Offer til Gudebilledet og frydede sig ved deres Hænders Gerninger.
\par 42 Men Gud vendte sig fra dem og gav dem hen til at tjene Himmelens Hær, som der er skrevet i Profeternes Bog: "Have I vel, Israels Hus! bragt mig Slagtofre og andre Ofre i fyrretyve År i Ørkenen?
\par 43 Og I bare Moloks Telt og Guden Remfans Stjerne, de Billeder, som I havde gjort for at tilbede dem; og jeg vil flytte eder bort hinsides Babylon."
\par 44 Vore Fædre i Ørkenen havde Vidnesbyrdets Tabernakel, således som han, der talte til Moses, havde befalet at gøre det efter det Forbillede, som han havde set.
\par 45 Dette toge også vore Fædre i Arv og bragte det under Josva ind i Landet, som Hedningerne besade, hvilke Gud fordrev fra vore Fædres Åsyn indtil Davids Dage,
\par 46 som vandt Nåde for Gud og bad om at måtte finde en Bolig for Jakobs Gud.
\par 47 Men Salomon byggede ham et Hus.
\par 48 Dog, den Højeste bor ikke i Huse gjorte med Hænder, som Profeten siger:
\par 49 "Himmelen er min Trone, og Jorden mine Fødders Skammel, hvad Hus ville I bygge mig? siger Herren, eller hvilket er min Hviles Sted?
\par 50 Har ikke min Hånd gjort alt dette?"
\par 51 I hårde Halse og uomskårne på Hjerter og Øren! I stå altid den Helligånd imod; som eders Fædre, således også I.
\par 52 Hvem af Profeterne er der, som eders Fædre ikke have forfulgt? og de ihjelsloge dem, som forud forkyndte om den retfærdiges Komme, hvis Forrædere og Mordere I nu ere blevne,
\par 53 I, som modtoge Loven under Engles Besørgelse og have ikke holdt den!"
\par 54 Men da de hørte dette, skar det dem i deres Hjerter, og de bede Tænderne sammen imod ham.
\par 55 Men som han var fuld af den Helligånd, stirrede han op imod Himmelen og så Guds Herlighed og Jesus stående ved Guds højre Hånd.
\par 56 Og han sagde: "Se, jeg ser Himlene åbnede og Menneskesønnen stående ved Guds højre Hånd."
\par 57 Men de råbte med høj Røst og holdt for deres Øren og stormede endrægtigt ind på ham.
\par 58 Og de stødte ham ud uden for Staden og stenede ham. Og Vidnerne lagde deres Klæder af ved en ung Mands Fødder, som hed Saulus.
\par 59 Og de stenede Stefanus, som bad og sagde: "Herre Jesus, tag imod min Ånd!"
\par 60 Men han faldt på Knæ og råbte med høj Røst: "Herre, tilregn dem ikke denne Synd!" Og som han sagde dette, sov han hen.

\chapter{8}

\par 1 Men Saulus fandt Behag i hans Mord. Og på den Dag udbrød der en stor Forfølgelse imod Menigheden i Jerusalem, og de adspredtes alle over Judæas og Samarias Egne, undtagen Apostlene.
\par 2 Men gudfrygtige Mænd begravede Stefanus og holdt en stor Veklage over ham.
\par 3 Men Saulus plagede Menigheden og gik ind i Husene og trak både Mænd og Kvinder frem og lod dem sætte i Fængsel.
\par 4 Imidlertid gik de, som bleve adspredte, omkring og forkyndte Evangeliets Ord.
\par 5 Da kom Filip til Byen Samaria og prædikede Kristus for dem.
\par 6 Og Skarerne gave endrægtigt Agt på det, som blev sagt af Filip, idet de hørte og så de Tegn, som han gjorde.
\par 7 Thi der var mange, som havde urene Ånder, og af hvem disse fore ud, råbende med høj Røst; og mange værkbrudne og lamme bleve helbredte.
\par 8 Og der blev en stor Glæde i denne By.
\par 9 Men en Mand, ved Navn Simon, var i Forvejen i Byen og drev Trolddom og satte Samarias Folk i Forbavselse, idet han udgav sig selv for at være noget stort.
\par 10 På ham gave alle Agt, små og store, og sagde: "Det er ham, som man kalder Guds store Kraft."
\par 11 Men de gave Agt på ham, fordi han i lang Tid havde sat dem i Forbavselse ved sine Trolddomskunster.
\par 12 Men da de troede Filip, som forkyndte Evangeliet om Guds Rige og Jesu Kristi Navn, lode de sig døbe, både Mænd og Kvinder.
\par 13 Men Simon troede også selv, og efter at være døbt holdt han sig nær til Filip; og da han så Tegn og store, kraftige Gerninger ske, forbavsedes han højligt.
\par 14 Men da Apostlene i Jerusalem hørte, at Samaria havde taget imod Guds Ord, sendte de Peter og Johannes til dem,
\par 15 og da disse vare komne derned, bade de for dem om, at de måtte få den Helligånd;
\par 16 thi den var endnu ikke falden på nogen af dem, men de vare blot døbte til den Herres Jesu Navn.
\par 17 Da lagde de Hænderne på dem, og de fik den Helligånd.
\par 18 Men da Simon så, at den Helligånd blev given ved Apostlenes Håndspålæggelse, bragte han dem Penge og sagde:
\par 19 "Giver også mig denne Magt, at, hvem jeg lægger Hænderne på, han må få den Helligånd."
\par 20 Men Peter sagde til ham: "Gid dit Sølv må gå til Grunde tillige med dig, fordi du mente at kunne erhverve Guds Gave for Penge.
\par 21 Du har ikke Del eller Lod i dette Ord; thi dit Hjerte er ikke ret for Gud.
\par 22 Omvend dig derfor fra denne din Ondskab og bed Herren, om dog dit Hjertes Påfund måtte forlades dig.
\par 23 Thi jeg ser, at du er stedt i Bitterheds Galde og Uretfærdigheds Lænke."
\par 24 Men Simon svarede og sagde: "Beder I for mig til Herren, for at intet af det, som I have sagt, skal komme over mig."
\par 25 Men da de havde vidnet og talt Herrens Ord, vendte de tilbage til Jerusalem, og de forkyndte Evangeliet i mange af Samaritanernes Landsbyer.
\par 26 Men en Herrens Engel talte til Filip og sagde: "Stå op og gå mod Syd på den Vej, som går ned fra Jerusalem til Gaza; den er øde."
\par 27 Og han stod op og gik. Og se, der var en Æthioper, en Hofmand, en mægtig Mand hos Kandake, Æthiopernes Dronning, som var sat over alle hendes Skatte; han var kommen til Jerusalem for at tilbede.
\par 28 Og han var på Hjemvejen og sad på sin Vogn og læste Profeten Esajas.
\par 29 Men Ånden sagde til Filip:"Gå hen og hold dig til denne Vogn!"
\par 30 Og Filip løb derhen og hørte ham læse Profeten Esajas; og han sagde: "Forstår du også det, som du læser?"
\par 31 Men han sagde: "Hvorledes skulde jeg kunne det, uden nogen vejleder mig?" Og han bad Filip stige op og sætte sig hos ham.
\par 32 Men det Stykke af Skriften, som han læste, var dette: "Som et Får blev han ført til Slagtning, og som et Lam er stumt imod den, der klipper det, således oplader han ej sin Mund.
\par 33 I Fornedrelsen blev hans Dom taget bort; hvem kan fortælle om hans Slægt, efterdi hans Liv borttages fra Jorden?"
\par 34 Men Hofmanden talte til Filip og sagde: "Jeg beder dig, om hvem siger Profeten dette? om sig selv eller om en anden?"
\par 35 Da oplod Filip sin Mund, og idet han begyndte fra dette Skriftsted, forkyndte han ham Evangeliet om Jesus.
\par 36 Men som de droge frem ad Vejen, kom de til noget Vand; og Hofmanden siger: "Se, her er Vand, hvad hindrer mig fra at blive døbt?"
\par 37 (Men Filip sagde: "Dersom du tror, af hele dit Hjerte, kan det ske." Men han svarede og sagde: "Jeg tror, at Jesus Kristus er Guds Søn.")
\par 38 Og han bød, at Vognen skulde holde, og de stege begge ned i Vandet, både Filip og Hofmanden; og han døbte ham
\par 39 Men da de stege op af Vandet, bortrykkede Herrens Ånd Filip, og Hofmanden så ham ikke mere; thi han drog sin Vej med Glæde.
\par 40 Men Filip blev funden i Asdod, og han drog omkring og forkyndte Evangeliet i alle Byerne, indtil han kom fil Kæsarea.

\chapter{9}

\par 1 Men Saulus, som endnu fnøs med Trusel og Mord imod Herrens Disciple, gik til Ypperstepræsten
\par 2 og bad ham om Breve til Damaskus til Synagogerne, for at han, om han fandt nogle, Mænd eller Kvinder, som holdt sig til Vejen, kunde føre dem bundne til Jerusalem.
\par 3 Men da han var undervejs og nærmede sig til Damaskus, omstrålede et Lys fra Himmelen ham pludseligt.
\par 4 Og han faldt til Jorden og hørte en Røst, som sagde til ham: "Saul! Saul! hvorfor forfølger du mig?"
\par 5 Og han sagde: "Hvem er du, Herre?" Men han svarede: "Jeg er Jesus, som du forfølger.
\par 6 Men stå op og gå ind i Byen, og det skal siges dig, hvad du bør gøre."
\par 7 Men de Mænd, som rejste med ham, stode målløse, da de vel hørte Røsten, men ikke så nogen.
\par 8 Og Saulus rejste sig op fra Jorden; men da han oplod sine Øjne, så han intet. Men de ledte ham ved Hånden og førte ham ind i Damaskus.
\par 9 Og han kunde i tre Dage ikke se, og han hverken spiste eller drak.
\par 10 Men der var en Discipel i Damaskus, ved Navn Ananias, og Herren sagde til ham i et Syn: "Ananias!" Og han sagde: "Se, her er jeg, Herre!"
\par 11 Og Herren sagde til ham: "Stå op, gå hen i den Gade, som kaldes den lige, og spørg i Judas's Hus efter en ved Navn Saulus fra Tarsus; thi se, han beder.
\par 12 Og han har i et Syn set en Mand, ved Navn Ananias, komme ind og lægge Hænderne på ham, for at han skulde blive seende."
\par 13 Men Ananias svarede: "Herre! jeg har hørt af mange om denne Mand, hvor meget ondt han har gjort dine hellige i Jerusalem.
\par 14 Og her har han Fuldmagt fra Ypperstepræsterne til at binde alle dem, som påkalde dit Navn."
\par 15 Men Herren sagde til ham: "Gå; thi denne er mig et udvalgt Redskab til at bære mit Navn frem både for Hedninger og Konger og Israels Børn;
\par 16 thi jeg vil, vise ham hvor meget han bør lide for mit Navns Skyld."
\par 17 Men Ananias gik hen og kom ind i Huset og lagde Hænderne på ham og sagde: "Saul, Broder! Herren har sendt mig, den Jesus, der viste sig for dig på Vejen, ad hvilken du kom, for at du skal blive seende igen og fyldes med den Helligånd."
\par 18 Og straks faldt der ligesom Skæl fra hans Øjne, og han blev seende, og han stod op og blev døbt.
\par 19 Og han fik Mad og kom til Kræfter. Men han blev nogle Dage hos Disciplene i Damaskus.
\par 20 Og straks prædikede han i Synagogerne om Jesus, at han er Guds Søn.
\par 21 Men alle, som hørte det, forbavsedes og sagde: "Er det ikke ham, som i Jerusalem forfulgte dem, der påkaldte dette Navn, og var kommen hertil for at føre dem bundne til Ypperstepræsterne?"
\par 22 Men Saulus voksede i Kraft og gendrev Jøderne, som boede i Damaskus, idet han beviste, at denne er Kristus.
\par 23 Men da nogle Dage vare forløbne, holdt Jøderne Råd om at slå ham ihjel.
\par 24 Men Saulus fik deres Efterstræbelser at vide. Og de bevogtede endog Portene både Dag og Nat, for at de kunde slå ham ihjel.
\par 25 Men hans Disciple toge ham ved Nattetid og bragte ham ud igennem Muren, idet de firede ham ned i en Kurv.
\par 26 Men da han kom til Jerusalem, forsøgte han at holde sig til Disciplene; men de frygtede alle for ham, da de ikke troede, at han var en Discipel.
\par 27 Men Barnabas tog sig af ham og førte ham til Apostlene; og han fortalte dem, hvorledes han havde set Herren på Vejen, og at han havde talt til ham, og hvorledes han i Damaskus havde vidnet frimodigt i Jesu Navn.
\par 28 Og han gik ind og gik ud med dem i Jerusalem
\par 29 og vidnede frimodigt i Herrens Navn. Og han talte og tvistedes med Hellenisterne; men de toge sig for at slå ham ihjel.
\par 30 Men da Brødrene fik dette at vide, førte de ham ned til Kæsarea og sendte ham videre til Tarsus.
\par 31 Så havde da Menigheden Fred over hele Judæa og Galilæa og Samaria, og den opbyggedes og vandrede i Herrens Frygt, og ved den Helligånds Formaning voksede den.
\par 32 Men det skete, medens Peter drog omkring alle Vegne, at han også kom ned til de hellige, som boede i Lydda.
\par 33 Der fandt han en Mand ved Navn Æneas, som havde ligget otte År til Sengs og var værkbruden.
\par 34 Og Peter sagde til ham: "Æneas! Jesus Kristus helbreder dig; stå op, og red selv din Seng!" Og han stod straks op.
\par 35 Og alle Beboere af Lydda og Saron så ham, og de omvendte sig til Herren.
\par 36 Men i Joppe var der en Discipelinde ved Navn Tabitha, hvilket udlagt betyder Hind; hun var rig på gode Gerninger og gav mange Almisser.
\par 37 Men det skete i de Dage, at hun blev syg og døde. Da toede de hende og lagde hende i Salen ovenpå.
\par 38 Men efterdi Lydda var nær ved Joppe, udsendte Disciplene, da de hørte, at Peter var der, to Mænd til ham og bade ham: "Kom uden Tøven over til os!"
\par 39 Men Peter stod op og gik med dem. Og da han kom derhen, førte de ham op i Salen ovenpå, og alle Enkerne stode hos ham, græd og viste ham alle de Kjortler og Kapper, som "Hinden" havde forarbejdet, medens hun var hos dem.
\par 40 Men Peter bød dem alle at gå ud, og han faldt på Knæ og bad; og han vendte sig til det døde Legeme og sagde: "Tabitha, stå op!" Men hun oplod sine Øjne, og da hun så Peter, satte hun sig op.
\par 41 Men han gav hende Hånden og rejste hende op, og han kaldte på de hellige og Enkerne og fremstillede hende levende for dem.
\par 42 Men det blev vitterligt over hele Joppe, og mange troede på Herren.
\par 43 Og det skete, at han blev mange Dage i Joppe hos en vis Simon, en Garver.

\chapter{10}

\par 1 Men en Mand i Kæsarea ved Navn Konelius, en Høvedsmand ved den Afdeling, som kaldes den italienske,
\par 2 en from Mand, der frygtede Gud tillige med hele sit Hus og gav Folket mange Almisser og altid bad til Gud,
\par 3 han så klarlig i et Syn omtrent ved den niende Time på Dagen en Guds Engel, som kom ind til ham og sagde til ham: "Kornelius!"
\par 4 Men han stirrede på ham og blev forfærdet og sagde: "Hvad er det, Herre?" Han sagde til ham: "Dine Bønner og dine Almisser ere opstegne til Ihukommelse for Gud.
\par 5 Og send nu nogle Mænd til Joppe, og lad hente en vis Simon med Tilnavn Peter.
\par 6 Han har Herberge hos en vis Simon, en Garver, hvis Hus er ved Havet."
\par 7 Men da Engelen, som talte til ham, var gået bort, kaldte han to af sine Husfolk og en gudfrygtig Stridsmand af dem, som stadig vare om ham.
\par 8 Og han fortalte dem det alt sammen og sendte dem til Joppe.
\par 9 Men den næste Dag, da disse vare undervejs og nærmede sig til Byen, steg Peter op på Taget for at bede ved den sjette Time.
\par 10 Og han blev meget hungrig og vilde have noget at spise; men medens de lavede det til, kom der en Henrykkelse over ham,
\par 11 og han så Himmelen åbnet og noget, der dalede ned, ligesom en stor Dug, der ved de fire Hjørner sænkedes ned på Jorden;
\par 12 og i denne var der alle Jordens firføddede Dyr og krybende Dyr og Himmelens Fugle.
\par 13 Og en Røst lød til ham: "Stå op, Peter, slagt og spis!"
\par 14 Men Peter sagde: "Ingenlunde, Herre! thi aldrig har jeg spist noget vanhelligt og urent."
\par 15 Og atter for anden Gang lød der en Røst til ham: "Hvad Gud har renset, holde du ikke for vanhelligt!"
\par 16 Og dette skete tre Gange, og straks blev dugen igen optagen til Himmelen.
\par 17 Men medens Peter var tvivlrådig med sig selv om, hvad det Syn, som han havde set, måtte betyde, se, da havde de Mænd, som vare udsendte af Kornelius, opspurgt Simons Hus og stode for Porten.
\par 18 Og de råbte og spurgte, om Simmon med Tilnavn Peter havde Herberge der.
\par 19 Men idet Peter grublede over Synet, sagde Ånden til ham: "Se, der er tre Mænd, som søge efter dig;
\par 20 men stå op, stig ned, og drag med dem uden at tvivle; thi det er mig, som har sendt dem."
\par 21 Så steg Peter ned til Mændene og sagde: "Se, jeg er den, som I søge efter; hvad er Årsagen, hvorfor I ere komne?"
\par 22 Men de sagde: "Høvedsmanden Kornelius, en retfærdig og gudfrygtig Mand, som har godt Vidnesbyrd af hele Jødernes Folk, har at en hellig Engel fået Befaling fra Gud til at lade dig hente til sit Hus og høre, hvad du har at sige."
\par 23 Da kaldte han dem ind og gav dem Herberge. Men den næste Dag stod han op og drog bort med dem, og nogle af Brødrene fra Joppe droge med ham.
\par 24 Og den følgende Dag kom de til Kæsarea. Men Kornelius ventede på dem og havde sammnenkaldt sine Frænder og nærmeste Venner.
\par 25 Men da det nu skete, at Peter kom ind, gik Kornelius ham i Møde og faldt ned for hans Fødder og tilbad ham.
\par 26 Men Peter rejste ham op og sagde: "Stå op! også jeg er selv et Menneske."
\par 27 Og under Samtale med ham gik han ind og fandt mange samlede.
\par 28 Og han sagde til dem: "I vide, hvor utilbørligt det er for en jødisk Mand at omgås med eller komne til nogen, som er af et fremmede Folk; men mig har Gud vist, at jeg ikke skulde kalde noget Menneske vanhelligt eller urent.
\par 29 Derfor kom jeg også uden Indvending, da jeg blev hentet; og jeg spørger eder da, af hvad Årsag I hentede mig?"
\par 30 Og Kornelius sagde: "For fire Dage siden fastede jeg indtil denne Time, og ved den niende Time bad jeg i mit Hus; og se, en Mand stod for mig i et strålende Klædebon,
\par 31 og han sagde: Kornelius! din Bøn er hørt, og dine Almisser ere ihukommede for Gud.
\par 32 Send derfor Bud til Joppe og lad Simon med Tilnavn Peter kalde, til dig; han har Herberge i Garveren Simons Hus ved Havet; han skal tale til dig, når han kommer.
\par 33 Derfor sendte jeg straks Bud til dig, og du gjorde vel i at komme.
\par 34 Men Peter oplod Munden og sagde: "Jeg forstår i Sandhed, at Gud ikke anser Personer;
\par 35 men i hvert Folk er den, som frygter ham og gør Retfærdighed, velkommen for ham;
\par 36 det Ord, som han sendte til Israels Børn, da han forkyndte Fred ved Jesus Kristus: han er alles Herre.
\par 37 I kende det, som er udgået over hele Judæa, idet det begyndte fra Galilæa, efter den Dåb, som Johannes prædikede,
\par 38 det om Jesus fra Nazareth, hvorledes Gud salvede ham med den Helligånd og Kraft, han, som drog omkring og gjorde vel og helbredte alle, som vare overvældede af Djævelen; thi Gud var med ham;
\par 39 og vi ere Vidner om alt det, som han har gjort både i Jødernes Land og i Jerusalem, han, som de også sloge ihjel, idet de hængte ham på et Træ.
\par 40 Ham oprejste Gud på den tredje dag og gav ham at åbenbares,
\par 41 ikke for hele Folket, men for de Vidner, som vare forud udvalgte af Gud, for os, som spiste og drak med ham, efter at han var opstanden fra de døde.
\par 42 Og han har påbudt os at prædike for Folket og at vidne, at han er den af Gud bestemte Dommer over levende og døde.
\par 43 Ham give alle Profeterne det Vidnesbyrd, at enhver, som tror på ham, skal få Syndernes Forladelse ved hans Navn."
\par 44 Medens Peter endnu talte disse Ord, faldt den Helligånd på alle dem, som hørte Ordet.
\par 45 Og de troende af Omskærelsen, så mange, som vare komne med Peter, bleve meget forbavsede over, af den Helligånds Gave var bleven udgydt også over Hedningerne;
\par 46 thi de hørte dem tale i Tunger og ophøje Gud.
\par 47 Da svarede Peter: "Mon nogen kan formene disse Vandet; så de ikke skulde døbes, de, som dog havde fået den Helligånd lige så vel som vi?"
\par 48 Og han befalede, af de skulde døbes i Jesu Kristi Navn. Da bade de ham om at blive der nogle Dage.

\chapter{11}

\par 1 Men Apostlene og de Brødre, som vare rundt om i Judæa, hørte, at også Hedningerne havde modtaget Guds Ord.
\par 2 Og da Peter kom op til Jerusalem, tvistedes de af Omskærelsen med ham og sagde:
\par 3 "Du er gået ind til uomskårne Mænd og har spist med dem."
\par 4 Men Peter begyndte og forklarede dem det i Sammenhæng og sagde:
\par 5 "Jeg var i Byen Joppe og bad; og jeg så i en Henrykkelse et Syn, noget, der dalede ned, ligesom en stor Dug, der ved de fire Hjørner sænkedes ned fra Himmelen, og den kom lige hen til mig.
\par 6 Jeg stirrede på den og betragtede den og så da Jordens firføddede Dyr og vilde Dyr og krybende Dyr og Himmelens Fugle.
\par 7 Og jeg hørte også en Røst, som sagde til mig: Stå op, Peter, slagt og spis!
\par 8 Men jeg sagde: Ingenlunde, Herre! thi aldrig kom noget vanhelligt eller urent i min Mund.
\par 9 Men en Røst svarede anden Gang fra Himmelen: Hvad Gud har renset, holde du ikke for vanhelligt!
\par 10 Og dette skete tre Gange; så blev det igen alt sammen draget op til Himmelen.
\par 11 Og se, i det samme stode tre Mænd ved det Hus, i hvilket jeg var, som vare udsendte til mig fra Kæsarea.
\par 12 Men Ånden sagde til mig, at jeg skulde gå med dem uden at gøre Forskel. Men også disse seks Brødre droge med mig, og vi gik ind i Mandens Hus.
\par 13 Og han fortalte os, hvorledes han havde set Engelen stå i hans Hus og sige: Send Bud til Joppe og lad Simon med Tilnavn Peter hente!
\par 14 Han skal tale Ord til dig, ved hvilke du og hele dit Hus skal frelses.
\par 15 Men idet jeg begyndte at tale, faldt den Helligånd på dem ligesom også på os i Begyndelsen.
\par 16 Og jeg kom Herrens Ord i Hu, hvorledes han sagde: Johannes døbte med Vand, men I skulle døbes med den Helligånd.
\par 17 Når altså Gud gav dem lige Gave med os, da de troede på den Herre Jesus Kristus,hvem var da jeg, at jeg skulde kunne hindre Gud?"
\par 18 Men da de hørte dette, bleve de rolige, og de priste Gud og sagde: "Så har Gud også givet Hedningerne Omvendelsen til Liv."
\par 19 De, som nu vare blevne adspredte på Grund af den Trængsel, som opstod i Anledning af Stefanus, vandrede om lige til Fønikien og Kypern og Antiokia, og de talte ikke Ordet til nogen uden til Jøder alene.
\par 20 Men iblandt dem var der nogle Mænd fra Kypern og Kyrene, som kom til Antiokia og talte også til Grækerne og forkyndte Evangeliet om den Herre Jesus.
\par 21 Og Herrens Hånd var med dem, og et stort Antal blev troende og omvendte sig til Herren.
\par 22 Men Rygtet om dem kom Menigheden i Jerusalem for Øre, og de sendte Barnabas ud til Antiokia.
\par 23 Da han nu kom derhen og så Guds Nåde, glædede han sig og formanede alle til med Hjertets Forsæt at blive ved Herren.
\par 24 Thi han var en god Mand og fuld af den Helligånd og Tro. Og en, stor Skare blev ført til Herren.
\par 25 Men han drog ud til Tarsus for at opsøge Saulus; og da han fandt ham, førte han ham til Antiokia.
\par 26 Og det skete, at de endog et helt År igennem færdedes sammen i Menigheden og lærte en stor Skare, og at Disciplene først i Antiokia bleve kaldte Kristne.
\par 27 Men i disse Dage kom der Profeter ned fra Jerusalem til Antiokia
\par 28 Og en af dem, ved Navn Agabus, stod op og tilkendegav ved Ånden, at der skulde komme en stor Hungersnød over hele Verden, hvilken også kom under Klaudius.
\par 29 Men Disciplene besluttede at sende, hver efter sin Evne, noget til Hjælp for Brødrene, som boede i Judæa;
\par 30 hvilket de også gjorde, og de sendte det til de Ældste ved Barnabas og Saulus's Hånd.

\chapter{12}

\par 1 På den Tid lagde Kong Herodes den for at mishandle dem,
\par 2 og Jakob, Johannes's Broder, lod han henrette med Sværd.
\par 3 Og da han så, at det behagede Jøderne, gik han videre og lod også Peter gribe. Det var de usyrede Brøds Dage.
\par 4 Og da han havde grebet ham, satte han ham i Fængsel og overgav ham til at bevogtes af fire Vagtskifter, hvert på fire Stridsmænd, da han efter Påsken vilde føre ham frem for Folket.
\par 5 Så blev da Peter bevogtet i Fængselet; men der blev af Menigheden holdt inderlig Bøn til Gud for ham.
\par 6 Men da Herodes vilde til at føre ham frem, sov Peter den Nat imellem to Stridsmænd, bunden med to Lænker, og Vagter foran Døren bevogtede Fængselet.
\par 7 Og se, en Herrens Engel stod der, og et Lys strålede i Fangerummet, og han slog Peter i Siden og vækkede ham og sagde: "Stå op i Hast!" og Lænkerne faldt ham af Hænderne.
\par 8 Og Engelen sagde til ham: "Bind op om dig, og bind dine Sandaler på!" Og han gjorde så. Og han siger til ham: "Kast din Kappe om dig, og følg mig!"
\par 9 Og han gik ud og fulgte ham, og han vidste ikke, at det, som skete ved Engelen, var virkeligt, men mente, at han så et Syn.
\par 10 Men de gik igennem den første og den anden Vagt og kom til den Jernport, som førte ud til Staden; denne åbnede sig for dem af sig selv, og de kom ud og gik en Gade frem, og straks skiltes Engelen fra ham.
\par 11 Og da Peter kom til sig selv, sagde han: "Nu ved jeg i Sandhed, at Herren udsendte sin Engel og udfriede mig af Herodes's Hånd og al det jødiske Folks Forventning."
\par 12 Og da han havde besindet sig, gik han til Marias Hus, hun, som var Moder til Johannes, med Tilnavn Markus, hvor mange vare forsamlede og bade.
\par 13 Men da han bankede på Døren til Portrummet, kom der en Pige ved Navn Rode for at høre efter.
\par 14 Og da hun kendte Peters Røst, lod hun af Glæde være at åbne Porten, men løb ind og forkyndte dem, at Peter stod uden for Porten.
\par 15 Da sagde de til hende: "Du raser." Men hun stod fast på, at det var således. Men de sagde: "Det er hans Engel."
\par 16 Men Peter blev ved at banke på, og da de lukkede op, så de ham og bleve forbavsede.
\par 17 Da vinkede han til dem med Hånden, at de skulde tie, og fortalte dem, hvorledes Herren havde ført ham ud af Fængselet, og han sagde: "Forkynder Jakob og Brødrene dette!" Og han gik ud og drog til et andet Sted.
\par 18 Men da det blev Dag, var der ikke liden Uro iblandt Stridsmændene over, hvad der var blevet af Peter.
\par 19 Men da Herodes søgte ham og ikke fandt ham, forhørte han Vagten og befalede, at de skulde henrettes. Og han drog ned fra Judæa til Kæsarea og opholdt sig der.
\par 20 Men han lå i Strid med Tyrierne og Sidonierne. Men de kom endrægtigt til ham og fik Blastus, Kongens Kammerherre, på deres Side og bade om Fred, fordi deres Land fik Næringsmidler tilførte fra Kongens Land.
\par 21 Men på en fastsat Dag iførte Herodes sig en Kongedragt og satte sig på Tronen og holdt en Tale til dem,
\par 22 Og Folket råbte til ham: "DeterGudsRøstog ikke et Menneskes."
\par 23 Men straks slog en Herrens Engel ham, fordi han ikke gav Gud Æren; og han blev fortæret af Orme og udåndede.
\par 24 Men Guds Ord havde Fremgang og udbredtes.
\par 25 Og Barnabas og Saulus vendte tilbage fra Jerusalem efter at have fuldført deres Ærinde, og de havde Johannes, med Tilnavn Markus, med sig.

\chapter{13}

\par 1 Men i Antiokia, i den derværende Menighed, var der Profeter og Lærere, nemlig Barnabas og Simeon, med Tilnavn Niger, og Kyrenæeren Lukius og Manaen, en Fosterbroder af Fjerdingsfyrsten Herodes, og Saulus.
\par 2 Medens de nu holdt Gudstjeneste og fastede, sagde den Helligånd: "Udtager mig Barnabas og Saulus til den Gerning, hvortil jeg har kaldet dem."
\par 3 Da fastede de og bade og lagde Hænderne på dem og lode dem fare.
\par 4 Da de nu således vare udsendte af den Helligånd, droge de ned til Seleukia og sejlede derfra til Kypern.
\par 5 Og da de vare komne til Salamis, forkyndte de Guds Ord i Jødernes Synagoger; men de havde også Johannes til Medhjælper.
\par 6 Og da de vare dragne igennem hele Øen indtil Pafus, fandt de en Troldkarl, en falsk Profet, en Jøde, hvis Navn var Barjesus.
\par 7 Han var hos Statholderen Sergius Paulus, en forstandig Mand. Denne kaldte Barnabas og Saulus til sig og attråede at høre Guds Ord.
\par 8 Men Elimas, Troldkarlen, (thi dette betyder hans Navn), stod dem imod og søgte at vende Statholderen bort fra Troen.
\par 9 Men Saulus, som også kaldes Paulus, blev fyldt med den Helligånd, så fast på ham og sagde:
\par 10 "O, du Djævelens Barn, fuld af al Svig og al Underfundighed, du Fjende af al Retfærdighed! vil du ikke holde op med at forvende Herrens de lige Veje?
\par 11 Og nu se, Herrens Hånd er over dig, og du skal blive blind og til en Tid ikke se Solen." Men straks faldt der Mulm og Mørke over ham, og han gik omkring og søgte efter nogen, som kunde lede ham.
\par 12 Da Statholderen så det, som var sket, troede han, slagen af Forundring over Herrens Lære.
\par 13 Paulus og de, som vare med ham, sejlede da ud fra Pafus og kom til Perge i Pamfylien. Men Johannes skiltes fra dem og vendte tilbage til Jerusalem.
\par 14 Men de droge videre fra Perge og kom til Antiokia i Pisidien og gik ind i Synagogen på Sabbatsdagen og satte sig.
\par 15 Men efter Forelæsningen af Loven og Profeterne sendte Synagogeforstanderne Bud hen til dem og lode sige: "I Mænd, Brødre! have I noget Formaningsord til Folket, da siger frem!"
\par 16 Men Paulus stod op og slog til Lyd med Hånden og sagde: "I israelitiske Mænd og I, som frygte Gud, hører til!
\par 17 Dette Folks, Israels Gud udvalgte vore Fædre og ophøjede Folket i Udlændigheden i Ægyptens Land og førte dem derfra med løftet Arm.
\par 18 Og omtrent fyrretyve År tålte han deres Færd i Ørkenen.
\par 19 Og han udryddede syv Folk i Kanåns Land og fordelte disses Land iblandt dem,
\par 20 og derefter i omtrent fire Hundrede og halvtredsindstyve År gav han dem Dommere indtil Profeten Samuel.
\par 21 Og derefter bade de om en Konge; og Gud gav dem Saul, Kis's Søn, en Mand af Benjamins Stamme, i fyrretyve År.
\par 22 Og da han havde taget ham bort, oprejste han dem David til Konge, om hvem han også vidnede, og sagde: "Jeg har fundet David, Isajs Søn, en Mand efter mit Hjerte, som skal gøre al min Villie."
\par 23 Af dennes Sæd bragte Gud efter Forjættelsen Israel en Frelser, Jesus,
\par 24 efter at Johannes forud for hans Fremtræden havde prædiket Omvendelses-Dåb for hele Israels Folk.
\par 25 Men da Johannes var ved at fuldende sit Løb, sagde han: "Hvad anse I mig for at være? Mig er det ikke; men se, der kommer en efter mig, hvis Sko jeg ikke er værdig at løse."
\par 26 I Mænd, Brødre, Sønner af Abrahams Slægt, og de iblandt eder, som frygte Gud! Til os er Ordet om denne Frelse sendt.
\par 27 Thi de, som bo i Jerusalem, og deres Rådsherrer kendte ham ikke; de dømte ham og opfyldte derved Profeternes Ord, som forelæses hver Sabbat.
\par 28 Og om end de ingen Dødsskyld fandt hos ham, bade de dog Pilatus, at han måtte blive slået ihjel.
\par 29 Men da de havde fuldbragt alle Ting, som ere skrevne om ham, toge de ham ned af Træet og lagde ham i en Grav.
\par 30 Men Gud oprejste ham fra de døde,
\par 31 og han blev set i flere Dage af dem, som vare gåede med ham op fra Galilæa til Jerusalem, dem, som nu ere hans Vidner for Folket.
\par 32 Og vi forkynde eder den Forjættelse, som blev given til Fædrene, at Gud har opfyldt denne for os, deres Børn, idet han oprejste Jesus;
\par 33 som der også er skrevet i den anden Salme: "Du er min Søn, jeg har født dig i Dag."
\par 34 Men at han har oprejst ham fra de døde, så at han ikke mere skal vende tilbage til Forrådnelse, derom har han sagt således: "Jeg vil give eder Davids hellige Forjættelser, de trofaste."
\par 35 Thi han siger også i en anden Salme: "Du skal ikke tilstede din hellige at se Forrådnelse."
\par 36 David sov jo hen, da han i sin Livstid havde tjent Guds Rådslutning, og han blev henlagt hos sine Fædre og så Forrådnelse;
\par 37 men den, som Gud oprejste, så ikke Forrådnelse.
\par 38 Så være det eder vitterligt, I Mænd, Brødre! at ved ham forkyndes der eder Syndernes Forladelse;
\par 39 og fra alt, hvorfra I ikke kunde retfærdiggøres ved Mose Lov, retfærdiggøres ved ham enhver, som tror.
\par 40 Ser nu til, at ikke det, som er sagt ved Profeterne, kommer over eder:
\par 41 "Ser, I Foragtere, og forundrer eder og bliver til intet; thi en Gerning gør jeg i eders Dage, en Gerning, som I ikke vilde tro, dersom nogen fortalte eder den."
\par 42 Men da de gik ud, bad man dem om, at disse Ord måtte blive talte til dem på den følgende Sabbat.
\par 43 Men da Forsamlingen var opløst, fulgte mange af Jøderne og af de gudfrygtige Proselyter Paulus og Barnabas, som talte til dem og formanede dem til at blive fast ved Guds Nåde.
\par 44 Men på den følgende Sabbat forsamledes næsten hele Byen for at høre Guds Ord.
\par 45 Men da Jøderne så Skarerne, bleve de fulde af Nidkærhed og modsagde det, som blev talt af Paulus, ja, både sagde imod og spottede.
\par 46 Men Paulus og Barnabas talte frit ud og sagde: "Det var nødvendigt, at Guds Ord først skulde tales til eder; men efterdi I støde det fra eder og ikke agte eder selv værdige til det evige Liv, se, så vende vi os til Hedningerne.
\par 47 Thi således har Herren befalet os: "Jeg har sat dig til Hedningers Lys, for at du skal være til Frelse lige ud til Jordens Ende."
\par 48 Men da Hedningerne hørte dette, bleve de glade og priste Herrens Ord, og de troede, så mange, som vare bestemte til evigt Liv,
\par 49 og Herrens Ord udbredtes over hele Landet.
\par 50 Men Jøderne ophidsede de fornemme gudfrygtige Kvinder og de første Mænd i Byen; og de vakte en Forfølgelse imod Paulus og Barnabas og joge dem ud fra deres Grænser.
\par 51 Men de rystede Støvet af deres Fødder imod dem og droge til Ikonium.
\par 52 Men Disciplene bleve fyldte med Glæde og den Helligånd.

\chapter{14}

\par 1 Men det skete i Ikonium, at de sammen gik ind i Jødernes Synagoge og talte således, at en stor Mængde,både af Jøder og Grækere, troede.
\par 2 Men de Jøder, som vare genstridige, ophidsede Hedningernes Sind og satte ondt i dem imod Brødrene.
\par 3 De opholdt sig nu en Tid lang der og talte med Frimodighed i Herren, som gav sin Nådes Ord Vidnesbyrd, idet han lod Tegn og Undere ske ved deres Hænder.
\par 4 Men Mængden i Byen blev uenig, og nogle holdt med Jøderne, andre med Apostlene.
\par 5 Men da der blev et Opløb, både af Hedningerne og Jøderne med samt deres Rådsherrer, for at mishandle og stene dem,
\par 6 og de fik dette at vide, flygtede de bort til Byerne i Lykaonien, Lystra og Derbe, og til det omliggende Land,
\par 7 og der forkyndte de Evangeliet.
\par 8 Og i Lystra sad der en Mand, som var kraftesløs i Fødderne, lam fra Moders Liv, og han havde aldrig gået.
\par 9 Han hørte Paulus tale; og da denne fæstede Øjet på ham og så, at han havde Tro til at frelses, sagde han med høj Røst:
\par 10 "Stå ret op på dine Fødder!" Og han sprang op og gik omkring.
\par 11 Men da Skarerne så, hvad Paulus havde gjort, opløftede de deres Røst og sagde på Lykaonisk: "Guderne ere i menneskelig Skikkelse stegne ned til os."
\par 12 Og de kaldte Barnabas Zens, men Paulus Hermes, fordi han var den, som førte Ordet.
\par 13 Men Præsten ved Zenstemplet, som var uden for Byen, bragte Tyre og Kranse hen til Portene og vilde ofre tillige med Skarerne.
\par 14 Men da Apostlene, Barnabas og Paulus, hørte dette, sønderreve de deres Klæder og sprang ind i Skaren,
\par 15 råbte og sagde: "I Mænd! hvorfor gøre I dette? Vi ere også Mennesker, lige Kår undergivne med eder, og vi forkynde eder Evangeliet om at vende om fra disse tomme Ting til den levende Gud, som har gjort Himmelen og Jorden og Havet og alt, hvad der er i dem;
\par 16 han, som i de forbigangne Tider lod alle Hedningerne vandre deres egne Veje,
\par 17 ihvorvel han ikke lod sig selv være uden Vidnesbyrd, idet han gjorde godt og gav eder Regn og frugtbare Tider fra Himmelen og mættede eders Hjerter med Føde og Glæde."
\par 18 Og det var med Nød og næppe, at de ved at sige dette afholdt Skarerne fra at ofre til dem.
\par 19 Men der kom Jøder til fra Antiokia og Ikonium, og de overtalte Skarerne og stenede Paulus og slæbte ham uden for Byen i den Tro, at han var død.
\par 20 Men da Disciplene omringede ham, stod han op og gik ind i Byen. Og den næste Dag gik han med Barnabas bort til Derbe.
\par 21 Og da de havde forkyndt Evangeliet i denne By og vundet mange Disciple, vendte de tilbage til Lystra og Ikonium og Antiokia
\par 22 og styrkede Disciplenes Sjæle og påmindede dem om at blive i Troen og om, at vi må igennem mange Trængsler indgå i Guds Rige.
\par 23 Men efter at de i hver Menighed havde udvalgt Ældste for dem, overgave de dem under Bøn og Faste til Herren, hvem de havde givet deres Tro.
\par 24 Og de droge igennem Pisidien og kom til Pamfylien.
\par 25 Og da de havde talt Ordet i Perge, droge de ned til Attalia.
\par 26 Og derfra sejlede de til Antiokia, hvorfra de vare blevne overgivne til Guds Nåde til den Gerning, som de havde fuldbragt.
\par 27 Men da de kom derhen og havde forsamlet Menigheden, forkyndte de, hvor store Ting Gud havde gjort med dem, og at han havde åbnet en Troens Dør for Hedningerne.
\par 28 Men de opholdt sig en ikke liden Tid sammen med Disciplene.

\chapter{15}

\par 1 Og der kom nogle ned fra Judæa, som lærte Brødrene: "Dersom I ikke lade eder omskære efter Mose Skik, kunne I ikke blive frelste."
\par 2 Da nu Paulus og Barnabas kom i en ikke ringe Splid og Strid med dem, så besluttede man, at Paulus og Barnabas og nogle andre af dem skulde drage op til Jerusalem til Apostlene og de Ældste i Anledning af dette Spørgsmål.
\par 3 Disse bleve da sendte af Sted af Menigheden og droge igennem Fønikien og Samaria og fortalte om Hedningernes Omvendelse, og de gjorde alle Brødrene stor Glæde.
\par 4 Men da de kom til Jerusalem, bleve de modtagne af Menigheden og Apostlene og de Ældste, og de kundgjorde, hvor store Ting Gud havde gjort med dem.
\par 5 Men nogle af Farisæernes Parti, som vare blevne troende, stode op og sagde: "Man bør omskære dem og befale dem at holde Mose Lov."
\par 6 Men Apostlene og de Ældste forsamlede sig for at overlægge denne Sag.
\par 7 Men da man havde tvistet meget herom, stod Peter op og sagde til dem: "I Mænd, Brødre! I vide, at for lang Tid siden gjorde Gud det Valg iblandt eder, at Hedningerne ved min Mund skulde høre Evangeliets Ord og tro.
\par 8 Og Gud, som kender Hjerterne, gav dem Vidnesbyrd ved at give dem den Helligånd lige så vel som os.
\par 9 Og han gjorde ingen Forskel imellem os og dem, idet han ved Troen rensede deres Hjerter.
\par 10 Hvorfor friste I da nu Gud, så I lægge et Åg på Disciplenes Nakke, som hverken vore Fædre eller vi have formået at bære?
\par 11 Men vi tro, at vi bliver frelste ved den Herres Jesu Nåde på samme Måde som også de."
\par 12 Men hele Mængden tav og de hørte Barnabas og Paulus fortælle, hvor store Tegn og Undere Gud havde gjort iblandt Hedningerne ved dem.
\par 13 Men da de havde hørt op at tale, tog Jakob til Orde og sagde: "I Mænd, Brødre, hører mig!"
\par 14 Simon har fortalt, hvorledes Gud først drog Omsorg for at tage ud af Hedninger et Folk for sit Navn.
\par 15 Og dermed stemme Profeternes Tale overens, som der er skrevet:
\par 16 "Derefter vil jeg vende tilbage og atter opbygge Davids faldne Hytte, og det nedrevne af den vil jeg atter opbygge og oprejse den igen,
\par 17 for at de øvrige af Menneskene skulle søge Herren, og alle Hedningerne, over hvilke mit Navn er nævnet, siger Herren, som gør dette."
\par 18 Gud kender fra Evighed af alle sine Gerninger.
\par 19 Derfor mener jeg, at man ikke skal besvære dem af Hedningerne, som omvende sig til Gud,
\par 20 men skrive til dem, at de skulle afholde sig fra Besmittelse med Afguderne og fra Utugt og fra det kvalte og fra Blodet.
\par 21 Thi Moses har fra gammel Tid i hver By Mennesker, som prædike ham, idet han oplæses hver Sabbat i Synagogerne."
\par 22 Da besluttede Apostelene og de Ældste tillige med hele Menigheden at udvælge nogle Mænd af deres Midte og sende dem til Antiokia tillige med Paulus og Barnabas, nemlig Judas, kaldet Barsabbas, og Silas, hvilke Mænd vare ansete iblandt Brødrene.
\par 23 Og de skreve således med dem: "Apostlene og de Ældste og Brødrene hilse Brødrene af Hedningerne i Antiokia og Syrien og Kilikien.
\par 24 Efterdi vi have hørt, at nogle, som ere komne fra os, have forvirret eder med Ord og voldt eders Sjæle Uro uden at have nogen Befaling fra os,
\par 25 så have vi endrægtigt forsamlede, besluttet at udvælge nogle Mænd og sende dem til eder med vore elskelige Barnabas og Paulus,
\par 26 Mænd, som have vovet deres Liv for vor Herres Jesu Kristi Navn.
\par 27 Vi have derfor sendt Judas og Silas, der også mundtligt skulle forkynde det samme.
\par 28 Thi det er den Helligånds Beslutning og vor, ingen videre Byrde at pålægge eder uden disse nødvendige Ting:
\par 29 At I skulle afholde eder fra Afgudsofferkød og fra Blod og fra det kvalte og fra Utugt. Når I holde eder derfra, vil det gå eder godt.
\par 30 Så lod man dem da fare, og de kom ned til Antiokia og forsamlede Mængden og overgave Brevet.
\par 31 Men da de læste det, bleve de glade over Trøsten.
\par 32 Og Judas og Silas, som også selv vare Profeter, opmuntrede Brødrene med megen Tale og styrkede dem.
\par 33 Men da de havde opholdt sig der nogen Tid, lode Brødrene dem fare med Fred til dem, som havde udsendt dem.
\par 34 (Men Silas besluttede at blive der.)
\par 35 Men Paulus og Barnabas opholdt sig i Antiokia, hvor de tillige med mange andre lærte og forkyndte Herrens Ord.
\par 36 Men efter nogen Tids Forløb sagde Paulus til Barnabas: "Lader os dog drage tilbage og besøge vore Brødre i hver By, hvor vi have forkyndt Herrens Ord, for at se, hvorledes det går dem."
\par 37 Men Barnabas vilde også tage Johannes, kaldet Markus, med.
\par 38 Men Paulus holdt for, at de ikke skulde tage den med, som havde forladt dem i Pamfylien og ikke havde fulgt med dem til Arbejdet.
\par 39 Der blev da en heftig Strid, så at de skiltes fra hverandre, og Barnabas tog Markus med sig og sejlede til Kypern.
\par 40 Men Paulus udvalgte Silas og drog ud, anbefalet af Brødrene til Herrens Nåde.
\par 41 Men han rejste omkring i Syrien og Kilikien og styrkede Menighederne.

\chapter{16}

\par 1 Og han kom til Derbe og Lystra, og se, der var der en Discipel ved Navn Timotheus, Søn af en troende Jødinde og en græsk Fader.
\par 2 Han havde godt Vidnesbyrd af Brødrene i Lystra og Ikonium.
\par 3 Ham vilde Paulus have til at drage med sig, og han tog og omskar ham for Jødernes Skyld, som vare på disse Steder; thi de vidste alle, at hans Fader var en Græker.
\par 4 Men alt som de droge igennem Byerne, overgave de dem de Bestemmelser at holde, som vare vedtagne af Apostlene og de Ældste i Jerusalem,
\par 5 Så styrkedes Menighederne i Troen og voksede i Antal hver Dag.
\par 6 Men de droge igennem Frygien og det galatiske Land, da de af den Helligånd vare blevne forhindrede i at tale Ordet i Asien.
\par 7 Da de nu kom hen imod Mysien, forsøgte de at drage til Bithynien; og Jesu Ånd tilstedte dem det ikke.
\par 8 De droge da Mysien forbi og kom ned til Troas.
\par 9 Og et Syn viste sig om Natten for Paulus: En makedonisk Mand stod der og bad ham og sagde: "Kom over til Makedonien og hjælp os!"
\par 10 Men da han havde set dette Syn, ønskede vi straks at drage over til Makedonien; thi vi sluttede, at Gud havde kaldt os derhen til at forkynde Evangeliet for dem.
\par 11 Vi sejlede da ud fra Troas og styrede lige til Samothrake og den næste Dag til Neapolis
\par 12 og derfra til Filippi, hvilken er den første By i den Del af Makedonien, en Koloni. I denne By opholdt vi os nogle Dage.
\par 13 Og på Sabbatsdagen gik vi uden for Porten ved en Flod, hvor vi mente, at der var et Bedested", og vi satte os og talte til de Kvinder, som kom sammen.
\par 14 Og en Kvinde ved Navn Lydia, en Purpurkræmmerske fra Byen Thyatira, en Kvinde, som frygtede Gud, hørte til, og hendes Hjerte oplod Herren til at give Agt på det, som blev talt af Paulus.
\par 15 Men da hun og hendes Hus var blevet døbt, bad hun og sagde: "Dersom I agte mig for at være Herren tro, da kommer ind i mit Hus og bliver der!" Og hun nødte os.
\par 16 Men det skete, da vi gik til Bedestedet, at en Pige mødte os, som havde en Spådomsånd og skaffede sine Herrer megen Vinding ved at spå.
\par 17 Hun fulgte efter Paulus og os, råbte og sagde: "Disse Mennesker ere den højeste Guds Tjenere, som forkynde eder Frelsens Vej."
\par 18 Og dette gjorde hun i mange Dage. Men Paulus blev fortrydelig derover, og han vendte sig og sagde til Ånden: "Jeg byder dig i Jesu Kristi Navn at fare ud af hende." Og den for ud i den samme Stund.
\par 19 Men da hendes Herrer så, at deres Håb om Vinding var forsvundet, grebe de Paulus og Silas og slæbte dem hen på Torvet for Øvrigheden.
\par 20 Og de førte dem til Høvedsmændene og sagde: "Disse Mennesker, som ere Jøder, forvirre aldeles vor By.
\par 21 og de forkynde Skikke, som det ikke er tilladt os, der ere Romere, at antage eller øve."
\par 22 Og Mængden rejste sig imod dem, og Høvedsmændene lode Klæderne rive af dem og befalede at piske dem.
\par 23 Og da de havde givet dem mange Slag, kastede de dem i Fængsel og befalede Fangevogteren at holde dem sikkert bevogtede.
\par 24 Da han havde fået sådan Befaling, kastede han dem i det inderste Fængsel og sluttede deres Fødder i Blokken.
\par 25 Men ved Midnat bade Paulus og Silas og sang Lovsange til Gud; og Fangerne lyttede på dem.
\par 26 Men pludseligt kom der et stort Jordskælv, så at Fængselets Grundvolde rystede, og straks åbnedes alle Dørene, og alles Lænker løstes.
\par 27 Men Fangevogteren for op at Søvne, og da han så Fængselets Døre åbne, drog han et Sværd og vilde dræbe sig selv, da han mente, at Fangerne vare flygtede.
\par 28 Men Paulus råbte med høj Røst og sagde: "Gør ikke dig selv noget ondt; thi vi ere her alle."
\par 29 Men han forlangte Lys og sprang ind og faldt skælvende ned for Paulus og Silas.
\par 30 Og han førte dem udenfor og sagde: "Herrer! hvad skal jeg gøre, for at jeg kan blive frelst?"
\par 31 Men de sagde: "Tro på den Herre Jesus Kristus, så skal du blive frelst, du og dit Hus."
\par 32 Og de talte Herrens Ord til ham og til alle dem, som vare i hans Hus.
\par 33 Og han tog dem til sig i den samme Stund om Natten og aftoede deres Sår; og han selv og alle hans blev straks døbte.
\par 34 Og han førte dem op i sit Hus og satte et Bord for dem og frydede sig over, at han med hele sit Hus var kommen til Troen på Gud.
\par 35 Men da det var blevet Dag, sendte Høvedsmændene Bysvendene hen og sagde: "Løslad de Mænd!"
\par 36 Men Fangevogteren meldte Paulus disse Ord: "Høvedsmændene have sendt Bud, at I skulle løslades; så drager nu ud og går bort med Fred!"
\par 37 Men Paulus sagde til dem: "De have ladet os piske offentligt og uden Dom, os, som dog ere romerske Mænd, og kastet os i Fængsel, og nu jage de os hemmeligt bort! Nej, lad dem selv komme og føre os ud!"
\par 38 Men Bysvendene meldte disse Ord til Høvedsmændene; og de bleve bange, da de hørte, at de vare Romere.
\par 39 Og de kom og gave dem gode Ord, og de førte dem ud og bade dem at drage bort fra Byen.
\par 40 Og de gik ud af Fængselet og gik ind til Lydia; og da de havde set Brødrene. formanede de dem og droge bort.

\chapter{17}

\par 1 Men de rejste igennem Amfipolis og Apollonia og kom til Thessalonika, hvor Jøderne havde en Synagoge.
\par 2 Og efter sin Sædvane gik Paulus ind til dem, og på tre Sabbater samtalede han med dem ud fra Skrifterne,
\par 3 idet han udlagde og forklarede, at Kristus måtte lide og opstå fra de døde, og han sagde: "Denne Jesus, som jeg forkynder eder, han er Kristus."
\par 4 Og nogle af dem bleve overbeviste og sluttede sig til Paulus og Silas, og tillige en stor Mængde at de gudfrygtige Grækere og ikke få af de fornemste Kvinder.
\par 5 Men Jøderne bleve nidkære og toge med sig nogle slette Mennesker af Lediggængerne på Torvet, rejste et Opløb og oprørte Byen; og de stormede Jasons Hus og søgte efter dem for at føre dem ud til Folket.
\par 6 Men da de ikke fandt dem, trak de Jason og nogle Brødre for Byens Øvrighed og råbte: "Disse, som have bragt hele Verden i Oprør, ere også komne hid;
\par 7 dem har Jason taget ind til sig; og alle disse handle imod Kejserens Befalinger og sige, at en anden er Konge, nemlig Jesus."
\par 8 Og de satte Skræk i Mængden og Byens Øvrighed, som hørte det.
\par 9 Og denne lod Jason og de andre stille Borgen og løslod dem.
\par 10 Men Brødrene sendte straks om Natten både Paulus og Silas bort til Berøa; og da de vare komne dertil,gik de ind i Jødernes Synagoge.
\par 11 Men disse vare mere velsindede end de i Thessalonika, de modtoge Ordet med al Redebonhed og ransagede daglig Skrifterne, om disse Ting forholdt sig således.
\par 12 Så troede da mange af dem og ikke få af de fornemme græske Kvinder og Mænd.
\par 13 Men da Jøderne i Thessalonika fik at vide, at Guds Ord blev forkyndt af Paulus også i Berøa, kom de og vakte også der Røre og Bevægelse iblandt Skarerne.
\par 14 Men da sendte Brødrene straks Paulus bort, for at han skulde drage til Havet; men både Silas og Timotheus bleve der tilbage.
\par 15 Og de, som ledsagede Paulus, førte ham lige til Athen; og efter at have fået det Bud med til Silas og Timotheus, at de snarest muligt skulde komme til ham, droge de bort.
\par 16 Medens nu Paulus ventede på dem i Athen, harmedes hans Ånd i ham, da han så, at Byen var fuld af Afgudsbilleder.
\par 17 Derfor talte han i Synagogen med Jøderne og de gudfrygtige og på Torvet hver Dag til dem, som han traf på.
\par 18 Men også nogle af de epikuræiske og stoiske Filosoffer indlode sig i Ordstrid med ham; og nogle sagde: ""Hvad vil denne Ordgyder sige?"" men andre: ""Han synes at være en Forkynder af fremmede Guddomme;"" fordi han forkyndte Evangeliet om Jesus og Opstandelsen.
\par 19 Og de toge ham og førte ham op på Areopagus og sagde: ""Kunne vi få at vide, hvad dette er for en ny Lære, som du taler om?
\par 20 Thi du bringer os nogle fremmede Ting for Øren; derfor ville vi vide, hvad dette skal betyde.""
\par 21 Men alle Atheniensere og de fremmede, som opholdt sig der, gave sig ikke Stunder til andet end at fortælle eller høre nyt.
\par 22 Men Paulus stod frem midt på Areopagus og sagde: ""I athemiensiske Mænd! jeg ser, at I i alle Måder ere omhyggelige for eders Gudsdyrkelse.
\par 23 Thi da jeg gik omkring og betragtede eders Helligdommen, fandt jeg også et Alter, på hvilket der var skrevet: ""For en ukendt Gud."" Det, som I således dyrke uden at kende det, det forkynder jeg eder.
\par 24 Gud, som har gjort Verden og alle Ting, som ere i den, han, som er Himmelens og Jordens Herre, bor ikke i Templer, gjorte med Hænder,
\par 25 han tjenes ikke heller af Menneskers Hænder som en, der trænger til noget, efterdi han selv giver alle Liv og Ånde og alle Ting.
\par 26 Og han har gjort, at hvert Folk iblandt Mennesker bor ud af eet Blod på hele Jordens Flade, idet han fastsatte bestemte Tider og Grænserne for deres Bolig,
\par 27 for at de skulde søge Gud, om de dog kunde føle sig frem og finde ham, skønt han er ikke langt fra hver enkelt af os;
\par 28 thi i ham leve og røres og ere vi, som også nogle af eders Digtere have sagt: Vi ere jo også hans Slægt.
\par 29 Efterdi vi da ere Guds Slægt, bør vi ikke mene, at Guddommen er lig Guld eller Sølv eller Sten, formet ved Menneskers Kunst og Opfindsomhed.
\par 30 Efter at Gud altså har båret over med disse Vankundighedens Tider, byder han nu Menneskene at de alle og alle Vegne skulle omvende sig.
\par 31 Thi han har fastsat en Dag, på hvilken han vil dømme Jorderige med Retfærdighed ved en Mand, som han har beskikket dertil, og dette har han bevist for alle ved at oprejse ham fra de døde."
\par 32 Men da de hørte om de dødes Opstandelse, spottede nogle; men andre sagde: "Ville atter høre dig om dette."
\par 33 Således gik Paulus ud fra dem.
\par 34 Men nogle Mænd holdt sig til ham og troede; iblandt hvilke også var Areopagiten Dionysius og en Kvinde ved Navn Damaris og andre med dem.

\chapter{18}

\par 1 Derefter forlod Paulus Athen og kom til Korinth
\par 2 Der traf han en Jøde ved Navn Akvila, født i Pontus, som nylig var kommen fra Italien med sin Hustru Priskilla, fordi Klaudius havde befalet, at alle Jøderne skulde forlade Rom. Til disse gik han.
\par 3 Og efterdi han øvede det samme Håndværk, blev han hos dem og arbejdede; thi de vare Teltmagere af Håndværk.
\par 4 Men han holdt Samtaler i Synagogen på hver Sabbat og overbeviste Jøder og Grækere.
\par 5 Men da Silas og Timotheus kom ned fra Makedonien, var Paulus helt optagen af at tale og vidnede for Jøderne, at Jesus er Kristus.
\par 6 Men da de stode imod og spottede, rystede han Støvet af sine Klæder og sagde til dem: "Eders Blod komme over eders Hoved! Jeg er ren; herefter vil jeg gå til Hedningerne."
\par 7 Og han gik bort derfra og gik ind til en Mand ved Navn Justus, som frygtede Gud, og hvis Hus lå ved Siden af Synagogen.
\par 8 Men Synagogeforstanderen Krispus troede på Herren tillige med hele sit Hus, og mange af Korinthierne, som hørte til, troede og bleve døbte.
\par 9 Men Herren sagde til Paulus i et Syn om Natten: "Frygt ikke, men tal og ti ikke,
\par 10 eftersom jeg er med dig, og ingen skal lægge Hånd på dig for at gøre dig noget ondt; thi jeg har et talrigt Folk i denne By."
\par 11 Og han slog sig ned der et År og seks Måneder og lærte Guds Ord iblandt dem.
\par 12 Men medens Gallio var Statholder i Akaja, stode Jøderne endrægtigt op imod Paulus og førte ham for Domstolen og sagde:
\par 13 "Denne overtaler Folk til en Gudsdyrkelse imod Loven."
\par 14 Og da Paulus vilde oplade Munden, sagde Gallio til Jøderne: " Dersom det var nogen Uret eller Misgerning, I Jøder! vilde jeg, som billigt var, tålmodigt høre på eder.
\par 15 Men er det Stridsspørgsmål om Lære og Navne og om den Lov, som I have, da ser selv dertil; thi jeg vil ikke være Dommer over disse Ting."
\par 16 Og han drev dem bort fra Domstolen.
\par 17 Men alle grebe Synagogeforstanderen Sosthenes og sloge ham lige for Domstolen; og Gallio brød sig ikke om noget af dette.
\par 18 Men Paulus blev der endnu i mange dage; derefter tog han Afsked med Brødrene og sejlede bort til Syrien og med ham Priskilla og Akvila, efter at han havde ladet sit Hår klippe af i Kenkreæ; thi han havde et Løfte på sig.
\par 19 Men de kom til Efesus; og der lod han hine blive tilbage; men han selv gik ind i Synagogen og samtalede med Jøderne.
\par 20 Men da de bade ham om at blive i længere Tid, samtykkede han ikke;
\par 21 men han tog Afsked og sagde: " (Jeg må endelig holde denne forestående Højtid i Jerusalem; men) jeg vil atter vende tilbage til eder, om Gud vil." Og han sejlede ud fra Efesus
\par 22 og landede i Kæsarea, drog op og hilste på Menigheden og drog så ned til Antiokia.
\par 23 Og da han havde opholdt sig der nogen Tid, drog han bort og rejste fra Sted til Sted igennem det galatiske Land og Frygien og styrkede alle Disciplene.
\par 24 Men en Jøde ved Navn Apollos, født i Aleksandria, en veltalende Mand, som var stærk i Skrifterne, kom til Efesus.
\par 25 Denne var undervist om Herrens Vej, og brændende i Ånden talte og lærte han grundigt om Jesus, skønt han kun kendte Johannes's Dåb.
\par 26 Og han begyndte at tale frimodigt i Synagogen. Men da Priskilla og Akvila hørte ham,toge de ham til sig og udlagde ham Guds Vej nøjere.
\par 27 Men da han vilde rejse videre til Akaja, skrev Brødrene til Disciplene og opmuntrede dem til at tage imod ham. Da han var kommen derhen, var han ved Guds Nåde de troende til megen Nytte;
\par 28 thi han gendrev Jøderne offentligt med stor Kraft og beviste ved Skrifterne, at Jesus er Kristus.

\chapter{19}

\par 1 Men det skete, medens Apollos var i Korinth, at Paulus efter at være dragen igennem de højereliggende Landsdele kom ned til Efesus
\par 2 og fandt nogle Disciple, og han sagde til dem: "Fik I den Helligånd, da I bleve troende?" Men de sagde til ham: "Vi have ikke engang hørt, at der er en Helligånd."
\par 3 Og han sagde: "Hvortil bleve I da døbte?" Men de sagde: "Til Johannes's Dåb."
\par 4 Da sagde Paulus: "Johannes døbte med Omvendelses-Dåb, idet han sagde til Folket, at de skulde tro på den, som kom efter ham, det er på Jesus."
\par 5 Men da de hørte dette, lode de sig døbe til den Herres Jesu Navn.
\par 6 Og da Paulus lagde Hænderne på dem, kom den Helligånd over dem, og de talte i Tunger og profeterede.
\par 7 Men de vare i det hele omtrent tolv Mand.
\par 8 Og han gik ind i Synagogen og vidnede frimodigt i tre Måneder, idet han holdt Samtaler og overbeviste om det, som hører til Guds Rige.
\par 9 Men da nogle forhærdede sig og strede imod og over for Mængden talte ilde om Vejen, forlod han dem og skilte Disciplene fra dem og holdt daglig Samtaler i Tyrannus's Skole.
\par 10 Men dette varede i to År, så at alle, som boede i Asien, både Jøder og Grækere, hørte Herrens Ord.
\par 11 Og Gud gjorde usædvanlige kraftige Gerninger ved Paulus's Hænder,
\par 12 så at man endog bragte Tørklæder og Bælter fra hans Legeme til de syge, og Sygdommene vege fra dem, og de onde Ånder fore ud.
\par 13 Men også nogle af de omløbende jødiske Besværgere forsøgte at nævne den Herres Jesu Navn over dem, som havde de onde Ånder, idet de sagde: "Jeg besværger eder ved den Jesus, som Paulus prædiker."
\par 14 Men de, som gjorde dette, vare syv Sønner af Skeuas, en jødisk Ypperstepræst,
\par 15 Men den onde Ånd svarede og sagde til dem: "Jesus kender jeg, og om Paulus ved jeg; men I, hvem ere I?"
\par 16 Og det Menneske, i hvem den onde Ånd var, sprang ind på dem og overmandede dem begge og fik sådan Magt over dem, at de flygtede nøgne og sårede ud af Huset.
\par 17 Men dette blev vitterligt for alle dem, som boede i Efesus, både Jøder og Grækere; og der faldt en Frygt over dem alle, og den Herres Jesu Navn blev ophøjet,
\par 18 og mange af dem, som vare blevne troende, kom og bekendte og fortalte om deres Gerninger.
\par 19 Men mange af dem, som havde drevet Trolddom, bare deres Bøger sammen og opbrændte dem for alles Øjne; og man beregnede deres Værdi og fandt dem halvtredsindstyve Tusinde Sølvpenge værd.
\par 20 Så kraftigt voksede Herrens Ord og fik Magt.
\par 21 Men da dette var fuldbragt, satte Paulus sig for i Ånden, at han vilde rejse igennem Makedonien og Akaja og så drage til Jerusalem, og han sagde: "Efter at jeg har været der, bør jeg også se Rom."
\par 22 Og han sendte to af dem, som gik ham til Hånde, Timotheus og Erastus, til Makedonien; men selv blev han nogen Tid i Asien.
\par 23 Men på den Tid opstod der et ikke lidet Oprør i Anledning af Vejen.
\par 24 Thi en Sølvsmed ved Navn Demetrius gjorde Artemistempler af Sølv og skaffede Kunstnerne ikke ringe Fortjeneste.
\par 25 Disse samlede han tillige med de med sådanne Ting sysselsatte Arbejdere og sagde: "I Mænd! I vide, at vi have vort Udkomme af dette Arbejde.
\par 26 Og I se og høre, at ikke alene i Efesus, men næsten i hele Asien har denne Paulus ved sin Overtalelse vildledt en stor Mængde, idet han siger, at de ikke ere Guder, de, som gøres med Hænder.
\par 27 Men der er ikke alene Fare for, at denne vor Håndtering skal komme i Foragt, men også for, at den store Gudinde Artemis's Helligdom skal blive agtet for intet, og at den Gudindes Majestæt, hvem hele Asien og Jorderige dyrker, skal blive krænket."
\par 28 Men da de hørte dette, bleve de fulde af Vrede og råbte og sagde: "Stor er Efesiernes Artemis!"
\par 29 Og Byen kom i fuldt Oprør, og de stormede endrægtigt til Teatret og reve Makedonierne Hajus og Aristarkus, Paulus's Rejsefæller, med sig.
\par 30 Men da Paulus vilde gå ind iblandt Folkemængden, tilstedte Disciplene ham det ikke.
\par 31 Men også nogle af Asiarkerne, som vare hans Venner, sendte Bud til ham og formanede ham til ikke at vove sig hen til Teatret.
\par 32 Da skrege nogle eet, andre et andet; thi Forsamlingen var i Forvirring, og de fleste vidste ikke, af hvad Årsag de vare komne sammen.
\par 33 Men de trak Aleksander, hvem Jøderne skøde frem, ud af Skaren; men Aleksander slog til Lyd med Hånden og vilde holde en Forsvarstale til Folket.
\par 34 Men da de fik at vide, at han var en Jøde, råbte de alle med een Røst i omtrent to Timer: "Stor er Efesiernes Artemis!"
\par 35 Men Byskriveren fik Skaren beroliget og sagde: "I Mænd i Efesus! hvilket Menneske er der vel, som ikke ved, at Efesiernes By er Tempelværge for den store Artemis og det himmelfaldne Billede?
\par 36 Når altså dette er uimodsigeligt, bør I være rolige og ikke foretage eder noget fremfusende.
\par 37 Thi I have ført disse Mænd hid, som hverken er Tempelranere eller bespotte eders Gudinde.
\par 38 Dersom nu Demetrius og hans Kunstnere have Klage imod nogen, da holdes der Tingdage, og der er Statholdere; lad dem kalde hinanden for Retten!
\par 39 Men have I noget Forlangende om andre Sager, så vil det blive afgjort i den lovlige Forsamling.
\par 40 Vi stå jo endog i Fare for at anklages for Oprør for, hvad der i Dag er sket, da der ingen Årsag er dertil; herfor, for dette Opløb, ville vi ikke kunne gøre regnskab."
\par 41 Og da han havde sagt dette, lod han Forsamlingen fare.

\chapter{20}

\par 1 Men efter at dette Røre var stillet, lod Paulus Disciplene hente og formanede dem, tog Afsked og begav sig derfra for at rejse til Makedonien.
\par 2 Og da han var dragen igennem disse Egne og havde formanet dem med megen Tale, kom han til Grækenland.
\par 3 Der tilbragte han tre Måneder, og da Jøderne havde Anslag for imod ham, just som han skulde til at sejle til Syrien, blev han til Sinds at vende tilbage igennem Makedonien.
\par 4 Men Pyrrus's Søn Sopater fra Berøa og af Thessalonikerne Aristarkus og Sekundus og Kajus fra Derbe og Timotheus og af Asiaterne Tykikus og Trofimus fulgte med ham til Asien.
\par 5 Disse droge forud og biede på os i Troas;
\par 6 men vi sejlede efter de usyrede Brøds Dage ud fra Filippi og kom fem Dage efter til dem i Troas, hvor vi tilbragte syv Dage.
\par 7 Men på den første Dag i Ugen, da vi vare forsamlede for at bryde Brødet, samtalede Paulus med dem, da han den næste Dag vilde rejse derfra, og han blev ved med at tale indtil Midnat.
\par 8 Men der var mange Lamper i Salen ovenpå, hvor vi vare samlede.
\par 9 Og der sad i Vinduet en ung Mand ved Navn Eutykus; han faldt i en dyb Søvn, da Paulus fortsatte Samtalen så længe, og overvældet af Søvnen styrtede han ned fra det tredje Stokværk og blev tagen død op.
\par 10 Men Paulus gik ned og kastede sig over ham og omfavnede ham og sagde: "Larmer ikke; thi hans Sjæl er i ham."
\par 11 Men han gik op igen og brød Brødet og nød deraf og talte endnu længe med dem indtil Dagningen, og dermed drog han bort.
\par 12 Men de bragte det unge Menneske levende op og vare ikke lidet trøstede.
\par 13 Men vi gik forud til Skibet og sejlede til Assus og skulde derfra tage Paulus med; thi således havde han bestemt det, da han selv vilde gå til Fods.
\par 14 Da han nu stødte til os i Assus, toge vi ham om Bord og kom til Mitylene.
\par 15 Og vi sejlede derfra og kom den næste Dag lige udfor Kios; Dagen derpå lagde vi til ved Samos og kom næste Dag til Milet.
\par 16 Thi Paulus havde besluttet at sejle Efesus forbi, for at det ikke skulde hændes, at han blev opholdt i Asien; thi han hastede for at komme til Jerusalem på Pinsedagen, om det var ham muligt.
\par 17 Men fra Milet sendte han Bud til Efesus og lod Menighedens Ældste kalde til sig.
\par 18 Og da de kom til ham, sagde han til dem: "I vide, hvorledes jeg færdedes iblandt eder den hele Tid igennem fra den første Dag, jeg kom til Asien,
\par 19 idet jeg tjente Herren i al Ydmyghed og under Tårer og Prøvelser, som timedes mig ved Jødernes Efterstræbelser;
\par 20 hvorledes jeg ikke har unddraget mig fra at forkynde eder noget som helst af det, som kunde være til Gavn, og at lære eder offentligt og i Husene,
\par 21 idet jeg vidnede både for Jøder og Grækere om Omvendelsen til Gud og Troen på vor Herre Jesus Kristus.
\par 22 Og nu se, bunden af Ånden drager jeg til Jerusalem uden at vide, hvad der skal møde mig,
\par 23 kun, at den Helligånd i hver By vidner for mig og siger, at Lænker og Trængsler vente mig.
\par 24 Men jeg agter ikke mit Liv noget værd for mig selv, for at jeg kan fuldende mit Løb og den Tjeneste, som jeg har fået af den Herre Jesus, at vidne om Guds Nådes Evangelium.
\par 25 Og nu se, jeg ved, at I ikke mere skulle se mit Ansigt, alle I, iblandt hvem jeg gik om og prædikede Riget.
\par 26 Derfor vidner jeg for eder på denne Dag, at jeg er ren for alles Blod;
\par 27 thi jeg unddrog mig ikke fra at forkynde eder hele Guds Råd.
\par 28 Så giver Agt på eder selv og den hele Hjord, i hvilken den Helligånd satte eder som Tilsynsmænd, til at vogte Guds Menighed, som han erhvervede sig med sit eget Blod.
\par 29 Jeg ved, at der efter min Bortgang skal komme svare Ulve ind iblandt eder, som ikke ville spare Hjorden.
\par 30 Og af eders egen Midte skal der opstå Mænd, som skulle tale forvendte Ting for at drage Disciplene efter sig.
\par 31 Derfor våger og kommer i Hu, at jeg har ikke ophørt i tre År, Nat og Dag, at påminde hver enkelt med Tårer.
\par 32 Og nu overgiver jeg eder til Gud og hans Nådes Ord, som formår at opbygge eder og at give eder Arven iblandt alle de helligede.
\par 33 Jeg har ikke begæret nogens Sølv eller Guld eller Klædebon.
\par 34 I vide selv, at disse Hænder have tjent for mine Fornødenheder og for dem, som vare med mig.
\par 35 Jeg viste eder i alle Ting, at således bør vi arbejde og tage os af de skrøbelige og ihukomme den Herres Jesu Ord, at han selv har sagt: "Det er saligere at give end at tage."
\par 36 Og da han havde sagt dette, faldt han på sine Knæ og bad med dem alle.
\par 37 Og de brast alle i heftig Gråd, og de faldt Paulus om Halsen og kyssede ham.
\par 38 Og mest smertede dem det Ord, han havde sagt, at de ikke mere skulde se hans Ansigt. Så ledsagede de ham til Skibet.

\chapter{21}

\par 1 Men da vi havde revet os løs fra dem og vare afsejlede, droge vi lige til Kos, og den næste Dag til Rodus og derfra til Patara.
\par 2 Og da vi fandt et Skib, som skulde gå lige til Fønikien, gik vi om Bord og afsejlede.
\par 3 Men da vi havde fået Kypern i Sigte og vare komne den forbi til venstre for os, sejlede vi til Synen og landede i Tyrus; thi der skulde Skibet losse sin Ladning.
\par 4 Og vi opsøgte Disciplene og bleve der syv Dage; disse sagde ved Ånden til Paulus, at han ikke skulde drage op til Jerusalem.
\par 5 Men da vi havde tilendebragt disse Dage, droge vi derfra og rejste videre, idet de alle, med Hustruer og Børn, ledsagede os uden for Byen; og efter at have knælet på Strandbredden og holdt Bøn
\par 6 toge vi Afsked med hverandre; og vi gik om Bord i Skibet, men de vendte tilbage til deres Hjem.
\par 7 Men vi fuldendte Sejladsen og kom fra Tyrus til Ptolemais, og vi hilste på Brødrene og bleve een Dag hos dem.
\par 8 Og den følgende Dag droge vi derfra og kom til Kæsarea, og vi gik ind i Evangelisten Filips Hus, han, som var en af de syv, og bleve hos ham.
\par 9 Men denne havde fire ugifte Døtre, som profeterede.
\par 10 Men da vi bleve der flere Dage, kom der en Profet ned fra Judæa ved Navn Agabus.
\par 11 Og han kom til os og tog Paulus's Bælte og bandt sine egne Fødder og Hænder og sagde: "Dette siger den Helligånd: Den Mand, hvem dette Bælte tilhører, skulle Jøderne binde således i Jerusalem og overgive i Hedningers Hænder."
\par 12 Men da vi hørte dette, bade såvel vi som de der på Stedet ham om ikke at drage op til Jerusalem.
\par 13 Da svarede Paulus: "Hvad gøre I, at I græde og gøre mit Hjerte modløst? thi jeg er rede til ikke alene at bindes, men også at dø i Jerusalem for den Herres Jesu Navns Skyld."
\par 14 Da han nu ikke vilde lade sig overtale, bleve vi stille og sagde: "Herrens Villie ske!"
\par 15 Men efter disse Dage gjorde vi os rede og droge op til Jerusalem.
\par 16 Og også nogle af Disciplene fra Kæsarea rejste med os og bragte os til Mnason, en Mand fra Kypern, en gammel Discipel, hos hvem vi skulde have Herberge.
\par 17 Da vi nu kom til Jerusalem, modtoge Brødrene os med Glæde.
\par 18 Og Dagen efter gik Paulus ind med os til Jakob, og alle de Ældste kom derhen.
\par 19 Og da han havde hilst på dem, fortalte han Stykke for Stykke, hvad Gud havde gjort iblandt Hedningerne ved hans Tjeneste.
\par 20 Men da de hørte dette, priste de Gud og de sagde til ham: "Broder! du ser, hvor mange Tusinder der er af Jøderne, som have antaget Troen, og de ere alle nidkære for Loven.
\par 21 Men de have hørt om dig, at du lærer alle Jøderne ude iblandt Hedningerne at falde fra Moses og siger, at de ikke skulle omskære Børnene, ej heller vandre efter Skikkene.
\par 22 Hvad er der da at gøre? Der må sikkert komme mange Mennesker sammen; thi de ville få at høre, at du er kommen.
\par 23 Gør derfor dette, som vi sige dig: Vi have her fire Mænd, som have et Løfte på sig.
\par 24 Tag dem med dig, og rens dig sammen med dem,, og gør Omkostningen for dem, for at de kunne lade deres Hoved rage; så ville alle erkende, at det, som de have hørt om dig, ikke har noget på sig, men at du også selv vandrer således, at du holder Loven.
\par 25 Men om de Hedninger, som ere blevne troende, have vi udsendt en Skrivelse med den Afgørelse, at de intet sådant skulle holde, men kun vogte sig for Afgudsofferkød og Blod og det kvalte og Utugt."
\par 26 Da tog Paulus Mændene med sig næste dag, og efter at have renset sig sammen med dem gik han ind i Helligdommen og anmeldte Renselsesdagenes Udløb, da Offeret blev bragt for hver enkelt af dem.
\par 27 Men da de syv Dage næsten vare til Ende, satte Jøderne fra Asien, som havde set ham i Helligdommen, hele Mængden i Oprør og lagde Hånd på ham
\par 28 og råbte: "I israelitiske Mænd, kommer til Hjælp! Denne er det Menneske, som alle Vegne lærer alle imod Folket og Loven og dette Sted; og tilmed har han også ført Grækere ind i Helligdommen og gjort dette hellige Sted urent;"
\par 29 de havde nemlig i Forvejen set Efesieren Trofimus i Staden sammen med ham, og ham mente de, at Paulus havde ført ind i Helligdommen.
\par 30 Og hele Staden kom i Bevægelse, og Folket stimlede sammen; og de grebe Paulus og slæbte ham uden for Helligdommen, og straks bleve Dørene lukkede.
\par 31 Og da de søgte at slå ham ihjel, gik der Melding op til Krigsøversten for Vagtafdelingen, at hele Jerusalem var i Oprør.
\par 32 Han tog straks Stridsmænd og Høvedsmænd med sig og ilede ned imod dem. Men da de så Krigsøversten og Stridsmændene, holdt de op at slå Paulus.
\par 33 Da trådte Krigsøversten til, greb ham og befalede, at han skulde bindes med to Lænker, og han spurgte, hvem han var, og hvad han havde gjort.
\par 34 Da råbte nogle i Skaren eet, andre et andet til ham; men da han ikke kunde få noget pålideligt at vide på Grund af Larmen, befalede han at føre ham ind i Borgen,
\par 35 Men da han kom på Trappen, gik det således, at han måtte bæres af Stridsmændene på Grund af Skarens Voldsomhed;
\par 36 thi Folkemængden fulgte efter og råbte: "Bort med ham!"
\par 37 Og da Paulus var ved at blive ført ind i Borgen, siger han til Krigsøversten: "Er det mig tilladt at sige noget til dig?" Men han sagde: "Forstår du Græsk?
\par 38 Er du da ikke den Ægypter, som for nogen Tid siden gjorde Oprør og førte de fire Tusinde Stimænd ud i Ørkenen?"
\par 39 Men Paulus sagde: "Jeg er en jødisk Mand fra Tarsus, Borger i en ikke ubekendt By i Kilikien. Men jeg beder dig, tilsted mig at tale til Folket!"
\par 40 Og da han tilstedte det, stod Paulus frem på Trappen og slog til Lyd med Hånden for Folket. Men da der var blevet dyb Tavshed, tiltalte han dem i det hebraiske Sprog og sagde:

\chapter{22}

\par 1 " I Mænd, Brødre og Fædre! hører nu mit forsvar over for eder!"
\par 2 Men da de hørte, at han talte til dem i det hebraiske Sprog, holdt de sig end mere stille. Og han siger:
\par 3 "Jeg er en jødisk Mand, født i Tarsus i Kilikien, men opfostret i denne Stad, oplært ved Gamaliels Fødder efter vor Fædrenelovs Strenghed og nidkær for Gud, ligesom I alle ere i Dag.
\par 4 Og jeg forfulgte denne Vej indtil Døden, idet jeg lagde både Mænd og Kvinder i Lænker og overgav dem til Fængsler,
\par 5 som også Ypperstepræsten vidner med mig og hele Ældsterådet, fra hvem jeg endog fik Breve med til Brødrene i Damaskus og rejste derhen for også at føre dem, som vare der, bundne til Jerusalem, for at de måtte blive straffede.
\par 6 Men det skete, da jeg var undervejs og nærmede mig til Damaskus, at ved Middag et stærkt Lys fra Himmelen pludseligt omstrålede mig.
\par 7 Og jeg faldt til Jorden og hørte en Røst, som sagde til mig: Saul! Saul! hvorfor forfølger du mig?
\par 8 Men jeg svarede: Hvem er du, Herre? Og han sagde til mig: Jeg er Jesus af Nazareth, som du forfølger.
\par 9 Men de, som vare med mig, så vel Lyset, men hørte ikke hans Røst, som talte til mig.
\par 10 Men jeg sagde: Hvad skal jeg gøre, Herre? Men Herren sagde til mig: Stå op og gå til Damaskus; og der skal der blive talt til dig om alt, hvad der er bestemt, at du skal gøre.
\par 11 Men da jeg havde mistet Synet ved Glansen af hint Lys, blev jeg ledet ved Hånden af dem, som vare med mig, og kom således ind i Damaskus.
\par 12 Men en vis Ananias, en Mand, gudfrygtig efter Loven, som havde godt Vidnesbyrd af alle Jøderne, som boede der,
\par 13 kom til mig og stod for mig og sagde: Saul, Broder, se op! Og jeg så op på ham i samme Stund.
\par 14 Men han sagde: Vore Fædres Gud har udvalgt dig til at kende hans Villie og se den retfærdige og høre en Røst af hans Mund.
\par 15 Thi du skal være ham et Vidne for alle Mennesker om de Ting, som du har set og hørt.
\par 16 Og nu, hvorfor tøver du? Stå op, lad dig døbe og dine Synder aftvætte, idet du påkalder hans Navn!
\par 17 Og det skete, da jeg var kommen tilbage til Jerusalem og bad i Helligdommen, at jeg faldt i Henrykkelse
\par 18 og så ham, idet han sagde til mig: Skynd dig, og gå hastigt ud af Jerusalem, thi de skulle ikke af dig modtage Vidnesbyrd om mig.
\par 19 Og jeg sagde: Herre! de vide selv, at jeg fængslede og piskede trindt om i Synagogerne dem, som troede på dig,
\par 20 og da dit Vidne Stefanus's Blod blev udgydt, stod også jeg hos og havde Behag deri og vogtede på deres Klæder, som sloge ham ihjel.
\par 21 Og han sagde til mig: Drag ud; thi jeg vil sende dig langt bort til Hedninger."
\par 22 Men de hørte på ham indtil dette Ord, da opløftede de deres Røst og sagde: "Bort fra Jorden med en sådan! thi han bør ikke leve.""
\par 23 Men da de skrege og reve Klæderne af sig og kastede Støv op i Luften,
\par 24 befalede Krigsøversten, at han skulde føres ind i Borgen, og sagde, at man med Hudstrygning skulde forhøre ham, for at han kunde få at vide, af hvad Årsag de således råbte imod ham.
\par 25 Men da de havde udstrakt ham for Svøberne, sagde Paulus til den hosstående Høvedsmand: "Er det eder tilladt at hudstryge en romersk Mand, og det uden Dom?"
\par 26 Men da Høvedsmanden hørte dette, gik han til Krigsøversten og meldte ham det og sagde: "Hvad er det, du et ved at gøre? denne Mand er jo en Romer."
\par 27 Men Krigsøversten gik hen og sagde til ham: "Sig mig, er du en Romer?" Han sagde: "Ja."
\par 28 Og Krigsøversten svarede: "Jeg har købt mig denne Borgerret for en stor Sum," Men Paulus sagde: "Jeg er endog født dertil."
\par 29 Da trak de, som skulde til at forhøre ham, sig straks tilbage fra ham. Og da Krigsøversten fik at vide, at han var en Romer, blev også han bange, fordi han havde bundet ham.
\par 30 Men den næste Dag, da han vilde have noget pålideligt at vide om, hvad han anklagedes for af Jøderne, løste han ham og befalede, at Ypperstepræsterne og hele Rådet skulde komme sammen, og han førte Paulus ned og stillede ham for dem.

\chapter{23}

\par 1 Da så Paulus fast på Rådet og sagde: "I Mænd, Brødre! jeg har med al god Samvittighed vandret for Gud indtil denne Dag."
\par 2 Men Ypperstepræsten Ananias befalede dem, som stode hos ham, at slå ham på Munden.
\par 3 Da sagde Paulus til ham: "Gud skal slå dig, du kalkede Væg! Og du sidder for at dømme mig efter Loven, og tvært imod Loven befaler du, at jeg skal slås."
\par 4 Men de, som stode hos, sagde: "Udskælder du Guds Ypperstepræst?"
\par 5 Og Paulus sagde: "Brødre! jeg vidste ikke, at han er Ypperstepræst; thi der er skrevet: En Fyrste i dit Folk må du ikke tale ondt imod."
\par 6 Men da Paulus vidste, at den ene Del bestod af Saddukæere, men den anden af Farisæere, råbte han i Rådet: "I Mænd, Brødre! jeg er en Farisæer, Søn af Farisæere, for Håb og for dødes Opstandelse er det, jeg dømmes."
\par 7 Men da han udtalte dette, opkom der Splid imellem Farisæerne og Saddukæerne, og Mængden blev uenig.
\par 8 Thi Saddukæerne sige, at der ingen Opstandelse er, ej heller nogen Engel eller Ånd; men Farisæerne hævde begge Dele.
\par 9 Men der opstod en stærk Råben; og nogle af de skriftkloge af Farisæernes Parti stode op, strede heftigt og sagde: "Vi finde intet ondt hos dette Menneske; men hvad om en Ånd eller en Engel har talt til ham!"
\par 10 Men da der blev stærk Splid frygtede Krigsøversten, at Paulus skulde blive sønderslidt af dem, og befalede Krigsfolket at gå ned og rive ham ud fra dem og føre ham ind i Borgen.
\par 11 Men Natten derefter stod Herren for ham og sagde: "Vær frimodig, thi ligesom du har vidnet om mig i Jerusalem, således skal du også vidne i Rom."
\par 12 Men da det var blevet Dag, sloge Jøderne sig sammen og forpligtede sig under Forbandelser til hverken at spise eller drikke, førend de havde slået Paulus ihjel.
\par 13 Og de, som havde indgået denne Sammensværgelse, vare flere end fyrretyve i Tal.
\par 14 Disse gik da til Ypperstepræsterne og de Ældste og sagde: "Vi have under Forbandelser forpligtet os til ikke at smage noget, førend vi have slået Paulus ihjel.
\par 15 Så giver nu I tillige med Rådet Krigsøversten Meddelelse, for at han må føre ham ned til eder, som om I ville undersøge hans Sag nøjere; men vi ere rede til at slå ham ihjel, førend han kommer derhen."
\par 16 Men Paulus's Søstersøn, som havde hørt om dette Anslag, kom og gik ind i Borgen og fortalte Paulus det.
\par 17 Men Paulus kaldte en af Høvedsmændene til sig og sagde: "Før denne unge Mand hen til Krigsøversten; thi han har noget at melde ham."
\par 18 Da tog han ham og førte ham til Krigsøversten og siger: "Den fangne Paulus kaldte mig og bad mig føre denne unge Mand til dig, da han har noget at tale med dig om."
\par 19 Men Krigsøversten tog ham ved Hånden, gik hen til en Side og spurgte: "Hvad er det, som du har at melde mig?"
\par 20 Men han sagde: "Jøderne have aftalt at bede dig om at lade Paulus føre ned for Rådet i Morgen under Foregivende af at ville have nøjere Underretning om ham.
\par 21 Lad du dig nu ikke overtale af dem; thi mere end fyrretyve Mænd af dem lure på ham, og de have under Forbandelser forpligtet sig til hverken at spise eller at drikke, førend de have slået ham ihjel; og nu ere de rede og vente på dit Tilsagn."
\par 22 Da lod Krigsøversten det unge Menneske fare og bød ham: "Du skal ingen sige, at du har givet mig dette til Kende."
\par 23 Og han kaldte et Par af Høvedsmændene til sig og sagde: "Gører to Hundrede Stridsmænd rede til at drage til Kæsarea og halvfjerdsindstyve Ryttere og to Hundrede Spydkastere fra den tredje Time i Nat; "
\par 24 og at de skulde bringe Lastdyr for at kunne lade Paulus ride og føre ham sikkert til Landshøvdingen Feliks.
\par 25 Og han skrev et Brev af følgende Indhold:
\par 26 "Klaudius Lysias hilser den mægtigste Landshøvding Feliks.
\par 27 Denne Mand havde Jøderne grebet og vilde have slået ham ihjel; men jeg kom til med Krigsfolket og udfriede ham, da jeg erfarede, at han var en Romer.
\par 28 Og da jeg vilde vide Årsagen, hvorfor de anklagede ham, førte jeg ham ned for deres Råd
\par 29 og fandt ham anklaget i Anledning af nogle Stridsspørgsmål i deres Lov, men uden nogen Beskyldning, som fortjente Død eller Fængsel.
\par 30 Men da jeg har fået Underretning om, at der skulde være et hemmeligt Anslag af Jøderne imod Manden, har jeg straks sendt ham til dig efter også at have befalet Anklagerne at fremføre for dig, hvad de have imod ham."
\par 31 Da toge Stridsmændene Paulus, som det var dem befalet, og førte ham om Natten til Antipatris.
\par 32 Men næste Dag lode de Rytterne drage videre med ham og vendte selv tilbage til Borgen.
\par 33 Da hine nu kom til Kæsarea og havde overgivet Landshøvdingen Brevet, fremstillede de også Paulus for ham.
\par 34 Men da han havde læst Brevet og spurgt, fra hvilken Provins han var, og havde erfaret, at han var fra Kilikien, sagde han:
\par 35 "Jeg vil forhøre dig, når også dine Anklagere komme til Stede." Og han bød, at han skulde holdes bevogtet i Herodes's Borg.

\chapter{24}

\par 1 Men fem Dage derefter drog Ypperstepræsten Ananias ned med nogle Ældste og en Taler, Tertullus, og disse førte Klage for Landshøvdingen imod Paulus.
\par 2 Da han nu var kaldt ind, begyndte Tertullus at anklage ham og sagde:
\par 3 "At vi ved dig nyde megen Fred, og at Forbedringer i alle Retninger og alle Vegne skaffes dette Folk ved din Omsorg, mægtigste Feliks! det erkende vi med al Taknemmelighed.
\par 4 Men for at jeg ikke skal opholde dig for længe, beder jeg, at du efter din Mildhed vil høre os kortelig.
\par 5 Vi have nemlig fundet, at denne Mand er en Pest og en Oprørsstifter iblandt alle Jøderne hele Verden over, samt er Fører for Nazaræernes Parti,
\par 6 ja, han har endog forsøgt at vanhellige Helligdommen. Vi grebe ham da også og vilde have dømt ham efter vor Lov.
\par 7 Men Krigsøversten Lysias kom til og borttog ham med megen Vold af vore Hænder
\par 8 og bød hans Anklagere komme til dig. Af ham kan du selv, når du undersøger det, erfare alt det, hvorfor vi anklage ham."
\par 9 Men også Jøderne stemmede i med og påstode, at dette forholdt sig således.
\par 10 Og Paulus svarede, da Landshøvdingen gav ham et Vink, at han skulde tale: "Efterdi jeg ved, at du i mange År har været Dommer for dette Folk, vil jeg frimodigt forsvare min Sag,
\par 11 da du kan forvisse dig om, at det er ikke mere end tolv Dage, siden jeg kom op for at tilbede i Jerusalem.
\par 12 Og de have ikke fundet mig i Ordveksel med nogen eller i Færd med at vække Folkeopløb, hverken i Helligdommen eller i Synagogerne eller omkring i Staden.
\par 13 Og de kunne ej heller bevise dig det, som de nu anklage mig for.
\par 14 Men dette bekender jeg for dig, at jeg efter den Vej, som de kalde et Parti, tjener vor fædrene Gud således, at jeg tror på alt det, som står i Loven, og det, som er skrevet hos Profeterne,
\par 15 og har det Håb til Gud, som også disse selv forvente, at der skal komme en Opstandelse både af retfærdige og af uretfærdige.
\par 16 Derfor øver også jeg mig i altid at have en uskadt Samvittighed for Gud og Menneskene.
\par 17 Men efter flere Års Forløb er jeg kommen for at bringe Almisser til mit Folk og Ofre,
\par 18 hvad de fandt mig i Færd med, da jeg var bleven renset i Helligdommen, og ikke med Opløb og Larm; men det var nogle Jøder fra Asien,
\par 19 og de burde nu være til Stede hos dig og klage, om de have noget på mig at sige.
\par 20 Eller lad disse her selv sige, hvad Uret de have fundet hos mig, da jeg stod for Rådet,
\par 21 uden det skulde være dette ene Ord, som jeg råbte, da jeg stod iblandt dem: Jeg dømmes i Dag af eder for dødes Opstandelse."
\par 22 Nu udsatte Feliks Sagen, da han vidste ret god Besked om Vejen, og sagde: "Når Krigsøversten Lysias kommer herned, vil jeg påkende eders Sag."
\par 23 Og han befalede Høvedsmanden, at han skulde holdes bevogtet, men med Lempelse, og at han ikke måtte forbyde nogen af hans egne at gå ham til Hånde.
\par 24 Men nogle Dage efter kom Feliks med sin Hustru Drusilla, som var en Jødinde, og lod Paulus hente og hørte ham om Troen på Kristus Jesus.
\par 25 Men da han talte med ham om Retfærdighed og Afholdenhed og den kommende Dom, blev Feliks forfærdet 6g svarede: "Gå for denne Gang; men når jeg får Tid, vil jeg lade dig kalde til mig."
\par 26 Tillige håbede han også, at Paulus skulde give ham Penge; derfor lod han ham også oftere hente og samtalede med ham.
\par 27 Men da to År vare forløbne, fik Feliks Porkius Festus til Efterfølger; og da Feliks vilde fortjene sig Tak af Jøderne, lod han Paulus blive tilbage i Lænker.

\chapter{25}

\par 1 Da Festus nu havde tiltrådt sit Landshøvdingembede, drog han efter tre Dages Forløb fra Kæsarea op til Jerusalem.
\par 2 Da førte Ypperstepræsterne og de fornemste af Jøderne Klage hos ham imod Paulus og henvendte sig til ham,
\par 3 idet de med ondt i Sinde imod Paulus bade ham om at bevise dem den Gunst, at han vilde lade ham hente til Jerusalem; thi de lurede på at slå ham ihjel på Vejen.
\par 4 Da svarede Festus, at Paulus blev holdt bevogtet i Kæsarea, men at han selv snart vilde drage derned.
\par 5 "Lad altså," sagde han, "dem iblandt eder, der have Myndighed dertil, drage med ned og anklage ham, dersom der er noget uskikkeligt ved Manden."
\par 6 Og da han havde opholdt sig hos dem ikke mere end otte eller ti Dage, drog han ned til Kæsarea, og den næste Dag satte han sig på Dommersædet og befalede, at Paulus skulde føres frem.
\par 7 Men da han kom til Stede, stillede de Jøder, som vare komne ned fra Jerusalem sig omkring ham og fremførte mange og svare Klagemål, som de ikke kunde bevise,
\par 8 efterdi Paulus forsvarede sig og sagde: "Hverken imod Jødernes Lov eller imod Helligdommen eller imod Kejseren har jeg syndet i noget Stykke."
\par 9 Men Festus. som vilde fortjene sig Tak af Jøderne, svarede Paulus og sagde: "Er du villig til at drage op til Jerusalem og der stå for min Domstol i denne Sag?"
\par 10 Men Paulus sagde: "Jeg står for Kejserens Domstol, og der bør jeg dømmes. Jøderne har jeg ingen Uret gjort, som også du ved helt vel.
\par 11 Dersom jeg så har Uret og har gjort noget, som fortjener Døden, vægrer jeg mig ikke ved at dø; men hvis det, hvorfor disse anklage mig, intet har på sig, da kan ingen prisgive mig til dem. Jeg skyder mig ind under Kejseren."
\par 12 Da talte Festus med sit Råd og svarede: "Du har skudt dig ind under Kejseren; du skal rejse til Kejseren."
\par 13 Men da nogle Dage vare forløbne, kom Kong Agrippa og Berenike til Kæsarea og hilste på Festus.
\par 14 Og da de opholdt sig der i flere Dage, forelagde Festus Kongen Paulus's Sag og sagde: "Der er en Mand, efterladt af Feliks som Fange;
\par 15 imod ham førte Jødernes Ypperstepræster og Ældste Klage, da jeg var i Jerusalem, og bade om Dom over ham.
\par 16 Dem svarede jeg, at Romere ikke have for Skik at prisgive noget Menneske, førend den anklagede har Anklagerne personligt til Stede og får Lejlighed til at forsvare sig imod Beskyldningen.
\par 17 Da de altså kom sammen her, tøvede jeg ikke, men satte mig den næste Dag på Dommersædet og bød, at Manden skulde føres frem.
\par 18 Men da Anklagerne stode omkring ham, fremførte de ingen sådan Beskyldning, som jeg havde formodet;
\par 19 men de havde nogle Stridsspørgsmål med ham om deres egen Gudsdyrkelse og om en Jesus, som er død, men som Paulus påstod er i Live.
\par 20 Men da jeg var tvivlrådig angående Undersøgelsen heraf, sagde jeg, om han vilde rejse til Jerusalem og der lade denne Sag pådømme.
\par 21 Men da Paulus gjorde Påstand på at holdes bevogtet til Kejserens Kendelse, befalede jeg, at han skulde holdes bevogtet, indtil jeg kan sende ham til Kejseren."
\par 22 Da sagde Agrippa til Festus: "Jeg kunde også selv ønske at høre den Mand." Men han sagde: "I Morgen skal du få ham at høre."
\par 23 Næste Dag altså, da Agrippa og Berenike kom med stor Pragt og gik ind i Forhørssalen tillige med Krigsøversterne og Byens ypperste Mænd, blev på Festus's Befaling Paulus ført frem.
\par 24 Og Festus siger: "Kong Agrippa, og alle I Mænd, som ere med os til Stede! her se I ham, om hvem hele Jødernes Mængde har henvendt sig til mig, både i Jerusalem og her, råbende på, at han ikke længer bør leve.
\par 25 Men jeg indså, at han intet havde gjort, som fortjente Døden, og da han selv skød sig ind under Kejseren, besluttede jeg at sende ham derhen.
\par 26 Dog har jeg intet sikkert at skrive om ham til min Herre. Derfor lod jeg ham føre frem for eder og især for dig, Kong Agrippa! for at jeg kan have noget at skrive, når Undersøgelsen er sket.
\par 27 Thi det synes mig urimeligt at sende en Fange uden også at angive Beskyldningerne imod ham."

\chapter{26}

\par 1 Men Agrippa sagde til Paulus: "Det tilstedes dig at tale om dig selv." Da udrakte Paulus Hånden og forsvarede sig således:
\par 2 "Jeg agter mig selv lykkelig, fordi jeg i Dag skal forsvare mig for dig angående alle de Ting, for hvilke jeg anklages af Jøderne, Kong Agrippa!
\par 3 navnlig fordi du er kendt med alle Jødernes Skikke og Stridsspørgsmål; derfor beder jeg dig om, at du tålmodigt vil høre mig.
\par 4 Mit Levned fra Ungdommen af, som fra Begyndelsen har været ført iblandt mit Folk og i Jerusalem, vide alle Jøderne Besked om;
\par 5 thi de kende mig i Forvejen lige fra først af (om de ellers ville vidne), at jeg har levet som Farisæer efter det strengeste Parti i vor Gudsdyrkelse.
\par 6 Og nu står jeg her og dømmes for Håbet på den Forjættelse, som er given af Gud til vore Fædre,
\par 7 og som vort Tolvstammefolk håber at nå frem til, idet de tjene Gud uafladeligt Nat og Dag; for dette Håbs Skyld anklages jeg af Jøder, o Konge!
\par 8 Hvor kan det holdes for utroligt hos eder, at Gud oprejser døde?
\par 9 Jeg selv mente nu også at burde gøre meget imod Jesu, Nazaræerens Navn,
\par 10 og det gjorde jeg også i Jerusalem; og jeg indespærrede mange af de hellige i Fængsler, da jeg havde fået Fuldmagt dertil af Ypperstepræsterne, og når de bleve slåede ihjel, gav jeg min Stemme dertil.
\par 11 Og i alle Synagogerne lod jeg dem ofte straffe og tvang dem til at tale bespotteligt, og rasende end mere imod dem forfulgte jeg dem endog til de udenlandske Byer.
\par 12 Da jeg i dette Øjemed drog til Damaskus med Fuldmagt og Myndighed fra Ypperstepræsterne,
\par 13 så jeg undervejs midt på Dagen, o Konge! et Lys fra Himmelen, som overgik Solens Glans, omstråle mig og dem, som rejste med mig.
\par 14 Men da vi alle faldt til Jorden, hørte jeg en Røst, som sagde til mig i det hebraiske Sprog: Saul! Saul! hvorfor forfølger du mig? det bliver dig hårdt at stampe imod Brodden.
\par 15 Og jeg sagde: Hvem er du, Herre? Men Herren sagde: Jeg er Jesus, som du forfølger.
\par 16 Men rejs dig og stå på dine Fødder; thi derfor har jeg vist mig for dig, for at udkåre dig til Tjener og Vidne, både om det, som du har set, og om mine kommende Åbenbaringer for dig,
\par 17 idet jeg udfrier dig fra Folket og fra Hedningerne, til hvilke jeg udsender dig
\par 18 for at oplade deres Øjne, så de må omvende sig fra Mørke til Lys og fra Satans Magt til Gud, for at de kunne få Syndernes Forladelse og Lod iblandt dem, som ere helligede ved Troen på mig.
\par 19 Derfor, Kong Agrippa! blev jeg ikke ulydig imod det himmelske Syn;
\par 20 men jeg forkyndte både først for dem i Damaskus og så i Jerusalem og over hele Judæas Land og for Hedningerne, at de skulde fatte et andet Sind og omvende sig til Gud og gøre Gerninger, Omvendelsen værdige.
\par 21 For denne Sags Skyld grebe nogle Jøder mig i Helligdommen og forsøgte at slå mig ihjel.
\par 22 Det er altså ved den Hjælp, jeg har fået fra Gud, at jeg har stået indtil denne Dag og vidnet både for små og store, idet jeg intet siger ud over det, som både Profeterne og Moses have sagt skulde ske,
\par 23 at Kristus skulde lide, at han som den første af de dødes Opstandelse skulde forkynde Lys både for Folket og for Hedningerne."
\par 24 Men da han forsvarede sig således, sagde Festus med høj Røst: "Du raser, Paulus! den megen Lærdom gør dig rasende."
\par 25 Men Paulus sagde: "Jeg raser ikke, mægtigste Festus! men jeg taler sande og betænksomme Ord.
\par 26 Thi Kongen ved Besked om dette, og til ham taler jeg frimodigt, efterdi jeg er vis på, at slet intet af dette er skjult for ham; thi dette er ikke sket i en Vrå.
\par 27 Tror du, Kong Agrippa, Profeterne? Jeg ved, at du tror dem."
\par 28 Men Agrippa sagde til Paulus: "Der fattes lidet i, at du overtaler mig til at blive en Kristen."
\par 29 Men Paulus sagde: "Jeg vilde ønske til Gud, enten der fattes lidet eller meget, at ikke alene du, men også alle, som høre mig i Dag, måtte blive sådan, som jeg selv er, på disse Lænker nær."
\par 30 Da stod Kongen op og Landshøvdingen og Berenike og de, som sade hos dem.
\par 31 Og da de gik bort, talte de med hverandre og sagde: "Denne Mand gør intet, som fortjener Død eller Lænker."
\par 32 Men Agrippa sagde til Festus: "Denne Mand kunde være løsladt, dersom han ikke havde skudt sig ind under Kejseren."

\chapter{27}

\par 1 Men da det var besluttet, at vi skulde afsejle til Italien, overgave de både Paulus og nogle andre Fanger til en Høvedsmand ved Navn Julius af den kejserlige Afdeling.
\par 2 Vi gik da om Bord på et adramyttisk Skib, som skulde gå til Stederne langs med Asiens Kyster, og vi sejlede af Sted; og Aristarkus, en Makedonier fra Thessalonika, var med os.
\par 3 Og den næste Dag anløb vi Sidon. Og Julius, som behandlede Paulus venligt. tilstedte ham at gå hen til sine Venner og nyde Pleje.
\par 4 Og vi fore bort derfra og sejlede ind under Kypern, fordi Vinden var imod.
\par 5 Og vi sejlede igennem Farvandet ved Kilikien og Pamfylien og kom til Myra i Lykien.
\par 6 Og der fandt Høvedsmanden et aleksandrinsk Skib, som sejlede til Italien, og bragte os over i det.
\par 7 Men da Sejladsen i mange Dage gik langsomt, og vi med Nød og næppe nåede henimod Knidus (thi Vinden føjede os ikke), holdt vi ned under Kreta ved Salmone.
\par 8 Med Nød og næppe sejlede vi der forbi og kom til et Sted, som kaldes "Gode Havne", nær ved Byen Lasæa.
\par 9 Men da en rum Tid var forløben, og Sejladsen allerede var farlig, såsom endog Fasten allerede var forbi, formanede Paulus dem og sagde:
\par 10 "I Mænd! jeg ser, at Sejladsen vil medføre Ulykke og megen Skade, ikke alene på Ladning og Skib, men også på vort Liv."
\par 11 Men Høvedsmanden stolede mere på Styrmanden og Skipperen end på det, som Paulus sagde.
\par 12 Og da Havnen ikke egnede sig til Vinterleje, besluttede de fleste, at man skulde sejle derfra, om man muligt kunde nå hen og overvintre i Føniks, en Havn på Kreta, som vender imod Sydvest og Nordvest,
\par 13 Da der nu blæste en Sønden: vind op, mente de at have nået deres Hensigt, lettede Anker og sejlede langs med og nærmere ind under Kreta.
\par 14 Men ikke længe derefter for der en heftig Storm ned over den, den såkaldte "Eurakvilo".
\par 15 Og da Skibet reves med og ikke kunde holde op imod Vinden, opgave vi det og lode os drive.
\par 16 Men da vi løb ind under en lille Ø, som kaldes Klavde, formåede vi med Nød og næppe at bjærge Båden.
\par 17 Men efter at have trukket den op, anvendte de Nødmidler og omsurrede Skibet; og da de frygtede for, at de skulde blive kastede ned i Syrten, firede de Sejlene ned og lode sig således drive.
\par 18 Og da vi måtte kæmpe hårdt med Stormen, begyndte de næste Dag at kaste over Bord.
\par 19 Og på den tredje Dag udkastede de med egne Hænder Skibets Redskaber.
\par 20 Men da hverken Sol eller Stjerner lode sig se i flere Dage, og vi havde et Uvejr over os; som ikke var ringe, blev fra nu af alt Håb om Redning os betaget.
\par 21 Og da man længe ikke havde taget Føde til sig, så stod Paulus frem midt iblandt dem og sagde: "I Mænd! man burde have adlydt mig og ikke været sejlet bort fra Kreta og have sparet os denne Ulykke og Skade.
\par 22 Og nu formaner jeg eder til at være ved godt Mod; thi ingen Sjæl af eder skal forgå, men alene Skibet.
\par 23 Thi i denne Nat stod der en Engel hos mig fra den Gud, hvem jeg tilhører, hvem jeg også tjener, og sagde:
\par 24 "Frygt ikke, Paulus! du skal blive stillet for Kejseren; og se,Gud har skænket dig alle dem, som sejle med dig."
\par 25 Derfor, I Mænd! værer ved godt Mod; thi jeg har den Tillid til Gud, at det skal ske således, som der er blevet talt til mig.
\par 26 Men vi må strande på en Ø."
\par 27 Men da den fjortende Nat kom, og vi dreve i det adriatiske Hav, kom det Skibsfolkene for ved Midnatstid, at der var Land i Nærheden.
\par 28 Og da de loddede, fik de tyve Favne, og da de lidt længere fremme atter loddede, fik de femten Favne.
\par 29 Og da de frygtede, at vi skulde støde på Skær, kastede de fire Ankre ud fra Bagstavnen og bade til, at det måtte blive Dag.
\par 30 Men da Skibsfolkene gjorde Forsøg på at flygte fra Skibet og firede Båden ned i Søen under Påskud af, at de vilde lægge Ankre ud fra Forstavnen,
\par 31 da sagde Paulus til Høvedsmanden og til Stridsmændene: "Dersom disse ikke blive i Skibet, kunne I ikke reddes."
\par 32 Da kappede Stridsmændene Bådens Tove og lode den falde ned.
\par 33 Men indtil det vilde dages, formanede Paulus alle til at tage Næring til sig og sagde: "Det er i Dag den fjortende Dag, I have ventet og tilbragt uden at spise og intet taget til eder.
\par 34 Derfor formaner jeg eder til at tage Næring til eder, thi dette hører med til eders Redning; ikke et Hår på Hovedet skal gå tabt for nogen af eder."
\par 35 Men da han havde sagt dette, tog han Brød og takkede Gud for alles Øjne og brød det og begyndte at spise.
\par 36 Da bleve de alle frimodige og toge også Næring til sig.
\par 37 Men vi vare i Skibet i alt to Hundrede og seks og halvfjerdsindstyve Sjæle.
\par 38 Og da de vare blevne mættede med Føde, lettede de Skibet ved at kaste Levnedsmidlerne i Søen.
\par 39 Men da det blev Dag, kendte de ikke Landet; men de bemærkede en Vig med en Forstrand, som de besluttede, om muligt, at sætte Skibet ind på.
\par 40 Og de kappede Ankrene, som de lode blive i Søen, og løste tillige Rortovene, og idet de satte Råsejlet til for Vinden, holdt de ind på Strandbredden.
\par 41 Men de stødte på en Grund med dybt Vand på begge Sider, og der satte de Skibet, og Forstavnen borede sig fast og stod urokkelig, men Bagstavnen sloges sønder af Bølgernes Magt.
\par 42 Det var nu Stridsmændenes Råd, at man skulde ihjelslå Fangerne, for at ingen skulde svømme bort og undkomme.
\par 43 Men Høvedsmanden, som vilde frelse Paulus, forhindrede dem i dette Forehavende og bød, at de, som kunde svømme, skulde først kaste sig ud og slippe i Land,
\par 44 og de andre bjærge sig, nogle på Brædder, andre på Stykker af Skibet. Og således skete det, at alle bleve reddede i Land.

\chapter{28}

\par 1 Og da vi nu vare reddede, så fik vi at vide, at Øen hed Malta.
\par 2 Og Barbarerne viste os en usædvanlig Menneskekærlighed; thi de tændte et Bål og toge sig af os alle for den frembrydende Regns og Kuldens Skyld.
\par 3 Men da Paulus samlede en Bunke Ris og lagde på Bålet, krøb der en Øgle ud på Grund af Varmen og hængte sig fast ved hans Hånd.
\par 4 Da nu Barbarerne så Dyret hænge ved hans Hånd, sagde de til hverandre: "Sikkert er denne Mand en Morder, hvem Gengældelsen ikke har tilstedt at leve, skønt han er reddet fra Havet."
\par 5 Men han rystede Dyret af i Ilden, og der skete ham intet ondt.
\par 6 Men de ventede, at han skulde hovne op eller pludseligt falde død om. Men da de havde ventet længe og så, at der ikke skete ham noget usædvanligt, kom de på andre Tanker og sagde, at han var en Gud.
\par 7 Men i Omegnen af dette Sted havde Øens fornemste Mand, ved Navn Publius, nogle Landejendomme. Han tog imod os og lånte os venligt Herberge i tre Dage.
\par 8 Men det traf sig, at Publius's Fader lå syg af Feber og Blodgang.
\par 9 Da dette var sket, kom også de andre på Øen, som havde Sygdomme, til ham og bleve helbredte.
\par 10 De viste os også megen Ære, og da vi sejlede bort, bragte de om Bord i Skibet, hvad vi trængte til.
\par 11 Men efter tre Måneders Forløb sejlede vi da bort i et aleksandrinsk Skib, som havde haft Vinterleje ved Øen og førte Tvillingernes Mærke.
\par 12 Og vi løb ind til Syrakus, hvor vi bleve tre Dage.
\par 13 Derfra sejlede vi videre og kom til Regium, og efter en Dags Forløb fik vi Søndenvind og kom den næste Dag til Puteoli.
\par 14 Der fandt vi Brødre og bleve opfordrede til at blive hos dem i syv Dage. Og så droge vi til Rom.
\par 15 Og Brødrene derfra, som havde hørt om os, kom os i Møde til Appius's Forum og Tres-Tabernæ. Og da Paulus så dem, takkede han Gud og fattede Mod.
\par 16 Men da vi kom til Rom, (overgav Høvedsmanden Fangerne til Høvdingen for Livvagten. Dog) blev det tilstedt Paulus at bo for sig selv sammen med den Stridsmand, der bevogtede ham.
\par 17 Men efter tre Dages Forløb skete det, at han sammenkaldt de fornemste iblandt Jøderne. Men da de vare forsamlede, sagde han til dem: "I Mænd, Brødre! uagtet jeg intet har gjort imod vort Folk eller de fædrene Skikke, er jeg fra Jerusalem overgiven som Fange i Romernes Hænder,
\par 18 og disse vilde efter at have forhørt mig løslade mig, efterdi der ikke var nogen Dødsskyld hos mig.
\par 19 Men da Jøderne gjorde Indsigelse, nødtes jeg til at skyde mig ind under Kejseren, dog ikke, som om jeg havde noget at anklage mit Folk for.
\par 20 Af denne Årsag har jeg altså ladet eder kalde hid for at se og tale med eder; thi for Israels Håbs Skyld er jeg sluttet i denne Lænke."
\par 21 Men de sagde til ham: "Hverken have vi fået Brev fra Judæa om dig, ikke heller er nogen af Brødrene kommen og har meddelt eller sagt noget ondt om dig.
\par 22 Men vi ønske at høre af dig, hvad du tænker; thi om dette Parti er det os bekendt, at det alle Vegne finder Modsigelse."
\par 23 Efter så at have aftalt en Dag med ham, kom de til ham i Herberget i større Tal, og for dem forklarede han og vidnede om Guds Rige og søgte at overbevise dem om Jesus, både ud af Mose Lov og af Profeterne, fra årle om Morgenen indtil Aften.
\par 24 Og nogle lode sig overbevise af det, som blev sagt, men andre troede ikke.
\par 25 Og under indbyrdes Uenighed gik de bort, da Paulus havde sagt dette ene Ord: "Rettelig har den Helligånd talt ved Profeten Esajas til eders Fædre og sagt:
\par 26 "Gå hen til dette Folk og sig: I skulle høre med eders Øren og ikke forstå og se med eders Øjne og ikke se;
\par 27 thi dette Folks Hjerte er blevet sløvet, og med Ørene høre de tungt, og deres Øjne have de tillukket, for at de ikke skulle se med Øjnene og høre med Ørene og forstå med Hjertet og omvende sig, så jeg kunde helbrede dem."
\par 28 Derfor være det eder vitterligt, at denne Guds Frelse er sendt til Hedningerne; de skulle også høre."
\par 29 Og da han havde sagt dette, gik Jøderne bort, og der var stor Trætte imellem dem indbyrdes.
\par 30 Men han blev hele to År i sit lejede Herberge og modtog alle, som kom til ham,
\par 31 idet han prædikede Guds Rige og lærte om den Herre Jesus med al Frimodighed, uhindret.



\end{document}