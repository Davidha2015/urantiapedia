\begin{document}

\title{Romans}


\chapter{1}

\par 1 Paulus, Jesu Kristi Tjener, Apostel ifølge Kald, udtagen til at forkynde Guds Evangelium,
\par 2 hvilket han forud forjættede ved sine Profeter i hellige Skrifter,
\par 3 om hans Søn, født af Davids Sæd efter Kødet,
\par 4 kraftelig bevist som Guds Søn efter Helligheds Ånd ved Opstandelse fra de døde, Jesus Kristus, vor Herre,
\par 5 ved hvem vi have fået Nåde og Apostelgerning til at virke Tros-Lydighed iblandt alle Hedningerne for hans Navns Skyld,
\par 6 iblandt hvilke også I ere Jesu Kristi kaldede:
\par 7 Til alle Guds elskede, som ere i Rom, kaldede hellige. Nåde være med eder og Fred fra Gud vor Fader og den Herre Jesus Kristus!
\par 8 Først takker jeg min Gud ved Jesus Kristus for eder alle, fordi eders Tro omtales i den hele Verden.
\par 9 Thi Gud er mit Vidne, hvem jeg i min Ånd tjener i hans Søns Evangelium, hvor uafladeligt jeg kommer eder i Hu,
\par 10 idet jeg bestandig i mine Bønner beder om, at jeg dog endelig engang måtte få Lykke til Ved Guds Villie at komme til eder.
\par 11 Thi jeg længes efter at se eder, for at jeg kunde meddele eder nogen åndelig Nådegave, for at I måtte styrkes,
\par 12 det vil sige, for sammen at opmuntres hos eder ved hinandens Tro, både eders og min.
\par 13 Og jeg vil ikke, Brødre! at I skulle være uvidende om, at jeg ofte har sat mig for at komme til eder (men hidindtil er jeg bleven forhindret), for at jeg måtte få nogen Frugt også iblandt eder, ligesom iblandt de øvrige Hedninger.
\par 14 Både til Grækere og Barbarer, både til vise og uforstandige står jeg i Gæld.
\par 15 således er jeg, hvad mig angår, redebon til at forkynde Evangeliet også for eder, som ere i Rom.
\par 16 Thi jeg skammer mig ikke ved Evangeliet; thi det er en Guds Kraft til Frelse for hver den, som tror, både for Jøde først og for Græker.
\par 17 Thi deri åbenbares Guds Retfærdighed af Tro for Tro, som der er skrevet: "Men den retfærdige skal leve af Tro."
\par 18 Thi Guds Vrede åbenbares fra Himmelen over al Ugudelighed og Uretfærdighed hos Mennesker, som holde Sandheden nede ved Uretfærdighed;
\par 19 thi det, som man kan vide om Gud, er åbenbart iblandt dem; Gud har jo åbenbaret dem det.
\par 20 Thi hans usynlige Væsen, både hans evige Kraft og Guddommelighed, skues fra Verdens Skabelse af, idet det forstås af hans Gerninger, så at de have ingen Undskyldning.
\par 21 Thi skønt de kendte Gud, så ærede eller takkede de ham dog ikke som Gud, men bleve tåbelige i deres Tanker, og deres uforstandige Hjerte blev formørket.
\par 22 Idet de påstode at være vise, bleve de Dårer
\par 23 og omskiftede den uforkrænkelige Guds Herlighed med et Billede i Lighed med et forkrænkeligt Menneske og Fugle og firføddede og krybende dyr.
\par 24 Derfor gav Gud dem hen i deres Hjerters Begæringer til Urenhed, til at vanære deres Legemer indbyrdes,
\par 25 de, som ombyttede Guds Sandhed med Løgnen og dyrkede og tjente Skabningen fremfor Skaberen, som er højlovet i Evighed! Amen.
\par 26 Derfor gav Gud dem hen i vanærende Lidenskaber; thi både deres Kvinder ombyttede den naturlige Omgang med den unaturlige,
\par 27 og ligeså forlode også Mændene den naturlige Omgang med Kvinden og optændtes ideres Brynde efter hverandre, så at Mænd øvede Uterlighed med Mænd og fik deres Vildfarelses Løn, som det burde sig, på sig selv.
\par 28 Og ligesom de forkastede af have Gud i Erkendelse, således gav Gud dem hen i et forkasteligt Sind til at gøre det usømmelige,
\par 29 opfyldte med al Uretfærdighed, Ondskab, Havesyge, Slethed; fulde af Avind, Mord, Kiv, Svig, Ondsindethed;
\par 30 Øretudere, Bagvaskere, Gudshadere, Voldsmænd, hovmodige; Pralere, opfindsomme på ondt, ulydige mod Forældre,
\par 31 uforstandige, troløse, ukærlige, ubarmhjertige;
\par 32 - hvilke jo, skønt de erkende Guds retfærdige Dom, at de, der øve sådanne Ting, fortjene Døden, dog ikke alene gøre det, men også give dem, som øve det, deres Bifald.

\chapter{2}

\par 1 Derfor er du uden Undskyldning, o Menneske! hvem du end er, som dømmer; thi idet du dømmer den anden, fordømmer du dig selv; thi du, som dømmer, øver det samme.
\par 2 Vi vide jo, at Guds Dom er, stemmende med Sandhed, over dem, som øve sådanne Ting.
\par 3 Men du, o Menneske! som dømmer dem, der øve sådanne Ting, og selv gør dem, mener du dette, at du skal undfly Guds Dom?
\par 4 Eller foragter du hans Godheds og Tålmodigheds og Langmodigheds Rigdom og ved ikke, at Guds Godhed leder dig til Omvendelse?
\par 5 Men efter din Hårdhed og dit ubodfærdige Hjerte samler du dig selv Vrede på Vredens og Guds retfærdige Doms Åbenbarelses Dag,
\par 6 han, som vil betale enhver efter hans Gerninger:
\par 7 dem, som med Udholdenhed i god Gerning søge Ære og Hæder og Uforkrænkelighed, et evigt Liv;
\par 8 men over dem, som søge deres eget og ikke lyde Sandheden, men adlyde Uretfærdigheden, skal der komme Vrede og Harme.
\par 9 Trængsel og Angst over hvert Menneskes Sjæl, som øver det onde, både en Jødes først og en Grækers;
\par 10 men Ære og Hæder og Fred over hver den, som gør det gode, både en Jøde først og en Græker!
\par 11 Thi der er ikke Persons Anseelse hos Gud.
\par 12 Thi alle de, som have syndet uden Loven, de skulle også fortabes uden Loven; og alle de, som have syndet under Loven, de skulle dømmes ved Loven;
\par 13 thi ikke Lovens Hørere ere retfærdige for Gud, men Lovens Gørere skulle retfærdiggøres
\par 14 thi når Hedninger, som ikke have Loven, af Naturen gøre, hvad Loven kræver, da ere disse; uden at have Loven sig selv en Lov;
\par 15 de vise jo Lovens Gerning skreven i deres Hjerter, idet deres Samvittighed vidner med, og Tankerne indbyrdes anklage eller også forsvare hverandre
\par 16 på den Dag, da Gud vil dømme Menneskenes skjulte Færd ifølge mit Evangelium ved Jesus Kristus.
\par 17 Men når du kalder dig Jøde og forlader dig trygt på Loven og roser dig af Gud
\par 18 og kender hans Villie og værdsætter de forskellige Ting, idet du undervises af Loven,
\par 19 og trøster dig til at være blindes Vejleder, et Lys for dem, som ere i Mørke,
\par 20 uforstandiges Opdrager, umyndiges Lærer, idet du i Loven har Udtrykket for Erkendelsen og Sandheden,
\par 21 du altså som lærer andre, du lærer ikke dig selv! Du, som prædiker, at man ikke må stjæle, du stjæler!.
\par 22 Du, som siger, at man ikke må bedrive Hor, du bedriver Hor! Du, som føler Afsky for Afguderne, du øver Tempelran!
\par 23 Du, som roser dig af Loven, du vanærer Gud ved Overtrædelse af Loven!
\par 24 Thi "for eders Skyld bespottes Guds Navn iblandt Hedningerne", som der er skrevet.
\par 25 Thi vel gavner Omskærelse, om du holder Loven; men er du Lovens Overtræder, da er din Omskærelse bleven til Forhud.
\par 26 Dersom nu Forhuden holder Lovens Forskrifter, vil da ikke hans Forhud blive regnet som Omskærelse?
\par 27 Og når den af Natur uomskårne opfylder Loven, skal han dømme dig, som med Bogstav og Omskærelse er Lovens Overtræder.
\par 28 Thiikke den er Jøde, som er det i det udvortes, ej heller er det Omskærelse, som sker i det udvortes, i Kød;
\par 29 men den, som indvortes er Jøde, og Hjertets Omskærelse i Ånd, ikke i Bogstav - hans Ros er ikke af Mennesker, men af Gud,

\chapter{3}

\par 1 Hvad er da Jødens Fortrin? eller hvad gavner Omskærelsen?
\par 2 Meget alle Måder; først nemlig dette, at Guds Ord ere blevne dem betroede.
\par 3 Thi hvad? om nogle vare utro, skal da deres Utroskab gøre Guds Trofasthed til intet?
\par 4 Det være langt fra! Gud må være sanddru, om end hvert Menneske er en Løgner, som der er skrevet: "For at du må kendes retfærdig i dine Ord og vinde, når du går i Rette."
\par 5 Men dersom vor Uretfærdighed beviser Guds Retfærdighed, hvad skulle vi da sige? er Gud da uretfærdig, han, som lader sin Vrede komme? (Jeg taler efter menneskelig Vis).
\par 6 Det være langt fra! Thi hvorledes skal Gud ellers kunne dømme Verden?
\par 7 Men dersom Guds Sanddruhed ved min Løgn er bleven ham end mere til Forherligelse, hvorfor dømmes da jeg endnu som en Synder?
\par 8 Og hvorfor skulde vi da ikke, som man bagvasker os for, og som nogle sige, at vi lære, gøre det onde, for at det gode kan komme deraf? Sådannes Dom er velforskyldt.
\par 9 Hvad da? have vi noget forud? Aldeles ikke; vi have jo ovenfor anklaget både Jøder og Grækere for alle at være under Synd,
\par 10 som der er skrevet: "Der er ingen retfærdig, end ikke een;
\par 11 der er ingen forstandig, der er ingen, som søger efter Gud;
\par 12 alle ere afvegne, til Hobe ere de blevne uduelige, der er ingen, som øver Godhed, der er end ikke een."
\par 13 "En åbnet Grav er deres Strube; med deres Tunger øvede de Svig;" "der er Slangegift under deres Læber;"
\par 14 "deres Mund er fuld af Forbandelse og Beskhed;"
\par 15 "rappe ere deres Fødder til at udøse Blod;
\par 16 der er Ødelæggelse og Elendighed på deres Veje,
\par 17 og Freds Vej have de ikke kendt."
\par 18 "Der er ikke Gudsfrygt for deres Øjne."
\par 19 Men vi vide, at alt, hvad Loven siger, taler den til dem, som ere under Loven, for at hver Mund skal stoppes og hele Verden blive strafskyldig for Gud,
\par 20 efterdi intet Kød vil blive retfærdiggjort for ham af Lovens Gerninger; thi ved Loven kommer Erkendelse af Synd.
\par 21 Men nu er uden Lov Guds Retfærdighed åbenbaret, om hvilken der vidnes af Loven og Profeterne.
\par 22 nemlig Guds Retfærdighed ved Tro på Jesus Kristus, for alle og over alle dem, som tro; thi der er ikke Forskel.
\par 23 Alle have jo syndet, og dem fattes Æren fra Gud,
\par 24 og de blive retfærdiggjorte uforskyldt af hans Nåde ved den Forløsning, som er i Kristus Jesus,
\par 25 hvem Gud fremstillede som Sonemiddel ved Troen på hans Blod for at vise sin Retfærdighed, fordi Gud i sin Langmodighed havde båret over med de forhen begåede Synder,
\par 26 for at vise sin Retfærdighed i den nærværende Tid, for at han kunde være retfærdig og retfærdiggøre den, som er af Tro på Jesus.
\par 27 Hvor er så vor Ros? Den er udelukket. Ved hvilken Lov? Gerningernes? Nej, men ved Troens Lov.
\par 28 Vi holde nemlig for, at Mennesket bliver retfærdiggjort ved Tro, uden Lovens Gerninger.
\par 29 Eller er Gud alene Jøders Gud? mon ikke også Hedningers? Jo, også Hedningers;
\par 30 så sandt som Gud er een og vil retfærdiggøre omskårne af Tro og uomskårne ved Troen.
\par 31 Gøre vi da Loven til intet ved Troen? Det være langt fra! Nej, vi hævde Loven.

\chapter{4}

\par 1 Hvad skulle vi da sige, at vor Stamfader Abraham har vundet efter Kødet?
\par 2 Thi dersom Abraham blev retfærdiggjort af Gerninger, har han Ros, men ikke for Gud.
\par 3 Thi hvad siger Skriften?"Og Abraham troede Gud, og det blev regnet ham til Retfærdighed."
\par 4 Men den, som gør Gerninger, tilregnes Lønnen ikke som Nåde, men som Skyldighed;
\par 5 den derimod, som ikke gør Gerninger, men tror på ham, som retfærdiggør den ugudelige, regnes hans Tro til Retfærdighed;
\par 6 ligesom også David priser det Menneske saligt, hvem Gud tilregner Retfærdighed uden Gerninger:
\par 7 "Salige de, hvis Overtrædelser ere forladte, og hvis Synder ere skjulte;
\par 8 salig den Mand, hvem Herren ikke vil tilregne Synd."
\par 9 Gælder da denne Saligprisning de omskårne eller tillige de uomskårne? Vi sige jo: Troen blev regnet Abraham til Retfærdighed.
\par 10 Hvorledes blev den ham da tilregnet? da han var omskåren, eller da han havde Forhud? Ikke da han var omskåren, men da han havde Forhud.
\par 11 Og han fik Omskærelsens Tegn som et Segl på den Troens Retfærdighed, som han havde som uomskåren, for at han skulde være Fader til alle dem, som tro uden at være omskårne, for at Retfærdighed kan blive dem tilregnet,
\par 12 og Fader til omskårne,til dem, som ikke alene have Omskærelse, men også vandre i den Tros Spor, hvilken vor Fader Abraham havde som uomskåren.
\par 13 Thi ikke ved Lov fik Abraham eller hans Sæd den Forjættelse, at han skulde være Arving til Verden, men ved Tros-Retfærdighed.
\par 14 Thi dersom de, der ere af Loven, ere Arvinger, da er Troen bleven tom, og Forjættelsen gjort til intet.
\par 15 Thi Loven virker Vrede; men hvor der ikke er Lov, er der heller ikke Overtrædelse.
\par 16 Derfor er det af Tro, for at det skal være som Nåde, for at Forjættelsen må stå fast for den hele Sæd, ikke alene for den af Loven, men også for den af Abrahams Tro, han, som er Fader til os alle
\par 17 (som der er skrevet: "Jeg har sat dig til mange Folkeslags Fader"), over for Gud, hvem han troede, ham, som levendegør de døde og kalder det, der ikke er, som om det var.
\par 18 Og han troede imod Håb med Håb på, at, han skulde blive mange Folkeslags Fader, efter det, som var sagt: "Således skal din Sæd være;"
\par 19 og uden at blive svag i Troen så han på sit eget allerede udlevede Legeme (han var nær hundrede År) og på, at Saras Moderliv var udlevet;
\par 20 men om Guds Forjættelse tvivlede han ikke i Vantro, derimod blev han styrket i Troen, idet han gav Gud Ære
\par 21 og var overbevist om, at hvad han har forjættet, er han mægtig til også at gøre.
\par 22 Derfor blev det også regnet ham til Retfærdighed.
\par 23 Men det blev, ikke skrevet for hans Skyld alene, at det blev ham tilregnet,
\par 24 men også for vor Skyld, hvem det skal tilregnes, os, som tro på ham, der oprejste Jesus, vor Herre, fra de døde,
\par 25 ham, som blev hengiven for vore Overtrædelsers Skyld og oprejst for vor Retfærdiggørelses Skyld.

\chapter{5}

\par 1 Altså retfærdiggjorte af Tro have vi Fred med Gud ved vor Herre Jesus Krist,
\par 2 ved hvem vi også have fået Adgang ved Troen til denne Nåde, hvori vi stå, og vi rose os af Håb om Guds Herlighed;
\par 3 ja, ikke det alene, men vi rose os også af Trængslerne, idet vi vide, at Trængselen virker Udholdenhed,
\par 4 men Udholdenheden Prøvethed, men Prøvetheden Håb,
\par 5 men Håbet beskæmmer ikke; thi Guds Kærlighed er udøst i vore Hjerter ved den Helligånd, som blev given os.
\par 6 Thi medens vi endnu vare kraftesløse, døde Kristus til den bestemte Tid for ugudelige.
\par 7 Næppe vil nemlig nogen dø for en retfærdig - for den gode var der jo måske nogen, som tog sig på at dø -,
\par 8 men Gud beviser sin Kærlighed over for os, ved at Kristus døde for os, medens vi endnu vare Syndere.
\par 9 Så meget mere skulle vi altså, da vi nu ere blevne retfærdiggjorte ved hans Blod, frelses ved ham fra Vreden.
\par 10 Thi når vi, da vi vare Fjender, bleve forligte med Gud ved hans Søns Død, da skulle vi meget mere, efter at vi ere blevne forligte, frelses ved hans Liv,
\par 11 ja, ikke det alene, men også således, at vi rose os af Gud ved vor Herre Jesus Kristus, ved hvem vi nu have fået Forligelsen.
\par 12 Derfor, ligesom Synden kom ind i Verden ved eet Menneske, og Døden ved Synden, og Døden således trængte igennem til alle Mennesker, efterdi de syndede alle
\par 13 thi inden Loven var der Synd i Verden; men Synd tilregnes ikke.
\par 14 dog herskede Døden fra Adam til Moses også over dem, som ikke syndede i Lighed med Adams Overtrædelse, han, som er et Forbillede på den, der skulde komme.
\par 15 Men det er ikke således med Nådegaven som med Faldet; thi døde de mange ved den enes Fald, da har meget mere Guds Nåde og Gaven i det ene Menneskes Jesu Kristi Nåde udbredt sig overflødig, til de mange.
\par 16 Og Gaven er ikke som igennem en enkelt, der syndede; thi Dommen blev ud fra en enkelt til Fordømmelse, men Nådegaven blev ud fra mange Fald til Retfærdiggørelse.
\par 17 Thi når på Grund af dennes Fald Døden herskede ved den ene, da skulle meget mere de, som modtage den overvættes Nåde og Retfærdigheds Gave, herske i Liv ved den ene, Jesus Kristus.
\par 18 Altså, ligesom det ved eens Fald blev for alle Mennesker til Fordømmelse, således også ved eens Retfærdighed for alle Mennesker til Retfærdiggørelse til Liv.
\par 19 Thi ligesom ved det ene Menneskes Ulydighed de mange bleve til Syndere, så skulle også ved den enes Lydighed de mange blive til retfærdige.
\par 20 Men Loven kom til, for at Faldet kunde blive større; men hvor Synden blev større, der blev Nåden end mere overvættes,
\par 21 for at, ligesom Synden herskede ved Døden, således også Nåden skulde herske ved Retfærdighed til et evigt Liv ved Jesus Kristus, vor Herre.

\chapter{6}

\par 1 Hvad skulle vi da sige? skulde vi blive ved i Synden, for at Nåden kunde blive desto større?
\par 2 Det være langt fra! Vi, som jo ere døde fra Synden, hvorledes skulle vi endnu leve i den?
\par 3 Eller vide I ikke, at vi, så mange som bleve døbte til Kristus Jesus, bleve døbte til hans Død?
\par 4 Vi bleve altså begravne med ham ved Dåben til Døden, for at, ligesom Kristus blev oprejst fra de døde ved Faderens Herlighed, således også vi skulle vandre i et nyt Levned.
\par 5 Thi ere vi blevne sammenvoksede med ham ved hans Døds Afbillede, skulle vi dog også være det ved hans Opstandelses,
\par 6 idet vi erkende dette, at vort gamle Menneske blev korsfæstet med ham, for at Syndens Legeme skulde blive til intet, for at vi ikke mere skulde tjene Synden.
\par 7 Thi den, som er død, er retfærdiggjort fra Synden.
\par 8 Men dersom vi ere døde med Kristus, da tro vi, at vi også skulle leve med ham,
\par 9 efterdi vi vide, at Kristus, efter at være oprejst fra de døde, ikke mere dør; Døden hersker ikke mere over ham.
\par 10 Thi det, han døde, døde han een Gang fra Synden; men det, han lever, lever han for Gud.
\par 11 Således skulle også I anse eder selv for døde fra Synden, men levende for Gud i Kristus Jesus.
\par 12 Så lad da ikke Synden herske i eders dødelige Legeme, så I lyde dets Begæringer;
\par 13 fremstiller ej heller eders Lemmer for Synden som Uretfærdigheds Våben; men fremstiller eder selv for Gud som sådanne, der fra døde ere blevne levende,og eders Lemmer som Retfærdigheds Våben for Gud.
\par 14 Thi Synd skal ikke herske over eder I ere jo ikke under Lov, men under Nåde.
\par 15 Hvad da? skulde vi Synde, fordi vi ikke ere under Lov, men under Nåde? Det være langt fra!
\par 16 Vide I ikke, at når I fremstille eder for en som Tjenere til Lydighed, så ere I hans Tjenere, hvem I lyde, enten Syndens til Død, eller Lydighedens til Retfærdighed?
\par 17 Men Gud ske Tak, fordi I have været Syndens Tjenere, men bleve af Hjertet lydige imod den Læreform, til hvilken I bleve overgivne.
\par 18 Og frigjorde fra Synden bleve I Retfærdighedens Tjenere.
\par 19 Jeg taler på menneskelig Vis på Grund af eders Køds Skrøbelighed. Ligesom I nemlig fremstillede eders Lemmer som Tjenere for Urenheden og Lovløsheden til Lovløshed, således fremstiller nu eders Lemmer som Tjenere for Retfærdigheden, til Helliggørelse!
\par 20 Thi da I vare Syndens Tjenere, vare I frie over for Retfærdigheden.
\par 21 Hvad for Frugt havde I da dengang? Ting, ved hvilke I nu skamme eder; Enden derpå er jo Død.
\par 22 Men nu, da I ere blevne frigjorde fra Synden og ere blevne Guds Tjenere, have I eders Frugt til Helliggørelse og som Enden derpå et evigt Liv;
\par 23 thi Syndens Sold er Død, men Guds Nådegave er et evigt Liv i Kristus Jesus, vor Herre.

\chapter{7}

\par 1 Eller vide I ikke, Brødre! (thi jeg taler til sådanne, som kender Loven) at Loven hersker over Mennesket, så lang Tid han lever?
\par 2 Den gifte Kvinde er jo ved Loven bunden til sin Mand, medens han lever; men når Manden dør, er hun løst fra Mandens Lov.
\par 3 Derfor skal hun kaldes en Horkvinde, om hun bliver en anden Mands, medens Manden lever: men når Manden dør, er hun fri fra den Lov, så at hun ikke er en Horkvinde, om hun bliver en anden Mands.
\par 4 Altså ere også I, mine Brødre! gjorte døde for Loven ved Kristi Legeme, for at I skulle blive en andens, hans, som blev oprejst fra de døde, for at vi skulle bære Frugt for Gud.
\par 5 Thi da vi vare i Kødet, vare de syndige Lidenskaber, som vaktes ved Loven, virksomme i vore Lemmer til at bære Frugt for Døden,
\par 6 Men nu ere vi løste fra Loven, idet vi ere bortdøde fra det, hvori vi holdtes nede, så at vi tjene i Åndens nye Væsen og ikke i Bogstavens gamle Væsen.
\par 7 Hvad skulle vi da sige? er Loven Synd? Det være langt fra! Men jeg kendte ikke Synden uden ved Loven; thi jeg kendte jo ikke Begæringen, hvis ikke Loven sagde: "Du må ikke begære."
\par 8 Men da Synden fik Anledning, virkede den ved Budet al Begæring i mig; thi uden Lov er Synden død.
\par 9 Og jeg levede engang uden Lov, men da Budet kom, levede Synden op;
\par 10 men jeg døde, og Budet, som var til Liv, det fandtes at blive mig til Død;
\par 11 thi idet Synden fik Anledning, forførte den mig ved Budet og dræbte mig ved det.
\par 12 Altså er Loven vel hellig, og Budet helligt og retfærdigt og godt.
\par 13 Blev da det gode mig til Død? Det være langt fra! Men Synden blev det, for at den skulde vise sig som Synd, idet den ved det gode virkede Død for mig, for at Synden ved Budet skulde blive overvættes syndig.
\par 14 Thi vi vide, at Loven er åndelig; men jeg er kødelig, solgt under Synden.
\par 15 Thi jeg forstår ikke, hvad jeg udfører; thi ikke det, som jeg vil, øver jeg, men hvad jeg hader, det gør jeg.
\par 16 Men når jeg gør det, jeg ikke vil, så samstemmer jeg med Loven i, at den er god.
\par 17 Men nu er det ikke mere mig, som udfører det, men Synden, som bor i mig.
\par 18 Thi jeg ved, at i mig, det vil sige i mit Kød, bor der ikke godt; thi Villien har jeg vel, men at udføre det gode formår jeg ikke;
\par 19 thi det gode, som jeg vil, det gør jeg ikke; men det onde, som jeg ikke vil, det øver jeg.
\par 20 Dersom jeg da gør det, som jeg ikke vil, så er det ikke mere mig, der udfører det, men Synden, som bor i mig.
\par 21 Så finder jeg da den Lov for mig, som vil gøre det gode, at det onde ligger mig for Hånden
\par 22 Thi jeg glæder mig ved Guds Lov efter det indvortes Menneske;
\par 23 men jeg ser en anden Lov i mine Lemmer, som strider imod mit Sinds Lov og tager mig fangen under Syndens Lov, som er i mine Lemmer.
\par 24 Jeg elendige Menneske! hvem skal fri mig fra dette Dødens Legeme?
\par 25 Gud ske Tak ved Jesus Kristus, vor Herre! Altså: jeg selv tjener med Sindet Guds Lov, men med Kødet Syndens Lov.

\chapter{8}

\par 1 Så er der da nu ingen Fordømmelse for dem, som ere i Kristus Jesus.
\par 2 Thi Livets Ånds Lov frigjorde mig i Kristus Jesus fra Syndens og Dødens Lov.
\par 3 Thi det, som var Loven umuligt, det, hvori den var afmægtig ved Kødet, det gjorde Gud, idet han sendte sin egen Søn i syndigt Køds Lighed og for Syndens Skyld og således domfældte Synden i Kødet,
\par 4 for at Lovens Krav skulde opfyldes i os, som ikke vandre efter Kødet, men efter Ånden.
\par 5 Thi de, som lade sig lede af Kødet, hige efter det kødelige; men de, som lade sig lede af Ånden, hige efter det åndelige.
\par 6 Thi Kødets Higen er Død, men Åndens Higen er Liv og Fred,
\par 7 efterdi Kødets Higen er Fjendskab imod Gud, thi det er ikke Guds Lov lydigt, det kan jo ikke heller være det.
\par 8 Og de, som ere i Kødet, kunne ikke tækkes Gud.
\par 9 I derimod ere ikke i Kødet, men i Ånden, om ellers Guds Ånd bor i eder. Men om nogen ikke har Kristi Ånd, så hører han ham ikke til.
\par 10 Men om Kristus er i eder, da er vel Legemet dødt på Grund at Synd, men Ånden er Liv på Grund af Retfærdighed.
\par 11 Men om hans Ånd, der oprejste Jesus fra de døde, bor i eder, da skal han, som oprejste Kristus fra de døde, levendegøre også eders dødelige Legemer ved sin Ånd, som bor i eder.
\par 12 Altså, Brødre! ere vi ikke Kødets Skyldnere, så at vi skulde leve efter Kødet;
\par 13 thi dersom I leve efter Kødet, skulle I dø, men dersom l ved Ånden døde Legemets Gerninger, skulle I leve.
\par 14 Thi så mange som drives af Guds Ånd, disse ere Guds Børn.
\par 15 I modtoge jo ikke en Trældoms Ånd atter til Frygt, men I modtoge en Sønneudkårelses Ånd, i hvilken vi råbe: Abba, Fader!
\par 16 Ånden selv vidner med vor Ånd, at vi ere Guds Børn.
\par 17 Men når vi ere Børn, ere vi også Arvinger, Guds Arvinger og Kristi Medarvinger, om ellers vi lide med ham for også at herliggøres med ham.
\par 18 Thi jeg holder for, at den nærværende Tids Lidelser ikke ere at regne imod den Herlighed, som skal åbenbares på os.
\par 19 Thi Skabningens Forlængsel venter på Guds Børns Åbenbarelse.
\par 20 Thi Skabningen blev underlagt Forfængeligheden, ikke med sin Villie, men for hans Skyld, som lagde den derunder,
\par 21 med Håb om, at også Skabningen selv skal blive frigjort fra Forkrænkelighedens Trældom til Guds Børns Herligheds Frihed.
\par 22 Thi vi vide, at hele Skabningen tilsammen sukker og er tilsammen i Veer indtil nu.
\par 23 Dog ikke det alene, men også vi selv, som have Åndens Førstegrøde, også vi sukke ved os selv, idet vi forvente en Sønneudkårelse, vort Legemes Forløsning.
\par 24 Thi i Håbet bleve vi frelste. Men et Håb, som ses, er ikke et Håb; thi hvad en ser, hvor kan han tillige håbe det?
\par 25 Men dersom vi håbe det, som vi ikke se, da forvente vi det med Udholdenhed.
\par 26 Og ligeledes kommer også Ånden vor Skrøbelighed til Hjælp; thi vi vide ikke, hvad vi skulle bede om, som det sig bør, men Ånden selv går i Forbøn for os med uudsigelige Sukke.
\par 27 Og han, som ransager Hjerterne, ved, hvad Åndens Higen er, at den efter Guds Villie går i Forbøn for hellige.
\par 28 Men vi vide, at alle Ting samvirke til gode for dem, som elske Gud, dem, som efter hans Beslutning ere kaldede.
\par 29 Thi dem, han forud kendte, forudbestemte han også til at blive ligedannede med hans Søns Billede, for at han kunde være førstefødt iblandt mange Brødre.
\par 30 Men dem, han forudbestemte, dem kaldte han også; og dem, han kaldte, dem retfærdiggjorde han også; men dem, han retfærdiggjorde, dem herliggjorde han også.
\par 31 Hvad skulle vi da sige til dette? Er Gud for os, hvem kan da være imod os?
\par 32 Han, som jo ikke sparede sin egen Søn, men gav ham hen for os alle, hvorledes skulde han ikke også med ham skænke os alle Ting?
\par 33 Hvem vil anklage Guds udvalgte? Gud er den, som retfærdiggør.
\par 34 Hvem er den, som fordømmer? Kristus er den, som er død, ja, meget mere, som er oprejst, som er ved Guds højre Hånd, som også går i Forbøn for os.
\par 35 Hvem skal kunne skille os fra Kristi Kærlighed? Trængsel eller Angst eller Forfølgelse eller Hunger eller Nøgenhed eller Fare eller Sværd?
\par 36 som der er skrevet: "For din Skyld dræbes vi den hele Dag, vi bleve regnede som Slagtefår."
\par 37 Men i alt dette mere end sejre vi ved ham, som elskede os.
\par 38 Thi jeg er vis på, at hverken Død eller Liv eller Engle eller Magter eller noget nærværende eller noget tilkommende eller Kræfter
\par 39 eller det høje eller det dybe eller nogen anden Skabning skal kunne skille os fra Guds Kærlighed i Kristus Jesus, vor Herre.

\chapter{9}

\par 1 Sandhed siger jeg i Kristus, jeg lyver ikke, min Samvittighed vidner med mig i den Helligånd,
\par 2 at jeg har en stor Sorg og en uafladelig Kummer i mit Hjerte.
\par 3 Thi jeg kunde ønske selv at være bandlyst fra Kristus til Bedste for mine Brødre, mine Frænder efter Kødet,
\par 4 de, som jo ere Israeliter, hvem Sønneudkårelsen og Herligheden og Pagterne og Lovgivningen og Gudstjenesten og Forjættelserne tilhøre,
\par 5 hvem Fædrene tilhøre, og af hvem Kristus er efter Kødet, han, som er Gud over alle Ting, højlovet i Evighed! Amen.
\par 6 Ikke dog som om Guds Ord har glippet; thi ikke alle, som stamme fra Israel, ere Israel;
\par 7 ej, heller ere alle Børn, fordi de ere Abrahams Sæd, men: "I Isak skal en Sæd få Navn efter dig."
\par 8 Det vil sige: Ikke Kødets Børn ere Guds Børn, men Forjættelsens Børn regnes for Sæd.
\par 9 Thi et Forjættelsesord er dette: "Ved denne Tid vil jeg komme, så skal Sara have en Søn."
\par 10 Men således skete det ikke alene dengang, men også med Rebekka, da hun var frugtsommelig ved een, Isak, vor Fader.
\par 11 Thi da de endnu ikke vare fødte og ikke havde gjort noget godt eller ondt, blev der, for at Guds Udvælgelses Beslutning skulde stå fast, ikke i Kraft af Gerninger, men i Kraft af ham, der kalder,
\par 12 sagt til hende: ""Den ældste skal tjene den yngste,""
\par 13 som der er skrevet: ""Jakob elskede jeg, men Esau hadede jeg."
\par 14 Hvad skulle vi da sige? mon der er Uretfærdighed hos Gud? Det være langt fra!
\par 15 Thi han siger til Moses: "Jeg vil være barmhjertig imod den, hvem jeg er barmhjertig imod, og forbarme mig over den, hvem jeg forbarmer mig over."
\par 16 Altså står det ikke til den, som vil, ej heller til den, som løber, men til Gud, som er barmhjertig.
\par 17 Thi Skriften siger til Farao: "Netop derfor lod jeg dig fremstå, for at jeg kunde vise min Magt på dig, og for at mit Navn skulde forkyndes på hele Jorden."
\par 18 Så forbarmer han sig da over den, som han vil, men forhærder den, som han vil.
\par 19 Du vil nu sige til mig: Hvad klager han da over endnu? thi hvem står hans Villie imod?
\par 20 Ja, men, hvem er dog du, o Menneske! som går i Rette med Gud? mon noget, som blev dannet, kan sige til den, som dannede det: Hvorfor gjorde du mig således?
\par 21 Eller har Pottemageren ikke Rådighed over Leret til af den samme Masse at gøre et Kar til Ære, et andet til Vanære?
\par 22 Men hvad om nu Gud, skønt han vilde vise sin Vrede og kundgøre sin Magt, dog med stor Langmodighed tålte Vredes-Kar, som vare beredte til Fortabelse,
\par 23 også for at kundgøre sin Herligheds Rigdom over Barmhjertigheds-Kar, som han forud havde beredt til Herlighed?
\par 24 Og hertil kaldte han også os, ikke alene af Jøder, men også af Hedninger,
\par 25 som han også siger hos Hoseas: "Det, som ikke var mit Folk, vil jeg kalde mit Folk, og hende, som ikke var den elskede; den elskede;
\par 26 og det skal ske, at på det Sted, hvor der blev sagt til dem: I ere ikke mit Folk, der skulle de kaldes den levende Guds Børn."
\par 27 Men Esajas udråber over Israel: "Om end Israels Børns Tal var som Havets Sand, så skal kun Levningen frelses.
\par 28 Thi idet Herren opgør Regnskab og afslutter det i Hast, vil han fuldbyrde det på Jorden."
\par 29 Og som Esajas forud har sagt: "Dersom den Herre Zebaoth ikke havde levnet os en Sæd, da vare vi blevne som Sodoma og gjorte lige med Gomorra."
\par 30 Hvad skulle vi da sige? At Hedninger, som ikke jagede efter Retfærdighed, fik Retfærdighed; nemlig Retfærdigheden af Tro;
\par 31 men Israel, som jagede efter en Retfærdigheds Lov, nåede ikke til en sådan Lov.
\par 32 Hvorfor? fordi de ikke søgte den af Tro, men som af Geringer. De stødte an på Anstødsstenen,
\par 33 som der er skrevet: "Se, jeg sætter i Zion en Anstødssten og en Forargelses Klippe; og den, som tror på ham, skal ikke blive til Skamme."

\chapter{10}

\par 1 Brødre!mit Hjertes Ønske og Bøn til Gud for dem er om deres Frelse.
\par 2 Thi jeg giver dem det Vidnesbyrd, at de have Nidkærhed for Gud, men ikke med Forstand;
\par 3 thi da de ikke kendte Guds. Retfærdighed og tragtede efter at opstille deres egen Retfærdighed, så bøjede de sig ikke under Guds Retfærdighed.
\par 4 Thi Kristus er Lovens Ende til Retfærdighed for hver den, som tror.
\par 5 Moses skriver jo, at det Menneske, som gør den Retfærdighed, der er af Loven, skal leve ved den.
\par 6 Men Retfærdigheden af Tro siger således: Sig ikke i dit Hjerte: Hvem vil fare op til Himmelen? nemlig for at hente Kristus ned;
\par 7 eller: Hvem vil fare ned i Afgrunden? nemlig for at hente Kristus op fra de døde.
\par 8 Men hvad,siger den? Ordet er dig nær, i din Mund og i dit Hjerte, det er det Troens Ord, som vi prædike.
\par 9 Thi dersom du med din Mund bekender Jesus som Herre og tror i dit Hjerte, at Gud oprejste ham fra de døde, da skal du blive frelst.
\par 10 Thi med Hjertet tror man til Retfærdighed, og med Munden bekender man til Frelse.
\par 11 Skriften siger jo: "Hver den, som tror på ham, skal ikke blive til Skamme."
\par 12 Thi der er ikke Forskel på Jøde og Græker; thi den samme er alles Herre, rig nok for alle dem, som påkalde ham.
\par 13 Thi hver den, som påkalder Herrens Navn, skal blive frelst.
\par 14 Hvorledes skulde de nu påkalde den, på hvem de ikke have troet? og hvorledes skulde de tro den, som de ikke have hørt? og hvorledes skulde de høre, uden der er nogen, som prædiker?
\par 15 og hvorledes skulde de prædike, dersom de ikke bleve udsendte? Som der er skrevet: "Hvor dejlige. ere deres Fødder, som forkynde godt Budskab."
\par 16 Dog ikke alle løde Evangeliet; thi Esajas siger: "Herre! hvem troede det, (han hørte af os?")
\par 17 Altså kommer Troen af det. som høres, men det, som høres, kommer igennem Kristi Ord.
\par 18 Men jeg siger: Have de ikke hørt? Jo vist, "over hele Jorden er deres Røst udgået og til Jorderiges Grænser deres Ord."
\par 19 Men jeg siger: Har Israel ikke forstået det? Først siger Moses: "Jeg vil gøre eder nidkære på et Folk, som ikke er et Folk, imod et uforstandigt Folk vil jeg opirre eder."
\par 20 Men Esajas drister sig til at sige: "Jeg blev funden af dem, som ikke søgte mig; jeg blev åbenbar for dem. som ikke spurgte efter mig."
\par 21 Men om Israel siger han: "Den hele Dag udstrakte jeg mine Hænder imod et ulydigt og genstridigt Folk."

\chapter{11}

\par 1 Jeg siger da: Mon Gud har forskudt sit folk? det være langt fra! Thi også jeg er en Israelit, af "Abrahams Sæd, Benjamins Stamme.
\par 2 Gud har ikke forskudt sit Folk, som han forud kendte. Eller vide I ikke, hvad Skriften siger i Stykket om Elias? hvorledes han træder frem for Gud imod Israel, sigende:
\par 3 "Herre! dine Profeter have de ihjelslået, dine Altre have de nedbrudt, og jeg er den eneste, der er levnet, og de efterstræbe mit Liv."
\par 4 Men hvad siger det guddommelige Gensvar til ham?"Jeg har levnet mig selv syv Tusinde Mænd, som ikke have bøjet Knæ for Bål."
\par 5 Således er der også i den nærværende Tid blevet en Levning som et Nådes-Udvalg.
\par 6 Men er det af Nåde, da er det ikke mere af Gerninger, ellers bliver Nåden ikke mere Nåde.
\par 7 Hvad altså? Det, Israel søger efter, har det ikke opnået, men Udvalget har opnået det; de øvrige derimod bleve forhærdede,
\par 8 som der er skrevet: "Gud gav dem en Sløvheds Ånd, Øjne til ikke at se med, Øren til ikke at høre med indtil den Dag i Dag."
\par 9 Og David siger: "Deres Bord vorde til Snare og til Fælde og til Anstød og til Gengældelse for dem;
\par 10 deres Øjne vorde formørkede, så de ikke se, og bøj altid deres Ryg!"
\par 11 Jeg siger da: Mon de have stødt an, for at de skulde falde? Det være langt fra! Men ved deres Fald er Frelsen kommen til Hedningerne, for at dette kunde vække dem til Nidkærhed.
\par 12 Men dersom deres Fald er Verdens Rigdom, og deres Tab er Hedningers Rigdom, hvor meget mere skal deres Fylde være det!
\par 13 Og til eder, I Hedninger, siger jeg: For så vidt jeg nu er Hedningeapostel, ærer jeg min Tjeneste,
\par 14 om jeg dog kunde vække min Slægt til Nidkærhed og frelse nogle af dem.
\par 15 Thi dersom deres Forkastelse er Verdens Forligelse, hvad bliver da deres Antagelse andet end Liv ud af døde?
\par 16 Men dersom Førstegrøden er hellig, da er Dejgen det også; og dersom Roden er hellig, da ere Grenene det også.
\par 17 Men om nogle af Grenene bleve afbrudte, og du, en vild Oliekvist, blev indpodet iblandt dem og blev meddelagtig i Olietræets Rod og Fedme,
\par 18 da ros dig ikke imod Grenene; men dersom du roser dig, da bærer jo ikke du Roden, men Roden dig.
\par 19 Du vil vel sige: Grene bleve afbrudte, for at jeg skulde blive indpodet.
\par 20 Vel! ved deres Vantro bleve de afbrudte, men du står ved din Tro; vær ikke overmodig, men frygt!
\par 21 Thi når Gud ikke sparede de naturlige Grene, vil han heller ikke spare dig.
\par 22 Så se da Guds Godhed og Strenghed: Over dem, som faldt, er der Strenghed, men over dig Guds Godhed, hvis du bliver i hans Godhed; ellers skal også du afhugges.
\par 23 Men også hine skulle indpodes, dersom de ikke blive i Vantroen; thi Gud er mægtig til atter at indpode dem.
\par 24 Thi når du blev afhugget af det Olietræ, som er vildt af Naturen, og imod Naturen blev indpodet i et ædelt Olietræ, hvor meget mere skulle da disse indpodes i deres eget Olietræ, som de af Natur tilhøre!
\par 25 Thi jeg vil ikke, Brødre! at I skulle være uvidende om denne Hemmelighed, for at I ikke skulle være kloge i eders egne Tanker, at Forhærdelse delvis er kommen over Israel, indtil Hedningernes Fylde er gået ind;
\par 26 og så skal hele Israel frelses, som der er skrevet: "Fra Zion skal Befrieren komme, han skal afvende Ugudeligheder fra Jakob;
\par 27 og dette er min Pagt med dem, når jeg borttager deres Synder."
\par 28 Efter Evangeliet er de vel Fjender for eders Skyld, men efter Udvælgelsen ere de elskede for Fædrenes Skyld;
\par 29 thi Nådegaverne og sit Kald fortryder Gud ikke.
\par 30 Thi ligesom I tilforn bleve ulydige imod Gud, men nu fik Barmhjertighed ved disses Ulydighed,
\par 31 således bleve også disse nu ulydige, for at også de måtte få Barmhjertighed ved den Barmhjertighed, som er bleven eder til Del.
\par 32 Thi Gud har indesluttet alle under Ulydighed, for at han kunde forbarme sig over alle.
\par 33 O Dyb af Guds Rigdom og Visdom og Kundskab! hvor uransagelige ere hans Domme, og hans Veje usporlige!
\par 34 Thi hvem har kendt Herrens Sind? eller hvem blev hans Rådgiver?
\par 35 eller hvem gav ham først, så at der skulde gives ham Gengæld derfor?
\par 36 Thi af ham og ved ham og til ham ere alle Ting; ham være Ære i Evighed! Amen.

\chapter{12}

\par 1 Jeg formaner eder altså, Brødre! ved Guds Barmhjertighed, til at fremstille eder Legemer som et levende, helligt, Gud velbehageligt Offer; dette er eders fornuftige Gudsdyrkelse.
\par 2 Og skikker eder ikke lige med denne Verden, men vorder forvandlede ved Sindets Fornyelse, så I må skønne, hvad der er Guds Villie, det gode og velbehagelige og fuldkomne.
\par 3 Thi ved den Nåde, som er given mig, siger jeg til enhver iblandt eder, at han ikke skal tænke højere om sig selv, end han bør tænke, men tænke med Betænksomhed, efter som Gud tildelte enhver Troens Mål.
\par 4 Thi ligesom vi have mange Lemmer på eet Legeme, men Lemmerne ikke alle have den samme Gerning,
\par 5 således ere vi mange eet Legeme i Kristus, men hver for sig hverandres Lemmer.
\par 6 Men efterdi vi have forskellige Nådegaver efter den Nåde, som er given os, det være sig Profeti, da lader os bruge den i Forhold til vor Tro;
\par 7 eller en Tjeneste, da lader os tage Vare på Tjenesten; eller om nogen lærer, på Lærergerningen;
\par 8 eller om nogen formaner, på Formaningen; den, som uddeler, gøre det med Redelighed; den, som er Forstander, være det med Iver; den, som øver Barmhjertighed, gøre det med Glæde!
\par 9 Kærligheden være uskrømtet; afskyer det onde, holder eder til det gode;
\par 10 værer i eders Broderkærlighed hverandre inderligt hengivne; forekommer hverandre i at vise Ærbødighed!
\par 11 Værer ikke lunkne i eders Iver; værer brændende i Ånden; tjener Herren;
\par 12 værer glade i Håbet, udholdende i Trængselen, vedholdende i Bønnen!
\par 13 Tager Del i de helliges Fornødenheder; lægger Vind på Gæstfrihed!
\par 14 Velsigner dem, som forfølge eder, velsigner og forbander ikke!
\par 15 Glæder eder med de glade, og græder med de grædende!
\par 16 Værer enige indbyrdes; tragter ikke efter de høje Ting, men holder eder til det lave; vorder ikke kloge i eders egne Tanker!
\par 17 Betaler ikke nogen ondt for ondt; lægger Vind på, hvad der er godt for alle Menneskers Åsyn!
\par 18 Dersom det er muligt - såvidt det står til eder - da holder Fred med alle Mennesker:
\par 19 Hævner eder ikke selv, I elskede! men giver Vreden Rum; thi der er skrevet: "Mig hører Hævnen til, jeg vil betale, siger Herren."
\par 20 Nej, dersom din Fjende hungrer, giv ham Mad; dersom han tørster, giv ham Drikke; thi når du gør dette, vil du samle gloende Kul på hans Hoved.
\par 21 Lad dig ikke overvinde af det onde, men overvind det onde med det gode!

\chapter{13}

\par 1 Hver Sjæl underordne sig de foresatte Øvrigheder; thi der er ikke Øvrighed uden af Gud, men de, som ere, ere indsatte af Gud,
\par 2 så at den, som sætter sig imod Øvrigheden, modstår Guds Ordning; men de, som modstå, skulle få deres Dom.
\par 3 Thi de styrende ere ikke en Skræk for den gode Gerning, men for den onde. Men vil du være uden Frygt for Øvrigheden, så gør det gode, og du skal få Ros af den.
\par 4 Thi den er en Guds Tjener, dig til gode. Men dersom du gør det onde, da frygt; thi den bærer ikke Sværdet forgæves; den er nemlig Guds Tjener, en Hævner til Straf for den, som øver det onde.
\par 5 Derfor er det nødvendigt at underordne sig, ikke alene for Straffens Skyld, men også for Samvittighedens.
\par 6 Derfor betale I jo også Skatter; thi de ere Guds Tjenere, som just tage Vare på dette.
\par 7 Betaler alle, hvad I ere dem skyldige: den, som I ere Skat skyldige, Skat; den, som Told, Told; den, som Frygt, Frygt; den, som Ære, Ære.
\par 8 Bliver ingen noget skyldige, uden det, at elske hverandre; thi den, som elsker den anden, har opfyldt Loven.
\par 9 Thi det: "Du må ikke bedrive Hor; du må ikke slå ihjel; du må ikke stjæle; du må ikke begære," og hvilket andet bud der er, det sammenfattes i dette Ord: "Du skal elske din Næste som dig selv,"
\par 10 Kærligheden gør ikke ondt imod Næsten; derfor er Kærligheden Lovens Fylde.
\par 11 Og dette just, fordi I kende Tiden, at det alt er på Tide, at I skulle stå op af Søvne; thi nu er vor Frelse nærmere, end da vi bleve troende.
\par 12 Natten er fremrykket, og Dagen er kommen nær. Lader os derfor aflægge Mørkets Gerninger og iføre os Lysets Våben;
\par 13 lader os vandre sømmeligt som om Dagen, ikke i Svir og Drik, ikke i Løsagtighed og Uterlighed, ikke i Kiv og Avind;
\par 14 men ifører eder den Herre Jesus Kristus, og drager ikke Omsorg for Kødet, så Begæringer vækkes!

\chapter{14}

\par 1 Men tager eder af den, som er skrøbelig i Troen, og dømmer ikke hans Meninger!
\par 2 En har Tro til at spise alt; men den skrøbelige spiser kun Urter.
\par 3 Den, som spiser, må ikke ringeagte den, som ikke spiser; og den, som ikke spiser, må ikke dømme den, som spiser; thi Gud har taget sig af ham.
\par 4 Hvem er du, som dømmer en andens Tjener? For sin egen Herre står eller falder han; men han skal blive stående, thi Herren er mægtig til at lade ham stå.
\par 5 En agter den ene Dag fremfor den anden, en anden agter alle dage lige; enhver have fuld Vished i sit eget Sind!
\par 6 Den, som lægger Vægt på Dagen, han gør det for Herren. Og den, som spiser, gør det for Herren, thi han takker Gud; og den, som ikke spiser, gør det for Herren og takker Gud.
\par 7 Thi ingen af os lever for sig selv, og ingen dør for sig selv;
\par 8 thi når vi leve, leve vi for Herren, og når vi dø, dø vi for Herren; derfor, enten vi leve, eller vi dø, ere vi Herrens.
\par 9 Dertil er jo Kristus død og bleven levende, at han skal herske både over døde og levende.
\par 10 Men du, hvorfor dømmer du din Broder? eller du, hvorfor ringeagter du din Broder? Vi skulle jo alle fremstilles for Guds Domstol.
\par 11 Thi der er skrevet: "Så sandt jeg lever, siger Herren, for mig skal hvert Knæ bøje sig, og hver Tunge skal bekende Gud."
\par 12 Altså skal hver af os gøre Gud Regnskab for sig selv.
\par 13 Derfor,lader os ikke mere dømme hverandre, men dømmer hellere dette, at man ikke må give sin Broder Anstød eller Forargelse.
\par 14 Jeg ved og er vis på i den Herre Jesus, at intet er urent i sig selv; dog, for den, som agter noget for urent, for ham er det urent.
\par 15 Thi dersom din Broder bedrøves for Mads Skyld, da vandrer du ikke mere i Kærlighed. Led ikke ved din Mad den i Fordærvelse, for hvis Skyld Kristus er død.
\par 16 Lader derfor ikke eders Gode blive bespottet!
\par 17 Thi Guds Rige består ikke i at spise og drikke, men i Retfærdighed og Fred og Glæde i den Helligånd.
\par 18 Thi den, som deri tjener Kristus, er velbehagelig for Gud og tækkelig for Menneskene.
\par 19 Derfor, lader os tragte efter det, som tjener til Fred og indbyrdes Opbyggelse!
\par 20 Nedbryd ikke Guds Værk for Mads Skyld! Vel er alt rent, men det er ondt for det Menneske, som spiser med Anstød.
\par 21 Det er rigtigt ikke at spise Kød eller at drikke Vin eller at gøre noget, hvoraf din Broder tager Anstød.
\par 22 Den Tro, du har, hav den hos dig selv for Gud! Salig er den, som ikke dømmer sig selv i det, som han vælger.
\par 23 Men den, som tvivler, når han spiser, han er domfældt, fordi det ikke er at Tro; men alt det, som ikke er af Tro, er Synd.

\chapter{15}

\par 1 Men vi, som ere stærke, bør bære de svages Skrøbeligheder og ikke være os selv til Behag.
\par 2 Enhver af os være sin Næste til Behag til det gode, til Opbyggelse.
\par 3 Thi også Kristus var ikke sig selv til Behag; men, som der er skrevet: "Deres Forhånelser, som håne dig, ere faldne på mig."
\par 4 Thi alt, hvad der er skrevet tilforn, det er skrevet til vor Belæring, for at vi skulle have Håbet ved Udholdenheden og Skrifternes Trøst.
\par 5 Men Udholdenhedensog Trøstens Gud give eder at være enige indbyrdes, som Kristus Jesus vil det,
\par 6 for at I endrægtigt med een Mund kunne prise Gud og vor Herres Jesu Kristi Fader.
\par 7 Derfor tager eder af hverandre, ligesom også Kristus har taget sig af os, til Guds Ære.
\par 8 Jeg siger nemlig, at Kristus er bleven Tjener for omskårne for Guds Sanddruheds Skyld for at stadfæste Forjættelserne til Fædrene;
\par 9 men at Hedningerne skulle prise Gud for hans Barmhjertigheds Skyld, som der er skrevet: "Derfor vil jeg bekende dig iblandt Hedninger og lovsynge dit Navn,"
\par 10 Og atter siges der: "Fryder eder, I Hedninger, med hans Folk!"
\par 11 Og atter: "Lover Herren, alle Hedninger, og alle Folkene skulle prise ham."
\par 12 Og atter siger Esajas: "Komme skal Isajs Rodskud og han, der rejser sig for at herske over Hedninger; på ham skulle Hedninger håbe."
\par 13 Men Håbets Gud fylde eder med al Glæde og Fred, idet I tro, for at I må blive rige i Håbet ved den Helligånds Kraft!
\par 14 Men også jeg, mine Brødre! har selv den Forvisning om eder, at I også selv ere fulde af Godhed, fyldte med al Kundskab, i Stand til også at påminde hverandre.
\par 15 Dog har jeg for en Del tilskrevet eder noget dristigere for at påminde eder på Grund af den Nåde, som er given mig fra Gud
\par 16 til iblandt Hedningerne at være en Kristi Jesu Offertjener, der som Præst betjener Guds Evangelium, for at Hedningerne må blive et velbehageligt Offer, helliget ved den Helligånd.
\par 17 Således har jeg min Ros i Kristus Jesus af min Tjeneste for Gud.
\par 18 Thi jeg vil ikke driste mig til at tale om noget af det, som Kristus ikke har udført ved mig til at virke Hedningers Lydighed, ved Ord og Handling,
\par 19 ved Tegns og Undergerningers Kraft, ved Guds Ånds Kraft, så at jeg fra Jerusalem og trindt omkring indtil Illyrien har til fulde forkyndt Kristi Evangelium;
\par 20 dog således, at jeg sætter min Ære i at forkynde Evangeliet ikke der, hvor Kristus er nævnet, for at jeg,ikke skal bygge på en andens Grundvold,
\par 21 men, som der er skrevet: "De, for hvem der ikke blev kundgjort om ham, skulle se, og de, som ikke have hørt, skulle forstå."
\par 22 Derfor er jeg også de mange Gange bleven forhindret i at komme til eder.
\par 23 Men nu, da jeg ikke mere har Rum i disse Egne og i mange År har haft Længsel efter at komme til eder,
\par 24 vil jeg, når jeg rejser til Spanien, komme til eder; thi jeg håber at se eder på Gennemrejsen og af eder at blive befordret derhen, når jeg først i nogen Måde er bleven tilfredsstillet hos eder.
\par 25 Men nu rejser jeg til Jerusalem i Tjeneste for de hellige.
\par 26 Thi Makedonien og Akaja have fundet Glæde i at gøre et Sammenskud til de fattige iblandt de hellige i Jerusalem.
\par 27 De have nemlig fundet Glæde deri, og de ere deres Skyldnere. Thi ere Hedningerne blevne delagtige i hines åndelige Goder, da ere de også skyldige at tjene dem med de timelige.
\par 28 Når jeg da har fuldbragt dette og beseglet denne Frugt for dem, vil jeg derfra drage om ad eder til Spanien.
\par 29 Men jeg ved, at når jeg kommer til eder, skal jeg komme med Kristi Velsignelses Fylde.
\par 30 Men jeg formaner eder,Brødre! ved vor Herre Jesus Kristus og ved Åndens Kærlighed til med mig at stride i eders Bønner for mig til Gud,
\par 31 for at jeg må udfries fra de genstridige i Judæa, og mit Ærinde til Jerusalem må blive de hellige kærkomment,
\par 32 for at jeg kan komme til eder med Glæde, ved Guds Villie, og vederkvæges med eder.
\par 33 Men Fredens Gud være med eder alle! Amen.

\chapter{16}

\par 1 Men jeg anbefaler eder Føbe, vor Søster, som er Tjenerinde ved Menighed i Kenkreæ,
\par 2 for at I må modtage hende i Herren, som det sømmer sig de hellige, og yde hende Bistand, i hvad som helst hun måtte trænge til eder; thi også hun har været en Hjælperske for mange og for mig selv med.
\par 3 Hilser Priska og Akvila, mine Medarbejdere i Kristus Jesus,
\par 4 som jo for mit Liv have sat deres egen Hals i Vove, hvem ikke alene jeg takker, men også alle Hedningernes Menigheder;
\par 5 og hilser Menigheden i deres Hus! Hilser Epænetus, min elskede, som er Asiens Førstegrøde for Kristus.
\par 6 Hilser Maria, som har arbejdet meget for eder.
\par 7 Hilser Andronikus og Junias, mine Frænder og mine medfangne, som jo ere navnkundige iblandt Apostlene og tilmed have været i Kristus før mig.
\par 8 Hilser Ampliatus, min elskede i Herren!
\par 9 Hilser Urbanus, vor Medarbejder i Kristus, og Stakys, min elskede!
\par 10 Hilser Apelles, den prøvede i Kristus. Hilser dem, som ere af Aristobulus's Hus.
\par 11 Hilser Herodion, min Frænde! Hilser dem af Narkissus's Hus, som ere i Herren.
\par 12 Hilser Tryfæna og Tryfosa, som arbejde i Herren. Hilser; Persis, den elskede, som jo har arbejdet meget i Herren.
\par 13 Hilser Rufus, den udvalgte i Herren, og hans og min Moder!
\par 14 Hilser Asynkritus,Flegon, Hermes, Patrobas, Hermas og Brødrene hos dem!
\par 15 Hilser Filologus og Julia, Nereus og hans Søster og Olympas og alle de hellige hos dem!
\par 16 Hilser hverandre med et helligt Kys! Alle Kristi Menigheder hilse eder!
\par 17 Men jeg formaner eder, Brødre! til at give Agt på dem, som volde Splittelseme og Forargelseme tvært imod den Lære, som I have lært, og viger bort fra dem!
\par 18 Thi sådanne tjene ikke vor Herre Kristus, men deres egen Bug, og ved søde Ord og skøn Tale forføre de troskyldiges Hjerter.
\par 19 Eders Lydighed er jo kommen alle for Øre; derfor glæder jeg mig over eder. Men jeg vil, at I skulle være vise med Hensyn til det gode og enfoldige med Hensyn til det onde.
\par 20 Men Fredens Gud skal hastelig knuse Satan under eders Fødder. Vore Herres Jesu Kristi Nåde være med eder!
\par 21 Timotheus, min Medarbejder, og Lukius og Jason og Sosipater, mine Frænder, hilse eder.
\par 22 Jeg,Tertius, som har nedskrevet dette Brev, hilser eder i Herren.
\par 23 Kajus, min og den hele Menigheds Vært, hilser eder. Erastus, Stadens Rentemester, hilser eder, og Broderen Kvartus.
\par 24 Vor Herres Jesu Kristi Nåde være med eder alle! Amen.)
\par 25 Men ham, som kan styrke eder i mit Evangelium og Forkyndelsen af Jesus Kristus, i Overensstemmelse med Åbenbarelse af en Hemmelighed, som var fortiet fra evige Tider,
\par 26 men nu er bragt for Dagen og ved profetiske Skrifter efter den evige Guds Befaling kundgjort for alle Hedningerne til TrosLydighed:
\par 27 Den ene vise Gud ved Jesus Kristus, ham være Ære i Evigheders Evighed! Amen.



\end{document}